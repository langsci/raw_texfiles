\chapter{Syntax and clause structure}\label{chapter5}

This chapter is devoted to the syntactic phenomena that are present in the Mandan corpus. Mandan, like other Siouan languages, is noted for having complex verbal constructions that are complicated by the frequent omission of overt nominal constructions in the discourse. Discourse structure in Mandan is discussed in detail in Chapter \ref{chapter6}, while this chapter concerns itself with the grammar of phrases and clauses.

The syntax of nouns and elements associated with nominals has a very predictable and inviolable word order in Mandan. Throughout the corpus, there is no deviation from the order that is described in the section on nominal constructions below. The lack of malleability in the ordering of words within a nominal construction is indicative of the lack of flexibility in the semantics or pragmatics associated with such constructions. Overall, the syntax of nominal constructions is quite rigid, which contrasts with the fluidity of word order within uniclausal and multiclausal constructions. 

One of the more integral aspects of Mandan syntax is its system of switch-reference and reliance on clausal adjuncts to play the discourse role that coordination might otherwise play in other languages. Switch-reference and other interclausal relationships is discussed in this chapter at length. The default word order in Mandan is subject--object--verb, where the doer of the action or the experiencer of the state is the initial element, while the action or state is the final element in a sentence. There are numerous exceptions to this default order due to various topicalization strategies, which are discussed in the section on topicalization below.

\section{Nominal constructions}\label{Ch5Nouns}

Nouns often appear throughout the corpus with no overt morphology. The interaction of multiple words within a nominal construction is highly regimented. The noun, being the most salient constituent of a nominal construction, appears as the initial element of any nominal construction. Nouns can be followed by other elements, which are discussed at length in the sections below.

\subsection{Noun phrases}\label{Ch5NounPhrases}
\largerpage
The noun phrase in Mandan uniformly involves the noun or a compound noun being the leftmost element in its domain. Any adjunct materials, such as stative verbs that are functioning in an adjectival capacity or relative clauses, will immediately follow the noun being modified. Articles, determiners, and other enclitics will appear after any adjuncts that may be present. A template for noun phrases in Mandan appears in \tabref{TabNounPhrase} below.

\begin{table}
\caption{Noun phrase template}\label{TabNounPhrase}
\begin{tabular}{llllll}
\lsptoprule
0&1&2&3&4&5\\
\midrule
Noun&Stative verb&\textsc{inf}&\textsc{def}&\textsc{dem}&\textsc{top}\\
Compound noun&Relative clause&\\
\lspbottomrule
\end{tabular}
\end{table}

As \tabref{TabNounPhrase} shows, a noun phrase can be expanded to include adjunct materials plus enclitics. This template does not have any exceptions, so the aforementioned topic marker will always precede the definite article, the definite article will always precede the demonstrative, and the demonstrative will always precede the topic marker.

We can see some examples of noun phrases that conform to the above template in (\ref{NounExamples}) below.

\begin{exe}

\item\label{NounExamples} Examples of noun phrases

\begin{xlist}

    \item\label{NounExamplesA} \glll maná terés\\
    wrą trE=s\\
    \textnormal{tree} \textnormal{be.big.around}=def\\
    \glt `the big trees' \citep[21]{hollow1973a}

    \item\label{NounExamplesB} \glll mí' pshíireena\\
    wį' pshii=ee=rą\\
    \textnormal{stone} \textnormal{be.flat}=dem.dist=top\\
    \glt `the flat stone' \citep[316]{hollow1973b}

    \item\label{NounExamplesC} \glll kowóokih ~ ~ ~ ~ ~ ~ ~ ~ ~ ~ ~ ~ ~ ~ koxamáhseena\\
    ko-wV-o-kih ~ ~ ~ ~ ~ ~ ~ ~ ~ ~ ~ ~ ~ ~ ko-xwąh=s=ee=rą\\
    3poss.pers-unsp-pv.irr-\textnormal{man's.brother.in.law} ~ ~ ~ ~ ~ ~ ~ ~ ~ ~ ~ ~ ~ ~ rel-\textnormal{be.small}=def=dem.dist=top\\
    \glt `his brother-in-law who is smallest' \citep[133]{hollow1973a}
    
\end{xlist}

\end{exe}

Besides nouns and adjuncts, the most common kind of formative found within a noun phrase is the definite article =\textit{s}. Definiteness is not obligatorily marked in Mandan, as discussed earlier in \sectref{SubSecArticles}. See \sectref{SubSubSecDeicticDeterminers} for further information on the use of deictic determiners and \sectref{SubSecTopics} for more on topic marking.

\subsection{Quantifiers and numerals}

In Mandan, quantifiers and numerals will always appear to the right of a noun phrase. We can see instances of quantifiers and numerals being used with noun phrases in the examples in (\ref{Ch5Quantifiers}) below.

\begin{exe}
    \item\label{Ch5Quantifiers} Examples of quantifier placement with noun phrases

    \begin{xlist}
        \item\label{Ch5QuantifiersA} \glll tamáahkeres \textbf{ą́ąwe}\\
        ta-wąąh=krE=s \textbf{ąąwe}\\
        al-\textnormal{arrow}=3pl=def \textnormal{\bfseries all}\\
        \glt `\textbf{all} his arrows' \citep[155]{hollow1973a}

        \item\label{Ch5QuantifiersB} \glll mí'h óshi \textbf{tóop}\\
        wį'h o-shi \textbf{toop}\\
        \textnormal{blanket} pv.irr-\textnormal{be.good} \textnormal{\bfseries four}\\
        \glt `\textbf{four} blankets that are good' \citep[35]{hollow1973b}

        \item\label{Ch5QuantifiersC} \glll manáwerexe ko'ų́'st kotké kokámix ~ ~ ~ koxtés \textbf{kixų́ųh}\\
        wrą\#wrex=E ko-ų't=t ko-tke ko-kawįx ~ ~ ~ ko-xtE=s \textbf{kixųųh}\\
        \textnormal{wood}\#\textnormal{kettle}=sv rel-\textnormal{be.in.past}=loc rel-\textnormal{be.heavy} rel-\textnormal{be.round} ~ ~ ~ rel-\textnormal{be.big}=def \textnormal{\bfseries five}\\
        \glt `\textbf{five} big, round, heavy, old drums' \citep[21]{mixco1997a}

        \item\label{Ch5QuantifiersD} \glll manáwerexe ko'ų́'st kotké kokámix ~ ~ ~ ~ ~  \textbf{hų́}\\
        wrą\#wrex=E ko-ų't=t ko-tke ko-kawįx ~ ~ ~ ~ ~ \textbf{hų} \\
    \textnormal{wood}\#\textnormal{kettle}=sv rel-\textnormal{be.in.past}=loc rel-\textnormal{be.heavy} rel-\textnormal{be.round} ~ ~ ~ ~ ~ \textnormal{\bfseries many}\\
        \glt `\textbf{many} round, heavy, old drums' \citep[21]{mixco1997a}

    \item\label{Ch5QuantifiersE} \glll míih máakahe \textbf{są́ąkare}\\
    wįįh wąąkah=E \textbf{sąąka=re}\\
    \textnormal{woman} \textnormal{these}=sv \textbf{\textnormal{few}=dem.prox}\\
    \glt `these \textbf{few} women here' \citep[53]{hollow1973a}

    \item\label{Ch5QuantifiersF} \glll pt\'{ı̨}į \textbf{są́ąkana}\\
    ptįį \textbf{sąąka=rą}\\
    \textnormal{buffalo} \textnormal{\bfseries few}\textbf{=top}\\
    \glt `a few buffalo' \citep[79]{hollow1973b}

    \item\label{Ch5QuantifierG} \glll wáahokshuke \textbf{hų́nus}\\
    waa-hok\#kshuk=E \textbf{hų=rų=s}\\
    nom-\textnormal{voice}\#\textnormal{be.narrow}=sv \textnormal{\bfseries many}\textbf{=anf=def}\\
    \glt `those many animals' \citep[45]{hollow1973a}
    \end{xlist}
\end{exe}

In each of the examples above in (\ref{Ch5Quantifiers}), the quantifier or numeral appears to the right edge of the noun phrases. Noun phrases can contain appear with articles or demonstratives. It is worth noting that we do not see any instances of topicalized nominal constructions that appear under the scope of a quantifier or numeral. It is unclear whether this lack of =\textit{na} or =\textit{nu} marking within quantified nominal constructions is prohibited in Mandan or such constructions just never appeared in the corpus. Like many issues of Mandan grammar, the lack of L1 speakers inhibits a definitive answer. Quantifiers themselves, however, can bear topic marking, as we see in (\ref{Ch5QuantifiersF}) and (\ref{Ch5QuantifierG}) above. The fact that quantifiers can bear topic marking highlights the fact that morphological topic marking in Mandan is restricted to an entire constituent and not just components thereof, i.e., the whole nominal construction bears the topic marker rather than, say, the noun alone. Thus, quantifiers and numerals are treated as extensions of the noun phrase with respect to topic marking in Mandan.

\subsection{Possession}\label{Ch5Possession}

All nouns in Mandan are classified by whether they take alienable possession or inalienable possession. The morphological behavior of possessed nouns has already been discussed in \sectref{SecPossession}, so the description here focuses on the syntactic behavior of possession.

When a noun is possessed, the first noun is always the possessor, followed by the possessee. As previously described, possession marking appears on the possessee. However, overt possessors may not be present in the syntax, as Mandan frequently omits overt arguments that are judged to be obvious from the context of the discourse. Speech act participants that possess a noun -- first and second person possessors -- never have an overt nominal element appear before the possessed noun. A noun possessed by a first or second person possessor appear with only the possessive marking prefixed upon it and no preceding nominal construction. We can see examples of inalienable possession in the examples in (\ref{Ch5Inalienable}) below.

\begin{exe}
    \item\label{Ch5Inalienable} Examples of inalienable possession

    \begin{xlist}
        \item\label{Ch5InalienableA} \glll mí'maa\\
        w'\~~-iwąą\\
        1poss-\textnormal{body}\\
        \glt `my body' \citep[96]{hollow1970}

        \item\label{Ch5InalienableB} \glll numá'k imáare\\
        ruwą’k iwąą=E\\
        \textnormal{man} \textnormal{body}=sv\\
        \glt `a man's body' \citep[202]{hollow1973b}

        \item\label{Ch5InalienableC} \glll riráse\\
        ri-ras=E\\
        2poss-\textnormal{name}=sv\\
        \glt `your name' \citep[14]{hollow1973a}

        \item \glll koshų́ųkas ráse\\
        ko-shųųka=s ras=E\\
        3poss-\textnormal{younger.brother}=def \textnormal{name}=sv\\
        \glt `her younger brother's name' \citep[285]{hollow1973b}

    \end{xlist}
\end{exe}

As shown in (\ref{Ch5Alienable}) below, all possessors are able to be marked for definiteness, as are possessees. We can see this double definite marking in nouns involving alienable possession below in (\ref{Ch5AlienableB}), along with other examples involving alienable possession.

\begin{exe}
    \item\label{Ch5Alienable} Examples of alienable possession


    \begin{xlist}

        \item\label{Ch5AlienableA} \glll ptasúk\\
        p-ta-suk\\
        1poss-al-\textnormal{child}\\
        \glt `my child'
    
        \item\label{Ch5AlienableB} \glll Kóoxą'te Míihs tasúkseena\\
        kooxą'te wįįh=s ta-suk=s=ee=rą\\
        \textnormal{corn} \textnormal{woman}=def al-\textnormal{child}=def=dem.dist=top\\
        \glt `Corn Woman's child' \citep[112]{hollow1973a}

        \item\label{Ch5AlienableC} \glll nitámi'tis\\
        rį-ta-wį'\#ti=s\\
        2poss-al-\textnormal{stone}\#\textnormal{dwell}=def\\
        \glt `your village' \citep[217]{hollow1973a}

        \item\label{Ch5AlienableD} \glll Rá'puse tamí'ti\\
        ra'-pus=E ta-wį'\#ti\\
        ins.heat-\textnormal{be.speckled}=sv al-\textnormal{stone}\#\textnormal{dwell}\\
        \glt `Speckled Arrow's village' \citep[135]{hollow1973a}
    \end{xlist}
\end{exe}

Further complicating the matter of how to express possession in Mandan is the fact that many nouns are structurally derived from nominalized relative clauses that is headed by a preverb. Mandan shares this tendency to blur the line between noun and verb with other Siouan languages, where nouns are often a lexicalized description thereof, e.g., \textit{ína'ka} `corn grinder' is composed of the instrumental preverb \textit{i}-, the verb \textit{ná'} `grind', and the habitual aspect marker =\textit{ka}. Thus, this noun is most literally translated as 'what you grind [corn] with.' Therefore, possessive marking for such nouns overlaps with subject marking for verbs, as seen in the examples in (\ref{Ch5PossessionRel}) below.

\begin{exe}
    \item\label{Ch5PossessionRel} Possession marking on nouns derived from relative clauses

    \begin{xlist}
        \item\label{Ch5PossessionRelA} \glll óti\\
        o-ti\\
        pv.loc-\textnormal{dwell}\\
        \glt `his/her house' \citep[251]{hollow1970}

        \item\label{Ch5PossessionRelB} \glll ówati\\
        o-wa-ti\\
        pv.loc-1a-\textnormal{dwell}\\
        \glt `my house' \citep[251]{hollow1970}

        \item\label{Ch5PossessionRelC} \glll \'{ı̨}'karehirit\\
        į'-ka-rehrit\\
        pv.rflx-ins.frce-\textnormal{fan}\\
        \glt `his/her fan' \citep[87]{hollow1970}

        \item\label{Ch5PossessionRelD} \glll \'{ı̨}'mikarehirit\\
        į'-wį-ka-rehrit\\
        pv.rflx-1s-ins.frce-\textnormal{fan}\\
        \glt `my fan' \citep[87]{hollow1970}

        \item\label{Ch5PossessionRelE} \glll ímashut\\
        i-wąshut\\
        pv.ins-\textnormal{clothe}\\
        \glt `his/her clothes, shirt, dress, coat' \citep[97]{hollow1970}

        \item\label{Ch5PossessionRelF} \glll ríimashut\\
        rV-i-wąshut\\
        1s.pl-pv.ins-\textnormal{clothe}\\
        \glt `our clothes, shirt(s), dress(es), coat(s)' \citep[97]{hollow1970}
    \end{xlist}
\end{exe}

Nouns derived from relative clauses in Mandan are lexicalized to the extent that there is a shared concept of what the noun being described is. The descriptive nature of Mandan can create situations where the same nominalized relative clause can be interpreted in different ways depending on the context, e.g., \textit{wáakapus} is literally the nominalizer \textit{waa}- plus the verb \textit{kapús} `write, draw.' However, this one word can variously mean `book', `paper', `school', `writing', `homework', \textit{inter alios}.
Though these constructions take verbal person marking to express possession, they are clearly treated as nominal elements within the grammar, as they conform to the same template for nominal syntax and morphology described throughout this book. In the example in (\ref{HouseWithMorphology}) below, \textit{óti} `house' bears a definite article and a distal demonstrative, plus the whole construction is able to be taken under the scope of the locative postposition =\textit{t}.

\begin{exe}
    \item\label{HouseWithMorphology} \glll ótiseet\\
    o-ti=s=ee=t\\
    pv.loc-\textnormal{dwell}=def=dem.dist=loc\\
    \glt `into the house' \citep[202]{hollow1973a}
\end{exe}

The Mandan language vigorously employs this ability to create nouns from relative clauses throughout the extant corpus. A simple perusal of \citeapos{hollow1970} dictionary demonstrates that this class of noun represents a substantial percentage of the lexicon. Any future attempts at producing a Mandan dictionary should ensure that the morphological complexities of such nouns be explicitly clear to learners, though that effort lies outside the scope of the present book.


\subsection{Postpositional phrases}\label{Ch5PostpositionalPhrasesSec}

Like other languages with a default subject--object--verb word order, Mandan features postpositions instead of prepositions. Postpositions will always be the rightmost element of any kind of nominal construction. Most postpositions are independent words, though there are a small set of postpositions that are enclitics that can never appear independent of the noun over which they have scope. 

Postpositions in Mandan behave like stative verbs in that take stative person marking for any first or second person arguments. 
We can see some examples of postpositional phrases below in (\ref{Ch5PostpositionalPhrases}) where the postpositions themselves are independent words, as well as instances in (\ref{Ch5PostpositionalPhrasesB}) and (\ref{Ch5PostpositionalPhrasesD}) where stative person marking appears on the postposition to indicate that the postposition is taking a non-third person argument.

\begin{exe}
    \item\label{Ch5PostpositionalPhrases} Examples of free postpositional phrases

    \begin{xlist}
        \item\label{Ch5PostpositionalPhrasesA} \glll Kóoxą’te Míihs ų́ųpa\\
        kooxą'te wįįh=s ųųpa\\
        \textnormal{corn} \textnormal{woman}=def \textnormal{with}\\
        \glt `with Corn Woman' \citep[112]{hollow1973a}

        \item\label{Ch5PostpositionalPhrasesB}
        \glll mú'pa\\
        w'-ųųpa\\
        1s-\textnormal{with}\\
        \glt `with me' \citep[284]{hollow1973b}

        \item\label{Ch5PostpositionalPhrasesC}
        \glll taté ų́'t\\
        tate ų't\\
        \textnormal{father}.voc \textnormal{towards}\\
        \glt `to father' \citep[25]{kennard1936}

        \item\label{Ch5PostpositionalPhrasesD}
        \glll mi'ų́'t\\
        wį-ų't\\
        1s-\textnormal{towards}\\
        \glt `to me' \citep[122]{hollow1973b}

        \item\label{Ch5PostpositionalPhrasesE} \glll ókxųhe máapes\\
        o-kxųh=E wąąpE=s\\
        pv.loc-\textnormal{lie.down}=sv \textnormal{under}=def\\
        \glt `under her bed' \citep[321]{hollow1973b}

    \end{xlist}

\end{exe}

As stated above, there is a small set postpositions that occur in Mandan as enclitics. Such postpositions never appear without being bound to a preceding word. We also never see any instances of these enclitic postpositions being used with first or second person pronominal prefixes. It is unclear if such constructions would be ungrammatical or if the semantics of this small class of postpositions limits the contexts under which anyone would need to produce such utterances. A set of examples of enclitic postpositions with nominal complements appears in (\ref{Ch5PostpositionalPhrases2}) below.

\newpage

\begin{exe}

\item\label{Ch5PostpositionalPhrases2} Examples of enclitic postpositional phrases

    \begin{xlist}
        \item\label{Ch5PostpositionalPhrases2A} \glll mústaa\\
        wųt=taa\\
        \textnormal{garden}=loc\\
        \glt `in the garden' \citep[84]{hollow1973a}

        \item\label{Ch5PostpositionalPhrases2B} \glll mí'tit\\
        wį'\#ti=t\\
        \textnormal{stone}\#\textnormal{dwell}=loc\\
        \glt `to the village' \citep[107]{hollow1973a}

        \item\label{Ch5PostpositionalPhrases2C} \glll ómanatsoohąą\\
        o-wrąt=s=oo=hąą\\
        pv.irr-\textnormal{axe}=def=dem.mid=ins\\
        \glt `with the axe' \citep[25]
        {kennard1936}
    \end{xlist}

\end{exe}

Postpositions in Mandan often form strings where an enclitic postposition will combine with a free postposition. These compound postpositions typically involve a single postpositional root plus some form of the locative enclitic, either =\textit{taa} or =\textit{t}. Another common compound postposition is \textit{ó'harani} `from', which is morphologically composed of the verb \textit{ó'} `be' plus the causative and same-subject switch-reference marker. This construction is no longer decomposable semantically for speakers and has lexicalized to indicate direction away from a person or place or temporal distance from a point in time.  We can see examples of compound postpositional phrases in the examples in (\ref{Ch5PostpositionalPhrases3}) below.

\begin{exe}

\item\label{Ch5PostpositionalPhrases3} Examples of compound postpositional phrases

    \begin{xlist}
        \item\label{Ch5PostpositionalPhrases3A} \glll miníxte áakihąą\\
        wrį\#xtE aaki=hąą\\
        \textnormal{water}\#\textnormal{be.big} \textnormal{above}=ins\\
        \glt `on top of the lake' \citep[212]{hollow1973a}

        \item\label{Ch5PostpositionalPhrases3B} \glll \'{ı̨}'kapxe maná ówa'shkap kú'shtaa\\
        į'-kapxe wrą o-wa'-shkap ku'sh=taa\\
        pv.rflx-\textnormal{earn} \textnormal{wood} pv.loc-ins.prce-\textnormal{pinch} \textnormal{inside}=loc\\
        \glt `inside the sweat lodge stake holes' \citep[119]{hollow1973a}

        \item\label{Ch5PostpositionalPhrases3C} \glll mí'ti ų́ųpat ó'harani\\
        wį'\#ti ųųpat o'\#hrE=rį\\
        \textnormal{stone}\#\textnormal{dwell} \textnormal{other} \textnormal{be}\#caus=ss\\
        \glt `from another village' \citep[138]{hollow1973b}
    \end{xlist}
\end{exe}

Compound postpositions are treated as a single word with respect to their status as a syntactic and prosodic unit. 

\subsection{Coordination}\label{Ch5NounCoord}

The coordination of nominal compounds in Mandan is similar to that of verbal coordination in that the same-subject switch-reference marker =\textit{ni} is employed. In such constructions, the same-subject switch-reference marker appears after every noun in the coordinated phrase. We can see examples of coordinated structures in the data in (\ref{ExNounCoord}) below.

\begin{exe}

\item\label{ExNounCoord} Examples of =\textit{ni} with nominal coordination

\begin{xlist}

\item\label{ExNounCoordA} \glll shehékini xarátani ą́ąwe kihíkereoomaks.\\
shehek=rį xratE=rį ąąwe ki-hi=krE=oowąk=s\\
\textnormal{coyote}=ss \textnormal{wolf}=ss \textnormal{all} rflx-\textnormal{arrive.there}=3pl=narr=def\\
\glt `The wolf and coyote all got together.' \citep[43]{hollow1973a}

\item\label{ExNounCoordB} \glll mí'hini wáa'opąąpi túwaharani...\\
wį'h=rį waa-o-pąąpi tu\#wa-hrE=rį\\
\textnormal{robe}=ss nom-pv.irr-\textnormal{be.thin} \textnormal{be.some}\#1a-caus=ss\\
\glt `I made a robe and some calico goods and...' \citep[62]{hollow1973a}

\item\label{ExNounCoordC} \glll ímaareksukini patóhanashini míihąni matewé ~ ~ ó'roshka\\
i-wąąreksuk=rį pa\#toh=rąsh=rį wiihą=rį wa-t-we ~ ~ o'=oshka\\
pv.poss-\textnormal{bird}=ss \textnormal{head}\#\textnormal{blue/green}=att=ss \textnormal{goose}=ss unsp-wh-indf ~ ~ \textnormal{be}=emph\\
\glt `birds, ducks, geese, and whatever it is' \citep[91]{hollow1973a}

\end{xlist}


\end{exe}

Another strategy for expressing nominal coordination is the construction \textit{éeheni}. Literally, this is composed of the preverb \textit{ée}- plus the verb \textit{hé} `say' and same-subject switch-reference marker. It is likely derived from some periphrastic hearsay construction akin to `one says X and...' or `they say X and...', but it has become lexicalized as a true coordinator. However, this coordinator is exclusively used for nominal constructions and is not used to link verbal constructions. We can see examples of \textit{éeheni} in (\ref{ExNounCoord2}) below from the corpus. 

\newpage

\begin{exe}

\item\label{ExNounCoord2} Examples of nominal coordination with \textit{éeheni}



\begin{xlist}
\item\label{ExNounCoord2A} \glll Kinúma'kshi éeheni Numá'k Máxana ~ ~ ~ ~ ~ ~ ~ ~ ~ ~ ~ ~ ~ ~ íwahekanashe\\
ki-ruwą'k\#shi ee-he=rį ruwą'k wąxrą ~ ~ ~ ~ ~ ~ ~ ~ ~ ~ ~ ~ ~ ~ i-wa-hek=rąsh=E\\
mid-\textnormal{man}\#\textnormal{be.good} pv-\textnormal{say}=ss \textnormal{man} \textnormal{one} ~ ~ ~ ~ ~ ~ ~ ~ ~ ~ ~ ~ ~ ~ pv.ins-1a-\textnormal{know}=att=sv\\
\glt `what I know about Royal Chief and Lone Man' \citep[1]{hollow1973a}

\item\label{ExNounCoord2B} \glll na'é éeheni taté éeheni mishų́ųka éeheni ą́ąwe máakahoomakak, ą́'teetaa\\
rą'e ee-he=rį tatE ee-he=rį wį-shųųka ee-he=rį ąąwe wąąkah=oowąk=ak ą't=ee=taa\\
\textnormal{mother}.voc pv-\textnormal{say}=ss \textnormal{father}.voc pv-\textnormal{say}=ss 1poss-\textnormal{younger.brother} pv-\textnormal{say}=ss \textnormal{all} \textnormal{lie}.aux.pos=narr=ds dem.anap=dem.dist=loc\\
\glt `my mother, father, and brother all must be living there, over there' \citep[77]{hollow1973a}

\item\label{ExNounCoord2C} \glll máamanahku éeheni wáahokshuke wáateenixa núuniha kisúkini...\\
waawarąhku ee-he=rį waa-hok\#shuk=E waa-tee=rįx=E ruurįh=E=$\varnothing$ ki-suk=rį\\
\textnormal{deer} pv-\textnormal{say}=ss nom-\textnormal{voice}\#\textnormal{be.narrow}=sv neg=\textnormal{die}=neg=sv \textnormal{be.there}.pl.dur.aux=sv=cont vert-\textnormal{exit}=ss\\
\glt `deer and animals that were not dead came back out and...' \citep[150]{hollow1973a}

\end{xlist}
    
\end{exe}

One final strategy for coordinating noun phrases is use \textit{inák} `again' after each of the coordinated noun phrases. This tactic for indicating coordination appears to be used in longer lists, and as such, it is rare in the corpus. However, when asked to produce a sentence with a longer set of coordinated noun phrases, the utterance in (\ref{ExBensonCoord}) below is what Mr. Edwin Benson produced.

\begin{exe}

\item\label{ExBensonCoord} Example of nominal coordination with \textit{inák}

\glll Pt\'{ı̨}į inák, mapáakokohka shóte inák, ~ ~ míihą inák, pusé tóhe inák, minísweerut psí inák,  ~ ~ ~ ~ ~ ~  miníse xíi ropxímina'kitaa sí ráahini ~ ~ ~ éerehkere're.\\
ptįį irąk wą-paa-ko$\sim$kok=ka shot=E irąk ~ ~ wiihą irąk puse toh=E irąk wrįs\#wee\#rut psi irąk  ~ ~ ~ ~ ~ ~  wrįs=E xii ropxi\#wirą'ki=taa si=$\varnothing$ rEEh=rį ~ ~ ~ ee-reh=krE=o're\\
\textnormal{buffalo} \textnormal{again} unsp-\textnormal{be.bitter}\#aug$\sim$\textnormal{antelope}=hab \textnormal{be.white}=sv \textnormal{again} ~ ~ \textnormal{goose} \textnormal{again} \textnormal{cat} \textnormal{be.blue} \textnormal{again} \textnormal{horse}\#\textnormal{feces}\#\textnormal{eat} \textnormal{be.black} \textnormal{again} ~ ~ ~ ~ ~ ~ \textnormal{horse} \textnormal{be.buckskin.color} \textnormal{hide}\#\textnormal{boat}=loc \textnormal{travel}=cont \textnormal{go.there}=ss ~ ~ ~ pv-\textnormal{want}=3pl=ind.f\\
\glt `A buffalo, a white butterfly, a goose, a blue cat, a black dog, and a buckskin horse want to go traveling.' \citep[9]{bensonspottedbearkasak2016}
\end{exe}

In all of the instances of coordination discussed here, coordination involves polysyndeton, i.e., the repetition of a coordinating morphological element for each item that is syntactically coordinated. In instances of coordinated nominal constructions, the only element that does not feature a morphological marker of coordination is the final element in the series. Coordination does not involve monosyndeton in series of three or more.

Disjunction and coordination have identical syntactic structures in Mandan, and they employ identical morphology. Thus, it is only apparent whether =\textit{ni} indicates `and' or `or' from the context, as we can see in the example in (\ref{Ch5ExNIDisjunction}) below.

\begin{exe}
    \item\label{Ch5ExNIDisjunction} Example of nominal disjunction with =\textit{ni}

    \glll tamáana kíimanashini kúupa tewé hą́kerookto'sh\\
    ta-waarą kiiwą=rąsh=rį kuupa t-we hąk=roo=kt=o'sh\\
    al-\textnormal{winter} \textnormal{six}=att=ss \textnormal{seven} wh-indf pos.stnd=dem.mid=pot=ind.m\\
    \glt `[I don't know] whether he was six or seven years old' \citep[195]{hollow1973a}
\end{exe}

Nominal coordination strongly parallels both verbal and clausal coordination, which is explained in greater detail in \sectref{Ch5Coordination}. One caveat is that nominal coordination will exclusively involve the same-subject switch-reference marker =\textit{ni}. There is no instance of nomaminal coordination where the different-subject switch-reference marker =\textit{ak} is involved.

\section{Verbal constructions}\label{Ch5Verbs}

The true weight of Mandan discourse is carried by the verb. As discussed at length in Chapter \ref{chapter3}, Mandan has an elaborate set of verbal morphology. The template for verbs in Mandan appears again in \tabref{prefixfieldmandanREredux2} below.

\begin{table}
\caption{Prefix field in Mandan}\label{prefixfieldmandanREredux2}

\fittable{\scshape
\begin{tabular}{llllllllllll}
    \lsptoprule
    11  & 10  & 9      & 8    & 7   & 6          & 5      	& 4		& 3			& 2     & 1     & 0    \\
    \midrule
    rel & neg & unsp   & 1pl & pv.irr & pv.loc   & 1sg 		& 2sg		& suus		& iter  & ins & stem \\
    ~   & ~   & ~ & ~    & ~   & pv.ins 	 & ~      	& 2pl		& mid			& incp   & ~     & ~    \\
    ~   & ~   & ~      & ~    & ~   & pv.tr 	 & ~      	& ~		& recp		& ~  & ~     & ~    \\
    \lspbottomrule
    \end{tabular}}
\end{table}

Mandan is a language with a default subject--verb--object word order. As such, verbs are typically the last element in an utterance. The template above highlights how much information verbs carry in just the prefix field alone. When taking aspectual, evidential, modal, negational, and sentential enclitics discussed in \sectref{SecPhrasalMorphology}, the information load placed on verbs becomes all the more evident.

\subsection{Verb phrases}\label{Ch5VerbPhrases}

A typical Mandan clause features a single verb within its verb phrase. That is to say, most Mandan clauses do not involve auxiliary verbs. Given the propensity for omitting overt arguments from the discourse, many sentences in Mandan are composed of a single word: the verb. Numerous examples throughout this book have involved a single word bearing the information load of an entire sentence. Mandan grammar likewise is able to treat nouns as stative verbs when no other verbal material is present. We can see several examples of this phenomenon in (\ref{C5VerbOnlySentences}) below, where a single verb contains sufficient information to constitute a whole sentence.

\begin{exe}
    \item\label{C5VerbOnlySentences} Examples of one-word sentences

    \begin{xlist}
        \item\label{C5VerbOnlySentencesA} \glll Shí're.\\
        shi-o're\\
        \textnormal{be.good}=ind.f\\
        \glt `It is good.' \citep[15]{hollow1976}

        \item\label{Ch5VerbOnlySentencesB} \glll Wakaráho'sh.\\
        wa-krah=o'sh\\
        1a-\textnormal{be.afraid}=ind.m\\
        \glt `I am afraid.' \citep[4]{hollow1973b}

        \item\label{Ch5VerbOnlySentencesC} \glll Éeminpeso'sh.\\
        ee-w-rį-pE=s=o'sh\\
        pv-1a-2s-\textnormal{say}.1a=def=ind.m\\
        \glt `I definitely said it to you.' \citep[320]{hollow1973b}

        \item\label{Ch5VerbOnlySentencesD} \glll Istų́hoomako'sh\\
        istųh=oowąk=o'sh\\
        \textnormal{night}=narr=ind.m\\
        \glt `It was night.' \citep[171]{hollow1973b}
    \end{xlist}
\end{exe}

In verb phrases where an auxiliary verb occurs, the main verb will precede the auxiliary verb. For example, we see in (\ref{Ch5AuxiliaryVerbsA}) below, where the visual evidential auxiliary \textit{ishí} appears after the stative verb \textit{psí} `be black.' Likewise, the verb \textit{kú'} `give' is used in benefactive constructions, as we can see in (\ref{Ch5AuxiliaryVerbsB}). Subject marking must be present on each of the verbs within the verb phrase, though object marking is only marked once.

\begin{exe}
    \item\label{Ch5AuxiliaryVerbs} Examples of auxiliary verbs in a verb phrase
    \begin{xlist}
        \item\label{Ch5AuxiliaryVerbsA} \glll psí ishí'sh\\
        psi ishi=o'sh\\
        \textnormal{be.black} vis=ind.m\\
        \glt `it must be black' \citep[123]{hollow1973a}

        \item\label{Ch5AuxiliaryVerbsB} \glll warúsha minikú'kto’sh\\
        wa-ru-shE w-rį-ku'=kt=o'sh\\
        1a-ins.hand-\textnormal{hold} 1a-2s-\textnormal{give}=pot=ind.m\\
        \glt `I will take it for you' \citep[307]{hollow1973b}
    \end{xlist}
\end{exe}

Other verbs that have been talked about in the literature as auxiliaries -- for example, the positionals \textit{hą́k}, \textit{mák}, and \textit{nák} -- actually introduce multiclausal constructions rather than functioning as true auxiliary verbs. This topic is explored further in \sectref{Ch5Multiclausal}.

\subsection{Simple clause structure}\label{Ch5Clauses}
\largerpage
Mandan is like other Siouan languages in that its default word order is subject--object--verb. One fortunate aspect of the Mandan corpus is that it is predominantly composed of free speech in the form of traditional and personal narratives. Therefore, there is a large amount of data to back up this core aspect of Mandan grammar. We can see examples of this canonical word order in (\ref{Ch5SOVWordOrder}) below. Subjects are indicated in bold, while direct objects are underlined.

\newpage

\begin{exe}
    \item\label{Ch5SOVWordOrder} Examples of SOV word order

    \begin{xlist}
        \item\label{Ch5SOVWordOrderA} \glll \textbf{Koshų́ųkas} \uline{máamanahku} \uline{hų́} téeheres\\
        ko-shųųka=s waawarąhku hų tee\#hrE=s\\
        3poss.pers-\textnormal{younger.brother}=def \textnormal{deer} \textnormal{many} \textnormal{die}\#caus=def\\
        \glt `\textbf{Her brother} killed \uline{a lot of deer}.' \citep[202]{hollow1973b}

        \item\label{Ch5SOVWordOrderB} \glll \textbf{Kinúma'kshiseena} \uline{Numá'k} \uline{Máxanas} ~ ~ ~ ~ ~ ~ ~ ~ pahų́hanashoomaks\\
        {ki-ruwą'k\#shi=s=ee=rą} {ruwą'k} {wąxrą=s} ~ ~ ~ ~ ~ ~ ~ ~ pa-hųh=rąsh=oowąk=s\\
        {mid-}\textnormal{man}{\#}\textnormal{be.good}{=def=dem.dist=top} {\textnormal{man}} {\textnormal{one}=def} ~ ~ ~ ~ ~ ~ ~ ~ ins.push-\textnormal{get.ahead.of}=att=narr=def\\
        \glt `\textbf{Royal Chief} got ahead of \uline{Lone Man}.' \citep[9]{hollow1973a}

        \item\label{Ch5SOVWordOrderC} \glll \textbf{Paxirúukeena} \uline{miní} h\'{ı̨}įhere\\
        paxruuk=ee=rą wrį hįį\#hrE\\
        \textnormal{cornsilk}=dem.dist=top \textnormal{water} \textnormal{drink}\#caus\\
        \glt `\textbf{Cornsilk} made him drink \uline{water}' \citep[133]{hollow1973a}

        \item\label{Ch5SOVWordOrderD} \glll \textbf{Komíihere} \uline{maná} \uline{ósasak} ~ ~ ~ ~ ~ ~ ~ ~ rutą́ąnik.\\
        ko-wįįh=re wrą o-sa$\sim$sak ~ ~ ~ ~ ~ ~ ~ ~ ru-tąą=rįk\\
        3poss.pers-\textnormal{man's.sister}=dem.prox \textnormal{wood} pv.irr-aug$\sim$\textnormal{dry} ~ ~ ~ ~ ~ ~ ~ ~ ins.hand-\textnormal{drag}=iter\\
        \glt `\textbf{This sister of his} was dragging \uline{dry wood}.' \citep[196]{hollow1973a}
        
    \end{xlist}
\end{exe}

We have seen throughout this book that many overt nominal arguments are omitted when Mandan speakers produce utterances. As such, it is less common to see this kind of explicit SOV sentence structure in a running narrative, as listeners are trusted to keep track of who is doing what to whom. It is very common to not hear an overt subject for an extended period of a narrative, especially when there are only one or two individuals to keep track of.

In sentences with indirect objects, the indirect object is almost never explicitly present in the syntax. Furthermore, it is rare for even two arguments to be explicitly present in the syntax. We can infer, however, that in sentences with indirect objects that the indirect object will precede the direct object in the syntax. We can see in example (\ref{Ch5IndirectObjectOrder}) below that there is an omitted subject `he' and the indirect object being marked is second person singular. The noun \textit{íwarahere} `for your food' bears the directional preverb \textit{í}-, indicating that the action is being done for the sake of the second person argument having food. The direct object \textit{pt\'{ı̨}į} `cow' is the direct object, as it is the entity to whom the action is happening. The direct object is underlined, while the indirect object is underlined.

\begin{exe}
\item\label{Ch5IndirectObjectOrder} Ordering of objects in ditransitive constructions

\glll \uline{Íwarahere} \textbf{pt\'{ı̨}i}̨ téeharani niku'́kto'sh.\\
i-wa-ra-hrE ptįį tee\#hrE=rį rį-ku'=kt=o'sh\\
pv.dir-unsp-2a-caus \textnormal{buffalo} \textnormal{die}\#caus=ss 1s-\textnormal{give}=pot=ind.m\\
\glt `He will kill \textbf{a cow} \uline{for your food}.' \citep[60]{hollow1973a}
\end{exe}

Direct objects appear immediately before the verb in neutral utterances where there is no topicalization causing another argument to be moved to a more prominent position within the clause. In other ditransitive constructions, such as those where an applicative preverb adds an additional argument to the syntax, we likewise see that direct objects are the closest argument to the verb, while the tertiary argument appears after the subject but before the direct object.\footnote{I am simply using the term tertiary argument as a cover term for arguments that are introduced by a ditransitive construction that are not subjects or direct objects. I imply no theoretical motivation behind the use of this term.} 

In the example below, Royal Chief is participating in an game with some children where they are throwing their eyes at a tree and then calling them back. Royal Chief is the unspoken subject in this sentence, and we can assume that he would be the initial element in (\ref{Ch5ApplicativeArgumentsEx}) if the narrator had chosen to include him. The direct object \textit{istámi'} `eye' is the closest element to the verb, as expected. The applicative argument introduced by the directional preverb \textit{í}- is the postpositional phrase \textit{skiskíka kaxtékseet} `to the willow bunch.' Like the indirect object in (\ref{Ch5IndirectObjectOrder}) above, we see that the tertiary argument appears before the direct object below in (\ref{Ch5ApplicativeArgumentsEx}). The infrequency of overt nominal constructions in Mandan discourse can cause drawing generalizations about syntax to be opaque, but the pattern holds that when a ditransitive verb occurs with overt arguments in the corpus, the direct object is always immediately before the verb, while the other non-subject argument appears immediately before the direct object. The direct object is again underlined, while the tertiary argument is shown in bold.

\begin{exe}
    \item\label{Ch5ApplicativeArgumentsEx} Ordering of applicative arguments in ditransitive constructions

    \glll \textbf{Skiskíka} \textbf{kaxtékseet} \uline{istámi'} íkų'teoomako'sh.\\
    skiskika kaxtek=s=ee=t istawį' i-kų'tE=oowąk=o'sh\\
    \textnormal{willow} \textnormal{bunch}=def=dem.dist=loc \textnormal{eye} pv.dir-\textnormal{throw}=narr=ind.m\\
    \glt `He threw \textbf{his eyes} \uline{to the willow bunch}.' \citep[34]{hollow1973a}
\end{exe}

Adjunct material that affect the overall clause, such as temporal or deictic adverbials, have two typical placements within a Mandan clause. Adjuncts either directly precede or directly follow the subject. In (\ref{Ch5AdjunctWordOrderA}), we see the deictic adjunct \textit{óo} `there' at the beginning of the sentence, appearing before the subject \textit{Kawóoxohkare} `this Swallower.' The placement of adjuncts in (\ref{Ch5AdjunctWordOrderA}) contrasts with the placement of the adverb \textit{inák} `again' in (\ref{Ch5AdjunctWordOrderB}). In the latter, the adverb appears after the subject \textit{rétaaseena} `the other one.' In the free translation that follows, I have rendered it is `The other one asked \textbf{again} for his food', rather than `The other one asked for his food \textbf{again}.' Given that English allows for the syntactic flexibility to render the same core proposition with two different syntactic outputs, it is likely the case that Mandan features a similar flexibility with adjuncts that have semantic scope over the whole proposition. There may be a slight pragmatic difference in choosing to place the adjunct before the subject or after the subject, but there are no L1 speakers to elucidate this issue. We also see in (\ref{Ch5AdjunctWordOrderC}) situations where two different adjuncts appear in different spots within the sentence, i.e., before the subject and after it. The adjunct in each example is shown in bold, though in (\ref{Ch5AdjunctWordOrderC}), the second adjunct is underlined.

\begin{exe}
    \item\label{Ch5AdjunctWordOrder} Word order with adverbial adjuncts

    \begin{xlist}
        \item\label{Ch5AdjunctWordOrderA} \glll \textbf{Óo} Kawóoxohkare óti ~ ~ ~ ó'roomako'sh.\\
        oo ka-wV-o-xok=ka=re o-ti ~ ~ ~ o'=oowąk=o'sh\\
        dem.mid agt-unsp-pv.loc-\textnormal{swallow}=hab=dem.prox pv.loc-\textnormal{dwell} ~ ~ ~ \textnormal{be}=narr=ind.m\\
        \glt `It was this Swallower's house \textbf{there}.' \citep[149]{hollow1973a}

        \item\label{Ch5AdjunctWordOrderB} \glll Rétaaseena \textbf{inák} waheré   ~ ~ ~ ~ ~ ~ ~ ~ ~ ~ ~ ~ waká'roomako'sh.\\
        retaa=s=ee=rą irąk wa-hrE   ~ ~ ~ ~ ~ ~ ~ ~ ~ ~ ~ ~ wa-ka'-oowąk=o'sh\\
        \textnormal{be.another}=def=dem.dist=top \textnormal{again} unsp-caus  ~ ~ ~ ~ ~ ~  ~ ~ ~ ~ ~ ~ unsp-\textnormal{possess}=narr=ind.m\\
        \glt `The other one asked \textbf{again} for his food.' \citep[93]{hollow1973b}
        
        \item\label{Ch5AdjunctWordOrderC} \glll \textbf{Háa} \textbf{áakitaa} \textbf{ó'harani} máaheena  ~ ~ ~ ~ ~ ~  ~ ~ ~ ~ ~ ~ \uline{ímaapetaa} húuroomako'sh.\\
        haa aaki=taa o'\#hrE=rį wąąh=ee=rą  ~ ~ ~ ~ ~ ~  ~ ~ ~ ~ ~ ~ i-wąąpE=taa huu=oowąk=o'sh\\
        \textnormal{cloud} \textnormal{above}=loc \textnormal{be}\#caus=ss \textnormal{arrow}=dem.dist=top  ~ ~ ~ ~ ~ ~  ~ ~ ~ ~ ~ ~ pv.dir-\textnormal{down}=loc \textnormal{come.here}=narr=ind.m\\
        \glt `An arrow came \uline{downward} \textbf{from heaven}.' \citep[154]{hollow1973a}
        
    \end{xlist}
\end{exe}

It is not obvious which of these two locations is the true default for sentential adjunct placement, as Mandan makes frequent use of topicalization to shift elements to the front or back of a sentence to place increased levels of attention to such constitutents. More information on topicalization apppears in \sectref{Ch5TopicAndFocus}.

Overall, when Mandan speakers have been asked to produce sentences without any overt emphasis on any one element, the default word order will always be subject first and verb last. Direct objects will always precede the verb, while indirect objects or arguments of an applicative preverb will appear before the direct object but after the subject. Sentential adjuncts appear either sentence-initially or after the subject. With respect to adjuncts, it is still unclear whether the position after the subject is the most neutral and the sentence-initial position is due to topicalization.

\subsection{Negation}\label{Ch5Negation}

Negation in Mandan is unique within Siouan languages in that verbal negation requires two formatives to convey the proper semantics. As discussed previously in \sectref{SubsubsecNegative} and \sectref{Ch3NegativeEnclitics}, Mandan negation is morphological realized by the joint appearance of the negative prefix \textit{waa}- before the verb stem and a negative enclitic, =\textit{xi} or =\textit{nix} after the verb stem.

There is no change in word order due to verbal negation, given that negation involves inflectional morphology on the verb itself rather than a free word or particle somewhere else in the sentence for statements or questions. Nouns are treated as stative verbs when negated, so no copular constructions are required to negate them. We can instances of negation in the examples in (\ref{Ch5ExNegation}) below with the negated verbs shown in bold.

\begin{exe}

\item\label{Ch5ExNegation} 

    \begin{xlist}
        \item\label{Ch5ExNegationA} \glll Némak \textbf{wáahokinixo'sh}.\\
        re\#wąk waa-hok=rįx=o'sh\\
        dem.prox\#pos.lie neg-\textnormal{story}=neg=ind.m\\
        \glt `This is not a story.' \citep[1]{trechter2012b}
        
        \item\label{Ch5ExNegationB} \glll Kó'ts \textbf{wáahaxikereroomako'sh}.\\
        ko-at=s waa-hE=xi=krE=oowąk=o'sh\\
        3poss.pers=\textnormal{father}=def neg-\textnormal{see}=neg=3pl=narr=ind.m\\
        \glt `They did not see their father.' \citep[209]{hollow1973a}

        \item\label{Ch5ExNegationC} \glll Koníhka, wáa'eepes \textbf{wáarakina'nixa'shka}?\\
        ko-rįk=ka waa-ee-pE=s waa-ra-kirą'=rįx=a'shka\\
        rel-\textnormal{be.small}=hab nom-pv-\textnormal{say}.1a=def neg-2a-\textnormal{tell}=neg=psbl\\
        \glt `Dear, didn't you tell what I said?' \citep[234]{hollow1973b}

        \item\label{Ch5ExNegationD} \glll Tashkák \textbf{máamanarutinixo'sha}?\\
        tashka=k waa-w-rą-rut=rįx=o'sha\\
        \textnormal{how}=ds neg-1s-2a-\textnormal{eat}=neg=int.m\\
        \glt `Why don't you eat me?' \citep[58]{hollow1973b}
        
    \end{xlist}


\end{exe}

The data thus far has been restricted to statements and questions because negation for imperatives functions differently in Mandan. Mandan is the only Siouan language to have a negative imperative particle, \textit{káare}, that appears before the verb as a free word. Furthermore, this negative imperative particle is the sole morphological manifestation of negation in imperatives, as there is no =\textit{nix} or =\textit{xi} enclitics with commands. Generally, \textit{káare} appears before the verb, but preceding direct objects. Mandan speakers appear to have some freedom with where \textit{káare} can go in a sentence, as it can appear sentence-initially ahead of adjuncts like postpositional phrases or adverbials. It is not clear whether this flexibility is due to \textit{káare} having a fixed position in a sentence and other words being clefted to different positions to the left edge of the sentence or if \textit{káare} is likewise subject to topicalization by Mandan speakers. As is often the case, the lack of L1 speakers at this point in time has made this question more opaque. The negative imperative particle in (\ref{Ch5ExImperativeNegatives}) below has appears in bold to highlight the variability of its placement.

\begin{exe}
    \item\label{Ch5ExImperativeNegatives} Examples of negation of imperatives

    \begin{xlist}

    \item\label{Ch5ExImperativeNegativesA} \glll \textbf{Káare} ų́'shka ísekta!\\
    kaare ų'sh=ka i-sek=ta\\
    imp.neg \textnormal{be.thus}=hab pv.ins-\textnormal{make}=imp.m\\
    \glt `Don't do such things!' \citep[157]{trechter2012b}

    \item\label{Ch5ExImperativeNegativesB} \glll Ų́'shka \textbf{káare} ísekana!\\
    ų'sh=ka kaare i-sek=rą\\
    \textnormal{be.thus}=hab imp.neg pv.ins-\textnormal{make}=imp.f\\
    \glt `Don't do such things!' \citep[158]{trechter2012b}

    \item\label{Ch5ExImperativeNegativesC} \glll \textbf{Káare} ótaamaharata, mishų́ųka!\\
    kaare o-taa\#wą-hrE=ta wį-shųųka\\
    imp.neg pv.loc-\textnormal{be.facing}\#1s-caus=imp.m 1poss.\textnormal{younger.brother}\\
    \glt `Don't point it at me, brother!' \citep[167]{hollow1973a}

     \item\label{Ch5ExImperativeNegativesD} \glll Nitámaanuka'teet \textbf{káare} ~ ~ ~ ~ ~ ~ ótaamaharata!\\
    rį-ta-waarųka=ą't=ee=t kaare ~ ~ ~ ~ ~ ~ o-taa\#wą-hrE=ta\\
    2poss-al-\textnormal{man's.friend}=dem.anap=dem.dist=loc imp.neg ~ ~ ~ ~ ~ ~ pv.loc-\textnormal{be.facing}\#1s-caus=imp.m\\
    \glt `Don't face it towards your friend!' \citep[167]{hollow1973a}

    \item\label{Ch5ExImperativeNegativesE} \glll ahkó maná óshote óhųtaa \textbf{káare} ~ ~ ~ ~ ~  ráahinista!\\
    ahko wrą o-shot=E o-hų=taa kaare ~ ~ ~ ~ ~  rEEh=rįt=ta\\
    \textnormal{over.there} \textnormal{wood} pv.irr-\textnormal{be.white}=sv pv.irr-\textnormal{many}=loc imp.neg ~ ~ ~ ~ ~  \textnormal{go.there}=2pl=imp.m\\
    \glt `Don't go over there where there is a lot of white wood!' \citep[168]{hollow1973a}

    \item\label{Ch5ExImperativeNegativesF} \glll \textbf{Káare} xamáhashka ísįįre wáarute ~ ~ ~ ~ ~ ~ ~ ~ ~ ~ ~ míkinista!\\
    kaare xwąh-aska i-sįį=E waa-rut=E ~ ~ ~ ~ ~ ~ ~ ~ ~ ~ ~  wįk=rįt=ta\\
    imp.neg \textnormal{be.small}-emph pv.poss-\textnormal{fat}=sv nom-\textnormal{eat}=sv ~ ~ ~ ~ ~ ~ ~ ~ ~ ~ ~ \textnormal{be.none}=2pl=imp.m\\
    \glt `Don't eat even a little bit of its fat!' \citep[141]{hollow1973b}

    \item\label{Ch5ExImperativeNegativesG} \glll Mashéhųpe \textbf{káare} ínupha rústa!\\
    wą-she\#hųp=E kaare i-rųp\#ha rut=ta\\
    unsp-\textnormal{be.red}\#\textnormal{moccasin}=sv imp.neg pv.ord-\textnormal{two}\#\textnormal{times} \textnormal{eat}=ta\\
    \glt `Don't eat cornmush a second time!' \citep[176]{hollow1973b}
    
    \end{xlist}
    
\end{exe}

In each of the examples in (\ref{Ch5ExImperativeNegatives}) above, \textit{káare} appears at various locations before the verb bearing an imperative allocutive marker, =\textit{na} or =\textit{ta}. The particle may appear sentence-initially, before the direct object or after the direct object. The one location where \textit{káare} never appears is after the imperative, meaning that this item is not subject to right dislocation or postposing as a clarifying topic.

The above data deals with verbal negation. Another kind of negation that occurs in Mandan is one that affects nouns that I am calling existential negation. Existential negation is most accomplished through the use of the verb \textit{mik} `be none.' This verb is takes a noun as its complement, though these nouns are typically nominalized clauses. These nominalized clauses usually bear the nominalizer \textit{waa}- and have the stem vowel =\textit{e}. We can see instances of this treatment of nominal negation in (\ref{Ch5ExNegation2}) below.

\begin{exe}
    \item\label{Ch5ExNegation2} Examples of nominal negation

    \begin{xlist}
            \item\label{Ch5ExNegation2A} \glll Wáa'eepe \textbf{mikó'sh}.\\
            waa-ee-pE wįk=o'sh\\
            nom-pv-\textnormal{say}.1a=sv \textnormal{be.none}=ind.m\\
            \glt `I said nothing.' \citep[41]{hollow1973a}

        \item\label{Ch5ExNegation2B} \glll Wáateehą nuhų́ųre \textbf{mikóote'sh}.\\
        waa-teehą rų-hųų=E wįk=ootE=o'sh\\
        nom-\textnormal{be.far.away} 1pl.poss-\textnormal{mother}=sv \textnormal{be.none}=evid=ind.m\\
        \glt `Our mother clearly has not been around for a long time.' \citep[145]{hollow1973a}

        \item\label{Ch5ExNegation2C} \glll máanuhe \textbf{mikó'sh}\\
        waa-rų-hE wįk=o'sh\\
        nom-1a.pl-\textnormal{see} \textnormal{be.none}=ind.m\\
        \glt `we (du.) saw nothing' \citep[186]{hollow1973b}

        \item\label{Ch5ExNegation2D} \glll máanurutirishe \textbf{miká}\\
        waa-rų-ru-trish=E wįk=E=$\varnothing$\\
        nom-1a.pl-ins.hand-\textnormal{shake}=sv \textnormal{be.none}=sv=cont\\
        \glt `we could not budge them at all' \citep[52]{hollow1973a}

        
        
    \end{xlist}
\end{exe}

In situations where a nominalized clause is used with the negative \textit{mik}, there is a sense of exhaustiveness conveyed in the nominal negation versus the verbal negation, which seems to convey a sense of discrete negation. For example, in (\ref{Ch5ExNegation2D}), the reading here is that not only were the characters expressing that they could not move the object, but there was simply no moving them in any way. Thus, nominal negation is often used in instances where a negative polarity item in English would be used, e.g., `nothing', `at all', etc.


\subsection{Benefactive constructions}\label{Ch5Benefactives}

Most Mandan verbs are maximally transitive. That is, most verbs take no more than two arguments: a subject and a direct object. Some verbs, namely those that bear an applicative preverb, are able to project a third argument and some of those verbs are able to take an indirect object, more specifically. However, if a speaker wishes to add an indirect object to a sentence when the verb itself does not intrinsically take one, then an indirect object can only be inserted into the syntax by way of a benefactive construction. Benefactive constructions involve the use of the verb \textit{kų́'} `give' with a lexical verb. In such constructions, \textit{kų́'} no longer has the semantics of `give', but serves to indicate that a third argument is being added to the syntax. These benefactive constructions rely on \textit{kų́'} acting as an auxiliary verb. We can see several examples of benefactive constructions in the data in (\ref{Ch5ExBenefactives}) below. We can see that instances of benefactive constructions that bear first or second person subject marking do so on both the main verb and the auxiliary, while the indirect object is only marked on the auxiliary.

\begin{exe}
    \item\label{Ch5ExBenefactives} Examples of benefactive constructions

    \begin{xlist}
        \item\label{Ch5ExBenefactivesA} \glll wóorut kéeka' kų́'kereroomako'sh\\
        wV-o-rut keeka' kų'=krE=oowąk=o'sh\\
        unsp-pv.irr-\textnormal{eat} \textnormal{keep} \textnormal{give}=3pl=narr=ind.m\\
        \glt `they kept food for him' \citep[109]{hollow1973a}

        \item\label{Ch5ExBenefactivesB} \glll Miníke, Pą́ąhį' Pa, kihké'roo makų́'ro'sh.\\
        wį-rįk=E pąąhį' pa kihke'=oo wą-kų'=o'sh\\
        1poss-\textnormal{son}=sv \textnormal{porcupine} \textnormal{head} \textnormal{dig.up}=dem.mid 1s-\textnormal{give}=ind.m\\\
        \glt `My son, Porcupine Head, dug them up for me.' \citep[55]{hollow1973a}

        \item\label{Ch5ExBenefactivesC} \glll Ų́'shka wahará minikų́'kto'sh.\\
        ų'sh=ka wa-hrE w-rį-kų'=kt=o'sh\\
        \textnormal{be.thus}=hab 1a-caus 1a-2s-\textnormal{give}=pot=ind.m\\\
        \glt `I will do it that way for you.' \citep[138]{hollow1973a}

        \item\label{Ch5ExBenefactivesD} \glll Mató Psí téehere rokų́'roomako'sh.\\
        wąto psi tee\#hrE ro-kų'=oowąk=o'sh\\
        \textnormal{bear} \textnormal{be.black} \textnormal{die}\#caus 1s.pl-\textnormal{give}=narr=ind.m\\
        \glt `He killed Black Bear for us.' \citep[160]{hollow1973b}
    \end{xlist}
\end{exe}

One peculiarity in the benefactive constructions seen in the corpus is that nearly half of all instances of benefactive constructions with a third person plural benefactive recipient do not use \textit{kų́'} as an auxiliary. Instead, those third person plural benefactive recipients employ \textit{káhere} instead. We can see examples of benefactives with \textit{káhere} in (\ref{Ch5ExBenefactives2}) below.

It is possible that there was once a more regularized conjugation where singular arguments took one benefactive auxiliary, while plural ones took another. We see that Hidatsa features a remarkably similar split between singular and plural recipients. The language attrition that took place after the smallpox epidemics in the early 1800s killed most L1 speakers could have caused a once-productive feature of the grammar to become regularized by subsequent speakers.\footnote{\citet[541]{park2012} describes benefactive constructions in Hidatsa, where \textit{gú'} `give', a clear cognate with Mandan \textit{kų́'}, is used when the benefactive entity is singular. There is a suppletive plural benefactive form \textit{gahée} in Hidatsa, which itself also appears to be a clear cognate with Mandan \textit{káhere}, given that the \textit{hée} in Hidatsa is the direct causative and \textit{heré} is the Mandan causative. The \textit{ga}- element in Hidatsa does not appear to have any synchronic place in the lexicon as a formative on its own, but \textit{ká} in Mandan is used as a pro-verb that serves to connect one sentence to the preceding one, as we see in \textit{káni} `and then.'}

\begin{exe}
    \item\label{Ch5ExBenefactives2} Examples of third person plural benefactive constructions

    \begin{xlist}
        \item\label{Ch5ExBenefactives2A} \glll míihere ré wóo'ipke ~ ~ ~ ~ ~ ~ ~ ~ ~ ókįhara káharani...\\
        wįįh=re re wV-o-i-pke ~ ~ ~ ~ ~ ~ ~ ~ ~ o-kį\#hrE=$\varnothing$ ka\#hrE=rį\\
        \textnormal{woman}=dem.prox dem.prox unsp-pv.irr-pv.ins-\textnormal{taste} ~ ~ ~ ~ ~ ~ ~ ~ ~ pv.loc-\textnormal{be.cooking}\#caus=cont prov\#caus=ss\\
        \glt `this woman here made dried meat for them and...' \citep[204]{hollow1973a}

        \item\label{Ch5ExBenefactives2B} \glll numá'kseena manáktetaa í'aakanake ~ ~ ~ ~ ~ réehara káhereroomako'sh\\
        ruwą'k=s=ee=rą wrąkte=taa i-aaki\#rąk=E ~ ~ ~ ~ ~ rEEh\#hrE=$\varnothing$ ka\#hrE=oowąk=o'sh\\
        \textnormal{man}=def=dem.dist=top \textnormal{altar}=loc pv.dir-\textnormal{above}\#pos.sit=sv ~ ~ ~ ~ ~ \textnormal{go.there}\#caus prov\#caus=narr=ind.m\\
        \glt `the man put a sitting place at the altar for them' \citep[174]{hollow1973b}

        \item\label{Ch5ExBenefactives2C} \glll rushínashhąą k\'{ı̨}įka káharata!\\
        ru-shi=rąsh=hąą kįįk=E=$\varnothing$  ka\#hrE=ta\\
        ins.hand-\textnormal{be.good}=att=ins \textnormal{finish}=sv=$\varnothing$ prov\#caus=imp.m\\
        \glt `let it be completed perfectly for them!' \citep[262]{trechter2012}
    \end{xlist}
\end{exe}

All Mandan speakers in the twentieth and twenty-first centuries were also fluent Hidatsa speakers, and the fact that Hidatsa also features cognate stems for singular versus plural benefactive recipients makes it difficult to assertation whether this similarity is an inherited feature of these languages or caused by language contact. Hidatsa's closest relative, Crow, does not feature different auxiliary stems for benefactive constructions \citep[145]{graczyk2007}. Therefore, the origin of this singular versus plural recipient stem for benefactives remains up for debate.

\subsection{Topicalization}\label{Ch5TopicAndFocus}
\largerpage
A pervasive syntactic process in Mandan is the heavy use of topicalization to emphasize some nominal element. Topicalization involves moving a nominal construction from its default position to either the beginning of the sentence or end of the sentence. Furthermore, Mandan frequently appeals to left dislocation and right dislocation, where some nominal element within the clause is reiterated or made more explicit at an edge of the utterance. The most common tactic of putting emphasis on a nominal construction in Mandan involves moving an element to the leftmost position within a sentence, i.e., topicalization.

As discussed previously in \sectref{SubSecTopics}, all semantic roles that a nominal element can play in a proposition can be morphologically marked as a topic with either =\textit{na} or =\textit{nu}. However, not all elements marked with topic morphology are topicalized. Nominal constructions that are not subjects but appear sentence-initially are able to appear there because these non-subjects are topicalized. It is likely that subjects can also be topicalized, as \citet{kasak2022} points out that there are different intonational patterns for topicalized elements versus focused elements in spoken Mandan, but an extensive survey of Mandan intonation patterns has not been conducted and is outside the scope of the present work.

Examples of topicalized structures appear in the data below, where the initial element has been preposed at the beginning of the sentence, outside the typical word order that has been described above in \sectref{Ch5Clauses}. In (\ref{Ch5ExTopicalizedC}), we see a \textit{wh}-word topicalized: \textit{tewéteroo} `where around here.' The subject--object--word order does not change in Mandan; statements and questions all employ this default word order. Likewise, unlike languages like English, \textit{wh}-words are not topicalized in Mandan. All question words appear canonically \textit{in situ} within the syntax. In (\ref{Ch5ExTopicalizedC}), the default expected position of the \textit{wh}-word is after the subject, \textit{ptamáah tóops} `my four arrows.' However, given that the speaker is highlighting the topicality of the question over the subject, the \textit{wh}-word appears in sentence-initial position.

The examples in (\ref{Ch5ExTopicalizedA}) and (\ref{Ch5ExTopicalizedB}) below are more obvious in how they flout standard word order. In (\ref{Ch5ExTopicalizedA}), for instance, this sentences involves a direct object that precedes all other adjunct material that would normally appear either before or after a subject. The direct object should appear immediately before the verb, yet in this example, it is the first element in the entire sentence. This irregular word order can only be due to topicalization. Likewise, in (\ref{Ch5ExTopicalizedB}), the indirect object appears in sentence-initial position. The adjunct element \textit{tashkák} `how' should appear before the indirect object, since it should either precede or follow where the subject normally appears. The absence of an overt subject should render this word sentence-initial, but that is not what we see here. Instead, the topicalized indirect object appears in sentence-initial position.

\begin{exe}
    \item\label{Ch5ExTopicalized} Examples of topicalized elements

    \begin{xlist}
        \item\label{Ch5ExTopicalizedA} Direct object

        \glll \textbf{Tawáarukiriihere} róo hú'taa ~ ~ ~ ~ ~ ~ ~ ~ wá'pąxoomako'sh\\
        ta-waa-ru-kriih\#hrE roo hu'=taa ~ ~ ~ ~ ~ ~ ~ ~   wa'-pąx=oowąk=o'sh\\
        al-nom-ins.hand-\textnormal{be.smooth}\#caus dem.mid \textnormal{be.near}=loc ~ ~ ~ ~ ~ ~ ~ ~ ins.prce-\textnormal{be.upright}=narr=ind.m\\
        \glt `\textbf{His staff} he stood up here by him.' \citep[1]{hollow1973a}

        \item\label{Ch5ExTopicalizedB} Indirect object

        \glll \textbf{Numá'kaaki} tashkák wáa'ąąwe rakų́'kina'ka'na?\\
        ruwą'k-aaki tashka=ak waa-ąąwe ra-kų'\#kirą'=ka=o'rą\\
        \textnormal{person}-coll \textnormal{how}=ds nom-\textnormal{all} 2a-\textnormal{give}\#\textnormal{tell}=hab=int.f\\
        \glt `\textbf{To people} why do you always tell everything?' \citep[213]{hollow1973a}

        \item\label{Ch5ExTopicalizedC} Location

        \glll \textbf{Tewéteroo} ptamáah tóops ó'ro'sha?\\
        t-we=t=roo p-ta-wąąh toop=s o'=o'sha\\
        wh-indf=loc=dem.mid 1sg.poss-al-\textnormal{arrow} \textnormal{four}=def \textnormal{be}=int.m\\
        \glt `\textbf{Where around here} are my four arrows?' \citep[33]{hollow1973a}
    \end{xlist}
\end{exe}

Another common manifestation of emphasizing some topic in Mandan goes beyond just encliticizing =\textit{na} or =\textit{nu} onto a nominal construction. The fact that Mandan speakers often omit most -- if not all -- explicit nominal constructions when speaking in the narrative means that listeners are forced to either infer what nominal elements the speaker is referring to or to recall what arguments have already been introduced into the discourse. Thus, mentioning an overt argument can confer great importance onto said nominal construction. When mentioning an overt argument that is also a topic that the speaker wishes to emphasize, a Mandan speaker often does so through dislocation of some kind.

A left dislocation introduces an argument at the left edge of the sentence that is present elsewhere in the sentence. In many instances, it is difficult to distinguish left dislocation from topicalization in Mandan due to the fact that overt nominals are often omitted. Therefore, it is unclear whether the element appearing at the left edge of the sentence is merely being topicalized or if there is a more covert left dislocation present, where a nominal in the sentence is being omitted but is still present in the syntax in and referenced by the left dislocated element. We can see an obvious instance of left dislocation in example (\ref{Ch5ExLeftDislocation}) below, where the distal demonstrative \textit{ée} is the subject of the verb, while the more elaborate nominal construction \textit{kowóoroo xíkanashs} `that no-good husband of hers' is a constituent at the left edge of the sentence that is the same entity as \textit{ée} `he, that one.'

\begin{exe}

    \item\label{Ch5ExLeftDislocation} Example of left dislocation

        \glll \textbf{Kowóoroo} \textbf{xíkanashs}, \uline{ée} wáa'o'nix ishí're.\\
        ko-wooroo xik=rąsh=s ee waa-o'-rįx ishi=o're\\
        3poss.pers-\textnormal{husband} dem.dist neg-\textnormal{be}=neg vis=ind.f\\
        \glt `\textbf{That no-good husband of hers}, it must not be \uline{him}.' \citep[134]{hollow1973a}

        
\end{exe}

Right dislocation in Mandan is just as pervasive in the corpus as left dislocation, where an argument is brought up in a clause, only to have some elaborating or clarifying element added to the end of the clause that refers back to an established argument. Given that Mandan is a topic-prominent language, deviations from the standard subject--object--verb word order do not affect the illocutionary force of a statement, as we see in languages like English, e.g., `They have gone home' is a statement versus `Have they gone home?' is a question, where the same syntactic elements are reordered to encode for illocutionary force. Instead, Mandan word order is altered by the weight of topic-driven pragmatics. Placing a right dislocated nominal at the end of a sentence serves to reinforce who or what the speaker wishes to emphasize. 

We see an example of right dislocation in the data below in (\ref{Ch5ExRightDislocation}), where the right dislocation is an entire postpositional phrase. In the main clause, the postpositional phrase \textit{óo ó'harani} `from there' provides the origin of buffalo. The speaker adds the right dislocated postpositional phrase \textit{máa'ąk kú'sht ó'harani} `from underneath the earth' to expound upon the nominal element that is already present in the syntax. The right dislocated phrase appears in bold, while the constituent in the main clause that it is referencing is underlined.

\begin{exe}
    \item\label{Ch5ExRightDislocation} Example of right dislocation

            \glll \uline{óo} \uline{ó'harani} pt\'{ı̨}įtkushkeres ą́ąwe  ~ ~ ~ ~ ~ ~  ~ ~ ~ ~ ~ ~ súhkereroomako'sh, \textbf{máa'ąk} \textbf{kú'sht} \textbf{ó'harani}.\\
        oo o'\#hrE=rį ptįį\#tkush=krE=s ąąwe  ~ ~ ~ ~ ~ ~  ~ ~ ~ ~ ~ ~ suk=krE=oowąk=o'sh waa'ąk ku'sh=t o'\#hrE=rį\\
        dem.mid \textnormal{be}\#caus=ss \textnormal{buffalo}\#\textnormal{be.true}=3pl=def \textnormal{all}  ~ ~ ~ ~ ~ ~  ~ ~ ~ ~ ~ ~ \textnormal{exit}=3pl=narr=ind.m \textnormal{earth} \textnormal{inside}=loc \textnormal{be}\#caus=ss\\
        \glt `From there, all the buffalo came \uline{from there}, \textbf{from underneath the earth}.' \citep[114]{hollow1973b}
\end{exe}

Multiple nominal entities can be topicalized and dislocated in Mandan. Topicalization and dislocation as syntactic processes do not always overlap with the use of topic-marking morphology like =\textit{na} and =\textit{nu}, but there are instances where the speaker wishes to highlight the importance of some element and it is both topicalized and dislocated. In the example in (\ref{Ch5ExDoubleTopics}) below, the subject of the matrix verb \textit{íko'tseena} `that father of hers' is represented as topicalized with the topic marker =\textit{na}, but also as the right dislocated \textit{kó'tseena} `her father' at the end of the sentence.

\begin{exe}
    \item\label{Ch5ExDoubleTopics} Topicalized subject and right dislocated subject
    
    \glll \textbf{íko'tseena} ``mí'shak, ~ ~ ~ ~ ~ ~ ~ maní'o'na ą́'skarahara'shka éewaharani  ~ ~ ~ ~  ~ ~ ~ ~ minikímaxani,'' éeheeroomako'sh, ~ ~ ~ ~ ~ ~ ~ ~ ~ ~ ~ ~ ~ ~ ~ ~ ~ ~ ~ ~ \textbf{kó'tseena}\\
    \textbf{i-ko-at=s=ee=rą} w'\~~-ishak ~ ~ ~ ~ ~ ~ ~ wa-rį-o'=rą ą'ska\#ra-hrE=ą'shka ee-wa-hrE=rį  ~ ~ ~ ~ ~ ~  ~ ~  w-rį-kiwąxE=rį ee-hEE=oowąk=o'sh ~ ~ ~ ~ ~ ~ ~ ~ ~ ~ ~ ~ ~ ~ ~ ~ ~ ~ ~ ~ \textbf{ko-at=s=ee=rą}\\
    \textbf{pv.poss-3poss.pers-\textnormal{\bfseries father}=def=dem.dist=top} 1poss-\textnormal{pro} ~ ~ ~ ~ ~ ~ ~ unsp-2s-\textnormal{be}=top \textnormal{be.that.way}\#2a-caus=psbl pv-1a-caus=ss  ~ ~ ~ ~ ~ ~  ~ ~  1a-2s-\textnormal{ask}=ss pv-\textnormal{say}=narr=ind.m ~ ~ ~ ~ ~ ~ ~ ~ ~ ~ ~ ~ ~ ~ ~ ~ ~ ~ ~ ~ \textbf{3poss.pers-\textnormal{\bfseries father}=def=dem.dist=top}
\\
    \glt `\textbf{that father of hers} said, ``me, I thought that you were the one who maybe did something so I asked you,'' \textbf{her father} did.' \citep[238]{hollow1973b}
\end{exe}

Obvious examples of double cleft constructions like topicalization or dislocation are not common in the corpus, but they do occur. It is especially likely to occur when a speaker is using a direct quote or speaking freely instead of speaking as the narrator of a traditional narrative.

\subsection{Illocuationary force and allocutive agreement}\label{Ch5IllocutionaryForce}

Word order in Mandan is subject--object--verb as a default, but the syntax can be flexible to take the topicality of an element into account. In Mandan, word order does not convey any aspect of the intention of the utterance. The ordering of words is the same in a statement, a question, or a command for Mandan speakers.

Instead of encoding the illocutionary force of a speech act through the syntax, Mandan instead encodes the intent of the speaker through allocutive agreement markers. These markers were previously discussed in \sectref{Ch3SubSubSecAllocutive}, but are reproduced in \tabref{allocutivetable2} below.

\begin{table}
\caption{Mandan allocutive agreement markers} 
\label{allocutivetable2}

	\begin{tabular}{llll}
	\lsptoprule
		~		&
			Indicative		&
			Interrogative	&	
			Imperative\\
			\midrule
		Male	&	
			=\textit{o'sh}	&	
			=\textit{o'sha}	&	
			=\textit{ta}\\
		Female/Non-male	&	
			=\textit{o're}	&	
			=\textit{o'na}	&	
			=\textit{na}\\
			\lspbottomrule
		\end{tabular}

\end{table}

These allocutive agreement markers are exclusively found on the matrix verb within an utterance. Mandan is similar to most other Siouan languages in this respect, as Lakota \citep{trechter1995}, Biloxi \citep{dorseyswanton1912}, Chiwere \citep{greer2016}, \textit{inter alios} all feature allocutive agreement markers on matrix verbs. Mandan does differ from other Siouan languages, though, in that the Mandan language exclusively marks allocutive agreement with the gender of the addressee; the gender of the speaker is not reflected in this kind of utterance-level morphology. We can see examples of different allocutive agreement markers below and notice that word order is unaffected by the illocutionary intent of the speaker. The only major change between the three examples in (\ref{Ch5WordOrderIllocutionaryForce}) is the allocutive agreement marker at the end of the sentence.

\begin{exe}
    \item\label{Ch5WordOrderIllocutionaryForce} Examples of consistent word order

    \begin{xlist}
        \item\label{Ch5WordOrderIllocutionaryForceA} Statements

        \glll Súks weréhe pataráko'sh $\sim$ pataráko're.\\
        suk=s wreh=E pa-trak=o'sh ~ pa-trak=o're\\
        \textnormal{child}=def \textnormal{door}=sv ins.frce-\textnormal{shut}=ind.m ~ ins.frce-\textnormal{shut}=ind.f\\
        \glt `The child shut the door.'

        \item\label{Ch5WordOrderIllocutionaryForceB} Questions

        \glll Súks weréhe pataráko'sha $\sim$ pataráko'na?\\
        suk=s wreh=E pa-trak=o'sha ~ pa-trak=o'rą\\
        \textnormal{child}=def \textnormal{door}=sv ins.frce-\textnormal{shut}=int.m ~ ins.frce-\textnormal{shut}=int.f\\
        \glt `Did the child shut the door?'

        \item\label{Ch5WordOrderIllocutionaryForceC} Commands

        \glll Súke, weréhe patarákta $\sim$ patarákana!\\
        suk=E wreh=E pa-trak=ta ~ pa-trak=rą\\
        \textnormal{child}=sv \textnormal{door}=sv ins.frce-\textnormal{shut}=imp.m ~ ins.frce-\textnormal{shut}=imp.f\\
        \glt `Child, shut the door!'

    \end{xlist}
\end{exe}

Deviations from this subject--object--verb word order are explicitly due to the speaker wishing to affect the information structure of a given utterance. Topicalization or dislocation are frequent reasons for not adhering to the default word order, and it should be assumed that sentences whose word order falls outside this norm have divergent word orders due to these two processed, as described in \sectref{Ch5TopicAndFocus}.

\section{Multiclausal structures}\label{Ch5Multiclausal}

\citet{carmodykasak2023} find that even short narratives in Mandan typically involve sentences with more than three clauses. The corpus is dominated by multiclausal structures, and it is not uncommon for a single sentence to contain ten or more clauses before reaching the matrix verb. A complete sentence in Mandan must have an allocutive agreement marker to signal to the listener that the utterance is finished, but there are a variety of other constructions that can appear in a Mandan sentence before reaching the end of the utterance.


\subsection{Switch-reference}\label{Ch5switchreference}

One of the most prolific interclausal structures in Mandan is that of switch-reference. Switch-reference is a system whereby there is some formative that serves to signal that the subject of one clause either shares or has a different subject than an adjascent clause \citep[ix]{haimanmunro1983}.\footnote{This definition holds for canonical switch-reference, versus non-canonical switch-reference, which marks clauses as having some kind of topic pivot or not. I assume that switch-reference in Mandan is canonical, due to the extreme predictability of these formatives tracking the subject of each clause.} Mandan is not unique in Siouan in having a system of switch-reference, as it has been described at length in Hidatsa \citep{boyle2007,boyle2011}, Crow \citep{graczyk2007}, and Biloxi \citep{graczyk1997,torres2010}.

\citet{kasak2019} describes the distribution of switch-reference markers in Mandan, and has done so here in \sectref{SecSwitchReference}. There are two switch-reference markers that are identified in (\ref{switchreferencemarkersinMandanRedux}).



\begin{exe}

\item\label{switchreferencemarkersinMandanRedux} Switch-reference markers in Mandan

\begin{tabular}{lll}
	
	{=\textit{ak}}& /=ak/ & different-subject switch-reference marker\\
	{=\textit{ni}}&/=rį/	&	same-subject switch-reference marker\\

\end{tabular}

\end{exe}

Switch-reference markers are commonly translated as conjunctions like `and' into English by Mandan speakers throughout the corpus. \citet{kasak2019} argues that these structures are not actually instances of coordination, however, but are clausal adjuncts. The primary motivation behind this analysis is that true coordination would mean that verbs with switch-reference marking should have a full range of number, aspect, mood, and evidentiality enclitics, but we only see such enclitics on the matrix verb of a sentence featuring a verb with switch-reference marking. Thus, verbs marked with switch-reference are inflectionally impoverished and are not finite verbs in the way that matrix verbs are. 

2Verbs with switch-reference marking will maximally carry subject and object prefixes, but will not have any corresponding number or other agreement enclitics. We can see examples of same-subject switch-reference constructions in (\ref{Ch5ExSameSubjectSR}) below with the clauses marked for switch-reference shown in bold, but different-subject switch-reference constructions are underlined. In (\ref{Ch5ExSameSubjectSRA}) and (\ref{Ch5ExSameSubjectSRB}), the same subject is engaging in all the actions that are part of the sentence, which means that all such non-matrix verbs end in =\textit{ni}. Conversely, in (\ref{Ch5ExSameSubjectSCC}), each clause has a different subject than the one proceeding it. To make the translations easier to follow in terms of coindexation, subscript letters appear next to the subject in question to ensure that the motivation for switching to the different-subject switch-reference marker =\textit{ak} is needed.

\begin{exe}
    \item\label{Ch5ExSameSubjectSR} Examples of same-subject switch-reference

    \begin{xlist}
        \item\label{Ch5ExSameSubjectSRA} \glll Miní \textbf{kih\'{ı̨}įni} pt\'{ı̨}į koníhkas ~ ~ ~ ~ ~ ~ ~ ~ ~ ~ \textbf{pakanąhini} \textbf{kirútini} náakeroomako'sh.\\
        wrį ki-hįį=rį ptįį ko-rįk=ka=s ~ ~ ~ ~ ~ ~ ~ ~ ~ ~ pa-krąh=rį ki-rut=rį rąąkE=oowąk=o'sh\\
        \textnormal{water} suus-\textnormal{drink}=ss \textnormal{buffalo} rel-\textnormal{be.little}=hab=def ~ ~ ~ ~ ~ ~ ~ ~ ~ ~ ins.push-\textnormal{butcher}=ss suus-\textnormal{eat}=ss \textnormal{sit}.aux=narr=ind.m\\
        \glt `He\textsubscript{i} sat down and [he\textsubscript{i}] drank the water, [he\textsubscript{i}] butchered the calf, and [he\textsubscript{i}] ate it.' \citep[212]{hollow1973a}

        \item\label{Ch5ExSameSubjectSRB} \glll Numá'k Máxana \textbf{ų́ųpani}, ípashahąkt ~ ~ ~ ~ ~ ~ ~ ~ ~ ~ ráaha, miníkere, miníxuxtekere ~ ~ ~ ~ ~ ~ ~ ~ ~ ~ \textbf{ísekini} íkų'hąą ísekoomaksįh\\
        ruwą'k wąxrą ųųpa=rį i-pasha\#hąk=t ~ ~ ~ ~ ~ ~ ~ ~ ~ ~ rEEh=E=$\varnothing$ wrį=krE wrį\#xuu-xtE=krE ~ ~ ~ ~ ~ ~ ~ ~ ~ ~ i-sek=rį i-kų'=hąą i-sek=oowąk=sįh\\
        \textnormal{man} \textnormal{one} \textnormal{with}=ss pv.dir-\textnormal{north}\#pos.stnd=loc ~ ~ ~ ~ ~ ~ ~ ~ ~ ~ \textnormal{go.there}=sv=cont \textnormal{water}=3pl \textnormal{water}\#\textnormal{be.shallow}-aug=3pl ~ ~ ~ ~ ~ ~ ~ ~ ~ ~ pv.ins-\textnormal{make}=ss pv.dir-\textnormal{be.yonder}=ins pv.ins-\textnormal{make}=narr=ints\\
        \glt `Lone Man\textsubscript{i} was with him and, going to the north, he\textsubscript{i} made lakes, large shallow lakes, and he\textsubscript{i} made it all over.' \citep[11]{hollow1973a}

        \item\label{Ch5ExSameSubjectSCC} \glll Míihs \uline{ratóoniitek}, téehį kų́'hs ~ ~ ~ ~ ~ ~ koxamáhs wáaka're \uline{pkaminíshak}, ~ ~ wáa'oka're \uline{óksuuherek}, \uline{ptą́ąrak} ~ ~ ~ ą́ąwe xką́hkereroomako'sh.\\
        wįįh=s ratoo=rįįtE=ak teehį k'-ųųh=s ~ ~ ~ ~ ~ ~ ko-xwąh=s waa-ka'=E k-pa-wrįsh=ak ~ ~ waa-o-ka'=E o-k-suu\#hrE=ak ptąą=ak ~ ~ ~ ąąwe xkąh=krE=oowąk=o'sh\\
        \textnormal{woman}=def \textnormal{be.old}=cel=ds \textnormal{so.then} 3poss.pers-\textnormal{wife}=def ~ ~ ~ ~ ~ ~ rel-\textnormal{be.small}=def nom-\textnormal{possess}=sv suus-ins.push-\textnormal{be.folded.up}=ds ~ ~ nom-pv.irr-\textnormal{possess}=sv pv.loc-suus-\textnormal{be.filled}\#caus=def \textnormal{fall}=ds ~ ~ ~ \textnormal{all} \textnormal{move}=3pl=narr=ind.m\\
        \glt `The woman\textsubscript{i} was a bit older, so then the younger wife\textsubscript{j} packed her belongings, she\textsubscript{i} put her belongings in there, and fall\textsubscript{k} came, then they\textsubscript{i+j} all moved.' \citep[13]{trechter2012b}
        



        \item\label{Ch5ExSameSubjectSRD} \glll Manároo mashkáshkapkas rúta ~ ~ ~ ~ ~ ~ ~ ~ ~  hą́ąka \textbf{éexohiharani} \uline{réehak} ų́ųshe \uline{shirúurak} ~ ~ ~ ~ ~ kirúkaxa hą́hka'sh.\\
        wrą=oo wą-shka$\sim$shkap=ka=s rut=E=$\varnothing$ ~ ~ ~ ~ ~ ~ ~ ~ ~ hąąkE=$\varnothing$ eexi\#o-hi\#hrE=rį rEEh=ak ųųsh=E shruu=ak ~ ~ ~ ~ ~ ki-ru-kax=E hąk=ka=o'sh\\
        \textnormal{tree}=dem.mid unsp-aug-\textnormal{prick}=hab=def \textnormal{eat}=sv=cont ~ ~ ~ ~ ~ ~ ~ ~ ~ \textnormal{stand}.aux=cont \textnormal{belly}\#pv.loc-\textnormal{be.full}\#caus=ss \textnormal{go.there}=ds \textnormal{anus}=sv \textnormal{be.itchy}=ds ~ ~ ~ ~ ~ suus-ins.hand-\textnormal{scratch}=sv pos.stnd=hab=ind.m\\
        \glt `In the woods, he\textsubscript{i} was eating tomatoes, [he\textsubscript{i} was] filling his belly and as he\textsubscript{i} went along, his anus\textsubscript{j} itched, so he\textsubscript{i} keep on scratching it.' \citep[14]{hollow1973a}

        \item\label{Ch5ExSameSubjectSRE} \glll \uline{Óshikerek} \textbf{kawéhini} wakí're ~ ~ ~ ~ \uline{kaskék} xtéxteroomako'sh.\\
        o-shi=krE=ak ka-weh=rį wa-ki'=E ~ ~ ~ ~ ka-skE=ak xtE-xtE=oowąk=o'sh\\
        pv.irr-\textnormal{be.good}=3pl=ds ins.frce-\textnormal{pick}=ss unsp-\textnormal{pack.on.back}=sv ~ ~ ~ ~ ins.frce-\textnormal{tie}=ss \textnormal{be.big}-aug=narr=ind.m\\
        \glt `They\textsubscript{i} were good, so he\textsubscript{j} picked them up and he\textsubscript{j} tied them in packs, then it\textsubscript{k} [the bunch of packs] was really big.' \citep[216]{hollow1973a}
    \end{xlist}
\end{exe}

We can see in (\ref{Ch5ExSameSubjectSRD}) and (\ref{Ch5ExSameSubjectSRE}) above that a subject change warrants a change in switch-reference marking. For example, in (\ref{Ch5ExSameSubjectSRD}), we see that an unspoken person is the subject of the first three clauses. The actions of standing there eating, filling his belly, and going along were all done by the same subject. After that first subject begins to go along, the subject of the next clause changes to a body part, which starts to itch. The following clause is the first subject itching himself, so different-subject marking on \textit{shirúurak} `it is itchy' is required to indicate that the body part in question is no longer the subject.

Semantically, switch-reference in Mandan plays a major role in expressing that there is a connection between two clauses. Switch-reference can indicate that clauses are either contemporaneous or sequential, coordinated or disjoined. Context is key in understanding what the relationship is between a clause bearing switch-reference marking and the following clause. This flexibility in meaning makes switch-reference the most frequent means of expressing an interclausal relationship in the language.

\subsection{Serial verb constructions}\label{Ch5SerialVerb}

Despite the frequency of switch-reference in the corpus, not all interclausal relationships are expressed through switch-reference. Verb serialization in Mandan involves a sequence of two or more verbs within a clause that are treated as a single event. Such events serve as a singular predicate \citep{aikhenvald2018}.

Due to the fact that such constructions are part of a single event, they must all necessarily share the same subject. These serial verb constructions also signify a contemporaneous act or state with the other serialized verbs. Serial verb constructions trigger ablaut in the final vowel where applicable. Consonant-final stems often bear a stem vowel to make it clear that the verb is part of a serial verb construction. Stems that end in a long vowel or a consonant will take the stem vowel /=E/, whereupon ablaut can take place and signify that a verb is part of a serial verb construction. Vowel-final stems that do not end in /E/ or /EE/ are less obviously part of serial verb constructions, because the addition of /=E/ is deleted by the phonotactic constraint against placing two short vowels next to each other in the post-verbal field, as previously discuseed in \sectref{shortvoweldeletion}. Most typically, serial verbs constructions appear as a dyad, but there is no practical limit to the number of verbs that can be serialized.

We can see examples of serial verb constructions in the data in (\ref{Ch5ExSerialVerbs}) below, where serialized verbs are shown in bold.

\newpage

\begin{exe}
    \item\label{Ch5ExSerialVerbs} Examples of serial verb constructions

    \begin{xlist}
        \item\label{Ch5ExSerialVerbsA} \glll tawáarukiriihs \textbf{kahóora} ~ ~ ~ ~ ráahini nákak...\\
        ta-waa-ru-kriih=s ka-hoo=E=$\varnothing$ ~ ~ ~ ~ rEEh=rį rąk=ak\\
        al-nom-ins.hand-\textnormal{be.smooth}=def ins.frce-\textnormal{fall.down}=sv=cont ~ ~ ~ ~ \textnormal{go.there}=ss pos.sit=ds\\
        \glt `his staff went \textbf{dropping down} and it sat there...' \citep[1]{hollow1973a}

        \item\label{Ch5ExSerialVerbsB} \glll wiróke koratitaahąą \textbf{óraraxi} írakų'kto'sh\\
        wi-rok=E ko-rati=taa=hąą o-ra-ra-xi=$\varnothing$ i-ra-kų'=kt=o'sh\\
        1poss-\textnormal{leg}=sv rel-\textnormal{right.side}=loc=ins pv.irr-2a-ins.foot-\textnormal{kick}=cont pv.ins-2a-\textnormal{give}=pot=ind.m\\
        \glt `you will pretend \textbf{to kick} my leg on the right side' \citep[62]{hollow1973b}

        \item\label{Ch5ExSerialVerbsC} \glll máareksuks máaskaps \textbf{rúta} ~ ~ ~ ~ ~ ~ ~ ~ ~ ~ ~ ~ ~ ~ ~ ~ óshapoomako'sh\\
        wąąreksuk=s wąą-skap=s rut=E=$\varnothing$ ~ ~ ~ ~ ~ ~ ~ ~ ~ ~ ~ ~ ~ ~ ~ ~ o-shap=oowąk=o'sh\\
        \textnormal{bird}=def nom-\textnormal{be.wet}=def \textnormal{eat}=sv=cont ~ ~ ~ ~ ~ ~ ~ ~ ~ ~ ~ ~ ~ ~ ~ ~ pv.loc-\textnormal{be.lively}=narr=ind.m\\
        \glt `the bird was rapidly \textbf{eating} the meat' \citep[132]{hollow1973a}

        \item\label{Ch5ExSerialVerbsD} \glll Óo ó'harani \textbf{wará'taxa} ~ ~ ~ ~ téroomako'sh.\\
        oo o'\#hrE=rį wa-ra'-tax=E=$\varnothing$ ~ ~ ~ ~ tE=oowąk=o'sh\\
        dem.mid \textnormal{be}\#caus=ss unsp-ins.mth-\textnormal{make.loud.noise}=sv=cont ~ ~ ~ ~ \textnormal{stand}=narr=ind.m\\
        \glt `After that, he stood there \textbf{crying}.' \citep[270]{hollow1970}

    \end{xlist}
\end{exe}

Mandan can often encode the same meaning using either a serial verb construction or a switch-reference construction. We can see examples of the interaction between switch-reference constructions and serial verb constructions in (\ref{Ch5SerialVsSR}) below.

\newpage

\begin{exe}

    \item\label{Ch5SerialVsSR} Example of switch-reference and serial verbal construction with the same meaning

    \begin{xlist}
    \item\label{Ch5SerialVsSRA} \glll wawárutani manáake'sh\\
    wa-wa-rutE=rį wa-rąąkE=o'sh\\
    unsp-1a-\textnormal{eat}=ss 1a-\textnormal{sit}.pos.aux=ind.m\\
    \glt `I sat there eating.'

    \item\label{Ch5SerialVsSRB} \glll wawáruta manáake'sh\\
    wa-wa-rutE=$\varnothing$ wa-rąąkE=o'sh\\
    unsp-1a-\textnormal{eat}=cont 1a-\textnormal{sit}.pos.aux=ind.m\\
    \glt `I sat there eating.'

    \end{xlist}

\end{exe}

Mandan seems to share this characteristic with Hidatsa, where a serial verb construction and a same-subject switch-reference construction can be used interchangeably to express the same concept \citep[541]{park2012}. The prolific bilingualism of Mandan speakers over the past two hundred years has blurred whether this grammatical similarity between Mandan and Hidatsa is due to language contact or an inherited grammatical construction from an ancestor language. Furthermore, if language contact is the cause of this shared feature, it is not obvious if it originated in Mandan and spread to Hidatsa or if Hidatsa was the origin and it then spread to Mandan. It origin aside, it is not clear if this similarity with Hidatsa is actually superficial and these constructions are actually distinct. It is possible that there is a register difference or if the semantics or pragmatics are distinct enough to encode for a slightly different situation in a construction like (\ref{Ch5SerialVsSRA}) versus one in (\ref{Ch5SerialVsSRB}). 

\subsection{Auxiliary verb constructions}\label{Ch5AuxVerbs}

Benefactive constructions are special kind of auxiliary verb construction. The purpose of benefactives is to add an argument to a proposition. The other auxiliary verb constructions serve to provide aspectual semantics to a proposition. Specifically, Mandan makes frequent use of auxiliaries to convey progressive aspect or habitual aspect.

These auxiliaries were previously discussed in \sectref{syntacticallyconditionedablaut}, where they all collectively cause syntactially conditioned ablaut in the lexical verb. The list of auxiliaries is reproduced in (\ref{ablautsyntaxRedux}) below.

\newpage

\begin{exe}
\item\label{ablautsyntaxRedux} Ablaut-triggering auxiliary verbs

\begin{xlist}
\item /hąąkE/ \textit{hą́ąke} `standing' positional auxiliary verb
\item /rąąkE/ \textit{náake} `sitting' positional auxiliary verb
\item /rąąkah/ \textit{náakah} `sitting' habitual auxiliary verb
\item /ruurįh/ \textit{núunih} plural durational auxiliary verb
\item /wąąkE/ \textit{máake} `lying' positional auxiliary verb
\item /wąąkah/ \textit{máakah} `lying' habitual auxiliary verb
\end{xlist}

\end{exe}

The list of auxiliaries above are mostly derived from positionals, as previously discussed in \sectref{SubSubSecPositionals}. The diversity in the morphological realization of positionals can cause some confusion, given that many of the texts in the corpus do not distinguish between vowel length. Positional determiners are those with short vowels, and they are reduced versions of these lexical positional verbs. The morphological split reflects the semantic split that occurred at an earlier stage of language development in Pre-Mandan, where the Proto-Siouan *wų́ų-kE `lie', *rą́ą-kE `sit', and *hą́ą-kE `stand' underwent semantic bifurcation, where the literal, lexical semantics of `lie', `sit', and `stand' remained in Mandan, but there was also a shift to a more existential reading for all three verbs. This existential reading became associated with habitual or progressive acts, and as such, another semantic bifurcation took place, i.e., the creation of periphrastic aspect marking through the use of an auxiliary verb.

The positional auxiliaries \textit{hą́ąke}, \textit{máake}, and \textit{náake} can all be used to express an existential proposition. However, these existential meanings still have a progressive aspectual reading, i.e., these positionals are intrinsically progressive, while \textit{hą́k}, \textit{mák}, and \textit{nák} have non-specific aspect marking. While both \textit{hą́k} and \textit{hą́ąke} mean `stand', the former is used to express a discrete action, while the latter is used to express an activity that was in progress or that took place over a period of time. We can see examples of this existential use of these auxiliaries in (\ref{Ch5ExPositionalsExistential}) below.

\begin{exe}
    \item\label{Ch5ExPositionalsExistential} Examples of the existential use of positional auxiliaries

    \begin{xlist}

    \item\label{Ch5ExPositionalsExistentialA} \glll \textbf{Ną́akani} rá'hashoomako'sh.\\
    rąąkE=rį ra'-hash=oowąk=o'sh\\
    \textnormal{sit}.pos.aux=ss ins.heat-\textnormal{be.disintegrated}=narr=ind.m\\
    \glt `He was [sitting] there and he burned it up.' \citep[182]{hollow1973b}

    \item\label{Ch5ExPositionalsExistentialB} \glll kohų́ųre wóorut ísekini ~ ~ ~ ~ ~ ~ ~ ~ ~ \textbf{máakeki}, kí'hoomako'sh\\
    ko-hųų=E wV-o-rut i-sek=rį ~ ~ ~ ~ ~ ~ ~ ~ ~ wąąkE=ki ki'h=oowąk=o'sh\\
    3poss.pers-\textnormal{mother}=sv unsp-pv.irr-\textnormal{eat} pv.ins-\textnormal{make}=ss ~ ~ ~ ~ ~ ~ ~ ~ ~ \textnormal{lie}.pos.aux=cond \textnormal{arrive.back.there}=narr=ind.m\\
    \glt `he got back when his mother was [lying] there and she was making dinner' \citep[67]{hollow1973a}

    \item\label{Ch5ExPositionalsExistentialC} \glll \textbf{Hą́ąkenashki}, máxanana ``ómiko'sh. ~ ~ ~ ~ ~ ~ Wáahųnash téereherenashini ~ ~ ~ ~ ~ ~ ~ ~ ~ ~ ~ ~ ~ ~ ~ ~ ~ ~ wáa'orahąąhkaxi'sh.''\\
    hąąkE=rąsh=ki wąxrą=rą o-wįk=o'sh ~ ~ ~ ~ ~ ~ waa-hų=rąsh tee\#re-hrE=rąsh=rį ~ ~ ~ ~ ~ ~ ~ ~ ~ ~ ~ ~ ~ ~ ~ ~ ~ ~ waa-o-ra-hąąk=ka=xi=o'sh\\
    \textnormal{stand}.pos.aux=att=cond \textnormal{one}=top pv.irr-\textnormal{be.none}=ind.m ~ ~ ~ ~ ~ ~ nom-\textnormal{many}=att \textnormal{die}\#2a-caus=att=ss ~ ~ ~ ~ ~ ~ ~ ~ ~ ~ ~ ~ ~ ~ ~ ~ ~ ~ neg-pv.irr-2a-\textnormal{not.know}=hab=neg=ind.m\\
    \glt `When he was [standing] there, one of them was like, ``I don't think so. You aren't capable of killing that many.{''}' \citep[7]{trechter2012b}

    \end{xlist}
\end{exe}

When these positional auxiliaries are used in conjunction with another verb, they result in a progressive reading of the proposition. As previously stated, these constructions trigger ablaut in the final vowel of the lexical verb. If the verb stem ends in a short vowel, no ablaut occurs due to the fact that only /E/ and /EE/ can undergo ablaut. Furthermore, if we assume that a stem vowel /=E/ is encliticized onto the stem, hiatus resolution rules will delete the second short vowel in a sequent of two short vowels, eliminating the /=E/ and preserving the final short vowel of the stem. Therefore, ablaut most often occurs when the stem ends in a consonant or a long vowel, which allows the /=E/ to manifest without syncope and be subject to ablaut. 

In (\ref{Ch5ProgressivePositionals}), we see that the verb immediately preceding the positional auxiliary gets progressive aspect. However, this auxiliary is able to take scope over multiple predicates, as we see in (\ref{Ch5ProgressivePositionalsB}), where the positional auxiliary \textit{hą́ąka} bestows progressive aspect not only on the immediately preceding \textit{sí} `travel', but also on \textit{íxatanashini} `look over' in the previous clause. Thus, in same-subject switch-reference constructions, multiple verbs can receive progressive aspectual semantics through this construction. The overwhelming majority of instances of positional auxiliaries being used in the corpus involve a single verb, however.

We can see further examples of these progressive constructions with positional auxiliaries in the data below in (\ref{Ch5ProgressivePositionals}). The elements that bear progressive semantics are marked in bold.

\begin{exe}
    \item\label{Ch5ProgressivePositionals} Examples of progressive constructions with positional auxiliaries

    \begin{xlist}
        \item\label{Ch5ProgressivePositionalsA} \glll máareksuks katą́ąhini \textbf{rúta} \textbf{náakek}\\
        wąąreksuk=s ka-tąąh=rį rutE rąąkE=ak\\
        \textnormal{bird}=def ins.frce-\textnormal{pound}=ss \textnormal{eat} \textnormal{sit}.pos.aux=ds\\
        \glt `he pounded the birds and was there eating' \citep[146]{hollow1973a}

        \item\label{Ch5ProgressivePositionalsB} \glll máa'ąk ų́ųpat \textbf{íxatanashini} \textbf{sí} ~ ~ ~ ~ ~ ~ ~ ~ ~ ~ ~ ~ ~ ~ ~ \textbf{hą́ąka} inák kúhoomaksįh\\
        waa'ąk ųųpat i-xat=rąsh=rį si ~ ~ ~ ~ ~ ~ ~ ~ ~ ~ ~ ~ ~ ~ ~ hąąkE=$\varnothing$ irąk kuh=oowąk=o'sh\\
        \textnormal{land} \textnormal{be.different} pv.dir-\textnormal{look.at}=att=ss \textnormal{travel} ~ ~ ~ ~ ~ ~ ~ ~ ~ ~ ~ ~ ~ ~ ~ \textnormal{stand}.pos.aux=cont \textnormal{again} \textnormal{come.back.here}=narr=ind.m\\
        \glt `he was looking over different lands and traveling around, and then he came back again' \citep[8]{hollow1973a}

        \item\label{Ch5ProgressivePositionalsC} \glll Káare \textbf{rá'taxa} \textbf{hą́ąkata}!\\
        kaare ra'-tax=E hąąkE=ta\\
        imp.neg ins.heat-\textnormal{make.loud.noise}=sv \textnormal{stand}.pos.aux=imp.m\\
        \glt `Don't be crying!' \citep[322]{hollow1973b}

        \item\label{Ch5ProgressivePositionalsD} \glll Káni míihs \textbf{wíihara} \textbf{náakeroomako'sh}.\\
        ka=rį wįįh=s wV-i-hrE rąąkE=oowąk=o'sh\\
        prov=ss \textnormal{woman}=def nom-pv.ins-caus \textnormal{sit}.pos.aux=narr=ind.m\\
        \glt `And then, the woman was doing quillwork.' \citep[46]{hollow1973b}

    \end{xlist}
\end{exe}

There is a variant on these positional auxiliaries that carries habitual aspectual semantics, i.e., \textit{hą́ąkah}, \textit{máakah}, and \textit{náakah}. These auxiliaries appear to be derived from the base 
positional verbs, and the final /E/ in each has undergone ablaut with the addition of some formative that begins with /h/. Generally, when a Mandan verb ends with /h/, it is because that /h/ is a reflex of the Proto-Siouan stem augment *-hE. It is not clear if this /h/ is historically linked to this stem augment, or if it is tied to the simultaneous aspectual enclitic /=hąą/, which is an ablaut-triggering enclitic. The synchronic manifestation of habitual aspect marking in Mandan is /=ka/, which does not trigger ablaut, so we are left with the mystery of whether the additional morphology that is present in these habitual positional auxiliaries are a reflex of an older system from Pre-Mandan or even Proto-Siouan, or if they are innovations that developed after the proliferation of /=ka/ as the habitual aspect marker. These verbs maintain the progressive aspectual semantics of the base auxiliary with the added semantics of these activities being habitual. 

We can see examples of these constructions in (\ref{Ch5HabitualPositionals}) below. Like the simple progressive constructions above, we can see examples of these habitual positional auxiliaries being used in situations where multiple verbs receive aspectual semantics from the same auxiliary, such as in (\ref{Ch5HabitualPositionalsB}). Such constructions are rare in the corpus, but they clearly do occur. Within the corpus, there are no attested examples of an analogous \textit{hą́ąkah} form of \textit{hą́ąke}. It is not clear whether such a form exists but is uncommon, or if there is a gap in this paradigm. It is worth noting that \textit{hą́ąke} is less common overall than either \textit{máake} and \textit{náake} in the corpus, and the habitual forms are even more exceptional.

\begin{exe}

    \item\label{Ch5HabitualPositionals} Examples of habitual positional auxiliaries

    \begin{xlist}
        \item\label{Ch5HabitualPositionalsA} \glll \textbf{Kará'taxa} \textbf{náakaha} ~ ~ ~ inák koshų́ųkas húuroomako'sh.\\
        ka-ra'-tax=E rąąkah=E=$\varnothing$ ~ ~ ~ irąk ko-shųųka=s huu=oowąk=o'sh\\
        ins.frce-ins.heat-\textnormal{make.loud.noise}=sv \textnormal{sit}.pos.aux.hab=sv=cont ~ ~ ~ \textnormal{again} 3poss.pers-\textnormal{younger.brother}=def \textnormal{come.here}=narr=ind.m\\
        \glt `He was there crying, so his younger brother came again.' \citep[221]{hollow1973b}

        \item\label{Ch5HabitualPositionalsB} \glll Óo ó'harani  wóo'ipkekeres ą́ąwe ~ ~ ~ ~ xamáhana \textbf{rupáaxini} \textbf{ų́'shkahara} ~ ~ ~ ~ ~ ~ ~ ~ \textbf{máakahini} ą́ąwe rushínashhąą ~ ~ ~ ~ ~ ~ ~ ~ ~ ~ k\'{ı̨}'hkereroomako'sh.\\
        oo o'\#hrE=rį wV-o-i-pke=krE=s ąąwe ~ ~ ~ ~ xwąh=rą ru-paax=rį ų'sh=ka\#hrE ~ ~ ~ ~ ~ ~ ~ ~ wąąkah=rį ąąwe ru-shi=rąsh=hąą ~ ~ ~ ~ ~ ~ ~ ~ ~ ~ kį'k=krE=oowąk=o'sh\\
        dem.mid \textnormal{be}\#caus=ss nom-pv.irr-pv.ins-\textnormal{taste}=3pl=def \textnormal{all} ~ ~ ~ ~ \textnormal{be.small}=top ins.hand-\textnormal{be.broken}=ss \textnormal{be.thus}=hab\#caus ~ ~ ~ ~ ~ ~ ~ ~ \textnormal{lie}.pos.aux.hab=ss \textnormal{all} ins.hand-\textnormal{be.good}=att=ins ~ ~ ~ ~ ~ ~ ~ ~ ~ ~ \textnormal{finish}=3pl=narr=ind.m\\
        \glt `From there, he kept doing it that way and breaking up all the dry meats into small pieces and they finished it all nicely.' \citep[224]{hollow1973b} 

        \item\label{Ch5HabitualPositionalsC} \glll Ó'sh, téehą \textbf{óminitaa} ~ ~ ~ ~ ~ ~ ~ ~ ~ ~ ~ ~ ~ ~ ~ ~ \textbf{mamáakahinito'sh}.\\
        o'sh teehą o-w-rį=taa ~ ~ ~ ~ ~ ~ ~ ~ ~ ~ ~ ~ ~ ~ ~ ~ wa-wąąkah=rįt=o'sh\\
        \textnormal{gosh} \textnormal{be.long.distance} pv.loc-1a-2s=loc ~ ~ ~ ~ ~ ~ ~ ~ ~ ~ ~ ~ ~ ~ ~ ~ 1a-\textnormal{lie}.pos.aux.hab=2pl=ind.m\\
        \glt `Gosh, I have been staying with you all for a long time.' \citep[31]{hollow1973a}
    \end{xlist}
\end{exe}


The lone non-positional auxiliary is the plural durational auxiliary \textit{núunih}. It is most typically translated by Mandan speakers as `they are there.' On its own, this verb is used to express that a plural subject exists at a specific point, as determined by the narrative. We can see examples of the non-auxiliary use of \textit{núunih} `be.\textsc{pl} there' below. Each instance of the verb in question is shown in bold in (\ref{Ch5ExNuunih1}) below.

\begin{exe}
    \item\label{Ch5ExNuunih1} Examples of non-auxiliary use of \textit{núunih}

    \begin{xlist}
        \item\label{Ch5ExNuunih1A} \glll íwahuurahka \textbf{núunihoomako'sh}\\
        i-wa-huu=ahka ruurįh=oowąk=o'sh\\
        pv.poss-unsp-\textnormal{bone}=\textnormal{only} \textnormal{be.there}.pl.dur=narr=ind.m\\
        \glt `only his bones were there' \citep[199]{hollow1973b}

        \item\label{Ch5ExNuunih1B} \glll Karóotiki róoktini \textbf{núunihkereroomako'sh}.\\
        ka=rooti=ki rookti=rį ruurįh=krE=oowąk=o'sh\\ prov=evid=cond \textnormal{make.camp}=ss \textnormal{be.there}.pl.dur=3pl=narr=ind.m\\
        \glt `And then they camped and they were there.' \citep[171]{hollow1973b}

        \item\label{Ch5ExNuunih1C} \glll Nihų́ųxi'he ítoop ą́'teroo ~ ~ ~ ~ ~ ~ ~ ~ \textbf{núuniho're}.\\
        rį-hųų\#xi'h=E i-toop ą't=roo ~ ~ ~ ~ ~ ~ ~ ~ ruurįh=o're\\
        2poss-\textnormal{mother}\#\textnormal{be.old}=sv pv.poss-\textnormal{four} dem.anap=dem.mid ~ ~ ~ ~ ~ ~ ~ ~ \textnormal{be.there}.pl.dur=ind.f\\
        \glt `Your four grandmothers are there.' \citep[104]{hollow1973a}

    \end{xlist}
\end{exe}

When used as an auxiliary verb, \textit{núunih} has similar semantics as a progressive positional auxiliary does. That is to say, they express an ongoing event or state. However, there appears to be a slight aspectual difference between the positional auxiliaries and \textit{núunih} in that \textit{núunih} is used only for plural subjects where a proposition takes place over a prolonged duration. We can see examples of this use of the durational auxiliary below. The instances of durational constructions are shown in (\ref{Ch5NuunihAux}) in bold.

\begin{exe}

\item\label{Ch5NuunihAux} Examples of \textit{núunih} as an auxiliary

\begin{xlist}
    \item\label{Ch5NuunihAuxA} \glll \textbf{Súka} \textbf{núuniha} hų́xte súhkereroomako'sh.\\
    suk=E ruurįh=E=$\varnothing$ hų-xtE suk=krE=oowąk=o'sh\\
    \textnormal{exit}=sv \textnormal{be.there}.pl.dur=sv=cont \textnormal{many}-aug \textnormal{exit}=3pl=narr=ind.m\\
    \glt `They were coming out, a lot of them came out.' \citep[109]{hollow1973b}

    \item\label{Ch5NuunihAuxB} \glll \textbf{téera} \textbf{núunihkereka'ehe}\\
    tee=E ruurįh=krE=ka'ehe\\
    \textnormal{be.dead}=sv \textnormal{be.there}.pl.dur=3pl=quot\\
    \glt `there were dead there, it is said' \citep[166]{hollow1973b}

    \item\label{Ch5NuunihAuxC} \glll wáakiruxka'eshka ótis íki'ų́'taa ~ ~ ~ ~ \textbf{íkitaara} \textbf{núuniherenashoomako'sh}.\\
    waakruxka-eshka o-ti=s i-ki-ų'=taa ~ ~ ~ ~ i-ki-taa=E ruurįh\#hrE=rąsh=oowąk=o'sh\\
    \textnormal{snake}-smlt pv.loc-\textnormal{dwell}=def pv.dir-vert-\textnormal{be.closer}=loc ~ ~ ~ ~ pv.dir-mid-\textnormal{peek}=sv \textnormal{be.there}.pl.dur\#caus=att=narr=ind.m\\
    \glt `he made them look like they were snakes peeking back towards their den' \citep[69]{trechter2012b}
\end{xlist}

\end{exe}

One peculiarity of the auxiliary verb \textit{núunih} is that it is intrinsically plural. However, we do see instances of additional plural marking in the corpus, such as in (\ref{Ch5NuunihAuxB}). In (\ref{Ch5NuunihAuxB}), the third person plural enclitic =\textit{kere} appears after the auxiliary. It is unclear if this plural marking is optional for all speakers or if it is marking another plural argument that is otherwise not obvious from the context.

\subsection{Causatives}\label{Ch5SubSecCausatives}

Mandan employs the causative verb \textit{heré} to create constructions where an agent is involved in making a proposition happen. Causative constructions are complex words where the causative verb becomes part of the same prosodic word as the verb being causativized. The causative verb will bear agreement marking for the causer and the causee. No other elements are able to appear between the causative verb and the causativized verb, and the causativized verb and causative verb will share a single primary stress.

The semantics of causatives in Mandan are more broad than that of `X made Y do Z.' Causatives in the corpus are often translated as `X told Y to do Z', `X allowed Y to do Z', or even `X wanted Y to do Z.' Thus, causatives can indicate direct causation in the sense that someone is causing someone or something else to do something or be a certain way. However, causatives can also indicate indirect causation, as someone may ask or implore or require someone to do something or be a certain way. We can see examples of causative constructions in the data below. The causative constructions have been shown in (\ref{Ch5Causatives}) in bold.

\begin{exe}
    \item\label{Ch5Causatives} Examples of causative constructions

    \begin{xlist}
        \item\label{Ch5CausativesA} \glll Ptamíihe, wíipe \textbf{wáa'isekminihere're}.\\
        p-ta-wįįh=E wiipe waa-i-sek\#w-rį-hrE=o're\\
        1poss-al-\textnormal{woman}=sv \textnormal{cornballs} \textnormal{some}-pv.ins-\textnormal{make}\#1a-2s-caus=ind.f\\
        \glt `My sister, I want you to make some cornballs.' \citep[299]{hollow1973b}

        \item\label{Ch5CausativesB} \glll Ratóoreena waktáni íshąąhe ~ ~ ~ ~ ~ ~ ~ ~ \textbf{kaxípmaherek}...\\
        ratoo=ee=rą wa-ktE=rį i-shąąh=E ~ ~ ~ ~ ~ ~ ~ ~ ka-xip\#wą-hrE=ak\\
        \textnormal{be.old}=dem.dist=top unsp-\textnormal{kill}=ss pv.poss-\textnormal{be.across}=sv ~ ~ ~ ~ ~ ~ ~ ~ ins.frce-\textnormal{peel.skin}\#1s-caus=ds\\
        \glt `That old one killed something and he made me skin half of it...' \citep[42]{hollow1973a}

        \item\label{Ch5CausativesC} \glll Na'é \textbf{réehrohereso'sh}.\\
        rą'e rEEh\#ro-hrE=s=o'sh\\
        \textnormal{mother}.voc \textnormal{go.there}\#1s.pl-caus=def=ind.m\\
        \glt `Mother told us to go there.' \citep[166]{hollow1973a}

        \item\label{Ch5CausativesD} \glll \textbf{kikų́'minihere'sh}\\
        ki-kų'\#w-rį-hrE=o'sh\\
        suus-\textnormal{give}\#1a-2s-caus=ind.m\\
        \glt `I told you to give it to him.' \citep[59]{hollow1973a}

    \end{xlist}
\end{exe}

Negation in causative constructions depends on the intended semantics of what part of the proposition is being negated. The negative prefix \textit{waa}- always appears in initial position within the overall causative complex, which means that \textit{waa}- always appears prefixed onto the causativized verb. The negative enclitics =\textit{xi} and =\textit{nix}, however, can appear either after the causativized verb or on the causative verb itself. Again, the placement of the enclitics indicate what component of the proposition is negated. In (\ref{Ch5CausativesE}) below, we see that the negation enclitic =\textit{nix} appears on the causativized verb. This placement indicates that the entire proposition is not negated, just that it was caused to not be the case. Thus, \textit{wáara'hashinixhara} is more literally `it made him not be burnt up completely.' We can contrast this placement of a negative enclitic with that of the one in (\ref{Ch5CausativesF}), where the =\textit{xi} after the causative indicates that the causers did not make the situation be a certain way. Negative causative imperatives, however, have no such variability in what can be negated within a causative construction. The locus of negation in these constructions is from the negative imperative \textit{káare}, and no negative enclitics appear on the imperative itself. Thus, negation in imperative causative constructions is always on the causative element, never on the causativized element in the corpus. We see this in (\ref{Ch5CausativesG}).

\begin{exe}
    \item\label{Ch5Causatives2} Examples of negation in causative constructions

    \begin{xlist}
        \item\label{Ch5CausativesE} \glll \textbf{wáara'hashinixhara}, ~ ~ ~ ~ ~ ~ ~ ~ ~ ~ ~ ~ rá'pus'harani réehak...\\
	waa-ra'-hash=rįx\#hrE=$\varnothing$ ~ ~ ~ ~ ~ ~ ~ ~ ~ ~ ~ ~  ra'-pus\#hrE=rį rEEh=ak\\
	neg-ins.heat-\textnormal{be.disintegrated}=neg\#caus=cont ~ ~ ~ ~ ~ ~ ~ ~ ~ ~ ~ ~ ins.heat-\textnormal{be.streaked}\#caus=ss \textnormal{go.there}=ds\\
	\glt `causing him to not be burnt up completely, it just scorched him in streaks and went' \citep[154]{hollow1973a}

    \item\label{Ch5CausativesF} \glll \textbf{wáatashkamaharaxikere'sh}\\
    waa-tashka\#wą-hrE=xi=krE=o'sh\\
    neg-\textnormal{how}\#1s-caus=neg=3pl=ind.m\\
    \glt `they could not do anything about me' \citep[319]{hollow1973b}

    \item\label{Ch5CausativesG} \glll Káare \textbf{súkharanista}!\\
    kaare suk\#hrE=rįt=ta\\
    neg.imp \textnormal{exit}\#caus=2pl=imp.m\\
    \glt `Don't let him go out!' \citep[57]{trechter2012b}
        
    \end{xlist}
\end{exe}

In all the examples above in (\ref{Ch5Causatives}), only the causative verb bears any agreement morphology for the arguments involved in the proposition. Speakers can optionally mark the causativized verb for its agent or experiencer instead of the causative verb itself. We can see an example of variable agreement placement in the data below, where each causative construction involves a third person plural argument that acts as the subject of the causativized verb. The third person plural enclitic =\textit{kere} appears on the causativized verb in (\ref{Ch5CausativeAgreeA}), yet it appears on the causative verb in (\ref{Ch5CausativeAgreeB}). The translations below are given by the consultants working with \citet{hollow1973a}, so it is unclear if there are any semantic or pragmatic differences in why plural marking takes place on the causativized verb in one construction but on the causative verb in the other.

\begin{exe}
    \item\label{Ch5CausativeAgree} Examples of variable agreement marking with causative constructions

    \begin{xlist}
        \item\label{Ch5CausativeAgreeA} \glll Károotiki wíipes pą́ątaha \textbf{xtékereharani} ~ tóop ísekoomako'sh.\\
        ka=ooti=ki wiipe=s pą́ątah=E=$\varnothing$ xtE=krE\#hrE=rį ~ toop i-sek=oowąk=o'sh\\
        prov=evid=cond \textnormal{cornball}=def \textnormal{mix}=sv=cont \textnormal{be.big}=3pl\#caus=ss ~ \textnormal{four} pv.ins-\textnormal{make}=narr=ind.m\\
        \glt `And then, mixing up the cornballs, she made them big and she made four.' \citep[268]{hollow1973b}

        \item\label{Ch5CausativeAgreeB} \glll í'aaki'esh \textbf{húuherekerek} ~ ~ ~ ~ ~ ~ ~ ~ ~ ~ ~ ~ ~ ~ ~ ~ minisweerutseena wakíkų'teroomako'sh\\
        i-aaki-esh huu\#hrE=krE=ak ~ ~ ~ ~ ~ ~ ~ ~ ~ ~ ~ ~ ~ ~ ~ ~ wrįs\#wee\#rut=s=ee=rą wa-kikų'tE=oowąk=o'sh\\
        pv.dir-\textnormal{above}-smlt \textnormal{come.here}\#caus=3pl=ds ~ ~ ~ ~ ~ ~ ~ ~ ~ ~ ~ ~ ~ ~ ~ ~ \textnormal{dog}\#\textnormal{feces}\#\textnormal{eat}=s=dem.dist=top unsp-\textnormal{help}=narr=ind.m\\
        \glt `he made them come further up and the dog helped them' \citep[181]{hollow1973a}
    
    \end{xlist}
\end{exe}

Agreement marking on the causativized verb is rare in the corpus, as the causative verb is the expected locus of agreement marking. It is unclear if there is any pragmatic reason for using one tactic for agreement marking or another, or if both are equally grammatical and simply a matter of personal style on the part of the speaker.\footnote{This floating agreement marking is reminiscint of clitic raising in Romance languages like Spanish, where clitics can appear before the finite verb or after an infinitive verb and have the same meaning, e.g., Spanish \textit{\textbf{lo} quiero comer} versus \textit{quiero comer\textbf{lo}} `I want to eat it.'}

\subsection{Desideratives}\label{Ch5Desideritives}

Switch-reference is the most common tactic to connect clauses in Mandan. Constructions involving switch-reference consistently indicate a link between two different propositions by indicating whether there is a shared subject or not. We see switch-reference marking co-opted in a construction that would typically be associated with subordination in other languages, i.e., raising, where the subject of a subordinate clause is treated as the direct object of the superordinate clause. 

In desiderative constructions involving the verb \textit{éereh} `to want', this desiderative verb appears after a clause bearing same-subject switch-reference marking if the subject of the raised verb is the same as that of \textit{éereh}, while the conditional =\textit{ki} is used if the subjects do not match. We can see examples of desideratives involving the same subject in (\ref{Ch5Desideratives}) below, where the desiderative constructions are shown in bold.

\begin{exe}
    \item\label{Ch5Desideratives} Examples of same-subject desideratives

    \begin{xlist}
        \item\label{Ch5DesiderativesA} \glll Ptamáahe \textbf{íwasekini} \textbf{éewereho're}.\\
        p-ta-wąąh=E i-wa-sek=rį ee-we-reh=o're\\
        1poss-al-\textnormal{arrow}=sv pv.ins-1a-\textnormal{make}=ss pv-1a-\textnormal{want}=ind.f\\
        \glt `I want to make my arrows.' \citep[284]{hollow1973b}

        \item\label{Ch5DesiderativesB} \glll Watewé \textbf{írasekini} \textbf{éerareho'sha}?\\
        wa-t-we i-ra-sek=rį ee-ra-reh=o'sha\\
        unsp-wh-indf pv.ins-2a-\textnormal{make}=ss pv-2a-\textnormal{want}=int.m\\
        \glt `What do you want to do?' \citep[305]{hollow1973b}

        \item\label{Ch5DesiderativesC} \glll ą́ąwena \textbf{kų́'hini} \textbf{éerehkerektiki}, ~ ~ ~ ~ ~ ~ ~ ~ ~ ~ kirúkąha náaka...\\
        ąąwe=rą k'-ųųh=rį ee-reh=krE=kti=ki ~ ~ ~ ~ ~ ~ ~ ~ ~ ~ ki-ru-kąh=E rąąkE=$\varnothing$\\
        \textnormal{all}=top 3poss.pers-\textnormal{wife}=ss pv-\textnormal{want}=3pl=pot=cond ~ ~ ~ ~ ~ ~ ~ ~ ~ ~ iter-ins.hand-\textnormal{refuse}=sv \textnormal{sit}.pos.aux=cont\\
        \glt `when everyone wanted to be able to marry her, she would keep refusing them...' \citep[125]{hollow1973a}

        \item\label{Ch5DesiderativesD} \glll ninúpshashka \textbf{raráahini} \textbf{éerehinitki}, ~ ~ ~ ~ ~ ~ ~ ~ ptúuhąhka wawí wahékto'sh\\
        rį-rųp-sha-shka ra-rEEh=rį ee-reh=rįt=ki ~ ~ ~ ~ ~ ~ ~ ~ p-tuuhąk=ka wa-wi=$\varnothing$ wa-hE=kt=o'sh\\
        2s-\textnormal{two}-coll-ints.coll 2a-\textnormal{go.there}=ss pv-\textnormal{want}=2pl=cond ~ ~ ~ ~ ~ ~ ~ ~ 1poss-\textnormal{sister's.child}=hab 1a-\textnormal{watch.over?}=cont 1a-\textnormal{see}=pot=ind.m\\
        \glt `if both of you want to go there, I will look after my nephews' \citep[64]{hollow1973b}  

        \item\label{Ch5DesiderativesE} \glll Kináatani \textbf{karóoskani} \textbf{éerehak}, máa'ąks ~ ~ ~ ~ ~ ókarooske náaku míkoomako'sh.\\
        ki-nąątE=rį ka-rooskE=rį ee-reh=ak waa'ąk=s ~ ~ ~ ~ ~ o-ka-rooskE rąąku wįk=oowąk=o'sh\\
        iter-\textnormal{stand.up}=ss ins.frce-\textnormal{get.down}=ss pv-\textnormal{want}=ds \textnormal{land}=def ~ ~ ~ ~ ~ pv.loc-ins.frce-\textnormal{get.down} \textnormal{road} \textnormal{be.none}=narr=ind.m\\
        \glt `He stood up again and he wanted to get down, but there was no path for him to get down the hill.' \citep[210]{hollow1973a}
    \end{xlist}
\end{exe}

Different-subject desideratives are distinct from same-subject desideratives in that they do not employ switch-reference marking to express the connection between the two clause. Instead, different-subject desideratives use the conditional enclitic =\textit{ki}. We can see examples of different-subject desiderative constructions below. The constructions in question are shown in bold.

\begin{exe}
    \item\label{Ch5DesidDS} Examples of different-subject desideratives

    \begin{xlist}
        \item\label{Ch5DesidDSA} \glll ní'o'na \textbf{áarakxųhki} \textbf{éewereho'sh}\\
        rį-o'=rą aa-ra-kxųh=ki ee-we-reh=o'sh\\
        2s-\textnormal{be}=top pv.tr-2a-\textnormal{lie.down}=cond pv-1a-\textnormal{want}=ind.m\\
        \glt `you are the one who I want to sleep with her' \citep[75]{trechter2012b}

        \item\label{Ch5DesidDSB} \glll ptéhini \textbf{nukíkiki} \textbf{éewereho'sh}\\
        ptEh=rį rų-kiki=ki ee-we-reh=o'sh\\
        \textnormal{run}=ss 1a.pl-\textnormal{race}=cond pv-1a-\textnormal{want}=ind.m\\
        \glt `I want us to run a race' \citep[39]{hollow1973a}

        \item\label{Ch5DesidDSC} \glll pą́ąhį's róotki, pą́ąhį' ~ ~ ~ ~ ~ ~ ~ ~ ~ ~ ~ ~ ~ ~ ~ ~ ~ ~ ~ ~ \textbf{ókaptihki} \textbf{éerehini}...\\
        pąąhį'=s rootki=$\varnothing$ pąąhį' ~ ~ ~ ~ ~ ~ ~ ~ ~ ~ ~ ~ ~ ~ ~ ~ ~ ~ ~ ~ o-ka-ptik=ki ee-reh=rį\\
        \textnormal{porcupine}=def \textnormal{hit}=cont \textnormal{porcupine} ~ ~ ~ ~ ~ ~ ~ ~ ~ ~ ~ ~ ~ ~ ~ ~ ~ ~ ~ ~ pv.loc-ins.frce-\textnormal{be.falling.down}=cond pv-\textnormal{want}=ss\\
        \glt `hitting the porcupine, she wanted the porcupine to fall off...' \citep[65]{hollow1973a}

        \item\label{Ch5DesidDSD} \glll órati róo \textbf{órahereki} \textbf{éewereho'sh}\\
        o-ra-ti roo o-ra-hrE=ki ee-we-reh=o'sh\\
        pv.loc-2a-\textnormal{dwell} dem.mid pv.irr-2a-caus=cond pv-1a-\textnormal{want}=ind.m\\
        \glt `I want you to make your house here' \citep[299]{hollow1973b}
        
    \end{xlist}
\end{exe}

Different-subject desideratives are uncommon in the corpus. This rarity could be due to the fact that causative constructions often carry out a similar function. Same-subject desideratives are extremely common throughout the corpus. It is not clear what the cause of this asymmetry is. One possibility is that same-subject desideratives in the corpus are so common due to the fact that the corpus is built from traditional narratives that involve a single undertaking a task over an extended period of time by themselves. As such, if an individual is the sole focus of the discourse for an extended period of time, then that individual would naturally be expressing what they want and their own desires, as there is no other individual in the discourse upon which to express other desires.


\subsection{Clausal coordination and connections}\label{Ch5Coordination}

\citet[24]{kennard1936} states that =\textit{ni} is a general connective element, and he often translates it as `and'. \citet{hollow1973a,hollow1973b} and \citet{trechter2012b} likewise also translate =\textit{ni} as `and' in their interlinear gloss of Mandan narratives. However, \citet[472]{hollow1970} defines this element as an infinitive or conjunctive. Conversely, =\textit{ak} is often translated as non-finite clauses that have no overt subject, given that \citet[22]{kennard1936} defines it as a past particle marker, a treatment that is echoed in \citet[430]{hollow1970}.

Despite the fact that these two elements are often translated by researchers and Mandan speakers as conjunctions, switch-reference markers do not behave as coordinators in terms of how they function morphologically and syntactically. \citet[53]{kasak2019} argues that switch-reference in Mandan marks adjunct clauses. Assuming this analysis, that leaves the question open as to how Mandan forms coordinated clausal constructions.

\subsubsection{Clausal coordination}\label{Ch5SubsecCoordination}

Analysis of the corpus reveals that coordination is not a common tactic. Within the narratives from \citet{hollow1973a,hollow1973b}, there are several times that \textit{éeheni} is used in a way that resembles a coordinator like `and'. However, this form seems to have been lexicalized as a postposition that means `with'. We have previously seen this use in nominal constructions, as discussed earlier in \sectref{Ch5NounCoord}. In (\ref{Ch5EeheniNoun}) below, we can see an example of this use of \textit{éeheni}.

\begin{exe}
    \item\label{Ch5EeheniNoun} Example of \textit{éeheni} as a postposition

    \glll ká'herektiki, \textbf{hų́pinihe} \textbf{éeheni} íkihįįra ~ ~ ~ máakaha.\\
    ka'\#hrE=kti=ki hųprįh=E ee-he=rį i-ki-hįį=E ~ ~ ~ wąąkah=E=$\varnothing$\\
    \textnormal{possess}\#caus=pot=cond \textnormal{soup}=sv pv-\textnormal{say}=ss pv.ins-iter-\textnormal{drink}=sv ~ ~ ~ \textnormal{lie}.aux.pos.hab=sv=cont\\
    \glt `whenever she would give it to them, they would be drinking it with soup' \citep[207]{hollow1973a}
\end{exe}

This \textit{éeheni} has also been translated as `both' elsewhere in the corpus, such as the example below. We can also see in (\ref{CH5EeheniBoth}) that the construction `both mother and father' does not feature an overt coordinator. Furthermore, this treatment of \textit{éeheni} as a quantifier is restricted to the narratives in \citet{hollow1973b}, which are just re-elicitations from \citet{kennard1934}. It could be that treating \textit{éeheni} as a quantifier could be a feature of older varieties of Mandan, given the fact that \citeapos{kennard1934} worked with consultants who were prior to the reservation period up to the late nineteenth century. It is also possible that the treatment of \textit{éeheni} as `both' is an artifact of translation.


\begin{exe}
    \item\label{CH5EeheniBoth} Example of \textit{éeheni} as a quantifier

    \glll \textbf{na'é} \textbf{taté} \textbf{éeheni} wáarokų're tékara ~ ~ makų́'kereka'sh\\
    rą'e tate ee-he=rį waa-ro-kų'=E te=krE=$\varnothing$ ~ ~ wą-kų'=krE=ka=o'sh\\
    \textnormal{mother}.voc \textnormal{father}.voc pv-\textnormal{say}=ss nom-1s.pl-\textnormal{give}=sv \textnormal{stand}=3pl=cont ~ ~ 1s-\textnormal{give}=3pl=hab=ind.m\\
    \glt `both mother and father are always standing there and give me what he gives us' \citep[128]{hollow1973b}
\end{exe}


What is evident is that there is no overt morpho-syntactic construction for clausal coordination that is clearly identifiable in Mandan within the same utterance. Mandan nominal constructions have a variety of coordination strategies, i.e., asyndeton, monosyndeton, and polysyndeton. Clauses lack any means to express coordination beyond juxtaposition, but it becomes difficult to argue whether strings of verbs in juxtaposition with each other are truly coordinated or if they are involved in a series of fragments or pivots by a speaker who started to say one thing but shifted to rephrase what they were talking about. For this reason, I argue that there is no true clausal coordination in Mandan. The function that clausal coordination plays in languages like English are carried out by switch-reference and serial verb constructions in Mandan. Both of these constructions are structurally adjunct clauses. A proposition like `She walked into the garden and sang' in English, which involves two coordinated clauses, could be rephrased as a single matrix clause with an adjunct clause that has a non-finite verb, i.e., `Walking into the garden, she sang.' Languages like English has at least two strategies to encode the proposition here, while languages like Mandan have just the one.\footnote{\citet[72]{lefebvremuysken1988} note that Quechua lacks a true coordinator, where the comitative case marker -\textit{wan} is often used to coordinate nouns. Many Quechua varieties borrow Spanish coordinators like \textit{y} `and' or \textit{o} `or', given the fact that Quechua lacks morpho-syntactic markers of coordination. Quechua also appears to rely on adjunct clauses to express relationships between clauses in ways similar to Mandan, but that topic is beyond the present scope of this work.}

\subsubsection{Clausal subordination and adjunction}\label{Ch5Subordination}

Clauses within the same sentence in Mandan are linked by producing a clause that ends in a complementizer and juxtaposing that clause with a matrix clause that bears some kind of illocutionary force marker. In \sectref{SecComplementizers}, I extensively detail the variety and functions of these enclitic complementizers, so I will not repeat them all here. However, it is worth noting that the syntactic behavior of clauses bearing these subordinators and adjunct-marking complementizers is very consistent.

Non-matrix clauses in Mandan routinely appear before matrix clauses, regardless of what complementizer appears on the non-matrix clause. We can see examples of this ordering in (\ref{Ch5SubordinationOrder}) below, where the matrix verb appears in bold and the non-matrix clauses appear within square brackets.

\begin{exe}
    \item\label{Ch5SubordinationOrder} Examples of clause ordering

    \begin{xlist}
        \item\label{Ch5SubordinationOrderA} \glll \textnormal{[}pt\'{ı̨}įre hų́so'nik\textnormal{]}, \textbf{níhkipatuna} \textbf{mí'he} \textbf{ruh\'{ı̨}įthereoomakosh}\\
        ptįį=E hų=so'rįk rįk=ka\#i-paturą wį'h=E ru-hįįt\#hrE=oowąk=o'sh\\
        \textnormal{buffalo}=sv \textnormal{many}=comp.caus \textnormal{offspring}=hab\#pv.ord-\textnormal{be.two.years.old} \textnormal{robe}=sv ins.hand-\textnormal{tan.hide}\#caus=narr=ind.m\\
        \glt `\textbf{he had her tan a two-year-old calf robe}, [since there were many buffalo]' \citep[280]{hollow1973b}
        
        \item\label{Ch5SubordinationOrderB} \glll \textnormal{[}róo wakxų́hki\textnormal{]}, \textbf{ó'iraheką't}\\
        roo wa-kxųh=ki o-i-ra-hek=ą't\\
        dem.mid 1a-\textnormal{lie.down}=cond pv.irr-pv.ins-2a-\textnormal{know}=hyp\\
        \glt `\textbf{you would know it}, [if I lay down here]' \citep[1]{hollow1973a}

        \item\label{Ch5SubordinationOrderC} \glll \textnormal{[}Úkereshka'nik\textnormal{]}, \textbf{wáateenixka'sh}.\\
        u=krE=shka'rįk waa-tee=rįx=ka=o'sh\\
        \textnormal{shoot}=3pl=disj neg-\textnormal{die}=neg=hab=ind.m\\
        \glt `\textbf{He never dies}, [but they would shoot him].' \citep[117]{hollow1973b}

        \item\label{Ch5SubordinationOderD} \glll \textnormal{[}óo wíiwaraxirutini\textnormal{]} \textbf{óo} ~ ~ ~ ~ ~ ~ ~ ~ \textbf{wahúuro'sh}\\
        oo wV-i-wa-ra-xrut=rį oo ~ ~ ~ ~ ~ ~ ~ ~ wa-huu=o'sh\\
        dem.mid \textnormal{some}-ins.dir-1a-ins.foot-\textnormal{drive.herd}=ss dem.mid ~ ~ ~ ~ ~ ~ ~ ~ 1a-\textnormal{come.here}=ind.m\\
        \glt `\textbf{I will bring them there} [by driving some there]' \citep[255]{trechter2012b}

        \item\label{Ch5SubordinationOderE} \glll \textbf{Óo} \textbf{ó'harani} \textnormal{[}máapsitaaki\textnormal{]}, \textnormal{[}ráaha ~ ~ ~ ~ ~ ~ ~ ~ ~ ~ núuniha\textnormal{]} \textbf{wíiratąąre} \textbf{hékereroomako'sh}.\\
        oo o'\#hrE=rį wąąpsi=taa=ki rEEh=E ~ ~ ~ ~ ~ ~ ~ ~ ~ ~ ruurįh=E=$\varnothing$ wiiratąą=E hE=krE=oowąk=o'sh\\
        dem.mid \textnormal{be}\#caus=ss \textnormal{morning}=loc=cond \textnormal{go.there}=sv ~ ~ ~ ~ ~ ~ ~ ~ ~ ~ \textnormal{be}.pl.dur=sv=cont \textnormal{enemy}=sv \textnormal{see}=3pl=narr=ind.m\\
        \glt `\textbf{From there, they saw the enemy} [when it was morning] [as they were going along].' \citep[98]{hollow1973a}
        
        
    \end{xlist}
\end{exe}

The general ordering of clauses is clear from the examples above. In situations where there are multiple clauses within the same sentence, as in (\ref{Ch5SubordinationOderE}), all non-matrix clauses will still routinely precede the matrix verb. Topicalized elements, such as locatives and adverbs, may precede even the non-matrix verbs, as is the case in the aforementioned example, where \textit{óo ó'harani} `from there' is associated with the action in the matrix clause \textit{wíiratąąre} \textit{hékereroomako'sh} `they say the enemy', rather than in the following non-matrix clause \textit{máapsitaaki} `when it was morning.'

Non-matrix clauses are able to be postposed after matrix verbs, if the speaker wishes to accentuate that particular clause or add it as an addendum. The matrix clauses in the examples in (\ref{Ch5PostposedClauses}) below are shown in bold, with the postposed non-matrix clauses appearing with an underline beneath. The typical preposed non-matrix clauses appear within square brackets.

\largerpage
\begin{exe}
    \item\label{Ch5PostposedClauses} Examples of postposted non-matrix clauses

    \begin{xlist}
        \item\label{Ch5PostposedClausesA} \glll Hiré, \textnormal{[}ráahini\textnormal{]} \textnormal{[}ísekini\textnormal{]} \textnormal{[}éerehak\textnormal{]} ~ ~ ~ ~ ~ ~ ~ ~ ~ ~ ~ ~ ~  \textbf{míhka'eheroo}, \uline{míkini}.\\
        hire rEEh=rį i-sek=rį ee-reh=ak ~ ~ ~ ~ ~ ~ ~ ~ ~ ~ ~ ~ ~ wįk=ka'ehe=roo wįk=rį\\
        \textnormal{now} \textnormal{go.there}=ss pv.ins-\textnormal{make}=ss pv-\textnormal{want}=ds ~ ~ ~ ~ ~ ~ ~ ~ ~ ~ ~ ~ ~ \textnormal{be.none}=quot=dem.mid \textnormal{be.none}=ss\\
        \glt `Now, [he went] and [he wanted [to do it]], though \textbf{they say that it did not work then}, \uline{there being nothing there}.' \citep[17]{hollow1973a}

        \item\label{Ch5PostposedClausesB} \glll \textnormal{[}Máatah íwokahąą kasími\textnormal{]} ~ ~ ~ ~ ~ ~ ~ ~ ~ ~ ~ ~ ~ ~ ~ \textbf{wa'éroomako'sh}, \uline{wáasherok}.\\
        wąątah i-woka=hąą ka-si=wį=$\varnothing$ ~ ~ ~ ~ ~ ~ ~ ~ ~ ~ ~ ~ ~ ~ ~ wa-E=oowąk=o'sh waa-shro=ak\\
        \textnormal{river} pv.dir-\textnormal{edge}=ins inch-\textnormal{travel}=prog=cont ~ ~ ~ ~ ~ ~ ~ ~ ~ ~ ~ ~ ~ ~ ~ unsp-\textnormal{hear}=narr=ind.m \textnormal{some}-\textnormal{shout}=ds\\
        \glt `[While traveling along the river edge], \textbf{he heard something}, \uline{someone} \uline{shouting}.' \citep[28]{hollow1973a}

        \item\label{Ch5PostposedClausesC} \glll \textnormal{[}máakaha\textnormal{]} \textbf{hiré'oshka} \textbf{máakahkerekto'sh}, ~ ~ ~ ~ \uline{tamí'tikereso'nik}.\\
        wąąkah=E=$\varnothing$ hire-oshka wąąkah=krE=kt=o'sh ~ ~ ~ ~ ta-wį'\#ti=krE=so'rįk\\
        \textnormal{lie}.aux.hab=sv=cont \textnormal{now}-emph \textnormal{lie}.aux.hab=3pl=pot=ind.m ~ ~ ~ ~ al-\textnormal{stone}\#\textnormal{dwell}=3pl=comp.caus\\
        \glt `[Being there], \textbf{they must live there even now}, \uline{since it is their village}. \citep[91]{hollow1973b}
        
    \end{xlist}
\end{exe}

It is uncommon for non-matrix verbs to appear after matrix verbs in Mandan, but they do appear in the corpus. These constructions provide additional evidence against an analysis of switch-reference markers indicating coordinated clauses. If these markers really did indicate coordination, it would be unexpected that we could simply move the coordinator-bearing element (i.e., the element bearing =\textit{ni} or =\textit{ak}) elsewhere in the utterance. If the =\textit{ni} or =\textit{ak} corresponded to a coordinator like `and', we would not be able to move a coordinated element outside of the coordination phrase. Mandan is a head-final language where the heads of phrases are always the final element in their respective domains. If switch-reference markers were the heads of coordination phrases, they would be unique within the language in that they were not head-final, but head-initial like English. We would expect a coordinator to appear after the final element in the coordination phrase, but we instead see =\textit{ni} and =\textit{ak} in a variety of positions within a series of clauses, namely before or after a matrix verb.

If switch-reference markers do indeed indicate coordination, then the data in 
(\ref{Ch5PostposedClauses}) should be ungrammatical, since the coordinated elements are illicitly sequenced for coordination. We can see this fact exemplified below in (\ref{SRisntCoord}), where a grammatical coordinated phrase appears in (\ref{SRisntCoordA}), but an ungrammatical one appears in (\ref{SRisntCoordB}). The construction in (\ref{SRisntCoordB}) is illicit in English, but it mirrors the constructions we see above in (\ref{Ch5PostposedClauses}).

\begin{exe}
    \item\label{SRisntCoord} Evidence against switch-reference as coordination in Mandan

    \begin{xlist}
    \item\label{SRisntCoordA} {~[}They took it] and [they went home].
    \item\label{SRisntCoordB} *[They went home] {[}they took it] and.
    \end{xlist}
\end{exe}

As \sectref{Ch5SubsecCoordination} argues, there is no obvious morpho-syntactic mechanism for clausal coordination in Mandan. Therefore, we can assume by the existence of postposed clauses bearing switch-reference marking or bearing other complementizers that we are not dealing with clausal coordination. Instead, we are looking at a relationship where a single clause is designated as the superordinate matrix clause and all other clauses present in that sentence must necessarily be non-matrix.

The constructions in (\ref{Ch5PostposedClauses}) are evidence that Mandan relies on clausal adjunction as a strategy for expressing clausal relationships rather than clausal adjunction. Furthermore, the fluidity with which Mandan sentences are created in the corpus suggests that non-matrix clauses are adjuncts. We can see in the example below that the Mandan data from (\ref{Ch5PostposedClausesC}) is represented in English in two different ways. The example in (\ref{AdjunctClausesMandanB}) reflects the data seen above, while preposing the non-matrix clause does not affect the grammaticality of the example in (\ref{AdjunctClausesMandanA}). The word order in (\ref{AdjunctClausesMandanA}) is reflective of the general word order of non-matrix clauses with respect to matrix clauses in Mandan, but the order in (\ref{AdjunctClausesMandanB}) is clearly possible, given the data above.

\begin{exe}
    \item\label{AdjunctClausesMandan} Evidence for non-matrix clauses as adjuncts in Mandan

    \begin{xlist}
    \item\label{AdjunctClausesMandanA} {[}Since it is their village], \textbf{they must live there even now}.
    \item\label{AdjunctClausesMandanB} \textbf{They must live there even now}, {[}since it is their village].
    \end{xlist}
\end{exe}

It is unclear what kinds of constructions are true subordination in Mandan, given the fact that non-matrix clauses in Mandan all appear to be optional elements and can be excluded from a sentence without affecting its grammaticality. There is no clear evidence that non-matrix clauses in Mandan are embedded within the matrix clauses. Further study of the corpus and examination of the data could result in the discover of clauses that are complements of a verb instead of adjuncts, but the overwhelming preponderance of non-matrix clauses that bear a complementizer in Mandan exhibit characteristics of clausal adjunction instead of subordination.


\subsection{Relativization}\label{Ch5Relativization}

Mandan makes heavy usage of relative clauses to describe nominals and to act as nominals in their own right. Like other non-matrix clauses in Mandan, relative clauses are adjuncts, where they are optional to the syntax. The morphological manifestations of relativization has been discussed in \sectref{SubsubsecPreverbs} and more extensively in \sectref{SubsubsecRelativized}. The purpose of this section is to outline the syntactic patterns of relativization in the corpus, not to review the ways in which relativization is morphologically marked.

Like other Siouan languages, relative clauses in Mandan can be divided into lexically headed types and non-lexically headed types, e.g., Crow \citep[252]{graczyk1997}, Hidatsa \citep[296]{boyle2007}, Lakota \citep[79]{ingham2003}, \textit{inter alios}. The subsections below outline the behavior of these two relative clause types in Mandan.

\subsubsection{Lexically headed relative clauses}\label{Ch5SubSecHeadedRCs}

Relative clauses, like verbal adjuncts, appear after the nominal to which they are adjoined. The lexical head of a relative clause is a noun or even a nominalized relative clause.

The most transparent manifestation of relativization is the presence of the relativizer \textit{ko}-. This prefix appears as the leftmost element within the verbal template and serves to indicate that the verb in question is part of a relative clause. We can see examples of this relativizer in the examples below in (\ref{Ch5HeadedRCs}). The lexical head of the relative clause is shown in angled brackets, with the relative clause itself also within angled brackets but shown in bold.

\begin{exe}
    \item\label{Ch5HeadedRCs} Examples of lexically headed relative clauses with \textit{ko}-

    \begin{xlist}
        \item\label{Ch5HeadedRCsA} \glll \textnormal{[}róo numá'ks \textnormal{[}\textbf{kotíseena}\textnormal{]]} mí'he ~ ~ ~ ~ paxápini ná'teroomao'sh\\
        roo ruwą'k=s ko-ti=s=ee=rą wį'h=E ~ ~ ~ ~ pa-xap=rį rą'tE=oowąk=o'sh\\
        dem.mid \textnormal{man}=def rel-\textnormal{arrive.there}=def=dem.dist=top \textnormal{robe}=sv ~ ~ ~ ~ ins.push-\textnormal{peel}=ss \textnormal{stand.up}=narr=ind.m\\
        \glt `[This man [\textbf{who arrived}]] threw off his robe and got up.' \citep[94]{hollow1973b}

        \item\label{Ch5HeadedRCsB} \glll róo \textnormal{[}numá'kaaki \textnormal{[}\textbf{kosúhkeres}\textnormal{]]}, \textnormal{[}kosúkanashkere\textnormal{]}'re,  ~ ~ ~ óti íkisekini\\
        roo ruwą'k-aaki ko-suk=krE=s ko-suk=rąsh=krE=o're ~ ~ ~ o-ti i-ki-sek=rį\\
        dem.mid \textnormal{person}-coll rel-\textnormal{exit}=3pl=def rel-\textnormal{exit}=att=3pl=ind.f ~ ~ ~ pv.loc-\textnormal{dwell} pv.ins-iter-\textnormal{make}=ss\\
        \glt `there [the people [\textbf{who came out}]] were [the ones who came out], repairing their homes' \citep[198]{trechter2012b}

        \item\label{Ch5HeadedRCsC} \glll \textnormal{[}kų́'hs \textnormal{[}\textbf{koxamáhsįh}\textnormal{]]} inák rátinik...\\
        k'-ųųh=s ko-xwąh=sįh irąk rat=rįk\\
        3poss.pers-\textnormal{wife}=def rel-\textnormal{be.small}=ints \textnormal{again} \textnormal{call.name}=iter\\
        \glt `he once again kept calling the name of [the wife [\textbf{}\textbf{who was young}]]...' \citep[32]{trechter2012b}

        \item\label{Ch5HeadedRCsD} \glll Óo ó'harani \textnormal{[}ų́'te \textnormal{[}\textbf{koréehraheres}\textnormal{]]} ~ ~ ~ ~ ~ ~ ~ ~ rakirúshekto'sh.\\
        oo o'\#hrE=rį ų't=E ko-rEEh\#ra-hrE=s ~ ~ ~ ~ ~ ~ ~ ~ ra-k-ru-shE=kt=o'sh\\
        dem.mid \textnormal{be}\#caus=ss \textnormal{be.first}=sv rel-\textnormal{go.there}\#2a-caus=def ~ ~ ~ ~ ~ ~ ~ ~ 2a-vert-ins.hand-\textnormal{grasp}=pot=ind.m\\
        \glt `From there, you should take [the first one [\textbf{that you put down}]] back.' \citep[228]{hollow1973b}

        \item\label{Ch5HeadedRCsE} \glll \textnormal{[}kotámiihs ~ ~ ~ ~ ~ ~ ~ ~ ~ ~ ~ ~ ~ ~ ~ ~ ~ ~ ~ ~ ~ ~ ~ ~  \textnormal{[}\textbf{kíihkarahseena}\textnormal{]]} ``hiré ~ ~ ~ ~ ~ ~ ~ ~ ~ ~ ptamíihe, wasíi warého'xere're, káni ~ ~ ~ téehąki ówakiri'eshka're,'' ~ ~ ~ ~ ~ éeheka'ehe\\
	ko-ta-wįįh=s ~ ~ ~ ~ ~ ~ ~ ~ ~ ~ ~ ~ ~ ~ ~ ~ ~ ~ ~ ~ ~ ~ ~ ~ kV-i-k-krah=s=ee=rą hire ~ ~ ~ ~ ~ ~ ~ ~ ~ ~ p-ta-wįįh=E wa-sii wa-reh=o'xre=o're ka=rį ~ ~ ~ teehą=ki o-wa-kri-eshka=o're ~ ~ ~ ~ ~ ee-he=ka'ehe\\
	3poss.pers-al-\textnormal{woman}=def ~ ~ ~ ~ ~ ~ ~ ~ ~ ~ ~ ~ ~ ~ ~ ~ ~ ~ ~ ~ ~ ~ ~ ~ rel-pv.ins-mid-\textnormal{be.afraid}=def=dem.dist=top \textnormal{now} ~ ~ ~ ~ ~ ~ ~ ~ ~ ~ 1poss-al-\textnormal{woman}=sv unsp-\textnormal{travel} 1a-\textnormal{think}=dub=ind.f prov=ss ~ ~ ~ \textnormal{be.long.distance}=cond pv.irr-1a-\textnormal{arrive.back.here}-smlt=ind.f ~ ~ ~ ~ ~ pv-\textnormal{say}=quot\\
	\glt 	`He told [the sister [\textbf{that he was afraid of}]] ``now, my sister, I am thinking of traveling and I will come back after a long time''' \citep[281]{hollow1973b}

    \item\label{Ch5HeadedRCsF} \glll \textnormal{[}Wáaratookaxi'h \textnormal{[}\textbf{kowápashiriihshis}\textnormal{]]} ~ kirúharanista!\\
    waa-ratoo=ka\#xi'h ko-wa-pa-shriih\#shi=s ~ kru\#hrE=rįt=ta\\
    nom-\textnormal{be.mature}=hab\#\textnormal{be.old} rel-unsp-ins.push-\textnormal{think}\#\textnormal{be.good}=def ~ \textnormal{call}\#caus=2pl=imp.m\\
    \glt `Call him [an old man [\textbf{who has good ideas}]].' \citep[167]{hollow1973b}
    \end{xlist}
    
\end{exe}

These kinds of relative clauses are found throughout the corpus. They are especially common when used with stative verbs to convey superlative or comparative semantics, as discussed previously in \sectref{SubsubsecRelativized}. However, a more common kind of relative clause is one where the there is no relativizer \textit{ko}-, but the relativizer is the preverb. We can see examples of these relativized preverb constructions in (\ref{Ch5RCExamples}) below, where the head of the relative clause appears within square brackets, and the relative clause also appears within square brackets but bolded.

\begin{exe}
    \item\label{Ch5RCExamples} Examples of lexically headed relative clauses with preverbs

    \begin{xlist}
        \item\label{Ch5RCExamplesA} \glll \textnormal{[}Súknuma'k \textnormal{[}\textbf{ówaahįįkuunashkereka}\textnormal{]]} ~ ~ ~ ~ ~ ~ ~ ~ ~ ~ ~ ~ ~ \'{ı̨}'xtitaa réeherekere'sh.\\
        suk\#ruwą'k o-waa-hįįkuu=rąsh=krE=ka ~ ~ ~ ~ ~ ~ ~ ~ ~ ~ ~ ~ ~ į'-xti=taa rEEh\#hrE=krE=o'sh\\
        \textnormal{child}\#\textnormal{man} pv.irr-neg-\textnormal{be.difficult}=att=3pl=hab ~ ~ ~ ~ ~ ~ ~ ~ ~ ~ ~ ~ ~ pv.rflx-\textnormal{be.pointed}=loc \textnormal{go.there}\#caus=3pl=ind.m\\
        \glt `They put [young men [\textbf{who are clever}]] at the point.' \citep[210]{hollow1973b}

        \item\label{Ch5RCExamplesB} \glll kawóomihka \textnormal{[}áakinuma'kaaki ~ ~ ~ ~ ~ ~ ~ ~ ~ ~ \textnormal{[}\textbf{óroos}\textnormal{]]} kawóomihka ~ ~ ~ ~ ~ ~ ~ ~ ~ ~ ~ ~ ~ túherekarani...\\
        ka-wV-o-wįh=ka aaki\#ruwą'k-aaki ~ ~ ~ ~ ~ ~ ~ ~ ~ ~ o-roo=s ka-wV-o-wįh=ka ~ ~ ~ ~ ~ ~ ~ ~ ~ ~ ~ ~ ~ tu\#hrE=krE=rį\\
        agt-unsp-pv.loc-\textnormal{point.at}=hab \textnormal{above}\#\textnormal{person}-coll ~ ~ ~ ~ ~ ~ ~ ~ ~ ~ pv.irr-\textnormal{speak}=def agt-unsp-pv.loc-\textnormal{point.at}=hab ~ ~ ~ ~ ~ ~ ~ ~ ~ ~ ~ ~ ~  \textnormal{be.some}\#caus=3pl=ss\\
        \glt `the teachers made some of the [Native Americans [\textbf{who speak their language]}] teachers and...' \citep[236]{trechter2012b}

        \item\label{Ch5RCExamplesC} \glll komíihere \textnormal{[}mána \textnormal{[}\textbf{ósasak}\textnormal{]]} ~ ~ ~ ~ ~ ~ ~ ~ rutą́ąnik...\\
        ko-wįįh=re wrą o-sa$\sim$sak ~ ~ ~ ~ ~ ~ ~ ~ ru-tąą=rįk\\
        3poss.pers-\textnormal{woman}=dem.prox \textnormal{wood} pv.irr-aug$\sim$\textnormal{be.dry} ~ ~ ~ ~ ~ ~ ~ ~ ins.hand-\textnormal{drag}=iter\\
        \glt `his sister would be dragging [wood [\textbf{that was dry}]]...'

        \item\label{Ch5RCExamplesD} \glll \textnormal{[}máareksukeena \textnormal{[}\textbf{``píska''}, \textbf{éeheekerekas}\textnormal{]]} ~ ~ ~ ~ ~ ~ ~ ~ ~ ~ ų́'shkanak kí'kare'ra máakahak...\\
        wąąreksuk=ee=rą pis=ka ee-hee=krE=ka=s ~ ~ ~ ~ ~ ~ ~ ~ ~ ~ ų'sh=ka\#rąk ki'kare'=E wąąkah=ak\\
        \textnormal{bird}=dem.dist=top \textnormal{sniffle}=hab pv-\textnormal{say}=3pl=hab=def ~ ~ ~ ~ ~ ~ ~ ~ ~ ~ \textnormal{be.thus}=hab\#pos.sit \textnormal{fly}=sv \textnormal{lie}.pos.aux.hab=ds\\
        \glt `[a bird [\textbf{that they call ``sniffler{''}}]], that sort [of bird] was there flying around...' \citep[18]{hollow1973a}

        \item\label{Ch5RCExamplesE} \glll \textnormal{[}wáa'aahuu \textnormal{[}\textbf{áahinashs}\textnormal{]]} ~ ~ ~ ~ ~ ~ ~ ~ ~ ~ óhi, kų́'here, máapsitaarak, ~ ~ ~ ~ súkini réehoomako'sh\\
        waa-aa-huu aa-hi=rąsh=s ~ ~ ~ ~ ~ ~ ~ ~ ~ ~ o-hi k'-ųųh=re wąąpsi=taa=ak ~ ~ ~ ~ suk=rį  rEEh=oowąk=o'sh\\
        nom-pv.tr-\textnormal{come.here} pv.tr-\textnormal{arrive.there}=att=def ~ ~ ~ ~ ~ ~ ~ ~ ~ ~ pv.irr-\textnormal{arrive.there} 3poss.pers-\textnormal{wife}=dem.prox \textnormal{morning}=loc=ds ~ ~ ~ ~ \textnormal{exit}=ss \textnormal{go.there}=narr=ind.m\\
        \glt `when he arrived with [the belongings [\textbf{that he arrived with}]], the wife went out and left in the morning' \citep[67]{trechter2012b}

        \item\label{Ch5RCExamplesF} \glll \textnormal{[}Ą́'t \textnormal{[}\textbf{í'ą'ska}\textnormal{]]}, ptamíihe ~ ~ ~ ~ ~ ~ ~ ~ áahuurak, ó'ro'sh.\\
        ą't i-ą's=ka p-ta-wįįh=E ~ ~ ~ ~ ~ ~ ~ ~ aa-huu=ak o'=o'sh\\
        dem.anap pv.dir-\textnormal{be.near}=hab 1poss-al-\textnormal{man's.sister}=sv ~ ~ ~ ~ ~ ~ ~ ~ pv.tr-\textnormal{come.here}=ds \textnormal{be}=ind.m\\
        \glt `\textnormal{[}That one [\textbf{who is that way}], my sister brought him, so he is here.' \citep[129]{hollow1973a}
        
    \end{xlist}
\end{exe}

The majority of cases where a preverb is used to indicate a relative clause involves the irrealis preverb \textit{ó}-, however, there are numerous examples of the other preverbs being used in a similar fashion throughout the corpus. The use of \textit{ó}- to create relative clauses is reminiscent of the relativizers \textit{agu}- and \textit{aru}- in Hidatsa, which also have the allomorph \textit{oo}- before certain stems \citep[40]{boyle2007}, though it is not clear whether this /o/-shaped relativization marker is parallel evolution or influence from one language upon the other. 

One final source of relative clauses in the corpus involves verbs that bear no relativization morphology at all. In these constructions, the only indication that the clauses are relativized comes from the translations offered by the transcriber. We can see an example of one such instance in (\ref{RCwithoutMorphology}) below with its original translation represented in the two-line interlinear gloss style that \citet[61]{hollow1973a} uses.

\begin{exe}
    \item\label{RCwithoutMorphology} Example of relative clauses without relativization marking
    
    \gll Mishų́ųkak, koník  koxamáhere, ráse ínupshashka, ą́'t, ráse túkere'sh.\\
    \textnormal{my brother} \textnormal{his son} \textnormal{the youngest} \textnormal{his names} \textnormal{both of them} \textnormal{those} \textnormal{names} \textnormal{that he got}\\
    \glt `My brother's youngest son, both of his names, those ones, they are the names that he has.' \citep[61]{hollow1973a}
\end{exe}

The translation of the example above appears to be periphrastic in that there is no relativization marking whatsoever present.\footnote{This example has appeared previously in (\ref{ExAnaphoricDemPl2}) in Chapter \ref{chapter4} with my own interpretation, given its context in the narrative.} Instances such as the one above are almost certainly artifacts of translation rather than true instances of relativization. The corpus is generated from narratives that are delivered initially in Mandan, then translated into English. In the case of \citet{hollow1973a}, the Mandan consultants give free translations into English afterward the initial Mandan telling, and then \citeauthor{hollow1973a} goes back later to give a word by word translation of what has been transcribed with the help of his consultants or other Mandan community members. \citet{trechter2012,trechter2012b} employs a similar strategy with Mr. Edwin Benson and Mr. Corey Spotted Bear. With these instances of relative clauses in the corpus being the result of paraphrasing from Mandan into English rather than direct translations, I argue that Mandan has only the two strategies for relativization that are described above: using \textit{ko}- or a preverb.

The choice of which relativizer to use is not completely obvious. \citet[15]{kennard1936} states that \textit{ko}- is used for constructions where the action involves an agent, and \citet[451]{hollow1970} similarly describes \textit{ko}- as an agentive relativizer, but for non-stative verbs. We see in (\ref{RCsStative}) that \textit{ko}- can be used with stative verbs as well, however.

\begin{exe}
    \item\label{RCsStative} Examples of stative verbs bearing \textit{ko}-

    \begin{xlist}
    \item\label{RCsStativeA} \glll manáwerexe \textbf{ko'ų́'st} \textbf{kotké} \textbf{kokámix} ~ ~ ~ \textbf{koxtés} kixų́ųh\\
        wrą\#wrex=E ko-ų't=t ko-tke ko-kawįx ~ ~ ~ ko-xtE=s {kixųųh}\\
        \textnormal{wood}\#\textnormal{kettle}=sv rel-\textnormal{be.in.past}=loc rel-\textnormal{be.heavy} rel-\textnormal{be.round} ~ ~ ~ rel-\textnormal{be.big}=def \textnormal{five}\\
        \glt `{five} \textbf{big}, \textbf{round}, \textbf{heavy}, \textbf{old} drums' (lit. `five drums that are big, that are round, that are round, that are heavy, and that are old.') \citep[21]{mixco1997a}

        \item\label{RCsStativeB} \glll ímashut \textbf{ko'áaki}\\
        i-wąshut ko-aaki\\
        pv.ins-\textnormal{clothe} rel-\textnormal{above}\\
        \glt `top coat' (lit. `shirt \textbf{that is on top'}) \citep[56]{hollow1970}
    \end{xlist}
\end{exe}

In the examples above, we see stative verbs being used with \textit{ko}-, which is something that \citet[451]{hollow1970} states provides nominalizing semantics. This use of \textit{ko}- has already been detailed in \sectref{SubsubsecRelativized}. However, relative clauses in general can be treated as nominals in Mandan, as detailed further in \sectref{Ch5HeadlessRCs} below.

\subsubsection{Lexically headless relative clauses}\label{Ch5HeadlessRCs}

Relative clauses happen throughout the corpus with great frequency. One of the most productive methods of adding nouns to the lexicon is to describe what that entity does or what it is for. These novel terms are often derived from a relative clause. Furthermore, these relative clauses are typically absent a lexical head. Headless relative clauses can be seen throughout \citeapos{hollow1970} dictionary. Several examples of headless relative clauses treated as nouns appear below.

In (\ref{Ch5HeadlessRCExs}), we see numerous instances of a noun that is compositionally a relative clause that employs a preverbal relativizer. In each of the examples below, there is no lexical head to which each relative clauses adjoins. For example, \textit{ó'aakų} `a shadow' in (\ref{Ch5HeadlessRCExsA}) involves the irrealis preverb \textit{o}- prefixed onto the verb \textit{áakų} `cast a shadow.' The literal translation of this item is `when a shadow is cast', but it is the case that Mandan speakers have lexicalized this term to mean a specific noun. We can see other similar nouns that occur in the lexicon that are transparently relative clauses in (\ref{Ch5HeadlessRCExs}) below.

\begin{exe}
    \item\label{Ch5HeadlessRCExs} Examples of nouns that are headless relative clauses

    \begin{xlist}
        \item\label{Ch5HeadlessRCExsA} \glll ó'aakų\\
        o-aakų\\
        pv.irr-\textnormal{cast.a.shadow}\\
        \glt `a shadow' (lit. `when a shadow is cast') \citep[57]{hollow1970}

        \item\label{Ch5HeadlessRCExsB} \glll ímikiha\\
        i-wį-ki-hE\\
        pv.ins-1s-rflx-\textnormal{see}\\
        \glt `a mirror' (lit. `what I see myself with') \citep[71]{hollow1970}

        \item\label{Ch5HeadlessRCExsC} \glll íhįįka\\
        i-hįį=ka\\
        pv.ins-\textnormal{drink}=hab\\
        \glt `a pipe' (lit. `what one smokes with') \citep[74]{hollow1970}

        \item\label{Ch5HeadlessRCExsD} \glll óhop\\
        o-hop\\
        pv.loc-\textnormal{be.hollow}\\
        \glt `a hole' (lit. `where it is hollow') \citep[77]{hollow1970}

        \item\label{Ch5HeadlessRCExsE} \glll írukaxka\\
        i-ru-kEx=ka\\
        pv.ins-ins.hand-\textnormal{scrape}=hab\\
        \glt `a rake' (lit. `what one holds in one's hands to scrape with') \citep[107]{hollow1970}

        \item\label{Ch5HeadlessRCExsF} \glll ókso\\
        o-kso\\
        pv.irr-\textnormal{spit}\\
        \glt `saliva' (lit. `when one spits') \citep[121]{hollow1970}

        \item\label{Ch5HeadlessRCExsG} \glll ópshii\\
        o-pshii\\
        pv.loc-\textnormal{be.flat}\\
        \glt `plains' (lit. `where it is flat') \citep[445]{hollow1970}
    \end{xlist}
\end{exe}

One salient quality of these relative clauses is that the lack any kind of lexical head. There is no overt noun that can be inserted to maintain the same lexical semantics. For example, the word \textit{ópshii} `plains' in (\ref{Ch5HeadlessRCExsG}) does not have the same meaning when used as a \textit{bona fide} relative clause, as we see in (\ref{RCHeadlessLexicalChange}) below.

\begin{exe}
    \item\label{RCHeadlessLexicalChange} Example of polysemy of certain relative clauses

    \glll \textnormal{[}mí'tis nátoo \textnormal{[}\textbf{ópshiitaa}\textnormal{]]}, óo ~ ~ ~ ~ ó'harani wíiskek náapextekereroomako'sh\\
    wį'\#ti=s rąt=oo o-pshii=taa oo ~ ~ ~ ~ o'\#hrE=rį wV-i-skE=k rąąp(E)-xtE=krE=oowąk=o'sh\\
    \textnormal{stone}\#\textnormal{dwell}=def \textnormal{be.middle.of}=dem.mid pv.loc-\textnormal{be.flat}=loc dem.mid ~ ~ ~ ~ \textnormal{be}\#caus=ss unsp-pv.ins-\textnormal{jump}=hab \textnormal{dance}-aug=3pl=narr=ind.m\\
    \glt `[The village center there [\textbf{where it was flat}]], they danced a lot to praise songs from there' \citep[251]{hollow1973b}
\end{exe}

The use of \textit{ópshiitaa} in (\ref{RCHeadlessLexicalChange}) above obviously excludes the reading of \textit{ópshii} as `plains', given the context. A village center is clearly not big enough to encompass an extended geographical features like a plain, so this polyseme must indicate the meaning of `where it is flat' instead. The use of \textit{ópshii} `where it is flat' with a lexical head in (\ref{RCHeadlessLexicalChange}) above makes it obvious that this polyseme cannot be referring to the plains. Relative clauses without a lexical head like the one in (\ref{RCHeadlessLexicalChange}) above are very common throughout the corpus.

In each of the examples in (\ref{Ch5HeadlessRCsInCorpus}) below, headless relative clauses are represented with the null symbol, $\varnothing$, in place of where a lexical head should be, followed by the relative clause in square brackets and in bold.

\begin{exe}
    \item\label{Ch5HeadlessRCsInCorpus} Examples of headless relative clauses

    \begin{xlist}
        \item\label{Ch5HeadlessRCsInCorpusA} \glll ptasúknuma'ke ą́ąwe \textnormal{[}$\varnothing$ \textnormal{[}\textbf{ótawiiratąąre}\textnormal{]]} kitúni...\\
        p-ta-suk\#ruwą'k=E ąąwe ~ o-ta-wiiratąą=E ki-tu=rį\\
        1poss-al-\textnormal{child}\#\textnormal{man}=sv \textnormal{all} ~ pv.loc-al-\textnormal{enemy}=sv iter-\textnormal{some}=ss\\
        \glt `I have all my young men [$\varnothing$ [\textbf{where their enemies are}]]...' \citep[56]{hollow1973a}

        \item\label{Ch5HeadlessRCsInCorpusB} \glll \textnormal{[}$\varnothing$ \textnormal{[}\textbf{watámaana} \textbf{ókimimanashini}, \textbf{téetokinashini}, ~ ~ ~ ~ ~ ~ ~ ~ ~ ~ \textbf{ų́'kanashoo}\textnormal{]]} wíipto íminixkere...\\
        ~ waa-ta-waarą o-kiiwą=rąsh=rį teetoki=rąsh=rį ~ ~ ~ ~ ~ ~ ~ ~ ~ ~ ų'=ka=rąsh=oo wiipto i-wrįx=krE\\
        ~ unsp-ta-\textnormal{winter} pv.irr-\textnormal{six}=att=ss \textnormal{eight}=att=ss ~ ~ ~ ~ ~ ~ ~ ~ ~ ~ \textnormal{be.thus}=hab=dem.mid \textnormal{ball} pv.ins-\textnormal{play}=3pl\\
        \glt `[The ones [\textbf{who were about six or eight or something like that}]] played ball...' \citep[126]{hollow1973a}

       

        \item\label{Ch5HeadlessRCsInCorpusD} \glll Mákak \textnormal{[}$\varnothing$ \textnormal{[}\textbf{Kowóoxohkas}\textnormal{]]} ~ ~ ~ ~ ~ ~ ~ ~ ~ ~ téehereroomaksįh.\\
        wąk=ak ~ ko-wV-o-xok=ka=s ~ ~ ~ ~ ~ ~ ~ ~ ~ ~ tee\#hrE=oowąk=sįh\\
        pos.lie=ds ~ rel-unsp-pv.loc-\textnormal{swallow}=hab=def ~ ~ ~ ~ ~ ~ ~ ~ ~ ~ \textnormal{die}\#caus=narr=ints\\
        \glt `He was there and [the One [\textbf{who Swallows}\footnotemark]] killed him.' \citep[97]{hollow1973a}
        \footnotetext{The translation typically given for \textit{Kowóoxohkas} in Mandan narratives is `the Swallower', a monster who has a man-like shape, but with no head and a mouth that goes from shoulder to shoulder. A similar figure appears in Hidatsa narratives as \textit{Íihdia} `Big Mouth' \citep[136]{matthews1878}.}

        \item\label{Ch5HeadlessRCsInCorpusC} \glll káni \textnormal{[}$\varnothing$ \textnormal{[}\textbf{óranuunihinits}\textnormal{]]} áakihąą ~ ~ ~ ~ ~ ~ ~ ~ ~ waakí'kare'ra máanaakiki\\
        ka=rį ~ o-ra-ruurįh=rįt=s aaki=hąą ~ ~ ~ ~ ~ ~ ~ ~ ~ waa-ki'kare'=E waa-rąąkE=ki\\
        prov=ss ~ pv.loc-2a-\textnormal{be}.aux.pl.dur=2pl=def \textnormal{above}=ins ~ ~ ~ ~ ~ ~ ~ ~ ~ \textnormal{someone}-\textnormal{fly}=sv \textnormal{someone}-\textnormal{sit}.pos.aux=cond\\
        \glt `and when there is someone flying above [$\varnothing$ [where you are]]' \citep[130]{hollow1973a} 

        \item\label{Ch5HeadlessRCsInCorpusE} \glll \textnormal{[}$\varnothing$ \textnormal{[}\textbf{Koháni} \textbf{éerehs}\textnormal{]]} ótoomako'sh.\\
        ~ ko-hE=rį ee-reh=s o=t=oowąk=o'sh\\
        ~ rel-\textnormal{see}=ss pv-\textnormal{want}=def pv.loc=loc=narr=ind.m\\
        \glt `[The one [\textbf{who he wanted to see}]] was with them.' \citep[127]{hollow1973a}

        \item\label{Ch5HeadlessRCsInCorpusF} \glll Kixéektek, \textnormal{[}$\varnothing$ \textnormal{[}\textbf{mí'ti} \textbf{kotkás}\textnormal{]]} ą́ąwena pó xtés wakirútoomako'sh.\\
        ki-xee=ktE=ak ~ wį'\#ti ko-tka=s ąąwe=rą po xtE=s wa-k-rut=oowąk=o'sh\\
        mid-\textnormal{be.slow}=pot=ds ~ \textnormal{stone}\#\textnormal{dwell} rel-\textnormal{reside.in}=def \textnormal{all}=top \textnormal{fish} \textnormal{be.big}=def unsp-iter-\textnormal{eat}=narr=ind.m\\
        \glt `When he could stop, he would eat all [the ones [\textbf{who lived in the village}]]'s big fish.' \citep[201]{hollow1973b}

        \item\label{Ch5HeadlessRCsInCorpusG} \glll Károotiki, \textnormal{[}$\varnothing$ \textnormal{[}\textbf{koxópiniseena}\textnormal{]]}: ``Xéepa, ~ ~ ~ ~ ~ núunihinista!''\\
        ka=ooti=ki ~ ko-xoprį=s=ee=rą xeepa ~ ~ ~ ~ ~ ruurįh=rįt=ta\\
        prov=evid=cond ~ rel-\textnormal{be.holy}=def=dem.dist=top \textnormal{hold.up} ~ ~ ~ ~ ~ \textnormal{be}.aux.pl.dur=2pl=imp.m\\
        \glt `And then, the holy one was like, ``wait a minute, stay there!{''}' \citep[259]{hollow1973b}

        \item\label{Ch5HeadlessRCsInCorpusH} \glll \textnormal{[}$\varnothing$ \textnormal{[}\textbf{Kíisehka}\textnormal{]]}, kakí'ųųtka ~ ~ ~ ~ ~ ~ ~ ~ ~ ~ ~ ~ ~ ~ ~ túkereka'sh.\\
        ~ kV-i-sek=ka ka-ki-ųųt=ka ~ ~ ~ ~ ~ ~ ~ ~ ~ ~ ~ ~ ~ ~ ~ tu=krE=ka=o'sh\\
        ~ rel-pv.ins-\textnormal{make}=hab agt-mid-\textnormal{be.first}=hab ~ ~ ~ ~ ~ ~ ~ ~ ~ ~ ~ ~ ~ ~ ~ \textnormal{be.some}=3pl=hab=ind.m\\
        \glt `[One [\textbf{who does it}]], there is always a leader.' \citep[238]{trechter2012b}
    \end{xlist}
\end{exe}

The English translation for relative clauses bearing \textit{ko}- typically follow some formula of `the one who \textit{ko}-X', where X is whatever the relativized predicate is. The translations of these relative clauses have a head (i.e., `the one(s)'), but there is no such head present in the syntax in Mandan in the examples above. Furthermore, many of these headless relative clauses are treated as nominals on their own. We see in (\ref{Ch5HeadlessRCsInCorpusD}) through (\ref{Ch5HeadlessRCsInCorpusG}) that the verb of the headless relative clause bears the definite article =\textit{s} `the', and (\ref{Ch5HeadlessRCsInCorpusG}) in particular bears the topic marker =\textit{na}. These are all formatives associated with nominals. Headless relative clauses stand as one of the most productive methods of building novel nominals in Mandan.

The majority of headless relative clauses in the corpus involve a preverb. One reason for this distribution is that \textit{ko}- is most often associated with relative clauses with an animate subject. The irrealis preverb \textit{o}- can be used in similar ways throughout the corpus, though it is not clear if there is a semantic or pragmatic reason for using \textit{ko}- versus \textit{o}- for relative clauses that modify a human or otherwise animate entity involving a stative verb. One possible difference is that \textit{ko}- might imply some unique characteristic about the entity in question, as we saw in example (\ref{RCsStative}), where each adjectival use of a stative verb received a \textit{ko}-, while \textit{o}- might imply a quality that is not permanent or is subjective, as we see in (\ref{Ch5RCExamplesA}). Without an L1 speaker to elucidate the strategy for selecting one relativizer over another, we can only postulate why speakers opt for \textit{ko}- in some in some instances, but \textit{o}- in others.
