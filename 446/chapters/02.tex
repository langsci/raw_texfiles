\chapter{Phonetics and phonology}\label{chapter2}

This chapter describes the sound system of Mandan. Very little attention has been paid to phonetic and phonological matters in this language, as previous efforts have focused on the collection of narratives  (e.g., \citealt{kennard1934}; \citealt{hollow1973a,hollow1973b}; \citealt{trechteretal2009}; \citealt{trechter2012,trechter2012b}), the creation of brief grammatical sketches (e.g., \citealt{kennard1936}, \citealt{coberly1979}, \citealt{carter1991b}, or \citealt{mixco1997a}), or word and affix lists (e.g., \citealt{maximilian1839}, \citealt{will1906}, and \citealt{hollow1970}). Pedagogical materials created for Mandan likewise mention little about the relevant sounds in this language (e.g., \citealt{hollow1976}, \citealt{littleowl1992}, and \citealt{benson2000}).

There are two overall goals for this chapter. One goal is to provide an overview of the phonemic inventory of Mandan, as well as allophonic alternations and phonological processes that influence the realization of surface forms. This chapter also deals with information that has not been previously described, such as stress assignment and environments where nasal harmony is blocked. Phonological processes are described in a way that is meant to be theoretically neutral and accessible to a readership consisting of community members while also presenting data in a way such that linguists and other scholars can contextualize these phonological processes within the typology. One additional contribution that this work provides to the understanding of Mandan is the inclusion of phonetic data analyzed using \citeauthor{boersmaweenik2016}'s (\citeyear{boersmaweenik2016}) program Praat to conduct waveform and spectrographic analyses.

I begin this chapter by examining the consonant inventory of Mandan in \sectref{consonants}, before moving on to vowels in \sectref{SecVowels}. The orthography used for Mandan in this book is explained in \sectref{SecOrthography}. An examination of the phonotactics of Mandan and its possible consonant clusters appears in \sectref{SecPhonotactics}. In \sectref{boundaryindependent}, I describe those morpho-phonological phenomena that do not take internal word structure into account, and then document those phenomena that treat internal word boundaries as blocking environments in \sectref{boundarydependent}.


\section{Consonants}\label{consonants}

Most description of Mandan grammar has revolved around its morphology (\citealt{kennard1936}, \citealt{hollow1970}). The analysis on its sound system, and its consonants in particuarly, has been minimal. This section aims to collect what little description has been done on Mandan consonants to argue for the phonemic inventory shown in \tabref{consonantinventory}, and to contrast this inventory with the descriptions by previous researchers and their subsequent transcriptions of Mandan. First, I propose an inventory of consonants, then describe its plosives in order to resolve that there are no voiced stops in Mandan and that there is a single stop series. I investigate whether the affricate /tʃ/ as described in \citet{kennard1936} is actually present in Mandan, and discount it as the misperception of certain consonant clusters. I show that Mandan has the same set of plain fricatives that are common to most Siouan languages, and also show that there are no underlying nasal consonants in Mandan, as all surface nasals can be attributed to nasal harmony with a following nasal vowel.

Mandan is similar to other languages of the northern Great Plains in that it has a relatively small consonant inventory within \citeapos{maddieson2013b} typology of consonant inventory sizes.\footnote{Languages of the northern Great Plains have small consonant inventories going off the typology (i.e., fewer than 14 consonants) per \citet{maddieson2013b}, with \ili{Hidatsa} having 10 \citep[27]{boyle2007}, \ili{Crow} with 11 \citep[12]{graczyk2007}, \ili{Arikara} with 12 \citep[1]{parksbeltranwaters1979}, \ili{Cheyenne} with 11 \citep[214]{leman2013}, \ili{Arapaho} with 12 \citep[4]{picard1994}, \ili{Plains Cree} with 10 \citep[8]{wolfart1981}, and \ili{Pawnee} with 8 \citep[13]{parks1976}. To the best of my knowledge, there has not been much in-depth study to determine whether this phenomenon is part of some language area effect, particularly since the ancestor languages of the languages mentioned above are reconstructed with drastically larger consonant inventories.} In particular, the number of phonemic consonants is quite low, with only 10. Mandan is likewise noteworthy for having no underlying nasal consonants, despite the fact that nasal consonants are some of the most commonly encountered sounds in the language. This lack of underlying nasal consonants is a holdover from Proto-Siouan, which likewise lacked such consonants \citep{rankinetalnd}. Missouri Valley languages share this feature with Mandan, which is one of the numerous reasons why previous researchers have grouped them together \citep{rankin2010}.

\subsection{Consonant inventory}\label{inventory}

\tabref{consonantinventory} presents a summary of the phonemic consonant inventory of Mandan, with the corresponding orthographic equivalent represented in angled brackets. Allophony is not represented here, but is addressed in subsequent sections.

\begin{table}
\begin{center}
\caption{Consonant inventory} \label{consonantinventory}
\begin{tabular}{lccccc}
\lsptoprule
          & Bilabial & Alveolar & Postalveolar & Velar & Glottal \\
\midrule
Plosive   & p        & t        &               & k     & ʔ $\langle$'$\rangle$    \\
Fricative &          & s        & ʃ $\langle$sh$\rangle$          & x     & h       \\
Sonorant  & w        & ɾ $\langle$r$\rangle$         &          	&       &         \\ \lspbottomrule
\end{tabular}
\end{center}

\end{table}

The consonant inventory listed in \tabref{consonantinventory} does not differ from the inventory given in \citet[14]{hollow1970}. However, the inventory above is smaller than those given in \citet[2]{kennard1936} and \citet[13]{mixco1997a}, who identify 13 and 11 consonants, respectively. \citeauthor{kennard1936}'s inventory includes /\textsuperscript{n}d n m/, and he does not take allophony into account, however. As I demonstrate below in \sectref{sonorants}, allophony can account for the presence of surface [\textsuperscript{n}d n] in Mandan, with these sounds as allophones of /ɾ/ and [m] as an allophone of /w/. \citeauthor{mixco1997a} includes an affricate /tʃ/ (see \sectref{affricates}).

Of note is that Mandan has a drastically diminished consonant inventory when compared to other Siouan languages. The neighboring Lakota have a four-way stop distinction and a two-way fricative distinction, while the Omaha to the south have a five-way stop distinction. The inventory of consonants described in \citet{rankinetalnd} for Proto-Siouan appears below in \tabref{PSiconsonants}.

\begin{table}
\begin{center}
\caption{Proto-Siouan consonant inventory}
\label{PSiconsonants}
\begin{tabular}{llccccc}
\lsptoprule
\multicolumn{2}{l}{}                     & \multicolumn{1}{l}{Bilabial} & \multicolumn{1}{l}{Coronal} & \multicolumn{1}{l}{Palatal} & \multicolumn{1}{l}{Velar} & \multicolumn{1}{l}{Glottal} \\
\midrule
\multirow{4}{*}{Plosive}   & plain         & *p                            & *t                           &                              & *k                         & *ʔ                           \\ %\cline{2-7}
                           & glottalized   & *pʔ                           & *tʔ                          &                              & *kʔ &                              \\ %\cline{2-7}
                           & preaspirated  & *hp                           & *ht                          &                              & *hk                        &                              \\ %\cline{2-7}
                           & postaspirated & *ph                           & *th                          &                              & *kh                        &                              \\
\multirow{2}{*}{Fricative} & plain         &                               & *s                           & *š                           & *x                         & *h                           \\ %\cline{2-7}
                           & glottalized   &                               & *sʔ                          & *šʔ                          & *xʔ                        &                              \\
\multicolumn{2}{l}{Sonorant}             & *w                            & *r                           & *y                           &                            &                              \\
\multicolumn{2}{l}{Obstruent}            & *W                            & *R                           &                              &                            &                              \\ \lspbottomrule
\end{tabular}
\end{center}
\end{table}

The two aspirated stop series collapsed with the plain stops in Mandan, and the glottalized series for the stops underwent a particular metathesis, where the glottalized component of the stop became a coda glottal, i.e., PSi *CʔV $>$ Man CVʔ, as we see in \textit{kú'} `give' < PSi *kʔu. Proto-Siouan glottalized fricatives likewise underwent this change in Mandan, e.g., PSi *šʔehe `drip' > Man \textit{shé'he} `drip' and Lak \textit{oyúš'e} `make a drop into something using one's hands.' Mandan shares this typologically marked sound change with Missouri Valley Siouan, yet another reason to posit a closer genetic relationship between these two branches of the Siouan language family. Further work is needed to explore additional evidence for the closer relationship between Mandan and Missouri Valley Siouan.

\subsection{Plosives}\label{plosives}

This subsection serves to describe the plosives present in Mandan, which has only a single stop series. Other branches of Siouan have several contrasting stop series. Osage exemplifies this robust distinction with its five-way stop series: plain, preaspirated, postaspirated, voiced, and ejective \citep[17]{quintero2004}. Mandan is similar to Crow in that it only has a single stop series, though it differs from Hidatsa, which has re-innovated a postaspirated distinction (\citealt[27]{boyle2007}, \citealt[12]{graczyk2007}).\footnote{The postaspirated stops in Hidatsa correspond to geminate stops in Crow, e.g., Hid \textit{-taa} [-tʰaː] \textsc{neg} and Cro \textit{-ssaa} [-sːaː] \textsc{neg}, but both in turn originate from simple stops or sonorants. The negative suffix, for example, comes from PSi *rį \textsc{neg} plus the adverbial *haa. The short vowel syncopates before the syllable with the long vowel, creating an **rh cluster. The **r fortifies to ***t, yielding the aspirated [tʰ] \citep[8]{jones1983}. Mandan treats PSi *rh clusters similarly, but only has the plain stops, e.g., PSi *re-híi `arrive here' > **rhii > Man \textit{tí} [ti] `arrive here.'} The collapse of the Proto-Siouan stop series must have occurred before the ancestor language of Mandan and Missouri Valley Siouan (i.e., Crow and Hidatsa), given that all three languages involve a merger of *C, *hC, and *Ch to /C/, as well as metathesis of glottal elements in *CʔV to /CVʔ/ \citep{rankinetalnd}. There is no evidence that Mandan has innovated any other stop series since these sound changes in Proto-Mandan-Missouri Valley.

\subsubsection{Supralaryngeal stops}\label{supraplosives}

We can divide the stops in Mandan into two varieties: supralaryngeal and sublaryngeal. Historically, the sublaryngeal stop (i.e., the glottal stop) has often not be transcribed and can be difficult to discern by ear for some. Supralaryngeal stops have been the subject of more description and are treated to a separate analysis here.

The earliest description of Mandan stops comes from Prince \citet{maximilian1839}, who lived among the Mandan people in the early 1830s. Though he transcribed Mandan using German orthography, the only voiced stop he consistently writes is $\langle$d$\rangle$ between vowels where [ɾ] appears today. This treatment of the flap as a stop is continued by the trader and translator \citet{kipp1852}. \citet{will1906} describe Mandan as having voiced and unvoiced stops in their grammatical sketch: /p b t d k ɡ/. \citeapos{kennard1936} grammar describes only one voiced stop, $\langle$d$\rangle$, which he describes as a pre-nasalized stop [ⁿd] that occurs in complementary distribution with [ɾ]. He has this to say about the quality of these stops: ``the series of stops are all slightly aspirated. The degree of aspiration varies with the position of the sound, being more pronounced in initial and medial position than final. Acoustically, the aspiration is stronger when the stop precedes a close vowel than an open, although each represents only one phoneme'' \citep[2]{kennard1936}.

\citet{torres2013a} conducts a phonetic investigation of the quality of stops in Mandan, using a 30-minute recording from the 1980s of Mrs. Otter Sage, along with a transcription produced by \citet{trechteretal2009} as a corpus. \citeauthor{torres2013a} finds that the stops in Mandan are clearly voiceless, with the measured voice onset times (VOTs) appearing below.

\begin{table}
\begin{center}
\caption{Measured means: VOT per \citet[29]{torres2013a}}
\label{torresVOT}
\begin{tabular}{lrrr}
\lsptoprule
{Stop}	&	{Number of tokens} 	&	{Mean (ms)}	&	{St.Dev.}\\
\midrule
{/p/}	&	35	&	12.8	&	11.4\\
{/t/}	&	255&	10.9	&	6.1\\
{/k/}	&	562&	25.8	&	13.3\\
\lspbottomrule
\end{tabular}
\end{center}
\end{table}



My own findings conform to those in \citet{torres2013a} in that stops are unaspirated. Under \citeapos{choladefoged1999} typology of aspiration, all of these stops fall into the expected range for being unambiguously unaspirated, where VOT values of approximately 30ms form the upper bounds of what most languages treat as a voiceless unaspirated stop. We can compare these numbers with those of a Siouan language that contrasts aspirated and unaspirated stops like Lakota, where we can see that Mandan's stops clearly pattern as plain stops.

\begin{table}
\caption{Mean (ms) VOT for Lakota stops \citep{torres2013b}}
\label{torresVOT2}
\begin{tabular}{lrrr}
\lsptoprule
{Stop}			&	{Bilabial}	&	{Alveolar}	&	{Velar}\\
\midrule
{plain}			&	13				&	14				&	31\\
{aspirated}	    &	94				&	95				&	114\\
\lspbottomrule
\end{tabular}

\end{table}

\citet{torres2013a,torres2013b} conclusively demonstrates that the supralaryngeal stops in Mandan are unaspirated. What she does not look at, however, is to what degree these stops are voiced.

Analysis of recorded data from Mandan speakers born prior to 1940 shows that there is no allophonic voicing of these plosives in intervocalic environments. Contra \citeapos{kennard1936} observations, there is no perceptually significant aspiration, i.e., stops are unaspirated. To illustrate this behavior of both singleton stops and those in clusters, examples of /t/ in the full range of possible stop environments appear below: word-initially, word-finally, intervocalically, and elements in a consonant cluster.

In \figref{wordinitialT} below, we can see that the word-initial /t/ in the word \textit{túroote} `must be some' has only a slightly positive VOT. For this particular word, the VOT is 9.1ms, which is lower than the average of 12.2ms for word-initial /t/, but still within a single standard deviation of 5.6ms \citep[29]{torres2013a}. The intervocalic /t/ has an even shorter VOT of 7.8ms. For the word \textit{íkimaapet} `downward' in \figref{wordfinalT}, where /t/ appears word-finally, there is no perceptable VOT between the release of the /t/ and the following segment. %This zero VOT is well below the mean for word-final /t/.
\footnote{These figures are created in Praat, version 6.4.12 \citep{boersmaweenik2016}, using \citeauthor{garcia2022}'s (\citeyear{garcia2022}) Praat script for creating customizable Praat pictures}

\begin{figure}
\caption{\#t and {\longrule}t{\longrule} in \textit{túroote} (MG)}\label{wordinitialT}
\includegraphics[scale=0.55]{figures/turoote2.png}
\end{figure}

In addition to demonstrating the lack of aspiration, contra \citeapos{kennard1936} description, the examples below act as a representative example of the behavior of voicing with respect to supralaryngeal stops in Mandan. Though the quality of certain recordings makes analysis difficult due to background noise or issues with older recording equipment that cause a band of energy to appear on the spectrogram where the voicing bar should be, we can compare this band with the lack of periodicity to any stops on the accompanying waveform to show that stops in Mandan are robustly voiceless, even in intervocalic environments.


\begin{figure}
\caption{t{\#} in \textit{íkimaapet} (MG)}\label{wordfinalT}
\includegraphics[scale=0.55]{figures/ikimaaNpet2.png}
\end{figure}






\begin{figure}
\caption{/Ct/ cluster in \textit{kúupatkush} (EB)}\label{CTcluster}
\includegraphics[scale=0.55]{figures/kuupatkush2.png}
\end{figure}


\begin{figure}
\caption{/tC/ cluster in \textit{ptáhta} (EB)}\label{TCcluster}
\includegraphics[scale=0.55]{figures/ptahta2.png}
\end{figure}


We see no significant bursts after /t/ in Figures \ref{CTcluster} and \ref{TCcluster} to signify aspiration, regardless of what position the /t/ takes. In \figref{CTcluster}, where we have a cluster-final and word-final /t/ in \textit{kúupatkush} `just seven', we see a VOT of 16.8ms for the cluster-final /t/. In \figref{TCcluster}, the VOT has a value of 16.2ms. Both of these values are in line with other values for unaspirated /t/ in \citet[219]{choladefoged1999}, as well as the data presented in \citet{torres2013a}.

When looking at materials written by Mandan speakers themselves, there are inconsistencies in the ways that they have recorded these stops. The acoustic analyses presented above show that the supralaryngeal plosives /p t k/ in Mandan are always voiceless and are on average without perceivable aspiration. This factor can cause L1 English-speakers to interpret these sounds as voiced. In pedagogical materials used at the school in Twin Buttes, ND or at the tribal college in New Town, ND, these sounds have often been written out as $\langle$b d g$\rangle$ by L1 instructors and heritage learners (e.g,. \citealt{littleowl1992} and \citealt{benson2000}).

The singleton supralarygneal stops can appear in any position within a syllable or word. This distribution is visible in \tabref{suprastops}.

\begin{table}
\caption{Supralaryngeal stops}\label{suprastops}
\begin{tabular}{rlll}
\lsptoprule
                   \multicolumn{4}{c}{/p/}                      \\
\midrule
{\#{\longrule}}: & [ˈpo]      & \textit{pó}       & `fish'       \\
V{\longrule}:   & [ˈkuːpa]   & \textit{kúupa}    & `seven'      \\
{\longrule}:    & [ˈnũp]    & \textit{núp}      & `two'        \\
\midrule
                   \multicolumn{4}{c}{/t/}                      \\
\midrule
{\#{\longrule}}: & [ˈtoːhe]   & \textit{tóohe}    & `blue/green' \\
V{\longrule}:   & [ˈoti]     & \textit{óti}      & `house'      \\
{\longrule}:    & [ˈãʔt]    & \textit{ą́'t}     & `that one'   \\
\midrule
                   \multicolumn{4}{c}{/k/}                      \\
\midrule
{\#{\longrule}}: & [ˈkoːɾe]   & \textit{kóore}    & `squash'     \\
V{\longrule}:   & [ˈpsãːka] & \textit{psą́ąka} & `frog'       \\
{\longrule}:    & [ˈsuk]     & \textit{súk}      & `child'     \\
\lspbottomrule
\end{tabular}
\end{table}

A detailed account of possible consonant clusters is provided in \sectref{consonantclusters}.

There is no allophony for singleton supralaryngeal stops based on their position within a word or syllable. When pseudo-geminate clusters arise through compounding or other morphological operations, however, lenition of the first segment occurs to prevent surface [C\textsubscript{i}C\textsubscript{i}] sequences. See \sectref{geminatedissimilation} for further explanation of pseudo-geminate dissimilation.

\subsubsection{Glottal stop}\label{glottalstop}

One sound that is often omitted from transcriptions in the corpus is the glottal stop /ʔ/. This omission is due to the fact that words containing /ʔ/ are conflated with having a long vowel instead of a /Vʔ/ sequence, or because there are difficulties perceiving the glottal stop when word-final (\citealt{hollow1970}, \citeauthor{boyle2007} p.c.). This subsection aims to demonstrate that the glottal stop occurs more often than described in previous work, and is not deleted word-finally, contrary to \citet[43]{hollow1970}.

The glottal stop /ʔ/ is a distinctive phoneme in Mandan, and is often found as the first element of consonant clusters, most notably the allocutive markers, e.g., the sentence-final male-addressee indicative marker =\textit{o'sh} or its female-addressee counterpart =\textit{o're}. The /ʔ/ can only appear as a coda element, and can occur word-finally, as seen in the first two examples in (\ref{codaglottals}) below, or as the first segment in a cluster, as seen in the remaining examples in (\ref{codaglottals}). There is no word-initial /ʔ/ in Mandan, nor does it appear as the second element of a cluster.
\begin{exe}

\newpage
\item\label{codaglottals} Coda glottal stops\footnote{Several of these underlying forms involve the stem vowel /=E/. This enclitic features the ablaut vowel /E/, which is explained later on in \sectref{ablaut}.}

\begin{xlist}

\item 	\textit{pí'}\\
	{[}ˈpiʔ]\\
        /piʔ/\\
	`liver'

\item 	\textit{ké'}\\
    {[}ˈkeʔ]\\
    /keʔ/\\
    `to dig'

\item 	\textit{mí'he}\\
	{[}ˈmĩʔ.he]\\
        /wĩʔh=E/\\
	`shawl'

\item 	\textit{sé'sh}\\
    {[}ˈseʔʃ]\\
    /se=oʔʃ/\\
    `it is red'

\item 	\textit{shí're}\\
        {[}ˈʃiʔ.ɾe]\\
        /ʃi=oʔɾe/\\
	`it is good'

\end{xlist}

\end{exe}

The only position where /ʔ/ does not appear is word-initially or root-initially. Phonemic /ʔ/ is only found as an element of a syllable coda. Glottal stops are possible intervocalically, but such segments are epenthetic and a repair mechanism to prevent hiatus. This kind of epenthesis is discussed further in \sectref{epentheticprocesses} and \sectref{glottalstopmetathesis}.

In previous descriptions of Mandan, marking of the glottal stop is sporadic, with \citeauthor{hollow1970} (\citeyear{hollow1970,hollow1973a,hollow1973b}) and \citet{hollow1976} being the most consistent. However, \citeauthor{hollow1970} frequently does not mark word-final glottal stops, even going so far as to propose a rule to delete glottal stops in word-final environments. However, word-final /ʔ/ is typically present, as shown below for \textit{wará' oráakini} `and he built a fire.' The /ʔ/ in \textit{wará'} `fire' is weakly present before the initial /o/ in \textit{oráakini} `he built it', manifesting with drastically reduced closure and causing the preceding vowel to have creaky voice. This glottal stop is realized by a reduction in amplitude and periodicity, but phonation is still taking place, which accounts for why it is still voiced (see \figref{AEglottals1}). The adduction of the glottal folds is insufficient in this lenited position to completely block expiration, which causes the /ʔ/ to have an almost approximant-like appearance on the waveform and spectrogram. %While it is true that the glottal closure is not always one manifestation of a word-final /ʔ/, there is always some phonetic cue that it is there.

The examples   in \figref{AEglottals1} and \figref{OSglottals1} highlight an important and overlooked fact about /ʔ/ in Mandan: phonetically, glottal stops have a large variation in how they are realized. Glottal stops are often weakly produced, as exemplified in \figref{AEglottals1}, where the glottal stop appears between two vowels and manifests are a drop in the frequency of the adjacent vowel formants and the voicing is present due to the lack of a complete glottal closure.

\begin{figure}

\caption{Word-final /ʔ/ in \textit{wará' oráakini} (AE)}\label{AEglottals1}
\includegraphics[scale=0.55]{figures/waraPoraakini2.png}
\end{figure}

Glottal stops can also be realized as the expected complete closure at the glottis, as we see in \figref{OSglottals1}   with the word \textit{ní'ro'sh} `he climbed.'  In this word, the first glottal stop is a complete closure. The second glottal stop, however, is shorter in the duration of its closure, so some of the phonation from the preceding vowel carries through, creating an echo vowel.

\begin{figure}
\caption{Glottal stops in \textit{ní'ro'sh} (OS)}\label{OSglottals1}
\includegraphics[scale=0.55]{figures/niProPshOS2.png}

\end{figure}

Other possible realizations of /ʔ/ appear in the word \textit{rá'kakshe'sh} `you met him', shown   in \figref{OSglottals2}. Like in \figref{OSglottals1} above, there is an instance of underlying /Vʔ/ manifesting as [VʔV]. However, the glottal stop itself has a low degree of closure. The second /ʔ/ similarly is not a complete stop, but it has the added effect of causing the preceding /e/ to become creaky voiced do to anticipatory co-articulation of the adducting glottal folds.

\begin{figure}
\caption{Glottal stops in \textit{rá'kakshe'sh} (OS)}\label{OSglottals2}
\includegraphics[scale=0.55]{figures/raPkakshePshOS2.png}

\end{figure}

\citet[94]{pierrehumberttalkin1992} state that the variations in the realization of /ʔ/ in other languages they have investigated are unsurprising, noting that a complete obstruction of airflow at the glottis for a glottal stop is quite unusual. Their analysis shows that it is typologically expected that languages feature this degree of variation in the production of /ʔ/, so Mandan is not typologically unusual with respect to its treatment of glottal stops.\footnote{Of particular note are Mayan languages, which likewise can have /ʔC/ clusters that have a wide variation of realizations \citep{baird2010, bennettTA}.}

Overall, /ʔ/ has a variety of realizations, which is likely the reason why so many previous researchers have either inconsistently transcribed it or have omitted it altogether. The one observation that holds across all descriptions of the glottal stop in Mandan, however, is that it is restricted to coda positions in surface representations. Moreover, these glottal stops are salient to the phonology of Mandan in that they contribute to syllable weight and thus affect the placement of primary stress (see \sectref{glottalstopmetathesis} for further discussion of how the introduction of /ʔ/ can cause long vowels to truncate and \sectref{intrusiveR} for explanation how epenthetic [ɾ] resolves hiatus in the environment of both long vowels and syllables with coda glottal stops).

\subsection{Affricates}\label{affricates}

Mandan has been described as having a single affricate, /tʃ/, in three previous works: \citet{will1906}, \citet{kennard1936}, and \citet{mixco1997a}. \citet[14]{hollow1970} does not mention this sound in his description of Mandan phonology in the first chapter of his dissertation and does not address why he omits it, given that he frequently cites \citeapos{kennard1936} interpretations of what functions particular pieces of morphology play. The /tʃ/ is sparsely attested within \citeapos{kennard1934} texts, and \citet[190]{will1906} even note that this sound is seldom heard. Despite this sound being described in other works, I argue here that there is no /tʃ/ in Mandan, though a heterosyllabic [t.ʃ] cluster is possible.

\citet[190]{will1906} list /tʃ/ in their sketch of Mandan, using the digraph $\langle$tc$\rangle$, as was convention for Americanist transciption at the time. The words they compiled consist of those they personally collected from Mandan consultants while doing archeological fieldwork on the reservation, as well as several words copied from \citeapos{maximilian1839} wordlist. While they do state that this sound is poorly attested, upon further analysis, we can see in (\ref{ExWillSpindenTC}) that all instances of $\langle$tc$\rangle$ in their survey of Mandan vocabulary are typically the result of the failure to hear a word break or mistaking adjacent /k/ and /ʃ/ segments for /tʃ/. The original transcirptions by \citeauthor{will1906} appear in angled brackets, followed by orthographic representation used in this book, then the surface and underlying representations of each word, and finally the gloss.

\begin{exe}

\item\label{ExWillSpindenTC} Examples of $\langle$tc$\rangle$ in \citet{will1906}

\begin{xlist}
\item $\langle$cantcuke$\rangle$\\
    \textit{shų́tkshuke}\\
    {[}ˈʃũt.kʃu.ke]\\
    /ʃũt\#kʃuk=E/\\
    `muskrat (lit. narrow tail)'

\item $\langle$hirutcote$\rangle$\\
        \textit{hirútshote}\\
	{[}ˈhⁱɾut.ʃo.te]\\
	/hɾut\#ʃot=E/\\
	`grey fox'

\item $\langle$antcihc$\rangle$\\
        \textit{ą́'t shí'sh}\\
	{[}ãʔt ˈʃiʔʃ]\\
	/ãʔt ʃi=oʔʃ/\\
	`it is good'

\item $\langle$Mantaktcuka$\rangle$\\
        \textit{Máatah Kshúke}\\
	{[}ˈmãː.tah.kʃu.ke]\\
	/wãːtah\#kʃuk=E/\\
	`Little Missouri River'

\end{xlist}

\end{exe}

\citet{kennard1936} is the first researcher to accumulate large amounts of Mandan data, depositing around three hundred typed pages of Mandan narratives and English free translations at the American Philosophical Society. He is also the first to describe the grammar of Mandan at any length. In his grammar, he refers to the suffix /-tʃ/ as the intentive marker. Like \citet{will1906}, he uses the digraph $\langle$tc$\rangle$ to mark the affricate /tʃ/, though it is also unclear when $\langle$tc$\rangle$ is an affricate and when it is a sequence of /t/ followed by /ʃ/.

For the most part, /tʃ/ is found almost only in what \citet[19]{kennard1936} calls the intentive marker $\langle$-tc$\rangle$, e.g., $\langle$ma$'$mąkotc$\rangle$ `I'll be there.' I suggest that this intentive maker Kennard was hearing was not actually [tʃ], but some casual speech phenomenon where the sequence of the modal \textit{=kt} and the allocutive marker \textit{=o'sh} were sometimes not clearly articulated at the end of an utterance. In the recordings I have, it is very common for speakers to cease phonation leading up to sentence-final morphology like the male-addressee declarative marker \textit{=o'sh} or its female-addressee counterpart \textit{=o're}. In fact, since there are no instances of /tʃ/ found in any of the recordings I have personally collected or in recordings collected by others, I discount the notion that there is an affricate /tʃ/ in Mandan and furthermore claim that all such instances of it in \citet{kennard1934,kennard1936} are all instances of misanalysed consonant clusters or typographic error.

To demonstrate this, I appeal to \citeapos{kennard1934} set of 28 narratives, which he elicited and transcribed in the early 1930s. \citet{hollow1973b} later re-elicited 24 of those same texts in the late 1960s and early 1970s. The fact that we have two sets of texts that are mostly identical provides us with an excellent means to evaluate which transcription is more faithful to the spoken Mandan each researcher was recording. In every instance where \citeauthor{kennard1936} marks this sound, \citeauthor{hollow1970}'s re-elicitation has the potential marker \textit{=kt} plus the masculine indicative marker \textit{=o'sh} instead. Examples of the relationship between \citeauthor{kennard1936}'s /tʃ/ and \citeauthor{hollow1970}'s potential mood marker \textit{=kt} followed by the male-addressee declarative marker \textit{=o'sh} appear in (\ref{noTS1}) below. The original translations from \citeauthor{kennard1934} remain as-is, and the relevant segments are shown in bold.\footnote{I have altered the transcription from \citet{hollow1973b} to fit the Mandan orthography used throughout this work due to the fact that \citeauthor{hollow1970}'s work is much more thoroughly transcribed than \citeauthor{kennard1936}'s, as well as the fact that the point of this portion is to demonstrate that all of \citeauthor{kennard1936}'s $\langle$tc$\rangle$ are really typographic or perception errors.}


\begin{exe}
\item\label{noTS1} Kennard's $\langle$tc$\rangle$ as /=kt=oʔʃ/

	\begin{xlist}

	\item\label{noTS1a} $\langle$wahú:ki kahó:nihàrani mánanòt\textbf{kisotc}.$\rangle$\\
			`If anyone comes, fall over and strike me.' \citep[257]{kennard1934}

\item\label{noTS1b}
				\glll	Wáahuuki kahóoniharani ~ ~ ~ ~ ~ ~ ~ ~ ~ ~ manárootki\textbf{kto'sh}.\\
				waa-huu=ki ka-hoo-rį-hrE=rį ~ ~ ~ ~ ~ ~ ~ ~ ~ ~ w-ra-rootki=\textbf{kt=o'sh}\\
				\textnormal{someone}-\textnormal{come.here}=\textsc{cond} \textsc{ins.frce}-\textnormal{fall}-\textsc{2s-caus=ss} ~ ~ ~ ~ ~ ~ ~ ~ ~ ~ \textsc{1s-2a}-\textnormal{hit}=\textsc{\textbf{pot=ind.m}}\\
				\glt	`If someone comes, you should fall and hit me.' \citep[62]{hollow1973b}

	\end{xlist}

\end{exe}

The $\langle$-kisotc$\rangle$ in (\ref{noTS1a}) is probably a misperceived \textit{=kt=o'sh}. Other instances in which we see /tʃ/ in \citet{kennard1934} are when there is a typographic error, as seen in the case below where a /t/ that should follow after the /ʃ/ is written as if they were an affricate. The passage in (\ref{noTS2}) below is an excerpt from \citeapos{kennard1934} elicitation of the tale ``Old Woman's Grandson,'' followed by \citeapos{hollow1973b} re-elicitation.\footnote{There is only a single word difference (i.e., \textit{hékarani} `they looked at it and') between these two versions, but the material relevant to demonstrating that there is no /tʃ/ in Mandan still stands.} Kennard's free translation is preserved, though I have altered Hollow's. Relevant segments are shown in bold.

\begin{exe}

\item\label{noTS2}

	\begin{xlist}

	\item $\langle$iwahúrɛ rátirika cíhɛrɛk ų́ka tákaha kiwára\textbf{tcų̀k}i ų́ka \\
	 orúsanakɛrɛròmakoc$\rangle$\\
	`The bones were scattered. After they burned to ashes again, they left it.' \citep[275]{kennard1934}

	\item \glll íwahuure rá'tirikaa shíherek, ~ ų́'ka ~ hékarani kiwará'\textbf{shųt}ki, ų́'ka ~ ~ ~ ~ ~ ~ ~ ~ órusanahkereroomako'sh\\
		i-wa-huu=E ra'-trik=E=$\varnothing$ shi-hrE=ak ~ ų'ka ~  hE=krE=rį ki-wra'-\textbf{shųt}=ki ų'ka ~ ~ ~ ~ ~ ~ ~ ~ o-ru-srąk=krE=oowąk=o'sh\\
		\textsc{3poss-unsp}-\textnormal{bone}=\textsc{sv} \textsc{ins.fire}-\textnormal{be.powdery}=\textsc{sv=cont} \textnormal{good}-\textsc{caus=ds} ~ \textnormal{be.farther} ~ \textnormal{see}=\textsc{3pl=ss} \textsc{mid}-\textnormal{fire}-\textbf{\textnormal{tail}}=\textsc{cond} \textnormal{be.farther} ~ ~ ~ ~ ~ ~ ~ ~ \textsc{pv.loc-ins.hand}=\textnormal{short}=\textsc{3pl=narr=ind.m}\\
	\glt `His bones having been nicely burnt to a fine powder, they then looked at it, and when it [his bones] became ashes, they then left it there.' \citep[136]{hollow1973b}

	\end{xlist}

\end{exe}

In the data above, \citeapos{kennard1934} $\langle$kiwáratcų̀ki$\rangle$ is really \textit{kiwára'shųtki} `when [his bones] became ashes', where `ash' is literally `fire's tail.' At the boundary between the two words in the compound \textit{wára'} `fire' + \textit{shų́t} `tail', we see a heterosyllabic [ʔ.ʃ] cluster. This cluster is misperceived by Kennard as [tʃ]. All instances of $\langle$tc$\rangle$ in \citet{kennard1934,kennard1936} can thus be explained as surface clusters of [tʃ] stemming from morphologically complex words, misperceptions of a voiceless stop followed by [ʃ], or misperceptions of the modal \textit{=kt} followed by the male-addressee indicative marker \textit{=o'sh}.

\citet[26]{mixco1997a} gives only one example sentence with /tʃ/ in his grammar, and it is not clear if it is from his own field work or from \citet{kennard1934}. Furthermore, there simply are no examples of it in any of the recordings analyzed for this book, which includes speakers born between the 1860s and the 1960s. Given the fact that every instance of $\langle$tc$\rangle$ in \citet{kennard1934,kennard1936} equates with either \textit{=kt} plus \textit{=o'sh} in the \citet{hollow1973b} re-elicitations, a cluster consisting of /t/ and /ʃ/, or simply a misperception of a voiceless stop followed by /ʃ/, it is just not the case that /tʃ/ is present in Mandan, despite being reported in previous works. Mandan has no affricates.

\subsection{Fricatives}\label{fricatives}

While Proto-Siouan had two different fricative series, a plain fricative and a glottalized fricative, Mandan has only one \citep{rankinetalnd}. Previous phonetic work on Mandan consonants has focused solely on the supralaryngeal stops (e.g., \citealt{torres2013a}), and as such, the fricatives merit additional attention, which is given here.

\subsubsection{Supralaryngeal fricatives}\label{suprafrics}

All published sources that give a description of the sound system of Mandan agree on the inventory of supralaryngeal fricatives: /s ʃ x/. The singleton fricatives can appear in the onset or coda within a syllable or word. This distribution is visible in \tabref{suprafrictable}.

\begin{table}
\caption{Supralaryngeal fricatives}\label{suprafrictable}
\begin{tabular}{rlll}
\lsptoprule
                   \multicolumn{4}{c}{/s/}                      \\
\midrule
{\#{\longrule}}: & [ˈ\textbf{s}i] &\textit{\textbf{s}í} &`feather'\\
V{\longrule}:   & [ˈo.\textbf{s}u] &\textit{ó\textbf{s}u} &`hole'      \\
{\longrule}:    & [ˈpu\textbf{s}] &\textit{pú\textbf{s}} &`cat'\\
\midrule
                   \multicolumn{4}{c}{/ʃ/}                      \\
\midrule
{\#{\longrule}}: & [ˈ\textbf{ʃ}i] &\textit{\textbf{sh}í} &`foot'\\
V{\longrule}:   & [ˈo.\textbf{ʃ}ᵉɾop] &\textit{ó\textbf{sh}erop} &`swallow'\\
{\longrule}:    & [ˈtku\textbf{ʃ}] &\textit{tkú\textbf{sh}} &`real, true'\\
\midrule
                   \multicolumn{4}{c}{/x/}                      \\
\midrule
{\#{\longrule}}: & [ˈ\textbf{x}ih] &\textit{\textbf{x}íh} &`old'\\
V{\longrule}:   & [ˈoː.\textbf{x}a] &\textit{óo\textbf{x}a} &`fox'\\
{\longrule}:    & [ˈko\textbf{x}] &\textit{kó\textbf{x}} &`buzz'\\
\lspbottomrule
\end{tabular}
\end{table}

Root- and affix-internally, supralaryngeal fricatives can be elements in consonant clusters, either as the first or second segment in the cluster. To demonstrate this distribution, instances of /x/ will be given for all five possible positions: word-initial, word-final, intervocalic, cluster-initial, and cluster-final. This fricative is used as an exemplar due to its high frequency within the corpus.\footnote{The velar fricative /x/ is impressionistically much longer than the other fricatives, } As was the case in \sectref{supraplosives}, the purpose of \figref{initialX}, \figref{finalX}, and \figref{intervocalicX}  is to demonstrate that fricatives likewise do not display any voicing assimilation, regardless of their environment.

In the example in \figref{initialX}, we see \textit{xíko'sh} `it is bad (male addressee).' This figure shows that /x/, like all fricatives, is able to appear word-initially. In addition, this figure visibly shows the fact that the /x/ is voiceless. We see a similar behavior for word-final /w/ in the word \textit{miníx} `play' in \figref{finalX} below.


\begin{figure}

\caption{\#x in \textit{xíko'sh} (OS)}\label{initialX}

\includegraphics[scale=0.55]{figures/xikoPshOS2.png}
\end{figure}



\begin{figure}
\caption{x{\#} in \textit{miníx} (EB)}\label{finalX}
\includegraphics[scale=0.55]{figures/minixEB2.png}

\end{figure}

Looking at the examples above, we can see that /x/ maintains its voicelessness even in non-peripheral positions. Similar to the behavior of plosives, the /x/ in \textit{\'{ı̨}įxa} `alone' does not undergo voicing assimilation intervocalically, as shown in \figref{intervocalicX}.\footnote{The voiceless velar fricative in Mandan appears to be in free variation with a uvular realization. This sound occurs most often as [x], but it can be articulated as [χ] without a perceived difference by speakers. Due to its status as the most frequent manifestation of this fricative, I uniformly transcribe this sound as /x/ in the present work.}


\begin{figure}[t]
\caption{{\longrule}x{\longrule} in \textit{\'{ı̨}įxa} (EB)}\label{intervocalicX}
\includegraphics[scale=0.55]{figures/iiNxaEB2.png}
\end{figure}

The /x/ is a common element in consonant clusters, as we can see in the word \textit{xtą́ąte} `thunderbird' in \figref{XCcluster} and the word \textit{íxkąhta} `laugh!' in \figref{CXcluster}. Both of these figures exemplify that /x/ retains its characteristic lack of voicing, a behavior shared with other supralaryngeal fricatives.

To date, there has been no extensive phonetic analysis done on supralaryngeal fricatives in Mandan, or fricatives in general, for that matter. Additional work is needed to verify the impressionistic descriptions above regarding the frequency of these segments in the corpus (i.e., how much more frequent is /x/ than other fricatives), as well as what kind of durational differences there are, given the observation that /x/ seems much longer than other fricatives. The investigation of these points are outside of the scope of this book, but are worthy topics of investigation in the future to add to the typology of sound frequency and address the question why some fricatives are perceptibly longer than others.


\begin{figure}

\caption{/xC/ in \textit{xtą́ąte} (AE)}\label{CXcluster}
\includegraphics[scale=0.55]{figures/xtaaNteAE2.png}
\end{figure}

\begin{figure}
\caption{/Cx/ in \textit{íkxąhta} (EB)}\label{XCcluster}
\includegraphics[scale=0.55]{figures/ikxaNhtaEB2.png}

\end{figure}

The discussion above described the quality and behavior of supralaryngeal fricatives. Mandan also has a glottal fricative, /h/, which is treated in the subsection below.

\subsubsection{Glottal fricative}\label{glottalfric}

The glottal fricative /h/ is a frequently encountered sound in Mandan. This fricative commonly appears in word-initial position, as we see in the first three examples in (\ref{codaH}) below. An /h/ can appear word-finally as well, which we can see in the final three examples in (\ref{codaH}). Previous researchers omit or sporadically transcribe word-final /h/ in the corpus, and \citet[43]{hollow1970} even argues that word-final /h/ is deleted. This subsection serves to show that this is not the case, and /h/ is always phonetically present, though it can have variable realization, leading to researchers interpreting it as not being there.

\begin{exe}

\item\label{codaH} Distribution of /h/

\begin{xlist}

\item 	\textit{Hereróoka}\\
	{[}hᵉɾe.ˈɾoː.ka]\\
        /hɾeɾoːka/\\
		`Crow tribe'

\item 	\textit{hahó}\\
	{[}ha.ˈɦo]\\
        /haho/\\
	`thank you'

\item 	\textit{hų́ų}\\
	{[}ˈhũː]\\
        /hũː/\\
	`yes'

\item 	\textit{xą́h}\\
	{[}ˈxãh]\\
    /xãh/\\
	`grass'

\item 	\textit{ráahta}\\
	{[}ˈⁿdaːh.ta]\\
        /ɾEːh=ta/\\
	`go there!'

\item 	\textit{istų́h}\\
        /istũh/\\
        {[}i.ˈstũh]\\
	`night'

\end{xlist}

\end{exe}

An /h/ is able to appear in consonant clusters. Any clusters involving /h/ are due to affixation or compounding, with the exception of /ʔh/ clusters, which do occur in non-decomposable stems. In this respect, the distribution of /h/ differs from that of other fricatives.

\citet[43]{hollow1970} states that /h/ is deleted word-finally and optionally before a consonant. However, in the recorded data being analyzed, instrumentation shows this claim to be false. We have already seen the /h/ in \textit{ráahta} `go there!' earlier in \figref{TCcluster}, and we can see that word-final /h/ is likewise preserved in \textit{míih} `woman' in \figref{finalH}.


\begin{figure}

\caption{Realization of h{\#} in \textit{míih} (OS)}\label{finalH}
\includegraphics[scale=0.55]{figures/miihOS2.png}
\end{figure}

One major issue with \citeauthor{hollow1970}'s interpretation of Mandan phonology is that he did not distinguish short and long vowels. In addition to this issue, he worked with elderly speakers whose normal cadence tended to be more creaky or breathy, which can obfuscate these coda glottal segments. However, using instrumentation, we can see that these elements are not deleted word-finally, but are always present.

The /h/ is unique in Mandan as being the only segment that becomes voiced intervocalically. An example of this appears in \figref{intervocalicH1} for the word \textit{shehéks} `the coyote.' However, this voicing assimilation only happens when /h/ precedes a syllable bearing primary stress. This behavior even takes place across word boundaries, as seen in Figures \ref{intervocalicH2} and \ref{intervocalicH3}, where we have two examples with intervocalic /h/: \textit{míih éexixte'na} `a pregnant woman' and \textit{máatah íwokahąą} `along the river edge.' In both examples, we have a word-final /h/ that is voiced. For both of these examples, this /h/ precedes a syllable bearing primary stress. In \figref{intervocalicH3}, we see an intervocalic /h/ in \textit{íwokahąą} `along the edge.' Here, the /h/ does not precede an onsetless syllable with primary stress and as such, it does not undergo voicing.



\begin{figure}
\caption{Intervocalic /h/ voicing in \textit{shehéks} (EB)}\label{intervocalicH1}
\includegraphics[scale=0.55]{figures/sheheksEB2.png}

\end{figure}


When not followed by a word that begins with a vowel, it is not always obvious when a speaker is producing a word-final /h/ due to the fact that many of the recorded speakers of Mandan have had rather breathy voices. This tendency towards breathy voice can obscure word-final /h/ for some listeners, which is what leads \citet[43]{hollow1970} to his conclusion that they are deleted, but phonetic instrumentation reveals that they are always present.


\begin{figure}

\caption{/h/ voicing in \textit{míih éexixte'na} (OS)}\label{intervocalicH2}
\includegraphics[scale=0.55]{figures/miiheexixteenaOS2.png}
\end{figure}


\begin{figure}
\caption{/h/ voicing in \textit{máatah íwokahąą} (OS)}\label{intervocalicH3}
\includegraphics[scale=0.55]{figures/maatahiwokahaaOS2.png}
\end{figure}

This subsection has shown that /h/ has a slightly different behavior than other fricatives when they are intervocalic and in the environment of primary stress, where they become voiced, [ɦ]. The subsection that follows deals with the only consistently voiced consonants in Mandan, sonorants.\footnote{Additional phonetic work is needed to determine whether the vowels before a coda /h/ are methodically assimilating the [$+$spread glottis] feature and picking up breathy voice rather than modal voice, or if this is simply a tendency observed when going through older recordings. The scope of voice differences on vowels is beyond the scope of the present work, but seeing if there is a parallel with the creaky voice observed on vowels with coda /ʔ/ is worth investigating.}

\subsection{Sonorants}\label{sonorants}

Of all the consonants in Mandan, the sonorants /ɾ w/ are among the most common. Both \citet[190]{will1906} and \citet[2]{kennard1936} list /w m ɾ n/ as being separate phonemes in their grammatical sketches. \citet[18]{hollow1970} later argues that all surface nasal consonants are the result of nasal harmony with a following nasal vowel. Any operation that might syncopate such an underlying nasal vowel results in [ɾ w] on the surface instead of [n m]. Subsequent researchers such as \citet{coberly1979}, \citet{carter1991a,carter1991b}, and \citet{mixco1997a,mixco1997b} all adopt this analysis. I likewise adopt this analysis and assume that all sonorants in Mandan are ultimately oral and are realized as nasal only due to the regressive nasal harmony that affects all voiced segments. Nasal harmony is discussed in greater detail in \sectref{nasalharmony}.

With this analysis in mind, nasal consonants are treated as allophones of their oral sonorant equivalents, i.e., only /ɾ w/ are phonemic in Mandan. Below in (\ref{sonorantnormal}), we can see examples of oral sonorants becoming nasal sonorants in the data below, where the first person active prefix /wa-/ and second person active prefix /ɾa-/ become nasalized due to the following nasal segments, becoming [mã-] and [nã-], respectively (see \sectref{nasalharmony} for discussion of nasal harmony, which targets vowels and sonorants).

\begin{exe}
\item\label{sonorantnormal} Examples of complementary distribution of oral and nasal sonorants

	\begin{xlist}
	\item\label{sonorantnormal1} \glll \textbf{wa}ráko'sh\\
		\textbf{wa}-rak=o'sh\\
		\textbf{\textsc{1a}}-\textnormal{bury}=\textsc{ind.m}\\
		\glt `I buried it'

	\item\label{sonorantnormal2} \glll í\textbf{ra}heko'sh\\
		i-\textbf{ra}-hek=o'sh\\
		\textsc{pv.ins-\textbf{2a}}-\textnormal{know}=\textsc{ind.m}\\
		\glt `you knew it' \citep[71]{hollow1970}

	\item\label{sonorantnormal3} \glll \textbf{ma}náasko'sh\\
		\textbf{wa}-rąąkE=o'sh\\
		\textsc{\textbf{1a}}-\textnormal{sit}.\textsc{pos.aux=ind.m}\\
		\glt `I always manage [to do it]' \citep[54]{hollow1973a}

	\item\label{sonorantnormal4} \glll ó\textbf{na}naake'sh\\
		o-\textbf{ra}-rąąkE=o'sh\\
		\textsc{pv.irr-\textbf{2a}}-\textnormal{sit}.\textsc{pos.aux=ind.m}\\
		\glt `you will be [one]' \citep[187]{hollow1973a}

	\end{xlist}
\end{exe}

In (\ref{sonorantnormal}) above, we see nasal [mã-] and [nã-] where we  otherwise expect to see oral [wa-] and [ɾa-]. This change is not allomorphic in nature, but is purely phonological, owing to the fact that these prefixes precede a syllable that contains a nasal vowel. The nasality spreads leftward, causing /w/ to become [m] and /ɾ/ to become [n]. Thus, all instances of nasal consonants are due to the influence of following vowels bearing underlying nasality. These sounds are predictable and are best described as being in complementary distribution.

In addition to its nasal allophone, /ɾ/ has a word-initial allophone where it fortifies to a prenasalized voiced stop, [ⁿd]. \citet[52]{hollow1970} describes this allophone, but does not transcribe any differences between [ɾ] and [ⁿd] in his dictionary or the narratives he recorded. Only \citet[3]{kennard1936} records [ɾ] versus [ⁿd] in his grammar and his transcribed narratives.

\begin{exe}

\item\label{prenasalizedR} [ɾ] versus [ⁿd] in \citet{kennard1936}

	\begin{xlist}

	\item\label{prenasalizedR1} $\langle$do$'$pxani ma$'$hisɛks 'ųc dapį$'$tka'\textsuperscript{ɛhɛ}$\rangle$

	\glll róopxani máah íseks, ų́'sh, ~ ~ ~ ~ ~ ~ ~ ~ ~ ~ ~ ~ ~ ~ ~ rap\'{ı̨}įtka'ehe\\
		roopxE=rį wąąh i-sek=s ų'sh ~ ~ ~ ~ ~ ~ ~ ~ ~ ~ ~ ~ ~ ~ ~ ra-pįįt=ka'ehe\\
		\textnormal{enter}=\textsc{ss} \textnormal{arrow} \textsc{pv.ins}-\textnormal{make}=\textsc{def} \textnormal{be.thus} ~ ~ ~ ~ ~ ~ ~ ~ ~ ~ ~ ~ ~ ~ ~ \textsc{ins.foot}-\textnormal{be.scattered}=\textsc{quot}\\
		\glt `he entered and made the arrows, so he scattered them with his foot, it is said' \citep[38]{kennard1936}

	\item\label{prenasalizedR2} $\langle$oma$'$papi\textsuperscript{$\backprime$}rək  si:$'$rɛna ɛ$'$na na$'$tka dukci$'$cka'\textsuperscript{ɛhɛ}$\rangle$

	\glll óo máapapirak síireena, éena ~ ~ ~ ~ nátka rukshíshka'ehe\\
		oo wąąpapirak sii=ee=rą ee=rą ~ ~ ~ ~ rąt=ka ru-kshish=ka'ehe\\
		\textsc{dem.mid} \textnormal{weasel} \textnormal{be.yellow}=\textsc{dem.dist=top} \textsc{dem.dist=top} ~ ~ ~ ~ \textnormal{be.in.middle}=\textsc{hab} \textsc{ins.hand}-\textnormal{worry}=\textsc{quot}\\
		\glt `That yellow weasel there worried him, it is said' \citep[38]{kennard1936}

	\end{xlist}

\end{exe}

Each instance of $\langle$d$\rangle$ in (\ref{prenasalizedR}) above appears only word-initially, while $\langle$r$\rangle$ occurs word-internally. This behavior is consistent with the phonological rule in \citet[52]{hollow1970}. This rule is not completely consistent, however, as in rapid speech, this word-initial /r/, i.e., [ⁿd], can alternatively be realized as [ɾ].\footnote{While working with Mr. Edwin Benson to elicit recordings for the Level 1 Mandan textbook, I would ask him to say an item three times in a row. Typically, the first time Mr. Benson would say a word beginning with /ɾ/, he would pronounce it as [ⁿd], but subsequent iterations often vacillated between [ɾ] and [ⁿd].} For speakers born before the twentieth century, there is also an analogous [ᵐb] for word-initial /w/. \citet{maximilian1839} and \citet{will1906} both transcibe this sound as $\langle$b$\rangle$.

\begin{exe}
\item\label{mbexample} Example of word-initial [ᵐb] in older Mandan in the word `charcoal'\footnote{A note on \citeauthor{maximilian1839}'s orthography: being a native German speaker, he transcribes what he hears using German orthographic principles. For example, he writes $\langle$ch$\rangle$ for [x], and vacillates between $\langle$e$\rangle$ and $\langle$ä$\rangle$ to express [e]. The doubling of the $\langle$ch$\rangle$ here indicates that he perceives the initial vowel to be short, which conforms to the same length this root has in modern Mandan.}
\begin{xlist}

\item $\langle$bä́chchä$\rangle$ \citep[236]{maximilian1839}\\{}[ˈᵐbexe]

\item \textit{wéxe} \citep[285]{hollow1970}\\{}[ˈwexe]

\end{xlist}

\end{exe}

In \citeapos{maximilian1839} description of Mandan, the oldest recorded source of Mandan language, this allophony is not consistent. Most words that begin with /w/ are transcribed with $\langle$w$\rangle$ instead of $\langle$b$\rangle$. One possibility is that this variation was optional, or perhaps more closely associated with one subgroup or dialect of the Mandan. \citeauthor{maximilian1839} does not elaborate on how he collected his data and specifically from whom, so the context for which he writes forms beginning with $\langle$b$\rangle$ versus those with $\langle$w$\rangle$ are left to conjecture. His word lists and paradigms only contain notes of what villages the speakers came from when there was a lexical or grammatical difference (i.e., Nuu'etaa versus Ruptaa). He did not comment on whether there was any difference in the pronunciations of words by one group or another.

In the corpus, the distribution of [ᵐb] for /w/ as a pre-nasalized stop parallels utterance-initial [ⁿd] for /ɾ/, or at least /ɾ/ when said after a long pause. This variant of /w/ is not fully productive nor fully predictable in speakers born during the twentieth century and as such, the present work does not treat [ᵐb] as a full allophone of /w/ in contemporary Mandan, with its status marginal at best. This variant does appear sporadically in spontaneous speech, so it may be treated as a possible variant of /w/ in word- or utterence-initial environments, though [ᵐb] seems to be dispreferred by the speakers for whom we have recordings.



When compared to other consonants in Mandan, the distribution of sonorants within the syllable is much more restricted. A sonorant can only appear as the sole segment in the syllable onset. This behavior is described in greater detail in \sectref{dorseyslaw}.



\tabref{sonoranttable}   summarizes the pattern in Mandan that there are no underlying nasal consonants, and that nasal consonants appear due to assimilation with a following nasal vowel.\footnote{The phonemes /w/ and /ɾ/ can also appear as [m] and [n] without an accompanying nasal vowel if they are utterance-initial. This is discussed further in \sectref{sonorantfortition}.} \citeauthor{hollow1970}'s (\citeyear[18]{hollow1970}) analysis holds, as [m] and [n] cannot appear without an accompanying nasal vowel. Furthermore, the word-initial prenasalized allophone [ⁿd] of /ɾ/ never appears when a word-initial /ɾ/ occurs with a following nasal vowel, i.e., the /ɾ/ in /ɾãtka/ `heart' will always be realized as [n] and never [ⁿd].  For the purpose of taking historical data into account, the [ᵐb] variant of /w/ is included, though its status as a true and conditioned allophone is marginal at best in modern Mandan. There are no instances of word-final /w/ or /ɾ/ in Mandan (see \sectref{dorseyslaw}), so no such examples appear in \tabref{sonoranttable} below.

\begin{table}
\caption{Distribution of /w/ and /ɾ/}\label{sonoranttable}
\begin{tabular}{lllllll}
\lsptoprule
	~&
	&
	Initial&
	~&
	&
	Intervocalic&
	~\\
\midrule

[w]&
	[ˈ\textbf{w}iː.pe]&
	\textit{\textbf{w}íipe}&
	`cornball'&
	[ˈãː.\textbf{w}e]&
	\textit{ą́ą\textbf{w}e}&
	`all'
	\\

$\dagger$[ᵐb]&
	[ˈ\textbf{ᵐb}e.xe]&
	\textit{\textbf{w}éxe}&
	`charcoal'&
	---&
	---&
	---\\

[m]&
	[ˈ\textbf{m}\textsuperscript{ĩ}nĩ]&
	\textit{\textbf{m}iní}&
	`water'&
	[ĩ.ˈ\textbf{m}ãː.ɾe]&
	\textit{í\textbf{m}aare}&
	`body'\\

[ɾ]&
	---&
	---&
	---&
	[ˈwᵃ\textbf{ɾ}aʔ]&
	\textit{{w}a\textbf{r}á'}&
	`fire'\\

[ⁿd]&
	[ˈ\textbf{ⁿd}eː.sik]&
	\textit{\textbf{r}éesik}&
	`tongue'&
	---&
	---&
	---\\

[n]&
	[ˈnã.tka]&
	\textit{\textbf{n}ą́tka}&
	`heart'&
	[ĩ.ˈ\textbf{n}ãk]&
	\textit{i\textbf{n}ák}&
	`again'\\
\midrule\hline

\end{tabular}

\end{table}

This distribution of sonorants with nasal vowels is noteworthy in that it is consistent with the description of the phonological system of Proto-Siouan proposed in \citet{rankinetalnd}, where all nasal consonants in modern Siouan languages can be reconstructed back to oral sonorants that come into contact with a nasal vowel. To my knowledge, Mandan is alone within the Siouan language family in preserving this archaic feature of Proto-Siouan phonology.

\section{Vowels}\label{SecVowels}
\largerpage
In \sectref{consonants}, I noted that there has been very little attention paid to the description of consonants in Mandan. Here, I explain that even less attention has been paid to the quality of its vowels. This section serves to provide a summary of the vowel inventory in Mandan.

The Mandan vowel system is quite similar to that of other Siouan languages in that there are five vowels that contrast by length: two high vowels, two mid vowels, and a low vowel. In addition to these oral vowels, there are three nasal vowels, all of which also contrast by length. The vowel inventory of Mandan is identical to that of Proto-Siouan in this respect \citep{rankinetalnd}.

\subsection{Monophthongs}

The distribution of oral and nasal vowels in Mandan appears in Figures \ref{oralvowelchart} and \ref{nasalvowelchart} below. All the vowels shown below have long counterparts, which are not depicted on these vowel quadrilaterals. Minimal and near-minimal pairs and quadruplets for vowel length and nasality likewise appear   in \tabref{vowelminimalpairs}. The cells for nasal mid vowels have been left blank, since Mandan does not have underlying nasal mid vowels nor does it permit nasal spreading onto mid vowels. This restriction against nasal mid vowels is shared by many other Siouan languages and is thought to be a holdover from Proto-Siouan, though \citet[27]{rood1983} posits that Pre-Proto-Siouan could have had nasal mid vowels, which later merged with an oral vowel later in the development of the language.

Figures \ref{oralvowelchart} and \ref{nasalvowelchart} provide idealized realizations of Mandan vowels. Vowels in Mandan do not vary significantly in quality based on position within a particular syllable shape (i.e., in an open syllable versus a closed syllable), nor when in the environment of a stressed syllable. In this respect, the production of Mandan vowels is quite consistent.


\begin{figure}
\subfigure[oral vowels\label{oralvowelchart}]{
\begin{vowel}
	\putcvowel[l]{i}{1}
	\putcvowel[r]{u}{8}
	\putcvowel[l]{e}{2}
	\putcvowel[r]{o}{7}
	\putcvowel[l]{a}{4}
	\end{vowel}
}
\subfigure[nasal vowels\label{nasalvowelchart}]{
\begin{vowel}
	\putcvowel[l]{ĩ}{1}
	\putcvowel[r]{ũ}{8}
	\putcvowel[l]{ã}{4}
\end{vowel}
}
\caption{Mandan vowels}
\end{figure}





\begin{table}
\caption{Contrastive vowel qualities}\label{vowelminimalpairs}
\begin{tabular}{lllll}
\lsptoprule
  & Oral short                                 & Oral long                        & Nasal short                                  & Nasal long                             \\
\midrule
a & \textit{raké'he} & \textit{ráakana}          & \textit{naké} & \textit{náake}           \\
~	& [ⁿda.ˈkeʔ.he] & [ˈⁿdaː.k\textsuperscript{ã}nã] & [nã.ˈke] & [ˈnãː.ke]\\
~&`be angry with'& `hail'&`breechcloth' & `be alive' \\
\tablevspace
e & \textit{éreh}& \textit{réeh} &--- &--- \\
~& [ˈe.ɾeh] & [ˈⁿdeːh] & ~& ~\\
~& `want' & `go there' & & \\
\tablevspace
i & \textit{sí} & \textit{síi} & \textit{s\'{ı̨}h} & \textit{s\'{ı̨}į}  \\
~& [ˈsi] & [ˈsiː] & [ˈsĩh] & [ˈsĩː]\\
~&`foot' &`yellow'&`be strong'&`tallow'\\
\tablevspace
o & \textit{kók} & \textit{kóo} &---& ---\\
~& [ˈkok] & [ˈkoː] & ~&~\\
~&`pronghorn'&`squash' & &  \\
\tablevspace
u & \textit{húpinih} & \textit{húu} & \textit{hų́}  & \textit{hų́ų} \\
~& [ˈhu.p\textsuperscript{ĩ}nĩh] & [ˈhuː] & [ˈhũ] & [ˈhũː]\\
~&`soup'&`bone' &`many'&`yes'\\
\lspbottomrule
\end{tabular}
\end{table}

In \tabref{vowelminimalpairs}, we see examples of the phonemic vowels found in Mandan. These monophthongs have no restriction preventing them from appearing in any kind of syllable (i.e., open or closed) or position within a word. The one restriction placed upon the distribution of vowels in Mandan is that there can be no VV sequences, i.e., hiatus is forbidden in Mandan. To my understanding, there have been no phonetic studies on Mandan vowel quality to date. As such, there is room to investigate the impressionistic observations discussed here regarding the invariability of vowel quality and the lack of allophony in future work on Mandan.



\subsection{Diphthongs}\label{subsectiondiphthongs}

In the description of monophthongs in Mandan above, I claim that VV sequences in Mandan are illicit. The purpose of the following subsection is to examine whether the aforementioned observation holds.

Mandan has no native lexical items containing a diphthong, and in cases where two vowels would otherwise come into contact due to compounding or affixation, there is some epenthetic process to prevent hiatus (see further discussion of hiatus resolution in \sectref{epentheticprocesses}). The sole exception to this restriction against diphthongs in the entire corpus is the word \textit{háu} [ˈhau̯] `hello, yes.'


\begin{figure}
\caption{Diphthong /au̯/ in \textit{háu} (OS)}\label{OSdiphthong}
\includegraphics[scale=0.55]{figures/hauOS2.png}

\end{figure}

\newpage
This word is not native to Mandan, and it is found in many other languages of the Plains.\footnote{The use of this item is not restricted to Plains peoples, but has been attested in the Great Lakes region as early as 1636, as the French Jesuit missionary and martyr Saint Jean de Brébeuf notes that the Wyandot (also known as the \ili{Huron} or the \ili{Wendat}, an Iroquoian-speaking people who historically occupied the land on the northern shores of Lake Ontario in what is now Canada) use this interjection to express approval or affirmation, as well as to punctuate that a speaker is finished speaking or to affirm that one has heard the speaker finish speaking \citep[146]{axtell1981}. The former is identical to how it is used in Mandan, though it is not clear that this term originated in \ili{Wyandot} either, as that language likewise has no diphthongs \citep[325]{julian2010}.} No diphthong is otherwise permitted in Mandan, so the acceptability of this diphthong is due to either its status as a loanword borrowed wholesale or its status as an interjection. With respect to the latter, it is typologically common for interjections to have anomalous phonology or morphology \citep[105]{ameka1992}, so it may be for this reason that \textit{háu} can have two differing vowels in a syllable nucleus while other lexical items cannot.

\subsection{Dorsey's Law excrescent vowels}\label{dorseyslaw}

There have been vowels in superscript represented in phonetic notation throughout this work. These superscript IPA characters represent excrescent vowels due to Dorsey's Law. Dorsey's Law inserts a copy vowel between two consonants. This phenomenon was introduced as a phonological rule in (\ref{chapter1dorseyslaw}) from Chapter \ref{chapter1}, which is reproduced below. In Mandan, this behavior is seen whenever a cluster has a sonorant as its second element.

\begin{exe}

\item\label{chapter1dorseyslawREDUX} \textbf{Dorsey's Law}\\
/CRV\textsubscript{1}/ $\to$ [CV\textsubscript{1}RV\textsubscript{1}]\\
	\textit{Insert a copy of the following vowel between a consonant-sonorant cluster.}

\end{exe}

For (\ref{dorseyslaw1a}) through (\ref{dorseyslaw1e}), we have simplex words with an excrescent vowel that shares certain features with a following vowel. For example, the datum in (\ref{dorseyslaw1a}), /wɾã/ `tree, wood', undergoes nasal harmony and then an intrusive vowel interrupts the cluster that contains a sonorant, yielding [m\textsuperscript{ã}nã]. This vowel is not syllabic: i.e, not *[mã.ˈnã]. For further elaboration on why these vowels in Mandan are not syllabic, see \sectref{stress}.

The data below show that this excrescent vowel is triggered whenever the conditions are created where a consonant cluster has a sonorant as its second element. This intrusive vowel appears both within a lexical root as well as when enclitics are added, as seen in (\ref{dorseyslaw1h}), where a Dorsey's Law vowel occurs both within the lexical root /sɾãh/ `leave' and the cluster created with the same-subject switch reference marker /=ɾĩ/ encliticized onto the stem. Dorsey's Law vowels are produced within a root and across affix and enclitic boundaries. However, the data in (\ref{dorseyslaw1i}) through (\ref{dorseyslaw1k}) show that no intrusive vowels appear across a word boundary in the case of compounds.

\begin{exe}
\item\label{dorseyslaw1} Dorsey's Law vowels in Mandan
	\begin{xlist}
	\item\label{dorseyslaw1a} /wɾã/ $\to$ [ˈm\textsuperscript{ã}nã] `tree, wood'
	\item\label{dorseyslaw1b} /wɾaʔ/ $\to$ [ˈw\textsuperscript{a}ɾaʔ] `fire'
	\item\label{dorseyslaw1c} /ʃɾuk/ $\to$ [ˈʃ\textsuperscript{i}ɾuk] `be wise, well-behaved'
	\item\label{dorseyslaw1d} /ʃɾeːk/ $\to$ [ˈʃᵉɾeːk] `warhoop'
	\item\label{dorseyslaw1e} /paxɾuːk/ $\to$ [pa.ˈxⁱɾuːk] `corn silk'
	\item\label{dorseyslaw1f} /ɾa-ɾEːh=ɾĩt=oʔʃ/ $\to$ [ⁿda.ˈɾaː.h\textsuperscript{ĩ}nĩ.toʔʃ] `you all went there'
	\item\label{dorseyslaw1g} /kʃãt=ɾã/ $\to$ [ˈkʃã.t\textsuperscript{ã}nã] `watch outǃ'
	\item\label{dorseyslaw1h} /ɾo-ɾa-ɾu-sɾãh=ɾĩ/ $\to$ [ⁿdo.ˈɾa.ɾu.s\textsuperscript{ã}nã.h\textsuperscript{ĩ}nĩ] `you leave us and...'
	\item\label{dorseyslaw1i} /suk\#ɾuwãʔk/ $\to$ [ˈsuk.nũ.mãʔk] `young man, boy'
	\item\label{dorseyslaw1j} /wĩ-hãp\#wãʔk/ $\to$ [mĩ.ˈhãp.mãʔk] `today'
	\item\label{dorseyslaw1k} /iːh\#wɾĩ/ $\to$ [ˈiːh.m\textsuperscript{ĩ}nĩ] `saliva'
	\end{xlist}
\end{exe}

\citet[388]{hall2006} takes a look at the typology of excrescent vowels and argues that excrescent vowels behave very differently from epenthetic vowels in that excrescent vowels are extrametrical and are restricted to the gestural layer without adding an additional segment to the surface representation. The phasing of tautosyllabic consonants is altered slightly to increase perceptibility of the constituent consonants in a cluster \citep{silverman1995,wright1996,chitoranetal2002}. This behavior differs from that of epenthesis, which is phonological in nature and serves to repair an illicit sequence of segments.

In languages featuring Dorsey's Law vowels, speakers are sometimes aware of the intrusive vowels. Other Siouan languages that have Dorsey's Law vowels reflect these sounds differently in their orthographies. We can see examples of the presence or absence of Dorsey's Law vowels in the Lakota and Hoocąk data in (\ref{siouandorsey}) below.

\begin{exe}
\item\label{siouandorsey} Dorsey's Law vowels in orthographies of other Siouan languages

	\begin{xlist}
	\item\label{siouandorsey1} Lakota\\\textit{waglé} {[}wa.ˈɡᵉle] \\
	`I am going back' \citep[122]{ullrich2011}

	\item\label{siouandorsey2} Hoocąk\\ \textit{hakewe} {[}ha.ˈkᵉwe] \\
	`six' \citep[26]{miner1979}

	\end{xlist}

\end{exe}

Lakota speakers routinely produce Dorsey's Law vowels when speaking, but speakers are not typically aware of this fact (\citeauthor{mirzayan2010} p.c.). The most widely used orthography for Lakota and Dakota used in \citeapos{ullrich2011} dictionary follows on the tendency of earlier orthographies such as that used in \citeapos{boasdeloria1941} grammar of the Dakota language to not transcribe the intrusive vowel.\footnote{\citet[5]{boasdeloria1941} do note that certain clusters contain a weak copy of the following vowel, and this intrusive vowel is typically represented by a period between the consonants in their orthography, e.g., $\langle$g.li$\rangle$ [ɡⁱli] `to have come back here.'} In Hoocąk, speakers are generally aware of these copy vowels, and this fact is reflected in the orthography used by the Wisconsin Hoocąk (\citeauthor{lundquist2015} p.c.).

Mandan speakers appear to uniformly be aware of these vowels, as we can look at orthographies used by native speakers and see that we do not have surface clusters involving sonorants, but a sequence of graphemes with vowels written between a consonant and a sonorant. We can see this in the data below, which are taken from textbooks created by L1 speakers for use in classrooms around the Fort Berthold Reservation. Both examples below involve a word that has a Dorsey's Law vowel. While these excrescent vowels play no role in stress assignment and are shorter than phonemic short vowels, these vowel sounds are still both perceptible and salient for speakers, who consistently transcribe them in home orthographies, as shown below.


\begin{exe}
\item\label{dorseyslaworthography1} Vowel excrescence in \textit{minís} /wɾĩs/ [ˈm\textsuperscript{ĩ}nĩs] `horse'

\begin{xlist}
	\item\label{dorseyslaworthography1a} $\langle$ma-nees$\rangle$ \citep[28]{benson2000}
	\item\label{dorseyslaworthography1b} $\langle$meníss$\rangle$ \citep[4]{littleowl1992}

\end{xlist}

\item\label{dorseyslaworthography2} Vowel excrescence in \textit{húpinih(e)} /hupɾĩh=E/ [ˈhu.p\textsuperscript{ĩ}nĩh$\sim$ˈhu.p\textsuperscript{ĩ}nĩ.he] `soup'
\begin{xlist}
	\item\label{dorseyslaworthography2a} $\langle$who pe ne hea$\rangle$ [ˈhu.p\textsuperscript{ĩ}nĩ.he] \citep[2]{benson1999}
	\item\label{dorseyslaworthography2b} $\langle$húpi'ni$\rangle$ [ˈhu.p\textsuperscript{ĩ}nĩh] \citep[23]{littleowl1992}

\end{xlist}

\end{exe}

The Dorsey's Law vowels in Mandan are more centralized than their phonemic counterparts, so the production of these vowels varies between utterances of particular tokens. Some speakers more strongly centralize their Dorsey's Law vowels, which has led some researchers and learners to transcribe these sounds as [a$\sim$ə], e.g.,  $\langle$manís$\rangle$ for `horse' (\citeauthor{park2012} p.c.).

This variability in the realization of Dorsey's Law vowels has led to inconsistencies in past transcriptions of Mandan. \citet[391]{hall2006} notes that excrescent vowels typologically have a highly variable duration and the quality of these vowels is influenced by a nearby vowel or consonant. These factors differ from the typological behavior of epenthetic vowels, whose phonetic properties are more predictable. Furthermore, these excrescent vowels are not visible to phonological processes like syllabification and stress assignment. The blindness of phonological processes to these vowels demonstrates their extraphonologicality, which is addressed further in \sectref{stress}.


\section{Orthography}\label{SecOrthography}
\largerpage
The orthography used throughout this book largely follows the orthography used in \citet{kasak2014}. This orthography is amended slightly from the orthography used by the Nu'eta Language Initiative after a discussion between me and Corey Spotted Bear in Twin Buttes, ND in the spring of 2016. This orthography differs from those used in \citeapos{hollow1970} dictionary and \citeapos{hollow1976} textbook, as well as from \citet{mixco1997a,mixco1997b}. Other orthographies exist for Mandan, such as home orthographies used by speakers to record their own language or share it with others. Previous Mandan teachers in Twin Buttes or New Town have also employed their own orthographies. Efforts to promote a consensus Mandan orthography on the Fort Berthold Indian Reservation are ongoing.

The graphic representation of Mandan utilized here is ultimately a mix of Americanist and English-oriented notation. Unlike the orthographies used in \citet{hollow1970} or \citet{mixco1997a}, the orthography herein is not phonemic, nor is there a one-to-one relationship between phones and graphemes. Previous orthographies were tailor made to reveal as much about the underlying morphology and phonology as possible, such as \citeapos{hollow1970} dictionary recording entries with assumed underlying forms, which are not immediately useful to learners if they do not read the chapter of his dictionary on how to convert the underlying representation into a surface one. The orthography used herein represents the surface form of each word, given that being able to immediately say a word is of the highest priority to heritage learners and community members.

A guide on how to read the present orthography is summarized below in (\ref{ExMandanOrthography}), with the orthographic form depicted in angled brackets, $\langle\rangle$, and the phonetic form depicted in square backets, [].

\begin{exe}

\item\label{ExMandanOrthography} Overview of Mandan orthography

\begin{itemize}

\item Vowel length is marked with digraphs, e.g., /a/ is $\langle$a$\rangle$, while /aː/ is $\langle$aa$\rangle$.

\item Excrescent vowels are marked with non-superscript vowels, e.g., the word [ˈwᵉɾoːk] `bull' is $\langle$weróok$\rangle$.

\item The postalveolar fricative /ʃ/ is written as the digraph $\langle$sh$\rangle$, e.g., the word [ˈʃi] `foot' is $\langle$shí$\rangle$.

\item The nasalized allophones of /w/ and /ɾ/ are written as $\langle$m$\rangle$ and $\langle$n$\rangle$, respectively, and the following vowel is assumed to be nasal unless otherwise stated, e.g., the word \textit{maná} `wood, tree' is [ˈm\textsuperscript{ã}nã] and not *[ˈm\textsuperscript{a}na].

\item Word-initial [ⁿd] for /ɾ/ is written as $\langle$r$\rangle$, e.g., the word [ˈⁿdaː.hta] `go'  is $\langle$ráahta$\rangle$.

\item Nasal vowels not preceded by $\langle$m$\rangle$ or $\langle$n$\rangle$ are marked by an ogonek, e.g., the word [ˈãː.we] `all' is  $\langle$ą́ąwe$\rangle$, but the word [ˈmãː.nã] `winter, year' is $\langle$máana$\rangle$.


\item The glottal stop /ʔ/ is represented with an apostrophe, e.g., the word [ˈʃiʔ.ɾe] `it is good' (female addressee) is $\langle$shí're$\rangle$.

\item Remaining consonants are equivalent to the IPA, e.g., /x/ is $\langle$x$\rangle$, /s/ is $\langle$s$\rangle$, etc.

\item An apostrophe appears between $\langle$s$\rangle$ and $\langle$h$\rangle$ to distinguish between $\langle$sh$\rangle$ as /ʃ/, e.g., \textit{tashká'sha?} [ta.ˈʃkaʔ.ʃa] `how are you?') and $\langle$s'h$\rangle$ as the cluster /s.h/ (e.g., \textit{kapús'here'sh} [ka.ˈpus.hᵉɾeʔʃ] `he made it streaked.'\footnote{A reviewer raises the issue of possible ambiguity that can be introduced by the convention of using an apostrophe to indicate a /s.h/ cluster versus $\langle$sh$\rangle$ /ʃ/ for the postalveolar fricative. There should be no ambiguity in this convention, as there is no way for a glottal stop to appear in an interconsonantal environment. The cluster [sʔh] is simply an illicit sequence in Mandan and will never occur.}

\item An acute accent marks primary stress; for stressed long vowels, the the acute accent is placed on the first vowel in the digraph, e.g., the word [ĩ.ˈnãk] `again' is $\langle${inák}$\rangle$ and [ˈaː.ki.taː] `above' is $\langle$áakitaa$\rangle$.

\item The underlying ablaut vowel is written as /E/ in underlying representations or in dictionary entries but is written with its surface form in the orthography, e.g., the word /ɾEːh=ta/ `go!' is [ˈⁿdaː.hta] $\langle${ráahta}$\rangle$, but the word /ɾEːh=oʔʃ/ `he goes there' is [ˈⁿdeː.hoʔʃ] $\langle${réeho'sh}$\rangle$. The default value of /E(ː)/ is [e(ː)], but it can become [a(ː)] if  triggered by any of the conditions discussed in \sectref{ablaut}.

\end{itemize}

\end{exe}

Throughout this work, two different orthographies may be used in examples relating to underlying form: the Mandan orthography above and the IPA. IPA will only be used when in the context of explanations of phonological and phonetic matters. In glossed examples, underlying representation of lexical items will be given in Mandan orthography.

This orthography is summarized in the table below:

\begin{table}
\caption{Summary of orthographic conventions}\label{summaryoforthography}
\begin{tabular}{lllllllllllllllllll}
\lsptoprule
\multicolumn{4}{l}{Consonants}\\
\midrule
IPA			&	p&	t&	 k&	ʔ& s&		ʃ&		x&	h&	w&	m&	ɾ& ⁿd& n\\
Orth.&	p&	t&	k&		'&	s&		sh&	x&	h&	w&	m&	r&	r&	n\\
\midrule
\multicolumn{4}{l}{Vowels}\\
\midrule
IPA&				a&	aː& ã& ãː&e&eː&i&iː&ĩ&ĩː&o&oː&u&uː&ũ&ũː\\
Orth.&	a&aa&ą&ąą& e&ee&i&ii&į&įį&o&oo&u&uu&ų&ųų\\
\lspbottomrule
\end{tabular}
\end{table}

It bears mentioning that different individuals and organizations have differing conventions on how to indicate this particular cluster, but I employ this convention throughout the present work nonetheless. Various Mandan individuals or organizations have differing conventions for writing their language, and I wish to iterate that I do not claim that the orthography I use here is the only one that should be used. This orthography is simply one that I use to map sounds to writing in a way that makes sense from my point of view, and I acknowledge that alternative points of view exist and are valid.


\section{Phonotactics}\label{SecPhonotactics}

This section details the structure of possible syllables in Mandan.

\subsection{Consonant clusters}\label{consonantclusters}

Mandan, like many Siouan languages, allows for CC consonant clusters. \citet[16]{hollow1970} states that within a root or affix, there can be at most one consonant cluster. Exceptions to this generalization are compounds or words that were compounds historically, but are no longer considered compounds for modern speakers. Observed data from the corpus show this generalization to be true.

In his grammar, \citet[17]{hollow1970} lists the possible underlying consonant clusters attested in his data. These possible clusters appear in \tabref{hollowclusters} below. Given that \citet{hollow1970} records all lexical items in underlying notation, the data in \tabref{hollowclusters} do not capture the full range of consonant clusters found in the surface forms of Mandan words in the corpus. Furthermore, Hollow does not include the glottal stop in his description of consonant clusters, even though there are numerous words he posits having underlying /ʔC/ clusters, where /C/ is any consonant. Another issue with the above inventory of possible clusters is that it excludes any clusters formed due to some morphological operation, i.e., affixation, compounding, and incorporation. Given that noun-noun, noun-verb, and verb-verb compounds are common in Mandan, the productivity of this process inevitably yields environments where a wider range of consonants come into contact and can form clusters.

\begin{table}
\caption{Underlying consonant clusters given in \citet[17]{hollow1970}}\label{hollowclusters}
\begin{tabular}{clllllllll}
\lsptoprule
\diagbox[innerwidth=0.75cm]{C\textsubscript{1}}{C\textsubscript{2}}&\textbf{p}&\textbf{t}&\textbf{k}&\textbf{s}&\textbf{ʃ}&\textbf{x}&\textbf{h}&\textbf{w}&\textbf{ɾ}\\
\midrule
\textbf{p}&	--& 	pt&	pk&	ps&	pʃ&	px&	--&	--&	pɾ\\
\textbf{t}&	--& 	--&	tk&	--&	--&	--&	--&	--&	tɾ\\
\textbf{k}& --& 	kt&	--&	ks&	kɾ&	kx&	--&	kw&	kɾ\\
\textbf{s}& --& 	st&	sk&	--&	--&	--&	--&	--&	sɾ\\
\textbf{ʃ}&--&ʃt&ʃk&--&--&	--&	--&	--&	ʃɾ\\
\textbf{x}& --& 	xt&	xk&	--&	--&	--&	--&	xw&	xɾ\\
\textbf{h}& --& 	--&	--&	--&	--&	--&	--&	--&	hɾ\\
\textbf{w}& --& 	--&	--&	--&	--&	--&	--&	--&	wɾ\\
\textbf{ɾ}&--&--&	--&	--&	--&	--&	ɾh&--& --\\
\lspbottomrule
\end{tabular}
\end{table}



Looking just at surface consonant clusters, those shown below are attested in the corpus. This distribution includes both phonemic and allophonic instances of the consonants discussed in \sectref{consonants}. An asterisk indicates that this cluster is only attested in morphologically complex conditions, i.e., affixation, compounding, or incorporation.

\begin{table}
\caption{Surface consonant clusters}\label{surfaceclusters}
\label{clusterstable}
\begin{tabular}{cllllllllllll}
\lsptoprule
 \diagbox[innerwidth=0.75cm]{C\textsubscript{1}}{C\textsubscript{2}}& \textbf{p} & \textbf{t} & \textbf{k} & \textbf{ʔ} & \textbf{s} & \textbf{ʃ} & \textbf{x} & \textbf{h} & \textbf{w} & \textbf{m} & \textbf{ɾ} & \textbf{n} \\
\midrule
\textbf{p} &--&pt&pk&--&ps&pʃ&px&ph*&pw*&pm*&pɾ*&pn*\\
\textbf{t} &tp*&--&tk&--&ts*&tʃ*&tx&th*&tw*&tm*&tɾ*&tn*\\
\textbf{k} &kp*&kt&--&--&ks&kʃ&kx&kh*&kw*&km*&kɾ*&kn*\\
\textbf{ʔ} &ʔp&ʔt&ʔk&--&ʔs&ʔʃ&ʔx&ʔh&ʔw*&ʔm*&ʔɾ&ʔn\\
\textbf{s} &sp*&st&sk&--&--&sʃ*&sx*&sh*&sw*&sm*&sɾ*&sn*\\
\textbf{ʃ} &ʃp*&ʃt*&ʃk&--&ʃs*&--&ʃx*&ʃh*&ʃw*&ʃm*&ʃɾ*&ʃn*\\
\textbf{x} &xp&xt&xk&--&xs*&xʃ*&--&xh*&xw*&xm*&xɾ*&xn*\\
\textbf{h} &hp&ht*&hk&--&hs*&hʃ*&hx*&--&hw*&hm*&hɾ*&hn*\\
\textbf{w} &--&--&--&--&--&--&--&--&--&--&wɾ&--\\
\textbf{m} &--&--&--&--&--&--&--&--&--&--&--&mn\\
\textbf{ɾ} &--&--&--&--&--&--&--&--&--&--&--&--\\
\textbf{n} &--&--&--&--&--&--&--&--&--&--&--&--\\
\lspbottomrule
\end{tabular}
\end{table}

We can see a much wider range of attested clusters above than in \tabref{surfaceclusters} from \citet{hollow1970}. The large discrepancy between the Tables \ref{hollowclusters} and \ref{surfaceclusters} is due to the fact that there are a number of clusters attested in \citeauthor{hollow1970}'s own dictionary that are not reflected in his original description of possible consonant clusters, along with \citeauthor{hollow1970}'s focus only on underlying clusters rather than surface forms. Three initial phonotactic restrictions in Mandan can be derived from \tabref{surfaceclusters}:

\begin{exe}

\item\label{phonotacticrestrictions} General phonotactic restrictions

\begin{xlist}

\item No [C\textsubscript{i}C\textsubscript{i}] sequences are permitted.

\item No cluster may have a glottal stop as its second element.

\item No cluster may have a sonorant as its initial element except for /wɾ/.\footnote{See \sectref{dorseyslaw} and \sectref{secdorseystress} for further discussion of how these clusters are interrupted by excrescent vowels due to Dorsey's Law, but they are still tautosyllabic. Note that both [wɾ] and [mn] are derived from /wr/.}

\end{xlist}

\end{exe}

The first restriction in (\ref{phonotacticrestrictions}) seems motivated by a dispreference for surface geminates or pseudo-geminates. This phonotactic restriction also a motivating factor for why /tt/ or /kk/ clusters that arise through morphological processes involve lenition of the initial element of the cluster, i.e., /tt/ becomes [st] and /kk/ becomes [hk]. The process of surface /C\textsubscript{i}C\textsubscript{i}/ dissimilation is explained further in \sectref{geminatedissimilation}.

With respect to the second restriction, there are no root-initial glottal stops in Mandan, so there is no way for a glottal stop to appear as the second element of a cluster through affixation or compounding. Underlying Glottal stops can only appear in the coda or epenthetically between vowels to prevent hiatus, but under certain conditions, the vowel preceding the glottal stop can syncopate and cause the glottal stop to metathesize with the following vowel, turning what was once an onset into a coda. This behavior is described in greater detail in \sectref{glottalstopmetathesis}. %This metathesis is a last-resort operation to avoid violating the phonotactic restriction against having a consonant cluster where the second element is a [ʔ].

The final restriction is due to the fact there are no sonorant codas in any Mandan root or affix. Historically, Proto-Siouan did not have any sonorant codas \citep{rankinetalnd}. Reconstructed roots have a basic CV(ː) syllable structure, where sonorants can only appear as onsets. Mandan has inherited this same distribution of sonorants. Unlike other Siouan languages like Crow \citep{graczyk2007} or Lakota \citep{ingham2003}, Mandan did not innovate any sonorant codas. This phonotactic restriction originates in the diachrony and there is no synchronic process to avoid sonorant-obstruent clusters.

Examples of attested consonant clusters appear below on Tables \ref{CCexamplesp}--\ref{CCexamplesʔ} for stops and \ref{SCexampless}--\ref{SCexamplesh} for fricatives.

\begin{table}
\caption{Examples of p-initial clusters}\label{CCexamplesp}
\begin{tabularx}{\textwidth}{lQQl}
\lsptoprule
\textbf{~}&\textbf{Pronunciation}&\textbf{Orthography}&\textbf{Gloss}\\
\midrule
\textbf{pt}&[ˈptĩː.ɾe]&\textit{pt\'{ı̨}įre}&`buffalo'\\
\textbf{pk}&[ˈpke]&\textit{pké}&`snapping turtle'\\
\textbf{ps}&[ˈpsãː.ka]&\textit{psą́ąka}&`frog'\\
\textbf{pʃ}&[ˈpʃãː.ʃe]&\textit{pshą́ąshe}&`sweetgrass'\\
\textbf{px}&[ˈpxiʔʃ]&\textit{pxí'sh}&`he sneezes'\\
\textbf{ph}&[ˈtoːp.ha]&\textit{tópha}&`four times'\\
\textbf{pw}&[ki.ˈʃop.wa.h\textsuperscript{e}ɾeʔʃ]&\textit{kishópwahere'sh}&`I rounded them up'\\
\textbf{pm}&[mĩ.ˈhãp.mãʔk]&\textit{mihą́pma'k}&`today'\\
\textbf{pɾ}&[ki.ˈʃop.ɾa.h\textsuperscript{e}ɾeʔʃ]&\textit{kishóprahere'sh}&`you rounded them up'\\
\textbf{pn}&[ki.ˈʃop.nũ.h\textsuperscript{e}ɾeʔʃ]&\textit{kishópnuhere'sh}&`we rounded them up'\\
\lspbottomrule
\end{tabularx}
\end{table}

\begin{table}
\caption{Examples of t-initial clusters}\label{CCexamplest}
\begin{tabularx}{\textwidth}{llQl}
\lsptoprule
\textbf{~}&\textbf{Pronunciation}&\textbf{Orthography}&\textbf{Gloss}\\
\midrule
\textbf{tp}&[ko.ˈɾet.pa.ʃa.hãkt]&\textit{korétpashahąkt}&`northward'\\
\textbf{tk}&[ˈtkuʃ]&\textit{tkúsh}&`true'\\
\textbf{ts}&[ˈwaʔts]&\textit{wá'ts}&`my father'\\
\textbf{tʃ}&[ˈhⁱɾut.ʃo.te]&\textit{hirútshote}&`grey fox'\\
\textbf{tx}&[ˈwaː.ʔat.xihs]&\textit{wáa'atxihs}&`government, the president'\\
\textbf{th}&[ˈo.ka.pat.h\textsuperscript{e}ɾe]&\textit{ókapathere}&`to plant'\\
\textbf{tw}&[ˈo.ka.pat.wa.h\textsuperscript{e}ɾeʔʃ]&\textit{ókapatwahere'sh}& `I planted it'\\
\textbf{tm}&[ˈo.ka.pat.mã.h\textsuperscript{e}ɾeʔʃ]&\textit{ókapatmahere'sh}& `he made me plant it'\\
\textbf{tɾ}&[ˈo.ka.pat.ɾa.h\textsuperscript{e}ɾeʔʃ]&\textit{ókapatrahere'sh}& `you planted it'\\
\textbf{tn}&[ˈo.ka.pat.nĩ.h\textsuperscript{e}ɾeʔʃ]&\textit{ókapatnihere'sh}& `he made you plant it'\\
\lspbottomrule
\end{tabularx}
\end{table}

\begin{table}
\caption{Examples of t-initial clusters}\label{CCexamplesk}
\begin{tabularx}{\textwidth}{lQQl}
\lsptoprule
\textbf{~}&\textbf{Pronunciation}&\textbf{Orthography}&\textbf{Gloss}\\
\midrule
\textbf{kp}&[ˈwᵉɾoːk.pa]&\textit{Weróokpa}&`Buffalo Bull Head' (name)\\
\textbf{kt}&[ktaː.hoʔʃ] &\textit{ktáaho'sh} &`he is frozen'\\
\textbf{ks}&[ˈksek] &\textit{ksék} &`crooked'\\
\textbf{kʃ}&[ˈkʃĩː.kʃe] &\textit{kshı̨́įkshe} &`lightning'\\
\textbf{kx}&[ˈkxaa.kxeʔʃ] &\textit{kxáakxe'sh} &`it is spotted'\\
\textbf{kh}&[ˈsuk.h\textsuperscript{e}ɾe]&\textit{súkhere} &`to make him exit'\\
\textbf{kw}&[ˈsuk.wa.h\textsuperscript{e}ɾeʔʃ] &\textit{súkwahere'sh} &`I made him exit'\\
\textbf{km}&[ˈsuk.mĩː.he] &\textit{súkmiihe} &`young woman, girl'\\
\textbf{kɾ}&[ˈsuk.ɾa.h\textsuperscript{e}ɾeʔʃ]&\textit{súkrahere'sh} &`I made him exit'\\
\textbf{kn}&[ˈsuk.nũ.mãʔk] &\textit{súknuma'k} &`young man, boy'\\
\lspbottomrule
\end{tabularx}
\end{table}


\begin{table}
\caption{Examples of ʔ-initial clusters}\label{CCexamplesʔ}
\begin{tabularx}{\textwidth}{lQQl}
\lsptoprule
\textbf{~}&\textbf{Pronunciation}&\textbf{Orthography}&\textbf{Gloss}\\
\midrule
\textbf{ʔp}&[ˈmũʔpa]&\textit{mú'pa}&`with me'\\
\textbf{ʔt}&[ˈãʔt] &\textit{ą́'t} &`that (far away)'\\
\textbf{ʔk}&[nũ.ˈmãʔk] &\textit{numá'k} &`man, person'\\
\textbf{ʔs}&[ˈmãʔs] &\textit{ma's} &`spoon'\\
\textbf{ʔʃ}&[ˈhũʔʃ] &\textit{hų́'sh} &`there are many'\\
\textbf{ʔx}&[ˈtuʔ.x\textsuperscript{e}ɾeʔ.ɾe] &\textit{tú'xere're} &`there might be some'\\
\textbf{ʔh}&[ˈmĩʔ.he] &\textit{mí'he} &`blanket, shawl, robe'\\
\textbf{ʔɾ}&[ˈʃiʔ.ɾe] &\textit{shí're} &`it is good'\\
\textbf{ʔn}&[ˈʃiʔ.nã] &\textit{shí'na?} &`is it good?'\\
\lspbottomrule
\end{tabularx}
\end{table}

\begin{table}
\caption{Examples of s-initial clusters}\label{SCexampless}
\begin{tabularx}{\textwidth}{llQl}
\lsptoprule
\textbf{Cluster}&\textbf{Pronunciation}&\textbf{Orthography}&\textbf{Gloss}\\
\midrule
\textbf{sp}&[pu.ˈspu.se]&\textit{puspúse}&`kitten'\\
\textbf{st}&[i.ˈstã.mĩʔ]&\textit{istámi'} &`eye'\\
\textbf{sk}&[ra.ˈskiː.koʔʃ]&\textit{raskíiko'sh}&`he squeezes it with teeth'\\
\textbf{sʃ}&[mãː.mĩks.ʃkaʔ.nĩk]&\textit{máamikssha'nik}&`but there's no such thing'\\
\textbf{sx}&[ki.ˈki.ɾas.xteʔʃ]&\textit{kikírasxte'sh}&`she really liked him'\\
\textbf{sh}&[ka.ˈpus.h\textsuperscript{e}ɾeʔʃ]&\textit{kapús'here'sh}& `he made him write'\\
\textbf{sw}&[ˈm\textsuperscript{ĩ}nĩs.weː.ɾut]&\textit{minísweerut}& `dog'\\
\textbf{sm}&[ka.ˈpus.mã.h\textsuperscript{e}ɾeʔʃ]&\textit{kapúsmahere'sh}& `he made me write'\\
\textbf{sɾ}&[ka.ˈpus.ɾa.h\textsuperscript{e}ɾeʔʃ]&\textit{kapúsrahere'sh}& `you made him write'\\
\textbf{sn}&[ka.ˈpus.nĩ.h\textsuperscript{e}ɾeʔʃ]&\textit{kapúsnihere'sh}& `he made you write'\\
\lspbottomrule
\end{tabularx}
\end{table}


\begin{table}
\caption{Examples of ʃ-initial clusters}\label{SCexamplesʃ}
\begin{tabularx}{\textwidth}{lllQ}
\lsptoprule
\textbf{Cluster}&\textbf{Pronunciation}&\textbf{Orthography}&\textbf{Gloss}\\
\midrule
\textbf{ʃp}&[i.ˈʃpa.ɾi.ʔoː.ɾi] & \textit{ishpári'oori}&`Mexican'\newline (< Fr. \textit{espagnol})\\
\textbf{ʃt}&[ˈo.p\textsuperscript{a}ɾaʃt]&\textit{óparasht}& `on a high ridge'\\
\textbf{ʃk}&[ta.ˈʃkaʔ.ʃa]&\textit{tashká'sha?}&`how is it?'\\
\textbf{ʃs}&[pa.ˈweʃs]&\textit{pawéshs}&`the one he cut'\\
\textbf{ʃx}&[ˈʃi.nãʃ.xte.ɾoː.mã.koʔʃ]&\textit{shíxteroomako'sh}&`he was really good-looking'\\
\textbf{ʃh}&[pa.ˈweʃ.h\textsuperscript{e}ɾe.ki]&\textit{pawéshhereki!}&`let him cut it!'\\
\textbf{ʃw}&[pa.ˈweʃ.wa.h\textsuperscript{e}ɾeʔʃ]&\textit{pawéshwahere'sh}&`I made him cut it'\\
\textbf{ʃm}&[pa.ˈweʃ.mã.h\textsuperscript{e}ɾeʔʃ]&\textit{pawéshmahere'sh}&`He made me cut it'\\
\textbf{ʃɾ}&[pa.ˈweʃ.ɾa.h\textsuperscript{e}ɾeʔʃ]&\textit{pawéshrahere'sh}&`you made him cut it'\\
\textbf{ʃn}&[pa.ˈweʃ.nĩ.h\textsuperscript{e}ɾeʔʃ]&\textit{pawéshrahere'sh}&`he made you cut it'\\
\lspbottomrule
\end{tabularx}
\end{table}


\begin{table}
\caption{Examples of x-initial clusters}\label{SCexamplesx}
\begin{tabularx}{\textwidth}{lQQl}
\lsptoprule
\textbf{Cluster}&\textbf{Pronunciation}&\textbf{Orthography}&\textbf{Gloss}\\
\midrule
\textbf{xp}&[ˈmã.xpe]&\textit{maxpé}&`nine'\\
\textbf{xt}&[ˈxteʔʃ]&\textit{xté'sh}&`it is big'\\
\textbf{xk}&[ˈxkã.hoʔ.ɾe]&\textit{xką́ho're}&`she moves away'\\
\textbf{xs}&[ⁿdu.ˈptu.xsoʔʃ]&\textit{ruptúxso'sh}&`he made it fine'\\
\textbf{xʃ}&[ˈpax.ʃo.wok]&\textit{páxshowok}&`bowl'\\
\textbf{xh}&[ˈm\textsuperscript{ĩ}nĩx.h\textsuperscript{e}ɾe.k\textsuperscript{e}ɾeʔʃ]&\textit{miníxherekere'sh}&`he plays with them'\\
\textbf{xw}&[ˈm\textsuperscript{ĩ}nĩx.wa.h\textsuperscript{e}ɾeʔʃ]&\textit{miníxwahere'sh}&`I play with him'\\
\textbf{xm}&[ˈm\textsuperscript{ĩ}nĩx.mã.h\textsuperscript{e}ɾeʔʃ]&\textit{miníxmahere'sh}&`He plays with me'\\
\textbf{xɾ}&[ˈm\textsuperscript{ĩ}nĩx.ɾa.h\textsuperscript{e}ɾeʔʃ]&\textit{miníxrahere'sh}&`You play with him'\\
\textbf{xn}&[ˈm\textsuperscript{ĩ}nĩx.nĩ.h\textsuperscript{e}ɾeʔʃ]&\textit{miníxnihere'sh}&`He plays with you'\\
\lspbottomrule
 \end{tabularx}
 \end{table}

\begin{table}
\caption{Examples of h-initial clusters}\label{SCexamplesh}
\begin{tabularx}{\textwidth}{lQQl}
\lsptoprule
\textbf{Cluster}&\textbf{Pronunciation}&\textbf{Orthography}&\textbf{Gloss}\\
\midrule
\textbf{hp}&[ˈi.hp\textsuperscript{a}ɾak]&\textit{íhparak}&`belt'\\
\textbf{ht}&[ˈⁿdaː.hta]&\textit{ráahta!}&`go there!'\\
\textbf{hk}&[ˈʃi.hka]&\textit{shíhka}&`dwarf, small dog'\\
\textbf{hs}&[ˈmĩːhs]&\textit{míihs}&`the woman'\\
\textbf{hʃ}&[ki.ˈxũː.hʃa]&\textit{kixų́ųhsha}&`five of them'\\
\textbf{hx}&[ˈxi.hxteʔʃ]&\textit{xíhxte'sh}&`he is really old'\\
\textbf{hw}&[ˈⁿdeːh.wa.h\textsuperscript{e}ɾeʔʃ]&\textit{réehwahere'sh}&`I put it there'\\
\textbf{hm}&[ˈⁿdeːh.mã.h\textsuperscript{e}ɾeʔʃ]&\textit{réehmahere'sh}&`He told me to go'\\
\textbf{hɾ}&[ˈⁿdeːh.ɾa.h\textsuperscript{e}ɾeʔʃ]&\textit{réehrahere'sh}&`You put it there'\\
\textbf{hn}&[ˈⁿdeːh.nĩ.h\textsuperscript{e}ɾeʔʃ]&\textit{réehnihere'sh}&`He told you to go'\\
\lspbottomrule
\end{tabularx}
\end{table}


As both Tables \ref{CCexamplesp}--\ref{CCexamplesʔ}   and \ref{SCexampless}--\ref{SCexamplesh} demonstrate, there is a large variety of permitted consonant clusters in Mandan. What has not been addressed so far is the issue of triconsonantal clusters. Neither \citet[15]{hollow1970} nor \citet[9]{mixco1997a} outright state that CCC clusters are not permitted, but in \citeauthor{hollow1970}'s dictionary, there are several instances where a CC sequence appears in what should be a CCC, as the examples below demonstrate.

\begin{exe}
\item\label{CCCclusters} Lack of CCC clusters in \citet{hollow1970}

	\begin{xlist}


	\item $\langle$šų́kšuk$\rangle$ $\to$	{[}ˈʃũt.kʃuk]\\
	{/}shųt/ `tail' + /kshuk/ `narrow'\\
	`muskrat'

	\item $\langle$mą́takšuks$\rangle$ $\to$ 	{[}ˈmãː.tah.kʃuks]\\
	{/}wąątah/ `river' + /kshuk/ `narrow' + /=s/ \textsc{def}\\
	`Little Missouri River'

	\item $\langle$pasą́xte$\rangle$ $\to$ 	{[}ˈpa.ˈsãh.xte]\\
	{/}pasąh/ `creek' + /xte/ `big'\\
	`Cannonball River'

	\end{xlist}


\end{exe}

In his own recordings, the deleted first consonant of these clusters is pronounced, though the first consonant may be unreleased in fast speech.  Given that CCC clusters are entirely due to compounding, it is difficult find the full extent to whether there are any illicit CCC clusters, or whether one can simply combine any viable CC cluster after any coda.

\subsection{Syllable structure}\label{syllablestructure}

\citet[15]{hollow1970} and \citet[21]{coberly1979} both state that the Mandan syllable has the following shape:

\begin{exe}
\item\label{mandansyllable1} Syllable structure per \citet[15]{hollow1970}

(C)(C)V(C)
\end{exe}

A syllable under this analysis can have an onset with no more than two consonants and a coda with no more than one. Neither \citeauthor{hollow1970} nor \citeauthor{coberly1979} record vowel length, so their analysis of Mandan syllable structure does not take vowel length into account.

\citet{carter1983} tries to reconcile the syllable structure in (\ref{mandansyllable1}) above with the fact that certain lexical items in \citeauthor{hollow1970}'s dictionary have underlying forms that end in /ʔ/ followed by a consonant. For example, \citet[246]{hollow1970} states that the underlying form of `to pick at, tamper with' is $\langle$tą́ʔx$\rangle$. \citeauthor{carter1991a} cites examples of words like these to suggest that Proto-Siouan also had root that had glottal stop codas. Much later after the publication of his article on root-final consonant clusters in Mandan and following subsequent fieldwork, \citeauthor{carter1991a} states that there is an issue with words that \citeauthor{hollow1970} transcribes as having underlying /ɾ/ codas: these word have no [ɾ] articulated when said in isolation, but when additional morphology is added to such words, the supposed /ɾ/ appears on the surface. \citet{carter1991a} points out that these words all happen to feature long vowels or have a coda glottal stop. As such, the [ɾ] that appears between the root and subsequent morphology is not actually underlying. Rather, it is an linking segment that appears between long vowels or coda glottal stops \citep[487]{carter1991a}.

\begin{exe}

\item\label{cartersyllablerule} \textbf{Carter's [ɾ] Insertion Rule}\\
	$\varnothing$ $\to$ [ɾ] / Vː{\longrule}V\\
	\textit{Add an epenthetic} [ɾ] \textit{between a long vowel and a short vowel.}

\end{exe}

Glottal stops seem to pattern with long vowels in two different ways. Firstly, they both trigger this linking [ɾ] when additional morphological material appears to the right of them. This process is described in \sectref{intrusiveR}. Secondly, both mark a syllable as heavy and draw primary stress. Mandan features left-aligned iambic footing for primary stress, i.e., primary stress defaults to the second syllable, but can appear on the first syllable if the first syllable is heavy. An initial syllable with a coda glottal stop receives primary stress over one without a coda glottal in the same way a syllable with a long vowel does. Stress is explained further in \sectref{stress}.

Given this information, we can revise \citet{hollow1970} and \citeapos{coberly1979} description of Mandan syllable structure

\begin{exe}
\item\label{mandansyllable2} Revised syllable structure for Mandan per \citet[485]{carter1991a}

(C)(C)V($\substack{\text{V}\\\text{ʔ}}$)(C)

\end{exe}

This revised structure accounts for the fact that Mandan does not permit a glottal stop coda in a syllable containing a long vowel. If both long vowels and coda glottal stops are considered heavy (i.e., add a mora to a syllable), the motiving factor to account for the lack of VVʔ-type syllables is that such a structure would be superheavy. Having a trimoraic syllable is not permitted in Mandan; a syllable may contain a long vowel or a coda glottal stop, but never both.

Aside from the stipulation regarding what the nucleus of a syllable in Mandan may look like, the structure above in (\ref{mandansyllable2}) does not alter previous analyses of how onsets and codas may be formed. What has changed is what counts as a syllable nucleus in Mandan phonology. Instead of being treated as a consonant, a coda glottal fills the same slot as a vowel in the syllable skeletal structure, i.e., it creates a bimoraic syllable in the same way a long vowel does.

Below, we can see examples of the different attested iterations of the syllable structure in Mandan. Instead of being limited to a single consonant in the coda (excluding glottal stops), we see that all combinations of syllables with onsetless and simple onsets have a confirmed complex coda. There are words that allow for both a complex onset and complex coda, but they are always morphologically complex. No instance of a CCVʔCC syllable has been found, and it is unlikely that such a combination exists due to the fact that roots containing a coda glottal are rare and none of them have complex onsets. This distribution of roots with coda /ʔ/ therefore rules out CCVʔCC by the lack of items that could possibly be combined to yield such a structure rather than being ruled out due to any phonotactic constraint.

If we take the structure of the Mandan syllable in (\ref{mandansyllable2}) into account, there is a wider range of possible syllable shapes in Mandan than the one originally suggested by \citet[15]{hollow1970}. \citeapos{carter1991a} analysis best fits the data, as \tabref{mandansyllable3} illustrates, since each possible syllable type under this revise syllable schema is attested in the language.



\begin{table}[t]
\singlespacing
\caption{Possible syllables in Mandan}\label{mandansyllable3}

\begin{tabular}{llll}
\lsptoprule
\textbf{Syllable}&
	\textbf{Example}&
	\textbf{Orthography}&
	\textbf{Gloss}\\
\midrule
V&
	[ˈi.ʔahe]&
	\textit{í'ahe}&
	`skin, shell'\\
VV&
	[ˈoː]&
	\textit{óo}&
	`there'\\
Vʔ&
	[ˈĩʔ.tu]&
	\textit{\'{ı̨}'tu}&
	`to be born'\\
VC&
	[ˈũk]&
	\textit{ų́k}&
	`hand'\\
VVC&
	[ˈoːt]&
	\textit{óot}&
	`mix'\\
VʔC&
	[ˈãʔt]&
	\textit{ą́'t}&
	`that (distal)'\\
VCC&
	[ˈũks]&
	\textit{ų́ks}&
	`the hand'\\
VVCC&
	[ˈaːps]&
	\textit{áaps}&
	`the leaf'\\
VʔCC&
	[ˈũʔst]&
	\textit{ų́'st}&
	`long ago'\\
\midrule
CV&
	[mã.ˈto]&
	\textit{mató}&
	`bear'\\
CVV&
	[ˈⁿdoː]&
	\textit{róo}&
	`here'\\
CVʔ&
	[ˈkũʔ]&
	\textit{kų́'}&
	`give'\\
CVC&
	[ˈkok]&
	\textit{kók}&
	`pronghorn antelope'\\
CVVC&
	[ˈmĩːh]&
	\textit{míih}&
	`woman'\\
CVʔC&
	[ˈʃiʔʃ]&
	\textit{shí'sh}&
	`it is good'\\
CVCC&
	[ˈkoks]&
	\textit{kóks}&
	`the pronghorn antelope'\\
CVVCC&
	[ˈmĩːhs]&
	\textit{míihs}&
	`the woman'\\
CVʔCC&
	[ˈkuʔʃt]&
	\textit{kú'sht}&
	`inside'\\
\midrule
CCV&
	[ˈpsi.ɾoː.mã.koʔʃ]&
	\textit{psíroomako'sh}&
	`it was black'\\
CCVV&
	[ˈpʃiː]&
	\textit{pshíi}&
	`be flat, low'\\
CCVʔ&
	[ˈxik.ʃkaʔ.nĩk]&
	\textit{xíkshka'nik}&
	`but it was bad'\\
CCVC&
	[ˈxkek]&
	\textit{xkék}&
	`star'\\
CCVVC&
	[ˈskũːh]&
	\textit{skų́ųh}&
	`be sweet'\\
CCVʔC&
	[ˈʃkeʔʃ]&
	\textit{shké'sh}&
	`he jumps'\\
CCVCC&
	[ˈxkeks]&
	\textit{xkéks}&
	`the star'\\
CCVVCC&
	[ˈptĩːst]&
	\textit{pt\'{ı̨}įst}&
	`to the buffalo'\\
\lspbottomrule
\end{tabular}
\end{table}

The presence of the forms above demonstrate that we require one final revision of what a syllable may look like in Mandan. While \citeauthor{hollow1970} is correct in that a bare lexical stem will maximally have a simple coda, complex codas are possible given morphologically complex structures, e.g., \textit{míihs} /wįįh=s/ `the woman', where \textit{míih} `woman' is followed by the definite article \textit{=s}, resulting in a coda cluster. As such, we must revise the possible structure of the Mandan syllable one final time, with said structure appearing below.

\begin{exe}
\item\label{mandansyllablefinal} Final revised structure of Mandan syllables

(C)(C)V($\substack{\text{V}\\\text{ʔ}}$)(C)(C)

\end{exe}

A syllable in Mandan can have at most two consonants in the onset, maximally two consonants in the coda, provided that no additional morphology adds . The nucleus of a Mandan syllable minimally contains one vocalic moraic component (i.e., a short vowel), though can maximally contain two (i.e., a long vowel or a coda glottal stop).

The previously seen data in (\ref{CCCclusters}) show that CCC clusters are possible in Mandan. However, such clusters will always be syllabified with the underlying morphological structure in mind. Compound words do not syllabify across word boundaries, e.g., a compound word /VC\#CCVC/ will syllabify with the initial consonant forming the coda of the preceding syllable and the remaining two consonants in the cluster forming the onset of the following syllable, i.e., /CCC/ $\to$ [C.CC]. As such, syllabification of complex clusters can be used as a test for internal word boundaries in Mandan, though the presence of such clusters in the first place are always the result of the application of some morphological process like compounding or encliticization. %This pattern is mirrored by disyllabic consonant clusters of the type CC. For example, \textit{nátka} `heart, center' is realized as [ˈnãt.ka], not *[ˈnã.tka], even though [tk] is a viable onset in other words, e.g., \textit{tké'sh} `it is heavy.'

\section{Word boundary-independent morpho-phonology}\label{boundaryindependent}
\largerpage
The vast majority of the study of Mandan has been spent translating texts and breaking down its morphology. Very little attention has been paid to the interaction between its morphology and its phonology. In looking at the morpho-phonological processes present in Mandan, we can divide all attested phenomena in this language into two categories: phonemena that are word boundary-independent (i.e., phenomena that occur regardless of whether they occur in the environment of a morphological word boundary) versus those that are word boundary-dependent (i.e., phenomena that only happen at a morphological word boundary). The data present here are descriptive in nature and is not intended to introduce a theoretical claim. As such, this section utilizes a rule-based account of these phenomena for the sake of parsimony.\footnote{I understand that the choice to implement a rule-based account for the phonological processes here is a kind of theoretical stance, but I do not imply that alternative theoretical accounts, such as ones based in Optimality Theory \citep{princesmolensky1993}, cannot account for the interaction between these processes. I employ an OT-based account of these phenomena in \citet{kasak2019}, but the point of the present work is make the phonological processes described here as understandable to a wide audience as possible and to minimize any kind of theoretical machinery in order to explain why a certain segment changes in particular environments.}

The aim of this section is to describe the processes below and demonstrate that they are unaffected by morphological word boundaries. This lack of word boundary-sensitivity is important to the case built in \sectref{boundarydependent} and the overall case against a phonological motivation for affix ordering in Mandan.

\subsection{Pseudo-geminate dissimilation}\label{geminatedissimilation}

Mandan features numerous examples of consonant clusters, but [C\textsubscript{i}C\textsubscript{i}] clusters are forbidden. \citet[37]{hollow1970} notes that instances of /tt/ clusters dissimilate to [st], but otherwise states that all other instances of /C\textsubscript{i}C\textsubscript{i}/ involve the deletion of the first consonant in the cluster to simplify potential

Both the processes of adding postverbal morphology to a word and compounding often creates environments where such clusters are underlyingly present. In examples (\ref{CClenition}) below, we see /C\textsubscript{i}C\textsubscript{i}/ environments involving stops. While (\ref{CClenitionexFricatives}) involve fricatives, each datum is accompanied by an example that does not feature a /C\textsubscript{i}C\textsubscript{i}/ environment to demonstrate the normal distribution of the obstruent and juxtapose how pseudo-geminiate dissimilation occurs for stops versus fricatives, i.e., stops lenite the first segment while fricatives delete the first segment.

\begin{exe}
\item\label{CClenition} /C\textsubscript{i}C\textsubscript{i}/ resolution $\Rightarrow$ lenition
	\begin{xlist}

	\item\label{CClenitionex1a} \glll ná\uline{t}haa\\
		rąt=haa\\
		\textnormal{middle}=loc\\
		\glt `to be between two' \citep[170]{hollow1970}

	\item\label{CClenitionex1} \glll maná\uline{s}taa\\
	wą-rąt=taa\\
	1s-\textnormal{middle}=loc\\
	\glt `I am in the middle' \citep[37]{hollow1970}

	\item\label{CClenitionex2a} \glll sú\uline{k}numa'k\\
		suk\#ruwą'k\\
		\textnormal{child}\#\textnormal{man}\\
		\glt `boy' \cite[220]{hollow1970}

	\item\label{CClenitionex2} \glll sú\uline{h}keres\\
	suk=krE=s\\
	\textnormal{child}=3pl=def\\
	\glt `the chidren' \citep[178]{hollow1973a}

	\item\label{CClenitionex3} \glll Wará\uline{p}epasąhs\\
	wrap(E)\#pasąh=s\\
	\textnormal{beaver}\#\textnormal{creek}=def\\
	\glt `Beaver Creek' \citep[67]{hollow1973a}
 \end{xlist}
 \end{exe}

 \newpage
 \begin{exe}

    \item\label{CClenitionexFricatives}  /C\textsubscript{i}C\textsubscript{i}/ resolution $\Rightarrow$ deletion

        \begin{xlist}

	\item\label{CClenitionex4a} \glll máataawerex\uline{s}\\
	wąątaa\#wrex=s\\
	\textnormal{clay}\#\textnormal{kettle}=def\\
	\glt `the clay kettle' \cite[266]{hollow1973b}


	\item\label{CClenitionex4} \glll miní\uline{s}\\
	wrįs=s\\
	\textnormal{horse}=def\\
	\glt `the horse' \citep[251]{hollow1973b}

	\item\label{CClenitionex5a} \glll ísekana\uline{sh}oomaks\\
	i-sek=rąsh=oowąk=s\\
	pv.ins-\textnormal{make}=typ=narr=def\\
	\glt `that which he kind of made' \citep[10]{hollow1973a}

	\item\label{CClenitionex5} \glll éeheena\uline{sh}ka'nik\\
	ee-hee=rąsh=shka'rįk\\
	pv-\textnormal{say}=typ=disj\\
	\glt `even though he said it' \citep[41]{hollow1973a}

	\item\label{CClenitionex6a} \glll pá\uline{x}showok\\
		pax\#showok\\
		\textnormal{dish}\#\textnormal{be.deep}\\
		\glt `bowl' \citep[138]{hollow1970}

	\item\label{CClenitionex6} \glll pá\uline{x}te\\
	pax\#xte\\
	\textnormal{bowl}\#\textnormal{big}\\
	\glt `dishpan' \citep[52]{hollow1973a}

	\item\label{CClenitionex7a} \glll rée\uline{h}o'sh\\
	rEEh=o'sh\\
	\textnormal{go.there}=ind.m\\
	\glt `he went there' \citep[175]{hollow1970}

	\item\label{CClenitionex7} \glll rée\uline{h}erek\\
	rEEh\#hrE=ak\\
	\textnormal{go.there}\#caus=ds\\
	\glt `having put it on [himself]' \citep[16]{hollow1973a}

	\end{xlist}

\end{exe}

 Three patterns emerge in the data above. Firstly, there are no /C\textsubscript{i}C\textsubscript{i}/ clusters involving bilabial stops. \citet[10]{torres2013a} notes that /p/ does not occur in word-final position in the narrative she examines, and I can corroborate that /p/ is indeed rare in word-final positions from my own personal experience and fieldwork. Mandan words with stem-final /p/ seem to vacillate between ending in [p] or with [pe] in the corpus, and this variability is not conditioned by any environment. Modern Mandan stems ending with /p/ often feature a final [e] when uttered in isolation, but when placed within the context of a sentence, both variants appear with no reported change in meaning. Below are some examples from the corpus that vary between forms ending with [p] and [pe].

\begin{exe}
\item\label{fireshoe} Alternation between stem-final [p] and [pe] in nouns

\begin{xlist}
	\item\label{fireshoe1} \glll Hų́pwara're\\
	hųp(E)\#wra'=E\\
	\textnormal{shoe}{\#}\textnormal{fire}=sv\\
	\glt `Fire Shoe' (name) \citep[155]{hollow1973a}

	\item\label{fireshoe2} \glll Hų́pewara're\\
	hųp(E)\#wra'=E\\
	\textnormal{shoe}{\#}\textnormal{fire}=sv\\
	\glt `Fire Shoe' (name) \citep[168]{hollow1973a}

\end{xlist}
\item\label{dancep} Alternation between stem-final [p] and [pe] in verbs

\begin{xlist}
	\item\label{dancep1} \glll máana'pe'sh\\
	waa-rą'p(E)=o'sh\\
	unsp-\textnormal{dance}=ind.m\\
	\glt `he dances' \citep[167]{hollow1970}

	\item\label{dancep2} \glll máana'po'sh\\
	waa-rą'p(E)=o'sh\\
	unsp-\textnormal{dance}=ind.m\\
	\glt `he dances' \citep[167]{hollow1970}
\end{xlist}
\end{exe}

Since speakers seem to produce both forms without a change in meaning, one explanation is that there is a trend in Mandan that disfavors bilabial stop codas. Historically, the stem vowel complementizer /=E/ has likely been reanalyzed as part of the root for such words. With the passing of the last L1 speaker in 2016, there is no way to definitively rule out the possibility of such a cluster being possible, but in the entire corpus, no /pp/ cluster is attested. I posit that this lack of /pp/ clusters arises from language change rather than some synchronic phonological conspiracy to prevent such clusters, as all lexical stems in the corpus that end in /p/ have a variant that also ends in [pe]. Given time, it is possible that Mandan would have shifted to make word-final /p/ illicit, since instances of variants ending in [pe] are more common than those with just [p] in the corpus.\footnote{This dispreference for word-final /p/ may be restricted to Nuu'etaa variety, as  \citet{maximilian1839} does not show this same variation between stem-final /p/ and [pe] in the Ruptaa variety. There are insufficient data, however, to conclusively say if preference for adding the stem vowel /=E/ onto /p/-final stems was not also present in Ruptaa.}

The second pattern to emerge from (\ref{CClenition}) is that the other stops fricativize the initial element in the /C\textsubscript{i}C\textsubscript{i}/ cluster. Just as \citet[37]{hollow1970} describes, /tt/ clusters become [st]. However, \citeauthor{hollow1970} does not notice that /kk/ clusters become [hk], as shown in (\ref{CClenitionex2}), though in \citet{hollow1973a}, he does transcribe /kk/ clusters as [hk] in a plurality of cases. The third and final pattern that we see in (\ref{CClenitionex4}) through (\ref{CClenitionex7}) regarding the spellout of /C\textsubscript{i}C\textsubscript{i}/ is that in clusters where C\textsubscript{i} is a fricative, the cluster is simplified to a singleton, i.e., /S\textsubscript{i}S\textsubscript{i}/ $\to$ [S\textsubscript{i}], where S is any fricative.

Given these three patterns, we can make an overall observation regarding the realization of /C\textsubscript{i}C\textsubscript{i}/ clusters: the first element in such clusters undergoes lenition. The main distinction between the three patterns described above is that the first element in underlying stop clusters is spirantized, while the first element in underlying fricative clusters are elided. Clearly, there is a conspiracy in Mandan to prevent pseudo-geminates given the fact that such sequences never occur in the corpus, even though there are frequent conditions where /C\textsubscript{i}C\textsubscript{i}/ clusters can arise through morphological processes, such as compounding, suffixation, and encliticization. This conspiracy seems motivated by the Obligatory Contour Principle (OCP) in the sense of \citet{mccarthy1986}, i.e., certain adjacent identical elements are prohibited at the surface level. It is preferable to increase the sonority of the first element of a pseudo-geminate (e.g., through fricativizating or eliding it) than it is to have identical obstruents come into contact.

We can summarize instances of pseudo-geminate dissimilation by appealing to the rules in (\ref{rulesPseudoGeminate}) below.

\begin{exe}
\item\label{rulesPseudoGeminate} Rules for pseudo-geminate dissimilation

	\begin{xlist}

	\item\label{rulesPseudoGeminate1} \textbf{/t/ Lenition Rule}\\
	 /t/ $\to$ [s] / {\longrule}t\\
	\textit{In a} /tt/ \textit{sequence, the first} /t/ \textit{becomes} [s].

	\item\label{rulesPseudoGeminate2} \textbf{/k/ Lenition Rule}\\
	/k/ $\to$ [h] / {\longrule}k\\
	\textit{In a} /kk/ \textit{sequence, the first} /k/ \textit{becomes} [h].

	\item\label{rulesPseudoGeminate3} \textbf{Fricative Deletion Rule}\\
	/S\textsubscript{i}/ $\to$ $\varnothing$ / {\longrule}S\textsubscript{i}\\
	\textit{When two identical fricatives form a cluster, the first one is deleted.}

	\end{xlist}

\end{exe}

A restriction against [C\textsubscript{i}C\textsubscript{i}] sequences alone is insufficient to account for this distribution of OCP-motivated behaviors, given that plosive-plosive and fricative-fricative pseudo-geminate clusters each have differing strategies for the resolution of pseudo-geminates. We know that /T\textsubscript{i}T\textsubscript{i}/ clusters, where T is a plosive, become either [st] or [hk], depending on the identity of the second T.\footnote{There is no instance of a /pp/ cluster in the corpus, and there are no remaining L1 speakers from whom to seek judgments, so we cannot definitively say that /pp/ is illicit or that /pp/ clusters involve lenition of  some time. However, the overall trend we see is that [C\textsubscript{i}C\textsubscript{i}] is not permissible for any other obstruent, so we can infer that [pp] is likewise prohibited in surface representation. The question is whether the initial /p/ would lenite to [h] like we see in /kk/ $\to$ [hk] or to something else. \citet{rankin2015} remark that /hC/ sequences in Proto-Missouri Valley Siouan originate from /C\textsubscript{i}C\textsubscript{i}/ sequences. The /k/ Lenition rule in Mandan is strikingly similar, and the /t/ Lenition differs only in that the first /t/ in a /tt/ sequence is not totally debuccalized. Sharing this process with Hidatsa and Crow to the exclusion of the rest of the Siouan family adds weight to Mandan being more closely related to Missouri Valley Siouan than to other branches of the Siouan language family.} Furthermore, we know that /S\textsubscript{1}S\textsubscript{1}/ clusters elide the first fricative to become [S\textsubscript{1}]. Each of these three strategies relies on increasing the sonority of the initial element in the cluster, with stops becoming fricatives, and fricatives eliding since there are no corresponding voiceless sonorants to which they can lenite, e.g., /ss/ $\to$ [ɾ̥s] is not a possible input-to-output relationship in Mandan. This lack of a viable segment to which an initial fricative can lenite is likely the motivation for deletion.

\subsection{Initial sonorant fortition}\label{sonorantfortition}

Sonorants have already been discussed in \sectref{sonorants}, but there is an additional set of realizations for sonorants independent of the description above because Mandan sonorants are also sensitive to prosodic boundaries.

The sonorants in Mandan are the phonemes that have the highest degree of variability in terms of allophony. In \sectref{sonorants}, we saw that sonorants are sensitive to their position within a word or phrase, e.g., \citet{hollow1970} notes that /ɾ/ is realized as a flap word-internally, but word-initially, it fortifies to a [ⁿd] when followed by an oral vowel. The /ɾ/ will never become a prenasalized stop when before a nasal vowel. This word-initial fortition pattern for /ɾ/ is optionally mirrored in /w/, which can turn to [ᵐb] at the left edge of a word when preceding an oral vowel, though [w] is much more common in recordings of speakers born in the twentieth century.

In each of the examples below, it is not possible for a prenasalized stop to appear anywhere other than word-initially. The data in (\ref{wordfortR1}) and (\ref{wordfortW1}) show that sonorants fortify word-initially when followed by an oral vowel, but in (\ref{wordfortR2}) and (\ref{wordfortW2}), sonorants remain sonorants intervocalically. When a sonorant is followed by a nasal vowel, nasal harmony takes place according to the rules described in \sectref{nasalharmony}, as we see in (\ref{wordfortR3}) and (\ref{wordfortW3}).

\begin{exe}
\item\label{wordfortR} Word-initial fortition for /ɾ/
	\begin{xlist}
	\item\label{wordfortR1} \glll {raskápo'sh}~\textnormal{[ⁿda.ˈska.poʔʃ], *[ɾa.ˈska.poʔʃ]}\\
	ra-skap=o'sh\\
	{ins.mth}-\textnormal{be.wet}={ind.m}\\
	\glt `he tasted it'
	\item\label{wordfortR2} \glll {óraskapo'sh}~\textnormal{[ˈo.ɾa.ska.poʔʃ], *[ˈo.ⁿda.ska.poʔʃ]}\\
	o-ra-skap=o'sh\\
	{pv.irr-ins.mth}-\textnormal{be.wet}={ind.m}\\
	\glt `he will taste it'
	\item\label{wordfortR3} \glll {nuráskapo'sh}~\textnormal{[nũ.ˈɾa.ska.poʔʃ], *[ⁿdũ.ˈɾa.ska.poʔʃ]}\\
	rų-ra-skap=o'sh\\
	{1a.pl-ins.mth}-\textnormal{be.wet}={ind.m}\\
	\glt `we tasted it'
	\end{xlist}
\item\label{wordfortW} Word-initial fortition for /w/
	\begin{xlist}
	\item\label{wordfortW1} \glll {wáarehe'sh}~\textnormal{[ˈwaː.ɾe.heʔʃ $\sim$ ˈᵐbaː.ɾe.heʔʃ]}\\
	waarehe=o'sh\\
	\textnormal{understand}={ind.m}\\
	\glt `he understands it'
	\item\label{wordfortW2} \glll {ówaarehe'sh}~\textnormal{[ˈo.waː.ɾe.heʔʃ], *[ˈo.ᵐbaː.ɾe.heʔʃ]}\\
	o-waarehe=o'sh\\
	pv.irr-\textnormal{understand}={ind.m}\\
	\glt `he will understand it'
	\item\label{wordfortW3} \glll {mawáarehe'sh}~\textnormal{[mã.ˈwaː.ɾe.heʔʃ], *[ᵐbã.ˈwaː.ɾe.heʔʃ]}\\
	wą-waarehe=o'sh\\
	1s-\textnormal{understand}={ind.m}\\
	\glt `he understands me'
	\end{xlist}
\end{exe}

Nasal stops, however, are possible when a sonorant appears utterance-initially or when there is some intonational break, such as when a sonorant is the initial element in a right dislocated phrase, or part of a series of connected independent clauses as seen below. These cases of initial fortition can be seen throughout the corpus, as exemplified by the example from \citet{lowie1913} in (\ref{uttfort}).

\begin{exe}
\item\label{uttfort} Utterance-initial fortition for /ɾ/ and /w/

	\glll \uline{N}éhąk nihų́po'sh. \uline{N}éhąk ~ ~ ~ nihų́ųshi'sh. Ínimashuto'sh. Manániho'sh.  ~ ~ \uline{M}aráraapininu'sh. ~ ~ ~ ~ ~ ~ ~ ~ ~ ~ ~ ~ ~ ~ ~ ~ ~ ~ ~ \uline{M}óorakaske'sh. \uline{M}óorakiru'sh. ~ ~ ~ ~ ~ ~ Ní'takorashiipo'sh.\\
	re=hąk rį-hųp=o'sh re=hąk ~ ~ ~ rį-hųųshi=o'sh i-rį-wąshut=o'sh wa-ra-rįh=o'sh ~ ~ wa-ra-raaprį=rų=o'sh ~ ~ ~ ~ ~ ~ ~ ~ ~ ~ ~ ~ ~ ~ ~ ~ ~ ~ ~ wa-o-ra-ka-skE=o'sh wa-o-ra-kru=o'sh ~ ~ ~ ~ ~ ~ r'-įįtake\#o-ra-shiip=o'sh\\
	dem.prox=psnl.stnd 2poss-\textnormal{moccasin}=ind.m dem.prox=psnl.stnd ~ ~ ~ 2poss-\textnormal{leggings}=ind.m pv.ins-2s-\textnormal{clothe}=ind.m unsp-2a-\textnormal{wrap}=ind.m ~ ~ unsp-2a-\textnormal{wear.around.neck}=anf=ind.m ~ ~ ~ ~ ~ ~ ~ ~ ~ ~ ~ ~ ~ ~ ~ ~ ~ ~ ~ unsp-pv.irr-2a-ins.frce-\textnormal{tie}=ind.m unsp-pv.irr-2a-\textnormal{coil}=ind.m ~ ~ ~ ~ ~ ~ 2poss-\textnormal{forehead}\#pv.irr-2a-\textnormal{be.rough}=ind.m\\
	\glt	`These here are your moccasins. These here are your leggings. This is your shirt. This is your blanket. This is that necklace of yours. This is your earring. This is your head ornament. These are your face-pendants.' \citep[358]{lowie1913}\footnote{\citet{lowie1913} glosses this selection as a single long sentence, but the instances of utterance-initial [m] and [n] along with the sentence-final male addressee markers =\textit{o'sh} strongly implies the presence of intonational breaks due to the start of new sentences within the narrative. I have thus revised the presentation of these clauses as separate sentences, rather than a chain of clauses as \citeauthor{lowie1913} does.}

\end{exe}

In data obtained from the early twentieth century from \citet{kennard1936}, \citet{densmore1923}, and especially \citet{lowie1913} (i.e., from speakers born in the middle of the nineteenth century before the Reservation period), utterance-initial fortition is more common but still not entirely consistent. Data collected by \citet{hollow1970} and \citet{trechter2012} show that utterance-initial fortition varies by speaker, but this fortition is less common overall to the point of being negligible in some speakers.

We can contrast utterance-initial fortition with word-initial fortition, where /ɾ/ to [ⁿd] is obligatory with /w/ to [ᵐb] being optional and uncommon. This contrast in fortitions suggests a change had taken place from the nineteenth to the twentieth century with respect to how strictly these fortitions took place. A similar process can be seen in contemporary Hidatsa, where older sources state that /ɾ/ and /w/ undergo identical fortitions to [n] and [m] after an intonational break (\citeauthor{boyle2007} \citeyear{boyle2007}:26, \citeauthor{park2012} \citeyear{park2012}:22), though my own fieldwork and that of \citeauthor{park2012} note that speakers born after the middle part of the twentieth century spontaneously produce [n] and [w] as word-initial fortition or even at affix boundaries word-internally.\footnote{\citet[183]{lowie1913} remarks that Hidatsa speakers occasionally produce [β] for /w/ before /i/, which is something not otherwise observed in Mandan. Personal fieldwork that I have done corroborates that certain speakers from the \textit{Xóshga} band of Hidatsa will produce a more approximant-like [β̞] to a simple [b] instead of /w/ at times, though there does not appear to be a specific conditioning environment.} I have argued that Mandan and Hidatsa, along with Crow, are more closely genetically related to each other than to other Siouan languages \citep{kasak2015}, but it is unclear if this sonorant fortition is a feature inherited from a common ancestor or an innovation that was transmitted from one to another.\footnote{\citet{graczyk2007} notes that Crow also has two sonorants that have differing allophones depending on their position within a word. \citet[3]{lowie1942} writes in his Crow grammar that word-initial sonorants are ``weakly nasalized'' to the point of having the quality of [ᵐb] and [ⁿd], respectively. \citeauthor{lowie1913} had been working with Crow speakers since 1907, and many of his consultants had been born in the early- to mid-nineteenth century. My own fieldwork among contemporary Crow speakers reveals that word-initial sonorants are plain voiced stops without any prenasalized quality with strongly positive VOTs. It is interesting to note that both these Siouan languages on the Upper Missouri have been gradually losing allophonic prenasalized stops over the course of the twentieth and twenty-first centuries. Crow has shifted towards plain voiced stops, while Mandan only retains the prenasalized coronal allophone [ⁿd] word-initially, with /w/ remaining [w] word-initially.} Given that all speakers of Mandan for the past century have also been fluent in Hidatsa, and both groups had been bilingual by tradition for at least a century prior to the Reservation period, the directionality of such a feature is unclear. The more articulated types of fortition found in Mandan suggests that this areal feature was transmitted from Mandan to Hidatsa and Crow, but without older data, this position is conjectural.

What is not conjecture is that the two kids of fortition in Mandan have different triggering conditions. These conditions are dependent on where within the prosodic hierarchy sonorant-initial words fall (\citealt{selkirk1986}, \citealt{selkirk2011}; \citealt{nesporvogel1986}; \citealt{beckmanpierrehumbert1986}; \textit{inter alios}). The prosodic hierarchy per \citet[221]{nesporvogel1986} appears below.

\begin{exe}

\item The prosodic hierarchy

Utterance (Utt) $\gg$ intonational phrase (ιP) $\gg$ phonological phrase (φP) $\gg$ prosodic word (ω) $\gg$ foot (Ft) $\gg$ syllable (σ) $\gg$ mora (μ)

\end{exe}

An ιP is often associated with clause-level syntactic structures, while a φP is a subcomponent of a clause, such as a syntactic phrase below the level of a complementizer phrase. An utterance can be composed of multiple clause-level syntactic structures. An example of this structure appears below with an example utterance in English broken down according to this hierarchy in (\ref{ExProsodicHierarchyExample}):

\begin{exe}

\item\label{ExProsodicHierarchyExample} Example of the prosodic hierarchy\\

\begin{tikzpicture}[frontier/.style={distance from root=6cm}]

\Tree[.Utt [.ιP [.φP [.ω [.σ A ] [.Ft [.σ Brì ] [.σ tish ] ] ] [.ω [.Ft [.σ bá ] [.σ ker ] ] ] ] [.φP [.ω [.Ft [.σ bròught ] ] ] [.ω [.σ the ] [.Ft [.σ brów ] [.σ nies ] ] ] ] ] ]
\end{tikzpicture}

\end{exe}

As previously discussed, /ɾ/ becomes [ⁿd] word-initially, but can optionally become [n] after an intonational or utterance break. For this reason, we can say that one kind of fortition works on the level of the prosodic word, while the other works on the level of the utterance. For speakers born before the twentieth century, /w/ becomes [m] after an intonational break or utterance break, but seems to be optional among speakers born after the beginning of the twentieth century. A summary of the rules for initial sonorant fortition appear in (\ref{sonorantallophonyfinal}) below.

\begin{exe}
\item\label{sonorantallophonyfinal} Rules for initial sonorant fortition

	\begin{xlist}
		\item \textbf{Word-initial /ɾ/-Fortition Rule}\\
		/ɾ/ $\to$ [ⁿd] / \#{\longrule}\\
		\textit{Underlying} /ɾ/ \textit{becomes} [ⁿd] \textit{word-initially.}

		\item \textbf{Word-initial /w/-Fortition Rule} (archaic)\\
		/w/ $\to$ [ᵐb] / \#{\longrule}\\
		\textit{Underlying} /w/ \textit{becomes} [ᵐb] \textit{word-initially in older varities of Mandan.}

		\item \textbf{Utterance-initial /ɾ/-Fortition Rule} (optional)\\
		/ɾ/ $\to$ [n] / \textsubscript{Utt}[{\longrule}\\
		\textit{Underlying} /ɾ/ \textit{optionally becomes} [n] \textit{utterance-initially.}

		\item \textbf{Utterance-initial /w/-Fortition Rule} (optional)\\
		/w/ $\to$ [m] / \textsubscript{Utt}[{\longrule}\\
		\textit{Underlying} /w/ \textit{optionally becomes} [m] \textit{utterance-initially.}
	\end{xlist}

\end{exe}

Initial sonorant fortition is a kind of boundary sensitivity, but it is not discussed in the word boundary-sensitive morpho-phonology section because this sensitivity is strictly prosodic. The main contribution of this description is that it clarifies the distribution of these allophones for sonorants and allows us to make predictions about the prosodic and narrative structure of Mandan sentences by pointing to where word-initial [m] and [n] appear without an accompanying nasal vowel within an utterance versus their word-initial counterparts. The presence for initial nasal consonants without an accompanying nasal vowel is thus a metric for gauging whether a speaker is beginning a new utterance or discontinuing the previous thought. Likewise, prenasalization and obstruentization of [ɾ] is a very productive metric for detecting where the left edge of a word boundary begins in Mandan.

\subsection{Ablaut}\label{ablaut}

As \citet{rood1983} and \citet{jones1983b} note, ablaut is a morpho-phonological feature of all Siouan languages. This system involves apophonic alternation of a vowel within a lexical stem before certain enclitics or when an ablaut vowel is placed into a certain prosodic or syntactic environment. An example of this process in Mandan can be seen below in (\ref{AblautExampleMandan}) with the ablauting elements shown in bold.



\begin{exe}

\item\label{AblautExampleMandan} Example of Mandan ablaut

	\begin{xlist}

	\item\label{AblautExampleMandana}
	\glll rar\textbf{ée}ho're\\
		ra-rEEh=o're\\
		2a-\textnormal{go.there}=ind.f\\
		\glt `you went there'

	\item\label{AblautExampleMandanb}
	\glll rar\textbf{áa}hinito're\\
		ra-rEEh=rįt=o're\\
		2a-\textnormal{go.there}=2pl=ind.f\\
		\glt `you (pl.) went there'

	\end{xlist}

\end{exe}

Across the Siouan language family, there are three recognized ablaut grades: \textit{e}-grade, \textit{a}-grade, and \textit{į}-grade. The \textit{e}-grade and \textit{a}-grade distinction is still productive in Mandan. There are some fossilized examples of the \textit{į}-grade ablaut in Mandan, but it is highly restricted in its distribution. Further explanation on this \textit{į}-grade ablaut in Mandan can be seen in \sectref{allomorphkti} and \sectref{encliticOOTE}. The Dakotan branch of Mississippi Valley Siouan and Biloxi of Ohio Valley Siouan have preserved all three grades, though it is most prodigiously attested in Dakotan. An example of this ablaut in Lakota appears in (\ref{lakotaablaut}) below.

\begin{exe}
\item\label{lakotaablaut} Three-way ablaut distinction in Lakota \citep[754]{ullrich2011}

	\begin{xlist}

	\item\label{lakotaablaut1} \textit{a}-grade

	\glll	Šúŋka waŋ \textbf{sápa} čha waŋbláke.\\
	šúŋka =waŋ sápA =čha waŋ-w-yákA\\
	\textnormal{dog} =indf \textnormal{be.black} =rel pv.indf-1a-\textnormal{see}\\
	\glt `I saw a black dog'

	\item\label{lakotaablaut2} \textit{e}-grade

	\glll Šúŋka kiŋ hé \textbf{sápe}.\\
	šúŋka =kiŋ hé sápA\\
	\textnormal{dog} =def dem.prox \textnormal{be.black}\\
	\glt `The dog [here] was black'

	\item\label{lakotaablaut3} \textit{į}-grade

	\glll Šúŋka kiŋ hé \textbf{sápiŋ} na tȟáŋka.\\
	šúŋka kiŋ hé \textbf{sápA} =na tȟáŋka\\
	\textnormal{dog} =def dem.prox \textnormal{be.black} =\textnormal{and} \textnormal{be.big}\\
	\glt `The dog [here] was black and big'

	\end{xlist}

\end{exe}

Throughout this book, the underlying ablaut vowel has been represented by /E/ or /Eː/, following similar conventions for other Siouan languages, where underlying ablaut vowels are represented by a capital vowel. In Mandan, the default realization of the ablaut vowel is [e] or [eː], depending on its underlying length. It should be noted that there are very few instances where the ablaut vowel is long, with the common motion verb /ɾEːh/ `go there' being the most frequent example. It is important to identify whether a root contains an ablaut vowel or not whenever encountering a novel word with [e] or [eː], as there is a set of enclitics that will trigger ablaut and cause the /E/ or /Eː/ to be realized as [a] or [aː] instead. That set of enclitics is listed below.

\subsubsection{Morphologically conditioned ablaut}\label{morphologicallyconditionedablaut}

There are two conditioning factors for ablaut: one factor involves whether an ablaut vowel is followed by an ablaut-triggering enclitic or undergoes reduplication and the other is when the ablaut vowel is in certain syntactic environments. The overwhelming majority of instances where ablaut occurs in the corpus is due to the presence of an enclitic. Ablauting can occur on both lexical stems and on enclitics themselves when followed by other ablaut-inducing enclitics. The following enclitics all trigger ablaut.

\begin{exe}
\item\label{ablautenclitics} Ablaut-triggering enclitics

\begin{tabular}{lll}
/=a'shka/ 	& \textit{=a'shka}	&	possible modal\\
/=awį/		&	\textit{=ąmi}	&	continuous aspectual\\
/=ą't/		&	\textit{=ą't}	&	hypothetical complementizer\\
/=haa/		&	\textit{=haa}	&	simultaneous aspect\\
/=ta/			&	\textit{=ta}		&	male-directed imperative marker\\
/=rą/ 		&	\textit{=na}		&	female-directed imperative marker\\
/=rąątE/	&	\textit{=naate}	&	prospective aspectual\\
/=rį/			&	\textit{=ni}		&	same-subject switch-reference marker\\
/=rįįtE/	&	\textit{=niite}	&	celerative aspectual\\
/=rįt/		&	\textit{=nit}	&	second person plural\\
/=rįx/		&	\textit{=nix}	&	negative\\
/=rįk/		&	\textit{=nik}	&	iterative complementizer\\
/=skee/		&	\textit{=skee}	&	iterative aspect\\
/=xi/			&	\textit{=xi}		&	negative\\
/=$\varnothing$/ & --			&	continuous aspectual\\
\end{tabular}

\end{exe}

The trend we see in (\ref{ablautenclitics}) is that the overwhelming majority of enclitics that trigger ablaut involve an underlying nasal. \citet[28]{rood1983} posists that those enclitics that do not have an overt nasal may have at one time in Pre-Proto-Siouan had such an element. For example, the reflexes of the Proto-Siouan future enclitic *ktE in Lakota and Biloxi triggers \textit{į}-grade ablaut, so Pre-Proto-Siouan may have had **inktE or **įktE to condition this apophony. %The fact that the \textit{a}-grade in Mandan uniformly goes to a long vowel, regardless of the length of the underlying ablaut vowel, suggests that this process does originate from some underlying nasal segment that has since been deleted, resulting in compensatory lengthing of a preceding short vowel.
Additional work on the origins of ablaut in Siouan languages is needed.

Mandan only has \textit{e}- and \textit{a}-grade ablaut. The \textit{e}-grade is the default realization. We can see examples of ablaut in Mandan at work below. The data below contain examples of both \textit{e}-grade and \textit{a}-grade ablaut in Mandan with doublets showing an example of the default \textit{e}-grade accompanied by an example with an \textit{a}-grade form that has been conditioned by a morphological item specified above.

\begin{exe}
\item\label{ablautexamples} Examples of \textit{e}-grade and \textit{a}-grade ablaut

\begin{xlist}
	\item\label{ablautexamples1} /hE/ `see'
		\begin{xlist}
		\item /hE=o'sh/ $\to$ \textit{h\uline{é}'sh} `he sees it'
		\item /hE=ta/ $\to$ \textit{h\uline{á}ta} `look!'
		\end{xlist}

	\item\label{ablautexamples2} /ru-shE/ `take [by hand]'
		\begin{xlist}
		\item /ru-shE=o're/ $\to$ \textit{rush\uline{é}'re} `she took it'
		\item /ru-shE=rą/ $\to$ \textit{rush\uline{á}na} `take it!'
		\end{xlist}

	\item\label{ablautexamples3} /ptEh/ `run'
		\begin{xlist}
		\item /ptEh=ak/ $\to$ \textit{pt\uline{é}hak} `he ran and...'
		\item /ptEh=rįįtE=o're/	$\to$	 \textit{pt\uline{á}hiniito're} `she quickly ran'
		\end{xlist}

	\item\label{ablautexamples4} /rEEh/ `go there'
		\begin{xlist}
		\item /rEEh=o're/ $\to$ \textit{r\uline{ée}ho're} `she went there'

		\item /rEEh=haa/	$\to$ \textit{r\uline{áa}haa} `while going'
		\end{xlist}

\end{xlist}

\end{exe}

Whether a vowel ablauts or not is lexically determined. It is not the case that all /e eː/ vowels ablaut to [a aː] when followed by the enclitics in (\ref{ablautenclitics}).

\begin{exe}
\item\label{noablautstems} Stems with /e eː/ without ablaut

	\begin{xlist}

	\item\label{noablautstems1} /ee-reh/ `want to'

		\begin{xlist}

		\item /ee-reh=o'sh/ $\to$ \textit{éer\uline{e}ho'sh} `he wanted to'

		\item /ee-reh=rį/ $\to$ \textit{éer\uline{e}hini} `he wanted to and...'

		\end{xlist}

	\item\label{noablautstems2} /tee/ `die'

		\begin{xlist}

		\item /tee=o're/ $\to$ \textit{t\uline{ée}ro're} `he died'

		\item /waa-tee=rįx=o're/ $\to$ \textit{wáat\uline{ee}nixo're} `he didn't die'

		\end{xlist}

	\end{xlist}



	\item\label{noablautstems3} /ee-he/ `say'

		\begin{xlist}

		\item /ee-he=o'sh/ $\to$ \textit{éeh\uline{e}'sh} `he said it'

		\item /ee-rį-he=rį/ $\to$ \textit{éenih\uline{e}ni} `he said to you and...'

		\end{xlist}

\end{exe}

In the data above, there is not a single instance of /e eː/ ablauting to [aː]. These data, along with many others, demonstrate that this apophony does not apply to all stems involving /e eː/, but only a subset. These verbs must be learned, and are not intuitive, as these ablauting verbs are inherited from stems that also ablauted in Proto-Siouan. Further complicating this dichotomy between ablauting and non-ablauting formatives is the fact that not all speakers treat the same set of  enclitics as ablaut-inducing. The set of enclitics in (\ref{ablautenclitics}) is the superset of all enclitics that cause precedings /E/ or /Eː/ to ablaut, but some speakers have smaller subsets of ablauting enclitics. For example, the negative enclitic triggers ablaut for some speakers, but not for others. We can see this speaker-dependent ablaut in the data in (\ref{ablautnegation}) below.


\begin{exe}
\item\label{ablautnegation} Negation-triggered ablaut

	\begin{xlist}

	\item\label{ablautnegation1}
	\glll wáawaruut\uline{e}xikani~\textnormal{(OS)}\\
	waa-wa-ruutE=xi=ka=rį\\
	neg-1a-\textnormal{eat}=neg=hab=ss\\
	\glt `I didn't eat any and...' \citep[46]{hollow1973a}

	\item\label{ablautnegation2}
	\glll wáah\uline{a}xiniitek~\textnormal{(AE)}\\
	waa-hE=xi=rįįtE=ak\\
	neg-\textnormal{see}=neg=cel=ds\\
	\glt `she did not look quick enough...' \citep[167]{hollow1973a}

	\end{xlist}

\end{exe}

Mrs. Annie Eagle consistently ablauts /E/ and /Eː/ to [aː] before the negation markers /=xi/ or /=rįx/ throughout the corpus, while this is not the case for Mrs. Otter Sage. These two consultants are descended from different bands of Mandan, so it is not possible to tell if this tendency to ablaut before negation is a characteristic of one variety of Mandan or another, or if this linguistic variation is at the level of the individual.\footnote{\citet[37]{mixco1997a} states that stems ending in short vowels lengthen that vowel when the negative enclitic =\textit{xi} is added. I do not find this to be a straightforward rule, but a tendency for some speakers. There are two possible scenarios that make sense in light of what we know about the Mandan language. Firstly, it is possible that the vowel lengthening is an older productive process that has been lost by some speakers. Secondly,  it could also be an innovation by another group of Mandan speakers. The fact that Mrs. Otter Sage in (\ref{ablautnegation1}) and Mrs. Annie Eagle in (\ref{ablautnegation2}) both have differing strategies for dealing with pre-\textit{xi} lengthening---despite the fact that they are both sisters, daughters of Crow's Heart---suggests that this issue is not so cut and dry. Further phonetic analysis of Mandan recordings is needed to resolve this quandary.}

\begin{exe}
\item\label{ablautiterative} Iterative-triggered ablaut

	\begin{xlist}

	\item\label{ablautiterative1}
	\glll kisúk  \'{ı̨}'h\uline{e}r\uline{e}skeeroomako'sh~\textnormal{(WF)}\\
	ki-suk į'-hrE=skee-oowąk=o'sh\\
	mid-\textnormal{child} pv.rflx-caus=iter=narr=ind.m\\
	\glt `he became a child again' \citep[150]{hollow1973b}

	\item\label{ablautiterative2}
	\glll rakú'k\uline{a}r\uline{a}skeenito'sh~\textnormal{(AE)}\\
	ra-ku'=krE=skee=rįt=o'sh\\
	2a-\textnormal{give}=3pl=iter=2pl=ind.m\\
	\glt `you (pl.) are giving it to them again \citep[453]{hollow1970}

	\item\label{ablautiterative3}
	\glll karátaxa máak\uline{a}skeeki~\textnormal{(EB)}\\
	k-ra-tax=E wąąkE=skee=ki\\
	iter-ins.mth-\textnormal{make.loud.noise}=sv \textnormal{lie}.pos.aux=inter=cond\\
	\glt `when he continued crying' \citep[244]{trechter2012b}

	\item\label{ablautiterative4}
	\glll warúut\uline{e}skeeni~\textnormal{(EB)}\\
	wa-ruutE=skee=rį\\
	unsp-\textnormal{eat}=iter=ss\\
	\glt `he was hungry again and...' \citep[101]{trechter2012b}

	\end{xlist}

\end{exe}

Mr. Walter Face (1890--1965) consulted with \citet{kennard1936} on the latter's Mandan grammar sketch, and the iterative /=skee/ does not trigger ablaut for him in (\ref{ablautiterative1}). Mrs. Annie Eagle consistently ablauts before /=skee/ as in (\ref{ablautiterative2}), while Mr. Edwin Benson vacillates between ablauting and not ablauting, as seen in (\ref{ablautiterative3}) and (\ref{ablautiterative4}) above. All nasal enclitics in (\ref{ablautenclitics}) otherwise trigger ablaut.

One enclitic with an underlying nasal that never triggers ablaut is the attitudinal /=rąsh/, which adds the sense of hedging a statement in a way similar to English `sort of', `kind of', or `like.'

\begin{exe}
\item\label{noablaut} Lack of ablaut before a nasal enclitic

	\begin{xlist}
	\item\label{noablaut1}
	\glll náak\uline{e}nashini \textnormal{(EB)}\\
	rąąkE=rąsh=rį\\
	\textnormal{sit}.pos.aux=att=ss\\
	\glt `he was sitting and...' \citep[110]{trechter2012b}

	\item\label{noablaut2}
	\glll írush\uline{e}nashoomaksįh \textnormal{(OS)}\\
	i-ru-shE=rąsh=oowąk=sįh\\
	pv.ins-ins.hand-\textnormal{grasp}=narr=ints\\
	\glt `he was kind of holding him' \citep[45]{hollow1973a}

	\item\label{noablaut3}
	\glll ówa'pxenashoo \textnormal{(MG)}\\
	o-wa'-pxE=rąsh=oo\\
	pv.loc-pv.prce-\textnormal{stumble}=att=dem.mid\\
	\glt `she kind of broke through' \citep[81]{hollow1973a}

	\end{xlist}

\end{exe}

The fact that immediately preceding a nasal enclitic does not automatically trigger ablaut signifies that this process is rooted in more than just a simple phonological rule. The topicalizer /=rą/ likewise seems to not trigger ablaut, but it is difficult to say whether this is the case or not due to the fact that the topicalizer is almost always accompanied by some deictic demonstrative, which all feature a mid vowel.

\begin{exe}
\item\label{turtlefootnote} \glll	koshų́ųkaseena\\
	ko-shųųka=s=ee=rą\\
	3poss.al.pers=\textnormal{younger.brother}=def=dem.dist=top\\
\glt	`his younger brother' \citep[141]{trechter2012b}
\end{exe}

The number of instances of the topicalizer without an accompanying deictic demonstrative is very limited in the corpus, but in these situations, the topicalizer does not trigger ablaut:

\begin{exe}
\item\label{somethingoodfootnote} \glll wáashixtena\\
	waa-shi-xtE=rą\\
	nom-\textnormal{be.good}-aug=top\\
\glt `something good' \citep[127]{trechter2012b}
\end{exe}

As \citet{rankinetalnd} point out, Proto-Siouan featured the same three apex nasal vowels that Mandan possesses: i.e. /*ą *į *ų/. \citet{rood1983} posits that Pre-Proto-Siouan may have also had nasal mid vowel or nasal segements that disappeared but left the nasal feature behind on preceding vowels, and that some merger of these nasal mid vowels with oral vowels is what has caused the ablaut reconstructed in Proto-Siouan and in its daughter languages. These former nasal elements may have formerly been part of enclitics that caused nasalization of a mid vowel, but then those nasal mid vowels were lost to subsequent mergers by the time of early Proto-Siouan.

\citet{jones1983b} agrees that ablaut might be explained historically as remnants of a formerly regular phonological system. One piece of evidence to support the notion that nasality was a key component in creating this ablaut system can be seen with Mandan, where the overwhelming majority of ablaut-triggering enclitics are inherited from Proto-Siouan. These same enclitics were either enclitics or became encliticized during the development into Mandan. For example, the negative /=rįx/ is a cognate of /=šni/ in Lakota \citep{rankin2015}, and /=šni/ is also triggers ablaut in Lakota and Dakota \citep[754]{ullrich2011}. The enclitics that bear nasals that do not trigger ablaut seem to have been innovated from some element that is not clearly attributable to Proto-Siouan, e.g., /=rąsh/ does not have any other cognates in Siouan \citep{rankin2010}. These innovations have not become incorporated into what was a formerly regular morpho-phonological process since the conditions for creating ablaut had long since grammaticalized. More work is needed to flesh out this idea of ablaut being conditioned by nasality in Proto- and Pre-Proto-Siouan.

The other morphologically conditioned environment where ablaut occurs is when a sequence containing an ablaut vowel is reduplicated. Reduplication in Mandan copies the onset and the first mora of the syllable nucleus and prefixes this reduplicated formative onto the root of the word. In these constructions, the ablaut vowel in the reduplicated formative becomes [a] and the ablaut vowel in the root remains unchanged. We can see examples of reduplication-induced ablaut in the data below.


\begin{exe}
\item\label{ablautredup} Reduplication-induced ablaut in \citet{hollow1970}

	\begin{xlist}

	\item\label{ablautredup1}
	\glll kxakxé'sh\\
	kxE$\sim$kxE=o'sh\\
	dist$\sim$\textnormal{be.spotted}=ind.m\\
	\glt `it is spotted [in color]' \citep[126]{hollow1970}

	\item\label{ablautredup2}
	\glll Masáse\\
	wą-sE$\sim$sE\\
	\textsc{unsp}-\textsc{dist}$\sim$\textnormal{be.red}\\
	\glt `Red Butte' \citep[29]{hollow1970}

	\item\label{ablautredup3}
	\glll wíikxakxeka\\
	waa-i-kxE$\sim$kxE=ka\\
	nom-pv.ins-dist$\sim$\textnormal{be.spotted}=hab\\
	\glt `magpie' \citep[289]{hollow1970}

	\item\label{ablautredup4}
	\glll rashkáshke\\
	ra-shkE$\sim$shkE\\
	ins.foot-dist$\sim$\textnormal{jump}\\
	\glt `to tiptoe' \citep[231]{hollow1970}

	\item\label{ablautredup5}
	\glll xkaxką́ho'sh\\
	xkąh$\sim$xkąh=o'sh\\
	aug$\sim$\textnormal{move}=ind.m\\
	\glt `he is ambitious' \citep[316]{hollow1970}

	\item\label{ablautredup6}
	\glll rarúksiksįįro'sha?\\
	ra-ru-ksįį$\sim$ksįį=o'sha\\
	2a-ins.hand-dist$\sim$\textnormal{tickle}=int.m\\
	\glt `did you tickle her?' \citep[476]{hollow1970}

	\end{xlist}

\end{exe}

In (\ref{ablautredup1}) through (\ref{ablautredup4}), we see the expected behavior for the ablaut vowel when reduplicated. The reduplicated (i.e., prefixed) version of the verb undergoes ablaut, turning /E/ to [a], while the vowel in the original verbal root remains [e]. This prefixal reduplication indicates some kind of intensity or distributed quality of a state or action, and is quite productive. However,  we can see a possible additional variety of ablaut occurring in (\ref{ablautredup5}) and (\ref{ablautredup6}) whenever a formative with an underlying nasal is reduplicated. When prefixal reduplication takes place, only the onset and a single mora are reduplicated. No nasal quality is reduplicated onto the new prefix, so an underlying nasal vowel becomes an oral vowel. Similarly, a long vowel will become a short vowel. We can see this demonstrated in (\ref{ablautredup5} where the stem /xkąh/ `move' is reduplicated as [xka], with the vowel lacking its original nasal feature. In the case of (\ref{ablautredup6}), we see a long nasal vowel become a short oral vowel in addition to the underlying nasality of the vowel not being copied over to the reduplicated morph.\footnote{I assume throughout this book that nasality is underlying present on vowels, rather than arguing for some kind of underlying floating [+nasal] or underspecified /N/ that triggers regressive nasal harmony, i.e., */ksiiN/ instead of /ksįį/ `tickle'. I take this behavior for reduplication to mean that only certain articulatory gestures and their durations are truly copied to become reduplicated prefixally and not that some additional underlying feature is needed to allow such a pattern.}

\subsubsection{Syntactically conditioned ablaut}\label{syntacticallyconditionedablaut}

Enclitics are not the sole cause of ablaut in Mandan, as several auxiliary verbs can likewise serve as triggers. As is the case with enclitic-conditioned ablaut, the ablaut occurs in the environment where the auxiliary verb that immediately follows an underlying /E Eː/ is nasalized. These auxiliary verbs are all positional or existential in nature and convey a progressive reading to the subordinate verb. The these auxiliary verbs trigger ablaut and are listed below.

\begin{exe}
\item\label{ablautsyntax} Ablaut-triggering auxiliary verbs

\begin{tabular}{lll}
/hąąkE/ 	&	\textit{hą́ąke}	&	`standing' positional auxiliary verb\\
/rąąkE/		&	\textit{náake}	&	`sitting' positional auxiliary verb\\
/rąąkah/	&	\textit{náakah}	&	`sitting' habitual auxiliary verb\\
/ruurįh/	&	\textit{núunih}&	plural durational auxiliary verb\\
/wąąkE/		&	\textit{máake}	&	`lying' positional auxiliary verb\\
/wąąkah/	&	\textit{máakah}	&	`lying' habitual auxiliary verb\\
\end{tabular}

\end{exe}

Two of the these possitionals have a variant that expresses a habitual progressive action, one for sitting and one for lying. Examples of these auxiliaries co-ocurring with other verbs that have progressive readings appear below.

\begin{exe}
\item\label{auxablaut} Auxiliary-induced ablaut

	\begin{xlist}

	\item\label{auxablaut1}
	\glll wakí'karaar\textbf{a} hą́ąkeroomako'sh\\
	wa-ki'kraa=\textbf{E} hąąkE=oowąk=o'sh\\
	unsp-\textnormal{look.for}=\textbf{sv} \textnormal{stand}.pos.aux=narr=ind.m\\
	\glt `he was looking around' \citep[139]{hollow1973a}

	\item\label{auxablaut2}
	\glll 	píir\textbf{a} náakaani\\
	pii=\textbf{E} rąąkE=rį\\
	\textnormal{devour}=\textbf{sv} \textnormal{sit}.pos.aux=ss\\

	\glt `he was sitting there eating it up and...' \citep[122]{hollow1973b}

	\item\label{auxablaut3}
	\glll 	karátax\textbf{a} náakahoomako'sh\\
	ka-ra-tax=\textbf{E} rąąkah=oowąk=o'sh\\
	ins.frce-ins.mth-\textnormal{make.loud.noise}=\textbf{sv} \textnormal{sit}.pos.aux.hab=narr=ind.m\\
	\glt  `he was sitting there crying' \citep[222]{hollow1973b}

	\item\label{auxablaut4}
	\glll 	íra'reshh\uline{a}r\textbf{a} máakahąą\\
	i-ra'-resh\#hrE wąąkE=$\varnothing$\\
	pv.ins-ins.heat-\textnormal{be.hot}\#caus \textnormal{lie}.pos.aux=sim\\
	\glt `while making him hot' \citep[204]{hollow1973b}

	\item\label{auxablaut5}
	\glll íkih\textbf{a} má'kahaa\\
	i-kihE=\textbf{E} wą'kE=haa\\
	pv.ins-\textnormal{wait}=\textbf{sv} \textnormal{lie}.pos.aux=sim\\
	\glt `while they waited there' \citep[298]{hollow1973b}

	\item\label{auxablaut6}
	\glll  ų́'shkah\textbf{a}r\textbf{aa} má'kahini\\
	ų'sh=ka\#hrE=\textbf{E} wą'kah=rį\\
	\textnormal{be.thus}={hab}\#\textsc{caus}=\textbf{sv} \textnormal{lie}.{pos.aux.hab=ss}\\
	\glt `he did it that way and...' \citep[224]{hollow1973b}

	\item\label{auxablaut7}
	\glll íkih\textbf{a} núunihoomako'sh\\
	i-kihE=\textbf{E} ruurįh=oowąk=o'sh\\
	pv.ins-\textnormal{wait}=\textbf{sv} \textnormal{be.there}.pl.dur.aux=narr=ind.m\\
	\glt `they were there waiting' \citep[258]{hollow1973b}

	\end{xlist}

\end{exe}

These auxiliaries are sometimes combined with a verb that bears the simulatenous aspectual /=haa/, and in fast speech, it is sometimes difficult to perceive [h] as part of a consonant cluster. As such, it is often not clear in \citeapos{hollow1970} transcriptions whether he intends to mark a stem vowel /=E/ at the end of a verb or he cannot hear the juncture between the consonant and the [h] in /=haa/. \citeapos{kennard1936} transcriptions fare better in this respect. As such, the majority of the examples above in (\ref{auxablaut}) contain morphology such that we can distinguish between /=E/ and /=haa/.

The benefactive auxiliary \textit{kú'} `give' can also trigger ablaut, but it does not do so consistently. All speakers in the corpus have multiple instances of \textit{kú'} triggering and not triggering ablaut within the same narrative. However, the lack of ablaut before before \textit{kú'} in \citeauthor{kennard1934}'s (\citeyear{kennard1934,kennard1936}) transcribed narratives is much rarer than it is in \citeauthor{hollow1970}'s (\citeyear{hollow1970,hollow1973a,hollow1973b}). Both of the examples in (\ref{ablautkuP}) come from Mrs. Annie Eagle, \citeapos{hollow1970} primary consultant and the source for the plurality of the extant recorded sources of Mandan.

\begin{exe}
\item\label{ablautkuP} Variability in benefactive-triggered ablaut for Annie Eagle

	\begin{xlist}

	\item\label{ablautkuP1} \glll karúxų'he makú're mikák\\
	ka-ru-xų'h=E wą-ku'=E wįk=ak\\
	ins.frce-ins.hand-\textnormal{plow}=sv 1s-\textnormal{give}=sv \textnormal{be.none}=ds\\
	\glt `there was no one to plow for me' \citep[54]{hollow1973a}

	\item\label{ablautkuP2} \glll ą́ąwe rusháa makú'ta\\
	ąąwe ru-shE wą-ku'=ta\\
	\textnormal{all} ins.hand-\textnormal{take} 1s-\textnormal{give}=imp.m\\
	\glt `take all of it for me' \citep[78]{hollow1973a}

	\end{xlist}

\end{exe}

 Her own variability in ablauting before \textit{kú'} underscores the instability of its productivity, especially since she ablauts before certain enclitics more consistently than other speakers who contributed data. This pattern suggests that \textit{kú'} had begun being reanalyzed as an auxiliary that did not trigger ablaut, or at least optionally so, for speakers born around the turn of the twentieth century.\footnote{There are few instances of other auxiliaries not triggering ablaut in the corpus, but these examples can likely be attributed to a break in the prosody, i.e., these counterexamples are likely fragments or left dislocated elements.}

The final syntactic environment that triggers ablaut is when the stem vowel is added clause-finally to indicate that the ablauted verb is the first in a sequence, with the following verb taking place afterwards. We can see this clause-final ablaut in data below.

\begin{exe}

\item\label{CPfinalablaut} Sequence marking with ablaut

	\begin{xlist}

	\item\label{CPfinalablaut1} \glll Náak\textbf{a} inák waherés waká'roomako'sh.\\
	rąąkE=\textbf{$\varnothing$} irąk wa-hrE=s wa-ka'=oowąk=o'sh\\
	\textnormal{sit}.pos.aux=\textbf{cont} \textnormal{again} unsp-caus=def unsp-\textnormal{possess}=narr=ind.m\\
	\glt `He was sitting and then he asked for his food again' \citep[94]{hollow1973b}

	\item\label{CPfinablaut2}
	\glll Kxų́hini máapsitaar\textbf{a} inák ráahąmi\\
	kxųh=rį wąąpsi=taa=E=\textbf{$\varnothing$} irąk rEEh=awį\\
	\textnormal{lie.down}=ss \textnormal{morning}=loc=sv=\textbf{cont} \textnormal{again} \textnormal{go.there}=cont\\
	\glt `She lay down and once it was morning she went along again' \citep[103]{hollow1973a}

	\end{xlist}

\end{exe}

This ablaut is reminiscent of the process by which certain conjunctions in Lakota trigger \textit{e}-grade ablaut \citep[754]{ullrich2011}. This ablaut serves to temporally connect one clause with the following clause, albeit with a phonetically null coordinator. Only the ablaut itself overtly conveys that there is a relationship between the two clauses.

Similar to initial sonorant fortition, ablaut is a process that does have a kind of boundary sensitivity, but its sentitivity has to do with certain enclitics triggering ablaut and certain syntactic constructions triggering ablaut. Ablaut is not sensitive to word-internal boundaries as described in \sectref{boundarydependent} below.

\section{Word boundary-dependent morpho-phonology}\label{boundarydependent}
\largerpage
In previous published descriptions of the phonology of Mandan, there has been minimal attention paid to fine details of the interaction between its phonology and morphology. Specifically, there are various phonological processes that are described as being regular, but several systematic exceptions appear throughout the corpus. \citet[35]{hollow1970} and \citet[12]{mixco1997a} likewise state that there are phenomena that they do not address and leave these questions open for future research.


This section serves to address these open questions and underdescribed phenomena and to demonstrate that they are not actually exceptions, but instances of regular phonological processes being blocked by word boundaries that are word-internal. As such, this section is based on theoretical approaches put forth in \citet{kasak2019}, which attributes these irregularities to word boundary-sensitive phenomena. The processes that are sensitive to word-internal boundaries are hiatus resolution (see \sectref{epentheticprocesses}), metathesis (see \sectref{metathesis}), nasal harmony (see \sectref{nasalharmony}), and primary stress assignment (see \sectref{stress}).

\subsection{Hiatus resolution processes}\label{epentheticprocesses}

Mandan does not permit [V.V] or [VV̯] sequences.\footnote{The sole exception to this statement is the diphthong in the greeting \textit{háu}, as detailed previously in \sectref{subsectiondiphthongs}.} This conspiracy to prevent hiatus or diphthongs results in several different strategies for resolving /VV/ sequences. In \citet{hollow1970} and \citet{mixco1997a}, discussion of consonantal epenthesis is brief and restricted to showing a single example of a proposed phonological rule to insert glottal stops between two vowels. I elaborate upon the conditions under which this epenthesis occurs, and also define a second kind of epenthesis that was first suggested by \citet{carter1991a}: that the root-final /ɾ/ that \citeauthor{hollow1970} posits for a large number of words is not really part of that root, but is in fact an epenthetic segment that separates a long vowel from an element in the postverbal field.

In this subsection, I delve into the ways in which Mandan resolves hiatus, concluding that there are three ways hiatus is resolved in Mandan, with the data in (\ref{multiepenthesis}) showing an example that contains two epenthetic processes in Mandan. Epenthesis within a morphological word (i.e., a word that involves a stem plus affixation) resolves hiatus by inserting [ʔ], while hiatus between an enclitic and a stem where a long vowel is involved is resolved by inserting [ɾ]. In the example below, brackets represent the boundaries of the morphological word. Enclitics exist outside the bounds of this morphological word.

\begin{exe}

	\item\label{multiepenthesis} Two kinds of epenthesis to resolve hiatus

	\glll wapákanakini\textbf{'}eshkakere\textbf{r}oomako'sh\\
		{[}wa-pa-krąkrį-eshka{]}=krE=oowąk=o'sh\\
		unsp-ins.push-\textnormal{butcher}-sim=3pl=narr=ind.m\\
		\glt `they were kind of butchering them' \citep[86]{hollow1973b}

\end{exe}

This topic of epentheis receives a considerable amount of attention here, because the differing strategies for resolving hiatus in Mandan depending on the level on which the domain occurs: the morphological word (i.e., the stem and affixes) or the word in the phrase structure (i.e., the stem and enclitics). The argument that follows in \sectref{intrusiveglottal} is that the [ʔ]-insertion described in \citet{hollow1970} is restricted to word-internal hiatus. In \sectref{intrusiveR} and \sectref{shortvoweldeletion}, I argue that virtually all postverbal morphology in Mandan is actually enclitic in nature, and that this linking [ɾ] only occurs across enclitic boundaries from a lexical item onto a functional item that is an enclitic.

When hiatus occurs across an enclitic boundary involving two short vowels, the final short vowel is deleted, while the linking [ɾ] occurs in the environment of a long vowel at an enclitic boundary. \citet{carter1991a} is the first to posit that the root-final rhotic that \citet{hollow1970} describes is not part of the underlying representation, but the work herein is the first to take that assumption and extend the analysis to show that [ɾ] is not simply epenthetic for post-verbal elements, but specifically for enclitics in the environment of a heavy syllable. The fact that [ɾ]-insertion is predictable in Mandan provides phonological evidence for sensitivity to the morphological domains proposed in this book, i.e., internal word boundaries within a morphological word. The argumentation for this interpretation of the structure of postverbal elements in Mandan will be presented in more detail in \sectref{SyntaxAffixation} in the following chapter.

\subsubsection{Epenthetic [ʔ]}\label{intrusiveglottal}

\citet[47]{hollow1970} states that [V.V] sequences are illicit on the surface. To avoid hiatus, Mandan has an epenthetic [ʔ] that acts as the onset of the following syllable. One issue with \citeauthor{hollow1970}'s description of this process is that he does not distinguish between long and short vowels. As such, it is not clear whether this kind of epenthesis is exclusive to environments involving vowels of one particular length or must occur with both.

Several examples of this epenthesis rule at work appear below.

\begin{exe}
\item\label{intrusiveglottal1} Instances of [ʔ] epenthesis

\begin{xlist}

\item\label{glottalep1}
\glll	psi'éshka\\
	psi-eshka\\
	\textnormal{be.black}-smlt\\
\glt	`just black' \citep[74]{hollow1973a}

\item\label{glottalep2}
\glll	warú'uuxo'sh\\
	wa-ru-uux=o'sh\\
	1a-ins.hand-\textnormal{be.broken}=ind.m\\
\glt `I broke it with my hands' \citep[47]{hollow1970}


\item\label{glottalep3}
\glll	wa'ípteh\\
	wa-i-ptEh\\
	unsp-pv.ins-\textnormal{run}\\
\glt	`automobile' \citep[338]{hollow1970}

\item\label{glottalep4}
\glll	wáa'oshi\\
	waa-o-shi\\
	nom-pv.irr-\textnormal{be.good}\\
\glt	`good things' \citet[132]{hollow1973a}

\item\label{glottalep5}
\glll	wa'áahuuroomako'sh\\
	wa-aa-huu=oowąk=o'sh\\
	unsp-pr.tr-\textnormal{come.here}=narr=ind.m\\
\glt	`he brought them some' \citep[177]{hollow1973a}

\item\label{glottalep6}
\glll	wáa'aahuuki\\
	waa-aa-huu=ki\\
	neg-pv.tr-\textnormal{come.here}=cond\\
\glt	`if he brings it' \citep[120]{hollow1973a}

\end{xlist}
\end{exe}

Each instance of [ʔ]-insertion in (\ref{intrusiveglottal1}) occurs regardless of whether the surrounding vowels are long or short. Furthermore, this kind of epenthesis can occur pre- or post-tonically. As such, [ʔ]-insertion is purely conditioned by underlying /V.V/ sequences within the boundaries of a morphological word, i.e., within the scope of the stem plus affixes but excluding enclitics.

When observing the distribution of where this epenthesis is most likely to occur, it is most commonly associated with the prefix field. Very rarely will a postverbal element participate in [ʔ]-insertion. In fact, the only morphological item following a verbal root that triggers [ʔ]-insertion is the similitive suffix \textit{-eshka} and its alternative form \textit{-esh}. This restriction is caused by the fact that the similitive suffix is the only productive suffix that is vowel-initial (see \sectref{SecSuffixField} for additional description of suffixes in Mandan).\footnote{There is another vowel-initial suffix, the collective \textit{-aaki}, but it appears restricted to compounds involving the word \textit{numá'k} `person'. As such, there are no examples in the corpus showing what happens when \textit{-aaki} comes into contact with a vowel-final stem, though the behavior of the similitive suggests that it would trigger [ʔ] epenthesis.}

Examples of [ʔ]-insertion in the suffix field appear below. In the first two examples, we see the expected behavior of glottal stop insertion when hiatus occurs in the prefix field, and the last two examples show glottal stops resolving hiatus involving suffixes. We can surmise from this distribution that [ʔ]-epenthesis is not restricted to the prefix field alone, as the few genuine suffixes that exist in Mandan resolve hiatus-creating conditions in an identical manner.

\begin{exe}
\item\label{glottalstopsuffixCh2} Examples of [ʔ]-insertion with affixes

	\begin{xlist}

	\item\label{glottalstopsuffixCh2A} \glll wáa'ishąhemik\\
		waa-ishąhe\#wįk\\
		nom-\textnormal{price}\#\textnormal{be.none}\\
		\glt `credit, debt' \citep[288]{hollow1970}

	\item\label{glottalstopsuffixCh2B} \glll ra'úux\\
		ra-uux\\
		ins.mth-\textnormal{be.broken}\\
		\glt `to break something between one's teeth' \citep[264]{hollow1970}

	\item\label{glottalstopsuffixCh2C} \glll psi'éshkaso'sh\\
		psi-eshka=s=o'sh\\
		\textnormal{be.black}-sim=def=ind.m\\
		\glt `it was definitely kind of black' \citep[434]{hollow1970}

	\item\label{glottalstopsuffixCh2D} \glll tashká'eshka?\\
		tashka-eshka\\
		\textnormal{how}-sim\\
		\glt `how come?' \citep[42]{hollow1973a}

	\end{xlist}

\end{exe}

There has been some debate among Siouanists regarding whether we can demonstrate that modern Siouan languages have true underlying onset glottal stops (\citeauthor{larson2016} p.c., \citeauthor{mirzayan2010} p.c., \citeauthor{ullrich2011} p.c.), since there is clear evidence for certain Proto-Siouan roots being reconstructed with word-initial *ʔ \citep{rankin2015}. In some other Siouan languages, like Lakota, there is a small set of roots that are assumed to be [ʔ]-initial. A number of prefixes have allomorphs that are specific to vowel-initial stems and as such, we can point to these instances of allomorphy as evidence that we do not have underlying /ʔ/ or epenthetic [ʔ] to satisfy a requirement to have an onset.

The examples below demonstrate that there are special allomorphs in Mandan for consonant-initial stems versus vowel-initial stems. In (\ref{nowordinitialglottals1}), the underlying vowel in the first person active plural prefix /rV-/ copies the following vowel, creating a single long vowel, while the integrity of the underlying vowel in /rų-/ is maintained when the stem is consonant-initial in (\ref{nowordinitialglottals2}). Similarly, the alienable possession prefix /ta-/ is fully realized in (\ref{nowordinitialglottals3}), but when prefixed onto a vowel-initial stem, the allomorph /tV-/ harmonizes with the initial vowel of the stem to produce a long vowel like in (\ref{nowordinitialglottals4}). There are other allomorphic alternations that behave similar to the prefixes above, which are explained more thoroughly in the following chapter. The point still stands, however, that we cannot make a case for these intervocalic glottal stops being present in the underlying representation. As such, these instances of /ʔ/ must be epenthetic.


\begin{exe}

\item\label{nowordinitialglottals} Vowel-initial stems and allomorphy

\begin{xlist}

\item\label{nowordinitialglottals1}
\glll ríisehka'sh\\
	rV-i-sek=ka=o'sh\\
	1a.pl-pv.ins-\textnormal{make}=hab=ind.m\\
\glt	`we [always] made it' \citep[356]{lowie1913}

\item\label{nowordinitialglottals2}
\glll	nuhé'sh\\
	rų-hE=o'sh\\
	1a.pl-\textnormal{see}=ind.m\\
\glt `we see it' \citep[71]{hollow1970}

\item\label{nowordinitialglottals3}
\glll	tamí'ti\\
	$\varnothing$-ta-wį'ti\\
	3poss-al-\textnormal{village}\\
\glt	`his village' \citep[482]{hollow1970}

\item\label{nowordinitialglottals4}
\glll	tóominike\\
	$\varnothing$-tV-owrįk=E\\
	3poss-al-\textnormal{beans}=sv\\
\glt	`her beans' \citep[482]{hollow1970}

\end{xlist}

\end{exe}

We can deduce from the data above that Mandan does not have word-initial /ʔ/, given the presence of allomorphy for prefixes that select for consonant-initial and vowel-initial stems. As such, these [ʔ] segments are not underlying and are being added epenthetically when hiatus occurs within the domain of a morphological word. We can codify this rule as follows:

\begin{exe}

\item\label{GlottalStopEpenthesisRule} \textbf{[ʔ]-Epenthesis Rule}\\
	$\varnothing$ $\to$ [ʔ] / V(ː){\longrule}V(ː) (word-bounded)\\
	\textit{Insert a} [ʔ] \textit{between two vowels of any length within the domain of a morphological word, i.e., between a stem and an affix.}

\end{exe}

This [ʔ]-insertion is not the only kind of epenthesis in Mandan, as [ɾ] also appears in certain conditions to prevent hiatus. This process is described below.

\subsubsection{Epenthetic [ɾ]}\label{intrusiveR}

\citet{carter1991a} is the first to propose that all instances of what \citet{hollow1970} interprets as underlying root-final /ɾ/ in Mandan are really just due to the fact that those roots contain long vowels, though \citet{mixco1997a} includes \citeauthor{hollow1970}'s stem-final /ɾ/ in his grammar. \citeauthor{carter1991a} argues that [ɾ] is epenthetic by analyzing data collected by Prince \citet{maximilian1839} on dialectal differences between Mandan villages. In looking at the forms below, \citeauthor{carter1991a} points out that both dialects differ in how they handle hiatus resolution after a verb stem: Nuu'etaare will delete the second of two short vowels, while [ɾ] appears after a stem with a long vowel or one that ends in [ʔ]. Ruptaare, however, appears to insert [ʔ] between short vowels and harmonizes the following short vowel to the preceding long vowel.

\begin{exe}
\item\label{dialectRs} Differing hiatus strategies in the Nuu'etaa and Ruptaa dialects

\begin{tabular}{llll}
\textbf{Underlying}&
	\textbf{Nuu'etaare}&
	\textbf{Ruptaare}&
	\textbf{Gloss}\\
/wa-he=o'sh/&
	\textit{wahé'sh}&
	\textit{wahé'osh}&
	`I saw'\\
/kri=o'sh/&
	\textit{kirí'sh}&
	\textit{kirí'osh}&
	`he arrived there'\\
/tee=o'sh/&
	\textit{téero'sh}&
	\textit{tée'e'sh}(?)&
	`he died'\\
/huu=o'sh/&
	\textit{húuro'sh}&
	\textit{húu'u'sh}(?)&
	`he came here'\\
/kihkra'=o'sh/&
	\textit{kihkará'ro'sh}&
	\textit{kihkará'a'sh}(?)&
	`he looked for it'\\
/wa-hrą'=o'sh/&
	\textit{wahaná'ro'sh}&
	\textit{wahaná'a'sh}(?)&
	`I sleep'\\


\end{tabular}

\end{exe}

\citet[487]{carter1991a} merely posits that [ɾ] is epenthetic in Mandan, not commenting on the conditions under which it occurs, aside from postverbally when hiatus involves a stem that ends with a long vowel or a glottal stop.\footnote{The Ruptaare forms in (\ref{dialectRs}) have been slightly altered here. \citet{carter1991a} suggests that the male-addressee indicative enclitic /=o'sh/ becomes /=sh/, but I suggest that the vowel in the enclitic merely harmonizes with the preceding vowel. More work on \citeapos{maximilian1839} data is needed, but such work is ultimately outside the scope of this book.} One reason why \citeauthor{carter1991a} reached this conclusion may stem from the fact that he consistently recorded the difference between long and short vowels. \citeauthor{hollow1970} does not record long vowels, and as such he proposes that there exist minimal pairs between lexical items where some forms seem to have an [ɾ] that appears in certain conditions, while others do not. When word-final, \citeauthor{hollow1970} states that these flaps undergo apocape, but can otherwise be realized with the addition of post-verbal morphological material.

\begin{exe}
\item\label{hollowRs} Examples of what \citet{hollow1970} proposes are root-final [ɾ]

\begin{tabular}{lll}
\textbf{\citeauthor{hollow1970}'s proposed form}&
	\textbf{Actual form}&
	\textbf{Gloss}\\
$\langle$sí$\rangle$&
	[ˈsi]&
	`to hire someone'\\
$\langle$sír$\rangle$&
	[ˈsiː]&
	`to be yellow'\\
$\langle$pé$\rangle$&
	[ˈpe]&
	`head louse'\\
$\langle$pér$\rangle$&
	[ˈpeː]&
	`to break something off'\\
\end{tabular}

\end{exe}

Subsequent fieldwork yields an explanation for why certain lexical items seemingly feature this root-final sonorant: all words that \citeauthor{hollow1970} analyzes as having an underlying root-final $\langle$r$\rangle$ actually contain long vowels, which corroborates \citeauthor{carter1991a}'s account of Mandan vowels over that of \citeauthor{hollow1970}. These long vowels, when followed by vowel-initial enclitic morphology, trigger an epenthetic [ɾ] to prevent hiatus between a long vowel and another vowel post-verbally. As I have mentioned in \sectref{glottalstop}, all phonemic glottal stops appear in coda positions in their underlying form, and the stems ending in  /ʔ/ in the data in (\ref{dialectRs}) show that these stems pattern with those ending in long vowels.

The distinction between this kind of epenthesis and the one described above in \sectref{intrusiveglottal} is the domain in which each epenthesis is active. For [ʔ]-epenthesis, the relevant domain is that of the morphological word; any prefix or derivational suffix will result in [ʔ]-insertion to prevent hiatus. For [ɾ]-epenthesis, the overall prosodic word is its designated domain. Namely, [ɾ]-insertion only occurs to prevent hiatus between a long vowel and another vowel at the boundary of a lexical item and a functional item. Similar to r-intrusion in certain English varieties \citep{gick1999}, this [ɾ]-epenthesis in Mandan occurs at the right boundary of a lexical item to prevent hiatus. However, it is not the case that [ɾ]-insertion appears merely between two words.

All three of the examples in (\ref{nointrusiveR}) below highlight the fact that [ɾ]-epenthesis is not simply triggered to prevent hiatus across a word boundary. In (\ref{nointrusiveR1}) and (\ref{nointrusiveR2}), each word has a vowel-vowel contact with an adjacent word, and yet no [ɾ]-insertion takes place. In particular, we see no [ɾ]-epenthesis in (\ref{nointrusiveR2}), even though the word \textit{Núu'etaa} `Mandan' ends in a long vowel. Similarly, even within a postpositional phrase, we do not see intrusive [ɾ] appear between the final vowel of a noun and the first vowel of a postposition, which can be seen in (\ref{nointrusiveR3}). As such, it is not enough to state that [ɾ]-epenthesis occurs at the right edge of a word boundary, but that its presence is motivated by the underlying syntagmatic structure of a phrase. We can see examples of this below, where we see examples of words that are vowel-final followed by vowel-initial words, yet no epenthetic elements are introduced to avoid a vowel abutting another vowel across a word boundary.

\begin{exe}
\item\label{nointrusiveR} Lack of [ɾ]-epenthesis with word-boundary hiatus

\begin{xlist}
\item\label{nointrusiveR1}
\glll Matéwe íresekini érereho'sha?\\
	watewe i-ra-sek=rį e-ra-rEh=o'sha\\
	\textnormal{what} pv.ins-2a-\textnormal{make}=ss pv-2a-\textnormal{want}=int.m\\
\glt	`What do you want to do?' \citep[3]{hollow1973b}


\item\label{nointrusiveR2}
\glll	Núu'etaa ı̨́'ksahe íwarooni éwereho'sh.\\
	rųų'etaa į'-ksah=E i-wa-roo=rį e-wa-rEh=o'sh\\
	\textnormal{Mandan} pv.rflx-\textnormal{ways}=sv pv.ins-1a-\textnormal{speak}=ss pv-1a-\textnormal{want}=ind.m\\
\glt	`I want to talk about the Mandan ways.' \citep[47]{hollow1973a}

\item\label{nointrusiveR3}
\glll miní íku'shtaa\\
	wrį i-ku'sh=taa\\
	\textnormal{water} dir-\textnormal{be.inside}=loc\\
\glt	`under water' \citep[10]{hollow1973b}
\end{xlist}
\end{exe}

This notion of postverbal material as enclitics is expanded upon throughout Chapter \ref{chapter3}, but we can see that the epenthetic [ɾ] is only spelled out to prevent hiatus between a long vowel and some postverbal element with a vowel on its margin where said element is prosodically dependent on the verb. More specifically, [ɾ] appears to prevent hiatus between a long vowel and another vowel at an enclitic boundary. The ordering of these enclitics reflects the underlying syntactic structure, making these enclitics simple clitics per \citeapos{anderson2005} definition, where each clitic is an element that is prosodically dependent upon an adjacent word to be realized.

In the example in (\ref{theyfixedthehouseagain1}), we see an utterance with several enclitics.\footnote{I also argue in \citet{kasak2019} that enclitics in Mandan are what \citet{anderson2005} dubs phonological clitics, and remain in-situ within the structure, so these enclitics are not simple targeting a particular location within the structure; they are already present in the structural position in which they appear and are simply relying on an adjascent word to be prosodically realized. The implication of this analysis is that enclitic boundaries in Mandan always coincide with a phrasal boundary.} Hiatus occurs due to the accretion of enclitic materials onto the stem, /=krE/ \textsc{3pl} and /=oowąk/ \textsc{narr}.\footnote{I argue in \citet{kasak2019}, from a theoretical perspective, that the enclitic associated with the number of the subject appears in T, which is plural in this case. Plural marking for objects appears in v, though the singular object in the example above causes no such marker to be realized. The narrative evidential appears after subject marking, and the allocutive agreement marker appears in C, where it is in complementary distribution with other complementizers. This syntactic structure assumes a Minimalist framework as proposed by \citet{chomsky1993,chomsky1995}. I attempt to spend as little time involving this framework into the present book so as to not detract from the descriptive goal of this grammar. The framework accounting for the phrasal relationships here is less important than the overall notion that there is a structural relation that enclitics have to the words to which they adjoin that is different from the one that affixes have to their stems. For this reason, this work continues to use a hyphen to indicate the boundary between stems and affixes and an equal sign to indicate the boundary between stems and enclitics. An enclitic boundary is synonymous here with a phrasal boundary.} As such, this [ɾ]-intrusion takes place at the enclitic boundary to repair this illicit construction. \citet{hollow1970} argues that this [ɾ] is an underlying element of the coda of specific roots, but if that were so, the flap should not be expected in the example above, because the third person plural marker is underlyingly /=kɾE/, morphologically speaking. \citeauthor{hollow1970} does not posit that this formative has an underlying root-final flap, but instead argues that the narrative evidential \textit{=oomak} has an allomorph \textit{=roomak} when added to a vowel-final stem. Rather than adhering to \citeauthor{hollow1970}'s multiple stipulations regarding the appearance of [ɾ], we can take a more parsimonious approach and conclude that all instances of [ɾ] appearing intervocalically are due to the same motivating factor: to prevent hiatus at the right edge of a stem across an enclitic boundary that involves a long vowel.

\begin{exe}

\item\label{theyfixedthehouseagain1} \glll inák óti íkisehkereroomako'sh\\
	irąk o-ti i-ki-sek=krE=oowąk=o'sh\\
	\textnormal{again} pv.loc-\textnormal{live} pv.ins-itr-\textnormal{make}=3pl=narr=ind.m\\
\glt	`they fixed the house again' \citep[157]{hollow1973a}

\end{exe}





Further evidence that this flap is truly epenthetic rather than some underlying morphological element that is realized in particular conditions comes from \citet{maximilian1839}. His notes explicitly state that there is a variety of differences between the grammar of the Ruptaare village and Like-A-Fishhook, the village where he has taken up residence and whose population consists of Nuu'etaare speakers.

\Citet{carter1991a} is the first to sift through \citeauthor{maximilian1839}'s data in an attempt to reconstruct what the modern Mandan forms might look like for both dialects. In doing so, there is a clear pattern that emerges: in instances where Nuu'etaare involves an intrusive [ɾ], this sound does not appear in Ruptaare. All instances of underlying /Vː.V/ sequences are realized with the short vowel being deleted. Furthermore, in cases where Nuu'etaare would avoid hiatus by deleting a vowel from an enclitic in the case of /V.V/ sequences, Ruptaare simply inserts a glottal stop. These differences can be seen in (\ref{mandandialects}) below.

\begin{exe}
\item\label{mandandialects} Dialect differences in \citeapos{maximilian1839} grammar via \citet[486]{carter1991a}

\begin{tabular}{llll}
\textbf{Nuu'etaare}&
	\textbf{Ruptaare}&
	\textbf{Morphology}&
	\textbf{Gloss}\\
\textit{wahé'sh}&
	\textit{wahé'o'sh}&
	/wa-hE=o'sh/&
	`I saw'\\
\textit{sı̨́įho'sh}&
	\textit{sı̨́įho'sh}&
	/sįįh=o'sh/&
	`he begged'\\
\textit{téero'sh}&
	\textit{tée'sh}&
	/tee=o'sh/&
	`he died'
\end{tabular}
\end{exe}

As shown above, the Ruptaare dialect treats the male-addressee indicative marker \textit{=o'sh} differently than the Nuu'etaare dialect. Specifically, when a stem ends in a short vowel or a consonant, both forms end in \textit{=o'sh}, with the short stem getting an epenthetic glottal stop. A stem ending in a long vowel will delete the initial /o/ in \textit{=o'sh}, creating a [VVʔ] sequence. In Nuu'etaare, when a stem ends in a short vowel, the /o/ in \textit{=o'sh} is deleted to avoid hiatus across an enclitic boundary, but remains when the stem ends in a consonant. However, when the stem ends in a long vowel, the epenthetic [ɾ] appears to prevent hiatus. It seems that Ruptaare did not permit /Vː.V/ sequences and preferred to have [VVʔ] sequences instead. For Nuu'etaare, the variety of Mandan addressed throughout this work, instead of a [VVʔ] or [VVʔV] sequence, a [VVɾV] sequence is preferred instead.

The overall takeaway here is that this [ɾ] is not just the case of an underlying coda surfacing in an intervocalic environment. It is conditioned by two factors. Firstly, a stem must undergo the addition of some postverbal element. Secondly, either the stem or the postverbal element must involve a long vowel that would otherwise cause hiatus.

The two competing systems for resolving hiatus that we see in \citet{maximilian1839} have collapsed in present-day Mandan, where modern Ruptaare speakers make exclusive use of the epenthetic [ɾ] like in Nuu'etaare, e.g., \citet[10]{littleowl1992} realizes /siː=oʔʃ/ `it is yellow' as \textit{síiro'sh} rather than the $\dagger$\textit{síi'sh} we would expect in the Ruptaare found in \citeauthor{maximilian1839}'s notes. Given this work's focus on modern Mandan and the apparent dialect coalescence that took place following the last smallpox epidemic, we can appeal to a single system for dealing with enclitic-boundary epenthesis versus the dueling systems of older Ruptaare and Nuu'etaare.

We have already seen instances of glottal stop epenthesis occurring word-internally in the environment of a long vowel, so we cannot simply say that [ʔ] epenthesis involves short vowels and [ɾ] epenthesis merely involves long vowels. The glottal stop epenthesis described in \sectref{intrusiveglottal} and the flap epenthesis described herein take place in different environments, i.e., the intrusive [ʔ] occurs to prevent hiatus within a morphological word, while the intrusive [ɾ] takes place to prevent hiatus involving a long vowel at the right margin of a phrase. Furthermore, [ɾ]-epenthesis is not just found in contexts where two underlying vowels come into contact, but when an underlying glottal stop abuts an onsetless enclitic. This behavior can be seen below, where both examples in (\ref{KennardEp}) feature stems that end in a heavy syllable, i.e., a syllable that contains a long vowel or a coda glottal stop followed by an enclitic that begins with a vowel.


\begin{exe}

\item\label{KennardEp} Instances of [ɾ]-epenthesis in \citet[17]{kennard1936}

\begin{xlist}

\item\label{KennardEp1} \textit{kixée} `to quit' $\to$ \textit{kixéero'sh} `he quit'

\item\label{KennardEp2} \textit{kiná'} `to tell' $\to$ \textit{kiná'ro'sh} `he told it'

\end{xlist}

\end{exe}

In the examples above, we have the same structure, where a verbal stem with the shape of [ X ] involves the morphological adjunction of an enclitic =Y, giving the structure [ [ X ] =Y ]. If we were to apply [ʔ] epenthesis to resolve hiatus above, we would wind up with *\textit{kixée'o'sh} for (\ref{KennardEp1}), and  we should not technically need epenthesis for (\ref{KennardEp2}), since the stem ends in a glottal stop, so any enclitic added onto such a stem should satisfy the requirement to avoid hiatus with *\textit{kiná'o'sh}. A likely reason why *\textit{kiná'o'sh} is unacceptable in that the glottal stop at the right edge of \textit{kina'} `tell' would be syllabified as an onset. As discussed in \sectref{glottalstop}, underlying glottal stops in Mandan only occur in coda positions and are never resyllabified as an onset.

A likely reason why *\textit{kiná'o'sh} is unacceptable is that syllabifying that word would result in the underlying glottal stop shifting from a coda to an onset, i.e., the [ʔ] is aligned to the left edge of the right edge of a phrase. This change would affect the syllable structure of the final syllable in /kiɾãʔ/, as the glottal stop is moraic and part of the nucleus. Given that every phonemic /ʔ/ adds to syllable weight and whose syllables pattern with long vowels, I assume that re-syllabifying what should be a coda /ʔ/ to an onset would also violate the identity of an underlying bimoraic sequence, which is being changed in the output. Parsing an underlying /ʔ/ from a coda into an onset may also violate the restriction against hiatus for similar reasons, due to the fact that /Vʔ/ patterns with /Vː/, rendering any /Vʔ=Vː/ sequences equal to /Vː=Vː/ ones. This analysis explains why stems ending in glottal stops and those ending in long vowels undergo [ɾ]-epenthesis.

In addition to glottal stops not being syllabified as onsets with the addition of enclitics, we likewise do not see [ʔ] epenthesis with /Vː=Vː/ sequences. While /ʔ/ cannot re-syllabify to an onset due to its status as part of a syllable nucleus (i.e., it is treated as equivalent to a long vowel by the phonology), there should be nothing preventing [ʔ] epenthesis from occurring at enclitic boundaries. Featurally, there is no impediment to inserting a glottal stop to prevent hiatus, as we have seen above, but there must also be some factor preventing [ʔ] as a viable option to prevent hiatus. I propose that Mandan also has a restriction against glottal stops appearing between enclitic boundaries. This restriction accounts for the fact that /ʔ/ is unable to become an onset, as well as the fact that epenthetic [ʔ] cannot occur at an enclitic boundary.

According to this restriction, having a glottal stop after a enclitic boundary edge is marked. This restriction motivates the need for another kind of epenthesis. Not permitting [ʔ] at the right edge of a phrase prevents underlying glottal stops from being assigned to an onset position, since the addition of a vowel-initial enclitic means that the glottal stop would have to cross an enclitic boundary. By keeping underlying /ʔ/ in coda positions and preventing epenthetic glottal stops from generating at the point of hiatus, [ɾ]-epenthesis is the only viable solution. The motivation for having competing epentheses could come from the fact that [ɾ] is only found word-internally, never initially or finally. The flap could serve as a cue to the listener that the speaker has not finished a word and moved on to another word.\footnote{We see similar processes crosslinguistically, as \citet{uffmann2007} argues that rhotic epenthesis is typologically found in peak positions to break hiatus, versus glottal stop epenthesis, which is found in marginal positions within a prosodic structure.} This cue might be useful for listeners, given the large amount of homophony in Mandan that has been caused by the historical merger of all Proto-Siouan obstruents to the plain series, as well as the merger of Proto-Siouan *y and *r to Mandan /ɾ/.\footnote{There is the possibility that the similitive suffix -\textit{esh}/-\textit{eshka} could be added to a /ʔ/-final stem, but there are no instances of such a construction in the corpus, and there are no longer any L1 speakers to give judgments about whether a /Vʔ-eʃka/ sequence would yield [V.ʔe.ʃka], [Vʔ.e.ʃka], [Vʔ.ɾe.ʃka], or something else. I conjecture that speakers might use [ɾ]-epenthesis as a last resort, but there is no conclusive evidence for this in the corpus.}

One caveat to the above claim is that Mandan has a single word that seems to optionally take the glide [w] instead of [ɾ]: `mother', which is shown in (\ref{mandanmother}) below. The variant with [ɾ] is the only lexical item found in the corpus that behaves in this manner, and modern speakers will only use this form in spontaneous speech. Likewise, \citet{maximilian1839} only gives \textit{kohų́ųre} for `his mother', so it is unclear if employing [w] instead of [ɾ] represents a holdover from a non-Nuu'etaare variety of Mandan, if it is a fossilized lexical alternative to \textit{kohų́ųre}, or if it is an innovation in Mandan where epenthetic elements across enclitic boundaries are assimilating features of the preceding vowel. The morpho-phonological breakdown of this doublet with both the [ɾ] and [w] forms is shown below.

\begin{exe}

\item\label{mandanmother} Variable realization of enclitic boundary epenthesis

\glll kohų́ųre~$\sim$~kohų́ųwe\\
	ko-hųų=E\\
	3poss.pers-\textnormal{mother}=sv\\
\glt	`his mother' \citep[83]{hollow1970}
\end{exe}

Other Siouan languages, such as Lakota, employ epenthetic glides intervocalically to avoid hiatus. Thus, it is possible that what began as an excrescent process of glide insertion evolved to where that glide depended on the previous vowel to determine its quality, i.e., back vowels are followed by [w] and front vowels followed by [j]. All other instances of hiatus are resolved by a glottal stop (\citeauthor{mirzayan2010} p.c.).

\largerpage
\begin{exe}
\item Hiatus resolution in Lakota
\begin{xlist}
\item \glll iyápȟa~\textnormal{[i.ˈja.pˣa]}\\
	 i-a-pȟa\\
	 pv.ins-pv.supe-\textnormal{strike}\\
\glt	`to strike someone unintentionally' \citep[249]{ullrich2011}
\item \glll owóškate~\textnormal{[o.ˈwo.ʃka.te]}\\
	 o-o-škatA\\
	 pv.loc-pv.ill-\textnormal{play}\\
\glt	`playground, park, recreation area' \citep[442]{ullrich2011}
\item \glll waáiye~\textnormal{[wa.ˈʔa.ʔi.je]}\\
	wa-a-iyA\\
	unsp-pv.supe-\textnormal{speak}\\
\glt	`to gossip about people' \citep[579]{ullrich2011}

\end{xlist}
\end{exe}

However, this system appears to be in flux in many varieties of Lakota and Dakota. There are words that seem to categorically resist glide insertion like in (\ref{lakotaEp1}) below. To whit, there are also words that allow for either the expected glide or glottal stop to avoid hiatus, such as (\ref{lakotaEp2}). Quizzically, there are also words that permit any epenthetic segment and are considered valid for all speakers as in (\ref{lakotaEp3}) below. These pronunciations come from \citeauthor{mirzayan2010} (p.c.).

\begin{exe}
\item\label{lakotaEp} Unexpected epenthesis in Lakota
\begin{xlist}

\item\label{lakotaEp1} \glll iúŋšila~\textnormal{[i.ˈʔũ.ʃi.la, *i.ˈjũ.ʃi.la]}\\
	i-uŋšila\\
	pv.ins-\textnormal{take.pity.on}\\
\glt	`to take pity on someone by means of something' \citep[241]{ullrich2011}

\item\label{lakotaEp2} \glll oíhaŋke~\textnormal{[o.ˈwi.hã.ke $\sim$ o.ˈʔi.hã.ke]}\\
	o-i-haŋke\\
	pv.loc-pv.dir-\textnormal{portion}\\
\glt	`to come to an end' \citep[413]{ullrich2011}

\item\label{lakotaEp3} \glll huókaȟmi~$\sim$~huyókaȟmi~\textnormal{[hu.ˈʔo.ka.xᵊmi/hu.ˈwo.ka.xᵊmi $\sim$ hu.ˈjo.ka.xᵊmi]}\\
	hu-o-ka-ȟmi\\
	\textnormal{bone}-pv.loc-ins.frce-\textnormal{be.crooked}\\
\glt	`popliteal fossa' \citep[177]{ullrich2011}

\end{xlist}
\end{exe}

The variability in the epenthetic consonant utilized in Lakota appears to be more extreme than the one-word variation in epenthesis we see in Mandan with /hũː/ `mother'. The system of hiatus resolution in Lakota is more strongly informed by the quality of the vowel preceding the hiatus, and as such, more closely aligned with its phonetic characteristics. Mandan, on the other hand, seems to have a system of hiatus resolution that is independent of the features of its surrounding vowels (i.e., phonetically unmotivated) and reliant on a system of a single, specific consonant being inserted to avoid hiatus at enclitic boundaries (i.e., phonologically motivated). The existence of \textit{kohų́ųwe} raises the question of whether this doublet is evidence of an older system more similar to that in Lakota, where hiatus is resolved by inserting a glide that shares features with non-low vowels, or whether it is part of an innovation by way of an incomplete phonological change where different kinds of epenthetic segments were possible, but the collapse in the population of Mandan speakers and the leveling of dialects reversed this change. This work argues in favor of the latter scenario. This particular topic is addressed further in \sectref{stemvowel}, and does not detract from the overall point of the argumentation above: modern Mandan has a productive and predictable process of inserting [ɾ] to break hiatus involving long vowels at boundaries involving enclitics.\footnote{Another possibility is that \textit{kohų́ųwe} has a different morphology altogether, and the element \textit{-we} is actually historically related to the indefinite \textit{-we} that appears on pronominals and quantifiers, e.g., \textit{kotewé} /ko-t-we/ \textsc{rel-wh-indf} `who.'}

Given what \citet{maximilian1839} writes about the Ruptaare variety of Mandan, which only has [ʔ] epenthesis and no [ɾ] epenthesis, the most likely scenario is that Ruptaare completely lost any kind of sonorant epenthesis that was part of a Pre-Mandan language by the time the Mandan had settled into two villages in the early 1800s, while Nuu'etaare had a robustly established epenthetic [ɾ] by that same time period. Aside from `mother', there could have been more doublets that survived into modern Mandan, but current speakers have been unable to recall any others.

The data presented thus far regarding the status of epenthetic [ɾ] refutes \citet{hollow1970} and \citeapos{carter1983} hypothesis that there are stems in Mandan that end in the coronal flap. This [ɾ] is a predictable epenthetic element that occurs due to the hiatus caused by certain enclitics, e.g., the narrative evidential \textit{-oomak}, which \citet{hollow1970} analyzes as having an allomorph \textit{-roomak}. Instead of attributing the numerous instances of stems and postverbal elements that appear with [ɾ] in some environments and lack it in others to phonology and allomorphy, we can accurately pin this elusive flap solely on phonology.

We have two distinct epenthetic processes in Mandan that act on two different domains: [ʔ]-insertion prevents hiatus within the domain of a morphological word, while [ɾ]-insertion prevents hiatus between the domain of a morphological word and an enclitic. We can formalize these observations in the phonological rule depicted below:

\begin{exe}

\item\label{REpenthesisRule} \textbf{[ɾ]-Epenthesis Rule}\\
	$\varnothing$ $\to$ [ɾ] / V(ː/ʔ)={\longrule}V(ː/ʔ)\\
	\textit{Insert a} [ɾ] \textit{between two vowels when an enclitic is added to a stem that either ends in a heavy open syllable (i.e., one that is long with no coda or one that is short with a coda [ʔ]) abutting an onsetless syllable or a stem ending in a light open syllable abutting an onsetless heavy syllable.}

\end{exe}


This [ɾ]-epenthesis is triggered by the presence of long vowels; a different process resolves hiatus between two short vowels between the domain of a morphological word and an enclitic, which is described below.

\subsubsection{Short vowel elision}\label{shortvoweldeletion}

While the preferred method of resolving hiatus in Mandan is to produce an epenthetic consonant, there are instances where hiatus is resolved by vowel deletion, as exemplified below, where each of the examples in (\ref{deletionexamples1}) features clause-final morphology that begins with an onsetless formative. Furthermore, the short vowel in these formatives is elided when following a stem with an open syllable that also contains a short vowel. This environment differs from the one previously described in \sectref{intrusiveR} in that epenthetic [ɾ] occurs to prevent hiatus at an enclitic boundary involving a long vowel or a stem ending in a glottal stop, whereas the deletions above all involve hiatus featuring only short vowels both stem-finally and enclitic-initially. We can see examples of vowels being deleted at enclitic boundaries in the data below.

\begin{exe}
\item\label{deletionexamples1} Examples of hiatus resolution via vowel deletion

\begin{xlist}
\item\label{deletionexamples1a}
\glll ą́'skere tú'sh\\
	ą's=krE tu=o'sh\\
	\textnormal{horn}=3pl \textnormal{be.some}=ind.m\\
\glt	`there were some horns' \citep[20]{trechter2012b}

\item\label{deletionexamples1b}
\glll Rémak watéwe'na?\\
	re=wąk watewe=o'rą\\
	dem.prox=psnl.lie \textnormal{what}=int.f\\
\glt	`What is this?' \citep[7]{kasak2014}

\item\label{deletionexamples1c}
\glll	kixéekerek\\
	ki-xee=krE=ak\\
	mid-\textnormal{be.quiet}=3pl=ds\\
\glt	`them having quit' \citep[430]{hollow1970}

\end{xlist}

\end{exe}

If we apply the processes utilized thus far to resolve hiatus between two short vowels, we can see that they cannot account for short vowel elision. In the Ruptaare variety of Mandan during \citeapos{maximilian1839} time, this type of hiatus was resolved by inserting a glottal stop, e.g., \textit{shí'osh} /shi=o'sh/ `it is good' in Ruptaare versus \textit{shí'sh} in the Nuu'etaare of that time and in modern Mandan. As such, nineteenth century Ruptaare would select the glottal stop as the sole epenthetic segment.


This historical evidence, along with the data presented above in (\ref{deletionexamples1}), demonstrates that modern Mandan morpho-phonology prioritizes retaining long vowels over deleting them more than it does for short vowels. This pattern of hiatus resolution protects long vowels over short vowels. As I have argued throughout this chapter, underlying glottal stops are moraic and pattern with long vowels. Therefore, the phonological rules of Mandan require that any contiguous sequence of moraic elements (i.e., /Vː/ and /Vʔ/) present in the input must be reflected in the output. All short vowels that are not followed by a coda glottal stop are monomoraic and are candidates for elision to avoid hiatus.

Bearing in mind that long and short vowels are treated differently with respect to hiatus resolution in Mandan, we must conclude that preserving the status of a heavy syllable has higher priority than preserving the status of a light syllable. These priorities will result in elision for short vowel-initial enclitics when following a stem ending in a short vowel. When adjoining to stems ending in long vowels or a short vowel and a [ʔ] coda, such environments will always trigger [ɾ] epenthesis between the stem and the enclitic.

Elision of short vowels that have a coda glottal stop is likewise prohibited for two reasons. Firstly, coda glottal stops are moraic and Mandan prioritizes maintaining the integrity of heavy syllables over light syllables. Secondly, deleting a short vowel that shares a syllable with a coda [ʔ] can result in a heavy syllable abutting a moraic element, producing a trimoraic sequence.

For example, the word \textit{síi} `be yellow' when combined with the male-addressee indicative marker \textit{=o'sh} results in \textit{síiro'sh} `it is yellow', with [ɾ]-epenthesis preventing hiatus between the final long vowel of the stem and the initial short vowel of the enclitic. We cannot simply delete the initial /o/ in \textit{=o'sh} to produce \textit{*síi'sh}. A word like \textit{*síi'sh} is wholly ungrammatical in Mandan. Coda glottal stops are only permitted after short vowels. This restriction is due to the fact that coda glottal stops add to syllable weight, which is explained further in \sectref{stress}. As such, a [Vʔ] syllable is already heavy, so a long vowel plus a [ʔ] results in a superheavy syllable. Mandan clearly does not permit trimoraic syllables, so a superheavy syllable must be avoided, thus triggering [ɾ]-epenthesis.

It is clear that Mandan robustly elides short vowels in short vowel-initial enclitics that follow short vowel-final stems, as the example below shows.

\begin{exe}

\item\label{shortvowelsgone} Elision of multiple short vowels to avoid hiatus

 \glll Tewét tú'xere'sha?\\
	t-we=t tu=o'xrE=o'sha\\
	wh-indf=loc \textnormal{be.some}=dub=int.m\\
\glt	`Where would there possibly be any?' \citep[34]{mixco1997a}

\end{exe}

The root \textit{tú} `to be some' ends in a short vowel, and the epistemic modal \textit{=o'xere} begins in a short vowel. The short vowel in \textit{=o'xere} is deleted, yielding a stem of \textit{tú'xere}. This new stem likewise ends in a short vowel, so the male-directed allocutive interrogative enclitic \textit{=o'sha} must in turn lose its initial /o/, yielding the proper output: \textit{tú'xere'sha}. This process can be visualized below. Stem 1 represents the root \textit{tú}, while Stem 2 is the result of concatenating \textit{tú} plus \textit{=o'xere}, onto which \textit{=o'sha} must cliticize. The combination of Stem 1 with the enclitic \textit{=o'xere} results in the deletion of the second short vowel in a /V\textsubscript{1}V\textsubscript{2}/ sequence, yielding a new stem, \textit{tu'xere}. Likewise, this Stem 2 \textit{tu'xere} concatenates with the enclitic \textit{=o'sha}, which must undergo another deletion of the second short vowel in a /V\textsubscript{1}V\textsubscript{2}/ sequence. The end result of this process is the surface form \textit{tú'xere'sha}.

\begin{exe}

\item\label{structureofTuPxerePsha} Structural composition of \textit{tú'xere'sha}

[ [ [ \textit{tu} ]\textsubscript{Stem 1} \textit{=o'xere} ]\textsubscript{Stem 2} \textit{=o'sha} ]

\end{exe}

As each postverbal element cliticizes onto its stem, the enclitic maximizes the segments already present in the stem and elides its initial vowel to avoid hiatus. As such, there is a conspiracy to maximize the segments in the stem, not necessarily a specific lexical item, e.g., a root or formative. Once an enclitic is properly realized with its accompanying stem, it forms a new stem, and any ensuing enclitics must be realized with that new stem in mind. This rule for short vowel deletion as a strategy for hiatus avoidance is formalized in (\ref{ShortVowelDeletionRule}) below:

\begin{exe}

\item\label{ShortVowelDeletionRule} \textbf{Short Vowel Deletion Rule}\\
	/V\textsubscript{2}/ $\to$ $\varnothing$ / V\textsubscript{1}={\longrule}\\
	\textit{Delete an initial short vowel in an enclitic where that enclitic abuts a stem ending in a short vowel.}

\end{exe}

This short vowel deletion rule specifically targets environments where two short vowels come into contact across an enclitic boundary. Word-internal hiatus is resolved through [ʔ]-epenthesis, as described in \sectref{intrusiveglottal} above.

\subsubsection{Summary of hiatus resolution processes}\label{SubsubsecSummaryofHiatusResolutionProcesses}

The productivity of these processes is evident in their ability to handle the full range of situations in which hiatus occurs: word-internally and at enclitic boundaries. The three different tactics for hiatus resolution are predictable and dependent on the precise cause of the hiatus. Word-internal morphological concatenation involves [ʔ] epenthesis, as it is the least-marked segment according to \citet{princesmolensky1993} and \citeapos{lombardi2001} markedness hierarchy. Hiatus caused by enclitics becoming prosodically linked to stems ending in vowels or glottal stops have two different strategies: short vowels outside of the stem are elided, while an epenthetic [ɾ] breaks up hiatus between long vowels or stems ending in glottal stops and the following enclitics to avoid creating constructions where a [ʔ] appears at the edge of an enclitic boundary, rendering epenthetic [ʔ] at enclitic boundaries impossible, disqualifying any other possible wordforms that delete material from a stem in favor of preserving material by generating the alternative epenthetic [ɾ] before an enclitic.



\subsection{Metathetical processes}\label{metathesis}

In addition to having two different forms of epenthesis, Mandan also features two different kinds of metathesis. Both processes are repair mechanisms to avoid illicit clusters, and both of them are predictable synchronic processes. One metathetical process prevents glottal stops from being the final element of a consonant cluster (see \ref{glottalstopmetathesis}), while the other process prevents surface realizations of [kp] sequences (see \ref{velarlabialmetathesis}). Both processes are noteworthy in that they are restricted to preventing illicit clusters within a single morphological word. Metathesis does not occur across a word boundary because metathesis is sensitive to word boundaries.

\subsubsection{Glottal stop metathesis}\label{glottalstopmetathesis}

In \sectref{glottalstop}, I have explained that glottal stops do not occur word-initially or root-initially. This restriction against word-initial glottal stops is rooted in a historical change where Proto-Siouan words that began with *ʔ metathesized with a vowel, moving the glottal stop from the onset to the coda.



\begin{exe}

\item\label{glottalchange} Historical glottal stop metathesis

\begin{tabular}{ll}
\textbf{Proto-Siouan}&
	\textbf{Mandan}\\
*ʔa + *-t(a)&	\textit{ą́'t}$\sim$\textit{á't}\\
	~~~\textsc{dem} + \textsc{loc}&~~~`that one (far away)'\\
*wa-ʔįį-he&	\textit{mí'he}\\
	~~~\textsc{inan.clf}-wear.around.shoulders-\textsc{nom}&
	~~~`shawl, blanket'\\
*ki-ʔųų-te&\textit{kų́'te}\\
~~~\textsc{dat}-throw-\textsc{aug}&~~~`to throw something'\\
*ʔoo&	\textit{ó'}\\
	~~~`be.\textsc{pl}'&~~~`to be'\\
*xʔehe&	\textit{xé'he}\\
	~~~`drip'& 	~~~`rain'\\
\end{tabular}

\end{exe}

Mandan not only does not permit glottal stops in word-initial positions, it does not permit /ʔ/ to be part of an onset cluster.  In each of the Proto-Siouan forms above, we can see a word- or root-initial glottal stop manifest as a coda glottal stop in Mandan. The distal demonstrative *ʔa combines with the locative *-ta to become \textit{ą́'t} in modern Mandan. Similarly, the *ʔ in PSi *wa-ʔįį-he metathesizes with the vowel in its syllable nucleus. The plural copula *ʔoo transparently metathesizes with the vowel, producing \textit{ó'} `to be' in Mandan. Metathesizing the /ʔ/ to a coda position in a syllable containing a long vowel also truncates the long vowel to avoid creating trimoraic syllables.

Mandan productively inserts glottal stops to prevent hiatus within a stem, as described in \sectref{intrusiveglottal}. This is a common source of surface glottal stops. Another major source of surface glottals involves the allophones of certain prefixes. These prefixes, such as the first person stative marker \textit{ma}- /wą-/ or the second person possessive \textit{ni}- /rį-/, are alternatively realized as underlying /Cʔ-/ clusters instead of /CV-/ sequences, i.e., /wą-/ has the allomorph /w'-/ and /rį-/ has the allomorph /r'-/ before vowel-initial stems. These /Cʔ-/ clusters are not permitted in surface representations in Mandan due to the constraint against consonant clusters having a [ʔ] as their second element, so metathesis takes place as a repair mechanism. We can see examples of this process in contemporary Mandan in the examples below.



\begin{exe}

\item\label{glottalstopmetathesis1} Synchronic [ʔ] metathesis

\begin{xlist}
\item\label{glottalstopmetathesis1a}
\glll wá'ts\\
	w'-at=s\\
	1poss=\textnormal{father}=def\\
\glt	`my father'

\item\label{glottalstopmetathesis1b}
\glll rá're\\
	r'-aa=E\\
	2poss-\textnormal{arm}=sv\\
\glt	`your arm'

\item\label{glottalstopmetathesis1c}
\glll	wá'kana'ko'sh\\
	w'-aaki-rą'k=o'sh\\
	1s-\textnormal{be.above}-psnl.sit=ind.m\\
\glt	`I ride [on horseback]'

\item\label{glottalstopmetathesis1d}
\glll	kų́'he\\
	k'-ųh=E\\
	3poss.pers-\textnormal{wife}=sv\\
\glt	`his wife'

\item\label{glottalstopmetathesis1e}
\glll	ko'áakis\\
	ko-aaki=s\\
	rel-\textnormal{be.above}=def\\
\glt	`the one on top'


\end{xlist}

\end{exe}

In (\ref{glottalstopmetathesis1a}) through (\ref{glottalstopmetathesis1d}), the prefixes involved have allomorphs that consist of a consonant and a glottal stop, e.g., the third person personal possessive \textit{ko}- /ko-/ has the allomorph /k'-/ when it appears before a nasal vowel. The phonetically identical relativizer \textit{ko-} does not have any allomorphs, and as such does not trigger glottal stop metathesis. A similar argument can be made that this allomorphy occurs even when the prefix is added to a stem beginning with a long vowel, such as in (\ref{glottalstopmetathesis1c}), where /wą-/ becomes /w'-/ before the stem \textit{áakana'ko'sh}, yielding \textit{wá'kana'ko'sh} `I ride on horseback.' The long vowel contracts when followed by a coda glottal stop to avoid creating a trimoraic syllable. This process of long vowel truncation is likewise seen in (\ref{glottalchange}) above, such as in \textit{mí'he} `shawl', where the long vowel in *ʔįį `to wear about the shoulders' becomes short when the glottal stop metathesizes to a coda position. While the data in (\ref{glottalchange}) demonstrate that glottal metathesis is a diachronic sound change that occurred at a stage in development prior to modern Mandan, (\ref{glottalstopmetathesis1}) shows that glottal stop metathesis is still a productive part of modern Mandan grammar.

As argued above, we can classify these differences in surface forms involving certain prefixes as allomorphic, and not some phonological process whereby the vowel in the prefix is syncopated. While syncope may have been part of a regular phonological rule at some point in the diachrony, this syncope became less regular over time and certain formatives came to be reanalyzed as having different forms when added to vowel-initial stems.


A slightly more complicated situation arises when these underlying /Cʔ-/ allomorphs prefix onto stems that begin with long vowels. The noun \textit{áare} `arm' is one such stem, where the first person possessive marker \textit{mi-} /wį-/ appears as its allomorph /w'-/ due to its prefixing onto a vowel-initial environment. Metathesis must occur to avoid violating the phonotactic constraint against having a glottal stop as the second element of a consonant cluster, but the metathesis sets up the condition for the syllable to contain a long vowel and a glottal stop coda. Such sequences are illicit in Mandan, due to the fact that [Vːʔ] syllables would be trimoraic and would thus violate the phonotactic constraint against having a long vowel in a syllable containing a moraic glottal stop in the coda.

The actual output of \textit{áare} with first person possession marking is \textit{wá're} `my arm.' Any possible word form with a surface [Cʔ] is not viable, so the [ʔ] must metathesize with the long vowel. When /ʔ/ enters the coda, the long vowel truncates to avoid violating the restriction against trimoraic syllables. Normally, Mandan prioritizes the preservation of heavy syllables, as we have seen in how hiatus involving heavy syllables is treated in this chapter, but it appears that a superheavy syllable is ungrammatical and must be repaired by eliminating one of the morae.

The primary driver of glottal stop metathesis is a restriction against surface [Cʔ] clusters. It is thus preferable to metathesize the /ʔ/ with its following vowel than it is to preserve the underlying linearity. All glottal stop metathesis is word-internal, and does not happen across word boundaries. No synchronic Mandan words begin with word-initial glottal stops due to the diachronic metathesis described in (\ref{glottalchange}), so we are unable to test whether or not this metathetical process would take place across word boundaries, such as with compounds. Given the behavior of the other metathetical process in Mandan, it is likely that a word boundary would be a blocking mechanism for metathesis, as this is precisely the same condition that blocks velar-biliabial metathesis.


\subsubsection{Velar-bilabial stop metathesis}\label{velarlabialmetathesis}

The other instance of metathesis found in Mandan is likewise conditioned by the conspiracy in Mandan phonotactics to avoid a marked cluster. Instances of tautosyllabic [kp] clusters on the surface are illicit, and when /k/ and /p/ come into contact through some morphological operations, the [p] must precede the [k] on the surface, as demonstrated by (\ref{kpmetaex1a}) through (\ref{kpmetaex1c}) below. This restriction against velar-labial clusters is limited to velar and bilabial stops, not all segments that are velar or bilabial in general. Velar fricatives are permissible before bilabial stops, which (\ref{kpmetaex1d}) shows. As such, this metathesis is specific to the combination of /k/ and /p/, rather than all velars followed by /p/.\footnote{Clusters involving the velar stop plus the labiovelar glide, /kw/, are also possible, but a Dorsey's Law vowel will inevitably cause an excrescent vowel to interrupt the /k/ and the /w/: /i-kwa=taa/ $to$ [ˈi.kᵃwa.taː] `against' \citep[125]{hollow1970}. See \sectref{dorseyslaw} for further discussion of Dorsey's Law vowels and the behavior of sonorants in clusters.}

\begin{exe}
\item\label{kpmetaex1} Examples of /kp/ metathesis
	\begin{xlist}
	\item\label{kpmetaex1a}
	\glll ų́pka\\
	ųk-pa\\
	\textnormal{hand}-\textnormal{head}\\
	\glt `thumb' \citep[35]{hollow1970}
	\item\label{kpmetaex1b}
	\glll pka'úux\\
	k-pa-uux\\
	suus-ins.push-\textnormal{be.broken}\\
	\glt `to break something of one's own' \citep[263]{hollow1970}
	\item\label{kpmetaex1c}
	\glll órapkakishinite\\
	o-ra-k-pa-kish=rįt=E\\
	pv.irr-2a-mid-ins.push-\textnormal{wipe}=2pl=sv\\
	\glt `that you (pl.) wipe them out'	 \citep[210]{trechter2012b}
	\item\label{kpmetaex1d}
	\glll maxpé\\
	wąxpe\\
	\textnormal{nine}\\
	\glt `nine' \citep[21]{hollow1976}
	\end{xlist}
\end{exe}

The ban on [kp] clusters can be motivated by a phonotactic restriction in Mandan, where such clusters are highly marked. For that, the following rule applies:

\begin{exe}

\item\label{KPmetathesizeRule} \textbf{/kp/ Metathesis Rule}\\
	/kp/ $\to$ [pk]\\
	\textit{Whenever an underlying} /kp/ \textit{cluster occurs, the} /k/ \textit{becomes the second element in the cluster, resulting in a surfce} [pk] \textit{cluster.}

\end{exe}

One exception to this generalization is that [kp] is possible in compounds when each segment is at a word boundary. We can see these exceptions for [kp] metathesis for compound nouns in (\ref{kpsurface}) below.



\begin{exe}
\item\label{kpsurface} Surface [kp] in compounds

\begin{xlist}

\item\label{kpsurface1} \glll manúuxikpa\\
warųųxik\#pa\\
\textnormal{ghost}\#\textnormal{head}\\
\glt `skull' \citep[7]{hollow1973b}

\item\label{kpsurface2} \glll Weróokpa\\
wrook\#pa\\
\textnormal{buffalo.bull}\#\textnormal{head}\\
\glt `Buffalo Bull Head' \citep[xviii]{densmore1923}

\end{xlist}

\end{exe}

\largerpage
The fact that [kp] is permissible in (\ref{kpsurface1}) but not in (\ref{kpmetaex1}) indicates that the phonotactic restriction against [kp] pertains to [kp] clusters within a single stem. For example, \textit{pka'úx} `to break one's own' cannot be *\textit{kpa'úux} because its morphological structure is simplex, i.e., [k-pa-uux]. The word \textit{manúuxikpa} `skull' allows for [kp] due to the fact that it has a compound structure, which is shown in (\ref{ExSkullStructure}) below.

\begin{exe}
\item\label{ExSkullStructure} Morphological structure of \textit{manúuxikpa} `skull'\\
{[}[warųųxik]\textsubscript{word}[pa]\textsubscript{word}]\textsubscript{word}
\end{exe}

Given examples like the ones in (\ref{kpsurface}) and (\ref{ExSkullStructure}) above, we cannot say that this constraint in Mandan rules out [kp] clusters across the board, but rather that Mandan grammar specifically considers stem-internal [kp] clusters to be illicit.

\subsubsection{Summary of metathetical processes}\label{metathesissummary}

Throughout \sectref{metathesis}, I have demonstrated that there is not just one but two different metathetical processes in Mandan. Both these processes serve to avoid illicit surface clusters: sequences of underlying /pk/ that become [kp] and clusters whose last element is the glottal stop, i.e., [Cʔ] sequences, where C is any consonant. Both of these metathetical processes take place as operations of last resort. Instead of appealing to yet another kind of epenthesis or simplifying these clusters by deleting one segment, Mandan transposes one of the consonants in these clusters. For clusters ending in [ʔ], the [ʔ] metathesizes with the following vowel. Clusters involving /kp/ involve the order of these consonants reversing to [pk]. The only exception for this is when /kp/ clusters occur in compounds.


\subsection{Nasal harmony}\label{nasalharmony}

The nasal sonorants [m] and [n] are some of the most common surface segments in Mandan. In \citet{kennard1936}, these two nasal consonants are listed as being phonemic, i.e., /m/ and /n/. However, \citet{hollow1970} demonstrates that all surface nasal consonants arise due to contact with an underlying nasal vowel. This lack of phonemic nasal consonants in Mandan is typologically rare, with perhaps fewer than two percent of languages sharing this gap in their phonemic inventories \citep{sampson1999,maddieson2013}.

As previously discussed, \citet{hollow1970} and \citet{mixco1997a} state that [m] and [n] are allophones of /w/ and /ɾ/, respectively, and are only realized as nasals when they occur before a nasal vowel through regressive nasal assimilation, as seen below with the nasal spread underlined.

\begin{exe}
\item\label{firstnasals} Regressive nasal assimilation
	\begin{xlist}
	\item\label{firstnasals1}
	\glll \textnormal{[ˈ\uline{nãː.m\textsuperscript{ĩ}nĩ}]}\\
	/ɾaːwɾĩ/\\
	\textnormal{three}\\
	\glt `three'

	\item\label{firstnasals2}
	\glll \textnormal{[\uline{mã.ˈnãː}.teʔʃ]}\\
	/wa-ɾããtE=oʔʃ/\\
	1a-\textnormal{stand.up}=ind.m\\
	\glt  `I stand up' \citep[173]{hollow1970}

	\item\label{firstnasals3}
	\glll \textnormal{[ˈ\uline{mãː.mã.nã.nũː.nĩ}.x\uline{\textsuperscript{ĩ}nĩ}.stoʔʃ]}\\
	 /waː-wa-ɾa-ɾũː=ɾĩx=ɾĩt=t=oʔʃ/\\
	 neg-unsp-2a-\textnormal{abduct}=neg=2pl=pot=ind.m\\
	 \glt `thou shalt not commit adultery' \citep[22]{hollow1970}
	\end{xlist}
\end{exe}

In (\ref{firstnasals1}), the nasality from the underlying /ĩ/ in \textit{náamini} `three' spreads leftward, adding [$+$nasal] to voiced segments. Since the only underlyingly voiced consonants in Mandan are sonorants, /w/ and /ɾ/ are able to take on this nasal feature. Nasality is able to spread from a stem to a prefix, which (\ref{firstnasals2}) demonstrates. In (\ref{firstnasals2}), not only does nasality spread leftward from /ãː/ onto the /ɾ/ to make it [n], but this harmony continues past the boundary of the stem and onto the /a/ in the first person active prefix /wa/, which then causes the /w/ to nasalize to [m]. The last example in (\ref{firstnasals3}) illustrates that this nasal harmony can cause distant segments that are not in contact with a syllable bearing an underlying nasal vowel to pick up nasal features.

\citet[21]{hollow1970} describes regressive nasal assimilation in Mandan as being optional across morpheme boundaries, but obligatory within a morpheme, which he codifies in (\ref{RNArule}) below:

\begin{exe}
\item\label{RNArule} \citeauthor{hollow1970}'s (\citeyear[21]{hollow1970}) Regressive Nasal Assimilation Rule\\

$\begin{Bmatrix}
\textrm{Resonant consonants}\\
\textrm{Apex vowels}
\end{Bmatrix}$ $\to$ [+nasal] / {\longrule}[+nasal]
\\
\end{exe}

The apex vowels consist of non-mid vowels, i.e., /a i u/. \citet{hollow1970} classifies /w/ and /ɾ/ as resonants, and also places /h/ in that category. \citet{hollow1970} does not elaborate on why he classifies /h/ as a resonant with the two sonorant consonants. One possibility is that /h/ can optionally become voiced intervocalically, and nasality only spreads along voiced segments, e.g., /paahį'/ `porcupine' can be realized as either [ˈpaː.hĩʔ] or [ˈpãː.ɦ̃ĩʔ].\footnote{The fully nasalized version of `porcupine' is much more common in the corpus, i.e., [ˈpãː.ɦ̃ĩʔ].}

The rule given in (\ref{RNArule}) stipulates that leftward nasal harmony spreads along voiced segments that are not mid vowels. As such, any voiceless segment or mid vowel will act as a blocking mechanism for nasal harmony. This behavior is demonstrated below, again with the nasal spread highlighted with an underline.


\begin{exe}
\item\label{secondnasals} Blocking environments for regressive nasal harmony

	\begin{xlist}
	\item\label{secondnasals1}
	\glll \textnormal{[\uline{nũ.ˈmã}ʔk]}\\
	/ɾuwãʔk/\\
	\textnormal{man}\\
	\glt `man'

	\item\label{secondnasals2}
	\glll \textnormal{[i.ˈst\uline{ã.mĩ}ʔ]}\\
	 /istawĩʔ/\\
	\textnormal{eye}\\
	\glt `eye'

	\item\label{secondnasals3}
	\glll \textnormal{[ˈ\uline{ĩː}.ta.h\uline{\textsuperscript{ĩ}nũ}]}\\
	/ĩːtahɾũ/\\
	\textnormal{neck}\\
	\glt `neck'

	\item\label{secondnasals4}
	\glll \textnormal{[\uline{nã.ˈnĩ}.hoʔʃ]}\\
	 /ɾa-ɾĩh=oʔʃ/\\
	 2a-\textnormal{breathe}=ind.m\\
	 \glt `you breathe' \citep[181]{hollow1970}

	\item\label{secondnasals5}
	\glll \textnormal{[ˈ\uline{mĩ}ʔ.ti.k\textsuperscript{e}ɾe.seː.\uline{nã}]}\\
	/wĩʔti=kɾE=s=eː=ɾã/\\
	\textnormal{village}=3pl=def=dem.dist=top\\
	\glt `the villagers [there]' \citep[313]{hollow1973b}

	\item\label{secondnasals6}
	\glll \textnormal{[ˈoː.xa.ɾeː.\uline{nã}]}\\
	/oːxa=eː=ɾã/\\
	\textnormal{fox}=dem.dist=top\\
	\glt `the fox [there]' \citep[63]{hollow1973b}


	\item\label{secondnasals7}
	\glll \textnormal{[ˈ\uline{mãː.m\textsuperscript{ĩ}nĩ}.haː.xiʔ.ɾe]}\\
	/waː-w-ɾĩ-hE=xi=oʔɾe/\\
	neg-1a-2s-\textnormal{see}=neg=ind.f\\
	\glt `I did not see you' \citep[72]{hollow1973a}

	\item\label{secondnasals8}
	\glll \textnormal{[i.ˈʃka} \textnormal{\uline{ˈnãː.m\textsuperscript{ĩ}nĩ}.k\textsuperscript{e}ɾe]}\\
	/iʃka ɾaːwɾĩ=kɾE/\\
	\textnormal{only} \textnormal{three}=3pl\\
	\glt `only three of them' \citep[168]{trechter2012b}

	\end{xlist}
\end{exe}

In (\ref{secondnasals1}), (\ref{secondnasals4}), and (\ref{secondnasals7}), we see nasal harmony spread leftward from an underlying nasal vowel to the left edge of the word no differently than we saw in (\ref{firstnasals}). However, we see in (\ref{secondnasals2}) and (\ref{secondnasals3}) that a voiceless segment is preventing the spread of the [+nasal] feature. The mid vowels in (\ref{secondnasals5}) and (\ref{secondnasals6}) likewise prevent the leftward spread of nasality. Word boundaries also prevent the spread of nasal harmony, even if there are segments that are viable targets for this process, as we see in (\ref{secondnasals8}).

The data above in (\ref{secondnasals}) show that nasal harmony is a process that occurs not just within a single root or formative. This process is not strictly a local process. Rather, nasality can spread leftward along possible segments until reaching some featurally motivated blocking environment. The data above demonstrate that there are three blocking environments:

\begin{exe}

\item\label{ThreeBlockingEnvironments} Blocking environments for nasal harmony

	\begin{xlist}

	\item Mid vowels: Mandan does not permit nasal mid vowels, so nasal harmony cannot manifest on /e eː o oː/.

	\item Voiceless consonants: Nasal harmony can only manifest on voiced segments, and the only voiced segments in Mandan are vowels and /w/ and /ɾ/, so anything that is [$-$voice] blocks the spread of nasal harmony.

	\item Word boundaries: Nasal harmony is a word-level process and it cannot spread from one word to another.

	\end{xlist}

\end{exe}

These constraints have so far been able to accurately predict the actual output with respect to nasal harmony, even when several of the examples have shown that nasal harmony is able to spread across long distances, provided that the right featural environment exists, i.e., non-nasal voiced segments that are not mid vowels. These constraints notwithstanding, there are a large number of words where nasal harmony does not occur where expected. Nasal spread is highlighted by an underline.

\begin{exe}
\item\label{nonasalharmony} Unexpected lack of nasal harmony
	\begin{xlist}
	\item\label{nonasalharmony1}
	\glll	 \textnormal{[ˈi.\uline{mã}.ʃut], *[ˈ\uline{ĩ.mã}.ʃut]}\\
	 /i-wãʃut/\\
	 {pv.ins}-\textnormal{clothe}\\
	 \glt `clothes, shirt, dress, coat'

	\item\label{nonasalharmony2}
	\glll \textnormal{[ˈi.\uline{mĩ}.ka.wa.tke], *[ˈ\uline{ĩ.mĩ}.ka.wa.tke]}\\
	/i-wĩ-ka-watke/\\
	{pv.ins-1poss-ins.frce-}\textnormal{put}\\
	\glt `my driftwood doctor' \citep[87]{hollow1970}%\footnote{A driftwood doctor is a ceremonial item crafted from driftwood.}

	\item\label{nonasalharmony3}
	\glll \textnormal{[ˈaː.\uline{m\textsuperscript{ĩ}nĩ}.ɾeː.htoʔʃ], *[ˈ\uline{ãː.m\textsuperscript{ĩ}nĩ}.ɾeː.htoʔʃ]}\\
	/aː-w-ɾĩ-ɾEːh=kt=oʔʃ/\\
	{pv.tr-1a-2s}-\textnormal{go.there}={ind.m}\\
	\glt `I will take you' \citep[3]{hollow1973b}

	\item\label{nonasalharmony4}
	\glll \textnormal{[ˈⁿdiː.\uline{m\textsuperscript{ĩ}nĩ}.xoʔʃ], *[ˈ\uline{nĩː.m\textsuperscript{ĩ}nĩ}.xoʔʃ]}\\
	 /ɾV-i-wɾĩx=oʔʃ/\\
	 {1a.pl-pv.ins}-\textnormal{play}={ind.m}\\
	 \glt `we played' \citep[305]{hollow1970}

	\item\label{nonasalharmony5}
	\glll \textnormal{[ˈⁿdaː.haː.\uline{mĩː}], *[ˈⁿdaː.h\uline{ãː.mĩ}]}\\
	/ɾEːh=haː=awĩː/\\
	\textnormal{go.there}={sim=cont}\\
	\glt `while he was going along' \citep[269]{hollow1973b}

	\item\label{nonasalharmony6}
	\glll \textnormal{[ˈhuː.\uline{nĩ}], *[ˈh\uline{ũː.nĩ}]}\\
	/huː=ɾĩ/\\
	\textnormal{come.here}=ss\\
	\glt `he came and...' \citep[230]{hollow1973b}

	\item\label{nonasalharmony7}
	\glll \textnormal{[ˈpxi.\uline{nãː}.teʔʃ], *[ˈpx\uline{ĩ.nãː}.teʔʃ]}\\
	/pxi=ɾãːtE=oʔʃ/\\
	\textnormal{sneeze}={prsp=ind.m}\\
	\glt `he almost sneezed' \citep[468]{hollow1970}

	\item\label{nonasalharmony8}
	\glll \textnormal{[ˈⁿdaː.h\textsuperscript{a}ɾaː.\uline{nĩ}], *[ˈⁿdaː.h\uline{\textsuperscript{ã}nãː.nĩ}]}\\
	/ɾEːh\#hɾE=ɾĩ/\\
	\textnormal{go.there}\#{caus=ss}\\
	\glt  `he put it there and...' \citep[26]{hollow1973a}

	\item\label{nonasalharmony9}
	\glll \textnormal{[ˈ\uline{nũː}.ʔe.taː.\uline{mĩː}hs], *[ˈ\uline{nũː}.ʔe.t\uline{ãː.mĩː}hs]}\\
	/ɾũːʔetaː\#wĩːh=s/\\
	\textnormal{Mandan}\#\textnormal{woman}={def}\\
	\glt `the Mandan woman' \citep[87]{hollow1973b}

	\item\label{nonasalharmony10}
	\glll \textnormal{[ˈⁿdeː.hka} \textnormal{\uline{mã}.ˈhe.ki], *[ˈⁿdeː.hk\uline{ã mã}.ˈhe.ki]}\\
	/ɾEːh=ka wã-hE=ki/\\
	\textnormal{go.there}=hab {1s}-\textnormal{see}={cond}\\
	\glt `if she saw me now' \citep[21]{hollow1973a}
	\end{xlist}
\end{exe}

In all of the words in (\ref{nonasalharmony}), the featural conditions are met such that nasal harmony should occur. All feature an apex vowel that precedes a syllable containing an underlying nasal vowel, such as /huː=ɾĩ/ `he came and...', and nasal harmony resolves as predicted within the syllable containing /ĩ/, yielding [nĩ]. However, nasal harmony does not spread onto the /uː/, despite the fact that /u uː/ can participate in nasal harmony under normal circumstances, e.g., /ɾuwãʔk/ $\to$ [\uline{nũ.ˈmã}ʔk] `man.'

There are three observations in (\ref{nonasalharmony}) above for what blocks the spread of nasal harmony: preverbs, enclitic boundaries, and word boundaries. In (\ref{nonasalharmony1}) through (\ref{nonasalharmony3}), nasality spreads leftward from a morphological element towards the left edge of the word, but does not trigger nasal harmony onto a preverb, even when that preverb is an apex vowel. Similarly, in (\ref{nonasalharmony4}) through (\ref{nonasalharmony8}), nasality spreads leftward from an enclitic and moves to the edge of the enclitic, but does not move onto its stem, regardless of whether that stem is a lexical root or another enclitic. The final observation is that nasal harmony is blocked from spreading past a word boundary. We see this blocking effect in compound nouns like the one in (\ref{nonasalharmony9}) as well as in independent prosodic words like in (\ref{nonasalharmony10}).

In all the examples in (\ref{nonasalharmony}) above, there must be some morphologically motivated reason for why nasal harmony does not occur. As such, we cannot attribute these non-featural blocking environments to the phonology alone.\footnote{The position taken in \citet{kasak2019} is that words in Mandan can have more articulated internal structure, i.e., composite words, where there is a morphological head and then other material within the domain of the overall word that is not a word itself. This structure resembles a compound in that there is some word that is the head of the overall word (i.e., the word that determines its semantic category), resulting in a structure like [ X [ Y ]\textsubscript{Head} ]. This analysis is not vital for understanding the ``unexpected'' blocking environments described above.} Enclitic boundaries, word boundaries, and preverbs are also blocking environments for nasal harmony in Mandan. Given the data we have seen in (\ref{nonasalharmony}) above, combined with those we saw previously in (\ref{secondnasals}), we can revise the blocking environments:

\largerpage
\begin{exe}

\item\label{ThreeBlockingEnvironmentsRedux} Blocking environments for nasal harmony (revised)

	\begin{xlist}

	\item Mid vowels: Mandan does not permit nasal mid vowels, so nasal harmony cannot manifest on /e eː o oː/.

	\item Voiceless consonants: Nasal harmony can only manifest on voiced segments, and the only voiced segments in Mandan are vowels and /w/ and /ɾ/, so anything that is [$-$voice] blocks the spread of nasal harmony.

	\item Word boundaries: Nasal harmony is a word-level process and it cannot spread from one word to another.

	\item Enclitic boundaries: Nasal harmony cannot spread leftward across an enclitic boundary.

	\item Preverbs: Mandan does not permit nasality to spread onto a preverb.

	\end{xlist}

\end{exe}

The principal issue with previous researchers describing nasal harmony in Mandan is that their generalizations did not capture the full range of blocking environments for this phonological process. Phonology alone is unable to account for why nasality cannot continue spreading leftward across certain environments. As such, it is only through examination of Mandan's morphology and morphological junctures within words that the pattern becomes clear. Mandan is not the only Siouan language to exhibit nasal harmony, but it the only Siouan language that has been documented to have purely regressive nasal harmony. Hoocąk and Lakota have both been described as having systems of nasal harmony, but more research is needed to investigate how Mandan's system of nasal harmony fits within the typology of nasal spreading in Siouan \citep{kasaklundquist2019,panick2021}.


\subsection{Stress}\label{stress}

Mandan has a robustly predictable system of primary stress assignment. Primary stress is iambic and weight-sensitive, so primary stress typically appears on the second syllable, unless the first syllable is heavy, in which case the word has first-syllable stress. There is no primary stress on the third syllable or beyond. Words containing a single light syllable can still bear primary stress, so we can tell that footing for primary stress does not cross word boundaries, i.e., a poorly formed iamb is preferable to a well-formed iamb that is footed across a word boundary.

This behavior for stress is not remarkable by itself. What is noteworthy, however, is that preverbs typically appear with primary stress, even though most preverbs are light syllables. Even if forming an iamb is possible, a word with a preverb with no preceding morphological material will always feature first-syllable stress, despite the fact that a well-formed iamb is possible. This stress assignment behavior seems unexpected at first, but if we analyze words with preverbs as having internal word boundaries, then this generalization about stress not being footed across word boundaries readily explains the presence of a light first syllable bearing stress over even a heavy second syllable.

\subsubsection{Previous descriptions of Mandan stress}\label{previousstress}

One aspect of the sound system of Mandan that has been treated with the highest degree of inconsistency is that of word-level prominence marking. The aristocrat and adventurer Prince \citet{maximilian1839} uses German orthographic conventions to transcribe Mandan, and typically marks where he hears primary stress with an acute accent mark. The first line in (\ref{maximilianstress}) below is \citeauthor{maximilian1839}'s transcription, followed by the transcription employed under the schema established in this book, then the phonetic and phonemic representations of that word, and finally its gloss.

\begin{exe}
\item\label{maximilianstress} Accentuation in \citet{maximilian1839}

\begin{xlist}
\item $\langle$tapsá$\rangle$\\
	\textit{tapsá}\\
	{[}(ta.ˈpsa)]\\
        /tapsa/\\
	`ash tree'

\item $\langle$máhnu$\rangle$\\
    \textit{máanu}\\
    {[}(ˈmãː).nũ]\\
    /waːɾũ/\\
    `turkey'



\item $\langle$uihkchák-chäkä$\rangle$\\
	\textit{wíikxaakxeka}\\
	{[}(ˈwiː).kxaː.kxe.ka]\\
        /waa-i-kxE$\sim$kxEka/\\
	`magpie'

\item $\langle$scháh-hä$\rangle$\\
	\textit{shá'he}\\
	{[}(ˈʃaʔ).he]\\
        /ʃaʔh=E/\\
        `hoof'
\end{xlist}

\end{exe}

\citet{maximilian1839} is often able to discern vowel length, which he typically marks in the German orthographic custom by following a vowel with an $\langle$h$\rangle$. However, he does sometimes conflate vowel length and stress, as we see in \textit{wíikxaakxeka} `magpie' in (\ref{maximilianstress}) above.

The second-oldest Mandan word list by the trader \citet{kipp1852}, whose wife was Mandan, sparingly uses diacritics. When \citeauthor{kipp1852} does include diacritic marks, an acute accent typically appears on a word-final $\langle$e$\rangle$, and it is unclear whether this acute accent mark is meant to indicate that the sound is not silent as in English, if it is to emulate the sound of the $\langle$e$\rangle$ with \textit{accent aigu} as in French (i.e., [e] instead of [ɛ]), or if the acute accent mark indicates stress. The data in (\ref{kipptrans}) within angled brackets represents \citeauthor{kipp1852}'s transcription, followed by the transcription used within this book, then the phonetic and underlying representation of these words, and finally the gloss.

\begin{exe}
\item\label{kipptrans} Accentuation in \citet{kipp1852}

\begin{xlist}
\item $\langle$warade$\rangle$\\
    \textit{wará're}\\
    {[}(ˈwᵃɾaʔ).ɾe]\\
    /wɾaʔ=E/\\
    `fire'

\item $\langle$wahe$\rangle$\\
	\textit{wá'he}\\
 	{[}(ˈwaʔ).he]\\
	/waʔh=E/\\
	`snow'




\item $\langle$xaxe$\rangle$\\
	\textit{xą́he}\\
	{[}(ˈxã).he]\\
	/xãh=E/\\
	`grass'

\item $\langle$xooré$\rangle$\\
	\textit{xóore}\\
	{[}(ˈxoː).ɾe]\\
	/xoː=E/\\
	`ice'

\item $\langle$äapé$\rangle$\\
	\textit{áape}\\
	{[}(ˈaː).pe]\\
        /aːp=E/\\
	`leaf'

\item $\langle$manisérute$\rangle$\\
	\textit{miníseerute}\\
	{[}(ˈm\textsuperscript{ĩ}nĩs).eː.ɾu.te]\\
	/wɾĩs\#eː-rut=E/\\
	`dog'
\end{xlist}

\end{exe}

In the case of the last two words above, `leaf' and `ice', both words contain a heavy syllable followed by an unstressed final syllable. \citet{kipp1852} includes an $\langle$é$\rangle$ word-finally, even though that syllable should not bear any kind of stress, while he neglects to put any diacritic on the final vowel in `fire'. Given this inconsistency throughout his wordlist, and given the lack of accentuation on any other vowel, we cannot surmise precisely what \citeauthor{kipp1852}'s intentions are. This wordlist was created at the behest of \citet{schoolcraft1853}, who was tasked with creating a comprehensive survey of the indigenous peoples of the United States and their languages through the Bureau of Indian Affairs.

Following his fieldwork on the Fort Berthold Reservation, \citet{kennard1936} publishes a brief grammatical sketch of Mandan wherein he states that stress is not predictable and can shift along a stem when affixes are added. \citeauthor{kennard1936} gives the following examples of how affixation affects stress placement. In (\ref{kennardstress}) below, we see \citeauthor{kennard1936}'s transcription, followed by the transcription used in this book, then the phonetic and phonemic representations, and finally the gloss.



\begin{exe}
\item\label{kennardstress} Accentuation in \citet{kennard1936}


\begin{xlist}
\item $\langle$númąk$\rangle$\\
	\textit{numá'k}\\
 	{[}(nũ.ˈmãʔk)]\\
	/ɾuwãʔk/\\
	`man'

\item $\langle$numą́kci$\rangle$\\
	\textit{numá'kshi}\\
 	{[}(nũ.ˈmãʔk).ʃi]\\
	/ɾuwãʔk\#ʃi/\\
	`chief'

\item $\langle$kínumą̀kci$\rangle$\\
	\textit{Kinúma'kshi}\\
 	{[}(ki.ˈnũ).mãʔk.ʃi]\\
	/ki-ɾuwãʔk\#ʃi/\\
	`Royal Chief'

\item $\langle$númąkàki$\rangle$\\
	\textit{numá'kaaki}\\
 	{[}(nũ.ˈmãʔ).kaː.ki]\\
	/ɾuwãʔk-aːki/\\
	`people, humans'

\item $\langle$ákinumą̀kaki$\rangle$\\
	\textit{Áakinuma'kaaki}\\
  	{[}(ˈaː).ki.nũ.mãʔ.kaː.ki]\\
	/aːki\#ɾuwãʔk-aːki/\\
	`Native American'

\end{xlist}
\end{exe}

Though \citeapos{kennard1936} original orthography attempts to indicate primary and secondary stress, he is very inconsistent in where he marks stress. There are numerous instances where the same word has stress on differing syllables within the same sentence. This wide discrepancy in stress marking largely comes from \citeauthor{kennard1936} conflating stress and phrasal pitch accent.

The most divergent approach to dealing with Mandan stress is found in \citet{hollow1970}, who states that Mandan does not have vowel length, contra \citet{kennard1936} and \citet{maximilian1839}. \citeauthor{hollow1970} proposes that stems in Mandan either have or lack underlying stress. Furthermore, \citet[50]{hollow1970} proposes the following system for stress:

\begin{quote}
In two syllable roots with stress on the second syllable, and with stressed monosyllabic verb roots preceded by instrumental prefixes, stress may be moved to the first root syllable or to the instrumental prefix if the stress movement would result in the stress being placed on the second syllable of the derived form. Details of stress placement under these conditions have not been worked out.
\end{quote}

\noindent \citeapos{hollow1970} generalization is codified with two rules: one to place primary stress, and another to remove the underlying stress from the remaining elements in a word.

\begin{exe}
\item\label{hollowstressrules} Stress assignment rules in \citet{hollow1970}

\begin{xlist}
\item\label{hollowstressrules1} \textbf{Primary Stress Assignment Rule}\\
V $\to$ [+stress] / \#C\textsubscript{0}{\longrule}C\textsubscript{0}V\textsuperscript{[$-$stress]}\\
	\textit{Assign stress to the first syllable preceding an unstressed syllable.}

\item\label{hollowstressrules2} \textbf{Underlyinig Stress Deletion Rule}\\
V $\to$ [$-$stress] / V\textsuperscript{[+stress]}Σ\textsubscript{0}{\longrule}\#\\
	\textit{Delete the [$+$stress] feature from any syllable that appears after a stressed syllable.}

\end{xlist}

\end{exe}

We can see the application of these stress rules at work below in an example that \citet[50]{hollow1970} gives. The data here use \citeauthor{hollow1970} underlying representation and phonetic interpretation of the words \textit{wáaratookaxi'h} `old man.'

\begin{exe}

\item\label{hollowrules} Stress assignment and deletion in \citet{hollow1970}

\begin{tabular}{llllllll}
/wá-& +&
	ratór& +&
	-ka& +&
	xíh/&\\
wá-& +&
	ratór& +&
	-ka& +&
	xí&	final resonant deletion\\
wá-& +&
	rató& +&
	-ka& +&
	xí&	preconsonantal /r/ deletion\\
wá-& +&
	rato& +&
	-ka& +&
	xi&	stress\\
\multicolumn{8}{l}{$<$wáratokaxi$>$}
\end{tabular}
\end{exe}

\citet{hollow1970} is generally correct in where he places primary stress in his transcribed narratives, but the application of these stress rules is inconsistent throughout his dictionary and grammar. Furthermore, he states that stress can fall on vowels he describes as epenthetic, which spectral analysis shows not to be the case. \citet{mixco1997a} largely follows \citeapos{hollow1970} interpretation of some roots having underlying stress, but acknowledges that certain roots contain underlyingly long vowels.

Some contemporaries who have worked with Mandan have even described Mandan as having a pitch accent system rather than a stress accent system (\citeauthor{boyle2007} p.c., \citeauthor{park2012} p.c.). \citeauthor{park2012} (p.c.) has even gone so far as to posit that Mandan has a pitch accent system whereby high tone can be found on multiple adjacent morae, with some long vowels having pitch contour differences. \citeauthor{park2012}'s transcription of Mandan words he describes as having a pitch accent appears in (\ref{indrekmandanpitch}) with his interpretation of that Mandan word depicted in angled brackets, followed by the orthography used in this work, then the phonetic and underlying representations, and then the gloss.

\begin{exe}
\item\label{indrekmandanpitch} \citeauthor{park2012}'s (p.c.) pitch marking

\begin{xlist}
\item $\langle$taxáráxe$\rangle$\\
	\textit{taxaráxe}\\
 	{[}(ta.ˈxᵃɾa).xe]\\
	/ta-xɾax=E/\\
	`his chest'

\item $\langle$tóóp$\rangle$\\
	\textit{tóp}\\
  	{[}(ˈtop)] $\sim$ [(ˈtoːp)]\\
	/top $\sim$ toːp/\\
	`four'
\end{xlist}

\end{exe}

The interaction between pitch and stress in Mandan is discussed in \sectref{secdorseystress}, wherein I explain that the perception of stress or high tone on Dorsey's Law vowels is related to the physiological process of F0 undershoot on the way to the target vowel. The perception of a long rising vowel for \textit{tóp} is an elicitation effect caused by one speaker emphasizing it when in isolation. More on the interaction of pitch and stress is discussed in \sectref{secdorseystress} below.

Overall, any researcher who has discussed stress in Mandan has stated that stress assignment is something that must be worked out in future. No author revisits the issue of stress in subsequent scholarships, and as such, it is left to this work to re-examine it. In this section, I discuss the overarching pattern for primary stress assignment in Mandan, as well as secondary stress, which is a topic heretofore untouched by previous scholars. I argue that primary stress assignment is quite regular and generally predictable, with the exception of certain fossilized compounds where stress seems to be lexical synchronically but still adhere to regular stress placement rules if viewed diachronically.

\subsubsection{Default primary stress assignment}\label{primarystress}

In their analysis of the sound system of Proto-Siouan, \citet{rankinetalnd} note that primary stress in Proto-Siouan is overwhelmingly on the second syllable, unless the first syllable bears a long vowel.\footnote{We can reconstruct stress in Proto-Siouan, in part, through Carter's Law, where a plain stop in Pre-Proto-Siouan becomes pre-aspirated in Proto-Siouan \citep{rankinetalnd}.

\begin{exe}
\item[(2.i)] Carter's Law:

	Pre-Proto-Siouan **C > Proto-Siouan *hC / {\longrule}V́(V)

\end{exe}

Preaspirated stops have different reflexes than plain stops in daughter languages, so this former allophony in Pre-Proto-Siouan had become reanalyzed as being a phonemic difference in Proto-Siouan (\citeauthor{rankin2010} p.c.). The one Siouan language that flouts this stress pattern is Crow, which developed a Japanese-style pitch accent system where high pitch originates on a mora and spreads towards the leftmost bimoraic syllable bearing a long vowel: /maa-iihulí\#shoopé/ $\to$ [bááííhúlíshóópé] `table', but /ana-maa-chimmí-uu=a/ $\to$ [ammááchímmúua] `school' (cf. \citealt{graczyk2007} and \citealt{wallace1993}).} This pattern holds true even today for many Siouan languages across all branches of the Siouan language family, e.g., Hidatsa \citep{boyleetal2016}, Lakota \citep{boasdeloria1941}, Tutelo \citep{oliverio1997}, and Ioway-Otoe \citep{whitman1947}. With this family-wide pattern in mind, I have proposed that Mandan features a similar system of primary stress assignment \citep{kasakqp1}. Stress in Mandan is robustly drawn to the second syllable when a word begins with two light syllables, but when the first when the word begins with a heavy syllable, then that heavy syllable takes stress. This pattern is demonstrated below.

The first four items in (\ref{secondsyllable}) involve two initial light syllables. In such cases, stress falls upon the second syllable. Similarly, in cases where the second syllable is heavy (i.e., it contains a long vowel or coda [ʔ]), stress still falls on the second syllable. We can contrast this pattern with the one seen in (\ref{firstsyllable}), where an initial heavy syllable attracts stress, even if the word begins with two heavy syllables, as we see in \textit{kóoxą'te} `corn'. The vast majority of morphologically simple words in Mandan conform to this pattern.

\begin{exe}
\item\label{secondsyllable} Second-syllable stress

\begin{xlist}
\item 	\textit{ishák}\\
    {[}(i.ˈʃak)]\\
    /iʃak/\\
    `he, she, they'

\item \textit{tashká}\\
    {[}(ta.ˈʃka)]\\
    /taʃka/\\
    `how'

\item \textit{restą́}\\
    {[}(ⁿde.ˈstã)]\\
    /ɾestã/\\
    `bullsnake'

\item \textit{pasą́h}\\
    {[}(pa.ˈsãh)]\\
    /pasãh/\\
    `creek, stream'

\item \textit{Aríkara}\\
    {[}(a.ˈɾi).kᵃɾa]\\
    /aɾikɾa/\\
    `Arikara' (< Ar. \textit{arikaraánu'} `stag')

\item \textit{Ihą́tu}\\
    {[}(i.ˈhã).tu]\\
    /ihãtu/\\
    `Yankton' (< Dak. \textit{Iháŋktȟuŋwaŋ} `village at the end')

\item \textit{imáare}\\
    {[}(ĩ.ˈmãː).ɾe]\\
    /iwãː=E/\\
    `a body'

\item \textit{paxáare}\\
    {[}(pa.ˈxaː).ɾe]\\
    /paxaː=E/\\
    `beloved'

\item \textit{tamí'ti}\\
    {[}(ta.ˈmĩʔ).ti]\\
    /ta-wĩʔ\#ti/\\
    `his/her/their village'

\item \textit{rup\'{ı̨}'xe}\\
    {[}(ⁿdu.ˈpĩʔ).xe]\\
    /ɾu-pĩʔx=E/\\
    `to scatter (by hand)'

\end{xlist}

\item\label{firstsyllable} First-syllable stress

\begin{xlist}

\item \textit{pą́ąpi}\\
    {[}(ˈpãː).pi]\\
    /pãːpi/\\
    `thin'

\item \textit{pt\'{ı̨}įre}\\
    {[}(ptĩː).ɾe]\\
    /ptĩː-E/\\
    `a buffalo'

\item \textit{wáaxtik}\\
    {[}(ˈwaː).xtik]\\
    /waːxtik/\\
    `jackrabbit'

\item \textit{áakitaa}\\
    {[}('aː).ki.taː]\\
    /aːki=taː/\\
    `above'

\item \textit{kóoxą'te}\\
    {[}(ˈkoː).xãʔ.te]\\
    /koːxãʔte/\\
    `corn'

\item \textit{mí'he}\\
    {[}(ˈmĩʔ).he]\\
    /wĩʔh=E/\\
    `robe'

\item \textit{ná'ro'sh}\\
    {[}(ˈnãʔ).ɾoʔʃ]\\
    /ɾãʔ=oʔʃ/\\
    `it aches'

\item \textit{wá'kup}\\
    {[}(ˈwaʔ).kup]\\
    /wʔ-aːkup/\\
    `war bonnet'

\item \textit{sé'ro're}\\
    {[}(ˈseʔ).ɾoʔ.ɾe]\\
    /seʔ=oʔɾe/\\
    `it came apart'

\end{xlist}

\end{exe}



One of the major phonetic cues for primary stress is a raised F0 value. We can see the increase in pitch for the third person pronoun \textit{ishák} in \figref{ishak} and the interrogative \textit{tashká} in \figref{tashka} on the spectrograms below. These two words are typical LĹ iambs, so we see a pitch curve start at a significantly lower level on the first syllable, then rise to a peak within the begining of the vowel window and then dropping. This pitch curve is a regular process in Mandan in that the target for the F0 peak aligns to the left edge of the vowel in the syllable bearing primary stress.

\begin{figure}
\caption{\textit{ishák} \textsc{3.pro}}\label{ishak}
\includegraphics[scale=0.5]{figures/ishak.png}
\end{figure}


\begin{figure}
\caption{\textit{tashká} `how'}\label{tashka}
\includegraphics[scale=0.5]{figures/tashka.png}
\end{figure}

A similar behavior for F0 can be seen on the pitch contour for the H́ iambs with \textit{áakitaa} `above' in \figref{Aakitaa} and \textit{ní'ni} `he climbed and...' in \figref{niPni}. The pitch quickly reaches its peak within the first half of the vowel window for \textit{áakitaa}. Following this prominence peak, the pitch levels off for the rest of the word. For \textit{ní'ni}, F0 peaks within the window for [ĩ] before the transition to the [ʔ], after which the pitch falls and then levels off for the rest of the word in a manner similar to what we see in \textit{áakitaa}.

We likewise see a noticeable drop in F0 after the primary stress in these words, a pattern likewise observed in LH́ iambs, as seen in \figref{pataaNta} with \textit{patą́ąta} `push!' and in \figref{numaPk} with \textit{numá'k} `person'. In \textit{patą́ąta}, we once again see a sharp rise in pitch in the first syllable, and then F0 peaks within the first half of the window for [ãː]. After the peak, the pitch for the following syllable drops significantly. We see a slightly different behavior for the pitch in \textit{numá'k}. With the pitch curve in \textit{numá'k}, we do not notice as drastic a rise and fall in pitch. This behavior will be discussed further in \sectref{unexpectedstress}.



\begin{figure}
\caption{\textit{áakitaa} `above'}\label{Aakitaa}
\includegraphics[scale=0.5]{figures/Aakitaa.png}
\end{figure}
\begin{figure}
\caption{\textit{ní'ni} `he climbed and...'}\label{niPni}
\includegraphics[scale=0.5]{figures/niPni.png}
\end{figure}

\begin{figure}
\caption{\textit{patą́ąta} `pushǃ'}\label{pataaNta}
\includegraphics[scale=0.5]{figures/pataaNta.png}
\end{figure}
\begin{figure}
\caption{\textit{numá'k} `man, person'}\label{numaPk}
\includegraphics[scale=0.5]{figures/numaPk.png}
\end{figure}

With this stress assignment behavior in mind, we can pose the phonological rule in (\ref{PrimaryStressSPERule}).

\begin{exe}

\item\label{PrimaryStressSPERule} \textbf{Primary Stress Assignment Rule}\\
	\phonl{σ}{\phonfeat{$+$main\\$+$stress}}{\#(σ\textsubscript{μ})}\\
	\textit{Assign primary stress to the second syllable in a word, unless the first syllable is heavy.}

\end{exe}

The rule above in (\ref{PrimaryStressSPERule}) accounts for the pattern seen thus far by ruling out stress that is not part of an iambic foot aligned to the left edge of a word. This rule accounts for the distribution of primary stress we saw previously in (\ref{secondsyllable}) and (\ref{firstsyllable}), where first-syllable stress is associated with heavy syllables, versus second syllable stress, where the first syllable is light. Furthermore, while Mandan primary stress assignment is weight sensitive, a heavy syllable will not drag stress rightward if preceded by two light syllables. We can see this in the data in (\ref{LLHstress}) below.


\begin{exe}

\item\label{LLHstress} Primary stress in LLH words

\begin{xlist}

\item \textit{wakíxeekto'sh}\\
    {[}(wa.ˈki).xeː.ktoʔʃ]\\
    /wa-ki-xeː=kt=oʔʃ/\\
    `I'll give up'

\item \textit{rarúxąąho'sh}\\
    {[}(ⁿda.ˈɾu).xãː.hoʔʃ]\\
    /ɾa-ɾu-xãːh=oʔʃ/\\
    `you reach for it'

\item \textit{kotámiihe}\\
    {[}(ko.ˈta).mĩː.he]\\
    /ko-ta-wĩːh=E/\\
    `his sister'

\item \textit{wakátą'xo'sh}\\
    {[}(wa.ˈka).tãʔ.xoʔʃ]\\
    /wa-ka-tãʔx=oʔʃ/\\
    `I hammer at it'

\item \textit{nupásų'ro'sh}\\
    {[}(nũ.ˈpa).sũʔ.ɾoʔʃ]\\
    /ɾũ-pa-sũʔ=oʔʃ/\\
    `we swim'

\item \textit{Kinúma'kshis}\\
	{[}(ki.ˈnũ).mãʔk.ʃis]\\
        /ki-ɾuwãʔk\#ʃi=s/\\
	`Royal Chief, First Creator'\footnotemark

\end{xlist}

\end{exe}

\footnotetext{Several different translations appear throughout this work for Kinúma'kshi, a cultural hero that plays a major role in traditional Mandan narratives. Mrs. Mattie Grinnell typically translates his name as meaning `Royal Chief', while Mrs. Annie Eagle often translates his name as `First Creator.' In \citeapos{hollow1973a} transcribed narratives, he always translates this name as `Old Man Coyote.' Many other Plains groups have a cultural figure, Coyote, that either plays a role in the creation of the world or acts as a trickster who goes on adventures that serve as a fable to bestow some lesson upon the listener. In the Mandan narratives collected by \citet{kennard1934} and \citet{hollow1973a,hollow1973b}, Kinúma'kshi plays both roles. See \citet{erdoesortiz1998} for more information on Coyote figures in the cultures of different Plains groups.}

In each of the examples above in (\ref{LLHstress}), the presence of a heavy third syllable does not affect the placement of stress. \citeapos{prince1990} Weight-to-Stress Principle (WSP) holds that if a syllable is heavy, it tends to be stressed. In the data above, the generalization of the WSP does not hold. Therefore, we can maintain our established formulation of the rule for primary stress assignment from (\ref{PrimaryStressSPERule}).

\subsubsection{Secondary stress assignment}\label{secondarystress}

In addition to primary stress, the constraints discussed above also play into the assignment of secondary stress. The directionality of all stress in Mandan is left-aligned, so a foot bearing secondary stress will align its left edge to the right edge of a foot bearing primary stress. We can see this pattern in (\ref{secondarystressexamples}) below.\footnote{Secondary stress is historically not marked in Mandan orthography, with the only researcher who attempts to do this being \citet{kennard1936}. However, \citeauthor{kennard1936} often conflates stress and vowel length, so his secondary stress marking is not reliable. In the orthography used by Mandan language learners on the Fort Berthold Indian Reservation, secondary stress is not recorded, and primary stress marking is often omitted as well.}

\begin{exe}
\item\label{secondarystressexamples} Examples of secondary stress assignment

\begin{xlist}
\item 	\textit{máareksuk}\\
    {[}(ˈmãː).(ɾe.ˌksuk)]\\
    /wãːɾeksuk/\\
    `bird'

\item 	\textit{xóoxixąąka}\\
    {[}(ˈxoː).(xi.ˌxãː).ka]\\
    /xoːxixãː=ka/\\
    `crow'

\item 	\textit{rúuhaare}\\
	{[}(ˈⁿduː).(ˌhaː).ɾe]\\
        /ɾuːxaː=E/\\
	`buzzard'

\item   \textit{mashkáshkapka}\\
	{[}(mã.ˈʃka).(ʃka.ˌpka)]\\
        /wã-ʃka$\sim$ʃkap=ka/\\
	`rosehips, tomatoes'
\end{xlist}

\end{exe}

In each of the examples above, we see instances of varying combinations of syllables with differing weights. Regardless of whether a word consists of only light syllables (e.g., \textit{mashkápshkapka}) or mostly heavy syllables (e.g., \textit{rúuhaare}), the iambic foot with secondary stress is always adjacent to the iambic foot bearing primary stress, even if this juxtaposition creates instances where a stressed syllable abuts another stressed syllable.

\largerpage
The fact that heavy syllables can occur adjacent to one another means that stressed syllables can also be adjacent, as we see in the word \textit{rúuhaare} `buzzard', which has primary stress on the first syllable, but secondary stress on the second syllable. For these syllables to follow one after the other, Mandan must allow for stress clash, where two successive syllables can bear stress.

These observations lead us to the formulation of the following rule for secondary stress:

\begin{exe}

\item\label{SecondaryStressSPERule} \textbf{Secondary Stress Assignment Rule}\\
	\phonl{σ}{\phonfeat{$+$stress}}{\phonfeat{σ\\$+$main\\$+$stress}(σ\textsubscript{μ})}\\
	\textit{Assign secondary stress to the second syllable after the syllable bearing primary stress, unless the syllable following the primary stress is heavy.}

\end{exe}

It is not obvious that Mandan has iterative secondary stress or any kind of tertiary stress, as there is a strong tendency for the overall intensity of a word to drop after three syllables \citep{kasak2022}. Secondary stress does appear to be optional, however. Secondary stress also is associated with an increase in F0, but not to the same degree as primary stress. As we saw in \textit{áakitaa} in \figref{Aakitaa}, we would expect secondary stress on the final syllable. This word has an overall HLH structure, which should result in a (H́)(LH̀) parsing. However, there is no statistically significant difference in the mean F0 for the second syllable [ki] and the third syllable [taː], which is where we expect secondary stress to manifest.


\subsubsection{Unexpected stress assignment}\label{unexpectedstress}

\citet[35]{hollow1970} notes that there are several phonological issues that remain to be dealt with in Mandan. Among those problems, understanding the motivation behind unexpected stress assignments is one of the most daunting. The material below tackles this issue of ``unexpected'' stress assignment, and explains the factors that yield the surface stress assignments that we see in Mandan and why the stress we see there is not unmotivated.

\paragraph{Stress assignment and Dorsey's Law vowels}\label{secdorseystress}

\noindent As I have argued above, there is small set of rules that capture the pattern of stress assignment in Mandan. There are many words in the corpus that seem to flout this pattern in \citeapos{hollow1970} transcriptions, however, that seemingly call this argument into question. The data in (\ref{hollowdorseystress}) below appear the same as \citeauthor{hollow1970} transcribes them.\footnote{\citet{hollow1970} does not transcribe vowel length, instead postulating that certain syllables in Mandan bear underlying stress. This postulation is not borne out by the phonetic data; Mandan definitely has phonemic long and short vowels, and stress is predictable if one is familiar with the underlying morphology, as described in \sectref{primarystress} and \sectref{secondarystress} above.} \citeauthor{hollow1970}'s transcription appears in angled brackets, followed by the orthographic interpretation consistent with the one in this book, then the phonetic and phonemic representation of of each word, and finally the gloss for each word.

\largerpage
\begin{exe}
\item\label{hollowdorseystress} Exceptional first-syllable stress in \citet{hollow1970}

\begin{xlist}
\item 	$\langle$m\textsuperscript{í}nį$\rangle$\\
    \textit{miní}
    {[}(ˈm\textsuperscript{ĩ}nĩ)]\\
    /wɾĩ/\\
    `water'

\item 	$\langle$w\textsuperscript{é}rok$\rangle$\\
    \textit{weróok}\\
    {[(ˈw\textsuperscript{e}ɾoːk)]}\\
    /wɾoːk/\\
    `buffalo bull'

\item 	$\langle$w\textsuperscript{á}rap$\rangle$\\
    \textit{waráp}\\
    {[(ˈw\textsuperscript{a}ɾap)]}\\
    /wɾap/\\
    `beaver'

\item 	$\langle$p\textsuperscript{é}reʔš$\rangle$\\
    \textit{peré'sh}\\
    {[(ˈp\textsuperscript{e}ɾeʔʃ)]}\\
    /pɾe=oʔʃ/\\
    `he licks it'

\item $\langle$h\textsuperscript{á}nąʔre$\rangle$\\
    \textit{haná're}\\
    {[(ˈh\textsuperscript{ã}nãʔ).ɾe]}\\
    /hɾãʔ=E/\\
    `to sleep'



\item 		$\langle$k\textsuperscript{á}rašekoʔš$\rangle$\\
    \textit{karáasheko'sh}\\
    {[(ˈkaɾaː).ʃe.koʔʃ]}\\
    /kɾaːʃek=oʔʃ/\\
    `it clears up [after a storm]'

\item 	$\langle$x\textsuperscript{á}mah$\rangle$\\
    \textit{xamáh}\\
    {[(ˈx\textsuperscript{ã}mãh)]}\\
    /xwãh/\\
    `small'
\end{xlist}

\end{exe}

These CVRV sequences in \citeapos{hollow1970} transcriptions raise cause for concern over whether there is a single phonological process for stress assignment in Mandan. Further complicating matters is how common such sequences are, with RVRV making up an enormous portion of the corpus due to most of the prefix field consisting of prefixes with a basic /RV-/ shape. RVRV words often manifest in
\citeauthor{hollow1970}'s transcriptions as having stress on either syllable in different places in his narratives, e.g., both $\langle$m\textsuperscript{í}ni$\rangle$ and $\langle$m\textsuperscript{i}ní$\rangle$ for /wɾĩ/ 'water' appear, indicating he perceived stress after the first and second consonant in different instances. Furthermore, he often writes this word without any accentuation at all, suggesting that he either could not determine where the primary stress should fall or might have considered the word to have no underlying stress whatsoever. \citet[5]{kennard1936} likewise remarks that some words in Mandan seem to have an accent that is ``evenly distributed.'' These words are given below, with \citeauthor{kennard1936}'s original transcription appearing in angled brackets, the updated orthography below that, then the phonetic and underlying representations, and finally the gloss for each word.

\begin{exe}
\item\label{kennardaccentless} Accentless words in \citet{kennard1936}


\begin{xlist}
\item $\langle$manace$\rangle$\\
	\textit{manáshe}\\
 	{[}(ˈm\textsuperscript{ã}nã).ʃe]\\
	/wɾãʃ=E/\\
	`tobacco'

\item $\langle$natore$\rangle$\\
	\textit{ratóore}\\
	{[}(ⁿda.ˈtoː).ɾe]\\
	/ɾatoː=E/\\
	`male's father's older brother (voc.), elder (voc.)'

\end{xlist}

\end{exe}

There may have been more words \citet{kennard1936} considered to be accentless, but he only provides two in his grammar. With respect to `tobacco', it is a citation form where boundary tones and a Dorsey's Law vowel are interfering with the perception of stress. For \textit{ratóore}, it is a vocative form, so the intonational contour of the word will involve high tone at the right edge of the word, which is confounding the perception of the primary stress on the second syllable, which likewise will have a higher F0 than the first syllable.

Taking the first example of an accentless word from \citet{kennard1936} and combining it with the data in (\ref{hollowdorseystress}) in \citet{hollow1970}, these exceptions all share one thing in common: they have word-initial CR clusters in their underlying representations. Furthermore, the stress is being marked on a vowel that \citet{hollow1970} describes as epenthetic, i.e., the Dorsey's Law vowel. As previously discussed in \sectref{dorseyslaw}, my interpretation is that these sounds are not epenthetic at all. Rather, Dorsey's Law vowels in Mandan are intrusive sounds that are extraphonological in nature following the analysis of intrusive vowels by \citet{hall2006}. That is to say, phonological processes like stress assignment do not take these sounds into account when evaluating syllables and morae for stress assignment because these excrescent vowels are a phonetic, post-phonological phenomenon.

Dorsey's Law vowels are not treated phonologically as syllables in their own right, but acoustically they are vowel sounds that spill over between an articulatory gap between a consonant cluster involving a sonorant. As such, the intrusive vowel sound is really just an extension of the vowel that follows. Any intrusive vowel is tautosyllabic with the vowel whose features it shares. This tautosyllabicity is the reason for the varying perception of where stress falls in words beginning with underlying /CRV/ sequences. Primary stress places an articulatory target on a particular syllable in regards to pitch. Voiced consonants are articulatorilly more conducive to the production of pitch, and the only voiced sounds in Mandan are sonorants.

This conduciveness of pitch production allows for speakers to more easily undershoot the F0 target. This process is observed in \figref{numaPk}. Peak F0 is reached within the [m], right as the vocal cavity opens to produce the [ã] in \textit{numá'k}, which is the true target of this high pitch by virtue of it being the head of the iambic foot in [(nũ.ˈmãʔk)].

In all the figures discussed so far, we see a clear pattern: peak F0 is highest on the syllable bearing primary stress, along with a higher average F0 over the duration of the stressed vowel. The evidence presented herein points to F0 being a key component of primary stress marking in Mandan. With this pattern in mind, let us now compare these words with predictable stress manifesting on the expected syllable to those where stress seemingly appears sooner.

In the figures below, we can see a relatively small rise and fall in F0 from the start of the word that reaches its peak before the target vowel, such as in \figref{manaN}, or a flat F0 that prematurely reaches its peak and maintains it until it reaches the target vowel, like in \figref{miniN}.


\begin{figure}
\caption{\textit{maná} `wood, tree'}\label{manaN}
\includegraphics[scale=0.5]{figures/manaN.png}
\end{figure}

\begin{figure}
\caption{\textit{miní} `water, liquid'}\label{miniN}
\includegraphics[scale=0.5]{figures/miniN.png}
\end{figure}


The data in the figures above represent typical behavior for F0 in words beginning with consonant clusters consisting of two sonorants.
F0 behaves in a similar manner in words that begin with clusters that produce Dorsey's Law vowels where the first element of the cluster is not a sonorant. In \figref{terekerek}, we see the word [(ˈtᵉɾe).kᵉɾek] /tɾE=kɾE=ak/ `them being big around' with F0 starting high through the first vocal element until it peaks at the target vowel. The word [(x\textsuperscript{ã}mã).he] /xwãh=E/ `small' likewise begins with an early F0 peak that plateaus until the target vowel, then drops sharply.


\begin{figure}
\caption{\textit{terékerek} `them being big around'}\label{terekerek}
\includegraphics[scale=0.5]{figures/terekerek.png}
\end{figure}
\begin{figure}
\caption{\textit{xamáhe} `small'}\label{xamahe}
\includegraphics[scale=0.5]{figures/xamahe.png}
\end{figure}


Regardless of whether a word begins with a stop-sonorant, fricative-sonorant, or a sonorant-sonorant cluster, the behavior of F0 is identical: F0 approaches or achieves peak F0 during the first vocalic window and then begins to fall somewhere between the transition between the sonorant and the second vocalic window. Since Dorsey's Law vowels are phonologically part of the same syllable as the following vowel, the phonetic correlates of stress apply to these excrescent vowels as well. As such, the heightened F0 associated with primary stress likewise affects Dorsey's Law vowels. Past interpretations of first syllable stress come from this process; previous scholars write stress in words like \textit{miní} `water' and \textit{maná} `tree, wood' as having first syllable stress because they do have first syllable stress due to the fact that the excrescent vowels are not assigned syllables of their own.

We can likewise tell that these intrusive vowels are not treated like syllables phonologically because there are instances of third- or even fourth-syllable stress in \citet{hollow1970}. In (\ref{thirdfourthstress}), we see instances of \citeauthor{hollow1970}'s transcription in angled brackets, then the orthography used within this book, followed by the phonetic and underlying representations, and finally the gloss of each word.

\begin{exe}
\item\label{thirdfourthstress} Third- and fourth-syllable stress in \citet{hollow1970}

\begin{xlist}
\item $\langle$paxⁱrúke$\rangle$\\
	\textit{paxirúuke}\\
 	{[}(pa.ˈxⁱɾuː).ke]\\
	/paxɾuːk=E/\\
        `corn silk'

\newpage
\item $\langle$mⁱnįkúktoʔš$\rangle$\\
	\textit{minikų́'kto'sh}\\
	{[}(m\textsuperscript{ĩ}nĩ.ˈkũʔ).ktoʔʃ]\\
	/w-ɾĩ-kũʔ=kt=oʔʃ/\\
        `I will give it to you'

\item $\langle$kⁱnįkⁱn\'{ı̨}k$\rangle$\\
    \textit{kinikiník}\\
    {[}(k\textsuperscript{ĩ}nĩ.ˈk\textsuperscript{ĩ}nĩk)]\\
    /kɾĩkɾĩk/\\
    `kinnikinnick' (< PAlg *kerek-en- `mix by hand')
\end{xlist}

\end{exe}

All of these words that deviate from the expected first- or second-syllable stress in (\ref{thirdfourthstress}) above are actually typical iambs that happen to have one or more clusters that trigger a Dorsey's Law vowel. These excrescent vowels are tautosyllabic with the underlying vowel, and as such are not treated as belonging to different syllables for the purposes of footing. These words alternatively appear with stress on the Dorsey's Law vowel in \citeapos{hollow1970} transciptions. The spectrogram in \figref{paxiruuke} below illustrates why previous attempts to pin down Mandan stress placement have yielded varying results. The underlying cluster /xɾ/ in \textit{paxirúuke} `cornsilk' features a high F0 that peaks and levels out for the duration of both vocalic windows in the second syllable of [(pa.ˈxⁱɾuː).ke] before falling steeply in the third syllable.

\begin{figure}
\caption{\textit{paxirúuke} `cornsilk'}\label{paxiruuke}
\includegraphics[scale=0.5]{figures/paxiruuke.png}
\end{figure}

This misperceived stress is caused by the same factors that result in stress being perceived on Dorsey's Law vowels as discussed above. There is a premature peak in F0 on the excrescent vowel, and this high pitch typically plateaus onto the target vowel. Since there is such a stark contrast between the high pitch on the excrescent vowel and the preceding vowel, listeners may interpret this change as the cue for primary stress and transcribe stress too early in the word (e.g., \citealt{kennard1936} and \citealt{hollow1970}). Alternatively, this pitch plateau can be taken as a sign that there is some kind of pitch spreading along voiced segments for those scholars who have described Mandan as a language with a Tokyo Japanese-style pitch accent (e.g., \citeauthor{park2012} p.c.).

As the data presented here demonstrate, stress in Mandan is predictable once one analyzes certain sequences as being underlying clusters that trigger excrescent vowel insertion due to Dorsey's Law. This same pattern holds for other Siouan languages that feature Dorsey's Law vowels such as Hoocąk and Dakota, as we can see in (\ref{otherdorsey}) below.\footnote{The depiction of stress in Hoocąk takes into account that the first mora in a word is typically unfooted, and that primary stress falls on the third mora of a word. Stress skips over Dorsey's Law vowels in this language. See \citet{miner1979,miner1981} and \citet{halewhiteeagle1980} for further detail on iambic stress in Hoocąk.} The original authors' transcriptions appear in angled brackets, then phonetic and phonemic representations directly follow, and finally the gloss for each word.



\begin{exe}
\item\label{otherdorsey} Iambic stress and Dorsey's Law vowels in other Siouan languages

    \begin{xlist}
    \item\label{otherdorsey1} Hoocąk \\
    $\langle$hikorohó$\rangle$\\
    {[}<hi>.(kᵒɾo.ˈho)]\\
    /hi-kɾo-ho/\\
    `he gets dressed' \citep[128]{halewhiteeagle1980}

    \item\label{otherdorsey2} Dakota \\
    $\langle$wab.lúġa$\rangle$\\
    {[}(wa.ˈbᵘlu).ɣa]\\
    /wa-w-yuɣa/\\
    `I separate it from its outer covering' \citep[9]{boasdeloria1941}

	\end{xlist}
\end{exe}

In each of the examples in (\ref{otherdorsey}) above, we see left-aligned iambic footing for primary stress. Hoocąk leaves the first mora of a word unparsed, but otherwise we see that the Dorsey's Law vowel in /kɾo/ is not taken into account for counting morae for footing purposes. The Dakota example demonstrates that typical primary stress assignment in Dakota functions in the same way as in Mandan, where excrescent vowels are not treated as syllables when stress is assigned.

The overall takeaway from the data presented herein is that the wide variety of transcriptions for words in Mandan can be accounted for once the underlying structure of a word is considered. Namely, when an underlying consonant cluster ends with a sonorant, an excrescent vowel that copies features of the syllable nucleus will be inserted between the sonorant and the other consonant. This vocalic intrusion is not phonological in nature, as we can tell from the fact that the phonology of the language is blind to it. This excrescent Dorsey's Law vowel exhibits many of the same phonetic correlates of stress as the original syllable nucleus because for all intents and purposes it is actually the same vowel. Stress cannot be assigned to these vowels alone due to the fact that excrescent vowels are tautosyllabic with the vowels whose features they are copying.


\paragraph{Stress assignment and pre- and post-verbal elements}\label{ParaPreverbStress}

While many of the inconsistent instances of stress marking in \citet{kennard1936} and \citet{hollow1970} are due to their perception of the interaction between F0 and Dorsey's Law vowels, there is one other source of unexpected stress assignment to be found in Mandan. These instances of unexpected stress stem from how stress interacts with preverbs and postverbal clitics. The observed pattern is twofold: firstly, that preverbs will always draw primary stress, even if they are short vowels, and secondly, that primary stress will never be placed onto an enclitic.


\begin{exe}
\item\label{preverbstress} Preverbs and primary stress

	\begin{xlist}
	\item\label{preverbstress1} /i-wa-tee=o'sh/ $\to$ \textit{íwateero'sh} `I like her'
	\item\label{preverbstress2} /o-wą-shraa=o'sh/ $\to$ \textit{ómasharaaro'sh} `I slid'
	\item\label{preverbstress3} /e-wa-he=o'sh/ $\to$ \textit{éepe'sh} `I said it'
	\item\label{preverbstress4} /aa-wa-rEEh=o'sh/ $\to$ \textit{áawareeho'sh} `I brought it'
	\end{xlist}

\item\label{postverbalcliticsstress} Enclitics and primary stress

	\begin{xlist}
	\item\label{postverbalcliticsstress1} /tu=ootE/ $\to$ \textit{túroote} `there must be some'
	\item\label{postverbalcliticsstress2} /hE=oowąk=o'sh/ $\to$ \textit{héroomako'sh} `he saw it'
	\item\label{postverbalcliticsstress3} /hi=rįįtE=o'sh/ $\to$ \textit{híniiteroomako'sh} `he got there fast'
	\item\label{postverbalcliticsstress4} /pxik=o'sh/ $\to$ \textit{pxíko'sh} `it worked loose and fell'
	\item\label{postverbalcliticsstress5} /wrįs=E/ $\to$ \textit{miníse} `horse'
	\end{xlist}

\end{exe}

The preverbs in (\ref{preverbstress}) all draw stress, even though most of them do not satisfy the requirement to form iambic feet, i.e., most preverbs are short vowels and should therefore be passed over for primary stress. The transitivizer preverb \textit{aa}- in (\ref{preverbstress4}) follows typical stress assignment, since it is a heavy syllable, but the other preverbs do not. This raises the question of whether \citeapos{hollow1970} analysis that certain morphological items carry underlying stress is correct. Furthermore, these data raise the question of whether footing is always iambic.

An additional complication is the fact that the data in (\ref{postverbalcliticsstress}) show that primary stress cannot fall on an enclitic. In (\ref{postverbalcliticsstress1}) through (\ref{postverbalcliticsstress4}), the second syllable in each word is heavy and should attract stress to become a LH́ iamb. Similarly, the data in (\ref{postverbalcliticsstress5}) are such that stress should fall on the second syllable to form a LĹ iamb. In none of the data in (\ref{postverbalcliticsstress}), however, do we see expected iambic footing.

Combining what we see in (\ref{preverbstress}) and (\ref{postverbalcliticsstress}), we must account for how such stress assignment is possible. One explanation for the data in (\ref{postverbalcliticsstress}) is that stress does not fall on enclitics due to the fact they are prosodically deficient by virtue of the fact enclitics in Mandan are phonological clitics, i.e., they rely on prosodically adjoining to a prosodic word to be phonetically realized. As such, we may motivate the stress pattern with respect to (\ref{postverbalcliticsstress}). This same explanation does not hold for the data involving preverbs in (\ref{preverbstress}), as preverbs are not proclitics, and can be surrounded by other prefixes. We can see that preverbs can both follow and precede inflectional prefixes in the examples in (\ref{preverbplusprefixes}) below.



\begin{exe}
\item\label{preverbplusprefixes} Preverbs in Mandan plus other prefixes

	\begin{xlist}
	\item\label{preverbplusprefixes1} \glll	wáa'orakaraahinixiniite'sh\\
	waa-o-ra-ki-rEEh=rįx=rįįtE=o'sh\\
	neg-pv.irr-2a-vert-\textnormal{go.back.there}=neg=cel=ind.m\\
	\glt `you're not going to go just yet' \citep[216]{hollow1973b}

	\item\label{preverbplusprefixes2} \glll	karóoruxihka\\
	ka-ro-o-ru-xik=ka\\
	agt-1s-pv.irr-ins.hand-\textnormal{be.bad}=hab\\
	\glt `the ones bad to us' \citep[45]{hollow1973b}

	\item\label{preverbplusprefixes3} \glll	wáa'iwahekinixo'sh\\
	waa-i-wa-hek=rįx=o'sh\\
	neg-pv.ins-1a-\textnormal{know}=neg=ind.m\\
	\glt `I don't know' \citep[47]{hollow1973a}
	\end{xlist}
\end{exe}

All of the data above features expected iambic footing, either by having H́ feet like in (\ref{preverbplusprefixes1}) and (\ref{preverbplusprefixes3}) or LH́ feet like in (\ref{preverbplusprefixes2}). Each example above has an inflectional prefix flanking a preverb and, except for (\ref{preverbplusprefixes2}), the preverb is not drawing primary stress. In (\ref{preverbplusprefixes2}), we may say that stress is technically on the preverb, but that same syllable also contains the first person plural stative pronominal prefix \textit{ro}-, which blends with the irrealis preverb \textit{o}- to form a single syllable. Adding additional prefixes to the left of a preverb draws primary stress away from it if a heavy syllable occurs at the left edge of the word. Otherwise, in the case of a light syllable preceding the preverb, iambic footing takes place normally.

These data show us that preverbs by themselves are not underlyingly stressed. The data do point to the fact that there is something about preverbs that prevents normal iambic footing unless additional morphological material precedes a preverb. Once additional elements are prefixed onto a stem bearing a preverb, the expected iambic stress pattern resumes. It is not the case that preverbs suddenly cause footing to go from iambic to trochaic, but the question remains as to what is preventing the expected footing to occur past a preverb and into the rest of the stem.

As noted in \sectref{nasalharmony}, preverbs also act as a barrier to nasal harmony spreading. This same barrier appears to block the left-to-right directionality of footing for primary stress, as well as right-to-left nasal harmony. I argue in Chapter \ref{chapter4} that this barrier is really a word boundary within the greater morphological word. Preverbs are not words themselves, but are morphologically part of a composite word. Footing and nasal harmony are unable to cross a word boundary, even an internal word boundary, and this internal word boundary in turn helps explain why \citet{hollow1970} may have posited that preverbs are underlyingly stressed. There really is only a single process for stress assignment in Mandan, but this internal word boundary obfuscates this regular stress pattern.

\paragraph{Stress assignment in fossilized compounds}

There are a few words in Mandan that do not involve preverbs or enclitics that still display unexpected stress. These irregularities can be explained if we analyze them as adhering to the same rules as compounds in contemporary Mandan. In a compound word, primary stress is footed from the left edge of the word. Mandan does not allow footing across a word boundary, so if the first word contains a single light syllable, that syllable is marked with primary stress.

\begin{exe}
\item\label{normalcompounds} Stress assignment in compounds

	\begin{xlist}
	\item /suk\#ɾuwãʔk/ $\to$ \textit{súknuma'k} `young man' (lit. `child + man')

	\item /pax\#ʃowok/ $\to$ \textit{páxshowok} `bowl' (lit. `pot + shallow')

	\item /wɾãʃ\#oːt/ $\to$ \textit{manáshoot} `red willow' (lit. `tobacco + mix')

	\end{xlist}

\end{exe}

Modern speakers recognize the words above as being composed of two other words. However, the words below in (\ref{remainingunexpected}) are seen as being simplex words (\citeauthor{benson2000} p.c.).

\begin{exe}
\item\label{remainingunexpected} Words with unexpected stress

\begin{xlist}
\item /hũpɾĩh=E/ $\to$ \textit{hų́pinihe} `soup'

\item /kɾãhɾĩ/ $\to$ \textit{kanáhini} `grain, seed'

\item /hãxuɾaː=E/ $\to$ \textit{hą́xuraare} `bat'

\item /xopɾĩ/ $\to$ \textit{xópini} `be holy, sacred'

\item /wĩɾãʔki/ $\to$ \textit{mína'ki} `sun, moon, orb, boat, vehicle'
\end{xlist}

\end{exe}

The list above is not exhaustive, but does serve as a jumping point to show why these words have unexpected first syllable stress. Diachronically, each of these words is a compound. The word \textit{xópini} `be holy, sacred' is comprised of reflexes of PSi *xopE `holy' and *rį `be, exist.' Similarly, \textit{kanáhini} `grain, seed' has PSi *rį `be, exist' compounded at the end, with what appears to be a reflex of *krą `put' and the stem augment *-hE. The word for `soup' likewise is a compound of PSi *hųpV `juice, liquid' plus the PSi verb *rį `be, exist.' The root-final /h/ is likely another instance of the stem augment *-hE.

The word \textit{hą́xuraare} `bat' is an old compound that consists of PSi *hą `night, darkness' and *xuraa `eagle'. Neither of these words exist separately in modern Mandan, and as \citet{rankin2015} note, reflexes of these two Proto-Siouan roots only occur in compounds. In a similar manner, \textit{mína'ki} `sun, moon, orb' consists of PSi *wį `sun, moon, orb' plus the `sitting' positional *Raa-ke. The final vowel may be a fossilized remnant of the Proto-Siouan determiner *ki$\sim$*kį, though both \textit{mína'k} and \textit{mína'ki} appear in the Mandan corpus.

What these words with exceptional stress show is that they still maintain the expected pattern of stress assignment for compounds, even though the individual elements of those compounds are no longer analyzed as words. One can argue that stress in these words is now lexical, but we can also posit that these lemma are still compounds, even if one or more elements are no longer available in the lexicon. Either way, the motivation for stress placement in words like those in (\ref{remainingunexpected}) is not purely arbitrary. After all, even the few borrowings into Mandan ignore the stress of the original words and Mandanize them.


\begin{exe}
\item\label{mandanborrowings} Stress in borrowings

\begin{xlist}

\item\label{mandanborrowings1} Arikara to Mandan \citep{parksdemallie2002}

\textit{arikaráanu'} [(ˌə.ɾɪ).kə.(ˈɾaː.nʊʔ)] `stag' $\to$\\
	\textit{Aríkara} [(a.ˈɾi).kᵃɾa] `Arikara'\footnote{A traditional etymology behind this exonym for the Arikara is that a Mandan man first encountered an Arikara hunter who had shot and killed a stag near a Mandan village. When asked who he was and where he was from, the hunter just pointed to his kill and said the Arikara word \textit{árikaraaru'} `stag', not knowing exactly what the Mandan man was asking him. This word is also noteworthy in that it follows the tendency to borrow no more than three syllables of a loanword, even though all the sounds of the original Arikara word \textit{árikaraaru'} are possible for Mandan speakers.}


\item\label{mandanborrowings2} Lakota to Mandan \citep[208]{ullrich2011}

\textit{Iháŋktȟuŋwaŋ} [(i.ˈhã).(ktˣũ.ˌwã)] `Yankton' (lit. `those who dwell on the edge') $\to$\\
	\textit{Ihą́tu} [(i.ˈhã).tu] `Yankton'\footnote{Like the Arikara borrowing, this word is shortened to three syllables. With the exception of the [wã] syllable, which would be realized as [mã], this word is largely pronounceable for Mandan speakers, so this simplification is unnecessary. The Mandanized version of this loanword also conforms to the pattern that Mandan only allows one underlying nasal per root, as the last syllable is an oral vowel, unlike the source word.}

\item\label{mandanborrowings3} Omaha to Mandan \citep{larson2005}

\textit{Umóⁿhoⁿ} [(u.ˈmã).hã] `upstream, Omaha' $\to$\\
	\textit{Ómahą} [(ˈo).(mã.ˌhã)] `Omaha'\footnote{The \textit{u-} in Omaha is a locative preverb that is a cognate with the locative preverb \textit{o-} in Mandan. In Omaha, this preverb does not block iambic footing, which is why we see second syllable stress in Omaha. However, Mandan speakers analyzed it as a locative preverb, which caused the word to have first-syllable stress like other Mandan stems with preverbs. A reflex of the PSi word *wąhą `upstream, upwind' is not present in Mandan otherwise.}

\item\label{mandanborrowings4} French to Mandan \citep[4]{littleowl1992}

\textit{espagnol} [ɛs.(pa.ˈɲɔl)] `Spanish' $\to$\\
	\textit{íspari'oori} [(ˈi).(spa.ˌɾi).(ˌʔoː).ɾi] `Mexican'\footnote{The initial /i/ in this word is taken to be the instrumental preverb \textit{i}-, since the vast majority of Mandan words with word-initial /i/ involve that preverb. Analyzing the initial /i/ as a preverb causes first syllable stress instead of the expected second syllable iamb due to the internal word boundary found after preverbs in Mandan. There are at least three versions of this word in use. This version comes from \citet[4]{littleowl1992}, but \citet[95]{hollow1970} gives [ʃ] for the fricative, i.e., [(ˈi).(ʃpa.ˌɾi).(ˌʔoː).ɾi]. Given the fact that French /s/ becomes Mandan /ʃ/ in this word, it is possible that this loan entered Mandan from French via Hidatsa, where there is no /s/, only /ʃ/. The word for `Mexican' in contemporary Hidatsa is [(i.ˈʃpɑ).(ɾi.ˌʔoː).ɾɪ], which is often shortened to just [(i.ˈʃpɑ).ɾɪ]. The truncated Hidatsa version is also found in Mandan. The version of this word in Mandan with the [ʃ] is likely contamination from Hidatsa, since virtually all L1 speakers of Mandan over the past century have also been speakers of Hidatsa, which has been used more widely than Mandan throughout the Fort Berthold Reservation since the construction of the Garisson Dam.}

\newpage
\item\label{mandanborrowings5} Hidatsa to Mandan \citep[193]{matthews1877}

\textit{miridáari} [(mɪ.ˌɾi).(ˈtɑː).ɾɪ] `cross over water' $\to$\\
	\textit{Minítaari} [(ˈm\textsuperscript{ĩ}nĩ).(ˈtaː).ɾi] `Hidatsa'\footnote{The original Hidatsa word is a compound of \textit{mirí} `water' and \textit{dáari} `cross a river', and has anapestic stress due to syllable weight outranking a left-aligned iamb in Hidatsa stress assignment \citep{boyleetal2016}. The majority of Mandan speakers over the past century have also been Hidatsa speakers and have therefore recognized this word as a compound. This word is treated as a compound in Mandan, which prevents normal iambic footing because of a word boundary after \textit{miní} `water'. If this word were not analyzed as a compound, we would otherwise expect stress on the second syllable [taː].}

\end{xlist}

\end{exe}

All of the words in (\ref{mandanborrowings}) are consistent with Mandan primary stress assignment. In most of these words, we see primary stress on a different syllable than the original word, so we can assume that stress is not lexical and is predictable. Furthermore, several of these words are reanalyzed as having complex internal morphology, which causes a shift in primary stress. We see this behavior in `Omaha' and `Mexican', where the initial vowel is treated like a preverb. Similarly, the compound in `Hidatsa' is borrowed as a compound rather than a simplex word.

\subsubsection{Summary of stress assignment}

The overall pattern we see as we go through the corpus is that primary stress is predictable if the underlying morphology is known. Mandan utilizes a very regular stress accent as a system of word-level prominence rather than a pitch accent system.\footnote{The only Siouan language that demonstrably has a pitch accent system is Crow, where there is a single mora per root that bears an underlying high tone that spreads leftward to create a high pitch plateau \citep{graczyk2007,wallace1993}. \citet{boyle2007} and \citet{park2012} argue that Hidatsa also has a pitch accent system similar to that in Crow, but \citet{boyleetal2016} finds that Hidatsa has a very predictable stress system using phonetic instrumentation that is not dissimilar from the one in Mandan.} Stress in Mandan involves left-aligned, weight-sensitive iambs. Primary stress falls on the leftmost iamb, and secondary accent falls on each subsequent iamb.

Heavy syllables attract stress, with all syllables containing a long vowel or a coda /ʔ/ being treated as heavy. All other syllable types are light. A word that begins with two light syllables will have the primary stress fall on the second syllable in keeping with the iambic stress pattern of Mandan. Initial heavy syllables attract primary stress due to the fact that they are bimoraic and therefore satisfy the requirement for an iambic foot for primary stress. Dorsey's Law vowels are excrescent and do not factor into syllabification due to the fact that they are not visible to the phonology. This extraphonological status is why such vowels are represented as superscript vowels throughout this book.

Exceptions to this pattern are found in words with complex internal structures, i.e., compounds and composites. Footing does not cross word boundaries. A compound like \textit{súknuma'k} `young man' has primary stress on the first syllable instead of the expected second syllable because there is a word boundary after \textit{súk} `child, young.' Iambic footing cannot cross a word boundary, so it is preferable to have a deficient iamb than to cross a boundary, resulting in primary stress on the first syllable.

A composite word whose initial element has only a light first syllable (e.g., one with a preverb like \textit{i-} or  \textit{o}-) likewise can have that first syllable bear primary stress in spite of the presence of a second syllable that would otherwise attract primary stress. A word like \textit{íseko'sh} `he did it' contains the instrumental preverb \textit{i}- and the root \textit{sek} `do, make', plus the male-addressee indicative enclitic =\textit{o'sh}. Preverbs act as internal word boundaries for the purpose of stress assignment, so we again must have first-syllable stress despite the fact that there are two light syllables at the left edge of the word. See \citet{kasak2019} for further argumentation about the theoretical underpinnings of this analysis.\footnote{It is not clear how productive this process of preverbs blocking primary stress assignment is throughout the Siouan language family. Further study of prosodic processes within other Siouan languages if needed to determine whether Mandan is exceptional in this respect, or if Mandan is one of several Siouan languages that are sensitive to the boundary between preverbs and the rest of the verbal complex. This research question lies outside the scope of the present work.}

\section{Sound symbolism}\label{soundsymbolism}

Mandan is like most other Siouan languages in that it features sound symbolism \citep[107]{parksrankin2001}. Sound symbolism involves changing the supralaryngeal fricatives in a word to express less or greater intensity of an action or of the quality of some state. This iconicity is lexically determined and is not a productive part of synchronic Mandan. Sound symbolism was a feature of Proto-Siouan, where many lexical and morphological items can be reconstructed with doublets or triplets, differing in which of the three Proto-Siouan supralaryngeal fricatives *s *š *x is present in the reconstructed formative.


These same three fricatives in Mandan can be found with the \textit{s}-grade being associated with small objects or limited, precise actions, bright or intense colors, and quiet or high-pitched noises. On the opposite end of the spectrum is the \textit{x}-grade, which is typically associated with large objects or extreme, imprecise actions, dark or dull colors, and loud noises. The \textit{sh}-grade is considered to be a middle quality in cases where forms for all three fricatives exist. However, most cases of sound symbolism involve doublets rather than full triplets. We can see some examples of sound symbolism   in \tabref{soundsymbolismtable}.



Sound symbolism is typically seen on verbs. In particular, these verbs tend to be stative, since they are used to describe something that has a gradable quality. There are active verbs that display sound symbolism, but they are a minority of cases. The examples in \tabref{soundsymbolismtable} are not all-inclusive of all instances of sound symbolism in Mandan, as there are numerous doublets and triplets in the language. Where a gap in the paradigm exists, one cannot simply offer a hypothetical form, as speakers consider those to be ill-formed. This consistent judgment against proposing a missing form in the case of a doublet reinforces the fact that sound symbolism is not a productive feature of contemporary Mandan. Furthermore, the lack of L1 speakers prevents us from exhaustively investigating all Mandan roots involving the /s/, /ʃ/, and /x/ sounds possible sound symbolism pairs or triplets that do not appear in the corpus.





Most Mandan roots are monosyllabic, and there are numerous homophones. This homophony may be obscuring more cases of sound symbolism. Another factor in preventing more cases of sound symbolism from being identified is the fact that there may be semantic change involved where there is some metaphorical connection between one form and a possible doublet or triplet, e.g., the root \textit{siip} of \textit{rusíip}  `blink' and \textit{xíip} `wrinkled.'

\begin{table}
\caption{Examples of sound symbolism in Mandan}\label{soundsymbolismtable}
\begin{tabular}{lll}
\lsptoprule
\textit{\textbf{s}}&\textit{\textbf{sh}}&\textit{\textbf{x}}\\
\midrule
\textit{pų́ųs}&\textit{pų́ųsh}&\textit{pų́ųx}\\
`striped'&`spotted'&`dappled'\\
\tablevspace
\textit{síi}&\textit{shíi}&\textit{xíi}\\
`yellow'&`tawny'&`brown'\\
\tablevspace
\textit{ské}&\textit{shké}&\textit{xké}\\
`tie, twist, braid'&`weave, twine'&`pluck'\\
\tablevspace
\textit{ksíp}&\textit{kshíp}&---\\
`go under water'&`drown, go under water'&~\\
\tablevspace
\textit{sé}&\textit{shé}&---\\
`red'&`pink'&~\\
\tablevspace
\textit{síh}&\textit{shíh}&---\\
`keen'&`sharp'&~\\
\tablevspace
---&\textit{ná'resh}&\textit{ná'rex}\\
~&`hot'&`lukewarm'\\
\tablevspace
---&\textit{púushak}&\textit{púuxak}\\
~&`coarse sand'&`fine sand'\\
\tablevspace
---&\textit{shóot}&\textit{xóot}\\
~&`white'&`gray'\\
\tablevspace
\textit{hą́s}&\textit{hą́sh}&~\\
`bullberry'&`grape'&~\\
\tablevspace
\textit{rusáp}&---&\textit{ruxáp}\\
`pull, tug on'&~&`tear off'\\
\tablevspace
\textit{sé'h}&---&\textit{xé'h}\\
`leak, drip'&~&`be raining'\\
\tablevspace
\textit{seróo}&---&\textit{xeróo}\\
`jingle'&~&`rattle'\\
\lspbottomrule
\end{tabular}
\end{table}


Almost all instances of sound symbolism in Mandan involve a single fricative changing to reflect some iconic relationship between the size, intensity, or degree of a quality or action. A small number of words in Mandan that feature disyllabic roots with the same fricative display sound symbolism. \textit{Są́ąsi} `smooth [like ice]' and  \textit{shą́ąshi} `smooth [like skin]' both involve the same fricative, where a change in its quality indicates a change in the degree of smoothness. The antonyms \textit{sasáp} `rough [like a file]'  and \textit{xaxáp}  `rough [like sandpaper]' likewise involve a complete change of all fricatives to indicate a change in the degree of roughness. It is not clear if these instances of double sound symbolism arose through some diachronic process of compounding or accretion of other morphology or if these etyma are inherited from Proto-Siouan. The Comparative Siouan Dictionary only has two lemmata	 with double sound symbolism: *SiSi- `bend' and *SuuSE `crush' \citep{rankin2015}. Further study of sound symbolism throughout the Siouan language family is needed to investigate how widespread this double sound symbolism is, given the fact that the two Mandan words with double sound symbolism mentioned here do not have any identified cognates in other Siouan languages.
