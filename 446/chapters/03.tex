\chapter{Verbal morphology}\label{chapter3}

This chapter addresses the issue of Mandan verbal morphology. Specifically, this chapter delves into the distribution of inflectional affixes and other agreement morphology, as well as the preponderance of postverbal enclitics possible in Mandan. Derivational affixes are also discussed here, as is cursory overview of the syntax of Mandan and the structure of the Mandan clause.

Much of the extant literature on Mandan deals with topics presented herein, as most previous scholars have been primarily concerned with glossing narratives \citep{kennard1936,hollow1970,coberly1979,mixco1997a}. Very little has been done to investigate issues in the behavior of verbs intra- and inter-clausally, though \citet{mixco1997b} argues that many of the participial markers described by \citet{kennard1936} and \citet{hollow1970} are really switch-reference markers. Previous explanations of Mandan have also described varying amounts of tense morphology, which herein is interpreted as being evidentials. I argue that Mandan, like most Siouan languages, lacks dedicated tense morphology, and that such formatives are truly aspectuals or evidentials.

This chapter attempts to document all morphology present in the corpus, explain its usage, and provide examples that might be used in the future study of Mandan or in certain phenomena present in Mandan. This chapter breaks down the distribution and behavior of the affixes observed in Mandan into three divisions: the prefix field in \sectref{SecPrefixField}, the suffix field in \sectref{SecSuffixField}, and phrasal morphology (i.e., enclitics) in \sectref{SecPhrasalMorphology}. I argue that much of the post-verbal morphology in Mandan is actually implemented by enclitics and not true suffixes. I conclude this chapter by noting that the ordering of enclitics can be changed in certain contexts that reflect the scopal relationships intended by the speaker. This limited degree of flexibility in morphological order is restricted to enclitics and is not reflected in the ordering of the prefix field.

\section{Prefix field}\label{SecPrefixField}

Mandan has a rigid ordering for its prefixes. This kind of morphology is often referred to as templatic morphology, since morphological items appear in specific ordering with respect to one another. \tabref{prefixfieldmandanredux} below represents the order in which we see prefixes ordered in Mandan \citep[8]{kasak2019}.

\begin{table}
\caption{Prefix field in Mandan}\label{prefixfieldmandanredux}

\fittable{\scshape
    \begin{tabular}{llllllllllll}
    \lsptoprule
    11  & 10  & 9      & 8    & 7   & 6          & 5      	& 4		& 3			& 2     & 1     & 0    \\
\midrule
    rel & neg & unsp   & 1pl & pv.irr & pv.loc   & 1sg 		& 2sg		& suus		& iter  & ins & stem \\
    ~   & ~   & ~ & ~    & ~   & pv.ins 	 & ~      	& 2pl		& mid			& incp   & ~     & ~    \\
    ~   & ~   & ~      & ~    & ~   & pv.tr 	 & ~      	& ~		& recp		& ~  & ~     & ~    \\
\lspbottomrule
    \end{tabular}}
\end{table}

The prefix field in Mandan alternates between inflectional and derivational morphology, i.e., we see affix orderings like \textsc{derivation-inflection-stem}. Slots 1 (see \sectref{SubsubsecInstrumentalPrefixes}), 2 (see \sectref{SubsubsecAspectualPrefixes}), and 3 (see \sectref{SubsubsecVoicePrefixes}) are all voice or aspectual markers, with slots 6 (see \sectref{ParaApplicatives}) and 7 (see \sectref{ParaIrrealis}) being mood or applicative preverbs. Person marking takes place in slots 4 (see \sectref{SubsubSecondPerson}), 5 (see \sectref{SubsubFirstPersonSingular}), and 8 (see \sectref{SubsubFirstPersonPlural}), while an unspecified argument marker appears in slot 9 (see \sectref{SubsubsUnspecifiedArgument}). Negation is marked in slot 10 (see \sectref{SubsubsecNegative}), and relativization appears in slot 11 (see \sectref{SubsubsecRelativized}).

In this section, I outline the prefix inventory of Mandan, starting with those prefixes that appear closest to the verbal stem, addressing each prefix by which templatic slot it is associated.

\subsection{Derivational prefixes}\label{SubsecDerivationalPrefixes}

The bulk of prefixal slots is occupied by derivational material.\footnote{My assumptions regarding why these prefixes are derivational have been elaborated upon more fully in \citet[5\textit{ff}]{kasak2019}.} The majority of these derivational affixes affect the valency of the verb, e.g., the prefix \textit{ru-} \textsc{ins.hand} indicates that an action is being done using the agent's hand, where the instrument is only covertly present in the sentence, not overtly.

\subsubsection{Instrumental prefixes (Slot 1)}\label{SubsubsecInstrumentalPrefixes}

Mandan, like every other Siouan language, has a number of instrumental prefixes. These prefixes indicate the manner by which an action occurs when added to a transitive verb. When an instrumental occurs with an intransitive verb, that verb becomes transitive. The tendency to transitivize intransitive verbs, however, is not universal. There are numerous instances of an intransitive verb bearing instrumental morphology and still preserving their status as intransitive verbs. Instrumental prefixes are lexically determined, and are not productive in modern Mandan. A list of instrumental prefixes in Mandan appears in (\ref{instrumentalprefixes}) below.

\begin{exe}
\item\label{instrumentalprefixes} Mandan instrumental prefixes

    \begin{xlist}
	\item	\textit{ka-}	`by force'
	\item 	\textit{pa-}	`by pushing'
	\item	\textit{ra-}\textsubscript{1}	`by foot'
	\item	\textit{ra-}\textsubscript{2}	`by mouth'
	\item	\textit{ra'-}	`by heat'
	\item	\textit{ru-}	`by hand'
	\item	\textit{wa'-}	`by piercing'
    \end{xlist}
\end{exe}

All seven of these instrumentals are inherited from Proto-Siouan, as cognates of these instrumentals can also be found in most other Siouan languages. Examples and semantic peculiarities of each instrumental follow below.% and a list of attested forms with instrumental morphology appears in Appendix \ref{appendixA}.

\subsubsubsection{`By force' instrumental: \textit{ka-}}\label{ParaByForce} 

This instrumental prefix often accompanies verbs that deal with cutting or striking actions. Actions bearing \textit{ka}- often involve some kind of sudden movement. This prefix is a reflex of Proto-Siouan *raka- \textsc{ins.frce}, and typically manifests as \textit{ka-}. However, when combined with a reflexive marker, the reflexive or \textit{suus} marker \textit{ki}- plus \textit{ka}- become \textit{kara}- /k-ra-/, where \textit{ka}- has an allomorph \textit{ra}-. This relic trace of the fuller *raka- formative is likewise found in Dakotan and Dhegihan languages, and these languages likewise form a portmanteau of the reflexive plus force instrumental like Mandan \citep{rankin2015}. Examples of the `by force' instrumental appear in (\ref{kaexamples}) below.

\begin{exe}
\item\label{kaexamples} Examples of \textit{ka-}

	\begin{xlist}
	
	\item \glll \textbf{ka}háaro'sh\\
	\textbf{ka}-haa=o'sh\\
	\textbf{ins.frce}-\textnormal{be.separated}=ind.m\\
	\glt `he cuts meat from the bone' \citep[68]{hollow1970}
	
	\item \glll í\textbf{ka}hįto'sh\\
	i-\textbf{ka}-hįt=o'sh\\
	pv.ins-\textbf{ins.frce}-\textnormal{soften.hide}=ind.m\\
	\glt `she softens a hide by beating it' \citep[77]{hollow1970}
	
	\item \glll \textbf{ka}hóora ráahini\\
	\textbf{ka}-hoo=E rEEh=rį\\
	\textbf{ins.frce}-\textnormal{fall.over} \textnormal{go.there}=ss\\
	\glt `it was falling down and...' \citep[1]{hollow1973a}
	
	\item \glll karó\textbf{ka}hashka\\
	ka-ro-\textbf{ka}-hash=ka\\
	agt-1s.pl-\textbf{ins.frce}-\textnormal{be.distintegrated}=hab\\
	\glt `the one who slaughters us' \citep[146]{hollow1973a}
	
	\item \glll ra\textbf{ká}shihe\\
	ra-\textbf{ka}-shih=E\\
	2a-\textbf{ins.frce}-\textnormal{be.sharp}=sv\\
	\glt `you sharpen it' \citep[189]{hollow1973a}

	\item \glll ówa\textbf{ka}ptihki\\
	o-wa-\textbf{ka}-ptik=ki\\
	pv.irr-1a-\textbf{ins.frce}-\textnormal{fall.down}=cond\\
	\glt `if I knock it down' \citep[65]{hollow1973a}

	\item \glll \textbf{kará}xkąho'sh\\
	k-\textbf{ra}-xkąh=o'sh\\
	suus-\textbf{ins.frce}-\textnormal{move}=ind.m\\
	\glt `he shook something of his own' \citep[448]{hollow1970}
	
	\end{xlist}

\end{exe}

\subsubsubsection{`By pushing' instrumental: \textit{pa-}}\label{ParaByPushing}

This instrumental typically implies a pushing-type action, frequently involving the movement of a cutting instrument. The use of \textit{pa-} versus \textit{ka-} when used for cutting motions indicates the difference in intensity of the cutting motion, i.e., butchering a carcass versus mincing food. \citet[461]{hollow1970} notes that this instrumental is often associated with motions away from the body. This formative is a reflex of Proto-Siouan *pa- \textsc{ins.push}. Examples of the `by pushing' instrumental appear in (\ref{paexamples}) below.

\begin{exe}
\item\label{paexamples} Examples of \textit{pa-}

	\begin{xlist}
	
	\item \glll ówa\textbf{pa}weshto'sh\\
	o-wa-\textbf{pa}-wesh=kt=o'sh\\
	pv.irr-1a-\textbf{ins.push}-\textnormal{cut}=pot=ind.m\\
	\glt `I might be going to cut it' \citep[454]{hollow1970}

	\item \glll nu\textbf{pá}minishinito'sh\\
	rų-\textbf{pa}-wrįsh=rįt=o'sh\\
	1a.pl-\textbf{ins.push}-\textnormal{be.folded}=2pl=ind.m\\
	\glt `we (pl.) rolled it up' \citep[462]{hollow1970}
	
	\item \glll ó\textbf{pa}xiruukini\\
	o-\textbf{pa}-xruuk=rį\\
	pv.loc-\textbf{ins.push}-\textnormal{take.off}=ss\\
	\glt `she took it off [the wall]' \citep[91]{hollow1973a}


	\item \glll ó\textbf{pa}ptiktiki\\
	o-\textbf{pa}-ptik=kti=ki\\
	pv.loc-\textbf{ins.push}-\textnormal{have.fallen.down}=pot=cond\\
	\glt `whenever he pushes them' \citep[105]{hollow1973a}

	\item \glll miní wa\textbf{pá}shų'ro'sh\\
	wrį wa-\textbf{pa}-shų'=o'sh\\
	\textnormal{water} 1a-\textbf{ins.push}-\textnormal{thresh}=ind.m\\
	\glt `I swim' \citep[241]{hollow1970}
	
	\item \glll Máatah Ó\textbf{pa}tarak\\
	wąątah o-\textbf{pa}-trak\\
	\textnormal{Missouri.River} pv.loc-\textbf{ins.push}-\textnormal{block}\\
	\glt `Garrison Dam' \citep[256]{hollow1970}
	
	\item \glll \textbf{pa}są́ąsi'sh\\
	\textbf{pa}-sąąsi=o'sh\\
	\textbf{ins.push}-\textnormal{be.smooth}=ind.m\\
	\glt `he polishes it' \citep[461]{hollow1970}
		
	\end{xlist}

\end{exe}

\subsubsubsection{`By foot' instrumental: \textit{ra-}\textsubscript{1}}\label{ParaByFoot}

The instrumental \textit{ra-} is homophonous with the `by mouth' instrumental. The `by foot' instrumental is a reflex of the Proto-Siouan instrumental *rąą \textsc{ins.foot}.  The \textit{ra-} almost always indicates that an action is done using one or both feet, though there are some verbs where the understanding is that the action happened using the legs instead. The semantics of whether the feet are involved or the legs must be learned for each word, but the default reading is that of an action done by the use of one's foot or feet.

This instrumental is nasalized in Core Siouan (i.e., Mississippi Valley and Ohio Valley Siouan), but not nasalized in Peripheral Siouan (i.e., Missouri Valley, Mandan, and Catawban). Catawba uses a prosodically reduced form of the verb \textit{daa'} `go by foot' \citep{rankin2015}. The length of the vowel and its orality could be influenced from the Hidatsa and Crow analogs, \textit{ara-} and \textit{ala-}, respectively. There could also have been some contamination with the oral vowel in `go' \textit{réeh}-/\textit{ráah}- that caused the vowel in the instrumental to oralize. Nonetheless, this instrumental is distinct from its homophonous counterpart \textit{ra-} \textsc{ins.mth}, as it is lexically selected for by certain verbs. We can see examples of the `by foot' instrumental in (\ref{ra1examples}) below.

\begin{exe}
\item\label{ra1examples} Examples of \textit{ra}- `by foot'

	\begin{xlist}
	
	\item \glll \textbf{ra}kéxo'sh\\
	\textbf{ra}-kEx=o'sh\\
	\textbf{ins.foot}-\textnormal{scrape}=ind.m\\
	\glt `he scrapes it [with his feet]' \citep[107]{hollow1970}

	\item \glll óra\textbf{ra}xih írakų'kto'sh\\
	o-ra-\textbf{ra}-xih i-ra-kų'=kt=o'sh\\
	pv.irr-2a-\textbf{ins.foot}-\textnormal{kick} pv.ins-2a-\textnormal{give}=pot=ind.m\\
	\glt `you'll pretend to kick' \citep[62]{hollow1973b}
	
	\item \glll \textbf{ra}pı̨́'xo'sh\\
	\textbf{ra}-pį'x=o'sh\\
	\textbf{ins.foot}-\textnormal{be.scattered}=ind.m\\
	\glt `he scatters it [with his foot]' \citep[146]{hollow1970}

	\item \glll \textbf{ra}shų́'ro'sh\\
	\textbf{ra}-pa-shų'=o'sh\\
	\textbf{ins.foot}-\textnormal{thresh}=ind.m\\
	\glt `he threshes corn with his feet' \citep[221]{hollow1970}

	\item \glll wa\textbf{rá}pįįto'sh\\
	wa-\textbf{ra}-pįįt=o'sh\\
	1a-\textbf{ins.foot}-\textnormal{scatter}=ind.m\\
	\glt `I scatter it with my foot' \citep[464]{hollow1970}

	\item \glll ki\textbf{rá}siruutoomako'sh\\
	ki-\textbf{ra}-sruut=oowąk=o'sh\\
	itr-\textbf{ins.foot}-\textnormal{be.slippery}=narr=ind.m\\
	\glt 	`he slipped again' \citep[11]{kennard1936}

	\item \glll \textbf{ra}shóho'sh\\
	\textbf{ra}-shoh=o'sh\\
	\textbf{ins.foot}-\textnormal{be.pointed}=ind.m\\
	\glt `he stretches his legs' \citep[232]{hollow1970}
	
	\end{xlist}

\end{exe}

\subsubsubsection{`By mouth' instrumental: \textit{ra-}\textsubscript{2}}\label{ParaByMouth}

The `by mouth' instrumental \textit{ra-} is a reflex of Proto-Siouan *ra- \textsc{ins.mth}. Actions involving an agent's lips, mouth, teeth, or tongue will often display this instrumental. Certain stems may involve either \textit{ra-} \textsc{ins.foot} or \textit{ra}- \textsc{ins.mth}, making these verbs homophonous, e.g., the verb \textit{rashkápo'sh} can mean either `he pinches it between his toes' or `he nibbles on it.' Context is clearly the tiebreaker in such instances, and in isolation, it is impossible to conclusively tell which \textit{ra}- the speaker intends. The only possible clue might be frequency, as the \textit{ra}- `by foot' instrumental appears more often in the corpus than the \textit{ra}- `by mouth.' Examples of \textit{ra}- \textsc{ins.mth} appear in (\ref{ra2examples}) below.

\begin{exe}
\item\label{ra2examples} Examples of \textit{ra}- `by mouth'

	\begin{xlist}
	
	\item \glll \textbf{ra}hópo'sh\\
	\textbf{ra}-hop=o'sh\\
	\textbf{ins.mth}-\textnormal{be.hollow}=ind.m\\
	\glt `he nibbles a hole' \citep[77]{hollow1970}

	\item \glll \textbf{ra}káxo'sh\\
	\textbf{ra}-kax=o'sh\\
	\textbf{ins.mth}-\textnormal{eat.corn.from.cob}=ind.m\\
	\glt	`he eats corn from the cob' \citep[104]{hollow1970}

	\item \glll ó\textbf{ra}tke mikó'sh\\
	o-\textbf{ra}-tke wįk=o'sh\\
	pv.irr-\textbf{ins.mth}-\textnormal{touch} \textnormal{be.none}=ind.m\\
	\glt `it has no flavor' \citep[138]{hollow1973b}

	\item \glll wa\textbf{rá}xkiho'sh\\
	wa-\textbf{ra}-xkih=o'sh\\
	unsp-\textbf{ins.mth}-\textnormal{be.cracked}=ind.m\\
	\glt `he cracks something between his teeth' \citep[465]{hollow1970}

	\item \glll í\textbf{ra}xąko'sh\\
	i-\textbf{ra}-xąk=o'sh\\
	pv.ins-\textbf{ins.mth}-\textnormal{be.torn}=ind.m\\
	\glt `he tears it open with his teeth' \citep[309]{hollow1970}
	
	\item \glll wa\textbf{rá}xtuxte'sh\\
	wa-\textbf{ra}-xtuxte=o'sh\\
	1a-\textbf{ins.mth}-\textnormal{chew}=ind.m\\
	\glt `I chew it' \citep[330]{hollow1970}

	\item \glll ra\textbf{rá}'uuxo'sha?\\
	ra-\textbf{ra}-uux=o'sha\\
	2a-\textbf{ins.mth}-\textnormal{be.broken}=int.m\\
	\glt `are you going to break it between your teeth?' \citep[465]{hollow1970}
	
	\end{xlist}


\end{exe}

\subsubsubsection{`By heat' instrumental: \textit{ra'-}}\label{ParaByHeat}

This instrumental has at times been confused for one of the \textit{ra}- instrumentals due to the fact that many scholars did not hear the coda glottal stop, as well as the fact that many speakers tend not to realize the glottal stop with a full glottal occlusion. The use of creaky voice is often the only clue that this is the `by heat' instrumental in many tokens in the corpus. This instrumental is a reflex of Proto-Siouan *aRaa \textsc{ins.temp}, which denotes an action taken using either extreme heat and extreme cold. In some Siouan languages, this instrumental is still used with extreme cold \citep{rankin2015}, but this is not the case in Mandan. The \textit{ra'-} prefix can only be used with actions or states involving heat. Examples of the `by heat' instrumental appear in (\ref{ra'examples}) below.

\begin{exe}
\item\label{ra'examples} Examples of \textit{ra'}-

	\begin{xlist}
	
	\item \glll ka\textbf{rá'}ptewaherekto'sh\\
	ka-\textbf{ra'}-ptE\#wa-hrE=kt=o'sh\\
	incp-\textbf{ins.heat}-\textnormal{be.burning}\#1a-caus=pot=ind.m\\
	\glt `I will cause it to start to burn' \citep[47]{hollow1973b}
	
	\item \glll \textbf{rá'}sako'sh\\
	\textbf{ra'}-sak=o'sh\\
	\textbf{ins.heat}-\textnormal{be.dry}=ind.m\\
	\glt `it dried up in the fire' \citep[198]{hollow1970}
	
	\item \glll ma\textbf{rá'}resho'sh\\
	wą\textbf{-ra}'-resh=o'sh\\
	1s-\textbf{ins.heat}-\textnormal{be.hot}=ind.m\\
	\glt `I am hot' \citep[463]{hollow1970}
	
	\item \glll \textbf{rá'}xuunuhere'sh\\
	\textbf{ra'}-xuu\#rų-hrE=o'sh\\
	\textbf{ins.heat}-\textnormal{be.charred}\#1a.pl-caus=ind.m\\
	\glt `we charred it and made it brittle' \citep[330]{hollow1970}
	
	\item \glll \textbf{rá'}xerephere\\
	\textbf{ra'}-xrep\#hrE\\
	\textbf{ins.heat}-\textnormal{scab}\#caus\\
	\glt `boil off the bone' \citep[12]{kennard1936}
	
	\item \glll mí's \textbf{rá'}seso'nik\\
	wį'=s \textbf{ra}'-sE=so'rįk\\
	\textnormal{stone}=def \textbf{ins.heat}-\textnormal{be.red}=comp.caus\\
	\glt `since the rock was red hot' \citep[97]{hollow1973a}
	
	\item \glll rúut \textbf{rá'}sitwahara minikú'nito'sh\\
	ruut \textbf{ra'}-sit\#wa-hrE w-rį-ku'=rįt=o'sh\\
	\textnormal{rib} \textbf{ins.heat}-\textnormal{roast}\#1a-caus 1a-2s-\textnormal{give}=2pl=ind.m\\
	\glt `I roasted the ribs for you (pl.)' \citep[177]{hollow1973a} 
	
	\end{xlist}

\end{exe}


\subsubsubsection{`By hand' instrumental: \textit{ru-}}\label{ParaByHand}

One of the most commonly encountered instrumentals is \textit{ru}-, indicating that an action is taking place using the agent's hands. It is a reflex of Proto-Siouan *ru- \textsc{ins.hand}, which has cognates in every Siouan language. We see examples of the `by hand' instrumental in the data in (\ref{ruexamples}) below.

\begin{exe}
\item\label{ruexamples} Examples of \textit{ru-}

	\begin{xlist}
	
	\item \glll í\textbf{ru}xąko'sh\\
	i-\textbf{ru}-xąk=o'sh\\
	pv.ins-\textbf{ins.hand}-\textnormal{be.torn}=ind.m\\
	\glt `he tears it open [with his hands]' \citep[309]{hollow1970}
	
	\item \glll nu\textbf{rú}sįko'sh\\
	rų-\textbf{ru}-sįk=o'sh\\
	1a.pl-\textbf{ins.hand}-\textnormal{be.squeezed}=ind.m\\
	\glt `we squeezed/choked something' \citep[206]{hollow1970}
	
	\item \glll \textbf{ru}h\'{ı̨}įto'sh\\
	\textbf{ru}-hįįt=o'sh\\
	\textbf{ins.hand}-\textnormal{tan.hide}=ind.m\\
	\glt 	`she tans a hide' \citep[75]{hollow1970}
		
	\item \glll wáa'owa\textbf{ru}shaaxi're\\
	waa-o-wa-\textbf{ru}-shE=xi=o're\\
	neg-pv.irr-1a-\textbf{ins.hand}-\textnormal{take}=neg=ind.f\\
	\glt `I won't take it' \citep[131]{hollow1973a}
	
	\item \glll \textbf{ru}sé'rak\\
	\textbf{ru}-se'=ak\\
	\textbf{ins.hand}-\textnormal{open}=ds\\
	\glt `having opened it' \citep[137]{hollow1973a}
	
	\item \glll waa'i\textbf{ru}ptini\\
	waa-i-\textbf{ru}-pti=rį\\
	nom-pv.ins-\textbf{ins.hand}-\textnormal{carry}=ss\\
	\glt `a pot holder and...' \citep[149]{hollow1973a}
	
	\item \glll ą́ąwe óshiriihaa ~ ~ ~ ~ ~ ~ ~ ~ ~ ~ ~ ~ ~ ~ ~ ~ ~ ~ ~ ~ ~ ~ ~ ~ íki\textbf{ru}xkekerekaroomako'sh\\
	ąąwe o-shriih=haa ~ ~ ~ ~ ~ ~ ~ ~ ~ ~ ~ ~ ~ ~ ~ ~ ~ ~ ~ ~ ~ ~ ~ ~ i-k\textbf{i-}ru-xke=krE=ka=oowąk=o'sh\\
	\textnormal{all} pv.loc-\textnormal{be.scattered}=sim ~ ~ ~ ~ ~ ~ ~ ~ ~ ~ ~ ~ ~ ~ ~ ~ ~ ~ ~ ~ ~ ~ ~ ~ pv.ins-vert-\textbf{ins.hand}-\textnormal{pluck}=3pl=hab=narr=ind.m\\
	\glt `they would all pull [their heads] back while scattering' \citep[45]{hollow1973a}
	
	\end{xlist}

\end{exe}

In the overwhelming majority of cases, words bearing \textit{ru-} transparently involve the semantics of an action that takes place by way of an agent's hand. However, there are numerous cases where there is no obvious connection to an action or state taking place by way of an agent's hand, sometimes not even having any kind of an agent. Some instrumentals have similar opaque connections between the meaning of the resulting lexical item and the manner of action they denote, but \textit{ru-} has a much higher number of these semantically non-transparent roots than any other instrumental. We can see examples of some of these non-transparent roots with \textit{ru}- in (\ref{ruexceptions}) below.

\begin{exe}
\item\label{ruexceptions} Non-transparent semantics for \textit{ru}-

	\begin{xlist}

	\item \glll \textbf{ru}siríxo'sh\\
	\textbf{ru}-srix=o'sh\\
	\textbf{ins.hand}-\textnormal{splash}=ind.m\\
	\glt `it splashes' \citep[218]{hollow1970}
	
	\item \glll \textbf{ru}kóho'sh\\
	\textbf{ru}-koh=o'sh\\
	\textbf{ins.hand}-\textnormal{be.vacant}=ind.m\\
	\glt `he moves sideways, makes a space' \citep[114]{hollow1970}
	
	\item \glll hą́sh í\textbf{ru}mini\\
	hąsh i\textbf{-r}u-wrį\\
	\textnormal{grape} pv.ins-\textbf{ins.hand}-\textnormal{be.twisted}\\
	\glt `grape vine' \citep[196]{trechter2012b}

	\item \glll \textbf{ru}minísho'sh\\
	\textbf{ru}-wrįsh=o'sh\\
	\textbf{ins.hand}-\textnormal{be.rolled.up}=ind.m\\
	\glt `he goes around in circles' \citep[305]{hollow1970}

	\item \glll óma\textbf{ru}xke'sh\\
	o-wą-\textbf{ru}-xke=o'sh\\
	pv.loc-1s-\textbf{ins.hand}-\textnormal{sink}=ind.m\\
	\glt `I am sinking' \citep[317]{hollow1970}

	\item \glll \textbf{ru}xóko'sh\\
	\textbf{ru}-xok=o'sh\\
	\textbf{ins.hand}-\textnormal{be.idle}=ind.m\\
	\glt `he forbids someone from working' \citep[319]{hollow1970}
	
	\item \glll ki\textbf{ru}xų́'ro'sh\\
	k-\textbf{ru}-xų'=o'sh\\
	mid-\textbf{ins.hand}-\textnormal{plow}=ind.m\\
	\glt `he frowns' \citep[331]{hollow1970}
	
	\end{xlist}

\end{exe}

It is possible that several of these items originate from some metaphorical usage (e.g., \textit{ruxók} involving someone holding up their hand to prohibit someone else from doing something), whereas for others it is less clear (e.g., \textit{rusiríx} describes something that a liquid does, not what someone is doing to the liquid, so there should be no agent involved).

\subsubsubsection{`By piercing' instrumental: \textit{wa'-}}\label{ParaByPiercing}

This instrumental is the least common of all seven instrumentals in Mandan. It is a reflex of Proto-Siouan *Wa- \textsc{ins.cut}, which is not restricted to just cutting actions in Mandan. Actions involving \textit{wa'-} all make use of some sharp and pointed object. We can see examples of the `by piercing' instrumental in (\ref{wa'examples}) below. 

\begin{exe}
\item\label{wa'examples} Examples of \textit{wa'}-

	\begin{xlist}
	
	\item \glll \textbf{wá'}hopo'sh\\
	\textbf{wa'}-hop=o'sh\\
	\textbf{ins.prce}-\textnormal{be.hollow}=ind.m\\
	\glt `he makes a hole with an awl' \citep[490]{hollow1970}
	
	\item \glll \textbf{wá'}skįh\\
	\textbf{wa'}-skįh\\
	\textbf{ins.prce}-\textnormal{cut.open}\\
	\glt `cut open' \citep[12]{kennard1936}
	
	\item \glll \textbf{wá'}tke'sh\\
	\textbf{wa'}-tkE=o'sh\\
	\textbf{ins.prce}-\textnormal{poke}=ind.m\\
	\glt `he scrapes hair from hide with pointed object' \citep[253]{hollow1970}

	\item \glll \textbf{wa'}úux\\
	\textbf{wa'}-uux\\
	\textbf{ins.prce}-\textnormal{be.broken}\\
	\glt `he breaks something with a pointed object' \citep[263]{hollow1970}
	
	\item \glll \textbf{wá'}pshako'sh\\
	\textbf{wa'}-pshak=o'sh\\
	\textbf{ins.prce}-\textnormal{be.ripped}=ind.m\\
	\glt `he cut through it, opened it with a point' \citep[153]{hollow1970}
	
	\item \glll íra\textbf{wa'}tereko'sh\\
	i-ra-\textbf{wa'}-trek=o'sh\\
	pv.ins-2a-\textbf{ins.prce}-\textnormal{sew}=ind.m\\
	\glt `you sew it' \citep[97]{hollow1970}
	
	\item \glll wáa'i\textbf{wa'}shkap\\
	waa-i-\textbf{wa'}-shkap\\
	nom-pv.ins-\textbf{ins.prce}-\textnormal{pinch}\\
	\glt `a pin' \citep[230]{hollow1970}
	
	\end{xlist}

\end{exe}

\subsubsection{Aspectual prefixes (Slot 2)}\label{SubsubsecAspectualPrefixes}

Aspect marking appears in the second prefixal slot in Mandan. Both iterativity and inceptivity can be expressed with the prefix \textit{ki-}. Furthermore, the prefix \textit{ki-} is homophonous with the middle voice marking prefix \textit{ki-}, which is described in \sectref{SubsubsecVoicePrefixes}. The iterative interpretation of \textit{ki}- is very productive in Mandan, and \citet[11]{kennard1936} notes that this formative is used in a way similar to the prefix `re-' in English. The same cannot be definitively said about the inceptive interpretation, which does not appear to be productive, and is only sparsely attested in the corpus.

\subsubsubsection{Iterative aspectual: \textit{ki-}\textsubscript{1}}

The iterative prefix has an allophone of \textit{k}- before stems beginning with oral sonorants as well as the instrumental \textit{pa-}. The iterative can convey the meaning of `once more' as well as `over and over again.' The most common interpretation of the iterative observed in the corpus is `once more.' \citet[29]{mixco1997a} notes that iterative marking can co-occur with the free adverb \textit{inák} `again'. Throughout the corpus, though, the most frequent manifestation of iterativity is expressed with the adverb \textit{inák} instead of derivationally on the verb itself. We can see both allomorphs of iterative \textit{ki}- in the examples in (\ref{iterativeexamples}) below.

\begin{exe}
\item\label{iterativeexamples} Examples of iterative \textit{ki-}

	\begin{xlist}
	
	\item \glll óti í\textbf{ki}sehkereroomako'sh\\
	o-ti i-\textbf{ki}-sEk=krE=oowąk=o'sh\\
	pv.loc-\textnormal{reside} pv.ins-itr-\textnormal{make}=3pl=narr=ind.m\\
	\glt `they fixed the house' [lit. `they re-made the house'] \citep[157]{hollow1973a}
	
	\item \glll wáa'o\textbf{ki}naataaxi\\
	waa-o-\textbf{ki}-rąątE=xi\\
	neg-pv.irr-itr-\textnormal{stand}.aux=neg\\
	\glt `he won't be getting up again' \citep[2]{hollow1973a}
	
	\item \glll wa\textbf{kí}kuuho'sh\\
	wa-\textbf{ki}-kuuh=o'sh\\
	1a-\textbf{iter}-\textnormal{come.back.here}=ind.m\\
	\glt `I came back again' \citep[450]{hollow1970}
	
	\item \glll \textbf{ke}répo'sh\\
	\textbf{k}-rep=o'sh\\
	\textbf{iter}-\textnormal{be.fat}=ind.m\\
	\glt `he is fat again' \citep[450]{hollow1970}
	
	\item \glll wa\textbf{ki}rú'uuxo'sh\\
	wa-\textbf{k}-ru-uux=o'sh\\
	1a-\textbf{iter}-ins.hand-\textnormal{be.broken}=ind.m\\
	\glt `I break it again' \citep[489]{hollow1970}
	
	\item \glll \textbf{ke}róoro'sh\\
	\textbf{k}-roo=o'sh\\
	\textbf{iter}-\textnormal{talk}=ind.m\\
	\glt `he is talking again' \citep[449]{hollow1970}
	
%	\item \glll pka'ux\\
%	k-pa-ux\\
%	itr-ins.frce-\textnormal{be.broken}\\
%	\glt `he is talking again' \citep[449]{hollow1970}
	
	\end{xlist}

\end{exe}

\citet[11]{kennard1936} points out that there are certain verbs for which \textit{ka-} is the iterative prefix, but only when prefixing onto a stem beginning with [ⁿd]. After going over \citeauthor{hollow1970}'s recordings, the \textit{ka-} in these instances turns out to not be [ka]. In reality, this is a situation where he is perceiving an intrusive vowel as [a$\sim$ə] due to the more centralized realization of Dorsey's Law vowels.\footnote{Refer back to \sectref{dorseyslaw} for further explanation of the tautosyllabic status of excrescent vowels in Mandan.} In the examples in (\ref{kennarditerative}) below, \citeauthor{kennard1936}'s transcription appears in angled brackets with the current Mandan orthography and morphological breakdown appearing beneath it.

\begin{exe}
\item\label{kennarditerative} Iterative \textit{ka-} in \citet{kennard1936}

\begin{xlist}

\item $\langle$karo$'$pxani$\rangle$\\
    \glll keropxáani\\
    k-ropxE=rį\\
    itr-\textnormal{enter}=ss\\
    \glt `he went in again'

\item $\langle$ta$'$menis karo$'$tkika'\textsuperscript{ehe}$\rangle$\\
    \glll taminís keróotkika'ehe\\
    ta-wrįs k-rootki=ka'ehe\\
    al-\textnormal{horse} itr-\textnormal{strike}=quot\\
    \glt `he struck his horse again'

\item $\langle$karo$'$ktike\textsuperscript{$\backprime$}reka'\textsuperscript{ehe}$\rangle$\\
    \glll keróotkikereka'ehe\\
    k-rookti=krE=ka'ehe\\
    itr-\textnormal{make.camp}=3pl=quot\\
    \glt `they camped again'

\item $\langle$kara$'$cikoc$\rangle$\\
    \glll karashíko'sh\\
    k-ra-shik=o'sh\\
    itr-ins.foot-\textnormal{knock.over}=ind.m\\
	\glt `he kicked it again'

\end{xlist}

\end{exe}

Each of the examples in (\ref{kennarditerative}) is really an underlying /k-/ that triggers a Dorsey's Law vowel before the sonorant-initial root, e.g., \textit{karashíko'sh} `he kicked it again' is [kᵃɾa.ˈʃi.koʔʃ], not *[ka.ɾa.ˈʃi.koʔʃ]. This allomorphy does not seem to be phonologically conditioned, given that it is triggered before stems beginning with /ɾ w/ and the instrumental \textit{pa}- but not other stems beginning with /p/, e.g., \textit{kipáxo'sh} `it is broken again' and not *\textit{kapáxo'sh}, but \textit{wapká'uuxo'sh} `I broke it again' and not *\textit{wakípa'uuxo'sh}.%\footnote{Refer back to \sectref{velarlabialmetathesis} for an explanation for why the /k-p/ sequence resulted in metathesis, i.e., [pk].}

\subsubsubsection{Inceptive aspectual: \textit{ka-} and \textit{ki-}\textsubscript{2}}\label{inceptiveprefix}

Inceptive aspect is marked very sparingly in the corpus. Virtually every instance of inceptivity involves a verb of motion, with most of the non-motion verbs being from \citeapos{hollow1973b} re-elicitation of \citeapos{kennard1934} narratives. It is possible that this prefix is less productive in the speech of Mandan speakers who were born around the turn of the twentieth century, but the lack of additional data relegates this explanation to the realm of conjecture. However, the fact remains that inceptive marking is more widely attested in narratives from speakers born in the middle of the nineteenth century than it is for speakers born at the turn of the twentieth century and onward.

While \citet{hollow1973a} documents numerous examples of inceptive \textit{ka-} in his transcribed narratives, it is not included in his list of morphology in the back of his dictionary \citep{hollow1970}. The inceptive aspect marker falls in the same slot as the iterative aspect marker, and as such, they cannot co-occur. It would be formally possible for multiple manifestations of iterativity to be present on a single verb (i.e., with the prefix \textit{ki-} and the enclitic \textit{=ske}), but no such redundant marking of inceptivity is documented in the corpus.

The narratives in \citet{hollow1973a} show that \textit{ka-} is the most frequent shape for the inceptive marker, but a few verbs take \textit{ki}- instead. This difference appears to be lexical, as there are no other transparent conditioning factors that might suggest a phonological motivation. We can see examples of both inceptive \textit{ka-} and \textit{ki-} in (\ref{inceptiveexamples}) and (\ref{inceptiveexamples2}), respectively, below.

\begin{exe}
\item\label{inceptiveexamples} Examples of inceptive \textit{ka-}

	\begin{xlist}
	
	\item \glll \textbf{ka}súkini\\
	\textbf{ka}-suk=rį\\
	\textbf{incp}-\textnormal{come.out}=ss\\
	\glt `he appeared and...' \citep[45]{hollow1973a}
	
	\item \glll \textbf{ka}síi ráahini\\
	\textbf{ka}-sii rEEh=rį\\
	\textbf{incp}-\textnormal{travel} \textnormal{go.there}=ss\\
	\glt `he went traveling and...' \citep[1]{hollow1973a}
	
	\item \glll \textbf{ka}rópxekereroomako'sh\\
	\textbf{ka}-ropxE=krE=oowąk=o'sh\\
	\textbf{incp}-\textnormal{go.in}=3pl=narr=ind.m\\
	\glt `they began to go in' \citep[174]{hollow1973b}
	
	\item \glll ą́ąwe \textbf{ka}róokereroomako'sh\\
	ąąwe \textbf{ka}-roo=krE=oowąk=o'sh\\
	\textnormal{all} \textbf{incp}-\textnormal{talk}=3pl=narr=ind.m\\
	\glt `everyone started talking' \citep[149]{hollow1973b}
	
	\item \glll \textbf{ka}ní'roomako'sh\\
	\textbf{ka}-rį'=oowąk=o'sh\\
	\textbf{incp}-\textnormal{climb}=narr=ind.m\\
	\glt `he started to climb ashore' \citep[270]{hollow1973b}
	
	\item \glll \textbf{ka}ráahaa\\
	\textbf{ka}-rEEh=haa\\
	\textbf{incp}-\textnormal{go.there}=sim\\
	\glt `she started to go' \citep[275]{hollow1973b}
	
	\item \glll \textbf{ka}rá'ptewaherekto'sh\\
	\textbf{ka}-ra'-ptE\#wa-hrE=kt=o'sh\\
	\textbf{incp}-ins.heat-\textnormal{burn}\#1a-caus=pot=ind.m\\
	\glt 	`I will cause it to start to burn' \citep[47]{hollow1973b}
	\end{xlist}

\item\label{inceptiveexamples2} Examples of inceptive \textit{ki-}

	\begin{xlist}
	
	\item\label{inceptiveexamples2a} \glll ra\textbf{ka}rátaxa raháarootiki\\
	ra-\textbf{k}-ra-tax=E ra-haa=ooti=ki\\
	2a-incp-ins.mth-\textnormal{make.loud.noise}=sv 2s-\textnormal{start}=evid=cond\\
	\glt `when you start crying' \citep[185]{hollow1973a}
	
	\item \glll \textbf{ki}xką́harani\\
	\textbf{ki}-xkąh\#hrE=rį\\
	\textbf{incp}-\textnormal{move}\#caus=ss\\
	\glt `she began chasing him and...' \citep[222]{hollow1973a}
	
	\item \glll \textbf{ki}kanáani ki\textbf{kí}napaa máakaahaa\\
	\textbf{ki}-krąą=rį ki$\sim$\textbf{ki}-rąp=E wąąkE=haa\\
	\textbf{incp}-\textnormal{sing}=ss aug$\sim$\textbf{incp}-\textnormal{dance}=sv \textnormal{lying}.aux=sim\\
	\glt `they would start to sing and just start to dance' \citep[165]{trechter2012b}
	
	\end{xlist}

\end{exe}

Very few examples of inceptive \textit{ki-} appear in the corpus, so it is not evident how many other verbs fall into the class of \textit{ki-} inceptives versus \textit{ka-} inceptives. We can see above in (\ref{inceptiveexamples2a}) that inceptive \textit{ki-} behaves like iterative \textit{ki-} with respect to allophonic realization as /k-/ when before sonorant-initial stems. We can be certain that this example is truly \textit{ki-} and not \textit{ka-} by observing that stress appears on the third vocalic element, indicating that the $\langle$ka$\rangle$ in (\ref{inceptiveexamples2a}) is a Dorsey's Law vowel and therefore extrametrical. It is possible that we may see additional variation in \citeapos{bowers1971} recordings of Mrs. Annie Eagle and Mrs. Otter Sage, but the transcription and glossing of the 100-plus hours of recordings remains a task for the future.

The overall pattern we see with respect to aspectual prefixes is that \textit{ki-} is the sole iterative prefix, while \textit{ka-} is the primary inceptive prefix with certain tokens bearing \textit{ki-} instead. The lack of L1 speakers means that we may not know if there is any difference in meaning between inceptive \textit{ka-} and \textit{ki-}, or if the distinction lies more in a difference in registers or varieties of Mandan, i.e., Núu'etaa versus Rúptaa varieties. However, it is worth noting that there is no overlap in the words that bear \textit{ka-} as the inceptive marker versus those that bear \textit{ki-}, so this variation may ultimately be ascribed to the lexicon.

An additional unknown is whether stacking these aspectuals is permissible. English allows for constructions like `she began writing again' or `we restarted an annual tradition.' If these aspectuals are stackable, is there any asymmetry in how they can be concatenated? We do not know if Mandan permits constructions like *?\textit{kikásii} `re-begin traveling' or  *?\textit{kakísii} `start traveling again', though their absences from the corpus suggest that these forms may be marked at the very best or illicit at the very worst.

\subsubsection{Voice prefixes (Slot 3)}\label{SubsubsecVoicePrefixes}

This group of derivational prefixes is especially common, being found frequently throughout the corpus. The prefix \textit{ki-} has been described as reflexive, reciprocal, middle voice, vertative, and \textit{suus} in Mandan by various authors (\citealt{kennard1936}, \citealt{hollow1970}, \citealt{coberly1979}, \textit{inter alios}). One issue that we must deal with is how to properly describe what \citet[22]{mixco1997a} describes as ``polysemous \textit{ki-}.''

A portion of this polysemous \textit{ki-} has been addressed in \sectref{inceptiveprefix}, where \textit{ki-} can be an inceptive marker. Herein I present the different manifestations of \textit{ki-} and its numerous uses.

\subsubsubsection{Middle voice marker: \textit{ki-}\textsubscript{1}}

This prefix is found whenever there is some dynamic change to a state or an action. The middle voice marker frequently carries an anticausative meaning, i.e., the subject of the verb undergoes some change in state without necessarily having an agent enacting that change, e.g., \textit{kishí} `become good', `get better', or `heal'. We can see examples of this use of \textit{ki}- in the data in (\ref{middlevoiceexamples}) below.

\newpage

\begin{exe}
\item\label{middlevoiceexamples} Examples of middle voice \textit{ki-}

	\begin{xlist}
	\item \glll wa\textbf{kí}kiiraso'sh\\
	wa-\textbf{ki}-kiiras=o'sh\\
	1a-\textbf{mid}-\textnormal{be.stingy.with}=ind.m\\
	\glt `I love him, I am stingy with it' \citep[111]{hollow1970}
	
	\item \glll p\textbf{k}akíisho'sh\\
	\textbf{k}-pa-kiish=o'sh\\
	\textbf{mid}-ins.push-\textnormal{wipe}=ind.m\\
	\glt `he pays off a debt' \citep[111]{hollow1970}
	
	\item \glll ra\textbf{kí}kiishke'sh\\
	ra-\textbf{ki}-kiishkE=o'sh\\
	1a-\textbf{mid}-\textnormal{consider}=ind.m\\
	\glt `you think about it, taste it' \citep[112]{hollow1970}
	
	\item \glll \textbf{ki}rúxų'ro'sh\\
	\textbf{ki}-ru-xų'=o'sh\\
	\textbf{mid}-ins.hand-\textnormal{plow}=ind.m\\
	\glt `he frowns' \citep[331]{hollow1970}
	
	\item \glll \textbf{ki}míkhere\\
	\textbf{ki}-wįk\#hrE\\
	\textbf{mid}-\textnormal{be.none}\#caus\\
	\glt `remove' \citep[113]{hollow1970}
	
	\item \glll \textbf{ki}xéero'sh\\
	\textbf{ki}-xee=o'sh\\
	\textbf{mid}-\textnormal{be.quiet}=ind.m\\
	\glt `he quits, surrenders' \citep[312]{hollow1970}
	
	\item \glll \textbf{ki}xíko'sh\\
	\textbf{ki}-xik=o'sh\\
	\textbf{mid}-\textnormal{be.bad}=ind.m\\
	\glt `it soured (of cream), spoiled' \citep[314]{hollow1970}
	
	\item \glll \textbf{ki}xté'sh\\
	\textbf{ki}-xtE=o'sh\\
	\textbf{mid}-\textnormal{be.big}=ind.m\\
	\glt `it gets bigger' \citep[326]{hollow1970}
	
	\end{xlist}

\end{exe}

Some of the examples above are more idiomatic or metaphorical in nature, e.g., \textit{kirúxų'} `frown' literally means `becomes plowed', likely referring the motion of the mouth. The word \textit{pkakíish} `pay off a debt' literally means to `have something get pushed clear', likely referencing the clearing of a ledger. Most instances of middle voice throughout the corpus are transparent, however, and will carry these semantics.

A small number of instances of middle voice exist that suggest that the middle voice can also be used to express accidental actions, as seen in (\ref{middlevoiceaccident}) below.


\begin{exe}
\item\label{middlevoiceaccident} Middle voice to denote accidental actions

	\begin{xlist}

	\item \glll p\textbf{k}ashų́ho'sh\\
	\textbf{k}-pa-shųh=o'sh\\
	\textbf{mid}-ins.push-\textnormal{spill}=ind.m\\
	\glt `he spills something accidentally' \citep[288]{hollow1970}
	
	\item \glll p\textbf{k}axkího'sh\\
	\textbf{k}-pa-xkih=o'sh\\
	\textbf{mid}-ins.push-\textnormal{be.split}=ind.m\\
	\glt `he split it accidentally' \citep[318]{hollow1970}
	
	\end{xlist}
	
\end{exe}

This middle voice marking is more akin to an impersonal construction rather than truly being accident-denoting morphology. As such, a more literal interpretation of the sentences above would be `something got spilled' and `something got split', respectively. Rather than ascribing grammatical subjecthood to the actual agent who affected this change, these constructions shift the grammatical role of the agent from the person who did the action to the action happening on its own. It is thus possible for speakers of Mandan to downgrade their own agency for an action and simply make use of the middle voice to express that something has happened, but without morphologically attaching a cause to this event.

\subsubsubsection{\textit{Suus} marker: \textit{ki-}\textsubscript{2}}

Like all other Siouan languages \citep[22]{mixco1997a}, Mandan is able to use the polysemous \textit{ki-} to indicate that the direct object is possessed by the subject. Certain authors refer to this as a reflexive, of which it certainly is one species, but calling it a reflexive does not fully explain what it does. After all, this \textit{suus} marker shows that the direct object of an action is not the agent, but something of the agent's very own. We can see this relationship in the data in (\ref{suusexamples}) below.

\newpage

\begin{exe}
\item\label{suusexamples} Examples of \textit{suus}-marking \textit{ki-}

	\begin{xlist}
	
	
%	\item \glll pkaminísho'sh\\
%	k-pa-wrįsh=o'sh\\
%	suus-ins.push-\textnormal{be.folded.up}=ind.m\\
%	\glt `he folded up something of his own' \citep[305]{hollow1970}
	
	\item \glll wapáminishs \textbf{ki}rúxotki\\
	wa-pa-wrįsh=s \textbf{k}-ru-xot=ki\\
	unsp-ins.push-\textnormal{be.folded}=def \textbf{suus}-ins.hand\textnormal{untie}=cond\\
	\glt `when she untied her bundle' \citep[11]{kennard1936}
	
	\item \glll tamáahį \textbf{ki}rúsheka'ehe\\
	ta-wąąhį \textbf{k}-ru-shE=ka'ehe\\
	al-\textnormal{knife} \textbf{suus}-ins.hand-\textnormal{grasp}=quot\\
	\glt `he took his knife, it is said' \citep[11]{kennard1936}
	
	\item \glll ówa\textbf{ki}pka'uxo'sh\\
	o-wa-\textbf{ki}-k-pa-ux=o'sh\\
	pv.irr-1a-\textbf{suus}-itr-ins.push-\textnormal{be.broken}=ind.m\\
	\glt `I am going to break something of my own again' \citep[450]{hollow1970}
	
	\item \glll istámis \textbf{ki}rusá'roomako'sh\\
	istawį=s\textbf{ }k-ru-sa'=oowąk=o'sh\\
	\textnormal{eye}=def \textbf{suus}-ins.hand-\textnormal{wash}=narr=ind.m\\
	\glt `he washed his eyes' \citep[37]{hollow1973a}
	
	\item \glll \textbf{ki}ká'ro'sh\\
	\textbf{ki}-ka'=o'sh\\
	\textbf{suus}-\textnormal{have}=ind.m\\
	\glt `he keeps it' \citep[102]{hollow1970}
	
	\end{xlist}

\end{exe}

This \textit{suus} marker often accompanies verbs where the subject affects some kind of change upon their own possession, e.g., picking it up or hitting it. It is not used in the corpus with verbs where the direct object is not physically affected by the action. This lack of \textit{suus} with perception verbs does not indicate that such constructions are impossible, but the pattern of verbs that can take \textit{suus} marking suggests that there is a restriction on the kinds of actions that can have a \textit{suus} direct object.

\subsubsubsection{Reflexive and reciprocal marker: \textit{ki-}\textsubscript{3}}\label{reflexivemarker}

\citet[31]{kennard1936} lists \textit{ki-} as the reflexive marker, while \citet[440]{hollow1970} argues that \textit{íki-} is really the reflexive marker. It is unlikely that \citeauthor{hollow1970} is correct, since the one verb form he uses to justify the base shape of the reflexive in Mandan actually has an instrumental preverb on it that is not attached to the reflexive. The word \textit{íkihe'sh} does mean `he sees himself', but the addition of the instrumental preverb indicates the presence of a covert instrument, such as a mirror or a pool of water in which he sees his reflection. In the corpus, each instance of $\langle$íki$\rangle$ in \citeapos{hollow1970} transcription actually contains an instrumental preverb. We can see other examples of \textit{ki}- being used as a clear reflexive marker in the data in (\ref{reflexiveexamples}) below.


\begin{exe}
\item\label{reflexiveexamples} Examples of reflexive \textit{ki-}

	\begin{xlist}
	
	\item \glll mi\textbf{kí}he'sh\\
	wį-\textbf{ki}-hE=o'sh\\
	1s-\textbf{rflx}-\textnormal{see}=ind.m\\
	\glt `I see myself' \citep[23]{mixco1997a}
	
	\item \glll mí'shak ími\textbf{ki}sehki\\
	w'\~~-ishak i-wį-\textbf{ki}-sEk=ki\\
	1s-pro pv.ins-1s-\textbf{rflx}-\textnormal{make}=cond\\
	\glt `when I fix myself' \citep[127]{hollow1970}
	
	\item \glll ni\textbf{kí}paweshoote'sh\\
	rį-\textbf{ki}-pa-wesh-ootE=o'sh\\
	2s-\textbf{rflx}-ins.push-\textnormal{cut}=evid=ind.m\\
	\glt `you must have cut yourself' \citep[11]{kennard1936}
	
	\item \glll nu\textbf{kí}rushkapo'sh\\
	rų-\textbf{ki}-ru-shkap=o'sh\\
	1s.pl-\textbf{rflx}-ins.hand-\textnormal{pinch}=ind.m\\
	\glt `we pinch ourselves (each other)' \citep[440]{hollow1970}
	
	\item \glll í\textbf{ki}he'sh\\
	i-\textbf{ki}-hE=o'sh\\
	pv.dir-\textbf{rflx}-\textnormal{see}=ind.m\\
	\glt `he sees himself' \citep[440]{hollow1970}
	
	\item \glll í\textbf{ki}rookereroomako'sh\\
	i-\textbf{ki}-roo=krE=oowąk=o'sh\\
	pv.ins-\textbf{rflx}-\textnormal{talk}=3pl=narr=ind.m\\
	\glt `they argued about it' \citep[24]{hollow1973a}
	
	\end{xlist}

\end{exe}


The pronominals used in reflexive marking are all stative prefixes. The first person singular marker before a reflexive is \textit{mi}-, which is not a typical first person singular stative prefix. This allomorph is explained further in \sectref{wiPprefix}. The second person \textit{ni}- is likewise the second person stative. The first person plural active \textit{nu}- is homophonous with the first person plural stative in reflexive constructions.

\citet[11]{kennard1936} gives \textit{kiki-} as the form for the reciprocal. \citet[440]{hollow1970} notes that reciprocal and reflexive acts are ambiguous. The presence of \textit{kiki-} is attested in \citet[23]{mixco1997a}, though he suggests that it is merely \textit{ki}- concatenated with itself. I likewise take this view, hypothesizing that this reciprocal meaning is accomplished through reduplication. Prefixal reduplication in Mandan add an augmentative meaning to the item being reduplicated. This process typically targets verbs and nouns, but other prefixes can be reduplicated as well for emphasis. This pattern of reduplicating prefixes is observed throughout the corpus, typically on pronominal morphology to emphasize who is doing an action. 

There is no dedicated reciprocal marking in Mandan, though a reciprocal interpretation can be achieved by reduplicating the reflexive. We can see this behavior in (\ref{reciprocalexamples}) below, where each of the utterances carries a reciprocal reading that has been confirmed by the consultant who provided the data from each of the respective narratives that have been cited, but most do not bear double \textit{ki-} markers. 

\begin{exe}
\item\label{reciprocalexamples} Examples of reciprocal marking

	\begin{xlist}
	
	\item\label{reciprocalexamples1} \glll numá'k \textbf{ki}kíhekere'sh\\
	ruwą'k \textbf{ki}$\sim${ki}-hE=o'sh\\
	\textnormal{man} \textbf{recp}$\sim${rflx}-\textnormal{see}=ind.m\\
	\glt 	`the men see each other' \citep[23]{mixco1997a}
	
	\item\label{reciprocalexamples2} \glll \textbf{ki}kíraksąąkereroomako'sh\\
	\textbf{ki}-kiraksąą=krE=oowąk=o'sh\\
	\textbf{rflx}-\textnormal{make.war}=3pl=narr=ind.m\\
	\glt `they fought each other' \citep[40]{hollow1973b}
	
	\item\label{reciprocalexamples3} \glll \textbf{kikí}xkąharani\\
	\textbf{ki}$\sim$ki-xkąh\#hrE=rį\\
	\textbf{recp}$\sim$rflx-\textnormal{move}\#caus=ss\\
	\glt `they were chasing each other and...' \citep[107]{trechter2012b}
	
	\item\label{reciprocalexamples4} \glll \textbf{ki}káni\\
	\textbf{ki}-kE=rį\\
	\textbf{rflx}-\textnormal{pluck}=ss\\
	\glt `they shot [arrows] at each other and...' \citep[116]{trechter2012b}
	
	\end{xlist}

\end{exe}

The reflexive can situationally carry reciprocal meaning, though it seems likely that reduplicating the reflexive to ensure a reciprocal reading is done for emphasis. In cases like (\ref{reciprocalexamples4}), the single use of the reflexive could yield two readings: `they shot [arrows] at each other' or `they shot themselves [with their own arrows].' The more likely reading of this example is that the actors involves were firing arrows at others, rather than at their own persons. As such, the reduplicated reciprocal construction is unnecessary.

Another possible reason for why double \textit{ki-} marking is observed so rarely is because of the augmentative reading that prefixal reduplication bestows upon an item. As such, in (\ref{reciprocalexamples3}), the use of \textit{kiki-} could serve to indicate that the actors involved were not involved in a single chase action, but were chasing each other all over. Nevertheless, \citeapos{hollow1970} observation holds that \textit{kiki-} is quite rare, and \citeapos{mixco1997a} position that a reciprocal reading can be conveyed through reduplicating the reflexive \textit{ki-} appears to bear out in the data. 
\subsubsubsection{Vertitive marker: \textit{ki}-\textsubscript{4}}

The use of the term vertitive is ascribed to Terrance Kaufman, who coined it for describing cislocative motion in Mayan languages, and this term was passed on to his student \citeauthor{hollow1970} and onto Siouanists at large (\citeauthor{rankin2010} p.c.). The Siouanist definition of this term deals with motion back to a source. \citet{taylor1976} reconstructs core motion verbs in Proto-Siouan as having vertitive and non-vertitive forms, though his interpretation of vertitivity is restricted to meaning `back'.

\citet[238]{quintero2004} defines the vertitive as a kind of \textit{suus} marker, where an agent returns to their home. In practice, the vertitive is often used to indicate motion homeward in Mandan, but it is also found in narratives where an agent has no clear home, as in the case of the cultural figure \textit{Kinúma'kshi} `Royal Chief, Old Man Coyote', who is constantly traveling the land. As such, this \textit{suus} interpretation does not hold in Mandan.

\citet{cumberland2005} is really the first Siouanist to contextualize the complex use of the vertitive within a narrative structure. While driving around with a consultant, \citeauthor{cumberland2005} (p.c.) noticed that her consultant had started using vertitive motion verbs as soon as they reached the mid-point of their journey around the reservation and began the process of returning to the place whence they had originally departed. This vertitive marking did not indicate that they were returning home, but rather returning to a particular origin point that has been established.

The vertitive does not mark a return home, but a return to some deictic center. The type of motion verb used depends on the relationship between a deictic center (DC) and a base. The DC is the geospatial point of perspective from where motion verbs are interpreted. Furthermore, the location designated as the deictic center can shift during discourse, causing a shift in the base to which the agent is returning. The base is the location associated with the traveler, and can be permanent or temporary, e.g., permanent like a home or temporary like a social gathering the traveler has just attended. All verbs of motion deal with movement towards or away from one or both of these two variables. This pattern in Assiniboine is summarized in \tabref{assiniboineverbs} below.

\begin{table}
\caption{Verbs of motion in Assiniboine \citep[287]{cumberland2005}}\label{assiniboineverbs} 

    \begin{tabular}{lllll}
    \lsptoprule
~&
	~&
	\textbf{Departure}&
	\textbf{Progress}&
	\textbf{Arrival}\\
\midrule
$\leftarrow$DC&
	$\leftarrow$base&
	\textit{iyáyA}&
	\textit{yÁ}&
	\textit{í}\\
~&
	~&
	`leave here&
	`to go there'&
	`arrive there'\\
~&
	~&
	go there'&
	~&
	~\\
\tablevspace
~&
	$\rightarrow$base&
	\textit{kʰiknÁ}&
	\textit{knÁ}&
	\textit{kʰi}\\
~&
	~&
	`leave here&
	`go back there'&
	`arrive back there'\\
~&
	~&
	to go back there'&
	~&
	~\\
\tablevspace
$\rightarrow$DC&
	$\leftarrow$base&
	\textit{hiyú}&
	\textit{ú}&
	\textit{hí}\\
~&
	~&
	`leave there&
	`come here'&
	`arrive here'\\
~&
	~&
	to go here'&
	~&
	~\\
\tablevspace
~&
	$\rightarrow$base&
	\textit{knicú}&
	\textit{kú}&
	\textit{kní}\\
~&
	~&
	`leave there&
	`come back here'&
	`arrive back here'\\
~&
	~&
	to go back here'&
	~&
	~\\

\lspbottomrule
    \end{tabular}
\end{table}

\citet{taylor1976} argues this tripartite Departure-Progress-Arrival paradigm is an innovation of Dakotan languages, but I argue that this tripartite distinction is truly a Proto-Siouan feature, as cognates of most the above forms exist in other Siouan languages and Catawban \citep{kasak2013b}. Furthermore, Mandan has cognates of all the motion verbs found in Dakotan languages like Assiniboine, as can been seen in \tabref{mandanmotionverbtable}. \citet{kennard1936} and \citet{hollow1970} only list the Progress and Arrival series as motion verbs, but the corpus and judgments from speakers reveal that the Departure series exists in Mandan as well. The semantics of the Departure series convey motions that are underway, as we can see in the examples in (\ref{departureverbsmandan}) below.

\begin{exe}
\item\label{departureverbsmandan} Departure-series verbs in Mandan

	\begin{xlist}
	
	\item \glll \textbf{tíhuu}ro'sh\\
	ti\#huu=o'sh\\
	\textnormal{arrive.here}\#\textnormal{come.here}=ind.m\\
	\glt `he is arriving' \citep[37]{kennard1936}
	
	\item \glll \textbf{híreeh}ki ``mmmm'' éeheni\\
	hi\#rEEh=ki ~ ee-he=rį\\
	\textnormal{arrive.there}\#\textnormal{go.there}=cond ~ pv-\textnormal{say}=ss\\
	\glt `When he was coming, he said ``mmmm''' \citep[73]{hollow1973b}

        \item \glll \textbf{kíkereeh}ak karópxani...\\
        ki\#rEEh=ak ka-ropxE=rį\\
        \textnormal{arrive.there}.vert\#vert-\textnormal{go.there}=ds incp-\textnormal{enter}=ss\\
        \glt `when they arrived back and entered...' \citep[72]{trechter2012b}
        
	
	\end{xlist}

\end{exe}

\citeapos{kasak2012} summary of motion verbs in Mandan appears in \tabref{mandanmotionverbtable}. We can see that most of these lexical items have transparent morphology for vertative forms, consisting of the verb root preceded by either /ki-/ or /k-/. However, the vertative forms for \textit{húu} `come here' and \textit{tí} `arrive here' show remnants of older morphology. Instead of having /ki-/ plus /huuu/, we have \textit{kúh} `come back here.' This form shows a remnant of the Proto-Siouan stem augment *-he, indicating that there may have been some alternative form for this verb. We see an analogous trajectory for \textit{réeh} `go there', which is composed of PSi *rEE `go' plus the stem augment PSi *-he. There is no form of \textit{réeh} that appears without the stem augment in Mandan.

We also see the vertitive form of \textit{tí} `arrive here' realized as \textit{kirí}. We can reconstruct \textit{tí} from Proto-Siouan *re `here' plus *hi(i) `arrive' \citep{kasak2013b,rankin2015}. The syncope of the initial vowel creates a *rh sequence, which is realized as /t/ in Mandan, whence \textit{tí} /ti/ `arrive here.' Adding the vertitive PSi *ki- to this stem results in further syncope, where we have a *krh sequence in some stage in the development of the Mandan language, which is realized as /kr/ in contemporary Mandan, whence \textit{kirí} /kri/ `arrive back here.'

To further illustrate the semantics of these verbs and their relationships to each other, an adaption of \citeauthor{cumberland2005}'s (\citeyear{cumberland2005}:297) visual interpretation of vertitivity for Mandan appears in \figref{visualmandanmotion}. On this visual interpretation of vertitivity in Mandan, a solid line indicates motion away from a base; a dotted line indicates motion towards a base. Like in Assiniboine, a base in Mandan is some location to which the traveler belongs as determined within the discourse. All motion is done with respect to or from a base and a potentially shifting deictic center \citep[287]{cumberland2005}.


\begin{table}
\caption{Verbs of motion in Mandan} \label{mandanmotionverbtable}

    \begin{tabular}{lllll}
    \lsptoprule
~&
	~&
	\textbf{Departure}&
	\textbf{Progress}&
	\textbf{Arrival}\\
$\leftarrow$DC&
	$\leftarrow$base&
	\textit{híreeh}&
	\textit{réeh}&
	\textit{hí}\\
~&
	~&
	`leave here&
	`go there'&
	`arrive there'\\
~&
	~&
	to go there'&
	~&
	~\\
~&
	$\rightarrow$base&
	\textit{kíkereeh}&
	\textit{keréeh}&
	\textit{kí}\\
~&
	~&
	`leave here&
	`go back there'&
	`arrive back there'\\
~&
	~&
	to go back there'&
	~&
	~\\
$\rightarrow$DC&
	$\leftarrow$base&
	\textit{tíhuu}&
	\textit{húu}&
	\textit{tí}\\
~&
	~&
	`leave there&
	`come here'&
	`arrive here'\\
~&
	~&
	to go here'&
	~&
	~\\
~&
	$\rightarrow$base&
	\textit{kiríkuh}&
	\textit{kúh}&
	\textit{kirí}\\
~&
	~&
	`leave there&
	`come back here'&
	`arrive back here'\\
~&
	~&
	to go back here'&
	~&
	~\\

\lspbottomrule
    \end{tabular}
\end{table}

\begin{figure}
\caption{Visual representation of Mandan motion verbs}\label{visualmandanmotion}
\includegraphics[scale=0.85]{figures/mandanmotionverbs.png}

\end{figure}

The vertitive \textit{ki-} is more than just some kind of reflexive marker that marks going back to `one's own' place. This prefix indicates motion towards some base that is contextualized within the narrative. The presence of \textit{ki-} \textsc{vert} often indicates that a speaker is indicating homeward motion, but this use is not exclusive, and does not fully capture the range of possibilities for vertitive marking. Vertitive marking appears not only on the classic verbs of motion shown above, but can also appear on other motion verbs in Mandan, as we can see in (\ref{othervertitive}) below.

\begin{exe}
\item\label{othervertitive} Vertitive usage on non-tripartite motion verbs

	\begin{xlist}
	
	\item \glll \textbf{ki}ptą́ho'sh\\
	\textbf{ki}-ptąh=o'sh\\
	\textbf{vert}-\textnormal{turn.away}=ind.m\\
	\glt `he turns back' \citep[155]{hollow1970}
	
	\item \glll áa\textbf{ke}ropxe\\
	aa-\textbf{k}-ropxE\\
	pv.tr-\textbf{vert}-\textnormal{enter}\\
	\glt `bring something back in' \citep[189]{hollow1970}
	
	\item \glll \textbf{ki}súkherek\\
	\textbf{ki}-suk\#hrE=ak\\
	\textbf{vert}-\textnormal{come.out}\#caus=ds\\
	\glt `having made him get back out' \cite[144]{hollow1973a}
	
	\item \glll í\textbf{ki}'aakit háa'aakit ráahini\\
	i-\textbf{ki}-aaki=t haa\#aaki=t rEEh=rį\\
	pv.dir-\textbf{vert}-\textnormal{be.above}=loc \textnormal{cloud}\#\textnormal{be.above}=t \textnormal{go.there}=ss\\
	\glt `they went back upward to heaven and...' \citep[153]{hollow1973a}
	
	\item \glll \textbf{ki}kú'kereroote'sh\\
	\textbf{ki}-ku'=krE=ootE=o'sh\\
	\textbf{vert}-\textnormal{give}=3pl=evid=ind.m\\
	\glt `they must have given it back' \citep[474]{hollow1970}
	
	\item \glll \textbf{ki}rúsheka'ehe\\
	\textbf{k}-ru-shE=ka'ehe\\
	\textbf{vert}-ins.hand-\textnormal{grasp}=quot\\
	\glt `he took them back, it is said' \cite[16]{hollow1973a}
	
	\end{xlist}

\end{exe}

Most of the examples above involve \textit{ki-} prefixing onto a verb of motion, but we can see that the vertitive marker can also appear on verbs that indirectly deal with motion, like \textit{kikú'} `give back' or \textit{kirúshe} `take back.' Because of the polysemy of \textit{ki-}, the vertitive is often grouped with the other voice markers. It is not fully clear if this affix truly belongs in slot 2 or 3. While this prefix does deal with properties of the subject of the verb rather than the act itself (i.e., going back to a base associated with the subject), it also has an iterative association, given that an agent is once again in motion towards a base. The fact that the iterative and vertitive \textit{ki-} prefixes are homophonous means it is not obvious which one is which, creating situations where speakers disagree which \textit{ki-} is vertative and which is iterative. We can see both \textit{ki-} prefixes co-occur in constructions like the one in (\ref{2kis}) below.


\begin{exe}
\item\label{2kis} Multiple \textit{ki-} marking

\glll \textbf{ki}kíku'ro'sh\\
	\textbf{ki}-ki-ku'=o'sh\\
	\textbf{vert}-itr-\textnormal{give}=ind.m\\
	\glt `he gave it back again'

\end{exe}

When presented with examples like the one above, speakers recognize it and its intended meaning, but are unable to specify which \textit{ki-} is indicating that the subject is giving the object back and which one indicates that this action is happening again. The vertitive in the tripartite motion verbs (e.g., \textit{keréeh} `go back there') seems to be a fossilized formative, as we can have constructions like those below, where the iterative seemingly precedes a vertitive form. Examples of these vertitive verbs with iterative marking appear in (\ref{iterativevertitive}) below.

\begin{exe}
\item\label{iterativevertitive} Iterative marking with vertitive verbs

\begin{xlist}

\item \glll wa\textbf{kí}kuho'sh\\
	wa-\textbf{ki}-kuh=o'sh\\
	1a-\textbf{iter}-\textnormal{come.back.here}=ind.m\\
	\glt `I came back again' \citep[12]{mixco1997a}
	
	\item \glll \textbf{ki}kiríkerek\\
	\textbf{ki}-kri=krE=ak\\
	\textbf{iter}-\textnormal{arrive.back.here}=3pl=ds\\
	\glt `when they got back' \citep[39]{kennard1936}
	
	\end{xlist}
	
\end{exe}

In each of the examples in (\ref{iterativevertitive}) above, iterative marking precedes a vertitive verb. In \citet{hollow1970} and \citet{mixco1997a}, such cases are glossed as if the vertitive verbs are morphologically deconstructable, synchronically. There is no set of rules in contemporary Mandan that turns an input like /ki-huu/ `\textsc{vert}-come.here' into [kuh]. Historically, there was likely a process in Pre-Mandan where the vertitive \textit{ki-} became /k-/ before an /h/, yielding *k-huu, which then triggered metathesis of the /h/ to avoid aspiration, yielding the modern \textit{kúh} `come back here.' This process, however, is not something modern speakers are able to access, so these verbs are treated as atomic lexical items herein. 

There are no examples of clearly concatenated iterative and vertitive prefixes in the corpus, with iterativity being expressed on vertitive-marked actions with the adverb \textit{inák} `again'. Though there is no definitive evidence from speakers or from the corpus, the vertitive marker is considered here to be in slot 3 of the template due to its close semantic alignment with the agent argument, which the other voice markers here likewise share.

\subsubsection{Preverbal prefixes (Slots 6 and 7)}\label{SubsubsecPreverbs}

Preverbs are frequent elements of words in Mandan. I refer to these prefixes as preverbs because they are elements that must appear before a verb and are not verbs themselves, yet they are integral parts of any verb that lexically selects for them. \citet{rankinetal2003} and \citet{helmbrecht2008} agree that these elements likely originate from Proto-Siouan or Pre-Proto-Siouan postpostpositions. These postpositions became reanalyzed over time as integral elements of a verb instead of being associated with a noun. There are still traces of these preverbs as true postpositions in a few Siouan languages. Crow, for example, makes use of the instrumental postposition \textit{ii} productively. We can see an example of one of these true postpositions in Crow in (\ref{crowiiexample}) below.

\begin{exe}

\item\label{crowiiexample} Example of postpositional instrumental \textit{ii}

\gll hinne shikáakee-sh baap-tatchée \textnormal{[}iseé ii\textnormal{]\textsubscript{PP}} ihchilasshihk-a-lahkú-k\\
this boy-\textsc{det} day-every his.arrow \textsc{ins} practice-\textsc{cont}-continue-\textsc{decl}\\
\glt `every day this boy kept practicing [with his arrows]\textsubscript{PP}' \citep[377]{graczyk2007}

\end{exe}

In Mandan, there are nine preverbs, shown in (\ref{mandanapplicativelist}) below.

\begin{exe}
\item\label{mandanapplicativelist} Preverbs in Mandan

\begin{xlist}
\item \textit{aa}-                 transitivizer
\item \textit{ee}-                  generic preverb
\item \textit{i}-\textsubscript{1} directional 
\item \textit{i}-\textsubscript{2} instrumental
\item \textit{i}-\textsubscript{3} ordenalizer  
\item \textit{i}-\textsubscript{4} possessive
\item \textit{į'}-                reflexive
\item \textit{o}-\textsubscript{1} irrealis 
\item \textit{o}-\textsubscript{2} locative
           
\end{xlist}

\end{exe}

The instrumental postposition \textit{ii} in Crow is cognate with the instrumental preverb \textit{i-} in Mandan, though most preverbs in Mandan involve short vowels. Five of these preverbs have cognates across the Siouan language family, while three appear to be specific to Mandan. One phonological factor that all preverbs share is they have a tendency to draw stress, as described in \sectref{ParaPreverbStress}. Preverbs likewise act as a boundary for the leftward spread of nasal harmony. This section gives examples involving preverbs and their interaction with inflectional morphology.

\subsubsubsection{Applicative preverbs (Slot 6)}\label{ParaApplicatives}

The majority of preverbs in Mandan can be considered applicatives. Siouanist literature typically calls all preverbs applicatives, due to the fact that these elements indicate that the number of arguments that the root verb takes is increased by one. These applicative preverbs are as follows:

\subsubsubsubsection{Transitivizer preverb: \textit{aa-}}\label{SubParaTransitivizer}

The transitivizer \textit{aa-} is used to turn a subset of intransitive verbs into transitive ones. It predominantly occurs with verbs of motion to give the verb a comportatitve (i.e., an action done while carrying something or bringing someone along) reading that can mean to travel with someone or to take or carry something. This preverb is ambiguous over whether the intended reading is that someone is doing something with another person or thing, or whether someone is doing an action while carrying or being in physical possession of the direct object. We can see examples of the transitivizer in (\ref{TransitivePreverb}) below.

\begin{exe}

\item\label{TransitivePreverb} Examples of the transitivizer \textit{aa-}

\begin{xlist}

\item \glll nihų́pe r\textbf{áa}huuro're\\
	rį-hųp(E) r-\textbf{aa}-huu=o're\\
	2pos-\textnormal{shoe} 1a.pl-\textbf{pv.tr}-\textnormal{come.here}=ind.f\\
	\glt 	`we brought your shoes' \citep[169]{hollow1973a}
	
\item \glll tasúhkeres wa'\textbf{áa}huuroomako'sh\\
	ta-suk=krE=s wa-\textbf{aa}-huu=oowąk=o'sh\\
	al-\textnormal{child}=3pl=def unsp-\textbf{pv.tr}-\textnormal{come.here}=narr=ind.m\\
	\glt `he brought his children' \citep[177]{hollow1973a}
	
\item \glll hų́ \textbf{áa}ki'hkarani\\
	hų \textbf{aa}-ki'h=krE=rį\\
	\textnormal{be.many} \textbf{pv.tr}-\textnormal{arrive.back.there}=3pl=ss\\
	\glt 	`they brought a lot home and...' \citep[184]{hollow1973a}
	
\item \glll taté éeheni na'é'na ~ ~ ~ ~ ~ ~ ~ ~ ~ ~ ~ ~ ~ ~ ~  \textbf{áa}nihuukere'sh\\
	tatE ee-he=rį rą'e=o'=rą ~ ~ ~ ~ ~ ~ ~ ~ ~ ~ ~ ~ ~ ~ ~  \textbf{aa}-rį-huu=krE=o'sh\\
	\textnormal{father}.voc pv-\textnormal{say}=ss \textnormal{mother}.voc=\textnormal{be}=top ~ ~ ~ ~ ~ ~ ~ ~ ~ ~ ~ ~ ~ ~ ~  \textbf{pv.tr}-2s-\textnormal{come.here}=3pl=ind.m\\
	\glt `it was father and mother who brought you here' \citep[211]{hollow1973a}
	
\item \glll íxe'hąką't ~ ~ ~ ~ ~ ~ ~ ~ ~ ~ ~ ~ ~ ~ ~ ~ ~ ~ ~ ~  \textbf{áa}wakuho're\\
	i-xe'h\#hąk=ą't ~ ~ ~ ~ ~ ~ ~ ~ ~ ~ ~ ~ ~ ~ ~ ~ ~ ~ ~ ~  \textbf{aa}-wa-kuh=o're\\
	pv.ins-\textnormal{be.dripping}\#pos.stnd=dem.anap ~ ~ ~ ~ ~ ~ ~ ~ ~ ~ ~ ~ ~ ~ ~ ~ ~ ~ ~ ~ \textbf{pv.tr}-1a-\textnormal{come.back.here}=ind.f\\
	\glt `I brought that basket back' \citep[92]{hollow1973a}
	
\item \glll inák \textbf{áa}sukini patíhka'eheero'sh\\
	irąk \textbf{aa}-suk=rį pa-tik=ka'ehee=o'sh\\
	\textnormal{again} \textbf{pv.tr}-\textnormal{come.out}=ss ins.push-\textnormal{throw}=quot=ind.m\\
	\glt 	`she took it out again and threw it away' \citep[122]{hollow1973a}
	
\item \glll \textbf{áa}mati'sh\\
	\textbf{aa}-wą-ti=o'sh\\
	\textbf{pv.tr}-1s-\textnormal{arrive.here}=ind.m\\
	\glt `he brought me' \citep[13]{kennard1936}

\end{xlist}

\end{exe}

There is only one exception to the generalization that \textit{aa-} appears only with motion verbs, which appears in (\ref{aakxuNh}) below.

\begin{exe}
\item\label{aakxuNh} \glll \textbf{áa}kxųho'sh\\
	\textbf{aa}-kxųh=o'sh\\
	\textbf{pv.tr}-\textnormal{lie.down}=ind.m\\
	\glt `he lies with someone' \citep[127]{hollow1970}

\end{exe}

\citet[429]{hollow1970} remarks that this term is a euphemism for sexual intercourse, and that it is a calque from English. There are no other instances of \textit{aa-} with any other verb besides a motion verb, and this preverb is not generally productive in contemporary Mandan.

\subsubsubsubsection{Generic preverb: \textit{ee-}}\label{SubParaGeneric}

The generic preverb \textit{ee-} is referred to as `generic' due to the fact that it does not convey any meaning of its own. It is semantically bleached in Mandan, and has a very limited distribution. Diachronically, this applicative can be reconstructed back to the preverb *e(e) in Proto-Siouan \citep{rankin2015}, which may originate in the Proto-Siouan distal demonstrative *ʔee. Another possibility is that the Siouan languages that permit multiple preverbs have the combination of \textit{aa-} and \textit{i-} become \textit{e-}.

In Mandan, there are only two verbs with the generic preverb: \textit{éehee} `say' and \textit{éereh} `think, want.' The verb `say' in Mandan is also the only truly irregular verb. Its conjugation appears in \tabref{mandantosay} below.

\begin{table}
\caption{Conjugation of \textit{éehee} `say'} \label{mandantosay}

    \begin{tabular}{llll}\hline\hline
~&
	\textbf{Singular}&
	\textbf{Dual}&
	\textbf{Plural}\\
1&
	\textit{éepe'sh}&
	\textit{réeheero'sh}&
	\textit{réehaanito'sh}\\
2&
	\textit{éete'sh}&
	~&
	\textit{éetaanito'sh}\\
3&
	\textit{éeheero'sh} $\sim$ \textit{éehe'sh}&
	~&
	\textit{éeheekere'sh} $\sim$ \textit{éehekere'sh}\\
\lspbottomrule
    \end{tabular}
\end{table}

The root verb *hee in Proto-Siouan *e(e)-hee `say' contains a long vowel in its root, which is generally preserved in the third person forms. This long vowel is contracted in first and second person singular forms, and \citet{hollow1970} reports that a short vowel is permissible for the third person singular, i.e., \textit{éehe'sh} `he said it.' However, the overwhelming majority of instances of third person singular forms of \textit{éehee} `say' are \textit{éeheero'sh}.

This verb sporadically ablauts; first and second person will ablaut but this does not happen for third person forms. The unexpected first and second person singular forms are descended from a subset of Proto-Siouan verbs that took reduced inflectional prefixes. For example, instead of the typical first person *wa-, some verbs took only *w-, and the same pattern held for second person *ya- and *y-. The cluster *wh obstruentized to [p] and clusters *rh and *yh became [t] in Mandan and became aspriated stops in other daughter languages, such as Lakota, e.g., Proto-Siouan *e(e)-w-hee > Lakota \textit{epȟé} `I say it.' What may have formerly been a phonologically predictable process at an earlier stage in the evolution of the language has become reanalyzed as irregular verb forms in contemporary Mandan.

An additional peculiarity is that when this verb is marked for an animate object, the irregular stems remain. Normally, marking first and second person arguments involves prefixing the appropriate pronominals, but on top of these pronominals, Mandan preserves the irregular first and second person singular stems on top of normal prefixation. This means that the active argument (i.e., the subject) is marked twice: once prefixally and again suppletively on the verb stem. Examples of this irregular double marking on the verb \textit{éehee} `say' appear in (\ref{irregularsay}) below.

\begin{exe}
\item\label{irregularsay} Irregularities in inflectional markers for \textit{éehee} `say'

	\begin{xlist}
	
	\item \glll éeminipe'sh\\
	ee-w-rį-pE=o'sh\\
	pv-1a-2s-\textnormal{say}.1a=ind.m\\
	\glt `I said it to you'
	
	\item \glll éemanate'sh\\
	ee-w-rą-tE=o'sh\\
	pv-1s-2a-\textnormal{say}.2a=ind.m\\
	\glt `you said it to me'
	
	\item \glll *éeminiheero'sh\\
	ee-w-rį-hee=o'sh\\
	pv-1a-2s-\textnormal{say}=ind.m\\
	\glt `I said it to you'
	\end{xlist}

\end{exe}

The only other verb to take the generic preverb \textit{e-} can also be traced back to having this preverb in Proto-Siouan, with PSi *e(e)-yehe `think' remaining relatively unchanged in modern Mandan \textit{éereh} `think, want.' Examples of this verb appear in (\ref{ExThink}) below.

\begin{exe}

\item\label{ExThink} Examples of \textit{éreh} `want, think'

	\begin{xlist}
	
	\item \glll riréesika'shka ó'\textbf{ee}reho'sh\\
	ri-reesik=a'shka o-\textbf{ee}-reh=o'sh\\
	2poss-\textnormal{tongue}=psbl pv.irr-\textbf{pv}-\textnormal{think}=ind.m\\
	\glt `he'll think that it might be your tongue' \citep[189]{hollow1973a}
	
	\item \glll Kinúma'kshi éeheni Numá'k Máxana ~ ~ ~ ~ ~ ~ ~ ~ ~ ~ ~ ~ ~ ~ ~ ~ íwahekanashe wakína'ni \textbf{ée}wereho'sh\\
	ki-ruwą'k ee-he=rį ruwą'k wąxrą ~ ~ ~ ~ ~ ~ ~ ~ ~ ~ ~ ~ ~ ~ ~ ~ i-wa-hek=rąsh=E wa-kirą'=rį \textbf{ee}-w-reh=o'sh\\
	mid-\textnormal{man}\#\textnormal{be.good} pv-\textnormal{say}=ss \textnormal{man} \textnormal{one} ~ ~ ~ ~ ~ ~ ~ ~ ~ ~ ~ ~ ~ ~ ~ ~ pv.ins-1a-\textnormal{know}=att=sv 1a-\textnormal{tell}=ss \textbf{pv}-1a-\textnormal{think}=ind.m\\
	\glt `I want to tell about what I sort of know about First Creator and Lone Man' \citep[1]{hollow1973a}
	
	\item \glll raráahini \textbf{ée}rereho'sh\\
	ra-rEEh=rį \textbf{ee}-r-reh=o'sh\\
	2a-\textnormal{go.there}=ss \textbf{pv}-2a-\textnormal{think}=ind.m\\
	\glt `you want to go' \citep[31]{hollow1973a}
	
	\item \glll xé'hini \textbf{ée}reho'sh\\
	xe'h=rį \textbf{ee}-reh=o'sh\\
	\textnormal{rain}=ss \textbf{pv}-\textnormal{think}=ind.m\\
	\glt `it might rain' \citep[182]{hollow1970}
	
	\end{xlist}

\end{exe}

This verb is often used to denote potential, most typically with respect to stating what one wants to do or what one may do. It is also possible to create impersonal constructions expressing the potential for some non-agentive act to happen, especially with weather verbs. Like \textit{aa-}, the preverb \textit{ee-} is not productive in Mandan.

\subsubsubsubsection{Directional preverb: \textit{i-}\textsubscript{1}}\label{SubParaDirectional}

The directional preverb is found on stative verbs alongside the locative postpositions \textit{=t} and \textit{=taa}. This preverb is a reflex of the Proto-Siouan directional applicative *ii-, and it is only used to express motion towards or away from a location that is incomplete. Unlike most other preverbs, the directional \textit{i-} is very productive in contemporary Mandan and can occur in novel constructions. This preverb may appear on nominal elements and stative verbs, including verbalized nouns, as seen in (\ref{directionalexamples}) below. 

\begin{exe}
\item\label{directionalexamples} Examples of directional \textit{i-}

\begin{xlist}

\item \glll \textbf{í}maataht waréeh íwateero'sh\\
	\textbf{i}-wąątah=t wa-rEEh i-wa-tee=o'sh\\
	\textbf{pv.dir}-\textnormal{river}=loc 1a-\textnormal{go.there} pv.ins-1a-\textnormal{like}=ind.m\\
	\glt `I'd like to go to the river' \citep[35]{hollow1973a}

\item \glll \textbf{í}mi'tit karáahini tasúhkeres ~ ~ ~ ~ ~ ~ ~ wa'áahuuroomako'sh\\
	\textbf{i}-wį'\#ti=t krEEh=rį ta-suk=krE=s ~ ~  ~ ~ ~ ~ ~ wa-aa-huu=oowąk=o'sh\\
	\textbf{pv.dir}-\textnormal{stone}\#\textnormal{abide}=loc \textnormal{go.back.there}=ss al-\textnormal{child}=3pl=def ~ ~ ~ ~ ~ ~ ~ unsp-pv.tr-\textnormal{come.here}=narr=ind.m\\
	\glt `he went back to the village and brought his children' \citep[177]{hollow1973a}
	
\item \glll mí'shak, \textbf{í}pashahąkt náaketaa máa'ąk ~ ~ ~ ~ ~ ~ ~ ~ ~ ~ ~ ~ ~ íwasekto'sh\\
	w'\~~-ishak \textbf{i}-pashahąk=t rąąkE=taa wąą'ąk ~ ~ ~ ~ ~ ~ ~ ~ ~ ~ ~ ~ ~ i-wa-sek=kt=o'sh\\
	1s-pro \textbf{pv.dir}-\textnormal{north}=loc \textnormal{sit}.aux=loc \textnormal{earth} ~ ~ ~ ~ ~ ~ ~ ~ ~ ~ ~ ~ ~ pv.ins-1a-\textnormal{make}=pot=ind.m\\
	\glt `as for myself, I'll make land that way to the north' \citep[9]{hollow1973a}

\item \glll máahsi máakahe rá'tseena ~ ~ ~ ~ ~ ~ ~ ~ káherekto'sh, \textbf{í}numa'ktaa\\
	wąąh\#si wąąkahE r'-at=s=ee=rą ~ ~ ~ ~ ~ ~ ~ ~ ka'\#hrE=kt=o'sh \textbf{i}-ruwą'k=taa\\
	\textnormal{arrow}\#\textnormal{feather} \textnormal{these} 2poss-\textnormal{father}=def=dem.dist=top ~ ~ ~ ~ ~ ~ ~ ~  \textnormal{have}\#caus=pot=ind.m \textbf{pv.dir}-\textnormal{man}=loc\\
	\glt `your father should give these eagle feathers away to the men' \citep[226]{hollow1973b}
	
\item \glll óo ó'harani \textbf{í}miisihąktaa máatah ~ ~ ~ ~ ~ ~ ~ ~ ~ ~ ~ ~ ~ ~ rukxą́hkereroomako'sh\\
	oo o'\#hrE=rį \textbf{i}-wįįsihąk=taa wąątah ~ ~ ~ ~ ~ ~ ~ ~ ~ ~ ~ ~ ~ ~ ru-kxąh=krE=oowąk=o'sh\\
	dem.mid \textnormal{be}\#caus=ss \textbf{pv.dir}-\textnormal{west}=loc \textnormal{river} ~ ~ ~ ~ ~ ~ ~ ~ ~ ~ ~ ~ ~ ~  ins.hand-\textnormal{ford}=3pl=narr=ind.m\\
	\glt `they crossed the river from there to the west' \citep[253]{hollow1973b}
	
\item \glll pkés \textbf{í}wara't ráahini érehka'ehe\\
	pke=s \textbf{i}-wra'=t rEEh=rį e-reh=ka'ehe\\
	\textnormal{turtle}=def \textbf{pv.dir}-\textnormal{fire}=loc \textnormal{go.there}=ss pv-\textnormal{think}=quot\\
	\glt `the turtle wanted to go to the fire, it is said' \citep[167]{hollow1973b}
	
\item \glll karóotiki roo numá'ks \textbf{í}rextaa ~ ~ ~ ~ ~ ~ ~ ~ ~ ~ ~ áareehkaroomako'sh\\
	ka=ooti=ki roo ruwą'k=s \textbf{i}-rex=taa ~ ~ ~ ~ ~ ~ ~ ~ ~ ~ ~ aa-rEEh=ka=oowąk=o'sh\\
	prov=evid=cond dem.mid \textnormal{man}=def  \textbf{pv.dir}-\textnormal{light}=loc ~ ~ ~ ~ ~ ~ ~ ~ ~ ~ ~ pv.tr-\textnormal{go.there}=hab=narr=ind.m\\
	\glt 	`and then he kept taking him towards the light' \citep[95]{hollow1973b}

\end{xlist}

\end{exe}

There is no specific postposition that indicates motion away from somewhere, but motion away is periphrastically marked using a demonstrative like \textit{oo} or \textit{roo} `that, there, then' followed by a causativized verb \textit{ó'} `be'. This construction always bears the same-subject switch-reference marker, and such it is likely that \textit{ó'harani} is really just a singular lexical item that can be treated like a unit to mean `from.'

Unlike most other preverbs, the directional \textit{i-} is very productive. It may appear on any stative verb, including verbalized nouns. 

\subsubsubsubsection{Instrumental preverb: \textit{i-}\textsubscript{2}}\label{SubParaInstrumental}

The instrumental preverb is the most common of all the preverbs, with an enormous amount of nouns and verbs lexically selecting for it. It is homophonous with the directional and ordinal preverbs. The distinction between PSi *i \textsc{pv.dir} and PSi *ii \textsc{pv.ins} has been lost in most daughter languages, with the vowel becoming short in all branches of Siouan except for Missouri Valley.

While this preverb may have introduced an instrumental non-core argument in Pre-Mandan, many instances of \textit{i-} have no obvious instrument, as we can see in (\ref{instrumentalexamples}) below. It is possible that the instruments are covert, given that Mandan is an aggressively {pro}-drop language, being able to omit subjects, direct objects, and indirect objects to be inferred by context.

\begin{exe}

\item\label{instrumentalexamples} Examples of instrumental \textit{i-}

	\begin{xlist}
	
%	\item numá'ką't mí'hąą waróotki'sh\\
%	ruwą'k=ą't wį'=hąą wa-rootki=o'sh\\
%	\textnormal{man}=dem.dist \textnormal{stone}=ins 

	\item\label{instrumentalexamples1} \glll Pt\'{ı̨}įmiihs tasúke ~ ~ ~ ~ ~ ~ ~ ~ ~ ~ ~ ~ ~ ~ ~ ~ ~ ~ ~ ~ wáarumixeena kú'rak \textbf{í}minixak\\
	ptįį\#wįįh=s ta-suk=E ~ ~ ~ ~ ~ ~ ~ ~ ~ ~ ~ ~ ~ ~ ~ ~ ~ ~ ~ ~  waa-ru-wrįx=ee=rą ku'=ak \textbf{i}-wrįx=ak\\
	\textnormal{buffalo}\#\textnormal{woman}=def al-\textnormal{child}=sv ~ ~ ~ ~ ~ ~ ~ ~ ~ ~ ~ ~ ~ ~ ~ ~ ~ ~ ~ ~  nom-ins.hand-\textnormal{be.circular}=dem.dist=top \textnormal{give}=ds \textbf{pv.ins}-\textnormal{play}=ds\\
	\glt `Cow Woman's child was playing with a hoop he had been given' \citep[112]{hollow1973a}
	
	\item\label{instrumentalexamples2} \glll náxihe, nitáxe'hąką't ~ ~ ~ ~ ~ ~ ~ ~  \textbf{í}wakikiishkekto're\\
	rą\#xih=E rį-ta-xe'h\#hąk=ą't ~ ~ ~ ~ ~ ~ ~ ~ \textbf{i}-wa-kikiishkE=kt=o're\\ 
	\textnormal{mother}.voc\#\textnormal{be.old}=sv 2poss-al-\textnormal{hang}\#pos.stnd=dem.anap ~ ~ ~ ~ ~ ~ ~ ~ \textbf{pv.ins}-1a-\textnormal{try}=pot=ind.f\\
	\glt `grandmother, I'll try it with that basket of yours' \citep[148]{hollow1973a}
	
	\item\label{instrumentalexamples3} \glll %wáapaksąhe míhkaraani roo ~ ~ ~ ~ ~ ~ ~ ~ ~~ \textbf{í}katehere \textbf{í}seks ~ ~ ~ ~ ~ ~ ~ 
	óo míihseena \textbf{í}kiri hų́ ~ ~ ~ ~ \textbf{í}rushaahaa ų́'shkaharani\\
	%waa-pa-ksąh=E wįk=krE=rį roo ~ ~ ~ ~ ~ ~ ~ ~ ~  \textbf{i}-ka-te\#hrE \textbf{i}-sek=s ~ ~ ~ ~ ~ ~ 
	oo wįįh=s=ee=rą \textbf{i}-kri hų ~ ~ ~ ~ \textbf{i}-ru-shE=haa ų'sh=ka\#hrE=rį\\
	%nom-ins.push-\textnormal{be.worried}=sv \textnormal{be.none}=3pl=ss dem.mid ~ ~ ~ ~ ~ ~ ~ ~ ~ \textbf{pv.ins}-ins.frce-\textnormal{pound}\#caus \textbf{pv.ins}-\textnormal{make}=def ~ ~ ~ ~ ~ ~ ~ ~ ~ ~ ~ ~ 
	dem.mid \textnormal{woman}=def=dem.dist=top \textbf{pv.ins}-\textnormal{be.grease} \textnormal{be.many} ~ ~ ~ ~ \textbf{pv.ins}-ins.hand-\textnormal{grasp}=sim \textnormal{be.thus}=hab\#caus=ss\\
	\glt `The girl there was mixing it like this with a lot of tallow' \citep[208]{hollow1973b}
	
	\item\label{instrumentalexamples4} \glll 
	ą́ąwe wáa'\textbf{i}wahekinixo'sh\\
	ąąwe wa\textbf{a}-i-wa-hek=rįx=o'sh\\
	\textnormal{all} neg-\textbf{pv.ins}-1a-\textnormal{know}=neg=ind.m\\
	\glt 	`I don't know all of it' \citep[47]{hollow1973a}
	
	\item\label{instrumentalexamples5} \glll 
	weréh ų́ųptaa xték \textbf{í}mitaarak numá'ks ~ ~ ~ ~ ~ ~ ~ ~ ~ ~ rá'ke'ho're\\
	wreh ųųptaa xtE=ak \textbf{i}-wį-taa=ak ruwą'k=s ~ ~ ~ ~ ~ ~ ~ ~ ~ ~ ra'-ke'h=o're\\
	\textnormal{door} \textnormal{next.to} \textnormal{be.big}=ds \textbf{pv.ins}-1s-\textnormal{peek}=ds \textnormal{man}=def ~ ~ ~ ~ ~ ~ ~ ~ ~ ~  ins.heat-\textnormal{be.angry}=ind.f\\
	\glt `I peeked in right next to the big door and the man got mad' \citep[98]{hollow1973a}
	
	\item\label{instrumentalexamples6} \glll 
	{mí'se~\textbf{í}hįįk}\\
	wį'\#sE\#\textbf{i}-hįį=k\\
	\textnormal{stone}\#\textnormal{red}\#\textbf{pv.ins}-\textnormal{drink}=hab\\
	\glt `Catlinite' (lit. `red pipe stone') \citep[439]{hollow1970}
	
	\item\label{instrumentalexamples7} \glll maná~\textbf{í}kawesh\\
	wrą\#\textbf{i}-ka-wesh\\
	\textnormal{wood}\#\textbf{pv.ins}-ins.frce-\textnormal{cut}\\
	\glt `axe' (lit. `chop wood with it') \citep[439]{hollow1970}
	
	\item\label{instrumentalexamples8} \glll w\textbf{íi}pashih\\
	waa-\textbf{i}-pa-shih\\
	nom-\textbf{pv.ins}-ins.push-\textnormal{be.sharp}\\
	\glt `a file' (lit. `something that makes it sharp') \citep[439]{hollow1970}
	
	\end{xlist}

\end{exe}

The instrumental preverbs in (\ref{instrumentalexamples1}) and (\ref{instrumentalexamples2}) take overt instruments, while only one in (\ref{instrumentalexamples3}) has an overt instrument, i.e., a lot of tallow to make the pemmican. The word `tallow' also involves an instrumental preverb, though the preverb serves no other purpose but to nominalize the verb `be grease.' The \textit{i-} in (\ref{instrumentalexamples4}) and (\ref{instrumentalexamples5}) lack any clear possible instrument, which suggests that these verbs are lexically selecting for this preverb for some historical reason that is no longer transparent. We see the use of \textit{i-} as a nominalizer again in (\ref{instrumentalexamples6}) through (\ref{instrumentalexamples8}).

\subsubsubsubsection{Ordinal preverb: \textit{i-}\textsubscript{3}}\label{SubParaOrdinal}

The ordinal preverb \textit{i-} has a limited scope of usage. Its sole use is to turn a cardinal number into ordinals. We see examples of its use in (\ref{ordinalexamples}) below.

\begin{exe}
\item\label{ordinalexamples} Examples of ordinal \textit{i-}

	\begin{xlist}

	\item \glll kit\'{ı̨}hka \textbf{í}naaminihak órookere ~ ~ ~ ~ ~ ~ ~ áanakoomako'sh\\
	ki-tįh=ka \textbf{i}-raawrį\#ha=ak o-roo=krE ~ ~ ~ ~ ~ ~ ~ E\#rąk=oowąk=o'sh\\
	mid-\textnormal{stick.out}=hab \textbf{pv.ord}-\textnormal{three}\#\textnormal{times}=ds pv.irr-\textnormal{talk}=3pl ~ ~ ~ ~ ~ ~ ~  \textnormal{hear}\#pos.sit=narr=ind.m\\
	\glt `it became clearer the third time he heard what they were saying' \citep[108]{hollow1973a}

	
	\item \glll \textbf{í}teetoki\\
	\textbf{i}-teetoki\\
	\textbf{pr.ord}-\textnormal{eight}\\
	\glt `the eighth' \citep[439]{hollow1970}
	
	\item \glll \textbf{í}nupha\\
	\textbf{i}-rųp\#ha\\
	\textbf{pr.ord}-\textnormal{two}\#\textnormal{times}\\
	\glt `the second time' \citep[440]{hollow1970}
	
	\item \glll hą́p \textbf{í}kixųh\\
	hąp\textbf{ }i-kixųh\\
	\textnormal{day} \textbf{pv.ord}-\textnormal{five}\\
	\glt `Friday (lit. `fifth day')' \citep[439]{hollow1970}

	\end{xlist}

\end{exe}

Other Siouan languages, such as Hidatsa \citep{park2012} and Lakota \citep{ullrich2011}, describe ordinal numbers as being constructed by adding the instrumental preverb with the cardinal number. Its widespread use to ordinalize cardinal numbers makes it clear that this is a pan-Siouan characteristic of some Proto-Siouan element, though it is not clear if this preverb evolved from PSi *ii- \textsc{pv.ins}, PSi *i- \textsc{pv.dir}, or some possible third item that has yet to be reconstructed. The ordinalizer \textit{i-} is therefore classed as its own preverb here in order to highlight the semantic contrast it has with the other polysemous \textit{i-} preverbs. 

\subsubsubsubsection{Possessive preverb: \textit{i-}\textsubscript{4}}\label{SubParaPossessive}

The final polysemous \textit{i-} preverb is the possessive. The possessive \textit{i-} primarily marks some established complex noun where the second element is marked as being inalienably possessed by the first. There are also a few idiomatic uses of the possessive \textit{i-} where the first noun has been elided but still understood, as we can see in (\ref{possessivepreverb}) below.

\begin{exe}
\item\label{possessivepreverb} Examples of possessive \textit{i-}

	\begin{xlist}
	
	\item \glll pó~\textbf{í}shųt\\
	po\#\textbf{i}-shųt\\
	\textnormal{fish}\#\textbf{pv.poss}-\textnormal{tail}\\
	\glt `fish tail' \citep[438]{hollow1970}
	
	\item \glll núutka~\textbf{í}hįse\\
	rųųtka\#\textbf{i}-hįs=E\\
	\textnormal{throat}\#\textbf{pv.poss}-\textnormal{long.muscle}=sv\\
	\glt `sternocleidomastoid muscle' \citep[75]{hollow1970}
	
	\item \glll tamáshka~\textbf{í}pa\\
	ta-wąshka\#\textbf{i}-pa\\
	al-\textnormal{breast}\#\textbf{pv.poss}-\textnormal{head}\\
	\glt `nipple' \citep[142]{hollow1970}
	
	\item \glll rók~\textbf{í}wahuu\\
	rok\#\textbf{i}-wa-huu\\
	\textnormal{leg}\#\textbf{pv.poss}-unsp-\textnormal{bone}\\
	\glt `thigh bone' \citep[187]{hollow1970}
	
	\item \glll mató~\textbf{í}werook\\
	wąto\#\textbf{i}-wrook\\
	\textnormal{bear}\#\textbf{pv.poss}-\textnormal{male.animal}\\
	\glt `male bear' \citep[306]{hollow1970}
	
	\item \glll psh\'{ı̨}įxaa~\textbf{í}miihka\\
	pshįįxaa\#\textbf{i}-wįįh=ka\\
	\textnormal{sage}\#\textbf{pv.poss}-\textnormal{woman}=hab\\
	\glt `female sage plant' \citep[286]{hollow1970}
	
	\item \glll \textbf{í}pirak\\
	\textbf{i}-pirak\\
	\textbf{pv.poss}-\textnormal{ten}\\
	\glt `tribal council' (lit. `those of the ten') \citep[420]{hollow1970}
	
	\item \glll \textbf{í}numa'k\\
	\textbf{i}-ruwą'k\\
	\textbf{pv.poss}-\textnormal{man}\\
	\glt `paterfamilias' \citep[438]{hollow1970}
	
	\end{xlist}

\end{exe}

Each of the compounds above shows that the initial noun is the possessor, and that the second noun is tied to the first. In the case of \textit{ípirak} `tribal council' and \textit{ínuma'k} `paterfamilias', the possessor has been omitted.

This preverb originates from the third person inalienable possessor prefix *i- in Proto-Siouan. This preverb is likewise used to denote inalienable possession in Mandan, though it is no longer productive. In some other Siouan languages like Hidatsa (\citeauthor{boyle2007} p.c.), there are some nouns where the Proto-Siouan third person possessive prefix *i- has been reanalyzed as part of the stem, causing irregular possessive marking. A limited number of such nouns also exists in Mandan, where what used to be a possessive is now considered part of the root, e.g., the *i in PSi *išta `eye' > Mandan \textit{istá} `face'.  Possessive marking in Mandan, however, is quite regular. Third person possession in Mandan is not morphologically marked, aside from cases where possession is intrinsic, as we see in the case of the compounds above.

There are some vestiges of PSi *i- \textsc{3poss} in independent pronominal constructions like the ones we see below in (\ref{possessivearchaisms}).

\begin{exe}

\item\label{possessivearchaisms} Fossilized remains of PSi *i- \textsc{3poss}

	\begin{xlist}
	
	\item\label{possessivearchaisms1} \glll í'o'na\\
	i-o'=rą\\
	pv.poss-\textnormal{be}=top\\
	\glt `as for him/her/them...' \citep[88]{hollow1970}
	
	\item\label{possessivearchaisms2} \glll mí'o'na\\
	w'-i-o'=rą\\
	1s-pv.poss-\textnormal{be}=top\\
	\glt `as for me' \citep[244]{hollow1973b}
	
	\item\label{possessivearchaisms3} \glll í'o'rak\\
	i-o'=ak\\
	pv.poss-\textnormal{be}=ds\\
	\glt `he/she/they is/are the one(s)' \citep[88]{hollow1970}
	
	\item\label{possessivearchaisms7} \glll ée'o'rak\\
	ee\#o'=ak\\
	dem.dist\#\textnormal{be}=ds\\
	\glt `he is the one' \citep[212]{hollow1973a}
	
	\item\label{possessivearchaisms4} \glll ní'o'rak\\
	r'-i-o'=ak\\
	2s-pv.poss-\textnormal{be}=ds\\
	\glt `you are the one' \citep[105]{hollow1973a}
	
	\item\label{possessivearchaisms5} \glll ishák\\
	ishak\\
	\textnormal{pro}\\
	\glt `he/she/it/they' \citep[1]{hollow1973a}
	
%	\item\label{possessivearchaisms6} \glll mí'shak\\
%	w-ishak\\
%	1poss-pro\\
%	\glt `I/myself' \citep[128]{hollow1973a}
	
	\end{xlist}

\end{exe}

The examples in (\ref{possessivearchaisms1}) and (\ref{possessivearchaisms3}) above utilize the possessive preverb when making a pronominal-type construction, but switch back to typical first and second person possessives for marking pronouns in (\ref{possessivearchaisms2}) and (\ref{possessivearchaisms4}). In the bare pronoun \textit{ishák}, we see a stem-initial [i] that is a fossilized third person possessive. The fact that it does not shift stress to the first syllable indicates that speakers no longer treat it as an analyzable unit within the overall lexical item, but its origin is clearly from PSi *i-. Since *i- is no longer productive as a general third person possessive marker, there is also a competing form with the distal demonstrative \textit{ee} being incorporated into the pronominal construction in lieu of the possessive preverb. Constructions with \textit{ee}, like the one in (\ref{possessivearchaisms7}), are more common in the corpus than those with the preverb \textit{i-}, suggesting that the forms with \textit{i-} are more archaic.

\subsubsubsubsection{Reflexivizer preverb: \textit{į'-}}\label{SubParaReflexivizer}

The reflexivizer \textit{į'-} is very uncommon, and appears on only a few verbs. Its origin is unclear, as it has no parallels to other Siouan preverbs or pronominals. There are several body parts pertaining to the face and head that begin with /į/ or /įį/, so it is possible that this preverb is a contracted version of one of these nouns that has been reanalyzed as having a reflexive meaning. This preverb is most commonly seen on the auxiliary \textit{\'{ı̨}'here} `become, pretend', consisting of the causative \textit{heré} along with the reflexivizer preverb, and can also be used when one is causing something to happen to oneself or something owned by oneself. We never see the reflexive prefix \textit{ki-} used with the causative. We can see that \textit{į'-} functions like other preverbs in the examples in (\ref{reflexivizerexamples}) below.

\begin{exe}

\item\label{reflexivizerexamples} Examples of reflexivizer \textit{į'}-

	\begin{xlist}
	
		
	\item \glll \textbf{\'{ı̨}'}mikihąąxiko'sh\\
	\textbf{į'}-wį-ki-hąąxik=o'sh\\
	\textbf{pv.rflx}-1s-rflx-\textnormal{not.know}=ind.m\\
	\glt `I fainted' \citep[170]{trechter2012b}
	
		\item \glll táani tasúk \textbf{\'{ı̨}'}tuherék\\
	tE=rį ta-suk \textbf{į'}-tu\#hrE=ak\\
	\textnormal{stand}=ss al-\textnormal{child} \textbf{pv.rflx}-\textnormal{be.some}\#caus=ds\\
	\glt `she stood there and gave birth to her child' \citep[111]{hollow1973a}
	
	\item \glll maná kashíhs ké'ka'ni ée o'haraa réesiks wá'shkap \textbf{\'{ı̨}'}harani\\
	wrą ka-shih=s ke'=ka'=rį ee o'=hrE=$\varnothing$ reesik=s wa'-shkap \textbf{į'}-hrE=rį\\
	\textnormal{wood} ins.frce-\textnormal{be.sharp}=def \textnormal{keep}\#\textnormal{have}=ss dem.dist \textnormal{be}\#caus=cont \textnormal{tongue}=def ins.prce-\textnormal{prick} \textbf{pv.rflx}-caus=ss\\
	\glt `he had been keeping the sharp stick and with that he pinched his own tongue' \citep[191]{hollow1973a}
	
	\item \glll ų́'sh rushá pawésh \textbf{\'{ı̨}'}hereroomako'sh\\
	ų'sh ru-shE pa-wesh \textbf{į'}-hrE=oowąk=o'sh\\
	\textnormal{thus} ins.hand-\textnormal{grasp} ins.push-\textnormal{cut} \textbf{pv.rflx}-caus=narr=ind.m\\
	\glt `so he took it and then he pretended to cut it' \citep[191]{hollow1973a}
	
	\item \glll Numá'k Máxana níinami \textbf{\'{ı̨}'}kahekoomaksįh\\
	ruwą'k wąxrą rįį=awį \textbf{į'}-ka-hek=oowąk=sįh\\
	\textnormal{man} \textnormal{one} \textnormal{walk}=cont \textbf{pv.rflx}-incp-\textnormal{know}=narr=ints\\
	\glt `Lone Man was walking along and became aware of himself' \citep[5]{hollow1973a}
	
	\item \glll ítopsha íhaa'aakit ~ ~ ~ ~ ~ ~ ~ ~ ~ ~ ~ ~ ~ ~ ~ keréehkereroomako'sh, kixkék \textbf{\'{ı̨}'}harani\\
	i-top-sha i-haa\#aaki=t ~ ~ ~ ~ ~ ~ ~ ~ ~ ~ ~ ~ ~ ~ ~  krEEh=krE=oowąk=o'sh ki-kxek \textbf{į'}-hrE=rį\\
	pv.poss-\textnormal{four}-coll pv.dir-\textnormal{cloud}\#\textnormal{be.above}=loc ~ ~ ~ ~ ~ ~ ~ ~ ~ ~ ~ ~ ~ ~ ~  \textnormal{go.back.there}=3pl=narr=ind.m mid-\textnormal{star} \textbf{pv.rflx}-caus=ss\\
	\glt `All four of them returned to heaven, having turned into stars' \citep[175]{hollow1973a}
	

%	
%	\item \glll \uline{\'{ı̨}'}mikareehirit\\
%	į'-wį-ka-reehrit\\
%	pv.rflx-1poss-ins.frce-\textnormal{fan}\\
%	\glt `my fan' \citep[87]{hollow1970}

	
\end{xlist} 

\end{exe} 

This preverb is not productive in contemporary Mandan, with the reflexive prefix \textit{ki-} being the most typical realization of reflexivity.

\subsubsubsubsection{Locative preverb: \textit{o-}\textsubscript{1}}\label{SubParaLocative}

The locative preverb \textit{o-} has cognates in all Siouan languages, and is a reflex of PSi *o- \textsc{pv.ines}, where it carried an inessive meaning. Several other Siouan languages still preserve the semantics of the inessive preverb, indicating an action into a particular place. In Mandan, this preverb bears a more generalized locative reading, and is often used to create relative clauses to describe where an action is taking place. We can see examples of this preverb in (\ref{locativeexamples}) below.

\newpage
\begin{exe}

\item\label{locativeexamples} Examples of locative \textit{o-}

	\begin{xlist}
	
	\item \glll páaxu~\textbf{ó}hop\\
	paaxu\#\textbf{o}-hop\\
	\textnormal{nose}\#\textbf{pv.loc}-\textnormal{be.hollow}\\
	\glt `nostril' \citep[77]{hollow1970}
	
	\item \glll miní~\textbf{ó}ropxe\\
	wrį\#\textbf{o}-ropxE\\
	\textnormal{water}\#\textbf{pv.loc}-\textnormal{enter}\\
	\glt 	`bathtub' \citep[189]{hollow1970}
	
	\item \glll istámi'~\textbf{ó}sanake\\
	istawį\textbf{\#}o-srąk=E\\
	\textnormal{eye}\#\textbf{pv.loc}-\textnormal{be.round}\\
	\glt `eyeball' \citep[216]{hollow1970}
	
	\item \glll \textbf{ó}wati\\
	\textbf{o}-wa-ti\\
	\textbf{pv.loc}-1a-\textnormal{reside}\\
	\glt `my house' \citet[251]{hollow1970}
		
	\item \glll súks xamáha shí \textbf{ó}kashųka ~ ~ ~ ~ ~ ~ ~ ~ ~ ~  máaptet\\
	suk=s xwąh=E=$\varnothing$ shi \textbf{o}-ka-shųk=E ~ ~ ~ ~ ~ ~ ~ ~ ~ ~  wąąpte=t\\
	\textnormal{child}=def \textnormal{little}=sv=cont \textnormal{foot} \textbf{pv.loc}-ins.frce-\textnormal{hang}=sv ~ ~ ~ ~ ~ ~ ~ ~ ~ ~ \textnormal{river.bank}=loc\\
	\glt `the kids' feet were hanging over the river bank a little bit' \citep[181]{hollow1970}
	
	\item \glll wáa'iparaare ímikak roo ų́'sh ~ ~ ~ ~ ~ ~  mí'reenus warúshani ~ ~ ~ ~ ~ ~ ~ ~ ~ ~ ~ re'éshkawahara íwarootkik \textbf{ó}seroopo're\\
	waa-i-praa=E i-wįk=ak roo ų'sh ~ ~ ~ ~ ~ ~  wį'=ee=rų=s wa-ru-shE=rį ~ ~ ~ ~ ~ ~ ~ ~ ~ ~ ~ re-eshka\#wa-hrE=$\varnothing$ i-wa-rootki=ak \textbf{o}-sroop=o're\\
	nom-pv.ins-\textnormal{be.big}=sv pv.ins-\textnormal{be.none}=ds dem.mid \textnormal{be.thus} ~ ~ ~ ~ ~ ~  \textnormal{stone}=dem.dist=anf=def 1a-ins.hand-\textnormal{grasp}=ss ~ ~ ~ ~ ~ ~ ~ ~ ~ ~ ~ dem.prox-smlt\#caus pv.dir-1a\textnormal{strike}=ds \textbf{pv.loc}-\textnormal{swallow}=ind.f\\
	\glt `It sure was big, so I took the aforementioned stone just like this one and I put it in his mouth, and he swallowed it' \citep[99]{hollow1973a}
	
	\item \glll tawáa'irukiriihs ~ ~ ~ ~ ~ ~ ~ ~ ~ ~ ~ ~ ~ ~ ~ ~ ~  \textbf{ó}ptikanashini máhki, ishák, ~ ~ ~ ~ ~ ~ ~ ~ ~  Kinúma'kshis, kiwarátanashoomaks\\
	ta-waa-i-ru-kriih=s ~ ~ ~ ~ ~ ~ ~ ~ ~ ~ ~ ~ ~ ~ ~ ~ ~ \textbf{o}-ptik=rąsh=rį wąk=ki ishak ~ ~ ~ ~ ~ ~ ~ ~ ~  ki-ruwą'k\#shi=s ki-wrat=rąsh=oowąk=s\\
	al-nom-pv.ins-ins.hand-\textnormal{be.lined.up}=def ~ ~ ~ ~ ~ ~ ~ ~ ~ ~ ~ ~ ~ ~ ~ ~ ~ \textbf{pv.loc}-\textnormal{have.fallen.down}=att=ss pos.lie=cond pro ~ ~ ~ ~ ~ ~ ~ ~ ~  mid-\textnormal{man\#be.good}=def mid-\textnormal{dirt}=att=narr=def\\
	\glt `His staff had fallen down when he was laying there, [for] he, First Creator, had turned into dirt' \citep[8]{hollow1973a}
	
	\item \glll są́ąka r\textbf{óo}napini ą́ąwe nurúha'ni ~ ~ nukúho'sh\\
	sąąka rV-\textbf{o}-rąp=rį ąąwe rų-ru-ha'=rį ~ ~  rų-kuh=o'sh\\
	\textnormal{be.few} 1a.pl-\textbf{pv.loc}-\textnormal{find}=ss \textnormal{all} 1a.pl-ins.hand-\textnormal{pick.berries}=ss ~ ~ 1a.pl-\textnormal{come.back.here}=ind.m\\
	\glt `we found a few, picked them all, and came back' \citep[52]{hollow1970}

	\end{xlist}

\end{exe}

This preverb mirrors the possessive \textit{i-} and instrumental \textit{i-} in that it is used in compounds to express a relationship between two nominal elements. The locative preverb can likewise create a relative clause that is treated like a noun, e.g., \textit{ówati} `my house' (lit. `where I live'). %The word `nostril' in Mandan is \textit{páaxu óhop}, literally `where the nose is hollow.' 

\subsubsubsection{Irrealis preverb (Slot 7): \textit{o-}\textsubscript{2}}\label{ParaIrrealis}

While there are a large number of applicative preverbs, they can be preceded by the irrealis preverb \textit{o-}. In previous grammatical sketches of Mandan, this preverb has been treated as a future marker \citep{kennard1936,hollow1970,mixco1997a}. This preverb is found outside of contexts where there is no future reading, as we can see in (\ref{irrealisnotfuture}) below.

\begin{exe}

\item\label{irrealisnotfuture} Non-future use of \textit{o-}

\begin{xlist}

\item\label{irrealisnotfuture1} 
	\glll Róoniire írasiinitki, ~ ~ ~ ~ ~ ~ ~ ~ ~ ~ ~ ~ ~ \textbf{ó}rakiikirixinitą't, tashká'eshkak roo ~ ~ ~ ~ ~ résh nanúhinito'sha?\\
	rV-o-rįį=E i-ra-sii=rįt=ki ~ ~ ~ ~ ~ ~ ~ ~ ~ ~ ~ ~ ~  \textbf{o}-ra-kiikrix=rįt=ą't tashka-eshka=ak roo ~ ~ ~ ~ ~ resh ra-rųh=rįt=o'sha\\
	1a.pl-pv.loc-\textnormal{walk}=sv pv.dir-2a-\textnormal{travel}=2pl=cond  ~ ~ ~ ~ ~ ~ ~ ~ ~ ~ ~ ~ ~  \textbf{pv.irr}-2a-\textnormal{catch.up.with}=2pl=hyp \textnormal{how}-smlt=ds dem.mid ~ ~ ~ ~ ~ \textnormal{this.way} 2a-\textnormal{be.here}=2pl=int.m\\
\glt 	`If you (pl.) had followed our tracks, you might have caught up with us, so how come you're still here like this?' \citep[208]{hollow1973a}

\item\label{irrealisnotfuture2} \glll ``roo wakxų́hki \textbf{ó}'iraheką't'' éeheni ~ Kinúma'kshi kxų́hoomako'sh\\
    roo wa-kxųh=ki \textbf{o}-i-ra-hek=ą't ee-he=rį ~ ki-ruwą'k\#shi kxųh=oowąk=o'sh\\
	dem.mid 1a-\textnormal{lie.down}=hyp \textbf{pv.irr}-pv.ins-2a-\textnormal{know}=cons pv-\textnormal{say}=ss ~ mid-\textnormal{man\#be.good} \textnormal{lie.down}=narr=ind.m\\
	\glt {`}{``}If I lie down here, you would know,'' he said and First Creator lay down.' \citep[1]{hollow1973a}
	
	\item\label{irrealisnotfuture3} \glll ishák ítaa \textbf{ó}rushenikini ~ ~ ~ ~ ~ ~ ~ ~ ~ ~ kí'hka'eheroo\\
	ishak i-tE \textbf{o}-ru-shE=rįk=rį ~ ~ ~ ~ ~ ~ ~ ~ ~ ~  ki'h=ka'ehe=oo\\
	pro pv.ins-\textnormal{stand} \textbf{pv.irr}-ins.hand-\textnormal{grasp}=itr=ss ~ ~ ~ ~ ~ ~ ~ ~ ~ ~  \textnormal{arrive.back.here}=quot=dem.mid\\
	\glt `He was tired and could not take any more, so he got back, it is said now.' \citep[124]{hollow1973a}
	
	\end{xlist}
	
\end{exe}

We see conditional constructions in (\ref{irrealisnotfuture1}) and (\ref{irrealisnotfuture2}), where \textit{o-} is found following a conditional clause. Furthermore, we can see in (\ref{irrealisnotfuture2}) that the \textit{o-} is able to precede other preverbs, as in \textit{ó'iraheką't} `you would know.' In cases where both \textit{o-} preverbs appear, they are realized as a single syllable with a long [oː], as we can see in (\ref{multipleOpreverbs}) below.

\begin{exe}
\item\label{multipleOpreverbs} Instances of sequential \textit{o-} preverbs

	\begin{xlist}
	
%	\item \glll \uline{óo}rushųko'sh\\
%	o-o-ru-shųk=o'sh\\
%	pv.irr-pv.loc-ins.hand-\textnormal{hang}=ind.m\\
%	\glt `it will hang down' \citep[5]{kennard1936}
	
	\item \glll \textbf{óo}kaptiko'sh\\
	\textbf{o}-\textbf{o}-ka-ptik=o'sh\\
	\textbf{pv.irr}-\textbf{pv.loc}-ins.frce-\textnormal{have.fallen.down}=ind.m\\
	\glt `he will shoot it down' \citep[5]{kennard1936}
	
	\item \glll \textbf{óo}wakakąko'sh\\
	\textbf{o}-\textbf{o}-wa-ka-kąk=o'sh\\
	\textbf{pv.irr}-\textbf{pv.loc}-ins.frce-\textnormal{be.tight}=ind.m\\
	\glt `I will be mired' \citep[5]{kennard1936}
	
	\end{xlist}

\end{exe}

The use of \textit{o-} in (\ref{irrealisnotfuture3}) carries an even more unambiguously modal reading, rather than a temporal one. The narrator is describing a situation in the past and uses \textit{o-} despite that lack of any future reading. We can also see that \textit{o-} is not required in conditional constructions, given the data below, where only the first example bears \textit{o}-, yet all subsequent examples still have a conditional reading, as we can see in (\ref{ifconstructions}) below.

The first two examples above feature the irrealis preverb \textit{o-}, while the second pair do not. All of the data contain conditionals constructions that imply some future consequence to a conditional clause. The irrealis preverb can certainly used to give future readings to an action or state, but it can also be used to mark hypothetical situations, as we see in (\ref{ifconstructions}) below. 


\begin{exe}

\item\label{ifconstructions} Conditional constructions with and without \textit{o-}

	\begin{xlist}
	
	\item \glll tópha wahúuki, ų́'ka miní r\textbf{óo}ropxe're\\
	top\#ha wa-huu=ki ų'ka wrį r\textbf{V}-o-ropxE=o're\\
	\textnormal{four\#times} 1a-\textnormal{come.here}=cond \textnormal{then} \textnormal{water} 1a.pl-\textbf{pv.irr}-\textnormal{enter}=ind.f\\
	\glt `if I come four times, then we'll go swimming' \citep[106]{hollow1973a}
	
	\item \glll wáarahuunixki \textbf{ó}xiko'sh\\
	waa-ra-huu=rįx=ki \textbf{o}-xik=o'sh\\
	neg-2a-\textnormal{come.here}=neg=cond \textbf{pv.irr}-\textnormal{be.bad}=ind.m\\
	\glt `if you don't come, it will be bad' \citep[53]{hollow1973b}
	
	
	\item \glll tashká, waheréki, taté rásą't ~ ~ ~ ~ ~ ~ ~ ~ ~ ~ kit\'{ı̨}ho'xere\\
	tashka wa-hrE=ki tatE ras=ą't ~ ~ ~ ~ ~ ~ ~ ~ ~ ~ ki-tįh=o'xrE\\
	\textnormal{how} 1a-caus=cond \textnormal{father}.voc \textnormal{name}=dem.anap ~ ~ ~ ~ ~ ~ ~ ~ ~ ~ mid-\textnormal{stick.out}=dub\\
	\glt `how will my father's name come out if I do it?' \citep[61]{hollow1973a}
	
	\item \glll kotewé ų́ųte rupáskįhki ~ ~ ~ ~ ~ ~ ~  tapt\'{ı̨}įkto'sh\\
	ko-t-we ųųt=E ru-pa-skįh=ki ~ ~ ~ ~ ~ ~ ~  ta-ptįį=kt=o'sh\\
	rel-wh-indf \textnormal{be.first}=sv ins.hand-ins.push-\textnormal{cut.open}=cond ~ ~ ~ ~ ~ ~ ~  al-\textnormal{buffalo}=pot=ind.m\\
	\glt `if someone slashes it first, it will be his buffalo' \citep[7]{hollow1973b}
	
	\end{xlist}

\end{exe}


Another use for the irrealis preverb is to create relative clauses and nominalize verbs in a similar manner to the locative and instrumental preverbs. We can see examples of this behavior in (\ref{relativeclauseirrealis}) below.

\begin{exe}
\item\label{relativeclauseirrealis} Relativization and nominalization with \textit{o-}

	\begin{xlist}
	
	\item \glll \textbf{ó}'į'tu\\
	\textbf{o}-į'-tu\\
	\textbf{pv.irr}-pv.rflx-\textnormal{be.some}\\
	\glt `birth, birthday (lit. `when one is born')' \citep[96]{hollow1970}
	
%	\item \glll \uline{ó}kape\\
%	o-kap=E\\
%	pv.irr-\textnormal{remain}=sv\\
%	\glt `remainder (' \citep[101]{hollow1970}
	
	\item \glll wáa'\textbf{o}kiraksąąmik\\
	waa-\textbf{o}-kiraksąą\#wįk\\
	neg-\textbf{pv.irr}-\textnormal{make.war}\#\textnormal{be.none}\\
	\glt `peace (lit. `when there is no war')' \citep[111]{hollow1970}
	
	\item \glll hą́sh~\textbf{ó}sek\\
	hąsh\#\textbf{o}-sEk\\
	\textnormal{grape}\#\textbf{pv.irr}-\textnormal{be.dry}\\
	\glt `partially raisined grapes (lit. `when the grape is dried out')' \citep[69]{hollow1970}
	
	\item \glll wáa'\textbf{o}haaxi\\
	waa-\textbf{o}-hE=xi\\
	neg-\textbf{pv.irr}-\textnormal{see}=neg\\
	\glt `an invisible thing (lit. `when one cannot see it')' \citep[71]{hollow1970}
	
	\item \glll súk~\textbf{ó}hųųkamik\\
	suk\#\textbf{o}-hųųka\#wįk\\
	\textnormal{child}\#\textbf{pv.irr}-\textnormal{parent}\#\textnormal{be.none}\\
	\glt `orphan (lit. `when a child is parentless')' \citep[83]{hollow1970}
	
	
	\end{xlist}

\end{exe}

Each of the tokens involving the irrealis preverb above is able to stand on its own as a relative clause, but these relative clauses are also able to then take nominal morphology such as articles, demonstratives, and postpositions. The irrealis preverb is highly productive in Mandan, and is frequently used to coin new words on the fly, or even as circumlocution for when a speaker does not remember a word but still wants to describe it. The literal nature of Mandan words means that there are often numerous possible ways to express a single concept, so the irrealis preverb frequently is employed to great effect.

\subsection{Inflectional prefixes}\label{Ch2InflectionalPrefixes}

Mandan makes heavy use of person and number marking on verbs. Unlike derivational prefixes, inflectional prefixes can also display a high degree of allomorphy. Most of these allomorphs are phonetically similar to a default formative, so it is likely that this allomorphy was at one time phonologically predictable at some point in the history of the language, either in Pre-Mandan or in the protolanguage shared between Mandan and Missouri River Siouan (i.e., Hidatsa and Crow). This reliance on heavily-affixing verbal elements allows Mandan to omit overt nominal arguments. Mandan can take {pro}-drop to the extreme at times. We can compare the two sentences in (\ref{prodropmandan}) below.

\begin{exe}

\item\label{prodropmandan} {Pro}-drop in Mandan

	\begin{xlist}
	
	\item \glll míiho'na\textnormal{\textsubscript{i}} páaxu shishíhka\textnormal{\textsubscript{j}} wará'nast\textnormal{\textsubscript{k}} ~ ~ ~ ~ ~  íkųųtekereroomako'sh\\
	wįįh=o'=rą paaxu shi$\sim$shih=ka wra'=rąt=t ~ ~ ~ ~ ~ i-kųųtE=krE=oowąk=o'sh\\
	\textnormal{woman}=\textnormal{be}=top \textnormal{nose} aug$\sim$\textnormal{be.sharp}=hab \textnormal{fire}=\textnormal{middle}=loc ~ ~ ~ ~ ~ pv.dir-\textnormal{throw}=3pl=narr=ind.m\\
	\glt `it was those women who threw the mosquito in the middle of the fire' \citep[153]{hollow1973b}
	
	\item \glll	\textnormal{pro\textsubscript{i}} \textnormal{pro\textsubscript{j}} \textnormal{pro\textsubscript{k}} íkųųtekereroomako'sh\\
	pro pro pro i-kųųtE=krE=oowąk=o'sh\\
	1a 3s 3s.loc pv.dir-\textnormal{throw}=3pl=narr=ind.m\\
	\glt `they threw it there'
	
	\end{xlist}

\end{exe}

The third person plural enclitic \textit{=kere} marks a third person plural subject. There is no third person singular marking in Mandan, but the directional preverb \textit{i-} indicates that there is a specific direction towards which the direct object is thrown. The presence of all this morphology supplies enough information that overt nominal arguments are not necessary to express the notion that the verb is ditransitive and what inflectional features these arguments have. Once a nominal element has been introduced into the discourse, subsequent references to it are typically elided. Subjects are most commonly dropped, with a system of switch-reference marking clarifying who is doing the action when both subjects are third person. Because of this strong preference for {pro}-drop, Mandan relies heavily upon context and inference, along with inflectional morphology, to convey who is doing an action, and who or what is undergoing said action.

We can divide the inflectional prefixes into two distinct groups: inner prefixes and outer prefixes. This distinction is drawn from the observation that certain prefixes will always appear after a preverb but before a verbal root (i.e., inner prefixes), while other prefixes appear before preverbs at the leftmost edge of the word (i.e., outer prefixes).

One major division between the kinds of prefixes is that pronominals in Mandan reflect the thematic role that an argument plays. Mandan has an active-stative alignment. Active marking typically indicates a semantic agent, i.e., someone who is undertaking an action. Stative marking is for arguments that lack any agency over the act. For example,  in some active-stative languages, the marking for subjects may either be active or stative, depending on the volitionality, e.g., `I coughed' may normally be stative, but if the speaker wishes to convey that this cough was intentionally and controllably done, the active may be used. 

Mandan is a split-S language, meaning its subjects are lexically split between active or stative. That is to say, a verb is lexically categorized for whether its subject takes active marking or stative marking. Some verbs that lack a semantic agent still take active marking, such as \textit{írukap} `be unable', though it is the case that no verbs with semantic agents take stative subjects. There are nine different inflectional prefixes, which are summarized in (\ref{mandaninflectionalprefixes}) below.

\begin{exe}
\item\label{mandaninflectionalprefixes} Inflectional prefixes in Mandan

	\begin{xlist}
	\item \textit{wa-}\textsubscript{1} first person singular active
        \item \textit{ma-} first person singular stative
        \item \textit{ra-} second person active
        \item \textit{nu-} first person plural active
        \item \textit{ro-} first person plural stative
        \item \textit{wa-}\textsubscript{2} unspecified argument stative
        \item \textit{ni-} second person stative 
        \textit{waa-}	negative
	\item \textit{ko-} relativizer
	\end{xlist}
\end{exe}

A description of the distribution and variation of these prefixes appears in the sections that follow.

\subsubsection{Second person prefix (Slot 4)}\label{SubsubSecondPerson}

When multiple inflectional prefixes occur on the same verbal stem, the one closest to the root will be the second person marker. Unlike first person, the second person prefix does not encode number. There are two main prefixes that indicate a second person argument:

\begin{exe}
\item\label{secondpersonmain} Default second person markers

\begin{xlist}
    \item \textit{ra-} second person active
    
    \item \textit{ni-} second person stative
\end{xlist}

\end{exe}

We can see examples of these prefixes at work in \sectref{Para2A} and \sectref{Para2S} below.

\subsubsubsection{Second person active prefix: \textit{ra}-}\label{Para2A}

The default marking for a second person active argument is the prefix \textit{ra}-. This prefix is a reflex of the Proto-Siouan second person active marker *ya-, since PSi *y merged with *r in Mandan and Missouri Valley Siouan. It is found frequently throughout the corpus, and we can see examples of this marker in (\ref{2Amarking}) below.

\begin{exe}

\item\label{2Amarking} Examples of second person active \textit{ra}-

	\begin{xlist}
	
	\item \glll áakahąktaahąą \textbf{ra}réehki, mí'shak, pasháhąktaahąą ~ ~ weréehto'sh\\
	aakahąk=taa=hąą \textbf{ra}-rEEh=ki w'-ishak pashahąk=taa=hąą ~ ~ we-rEEh=kt=o'sh\\
	\textnormal{south}=loc=ins \textbf{2a}-\textnormal{go.there}=cond 1poss-\textnormal{pro} \textnormal{north}=loc=ins ~ ~  1a-\textnormal{go.there}=pot=ind.m\\
	\glt `if you go to the south side, me, I'll go to the north side' \citep[3]{hollow1973a}
	
	\item \glll wáa'i\textbf{ra}seke shí'sh\\
	waa-i-\textbf{ra}-sek=E shi=o'sh\\
	nom-pv.ins-\textbf{2a}-\textnormal{make}=sv \textnormal{be.good}=ind.m\\
	\glt `what you made is good' \citep[11]{hollow1973a}
	
	\item \glll máa'ąke \textbf{ra}ké'ra ó\textbf{ra}kxųh hą́ąka ~ ~ ~ ~  íninaahki, ó'ųųka'sh\\
	wąą'ąk=E \textbf{ra}-ke'=E=$\varnothing$ o-\textbf{ra}-kxųh hą́ąkE=$\varnothing$ ~ ~ ~ ~  i-rį-rąą=ki o-ųųka=o'sh\\
	\textnormal{earth}=sv \textbf{2a}-\textnormal{dig}=sv=cont pv.irr-\textbf{2a}-\textnormal{lie.down} \textnormal{stand}.aux=cont ~ ~ ~ ~ pv.ins-2s-\textnormal{be.out.of.sight}=cond pv.irr-\textnormal{be.enough}=ind.m\\
	\glt `when you are out of sight, digging out a space as big as you when you lie down, that will be enough' \citep[25]{hollow1973a}
	
	\item \glll súkite, matewé í\textbf{ra}sekinito'sha?\\
	suk=rįt=E wa-t-we i-\textbf{ra}-sek=rįt=o'sha\\
	\textnormal{child}=2pl=sv unsp-wh-indf pv.ins-\textbf{2a}-\textnormal{make}=2pl=int.m\\
	\glt `children, what are you doing?' \citep[28]{hollow1973a}
	
	\item \glll tashká'eshka \textbf{ra}rátaxo'sha?\\
	tashka-eshka \textbf{ra}-ra-tax=o'sha\\
	\textnormal{how}-smlt \textbf{2a}-ins.mth-\textnormal{make.loud.noise}=int.m\\
	\glt `how come you are crying?' \citep[42]{hollow1973a}
	
	\item \glll wa\textbf{rá}raapiniirą't ké'ka'harata!\\
	wa-ra-raaprįį=ą't ke'\#ka'\#hrE=ta\\
	unsp-\textbf{2a}-\textnormal{be.around.neck}=dem.anap \textnormal{keep}\#\textnormal{have}\#caus=imp.m\\
	\glt `let him have that necklace of yours!' \citep[58]{hollow1973a}
	
	\item \glll nukeréehki, wa\textbf{rá}ruusto'sh\\
	rų-krEEh=ki wa-\textbf{ra}-ruut=kt=o'sh\\
	1a.pl-\textnormal{go.back.there}=cond unsp-\textbf{2a}-\textnormal{eat}=pot=ind.m\\
	\glt `when we get home, you can eat' \citep[87]{hollow1973a}
	
	\item \glll náxihe, ítewetaa \textbf{ra}réeho'na?\\
	rą\#xih=E i-t-we=taa\textbf{ r}a-rEEh=o'rą\\
	\textnormal{mother}.voc\#\textnormal{be.old}=sv pv.dir-wh-indf=loc \textbf{2a}-\textnormal{go.there}=int.f\\
	\glt `grandmother, where are you going?' \citep[89]{hollow1973a}
	
	\end{xlist}
\end{exe}

The overwhelming majority of second person active marking is carried out with the \textit{ra-} prefix, but there are three allomorphs for this formative: /r'-/, /re-/, and /rą-/. 

\subsubsubsubsection{Allomorph /r'-/}\label{rglottal}

An underlying /r'-/ allomorph appears with vowel-initial stems. The glottal stop then metathesizes with the following vowel and causes long vowels to contract, as described in \sectref{glottalstopmetathesis}. This is a completely predictable process for determining /ra-/ versus /r'-/. We can see examples of this allomorph in the data in (\ref{vowelinitial2A}) below.


\begin{exe}
\item\label{vowelinitial2A} Second person active marking before vowel-initial stems

	\begin{xlist}
	
	\item \glll ní'maare íkų'hąą ~ ~ ~ ~ ~ ~ ~ ~ ~ ~ ~ ~ ~ ~ ~ ~ ~ ~ tákraharani nitáxaraxeroo ~ ~ ~ ~ manúuxikpa \textbf{r}á'kisekto'sh\\
	r'-iwąą=E i-kų'=hąą ~ ~ ~ ~ ~ ~ ~ ~ ~ ~ ~ ~ ~ ~ ~ ~ ~ ~ tak\#ra-hrE=rį rį-ta-xrax=roo ~ ~ ~ ~  wa-rųų\#xik\#pa \textbf{r'}-aaki\#isek=kt=o'sh\\
	2poss-\textnormal{body}=sv pv.dir-\textnormal{be.all.over}=loc ~ ~ ~ ~ ~ ~ ~ ~ ~ ~ ~ ~ ~ ~ ~ ~ ~ ~  \textnormal{be.painted.with.white.clay}\#2a-caus=ss 2poss-al-\textnormal{chest}=dem.mid ~ ~ ~ ~ unsp-\textnormal{be.fog}\#\textnormal{be.bad}\#\textnormal{head} \textbf{2a}-\textnormal{be.above}\#pv.ins-\textnormal{make}=pot=ind.m\\
	\glt `you should paint your body all over with white clay and paint a skull on your chest' \citep[98]{hollow1973b}
	
	\item \glll \textbf{r}á's raráahaarami míihna'k íma'pet háa ná'ko'sh\\
	\textbf{r'}-as ra-rEEh=haa=awį wįįh\#rą'k i-wą'pe=t hE rą'k=ind.m\\
	\textbf{2a}-\textnormal{follow} 2a-\textnormal{go.there}=sim=cont \textnormal{woman}\#pos.sit pv.dir-\textnormal{below}=loc \textnormal{see} pos.sit=ind.m\\
	\glt `as you follow it while you keep going down, this woman sits looking below' \citep[82]{hollow1973a}
	
	\item \glll Óo o'harani \textbf{r}á'kani ``ptamí'tis ~ ~ ~ ~ í'ų'taa wakeréeho'sh,'' éetekto'sh\\
	oo o'\#hrE=rį \textbf{r'}-aakE=rį p-ta-wį'\#ti=s ~ ~ ~ ~ i-ų'=taa wa-krEEh=o'sh ee-te=kt=o'sh\\
	dem.dist \textnormal{be}\#caus=ss \textbf{2a}-\textnormal{step.on}=ss 1poss-al-\textnormal{stone}\#\textnormal{reside}=def ~ ~ ~ ~ pv.dir-\textnormal{be.closer}=loc 1a-\textnormal{go.back.there}=ind.m pv-\textnormal{say}.2s=pot=ind.m\\
	\glt `from there you should step on it and say, ``I am going back toward my village.''' \citep[227]{hollow1973b}
	
	\item\label{youarelost} \glll \textbf{r}í'mahąpo'sh\\
	\textbf{r'}-iiwahąp=o'sh\\
	\textbf{2a}-\textnormal{be.lost}=ind.m\\
	\glt `you are lost' \citep[96]{hollow1970}
	

	
	\end{xlist}

\end{exe}

As we see in (\ref{youarelost}), the semantics of the verb \textit{íimahąp} `be lost' should require a stative subject, but the verb only takes active marking. This particular verb is one of several that are lexically marked to take active subjects. The allophonic variation between /ra-/ and /r'-/ has a small set of exceptions, including the verb \textit{é} `hear' and the verb \textit{ú} `shoot, wound', where the /r'-/ allomorph is not never found in the corpus, as we see in (\ref{2Ahear}) below.

\begin{exe}
\item\label{2Ahear} Use of \textit{ra-} before vowel-initial stem 

	\begin{xlist}
	
	\item \glll waróore \textbf{ra}'éki, óshi'sh\\
	wa-roo=E \textbf{ra}-E=ki o-shi=o'sh\\
	unsp-\textnormal{talk}=sv \textbf{2a}-\textnormal{hear}=cond pv.irr-\textnormal{be.good}=ind.m\\
	\glt `if you hear what I say, it will be good' \citep[240]{hollow1973b}
	
	\item \glll \textbf{ra}'ú'sh\\
	\textbf{ra}-u=o'sh\\
	\textbf{2a}-\textnormal{wound}\\
	\glt `you wound him'
	
	\end{xlist}
	
\end{exe}

These verbal roots are both monosyllables with no onset, so it is the case that /r'-/ is only a viable allomorph of /ra-/ on vowel-initial stems that are polysyllabic.

\subsubsubsubsection{Allomorph /re-/}\label{allomorphRE}

The allomorph /re-/, as \citet[5]{kennard1936} describes it, is less predictable. In transcriptions of narratives provided by speakers born during the middle of the nineteenth century, /re-/ is sometimes used before verbal stems that begin with a sonorant and /e eː/. 

This trend towards local vowel harmony seems to be an incomplete change, as it not consistent, even within data given by the same speaker. This allomorph also does not occur when there is intervening morphology between the inflectional prefix and the verbal root, and it is quite rare in data collected from speakers born around the turn of the twentieth century or later. We can see examples of \textit{re-} instead of the expected \textit{ra-} in the examples in (\ref{ravariants}) below.

\begin{exe}
\item\label{ravariants} Use of \textit{re}- instead of \textit{ra}-
	
	\begin{xlist}
	
	\item \glll nihų́ųxihe hiré \textbf{re}réehki óo inák ~ ~ ~ ~ órahi're\\
	rį-hųų\#xih=E hire \textbf{re}-rEEh=ki oo irąk ~ ~ ~ ~ o-ra-hi=o're\\
	2poss-\textnormal{mother}\#\textnormal{be.old}=sv \textnormal{now} \textbf{2a}-\textnormal{go.there}=cond dem.mid \textnormal{again} ~ ~ ~ ~ pv.irr-2a-\textnormal{arrive.there}=ind.f\\
	\glt `if you go to your grandmother now, you will get there again' \citep[102]{hollow1973a}
	
	\item \glll ratóore, wáashi nuharánitak, ų́'sh ~ ~ ~ ~ ~ ~ ~ ~ ~ ~ ro\textbf{rá}rusanaahini \textbf{ra}ráahini ~ ~ ~ ~ ~ ~ ~ ~ ~ ~ ée\textbf{re}reho'sh\\
	ratoo=E waa-shi rų-hrE=rįt=ak ų'sh ~ ~ ~ ~ ~ ~ ~ ~ ~ ~ ro-\textbf{ra}-ru-srąąh=rį \textbf{ra}-rEEh=rį ~ ~ ~ ~ ~ ~ ~ ~ ~ ~ ee-\textbf{re}-reh=o'sh\\
	\textnormal{be.old}=sv nom-\textnormal{be.good} 1a.pl-caus=2pl=ds \textnormal{be.thus} ~ ~ ~ ~ ~ ~ ~ ~ ~ ~ 1s.pl-2a-ins.hand-\textnormal{leave.behind}=ss 2a-\textnormal{go.there}=ss ~ ~ ~ ~ ~ ~ ~ ~ ~ ~ pv-2a-\textnormal{think}=ind.m\\
	\glt 	`elder, we (pl.) are having a good time, so now you want to go and leave us' \citep[31]{hollow1973a}
	
	\item \glll wáashi \textbf{re}heré'sh\\
	waa-shi \textbf{re}-hrE=o'sh\\
	nom-\textnormal{be.good} \textbf{2a}-caus=ind.m\\
	\glt `thank you (lit. `you did something good')' \citep[35]{hollow1973a}
	
	\end{xlist}
	
\end{exe}

The default prefix \textit{ra-} is used almost exclusively instead of \textit{re-} in contemporary Mandan. The one exception to this tendency is the verb \textit{éereh} `want, think.' There are more instances of \textit{éerereh} for `you want' in the corpus than \textit{éerareh}, so it seems that this verb is developing into having a slightly irregular conjugation paradigm for some speakers, though both forms are valid.

\subsubsubsubsection{Allomorph /rą-/}

This allomorph of \textit{ra-} is only found when preceded by a first person singular stative prefix. The default first person singular stative prefix is /wą-/, though it has an allomorph of /w-/ before a second person active prefix (see \sectref{allomorphw}). This sequence is realized as [m\textsuperscript{ã}nã]. It is possible that the nasalization on the second person form is a remnant of the nasality on the first person singular stative /wą-/ during a time in the evolution of the Mandan language when the nasal vowel did not syncopate before /ra-/. Being in contact with a nasal vowel could have caused the /r/ in \textit{ra-} to have some nasality bleed over, which in turn caused it to be reanalyzed as /rą-/.

It is also possible that this process of /rą-/ developing as an allophone to /ra-/ after /wą-/ occurred in Pre-Mandan or some earlier stage of development. Progressive nasal harmony is common in other Siouan languages \citep{kasaklundquist2019}. Contemporary Mandan, however, exclusively engages in regressive nasal harmony instead of progressive nasal harmony, as previously outlined in \sectref{nasalharmony}. As such, this allomorph could be a holdover from some earlier system. We can see examples of this allomorph in (\ref{na2A}) below.

\begin{exe}
\item\label{na2A} Examples of /rą-/ instead of \textit{ra-}

	\begin{xlist}
	
	\item \glll \'{ı̨}'saara ma\textbf{na}ké'kakto'sh\\
	į'-saa=E=$\varnothing$ w-\textbf{rą}-ke'\#ka'=kt=o'sh\\
	pv.rflx-\textnormal{be.hurried}=sv=cont 1s-\textbf{2a}-\textnormal{keep}\#\textnormal{have}=hab=pot=ind.m\\
	\glt `you should keep me hurrying' \citep[223]{hollow1973a}
	
	\item \glll húutahak! ma\textbf{na}kíkųųtekto'sh\\
	huu=ta=hak w-\textbf{rą}-kikųųtE=kt=o'sh\\
	\textnormal{come.here}=imp.m=pol 1s-\textbf{2a}-\textnormal{help}=pot=ind.m\\
	\glt `please, come on! you can help me' \citep[41]{hollow1973a}
	
	\item \glll weréxanash írapawehaa ~ ~ ~ ~ ~ ~ ~ ~ ~ ~ ~ ~ ~ ~ ~ h\'{ı̨}įma\textbf{na}herekto're\\
	wrex=rąsh i-ra-pa-weh=haa ~ ~ ~ ~ ~ ~ ~ ~ ~ ~ ~ ~ ~ ~ ~  hįį\#w-\textbf{rą}-hrE=kt=o're\\
	\textnormal{kettle}=att pv.ins-2a-ins.push-\textnormal{hold.up}=sim ~ ~ ~ ~ ~ ~ ~ ~ ~ ~ ~ ~ ~ ~ ~ \textnormal{drink}\#1s-\textbf{2a}-caus=pot=ind.m\\
	\glt `you should let me drink while you hold out the pail' \citep[131]{hollow1973a}
	
	\item \glll ma\textbf{na}karápxaani éererehtiki ~ ~ ~ ~ ~ ~ ~ ~ ~ ~ húuni ráahta!\\
	w-\textbf{rą}-k-ra-pxE=rį e-re-reh=kti=ki ~ ~ ~ ~ ~ ~ ~ ~ ~ ~ huu=rį rEEh=ta\\
	1s-2a-mid-ins.frce-\textnormal{stumble}=ss pv-\textbf{2a}-\textnormal{want}=pot=cond ~ ~ ~ ~ ~ ~ ~ ~ ~ ~ \textnormal{come.here}=ss \textnormal{go.there}=imp.m\\
	\glt `you can grab me whenever you want, so come on and do it!' \citep[159]{hollow1973b}
	
	\end{xlist}

\end{exe}

\subsubsubsection{Second person stative prefix: \textit{ni}-}\label{Para2S}

The prefix \textit{ni}- is the stative counterpart to \textit{ra-}. This person marker is often the subject of stative verbs or an object of transitive or ditransitive verbs. We can see examples of \textit{ni}- in (\ref{2Smarking}) below.



\begin{exe}

\item\label{2Smarking} Examples of second person stative \textit{ni}-

	\begin{xlist}
	
	\item \glll rapéna'ni \textbf{ni}xíko'sh\\
	ra-perą'=rį \textbf{rį}-xik=o'sh\\
	2a-\textnormal{be.slow}=ss \textbf{2s}-\textnormal{be.bad}=ind.m\\
	\glt `you are slow and bad' \citep[163]{hollow1973b}
	
	\item \glll tashká'sha, ínuma'k \textbf{ni}są́ąkanito'sh\\
	tashka=o'sha i-ruwą'k \textbf{rį}-sąąka=rįt=o'sh\\
	\textnormal{how}=int.m pv.ord-\textnormal{man} \textbf{2s}-\textnormal{be.few}=2pl=ind.m\\
	\glt `there are so few of you men, [so] how can this be?' \citep[151]{hollow1973b}
	
	\item \glll istų́hma'k shíharanista, \textbf{ni}núma'kinito'sh\\
	istųh\#wą'k shi\#hrE=rįt=ta \textbf{rį}-ruwą'k=rįt=o'sh\\
	\textnormal{night}\#pos.lie \textnormal{be.good}\#caus=2pl=imp.m \textbf{2s}-\textnormal{man}=2pl=ind.m\\
	\glt `be careful tonight, [because] you (pl.) are men' \citep[258]{hollow1973b}
	
	\item \glll í\textbf{ni}kxąhinisto'sh, numá'kaaki ~ ~ ~ ~ ~ ~ ~ ~ ~ ~ ~ ~ ~ ~ ~ í\textbf{ni}hekinitki\\
	i-\textbf{rį}-kxąh=rįt=kt=o'sh ruwą'k-aaki ~ ~ ~ ~ ~  ~ ~ ~ ~ ~ ~ ~ ~ ~ ~ i-\textbf{rį}-hek=rįt=ki\\
	pv.ins-\textbf{2s}-\textnormal{laugh}=2pl=pot=ind.m \textnormal{man}-coll ~ ~ ~ ~ ~ ~ ~ ~ ~ ~ ~ ~ ~ ~ ~ pv.ins-\textbf{2s}-\textnormal{know}=2pl=cond\\
	\glt `if people know about you (pl.), they will laugh at you' \citep[28]{hollow1973a}
	
	\item \glll ímikixkahini í\textbf{ni}'ų'taa waptáhini wahúukto'sh\\
	i-wį-ki-xkah=rį i-\textbf{rį}-ų'=taa wa-ptEh=rį wa-huu=kt=o'sh\\
	pv.ins-1s-rflx-\textnormal{put.on.regalia}=ss pv.dir-\textbf{2s}-\textnormal{be.closer}=loc 1a-\textnormal{run}=ss 1a-\textnormal{come.here}=pot=ind.m\\
	\glt `I will dress up and come running towards you' \citep[70]{hollow1973b}
	
	\item \glll keréeh\textbf{ni}herekto'sh\\
	krEEh\#\textbf{rį}-hrE=kt=o'sh\\
	\textnormal{go.back.there}\#\textbf{2s}-caus=pot=ind.m\\
	\glt `he will send you home' \citep[219]{hollow1973b}
	
	\end{xlist}

\end{exe}

The first half of the examples above show stative subjects, while the second half show other stative arguments. There are two others allomorph for \textit{ni-}: /r'\~~-/ and /rų-.

\subsubsubsubsection{Allomorph /r'\~~-/}\label{rglottalnasal}

Very few verbal roots in Mandan are vowel-initial where that vowel is not a preverb. These verbs take a second person stative prefix that is similar to the one described in \sectref{rglottal}. The variant of \textit{ni-} that appears before vowel-initial stems is /r'\~~-/, which has the basic shape of the vowel-initial active /r'-/, except for the fact that this prefix causes the vowel that follows it to become nasalized.

We can see the impact this floating nasal has on the surface representation of the verbs in (\ref{rglottalexamples}) below.

\begin{exe}
\item\label{rglottalexamples} Second person stative marking before vowel-initial stems

	\begin{xlist}

	\item\label{youarealive} \glll \textbf{n}í'niso'sh\\
	\textbf{r'\~~}-irįs=o'sh\\
	\textbf{2s}-\textnormal{be.alive}=ind.m\\
	\glt `you are alive' \citep[91]{hollow1970}
	
	\item\label{youyawn} \glll \textbf{n}í'wereero'sh\\
	\textbf{r'\~~}-iiwree=o'sh\\
	\textbf{2s}-\textnormal{yawn}=ind.m\\
	\glt `you yawn' \citep[98]{hollow1970}
	
	\item\label{youcoveredyourselves} \glll tashká'eshkak mí'he tatą́ąhąą \textbf{n}í'hinito'na?\\
	tashka-eshka=ak wį'h=E ta$\sim$tąą=hąą \textbf{r'i\~~}-ih=rįt=o'rą\\
	\textnormal{how}-smlt=ds \textnormal{robe}=sv r$\sim$\textnormal{be.different}=ins \textbf{2s}-\textnormal{be.draped.in}=int.f\\
	\glt `how come you covered yourself with all kinds of different robes?' \citep[237]{hollow1973b}
	
	\end{xlist}
	
\end{exe}

While the nasality on the prefix for example (\ref{youarealive}) above can possibly be explained by nasal harmony spreading leftward from the root, there are no viable sources of nasal spread that can move from the roots \textit{íiweree} `yawn' or \textit{íh} `be draped in something.' Let us compare the data in (\ref{rglottalexamples}) above to the following example in (\ref{nostativeniP}) where we see no nasality on the prefix.

\begin{exe}
\item\label{nostativeniP} Use of /r'-/ instead of /r'\~~-/

	\glll \textbf{r}í'sąąro'sh\\
	\textbf{r'}-isąą=o'sh\\
	\textbf{2s}-\textnormal{be.in.a.hurry}=ind.m\\
	\glt `you are in a hurry' \citep[92]{hollow1970}

\end{exe}

The lack of nasality on the first syllable indicates that we are looking at a verb that takes active subjects. Thus, we can use this floating nasal on the prefix as a test for whether a verb takes an active or stative subject.

\subsubsubsubsection{Allomorph /rų-/}\label{nu2S}

When first person plural active agents act upon second person arguments (i.e., when the first person plural \textit{nu-} immediately precedes a second person prefix), the second person stative is realized as \textit{nu-} as well. We can see examples of this allomorph in the data in (\ref{nu2Sexamples}) below.

\begin{exe}

\item\label{nu2Sexamples} Examples of \textit{nu-} as a second person stative marker

	\begin{xlist}
	
	\item \glll nu\textbf{nú}kina'kto'sh\\
	rų-\textbf{rų}-kirą'=kt=o'sh\\
	1a.pl-\textbf{2s}-\textnormal{tell}=pot=ind.m\\
	\glt `we will tell you' \citep[10]{kennard1936}
	
	\item \glll máa'ąke nu\textbf{nú}ku'nitki, ónitki raxką́hini raráahini\\
	wąą'ąk=E rų-\textbf{rų}-ku'=rįt=ki o-rį-tki ra-xkąh=rį ra-rEEh=rį\\
	\textnormal{earth}=sv 1a.pl-\textbf{2s}-\textnormal{give}=2pl=cond pv.loc-2s-\textnormal{be.allotted} 2a-\textnormal{move}=ss 2a-\textnormal{go.there}=ss\\
	\glt `as soon as we give you the land, you go move to your allotment and...' \citep[217]{trechter2012b}
	
	\end{xlist}

\end{exe}

This allomorph is very uncommon in the corpus, but common in conversation, due to the fact that the corpus consists mostly of traditional narratives about cultural figures. The /rų-/ form can only be realized if there is no preverb between the first person plural active \textit{nu-} and the second person stative \textit{nu-}. If a preverb is placed between these two preverbs, the second person stative reverts to its default shape, \textit{ni-}.

\subsubsection{First person singular prefix (Slot 5)}\label{SubsubFirstPersonSingular}

First person marking differs from second person marking in that there are dedicated first person singular and first person plural forms.  First person prefixes will always precede a second person prefix, though the specific position within the prefix field depends on whether the first person argument is singular or plural. The two prefixes indicating a first person singular argument appear in (\ref{firstpersonsingularmain}).

\begin{exe}
\item\label{firstpersonsingularmain} Default first person singular person markers

\begin{xlist}
	\item \textit{wa-} first person singular active
	\item \textit{ma-} first person singular stative
\end{xlist}

\end{exe}

Both of these prefixes have a number of allomorphs. The large degree of allomorphs that both the second person and first person singular prefixes have in Mandan is not unlike the large variation found in some other Siouan languages. This variation is taken to be a symptom of the fact that these prefixes are likely the earliest pieces of inflectional morphology to develop onto Proto-Siouan or Pre-Proto-Siouan stems, with material to the left of the inner pronominals being incorporated into the verbal complex at later stages in the development of various daughter languages.

We can see examples of these prefixes at work in \sectref{Para1A} and \sectref{Para1S} below.

\subsubsubsection{First person singular active prefix: \textit{wa}-}\label{Para1A}

The most common realization of a first person singular active argument is the prefix \textit{wa-}. This prefix is a reflex of the Proto-Siouan first person singular active marker *wa-. We see examples of this prefix in (\ref{1Aexamples}) below.

\begin{exe}

\item\label{1Aexamples} Examples of first person singular active \textit{wa-}

	\begin{xlist}
	
	\item \glll rá'skamak wáa\textbf{wa}he'sh, manápusheke\\
	ra'ska\#wąk waa-\textbf{wa}-hE=o'sh wrą\#pushek=E\\
	\textnormal{summer}\#pos.lie \textnormal{some}-\textbf{1a}-\textnormal{see}=ind.m \textnormal{tree}\#\textnormal{juneberry}=sv\\
	\glt `this summer I saw some of them, juneberries' \citep[52]{hollow1973a}

	\item \glll \textbf{wa}'áani hiré \textbf{wa}húuro'sh\\
	\textbf{wa}-E=rį hire \textbf{wa}-huu=o'sh\\
	\textbf{1a}-\textnormal{hear}=ss \textnormal{now} \textbf{1a}-\textnormal{come.here}=ind.m\\
	\glt `I heard it and now I came' \citep[41]{hollow1973a}
	
	\item \glll wáahokshukanashe hiré ą́ąwe í\textbf{wa}seko'sh\\
	waa-ho\#kshuk=rąsh=E hire ąąwe i-\textbf{wa}-sek=o'sh\\
	nom-\textnormal{voice}\#\textnormal{be.narrow}=att=sv \textnormal{now} \textnormal{all} pv.ins-\textbf{1a}-\textnormal{make}=ind.m\\
	\glt `I made all the small creatures now' \citep[11]{hollow1973a}
	
	\item \glll masásaks ų́'taa \textbf{wa}ráahini ~ ~ ~ ~ ~ ~ ~ ~ ~ ~ \textbf{wa}hík numákaaki hų́'re\\
	wą-sa$\sim$sak=s ų'=taa \textbf{wa}-rEEh=rį ~ ~ ~ ~ ~ ~ ~ ~ ~ ~ \textbf{wa}-hi=ak ruwą'k-aaki hų=o're\\
	unsp-aug$\sim$\textnormal{be.dry}=def \textnormal{be.closer}=loc \textbf{1a}-\textnormal{go.there}=ss ~ ~ ~ ~ ~ ~ ~ ~ ~ ~ \textbf{1a}-\textnormal{arrive.there}=ds \textnormal{man}-coll \textnormal{be.many}=ind.f\\
	\glt `I went to the badlands and when I got there, there were many people' \citep[318]{hollow1973b}
	
	\item \glll ą́'ska \textbf{wa}pų́'h shí\textbf{wa}hereka'sh\\
	ą'ska \textbf{wa}-pų'h shi\#\textbf{wa}-hrE=ka=o'sh\\
	\textnormal{that.way} \textbf{1a}-\textnormal{doctor} \textnormal{be.good}\#\textbf{1a}-caus=hab=ind.m\\
	\glt `I am able to doctor that way' \citep[25]{hollow1973a}
	
	\item \glll ptaníshkere máa'ąhku'shtaa reeh\textbf{wa}here'sh\\
	p-ta-rįshkrE wąą'ąk=ku'sh=taa rEEh\#\textbf{wa}-hrE=o'sh\\
	1poss-al-\textnormal{medicine} \textnormal{earth}=\textnormal{be.inside}=loc \textnormal{go.there}\#\textbf{1a}-caus-ind.m\\
	\glt `I put my medicine under the ground' \citep[48]{hollow1973a}
	
	\item \glll wáa'o\textbf{wa}raahinixo'sh\\
	waa-o-\textbf{wa}-rEEh=rįx=o'sh\\
	neg-pv.irr-\textbf{1a}-\textnormal{go.there}=neg=ind.m\\
	\glt `I am not going to go there' \citep[48]{hollow1973a}
	
		
	\item \glll \textbf{wa}hík, manáxot raxápaa náakek, ~ ~ wáa\textbf{wa}ka'rak, máamaku'nixo're\\
	\textbf{wa}-hi=ak wrą\#xot ra-xap=E rąąkE=ak ~ ~  waa-\textbf{wa}-ka'=ak waa-wą-ku'=rįx=o're\\
	\textbf{1a}-\textnormal{arrive.there}=ds \textnormal{wood}\#\textnormal{gray} ins.foot-\textnormal{be.peeling}=sv \textnormal{sit}.aux=ak ~ ~  \textnormal{some}-\textbf{1a}-\textnormal{have}=ds neg-1s-\textnormal{give}=neg=ind.f\\
	\glt `I arrived as she was peeling gray wood, so I asked for some, but she did not give me any' \citep[121]{hollow1973a}
	
	\end{xlist}

\end{exe}

While the majority of situations where a first person singular active argument is present involves \textit{wa-}, there are three other allomorphs: /w'-/, /we-/, and /w-/.

\subsubsubsubsection{Allomorph /w'-/}

This formative mirrors the distribution of /r'-/, described in \sectref{rglottal}. Whenever the first person singular active pronominal appears before a vowel-initial stem, /w'-/ is used instead of /wa-/. Mandan does not permit [Cʔ] clusters, as outlined in \sectref{glottalstopmetathesis}, so this prefix will be realized as [w] that shares a syllable with a coda [ʔ].

We can see examples of this allomorph in (\ref{wglottal}) below. Note that in cases where /w'-/ occurs before a stem beginning with a long vowel, the /ʔ/-metathesis will result in the truncation of the long vowel due to a restriction on superheavy syllables in Mandan, i.e., /VVʔ/ $\to$ /Vʔ/.



\begin{exe}
\item\label{wglottal} Use of /w'-/ for first person singular active

	\begin{xlist}
	
	\item \glll \textbf{w}í'mahąpo'sh\\
	\textbf{w'}-iiwąhąp=o'sh\\
	\textbf{1a}-\textnormal{be.lost}=ind.m\\
	\glt `I am lost' \citep[96]{hollow1970}
	
	\item \glll \textbf{w}á'kana'k\\
	\textbf{w'}-aaki\#rą'k\\
	\textbf{1a}-\textnormal{be.above}\#pos.sit\\
	\glt `I ride horseback' \citep[59]{hollow1970}
	
	\item \glll \textbf{w}á'keroomako'sh\\
	\textbf{w'}-aakE=oowąk=o'sh\\
	\textbf{1a}-\textnormal{step.on}=narr=ind.m\\
	\glt `I stepped on it' \citep[128]{trechter2012b}
	
	\item \glll \textbf{w}á'kakshe'sh\\
	\textbf{w'}-aakakshE=o'sh\\
	\textbf{1a}-\textnormal{meet}=ind.m\\
	\glt `I met him' \citep[3]{kennard1936}
	
	\end{xlist}

\end{exe}

Like with /r'-/, /w'-/ does not appear on open monosyllable roots, e.g., \textit{é} `hear' is \textit{wa'é} for `I hear', never *\textit{wé'}.

\subsubsubsubsection{Allomorph /we-/}\label{allomorphWE}

The allomorph /we-/ is analogous to the /re-/ allomorph in \sectref{allomorphRE}. It sparingly appears before verb roots that begin with sonorants and have /e eː/ in the root. It most commonly occurs with the verb \textit{éereh} `think, want' to the point that the majority of the tokens of \textit{éereh} that are conjugated for first person singular active subjects have \textit{we-} instead of \textit{wa}-.

Other verbs take \textit{we-} sparingly, so it is not completely predictable, but it happens with \textit{éereh} so often that we can say that this allomorph is likely part of the conjugation paradigm of this particular verb. It is not clear if this is a recent change in Mandan verbal morphology, or if the \textit{wa}-/\textit{we}- distribution was more predictable in the past. The other verb that commonly takes \textit{we}- is \textit{réeh} `go there.' The \textit{we}-appears  to occur with verbs whose roots have a closed syllable containing /e/ or /ee/. Verbs whose roots have open syllables do not take \textit{we}-, as we can see in (\ref{WEexamples}) below.



\begin{exe}

\item\label{WEexamples} Examples of \textit{we-}

	\begin{xlist}
	
	\item \glll ówa'ek wakína'ni ée\textbf{we}reho'sh\\
	o-wa-E=ak wa-kirą'=rį ee-\textbf{we}-reh=o'sh\\
	pv.irr-1a-\textnormal{hear}=ds 1a-\textnormal{tell}=ss pv-\textbf{1a}-\textnormal{want}=ind.m\\
	\glt `I want to tell what I heard' \citep[47]{hollow1973a}
	
	\item \glll í\textbf{we}heko'sh\\
	i-\textbf{we}-hek=o'sh\\
	pv.ins-\textbf{1a}-\textnormal{know}=ind.m\\
	\glt `I know it' \citep[5]{kennard1936}
	
	\item \glll háki, nitų́ųminike áa\textbf{we}reehki, ~ ~ ~ ~ ~ ~ ~ ą́'teena ísekto'sh\\
	ha=ki rį-tuuwrįk=E aa-\textbf{we}-rEEh=ki ~ ~ ~ ~ ~ ~ ~ ą't=ee=rą i-sek=kt=o'sh\\
	prov=cond 2poss-\textnormal{clan.aunt}=sv pv.tr-\textbf{1a}-\textnormal{go.there}=cond ~ ~ ~ ~ ~ ~ ~ dem.anap=dem.dist=top pv.ins-\textnormal{make}=pot=ind.m\\
	\glt `So, if I take him to your clan aunt, that one should do it' \citep[57]{hollow1973a}
	
	\end{xlist}

\end{exe}

The \textit{we-} allomorph of \textit{wa-} is less frequently encountered than the \textit{re-} allomorph of \textit{ra-} in the corpus. It is not clear whether this asymmetry is significant, but it is the case that both \textit{we-} and \textit{re-} seem to be lexically conditioned rather than be morphologically or phonologically conditioned.

\subsubsubsubsection{Allomorph /w-/}\label{allomorphw}

In contemporary Mandan, the \textit{wa-} prefix cannot precede a second person prefix. When a first person singular active argument acts upon a second person argument, we must use the allomorph /w-/ instead of \textit{wa-}. Many daughter languages of Proto-Siouan developed from proto-languages that had a productive phonological process whereby inflectional prefixes beginning with a sonorant syncopated their short vowel before another sonorant. 

This process of pre-sonorant syncope is no longer productive in Mandan, but it has left its mark in instances such as /w-/. When combined with \textit{ni-}, the /w-/ nasalizes to [m] and an excrescent Dorsey's Law vowel appears between the /w-/ and the /rį-/ to create a sequence of [m\textsuperscript{ĩ}nĩ]. We see examples of this /w-/ allomorph in (\ref{1Aw}) below.



\begin{exe}
\item\label{1Aw} Examples of /w-/ for first person singular active

	\begin{xlist}
	
	\item \glll \textbf{m}ini'áashko's \\
	\textbf{w}-rį-E=ashko'=s\\
	\textbf{1a}-2s-\textnormal{hear}=emph=def\\
	\glt `I heard you' \citep[41]{hollow1973a}
	
% 	\item \glll \textbf{m}ininíiko'sh\\
% 	\textbf{w}-rį-rįįk=o'sh\\
% 	\textbf{1a}-2s-\textnormal{offspring}=ind.m\\
% 	\glt `you are my son' \citep[217]{hollow1973a}
	
	\item \glll \textbf{m}inikímaaxe'sh\\
	\textbf{w}-rį-kiwąąxE=o'sh\\
	\textbf{1a}-2s-\textnormal{ask}=ind.m\\
	\glt `I asked for you' \citep[131]{hollow1973a}
	
	\item \glll wáa'i\textbf{m}inirats ą́ąwe, miníike, ~ ~ ~ ~ ~ ~ ~ ~ raká'kto'sh\\
	waa-i-\textbf{w}-rį-rat=s ąąwe wį-rįįk=E ~ ~ ~ ~ ~ ~ ~ ~ ra-ka'=kt=o'sh\\
	nom-pv.ins-\textbf{1a}-2s-\textnormal{promise}=def \textnormal{be.many} 1poss-\textnormal{son}=sv ~ ~ ~ ~ ~ ~ ~ ~ 2a-\textnormal{have}=pot=ind.m\\
	\glt `You will have everything I promised you, my son' \citep[192]{hollow1973a}
	
	\item \glll wiráse \textbf{m}inikína'so'sh\\
	wi-ras=E \textbf{w}-rį-kirą'=s=o'sh\\
	1poss-\textnormal{name}=sv \textbf{1a}-2s-\textnormal{tell}=def=ind.m\\
	\glt `I told you my name' \citep[58]{hollow1973a}
	
	\item \glll réeh\textbf{m}inihereki, shų́hą't ísi ~ ~ ~ ~ ~ ~ ~ ~ raréehto'sh\\
	rEEh\#\textbf{w}-rį-hrE=ki shųh=ą't i-si ~ ~ ~ ~ ~ ~ ~ ~ ra-rEEh=kt=o'sh\\
	\textnormal{go.there}\#\textbf{1a}-2s-caus=hyp \textnormal{sinew}=dem.anap {pv.ins}-\textnormal{travel} ~ ~ ~ ~ ~ ~ ~ ~ 2a-\textnormal{go.there}=pot=ind.m\\
	\glt `when I send you there, you follow that sinew' \citep[309]{hollow1973a}
	
	\end{xlist}
	
\end{exe}

The prefix /w-/ will never be realized without nasalization due to the fact that it must always appear before /rį-/, which will spread its [$+$nasal] feature leftward according to the conditions laid out in \sectref{nasalharmony}. This allomorph is always tautosyllabic with /rį-/, which is what leads \citet[10]{kennard1936} to treat this combination as a portmanteau, rather than two discrete morphological items that merely share a syllable.

\subsubsubsection{First person singular stative prefix: \textit{ma}-}\label{Para1S}

The first person singular stative prefix has a similar phonological shape as the first person singular active marker, with the only exception being that it has an underlying nasal vowel instead of an oral one. The first person singular stative prefix is used to mark non-agentive subjects, as well as all other non-subject arguments that bear first person singular semantic features. We can see examples of this prefix in (\ref{MAexamples}) below.

\begin{exe}

\item\label{MAexamples} Examples of \textit{ma-}

	\begin{xlist}
	
	\item \glll manáseena ``\textbf{ma}wáaxe'sh,'' éeheroomako'sh\\
	wrą=s=ee=rą \textbf{wą}-waaxe=o'sh ee-he-oowąk=o'sh\\
	\textnormal{tree}=def=dem.dist=top \textbf{1s}-\textnormal{cottonwood}=ind.m pv-\textnormal{say}=narr=ind.m\\
	\glt `the tree said,`I am a cottonwood''' \citep[36]{hollow1973a}
	
	\item \glll súhkaratoohereka \textbf{ma}'ų́'shka'sh\\
	suk\#k-ratoo\#hrE=ka \textbf{wą}-ų'sh=ka=ind.m\\
	\textnormal{child}\#mid-\textnormal{be.old}\#caus=hab \textbf{1s}-\textnormal{be.thus}=hab=ind.m\\
	\glt `I am the child-rearing kind [of person]' \citep[113]{hollow1973a}
	
	\item \glll hiré watéhąka numá'k í\textbf{ma}huuka'sh\\
	hire wa-te\#hąk=E=$\varnothing$ ruwą'k i-\textbf{wą}-huuka=o'sh\\
	\textnormal{here} 1a-\textnormal{stand}\#pos.std=sv=cont \textnormal{man} pv.ins-\textbf{1s}-\textnormal{be.brave}=ind.m\\
	\glt `I am a brave man, standing here' \citep[91]{trechter2012b}
	
	\item \glll ma\textbf{má}xikanasho'sh\\
	wą$\sim$\textbf{wą}-xik=rąsh=o'sh\\
	aug$\sim$\textbf{1s}-\textnormal{be.bad}=att=ind.m\\
	\glt `I am kind of sick' \citep[107]{hollow1973b}
	
	\item \glll ``wáa'aahuuki, órara'kto're'' ~ ~ ~ ~  ée\textbf{ma}heerak\\
	waa-aa-huu=ki o-ra-ra'k=kt=o're ~ ~ ~ ~  ee-\textbf{wą}-hee=ak\\
	\textnormal{some}-pv.tr-\textnormal{come.here}=cond pv.loc-2a-\textnormal{make.a.fire}=pot=ind.f ~ ~ ~ ~  pv-\textbf{1s}-\textnormal{say}=ds\\
	\glt `{``}if he brings some, you can build a fire,'' she said to me' \citep[120]{hollow1973a}
	
	\item \glll rá'ts wáa'owakiniire ą́ąwe \textbf{ma}séero'sh\\
	r'-at=s waa-o-wa-ki-rįį=E ąąwe \textbf{wą}-see=o'sh\\
	2poss-\textnormal{father}=def nom-pv.irr-1a-rflx-\textnormal{run}=sv \textnormal{all} \textbf{1s}-\textnormal{defeat}=ind.m\\
	\glt `your father beat me every time I raced him' \citep[124]{hollow1973a}
	
	\item \glll hí hą́shkakere'sh, \textbf{ma}pí'kto'sh\\
	hi hąsh=ka=krE=o'sh \textbf{wą}-pi'=kt=o'sh\\
	\textnormal{tooth} \textnormal{be.long}=hab=3pl=ind.m \textbf{1s}-\textnormal{devour}=pot=ind.m\\
	\glt `his teeth are long [and] he might eat me up' \citep[143]{hollow1973a}
	
	\item \glll káare ótaa\textbf{ma}harata, ~ ~ ~ ~ ~ ~ ~ ~ ~ ~ ~ ~ ~ ~ ~ ~ ~ ~ ~ ~ mishų́ųkaǃ\\
	kaare o-taa\#\textbf{wą}-hrE=ta ~ ~ ~ ~ ~ ~ ~ ~ ~ ~ ~ ~ ~ ~ ~ ~ ~ ~ ~ ~ wį-shųųka\\
	imp.neg pv.loc-\textnormal{be.pointing}\#\textbf{1s}-caus=imp.m ~ ~ ~ ~ ~ ~ ~ ~ ~ ~ ~ ~ ~ ~ ~ ~ ~ ~ ~ ~ 1poss-\textnormal{male's.younger.brother}\\
	\glt `do not point it at me, my brother!' \citep[167]{hollow1973a}
	\end{xlist}

\end{exe}

The first half of the examples above demonstrate that \textit{ma}- is used for verbs that take stative subjects, while the second half of the data above highlights that \textit{ma}- can be used for both direct and indirect objects. 

While PSi *wą can be reconstructed as a possible first person singular stative marker in Proto-Siouan, it is a much more marked variant, with reflexes of *wį being the norm across most daughter languages. It is not clear whether there was a semantic distinction between these two formatives or if the difference between them may have originally been constrained by some aspect of the grammar of Proto-Siouan or Pre-Proto-Siouan. 

This confusion between PSi *wą and *wį surfaces in Mandan, where speakers sometimes replace \textit{ma-} with \textit{mi}-. This allomorphy is described below, as well as the allomorphy of \textit{ma-} with /w'\~~-/ and /w-/.

\subsubsubsubsection{Allomorph /w'\~~-/}

For vowel-initial verbal stems, we cannot use \textit{ma-}, but its allomorph /w'\~~-/ instead. This prefix is similar to the first person singular active variant /w'-/ from (\ref{wglottal}) in \sectref{SubSecAllomorphw'}, but this prefix bears a floating nasal. This floating nasal causes the syllable this formative prefixes onto to become nasalized, which then spreads its nasal feature leftward to cause the underlying /w/ in /w'\~~-/ to be realized as [m]. We can see this allomorph in the data in (\ref{wPNexamples}) below.

\begin{exe}

\item\label{wPNexamples} Examples of /w'\~~-/ for first person singular stative

	\begin{xlist}
	
	\item \glll \textbf{m}í'ktąho'sh\\
	\textbf{w'\~~}-iktąh=o'sh\\
	\textbf{1s}-\textnormal{be.cold}=ind.m\\
	\glt `I am cold' \citep[88]{hollow1970}
	
	\item \glll \textbf{m}í'niso'sh\\
	\textbf{w'\~~}-irįs=o'sh\\
	\textbf{1s}-\textnormal{be.alive}=ind.m\\
	\glt `I am alive' \citep[91]{hollow1970}
	
	\item \glll \textbf{m}í'wereero'sh\\
	\textbf{w'\~~}-iiwree=o'sh\\
	\textbf{1s}-\textnormal{yawn}-ind.m\\
	\glt `I yawned' \citep[98]{hollow1970}
	
	\item \glll mishų́ųka \textbf{m}ú'pa ómahikxikanashaa\\
	wį-shųųka \textbf{w'\~~}-ųųpa o-wą-hikxik=rąsh=E=$\varnothing$\\
	1poss-\textnormal{younger.brother} 1a-\textnormal{with} pv.loc-{1s}-\textnormal{be.poor}=att=sv=cont\\
	\glt `my brother is sort of poor with me...' \cite[284]{hollow1973b}
	
	\end{xlist}

\end{exe}

The limited number of verbs that have vowel-initial roots means that this prefix is not common. However, there are enough examples to know that we tell the difference between a vowel-initial stem with an active versus a stative subject. \citet[34]{hollow1970} does not describe this distinction, writing instead that /wa-/ goes to /w'-/ before a vowel with [$-$round] features. \citeauthor{hollow1970}'s dictionary is one of the few sources of full sets of conjugation paradigms, but he does not identify any cause for why certain vowel-initial stems become nasalized while others do not. By winnowing away at the differences between the oral and nasal realizations of these vowel-initial stem prefixes, we can justify the distinction between active and stative verbs as being caused by a floating nasal in the stative prefixes, while the active prefixes do not automatically trigger nasalization.

\subsubsubsubsection{Allomorph /w-/}

The default first person singular stative prefix \textit{ma-} can never appear before a second person active prefix. When a second person active argument acts upon a first person singular stative argument, then \textit{ma-} is realized as /w-/ instead. \citet[10]{kennard1936} treats the ensuing [m\textsuperscript{ã}nã] syllable as a portmanteau (i.e., /wɾã/), but this sequence is not a single morphological item. We can see examples of /w-/ in (\ref{ExwPrefixStative}) below.

\begin{exe}

\item\label{ExwPrefixStative} Examples of /w-/ as first person singular stative prefix

	\begin{xlist}
	
	\item \glll ptawíihąka máxana raharáa \textbf{m}anakú'kto'sh\\
	p-ta-wiihąka wąxrą ra-hrE=$\varnothing$ \textbf{w}-rą-ku'=kt=o'sh\\
	1poss-al-\textnormal{grandchild} \textnormal{one} 2a-caus=cont \textbf{1s}-2a-\textnormal{give}=pot=ind.m\\
	\glt `you can make one grandchild keep it for me' \citep[61]{hollow1973a}
	
	\item \glll ráahta! wáa'i\textbf{m}anasąąpekto'sh\\
	rEEh=ta waa-i-\textbf{w}-rą-sąąpe=kt=o'sh\\
	\textnormal{go.there}=imp.m neg-pv.ins-\textbf{1s}-2a-\textnormal{go.around}=pot=ind.m\\
	\glt `go! you should not go around me' \citep[147]{hollow1973a}
	
	\item \glll \textbf{m}anatéexikini ą́'shkarahere're\\
	\textbf{w}-rą-tee\#xik=rį ą'shka\#ra-hrE=o're\\
	\textbf{1s}-2a-\textnormal{like}\#\textnormal{be.bad}=ss \textnormal{be.that.way}\#2a-caus=ind.f\\
	\glt `you do not like me, so you did it like that' \citep[71]{hollow1973a}
	
	\item \glll ``numá'kshiki ráse núpo'sh,'' ~ ~ ~ ~ ~ ~ ~ ~ ~ ~ ~ ~ ~ ~ ~ ée\textbf{m}anateso'sh\\
	ruwą'k\#shi=ki ras=E rųp=o'sh ~ ~ ~ ~ ~ ~ ~ ~ ~ ~ ~ ~ ~ ~ ~ ee-\textbf{w}-rą-te=s=o'sh\\
	\textnormal{man}\#\textnormal{be.good}=cond \textnormal{name}=sv \textnormal{two}=ind.m ~ ~ ~ ~ ~ ~ ~ ~ ~ ~ ~ ~ ~ ~ ~ pv-\textbf{1s}-2a-\textnormal{say}.2s=def=ind.m\\
	\glt `you said to me, ``if he is a chief, then he has two names{''}' \citep[57]{hollow1973b}
	
	\item \glll wawáruutanashak í\textbf{m}anapse'sh\\
	wa-wa-ruut=rąsh=ak i-\textbf{w}-rą-psE=o'sh\\
	unsp-1a-\textnormal{eat}=att=ds pv.ins-\textbf{1s}-2a-\textnormal{bother}=ind.m\\
	\glt `you are bothering me while I am eating' \citep[133]{hollow1973b}
	
	\item \glll hiré'oshka máa\textbf{m}anakarahinixo'sh\\
	hire-oshka waa-\textbf{w}-rą-krah=rįx=o'sh\\
	\textnormal{now}-emph neg-\textbf{1s}-2a-\textnormal{be.afraid.of}=neg=ind.m\\
	\glt `you are not afraid of me even now' \citep[96]{hollow1973b}
	
	\end{xlist}

\end{exe}

The distinction between the default /wą-/ and /w-/ is that the /w-/ involves a Dorsey's Law vowel, while the default is a full, phonological vowel. We can perceive the distinction between these two allomorphs by observing stress placement and recording vowel duration, sincle Dorsey's Law vowels are systematically shorter than phonemic short vowels. Stress assignment does not take Dorsey's Law vowels into account, while underlying vowels are factored into footing. The intrusive vowel in /w-/ will never affect stress assignment, while the /ą/ in /wą-/ always will.

\subsubsubsubsection{Allomorph /wį-/}\label{wiPprefix}

This variant of \textit{ma}- appears sporadically throughout the corpus. We can predictably see it used with reflexives in (\ref{wiNexamples}) below.

\begin{exe}

\item\label{wiNexamples} Examples of /wį-/

	\begin{xlist}

	\item \glll \textbf{mi}kíhe'sh\\
	\textbf{wį}-ki-hE=o'sh\\
	1s-rflx-\textnormal{see}=ind.m\\
	\glt `I see myself' \citep[440]{hollow1970}
	
	\item \glll mí'shak í\textbf{mi}kisehki nuréehto're\\
	w'\~~-ishak i-\textbf{wį}-ki-sek=ki rų-rEEh=kt=o're\\
	1s-pro pv.ins-\textbf{1s}-rflx-\textnormal{make}=conj 1a.pl-\textnormal{go.there}=pot=ind.f\\
	\glt `Me, I am going to fix myself up and we will go' \citep[127]{hollow1973a}
	
	\item \glll mú'ka, í\textbf{mi}kaheko'sh\\
	w'-ųųka i-\textbf{wį}-ka-hek=o'sh\\
	1poss-\textnormal{older.brother} pv.ins-\textbf{1s}-incp-\textnormal{know}=ind.m\\
	\glt `my brother, I have come to my senses' \citep[144]{hollow1973a}
	
% 	\item \glll í\textbf{mi}mashut\\
% 	i-\textbf{wį}-wąshut\\
% 	pv.ins-1s-\textnormal{clothe}\\
% 	\glt `my clothes' \citep[97]{hollow1970}
	
	\item \glll mí\textbf{mí}'ratooro'sh\\
	wį$\sim$\textbf{wį}'~-ratoo=o'sh\\
	aug$\sim$\textbf{1s}-\textnormal{be.old}=ind.m\\
	\glt `I am the oldest' \citep[6]{hollow1973a}
	
	\item \glll \textbf{mí'}ma'o'ro'sh\\
	\textbf{wį'}$\sim$wą-o'=o'sh\\
	aug$\sim$\textbf{1s}-\textnormal{be}=ind.m\\
	\glt `I am the one' \citep[121]{hollow1973a}
	
	\end{xlist}

\end{exe}

The use of \textit{mi-} before reflexives is documented in previous literature \citep{kennard1936,hollow1970,mixco1997a}. However, there is periodic alternation between \textit{mi-} and \textit{ma-}. Mr. Edwin Benson, the last L1 speaker of Mandan, would sometimes vacillate between \textit{mi-}, \textit{mii-}, and \textit{mi'-} for the first person singular stative when giving elicitations. It is not clear if this alternation with \textit{ma-} means that \textit{mi-} is in free variation with \textit{ma-}, or if this is an artifact of language contact with Hidatsa, whose first person stative marker is \textit{mii-}. Virtually all native speakers of Mandan have also spoken Hidatsa as well since at least the beginning of the twentieth century, so it is plausible that this alternation between different manifestations of the first person singular stative marker is due to the prevalence of Hidatsa usage on the Fort Berthold Indian Reservation.\footnote{Mr. Benson had remarked several times in the past that he was more accustomed to speaking Hidatsa in his later years than Mandan. \citeauthor{park2012} (p.c.) remarks that he had sometimes used Hidatsa as a medium of conversation with Mr. Benson a number of times to elicit Mandan data and narratives.}

One piece of evidence that they are interchangeable for some speakers is the fact that there are a number of examples of emphatic reduplication in the corpus where the reduplicated element is one prefix, and the base element is the other prefix, as we see in \textit{mí'ma'o'ro'sh} `I am the one' or `it is me.' 

Another possibility is that there is an analogical change where first person singular stative in Mandan is \textit{ma-}, but second person stative is \textit{ni-}, and speakers are replacing the vowel in /wą-/ with /į/ to bring it more in line with the phonological shape of the second person stative. The scarcity of \textit{mi}-type prefixes in place of \textit{ma-} in the corpus and the lack of L1 speakers renders it difficult to accurately assess what conditions outside of reflexives that \textit{mi-} is used instead of \textit{ma-}. 

\subsubsection{First person plural prefix (Slot 8)}\label{SubsubFirstPersonPlural}

The first person plural prefixes are the first of the outer pronominals. They will always appear to the left of a preverb. \citeauthor{rankinetalnd} (\citeyear{rankinetalnd}, p.c.) believes that the difference in first person singular and non-singular marking in Siouan is due to a pronominal element being grammaticalized onto the verb stem late in the development from Proto-Siouan into its daughter languages. From there, certain languages lost this dedicated first person plural marking, and transfered plural-marking to enclitics. 

Mandan retains a reflex of the Proto-Siouan first person marker *rų-, which is the first person plural active prefix \textit{nu-}. In \citet{kasak2015,kasak2016}, I make the case that Yuchi is a Siouan language that has undergone much lexical, morphological, and phonological innovation since splitting from Proto-Siouan. This particular morphological item is one piece of support for this hypothesis. All other Siouan languages have *ų- as their first person plural marker, but Yuchi has \textit{õ-} as its first person inclusive prefix and \textit{rõ-} as its first person exclusive prefix. The Mandan form appears to be cognate with the exclusive form, but Mandan \textit{nu-} carries an inclusive reading, which suggests that PSi *ų- merged with *rų- in Mandan, but the reverse happened in Core Siouan, i.e., Mississippi Valley and Ohio Valley Siouan merged the exclusive marker into the inclusive marker. 

We can compare first person plural marking in several Siouan languages below. Both Biloxi and Hidatsa have a generalized first person prefix and express plurality through enclitics.\footnote{\citet[46]{einaudi1976} observes that plural marking in Biloxi is optional once a subject has been established as being plural.} Mandan and Lakota retain the use of dedicated first person plural prefixes. In (\ref{1stplmarking}) below, we see that both languages also permit a dual reading by simply adding the first person plural prefix without an accompanying plural enclitic. 

\begin{exe}
\item\label{1stplmarking} First person plural marking in Siouan

	\begin{xlist}
	
	\item Biloxi\footnote{I have altered the orthography in \citet{einaudi1976} to conform to the orthography that is found in \citeapos{kaufman2011} dictionary.}\\
        \glll ąkidêê(tu)\\
	ą-ki-dêê=tu\\
	1a.pl-vert-\textnormal{go.there}=pl\\
	\glt `we go there' \citep[77]{einaudi1976} 
	
	\item Hidatsa\\
        \glll mú'shia'c\\
	w-u'shia='a=c\\
	1a-\textnormal{arrive}=pl=ind\\
	\glt `we arrived' (Bird Bear p.c.) 
	
	\item Lakota\\
        \glll uŋyé $\sim$ uŋyáŋpi\\
	uŋ-yA ~ uŋ-yA=pi\\
	1a.pl-\textnormal{go.there} ~ 1a.pl-\textnormal{go.there}=pl\\
	\glt `we (du.) went there' vs `we (pl.) went there' \citep[695]{ullrich2011} 
	
	\item Mandan\\
        \glll nuréeho'sh $\sim$ nuráahinito'sh\\
	rų-rEEh=o'sh ~ rų-rEEh=rįt=o'sh\\
	1a.pl-\textnormal{go.there}=ind.m ~ 1a.pl-\textnormal{go.there}=2pl=ind.m\\
	\glt `we (du.) went there' vs `we (pl.) went there'  
	\end{xlist}

\end{exe}

Mandan has two main prefixes that indicate a first person plural argument, which are shown in (\ref{Defaultfirstpersonpluralmarkers}) below. 

\begin{exe}
\item\label{Defaultfirstpersonpluralmarkers} Default first person plural markers

	\begin{xlist}
	\item \textit{nu}- first person plural active
	\item \textit{ro-} first person plural stative
	\end{xlist}

\end{exe}

Examples of the prefixes above appear in \sectref{Para1PLA} and \sectref{Para1PLS}.

\subsubsubsection{First person plural active prefix: \textit{nu-}}\label{Para1PLA}

The default allomorph \textit{nu}- common in conversational Mandan and in the corpus. This prefix is in complementary distribution with the first person singular prefixes, though it can co-occur with second person prefixes. We see this formative in the examples in (\ref{NUexamples}) below.

\begin{exe}

\item\label{NUexamples} Examples of first person plural active prefix \textit{nu}-
	
	\begin{xlist}
	
	\item \glll máa\textbf{nu}he mikó'sh\\
	waa-\textbf{rų}-hE wįk=o'sh\\
	nom-\textbf{1a.pl}-\textnormal{see} \textnormal{be.none}=ind.m\\
	\glt `we (du.) saw nothing' \citep[186]{hollow1973b}
	
	\item \glll hiré máa\textbf{nu}xkąhinixo'sh\\
	hire waa-\textbf{rų}-xkąh=rįx=o'sh\\
	\textnormal{now} neg-\textbf{1a.pl}-\textnormal{move}=neg=ind.m\\
	\glt `now we (du.) will not break camp' \citep[195]{hollow1973b}
	
	\item \glll ``tópha náhki, \textbf{nu}tíkto'sh,'' ~ ~ ~ ~ ~ ~ ~ ~ ~ ~ éeheeroomako'sh\\
	top\#ha rąk=ki \textbf{rų}-ti=kt=o'sh ~ ~ ~ ~ ~ ~ ~ ~ ~ ~ ee-hee=oowąk=o'sh\\
	\textnormal{four}\#\textnormal{times} pos.sit=cond \textbf{1a.pl}-\textnormal{arrive.here}=pot=ind.m ~ ~ ~ ~ ~ ~ ~ ~ ~ ~ pv-\textnormal{say}=narr=ind.m\\
	\glt `{``}we (du.) will arrive when it is the fourth time,'' he said' \citep[243]{hollow1973b}
	
	\item \glll \textbf{nu}kirúharanik \textbf{nu}kúhka'sh\\
	\textbf{rų}-k-rhu\#hrE=rįk \textbf{rų}-kuh=ka=o'sh\\
	\textbf{1a.pl}-vert-seq\#caus=itr \textbf{1a.pl}-\textnormal{come.back.here}=hab=ind.m\\
	\glt `we (du.) always come home when we head back here' \citep[151]{hollow1973b}
	
	\item \glll hóoraka ptą́ąka mú'pani ~ ~ ~ ~ ~ ~ ~ ~ warúha' \textbf{nu}reeho'sh\\
	hóoraka p-(ta)-tąąka w'-ųųpa=rį ~ ~ ~ ~ ~ ~ ~ ~ wa-ru-ha' \textbf{rų}-rEEh=o'sh\\
	\textnormal{yesterday} 1poss-al-\textnormal{woman's.younger.sister} 1a-\textnormal{with}=ss ~ ~ ~ ~ ~ ~ ~ ~  unsp-ins.hand-\textnormal{pick.berries} \textbf{1a.pl}-\textnormal{go.there}=ind.m\\
	\glt `yesterday I went berry-picking with my sister' \citep[52]{hollow1973a}
	
	\item \glll xamáhe \textbf{nu}rúshektiki, ~ ~ ~ ~ ~ ~ ~ ~ ~ ~ ~ ~ ~ ~ ~ íhehka'sh\\
	xwąh=E \textbf{rų}-ru-shE=kti=ki ~ ~ ~ ~ ~ ~ ~ ~ ~ ~ ~ ~ ~ ~ ~ i-hek=ka=o'sh\\
	\textnormal{be.little}=sv \textbf{1a.pl}-ins.hand-\textnormal{grasp}=pot=cond ~ ~ ~ ~ ~ ~ ~ ~ ~ ~ ~ ~ ~ ~ ~ pv.ins-\textnormal{know}=hab=ind.m\\
	\glt `whenever we (du.) take a little, he always knows' \citep[116]{hollow1973b}
	\largerpage
	\item \glll óo ų́ųpaną \textbf{nu}rúsanaahini ~ ~ ~ ~ ~ ~ ~ ~ ~ ~ \textbf{nu}húuro'sh\\
	oo ųųpa=ną \textbf{rų}-ru-srąąh=rį ~ ~ ~ ~ ~ ~ ~ ~ ~ ~ \textbf{rų}-huu=o'sh\\
	dem.mid \textnormal{elk}=top \textbf{1a.pl}-ins.hand-\textnormal{leave.behind}=ss ~ ~ ~ ~ ~ ~ ~ ~ ~ ~  \textbf{1a.pl}-\textnormal{come.here}=ind.m\\
	\glt `we (du.) left an elk there and came' \citep[180]{hollow1973b}
	
	\item \glll \textbf{nu}páminishinito'sh\\
	\textbf{rų}-pa-wrįsh=rįt=o'sh\\
	\textbf{1a.pl}-ins.push-\textnormal{be.rolled.up}=2pl=ind.m\\
	\glt `we (pl.) rolled it up' \citep[462]{hollow1970}
	
	\item \glll \textbf{nu}'áanito'sh\\
	\textbf{rų}-E=rįt=o'sh\\
	\textbf{1a.pl}-\textnormal{hear}=2pl=ind.m\\
	\glt `we (pl.) hear it' \citep[473]{hollow1970}
	
	\item \glll máa\textbf{nu}haanitinixo'sh\\
	waa-\textbf{rų}-hE=rįt=rįx=o'sh\\
	neg-\textbf{1a.pl}-\textnormal{see}=2pl=neg=ind.m\\
	\glt `we (pl.) did not see it'
	
	\end{xlist}

\end{exe}

Whenever \textit{nu}- appears without the second person plural enclitic \textit{=nit}, \textit{nu-} typically carries a dual inclusive reading, i.e., the speaker and the addressee only. The enclitic \textit{=nit} grants a plural reading, and does not automatically give an inclusive reading.

\subsubsubsubsection{Allomorph /rV-/}\label{rVallomorph}

As an outer pronominal, \textit{nu}- often comes into contact with preverbs. All preverbs in Mandan lack an onset, and the frequency at which \textit{nu-} abutted these preverbs has caused Mandan to develop an allomorph where the underlying nasal vowel of /rų-/ \textit{nu}- harmonizes with the following vowel. As we see in (\ref{rV1examples}), first person active plural marking onto a vowel-initial stem causes the initial vowel to lengthen, and the lack of an underlying nasal does not cause the /r/ to nasalize.

\begin{exe}

\item\label{rV1examples} Examples of /rV-/

	\begin{xlist}
	

	
	\item \glll są́ąka \textbf{ró}onapini ą́ąwe nurúha'ni ~ ~ nukúho'sh\\
	sąąka \textbf{rV}-o-rąp=rį ąąwe rų-ru-ha'=rį ~ ~  rų-kuh=o'sh\\
	\textnormal{be.few} \textbf{1a.pl}-pv.loc-\textnormal{find}=ss \textnormal{all} 1a.pl-ins.hand-\textnormal{pick.berries}=ss ~ ~  1a.pl-\textnormal{come.back.here}=ind.m\\
	\glt `we found a few, we picked everything, and we came back' \citep[52]{hollow1973a}
	
	\item \glll wáa'oksąą íseke s\'{ı̨}hanashak ~ ~ ~ ~ ~ ~ ~ ~ ~ ~ \textbf{rí}ihekinito'sh\\
	waa-o-ksąą i-sek=E sįh=rąsh=ak ~ ~ ~ ~ ~ ~ ~ ~ ~ ~  \textbf{rV}-i-sek=rįt=o'sh\\
	nom-pv.irr-\textnormal{trouble} pv.ins-\textnormal{make}=sv \textnormal{be.strong}=att=ds ~ ~ ~ ~ ~ ~ ~ ~ ~ ~  \textbf{1a.pl}-pv.ins-\textnormal{know}=2pl=ind.m\\
	\glt `he does crooked things all the time and we know it' \citep[43]{hollow1973a}
	
	\item\label{weareinahurry} \glll \textbf{rí}isąąro'sh\\
	\textbf{rV}-isąą=o'sh\\
	\textbf{1a.pl}-\textnormal{be.in.a.hurry}=ind.m\\
	\glt `we are in a hurry' \citep[92]{hollow1970}
	
	\end{xlist}

\end{exe}

This formative is mostly seen in conjunction with preverbs, but /rV-/ also appears when used with any of the few vowel-initial verbal roots in Mandan as we can see in (\ref{weareinahurry}). With vowel-initial stems, like the one in (\ref{youareinahurry}) below, the main difference between a second person active form and a first person plural active form is whether the first syllable involves a short vowel and a coda glottal or a long vowel.

\begin{exe}

\item\label{youareinahurry} \glll \textbf{rí'}sąąro'sh\\
	\textbf{r'}-isąą=o'sh\\
	\textbf{2s}-\textnormal{be.in.a.hurry}=ind.m\\
	\glt `you are in a hurry' \citep[92]{hollow1970}

\end{exe}

As we can see in (\ref{youareinahurry}), the /r'-/ prefix causes metathesis with the glottal stop and the initial vowel, creating a closed syllable, but with the /rV-/ in (\ref{weareinahurry}), the syllable remains open and the initial vowel lengthens. The distinction between these two words can be minimal or nonexistent in fast speech where [Vʔ] can be realized as [Vː]. Context certainly helps clarify what subject marking a speaker intends in cases like this one.

\subsubsubsubsection{Allomorph /r-/}
\largerpage
Mandan prohibits trimoraic syllables, as discussed in \sectref{shortvoweldeletion}, but some preverbs and vowel-initial verbal roots begin with long vowels. We cannot use /rV-/ in these contexts, with /r-/ being used instead, as we see in (\ref{ExRonly}) below.

\newpage
\begin{exe}

\item\label{ExRonly} Examples of first person plural active /r-/

	\begin{xlist}
	
	\item \glll nuráahini Pą́ąhį'~Shųts óo ~ ~ ~ ~ ~ ~ ~ ~ ~ ~ ~ ~ ~ \textbf{r}óorookti'sh\\
	rų-rEEh=rį pąąhį'\#shųt=s oo ~ ~ ~ ~ ~ ~ ~ ~ ~ ~ ~ ~ ~  \textbf{r}-o-o-rootki=o'sh\\
	1a.pl-\textnormal{go.there}=ss \textnormal{porcupine}\#\textnormal{tail}=def dem.mid ~ ~ ~ ~ ~ ~ ~ ~ ~ ~ ~ ~ ~ \textbf{1a.pl}-pv.irr-pv.loc-\textnormal{hit}=ind.m\\
	\glt `we will go and camp there at Porcupine Tail' \citep[254]{hollow1973b}
	
	\item \glll na'é, nihų́pe \textbf{r}áahuuro're\\
	rą-E rį-hųp(E) \textbf{r}-aa-huu=o'sh\\
	\textnormal{mother}.voc=sv 2poss-\textnormal{shoe} \textbf{1a.pl}-pv.tr-\textnormal{come.here}=ind.m\\
	\glt `mother, we brought your shoes' \citep[147]{hollow1973a}
	
	\item \glll maná teréekerek nurúskopini ~ ~ ~ ~ ~ ~ ~ ~ \textbf{r}éerehini máanurutirishe miká\\
	wrą tree=krE=ak rų-ru-skop=rį ~ ~ ~ ~ ~ ~ ~ ~  \textbf{r}-ee-reh=rį waa-rų-ru-trish=E wįk=E=$\varnothing$\\
	\textnormal{tree} \textnormal{be.big.around}=3pl=ds 1a.pl-ins.hand-\textnormal{be.bent}=ss ~ ~ ~ ~ ~ ~ ~ ~  \textbf{1a.pl}-pv-\textnormal{want}=ss nom-1a.pl-ins.hand-\textnormal{shake}=sv \textnormal{be.none}=sv=cont\\
	\glt `the trees were big, so we wanted to bend them, but we could not budge them' \citep[52]{hollow1973a}

	\end{xlist}
	
\end{exe}

This allomorph is not seen often in the corpus due to the fact that there is only one preverb that has an underlying long vowel, and there are no attested active-marking verbs that begin with long vowels. The nature of /rV-/ also can make it ambiguous whether there is an irrealis preverb plus a locative preverb with /r-/ or a locative preverb with /rV-/, since they will be homophonous, e.g., \textit{róorootki'sh
} `we will camp there' can likewise be `we camp there.' The corpus includes glosses beneath each word, so we can glean the intent of the speaker who went through and explained what he or she had meant, but multiple interpretations are possible in the narratives that have yet to be glossed with a Mandan speaker, such as the \citet{bowers1971} recordings.

\subsubsubsection{First person plural stative prefix: \textit{ro-}}\label{Para1PLS}

The prefix \textit{ro-} is not common in the corpus, due to the fact that the corpus is mostly comprised of narratives where a single cultural figure is on a journey and interacts with maybe one other figure at a time. This is another example of morphology that is more common in conversational Mandan than the corpus would otherwise suggest. This particular formative does not have an obvious Proto-Siouan origin. Catawba has similar \textit{nu-} and \textit{do-} first person plural object markers, the latter bearing the strongest similarity to the Mandan \textit{ro-} [ⁿdo] \citep{rankin2015}. 

If we assume that Catawban and Mandan are both part of the Peripheral Siouan phylogenetic group of languages, following the proposal laid down in \citet{kasak2015}, we can hypothesize that these similarities are either due to shared innovations in a common ancestor that had already split from Core Siouan, or that these languages share an archaism that has been lost in other daughter languages of Proto-Siouan, since other Siouan languages do not have dedicated morphology to active versus stative first person plural marking. We can see examples of \textit{ro}- in the examples in (\ref{ExROexamples}) below.

\begin{exe}

\item\label{ExROexamples} Examples of first person plural stative \textit{ro-}

	\begin{xlist}
	
	\item \glll waptáhehki \textbf{ro}kíikirixaani rápena'ro'sh\\
	wa-ptEh=ki \textbf{ro}-kiikrixE=rį ra-perą'=o'sh\\
	1a-\textnormal{run}=cond \textbf{1s.pl}-\textnormal{catch.up.to}=ss 2a-\textnormal{be.slow}=ind.m\\
	\glt `if I run away, they will catch us since you are slow' \citep[163]{hollow1973b}
	
	\item \glll wáa'oxikt nuréehki, \textbf{ro}kirushaata\\
	waa-o-xik=t rų-reeh=ki \textbf{ro}-k-ru-shE=ta\\
	nom-pv.loc-\textnormal{be.bad} 1a.pl-\textnormal{go.there}=cond \textbf{1s.pl}-vert-\textnormal{grasp}=ind.m\\
	\glt `if we go to a bad place, take us back' \citep[45]{hollow1973b}
	
	\item \glll óparashtaa íshąątaa nákini ~ ~ ~ ~ ~ ~ ~ ~ ~ \textbf{ro}kirúherektiki nuréehka'sh\\
		o-prash=taa i-shąą=taa rąk=rį ~ ~ ~ ~ ~ ~ ~ ~ ~  \textbf{ro}-kru\#hrE=kti=ki rų-rEEh=ka=o'sh\\
		pv.loc-\textnormal{be.pointed}=loc pv.dir-\textnormal{across}=loc pos.sit=ss ~ ~ ~ ~ ~ ~ ~ ~ ~ \textbf{1s.pl}-vert+seq\#caus=pot=cond 1a.pl-\textnormal{go.there}=hab=ind.m\\
	\glt	`whenever he calls us across to the ridge there, we always go' \citep[151]{hollow1973b}

	\item \glll nu'ų́'taa waxópininite, éetaanik, ~ ~ ~ ~ ~ ~ ~ ~ ~ ~ wa\textbf{ró}ruute  \textbf{ro}kú'ka'sh\\
	rų-ų'=taa wa-xoprį=rįt=E ee-tE=rįk  ~ ~ ~ ~ ~ ~ ~ ~ ~ ~  wa-\textbf{ro}-ruutE ro-ku'=ka=o'sh\\
	1a.pl-\textnormal{be.closer}=loc unsp-\textnormal{be.holy}=2pl=sv pv-\textnormal{say}.2a=itr  ~ ~ ~ ~ ~ ~ ~ ~ ~ ~  unsp-1s.pl-\textnormal{eat} \textbf{1s.pl}-\textnormal{give}=hab=ind.m\\
	\glt `you always give something to eat to us holy spirits that you call' \citep[176]{hollow1973b}
	
	\item \glll na'é réeh\textbf{ro}hereso'sh\\
	rą-E rEEh\#\textbf{ro}-hrE=s=o'sh\\
	\textnormal{mother}.voc=sv \textnormal{go.there}\#\textbf{1s.pl}-caus=def=ind.m\\
	\glt `mother told us to go' \citep[166]{hollow1973a}
	
	\item \glll ka\textbf{ró}kahashka\\
	ka-\textbf{ro}-ka-hash=ka\\
	agt-\textbf{1s.pl}-ins.frce-\textnormal{slaughter}=hab\\
	\glt `the one who slaughters us' \citep[146]{hollow1973a}
	
	\end{xlist}

\end{exe}

This prefix is not described in \citeapos{hollow1970} dictionary, but it is discussed in \citeapos{kennard1936} grammar. \citeauthor{hollow1970} explicitly defines \textit{nu}- as the sole marker of first person plural, regardless of what role the first person plural argument is playing in the clause. However, in the narratives collected in \citet{hollow1973a,hollow1973b}, \textit{ro}- does occur with `us' appearing in the translation instead of `we.'

\subsubsubsubsection{Allomorph /rV-/}

Much like \textit{nu-}, \textit{ro-} cannot appear before a vowel-initial stem. This prefix, too, has an allomorph where the underlying vowel harmonizes the following vowel to create a single long vowel: /rV-/. We can see examples involving this prefix in (\ref{ExRVS}) below.

\begin{exe}
\item\label{ExRVS} Examples of first person plural stative allomorph /rV-/

	\begin{xlist}
	
	\item \glll éena \textbf{ró}okaweho'sh, ~ ~ ~ ~ ~ ~ ~ ~ ~ ~ ~ ~ ~ ~ ~ manáhįtahimi'kshukeena\\
	ee=rą \textbf{rV}-o-ka-weh=o'sh ~ ~ ~ ~ ~ ~ ~ ~ ~ ~ ~ ~ ~ ~ ~  wrą=hį\#ta-hi\#wį'\#kshuk=ee=rą\\
	dem.dist=top \textbf{1s.pl}-ins.frce-\textnormal{chose}=ind.m ~ ~ ~ ~ ~ ~ ~ ~ ~ ~ ~ ~ ~ ~ ~  \textnormal{wood}=\textnormal{with}\#al-\textnormal{tooth}\#\textnormal{stone}\#\textnormal{be.narrow}=dem.dist=top\\
	\glt `then he chooses it for us, a stone-pointed lance' \citep[151]{hollow1973b}
	
	\item \glll mí'ti kí'hini \textbf{rí}iruptaahini ~ ~ ~ ~ numá'kaaki rokaraahini téeroharani éerehini\\
	wį'\#ti ki'h=rį \textbf{rV}-i-ru-ptEh=rį ~ ~ ~ ~  ruwą'k-aaki ro-kraah=rį tee\#ro-hrE=rį ee-reh=rį\\
	\textnormal{stone}\#\textnormal{reside} \textnormal{arrive.back.here}=ss \textbf{1s.pl}-pv.ins-ins.hand-\textnormal{run}=ss ~ ~ ~ ~ \textnormal{man}-coll 1s.pl-\textnormal{be.afraid.of}=ss \textnormal{die}\#1s.pl-caus=ss pv-\textnormal{want}=ss\\
	\glt `he got back to the village and blamed us, so the people were afraid of us and wanted to kill us and...' \citep[186]{hollow1973a}
	
	\item \glll numá'kaaki ą́ąwe rokaraahkarani ~ ~ ~ ~ ~ ~ ~ ~ ~ ~ ~ ~ ~ ~ ~ \textbf{rí}iruksahąmika réehkere'sh\\
	ruwą'k-aaki ąąwe ro-kraah=krE=rį ~ ~ ~ ~ ~ ~ ~ ~ ~ ~ ~ ~ ~ ~ ~ \textbf{rV}-i-ru-ksah=awį=ka rEEh=krE=o'sh\\
	\textnormal{man}-coll \textnormal{all} 1s.pl-\textnormal{be.afraid.of}=3pl=ss ~ ~ ~ ~ ~ ~ ~ ~ ~ ~ ~ ~ ~ ~ ~ \textbf{1s.pl}-pv.ins-ins.hand-\textnormal{go.away}=cont=hab \textnormal{go.there}=3pl=ind.m\\
	\glt `all the people were afraid of us and went, leaving us behind' \citep[184]{hollow1973a}
	
	\end{xlist}

\end{exe}

The homophony between the first person plural active /rV-/ and the first person plural stative /rV-/ above can make it challenging to identify which argument /rV-/ represents in isolation. The context in which such words appear is crucial for informing a listener what the intended argument is that is being marked.

\subsubsubsubsection{Allomorph /r-/}

This allomorph of \textit{ro}- is homophonous with the /rV-/ allomorph of \textit{nu}-. This prefix is used before vowel-initial stems that begin with long vowels. If no other pronominal marking is present on the verb, /rV-/ can ambiguously indicate a first person plural active argument or a first person plural stative argument. We can see an example of this ambiguity in the examples in (\ref{RVambiguity}) below.

\begin{exe}

\item\label{RVambiguity} Ambiguity with /rV-/ marking

	\begin{xlist}
	
	\item\label{ambiguousrV1} \glll ráa\textbf{ni}raahinito'sh\\
	r-aa-\textbf{rį}-rEEh=rįt=o'sh\\
	1a.pl-pv.tr-\textbf{2s}-\textnormal{go.there}=2pl=ind.m\\
	\glt `we brought you there'
	
	\item\label{ambiguousrV2} \glll ráa\textbf{ra}raahinito'sh\\
	r-aa-\textbf{ra}-rEEh=rįt=o'sh\\
	1s.pl-pv.tr-\textbf{2a}-\textnormal{go.there}=2pl=ind.m\\
	\glt `you brought us there'
	
	\item\label{ambiguousrV3} \glll ráaraahinito'sh\\
	r-aa-rEEh=rįt=o'sh\\
	1a.pl/1s.pl-pv.tr-\textnormal{go.there}=2pl=ind.m\\
	\glt `he/she/it brought us there' or `we brought him/her/it there'
	
	\end{xlist}

\end{exe}

The /r-/ alone does not tell us if it refers to an active or stative argument, but the presence of second person marking clarifies what role /r-/ plays. The verb marked with \textit{ni-} in (\ref{ambiguousrV1}) must have its /r-/ refer to an active argument, as \textit{ni-} is stative, and vice versa for \textit{ra-} and its /r-/ in (\ref{ambiguousrV2}). If there were no second person argument involved in these sentences, then there would be no way to know whether /r-/ is referring to people doing an action or to whom the action is done, as we see in (\ref{ambiguousrV3}). The intended reading would need some kind of context to remedy this ambiguity.

\subsubsubsubsection{Allomorph /rų-/}

This allomorph of \textit{ro-} is used only with reflexives. The shape of this formative differs from the other reflexive allomorphs \textit{mi-} and \textit{ni-} in that \textit{nu-} does not resemble stative marking, but active marking. There likely was paradigmatic instability at some point in pre-modern Mandan where there was a shift away from stative marking, and only the first person plural stative before a reflexive is identical with its active counterpart. It is unclear how this process worked in Proto-Siouan, given the fact that there is no default pattern that we observe across the language family, even within the same branch. 

In Mississippi Valley Siouan languages, Lakota  marks reflexive subjects with stative pronominals \citep[23]{ingham2003}, while Ioway-Oto and Osage use active pronominals for subject marking (\citealt[244]{whitman1947}, \citealt[244]{quintero2004}). The Ohio Valley Siouan language Tutelo uses dative prefixes with reflexives \citep[77]{oliverio1997}, while the Missouri Valley Siouan language Crow uses active prefixes to mark reflexive subjects \citep[149]{graczyk2007}. It is difficult to pinpoint what the original system of reflexive subject marking was in Proto-Siouan, and the mixed paradigm in Mandan suggests there this system may not have been particularly uniform across the language family. It is worth noting that the number of Siouan languages use active marking for reflexive subjects.

We can see the use of \textit{nu-} as a first person plural stative marker in the data in (\ref{ExNUreflexive}) below.



\begin{exe}

\item\label{ExNUreflexive} First person plural stative marking with reflexives

	\begin{xlist}

	\item \glll \textbf{nu}kirúskapo'sh\\
	\textbf{rų}-k-ru-skap=o'sh\\
	\textbf{1s.pl}-rflx-ins.hand-\textnormal{pinch}=ind.m\\
	\glt `we pinch ourselves' \citep[440]{hollow1970}
	
	\item \glll \textbf{nu}kíhe'sh\\
	\textbf{rų}-ki-hE=o'sh\\
	\textbf{1s.pl}-rflx-\textnormal{see}=ind.m\\
	\glt `we see ourselves' \citep[475]{hollow1970}
	
	\end{xlist}

\end{exe}

As discussed in \sectref{reflexivemarker}, reflexive marking on verbs with plural subjects can also give a reciprocal reading. Thus, both of the sentences above can be interpreted as `we pinch each other' and `we see each other', respectively. Mandan has no devoted reciprocal marking morphology, so any reciprocal reading is typically left to context.

\subsubsection{Unspecified argument prefix (Slot 9)}\label{SubsubsUnspecifiedArgument}

One of the more difficult inflectional prefixes to explain is the unspecified argument marker \textit{wa}-. In much of the Siouanist literature, this prefix is referred to as the absolutive marker. Calling this formative absolutive does not mean that Siouanists believe that these languages have an ergative-absolutive alignment, though historically this was the case at one time. Siouanists that studied at the University of California, Berkeley under Terrence Kaufman had been introduced to ergative-absolutive languages through Kaufman's work on Mayan languages. While Siouanists came to agree that active-stative was the more accurate description of the alignment system found across the language family, the term absolutive remained in the literature for the sake of convention (\citeauthor{rankin2010} p.c.).

In the work presented here, I employ the term that \citet{mixco1997a} uses in his Mandan grammar: unspecified argument. This term captures the actual function of this prefix in Mandan, which is to mark some indefinite non-subject argument. Some of the issues with identifying the unspecified argument marker in Mandan have been the inconsistency (e.g., \citealt{kennard1936}, \citealt{mixco1997a}, and \citealt{trechter2012b}) of long vowel marking, or even the disavowal of vowel length (e.g., \citealt{hollow1970} and \citealt{coberly1979}).

Historically, this prefix originates as a merger between the *wi- and *wa- classifiers in Proto-Siouan, where *wi- marks animate non-human arguments and *wa- marks inanimate arguments. After *wi- merged with *wa-, *wa- became used to mark not only nominal stems, but verbal stems as well. This process seems to have taken place before late Proto-Siouan, as this behavior is observed in every branch of the Siouan language family, where some reflex of *wa- has become part of the outer pronominal set \citep{rankinetalnd}. The meaning of this element varies from language to language. Tutelo treats its \textit{wa-} in a manner similar to that in Mandan \citep[87]{oliverio1997}. In Dakotan, \textit{wa-} acts as an indefinite object \citep[16]{ingham2003}, but in Dhegihan and Hoocąk-Chiwere, \textit{wa-} is a third person plural object marker (\citealt[75]{quintero2004}, \citealt[286]{helmbrechtlehmann2008}). Crow and Hidatsa both have an indefinite object marker that doubles as a nominalizer (\citealt[242]{boyle2007}, \citealt[195]{graczyk2007})

One bit of confusion found in previous grammars of Mandan is that this $\langle$wa$\rangle$ in others' transcriptions maps to two different phonetic realizations: [wa-] and [waː-]. Furthermore, the [waː-] really has multiple meanings, the most common of which is that it acts as a nominalizer. The different uses of these formatives is summarized in (\ref{ExAlltheWAs}) below.

\begin{exe}

\item\label{ExAlltheWAs} Items conflated as `absolutive' in \citet{hollow1970}

	\begin{xlist}
	\item \textit{wa}- unspecified argument prefix
 	\item \textit{waa}-\textsubscript{1} nominalizer
	\item \textit{waa}-\textsubscript{2} partitive marker
	\item \textit{waa}-\textsubscript{3} indefinite subject prefix
	
	\end{xlist}

\end{exe}

This conflation in previous scholars' work is due to the fact that \textit{wa-} \textsc{unsp}, \textit{waa-} \textsc{nom}, and \textit{waa-} \textsc{part} have similar semantics. Namely, these formatives all indicate some kind of indefinite characteristic. In particular, this confusion arises from the nominalizer \textit{waa-} having subject semantics for the nominalized element (e.g., \textit{wáashi} `something that is good') versus an action undertaken by someone else  (e.g., \textit{wapápshii} `baking powder bread', which is literally `push something flat'). Both \textit{wáashi} and \textit{wapápshii} are treated as nouns syntactically and can receive noun morphology. Thus, while both items are nominalized verbs, the process of how they they are formed differs: \textit{waa-} nominalizes a verbal complex by taking the place of a subject, while \textit{wa-} can be found on deverbals as a stand-in for an object.

\subsubsubsection{Unspecified argument prefix: \textit{wa-}}\label{SubSubSubSecUNSP}

The unspecified argument marker is often found in the corpus when an agent does an action to `something' or `someone' without specifying what or whom that object is.  It is also found with certain verbs that seem to always require an overt object. If no overt nominal object is present, then \textit{wa-} substitutes for whatever it is that is serving as a direct or indirect object, as we can see in (\ref{UNSPwithWA}) below.

\begin{exe}

\item\label{UNSPwithWA} Examples of unspecified argument \textit{wa-} marking

	\begin{xlist}

	\item \glll \textbf{wa}kósh ówaku'ro'sh\\
	\textbf{wa}-kosh o-wa-ku'=o'sh\\
	\textbf{unsp}-\textnormal{whistle} pv.irr-1a-\textnormal{give}=ind.m\\
	\glt `I am going to whistle for someone' \citep[487]{hollow1970}
	
	\item \glll máa\textbf{ma}nanuunixinisto'sh\\
	waa-\textbf{wa}-ra-rųų=rįx=rįt=kt=o'sh\\
	neg-\textbf{unsp}-2a-\textnormal{abduct}=neg=2pl=pot=ind.m\\
	\glt `you shall not commit adultery' \citep[22]{hollow1970}
	
	\item \glll \textbf{Wa}xíhkina'\\
	\textbf{wa}-xik\#kirą'\\
	\textbf{unsp}-\textnormal{be.bad}\#\textnormal{tell}\\
	\glt `Bad News Clan' \citep[30]{bowers1950}
		
	\item \glll \textbf{wa}húu~íkiri\\
	\textbf{wa}-huu\#i-kri\\
	\textbf{unsp}-\textnormal{bone}\#pv.ins-\textnormal{be.grease}\\
	\glt `marrow' \citep[82]{hollow1970}

	\item \glll ta\textbf{wá}kapxe\\
	ta-\textbf{wa}-kapxe\\
	al-\textbf{unsp}-\textnormal{earn}\\
	\glt `his wages' \citep[101]{hollow1970}
	
	\item \glll \textbf{wa}ká're\\
	\textbf{wa}-ka'=E\\
	\textbf{unsp}-\textnormal{have}=sv\\
	\glt 	`property' \citep[102]{hollow1970}
		
	\item \glll \textbf{wa}pápshiire\\
	\textbf{wa}-pa-pshii=E\\
	\textbf{unsp}-ins.push-\textnormal{be.flat}=sv\\
	\glt `baking soda bread' \citep[14]{benson1999}
	
	\item \glll hų́ų, minikína'nisto'sh, \textbf{wa}wákanaaki, ~ ~ ~ ~ ~ ~ ~ ~ ~ ~ \textbf{ma}nánapaanitki\\
	hųų w-rį-kirą'=rįt=kt=o'sh \textbf{wa}-wa-krąą=ki ~ ~ ~ ~ ~ ~ ~ ~ ~ ~ \textbf{wa}-ra-rąpE=rįt=ki\\
	\textnormal{yes} 1a-2s-\textnormal{tell}=2pl=pot=ind.m \textbf{unsp}-1a-\textnormal{sing}=cond ~ ~ ~ ~ ~ ~ ~ ~ ~ ~ \textbf{unsp}-2a-\textnormal{dance}=2pl=cond\\
	\glt `yes, I will tell it to you (pl.), when I sing and when you (pl.) dance' \citep[32]{hollow1973b}
	
\item \glll káni ``tashká waheréki \textbf{wa}wáruuto'xara'shka,'' ~ ~ ~ ~ ~ ~ ~ ~ ~  éerehoomako'sh\\
	ka=rį tashka \textbf{wa}-hrE=ki wa-wa-ruut=o'xrE=a'shka  ~ ~ ~ ~ ~ ~ ~ ~ ~ ee-reh=oowąk=o'sh\\
	pros=ss \textnormal{how} 1a-caus=cond \textbf{unsp}-1a-\textnormal{eat}=dub=psbl ~ ~ ~ ~ ~ ~ ~ ~ ~  pv-\textnormal{say}=narr=ind.m\\
	\glt `and how might I be going to eat if I do that?' \citep[46]{hollow1973b}
	
	\item \glll máatah íwakahąą kasíimi \textbf{wa}'éroomako'sh\\
	wąątah i-waka-hąą ka-sii=awį \textbf{wa}-E=oowąk=o'sh\\
	\textnormal{river} pv.poss-\textnormal{edge}=ins incp-\textnormal{travel}=cont \textbf{unsp}-\textnormal{hear}=narr=ind.m\\
	\glt `traveling along the river's edge, he heard something' \citep[28]{hollow1973a}

	
	\end{xlist}
	
\end{exe}

The unspecified argument prefix appears frequently throughout the corpus and is one of the most common prefixal elements in Mandan. It is typically used on verbal elements, though it does show up on certain nominalized verbs. In these verbs, the \textit{wa-} is a non-agent argument. No overt subject marking is present on these nominalizations, which contrasts with \textit{waa-}, where \textit{waa-} takes the place of a subject.

In a way, \textit{wa-} is used employed to nominalize verbs in a similar manner to preverbs. Mandan has no dedicated third person marking in its prefix field, so a verb with a preverb along with the habitual aspect marker is often the form nominalized verbs take. We can see examples of this nominalizing process through the use of preverbs in (\ref{ExPreverbNominals}) below.

\begin{exe}
\item\label{ExPreverbNominals} Nominalizations using preverbs

	\begin{xlist}
	
	\item \glll íkakiishka\\
	i-ka-kiish=ka\\
	pv.ins-ins.frce-\textnormal{brush.off}=hab\\
	\glt `broom' (lit. `what one sweeps it with') \citep[112]{hollow1970}
	
	\item \glll pá~íwa'xų'ka\\
	pa\#i-wa'-xų'=ka\\
	\textnormal{head}\#pv.ins-ins.prce-\textnormal{plow}=hab\\
	\glt `comb' (lit. `what one plows rows on the head with') \citep[142]{hollow1970}
	
	\item \glll manásh~írushtat\\
	wrąsh\#i-ru-shtat\\
	\textnormal{tobacco}\#pv.ins-ins.hand-\textnormal{twist}\\
	\glt `cigarette' (lit. `what one twists the tobacco by hand with') \citep[238]{hollow1970}
	
	\item \glll ítkek\\
	i-tkE=k(a)\\
	pv.ins-\textnormal{scape}=hab\\
	\glt `hide scraper' (lit. `what one scrapes it with') \citep[253]{hollow1970}
	
	\end{xlist}

\end{exe}

The examples above can either be treated as predicates or arguments, depending on the context. The behavior of \textit{wa-}, along with the use of preverbs to create relative clauses that can be treated as nominal elements, receiving nominal morphology. The presence of \textit{wa-} in many Mandan nouns underscores the highly verbal nature of the Mandan language, and that syntactic categories are not so cut and dried (see \sectref{allomorphgrammaticalized} to see grammaticalized unspecified argument markers on nouns or the use of the nominalizer \textit{waa}- with verbs in \sectref{nominalizer1}).

\subsubsubsubsection{Allomorph /w'-/}\label{SubSecAllomorphw'}

In situations where an unspecified argument marker is placed before a vowel-initial stem that is not a preverb, the allomorph /w'-/ is used instead. We can see this allomorph of \textit{wa}- in the data in (\ref{ExWPunsp}) below.

\begin{exe}

\item\label{ExWPunsp} Examples of unspecified argument marker as /w'-/

	\begin{xlist}
	
	\item \glll \textbf{w}á'kupe\\
	\textbf{w'}-aakup=E\\
	\textbf{unsp}-\textnormal{cover.the.head}=sv\\
	\glt `cap, war bonnet' \citep[58]{hollow1970}
	
	\item \glll \textbf{m}a'se\\
	\textbf{w'}-ąs=E\\
	\textbf{unsp}-\textnormal{horn}=sv\\
	\glt `horn spoon' \citep[60]{hollow1970}

	\end{xlist}

\end{exe}

The set above represents the sum total of cases where the unspecified argument marker is realized as /w'-/ instead of /wa-/ in the corpus. Before preverbs, \textit{wa-} is exclusively used, as we can see in (\ref{ExWAvsPV}) below.

\begin{exe}

\item\label{ExWAvsPV} \glll \textbf{wa}'íwaseko'sh, *wí'waseko'sh\\
	\textbf{wa}-i-wa-sek=o'sh\\
	\textbf{unsp}-pv.ins-\textnormal{make}=ind.m\\
	\glt `I am working' \citep[203]{hollow1970}

\end{exe}

The extremely limited set of verbs beginning with a vowel that is not a preverb and can take unspecified argument marking means that this particular allomorph is exceedingly rare.

\subsubsubsubsection{Allomorph /wą-/}\label{allomorphgrammaticalized}

In certain words, the unspecified argument prefix \textit{wa-} has developed a nasal underlying vowel. We can look at older Mandan sources in (\ref{waorma}) and see what once was [wa] is now [mã] in contemporary Mandan.

\begin{exe}

\item\label{waorma} Change from /wa-/ to /wą-/

	\begin{xlist}
	
	\item `white person'
	
		\begin{xlist}
		
		\item \textit{washí} $\leftarrow$ /wa-shi/ `white person' (lit. `someone who has everything good' or `someone who is dressed well') \citep[246]{maximilian1839}
		
		\item \textit{mashí} $\leftarrow$ /wą-shi/ `white person' \citep[276]{hollow1970} 
		
		\end{xlist}
		
	\item `leather'
	
		\begin{xlist}
		
		\item \textit{wapą́ąpi} $\leftarrow$ /wa-pąąpi/ `leather' (lit. `something made thin') \citep[249]{maximilian1839}
		
		\item \textit{wapą́ąpi} $\leftarrow$ /wa-pąąpi/ `buckskin' \citep[136]{hollow1970}
		
		\item \textit{mapą́ąpi} $\leftarrow$ /wą-pąąpi/ `deer, buckskin' (\citeauthor{benson2000} p.c.)
		
		\end{xlist}
	
	\end{xlist}

\end{exe}

The tendency to fortify sonorants to nasal stops when utterance-initial has caused certain words beginning with sonorants to become reanalyzed as if the source of those nasal stops was from an underlying nasal vowel. We can see that in the early nineteenth century, the modern word for someone of European descent is \textit{washí} [wa.ˈʃi], where the first vowel is definitely oral. All modern speakers use \textit{mashí} [mã.ˈʃi]. Other words in Mandan have reanalyzed the unspecified argument prefix as having an underlying nasal vowel, such that there is a class of nouns where this nasalization has become lexicalized. Mr. Edwin Benson explained the origin of the term for `white person', and his interpretation matches up with \citeapos{maximilian1839}. The data in (\ref{waorma}) show that certain words have shifted universally towards taking /wą-/ as their unspecified argument prefix, but other words may take either /wa-/ or /wą-/, depending on the speaker.\footnote{Mr. \citeauthor{benson2000} (p.c.) reported that this term came about after seeing European-American cavalry in the 1800s in their blue uniforms. Several speakers of Mandan have provided an alternative explanation that the word for `white person' comes from \textit{mashí'na}, meaning `generous', which may be related to the /wa-shi/ `\textsc{unsp}-be.good' explanation found in \citet{maximilian1839} anyway. Similar terms with supposedly different etymologies for `white person' found in Hidatsa (i.e., \textit{mashíi} /washii/) and Lakota/Dakota (i.e., \textit{wašíču}) raise the question of where this word originates and whether this explanation in Mandan is the original or if this is a folk etymology.}

We can see other nouns in Mandan that share this use of /wą-/ for the unspecified argument prefix in (\ref{manotwa}) below.

\begin{exe}

\item\label{manotwa} Contemporary nouns with /wą-/ instead of /wa-/

	\begin{xlist}
	
	\item \textit{mapáakokohka} $\leftarrow$ /wą-paa-ko$\sim$kok=ka/ `butterfly' \citep[270]{hollow1970}
	
	\item \textit{mapíhka} $\leftarrow$ /wą-pih=ka/ `beetle, stink bug' \citep[490]{hollow1970}
	
	\item \textit{maxáxaare} $\leftarrow$ /wą-xa$\sim$xaa=E/ `mountain' \citep[490]{hollow1970}
	
	\item \textit{maxópinixte} $\leftarrow$ /wą-xoprį\#xtE/ `God' \citep[490]{hollow1970}
	
	\item \textit{mapí'ksok} $\leftarrow$ /wą-pi'ksok/ `wild strawberry, raspberry' \citep[272]{hollow1970}
	
	\end{xlist}

\end{exe}

None of the words in (\ref{manotwa}) above appear anywhere in the corpus with /wa-/, and will always involve /wą-/.  There is no common element that connects these data, so we can say that this is an incomplete morphological change, and one that may not be entirely stable. It is likely the case that this element has become grammaticalized onto these stems, but since we do not see instances of /wą-/ interacting with other person-marking prefixes in the corpus, it is not possible to conclusively state that these forms are not complex, i.e., composed of multiple discrete morphological elements.

\subsubsubsection{Nominalizer prefix: \textit{waa-}\textsubscript{1}}\label{nominalizer1}

The nominalizer \textit{waa}- is not inflectional morphology, but is included in this section to highlight its difference from the unspecified argument prefix. All past scholars have treated these two prefixes as being one and the same, but using instrumentation such as Praat \citep{boersmaweenik2016}, we can see that there is a distinct length difference between these two formatives.

The semantics of the nominalizer are different from that of the unspecified argument marker in that, while both can be used to nominalize a verb, the nominalizer grants a subject-type reading to the verb it nominalizes. This is often rendered into English as `something/someone that X-es', where the X stands for a verb. We can take the word \textit{wáarokhuutop} `table' in (\ref{ExTable}) as an example of this nominalizer in action.

\begin{exe}

\item\label{ExTable} \glll wáarokhuutop\\
	waa-rok\#huu\#top\\
	nom-\textnormal{leg}\#\textnormal{bone}\#\textnormal{four}\\
	\glt `table' (lit. `something that has four leg bones')

\end{exe}

The word `table' begins with a nominalizer, and turns the entire determiner phrase into a single phonological word that means `something that has four leg bones.' We can add additional nominal morphology, such as determiners or quantifiers, e.g., \textit{wáarokhuutop tóp} `four tables', or literally `four somethings that have four leg bones.' Since Mandan does not easily accept loanwords into its lexicon, novel items and concepts will require a novel word that describes what this new item or concept is. As such, the corpus is filled with tokens containing the nominalizer. We can see several examples of \textit{waa-} used in this way in (\ref{ExNomWAA}) below.

\begin{exe}

\item\label{ExNomWAA} Examples of nominalizer \textit{waa-} 

	\begin{xlist}
	
	\item \glll wáara'xuu\\
	waa-ra'-xuu\\
	nom-ins.heat-\textnormal{be.charred}\\
	\glt `coffee (lit. `something that is charred')' \citep[330]{hollow1970}
	
	\item \glll wáaxte\\
	waa-xtE\\
	nom-\textnormal{be.big}\\
	\glt `a lot (`something that is big')' \citep[328]{hollow1970}
	
	\item \glll wáa'atxi'hs\\
	waa-at\#xi'h=s\\
	nom-\textnormal{father}\#\textnormal{be.old}=def\\
	\glt `the President (lit. `someone who is the grandfather)' \citep[61]{hollow1970}
	
	\item \glll wáa'iniire\\
	waa-irįį=E\\
	nom-\textnormal{grow}=sv\\
	\glt `plant (lit. `something that grows')' \citep[62]{hollow1970}
	
	\item \glll wáa'opakirii\\
	waa-o-pa-krii\\
	nom-pv.loc-ins.push-\textnormal{line.up}\\
	\glt `a number (lit. `something that counts')' \citep[119]{hollow1970}
	
	\item \glll wíikapus\\
	wV-i-ka-pus\\
	nom-pv.ins-ins.frce-\textnormal{be.marked}\\
	\glt 	`pencil (lit. `something that makes marks')' \citep[77]{hollow1970}
	
	\end{xlist}

\end{exe}

In each of the examples above, the \textit{waa-} acts as an indefinite, unspecified subject. Both active and stative verbs are able to be nominalized with \textit{waa-}. The nominalizer also can optionally contract before a vowel-initial stem, e.g., \textit{wáa'ikapus} and \textit{wíikapus} both mean `writing utensil.' This allomorph is completely optional, and speakers have been known to spontaneously switch from a contracted /wV-/ to a full /waa-/ or vice versa during data elicitation sessions.

\subsubsubsection{Partitive prefix: \textit{waa-}\textsubscript{2}}

An additional use of \textit{waa-} is to give a partitive meaning to some object. This prefix often is accompanied by the word `some' in the gloss, as we see in (\ref{ExWAApart}) below.

\begin{exe}

\item\label{ExWAApart} Partitive \textit{waa-}

	\begin{xlist}
	
	\item \glll \textbf{máa}maku'nista, \textbf{wáa}waruusto'sh!\\
	\textbf{waa}-wą-ku'=rįt=ta \textbf{waa}-wa-ruut=kt=o'sh\\
	\textbf{prtv}-1s-\textnormal{give}=2pl=imp.m \textbf{prtv}-1a-\textnormal{eat}=pot=ind.m\\
	\glt `give me some, [because] I want to eat some!' \citep[15]{hollow1973a}
	
	\item \glll nuharáa íxike \textbf{máa}nuruha'ni ~ ~ ~ ~ ~ nukirí'sh\\
	rų-hrE=$\varnothing$ i-xik=E \textbf{waa}-rų-ha'=rį ~ ~ ~ ~ ~ rų-kri=o'sh\\
	1a.pl-caus=cont pv.ins-\textnormal{be.bad}=sv \textbf{prtv}-1a.pl-\textnormal{pick.berries}=ss ~ ~ ~ ~ ~ 1a.pl-\textnormal{arrive.back.here}=ind.m\\
	\glt `we were doing it and we barely picked some and we came back' \citep[52]{hollow1973a}
	
	\item \glll hiré rá'skamak \textbf{wáa}wahe'sh, manápusheke\\
	hire ra'ska\#wąk \textbf{waa}-wa-hE=o'sh wrą\#pushek=E\\
	\textnormal{now} \textnormal{summer}\#pos.lie \textbf{prtv}-1a-\textnormal{see}=ind.m \textnormal{tree}\#\textnormal{juneberry}=sv\\
	\glt `I saw some now this summer, juneberries that is' \citep[52]{hollow1973a}
	
	\item \glll nitawiihąka máakahe \textbf{wáa}'o'kiharaa ká'harani ~ ~ ~ ~ ~ éexi rúut íteekereka're\\
	rį-ta-wiihąka wąąkahe \textbf{waa}-o'ki-hrE ka'\#hrE=rį ~ ~ ~ ~ ~  eexi ruut i-tee=krE=ka=o're\\
	2poss-al-\textnormal{grandchild} \textnormal{those} \textbf{prtv}-\textnormal{cook}\#caus \textnormal{have}\#caus=ss ~ ~ ~ ~ ~ \textnormal{paunch} \textnormal{eat} pv.ins-\textnormal{like}=3pl=hab=ind.f\\
	\glt `those granddaughters of yours cooked some for her and want to eat paunch' \citep[72]{hollow1973a}
	
	\item \glll máhe, \textbf{wáa}'aawakuhini ~ ~ ~ ~ ~ ~ ~ ~ ~ ~ ~ ~ ~ ~ ~ ówakįhereki nurúutinisto'sh\\
	wąh=E \textbf{waa}-aa-wa-kuh=rį ~ ~ ~ ~ ~ ~ ~ ~ ~ ~ ~ ~ ~ ~ ~  o-wa-kį\#hrE=ki rų-ruut=rįt=kt=o'sh\\
	\textnormal{turnip}=sv \textbf{prtv}-pv.tr-1a-\textnormal{come.back.here}=ss ~ ~ ~ ~ ~ ~ ~ ~ ~ ~ ~ ~ ~ ~ ~ pv.loc-1a-\textnormal{boil}\#caus=cond 1a.pl-\textnormal{eat}=2pl=pot=ind.m\\
	\glt `turnips, I will bring some back and we will eat them when I cook them' \citep[75]{hollow1973a}
	
	\end{xlist}

\end{exe}

It is likely that this \textit{waa-} originated as an enclitic quantifier with a structure in Pre-Mandan like \textit{máhe waa=} `some turnips', but the /waa/ become reanalyzed as being an integral part of the verb. \citet[480]{park2012} describes a similar use of the quantifier \textit{nuwa}, which encliticized onto an overt nominal, but is also realized within the verbal complex when referencing a covert DP. Contemporary Mandan does not have an enclitic for the partitive; it will always be realized as part of the verb as a prefix. This prefix is mutually exclusive with the unspecified argument prefix, suggesting that they both compete for the same slot because they are both a kind of inflectional prefix that is agreeing with an indefinite or non-specific argument.

\subsubsubsection{Indefinite subject \textit{waa-}\textsubscript{3}}\label{SubSubSubSectionIndefiniteSubject}

Another use for \textit{waa-} that seems to have occured more recently in Mandan than the other two \textit{waa-} prefixes is to mark indefinite subjects. Words bearing this \textit{waa}- are often glossed as `someone' or `somebody' doing an action. We can tell that they are bound elements rather than being independent DPs by observing that they take primary stress and that the following stem can take secondary stress. This behavior of stress assignment shows that this \textit{waa}- is being treated as a prefix and not a free DP or proclitic, as clitics may not take primary stress in Mandan. We can see examples of this indefinite subject reading of \textit{waa}- in (\ref{ExWAAindef}) below.



\begin{exe}

\item\label{ExWAAindef} Examples of the indefinite subject \textit{waa-}

	\begin{xlist}
	
	\item \glll weréhe kasé'harani ~ ~ ~ ~ ~ ~ ~ ~ ~ ~ ~ ~ ~ ~ ~ \textbf{wáa}huuki, káare rusé'na!\\
	wreh=E ka-se'\#hrE=rį ~ ~ ~ ~ ~ ~ ~ ~ ~ ~ ~ ~ ~ ~ ~ \textbf{waa}-huu=ki kaare ru-se'=rą\\
	\textnormal{door}=sv ins.frce-\textnormal{have.come.open}\#caus=ss ~ ~ ~ ~ ~ ~ ~ ~ ~ ~ ~ ~ ~ ~ ~ \textbf{\textnormal{someone}}-\textnormal{come.here}=cond imp.neg ins.hand-\textnormal{have.come.open}=imp.f\\
	\glt `lock the door and if someone comes, do not open it!' \citep[137]{hollow1973a}

	\item \glll ishtų́here'eshkaki, \textbf{wáa}huura ~ ~ ~ ~ ~ ~ ~ ~ ~ ~ éroomako'sh\\
	ishtų́h\#hrE-eshka=ki \textbf{waa}-huu=E=$\varnothing$ ~ ~ ~ ~ ~ ~ ~ ~ ~ ~ E=oowąk=o'sh\\
	\textnormal{night}\#caus-smlt=cond \textbf{\textnormal{someone}}-\textnormal{come.here}=sv=cont ~ ~ ~ ~ ~ ~ ~ ~ ~ ~ \textnormal{hear}=narr=ind.m\\
	\glt `when it was about evening time, he heard someone coming' \citep[200]{hollow1973b}
	
	\item \glll óo ó'harani \textbf{wáa}rataxa ~ ~ héroomako'sh\\
	oo o'\#hrE=rį \textbf{waa}-ra-tax=E=$\varnothing$ ~ ~   hE=oowąk=o'sh\\
	dem.mid \textnormal{be}\#caus=ss \textbf{\textnormal{someone}}-ins.mth-\textnormal{make.loud.noise}=sv=cont ~ ~ \textnormal{see}=narr=ind.m\\
	\glt `from there, he heard someone crying' \citep[249]{hollow1973b}
	
	\item \glll \textbf{wáa}ti ishíka'sh\\
	\textbf{waa}-ti ishi=ka=o'sh\\
	\textbf{\textnormal{someone}}-\textnormal{arrive.here} vis=hab=ind.m\\
	\glt `someone must be coming here' \citep[142]{hollow1973a}

	\end{xlist}

\end{exe}

This element seems to be able to be used as an unbound item that is usually combined with the topic marker. This element also has underlying form of /waa/, but it does not necessarily refer to a subject. Typically, if this formative is used as a subject, the verb is marked for as being singular. However, if there is a plural reading intended, the verb can take plural morphology, as we see in (\ref{nominalwaa}) below.



\begin{exe}

\item\label{nominalwaa} Examples of nominal \textit{waa}

	\begin{xlist}
	
	\item \glll \textbf{máa}na ípakixtiki ~ ~ ~ ~ ~ ~ ~ ~ ~ ~ ~ ~  íhehka'sh\\
	\textbf{waa}=rą i-pa-kixti=ki ~ ~ ~ ~ ~ ~ ~ ~ ~ ~ ~ ~  i-hek=ka=o'sh\\
	\textbf{\textnormal{someone}}=top pv.ins-ins.push-\textnormal{approach}=cond ~ ~ ~ ~ ~ ~ ~ ~ ~ ~ ~ ~ pv.ins-\textnormal{know}=hab=ind.m\\
	\glt `if someone gets close, he always knows' \citep[155]{hollow1973b}
	
	\item \glll \textbf{máa}na ų́ųpani ropxékerekaroomako'sh\\
	\textbf{waa}=rą ųųpa=rį ropxE=krE=ka=oowąk=o'sh\\
	\textbf{\textnormal{someone}}=top \textnormal{with}=ss \textnormal{enter}=3pl=hab=ind.m\\
	\glt `they usually go in with someone' \citep[171]{hollow1973b}
	
	\item \glll hiré'oshka, \textbf{máa}na kisúkini\\
	hire-oshka \textbf{waa}=rą ki-suk=rį\\
	\textnormal{now}-emph \textbf{\textnormal{someone}}=top vert-\textnormal{exit}=ss\\
	\glt `even now, someone goes back out and...' \citep[207]{hollow1973b}
	
	\item\label{nominalwaaD} \glll \textbf{máa}na ókų'hkerektiki,~ ~ ~ ~ ~ ~ ~ ~ ~ ~  warúkahsįhxteka'eheero'sh\\
	\textbf{waa}=rą o-k'-ųh=krE=kti=ki ~ ~ ~ ~ ~ ~ ~ ~ ~ wa-rukah-sįh-xtE=ka'ehEE=o'sh\\
	\textbf{\textnormal{someone}}=top pv.loc-3poss.pers-\textnormal{wife}=3pl=pot=cond ~ ~ ~ ~ ~ ~ ~ ~ ~ unsp-\textnormal{refuse}-ints-aug=quot=ind.m\\
	\glt `whenever someone would try to marry her, she always strongly refused, it is said' \citep[101]{hollow1973a} 
	
	\item\label{waaexample} \glll ptamíihe, máatki \textbf{máa}taa ~ ~ ~ ~ ~ ~ ~ ~ ~ ~ wáa'owaraahinixo're\\
	p-ta-wįįh=E wąątki \textbf{waa}=taa ~ ~ ~ ~ ~ ~ ~ ~ ~ ~  waa-o-wa-rEEh=rįx=o're\\
	1poss-al-\textnormal{male's.sister}=sv \textnormal{tomorrow} \textbf{\textnormal{somewhere}}=loc ~ ~ ~ ~ ~ ~ ~ ~ ~ ~  neg-pv.irr-1a-\textnormal{go.there}=neg=ind.f\\
	\glt `my sister, I am not going anywhere tomorrow' \citep[284]{hollow1973a}
	
	\item\label{waainside} \glll í\textbf{waa}t óreeho'sh\\
	i-\textbf{waa}=t o-rEEh=o'sh\\
	pv.dir-\textbf{\textnormal{somewhere}}=loc pv.irr-\textnormal{go.there}=ind.m\\
	\glt `he is going away' \citep[265]{hollow1970}
	
	\item\label{waaalone} \glll \textbf{wáa} warúute tú éheeni ko'ó'kto'sh\\
	\textbf{waa} wa-ruutE tu e-hee=rį ko-o'=kt=o'sh\\
	\textbf{\textnormal{someone}} unsp-\textnormal{eat} \textnormal{be.some} pv-\textnormal{say}=ss rel-\textnormal{be}=pot=ind.m\\
	\glt `there may be someone who is hungry' \citep[159]{trechter2012b}
	
	
	\end{xlist}

\end{exe}

We can see that \textit{máana} `someone' bears topic marking, and can be used in contexts other than subjects. Furthermore, \textit{maa-} is able to be used to substitute for unknown places as well, as in \textit{máataa} `somewhere'. These `someone' and `somewhere' words tend to be the initial element of an intonational phrase, which triggers intonational phrase-initial fortition. Additional morphology added before the /w/, like in (\ref{waainside}), removes the conditions for intonational phrase-initial fortition, causing the /w/ to be realized as [w] instead of [m]. Furthermore, we see a few instances of /waa/ alone in the corpus, as demonstrated in (\ref{waaalone}), where \textit{wáa} appears without the topic marker. Thus, while \textit{waa} often appears in the corpus with an initial [m], this [m] is completely due to this item being used as some kind of topic or emphasized element, which places it at the rightmost edge of its own intonational phrase. We can see the prosodic structure of (\ref{waaexample}) in the example below.

\begin{exe}
\item Prosodic structure of (\ref{waaexample})

\textsubscript{Utt}( \textsubscript{ιP}( \textit{ptamíihe,} )\textsubscript{ιP} \textsubscript{ιP}( \textit{máatki} )\textsubscript{ιP} \textsubscript{ιP}( \textit{máataa} \textit{wáa'owaraahinixo're} )\textsubscript{ιP} )\textsubscript{Utt}

\end{exe}

This /waa/ generally combines with enclitics, e.g., the locative =\textit{taa}, showing that it is a lexical root, likely expressing an uncertain person or place. This element is clearly nominal in nature, as it is only seen with nominal markings, e.g., topic and locative enclitics. As such, these constructions appear to be vestigial indefinite or partitive pronouns, which have generally been reinterpreted as being either partitive or indefinite subject markers in contemporary Mandan. 

\subsubsubsection{Summary of \textit{wa-} and \textit{waa-} prefixes}

This confusion between the unspecified argument prefix \textit{wa-} and the various \textit{waa-} formatives stems from their common trait of marking indefiniteness for a particular argument. The different phonetic shapes is one clue for which version of an indefinite that the speaker wishes to convey, but the fact that this distinction comes from vowel length and not any other cue has made it difficult for past researchers who are inconsistent about marking vowel length (e.g., \citealt{kennard1936} and \citealt{trechter2012b}) or impossible for those who do no recognize vowel length at all (e.g., \citealt{hollow1970} and \citealt{coberly1979}).

Other Siouan languages have a prefix that is cognate with the indefinite argument marker \textit{wa-} in Mandan, with that same vowel length. We can thus suppose that there was some element *wa- in Proto-Siouan. The presence of *waa- has not been discussed in the literature as a separate piece of morphology in Proto-Siouan, however. According to \citet{kasak2015}, the closest relatives to Mandan are Hidatsa and Crow, which both have productive cognates of Mandan \textit{waa-}, with those cognates being /waa-/ in Hidatsa and /maa-/ in Crow. This element fulfills all the roles that both \textit{wa-} and \textit{waa-} play in Mandan, but there are fossilized remnants of *wa- in certain lexical items, such as the Hidatsa word \textit{madú} `be some', which is a cognate of the Mandan \textit{tú} `be some.' The initial syllable in the Hidatsa word bears the short vowel of *wa- instead of the contemporary /waa-/ prefix, indicating it was a morphologically complex construction in Proto-Missouri Valley Siouan (e.g., PSi *wa-tu > PMsrV **wa-tu > Hid \textit{madu}). We can surmise that *wa- merged with *waa- in Proto-Missouri Valley but remained distinct in Mandan.

Further evidence that Mandan did not innovate this *wa- versus *waa- distinction can be seen in the Ohio Valley language Tutelo, where \citet{oliverio1997} marks the absolutive prefix as either \textit{wa-} or \textit{waa-} with a high level of inconsistency. One motivating factor for this length difference is that Tutelo also inherited this distinction between PSi *wa- and *waa-. Catawba has a proclitic \textit{pa} that is used for indefinite plural objects that is a cognate with PSi *wa-, though its status as a proclitic seems to be an atavism where this Proto-Siouan prefix has ceased being an integral part of the morphological word. 

Mandan seems to be the only language that can allow a morphologically unbound item \textit{wáa} to mean an indefinite subject. Generally speaking, this element appears as an inflectional prefix on the verb in the same verbal slot as the unspecified argument marker. It seems that \textit{wáa} is used as an independent word only as a last resort for when a situation where an indefinite subject is acting on an indefinite object like in (\ref{waaalone}), where the unspecified argument marker \textit{wa-} is marked on the verb, precluding any other indefinite marker from appearing. Thus, it is not possible for multiple instances of indefiniteness to be marked on a Mandan verb, i.e., the construction *\textit{wáawaruute tú} `there is someone who is hungry' is illicit because \textit{warúute} `be hungry' already bears an unspecified argument marker. When such situations where multiple indefinite arguments arise, the indefinite object is marked on the verb and the indefinite subject manifests as an independent argument. Otherwise, all instances of \textit{wa-} and \textit{waa-} can only appear in the same slot within the prefix template.

\subsubsection{Negative prefix (Slot 10)}\label{SubsubsecNegative}

The negative prefix \textit{waa-} always co-occurs with a negative enclitic, such as \textit{=nix} or \textit{=xi}.\footnote{These two enclitics are not phonologically related in the synchrony, but are diachronically related. One negation marker in Proto-Siouan is *-ši, which can be realized as *-xi due to sound symbolism, a consonantal ablaut that changes the place of articulation of a fricative to indicate the level of intensity of a state or action (cf. \sectref{soundsymbolism}). Another negation marker is PSi -rį. The /=xi/ in Mandan is an obvious reflex of the \textit{x}-grade form of *-aši. The /=rįx/, however, is actually a combination of the two: PSi *-rį-axi, where the final vowel in *-axi is deleted and the initial vowel is deleted to avoid hiatus.} This circumfix-like behavior is due to Mandan reanalyzing the indefinite argument \textit{wáa} as being part of the verbal complex. It is likely that this morphologically unbound element gradually became grammaticalized into an inflectional prefix due to the frequency of this indefinite \textit{wáa} occurring with negative constructions. Circumfixal or double marking of negation is not typologically uncommon \citep[630]{caffareletal2004}, though Mandan is the only Siouan language besides Biloxi (cf. \citealt[86]{einaudi1976}) to mark negation twice within the verbal complex, as seen in (\ref{waanegation}) below.

\begin{exe}
\item\label{waanegation} Examples of negative \textit{waa-}

	\begin{xlist}
	
	\item\label{waanegation1}
	\glll \textbf{wáa}'owaraahinixo'sh\\
	\textbf{waa}-o-wa-rEEh=rįx=o'sh\\
	\textbf{neg}-pv.irr-1a-\textnormal{go.there}=neg=ind.m\\
	\glt `I am not going to go' \citep[48]{hollow1973a}
	
	\item\label{waanegation2}
	\glll téehą \textbf{wáa}raki'hinixak tashkák ~ ~ ~ ~ ~ éewereho'sh\\
	teehą \textbf{waa}-ra-ki'h=rįx=ak tashka=ak ~ ~ ~ ~ ~  ee-we-reh=o'sh\\
	\textnormal{be.long.distance} \textbf{neg}-2a-\textnormal{arrive.back.there}=neg=ds \textnormal{how}=ds ~ ~ ~ ~ ~ pv-1a-\textnormal{think}=ind.m\\
	\glt `I wondered why you had not returned for so long' \citep[318]{hollow1973b}
	
	\item\label{waanegation3}
	\glll wáatishi'sh, kashká \textbf{wáa}rakina'nixo'sh\\
	waa-ti=ishi=o'sh kashka \textbf{waa}-ra-kirą'=rįx=o'sh\\
	\textnormal{someone}-\textnormal{arrive.here}=vis=ind.m \textnormal{but} \textbf{neg}-2a-\textnormal{tell}=neg=ind.m\\
	\glt `someone must have been here, but you are not telling' \citep[162]{hollow1973a}

	\item\label{waanegation4}
	\glll warúshaani \textbf{máa}mahaxik í'ų'taa ~ ~ ~    résh ótaawaherektak, ~ ~ ~ ~ ~ ~ ~ ~ ~ ~ ~ ~ ~ ~ ~ xé'hąkseet téeroomako're\\
	wa-ru-shE=rį \textbf{waa}-wą-hE=xi=ak i-ų'=taa ~ ~ ~ resh o-taa\#wa-hrE=kt=ak ~ ~ ~ ~ ~ ~ ~ ~ ~ ~ ~ ~ ~ ~ ~ xe'h\#hąk=s=ee=t tee=oowąk=o're\\
	1a-ins.hand-\textnormal{take}=ss \textbf{neg}-1s-\textnormal{see}=neg=ds pv.dir-\textnormal{be.closer}=loc ~ ~ ~ \textnormal{this.way} pv.loc-\textnormal{be.facing}\#1a-caus=pot=ds ~ ~ ~ ~ ~ ~ ~ ~ ~ ~ ~ ~ ~ ~ ~ \textnormal{be.dripping}\#pos.stnd=def=dem.dist=loc \textnormal{die}=narr=ind.f\\
	\glt `I took it and, while she was not looking at me, when I faced it this way toward her, she was dead in that basket' \citep[92]{hollow1973a}
	
	\item\label{waanegation5}
	\glll \textbf{wáa}raraahinixki, óxiko'sh\\
	\textbf{waa}-ra-rEEh=rįx=ki o-xik=o'sh\\
	\textbf{neg}-2a-\textnormal{go. there}=neg=cond pv.irr-\textnormal{be.bad}=ind.m\\
	\glt `if you do not go, it will be bad' \citep[113]{hollow1973a}
	
	\item\label{waanegation6}
	\glll ą́'t minikíkųųteki \textbf{wáa}'oteeniharaxi'sh\\
	ą't w-rį-kikųųtE=ki \textbf{waa}-o-tee\#rį-hrE=xi=o'sh\\
	dem.anap 1a-2s-\textnormal{help}=cond \textbf{neg}-pv.irr-\textnormal{die}\#2s-caus=neg=ind.m\\
	\glt `That one will not kill you if I help you' \citep[113]{hollow1973a}
	
	\item\label{waanegation7}
	\glll \textbf{wáa}ra'hashinixharaa, ~ ~ ~ ~ ~ ~ ~ ~ ~ ~ ~ ~ rá'pus'harani réehak...\\
	\textbf{waa}-ra'-hash=rįx\#hrE=$\varnothing$ ~ ~ ~ ~ ~ ~ ~ ~ ~ ~ ~ ~  ra'-pus\#hrE=rį rEEh=ak\\
	\textbf{neg}-ins.heat-\textnormal{be.disintegrated}=neg\#caus=cont ~ ~ ~ ~ ~ ~ ~ ~ ~ ~ ~ ~ ins.heat-\textnormal{be.streaked}\#caus=ss \textnormal{go.there}=ds\\
	\glt `not burning him up, it just scorched him in streaks and went...' \citep[154]{hollow1973a}
	
	\item\label{waanegation8}
	\glll miníseet kiskéktiki, ~ ~ ~ ~ ~ ~ ~ ~ ~ ~ ~ írexseena ~ ~ ~ ~ ~ ~ ~ ~ ~ ~ ~ ~ ~ ~ ~ ~ ~ ~ ~ \textbf{wáa}ksipharaxiktiki, íkxąhini...\\
	wrį=s=ee=t ki-skE=kti=ki ~ ~ ~ ~ ~ ~ ~ ~ ~ ~ ~  i-rex=s=ee=rą ~ ~ ~ ~ ~ ~ ~ ~ ~ ~ ~ ~ ~ ~ ~ ~ ~ ~ ~ \textbf{waa}-ksip\#hrE=xi=kti=ki i-kxąh=rį\\
	\textnormal{water}=def=dem.dist=loc refl-\textnormal{jump}=pot=cond ~ ~ ~ ~ ~ ~ ~ ~ ~ ~ ~  pv.ins-\textnormal{be.glistening}=def=dem.dist=top ~ ~ ~ ~ ~ ~ ~ ~ ~ ~ ~ ~ ~ ~ ~ ~ ~ ~ ~  \textbf{neg}-\textnormal{go.underwater}\#caus=neg=pot=cond pv.ins-\textnormal{laugh}=ss\\
	\glt `he laughed when he could not push the bladder underwater when he jumped into the water and...' \citep[164]{hollow1973a}
	
	\item\label{waanegation9}
	\glll éexi rá'xųtak máamanahku éeheni ~ ~ ~ ~  numá'kaaki ópi' éeheni ą́ąwe kisúhkereka'ehe, ~ ~ ~ \textbf{wáa}teenixa núunihkereroo\\
	eexi ra'-xųt=ak waawarąhku ee-he=rį ~ ~ ~ ~ ruwą'k-aaki o-pi' ee-he=rį ąąwe ki-suk=krE=ka'ehe ~ ~ ~   \textbf{waa}-tee-rįx=E=$\varnothing$ ruurįh=krE=oo\\
	\textnormal{belly} ins.heat-\textnormal{be.disintegrated}=ds \textnormal{white.tailed.deer} pv-\textnormal{say}=ss ~ ~ ~ ~  \textnormal{man}-coll pv.irr-\textnormal{devour} pv-\textnormal{say}=ss \textnormal{all} vert-\textnormal{exit}=3pl=quot ~ ~ ~  \textbf{neg}-\textnormal{be.dead}=neg=sv=cont \textnormal{be.there}.pl.dur.aux=3pl=dem.mid\\
	\glt `his stomach burst and all the white tail deer and people that he ate up all came out, it is said, [because] they were not dead there' \citep[171]{hollow1973a}
	
	
	\item\label{waanegation10}
	\glll taté \textbf{wáa}'isekinix rokú' ísekini réeho'sh\\
	tatE \textbf{waa}-i-sek=rįx ro-ku' i-sek=rį rEEh=o'sh\\
	\textnormal{father}.voc \textbf{neg}-pv.ins-\textnormal{do}=ss 1s.pl-\textnormal{give} pv.ins-\textnormal{do}=ss \textnormal{go.there}=ind.m\\
	\glt `father should not have done to us what he went and did' \citep[184]{hollow1973a}
	
	\end{xlist}

\end{exe}

For simplex verbs, like in (\ref{waanegation1}) through (\ref{waanegation5}), where the \textit{waa-} is placed is unambiguous, i.e., at the leftmost edge of the word. When we look at serial verbs like causative constructions, however, we can see that the \textit{waa-} will always prefix onto the left edge of the overall word, rather than the causative itself. We can see this behavior in (\ref{waanegation6}) through (\ref{waanegation8}). Previous scholars vacillate between transcribing causative constructions as one or two words. The fact that the \textit{waa-} becomes prefixed onto the left edge of the causativzed verb is evidence that causative constructions are analyzed as a single morphological word. The negative enclitics \textit{=nix} or \textit{=xi}, however, can appear on either the causative or on the causativized verb. This variation in enclitic placement is caused by the semantic scope, and will be discussed further in \sectref{SecPhrasalMorphology}.

\subsubsection{Relativized prefix (Slot 11)}\label{SubsubsecRelativized}

The prefix \textit{ko-} marks a relativized construction. This prefix comes from the Proto-Siouan word *ko, a demonstrative. In other Siouan languages, like Crow or Biloxi, the reflexes of PSi *ko are still unbound elements. Crow, in particular, is interesting because the reflexes of *ko do not appear where other demonstratives would in a head-final, left-branching language (i.e., at the right edge of a determiner phrase), but are always DP-initial \citep{graczyk2007}. These reflexes of *ko (i.e., \textit{ko} and \textit{kon}) do not bear a lexical pitch accent, so they appear to be first-position clitics rather than simple determiners in that they will always be in the first position within a DP shell. We can see the treatment of these unstressed first-position demonstratives in Crow in (\ref{ExCrowKOreflex}) below.

\begin{exe}
\item\label{ExCrowKOreflex} Reflexes of PSi *ko in Crow

\glll \textnormal{[}\textbf{ko} bachéesh\textnormal{]\textsubscript{DP}} \textnormal{[}\textbf{kon}\textnormal{]\textsubscript{DP}} díak\\
\textbf{ko}= machée=sh \textbf{kon}= nía=k\\
\textbf{dem} \textnormal{man}=det.def \textbf{dem.agt} \textnormal{do}=ind\\
\glt `that man is the one who did it' \citep[222]{graczyk2007}

\end{exe}

In a DP with an overt nominal like in [\textit{ko bachéesh}]\textsubscript{DP} `that man', the demonstrative appears at the leftmost edge of the DP shell. A demonstrative may also appear without an overt nominal, as \textit{kon} does in the example above, but in cases such as these, the demonstratives rely on the following word to be prosodically realized, as we can tell by its lack of an underlying pitch accent. The fact that the reflex of PSi *ko is DP-initial and prosodically deficient in Crow contrasts with \textit{ko} in Biloxi, which is always the rightmost element in a DP \citep{torres2010}. We can see this DP-final Biloxi \textit{ko} in (\ref{ExBiloxiKOreflex}) below.

\begin{exe} 
\item\label{ExBiloxiKOreflex} Reflexes of PSi *ko in Biloxi

\glll \textnormal{[}Opanaskêhǫna \textbf{ko}\textnormal{]\textsubscript{DP}} naxê'ǫǫką ~ ~ ~ ~ ~ ~ ~ ~ ~ ~ ~ ~ ~ ~ ~ ~ ~ ~ ~ ~ tandoyą kidi dąde\\
	o\#pa\#naskê\#hǫna \textbf{ko} naxê=ǫǫ=ką ~ ~ ~ ~ ~ ~ ~ ~ ~ ~ ~ ~ ~ ~ ~ ~ ~ ~ ~ ~ tando=yą ki-di dąde\\
	\textnormal{fish}\#\textnormal{head}\#\textnormal{long}\#\textnormal{be.like} \textbf{dem} \textnormal{hear}=pst=ds ~ ~ ~ ~ ~ ~ ~ ~ ~ ~ ~ ~ ~ ~ ~ ~ ~ ~ ~ ~  \textnormal{female's.younger.brother}=det.def vert-\textnormal{arrive.here} irr\\
	\glt `That Very-Long-Headed-Fish heard that her brother would be coming back' \citep[128]{torres2010}

\end{exe}

The behavior of \textit{ko} in Biloxi is in line with the expected distribution of a demonstrative within a DP shell in a head-final, left-branching language (i.e., at the right edge of a DP). We can certainly see that reflexes of PSi *ko in both Crow and Biloxi have similar functions, but the stark difference lies in their distribution and prosodic behavior. 

Under the assumption that Crow and Hidatsa are the closest relatives to Mandan, we can assume that their common ancestor language likewise had a constraint where this determiner was always DP-initial, and that this determiner in Mandan eventually became reanalyzed as an inflectional marker on the verb, rather than discrete lexical item. We can see examples of \textit{ko-} in Mandan below.

The prefix \textit{ko-} in Mandan is not commonly seen in the corpus, as preverbs and unspecified argument markers have the ability to relativize a clause, with the locative \textit{o-} in particular being commonly seen when describing places. Constructions with \textit{ko-} typically make reference to an agentive argument, though this argument is not always animate. We can see examples of the relativizer \textit{ko}- in the data in (\ref{ExMandanKOreflex}) below.

\begin{exe}
\item\label{ExMandanKOreflex} Examples of relativizer \textit{ko-} in Mandan

	\begin{xlist}
	
	\item \glll mí'ti kų́'haa áani máapehekere, mí'ti ~ ~ ~  ~ ~ ~ \textbf{ko}táakeres ą́ąwe\\
	wį'\#ti kųu'=haa E=rį wąąpehe=krE wį'\#ti ~ ~ ~ ~ ~ ~ \textbf{ko}-taa=krE=s ąąwe\\
	\textnormal{stone}\#\textnormal{dwell} \textnormal{be.entire}=sim \textnormal{hear}=ss \textnormal{mourn}=3pl \textnormal{stone}\#\textnormal{dwell} ~ ~ ~ ~ ~ ~  \textbf{rel}-loc=3pl=def \textnormal{all}\\
	\glt `the entire village heard it and they mourned, all the ones who lived in the village' \citep[166]{hollow1973b}
	
	\item \glll hirée róo nútaa óxkąhe \textbf{ko}ráshitaa, ~ ~ ~ ~ ~ ~ ~ ~ kúhkeres, ríikihekto'sh\\
	hiree roo rų-taa o-xkąh=E \textbf{ko}-rashi=taa ~ ~ ~ ~ ~ ~ ~ ~ kuh=krE=s rV-i-kihE=kt=o'sh\\
	\textnormal{now} dem.mid 1a.pl-loc pv.irr-\textnormal{move}=sv \textbf{rel}-\textnormal{behind}=loc ~ ~ ~ ~ ~ ~ ~ ~  \textnormal{come.back.here}=3pl=def 1a.pl-pv.ins-\textnormal{wait}=pot=ind.m\\
	\glt 	`We are here now, [so] we will wait for the travelers who are behind, the ones coming back' \citep[194]{hollow1973b} 

	\item \glll miníike, \textbf{ko}xáwaawaheres, hirée kirí'sh\\
	wį-rįįk=E, \textbf{ko}-xwaa\#wa-hrE=s hiree kri=o'sh\\
	1poss-\textnormal{son}=sv \textbf{rel}-\textnormal{be.lost}\#1a-caus=def \textnormal{now} \textnormal{arrive.back.here}=ind.m\\
	\glt `my son, the one I lost, is now back' \citep[229]{hollow1973b}
	
	\item \glll kixéektek, mí'ti \textbf{ko}tkás ą́ąwe'na pó ~ ~ ~ ~ xtes wakirúutoomako'sh\\
	ki-xee=ktek wį'\#ti \textbf{ko}-tka=s ąąwe=rą po ~ ~ ~ ~ xtE=s wa-k-ruut=oowąk=o'sh\\
	mid=\textnormal{be.slow}=pot+ds \textnormal{stone}\#\textnormal{dwell} \textbf{rel}-\textnormal{reside}=def \textnormal{all}=top \textnormal{fish} ~ ~ ~ ~  \textnormal{be.big}=def unsp-inch-\textnormal{eat}=narr=ind.m\\
	\glt `when he stopped, the big fish ate all the ones who lived in the village' \citep[201]{hollow1973b}
	
	\item \glll hą́p téehą nutékto'sh, háki ~ ~ ~ ~ ~ \textbf{ko}pą́to'na nuhékereki\\
	hąp(E) teehą rų-te=kt=o'sh ha=ki ~ ~ ~ ~ ~  \textbf{ko}-pąt=o'=rą rų-hE=krE=ki\\
	\textnormal{day} \textnormal{be.long.distance} 1a.pl-\textnormal{stand}=pot=ind.m pro.v=cond ~ ~ ~ ~ ~  \textbf{rel}-\textnormal{show.up}=\textnormal{be}=top 1a.pl-\textnormal{see}=3pl=cond\\
	\glt `we will be there for a long time, so let the ones to come see us' \citep[206]{hollow1973b}
	

	
	\item \glll karóotiki \textbf{ko}máa'ąktaa ~ ~ ~ ~ ~ ~ ~ ~ ~ ~ ~ ~ ~ ~ ~ ~ ~ ~ ~ ~ hą́ąkeseena ų́ke ~ ~ ~ ~ ~ ~ ~ ~ ~ ~ ~ ~ ~ ~ ~ írusheroomako'sh\\
		ka-ooti=ki \textbf{ko}-wąą'ąk=taa ~ ~ ~ ~ ~ ~ ~ ~ ~ ~ ~ ~ ~ ~ ~ ~ ~ ~ ~ ~ hąąkE=s=ee=rą ųk=E ~ ~ ~ ~ ~ ~ ~ ~ ~ ~ ~ ~ ~ ~ ~ i-ru-shE=oowąk=o'sh\\
		prov=evid=cond \textbf{rel}-\textnormal{earth}=loc ~ ~ ~ ~ ~ ~ ~ ~ ~ ~ ~ ~ ~ ~ ~ ~ ~ ~ ~ ~ \textnormal{be.standing}.aux=def=dem.dist=top \textnormal{hand}=sv ~ ~ ~ ~ ~ ~ ~ ~ ~ ~ ~ ~ ~ ~ ~ pv.ins-ins.hand-\textnormal{hold}=narr=ind.m\\
		\glt `and then he got hold of the one on the ground's hand' \citep[273]{hollow1973b}
		
	\item\label{relg} \glll kotámiihs ~ ~ ~ ~ ~ ~ ~ ~ ~ ~ ~ ~ ~ ~ ~ ~ ~ ~ ~ ~ ~ ~ ~ ~ \textbf{kí}ihkarahseena ``hiré ~ ~ ~ ~ ~ ~ ~ ~ ~ ~ ptamíihe, wasíi warého'xere're, káni ~ ~ ~ téehąki ~ ~ ~ ~  ówakiri'eshka're,'' éeheka'ehe\\
	ko-ta-wįįh=s ~ ~ ~ ~ ~ ~ ~ ~ ~ ~ ~ ~ ~ ~ ~ ~ ~ ~ ~ ~ ~ ~ ~ ~ \textbf{kV}-i-k-krah=s=ee=rą hire ~ ~ ~ ~ ~ ~ ~ ~ ~ ~ p-ta-wįįh=E wa-sii wa-reh=o'xre=o're ka=rį ~ ~ ~ teehą=ki ~ ~ ~ ~ o-wa-kri-eshka=o're ee-he=ka'ehe\\
	3poss.pers-al-\textnormal{woman}=def ~ ~ ~ ~ ~ ~ ~ ~ ~ ~ ~ ~ ~ ~ ~ ~ ~ ~ ~ ~ ~ ~ ~ ~ \textbf{rel}-pv.ins-mid-\textnormal{be.afraid}=def=dem.dist=top \textnormal{now} ~ ~ ~ ~ ~ ~ ~ ~ ~ ~ 1poss-al-\textnormal{woman}=sv unsp-\textnormal{travel} 1a-\textnormal{think}=dub=ind.f prov=ss ~ ~ ~ \textnormal{be.long.distance}=cond ~ ~ ~ ~  pv.irr-1a-\textnormal{arrive.back.here}-smlt=ind.f pv-\textnormal{say}=quot\\
	\glt 	`He told the sister he was afraid of, ``now, my sister, I am thinking of traveling and I will come back after a long time''' \citep[281]{hollow1973b}
	
	\item\label{relh} \glll máa'ąk íwaxarats \textbf{ko}'ų́ųte hík ~ ~ ~  roką́ąkaxihs máa'ųst séharani réehoote, ~ ~ ~ ~ ~ ~ ~ se'ésh náhka'eheroo\\
		wąą'ąk i-wa-xrat=s \textbf{ko}-ųųt=E hi=ak ~ ~ ~  rokąąka\#xih=s waa-ųst se\#hrE=rį rEEh=ootE ~ ~ ~ ~ ~ ~ ~  se-esh rąk=ka'ehe=oo\\
		\textnormal{land} pv.ins-unsp-\textnormal{hold.up}=def \textbf{rel}-\textnormal{be.first}=sv \textnormal{arrive.there}=ds ~ ~ ~ \textnormal{old.woman}\#\textnormal{old}=def nom-\textnormal{old} \textnormal{be.red}\#caus=ss \textnormal{go.there}=evid ~ ~ ~ ~ ~ ~ ~   \textnormal{be.red}-aprx pos.sit=quot=dem.mid\\
		\glt `the old lady, the one who got to the land that holds it up first, made it red and went now that she made it all red, it is said' \citep[123]{hollow1973a}
	
	\end{xlist}

\end{exe}

There is an optional allophone /kV-/ before vowel-initial stems containing preverbs, as seen in (\ref{relg}). However, the most common realization is as \textit{ko-}, even before vowel-initial stems, which we see in (\ref{relh}). The /kV-/ variant is more common in fast speech. The relativizer \textit{ko-}, seen below in (\ref{inanimateko}), is most commonly employed when referring to animate subjects, but inanimate subjects are also possible.

\newpage
\begin{exe}

\item\label{inanimateko} Inanimate referents for \textit{ko-}

	\begin{xlist}
	
	\item \glll ímashut ko'áaki\\
	i-wąshut ko-aaki\\
	pv.ins-\textnormal{clothe} rel-\textnormal{be.above}\\
	\glt `overcoat' (lit. `clothing that is on top')
	
	\item \glll maná weréxe ko'ų́st, kotké, kokámix, koxtés kixų́ųh\\
	wrą wrex=E ko-ųst ko-tkE ko-kawįx ko-xtE=s kixųųh\\
	\textnormal{wood} \textnormal{kettle}=sv rel-\textnormal{be.old} rel-\textnormal{be.heavy} rel-\textnormal{be.round} rel-\textnormal{be.big}=def \textnormal{five}\\
	\glt `five big, round, heavy, old drums' \citep[21]{mixco1997a}
	
	\end{xlist}

\end{exe} 

The majority of instances where \textit{ko}- appears in the corpus involve an active verb. All of the verbs in (\ref{inanimateko}) above, in contrast, are stative. The use of \textit{ko-} in these instances is less frequent when used adjectivally in casual speech. It is very common to omit \textit{ko}- in casual speech involving stative verbs functioning as adjectives, with speakers producing one token with \textit{ko}- and then immediately repeating themselves and dropping the \textit{ko}- with no effect on its intended reading.\footnote{During my own fieldwork in Hidatsa, I discovered that \textit{agu-}, the relativizer in that language that is a cognate, has an identical distribution as Mandan \textit{ko}-. All Hidatsa speakers I worked with would use the \textit{agu-} when using a stative verb adjectivally in one sentence, and then when asked to repeat it, the \textit{agu-} would often be dropped. However, when asked which way was the way they intended, consultants consistently said that the \textit{agu-} was mandatory. Mandan speakers have said the same thing about the presence of \textit{ko-}. Given the fact that the Mandan and Hidatsa peoples have been living in close proximity for so many centuries and that virtually all speakers of Mandan over the past hundred years have also been speakers of Hidatsa, it is unclear if this is a borrowing from Hidatsa into Mandan or if both languages inherited a similar propensity for treating relativized stative verbs from a common ancestor. Investigating the behavior of relativized stative verbs in Crow may offer some insight into this matter, as the split between the Hidatsa and Crow happened before the reservation period for the Mandan and Hidatsa.} However, the use of \textit{ko-} in some cases can mark the difference between a lexicalized noun-verb combination and a noun and a stative verb used adjectivally. In (\ref{adjko}) below, we can see a triplet where the semantics of the words \textit{máareksuk} `bird' and \textit{tóh} `blue' change depending on whether these words are compounded or the \textit{tóh} is relativized.

\newpage
\begin{exe}

\item\label{adjko} Adjectival use of stative verbs with and without \textit{ko-}

	\begin{xlist}
	
	\item\label{adjko1} \glll máareksuk~tóh \textnormal{or} máareksuktoh\\
	wąąreksuk\#toh\\
	\textnormal{bird}\#\textnormal{be.blue/green}\\
	\glt `bluejay'
	
	\item\label{adjko2} \glll máareksuk tóh\\
	wąąreksuk toh\\
	\textnormal{bird} \textnormal{be.blue/green}\\
	\glt `blue bird'
	
	\item\label{adjko3} \glll máareksuk kotóh\\
	wąąreksuk ko-toh\\
	\textnormal{bird} rel-\textnormal{be.blue/green}\\
	\glt `blue bird, a bird that is blue'
	
	\end{xlist}

\end{exe}

Orthographically, there is no distinction between \textit{máareksuk tóh} for `blue jay' or `blue bird', but prosodically, `blue bird' has a primary stress on both words, while there is a single primary stress in `blue jay', indicating that it is a single compound word rather than a DP with a stative verb adjunct. Either \textit{máareksuk tóh} or \textit{máareksuk kotóh} can be used to refer to a blue bird. Like Crow, the presence of the \textit{ko-} differs from other relativized constructions in that it acts to accentuate the predicate being relativized: \textit{máareksuk tóh} `blue bird' versus \textit{máareksuk kotóh} `a bird that is blue.'

One additional use of the relativizer is in constructions involving comparisons. The \textit{ko-} can be found on both comparatives and superlatives, as we see in (\ref{compko}) below.


\begin{exe}

\item\label{compko} Examples of comparative and superlative \textit{ko-}

	\begin{xlist}
	
	\item\label{compko1} \glll Minís koshíkeres ó'harani ~ ~ ~ ~ ~ ~ ~ ~ ~ ~ ~ ~ ~ ~ ~ ~ ~ ~ ~ ~ xką́herekereroomako'sh\\
	wrįs ko-shi=krE=s o'\#hrE=rį ~ ~ ~ ~ ~ ~ ~ ~ ~ ~ ~ ~ ~ ~ ~ ~ ~ ~ ~ ~ xkąh\#hrE=krE=oowąk=o'sh\\
	\textnormal{horse} rel-\textnormal{be.good}=3pl=def \textnormal{be}\#caus=ss ~ ~ ~ ~ ~ ~ ~ ~ ~ ~ ~ ~ ~ ~ ~ ~ ~ ~ ~ ~ \textnormal{move}\#caus=3pl=narr=ind.m\\
	\glt `they chased the better horses from there' \citep[84]{hollow1973b}
	
	\item\label{compko2} \glll taminís koshí térootiki, wapáweshini ~ ~ warúshaani wahúukto'sh\\
	ta-wrįs ko-shi te=ooti=ki wa-pa-wesh=rį ~ ~ wa-ru-shE=rį wa-huu=kt=o'sh\\
	3poss.al=\textnormal{horse} rel-\textnormal{be.good} \textnormal{stand}=evid=cond 1a-ins.push-\textnormal{cut}=ss ~ ~ 1a-ins.hand-\textnormal{hold}=ss 1a-\textnormal{come.here}=pot=ind.m\\
	\glt `when his best horse is there, I will cut it loose and come take it' \citep[259]{hollow1973b}
	
	\end{xlist}	

\end{exe}

\citet[22]{mixco1997a} notes that comparatives can also be periphrastically constructed when two nominals are being compared, with the first clause stating a quality and the second clause stating that this quality exceeds that of the second nominal. We can see examples of these periphrastic comparatives in (\ref{comparatives1}) below.

\begin{exe}

\item\label{comparatives1} Periphrastic comparatives

\begin{xlist}

\item\label{comparatives1a} \glll ą́'te, imáare hą́ska'sh; makáhų'ho'sh\\
	ą't=E iwąą=E hąska=o'sh wą-kahų'h=o'sh\\
	dem.anap=sv \textnormal{body}=sv \textnormal{be.long}=ind.m 1s-\textnormal{exceed}=ind.m\\
	\glt `he is taller than me [lit. his body is long; he exceeds me]' \citep[22]{mixco1997a}
	
\item\label{comparatives1b} \glll ą́'te, imáare kohą́ska'sh\\
	ą't=E iwąą=E ko-hąska=o'sh\\
	dem.anap=sv \textnormal{body}=sv rel-\textnormal{be.long}=ind.m\\
	\glt `he is the tallest [lit. his body is the one that is long]' \citep[22]{mixco1997a}
	
\end{xlist}

\end{exe}

The use of the relativizer to form comparatives is more common than the periphrastic construction seen in (\ref{comparatives1a}). No instances of this periphrastic construction appear in the corpus, and have only been documented in conversations with native speakers while eliciting comparatives.

\section{Suffix field}\label{SecSuffixField}

The suffix field in Mandan is extremely limited when compared to the prefix field. Many Siouan grammars alternate between describing post-verbal morphology as enclitics or suffixes, even when describing the same language. In Mandan, most post-verbal elements have traditionally been described as suffixes by \citet{hollow1970} and \citet{mixco1997a}. In Lakota, by contrast, \citet{ingham2003} and \citet{mirzayan2010} describe most post-verbal elements as enclitics. Many of these morphological items are cognates between these two languages, so the question arises as to whether Mandan truly has a large suffix field, or if the suffix field is more limited and there exists an enclitic field as well.\footnote{I elaborate more upon why I classify most postverbal elements in Mandan as enclitics in \citet{kasak2019}. For the purposes of this book, I shall maintain that assumption without further comment so as to not obfuscate the descriptive narrative of this grammar with morphological theory.}

I have glossed the data throughout this work as if most post-verbal elements are enclitics. The determining factor in deciding if an item is a suffix or an enclitic is whether hiatus between a verbal root and a post-verbal element is resolved with a glottal stop or a flap. We have previously seen the different behavior of word-internal hiatus resolution versus word-enclitic hiatus resultion in \sectref{epentheticprocesses}. A glottal stop occurs to prevent hiatus when prefixes or suffixes are added to a stem, as we saw in \ref{intrusiveglottal}. We see examples of how hiatus is treated in the suffix field in (\ref{roothiatus}) below.

\begin{exe}
\item\label{roothiatus} Hiatus between roots and affixes

	\begin{xlist}
	
	\item \glll ki'ų́ųpa\\
		ki-ųųpa\\
		suus-\textnormal{with}\\
		\glt `[something of her own] with her' \citep[219]{hollow1973a}
	
	\item \glll íki'aakit\\
	i-ki-aaki=t\\
	pv.dir-vert-\textnormal{be.above}=loc\\
	\glt `back upward' \citep[153]{hollow1973a}
	
	\item \glll ótu'eshkat\\
	o-tu-eshka=t\\
	pv.irr-\textnormal{be.some}-smlt=loc\\
	\glt `where there would be some like that' \citep[122]{hollow1973a}
	
	\item \glll róo'oshka\\
	roo-oshka\\
	dem.mid-emph\\
	\glt `right here' \citep[183]{hollow1973a}
		
	\end{xlist}

\end{exe}

In each of the examples in (\ref{roothiatus}) above, a glottal stop indicated by $\langle$'$\rangle$ appears at the juncture of a root and an affix. We can contrast this treatment of word-internal hiatus with hiatus found at the juncture of a word and an enclitic.

\newpage
\begin{exe}

\item\label{enclitichiatus} Hiatus between a root and an enclitic

	\begin{xlist}
	
	\item \glll tíroote'sh\\
	ti=ootE=o'sh\\
	\textnormal{arrive.here}=evid=ind.m\\
	\glt `she must have arrived here' \citep[127]{hollow1973a}
	
	\item \glll nátka xikxtéroomako'sh\\
	rąt=ka xik-xtE=oowąk=o'sh\\
	\textnormal{be.in.middle}=hab \textnormal{be.bad}-aug=narr=ind.m\\
	\glt `she felt really sorry for him' \citep[129]{hollow1973a}
	
	\item \glll warápiniira't\\
	wa-ra-prįį=ą't\\
	unsp-2a-\textnormal{wear.around.neck}=dem.anap\\
	\glt `that necklace of yours' \citep[58]{hollow1973a}
	
	\end{xlist}

\end{exe}

There are two different tactics for dealing with hiatus involving enclitics. If an enclitic beginning with a short vowel comes into contact with a stem ending in a short vowel, the enclitic elides that short vowel. However, if hiatus takes place and involves a long vowel, a [ɾ] is inserted to break up the two vowels and nothing is elided. We see this pattern clearly in (\ref{enclitichiatus}) above (see \sectref{epentheticprocesses} for more explicit argumentation for using phonological processes to identify morphological boundaries).

By looking at this criterion, we can assume that those post-verbal elements are not true suffixes, but are phrasal morphology (i.e., enclitics). An additional piece of evidence that these are true suffixes and not enclitics is the fact that they are wholly derivational in nature, versus enclitics which are generally inflectional in nature. Furthermore, the suffixes that appear in Mandan are templatic in nature in that they always appear in a prescribed order with respect to one another. The enclitics, on the other hand, have some degree of freedom in their ordering that depends on the intended semantic reading. %The fact that they are derivational means that these formatives are closer to the verb root, since the Lexicalist Hypothesis holds that derivation takes place in the lexicon, and then after exiting derivations, inflection can take place. 
With these assumptions in mind, the following suffixes in (\ref{mandansuffixlist}) exist in Mandan.

\begin{exe}
\item\label{mandansuffixlist} List of suffixes

\begin{tabular}{lll}
%\textit{-aaki}	&	collective 1 (\textsc{coll})
%	& (see \sectref{suffixcollective1})\\
\textit{-esh}	&	similitive 1 (\textsc{smlt})
	&	(see \sectref{suffixsimilitive1})\\
\textit{-eshka}	&	similitive 2 (\textsc{smlt})
	&	(see \sectref{suffixsimilitive2})\\
%\textit{-sha}	&	collective 2 (\textsc{coll})
%	& (see \sectref{suffixcollective2})\\
\textit{-sįh}	&	intensifier (\textsc{ints})
	&	(see \sectref{suffixintensifier})\\
\textit{-xte}	&	augmentative (\textsc{aug})
	& (see \sectref{augmentativesuffix})\\
\end{tabular} 

\end{exe}

This list of suffixes is massively reduced compared to the one given in \citet[15]{mixco1997a}. \citeauthor{mixco1997a}'s proposed suffix order appears in \tabref{suffixfieldmandan}.

\begin{table}
\caption{Suffix field in Mandan per \citet{mixco1997a}} \label{suffixfieldmandan}

\fittable{\scshape
    \begin{tabular}{lllllllllllll}
    \lsptoprule
    0&
    1&
    2&
    3&
    4&
    5&
    6&
    7&
    8&
    9&
    10&
    11&
    12\\\hline
    	root&
    	sv&
    	neg&
    	smlt&
    	att&
    	evid&
    	asp&
    	pl&
    	asp&
    	emph&
    	asp&
    	asp&
    	conj\\
    	~&~&~&~&~&~&~&~&~&~&~&~&frce\\
    	~&~&~&~&~&~&~&~&~&~&~&~&mod\\
		~&~&~&~&~&~&~&~&~&~&~&~&pst\\
    	~&~&~&~&~&~&~&~&~&~&~&~&quot
    	\\\hline
\lspbottomrule
    \end{tabular}}
\end{table}

The ordering in \figref{suffixfieldmandan} is proffered as a comprehensive ordering of suffixes in Mandan, but throughout \citeapos{mixco1997a} grammar, these items do not always appear in their designated suffix slot. Furthermore, there is a large number of post-verbal elements not accounted for in \figref{suffixfieldmandan}.

Under the definition of what is a suffix versus what is an enclitic in \citet{kasak2019}, the number of true suffixes in Mandan is quite low, as seen in (\ref{mandansuffixlist}). All suffixes are derivational in nature. Examples of all four suffixes appear in the subsections below.

\subsection{Augmentative suffix: -\textit{xte}}\label{augmentativesuffix}

The augmentative suffix \textit{-xte} is descended from the Proto-Siouan augmentative *-xtE$\sim$*-xti, which has numerous cognates throughout the language family. This augmentative suffix also exists in Mandan as a stative verb \textit{xté} `be big.' It is not clear whether Proto-Siouan also had an independent verb that became grammaticalized as an augmentative suffix or if Mandan innovated a separate verb from the augmentative by analyzing instances of it as serial verb constructions, and as such, we cannot concretely say if this dual purpose for /xtE/ in Mandan is an innovation or an archaism. However, one piece of evidence that Mandan did not innovate this dual usage can be seen in Missouri Valley Siouan. 

The Hidatsa word for `big' is \textit{ihdía}. Phonetic analysis shows that the $\langle$h$\rangle$ is really a lowered [x̞] due to the surrounding high vowels drawing the body of the tongue forward, making this word cognate with Mandan \textit{xté}.\footnote{We can make this judgment by observing that the formants have more energy in the higher bands rather than the diffused energy we see with [h] (\citeauthor{torres2013a} p.c.). This behavior means that the word is really [i.ˈx̞tiə̯].} There is no record of this verb also being used as an augmentative in modern Hidatsa, but Crow does have an augmentative suffix \textit{-shta} that is cognate with Mandan \textit{-xte}. I have argued that Crow and Hidatsa are the closest relatives of Mandan, so it suggests that the dual use of PSi *-xtE as an augmentative and a lexical verb stems from their common ancestor \citep{kasak2015}. We can see this behavior of \textit{-xte} in (\ref{Ch3ExAugmentativeXTE}) below.

\begin{exe}
\item\label{Ch3ExAugmentativeXTE} Examples of augmentative \textit{-xte}

	\begin{xlist}
	
% 	\item \glll wáaxik\textbf{xte}\\
% 	waa-xik-\textbf{xtE}\\
% 	nom-\textnormal{be.bad}-\textbf{aug}\\
% 	\glt `something really bad' \citep[46]{hollow1973a}
	
	\item \glll wóoruut shi\textbf{xté}'sh, ą́'t, ~ ~ ~ ~ ~ ~ ~ ~ ~ ~ ~ ~ ~ manápusheką't\\
	waa-o-ruut shi-\textbf{xtE}=o'sh ą't ~ ~ ~ ~ ~ ~ ~ ~ ~ ~ ~ ~ ~  wrą\#pushek=ą't\\
	nom-pv.irr-\textnormal{eat} \textnormal{be.good}-\textbf{aug}=ind.m dem.anap ~ ~ ~ ~ ~ ~ ~ ~ ~ ~ ~ ~ ~  \textnormal{tree}\#\textnormal{juneberry}=dem.anap\\
	\glt `Those are good eating, them, those juneberries' \citep[53]{hollow1973a}
	
	\item \glll ishák kohų́ųxiho'na, ké'kani ~ ~ ~ ~ ~ ~ ~ ~ ~  kikíiras\textbf{xte}'sh, Numá'kshikaraha\\
	ishak ko-hųų\#xih=o'=rą ke'\#ka'=rį ~ ~ ~ ~ ~ ~ ~ ~ ~   ki-kiiras-\textbf{xtE}=o'sh ruwą'k\#shi\#krah=ha\\
	3pro 3poss.al.pers-\textnormal{mother}\#\textnormal{be.old}=\textnormal{be}=top \textnormal{keep}\#\textnormal{have}=ss ~ ~ ~ ~ ~ ~ ~ ~ ~ mid-\textnormal{be.stingy.with}-\textbf{aug}=ind.m \textnormal{man}\#\textnormal{be.good}\#\textnormal{be.afraid.of}=sim\\
	\glt `them, those grandmothers of his, they kept him and sure did love Afraid-to-be-Chief' \citep[64]{hollow1973a}
	
	\item \glll níikasiiseena kóos ~ ~ ~ ~ ~ ~ ~ ~ ~ ~ ítee\textbf{xte}ka'eheero'sh\\
	rįįka\#sii=s=ee=rą koo=s ~ ~ ~ ~ ~ ~ ~ ~ ~ ~  i-tee-\textbf{xtE}=ka'ehee=o'sh\\
	\textnormal{offspring}\#\textnormal{be.yellow}=def=dem.dist=top \textnormal{squash}=def ~ ~ ~ ~ ~ ~ ~ ~ ~ ~ pv.ins-\textnormal{like}-\textbf{aug}=quot=ind.m\\
	\glt `that young calf there really liked the squash, it is said' \citep[112]{hollow1973a}
	
	\item \glll ``Náxihe, nuwáruute\textbf{xtaa}ni ~ ~ ~ ~ ~ ~ ~ ~ ~ ~ nuhúuro'sh,'' éeheroomako'sh.\\
	rą\#xih=E rų-waruutE-\textbf{xtE}=rį ~ ~ ~ ~ ~ ~ ~ ~ ~ ~  rų-huu=o'sh ee-he=oowąk=o'sh\\
	\textnormal{mother}.voc\#\textnormal{be.old}=sv 1a.pl-\textnormal{be.hungry}.1a.pl-\textbf{aug}=ss ~ ~ ~ ~ ~ ~ ~ ~ ~ ~ 1a.pl-\textnormal{come.here}=ind.m pv-\textnormal{say}=narr=ind.m\\
	\glt `He said, ``Grandmother, we came really hungry.{''}' \citep[266]{hollow1973a}
	
	\end{xlist}

\end{exe}

\largerpage
As we can see in the data in (\ref{Ch3ExAugmentativeXTE}) above, the underlying form of \textit{-xte} is identical to the lexical verb `be big' /xtE/, given the fact that enclitics that normally trigger ablaut do so. The augmentative is the only suffix that can be ablauted.

% \subsection{Collective suffixes: \textit{-aaki}, \textit{-sha}, \textit{-shka}}\label{suffixcollective}

% There are two competing collective suffixes in Mandan. Each of these suffixes is very restricted in where they can appear.

% \subsubsection{Collective 1: \textit{-aaki}}\label{suffixcollective1}

% The first collective suffix \textit{-aaki} is attested with a single stem: \textit{numá'k} `person, man.'

% \begin{exe}

% \item\label{coll1} Collective 1 examples

% 	\begin{xlist}
	
% 	\item \glll numá'k\textbf{aaki} máamikoomako'sh\\
% 	ruwą'k-\textbf{aaki} waa-wįk=oowąk=o'sh\\
% 	\textnormal{person}-\textbf{coll} \textnormal{some}-\textnormal{be.none}=narr=ind.m\\
% 	\glt `there were no people' \citep[178]{hollow1973a}
	
% 	\item \glll áakinuma'k\textbf{aaki}\\
% 	aaki\#ruwą'k-\textbf{aaki}\\
% 	\textnormal{be.above}\#\textnormal{person}-\textbf{coll}\\
% 	\glt `Native American(s)' \citep[220]{trechter2012b}\footnote{This word for `Native American' is a contraction of the term \textit{máa'ąk áaki numá'kaaki} `people on the land.' This term is cognate with the Hidatsa term for Native American \textit{(awa')áagaaruxbaaga} `people on the land', and is similar to terms for indigenous peoples found in nearby languages, e.g., Lakota \textit{ikčé wičáša} `ordinary people.'}
	
% 	\item \glll ómahą numá'k\textbf{aaki}\\
% 	owąhą ruwą'k-\textbf{aaki}\\
% 	\textnormal{Omaha} \textnormal{person}-\textbf{coll}\\
% 	\glt `Omaha tribe' \citep[431]{hollow1970}
		
% 	\end{xlist}

% \end{exe}

% This suffix \textit{-aaki} originates from the stative verb \textit{áaki} `be above.' This verb appears in compounds where it serves to intensify another stative verb, but this pattern does not seem to be productive in modern Mandan. Furthermore, \textit{áaki} serves as the initial element in all such compounds.

% \begin{exe}
% \item Compound with \textit{áaki}

% \glll	áakana'ro'sh\\
% 	aaki\#rą'=o'sh\\
% 	\textnormal{be.above}\#\textnormal{ache}=ind.m\\
% 	\glt `he is sick' \citep[168]{hollow1970}

% \end{exe}

% It is possible that the collective \textit{-aaki} could have been used metaphorically to describe a large number in the past. This suffix is not otherwise productive in modern Mandan.

% \subsubsection{Collective 2: \textit{-sha}}\label{suffixcollective2}

% This collective suffix comes from the Proto-Siouan collective *-sa. This suffix is restricted to numerals.

% \begin{exe}
% \item Examples of collective \textit{-sha}

% 	\begin{xlist}
	
% 	\item \glll núup\textbf{sha}\\
% 	rųųp-\textbf{sha}\\
% 	\textnormal{two}-\textbf{coll}\\
% 	\glt `both of them, two of them, twins' \citep[481]{hollow1970}
	
% 	\item \glll kixų́ųh\textbf{sha}\\
% 	kixųųh-\textbf{sha}\\
% 	\textnormal{five}-\textbf{coll}\\
% 	\glt `five of them' \citep[481]{hollow1970}
	
% 	\item \glll náamini\textbf{sha}\\
% 	raawrį-\textbf{sha}\\
% 	\textnormal{three}-\textbf{coll}\\
% 	\glt `three of them' \citep[481]{hollow1970}
	
% 	\item \glll nunáamini\textbf{sha} shí'sh\\
% 	rų-raawrį-\textbf{sha} shi=o'sh\\
% 	1a.pl-\textnormal{three}-\textbf{coll} \textnormal{be.good}=ind.m\\
% 	\glt `the three of us are good' \citep[481]{hollow1970}
	
% 	\end{xlist}

% \end{exe}

% This collective suffix often co-occurs with the ordinal preverb \textit{i-}.

% \begin{exe}
% \item Collective suffix with ordinal preverb

% 	\begin{xlist}
	
% 	\item \glll \textbf{í}toop\textbf{sha} íhaa'aakit ~ ~ ~ ~ ~ ~ ~ ~ ~ ~ ~ ~ ~ ~ keréehkereroomako'sh\\
% 	\textbf{i}-toop-\textbf{sha} i-haa\#aaki=t ~ ~ ~ ~ ~ ~ ~ ~ ~ ~ ~ ~ ~ ~  krEEh=krE=oowąk=o'sh\\
% 	\textbf{pv.ord}-\textnormal{four}-\textbf{coll} pv.dir-\textnormal{cloud}\#\textnormal{be.above}=dir ~ ~ ~ ~ ~ ~ ~ ~ ~ ~ ~ ~ ~ ~ \textnormal{go.back.there}=3pl=narr=ind.m\\
% 	\glt `all four of them returned to heaven' \citep[175]{hollow1973a}
	
% 	\item \glll \textbf{í}nuup\textbf{sha} tasúke túkerek\\
% 	\textbf{i}-rųųp-\textbf{sha} ta-suk=E tu=krE=ak\\
% 	\textbf{pv.ord}-\textnormal{two}-\textbf{coll} 3poss.al-\textnormal{child}=sv \textnormal{be.some}=3pl=ds\\
% 	\glt `both of them had children' \citep[111]{hollow1973a}
	
% 	\item \glll \textbf{í}nuup\textbf{sha} ráse ísekwahere'sh\\
% 	\textbf{i}-rųųp-\textbf{sha} ras=E i-sek\#wa-hrE=o'sh\\
% 	\textbf{pv.ord}-\textnormal{two}-\textbf{coll} \textnormal{name}=sv pv.ins-\textnormal{make}\#1a-caus=ind.m\\
% 	\glt `I gave both of them their names' \citep[64]{hollow1973a} 
	
% 	\end{xlist}

% \end{exe}

% The collective \textit{-sha} is sometimes accompanied by the suffix \textit{-shka}. This suffix serves to emphasize the collective reading. This suffix comes from the Proto-Siouan suffix *-ska, which historically a similitive marker. Traces of this *-ska can be seen in other productive suffixes, like the emphatic \textit{-oshka} or the similitive suffix \textit{-eshka}, or on the interrogative word \textit{tashká} `how', where \textit{tá} is `what'. The collective suffix can appear with this emphatic \textit{-shka}

% \begin{exe}

% \item Examples of intensified collective suffixes

% 	\begin{xlist}
	
% 	\item \glll \textbf{í}nuup\textbf{shashka}na hų́pe ké'ka'rak ~ ~ ~ ~ ~ ~ ~ ~ kú'kerek\\
% 	\textbf{i}-rųųp-\textbf{sha-shka}=rą hųp=E ke'\#ka'=ak ~ ~ ~ ~ ~ ~ ~ ~  ku'=krE=ak\\
% 	pv.ord-\textnormal{two}-\textbf{coll-ints.coll}=top \textnormal{shoe}=sv \textnormal{keep}\#\textnormal{have}=ds ~ ~ ~ ~ ~ ~ ~ ~ \textnormal{give}=3pl=ds\\
% 	\glt `both of them kept shoes for him' \citep[109]{hollow1973a}
	
% 	\item \glll \textbf{í}toop\textbf{shashka} Máarepaaxu í'ų'taa ~  minípashų'ni réehkereroomako'sh\\
% 	\textbf{i}-toop-\textbf{sha-shka} wąą=E\#paaxu i-ų'=taa ~  wrį\#pa-shų'=rį rEEh=krE=oowąk=o'sh\\
% 	\textbf{pv.ord}-\textnormal{four}-\textbf{coll-ints.coll} \textnormal{eagle}=sv\#\textnormal{nose} pv.dir-\textnormal{be.closer}=loc ~  \textnormal{water}\#ins.push-\textnormal{thresh.with.feet}=ss \textnormal{go.there}=3pl=narr=ind.m\\
% 	\glt `the four of them were swimming toward Eagle Nose' \citep[295]{hollow1973b}
	
% 	\item \glll	óo ó'harani numá'kaaki hų́keres sheréekini ~  ~ tóop\textbf{shashka} kaháshkereroomako'sh\\
% 	oo o'\#hrE=rį ruwą'k-aaki hų=krE=s shreek=rį ~ ~   toop-\textbf{sha-shka} ka-hash=krE=oowąk=o'sh\\
% 	dem.mid \textnormal{be}\#caus=ss \textnormal{person}-coll \textnormal{many}=3pl=def \textnormal{war.whoop}=ss ~ ~  \textnormal{four}-\textbf{coll}-\textbf{ints.coll} ins.frce-\textnormal{be.disintegrated}=3pl=narr=ind.m\\
% 	\glt `from there, the whole lot of people war whooped and slaughtered all four of them' \citep[255]{hollow1973b}
	
% 	\end{xlist}

% \end{exe}

% The intensified collective /-sha-shka/ can appear with or without the ordinal preverb \textit{i-}. The presence of the preverb does not seem to alter the reading, though it seems to serve to intensify the collective meaning.

\subsection{Intensifier suffix: \textit{-sįh}}\label{suffixintensifier}

The intensifier \textit{-sįh} can appear on either verbal or nominal elements. There is also a lexical verb \textit{s\'{ı̨}h} `be strong', which takes active pronominals despite having semantics that are more in line with stative verbs, e.g., \textit{was\'{ı̨}ho'sh} `I am strong.' The use of active pronominals may stem from the fact that there has been a semantic shift in the meaning of this verb in Mandan, given that its cognates in Omaha-Ponca and Osage are \textit{uáⁿsi} and \textit{áowisi}, respectively, which both mean `jump'. If the Omaha-Ponca and Osage cognates are closer to the Proto-Siouan use of this term, then it would make more sense why this verb takes active pronominals, as jumping can be something that involves agency. In addition to this suffixal and free contrast for \textit{s\'{ı̨}h} and \textit{-sįh}, there is also a clause-final enclitic \textit{=sįh} that can co-occur with the narrative evidential \textit{=oomak}, which will be discussed further in \sectref{SecPhrasalMorphology}. We can see instances of this suffix in (\ref{Ch3ExSIH}) below.

\begin{exe}
\item\label{Ch3ExSIH} Examples of the intensifier suffix \textit{-sįh}

	\begin{xlist}
	
	\item \glll wáaka'\textbf{sįh}o'sh\\
	waa-ka'-\textbf{sįh}=o'sh\\
	\textnormal{something}-\textnormal{have}-\textbf{ints}=ind.m\\
	\glt `he begs for something' \citep[210]{hollow1970}
	
	\item \glll shehék\textbf{sįh}\\
	shehek-\textbf{sįh}\\
	\textnormal{coyote}-\textbf{ints}\\
	\glt `a liar' \citep[210]{hollow1970}
	
	\item \glll wáashereek xtena áakit e\textbf{s\'{ı̨}h}oomako'sh\\
	waa-shreek xtE=rą aaki=t e-\textbf{sįh}=oowąk=o'sh\\
	nom-\textnormal{noise} \textnormal{be.big}=top \textnormal{be.above}=loc \textnormal{hear}-\textbf{ints}=narr=ind.m\\
	\glt `he really heard a big noise above him' \citep[281]{hollow1973b}
	
		\item \glll {máa}na ókų'hkerektiki, ~ ~ ~ ~ ~ ~ ~ ~ warúkah\textbf{sįh}xteka'eheero'sh\\
	waa=rą o-k'-ųh=krE=kti=ki ~ ~ ~ ~ ~ ~ ~ ~ wa-rukah-\textbf{sįh}-xtE=ka'ehEE=o'sh\\
	\textnormal{someone}=top pv.loc-3poss.pers-\textnormal{wife}=3pl=pot=cond ~ ~ ~ ~ ~ ~ ~ ~ unsp-\textnormal{refuse}-\textbf{ints}-aug=quot=ind.m\\
	\glt `whenever someone would try to marry her, she always strongly refused, it is said' \citep[101]{hollow1973a} 
	
	\end{xlist}

\end{exe}

The intensifier is able to co-occur with the augmentative for further emphasis. While the intensifier is much rarer in the corpus when compared to the augmentative, both are productive in modern Mandan. When both suffixes do co-occur, the intensifier will always precede the augmentative. This suffix is homophonous with the intensive indicative complementizer =\textit{sįh}, which is discussed later on in \sectref{intensivecomplementizer}.

\subsection{Similitive suffixes: \textit{-esh} and \textit{-eshka}}\label{suffixsimilitive1}

There are two similitive suffixes in Mandan: \textit{-esh} and \textit{-eshka}. Both appear to be related to the Proto-Siouan similitives *-se and *-ska. Given the fact that there are two similitives in Mandan, it is possible that the Proto-Siouan *-ska is actually decomposable into *-s-ka, where *-ka is a determiner or a distal locative. The *-s looks to be a determiner that is cognate with the indicative enclitic =\textit{c} in Hidatsa, and the male addressee indicative enclitic \textit{=o'\textbf{sh}} in Mandan.\footnote{Both Mandan and Hidatsa appropriated the PSi *-s element as a declarative marker, but Pre-Modern Mandan appears to have undergone a stage where non-imperative clauses required the copula \textit{ó'} `be' along with certain determiners, demonstratives, or locatives that became re-analyzed as illocutionary force markers and allocutive morphology. In modern Mandan, these historically distinct pieces of morphology are no longer individually discrete, but form whole morphological items, e.g., \textit{=o'sh} marking male-addressee indicative utterances or \textit{o're} marking female-addressee indicative utterances. We can also see hints of this *-s on certain adverbial morphology like \textit{résh} `this way, like this' (PSi *re `this' + *-s) or \textit{ų́'sh} `thus, like that, so' (PSi *ʔųų `be.\textsc{pl} + *-s).} 

\subsubsection{Similitive 1: \textit{-esh}}

In order to grant a similitive reading to a word, the suffix \textit{-esh} can be added to a root. The roots that take this suffix are mostly stative verbs or nouns being used as stative verbs. We can see examples of -\textit{esh} in (\ref{smlt1}) below.

\begin{exe}
\item\label{smlt1} Examples of the similitive suffix \textit{-esh}

	\begin{xlist}
	
	\item \glll íirapsi'\textbf{esh} máakeroomako'sh\\
	ii\#ra-psi-\textbf{esh} wąąkE=oowąk=o'sh\\
	\textnormal{blood}\#mut-\textnormal{be.black}-\textbf{smlt} \textnormal{be.lying}.aux=narr=ind.m\\
	\glt `black blood was there' \citep[132]{hollow1973a}
	
	\item \glll íku'\textbf{esh}a ráahini éeta!\\
	i-ku'-\textbf{esh}=E rEEh=rį ee=ta\\
	pv.dir-\textnormal{be.further.away}-\textbf{smlt}=sv \textnormal{go.there}=ss \textnormal{defecate}=imp.m\\
	\glt `Go a little farther away and relieve yourself' \citep[11]{hollow1973b}
	
	\end{xlist}

\end{exe}

This particular monosyllabic realization of the similitive is very rare when compared to \textit{-eshka}. The presence of the habitual \textit{=ka} suggests that there may be a slight semantic difference between these two kinds of similitive suffixes, but that difference may have been historical, as \textit{-eshka} is the predominant similitive marker. 

\subsubsection{Similitive 2: \textit{-eshka}}\label{suffixsimilitive2}

The overwhelming majority of cases where the similitive appears in Mandan involves the suffix \textit{-eshka}. Words bearing this suffix can have a meaning of `X-like', where X is the stative verb or noun in question, as we can see in (\ref{smlt2}) below.

\begin{exe}
\item\label{smlt2} Examples of the similitive suffix \textit{-eshka}

	\begin{xlist}
	
	\item \glll ráahąmi íku'\textbf{eshka} réehak...\\
	rEEh=awį i-ku'-\textbf{eshka} rEEh=ak\\
	\textnormal{go.there}=cont dir-\textnormal{be.further.away}=\textbf{smlt} \textnormal{go.there}=ds\\
	\glt `Going there, when he had gone a little ways...' \citep[68]{hollow1973b}
	
	\item \glll íhek\textbf{eshka}kerek\\
	i-hek-\textbf{eshka}=krE=ak\\
	pv.ins-\textnormal{know}-\textbf{smlt}=ds\\
	\glt `they kind of knew it' \citep[152]{hollow1973a}

	\item \glll rúute nát\textbf{eshka} rusháani...\\
	ruut=E rąt-\textbf{eshka} ru-shE=rį\\
	\textnormal{rib}=sv \textnormal{be.in.middle}-\textbf{smlt} ins.hand-\textnormal{grasp}=ss\\
	\glt `he took her ribs on both sides...' \citep[176]{hollow1973a}
	
	\end{xlist}

\end{exe}

\subsection{Summary of suffixes in Mandan}

Contrary to what has been proposed in previous analyses in Mandan, very little of the post-verbal morphological material can actually be classified as true suffixes. Most of the formatives following a stem are really enclitics. In \sectref{boundarydependent} in the previous chapter, I have argued that morpho-phonological processes like epenthesis are sensitive to whether there is a word-boundary or not at the locus of hiatus. Namely, [ʔ] is used to prevent hiatus at affix boundaries (i.e., word-internal boundaries caused by affixation), while [ɾ] resolves hiatus between a stem and a morphological item outside the scope of a word boundary (i.e., word-external boundaries caused by encliticization). Suffixes in Mandan will always incur [ʔ]-epenthesis to resolve hiatus. Any formative that does not trigger [ʔ]-epenthesis, therefore, must not be a suffix.

\section{Enclitic field}\label{SecPhrasalMorphology}

The purpose of this section is to describe the enclitic morphology present in Mandan, and then discuss the ways in which it differs from true affixation. Namely, enclitics may appear in different orders with respect to the stem in order to reflect a difference in underlying semantic scope. My criteria for judging whether an item is morphologically part of the word (i.e., is an affix) or not (i.e., is an enclitic) are based on phonological tests laid out in \sectref{epentheticprocesses} and \sectref{primarystress}. Namely, we can tell whether a morphological item is an affix or a clitic by what kind of hiatus repair mechanism it avails itself of (i.e., [ʔ]-epenthesis for affixes and [ɾ]-epenthesis for clitics), as well as blocking conditions for primary stress (i.e., suffixes can take primary stress because they appear within the domain of the word, while enclitics cannot take primary stress, because footing cannot cross a phrasal boundary). Thus, when we see primary stress after a verb root, we can tell that that element is a suffix, but when a well-formed iambic foot does not occur in favor of a deficient iamb, we can tell that footing did not occur due to it being blocked by a phrasal boundary.

In \sectref{SecSuffixField} above, I outline the few instances of genuine suffixation that exist in Mandan. These suffixes appear frequently throughout the corpus, but are still relatively infrequent when compared to the other post-verbal material that has been surveyed. The question of what the difference is between these formatives and true suffixes revolves around the kinds of morpho-phonological behaviors we observe. It is robustly the case that Mandan prefers left-aligned weight-sensitive iambs for primary stress placement and likewise that Mandan does not permit two vowels in adjacent syllables to come into contact. A large number of words appear to flout these two major characteristics of Mandan morpho-phonology. Two examples of words with unexpected stress appear in (\ref{badstresschap3}) below.

\begin{exe}
\item\label{badstresschap3} Examples of unexpected non-iambic primary stress

	\begin{xlist}
	
	\item \glll múp $\sim$ múpe *mupé\\
	wųp ~ wųp=E\\
	\textnormal{corn.mush} ~ \textnormal{corn.mush}=sv\\
	\glt `corn mush' \citep[274]{hollow1970}

	\item \glll ínaare *ináare\\
	i-rąą=E\\
	pv.ins-\textnormal{rattle}=sv\\
	\glt `a rattle' \citep[93]{hollow1970}
	
	\end{xlist}

\end{exe}

My argument in \sectref{stress} about why stress assignment in Mandan is not as irregular as it might seem at first glance stems from the fact that we can analyze cases like that as being composed of non-simplex words. That is to say, these words are composites in the sense of \citet[310]{anderson1992}, where a single morphological word contains internal word boundaries. Primary stress assignment is unable to cross a word boundary, which explains why neither of the words above demonstrate second-syllable stress, even though a language with left-aligned iambic footing should be preferred as such. The words in (\ref{badstresschap3}) above really have the underlying morphological structures seen in (\ref{badstresschap3b}) below.

\begin{exe}
\item\label{badstresschap3b} Underlying structures for (\ref{badstresschap3})

	\begin{xlist}
	
	\item\label{badstresschap3b1}  {[}\textit{múp}] $\sim$ [\textit{múp}]=\textit{e}, but *[\textit{mup}]=\textit{é}


	\item\label{badstresschap3b2}{[}\textit{í}[\textit{naa}]]=\textit{re}, but *[\textit{i}[\textit{náa}]]=\textit{re}
	
	\end{xlist}

\end{exe}

The stress in (\ref{badstresschap3b1}) cannot fall on the final syllable because that would cause footing across a right word boundary onto an enclitic. Similarly, the primary stress in (\ref{badstresschap3b2}) is trapped on an ill-formed iamb because footing cannot cross a left word boundary. Thus, whether it is a word-internal word boundary due to a word being a composite or a word-external word boundary due to the presence of enclitic materials, a word boundary will always act as an impediment for foot formation.

The list of enclitics found here comes from those formatives in the corpus that do not meet the conditions for suffixhood. That is, these morphological items either resist primary stress assignment when in second-syllable positions or trigger [ɾ]-epenthesis to repair hiatus (cf. \sectref{epentheticprocesses}). These enclitics fall into six different categories: aspectuals, evidentials, number markers, negation, modals, and complementizers.

\subsection{Aspectual enclitics}

Siouan languages are typically described as not marking tense on verbs, as we can see in languages like Lakota \citep[27]{ullrichblackbear2016} or Crow \citep[7]{graczyk2007}. The timeframe in discussion is usually left up to the context, expressed periphrastically through auxiliaries, or clarified through the use of temporal adverbials or other deictic elements \citep{rankin1977}.

Previous scholars, however, have described Mandan as a language featuring tense morphology \citep{kennard1936,hollow1970,mixco1997a}. Three endings that are often described as marking tense are \textit{=s}, which is generally called a preterite or past tense marker; \textit{=oomak}, which is traditionally referred to as a narrative past marker; and \textit{=kt}, which previous works often describe as a future tense or potential marker. \citet[454]{hollow1970} even notes that these three formatives are in complementary distribution. \citet{hollow1970} also describes the preverb \textit{o-} as a future tense prefix, bringing the number of formatives that he ascribes as encoding tense features to four. If we follow \citeapos{hollow1970} analysis, these four tenses consist of two different past tenses and two different future tenses. We can see examples of these so-called tense markers below in (\ref{hollowtense}), taken directly from the morphological analysis section of his grammar.

\begin{exe}
\item\label{hollowtense} Tense marking per \citet{hollow1970}

	\begin{xlist}
	
	\item Past tense
	
	\glll rupshásh\textbf{s}o'sh\\
		ru-pshash=\textbf{s}=o'sh\\
		ins.hand-\textnormal{be.hard}=\textbf{def}=ind.m\\
	\glt	`he made it hard' \citep[479]{hollow1970}

	\item Narrative past tense
	
	\glll	rupshásh\textbf{oomak}o'sh\\
		ru-pshash=\textbf{oowąk}=o'sh\\
		ins.hand-\textnormal{be.hard}=\textbf{narr}=ind.m\\
	\glt	`he hardened it' \citep[475]{hollow1970}

	\item Future tense 1
	
	\glll	\textbf{ó}manakikųųte'sha?\\
		\textbf{o}-w-rą-kikųųtE=o'sha\\
		\textbf{pv.irr}-1s-2a-\textnormal{help}=int.m\\
	\glt	`will you help me?' \citep[456]{hollow1970}

	\item Future tense 2

	\glll áawakereehske\textbf{kt}o'sh\\
		aa-wa-k-rEEh=ske=\textbf{kt}=o'sh\\
		pv.tr-1a-vert-\textnormal{go.there}=itr=\textbf{pot}=ind.m\\
	\glt	`I might take it back again' \citep[480]{hollow1970}
	
	\end{xlist}

\end{exe}

\citet[455]{hollow1970} remarks that \textit{o-} is also a ``true future'' marker if the likelihood of an event happening is certain, though if the likelihood is less certain, \textit{o-} may be used in tandem with the potential marker \textit{=kt}. The \textit{o-} may also be omitted altogether when \textit{=kt} is present. We can see an example of this co-occurrence in (\ref{lotsoffuture}) below.

\begin{exe}

\item\label{lotsoffuture} Multiple ``future'' marking

	\glll óo t\'{ı̨}'xtike \textbf{ó}kasaa\textbf{kt}o'sh\\
		oo tV-į'-xtike \textbf{o}-ka-saa=\textbf{kt}=o'sh\\
		dem.mid al-pv.rflx-\textnormal{quiver} \textbf{pv.irr}-ins.frce-\textnormal{hang.up}=\textbf{pot}=ind.m\\
	\glt	`he will hang his arrow sack there' \citep[116]{hollow1973a}

\end{exe}

This complementary distribution of tense markers as described by \citet{hollow1970} makes sense if the three endings mentioned above are true tense markers, but as the data in (\ref{multipletense}) below show, these morphological items are able to co-occur. 

\begin{exe}
\item\label{multipletense} Conflicting tense marking

	\begin{xlist}
	
	\item \glll ómaake kotewét ó'ki, ~ ~ ~ ~ ~ ~ ~ ~ ~ ~ ~ ~ ~ ~ ~  kų́'hą't pkahų́\textbf{st}o'sh\\
	o-wąąk=E ko-t-we=t o'=ki ~ ~ ~ ~ ~ ~ ~ ~ ~ ~ ~ ~ ~ ~ ~   k'-ųh=ą't k-pa-hų=\textbf{s}=\textbf{t}=o'sh\\
	pv.irr-pos.lie=sv rel-wh-indf=loc \textnormal{be}=cond ~ ~ ~ ~ ~ ~ ~ ~ ~ ~ ~ ~ ~ ~ ~ 3poss.pers-\textnormal{wife}=dem.anap mid-ins.push-\textnormal{be.near}=\textbf{def}=\textbf{pot}=ind.m\\
	\glt `if it is somewhere by her bed, his wife will be by her' \citep[117]{hollow1973a}

	\item \glll íminikawoxsha\textbf{st}o'sh\\
	i-w-rį-ka-woxsha=\textbf{s}=\textbf{t}=o'sh\\
	pv.ins-1a-2s-ins.frce-\textnormal{take.good.care.of}=\textbf{def}=\textbf{pot}=ind.m\\
	\glt `I will take good care of you' \citep[14]{trechter2012b}
	
	\end{xlist}

\end{exe}

The marker \textit{=s} is analyzed as a preterite marker by \citet[478]{hollow1970}. If this formative truly did mark tense and were in complementary distribution with \textit{=oomak} and \textit{=kt}, then these constructions should be ungrammatical, as well have issues with their semantics, given the fact that there is simultaneous marking of both so-called past and future. Their presence in the corpus raises the question of how well previous analyses of these markers fit the data. Contemporary discussion of grammatical categories in Siouan languages centers around a language family that robustly marks aspect on verbs and relegates tense marking to adverbial or deictic function words, or tense is simply inferred by context \citep[105]{parksrankin2001}. While previous works on Mandan describe the languages as having tense marking, a survey of Siouan literature strongly suggests a uniform tendency towards overtly marking aspect rather than tense throughout the language family. Evidence for \citeapos{parksrankin2001} claim can be found in Biloxi \citep[2]{torres2010}, Catawba \citep[2]{rudes2007a}, Crow \citep[305]{graczyk2007}, Kanza \citep[vii]{cumberlandrankin2012}, and Lakota \citep[28]{ingham2003}.

Going off the data in the corpus, along with looking broadly at the Siouan language family in general, I argue that Mandan has no morphological manifestation of tense on its verbs. Rather, postverbal elements that have previously been assumed to relate to tense are actually evidential or mood markers that contextualize a speaker's knowledge of the events being described or attitude towards the likelihood of an event. Below in (\ref{aspectualenclitics}) is a list of all enclitics that mark aspect in Mandan that appear in the corpus.


\begin{exe}
\item\label{aspectualenclitics} List of aspectual enclitics

\begin{xlist}

\item /=aahka/ retrospective (\textsc{rtro})
\item /=awį/ continuative 1 (\textsc{cont})
\item /=haa/ simultaneous (\textsc{sim})
\item /=ka/ habitual (\textsc{hab})
\item /=rąątE/ prospective (\textsc{prsp})
\item /=rįįtE/ celerative (\textsc{cel})
\item /=ske/ iterative (\textsc{iter})
\item /=$\varnothing$/ continuative 2 (\textsc{cont})

\end{xlist}

\end{exe}

The information that follows highlights the uses of these aspectual enclitics in the corpus, as well as points to cognates in other Siouan languages where establishing cognacy is possible. Relevant morphology appears in bold in the data below.

\subsubsection{Retrospective aspectual enclitic: =\textit{aahka}}\label{Sec:Retrospective}

\citet[23]{kennard1936} first describes the element =\textit{aahka} as having multiple uses in Mandan. It may express ability to do an action or it can convey that an action has happened at a specific point, often translated as `just then' or `at that moment.' \citet{hollow1970} does not describe it, but =\textit{aahka} does appear in his transcribed texts. \citet[54]{mixco1997a} likewise notes that this formative is polysemous. This enclitic seems to be either a borrowing or a cognate with Hidatsa \textit{áagaa}/\textit{áahgaa} `be on top of.' It is rare for Mandan to have a simplex formative with a /hC/ cluster, whereas Hidatsa and Crow are rich with words with /hC/ clusters, which suggests that =\textit{aahka} is a borrowing. Another possible scenario is that this enclitic comes from a combination of the postposition \textit{áaki} `be on top' and the habitual =\textit{ka}. The loss of the /i/ in \textit{áaki} would create an underlying /kk/ cluster, which would become [hk] on the surface: Pre-Mandan *aaki-ka > contemporary Mandan \textit{aahka}.

This enclitic is very rare in the corpus when used to denote retrospective aspect, i.e., when describing an action that has just happened. We can see the behavior of this enclitic in the data in (\ref{retrospectiveaspectualenclitic}) below.

\begin{exe}

\item\label{retrospectiveaspectualenclitic} Examples of the retrospective aspectual enclitic =\textit{aahka}

	\begin{xlist}

 	\item\label{retrospectiveaspectualenclitic1}
	\glll róotkiki éreh\textbf{aahka}, í'aakit réehtiki órootki xíkąmi, áakitaa kaní', ráate...\\
	rootki=ki e-reh=\textbf{aahka} i-aaki=t rEEh=kt(i)=ki o-rootki xik=awį aaki=taa ka-rį' raat=E\\
	\textnormal{hit}=cond pv-\textnormal{want}=\textbf{rtro} pv.dir-\textnormal{be.above}=loc \textnormal{go.there}=pot=cond pv.loc-\textnormal{hit} \textnormal{be.bad}=cont \textnormal{be.above}=loc ins.frce-\textnormal{climb} \textnormal{be.high.up}=sv\\
	\glt `just when she wanted to hit him, he went upward where she could not hit him, she climbed up, way up...' \citep[99]{hollow1973a}
	
	\item\label{retrospectiveaspectualenclitic2}
	\glll mí'maa ó'r\textbf{aahka}'sh, inák\\
	w'-iwąą o'=\textbf{aahka}=o'sh irąk\\
	1poss-\textnormal{body} \textnormal{be}=\textbf{rtro}=ind.m \textnormal{again}\\
	\glt `he is just about my size, too' \citep[142]{hollow1973a}
	




	\item\label{retrospectiveaspectualenclitic3}
	\glll éeheer\textbf{aahka}\\
	ee-hee=\textbf{aahka}\\
	pv-\textnormal{speak}=\textbf{rtro}\\
	\glt `just as he said it' \citep[23]{kennard1936}
	
	\item\label{retrospectiveaspectualenclitic4}
	\glll kirúsher\textbf{aahka}\\
	ki-ru-shE=\textbf{aahka}\\
	vert-ins.hand-\textnormal{hold}=\textbf{rtro}\\
	\glt `just as he took it back' \citep[23]{kennard1936}
	
	\item\label{retrospectiveaspectualenclitic5}
	\glll óo tís\textbf{aahka}k\\
	oo ti=s=\textbf{aahka}=ak\\
	dem.mid \textnormal{arrive.there}=def=\textbf{rtro}=ds\\
	\glt `having just arrived there' \citep[23]{kennard1936}

	\item\label{retrospectiveaspectualenclitic6}
	\glll \textbf{áahka} mí'ti xténa héroomako'sh\\
	\textbf{aahka} wį'\#ti xtE=rą he=oowąk=o'sh\\
	\textbf{rtro} \textnormal{stone}\#\textnormal{reside} \textnormal{be.big}=top \textnormal{see}=narr=ind.m\\
	\glt `he just saw a big village' \citep[105]{hollow1973b}

	\item\label{retrospectiveaspectualenclitic7}
	\glll réehkere \textbf{áahka}, hų́pe pí'hąmi, húurak, hų́pe ~ ~ ~ wará' kapiríhe shúupa \'{ı̨}'sąąpek, ~ ~ ~ ~ ~ Shíwará're ée ó'ka'eheeroo\\
	rEEh=krE \textbf{aahka} hųp(e) pi'h=awį huu=ak hųp(e) ~ ~ ~ wra' ka-prih=E shuupa į'-sąąpE=ak ~ ~ ~ ~ ~ shi\#wra'=E ee o'=ka'ehee=oo\\
	\textnormal{go.there}=3pl \textbf{rtro} \textnormal{shoe} \textnormal{smoke}=cont \textnormal{come.here}=ds \textnormal{shoe} ~ ~ ~ \textnormal{fire} ins.frce-\textnormal{spread.flat}=sv \textnormal{shin.bone} pv.rflx-\textnormal{around}=ds ~ ~ ~ ~ ~ \textnormal{foot}\#\textnormal{fire}=sv dem.dist \textnormal{be}=quot=dem.mid\\
	\glt `they just went, his shoes kept on smoking, he came and his shoes were flaming around his ankles, and it was Fire Shoe, it is said' \citep[146]{hollow1973a}
	
	\end{xlist}

\end{exe}

In the data in (\ref{retrospectiveaspectualenclitic}) above, we see two different manifestations of the retrospective aspectual: one as an enclitic and another as a free adverbial. In (\ref{retrospectiveaspectualenclitic2}) through (\ref{retrospectiveaspectualenclitic5}), we see =\textit{aahka} behave like a typical enclitic in that it triggers [ɾ]-epenthesis to prevent hiatus. In (\ref{retrospectiveaspectualenclitic6}), we see a full prosodic word appearing at the left edge of the clause, which is where we would expect an adverbial to appear that has scope over the whole proposition.

Given that Mandan is such an SOV language, the verb is typically the last thing to appear in a clause. This is not the case in (\ref{retrospectiveaspectualenclitic7}), where the free adverbial \textit{áahka} follows the verb it is modifying. We see a similar behavior in the iterative adverbial \textit{inák} `again, also' in (\ref{retrospectiveaspectualenclitic2}). In both cases, the adverbial appears dislocated to the right, since it is parenthetical information. When \textit{áahka} appears as a free adverbial in the corpus, it is most commonly found in a right-dislocated position similar to (\ref{retrospectiveaspectualenclitic7}). The fact that this element exists with both bound and unbound forms suggests that Mandan has been undergoing a morphological reanalysis where the unbound item has become most commonly associated with the verb it modifies and become subsumed in the enclitic field due to the frequency with which it appears postposed after the verb. It is likely that this same process is what has led to the large amount of enclitics in Mandan and other Siouan languages.

\subsubsection{Continuative aspectual enclitics: =\textit{ąmi} and =∅}

There are two different manifestations of the continuative aspect. One is an overt enclitic, =\textit{ąmi}, whereas the other involves ablauting the final stem vowel in a complementizer phrase. A description of each of these aspectual enclitics appears in \sectref{Ch3SubSubContinuative1} and \sectref{Ch3SubSubContinuative2} below.

\subsubsubsection{Continuative 1: =\textit{ąmi}}\label{Ch3SubSubContinuative1}

The continuative enclitic \textit{=ąmi} has cognates across the Siouan language family. In Missouri Valley Siouan, Hidatsa has the future enclitic \textit{=wi} and Crow has the enclitic \textit{=wis}, which is typically glossed as `probably', and the desiderative \textit{bia}. Crow likewise has a continuative \textit{=dawi}. In Ohio Valley Siouan, Ofo has \textit{-(a)b\u{i}} to mark future events, and Tutelo has \textit{-(a)pi} as a desiderative marker. In Catawba, \citet[54]{rudes2007a} glosses \textit{-wee} as a potential mood marker, but it is often glossed as `probably' by L1 consultants in the Catawba source texts.

There is no reconstructed form in Proto-Siouan in \citet{rankin2015}. Missouri Valley Siouan lost nasalization, but the presence of nasality in Mandan but no nasality in Ohio Valley Siouan suggests that there may have been competing forms *awi and *awį in the proto-language. All other Siouan languages use this formative to mark some kind of future or possible event. Crow has \textit{=dawi} as a true continuative and clear cognate with Mandan /=awį/, though this formative in Crow can also have inchoative or inceptive semantics \citep[307]{graczyk2007}. Mandan has undergone a semantic change and no longer has a future or potential reading for this formative. Verbs bearing the continuative aspectual enclitic show that some action keeps happening or happens over and over, typically before the following action cuts off the continuative action. This behavior is shown in the examples in (\ref{continuativeexamples}) below.

\begin{exe}

\item\label{continuativeexamples} Examples of the continuative enclitic \textit{=ąmi}

	\begin{xlist}
	
	\item\label{continuativeexamples1}
	
	\glll húur\textbf{ąmi}, súkseet ~ ~ ~ ~ ~ ~ ~ ~ ~ ~ ~ ~ ~ ~ ~ híroomako'sh\\
	huu=\textbf{awį} suk=s=ee=t ~ ~ ~ ~ ~ ~ ~ ~ ~ ~ ~ ~ ~ ~ ~ hi=oowąk=o'sh\\
	\textnormal{come.here}=\textbf{cont} \textnormal{child}=def=det.dist=loc ~ ~ ~ ~ ~ ~ ~ ~ ~ ~ ~ ~ ~ ~ ~ \textnormal{arrive.there}=narr=ind.m\\
	\glt `He kept coming, and then he got to the child' \citep[18]{hollow1973a}
	
	\item\label{continuativeexamples2}
	
	\glll ráah\textbf{ąmi}, tís híreehoomako'sh\\
	rEEh=\textbf{awį} ti=s hi\#rEEh=oowąk=o'sh\\
	\textnormal{go.there}=\textbf{cont} \textnormal{house}=def \textnormal{arrive.there}\#\textnormal{go.there}=narr=ind.m\\
	\glt `He was going along, and he approached the house' \citep[46]{hollow1973b}
	
	\item\label{continuativeexamples3}
	
	\glll ų́'shką\textbf{mi}, róotkikereroomako'sh\\
	ų'sh=ka=\textbf{awį} rootki=krE=oowąk=o'sh\\
	\textnormal{be.thus}=hab=\textbf{cont} \textnormal{strike}=3pl=narr=ind.m\\
	\glt `Continuing that way, they made camp' \citep[49]{hollow1973b}
	
	\item\label{continuativeexamples4}
	
	\glll ``mikó'sh, nukaráho'sh,'' éeheer\textbf{ąmi}\\
	wįk=o'sh rų-krah=o'sh ee-hee=\textbf{awį}\\
	\textnormal{be.none}=ind.m 1a-\textnormal{be.afraid.of}=ind.m pv-\textnormal{say}=\textbf{cont}\\
	\glt ` ``no, we are afraid,'' they kept saying' \citep[22]{hollow1973a}
	
	\item\label{continuativeexamples5}
	
	\glll ká'hara\textbf{mi}kereka'ehe\\
	ka'\#hrE=\textbf{awį}=krE=ka'ehe\\
	\textnormal{possess}\#caus=\textbf{cont}=3pl=quot\\
	\glt `they kept giving it to them, it is said' \citep[30]{kennard1936}
	
	\item\label{continuativeexamples6}
	
	\glll ptáah\textbf{ąmi}\\
	ptEh=\textbf{awį}\\
	\textnormal{run}=\textbf{cont}\\
	\glt `continuing to run along' \citep[30]{kennard1936}
	
	\end{xlist}

\end{exe}

\citet[30]{kennard1936} lists this enclitic only as \textit{=mi} in his grammar. However, when a stem ends in a consonant, he posits that there is a simultaneous \textit{=haa} enclitic between the stem and the continuative aspectual. This analysis does not hold, as all of the verb roots where he places a simultaneous marker end in /h/, as we see in (\ref{continuativeexamples6}), where the stem is /ptEh/ `run'. \citeauthor{kennard1936} often does not transcribe nasal vowels in the environment of a surface nasal stop, so it is unclear whether he perceives the overt [ã] in the enclitic. Furthermore, as discussed in \sectref{ablaut}, the continuative enclitic triggers ablaut on any preceding formative containing an ablaut vowel, i.e., /E/ like in /ptEh/ `run' or /EE/ like in /rEEh/ `go there' due to the underlying nasality in this formative.

\citet[51]{mixco1997a} likewise combines this marker with the simultaneous marker in his morphological analyses of Mandan narratives, stating that the /h/ in the simultaneous /=haa/ is deleted with consonant-final stems. However, this analysis likewise does not hold, as the concatenation of a stem ending with a long vowel with an enclitic beginning with a short vowel would trigger the insertion of an epenthetic [ɾ] between the long vowel and the short vowel like all enclitics described in \sectref{epentheticprocesses} do, i.e., /Vː\textsubscript{1}=V\textsubscript{2}/ are realized as [Vː\textsubscript{1}ɾV\textsubscript{2}]. The lack of epenthetic [ɾ] in these situations indicates that there cannot be an underlying long vowel as there is in /=haa/ present before the continuative aspectual enclitic, so what both \citeauthor{kennard1936} and \citeauthor{mixco1997a} are both perceiving is simply the /a/ in /=awį/.

We can see from the data in (\ref{continuativeexamples}) above that /=awį/ experiences typical behavior for vowel-initial enclitics, in that the initial vowel of the enclitic deletes before a stem that ends in an underlying short vowel, as shown in (\ref{continuativeexamples3}) and (\ref{continuativeexamples5}). The /a/ in /=awį/ is still deleted after the causative /hrE/, which ends in an underlying short vowel. Similarly, we notice that regressive nasal harmony does not pass an enclitic boundary in (\ref{continuativeexamples1}) and (\ref{continuativeexamples4}). We can thus conjecture that nasal harmony must take place at an earlier cycle than [ɾ]-epenthesis, else we would have \textit{húunami} and \textit{éheenami} in the previously mentioned examples.

\subsubsubsection{Continuative 2: =∅}\label{ParaContinuative2}\label{Ch3SubSubContinuative2}

In his grammar, \citet{mixco1997a} assumes that the simultaneous aspectual \textit{=haa} has an allomorph \textit{=aa} that appears on consonant-final roots.\footnote{This is never clearly stated in \citeapos{mixco1997a} grammar, but by looking at the distribution of anything glossed with \textsc{sim}, this pattern becomes clear.} I argue that \citet{mixco1997a,mixco1997b} glosses many examples with the simultaneous aspectual /=haa/, but he is actually confounding /=haa/ with stem vowel /=E/ followed by the phonologically null continuative aspectual enclitic /=$\varnothing$/. This /=$\varnothing$/ is lexically selected to trigger ablaut, which causes the stem vowel to ablaut to [aː]. This conflation between /=haa/ and /=$\varnothing$/ can be discerned by observing the interaction between vowel-final stems and the presence of word-final [aː], where \citet{mixco1997a} would predict we should see /=haː/, but we see /E/ turn to [aː] instead. To demonstrate this distinction, examples of /=$\varnothing$/ appear in (\ref{continuativeablaut}) below.



\begin{exe}

\item\label{continuativeablaut} Examples of continuative ablaut in Mandan

	\begin{xlist}
	
	\item\label{continuativeablaut1}
	
	\glll ó'harani wáaratax\textbf{a} ~ ~ ~ ~ ~ ~ ~ ~ ~ ~ héroomako'sh\\
	o'\#hrE=rį waa-ra-tax=E=$\varnothing$ ~ ~ ~ ~ ~ ~ ~ ~ ~ ~  hE=oowąk=o'sh\\
	\textnormal{be}\#caus=ss unsp-ins.mth-\textnormal{make.loud.noise}=sv=\textbf{cont} ~ ~ ~ ~ ~ ~ ~ ~ ~ ~ \textnormal{see}=narr=ind.m\\
	\glt `from there, he saw him crying a lot' \citep[14]{hollow1973b}

	\item\label{continuativeablaut2}

	\glll máa'ąk ų́ųpat íxatanashini sí ~ ~ ~ ~ ~ ~ ~ ~ ~ ~ hą́ąk\textbf{a} inák kúhoomaksįh\\
	wąą'ąk ųųpat i-xat=rąsh=rį si ~ ~ ~ ~ ~ ~ ~ ~ ~ ~ hąąkE=$\varnothing$ irąk kuh=ooąk=sįh\\
	\textnormal{land} \textnormal{be.different} pv.ins-\textnormal{look.at}=att=ss  \textnormal{travel} ~ ~ ~ ~ ~ ~ ~ ~ ~ ~ \textnormal{be.standing}.aux=\textbf{cont} \textnormal{again} \textnormal{come.back.here}=narr=ints\\
	\glt `he was looking over different lands and, traveling around, he came back again' \citep[8]{hollow1973a} 
	

	\item\label{continuativeablaut4}
	
	\glll manáat\textbf{a} waté'sh\\
	wa-rąątE=$\varnothing$ wa-te=o'sh\\
	1a-\textnormal{get.up}=\textbf{cont} 1a-\textnormal{stand}=ind.m\\
	\glt `I am standing up' \citep[173]{hollow1970}

	\item\label{continuativeablaut3}
	
	\glll ``ą́'skak mi'ó'ro'sh, korátoore, éepes'' éeheni  waráte rá'xį'k\textbf{a} kįnáateroomaks\\
	ą'ska=ak wį-o'=o'sh ko-ratoo=E ee-w-he=s ee-he=rį wrat=E ra'-xį'k=E=$\varnothing$ ki-rąątE=oowąk=s\\
	\textnormal{be.like.this}=ds 1s-\textnormal{be}=ind.m rel-\textnormal{be.old}=sv pv-1a-\textnormal{say}=def pv-\textnormal{say}=ss \textnormal{dust}=sv ins.heat-\textnormal{storm}=sv=\textbf{cont} itr-\textnormal{get.up}=narr=def\\
	\glt `{``}I am so, older, I said,'' he said and, there being a cloud of dust, he got back up' \citep[9]{hollow1973a}

	\item\label{continuativeablaut6}
	
	\glll suk ríirush\textbf{a} nuháaro'xere'sha?\\
	suk rV-i-ru-shE=$\varnothing$ rų-haa=o'xrE=o'sha\\
	\textnormal{child} 1a.pl-pv.ins-ins.hand-\textnormal{hold}=\textbf{cont} 1a.pl-\textnormal{separate}=dub=int.m\\
	\glt `we shouldn't be holding onto this child, should we?' \citep[160]{hollow1973a} 

	\item\label{continuativeablaut7}

	\glll shé'xt\textbf{a} mí'ksuke ~ ~ ~ ~ ~ ~ ~ ~ ~ ~ ~ ~ ~ ~ ~ ~ ~ ~ ~ ~  áareehereroomako'sh\\
	she'-xtE=$\varnothing$ wį'\#ksuk=E ~ ~ ~ ~ ~ ~ ~ ~ ~ ~ ~ ~ ~ ~ ~ ~ ~ ~ ~ ~   aa-rEEh\#hrE=oowąk=o'sh\\
	\textnormal{wind}-aug=\textbf{cont} \textnormal{stone}\#\textnormal{be.narrow}=sv ~ ~ ~ ~ ~ ~ ~ ~ ~ ~ ~ ~ ~ ~ ~ ~ ~ ~ ~ ~  pv.tr-\textnormal{go.there}\#caus=narr=ind.m\\
	\glt `being so windy, [the wind] picked up flint rocks' \citep[49]{hollow1973a}

	\item\label{continuativeablaut8}

	\glll nuh\textbf{ará} íxike ~ ~ ~ ~ ~ ~ ~ ~ ~ ~ ~ ~ ~ ~ ~ ~ ~ ~ ~ ~ máanuruhaani nukirí'sh\\
	rų-hrE=$\varnothing$ i-xik=E ~ ~ ~ ~ ~ ~ ~ ~ ~ ~ ~ ~ ~ ~ ~ ~ ~ ~ ~ ~ waa-rų-ru-haa=rį rų-kri=o'sh\\
	1a.pl-caus=\textbf{cont} pv.ins-\textnormal{be.bad}=sv ~ ~ ~ ~ ~ ~ ~ ~ ~ ~ ~ ~ ~ ~ ~ ~ ~ ~ ~ ~ part-1a.pl-ins.hand-\textnormal{separate}=ss 1a.pl-\textnormal{arrive.back.here}=ind.m\\
	\glt `doing it, we barely picked any [berries] and we came back' \citep[52]{hollow1973a}
	
	\item\label{continuativeablaut5}
	
	\glll sk\textbf{á} réeho'sh\\
	skE=$\varnothing$ rEEh=o'sh\\
	\textnormal{jump}=\textbf{cont} \textnormal{go.there}=ind.m\\
	\glt `he dashes off [lit. he goes there jumping]' \citep[211]{hollow1970}

	\newpage
	\item\label{continuativeablaut9}

	\glll súhkereseena rá'taxak, ~ ~ ~ ~ ~ ~ ~ ~ ~ Kįnúma'kshi kas\textbf{í}, máatah íwokahąą ~ ~ ~ ~  kasí\textbf{mi}roomak'osh\\
	suk=krE=s=ee=rą ra'-tax=ak ~ ~ ~ ~ ~ ~ ~ ~ ~ ki-ruwą'k\#shi ka-si=$\varnothing$ wąątah i-woka=hąą ~ ~ ~ ~  ka-si=awį=oowąk=o'sh\\
	\textnormal{child}=3pl=def=dem.dist=top ins.heat-\textnormal{make.loud.noise} ~ ~ ~ ~ ~ ~ ~ ~ ~ mid-\textnormal{man}\#\textnormal{be.good} incp-\textnormal{travel}=\textbf{cont} \textnormal{river} pv.dir-\textnormal{edge}=loc ~ ~ ~ ~  incp-\textnormal{travel}=\textbf{cont}=narr=ind.m\\
	\glt `the children cried and, with Old Man Coyote traveling, he set off traveling along the river edge' \citep[31]{hollow1973a}
	
	\end{xlist}

\end{exe}

Clauses bearing this kind of continuative marking are never matrix clauses. That is, this marker only appears on parenthetical clauses that serve as adjuncts that describe some kind of action or state that is continuing to take place while the matrix clause does. We can see this difference between the ablaut continuative and the overt =\textit{ąmi} continuative in (\ref{continuativeablaut9}), where both continuatives are present. Normally, we would expect to see a vowel change with an ablauted continuative, yet we see \textit{kasí} remain unchanged. In the corpus, Mrs. Otter Sage translates this word as `went traveling', which shows that \textit{kasí} has a continuative reading. Since /i/ cannot ablaut, the phonological shape of the word does not change, though its meaning does. Only /E/ and /EE/ can ablaut in Mandan, so this morphological distinction is only phonetically perceived when a formative with an ablaut vowel is followed by this /=$\varnothing$/ continuative marking. It is possible that these constructions may also have an added stem vowel /=E/, but any phonological realization of the stem vowel following a stem ending with a short vowel would result in the deletion of the stem vowel for the reasons described previously in \sectref{epentheticprocesses}, i.e., /ka-si=E/ $\to$ [ka.ˈsi], because the /VV/ sequence simplifies to just [V], eliminating the stem vowel /=E/. The matrix verb \textit{kasímiroomako'sh} `he was traveling' is marked with the overt continuative aspectual enclitic =\textit{ąmi}, reinforcing the fact that it is an enclitic, because the initial vowel of /awį/ is deleted when in contact with the /i/ in /si/.

Mandan shares this continuative ablaut with Hidatsa, where any stem-final vowel can undergo apophony. We can see examples of this process in (\ref{hidatsacontinuative}) below.\footnote{The examples from \citet{park2012} have been slightly modified to conform to the orthography being used at the time of this writing in the Hidatsa classes at Mandaree High School in Mandaree, ND.}

\begin{exe}

\item\label{hidatsacontinuative} Examples of continuative ablaut in Hidatsa

	\begin{xlist}
	
	\item\label{hidatsacontinuative1} 
	
	\glll gar\textbf{ía} háhguc, adáashigua\\
	garee=$\varnothing$ hahgu=c adaashi=gua\\
	\textnormal{vomit}=\textbf{cont} \textnormal{be.at}=decl \textnormal{outside}=loc\\
	\glt `he is vomiting outside' \cite[53]{park2012}
	
	\item\label{hidatsacontinuative2}
	
	\glll mirée rushg\textbf{á} náagihgeeta!\\
	wiree ru-shgi=$\varnothing$ raagi\#hgee=taa=$\varnothing$\\
	\textnormal{door} ins.hand-\textnormal{twist}=\textbf{cont} \textnormal{sit}.sg\#3caus.indr=neg=imp.pl\\
	\glt `don't leave the door open!' \citep[215]{park2012}
	
	\item\label{hidatsacontinuative3}
	
	\glll awáhs\textbf{ia} naharéec\\
	awa\#hsii=$\varnothing$ raharee=c\\
	\textnormal{land}\#\textnormal{be.hazy}=\textbf{cont} \textnormal{stand}=decl\\
	\glt `the haze is continuing' \citep[269]{park2012}
	
	\item\label{hidatsacontinuative4}
	
	\glll aracóoc\textbf{a} níirag néec\\
	aracooci=$\varnothing$ riiri=g ree=c\\
	\textnormal{shuffle.feet}=\textbf{cont} \textnormal{walk}=ss \textnormal{go}=decl\\
	\glt `he went shuffling his feet' \citep[539]{park2012}
	
	\item\label{hidatsacontinuative5}
	
	\glll caw\textbf{áa}'ii aruwaríahi ishíac\\
	cawee=$\varnothing$=ii aru-wa-iriahi ishia=c\\
	\textnormal{be.hot}=\textbf{cont}=ints irr=1a-\textnormal{breathe} \textnormal{be.bad}=decl\\
	\glt `it is so hot that it is hard for me to breathe' \citep[230]{park2012}
	
	\end{xlist}

\end{exe}

In the examples above, Hidatsa demonstrates that it shares a similar process whereby the continuative is marked through ablaut of the final vowel of a verbal stem. Unlike Mandan, Hidatsa is able to ablaut a non-final element within the verbal complex, as we see in (\ref{hidatsacontinuative5}), where the underlying /ee/ in \textit{cawée} `be hot' ablauts to /aa/ in \textit{cawáa'ii} `it keeps being really hot' and is still followed by the intensifier \textit{=ii}. In all the Mandan data in (\ref{continuativeablaut}), ablaut is restricted to elements in a clause-final position. We can see this behavior in (\ref{clausefinalablaut}) below, where (\ref{continuativeablaut5}) is repeated and its internal syntactic structure is shown.

\begin{exe}

\item\label{clausefinalablaut} Structure of continuative ablaut clauses

	\begin{xlist}
	
	\item\label{clausefinalablaut1}
	
	\glll sk\textbf{á} réeho'sh\\
	skE=$\varnothing$ rEEh=o'sh\\
	\textnormal{jump}=\textbf{cont} \textnormal{go.there}=ind.m\\
	\glt `he dashes off [lit. he goes there jumping]' \citep[211]{hollow1970}

	\item\label{clausefinalablaut2}
	
	{[} {[} \textit{ská} {]}\textsubscript{Adjunct clause} \textit{réeho'sh} {]}\textsubscript{Main clause} 

	\end{xlist}

\end{exe}

\subsubsubsection{Summary of continuatives}\label{ParaContinuativeSummary} Both continuatives can be found in adjunct clauses, typically with minimal additional morphology, but only =\textit{ąmi} is able to concatenate with additional formatives. The ablaut continuative is found only at the right edge of a clause, and ablaut only occurs on stems ending with /E/ or /EE/. A continuative reading can be present on stems that have no actual ablauted vowels provided that those stems end in a short vowel.

\subsubsection{Habitual aspectual enclitic: \textit{=ka}}\label{habitualaspectualenclitic}

Many Proto-Siouan roots are reconstructed containing *-ka \citep{rankin2015}. This element in Proto-Siouan is a derivational suffix that provides an attributive meaning, e.g., PSi *wi-roo-ka \textsc{anim.cl}-male-\textsc{att} > Mandan \textit{weróok} /wrook/ `buffalo bull.' In Mandan, an identical reflex of this attributive derivational suffix can still be seen in nominal morphology, where novel nouns can be created with the attributive \textit{=ka}/=\textit{k}, but when used with verbs, a phonetically identical enclitic \textit{=ka} marks habitual actions or states. This enclitic is often translated by speakers into English as `usually' or `always' with positive verbs and `never' with negative verbs. Propositions can also have an imperfective reading of actions that `used to' happen. We can see examples of \textit{-ka} in (\ref{habitualaspectualencliticEx}) below.



\begin{exe}

\item\label{habitualaspectualencliticEx} Examples of the habitual aspectual enclitic =\textit{ka}

	\begin{xlist}
	
	\item\label{habitualaspectualenclitic1}
	
	\glll wasí wáawahąąkexi\textbf{ka}'sh\\
	wa-si waa-wa-hąąkE=xi=\textbf{ka}=o'sh\\
	uns-\textnormal{travel} neg-1a-\textnormal{stand}.aux=neg=\textbf{hab}=ind.m\\
	\glt `I never go around traveling' \citep[54]{hollow1973a}

	\item\label{habitualaspectualenclitic2}

	\glll máanikshątinix\textbf{ka}'sh\\
	waa-rį-kshąt=rįx=\textbf{ka}=o'sh\\
	neg-2s-\textnormal{be.wise}=neg=\textbf{hab}=ind.m\\
	\glt `you are not usually careful' \citep[444]{hollow1970}

	\item\label{habitualaspectualenclitic3}

	\glll wáa'onatka óxik hąk waká'ni ~ ~ ~ ~  wahą́ąke\textbf{ka}'sh\\
	waa-o-rątka o-xik hąk wa-ka'=rį ~ ~ ~ ~  wa-hąąkE=\textbf{ka}=o'sh\\
	nom-pv.irr-\textnormal{heart} pv.irr-\textnormal{be.bad} \textnormal{be.standing}.aux 1a-\textnormal{possess}=ss ~ ~ ~ ~  1a-\textnormal{be.standing}.aux=\textbf{hab}=ind.m\\
	\glt `I always have this bad feeling [for them]' \citep[56]{hollow1973a}

	\item\label{habitualaspectualenclitic4}

	 \glll íwapashiri\textbf{ka}so'sh\\
	 i-wa-pa-shrih=\textbf{ka}=s=o'sh\\
	 pv.ins-1a-ins.prce-\textnormal{think}=\textbf{hab}=ind.m\\
	\glt `I used to think about it' \citep[444]{hollow1970}

	\item\label{habitualaspectualenclitic5}
	\glll xamáhe nurúshektiki ~ ~ ~ ~ ~ ~ ~ ~ ~ ~ ~ ~ ~ ~ ~ íheh\textbf{ka}'sh\\
	xwąh=E rų-ru-shE=kti=ki ~ ~ ~ ~ ~ ~ ~ ~ ~ ~ ~ ~ ~ ~ ~  i-hek=\textbf{ka}=o'sh\\
	\textnormal{be.small}=sv 1a.pl-ins.hand-\textnormal{hold}=pot=cond ~ ~ ~ ~ ~ ~ ~ ~ ~ ~ ~ ~ ~ ~ ~ pv.ins-\textnormal{know}=\textbf{hab}=ind.m\\
	\glt `whenever we take a little bit, he always knows' \citep[116]{hollow1973b}
	
	\end{xlist}

\end{exe}

This enclitic always marks habitual actions, whether this action is something that always happens or never happens. It is also used when describing habitual actions in the past that no longer take place. \citet[19]{kennard1936} does not analyze =\textit{ka} as a single element he calls the customary marker, but analyzes the sequence of the habitual and the gender-oriented indicative markers as being a single unit, i.e., /=ka=o'sh/ is /=ka'sh/ per \citeauthor{kennard1936}. When the definite marker /=s/ is present, \citeauthor{kennard1936} likewise analyzes /=ka=s/ as a single element that he calls the usitative. There does not seem to be a clear difference between these two constructions in \citeapos{kennard1936} grammar.

\citet[444]{hollow1970} argues that this enclitic is a single discrete element, but calls it an imperfective aspectual. This analysis does not entirely fit the actual distribution and fuction of this enclitic, especially when \citeauthor{hollow1970} states that an imperfective action can take preterite marking, where he analyzes the definite marker /=s/ as preterite. As discussed at the beginning of this subsection in example (\ref{hollowtense}), it is not the case that Mandan morphologically marks tense, so /=s/ cannot be a true preterite marker. Furthermore, preterite marking is inherently a combination of the perfective aspect and past tense, so it would be contradictory for a true imperfective to take preterite marking.

For the sake of eliminating any confusion over the primary role this enclitic plays in expressing habitual or repeated actions, I refer to it as a habitual aspect marker, which lines up with the description of =\textit{ka} in \citet[27]{mixco1997a}. It is likewise not accurate to simply call this enclitic a true imperfective, as it cannot be used to describe actions that are continuous until cut off by an intervening action. We can observe this lack of =\textit{ka}-marking in (\ref{notimperfective}) below, where one imperfective action is taking place (i.e., facing the basket towards the old woman) and is then interrupted by a perfective action (i.e., the spirit of the old woman shooting into the basket).

\begin{exe}

\item\label{notimperfective} Lack of imperfective marking on an imperfective action

\glll roką́ąkaxi'hseet ~ ~ ~ ~ ~ ~ ~ ~ ~ ~ ~ ~ ~ ~ ~ ~ ~ ~ ~  ókitataaherek, shé ~ ~ ~ ~ ~ ~ ~ ~ ~ ~ ~ ~ ~ ~ ~ ~ karópxeka'eheero'sh, ~ ~ ~ ~ ~ ~ ~ ~ ~ ~ ~ ~ ~ ~ ~ ~ ~ ~ ~ ~ ~ ~ ~ ~ ~ taxé'hąkseet\\
	rokąąka\#xi'h=s=ee=t ~ ~ ~ ~ ~ ~ ~ ~ ~ ~ ~ ~ ~ ~ ~ ~ ~ ~ ~  o-ki-ta$\sim$taa\#hrE=ak she ~ ~ ~ ~ ~ ~ ~ ~ ~ ~ ~ ~ ~ ~ ~ ~ ka-ropxE=ka'ehee=o'sh ~ ~ ~ ~ ~ ~ ~ ~ ~ ~ ~ ~ ~ ~ ~ ~ ~ ~ ~ ~ ~ ~ ~ ~ ~ ta-xe'\#hąk=s=ee=t\\
	\textnormal{old.woman}\#\textnormal{be.old}=def=dem.dist=loc ~ ~ ~ ~ ~ ~ ~ ~ ~ ~ ~ ~ ~ ~ ~ ~ ~ ~ ~  pv.loc-mid-cont$\sim$\textnormal{be.facing}\#caus=ds \textnormal{wind} ~ ~ ~ ~ ~ ~ ~ ~ ~ ~ ~ ~ ~ ~ ~ ~ ins.frce-\textnormal{enter}=quot=ind.m ~ ~ ~ ~ ~ ~ ~ ~ ~ ~ ~ ~ ~ ~ ~ ~ ~ ~ ~ ~ ~ ~ ~ ~ ~ 3poss.al-\textnormal{hang}\#\textnormal{stand}.pos=def=dem.dist=loc\\
	\glt `he was making it face the old woman, and her spirit went in, it is said, into his basket' \citep[148]{hollow1973a}

\end{exe}

The lack of any overt marking of imperfective aspect for the first action (i.e., making the basket face the old woman) highlights the fact that the semantics of =\textit{ka} are narrower than those of a true imperfective. As such, treating this enclitic as marking habitual actions or states more accurately encapsulates the observed usage in the corpus.

\subsubsection{Prospective aspectual enclitic: \textit{=naate}}\label{prospectiveaspectualenclitic}

The prospective aspectual enclitic =\textit{naate} has cognates across the Siouan language family. In Hidatsa, \citet[165]{boyle2007} describes the approximative =\textit{raa} as conveying the sense of `almost'. Crow has a cognate =\textit{laa} \citep[164]{graczyk2007}. There are two elements in Biloxi that have a match with the meaning of this formative in Mandan and Missouri Valley: the enclitic =\textit{nąąteke} `nearly' and the adverb \textit{yąąxa} `almost' \citep{kaufman2011}. %Both Mandan and Missouri Valley merge Proto-Siouan *r and *y with /r/, while Biloxi maintains a distinction between *r and *y, where *r becomes /d/ before oral vowels and /n/ before nasal vowels, while *y remains /y/.

The presence of a doublet in Biloxi with two different reflexes complicates a possible Proto-Siouan reconstruction. The varied material present in the Biloxi forms and the shorter forms in Crow and Hidatsa suggest that this was a complex construction in Proto-Siouan, *rąą-tE, where the initial element looks to be related to *rąą- `by foot' plus a *-tE stem augment.\footnote{This stem augment is never discussed in \citet{rankin2015}, though it does bear resemblance to the Mandan verb \textit{té} `stand', which itself is derived from the Proto-Siouan standing classifier for inanimate entitives PSi *rahE, i.e., *rahE > **rhE > Man. \textit{té}. This stem augment *-tE appears on several enclitics, but it is not clear what those enclitics might have in common to trigger the addition of *-tE to the stem.} Mandan uses this enclitic to mark a situation that almost came to pass, and is nearly always translated with the English word `almost'. We can see the behavior of this enclitic in (\ref{prospectiveaspectual}) below.

\begin{exe}

\item\label{prospectiveaspectual} Examples of the prospective aspectual enclitic =\textit{naate}

	\begin{xlist}
	
	\item\label{prospectiveaspectual1} 
	\glll ų́'sh, áakee\textbf{naataa}ni róo ~ ~ ~ ~ ~ ~ ~ ~ ~ ~ ~ ~ ~ ~ ~  áakeeroomaksįh\\
	ų'sh aa-kee=\textbf{rąątE}=rį roo ~ ~ ~ ~ ~ ~ ~ ~ ~ ~ ~ ~ ~ ~ ~  aa-kee=oowąk=sįh\\
	\textnormal{be.thus} pv.tr-\textnormal{move.away}=\textbf{prsp}=ss dem.mid ~ ~ ~ ~ ~ ~ ~ ~ ~ ~ ~ ~ ~ ~ ~ pv.tr-\textnormal{move.away}=narr=ints\\
	\glt `So, I almost stepped on her and stepped on her here' \citep[71]{hollow1973a}

	\item\label{prospectiveaspectual2} 
	\glll súkxikanashnak pá ~ ~ ~ ~ ~ ~ ~ ~ ~ ~ ~ ~ ~ ~ ~ ~ ~ ~ ~ ~ ~ ~ ~ ~ ~ ~ ~ ~ ~ ~ warú'uuxa\textbf{naate}'sh\\
	suk\#xik=rąsh\#rąk pa ~ ~ ~ ~ ~ ~ ~ ~ ~ ~ ~ ~ ~ ~ ~ ~ ~ ~ ~ ~ ~ ~ ~ ~ ~ ~ ~ ~ ~ ~ wa-ru-uux=\textbf{rąątE}=o'sh\\
	\textnormal{child}\#\textnormal{be.bad}=att\#pos.sit \textnormal{head} ~ ~ ~ ~ ~ ~ ~ ~ ~ ~ ~ ~ ~ ~ ~ ~ ~ ~ ~ ~ ~ ~ ~ ~ ~ ~ ~ ~ ~ ~ 1a-ins.hand-\textnormal{be.broken}=\textbf{prsp}=ind.m\\
	\glt `I almost broke this no-good child's head' \citep[158]{hollow1973b}
	
	\item\label{prospectiveaspectual3}
	\glll miní wáaxtaani watúshak ~ ~ ~ ~ ~ ~ ~ ~ ~ ~ ~ ~ ~ ~ ~ ~ ~ ~ ~ ~  óruxke\textbf{naate}roomako'sh\\
	wrį waa-xtE=rį wa-tush=ak ~ ~ ~ ~ ~ ~ ~ ~ ~ ~ ~ ~ ~ ~ ~ ~ ~ ~ ~ ~  o-ru-xke=\textbf{rąątE}=oowąk=o'sh\\
	\textnormal{water} nom-\textnormal{be.big}=ss unsp-\textnormal{be.fast}=ds ~ ~ ~ ~ ~ ~ ~ ~ ~ ~ ~ ~ ~ ~ ~ ~ ~ ~ ~ ~  pv.loc-ins.hand-\textnormal{sink}=\textbf{prsp}=narr=ind.m\\
	\glt `the water was really fast and he almost sank' \citep[296]{hollow1973b}

	\item\label{prospectiveaspectual4}
	\glll wáatoohe míka\textbf{naate}'sh\\
	waa-tooh=E wįk=\textbf{rąątE}=o'sh\\
	nom-\textnormal{be.blue/green}=sv \textnormal{be.none}=\textbf{prsp}=ind.m\\
	\glt `there is almost no blue' \citep[468]{hollow1970}

	\item\label{prospectiveaspectual5}
	\glll	mú'pka ótaa waheré\textbf{naate}'sh\\
	w'-ųk\#pa o-taa wa-hrE=\textbf{rąątE}=o'sh\\
	1poss-\textnormal{hand}\#\textnormal{head} pv.loc-\textnormal{be.facing} 1a-caus=\textbf{prsp}=ind.m\\
	\glt `I almost pointed my thumb at him' \citep[133]{hollow1973b}

	\end{xlist}


\end{exe}

The enclitic preserves the formative-final ablaut vowel from Proto-Siouan, as we see in (\ref{prospectiveaspectual1}), where the same-subject switch-reference marker =\textit{ni} triggers /E/ to become [aa]. However, unlike most other enclitics that contain nasal elements, =\textit{naate} itself does not trigger ablaut in a preceding stem, as we see in (\ref{prospectiveaspectual5}), where the underlying ablaut vowel in the causative /hrE/ does not become [aa]. The ability to be ablauted but not cause ablaut is consistent across the corpus for all speakers.

\subsubsection{Celerative aspectual enclitic: =\textit{niite}}

The celerative aspectual enclitic =\textit{niite} is first identified in \citet[30]{hollow1973a}, which he describes as marking actions that happen quickly or states that arise suddenly. This description matches the usage of this enclitic throughout the corpus. This enclitic seems to be derived from Proto-Siouan *rįį- `walk'. Other Siouan languages use particular motion verbs to mark aspect periphrastically, such as Lakota, as seen in the data in (\ref{lakotacelerative}) below.

\begin{exe}

\item\label{lakotacelerative} Aspectual auxiliaries in Lakota

	\begin{xlist}
	
	\item\label{lakotacelerative1}
	\glll kaksá iyéwaye\\
	ka-ksA i-yA\#wa-yA\\
	ins.frce-\textnormal{cut} pv.dir-\textnormal{speak}\#1a-\textnormal{go}\\
	\glt `I suddenly cut it off' [lit. `I sent it away cutting'] \citep[38]{ingham2003}

	\item\label{lakotacelerative2}
	\glll kaksá iyáye\\
	ka-ksA i-yA$\sim$yA\\
	ins.frce-\textnormal{cut} pv.dir-r$\sim$\textnormal{go}\\
	\glt `[the clouds] cleared away' (lit. [the clouds] went away cutting') \citep[38]{ingham2003}
	
	\end{xlist}

\end{exe}

Both of the examples above show a motion verb combining with an action to convey the sense that the action has happened suddenly as in (\ref{lakotacelerative1}), or that the action happened quickly as in (\ref{lakotacelerative2}). Both of these uses of motion verbs in Lakota could parallel the use of \textit{nii} `walk' in Pre-Mandan, where an auxiliary could have become reanalyzed as an enclitic. The presence of the final syllable on =\textit{niite} suggests that there was a *-tE stem augment on the auxiliary verb at some point. The motivation for how certain stem augments in Proto-Siouan become part of certain stems is not known. \citet[512]{jones1991} proposes that Proto-Siouan roots have a basic shape of CV(V), and that any consonant clusters or root extensions are due to remnants of ancient morphology. Certain root extensions are common throughout Siouan and seem obvious, as in the case of stems ending in /k/ or /ka/, which derive from the same source as the habitual enclitic in Mandan that was discussed previously in \sectref{habitualaspectualenclitic}. The *-tE stem augment appears in several Mandan enclitics, but its origin is not well understood.\footnote{Proto-Siouan likewise has several stem augments that are poorly understood, such as *-he, *-te, and *-re. \citet{jones1991} argues that Proto-Siouan had a system of root extensions that have become fossilized onto lexical stems, in many cases obscuring the semantics conveyed by adding these morphological items, but more work is needed in comparative Siouan to posit what role these elements played in the protolanguage.}

While the entirety of the Proto-Siouan origin of =\textit{niite} is opaque, its usage is not. Any action that happens quickly or state that arises suddenly will take the celerative aspectual enclitic. We can see examples of its use in (\ref{celerativeaspectual}) below.

\begin{exe}

\item\label{celerativeaspectual} Examples of the celerative aspectual enclitic =\textit{niite}

	\begin{xlist}
	
	\item\label{celerativeaspectual1}
	\glll wáaxnake ímisąąpa ~ ~ ~ ~ ~ ~ ~ ~ ~ ~ ~ ~ ~ waráahi\textbf{niite}kto'sh\\
	waax\#rąk=E i-wį-sąąpE=$\varnothing$ ~ ~ ~ ~ ~ ~ ~ ~ ~ ~ ~ ~ ~  wa-rEEh=\textbf{rįįtE}=kt=o'sh\\
	\textnormal{cottonwood}\#pos.sit=sv pv.dir-1s-\textnormal{around}=cont ~ ~ ~ ~ ~ ~ ~ ~ ~ ~ ~ ~ ~  1a-\textnormal{go.there}=\textbf{cel}=pot=ind.m\\
	\glt `I want to go quickly around that cottonwood tree' \citep[147]{hollow1973a}

	\item\label{celerativeaspectual2}
	\glll kotewé k\'{ı̨}'ki\textbf{niite}ki, íshąhe, ~ ~ ~ ~ ~ ~ ~ ~ ~ ~ ~ ~ ~ ~ ~ kaxípa ~ ~ ~ ~ ~  k\'{ı̨}'ki\textbf{niite}ki, ée ~ ~ ~ ~ ~ ~ ~ ~ wáaka'ro'sh\\
	ko-t-we kį'k=\textbf{rįįtE}=ki i-shąh=e ~ ~ ~ ~ ~ ~ ~ ~ ~ ~ ~ ~ ~ ~ ~ ka-xip=E=$\varnothing$ ~ ~ ~ ~ ~ kį'k=\textbf{rįįtE}=ki ee ~ ~ ~ ~ ~ ~ ~ ~ waa-ka'=o'sh\\
	rel-wh-indf \textnormal{finish}=\textbf{cel}=cond 3poss-\textnormal{side}=sv ~ ~ ~ ~ ~ ~ ~ ~ ~ ~ ~ ~ ~ ~ ~ ins.frce-\textnormal{skin}=sv=cont ~ ~ ~ ~ ~ \textnormal{finish}=\textbf{cel}=cond dem.dist ~ ~ ~ ~ ~ ~ ~ ~ nom-\textnormal{possess}=ind.m\\
	\glt `if someone finishes quickest, if [whoever] finishes skinning his side quickest, the [whole thing] is his' \citep[42]{hollow1973a}

	\item\label{celerativeaspectual3}
	\glll náata\textbf{niita}ni éerehanashki, mí'seena ~ ~ ~ ~ írushenashoomaksįh\\
	rąątE=\textbf{nįįtE}=rį ee-reh=rąsh=ki wį'=s=ee=rą ~ ~ ~ ~ i-ru-shE=rąsh=oowąk=sįh\\
	\textnormal{stand}=\textbf{cel}=ss pv-\textnormal{want}=att=cond \textnormal{stone}=def=dem.dist=top ~ ~ ~ ~  pv.ins-ins.hand-\textnormal{hold}=att=narr=ints\\
	\glt `when he tried to get up in a hurry, the rock was kind of holding him' \citep[45]{hollow1973a}

	\item\label{celerativeaspectual4}
	\glll pí'\textbf{niite}karani róo Kįnúma'kshi wáaheres ~ ~ ~ rusháni éerehkerek...\\
	pi'=\textbf{rįįtE}=krE=rį roo ki-ruwą'k\#shi waa-hrE=s ~ ~ ~ ru=shE=rį ee-reh=krE=ak\\
	\textnormal{devour}=\textbf{cel}=3pl=ss dem.mid mid-\textnormal{man}\#\textnormal{be.good} nom-caus=def ~ ~ ~  ins.hand-\textnormal{hold}=ss pv-\textnormal{want}=3pl=ds\\
	\glt `they ate [their mother's food] up quickly and they wanted to take Old Man Coyote's food here...' \citep[27]{hollow1973a}
	
	\end{xlist}

\end{exe}

Unlike the prospective aspectual enclitic =\textit{naate}, the celerative aspectual triggers ablaut, as we can see in (\ref{celerativeaspectual3}) above. It is not clear why =\textit{niite} causes ablaut but =\textit{naate} does not, especially since ablaut is predominantly triggered by morphology bearing nasal segments. One possibility is that this formative is a relatively recent addition to the enclitic field and that this tardiness is what has exempted it from triggering ablaut, versus the majority of ablaut-triggering enclitics which seems to also trigger ablaut in other Siouan languages.\footnote{\citet{jones1983b} and \citet{rood1983} both bring up some possible scenarios in Pre-Proto-Siouan where ablaut could have been a regular phonological phenominon that became morphologized after a sound change, but to the best of my knowledge, little work has been done recently to investigate this hypothesis. The crux of their theory is that Siouan ablaut originated from a collapse of possible nasal vowels in Pre-Proto-Siouan, where nasal mid vowels *ę and *ǫ merged with oral vowels, resulting in apophony in Pro-Siouan.}

\subsubsection{Iterative aspectual enclitic: =\textit{ske}}\label{iterativeaspectualenclitic}

Mandan can mark iterative aspect derivationally on a verb through the prefix \textit{ki}-, which has cognates across the Siouan language family. Another way in which iterativity manifests in Mandan is through the iterative aspectual enclitic =\textit{ske}. Hoocąk has a cognate =\textit{šge} `too, also' (\citeauthor{lundquist2015} p.c.). There seems to be a semantic difference between these two enclitics, but as previously seen in (\ref{retrospectiveaspectualenclitic2}) the iterative adverbial \textit{inák} `again' in Mandan can also have the meaning `too' or `also'. Examples of this iterative enclitic =\textit{ske} appear in (\ref{iterativeaspectualencliticEx}) below.

\begin{exe}
\item\label{iterativeaspectualencliticEx} Examples of the iterative aspectual enclitic \textit{=ske}
	\begin{xlist}

	\item\label{iterativeaspectualenclitic1} \glll kasúk\textbf{ske}roomako'sh\\
		ka-suk=\textbf{ske}=oowąk=o'sh\\
		iter-\textnormal{exit}=\textbf{iter}=narr=ind.m\\
		\glt `he came out again' \citep[152]{hollow1973b}
	
	\item\label{iterativeaspectualenclitic2} \glll Roką́ąkakotawiihą'kas ~ ~ ~ ~ ~ ~ ~ ~ ~ ~ ~ ~ ~ ~ ~   wáakimaaxe\textbf{ske}roomako'sh\\
			rokąąka\#ko-ta-wiihą'ka=s ~ ~ ~ ~ ~ ~ ~ ~ ~ ~ ~ ~ ~ ~ ~  waa-kiwąąxE=\textbf{ske}=oowąk=o'sh\\
			\textnormal{old.woman}\#3poss.pers-al-\textnormal{grandchild}=def ~ ~ ~ ~ ~ ~ ~ ~ ~ ~ ~ ~ ~ ~ ~  unsp-\textnormal{ask}=\textbf{iter}=narr=ind.m\\
			\glt `Old Woman's Grandson asked her again' \citep[138]{hollow1973b}

	\item\label{iterativeaspectualenclitic3} \glll kisúk ı̨́'here\textbf{ske}roomako'sh\\
		ki-suk į'-hrE=\textbf{ske}=oowąk=o'sh\\
		mid-\textnormal{child} pv.rflx-caus=\textbf{iter}=narr=ind.m\\
		\glt `he turned into a child again' \citep[150]{hollow1973b}

	\item\label{iterativeaspectualenclitic4}
	\glll wáarakų'\textbf{ske}nitinixo'sha?\\
	waa-ra-kų'=\textbf{ske}=rįt=rįx=o'sha\\
	neg-1a-\textnormal{give}=\textbf{iter}=2pl=neg=int.m\\
	\glt `you (pl.) haven't given it to him again?' \citep[480]{hollow1970}

		\item\label{iterativeaspectualenclitic5}
		\glll rakų́'karaa\textbf{ske}nito'sh\\
		ra-kų'=krE=\textbf{ske}=rįt=o'sh\\
		1a-\textnormal{give}=3pl=\textbf{iter}=2pl=ind.m\\
		\glt `you (pl.) are giving it to them again' \citep[453]{hollow1970}

	\item\label{iterativeaspectualenclitic6}
	\glll karátaxaa máakaa\textbf{ske}ki\\
	k-ra-tax=E=$\varnothing$ wąąkE=\textbf{ske}=ki\\
	iter-ins.mth-\textnormal{make.loud.noise}=sv=cont \textnormal{lie}.aux=\textbf{iter}=cond\\
	\glt `when he continued crying' \citep[244]{trechter2012b}
	
	\end{xlist}
\end{exe}

As previously discussed in \sectref{morphologicallyconditionedablaut}, some speakers treat this as an ablaut-trig\-gering enclitic, while others do not. It is not clear if this distinction follows old Núu'etaa or Rúptaa lines and is dialectal, or if the difference is idiolectal or familiolectal. Neither \citet[480]{hollow1970} nor \citet[29]{mixco1997a} mention what difference exists between the iterative prefix \textit{ki}- and the iterative aspectual enclitic =\textit{ske}, with both authors questioning whether there is a difference. \citet[11]{kennard1936} suggests that the difference between the two is equivalent to the difference between using the English prefix `{re}-' as in `{reconvene}' and simply using the adverb `again', with \textit{ki}- being `re' and =\textit{ske} being `again'. Furthermore, \citeauthor{kennard1936} states that =\textit{ske} is not a true repetitive (i.e., action that happens over and over again), but that the action has repeated perhaps once. Work with contemporary Mandan speakers has not settled the difference between these two formatives, and it may certainly be the case that it is a difference in stylistics rather than semantics.

\subsection{Evidential and modal enclitics}

Several of the grammars on Siouan languages that have been published in the last decade or so (e.g., \citeapos{graczyk2007} grammar of Crow and \citeapos{park2012} grammar of Hidatsa) have described much of what had been previously described by earlier scholars as tense markers as really being evidentials or modals. Re-examining similar markers in Mandan, we can make a similar case that many of the verbal endings that \citet{hollow1970} and \citet{mixco1997a} call tense markers are truly evidentials or modals. A list of these enclitics appears in (\ref{listofevidentials}) below.

\begin{exe}

\item\label{listofevidentials} Evidential and modal enclitics in Mandan

\begin{xlist}

	\item /=aahka/ dynamic modal (\textsc{able})
	\item /=a'shka/ possible modal (\textsc{psbl})
	\item /=ishi/ visual evidential (\textsc{vis})
	\item /=ka'ehe/ quotative evidential (\textsc{quot})
	\item /=kt/	potential modal (\textsc{pot})
        \item /=ootE/	indirect evidential (\textsc{evid})
	\item /=oowąk/ narrative evidential (\textsc{narr})
	\item /=o'xrE/ dubitative modal (\textsc{dub})
	\item /=rąsh/ attitudinal evidential (\textsc{att})
%	/=rįkuk/ & incredulative modal (\textsc{incd})\\
	\item /=s/ definite evidential (\textsc{def})

\end{xlist}

\end{exe}

The following subsections illustrate the usage of these enclitics within the corpus. These are some of the most common post-verbal elements in Mandan, as most non-direct evidence is accompanied by some kind of evidential enclitic. Mandan is similar to Hidatsa in this respect \citep[220]{park2012}, though it is not clear if this extensive use of evidentials is a shared trait with Hidatsa or a carry-over effect of one language influencing another due to hundreds of years of living closely together and intermarriage between the two groups. While several of these enclitics appear extensively throughout the corpus (e.g., the narrative evidential =\textit{oomak} and the potential modal =\textit{kt}), there are some that scarcely occur (e.g., the possible modal =\textit{a'shka}).

\subsubsection{Dynamic modal enclitic: =\textit{aahka}}

The dynamic modal =\textit{aahka} is homophonous with the retrospective aspectual =\textit{aahka}. No data includes instances of both these markers in their respective roles being used on the same verb, however. It is unclear whether Mandan allows both manifestations of =\textit{aahka} to appear simultaneously like other homophonous formatives (e.g., the polysemous \textit{ki}- prefix, which can be iterative, middle voice, vertitive, etc.). The lack of L1 speakers at this point leaves any discussion of multiple =\textit{aahka} marking as purely hypothetical, given the lack of such in the corpus. As previously discussed in \sectref{Sec:Retrospective}, this enclitic is likely derived from some form of the term \textit{áaki} `be above, on top of.' We can see examples of the dynamic modal enclitic in the data below in (\ref{dynamicmodal}).

\begin{exe}

\item\label{dynamicmodal} Examples of the dynamic modal enclitic =\textit{aahka}

	\begin{xlist}
	
	\item\label{dynamicmodal1}
	\glll káare mishų́ųkas ka'óot\textbf{aahka}ni kixawáaro'sh\\
	kaare wį-shųųka=s ka-oot=\textbf{aahka}=rį ki-xwaa=o'sh\\
	neg.imp 1poss-\textnormal{male's.younger.brother}=def ins.frce-\textnormal{mix}=\textbf{able}=ss mid-\textnormal{disappear}=ind.m\\
	\glt `[I said] don't let my brother get hurt, and he got hurt and died' \citep[63]{hollow1973a}

	\item\label{dynamicmodal2}
	\glll íkixųųh-haa-pirák kapéeka'eheero'sh, ~ ~ ~ ~ ~ ~ ~ numá'ke írupa óruxok\textbf{aahka}\\
	i-kixųųh=haa\#pirak ka-pee=ka'ehe(e)=o'sh ~ ~ ~ ~ ~ ~ ~ ruwą'k=E i-ru-pa o-ru-xok=\textbf{aahka}\\
	pv.num-\textnormal{five}=sim\#\textnormal{ten} ins.frce-\textnormal{be.distributed}=quot=ind.m ~ ~ ~ ~ ~ ~ ~ \textnormal{man}=sv pv.ins-ins.hand-\textnormal{pull?} pv.irr-ins.hand-\textnormal{lift}=\textbf{able}\\
	\glt `there were fifty were left, it is said, men who could lift a gun' \citep[47]{hollow1973a}

	\item\label{dynamicmodal3}
	\glll ``máa'ųst niníir\textbf{aahka}ki'' éeheerak...\\
	waa-ųt=t rį$\sim$rįį=\textbf{aahka}=ki ee-hee=ak\\
	nom-\textnormal{be.in.past}=loc r$\sim$\textnormal{walk}=\textbf{able}=cond pv-\textnormal{say}=ds\\
	\glt ` `if only [the child] could walk already', he said and...' \citep[160]{hollow1973a}

	\end{xlist}

\end{exe} 

This modal almost always corresponds to the English modal `can' or `could' in the sense of conveying one's ability to perform an action or allow an action to come to pass.

\subsubsection{Possible modal enclitic: =\textit{a'shka}}
\largerpage
Possibility in Mandan can be expressed through the use of =\textit{a'shka} as an enclitic or a free modal \textit{á'shka} `maybe'. This formative is relatively rare in the corpus. We see examples of the possible modal in (\ref{possiblemodal}) below.

\begin{exe}

\item\label{possiblemodal} Examples of the possible modal enclitic =\textit{a'shka}

	\begin{xlist}
	
	\item\label{possiblemodal1} \glll riréesik\textbf{a'shka} ó'{e}reho'sh\\
	ri-reesik=\textbf{a'shka} o-e-reh=o'sh\\
	2poss-\textnormal{tongue}=\textbf{psbl} pv.irr-pv-\textnormal{think}=ind.m\\
	\glt `he'll think that it might be your tongue' \citep[189]{hollow1973a}

	\item\label{possiblemodal2}
	\glll káni ``tashká waheréki wawáruuto'xara\textbf{`shka},'' ~ ~ ~ ~ ~ ~ ~  éerehoomako'sh\\
	ka=rį tashka wa-hrE=ki wa-wa-ruut=o'xrE\textbf{=a'shka} ~ ~ ~ ~ ~ ~ ~ ee-reh=oowąk=o'sh\\
	pros=ss \textnormal{how} 1a-caus=cond unsp-1a-\textnormal{eat}=dub=\textbf{psbl} ~ ~ ~ ~ ~ ~ ~  pv-\textnormal{think}=narr=ind.m\\
	\glt `and how might I be going to eat if I do that?' \citep[46]{hollow1973b}

	\item\label{possiblemodal3}
	\glll mí'shak, maní'o'na ą́'skarahara\textbf{'shka}, éwaharani...\\
	w'\~~-ishak wa-rį-o'=rą ą'ska\#ra-hrE=\textbf{a'shka} e-wa-hrE=rį\\
	1s-\textnormal{pro} unsp-1s-\textnormal{be}=top \textnormal{be.near}\#2a-caus=\textbf{psbl} pv-1a-caus=ss\\
	\glt `me? you were the one who maybe did something, I thought and...' \citep[238]{hollow1973b}

	\item\label{possiblemodal4}
	\glll súkxiknak téewahere\textbf{'shka} ~ ~ ~ ~ ~ ~ ~ ~ ~ ~ ~ ~ ~ ~ ~  wáa'iwapkaaxi'sh\\
	suk\#xik\#rąk tee\#wa-hrE=\textbf{a'shka} ~ ~ ~ ~ ~ ~ ~ ~ ~ ~ ~ ~ ~ ~ ~  waa-i-wa-pkE=xi=o'sh\\
	\textnormal{child}\#\textnormal{be.bad}\#pos.sit \textnormal{die}\#1a-caus=\textbf{psbl} ~ ~ ~ ~ ~ ~ ~ ~ ~ ~ ~ ~ ~ ~ ~  neg-pv.ins-1a-\textnormal{taste}=neg=ind.m\\
	\glt `maybe me killing that no-good kid will not do me any good' \citep[132]{hollow1973b}

	\item\label{possiblemodal5}
	\glll ``nutámaanuks kirí \textbf{á'shka},'' ~ ~ ~ ~ ~ ~ ~ ~ ~ ~ éehekereroomako'sh\\
	rų-ta-waarųk=s k-ri \textbf{a'shka} ~ ~ ~ ~ ~ ~ ~ ~ ~ ~  ee-he=krE=oowąk=o'sh\\
	1a.pl-al-\textnormal{male's.friend}=def vert-\textnormal{arrive.there} \textbf{psbl} ~ ~ ~ ~ ~ ~ ~ ~ ~ ~ pv-\textnormal{say}=3pl=narr=ind.m\\
	\glt `{``}I wonder if our friend got back', they said' \citep[214]{hollow1973b}

	\end{xlist}

\end{exe}

Generally, this enclitic triggers ablaut. There are only a few instances in the corpus where =\textit{a'shka} does not result in ablaut, like in (\ref{possiblemodal4}). Most enclitics that trigger ablaut have historically featured nasal segments, but this is not the case with =\textit{a'shka}. This word does bear similarity to the word \textit{ą́'ska} `be near, correct', so it is possible that =\textit{a'shka} is a phonologically reduced form that has lost its nasal feature as an enclitic, but the historical nasality has caused it to be classified as an ablaut-triggering enclitic. Furthermore, the mismatch in fricatives could be due to a fricative alternation caused by the sound symbolism Mandan has, as described in \sectref{soundsymbolism}.

Another use of this formative is as a free formative \textit{á'shka} `maybe', as seen in (\ref{possiblemodal5}). Like the retrospective aspectual \textit{áahka} `just', speakers may treat the possible modal as an enclitic or an unbound element, though when used as an unbound element, it is always postposed after the matrix verb. This use of \textit{á'shka} is rarely observed in the corpus, and only appears in direct quotations, though it is common enough in spoken conversations. In (\ref{possiblemodal5}), the speakers are wondering aloud to themselves. Normally, almost every complete sentence in Mandan must be marked for the sex of the listener (i.e., a female listener or male listener). However, the possible modal seems to preclude the ability to have this allocutive agreement.

Examples of a question with a listener orientation and another no listener orientation appear in (\ref{directindirectquestions}) below. The first question in (\ref{directindirectquestions1}) below is directed to a specific male listener, and as such, it bears the male-oriented indicative enclitic =\textit{o'sh}. The same is not true of the question in (\ref{directindirectquestions2}), where the speaker is not addressing anyone, but wondering aloud `what time is it?' 

\begin{exe}

\item\label{directindirectquestions} Direct versus indirect questions

	\begin{xlist}
	
	\item\label{directindirectquestions1}
	\glll ótąąs hí'sha?\\
	o-tąą=s hi=o'sha\\
	pv.loc-\textnormal{how.many}=def \textnormal{arrive.there}=int.m\\
	\glt `what time is it? [asked to a male listener]' (\citeauthor{benson2000} p.c.)
	
	\item\label{directindirectquestions2}
	\glll ótąąs hí á'shka?\\
	o-tąą=s hi a'shka\\
	pv.loc-\textnormal{how.many}=def \textnormal{arrive.there} psbl\\
	\glt `what time is it? [asked to no one in particular]' (\citeauthor{benson2000} p.c.)
	\end{xlist}

\end{exe}

Consulting with a Mandan speaker reveals that the sentence in (\ref{directindirectquestions2}) can also mean `I wonder what time it is.' When used this way, \textit{á'shka} does not necessarily imply that the speaker is asking the question to anyone in particular, and with no listener selected, there can be no allocutive agreement marking on the clause. As such, this use of \textit{á'shka} may indicate something akin to impersonal speech or possibly even self-directed speech.
\newpage

\subsubsection{Visual evidential enclitic: =\textit{ishi}}
\largerpage
In the corpus, most of the data come from traditional narratives. As such, very little of it is first-hand information. However, when a figure in the narrative is speaking out loud, or in the case of language consultants speaking extemporaneously on a topic that is not a traditional narrative, the visual evidential =\textit{ishi} appears. This evidential is used to express that the information being reported come from personal, eye-witness testimony. Another use of this enclitic is to convey that some can tell that something is the case simply by looking at it, e.g., someone must be lost because they appear to be from the speaker's perspective. We can see examples of this evidential in (\ref{visualevidential}) below.

\begin{exe}

\item\label{visualevidential} Examples of the visual evidential enclitic =\textit{ishi}

	\begin{xlist}
	
	\item\label{visualevidential1}
	\glll ų́'sh keréeh\textbf{ishi}kere'sh\\
	ų'sh k-rEEh=\textbf{ishi}=krE=o'sh\\
	\textnormal{be.thus} vert-\textnormal{go.there}=\textbf{vis}=3pl=ind.m\\
	\glt `so, they clearly must have gone home' \citep[265]{hollow1973b}

	\item\label{visualevidential2}
	\glll mí'ti rúutakt ísuk\textbf{ishi}'re\\
	wį'\#ti ruutak=t i-suk=\textbf{ishi}=o're\\
	\textnormal{stone}\#\textnormal{dwell} \textnormal{be.far.away}=loc pv.dir-\textnormal{exit}=\textbf{vis}=ind.f\\
	\glt `he obviously must belong to the village over there' \citep[157]{hollow1973b}

	\item\label{visualevidential3}
	\glll kowóorooxikanashs ée ~ ~ ~ ~ ~ ~ ~ ~ ~ ~ ~ ~ ~ ~ ~ wáa'o'nix\textbf{ishi}'re\\
	ko-wooroo\#xik=rash=s ee ~ ~ ~ ~ ~ ~ ~ ~ ~ ~ ~ ~ ~ ~ ~ waa-o'=rįx=\textbf{ishi}=o're\\
	3poss.pers-\textnormal{husband}\#\textnormal{be.bad}=att=def dem.dist ~ ~ ~ ~ ~ ~ ~ ~ ~ ~ ~ ~ ~ ~ ~ neg-\textnormal{be}=neg=\textbf{vis}=ind.f\\
	\glt `it must not be her no-good husband over there after all' \citep[133]{hollow1973a}

	\item\label{visualevidential4}
	\glll wáati\textbf{shi}ka'sh\\
	waa-ti=\textbf{ishi}=ka=o'sh\\
	\textnormal{someone}-\textnormal{arrive.there}=\textbf{vis}=hab=ind.m\\
	\glt `someone must be coming here' \citep[142]{hollow1973a}

	\item\label{visualevidential5}
	\glll ée shanahkere, pt\'{ı̨}į \textbf{ishí}'sh\\
	ee shrąk=krE ptįį \textbf{ishi}=o'sh\\
	dem.dist \textnormal{be.around}=\textnormal{only} \textnormal{buffalo} \textbf{vis}=ind.m\\
	\glt `nothing but round things over there, they must be buffalo' \citep[212]{hollow1973b}

	
	
	\end{xlist}

\end{exe}

In examples (\ref{visualevidential1}) through (\ref{visualevidential4}), we can see that =\textit{ishi} serves to inform the listener that the speaker can attest to the veracity of the statement through visual evidence. In (\ref{visualevidential1}), for instance, the speaker conjectures that the other people have gone home due to the fact that a visual inspection reveals that they are no longer there. Similarly, in (\ref{visualevidential3}), the speaker can tell that the man she is looking at is not her relative's husband, as she can tell by looking at him.

Like several other enclitics, there is a free form, \textit{ishí}. There is no obvious Proto-Siouan origin for this evidential, so it is not clear whether this is a piece of inherited Proto-Siouan morphology or a Mandan innovation. Furthermore, internally reconstructing this formative is tenuous. One possible analysis is that it comes from Pre-Mandan *i-ši, where *ši is modern Mandan \textit{shí} `good', and *i is a preverb. This construction could have been used periphrasically to emphasize that a statement has been visually attested. Over time, this verb could have lost its internal morphological boundaries (i.e., *i-ši > **iši) and then become analyzed as simplex verb, like we see in (\ref{visualevidential5}). The Mandan verb complex is so elaborate that speakers could have begun to interpret this evidentiality-denoting verb as being a simple evidential enclitic, which is how =\textit{ishi} typically is realized in the corpus. This process is reminiscent of \citeauthor{chafe1999}'s (\citeyear[39]{chafe1999}) notion of florescence, which holds that particular features of a grammar may come to dominate the form that a language takes. In this sense, Mandan encodes so much information in its verb complex that elements that exist outside of it (e.g., auxiliary verbs) can come to be subsumed by it and reanalyzed as being part of the verb itself, rather than an auxiliary that accompanies a verb. Another possible origin is the Proto-Siouan locative *\v{s}i, and this locative became lexicalized and acquired verbal morphology in the form of the instrumental or directional preverb \textit{i}-.

\subsubsection{Quotative evidential enclitic: =\textit{ka'ehe}}

The corpus mostly consists of traditional narratives of the Mandan people. As such, most of the events that appear in the corpus are not first-hand information. One way for the speaker to indicate that the information being presented is reported information is to include the quotative evidential enclitic =\textit{ka'ehe}. This marker is transparently composed of the habitual =\textit{ka} and a prosodically weakened verb \textit{éehee} `say.'\footnote{The \textit{éehee} is so prosodically weak in the speech of most Mandan speakers that \citet[19]{kennard1936} transcribed it in superscript, e.g., $\langle$xama$'$kɛrɛka'\textsuperscript{ɛhɛ}$\rangle$ `they were small, it is said', which would be \textit{xamahkereka'ehe} in the present orthography.} Speakers recognize this element as meaning `say', and there are instances of it being realized with long vowels in precise speech.

We see similar morphology across Siouan. There are similarities with the Hidatsa reportative =\textit{rahee} \citep [194]{boyle2007}, which appears to be made up of Proto-Siouan *yą-hee, where *yą is some kind of topic marker or demonstrative, and *hee is the verb `say'. Crow marks quoted speech both morphologically and periphrastically, with the enclitic =\textit{hcheilu} being made up of the indirect causative \textit{hche} and the plural allomorph of the habitual aspectual enclitic =\textit{ilu} cliticizing onto the verb or by adding \textit{huuk} `they say' at the end of the statement \citep[397]{graczyk2007}. Assiniboine has two functionally equal quotative markers: =\textit{hųštá} and =\textit{káya}, the latter being a combination of either the durative =\textit{ka} or a distal demonstrative \textit{ká} and the verb \textit{yÁ} `say' \citep[357]{cumberland2005}. Biloxi likewise has a construction derived from `they say' plus the habitual aspectual enclitic, =\textit{éetu=xaa} \citep[189]{dorseyswanton1912}, though it also has \textit{yeke}, which is an inferential marker that fulfills a similar role.

Languages across the entire Siouan language family have dedicated morphology or a periphrastic construction to denote quoted speech. However, the composition of the quotative or reportative enclitics is quite varied. As such, we cannot conclusively say that there was a quotative marker in Proto-Siouan, but it is likely that we are seeing a case of parallel development, since the element most of these constructions have in common is some form of the word `say'.  \citet[12]{rankin2010} likewise notes the presence of quotative markers in most branches of Siouan, but he states that these markers ``represent nice examples of parallel development, but, as such, they do not constitute evidence for subgrouping.'' We can see examples of the quotative evidential enclitic in (\ref{quotativeevidentialenclitic}) below.

\begin{exe}

\item\label{quotativeevidentialenclitic} Examples of the quotative evidential enclitic =\textit{ka'ehe}

	\begin{xlist}
	
	\item\label{quotativeevidentialenclitic1}
	\glll Numá'k Máxanas réehak, Kinúma'kshis ~ ~ ~ ~ ~ ~ ~ ~ ~ ~ wá'xokaa náate\textbf{ka'eehee}ro'sh\\
	ruwą'k wąxrą=s rEEh=ak ki-ruwą'k\#shi=s ~ ~ ~ ~ ~ ~ ~ ~ ~ ~  wa'-xok=E rąątE=\textbf{ka'eehee}=o'sh\\
	\textnormal{man} \textnormal{one}=def \textnormal{go.there}=ds mid-\textnormal{man}\#\textnormal{be.good}=def ~ ~ ~ ~ ~ ~ ~ ~ ~ ~ ins.prce-\textnormal{be.idle}=sv \textnormal{stand.up}=\textbf{quot}=ind.m\\
	\glt `with Lone Man having gone, First Creator jumped up, it is said' \citep[2]{hollow1973a}

	\item\label{quotativeevidentialenclitic2}
	\glll ``háu, wá'to'sh,'' éereh\textbf{ka'ehe}sįh\\
	hau w'-at=o'sh ee-reh=\textbf{ka'ehe}=sįh\\
	\textnormal{yes} 1poss-\textnormal{father}=ind.m pv-\textnormal{think}=\textbf{quot}=ints\\
	\glt ```yes, it is my father,'' he thought, it is said' \citep[5]{hollow1973a}

	\item\label{quotativeevidentialenclitic3}
	\glll ``hóo,'' éehe\textbf{ka'ehe}s\\
	hoo ee-he=ka'ehe=s\\
	\textnormal{yes} pv-\textbf{say}=quot=def\\
	\glt ```yes,'' they said, it is said' \citep[49]{hollow1973a}

	\item\label{quotativeevidentialenclitic4}
	\glll ų́'shkana áahi\textbf{ka'ehe}roo\\
	ų'sh=ka=rą aa-hi=\textbf{ka'ehe}=oo\\
	\textnormal{be.thus}=hab=top pv.tr-\textnormal{arrive.there}=\textbf{quot}=dem.mid\\
	\glt `so, he took her there, it is said, then' \citep[70]{hollow1973a}

	\item\label{quotativeevidentialenclitic5}
	\glll Rá'puse íkarapątkere\textbf{ka'eehee}rak, ~ ~ ~ ~ ~ ~  wakárahash waréeho'sh\\
	ra'-pus=E i-k-rapąt=krE=\textbf{ka'eehee}=ak ~ ~ ~ ~ ~ ~  wa-ka-ra-hash wa-rEEh=o'sh\\
	ins.heat-\textnormal{be.marked}=sv pv.ins-mid-\textnormal{increase}=3pl=\textbf{quot}=ds ~ ~ ~ ~ ~ ~  1a-incp-ins.foot-\textnormal{slaughter} 1a-\textnormal{go.there}=ind.m\\
	\glt `Speckled Arrow's [birds] have increased, it is said, and I am going to slaughter them' \citep[148]{hollow1973a}

	\item\label{quotativeevidentialenclitic6}
	\glll ráahini tamáahs kirúshe\textbf{ka'ehe}\\
	rEEh=rį ta-wąąh=s ki-ru-shE=\textbf{ka'ehe}\\
	\textnormal{go.there}=ss 3poss.al-\textnormal{arrow}=def vert-ins.hand-\textnormal{hold}=\textbf{quot}\\
	\glt `he went and took back his arrows, it is said' \citep[16]{hollow1973a}

	\item\label{quotativeevidentialenclitic7}
	\glll	máaxtik ókshuko'na óshiriihaa ~ ~ ~  ptéhkere\textbf{ka'ehe}\\
	wąąxtik o-kshuk-o'=rą o-shriih=E=$\varnothing$ ~ ~ ~  ptEh=krE=\textbf{ka'ehe}\\
	\textnormal{rabbit} pv.irr-\textnormal{be.narrow}-\textnormal{be}=top pv.loc-\textnormal{be.scattered}=sv=cont ~ ~ ~  \textnormal{run}=3pl=\textbf{quot}\\
	\glt `it was the cottontails who were running scattered, it is said' \citep[17]{hollow1973a}
	
	\end{xlist}

\end{exe}

The majority of instances of =\textit{ka'ehe} in the corpus involve no additional enclitic material after the quotative itself. We see this behavior in (\ref{quotativeevidentialenclitic6}) and (\ref{quotativeevidentialenclitic7}). When the quotative enclitic is not clause-final, it is because an allocutive declarative marker is following it, e.g., the =\textit{o'sh} we see in (\ref{quotativeevidentialenclitic1}). The quotative is also followed by the bound medial demonstrative =\textit{oo} to emphasize some place or time in which the quoted event happened, like in (\ref{quotativeevidentialenclitic4}), or it may be followed by the definite =\textit{s} like in (\ref{quotativeevidentialenclitic3}) or intensive =\textit{sįh} like in (\ref{quotativeevidentialenclitic2}) to highlight that, although the speaker is reporting information they have not personally witnessed, they can attest to the veracity of the claim nonetheless. Rarely, =\textit{ka'ehe} appears on non-matrix verbs, as we can see in (\ref{quotativeevidentialenclitic5}), where the veracity of the initial clause is hearsay (i.e. the number of birds that Speckled Arrow has has increased), but the proposition in the matrix clause is not hearsay (i.e., `I am going to slaughter them').

Like the verb \textit{éehee} `say', the realization of the quotative varies somewhat. When word-final, the last vowel in the stem is short, but when taking additional morphology, the vowel is long. This difference mirrors the alternation between the speakers treating the verb `say' as having an underlying short vowel or  a long vowel, i.e., /ee-he $\sim$ ee-hee/.

\subsubsection{Potential modal enclitic: =\textit{kt}}

The potential modal enclitic =\textit{kt} is one of the most robustly-attested pieces of postverbal morphology across the Siouan language family. This Proto-Siouan modal enclitic *ktE has reflexes in Missouri Valley in the desideritive =\textit{hdi} \citep[194]{park2012}, the potential \textit{ktA} in Lakota \citep[821]{ullrich2011}, the future -\textit{kje} in Hoocąk \citep[54]{helmbrechtlehmann2006}, the potential -\textit{tte} in Quapaw \citep[484]{rankin2005}, the optative -\textit{tE} in Biloxi \citep[31]{einaudi1976}, and the ability modal and future marker \textit{te} in Yuchi \citep[291]{linn2000}.

Mandan shares much in common with the various uses of the reflexes of *ktE in that =\textit{kt} is extremely versatile as a modal. Its most common usage is to provide a future reading for an event or state, and we can see this use in the data in (\ref{potentialfuture}) below.

\begin{exe}

\item\label{potentialfuture} Future readings for =\textit{kt}

	\begin{xlist}
	
	\item\label{potentialfuture1}
	\glll míiptos rushé\textbf{kt}o'sh\\
	wįįpto=s ru-shE=\textbf{kt}=o'sh\\
	\textnormal{ball}=def ins.hand-\textnormal{hold}=\textbf{pot}=ind.m\\
	\glt `she will take the ball' \citep[86]{hollow1973a}

	\item\label{potentialfuture3}
	\glll mí'shak hą'khaa íreehrahere\textbf{kt}o'sh\\
	w'\~~-ishak hą'k=haa i-rEEh\#ra-hrE=\textbf{kt}=o'sh\\
	1poss-pro stnd.pos=loc pv.ins-\textnormal{go.there}\#2a-caus=\textbf{pot}=ind.m\\
	\glt `you will put it on after me' \citep[109]{hollow1973a}

	\item\label{potentialfuture2}
	\glll tewé ų́ųte pask\'{ı̨}įhki ~ ~ ~ ~ ~ ~ ~ ~ ~ ~ ~ ~ ~ ~ ~ ~ ~ ~ tapt\'{ı̨}į\textbf{kt}o'sh\\
	t-we ųųt=E pa-skįįh=ki ~ ~ ~ ~ ~ ~ ~ ~ ~ ~ ~ ~ ~ ~ ~ ~ ~ ~ ta-ptįį=\textbf{kt}=o'sh\\
	wh-indf \textnormal{be.first}=sv ins.push-\textnormal{cut.open}=cond ~ ~ ~ ~ ~ ~ ~ ~ ~ ~ ~ ~ ~ ~ ~ ~ ~ ~ al-\textnormal{buffalo}=\textbf{pot}=ind.m\\
	\glt `whoever opens it first, it will be his buffalo' \citep[6]{hollow1973b}

	\item\label{potentialfuture4}
	\glll wáa'iminirats ą́ąwe, miníike, ~ ~ ~ ~ ~ ~ ~ ~ ~ ~ raká'\textbf{kt}o'sh\\
	waa-i-w-rį-rat=s ąąwe wį-rįįk=E ~ ~ ~ ~ ~ ~ ~ ~ ~ ~  ra-ka'=\textbf{kt}=o'sh\\
	nom-pv.ins-1a-2s-\textnormal{promise}=def \textnormal{all} 1poss-\textnormal{son}=sv ~ ~ ~ ~ ~ ~ ~ ~ ~ ~  2a-\textnormal{possess}=\textbf{pot}=ind.m\\
	\glt `you will have everything I promised you, my son' \citep[192]{hollow1973a}
	
	\end{xlist}

\end{exe}

Another common use of =\textit{kt} is to express permissive deontic modality and matches up with such uses of English `can'. The =\textit{kt} in (\ref{potentialdeontic}), for example, indicates that someone has permission to do something.



\begin{exe}

\item\label{potentialdeontic} Permissive deontic readings for =\textit{kt}

	\begin{xlist}
	
	\item\label{potentialdeontic1}
	\glll manakíkų'te\textbf{kt}o'sh, ókaxiipe\\
	w-rą-kikų'tE=\textbf{kt}=o'sh o-ka-xiip=E\\
	1s-2a-\textnormal{help}=\textbf{pot}=ind.m pv.irr-ins.frce-\textnormal{skin}=sv\\
	\glt `you can help me skin it' \citep[41]{hollow1973a}

	\item\label{potentialdeontic2}
	\glll wáa'aahuuki órara'\textbf{kt}o're\\
	waa-aa-huu=ki o-ra-ra'=\textbf{kt}=o're\\
	part-pv.tr-\textnormal{come.here}=cond pv.loc-2a-\textnormal{make.fire}=\textbf{pot}=ind.f\\
	\glt `you can make a fire if he brings some' \citep[120]{hollow1973a}

	\item\label{potentialdeontic3}
	\glll wíipe íwasehki rakéeka'ni ~ ~ ~ ~ ~ ~ ~ ~ ~ ~ ~ ~ ~ ~ ~  raráahinis\textbf{t}o'sh\\
	wiip(E) i-wa-sek=ki ra-keeka'=rį ~ ~ ~ ~ ~ ~ ~ ~ ~ ~ ~ ~ ~ ~ ~  ra-rEEh=rįt=\textbf{t}=o'sh\\
	\textnormal{cornball} pv.ins-1a-\textnormal{make}=cond 2a-\textnormal{have}=ss ~ ~ ~ ~ ~ ~ ~ ~ ~ ~ ~ ~ ~ ~ ~  2a-\textnormal{go.there}=2pl=\textbf{pot}=ind.m\\
	\glt `if I make cornballs, you can have them and go' \citep[268]{hollow1973b}

	\item\label{potentialdeontic4}
	\glll ómaniitaa miní tú téki ~ ~ ~ ~ ~ ~ ~ ~  rah\'{ı̨}į\textbf{kt}o'sh\\
	o-wa-rįį=taa wrį tu tE=ki ~ ~ ~ ~ ~ ~ ~ ~  ra-hįį=\textbf{kt}=o'sh\\
	pv.irr-1a-\textnormal{walk}=loc \textnormal{water} \textnormal{be.some} \textsc{stand.upright}=cond ~ ~ ~ ~ ~ ~ ~ ~ 2a-\textnormal{drink}=\textbf{pot}=ind.m\\
	\glt `there will be some water where I walk and you can drink it' \citep[114]{hollow1973a}
	
	\end{xlist}

\end{exe}

Another kind of necessitative deontic modality that =\textit{kt} can express is one that conveys an action that one `should' or `should not' engage in, per some set of rules, norms, and the like. We can see examples of this reading of =\textit{kt} in (\ref{necessitativedeontic}) below.

\begin{exe}

\item\label{necessitativedeontic} Necessitative deontic readings for =\textit{kt}

	\begin{xlist}
	
	\item\label{necessitativedeontic1}
	\glll wáa'oxikanashe wáarakaweh\textbf{t}o'sh\\
	waa-o-xik=rąsh=E waa-ra-ka-weh=\textbf{t}=o'sh\\
	nom-pv.irr-\textnormal{be.bad}=att=sv neg-2a-ins.frce-\textnormal{pick}=\textbf{pot}=ind.m\\
	\glt `you should not pick the ones that are no good' \citep[216]{hollow1973a}

	\item\label{necessitativedeontic2}
	\glll hįhé, mí'shak wawákereeh\textbf{t}o'sh\\
	hįhe w'\~~-ishak wa$\sim$wa-k-rEEh=\textbf{t}=o'sh\\
	\textnormal{well} 1s-pro aug$\sim$1a-vert-\textnormal{go.there}=\textbf{pot}=ind.m\\
	\glt `well, I should really go home' \citep[217]{hollow1973a}

	\item\label{necessitativedeontic3}
	\glll kúhki ní'o'na raróora  ~ ~ ~ ~ ~ ~ ~ ~ ~ ~ namáake\textbf{kt}o'sh\\
	kuh=ki rį-o'=rą  ra-roo=E=$\varnothing$  ~ ~ ~ ~ ~ ~ ~ ~ ~ ~ ra-wąąkE=\textbf{kt}=o'sh\\
	\textnormal{come.back.here}=cond 2s-\textnormal{be}=top 2a-\textnormal{talk}=sv=cont ~ ~ ~ ~ ~ ~ ~ ~ ~ ~ 2a-\textnormal{lie}=\textbf{pot}=ind.m\\
	\glt `when she comes back, you are the one who should be talking' \citep[221]{hollow1973a}

	\item\label{necessitativedeontic4}
	\glll wáa'inak rá'tere ~ ~ ~ ~ ~ ~ ~ ~ ~ ~ ~ ~ ~ ~ ~ ~ ~ ~ ~ ~  rapkáhųtinis\textbf{t}o'sh\\
	waa-irąk r'-at=re ~ ~ ~ ~ ~ ~ ~ ~ ~ ~ ~ ~ ~ ~ ~ ~ ~ ~ ~ ~  ra-k-pa-hųt=rįt=\textbf{t}=o'sh\\
	neg-\textnormal{again} 2poss-\textnormal{father}=dem.prox
 ~ ~ ~ ~ ~ ~ ~ ~ ~ ~ ~ ~ ~ ~ ~ ~ ~ ~ ~ ~  2a-mid-ins.push-\textnormal{make.up.with}=2pl=\textbf{pot}=ind.m\\
	\glt `you (pl.) should never make up with your father' \citep[194]{hollow1973a}
	
	\end{xlist}

\end{exe}

As the data in (\ref{necessitativedeontic1}) show, when the potential enclitic is present, a negative enclitic can be omitted so long as the negative prefix is present. The potential very rarely appears in negated propositions. 

\subsubsubsection{Allomorph /=t/}

Mandan is the only Siouan language that has reduced Proto-Siouan *ktE to a consonant cluster without a syllabic nucleus. We never see it appear in word-final position; there is always some other enclitic following it, e.g., indicative allocutionary markers. Furthermore, this enclitic becomes a simple /=t/ when cliticizing onto a stem that ends in a consonant that is not /ʔ/. Coda glottal stops are treated as being part of the syllable nucleus, and thus do not count as a consonant for the purpose of consonant cluster reduction. We can see this allomorphy at work in (\ref{potentialt}) below.

\begin{exe}

\item\label{potentialt} Examples of the allomorph =\textit{t}

	\begin{xlist}
	

	\item\label{potentialt1}
	\glll nukeréehki warárus\textbf{t}o'sh\\
	rų-k-rEEh=ki wa-ra-rut=\textbf{t}=o'sh\\
	1a.pl-vert-\textnormal{go.there}=cond unsp-2a-\textnormal{eat}=\textbf{pot}=ind.m\\
	\glt `let's go home and then you will eat' \citep[87]{hollow1973a}

	\item\label{potentialt2}
	\glll hiré nuxką́hinis\textbf{t}o'sh\\
	hire rų-kxąh=rįt=\textbf{t}=o'sh\\
	\textnormal{now} 1a.pl-\textnormal{move}=2pl=\textbf{pot}=ind.m\\
	\glt `we will take off now' \citep[82]{hollow1973b}

	\item\label{potentialt3}
	\glll máatki nuxką́h\textbf{t}o'sh\\
	wąątki rų-kxąh=\textbf{t}=o'sh\\
	\textnormal{tomorrow} 1a.pl-\textnormal{move}=\textbf{pot}=ind.m\\
	\glt `we will move tomorrow' \citep[48]{hollow1973b}

	\item\label{potentialt4}
	\glll sé waharáa minikų́'\textbf{kt}o'sh\\
	se wa-hrE w-rį-kų'=\textbf{kt}=o'sh\\
	\textnormal{be.red} 1a-caus 1a-2s-\textnormal{give}=\textbf{pot}=ind.m\\
	\glt `I might make it red for you' \citep[454]{hollow1970}

	\end{xlist}
	
\end{exe}

We can see in the data above in (\ref{potentialt}) that the potential modal enclitic is realized as =\textit{t} instead of =\textit{kt} when following a consonant-final stem. Again, the one exception to the rule that the \textit{=t} allomorph occurs after a consonant-final stem is a stem that ends in a glottal stop, as glottal stops are not treated phonologically as a consonant. Such glottal stops are considered part of the syllable nucleus instead. We see this behavior in (\ref{potentialt4}), where the verb \textit{kų́'} `give' is followed by =\textit{kt} and not =\textit{t}. Another property that the =\textit{t} allomorph has is that it triggers lenition in /t/-final stems, turning /t/ to [s] to prevent a surface [tt] cluster. This process has been described previously in \sectref{geminatedissimilation}.

\subsubsubsection{Allomorphs /=kte/ and /=te/}

The potential enclitic typically has the phonetic shape /=kt/ in Mandan, as discussed above. However, in certain constructions, we see remnants of the full Proto-Siouan *ktE, where a vowel will appear only before one particular enclitic. The =\textit{kte} allomorph is exclusively found before a single formative: the different-subject switch-reference marker =\textit{ak}. A consonant-final stem should yield [=ak] for this marker, but what we see instead is [=k], the allomorph used with stems that end in short vowels, and we observe /=kte/ or /=te/ instead. We can see examples of the /=kte/ and /=te/ allomorphs in (\ref{potentialkte}) below.



\begin{exe}

\item\label{potentialkte} Examples of the allomorph =\textit{kte}

	\begin{xlist}
	
	\item\label{potentialkte1}
	\glll ímate íkxąhkere'sh, ~ ~ ~ ~ ~ ~ ~ ~ ~ ~ ~ ~ ~ ~ ~ ~ ~ ~ ~ ~   ówaxohkere\textbf{kte}k\\
	i-wą-te i-kxąh=krE=o'sh ~ ~ ~ ~ ~ ~ ~ ~ ~ ~ ~ ~ ~ ~ ~ ~ ~ ~ ~ ~  o-wa-xok=krE=\textbf{kte}=ak\\
	pv.ins-1s-\textnormal{stand} pv.ins-\textnormal{laugh}=3pl=ind.m  ~ ~ ~ ~ ~ ~ ~ ~ ~ ~ ~ ~ ~ ~ ~ ~ ~ ~ ~ ~  pv.loc-1a-\textnormal{swallow}=3pl=\textbf{pot}=ds\\
	\glt `they made fun of me, so I will swallow them' \citep[149]{hollow1973a}

	\item\label{potentialkte2}
	\glll mí'mami'hs kawásųkhereki, ~  kishí\textbf{kte}k...\\
	w'\~~-i-wą-wį'h=s ka-wa'-sųk\#hrE=ki ~  ki-shi=\textbf{kte}=ak\\
	1poss-pv.ins-1s-\textnormal{robe}=def ins.frce-ins.prce-\textnormal{be.rinsed}\#caus=cond ~  mid-\textnormal{be.good}=\textbf{pot}=ds\\
	\glt `if my robe is rinsed, it must be all right...' \citep[17]{hollow1973a}

	\item\label{potentialkte3}
	\glll ha\textbf{kté}k súhkereseet tamáah ~ ~ ~ ~ ~  tóps ká'hereroomako'sh\\
	ha=\textbf{kte}=ak suk=krE=s=ee=t ta-wąąh ~ ~ ~ ~ ~  top=s ka'\#hrE=oowąk=o'sh\\
	prov=\textbf{pot}=ds \textnormal{child}=3pl=def=dem.dist=loc 3poss.al-\textnormal{arrow} ~ ~ ~ ~ ~ \textnormal{four}=def \textnormal{possess}\#caus-narr-ind.m\\
	\glt `so, he gave his four arrows to the children' \citep[31]{hollow1973a}

	\item\label{potentialkte4}
	\glll ą́ąwena wáa'omik\textbf{te}k íraseko'sh, ~ ~ ~ ~   Kinúma'kshi\\
	ąąwe=rą waa-o-wįk=\textbf{te}=ak i-ra-sek=o'sh ~ ~ ~ ~  ki-ruwą'k\#shi\\
	\textnormal{all}=top nom-pv.irr-\textnormal{be.none}=\textbf{pot}=ds pv.ins-2a-\textnormal{make}=ind.m ~ ~ ~ ~  mid-\textnormal{man}\#\textnormal{be.good}\\
	\glt `everything you do will be nothing, Royal Chief' \citep[47]{hollow1973b}
	
	\end{xlist}

\end{exe}

While /=kt/ is generally the expected reflex of Proto-Siouan *ktE, these data demonstrate that there is a small vestige of the fully-realized Proto-Siouan *ktE in Mandan when accompanied by the different-subject switch-reference marker. Furthermore, this =\textit{kte} allomorph is realized as =\textit{te} when cliticized onto a stem the ends in a consonant, as we see in (\ref{potentialkte4}). It is unclear if the loss of /k/ in *ktE was a regular process in Proto-Siouan or a Mandan-only innovation.

\subsubsubsection{Allomorph /=kti/}\label{allomorphkti}

As with the allomorph =\textit{kte}, we see additional evidence of residual retention of the ablaut vowel in *ktE into modern Mandan with =\textit{kti}. Mandan only possesses two ablaut grades: the e-grade and a-grade. The Dakotan branch plus Biloxi have a third: the i/į-grade. This third grade has a very limited distribution, with it being triggered by just the intensifier =\textit{xti} in Biloxi and the future \textit{ktA} and conjunction \textit{na} `and' in Dakotan \citep[29]{jones1983b}. One of the only relics of the i/į-grade ablaut in Mandan appears in the allomorph =\textit{kti}, which itself only occurs before the conditional complementizer =\textit{ki}. We see examples of this /=kti/ allomorph in (\ref{potentialkti}) below.

\largerpage
\begin{exe}

\item\label{potentialkti} Examples of the allomorph =\textit{kti}

	\begin{xlist}
	
	\item\label{potentialkti1}
	\glll ó'harani kishiniih\textbf{kti}ki, kóoxą'taanashini ~ ~ ~ ~ ~ ~ ~ ~ ~ ~ wahų́harani má'kekereroomako'sh\\
	o'\#hrE=rį ki-shrįįh=\textbf{kti}=ki, kooxą'tE=rąsh=rį ~ ~ ~ ~ ~ ~ ~ ~ ~ ~ wa-hų\#hrE=rį wą'kE=krE=oowąk=o'sh\\
	\textnormal{be}\#caus=ss mid-\textnormal{be.cold}=\textbf{pot}=cond \textnormal{corn}=att=ss ~ ~ ~ ~ ~ ~ ~ ~ ` ~ unsp-\textnormal{be.many}\#caus=ss \textnormal{lie}.aux=3pl=narr=ind.m\\
	\glt `when it got cold from there, they had lots of corn and they were there' \citep[91]{hollow1973b}

	\item\label{potentialkti2}
	\glll xamáhe nurúshe\textbf{kti}ki, ~ ~ ~ ~ ~ ~ ~ ~ ~ ~ ~ ~ ~ ~ ~  íhehka'sh\\
	xwąh=E rų-ru-shE=\textbf{kti}=ki ~ ~ ~ ~ ~ ~ ~ ~ ~ ~ ~ ~ ~ ~ ~  i-hek=ka=o'sh\\
	\textnormal{be.small}=sv 1a.pl-ins.hand-\textnormal{hold}=\textbf{pot}=cond ~ ~ ~ ~ ~ ~ ~ ~ ~ ~ ~ ~ ~ ~ ~  pv.ins-\textnormal{know}=hab=ind.m\\
	\glt `when we take a little, he always knows' \citep[116]{hollow1973b}

	\item\label{potentialkti3}
	\glll óparashtaa ishąątaa nákini ~ ~ ~ ~ ~ ~ ~ ~ ~ ~ ~ ~ ~ ~ ~ rokirúhere\textbf{kti}ki, nuréehka'sh\\
	o-prash=taa ishąą=taa rąk=rį ~ ~ ~ ~ ~ ~ ~ ~ ~ ~ ~ ~ ~ ~ ~  ro-kru\#hrE=\textbf{kti}=ki rų-rEEh=ka=o'sh\\
	pv.irr-\textnormal{be.pointed}=loc \textnormal{across}=loc sit.pos=ss ~ ~ ~ ~ ~ ~ ~ ~ ~ ~ ~ ~ ~ ~ ~ 1s.pl-\textnormal{be.called}\#caus=\textbf{pot}=cond 1a.pl-\textnormal{go.there}=hab=ind.m\\
	\glt `we always go when he calls us across the ridge there' \citep[151]{hollow1973b}

	\item\label{potentialkti4}
	\glll wará're íkarexwaheré\textbf{kti}ki, ~ ~ ~ ~ ~ ~ ~ ~ ~ ~ máamihka'sh\\
	wra'=E i-ka-rex\#wa-hrE=\textbf{kti}=ki ~ ~ ~ ~ ~ ~ ~ ~ ~ ~ waa-wįk=ka=o'sh\\
	\textnormal{fire}=sv pv.ins-ins.frce-\textnormal{glisten}\#1a-caus=\textbf{pot}=cond ~ ~ ~ ~ ~ ~ ~ ~ ~ ~  \textnormal{someone}-\textnormal{be.none}=hab=ind.m\\
	\glt `there was never anyone there when I lit the fire' \citep[203]{hollow1973b}
	
	\end{xlist}

\end{exe}

Unlike =\textit{kt} or =\textit{kte}, =\textit{kti} does not lose its /k/ when added onto a consonant-final stem, as we see in (\ref{potentialkti1}), where the verb \textit{kishiníih} `get cold' ends in an /h/. We should expect to lose the /k/ to avoid a CCC cluster, like in (\ref{potentialt1}) through (\ref{potentialt3}) above, but that is not the case. It is not clear if this is a rule restricted to /hkt/ sequences being permissible, or if this is an example of hyperarticulation on the part of the consultant.

\subsubsection{Indirect evidential enclitic: =\textit{oote}}\label{encliticOOTE}

Mandan has several evidentials that deal with marking first-hand versus second-hand knowledge. The indirect evidential =\textit{oote} marks a statement as being true through inference. \citet[474]{hollow1970} describes this as a perfective aspect marker, while \citet[17]{kennard1936} calls it both a completive and evidential marker. It is more appropriate to call this marker an evidential, as it conveys information that the speaker can infer to be the case. As such, the speaker has indirect knowledge of the event that has occurred, rather than direct, first-hand information. We can see this behavior for =\textit{oote} in (\ref{evidentialenclitic}) below.

\begin{exe}

\item\label{evidentialenclitic} Examples of the indirect evidential enclitic =\textit{oote}

	\begin{xlist}
	
	\item\label{evidentialenclitic1} 
	\glll nú'kas má'kah\textbf{oote} ~ ~ ~ ~ ~ ~ ~ ~ ~ ~ ípke'sh\\
	r'-ųųka=s wą'kah=\textbf{ootE} ~ ~ ~ ~ ~ ~ ~ ~ ~ ~  i-pke=o'sh\\
	2poss-\textnormal{male's.older.brother}=def \textnormal{lie}.aux.hab=\textbf{evid} ~ ~ ~ ~ ~ ~ ~ ~ ~ ~ pv.ins-\textnormal{smell}=ind.m\\
	\glt `it smells like your brother must be here' \citep[143]{hollow1973a}

	\item\label{evidentialenclitic2} 
	\glll téehąxte wahaná'r\textbf{oote}'sh\\
	teehą-xte wa-hrą'=\textbf{ootE}=o'sh\\
	\textnormal{be.far.away}-aug 1a-\textnormal{sleep}=\textbf{evid}=ind.m\\
	\glt `I must have slept for a really long time' \citep[145]{hollow1973a}

	\item\label{evidentialenclitic3} 
	\glll ``í'aakitaa áareehkerer\textbf{oota}'t'' érehini...\\
	i-aaki=taa aa-rEEh=krE=\textbf{ootE}=ą't e-reh=rį\\
	pv.dir-\textnormal{be.above}=loc pv.tr-\textnormal{go.there}=3pl=\textbf{evid}=hyp pv-\textnormal{think}=ss\\
	\glt `he thought ``they must have taken him upward'' and...' \citep[172]{hollow1973a}

	\item\label{evidentialenclitic4}
	\glll órataxe ér\textbf{oote}, húurąmi, ~   minísweeruts híroomako'sh, ~ ~ ~ ~ ~ ~ ~ ~ ~ ~ ~ ~ súhkereseetaa\\
	o-ra-tax=E E=\textbf{ootE} huu=awį ~   wrįs\#ee\#rut=s hi=oowąk=o'sh ~ ~ ~ ~ ~ ~ ~ ~ ~ ~ ~ ~  suk=krE=s=ee=taa\\
	pv.irr-ins.mth-\textnormal{make.loud.noise}=sv \textnormal{hear}=\textbf{evid} \textnormal{come.here}=cont ~  \textnormal{horse}\#\textnormal{feces}\#\textnormal{eat}=def \textnormal{arrive.there}=narr=ind.m ~ ~ ~ ~ ~ ~ ~ ~ ~ ~ ~ ~ \textnormal{child}=3pl=def=dem.dist=loc\\
	\glt `the dog must have heard their cries coming along to where the kids were' \citep[180]{hollow1973a}
	
	\end{xlist}

\end{exe}

No previous researcher notes that the final vowel in /=ootE/ is an ablaut vowel, as it rarely precedes ablaut-inducing enclitics in the corpus. This enclitic appears to have evolved from the medial demonstrative \textit{oo} plus the Proto-Siouan stem augment *-tE. We can see in (\ref{evidentialenclitic3}) that the conditional complementizer =\textit{ą't} triggers ablaut. The deleted vowel from =\textit{ą't} leaves a glottal stop to constrict the ablaut vowel [aː] to [a] to avoid a tautosyllabic [aːʔ] sequence, but the final vowel in =\textit{oote} otherwise behaves as expected for an ablaut vowel. Another piece of evidence of this vowel being an ablaut vowel is the fact that there is a special allomorph of =\textit{oote} that only appears when followed by the conditional =\textit{ki}, as shown in (\ref{igradeablautwithoote}) below, making this another relic of \textit{i}-grade ablaut from Proto-Siouan.

\newpage
\begin{exe}

\item\label{igradeablautwithoote} Examples of the allomorph =\textit{ooti}

	\begin{xlist}
	
	\item\label{igradeablautwithoote1}
	\glll manakų́'r\textbf{ooti}ki, róowa'oshi hų́ ~ ~ ~ ~ ~ ~ ~ ~ ~   íminirats óraka'ro'sh\\
	w-rą-kų'=\textbf{ooti}=ki roo\#wa-o-shi hų ~ ~ ~ ~ ~ ~ ~ ~ ~  i-w-rį-rat=ss o-ra-ka'=o'sh\\
	1s-2a-\textnormal{give}=\textbf{evid}=cond \textnormal{speak}\#1a-pv.irr-\textnormal{be.good} \textnormal{many} ~ ~ ~ ~ ~ ~ ~ ~ ~  pv.ins-1a-2s-\textnormal{promise}=def pv.irr-2a-\textnormal{possess}=ind.m\\
	\glt `when you give it to me, you will have many promises that I made to you' \citep[191]{hollow1973a}

	\item\label{igradeablautwithoote2}
	\glll míih\textbf{ooti}ki, rétaa míiptos ~ ~ ~ ~ ~ ~ ~ ~ ~ ~ ~ ~ ~ ~ ~ rushékto'sh\\
	wįįh=\textbf{ooti}=ki re=taa wįįpto=s ~ ~ ~ ~ ~ ~ ~ ~ ~ ~ ~ ~ ~ ~ ~   ru-shE=kt=o'sh\\
	\textnormal{woman}=\textbf{evid}=cond dem.prox=loc \textnormal{ball}=def ~ ~ ~ ~ ~ ~ ~ ~ ~ ~ ~ ~ ~ ~ ~  ins.hand-\textnormal{hold}=pot=ind.m\\
	\glt `if it is a girl, I will take this ball right here' \citep[86]{hollow1973a}
	
	\end{xlist}

\end{exe}

While most of the data above reflects an action that happened in the past, the completion of the action is not what is being accentuated, but rather that the speaker is expressing that the event has apparently or seemingly happened. Often, this enclitic is translated as `must' or `must have' in the corpus, which reflects that the speaker is marking some inferential knowledge. This can also reflect first-hand information where the speaker is witnessing something or someone unexpectedly, as we can see in (\ref{evidentialunexpected}) below.

\begin{exe}

\item\label{evidentialunexpected} Examples of marking unexpected events
	
	\begin{xlist}
	
	\item\label{evidentialunexpected1}
	\glll ní'o'r\textbf{oote}'sh\\
	rį-o'=\textbf{ootE}=o'sh\\
	2s-\textnormal{be}=\textbf{evid}=ind.m\\
	\glt `it is you' (Delores Sand p.c.)

	\item\label{evidentialunexpected2}
	\glll ní'o'r\textbf{oote}'re\\
	rį-o'=\textbf{ootE}=o're\\
	2s-\textnormal{be}=\textbf{evid}=ind.f\\
	\glt `it is you' (Delores Sand p.c.)	
	
	\end{xlist}

\end{exe}

The data above are a kind of greeting in Mandan, where the speaker is expressing that they can tell it is the listener that they are speaking to. It is not necessarily that they are surprised to see the speaker, but that they may not initially have been sure who it was going to be. A true mirative can be expressed with the =\textit{oomak} enclitic, as described in \sectref{SubNARR} below.

\subsubsection{Narrative evidential enclitic: =\textit{oomak}}\label{SubNARR}

This enclitic has traditionally been referred to in Mandan literature as the narrative past marker by both \citet[18]{kennard1936} and \citet[474]{hollow1970}. They both describe =\textit{oomak} as being used to describe events in the distant past, and it is seen extensively in traditional narratives. However, Mandan and most other Siouan languages are not truly tense-marking languages. As such, what this enclitic marks is that the speaker is privy to second- or third-hand knowledge of the event being described. We can see examples of this enclitic in (\ref{narrativeevidentialenclitic}) below.

\begin{exe}

\item\label{narrativeevidentialenclitic} Examples of the narrative evidential enclitic \textit{=oomak}

	\begin{xlist}

	\item\label{narrativeevidentialenclitic2}
	\glll Numá'k Máxana níirąmi \'{ı̨}'kahek\textbf{oomak}sįh\\
	ruwą'k wąxrą rįį=awį į'-ka-hek=\textbf{oowąk}=sįh\\
	\textnormal{man} \textnormal{one} \textnormal{walk}=cont pv.rflx-incp-\textnormal{know}=\textbf{narr}=ints\\
	\glt `Lone Man suddenly became aware of himself while walking along' \citep[5]{hollow1973a}
	
	\item\label{narrativeevidentialenclitic1}
	\glll Numá'k Máxana éeheni Kinúma'kshi ~ ~ ~ ~ ~ ~ ~ ~ ~ ~ ~ ~ íkirookerer\textbf{oomak}o'sh, kotewé ~ ~ ~ ~ ~ ~ ~ ~ ~ ~ ~ óratoore\\
	ruwą'k wąxrą ee-he=rį ki-ruwą'k\#shi ~ ~ ~ ~ ~ ~ ~ ~ ~ ~ ~ ~ i-ki-roo=krE=\textbf{oowąk}=o'sh ko-t-we ~ ~ ~ ~ ~ ~ ~ ~ ~ ~ ~ o-ratoo=E\\
	\textnormal{man} \textnormal{one} pv-\textnormal{say}=ss mid-\textnormal{man}\#\textnormal{be.good} ~ ~ ~ ~ ~ ~ ~ ~ ~ ~ ~ ~ pv.ins-mid-\textnormal{talk}=3pl=\textbf{narr}=ind.m rel-wh-indf ~ ~ ~ ~ ~ ~ ~ ~ ~ ~ ~ pv.irr-\textnormal{be.mature}=sv\\
	\glt `Lone Man and First Creator argued about which one was older' \citep[1]{hollow1973a}

	\item\label{narrativeevidentialenclitic3}
	\glll Kawóoxohkas óti óo ~ ~ ~ ~ ~ ~ ~  ó'r\textbf{oomak}s\\
	ka-wV-o-xok=ka=s o-ti oo ~ ~ ~ ~ ~ ~ ~ o'=\textbf{oowąk}=s\\
	agt-unsp-pv.loc-\textnormal{swallow}=hab=def   pv.loc-\textnormal{dwell} dem.mid ~ ~ ~ ~ ~ ~ ~  \textnormal{be}=\textbf{narr}=def\\
	\glt `the Swallower's house was there' \citep[170]{hollow1973a}
	
	\item\label{narrativeevidentialenclitic4}
	\glll ų́'taahaa kó'ts ~ ~ ~ ~ ~ ~ ~ ~ ~ ~ ~ ~ ~ ~ ~  wáahaaxikerer\textbf{oomak}o'sh\\
	ų'=taa=haa ko-at=s ~ ~ ~ ~ ~ ~ ~ ~ ~ ~ ~ ~ ~ ~ ~  waa-hE=xi=krE=\textbf{oowąk}=o'sh\\
	\textnormal{be.close}=loc=ins 3poss.pers-\textnormal{father}=def ~ ~ ~ ~ ~ ~ ~ ~ ~ ~ ~ ~ ~ ~ ~  neg-\textnormal{see}=neg=3pl=\textbf{narr}=ind.m\\
	\glt `that way, they did not see their father' \citep[209]{hollow1973a}

	\end{xlist}

\end{exe}

This evidential never appears word-finally, but almost always has an allocutive agreement marker like =\textit{o'sh}. More sparingly, we see the definite marker =\textit{s} or the intensifying complementizer =\textit{sįh} in sentence-final position to highlight the fact that while the information is hearsay, the speaker is putting some kind of credence in it. During a visit to Fort Berthold in the spring of 2016, I discussed this issue with Indrek \citeauthor{park2012} (p.c.), who, in his time living in Twin Buttes and working with the Nueta Language Initiative, has noticed that there is  a parallel between Mandan =\textit{oomak} and the narrative evidential in Hidatsa =\textit{waree}. Both languages can also use this evidential as a mirative, as we can see in (\ref{mandanmirative}) and (\ref{mandanmirativehidatsa}) below.


\begin{exe}

\item\label{mandanmirative} Mirative use of =\textit{oomak} in Mandan

	\begin{xlist}
	
	\item\label{mandanmirative1}
	\glll ní'o'roomako'sh\\
	rį-o'=oowąk=o'sh\\
	2s-\textnormal{be}=narr=ind.m\\
	\glt `oh, it is you' (\citeauthor{park2012} p.c.)
	
	\item\label{mandanmirative2}
	\glll rahúuroomako'sh\\
	ra-huu=oowąk=o'sh\\
	2a-\textnormal{come.here}=narr=ind.m\\
	\glt `ah, you have come' (\citeauthor{park2012} p.c.)	
	
	\end{xlist}

\item\label{mandanmirativehidatsa} Mirative use of =\textit{waree} in Hidatsa

	\begin{xlist}
	
	\item\label{mandanmirativehidatsa1}
	\glll niháariwareeg\\
		n-ihaari=waree=g\\
		2a-\textnormal{finish}=narr=ss\\
		\glt `oh, you made some!' \citep[256]{park2012}

	\item\label{mandanmirativehidatsa2}
	\glll	 ée, miihacúudiriawareec\\
		ee wii-hacuudi-ria=waree=c\\
		\textnormal{oh} 1s-\textnormal{slit}-rflx=narr=ind\\
		\glt `oh, I've cut myself somehow' \citep[256]{park2012}
	
	\end{xlist}
	
\end{exe}

Examples of how this one enclitic can have analogous functions appear above, where this enclitic in both languages conveys a sense of surprise or sudden realization. The difference between the indirect evidential =\textit{oote} and the narrative =\textit{oomak} seems to be slight in the data above. As we can see in \citeauthor{park2012}'s (\citeyear[256]{park2012}) dissertation, the narrative =\textit{waree} in Hidatsa has a very similar distribution to the Mandan =\textit{oomak}. This may be a case of parallel development in both languages, where a particular evidential marker is able to play different roles. Another possibility is that Hidatsa has influenced Mandan so that the narrative marker in Mandan has taken on a more Hidatsa-like distribution due to the fact that the overwhelming majority of L1 Mandan speakers over the past century have also grown up in households containing fluent Hidatsa speakers.

\subsubsection{Dubitative modal enclitic: =\textit{o'xere}}

The dubitative modal enclitic =\textit{o'xere} is first described in \citet[19]{kennard1936}, where the enclitic is analyzed as a kind of conditional that expresses wonder or doubt about whether or how an action can be committed. These translations are often accompanied by `would', as in `where would we get to?', seen below in (\ref{dubitativeenclitic1}). \citeauthor{kennard1936} also says that there is a shortened form of =\textit{o'xere}, =\textit{o'x}, and that it is much more common, but it does not appear at all in his own transcribed narratives or in the corpus in general. \citet[460]{hollow1970} analyzes this enclitic as an inchoative aspectual instead, stating that it conveys a sense of an action that is about to happen. Examples of =\textit{o'xere} from the corpus appear in (\ref{dubitativeenclitic}) below.

\begin{exe}

\item\label{dubitativeenclitic} Examples of the dubitative modal enclitic =\textit{o'xere}

	\begin{xlist}
	
	\item\label{dubitativeenclitic1}
	\glll nuhínit\textbf{o'xere}'sha?\\
	rų-hi=rįt=\textbf{o'xrE}=o'sha\\
	1a.pl-\textnormal{arrive.there}=2pl=\textbf{dub}=int.m\\
	\glt `where would we get to?' \citep[19]{kennard1936}

	\item\label{dubitativeenclitic2}
	\glll ų́'st ó'harani wáara'xuure ~ ~ ~ ~ ~ ~ ~ ~ ~ túk h\'{ı̨}įkere\textbf{'xere}'sh; máamiko'sh\\
	ų't=t o'\#hrE=rį waa-ra'=xuu=E ~ ~ ~ ~ ~ ~ ~ ~ ~  tu=ak hįį=krE=\textbf{o'xrE}=o'sh waa-wįk=o'sh\\
	\textnormal{be.in.past}=loc \textnormal{be}\#caus=ss nom-ins.heat-\textnormal{be.charred}=sv ~ ~ ~ ~ ~ ~ ~ ~ ~  \textnormal{be.some}=ds \textnormal{drink}=3pl=\textbf{dub}=ind.m part-\textnormal{be.none}=ind.m\\
	\glt `long ago they could not drink any coffee; there was none' \citep[204]{hollow1973a}

	\item\label{dubitativeenclitic3}
	\glll Numá'k Máxana ókapąte ~ ~ ~ ~ ~ ~ ~ ~ ~ ~ ~ ~ ~ ~ ~  íwaroor\textbf{o'xere}'sh\\
	ruwą'k wąxrą o-ka-pąt=E ~ ~ ~ ~ ~ ~ ~ ~ ~ ~ ~ ~ ~ ~ ~  i-wa-roo=\textbf{o'xrE}=o'sh\\
	\textnormal{man} \textnormal{one} pv.irr-ins.frce-\textnormal{cultivate}=sv ~ ~ ~ ~ ~ ~ ~ ~ ~ ~ ~ ~ ~ ~ ~ pv.ins-1a-\textnormal{speak}=\textbf{dub}=ind.m\\
	\glt `I am going to talk about Lone Man's origin' \citep[5]{hollow1973a}

	\item\label{dubitativeenclitic4}
	\glll tashá waheréki wawárut\textbf{o'xara}'shka\\
	tashka wa-hrE=ki wa-wa-rut=\textbf{o'xrE}=a'shka\\
	\textnormal{how} 1a-caus=cond unsp-1a-\textnormal{eat}=\textbf{dub}=psbl\\
	\glt `how can I go about eating one of those?' \citep[32]{hollow1973b}

	\item\label{dubitativeenclitic5}
	\glll inák wiráse ítąąhąą tú ~ ~ ~ ~ ~ ~ ~ ~ ~  \textbf{ó'xere}'sh\\
	irąk wi-ras=E i-tąą=hąą tu ~ ~ ~ ~ ~ ~ ~ ~ ~  \textbf{o'xrE}=o'sh\\
	\textnormal{another} 1poss-\textnormal{name}=sv pv.ins-\textnormal{be.different}=ins \textnormal{be.some} ~ ~ ~ ~ ~ ~ ~ ~ ~ \textbf{dub}=ind.m\\
	\glt `I do not have another different name' \citep[14]{hollow1973a}
	
	\item\label{dubitativeenclitic6}
	\glll rá'te túki ré'esh númi ~ ~ ~ ~ ~ ~  \textbf{ó'xere}'sh\\
	r'-at=E tu=ki re-esh rų=awį ~ ~ ~ ~ ~ ~  \textbf{o'xrE}=o'sh\\
	2poss-\textnormal{father}=sv \textnormal{be.some}=cond dem.prox-smlt 1a.pl=cont ~ ~ ~ ~ ~ ~ \textbf{dub}=ind.m\\
	\glt `if you had a father, we would not be this way' \citep[114]{hollow1973a}	

	\end{xlist}

\end{exe}

In all the data above, the speaker is conveying a sense of doubt or wonder over whether the action can happen or could have happened. \citet[460]{hollow1970} notes that =\textit{o'xere} never co-occurs with the indirect evidential enclitic =\textit{oote}, which likely stems from the fact that the former seems to indicate a lack of certainty to whether an action took place versus the latter seeming to mark that the speaker can infer that the action did happen. As such, there is some conjectural presupposition conveyed by both =\textit{o'xere} and =\textit{oote}, though =\textit{oote} comes with a sense of certainty on the part of the speaker, while =\textit{o'xere} does not.

Like several other modal and aspectual enclitics, there is a free form of the dubitative modal enclitic, as seen in (\ref{dubitativeenclitic5}) and (\ref{dubitativeenclitic6}). The enclitic itself seems to be composed of the verb \textit{ó'} `be' plus /xrE/. This second element resembles \textit{xeré} `be safe, out of danger', which can be used periphrastically to modify a verb, as shown in (\ref{auxiliaryo'xere}) below.

\begin{exe}

\item\label{auxiliaryo'xere} Periphrastic use of \textit{xeré} `be safe'

	\glll tashkáhara ríikirukxąh ~ ~ ~ ~ ~ ~ ~ ~ ~ ~ ~ ~ ~ ~ ~  ó\textbf{xere}'sha?\\
	tashka\#hrE=$\varnothing$ rV-i-k-ru-kxąh ~ ~ ~ ~ ~ ~ ~ ~ ~ ~ ~ ~ ~ ~ ~ o-\textbf{xrE}=o'sha\\
	\textnormal{how}\#caus=cont 1a.pl-pv.ins-rflx-ins.hand-\textnormal{move} ~ ~ ~ ~ ~ ~ ~ ~ ~ ~ ~ ~ ~ ~ ~ pv.irr-\textbf{\textnormal{be.safe}}=int.m\\
	\glt `how else could we cross [the river] safely?' \citep[269]{hollow1973b}

\end{exe}

It is possible that this verb could have served to introduce dubitative propositions periphrastically in Mandan, and this construction gradually became associated more closely with the verb it modified until it became an enclitic associated with the copula \textit{ó'}, which is a common element in many enclitics associated with  complementizers. This evolution from lexical verb to modal enclitic is likely how Mandan has acquired much of its rich enclitic field.

\subsubsection{Attitudinal evidential enclitic: =\textit{nash}}
\largerpage
The attitudinal evidential enclitic =\textit{nash} is one of the most common enclitics seen in the corpus. It has posed a challenge to past scholars in that its meaning is not easy to explain. \citet[23]{kennard1936} states that =\textit{nash} adds a quality of vagueness to a stem, while \citet[467]{hollow1970} calls =\textit{nash} a typifier, in that it indicates similarity with the named object, state, or action. \citet[35]{mixco1997a} describes this enclitic as an attitudinal marker, reflecting the attenuated force of the speaker's statements as merely speculative opinion. The work herein adheres to \citeauthor{mixco1997a}'s terminology, as this enclitic does reflect some aspect of the speaker's attitude towards a proposition. We can see how the attitudinal occurs in the data in (\ref{attitudinalenclitic}) below.

\begin{exe}

\item\label{attitudinalenclitic} Examples of the attitudinal evidential enclitic =\textit{nash}

	\begin{xlist}
	
	\item\label{attitudinalenclitic1}
 	\glll Kinúma'kshi éeheni Numá'k Máxana ~ ~ ~ ~ ~ ~ ~ ~ ~ ~ ~ ~ ~ ~ ~ íwaheka\textbf{nash}e wakína'ni {ée}wereho'sh\\
	ki-ruwą'k ee-he=rį ruwą'k wąxrą ~ ~ ~ ~ ~ ~ ~ ~ ~ ~ ~ ~ ~ ~ ~   i-wa-hek=\textbf{rąsh}=E wa-kirą'=rį ee-w-reh=o'sh\\
	mid-\textnormal{man}\#\textnormal{be.good} pv-\textnormal{say}=ss \textnormal{man} \textnormal{one} ~ ~ ~ ~ ~ ~ ~ ~ ~ ~ ~ ~ ~ ~ ~  pv.ins-1a-\textnormal{know}=\textbf{att}=sv 1a-\textnormal{tell}=ss pv-1a-\textnormal{think}=ind.m\\
	\glt `I want to tell about what I sort of know about First Creator and Lone Man' \citep[1]{hollow1973a}

	\item\label{attitudinalenclitic2}
 	\glll ų́'st ówatik ~ ~ ~ ~ ~ ~ ~ ~ ~ ~ ~ ~ ~ ~ ~ ~ ~ ~ ~ ~ wáanixika\textbf{nash}o'sh\\
 	 ų't=t o-wa-ti=ak ~ ~ ~ ~ ~ ~ ~ ~ ~ ~ ~ ~ ~ ~ ~ ~ ~ ~ ~ ~ waa-rį-xik=\textbf{rąsh}=o'sh\\
 	 \textnormal{be.in.past}=loc pv.irr-1a-\textnormal{arrive.there}=ds ~ ~ ~ ~ ~ ~ ~ ~ ~ ~ ~ ~ ~ ~ ~ ~ ~ ~ ~ ~ nom-2s-\textnormal{be.bad}=\textbf{att}=ind.m\\
 	 \glt `I would have gotten there a long time ago and you are kind of bad at this' \citep[152]{hollow1973a}

	\item\label{attitudinalenclitic3}
	\glll máakahoomaksįh, súk óhekxika\textbf{nash}keres  ~ ~ ų́'shka máakah\\
	wąąkah=oowąk=sįh suk o-hek\#xik=\textbf{rąsh}=krE=s  ~ ~  ų'sh=ka wąąkah\\
	\textnormal{lie}.aux.hab=narr=ints \textnormal{child} pv.irr-\textnormal{know}\#\textnormal{be.bad}=\textbf{att}=3pl=def ~ ~  \textnormal{be.thus}=hab \textnormal{lie}.aux.hab\\
	\glt `they were living there, those poor children were living so' \citep[203]{hollow1973a}

	\item\label{attitudinalenclitic4}
	\glll mamáhe\textbf{nash}inista!\\
	wą$\sim$wą-hE=\textbf{rąsh}=rįt=ta\\
	aug$\sim$1s-\textnormal{see}=\textbf{att}=2pl=imp.m\\
	\glt `you all have got to try to see for me' \citep[35]{hollow1973a}

	\item\label{attitudinalenclitic5}
	\glll karóotiki kinúuxik tú\textbf{nash}oomako'sh\\
	ka=ooti=ki ki-rųųxik tu=\textbf{rąsh}=oowąk=o'sh\\
	prov=evid=cond suus-\textnormal{ghost} \textnormal{be.some}=\textbf{att}=narr=ind.m\\
	\glt `then he got kind of scared' \citep[71]{hollow1973b}
	
	\item\label{attitudinalenclitic6}
	\glll maxópini\textbf{nash}ishi'sh\\
	wą-xoprį=\textbf{rąsh}=ishi=o'sh\\
	unsp-\textnormal{be.holy}=\textbf{att}=vis=ind.m\\
	\glt `he must be kind of holy' \citep[313]{hollow1973b}

	\item\label{attitudinalenclitic7}
	\glll péeha\textbf{nash}tiki, ą́ąwe óshiriihaa ~ ~ ~ ~ ~  íkiruxkekerekaroomako'sh\\
	peeh=\textbf{rąsh}=ti=ki ąąwe o-shriih=E=$\varnothing$ ~ ~ ~ ~ ~  i-k-ru-xke=krE=ka=oowąk=o'sh\\
	\textnormal{be.high}=\textbf{att}=pot=cond \textnormal{all} pv.irr-\textnormal{be.scattered}=sv=cont ~ ~ ~ ~ ~  pv.ins-mid-ins.hand-\textnormal{pluck}=3pl=hab=narr=ind.m\\
	\glt `whenever he hollered, everyone would pull back and scatter' \citep[45]{hollow1973a}
	
	\end{xlist}

\end{exe}

Evidentials may be used to specify the degree of precision or the degree of truth that a speaker wishes to bestow upon an utterance, in particular, that the speaker is unsure about the veracity of an utterance, wishes to hedge the precision or truth of an utterance, or simply because the speaker does not have a more precise way to articulating their point (\citealt[90]{mithun1986}, \citealt[3]{aikhenvald2005}). It is this use that =\textit{nash} fills: it expresses some aspect of the speaker's attitude regarding how precise or how true the statement is. What is noteworthy about this evidential is that it can co-occur with other evidentials, as we see in (\ref{attitudinalenclitic5}) and (\ref{attitudinalenclitic6}) above. In (\ref{attitudinalenclitic5}), the narrative evidential notes how the speaker came by the information (i.e., through hearsay or having heard it before), but the attitudinal questions how precise or sure they are about the proposition at hand. Similarly, in (\ref{attitudinalenclitic6}), the visual evidential =\textit{ishi} shows that the speaker can visually infer that the event has happened, but the =\textit{nash} hedges whether the speaker believes how accurate or how appropriate what is being said is. As such, this double evidential marking in Mandan is not contradictory: one evidential serves to inform how the speaker knows about the proposition, while =\textit{nash} serves to depreciate or downplay some aspect of the proposition.

Both Crow and Hidatsa share cognates with Mandan =\textit{nash}, where the approximative =\textit{aachí} in Crow marks similarity or conveys a sense of `kind of, sort of, like' to the affected verb or noun \citep[44]{graczyk2007}, while the compromisive =\textit{raci} in Hidatsa has a nearly identical usage as the attitudinal in Mandan. All these forms seem to be composed of Proto-Siouan *yą-se, where *yą is a topic marker or distal demonstrative pronoun and *se, which marks similarity. In both Mandan and Hidatsa, this enclitic is found often in casual conversations, and -- at least in Mandan -- is often chided as being informal or ``slangy.'' Some speakers are more prone to use =\textit{nash} more often than others, so its usage is highly subject to personal speech style rather than a language-wide tendency. The attitudinal =\textit{nash} also serves to widen the category of a noun. Examples of =\textit{nash} with nouns from the corpus appear in (\ref{attitudinalclasses}) below.

\begin{exe}

\item\label{attitudinalclasses} Use of =\textit{nash} to extend the class of nouns

	\begin{xlist}
	
	\item\label{attitudinalclasses1} 
	\glll wáa'oksąha\textbf{nash}\\
	waa-o-ksąh=\textbf{rąsh}\\
	nom-pv.irr-\textnormal{worry}=\textbf{att}\\
	\glt `sneaky kinds of things' \citep[80]{hollow1973a}

	\item\label{attitudinalclasses2} 
	\glll mashkáshkapka\textbf{nash}ini katéka\textbf{nash}ini rúta máakahoomako'sh\\
	wą-shka$\sim$shkap=ka=\textbf{rąsh}=rį katek=\textbf{rąsh}=rį rut=E=$\varnothing$ wąąkah=oowąk=o'sh\\
	unsp-dist$\sim$\textnormal{prick}=hab=\textbf{att}=ss \textnormal{chokecherry}=\textbf{att}=ss \textnormal{eat}=sv=cont \textnormal{lie}.aux.hab=narr=ind.m\\
	\glt `they had been eating rosehips and things like that, and chokecherries and things like that' \citep[279]{hollow1973b}

	\item\label{attitudinalclasses3} 
	\glll kóo\textbf{nash}e\\
	koo=\textbf{rąsh}=E\\
	\textnormal{squash}=\textbf{att}=sv\\
	\glt `squash and things like that' \citep[467]{hollow1970}

	\item\label{attitudinalclasses4} 
	\glll minísa\textbf{nash}e\\
	wrįs=\textbf{rąsh}=E\\
	\textnormal{horse}=\textbf{att}=sv\\
	\glt `horses and things like that' \citep[467]{hollow1970}

	\item\label{attitudinalclasses5} 
	\glll makxéxa\textbf{nash}ini...\\
	wą-kxek=\textbf{rąsh}=rį\\
	unsp-\textnormal{throw.out}=\textbf{att}=ss\\
	\glt `trash and stuff and...' \citep[110]{hollow1973a}
	
	\end{xlist}

\end{exe}

This enclitic does not trigger ablaut, despite the fact that it contains a nasal element. The reason why is not clear, especially since it seems to be an older element that is shared between Mandan and Crow-Hidatsa.

\subsubsection{Definite evidential enclitic: =\textit{s}}\label{SubSubSecDefiniteEvidential}

Previous works on Mandan have referred to \textit{=s} as a preterite marker \citep{kennard1936,hollow1970}. Much of the discussion in this part of the chapter has dealt with why Mandan is not a tense-marking language. Given that =\textit{s} does not mark tense, it must have a different function.

The definite article in Mandan is =\textit{s}, and it is homophonous with the verbal enclitic. This homophony is not coincidental, as =\textit{s} marks that the speaker is certain that an event has happened. As such, this enclitic marks definiteness on both nouns and verbs. We can see examples of this distribution of =\textit{s} in the data in (\ref{definiteevidential}) below.

\begin{exe}

\item\label{definiteevidential} Examples of the definite evidential enclitic =\textit{s}

	\begin{xlist}
	
	\item\label{definiteevidential1}
	\glll miníike, ``riréesike manakų́'ki'' ~ ~ ~ ~ ~ ~ ~ ~ ~ ~ éepe\textbf{s}o'sh\\
	wį-rįįk=E ri-reesik=E w-rą-kų'=ki ~ ~ ~ ~ ~ ~ ~ ~ ~ ~  ee-w-he=\textbf{s}=o'sh\\
	1poss-\textnormal{son}=sv 2poss-\textnormal{tongue}=sv 1s-2a-\textnormal{give}=cond ~ ~ ~ ~ ~ ~ ~ ~ ~ ~  pv-1a-\textnormal{say}=\textbf{def}=ind.m\\
	\glt `my son, I said, ``will you give me your tongue?''' \citep[190]{hollow1973a} 

	\item\label{definiteevidential2}
	\glll Kinúma'kshi\textbf{s}eena Numá'k Máxana\textbf{s} ~ ~ ~ ~ ~ ~ ~ ~ pahų́hanashoomak\textbf{s}\\
	ki-ruwą'k\#shi=\textbf{s}=ee=rą ruwą'k wąxrą=\textbf{s} ~ ~ ~ ~ ~ ~ ~ ~ pa-hųh=rąsh=oowąk=\textbf{s}\\
	mid-\textnormal{man}\#\textnormal{be.good}=\textbf{def}=dem.dist=top \textnormal{man} \textnormal{one}=\textbf{def} ~ ~ ~ ~ ~ ~ ~ ~  ins.push-\textnormal{be.ahead}=att=narr=\textbf{def}\\
	\glt `Old Man Coyote got ahead of Lone Man' \citep[9]{hollow1973a}

	\item\label{definiteevidential3}
	\glll na'é, éexi wáarahere tú\textbf{s}o're\\
	rą'e eexi waa-ra-hrE tu=\textbf{s}=o're\\
	\textnormal{mother}.voc \textnormal{belly} nom-2a-caus \textnormal{be.some}=\textbf{def}=ind.f\\
	\glt `mother, there is some paunch for you to eat' \citep[72]{hollow1973a}

	\item\label{definiteevidential4}
	\glll waxtáani óxkąhe warúutekere\textbf{s}o'sh\\
	wa-xtE=rį o-xkąh=E wa-ruutE=krE=\textbf{s}=o'sh\\
	unsp-\textnormal{be.big}=ss pv.irr-\textnormal{move}=sv unsp-\textnormal{eat}=3pl=\textbf{def}=ind.m\\
	\glt `the travelers were really hungry' \citep[80]{hollow1973b}

	\item\label{definiteevidential5}
	\glll wáa'ąąwe minikų́'ka\textbf{s}o'sh\\
	waa-ąąwe w-rį-kų'=ka=\textbf{s}=o'sh\\
	nom-\textnormal{all} 1a-2s-\textnormal{give}=hab=\textbf{def}=ind.m\\
	\glt `I always give everything to you' \citep[98]{hollow1973b}

	\item\label{definiteevidential6}
	\glll réeharaana, írahek\textbf{s}o're\\
	rEEh\#hrE=rą i-ra-hek=\textbf{s}=o're\\
	\textnormal{go.there}\#caus=imp.f pv.ins-2a-\textnormal{know}=\textbf{def}=ind.f\\
	\glt `go ahead, you know how to do it' \citep[181]{hollow1973b}
	
	\end{xlist}

\end{exe}

In all the examples above, the =\textit{s} serves to emphasize an action that the speaker knows has happened. In the case of (\ref{definiteevidential2}), the presence of the narrative and the definite together indicate that, although the speaker heard that it happened that way from another source, the speaker can attest that it definitely happened that way. Similarly, in (\ref{definiteevidential3}), there is no past tense reading, and the speaker is informing their mother that something is indeed the case. The same reading is present in (\ref{definiteevidential6}), where the speaker is exhorting another person to do something that they know how to do, and that the speaker knows they know how to do. 

It is true that this enclitic often appears on propositions that take place in the past, but that is also a side effect of the corpus consisting mostly of traditional Mandan narratives about cultural figures and their past deeds. When present in quoted speech, =\textit{s} can be used for events in any time setting, provided that the speaker is certain about the truth value of what they are saying.

Mandan shares this definite marker with Crow =\textit{sh} \citep[156]{graczyk2007}, and with Hidatsa =\textit{sh} \citep[530]{park2012}. In Crow and Hidatsa, this definite marker can likewise be seen on definite events that have taken place in the past, similar to its use in Mandan, though =\textit{sh} seems to imply some sequential relationship where the definite event has finished and a subsequent event begins. However, this enclitic is not required to give a past reading \textit{per se}, but indicates that an event has been completed or will be completed. In Mandan, this perfective reading is not inherent, as it serves mostly to emphasize how certain the speaker is of the veracity of an utterance.

\subsection{Number enclitics}

All Siouan languages have suffixes or enclitics to distinguish a singular subject from a plural subject \citep[106]{parksrankin2001}. Mandan is the only Siouan language besides Catawba to have dedicated postverbal marking for plural subjects and plural objects. Plural marking in Mandan for subjects and objects does not depend on what semantic role that argument plays in a proposition, but rather how that argument is involved in the discourse. A list of plural enclitics appears in (\ref{mandanpluralenclitics}) below.

\begin{exe}

\item\label{mandanpluralenclitics} Plural enclitics in Mandan

\begin{tabular}{ll}

	/=rįt/ & second person plural\\
	/=krE/ & third person plural\\ 

\end{tabular}

\end{exe}

These enclitics are explained in greater detail in the subsections below.

\subsubsection{Second person plural: \textit{=nit}}\label{subsecnit}

Whenever a second person plural argument is involved, the enclitic =\textit{nit} appears on the verb. In addition to marking plurality for second person arguments, =\textit{nit} is also used for first person plural marking. This behavior is due to the fact that the first person plural typically has an inclusive reading, and thus when the first person plural is used, a second person argument is also involved in the proposition by virtue of being grouped with the first person argument. Given this distribution, this enclitic marks plurality for speech act participants (i.e., both first person and second person arguments) rather than just for second person arguments. 

Because this enclitic has historically been referred to as the second person plural marker, I continue to refer to it by this nomenclature, though the fact that I call it a second person plural marker should not be confused with the fact that it really is a speech act participant plural marker. We can see the behavior of this enclitic in the data below, where =\textit{nit} appears with both first person plural arguments and second person plural arguments, regardless of what role said arguments play in a clause, i.e., agent or non-agent.

\begin{exe}

\item\label{2ndpersonplural} Examples of the second person plural enclitic =\textit{nit}

	\begin{xlist}
	
	\item\label{2ndpersonplural1} 
	\glll súki\textbf{nit}e, matewé íraseki\textbf{nit}o'sha?\\
	suk=\textbf{rįt}=E wa-t-we i-ra-sek=\textbf{rįt}=o'sha\\
	\textnormal{child}=\textbf{2pl}=sv unsp-wh-indf pv.ins-2a-\textnormal{make}=\textbf{2pl}=int.m\\
	\glt `children, what are you all doing?' \citep[28]{hollow1973a}

	\item\label{2ndpersonplural2} 
	\glll ó'sh, téehą óminitaa wamáakahi\textbf{nit}o'sh\\
	o'sh teehą o-w-rį-taa wa-wąąkah=\textbf{rįt}=o'sh\\
	\textnormal{ind.m} \textnormal{be.far.away} pv.loc-1a-2s-\textnormal{be.with} 1a-\textnormal{lie}.aux.hab=\textbf{2pl}=ind.m\\
	\glt `gosh, I have been staying with you all for a long time' \citep[31]{hollow1973a}

	\item\label{2ndpersonplural3} 
	\glll íshkanasha írawaaxani ~ ~ ~ ~ ~ ~ ~ ~ ~ ~ ranúunihi\textbf{nit}ki tóopa ná'hki ~ ~ ~ ~ ~ ~ ~ ~ ~ ~ ówahi'sh\\
	i-shka=rąsh=E=$\varnothing$ i-ra-waxE=rį ~ ~ ~ ~ ~ ~ ~ ~ ~ ~ ra-ruurįh=\textbf{rįt}=ki toopa rą'k=ki ~ ~ ~ ~ ~ ~ ~ ~ ~ ~  o-wa-hi=o'sh\\
	pv.ins-\textnormal{be.a.while}=att=sv=cont pv.ins-2a-\textnormal{stop}=ss ~ ~ ~ ~ ~ ~ ~ ~ ~ ~ 2a-\textnormal{be.there}.pl.dur.aux=\textbf{2pl}=cond \textnormal{four} sit.pos=cond ~ ~ ~ ~ ~ ~ ~ ~ ~ ~ pv.irr-1a-\textnormal{arrive.there}=ind.m\\
	\glt `if you (pl.) stop and are there in a little bit, after four days, I will be there' \citep[130]{hollow1973a}

	\item\label{2ndpersonplural4} 
	\glll rá'tere máa'ų'staa ~ ~ ~ ~ ~ ~ ~ ~ ~ ~ ~ ~ ~ ~ ~  óteeniharaa\textbf{nit}e íratso'nik...\\
	r'-at=re waa-ų't=taa ~ ~ ~ ~ ~ ~ ~ ~ ~ ~ ~ ~ ~ ~ ~  o-tee\#rį-hrE=\textbf{rįt}=E i-rat=so'rįk\\
	2poss-\textnormal{father}=dem.prox nom-\textnormal{be.in.past}=loc ~ ~ ~ ~ ~ ~ ~ ~ ~ ~ ~ ~ ~ ~ ~  pv.irr-\textnormal{die}\#2s-caus=\textbf{2pl}=sv pv.ins-\textnormal{promise}=comp.caus\\
	\glt `because that father of yours promised to kill you (pl.) long ago...' \citep[194]{hollow1973a}

	\item\label{2ndpersonplural5} 
	\glll nustámi nukirúshani manátaa ~ ~ ~ ~ ~ ~  róokasaa\textbf{nit}o'sh\\
	rų-ista\#wį rų-k-ru-shE=rį wrą=taa ~ ~ ~ ~ ~ ~  rV-o-ka-saa=\textbf{rįt}=o'sh\\
	1pl.poss-\textnormal{face}\#\textnormal{orb} 1a.pl-suus-ins.hand-\textnormal{hold}=ss \textnormal{tree}=loc ~ ~ ~ ~ ~ ~ 1a.pl-pv.loc-ins.frce-\textnormal{remove.meat.from.bone}=\textbf{2pl}=ind.m\\
	\glt `we are taking our eyes out to hang them on a tree' \citep[28]{hollow1973a}

	\item\label{2ndpersonplural6}
	\glll wáa'oksąh íseke s\'{ı̨}hanashak ~ ~ ~ ~ ~ ~ ~  ríiheki\textbf{nit}o'sh\\
	waa-o-ksąh i-sek=E sįh=rąsh=ak ~ ~ ~ ~ ~ ~ ~  rV-i-hek=\textbf{rįt}=o'sh\\
	nom-pv.irr-\textnormal{be.worried} pv.ins-\textnormal{make}=sv \textnormal{be.strong}=att=ds ~ ~ ~ ~ ~ ~ ~  1a.pl-pv.ins-\textnormal{know}=\textbf{2pl}=ind.m\\
	\glt `we know about the crooked things he does all the time' \citep[43]{hollow1973a}

	\item\label{2ndpersonplural7}
	\glll nukíkirakshiki\textbf{nis}to'sh!\\
	rų-ki$\sim$ki-ra-kshik=\textbf{rįt}=t=o'sh\\
	1a.pl=recp$\sim$rflx-ins.foot-\textnormal{wrestle}=\textbf{2pl}=pot=ind.m\\
	\glt `let's all wrestle each other!' \citep[152]{hollow1973b}

	\item\label{2ndpersonplural8}
	\glll róoraha\textbf{nit}o'sh\\
	rV-o-ra-hE=\textbf{rįt}=o'sh\\
	1s.pl-pv.irr-2a-\textnormal{see}=\textbf{2pl}=ind.m\\
	\glt `you are going to see us' \citep[477]{hollow1973a}
	
	\end{xlist}

\end{exe}

When =\textit{nit} appears, it is typically as a subject plural. Marking object plurality in Mandan is optional, so =\textit{nit} appears to mark objects only sparingly. Since both first and second persons compete for the same marker, it can be unclear in isolation whether the =\textit{nit} pluralizes the first or the second person argument in a proposition. We can see this ambiguity in (\ref{2ndpersonplural8}), where it is not immediately clear whether there is a singular second person acting upon more than two first persons, or if there are more than one second persons acting upon a pair of first persons.

Mandan distinguishes between first person plural and first person dual  like other Siouan languages with dedicated first person plural prefixes, such as Lakota \citep[761]{ullrich2011}. In the same manner as these other Siouan languages, this distinction is realized by first person plural bearing postverbal plural marking along with the first person plural prefix, while first person dual involves only the first person plural prefix and no additional marking following the verb. However, plurality is generally only marked on matrix verbs, so in situations where multiple clauses are chained together within the same sentence, only the matrix verb will bear \textit{=nit}. This lack of plural marking on those non-matrix verbs do not mark them as being dual rather than plural. 

We can see examples of this dual marking below in (\ref{2ndpersondual}), where the lack of \textit{=nit} reflects the presence of a first person dual argument. Such arguments are typically interpreted as being inclusive in nature, i.e., the other speech act participant is also the addressee.

\begin{exe}

\item\label{2ndpersondual} Examples of dual marking without =\textit{nit}

	\begin{xlist}
	
	\item\label{2ndpersondual1}
	\glll máahsi íip \textbf{nu}táshika'sh\\
	wąąh\#si iip \textbf{rų}-ta-shi=ka=o'sh\\
	\textnormal{arrow}\#\textnormal{feather} \textnormal{tail.feather} \textbf{1a.pl}-al-\textnormal{be.good}=hab=ind.m\\
	\glt `we (du.) always like eagle tail feathers' \citep[215]{hollow1973a}

	\item\label{2ndpersondual2}
	\glll \textbf{nu}kípiiro'sh, \textbf{nu}tápąąxe\\
	\textbf{rų}-ki-pii=o'sh \textbf{rų}-ta-pąąxE\\
	\textbf{1a.pl}-suus-\textnormal{devour}=ind.m \textbf{1a.pl}-al-\textnormal{potato}\\
	\glt `we (du.) ate our potatoes up' \citep[55]{hollow1973a}

	\item\label{2ndpersondual3}
	\glll na'é réeh\textbf{ro}hereso'sh\\
	rą'e rEEh\#\textbf{ro}-hrE=s=o'sh\\
	\textnormal{mother}.voc \textnormal{go.there}\#\textbf{1s.pl}-caus=def=ind.m\\
	\glt `mother told us (du.) to go' \citep[166]{hollow1973a}
	
	\end{xlist}

\end{exe}

In each of the sentences above, there is a first person dual subject or object, and the only morphological indication that the number is not plural is the lack of =\textit{nit}. In many situations, it is not possible to glean whether the first person argument is plural or dual when it is not a subject. Mandan often relies on context to fill in those kinds of details, and the language likewise relies heavily upon {pro}-dropping arguments, so listeners must be active to ascertain who is doing what to whom, or speakers must rely on their familiarity with traditional narratives to fill in gaps the speaker is leaving by omitting arguments.

Mandan is the only Siouan language to have a dedicated speech act participant plural marker. This enclitic likely developed from the Proto-Siouan second person stative marker *yį-, which became duplicated as a postverbal element that combined with the Proto-Siouan stem augment *-tE. The presence of the stem augment suggests that a Pre-Mandan **rįtE may have been an unbound element at some point in its development before becoming reanalyzed as an enclitic. The second person plural marker is an ablaut-triggering enclitic, likely stemming from the fact it contains a nasal vowel.

\subsubsection{Third person plural: \textit{=kere}}

The most common manifestation of plurality in the corpus is the third person plural marker =\textit{kere} /=krE/. It has several cognates across the Siouan language family: Hoocąk uses /-ire/ to mark plurals for third person subjects \citep[6]{lipkind1945}, and Tutelo uses =\textit{hele} /=hlE/ \citep[41]{oliverio1997}. The Catawban third person plural subject suffix -\textit{ʔi} may also be a cognate, since certain verb paradigms have a -\textit{hi} instead \citep[42]{rudes2007a}. This variant suggests that Proto-Siouan could have had *hirE as a third person plural marker, though the /k/ in Mandan is unexplained. Another possibility is that the \textit{suus} marker *ki- could have become associated with the *hirE third person plural at some point in Pre-Mandan. Before stems beginning with sonorants or *h, the \textit{suus} marker tended to syncopate the *i, becoming *k-, a tendency preserved in modern Mandan. Proto-Siouan *k, *kh, and *hk all collapsed into /k/ in Mandan, and short vowels tended to syncopate before a sonorant, so we could achieve the modern Mandan form if this enclitic underwent the steps shown in (\ref{possiblekre}) below.

\begin{exe}

\item\label{possiblekre} Possible evolution of =\textit{kere}

*ki-hirE > **k-hirE > ***k-hrE > ****k-rE > =\textit{kere} /=krE/

\end{exe}

There is no posited third person plural marker in \citet{rankin2015}, but the presence of cognates across the Siouan language family suggests that Proto-Siouan had a dedicated third person plural marker. Alternatively, there was some kind of periphrastic construction that expressed a third person subject in Proto-Siouan. In Mandan, this enclitic is used to mark both subjects and objects that are not speech act participants, as we can see in (\ref{3rdpersonplural}) below.

\begin{exe}

\item\label{3rdpersonplural} Examples of third person plural enclitic =\textit{kere}

	\begin{xlist}
	
	\item\label{3rdpersonplural1}
	\glll kirusanáhanashini réeh\textbf{kere}roomaks\\
	k-ru-srąh=rąsh=rį rEEh=\textbf{krE}=oowąk=s\\
	rflx-ins.hand-\textnormal{abandon}=att=ss \textnormal{go.there}=\textbf{3pl}=narr=ind.m\\
	\glt `they parted ways and went' \citep[9]{hollow1973a}

	\item\label{3rdpersonplural2}
	\glll kíihi\textbf{karaa}ni náakus íkiroo\textbf{kere}roomako'sh\\
	kiihi=\textbf{krE}=rį rąąku=s i-ki-roo=\textbf{krE}=oowąk=o'sh\\
	\textnormal{meet}=\textbf{3pl}=ss \textnormal{road}=def pv.ins-reflx-\textnormal{speak}=\textbf{3pl}=narr=ind.m\\
	\glt `they met and argued about the road' \citep[24]{hollow1973a}

	\item\label{3rdpersonplural3}
	\glll numá'kaaki ísek ą́ąwe k\'{ı̨}'h\textbf{kere}'sh\\
	ruwą'k-aaki i-sek ąąwe kį'k=\textbf{krE}=o'sh\\
	\textnormal{man}-coll pv.ins-\textnormal{make} \textnormal{all} \textnormal{finish}=\textbf{3pl}=ind.m\\
	\glt `they were all finished making people' \citep[13]{hollow1973a}

	\item\label{3rdpersonplural4}
	\glll káni súh\textbf{kere}s istámi kirúshaani ~ ~ ~ ~ ~ ~ ~ ímanastaa íkų'te\textbf{kere}roomako'sh\\
	ka=rį suk=\textbf{krE}=s ista\#wį ki-ru-shE=rį ~ ~ ~ ~ ~ ~ ~  i-wrą=s=taa i-kų'tE=\textbf{krE}=oowąk=o'sh\\
	prov=ss \textnormal{child}=\textbf{3pl}=def \textnormal{face}\#\textnormal{orb} suus-ins.hand-\textnormal{hold}=ss ~ ~ ~ ~ ~ ~ ~  pv.dir-\textnormal{tree}=def=loc pv.dir-\textnormal{throw}=\textbf{3pl}=narr=ind.m\\
	\glt `and then, the children took their eyes and threw them toward the tree' \citep[29]{hollow1973a}

	\item\label{3rdpersonplural5}
	\glll í'ųųtahąkt íkų'te\textbf{kere}roomako'sh\\
	i-ųųtahąk=t i-kų'tE=\textbf{krE}=oowąk=o'sh\\
	pv.dir-\textnormal{east}=loc pv.dir-\textnormal{throw}=\textbf{3pl}=narr=ind.m\\
	\glt `he threw them to the east' \citep[13]{hollow1973a}

	\item\label{3rdpersonplural6}
	\glll wáarakaakiri\textbf{karaa}nitinixo'sh\\
	waa-ra-k-aa-kri=\textbf{krE}=rįt=rįx=o'sh\\
	neg-2a-vert-pv.tr-\textnormal{arrive.back.here}=\textbf{3pl}=2pl=neg=ind.m\\
	\glt `you did not arrive back here with them having started out with them' \citep[447]{hollow1970}
	
	\end{xlist}
	
\end{exe}

We see =\textit{kere} used as an object marker in (\ref{3rdpersonplural5}) and (\ref{3rdpersonplural6}), but it is more common to omit the =\textit{kere} when not used to mark a subject. Context plays a large role in how =\textit{kere} is interpreted, as both (\ref{3rdpersonplural4}) and (\ref{3rdpersonplural5}) feature the same verb \textit{íkų'tekereroomako'sh}, where the meaning of the verb can be `they threw it', `they threw them', `he threw them', etc. The meaning is apparent when contextualized within the discourse, but separately, the dual use as subject and object plural marker can create ambiguous statements.

When not used in matrix clauses, there is a strong tendency to omit =\textit{kere}, as we see in (\ref{3rdpersonplural1}) where both verbs involve the same subjects, but only the matrix verb is marked with =\textit{kere}. When switch-reference markers are involved, there are far more instances of verbs without =\textit{kere} than there are verbs that include third person plural marking. Again, much of the informational load is left to context within the discourse. 

\subsection{Negative enclitics: \textit{=nix} and \textit{=xi}}\label{Ch3NegativeEnclitics}

Negation in Mandan involves multiple exponents. As discussed in \sectref{SubsubsecNegative}, the negative inflectional prefix \textit{waa}- appears on a negated verb. In addition to that negation prefix, there are two different negation enclitics that must be used in Mandan: =\textit{nix} and =\textit{xi}. Both of these enclitics come from Proto-Siouan negation markers. The Proto-Siouan negation enclitic is reconstructed as *-aši, with the initial vowel becomes reanalyzed as part of the stem in many modern Siouan languages, where it triggers ablaut \citep{rankin2015}.

Proto-Siouan and most modern Siouan languages have a fricative sound symbolism where an action or state can be increased or diminished depending on which fricative is used, e.g., \textit{síire} `yellow', \textit{shíire} `tawny', \textit{xíire} `brown' in Mandan (cf. \sectref{soundsymbolism}). The enclitic =\textit{xi} is an x-grade reflex of *-aši. The enclitic =\textit{nix} is actually a combination of a different Proto-Siouan negative marker, *-rį, plus another negative marker, *-axi, i.e., *-rį-axi > **-rį-xi > =\textit{nix}. The distribution of each of these enclitics appears in \sectref{SubSubNIX} and \sectref{SubSubXI} below.

\subsubsection{Allomorph /=rįx/}\label{SubSubNIX}
\largerpage
\citet[23]{kennard1936} states that =\textit{nix} appears on any consonant-final stem, as well as on any vowel-final stem ending in a non-high vowel. \citet[31]{hollow1970} states that =\textit{nix} is only for consonant-final stems. The distribution observed in the corpus is that =\textit{nix} actually appears on any stem that does not end in a short vowel. \citeauthor{hollow1970} analyzes all underlying heavy open syllables as having a /ɾ/ at the end because he does not perceive the difference between long and short vowels, where [ɾ] only arises through epenthesis at an enclitic boundary involving an open heavy syllable.\footnote{Refer back to \sectref{intrusiveR} for more discussion about why \citet{hollow1970} hypothesized stem-final /ɾ/ and why that analysis does not hold.} We can see this behavior for =\textit{nix} in (\ref{negativenix}) below, where each instance of =\textit{nix} occurs following a stem that does not end in an open syllable containing a short vowel.

\begin{exe}

\item\label{negativenix} Examples of the negative enclitic =\textit{nix}

	\begin{xlist}
	
	\item\label{negativenix1}
	\glll tí áakit ó'harani há náaka ~ ~ ~ ~ ~ ~ ~ ~ ~ ~ ~ ~ ~ ~  wáa'ooti\textbf{nix}oomako'sh\\
	ti aaki=t o'\#hrE=rį hE rąąkE=$\varnothing$ ~ ~ ~ ~ ~ ~ ~ ~ ~ ~ ~ ~ ~ ~ waa-oot=\textbf{rįx}=oowąk=o'sh\\
	\textnormal{dwell} \textnormal{be.on.top}=loc \textnormal{be}\#caus=ss \textnormal{see} \textnormal{sit}.aux=cont ~ ~ ~ ~ ~ ~ ~ ~ ~ ~ ~ ~ ~ ~  neg-\textnormal{mix}=\textbf{neg}=narr=ind.m\\
	\glt `looking from on top of the house, she was not in it' \citep[126]{hollow1973a}

	\item\label{negativenix2}
	\glll kotámiihseena inák miní ~ ~ ~ ~ ~ hų́ áaki'hini ~ ~ ~ ~ ~ ~ ~ ~ ~ ~ ~ ~ ~ ~ ~ ~ ~ ~ ~ ~ wáasuki\textbf{nix}kereroomako'sh\\
	ko-ta-wįįh=s=ee=rą irąk wrį ~ ~ ~ ~ ~ hų aa-ki'h=rį ~ ~ ~ ~ ~ ~ ~ ~ ~ ~ ~ ~ ~ ~ ~ ~ ~ ~ ~ ~ waa-suk=\textbf{rįx}=krE=oowąk=o'sh\\
	3poss.pers-al-\textnormal{male's.sister}=def=dem.dist=top \textnormal{again} \textnormal{water} ~ ~ ~ ~ ~ \textnormal{be.many} pv.tr-\textnormal{arrive.back.here}=ss ~ ~ ~ ~ ~ ~ ~ ~ ~ ~ ~ ~ ~ ~ ~ ~ ~ ~ ~ ~ neg-\textnormal{exit}=\textbf{neg}=3pl=narr=ind.m\\
	\glt `his sister also brought a lot of water and they did not go out' \citep[200]{hollow1973a}

	\item\label{negativenix3}
	\glll kowóorooxikanash ée wáa'o'\textbf{nix}ishi're\\
	ko-wooroo\#xik=rąsh ee waa-o'=\textbf{rįx}=ishi=o're\\
	3poss.pers-\textnormal{husband}\#\textnormal{be.bad}=att dem.dist neg-\textnormal{be}=neg=vis=ind.f\\
	\glt `that one must not be her no-good husband' \citep[133]{hollow1973a}

	\item\label{negativenix4}
	\glll wáakų'\textbf{nix}ishiso'sh\\
	waa-kų'=\textbf{rįx}=ishi=s=o'sh\\
	neg-\textnormal{give}=\textbf{neg}=vis=def=ind.m\\
	\glt `I knew he did not give it to him' \citep[441]{hollow1970}

	\item\label{negativenix5}
	\glll kashká'nik, wáatee\textbf{nix}karoomako'sh\\
	ka=shka'rįk waa-tee=\textbf{rįx}=ka=oowąk=o'sh\\
	prov=disj neg-\textnormal{die}=\textbf{neg}=hab=narr=ind.m\\
	\glt `however, she would never die' \citep[287]{hollow1973b}

	\item\label{negativenix6}
	\glll máaminihįį\textbf{nix}o'sh\\
	waa-wrį\#hįį=rįx=o'sh\\
	neg-\textnormal{water}\#\textnormal{drink}=\textbf{neg}=ind.m\\
	\glt `he fasts [lit. he does not drink water]' \citep[303]{hollow1970}
	
	\end{xlist}

\end{exe}

When it comes to determining whether to use =\textit{nix} or =\textit{xi}, =\textit{nix} has a wider distribution: it appears after consonant-final stems, long vowel-final stems, and glottal stop-final stems. Of the two realizations of negativity in the enclitic field, =\textit{nix} is the default negative enclitic in Mandan.

\subsubsection{Allomorph /=xi/}\label{SubSubXI}

As discussed above, =\textit{nix} is used in nearly every context to express negation on a proposition except for when a stem ends in a short vowel. This fact is somewhat obscured by the fact that both =\textit{nit} and =\textit{xi} are ablaut-triggering enclitics. We can see this behavior in the data below for those speakers whose grammar treats negation as an ablaut-triggering process. Of \citeapos{hollow1970} two main consultants, Mrs. Annie Eagle consistently ablauts /E/ and /EE/ before a negative enclitic, but Mrs. Otter Sage does not. This was previously discussed in \sectref{morphologicallyconditionedablaut}. The data included (\ref{negativexi}) below are restricted to examples that feature ablaut, as that is more common throughout the corpus across a greater number of speakers.\footnote{This omission of non-ablauting \textit{=xi} data does not signify that there is anything incorrect about not ablauting before a negative enclitic, but it is unclear whether this lack of ablaut is due to idiolectal, familiolectal, or dialectal differences.}

\begin{exe}

\item\label{negativexi} Examples of the negative enclitic =\textit{xi}

	\begin{xlist}
	
	\item\label{negativexi1}
	\glll wáa'eetaa ná'kak wáawahi\textbf{xi}'sh\\
	waa-ee=taa rą'k=ak waa-wa-hi=\textbf{xi}=o'sh\\
	nom-\textnormal{be.far.off}=loc pos.sit=ds neg-1a-\textnormal{arrive.there}=\textbf{neg}=ind.m\\
	\glt `because he was sitting far off, I did not get there' \citep[10]{hollow1973b}

	\item\label{negativexi2}
	\glll wáaroska\textbf{xi} áakit ó'harani ~ ~ ~ ~ ~ ~ ~ ~ ~ ~ hą́ąkeroomako'sh\\
	waa-roskE=\textbf{xi} aaki=t o'\#hrE=rį ~ ~ ~ ~ ~ ~ ~ ~ ~ ~  hąąkE=oowąk=o'sh\\
	neg-\textnormal{jump.down}=\textbf{neg} \textnormal{be.on.top}=loc \textnormal{be}\#caus=ss ~ ~ ~ ~ ~ ~ ~ ~ ~ ~ \textnormal{be.standing}.aux=narr=ind.m\\
	\glt `he did not come down from on top of there and stayed there' \citep[143]{hollow1973b}

	\item\label{negativexi3}
	\glll nitámi'tis wáateehą\textbf{xi}'sh\\
	rį-ta-wį'\#ti=s waa-teehą=\textbf{xi}=o'sh\\
	2poss-al-\textnormal{stone}\#\textnormal{dwell}=def neg-\textnormal{be.far}=\textbf{neg}=ind.m\\
	\glt `your village is not far' \citep[29]{hollow1973b}

	\item\label{negativexi4}
	\glll Kinúma'kshis shí máa'ąke ~ ~ ~ ~ ~ ~ ~ ~ ~ ~ ~ ~ ~ ~ ~ ~ ~ ~   wáahi\textbf{xi}roomako'sh\\
	ki-ruwą'k\#shi=s shi waa'ąk=E ~ ~ ~ ~ ~ ~ ~ ~ ~ ~ ~ ~ ~ ~ ~ ~ ~ ~  waa-hi=\textbf{xi}=oowąk=o'sh\\
	mid-\textnormal{man}\#\textnormal{be.good}=def \textnormal{foot} \textnormal{earth}=sv ~ ~ ~ ~ ~ ~ ~ ~ ~ ~ ~ ~ ~ ~ ~ ~ ~ ~  neg-\textnormal{arrive.there}=\textbf{neg}=narr=ind.m\\
	\glt `Old Man Coyote's feet did not touch the ground' \citep[49]{hollow1973b}

	\item\label{negativexi5}
	\glll mí'shak maná'teki, wáa'oraxaraa\textbf{xi}'sh\\
	w'\~~-ishak wa-rą'tE=ki waa-o-ra-xrE=\textbf{xi}=o'sh\\
	1s-pro 1a-\textnormal{stand.up}=cond neg-pv.irr-\textnormal{be.safe}=\textbf{neg}=ind.m\\
	\glt `if I get up, you are not going to live' \citep[123]{hollow1973b}
	
	\item\label{negativexi6}
	\glll wáa, ní'o'na máanitashi\textbf{xi} á'shka?\\
	waa rį-o'=rą waa-rį-ta-shi=\textbf{xi} a'shka\\
	\textnormal{uh} 2s-\textnormal{be}=top neg-2poss-al-\textnormal{be.good}=\textbf{neg} psbl\\
	\glt `uh, was it you who did not like him?' \citep[237]{hollow1973b}
	
	\end{xlist}

\end{exe}

Because of its status as an ablaut-triggering enclitic, =\textit{xi} often appears after phonetically long vowels despite the fact that it targets stems with underlyingly short vowels. Once a vowel ablauts, we do not see a switch from =\textit{xi} to =\textit{nix} instead. \citet[37]{mixco1997a} is the first scholar to suspect some connection between vowel lengthening and =\textit{xi}, though he did not ascribe this lengthening to ablaut alone, which is the case.

\subsection{Complementizer enclitics}\label{SecComplementizers}

Mandan typically requires that some element fill the complementizer spot within a syntactic structure for a complete utterance. I argue in \citet[305]{kasak2019} that Mandan has some kind of \textsc{Fill} C requirement, which accounts for the large amount of complementizer-level morphology by making it obligatory that all clauses bear some kind of complementizer morphology. The lack of a complementizer can often signal false starts, shifts in the discourse, or that speakers have uttered some kind of sentence fragment. Detaching this work from a theoretical working model of the development of enclitic morphology in Mandan, we can at least note that the large amount of material that must appear in clause-final position strongly suggests that there was an emphasis placed in Pre-Mandan on ensuring that certain information was encoded into the structure of a sentence through overt morphology, e.g., indicating whether an utterance is a statement or a question, or whether a clause was a matrix clause or an adjunct clause, and so on and so forth.

Furthermore, previous stages of Mandan must have had a similar requirement to shift information-carrying morphological items to the right edge of the clause, as we see vestiges of auxiliary verbs that have moved to superordinate positions within the clausal domain, creating ample space between the root verb and the complementizer to allow for numerous aspectual and other enclitics to develop. The evolution of these enclitics is explained below where a plausible account of their connection to Proto-Siouan or comparative Siouan morphology exists. 

This requirement to have some element in the clause-final complementizer position also helps explain the distribution of the stem vowel /=E/, which \citet[26]{kennard1936} incorrectly calls an indefinite article and which \citet[39]{hollow1970} deems to be a meaningless element that is purely optional. Under the analysis in \citet{kasak2019}, this complementizer, and the others listed in (\ref{listofcomplementizers}) below, are certainly not optional.

The most common complementizers that appear in the corpus and in conversational Mandan involve allocutive agreement (i.e., marking agreement with the sex of the listener) or switch-reference marking (i.e., marking the clause as having the same or different subjects as the one that follows). Other complementizers exist, including those that carry some kind of aspectual or modal reading. Several complementizers here are only sparsely attested in the corpus, as they are more typically found in conversational speech and not in the kind of linguistic data that is found in the kind of register that comprises most of the corpus, i.e., narrative discourse. A list of the complementizers observed in the corpus and discussed within this chapter appears in (\ref{listofcomplementizers}) below.

\largerpage
\begin{exe}

\item\label{listofcomplementizers} List of complementizer enclitics in Mandan

\begin{xlist}
    

	\item /=ak/	different-subject switch-reference marker (\textsc{ds})
 
	\item /=ą't/ hypothetical mood complementzier (\textsc{hyp})
 
	\item /=E/ stem vowel (\textsc{sv})
 
	\item /=haa/ simultaneous aspectual complementizer (\textsc{sim})

        \item /=hak/ politeness marker (\textsc{pol})

        \item /=ki/ conditional complementizer (\textsc{cond})

        \item /=o'rą/ female-addressee interrogative marker (\textsc{int.m})

        \item /=o're/ female-addressee indicative marker (\textsc{ind.f})

        \item /=o'sh/ male-addressee indicative marker (\textsc{ind.m})

        \item /=o'sha/ male-addressee interrogative marker (\textsc{int.m})

        \item /=rą/ female-addressee imperative enclitic (\textsc{imp.f})

        \item /=rį/ same-subject switch-reference marker (\textsc{ss})

        \item /=rįk/ iterative aspectual complementizer (\textsc{iter})

        \item /=rįkų'k/ incredulitive complementizer (\textsc{incd})

        \item /=shka'rįk/ disjunctive complementizer (\textsc{disj})

        \item /=sįh/ intensive indicative complementizer (\textsc{ints})

        \item /=so'rįk/ causational complementizer (\textsc{comp.caus})

        \item /=ta/ male-addressee imperative marker (\textsc{imp.m})
% 	/=o'tiki/& \\

\end{xlist}

\end{exe}

Most of the complementizers that appear in (\ref{listofcomplementizers}) are relatively rare in the corpus. Most clause-final marking contains switch-reference markers or allocutive agreement markers. As such, some of these complementizers have very few examples compared to others.

\subsubsection{Allocutive agreement markers}\label{Ch3SubSubSecAllocutive}

Almost every sentence in Mandan requires that the sentence end with an allocutive agreement marker. The term ``allocutive'' was coined by Prince \citet[19]{bonaparte1862} to describe the kind of agreement in Basque that marks the sex or social status of the listener. One major distinction between the allocutive agreement in Mandan versus other Siouan languages is that Mandan uses allocutivity to agree with the sex of the listener, while other Siouan languages agree with the sex of the speaker. Most sentences are ungrammatical if there is no allocutive agreement marker on the matrix verb. We can organize these allocutive agreement markers by the sex of the speaker and the illocutionary force behind the utterance, as shown in \tabref{allocutivetable} below.

\begin{table}
\renewcommand\thetable{3.6}
\caption{Mandan allocutive agreement markers} 
\label{allocutivetable}


	\begin{tabular}{llll}\hline\hline
		~		&
			Indicative		&
			Interrogative	&	
			Imperative\\
\midrule
		Male	&	
			=\textit{o'sh}	&	
			=\textit{o'sha}	&	
			=\textit{ta}\\
		Female/Non-male	&	
			=\textit{o're}	&	
			=\textit{o'na}	&	
			=\textit{na}\\
			\hline\hline
		\end{tabular}

\end{table}

The non-imperative allocutive markers are made up of the copular \textit{ó'} `be' plus a determiner, locative, or discourse particle that has come to be reanalyzed as carrying allocutive semantics. These enclitics are no longer decomposable by speakers and are treated as discrete units. Given the \textsc{Fill} C constraint in Mandan, \textit{ó'} must have at one point been used as an auxiliary along with finite verbs. Inflectional morphology must have remained on the lexical verb, as there is no evidence in the corpus that this \textit{ó'} ever bore person marking. Over time, instead of being analyzed as an independent word, the copula became reanalyzed as being an integral whole with its corresponding allocutive marker. 

When addressing a woman or a group of women (including one or more people who identify as \textit{míirek} `two-spirit'), the female-addressee markers must be used. Thus, a more accurate assessment of these allocutionary markers is that they indicate that the listener is a non-male person or a group of non-male people. The male-addressee markers are used for speech directed at individual men, mixed groups, male animals, or tobacco plants. All female animals and plants that are not tobacco are addressed with the female-addressee markers. The allocutive agreement markers act as honorifics of a sort, showing respect to the listener by acknowledging their role in the speech act, even if the topics of the speech act do not involve them directly, per se.

The behavior of each allocutive agreement marker is explained in the following subsections.

\subsubsubsection{Female-addressee interrogative marker: =\textit{o'na}}

The female-addressee interrogative marker serves to indicate that a question is being directed at a woman, a group of women, a non-male individual (e.g., a two-spirit), or a group of people that does not contain men. This enclitic appears to be a combination of the copula \textit{ó'} and the topic marker =\textit{na}, historically speaking. Contemporary speakers do not decompose this enclitic into two elements. There is no known cognate with this allocutive agreement marking in any other Siouan language, though other Siouan languages have their own strategy for encoding sex in different illocutionary contexts. We can see the female-addressee interrogative marker in the data below.

\begin{exe}

\item\label{oPna} Examples of the female-address interrogative marker =\textit{o'na}

	\begin{xlist}
	
	\item\label{oPna1}
	\glll ítewetaa raréeh\textbf{o'na}?\\
	i-t-we=taa ra-rEEh=\textbf{o'rą}\\
	pv.dir-wh-indf=loc 2a-\textnormal{go.there}=\textbf{int.f}\\
	\glt `where are you going?' \citep[103]{hollow1973a}
	
	\item\label{oPna2}
	\glll wáarapakiriir\textbf{o'na}?\\
	waa-ra-pa-krii=\textbf{o'rą}\\
	\textnormal{something}-2a-ins.push-\textnormal{count}=\textbf{int.f}\\
	\glt `are you counting something?' \citep[457]{hollow1970}
	
	\item\label{oPna3}
	\glll manakíkų'tekt\textbf{o'na}?\\
	w-rą-kikų'tE=kt=\textbf{o'rą}\\
	1s-2a-\textnormal{help}=pot=\textbf{int.f}\\
	\glt `will you help me?' \citep[457]{hollow1970}

	\item\label{oPna4}
	\glll nuréeht\textbf{o'na}?\\
	rų-rEEh=t\textbf{=o'ra}̨\\
	1a.pl-\textnormal{go.there}=pot=\textbf{int.f}\\
	\glt `let's go' \citep[458]{hollow1970}
	
	\end{xlist}

\end{exe}

Very few examples of this enclitic exist in the corpus, but it is obviously quite common in everyday speech. Most of the dialog in the corpus involves men speaking to other men or to mixed groups, which explains the paucity of natural data involving =\textit{o'na}. Pedagogical materials, such as \citet{hollow1976} which contain paradigms for learners, feature this and the other female-addressee markers heavily.

Note that Mandan has no dedicated first person plural imperative, so hortative constructions like `let's go' in (\ref{oPna4}) are one strategy of conveying first person plural imperative propositions by turning a suggestion or command into a yes-no question.

\subsubsubsection{Female-addressee indicative marker: =\textit{o're}}

The female-addressee indicative marker is used when making statements to a woman or a group of women. This marker is historically a combination of the verb \textit{ó'} `be' and another element. This second element may be the proximal demonstrative \textit{re}, which seems to be cognate with the Hidatsa focus marker =\textit{ri} \citep[70]{boyle2007}, as well as \citeapos{rankin2010} reconstruction of the Proto-Dhegihan female-speaker assertion marker *{\dh}e. The Biloxi focus marker \textit{-di} is likewise cognate \citep[39]{torres2010}, as well as the indicative marker -\textit{re(e)} in Catawba \citep[53]{rudes2007a}. This wide range of cognates suggests that there was some element in Proto-Siouan that served to mark a topicalized or focused element or indicate the indicative. It is possible that the same element performed both duties, giving us the range of reflexes that we see across the Siouan language family. We can see this enclitic in use below.

\begin{exe}

\item\label{oPre} Examples of the female-addressee indicative marker =\textit{o're}

	\begin{xlist}
	
	\item\label{oPre1}
	\glll ptamíihe, wawákte\textbf{'re}\\
	w-ta-wįįh=E wa-wa-ktE=\textbf{o're}\\
	1poss-al-\textnormal{male's.sister}=sv unsp-1a-\textnormal{kill}=\textbf{ind.f}\\
	\glt `my sister, I killed something' \citep[221]{hollow1973a}

	\item\label{oPre2}
	\glll wáa'iwakisekaa maná'ke\textbf{'re}\\
	waa-i-wa-ki-sek=E wa-rą'kE=\textbf{o're}\\
	\textnormal{something}-pv.ins-1a-itr-\textnormal{make}=sv 1a-\textnormal{sit}.aux=\textbf{ind.f}\\
	\glt `I am fixing something' \citep[222]{hollow1973a}

	\item\label{oPre3}
	\glll míkaa téer\textbf{o're}, éeheerak\\
	wįk=E=$\varnothing$ tee=\textbf{o're} ee-hee=ak\\
	\textnormal{be.none}=sv=cont \textnormal{die}=\textbf{ind.f} pv-\textnormal{say}=ds\\
	\glt `he died having said nothing' \citep[63]{hollow1973a}

	\item\label{oPre4}
	\glll rarúshaa namá'kekt\textbf{o're}\\
	ra-ru-shE ra-wą'kE=kt=\textbf{o're}\\
	2a-ins.hand-\textnormal{hold} 2a-\textnormal{lie}.aux=pot=\textbf{ind.f}\\
	\glt `you should be taking them' \citep[75]{hollow1973a}
	
	\end{xlist}

\end{exe}

This enclitic is uncommon in the corpus, as much of the corpus consists of traditional narratives involving male figures. Like the other non-imperative illocutionary markers, no other enclitic can appear after =\textit{o're}; it is the final element in a matrix clause.

\subsubsubsection{Male-addressee indicative marker: =\textit{o'sh}}

Of all the verbal morphology present in the corpus, the male-addressee indicative marker =\textit{o'sh} is one of the most frequent items to appear. It is used whenever speaking to a man, a group of men, or a mixed group. The reason this marker appears most often in the corpus is that the majority of scholars who have worked on Mandan have been men, and as such, their consultants have used male-addressee marking when speaking to them. It is interesting to note that \citeapos{trechter2012b} data also features male-addressee marking despite the fact she is a woman, but Mr. Edwin Benson seems to be telling his narratives not to her, but to people in general. This choice indicates that speakers have some pragmatic control over which allocutive agreement markers they use; the allocutive argeement is not restricted to those in earshot, else \citeauthor{trechter2012b}'s data would feature mostly female-addressee morphology. We can see the behavior of =\textit{o'sh} in the data below.

\begin{exe}

\item\label{oPSexamples} Examples of the male-addressee indicative marker =\textit{o'sh}

	\begin{xlist}
	
	\item\label{oPSexamples1}
	\glll manáhinii ą́ąwe tutúharani ~ ~ ~ ~ ~ ~ ~ ~ ~ ~ ~ ~ ~ ~ ~ ~ ~ ~ ~  kí'hoomak\textbf{o'sh}\\
	wrąh\#inįį ąąwe tu$\sim$tu\#hrE=rį ~ ~ ~ ~ ~ ~ ~ ~ ~ ~ ~ ~ ~ ~ ~ ~ ~ ~ ~ ki'h=oowąk=\textbf{o'sh}\\
	\textnormal{tree}\#\textnormal{grow} \textnormal{all} dist$\sim$\textnormal{be.some}\#caus=ss ~ ~ ~ ~ ~ ~ ~ ~ ~ ~ ~ ~ ~ ~ ~ ~ ~ ~ ~ \textnormal{arrive.back.here}=narr=\textbf{ind.m}\\
	\glt `he made the springs all over the place and came back' \citep[3]{hollow1973a}

	\item\label{oPSexamples2}
	\glll numá'kaaki sikereki, miní h\'{ı̨}įre ~ ~ ~ ~ ~ ~ ~ ~ ~ ~ ~ ~ ~ ~ ~ ~ ~ ~ óma'kekere\textbf{'sh}\\
	ruwą'k-aaki si=krE=ki wrį hįį=E ~ ~ ~ ~ ~ ~ ~ ~ ~ ~ ~ ~ ~ ~ ~ ~ ~ ~ o-wą'kE=krE=\textbf{o'sh}\\
	\textnormal{man}-coll \textnormal{travel}=3pl=cond \textnormal{water} \textnormal{drink}=sv ~ ~ ~ ~ ~ ~ ~ ~ ~ ~ ~ ~ ~ ~ ~ ~ ~ ~ pv.irr-\textnormal{lie}.aux=3pl=\textbf{ind.m}\\
	\glt `there will be water there to drink when people travel' \citep[4]{hollow1973a}

	\item\label{oPSexamples3}
	\glll Numá'k Máxana ókapąte ~ ~ ~ ~ ~ ~ ~ ~ ~ ~ ~ ~ ~ ~ ~  íwarooro'xere\textbf{'sh}\\
	ruwą'k wąxrą o-ka-pąt=E ~ ~ ~ ~ ~ ~ ~ ~ ~ ~ ~ ~ ~ ~ ~  i-wa-roo=o'xrE=\textbf{o'sh}\\
	\textnormal{man} \textnormal{one} pv.irr-ins.frce-\textnormal{cultivate}=sv ~ ~ ~ ~ ~ ~ ~ ~ ~ ~ ~ ~ ~ ~ ~  pv.ins-1a-\textnormal{speak}=dub=\textbf{ind.m}\\
	\glt `I am going to talk about Lone Man's origin' \citep[5]{hollow1973a}

	\item\label{oPSexamples4}
	\glll ímaataht waréeh íwateer\textbf{o'sh}\\
	i-wąątah=t wa-rEEh i-wa-tee=\textbf{o'sh}\\
	pv.dir-\textnormal{river}=loc 1a-\textnormal{go.there} pv.ins-1a-\textnormal{like}=\textbf{ind.m}\\
	\glt `I would like to go to the river' \citep[35]{hollow1973a}
	
	\end{xlist}

\end{exe}

This enclitic, like the other non-imperative allocutive agreement markers, contains a fossilized \textit{ó'} `be' plus another element. The /ʃ/ in the coda is cognate with the declarative marker in Hidatsa, =\textit{c} \citep[197]{boyle2007}. This element is also cognate with the Tutelo asssertion marker -\textit{se} \citep[121]{einaudi1976}. All three languages show reflexes of Proto-Siouan *-se. It is not clear if this declarative *-se is related to the similitive *-se, whose reflex is the fricative in the attitudinal =\textit{nash} in Mandan, or if these were two homophonous elements. What is clear is that the geographical distance between the Tutelo of Virginia and the Mandan and Hidatsa of North Dakota make this similarity too unlikely for it to be ascribed to contact.

\subsubsubsection{Male-addressee interrogative marker: =\textit{o'sha}}

When asking a question of a man, a group of men, or a mixed group, the male-addressee interrogative marker =\textit{o'sha} is required. It is similar in phonetic shape to the indicative marker for male addressees, and given the tendency to cease phonation towards the end of the word, the final vowel is sometimes not as audible as the preceding vowel. Unlike the female-addressee markers, there is no oral-nasal contrast in the indicative and interrogative for male addressees. Examples of =\textit{o'sha} appear in the data in (\ref{oPSaexamples}) below.

\newpage
\begin{exe}

\item\label{oPSaexamples} Examples of the male-addressee interrogative marker =\textit{o'sha}

	\begin{xlist}
	
	\item\label{oPSaexamples1}
	\glll tashká reheré\textbf{'sha}, wáa'ireseke?\\
	tashka re-hrE=\textbf{o'sha} waa-i-re-sek=E\\
	\textnormal{how} 2a-caus=\textbf{int.m} nom-pv.ins-2a-\textnormal{make}=sv\\
	\glt `how did you do it, what you made?' \citep[3]{hollow1973a}

	\item\label{oPSaexamples2}
	\glll mashkáshkapka, riráse tashkáhaa éeheer\textbf{o'sha}?\\
	wą-shka$\sim$shkap=ka ri-ras=E tashka=haa ee-hee=\textbf{o'sha}\\
	unsp-dist$\sim$\textnormal{prick}=hab 2poss-\textnormal{name}=sv \textnormal{how}=ins pv-\textnormal{say}=\textbf{int.m}\\
	\glt `tomato, how does one say your name?' \citep[14]{hollow1973a}

	\item\label{oPSaexamples3}
	\glll matewé órarukų'r\textbf{o'sha}?\\
	wa-t-we o-ra-ru-kų'=\textbf{o'sha}\\
	unsp-wh-indf pv.irr-2a-ins.hand-\textnormal{give}=\textbf{int.m}\\
	\glt `what will you give for it?' \citep[29]{hollow1973a}

	\item\label{oPSaexamples4}
	\glll rah\'{ı̨}įkt\textbf{o'sha}?\\
	ra-hįį=kt=\textbf{o'sha}\\
	2a-\textnormal{drink}=pot=\textbf{int.m}\\
	\glt `are you going to drink it?' \citep[454]{hollow1970}
	
	\end{xlist}

\end{exe}

This enclitic seems to have a similar origin as the male-addressee indicative marker =\textit{o'sh} in that it has a fossilized copular \textit{ó'}, along with another element. What is not clear is whether the /ʃa/ element at the end is historically one formative or two. That is, it is not clear if the interrogative originates from the indicative plus another element, or if the morphological material after /oʔ/ comes from a single formative. 

If the interrogative marker was historically formed from the indicative, then the likely candidate for what the final vowel is could be the Proto-Siouan demonstrative *ʔa, which has reflexes as both a prefixing and suffixing element across the Siouan language family \citep{rankin2015}. All /CʔV/ sequences in Mandan and Proto-Missouri Valley result in the glottal element undergoing metathesis with the following *a vowel to create a /CVʔ/ sequence. This utterance-final glottal could have been lost due to its weak prosodic position in the utterance, rendering it harder to perceive as phonation ceased, giving us the modern interrogative marker =\textit{o'sha}.

\citet{rankin2010} posits that the material after the fossilized copula is cognate with the Lakota dubitative enclitics =\textit{so} and =\textit{se}. The fricatives match up if they are both descended from Proto-Siouan *se, but it is not clear what has caused the differences in the vowels. Further study is needed of cross-Siouan verbal and nominal morphology to determine cognates for non-lexical material.

\subsubsubsection{Female-addressee imperative marker: =\textit{na}}

Unlike the indicative or the interrogative markers, the imperative =\textit{na} has no fossilized \textit{ó'} `be' as part of the enclitic. Phonetically, the female-addressee imperative marker is identical to the topic enclitic =\textit{na}, but unlike the topic enclitic, the imperative triggers ablaut. It is not clear if the imperative marker shares a common origin with the topic enclitic, but it would be consistent with other Siouan clause-level morphology to appropriate determiners and locatives as complementizers and other utterance-level morphology.

Biloxi has a hortative \textit{na} and a homophonous strong negative imperative \textit{na}, though this enclitic does not encode features of the listener or speaker \citep[91]{einaudi1976}. \citet[222]{greer2016} reports that the polite command marker for female speakers in Chiwere is -\textit{nɛ} and the direct command marker can be -\textit{rɛ} or -\textit{r\ae}. Both of these last two forms seem similar to the underlying phonological shape of the Mandan /=rą/, but it is unclear if the relationship between these markers is circumstantial or not. Furthermore, it is unclear what the relationship is between imperative =\textit{na} and interrogative =\textit{o'na}. We can see how =\textit{na} is used in the data in (\ref{femaleimperative}) below.

\begin{exe}

\item\label{femaleimperative} Examples of the female-address imperative marker =\textit{na}

	\begin{xlist}
	
	\item\label{femaleimperative1}
	\glll ``húunita\textbf{na}!'' éeheroomako'sh\\
	huu=rįt=\textbf{rą} ee-he=oowąk=o'sh\\
	\textnormal{come.here}=2pl=\textbf{imp.f} pv-\textnormal{say}=narr=ind.m\\
	\glt `he said ``come on!''' \citep[10]{hollow1973b}

	\item\label{femaleimperative2}
	\glll múupes inák wó'kiharaa\textbf{na}!\\
	wųųpE=s irąk w-o'ki\#hrE=\textbf{rą}\\
	\textnormal{cornmush}=def \textnormal{again} unsp-\textnormal{be.cooked}\#caus=\textbf{imp.f}\\
	\glt `cook the cornmush again! \citep[178]{hollow1973b}
	
	\item\label{femaleimperative3}
	\glll hiré máatahtaa réehmaharaa\textbf{na}!\\
	hire wąątah=taa rEEh\#wą-hrE=\textbf{rą}\\
	\textnormal{now} \textnormal{river}=loc \textnormal{go.there}\#1s-{caus}=\textbf{imp.f}\\
	\glt `put me in the river now!' \citep[322]{hollow1973b}

	\item\label{femaleimperative4}
	\glll káare ké'\textbf{na}!\\
	kaare ke'=\textbf{rą}\\
	imp.neg \textnormal{dig}=\textbf{imp.f}\\
	\glt `do not dig!' \citep[305]{hollow1973b}
	
	\end{xlist}

\end{exe}

When negating an imperative, the negative imperative proclitic \textit{káare} will typically appear in first position within a clause. No negation marking appears on the verb. Marking the imperative as plural simply requires the second person plural =\textit{nit} before =\textit{na}. The only element that can follow the imperative =\textit{na} is the politeness marker =\textit{hak}, per \citet[436]{hollow1970}. The politeness marker is discussed in greater detail in \sectref{SubSecPolitenessMarker}.



\subsubsubsection{Male-addressee imperative marker: =\textit{ta}}

A command given to a man, a group of men, or a mixed group will involve =\textit{ta}. This enclitic may appear with the politeness marker =\textit{hak} when speakers wish to soften the command or give a more jocular exhortation. We can see examples of =\textit{ta} in (\ref{imperativeta}) below.

\begin{exe}

\item\label{imperativeta} Examples of the male-addressee imperative marker =\textit{ta}

	\begin{xlist}

	\item\label{imperativeta1}
	\glll mamáhenashinis\textbf{ta}\\
	wa-wą-he=rąsh=rįt=\textbf{ta}\\
	unsp-1s-\textnormal{see}=att=2pl=\textbf{imp.m}\\
	\glt `you (pl.) have got to try to see for me' \citep[35]{hollow1973a}

	\item\label{imperativeta2}
	\glll shų́ųshuka ráah\textbf{ta}!\\
	shųųshuka rEEh=\textbf{ta}\\
	\textnormal{be.direct} \textnormal{go.there}=\textbf{imp.m}\\
	\glt `go straight ahead!' \citep[35]{hollow1973a}

	\item\label{imperativeta3}
	\glll káare ą́'skaharaa\textbf{ta}\\
	kaare ą'ska\#hrE=\textbf{ta}\\
	imp.neg \textnormal{be.near}\#caus=\textbf{imp.m}\\
	\glt `do not do it that way' \citep[38]{hollow1973a}

	\item\label{imperativeta4}
	\glll íshųųhe ą́ąwe rusháa makų́'\textbf{ta}\\
	i-shųųh=E ąąwe ru-shE wą-kų'=\textbf{ta}\\
	pv.poss-\textnormal{sinew}=sv \textnormal{all} ins.hand-\textnormal{hold} 1s-\textnormal{give}=\textbf{imp.m}\\
	\glt `take all the sinew for me' \citep[78]{hollow1973a}
	 
	\end{xlist}

\end{exe}

The male-addressee imperative marker =\textit{ta} is an ablaut-triggering enclitic, just like the female-addressee imperative marker =\textit{na}. However, there is no overt nasal element to this enclitic to explain why it triggers ablaut. Several researchers have raised possibilities over the origin of ablaut in Siouan, but all proposals to date have been very preliminary or simply do not have enough cross-linguistic data to support a strong conclusion about the morpho-phonological motivation behind the manifestation of ablaut in modern Siouan languages \citep{rood1983,jones1983b,rankin1995}. It is worth noting that these cursory studies on ablaut show that imperatives are one of the few conditions under which ablaut occurs across the Siouan language family. The Mandan =\textit{ta} has the cognate \textit{ta} `male to male imperative' in Biloxi \citep[88]{einaudi1976}. What is not obvious is whether this element stems from the Proto-Siouan locative *ta(a) or if it is a variant of a reduced form of the future or potential *ktE. 

\subsubsection{Switch-reference markers}\label{SecSwitchReference}

Several Siouan languages feature a system of switch-reference, including  Crow \citep{graczyk1987}, Biloxi \citep{graczyk1997}, and Hidatsa \citep{boyle2011}. In a seminal work on the topic, \citet[ix]{haimanmunro1983} define canonical switch-reference as category on the verb where there is a morphological indication of whether the subject of that verb is identical with the subject of another verb. \citet{mixco1997b} identifies Mandan as a switch-reference language, though he states there is a distinction between realis and irrealis switch-reference marking. That point of view is not supported by the data, as there is separate irrealis marking elsewhere in the verbal complex. However, the data do corroborate \citeauthor{mixco1997b}'s hypothesis that Mandan distinguishes between same-subject and different-subject clauses through switch-reference marking. As such, we can identify two switch-reference markers in Mandan: one that is used when the subjects of sequential clauses are the same and another that is used when the following clause has a different subject. We can see these two switch-reference markers in (\ref{switchreferencemarkersinMandan}) below.

\begin{exe}

\item\label{switchreferencemarkersinMandan} Switch-reference markers in Mandan

\begin{xlist}
    
\item /=ak/ different-subject switch-reference marker

\item /=rį/ same-subject switch-reference marker

\end{xlist}

\end{exe}

The precise syntax and narrative strategies of interclausal agreement between switch-reference markers and a superordinate clause in Mandan is discussed in greater detail \sectref{Ch5switchreference}.

\subsubsubsection{Different-subject switch-reference marker: \textit{=ak}}\label{SecDSmarker}

The different-subject switch-reference marker =\textit{ak} appears whenever the subject of the following verb is different from the subject of the verb bearing \textit{=ak}. These subjects can have all the same features (i.e., identical number, person, and the like), but they will always coindex different subjects. We can see examples of this coindexation (or lack thereof) marked with switch-reference markers in (\ref{DSmarking}) below.

\begin{exe}

\item\label{DSmarking} Examples of the different-subject switch-reference marker =\textit{ak}

	\begin{xlist}
	
	\item\label{DSmarking1}
	\glll óshi\textbf{k} íwaseko'sh\\
	o-shi=\textbf{ak} i-wa-sek=o'sh\\
	pv.irr-\textnormal{be.good}=\textbf{ds} pv.ins-1a-\textnormal{make}=ind.m\\
	\glt `it\textsubscript{\textit{i}} would be good and I\textsubscript{\textit{j}} made it' \citep[4]{hollow1973a}

	\item\label{DSmarking2}
	\glll éheer\textbf{ak} ``kotewé nurátoora'shka éeheki, ~ ~ ~ ~ ~ ~ ~ ~ ~ komíma'o'r\textbf{ak},'' éeheka'ehe\\
	e-hee=\textbf{ak} ko-t-we rų-ratoo=a'shka ee-he=ki ~ ~ ~ ~ ~ ~ ~ ~ ~ ko-wį$\sim$wą-o'=\textbf{ak} ee-he=ka'ehe\\
	pv-\textnormal{say}=\textbf{ds} rel-wh-indf 1a.pl-\textnormal{be.old}=psbl pv-\textnormal{say}=cond ~ ~ ~ ~ ~ ~ ~ ~ ~ rel-aug$\sim$1s-\textnormal{be}=\textbf{ds} pv-\textnormal{say}=quot\\
	\glt `he\textsubscript{\textit{i}} said it and he\textsubscript{\textit{j}} said ``if someone\textsubscript{\textit{k}} says who\textsubscript{\textit{l}} among us is the oldest, that person\textsubscript{\textit{j}} is me,'' it is said' \citep[6]{hollow1973a}

	\item\label{DSmarking3}
	\glll máamanap\textbf{ak} wáakanaarósh, míihą't\\
	waa-wa-rąp=ak waa=krąą=o'sh wįįh=ą't\\
	unsp-1a-\textnormal{dance}=\textbf{ds} unsp-\textnormal{sing}=ind.m \textnormal{woman}=dem.anap\\
	\glt `I\textsubscript{\textit{i}} danced and that woman\textsubscript{\textit{j}} sang' \citep[224]{mixco1997b}

	\item\label{DSmarking4}
	\glll ní'mahąp\textbf{ak} wahé'sh\\
	rį-iwąhąp=\textbf{ak} wa-hE=o'sh\\
	2s-\textnormal{be.lost}=\textbf{ds} 1a-\textnormal{see}=ind.m\\
	\glt `I\textsubscript{\textit{i}} see that you\textsubscript{\textit{j}} are lost' \citep[233]{mixco1997b}
	
	\end{xlist}

\end{exe}

The most likely reason that =\textit{ak} is used so extensively throughout Mandan discourse is that third person subjects come up quite often in traditional narratives, and there is no morphological marking of third person singular. As such, when multiple individuals are involved in the discourse, it can become confusing to keep track of who did what. When the listener hears =\textit{ak} they know that the next action or state involves a different subject.

The different-subject switch-reference marker has cognates in Missouri Valley languages: =\textit{ak}/=\textit{k} in Crow \citep{graczyk2007} and =\textit{ag}/=\textit{g} in Hidatsa \citep{boyle2007}. While these forms are all cognates, there is a semantic distinction: the different-subject marker in Mandan is the same-subject marker in both Crow and Hidatsa. This term may originate from the Proto-Siouan term *ake `across, over.' No other Siouan languages shares this element as a switch-reference marker, so it is quite likely an innovation from a proto-language ancestral to both Mandan and Missouri Valley, used to signal some characteristic about continuity of a topic that became associated with a change in subject. If this proposed evolution from *ake to =\textit{ak} holds, that means that Mandan is more in line with the original semantics of *ake, and that Missouri Valley Siouan altered the meaning to mean the opposite.\footnote{Reanalyzing an adposition to indicate switch-reference can begin in languages by having quasi-switch-reference like in English, where the preposition `with' can mark a non-finite clause as having a different subject than the matrix clause, e.g., \textit{With her\textsubscript{\textit{i}} being out of the picture, she\textsubscript{\textit{j}} could finally breathe a sigh of relief}.}

\subsubsubsection{Same-subject switch-reference marker: =\textit{ni}}\label{SecSSmarker}

In direct contrast to the different-subject switch-reference marker =\textit{ak}, the same-subject switch-reference marker =\textit{ni} indicates that the verbs bearing this complementizer share the same subject as the following verb. Historically, the same-subject switch-reference marker evolved from the Proto-Siouan verb *rį `be, exist.' Clauses bearing switch-reference marking are more morphologically reduced than matrix verbs, so it is likely the case that switch-reference clauses in Mandan are not finite. \citet{kennard1936} first proposes that switch-reference markers are really participles, given the fact that speakers tend to translate them into English as adjunct clauses using participles. This analysis is not too far from the interpretation of same-subject marked clauses that I propose in \sectref{Ch5switchreference}, given the origin of =\textit{ni}.\footnote{Use of a non-finite `be' is not unique to Mandan, as this is something that is possible in English. Quasi-switch-reference marking is used in English with a non-finite verb in an adjunct clause, where the subject of that clause is identical to the matrix clause, e.g., \textit{(PRO\textsubscript{\textit{i}}) being such a healthy person, I\textsubscript{\textit{i}} always avoided ice cream}. It is possible that this kind of construction gave way to reanalysis as indicating that one clause must necessarily have the same subject as the following one over time.}

This enclitic is one of the most common morphological items in the corpus, given that much of the corpus involves certain figures undertaking deeds or going on travels alone. We can see the behavior of =\textit{ni} in the data in (\ref{switchreferenSSexamples}) below.

\begin{exe}

\item\label{switchreferenSSexamples} Examples of the same-subject switch-reference marker =\textit{ni}

	\begin{xlist}
	
	\item\label{switchreferenSSexamples1} 
	\glll Kinúma'kshi ishák máa'ų'st ~ ~ ~ ~ ~ ~ ~ ~ ~ ~ ~ ~ ~ ~ ~ íwahuure rá'shoti\textbf{ni} ~ ~ ~ ~ ~ ~ ~ ~ ~ ~ ~ ~ ~ ~ ~ mákoomako'sh\\
	ki-ruwą'k\#shi ishak waa-ų't=t ~ ~ ~ ~ ~ ~ ~ ~ ~ ~ ~ ~ ~ ~ ~ i-wa-huu=E ra'-shot=\textbf{rį} ~ ~ ~ ~ ~ ~ ~ ~ ~ ~ ~ ~ ~ ~ ~ wąk=oowąk=o'sh\\
	mid-\textnormal{man}\#\textnormal{be.good} 3pro nom-\textnormal{be.in.past}=loc ~ ~ ~ ~ ~ ~ ~ ~ ~ ~ ~ ~ ~ ~ ~ pv.poss-unsp-\textnormal{bone}=sv ins.heat-\textnormal{be.white}=\textbf{ss} ~ ~ ~ ~ ~ ~ ~ ~ ~ ~ ~ ~ ~ ~ ~ lie.pos=narr=ind.m\\
	\glt `First Creator's bones\textsubscript{\textit{i}} had already turned white, and they\textsubscript{\textit{i}} were lying there' \citep[1]{hollow1973a}

	\item\label{switchreferenSSexamples2} 
	\glll háki nuráahi\textbf{ni} ríixatinisto'sh\\
	ha=ki rų-rEEh=\textbf{rį} rV-i-xat=rįt=t=o'sh\\
	prov=cond 1a.pl-\textnormal{go.there}=\textbf{ss} 1a.pl-pv.ins-\textnormal{inspect}=2pl=pot=ind.m\\
	\glt `so, we\textsubscript{\textit{i}} will go over there and we\textsubscript{\textit{i}} will look over it' \citep[11]{hollow1973a}

	\item\label{switchreferenSSexamples3}
	\glll minís éena waharáa\textbf{ni} waptého'sh\\
	wrįs ee=rą wa-hrE=\textbf{rį} wa-ptEh=o'sh\\
	\textnormal{horse} dem.dis=top 1a-\textnormal{see}=\textbf{ss} 1a-\textnormal{run}=ind.m\\
	\glt `I\textsubscript{\textit{i}} saw the horse and I\textsubscript{\textit{i}} ran away' \citep[5]{mixco1997b} 

	\item\label{switchreferenSSexamples4}
	\glll wáapshixi\textbf{ni} wáashotinixo'sh\\
	waa-pshi=xi=\textbf{rį} waa-shot=rįx=o'sh\\
	neg-\textnormal{be.black}=neg=\textbf{ss} neg-\textnormal{be.white}=neg=ind.m\\
	\glt `it\textsubscript{\textit{i}} is not black and it\textsubscript{\textit{i}} is not white' \citep[5]{mixco1997b}
	
	\end{xlist}

\end{exe}

In each of the examples above, we can see that the subjects are coindexed. The only determining factor over using =\textit{ni} or =\textit{ak} lies in whether the following subject matches the current subject. It is perfectly possible for a subject to switch its reference and then switch it back to the previous subject. We can see an example of this phenomenon in (\ref{SRswitchback}) below.

\begin{exe}

\item\label{SRswitchback} Switching between switch-reference markers

\glll miníseena ráahini maná nákak máapehaa ~ ~ ~ réeho'sh\\
	wrįs=s=ee=rą rEEh=rį wrą rąk=rį wąąpe=haa ~ ~ ~ rEEh=o'sh\\
	\textnormal{horse}=def=dem.dist=top \textnormal{go.there}=ss \textnormal{tree} pos.sit=ss \textnormal{under}=sim ~ ~ ~ \textnormal{go.there}=ind.m\\
	\glt `the horse passed beneath the tree' (lit. the horse\textsubscript{\textit{i}} went and a tree\textsubscript{\textit{j}} sat there and the horse\textsubscript{\textit{i}} went under it\textsubscript{\textit{j}})' \citep[226]{mixco1997b}

\end{exe}

The horse in the above example is the subject of the initial verb \textit{ráahini} `went' and the matrix verb \textit{réeho'sh} `went', while the subject of the second verb \textit{ná'kak} `sit' is the tree. Even though the first and last verbs have the same subjects, the fact that there is an intervening different subject necessitates the use of a different-subject switch-reference marker to indicate a transition from one subject to another. There is nothing about the construction in (\ref{SRswitchback}) that automatically tells the listener that the diffent-subject switch-reference marker is switching the reference back to the previous subject; it merely indicates that the current subject (i.e., the tree) is not the subject of the following verb. It is left to inference that the horse is the subject of the final verb, as it is equally plausible that a third subject could be involved. As such, the speaker and the listen are relying on the information that is already available in the discourse to both understand what subject is taking what action and when.



\subsubsection{Hypothetical mood complementizer: =\textit{ą't}}\label{SecHypothetical}
\largerpage
The hypothetical mood complementizer is derived from the anaphoric determiner \textit{ą't} `that'. Like many other Siouan languages, determiners and locatives are often reanalyzed as clause-level morphology. In this case, the hypothetical =\textit{ą't} indicates a kind of conditional reading where the speaker is expressing the conditions that could lead to an event or to mark contrary conditions. It is sparsely attested in the corpus. We can see examples of this enclitic in (\ref{hypotheticalexamples}) below.

\begin{exe}

\item\label{hypotheticalexamples} Examples of the hypothetical mood complementizer =\textit{ą't}

	\begin{xlist}
	
	\item\label{hypotheticalexamples1}
	\glll kiríkereki, ókina'kara\textbf{'t}\\
	kri=krE=ki o-kirą'=krE=\textbf{ą't}\\
	\textnormal{arrive.back.there}=3pl=cond pv.irr-\textnormal{tell}=3pl=\textbf{hyp}\\
	\glt `they would say so if they were to get back' \citep[20]{kennard1936}

	\item\label{hypotheticalexamples2}
	\glll róo wakxų́hki ó'irahek\textbf{ą't}\\
	roo wa-kxųh=ki o-i-ra-hek=\textbf{ą't}\\
	dem.mid 1a-\textnormal{lie.down}=cond pv.irr-pv.ins-2a-\textnormal{know}=\textbf{hyp}\\
	\glt `you would know it if I were to lie down here' \citep[1]{hollow1973a}

	\item\label{hypotheticalexamples3}
	\glll téehąt waréeh\textbf{ą't}\\
	teehą=t wa-rEEh=\textbf{ą't}\\
	\textnormal{be.far}=loc 1a-\textnormal{go.there}=\textbf{hyp}\\
	\glt `I would go a long way' \citep[146]{hollow1973b}
	
	\item\label{hypotheticalexamples4}
	\glll hiré tashká'eshkak ą́'ska rahereka'sha, mí'he ~ ~ ~ ~ ~  ~ tą́ąhąą ní'hka\textbf{'t}\\
	hire tashka-eshka=ak ą'ska ra-hrE=ka=o'sha wį'h=E ~ ~ ~ ~ ~ ~ tąą=hąą r'-įįh=ka=\textbf{ą't}\\
	\textnormal{now} \textnormal{how}-smlt=ds \textnormal{be.near} 2a-caus=hab=int.m \textnormal{robe}=sv ~ ~ ~ ~ ~ ~  \textnormal{be.different}=ins 2s-\textnormal{wear.about.shoulders}=hab=\textbf{hyp}\\
	\glt `how come you are always doing it this way now, would you always cover yourself with a different robe?' \citep[240]{hollow1973b}
	
	\end{xlist}

\end{exe}

In terms of usage, the hypothetical is often accompanied by a conditional complementizer =\textit{ki}, creating a construction where a condition is raised and then what would hypothetically happen is proposed. The hypothetical can also appear without a conditional, but in those cases, there is some implied conditional, like in (\ref{hypotheticalexamples3}), where the speaker is tied up, high above the ground and then looks down and wonders how far he would have to travel to get back down. The hypothetical can also be used independent of a conditional when soliciting reasons why someone does something, as we see in (\ref{hypotheticalexamples4}). The speaker confronts the listener who is always wearing a different robe after he comes home from sneaking out all night, and remarks why is it that he always would be wearing a different robe.

The hypothetical complementizer triggers ablaut for most speakers, but as we can see in (\ref{hypotheticalexamples3}), this is not the case for all speakers. As we have previously discussed with respect to negation, it seems that not all Mandan speakers treat ablauting enclitics the same. There is also a tendency for the nasalization in =\textit{ą't} to be pronounced very weakly, and it may be that nasal-initial enclitics in Mandan tend to lose nasality when cliticized.

\subsubsection{Stem vowel: =\textit{e}}\label{stemvowel}

One of the largest outstanding issues with Mandan morphology has been the status of the [e] that appears word-finally on nouns and verbs alike; sometimes, it is there, and other times, it is not. Speakers have not been able to articulate a meaning for this ending. \citet[26]{kennard1936} says that it is an indefinite article, while \citet[39]{hollow1970} says that it has no meaning and can just be optionally added at the end of any consonant-final stem. \citet[15]{mixco1997a} just calls it a stem vowel, and does not assign it any meaning and matches \citeauthor{hollow1970}'s opinion that it has no meaning of its own. However, when we start to match up the transcribed data with audio, a pattern begins to emerge. Namely, the stem vowel appears at the boundary of an intonational phrase, and there is a prosodic break between the item bearing the stem vowel and the rest of the utterance.

When working with learners, it is quite challenging to explain when this element must appear and when it must not. When eliciting words for a word list, items ending in a consonant or a long vowel usually have a stem vowel at the end. When these words are placed in the context of a sentence, the stem vowel does not appear unless there is some prosodic break. This distinction resembles the issue of citation forms versus stem forms in Crow. The citation form is used for free word forms when someone asks how to say something in Crow or give a one-word answer. The stem form is the form upon which all other morphology is added, and as such, citation forms are not common in daily discourse \citep[30]{graczyk2007}. The citation forms in Mandan are words bearing a stem vowel, because Mandan requires that some material be present in the complementizer position. The stem vowel acts as a complementizer, serving to mark the edge of an intonational phrase if no other material is available (see the data in (\ref{stemvowelexample1}) below). In this way, we can tell the difference between a fragment in Mandan and a complete utterance, because the fragment would lack the stem vowel. In the data below, we see pairs of words that have no difference in meaning, but have differences in form. Stem forms (i.e., those that can be present without /=E/) are those that consist of just the stem and can be found within some kind of phrase structure. Citation forms (i.e., those that end in /=E/) are those that can be found in isolation. We can see a comparison of some Mandan words in their stem forms and citation forms in (\ref{barevscitation}) below.

\begin{exe}

\item\label{barevscitation} Stem versus citation forms

	\begin{xlist}

	\item \textit{kók} $\sim$ \textit{kóke} `pronghorn antelope'
	
	\item \textit{ratáx} $\sim$ \textit{ratáxe} `to cry out'

	\item \textit{réeh} $\sim$ \textit{réehe} `to go there'

	\item \textit{músh} $\sim$ \textit{múshe} `buttocks'
		
	\item \textit{imáa} $\sim$ \textit{imáare} `body'

	\item \textit{ké'} $\sim$ \textit{ké're} `to dig'

	
	\end{xlist}

\end{exe}

The stem vowel, appearing in the complementizer position, triggers epenthetic [ɾ] when following a stem ending in a long vowel or a glottal stop (see discussion of hiatus resolution at phrasal boundary in \sectref{intrusiveR}). When looking across the language family, we see that this [ɾ]-epenthesis is not solely a Nu'etaare innovation,  but may ultimately be grounded in a similar process in Proto-Siouan.\footnote{As previously discussed by \citet{carter1991a}, the Ruptaare dialect does not have [ɾ]-epenthesis, but uses [ʔ]-epenthesis for both word-internal hiatus resolution as well as at enclitic boundaries.} Numerous reconstructed stems in Proto-Siouan typically end in what \citet{rankin2015} call the ``common suffix'' *-re. This common suffix is almost exclusively found after long vowels in Proto-Siouan, and in the cases that it is not, there is debate over the vowel's length. In most Siouan languages, this common suffix is not present, but in disparate branches of the family, we see that it survives and exists in contextually-dependent doublets much like in Mandan.

\citet{rankin2010} remarks that Mandan has been difficult to contextualize within the Siouan language family because so much of its morphology can be attributed to Proto-Siouan rather than being an innovation in Mandan. In this respect, Mandan shows itself to be particularly conservative, morphologically speaking. My proposal is that this common suffix represented a productive process in Proto-Siouan whereby there was some restriction on the environments where a long vowel could appear, and that this process is still productive in Mandan. Namely, there is a restriction against long vowels appearing at the right edge of some structural or prosodic domain, and as such, there is some epenthetic element that is generated to repair such illicit utterances. In Mandan, intonational phrases cannot end in a long vowel, which necessitates the insertion of the short stem vowel /{=E}/, as demonstrated in the example in (\ref{stemvowelexample1}) below.

\begin{exe}
\item\label{stemvowelexample1} Stem vowel encliticization at the edge of intonational phrases

\glll	Aríkara kirúuhka'eheero'sh, óhuur\textbf{e}.\\
	arikra kiruuh=ka'ehe=o'sh o-huu=\textbf{E}\\
	\textnormal{Arikara} \textnormal{refuse}=quot=ind.m pv.irr-\textnormal{come.here}=\textbf{sv}\\
\glt	`The Arikara refused to come, they say' \citep[48]{hollow1973a}
\end{exe}

Mandan normally allows the action being refused to be elided, so \textit{Aríkara kirúuhka'eheero'sh} in of itself is a complete utterance that is better translated as `The Arikara refused to do so, they say', and the word \textit{óhuure} `to come' is dislocated to the right as parenthetical information, indicating that it is an afterthought by the speaker. These two clauses are nested within different intonational phrases, and as such, the stem vowel is required with the right dislocated clause, and the presence of the stem vowel next to the long vowel triggers [ɾ]-epenthesis.

This same process has fossilized in other Siouan languages where the form with the stem vowel has become reanalyzed as being a single morphological item, e.g., Proto-Siouan *sii(-re) `yellow' becomes \textit{shíili} in Crow and \textit{cíiri} in Hidatsa, \textit{síidi} in Biloxi, and \textit{siri} `clear' in Catawba. The Mandan reflex of this is either \textit{síi} or \textit{síire}, depending on its context. Further research is needed into other Siouan languages to confirm this, but given the strong tendency among Siouan languages to have some manner of morphological material following the verb in complete sentences, it is likely that these unexplained ``optional'' vowels are not really optional at all, and that this common suffix in Siouanist literature is not a suffix, but a phrasal enclitic that is filling the complementizer slot.

The presence of the stem vowel is an indicator of a prosodic break, most commonly associated with a topicalized element or a parenthetical element. We can see this in the data in (\ref{SVcitationforms}) below, where the presence of /=E/ matches with the shifts in intonation patterns associated with topicalization.

\begin{exe}
\item\label{SVcitationforms} Prosodic breaks and /=E/
    \begin{xlist}
    \item\label{SVcitationforms1} \glll Kįnúma'kshi éeheni Numá'k Máxana íwahekanash\textbf{e} wakína'ni éewereho'sh\\
    ki-ruwą'k\#shi ee-he=rį ruwą'k wąxrą i-wa-hek=rąsh=E wa-kirą'=rį ee-we-reh=o'sh\\
    mid-\textnormal{man}\#\textnormal{good} pv-\textnormal{say}=ss \textnormal{man} \textnormal{one} pv.ins-1a-\textnormal{know}=att=sv 1a-\textnormal{tell}=ss pv-1a-\textnormal{want}=ind.m\\
    \glt `I want to tell what I know about First Creator and Lone Man' (lit. `what I know about First Creator and Lone Man, I want to tell it to you.') \citep[1]{hollow1973a}
    
    \item\label{SVcitationforms2} \glll ``mí'miratooro'sh,'' éeheerak, Kįnúma'kshi koratoor\textbf{e} ée ó'roomako'sh\\
    wį'$\sim$wį-ratoo=o'sh ee-hee=ak ki-ruwą'k\#shi ko-ratoo=\textbf{E} ee o'=oowąk=o'sh\\
    aug$\sim$1s-\textnormal{be.elder}=ind.m pv-\textnormal{say}=ds mid-\textnormal{man}\#\textnormal{good} rel-\textnormal{be.elder}=\textbf{sv} dem.dist \textnormal{be}=narr=ind.m\\
    \glt `{``}I am definitely the oldest,'' he said and First Creator was the one who was the elder.' (lit. `{``}I am definitely the oldest,'' he said, First Creator did, the elder one, he was the one.') \citep[2]{hollow1973a}
    
    \item\label{SVcitationforms3} \glll h\'{ı̨}į, tashká'sha, máa'ąk\textbf{e} írasek?\\
    hįį ta=shka=o'sha waa'ąk=\textbf{E} i-ra-sek\\
    \textnormal{well} wh=smlt=int.m \textnormal{earth}=\textbf{sv} pv.ins-2a-\textnormal{make}\\
    \glt `well, how is the land you made?' [lit. `well, how is it, the land, you made it...?'] \citep[10]{hollow1973a}
    
    \end{xlist}

\end{exe}

In (\ref{SVcitationforms1}), we see that the stem vowel marks the right edge of an intonational phrase. The literal translation of this example does a better job capturing the more layered structure of the sentence in the sense that it is not just clearly subject--object--verb; we have a fronted element (i.e., `[what I know about First Creator,] I want to tell it') rather than a subordinated element (i.e., `I want to tell [what I know about First Creator]'). We see a similar situation in (\ref{SVcitationforms2}), where the sentence ends with `he said, First Creator did, the elder one, he was the one', where we see First Creator topicalized, then we see `the elder one' topicalized as well, followed by `he was the one.' This style of speech does not conform one-to-one to academic English, rendering these kinds of subtleties of Mandan discourse structure opaque. The use of the stem vowel on `earth' in (\ref{SVcitationforms3}) and the fact that an enclitic-less \textit{írasek} `you made it' following it signals to the listener that this sentence involves the speaker is trailing off when narrating this situation, after adding these afterthoughts to clarify the question being asked.

The stem vowel acts as a complementizer that is used as a last resort to indicate the end of a clause at the end of an intonational phrase. There is thus the difference between Mandan speakers producing so-called ``citation forms'' during vocabulary elicitation that do not match the ``stem forms'' used in actual speech most of the time. We can see more about how these citation forms manifest in Mandan discourse structure in chapter \ref{chapter5}.

\subsubsection{Simultaneous aspectual complementizer: \textit{=haa}}\label{simultaneousaspectualenclitic}

As previously discussed in \sectref{ParaContinuative2}, many previous scholars have misanalyzed the ablauted stem vowel or the initial vowel in =\textit{ąmi} for =\textit{haa}, the simultaneous aspectual complementizer. In reality, this enclitic is quite rare in the corpus. The paucity of times that =\textit{haa} appears in the corpus lines up with the rarity of the adverbial subordinator or simultaneous marker =\textit{haa} in Hidatsa \citep[530]{park2012}. Consultants will typically translate clauses marked with \textit{=haa} as involving the English word `while'. This formative has cognates across the Siouan language family, typically realized as =\textit{ha} or =\textit{hą}, though this particular enclitic clearly derives from the Proto-Siouan adverbializer, *-haa \citep{rankin2015}. The behavior of this enclitic is observed in (\ref{simultaneousaspectualenclitic}) below.

\begin{exe}

\item\label{simultaneousaspectualencliticEx} Examples of the simultaneous aspectual enclitic =\textit{haa}

	\begin{xlist}
	
	\item\label{simultaneousaspectualenclitic1} 
	
	\glll inák kináaka\textbf{haa} kináhka kawópąx, ~ ~   Kįnúma'kshi, ishák máa'ų'st íwahuure ~ rá'shootini mákoomako'sh\\
	irąk ki-rąąkE=\textbf{haa} ki-rąk=ka ka-wopąx ~ ~  ki-ruwą'k\#shi ishak waa-ų't=t i-wa-huu=E ~ ra'-shoot=rį wąk=oowąk=o'sh\\
	\textnormal{again} mid-\textnormal{be.new}=\textbf{sim} mid-pos.sit=hab ins.frce-\textnormal{stand.upright} ~ ~  mid-\textnormal{man}\#\textnormal{be.good} 3pro nom-\textnormal{be.in.past}=loc 3poss-unsp-\textnormal{bone}=sv ~ ins.heat-\textnormal{be.white}=ss pos.lie=narr=ind.m\\
	\glt `while becoming like new again, [his staff] was there, he put it up, First Creator, his bones already turned white and were lying there' \citep[1]{hollow1973a}

	\item\label{simultaneousaspectualenclitic2} 

	\glll ų́'shkaherek, kikínaaka\textbf{haa}, ~ ~ ~ ~ ~ ~ ~ ~ ~ ~ ~ ~ ~ ~ ~ ~ ~ ~ ~ ~  ``mimíratooro'sh,'' éeheni...\\
	ų'sh=ka\#hrE=ak ki-ki-rąąkE=haa, ~ ~ ~ ~ ~ ~ ~ ~ ~ ~ ~ ~ ~ ~ ~ ~ ~ ~ ~ ~ wį$\sim$wį-ratoo=o'sh ee-he=rį\\
	\textnormal{be.thus}=hab\#caus=ds itr-mid-\textnormal{be.new}=sim ~ ~ ~ ~ ~ ~ ~ ~ ~ ~ ~ ~ ~ ~ ~ ~ ~ ~ ~ ~ aug$\sim$1s-\textnormal{be.mature}=ind.m pv-\textnormal{say}=ss\\
	\glt `having so done it, while it got new again, he said `I am the oldest' and...' \citep[2]{hollow1973a}
	

	\item\label{simultaneousaspectualenclitic3} 

	\glll xamá\textbf{haa} óxast ó'taaharani ~ ~ ~ ~ ~ ~ ~ ~ ~ ~  kohų́ųxi'hoo ~ ~ ~ ~ ~ ~ ~ ~ ~ ~ ~ ~ ~ ~ ~  ~ ~ ~ ~ ~  \'{ı̨}'pataxteka'sh\\
	xwąh=\textbf{haa} o-xat=t o'=taa\#hrE=rį ~ ~ ~ ~ ~ ~ ~ ~ ~ ~  ko-hųų\#xi'h=oo=rą ~ ~ ~ ~ ~ ~ ~ ~ ~ ~ ~ ~ ~ ~ ~  ~ ~ ~ ~ ~  į'-pa-ta-xtE=ka=o'sh\\
	\textnormal{be.small}=\textbf{sim} pv.irr-\textnormal{society}=loc \textnormal{be}=loc\#caus=ss ~ ~ ~ ~ ~ ~ ~ ~ ~ ~  3poss.pers-\textnormal{mother}\#\textnormal{be.old}=dem.mid=top ~ ~ ~ ~ ~ ~ ~ ~ ~ ~ ~ ~ ~ ~ ~ ~ ~ ~ ~ ~  pv.rflx-ins.push-\textnormal{push}-aug=hab=ind.m\\
	\glt `while he was small, when they had society doings, she made him be in them and his grandmother was always really proud of him' \citep[64]{hollow1973a}

	\item\label{simultaneousaspectualenclitic4} 
 	\glll ímashute kų́'shtaa\textbf{haa} íwataraakini ~ ~ ~ ~ ~ ká'ni náakaa\\
	i-wąshut=E kų'sh=taa=haa i-wa'-traak=rį ~ ~ ~ ~ ~ ka'=rį rąąkE=$\varnothing$\\
	pv.ins-\textnormal{clothe}=sv \textnormal{be.inside}=loc=\textbf{sim} pv.ins-ins.prce-\textnormal{shut}=ss ~ ~ ~ ~ ~  \textnormal{possess}=ss \textnormal{sit}.aux=sv\\
	\glt `her dress, while she had it inside, she sewed it on and she had it there' \citep[106]{hollow1973a}
	
	\item\label{simultaneousaspectualenclitic5} 
	\glll íkų'taa\textbf{haa} makú'ta!\\
	i-kų'tE=\textbf{haa} wą-ku'=ta\\
	pv.dir-\textnormal{throw}=\textbf{sim} 1s-\textnormal{give}=imp.m\\
	\glt `throw it to me!' \citep[132]{hollow1973a}

	\item\label{simultaneousaspectualenclitic6} \glll weréxanash írapawe\textbf{haa} ~ ~ ~ ~ ~ ~ ~ ~ ~ ~ ~ ~ ~ ~ ~ h\'{ı̨}įmanaherekto're\\
	wrex=rąsh i-ra-pa-weh=\textbf{haa} ~ ~ ~ ~ ~ ~ ~ ~ ~ ~ ~ ~ ~ ~ ~ hįį\#w-rą-hrE=kt=o're\\
	\textnormal{kettle}=att pv.ins-2a-ins.push-\textnormal{hold.up}=\textbf{sim} ~ ~ ~ ~ ~ ~ ~ ~ ~ ~ ~ ~ ~ ~ ~  \textnormal{drink}\#1s-2a-caus=pot=ind.m\\
	\glt `you should let me drink while you hold out the pail' \citep[131]{hollow1973a}

	\end{xlist}

\end{exe}

It is easy to confuse this enclitic with the ablauted continuative in casual speech, given that they both end in a long /aa/, in particular when the stem ends with /h/, as is the case in (\ref{simultaneousaspectualenclitic3}). This marker triggers ablaut in /E/- and /EE/-final stems, as we see in (\ref{simultaneousaspectualenclitic5}). The fact that /hh/ sequences in Mandan also simplify to [h] further obscures when a speaker is using /=haa/ versus an /h/-final stem that contains an ablauted /=E/. The semantics of both these enclitics seem quite similar. One distinction is that /=E/ with ablaut indicates a serial verb construction, while /=haa/ is often accompanied by an intonational shift or pause.

\subsubsection{Politeness marker: =\textit{hak}}\label{SubSecPolitenessMarker}

Mandan speakers tend to utilize very explicit and transparent strategies when it comes to verbal communication. The way to say `good-bye' in Mandan, for instance, is just to tell the people you are with that you are leaving, i.e., \textit{waréeho'sh} or \textit{waréeho're} `I am going.' One of the few overt manifestations of politeness in Mandan can be seen with imperatives. When issuing commands or requests, speakers may use the politeness marker =\textit{hak} after the imperative marker. This post-imperative enclitic is the closest equivalent to the English word `please' in Mandan, which serves to soften a command or to make a request sound less pressing. We can see its usage in the data in (\ref{politenessmarker}) below.

\begin{exe}

\item\label{politenessmarker} Examples of the politeness marker =\textit{hak}

	\begin{xlist}
	
	\item\label{politenessmarker1}
	\glll  húuta\textbf{hak}; manakíkų'tekto'sh\\
	huu=ta=\textbf{hak} w-rą-kikų'tE=kt=o'sh\\
	\textnormal{come.here}=imp.m=\textbf{pol} 1s-2a-\textnormal{help}=pot=ind.m\\
	\glt `come on, please; you can help me' \citep[41]{hollow1973a}

	\item\label{politenessmarker2}
	\glll waráahta\textbf{hak}!\\
	wa-rEEh=ta=\textbf{hak}\\
	unsp-\textnormal{go.there}=imp.m=\textbf{pol}\\
	\glt `go right on ahead!' \citep[265]{hollow1973b}

	\item\label{politenessmarker3}
	\glll makína'na\textbf{hak}\\
	wą-kirą'=rą=\textbf{hak}\\
	1s-\textnormal{tell}=imp.f=\textbf{pol}\\
	\glt `please tell me' \citep[436]{hollow1970}

	\item\label{politenessmarker4}
	\glll káare pawéshinista\textbf{hak}\\
	kaare pa-wesh=rįt=ta=\textbf{hak}\\
	imp.neg ins.push-\textnormal{cut}=2pl=imp.m=\textbf{pol}\\
	\glt `please don't any of you cut it' \citep[436]{hollow1970}
	
	\end{xlist}

\end{exe}

In the corpus, this enclitic is very rare, but it is not uncommon in daily conversations. The politeness marker =\textit{hak} can be used to soften a command, to demonstrate friendliness, or show respect. Even when used when speaking to elders, =\textit{hak} is not inherently used with formal register, so it is not the case that one must use =\textit{hak} exclusively when using imperatives with people to whom you are showing respect. This enclitic is not used outside of imperatives, and there is no equivalent of `please' when used with questions or entreaties. It is not clear what the origin of this enclitic is, though it is possibly related to the standing positional \textit{hą́k}. Another possibility is that it is related to the Proto-Siouan contrastive marker *ha, which is also seen in the Mandan pro-verb \textit{háki} `but, however.'

\subsubsection{Conditional complementizer: =\textit{ki}}

The conditional complementizer =\textit{ki} is often translated as `if' or `when' in the corpus. \citet{mixco1997a,mixco1997b} argues that this formative is a different-subject irrealis switch-reference marker. However, the conditional enclitic =\textit{ki} is used with both same-subject and different-subject clauses, so that analysis does not hold. The conditional complementizer is a reflex of the Proto-Siouan definite article *kį, which also serves as a subordinator. Mandan no longer uses a reflex of *kį as a definite article, but it has continued to use it as a complementizer, though its semantics have changed to only be used in conditional clauses. We can see this use of =\textit{ki} in (\ref{conditionalki}) below.

\begin{exe}

\item\label{conditionalki} Examples of the conditional complementizer =\textit{ki}

	\begin{xlist}
	
	\item\label{conditionalki1} 
	\glll áakotewe\textbf{ki} órookti óshik ~ ~ ~ ~ ~ ~ ~ ~ ~ ~ ~ ~ ~ ~ ~  íwaseko'sh\\
	aakotewe=\textbf{ki} o-rookti o-shi=ak ~ ~ ~ ~ ~ ~ ~ ~ ~ ~ ~ ~ ~ ~ ~  i-wa-sek=o'sh\\
	\textnormal{shelter}=\textbf{cond} pv.irr-\textnormal{camp} pv.irr-\textnormal{be.good}=ds ~ ~ ~ ~ ~ ~ ~ ~ ~ ~ ~ ~ ~ ~ ~  pv.ins-1a-\textnormal{make}=ind.m\\
	\glt `it would be good if they had a shelter for camping and I made it' \citep[3]{hollow1973a}

	\item\label{conditionalki2}
	\glll numá'kshi\textbf{ki} ráse núpo'sh, ótu'sh\\
	ruwą'k\#shi=\textbf{ki} ras=e rųp=o'sh o-tu=o'sh\\
	\textnormal{man}\#\textnormal{be.good}=\textbf{cond} \textnormal{name}=sv \textnormal{two}=ind.m pv.irr-\textnormal{be.some}=ind.m\\
	\glt `if he is a chief, then he has two names' \citep[14]{hollow1973a}

	\item\label{conditionalki3}
	\glll íninah\textbf{ki} ó'ų'ka'sh\\
	i-rį-rąk=\textbf{ki} o-ų'=ka=o'sh\\
	pv.ins-2s-sit.pos=\textbf{cond} pv.irr-\textnormal{be.thus}=hab=ind.m\\
	\glt `when you are out of sight, that will be enough' \citep[25]{hollow1973a}

	\item\label{conditionalki4}
	\glll Kinúma'kshi kasúh\textbf{ki} súk hų́'na ~ ~ ~ ~ ~ miníxa máakaho'sh\\
	ki-ruwą'k\#shi ka-suk=\textbf{ki} suk hų=o'=rą ~ ~ ~ ~ ~ wrįx=E wąąkah=o'sh\\
	mid-\textnormal{man}\#\textnormal{be.good} ins.frce-\textnormal{exit}=\textbf{cond} \textnormal{child} \textnormal{be.many}=\textnormal{be}=\textnormal{top} ~ ~ ~ ~ ~ \textnormal{play}=sv \textnormal{lie}.aux.hab=ind.m\\
	\glt `when Old Man Coyote peeked out, there were a lot of children who were playing there' \citep[28]{hollow1973a} 

	\item\label{conditionalki5}
	\glll síkere\textbf{ki} miní kíikaraakere'sh\\
	si=krE=\textbf{ki} wrį kiikraa=krE=o'sh\\
	\textnormal{travel}=3pl=\textbf{cond} \textnormal{water} \textnormal{look.for}=3pl=ind.m\\
	\glt `when they travel, they look for water' \citep[451]{hollow1970}

	\item\label{conditionalki6}
	\glll numá'kaaki éena máaskap ~ ~ ~ ~ ~ ~ ~ ~ ~ ~ ~ ~ ~ ~ ~ ~ ~ írukapkerekti\textbf{ki}, pt\'{ı̨}į ~ ~ ~ ~ ~ ~ ~ ~ ~ ~ ~ ~ ~ ~ híherekaroomako'sh\\
	nuwą'k-aaki ee=rą wąąskap ~ ~ ~ ~ ~ ~ ~ ~ ~ ~ ~ ~ ~ ~ ~ ~ ~ i-ru-kap=krE=kti=\textbf{ki} ptįį ~ ~ ~ ~ ~ ~ ~ ~ ~ ~ ~ ~ ~ ~  hi\#hrE=ka=oowąk=o'sh\\
	\textnormal{man}-coll dem.dist=top \textnormal{meat} ~ ~ ~ ~ ~ ~ ~ ~ ~ ~ ~ ~ ~ ~ ~ ~ ~  pv.ins-ins.hand-\textnormal{lack}=3pl=pot=\textbf{cond} \textnormal{buffalo} ~ ~ ~ ~ ~ ~ ~ ~ ~ ~ ~ ~ ~ ~  \textnormal{arrive.here}\#caus=hab=narr=ind.m\\
	\glt `whenever the people were hard up for meat, he always made the buffalo come' \citep[301]{hollow1973b}
	
	\end{xlist}

\end{exe}

As we can see in the data above for (\ref{conditionalki2}) and (\ref{conditionalki5}) and unlike the hypothetical modal enclitic, the conditional complementizer relies on temporal subordination. That is, it invokes a condition, and if met, the action or state in the superordinate clause does or would happen. Furthermore, this conditional is not restricted to irrealis propositions, as we see in (\ref{conditionalki5}) and (\ref{conditionalki6}), which describe situations that happen habitually or customarily. 

\subsubsection{Iterative aspectual complementizer: =\textit{nik}}

Mandan has several morphological markers of iterativity: the prefix \textit{ki}-, the aspectual enclitic =\textit{ske}, as well as the complementizer =\textit{nik}. The latter marks a subordinate clause and typically is used with superordinate clauses expressing habits or customs. It can be roughly glossed as `when', `whenever', or `each time.' We can see examples of =\textit{nik} in (\ref{nikexamples}) below.

\begin{exe}

\item\label{nikexamples} Examples of the iterative aspectual complementizer =\textit{nik}

	\begin{xlist}
	
	\item\label{nikexamples1}
	\glll ruptáahaa súki\textbf{nik}, réehkaroomaksįh\\
	ru-ptEh=haa suk=\textbf{rįk} rEEh=oowąk=sįh\\
	ins.hand-\textnormal{run}=sim \textnormal{exit}=\textbf{iter} \textnormal{go.there}=narr=ints\\
	\glt `he always leaves when he goes out turning around' \citep[88]{hollow1973a}

	\item\label{nikexamples2}
	\glll kirútiniitaa\textbf{nik}, ų́'sh hą́'ke'sh\\
	k-rut=rįįtE=\textbf{rįk} ų'sh hą'kE=o'sh\\
	suus-\textnormal{eat}=cel=\textbf{iter} \textnormal{be.thus} \textnormal{stand}.aux=ind.m\\
	\glt `that was the way it was when she would be nibbling at it' \citep[126]{hollow1973a}

	\item\label{nikexamples3}
	\glll óo íwaxekerektiki, órootki\textbf{nik}, ų́'sh ~ ~ ~ núunihkereka't\\
	oo i-waxE=krE=kti=ki o-rootki=\textbf{rįk} ų'sh ~ ~ ~  ruurįh=krE=ka=ą't\\
	dem.mid pv.ins-\textnormal{stop}=3pl=pot=cond pv.irr-\textnormal{camp}=\textbf{iter} \textnormal{be.thus} ~ ~ ~ \textnormal{be.there}.pl.dur.aux=3pl=hab=hyp\\
	\glt `whenever they stopped there, they were always there like that when they would camp' \citep[203]{hollow1973a}

	\item\label{nikexamples4}
	\glll	kimáto \'{ı̨}'haraa\textbf{nik} ~ ~ ~ ~ ~ ~ ~ ~ ~ ~ ~ ~ ~ ~ ~ ~ ~ ~ ~ ~ ~ ~ ~ ~ ~ íruxaxąhka'sh\\
	ki-wąto į'-hrE=\textbf{rįk} ~ ~ ~ ~ ~ ~ ~ ~ ~ ~ ~ ~ ~ ~ ~ ~ ~ ~ ~ ~ ~ ~ ~ ~ ~ i-ru-xa$\sim$xąh=ka=o'sh\\
	mid-\textnormal{bear} pv.rflx-caus=\textbf{iter} ~ ~ ~ ~ ~ ~ ~ ~ ~ ~ ~ ~ ~ ~ ~ ~ ~ ~ ~ ~ ~ ~ ~ ~ ~ pv.ins-ins.hand-aug$\sim$\textnormal{be.torn}=hab=ind.m\\
	\glt `every time he changes into a bear, he always tears them up' \citep[156]{hollow1973b}

	\end{xlist}

\end{exe}

Like the conditional, there is a causal relationship between a clause bearing the =\textit{nik} complementizer and its superordinate clause. The majority of the data show that =\textit{nik} occurs when something usually or always happens. It can be used with both realis and irrealis propositions. Most instances of =\textit{nik} involve the same subject for both the clause bearing =\textit{nik} and the superordinate clause, but as we see in (\ref{nikexamples2}), this complementizer is not like a switch-reference marker that depends on having or not having the same subjects.

It is likely that this complementizer is actually a combination of a reflex of Proto-Siouan *rį `be, exist' and the habitual =\textit{ka}, where the final vowel has been lost. This enclitic triggers ablaut in all possible instances in the corpus, which is to be expected, given the fact that it contains a nasal segment.

\subsubsection{Incredulative complementizer: =\textit{nikų'k}}

In his description of the incredulative in his grammar, \citet[20]{kennard1936} notes that this enclitic expresses disbelief about a proposition, real or imagined, on the part of the speaker. \citet[473]{hollow1970} amends this description by pointing out that a speaker can also use the incredulative to report second-hand information that the speaker does not believe or that an event has happened that is so surprising that the speaker cannot believe that it really happened and remains unconvinced. These descriptions encompass all the uses of the incredulative =\textit{nikų'k} found in the corpus. 

In many cases, indicative propositions are rendered as rhetorical questions in English, expressions of dubious possibility, or outright statements of disbelief. The incredulative =\textit{nikų'k} is not necessarily present in speech directed at any listener in particular, and while it can be translated as a question in English, it is used exclusively in declarations in Mandan. This complementizer is not a subordinator, and a clause ending in =\textit{nikų'k} is a complete utterance in of itself. As the data in (\ref{nikukexamples}) below show, =\textit{nikų'k} does not trigger ablaut, despite the fact that it features a nasal segment. We can see some examples of this complementizer in the following data in (\ref{nikukexamples}).

\begin{exe}

\item\label{nikukexamples} Examples of the incredulative complementizer =\textit{nikų'k}

	\begin{xlist}
	
	\item\label{nikukexamples1} 
	\glll ré'esh ų́kahanashe túkere\textbf{nikų'k}!\\
	re-esh ųk\#ah=rąsh=E tu=krE=rįkų'k\\
	dem.prox-smlt \textnormal{hand}\#\textnormal{be.covered}=att=sv \textnormal{be.some}=3pl=incd\\
	\glt `there couldn't have been some of his fingernails' \citep[150]{hollow1973a}

	\item\label{nikukexamples2} 
	\glll ináa, réhąk ímahąpi\textbf{nikų'k}!\\
	irąą re=hąk i-wąhąp=\textbf{rįkų'k}\\
	\textnormal{yikes}.f dem.prox=stnd.pos pv.ins-\textnormal{be.lost}=incd\\
	\glt `yikes, I can't believe this one has gotten lost!' \citep[172]{hollow1973a}

	\item\label{nikukexamples3} 
	\glll wáanuma'kaaki hų́nus ~ ~ ~ ~ ~ ~ ~ ~ ~ ~ ~ ~ ~ ~ ~ ~ ~ ~ ~ ~ wakaráahkakere\textbf{nikų'k}\\
	waa-nuwą'k-aaki hų=rų=s ~ ~ ~ ~ ~ ~ ~ ~ ~ ~ ~ ~ ~ ~ ~ ~ ~ ~ ~ ~ wa-kraah=ka=krE=\textbf{rįkų'k}\\
	nom-\textnormal{man}-coll \textnormal{be.many}=anf=def ~ ~ ~ ~ ~ ~ ~ ~ ~ ~ ~ ~ ~ ~ ~ ~ ~ ~ ~ ~ unsp-\textnormal{be.afraid}=hab=3pl=incd\\
	\glt `that bunch of people sure must have been afraid' \citep[178]{hollow1973a}

	\item\label{nikukexamples4} 
	\glll wáa'ą's kų́'hi\textbf{nikų'k}!\\
	waa-ą's k'-ųųh=\textbf{rįkų'k}\\
	nom-\textnormal{be.near} 3poss.pers-\textnormal{wife}=\textbf{incd}\\
	\glt `who would marry someone like that?' \citep[20]{kennard1936}
	
	\end{xlist}

\end{exe}

The most likely origin for the incredulative =\textit{nikų'k} is a combination of the Proto-Siouan negative *rį and the benefactive verb *kʔu `give', with the final /k/ coming from the habitual *ka that has lost its final vowel. Another possibility is that the /{kų'}/element is not originally derived from the benefactive verb `give', but is a reanalysis of the proposed Proto-Siouan dubitative or negative *ku$\sim$*kų. We see evidence of a cognate in Hoocąk \textit{šgųnį}, the weak dubitative, which is composed of all three proposed negative formatives *aši+kų+rį. As we saw in \sectref{SubSubNIX}, the ordering of negative exponents in Mandan is the reverse of that in Mississippi Valley, i.e., Mississippi Valley has PSi *aši-rį > Proto-Mississippi Valley *šnį, but Mandan has PSi *rį-axi > =\textit{nix} /=rįx/. Having a reflex of PSi *ku$\sim$kų appear after *rį would be consistent with what we see with respect to the ordering of negative Proto-Siouan reflexes in Mandan.

If the dubitative origin analysis above is correct, then this enclitic is noteworthy in that Mandan has no other reflexes of the dubitative element PSi *ku. The dubitative marker in Proto-Siouan itself is sparsely attested in the Siouan language family. This PSi *ku$\sim$kų appears only in Hoocąk-Chiwere and as a stem involved in a negative element in Biloxi \citep{rankin2015}. While it is not necessarily the case that this incredulative enclitic goes back to Proto-Siouan, it is plausibly reconstructable.

\subsubsection{Disjunctive complementizer: =\textit{shka'nik} or =\textit{skha}}

The disjunctive complementizer in Mandan juxtaposes two propositions where proposition A is true despite proposition B. This complementizer has similar semantics as `but', `although', or `even though' in English. The disjunctive can manifest in Mandan with one of two markers: the complex =\textit{shka'nik} or the simple =\textit{shka}. Neither \citet[22]{kennard1936} nor \citet[61]{mixco1997a} describe any functional difference between the two, and examination of the corpus corroborates this observation. One possibility is that the complex form has a stronger reading than the simple form, but no speakers have elaborated upon the difference. Examples of this complementizer appear in (\ref{SkaPnikexamples}) below.



\begin{exe}

\item\label{SkaPnikexamples} Examples of the disjunctive complementizer =\textit{shka'nik}

	\begin{xlist}
	
	\item\label{SkaPnikexamples1} 
	\glll í'ą'kanashoo íhaaxik\textbf{shka'nik}, ~ ~ ~ ~ ~ ~ ~ ~ kaxíp \'{ı̨}'hara má'kaha...\\
	i-ą'=ka=rąsh=oo i-haaxik=\textbf{shka'rįk}, ~ ~ ~ ~ ~ ~ ~ ~ ka-xip į'-hrE wą'kah=E=$\varnothing$\\
	pv.dir-\textnormal{be.near}=hab=att=dem.mid pv.ins-\textnormal{not.know}=\textbf{disj} ~ ~ ~ ~ ~ ~ ~ ~ ins.frce-\textnormal{skin} pv.rflx-caus \textnormal{lie}.aux.hab=sv=cont\\
	\glt `although they did not know exactly how to do it that way, they kept trying to skin it' \citep[197]{hollow1973a}

	\item\label{SkaPnikexamples2}
	\glll tópha kiná'\textbf{shka}, wáapaksąhe miká ~ ~ ~ nákini...\\
	top\#ha kirą'=\textbf{shka} waa-pa-ksąh=E wįk=E ~ ~ ~ rąk=rį\\
	\textnormal{four}\#\textnormal{times} \textnormal{tell}=\textbf{disj} nom-ins.push-\textnormal{be.worried}=sv \textnormal{be.none}=sv ~ ~ ~ sit.pos=ss\\
	\glt `even though he told it four times, he was not paying attention and...' \citep[156]{hollow1973a}

	\item\label{SkaPnikexamples3}
	\glll ą́'shka\textbf{shka'nik}, óminikiri'ro'sh\\
	ą'shka=\textbf{shka'rįk} o-w-rį-kri'=o'sh\\
	\textnormal{be.near}=\textbf{disj} pv.irr-1a-2s-\textnormal{defeat}=ind.m\\
	\glt `that may be so, but I will beat you' \citep[59]{hollow1973b}
	
	\item\label{SkaPnikexamples4}
	\glll úkere\textbf{shka'nik}, wáateenixka'sh\\
	u=krE=\textbf{shka'rįk} waa-tee=rįx=ka=o'sh\\
	\textnormal{wound}=3pl=\textbf{disj} neg-\textnormal{die}=neg=hab=ind.m\\
	\glt `they shot him, but he does not die' \citep[117]{hollow1973b}
	
	\end{xlist}

\end{exe}

The complex =\textit{shka'nik} looks to be made up of several elements. This formative includes the simple disjunctive or similitive =\textit{shka}, but we can see the presence of a glottal stop, which indicates an elided /o/ from the copula \textit{ó'} `be'. The final element that makes the last syllable is the iterative complementizer =\textit{nik}. Historically, this enclitic seems to have been made up of /=shka=o'=rįk/, suggesting that it was originally some kind of periphrastic construction in earlier stages of Mandan that has become reanalyzed as a single unit that is a discrete enclitic unto itself for modern speakers. It is not clear what the origin of the simple disjunctive =\textit{shka} is, as it could itself be made up of multiple Proto-Siouan elements, e.g., the /ka/ could be the attributive PSi *ka or even the habitual =\textit{ka}. Likewise, it could be related to the manner marker PSi *ska, though that would conflate the origin of the disjunctive marker with the similitive marker. Further examination of interclausal morphology across the Siouan language family is needed to determine if this element has roots in Proto-Siouan or if it is a Mandan innovation.

\subsubsection{Intensive indicative complementizer: =\textit{sįh}}\label{intensivecomplementizer}

The allocutive agreement markers for indicative utterances, =\textit{o'sh} and =\textit{o're}, occur on matrix verbs in the overwhelming majority of complete utterances in the corpus. However, if a speaker wishes to emphasize their point, they may choose to use the intensive complementizer =\textit{sįh} instead. This complementizer is homophonous and semantically similar to the intensifier suffix -\textit{sįh} described in \sectref{suffixintensifier}.

Instead of intensifying the action or the state, as the suffix -\textit{sįh} does, the intensive complementizer intensifies the entire proposition. The use of this enclitic can indicate that the speaker wishes to emphasize their point, or that they are vouching for the veracity of the statement. Both Hidatsa and Crow have a similar clause-final element: =\textit{sht} in Crow \citep[394]{graczyk2007} and =\textit{shd} in Hidatsa \citep[231]{park2012}. Both of these forms are reflexes of the Proto-Siouan augmentative *-xtE, whereas the intensive complementizer in Mandan comes from the Mandan verb \textit{s\'{ı̨}h} `be strong.' Several examples of the intensive complementizer appear in (\ref{intensivecomplementizerexamples}) below.

\begin{exe}

\item\label{intensivecomplementizerexamples} Examples of the intensive indicative complementizer =\textit{sįh}

	\begin{xlist}
	
	\item\label{intensivecomplementizerexamples1}
	\glll minís waká'ka\textbf{sįh}\\
	wrįs wa-ka'=ka=\textbf{sįh}\\
	\textnormal{horse} 1a-\textnormal{possess}=hab=\textbf{ints}\\
	\glt `I used to have a horse' \citep[27]{mixco1997a}
	
	\item\label{intensivecomplementizerexamples2}
	\glll ``Okípa'' wáa'eeheenixanashka\textbf{sįh}\\
	okipa waa-ee-hee=rįx=rąsh=ka=\textbf{sįh}\\
	\textnormal{Okipa.ceremony} neg-pv-\textnormal{say}=neg=att=hab=\textbf{ints}\\
	\glt `he never said ``Okipa''' \citep[29]{mixco1997a}

	\item\label{intensivecomplementizerexamples3}
	\glll máa'ąk íxatanashini réehoomak\textbf{sįh}\\
	wąą'ąk i-xat=rąsh=rį rEEh=oowąk=\textbf{sįh}\\
	\textnormal{land} pv.ins-\textnormal{look.around}=att=ss \textnormal{go.there}=narr=\textbf{ints}\\
	\glt `he went looking around the land once' \citep[6]{hollow1973a}

	\item\label{intensivecomplementizerexamples4}
	\glll tawáa'irukiriihs, ishák ~ ~ ~ ~ ~ ~ ~ ~ ~ ~ ~ ~ ~ ~ ~  náhka'ehe\textbf{sįh}\\
	ta-waa-i-ru-kriih=s ishak ~ ~ ~ ~ ~ ~ ~ ~ ~ ~ ~ ~ ~ ~ ~  rąk=ka'ehe=\textbf{sįh}\\
	al-{nom}-pv.ins-ins.hand-\textnormal{be.lined.up}=def pro ~ ~ ~ ~ ~ ~ ~ ~ ~ ~ ~ ~ ~ ~ ~  sit.pos=quot=\textbf{ints}\\
	\glt `his staff, it was right there, it is said' \citep[7]{hollow1973a}

	\end{xlist}

\end{exe}

This intensive indicative =\textit{sįh} is able to co-occur with evidentials that indicate that the speaker does not have first-hand knowledge of the event in the utterance. However, in a similar way to the definite =\textit{s}, speakers may use =\textit{sįh} to proclaim the truth of the statement. In this way, the intensive indicative produces a stronger assertion than one involving the definite =\textit{s}. This enclitic appears in the corpus rather sparingly. \citet[28]{mixco1997a} is the first to point out this formative, though he describes it as a combination of the definite =\textit{s} and what he describes as a coordinating conjunction \textit{hįį}. In Mandan, \textit{hįį} is used as a hedge in discourse, similar to English `uh', `um', or `well'. \citeauthor{mixco1997a} translates it as `and', but it is not a true coordinator or sentence connector. \citet{hollow1973a,hollow1973b} recognizes this and almost never transcribes \textit{hį} in his narratives, which also carried over to him omitting all hedges and filler elements like \textit{hį} from the corpus. The use of instrumentation like Praat allows us to see that there is frication at the end of this formative \citep{boersmaweenik2016}. The presence of /h/ after the vowel indicates that it cannot be a series of enclitics, but a single element that comes from a reanalysis of the word \textit{s\'{ı̨}h} `be strong.'

\subsubsection{Causational complementizer: =\textit{so'nik}}

The causational complementizer in Mandan is often used where `because' or `since' would be used in English. The enclitic =\textit{so'nik} indicates that a subordinated proposition B has happened as a result of proposition A being the case. We can see examples of this subordinator in the data in (\ref{soPnikexamples}) below.

\newpage
\begin{exe}

\item\label{soPnikexamples} Examples of the causational complementizer =\textit{so'nik}

	\begin{xlist}
	
	\item\label{soPnikexamples1}
	\glll wáashinash\textbf{so'nik}, ą́'t kirúto'xere'sh\\
	waa-shi=rąsh=\textbf{so'rįk} ą't k-rut=o'xrE=o'sh\\
	neg-\textnormal{be.good}=att=\textbf{comp.caus} dem.anap mid-\textnormal{eat}=dub=ind.m\\
	\glt `since he is no good, he will not get to eat that' \citep[43]{hollow1973a}

	\item\label{soPnikexamples2}
	\glll koshų́ųkas míih ~ ~ ~ ~ ~ ~ ~ ~ ~ ~ ~ ~ ~ ~ ~  áaki\textbf{so'nik}, íkxąhini...\\
	ko-shųųka=s wįįh ~ ~ ~ ~ ~ ~ ~ ~ ~ ~ ~ ~ ~ ~ ~  aa-\textbf{ki=so'rįk} i-kxąh=rį\\
	3poss.pers-\textnormal{male's.younger.brother}=def \textnormal{woman} ~ ~ ~ ~ ~ ~ ~ ~ ~ ~ ~ ~ ~ ~ ~ pv.tr-\textnormal{arrive.back.here}=\textbf{comp.caus} pv.ins-\textnormal{laugh}=ss\\
	\glt `since his brother brought a woman back, he was laughing at him and...' \citep[73]{hollow1973a}

	\item\label{soPnikexamples3}
	\glll karóotiki, kų́'h\textbf{so'nik}, má'keroomako'sh\\
	ka=oote=ki k'-ųųh=so'rįk wą'kE=oowąk=o'sh\\
	prov=evid=cond 3poss.pers-\textnormal{wife}=cond.caus \textnormal{lie}.aux=narr=ind.m\\
	\glt `and so, because she was his wife, he stayed there' \citep[28]{hollow1973b}

	\item\label{soPnikexamples4}
	\glll íwapashiriih\textbf{so'nik}, óshi'sh\\
	i-wa-pa-shriih=\textbf{so'rįk} o-shi=o'sh\\
	pv.ins-1s-ins.push-\textnormal{think.about}=\textbf{comp.caus} pv.irr-\textnormal{be.good}=ind.m\\
	\glt `because I thought it over, it will be good' \citep[210]{hollow1973b}

	\end{xlist}

\end{exe}

Clauses containing the enclitic =\textit{so'nik} cannot exist without a superordinate clause. That is, the causative complementizer introduces a reason for why the following proposition is the case. Without a superordinate clause, a clause bearing =\textit{so'nik} is an incomplete utterance.

Like other complementizers, =\textit{so'nik} appears to contain a fossilized remnant of the copula \textit{ó'} `be'. A some earlier stage in Mandan's development, this sequence was composed of three distinct elements: /=s=o'=rįk/, where =\textit{s} is the definite evidential, \textit{o'} is `be', and =\textit{nik} is the iterative complementizer. It is not clear how an iterative marker fits within the semantics of a causal subordinator like =\textit{so'nik}, though it could be the case that =\textit{nik} was at one point semantically broader and could be used for different kinds of subordination.

\section{Formative ordering}\label{SyntaxAffixation}

Over the course of this chapter, I have described the range of morphology present on Mandan verbs. While Mandan has a large inventory of prefixes, suffixes, and enclitics, its relatives Crow and Hidatsa boast a larger array of verbal morphology, as does Mandan's more distant cousin, Lakota. Mandan has a greater amount of distinct verbal morphology than other Siouan languages, like Tutelo or Biloxi. It is not immediately clear why Mandan has a smaller morphological inventory (excluding allomorphy) than other Siouan languages; this inventory size may be restricted by the fact that I have had to rely on a corpus elicited by other researchers that consists of traditional narratives. It is quite possible other enclitics had existed in contemporary Mandan, but would only likely had come out in conversation, and as such, these potential verbal morphological items not reflected in the corpus or in this book. The lack of L1 Mandan speakers means that there will likely be verbal morphology that was simply never recorded and thus lost.

Earlier in this chapter, I stated that prefixes in Mandan always occur in a proscribed order. This order is typically described as being templatic, in that we can conceptualize each prefix fitting in a particular slot in a template. This template appeared in \tabref{prefixfieldmandanredux}, which is repeated below.

\begin{table}
\caption{Prefix field in Mandan}\label{prefixfieldmandanREredux}

\fittable{\scshape
    \begin{tabular}{llllllllllll}
    \lsptoprule
    11  & 10  & 9      & 8    & 7   & 6          & 5      	& 4		& 3			& 2     & 1     & 0    \\
\midrule
    rel & neg & unsp   & 1pl & pv.irr & pv.loc   & 1sg 		& 2sg		& suus		& iter  & ins & stem \\
    ~   & ~   & ~ & ~    & ~   & pv.ins 	 & ~      	& 2pl		& mid			& incp   & ~     & ~    \\
    ~   & ~   & ~      & ~    & ~   & pv.tr 	 & ~      	& ~		& recp		& ~  & ~     & ~    \\
\lspbottomrule
    \end{tabular}}
\end{table}

Nowhere in the corpus do we see exceptions to this template. Likewise, we see some suffixes appear in a proscribed order. For example, the intensifier suffix -\textit{sįh} is always closer to the stem than the augmentative suffix -\textit{xte}. Some suffixes, like the collective suffix -\textit{aaki}, are so restricted in their use that there are simply no other instances of another suffix occurring alongside them to gauge how they would be ordered. As such, it is difficult to make a complete template for the suffix field in Mandan without having an L1 speaker to render grammaticality judgments. Nonetheless, the fact that certain suffixes must appear in specific orderings with respect to one another (e.g., the augmentative must always follow the intensifier and never the other way around) is consistent with the suffix field also being templatic in nature.

In this discussion of templatic morphology in Mandan, it has been assumed that the template is immutable. This observation holds for prefixes and true suffixes, but not so for enclitics, which do not have the same fixed ordering. I argue that enclitic order reflects the semantico-syntactic configuration of a proposition and that a change in enclitic order reveals the scopal relationships between that enclitic and its constituents. 

In the examples in (\ref{encliticshavenoorder}) below, we see two Mandan sentences that feature the same three enclitics that appear in a different order. The prospective aspectual enclitic =\textit{naate} appears in both, but its ordering with respect to subject marking and negation is different. This aspectual enclitic is shown in bold, the subject enclitic is underlined, and the negative enclitic appears with a double underline. The purpose of highlighting these various enclitics is to show their ordering with respect to one another. I have argued throughout this section that the ordering of enclitics reflects the underlying structure (contrary to the ordering of affixes), and I assume that the differing orders must therefore reflect differing underlying structures.

\begin{exe}

\item\label{encliticshavenoorder} Variable positioning of enclitics

	\begin{xlist}
	
	\item\label{encliticshavenoorder1}
	\glll wáa'okikashka\uline{\uline{xi}}nasha\textbf{naate}\uline{kere}'sh\\
	waa-o-ki-kashka=xi=rąsh=rąątE=krE=o'sh\\
	neg-pv.irr-mid-\textnormal{be.same}=neg=att=prsp=3pl=ind.m\\
	\glt `they \textbf{almost} were not sort of the same' \citep[30]{mixco1997a}
	
	Verb $\ll$ \uline{\uline{Negation}} $\ll$ \textbf{Aspect} $\ll$ \uline{Subject}

	
	\item\label{encliticshavenoorder2}
	\glll wáarakų'karaa\uline{nit}i\uline{\uline{nix}}a\textbf{naate}'sh\\
	waa-ra-kų'=krE=rįt=rįx=rąątE=o'sh\\
	neg-2a-\textnormal{give}=3pl=2pl=neg=prsp=ind.m\\
	\glt `you (pl.) \textbf{just about} did not give it to them' \citep[468]{hollow1970}

	Verb $\ll$ \uline{Subject} $\ll$ \uline{\uline{Negation}} $\ll$ \textbf{Aspect}
	
	\end{xlist}

\end{exe}

In (\ref{encliticshavenoorder1}), the negative enclitic appears immediately after the verb root and before the prospective aspectual enclitic, which in turn appears before the third person plural subject marker. This order indicates that the subject is more prominent in the structure than the aspectual, and that the negation is only scoping over the proposition (i.e., `be the same'), giving us a statement along the lines of `they almost were not the same.'

A drastically different enclitic order appears in (\ref{encliticshavenoorder2}), where the prospective enclitic appears after all plural marking, both subject and object, as well as after the negation enclitic =\textit{nix}. Just as we saw above, this enclitic order indicates what is scoping over what. The aspectual is farthest away from the stem, indicating that it is scoping over all other elements subordinate to it. Negation appears farther away from the stem than subject plural marking, which likewise signifies that the entire proposition involving the subject is being negated, not just the verb. Whereas negation appeared as the first postverbal element in (\ref{encliticshavenoorder1}), it appears much farther into the enclitic field in (\ref{encliticshavenoorder2}). This enclitic order provides a reading closer to `it was almost not the case that you gave it to them.' This reading is similar to the free translation that \citet[468]{hollow1970} provides, but differs slightly to emphasize that the prospective aspectual is scoping over the entire proposition, rather than just the act. This difference is subtle and could have been difficult to articulate, which is why it was not encoded as such in the transcribed data.

The data above are evidence that support my hypothesis that the order of enclitics in (\ref{encliticshavenoorder}) above is not random and can be tied to the underlying semantics of a clause. Both the negation enclitic and the prospective aspectual enclitic in these examples illustrate where each of these elements are in the structure and over what they have scope. The fact that negation is marked immediately after the first but before subject marking in (\ref{encliticshavenoorder1}) shows that negation appears low in the structure, scoping over just the verb, while its presence after the subject plural marker in (\ref{encliticshavenoorder2}) means that negation is taking place much higher in the structure and has scope over not just the verb but the whole inflectional phrase. Similarly, the prospective aspectual enclitic in (\ref{encliticshavenoorder1}) appears before the subject plural marker, which indicates that this aspect has scope over the verb but not the entire proposition. We can contrast this limited scope reading with a wide scope reading in (\ref{encliticshavenoorder2}), where the aspectual enclitic appears after all person marking and negation, showing that it has scope over the entire proposition.

I argue in \citet{kasak2019} that enclitics in Mandan are neither ordered in a template nor ordered at a whim. Rather, the ordering of enclitics corresponds to the intended semantics of what kind of scope that enclitic has over the other elements in verbal complex. Some of the changes in enclitic order have very small effects on the reading by a speaker, much in the same way that altering the order of adverbs in English can affect their semantics in slight ways. We can see in the examples below in (\ref{SlightSemantics}) how shifting the word \textit{only} around in a sentence can change the meaning, sometimes subtly, and sometimes in a major way.

\begin{exe}

\item\label{SlightSemantics} Changing semantics through adverb placement

    \begin{xlist}
    
    \item\label{SlightSemantics1} I \textit{only} want a cup of coffee.
    
    \item\label{SlightSemantics2} I want \textit{only} a cup of coffee
    
    \item\label{SlightSemantics3} I want a cup of \textit{only} coffee
    
    \end{xlist}

\end{exe}

In (\ref{SlightSemantics2}), the placement of \textit{only} indicates that there is something that I want, and the only thing it is is a cup of coffee. We can contrast this sentence with (\ref{SlightSemantics1}), where I am justifying my reason for being somewhere, e.g., of all the reasons why I am here at the coffee machine, I am only here because I want a cup of coffee, and not because I want to eavesdrop on my two coworkers sitting near it in the breakroom. Likewise, (\ref{SlightSemantics3}) differs from the previous two examples in that I am establishing that I want a cup of something, and only coffee will do (i.e., not water, not tea, not juice, etc.). This slight change in word order in English is analogous of the effect that moving an enclitic around in Mandan has. In general, object plural markers are closer to the verb stem than aspectual enclitics are, subject plural markers are farther away from the stem than aspectuals are, and negation will usually occur between aspectuals and subject marking. However, as we see above in (\ref{SlightSemantics}), small changes to this order can have an effect on the intended reading of an utterance.
