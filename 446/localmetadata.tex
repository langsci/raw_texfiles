\author{Ryan M. Kasak} %use this field for editors as well
\title{A grammar of Mandan}
\ISBNdigital{978-3-96110-495-6}
\ISBNhardcover{978-3-98554-124-9}
\BookDOI{10.5281/zenodo.14227513}
\typesetter{Ryan Kasak}
\proofreader{Aniefon Daniel,
Camil Staps,
Elliott Pearl,
Jean Nitzke,
Jeroen van de Weijer,
Katja Politt,
Lachlan Mackenzie,
Mary Ann Walter,
Mykel Brinkerhoff,
Sebastian Nordhoff,
Tihomir Rangelov,
Yvonne Treis,
}
% \lsCoverTitleSizes{51.5pt}{17pt}
\renewcommand{\lsSeries}{cogl}
\renewcommand{\lsSeriesNumber}{10}
\renewcommand{\lsID}{446}
\BackBody{This book presents an overview of the grammar of the Mandan language (Siouan), traditionally spoken in North Dakota along the Upper Missouri River in the United States. The last L1 speaker of Mandan passed away in 2016, so this book stems from personal fieldwork and fieldwork done by earlier researchers in the 20th century. This work has been written with a dual purpose: to be used by members of the Mandan community who wish to better understand the workings of their ancestral language, but also as a resource for linguists who wish to investigate several of the typologically notable features of the Mandan language.

Mandan is a highly agglutinating language with a complex system of templatic morphology, where affixes must appear in a specific ordering with respect to one another. This language displays a large repertoire of inflectional enclitics. These enclitics are less rigid in their ordering than affixes, because the placement of an enclitic is indicative of its semantic scope within the clause. A productive feature of Mandan verbal morphology is the compulsory presence of allocutive agreement markers at the end of matrix clauses, where these allocutive agreement markers encode the addressee's gender information. Other features of theoretical and typological interest are the presence two separate epenthetic processes that are sensitive to morphological boundaries, the treatment of excrescent vowels as extraphonological, the switch-reference system to link clauses together depending on whether the clauses have the same or different subject as the one that follows, and the extensive discussion of comparative Siouan morphology and lexicon.
}
