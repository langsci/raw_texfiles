\documentclass[output=paper
,modfonts
,nonflat]{langsci/langscibook} 

% \title{\textit{Most} vs. \textit{the most} in languages where \textit{the more} means \textit{most}} 

\markuptitle{\textit{Most} vs. \textit{the most} in languages where \textit{the more} means \textit{most}}{`{Most}' vs. `{the most}' in languages where `{the more}' means `{most}'}

\shorttitlerunninghead{\textnormal{Most} vs. \textnormal{the most} in languages where \textnormal{the more} means \textnormal{most}} 
\renewcommand{\lsCollectionPaperFooterTitle}{{\noexpand\itshape Most} vs. \noexpand\textit{the most} in languages where \noexpand\textit{the more} means \noexpand\textit{most}}
% \renewcommand{\lsCollectionPaperFooterTitle}{\noexpand\textit{Most} vs. \noexpand\textit{the most} in languages where \noexpand\textit{the more} means \noexpand\textit{most}}

\author{Elizabeth Coppock\affiliation{Boston University\\University of Gothenburg}\lastand Linnea Strand\affiliation{University of Gothenburg}}
\ChapterDOI{10.5281/zenodo.3252030}

% \epigram{}

\abstract{This paper focuses on languages in which a superlative interpretation is typically indicated merely by a combination of a \is{definiteness marking|(}definiteness marker with a \is{comparatives}comparative marker, including \ili{French}, \ili{Spanish}, \ili{Italian}, \ili{Romanian}, and \ili{Greek} (\textsc{\defcomp{} languages}). Despite ostensibly using definiteness markers to form the superlative, superlatives are not always definite-marked in these languages, and the distribution of definiteness-marking varies across languages. Constituency structure appears to vary across languages as well. To account for these patterns of variation, we identify conflicting pressures that all of the languages in consideration may be subject to, and suggest that different languages prioritize differently in the resolution of these conflicts. What these languages have in common, we suggest, is a mechanism of \isi{Definite Null Instantiation} for the \is{degrees}degree-type standard argument of the comparative. Among the parameters along which languages are proposed to differ is the relative importance of marking \isi{uniqueness} vs.\ avoiding \isi{determiners} with predicates of entities that are not individuals.}

\begin{document}

%Keywords: prenominal vs.\ postnominal word order, definiteness spreading, definiteness, superlatives, comparatives, quantity words

\maketitle
\section{Introduction} \is{comparatives|(}\is{superlatives|(}
In \ili{French}, placing a \is{definite articles}definite article before a comparative \is{adjectives}adjective, as in \REF{ex:coppockstrand:1}, suffices to produce a superlative interpretation:

\ea \label{ex:coppockstrand:1}
\gll Elle est \textbf{la} \textbf{plus} \textbf{grande}.\\
she is the \cmpr{} tall\\ \jambox{(\ili{French})}
\glt `She is \textbf{the tallest}.' 
\z

\ili{French} is not alone; other \ili{Romance} languages, as well as Modern \ili{Greek}, \ili{Maltese} and others, make do with the same limited resources. Some examples are given in \tabref{tab:coppockstrand:1}.\footnote{Besides \ili{Romance} languages, languages reported to use this strategy include 
Modern Standard Arabic\il{Arabic}, Assyrian Neo-Aramaic\il{Neo-Aramaic}, \ili{Middle Armenian}, Modern \ili{Greek}, \il{Hebrew!Biblical Hebrew}Biblical Hebrew, \ili{Livonian}, \ili{Maltese}, \il{Mixtec!Chalcatongo Mixtec}Chalcatongo Mixtec, \ili{Papiamentu}, \ili{Vlach Romani}, \ili{Russian}, and \ili{Tamashek} \citep{Bobaljik2012,Gorshenin2012}. Note however that \citeauthor{Gorshenin2012} has rather liberal criteria for a given construction being of this type; for \ili{Russian}, the example given is \textit{Etot \v{z}urnal \textbf{sam-yj} interesn-yj} `This magazine is the most interesting (one)'. \citet[129]{Gorshenin2012} describes \textit{sam-yj} as an ``emphatic pronoun'' and reasons that ``this pronoun indicates \isi{uniqueness}, particularity of the referent in some respect, and therefore it can be regarded as a functional equivalent of a \is{determiners}determiner in the corresponding superlative construction''.} This paper considers such languages, which we call \defcomp{} \textsc{languages}, against the background of a growing literature on cross-linguistic variation with respect to the relationship between definiteness-marking and the interpretation of superlatives.

\begin{table}[h]
 \caption{Comparative and superlative \is{degrees}degree of `tall' in selected \defcomp{} languages}
\label{tab:coppockstrand:1}
 \begin{tabularx}{.75\textwidth}{lllll} 
  \lsptoprule
  \textsc{language} & \textsc{pos} & \cmpr{} & \sprl{} \\\midrule
{\ili{English}}  & \textit{tall} & \textit{taller} & \textit{tallest} \\
{\ili{French}} & \textit{grande} & \textit{plus grande} & \textit{la plus grande} \\
{\ili{Spanish}}  & \textit{alto} & \textit{más alto} & \textit{el más alto} \\
{\ili{Romanian}} & \textit{inalt} & \textit{mai} \textit{inalt} & \textit{cea mai inalt} \\
{\ili{Italian}} & \textit{alto} & \textit{più alto} & \textit{il più alto} \\
{\ili{Greek}} & \textit{psilós} & \textit{pio psilós} & \textit{o pio psilós} \\
{\ili{Greek}} (alt 2) & \textit{psilós} & \textit{psilóteros} & \textit{o psilóteros} \\
\lspbottomrule
 \end{tabularx}
\end{table}


When it comes to the superlatives of ordinary gradable \isi{adjectives} like \textit{tall}, the interpretive contrast of interest is the distinction between so-called \is{absolute readings of superlatives|(}\textit{absolute} and \is{relative readings of superlatives|(}\textit{relative} readings of superlatives in the domain of \isi{quality superlatives}. In \ili{Swedish}, unlike \ili{English}, this interpretive distinction is signalled morphologically with definiteness:

\ea \label{ex:coppockstrand:2}
\begin{xlist}
\ex \label{ex:coppockstrand:2a}
\gll Gloria sålde \textbf{god-ast} \textbf{glass}.\\
    Gloria sold delicious-\sprl{} {ice cream}\\ \jambox{(\ili{Swedish})}
\glt `Gloria sold \textbf{the most delicious ice cream}.' (relative only)
\ex \label{ex:coppockstrand:2b}
\gll  Gloria sålde \textbf{den} \textbf{god-ast-e} \textbf{glass-en}.\\
    Gloria sold the delicious-\sprl{}-\wk{} {ice cream}-\defn{}\\
\glt `Gloria sold \textbf{the most delicious ice cream}.' (relative or absolute)
\end{xlist}
\z
 
As \citet{TelemanEtAlii1999} discuss, \REF{ex:coppockstrand:2a} means that Gloria sold more delicious ice cream than anyone else. It would not suffice for \REF{ex:coppockstrand:2a} to be true for there to be a salient set of ice creams of which Gloria sold the most delicious. If someone else sold that ice cream as well, then \REF{ex:coppockstrand:2a} would be false. In contrast, the \ili{English} gloss and the definite-marked example \REF{ex:coppockstrand:2b} could be true if both Gloria and someone else sold the ice cream that was more delicious than all other ice creams that are salient in the context. All that is required for that sentence to be true is that Gloria stands in the `sold' relation to the ice cream satisfying that description.

In \citegen{Heim1999} terms, \REF{ex:coppockstrand:2a} has a \textit{relative reading} (originally called a \textit{comparative reading} by \citealt{Szabolcsi1986}), and \REF{ex:coppockstrand:2b}, along with the \ili{English} gloss, is ambiguous between a relative reading and an \textit{absolute reading}. Relative readings are typically focus-sensitive, implying a comparison between the \isi{focus} (e.g. Gloria) and the focus-alternatives, and on such readings the superlative \is{noun phrases}noun phrase behaves like an \is{indefinites}indefinite despite the frequent presence of a \is{definite determiners}definite determiner \citep{Szabolcsi1986,CoppockBeaver2014}. On an absolute reading, comparisons are made only among elements satisfying the descriptive content of the modified \is{nouns}noun, and the definite behaves as a definite. The contrast between absolute and relative readings was discussed early on by \citet{Szabolcsi1986} with  reference to \ili{Hungarian}, and has been taken up in a fair amount of recent cross-linguistic research, mainly focused on \ili{English} \citep{Gawron1995,Heim1999,Hackl2000,SharvitStateva2002,Hackl2009,Teodorescu2009,Krasikova2012,Szabolcsi2012,Bumford2016,Wilson2016}, but also with reference to \ili{German} \citep{Hackl2009}, \ili{Swedish} \citep{CoppockJosefson2015}, other \ili{Germanic} languages \citep{Coppocktoappear}, \ili{Hungarian} \citep{FarkasKiss2000}, \ili{Romanian} \citep{Teodorescu2007}, \ili{Spanish} \citep{Rohena-Madrazo2007}, Arabic\il{Arabic} \citep{Hallman2016}, and \ili{Slavic} languages including \ili{Macedonian}, \ili{Czech}, \il{Serbo-Croatian}Serbian/Croatian and \ili{Slovenian} \citep{PanchevaTomaszewicz2012}. This paper extends this line of research insofar as it considers the \is{morphosyntax}morphosyntactic realization of both types of readings in \defcomp{} languages.

The landscape of possible interpretations is slightly different when it comes to  the superlatives of \isi{quantity words}, like \ili{English} \textit{much}, \textit{many}, \textit{little} and \textit{few}. In \ili{English}, \textit{the most} has a relative reading (`more than everybody else'), while bare \textit{most} has what is called a \is{proportional readings of superlatives|(}\textit{proportional} reading (`more than half', roughly). In this domain, there is an especially great deal of cross-linguistic variability. As \citet{Hackl2009} shows, \ili{German} \textit{die meisten}, lit. `the most', can be translated into \ili{English} either as \textit{most} or \textit{the most}. Even more dramatically, \ili{English} and \ili{Swedish} are near-opposites with respect to the impact of definiteness-marking on interpretation \citep{CoppockJosefson2015}; the definite \is{quantity superlatives}quantity superlative definite \textit{de flesta} has a proportional reading, corresponding to \ili{English} \textit{most}, while the bare \textit{flest} has a relative reading, corresponding to \ili{English} \textit{the most}. \citet{Coppocktoappear} shows that every possible correlation between definiteness and interpretation is attested among the \ili{Germanic} languages. So the quantity domain is one that appears to be particularly volatile.

We might expect the landscape of variation with respect to the definiteness-marking of superlatives to be rather dull and flat within the realm of \defcomp{} languages. If superlatives are formed with definiteness-markers, then definiteness-markers should always appear, regardless of what reading is involved. But this is not what we find.


We find in fact several departures from the dull and flat picture one might expect. First, as \citet{Dobrovie-SorinGiurgea2015} discuss, \ili{French} is one of the many languages of the world where \isi{quantity superlatives} do not have a proportional interpretation.

\ea[]{ \label{ex:coppockstrand:3}
\gll De tout les enfants de mon école, je suis celui qui joue \textbf{le} \textbf{plus} \textbf{d'instruments}.\\
of all the kids of my school, I am the.one who plays \defn{} \cmpr{} of.instruments\\ \jambox{(\ili{French})}
\glt `Of all the kids in my school, I'm the one who plays the most instruments.'}
\z

\ea[*]{\label{ex:coppockstrand:4}
\gll \textbf{Le} \textbf{plus} \textbf{de} \textbf{cygnes} sont blancs.\\
the more of swans are white\\ \jambox{(\ili{French})}
\glt Intended: `Most swans are white.'}
\z

Example \REF{ex:coppockstrand:3} shows that the \is{quantity superlatives}quantity superlative \textit{le plus} can be used with a relative interpretation (comparing the speaker to other kids in the school); \REF{ex:coppockstrand:4} shows that it does not have a proportional interpretation; this example does not mean `most swans are white'. Such languages are surprising from the perspective of \citet{Hackl2000,Hackl2009}, according to which the proportional readings of quantity superlatives are parallel to absolute readings of \is{quality superlatives}quality superlatives. \ili{Romanian} and \ili{Greek} are more well-behaved from that perspective; there, the superlative of `many' (literally `the more many') can have a proportional interpretation. For example, the \ili{Greek} sentence in \REF{ex:coppockstrand:5} is ambiguous as indicated:

\ea \label{ex:coppockstrand:5}
\gll Éfaga \textbf{ta} \textbf{perissotera} \textbf{biskóta}.\\
ate.1\sg{} the much.\cmpr{} cookies\\ \jambox{(\ili{Greek})}
\glt `I ate \textbf{the most cookies}' or `I ate \textbf{most of the cookies}.'
\z

This is one point of variation.\is{language internal variation}

Another point of variation is which types of superlatives are accompanied by definiteness-marking. We can distinguish between the following types:
\begin{itemize}
\item \is{quality superlatives}Quality superlatives
\begin{itemize}
\item \is{adjectives}Adjectival quality superlatives
\begin{itemize}
\item Predicative, as in \textit{She is \textnormal{(}\textbf{the}\textnormal{)} \textbf{tallest}.}
\item \is{adnominal superlatives}Adnominal; \is{absolute readings of superlatives|)}absolute reading, as in \textit{\textbf{The tallest girl} left.}
\item \is{adnominal superlatives}Adnominal; relative reading, e.g.\ \textit{I'm not the one with \textbf{the thinnest waist}.}
\end{itemize}
\item Adverbial quality superlatives, as in \textit{She runs \textbf{the fastest}.}
\end{itemize}
\item Quantity superlatives\is{quantity superlatives}
\begin{itemize}
\item \is{adnominal superlatives}Adnominal quantity superlatives
\begin{itemize}
\item Relative reading, as in \textit{I ate \textbf{the most cookies}.}
\item Proportional reading, as in \textit{I ate \textbf{most of the cookies}.}
\end{itemize}
\item \is{adverbial superlatives}Adverbial quantity superlatives, as in \textit{She talks \textbf{the most}.}
\end{itemize}
\end{itemize}

In \ili{French} and \ili{Romanian}, definiteness-marking appears on superlatives of all of these types. The same is not the case for \ili{Italian}, \ili{Spanish} and \ili{Portuguese}. Despite forming quality superlatives through the combination of a definiteness-marker with a comparative form, these languages do not use definiteness-marking for \is{adverbial superlatives}adverbial superlatives or quantity superlatives on relative readings (and they generally do not allow proportional readings for quantity superlatives at all). Sentence \REF{ex:coppockstrand:6} is an example from \ili{Italian} (cf.\ \citealt{deBoer1986}, \citealt{Dobrovie-SorinGiurgea2015}, i.a.):

\ea \label{ex:coppockstrand:6}
\gll Probabilmente è Hans che ha bevuto \textbf{più} \textbf{caffè}.\\
probably it.is Hans who has drunk \cmpr{} coffee\\ \jambox{(\ili{Italian})}
\glt `It is probably Hans who has drunk \textbf{the most coffee}.'
\z

(A comparative interpretation, `It is probably Hans who has drunk more coffee', is also available here, although the cleft construction strongly biases toward a superlative interpretation.) The same happens in \ili{Spanish} and \ili{Portuguese}.

In \ili{Greek}, as illustrated below, there is a split between \is{quantity superlatives}quantity and quantity \is{adverbial superlatives}adverbials (`talk the most' vs.\ `talk the fastest'): quantity adverbials are obligatorily definite-marked and quantity adverbials obligatorily lack definiteness-marking. All other superlatives have a definiteness marker, \is{relative readings of superlatives|)}relative and proportional readings of quantity superlatives included.

So, in all of these languages, superlatives are generally formed by combining a definiteness-marker with a comparative, yet in some of these languages, superlatives may lack a definiteness-marker. 
This is certainly surprising if the superlative interpretation is supposed to rest fully in the hands of the \is{definite determiners}definite determiner. 

\is{definite articles|(}Generally, there are several analytical options we could consider for \defcomp{} superlatives. The one we have just ruled out (at least for some of these languages) is that the definite article itself is the marker of the superlative. Another is that the comparative is lexically ambiguous between a comparative and a superlative. Another would build on the stance argued for by \citet{Bobaljik2012}, where superlatives are composed of comparatives and a bit that means `of all'. This latter piece could be taken to be silent in \defcomp{} languages; see \citet{Szabolcsi2012} for a formal analysis of \textit{the more} in \ili{English} along these lines. A fourth possibility is that a superlative interpretation arises more or less directly from the composition of a comparative meaning and the meaning of the definite article, just as the surface form suggests.

We show that a moderate instantiation of the last-mentioned strategy is viable, both for \defcomp{} languages and for certain cases in \ili{English} like \textit{the more qualified candidate} (\textit{of the two}). In a nutshell, the standard argument of the comparative is saturated by a \is{degrees}degree-type pronoun. So \textit{the more qualified candidate}, for example, denotes the candidate in the contextually-given \is{comparison classes}comparison class $\textbf{C}$ that is more qualified than contextually-given $\textbf{d}$, for appropriately chosen value of $\textbf{d}$. This is hypothesized to be possible in all of the languages under consideration (and even \ili{English}, manifest in expressions like \textit{the taller one of the two}).

This is the common core. But there are conflicting pressures that lead to variation with respect to whether definiteness-marking occurs. On the one hand, there is pressure to mark \isi{uniqueness} on phrases where uniqueness can be marked, and on the other hand, there is pressure to avoid \is{definiteness marking|)}definiteness-marking on descriptions of entities other than individuals. Different languages prioritize differently when it comes to resolving these conflicts. We suggest furthermore that \is{proportional readings of superlatives|)}proportional readings arise through \isi{grammaticalization}, but via different routes\largerpage for different languages.

The following sections will present data from \ili{Greek}, \ili{Romanian}, \ili{French}, and \il{Romance!Ibero-Romance}Ibero-Romance, in that order. These sections will lay out the basic facts concerning the \isi{morphosyntax} of superlatives in these languages. After a summary in \sectref{sec:coppockstrand:5}, compositional treatments of the various varieties will be sketched in \sectref{sec:coppockstrand:6}.

\section{\ili{Greek}} \label{sec:coppockstrand:2}

We begin with \ili{Greek}, where a definite article may combine with either a synthetic or periphrastic comparative to form the superlative. The
synthetic and periphrastic variants are in free variation. For example, the comparative form of \textit{psilós} `tall' has two varieties, \textit{psilóteros} and \textit{pio psilós}, and these can both combine with a \is{definite determiners}definite determiner to form a superlative. These two variants appear to be freely interchangeable, although the synthetic one may be slightly more commonplace. For all of the types of examples we elicited, many of which are presented below, both variants were judged to be acceptable.

\begin{table}[h]
\caption{Declension of the definite article in \ili{Greek}}
\label{tab:coppockstrand:2}
\begin{tabularx}{0.5\textwidth}{ l X X X }
\lsptoprule
\textsc{singular} \\\midrule
& \textsc{masc.} & \textsc{neut.} & \textsc{fem.}\\\midrule
\textsc{nom} & \textit{o} & \textit{to} & \textit{i}\\
\textsc{gen} & \textit{tou} & \textit{tou} & \textit{tis}\\
\textsc{acc} & \textit{to(n)} & to & \textit{ti(n)}\\\midrule
\textsc{plural} \\\midrule
& \textsc{masc.} & \textsc{neut.} & \textsc{fem.}\\\midrule
\textsc{nom} & \textit{oi} & \textit{ta} & \textit{oi}\\
\textsc{gen} & \textit{ton} & \textit{ton} & \textit{ton}\\
\textsc{acc} & \textit{tous} & \textit{ta} & \textit{tis} \\\lspbottomrule
\end{tabularx}
\end{table}

\subsection{Quality superlatives}\is{quality superlatives}

In \isi{adnominal superlatives}, there is always a definite article, which agrees in gender and \isi{number} with the modified \is{nouns}noun.\footnote{For reference, the inflectional paradigm for the definite article is as in \tabref{tab:coppockstrand:2}. We suppress the agreement features in our glosses for the sake of readability.}
The definite article is present regardless of whether an \is{absolute readings of superlatives|(}absolute or \is{relative readings of superlatives|(}relative interpretation is intended. Hence, example \REF{ex:coppockstrand:7} is ambiguous:\footnote{Thanks to Haris Themistocleous\ia{Themistocleus, Haris} and Stergios Chatzikyriakidis\ia{Chatzikyriakidis, Stergios} for judgments and discussion.}

\ea \label{ex:coppockstrand:7}
\gll O Stellios odigei \textbf{to} \textbf{pio} \textbf{grigoro} \textbf{aftokinito}.\\
the Stellios drives \defn{} \cmpr{} fast car\\
\glt `Stellios drives \textbf{the fastest car}.'
\z

Example \REF{ex:coppockstrand:8} strongly favors a relative interpretation; \is{definiteness marking}definiteness\hyp{}marking is obligatory here as well.

\ea \label{ex:coppockstrand:8}
\gll Den eimai ego afti me \textbf{ti} \textbf{leptoteri} \textbf{mesi} stin oikogeneia.\\
not I self she with \defn{} thin.\cmpr{} middle in family\\
\glt `I'm not the one with \textbf{the thinnest waist} in the family.'
\z

Note that the periphrastic variety \textit{ti pio lepti mesi} `the thinnest waist', lit. `the more thin waist', is equally acceptable here according to our consultants.

Absolute and relative readings of \isi{adnominal superlatives} are similar to each other and to ordinary \isi{adjectives} with respect to syntactic behavior as well. \ili{Greek} has a much-discussed construction in which the order of the adjective and the noun can be reversed called ``\isi{determiner spreading}''; see \citet[19]{Alexiadou2014} for an extensive list of references. The interpretive effect of determiner spreading is similar to that of placing an adjective postnominally in Romance: generally, it is restricted to restrictive modifiers \citep{AlexiadouWilder1998}. But unlike in Romance, this construction involves an extra \is{definite determiners}definite determiner, as can be seen in \REF{ex:coppockstrand:9}:

\ea \label{ex:coppockstrand:9}\il{Greek}
\begin{xlist}
\ex
\gll to kokino to podilato\\
 \defn{} red  \defn{} bicycle\\
\glt `the red bicycle'
\ex 
\gll  to podilato to kokino\\
  \defn{} bicycle  \defn{} red\\
\glt  `the red bicycle'
\end{xlist}
\z

\is{determiner spreading}Determiner spreading can involve superlatives; \citet{Alexiadou2014} discusses the example in \REF{ex:coppockstrand:10}, which has an \is{absolute readings of superlatives|)}absolute reading, referring to a particular cat:

\ea \label{ex:coppockstrand:10} \il{Greek}
\gll Spania haidevo \textbf{tin} \textbf{mikroteri} \textbf{ti} \textbf{gata}.\\
  seldom pet  \defn{} smallest the cat\\
  \glt `I seldom pet \textbf{the smallest cat}.'
\z

Intuitions appear to be somewhat murky when it comes to determiner spreading with relative readings, but example \REF{ex:coppockstrand:11}, a variant of \REF{ex:coppockstrand:8}, was judged as acceptable  by our consultants:

\ea \label{ex:coppockstrand:11}\il{Greek}
\gll Den eimai ego afti me \textbf{ti} \textbf{leptoteri} \textbf{ti} \textbf{mesi}  stin oikogeneia.\\
 not be.1\sg{} I she with the thin.\cmpr{}  \defn{} waist  in family\\
 \glt `I'm not the one with \textbf{the thinnest waist} in the family.'
\z

This evidence suggests that the comparative \is{adjectives}adjective in an \is{adnominal superlatives}adnominal superlative may be structurally analogous to an ordinary adjective in a \is{determiners}determiner-adjective-\is{nouns}noun sequences, and that the article is in its ordinary position.

\is{adverbial superlatives}Adverbial \isi{quality superlatives} are different, however; they do not involve a definite article, as can be seen in (\ref{ex:coppockstrand:12}) and (\ref{ex:coppockstrand:13}):

\ea\il{Greek} \label{ex:coppockstrand:12}
\gll I aderfi mou trechei \textbf{pio} \textbf{grigora}.\\
\defn{} sister my runs \cmpr{} fast\\
\glt `My sister runs \textbf{the fastest}.'
\z
 
\ea\il{Greek} \label{ex:coppockstrand:13}
\gll Pios tragoud\'ai \textbf{pio} \textbf{kal\'a}?\\
who sings more good\\
\glt `Who sings the best?' \citep[16, ex.\ 71]{Dobrovie-SorinGiurgea2015}\\
\z

Inserting a definite article before \textit{pio} is not possible in this sentence, e.g. *\textit{I aderfi mou trechei \textbf{to} pio grigora}. As \citet{Dobrovie-SorinGiurgea2015} point out, this shows that the definite article is not an integral part of superlative-marking in \ili{Greek}.

\subsection{Quantity superlatives}\is{quantity superlatives}

Like \isi{quality superlatives}, quantity superlatives are formed though the combination of a definite article with a comparative form, which may be either periphrastic, as in \REF{ex:coppockstrand:14}, or synthetic, as in \REF{ex:coppockstrand:15}. These two examples have \is{relative readings of superlatives|)}relative readings.

\ea \label{ex:coppockstrand:14}\il{Greek}
\gll Apó óla ta paidiá sto scholeío, egó paízo \textbf{ta} \textbf{pio} \textbf{pollá} \textbf{órgana}.\\
of all  \defn{} kids at school, I play \defn{} \cmpr{} many instruments\\
\glt `Of all the kids in my school, I'm the one who plays \textbf{the most instruments}.'
\z

\ea \label{ex:coppockstrand:15}\il{Greek}
\gll Eimai aftos pou pinei \textbf{ton}  \textbf{ligotero} \textbf{kafe}.\\
I he who drinks \defn{} little.\cmpr{} coffee\\
\glt `I am the one who drinks \textbf{the least coffee.}'
\z

\is{definiteness marking|(}Definiteness-marking is not optional here. Note that the word for `many' is transparently contained within the superlative phrase in \REF{ex:coppockstrand:14}.

Definite-marked \isi{quantity superlatives} are also regularly used for expressing a \is{proportional readings of superlatives}proportional interpretation. Sentences (\ref{ex:coppockstrand:16}--\ref{ex:coppockstrand:18}) are some examples from our data:

\ea \label{ex:coppockstrand:16}\il{Greek}
\gll \textbf{S-ta} \textbf{perissótera} \textbf{paidiá} sto scholeío mou arései na paízoun mousikí.\\
\dat-\defn{} many.\cmpr{} kids at school mine like to play music\\ 
\glt `Most of the kids in my school like to play music.'
\z


\ea \label{ex:coppockstrand:17}\il{Greek}
\gll I mamá éftiaxe biskóta chthes kai éfaga \textbf{ta} \textbf{perissótera}.\\
the mom made cookies yesterday and ate \defn{} many.\cmpr{}\\
\glt `Mom baked cookies yesterday and I ate \textbf{most of them}.'
\z


\ea \label{ex:coppockstrand:18}\il{Greek}
\gll Ípia epísis \textbf{to} \textbf{perissótero} \textbf{gála}.\\
drank also \defn{} much.\cmpr{} milk\\
\glt `I drank \textbf{most of the milk}, too.'
\z

Definiteness-marking is not optional here either.

Interestingly, there is a contrast between \is{quality superlatives}quality and quantity in the adverbial domain. \is{adverbial superlatives}Adverbial quantity superlatives appear to require a definite article, as in \REF{ex:coppockstrand:19}:\footnote{Thanks to a reviewer for pointing this out, and to Stavroula Alexandropoulou\ia{Alexandropoulou, Stavroula} for discussion.}

\ea \label{ex:coppockstrand:19} \il{Greek}
\gll O Pavlos milaei \textbf{to} \textbf{ligotero}.\\
 \defn{} Paul talks  \defn{} little.\cmpr{}\\
\glt `Paul talks the least'
\z

Removing the definite article in \REF{ex:coppockstrand:19} yields a comparative interpretation, `Paul talks less'. Notice that \textit{talk} is intransitive, so it is unlikely that \textit{to ligotero} is serving as the object of the verb. Further evidence that the construction in question is really adverbial comes from the fact that \is{definiteness marking|)}definite-marked quantity superlatives can be coordinated with non-definite-marked \is{adverbial superlatives}adverbial \isi{quality superlatives}, as is the case in \REF{ex:coppockstrand:20}:\is{definite articles|)}

\ea \label{ex:coppockstrand:20}\il{Greek}
\gll O Pavlos milaei {\ob}pio grigora apo olus ke \textbf{to} \textbf{perisotero}{\cb}.\\
\defn{} Paul talks [\cmpr{} fast of all.\acc{} and  \defn{} much.\cmpr{}]\\
\glt `Paul talks the fastest of all and the most'
\z

Thus \is{adverbial superlatives}adverbial \isi{quantity superlatives} pattern with \is{adnominal superlatives}adnominal quantity superlatives and \isi{quality superlatives}, and differently from adverbial quality superlatives.

Although quantity superlatives look morphologically very much like quality superlatives, there is a slight difference in their syntactic behavior.
\is{definiteness spreading|(}Definiteness spreading appears to be somewhat less acceptable with quantity superlatives than with quality superlatives. None of our consultants were entirely comfortable with examples (\ref{ex:coppockstrand:21}-\ref{ex:coppockstrand:22}) (although they were characterized as ``syntactically perfect''), and some rejected them:

\ea \label{ex:coppockstrand:21}\il{Greek}
\begin{xlist}
\ex[??]{
\gll Éfaga \textbf{ta} \textbf{perissotera} \textbf{ta} \textbf{biskóta}.\\
ate.1\sg{}  \defn{} much.\cmpr{} the cookies\\
\glt Intended: `I ate \textbf{the most cookies}' or `I ate \textbf{most of the cookies}.'}
\ex[??]{
\gll Éfaga \textbf{ta} \textbf{biskóta} \textbf{ta} \textbf{perissotera}.\\
ate.1\sg{}  \defn{} cookies  \defn{}  much.\cmpr{}\\
\glt Intended: `I ate \textbf{the most cookies}' or `I ate \textbf{most of the cookies}.'}
\end{xlist}
\z


\ea \label{ex:coppockstrand:22}\il{Greek}
\begin{xlist}
\ex[??]{
\gll Eimai aftos pou pinei \textbf{ton} \textbf{ligotero} \textbf{ton} \textbf{kafe}.\\
be.1\sg{} him who drinks  \defn{} little.\cmpr{}  \defn{} coffee\\
\glt `I'm the one who drinks \textbf{the least coffee}.'}
\ex[??]{ 
\gll Eimai aftos pou pinei \textbf{ton} \textbf{kafe} \textbf{ton} \textbf{ligotero}.\\
be.1\sg{} him who drinks   \defn{} coffee  \defn{} little.\cmpr{}\\
\glt `I'm the one who drinks \textbf{the least coffee}.'}
\end{xlist}
\z

So definiteness-spreading appears to be somewhat more restricted in the quantity domain.

However, \citet{Giannakidou2004} gives examples such as the following:

\ea[]{ \label{ex:coppockstrand:23}\il{Greek}
\gll \textbf{I} \textbf{perissoteri} \textbf{i} \textbf{fitites} efygan noris.\\
 \defn{} most  \defn{} students left early\\
\glt `Most of the students left early.'}
\z

It is unclear to us whether this should be seen as an instance of \isi{determiner spreading} or a construction in which \textit{i perissoteri} behaves as a \is{quantifiers}quantifier for which \textit{i fitites} serves as the restrictor. According to one native \ili{Greek} speaker we have consulted, the variant in \REF{ex:coppockstrand:23} is much better than a version in which the \is{nouns}noun precedes the \is{quantifiers}quantifier:

\ea[?]{ \label{ex:coppockstrand:24}\il{Greek}
\gll \textbf{I} \textbf{fitites} \textbf{i} \textbf{perissoteri} efygan noris.\\
  \defn{} students  \defn{} most left early \\}
\z

Example \REF{ex:coppockstrand:24} is fully acceptable only with comma intonation separating \textit{the students} from \textit{the most}, and serves as an answer to the question \textit{What happened with the students?}, rather than \textit{Who left early?} We see an even stronger contrast with \textit{ligotero} `less', which doesn't give rise to \is{proportional readings of superlatives}proportional readings. 

\ea[]{\label{ex:coppockstrand:25}\il{Greek}
\gll \textbf{Ton} \textbf{ligotero} \textbf{ton} \textbf{kafe} ton ipia egho.\\
 \defn{} less  \defn{} coffee it drink.1\sg{} I\\
\glt `I drink the least coffee.'}
\z

\ea[*]{\il{Greek}
\gll \textbf{Ton} \textbf{kafe} \textbf{ton} \textbf{ligotero} ton ipia egho.\\
 \defn{} coffee  \defn{} less it drink.1\sg{} I\\
}
\z 

Note that \REF{ex:coppockstrand:25} is ungrammatical without the subject pronoun \textit{egho}, even though \ili{Greek} is normally a pro-drop language; this is presumably because of the requirement of \isi{focus} for relative readings.

This evidence  suggests that the structure in \REF{ex:coppockstrand:23} is not actually a definiteness-spreading structure but actually one in which \textit{i fitites} behaves like a \is{partitives}partitive argument of \textit{i perissoteri}.  More generally, we take these facts to show that definiteness-spreading is not possible with \isi{quantity superlatives} in \ili{Greek}.

To summarize the situation for \ili{Greek}: \is{definiteness marking}definiteness-marking appears with every type of superlative \textit{except} \is{adverbial superlatives}adverbial \isi{quality superlatives}. This list includes \is{adnominal superlatives}adnominal \isi{quality superlatives} on both relative and proportional readings, and both \is{adnominal superlatives}adnominal and adverbial \isi{quantity superlatives}. \is{relative readings of superlatives}Relative and proportional readings are available for \is{adnominal superlatives}adnominal \isi{quantity superlatives} modifying both \is{mass/count distinction}\isi{mass nouns} and \isi{count nouns}. There is also full agreement with the noun in all cases where there is a noun to agree with. So quantity superlatives are morphologically very similar to quality superlatives overall. However, quantity superlatives differ from quality superlatives with respect to \is{definiteness spreading|)}definiteness-spreading, suggesting that the two types are not syntactically parallel. 


\section{Romanian} \label{sec:coppockstrand:3}\largerpage[2]

We turn now to \ili{Romanian}, which is like \ili{Greek} is some respects, but not in others. It uses \defcomp{} for both relative and proportional readings, but there is evidence that the \is{definite articles}definite article is more tightly knit with the comparative here than it is in \ili{Greek}.

\subsection{Quality superlatives}\is{quality superlatives}\largerpage[2]

Example \REF{ex:coppockstrand:27} shows a predicative use of a superlative in \ili{Romanian}, \REF{ex:coppockstrand:28} an attributive use, and \REF{ex:coppockstrand:29} an \is{adverbial superlatives}adverbial use.

\ea \label{ex:coppockstrand:27}\il{Romanian}
\gll Pentru că eram \textbf{cea} \textbf{mai} \textbf{entuziasmată}.\\
for that I.was \defn{} \cmpr{} enthustiastic\\
\glt `Because I (fem.) was \textbf{the most enthusiastic}.' 
\z

\ea \label{ex:coppockstrand:28}\il{Romanian}
\gll A scris \textbf{cea} \textbf{mai} \textbf{frumoasă} \textbf{compunere}.\\
has written \defn{} \cmpr{} beautiful composition.\acc\\
\glt `She wrote \textbf{the most beautiful composition}.'
\z

\ea \label{ex:coppockstrand:29}\il{Romanian}
\gll Sora mea poate alerga \textbf{cel} \textbf{mai} \textbf{repede}.\\
sister my can run \defn{} \cmpr{} fast\\
\glt `My sister can run \textbf{the fastest}.'
\z

In \REF{ex:coppockstrand:27} and \REF{ex:coppockstrand:28}, \textit{cea} is a feminine singular form of \textit{cel}.  In \REF{ex:coppockstrand:29}, we have the invariant, default form.\footnote{\citet[315]{PanaDindelgan2013} points out that adverbial \textit{cel} can receive dative case marking, so it is not entirely invariable.}
We will not gloss the agreement features, but simply refer the reader to the inflectional paradigm for the \is{demonstratives}demonstrative in \tabref{tab:coppockstrand:3}, taken from \citet[53]{Cojocaru2003}. 
Note also that the \is{adjectives}adjective \textit{frumosă} `beautiful' shows feminine singular agreement with the \is{nouns}noun \textit{compunere} `composition'.

\begin{table}[bh]
\caption{Inflectional paradigm for \textit{cel} in \ili{Romanian}}
\label{tab:coppockstrand:3}
\begin{tabularx}{0.55\textwidth}{ l X X }
\lsptoprule
\textsc{singular}\\\midrule
& \textsc{masc.}, \textsc{neut.} & \textsc{fem.}\\\midrule  
N., A. & \textit{cel} & \textit{cea}\\
G, D. & \textit{celui} & \textit{celei}\\\midrule
\textsc{plural}\\\midrule
& \textsc{masc.} & \textsc{fem.}, \textsc{neut.}\\\midrule
N., A. & \textit{cei} & \textit{cele}\\
G., D. & \textit{celor} & \textit{celor}\\\lspbottomrule
\end{tabularx}
\end{table}


We gloss \textit{cel} here as \defn, in order to bring out the parallels with other \defcomp{} languages, but it should be kept in mind that this element is not the most direct correlate of \ili{English} \textit{the} in the language. 
\textit{Cel} is not found in ordinary, simple \isi{definites}; instead a suffix is used. For example, in \REF{ex:coppockstrand:30a}, we have a feminine singular definite ending \textit{-a}, modified from the stem-inherent \textit{-ă} illustrated in \REF{ex:coppockstrand:30b}. We gloss this ending here as \defn{} as well.

\ea \label{ex:coppockstrand:30}\il{Romanian}
\begin{xlist}
\ex \label{ex:coppockstrand:30a}
\gll Carte-a e pe mas-a mare.\\
map-\defn{} is on table-\defn{} big\\
\glt `The map is on the big table.'

\ex \label{ex:coppockstrand:30b}
\gll Carte-a e pe o masă mare.\\
map-\defn{} is on a table big\\
\glt `The map is on a big table.'
\end{xlist}
\z

Note also that in traditional grammar (e.g.\ \citealt{Cojocaru2003}), \textit{cel} is classified as a \is{demonstratives}demonstrative, though it has additional functions as well. For instance, it can double a definite suffix \citep{Alexiadou2014}:

\ea \label{ex:coppockstrand:31}\il{Romanian}
\gll Legile {\op}cele{\cp} importante n'au fost votate.\\
laws-\defn{} (\defn{}) important have not\_been voted\\
\glt `The laws which were important have not been passed.'
\z

See \citet[53--62]{Alexiadou2014} for a recent discussion of this phenomenon and its relation to \ili{Greek} \isi{determiner spreading}.

As \REF{ex:coppockstrand:31} implies,
\ili{Romanian} has two word order options for \isi{adjectives}, including superlatives. This choice bears on the presence or absence of a definite suffix on the \is{nouns}noun. If the adjective precedes the modified noun as in \REF{ex:coppockstrand:28}, repeated in \REF{ex:coppockstrand:32a}, this noun remains uninflected. If the noun precedes the adjective, as in \REF{ex:coppockstrand:31} and \REF{ex:coppockstrand:32b}, the noun receives \is{definiteness marking}definiteness marking \citep[53]{Cojocaru2003}.

\ea \label{ex:coppockstrand:32}\il{Romanian}
\begin{xlist}
\ex \label{ex:coppockstrand:32a}
\gll A scris \textbf{cea} \textbf{mai} \textbf{frumoasă} \textbf{compunere}.\\
has written \defn{} \cmpr{} beautiful composition.\acc\\
\glt `She wrote \textbf{the most beautiful composition}.'

\ex \label{ex:coppockstrand:32b}
\gll A scris \textbf{compunere-a} \textbf{cea} \textbf{mai} \textbf{frumoasă}.\\
has written composition-\defn{} \defn{} \cmpr{} beautiful\\
\glt `She wrote \textbf{the most beautiful composition}.'
\end{xlist}
\z

According to \citet{Teodorescu2007}, the prenominal variant \REF{ex:coppockstrand:32a} and the postnominal variant \REF{ex:coppockstrand:32b} have the same interpretive options. The following is an example favoring a \is{relative readings of superlatives}relative interpretation; both orders, shown in \REF{ex:coppockstrand:33a} and \REF{ex:coppockstrand:33b}, are reportedly fine, although all four of the \ili{Romanian} speakers we consulted spontaneously translated the sentence indicated in the \ili{English} gloss using the prenominal variant \REF{ex:coppockstrand:33a}.\footnote{Thanks to Gianina Iordachioaia\ia{Iordachioaia, Gianina} for help and discussion.}


\ea \label{ex:coppockstrand:33}

\begin{xlist}

\ex \label{ex:coppockstrand:33a}\il{Romanian}
\gll Eu nu sunt cea din familie cu \textbf{cel} \textbf{mai} \textbf{subţire} \textbf{talie}.\\
I not be.1\sg{} \defn{} from family.\acc{} with \defn{} \cmpr{} thin waist\\
\glt `I am not the one in my family with \textbf{the thinnest waist}.'

\ex \label{ex:coppockstrand:33b}
\gll Eu nu sunt cea din familie cu \textbf{tali-a} \textbf{cea} \textbf{mai} \textbf{subtire}.\\
I not be.1\sg{} \defn{} from family.\acc{} with  waist-\defn{} \defn{} \cmpr{} thin\\
\glt `I am not the one in my family with \textbf{the thinnest waist}.'
\end{xlist}
\z

Note that postnominal \isi{adjectives} typically receive an intersective interpretation \citep{Cornilescu1992,MarchisAlexadiou2009,Teodorescu2007}:

\ea \label{ex:coppockstrand:34}\il{Romanian}
\begin{xlist}
\ex \label{ex:coppockstrand:34a}
\gll o poveste advărată\\
a story true\\
\glt `a story that is true' (not `quite a story')
\ex \label{ex:coppockstrand:34bb}
\gll o advărată poveste\\
a true story\\
\glt `a story that is true' or `quite a story'\label{rom:prenominal-story}
\ex \label{ex:coppockstrand:34b}
\gll Această poveste este advărată.\\
this story is true\\
\glt `This story is true.'
\end{xlist}
\z

The postnominal adjective in \REF{ex:coppockstrand:34a} has only the interpretation that the adjective in \REF{ex:coppockstrand:34b} has, while the prenominal adjective in \REF{ex:coppockstrand:34bb} can also have a non-intersective interpretation.
If this applies to superlatives, then the fact that both relative and \is{absolute readings of superlatives}absolute readings of superlatives are possible in post-nominal position suggests that both relative and absolute readings are, or can be, restrictive readings.

\citet{Dobrovie-SorinGiurgea2015} give a number of arguments that \textit{cel mai} + AP form a constituent that sits in the specifier of \is{determiner phrases}DP. One is the striking fact that
\textit{cel} can be preceded by an \is{indefinite articles}indefinite article as in \REF{ex:coppockstrand:35} \citep[15, ex.\ 64]{Dobrovie-SorinGiurgea2015}:

\ea \label{ex:coppockstrand:35}\il{Romanian}
\gll Există \^intotdeauna \textbf{un} \textbf{cel} \textbf{mai} \textbf{mic} \textbf{divizor} \textbf{comun} a două elemente.\\
exists always a \defn{} \cmpr{} small divisor common of two elements\\
\glt `There always exists \textbf{a smallest common factor} of two elements.'
\z

Their second argument is that \textit{cel} is always present in superlatives, both when the superlative is post-nominal as in \REF{ex:coppockstrand:32b}, and when it is \is{adverbial superlatives}adverbial as in \REF{ex:coppockstrand:36}. 

\ea \label{ex:coppockstrand:36}\il{Romanian}
\gll Vi fi premiat cel care va scrie \textnormal{\#}{\op}\textbf{cel}{\cp} \textbf{mai} \textbf{clar}.\\
will be awarded-prize \defn{} which will write \phantom{\#(}\defn{} more clearly\\
\glt `The one who writes \textbf{the most clearly} will be awarded a prize.' \citep[15, ex.\ 66]{Dobrovie-SorinGiurgea2015}
\z

Their third argument is that definite comparatives involve the suffix (which appears on the \is{adjectives}adjective preceding the head \is{nouns}noun) rather than \textit{cel}, as in \REF{ex:coppockstrand:37}:

\ea \label{ex:coppockstrand:37}\il{Romanian}
\gll ... dar cu \textbf{mult} \textbf{mai} \textbf{difficil-ul} \textbf{obiectiv} al ...\\
... but with much more difficult-the goal of ... \\
\glt `... but with \textbf{the much more difficult goal} of ...'
\z

So \textit{cel} must have some meaning or function distinct from the suffix. They also observe that the unmarked position of comparatives is postnominal, whereas the unmarked position for superlatives is prenominal, and note that \textit{cel} cannot be separated from a prenominal comparative by \isi{numerals} (though numerals can normally follow \textit{cel}), which can be seen in the contrast between \REF{ex:coppockstrand:38a} and \REF{ex:coppockstrand:38b}:

\ea \label{ex:coppockstrand:38}\il{Romanian}
\begin{xlist}
\ex[*]{ \label{ex:coppockstrand:38a}
\gll cei doi mai înalţi munţi\\
\defn{} two more high mountains\\
}
\ex[]{ \label{ex:coppockstrand:38b}
\gll cei mai înalţi doi munţi\\
\defn{} more high two mountains\\
\glt `the two highest mountains'}
\end{xlist}
\z

These arguments have us convinced that \textit{cel} in superlatives is not a direct dependent of the modified noun, but rather forms a phrase with the comparative marker and the \is{adjectives}adjective to the exclusion of the noun. So the structure of \textit{cea mai frumoasă compunere} `the most beautiful composition' appears to be:

\ea \label{ex:coppockstrand:39}\il{Romanian}
\begin{forest}
	calign primary angle=-60,
	calign secondary angle=60
	[~,calign=center, nice empty nodes 
		[ [\textit{cea mai frumoasă}, roof]
		]
		[\textit{compunere}
		]
	]
\end{forest}
\z



\subsection{Quantity superlatives}\is{quantity superlatives}

Now let us turn to quantity superlatives in \ili{Romanian}. As with \isi{quality superlatives}, \is{definiteness marking}definiteness-marking is ubiquitous, even with \is{adverbial superlatives}adverbials, as in \REF{ex:coppockstrand:40}:

\ea \label{ex:coppockstrand:40}\il{Romanian}
\gll  Personajele de care se râdea \textbf{cel} \textbf{mai} \textbf{mult} erau Leana şi nea Nicu.\\
characters of which they laughed \defn{} \cmpr{} much were Leana and uncle Nicu\\
\glt `The characters they laughed at the most were Leana and uncle Nicu.'
\z

And the \defcomp{} construction can have both \is{proportional readings of superlatives|(}proportional and \is{relative readings of superlatives}relative readings in \ili{Romanian}. Examples \REF{ex:coppockstrand:41} and \REF{ex:coppockstrand:42} have relative readings (the latter from \citealt[11]{Teodorescu2007}).

\ea \label{ex:coppockstrand:41}\il{Romanian}
\gll Eu sunt cel care canta la \textbf{cele} \textbf{mai} \textbf{multe} \textbf{instrumente}.\\
I am the which plays to \defn{} \cmpr{} {much} instruments\\
\glt `I am the one who plays \textbf{the most instruments}.'
\z

\ea \label{ex:coppockstrand:42}\il{Romanian}
\gll Dan a băut \textbf{cea} \textbf{mai} \textbf{multă} \textbf{bere}.\\
Dan has drunk \defn{} \cmpr{} much beer\\
\glt `Dan drank \textbf{the most beer}.'
\z

Example \REF{ex:coppockstrand:43} is a case with a proportional reading, using the \is{partitives}partitive \is{prepositions}preposition \textit{dintre}:\footnote{The \is{prepositions}preposition \textit{dintre} (\textit{din} with singular complements) is used in \ili{Romanian} to introduce an explicit \is{comparison classes}comparison class in superlative constructions, e.g.\ \textit{El scrie cel mai bine dintre toţi}, `He writes the best of all', lit. `He writes the more good among all' \citep[169]{Cojocaru2003}.  \textit{Dintre} is also used in \is{quantifiers}quantificational \is{partitives}partitive constructions, e.g.\ \textit{Unul dintre ei prezintă proiectul} `One of them is presenting the project'.}

\ea \label{ex:coppockstrand:43}
\gll \textbf{Cele} \textbf{mai} \textbf{multe} \textbf{dintre} \textbf{copiii} care merge la scoala mea place să se joace muzica.\\
\defn{} \cmpr{} {much} of kids.\defn{} who go at school mine like to \textsc{refl} play music \\
\glt `\textbf{Most of the kids} who go to my school like to play music.' 
\z

We also find non-partitive uses as in \REF{ex:coppockstrand:44} and \REF{ex:coppockstrand:45}:

\ea \label{ex:coppockstrand:44}\il{Romanian}
\gll \textbf{Cei} \textbf{mai} \textbf{mulţi} \textbf{elevi} din clasa mea au plecat devreme.\\
\defn{} \cmpr{} many students from class.the my have left early\\
\glt `\textbf{Most of the students} in my class have left early.'
\z

\ea \label{ex:coppockstrand:45}\il{Romanian}
\gll \textbf{Cele} \textbf{mai} \textbf{multe} \textbf{lebede} sunt albe.\\
\defn{} \cmpr{} many swans are white\\
\glt `\textbf{Most swans} are white.'
\z

But the syntactic position of the superlative phrase may not be the same as with \isi{quality superlatives}: in contrast to quality superlatives, \isi{quantity superlatives} are normally only permitted prenominally \citep[11]{Teodorescu2007}, as example \REF{ex:beer} shows.

\ea[*]{\il{Romanian}
\gll Dan a băut bere-a \textbf{cea} \textbf{mai} \textbf{multă}.\\
Dan has drunk beer-\defn{} \defn{} \cmpr{} much\\
\glt Intended:
`Dan drank \textbf{the most beer}.' \label{ex:beer}}
\z

\citet{Dobrovie-Sorin2015} does give the example of a postnominal \textit{cel mai mult}-construction in \REF{ex:coppockstrand:47a} and \REF{ex:coppockstrand:47b}, but says that  it does not give rise to a \is{relative readings of superlatives}relative \textit{or} proportional reading, but ``comparison between predefined groups'', where the \is{noun phrases}noun phrase refers to one of these groups.

\ea \label{ex:coppockstrand:47}\il{Romanian}
\begin{xlist}
\ex[]{\label{ex:coppockstrand:47a}
\gll  \textbf{Cele} \textbf{mai} \textbf{multe} \textbf{lebede} sunt albe.\\
\defn{} \cmpr{} many swans are white\\
\glt `\textbf{Most swans} are white.'}
\ex[?]{ \label{ex:coppockstrand:47b}
\gll \textbf{Lebedele} \textbf{cele} \textbf{mai} \textbf{multe} sunt albe.\\
swans.\defn{} \defn{} \cmpr{} many are  white\\
\glt `\textbf{The more/most numerous (group of) swans} are white.'}
\end{xlist}
\z

This reading is referential, and distinct from the proportional reading that arises in prenominal position, rather than \is{quantifiers}quantificational. 

Interestingly, \REF{ex:coppockstrand:42} above does not have a proportional interpretation. According to \citet{Dobrovie-Sorin2015}, this is tied to the fact that a \is{mass nouns}mass noun is involved. 
Indeed, in our data, a proportional interpretation, in the case of mass quantification (shown in \ref{ex:coppockstrand:48} and \ref{ex:coppockstrand:49}), typically involves a `majority' or `part' noun instead, just as in other \ili{Romance} languages:

\ea \label{ex:coppockstrand:48}\il{Romanian}
\gll Am baut \textbf{majoritatea} \textbf{laptelui}.\\
have drunk majority milk\\
\glt `I drank \textbf{most of the milk}.'
\z

\ea \label{ex:coppockstrand:49}\il{Romanian}
\gll Am baut \textbf{mai} \textbf{mare} \textbf{parte} \textbf{a} \textbf{laptelui}.\\
have drunk \cmpr{} big part \gen{} milk\\
\glt `I drank \textbf{most of the milk}.'
\z

Dobrovie-Sorin argues that \textit{cel mai mult} functions as a complex proportional \is{quantifiers}quantifier, one that expects a count down denotation as an argument. Providing further evidence for this view, she claims that a proportional reading is not \textit{always} available for \isi{count nouns}, either, pointing to a contrast in acceptability between \REF{ex:coppockstrand:50} and \REF{ex:coppockstrand:51}:

\ea[]{ \label{ex:coppockstrand:50}\il{Romanian}
\gll \textbf{Cei} \textbf{mai} \textbf{mulţi} \textbf{elevi} din clasa mea au plecat devreme.\\
\defn{} \cmpr{} many students.\defn{} of class.\defn{} my have left early.\\
\glt `\textbf{Most students} in my class left early.' \citep[395]{Dobrovie-Sorin2015}}
\z

\ea[*]{\label{ex:coppockstrand:51}\il{Romanian}
\gll \textbf{Cei} \textbf{mai} \textbf{mulţi} \textbf{băieţi} s-au adunat în sala asta.\\
\defn{} \cmpr{} many boys  \textsc{refl}-have gathered in room.\defn{} this.\\
\glt `\textbf{Most of the boys} have gathered in this room.' \citep[395]{Dobrovie-Sorin2015}}
\z 

She ascribes these differences to whether or not the nuclear scope is filled with a distributive predicate. The unacceptability of \REF{ex:coppockstrand:51} is explained under the assumption that the subject  \is{noun phrases}noun phrase is \is{quantifiers}quantificational rather than referential. This adds to the  evidence in favor of \citegen{Dobrovie-Sorin2015} idea that \textit{cel mai mult} has \is{grammaticalization}grammaticalized as a \is{determiners}proportional determiner.

To summarize:  superlatives are always definite in \ili{Romanian}. Evidence involving \isi{quality superlatives} suggests that the definite element is integrated more closely with the comparative element than with the modified noun, i.e. lower down in the structure, \is{definiteness marking}not signalling definiteness at the level of the full nominal.  Both \is{relative readings of superlatives}relative and proportional readings are available for \is{adnominal superlatives}adnominal \isi{quantity superlatives}, although the proportional readings are limited to \isi{count nouns}. The existence of proportional readings only with count nouns as well as the unacceptability of collective predicates suggests that \textit{cel mai mult} has grammaticalized into a \is{proportional readings of superlatives|)}proportional determiner \citep{Dobrovie-Sorin2015}.

\section{Ibero-Romance}\il{Romance!Ibero-Romance}

\subsection{Quality superlatives}\is{quality superlatives}

Predicative \is{adjectives}adjectival superlatives in \ili{Italian}, as in \REF{ex:coppockstrand:52}, and \ili{Spanish}, as in \REF{ex:coppockstrand:53}, normally involve a \is{definite articles|(}definite article:

\ea \label{ex:coppockstrand:52}
\gll Carla \`e \textbf{la} \textbf{pi\`u} \textbf{intelligente} di tutte queste studentesse.\\
Carla is \defn{} \cmpr{} intelligent of all these students\\ \jambox{(\ili{Italian})}
\glt`Carla is \textbf{the most intelligent} of all these students.' \citep[53]{deBoer1986}
\z

\ea \label{ex:coppockstrand:53}
\gll Ese carro es \textbf{el} \textbf{mejor}.\\
that car is \defn{} better\\ \jambox{(\ili{Spanish})}
\glt`That car is \textbf{the best}.' \citep[1]{Rohena-Madrazo2007}
\z

One exception, as illustrated in \REF{ex:coppockstrand:54}, is noted by \citet[53]{deBoer1986}, who gives the following predicative example without \is{definiteness marking|(}definiteness-marking.

\ea \label{ex:coppockstrand:54}
\gll il giorno in cui il nostro lavoro era \textbf{pi\`u} \textbf{faticoso}\\
\defn{} day in which \defn{} our work was \cmpr{} tiresome\\ \jambox{(\ili{Italian})}
\glt `the day on which our work was \textbf{most tiresome}'
\z

Here, even though the example is grammatically predicative, it has the flavor of a \is{relative readings of superlatives}relative reading, comparing days rather than  alternatives to the subject of the sentence \textit{il nostro lavoro} `our work'. The same example in \ili{French}, shown in \REF{ex:coppockstrand:55}, involves a definite article \ia{Cremers, Alexandre}(Alexandre Cremers, p.c.):

\ea \label{ex:coppockstrand:55}
\gll le jour où notre travail était \textbf{le} \textbf{plus} \textbf{fatiguant}\\
\defn{} day when our work was \defn{} \cmpr{} tiresome\\ \jambox{(\ili{French})}
\glt `the day on which our work was \textbf{most tiresome}'
\z

\citet[75]{Matushansky2008} reports a similar phenomenon in \ili{Spanish} presented in examples \REF{ex:coppockstrand:56} and \REF{ex:coppockstrand:57}:

\ea \label{ex:coppockstrand:56}
\gll la que es \textbf{m\'as} \textbf{alta}\\
\defn{} who is \cmpr{} tall\\ \jambox{(\ili{Spanish})}
\glt`the one who is \textbf{tallest}'
\z

\ea \label{ex:coppockstrand:57}
\gll la que est\'a \textbf{m\'as} \textbf{enojada}\\
\defn{} who is \cmpr{} annoyed\\ \jambox{(\ili{Spanish})}
\glt `the one who is \textbf{most annoyed}'
\z

In both these examples and in the \ili{Italian} example \REF{ex:coppockstrand:54}, \isi{uniqueness} is indicated with the help of a \is{relative clauses}relative clause. These patterns suggest that superlatives require marking of uniqueness in some fashion, not necessarily with an accompanying definite article.

As in \ili{French}, \isi{adnominal superlatives} can appear both pre- and post-nominally in \ili{Italian}, as the reader can see in \REF{ex:coppockstrand:58a} and \REF{ex:coppockstrand:58b}:

\ea \label{ex:coppockstrand:58}
\begin{xlist}
\ex \label{ex:coppockstrand:58a}
\gll La mamma fa \textbf{i} \textbf{biscotti} \textbf{più} \textbf{buoni} del mondo. \\
\defn{} mom makes \defn{} cookies \cmpr{} tasty of.\defn{} world\\ \jambox{(\ili{Italian})} 
\glt `Mom bakes \textbf{the yummiest cookies} in the whole world.'
\ex \label{ex:coppockstrand:58b}
\gll La mamma fa \textbf{i} \textbf{più} \textbf{buoni} \textbf{biscotti} del mondo. \\
\defn{} mom makes \defn{} \cmpr{} tasty cookies of.\defn{} world\\ 
\end{xlist}
\z

Normally, there is no definite article on a postnominal superlative in \ili{Italian}, although \citet{Plank2003} reports that both variants in \REF{ex:coppockstrand:59a} and \REF{ex:coppockstrand:59b} are acceptable, the latter ``putting greater emphasis on the \is{adjectives}adjective'':

\ea \label{ex:coppockstrand:59}
\begin{xlist}
\ex \label{ex:coppockstrand:59a}
\gll l'uomo \textbf{pi\`u} \textbf{forte}\\
\defn{}'man more strong\\ \jambox{(\ili{Italian})}
\glt `the \textbf{stronger} / \textbf{strongest} man'
\ex \label{ex:coppockstrand:59b}
\gll l'uomo \textbf{il} \textbf{pi\`u} \textbf{forte}\\
\defn{}'man the more strong\\
\glt`the \textbf{strongest} man'
\end{xlist}
\z

Example \REF{ex:coppockstrand:60} displays a postnominal superlative in \ili{Italian} with a \is{relative readings of superlatives}relative reading; here again there is no definite article:\footnote{According to \citet[11--12]{Cinque2010}, only the postnominal syntax is possible on \is{relative readings of superlatives}relative readings.  Here is a speculation as to how one might explain this in semantic/pragmatic terms: the prenominal position is normally hostile to non-restrictive modifiers in \ili{Italian} (e.g.\ *\textit{la presenza mera} vs. \textit{la mera presenza} `the mere presence'). \citet{Matushansky2008a} proposes that the modified noun saturates the \is{comparison classes}comparison class argument of a superlative, so that a superlative modifier combines with the \is{nouns}noun via \isi{Functional Application} rather than \isi{Predicate Modification}. This kind of analysis would yield an \is{absolute readings of superlatives}absolute reading; suppose this is how absolute readings arise. Then absolute readings would be non-restrictive and relative readings would be restrictive. Placing a superlative postnominally could then serve as an indication that an absolute reading is not intended.}

\ea \label{ex:coppockstrand:60}
\begin{xlist}
\ex[]{ 
\gll Non sono quello con \textbf{il} \textbf{girovita} \textbf{più} \textbf{sottile} in famiglia. \\ 
not am the.one with \defn{} waist \cmpr{} thin in family\\ \jambox{(\ili{Italian})}
\glt `I'm not the one with \textbf{the thinnest waist} in the family.'}
\ex[\#]{
\gll Non sono quello con \textbf{il}  \textbf{più} \textbf{sottile} \textbf{girovita} in famiglia.\\
not am the.one with \defn{} \cmpr{} thin waist  in family\\}
\end{xlist}
\z

\is{adverbial superlatives}Adverbial \isi{quality superlatives} systematically lack \is{definiteness marking|)}definiteness-marking in \ili{Italian}, as shown in example \REF{ex:coppockstrand:61} from \citet[53]{deBoer1986}:

\ea \label{ex:coppockstrand:61}
\gll Di tutte queste ragazze, Marisa lavora \textbf{più} \textbf{diligentemente}.\\ 
of all these kids Marisa works \cmpr{} diligently\\ \jambox{(\ili{Italian})}
\glt`Of all these kids, Marisa works \textbf{the most diligently}.'
\z

The same holds in \ili{Spanish}:

\ea \label{ex:coppockstrand:62}
\gll Juan es el que corre \textbf{m\'as} \textbf{r\'apido}.\\
Juan is \defn{} who runs \cmpr{} fast\\ \jambox{(\ili{Spanish})}
\glt`Joan is the one who runs \textbf{the fastest}.' \citep[1--2]{Rohena-Madrazo2007}
\z

As \citet{Rohena-Madrazo2007} notes, the \is{relative clauses}relative clause in \REF{ex:coppockstrand:62} is necessary in order for a superlative interpretation to arise. Example \REF{ex:coppockstrand:63} has only a comparative interpretation:

\ea \label{ex:coppockstrand:63}
\gll Juan corre \textbf{m\'as} \textbf{r\'apido}.\\
Juan  runs \cmpr{} fast\\ \jambox{(\ili{Spanish})}
\glt `Joan runs \textbf{faster}.'
\z

Thus a superlative interpretation does not freely arise on its own here; \isi{uniqueness} must somehow be signaled in the absence of a \is{determiners}determiner.

\subsection{Quantity superlatives}\is{quantity superlatives}\is{definite articles|)}

Naturally, we expect the definite article to mark the superlative \is{degrees}degree with quantity superlatives as it does with \isi{quality superlatives}. However, the definite article is sometimes absent even in superlative constructions. De Boer (\citeyear[53]{deBoer1986}) gives the example in \REF{ex:coppockstrand:64}; our informants consistently gave us translations like that in \REF{ex:coppockstrand:65} and \REF{ex:coppockstrand:66} for sentences involving \is{relative readings of superlatives}relative readings:

\ea \label{ex:coppockstrand:64}
\gll Dei nostri amici Luigi \`e quello che ha \textbf{pi\`u} \textbf{soldi}.\\
of.\defn{} our friends Luigi is the.one who has \cmpr{} money\\ \jambox{(\ili{Italian})}
\glt `Of our friends, Luigi is the one who has \textbf{the most money}.'
\z
  
\ea \label{ex:coppockstrand:65}
\gll Ma probabilmente è Hans che ha bevuto \textbf{più} \textbf{caffè}.\\
But probably it.is Hans who has drunk \cmpr{} coffee\\ \jambox{(\ili{Italian})}
\glt `But it is probably Hans who has drunk the most coffee.'
\z

\ea \label{ex:coppockstrand:66}
\gll Di tutti i ragazzi della mia scuola io sono quello che suona \textbf{più} \textbf{strumenti}.\\
of all \defn{} kids in.\defn{} my school I am the.one that plays \cmpr{} instruments\\ \jambox{(\ili{Italian})}
\glt `Of all the kids in my school, I'm the one who plays the most instruments.'
\z

Hence there is no overt morphological distinction between `more coffee' and `most coffee'.

Following \citet{BosqueBrucart1991}, \citet{Rohena-Madrazo2007} uses comparative and superlative ``codas'' to distinguish between comparative and superlative interpretations in \ili{Spanish}, as in \REF{ex:coppockstrand:67} and \REF{ex:coppockstrand:68} respectively:

\ea \label{ex:coppockstrand:67}
\gll \textbf{el} \textbf{ni\~no} \textbf{m\'as} \textbf{r\'apido} {\op}que todos nosotros{\cp}\\
\defn{} boy \cmpr{} fast (than all we)\\ \jambox{(\ili{Spanish})}
\glt `\textbf{the boy faster} (than all of us)'
\z

\ea \label{ex:coppockstrand:68}
\gll \textbf{el} \textbf{ni\~no} \textbf{m\'as} \textbf{r\'apido} {\op}de todos nosotros{\cp}\\
\defn{} boy \cmpr{} fast (of all we)\\ \jambox{(\ili{Spanish})}
\glt `\textbf{the fastest boy} (of all of us)'
\z

In \REF{ex:coppockstrand:67}, the boy is among `us', but not in \REF{ex:coppockstrand:68}. Using this technique, he shows that so-called ``free'' superlatives in \ili{Spanish}, as shown in \REF{ex:coppockstrand:69}, can be fronted before the verb, but comparatives cannot:\footnote{``Free superlatives'' include \isi{adverbial superlatives} like \textit{m\'as r\'apido} `the fastest' and \isi{quantity superlatives} like \textit{m\'as libros} `the most book'. In contrast, \is{incorporation}``incorporated superlatives'' such as \textit{el ni\~no m\'as r\'apido} `the fastest boy' are defined as being contained within an NP. The free/incorporated distinction in Spanish happens to draw a line between \is{adnominal superlatives}adnominal \isi{quality superlatives} on the one hand and \is{quantity superlatives}quantity and \isi{adverbial superlatives} on the other.}

\ea \label{ex:coppockstrand:69}
\gll Juan es el ni\~no que \textbf{m\'as} \textbf{libros} ley\'o {\op}de\textnormal{/*}que todos ellos{\cp}.\\
John is \defn{} boy that \cmpr{} books read (of/*than all them)\\ \jambox{(\ili{Spanish})}
\glt `Juan is the boy that read \textbf{the most books} (of/*than all of them).'
\z

This evidence suggests that the comparative and the superlative interpretations are really distinct.

Similarly, \textit{the most instruments} in `I'm the one who plays the most instruments' and \textit{the most coffee} in `Hans has drunk the most coffee' are translated without \is{definiteness marking|(}definiteness-marking in other \il{Romance!Ibero-Romance}Ibero-Romance languages, as we can see in the sets of examples in \REF{ex:coppockstrand:70} and \REF{ex:coppockstrand:71}:

\ea \label{ex:coppockstrand:70}
\settowidth\jamwidth{(\ili{Portuguese})}
\ea \textit{Yo soy el que toca \textbf{más instrumentos}.} \jam(\ili{Spanish})
\ex \textit{Eu sou o que toca \textbf{mais instrumentos}.} \jam(\ili{Portuguese})
\ex \textit{Jo sóc qui toca \textbf{més instruments}.} \jam (\ili{Catalan})
`I am the one who play \textbf{the most instruments}.'
\z 
\z

\ea \label{ex:coppockstrand:71}
\settowidth\jamwidth{(\ili{Portuguese})}
\ea \textit{Hans es el que ha bebido \textbf{más café}.} \jam(\ili{Spanish})
\ex \textit{Hans quem bebeu \textbf{mais café}.} \jam(\ili{Portuguese})
\ex \textit{Hans és probablement qui ha begut \textbf{més cafè}.} \jam(\ili{Catalan})
\sn`Hans is the one who has drunk \textbf{the most coffee}.'
\z
\z 

\is{adverbial superlatives}Adverbial \isi{quantity superlatives} also lack definiteness-marking, as \REF{ex:coppockstrand:72} and \REF{ex:coppockstrand:73} show:

\ea \label{ex:coppockstrand:72}
\gll ... uno che lavora \textbf{più} di tutti e parla \textbf{meno} di tutti. \\
... one who works \cmpr{} of all and speaks little.\cmpr{} of all\\ \jambox{(\ili{Italian})}
\glt `... one who works \textbf{most} of all and speaks \textbf{least} of all'
\z

\ea \label{ex:coppockstrand:73}
\gll Alberto es el que trabaja \textbf{m\'as}.\\
Alberto is \defn{} that works \cmpr\\ \jambox{(\ili{Spanish})}
\glt `Alberto is the one who works \textbf{the most}.' 
\z

Unlike in \ili{French} and \ili{Romanian}, a \is{definite articles|(}definite article would be ungrammatical preceding the comparative word here. Rather, \is{adverbial superlatives}adverbial \isi{quantity superlatives} the pattern of \is{adnominal superlatives}adnominal quantity superlatives here (as in all of the languages under consideration, in fact).

The \defcomp{} construction is generally not used to express \is{proportional readings of superlatives|(}proportional readings. Proportional \textit{most} is generally translated using other types of constructions, such as `the greater part' in \REF{ex:coppockstrand:74}:

\ea \label{ex:coppockstrand:74} 
\gll \textbf{Alla} \textbf{maggior} \textbf{parte} \textbf{dei} \textbf{bambini} nella mia scuola piace suonare.\\
of.\defn{} big.\cmpr{} part of.\defn{} kids in my school like play\\ 
\glt `\textbf{Most of the kids} in my school like to play (music).' \hfill (\ili{Italian})
\z

The same holds for the entire \il{Romance!Ibero-Romance}Ibero-Romance subfamily, as far as we can see, including \ili{Spanish}, \ili{Portuguese}, and \ili{Catalan}. For example,  \textit{most of the kids} in \textit{Most of the kids in my school like to play music} is translated using a majority \is{nouns}noun in these languages, as can be seen in \REF{ex:coppockstrand:75}:

\ea  \label{ex:coppockstrand:75}
\settowidth\jamwidth{(\ili{Portuguese})}
\ea \textit{\textbf{La} \textbf{mayoría} \textbf{de} \textbf{los} \textbf{niños}...} \jambox{(\ili{Spanish})}
\ex \textit{\textbf{A} \textbf{maioria} \textbf{das} \textbf{crianças}...} \jambox{(\ili{Portuguese})}
\ex \textit{\textbf{La majoria dels nens}...} \jambox{(\ili{Catalan})}
\sn `\textbf{Most of the kids}...'
\z
\z

However, according to \citet[20]{Dobrovie-SorinGiurgea2015}, ``\ili{Italian} allows the article and a proportional meaning in the \is{partitives}\textit{partitive} construction'':

\ea \label{ex:coppockstrand:76}
\gll \textbf{Il} \textbf{pi\`u} \textbf{degli} \textbf{uomini} predicano ciascuno la sua benignit\`a.\\
the more of.\defn{} men preach each the his kindness\\ \jambox{(\ili{Italian})}
\glt `Most men preach their own kindness.'
\z

\citet[21]{Dobrovie-SorinGiurgea2015} also write that this is possible with no overt \is{partitives}partitive complement.

\ea \label{ex:coppockstrand:77}
\gll Gli ospiti sono partiti. \textbf{I} \textbf{pi\`u} erano gi\`a stanchi.\\
\defn{} guests have left \defn{} \cmpr{} were already tired\\ \jambox{(\ili{Italian})}
\glt `The guests left. \textbf{Most (of them)} were already tired.'
\z

This shows that to the extent that proportional readings for \isi{quantity superlatives} are allowed in \ili{Italian}, they are signalled with the definite article. In this respect, \ili{Italian} is like \ili{Swedish}: definite for proportional and non-definite for \is{relative readings of superlatives}relative. But this construction appears more restricted than \ili{Swedish} \textit{de flesta} `most', given that it can only occur with partitive complements. Our \ili{Spanish} and \ili{French} informants do not accept the \defcomp{} construction in the same environment, so this appears to be specific to \ili{Italian} among the \il{Romance!Ibero-Romance}Ibero-Romance languages.

To summarize: \ili{Italian} and other \il{Romance!Ibero-Romance}Ibero-Romance languages use definiteness\hyp{}marking for \is{adnominal superlatives}adnominal \isi{quality superlatives}, and ordinary predicative quality superlatives, but not quantity superlatives, \isi{adverbial superlatives}, or predicative quality superlatives embedded in phrases uniquely characterizing a given \isi{discourse}discourse referent. Proportional readings are generally not available for \isi{quantity superlatives}, with the exception of \textit{il pi\`u} in \ili{Italian} accompanied by a partitive complement.

\section{Summary}
\label{sec:coppockstrand:5}

\tabref{tab:coppockstrand:4} gives a summary of the definiteness-marking patterns we have observed. For a set of languages in which superlatives are formed with the help of a definite article, there is a remarkable diversity of definiteness-marking patterns on superlatives.\is{definite articles|)}

\begin{table}[h]
 \caption{Definiteness-marking in superlatives in \defcomp{} languages}
 \label{tab:coppockstrand:4}
 \begin{tabularx}{\textwidth}{lccccccc}
 \lsptoprule
  & \ili{Greek} & \ili{Romanian} & \ili{French} & \ili{Italian} & \ili{Spanish} \\\midrule
 Qual./pred. & $+$ & $+$ & $+$ & $+$ & $+$ \\
 Qual./pred. (rel. clause) & $+$ & $+$ & $+$ & $-$ & $-$ \\
 Qual./prenom. & $+$ & $+$ & $+$ & $+$ & $+$\\
 Qual./postnom. & $+$ & $+$ & $+$ & $-$& $-$  \\
 Qual./adv. & $-$ & $+$ & $+$ & $-$& $-$ \\
 Quant./prop. & $+$ & $+$ & NA & $+$  & NA\\
 Quant./rel. & $+$ & $+$ & $+$ & $-$ & $-$ \\
  Quant./adv. & $+$ & $+$ & $+$ & $-$ & $-$ \\
 \lspbottomrule
 \end{tabularx}
\end{table}


The contrasts raise a number of questions, including:
\begin{itemize}
\item Why do \isi{quantity superlatives} in \il{Romance!Ibero-Romance}Ibero-Romance lack definiteness-marking, in contrast to \ili{Greek}, \ili{Romanian}, and \ili{French}?
\item Why are \isi{adverbial superlatives} marked definite in \ili{French} and \ili{Romanian}, but not \ili{Italian}, and why is there a split among adverbial superlatives in \ili{Greek}?
\item Why is definiteness-marking absent on predicative superlatives in \is{relative clauses}relative clauses in \ili{Italian}, but not in \ili{French}?
\item Why do \ili{Greek} and \ili{Romanian} allow \is{proportional readings of superlatives|)}proportional readings for \defcomp{} but not \ili{Spanish} or \ili{French}, and why is it limited to \is{partitives}partitive environments in \ili{Italian}?
\end{itemize}

We cannot address all of these issues adequately here. However, we will suggest a certain perspective that may bring some of this apparent chaos to order.

The perspective is as follows.  The variety of different \is{definiteness marking|)}definiteness-marking patterns we see suggests that the grammars of these languages may be pulled by a number of competing pressures. One pressure is to mark \isi{uniqueness} of a description overtly. Another pressure, we suggest, is to avoid combining a \is{definite determiners}definite determiner with a predicate of entities other than individuals, such as events or \is{degrees|(}degrees. In conjunction with certain additional assumptions regarding the semantics of various types of superlatives, these pressures result in a dispreference for certain patterns. These assumptions are made explicit in the following section.

\section{Formal analyses} \label{sec:coppockstrand:6}

\subsection{Quality superlatives}\is{quality superlatives}

\subsubsection{Prenominal quality superlatives}


To derive a superlative meaning for \defcomp{} constructions, let us start with the assumption that the basic meaning for a comparative like \ili{Greek} \textit{pio} is a function from measure functions to degrees to individuals to truth values, roughly following \citet{Kennedy2009}, \citet{AlrengaEtAlii2012}, and \citet{DunbarWellwood2016}, among others.\footnote{This presentation glosses over the fact that not all comparatives are alike. An illustration of this point of particular relevance to the case at hand are the detailed studies of comparison in \ili{Greek} by \citet{Merchant2009,Merchant2012}, where there are three \is{morphosyntax}morphosyntactic strategies for marking the standard: (i) the preposition \textit{apo} `from' introducing a phrasal standard; (ii) a genitive case marker, also introducing a phrasal standard; and (iii) a complex standard marker \textit{ap-oti} `from-wh' which introduces both reduced and unreduced clausal standards. \citet{Merchant2012} concludes that if all of the work is to be done by the comparative, then three different lexical entries for the comparative are needed. But there is hope for a unified analysis; the two phrasal comparatives differ only in the order in which they take their arguments, and \citet{Kennedy2009} shows that one of the phrasal meanings can be derived from the clausal meaning. Moreover, \citet{AlrengaEtAlii2012} offer a new perspective on the division of labor between the comparative and the standard marker, allowing for a unified view on the comparative morpheme across these constructions, with differences attributed to the standard markers. They use a lexical entry like \REF{ex:coppockstrand:78} for the comparative, and clausal and phrasal standard markers each combine with it appropriately in their own way. In light of this work, we may continue to operate under the assumption that \REF{ex:coppockstrand:78} constitutes a viable candidate for a unified treatment of the comparative morpheme across different types of constructions and across the languages under consideration.}

\ea \label{ex:coppockstrand:78}\il{Greek}
\textit{pio} $\leadsto \lambda g \lambda d \lambda x \ldot g(x) > d$
\z 

In \REF{ex:coppockstrand:78}, $g$ denotes a measure function, a function that maps individuals to degrees. A gradable \is{adjectives}adjective like \textit{long} is assumed to denote such a function.\footnote{The arrow $\leadsto$ signifies a translation relation from a natural language expression (part of an \is{logical form}LF representation) to an expression of a typed extensional language; we thus adopt an ``indirect interpretation'' framework, in which expressions of natural language are translated to a formal representation language. Within this framework we assume the standard rule of \isi{Functional Application}:

\begin{exe}
	\ex \textbf{Functional Application (Composition Rule)}\\
	If $\alpha \leadsto \alpha'$  and $\beta \leadsto \beta'$, and $\alpha'$ is of type $\<\sigma,\tau\>$ and $\beta'$ is of type $\sigma$, and $\gamma$ is a phrase whose only constituents are $\alpha$ and $\beta$, then $\gamma \leadsto \alpha'(\beta')$.
\end{exe}\vspace*{-\baselineskip}
}
Modulo\largerpage lambda-conversion, this yields the translation in \REF{ex:coppockstrand:79} for \textit{pio grigoro} `faster':

\ea \label{ex:coppockstrand:79} \textit{pio grigoro}  $\leadsto \lambda d \lambda x \ldot \const{fast}(x) > d$\label{longer}
\z 

The next ingredient is a meaning shift that we refer to as \isi{Definite Null Instantiation}, in homage to \citet{Fillmore1986}, as defined in \REF{ex:coppockstrand:80}. It takes any function and saturates its argument with an unbound variable.\footnote{Note that this meaning shift depends on the assumption that the $\leadsto$ relation is not a function; a given natural language expression can have multiple translations into the formal language and they need not be equivalent. See \citet{ParteeRooth1983} for precedent for this assumption.}

\ea \label{ex:coppockstrand:80}\is{Definite Null Instantiation}
\textbf{Definite Null Instantiation (Meaning Shift)}\\
If $\alpha \leadsto \alpha'$, and $\alpha'$ is an expression of type $\<\sigma,\tau\>$, then $\alpha \leadsto \alpha'(v)$ as well, where $v$ is an otherwise unused variable of type $\sigma$.
\z 

Applying this gives \REF{ex:coppockstrand:81}, where $\textbf{d}$ is an unbound degree-type variable:

\ea \label{ex:coppockstrand:81}\il{Greek}
\textit{pio grigoro}  (after DNI) $\leadsto \lambda x \ldot \const{fast}(x) > \textbf{d}$ 
\z 

We have written $\textbf{d}$ in bold-face in order to draw attention to the fact that it is unbound. (We could of course have chosen a variable other than $\textbf{d}$; all we needed was a degree variable that is not otherwise used.)  This description can combine with a \is{nouns}noun like \textit{aftokinito} `car' using \isi{Predicate Modification} to produce \REF{ex:coppockstrand:82}:

\ea \label{ex:coppockstrand:82}\il{Greek}
[\textit{pio grigoro}] \textit{aftokinito} $\leadsto \lambda x \ldot \const{fast}(x) > \textbf{d} \land \const{car}(x)$ 
\z 

If there is a \is{uniqueness}unique fastest car, then there will be a way of choosing a value for $\textbf{d}$ in such a way that this description picks it out. Hence, given an appropriate choice of value $\textbf{d}$, the \is{definite articles}definite article should be able to combine with this description to pick out the most qualified candidate. Normally, the range of potential referents will be limited to a class $\textbf{C}$, which we may suppose is referenced by the \is{definite determiners}definite determiner, as displayed in \REF{ex:coppockstrand:83}.


\ea \label{ex:coppockstrand:83}\il{Greek}
\textit{to} $\leadsto \lambda P_{\<\tau,t\>} \ldot \iota x_\tau \ldot P(x) \land \textbf{C}(x)$
\z 

Where $\tau$ is a variable over types, constrained in specific ways by different languages.
Applied to \textit{pio grigoro aftokinito}, this denotes the unique car in $\textbf{C}$ that is faster than $\textbf{d}$. The structure of the derivation is the one in \REF{ex:coppockstrand:84}.

\ea\il{Greek}  \label{ex:coppockstrand:84}
\begin{forest}
	[$e$
		[{$\<\<\tau,t\>,\tau\>$}[\textit{to}]]
		[{$\<e,t\>$\\(by Predicate Modification)}
			[{$\<e,t\>$}
				[{$\<\<\tau,d\>,\<\tau,t\>\>$\\$\Uparrow_{\textsc{dni}}$\\$\<d,\<\<\tau,d\>,\<\tau,t\>\>$} [\textit{pio}]]
				[{$\<e,d\>$} [\textit{grigoro}, roof]]
			]
			[{$\<e,t\>$}[\textit{aftokinito}]]
		]
	]
\end{forest}
\z 


This clearly gives an \is{absolute readings of superlatives}absolute superlative reading. What about \is{relative readings of superlatives|(}relative readings such as \REF{ex:coppockstrand:8}, with \textit{ti leptoteri mesi}  `the thinnest waist'? The analytical landscape is quite different under the assumption that there is no superlative morpheme. One influential analysis of the absolute vs.\ relative distinction, due to \citet{Szabolcsi1986} and developed in \citet{Heim1999}, holds that relative readings arise through movement of \textit{-est} at \is{logical form}LF to a position adjacent to the constituent of the sentence corresponding to one of the elements being compared, typically the \isi{focus}. With no \textit{-est} to undergo movement, this analytical route is not available to us. 

A prominent class of alternatives to the movement view is that \textit{-est} remains \textit{in situ}, the absolute vs.\ relative contrast resulting from different settings of the \is{comparison classes}comparison class \citep{Gawron1995,FarkasKiss2000,SharvitStateva2002,Gutierrez-Rexach2006,Teodorescu2009,PanchevaTomaszewicz2012,CoppockBeaver2014,CoppockJosefson2015}.  This type of approach is more amenable to the assumptions that we have made here. Although we have no superlative morpheme to provide a comparison class, the \is{definite articles}definite article is restricted to a contextually-determined domain $\textbf{C}$, and the contrast could concern the value of that contextually-set variable. On an relative reading of \textit{the fastest car}, for example,  $\textbf{C}$ might consist of cars standing in a salient correspondence relation to the focus alternatives.

\citet{Heim1999} notes that so-called ``upstairs \textit{de dicto}'' readings pose a challenge for the \textit{in situ} approach. The problem is that \textit{John wants to climb the highest mountain} can be true in a context where there is no specific mountain that John wants to climb, nor does John's desire pertain to the relative heights of mountains climbed by various competitors; it just so happens that he wants to climb a 5000 mountain (any such mountain), and the ambitions of the others in the context with respect to the heights of mountains they want to climb are not so great. This reading can be obtained by scoping just \textit{-est} over the intensional verb \textit{want}. Such a reading is apparently available in at least \ili{Greek} and \ili{French}, according to our informants.

Various responses to that challenge have been offered. \citet{SharvitStateva2002} offer an \textit{in situ} theory designed to handle these readings, but it relies on a non-standard \is{definite determiners}definite determiner, so that solution is not directly compatible with our analysis.  \citet{Solomon2011} points out that upstairs \textit{de dicto} readings can be handled if the comparison class is thought to be a set of degrees rather than individuals. This is more amenable to the assumptions we have made, and would only require us to allow for the possibility that the \is{definite articles}definite article combine directly with a $\textbf{d}$-saturated version of \cmpr{} that compares degrees rather than individuals and serve to pick out a specific \is{degrees|)}degree.

Other routes may be compatible with the analysis as it stands. \citet{CoppockBeaver2014} argue that the ``upstairs \textit{de dicto}'' phenomenon is part of a more general phenomenon that requires an explanation anyway, namely cases like \textit{Adrian wants to buy a jacket like Malte's}, discussed by \citet{Fodor1970} and in much subsequent literature under the heading of ``Fodor's puzzle''. If indeed upstairs \textit{de dicto} readings can be seen as an instance of Fodor's puzzle, then the problem can be explained away. Another alternative is offered by \citet{Bumford2016}, who posits a sort of definiteness that is subordinated to the modal element. Although Bumford's theory of the definite article is different from the simple one we have sketched here, his suggested approach for dealing with \isi{intensional contexts} may be viable even in the context of a more standard analysis. In any case, we believe it is an open question whether upstairs \textit{de dicto} readings can indeed be managed in the context of an \textit{in situ} approach using the sort of approach to the definite article that we have taken here, and the success of our analysis in dealing with them depends on a general solution to this problem.

Another fact to be accounted for is the fact that, as \citet{Szabolcsi1986} pointed out, superlatives on relative readings behave like \isi{indefinites}, suggesting that they are, in \citegen{CoppockBeaver2015} terms, \textit{indeterminate}. We refer to \citet{CoppockBeaver2014} for ideas on how to capture the indeterminacy of \is{relative readings of superlatives|)}relative readings in the context of an \textit{in situ} analysis.

Another question that this proposal raises is how to rule out overt standard phrases with comparatives that combine with definite articles. These are entirely\largerpage ungrammatical:

\ea[*]{ \label{ex:coppockstrand:85}
\gll Elle est la plus belle que \textnormal{\{}Marie, j'ai imagin\'e\textnormal{\}}.\\ 
she is the \cmpr{} beautiful than \{Marie, I've imagined\}\\ \jambox{(\ili{French})}
}
\z

The same is true for definite comparatives in \ili{English}, as \citet{LernerPinkal1995} observe:

\ea[]{ \label{ex:coppockstrand:86}
\textit{George owns the faster car {\op}\textnormal{*}than Bill{\cp}.}}
\z 

\citet{LernerPinkal1995} also observe that this is part of a larger pattern, where \is{determiners}weak determiners allow overt standard arguments and strong determiners disallow them:

\ea[]{ \label{ex:coppockstrand:87}
\textit{George owns a/some/a few faster car{\op}s{\cp} than Bill.}}
\z 

\ea[*]{\label{ex:coppockstrand:88}
\textit{George owns every/most faster car{\op}s{\cp} than Bill.}}
\z 

\citet{Beil1997} offers an explanation of this contrast on the basis of the fact that \is{determiners}strong determiners have a domain that has to be presupposed in previous context. \citet{Xiang2005} offers an alternative explanation, on which strong \isi{quantifiers} induce an \is{logical form}LF intervention effect blocking the movement that the \textit{than} phrase needs to undergo. This idea is quite compatible with the present analysis. In a case where \isi{Definite Null Instantiation} has applied, the target of comparison does not need to undergo movement, so no intervention effect is predicted to arise.

\subsubsection{Postnominal quality superlatives}\is{quality superlatives}

In all of the languages we have seen, there are constructions in which the superlative occurs post-nominally; (\ref{ex:coppockstrand:89}--\ref{ex:coppockstrand:92}) are some examples repeated from the discussions above.

\ea \label{ex:coppockstrand:89}
\gll Spania haidevo \textbf{tin} \textbf{mikroteri} \textbf{ti} \textbf{gata}.  \\
seldom pet \defn{} smallest \defn{} cat\\ \jambox{(\ili{Greek})}
\glt `I seldom pet \textbf{the smallest cat}.'
\z

\ea \label{ex:coppockstrand:90}
\gll A scris \textbf{compunere-a} \textbf{cea} \textbf{mai} \textbf{frumoasă}.\\
has written composition-\defn{} \defn{} \cmpr{} beautiful\\ \jambox{(\ili{Romanian})}
\glt `She wrote \textbf{the most beautiful composition}.'
\z

\ea \label{ex:coppockstrand:91}
\gll celui de la famille avec \textbf{la} \textbf{taille} \textbf{la} \textbf{plus} \textbf{fine} \\ 
the.one of the family with the waist \defn{} \cmpr{} {fine}\\ \jambox{(\ili{French})}
\glt `the one in the family with the thinnest waist.'
\z

\ea \label{ex:coppockstrand:92}
\gll La mamma fa \textbf{i} \textbf{biscotti} \textbf{più} \textbf{buoni} del mondo.\\
\defn{} mom makes \defn{} cookies \cmpr{} tasty of.\defn{} world\\ \jambox{(\ili{Italian})}
\glt `Mom bakes \textbf{the yummiest cookies} in the whole world.' 
\z

In \ili{Greek}, \ili{Romanian} and \ili{French}, the postnominal superlative is accompanied by a second \is{definiteness marking|(}definiteness-marker (this is specific to superlatives only in \ili{Romanian} and \ili{French}). For such cases, it is convenient to adopt \citegen{CoppockBeaver2015} predicative treatment of the \is{definite articles}definite article, whereby it denotes a function from predicates to predicates, presupposing \isi{uniqueness} but not existence. It is also important for our purposes to restrict the domain of a \is{definite determiners}definite determiner to a salient \is{comparison classes}comparison class $\textbf{C}$. Thus we adopt the lexical entry shown in \REF{ex:coppockstrand:93} for \ili{Romanian} \textit{cel}, for example.

\ea \label{ex:coppockstrand:93}\il{Romanian}
\textit{cel}$_{\textbf{C}}$ $\leadsto \lambda P \lambda x \ldot \partial(|P\cap \textbf{C}|\leq 1) \land P(x) \land \textbf{C}(x)$
\z 

(Here $\partial$ is the `partial' operator, whose scope is presupposed material. It evaluates to the `undefined' truth value unless its scope is true.) With this, we derive the interpretation in \REF{ex:coppockstrand:94} for the superlative phrase in \REF{ex:coppockstrand:90}:

\ea \label{ex:coppockstrand:94}\il{Romanian}
{\em cel$_{\textbf{\textnormal{C}}}$ mai frumoasă} $\leadsto \lambda x \ldot \partial(|\lambda x'\ldot \const{beautiful}(x') > \textbf{d} \land \textbf{C}(x)|\leq 1) \land \const{beautiful}(x) > \textbf{d} \land \textbf{C}(x)$ 
\z 

This description characterizes a composition $x$ in $\textbf{C}$ that is the only one whose beauty exceeds $\textbf{d}$. Combining this phrase with the \is{definite articles}definite article on the \is{nouns}noun yields a derivation of the following form for the the full \is{noun phrases}noun phrase (we assume that the suffix \textit{-a} in \textit{compunere-a} `the composition' is interpreted in D, and we represent it in \ref{ex:coppockstrand:95} as an \is{iota operator}iota operator for simplicity, although it can also be given a treatment along the lines of \ref{ex:coppockstrand:93}):\largerpage[2]

\ea\il{Romanian}\label{ex:coppockstrand:95}
\begin{forest}
	[{$e$}
		[{$\<\<\tau,t,\>,\tau\>$}[\textit{-a}]
		]
		[{$\<e,t\>$}
			[{$\<e,t\>$}[\textit{compunere}]
			]
			[{$\<e,t\>$} [\textit{cel}$_{\textbf{C}}$ \textit{mai frumoasă}, roof]
			]
		]
	]
\end{forest}\z 

\subsection{Quantity superlatives}\is{quantity superlatives}

The picture is much richer when it comes to quantity superlatives. In all of the languages we have considered, quantity superlatives differ at least to some extent from \isi{quality superlatives}, if not with respect \is{definiteness marking|)}definiteness-marking (as in \ili{Italian}) then with respect to \is{definiteness spreading}definiteness-spreading in object position (\ili{Greek}), use of a \is{pseudo-partitives}pseudopartitive construction (\ili{French}), or pre- vs. postnominal word order (\ili{Romanian}). We therefore posit that quantity superlatives are of a different semantic type from quality superlatives (across the board), namely: predicates of \is{degrees|(}degrees, rather than individuals. We have adopted a measure function approach to the semantics of gradable predicates, so that an \is{adjectives}adjective like \textit{tall} for example is translated as an expression of type $\<e,d\>$, mapping an individual to a degree. The parallel treatment for a \is{quantity words}quantity word like \textit{much} or \textit{many} would then be $\<d,d\>$; just as \textit{tall} maps an individual to its height, \textit{much} maps a quantity to its magnitude. The magnitude of a quantity might as well be seen as the quantity itself, so we will simply treat quantity words as identity functions on degrees. Thus for \ili{Greek}, we have \REF{ex:coppockstrand:96} and \REF{ex:coppockstrand:97}: 

\ea \label{ex:coppockstrand:96} \il{Greek}
\textit{pollá} $\leadsto \lambda d \ldot d$ 
\z 

\ea \label{ex:coppockstrand:97}\il{Greek}
\textit{pio pollá} (after DNI) $\leadsto \lambda d' \ldot d' > \textbf{d}$ 
\z 

Now, we cannot use \isi{Predicate Modification} to combine with the \is{nouns}noun (and this predicts that \isi{definiteness spreading} should be problematic.) Let us assume that what happens instead is that the degree predicate is linked to the nominal predicate by the same glue that holds a \is{pseudo-partitives}pseudopartitive together. We implement this with the composition rule called \is{Measure Identification|(}Measure Identification in \REF{ex:coppockstrand:98}. The result is a predicate that holds of some individual $x$ if the nominal predicate holds of $x$ and $x$ has an extensive measure satisfying the degree predicate. 

\ea \label{ex:coppockstrand:98}
\textbf{Measure Identification (Composition Rule)}\\
If $\gamma$ is a subtree whose only two immediate subtrees are $\alpha$ and $\beta$, and $\alpha\leadsto D$, where $D$ is of type $\<d,t\>$, and $\beta\leadsto P$, where $P$ is of type $\<\tau,t\>$, where $\tau$ is any type, then $$\gamma\leadsto \lambda v \ldot D(\mu_i(v)) \land P(v)$$ where $v$ is a variable of type $\tau$ and $\mu_i$ is a free variable over measure functions (type $\<\tau,d\>$).
\z 

We use $\mu_i$ to denote a contextually-salient measure function along the lines of \citet{Wellwood2014}, with $i$ as a free variable index presumed to be constrained by context. So given a predicate of degrees $D$ and a predicate of individuals $P$, this operation yields $\lambda x \ldot D(\mu_i(x)) \land P(x)$. 
\REF{ex:coppockstrand:99} is an example (assuming the plural is translated using the cumulativity operator $\cum$; cf.\ \citealt{Link1983}):

\ea \label{ex:coppockstrand:99}\il{Greek}
\textit{pio pollá órgana} $\leadsto \lambda x \ldot \mu_i(x) > \textbf{d} \land \cum\const{instrument}(x)$ 
\z 

This is the right sort of thing to combine with a \is{definite articles}definite article as long as $\textbf{d}$ is chosen appropriately. The definite article introduces a \is{comparison classes}comparison class $\textbf{C}$. So \textit{ta pio pollá órgana} will be predicted to denote the plurality of instruments in $\textbf{C}$ whose contextually-relevant extensive measure is $\textbf{d}$. The structure of the derivation is thus as in \REF{ex:coppockstrand:100}:

\ea \label{ex:coppockstrand:100}\il{Greek}
\begin{forest}
	[{$e$}
		[{$\<\<\tau,t\>,\tau\>$} [\textit{i}]
		]
		[{$\<e,t\>$\\(by Measure Identification)}
			[{$\<d,t\>$}
				[{$\<\<\tau,d\>,\<\tau,t\>\>$\\$\Uparrow_{\textsc{dni}}$\\$\<d,\<\<\tau,d\>,\<\tau,t\>\>$} [\textit{pio}]
				]
				[{$\<d,d\>$} [\textit{poll\'a}, roof]
				]
			]		
			[{$\<e,t\>$} [\textit{\'organa}, roof]
			]
		]
	]
\end{forest}
\z 

In \ili{Romanian}, the definite element \textit{cel} forms a constituent with the comparative element and the \is{quantity words}quantity word to the exclusion of the \is{nouns}noun. We therefore posit the structure in \REF{ex:coppockstrand:101} for the semantic derivation:\is{Measure Identification|)}

\ea \label{ex:coppockstrand:101}\il{Romanian}
\begin{forest}
	[{$\<e,t\>$\\(by Measure Identification)}
		[{$\<d,t\>$}
			[{$\<\<\tau,t\>,\<\tau,t\>\>$} [\textit{cele}]
			]
			[{$\<d,t\>$}
				[{$\<\<\tau,d\>,\<\tau,t\>\>$\\$\Uparrow_{\textsc{dni}}$\\$\<d,\<\<\tau,d\>,\<\tau,t\>\>\>$} [\textit{mai}]
				]
				[{$\<d,d\>$\\MP} [\textit{multe}]
				]
			]
		]
		[{$\<e,t\>$} [\textit{instrumente}, roof]
		]
	]
\end{forest}
\z 

The meaning for this expression as a whole characterizes a plurality of instruments whose measure is greatest among any of the degrees in the context. In the case of a \is{relative readings of superlatives}relative reading, the set of degrees that are salient in the context are aligned in a one-to-one relationship with some salient set of individuals, typically those individuals that are alternatives to the \is{focus}focused constituent.

\ili{French} has yet a different structure, involving a \is{pseudo-partitives}pseudopartitive, as illustrated in \REF{ex:coppockstrand:102}. 

\ea \label{ex:coppockstrand:102}
\gll Je suis celui qui joue \textbf{le} \textbf{plus} \textbf{d'instruments}. \\
I am the-one who plays \defn{} \cmpr{} of-instruments\\ \jambox{(\ili{French})}
\glt `I am the one who plays \textbf{the most instruments}.'
\z

Since \ili{French} does not use a word for \textit{many} parallel to \ili{Greek} \textit{poll\'a} or \ili{Romanian} \textit{mult}, we might posit either a silent underlying form with the same meaning, or we might imagine that \ili{French} simply makes do without such an element. In the latter case, it is convenient to treat \textit{plus} using the simplest imaginable lexical entry for comparison \citep{Heim2006,Beck2010}, namely \REF{ex:coppockstrand:103}:

\ea \label{ex:coppockstrand:103}\il{French}
\textit{plus} $\leadsto \lambda d \ldot \lambda d' \ldot d' > d$
\z 

Given this, we have the derivation in \REF{ex:coppockstrand:104}:

\ea \label{ex:coppockstrand:104} \il{French}
\begin{forest}
	[{$\<e,t\>$}
		[{$d$}
			[{$\<\<\tau,t\>,\tau\>$} [\textit{le}]
			]
			[{$\<d,t\>$\\$\Uparrow_{\textsc{dni}}$\\$\<d,\<d,t\>\>$} [\textit{plus}, roof]
			]
		]
		[{$\<d,\<e,t\>\>$}
			[{$\<\<e,t\>,\<d,\<e,t\>\>$} [\textit{de}]
			]
			[{$\<e,t\>$} [\textit{instruments}, roof]
			]
		]
	]
\end{forest}
\z 

We assume that the Meas head acts as glue, linking the degree denoted by \textit{le plus} with the denotation of the \is{noun phrases}noun phrase such that the noun phrase is constrained to have an extensive measure of that \is{degrees|)}degree. The resulting denotation is just the same as that posited for \ili{Romanian}.

Finally, we come to \ili{Italian}, which has the simplest overt form, as shown in \REF{ex:coppockstrand:66} above, repeated here as \REF{ex:coppockstrand:105}:

\ea \label{ex:coppockstrand:105}
\gll ... che suona \textbf{più} \textbf{strumenti}.\\
... that plays \cmpr{} instruments\\ \jambox{(\ili{Italian})}
\glt `...  who plays the most instruments.'
\z 

One possible analysis is the one in \REF{ex:coppockstrand:106}, using a lexical entry for \textit{pi\`u} like the one given for \ili{French} \textit{plus} above.

\ea \label{ex:coppockstrand:106}\il{Italian}
\begin{forest}
	[{$\<e,t\>$\\(by Measure Identification)}
		[{$\<d,t\>$\\$\Uparrow_{\textsc{dni}}$\\$\<d,dt\>$} [\textit{pi\`u}, roof]
		]
		[{$\<e,t\>$} [\textit{strumenti}, roof]
		]
	]
\end{forest}
\z 

\is{Measure Identification}The predicate that this derives holds of any plurality of instruments $x$ whose quantity exceeds $\textbf{d}$. This of course does not necessitate that there be no larger plurality of instruments in the context, so we have not captured a superlative interpretation. Assuming the same analysis carries over to \ili{Spanish}, it remains an open question why superlatives undergo fronting and comparatives do not.

\subsection{Adverbial superlatives}\is{adverbial superlatives}

For adverbial \isi{quantity superlatives}, we start with the assumption that a verb phrase denotes a property of events, translating to an expression of type $\<v,t\>$, and that the \defcomp{} construction combines with it via Measure Identification. For example, in \ili{Greek} we have \REF{ex:coppockstrand:107}:

\ea \label{ex:coppockstrand:107} \il{Greek}
\begin{forest}
	[{$\<v,t\>$\\(by Measure Identification)}
		[{$\<v,t\>$} [VP]
		]
		[{$\<d,t\>$}
			[{$\<\<\tau,t\>,\<\tau,t\>\>$} [\textit{i}]
			]
			[{$\<d,t\>$}
				[{$\<\<\tau,d\>,\<\tau,t\>\>$\\$\Uparrow_{\textsc{dni}}$\\$\<d,\<\<\tau,d\>,\<\tau,t\>\>\>$} [\textit{pio}]
				]
				[{$\<d,d\>$} [\textit{polla}]
				]
			]
		]
	]
\end{forest}
\z 

\is{adverbial superlatives}Adverbial \isi{quality superlatives}, on the other hand, involve gradable predicates that measure events as in \REF{ex:coppockstrand:108}: \is{Predicate Modification}

\ea \label{ex:coppockstrand:108}\il{Greek}
\begin{forest}
	[{$\<v,t\>$\\(by Predicate Modification)}
		[{$\<v,t\>$} [VP]
		]
		[{$\<v,t\>$}
			[{$\<\<\tau,d\>,\<\tau,t\>\>$\\$\Uparrow_{\textsc{dni}}$\\$\<d,\<\<\tau,d\>,\<\tau,t\>\>\>$} [\textit{pio}]
			]
			[{$\<v,d\>$} [\textit{grigora}]
			]
		]
	]
\end{forest}
\z 

We suggest that this difference in type underlies the contrast between \is{quantity superlatives}quantity and \is{quality superlatives}quality \isi{adverbial superlatives} in \ili{Greek}: the \ili{Greek} \is{definite determiners}definite determiner applies to predicates of type $\<d,t\>$ but not ones of type $\<v,t\>$.  In \ili{Italian}, neither type of adverbial superlative is \is{definiteness marking}marked definite; this can be understood as an aversion to definiteness-marking on predicates of both types. In \ili{French} and \ili{Romanian}, on the other hand, both types are definite, and this can be understood under the  lens of a maximally polymorphic definite determiner.

\subsection{Proportional readings}

Proportional readings for quantity superlatives are not fully available in \ili{French}, \ili{Spanish}, or \ili{Italian}, but they are available in \ili{Greek} and \ili{Romanian}. From a larger typological perspective, \ili{Greek} and \ili{Romanian} are the odd ones out; most languages lack proportional readings for the superlative of `many' \citep{CoppockEtAlii2017}. In line with \citet{CoppockEtAliiinprep}, 
 we suggest that this is related to our proposal that \isi{quantity words} typically denote predicates of \is{degrees|(}degrees rather than individuals, and their comparatives likewise compare degrees rather than individuals. A definite determiner that combines directly with the comparative of a quantity word after \isi{Definite Null Instantiation} produces a phrase denoting a degree or amount that is greatest among some contextually-salient set of degrees. Thus for example \textit{le plus} in \textit{le plus d'instruments} would a denotation like `the greatest number' or `the greatest amount'. Notice that the phrase \textit{the greatest number} only has a \is{relative readings of superlatives|(}relative reading. Consider \REF{ex:coppockstrand:109}:

\ea \label{ex:coppockstrand:109}
\textit{Maria has visited the greatest number of continents.}
\z 

This cannot mean that Maria has visited more than half of the continents. If \textit{le plus} means the same thing as \textit{the greatest number}, then it, too, should only have relative readings. According to \citet{CoppockEtAliiinprep}, the reason that such cases have only relative readings is related to a general constraint on the interpretation of superlatives. This view makes a distinction in principle between the entities that are actually measured by the gradable predicate to which superlative morphology attaches, the \textit{measured entities}, and what they call the \is{contrast sets}\textit{contrast set}, following \citet{CoppockBeaver2014}. On relative readings, the contrast set and the \isi{measured entities} are distinct and related by a salient association relation given by the sentence. On \is{absolute readings of superlatives}absolute readings, they are conflated. \citet{CoppockEtAliiinprep} posit a constraint on the contrast set, according to which it must consist of individuals. When the gradable predicate measures degrees rather than individuals, the contrast set must be distinct from the set of measured entities; hence a relative reading is forced.

How, then, do \is{proportional readings of superlatives}proportional readings arise? \citet{Dobrovie-SorinGiurgea2015} suggest that they arise through \isi{grammaticalization}, which requires full grammatical agreement (present in both \ili{Greek} and \ili{Romanian}), and is preempted by the \is{pseudo-partitives}pseudopartitive construction that \ili{French} uses with relative readings. On this perspective, it is a matter of historical accident whether a given language has developed a \is{determiners}proportional determiner from a \is{quantity superlatives}quantity superlative. We are sympathetic to this view. We would only note that if indeed \ili{Greek} and \ili{Romanian} involve different constituency relations when it comes to \is{relative readings of superlatives|)}relative readings, as suggested above, then the putative grammaticalization process must be of a different nature for the two languages. We would like to suggest that in \ili{Greek}, proportional readings arise through a process similar to the one envisioned by \citet{Hoeksema1983}, where the \is{quantity words}quantity word comes to denote a gradable predicate of (plural) individuals, and the \is{comparison classes}comparison class for the superlative is constituted by two non-overlapping pluralities, one consisting of atoms that satisfy the predicate in question and one consisting of atoms that do not. Such an analysis is consonant with the idea that the \is{definite determiners}definite determiner is in its ordinary position in \ili{Greek}, rather than more tightly integrated with the comparative marker. In \ili{Romanian}, on the other hand, there is a constituent containing the \is{definite articles}definite article, the comparative marker, and the quantity word; this phrase could potentially be reanalyzed as a complex determiner.


\section{Conclusion and outlook}

We have suggested that superlative interpretations arise in \defcomp{} languages with the help of an interpretive process called \isi{Definite Null Instantiation} for the target argument of a comparative. It is reasonable to ask whether this process is restricted to \defcomp{} languages or available more broadly. We suggest that it is available at least somewhat more broadly, and that \ili{English} is one of the languages that avails itself of it, in constructions like \textit{the taller of the two} (discussed from a formal semantic perspective by \citealt{Szabolcsi2012}). Why \ili{English} doesn't generally form superlatives using this strategy could be explained in terms of markedness; since there is a dedicated superlative morpheme in \ili{English}, it should be used whenever the \is{comparison classes}comparison class contains more than two members.

The pattern of variation suggests that a number of competing pressures are at play. One pressure is to \is{definiteness marking}mark \isi{uniqueness} of a description overtly. Another pressure is to avoid combining a \is{definite determiners}definite determiner with a predicate of entities other than individuals, such as events or degrees. We have assumed that quality adverbs denote gradable predicates of events, and that \isi{quantity words} denote predicates of degrees. The pressure to avoid combining definite determiners with predicates of events rules out definiteness-marking on \is{adverbial superlatives}adverbial \isi{quality superlatives}, and similarly for predicates of degrees and \isi{quantity superlatives}.

\is{Optimality Theory|(}\is{constraint ranking|(}In Optimality Theoretic terms, we might conceive of these forces as constraints that we could label *\textsc{def}/$d$ (``do not use a definite determiner with a predicate of degrees''), *\textsc{def}/$v$ (``do not use a definite determiner with a predicate of events'') and \textsc{mark-uniqueness}. \ili{Italian}  ranks the former two over the latter:
\begin{center}
*\textsc{def}/$d$,  *\textsc{def}/$v$  $>$ \textsc{mark-uniqueness}
\end{center}
while \ili{French} ranks the latter over the former two:
\begin{center}
\textsc{mark-uniqueness} $>$ *\textsc{def}/$d$,  *\textsc{def}/$v$ 
\end{center}
An \is{adverbial superlatives}adverbial superlative like \textit{le moins fort} (\ili{French}, lit.\ `the less fast') violates *\textsc{def}/$v$ but not \textsc{mark-uniqueness}, while one like \textit{m\'as r\'apido} (\ili{Spanish}, lit.\ `more fast') violates \textsc{mark-uniqueness} but not *\textsc{def}/$v$. \ili{Greek} draws the line at adverbial \isi{quality superlatives}, which suggests that it ranks \textsc{mark-uniqueness} over *\textsc{def}/$v$, but not over *\textsc{def}/$d$:
\begin{center}
*\textsc{def}/$d$ $>$ \textsc{mark-uniqueness} $>$  *\textsc{def}/$v$
\end{center}
Intuitively, \textsc{mark-uniqueness} should require that any descriptive phrase which is presupposed to apply to at most one individual is marked with a lexical item that conventionally signals this presupposition. But there may be slightly different shades of this constraint for different languages. Recall that in \ili{Italian} (and \ili{Spanish}), the definite article is normally used in predicative superlatives, presumably to distinguish between the comparative and the superlative interpretations. But the \is{relative clauses}relative clause construction serves to mark \isi{uniqueness} in some sense, rendering the \is{definite articles}definite article unnecessary. This sort of explanation could be made more precise by imagining a version of the \textsc{mark-uniqueness} constraint in Ibero-Romance that imposes slightly different requirements. Suppose that in \il{Romance!Ibero-Romance}Ibero-Romance, the operative \textsc{mark-uniqueness} constraint may be satisfied in some cases where a candidate phrase with \is{uniqueness}unique descriptive content is not actually marked as unique, as long as it is embedded in a larger phrase with unique descriptive content which \textit{is}. So Ibero-Romance might have a ``once per \is{discourse}discourse referent'' rule, while \ili{French} might have a ``once per phrase'' rule. Syntactic restrictions would presumably also come into play.\is{constraint ranking|)}\is{Optimality Theory|)}

This hypothesized difference could also apply to bare postnominal superlatives, which are found in \ili{Italian} but not \ili{French}. This idea would have to be evaluated in light of previous ideas regarding this contrast. According to \citet{Kayne2008}, the reason has to do with the licensing of \isi{bare nouns} in general. \citet[74--75]{Alexiadou2014} suggests an approach appealing to the richness of agreement features. \citet{Matushansky2008} argues that superlatives are always attributive modifiers of \isi{nouns}, so a nominal structure is projected around a superlative in the postnominal case; perhaps \ili{Italian} does not do that. We leave it to future research to compare among these possible explanations for the difference.

Future research on this topic should also bring into the discussion a wider range of languages that use this strategy.  For example, \citet{Plank2003} briefly discusses the very interesting case of \ili{Maltese}, which makes use of fronting to distinguish the superlative \is{degrees|)}degree \REF{ex:coppockstrand:110c} from the comparative \REF{ex:coppockstrand:110b}.

\ea \label{ex:coppockstrand:110}
\begin{xlist}
\ex \label{ex:coppockstrand:110a}
\gll  il-belt il-qawwi\\
\defn-city \defn-powerful\\ \jambox{(\ili{Maltese})}
\glt `the powerful city'
\ex \label{ex:coppockstrand:110b}
\gll  il-belt l-aqwa\\
\defn-city \defn-powerful.\cmpr{}\\
\glt `the more powerful city' 
\ex \label{ex:coppockstrand:110c}
\gll  l-aqwa belt\\
\defn{}-powerful.\cmpr{} city\\
\glt  `the most powerful city'.
\end{xlist}
\z

As \citet[361--362]{Plank2003} points out, ``Paradoxically, as a result of this fronting, NPs with superlatives thus end up less articulated than NPs with other \isi{adjectives} in normal postnominal position.'' Plank posits that ``Just like \textit{le plus jeune homme} [...] in \ili{French}, [superlatives in \ili{Maltese}] are in fact under-articulated: there ought to be two \is{definiteness marking}definiteness markers on the initial superlative, one by virtue of it being a superlative, another by virtue of it being NP-initial.'' Further issues for future work include whether and how the approach we have taken here, in terms of competing pressures, can be fruitfully applied to \ili{Maltese} and other \defcomp{} languages.

\section*{Acknowledgements}
We are very grateful to our consultants who have been so generous with their time, and to the organizers and participants of the Definiteness Across Languages conference in Mexico City, July 2016. Extra special thanks are due to Stavroula Alexandropoulou\ia{Alexandropoulou, Stavroula} for help with the \ili{Greek} judgments. This research was carried out under the auspices of the Swedish Research Council project 2015-01404 entitled \textit{Most and more: \is{quantity superlatives}Quantity superlatives across languages} awarded to PI Elizabeth Coppock at the University of Gothenburg.\is{superlatives|)}\is{comparatives|)}

\pagebreak\section*{Abbreviations}
\begin{multicols}{2}
\begin{tabbing}
	\cmpr{}\hspace{1em} \= comparative\\ \kill
	\sprl{} \> superlative\\
	\defn{} \> definite\\
	\wk{} \> weak ending
\end{tabbing}\end{multicols}

{\sloppy\printbibliography[heading=subbibliography,notkeyword=this]}
\end{document}
