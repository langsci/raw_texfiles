
%%%%%%%%%%%%%%%%%%%%%%%%%%%%%%%%%%%%%%%%%%%%%%%%%%%%
%%%                                              %%%
%%%                 Metadata                     %%%
%%%          fill in as appropriate              %%%
%%%                                              %%%
%%%%%%%%%%%%%%%%%%%%%%%%%%%%%%%%%%%%%%%%%%%%%%%%%%%%

\title{The Alor-Pantar languages}  %look no further, you can change those things right here.
\subtitle{History and typology\newlineCover {\large Second edition}}
\BackTitle{The Alor-Pantar languages}
\BackBody{The Alor-Pantar family constitutes the westernmost outlier group of Pa\-puan (Non-Austronesian) languages. Its twenty or so languages are spoken on the islands of Alor and Pantar, located just north of Timor, in eastern Indonesia. Together with the Papuan languages of Timor, they make up the Timor-Alor-Pantar family. The languages average 5,000 speakers and are under pressure from the local Malay variety as well as the national language, Indonesian.
This volume studies the internal and external linguistic history of this interesting group, and showcases some of its unique typological features, such as the preference to index the transitive patient-like argument on the verb but not the agent-like one; the extreme variety in morphological alignment patterns; the use of plural number words; the existence of quinary numeral systems; the elaborate spatial deictic systems involving an elevation component; and the great variation exhibited in their kinship systems.
Unlike many other Papuan languages, Alor-Pantar languages do not exhibit clause-chaining, do not have switch reference systems, never suffix subject indexes to verbs, do not mark gender, but do encode clusivity in their pronominal systems. Indeed, apart from a broadly similar head-final syntactic profile, there is little else that the Alor-Pantar languages share with Papuan languages spoken in other regions. While all of them show some traces of contact with Austronesian languages, in general, borrowing from Austronesian has not been intense, and contact with Malay and Indonesian is a relatively recent phenomenon in most of the Alor-Pantar region.

This is the second edition of the volume that was originally published in 2014. In this edition, typographical errors have been corrected, small textual improvements have been implemented, broken URL links repaired or removed, and references updated. The overall content of the chapters has not been changed.}
\dedication{This volume is dedicated to the memory of Mr. Anderias Malaikosa (1964--2011), whose love for the peoples and languages of his native Alor was an inspiration to linguists both within Alor and abroad.}
\typesetter{Sebastian Nordhoff; Timm Lichte}
\proofreader{Benedikt Singpiel}
\author{Marian Klamer}

\renewcommand{\lsISBNdigital}{978-3-944675-94-7}
\renewcommand{\lsISBNhardcover}{978-3-946234-67-8}
\renewcommand{\lsISBNsoftcover}{978-3-946234-91-3}
\renewcommand{\lsISBNsoftcoverus}{000-0-000000-00-0}
\renewcommand{\lsSeries}{sidl} % use lowercase acronym, e.g. sidl, eotms, tgdi
\renewcommand{\lsSeriesNumber}{3} %will be assigned when the book enters the proofreading stage
\renewcommand{\lsURL}{http://langsci-press.org/catalog/book/157} % contact the coordinator for the right number
\renewcommand{\lsBookDOI}{10.5281/zenodo.437098}

%<*coverdimen>
\setlength{\csspine}{25.0559784mm} % Please calculate: Total Page Number (excluding cover), usually (Total Page - 3) * 0.0572008 mm
\setlength{\bodspine}{20mm} % Please use BoD's algorithm: http://www.bod.de/hilfe/coverberechnung.html (German only, please contact LangSci staff for help)
%</coverdimen> 
		
\renewcommand{\lsImpressumCitationText}{Marian Klamer (ed.). 2017. \textit{The Alor-Pantar languages: History and typology.}
Second edition. (Studies in Diversity Linguistics 3). Berlin: Language Science Press.
}

\renewcommand{\lsYear}{2017}