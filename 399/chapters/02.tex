\chapter{The genius of the Slavic language according to Jan Herkel}
\label{ch:Van_Rooy}

{\Large Raf Van Rooy}\smallskip

\noindent KU Leuven\bigskip

\section{Introduction}

\begin{quote}
   Every verb is \is{Inflection!Verb inflection| see {Conjugation}}\is{Conjugation}inflected according to the aforementioned principles, but I am not unaware that some \is{Dialect}dialect grammarians will condemn these principles, but they alone will reach such a judgment of condemnation, since what one \is{Dialect}dialect approves, the other condemns, and vice versa. Nor indeed can reference grammarians sustain any other opinion. However, it remains an unshakable truth that in accordance with the \is{Genius!Genius of the Slavic language}genius of the \il{Slavic}Slavic language there is only one single form of verb \is{Conjugation}inflection. Let us see how six forms of \il{Russian}Russian can be reduced to a single form, a reduction which we will see confirmed by the usage in various other \is{Dialect}dialects.\footnote{\citet[146--147]{herkel_elementa_1826}: “Qualecunque verbum secundum praemissa \is{Conjugation}inflectitur principia; equidem non ignoro a nonnullis \is{Dialect}dialecticis isthaec principia damnatum iri; verum tale damnationis judicium nonnissi [\textit{sic}] dialectici ferent Grammatici, quod enim una \is{Dialect}dialectus approbat, id altera damnat, et vice versa, nec enim aliam Grammatici Referentes ferre possunt sententiam. Caeterum inconcussa veritas manet, e \is{Genius!Genius of the Slavic language}genio linguae \il{Slavic}Slavicae nonnisi unicam \is{Conjugation}inflectendorum verborum dari formam; videamus enim \il{Russian}Russorum 6 formas ad unam redactas, observabimus ipso usu variarum \is{Dialect}dialectorum hanc reductionem confirmari.” I always cite our \il{English}English translation in the main text. The present contribution is limited by my linguistic expertise: as a historiographer of linguistics with only a little \il{Old Church Slavonic}Old Church Slavonic, I analyze \ia{Herkel, Jan}Herkel’s conceptual toolkit revolving around the term \is{Genius}\textit{genius}, an important one in the history of linguistics. I thank \ia{Maxwell, Alexander}Alexander Maxwell for his insightful comments on earlier drafts of this chapter.}
\end{quote}

\noindent This remark on the \il{Slavic}Slavic verb reveals Jan \ia{Herkel, Jan}Herkel’s program of \is{Pan-Slavism!Linguistic Pan-Slavism}linguistic Pan-Slavism quite neatly, and is just one among many possible passages that could be cited to illustrate it. Time and again, \ia{Herkel, Jan}Herkel underscored the fundamental unity of all varieties of \il{Slavic}Slavic, which he typically referred to \is{Dialect}as “dialects”, a flexible but problematic term in itself which he did not care to define. His usage, however, suggests that he understood \is{Dialect}dialect as a language variety which diverged only superficially from the essential substance, or basic unity, of the underlying language. As such, his conceptualization of the \is{Dialect!Language-dialect distinction| see {Language/dialect dichotomy}}\is{Dialect!Language/dialect dichotomy}language-dialect distinction can be interpreted in terms of what I have elsewhere called “the Aristotelian criterion” \citep[109--124]{van_rooy_language_2020}. The \il{Slavic}Slavic \is{Dialect}dialects differed in their accidents, but their core was a unitary substance, which \ia{Herkel, Jan}Herkel did not cease to label “the \is{Genius!Genius of the Slavic language}genius of the \il{Slavic}Slavic language”. Remarkably enough, for \ia{Herkel, Jan}Herkel, this conceptualization did not have a strictly linear chronology: the various living \is{Dialect!Living dialect}dialects could help unravel this \is{Genius}genius, but this did not equal a kind of \il{Proto-Slavic}Proto-Slavic as one would expect from an account which has unmistakable historical-comparative traits. Such an approach would only become dominant later in the century, after historical-comparative linguistics had fully developed as an autonomous academic discipline thanks to the pioneering work of scholars like Rasmus Rask and Franz Bopp \citep{swiggers_indo-european_2017}. \ia{Herkel, Jan}Herkel, instead, was tracing an idealized form of \il{Slavic}Slavic, only existing on paper – a true \is{Pan-Slavism}\il{Slavic!Pan-Slavic}Pan-Slavic language that could serve as the common literary and written standard. His concern was, in other words, not with oral language, which remained too varied, but with written forms of language, especially in literature.

The search for a perfect language form, as Umberto \citet{eco_search_1995} has eloquently illustrated, was a long-standing endeavor in premodern reflections on the nature of language. The language spoken before the confusion of tongues at the Tower of Babel was often identified, or at least closely associated with perfection. In scientific writing, many scholars came to attribute this quality to \il{Latin}Latin, the supposed uniformity and sterility of which was considered a great asset (e.g. \cite{stroh_latein_2013}). \ia{Herkel, Jan}Herkel, by using a brand of nineteenth-century \il{Latin}Latin as the metalanguage for his \is{Pan-Slavism}Pan-Slavic program, inscribed himself in this tradition, and showed at the same time that he wanted his ideas to circulate among the intellectual élite of his days.\footnote{See my brief notes on \ia{Herkel, Jan}Herkel’s \il{Latin}Latin, discussed as part of our \hyperref[sec:Preface Latin discussion]{Preface}.} Even in the Middle Ages, however, the artificial nature of this scholarly and scientific \il{Latin}Latin, with its rigid grammar, had inspired certain authors to look for a more natural perfection elsewhere, in the vernacular languages, for instance. Indeed, \ia{Herkel, Jan}Herkel’s idealized Slavic recalls, in several ways, \ia{{Dante Alighieri}}Dante Alighieri’s \textit{volgare illustre}, a perfect yet fictional form of his native \il{Italian}Italian vernacular. Dante, whose imagination easily outshone \ia{Herkel, Jan}Herkel’s, powerfully pictured this \textit{volgare \il{Latin}illustre} as a tiger that is scented everywhere through the different levels of linguistic variation in \il{Italian}Italian but remains invisible:

\begin{quote}
    Now that we have hunted across the woodlands and pastures of all Italy without finding the panther we are trailing, let us, in the hope of tracking it down, carry out a more closely reasoned investigation, so that, by the assiduous practice of cunning, we can at last entice into our trap this creature whose scent is left everywhere but which is nowhere to be seen.\footnote{\ia{Dante Alighieri}Dante, \textit{De vulgari \il{Latin}eloquentia}, 1.16.1. Translation taken from the Princeton Dante Project <\url{https://dante.princeton.edu/cgi-bin/dante/DispMinorWork.pl?TITLE=V.E.\&REF=I\%20xvi\%201-6}> (last accessed 22 June 2023). The \il{Latin}Latin original reads: “Postquam venati saltus et pascua sumus Ytalie nec pantheram quam sequimur adinvenimus, ut ipsam reperire possimus, rationabilius investigemus de illa ut, solerti studio redolentem ubique et necubi apparentem nostris penitus irretiamus tenticulis.”}
\end{quote}

\noindent It is difficult to say whether \ia{Herkel, Jan}Herkel had read \textit{De vulgari eloquentia}, but as a man of learning he obviously knew \ia{Dante Alighieri}Dante, even applauding \ia{Linde, Samuel Gottlieb}Samuel Gottlieb Linde (1771--1847), author of a \il{Slavic}Slavic dictionary, for saying: “if the Italians, who are so very diverse in terms of \is{Dialect}dialect, have boasted a uniform written language since \ia{Dante Alighieri}Dante’s times, why should the Slavs not enjoy the same?”\footnote{\citet[23]{herkel_elementa_1826}: “si Itali \is{Dialect}dialecto distinctissimi a temporibus \ia{Dante Alighieri}Dante lingua uniformi scripturistica gloriantur, cur Slavi non gauderent?”} This quotation makes clear that \ia{Dante Alighieri}Dante and Herkel have another thing in common: their focus on written, literary forms of language. With \ia{Bacon, Roger}Roger Bacon, like \ia{{Dante Alighieri}}Dante a medieval pioneer in developing ideas on regional language variation, \ia{Herkel, Jan}Herkel shared the insistence on the \is{Dialect!Language/dialect dichotomy}Aristotelian criterion to distinguish languages from \is{Dialect}dialects, identified above and prominent especially in early modern scholarship from around the mid-sixteenth century \citep[190--194]{van_rooy_regional_2018}. \ia{Herkel, Jan}Herkel, in sum, inscribed himself in the European tradition of linguistic thought on the ideal language and on regional language variation.

\ia{Herkel, Jan}Herkel’s strong embrace of tradition does not mean that he lacked all originality. In one passage, he offered his readers a very clever take on historical semantics:

\begin{quote}
    Here we should remark that as long as the language and the people itself were still in their infancy, distinct ideas that shared some common characteristics were very often expressed by the same word. So much is clear both from books and from very ancient languages. Yet those ideas on the quality of objects were the most frequent, which most often applied to the physical condition of man, such as the idea of “good” and the idea of “bad”. Hence, they indicated everything that pleased them with the word for ‘good’, and everything that displeased them with the word for ‘bad.’ For instance, we know that ancient peoples, and especially the Slavs, delighted in the color white, and they expressed this idea with the word for ‘good’, equating the white with the beautiful. On the other hand, objects which triggered an unpleasant sensation, such as something bitter, or a burning sensation on the body, they indicated with the generic word for ‘bad.’\footnote{\citet[101]{herkel_elementa_1826}: “Hic animadvertendum est, quod, dum lingua, ipsaque gens adhuc in suis incunabulis exstitisset, saepissime ideae distinctae, in aliquibus tamen notis convenientes, eodem vocabulo expressae fuerint, id patet tum e libris, tum e linguis antiquissimis, ideae vero qualitatis objectorum illae erant frequentissimae, quae creberrime in physicum hominum statum egerunt, uti est idea boni et idea mali, hinc quidquid illis placuit, vocabulo boni, et quidquid displicuit, vocabulo mali indicarunt, sic scimus antiquas, potissimum Slavicas gentes albo colore delectatas fuisse, et hanc ideam vocabulo boni expresserunt, uti et id, quod formosum fuit; e contra, objecta, quae inamoenum sensum in iis excitarunt, uti quid amari, vel quid urentis corpus, vocabulo mali insigniverint tamquam generico.”}
\end{quote}

\noindent More pertinent to the core theme of his grammar, however, is that \ia{Herkel, Jan}Herkel further fleshed out earlier ideas about the relationships between \il{Slavic}Slavic varieties. He took the idea that \il{Slavic}Slavic tongues formed one language one step further. Among early modern thinkers, Slavic unity was widely accepted, and became a theme especially in the sixteenth century. This holds for scholars of western areas of Europe such as \ia{Bodin, Jean}Jean Bodin (1529/1530--1596) and scholars of Slavic background such as \ia{Mączyński, Jan}Jan Mączyński (ca. 1520--ca. 1587). Bodin, for instance, observed:

\begin{quote}
    For surely I hear that the Polish, Bohemians, Ruthenians, Lithuanians, Muscovites, Bosnians, Bulgarians, Serbs, Croats, Dalmatians, and Vandals use the same language of the Slavs, which is used in Scandia, and that they differ only in \is{Dialect}dialect.\footnote{\citet[439]{bodin_methodus_1566}: “sic enim audio Polonos, Bohemos, Rußios, Lithuanos, Moschouitas, Boßinios, Bulgaros, Seruios, Croatios, Dalmatas, Vandalos eade[m] Sclauorum vti lingua, quæ in Sca[n]dia vsurpatur, ac sola \is{Dialect}dialecto differre.”}
\end{quote}

\noindent Similarly, Mączyński wrote in his definition of \il{Latin}Latin \textit{dialectus}:

\begin{quote}
    The Greeks \is{Dialect}call “dialects” species of languages, a property of languages, like in our \il{Slavic}Slavic language, the Pole speaks differently, the Ruthenian differently, the Czech differently, the Illyrian differently, but it is nevertheless still one language. Only does every region have its own property, and likewise it was in the \il{Greek}Greek language.\footnote{\citet[\textit{s.v. \is{Dialect}dialectus}]{maczynski_lexicon_1564}: “Dialectos \is{Dialect}Graeci vocant linguarum species, Vlasność yęzyków yáko w nászim yęzyku \il{Slavic}Slawáckim ynáczey mowi Polak ynáczey Ruśyn, ynáczey Czech ynaczey Ilyrak, á wzdy yednak yeden yęzyk yest. Tylko ysz każda ziemiá ma swę wlasność, y tákże też w \il{Greek}Greckim yęzyku bylo.” The form \textit{wzdy} should be read as \textit{wżdy}. For this information, see \citet[33]{seldeslachts_every_2022}, whence also the \il{English}English translation is taken.}
\end{quote}

\noindent These scholars seem to have transposed the idea that a language such as \il{Greek}Ancient Greek or \il{German}German had different \is{Dialect}dialects to the level of the Slavic family of languages as a whole, counterintuitive for present-day readers, who perhaps would expect a projection of this idea on the lower level of individual \il{Slavic}Slavic tongues such as \il{Polish}Polish, \il{Bulgarian}Bulgarian, or \il{Russian}Russian (see \cite{maxwell_greece_2022}). However, for a long time there was a consensus that \il{Slavic}Slavic was the language, which appeared in different regional guises \is{Dialect}or “dialects” across a large portion of eastern Europe and beyond (see \cite{maxwell_effacing_2018}). This idea culminated in the nineteenth century, where it became known under the term \is{Pan-Slavism}\textit{Panslavismus} (see \cite[9]{kamusella_politics_2021}; and \ia{Maxwell, Alexander}\hyperref[ch:Maxwell]{Maxwell’s essay in this volume}). Present-day readers should therefore be weary of projecting their current ideas about the languagehood of individual \il{Slavic}Slavic varieties like \il{Slovak}Slovak back in time, as for centuries scholars believed them to be \is{Dialect}dialects of a superordinate \il{Slavic}Slavic language. The historical sources should be interpreted on their own terms, with careful attention to the conceptual toolkit their authors rely on, in order to avoid anachronistically attributing present-day ideas to earlier thinkers.

Generally, and probably rightly so, Jan \ia{Herkel, Jan}Herkel is credited with coining the influential term \is{Pan-Slavism}\textit{Panslavismus}. He did so at the very beginning of his grammar, entitled \textit{Elements of a universal \il{Slavic}Slavic language, drawn from the living \is{Dialect!Living dialect}dialects and based on sound logical principles}. Announced in the title, the idea of \is{Pan-Slavism}Pan-Slavism is crystallized in the \il{Latin}Latin term \is{Pan-Slavism}\textit{Panslavismus} on page 4, where \ia{Herkel, Jan}Herkel writes of the \il{Slavic}Slavic language that its

\begin{quote}
    genuine principles should preferably not be sought in one but in all \is{Dialect}dialects. Hence it also follows naturally that this language, as the original, should be cultivated by the common effort of the Slavic nations; only in this way, following the example of other nations, will flourish, even in the face of geographic, historic and political diversity, the greatly desired \textit{Union in Literature} among all Slavs, which is the true \is{Pan-Slavism}Pan-Slavism.\footnote{\citet[4]{herkel_elementa_1826}: “[…] \il{Latin}genuina principia non in una, sed in omnibus \is{Dialect}dialectis quaerenda esse pronum est; hinc suapte etiam fluit, Linguam hanc ut pote Originalem communi Slavicarum Nationum conatu esse colendam, hinc tantum efflorescet etiam penes diversitatem geographicam, historicam et politicam, ad exemplum aliarum Nationum, exoptata \textit{Unio in Litteratura} inter omnes Slavos, sive verus \is{Pan-Slavism}Panslavismus.”}
\end{quote}

\noindent For \ia{Herkel, Jan}Herkel, literature – understood broadly as the written word – presented itself as the domain where the Slavs should try to overcome their manifold differences, since in other domains changes were if not impossible (e.g. history and geography), at least unfeasible in the foreseeable future (e.g. spoken language and politics). To counter this diversity, he put forward a first proposal to make \il{Slavic}Slavic \is{Orthography!Orthographic unity}orthography and morphology homogeneous, not in order to offer a definitive solution but rather to stimulate the international community of scholars he was addressing with his \il{Latin}Latin work to reflect on and debate the matter. Or in \ia{Herkel, Jan}Herkel’s own words:

\begin{quote}
    […] I have decided to put forward in this booklet some proposals about a common method of writing \il{Slavic}Slavic and \is{Inflection}inflecting its parts of speech, yet in such a manner that I myself also invite men skilled in the philology and etymology of the \il{Slavic}Slavic tongue either to endorse my proposals or to refute them and formulate more suitable proposals.\footnote{\citet[3]{herkel_elementa_1826}: “[…] \il{Latin}quaedam de communi \il{Slavic}Slavice scribendi, partes Orationis \is{Inflection}inflectendi ratione hoc in opusculo proponere statui, ita tamen, ut philologice, et etymologice gnaros \il{Slavic}Slavici sermonis viros ipse ultro orando provocem ad ea, quae propono vel stabilienda, vel refellenda, et commodiora proponenda.”}
\end{quote}

\noindent However, in this contribution, I will not focus on \ia{Herkel, Jan}Herkel’s term \is{Pan-Slavism}\textit{Pansla\-vismus}, not only because \ia{Maxwell, Alexander}Alexander Maxwell has already discussed its historical context in detail in the previous chapter, but also because I argue that \is{Pan-Slavism}\textit{Panslavismus} is not the key term of \ia{Herkel, Jan}Herkel’s \textit{Elements}, despite its powerful legacy. Indeed, it is another term that already featured in the quotation with which I started my chapter: \is{Genius}\textit{genius}, which occurs 62 times in his grammar.

\section{Genius: a broad historical view on Herkel’s key term}
\label{sec:Section 2.2}

The \il{Latin}Latin word \is{Genius}\textit{genius} contains the stem \textit{gen}-, also found in \textit{gignere} (\textit{gi}-\textbf{\textit{gn}}-\textit{ere}), ‘to create, to engender’, and is probably related closely to \textit{gens}, ‘people’, if it is not derived from it. In fact, the original meaning is something like ‘the \is{Genius}spirit of a \textit{gens}’ or also of a particular place or person (\cite[\textit{s.v. \is{Genius}genius}]{lewis_latin_1879}; \cite[260--261]{de_vaan_etymological_2008}).\footnote{It is of note that \ia{Herkel, Jan}Herkel occasionally used \is{Genius}\textit{spiritus} as a synonym of \is{Genius}\textit{genius}. See \citet[54, 61, 138]{herkel_elementa_1826} and \citet{buzassyova_wissenschaftliche_2012}.} In the first half of the sixteenth century, and no later than the 1540s, the term \is{Genius}\textit{genius} came to be applied to \is{Genius}the “spirit” of a language, or in Toon \citeauthor{van_hal_genie_2013}’s (\citeyear[92]{van_hal_genie_2013}) phrasing: “the subtle properties of a certain language giving way to serious translation problems.”\footnote{The sixteenth century seems to have been an era which witnessed a large number of metalinguistic neologisms. An important term borrowed into \il{Latin}Latin around 1500 is \is{Dialect}\textit{dialectus}: see \citet{van_rooy__2019}.} Although especially popular in the early modern period, the underlying idea can be traced even further back in time: to the work of Early Christian authors, engaged in reading, interpreting and translating the Bible in its original languages. In that era, the untranslatable particularities of individual languages were expressed in the first place by the \il{Greek}Ancient Greek term \textit{idíōma} (ἰδίωμα), ‘property’ or ‘individuality of tongue’, which usage \ia{Origen}Origen (ca. 185--ca. 253) probably pioneered in his work, preserved only very fragmentarily today, partly in \il{Latin}Latin translation. In any case, according to the Early Christian \il{Latin}Latin author \ia{Jerome}Jerome, \ia{Origen}Origen advocated the view that in some cases it is impossible to translate a word or phrase in a text properly, “on account of the native \is{Idiom}idiom of each of the two languages”, and that in these cases it is better to leave the original expression untranslated.\footnote{Cited from \citet[52]{bartelink_hieronymus_1980}: \textit{Ep}. 26.1: “[…] licet […] \ia{Origen}Origenes \il{Latin}adserat, propter vernaculum linguae utriusque \is{Idiom}idioma non posse ita apud alios sonare, ut apud suos dicta sunt, et multo melius ininterpretata ponere, quam vim interpretatione tenuare.”} This concept found an expression in the phrase \textit{idíōma tês glṓttēs} (ἰδίωμα \il{Greek}τῆς γλώττης), which the Early Christian \il{Greek}Greek author \ia{{Theodoretus of Cyrrhus}}Theodoretus of Cyrrhus (ca. 393--ca. 458/466) adopted, applying it twice to the \is{Idiom}idiom of \il{Hebrew}Hebrew: “the property of that tongue spoke as follows”, and “doubling is a property of the \il{Hebrew}Hebrews’ tongue”.\footnote{\citet[15]{fernandez_marcos_theodoreti_1979}: “τὸ \il{Greek}τῆς γλώττης ἐκείνης ἰδίωμα οὕτως ἔφη […]”; \citet[81, 1145]{migne_patrologiae_1857}: “Ὁ \il{Greek}δὲ διπλασιασμὸς τῆς Ἑβραίων γλώττης ἰδίωμα […].”} It seems that the phrase got translated into \il{Latin}Latin as \textit{proprietas linguae}, probably from \ia{Origen}Origenes’ work, and that it became more popular in \il{Latin}Latin Early Christian works than it had been in \il{Greek}Greek Early Christian texts.\footnote{A proximity search for \textit{proprieta*} and \textit{linguae} that allowed a maximum of three intervening words, using the Cross Database Searchtool (CDS) of Brepolis \il{Latin}Latin databases <\url{https://clt.brepolis.net/cds/pages/Search.aspx}> (conducted on 26 June 2022), resulted in 48 hits, with 4 instances being recorded doubly; this brings the total number of hits to 44. Most notably, \ia{Rufinus}Rufinus’ translation of \ia{Origen}Origenes’ \textit{In Numeros homiliae} 27.13 (cited from the CDS) reads: “Verum ne huiusmodi expositio, quae per Hebraeorum nominum significantias currit, ignorantibus linguae illius proprietatem affectata videatur et violenter extorta, dabimus etiam in nostra lingua similitudinem, qua consequentiae huius ratio patescat.”} Its most eager users are \ia{Augustine}Augustine and especially the Bible translator \ia{Jerome}Jerome, who employed the phrase also in four prefatory letters accompanying the translation of Psalms.\footnote{The search reported in the previous note resulted in 8 hits for Augustine and 18 for \ia{Jerome}Jerome.}

The phrase \textit{proprietas linguae} continued being used throughout the Middle Ages.\footnote{76 hits in CDS as per a search conducted on 27 June 2022.} The idea that each language had its individuality became mixed up with the idea that a language had different varieties, which later would be \is{Dialect}called “dialects”. This confusion emerges most clearly from the works of thirteenth‐cen\-tury Franciscan friar \ia{Bacon, Roger}Roger Bacon:

\begin{quote}
    but it is impossible that the property of one language is preserved in another. For also \is{Idiom}idioms of the same language are varied among diverse [nations], as is clear from the \il{French}French language, which is varied by a manifold \is{Idiom}idiom among the Walloons, the Picards, the Normans, and the Burgundians. And what is properly said in the idiom of the Picards, becomes rough among the Burgundians, indeed even among the Walloons, who live nearer to Picardy; so how much more will this occur among different languages? Therefore, what is well done in one language, cannot be transferred to another according to the property it had from the start. (The \il{English}English translation of \ia{Bacon, Roger}Bacon’s original \il{Latin}Latin is quoted from \cite[35]{van_rooy_language_2020})
\end{quote}

\noindent The interference of \is{Idiom}the “idiom” \is{Dialect}and “dialect” concepts continued in the early modern period, even when the \il{Latin!Neo-Latin}Neo-Latin borrowing from \il{Greek}Greek \is{Dialect}\textit{dialectus} was adopted as metaterm to designate a variety of a “language”, however a scholar defined this conceptual distinction between “language” \is{Dialect!Language/dialect dichotomy}and “dialect”.\footnote{See the discussion in \citet[76--78]{van_rooy_language_2020}.} Perhaps as a strategy to circumvent the conceptual confusion \is{Idiom}of “idiom” \is{Dialect}and “dialect”, sixteenth-century scholars jargonized \is{Genius}\textit{genius} to convey a language’s individuality and \is{Idiomaticity}idiomaticity. They reasoned, it seems, that if in the \il{Latin!Classical Latin}classical Latin expression a people and a region can have their \is{Genius}\textit{genius}, their peculiar character, the same noun could be applied to a people’s language. Given the close tie scholars saw between a people, their territory, and their language, harking back to antiquity, the early modern association of the term \is{Genius}\textit{genius} with language comes as no surprise, all the more since in this period, as Joep \citet[17--32]{leerssen_imagology_2007} has argued, ethnic stereotypes were starting to crystallize.\footnote{For the age-old tie between language and ethnic character, see \citet{van_hal__2013}.}

It is surely no coincidence that one of the early adopters of the phrase \is{Genius}\textit{genius linguae}, ‘genius of a language’, Theodore Bibliander (1504/1509--1564), also conflated the concepts \is{Idiom}of “idiom” \is{Dialect}and “dialect” (\cite[84--85]{van_hal_genie_2013}; \cite[77--78]{van_rooy_language_2020}). Perhaps this confusion incited Bibliander to look for a clearer terminological apparatus that allowed for a distinction of the two concepts, although it must be stressed that his conceptual framework to refer to linguistic diversity and individuality is terribly chaotic, allowing for significant polysemy in linguistic metalanguage. He moreover was, as said, merely an early adopter of the phrase \is{Genius}\textit{genius linguae} in his \citeyear{bibliander_optimo_1542} commentary on \il{Hebrew}Hebrew grammars, where he, as can be expected, applied the phrase to the singularity of \il{Hebrew}Hebrew \citep[15]{bibliander_optimo_1542}. A search in Google Books for \textit{linguae} \is{Genius}\il{Latin}\textit{genius} reveals that, as early as 1529, an opponent of \ia{Luther, Martin}Martin Luther’s plan to translate the Bible into \il{German}German praised “that true and natural \is{Genius!Genius of the Latin language}genius of the \il{Latin}Latin language”.\footnote{\citet[Y1v]{of_saxony_epistolae_1529}: “Neq[ue] \il{Latin}sane hactenus vllu[m] libru[m] vllasue l[ite]ras legi, in q[ui]bus verus ille \& natiuus \il{Latin}latinæ linguæ \is{Genius!Genius of the Latin language}genius fuisset.”} Future research will no doubt be able to adjust and correct our image, especially once extensive and reliable corpora will be established. \is{Genius}The “genius” concept soon led to the idea that all languages had their \is{Genius}own “genius”, which a letter of the humanist \ia{Lipsius, Justus}Justus Lipsius (1547--1606) succinctly voiced in 1586: “every language has its own characteristics which cannot be torn away from it and simply transferred to another body.”\footnote{Quoted from \citet[88]{van_hal_genie_2013}: “Est \il{Latin}suus videlicet cuique linguae \is{Genius}Genius, quem non avellas, nec temere migrare iusseris in corpus alienum.”} \ia{Lipsius, Justus}Lipsius’ letters were widely read and probably stimulated the \is{Genius}\textit{linguae genius} collocation which can be found throughout the early modern era. It came to rival, and eventually overtake, the more ancient phrase \textit{linguae \il{Latin}proprietas}, to which it was more or less synonymous.\footnote{\citet[88--90]{van_hal_genie_2013}, where other (near-)synonyms are noted, such as \textit{indoles}. Van Hal also offers select examples for the entire early modern period.} In its \il{French}French version, the phrase \textit{génie de la langue} featured quite prominently in a speech by \ia{de Bourzeys, Amable}Amable de Bourzeys from 1635, edited only in modern times, and wrongly credited with coining the phrase. In it, one can read observations such as: “every language has its air and its particular \is{Genius}genius.”\footnote{Cited from \citet[84]{van_hal_genie_2013}: “Chaque \il{French}langue a son air et son genie particulier.”} Most notably, \ia{de Bourzeys, Amable}Bourzeys was exceptional in tying the \is{Genius}genius of a language explicitly to the “nation” speaking it, thus expanding the link that scholars had been making for centuries between language and its speech community \citep[242]{hullen_characterization_2001}.

However, \ia{de Bourzeys, Amable}Bourzeys worked in an age before nationalism. Even though many early moderns came to express increasing patriotic sentiments, the nation-state was still a long way in the future. In the process leading up to this, language became closely associated with national identity, crystallized in phrases such as \textit{ut \il{Latin}lingua, natio}, ‘like language, like nation.’\footnote{For France and Italy, respectively, see \citet{siouffi_genie_2010}, and \citet{gambarota_irresistible_2011}.} Around the turn of the nineteenth century, emergent nationalism probably further reinforced this association, as language became nationalized together with a host of other domains related to public life. This evolution, however, requires further analysis, but it is clear that it partly ran in parallel with the rationalization of language in the course of the early modern period. Starting from \ia{Scaliger, Julius Caesar}Julius Caesar Scaliger’s extensive description of \il{Latin}Latin and culminating in the Port Royal universal grammar, \is{Rationalism}rationalist approaches to language in early modernity have attracted extensive scholarly attention.\footnote{\citet{lardet_jules-cesar_2019}; \citet{arnauld_grammaire_1660}. See e.g. \citet[Chapter 15]{eco_search_1995}.} The ways in which \is{Rationalism}rationalism influenced the interpretation of existing metalinguistic terminology would benefit from a concerted investigation going beyond Germanic and Romance Europe, in general, and the eighteenth-century \is{Enlightenment}Enlightenment, in particular.\footnote{For some concepts as understood by the contributors to the \textit{Encylopédie}, see \citet{swiggers_grammaire_1986}.} The same holds for nationalism, and the combined impact of \is{Rationalism}rationalism, \is{Romanticism}Romanticism, and nationalism on linguistic ideas in the Slavic world has been poorly studied to this date. The analysis I present below of \ia{Herkel, Jan}Herkel’s term \is{Genius}\textit{genius} suggests that it might be fruitful to combine these perspectives in analyzing to what extent they influenced linguistic interpretations of \is{Genius}\textit{genius} in the Slavic world.

\section{The direct sources for Herkel’s \textit{genius}}

\ia{Herkel, Jan}Herkel was certainly not the first Slav to apply the term \is{Genius}\textit{genius} (and \is{Genius}\textit{spiritus}) to this language family, and Ľudmila \citet[136]{buzassyova_wissenschaftliche_2012} has claimed that his usage of the term “probably stems from \ia{Kopitar, Jernej}Jernej Kopitar’s workshop as well”, just as several other aspects of \ia{Herkel, Jan}Herkel’s work.\footnote{“Es \il{German}ist eine Kategorie, die wahrscheinlich auch aus \ia{Kopitar, Jernej}Jernej Kopitars Werkstatt stammt.” See e.g. \citet[48]{herkel_elementa_1826}, where \ia{Kopitar, Jernej}Kopitar is explicitly followed.} While I have illustrated above that the origin of the phrase harks back to the sixteenth century, \ia{Buzássyová, Ľudmila}Buzássyová may well be right in attributing \ia{Herkel, Jan}Herkel’s usage to \ia{Kopitar, Jernej}Kopitar’s influence. However, \ia{Kopitar, Jernej}Kopitar cannot have been \ia{Herkel, Jan}Herkel’s only source, since he used the \il{German}Germanized term \is{Genius}\textit{Genius} only sparingly in his \textit{Grammatik der Slavischen Sprache in Krain, Kärnten und Steyermark} of \citeyear{kopitar_grammatik_1808}, which did not give \ia{Herkel, Jan}Herkel much to go on, since \ia{Kopitar, Jernej}Kopitar employed it quite vaguely.\footnote{Searching the digitization of Google Books, I only came across three attestations in more than 500 pages: \citet[37]{kopitar_grammatik_1808}: “die \il{German}Stimme des Slavischen \is{Genius}Genius”; (\citeyear[128]{kopitar_grammatik_1808}): “von \il{German}einem \is{Genius}Genius der Slavischen Sprache hatte ihm nie geträumt”; (\citeyear[302]{kopitar_grammatik_1808}): “Ueberhaupt \il{German}scheint der \is{Genius}Genius des Slavischen Sprache auf einen ganz eigenen Weg zur Behandlung des Verbi hinzuweisen, welchen zu verfolgen wir jetzt nicht gerüstet sind.”} The term \is{Genius}\textit{genius} shows up more frequently in other works \ia{Herkel, Jan}Herkel cited in his \textit{Elements}. Indeed, it seems that he particularly followed the example of \ia{Dobrovský, Josef}Josef Dobrovský, who like his colleague \ia{Kopitar, Jernej}Kopitar was an advocate of \is{Orthography!Orthographic reform}orthographic reform in \il{Slavic}Slavic and an important source of \ia{Herkel, Jan}Herkel’s; \ia{Buzássyová, Ľudmila}Buzássyová acknowledges \ia{Herkel, Jan}Herkel’s debt to him, but not for the term \is{Genius}\textit{genius}.\footnote{See also \ia{Maxwell, Alexander}\hyperref[ch:Maxwell]{Maxwell’s essay in this volume}.} In \ia{Dobrovský, Josef}Dobrovský’s grammar of \il{Old Church Slavonic}Old Church Slavonic, written in \il{Latin}Latin and published in \citeyear{dobrovsky_institutiones_1822}, \is{Genius}\textit{genius} occurs six times, and in a more marked way than in \ia{Kopitar, Jernej}Kopitar’s grammar. For instance, an \il{Old Church Slavonic}Old Church Slavonic rendering of a \il{Greek}Greek composite word is said to be “wholly conform to the \is{Genius!Genius of the Slavic language}genius of the \il{Slavic}Slavic language”, a phrasing encountered also in \ia{Herkel, Jan}Herkel’s \textit{Elements}, where a certain form is said to “conform to the \is{Genius}genius of the language”.\footnote{\citet[455]{dobrovsky_institutiones_1822}: “[...] \is{Genius!Genius of the Slavic language}genio linguae \il{Slavic}slavicae prorsus conformi.” See our translation of \citet[93]{herkel_elementa_1826}: “[...] \is{Genius}genio linguae conformiter dictum est.”} \ia{Dobrovský, Josef}Dobrovský’s most telling observation involving the term \is{Genius}\textit{genius} regards the \il{Slavic}Slavic \is{Idiomaticity}idiomaticity that is not respected in rendering the \il{Greek}Greek New Testament:

\begin{quote}
    You have here a specimen of the grammatical corrections made by monks of the holy mountain, who push the \il{Greek}Greek subtleties too far through, thus violating the \is{Genius!Genius of the Slavic language}genius of the \il{Slavic}Slavic language because of an excessive reverence for Biblical Greekness.\footnote{\citet[713]{dobrovsky_institutiones_1822}: “Habes hic specimen correctionum grammaticarum monachorum sancti montis, \il{Greek}graecas subtilitates nimis prementium, itaque prae nimia in biblicam graecitatem reverentia \il{Slavic}\is{Genius!Genius of the Slavic language}Slavicae linguae genio vim inferentium.”}
\end{quote}

\noindent Notably, \ia{Dobrovský, Josef}Dobrovský applied the term \is{Genius}\textit{genius} also to \il{Greek}Greek in three instances, and clearly invoked it in the context of translation, whereas \ia{Herkel, Jan}Herkel did not, using it only in connection with \il{Slavic}Slavic. In one case, Herkel’s reasoning clearly followed that of \ia{Dobrovský, Josef}Dobrovský in the quotation above. When discussing two alternative \il{Slavic}Slavic translations of a New Testament passage, \ia{Herkel, Jan}Herkel preferred the option that “also occurs in the most ancient Bible as if it were an original \il{Slavic}Slavic expression”, whereas the other alternative “evokes the \il{Greek}Greek text more than the \is{Genius}\il{Slavic}Slavic genius”.\footnote{\citet[35]{herkel_elementa_1826}: “[...] haec loquendi ratio exstat in antiquissimis Bibliis tamquam originalis Slavica expressio, prior enim magis \il{Greek}Graecum textum, quam \il{Slavic}Slavicum \is{Genius}genium redolet.”} \citeauthor{dobrovsky_ausfuhrliches_1809} did not use the term in his earlier grammar handbook of \il{Czech}Czech (\citeyear{dobrovsky_ausfuhrliches_1809}), which he wrote in \il{German}German for both \il{German}German and \il{Czech}Czech speakers, and which \ia{Herkel, Jan}Herkel also quoted in his \textit{Elements}. This fact suggests that \is{Genius}\textit{genius} might have been typical especially of the \il{Latin}Latin metalanguage of \il{Slavic}Slavic grammar, although certainly not exclusively. In addition to \ia{Kopitar, Jernej}Kopitar, \ia{Herkel, Jan}Herkel no doubt also encountered the \il{German}German form \is{Genius}\textit{Genius} in Jerzy Samuel \citeauthor{bandtkie_polnische_1824}’s \il{Polish}Polish grammar (\citeyear{bandtkie_polnische_1824}), where it is used twice, though only once in the linguistic sense.\footnote{\citet[287]{bandtkie_polnische_1824}: “[...] in \il{German}jeder dieser vier Sprachen nach ihrem besondern \is{Genius}Genius [...].” The other instance occurs at page 30 but expresses the meaning of ‘exceptionally intelligent person’ (\citeyear[30]{bandtkie_polnische_1824}).} \is{Genius}\textit{Genius} likewise makes two appearances in Alexander \citeauthor{adamowicz_praktische_1796}’s \il{Polish}Polish grammar for \il{German}German learners (\citeyear{adamowicz_praktische_1796}), both times in the linguistic sense.\footnote{\citet[29]{adamowicz_praktische_1796}: “[...] sich überhaupt mit dem \is{Genius}Genius der polnischen Deklinationen bekannter gemacht hat”; (\citeyear[161]{adamowicz_praktische_1796}): “Ueberhaupt ist zu bemerken, daß wenn man den Infinitivum umschreiben kann, man allemal richtiger und dem \is{Genius}Genius der Sprache gemäßer sprechen wird.”} It does not feature in other sources of \ia{Herkel, Jan}Herkel’s, not even \citeauthor{puchmayer_lehrgebaude_1820}’s \il{Russian}Russian grammar (\citeyear{puchmayer_lehrgebaude_1820}), even though it is based on \ia{Dobrovský, Josef}Dobrovský’s \il{Czech}Czech grammar. It is also absent from Anton \citeauthor{bernolak_grammatica_1790}’s \textit{Grammatica \il{Slavic}slavica} of \citeyear{bernolak_grammatica_1790} and from \citeauthor{dainko_lehrbuch_1824}’s \il{Windic}Windic grammar of \citeyear{dainko_lehrbuch_1824}. As far as I can tell, even as prominent a scholar and compiler as \ia{Karadžić, Vuk}Vuk Karadžić (1787--1864), who insisted on the individuality of \il{Serbian}Serbian, and whom \citet[51]{herkel_elementa_1826} cited as “the illustrious \textit{Vuk}”, did not grant a central position to \is{Genius}a “genius”-like concept either, perhaps because he was not looking for \is{Pan-Slavism}Pan-Slavic commonalities.

\ia{Herkel, Jan}Herkel thus probably found the term \is{Genius}\textit{genius} and his key phrase \is{Genius!Genius of the Slavic language}\il{Slavic}\textit{genius Slavicae linguae} in several of his sources and naturally took them over. It seems that the term \is{Genius}\textit{genius} is particularly frequent in those grammatical works among \ia{Herkel, Jan}Herkel’s sources that appeared in the German-Slavic borderlands, and featured most prominently in \ia{Dobrovský, Josef}Dobrovský’s extensive \il{Old Church Slavonic}Old Church Slavonic grammar, written in \il{Latin}Latin and published in Vienna. These works were at least partly written with a non-Slavic – usually \il{German}German-speaking – target audience in mind, and it is not inconceivable to conjecture that the Slavs addressing these others adopted a concept widely applied to classical and vernacular languages and extended it to \il{Slavic}Slavic in order to make their readers feel at home. In the European tradition, especially under impulse of \is{Enlightenment}Enlightenment scholarship, \is{Genius}the “genius of a language” had become a very popular concept, closely associated with ethnic-regional groups and their different cognitive constitutions, as \ia{Beauzée, Nicolas}Nicolas Beauzée’s entry \textit{Grammaire} in the \textit{Encyclopédie} may suffice to illustrate:

\begin{quote}
    The diversity of climates; the political constitution of States; the revolutions that change its appearance; the status of the sciences, arts, and trade; religion and the degree to which one is attached to it; the opposing pretenses of nations, provinces, cities, families even: all that contributes to adopting a view of things that varies from here to there, from yesterday to tomorrow; and it’s the origin of the diversity of the \is{Genius}geniuses of languages.\footnote{My \il{English}English translation of the \il{French}French text, cited in \citet[780]{hasler_reichtum_2009}: “La diversité des climats; la constitution politique des Etats; les révolutions qui en changent la face; l’état des sciences, des arts, du commerce; la religion \& le plus ou le moins d’attachement qu’on y a; les prétentions opposées des nations, des provinces, des villes, des familles même: tout cela contribue à faire envisager les choses, ici sous un point de vûe, là sous un autre, aujourd’hui d’une façon, demain d’une maniere toute différente; \& c’est l’origine de la diversité des génies des langues.”}
\end{quote}

\noindent The concept \is{Genius}of “genius” gained further momentum due to \is{Romanticism}Romanticism, which partly reacted against \is{Rationalism}\is{Enlightenment}Enlightenment rationalism. In the case \is{Genius}of “genius”, however, the two trends coalesced, resulting in a concept that combined features of both, as in the case of \ia{Herkel, Jan}Herkel (see \hyperref[sec:Genius restyled]{my analysis in the next section}). Just like the \textit{Encyclopédie} is often used synonymously with the \is{Enlightenment}Enlightenment, \is{Romanticism}Romantic thought typically conjures up the name of Johann Gottfried von Herder (1744--1803). In his \textit{Abhandlung über den Ursprung der Sprache}, published in \citeyear{herder_abhandlung_1772}, \citeauthor{herder_abhandlung_1772} occasionally used the term \is{Genius}\textit{Genius}, but mainly in the meaning of a people’s rather than a language’s \is{Genius}spirit, with one notable exception.\footnote{I have found five instances. See \citet[81, 85 (twice), 185, 212]{herder_abhandlung_1772}.} Toward the end of his \textit{Abhandlung}, \citet[212]{herder_abhandlung_1772} directed criticism toward travel writers and missionaries; among them, “there have been so few true language philosophers, who would have been able or willing to report about the \is{Genius}genius and the characteristic grounds of their tribes’ languages, that one generally still errs in this respect”.\footnote{\citet[212]{herder_abhandlung_1772}: “Zudem sind unter den Reisebeschreibern und selbst Mißionarien so wenig wahre Sprachphilosophen gewesen, die uns von dem \is{Genius}Genius und dem charakteristischen Grunde ihrer Völkersprachen hätten Nachricht geben können oder wollen, daß man im Allgemeinen hier noch in der Irre gehet.”} This lack of data and expert analyses is also why Herder refrained from offering language genealogies in his work.

While the concept \is{Genius}of “genius”, as shown in \hyperref[sec:Section 2.2]{Section 2.2}, has roots as far back as the sixteenth century, it was popularized mostly by the grammatical and linguistic thought of seventeenth‐century France \citep[784]{hasler_reichtum_2009}. Soon, the concept gained a firm footing also in \il{German}German-speaking territories, thriving during the \is{Enlightenment}Enlightenment and the \is{Romanticism}Romantic movement, and through these in the Slavic world at the turn of the nineteenth century. There, \ia{Herkel, Jan}Herkel seems to have acted as an exceptionally eager adopter and promotor of the term \is{Genius}\textit{genius} as part of his \is{Pan-Slavism}Pan-Slavic project, hunting as it were for the core features of the \il{Slavic}Slavic language panther, to use \ia{{Dante Alighieri}}Dantean imagery.

In conclusion, \ia{Herkel, Jan}Herkel and his colleagues were partly writing for a non-Slavic target audience, and resorted to a concept firmly anchored in the European tradition of linguistic thought to talk about the \il{Slavic}Slavic linguistic situation and its basic unity. In the case of \ia{Herkel, Jan}Herkel, \is{Rationalism}\is{Enlightenment}Enlightenment ideas about the rationality of language seeped through in his use of \is{Genius}genius, which he actively restyled on a fundamental level, as I argue in the next section.

\section{\textit{Genius} restyled: a touchstone concept for Herkel}
\label{sec:Genius restyled}

\ia{Herkel, Jan}Herkel used the term \is{Genius}\textit{genius} 62 times on 164 pages, thus occurring, on average, once every three to four pages. By far the most often, the term \is{Genius}\textit{genius} appears in conjunction with the genitive \textit{linguae} (54 times), often accompanied by the \il{Slavic}adjective \textit{Slavicae} (25 times) – see \hyperref[fig:Figure 2.1]{Figure 2.1}.

\begin{figure}
    \caption{Collocations with \textit{genius} in Herkel’s \textit{Elements}}
    \label{fig:Figure 2.1}
    \begin{tikzpicture}
        \pie[sum=auto, radius=2, rotate=130, explode=0.1, text=legend, color={darkgray!70, darkgray!60, darkgray!50, darkgray!40, darkgray!30, darkgray!20, darkgray!10, lightgray!0}]
            {29/genius linguae, 
            25/genius Slavicae linguae, 
            2/genius dialecti, 
            2/genio conformiter {/} conformis, 
            1/genius Slavicus, 
            1/genius verborum in lingua Slavica, 
            1/genius linguae suae maternae, 1/genius eius},
    \end{tikzpicture}
\end{figure}



In all cases, it refers to \il{Slavic}Slavic, explicitly or implicitly. Telling is, for instance, \ia{Herkel, Jan}Herkel’s observation, quoted already above, that a certain \il{Slavic}Slavic Bible translation “evokes the \il{Greek}Greek text more than the \il{Slavic}Slavic \is{Genius}genius”, in which case he seems to have purposely contrasted the \il{Greek}Greek textual facts in the New Testament with the elusive \il{Slavic}Slavic \is{Genius}genius (see \cite[35]{herkel_elementa_1826}, quoted in \hyperlink{page.57}{footnote 31} above). Unlike \is{Pan-Slavism}\textit{Panslavismus}, featuring only once in his introduction, \is{Genius}\textit{genius} occurs throughout the \textit{Elements}, being concentrated especially in the sections devoted to \il{Slavic}Slavic \is{Inflection}inflection: especially noun \is{Declension!Noun declension}declensions (25 instances) and verb \is{Conjugation}conjugations (22~instances). This is not surprising, as \ia{Herkel, Jan}Herkel aimed to propose a new \is{Pan-Slavism}Pan-Slavic norm in \is{Orthography}orthography and grammar that corresponded as closely as possible to the \il{Slavic}Slavic \is{Genius}genius. He was looking for conformity among the \il{Slavic}Slavic \is{Dialect}dialects that reflected this \is{Pan-Slavism}Pan-Slavic \is{Genius!Genius of the Slavic language}genius, a supranational linguistic level that at the same time transcended and united the Slavic nations. This unity could be achieved with regard to language and literature, \ia{Herkel, Jan}Herkel made clear at the outset of his \textit{Elements}, as he cherished no hopes for political unification.\footnote{See on \ia{Herkel, Jan}Herkel’s \is{Pan-Slavism!Apolitical Pan-Slavism}apolitical Pan-Slavism and nationalism \ia{Maxwell, Alexander}\hyperref[ch:Maxwell]{Alexander Maxwell’s essay in this book}.} Instead, \ia{Herkel, Jan}Herkel launched his \is{Pan-Slavism}Pan-Slavic \is{Genius!Genius of the Slavic language}\textit{genius} as a \is{Touchstone concept}linguistic touchstone to develop a cultivated written language, based on \is{Rationalism}rational principles and approximating the invisible \is{Genius!Genius of the Slavic language}\textit{genius} of the \il{Slavic}Slavic language – and hence its original form – as closely as possible. In the remainder of my contribution, I will investigate how \ia{Herkel, Jan}Herkel interpreted the \is{Genius}\textit{genius} concept in the frame of his linguistic idealism.

\largerpage[-1]
The end of \ia{Herkel, Jan}Herkel’s section on the \is{Declension!Adjective declension}declension of adjectives features a key quotation for a better understanding of his \is{Genius}\textit{genius} concept: “That method of \is{Inflection}inflection which conforms to the \is{Genius}genius of the language and prevails in all \il{Slavic}Slavic \is{Dialect}dialects should be adopted.”\footnote{\citet[95]{herkel_elementa_1826}: “Ea \is{Inflection}inflectendi ratio est adoptanda, quae et \is{Genius}genio linguae conformis est, et in omnibus \il{Slavic}Slavicis \is{Dialect}dialectis viget […].”} A linguistic form adopted in \is{Pan-Slavism!Linguistic Pan-Slavism}\il{Slavic!Pan-Slavic}Pan-Slavic should conform to the principles found in the blue\-print that \il{Slavic}Slavic \is{Dialect}dialects shared. The full title of the \textit{Elements} already announced that the recovery of this blueprint should be “based on sound logical principles”.\footnote{\citet[title page]{herkel_elementa_1826} “[…] sanis logicae principiis suffulta.”} Nowhere in his book, however, did Herkel define these “logical principles”. Nevertheless, they can be gathered from passing observations, such as the following:

\begin{quote}
   Thus it follows that only one logical form of the neuter \is{Inflection}inflections can be established, from which it is clear that combining the \is{Dialect}dialects is absolutely indispensable for cultivating the \il{Slavic}Slavic language.\footnote{\citet[19]{herkel_elementa_1826}: “hinc sequitur \is{Inflection}inflectendorum neutrorum nonnisi unicam logicam posse stabiliri formam, ex quibus patet pro cultura linguae \il{Slavic}Slavicae combinationem \is{Dialect}dialectorum esse absolute necessariam.”}
\end{quote}

\noindent In order to arrive at a \is{Rationalism}rational \is{Pan-Slavism}\il{Slavic!Pan-Slavic}Pan-Slavic language form, one needed to compare the language facts across the different \il{Slavic}\is{Dialect}Slavic “dialects”, a complicated endeavor, as the logical principles have been obfuscated in the \is{Dialect}dialects:

\begin{quote}
    For all \is{Dialect}dialects are more or less burdened by various exceptions which emerged from diverging usage. Thus the cultivation of the \il{Slavic}Slavic language needs logical combination, and then rules will emerge that are firm, plain, clear, and beneficial for both Slavs and foreigners wanting to learn this language.\footnote{\citet[22]{herkel_elementa_1826}: “omnes enim \is{Dialect}dialecti plus minus variis onerantur exceptionibus a vago usu ortis; logica itaque combinatione opus est in cultura linguae \il{Slavic}Slavicae, et tunc orientur regulae firmae, planae, clarae, et hoc ipso et Slavis et exteris hanc linguam noscere volentibus proficuae.”}
\end{quote}

\noindent Indeed, most \is{Dialect}dialectal endings and the variation in them are “clearly not based on sound logic”.\footnote{\citet[46]{herkel_elementa_1826}: “in Logica sana plane non fundatae.”} \ia{Herkel, Jan}Herkel argued, instead, that \is{Rationalism}rational consideration of all \il{Slavic}Slavic language forms should lead to the development of logical principles of writing grounded in the \il{Slavic}Slavic \is{Genius}genius:

\begin{quote}
    No \il{Slavic}Slavic \is{Dialect}dialect, viewed in isolation from the others, can reasonably serve as the common literary \il{Slavic}Slavic language. Firstly, each \is{Dialect}dialect currently \linebreak{}abounds in foreign words, even though indigenous expressions are present in the other \is{Dialect}dialects. Secondly, the individual \is{Dialect}dialects lack thoughtful principles of writing. The reason is that the nations speaking individual \is{Dialect}dialects to a greater or lesser extent mixed with other peoples, and that mixing has greatly influenced the language itself. Hence it follows that [18] the \is{Genius!Genius of the Slavic language}genius of the original \il{Slavic}Slavic language does not consist of, and is not grounded on, any one \is{Dialect}dialect, but all of them. Thus not only \il{Old Church Slavonic}Church Slavonic is relevant, but also \il{Russian}Russian, \il{Polish}Polish, \il{Bohemian}Bohemian, \il{Pannonian}Pannonian, \il{Illyrian}Illyrian, and \il{Windic}Windic, together with their \is{Dialect!Subdialect}subdialects.\footnote{\citet[17--18]{herkel_elementa_1826}: “Dialectus \is{Dialect}\il{Latin}quaecunque \il{Slavic}Slavica ab aliis separata pro litteraria communi \il{Slavic}Slavica lingua sumi rationabiliter haud potest; nam 1mo quaevis \is{Dialect}dialectus, uti nunc sunt, scatet peregrinis plus minus vocabulis, licet originariae expressiones in aliis \is{Dialect}dialectis adsint; do singillativae \is{Dialect}dialecti criticis destituuntur scribendi principiis, ratio est, quia aliquae nationes singillativarum \is{Dialect}dialectorum sunt plus minus aliis gentibus mistae, quae commistio in linguam ipsam magnum habet influxum, hinc sequitur non in una \is{Dialect}dialecto, sed in omnibus consistere, ac fundari \is{Genius!Genius of the Slavic language}genium originariae linguae \il{Slavic}slavicae, adeoque huc spectant praeter \il{Old Church Slavonic}Ecclesiasticam, \il{Russian}Russica, \il{Polish}Polonica, \il{Bohemian}Bohemica, \il{Pannonian}Pannonica, \il{Illyrian}Illyrica, \il{Windic}Vindica, una cum suis \is{Dialect!Subdialect}subdialectis.”}
\end{quote}

\noindent Logical reasoning should be combined with civic cultivation of language, Her\-kel believed. This civic cultivation corresponds more or less to what scholars of language \is{Codification}standardization would today call “elaboration”, following Einar Haugen’s example: the use of a language form in a growing body of functions, mostly public-administrative and literary.\footnote{For the term “elaboration”, see the foundational contribution by \citet{haugen_dialect_1966}. See e.g. also \citet{joseph_eloquence_1987}.} In \ia{Herkel, Jan}Herkel’s words:

\begin{quote}
    The civic cultivation of language occurs when the use of a language prevails in a civil society. The more circumstances in which a language is used, addressing more diverse or even all possible subjects, the greater the civic cultivation of that language. Hence it follows that the larger a nation with the same language and civil society may be, the greater the civic cultivation of language within it, for we suppose the affairs of a great nation will also be great.\footnote{\citet[20]{herkel_elementa_1826}: “[…] \il{Latin}civilis linguae cultura tunc est, dum linguae alicuius usus in societate civili viget, et quo linguae alicujus usus in pluribus, vel plane omnibus negotiis occurrit, eo major est etiam ejusdem linguae civilis cultura, hinc sequitur, quo amplior aliqua natio ejusdem linguae ac societatis civilis est, eo ampliorem in ea esse civilem linguae culturam, nam ampla natio ampla supponitur habere etiam negotia […].”}
\end{quote}

\noindent As the last sentence indicates, \ia{Herkel, Jan}Herkel tied this civic cultivation to emergent national feelings, but he was clearly imagining the “nation” in a broader \is{Pan-Slavism}Pan-Slavic sense, rather than invoking any particularist national concept.

The original Slavic \is{Genius}genius, however, cannot be accessed directly. Uncovering it requires far-going linguistic comparison of \il{Slavic}Slavic \is{Dialect}dialects, an intellectual exercise which \ia{Herkel, Jan}Herkel aimed to initiate with his \textit{Elements}. In the study of \il{Slavic}Slavic verbs, for instance, one needed particular care to uncover the primal \is{Genius}genius of the language, since this domain of the language had been obfuscated by grammarians oriented toward other European languages:

\begin{quote}
    Before we turn to the \is{Conjugation}inflection of verbs, however, it is important to clear up the meaning and \is{Genius}genius of the \il{Slavic}Slavic verbs, since the \is{Genius}genius of this language differs from that of all other European languages. As some \il{Slavic}Slavic grammarians have followed the norms of other languages when composing their grammars, they have entangled themselves in inextricable difficulties.\footnote{\citet[103--104]{herkel_elementa_1826}: “Antequam tamen \is{Conjugation}inflexionem \il{Latin}verborum adgrediamur, interest vim, et \is{Genius}genium verborum in lingua \il{Slavic}Slavica eruderare, siquidem \is{Genius}genium hujus linguae differat ab omnibus Europaeis linguis, et ideo, quia nonnulli Grammatici \il{Slavic}Slavici cynosuram aliarum linguarum in concinnandis suis Grammaticis sunt secuti, inexplicabilibus semet involverunt difficultatibus.”}
\end{quote}

\noindent As such, \il{Slavic}Slavic grammarians “extended” the descriptive frameworks of other European languages to their native forms of speech, a linguistic approach that has attracted much attention in recent years, although not for the Slavophone sphere (see e.g. \cite{aussant_grammaire_2021}). Instead of extending classical grammar to \il{Slavic}Slavic languages, however, the \is{Pan-Slavism}\is{Genius!Genius of the Slavic language}Pan-Slavic genius needed to be sieved out of the various \is{Dialect}dialects, a metaphor evoked by the \il{Latin}Latin verb \textit{eruderare}, ‘to clear from rubbish.’ \ia{Herkel, Jan}Herkel apparently pictured himself as an archeologist going through the \is{Pan-Slavism}Pan-Slavic irregular bits and pieces he encountered in the \is{Dialect}dialects in order to create a grand palace of the \il{Slavic}Slavic language.\footnote{Cf. \citet[150]{herkel_elementa_1826}: “Thus if we consider the \il{Slavic}Slavic language through all its \is{Dialect}dialects as one language, then every irregularity dissipates like clouds at dawn. However, if we consider the \is{Dialect}dialects individually, they will be more or less overwhelmed with exceptions, and experience teaches us that every day new ones arise. For every \is{Dialect}dialect has its particularities, or so-called provincialisms either to greater or lesser extent.” (Original \il{Latin}Latin: “[…] ut adeo, si linguam \il{Slavic}Slavicam per omnes \is{Dialect}dialectos ut unam consideraverimus linguam, omnis irregularitas veluti nebulae orto sole dissipabuntur; si vero seorsivas dialectos spectemus, plus minus seorsivis obruuntur exceptionibus, novasque in dies oriri experientia docet; nam omnis \is{Dialect}dialectus habet suas singularitates, seu ita dictos Provincialismos jam majoris, jam minoris extensionis […].”)} This image finds confirmation in the full title of \ia{Herkel, Jan}Herkel’s book, cited in the introduction above and indicating that the author ‘dug out’ [\textit{eruere}] the “elements” of \il{Slavic}Slavic “from the living \is{Dialect!Living dialect}dialects”.

\ia{Herkel, Jan}Herkel in other words developed a historical-comparative mindset toward the \il{Slavic}Slavic \is{Dialect}dialects in order to lay bare the original \is{Pan-Slavism}Pan-Slavic \is{Genius!Genius of the Slavic language}genius. However, unlike co-eval scholars such as Rasmus Rask and Franz Bopp, who were turning historical-comparative grammar into a separate discipline \citep{swiggers_indo-european_2017}, \ia{Herkel, Jan}Herkel practiced his \is{Pan-Slavism}\il{Slavic!Pan-Slavic}Pan-Slavic grammar not for its own sake, but in service of literature and public uses. In one respect, \ia{Herkel, Jan}Herkel may have been a little ahead of \ia{Rask, Rasmus}Rask and \ia{Bopp, Franz}Bopp, in that he did not identify the original \il{Slavic}Slavic \is{Genius}\textit{genius} with an existing language such as the revered \il{Old Church Slavonic}Old Church Slavonic tongue, whereas Bopp initially strongly considered \il{Sanskrit}Sanskrit a principal candidate for protolanguage in the Indo-European family \citep[125, 155]{swiggers_lelaboration_1996}. Still, \ia{Herkel, Jan}Herkel’s approach was conceptually a much fuzzier one, in that he nowhere defined the \il{Slavic}Slavic \is{Genius}\textit{genius} in detail. Indeed, he did not even situate the Slavic \is{Genius}\textit{genius} within a clear chronology, but conceived the \is{Genius}\textit{genius} as an atemporal \is{Genius}spirit permeating all \il{Slavic}Slavic \is{Dialect}dialects. In laying bare the \is{Pan-Slavism}Pan-Slavic \is{Genius!Genius of the Slavic language}\textit{genius}, \ia{Herkel, Jan}Herkel moreover resorted to praising and rebuking words and forms across different \il{Slavic}Slavic \is{Dialect}dialects, reducing them as much as possible to simpler paradigms, based on the idea that the many different \is{Declension}declensions and \is{Conjugation}conjugations appearing in grammars of individual \il{Slavic}Slavic varieties are unnecessary complications reflecting only superficial variations on the grammatical patterns of the \is{Pan-Slavism}Pan-Slavic \is{Genius!Genius of the Slavic language}\textit{genius}.

The way \ia{Herkel, Jan}Herkel differed from \ia{Rask, Rasmus}Rask and \ia{Bopp, Franz}Bopp can be demonstrated excellently by his discussion of the \il{Slavic}Slavic dual. Although he noticed that “remnants of the dual […] are found in all \is{Dialect}dialects”,\footnote{\citet[36]{herkel_elementa_1826}: “[…] haec sunt manifesta dualis numeri vestigia, quae in omnibus \is{Dialect}dialectis existunt […].”} and that it “is in use among the Carinthians, Carniolans, and Styrians to this day”,\footnote{\citet[49]{herkel_elementa_1826}: “[…] apud Carinthios, Carniolos et Styrios hucdum est in usu […].”} he did not sense that the dual deserved a wide application in his \is{Pan-Slavism}\il{Slavic!Pan-Slavic}Pan-Slavic language form:

\begin{quote}
    since the dual number is currently not distinguished from the plural in the \il{Russian}Russian, \il{Polish}Polish, \il{Bohemian}Bohemian, and southern \is{Dialect}dialects, I judge that one needs to refrain from rigidly introducing it into use, so that, by all means, one makes no mistake when using the dual at a suitable place.\footnote{\citet[49]{herkel_elementa_1826}: “[…] quum vero nunc dualis numerus in \il{Russian}Russica, \il{Polish}Polonica, \il{Bohemian}Bohemica, et meridionalibus \is{Dialect}dialectis a plurali non distinguatur, ab eo stricte in usum inducendo supersedendum duxi, quin tamen erretur tunc, quum loco opportuno ejusdem usus fiat […].”}
\end{quote}

\noindent For this reason, \ia{Herkel, Jan}Herkel typically refrained from offering dual forms. When he discussed the indicative present using the verb \textit{nesu}, ‘to carry, to bring’, for example, he explained that “[t]he dual number is not shown, since no living \is{Dialect!Living dialect}dialect uses it except for \il{Windic}Windic”.\footnote{\citet[125]{herkel_elementa_1826}: “Dualis \il{Latin}numerus non exponitur, quia eo nulla viva \is{Dialect}dialectus praeter \il{Windic}Vindicam utitur.”} Contrary to the historical-comparative Indo-Europeanists, who wanted to arrive at the most ancient form of language and considered the dual to be integral part of it, \citet[27]{herkel_elementa_1826} dismissed the dual as an undesirable complication, despite its old age and the authority of \il{Old Church Slavonic}Old Church Slavonic, which he considered to be “not at all \is{Rationalism}rationally cultivated, as the various texts of the Bible prove”.\footnote{“[…] logice \il{Latin}tamen culta haud fuit, id probatur variis Bibliorum textibus.”}

In sum, whereas \ia{Rask, Rasmus}Rask and \ia{Bopp, Franz}Bopp studied linguistic variation mainly for the sake of knowledge, \ia{Herkel, Jan}Herkel’s intentions were both functional and pragmatic. While not challenging existing political structures or seeking to redraw borders,\footnote{Herkel’s pannationalism was hence not “high political” as defined by Alexander \ia{Maxwell, Alexander}Maxwell but rather focused on “low political” phenomena like language and literature. See \citet{maxwell_pan-nationalism_2022}.} Herkel’s \is{Pan-Slavism}Pan-Slavism was:

\begin{itemize}
    \item{rational, based as it was on logical principles that went against the grammarians’ fictions;}
    \item{literary rather than spoken, resulting in a focus on \is{Orthography}orthography and morphology;}
    \item{invisible but recoverable by \is{Rationalism}rationally comparing \is{Dialect}dialects, which should not be described using frameworks tailored to the classical, Germanic, and Romance languages;}
    \item{and original-idealistic, since \ia{Herkel, Jan}Herkel wanted to arrive at a uniform \il{Slavic!Pan-Slavic}Slavic that reunited the core features of all varieties, not least \il{Old Church Slavonic}Old Church Slavo\-nic, and hence reflected the primal properties of this language.}
\end{itemize}

As such, \ia{Herkel, Jan}Herkel combined \is{Enlightenment}Enlightenment with \is{Romanticism}Romanticism, looking for \is{Rationalism}rational principles to discover the unfathomable \is{Genius!Genius of the Slavic language}genius of the \il{Slavic}Slavic language, implicitly believed to resonate with the \is{Genius}spirit of the various Slavic peoples.\footnote{See e.g. \citet[144]{herkel_elementa_1826}, where \is{Rationalism}rationalism and \is{Romanticism}Romanticism go hand in hand: “In my humble opinion, I would judge that the \il{Bohemian!Polish-Bohemian}Polish-Bohemian form should be adopted, grounded in the \il{Old Church Slavonic}ancient \is{Dialect}dialect, and mixed in with the southerners’ sweetness […].” (Original \il{Latin}Latin: “Tenui opinione mea existimarem \il{Bohemian!Polish-Bohemian}Polono-Bohemicam formam assumendam, in veteri \il{Old Church Slavonic}\is{Dialect}dialecto fundatam, et svavitate meridionalium temperatam.”)} The impossibility of putting one’s finger on the \il{Slavic}Slavic \is{Genius}genius makes it somewhat awkward that \ia{Herkel, Jan}Herkel persistently used it as a \is{Touchstone concept}touchstone concept to measure the appropriateness – or even Slavicness – of specific linguistic forms. Where the term \is{Genius}\textit{genius} appears, it typically indicates that a form is in agreement with the \il{Slavic}Slavic \is{Genius}genius, or is grounded in it, with recourse to the metaphor that the \is{Genius}genius forms the fundament of all \il{Slavic}Slavic tongues. For instance, reflecting on the adjectival \is{Declension!Adjective declension}declension of animate nouns, \ia{Herkel, Jan}Herkel reasoned as follows:

\begin{quote}
    The \is{Inflection!Noun inflection| see {Noun declension}}\is{Declension!Noun declension}inflection \textit{krolovie} is adjectival, which the \is{Genius}genius of the language uses not only for the names of illustrious persons, as the Polish grammarian claims, but also for other masculine nouns, particularly those denoting a substance. That is clear from the \il{Old Church Slavonic}ancient \is{Dialect}dialect, in which is said also \textit{meчove} [‘swords’], \textit{deƶdove} [‘rains’], \textit{kamenove} [‘stones’]; thus \textit{rakove}, just like \textit{krolo\-vie}, conforms to the \is{Genius}genius of the language just as much as \textit{noƶi}, \textit{noƶove}, or \textit{noƶe}. For if only the adjectival \is{Declension!Noun declension}inflection is displayed in the dative singular, why would that \is{Declension!Noun declension}inflection be invalid in the plural?\footnote{\citet[38--39]{herkel_elementa_1826}: “\textit{krolovie} enim est adjectivalis \is{Declension!Noun declension}flexio, quam e \is{Genius}genio linguae, non tantum nomina Personarum honoratiorum, uti Polonus autumat, sed etiam aliorum masculinorum, maxime substantiam aliquam denotantium recipiunt, id patet e \il{Old Church Slavonic}veteri \is{Dialect}dialecto, in qua dicitur et \textit{meчove}, \textit{deƶdove}, \textit{kamenove}; ergo et \textit{rakove} prout et \textit{krolovie} \is{Genius}genio linguae conformiter dicitur prout et \textit{nozi}, \textit{nozove}, or \textit{noƶe}, nam si in dativo singulari tantum adjectivaliter exponitur, cur \is{Declension!Noun declension}inflexio haec in plurali respueretur?”}
\end{quote}

\noindent Less often, the \is{Genius}\textit{genius} of \il{Slavic}Slavic is contrasted with phenomena in individual \is{Dialect}dialects that are presented as later deviations from it, or with the grammarians’ fictions that did not correspond to actual usage, or with both, as in the case of the \il{Slavic}Slavic pluperfect, which, \ia{Herkel, Jan}Herkel argued, did not exist:

\begin{quote}
    The \is{Genius!Genius of the Slavic language}genius of the \il{Slavic}Slavic language rejects the expression of the pluperfect; for this reason there is not any mention of it in any grammars of any \is{Dialect}dialects. Yet some form it from the perfect by adding \textit{byl}, and in \il{Russian}Russian \textit{byvalo} for all genders. Thus \il{Polish}Polish \textit{xvalilem byl}, \il{Bohemian}Bohemian \textit{byl sem xvalil} [‘I had praised’]. But these expressions seem to have come in from the servile imitation of other languages because they are not grounded in the \il{Old Church Slavonic}ancient \is{Dialect}dialect, and also because they are not confirmed by the usage of the \is{Dialect}dialects themselves, unless one would slavishly imitate other languages. For a true Pole never uses it; \il{Russian}Russian \textit{byvalo}, for similar reasons, is only a grammarians’ fiction and does not reflect the \is{Genius}genius of the language.\footnote{\citet[140]{herkel_elementa_1826}: “Genium \is{Genius!Genius of the Slavic language}linguae \il{Slavic}Slavicae respuit expressionem plusquam perfecti, ideo in non nullarum \is{Dialect}dialectorum Grammaticis nec occurrit ejus mentio; aliqui tamen illud formant a perfecto addendo \textit{byl}, et \il{Russian}Russus \textit{byvalo}, pro omni genere. sic: \il{Polish}Polonus: \textit{xvalilem byl}, \il{Bohemian}Bohemus: \textit{Byl sem xvalil}; verum hae expressiones videntur e servili imitatione aliarum linguarum immigrasse, quia nec fundatur in \il{Old Church Slavonic}veteri \is{Dialect}dialecto, sed nec usu ipsissimarum \is{Dialect}dialectorum comprobatur, nisi quis serviliter alias imitetur linguas, originarius enim Polonus, nunquam eo utitur, ideo et \il{Russian}Russicum \textit{byvalo} est tantum Grammaticorum commentum, non vero linguae \is{Genius}genium […].”}
\end{quote}

\noindent \ia{Herkel, Jan}Herkel’s criticism of the grammarians’ fictions is, arguably, striking in that his \is{Pan-Slavism}Pan-Slavic proposal constituted a fiction itself, albeit a fiction transcending those of the grammarians by its alleged reflection of the \il{Slavic}Slavic \is{Genius}genius.

\section{Conclusion}

Although mainly remembered for his coining of the term \is{Pan-Slavism}\textit{Panslavismus}, \ia{Herkel, Jan}Herkel operated with a different keyword throughout his \textit{Elements}, as I have argued in this contribution: \is{Genius}\textit{genius}. Related to the idea of \is{Pan-Slavism}\textit{Panslavismus}, the \is{Genius}\textit{genius} captured the genuine forms the \il{Slavic}Slavic tongues shared. Indeed, in \ia{Herkel, Jan}Herkel’s eyes, everything that belonged to the \is{Genius!Genius of the Slavic language}genius of the \il{Slavic}Slavic language was \is{Pan-Slavism}Pan-Slavic. His supranational approach to Slavic unity despite diversity made his approach not political but literary, in keeping with the \textit{Cultorum Linguae et Literaturae Slavicae Unio} that he co-founded in 1834 and its literary almanac \textit{Zora}, where different \is{Orthography}orthographies were in use. Still, indirectly, a uniform \is{Pan-Slavism}\il{Slavic!Pan-Slavic}Pan-Slavic language would be conducive to political emancipation on the local level in the Hungary of his day and age, where local Slavs were subject to far-going \is{Magyarization}Magyarization (see \ia{Maxwell, Alexander}\hyperref[sec:1.2]{Maxwell in this volume}).

With his insistence on the \il{Slavic}Slavic \is{Genius}\textit{genius}, \ia{Herkel, Jan}Herkel adopted a term from European language studies that had risen to popularity since the sixteenth century and “extended” it to \il{Slavic}Slavic with greater emphasis than any of his predecessors. Along the way, it seems that he restyled the \is{Genius}\textit{genius} concept from “subtle properties of a certain language giving way to serious translation problems” \citep[92]{van_hal_genie_2013} to intrinsic, primal and charming properties of a language that constitute the foundations for the ideal, \is{Rationalism}rational form of that language. The issue of untranslatability that had been central to earlier discussions of the \is{Genius}geniuses of individual languages remained largely under \ia{Herkel, Jan}Herkel’s radar, with the exception of a brief allusion to the \il{Old Church Slavonic}Old Church Slavonic Bible text having a \il{Greek}Greek and hence un-Slavic air. Concomitantly, \ia{Herkel, Jan}Herkel’s \is{Genius}\textit{genius} did not concern solely the subtle properties and intricacies of a language, but its very core and essence, its intrinsic good properties. His interpretation of the \il{Slavic}Slavic \is{Genius}\textit{genius} did, however, have common ground with earlier conceptions in that it is unfathomable. One may approach it through \is{Rationalism}rational comparison of existing diversity, but one can never lay it bare entirely. \ia{Herkel, Jan}Herkel’s \il{Slavic}Slavic \is{Genius}\textit{genius}, then, resulted from fusing \is{Rationalism}Enlightenment rationalism with \is{Romanticism}Romantic sentiment, as he looked for logical principles that corresponded to what he believed to be the innate \is{Genius}spirit, the \is{Genius}genius, of the Slavic peoples and their essentially \is{Unitary Slavic language}unitary, \is{Pan-Slavism}\il{Slavic!Pan-Slavic}Pan-Slavic language.
