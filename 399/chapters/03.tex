\chapter{Elements of a universal Slavic language}

\section*{[Title page / p. 1]}
\addcontentsline{toc}{section}{[Title page / p. 1]}

Elements of a universal \il{Slavic}Slavic language, drawn from the living \is{Dialect!Living dialect}dialects and based on sound logical principles, the author being \ia{Herkel, Jan}Johannes Herkel from Pannonia. \\

\noindent In Buda, with the typeface of the Royal Hungarian University -- 1826.

\section*{[Imprimatur / p. 2]}
\addcontentsline{toc}{section}{[Imprimatur / p. 2]}

\begin{center}
    \textit{Admitted for print.}
\end{center}
\begin{flushright}
    Antonius Tumara signed with his own hand,
    
    censor and reviewer of books.
    
    On the 25\textsuperscript{th} day of February 1826.
\end{flushright}

\section*{Introduction [p. 3--4]}
\addcontentsline{toc}{section}{Introduction [p. 3--4]}

Not only the Slavic nations, but also other cultivated nations have a burning desire for a \is{Orthography!Orthographic unity}uniform script for the \il{Slavic}Slavic tongue, which should act like a suitable key, opening at last the door to this widely extended tongue. No wonder, for the Slavic nations are encouraged by a mutual love which nature has instilled in them, while other peoples are exhorted by the practical advantage of communicating with sixty million Europeans as conveniently as possible.

To this end, the Herculean work of the illustrious \ia{Linde, Samuel Gottlieb}Linde was published under the patronage of the most august monarchs of Great Russia and Prussia.\footnote{\ia{Linde, Samuel Gottlieb}Linde should be identified with the Swedish-German lexicographer of \il{Polish}Polish \ia{Linde, Samuel Gottlieb}Samuel Gottlieb Linde (1771--1847): see \hyperref[sec:3-3]{Section III, §8} below and our note there.} Inspired by this common desire, I have decided to put forward in this booklet some proposals about a common method of writing \il{Slavic}Slavic and \is{Inflection}inflecting its parts of speech, yet in such a manner that I myself also invite men skilled in the philology and etymology of the \il{Slavic}Slavic tongue either to endorse my proposals or to refute them and formulate more suitable proposals. For what is needed is a work perfect in its kind which commends itself both by the ease of its writing and comprehension and by the pleasantness of its expression. For this is not a matter [4] concerning a single \is{Dialect}dialect, but the \il{Slavic}Slavic language taken as a whole, whose genuine principles should preferably not be sought in one but in all \is{Dialect}dialects. Hence it also follows naturally that this language, as the original, should be cultivated by the common effort of the Slavic nations; only in this way, following the example of other nations, will flourish, even in the face of geographic, historic and political diversity, the greatly desired \textit{Union in Literature} among all Slavs, which is the true \is{Pan-Slavism}Pan-Slavism.

\section*{Section I. \textit{On the letters}. [p. 4--12]}
\addcontentsline{toc}{section}{Section I. \textit{On the letters}. [p. 4--12]}

\subsection*{\hspace*{\fill}{§. 1.}\hspace*{\fill}}

The sole impediment to the literature of the Slavic nations was, and remains, the diversity of writing, in terms of both letters and \is{Orthography!Orthographic diversity| see {Orthographic differences}}\is{Orthography!Orthographic differences}orthography; for the enormously extended Russian people, as well as the Serbs, use a double type of writing: church and civil. The \il{Old Church Slavonic}church letters are, to be sure, nothing else than the letters which \ia{Cyril}Cyril and \ia{Methodius}Methodius, the Apostles of the Slavs, took from the Greeks. They fashioned these same letters to express the sounds proper to the people -- or rather they had already anticipated them. For these men were philosophers; hence they acted in agreement with reason and devised proper letters in order to render the sounds proper to the people. And most certainly, if the Cyrillic letters indeed had been capable of great fluency, [5] no Slavic people would have ever adopted any others, given the enormous power they possess to express \il{Slavic}Slavic sounds. Yet because they lack both splendor in their external form and a swiftness in writing, most peoples gradually stopped using them, and even the Russians, under Peter the Great, introduced more practical letters for civil life. Most of these letters agree with their European counterparts in form but disagree in sound.

\subsection*{\hspace*{\fill}§. 2.\hspace*{\fill}}

In fact, the other Slavic nations use Latin letters, now called “European”, which also the \il{Bohemian}Bohemians and the \il{Pannonian}Pannonians have generally adopted, after abandoning the Gothic-Latin script. Yet even though they use the same letters, there is still such a great \is{Orthography!Orthographic differences}orthographic diversity in writing that texts written in the various \is{Dialect}dialects are barely understood. But also within individual \is{Dialect}dialects there is such a great variety in writing that even men from the same diocese employ three writing systems (or rather no system). The cause for this state of affairs is that the European alphabet lacks sufficient characters to express \il{Slavic}Slavic sounds, so it happens that people variously change the sound of the letters, overload them with diacritics, or laboriously wrench a typical \il{Slavic}Slavic sound from a conglomerate of several letters. A saddening example of this reality is offered by books composed in various \is{Dialect}dialects. For this reason, let us take [6] \ia{Cyril}Cyril as our guide in these matters, and adapt his argument to our own times as follows.

\subsection*{\hspace*{\fill}§. 3.\hspace*{\fill}}

A letter is nothing else than a symbol for a sound uttered by the human mouth. Hence, as many sounds there are in speech, exactly so many symbols have to be present, symbols which are called “letters”. The more distinctly the letters express the sounds of the language, the more perfect they will be. Thus it arises that one letter should never be confused with another, because then we would also confuse the sounds from which human speech is composed. For this reason, a letter should always retain the sound given to it. It is a poor way of writing if the sound of a letter depends on this or that word, on this or that vowel or consonant following or preceding it.

\subsection*{\hspace*{\fill}§. 4.\hspace*{\fill}}

Each language normally has, apart from the sounds it shares with other languages, also sounds that are peculiar to itself. That is what experience teaches us. By consequence, one should use common letters for common sounds, and likewise particular letters for particular sounds. Certainly, grammarians would indeed create great confusion if they adopted the letters of a certain people without providing particular letters for the sounds particular to their own language, instead making a great effort to find various subterfuges to express them. [7]

\subsection*{\hspace*{\fill}§. 5.\hspace*{\fill}}

The simpler, the more pleasant, the easier letters are to write, the more they are recommended for public use. For a pleasant script or typeface seduces even reluctant persons to read, since by our very nature we take delight not in ugliness but in beauty. Hence it is easy to judge: should we use the Cyrillic, Gothic-Latin, or indeed the more polished European letters?

\subsection*{\hspace*{\fill}§. 6.\hspace*{\fill}}

The European letters seem to be the most adequate for public usage, for the Gothic-Latin letters are disfigured by superfluous angles. The Cyrillic letters, in turn, have long been judged wholly inappropriate for common usage even by our brothers of the Eastern rite, not just by other Slavs. All these elements weigh in favor of adopting the more cultivated European letters, because, firstly, they enjoy the required qualities more than the other scripts, secondly because most Slavic nations are already using them, and finally because the whole of cultivated Europe has adopted these very same letters. However, these letters do not suffice to express all sounds of the \il{Slavic}Slavic tongue. Let us therefore either imitate Cyril and create new letters, or if any letters exist in particular \is{Dialect}dialects, let us adopt them, and if they are unpolished, let us polish them.

\subsection*{\hspace*{\fill}§. 7.\hspace*{\fill}}

The letters of the Europeans are the following: a, b, c, d, e, f, g, h, i, j, k, l, m, n, o, p, [8] q, r, s, t, u, v, x, y, z. Their number is 25, among which q = k, x = ks, and y = i have the same sound. So if you take away the superfluous letters, 22 remain. But these letters cannot express all \il{Slavic}Slavic sounds, as, for example:
\begin{enumerate}
    \item {\il{Russian}Russian ч is to the \il{Bohemian}Bohemian \textit{č}, the \il{German}German \textit{tsch}, the \il{Croatian}Croatian \textit{ch};}
    \item {\il{Russian}Russian ж is to the \il{Bohemian}Bohemian \textit{ž};}
    \item {\il{Russian}Russian ш is to the \il{Bohemian}Bohemian \textit{š}.}
    \item {Soft sounds are lacking in some \il{Slavic}Slavic \is{Dialect}dialects.}
    \item {Finally, the pleasant sound known to both the \il{Old Church Slavonic}ancient Slavs and to the \il{Russian}Russians and the \il{Windic}Winds which the \il{German}Germans express by means of \textit{ü}, but which the \il{Old Church Slavonic}ancient Slavs expressed by means of ы.}
\end{enumerate}

\subsection*{\hspace*{\fill}§. 8.\hspace*{\fill}}
\label{sec:3-1-8}

\indent \il{Russian}Russian ч is the most appropriate, since it is close to the European letters and as a letter is simple enough. Admittedly, the \il{Bohemian}Bohemian \textit{č} would be even more appropriate to express the essence of the sound because of the affinity between \textit{c} and \textit{č}, but the mark attached above deforms both writing and typeface, which is not the case with its Russian equivalent.

2. \il{Russian}Russian ж differs from the European letters, but \il{Bohemian}Bohemian \textit{ž} is European, especially if the mark is positioned at the middle: \textit{ƶ}.

3. \il{Russian}Russian ш (= \il{German}German \textit{sch}, \il{Bohemian}Bohemian \textit{š}) is too different from the form of the letter \textit{s}. In truth, \il{Windic}Windic 8 is nothing but a double S formed by means of one ductus.\footnote{The \il{Windic}Windic letter \ia{Herkel, Jan}Herkel here has in mind presumably comes from Peter \citeauthor{dainko_lehrbuch_1824}’s \citeyear{dainko_lehrbuch_1824} \textit{Lehrbuch der windischen Sprache}, where it refers to the sound [ʃ] (\citeyear[2]{dainko_lehrbuch_1824}). \ia{Dainko, Peter}Dainko’s letter looks like <8>. \ia{Dainko, Peter}Dainko’s letter as it appears in \ia{Herkel, Jan}Herkel’s typeface actually resembles <ȣ>. \ia{Herkel, Jan}Herkel also used <ȣ> to describe the letter <S> which, when doubled, produces <ȣ>. We surmised \ia{Herkel, Jan}Herkel’s meaning from context, drawing inspiration from \ia{Herkel, Jan}Herkel’s instruction that a benevolent reader will easily correct mistakes.} We would have no reason to scorn its use, save that the authority of the \il{Russian}Russian ш prevails. [9]

The letter \textit{x} = \textit{ks} is superfluous for us, since \textit{ks} performs its role. The \il{Dalmatian}Dalmatians use it instead of \textit{ƶ}, but the \il{Russian}Russians in place of \il{Bohemian}Bohemian \textit{ch}; it seems that at least in this case the \il{Russian}Russians should be imitated. Hence, the entire Slavic nation will equally write as well as say: \textit{žena} [‘woman’], \textit{duшa} [‘soul’], \textit{dux} [‘spirit’], \textit{чlovek} [‘person, human being’].

\subsection*{\hspace*{\fill}§. 9.\hspace*{\fill}}

Some \is{Dialect}dialects are used to softening the following letters: \textit{d}, \textit{l}, \textit{n}, \textit{t}, but this habit of softening smells merely like \is{Provincialism}provincialism in some \is{Dialect}dialects, to such an extent that in Pannonia itself the populace sometimes pronounces these letters soft before \textit{e}, \textit{i}, and other times hard. Hence, in order to designate these soft sounds, peculiar letters do not seem to be necessary at all. A \is{Dialect!Softening dialect}softening dialect will soften e.g. \textit{Nebo} [‘sky, heaven’], \textit{Niva} [‘field’] even without any symbol, but a \is{Dialect}dialect that does not soften would be brought into confusion. This habit of softening in some regions of Pannonia has developed to the point that the letters \textit{t} and \textit{d} have been entirely turned into the letter and the sound of the letter \textit{c}. This habit of speaking and writing is common to the \il{Polish}Poles. For instance, in original \il{Slavic}Slavic fashion the following should be written: \textit{napelniti} [‘to fill’], \textit{pokoiti} [‘to rest’], \textit{idjem} [‘I am going’], those who soften \textit{t} and \textit{d} mark them as \textit{$\tilde{t}$}, \textit{ď},\footnote{While in modern \is{Orthography}orthography one would expect the \textit{ť}, in \ia{Herkel, Jan}Herkel’s text the diacritics on the \textit{t} and \textit{d} are not the same.} but the \il{Polish}Poles write \textit{pokoic}, \textit{idzem} = this sounds like \textit{icem}.

\subsection*{\hspace*{\fill}§. 10.\hspace*{\fill}}

The \il{Bohemian}Bohemians use a triple \textit{i}, namely \textit{i}, \textit{j}, \textit{y}, all of which nonetheless always retain the sound \textit{i}, yet with [10] the following \is{Orthography}orthographic distinction: \textit{y} is employed after certain consonants, but they employ \textit{j} to soften a preceding consonant or to lengthen a syllable. However, the native \il{Bohemian}Bohemians themselves do not care much about this subtle distinction. It would certainly be more satisfying if \textit{j} were employed instead of \textit{g}. Then the excellent \il{Bohemian}Bohemian writers would be more easily understood by most Slavs. One should therefore write \textit{javor} [‘maple’], \textit{jagoda} [‘strawberry’], but not \textit{gavor}, \textit{gahoda}, and \textit{stoji} [‘he stands’] instead of \textit{stogi}.

But let us assign to the letter \textit{y} that sound, common among the \il{Windic}Winds, which approximates the \il{German}German \textit{ü}. This sound was known to the \il{Old Church Slavonic}ancient Slavs, the \il{Russian}Russians, and the \il{Windic}Winds. The other nations supply this sound only through \textit{i}, but the \il{Old Church Slavonic}ancient Slavs accurately distinguished \textit{i} from \textit{y} both in sound and in writing, such as \textit{vlk vyje}, ‘the wolf howls’, -- similarly \textit{dievica vienec vije} [‘the girl wreathes a wreath’], ‘I howl’ is written as ВЫШ, ‘I bind’, however, as ВИШ.\footnote{The capital \textit{Ш} is printed upside down here and elsewhere in \ia{Herkel, Jan}Herkel’s text. Only in the \il{Russian}Russian sample alphabet at the end of the text is the capital \textit{Ш} printed correctly. Because the lowercase ш is always printed correctly, we suspect a typesetter’s eccentricity.}

\subsection*{\hspace*{\fill}§. 11.\hspace*{\fill}}

The \il{Bohemian}Bohemians and \il{Polish}Poles usually pronounce the letter \textit{r}, especially before \textit{e} and \textit{i}, in a way not found among other Slavic nations. This sound used to be written as \textit{rz} even among the \il{Bohemian}Bohemians. So instead of \textit{Zverina} [‘wild game’], the \il{Polish}Pole writes \textit{Zverzina}, the \il{Bohemian}Bohemian drops the \textit{z} and notes the \textit{r} with a diacritic = \textit{Zveřina}. -- In the meantime, the diacritic itself above the letter \textit{r}, too, is superfluous, since a native \il{Bohemian}Bohemian or \il{Polish}Pole will read the letter \textit{r} before \textit{e} and \textit{i} only in the manner natural to him. It seems this nasality is related to [11] the old yer, Ъ, which in \il{Old Church Slavonic}ancient books used to be annexed to the consonants, and especially also to the letter \textit{r}.

\subsection*{\hspace*{\fill}§. 12.\hspace*{\fill}}

Thus we have the following letters, distinct in form and sound: a, b, c, ч, d, e, f, g, h, x, i, j, k, l, m, n, o, p, r, s, ш (\textit{š}), t, u, v, z, ƶ (ж) -- 27 in total, by means of which every authentic \il{Slavic}Slavic sound is very aptly expressed.

But perhaps some will say that there are more letters present than would be right, whereas others will maintain that there are fewer letters than necessary. The southerners will perhaps not be pleased by the letter \textit{x} instead of \il{Polish}Polish \textit{ch}, because for some the letter \textit{h} supplements this sound. They don’t say \textit{xudi} [‘poor’], \textit{xvalim} [‘please’], \textit{xram} [‘temple’] etc. deeply out of their throat, but only \textit{hudi}, \textit{hvalim}, \textit{hram}. What is more, some even change \textit{h} into \textit{f}, thus \textit{xvala} = \textit{hvala} = \textit{fala}. In addition, the \il{Dalmatian}Dalmatians have given the letter \textit{x} our sound \textit{ƶ}. Again, how great is the resulting diversity of writing, reading, and meaning! So the \il{Russian}Russian writes and reads \textit{xlieb} [‘bread’]. The \il{Dalmatian}Dalmatian reads this as \textit{ƶlieb}, which has a very different meaning [‘groove, gutter’]. Only agreement therefore will be able to establish a more extensive literature and culture, for both language and nation. In the meantime the \il{Polish}Poles can rightly complain about the neglect of their sweet sound \textit{dz}; but let it be supplanted with one single letter, namely with \textit{ꝺ}. This way, they will have a simple letter, but also it will facilitate reading among others. Among the \il{Polish}Poles, for instance, the \textit{d} in the word \textit{dielo}, coming before \textit{e} or \textit{i}, sounds almost like \textit{c}, and they show that by writing the letter \textit{z}, thus \textit{dzielo}. It would be more convenient to write \textit{ꝺielo}. [12]

But the question of what should be done with \textit{rz}, \textit{ą}, \textit{ę} is left for the \il{Polish}Poles themselves to decide. The \il{Bohemian}Bohemians, as I’ve said, have already eliminated the letter \textit{z} from \textit{r}. What is more, in recent times, they have also come to dislike the diacritic on \textit{ř} itself. Certainly, we don’t despair that the \il{Polish}Poles will also take this step. But it would be better to write \textit{ą} and \textit{ę} as they are pronounced, as follows: \textit{kvitnąl} [‘bloomed’], \textit{vziąl} [‘took’] = \textit{kvitnol}, \textit{vziol}. It is in any case not up to us to invent new sounds but rather to preserve the sounds that are fixed in the \is{Genius!Genius of the Slavic language}genius of the \il{Slavic}Slavic language. The objective is a common method of writing. Surely, more will be achieved with united strength, for when dispersed, or in plain opposition, they will sooner or later be extinguished.

\subsection*{\hspace*{\fill}§. 13.\hspace*{\fill}}

Where other \is{Dialect}dialects use the \textit{g}, the \il{Bohemian}Bohemians, \il{Moravian}Moravians, \il{Pannonian}Pannonians use the letter \textit{h}. So instead of \textit{grad} [‘city’], \textit{gruda} [‘heap, lump, clod’], \textit{griada} [‘garden bed, shaft’] they say and write = \textit{hrad}, \textit{hruda}, \textit{hriada} etc. On this account, since \textit{g} instead of \textit{h} is common to most nations as well as the \il{Old Church Slavonic}ancient Slavic, I for one see no reason why the \il{Bohemian}Bohemians and \il{Pannonian}Pannonians shouldn’t use this \textit{g} in the common kind of writing; among those who are concerned about the culture of their mother tongue, one should therefore write \textit{glava} [‘head’], \textit{jeden} [‘one’], but not \textit{hlava}, \textit{geden}; for in this way excellent \il{Bohemian}Bohemian books will be very easily understood by other fellow nationals. [13]

\section*{Section II. \textit{On diacritics}. [p. 13--16]}
\addcontentsline{toc}{section}{Section II. \textit{On diacritics}. [p. 13--16]}

\subsection*{\hspace*{\fill}§. 1.\hspace*{\fill}}

One should properly distinguish punctuation marks from diacritics. Without punctuation marks anything written, apart from very brief texts, would dissolve into disorder, since readers could take any meaning they please. Punctuation marks determine the meaning of speech put down on paper. Diacritics, however, are symbols attached to letters to prolong or shorten the vowel of a word etc. So what about these diacritics?

The origin of diacritics should be sought in the most ancient writing system, in which vowels were omitted and supplanted only by points. The following arguments seem to favor the preservation of diacritics. Several nations use them to more accurately depict how vowels are spoken. Additionally, diacritics can distinguish the meaning of ambiguous words. There are no other arguments in defense of diacritics.

It is important to consider how much weight to place on these arguments. Only one thing follows from the fact that many nations place diacritics above letters: those nations are either imitating the ancient way of writing, or are forced to use them because of some deficiency of their letters. If a language had enough letters, why would the typeface or script be deformed with useless marks? [14]

\subsection*{\hspace*{\fill}§. 2.\hspace*{\fill}}

By means of diacritics, a word acquires its due sound, but the \il{Slavic}Slavic language in its entirety has absolutely no need to use them, since it is already provided with sufficient letters. Let us take, for instance, the word \textit{ƶena} [‘woman’]. In the \il{Old Church Slavonic}ancient \is{Dialect}dialect, and most other \is{Dialect}dialects, it is written without any diacritic. Some \il{Russian}Russian grammarians have now burdened it with a diacritic: \textit{ƶenà}. The Serb also adds a diacritic, but not where the \il{Russian}Russian puts it: he writes \textit{ƶèna}, using a diacritic which is doubled in the vocative: \textit{ƶëna}.\footnote{The correct form would be \textit{ƶëno}.} Now on what grounds has the grammarian doubled this diacritic in the vocative? Perhaps because when someone is addressed, the voice of the person addressing him changes? But when someone is angry, is fawning, or shows any other emotion, the sound of the voice also changes. What a great crowd of diacritics would then have to be introduced!

\subsection*{\hspace*{\fill}§. 3.\hspace*{\fill}}

Then, by means of diacritics a word acquires its due sound; this observation would remain invalid even if all Slavic peoples lengthened or shortened every word equally; for then, on that very account, diacritics would be superfluous. Furthermore, the Slavic peoples show a great diversity of vowel lengthening and shortening. Indeed, I have heard a \il{Windic}Wind say the following: \textit{Nógí má bóléjó} [‘my feet/legs hurt’]. He lengthened [15] every syllable equally. However, such lengthening is not universal among all \il{Windic}Winds, for experience shows diverse prosody even among speakers of the same \is{Dialect}dialect. The reason for this is that the protraction or shortening of syllables is not grounded in the \is{Genius!Genius of the Slavic language}genius of the \il{Slavic}Slavic language, but in the manifold habits of speaking.

\subsection*{\hspace*{\fill}§. 4.\hspace*{\fill}}

Finally -- ambiguous words can be distinguished only if the diacritics are added. Yet ambiguous words are not distinguished by adding a mass of extra symbols, but rather by the context of conversation. The word \textit{vije} has very different meaning in \textit{Vlk vije}, ‘the wolf howls’, and \textit{dievica}, (\textit{panna}) \textit{vienec vije}, ‘the maiden wreathes a wreath’, but the meaning is absolutely clear from the context. The \il{Bohemian}Bohemians do indeed distinguish these meanings with their \is{Orthography}orthographic system, writing ‘to howl’ \textit{vyti} with \textit{y}, and \textit{vije} ‘wreathes’ with \textit{i}, but this distinct method of writing is right and correct only if it also produces a distinct sound. Otherwise, useless subtleties understood only by their fabricators will arise. The \il{Old Church Slavonic}ancient \is{Dialect}dialect and the \il{Russian}Russians write ‘I howl’ as ВЫШ, but ‘to wreathe’ as ВИШ, the different method of writing indicates a different sound, and indeed Ы = \textit{y} = \il{German}German \textit{ü}, [16] \il{French}French \textit{eu}; and thus the sound of \textit{vyti} is different from that in \textit{viti}. These arguments demonstrate that using diacritics in the \il{Slavic}Slavic language would just be multiplying entities beyond necessity,\footnote{\ia{Herkel, Jan}Herkel here refers to Ockham’s razor: \textit{Entia non sunt multiplicanda praeter necessitatem}.} slowing down the writer, troubling the reader, and spoiling handwriting and typefaces. So if there are sufficient letters to express the genuine \il{Slavic}Slavic sound, it follows that diacritics should be left to those peoples who are forced to use them because of the deficiency of their letters.

\section*{Section III. \textit{On the cultivation of language in general, then in particular}. [p. 16--25]}
\addcontentsline{toc}{section}{Section III. \textit{On the cultivation of language in general, then in particular}. [p. 16--25]}
\label{sec:3-3}

\subsection*{\hspace*{\fill}§. 1.\hspace*{\fill}}

With a uniform writing system, all Slavic peoples can engage in literary interaction among themselves. They understand one another very easily, for there is not such a great distinction between them as one observes among the \il{Italian}Italian or \il{German}German \is{Dialect}dialects. The Pannonian Slav in the Carpathian Mountains speaks as easily with the Bohemian and the \il{Polish}Pole as with his brother, and with the Russian as with his neighbor. This was abundantly shown by the passage through these regions of the Russian army, but anyway the \is{Dialect}dialect of some \il{Pannonian}Pannonians differs little from \il{Russian}Russian. They can likewise understand southern Slavs, perhaps with greater difficulty, but eight days of [17] conversation with them removes all difficulty (as I can attest from my own experience in Croatia), especially when both parties would abstain from foreign words. Now if oral communication is possible, a common writing system could much more easily be established. To that end, a uniform script is absolutely necessary, both in terms of letters and spelling. Without uniformity we have nothing, but with it, it will be clear that all \il{Slavic}Slavic \is{Dialect}dialects are but one single language. In particular, if foreign words are noted in \is{Dialect}dialect dictionaries and replaced by genuine words from another \is{Dialect}dialect, then variation will cease spontaneously, and there will be, as there once was, a single language. And if a philosophical grammar would accompany it, this original language, alive among numerous peoples of Europe, is bound to be very useful and flourish greatly.

\subsection*{\hspace*{\fill}§. 2.\hspace*{\fill}}

No \il{Slavic}Slavic \is{Dialect}dialect, viewed in isolation from the others, can reasonably serve as the common literary \il{Slavic}Slavic language. Firstly, each \is{Dialect}dialect currently abounds in foreign words, even though indigenous expressions are present in the other \is{Dialect}dialects. Secondly, the individual \is{Dialect}dialects lack thoughtful principles of writing. The reason is that the nations speaking individual \is{Dialect}dialects to a greater or lesser extent mixed with other peoples, and that mixing has greatly influenced the language itself. Hence it follows that [18] the \is{Genius!Genius of the Slavic language}genius of the original \il{Slavic}Slavic language does not consist of, and is not grounded on, any one \is{Dialect}dialect, but all of them. Thus not only \il{Old Church Slavonic}Church Slavonic is relevant, but also \il{Russian}Russian, \il{Polish}Polish, \il{Bohemian}Bohemian, \il{Pannonian}Pannonian, \il{Illyrian}Illyrian, and \il{Windic}Windic, together with their \is{Dialect!Subdialect}subdialects.

\subsection*{\hspace*{\fill}§. 3.\hspace*{\fill}}

If one particular \is{Dialect}dialect were taken as the basis for a common \il{Slavic}Slavic, it would first have to be purged of foreign expressions, and then its method of \is{Inflection}inflecting compared with that of the other \is{Dialect}dialects. Whatever \is{Dialect}dialect were chosen, it would be appropriate to consider the authentic words in each \is{Dialect}dialect and the original pattern of \is{Inflection}inflection, since otherwise a hodge-podge of useless rules and exceptions will arise. The following example will illustrate why. The \il{Bohemian}Bohemian \is{Dialect}dialect currently \is{Declension!Noun declension}inflects nouns of the neuter gender in various ways.\footnote{\citet[table insert at 234--235]{dobrovsky_ausfuhrliches_1809}.} For instance:

\newpage

\begin{longtable}{ l l l l }
    \lsptoprule
    \multicolumn{4}{ c }{Singular.} \\
    \midrule
    \textit{Nom}. \textit{Ac}. \textit{V}. & \textit{Pole} & \textit{Slovo} & \textit{Znameni} \\
    &  [‘field’], & [‘word’], & [‘sign’]. \\
    \textit{G}. & \textit{Pole}, & \textit{Slova}, & \textit{Znameni}. \\
    \textit{D}. & \textit{Poli}, & \textit{Slovu}, & \textit{Znameni}. \\
    \textit{Locative}. & \textit{v Poli}, & \textit{Slove}, & \textit{Znameni}. \\
    \textit{Instrumental}. & \textit{Polem}, & \textit{Slovem}, & \textit{Znamenim}. \\
    \lspbottomrule
    \\
    \lsptoprule
    \multicolumn{4}{ c }{Plural.} \\
    \midrule
    \textit{Nom}. \textit{A}. \textit{V}. & \textit{Pole}, & \textit{Slova}, & \textit{Znameni}. \\
    \textit{G}. & \textit{Poli}, & \textit{Slov}, & \textit{Znameni}. \\
    \textit{D}. & \textit{Polim}, & \textit{Slovum}, & \textit{Znamenim}. \\
    \textit{L}. & \textit{v Polix}, & \textit{Slovix}, & \textit{Znamenix}. \\
    \textit{I}. & \textit{Poli}, & \textit{Slovi}, & \textit{Znamenimi}. \\
    \lspbottomrule
\end{longtable}

[19] Here every word has a distinct \is{Declension}declension. The \il{Polish}Pole declines the very same words, but the paradigm follows only one norm of \is{Declension!Noun declension}inflection, as shown below.\footnote{\citet[122--123]{bandtkie_polnische_1808}. The \is{Declension!Noun declension}declension \ia{Herkel, Jan}Herkel ascribes to \textit{znamenie} ‘doctrine’ \ia{Bandtkie, Jerzy Samuel}Bandtkie actually gives for \textit{kazanie} ‘sermon’. \ia{Herkel, Jan}Herkel is obviously offering \textit{znamenie} for \il{Polish}Polish for the sake of parallelism with his earlier \il{Czech}Czech declension table.}

\begin{longtable}{ l l l l }
    \lsptoprule
    \multicolumn{4}{ c }{Singular.} \\
    \midrule
    \textit{N}. \textit{A}. \textit{Vo}. & \textit{Pole}, & \textit{Slovo}, & \textit{Znamenie}. \\
    \textit{G}. & \textit{Pola}, & \textit{Slova}, & \textit{Znamenia}. \\
    \textit{D}. & \textit{Polu}, & \textit{Slovu}, & \textit{Znameniu}. \\
    \textit{L}. & \textit{v Polu}, & \textit{Slovie}, & \textit{Znameniu}. \\
    \textit{I}. & \textit{Polem}, & \textit{Slovem}, & \textit{Zameniem}. \\
    \lspbottomrule
    \newpage
    \lsptoprule
    \multicolumn{4}{ c }{Plural.} \\
    \midrule
    \textit{Nom}. \textit{A}. \textit{Vo}. & \textit{Pola}, & \textit{Slova}, & \textit{Znamenia}. \\
    \textit{G}. & \textit{Pol}, & \textit{Slov}, & \textit{Znamien}. \\
    \textit{D}. & \textit{Polom}, & \textit{Slovom}, & \textit{Znameniom}. \\
    \textit{L}. & \textit{v Polax}, & \textit{Slovax}, & \textit{Znameniax}. \\
    \textit{I}. & \textit{Polami}, & \textit{Slovami}, & \textit{Znameniami}. \\
    \lspbottomrule
\end{longtable}

Though the \il{Polish}Pole here follows only one single \is{Declension!Noun declension}declension, he makes an exception for the word \textit{Slovo} in the locative singular, to which he appends not \textit{u}, as in the other cases, but \textit{e}. But if we would then consult the other \is{Dialect}dialects, both northern and southern, we find in everyday life the ending of the locative of the word \textit{slovo} as follows: \textit{v Boƶim slovu, v mojim slovu} etc. [‘in God’s word, in my word’]. Thus it follows that only one logical form of the neuter \is{Declension!Noun declension}inflections can be established, from which it is clear that combining the \is{Dialect}dialects is absolutely indispensable for cultivating the \il{Slavic}Slavic language. [20]

\subsection*{\hspace*{\fill}§. 4.\hspace*{\fill}}

But what do we mean by “cultivating the language”? The cultivation of language can be sometimes civic, sometimes rational. The civic cultivation of language occurs when the use of a language prevails in a civil society. The more circumstances in which a language is used, addressing more diverse or even all possible subjects, the greater the civic cultivation of that language. Hence it follows that the larger a nation with the same language and civil society may be, the greater the civic cultivation of language within it, for we suppose the affairs of a great nation will also be great. And if these affairs are conducted in the mother tongue, the language of the nation is cultivated and amplified in the military, in the home, outdoors, in training, at work, and in thousands of other activities. Hence it is clear that the smaller a nation is and the more insignificant its affairs, the less national language will undergo civic cultivation. The civic cultivation of language should therefore not be sought in the fancies of men of letters but in public usage. A man of letters devises words in vain if they do not find practical use, but diverse activities quickly generate terminology and put it to use. Languages are therefore enriched above all by the diversity of citizens’ occupations, and for this reason it becomes clear that the civic cultivation of language is to be measured by the number of citizens and their occupations.

\subsection*{\hspace*{\fill}§. 5.\hspace*{\fill}}

From this perspective, if the \il{Slavic}Slavic language is considered across its entire expanse, as a language [21] extending widely across eastern Europe and northern Asia: observe that many peoples of Europe speak it; the sacred Eastern rite is conducted in it, as is that of the Western Church, and of the Reformed rite. It is the language of the military, of trade, and thousands of other occupations. Yet consider also that this language is sundered into various \is{Dialect}dialects, so its civic cultivation depends on the situation of the peoples speaking it. For in one place it is the administrative language, at another the language of home or the family, at still another merely the language of servants, and the civic cultivation of the language is determined by this foundation.

\subsection*{\hspace*{\fill}§. 6.\hspace*{\fill}}

But what is meant by “logically cultivated language”? What is meant by “logically cultivated language” is abundantly indicated by the expression itself: a language should have firm principles, clear rules of writing, and not be crippled by a multitude of exceptions. The fewer exceptions to the rules, the firmer those rules will be. Hence a language absolutely does not deserve to be called “logically cultivated” if its grammar book is as long as or even longer than its dictionary. Such a language certainly lacked logical cultivation right at the start, thus it is wrong to believe that the cultivation of language resides in the idle subtlety of grammarians. The following example will confirm this truth. The \il{Pannonian}Pannonians express the adjective ‘long’ by means of the word \textit{dluhi}, but they form the comparative [22] by adding -\textit{шi} instead of -\textit{i}, as in \textit{xudi xudшi} [‘poor, poorer’], so also \textit{dluhi}, \textit{dluhшi} etc. But here already \is{Dialect}dialect grammarians form an exception and advocate \textit{dlukшi} or \textit{dluƶшi}, but where does this exception come from? Whence have \textit{ƶ} and \textit{k} intruded? It is only from the \is{Dialect}dialect, or rather the \is{Dialect!Subdialect}subdialect: for \il{Pannonian}Pannonians do not always say \textit{dluƶшi} or \textit{dlukшi}, they also say \textit{dluhшi} -- this is the same. But logic itself does not acknowledge this exception, since the regular pronunciation is established above all through usage itself. But \textit{dluhшi} is actually said, for \textit{ƶ} has crept in not so much by usage as by misuse; for \textit{dlukшi} is nothing else than a corruption of the original way of writing. For the \il{Old Church Slavonic}original Slavic was not \textit{dluhi} but \textit{dlugi}, and hence \textit{dlugшi}, not \textit{dlukшi}. There are thousands of exceptions of this type which constitute a true labyrinth for memory: for \textit{kniha} or \textit{kniga} [‘book’], for instance, the \is{Dialect}dialect experts give dative \textit{knize} instead of \textit{knihe} or \textit{knige} as in the other \is{Dialect}dialects etc. For all \is{Dialect}dialects are more or less burdened by various exceptions which emerged from diverging usage. Thus the cultivation of the \il{Slavic}Slavic language needs logical combination, and then rules will emerge that are firm, plain, clear, and beneficial for both Slavs and foreigners wanting to learn this language.

\subsection*{\hspace*{\fill}§. 7.\hspace*{\fill}}

For the logical cultivation and union of the \il{Slavic}Slavic \is{Dialect}dialects there are means available, for recently very erudite men [23] have accurately composed grammars of almost every \is{Dialect}dialect. And from these it is clear how much the \is{Dialect}dialects diverge from one another, or rather how much they vary, which is why the \il{Slavic}Slavic language does not need any more similar grammars. If the smaller Slavic nations cultivate their \is{Dialect}dialect grammatically with no regard for their relation to the other \is{Dialect}dialects, the growth of Slavic literature will be stunted. For the \il{Slavic}Slavic language is divided into various \is{Dialect}dialects, and their separate writing conventions also separate the language itself. For this reason grammars should be made which introduce a greater range of their language; for while civic cultivation does not flower among the \is{Dialect}dialects of smaller Slavic peoples, it is flourishing for the \il{Slavic}Slavic language viewed in its full extent. \is{Dialect}Dialect experts should therefore work to direct the various \is{Dialect}dialects like rivulets that will flow into one great stream.

\subsection*{\hspace*{\fill}§. 8.\hspace*{\fill}}

To this end, the very erudite \ia{Linde, Samuel Gottlieb}\textit{Samuel Bogomil Linde},\footnote{The \il{Slavic}Slavic alias of \ia{Linde, Samuel Gottlieb}Samuel Gottlieb Linde (1771--1847), who was born in Prussia to a Swedish father and a German mother. He was the author of a six-volume \il{Polish}Polish dictionary (\textit{Słownik języka polskiego}, \citeyear{linde_slownik_18071814}) that compared \il{Polish}Polish words to their counterparts in other \il{Slavic}Slavic languages.} rector of the Warsaw Lyce\-um, published a dictionary of all \il{Slavic}Slavic \is{Dialect}dialects in the \il{Polish}Polish \is{Dialect}dialect saying: if the \il{Italian}Italians, who are so very diverse in terms of \is{Dialect}dialect, have boasted a uniform written language since \ia{Dante Alighieri}Dante’s times, why should the Slavs not enjoy the same? Indeed, the illustrious \ia{Bandtkie, Jerzy Samuel}\textit{Samuel Bandtke},\footnote{Jerzy Samuel Bandtkie or Georg Samuel Bandtke (1768--1835) was a Polish philologist from Lublin. \ia{Herkel, Jan}Herkel quotes his \il{Polish}\textit{Polnische Grammatik für Deutsche} (\citeyear{bandtkie_polnische_1824}). We refer to this scholar as \ia{Bandtkie, Jerzy Samuel}Bandtkie, as is common in secondary literature, but preserve in our translation \ia{Herkel, Jan}Herkel’s form.} professor in Cracow, says in his work: there is no doubt that the \il{Slavic}Slavic \is{Dialect}dialects can be united, and that a common way of writing can be introduced. If the German nations, which are more diverse in \is{Dialect}dialects, have accomplished this, what impedes the Slavs? [24]

\subsection*{\hspace*{\fill}§. 9.\hspace*{\fill}}

For this goal, a common writing system is necessary, in terms of both letters and \is{Orthography}orthography, for instance: \textit{Ad} [‘hell’], \textit{Adám} [‘Adam’], \textit{Bog} [‘God’], \textit{brada} [‘beard’], \textit{brana} [‘gate’], \textit{cerkva} [‘church’], \textit{чast} [‘part’], \textit{чerv} [‘worm’], \textit{dvor} \linebreak{} [‘court’], \textit{den} [‘day’], \textit{deшt} [‘rain’], \textit{grad} [‘city’], \textit{grib} [‘mushroom’], \textit{jama} [‘pit’], \textit{jesen} [‘autumn’], \textit{izba} [‘cottage’], \textit{xram} [‘temple], \textit{xrom} [‘thunder’], \textit{kon} [‘horse’], \textit{kov}, \textit{kova} [‘metal’], \textit{libost} [‘pleasure’], \textit{liud} [‘people’], \textit{meч} [‘sword’], \textit{maso} \linebreak{}[‘meat’], \textit{mladost} [‘youth’], \textit{mleko} [‘milk’], \textit{nebo} [‘sky, heaven’], \textit{niva} [‘field’], \textit{oko} [‘eye’], \textit{orel} [‘eagle’], \textit{plod} [‘fruit’], \textit{roƶen} (\textit{roжen}) [‘skewer’], \textit{slovo} [‘word’], \textit{slava} [‘glory’], \textit{sused} [‘neighbor’], \textit{temno} [‘dark’], \textit{temnica} [‘dungeon’], \textit{ud} [‘limb’], \textit{vjek} [‘age’], \textit{zor} [‘view, look’], \textit{pozor} [‘attention’], \textit{ƶila} [‘vein’], \textit{ƶivot} [‘life’], and so on. Every northern, central, and southern Slav will very easily read, write, and understand these and similar words. This common writing system is necessary, since without it all literature, and indeed all culture, of the minor Slavic peoples will only remain \is{Pietism}a pious wish.\footnote{The phrase “a pious wish” is associated with \is{Pietism}pietism.}

\subsection*{\hspace*{\fill}§. 10.\hspace*{\fill}}

Furthermore, dictionaries of \is{Dialect}dialects are necessary, but fashioned in such a way that foreign words are accurately distinguished, and native words from the other \is{Dialect}dialects are put in their stead. For the \il{Slavic}Slavic language is a true cornucopia; it is indeed a fact that even native \il{Slavic}Slavic words themselves vary in the speech of those populations using different \is{Dialect}dialects. However, this variation consists mostly in the mutation of vowels, the consonants remain the same, as if they were the bones of the word. Hence it happens that the rougher \is{Dialect}dialects leave out the vowels, but soft \is{Dialect!Softening dialect}dialects insert [25] them and soften the consonants themselves. The \il{Illyrian}Illyrian \is{Dialect}dialects soften, as do \il{Polish}Polish and \il{Russian}Russian, so instead of \textit{smrt} [‘death’], \textit{srdce} [‘heart’], \textit{prst} [‘finger’], \textit{tvrdi} [‘hard’] one says in the \is{Dialect!Softening dialect}soft dialects: \textit{smert}, \textit{serdce}, \textit{serce}, \textit{serco}, \textit{perst}, \textit{tvardi} = \textit{tverdi} etc. Once there are particularist lexica, a universal etymological-philological dictionary will have to be compiled; if this were accompanied by a rational method of \is{Inflection}inflecting the parts of speech, then we can rightly claim the \il{Slavic}Slavic language has been cultivated both civically and logically. Reason restrains empty language rules, but instead establishes firm linguistic principles that conform to its \is{Genius}genius. Let us therefore proceed to the rational examination of the parts of speech.

\section*{Section IV. \textit{On the inflection of the parts of speech}. \linebreak{}[p. 25--76]}
\addcontentsline{toc}{section}{Section IV. \textit{On the inflection of the parts of speech}. [p. 25--76]}

\subsection*{\hspace*{\fill}§. 1.\hspace*{\fill}}

Henceforth, I will follow the method adopted by grammarians in transmitting and explaining languages, even though all might not approve, since our grammars are more occupied with terminology than with the subject matter itself. Some parts of speech are \is{Inflection}inflected, but others are immutable. For instance, nouns, pronouns, and verbs are mutable, but prepositions, adverbs, interjections, and conjunctions do not change. So seven parts of speech emerge, for the participle is not a distinct part of speech, since it either retains the value of a verb, and in this case it belongs to [26] the verb, or it assumes the form and nature of an adjective, and in this case it belongs to the noun. Otherwise, one could also call the so-called gerund and supine a distinct part of speech.

\subsection*{\hspace*{\fill}§. 2.\hspace*{\fill}}

The noun is a word by means of which an object or a property or quality thereof is designated, hence the name “substantive” and “adjective”. Examples of substantives are: \textit{bob} [‘bean’], \textit{pup} ‘navel’, \textit{um} ‘intellect’, \textit{pokoj} [‘peace’], \textit{len} [‘flax’], \textit{vol} [‘ox’], \textit{dol} [‘valley’], \textit{udol} [‘valley’], \textit{stol} [‘table’], \textit{san} ‘dignity’, \textit{mir} ‘peace’, \textit{zavjet} ‘treaty’, \textit{ad} ‘hell’, \textit{med} [‘honey’], \textit{sud} [‘law court’], \textit{xod} [‘pace, step’], \textit{zaxod} [‘circuit, (sun)set, latrine’], \textit{liepota} ‘grace’, \textit{krim} ‘lily’, \textit{kot} [‘cat’], \textit{kit} ‘whale’, \textit{sovjed} ‘council’, \textit{sovjest} ‘consciousness’, \textit{bies} ‘demon’, \textit{lis} ‘fox’, \textit{bieg} ‘course’, \textit{lug} ‘grove’, \textit{rog} [‘horn’], \textit{miex} ‘wineskin’, \il{Bohemian}\textit{чex} [‘Bohemian’], \textit{poslux} ‘aural witness’, \textit{bok} [‘side’], \textit{mak} [‘poppyseed’], \textit{zamok} [‘lock, castle’], \textit{rak} [‘crayfish’], \textit{lik} ‘choir’, \textit{tok} [‘flow’], \textit{potok} [‘stream’], \textit{otrok} [‘boy, servant’], \textit{ryk} ‘roaring’, \textit{vyk} ‘wolf howling’, \textit{tuk} ‘fat’, \textit{sok} ‘juice’, \textit{suk} ‘tree knot’, \textit{mol} [‘moth’], \textit{xmel} [‘hops’], \textit{kniaz} \linebreak{}[‘prince’], \textit{ƶeravel} ‘crane’, \textit{put} ‘road’, \textit{tat} ‘thief’, \textit{ziat} ‘son-in-law’, \textit{test} ‘father-in-law’, \textit{gost} ‘guest’, \textit{noƶ} [‘knife’], \textit{strax} [‘fear’], \textit{lemes} [‘plow blade’], \textit{meч} [‘sword’], \textit{plaч} [‘lamentation’] etc. These and other nouns have retained the same meanings which they had a thousand and more years ago, and in all \is{Dialect}dialects. Hence, \il{Old Church Slavonic}old Slavic should be regarded, as it were, as the nursery of the remaining \is{Dialect}dialects, which already flourished in civic fashion a thousand and more years ago. If only we possessed more monuments in this language than just the Bible [27]. Some peoples have more or less departed from the meaning and expression of \il{Old Church Slavonic}old Slavic. For instance, \textit{san} ‘dignity’ is unfamiliar to many: the \il{Pannonian}Pannonians express it by \textit{hodnost}. \textit{Tuk} ‘fat’ [noun], whence \textit{tuчni} ‘fat’ [adjective], is expressed by the \il{Moravian}Moravians by \textit{masnost} from \textit{maso} [‘flesh’], but not at all authentically, for \textit{masnost} and \textit{tuk} are not the same thing. For this reason one should always keep the \il{Old Church Slavonic}old Slavic in mind, and supplement from it in those cases where the authentic meaning has been lost.

\subsection*{\hspace*{\fill}§. 3.\hspace*{\fill}}

Should the \il{Old Church Slavonic}old Slavonic method of \is{Declension!Noun declension}inflecting nouns be strictly maintained? Not at all, for \il{Old Church Slavonic}old Slavic, into which the holy Bible was translated, once underwent civic cultivation, though it was not at all rationally cultivated, as the various texts of the Bible prove. In subsequent times, erudite men tried to draw up some grammatical rules from it. On the basis of Biblical texts, some derived 50 paradigms of \is{Declension!Noun declension}noun inflections, others reduced these to 40; the immortal \ia{Dobrovský, Josef}Dobrovský\footnote{Josef \ia{Dobrovský, Josef}Dobrovský (1753--1829) was one of the pioneers of \il{Slavic}Slavic comparative linguistics and authored a grammar of \il{Old Church Slavonic}Old Church Slavonic (\citeyear{dobrovsky_institutiones_1822}) and an outline for a general \il{Slavic}Slavic etymological dictionary (\citeyear{dobrovsky_entwurf_1813}), which made \ia{Dobrovský, Josef}Dobrovský a living monument for \ia{Herkel, Jan}Herkel.} limited these to nine forms, including two masculine, three neuter, and four feminine forms. There is a straightforward explanation for this great variation: the Bible was written in popular, and therefore very free, speech. In truth, the speech of the common people fluctuates; it changes, deletes, or adds vowels as it pleases. Hence it is not surprising to observe in the Bible as many as three \is{Declension!Noun declension}declensions of the same noun in the same case [28]: for instance, \textit{gosti} ‘guests’, or \textit{gostie}, or \textit{gostove}, or \textit{gostia}. Similarly, \il{Pannonian}Pannonians also commonly say \textit{priшli naшi hosti} [‘our guests arrived’], \textit{hostia}, \textit{hostove}, or \textit{hostie} etc. Yet no one will dare to condemn this or another way of talking, as everyone speaks in the way he has learned from hearing. The Bible was translated in this fashion. Those grammarians claiming that only this ending or that ending is authentic bring nothing useful to Slavic literature. It is no wonder, then, if some grammarians gather 50 \is{Declension!Noun declension}noun declensions from the Bible, and others 40.

\subsection*{\hspace*{\fill}§. 4.\hspace*{\fill}}

Since the \il{Old Church Slavonic}old Slavic language underwent no rational cultivation at all, it should not be treated as the exclusive norm from which \is{Inflection}inflections are derived. The \is{Dialect!Living dialect}living dialects and \is{Dialect!Subdialect}subdialects should also be consulted, and the principles of the language will eventually be discerned from all of them in combination. When considering this combination, it will be easiest to follow the general rule, so long as the \is{Dialect}dialects do not all share an exception in some noun or verb. But when in doubt, a single \is{Dialect}dialect will have to yield to the plurality of \is{Dialect}dialects. This law is just, founded as it is on sound reason. If reason will be observed, the labyrinth of grammars that torments natures longing for knowledge will vanish, and the language will obtain firm principles that are therefore easy to learn. [29]

\subsection*{\hspace*{\fill}§. 5.\hspace*{\fill}}

Because nouns have three genders, all \is{Dialect}dialect grammarians establish three \is{Declension!Noun declension}declensions, namely masculine, feminine, and neuter. The most recent grammar of the \il{Old Church Slavonic}ancient \is{Dialect}dialect divides the masculine nouns into two forms, and illustrates both by means of four paradigms, as follows:\footnote{\citet[466, 468]{dobrovsky_institutiones_1822}.}

\begin{longtable}{ l l l l l }
    \caption*{\is{Declension!Noun declension}\textit{Declension of the first form of masculine nouns}.} \\
    \noalign{\vspace{6pt}}
    \lsptoprule
    \multicolumn{5}{ c }{Singular.} \\
    \midrule
    \textit{N}. & \textit{Rab} & \textit{Sin} & \textit{Jarem} & \textit{Dom} \\
    & [‘slave, servant’], & [‘son’], & [‘yoke’], & [‘house’]. \\
    \textit{G}. & \textit{Raba}, & \textit{Sina}, & \textit{Jarma}, & \textit{Domu}. \\
    \textit{D}. & \textit{Rabu}, & \textit{Sinovi}, & \textit{Jarmu}, & \textit{Domu}. \\
    \textit{A}. & \textit{Rab}, & \textit{Sin}, & \textit{Jarem}, & \textit{Dom}. \\
    \textit{L}. & \textit{Rabje}, & \textit{Sinje}, & \textit{Jarmje}, & \textit{Domu}. \\
    \textit{I}. & \textit{Rabom}, & \textit{Sinom}, & \textit{Jarmom}, & \textit{Domom}. \\
    \lspbottomrule
    \\
    \lsptoprule
    \multicolumn{5}{ c }{Plural.} \\
    \midrule
    \textit{N}. & \textit{Rabi}, & \textit{Sinove}, & \textit{Jarmi}, & \textit{Domove}. \\
    \textit{G}. & \textit{Rab}, & \textit{Sinov}, & \textit{Jariem}, & \textit{Domov}. \\
    \textit{D}. & \textit{Rabom}, & \textit{Sinovom}, & \textit{Jarmom}, & \textit{Domom}. \\
    \textit{A}. & \textit{Raby}, & \textit{Sinovy}, & \textit{Jarmy}, & \textit{Domy}. \\
    \textit{L}. & \textit{Rabjex}, & \textit{Sinovjex}, & \textit{Jarmjex}, & \textit{Domjex}. \\
    \textit{I}. & \textit{Raby}, & \textit{Sinovy}, & \textit{Jarmi}, & \textit{Domy}. \\
    \lspbottomrule
\end{longtable}

\newpage

\begin{longtable}{ l l l l l }
    \caption*{{\is{Declension!Noun declension}\textit{Declension of the second form of masculine nouns}.}} \\
    \noalign{\vspace{6pt}}
    \lsptoprule
    \multicolumn{5}{ c }{Singular.} \\
    \midrule
    \textit{N}. & \textit{Car} & \textit{Vraч} & \textit{Kniaz} & \textit{Mravij} \\
    & [‘emperor’], & [‘physician’], & [‘prince’], & [‘ant’]. \\
    \textit{G}. & \textit{Carja}, & \textit{Vraчa}, & \textit{Kniaza}, & \textit{Mravija}. [30] \\
    \textit{D}. & \textit{Caru}, & \textit{Vraчevi}, & \textit{Kniaziu}, & \textit{Mraviju}. \\
    \textit{A}. & \multicolumn{4}{ l }{like the \textit{nominative} or the \textit{genitive}.} \\
    \textit{V}. & \textit{Carju}, & \textit{Vraчu}, & \textit{Knianzje}, & \textit{Mraviju}. \\ 
    \textit{L}. & \textit{Cari}, & \textit{Vraчi}, & \textit{Kniazi}, & \textit{Mravii}. \\
    \textit{I}. & \textit{Carem}, & \textit{Vraчem}, & \textit{Kniazem}, & \textit{Mraviem}. \\
    \lspbottomrule
    \\
    \lsptoprule
    \multicolumn{5}{ c }{Plural.} \\
    \midrule
    \textit{N}. & \textit{Carije}, & \textit{Vraчevje}, & \textit{Kniazi}, & \textit{Mravije}. \\
    \textit{G}. & \textit{Cary}, & \textit{Vraчjev}, & \textit{Kniaz}, & \textit{Mravij}. \\
    \textit{D}. & \textit{Cariem}, & \textit{Vraчjem}, & \textit{Kniaziem}, & \textit{Mraviem}. \\
    \textit{A}. & \textit{Carja}, & \textit{Vraчja}, & \textit{Knazja}, & \textit{Mravija}. \\
    \textit{V}. & \multicolumn{4}{ l }{like the \textit{nominative}.} \\ 
    \textit{L}. & \textit{Carjex}, & \textit{Vraчjex}, & \textit{Kniazjex}, & \textit{Mravijex}. \\
    \textit{I}. & \textit{Cary}, & \textit{Vraчi}, & \textit{Kniazi}, & \textit{Mravij}. \\
    \lspbottomrule
\end{longtable}

The \il{Old Church Slavonic}ancient \is{Dialect}dialect \is{Declension!Noun declension}inflected masculine nouns designating something perceptible by means of the senses in a twofold manner, namely commonly and in some cases adjectivally. The most common manner of \is{Declension!Noun declension}inflecting is \textit{dub} [‘oak’], G. \textit{duba}, D. \textit{dubu}, but the adjectival is \textit{dubovi}, \textit{muƶovi} [‘to the man’], \textit{vraчevi} etc. This can properly be called the adjectival way of \is{Declension!Noun declension}inflecting because adjectives denoting possession are formed from masculine substantives by adding the syllable \textit{ov}: for instance, from \textit{car} [‘emperor’], \textit{carov}, \textit{a}, \textit{o}, (\textit{e}), and \textit{Pavel}, \textit{Pavelov} or \textit{Pavlov}, \textit{a}, \textit{o}, (\textit{e}) derive \textit{carov dvor} [‘the emperor’s court’], \textit{Pavlov sin} [‘Pavel’s son’], \textit{kniazov dom} [‘the prince’s house’] etc. And for this reason the dative singular and the nominative plural are pronounced, in both the old and in the \is{Dialect!Living dialect}living dialects, in a twofold manner, [31] as follows: \textit{Caru}, \textit{Pavlu}, \textit{kameniu} [‘to the stone’], or \textit{Carovi}, \textit{Pavlovi}, \textit{kamenovi}; nominative plural, \textit{Cari} or \textit{Carove}, \textit{kameni} or \textit{kamenove}, \textit{kniazi} or \textit{kniazove} \textit{etc}. This twofold way of \is{Declension!Noun declension}inflecting is used among the Slavs everywhere, but grammarians describe this use with anxious rules. The \il{Polish}Pole, for instance, attributes an adjectival ending only to animate substantives denoting some excellence or dignity, such as \textit{Bogovi} [‘to God’], \textit{duxovi} [‘to the spirit’], \textit{Panovi} [‘to the Lord’], \textit{kastellanovi} [‘to the castellan’] etc. The \il{Bohemian}Bohemian restricts this use especially to monosyllabic words, yet makes an exception for \textit{Dux}, \textit{Bux}, and says that these have in the dative only \textit{Duxu} and \textit{Bohu}, an exception which is directly opposed to the rule of the \il{Polish}Poles, which contends that one should say \textit{Bogovi}, \textit{Duxovi} etc. Yet even the \il{Bohemian}Bohemian permits the adjectival plural in \textit{Bohove}, \textit{Duxove} -- see Dobrovský’s \ia{Dobrovský, Josef}grammar, pages 170 and 172.

\subsection*{\hspace*{\fill}§. 6.\hspace*{\fill}}

The \il{Pannonian}Pannonian grammarian restricts the adjectival form of \is{Declension!Noun declension}inflecting chiefly to verbal nouns ending in -\textit{el}, such as: \textit{uчitel} [‘teacher’] \textit{uчitelovi}, \textit{spasitel} [‘savior’] \textit{spasitelovi} etc. Yet common usage does not conform to the grammarian’s rules, but uses the adjectival ending freely, just like in \il{Old Church Slavonic}old Slavic. Indeed, if we closely inspect the Biblical record, it will be clear that masculine plurals took both the common and the adjectival form with no regard for the aforementioned restrictions [32], whether those words were animate or inanimate, monosyllabic or polysyllabic. For one reads in the dative \textit{dnevi} [‘to the day’], \textit{ognevi} [‘to the fire’], \textit{konevi} [‘to the horse’], \textit{kamenovi} [‘to the stone’], \textit{carovi} [‘to the emperor’], \textit{vinarovi} [‘to the winemaker’], \textit{gospodevi} [‘to the lord’], \textit{muƶevi} [‘to the man’], \textit{meчevi} [‘to the sword’], \textit{deƶdevi} [‘to the rain’], \textit{jezevi} [‘to the hedgehog’], \textit{molevi} [‘to the moth’], \textit{vrabievi} [‘to the sparrow’], \textit{smijevi} [‘to the snake’] instead of \textit{dnu}, \textit{ognu} etc., but in the nominative plural \textit{dni}, \textit{ogni} etc. or \textit{dneve}, \textit{ogneve}, \textit{koneve}, \textit{carjeve}, or by abbreviation \textit{dne}, \textit{ogne}, \textit{carje} etc. Hence, if I observe in the Bible this or another ending, I would shrink from asserting that this or that ending indicates a \is{Russianism}Russianism, \is{Polishism}Polishism, \is{Bohemianism}Bohemianism, or \is{Serbianism}Serbianism. If the Bible were to be translated into the \il{Pannonian}Pannonian \is{Dialect}dialect today, many different endings of the same case and word would certainly be used in its composition. One would say, for instance, \textit{s mojimi bratmi} [‘with my brothers’], as well as \textit{bratami}, or \textit{bratrami}, or \textit{bratrimi}, or \textit{bratji} etc. and the like. And if a thousand years hence such a composition were discussed, erudite men would torment themselves in vain trying to deduce fixed rules from it. The very same thing now also holds for the \il{Old Church Slavonic}ancient \is{Dialect}dialect of the Bible. One recent grammarian, for example, claims to establish an instrumental plural ending similar to that of the \textit{a}-stems or \textit{o}-stems, namely: \textit{s rabi}, \textit{sini}, \textit{jarmi}, \textit{domi}, \textit{Cari}, \textit{vraчi}, \textit{kniazi} [‘with the slaves, with the sons, with the yokes, with the homes, with the emperors, with the physicians, with the princes’] \textit{etc}., even though in the most \il{Old Church Slavonic}ancient manuscripts one also reads endings with \textit{mi}, as \textit{s gospodmi}, \textit{sinmi}, \textit{darmi}, \textit{muƶmi}, \textit{denmi}, \textit{liudmi}, \textit{stepenmi} [‘with the lords, with the sons, with the gifts, with the men, with the days, with the people, with the steps’] etc. And this seems to be the original ending of the instrumental plural, both because it appears in the most \il{Old Church Slavonic}ancient manuscripts, but also because it is endorsed by the common use of all \is{Dialect}dialects; for all [33] \is{Dialect}dialects append the characteristic -\textit{mi} to the instrumental plural. Only the \il{Bohemian}Bohemians like to abbreviate this ending. So in the \il{Old Church Slavonic}original Slavic one says: \textit{s meчami} [‘with the swords’], which becomes through syncope \textit{s meчmi}, and finally, after deletion of \textit{m}, \textit{s meчi}. Yet the \il{Bohemian}Bohemian grammarian warns that this abbreviation should be avoided, if any ambiguity could arise, since in this abbreviated form the instrumental is identical with the accusative (Dobrovský, \ia{Dobrovský, Josef}\textit{Lehrgebäude der Böhmischen Sprache}, page 175).

\subsection*{\hspace*{\fill}§. 7.\hspace*{\fill}}

From the aforementioned, and from the Biblical corpus, it is clear that masculine nouns designating something perceptible by means of the senses were \is{Declension!Noun declension}inflected by the \il{Old Church Slavonic}old Slavs also in adjectival fashion, namely by addition of the syllable -\textit{ov}. Hence, although the grammarian’s paradigm gives the dative \textit{rabu} and \textit{domu} for \textit{rab} and \textit{dom}, witnesses from antiquity, were they alive and present, would testify that \textit{rabovi} and \textit{domovi} were also in use. But the \is{Dialect!Living dialect}living dialects also demonstrate that \textit{domovi} is just as correct as \textit{domu}. The word \textit{put} ‘road’ is read in the genitive and dative as \textit{puti}, but it does not follow that, just as in the \is{Declension!Noun declension}inflection of modern \is{Dialect}dialects, the genitive \textit{puta} and the dative \textit{putu} were not also used. Modern \is{Dialect}dialects have these endings only in audible form, but hearing surely reflects usage. For it is certain that \il{Old Church Slavonic}old Slavic lacked a \is{Codification}codified grammar back when the Bible was translated into it; hence one reads without distinction e.g. \textit{grob}, loc. \textit{grobu} or \textit{grobje}  [‘in the grave’], \textit{domu} or \textit{domje}  [‘in the house’]; \textit{zakonu} or \textit{zakonje}  [‘in the law’]; \textit{uglu} [34] or \textit{uglje}  [‘in the corner’]; \textit{smjexu}, \textit{smexje} or \textit{smesje} [‘in the laughter’] etc. Hence it is no wonder if grammarians can derive from the Bible 50 or 40 \is{Declension!Noun declension}declension paradigms of the \il{Old Church Slavonic}ancient \is{Dialect}dialect, just as today one could also derive them from every spoken \is{Dialect}dialect.

\subsection*{\hspace*{\fill}§. 8.\hspace*{\fill}}

Although more than a thousand years have passed since the translation of the Bible into the \il{Slavic}Slavic \is{Idiom}idiom, still \il{Slavic}Slavic \is{Dialect}dialects currently thrive which barely deviate from the old language in form, and not at all in their essence. Consider the \is{Declension!Noun declension}inflection of masculine nouns, whose form of inflection is still observed today in the various \is{Dialect}dialects. All \is{Dialect}dialects and even \is{Dialect!Subdialect}subdialects should therefore be taken into consideration when devising a universal manner of writing.

\subsection*{\hspace*{\fill}§. 9.\hspace*{\fill}}

The \il{Russian}Russians establish for the \is{Declension!Noun declension}inflection of masculine nouns only two paradigms, for with them the \il{Slavic}Slavic language has achieved a greater rational cultivation than the \il{Old Church Slavonic}ancient \is{Dialect}dialect.\footnote{\citet[table insert at 204--205]{puchmayer_lehrgebaude_1820}.}

\begin{longtable}{ l l l l l }
    \lsptoprule
    & \multicolumn{2}{ c }{Singular.} & \multicolumn{2}{ c }{Plural.} \\
    \midrule
    \textit{N}.\textit{A}.\textit{V}. & \textit{Stol} & \textit{Korabl} & \textit{Stoli} & \textit{Korabli} \\
    & [‘table’], & [‘ship’]. & [‘tables’], & [‘ships’]. \\
    \textit{G}. & \textit{Stola}, & \textit{Korabilia}. & \textit{Stolov}, & \textit{Korablei}. \\
    \textit{D}. & \textit{Stolu}, & \textit{Korabliu}. & \textit{Stolam}, & \textit{Korabliam}. \\ 
    \textit{L}. & \textit{Stolie}, & \textit{Korablie}. & \textit{Stolax}, & \textit{Korabliax}. \\
    \textit{I}. & \textit{Stolom}, & \textit{Korablem}. & \textit{Stolomi},\footnote{We suspect this may be a typo for \textit{Stolami}.} & \textit{Korabliami}. \\
    \lspbottomrule
\end{longtable}

[35] Here we see that in the \il{Russian}Russian \is{Dialect}dialect the eight paradigms of the \il{Old Church Slavonic}ancient \is{Dialect}dialect have contracted into two forms, which other \is{Dialect}dialects have contracted into only one, since \textit{korabl} or \textit{korabel}, or \textit{korab} are \is{Declension!Noun declension}inflected like \textit{stol}. Indeed, the fact that this word \textit{korabl} in the oblique cases is augmented with the letter \textit{i} shows the softening pattern of the \il{Russian}Russian \is{Dialect!Softening dialect}dialect, yet whether that \textit{i} is inserted or not, the pattern is the same. Indeed, the fact that the instrumental case in one pattern is pronounced with -\textit{om}, but in the other with -\textit{em}, is not an essential distinction, but only a free pronunciation variant in which the \il{Russian}Russians themselves indulge. For instance, \textit{otec} [‘father’] they say as \textit{otcom} or \textit{otcem}, just as other \is{Dialect}dialects say \textit{stolem} or \textit{stolom}, \textit{korabljem} or \textit{korabliom} etc. Furthermore, some words in common usage are pronounced in the genitive singular with \textit{u}: for instance, \textit{vosk} [‘wax’], \textit{vosku}, \textit{lies} [‘forest’], \textit{liesu}, \textit{most} [‘bridge’], \textit{piesok} [‘sand’], \textit{riad} [‘row’], \textit{jad} [‘poison’], \textit{polk} ‘legion’, \textit{roj} [‘swarm’], \textit{boj} [‘struggle’], yet the more elevated style follows that general and rational way of writing, thus \textit{dom} takes genitive \textit{doma}, \textit{most mosta} etc. The genitive is namely always pronounced with -\textit{a}. In the \il{Old Church Slavonic}ancient Bible, the accusative singular of animate things is read the same as the nominative, but the \is{Dialect!Living dialect}living dialects, including \il{Russian}Russian, pronounce the nouns of animate things the same way as the genitive singular, as e.g. is read among the \il{Old Church Slavonic}ancients: \textit{Privjedox sin moj k tebje}, ‘I brought my son to you’. The more recent editions conform themselves to the \is{Dialect!Living dialect}living dialects: \textit{privjedox sina mojego k tebje}. This way of speaking also occurs in the most \il{Old Church Slavonic}ancient Bible as if it were an original \il{Slavic}Slavic expression, for the former evokes the \il{Greek}Greek text more than the \is{Genius!Genius of the Slavic language}Slavic genius. [36]

\subsection*{\hspace*{\fill}§. 10.\hspace*{\fill}}

Concerning the masculine plural among the \il{Russian}Russians, note the following: the regular nominative plural ending is \textit{i}, as with the other Slavs, but just as other \is{Dialect}dialects allow the so-called adjectival ending, as \textit{sini} or \textit{sinove} [‘sons’], \textit{muƶi} or \textit{muƶove} [‘husbands’], so do the \il{Russian}Russians change that \textit{e} into \textit{a} for the sake of euphony as follows: \textit{sinovia}, \textit{muƶovia}, \textit{stavotoja} [‘tasks’], or \textit{kumovja}, from \textit{kum} ‘godfather’. Yet the endings of the plural number of some words are in \textit{a}, so \textit{bok} [‘side’], \textit{rog} [‘horn’], \textit{rukav} [‘sleeve’], \textit{bereg} ‘river bank’, \textit{golos} ‘voice’, \textit{obraz} [‘image’] are \textit{boka}, \textit{roga} etc. in the plural, instead of \textit{boki}, \textit{rogi}, \textit{rukavi} etc. These are clearly remnants of the dual, which are found in all \is{Dialect}dialects. For example, \il{Pannonian}Pannonian has \textit{liudja} [‘people’], \textit{bratja} [‘brothers’], \textit{hostja} [‘guests’] etc. instead of \textit{ljudi}, \textit{hosti}, \textit{bratji}. Grammarians note this sort of nominative plural using \textit{a} with a diacritic \textit{à}, so as to distinguish it from the genitive singular. Hence it follows that the regular nominative plural ending, both in the \is{Dialect!Living dialect}living dialects and in the \il{Old Church Slavonic}ancient one, is either \textit{i} or the adjectival form; thus either \textit{sini}, \textit{kameni} [‘stones’], \textit{svati} [‘matchmakers’], or \textit{sinove}, \textit{kamenove}, \textit{svatove} etc.

\subsection*{\hspace*{\fill}§. 11.\hspace*{\fill}}

For \is{Declension!Noun declension}noun inflections following the pattern of \textit{korabl}, the genitive plural ending is -\textit{ov} or -\textit{ev}, and particularly nouns ending in -\textit{j}, such as \textit{pokoj} [‘peace’], \textit{zlodej} [‘thief’], take -\textit{ov} or -\textit{ev} in the genitive plural. Thus \textit{pokojev} [37], \textit{zlodejev} in \il{Russian}Russian, but in the other \is{Dialect}dialects \textit{pokojov}, \textit{zlodejov}, since other \is{Dialect}dialects pronounce \textit{korabl} in the genitive plural with -\textit{ov} or -\textit{ev}. Both \il{Polish}Polish as well as the southern \is{Dialect}dialects take this ending, yet some words in the southern \is{Dialect}dialects are pronounced differently in the genitive plural; for instance the genitive plural of \textit{gost} [‘guest’] in some \is{Dialect}dialects is \textit{gosti}, and \textit{gostov} in others. The genitive plural of \textit{muƶ} [‘man’], to give another example, is for the \il{Windic}Winds \textit{muƶ} in the style of neuter nouns, but other \is{Dialect}dialects say it regularly, and vice versa.\footnote{\citet[232]{kopitar_grammatik_1808}.} Combining these observations leads to the following general rule, which has no exceptions:

\newpage

\begin{longtable}{ l l l }
    \lsptoprule
    & Singular. & Plural. \\
    \midrule
    \textit{N}. & & \textit{i}, or adjectival -\textit{ve}. \\
    \textit{G}. & -\textit{a} & -\textit{ov} \\
    \textit{D}. & -\textit{u}, or -\textit{vi} & -\textit{om} \\ 
    \textit{A}. & animate -\textit{a} & inanimate like the nominative, or \\
    & inanimate like the nominative & animate like the genitive \\
    \textit{V}. & like the nominative & like the nominative \\
    \textit{L}. & -\textit{u}, or -\textit{e} & -\textit{ax} \\
    \textit{I}. & -\textit{om}, or -\textit{em}. & -\textit{ami}. \\
    \lspbottomrule
\end{longtable}

\subsection*{\hspace*{\fill}§. 12.\hspace*{\fill}}

The \il{Polish}Polish grammarian illustrates the \is{Declension!Noun declension}inflection of masculine nouns by means of twelve paradigms, but since most agree with each other, it is convenient to present only four, namely two of animate and two of inanimate objects, as follows:\footnote{\citet[47--49]{bandtkie_polnische_1808}.} [38]

\begin{longtable}{ l l l l l }
    \lsptoprule
    & \multicolumn{2}{ c }{Singular.} & \multicolumn{2}{ c }{Plural.} \\
    \midrule
    \textit{N}. & \textit{Krol} & \textit{Rak} & \textit{Krolovie}, & \textit{Raki}. \\
    & [‘king’], & [‘crayfish’], & & \\
    \textit{G}. & \textit{Krola}, & \textit{Raka}, & \textit{Krolov}, & \textit{Rakov}. \\
    \textit{D}. & \textit{Krolovi}, & \textit{Rakovi}, & \textit{Krolom}, & \textit{Rakom}. \\
    \textit{A}. & \textit{Krola}, & \textit{Raka}, & \textit{Krolov}, & \textit{Raki}. \\
    \textit{V}. & \textit{Krolu}, & \textit{Raku}, & \multicolumn{2}{ l }{like the nominative} \\ 
    \textit{I}. & \textit{Kroliem}, & \textit{Rakiem}, & \textit{Krolami}, & \textit{Rakami}. \\
    \textit{L}. & \textit{v}. \textit{Krolu}, & \textit{Rakiu}, & \textit{Krolax}, & \textit{Rakax}. \\
    \lspbottomrule
    \newpage
    \lsptoprule
    & \multicolumn{2}{ c }{Singular.} & \multicolumn{2}{ c }{Plural.} \\
    \midrule
    \textit{N}. & \textit{Noƶ} & \textit{Skarb} & \textit{Noƶe}, & \textit{Skarbi}. \\
    & [‘knife’], & [‘treasure’], & & \\
    \textit{G}. & \textit{Noƶa}, & \textit{Skarbu}, & \textit{Noƶov}, & \textit{Skarbov}. \\
    \textit{D}. & \textit{Noƶovi}, & \textit{Skarbovi}, & \textit{Noƶom}, & \textit{Skarbom}. \\
    \textit{A}. & \textit{Noƶ}, & \textit{Skarb}, & \textit{Noƶe}, & \textit{Skarbi}. \\
    \textit{V}. & \textit{Noƶu}, & \textit{Skarbie}, & \textit{Noƶe}, & \textit{Skarbi}. \\
    \textit{I}. & \textit{Noƶem}, & \textit{Skarbem}, & \textit{Noƶami}, & \textit{Skarbami}. \\
    \textit{L}. & \textit{Noƶu}, & \textit{Skarbie}, & \textit{Noƶax}, & \textit{Skarbax}. \\
    \lspbottomrule
\end{longtable}

Aminate nouns do not differ from inanimates in the singular, but they do in the plural, namely \textit{krol} is said \textit{krolovie} and \textit{rak raki}, but this difference is grounded only in varying usage, not in the \is{Genius!Genius of the Slavic language}genius of the \il{Slavic}Slavic language. For in other \is{Dialect}dialects the nominative plural of \textit{krol} is pronounced \textit{kroli}, just like \textit{rak}, \textit{raki}. The \is{Declension!Noun declension}inflection \textit{krolovie} is adjectival, which the \is{Genius}genius of the language uses not only for the names of illustrious persons, as the \il{Polish}Polish grammarian claims, but also for other masculine nouns, particularly those denoting a substance. That is clear from the \il{Old Church Slavonic}ancient \is{Dialect}dialect, in which is said also \textit{meчove} [‘swords’], \textit{deƶdove} [‘rains’], \textit{kamenove} [‘stones’]; thus \textit{rakove}, just like \textit{krolovie}, [39] conforms to the \is{Genius}genius of the language just as much as \textit{noƶi}, \textit{noƶove}, or \textit{noƶe}.\footnote{\ia{Herkel, Jan}Herkel’s original has \textit{nozi}, \textit{nozove}, not \textit{noƶi}, \textit{noƶove}, as one would expect.} For if only the adjectival \is{Declension!Noun declension}inflection is displayed in the dative singular, why would that inflection be invalid in the plural?

\subsection*{\hspace*{\fill}§. 13.\hspace*{\fill}}

The \is{Declension!Noun declension}inflection of inanimate masculine nouns is illustrated equally by two examples, namely \textit{skarb} and \textit{noƶ}, whose inflection agrees in essence both with each other and with the earlier paradigms, except that the accusative of inanimate nouns follows the nominative. While the grammarian distinguishes also the genitive singular endings, namely \textit{skarbu}, \textit{noƶa}, \il{Polish}Polish grammar is truly a torture of memory as far as the genitive endings -\textit{a} or -\textit{u} are concerned. This is unavoidable, for a \is{Dialect}dialect grammarian has to expound the \is{Dialect}dialect as it is, which is grounded in usage. Yet usage changes, so the rules of grammar also necessarily lack constancy. Thus in vain does a grammarian list 48 root word endings for which the genitive ending -\textit{u} is specified, for the assigned endings are merely weakened by so many exceptions, a matter which the erudite Bandtke treats in greater detail in his grammar, pages 53 to 90.{\enlargethispage{0.5mm}\footnote{\citeauthor{bandtkie_polnische_1824} in the \citeyear{bandtkie_polnische_1824} edition, especially at 58--62.}}

The genitive singular ending -\textit{a} for all masculine nouns is genuine and similar in all \is{Dialect}dialects, as the most ancient \il{Polish}Polish books themselves testify, in which one reads: \textit{Rim} > \textit{Rima} [‘Rome’], \textit{Dunaj} > \textit{Dunaja} [‘the Danube’], \textit{jastrab jastraba} [‘hawk’], [40] \textit{liud liuda} [‘people’], \textit{pokoj pokoja} [‘peace’], to which however the grammarian, following current usage, attributes the ending -\textit{u}. But a \is{Dialect}dialect grammarian can only honestly describe current usage as it is, it is not up to him to enquire into the causes of these or other endings, why the previous \textit{pokoja} is now said \textit{pokoju}. For a \is{Dialect}dialect grammarian, usage justifies the rule. Yet things are different with a rationally devised grammar of \il{Slavic}Slavic, or with a rationally cultivated language, for such a grammar examines usage strictly, harmonizes with the other \is{Dialect}dialects, pursues clarity, and unravels the firm principles of language as grounded in usage. There are already traces of such a cultivation in \il{Russian}Russian, where more elevated usage does not allow any other ending of the genitive of masculine nouns than -\textit{a}.

Furthermore, the locative of \textit{skarb} is produced with -\textit{e}, but of \textit{noƶ} with -\textit{u}. This only shows that the original locative ending is -\textit{e} or -\textit{u}. It does not show that some nouns like to take -\textit{u} as locative ending, but others -\textit{e}, because southern Slavs barely know any other locative ending than -\textit{u}. Thus it is clear that the masculine endings noted above are confirmed also by the very usage of the \il{Polish}Poles.

\subsection*{\hspace*{\fill}§. 14.\hspace*{\fill}}

The \il{Bohemian}Bohemian grammarian [Dobrovský] \ia{Dobrovský, Josef}establishes two forms for the masculine nouns, yet illustrates both of them by means of four paradigms: namely one form for animate beings, and another for inanimate objects. [41]

\enlargethispage{\baselineskip}

\begin{longtable}{ l l l l l }
    \lsptoprule
    & \multicolumn{4}{ c }{Singular.} \\
    \midrule
    \textit{N}. & \textit{Xlap} & \textit{Hraч} & \textit{Dub} & \textit{Meч} \\
    & [‘boy’], & [‘player’], & [‘oak’], & [‘sword’]. \\
    \textit{G}. & \textit{Xlapa}, & \textit{Hraчe}, & \textit{Dubu}, & \textit{Meчe}. \\
    \textit{D}. & \textit{Xlapu}, & \textit{Hraчi}, & \textit{Dubu}, & \textit{Meчi}. \\
    \textit{A}. & \textit{Xlapa}, & \textit{Hraчe}, & \textit{Dub}, & \textit{Meч}. \\
    \textit{V}. & \textit{Xlape}, & \textit{Hraчi}, & \textit{Dube}, & \textit{Meчi}. \\ 
    \textit{I}. & \textit{Xlapu}, & \textit{Hraчi}, & \textit{Dube}, & \textit{Meчi}. \\
    \textit{L}. & \textit{Xlapem}, & \textit{Hraчem}, & \textit{Dubem}, & \textit{Meчem}. \\
    \lspbottomrule
    \\
    \lsptoprule
    & \multicolumn{4}{ c }{Plural.} \\
    \midrule
    \textit{N}. & \textit{Xlapi}, & \textit{Hraчi}, & \textit{Dubi}, & \textit{Meчe}. \\
    \textit{G}. & \textit{Xlapů}, & \textit{Hraчů}, & \textit{Dubů}, & \textit{Meчů}. \\
    \textit{D}. & \textit{Xlapům}, & \textit{Hraчům}, & \textit{Dubům}, & \textit{Meчům}. \\
    \textit{A}. & \textit{Xlapi}, & \textit{Hraчe}, & \textit{Dubi}, & \textit{Meчe}. \\
    \textit{V}. & \multicolumn{4}{ l }{similar to the nominative.} \\
    \textit{I}. & \textit{Xlapix}, & \textit{Hraчix}, & \textit{Dubix}, & \textit{Meчix}. \\
    \textit{L}. & \textit{Xlapi}, & \textit{Hraчi}, & \textit{Dubi}, & \textit{Meчi}. \\
    \lspbottomrule
\end{longtable}

This diverse form of \is{Declension!Noun declension}inflecting nouns is \is{Dialect}dialectal, a property of \il{Bohemian}Bohemian alone, for in the other \is{Dialect}dialects these forms are reduced to one, with the following rule common to all \is{Dialect}dialects: the accusative of animate nouns resembles the genitive, and indeed the nominative takes no ending; genitive -\textit{a}; dative -\textit{u}. The accusative of animates follows the genitive, not the nominative; the vocative and the nominative, either -\textit{i},\footnote{The original edition has -\textit{u}, which must be a typo for -\textit{i}.} or -\textit{e}; the locative has -\textit{u}; the instrumental -\textit{m}. In the plural, the nominative takes -\textit{i}, or -\textit{ove} in the adjectival form, e.g. \textit{xlapove}, \textit{dubove}, \textit{hraчove}, \textit{meчove}, or by syncope, with -\textit{ov} omitted, e.g. from \textit{meчove}, \textit{meчe}, from \textit{deƶdove}, \textit{deƶde} [‘rains’]. The genitive takes -\textit{ov}; dative -\textit{om} or -\textit{am}; the accusative is like the nominative for inanimates and like the genitive for animates, or entirely like the nominative; the locative -\textit{ox}; and the instrumental -\textit{ami}. [42]

\subsection*{\hspace*{\fill}§. 15.\hspace*{\fill}}

Apart from this particularity, we observe in the \il{Bohemian}Bohemian \is{Dialect}dialect that the genitive singular of some nouns is produced by means of -\textit{e}, as \textit{haчe} [‘hooks’], \textit{meчe} [‘swords’], \textit{pritele} [‘friends’] etc. This genitive ending is not found in any other \is{Dialect}dialect, which is also why the utmost erudite author says that it is an innovation in place of -\textit{a}, just as also the genitive plural \textit{xlapů} [‘of the lads’] from \textit{xlapov}, in place of the original \textit{xlapuo}, was introduced relatively recently.

The \il{Bohemian}Bohemians mostly abbreviate the instrumental plural, as has been shown in the paradigm: \textit{xlapi} [‘with the guys/fellows’], \textit{dubi} [‘with the oaks’], \textit{hraчi} [‘with the hooks’], \textit{meчi} [‘with the swords’], instead of \textit{xlapmi}, \textit{dubmi} etc., or even \textit{xlapami} etc. Yet the grammarian of great erudition admonishes that this syncope should be avoided for fear that ambiguity could arise from it, and thus in essence \il{Bohemian}Bohemian, too, agrees with the other \is{Dialect}dialects.

\subsection*{\hspace*{\fill}§. 16.\hspace*{\fill}}

Almost all \il{Slavic}Slavic \is{Dialect}dialects are in use in Pannonia. For in almost every county they use a different accent and another prosody. And indeed: the central \il{Pannonian}Pannonians retain much from the \il{Old Church Slavonic}ancient \is{Dialect}dialect; at the borders of Galicia the Slavs are mixed with the \il{Polish}Poles, in the east with the \il{Russian}Russians, or \is{Magyarization}Magyarized with the Hungarians. In the counties of Sopron, Moson, Vas, and Zala, there are Winds and Croats; in Bács province and the Banat, there are Serbs and Bulgarians etc., but the more cultivated and indeed educated Slavs [43] take particular delight in \il{Bohemian}Bohemian books, because of the scarcity of books written in the \il{Pannonian}Pannonian \is{Dialect}dialect. The first to have broken the ice of this \is{Dialect}dialect was a man who has deserved well of \il{Pannonian}Pannonian literature, the late Master \ia{Bernolák, Anton}Bernolák, for he was the first to publish a grammar of the \il{Pannonian}Pannonian \is{Dialect}dialect, which had, however, not won universal approval, because he based his work on the \is{Dialect}dialect of one province only. But had he rigidly accommodated every \il{Pannonian}Pannonian \is{Dialect}dialect, the result would not have been a grammar, but utter chaos. Be that as it may, this indefatigable man nonetheless produced a \il{Pannonian}Pannonian grammar. So what does it say about the \is{Declension!Noun declension}inflection of masculine nouns? It determines two forms of masculine nouns, illustrated with three examples each.\footnote{{\citet[27--28]{bernolak_grammatica_1790}}.}

\begin{longtable}{ l l l l }
    \caption*{\is{Declension!Noun declension}Declension I of masculine animates.} \\
    \noalign{\vspace{6pt}}
    \lsptoprule
    \multicolumn{4}{ c }{Singular.} \\
    \midrule
    \textit{N}. & \textit{Sluha} & \textit{Sudce} & \textit{Pan} \\
    & [‘servant’], & [‘judge’], & [‘Lord’]. \\
    \textit{G}. & \textit{Sluhi}, & \textit{Sudca}, & \textit{Pana}. \\
    \textit{D}. & \textit{Sluhovi}, & \textit{Sudcovi}, & \textit{Panovi}. \\
    \textit{A}. & \textit{Sluhu}, & \textit{Sudca}, & \textit{Pana}. \\
    \textit{V}. & \textit{Sluho}, & \textit{Sudce}, & \textit{Pane}. \\ 
    \textit{L}. & \textit{Sluhovi}, & \textit{Sudcovi}, & \textit{Panovi}. \\
    \textit{I}. & \textit{Sluhom}, & \textit{Sudcom}, & \textit{Panom}. \\
    \lspbottomrule
    \newpage
    \lsptoprule
    \multicolumn{4}{ c }{Plural.} \\
    \midrule
    \textit{N}. & \textit{Sluhi}, & \textit{Sudci}, & \textit{Pani}. \\
    \textit{G}. & \textit{Sluhov}, & \textit{Sudcov}, & \textit{Panov}. \\
    \textit{D}. & \textit{Sluhom}, & \textit{Sudcom}, & \textit{Panom}. \\
    \textit{A}. & \textit{Sluhov}, & \textit{Sudcov}, & \textit{Panov}. \\
    \textit{V}. & \textit{Sluhi}, & \textit{Sudcov}, & \textit{Panov}. [44] \\
    \textit{L}. & \textit{Sluhox}, & \textit{Sudcox}, & \textit{Panox}. \\
    \textit{I}. & \textit{Sluhmi}, & \textit{Sudcmi}, & \textit{Panmi}. \\
    \lspbottomrule
\end{longtable}

\begin{longtable}{ l l l l }
    \caption*{\is{Declension!Noun declension}Declension II of inanimates.} \\
    \noalign{\vspace{6pt}}
    \lsptoprule
    \multicolumn{4}{ c }{Singular.} \\
    \midrule
    \textit{N}. & \textit{Dub} & \textit{Deƶd} & \textit{Dobitek} \\
    & [‘oak’], & [‘rain’], & [‘wealth’]. \\
    \textit{G}. & \textit{Duba}, & \textit{Deƶda}, & \textit{Dobitku}. \\
    \textit{D}. & \textit{Dubu}, & \textit{Deƶdu}, & \textit{Dobitku}. \\
    \textit{A}. & \textit{Dub}, & \textit{Deƶd}, & \textit{Dobitek}. \\
    \textit{V}. & \textit{Dube}, & \textit{Deƶdu}, & \textit{Dobitku}. \\ 
    \textit{L}. & \textit{Dube}, & \textit{Deƶdu}, & \textit{Dobitku}. \\
    \textit{I}. & \textit{Dubom}, & \textit{Deƶdom}, & \textit{Dobitkom}. \\
    \lspbottomrule
    \\
    \lsptoprule
    \multicolumn{4}{ c }{Plural.} \\
    \midrule
    \textit{N}. & \textit{Dubi}, & \textit{Deƶde}, & \textit{Dobitki}. \\
    \textit{G}. & \textit{Dubov}, & \textit{Deƶdov}, & \textit{Dobitkov}. \\
    \textit{D}. & \textit{Dubom}, & \textit{Deƶdom}, & \textit{Dobitkom}. \\
    \textit{A}. & \textit{Dubi}, & \textit{Deƶde}, & \textit{Dobitki}. \\
    \textit{V}. & \textit{Dubi}, & \textit{Deƶde}, & \textit{Dobitki}. \\
    \textit{L}. & \textit{Dubox}, & \textit{Deƶdox}, & \textit{Dobitkox}. \\
    \textit{I}. & \textit{Dubmi}, & \textit{Deƶdmi}, & \textit{Dobitkmi}. \\
    \lspbottomrule
\end{longtable}

\subsection*{\hspace*{\fill}§. 17.\hspace*{\fill}}

Before anything is said about these forms of masculine nouns, it is important to reflect first on this word \textit{Sluha}, or \textit{Sluga}. Almost every \is{Dialect}dialect \is{Declension!Noun declension}inflects this word differently. And indeed the \il{Russian}Russians have the words \textit{Sluga} ‘servant’, \textit{Vojvoda} or \textit{Bojvoda} ‘duke’, \textit{Vladika} ‘lord, ruler’, which take the feminine \is{Declension!Noun declension}declension forms because they end in -\textit{a}, [45] an ending which the \is{Genius!Genius of the Slavic language}genius of the \il{Slavic}Slavic language considers proper only for feminine nouns. Masculine nouns always end in a consonant, but the aforementioned words have been Greco-Latinized. For in original \il{Slavic}Slavic one says \textit{slug} or \textit{slux}; hence \textit{poslux} ‘earwitness’ comes from \textit{posluxati} ‘to listen’, or also ‘to obey’, which is for slaves. Then \textit{Vojvod}, \textit{Bojvod} for the \il{Old Church Slavonic}ancients and \textit{Vladik} in the nominative. And yet I do not deny that these words are already in the most \il{Old Church Slavonic}ancient books read with -\textit{a} as final vowel. Nonetheless, in the meantime, \textit{Vladika}, \textit{Vojvoda} instead of \textit{Vladik}, \textit{Vojvod} in the nominative is to every born Slav an act of violence, and at first sight this ending appears to oppose the \is{Genius}genius of the language. Hence, the \il{Russian}Russians have transferred these words \textit{Sluga}, \textit{Vojvoda}, \textit{Vladika}, as if they were adorned with female dress, to the feminine form of declining, without any exception. The grammarians of other \is{Dialect}dialects \is{Declension!Noun declension}inflect them as masculine nouns on account of their meaning, or as feminine nouns on account of their ending, while still others mix both forms. The \il{Pannonian}Pannonian grammarian completely opts for the paradigm of masculine nouns, but it is very clear that this paradigm cannot be sustained, since it consists only of foreign and Greco-Latinized words, such as \textit{armalista} [‘armalist’, a landless noble], \textit{gardista} [‘imperial guard’], \textit{Evangelista} [‘Evangelist’], \textit{Patriarcha} [‘patriarch’], \textit{Levita} [‘Levite’], for the \il{Slavic}Slavic ending for these words is: \textit{armalist}, \textit{gardist} etc.

\subsection*{\hspace*{\fill}§. 18.\hspace*{\fill}}

For the animate paradigm, \ia{Bernolák, Anton}Bernolák took the second word \textit{Sudce} [‘judge’]; this word is [46] genuinely \il{Slavic}Slavic, but the ending is \is{Dialect}dialectal, and in particular \il{Bohemian}Bohemian, because in \il{Slavic}Slavic nominative masculine nouns do not end in -\textit{e}, neither in the \il{Old Church Slavonic}ancient \is{Dialect}dialect nor in any other \is{Dialect!Living dialect}living dialect apart from \il{Bohemian}Bohemian. For this ending is proper to neuter nouns, and the genuine \il{Slavic}Slavic ending would \textit{sudec}, \textit{sudnik}, or \textit{sudiar} from \textit{suditi} [‘to judge’]. The third paradigm is \textit{pan} [‘lord’], which is the same as \textit{gospod}, \textit{gospon}, \textit{gospan}. With \textit{gos} cast away, \textit{Pan}, \textit{Ban} emerged. However, these paradigms differ by \is{Declension!Noun declension}inflection in grammatical exposition: e.g. \textit{sluha} [‘servant’] appears in the genitive as \textit{sluhi}, and in the accusative as \textit{sluhu}; but in the plural all three agree.

\subsection*{\hspace*{\fill}§. 19.\hspace*{\fill}}

However, the inanimate \is{Declension!Noun declension}declensions of \textit{dub} [‘oak’], \textit{deƶd} [‘rain’], \textit{dobitek} \linebreak{}[‘wealth’], but particularly of \textit{dub}, which is \textit{duba} in the genitive, are in direct contrast to the \il{Bohemian}Bohemian grammar, which likewise takes \textit{dub} as a paradigm and takes -\textit{u} as the genitive ending to the same word, thus \textit{dubu}. So what is in fact needed here? Who can be a fair judge? Furthermore, the locative of each paradigm is different, \textit{dube}, \textit{deƶdi}, \textit{dobitku}, and in the plural \textit{dubi}, \textit{deƶde} etc. These endings are certainly governed by rules and exceptions, but anyone who studies the \il{Slavic}Slavic language in the \il{Bohemian}Bohemian \is{Dialect}dialect will be buried under rules and exceptions which are merely one-sided, \is{Dialect}dialectal, and clearly not based on sound logic; let us therefore consult both usage and reason to remove these differences; let us call for help on other \is{Dialect}dialects in combination [47]. As far as usage is concerned, in the \il{Pannonian}Pannonian \is{Dialect}dialect itself (for I speak the same \is{Dialect}dialect as the \il{Pannonian}Pannonian grammarian, my birthplace not being far from his), the abovementioned sample words are even \is{Declension!Noun declension}declined differently from the forms displayed, thus one says \textit{Toho sluha plat} or \textit{togo sluga plat} [‘the servant’s salary’], thus the genitive singular does not always take -\textit{i} as shown in the paradigms. I have likewise heard: \textit{svojeho sluha sem videl} [‘I saw my servant’] etc., so the accusative is not always pronounced \textit{sluhu} like a feminine noun, but \textit{sluha} like a masculine noun. Then in the nominative plural not just \textit{sluhi} etc., but one also hears \textit{sluhove}, \textit{sudcove} [‘judges’], \textit{panove} [‘Lords’] in the manner of the \il{Old Church Slavonic}ancients, or \textit{sluhovja}, \textit{sudcovja}, \textit{panovja} in the \il{Russian}Russian manner. Among the \il{Pannonian}Pannonians not only the abbreviated locative is in use, as in \textit{sluhmi} etc., but also \textit{sluhami}, or entirely abbreviated to \textit{sluhi} etc. \textit{Deƶd} [‘rain’] in the nominative plural is not only \textit{deƶde}, but also \textit{deƶdi}, \textit{deƶdove}, for \textit{deƶde} is merely an abbreviation of \textit{deƶdove} etc. Hence even in usage all paradigms are reducible to one form; but if we would consider the ratio and combination of \is{Dialect}dialects, no more than one form of masculine nouns can be reasonably admitted. For one \is{Dialect}dialect has an exception that another does not have, and vice versa, which is why the six \il{Pannonian}Pannonian paradigms will follow one genuinely \il{Slavic}Slavic form, and specifically:{\enlargethispage{2mm}\footnote{\citet[27--28]{bernolak_grammatica_1790}.}}

\begin{longtable}{ l l }
    \lsptoprule
    \multicolumn{2}{ c }{Singular.} \\
    \midrule
    \textit{N}. & \textit{Slug}, \textit{sudec}, \textit{pan}, \textit{dub}, \textit{deƶd}, \textit{dobitek}. \\
    \textit{G}. & \textit{Sluga}, \textit{sudca}, \textit{pana}, \textit{duba}, \textit{deƶda}, \textit{dobitka}. [48] \\
    \textit{D}. & \textit{Slugu}, \textit{sudcu}, \textit{panu}, \textit{dubu}, \textit{deƶdu}, \textit{dobitku} or -\textit{ovi}. \\
    \textit{A}. & \textit{animate} like the \textit{genitive}, \textit{inanimate} like the \textit{nominative}. \\
    \textit{V}. & like the \textit{nominative in the manner of the \il{Russian}Russians} and \textit{southerners}. \\ 
    \textit{L}. & like the \textit{dative with} -\textit{u}. \\
    \textit{I}. & \textit{Slugom} etc. \\
    \lspbottomrule
    \newpage
    \lsptoprule
    \multicolumn{2}{ c }{Plural.} \\
    \midrule
    \textit{N}. & \textit{Slugi}, \textit{sudci}, \textit{pani}, \textit{deƶdi}, \textit{dobitki}, \textit{dubi}, or \textit{with} -\textit{ove}, \textit{dubove} etc. \\
    \textit{G}. & \textit{Slugov}, \textit{sudsov} etc. \\
    \textit{D}. & \textit{Slugom} or \textit{slugam} etc. \\
    \textit{A}. & \textit{of the animates} like the \textit{genitive}, \textit{of the inanimates} like the \textit{nominative}. \\
    \textit{V}. & like the \textit{nominative}. \\ 
    \textit{L}. & \textit{Slugox} or -\textit{ax} etc. \\
    \textit{I}. & \textit{Slugami}, \textit{dubami}, or \textit{abbreviated dubmi} or \textit{dubi}. \\
    \lspbottomrule
\end{longtable}

From these paradigms it is clear that the difference between the individual \is{Dialect}dialects is removed by combination, and that the language becomes rationally cultivated, achieving a greater clarity, ease, and sweetness. Is it clearer and easier to follow one set of \is{Declension!Noun declension}declensions, or six, seven, eight etc.? Is it sweeter to temper the clash of consonants, or to be overwhelmed by rules about when and where to delete an existing vowel from the root? All these things are present in \is{Dialect}dialect grammars to such an extent that they necessarily produce boredom in the readers themselves. Yet all these things cannot be removed from \is{Dialect}dialect grammars, for they are only \is{Dialect}dialect collectors making note of those things which they have observed.

\subsection*{\hspace*{\fill}§. 20.\hspace*{\fill}}

Among the southern \is{Dialect}dialects, let us look at the grammar of the most erudite \ia{Kopitar, Jernej}Kopitar as concerns the \is{Declension!Noun declension}declension of masculine nouns.\footnote{\citet[221]{kopitar_grammatik_1808}.} [49]

\begin{longtable}{ l l l l l }
    \lsptoprule
    & \multicolumn{2}{ c }{Singular.} & \multicolumn{2}{ c }{Plural.} \\
    \midrule
    \textit{N}. \textit{V}. & \textit{Rak} & \textit{Kraj} & \textit{Raki}, & \textit{Kraji}. \\
    & [‘crab’], & [‘edge, country’], & & \\
    \textit{G}. & \textit{Raka}, & \textit{Kraja}, & \textit{Rakov}, & \textit{Krajov}. \\
    \textit{D}. & \textit{Raku}, & \textit{Kraju}, & \textit{Rakam}, & \textit{Krajam}. \\ 
    \textit{A}. & \textit{Raka}, & \textit{Kraj}, & \textit{Rake}, & \textit{Kraje}. \\
    \textit{L}. & \textit{Raku} (\textit{i}), & \textit{Kraju} (\textit{i}), & \textit{Rakih}, & \textit{Krajih}. \\
    \textit{I}. & \textit{Rakam}, & \textit{Krajam}, & \textit{Rakmi} (\textit{ki}), & \textit{Krajmi} (\textit{ji}). \\
    \lspbottomrule
\end{longtable}

\ia{Kopitar, Jernej}Kopitar also displays the dual number, which is in use among the \il{Carinthian}Carinthians, \il{Carniolan}Car\-niolans, and \il{Styrian}Styrians to this day. It appears to have once been common to all \is{Dialect}dialects, judging by the vestiges which have remained in all \is{Dialect}dialects. For instance, a Carpathian \il{Pannonian}Pannonian commonly says: \textit{moji bratia} [‘with my brothers’], \textit{s mojima bratama} [‘with my brothers’], \textit{s mojima volama} [‘with my oxen’], \textit{s mojima ovcama} [‘with my sheep’] etc. Why not take into account the dual number, which in the nominative, accusative, and vocative ends with -\textit{a}, but in the instrumental with -\textit{ma}, while the remaining cases are \is{Declension!Noun declension}declined like the plural? But since the dual number is currently not distinguished from the plural in the \il{Russian}Russian, \il{Polish}Polish, \il{Bohemian}Bohemian, and southern \is{Dialect}dialects, I judge that one needs to refrain from rigidly introducing it into use, so that, by all means, one makes no mistake when using the dual at a suitable place, as in: \textit{s mojima oчima} [‘with my eyes’], \textit{nogama} [‘with legs, feet’], \textit{rakama} [‘with crabs’] etc.

\subsection*{\hspace*{\fill}§. 21.\hspace*{\fill}}

The \il{Windic}Windic paradigm shown below agrees with the other \is{Dialect}dialects. The locative plural is pronounced with \textit{h}, but only because the southern Slavs do not use the deeply guttural sound \textit{x}. Nothing in the rest [50] differs in essence from the general forms. In addition to the regular paradigms, \ia{Kopitar, Jernej}Kopitar also adds three irregular paradigms:\footnote{\citet[232]{kopitar_grammatik_1808}.}

\begin{longtable}{ l l l l }
    \lsptoprule
    \multicolumn{4}{ c }{Singular.} \\
    \midrule
    \textit{N}. & \textit{Moƶ} & \textit{Bog} & \textit{Tat} \\
    & ‘man’, & [‘God’], & ‘thief’. \\
    \textit{G}. & \textit{Moƶa}, & \textit{Boga}, & \textit{Tatova}, or \textit{Tatu}. \\
    \textit{D}. & \textit{Moƶu}, & \textit{Bogu}, & \textit{Tatu}, \textit{Tatovu} (\textit{i}). \\
    \textit{A}. & \textit{Moƶa}, & \textit{Boga}, & \textit{Tatu}, or \textit{Tatova}. \\
    \textit{L}. & \multicolumn{3}{ l }{like the dative}. \\
    \textit{I}. & \textit{Moƶam} etc. & & \\
    \lspbottomrule
    \newpage
    \lsptoprule
    \multicolumn{4}{ c }{Plural.} \\
    \midrule
    \textit{N}. & \textit{Moƶje}, & \textit{Bogovi}, & \textit{Tatovi}, \textit{Tatje}. \\
    \textit{G}. & \textit{Moƶ}, & \textit{Bogov}, & \textit{Tatov}. \\
    \textit{D}. & \textit{Moƶem}, & \textit{Bogovam}, & \textit{Tatovam}, \textit{Tatem}. \\
    \textit{A}. & \textit{Moƶe}, & \textit{Bogove}, & \textit{Tatove} (\textit{Tati}). \\
    \textit{L}. & \textit{Moƶeh}, & \textit{Bogovih}, & \textit{Tatovih}. \\
    \textit{I}. & \textit{Moƶmi}, & \textit{Bogovmi}, & \textit{Tatovmi}. \\
    \lspbottomrule
\end{longtable}

What can be said about this? The grammarian himself says that here no definite norm of \is{Declension!Noun declension}declension can be assigned, and thus no rule can be fixed. Whichever nouns are \is{Declension!Noun declension}inflected like \textit{Moƶ}, which like \textit{Bog}, and which like \textit{Tat}, is something that the very learned man could surely have judged; he could indeed have assigned some rules, but he did not do so for the very reason that he saw that such rules would soon be overthrown by exceptions. In the meantime, the three aforementioned paradigms are regular in some \is{Dialect}dialects, and those which are irregular in other \is{Dialect}dialects are in turn regular among the \il{Windic}Winds. But the words \textit{sluga} [‘servant’], \textit{vojvoda} [‘duke’], \textit{vladika} [‘ruler’], [51] \textit{starejшina} [‘elder’] (since there are no other masculine words that take the feminine form) have regular declensions among the \il{Windic}Winds, namely following the general form. On these words, the sharpest investigator of the \is{Genius!Genius of the Slavic language}genius of the \il{Slavic}Slavic language clearly states that the masculine words’ nominative ending with -\textit{a} does not at all agree with the \is{Genius!Genius of the Slavic language}genius of the \il{Slavic}Slavic language.\footnote{\ia{Herkel, Jan}Herkel is referring to \citet[233]{kopitar_grammatik_1808}, saying “die \il{German}Endigung auf Vocale ist, in unserem Dialekte, so wenig den masculinis eigen, dass nur eigene Nahmen, und ein Paar, so zu sagen, Ur-Slavische Substantive sie haben”. \ia{Kopitar, Jernej}Kopitar then gives a \textit{Musterbeispiel} on p. 234.}

\subsection*{\hspace*{\fill}§. 22.\hspace*{\fill}}

According to the illustrious \textit{Vuk}, the \il{Serbian}Serbian method of \is{Declension!Noun declension}inflecting masculine nouns is the following:\footnote{\citet[xxxvii]{karadzic_srpski_1818}.}

\newpage

\begin{longtable}{ l l l l }
    \lsptoprule
    \multicolumn{4}{ c }{Singular.} \\
    \midrule
    \textit{N}. & \textit{Jelen} & \textit{Kolaч} & \textit{Ora} \\
    & [‘stag’], & [‘cake’], & ‘walnut’. \\
    \textit{G}. & \textit{Jelena}, & \textit{Kolaчa}, & \textit{Oraa}. \\
    \textit{D}. & \textit{Jelenu}, & \textit{Kolaчu}, & \textit{Orau}. \\
    \textit{A}. & \textit{Jelena}, & \textit{Kolaч}, & \textit{Ora}. \\
    \textit{V}. & \textit{Jelenu}, & \textit{Kolaчu}, & \textit{Oraшe}. \\
    \textit{I}. & \textit{Jelenom}, & \textit{Kolaчom}, & \textit{Oraom}. \\
    \textit{L}. & \textit{Jelenu} & \textit{Kolaчu}, & \textit{Orau}. \\
    \lspbottomrule
    \\
    \lsptoprule
    \multicolumn{4}{ c }{Plural.} \\
    \midrule
    \textit{N}. & \textit{Jeleni}, & \textit{Kolaчi}, & \textit{Orasi}. \\
    \textit{G}. & \textit{Jelena}, & \textit{Kolaчa}, & \textit{Oraa}. \\
    \textit{D}. & \textit{Jelenima}, & \textit{Kolaчima}, & \textit{Orasima}. \\
    \textit{A}. & \textit{Jelene}, & \textit{Kolaчe}, & \textit{Orae}. \\
    \textit{V}. & \multicolumn{3}{ l }{\textit{like the nominative}.} \\
    \textit{I}. & \multicolumn{3}{ l }{\textit{like the dative}.} \\
    \textit{L}. & \multicolumn{3}{ l }{\textit{also like the dative}.} \\
    \lspbottomrule
\end{longtable}

[52] The \is{Declension!Noun declension}inflection of the singular is genuine. It therefore also agrees with the other \is{Dialect}dialects, but the plural genitive differs from others, as there is no distinction between the singular and plural genitive. It is true that the grammarian writes the genitive singular as follows: \textit{jelena}, but the plural as \textit{jelêna}, yet the symbol attached to the letter \textit{e}, the so-called \textit{sigla} or \textit{kamora} does not compensate for the absence of the characteristic -\textit{ov} in the genitive plural. The remaining plural cases such as the dative, locative, instrumental are borrowed from the dual. On the other hand, however, the dative also uses -\textit{om}. Thus in the recantation of the murder of Lazar at Kosovo Field, one reads: \textit{Dosta mesa i gavranom} [‘enough meat for the ravens’]. Additionally, the adjectival ending is in use, for they say \textit{volovi} [‘to the ox’], \textit{sokolovi} [‘to the falcon’], \textit{priatelovi} [‘to the friend’] etc. In the remaining \il{Illyrian}Illyrian \is{Dialect}dialects, too, there is the following peculiarity: that the masculine nouns ending in -\textit{l} are pronounced with -\textit{o}, as follows: \textit{soko} [‘falcon’], \textit{kotao} [‘boiler’] instead of \textit{sokol}, \textit{kotal} etc. Yet they accept this expunction of the \textit{l} only in the nominative, not in the remaining cases. The same happens in the past tenses of verbs, as follows:

\begin{longtable}{ l l }
    \textit{Junak koniu govorio (govoril)} & [The hero said to his horse]. \\
    \hspace*{5mm} \textit{Oj koniucu! dobro moje! etc}, & \hspace*{5mm} [Oh dear horse! My good thing! etc].
\end{longtable}

So from a comparison of the \is{Dialect}dialects it is clear that the following \is{Declension!Noun declension}declension of masculine nouns agrees with both the usage and the \is{Genius}genius of the language, without any exception:

\begin{longtable}{ l l l l }
    \lsptoprule
    \multicolumn{2}{ c }{Singular.} & \multicolumn{2}{ c }{Plural.} \\
    \midrule
    \textit{N}. & \textit{—} & \textit{N}. & -\textit{i} or -\textit{ove} \\
    \textit{G}. & -\textit{a} & \textit{G}. & -\textit{ov} \\
    \textit{D}. & -\textit{u} or -\textit{vi} & \textit{D}. & -\textit{om} or -\textit{am}. \\
    {[}53{]} & animates -\textit{a}, inanimates & \textit{A}. & like the nominative \\
    \textit{A}. & like the nominative & & \\
    \textit{V}. & like the nominative or -\textit{u}, -\textit{e} & \textit{V}. & like the nominative \\
    \textit{L}. & -\textit{u} & \textit{L}. & -\textit{ox}, or -\textit{ax}, or -\textit{ex}. \\
    \textit{I}. & -\textit{om} & \textit{I}. & -\textit{ami}. \\
    \lspbottomrule
\end{longtable}

\enlargethispage{\baselineskip}

\begin{footnotesize}
\begin{longtable}{ l l l l l l l }
    \lsptoprule
    \multicolumn{7}{ c }{Singular.} \\
    \midrule
    \textit{N}. & \textit{Sin} & \textit{Vojvod} & \textit{Posel} & \textit{Pritel} & \textit{Kamen} & \textit{Meч} \\
    & [‘son’], & [‘duke’], & [‘messenger’], & [‘friend’], & [‘stone’], & [‘sword’]. \\
    \textit{G}. & \textit{Sina}, & \textit{Vojvoda}, & \textit{Posela}, & \textit{Pritela}, & \textit{Kamena}, & \textit{Meчa}. \\
    \textit{D}. & \textit{Sinu}, & \textit{Vojvodu}, & \textit{Poselu}, & \textit{Pritelu}, & \textit{Kamenu}, & \textit{Meчu}. \\
    & or \textit{Sinovi} etc. & & & & \\
    \textit{A}. & \textit{Sina}, & \textit{Vojvoda}, & -\textit{a}, & -\textit{a}, & \textit{Kamen}, & \textit{Meч}. \\
    \textit{V}. & \multicolumn{6}{ l } {like the nominative.} \\
    or & \textit{Sinu}, & \textit{Vojvodu}, & -\textit{u}, & -\textit{u}, & -\textit{u}, & -\textit{u}. \\
    \textit{L}. & \textit{Sinu}, & -\textit{u}, & -\textit{u}, & -\textit{u}, & -\textit{u}, & -\textit{u}. \\
    \textit{I}. & \textit{Sinom}, & -\textit{m}, & -\textit{m}, & -\textit{m}, & -\textit{m}, & -\textit{m}. \\
    \lspbottomrule
    \\
    \lsptoprule
    \multicolumn{7}{ c }{Plural.} \\
    \midrule
    \textit{N}. & \textit{Sini}, & \textit{Vojvodi}, & \textit{Poseli}, & \textit{Priteli}, & \textit{Kameni}, & \textit{Meчi}, or -\textit{ove}. \\
    \textit{G}. & \textit{Sinov}, & \textit{Vojvodov}, & -\textit{ov}, & -\textit{ov}, & -\textit{ov}, & -\textit{ov}. \\
    \textit{D}. & \multicolumn{6}{ l }{\textit{Sinom}, or \textit{Sinam} everywhere -\textit{om}, or -\textit{am}.} \\
    \textit{A}. & \multicolumn{4}{ l }{\textit{Sinov}, or -\textit{Sini} etc.} & \textit{Kameni}, & \textit{Meчi}. \\
    \textit{V}. & \multicolumn{6}{ l } {\textit{like} the nominative.} \\
    \textit{L}. & \textit{Sinox}, & \multicolumn{5}{ l }{\textit{Vojvodox} etc. or everywhere with -\textit{ax}.} \\
    \textit{I}. & \multicolumn{6}{ l }{\textit{Sinami} etc. everywhere in the same fashion.} \\
    \lspbottomrule
\end{longtable}
\end{footnotesize}

\section*{\textit{On the inflection of feminine nouns}. [p. 53--66]}
\addcontentsline{toc}{section}{\indent On the inflection of feminine nouns. [p. 53--66]}

\subsection*{\hspace*{\fill}§. 1.\hspace*{\fill}}

The best-known grammarian of the \il{Old Church Slavonic}ancient \is{Dialect}dialect illustrates the \is{Declension!Noun declension}inflection of feminine nouns with the following four forms:\footnote{\citet[478]{dobrovsky_institutiones_1822}. Dobrovský depicts the declensions for \textit{волѧ} and \textit{ладїѧ} as variants of a single form, which probably explains why \ia{Herkel, Jan}Herkel asserts he will give four forms, and then presents five.}

\begin{longtable}{ l l l l l l }
    \lsptoprule
    \multicolumn{6}{ c }{Singular.} \\
    \midrule
    \textit{N}. & \textit{Voda} & \textit{Volja} & \textit{Ladija} & \textit{Cerkov} & \textit{Kost} \\
    & [‘water’], & [‘will’], & [‘boat’], & [‘church’], & [‘bone’]. \\
    \textit{G}. & \textit{Vodi}, & \textit{Volja}, & \textit{Ladija}, & \textit{Cerkve}, & \textit{Kosti}. \\
    \textit{D}. & [54] \textit{Vodje}, & \textit{Voli}, & \textit{Ladiji}, & \textit{Cerkvi}, & \textit{Kosti}. \\
    \textit{A}. & \textit{Vodu}, & \textit{Volju}, & \textit{Ladiju}, & \textit{Cerkov}, & \textit{Kost}. \\
    \textit{V}. & \textit{Vodo}, & \textit{Vole}, & \textit{Ladije}, & \textit{Cerkvi}, & \textit{Kosti}. \\
    \textit{I}. & \textit{Vodoju}, & \textit{Voleju}, & \textit{Ladieju}, & \textit{Cerkviju}, & \textit{Kostiu}. \\
    \lspbottomrule
    \\
    \lsptoprule
    \multicolumn{6}{ c }{Plural.} \\
    \midrule
    \textit{N}. \textit{A}. \textit{V}. & \textit{Vodi}, & \textit{Volja}, & \textit{Ladija}, & \textit{Cerkve}, & \textit{Kosti}. \\
    \textit{G}. & \textit{Vod}, & \textit{Vol}, & \textit{Ladij}, & \textit{Cerkvij}, & \textit{Kostij}. \\
    \textit{D}. & \textit{Vodam}, & \textit{Voljam}, & \textit{Ladijam}, & \textit{Cerkvam}, & \textit{Kostem}. \\
    \textit{L}. & \textit{Vodax}, & \textit{Voljax}, & \textit{Ladijax}, & \textit{Cerkvax}, & \textit{Kostjex}. \\
    \textit{I}. & \textit{Vodami}, & \textit{Voljami}, & \textit{Ladijami}, & \textit{Cerkvami}, & \textit{Kostmi}. \\
    \lspbottomrule
\end{longtable}

These \is{Declension!Noun declension}declension forms were collected from \il{Old Church Slavonic}ancient manuscripts, but I would probably not err to conclude that, strictly speaking, only these endings prevail for these and other feminine nouns. The erudite disagree among themselves, but I believe that their opinions can be reconciled, just as I have suggested for masculine nouns: i.e. the \il{Slavic}Slavic language received a Bible translation before its critical cultivation, thus there is no wonder that there are diverse readings in different manuscripts. But if we consider the modern \is{Dialect}dialects, they \is{Declension!Noun declension}inflect the five abovementioned forms according to one form only, and in particular according to the \il{Old Church Slavonic}ancient form \textit{voda}; hence it is clear that now, too, the \is{Dialect!Living dialect}living dialects have not departed from the \is{Genius}spirit of the \il{Old Church Slavonic}ancient \is{Dialect}dialect.

\subsection*{\hspace*{\fill}§. 2.\hspace*{\fill}}

The \il{Russian}Russians reduce the \is{Declension!Noun declension}inflections of feminine nouns to two forms, namely those ending in a consonant, and those ending in a vowel:\footnote{\citet[table insert at 204--205]{puchmayer_lehrgebaude_1820}.} [55]

\begin{longtable}{ l l l }
    \lsptoprule
    \multicolumn{3}{ c }{Singular.} \\
    \midrule
    \textit{N}. & \textit{Voda} & \textit{Trost} \\
    & [‘water’], & (‘reed’). \\
    \textit{G}. & \textit{Vodi}, & \textit{Trosti}. \\
    \textit{D}. & \textit{Vodje}, & \textit{Trosti}. \\
    \textit{A}. & \textit{Vodu}, & \textit{Trost}. \\
    \textit{L}. & \textit{Vodje}, & \textit{Trosti}. \\
    \textit{I}. & \textit{Vodoju}, & \textit{Trostju}. \\
    \lspbottomrule
    \\
    \lsptoprule
    \multicolumn{3}{ c }{Plural.} \\
    \midrule
    \textit{N}. & \textit{Vodi}, & \textit{Trosti}. \\
    \textit{G}. & \textit{Vod}, & \textit{Trostej}. \\
    \textit{D}. & \textit{Vodam}, & \textit{Trostam}. \\
    \textit{A}. & \textit{Vodi}, & \textit{Trosti}. \\
    \textit{L}. & \textit{Vodax}, & \textit{Trostjax}. \\
    \textit{I}. & \textit{Vodami}, & \textit{Trostami}. \\
    \lspbottomrule
\end{longtable}

To these feminine nouns also belong a few masculine nouns, which end in \linebreak{}-\textit{a}, such as \textit{Vojvoda}, \textit{Sluga}, \textit{Vladika} (‘ruler’), \textit{sudja} (instead of \textit{sudjar}). Some \linebreak{}\il{Russian}Russian nouns ending in -\textit{a} are moreover particular to both masculine and feminine classes, such as: \textit{Zapivoxa}, ‘drinker’ either masculine or feminine, \textit{Obƶora} ‘glutton’, \textit{Kusaka} ‘biter’, \textit{Zajka} ‘one who hesitates in speech’. But it is more in agreement with the \is{Genius}genius of the language to attribute to masculine nouns a distinct ending with a consonant, but to feminine nouns with the vowel -\textit{a}, such as: \textit{Zapivox}, \textit{Zapivoxa}, \textit{obƶor}, \textit{obƶora} etc. because indeed the genitive plural is formed from the nominative plural with the deletion of the letter \textit{i}, as in \textit{vod} [‘of the waters’], \textit{pil} [‘of the saws’] etc., for \textit{vodi}, \textit{pili} etc. Yet if this deletion would leave a collision of consonants, they are mitigated by interjected vowels: this way, \textit{doska} ‘pole, post’, in the plural \textit{doski}, [56] would be \textit{dosk} by deletion of \textit{i}, but a vowel is interjected: \textit{dosok}, \textit{vodki} [‘vodkas’], \textit{vodok}, \textit{igli} [‘needles’], \textit{ikri} [‘caviars’] to \textit{igol}, \textit{ikor} etc., which happens also in other \is{Dialect}dialects. Hence, this \il{Russian}Russian form of \is{Declension!Noun declension}inflecting feminine nouns is so much in agreement with the \is{Genius}genius of the language and measured by the rules of Logic and Philology, that all \is{Dialect}dialects, if any would differ, could safely adopt it. For they \is{Declension!Noun declension}inflect the nouns ending in a vowel in the same way as the \il{Old Church Slavonic}ancient \is{Dialect}dialect, as is clear from the paradigm of both \is{Dialect}dialects. But they \is{Declension!Noun declension}inflect those consonantal endings with much the same form as the \il{Old Church Slavonic}ancient and the \is{Dialect!Living dialect}living dialects, so much so that even the \il{Windic}Winds concur on this form; this way, for instance, \textit{reч} [‘word’], G. \textit{reчi}, D. \textit{reчi}, A. \textit{reч}, L. \textit{reчi}, I. \textit{reчio} etc. is like \textit{trost} or \textit{kost} in the \il{Old Church Slavonic}ancient \is{Dialect}dialect. But the reason that the instrumental ends in -\textit{o} is that the southerners take delight in the letter -\textit{o} instead of -\textit{u}; the same occurs also in verbs, instead of \textit{budu} [‘to be’], \textit{budo}, instead of \textit{pisaju} [‘I write’], \textit{pisajo} etc.

\enlargethispage{4mm}

\subsection*{\hspace*{\fill}§. 3.\hspace*{\fill}}

The \il{Polish}Poles likewise do not differ essentially from this pattern, as, for instance:\footnote{\citet[84 (\textit{ryba}), 88 (\textit{kość})]{bandtkie_polnische_1808}.}

\begin{longtable}{ l l l }
    \lsptoprule
    \multicolumn{3}{ c }{Singular.} \\
    \midrule
    \textit{N}. & \textit{Riba} & \textit{Kośc} (\textit{Kost}) \\
    & [‘fish’], & [‘bone’]. \\
    \textit{G}. & \textit{Ribi}, & \textit{Kosci}. \\
    \textit{D}. & \textit{Ribie}, & \textit{Kosci}. \\ 
    \textit{A}. & \textit{Ribę}, (\textit{Ribu}) & \textit{Kość}. \\
    \textit{V}. & \textit{Ribo}, & \textit{Kośći}. \\
    \textit{I}. & \textit{Ribą} (\textit{Ribom}), & \textit{Koscią} (\textit{Kosciom}). [57] \\
    \textit{L}. & \textit{Ribie}, & \textit{Kośći}. \\
    \lspbottomrule
    \\
    \lsptoprule
    \multicolumn{3}{ c }{Plural.} \\
    \midrule
    \textit{N}. & \textit{Ribi}, & \textit{Kosci}. \\
    \textit{G}. & \textit{Rib}, & \textit{Kości}. \\
    \textit{D}. & \textit{Ribom}, & \textit{Kościom}. \\ 
    \textit{A}. & \textit{Ribi}, & \textit{Kośći}. \\
    \textit{V}. & \textit{Ribi}, & \textit{Kosći}. \\
    \textit{I}. & \textit{Ribami}, & \textit{Kośćiami}. \\
    \textit{L}. & \textit{Ribax}, & \textit{Kośćiax}. \\
    \lspbottomrule
\end{longtable}

\newpage

Yet the \is{Declension!Noun declension}inflection of feminine nouns is illustrated with 16 paradigms, which nonetheless do not at all differ in essence from these two abovementioned forms, except that some nouns ending in a vowel have the dative singular identical to the genitive. For \textit{ziemia} [‘earth’], the genitive and dative are \textit{ziemii}, but in other \is{Dialect}dialects it is \is{Declension!Noun declension}inflected as \textit{riba}, namely genitive \textit{ziemi}, dative \textit{ziemie}. Additionally, some nouns in the nominative plural are pronounced with -\textit{e}, as: \textit{ziemie}, \textit{suknie} [‘dress’], \textit{sije} (\textit{шije}) ‘necks’. Finally, the dative plural takes -\textit{om} instead of -\textit{am}, but these little things do not make any difference, for vowels are easily changed in the mouth of speakers, as both taught by experience and testified in the most ancient \il{Polish}Polish books, for in \il{Polish}Polish books many feminine nouns are read in the genitive singular with -\textit{e}, which are, however, now pronounced with -\textit{i}. For example, \textit{tvierdza} [‘fortress’] in the genitive used to be \textit{tvierdze}, but now is \textit{tvierdzi}; \textit{zemie} is now \textit{ziemi}, \textit{pivnice} [‘beerhouse’] is now \textit{pivnici}, \textit{krvie} [‘blood’] is now \textit{krvi}, and vice versa, those which prevailed in the 17\textsuperscript{th} century have disappeared, however, in the 18\textsuperscript{th}, such as \textit{siestra} [‘sister’], \textit{ƶenia} [‘wife’] are now [58] \textit{sostra}, \textit{ƶona} etc. The genitive plural is also read in the fashion of masculine nouns taking -\textit{ov}, which has now entirely vanished; furthermore, the dative plural is now commonly in -\textit{om} in the fashion of masculine nouns, but previously it was pronounced as -\textit{am}, as in \textit{ribam}, \textit{pivnicam} etc. instead of modern \textit{ribom}, \textit{pivnicom} etc. But this modest diversity should be derived only from varying usage, for just as the Polish nation has undergone various vicissitudes, in the same way this \il{Slavic}Slavic \is{Dialect}dialect, too, has had various mutations, to such an extent that the language of a people, when critically examined, provides a transparent history of that same people. For in the 16\textsuperscript{th} century, the language of the \il{Polish}Poles flourished greatly. Then civil wars immeasurably weakened the nation along with its language; domestic factions divided it still more; contact with the \il{French}French has left a powerful mark on the very language itself, etc. Yet recent times promise a lot of good, for ample and at the same time powerful societies devoted to elevating the national culture are emerging, such as the Warsaw society \textit{Tovaristvo osviati, i culturi narodnej} [‘Society for national enlightenment and culture’].

\newpage

\subsection*{\hspace*{\fill}§. 4.\hspace*{\fill}}

The \il{Bohemian}Bohemians establish three declensions to \is{Declension!Noun declension}inflect feminine nouns.

\begin{longtable}{ l l l l }
    \lsptoprule
    \multicolumn{4}{ c }{Singular.} \\
    \midrule
    \textit{N}. & \textit{Kost} & \textit{Riba} & \textit{Zeme} \\
    & [‘bone’], & [‘fish’], & [‘land’]. \\
    \textit{G}. & \textit{Kosti}, & \textit{Ribi}, & \textit{Zeme}. \\
    \textit{D}. & \textit{Kosti}, & \textit{Ribe}, & \textit{Zemi}. [59] \\ 
    \textit{A}. & \textit{Kost}, & \textit{Ribu}, & \textit{Zemi}. \\
    \textit{V}. & \textit{Kost}, & \textit{Ribo}, & \textit{Zemi}. \\
    \textit{L}. & \textit{Kosti}, & \textit{Ribe}, & \textit{Zemi}. \\
    \textit{I}. & \textit{Kosti}, & \textit{Ribau}, & \textit{Zemi}. \\
    \lspbottomrule
    \\
    \lsptoprule
    \multicolumn{4}{ c }{Plural.} \\
    \midrule
    \textit{N}. & \textit{Kosti}, & \textit{Ribi}, & \textit{Zemi}. \\
    \textit{G}. & \textit{Kosti}, & \textit{Rib}, & \textit{Zemi}. \\
    \textit{D}. & \textit{Kostem}, & \textit{Ribam}, & \textit{Zemim}. \\ 
    \textit{A}. & \textit{Kosti}, & \textit{Ribi}, & \textit{Zeme}. \\
    \textit{V}. & \textit{Kosti}, & \textit{Ribi}, & \textit{Zeme}. \\
    \textit{L}. & \textit{Kostex}, & \textit{Ribax}, & \textit{Zemix}. \\
    \textit{I}. & \textit{Kostmi}, & \textit{Ribami}, & \textit{Zememi}. \\
    \lspbottomrule
\end{longtable}

\textit{Kost} and \textit{riba} are \is{Declension!Noun declension}inflected in the same way in the \il{Old Church Slavonic}ancient and other \is{Dialect}dialects, except that the locative singular in the \il{Old Church Slavonic}ancient \is{Dialect}dialect is not read as \textit{kosti} but as \textit{kostiu}. With \textit{zeme}, which is \textit{zemlia} for the \il{Russian}Russians, \textit{ziemia} for the \il{Polish}Poles, \textit{zem} for the \il{Pannonian}Pannonians, the \il{Bohemian}Bohemians have changed the -\textit{a} into -\textit{e} according to the \is{Genius}genius of the \is{Dialect}dialect, even though the feminine nouns in -\textit{e} are not observed anywhere in the other \is{Dialect}dialects. The \il{Pannonian}Pannonians, according to usage, \is{Declension!Noun declension}inflect \textit{ziemia} both like \textit{kost} = \textit{zem} and like \textit{voda} = \textit{ziemia}.

\subsection*{\hspace*{\fill}§. 5.\hspace*{\fill}}

The \il{Pannonian}Pannonian grammarian proposes the \is{Declension!Noun declension}inflection of feminine nouns in three forms, namely:\footnote{\citet[37]{bernolak_grammatica_1790}.}

\begin{longtable}{ l l l l }
    \lsptoprule
    \multicolumn{4}{ c }{Singular.} \\
    \midrule
    \textit{N}. & \textit{Ovca} & \textit{Osoba} & \textit{Чnost} \\
    & [‘sheep’], & [‘person’], & [‘virtue’]. \\
    \textit{G}. & \textit{Ovce}, & \textit{Osobe}, & \textit{Чnosti}. \\
    \textit{D}. & \textit{Ovci}, & \textit{Osobe}, & \textit{Чnosti}. [60] \\ 
    \textit{A}. & \textit{Ovcu}, & \textit{Osobu}, & \textit{Чnost}. \\
    \textit{V}. & \textit{Ovco}, & \textit{Osobo}, & \textit{Чnost}. \\
    \textit{L}. & \textit{Ovci}, & \textit{Osobi}, & \textit{Чnosti}. \\
    \textit{I}. & \textit{Ovcu}, & \textit{Osobu}, & \textit{Чnostu}. \\
    \lspbottomrule
    \\
    \lsptoprule
    \multicolumn{4}{ c }{Plural.} \\
    \midrule
    \textit{N}. & \textit{Ovce}, & \textit{Osobi}, & \textit{Чnosti}. \\
    \textit{G}. & \textit{Ovec}, & \textit{Osob}, & \textit{Чnosti}. \\
    \textit{D}. & \textit{Ovcam}, & \textit{Osobam}, & \textit{Чnostam}. \\ 
    \textit{A}. & \textit{Ovce}, & \textit{Osobi}, & \textit{Чnosti}. \\
    \textit{V}. & \textit{Ovce}, & \textit{Osobi}, & \textit{Чnosti}. \\
    \textit{L}. & \textit{Ovcax}, & \textit{Osobax}, & \textit{Чnostax}. \\
    \textit{I}. & \textit{Ovcami}, & \textit{Osobami}, & \textit{Чnostami}. \\
    \lspbottomrule
\end{longtable}

The grammarian distinguishes various cases for the words \textit{ovca} and \textit{osoba}, and also for other feminine nouns anxiously included in this category. These distinctions certainly prevail in usage, but a uniform model of \is{Declension!Noun declension}inflection also prevails. It follows the paradigm \textit{voda} of the \il{Old Church Slavonic}ancient \is{Dialect}dialect, although \textit{чnost} follows \textit{kost} [‘bone’] in the \il{Old Church Slavonic}ancient \is{Dialect}dialect. In the instrumental singular, furthermore, one also says \textit{ovcov}, \textit{osobov}, \textit{kostov} etc., whence it is clear that the various endings of separate \is{Dialect}dialects are not founded in the \is{Genius}genius of the language but in variable usage. For what one \is{Dialect}dialect expresses with the ending -\textit{i}, another does with -\textit{e} and vice versa, a fact which the \il{Carinthian!Slavo-Carinthian}Slavo-Carinthian \is{Dialect}dialect will immediately confirm. Its most erudite grammarian\footnote{\citet[243 (\textit{riba}, \textit{voda}), 252 (\textit{misel})]{kopitar_grammatik_1808}.} illustrates the feminine nouns with these paradigms more than any other:

\newpage

\begin{longtable}{ l l l l }
    \lsptoprule
    \multicolumn{4}{ c }{Singular.} \\
    \midrule
    \textit{N}. & \textit{Riba} & \textit{Voda} & \textit{Misel} \\
    & [‘fish’], & [‘water’], & [‘thought’]. \\
    \textit{G}. & \textit{Ribe}, & \textit{Vode}, & \textit{Misli}. [61] \\
    \textit{D}. & \textit{Ribi}, & \textit{Vodi}, & \textit{Misli}. \\ 
    \textit{A}. & \textit{Ribo}, & \textit{Vodo}, & \textit{Misel}. \\
    \textit{L}. & \textit{Ribi}, & \textit{Vodi}, & \textit{Misli}. \\
    \textit{I}. & \textit{Ribo}, & \textit{Vodo}, & \textit{Mislio}. \\
    \lspbottomrule
    \\
    \lsptoprule
    \multicolumn{4}{ c }{Plural.} \\
    \midrule
    \textit{N}. & \textit{Ribe}, & \textit{Vode}, & \textit{Misli}. \\
    \textit{G}. & \textit{Rib}, & \textit{Vod}, & \textit{Misel}. \\
    \textit{D}. & \textit{Ribam}, & \textit{Vodam}, & \textit{Mislim}. \\ 
    \textit{A}. & \textit{Ribe}, & \textit{Vode}, & \textit{Misli}. \\
    \textit{L}. & \textit{Ribah}, & \textit{Vodah}, & \textit{Mislih}. \\
    \textit{I}. & \textit{Ribami}, & \textit{Vodami}, & \textit{Mislimi}. \\
    \lspbottomrule
\end{longtable}

Here, the \is{Genius}spirit of \il{Slavic}Slavic \is{Declension!Noun declension}inflection is present, yet \is{Dialect}dialectally mixed, as in the genitive the grammarian says \textit{ribe}, \textit{vode} instead of \textit{ribi}, \textit{vodi}, and vice versa the accusative \textit{ribo}, \textit{vodo} instead of \textit{ribu}, \textit{vodu} etc. Furthermore, in the nominative plural \textit{ribe}, \textit{vode} instead of \textit{ribi}, \textit{vodi} etc.

\subsection*{\hspace*{\fill}§. 6.\hspace*{\fill}}

The \il{Illyrian}Illyrians acknowledge an equally double \is{Declension!Noun declension}inflection, one of words ending in a vowel, the other of words ending in a consonant, even though they often change that consonant in a vowel, as follows:

\newpage

\begin{longtable}{ l l l }
    \lsptoprule
    \multicolumn{3}{ c }{Singular.} \\
    \midrule
    \textit{N}. & \textit{Muka} & \textit{Misao} (\textit{Misal}, \textit{Misel}) \\
    & [‘pain’], & [‘thought’]. \\
    \textit{G}. & \textit{Muke}, & \textit{Misli}. \\
    \textit{D}. & \textit{Muki}, & \textit{Misli}. \\ 
    \textit{A}. & \textit{Muki}, & \textit{Misao}. [62] \\
    \textit{V}. & \textit{Muko}, & \textit{Misli}. \\
    \textit{I}. & \textit{Mukom}, & \textit{Misli}, \textit{Mislu}.  \\
    \textit{L}. & \textit{Muci}, & \textit{Misli}. \\
    \lspbottomrule
    \\
    \lsptoprule
    \multicolumn{3}{ c }{Plural.} \\
    \midrule
    \textit{N}. & \textit{Muke}, & \textit{Misli}. \\
    \textit{G}. & \textit{Muka}, & \textit{Misli}. \\
    \textit{D}. & \textit{Mukama}, & \textit{Mislima}. \\ 
    \textit{A}. & \textit{Muke}, & \textit{Misli}. \\
    \textit{V}. & \textit{Muke}, & \textit{Misli}. \\
    \textit{I}. & \textit{Mukama}, & \textit{Mislima}. \\
    \textit{L}. & \textit{Mukama}, & \textit{Mislima}. \\
    \lspbottomrule
\end{longtable}

This \is{Declension!Noun declension}inflection differs from the other \is{Dialect}dialects in the following respect. It gives the instrumental singular with -\textit{om} instead of -\textit{u} for nouns ending in -\textit{a}, but in usage this ending, too, is distributed indiscriminately. The plural cases have been borrowed from the obsolete dual, except the genitive plural, to which the \il{Serbian}Serbian \is{Dialect}dialect assigns the ending -\textit{a}: as in \textit{muka} [‘fly’], \textit{noga} [‘leg, foot’], \textit{ƶena} [‘woman’], \textit{kniga} [‘book’], \textit{smija} [‘snake’], instead of \textit{muk}, \textit{nog}, \textit{ƶien}, \textit{knig}, \textit{vod}, \textit{smij} in other \is{Dialect}dialects etc. However, these endings also appear in various places in other \il{Illyrian}Illyrian \is{Dialect}dialects, such as \textit{vatra}, ‘pyre, fire’, \textit{vatier} etc.

Hence, from the \is{Genius!Genius of the Slavic language}genius of the \il{Slavic}Slavic language, and from prevailing usage, two forms for \is{Declension!Noun declension}inflecting feminine nouns are brought to light, one for those ending in a vowel, the other for those ending in a consonant. For instance:

\newpage

\begin{longtable}{ l l l l l }
    \lsptoprule
    & \multicolumn{2}{ c }{Singular.} & \multicolumn{2}{ c }{Plural.} \\
    \midrule
    \textit{N}. & \textit{Brana} & \textit{Milost} & \textit{Brani}, & \textit{Milosti}. \\
    & [‘gate’], & [‘mercy’]. & & \\
    \textit{G}. & \textit{Brani}, & \textit{Milosti}. & \textit{Bran}, & \textit{Milosti}. [63] \\
    \textit{D}. & \textit{Branie}, & \textit{Milosti}. & \textit{Branam}, & \textit{Milostiam}. \\ 
    \textit{A}. & \textit{Branu}, & \textit{Milost}. & \textit{Brani}, & \textit{Milosti}. \\
    \textit{V}. & \textit{Brano}, & \textit{Milost}. & \textit{Brani}, & \textit{Milosti}. \\
    \textit{L}. & \textit{Branie}, & \textit{Milosti}. & \textit{Branax}, & \textit{Milostiax}. \\
    \textit{I}. & \textit{Branu}, & \textit{Milostiu}. & \textit{Branami}, & \textit{Milostiami}. \\
    \lspbottomrule
\end{longtable}

But if we consider the \is{Dialect}dialects in greater detail still, we will observe that the same feminine nouns are pronounced with a consonant in one \is{Dialect}dialect, but in another with a vowel, such as \textit{milostia} [‘mercy’], \textit{postelia} [‘bed’], \textit{zemia} [‘earth’], instead of \textit{milost}, \textit{postel}, \textit{zem} etc., to such an extent that, on this ground, feminine nouns ending in a consonant in some \is{Dialect}dialects are also \is{Declension!Noun declension}inflected as if they ended in a vowel. Thus \textit{milostia} can also be delightfully \is{Declension!Noun declension}inflected like \textit{brana}, and in this way feminine nouns are strictly speaking reducible to one single form, as in \textit{N}. \textit{milost} or \textit{milostia}, \textit{G}. \textit{milosti}. \textit{D}. \textit{milostie}. \textit{A}. \textit{milostiu}. \textit{V}. \textit{milostio} etc. as with \textit{brana}.

\subsection*{\hspace*{\fill}§. 7.\hspace*{\fill}}

Some observations on the word \textit{mati} [‘mother’], \textit{matier}, \textit{mama}, \textit{matka}, \textit{majka}. Nearly all \is{Dialect}dialect grammarians give this word an irregular \is{Declension!Noun declension}declension; it is important, therefore, to inquire after the reason why. This word has the same root as \il{German}German \textit{Mutter} and \il{Greek}Greek \textit{Meter} [μήτηρ]; hence, it follows that these languages are bound to each other by an important degree of kinship, since the first idea of a child, namely that of the mother, is expressed with nearly the same sound. Let us now then make an enquiry into the origin of this word [64] \textit{mati}, \textit{mama} etc. Certainly it is derived from nothing else than the word \textit{mati}, \textit{imati}, ‘to have’, or ‘to be freed by birth’, hence the way of speaking that is everywhere common among the Slavs: \textit{чto} or \textit{чo}, \textit{co ma tvoja sestra? Sina, dceru} [‘what did your sister have? A son, a daughter’] etc. Hence, she who held, she who carried someone below her heart, is naturally called their \textit{mati}, \textit{mama}, \textit{matier} etc. Let now the \il{German}Germans explain the origin of their \textit{Mutter} and the \il{Greek}Greeks of their \textit{Meter}, and then we will see which of these languages is the most original.

The noun for ‘daughter’ in the \il{Old Church Slavonic}ancient \is{Dialect}dialect is \textit{dшчi}, among the \il{Windic}Winds \textit{чi}, among the \il{Illyrian}Illyrians \textit{kчi}, \textit{ktji}, among the \il{Russian}Russians \textit{doч} etc. The \is{Dialect}dialect grammarians also count it, like ‘mother’, among the irregular nouns, for the reason that in the other cases the letter \textit{r} is inserted, a letter which the \il{Polish}Pole, the \il{Bohemian}Bohemian, and the \il{Pannonian}Pannonian already express in the nominative, and more precisely the \il{Polish}Pole says \textit{cora}, or with a certain tenderness \textit{corka}, \textit{coruska}, the \il{Bohemian}Bohemians say \textit{dcera}, the \il{Pannonian}Pannonians \textit{cera}. Now, if \textit{cora} or \textit{cera} is taken, every irregularity vanishes, and the result is:

\begin{longtable}{ l l l }
    \lsptoprule
    \multicolumn{3}{ c }{Singular.} \\
    \midrule
    \textit{N}. & \textit{Cora}, & \textit{Matier}. \\
    \textit{G}. & \textit{Cori}, & \textit{Matieri}. \\
    \textit{D}. & \textit{Core}, & \textit{Matieri}. \\ 
    \textit{A}. & \textit{Coru}, & \textit{Matier}, or \textit{Matieru}. \\
    \textit{V}. & \textit{Coro}, & \textit{Matiero}. \\
    \textit{L}. & \textit{Core}, & \textit{Matiere}.  \\
    \textit{I}. & \textit{Coru}, & \textit{Matieru}. \\
    \lspbottomrule
    \\
    \lsptoprule
    \multicolumn{3}{ c }{Plural.} \\
    \midrule
    \textit{N}. & \textit{Cori}, & \textit{Matieri}. [65] \\
    \textit{G}. & \textit{Cor}, & \textit{Matier}. \\
    \textit{D}. & \textit{Coram}, & \textit{Matieram}. \\ 
    \textit{A}. & \textit{Cori}, & \textit{Matieri}. \\
    \textit{V}. & \textit{Cori}, & \textit{Matieri}. \\
    \textit{L}. & \textit{Corax}, & \textit{Matierax}. \\
    \textit{I}. & \textit{Corami}, & \textit{Matierami}. \\
    \lspbottomrule
\end{longtable}

Likewise \textit{mama}, \textit{mami}, \textit{mami}, \textit{mamu}, \textit{mamo} etc.

There would surely be present both pleasantness and regularity, confirmed by the usage of the \il{Polish}Poles, \il{Bohemian}Bohemians, and \il{Pannonian}Pannonians; for the power of expressions does not consist in a multitude of rules weakened by exceptions and excep-\linebreak{}tions to exceptions, but in the \is{Genius}genius of the language itself. But it is the \is{Genius!Genius of the Slavic language}genius of the \il{Slavic}Slavic language to express at the same time affection of mind toward the object of speech, or alienation and contempt; hence emerge so-called diminutive expressions of tenderness, augmentatives, or contemptives, e.g. \textit{mama}, \textit{mamka}, \textit{mamiчka}, \textit{maminka}, \textit{mamuшka}, \textit{maminienka} are expressions of daughterly tenderness and are so diverse that I can neither express nor circumscribe these \linebreak{}expressions with \il{Latin}Latin words, as these things are understood by native Slavs.{\enlargethispage{\baselineskip}\footnote{The phrase “these things are understood by native Slavs” appears in \ia{Herkel, Jan}Herkel’s \il{Latin}Latin original as \textit{res natis nota est Slavis}. The possibility of a typographical error exists. The phrase \textit{res satis nota est Slavis} means ‘these things are sufficiently known to Slavs’, and would better match \il{Latin}Latin word order. \ia{Buzássyová, Ľudmila}Buzássyová’s \il{Slovak}Slovak translation is \textit{Sú to jednoducho veci známe rodeným Slovanom}, which might be glossed as ‘these are easy things known by every native Slav’. See \citet[95]{herkel_jan_2009}.}} The Slavs’ natural and original poetry is full of similar expressions, since expressions of this kind are very powerful for declaring one’s mood and giving color to an object.

\begin{small}
\begin{longtable}{ l l }
    \multicolumn{2}{ c }{\textit{\il{Russian}Russian song}.\footnote{\ia{Herkel, Jan}Herkel provided \il{Latin}Latin glosses for some \il{Slavic}Slavic words, which we offer in \il{English}English translation between brackets, as in \ia{Herkel, Jan}Herkel’s original. We offer an \il{English}English translation in a separate column.}} \\
    \\
    Vstala ja mlada mladenka, & {[}A young woman, I got up \\
    \hspace{0.5cm} Vstavala ranenko, [66] & \hspace{0.5cm} got up in the morning \\
    Po jutru rano vstavala, & I got up early in the morning \\
    \hspace{0.5cm} Druga (‘lover’) provadzala. & \hspace{0.5cm} To accompany my lover. \\
    Na kryleчuшke (‘threshold’) stojala, & I stood on the threshold \\
    \hspace{0.5cm} Platoчkom (‘kerchief’) maxala (‘to wave’) & \hspace{0.5cm} and waved with a kerchief \\
    Ja platoчkom to maxala,	& I waved with a kerchief \\
    \hspace{0.5cm} Чto by mil vrotil sia. & \hspace{0.5cm} So that he would come back \\
    Vroti sia moja nadeƶa, & Come back, my hope \\
    \hspace{0.5cm} Vroti sia serdeчko etc.{\enlargethispage{\baselineskip}\footnote{Excerpt from \textit{Roztaužená}, \citet[112]{celakovsky_slowanske_1822}.}} & \hspace{0.5cm} Come back, my heart etc.{]}\\
\end{longtable}

\begin{longtable}{ l l }
    \multicolumn{2}{ c }{\il{Bohemian}\textit{Bohemian song}.} \\
    \\
    Ukazte mi tu cestiчku, & [Show me the little path \\
    \hspace{0.5cm} Kadi nesli mu holчiчku, & \hspace{0.5cm} Where they took my little girl \\
    Cestiчka je provedoma,\footnote{\textit{provedoma} is probably a typo for \textit{povědoma}.} & The little path is well-known \\ 
    \hspace{0.5cm} Rozmarinku propletena. & \hspace{0.5cm} Entwined by rosemary. \\
    Ukaƶte mi kosteliчek, & Show me the little church \\
    \hspace{0.5cm} Kde leƶi moj Andieliчek etc.\footnote{Cf. \citet[4]{celakovsky_slowanske_1822}; \citet[97]{zbirt_bibliograficky_1895}.} & \hspace{0.5cm} Where my little angel lies etc.]
\end{longtable}
\end{small}

\section*{\textit{On the inflection of neuter nouns}. [p. 66--76]}
\addcontentsline{toc}{section}{\indent On the inflection of neuter nouns. [p. 66--76]}

\subsection*{\hspace*{\fill}§. 1.\hspace*{\fill}}

The ending of neuter nouns based on the \is{Genius}genius of the language is in -\textit{o}, or \linebreak{}-\textit{e}, and those denoting the young of animate beings or young animals ending in \linebreak{}-\textit{a}, such as \textit{telia} [‘calf’], \textit{dieta} [‘child’], \textit{gusia} [‘goslings’], \textit{jagnia} [‘lamb’], \textit{ƶrebia} [‘foal’] etc., or ending in -\textit{e}, such as \textit{telie}, \textit{dietie} etc., and these additionally take on an additional letter -\textit{t} in the remaining cases. But most nouns ending in -\textit{e} are verbal nouns, such as \textit{pisanie} [‘writing’], \textit{oranie} [‘ploughing’] etc. But they are \is{Declension!Noun declension}inflected in agreement with the \il{Old Church Slavonic}ancient \is{Dialect}dialect in the following way according to the most illustrious grammarian \textit{Dobrovszki}:{\enlargethispage{\baselineskip}\footnote{\citet[474--475]{dobrovsky_institutiones_1822}.}}

\begin{longtable}{ l l l l }
    \lsptoprule
    \multicolumn{4}{ c }{Singular.} \\
    \midrule
    \textit{N}. \textit{A}. \textit{V}. & \textit{Slovo} & \textit{Lice} & \textit{Uчenie} \\
    & [‘work’], & [‘cheek’], & [‘doctrine’]. \\
    \textit{G}. & \textit{Slova}, & \textit{Lica}, & \textit{Uчenija}. [67] \\
    \textit{D}. & \textit{Slovu}, & \textit{Licu}, & \textit{Uчeniju}. \\ 
    \textit{L}. & \textit{Slovie}, & \textit{Lici}, & \textit{Uчeniji}. \\
    \textit{I}. & \textit{Slovom}, & \textit{Licem}, & \textit{Uчenijem}. \\
    \lspbottomrule
    \\
    \lsptoprule
    \multicolumn{4}{ c }{Plural.} \\
    \midrule
    \textit{N}. \textit{A}. \textit{V}. & \textit{Slova}, & \textit{Lica}, & \textit{Uчenija}. \\
    \textit{G}. & \textit{Slov}, & \textit{Lic}, & \textit{Uчenij}. \\
    \textit{D}. & \textit{Slovom}, & \textit{Licjem}, & \textit{Uчenijem}. \\
    \textit{L}. & \textit{Slovix}, & \textit{Licjex}, & \textit{Uчeniix}. \\
    \textit{I}. & \textit{Slovi}, & \textit{Lici}, & \textit{Uчenii}. \\
    \lspbottomrule
\end{longtable}

This way of \is{Declension!Noun declension}inflecting corresponds to the \is{Dialect}dialects that are still alive now; in the meantime, it should be remarked that the instrumental plural is shown only in abbreviated fashion in the examples, namely: \textit{slovi}, \textit{lici} etc., for the full instrumental would be \textit{slovami}, \textit{licami}, \textit{uчenimi}; this syncope is also observed in the \is{Dialect!Living dialect}living dialects, especially \il{Bohemian}Bohemian and \il{Pannonian}Pannonian, without, however, rejecting the full expression, namely \textit{slovmi}, \textit{licami}, \textit{uчenimi}. That the very same thing occurred also among the ancients is clearly indicated by the most \il{Old Church Slavonic}ancient Bible, in which both the abbreviated and the full instrumental is read, a fact which the most illustrious \textit{Dobrovski} \ia{Dobrovský, Josef}on his part acknowledges, such as: \textit{igranmi} [‘with the games’], \textit{bezzakonmi} [‘with the criminals’], \textit{znamenmi} [‘with the signs’]. Furthermore, the locative is shown in -\textit{ex}, but the endings -\textit{ox}, -\textit{ax} are also abundant in the codices, for vowels change very easily in a \il{Slavic}Slavic mouth. This is confirmed not only by the usage of the \is{Dialect}dialects, but also by people of the same \is{Dialect!Living dialect}dialect living in each other’s neighborhood. In Pannonia, for instance, some say \textit{v Slovjex} [‘in the words’], others \textit{Slovax}, still others \textit{Slovox} etc. I believe that the same surely occurred also among the \il{Old Church Slavonic}ancients, which is why [68] grammarians torture themselves in vain by devising rules about when the vowel -\textit{e}, or -\textit{a}, or -\textit{o} should be put before the characteristic -\textit{x} of the instrumental.

\subsection*{\hspace*{\fill}§. 2.\hspace*{\fill}}

The \il{Russian}Russians \is{Declension!Noun declension}inflect neuter nouns in the following way:\footnote{\citet[table insert at 204--205]{puchmayer_lehrgebaude_1820}.}

\begin{longtable}{ l l l l l }
    \lsptoprule
    & \multicolumn{2}{ c }{Singular.} & \multicolumn{2}{ c }{Plural.} \\
    \midrule
    \textit{N}. \textit{A}. \textit{V}. & \textit{Dielo} & \textit{More} & \textit{Diela}, & \textit{Morja}. \\
    & [‘thing, affair’], & [‘sea’], & & \\
    \textit{G}. & \textit{Diela}, & \textit{Mora}, & \textit{Diel}, & \textit{Morej}. \\
    \textit{D}. & \textit{Dielu}, & \textit{Moru}, & \textit{Dielam}, & \textit{Morjam}.  \\ 
    \textit{L}. & \textit{Dielje}, & \textit{Morje}, & \textit{Dielax}, & \textit{Morjax}. \\
    \textit{I}. & \textit{Dielom}, & \textit{Morem}, & \textit{Dielami}, & \textit{Morjami}. \\
    \lspbottomrule
\end{longtable}

This way of \is{Declension!Noun declension}inflecting also corresponds to other \is{Dialect}dialects, but the grammarian observes some exceptions, namely: contemptuous augmentatives ending in -\textit{sko} form the nominative plural with -\textit{i}, for instance \textit{domisko} [from \textit{dom} ‘house’] or \textit{domiшko} in the nom. pl. is \textit{domiшki} instead of \textit{domiska} as in the other \is{Dialect}dialects, and also \textit{domiшчe}, \textit{domiшчi} instead of \textit{domiшчa}.

\subsection*{\hspace*{\fill}§. 3.\hspace*{\fill}}

The \il{Polish}Pole \is{Declension!Noun declension}inflects neuter nouns in the following way:{\footnote{\citet[106--107]{bandtkie_polnische_1808}.}}

\newpage

\begin{longtable}{ l l l l }
    \lsptoprule
    \multicolumn{4}{ c }{Singular.} \\
    \midrule
    \textit{N}. \textit{A}. \textit{V}. & \textit{Pole} & \textit{Kazanie} & \textit{Slovo} \\
    & [‘field’], & [‘sermon’], & [‘word’]. \\
    \textit{G}. & \textit{Pola}, & \textit{Kazania}, & \textit{Slova}. \\
    \textit{D}. & \textit{Polu}, & \textit{Kazaniu}, & \textit{Slovu}. \\ 
    \textit{In}. & \textit{Polem}, & \textit{Kazaniem}, & \textit{Slovem}. \\
    \textit{L}. & \textit{v Polu}, & \textit{Kazaniu}, & \textit{Slovje}. [69] \\
    \lspbottomrule
    \\
    \lsptoprule
    \multicolumn{4}{ c }{Plural.} \\
    \midrule
    \textit{N}. \textit{A}. \textit{V}. & \textit{Pola}, & \textit{Kazania}, & \textit{Slova}. \\
    \textit{G}. & \textit{Pol}, & \textit{Kazan}, & \textit{Slov}. \\
    \textit{D}. & \textit{Polam}, & \textit{Kazaniam}, & \textit{Slovam}. \\
    \textit{In}. & \textit{Polami}, & \textit{Kazaniami}, & \textit{Slovami}. \\
    \textit{L}. & \textit{Polax}, & \textit{Kazaniax}, & \textit{Slovax}. \\
    \lspbottomrule
\end{longtable}

The \il{Polish}Poles’ way of \is{Declension!Noun declension}inflecting corresponds to the \il{Russian}Russian. A small observation about the locative or prepositional case: the \il{Russian}Russians form the singular with -\textit{e}, but the \il{Polish}Poles sometimes with -\textit{e}, but sometimes with -u. The grammarian also tries to carve out rules, specifically for those ending in -\textit{e}, but these rules cannot have any stability even among the \il{Polish}Poles themselves, since \il{Polish}Polish writers of diverse periods have written in diverse ways. Yet from the \is{Genius}genius of the language it seems that the locative singular is identical to the dative. This is confirmed by the usage of all \is{Dialect}dialects, especially the southern, which is why the locative singular is either pronounced with -\textit{e} in the fashion of the \il{Russian}Russians, even though also among them the usage is mixed. It sometimes ends with -\textit{e}, and other times with -\textit{u}, even if the grammarian makes no mention of this usage, or is written with -\textit{u} in the fashion of the southerners. But the text of a song teaches us that the \il{Russian}Russians also pronounce the locative with -\textit{u}:\footnote{Since the \il{Russian}Russian preposition \textit{по} always takes the dative case, \ia{Herkel, Jan}Herkel has here not made a persuasive case that the locative resembles the dative.} \\

\noindent\hspace*{1cm}\textit{Ti razmiч} (a) \textit{moju kruчinu} (b) \textit{po чistomu poliu} etc. \\
\hspace*{1cm}[‘Spread you my grief along the open field’] \\
\hspace*{1cm}(a) \textit{razmitati} ‘to dissipate’ (b) \textit{kruчinu} ‘grief, sadness’ \\

\newpage

\subsection*{\hspace*{\fill}§. 4.\hspace*{\fill}}

The \il{Bohemian}Bohemians have four distinct forms for neuter \is{Declension!Noun declension}inflections, and those are:\footnote{\citet[table insert at 234--235]{dobrovsky_ausfuhrliches_1809}.} [70]

\begin{longtable}{ l l l l l }
    \lsptoprule
    \multicolumn{5}{ c }{Singular.} \\
    \midrule
    \textit{N}. \textit{A}. \textit{V}. & \textit{Pole} & \textit{Slovo} & \textit{Znameni} & \textit{Hause} \\
    & [‘field’], & [‘word’], & [‘sign’], & [‘gosling’]. \\
    \textit{G}. & \textit{Pole}, & \textit{Slova}, & \textit{Znameni}, & \textit{Hausete}. \\
    \textit{D}. & \textit{Poli}, & \textit{Slovu}, & \textit{Znameni}, & \textit{Hauseti}.  \\ 
    \textit{L}. & \textit{Poli}, & \textit{Slove}, & \textit{Znameni}, & \textit{Hauseti}. \\
    \textit{I}. & \textit{Polem}, & \textit{Slovem}, & \textit{Znamenim}, & \textit{Hausetem}. \\
    \lspbottomrule
    \\
    \lsptoprule
    \multicolumn{5}{ c }{Plural.} \\
    \midrule
    \textit{N}. \textit{A}. \textit{V}. & \textit{Pole}, & \textit{Slova}, & \textit{Znameni}, & \textit{Hausata}. \\
    \textit{G}. & \textit{Poli}, & \textit{Slov}, & \textit{Znameni}, & \textit{Hausat}. \\
    \textit{D}. & \textit{Polim}, & \textit{Slovum}, & \textit{Znamenim}, & \textit{Hausatum}.  \\ 
    \textit{L}. & \textit{Polix}, & \textit{Slovix}, & \textit{Znamenix}, & \textit{Hausatex}. \\
    \textit{I}. & \textit{Poli}, & \textit{Slovi}, & \textit{Znamenimi}, & \textit{Hausati}. \\
    \lspbottomrule
\end{longtable}

These four distinct \is{Declension!Noun declension}declensions are reduced in the other \is{Dialect}dialects to one norm, as is clear from the \il{Russian}Russian and \il{Polish}Polish forms, which the \il{Illyrian}Illyrian \is{Dialect}dialects also resemble.

\il{Bohemian}Bohemian \textit{hause}, and similar terms for young animals, are of neuter gender, and as far as concerns the form of the \is{Declension!Noun declension}declension, do not differ from other nouns. For this reason, nouns of this kind are only mentioned by grammarians because they receive the addition of one syllable in the oblique cases, such as: \textit{gusja} [‘goose’], \textit{ƶerebja} [‘foal’], \textit{golubja} [‘young pigeon’], \textit{ovчja} [‘lamb’], \textit{oslia} [‘little donkey’], \textit{otroчia} [‘small child’] etc. They are regularly declined like \textit{slovo}, N. \textit{golubja}, G. \textit{golubiata}, D. \textit{golubiatu} etc. \textit{gusjatu}, \textit{dietatu} [‘child’], \textit{tieliatu} [‘calf’]. This category contains some inanimate nouns which in certain \is{Dialect}dialects receive a suffix even in the nominative. For example, \textit{ramje} [‘shoulder’], \textit{semje} [‘seed’], \textit{imje} [‘name’], \textit{nebje} [‘heaven’], \textit{kolje} [71] (\textit{kolo}) [‘wheel’] are in the genitive \textit{remjena}, \textit{semjena}, \textit{imjena}, \textit{nebjesa}, \textit{koljesa}; but other \is{Dialect}dialects already have the suffix in the nominative, \textit{ramieno}, \textit{semieno}, \textit{imieno}, \textit{nebeso}, \textit{koleso}, and are hence declined following the pattern of \textit{slovo}, without any remark or exception.

\subsection*{\hspace*{\fill}§. 5.\hspace*{\fill}}

The \il{Pannonian}Pannonians’ neuter way of \is{Declension!Noun declension}inflecting shown in the grammar is the following:\footnote{\citet[41--42]{bernolak_grammatica_1790}.}

\begin{longtable}{ l l l l }
    \lsptoprule
    \multicolumn{4}{ c }{Singular.} \\
    \midrule
    \textit{N}. \textit{A}. \textit{V}. & \textit{Stavani} or (\textit{Stavana}) & \textit{Kura} & \textit{Srdce} \\
    & [‘building’], & [‘chicken’], & [‘heart’]. \\
    \textit{G}. & \textit{Stavani}, or -\textit{a} & \textit{Kurata}, & \textit{Srdca}. \\
    \textit{D}. & \textit{Stavani}, or -\textit{u} & \textit{Kuratu} (\textit{i}), & \textit{Srdcu} (\textit{i}). \\ 
    \textit{L}. & \textit{Stavani}, & \textit{Kuratu}, & \textit{Srdci}. \\
    \textit{I}. & \textit{Stavanim}, or -\textit{om}, & \textit{Kuratom}, & \textit{Srdcom}. \\
    \lspbottomrule
    \\
    \lsptoprule
    \multicolumn{4}{ c }{Plural.} \\
    \midrule
    \textit{N}. \textit{A}. \textit{V}. & \textit{Stavani}, (\textit{a}) & \textit{Kurata}, & \textit{Srdca}. \\
    \textit{G}. & \textit{Stavani}, & \textit{Kurat}, & \textit{Srdc}. \\
    \textit{D}. & \textit{Stavanim}, & \textit{Kuratam}, & \textit{Srdcam}. \\
    \textit{L}. & \textit{Stavanix} (\textit{ax}), & \textit{Kuratax}, & \textit{Srdcax}. \\
    \textit{I}. & \textit{Stavanmi}, & \textit{Kuratami}, & \textit{Srdcmi}. \\
    \lspbottomrule
\end{longtable}

The grammarian fixes three forms, but the observations of the grammarian himself already indicate that there must be only one. For the original ending is neither \textit{stavan} nor \textit{stavana} but \textit{stavanje}, and it is \is{Declension!Noun declension}inflected in the same form as the other \is{Dialect}dialects, since \textit{stavani}, namely with the -\textit{e} omitted, is the more recent \il{Bohemian}Bohemian ending.

The dative singular is exposed in two ways, namely [72] -\textit{u} and -\textit{i}, but the locative also takes this double ending, as proved by everyday use, since it is said: \textit{v srdcu} or \textit{srdci}, \textit{v kuratu} or \textit{kurati}. In addition, \textit{stavani} is put in the genitive plural only in \il{Bohemian}Bohemian fashion, but it is originally also said \textit{stavian} in the same form as the other \is{Dialect}dialects. And for this reason, one should be wary about following the grammarian of a certain \is{Dialect}dialect strictly, since \is{Dialect}dialect grammars rest not so much on the \is{Genius}genius of the language as on partial usage. It is therefore important to produce a grammar measured by the rules of Logic and Criticism, as Logic urges not to multiply rules unnecessarily -- -- otherwise we create unprofitable difficulties for ourselves and our descendants. One should therefore speak in conformity to the \is{Genius!Genius of the Slavic language}genius of the \il{Slavic}Slavic language and the other \is{Dialect}dialects.

\begin{longtable}{ l l l l }
    \lsptoprule
    \multicolumn{4}{ c }{Singular.} \\
    \midrule
    \textit{N}. \textit{A}. \textit{V}. & \textit{Stavanje}, & \textit{Serdce}, & \textit{Kurja}. \\
    \textit{G}. & \textit{Stavania}, & \textit{Serdca}, & \textit{Kurjata}. \\
    \textit{D}. & \textit{Stavaniu}, & \textit{Serdcu}, & \textit{Kurjatu}. \\ 
    \textit{L}. & \textit{v Stavaniu}, & \textit{Serdcu}, & \textit{Kurjatu}. \\
    \textit{I}. & \textit{Stavanim} (\textit{om}), & \textit{Serdcom}, & \textit{Kurjatom}. \\
    \lspbottomrule
    \\
    \lsptoprule
    \multicolumn{4}{ c }{Plural.} \\
    \midrule
    \textit{N}. \textit{A}. \textit{V}. & \textit{Stavania}, & \textit{Serdca}, & \textit{Kuriata}. \\
    \textit{G}. & \textit{Stavan}, & \textit{Serdec}, & \textit{Kuriat}. \\
    \textit{D}. & \textit{Stavaniom}, or -\textit{am}. \\
    \textit{L}. & \textit{Stavaniox}, or -\textit{ix}, or -\textit{ax}. \\
    \textit{I}. & \textit{Stavanami}. \\
    \lspbottomrule
\end{longtable}

Behold: a single \is{Declension!Noun declension}declension system, a regular one at that, and even sanctioned by usage! But the fact that [73] a double ending is in usage for certain cases, does not violate the general rule, but rather confirms it, just as the fact that vowels usually change in various \is{Dialect}dialects, as for instance with \textit{v stavanix}, \textit{stavanax}, \textit{stavanox}. A change of vowels of this kind was also known to the \il{Old Church Slavonic}ancient Slavs: this way, \textit{loƶesnex} [‘in the wombs’] is elsewhere in fact read \textit{loƶesnax}, \textit{serdcix}, \textit{serdcjex}, \textit{serdcjax}, \textit{serdcox}, \textit{bratox} [‘in the brothers’], \textit{koljenox} [‘in the knees’], \textit{selox} [‘in the villages’] etc. Nevertheless, the typical locative ending -\textit{x} always remains.

\newpage

\subsection*{\hspace*{\fill}§. 6.\hspace*{\fill}}

Let us look at the southern \is{Dialect}dialects, for instance \il{Serbian}Serbian.\footnote{\citet[xvi]{karadzic_srpski_1818}.}

\begin{longtable}{ l l l l }
    \lsptoprule
    \multicolumn{4}{ c }{Singular.} \\
    \midrule
    \textit{N}. \textit{A}. \textit{V}. & \textit{Pole} & \textit{Sretenje} & \textit{Ime} \\
    & [‘field’], & [‘meeting’], & [‘name’]. \\
    \textit{G}. & \textit{Pola}, & \textit{Sretenja}, & \textit{Imena}. \\
    \textit{D}. & \textit{Polu}, & \textit{Sretenju}, & \textit{Imenu}. \\ 
    \textit{I}. & \textit{Polem}, & \textit{Sretenjem}, & \textit{Imenom}. \\
    \textit{L}. & \textit{Polu}, & \textit{Sretenju}, & \textit{Imenu}. \\
    \lspbottomrule
    \\
    \lsptoprule
    \multicolumn{4}{ c }{Plural.} \\
    \midrule
    \textit{N}. \textit{A}. \textit{V}. & \textit{Pola}, & \textit{Sretenja}, & \textit{Imena}. \\
    \textit{G}. & \textit{Pola}, & \textit{Sretenja}, & \textit{Imena}. \\
    \textit{D}. & \textit{Polima}, & \textit{Sretinima}, & \textit{Imenima}. \\
    \textit{I}. & \textit{Polima}. & — & — \\
    \textit{L}. & \textit{Polima}. & — & — \\
    \lspbottomrule
\end{longtable}

The singular corresponds entirely to the other \is{Dialect}dialects, but the plural is the ancient dual, with the exception of the genitive, which absolutely does not correspond [74] to the \is{Genius!Genius of the Slavic language}genius of the \il{Slavic}Slavic language. Therefore, it also differs from all \is{Dialect}dialects. Even the \il{Slavonian}Slavonians, the \il{Serbian}Serbs’ neighbors, differ on this point.\footnote{\citet[39 (\textit{vrime}), 40 (\textit{serdce}, \textit{pivanje})]{lanosovic_anleitung_1795}.}

\begin{longtable}{ l l l l }
    \lsptoprule
    \multicolumn{4}{ c }{Singular.} \\
    \midrule
    \textit{N}. \textit{A}. \textit{V}. & \textit{Vreme} & \textit{Serdce} & \textit{Pivanje} \\
    & [‘time’], & [‘heart’], & [‘drinking’]. \\
    \textit{G}. & \textit{Vremena}, & \textit{Serdca}, & \textit{Pivanja}. \\
    \textit{D}. & \textit{Vremenu}, & \textit{Serdcu}, & \textit{Pivanju}. \\ 
    \textit{L}. & \textit{Vremenu}, & \textit{Serdcu}, & \textit{Pivanju}. \\
    \textit{I}. & \textit{Vremenom}, & \textit{Serdcom}, & \textit{Pivanjem}. \\
    \lspbottomrule
    \\
    \lsptoprule
    \multicolumn{4}{ c }{Plural.} \\
    \midrule
    \textit{N}. \textit{A}. \textit{V}. & \textit{Vremena}, & \textit{Serdca}, & \textit{Pivanja}. \\
    \textit{G}. & \textit{Vremenah}, & \textit{Serdcah}, & \textit{Pivanjeh}. \\
    \textit{D}. & \textit{Vremenom}, & \textit{Serdcom}, & \textit{Pivanjim}. \\
    \textit{L}. & \textit{Vremenah}, & \textit{Serdcih}, & \textit{Pivanjim}. \\
    \textit{I}. & \textit{Vremenama}, & \textit{Serdcima}, & \textit{Pivanima}. \\
    \lspbottomrule
\end{longtable}

The singular corresponds entirely with the other \is{Dialect}dialects, but some plural endings are taken from the dual, such as \textit{serdcama}. The genitive plural is pronounced with \textit{x} = \textit{h}, an ending common to all genders in the genitive plural, but this way of speaking can be reconciled neither with the \is{Genius}genius of the language, nor with the \is{Dialect!Living dialect}living dialects. The reason is that the entire plural is a mixture of the plural, the dual, and the \is{Declension!Adjective declension}inflection of the adjectives, for all adjectives have in the genitive plural the ending -\textit{x} (= \textit{h}).

Since among the \il{Windic}Winds the dual number is still alive, and accurately distinguished from the plural, let us now look at their most illustrious grammarian, and the greatest investigator of the \is{Genius!Genius of the Slavic language}genius of the \il{Slavic}Slavic language.{\enlargethispage{\baselineskip}\footnote{\citet[237 (\textit{serza}, \textit{délo}), 240--241 (\textit{séme})]{kopitar_grammatik_1808}.}} [75]

\begin{longtable}{ l l l l }
    \lsptoprule
    \multicolumn{4}{ c }{Singular.} \\
    \midrule
    \textit{N}. \textit{A}. \textit{V}. & \textit{Serce} & \textit{Delo} & \textit{Seme} \\
    & [‘heart’], & [‘work’], & [‘seed’]. \\
    \textit{G}. & \textit{Serca}, & \textit{Dela}, & \textit{Semena}. \\
    \textit{D}. & \textit{Sercu}, & \textit{Delu}, & \textit{Semenu}. \\ 
    \textit{L}. & \textit{Sercu} (\textit{i}), & \textit{Delu} (\textit{i}), & \textit{Semenu}. \\
    \textit{I}. & \textit{Sercom}, & \textit{Delom}, & \textit{Semenom}. \\
    \lspbottomrule
    \\
    \lsptoprule
    \multicolumn{4}{ c }{Plural.} \\
    \midrule
    \textit{N}. \textit{A}. \textit{V}. & \textit{Serca}, & \textit{Dela}, & \textit{Semena}. \\
    \textit{G}. & \textit{Serc}, & \textit{Del}, & \textit{Semen}. \\
    \textit{D}. & \textit{Sercam}, & \textit{Delam}, & \textit{Semenam}. \\
    \textit{L}. & \textit{Sercih}, & \textit{Delih}, & \textit{Semenih}. \\
    \textit{I}. & \textit{Serci}, & \textit{Delmi}, & \textit{Semeni}. \\
    \lspbottomrule
\end{longtable}

This way of \is{Declension!Noun declension}inflecting, proposed by the renowned \ia{Kopitar, Jernej}Kopitar, agrees with the other \is{Dialect}dialects, and moreover with the \is{Genius!Genius of the Slavic language}genius of the \il{Slavic}Slavic language. He presents the instrumental \textit{serci} only in the abbreviated form, for \textit{serci}, \textit{sercmi}, or \textit{sercami} is the same form, as the other grammarian of the same \is{Dialect}dialect presents the instrumental in full: \textit{letami} [‘by years’], instead of \textit{letmi}, or \textit{leti}.\footnote{\citet[151]{dainko_lehrbuch_1824}.} From all this it is clear that for the \is{Declension!Noun declension}inflection of neuter nouns there is only one unique form from the \is{Genius}genius of the language, and namely: 

\begin{longtable}{ l l l }
    \lsptoprule
    \multicolumn{3}{ c }{Singular.} \\
    \midrule
    \textit{N}. \textit{A}. \textit{V}. & — & \textit{Serce}. \\
    \textit{G}. & -\textit{a} & \textit{Serca}. \\
    \textit{D}. & -\textit{u} & \textit{Sercu}. \\ 
    \textit{L}. & -\textit{u} & \textit{Sercu}. \\
    \textit{I}. & -\textit{om} & \textit{Sercom}. \\
    \lspbottomrule
    \\
    \lsptoprule
    \multicolumn{3}{ c }{Plural.} \\
    \midrule
    \textit{N}. \textit{A}. \textit{V}. & -\textit{a} & \textit{Serca}. [76] \\
    \textit{G}. & — & \textit{Serc}. \\
    \textit{D}. & -\textit{am} & \textit{Sercam}. \\
    \textit{L}. & -\textit{ax}, -\textit{ix}, -\textit{ox} & \textit{Sercax}, \textit{Sercix}, \textit{Sercox}. \\
    \textit{I}. & -\textit{ami} & \textit{Sercami}. \\
    \lspbottomrule
\end{longtable}

\section*{Section V. \textit{On the inflection of adjectives}. [p. 76--96]}
\addcontentsline{toc}{section}{Section V. \textit{On the inflection of adjectives}. [p. 76--96]}

\subsection*{\hspace*{\fill}§. 1.\hspace*{\fill}}

If we take a look at the way of \is{Declension!Adjective declension}inflecting adjectives, we observe especially in southern \is{Dialect}dialects that adjectives are in harmony with nouns also in case endings, as follows: one says in the southern \is{Dialect}dialects \textit{dam ti sladkega vina} [‘I’ll give you sweet wine’], \textit{imam vernega prijatelia} [‘I have a true friend’] etc., where the northern \is{Dialect}dialects have \textit{sladkego}, \textit{vernego}. Now, from this fundamental agreement in the genitive plural, one should equally say \textit{lepov obrazov} [‘of the pretty pictures’], that is, if the adjective exactly corresponded also in its ending with the noun. But in all \is{Dialect}dialects adjectives are pronounced in the genitive plural with \textit{x} = \textit{h}, namely: \textit{lepix}, or \textit{lepih obrazov}. However, the \il{Slavonian}Slavonians entirely conform the nouns of all genders to this ending, as follows: \textit{vojakah} ‘of soldiers’, \textit{vremenah} ‘of times’, \textit{dievicah} ‘of girls’, \textit{milostih} ‘of graces’, \textit{obrazah} ‘of pictures’ etc. Hence it is clear that the \is{Declension!Adjective declension}inflection of adjectives is not guided by the \is{Declension!Noun declension}inflection of nouns, but rather rests on the \is{Inflection!Pronoun inflection| see {Pronoun declension}}\is{Declension!Pronoun declension}inflection of the third person pronoun, for the third [77] person pronoun’s mark of the genitive plural in all \is{Dialect}dialects is -\textit{x}, which in the southern \is{Dialect}dialects corresponds to the less guttural -\textit{h}. And for this reason, before we treat the \is{Declension!Adjective declension}inflection of adjectives, it is important to have a look at the \is{Declension!Pronoun declension}inflection of pronouns.

\subsection*{\hspace*{\fill}§. 2. \textit{On the pronoun}.\hspace*{\fill}}

Pronouns take up the role of nouns in speech; their endings in the \il{Old Church Slavonic}ancient \is{Dialect}dialect are the following:{\enlargethispage{\baselineskip}\footnote{\citet[490--497]{dobrovsky_institutiones_1822}.}}

\begin{table}
    \scalebox{0.82}{
    \begin{tabular}{ l l l l l l l l l l }
    \lsptoprule
    \multicolumn{10}{ c }{Singular.} \\
    \midrule
    & & & \textit{M}. & \textit{F}. & \multicolumn{2}{ c }{\textit{N}.} & & \\
    \textit{N}. & \textit{Az} & \textit{Ti}, & \textit{On}, & \textit{Ona}, & \textit{Ono}, & or & \textit{On}, & \textit{Ona}, & \textit{Ono}, \\
    \textit{G}. & \textit{Mne}, & \textit{Tebje}, & \textit{Onogo}, & \textit{Onaja}, & \textit{Onogo} & — & \textit{Jego}, & \textit{Jeja}, & \textit{Jemu} \\
    \textit{D}. & \textit{Mnje} (\textit{mi}) & \textit{Tebje} (\textit{ti}) & \textit{Onomu}, & \textit{Onoj}, & \textit{Onomu}, & or & \textit{Jemu}, & \textit{Jej}, & \textit{Jemu}. \\
    \textit{A}. & \textit{Mna}, & \textit{Tja}, & \textit{On} & \textit{Onu}, & \textit{Ono}, & — & \textit{Ji}, & \textit{Ju}, & \textit{Je} \\
    & & & (\textit{Onogo}), & & & & & & \\
    \textit{L}. & \textit{Mnje}, & \textit{Tebje}, & \textit{Onom}, & \textit{Onoj}, & \textit{Onom} & — & \textit{v Njem} & \textit{Njej} & \textit{Njem} \\
    \textit{I}. & \textit{Mnoja}, & \textit{Toboju}, & \textit{Onjem}, & \textit{Onoju}, & \textit{Onjem} & — & \textit{s Njim}, & \textit{Nju}, & \textit{Njim}. \\
    \lspbottomrule
    \\
    \lsptoprule
    \multicolumn{10}{ c }{Plural.} \\
    \midrule
    & & & \textit{M}. & \textit{F}. & \multicolumn{2}{ c }{\textit{N}.} & \textit{M}. & \textit{F}. & \textit{N}. \\
    \textit{N}. & \textit{Mi}, & \textit{Vi}, & \textit{Oni} & \textit{Oni} & \textit{Ona} & or & \textit{Oni}, & \textit{Oni}, & \textit{Ona}. \\
    \textit{G}. & \textit{Nas} & \textit{Vas} & \textit{Onjex} & & \multicolumn{2}{ c }{—} & \textit{Jix} & — & — \\
    \textit{D}. & \textit{Nam}, & \textit{Vam} & \textit{Onjem}, & \textit{Onjem}, & \textit{Ojem} & — & \textit{Jim} & — & — \\
    \textit{A}. & \textit{Nas}, & \textit{Vas} & \textit{Oni}, & \textit{Oni}, & \textit{Ona} & — & \textit{Ja}, or \textit{Jix} & — & — \\
    \textit{L}. & \textit{Nas}, & \textit{Vas} & \textit{Onjex} & — & — & — & \textit{Njix} & — & — \\
    \textit{I}. & \textit{Nami}, & \textit{Vami}, & \textit{Onjemi} & — & — & — & \textit{Nimi} & — & — \\
    \lspbottomrule
    \end{tabular}}
\end{table}

The first and second person pronouns cannot at all serve as the norm for \is{Declension!Adjective declension}inflecting adjectives, because they lack gender distinctions. The third person is the basis for \is{Declension!Adjective declension}inflecting adjectives; and its components are apparently \textit{ji}, \textit{ja}, \textit{jo}. [78]

The first person was expressed by the \il{Old Church Slavonic}ancients by \textit{az}, but in its place \textit{ja} has since prevailed in all \is{Dialect}dialects. Across the Carpathian Mountains \textit{jax} is also heard, especially when it is combined with a preterit, such as \textit{jax robil} ‘I did’, but this -\textit{x} seems to be a remnant of the \il{Old Church Slavonic}ancient preterit, namely \textit{robix}, instead of \textit{robil}.

\subsection*{\hspace*{\fill}§. 3.\hspace*{\fill}}

The \il{Russian}Russians \is{Declension!Pronoun declension}inflect in the following way:\footnote{\citet[227--228]{puchmayer_lehrgebaude_1820}.}

\begin{longtable}{ l l l l l l }
    \lsptoprule
    \multicolumn{6}{ c }{Singular.} \\
    \midrule
    & & & masc. & fem. & neut. \\
    \textit{N}. & \textit{Ja}, & \textit{Ti}, & \textit{On}, & \textit{Ona}, & \textit{Ono}. \\
    \textit{G}. & \textit{Mnja}, & \textit{Tebja}, & \textit{Jego}, & \textit{Jeja} (\textit{jeè}), & \textit{Jego}. \\
    \textit{D}. & \textit{Mnje}, & \textit{Tebje}, & \textit{Jemu}, & \textit{Jej}, & \textit{Jemu}. \\
    \textit{A}. & \textit{Mnja}, & \textit{Tebja}, & \textit{Jego}, & \textit{Jèè} (\textit{Jej}), & \textit{Ono}. \\
    \textit{L}. & \textit{v Mnje}, & \textit{Tebje}, & \textit{Nem}, & \textit{Nej}, & \textit{Nem}. \\
    \textit{I}. & \textit{so Mnoju}, & \textit{Toboju}, & \textit{s Nim}, & \textit{Neju}, & \textit{Nim}. \\
    \lspbottomrule
    \\
    \lsptoprule
    \multicolumn{6}{ c }{Plural.} \\
    \midrule
    & & & masc. & fem. & neut. \\
    \textit{N}. & \textit{Mi}, & \textit{Vi}, & \textit{Oni}, & \textit{Onje}, & \textit{Ona}. \\
    \textit{G}. & \textit{Nas}, & \textit{Vas}, & \textit{Jix}, & — & — \\
    \textit{D}. & \textit{Nam}, & \textit{Vam}, & \textit{Jim}. & — & — \\
    \textit{A}. & \textit{Nas}, & \textit{Vas}, & \textit{Jix}, & — & — \\
    \textit{L}. & \textit{v Nas}, & \textit{Vas}, & \textit{Nix}. & — & — \\
    \textit{I}. & \textit{Nami}, & \textit{Vami}, & \textit{Nimi}, & — & — \\
    \lspbottomrule
\end{longtable}

This way of \is{Declension!Pronoun declension}inflecting corresponds to the \il{Old Church Slavonic}ancient \is{Dialect}dialect, and also to very many \is{Dialect!Living dialect}living dialects and even to the \is{Genius!Genius of the Slavic language}genius of the \il{Slavic}Slavic language itself. The genitive singular in the feminine is expressed in a twofold way, namely [79] \textit{jeja} in the fashion of the \il{Old Church Slavonic}ancients or \textit{jee}, which in our fashion of writing is \textit{jej}; the \is{Dialect!Living dialect}living voice of the \il{Russian}Russians confirms that. The feminine accusative singular is likewise expressed with \textit{jee}, but the \il{Old Church Slavonic}ancient \is{Dialect}dialect does it with \textit{ju}, an ending which the \il{Russian}Russians also follow both in speech and in other pronouns, as follows: \textit{tu}, \textit{moju}, \textit{naшu} etc., which is why not the grammarian’s projection in \textit{jee} but rather \textit{ju} should be accepted, both because it is confirmed in the \il{Old Church Slavonic}ancient \is{Dialect}dialect and because the \il{Russian}Russians, too, speak like that, and so it will be:

\begin{longtable}{ l l l l l l l l }
    \lsptoprule
    \multicolumn{4}{ c }{Singular.} & \multicolumn{4}{ c }{Plural.} \\
    \midrule
    & [masc. & fem. & neut. & & masc. & fem. & neut.] \\
    \textit{N}. & \textit{On}, & \textit{Ona}, & \textit{Ona}. & \textit{N}. & \textit{Oni}, & \textit{Onje}, & \textit{Ona}. \\
    \textit{G}. & \textit{Jego}, & \textit{Jej}, & \textit{Jego}. & \textit{G}. & \textit{Ix}, & — & — \\
    \textit{D}. & \textit{Jemu}, & \textit{Jej}, & \textit{Jemu}. & \textit{D}. &\textit{Im}. & — & — \\
    \textit{A}. & \textit{Jego}, & \textit{Ju}, & \textit{Ono}. & \textit{A}. & \textit{Ix}. & \textit{Je}. & — \\
    \textit{L}. & \textit{v Nem}, & \textit{v Nej}, & \textit{v Nem}. & \textit{L}. & \textit{v Nix}. & — & — \\
    \textit{I}. & \textit{s Nim}, & \textit{s Neju}, & \textit{s Nim}. & \textit{I}. & \textit{s Nimi}, & — & — \\
    \lspbottomrule
\end{longtable}

And the \is{Declension!Adjective declension}inflection of adjectives is based on this form according to the \is{Genius}genius of the language, about which more below. Let us now look at what the \il{Polish}Polish grammarians say.

\subsection*{\hspace*{\fill}§. 4.\hspace*{\fill}}

Since for both \il{Polish}Poles and also the other Slavic Nations, first and second person pronouns correspond exactly to both the \il{Old Church Slavonic}ancient \is{Dialect}dialect and the \il{Russian}Russian, as far as their \is{Declension!Pronoun declension}inflection is concerned, let us now consider only the inflection of the 3\textsuperscript{rd} person pronouns as the supposed basis for \is{Declension!Adjective declension}inflecting adjectives. The \il{Polish}Pole inflects in the following way:\footnote{\citet[191]{bandtkie_polnische_1808}. \ia{Herkel, Jan}Herkel’s transcriptions are sometimes simplified, e.g. \ia{Herkel, Jan}Herkel gives \textit{jemu} where \ia{Bandtkie, Jerzy Samuel}Bandtkie has \textit{jemu} (\textit{mu}).} [80]

\begin{longtable}{ l l l l l l l l }
    \lsptoprule
    \multicolumn{4}{ c }{Singular.} & \multicolumn{4}{ c }{Plural.} \\
    \midrule
    \textit{N}. & \textit{On}, & \textit{Ona}, & \textit{Ono}. & \textit{N}. & \textit{Oni}, & \textit{One}, & \textit{One}. \\
    \textit{G}. & \textit{Jego}, & \textit{Jej}, & \textit{Jego}. & \textit{G}. & \textit{Jix}, & — & — \\
    \textit{D}. & \textit{Jemu}, & \textit{Jej}, & \textit{Jemu}. & \textit{D}. &\textit{Im}. & — & — \\
    \textit{A}. & \textit{Jego}, & \textit{Ją} (\textit{Ję}), & \textit{Ono}. & \textit{A}. & \textit{Ix}, & \textit{Je}. & — \\
    \textit{L}.* & \textit{v Nim}, & \textit{v Nej}, & \textit{v Nim}. & \textit{L}. & \textit{v Nix}. & — & — \\
    \textit{I}.* & \textit{s Njim}, & \textit{s Nju}, & \textit{s Njim}. & \textit{I}. & \textit{s Nimi}. & — & — \\
    \lspbottomrule
\end{longtable}

Every language of great originality has its own philosophy. This is illustrated especially in the \il{Slavic}Slavic language, if its multiple \is{Dialect}dialects would be combined. Thus the \il{Polish}Polish grammarian says of the chart shown above that the \il{Polish}Poles lack those cases marked with an asterisk.\footnote{The asterisks do not appear in \ia{Herkel, Jan}Herkel’s own text, and he does not seem to be referring to a paradigm of \ia{Bandtkie, Jerzy Samuel}Bandtkie’s marked by asterisks. We have inserted asterisks at the locative and instrumental level, since \citeauthor{bandtkie_polnische_1808} indicates in his \textit{Polnische Grammatik} (\citeyear[224--225]{bandtkie_polnische_1808}) that the locative “fehlt, weil er nur mit Präpositionen vorkommt” (\citeyear[225]{bandtkie_polnische_1808}), whereas the instrumental is obsolete, “weil es nicht ohne \textit{Præp}. vorkommt” (\citeyear[225]{bandtkie_polnische_1808}).} Another grammarian includes them.\footnote{\citet[212, fn. 220]{herkel_jan_2009}, has not been able to identify this \il{Polish}Polish grammarian. It seems to be a reference to Johann \citeauthor{moneta_polnische_1805}’s oft-published \textit{Polnische Grammatik}, e.g. in the edition of Daniel Vogel (\citeyear[184 (\textit{jego}, \textit{jemu})]{moneta_polnische_1805}).} But it is easy to decide which of them has it right; for if the aforementioned cases are found in speech, if the sisters of the \il{Polish}Polish \is{Dialect}dialect are not robbed of the marked cases, then it would be a contradiction to claim that \il{Polish}Polish lacks them. The grammarian explains the lack of these marked cases on the grounds that they occur only with prepositions.\footnote{\citet[225]{bandtkie_polnische_1808}.} However, a preposition does not produce a new case, but proves that this case should already be present. In the meantime, having said this between brackets, the \is{Declension!Pronoun declension}pronominal inflection of the \il{Polish}Poles offered above is consistent with the other \is{Dialect}dialects, and thus very much so with the \is{Genius!Genius of the Slavic language}genius of the \il{Slavic}Slavic language.

\subsection*{\hspace*{\fill}§. 5.\hspace*{\fill}}

The \is{Declension!Pronoun declension}inflection of the \il{Bohemian}Bohemians is the following:\footnote{\citet[282]{dobrovsky_ausfuhrliches_1809}.} [81]

\begin{longtable}{ l l l l l l l l }
    \lsptoprule
    \multicolumn{4}{ c }{Singular.} & \multicolumn{4}{ c }{Plural.} \\
    \midrule
    \textit{N}. & \textit{On}, & \textit{Ona}, & \textit{Ono}. & \textit{N}. & \textit{Oni}, & \textit{Oni}, & \textit{Ona}. \\
    \textit{G}. & \textit{Jeho}, & \textit{Ji}, & \textit{Jeho}. & \textit{G}. & \textit{Jix}, & — & — \\
    \textit{D}. & \textit{Jemu}, & \textit{Ji}, & \textit{Jemu}. & \textit{D}. &\textit{Jim}, & — & — \\
    \textit{A}. & \textit{Jej}, & \textit{Ji}, & \textit{Je}. & \textit{A}. & \textit{Je}, & \textit{Je}, & \textit{Ona}. \\
    \textit{L}. & \textit{v Nem}, & \textit{v Ni}, & \textit{v Nem}. & \textit{L}. & \textit{v Nix}, & — & — \\
    \textit{I}. & \textit{s Nim}, & \textit{Ni}, & \textit{s Nim}. & \textit{I}. & \textit{Nimi}, & — & — \\
    \lspbottomrule
\end{longtable}

The grammarian has explained the accusative of each number very \is{Dialect}dialectally, for the \il{Bohemian}Bohemian also says: \textit{Ja sem ho videl} [‘I saw him’], where \textit{ho} is an abbreviation of \textit{jeho}, which was originally \textit{jego}, whence \textit{go}, just as among the southerners \textit{jega} is abbreviated to \textit{ga}. So, similarly in the plural the \il{Bohemian}Bohemians say: \textit{Jix sem videl} [‘I saw them’]; for this reason, \il{Bohemian}Bohemian does not at all differ from the remaining \is{Dialect}dialects either. The grammarian of this \is{Dialect}dialect profoundly observes that in the nominative of this pronoun the double ending is also distinguished, namely: \textit{on}, \textit{ona}, \textit{ono} is a demonstrative pronoun, and \textit{ji}, \textit{je}, \textit{je}, or according to the \is{Genius}genius of the language \textit{ji}, \textit{ja}, \textit{jo} is strictly speaking a third person pronoun. This is why \textit{on} in the genitive is said \textit{onego}, but \textit{ji} is said \textit{jego}, expressions which are distinct in terms of both usage and meaning. Nonetheless, the grammarians attribute the nominative \textit{on}, \textit{ona}, \textit{ono} to both, since indeed the original components of the third person have vanished from the \is{Dialect}dialects: namely \textit{ji}, \textit{ja}, \textit{je} or \textit{jo} have been supplanted by \textit{on}, \textit{ona}, \textit{ono}.

\subsection*{\hspace*{\fill}§. 6.\hspace*{\fill}}

The \il{Pannonian}Pannonian \is{Declension!Pronoun declension}inflects the said pronoun in the following way:\footnote{\citet[64--65]{bernolak_grammatica_1790}.} [82]

\begin{longtable}{ l l l l }
    \lsptoprule
    \multicolumn{4}{ c }{Singular.} \\
    \midrule
    \textit{N}. & \textit{On}, & \textit{Ona}, & \textit{Ono}. \\
    \textit{G}. & \textit{Jeho}, & \textit{Jej}, & \textit{Jeho}. \\
    \textit{D}. & \textit{Jemu}, & \textit{Jej}, & \textit{Jemu}. \\
    \textit{A}. & \textit{Jeho}, & \textit{Ju}, & \textit{Jehu}, (\textit{ho}) \textit{ono}. \\
    \textit{L}. & \textit{Nom}, \textit{Nem}, & \textit{Nej}, & \textit{Nom}, \textit{Nem}. \\
    \textit{I}. & \textit{Snjim}, & \textit{Nju}, & \textit{Njim}. \\
    \lspbottomrule
    \\
    \lsptoprule
    \multicolumn{4}{ c }{Plural.} \\
    \midrule
    \textit{N}. & \textit{Oni}, & \textit{Oni}, & \textit{One}. \\
    \textit{G}. & \textit{Jix}, & — & — \\
    \textit{D}. & \textit{Jim}, & — & — \\
    \textit{A}. & \textit{Jix}, & \textit{Nje}, & \textit{Jix}. \\
    \textit{L}. & \textit{Nix}, & — & — \\
    \textit{I}. & \textit{Nimi}, & — & — \\
    \lspbottomrule
\end{longtable}

The nominative plural in the feminine is shown as \textit{oni}, but \textit{one} is also said. The plural neuter \textit{ona} would be more correct than \textit{one} in the \il{Polish}Polish fashion, since the \il{Bohemian}Bohemians also say the neuter as \textit{ona}.

The \il{Serbian}Serbians, the \il{Slavonian}Slavonians together with the \il{Croatian}Croatians and \il{Dalmatian}Dalmatians, \is{Declension!Pronoun declension}inflect in the following way:\footnote{\citet[xlix-l]{karadzic_srpski_1818}.}

\begin{longtable}{ l l l l }
    \lsptoprule
    \multicolumn{4}{ c }{Singular.} \\
    \midrule
    \textit{N}. & \textit{On}, & \textit{Ona}, & \textit{Ono}. \\
    \textit{G}. & \textit{Njega}, & \textit{Nje}, & \textit{Njega}. \\
    \textit{D}. & \textit{Njemu}, & \textit{Njoj}, & \textit{Njemu}. \\
    \textit{A}. & \textit{Njega}, & \textit{Nju}, & \textit{Njega}. \\
    \textit{L}. & \textit{u Njemu}, & \textit{Njoj}, & \textit{Njemu}. \\
    \textit{I}. & \textit{s Njime}, & \textit{Njom}, & \textit{Njime}. \\
    \lspbottomrule
    \\
    \lsptoprule
    \multicolumn{4}{ c }{Plural.} \\
    \midrule
    \textit{N}. & \textit{Oni}, & \textit{One}, & \textit{Ono}. [83] \\
    \textit{G}. & \textit{Njix}, & — & — \\
    \textit{D}. & \textit{Njim } or \textit{Njima}, & — & — \\
    \textit{A}. & \textit{Nje}, & — & \textit{Ona}. \\
    \textit{L}. & \textit{v Njix}, & — & — \\
    \textit{I}. & \textit{s Nimi}, & \textit{Njima}, & — \\
    \lspbottomrule
\end{longtable}

This way of \is{Declension!Pronoun declension}inflecting is in neat agreement with the \is{Genius!Genius of the Slavic language}genius of the \il{Slavic}Slavic language. However, the fact that the genitive ends in -\textit{a} in the masculine and neuter singular is explained by the grammarians as follows: the adjective follows the ending of the noun, as can be observed in the other cases. But this rule has its exceptions, for the neuter accusative ends in -\textit{a}, e.g. \textit{njega}, while the nouns are not pronounced that way; in the meantime, their plural becomes also confused with the dual, for their neighbors the \il{Carniolan}Carniolan and \il{Carinthian}Carinthian Slavs do not pronounce the dative and locative plural with -\textit{ma}, as the following figure shows:\footnote{\citet[281]{kopitar_grammatik_1808}.}

\enlargethispage{1mm}

\begin{longtable}{ l l l l }
    \lsptoprule
    \multicolumn{4}{ c }{Singular.} \\
    \midrule
    \textit{N}. & \textit{On}, & \textit{Ona}, & \textit{Ono}. \\
    \textit{G}. & \textit{Njega}, & \textit{Nje}, & \textit{Njega}. \\
    \textit{D}. & \textit{Njemu}, & \textit{Nji}, & \textit{Njemu}. \\
    \textit{A}. & \textit{Njega}, & \textit{Njo}, & \textit{Njega}. \\
    \textit{L}. & \textit{Njemu}, & \textit{Nji}, & \textit{Njemu}. \\
    \textit{I}. & \textit{Njim}, & \textit{Njo}, & \textit{Njim}. \\
    \lspbottomrule
    \\
    \lsptoprule
    \multicolumn{4}{ c }{Plural.} \\
    \midrule
    \textit{N}. & \textit{Oni}, & \textit{One}, & \textit{Ona}. \\
    \textit{G}. & \textit{Jih}, & — & — \\
    \textit{D}. & \textit{Jim}, & — & — \\
    \textit{A}. & \textit{Nje} or \textit{Jih}, & — & — [84] \\
    \textit{L}. & \textit{Njih}, & — & — \\
    \textit{I}. & \textit{Njimi}, & — & — \\
    \lspbottomrule
\end{longtable}

Some grammarians also take nouns as the basis for \is{Declension!Adjective declension}inflecting these adjectives, but the genitive plural with them always ends in -\textit{h} = -\textit{x} in adjectives, an ending that never occurs in nouns, and indeed in no other \is{Dialect}dialect except the \il{Slavonian}Slavonian. Both current usage among the \is{Dialect}dialects and \il{Old Church Slavonic}Old Church Slavonic show how far this expression deviates from the genuine \il{Slavic}Slavic ending.

\subsection*{\hspace*{\fill}§. 7.\hspace*{\fill}}

The abovementioned \is{Declension!Pronoun declension}inflections of the pronoun \textit{on} in the various \is{Dialect}dialects reveal the following form as agreeing with the \is{Genius!Genius of the Slavic language}genius of the \il{Slavic}Slavic language, which will serve as the basis for \is{Declension!Adjective declension}inflecting all adjectives:

\begin{longtable}{ l l l l }    
    \caption*{\textit{Norm for \is{Declension!Adjective declension}inflecting all adjectives}.} \\
    \noalign{\vspace{6pt}}
    \lsptoprule
    \multicolumn{4}{ c }{Singular.} \\
    \midrule
    & \textit{m}. & \textit{f}. & \textit{n}. \\
    \textit{N}. & \textit{On} (\textit{ji}), & \textit{Ona} (\textit{ja}), & \textit{Ono} (\textit{je}, \textit{jo}). \\
    \textit{G}. & \textit{Jego}, & \textit{Jej}, & \textit{Jego}. \\
    \textit{D}. & \textit{Jemu}, & \textit{Jej}, & \textit{Jemu}. \\
    \textit{A}. & \textit{Jego}, & \textit{Ju}, & \textit{Ono}. \\
    \textit{L}. & \textit{v Njem}, & \textit{Nej}, & \textit{Njem}. \\
    \textit{I}. & \textit{s Njim}, & \textit{s Nju}, & \textit{s Njim}. \\
    \lspbottomrule
    \newpage
    \lsptoprule
    \multicolumn{4}{ c }{Plural.} \\
    \midrule
    \textit{N}. & \textit{Oni} (\textit{ji}), & \textit{One} (\textit{je}), & \textit{Ona} (\textit{ja}). [85] \\
    \textit{G}. & \textit{Jix}, & — & — \\
    \textit{D}. & \textit{Jim}, & — & — \\
    \textit{A}. & \textit{Jix}, & \textit{Je}, & \textit{Ona}. \\
    \textit{L}. & \textit{v Nix}, & — & — \\
    \textit{I}. & \textit{s Nimi}. & — & — \\
    \lspbottomrule
\end{longtable}

\subsection*{\hspace*{\fill}§. 8.\hspace*{\fill}}

The other \is{Declension!Pronoun declension}pronouns, along with the \is{Declension!Adjective declension}adjectives, also follow this inflection, but the reason why I have established this inflection as the norm for adjectives is that all \il{Slavic}Slavic \is{Dialect}dialects in all three genders end genitive plural adjectives with -\textit{x},\linebreak{} i.e. in a way similar to the pronoun of the third person. Additionally, in very many \is{Dialect}dialects, and enormously widespread ones at that, adjectives also follow the \is{Declension!Adjective declension}inflection of the aforementioned pronoun. \il{Croatian}Croats and \il{Slavonian}Slavonians, however, also attribute the genitive plural ending -\textit{x} = -\textit{h} to nouns of all genders in that case, an ending which nonetheless seems to have crept in through abuse, because it is observed neither in the \il{Old Church Slavonic}ancient nor in the other \is{Dialect!Living dialect}living dialects. For instance, the \il{Slavonian}Slavonians say: \textit{vojakah} [‘of the soldiers’], \textit{vremenah} [‘of the times’], \textit{ƶenah} [‘of the women’], \textit{dievicah} [‘of the girls’], instead of \textit{vojakov}, \textit{vremien}, \textit{ƶien}, \textit{dievic} etc. But let us consider the \is{Declension!Adjective declension}inflection of the adjectives in the different \is{Dialect}dialects to see whether the proposed norm persists. In the \il{Old Church Slavonic}ancient \is{Dialect}dialect, as the most illustrious Dobrovský \ia{Dobrovský, Josef}asserts, some adjectives have a definite form, but others an indefinite form, and indeed, indefinite adjectives follow the \is{Declension!Noun declension}inflection of nouns, such as: [86]

\newpage

\begin{footnotesize}
\begin{longtable}{ l l }
    \lsptoprule
    \multicolumn{2}{ c }{Singular.} \\
    \midrule
    \textit{N}. & \textit{Meч Oster}, \textit{Plaч Gorek}, \textit{Dar Blag}, \textit{Slovo Lubavo}, \textit{Miaso Junчe}. \\
    & \hspace{0.5cm} [‘A sharp sword’, ‘a bitter tear weeping’, ‘a good gift’, ‘a word of love’, ‘bull’s flesh’.] \\
    \textit{G}. & \textit{Meчa Ostra}, \textit{Muƶa Premudra}, \textit{Raja Boƶija}, \textit{Vodi Mnogi} etc. \\
    & \hspace{0.5cm}[‘Of a sharp sword’, ‘of a wise man’, ‘of God’s paradise’, ‘of much water’.] \\
    \textit{D}. & \textit{Muƶu Pravenu}, \textit{Domu Carevu}, \textit{Licu Boƶiu}, \textit{Sinu Jedinorodnu} etc. \\
    & \hspace{0.5cm} [‘To a just man’, ‘to the tsar’s house’, ‘to God’s face’, ‘to the only begotten son’.] \\
    \textit{A}. & \textit{Glavu Zmievu}, \textit{Rizu novu}, \textit{Mzdu Proroчu}, \textit{Goru Sionju} etc. \\
    & \hspace{0.5cm} [‘The dragon’s head’, ‘a new robe’, ‘the prophet’s reward’, ‘the mountain of Zion’.] \\
    \textit{L}. & \textit{v Glasie Trubnie}, \textit{Na paƶiti Tuчnie}, \textit{Na Vode mnozie} etc. \\
    & \hspace{0.5cm} [‘In the trumpet’s sound’, ‘on a fat pasture’, ‘on much water’.] \\
    \textit{I}. & \textit{Mnogom Jazikom}, \textit{Boziem Slovom} etc. \\
    & \hspace{0.5cm} [‘With many tongues’, ‘with the word of God’.] \\
    \lspbottomrule
    \\
    \lsptoprule
    \multicolumn{2}{ c }{Plural.} \\
    \midrule
    \textit{N}. & \textit{Kniazi Judovi}, \textit{Rebra Sjeverova} etc. \\
    & [‘Priests of the Jews’, ‘ribs of the north’.] \\
    \textit{G}. & \textit{Krup Pшeniчen}, \textit{Kamien Ognien} etc. \\
    & [‘Of groats of wheat’, ‘of fiery stones’.] \\
    \textit{D}. & \textit{Dverom Zatverenam}, \textit{Diakonom Чistom} etc. \\
    & [‘To closed doors’, ‘to pure deacons’.] \\
    \textit{A}. & \textit{Ljudi Xrabri} [‘brave people’] etc. \\
    \textit{L}. & \textit{v Koziax Koƶax}, \textit{Po Mnozjex Dniex} etc. \\
    & [‘In goat’s skins’, ‘after many days’.] \\
    \textit{I}. & \textit{Mnogimi Slzami}, \textit{Nitmi Zlatimi} etc. \\
    & [‘With many tears’, ‘with golden threads’.] \\
    \lspbottomrule
\end{longtable}
\end{footnotesize}

Additionally, the pronominal \is{Declension!Adjective declension}inflection of the adjective follows the \is{Declension!Pronoun declension}inflection of the pronoun also in the \il{Old Church Slavonic}ancient \is{Dialect}dialect, thus the genitive singular is not \textit{meчa ostra}, but \textit{ostrego}, \textit{Muƶa premudrego}, \textit{raja boƶigo}, not \textit{krup pшeniчen}, but \textit{pшe\-niчnix}, \textit{kamien ognix}, or \textit{ognivix}. Note here that \textit{kamien} in the genitive plural derives from \textit{kamenie}, i.e. ‘a great number of stones’, but \textit{kamen} is ‘stone’ etc. Yet Logic cannot explain why Grammarians call the former form of \is{Declension!Adjective declension}inflecting adjectives indefinite, but the latter definite, [87] for every adjective combined with a noun determines the quality of the noun itself, since if I say \textit{meчa ostra} or \textit{meчa ostrego}, it always designates the quality of the sword etc. For this reason, adjectives combined with nouns are wrongly divided into definite and indefinite. The situation is different if a solitary adjective is purely and clearly regarded as a word expressing a quality, for instance \textit{nov} [‘new’], \textit{xud} [‘poor’], \textit{zelen} [‘green’], \textit{чerven} [‘red’] etc., \textit{uчen} [‘learned’], \textit{dober} [‘good’], \textit{oster} [‘sharp’] etc. Yet here it is understood what \textit{nov}, \textit{xud}, \textit{zelen} is; it is nevertheless left undefined to which object it applies. However, if I say \textit{novi dom} [‘new house’], \textit{xudi чelovek} [‘poor person’], or \textit{zeleni strom} [‘green tree’], or in the fashion of biblical expressions \textit{чelovik xud}, \textit{strom zelen}, \textit{plaч goreh} [‘bitter weeping’], \textit{dar blag} [‘good gift’] etc., the quality of the object is already defined. Hence it follows: whichever form ties an adjective to a noun, it always determines the meaning thereof.

For this reason, it is more correct to divide the \is{Declension!Adjective declension}inflection of the adjectives into the pronominal and nominal. The substantival inflection flourishes in all northern \is{Dialect}dialects, such as \il{Russian}Russian, \il{Polish}Polish, \il{Bohemian}Bohemian, \il{Pannonian}Pannonian. The nominal \is{Declension!Adjective declension}inflection was no less frequent in the \il{Old Church Slavonic}ancient \is{Dialect}dialect, but the southern \is{Dialect}dialects mix the pronominal with the nominal \is{Declension!Adjective declension}inflection, such as, for instance, \textit{dobre} or \textit{dobro sukno} [‘good cloth’]. All northerners use the genitive: \textit{dobrego sukna}, \textit{ƶutego sukna} [‘of yellow cloth’], but the southerners say: \textit{dobrega sukna}, \textit{ƶutega sukna}. Here, \textit{dobrega} is a mixed form. The southerners, and particularly the \il{Serbian}Serbs, also say \textit{ƶuta sukna}; this \is{Declension!Adjective declension}inflection is entirely nominal, [88] which the grammarians describe as an abbreviation from \textit{ƶutega}.

Now we must see which of the two ways of \is{Declension!Adjective declension}inflecting prevails. The biblical context indicates that the inflection was in mixed use in the \il{Old Church Slavonic}ancient \is{Dialect}dialect. As far as the \is{Dialect!Living dialect}living dialects are concerned, the northern \is{Dialect}dialects exclude the nominal inflection, the southern mix it with the pronominal.

\subsection*{\hspace*{\fill}§. 9.\hspace*{\fill}}

But let us return to our earlier topic and examine the \is{Declension!Adjective declension}inflection of the adjectives in the \is{Dialect!Living dialect}living dialects, and see whether they follow the \is{Declension!Noun declension}substantival or \is{Declension!Pronoun declension}pronominal way of inflecting. The \il{Russian}Russians, according to the grammarian, indeed inflect the adjectives as in the \il{Old Church Slavonic}ancient \is{Dialect}dialect, namely in the substantival and pronominal way. The substantival \is{Declension!Adjective declension}declension is as follows:\footnote{\citet[223]{puchmayer_lehrgebaude_1820}.}

\enlargethispage{0.5mm}

\begin{longtable}{ l l l l l l }    
    \lsptoprule
    & \multicolumn{3}{ c }{Singular.} & \multicolumn{2}{ c }{Plural.} \\
    \midrule
    & \textit{m}. & \textit{f}. & \textit{n}. & \multicolumn{2}{ c }{for all three genders} \\
    \textit{N}. & \textit{Dobr} [‘good’], & \textit{Dobra}, & \textit{Dobro}. & \textit{N}. & \textit{Dobri}. \\
    \textit{G}. & \textit{Dobra}, & \textit{Dobri}, & \textit{Dobra}. & \textit{G}. & \textit{Dobrix}. \\
    \textit{D}. & \textit{Dobru}, & \textit{Dobri}, & \textit{Dobru}. & \textit{D.} & \textit{Dobrim}. \\
    \textit{A}. & \textit{Dober} (\textit{a}), & \textit{Dobru}, & \textit{Dobro}. & \textit{A}. & \textit{Dobri}. \\
    \textit{I}. & \textit{Dobrim}, & \textit{Dobroju}, & \textit{Dobrim}. & \textit{I}. & \textit{Dobrimi}. \\
    \lspbottomrule
\end{longtable}

The accusative singular in the masculine ends in -\textit{a} when it concerns an animated object, as follows: \textit{Ja dvigal чeloveka mertva} [‘I moved a dead person’], not \textit{mertv}, because the accusative of animates takes the same form as the genitive. If, however, one would say: \textit{Ja dvigal чelovieka mertvego}, here the \is{Declension!Adjective declension}adjective is inflected as a pronoun. The grammarian [89] says that the former norm of inflecting is abbreviated, which the most illustrious Dobrovský \ia{Dobrovský, Josef}has called indefinite in the \il{Old Church Slavonic}ancient \is{Dialect}dialect. The southern Slavs contract this double way of inflecting into one form, such as in \il{Carinthian}Carinthian, where one says \textit{videl sem mertvega чelovjeka} [‘I saw a dead person’]; here, the ending of the genitive of the pronoun \textit{jego} is added to \textit{mertv}, yet with -\textit{o} changed into -\textit{a}, so that it is consistent with the noun as concerns the ending.

The pronominal inflection in \il{Russian}Russian is the following:\footnote{\citet[table insert at 222]{puchmayer_lehrgebaude_1820}.}

\begin{longtable}{ l l l l }    
    \lsptoprule
    \multicolumn{4}{ c }{Singular.} \\
    \midrule
    & \textit{m}. & \textit{f}. & \textit{n}. \\
    \textit{N}. & \textit{Dobrji}, & \textit{Dobroja}, & \textit{Dobroje}. \\
    \textit{G}. & \textit{Dobrago}, & \textit{Dobroj}, & \textit{Dobrago}. \\
    \textit{D}. & \textit{Dobromu}, & \textit{Dobroj}, & \textit{Dobromu}. \\
    \textit{A}. & \textit{Dobrj} (\textit{a}), & \textit{Dobroju}, & \textit{Dobroje}. \\
    \textit{L}. & \textit{Dobrom}, & \textit{Dobroj}, & \textit{Dobrom}. \\
    \textit{I}. & \textit{Dobrim}, & \textit{Dobroju}, & \textit{Dobrim}. \\
    \lspbottomrule
    \\
    \lsptoprule
    \multicolumn{4}{ c }{Plural.} \\
    \midrule
    \textit{N}. & \textit{Dobr}<\textit{i}>\textit{ji}, & \textit{Dobrije}, & \textit{Dobrija}. \\
    \textit{G}. & \textit{Dobrix}, & — & — \\
    \textit{D}. & \textit{Dobrim}. & & \\
    \textit{A}. & like the nominative. & & \\
    \textit{L}. & like the genitive. & &  \\
    \textit{I}. & \textit{Dobrimi}. & & \\
    \lspbottomrule
\end{longtable}

Here, the \is{Declension!Adjective declension}inflection of the adjective corresponds exactly with the \is{Declension!Pronoun declension}inflection of the pronoun; yet, a certain grammarian posits that the nominative plural number is \textit{dobrie}, \textit{dobrija}, \textit{dobrija}, complying with the grammarians [90] of the \il{Old Church Slavonic}ancient \is{Dialect}dialect, and loads final -\textit{a} with a diacritic; but if we would take a look at the usage of the \il{Russian}Russians, it will be clear that \textit{dobriji}, \textit{dobr}<\textit{i}>\textit{je}, \textit{dobrija} mostly prevails.

\subsection*{\hspace*{\fill}§. 10.\hspace*{\fill}}

In the \il{Polish}Polish \is{Dialect}dialect, adjectives only follow the pronominal inflection, such as, for instance:\footnote{\citet[24]{adamowicz_praktische_1796-1}.}

\begin{longtable}{ l l l l }    
    \lsptoprule
    \multicolumn{4}{ c }{Singular.} \\
    \midrule
    \textit{N}. & \textit{Grubi} & \textit{Gruba}, & \textit{Grube}. \\
    & [‘fat’], & & \\
    \textit{G}. & \textit{Grubego}, & \textit{Grubej}, & \textit{Grubego}. \\
    \textit{D}. & \textit{Grubemu}, & \textit{Grubej}, & \textit{Grubemu}. \\
    \textit{A}. & \textit{Grubi} or \textit{Grubego}, & \textit{Grubju}, & \textit{Grube}. \\
    \textit{L}. & \textit{Grubim}, & \textit{Grubej}, & \textit{Grubim}. \\
    \textit{I}. & \textit{Grubim}, & \textit{Grubju}, & \textit{Grubim}. \\
    \lspbottomrule
    \\
    \lsptoprule
    \multicolumn{4}{ c }{Plural.} \\
    \midrule
    \textit{N}. & \textit{Grubi}, & \textit{Grube}, & \textit{Grube}. \\
    \textit{G}. & \textit{Grubix}. & & \\
    \textit{D}. & \textit{Grubim}. & & \\
    \textit{A}. & like the nominative or genitive & & \\
    \textit{L}. & \textit{Grubimi}. & &  \\
    \textit{I}. & \textit{Grubix}. & & \\
    \lspbottomrule
\end{longtable}

This is in any case consistent with the \is{Declension!Pronoun declension}inflection of the pronoun, and with other \is{Dialect}dialects, and it would be the norm, if the nominative plural in the neuter would be pronounced not with -\textit{e} but with -\textit{a}.

The \il{Polish}Poles write the feminine accusative \textit{grubą}, which corresponds to \textit{grubju}. Furthermore, the grammarian notes, and the written record proves, that masculine adjectives [91] were once written in the nominative singular with -\textit{y}, which, because it corresponds completely to the letter -\textit{i} in speech, more recent and more rational writers write everywhere as -\textit{i}. Nevertheless, the grammarian still says that the adjective \textit{letni}, \textit{letnia}, \textit{letnie} [‘summer’] is still written \textit{letny}. I firmly believe, however, that this useless \is{Orthography}orthographic exception will shortly cease to exist, or rather that it has already ceased to exist, since no reason for this exception can be identified other than that authors write it that way. Yet unless it is also grounded in some peculiar pronunciation, we are but blind imitators of antiquity, not true and rational cultivators of the language.

\subsection*{\hspace*{\fill}§. 11.\hspace*{\fill}}

The grammarian of the \il{Bohemian}Bohemian \is{Dialect}dialect distinguishes three \is{Declension!Adjective declension}adjectival inflections, namely in addition to the nominal and the pronominal, he establishes a third inflection, now often still named as such, in which for all three genders in the nominative singular the adjectives end in -\textit{i}. Its form is as follows:\footnote{\citet[table between 270--271]{dobrovsky_ausfuhrliches_1809}.}

\begin{longtable}{ l l l l l }    
    \lsptoprule
    \multicolumn{3}{ c }{Singular.} & \multicolumn{2}{ c }{Plural.} \\
    \midrule
    & \textit{masc}. \textit{neut}. & \textit{fem}. & \multicolumn{2}{ c }{for all three genders} \\
    \textit{N}. & \textit{Boƶi} [‘divine’], & \textit{Boƶi}. & \textit{N}. & \textit{Boƶi}. \\
    \textit{G}. & \textit{Boƶiho}, & \textit{Boƶi}. & \textit{G}. & \textit{Boƶix}. \\
    \textit{D}. & \textit{Boƶimu}, & \textit{Boƶi}. & \textit{D.} & \textit{Boƶim}. \\
    \textit{A}. & \textit{Boƶi}, & \textit{Boƶi}. & \textit{A}. & \textit{Boƶi}. \\
    \textit{L}. & \textit{Boƶim}, & \textit{Boƶi}. & \textit{L}. & \textit{Boƶix}. \\
    \textit{I}. & \textit{Boƶim}, & \textit{Boƶi}. & \textit{I}. & \textit{Boƶimi}. \\
    \lspbottomrule
\end{longtable}

[92] This method of \is{Declension!Adjective declension}inflecting is peculiar to the \il{Bohemian}Bohemian \is{Dialect}dialect alone. It is nothing more than the regular way of inflecting, but with the following observation: in some cases the final ending is swallowed while speaking by the \il{Bohemian}Bohemians, and hence it can be called an abbreviated way of inflecting, such as, for instance, \textit{boƶi dar} [‘divine gift’] is not swallowed in the masculine, but in the feminine it is \textit{boƶi vula} [‘divine will’] instead of \textit{boƶia volia}, \textit{boƶi jmeno} [‘divine name’] instead of \textit{boƶio imeno}, and so on for the remaining cases. Experience proves that the \il{Bohemian}Bohemians, too, speak in the same form as the other \is{Dialect}dialects, since the \il{Bohemian}Bohemian grammarian also establishes the following regular norm of inflecting:\footnote{\citet[table between 270--271]{dobrovsky_ausfuhrliches_1809}.}

\newpage

\begin{longtable}{ l l l l }    
    \lsptoprule
    \multicolumn{4}{ c }{Singular.} \\
    \midrule
    \textit{N}. & \textit{Pravi} & \textit{Prava}, & \textit{Prave}. \\
    & [‘correct’], & & \\
    \textit{G}. & \textit{Praveho}, & \textit{Prave}, & \textit{Praveho}. \\
    \textit{D}. & \textit{Pravemu}, & \textit{Prave}, & \textit{Pravemu}. \\
    \textit{A}. & \textit{Pravi}, & \textit{Pravau}, & \textit{Prave}. \\
    \textit{L}. & \textit{Pravem}, & \textit{Prave}, & \textit{Pravem}. \\
    \textit{I}. & \textit{Pravim}, & \textit{Pravau}, & \textit{Provim} [\textit{sic}]. \\
    \lspbottomrule
    \\
    \lsptoprule
    \multicolumn{4}{ c }{Plural.} \\
    \midrule
    \textit{N}. & \textit{Pravi}, & \textit{Prave}, & \textit{Prava}. \\
    \textit{G}. & \textit{Pravih}, & — & — \\
    \textit{D}. & \textit{Pravim}, & — & — \\
    \textit{A}. & \textit{Prave} (\textit{ix}), & \textit{Prave}, & \textit{Prava}. \\
    \textit{L}. & \textit{Pravix}. \\
    \textit{I}. & \textit{Pravimi}. \\
    \lspbottomrule
\end{longtable}

This way of inflecting is regular, resting on a basis laid earlier and in harmony with the other \is{Dialect}dialects. The ending -\textit{o} follows the \is{Genius}genius of the language, however the neuter \is{Declension!Adjective declension}declension with -\textit{e} [93] is also characteristic, hence either \textit{prave} or \textit{pravo} would conform to the \is{Genius}genius of the language.

The accusative masculine singular conforms with the nominative, but that should only be understood for inanimate things, since for animate things it is pronounced in conformity with the genitive, as indeed in all \is{Dialect}dialects.

Furthermore, the feminine genitive, dative, locative is pronounced with an accentuated -\textit{é}, which according to the grammarian is pronounced as -\textit{ej}. Accordingly, it would be more correct to also write it like it is pronounced, since the mentioned cases are written in the other \is{Dialect}dialects with -\textit{j}, and in that way, subtle discrepancies between writing and pronunciation would vanish, and would be replaced by straightforwardness, accompanied by conformity between \is{Dialect}dialects. For that reason, one should not write \textit{pravé} in such cases, but \textit{pravej}.

\subsection*{\hspace*{\fill}§. 12.\hspace*{\fill}}

The \il{Pannonian}Pannonian grammarian establishes three paradigms for \is{Declension!Adjective declension}inflecting adjectives. Because the \il{Bohemian}Bohemians, too, establish three paradigms, one could be forgiven for believing that the three forms of \is{Declension!Adjective declension}inflecting adjectives are grounded in the \is{Genius!Genius of the Slavic language}genius of the \il{Slavic}Slavic language, but nothing is further from the truth, for the \il{Bohemian}Bohemian establishes the nominal, pronominal, and abbreviated forms. The \il{Pannonian}Pannonian grammarian does not even mention these, but his first paradigm is the following:\footnote{\citet[49]{bernolak_grammatica_1790}.}

\begin{longtable}{ l l l l }    
    \lsptoprule
    \multicolumn{4}{ c }{Singular.} \\
    \midrule
    & \textit{m}. & \textit{f}. & \textit{n}. \\
    \textit{N}. & \textit{Pekni} & \textit{Pekna}, & \textit{Pekne} (-\textit{o}). [94] \\
    & [‘nice, pretty’], & & \\
    \textit{G}. & \textit{Pekneho}, & \textit{Peknej}, & \textit{Pekneho}. \\
    \textit{D}. & \textit{Peknemu}, & \textit{Peknej}, & \textit{Peknemu}. \\
    \textit{A}. & \textit{Pekni} (-\textit{eho}), & \textit{Peknu}, & \textit{Pekne} (-\textit{o}). \\
    \textit{L}. & \textit{Peknem} (-\textit{om}), & \textit{Peknej}, & \textit{Peknem} (-\textit{om}). \\
    \textit{I}. & \textit{Peknim}, & \textit{Peknej}, & \textit{Peknim}. \\
    \lspbottomrule
    \\
    \lsptoprule
    \multicolumn{4}{ c }{Plural.} \\
    \midrule
    \textit{N}. & \textit{Pekni}, & \textit{Pekne}, & \textit{Pekne}. \\
    \textit{G}. & \textit{Peknix}, & — & — \\
    \textit{D}. & \textit{Peknim}, & — & — \\
    \textit{A}. & \textit{Pekne} (-\textit{ix}), & \textit{Pekne}, & \textit{Pekne}. \\
    \textit{L}. & \textit{Peknix}, & & \\
    \textit{I}. & \textit{Peknimi}. &  & \\
    \lspbottomrule
\end{longtable}

This form does not differ from the other \is{Dialect}dialects in any way. Those adjectives that take the second form are formed from masculine nouns, such as \textit{sinov}, -\textit{a}, -\textit{o} [‘filial, the son’s’], \textit{kozlov}, -\textit{a}, -\textit{o} [‘caprine, the goat’s’] etc. Those that take the third form, furthermore, are formed from feminine nouns, such as \textit{materin} [‘maternal, mother’s’], \textit{sestrin} [‘sororal, the sister’s’], \textit{tetkin} [‘materteral, the aunt’s’], -\textit{a}, -\textit{e}, -\textit{o}, but adjectives of this kind do not differ at all from the previous form shown in terms of \is{Declension!Adjective declension}inflection. So it is superfluous to establish among the \il{Pannonian}Pannonians three forms of \is{Declension!Adjective declension}inflecting adjectives, for all adjectives among the \il{Pannonian}Pannonians follow the form discussed above. This is also why the \il{Bohemian}Bohemian calls adjectives of that type, such as \textit{sestrin}, \textit{bratov} [‘fraternal, the brother’s’], \textit{otcov} [‘paternal, the father’s’], \textit{materin} etc. possessive adjectives, and does not attribute to them a particular way of \is{Declension!Adjective declension}inflecting other than the normal one, as among the \il{Pannonian}Pannonians. The \il{Pannonian}Pannonians, just like the \il{Polish}Poles, also pronounce the third person neuter plural nominative pronoun with -\textit{e}, which is also why the adjectives follow this [95] ending, such as \textit{one}, and hence also \textit{pekne} etc. Nevertheless, the authority of the \il{Old Church Slavonic}ancient \is{Dialect}dialect, and current usage in \il{Russian}Russian, \il{Bohemian}Bohemian, and other, southern \is{Dialect}dialects, leads me to posit the singular -\textit{o}, but plural -\textit{a}, and thus the form of \is{Declension!Adjective declension}inflecting discussed above corresponds exactly with the other \is{Dialect}dialects.

\subsection*{\hspace*{\fill}§. 13.\hspace*{\fill}}

Having clarified this, it remains to be investigated whether the nominal or pro\-nominal method of inflecting should be adopted. That method of \is{Inflection}inflection which conforms to the \is{Genius}genius of the language and prevails in all \il{Slavic}Slavic \is{Dialect}dialects should be adopted. Such, indeed, is the pronominal inflection, for it was common among the \il{Old Church Slavonic}ancient Slavs, and now prevails in all \is{Dialect!Living dialect}living dialects, as the \is{Declension!Adjective declension}inflection of adjectives has shown. These facts elicit the following universal method of inflection:

\begin{longtable}{ l l l l }    
    \lsptoprule
    \multicolumn{4}{ c }{Singular.} \\
    \midrule
    & \textit{m}. & \textit{f}. & \textit{n}. \\
    \textit{N}. & \textit{Dobri} & \textit{Dobra}, & \textit{Dobro} (\textit{e}). \\
    & [‘good’], & & \\
    \textit{G}. & \textit{Dobrego}, & \textit{Dobrej}, & \textit{Dobrego}. \\
    \textit{D}. & \textit{Dobremu}, & \textit{Dobrej}, & \textit{Dobremu}. \\
    \textit{A}. & \textit{Dobri}, or for animates: \textit{Dobrego}, & \textit{Dobru}, & \textit{Dobro} (\textit{e}). \\
    \textit{L}. & \textit{Dobrem}, & \textit{Dobrej}, & \textit{Dobrem} (\textit{om}). \\
    \textit{I}. & \textit{Dobrim}, & \textit{Dobru}, & \textit{Dobrim}. \\
    \lspbottomrule
    \\
    \lsptoprule
    \multicolumn{4}{ c }{[96] Plural.} \\
    \midrule
    & \textit{m}. & \textit{f}. & \textit{n}. \\
    \textit{N}. & \textit{Dobri}, & \textit{Dobre}, & \textit{Dobra}. \\
    \textit{G}. & \textit{Dobrix}. & — & — \\
    \textit{D}. & \textit{Dobrim}. & — & — \\
    \textit{A}. & \textit{Dobri}, or for animates: \textit{Dobrix}, & \textit{Dobre}, & \textit{Dobra}. \\
    \textit{L}. & \textit{Dobrix}. & & \\
    \textit{I}. & \textit{Dobrimi}. &  & \\
    \lspbottomrule
\end{longtable}

All adjectives follow this pronominal form without exception. The nominal paradigm seems to have come into \il{Slavic}Slavic from the \il{Greek}Greek text via Bible translation, but recent editors of the Bible have abandoned it, substituting pronominal forms as genuinely \il{Slavic}Slavic.

\section*{Section VI. \textit{On comparison}. [p. 96--103]}
\addcontentsline{toc}{section}{Section VI. \textit{On comparison}. [p. 96--103]}

\subsection*{\hspace*{\fill}§. 1.\hspace*{\fill}}

Some quality of an object of speech may be compared in relation to one or more objects similar to it, and hence grammarians distinguish between the degrees of comparison. And grammarians do indeed mark out three degrees, but without any foundation. For if one object is compared with another, yet neither of these two shows a greater degree [97] of the same quality, then the grammarian abandons the degrees of comparison. Hence, an adjective expressing the quality of a certain object is incorrectly designated the “positive degree”.\footnote{The base form of the adjective is known in \il{Latin}Latin as the \textit{gradus positivus}, ‘positive degree’. For example, “tall” is the positive degree, “taller” is the comparative, and “tallest” is the superlative. \ia{Herkel, Jan}Herkel seems to be criticizing the term \textit{gradus positivus} without offering an alternative.} But if some quality is observed to be greater in one of the two objects, then the expression of this quality can be called “the comparative”; all the more so, if it were compared with several objects as far as that quality is concerned, and if a quality would be observed to be more eminent in a certain object than all the others, the grammarians commonly call this degree “the superlative”. But this is nothing else than a greater existence of the same quality in one among several objects with the quality compared. So, for instance, if I would say: this youngster is more modest than his other schoolmates, or that youngster is the most modest of his schoolmates. These examples show that there is strictly only one degree of comparison, namely with one or more objects. Let us now see how the various \il{Slavic}Slavic \is{Dialect}dialects express comparison.

\subsection*{\hspace*{\fill}§. 2.\hspace*{\fill}}

In all \il{Slavic}Slavic \is{Dialect}dialects the degree of comparison is expressed in two ways, namely by adding the particle -\textit{шi} or -\textit{ji} to the adjective. The Bible reads thus: \textit{чistjei} [‘cleaner’], \textit{чistшi} [‘cleaner’], \textit{чistjejшi} [‘cleanest’]. The final expression, made by combining the two other forms, is sometimes called by grammarians the longer comparative, or sometimes the superlative. Yet the \il{Old Church Slavonic}ancient [98] Slavs, if they attributed to an object a greater excellence in some quality than to others, expressed the idea clearly with words, as follows: \textit{Preчistjei rucie tvoi} [‘your cleanest hands’]. The following, however, are particles denoting eminence: \textit{pre}-, \textit{vse}-, \textit{naj}- as follows: \textit{Premilostivjejшi Boƶe!} [‘most graceful God!’], \textit{Vsemilostivjejшi} [‘most graceful’], or also by combining the particles, as follows: \textit{Vsepresvietliejшi} [‘most illustrious’]. And this way of expressing a most eminent quality also prevails now in all \is{Dialect}dialects, but they are \is{Declension!Adjective declension}inflected as adjectives, since they are indeed adjectives expressing the quality of an object in a greater degree than the others.

\subsection*{\hspace*{\fill}§. 3.\hspace*{\fill}}

The \il{Russian}Russians’ expression of comparison agrees with the \il{Old Church Slavonic}ancient \is{Dialect}dialect, as follows: \textit{tonшi} [‘thinner’], \textit{mladшi} [‘younger’], \textit{starшi} [‘older’] from \textit{tonki}, \textit{mladi}, \textit{stari}, or by interjecting the syllable \textit{jei} to the former expression, as follows: \textit{starejшi}, \textit{tonчejшi}; this last expression the grammarians call the superlative. In the meantime, the \il{Russian}Russians also form the superlative by prefixing to the former expression: \textit{pre}-, \textit{vse}-, \textit{naj}-, as follows: \textit{najjadovitejшi} [‘most poisonous’], \textit{naive\-liчaiшi} [‘the greatest’], \textit{vsepokorneiшi} [‘most humble’].

\subsection*{\hspace*{\fill}§. 4.\hspace*{\fill}}

The \il{Polish}Poles likewise speak like this, yet they change the letter ȣ into \textit{s} according to the \is{Genius}genius of the \is{Dialect}dialect, as follows: \textit{grubi}, \textit{grubsi} [‘fatter’], \textit{bogati}, \textit{bogatsi} [‘richer’], \textit{prosti}, \textit{prostiejsi} (\textit{prośćiejszi}) [‘simpler’], \textit{najbogatsi} [‘richest’], \textit{naiglupsi} [‘stupidest’], \textit{najmilsi bratia} [99] (\textit{najmilsi bracia}) \textit{prenajveliebnejsi} [‘dearest \linebreak{} most respectable brothers’]. The \il{Bohemian}Bohemians likewise form the comparative like this, namely by adding \textit{шi}, or by a combination of consonants so that the euphony is not harmed: \textit{ejшi}, or only \textit{eji} in conformity with the \il{Old Church Slavonic}ancient and other \is{Dialect}dialects, as follows: \textit{hlubшi} [‘deeper’], \textit{dalшi} [‘further’], \textit{шirшi} [‘wider’], \textit{uzшi} [‘narrower’], \textit{kratшi} [‘shorter’], \textit{tmavejшi} [‘darker’], \textit{libejшi} [‘more pleasant’], \textit{чernejшi} [‘blacker’], \textit{xitrejшi} [‘cleverer’], \textit{чisteji} [‘cleaner’] \textit{pekneji} [‘nicer’], \textit{sladceji} [‘sweeter’] etc. The latter way of speaking takes place especially in adverbs; if the prefix \textit{naj}- (\textit{nej}- among the \il{Bohemian}Bohemians) is added, it will form the superlative. The \il{Pannonian}Pannonians likewise pronounce the comparative like this, namely through -\textit{шi}, or -\textit{ejшi}, so that \textit{tvrdi}, or among the \il{Pannonian}Pannonians who smoothen it \textit{tvardi}, \textit{tverdi} [‘hard’] becomes \textit{tvardшi}, \textit{najtvardшi}, and \textit{krasni} [‘beautiful’] becomes \textit{krasnejшi}, \textit{najkrasnejшi} etc.. An alternate form uses -\textit{eji}, thus \textit{perv} [‘early’], \textit{perveji}, \textit{najperveji}, \textit{dal} [‘far’], \textit{daleji}, \textit{najdaleji}; this likewise takes place especially in adverbs.

\subsection*{\hspace*{\fill}§. 5.\hspace*{\fill}}

The \il{Pannonian}Pannonian grammarian, just like other \is{Dialect}dialect grammarians, also establishes various rules for forming the comparative. He dwelt not so much on the final syllables, the essential syllables of the comparative that appear in all \is{Dialect}dialects, namely -\textit{ejшi} or -\textit{eji}, but on changing consonants in the adjective itself. For instance \textit{drahi} [‘dear’], \textit{dluhi} [‘long’], \textit{suxi} [‘dry’] etc.: the \il{Pannonian}Pannonian teaches that here \textit{h} changes into \textit{k}.\footnote{\citet[55]{bernolak_grammatica_1790}.} This mutation, however, is grounded neither in etymology nor in universal usage, but instead in various individual misuse, from which the grammarian forges a rule. Nevertheless, misuse can never rest on firm principles; for this reason, [100] the grammars of \is{Dialect}dialect scholars abound in such passages, with exceptions and exceptions to exceptions. Let us, therefore, examine the abovementioned adjectives to see whether the letter change of \textit{h} into \textit{k} is necessary at all. No, it is not: for if I say \textit{drahi}, \textit{dluhi}, \textit{suxi}, the comparative will be \textit{drahшi}, \textit{dluhшi}, \textit{suxшi}, confirmed by usage. But if I say \textit{dragi}, \textit{dlugi} etc., the comparative will likewise be regular, \textit{dragшi}, \textit{dlugшi}, or \textit{dragejшi} etc., and not \textit{drakшi}, \textit{dlukшi}, or \textit{draƶшi} etc., as grammarians burdened by varying usage would instruct. This way of teaching and writing does not suit philology, and overwhelms learners’ abilities, because it is grounded in vicious usage. 

The rule, therefore, must stand as long as all \is{Dialect}dialects do not overthrow it together with its basic principles, as appears to be the case in the following words: \textit{mal} [‘small’], \textit{zl} [‘bad’], \textit{dobr} [‘good’], \textit{velik} [‘big’]. But the \il{Russian}Russians also regularly say \textit{malejшi}, so the exception for \textit{menшi} is gone. For \textit{velik} the \il{Carinthian}Carinthians say \textit{vekшi}, \textit{velikejшi}, just like \textit{sladkejшi} [‘sweeter’], \textit{шirokejшi} [‘wider’], \textit{kratkejшi} [‘shorter’], \textit{tenkeiшi} [‘thinner’] etc. Yet there are some who categorize certain adjectives as irregular which in other \is{Dialect}dialects are regular. They change, for instance, \textit{dlug} [‘long’] into \textit{dolg}, from which they form the comparative \textit{dalшi} or \textit{dalji} [‘longer’], and behold what great confusion results when one departs from a fixed principle, since in other \is{Dialect}dialects \textit{dalшi}, \textit{daleji} means ‘farther’.

\subsection*{\hspace*{\fill}§. 6.\hspace*{\fill}}

Two more adjectives remain to be considered: \textit{zl} = \textit{zli} [‘bad’] [101] and \textit{dobr} = \textit{dober} = \textit{dobri} [‘good’]. In all \is{Dialect}dialects, they are irregular, but their irregularity, too, is different. We should therefore inquire into the cause of the irregularity and the difference. Here we should remark that as long as the language and the people itself were still in their infancy, distinct ideas that shared some common characteristics were very often expressed by the same word. So much is clear both from books and from very ancient languages. Yet those ideas on the quality of objects were the most frequent, which most often applied to the physical condition of man, such as the idea of “good” and the idea of “bad”. Hence, they indicated everything that pleased them with the word for ‘good’, and everything that displeased them with the word for ‘bad’. For instance, we know that ancient peoples, and especially the Slavs, delighted in the color white, and they expressed this idea with the word for ‘good’, equating the white with the beautiful. On the other hand, objects which triggered an unpleasant sensation, such as something bitter, or a burning sensation on the body, they indicated with the generic word for ‘bad’. Thus, \textit{dober} comes from \textit{doba}, which means ‘moment in time’ or ‘form’, which is why some peoples have taken \textit{osdoba} (‘decoration’) to mean ‘beautiful’, others as ‘white’, and hence \textit{dober} = \textit{dobri}. Some attribute the comparative \textit{lepшi} to it, others \textit{bolшi}. The word \textit{lepшi} comes from \textit{lep}, ‘beautiful’, but \textit{bolшi}, however, comes from \textit{biel}, \textit{bel}, \textit{bil}, \textit{bol}, ‘white’, which is why in \il{Slavic}Slavic mythology \textit{Belbog}, the ‘white god’, is the ‘good God’ etc. For this reason, confused expressions should be eliminated. From \textit{dobri} one should say \textit{dobriejшi}; from \textit{lepi}, -\textit{a}, -\textit{o} one should say \textit{lepшi}, from \textit{biel}, \textit{bol} one should say \textit{biolшi} etc. [102] Similarly, \textit{zl} = \textit{zl} ‘bad’, and what is bitter gives a bad sensation, and what burns gives a bad sensation, hence \textit{gor} = \textit{gorek} = \textit{gorki} ‘bitter’, the comparative of which is \textit{gorшi}, \textit{gorji}, \textit{gorkejшi}, \textit{gorjei}, and this word, designating a specific quality, they have also transformed into the generic. So the comparative of \textit{zli} is sometimes \textit{gorшi} instead of \textit{zlejшi}. These confused expressions of ideas should be restrained. Otherwise, if I would say \textit{Ten чelovjek jest lepшi} [‘this person is \textit{lepшi}’] etc., the \il{Illyrian}Illyrian will take it to mean ‘more beautiful’, but the \il{Polish}Pole to mean ‘better’. But if I would critically-etymologically say \textit{Ten чelovjek jest dobrejшi}, or \textit{dobreji} [‘this person is better’], everyone would understand the true meaning.

The \il{Carinthian}Carinthians, like the others, form the comparative with -\textit{шi} or -\textit{ji}, such as \textit{hitri} [‘clever’], \textit{hitrejшi}, or \textit{hitreji}; if \textit{naj} (among the \il{Windic}Winds \textit{nar}) is prefixed to it, it will be a superlative, as: \textit{najvisokeiшi} [‘tallest’], \textit{najшirokeiшi} [‘widest’] etc.

The \il{Croatian}Croatians, \il{Slavonian}Slavonians, \il{Dalmatian}Dalmatians prefer to express the comparative with -\textit{ji}; they form the superlative by prefixing \textit{naj}- to it, or the particles \textit{pri}- (\textit{pre}-), \textit{veчe}-, such as: \textit{sveti} [‘holy’], \textit{svetji} [‘holier’], \textit{najsvetji} [‘holiest’], \textit{sladki} [‘sweet’], \textit{sladji} [‘sweeter’], \textit{najsladji} [‘sweetest’], \textit{lepi} [‘beautiful’], \textit{lepшi} [‘more beautiful’], \textit{najlepшi} [‘most beautiful’], or \textit{prelepi}, such as: \textit{prilipia} (\textit{prelepia}) \textit{ƶena umerla jest} [‘the most beautiful woman is dead’].

\subsection*{\hspace*{\fill}§. 7.\hspace*{\fill}}

The Slavs also express adjectives in diminutive form for tenderness, such as \linebreak{}\textit{mali} [‘small’], \textit{malinki}, \textit{maleшenki} [‘dear little one’] etc. \textit{dobri} [‘good’], \linebreak{}\textit{dobruшki};\footnote{\il{Slavic}Slavic diminutives generally lack easy \il{English}English counterparts, which is why we refrain from glossing all diminutives. While the diminutive of \textit{mili} ‘dear’ has a counterpart in “darling”, the diminutive of \textit{dobri} ‘good’ can only be approximated in \il{English}English with a phrase such as “dear good one”. \il{Slavic}Slavic \textit{dobruшki} does, however, have a counterpart in \il{German}German \textit{Guterchen} from \textit{gut} ‘good’.} \textit{tenki} [‘thin’], \textit{tenuшki}, \textit{mili} [‘dear’] \textit{milunki}, \textit{miluшki} [‘darling’], \linebreak{}\textit{mladi} [‘young’], \textit{mladuшki}, or \textit{mladunki} = \textit{molodi}, \textit{molodenki}, \textit{moloduшki} etc. The \il{Russian}Russian sings as follows: [103]

\begin{longtable}{ l l }
    Maleшenki solovejko (\textit{a}) & [Dear little nightingale \\
    Чom ti ne шчebeчeш?	& Why don’t you sing? \\
    Uƶ rad by ja шчebetati & I would like to sing, \\
    da (\textit{b}) golosu ne maju. & But I have no voice. \\
    Molodenkij kozaчenko! & Dear young cossack, \\
    Чom ti ne ƶeniш sia? & Why don’t you marry? \\
    Uƶ rad by ja ƶeniti sia & I would like to marry \\
    da dolu (\textit{c}) ne maju. & But I have no fortune. \\
    Poterial ja svoju doliu & I lost my fortune \\
    Xodiuчi v dorogu (\textit{d}) etc. & Walking to distant lands.] \\
\end{longtable}

(\textit{a}) \textit{salavik} ‘nightingale’ (\textit{b}) \textit{da} ‘but, because’ (\textit{c}) ‘inheritance, fortune’ (\textit{d}) ‘distant, foreign regions’.\footnote{The \il{Russian}Russian \textit{дорогу} (acc.) means ‘road, path, way’. \ia{Herkel, Jan}Herkel wrongly glosses it as \il{Latin}\textit{exterae}, \textit{peregrinae orae}.}

\section*{Section VII. \textit{On verbs, and their inflection}. [p. 103--153]}
\addcontentsline{toc}{section}{Section VII. \textit{On verbs, and their inflection}. [p. 103--153]}

\subsection*{\hspace*{\fill}§. 1.\hspace*{\fill}}

Nothing in nature exists without reason, and hence without function, immediately apparent or not. However, that function is fulfilled in time; so time is thus that moment when something’s function is carried out, or shown to have already been carried out, or still needs to be carried out. Hence, grammarians divide time into present, past, and future, and the \is{Genius!Genius of the Slavic language}genius of the \il{Slavic}Slavic language rejects further grammatical divisions of time.

\subsection*{\hspace*{\fill}§. 2.\hspace*{\fill}}

Before we turn to the \is{Conjugation}inflection of verbs, however, it is important to clear up the meaning and \is{Genius!Genius of the Slavic language}genius of the \il{Slavic}Slavic verbs [104], since the genius of this language differs from that of all other European languages. As some \il{Slavic}Slavic grammarians have followed the norms of other languages when composing their grammars, they have entangled themselves in inextricable difficulties. In terms of meaning, verbs in the \il{Slavic}Slavic language are: 1) \textit{original verbs}; 2) \textit{verbs composed by adding a prefix to an original verb}; 3) \textit{frequentative verbs}; 4) \textit{double frequentative verbs}; 5) \textit{factitive verbs}; 6) \textit{instantaneous original verbs}; 7) \textit{instantaneous derivative verbs}. Original verbs are those which denote a simple function of an entity, and without any modification, such as: \textit{orati} [‘to plow’], \textit{tkati} [‘to weave’], \textit{spati} [‘to sleep’], \textit{sjeti} [‘to sow’], \textit{znati} [‘to know’], \textit{mereti} [‘to die’], \textit{duti} [‘to blow’], \textit{gniti} [‘rot’], \textit{ƶiti} [‘to live’], \textit{piti} [‘to drink’], \textit{kriti} [‘to cover’], \textit{ƶati} [‘to harvest (a crop)’], \textit{чati} [‘to start’] etc. From these the type 2 composites are made, such as: \textit{viorati} [‘to plow’], \textit{pritkati} [‘to attach’], \textit{nasjeti} [‘to sow’], \textit{prespati} [‘to spend the night’], \textit{uznati} [‘to recognize’], \textit{umereti} [‘to die’], \textit{naduti} [‘to inflate’], \textit{sgniti} [‘to decompose’], \textit{preƶiti} [‘to survive’], \textit{odpiti} [‘to sip’], \textit{odkriti} [‘to uncover’], \textit{zeƶati} [‘to complete the harvest’], \textit{naчati} [‘to begin’] etc. 3. Frequentative verbs are formed from original verbs by interjecting the syllable \textit{va}, such as: \textit{oravati} [‘to plow’], \textit{tkavati} [‘to weave’], \textit{spavati} [‘to sleep’], \textit{znavati} [‘to know’]; \textit{mereti} or \textit{mreti} [‘to die’] does not allow a frequentative, because death is unique, \textit{duvati} [‘to blow’], \textit{gnivati} [‘to rot’], \textit{ƶivati} = \textit{ƶiviti} [‘to live’], \textit{pijavati} [‘to drink frequently’], \textit{krivati} [‘to cover’], \textit{ƶavati} [‘to harvest regularly’], \textit{чavati} [‘to start’], hence \textit{naчavati} [‘to begin’] etc. 4. Double frequentative verbs are made by duplicating the syllable \textit{va}, such as: \textit{oravavati} [‘to plow regularly’], \textit{tkavavati} [‘to weave regularly’], \textit{spavavati} [‘to sleep regularly’] etc. Factitives are made with the preposition \textit{po}-, by interjecting the syllable -\textit{ju}, such as: \textit{poorujem} = \textit{pooruju} [‘I plow it all’], \textit{popijuju} [‘I drink it all up’], because indeed the use of these verbs is restricted only to the present tense. Because of this, they are replaced [105] by frequentatives in the perfect and future. 6. Instantaneous verbs denote the function of an entity that is instantaneous to such an extent that they do not even need any expression of the present, because their function is completed sooner than can be expressed by verbs: for instance, \textit{streliti} [‘to shoot’] occurs by a little pressure, \textit{dati} [‘to give’], \textit{kupiti} [‘to buy’] depends on voluntary assent.\footnote{The \il{Latin}Latin expression \textit{nutus voluntatis} comes from the Early Christian \il{Latin}Latin author \ia{Augustine}Augustine, who uses it almost forty times throughout his oeuvre.} And therefore, these verbs are instantaneous, and indeed original. Type 7 are instantaneous derivative verbs made from original verbs by interjecting the letter \textit{n}. Consider, for instance, the original verbs \textit{padati} [‘to fall’], \textit{mikati} [‘to move around’], \textit{dvigati} [‘to lift’], \textit{sekati} [‘to chop’], \textit{duxati} [‘to breathe’], \textit{pukati} [‘to burst’] etc.; interjecting the letter \textit{n} into them makes instantaneous verbs such as: \textit{padnuti}, \textit{miknuti}, \textit{dvignuti}, \textit{seknuti}, \textit{duxnuti}, \textit{puknuti} etc. And hence it is good to distinguish \textit{streliti} from \textit{strielati} = \textit{strielavati} [‘to shoot’], \textit{dati} from \textit{davati} [‘to give’], \textit{kupiti} from \textit{kupivati} [‘to buy’], similarly \textit{dixati} from \textit{dixnuti} [‘to breathe’] etc., because usage founded in the \is{Genius}genius of the language also accurately distinguishes these meanings. Otherwise, the \is{Genius}genius of the language will be violated, and solid and simple principles of the language will not be discovered. This is why works by grammarians who do not distinguish these verbs are entangled in extreme difficulties.

\subsection*{\hspace*{\fill}§. 3.\hspace*{\fill}}

The moods by means of which the function of entities is designated are usually these above all, namely: something is either indicated, or commanded, or wished. Hence, the moods of the grammarians emerged: indicative, imperative, optative. But the infinitive ceases to be a mood, [106] because it in no way defines anything, but is itself only a word, since in relation to mood, tense, person, and number it allows itself to be modified. So, if I would say “it is useful to work”, or “work is useful”, then here “to work” is a word, because it is used for the function of a certain entity, to be designated by variously modifying it, such as \textit{dielati} [‘to do’], \textit{pisati} [‘to write’], \textit{moliti} [‘to pray’] etc.; hence \textit{dielanje} [‘doing’], \textit{pisanie} [‘(act of) writing’], \textit{molienie} or \textit{modlienie} [‘prayer’] etc. are nouns designating a present function, but \textit{delanost} [‘artificiality’], \textit{psanost} [‘written edition’] etc. are nouns designating a past function. Also, four types of adjectives emerge, namely 1. of the present active function, such as: \textit{pisajuci} [‘writing’], \textit{delajuci}, -\textit{a}, -\textit{o} [‘doing’] etc., though in some \is{Dialect}dialects \textit{c} changes into \textit{ч}, such as \textit{pisajuчi}, \textit{delajuчi} etc. The \il{Russian}Russian sings as follows: \\

\noindent\hspace*{1cm}Oj! zaчuje stara mati \\
\hspace*{1cm}Sidiuчa u xati = xalupa = xatia. \\
\hspace*{1cm}[Oh! The old mother hears it, \\
\hspace*{1cm}Sitting by the hut = summer house = cottage.] \\

2. There are adjectives of a past active function, ending in -\textit{vsi}, -\textit{a}, -\textit{o}, but the feminine and neuter ending is sometimes neglected, and for every gender only the masculine ending is used, as in the Song of the \il{Russian}Russian woman deploring her lover going away:

\begin{small}
\begin{longtable}{ l l }
    Svisnuv kozak na konia & [A Cossack whistling to his horse, \\
    Budi zdrova moloda etc. & Farewell, young maiden etc. \\
    Vyiшla ruчki zalomavшi & She came out, wringing her hands, \\
    A tiazenko zaplakavшi etc. & Crying bitterly etc. \\
    Bielix ruчok ne lomaj & Do not wring your white hands, \\
    Чernix oчok ne uteraj & Do not rub your black eyes \\
    Budi zdrova, bo ja idu & Farewell, for I am going \\
    Uƶe za Dunaj. etc. [107] & Now beyond the Danube. etc.]\footnote{A noted folk song by \ia{Klymovskyi, Semen}Semen Klymovskyi, probably taken from \citet[156]{celakovsky_slowanske_1822}. Čelakovský presented the song \il{Little Russian}as “little Russian”. Variants appear in Ivan \citeauthor{pratsch_i_1806}, \textit{Ѣхавъ козакъ за Дунай} (\citeyear[75]{pratsch_i_1806}); “Song 91” (\citeyear[151]{pratsch__1821}); “Song 185” (\citeyear[174]{pratsch__1819}).} \\
\end{longtable}
\end{small}

Here \textit{zalomavшi}, \textit{zaplakavшi}, \textit{zaplakavшa}, \textit{zaplakavшo} is an adjective formed from the perfect \textit{zaplakav} [‘having wept’] etc. 3. There are adjectives of the present passive, such as \textit{znami}, -\textit{a}, -\textit{o} [‘(being) known’], \textit{vedomi}, \textit{a}, \textit{o} [‘(being) known’], which will be treated in greater detail below. 4. Adjectives of the preterit passive, such as: \textit{viden} [‘seen’], \textit{uчen} [‘learned’], \textit{nosen}, -\textit{a}, -\textit{o} [‘worn’], \textit{delan} [‘made’] etc. Grammarians have given these adjectives various names. They are sometimes called participles, sometimes transgressives, sometimes gerunds etc. They are perhaps called participles because they participate with verbs in the expression of tense and with nouns in \is{Declension!Noun declension}declension. In the meantime, expressions such as “participle”, “supine”, “gerund” etc., which convey no real meaning at all, should be obliterated. For instance, the gerund is nothing more than an adverb, on which more below.

\subsection*{\hspace*{\fill}§. 4.\hspace*{\fill}}

One grammarian has tried to force \il{Russian}Russian into an eightfold division of tense, including the present, indefinite perfect, simple perfect, perfect preterit, pluperfect, indefinite future, simple future, and future perfect. But all Slavic nations have rejected this contradictory fiction by common agreement, because they observed that several verbs are used for one meaning, for example, the verbs \textit{dvigati}, \textit{dvigavati}, \textit{sdvigati}, \textit{dvignuti} [‘to lift’] share one meaning. At any case, the grammarian was not a native \il{Russian}Russian, or if he was, he ignored the \is{Genius}genius of his mother tongue.\footnote{\ia{Herkel, Jan}Herkel probably refers to Johann \citeauthor{heym_russische_1794}’s \citeyear{heym_russische_1794} grammar that actually posits ten tenses for \il{Russian}Russian; see \citet[92--94]{heym_russische_1794}. Heym was born in Braunschweig and moved to Russia only in his twentieth year; he ultimately became professor and librarian at Moscow University. \ia{Herkel, Jan}Herkel probably based his discussion on \citet[303]{kopitar_grammatik_1808}.}

[108] But this should not surprise us, as we observe that the art of explaining languages makes hardly any progress, since the more grammarians amass pointless rules, the more they believe they have performed their duty. Likewise the Slavs blindly imitate the \il{Latin}Latins in the impetuous fabrication of terms itself. The \il{Latin}Latins call the various flections of nouns “cases”, the Slavs \textit{padez}; the nominative, genitive, dative, accusative, vocative, ablative are called \textit{imenitelni}, \textit{roditelni}, \textit{datelni}, \textit{vinitelni}, \textit{zvatelni} etc. However, these designations are so far from capturing the essence of the matter that they rather produce something ridiculous, for if I would say: “these oxen are my father’s”, \textit{Ti voli so mojego otca}, then here \textit{otca} would be the \textit{roditel} [‘genitor’] of ‘oxen’, or if I say “I love God”, \textit{Ja milujem} or \textit{liubim Boga}, here \textit{Boga} would be the \textit{vinitel} [‘accuser’]. -- At any rate, the following designations would better capture the idea of the matter, namely flexional endings: subjective, possessive, receptive, objective, appellative etc.

\subsection*{\hspace*{\fill}§. 5.\hspace*{\fill}}

The illustrious \ia{Dobrovský, Josef}\textit{Dobrovský}, that very famous researcher of the \il{Old Church Slavonic}ancient \is{Dialect}dialect, has established six forms of \is{Conjugation}verb inflections, where his predecessors proposed only two.\footnote{\citet[346--347, 384]{dobrovsky_institutiones_1822}.} Yet, they have illustrated them with more paradigms. Since, however, he observed that not all variations found in the Bible could be captured by means of the six established forms, he declared some of them \is{Polonism| see {Polishism}}\is{Polishism}Polonisms, some \is{Russianism}Russianisms, some \is{Serbism| see {Serbianism}}\is{Serbianism}Serbisms, and sometimes he blamed the carelessness of scribes. [109] Nonetheless, he faced difficulties which he attributed to none of these causes, and ultimately acknowledged that he did not entirely understand why things were written as he found them; see Part II, page 562.

\subsection*{\hspace*{\fill}§. 6.\hspace*{\fill}}

Several grammarians of the \il{Old Church Slavonic}ancient \is{Dialect}dialect establish two forms of verbs. They attribute three forms to the \il{Russian}Russian \is{Dialect}dialect. The \il{Polish}Poles recognize only one, \linebreak{}some \il{Bohemian}Bohemians divide the verbs into six forms. The \il{Pannonian}Pannonian imitates them. Southern grammarians, such as the \il{Serbian}Serbs, \il{Slavonian}Slavonians, \il{Croatian}Croatians, \il{Dalmatian}Dalmatians, and \il{Windic}Winds, prefer to illustrate the verbs with three forms, yet strictly speaking all verbs follow only one form. Now, what wisdom is there in such diverse grammatical opinions? The \il{Pannonian}Pannonian, as has been said, establishes six forms; hence it is important to consider the foundation on top of which those six forms have been built. For this reason, it will be worthwhile to show the original endings of each form of the verb from which the others are derived, so that one can plainly ask whether the established six forms cannot be reduced to one. They are as follows:\footnote{\citet[101--138]{bernolak_grammatica_1790}. As \ia{Buzássyová, Ľudmila}Buzássyová notes, \ia{Herkel, Jan}Herkel changes both \ia{Bernolák, Anton}Bernolák’s \is{Orthography}orthography and the names of \ia{Bernolák, Anton}Bernolák’s tenses. \ia{Bernolák, Anton}Bernolák gives: \il{Latin}Indefinitus modus \textit{wolať}, indicativus modus: Praeteritum Perfectum \textit{wolal sem}, Praeteritum Plusquam Perfectum \textit{Bol sem wolal}, Tempus Praesens \textit{wolám}, Tempus futurum \textit{Budem wolať}, Imperativus modus \textit{Wolag}, Gerundiuvus Modus Praeteritum \textit{wolawsi} (\textit{wolaw}), Gerundivus Modus Praesens \textit{Wolagic}, \textit{wolagice}, Participialis modus Praeteritum \textit{wolani}, Praesenc \textit{wolagici}, cf. \citet[216]{herkel_jan_2009}.}

\begin{table}
    \scalebox{0.81}{
    \begin{tabular}{ l l l l l l l l } 
    \lsptoprule
    & Infinitive & Present & Imperative & Perfect & Perfect & Present & Passive \\
    & & indicative & & & & participle & present \\
    & & & & & & & participle \\
    \midrule
    1 & \textit{volati} [‘to call’], & \textit{volam}, & \textit{volaj}, & \textit{volal}, & \textit{volav}, & \textit{volajuc}, & \textit{volan}. \\
    2 & \textit{lamati} [‘to break’], & \textit{lamem}, & \textit{lam}, & \textit{lamal}, & \textit{lamav}, & \textit{lamajuc}, & \textit{laman}. \\
    3 & \textit{sliшati} [‘to hear’], & \textit{sliшim}, & \textit{sliш}, & \textit{sliшal}, & \textit{sliшav}, & \textit{sliшic}, & \textit{sliшan}. \\
    & & & & & & & [110] \\
    4 & \textit{sati} [‘to sow’], & \textit{sejem}, & \textit{sej}, & \textit{sal}, & \textit{sav}, & \textit{sejic}, & \textit{sat}. \\
    5 & \textit{piti} [‘to drink’], & \textit{pijem}, & \textit{pi}, & \textit{pil}, & \textit{piv}, & \textit{pijic}, & \textit{pit}. \\
    6 & \textit{milovati} [‘to love’], & \textit{milujem}, & \textit{milui}, & \textit{miloval}, & \textit{miloval}, & \textit{milujic}, & \textit{milovan}. \\
    \lspbottomrule
    \end{tabular}}
\end{table}

The grammarian \textit{anxiously} specifies with prolix rules which verbs take which forms. For instance, the second form pertains to those verbs which end in -\textit{at} (-\textit{ati}) if preceded by the following syllables: \textit{ak}, \textit{am}, \textit{luh}, \textit{uh}, \textit{ip}, \textit{ok}, or the letters \textit{l}, \textit{r}, \textit{s}, \textit{t}, \textit{z}, but not if preceded by the vowels \textit{e}, \textit{i}; 2) verbs ending in -\textit{nut} (-\textit{nuti}) in the infinitive; 3) verbs ending in -\textit{et}, -\textit{st}, -\textit{zt} (-\textit{cti}, -\textit{zti}). These verb classifications, prepared with anxious care, do so very little to illuminate learners that they actually confuse them more, for another grammarian of the same \is{Dialect}dialect will establish other rules etc. Now, these things are not said with the aim of criticizing the author of the \il{Pannonian}Pannonian grammar, to whom immortal thanks are due as the most meritorious scholar of the \il{Pannonian}Pannonian \is{Dialect}dialect, but are emphasized in order to make apparent the extent to which \is{Dialect}dialect grammars are defective, because the \il{Slavic}Slavic languages, taken as a whole, are robbed of firm principles. For in the \il{Slavic}Slavic language, the \is{Dialect!Living dialect}living dialects show that there is no more than one form of \is{Conjugation}verb inflections, for the \il{Polish}Poles and southern Slavs, strictly speaking, recognize only one form, and the six \il{Pannonian}Pannonian forms can also be reduced to it according to the \is{Genius}genius of the language and usage. [111]

\subsection*{\hspace*{\fill}§. 7.\hspace*{\fill}}

The second form differs from the first in the present and the imperative, which in the first form is \textit{volam} [‘I call’], \textit{volaj} [‘call!’], but in the second form \textit{lamem} [‘I break’], \textit{lam} [‘break!’]. In Pannonia itself, the second form \textit{lamam}, \textit{lamaj} is also in use. Since it does not differ absolutely from the first, the second form is superfluous. After all, even if \textit{lamam} from \textit{lamati} were not in use, the second form would still be superfluous, for the terminations of the verbs are the same at the end. For this reason, they receive the same \is{Conjugation}inflectional form.

The third form is \textit{sliшati} [‘to hear’], but it is also said with \textit{a} changed to \textit{e} or \textit{i}, such as \textit{sliшiti}, \textit{sliшeti}. Hence, if I derive it from \textit{sliшati}, the present tense would be \textit{sliшam} in the fashion of the southern Slavs, but deriving it from \textit{sliшiti} results in \textit{sliшim} etc. One grammarian expounds the infinitive in the primitive form as \textit{sliшati} but the present as \textit{sliшim}; he is indeed correct in as far as he is a \is{Dialect}dialect grammarian, but not as a critical grammarian of \il{Slavic}Slavic. The imperative is expounded as \textit{sliш}, instead of \textit{sliшaj} of the southerners, or \textit{sliшej} of the \il{Bohemian}Bohemians, since the typical feature of the imperative of all verbs is -\textit{i}. That is clear from the \il{Old Church Slavonic}ancient as much as the \is{Dialect!Living dialect}living dialects, as the imperative of certain verbs lacking -\textit{i} is only an abbreviated way of speaking, and thus the third form likewise does not at all differ from the previous ones.

The fourth form is \textit{sati} [‘to sow’], but in \il{Slavic}Slavic also \textit{seti} is said, or \textit{sejeti}, or \textit{sjeti}, or \textit{sjejeti}, hence from \textit{sejeti} one should say \textit{sejem}, \textit{sej}, \textit{sel}, \textit{sejev}, [112] (\textit{sev}), \textit{sejen} etc., a regularity proved by usage in other \is{Dialect}dialects.

The fifth form is \textit{piti} [‘to drink’], hence \textit{piem}, or with an interjected \textit{j}, since the \il{Old Church Slavonic}ancients pronounced \textit{e} as \textit{je}. For this reason one can also write \textit{pijem}, \textit{pij}, \textit{pil}, \textit{piv}, \textit{pijan} etc. Likewise there is no difference from the previous ones.

The sixth form is \textit{milovati} [‘to love’], \textit{milujem}, \textit{miluj}, \textit{miloval} etc. This \is{Dialect}dialectal form is confused out of two verbs: \textit{miluti}, which is an original verb used by the southern Slavs with the frequentative form \textit{miluvati}, hence \textit{miluti} or \textit{militi}. In the \il{Windic}Windic fashion it means ‘to love’, ‘to have pity’, hence also among the \il{Pannonian}Pannonians: \textit{Boƶe! Smiluj sia nad nami} [‘God! Have mercy upon us’]. Hence \textit{miluti}, \textit{milujem}, but \textit{miluvati}, \textit{miluvam} should be said in the present; hence \textit{mila matka} [‘dear mother’], \textit{premileni sin} [‘dearest son’] is something else than \textit{milovana matka} [‘beloved mother’] etc. From these forms it is clear that \textit{miluti} and \textit{miluvati} are different verbs, and for this reason they also have a different meaning, even if \il{Pannonian}Pannonian usage has confused them.

\subsection*{\hspace*{\fill}§. 8.\hspace*{\fill}}

The grammarian expounds verbal adjectives, denoting a present action, or pre\-sent participles of some verbs with -\textit{ic}, such as: \textit{sejic} [‘sowing’], \textit{pijic} [‘drinking’], \textit{milujic} [‘loving’], but of other verbs with -\textit{uc}, such as: \textit{volajuc} [‘calling’], \textit{lamajuc} [‘breaking’]. Everyday usage in Pannonia, however, annuls this distinction. Participles shown taking -\textit{ic} may also end with -\textit{uc}, such as: \textit{sejuc}, \textit{pijuc}, \textit{milujuc}, and vice versa, but indeed also with -\textit{ac}, [113] giving \textit{sejac}, \textit{pijac} etc. Vowels change in the \il{Slavic}Slavic mouth, which is why the most pleasing forms should be adopted, not those which grammarians have been so anxiously establishing. Hence, the six \il{Pannonian}Pannonian forms, reflecting the \is{Genius!Genius of the Slavic language}genius of the \il{Slavic}Slavic language as revealed in the other \is{Dialect!Living dialect}living dialects, can be reduced to one, namely:

\begin{longtable}{ l l }
    \lsptoprule
    1. & \textit{Volati}, \textit{volam}, \textit{volaj}, \textit{volal}, \textit{volav}, \textit{volajuc}, \textit{volan}. \\
    2. & \textit{Lamati}, \textit{lamam}, \textit{lamaj}, \textit{lamal}, \textit{lamav}, \textit{lamajuc}, \textit{laman}. \\
    3. & \textit{Sliшati}, \textit{sliшam}, \textit{sliшaj}, \textit{sliшal}, \textit{sliшav}, \textit{sliшajuc}, \textit{sliшan}. \\
    4. & \textit{Sejeti}, \textit{sejem}, \textit{sej}, \textit{sejel}, \textit{sejev}, \textit{sejuc}, \textit{sejen}. \\
    5. & \textit{Piti}, \textit{pijem}, \textit{pij}, \textit{pil}, \textit{piv}, \textit{pijuc}, \textit{pijan}. \\
    6. & \textit{Miluti}, \textit{milujem}, \textit{miluj}, \textit{milul}, \textit{miluv}, \textit{milijuc}, \textit{milun}, or \textit{milen}. \\
    \lspbottomrule
\end{longtable}

See how six forms reduce to one, though some will say that this reduction feels forced, since the \il{Pannonian}Pannonian does not say \textit{pijan}, \textit{milun}, \textit{milen} etc., and use is the foremost grammarian etc. However, the \il{Slavic}Slavic language is taken here in general. Moreover, the southerners never say it otherwise than \textit{pijani чelovek} [‘drunk person’], and among the \il{Pannonian}Pannonians themselves \textit{milenka}, \textit{milunka} [‘sweetheart’] is used, which is a diminutive out of \textit{milena} [‘beloved’], but this is formed out of \textit{milen}, \textit{milena}, \textit{mileno} [‘loved’] etc. For here it is not the usage of one \is{Dialect}dialect that is taken as the basis, since no individual \is{Dialect}dialect allows for rational cultivation in the strict sense, because it considers only the usage in that [114] \is{Dialect}dialect. However, as \ia{Horace}Horace testifies, usage is the greatest tyrant in language, as it always introduces new rules burdened with exceptions, it mixes squares with circles. This cannot happen in the \il{Slavic}Slavic language if it is taken in general, if its \is{Genius!Genius of the Slavic language}genius is fully investigated, and if its various \is{Dialect}dialects are combined. For instance, the \il{Pannonian}Pannonian says \textit{pisati} [‘to write’] in the imperative as \textit{piш}, which is irregular, but other \is{Dialect}dialects take the regular \textit{pisaj}. The \il{Pannonian}Pannonian confuses \textit{miluti} with \textit{miluvati} and with \textit{miluvavati} as well, since it clearly occurs in schoolbooks that ‘I loved’ = \textit{miluvaval}, \textit{milovaval} etc.; that is, the original verb is confused with the double frequentative.

\subsection*{\hspace*{\fill}§. 9.\hspace*{\fill}}

Very many grammarians have rightly established the infinitive as the base form for the other forms, for the undetermined form of the verb lets itself be modified in relation to tense, person, number etc. For this reason, the other forms of the verb cannot be regarded as the base form, because they have already undergone some modification. Hence, the base form of the verb is the undetermined form of the verb ending in -\textit{ti}, such as: \textit{veriti} [‘believe’], \textit{uчiti} [‘teach’], \textit{kupiti} [‘buy’], \textit{шiti} [‘sew’]. Omitting the -\textit{ti} gives the imperative: \textit{veri}, \textit{kupi}, \textit{шi}; and thus it is very easy to form the principal forms of verbs from the infinitive, including the present, perfect, future, and verbal adjectives and nouns as well. The principal forms of the verbs in all \is{Dialect}dialects [115] end in the following way, namely: the present in -\textit{u} or -\textit{m}; the past tense in -\textit{x}, -\textit{l}, -\textit{v}; the imperative always in -\textit{i}, a letter which can also be written through \textit{j}, for it is much the same whether one writes \textit{stupai} or \textit{stupaj} [‘step!’]. The first person present is expressed with -\textit{u} or -\textit{m}. The \il{Old Church Slavonic}ancient \is{Dialect}dialect and \il{Russian}Russian in particular express the present through -\textit{u}, as in \textit{piu} [‘I drink’], \textit{шiu} [‘I sew’], but the other \is{Dialect}dialects, including \il{Pannonian}Pannonian, \il{Polish}Polish, and the southern \is{Dialect}dialects, use -\textit{m}, as in \textit{piem}, \textit{шiem}. Yet the \il{Polish}Poles leave out the letter \textit{m} after \textit{e}, and mark \textit{ę} with a cedilla: \textit{pię}, \textit{шię}, with the exception of \textit{viem} [‘I know’], \textit{jem} [‘I eat’], and those that take -\textit{am}, such as \textit{biegam} [‘I run’], \textit{poviedam} [‘I say’], \textit{scekam} [‘I bark’] etc. The \il{Bohemian}Bohemians pronounce some verbs with -\textit{u}, others with -\textit{m}, a few entirely with -\textit{i}, such as: \textit{ja piji}, ‘I drink’ etc. They thus have their own rules burdened with exceptions, which are founded uniquely on the varying usage of \is{Dialect}dialect. Hence it would be very well-considered to \is{Conjugation}conjugate all verbs with -\textit{u} or with -\textit{m}, to permit free choice between both endings in speaking and writing, because both are founded on the \is{Genius}genius of the language. For -\textit{u} is founded on the \il{Old Church Slavonic}ancient and the \il{Russian}Russian \is{Dialect}dialect, but -\textit{m} on all \is{Dialect}dialects, for the \il{Russian}Russians, too, pronounce some verbs with \textit{m}, such as: \textit{jam} [‘I eat’], \textit{snjem} [‘I will eat’], \textit{vjem} [‘I know’], \textit{dam} [‘I will give’], \textit{imam} [‘I have’] etc.\footnote{This list mainly contains \il{Old Church Slavonic}Old Church Slavonic forms, \textit{dam} being the only form that is also \il{Russian}Russian.}

\subsection*{\hspace*{\fill}§. 10.\hspace*{\fill}}

The expression of the perfect is already in the \il{Old Church Slavonic}ancient \is{Dialect}dialect read in a threefold form, as is clear from the records of the Bible, namely in -\textit{x}, -\textit{l}, -\textit{v}, such as \textit{tvorix}, \textit{чitax}, \textit{uчix}, or \textit{tvoriv}, \textit{чitav}, \textit{uчiv}, or \textit{tvoril}, \textit{чital}, \textit{uчil} [respectively ‘I created’, ‘I read’, ‘I learned’]. [116] Now, if we would consider the \is{Dialect!Living dialect}living dialects, the perfect with -\textit{x} has almost vanished, but there are traces of it in the southern \is{Dialect}dialects. For instance, the \il{Slavonian}Slavonian says: \textit{sluшah} [‘I listened’], \textit{uчih} [‘I learned’] etc. Among the \il{Pannonian}Pannonians, there are traces of it in \textit{byx} [‘I would’], expressing, so to speak, an auxiliary optative. Otherwise, in all \is{Dialect!Living dialect}living dialects, the perfect ends in -\textit{l}, for instance the \il{Russian}Russian says \textit{brati} [‘to take’], \textit{zvati} [‘to call’], \textit{ƶati} [‘to reap’] in the perfect as \textit{bral}, \textit{zval}, \textit{ƶal} etc. All \is{Dialect}dialects follow this way of speaking, founded, as it were, in the \is{Genius}genius of the language, so that, indeed, the letter -\textit{l} can be safely and universally established as the typical marker of the past tense. However, the \il{Illyrian}Illyrians confound it in the masculine with -\textit{o}, so instead of \textit{pekal}, \textit{pekao} [‘baked’] etc. But this substitution seems to have emerged from the kinship of sound between \textit{pekav} and \textit{pekao}, \textit{pisav} and \textit{pisao} [‘wrote’]. It is indeed clear from the fact that in the other genders the letter -\textit{l} is restored as typical feature of the perfect, such as \textit{snovao}, \textit{snovala}, \textit{snovalo} [‘warped; planned’], a way of writing that reflects a great partiality. Hence, the more recent and more cultivated \il{Illyrian}Illyrians should imitate the wholesome example of the most erudite \ia{Kopitar, Jernej}Kopitar, who, even though the \il{Windic}Winds pronounce the masculine perfect with -\textit{v}, has argued nonetheless that books should be written with -\textit{l}, to avoid separation from their fellow nationals through a special way of writing. For example, \textit{dal} [‘gave’], \textit{spal} [‘slept’] instead of \textit{dav}, \textit{spav}, even though \textit{dav}, \textit{spav} are also founded on the \is{Genius!Genius of the Slavic language}genius of the \il{Slavic}Slavic language, since the former forms occur among all Slavic nations. For instance, especially the \il{Russian}Russian says when two perfects occur together: \textit{Pristupiv skazal} [‘Having come near, he said’]\footnote{Mt 8:5 and 28:18 are possible sources for the example \textit{Pristupiv skazal}.} etc. instead of \textit{pristupil, i skazal} [‘he came near and said’].\footnote{In this passage, \ia{Herkel, Jan}Herkel confuses the gerund in \textit{-v}, a petrified form of the active past participle, with the preterit in \textit{-l}, whose final \textit{-l} in some \il{Slavic}Slavic languages evolved into \textit{-w} or \textit{-v}. These two forms have a completely different origin and function.} \il{Polish}Polish: \textit{Biegal po liesu vpav v jamu}, instead of: \textit{biegal po liesu i vpal} [117] (\textit{vpadel}) \textit{v jamu} [‘he ran through the forest and fell into a pit’]. Also in the Bible one reads as follows: \textit{zriv} [‘saw’], \textit{voliv} [‘chose’] etc. instead of \textit{zril}, \textit{volil} etc.\footnote{The words \textit{zriv} and \textit{voliv}, which according to \ia{Herkel, Jan}Herkel’s reasoning mean ‘saw’ and ‘chose’, actually mean ‘having seen’ and ‘having chosen’.} And it is called by grammarians the preterit gerund, and it means nothing else than a subject that has performed a certain function, and thus it is indeed a verb, if it retains the value of a verb, for instance if I would say: \textit{Aleksander sebrav vojsko iшel na Turka} [‘Alexander, having gathered an army, marched against the Turk’]; or if it would be treated as a subject having an adjective-like meaning which had performed a function, thus the \il{Polish}Pole: \textit{Krystus Pan wźiąwszj chleb lamal} or \textit{Kristus Pan vziavsi xleb lamal} [‘The Lord Christ, having taken the bread, broke it’]. Here, \textit{vziavsi} is a formal adjective derived from the verb \textit{vziati} [‘to take’], which is why one correctly says: \textit{Kristus pan vziav xlieb lamal}, or by agreeing in gender as a formal adjective: \textit{Kristus pan vziavsi xlieb lamal}.

\subsection*{\hspace*{\fill}§. 11.\hspace*{\fill}}

For the rest, the common and usual perfect ending among all Slavs is the letter -\textit{l}, with or without the auxiliary verb \textit{jesem}, \textit{jsem} [‘I am’], or through abbreviation \textit{sem}, \textit{som}. Such a way of speaking has flourished also in the \il{Old Church Slavonic}ancient \is{Dialect}dialect. This very fact is shown by the texts of the Bible, as one reads: \textit{чital} [‘I read’], \textit{tvoril} [‘I made’], or \textit{чital jesm} [‘I read’], \textit{tvoril jesm} [‘I made’]. Indeed, the renowned Dobrovský \ia{Dobrovský, Josef}ascribes to the \il{Russian}Russians Biblical texts using the perfect without \textit{jesm}, but elsewhere the very same man acknowledges that original perfects can be found in this way, such as at [2] Corinthians 2:5: \textit{Aшчeli kto oskorbil mene, ne mene oskorbi} [‘And if any one have caused grief, he hath not grieved me’] etc. Indeed, we can safely use the perfect ending in -\textit{l}, as [118] all \is{Dialect}dialects confirm.

\is{Dialect}Dialect grammarians also discuss the abbreviated perfect, such as the \il{Russian}Russian says \textit{strig} [‘I sheared’], \textit{liez} [‘I climbed’], instead of \textit{strigel}, \textit{liezel} etc. The \il{Bohemian}Bohemian says: \textit{spad} [‘I fell’], \textit{utek} [‘I escaped’], \textit{pribeh} [‘I came running’], \textit{virost} [‘I grew up’], \textit{zamk} [‘I locked’] etc. instead of \textit{spadel}, \textit{utekel}, \textit{pribehel}, \textit{virostel}, \textit{zamkel} etc. But the usage of swallowing in this way is not of such importance that it can claim a rule for itself, since some people of that very same \is{Dialect}dialect do not swallow, but follow the plain way of speaking. Yet grammarians who have collected similar ways of speaking, or rather swallowing, have done so reasonably. For according to the wise opinion of the famous \ia{Kopitar, Jernej}Kopitar, \is{Dialect}dialect grammarians represent their \is{Dialect}dialects, and from their diligent accounts the common council of \il{Slavic}Slavic philologists should reach a final judgment that is wholesome and conforms to our original language.

\subsection*{\hspace*{\fill}§. 12.\hspace*{\fill}}

The imperative mood, in the \il{Old Church Slavonic}ancient \is{Dialect}dialect, the \il{Russian}Russian \is{Dialect}dialect, and all southern \is{Dialect}dialects, always ends in -\textit{i}, and certainly sounds rather pleasant: \textit{lamaj} [‘break!’], \textit{pisaj} [‘write!’], \textit{sliшaj} [‘listen!’], which in \il{Pannonian}Pannonian is \textit{lam}, \textit{piш}, \textit{sliш} etc. It is formed simply by removing -\textit{ti} from the base form. If there is no vowel -\textit{i}, it must be added, but if it is present, then the imperative is expressed as follows: \textit{paliti} [‘to burn’], \textit{moliti} [‘to pray’], \textit{rastiti} [‘to grow’], \textit{zvoniti} [‘to ring’], \textit{stupati} [‘to step’], \textit{duti} [‘to blow’], \textit{pasti} [‘to pasture’], \textit{чuti} [‘to listen’] etc. will be \textit{pali}, \textit{moli}, \textit{rasti}, \textit{zvoni}, \textit{stupai}, \textit{dui}, \textit{pasi}, \textit{чui} etc. The \il{Pannonian}Pannonian even [119] says in the imperative: \textit{pal}, \textit{zvon}, in such a way, however, that he softens the \textit{l} and \textit{n}, letters which are usually not softened except before the vowels \textit{i}, \textit{e}. So it appears that among the \il{Pannonian}Pannonians this imperative vowel -\textit{i} in the imperative has only been swallowed by their habit of speaking quickly.

\subsection*{\hspace*{\fill}§. 13.\hspace*{\fill}}

Adjectives, as has been said, emerge from verbs in four ways, all of which the grammarians call participles. They are the present participle, the active preterit participle, the present and preterit passive participle, and finally the so-called gerund, which the \il{Pannonian}Pannonian grammarian has confused with the present participle. Other grammarians have denoted these participles with various other names because they have followed the norm of other languages, and in this way they have generated a great confusion. Let me cite the example of how the \il{Russian}Russian grammarian explains the meaning of the participles in his \il{German}German text: namely \textit{vodil}, -\textit{a}, -\textit{o} -- {\wieynk geführt} [‘led’]; \textit{vodim}, -\textit{a}, -\textit{o} -- {\wieynk geführt}; \textit{voden}, -\textit{a}, -\textit{o} -- likewise {\wieynk geführt}.\footnote{\citet[242]{puchmayer_lehrgebaude_1820}.} As far as the first is concerned, it cannot be called a participle, as it is a formal perfect, especially among the \il{Russian}Russians, \textit{ja vodil}, ‘I led’, \textit{ti vodil} [‘you led’], \textit{on vodil} [‘he led’], in the feminine \textit{ja vodila} [‘I led’] etc. \textit{Vodim}, -\textit{a}, -\textit{o} is the participle of the passive present, and means ‘I who am being led’. This way of speaking has disappeared from the other \is{Dialect}dialects, yet among the \il{Russian}Russians it prevails in elevated style, and this is the original \il{Slavic}Slavic expression. That is clear from the records of the \il{Old Church Slavonic}ancient \is{Dialect}dialect [120], such as \textit{nesom}, -\textit{a}, -\textit{o} = ‘I who am being carried’ etc. For this reason it should likewise be adopted in the other \is{Dialect}dialects, and esteemed as it were the greatest treasure in the language. The expression \textit{voden}, -\textit{a}, -\textit{o}, finally, corresponds to \il{German}German {\wieynk geführt}, but the earlier expressions do not.

Grammarians generally conflate the gerund with the present participle, yet a few distinguish them, such as the \il{Polish}Polish: the gerund \textit{xvaląc}, and present participle \textit{xvalaci}, -\textit{a}, -\textit{o} [‘praising’]. The southerners also follow this pattern. For instance, the revered \ia{Lanosović, Marijan}Lanashovich, grammarian of the \il{Slavonian}Slavonians, explains the gerund \textit{sluшajuчi} as {\wieynk zu hören} [‘to be heard’], and says that it is immutable, and that in that respect it differs from \textit{sluшajuчi}, -\textit{a}, -\textit{o} [‘listening’], the present participle.\footnote{\citet[104]{lanosovic_anleitung_1795}.} Furthermore, the \il{Polish}Pole does not explain the gerund \textit{xvaląc} as {\wieynk zu loben} [‘to be praised’], but as {\wieynk indem man lobt} [‘(in) praising’]. A great difference in meaning separates {\wieynk zu hören} and {\wieynk indem man hört}.

\subsection*{\hspace*{\fill}§. 14.\hspace*{\fill}}

The \il{Pannonian}Pannonian grammarian conflates the gerund with the present participle, as he attributes also a plural to the gerund. However, the gerund is nothing more than a formal adverb of manner, and just as adverbs are incapable of \is{Conjugation}inflection, this is also the case with the verbal adverb. The grammarian says: \textit{volajie} is a gerund, but in the plural it is \textit{volajice} [‘calling’]; however, this does not maintain the meaning of the adverb intact. For the meaning of the verbal adverb is different from the meaning of the present participle, so if I would say: \textit{sirota plaчuc po liesu bludila} [‘the orphan wandered weepingly in the forest’], here \textit{plaчuc} is an adverb; \textit{sirota plaчuca po liesu} [121] \textit{bludila} [‘the weeping orphan wandered in the forest’], here it is a present participle. The distinction of the expressions is very clear here, as the first conveys the way in which an orphan child has wandered in the forest, namely weeping, whereas the second expresses the quality of the orphan child, namely a weeping child. Now, the expression of manner, since it occurs in an adverb, remains unchanged for every gender, number, case. So according to the \il{Pannonian}Pannonian \textit{siroti plaчuc v liesu bludili} [‘the orphans wandered weeping in the forest’] should be said as: \textit{siroti plaчuce v liesu bludili}. But then \textit{plaчuce} ceases to be a gerund or adverb but is instead an adjective, since one describes the quality of weeping children etc. One must therefore distinguish a verbal adverb from a present participle, or adjective etc., because the more famous grammarians also distinguish these, as follows:

\begin{table}
\scalebox{0.99}{
\begin{tabular}{ l l }
    \lsptoprule
    \il{Russian}Russian \textit{veduч}, adverb, & \textit{veduшчii}, -\textit{aja}, -\textit{e}, adjective. \\
    \il{Polish}Polish \textit{xvaląc}, \textit{vedąc}, adverb, & \textit{xvaląci}, -\textit{a}, -\textit{o}, \textit{vedąci}, -\textit{a}, -\textit{o}, adjective. \\
    \il{Bohemian}Bohemian \textit{ƶenuc}, \textit{veduc}, adverb, & \textit{ƶenuci}, -\textit{a}, -\textit{o}, \textit{veduci}, -\textit{a}, -\textit{o}, adjective. \\
    \il{Pannonian}Pannonian \textit{milujuc}, \textit{veduc}, adverb, & \textit{milujuci}, -\textit{a}, -\textit{o}, \textit{viduci}, -\textit{a}, -\textit{o}, adjective. \\
    \il{Windic}Windic \textit{igrajoc}, \textit{vedoc}, adverb, & \textit{igrajoc}, -\textit{a}, -\textit{o}, \textit{vedoc}, -\textit{a}, -\textit{o}, adjective. \\
    \il{Illyrian}Illyrian \textit{sluшajuчi}, \textit{veduчi}, adverb, & \textit{sluшajuчi}, -\textit{a}, -\textit{o}, \textit{veduчi}, -\textit{a}, -\textit{o}, adjective. \\
    \lspbottomrule
\end{tabular}}
\end{table}

\subsection*{\hspace*{\fill}§. 15.\hspace*{\fill}}

We have seen of the perfect that if two perfect forms [122] come together, the former perfect also tends to be expressed with -\textit{v}, so, for instance, the \il{Russian}Russian: \textit{Vladimir sobrav vojsko poiшel v Xerson} [‘Vladimir, having gathered the army, left for Kherson’], instead of \textit{Vladimir sobral voisko}, \textit{i poiшel v Xerson} [‘Vladimir gathered the army and left for Kherson’]. It is also put before the future tense, namely, to indicate a future activity occurring after another activity has already finished, as the \il{Russian}Russian: \textit{Napisav pismo pogovoruju}, namely ‘after I have copied [the letter], I will speak’.\footnote{\ia{Herkel, Jan}Herkel glossed as \textit{postquam descripsero loquar}.} Some \il{Latin}Latin grammarians call this the future perfect of the subjunctive, others the future exact, but its true designation is to express the perfect function. If the aforementioned expression were treated as an adjective, it would indicate a property of an entity, because it has realized a certain function, such as: \textit{Brata svojego oplakavшu dievicu sem videl}, namely ‘I saw a maiden who mourned her brother’; here \textit{oplakavшu} is a formal adjective, namely: \textit{oplakavшi}, -\textit{a}, -\textit{o}.

\subsection*{\hspace*{\fill}§. 16.\hspace*{\fill}}

The Slavs also have a present participle of the passive, an adjective designating a certain entity which is brought about under the performance of another entity; this type of adjective was known in the \il{Old Church Slavonic}ancient \is{Dialect}dialect, and to the \il{Russian}Russians in the more sublime style. Its typical feature is the letter \textit{m}, as in:

\begin{longtable}{ l l l l l }
    \lsptoprule
    \textit{pisati} [‘to write’] & will be & \textit{pisam}, -\textit{a}, -\textit{o},\footnote{\ia{Herkel, Jan}Herkel does not offer the \il{Russian}Russian form here, which has no contraction (\textit{pisaem}).} & or & \textit{pisami}, -\textit{a}, -\textit{o}. \\
    \textit{nositi} [‘to wear’] & \multicolumn{1}{l}{—} & \textit{nosam}, -\textit{a}, -\textit{o}, & &	\textit{nosami}, -\textit{a}, -\textit{o}. \\
    \textit{voditi} [‘to lead’] & \multicolumn{1}{l}{—} & \textit{vodim}, -\textit{a}, -\textit{o},	& & \textit{vodimi}, -\textit{a}, -\textit{o}. \\
    \textit{znati} [‘to know’] & \multicolumn{1}{l}{—} & \textit{znam}, -\textit{a}, -\textit{o}, & & \textit{znami}, -\textit{a}, -\textit{o}. \\
    \textit{vjedeti} [‘to know’]\footnote{The verb \textit{znati} implies ‘know, be familiar with’, as with \il{German}German \textit{kennen}, while \textit{vjedeti} implies ‘know a fact’, as with \il{German}German \textit{wissen}.} & \multicolumn{1}{l}{—} & \textit{vjedom}, -\textit{a}, -\textit{o}, & & \textit{vjedomi}, -\textit{a}, -\textit{o}. [123] \\
    \lspbottomrule
\end{longtable}

Or before \textit{m} with \textit{e} put in front of it, such as \textit{pisaem}, \textit{nosaem}, \textit{vidiem} etc.

\subsection*{\hspace*{\fill}§. 17.\hspace*{\fill}}

But the participle of the passive preterit, or the adjective that indicates that an entity has already been constituted under the function of another entity, and ends in -\textit{n}, such as \textit{sliшan} [‘heard’], \textit{pisan} [‘written’], \textit{nosen} [‘worn’], \textit{voden} [‘led’], \textit{znan} [‘known’], \textit{viden} [‘seen’], -\textit{a}, -\textit{o} etc. It is clear that this participle is a formal adjective from the fact that it also receives the expression of comparison, such as \textit{uчen} [‘learned’], \textit{uчeneiшi} or \textit{uчeneji} [‘more learned’] etc. So then the question arises whether just like formal adjectives also the other participles can receive the expression of comparison, or not. I am so far from doing violence to the \is{Genius}genius of this language that I would rather try to disclose the original genius of the same language more fully. I do not see any reason why the other verbal adjectives, too, would not permit a comparative form, so long as the thing itself allows comparison, as is used in Pannonia: \textit{znamшi}, or \textit{znamejшi чelovek}, or ‘more renowned person’; \textit{vedomejшe hrjexi} or \textit{vedomejшi griexi} [‘more conscious sins’] etc. And it is clear from these examples that the present passive participle, even in usage itself, allows the expression of comparison. But what about active adjectives? On this topic one should seek the opinion of more learned men, but in my humble opinion, I would not deny them the expression of the comparative either, if the nature of the thing itself permits the adjectives to form a comparative, such as: \textit{verici} [‘believing’], \textit{vericejшi} or \textit{vericeji} [‘more believing’], \textit{milujici} [‘loving’], \textit{milujicejшi} [‘more loving’] [124] etc. And as far as the adjective of the perfect is concerned, one should believe the same; usage itself clearly indicates it, such as \textit{verivшi}, \textit{verivejшi чelovek}, ‘person who has believed more’. In Pannonia it is as follows: \textit{ƶivшi}, \textit{byvшi чelovek}, ‘person who has previously lived, or ‘has previously been’. In this usage, “he who has lived earlier” is clearly a comparison etc. In this way, \textit{uctivejшi}, \textit{lenivejшi}, \textit{horlivejшi чlovek} [‘a person who has previously been more respectful, lazy, ardent’] are comparatives derived from \textit{uctiv}, \textit{leniv}, \textit{horliv} etc.

\subsection*{\hspace*{\fill}§. 18.\hspace*{\fill}}

The passive preterit adjective also ends with the letter -\textit{t}, so the letter -\textit{n} cannot be its universal characteristic. One reads in the \il{Old Church Slavonic}ancient \is{Dialect}dialect \textit{jat} [‘taken’], \textit{naчat} [‘begun’], \textit{ƶat} [‘harvested’], \textit{prokliat} [‘cursed’], for instance: \textit{pojati vojak} [‘a captured soldier’], \textit{zeƶate ƶito} [‘harvested rye’], \textit{naчato vino} [‘opened wine’], \textit{pokliato mesto} [‘cursed place’] etc. But all these are expressed in the other \is{Dialect}dialects even more pleasantly with -\textit{n}, such as: \textit{pojani}, \textit{bijani}, \textit{pijani vojak}, \textit{zeƶano ƶito}, \textit{naчano vino}, \textit{pokliano mesto}. For \is{Dialect}dialects vary; they have sometimes adopted one ending, sometimes the other. Some \is{Dialect}dialects are accustomed to both, such as \textit{vikluvato kurja} or \textit{vikluvano kurja} [‘newly hatched chicken’], \textit{zemleno ƶito} [‘milled rye’], \textit{podpreti or podpreni dom} [‘supported house’], \textit{zeƶrata or zeƶrana ovca} [‘harvested fruit’], \textit{zapnuta}, \textit{zapnuna шata} [‘buttoned clothing’] etc. Here it would certainly be superfluous to specify as troublesome exceptions that some end in -\textit{t}, but others in -\textit{n}. For in one \is{Dialect}dialect we have a verb ending in -\textit{t}, but that same verb in another \is{Dialect}dialect with a certain pleasantness takes an -\textit{n} such as: [125] \textit{umreti чelovek}, elsewhere \textit{umreni чelovek} [‘dead person’], \textit{naduti miex}, elsewhere more pleasantly \textit{naduni miex} [‘inflated bellows’ ] etc.

\subsection*{\hspace*{\fill}§. 19.\hspace*{\fill}}

Let us now consider the entire form of \is{Conjugation}verb inflections in moods, tenses, persons, and numbers. There are three moods: indicative, imperative, optative; and the infinitive is the primary form for the other moods.

\subsubsection*{\textit{Indicative}.}

The threefold arrangement of time is expressed by the indicative mood, name\-ly: present, past, and future, since the \is{Genius!Genius of the Slavic language}genius of the \il{Slavic}Slavic language rejects any further division of time. The \il{Old Church Slavonic}Old Church \is{Dialect}dialect expresses the present in the following form:

Singular: 1\textsuperscript{st} person \textit{nesu} [‘carry, bring’]. 2\textsuperscript{nd} \textit{neseшi}. 3\textsuperscript{rd} \textit{neset}. Plural: 1\textsuperscript{st} \textit{nesem}. 2\textsuperscript{nd} \textit{nesete}. 3\textsuperscript{rd} \textit{nesut}. \il{Russian}Russian \textit{nesu}, \textit{neseш}, \textit{neset}, \textit{nesem}, \textit{nesete}, \textit{nesut}. The dual number is not shown, since no \is{Dialect!Living dialect}living dialect uses it except for \il{Windic}Windic.

\il{Bohemian}Bohemian: \textit{nesu}, \textit{neseш}, \textit{nese}, plural \textit{neseme}, \textit{nesete}, \textit{nesau}.

The extent of the difference between these three \is{Dialect}dialects is clear from comparing the \is{Conjugation}inflection. \il{Russian}Russian, following the \il{Old Church Slavonic}Church \is{Dialect}dialect, pronounces the first person of the plural with -\textit{m}, but in the most \il{Old Church Slavonic}ancient \is{Dialect}dialect the first person was also pronounced with -\textit{i}. There are still clear traces of this fact in the Bible; for instance, \textit{budemi} [‘we will’], \textit{poƶivemi} [‘we live’] can be read in the same [126] Bible, and for this reason the -\textit{m} ending of the first person plural is not universal even among the \il{Russian}Russians themselves, since it is also said with -\textit{i}, or with -\textit{o}, such as \textit{nesemi}, or \textit{nesemo}. The \il{Polish}Pole says \textit{niesię}, \textit{niesies}, \textit{niesie}, \textit{niesiemi}, \textit{niesiete} (\textit{niesiećie}), \textit{niesią}. The \il{Polish}Polish \is{Dialect}dialect counts among those that pronounce the first person singular with -\textit{m}, a letter which they elide before \textit{e} in writing, and they mark that \textit{e} with a \il{French}French cedilla, and if an \textit{a} precedes it, then they retain the letter -\textit{m}, such as: \textit{xvalam} [‘I praise’], \textit{sekam} [‘I chop’] etc. It otherwise agrees with the \il{Pannonian}Pannonian \is{Dialect}dialects, except that the \il{Polish}Polish \is{Dialect}dialect is softer, so it puts vowels in between consonants, changes the letter \textit{ш} into \textit{s}, \textit{ч} into \textit{c}, \textit{h} into \textit{g}, and \textit{t} before \textit{i} into \textit{ć}.

The \il{Pannonian}Pannonian is: \textit{niesem}, \textit{nieseш}, \textit{niese}, \textit{niesieme}, \textit{niesiete}, \textit{niesu}, or \textit{nesem}, \textit{neseш}, \textit{nese}, \textit{neseme}, \textit{nesete}, \textit{nesu}. All southerners also agree in the singular with the \il{Pannonian}Pannonian \is{Dialect}dialect, but in the plural they pronounce the first and third person with -\textit{o}, such as: \textit{nesiem}, \textit{nesies}, \textit{nesie}, \textit{nesiemo}, \textit{nesiete}, \textit{nesio}. Hence it follows that a double form of the present can be established, grounded in antiquity and the \is{Dialect!Living dialect}living dialects, namely the northern and the southern.

The first form: \textit{nesu}, \textit{neseш}, \textit{neset}, \textit{nesemi}, \textit{nesete}, \textit{nesut}.

The second form: \textit{nesiem}, \textit{nesieш}, \textit{nesie}, \textit{nesiemo}, \textit{nesiete}, \textit{nesio}.

In the first form, accordingly, the first plural is pronounced with -\textit{i}, because this ending is grounded in the most \il{Old Church Slavonic}ancient \is{Dialect}dialect, as is clear from the Bible. Also, the \il{Russian}Russians speak like that part of the time, and the \il{Polish}Poles always. Likewise, pleasant singing requires [127] an ending with a vowel. Finally, if it were pronounced with -\textit{m}, it would get confused with the second form of the first person singular. Hence, everything, of whatever \is{Dialect}dialectal \is{Conjugation}conjugation the form may be, can pleasingly follow this form, despite the anxious investigations of some grammarians into whether this or that takes the first, second, third, fourth, fifth or sixth form, since the \is{Genius!Genius of the Slavic language}genius of the \il{Slavic}Slavic language follows only one unique form of \is{Conjugation}inflecting, in the south: \textit{liubim} [‘I love’], \textit{liubiш}, \textit{liubi}, \textit{liubimo}, \textit{liubite}, \textit{liubio}; \textit{volam} [‘I call’], \textit{volaш}, \textit{vola}, \textit{volamo}, \textit{volate}, \textit{volajo}; \textit{maƶem} [‘I smear, spread’], \textit{maƶeш}, \textit{maƶe}, \textit{maƶemo}, \textit{maƶete}, \textit{maƶeio} or -\textit{jo}; or in the north \textit{maƶu}, \textit{maƶeш}, \textit{maƶet}, \textit{maƶemi}, \textit{maƶete}, \textit{maƶut} etc.

\subsection*{\hspace*{\fill}§. 20.\hspace*{\fill}}

It has been argued that the second form of the present, namely the southern one, should not be adopted because it is \il{Italian}Italian rather than \il{Slavic}Slavic, since the \il{Italian}Italian says: \textit{sentiámo} [‘we feel’], \textit{sentíte} [‘you feel’], \textit{sentino} [‘they feel’] etc. I respond: the fact that the southern form agrees entirely with \il{Italian}Italian does not mean that the \il{Slavic}Slavic form has an \il{Italian}Italian origin. For \il{Italian}Italian is a daughter of the \il{Latin}Latin, and when \il{Latin}Latin was still flourishing, and very much alive, very numerous Slavic tribes lived in both the eastern and the western empire. Indeed, Slavs lived even in the core parts of the empires themselves, which is also why men descending from the Slavic race ascended to the thrones of both empires, as confirmed by Justinian’s mother \textit{Bidlenica}, and his father \textit{Urpravda}. Furthermore, [128] the genuineness of the southerners’ \is{Conjugation}conjugations emerges from the fact that the \il{Bohemian}Bohemians pronounce the third person of the plural with -\textit{o}, even though they write it with -\textit{au}; as such, \textit{milujau} [‘they love’], \textit{majau} [‘they have’] is pronounced \textit{milujo}, \textit{majo}. Finally, the \il{Russian}Russians themselves, very remote from the \il{Italian}Italians, pronounce the first person most often with -\textit{o}, such as \textit{povidamo} [‘we see’], \textit{pisamo} [‘we write’], although \il{Russian}Russian grammarians do not even discuss it. The \il{Pannonian}Pannonians themselves also talk like that.

\subsection*{\hspace*{\fill}§. 21.\hspace*{\fill}}

We find, as noted above, the active expression of the past tense in the \il{Old Church Slavonic}ancient \is{Dialect}dialect takes -\textit{x}, but also -\textit{l}, with or without the auxiliary \textit{jesm} [‘I am’], e.g., \textit{чitax} [‘read’], \textit{liubix} [‘loved’], or \textit{чital}, \textit{liubil} or \textit{чital}, \textit{liubil jesm}, \textit{jesi}, \textit{jest} [‘I, you, he/she/it read, loved’]. In very many \is{Dialect}dialects, the auxiliary verb is abbreviated to \textit{sem}, \textit{si}, \textit{je}, in the plural \textit{jesmi}, \textit{jeste}, \textit{jesut} is abbreviated as \textit{sme}, (\textit{smi}), \textit{ste}, \textit{su}, or \textit{smo}, \textit{ste}, \textit{so}.

The \is{Dialect!Living dialect}living dialects do not use the -\textit{x} ending except for the \il{Slavonian}Slavonians, yet with -\textit{x} changed into -\textit{h}, such as \textit{liubih}, \textit{uчih} [‘loved, read’] instead of \textit{uчix} etc. but they also use the ending in -\textit{l}, but in such a way that, according to their \is{Dialect}dialectal custom, they pronounce the masculine singular -\textit{l} as -\textit{o}, such as \textit{jesam uчio}, \textit{uчila}, \textit{uчilo} [‘I (m./f./n.) learned’] etc. But because grammarians see different forms of the perfect in other languages, they have also tried to find them in their own \il{Slavic}Slavic \is{Dialect}dialects. The perfect, imperfect, and composite perfect draw their origin from this.

The \il{Russian}Russian expresses the perfect without an auxiliary verb, e.g. [129]

\begin{longtable}{ l l l l l l }
    \lsptoprule
    \textit{Ja}, \textit{ti}, \textit{on} etc. & \textit{uчil}, \textit{uчila}, -\textit{o}. & \multicolumn{2}{ c }{\textit{mi}, \textit{vi}, \textit{oni}} & \textit{uчili}. &	 	[‘taught; learned’] \\
    & \textit{dvigal}, -\textit{a}, -\textit{o}. & \multicolumn{1}{ l }{—} & \multicolumn{1}{ r }{—} & \textit{dvigali}. & [‘moved’] \\
    & \textit{vodil}, \textit{a}, \textit{o} etc. & \multicolumn{1}{ l }{—} & \multicolumn{1}{ r }{—} & \textit{vodili}. & [‘led’] \\
    \lspbottomrule
\end{longtable}

The \il{Polish}Pole expresses it as follows, bound with an auxiliary verb:

\begin{longtable}{ l l l l l }
    \lsptoprule
    & \multicolumn{3}{ c }{Singular.} & \multicolumn{1}{ c }{Plural.} \\
    \midrule
    & \multicolumn{1}{ c }{\textit{m}.} & \multicolumn{1}{ c }{\textit{f}.} & \multicolumn{1}{ c }{\textit{n}.} & \\
    1. & \textit{xvalilem} [‘praised’], & \textit{xvalilam}, & \textit{xvalilom}, & \textit{xvalilismi}. \\
    2. & \textit{xvaliles}, & \textit{xvalilas}, & \textit{xvalilos}, &	\textit{xvalilisće}. \\
    3. & \textit{xvalil}, &	\textit{xvalila}, &	\textit{xvalilo}, & \textit{xvalili}. \\
    \lspbottomrule
\end{longtable}

This is nothing else than \textit{xvalil}, -\textit{a}, -\textit{o}, appended with an abbreviated form of the auxiliary \textit{sem}, such as in the first and second persons, \textit{xvalil} (-\textit{em}), \textit{xvalila} (-\textit{m}), \textit{xvalilo} (-\textit{m}), \textit{xvalil} (-\textit{es}), where \textit{es} is a contraction from \textit{jesi}, \textit{xvalili} (\textit{smi}), \textit{xvalili} (\textit{sćie}) uses \textit{stie} instead of \textit{jeste}, \textit{ste}; the \textit{t} before a vowel is here softened as in other \is{Dialect}dialects, but the \il{Polish}Pole has changed it into \textit{ć}. The \il{Bohemian}Bohemian and the \il{Pannonian}Pannonian follow this way of \is{Conjugation}inflecting, yet they do not \is{Conjugation}conjugate the auxiliary verb in writing, even though they pronounce it in speech as one word, such as \textit{xvalilsem}, \textit{xvalilasem}, \textit{xvalilo sem}, \textit{si}, \textit{xvalilisme}, \textit{xvaliliste}, \textit{xvalili}. The \il{Bohemian}Bohemian additionally pronounces the neuter plural with -\textit{a}, which also happens among the \il{Serbian}Serbs and \il{Windic}Winds, such as \textit{hvalili}, -\textit{e}, -\textit{a}, \textit{smo}, \textit{ste}, \textit{so}. The \il{Windic}Wind says the following: \textit{domiska gorela so} [‘the little houses burned’]; the \il{Bohemian}Bohemian says \textit{domiska hořeli} or \textit{hořela} (\textit{sau}), the \il{Pannonian}Pannonian \textit{domiska horeli}, the \il{Polish}Pole, \textit{domiska goreli} (\textit{gorzeli}), the \il{Russian}Russian: \textit{domiska goreli}. [130]

\subsection*{\hspace*{\fill}§. 22.\hspace*{\fill}}

All southerners, both the \il{Windic}Winds and the other \il{Illyrian}Illyrians, express the perfect with -\textit{l} and the auxiliary verb:

\begin{longtable}{ l l l l l l l }
    \lsptoprule
    \multicolumn{7}{ c }{Singular.} \\
    \midrule
    \textit{m}. & \textit{jedel}, & \textit{vrel}, & \textit{igrol [\textit{sic}]}, & \textit{delal}, & \multirow{3}{*}{\begin{Huge} \} \end{Huge}} & \multirow{3}{2cm}{\textit{sem}, \textit{si je}, \textit{igral sem}, \textit{igrala si} etc.} \\
    \textit{f}. & \textit{jedla}, & \textit{vrela}, & \textit{igrala}, & \textit{delala}, & \\		
    \textit{n}. & \textit{jedlo}, & \textit{vrelo}, & \textit{igralo}, & \textit{delalo}, \\
    \lspbottomrule
    \\
    \lsptoprule
    \multicolumn{7}{ c }{Plural.} \\
    \midrule
    & & & & & \multirow{5}{*}{\begin{Huge} \} \end{Huge}} & \multirow{5}{3cm}{\textit{smo}, \textit{ste}, \textit{so}, \textit{igrali smo}, \textit{igrale smo}, feminine \textit{igrale ste}, \textit{igrale so}, which is likewise feminine.} \\
    \textit{m}. & \textit{jedli}, & \textit{vreli},	& \textit{igrali}, & \textit{delali},	& & \\
    \textit{f}. & \textit{jedle}, & \textit{vrele}, & \textit{igrale}, & \textit{delale},	\\
    \textit{n}. & \textit{jedle} (\textit{a}), & \textit{vrela}, & \textit{igrala}, & \textit{delala}, \\	
    & & & & & & \\
    \lspbottomrule
\end{longtable}

Since the typical feature of the perfect is -\textit{l}, it thus follows that the perfect can also be expressed without the auxiliary in the \il{Russian}Russian fashion, or with the auxiliary in the fashion of the other \is{Dialect}dialects, some of which sometimes insert the auxiliary, and other omit it, such as the \il{Pannonian}Pannonian: \textit{Ja videl mojego otca}, or \textit{ja sem videl mojego otca} [‘I saw my father’] etc.

\subsection*{\hspace*{\fill}§. 23.\hspace*{\fill}}

The most famous grammarian of the \il{Serbian}Serbs explains the expression of time passed in a triple way, namely:\footnote{\citet[lix]{karadzic_srpski_1818}.}

\begin{table}
\scalebox{0.89}{
\begin{tabular}{ l l l l l l l l l l l }
    \lsptoprule
    \multicolumn{11}{ c }{\textit{Imperfect}} \\
    \multicolumn{6}{ l }{1\textsuperscript{st} \textit{igra} [‘I played’], 2\textsuperscript{nd} \textit{igraшe}, 3\textsuperscript{rd} \textit{igraшe}} & \multicolumn{5}{ l }{Plural 1 \textit{igrasmo}, 2 \textit{igraste}, 3 \textit{igrau}. [131]} \\
    \\
    \multicolumn{11}{ c }{\textit{Simple perfect}} \\
    \multicolumn{6}{ l }{1\textsuperscript{st} \textit{igra}, 2\textsuperscript{nd} \textit{igra}, 3\textsuperscript{rd} \textit{igra}} & \multicolumn{5}{ l }{Plural 1 \textit{igrasmo}, 2 \textit{igraste}, 3 \textit{igraшe}.} \\
    \\
    \multicolumn{11}{ c }{\textit{Compositive perfect}} \\
    \multicolumn{6}{ c }{Singular.} & \multicolumn{5}{ c }{Plural.} \\
    \textit{m}. & \textit{igrao}, & \multirow{3}{*}{\begin{Huge} \} \end{Huge}} & & & & \textit{igrali}, & \multirow{3}{*}{\begin{Huge} \} \end{Huge}} & \\
    \textit{f}. & \textit{igrala}, & & \multicolumn{3}{ c }{{\rotatebox[origin=c]{90}{\textit{sem},}} {\rotatebox[origin=c]{90}{\textit{si},}} {\rotatebox[origin=c]{90}{\textit{je}}}} & \textit{igrale}, & & \multicolumn{3}{ c }{{\rotatebox[origin=c]{90}{\textit{smo},}} {\rotatebox[origin=c]{90}{\textit{ste},}} {\rotatebox[origin=c]{90}{\textit{su}}}} \\
    \textit{n}. & \textit{igralo}, & & & & & \textit{igrala}, \\
    \lspbottomrule
\end{tabular}}
\end{table}

The imperfect shown here, and the simple perfect, is nothing more than the remains of the perfect tense of the \il{Old Church Slavonic}ancient \is{Dialect}dialect, and indeed, in the \il{Old Church Slavonic}ancient \is{Dialect}dialect one said both \textit{igral} and \textit{igrax}, the final letter of which, -\textit{x}, is still preserved among the \il{Slavonian}Slavonians as -\textit{h}, \textit{igrah}, a letter which the \il{Serbian}Serbs have almost entirely banished from common speech. Hence, the abovementioned way of writing of the \il{Serbian}Serbs cannot serve as the norm, but the composite perfect agrees with the other \is{Dialect}dialects, which is the simple perfect among the \il{Russian}Russians, who say it without the auxiliary \textit{jesm} [‘I am’], as follows: \textit{ja}, \textit{ti}, (\textit{on}, \textit{a}, \textit{o}) \textit{igral}, -\textit{a}, -\textit{lo} etc. The Serb \is{Conjugation}inflects it in the plural as an adjective, \textit{igrali}, -\textit{e}, -\textit{a}, and rightly so. For if one would follow the gender in the singular in all \is{Dialect}dialects, why would one not also follow it in the plural, following the example of the \il{Serbian}Serbs, \il{Windic}Winds, and \il{Bohemian}Bohemians? For this reason, the perfect can be expressed with or without the auxiliary verb, taking gender into account in both numbers at the same time. [132]

\subsection*{\hspace*{\fill}§. 24.\hspace*{\fill}}

In some \is{Dialect}dialect grammars, discussions about expressing the future are lengthy, but all those discussions agree on the following point: that original, frequentative, or double frequentative verbs are expressed in the future by means of \textit{budem} (\textit{budu}), but the verbs composed out of original ones mark the future with a form of the present that is both original and already formed. For instance, the future of \textit{kopati} [‘to dig’] in the original is \textit{budem kopati} or \textit{budu kopati}, in the frequentative \textit{budu kopavati}, or \textit{budem kopavati}, and in the double frequentative \textit{budu kopavavati} is the future. From this, the momentaneous verb, formed by interjecting \textit{ni}, will be \textit{kopnu}, or \textit{kopnem} [‘I will dig’].

Now verbs that are originally momentaneous, \textit{kupiti} [‘to buy’], \textit{streliti} [‘to shoot’], \textit{dati} [‘to give’] etc. are expressed in the future as \textit{kupim}, \textit{strelim}, \textit{dam}, verbs which should be distinguished from \textit{kupivati}, \textit{strelavati} (abbreviated as \textit{strelati}), \textit{davati}, as they are already frequentatives, and for this reason they are expressed in the future by means of \textit{budu}. For it often happens that in \is{Dialect}dialectal use such verbs are substituted for each other; a substitution, however, which the \is{Genius}genius of the language, taken in the strict sense, cannot allow. So it happens that from the verbs \textit{duti} [‘to blow’], \textit{ƶuti} [‘to chew’], \textit{kuti} [‘to forge’] grammarians derive not the perfect forms \textit{dul}, \textit{ƶul}, \textit{kul}, but \textit{duval}, \textit{ƶuval}, \textit{kuval}, which are actually the perfects of verbs \textit{duvati}, \textit{ƶuvati}, \textit{kuvati} etc. Let us now see their use in various \is{Dialect}dialects, for instance: [133] 

\begin{footnotesize}
\begin{longtable}{ l l }
    \multicolumn{2}{ c }{\normalsize \il{Russian}\textit{Russian}.\footnote{Excerpt from \textit{Zpěw Ruský, 5.}, \citet[90--91]{celakovsky_slowanske_1825}.}} \\
    \\
    Matuшha! [\textit{sic}] na dvor gosti jedut. & [Mommy! Guests are coming to the yard, \\
    Sudarynja, na dvor gosti jedut. & Madam, guests are coming to the yard. \\
    Ditjatko! ne boj sia, ne vidam, & Little child! Fear not, I won’t give you away \\
    Svet miloje moje! ne bojsia ne vidam. & My dear light, fear not, I won’t give you away. \\
    Matuшka! na kryleчko (\textit{a}) gosti idut & Mommy! Guests are coming onto the porch, \\
    Sudarynja! na kryleчko gosti idut, & Madam! Guests are coming onto the porch, \\
    Ditjatko! nebojsia etc. & Little child! Fear not etc. \\
    Matuшka! v novu gornicu (\textit{b}) idut, & Mommy! They are coming into the new room, \\
    Sudarynja! v novu gornicu idut, & Madam! They are coming into the new room, \\
    Ditjatko etc. & Little child etc. \\
    Matuшka, za dubovoj stol sadjat sia, & Mommy, they are sitting on the oak table,  \\
    Sudarynja! za dubovoj stol sadjat sia & Madam, they are sitting on the oak table, \\
    Ditjatko! etc. & Little child! etc. \\
    Matuшka! obraz zo stieni snimajut, & Mommy! They are removing a picture from \\
    & \multicolumn{1}{ r }{the wall,} \\
    Sudarynja! zo steni snimajut, & Madam, from the wall, \\
    Ditjatko etc. & Little child etc. \\
    Matuшka menja blago slovjajut, & Matushka, they are wishing me well, \\
    Sudarinja, menja blago slovjajut, & Madam, they are wishing me well, \\
    Ditjatko! gospod Bog s toboju! & Little child! Lord God be with you! \\
    Svet miloje moje! gospod Bog stoboju & My light, my dear, Lord God be with you.] \\
\end{longtable}
\end{footnotesize}
(\textit{a}) \textit{sxodki} [‘little step; porch’]. (\textit{b}) an upper-floor guestroom.

\enlargethispage{\baselineskip}

\begin{small}
\begin{longtable}{ l l }
    \multicolumn{2}{ c }{\normalsize \il{Little Russian}\textit{Little Russian song}.\footnote{Excerpts from \textit{Maloruskijä dumki}, \citet[112, 114]{celakovsky_slowanske_1825}; for modernized \il{Ukrainian}Ukrainian versions, see \citet[80]{vlast_zhemchuhy_1923}; \textit{пiсня про милу}, \citet[85]{rudnyckyj_readings_1958}.}} \\
    \\
    Sivji konju, sivji konju! & [Grey horse, grey horse! \\
    \hspace*{0.5cm}Teƶko na tie bude, & \hspace*{0.5cm}It will be hard for you, \\
    Poidemo raƶom s vetrom & We will go along with the Wind \\
    \hspace*{0.5cm}Popasu nebude. & \hspace*{0.5cm}There will be no pasture. \\
    Bihaj konju rikal konju & Run horse, I said to the horse, \\
    \hspace*{0.5cm}Bo sie veчerije;\footnote{\ia{Herkel, Jan}Herkel drew this poem from \citet[114]{celakovsky_slowanske_1825}, who actually printed this couplet as \textit{Bihaj konju, bihaj konju / Bo vže vežerije}.} [134] & \hspace*{0.5cm}For the evening falls, \\
    Oj tam sedit moja mila, & Ah, my love sits there \\
    \hspace*{0.5cm}Kde z lisa zorije. (\textit{c}) & \hspace*{0.5cm}In the forest twilight \\
    Vidzu milu, vidzu lubku, & I see my dear, I see my love, \\
    \hspace*{0.5cm}Divit sje v okence, & \hspace*{0.5cm}Looking out the window \\
    Xoti, (\textit{d}) jak temno, xoti nevidno. & Though it is dark, though she \\
    & \hspace*{2.302cm}can’t be seen, \\
    \hspace*{0.5cm}Svitit sje jak sonce. & \hspace*{0.5cm}She shines like the sun \\
    Stanuv konik, stav sivenkij, & The little grey horse arrived, \\
    \hspace*{0.5cm}U miloji xaty (\textit{e}) & \hspace*{0.5cm}At my sweetheart’s cottage, \\
    Tu ja xoчu (\textit{f}) zavшe ƶiti. & I want to live here forever, \\
    \hspace*{0.5cm}Tu xoчu umerati. & \hspace*{0.5cm}Here I want to die.] \\
\end{longtable}
\end{small}
(\textit{c}) \textit{les}, \textit{lis} ‘forest’; \textit{zoriti} ‘to dawn’, hence ‘twilight’, \textit{zori}. (\textit{d}) \textit{xoti}, ‘although’, frequently used by \il{Pannonian}Pannonians as well. (\textit{e}) \textit{xati}, ‘cottage’ (\textit{f}) \textit{xoчu}, \textit{xoшчu}, ‘I want’.

\enlargethispage{\baselineskip}

\begin{footnotesize}
\begin{longtable}{ l l }
    \multicolumn{2}{ c }{\normalsize\il{Polish}\textit{Polish song}.} \\
    \multicolumn{2}{ c }{\normalsize\il{Polish}\textit{Polish tale}.\footnote{Excerpts from \textit{XIV, Molodec, duma z ruskiego}, \citet[201--202]{niemcewicz_bayki_1820}; cf. \citet[150--151]{celakovsky_slowanske_1822}. The quote bears both a \il{Latin}Latin and a \il{Polish}Polish title, in that order, hence the double title in our translation. \ia{Herkel, Jan}Herkel transcribed this poem from \ia{Čelakovský, František}Čelakovský, who in turn transcribed from \ia{Niemcewicz, Julian Ursyn}Niemcewicz. Niemcewicz’s original contains the word \textit{woyłok} ‘felt’. \ia{Čelakovský, František}Čelakovský’s version was printed with with several \il{Polish}Polish letters, including <ą, ć, ę, ń, ó, ś, ź>, but \ia{Čelakovský, František}Čelakovský replaced \il{Polish}Polish <ż> with \il{Bohemian}Bohemian <ž>, and \il{Polish}Polish <ł> with <l>. \ia{Čelakovský, František}Čelakovský thus printed \ia{Niemcewicz, Julian Ursyn}Niemcewicz’s \textit{woyłok} as \textit{woylok}, and in his \il{German}German translation glossed the word as \textit{seid’ner Teppich}, meaning ‘silk rug’. Since \ia{Herkel, Jan}Herkel wrote \textit{koniu moj} ‘my horse’ and \textit{vojne} ‘war’ where both \ia{Niemcewicz, Julian Ursyn}Niemcewicz and \ia{Čelakovský, František}Čelakovský wrote \textit{koniu móy} and \textit{woynę}, it seems \ia{Herkel, Jan}Herkel ought to have transcribed \ia{Čelakovský, František}Čelakovský’s \textit{woylok} as \textit{vojlok}. \ia{Herkel, Jan}Herkel’s \textit{vojak} ‘soldier’ thus appears a misprint.}} \\
    \\
    Juƶ mgla na morskiei opadla provodzi & [Already the fog on the sea has cleared, \\
    Juƶ ƶalosć padla v serce vojovnika. & already sorrow has fallen on the warrior’s heart \\
    Ze sinego mora, mgla sara nie sxodzi, & From the blue sea, the grey fog does not lift \\
    Ni ƶalosć ze serca molodca nieznika. & Sorrow persists in the young man’s heart. \\
    Nie jest to gviazda, co blisći na (luce) & It is not a star which shines on (the meadow), \\
    Lec jakies sviatlo zvodne latające. & But some kind of floating light is flying. \\
    Tam niedaleko byl vojak rozvarti, & Not far from there was a piece of felt spread out, \\
    Na nim leƶi molodec na ruku (ręku) oparti, & On it lies a young fellow leaning on his arm, \\
    Smiertelnu ranu v piersiax odniesionu, & The mortal wound he suffered in his breast, \\
    Prićiska xustku skrvavionu, & He staunches with a bloody handkerchief, \\
    Pri nim kon stoi dzielne Donu plemie, & Next to him stands a horse, the brave race of \\
    & \hspace*{3.85cm}the Don, bravely \\
    Ostrem kopitem mokru zriva zemiu & With his sharp hoof he tears up the wet earth, \\
    I tak, (jak poviesć niesie) [135] & And thus (the story continues) \\
    Cili sam movi, ci movic zdaje sie; & Either he himself speaks, or he seems to speak: \\
    Vstan moj molodcu, vstan moj dobri panie! & “Stand up, my youth; stand up, my good sir! \\
    Nie daj sie dluzej brocic tvojej ranie, & Don’t let your wound bleed any longer, \\
    Xvić sie za grivu moju dlugu. & Grab onto my long mane, \\
    Usiadz na viernego slugu, & Sit on your loyal servant, \\
    On cie zaniesie v dom tvoj ulubioni. & He will take you to your beloved home, \\
    Do tvej matki, do dzieci, do tvej peknej ƶoni. & To your mother, to your children, to your dear \\
    & \hspace*{4.7cm}wife.” \\
    Na ta slova molodec lzami tvar svu rosi. etc. & At these words the youth’s face becomes wet \\ 
    & \hspace*{3.128cm}with tears etc. \\
    Tak do konia premavia! O koniu moj dobri! & He speaks to the horse: “Oh my good horse, \\
    Nigdi nezmordovani, i ruci i xrobri, & Tireless, swift and brave, \\
    Cos v sluƶbie Cara mego, & Who went to war with me \\
    \hspace*{0.5cm}sedl (шel) so mnu na vojnu, & \hspace*{0.5cm}in the service of my tsar, \\
    I tam lotniejsi od vixrov salonix, & And there, faster than the wild winds, \\
    V posrod ognistix gradov vipusconix, & Dashing amidst the flaming hailstones, \\
    Juƶ jes mi niepotrebni, tak kazal los slepi, & I need you no longer, so blind fate has told me, \\
    Jdz v dalekie Donu a stepi, & Go to the distant Don steppes, \\
    Poviedz mej mlodej vdovie, & Tell my young widow, \\
    \hspace*{0.5cm} ƶem (ƶe sem) me slubi zmienil, & \hspace*{0.5cm} That I changed my vows, \\
    Ƶem (ƶe sem) sie juƶ s jinu oƶenil, & That I have already taken another \\
    Vziol s niju v posagu kilka piondzi ziemi, & I took a few acres of land from her as a dowry, \\
    Ze ostra sabla juƶ nas polucila, & The sharp saber brought us together, \\
    A gorejaca strela spac nas polozila. & But a flaming arrow has put us to sleep.] \\
\end{longtable}
\end{footnotesize}

This \il{Polish}Polish text is preserved intact, with the exception of \textit{z} after the letter \textit{r}, so \textit{potrebni} is written instead of \textit{potrzebni}; in some places \textit{ą}, \textit{ę} is omitted and written as it is pronounced, so \textit{vziol} instead of \textit{vziąl}, \textit{c} instead of \textit{t} is retained everywhere, but the entire text can nonetheless be understood very easily. The \il{Polish}Pole says the following: \textit{spać nas poloƶila} instead of \textit{spati nas poloƶila}. [136]

\begin{longtable}{ l l }
    \multicolumn{2}{ c }{\normalsize \textit{\il{Serbian}Serbian song}.\footnote{\textit{317. Жеља обога}, \citet[218]{karadzic_narodne_1824}; probably via \textit{11}, \citet[134]{celakovsky_slowanske_1825}.}} \\
    \\
    Devojчica voda gazi (\textit{g}) & [A girl wades through the water \\
    Nogi joj se beljo. & Her legs become white \\
    Za niom ide mlado momчe, (\textit{h}) & A young lad followed her \\
    Grohotom se smije: & He laughs out loud: \\
    Gazi gazi, dievocjчice; [\textit{sic}] & Hop, hop, little girl, \\
    Da bys moja byla, & That you may be mine, \\
    “Kad by znala, i videla, & “If I knew for sure \\
    Da by tvoja byla: & That I would be yours, \\
    Mlekom by se umivala, & I would wash with milk, \\
    Da by biela byla. & So I would be white, \\
    Ruƶom by se utirala. & I would rub myself with a rose, \\
    Da by rumena byla; & So I would be flushed, \\
    Svilom (\textit{i}) by se opasala, & I would wear silk, \\
    Da by tanka byla. & So I would be slender.”] \\
\end{longtable}

(\textit{g}) \textit{gaziti} ‘to wade through’. (\textit{h}) \textit{momeч}, or \textit{junak}, or \textit{molodec} ‘youth’. (\textit{i}) \textit{svil}, more original than in other \is{Dialect}dialects \textit{hodbab}.

\begin{small}
\begin{longtable}{ l l }
    \multicolumn{2}{ c }{\normalsize \textit{Another}.\footnote{\textit{272. Не гледај ме што сам малена}, \citet[182]{karadzic_narodne_1824}; probably via \citet[176]{celakovsky_slowanske_1822}. Čelakovský lists this among \il{Serbian}his “Serbian” songs.}}\\
    \\
    Dievojчice, sitna liubiчice! & [Girl, tiny violet! \\
    Liubio by te, ali si malena. & I would love you, but you are so small. \\
    “Liubi me dragij, byti чu i golema.(\textit{k}) & “Love me, darling, and I will be huge. \\
    Maleno je zrno biserovo (\textit{l}) & A pearl is small, \\
    Ale se nosi na gospodskom grlu. & But is worn on the necks of lords. \\
    Malena je ptica prepelica. & A quail bird is small, \\
    Al’ umori konia, i Junaka. & But it killed the horse, and the hero.”] \\
\end{longtable}
\end{small}

(\textit{k}) \textit{velika} ‘big’. (\textit{l}) \textit{biser} ‘gem’. \\

Let us now see the \il{Bohemian}Bohemian or \il{Pannonian}Pannonian, but with \textit{h} changed into \textit{g} in conformity with the other \is{Dialect}dialects. [137]

\begin{small}
\begin{longtable}{ l l }
    \multicolumn{2}{ c }{\normalsize \il{Bohemian}\textit{Bohemian dialect}.\footnote{\textit{8.}, \citet[12]{celakovsky_slowanske_1822}.}} \\ 
    \\
    Kdiƶ sem iшel skerz dubovi les, & [As I went through the oak forest, \\
    \hspace*{0.5cm}Pripadla mnia drimota; & \hspace*{0.5cm}A slumber fell over me, \\
    A za glavu mne do rana & And behind my head by morning \\
    \hspace*{0.5cm}Rozmarina vikvetla. (\textit{m}) & \hspace*{0.5cm}Rosemary bloomed. \\
    Porezal sem veski pruti & I cut all the stems, \\
    \hspace*{0.5cm}Do gromadi spleteni; & \hspace*{0.5cm}And wove them all together, \\
    Ti sem pustil po vodiчke, & I dropped them into the water, \\
    \hspace*{0.5cm}Po vodiчke studenej. & \hspace*{0.5cm}Into the cold water. \\
    Ta, ktera ji lovit bude, & She, who will catch it, \\
    \hspace*{0.5cm}Rozmarinu zelenu, & \hspace*{0.5cm}That green rosemary, \\
    Jiste ta ma mila bude & She will surely be my love, \\
    \hspace*{0.5cm}Za vodiчku studenu. & \hspace*{0.5cm}By the cold water. \\
    Iшli rano k rece panenki, (\textit{n}) & In the morning the maidens went to the river, \\
    \hspace*{0.5cm}Do vieder nabirali, & \hspace*{0.5cm}Collecting water in their buckets, \\
    A pruti kniem z rozmarinu & And stems of rosemary \\
    \hspace*{0.5cm}K samej lavce plinuli. & \hspace*{0.5cm}Floated down to the river-bank. \\
    Tu mlinarova Liduшka & There, the miller’s daughter Lidushka, \\
    \hspace*{0.5cm}Po nix se nagibala, & \hspace*{0.5cm}Leaned forward to gather them, \\
    A neшtiasna golubinka & And, unhappy little dove, \\
    \hspace*{0.5cm}Do vodiчki upadla. & \hspace*{0.5cm}She fell into the water. \\
    Zvoni, zvoni! smutni zvoni! & Ring, Ring! Sad bells, \\
    \hspace*{0.5cm}Co to asi znamena? & \hspace*{0.5cm}What can this mean? \\
    Povjedste mili ptaчkove, & Tell me, dear little birds, \\
    \hspace*{0.5cm}Snad to neni ma mila? & \hspace*{0.5cm}Is that perhaps my love? \\
    “Tvu milu, tve poteшenie & “Your love, your pleasure, \\
    \hspace*{0.5cm}Do rakve ti skladajo & \hspace*{0.5cm}Is being placed into a coffin, \\
    Чtiri muƶi v чernem ruxu, & By four men in black robes, \\
    \hspace*{0.5cm}Do grobu pokladajo. & \hspace*{0.5cm}Is being buried in a grave.” \\
    Ax moj Boƶe najmilejшi! & Ah, my most beloved God, \\
    \hspace*{0.5cm}Ti jsi mi vzal mu nevestu! [138] & \hspace*{0.5cm}You have taken from me my bride! \\
    Povjedste mili ptaчkove & Tell me, dear little birds, \\
    \hspace*{0.5cm}K jej grobiчku cestu (\textit{o}) & \hspace*{0.5cm}The way to her tomb, \\
    “Za verxem tam v kostoliчku, & “Behind the hill there in the little church, \\
    \hspace*{0.5cm}Spivajo v kuru kniezi (\textit{p}) & \hspace*{0.5cm}Sings a choir of priests, \\
    Pjet krokov za kostoliчkem & Five steps from the little church \\
    \hspace*{0.5cm}V grobje tva mila leƶi. & \hspace*{0.5cm}Your love lies in a grave. \\
    Budu plakat, a se sluzit, & I will weep and worship, \\
    \hspace*{0.5cm}Na ten tmavi grob sednu, & \hspace*{0.5cm}On the dark grave I will sit, \\
    A pre tobje ma panenko & And for you, my sweetheart \\
    \hspace*{0.5cm}Tieƶki gori (\textit{q}) ponesu. & \hspace*{0.5cm}Bear heavy sorrow. \\
    Tieƶki ja po nesu gori, & Heavy is the sorrow I bear, \\
    \hspace*{0.5cm}Aƶ mne smert visvobodi & \hspace*{0.5cm}Until death liberates me \\
    A z rosmarinu Veneчek & And lays a wreath of Rosemary \\
    \hspace*{0.5cm}Na moj prikrov poloƶi. & \hspace*{0.5cm}On my shroud.] \\
\end{longtable}
\end{small}

(\textit{m}) \textit{kvet}, with others \textit{svit} ‘bloom’. (\textit{n}) \textit{devojka}, ‘girl’. (\textit{o}) \textit{put}, \textit{draga} [‘way’]. (\textit{p}) \textit{sviaшчeniki} [‘clergymen’]. (\textit{q}) \textit{gora}, ‘pain; something adverse, bitter’.

\subsection*{\hspace*{\fill}§. 25.\hspace*{\fill}}

These songs show how little the \il{Slavic}Slavic \is{Dialect}dialects differ from each other, and they agree entirely in \is{Genius}spirit, but let us proceed further. Some grammarians make mention of a periphrastic future, by means of the verbs \textit{xoшчu}, \textit{imam}, ‘I want’, ‘I have’; but these verbs, when combined with infinitives according to the \is{Genius!Genius of the Slavic language}genius of the \il{Slavic}Slavic language, have a totally different meaning than when they strictly indicate the future, such as \textit{imam pisati}, \textit{sliшati}, \textit{dielati} means ‘I have to write’, etc. but \textit{xoшчu}, hence \textit{xчu}, \textit{xcu}, \textit{чu}, \textit{tju}, indicates the will to do something, but not strictly [139] the future tense of the verb. Admittedly, one also reads in the manuscripts \textit{ne imate vidjeti}, \textit{usliшati}, \textit{vjerovati}, namely ‘you will not see, hear, believe’ etc. In fact, this is a forced expression of the future following the \il{Greek}Greek text, as confirmed by the fact that it is not in common use in any \is{Dialect}dialect. The \il{Serbian}Serbs, however, like the expression of the future by means of \textit{xoшчu}, \textit{чu}, \textit{tju}, such as \textit{Ja чu napisat} [‘I will write’], instead of \textit{napiшu}, \textit{napiшem}, but the context itself teaches that this expression of the future is forced, for how can the will to write be expressed? Is it not with \textit{чu}, \textit{xчu} etc., for instance in \textit{Ja xoчu tvoje ime} (\textit{imeno}) \textit{napisati} [‘I want to write your name’].

The \il{Polish}Poles, the \il{Windic}Winds, and some \il{Illyrian}Illyrians form the future with the perfect added to the auxiliary verb \textit{budu}. Thus the \il{Polish}Pole: \textit{będę xvalil}, -\textit{a}, -\textit{lo} [‘I will praise’]; the \il{Illyrian}Illyrian: \textit{budem xvalio} (\textit{xvalil}), -\textit{la}, -\textit{lo} etc.; the \il{Windic}Wind: \textit{bom} (contracted out of \textit{budem}) \textit{hvalil}. Additionally, both \il{Polish}Poles and the \il{Illyrian}Illyrians, apart for said expression of the future, also form it from the infinitive added to the auxiliary. Thus the \il{Polish}Pole: \textit{będę xvalić} [‘I will praise’]; the \il{Illyrian}Illyrian: \textit{budem xvaliti}. Yet, the \il{Windic}Wind does not depart from its contracted way of speaking: \textit{bom}, \textit{boш}, \textit{bo igral}, -\textit{a}, -\textit{o}, \textit{bomo}, \textit{bote}, \textit{bodo igrali}, -\textit{e} [‘I, you, he/she/it, we, you, they will play’] etc. The formation of the future out of the perfect is also grounded in manuscripts in the \il{Old Church Slavonic}ancient \is{Dialect}dialect, such as: \textit{usnul budet} [‘he will have fallen asleep’], \textit{stvoril budet} [‘he will have created’] etc. And this is called the future exact by the grammarians. It occurs especially in cases where several future tense forms appear in the same sentence, such that the future exact appears only if there is another future following it. The illustrious Relkovich\footnote{Matija Antun Relković, (1732--1798), author of a Slavonic grammar and dictionary (\citeyear{relkovic_nova_1767}).} has approved this usage in his \il{Illyrian}Illyrian grammar as follows: \textit{kado mi budemo imali, damo vam} [‘after we have gotten it, we will give it to you’], [140] \textit{kdi mi budemo zoƶali ƶito, budemo vam ƶati} [‘after we have reaped the grain, we will reap for you’]. Other Slavs use the simple future here, such as: \textit{kdi}, or \textit{gdi zoƶnemo ƶito}, \textit{budemo vam ƶati} [‘after we have reaped the grain, we will reap for you’]. But the \il{Russian}Russians, who are more concise, use instead of this future exact the adjective of the preterit, as follows: \textit{zoƶavшi ƶito naшe}, \textit{budemo vam ƶati} [‘having reaped our grain, we will reap for you’].

\subsection*{\hspace*{\fill}§. 26.\hspace*{\fill}}

The \is{Genius!Genius of the Slavic language}genius of the \il{Slavic}Slavic language rejects the expression of the pluperfect; for this reason there is not any mention of it in any grammars of any \is{Dialect}dialects. Yet some form it from the perfect by adding \textit{byl}, and in \il{Russian}Russian \textit{byvalo} for all genders. Thus \il{Polish}Polish \textit{xvalilem byl}, \il{Bohemian}Bohemian \textit{byl sem xvalil} [‘I had praised’]. But these expressions seem to have come in from the servile imitation of other languages because they are not grounded in the \il{Old Church Slavonic}ancient \is{Dialect}dialect, and also because they are not confirmed by the usage of the \is{Dialect}dialects themselves, unless one would slavishly imitate other languages. For a true \il{Polish}Pole never uses it; \il{Russian}Russian \textit{byvalo}, for similar reasons, is only a grammarians’ fiction and does not reflect the \is{Genius}genius of the language. For it is nothing else than \textit{bylo} put in the neuter frequentative, so \textit{byti}, frequentative \textit{byvati}, ‘to frequently be’, whence among the \il{Pannonian}Pannonians it is also taken in the sense of ‘to inhabit’.

\subsection*{\hspace*{\fill}§. 27.\hspace*{\fill}}

The optative or subjunctive mood, also [141] called the conditional, is used when something is desired, and especially when two propositions are conjoined by means of particles. However, since in an expression of this kind the Slavs always use the auxiliary verb \textit{jesm} [‘I am’], it is important to clarify the meaning of this verb. This verb \textit{jesem}, often contracted to \textit{jsem}, \textit{sem}, \textit{som}, is an existential verb, but not a function verb. It therefore differs from the other verbs which by their very nature indicate some action, as it were, since some entity must exist before it can complete its action. Hence there seem to be as many different existential verbs as there are expressions of time. The present is as follows: \textit{jesem}, \textit{jesi}, \textit{jest}, \textit{jesmo}, \textit{jeste}, \textit{jeso} proceeds from the base form \textit{jesti}, which also signifies ‘to apply the means in order that a living being exists’, hence \il{Latin}Latin \textit{esse} also signifies ‘to exist’ sometimes, but at other times ‘to eat’. But the verb \textit{byti} ‘to exist’, whence \textit{byx}, \textit{byl} etc., is a verb of continuous existence, whence also the origin of the \il{Slavic}Slavic term for the Supreme Being: \textit{Byx}, \textit{Büx}, \textit{Box}, \textit{Bux}, \textit{Bog}. Hence the verb \textit{byti} has no etymological link with \textit{jesti}, but the verb \textit{byti} stands by itself, whence the frequentative \textit{byvati}, also in the sense of ‘to inhabit’, hence \textit{byeda}, or ‘the act of living in misery’, as well as \textit{byedovati} [‘to lament, to moan’] etc. Finally, to indicate future existence there is the verb \textit{budeti}, whence \textit{budu}, \textit{budem} [‘I will be’].

The \is{Conjugation}inflection of these three verbs, distinct from each other, is in truth also regular.

\begin{longtable}{ l l l l l }
    \lsptoprule
    \multicolumn{5}{ c }{Singular.} \\
    \midrule
    Northerners	& 1 \textit{jesem}, & 2 \textit{jesi}, & 3 \textit{jest}, & in abbreviated fashion \\
    Southerners	& 1 \textit{jesem}, & 2 \textit{jesi}, & 3 \textit{jest}, & \textit{sem}, \textit{si}, \textit{je} [142] \\
    \lspbottomrule
    \\
    \lsptoprule
    \multicolumn{5}{ c }{Plural.} \\
    \midrule
    Northerners	& \textit{jesmi}, & \textit{jeste}, & \textit{jesut}. & \\
    Southerners & \textit{jesmo}, & \textit{jeste}, & \textit{jeso}. & \textit{smo}, \textit{ste}, \textit{so}. \\
    \lspbottomrule
\end{longtable}

The \is{Conjugation}inflection of the verb \textit{byti} is likewise regular, and since it expresses only the sense of the preterit, by this very fact it cannot be used as a present. For this reason, the perfect ends just like in other verbs by deletion of -\textit{ti}, and adding -\textit{l}, as follows: Singular \textit{byti}, \textit{byl}, \textit{byla}, \textit{bylo}. Plural \textit{byli}, \textit{byle}, \textit{byla}, with or without an auxiliary verb, \textit{Ja jsem byl} [‘I have been’].

Finally, the \is{Conjugation}inflection of the verb \textit{budeti} likewise stays regular:

\begin{longtable}{ l l l l l }
    \lsptoprule
    \multicolumn{5}{ c }{Singular.} \\
    \midrule
    Northerners	& \textit{budu}, \textit{budeш}, \textit{budet}. & & & \\
    Southerners	& \textit{budem}, \textit{budeш}, \textit{bude}. & & & \\
    \lspbottomrule
    \\
    \lsptoprule
    \multicolumn{5}{ c }{Plural.} \\
    \midrule
    Northerners	& \textit{budemi}, \textit{budete}, \textit{budut}. & & & \\
    Southerners & \textit{budemo}, \textit{budete}, \textit{budo}. & & & \\
    \lspbottomrule
\end{longtable}

\subsection*{\hspace*{\fill}§. 28.\hspace*{\fill}}

No imperative can be formed from \textit{jesti} and \textit{byti}, since the imperative does not contain in itself the idea of a future function, but the imperative of \textit{budeti} is \textit{budi} ‘be’, and the \il{Old Church Slavonic}Church \is{Dialect}dialect confirms that this is the genuine form of the imperative, as follows: \textit{Boƶe Gospodin! budi milost tvoja na nas nynje, i vevjeki. Uшчedri, blagoslavi nji, i prosviti liчe svoje na nji, i omiluj} [‘Lord God! Thy mercy be on us now and forever. Take pity, bless us, and let thy face shine upon us, and have mercy’].

These words are found embroidered with a Phrygian needle on the remains of St. Stephen, the first king of Hungary. [143]

The \il{Russian}Russian says \textit{bud} but adds at the end a yer \textit{ь}, which indicates that the preceding letter should be softened. The \il{Pannonian}Pannonian also softens the \textit{d}; however, the \il{Pannonian}Pannonian only softens the \textit{d} if the following vowel is an -\textit{e} or -\textit{i}. Since therefore in \il{Pannonian}Pannonian \textit{bud} there is no [final] vowel, it follows that the vowel has been swallowed, a vowel which the authority of the \il{Old Church Slavonic}ancient \is{Dialect}dialect advises us to restore, as does that of other southern \is{Dialect}dialects, and the regularity of the language. One should therefore write: \textit{budi}. The same should be understood for all verbs in the imperative, thus not {\Blackletter pod} [‘come’], {\Blackletter brod} [‘wade’], {\Blackletter kru\'t} [‘twist’] in the fashion of the \il{Pannonian}Pannonians, but instead \textit{poidi}, \textit{brodi}, \textit{kruti}; for \textit{pot} is immediately derived from {\Blackletter pod} by eliding the vowel -\textit{i} before and after \textit{d}, as it comes from \textit{ideti}, and the preposition \textit{po}, whence \textit{poidem}, or \textit{poidu}, \textit{poidi}, but not \textit{pot}. The perfect indeed is \textit{poidel}, as it is analogous to \il{Windic}Windic \textit{naidel} from \textit{naideti} [‘to find’]. However, \textit{poiшel} as well as \textit{naiшel} and the like have their root in the infinitive \textit{poiшti}, \textit{naiшti}. For in the \is{Dialect}dialect around Cassovia [\il{Slovak}Slovak: Košice; \il{Hungarian}Hungarian: Kassa; \il{German}German: Kaschau] they say \textit{iшti} instead of \textit{jiti}, which is abbreviated and distorted from \textit{ideti}, whence the present is nowhere \textit{iшem} but \textit{idem}, among the \il{Polish}Poles \textit{idzem}.

\subsection*{\hspace*{\fill}§. 29.\hspace*{\fill}}

Some \is{Dialect}dialects add the particle \textit{nex} to the third person of the imperative: \textit{niex} among the \il{Polish}Poles, \textit{neka} among the \il{Serbian}Serbs, \textit{naj} among the \il{Windic}Winds. This particle emerged from the abbreviated imperative \textit{nexaj} from \textit{nexati} [‘to let’], \textit{lassen} in \il{German}German. But the particle \textit{nex} does not have a place in the imperative in the strict sense, for this particle is used neither in [144] the \il{Old Church Slavonic}ancient \is{Dialect}dialect, nor in \il{Russian}Russian nor in \il{Bohemian}Bohemian. \textit{Nex} is therefore a periphrasis, but not the expression of the imperative in the strict sense, so \textit{nex on bude}, instead of \textit{nexaj on bude}, ‘let him be’, or ‘that he may be’, \textit{lasse ihn seyen} among the \il{German}Germans. Such is also \textit{da}, \textit{da Bog}, similarly among the ancient \il{German}Germans {\wieynk wollte Gott} [‘God willing’], among the \il{Polish}Poles, the \il{Bohemian}Bohemians, and \il{Pannonian}Pannonians \textit{abyx}; here the letter \textit{d} is certainly left out, namely: \textit{da byx}, but this is abbreviated from \textit{da Box}, or \textit{Bog}, \textit{Byx} etc.

\subsection*{\hspace*{\fill}§. 30.\hspace*{\fill}}

The expression of the optative mood is always combined with the auxiliary verb \textit{byti}, so \il{Russian}Russian \textit{Ja}, \textit{ti}, \textit{on}, \textit{ona}, \textit{ono by byl}, -\textit{a} -\textit{o}; in the plural \textit{mi}, \textit{vi}, \textit{oni} etc. \textit{by byli} etc. The word \textit{by} remains invariable everywhere, but the annexed preterit \textit{byl} declines like an adjective. \il{Polish}Polish affixes to \textit{by} also a contraction of \textit{sem}, as follows: \textit{Ja bym byl}, \textit{ti bys byl}. \il{Bohemian}Bohemian has \textit{Ja byx byl}, \textit{ti bys byl}, \textit{on by byl} etc.; in the plural: \textit{byxom}, \textit{byste}, \textit{by bili}, -\textit{e}, -\textit{a}. \il{Illyrian}Illyrian: \textit{bih}, \textit{bi}, \textit{bi}, \textit{bismo}, \textit{biste}, \textit{biшa} with \textit{byl}, -\textit{a}, -\textit{o} added to it. Here all \is{Dialect}dialects agree in essence, as the root is everywhere the same, namely: \textit{byx}, or \textit{bily} or \textit{bi}, only \is{Dialect}dialectal variation occurs. The \il{Old Church Slavonic}ancient \is{Dialect}dialect has: \textit{byx}, \textit{by}, \textit{by}, \textit{byxom}, \textit{byste}, \textit{byшa}. In my humble opinion, I would judge that the \il{Bohemian!Polish-Bohemian}Polish-Bohemian form should be adopted, grounded in the \il{Old Church Slavonic}ancient \is{Dialect}dialect, and mixed in with the southerners’ sweetness, as follows: \textit{byx} (-\textit{m}), \textit{bys}, \textit{by}, \textit{bysmo} (-\textit{i}), \textit{byste}, \textit{byso} (-\textit{ut}), [145] which, if \textit{byl}, -\textit{a}, -\textit{o} is added to it, will be the present of the optative:

\begin{longtable}{ l l l l l l l l l l }
    \lsptoprule
    \textit{Ja}, & \multicolumn{1}{c}{—} & \multicolumn{1}{c}{—} & \textit{byx} (-\textit{m}) & \multirow{3}{*}{\rotatebox[origin=c]{90}{\footnotesize \textit{byl}, -\textit{a}, -\textit{o}}} & \textit{mi} & \multicolumn{1}{c}{—} & \multicolumn{1}{c}{—} & \textit{bysmo} (-\textit{i}), & \multirow{3}{*}{\rotatebox[origin=c]{90}{\footnotesize \textit{byli}, -\textit{e}, -\textit{a}.}} \\
    \textit{ti}, & \multicolumn{1}{c}{—} & \multicolumn{1}{c}{—} & \textit{bys} & & \textit{vi} & \multicolumn{1}{c}{—} & \multicolumn{1}{c}{—} & byste & \\	
    \textit{on}, & \textit{ona}, & -\textit{o}, & \textit{by} & & \textit{oni}, & \textit{one}, & \textit{ona} & \textit{by} (\textit{byso}) \\
    \lspbottomrule
\end{longtable}

In the plural, the adjectival perfect \textit{byli} is not altered in the \il{Old Church Slavonic}Church \is{Dialect}dialect, \il{Russian}Russian or \il{Polish}Polish, but the \il{Windic}Winds differentiate between the masculine \textit{byli}, feminine and neuter \textit{byle}; the \il{Bohemian}Bohemians differentiate masculine \textit{byli} from feminine \textit{byly} and neuter \textit{byla}. There are nonetheless \is{Dialect}dialects in which they say the masculine as \textit{byli}, the feminine as \textit{byle}, the neuter as \textit{byla}, and \textit{reasonably so}; for if in the singular it is declined as a formal adjective, why would it not also be declined in the plural? The rational grammar of the language certainly urges that as well, and the southern \is{Dialect}dialects, specifically \il{Serbian}Serbian, confirm it. The grammarians sometimes call this expression of the optative imperfect, sometimes perfect, misled as they are by the norm of other languages. Yet this is nonetheless the true and genuine present expression of the optative, for the Slav does not pronounce anywhere a desire in the present tense in any way other than the abovementioned expression, namely by adding \textit{by} etc. Thus ‘I would like to write’: \textit{Ja byx pisal}; ‘I would like to be’: \textit{Ja byx byl}, -\textit{a}, -\textit{o} etc. If \textit{byl} is moreover added to it, it will be an indication of a past desire: in this way, \textit{byl byx igral}, -\textit{a}, -\textit{o} [‘I would have liked to play’] etc. \textit{pisal} [‘… write’], \textit{mislel} [‘… think’], \textit{molil} [‘… pray’] etc. It is the true perfect of the optative, but not a pluperfect. For grammarians who call that expression of the optative “pluperfect”, abandon the perfect, but how can you lack a perfect but have a pluperfect? Etc. [146]

\subsection*{\hspace*{\fill}§. 31.\hspace*{\fill}}

The optative lacks a future in \il{Slavic}Slavic grammars because the present already expresses the idea of the future, for future, not present things are wished for, but the \il{Latin}Latins’ so-called conditional future, or preceding future is expressed among the Slavs either by a simple future or by the verbal adjective of the preterit or by the so-called future exact, for instance \textit{Postquam prandero}, \textit{scribam} [‘After I will have had breakfast, I will write’], in \il{Slavic}Slavic: \textit{Gdi odobjedujem}, \textit{na piшu pismo}, or \textit{odobjedavшi napiшu pismo}, likewise \textit{Kdi napiшu pismo}, \textit{odidem}, or \textit{napisavшi pismo odidem}, or \textit{Gdi napisal budem pismo}, \textit{odidu} etc. From this the expressions of the optative will now be very easily brought out.

\newpage

\begin{longtable}{ l l l l l l l l l }
    \lsptoprule
    \multicolumn{9}{c}{Present.} \\
    \midrule
    \multicolumn{5}{c}{Singular.} & \multicolumn{4}{c}{Plural.} \\
    \textit{Ja} & \multicolumn{3}{l}{\textit{byx}} & {\multirow{3}{*}{\rotatebox[origin=c]{90}{\scriptsize \textit{uчil}, -\textit{a}, -\textit{o}}}} & \textit{Mi} & \multicolumn{2}{l}{\textit{bismo} (\textit{i})} & {\multirow{3}{*}{\rotatebox[origin=c]{90}{\shortstack[l]{\scriptsize \textit{uчili} or \\ \scriptsize \textit{uчili}, -\textit{e}, -\textit{a}.}}}} \\
    \textit{ti} & \multicolumn{3}{l}{\textit{bys}} & & \textit{Vi} & \multicolumn{2}{l}{\textit{byste}} \\
    \textit{on} -\textit{a}, -\textit{o} & \multicolumn{3}{l}{\textit{by},} & & \textit{oni}, -\textit{e}, -\textit{a} & \multicolumn{2}{l}{\textit{by} (\textit{bysut}, \textit{byso})} \\
    \\
    & & & & \multicolumn{4}{c}{The perfect is made by adding \textit{byl}, -\textit{a}, -\textit{o}:} \\
    \multicolumn{5}{c}{Singular.} & \multicolumn{4}{c}{{[}Plural.{]}} \\
    \textit{Ja} & {\multirow{3}{*}{\rotatebox[origin=c]{90}{\scriptsize \textit{byl}, -\textit{a}, -\textit{o}}}} & \textit{byx} & {\multirow{3}{*}{\rotatebox[origin=c]{90}{\scriptsize \textit{uчil}, -\textit{a}, -\textit{o}}}} & & \textit{Mi} & {\multirow{3}{*}{\rotatebox[origin=c]{90}{\shortstack[l]{\scriptsize \textit{byli} or \\ \scriptsize \textit{byli}, -\textit{e}, -\textit{a}}}}} & \textit{bysmo} (\textit{i}) & {\multirow{3}{*}{\rotatebox[origin=c]{90}{\shortstack[l]{\scriptsize \textit{uчili} or \\ \scriptsize \textit{uчili}, -\textit{e}, -\textit{a}.}}}} \\
    \textit{ti} & & \textit{bys} & & & \textit{Vi} & & \textit{byste} \\	
    \textit{on}, -\textit{a}, -\textit{o} & & \textit{by} & & & \textit{oni} -\textit{e}, -\textit{a} & & \textit{by} (\textit{byso}, \textit{bysut}) \\	
    \lspbottomrule
\end{longtable}

\subsection*{\hspace*{\fill}§. 32.\hspace*{\fill}}

Every verb is \is{Conjugation}inflected according to the aforementioned principles, but I am not unaware that some [147] \is{Dialect}dialect grammarians will condemn these principles, but they alone will reach such a judgment of condemnation, since what one \is{Dialect}dialect approves, the other condemns, and vice versa. Nor indeed can reference grammarians sustain any other opinion. However, it remains an unshakable truth that in accordance with the \is{Genius!Genius of the Slavic language}genius of the \il{Slavic}Slavic language there is only one single form of \is{Conjugation}verb inflection. Let us see how six forms of \il{Russian}Russian can be reduced to a single form, a reduction which we will see confirmed by the usage in various other \is{Dialect}dialects.

\subsubsection*{\textit{The verbs of six forms reduced to one}. \textit{The infinitive}.}

\textit{Imati} [‘to have’], \textit{ljeti} [‘to pour’], \textit{pliuti} [‘to spit’], \textit{terpieti} [‘to suffer’], \textit{dojiti} [‘to milk’], \textit{ƶelati} [‘to wish’], \textit{tkati} [‘to weave’] etc. Casting the -\textit{ti} off from this and putting -\textit{u} or -\textit{m} will result in the present: \textit{imam}, \textit{liejem}, \textit{pliujem}, \textit{terpjem}, \textit{dojim}, \linebreak{}-\textit{iш}, -\textit{i}, \textit{dojimo}, \textit{dojite}, \textit{dojo}, or in northern fashion: \textit{doju}, -\textit{iш}, -\textit{i}, \textit{dojimi}, \textit{dojite}, \textit{dojut} etc. Now putting -\textit{l} instead of -\textit{ti} will give the perfect: \textit{imal}, \textit{liel}, \textit{pliul}, \textit{terpjel}, \textit{dojil}, \textit{ƶelal} etc., with or without the auxiliary, as follows: \textit{ja dojil} or \textit{ja sem dojil} [‘I milked’] etc.

The nature of the future has been explained rather abundantly. Since the aforementioned are basic verbs, adding \textit{budu} or \textit{budem} will give the future.

The imperative is formed by casting off -\textit{ti} and putting -\textit{i}, or -\textit{j}; hence I will either write \textit{imaj}, \textit{lieji}, \textit{pliuj} or \textit{imai}, \textit{liei}, \textit{pliui}, \textit{terpej}, \textit{tkaj}.

The present of the optative will be made from adding to the perfect \textit{imal} the auxiliary \textit{byx} as follows: \textit{ja byx imal}, \textit{dojil}, \textit{terpiel}, \textit{ƶelal} etc.; if one would again add \textit{byl}, -\textit{a}, -\textit{o} to that, it will become the perfect of the optative, as follows: \textit{Ja byx byl terpiel} [‘I would have suffered’]. [148]

Then the adjectives are derived by changing the present form of the infinitive from -\textit{iti} into -\textit{ici}, or -\textit{uci}, or -\textit{aci}; some \is{Dialect}dialects pronounce the \textit{c} as ч. Thus, \textit{imajici}, \textit{terpjejici}, \textit{dojici}, or \textit{dojaci}, \textit{ƶelajaci}, or \textit{ƶelujuci}, \textit{tkajuci} etc., or \textit{tkajuчi}, \textit{ƶelajuчi}, \textit{sluшajuчi} etc.

The adjective of the active perfect, with the -\textit{ti} removed, and replaced with -\textit{v}, thus: \textit{imav}, \textit{liev}, \textit{pliuv}, \textit{terpiev}, \textit{dojiv}, \textit{ƶelav}, \textit{tkav}.

The present participle of the passive puts -\textit{m} instead of -\textit{ti}: \textit{imam}, -\textit{a}, -\textit{o}, \textit{liem}, -\textit{a}, -\textit{o}, \textit{pilum}, -\textit{a}, \textit{dojim}, -\textit{a}, -\textit{o}, \textit{ƶelam}, -\textit{o}, -\textit{a} etc.

The participle of the passive preterit puts -\textit{n} instead of -\textit{ti}: \textit{iman}, \textit{liejen}, \textit{pliun}, \textit{terpjen}, \textit{dojin}, or \textit{dojen}, \textit{ƶelan}, \textit{tkan}, \textit{pisan}, \textit{чitan} etc.

The adverb is formed as a present participle of the active, only with the final -\textit{i} removed, as follows: \textit{imajuc}, \textit{tkajuc}, \textit{pliujuc}, \textit{dojuc}, or \textit{dojac}, \textit{sedac}, \textit{smiejac} etc.

Various verbal substantives, in turn, but especially those ending in -\textit{nie}, which denote the present function, so to speak, and those ending in -\textit{nost}, which denote the preterit function, emerge as follows: \textit{imanie}, \textit{lienie}, \textit{pliunie}, \textit{terpjenie}, \textit{dojenie}, \textit{ƶelanie}, \textit{tkanie}, \textit{uчenie} etc. But these and similar matters belong to the rational compilation of a dictionary.

Perhaps someone will, I think, reprove me for deriving an improper word like \textit{imanie}, but if we read and investigate other \is{Dialect}dialects, we will see that it is in use and means ‘to have something’, or more precisely ‘the having of something’, even if for \il{Pannonian}Pannonians \textit{imanie} might [149] seem a mere fiction. The same should be understood about the other words, and for that reason one should examine the \is{Genius}genius of the language by comparing \is{Dialect}dialects etc.

\subsection*{\hspace*{\fill}§. 33.\hspace*{\fill}}

In addition to the various regular \is{Conjugation}conjugation forms, \is{Dialect}dialect grammarians also list irregular verbs, though very often verbs that are irregular in one \is{Dialect}dialect are regular in another, and vice versa. For this reason, the goal and logic of language itself recommends adopting the regular form. In the \il{Old Church Slavonic}ancient \is{Dialect}dialect, the following verbs are reported as irregular: \textit{Jam} (\textit{jem}, \textit{jedem}) [‘I eat’], \textit{snjem} [‘I will eat’], \textit{vjem} [‘I know’], \textit{dam} [‘I will give’], \textit{idu} (\textit{idem}) [‘I go’], \textit{reku} [‘I say’], \textit{imu} [‘I have’], \textit{pnu} [‘I stretch’], \textit{ƶnu} [‘I harvest’], \textit{naчnu} [‘I start’], \textit{xoшчu} [‘I want’], \textit{чtu} [‘I read’], \textit{viƶdu} [‘I see’], \textit{vladu} [‘I rule’], \textit{iшчu} [‘I seek’].\footnote{\citet[537--543]{dobrovsky_institutiones_1822}.} More do not occur, but both these and others are variously read in various monuments of sacred antiquity, nor indeed do the monuments themselves agree. Hence, it is also difficult to judge whether this or that expression is more genuine than others, as various people wrote in various ways as they spoke. One reads as follows in the Venetian Psalter \textit{srebra svojego ne daśt v lixvu}, ‘he did not give his silver for interest’; see how \textit{dast} expresses the perfect; it is elsewhere read as \textit{dade}, and even as \textit{dal}, as in Psalm 48: \textit{Dal jesi veselje v serci mojem} ‘you have given gladness in my heart’.\footnote{In the King James Version, this is Psalm 4:7.} The imperative is read as \textit{daƶd}, \textit{dadi}, \textit{daj} etc., just as the most erudite Dobrovski \ia{Dobrovský, Josef}expounds in his grammar of the \il{Old Church Slavonic}ancient \is{Dialect}dialect. Let us now look at the use of the aforementioned verbs [150] in \is{Dialect!Living dialect}living dialects: \textit{Jam}, \textit{jem}, \textit{jedem}, from the base form \textit{jedeti}, \textit{jesti}, \textit{jeti}, \textit{jiti}, if the base form were \textit{jedeti}, it would be: \textit{jedem}, \textit{jedel}, \textit{jedev}, \textit{jeden}, \textit{jedenje}; if from \textit{jeti}, it would be \textit{jem}, \textit{jel} etc. For this is also the usage in Pannonia; \textit{snjem}, however, is entirely regular in the \is{Dialect}dialects. \textit{Vjiem} is abbreviated from \textit{vjedem}, from the base form \textit{vjedeti}, whence one says in the preterit \textit{vjedel}, \textit{vjedenie} etc. \textit{Dati}, however, follows \textit{ƶnuti}, \textit{rekati}, \textit{ideti}, \textit{pnuti}, \textit{ƶnuti},\footnote{\textit{Sic}, this example occurs twice in \ia{Herkel, Jan}Herkel’s original text.} \textit{naчnuti}, \textit{videte}, \textit{iskati}, \textit{vladati}, \textit{xoшчeti}, \textit{xceti}, \textit{чtiti}, \textit{ctiti}: it would not occur to anyone to count them among the irregular verbs, since if we regard all \is{Dialect}dialects as one language, they all follow one previously established form, shown here:

\begin{longtable}{ l l l l l l }
    \lsptoprule
    \textit{xcem} [‘I want’] & (\textit{u}) & \textit{xcel}, & \textit{xcev}, & \textit{xcen}, & \textit{xcenie} \\
    \textit{vidim} [‘I see’] & (\textit{u}) & \textit{videl}, & \textit{videv}, & \textit{viden}, & \textit{videnje} \\
    \textit{чtim} [‘I read’] & (\textit{u}) & \textit{чtil}, & \textit{чtiv}, & \textit{чten}, & \textit{чtenie} \\
    \lspbottomrule
\end{longtable}

\textit{Imu}, together with the composites \textit{poimu}, \textit{naimu}, \textit{zaimu}, is regular: \textit{pomu}, \textit{poimal}, \textit{poimav}, \textit{poimanie}, \textit{poiman}, \textit{poimaj} etc.; \textit{reku}, from \textit{rekati}, \textit{rekal}, \textit{rekav}, \textit{rekaj}, \textit{rekanie}, but from the base form \textit{reчeti}: \textit{reчem}, \textit{reчel}, \textit{reчen}, \textit{reчenie} etc. Thus if we consider the \il{Slavic}Slavic language through all its \is{Dialect}dialects as one language, then every irregularity dissipates like clouds at dawn. However, if we consider the \is{Dialect}dialects individually, they will be more or less overwhelmed with exceptions, and experience teaches us that every day new ones arise. For every \is{Dialect}dialect has its particularities, or so-called \is{Provincialism}provincialisms either to greater or lesser extent.

For instance, some \il{Russian}Russians are used to changing \textit{d} into \textit{ƶ}, \textit{ti} into \textit{ч}, [151] \textit{z} into \textit{ƶ}, \textit{s} into \textit{ш}, for instance from the base forms:

\newpage

\begin{longtable}{ l l l l l }
    \lsptoprule
    \textit{budjiti} [‘to waken’] & they form & \textit{buƶu}, & \textit{buƶen}, & \textit{buƶenie} \\
    \textit{krutiti} [‘to twist’] & — & \textit{kruчu}, & \textit{kruчen}, & \textit{kruчenie} \\
    \textit{grozjiti} [‘to threaten’] & — & \textit{groƶju}, & \textit{groƶjen},& \textit{groƶjenie} \\
    \textit{rosjiti} [‘to dew’] & — & \textit{roшu}, & \textit{roшien}, & \textit{roшienie} \\
    \textit{pustiti} [‘to let, to allow’]	& — & \textit{puшчu}, & \textit{puшчen}, & \textit{puшчenie} \\
    \textit{mysliti} [‘to think’] & — & \textit{myшliu}, & \textit{myшlien}, & \textit{myшlienie} \\
    \lspbottomrule
\end{longtable}

The \il{Polish}Pole counts the following as irregular verbs: \textit{iestem} ‘I am’, \textit{jem} ‘I eat’, \textit{viem} ‘I know’, \textit{smiem} ‘I dare’, \textit{idzem} ‘I go’, \textit{mam} ‘I have’, \textit{dajem} ‘I give’, \textit{vidzę} ‘I see’, \textit{mogę} ‘I can’; but in other \is{Dialect}dialects these verbs are regular, for instance: \textit{iestem} from the base form \textit{jesti} ‘to be’, in the present the \il{Polish}Poles improperly retain the letter \textit{t}, which as the infinitive ending is not retained in any other \is{Dialect}dialect. Hence it follows that \textit{jesem} etc. is more correct, and by adopting it every irregularity of this verb ceases to exist. \textit{Jem} is abbreviated out of \textit{jedem} from the base form \textit{jedeti}, \textit{jedel} etc. In \textit{jem} the root letter \textit{d} is elided, a fact made clear from the frequentative \textit{jedavam}, but abbreviated to \textit{jedam}, among the \il{Polish}Poles \textit{jadam}. \textit{Viem} has been cleared up above; moreover, \textit{smiem} and \textit{mam} are regular, namely: \textit{smiem}, \textit{smiel}, \textit{smiev}, \textit{smiej}, \textit{smien}, \textit{smienie}; \textit{mam}, \textit{mal}, \textit{mav}, \textit{maj}, \textit{man}, \textit{manie}. 

However, all \is{Dialect}dialect speakers pronounce \textit{ideti} in the perfect as \textit{iшel}, where its etymology disappears entirely with the letter \textit{d}, but the \il{Polish}Poles retain the letter \textit{d} also in the perfect, namely \textit{sedlem}. This variation has indeed arisen from the fact that different people derive them in different ways from various base forms, namely: \textit{ideti}, \textit{jiti}, \textit{jisti}, \textit{isti}, \textit{iшti}, as in upper [152] Pannonia near Galicia one says \textit{iшti}, from which the regular perfect formation \textit{iшel} is derived. Hence it is clear that usage alone does not dictate how languages work, but also rational rules. Let, however, no one judge me for explaining that the perfect \textit{idel} derives from \textit{idem}, since among the \il{Windic}Winds \textit{naidel}, from \textit{naideti}, is in use where the \il{Pannonian}Pannonian says \textit{naiшel} etc. As the frequentative of the verb \textit{idzem} the \il{Polish}Poles use \textit{xodzę} or \textit{xodim}, but \textit{xodim} is not its frequentative, but the frequentative of \textit{idem} is \textit{idievam}, \textit{idievavam}, that of \textit{xodim} is \textit{xodivam}, \textit{xodivavam}, as usage confirms. \textit{Dajem} and \textit{vidzę}, however, are regular in the other \is{Dialect}dialects. \textit{Mogę} (\textit{mogu}, \textit{moƶem}): the grammarian says that in the other \is{Dialect}dialects this verb lacks the expression of the imperative and the infinitive, but he does not give a reason why. In the meantime, the verb \textit{mogu} expresses ability, and hence has no imperative, since it does not depend on willpower, but on ability. In imitation of other verbs, however, what forbids us from saying \textit{mogej}, \textit{moƶej}? According to the \il{Polish}Polish grammarian, moreover, one sometimes hears the imperative \textit{modz}; and whence the letter \textit{d}? From the following: the root of this verb is \textit{moшч} in the \il{Old Church Slavonic}ancient \is{Dialect}dialect, in the others \textit{moc}, hence the base form \textit{mocti}, \textit{moceti}, whose imperative is \textit{moci} or \textit{moc}. \il{Polish}Polish \textit{modz} is analogous in terms of sound, but not in \is{Orthography}orthography. For among the Slavs \textit{c} changes easily into \textit{ƶ}, hence \textit{moƶeti} instead of \textit{moceti}, \textit{mocti}, but \textit{ƶ}, in turn, substitutes for \textit{g}, hence \textit{mogu}, \textit{mogel}. Furthermore, among the \il{Pannonian}Pannonians and \il{Bohemian}Bohemians \textit{g} is changed into \textit{h}, hence \textit{mohel} instead of \textit{mogel}, [153] which is why three base forms with the same meaning emerge, namely: \textit{moceti} [‘to be able to’], whence \textit{onemoceti} [‘to fall ill’]; \textit{moƶeti}, whence \textit{pomoƶeti} [‘to help’], and \textit{mogeti}, which are to a great extent analogous to \il{German}German {\wieynk mögen}  [‘to be able to’]. And from these base forms we will derive regularly formed expressions, namely:

\begin{longtable}{ l l l l l }
\lsptoprule
    \textit{mocem} & (\textit{u}) & \textit{mocel}, & \textit{mocen}, & mocenie, \\
    \textit{moƶem} & (\textit{u}) & \textit{moƶel}, & \textit{moƶen}, & \textit{moƶenie}, \\
    \textit{mogem} & (\textit{u}) & \textit{mogel}, & \textit{mogen}, & \textit{mogenie} etc. \\
    \lspbottomrule
\end{longtable}

The same picture emerges with the verbs \textit{goniti}, \textit{honiti} [both meaning ‘to chase’], \textit{ƶoniti}, \textit{ƶenuti} [both meaning ‘to marry’]. In particular, the letters \textit{ƶ}, \textit{g}, \textit{h} are used interchangeably in the various \is{Dialect}dialects. Hence, from \textit{ƶoniti ƶonim} or \textit{ƶoniem}, \textit{zonil} etc.; from \textit{goniti}, then, \textit{gonim} (\textit{honim}), \textit{gonil}, \textit{gonen}, \textit{gonenie} etc. From this verb \textit{ƶena}, \textit{ƶona} [‘woman, wife’] seems to draw its origin, since it was customary in the most ancient times among certain tribes to drive marriable virgins out to a certain place and wed them to the most promising males. This custom was native among the Slavic tribes, as shown by examples from recent times in Carpathian Pannonia. Also, in Croatia, among the populace, there are still women who are treated by their men almost as slaves. This is to say, men do not even eat with women, but their wives stand behind them while they are sitting at the table, and eat only if their husbands give them something.

\subsection*{\textit{On the passive expression of verbs}. [p. 153--157]}
\addcontentsline{toc}{section}{\indent On the passive expression of verbs. [p. 153--157]}

\subsection*{\hspace*{\fill}§. 1.\hspace*{\fill}}

Some \is{Dialect}dialect grammarians illustrate the passive expression of verbs by means of extensive paradigms; [154] but first it is important to consider whether the Slavs have their own and truly genuine expression of the passive voice, or indeed supplement it only through periphrasis. In the entire \il{Slavic}Slavic language, only two genuine passive expressions are found. Firstly, there is the present participle of the passive, which is observed in the Bible composed in the \il{Old Church Slavonic}ancient \is{Dialect}dialect. The \il{Russian}Russians also use the same in loftier style, but in the other \is{Dialect!Living dialect}living dialects few traces of it appear, except for \il{Pannonian}Pannonian phrases such as \textit{znami чelovek} ‘renowned person’, \textit{vedome hrjexi}, more correctly \textit{vjedomi griexi} [‘conscious sins’], \textit{vjedome vini} [‘well-known wines’]; \textit{vjedomi}, -\textit{a}, -\textit{o}, is derived from the verb \textit{vjedem} ‘I know’, the abbreviated form of which is \textit{vjem}; \textit{znami}, -\textit{a}, -\textit{o}, however, comes from \textit{znati} ‘to know’, hence the participle of the preterit \textit{znan}, -\textit{a}, -\textit{o}; \textit{vjeden}, -\textit{a}, -\textit{o}.

\subsection*{\hspace*{\fill}§. 2.\hspace*{\fill}}

Since therefore usage itself confirms the existence of a passive present adjective, it should clearly be considered a treasure of the language, as it were, characterized by -\textit{m}, where the preterit takes -\textit{n}. \is{Dialect}Dialect grammarians use the participle of the preterit alone to express various meanings of the passive by adding the auxiliary of the present, preterit, or future, such as the present: \textit{ja sem liuben} [‘I am being loved’], \textit{ja byl liuben} [‘I am being loved’], \textit{ja budem liuben} [‘I am being loved’]. But the \il{Russian}Russians accurately distinguish the present from the preterit, as follows: \textit{ja liubim jesm} [‘I am loved ’], or \textit{liubaem jesm} [‘I have been loved’], or \textit{liubomi}, -\textit{a}, -\textit{o} \textit{jesm} [‘I am loved’].

Numerous grammarians argue that the object expression [155] of the reflexive \textit{sia}, \textit{sa}, \textit{se} added to the verb is a passive expression, such as \textit{liubiti sia} [‘love oneself’], \textit{militi sia} [‘adore oneself’], \textit{dvigati sia} [‘move oneself’] etc. on the grounds that, for instance, \textit{ja sa menujem}, or \textit{zoviem} etc. corresponds to the \il{Latin}Latin \textit{ego nominor} [‘I am called’] etc., but it also corresponds to the active expression: \textit{ego me nomino} [‘I call myself’] etc. Hence, only this follows: if the subject coincides with the object in the same person, such as: \textit{Ja sia}; \textit{ti sia} etc.; \textit{mi vi}; \textit{oni}, \textit{one}, \textit{ona sia} (abbreviated out of \textit{sebia}), it fulfills roles that in other languages are expressed both by the active and the passive, but it does not follow that this would be the genuine passive way of speaking.

Many grammarians, especially ancient ones, connect that reflexive \textit{sebja} = \textit{sia} with the verb, and hence a reflexive verb has emerged with them, such as: \textit{lubitisia} [‘to be loved’], \textit{odreчatisia} [‘to disown’] etc. Thus \il{Russian}Russian has \textit{ja i moj prijatel odreчemsia} [‘I and my friend will disown each other’], instead of \textit{odreчem sia}. However, there are no reflexive verbs in the \il{Slavic}Slavic language, because they have no characteristic marker; for \textit{sia} is not a verbal marker but a pronoun in the so-called accusative case, a pronoun which always occurs in said case as long as the subject coincides with the object in terms of person, or as long as the nominative and the accusative are in the same person. Consider \textit{Ja liubim mojego otca} [‘I love my father’], here \textit{liubim} is not reflexive.\footnote{\ia{Herkel, Jan}Herkel confuses the \il{Russian}Russian first person singular \textit{ja ljublju} with the first person plural \textit{my ljubim}.} But if I would say \textit{moja sestra liubitsia} [‘my sister is loved’], or similarly \textit{dnem trudimsia a noчju pokoimsia} [‘we work during the day and rest at night’], instead of \textit{dnem trudime}, or \textit{trudimi}, or \textit{trudimo sia}, \textit{a noчju pokoimo sia}. For this reason, the words should not be [156] written together \textit{sviatitisia} [‘to sanctify’], \textit{xvalitisia} [‘to boast’], \textit{osobitisia} [‘to separate oneself, to stand apart’], \textit{gorbitisia} [‘to hunch; to slouch’], \textit{silitisia} [‘to struggle’], \textit{braчitsa} [‘to get married’] instead of \textit{osobiti}, \textit{gorbiti}, \textit{siliti}, \textit{braчiti} etc. \textit{sia}. For these verbs do not always occur with \textit{sia}; and even if they did, the reflexive \textit{sia} would still be a different word from the verb.

\subsection*{\hspace*{\fill}§. 3.\hspace*{\fill}}

We therefore have for the present and the passive perfect genuine passive expressions; for the future we lack a genuine expression. Nonetheless, in the usage of all Slavic nations the participle is employed by adding the auxiliary \textit{budu} or \textit{budem} to it. And in this way, the full expression of the passive emerges from the present participle and the preterit passive by adding the auxiliaries \textit{jesem}, \textit{byl}, \textit{budem}, as follows:

\subsubsection*{\textit{The present of the indicative}.}

\textit{Znam}, -\textit{a}, -\textit{o}, \textit{jesem}, \textit{jesi}, \textit{jest}, plural \textit{znami}, -\textit{e}, -\textit{a} \textit{jesmo}, \textit{jeste}, \textit{jeso} [‘I, you, he/she/it am/is/are being known; we, you, they are being known’] etc. Thus the \il{Russian}Russian: \textit{dvigaem}, -\textit{a}, -\textit{o} etc, \textit{jesem}, \textit{jesi}, \textit{jest} [‘I, you, he/she/it am/is/are being moved’] etc. The \il{Polish}Pole uses that participle of the preterit as follows: \textit{xvaloni}, -\textit{a}, -\textit{o} \textit{jestem} etc., but the \il{Polish}Polish grammarian himself spontaneously confesses that that expression does not correspond to the present but to the perfect, as \textit{ja jesem xvaloni} refers to the past, namely: ‘I have been praised’ etc.\footnote{\ia{Herkel, Jan}Herkel here somewhat misrepresents \ia{Bandtkie, Jerzy Samuel}Bandtkie, who forms the \il{Polish}Polish passive “through the auxiliary \textit{bydź} and its derivative \textit{bywać}, with the passive preterit participle, […] e.g. \textit{jestem prześladowany}, \textit{a}, \textit{e} ‘I am persecuted’”. See section 265--267 in \citet{bandtkie_polnische_1824}, quotation from (\citeyear[269]{bandtkie_polnische_1824}), see (\citeyear[269--271]{bandtkie_polnische_1824}).}

The perfect \textit{znan}, \textit{xvalen}, \textit{uчen}, \textit{vidjen}, -\textit{a} -\textit{o jesem}, \textit{jesi}, \textit{jest} [‘I have, you have, he/she/it has been known, praised, taught, seen’], or also by adding \textit{byl}, as follows: \textit{ja jesem vidien byl} [‘I have been seen’, masculine], \textit{ja jesem vidiena byla} [‘I have been seen’, feminine], \textit{ona vidiena byla} [‘she has been seen’] etc. [157]

The future: \textit{budu} (\textit{m}), -\textit{ш}, -\textit{e}, \textit{znani}, -\textit{a}, -\textit{o}, \textit{budemo}, \textit{budete}, \textit{budo znani}, -\textit{e}, -\textit{a} [‘I, you, he/she/it will be known; we, you, they will be known’] etc.

Optative: \textit{byl}, -\textit{a}, -\textit{o}, \textit{byx}, \textit{bys}, \textit{byl znan}, or -\textit{i}, -\textit{a}, -\textit{o} [‘were I, you, he/she/it to know’] etc.

The perfect is made by adding again \textit{byl} to the present, as follows: \textit{Byl byx znan byl}, \textit{byla byx znana byla}, \textit{bylo by znano bylo} [‘he/she/it would have been known’] etc. And this is the easy formation of all passive verbs, which by their very nature do not allow any other form of passive expression.

\section*{Section VIII. \textit{On the indeclinable parts of speech}. \linebreak{}[p. 157--164]}
\addcontentsline{toc}{section}{Section VIII. \textit{On the indeclinable parts of speech}. [p. 157--164]}

\subsection*{\hspace*{\fill}§. 1.\hspace*{\fill}}

We have dealt with the declinable parts of speech, since the preposition, the adverb, the interjection, and the conjunction are not \is{Inflection}inflected, and these matters are reserved for a universal dictionary. With regard to their rules, for instance, the prepositions will not be discussed, because Slavs everywhere agree on this point. In so far as certain adverbs are capable of forming a comparative, they follow the rules governing the comparative.

\subsection*{\hspace*{\fill}§. 2.\hspace*{\fill}}

Regarding syntax, since the \il{Slavic}Slavic language is original, we learn from experience that its syntax, too, is everywhere original and the same in the \is{Dialect}dialects. We may mention examples of its syntax drawn from [158] various \is{Dialect}dialects. Consider a Lord’s Prayer in the \il{Old Church Slavonic}ancient \is{Dialect}dialect, from a 1483 Glagolitic missal: \\
\\
\indent \textit{Otчe naш, iƶe jesi na nebiesiex, svati se ima tvoje, priidi carstvo tvoje, budi vola tvoja, jako na nebesi, i na zemli. Xlieb naш vsedanni, daj nam dnes, i odpusti nam dlgi naшe, jakoƶe i mi odpuшчaem dlƶnikom naшim, i ne vevedi nas v napast, izbavi nas od neprijazni.} \\
\\
It is also read differently in Matthew 6:9: \\
\\
\indent \textit{Otчe naш, iƶe jesi na nebiesiex, da svatisia imja tvoje, da priidet carstvi tvoje, da budiet volia tvoja, jako na nebesi, na zemli, xlieb naш nasuшчnii daƶd nam dnes, i ostavi nam dolgi naшija, jako i mi ostavlajem dolƶnikom naшim, i ne vevedi nas v napast, no i zbavi nas od lukavago. Jako tvoje jest Carstvije, i sila, i Slava vo vjeki Amin.} \\
\\
Popular usage, furthermore, demonstrates that the Lord’s Prayer is also read in another way. Let us look at a brief analysis of the former forms: \textit{iƶe}, ‘who, which’, is composed out of \textit{i} or \textit{ji} and \textit{ƶe}. \textit{Ji} is the root pronoun of the masculine third person, and \textit{ƶe} is a particle which in some \is{Dialect}dialects, such as the \il{Pannonian}Pannonian, is used as the conjunction ‘that’. This particle also appears as the postposition of other little words, such as: \textit{jakoƶe i mi odpuшчamo} etc. Instead of \textit{iƶe} one reads in missals also \textit{ki} in the fashion of the southerners, abbreviated out of \textit{keri}; \textit{carstvie} is also read as \textit{cartsvo} from \textit{car} or ‘emperor’. Similarly, the two Books of Kings are called [159] \textit{Carskie knigi}. \textit{Jako} is also read as \textit{jakoƶe} as well as \textit{kako}. \textit{Vsedanni} appears also as \textit{vsagdanni}, likewise with the \textit{g} changed into \textit{k}: \textit{vsakdanni}, or even \textit{vsakdaшni}. \textit{Dnes}, also \textit{danas}, \textit{dans}; \textit{otpuшчaem}, \textit{odpusчaemo}, \textit{otpuшчam}. \textit{Vevedi}, \textit{vavedi}. The word \textit{past} ‘temptation’ is used among the Carpathian \il{Pannonian}Pannonians in the meaning of ‘ambush, traps which are prepared on account of an ambush’. \textit{Neprijazen} in old books is read to mean ‘devil’. This word is likewise used among the aforementioned \il{Pannonian}Pannonians to mean ‘bad man’ in phrases like \textit{nepriaznik}, \textit{Boha priaznik}, \textit{Boha priajaznica} etc. Similar clarifications of words belong to the dictionary, since there are some words which in one \is{Dialect}dialect have one meaning, but in another have some other somewhat related meaning. In this way, for instance, \textit{kniaz} for some means ‘prince’, for others ‘priest’. This double use is very straightforward, as the Slavs are related to the Oriental Indians both by language and by mythology. Indeed, earlier among the Indians, just as among other peoples as well, the Civil Prince was in charge of sacred matters, too, whom the Slavs called \textit{kniaz}, to which there is an analogous form \textit{kagan}, corrupted from the \il{Greek}Greek, in place of \textit{kazar}. \textit{Kazar} derives from \textit{kazati}, and it is this form which has been transmitted to posterity, but \textit{kazati} means ‘to show something, to command’, whence the substantive \textit{kniaz} ‘civil commander’, and sometimes ‘a spiritual commander’, namely if he was also in charge of sacred matters. Some derive it from \textit{konati} [‘to do; to end’]. To this day, the \il{Bohemian}Bohemians, \il{Polish}Poles, and \il{Pannonian}Pannonians call a priest \textit{kniaz}, and indeed the \il{Bohemian}Bohemians presently call the Prince \textit{Kniƶe}, which in \il{Slavic}Slavic strictly means ‘young prince’ when compared to similar [160] constructions, such as: \textit{golub}, \textit{golubje} [‘pigeon, squab’]; \textit{gus}, \textit{gusje} [‘goose, gosling’]; \textit{ƶreb}, \textit{ƶrebje} [‘stallion, foal’], alternatively \textit{golubja}, \textit{ƶrebja}, \textit{gusja}, hence also \textit{kniaz}, \textit{kniƶe}, or \textit{kniƶa} means ‘young offspring of a prince’. However, a ‘priest’ is authentically called a \textit{pop} by other Slavs, from the very \il{Old Church Slavonic}ancient \textit{popa}, \textit{papa} ‘father’, but by others \textit{sviaшчenik}, or ‘consecrated priest’. But since \textit{kniazi} or ‘commanders’ have and have had many occupations, so in the old times they needed handbooks to record what needed to be recorded, and hence \textit{kniƶa} emerged, or also with \textit{ƶ} transformed in numerous \is{Dialect}dialects into \textit{g} or into \textit{h}, hence the derivation of \textit{kniga}, \textit{kniha}, and not, as the \textit{Tripartite of Languages} notes,{\enlargethispage{\baselineskip}\footnote{\citet{klaproth_tripartitum_1820}.}} from \il{German}German \textit{knicken} [‘to snap’].

\subsection*{\hspace*{\fill}§. 3.\hspace*{\fill}}

Let us look at the original \il{Russian}Russian text for the syntax: \textit{Anibal}, \textit{Amilcarov sin}, \textit{strax Italii}, \textit{prisjaƶni neprijatel Rimlanam umoril sebja jadom} [‘Hannibal, son of Hamilcar, terror of Italy, sworn enemy of the Romans, killed himself with poison’] etc. Everyone among the Slavs will easily understand this text, as it is expressed in genuine \il{Slavic}Slavic fashion, and \textit{Anibal}, for instance, is the subject, of which the description is: \textit{Amilkarov sin}, \textit{strax Italii}, \textit{prisjaƶni neprijatel Rimlanam}. The predicate is \textit{umoril}; the object is \textit{sebja}, the instrumental \textit{jadom}. Furthermore there are examples of the possessive ending, the so-called genitive: \textit{Tvorec neba}, \textit{i zemli} [‘creator of heaven and earth’], \textit{spasitel mira} [‘savior of the world’], \textit{otec naroda} [‘father of the nation’], \textit{liubitel nauk} [‘lover of science’], \textit{dviƶenje svietel nebeskix} [‘movement of heavenly bodies’], \textit{stado koz} [‘herd of goats’], \textit{gorst soli} [‘handful of salt’], \textit{чetvertnik krup or ovsa} [‘quarter of groats’ or ‘oats’], \textit{voz sjiena}, \textit{drov} [‘cart of hay, firewood’], \textit{boчka} [161] \textit{piva} [‘barrel of beer’], \textit{loƶka masla} [‘spoon of butter’], \textit{tysjaчa duш} [‘a thousand souls’] etc.

Examples of the receptive ending, or the dative: \textit{Boƶe!} or \textit{Bog! milostiv bud} (\textit{budi}) \textit{mnie grjeшnomu} [‘God have mercy on my soul’], \textit{podoben otcu} [‘similar to the father’], \textit{raven jemu lietam} [‘equal to him in age’] etc.

Of the instrumental: \textit{vysok rostom} [‘tall in stature’], \textit{bogat milostiu} [‘with the grace of God’], \textit{velik imenem i dielom} [‘great in name and deeds’], \textit{dik nravom} ‘savage by nature’, \textit{slab zdorovjem} [‘weak in health’].

Of the prepositional: \textit{skvoz ruku} [‘through the hand’]. \textit{Iti mimo xrama, dvora} [‘Go past the church, court’], \textit{mimo goroda} [‘past the city’]. \textit{U tebja}, \textit{u sebja byti} [‘To be at your place, at home’], \textit{u nog} [‘at the feet’], \textit{u dverej} [‘at the door’]. \textit{Volosi unego ljezut} (\textit{liezo}) [‘He is losing his hair’]. \textit{Do biela svieta spati} [‘To sleep until the light of day’]. \textit{Kriчat izo vsego gorla} [‘To shout at the top of your lungs’; literally ‘to shout out of the whole throat’]. \textit{Vozderƶavati sia od vina} [‘To abstain from wine’], \textit{Bez golovi, bez tebja} [‘Lost without you’; literally ‘headless without you’]. \textit{Iti podle kogo} [‘To walk beside somebody’]. \textit{Radi Boga}, or \textit{pre Boga} [‘For God’s sake’]. \textit{Protio rjeki, na protio togo} [‘Opposite the river, opposite that’]. \textit{Pri dvorje} [‘At court’]. \textit{Ko mnie, k sebie} [‘To me, to oneself’], \textit{on okolo tridcati liet} [‘he is about 30 years old’]. \textit{Udariti o kamen} [‘To hit a stone’], \textit{o Bogu o smerti govoriti} [‘To talk about God, about death’]. \textit{Jexat v Rigu, v Moskvu, v Pragu} [‘To drive to Riga, to Moscow, to Prague’], \textit{jexat na rynok} [‘to go to market’]. \textit{Na um priti} [‘To come to mind’]. \textit{Sukno na kaftan kupiti} [‘To buy cloth for a caftan’]. \textit{Zaplatit za Brata} [‘To pay for one’s brother’]. \textit{Pod derevom leƶit} [‘To lie under a tree’], \textit{pred dom viti}, or \textit{viideti} [‘to come out in front of the house’]. \textit{Po gorlo, po шeju ve vodje} \linebreak{}\newpage{}\noindent{}[‘Up to the throat, up to the neck in water’]. These and other examples agree exactly with other \is{Dialect}dialects, as far as case rules are concerned.

\subsection*{\hspace*{\fill}§. 4.\hspace*{\fill}}

The \il{Polish}Polish \is{Dialect}dialect would be understood very easily, if it could be represented with a common way of writing, namely if the composite letters would be removed, such as \textit{cz}, \textit{sz}, \textit{dz}, \textit{rz}, and if instead of accentuated \textit{ć} the original \textit{t} would be put, as follows: \textit{Idem do domu} (\textit{idzem}) [‘I’m going home’]. [162] \textit{Co tu mas? Nic niemam} [‘What have you got there? I don’t have anything’.] \textit{Ƶona sluxala}, \textit{dietie spalo, moƶ} (\textit{mąs?}) \textit{vidial} [‘The wife listened, the child slept, the husband saw’]. \textit{Sediac usnul} [‘He fell asleep sitting’]. \textit{Moji sinove! Budte poslusni Bogu, i otcu vasemu}. \textit{Bytie nase na Svietie jest krotkie, ƶitie ludske bylo pred tim nie tak krotkie, jak teraz. Xvaliti uчinki xvalebne jest vec xvalebna} [‘My sons! Obey God and your father. Our life on earth is short, in the past, human life was not so short. Praising deeds worthy of praise is a praiseworthy thing’]. \textit{Xvalivsi pilnost musjem byti pilnim} [‘Having praised diligence, I have to be diligent’]. \textit{Ten pan ukrivdil vsistkix podanix svojix} [‘That master wronged all of his servants’]. \textit{Kup sobie konia} [‘Buy yourself a horse’]. \textit{Slovik spieva vdiecnie v ogradi} [‘The nightingale sings beautifully in the garden’]. \textit{Slovo boƶie bede tervalo na vieki} [‘The word of God will last forever’]. \textit{Pravdivi krestianie so blagoslaveni, xvala vlasnix ust smerdi} [‘True Christians are blessed, praise from one’s own mouth stinks’]. \textit{Ma viele sciestia, ale malo rozumu} [‘He has much luck, but little understanding’]. \textit{Dna tretiego Marca} [‘On the third of March’]. \textit{Ten sie nie boji, co zlego nie broji} [‘He is not afraid of anything that does no harm’]. \textit{Nie bylo nikogo v izbie} [‘There was nobody in the room’]. \textit{Bogobojni krestianin xvali pana Boga sviego} [‘The God-fearing Christian praises his Lord God’]. \textit{Kristus pan urodil sie okolo roku} (\textit{godu}) \textit{ceteri tisiacnego po stvoreniu svieta} [‘Christ the Lord was born around the year four thousand after the creation of the world’] etc. Compared with other \is{Dialect}dialects, the \il{Polish}Polish \is{Dialect}dialect changes \textit{ч} into \textit{c}, \textit{ś} into \textit{s}, \textit{ƶ} into \textit{z}, \textit{t} into \textit{ć} before \textit{e} or \textit{i}, \textit{r} into \textit{rz} especially before vowels, and \textit{d} before \textit{e}, \textit{i} into \textit{dz}, which is \textit{c}. The sound \textit{dƶ} is only known to the \il{Polish}Poles and some \il{Pannonian}Pannonians. If this would be written by the \il{Polish}Poles with rounded \textit{∂}, then the \il{Polish}Polish \is{Orthography}orthography with its composite \textit{dz} could be supplanted, and it would become very easy to read for other Slavs. Thus \textit{po∂ielam} [‘I divide’], \textit{u∂ielam} [‘I give’], \textit{∂ietie} [‘child’] instead of \textit{podźielam}, \textit{udźielam}, \textit{dźietie} (\textit{dziećie}). \textit{Nie∂ela} [‘Sunday’] instead of \textit{niedziela}. Yet the change of \textit{ś} into \textit{s}, \textit{ƶ} into \textit{z}, \textit{ч} into \textit{c} could remain, since they are cognate sounds, such as \textit{zena}, or \textit{ƶona} [‘wife’], \textit{celo}, or \textit{чelo} [‘forehead’], \textit{Cloviek}, or \textit{Чloviek} [‘human being, person’], \textit{шata} or \textit{sata} [‘robe’] etc. We hope that this way of writing, or something similar to it, [163] will be adopted by the \il{Polish}Poles as far as possible; and indeed, that the first light of the union of the \il{Slavic}Slavic \is{Dialect}dialects would shine forth from the \il{Polish}Poles. Such a union would be the most efficient and indeed the sole means for advancing the \il{Slavic}Slavic language and people, the most extensive in the world, to the highest summit of culture.

\subsubsection*{Sample of the Pannonian dialect in the Universal Style}

\il{Pannonian}\textit{Jisti vladar juƶ na smertnej loƶe ƶivot svoj konajuci pred skonanim svojim svolal sinov svojix, a jim mnoge razdilne nauki daval, medzi jinimi verejnimi naukami tato byla najglavneiшa: dal kaƶdemu po prutu do ruki, a kazal, da by jedenkaƶdi svoj prut zlomil, чto laxko jeden kaƶdi udielal: po tim skasal vsi prouti sebrati, a do vjedna sviazati, a dal kaƶdemu, da by zviazek lamal; ale ƶaden zlomiti ne mogel; na to mudri vladar, a peчlivi otec ova zlata Slova mluvil: Premili sinove! Jednotu, a svojnost milujte; neb jeli jednotu budete medzi sebu imati, nepritelji vaшi vas neovladajo, po tej nauke blagoslaviv jim na vjeki usnul.}

[‘A certain ruler was already on his deathbed, as his life reached its end, and before his final passing he called his sons, and gave them various teachings, and among these public teachings one was the most important: he placed a stick in each of their hands and said that they should snap the stick, which each of them did easily. After that he said to gather all sticks together in a bundle, and gave everybody a chance to break the whole, but nobody could snap them; then the wise leader and caring father said these golden words: “Dear sons! Love unity and individuality,\footnote{\ia{Herkel, Jan}Herkel probably intended \textit{svornost} ‘concord’ instead of \textit{svojnost} ‘individuality’.} but if you remain united amongst yourselves, your enemies will not dominate you”, and after this teaching he blessed them and fell into eternal sleep’.]

\begin{center}
    II
\end{center}

\textit{Za starego vieku byla jedna kralica, koja mala tri prelepije dievice: milicu, krasicu a mudricu; vse tri byle bogate, okrem bogatstva milica byla pokorna, krasica uctiva, a mudrica umena. One matku, a matka je liubila} [164]\textit{, i nauчavala, medzi sebu takto govorile: mile sestri: mi poidemo za muƶi na tri strani: jedna k sjeveru, druga ku vixodu, tretia ku poldniu, nezabudnimo jedna na drugu, neb smo z jednej kervi, z jednej matieri. Ove rieчi sluшuc stara kralica, jejix matuшka od radosti omladnula, vidane sve ceri чasto naшtiovala, a vse liudstvo spjevanim svim rastomilim obveselavala.}

[‘In the olden days there was a queen who had three magnificent maidens: one kind, one beautiful, and one wise; all three were rich, apart from their wealth, the kind one was demure, the beautiful one was polite, and the wise one was clever. They loved their mother, their mother loved them, and taught them, and they spoke to each other like this: “Dear sisters, we are going to get married on three sides: one to the north, the second to the east, the third to the south, let us not forget each other, as we are of one blood, of one mother”. Hearing these words, the old queen, their mother, was rejuvenated with joy, she often visited her wedded daughters, and with her lovely singing delighted all the people’.]

\begin{longtable}{ c c c c c c c c c c c c }
    \lsptoprule
    \multicolumn{12}{ c }{\il{Russian}\il{Latin}\textit{Russian alphabet, and Latin}.} \\
    \midrule
    Aa, & Бб, & Ц, & Ч, & Д, & Ee, & Ф, & Г, & Х, & І, & И, & Ы, \\
    a,	& b, & c, & \textit{č}, & d, & e, & f, & g, & h(=ch), & i, & j, & ü(y), \\
    \multicolumn{12}{ c }{\vspace*{-2.3mm}} \\
    К, & Л, & М, & Н, & О, & П, & Р, & С, & Ш, & Щ, & Т, & У, \\
    k, & l, & m, & n, & o, & p, & r, & s, & \textit{š}, & \textit{šč}, & t, & u, \\
    \multicolumn{12}{ c }{\vspace*{-2.3mm}} \\
    В, & З, & Ж, & Я, & Ћ, & Ҍ, & Ю, \\
    v,& z, & ƶ, & ja, & tj, & je, & ju. \\
    \lspbottomrule
\end{longtable}

The following works concerning \il{Slavic}Slavic literature are being edited: firstly, \textit{Osmanida}, a \il{Slavic}Slavic epos by Gundeliч in Ragusa in 1826;\footnote{\citet{gundulic_osmanida_1826}; also printed in \il{Italian}Italian, \citet{gundulic_versione_1827}. See \citet{zlatar_slavic_1995}.} secondly, \textit{Časopis museumski}, in Prague 1827;\footnote{The \textit{Časopis Společnosti Wlastenského Museum w Čechách} began publishing in 1827. The journal has published regularly since then, under the titles \textit{Časopis Českého musea} (1831--1854), \textit{Časopis Musea království Českého} (1857--1922), \textit{Časopis Národního musea} (1923--1941, 1945--1976), \textit{Časopis Národního muzea v Praze} (1977--1992), and \textit{Časopis Národního muzea} (1992--present). See \citet[121--164]{spet_historie_1977}; \citet[5--26]{zlatar_slavic_1995}.} thirdly, the \textit{Ljetopis serbski} in Buda.\footnote{The \textit{Ljetopis Matice Srpske} began publishing in 1824. The journal has published continuously since then, under the titles \textit{Novij serbskij letopis} (1837--1855), \textit{Srbskij letopis} (1855--1865), \textit{Srbski letopis} (1865--1867), \textit{Srpski letopis}, 1867--1987), and \textit{Letopis Matice srpske}, (1987--present). See \citet[348--370]{kimball_serbian_1969}.} \\

The benevolent reader will easily correct any mistakes.