\chapter{The inspiration for and reception of Jan Herkel’s Pan-Slavism}
\label{ch:Maxwell}

{\Large Alexander Maxwell}\smallskip

\noindent Victoria University of Wellington\bigskip

\section{Introduction}

\noindent The term \is{Pan-Slavism}“Pan-Slavism” first appeared in print in Jan \citeauthor{herkel_elementa_1826}’s \citeyear{herkel_elementa_1826} \textit{Elementa Universalis Linguae Slavicae} [‘Elements of a Universal Slavic Language’]. This short book, written in \il{Latin}Latin, sought to reduce the grammatical and above all \is{Orthography!Orthographic differences}orthographic diversity in the Slavic world. Herkel’s linguistic ideas won no adherents, even among his friends and associates, yet the \is{Pan-Slavism}word “Pan-Slavism” has enjoyed great success. There are innumerable books \is{Pan-Slavism}about “Pan-Slavism” and “Pan-Slavs” in many different parts of the Slavic world, though subsequent scholarship typically invests the word with a meaning quite different from what Herkel originally intended. This essay seeks to explain \ia{Herkel, Jan}Herkel’s work within the context of Hungaro-Slavic linguistic thought. What did \ia{Herkel, Jan}Herkel hope to accomplish with his grammar? What impact did his ideas have? Why did the \is{Pan-Slavism}term “Pan-Slavism” strike such a chord when \ia{Herkel, Jan}Herkel’s actual ideas did not?

\section{Jan Herkel as a national activist}

\ia{Herkel, Jan}Jan Nepomuk Herkel (1786--ca. 1853), described on his birth certificate as “Geor\-gius Hrkel” \citep[404]{treimer_johannes_1931}, and remembered in Slovak historiography as Ján Herkeľ, was born on 22 January 1786 in Vavrečka (Vavrecska), near the White Orava river, around fifteen kilometers from the Polish frontier. He spent his early childhood in his hometown. Herkel was raised Roman Catholic, but additional documentary evidence about his family is scarce.

A short story from \citeyear{herkel_prameny_1836} might provide some information about \ia{Herkel, Jan}Herkel’s ancestors. \ia{Herkel, Jan}Herkel wrote it for the literary magazine \textit{Zora} when he was about fifty years old. The story, titled \textit{Pramény} [‘springs’ in the sense of ‘fountains’], is written in the third person for two pages, then shifts to a first-person narrative with a young male narrator. The final four pages are then told from the perspective of the narrator’s mother, who tells her son about the life of his grandfather. The narrator’s grandfather, a village official, led a difficult life. Oppressive imperial taxation provoked brigandage and eventually rebellion. The grandfather fled into the woods and built a hut for shelter. His wife died, his sons were drafted into both imperial and rebel armies. After the triumph of imperial forces, the grandfather returned to his native town, which war and famine had reduced to fifty inhabitants. Even the family’s swords suffered: at the end of the story, treasured heirlooms had been reduced to cabbage-cutting knives \citep[209--215]{herkel_prameny_1836}.

Herkel’s story provides at most problematic evidence about \ia{Herkel, Jan}Herkel’s family background. The story might be autobiographical, it might be fictionalized, or it might simply be fiction. The shifting narrative voices also conceal the author’s perspective. Contemporaries found the story baffling. Josef \citeauthor{chmelensky_literatura_1836}’s (\citeyear[213]{chmelensky_literatura_1836}) laconic review was “what \ia{Herkel, Jan}Jan Herkel’s \textit{Pramena} wants to say, I know not”.

Herkel received a good education for his era. He attended a Piarist secondary school in Ružomberok (Rózsahegy) and then worked a few years as a school\-teacher in the same Piarist school. In 1813, he travelled to Pest to do a law degree, which he completed in 1816. As a student, he became good friends with \ia{Hamuljak, Martin}Martin Hamuljak (1789--1859), who subsequently played an important role in the Slovak community of Pest \citep[3]{herkel_jan_2009}. \ia{Hamuljak, Martin}Hamuljak, like \ia{Herkel, Jan}Herkel, also came from the Orava region, and also studied law.

During the decade after the completion of his law degree, the period during which he composed \textit{Elementa Universalis Linguae Slavicae}, Herkel’s biography is somewhat obscure. Ľudmila \citet[3]{herkel_jan_2009} suggests that \ia{Herkel, Jan}Herkel spent most of his time in Budapest, but Karl \citet[404]{treimer_johannes_1931} thinks he spent some time in Croatia. He did not, however, support himself by practicing law. Instead, baron József von Wenckheim (1778--1830), who held a series of important administrative positions in the Banat, hired \ia{Herkel, Jan}Herkel to tutor his son and four daughters \citep[270]{wurzbach_wenckheim_1886}.

In 1831, the year after Baron Wenckheim died, Herkel bought a house in Pest. He settled permanently in Hungary’s greatest city (\cite[62]{matovcik_jan_1961}; \cite[113]{kerecman_historia_2011}), even if \citet[3]{herkel_jan_2009} reports that he frequently visited the Vavrečka region. His death date is not known. His last public act was to sign a petition for a Slovak censor in 1842 \citep[13]{matovcik_prispevok_1964}, though we will see that he was apparently discussing public affairs with \ia{Štúr, Ľudovít}Ľudovít Štúr in the mid-1840s. Ctiboh \citeauthor{zoch_abecedny_1853}’s “Alphabetical list of Slovak authors”, published in \citeyear{zoch_abecedny_1853}, listed Herkel’s birth year with no year of death, implying that Herkel was then still alive (\citeyear[270]{zoch_abecedny_1853}; see also \cite[13]{matovcik_prispevok_1964}). A 12 April 1865 letter by Michal Godra \ia{Godra, Michal}makes clear that Herkel had died by that date \citep[63]{matovcik_jan_1961}.

\ia{Herkel, Jan}Herkel first entered public life in 1826, about nine years after completing his legal studies. In that year, he not only published \textit{Elementa Universalis Linguae Slavicae} but also collaborated with \ia{Hamuljak, Martin}Hamuljak on a pamphlet for distribution at the Hungarian parliament, then meeting for the first time since 1812. The pamphlet itself has not survived, but Augustín Maťovčík found a draft manuscript in Hamuljak’s papers, now held in Martin at the literary archive of the Matica Slovenská. In \citeyear{matovcik_neznama_1969}, \citeauthor{matovcik_neznama_1969} published a brief introduction to the text (\citeyear[223--226]{matovcik_neznama_1969}), along with a \il{Slovak}Slovak translation by \ia{Havaš, Jozef}Jozef Havaš (\citeyear[224--235]{herkel_jegyzesek_1969}).

Herkel and \ia{Hamuljak, Martin}Hamuljak’s pamphlet responded to an \citeyear{thaisz_jelentes_1825} editorial published in the prestigious Hungarian journal \textit{Tudományos Gyüjtemény} [‘Scientific Collection’]. The editorial’s authors were András Thaisz (1789--1840) and Mátyás Trattner (1745--1828), respectively the journal’s editor and publisher \citep[118--127]{thaisz_jelentes_1825}. Theisz and Trattner advocated what historians of Hungary have come to \label{sec:1.2}\is{Magyarization}call “Magyarization”: the linguistic assimilation of all inhabitants of the Kingdom of Hungary, including the Slavs, to the language and culture of the Magyars, that is, of the ethnic Hungarians.

Theisz and Trattner’s (\citeyear[118]{thaisz_jelentes_1825}) watchword “let us speak frankly of spreading the Magyar language in Magyar” rings more sonorously in the original \il{Hungarian}Hungarian [\textit{a’ Magyar nyelvnek terjesztéséről magyarul és magyarán szólljunk}] since the Hungarian word here translated as “frankly” also derives from the \il{Hungarian}Hungarian endonym.\footnote{On the Hungarian endonym, see \citet[14--21]{maxwell_everyday_2019}.} They extolled Hungarian as the \textit{Nemzeti nyelv} [‘national language’] (\citeyear[118, 124, 126]{thaisz_jelentes_1825}) characteristic of the \textit{Magyar nemzet} [‘Hungarian nation’] (\citeyear[119, 120, 125]{thaisz_jelentes_1825}) and the \textit{Magyar nemesség} [‘Magyar aristocracy’] (\citeyear[126]{thaisz_jelentes_1825}). They wanted to make \il{Hungarian}Hungarian the “diplomatic language [\textit{diplomatikia nyelv}]” (\citeyear[126]{thaisz_jelentes_1825}), as the language of state administration was usually known, and forbid “educating our children in any foreign language, unless they already know the national language perfectly [\textit{idegen nyelvre […] tanittassuk gyermekeinket}]” (\citeyear[124]{thaisz_jelentes_1825}). They predicted that Hungary’s non-Magyars, whom they characterized as \textit{nem Nemzetek} [‘not nations’] (\citeyear[118]{thaisz_jelentes_1825}), would assimilate: “clever and honest Slavs [\textit{okos ’s \il{Hungarian}betsűletes Tótok}] will become perfect Magyars, and be joyfully welcomed like true sons of the homeland” (\citeyear[125]{thaisz_jelentes_1825}). Thaisz, an active member of Budapest’s Lutheran community \citep[176, 178]{laszlo_evangelikus_2004}, even attributed the spread of “Slavic books in Hungarian land” to Catholic conspiracy. They concluded by urging their readers: “let us be Magyars [\textit{legyünk Magyarok!}]” (\citeyear[126]{thaisz_jelentes_1825}).

\citet[227]{herkel_jegyzesek_1969}, addressing Theisz and ignoring Trattner, began by asking “who is to be understood under the name ‘not-nations’?” Herkel thought the phrase referred to the \textit{Tót nemzet}, whose rights he then defended. Indeed, since so few inhabitants of Hungary spoke the \il{Hungarian}Magyar language, Herkel attacked Magyar pretentions to unique nationhood: “if the Slav nation is a not-nation, then the Magyar must be a not-not-not nation!!! [\textit{a magyar nemzet kell lennie a nem-nem-nem nemzet!!!}]” (\citeyear[228]{herkel_jegyzesek_1969}). He also accused Thaisz of hypocrisy, since, in his essay, Thaisz expressed love for his own language and sought to promote it, while complaining “that the \textit{Tótok} love and promote their language” (\citeyear[229]{herkel_jegyzesek_1969}).

Though \citet[225]{matovcik_neznama_1969} characterized Herkel’s pamphlet as “a defence of the rights of the \il{Slovak}Slovak language and nation”, Herkel’s phrase \textit{Tót nemzet} actually contains some ambiguity. \ia{Havaš, Jozef}Havaš straightforwardly translated it as \textit{národ slovenský} [‘the Slovak nation’], but Peter \citet[612]{macho_k_2001} observed that at various points in the text Herkel refers to the accusative plural \textit{Tót nemzeteket}, a phrase which in context could only mean \textit{slovanské národy} [‘Slavic nations’]. Since Herkel’s ethnonym \textit{Tót} implied ‘Slav’ when appended to plural ‘nations’, Macho found it ambiguous in the singular, preferring to render it not as “Slovak”, but as \textit{Slovania}/\textit{Slováci} [‘Slavs/Slovaks’] (\citeyear[617]{macho_k_2001}), since “the terms \textit{tót} and \textit{szláv} appear to a great extent as synonyms” (\citeyear[615]{macho_k_2001}).

In some passages of \ia{Herkel, Jan}Herkel’s tract, the difference between “Slav” and “Slovak” may indeed be hard to distinguish since, as Macho observed, Slavs in Hungary were “\textit{de facto} Slovaks” (\citeyear[616]{macho_k_2001}). Yet Herkel claimed in one passage that the inhabitants of Hungary “for the most part use the \textit{Tót} language, in one or another \is{Dialect}dialects”. Here, the singular \textit{Tót} clearly \il{Slavic}meant ‘Slavic’, since Slovaks on their own could not claim a majority of Hungary’s population. Indeed, one 1790 statistical survey of the Habsburg monarchy claimed of the Slavs that “the language of this nation is spoken here [in Hungary] in various \is{Dialect}dialects, e.g. \il{Bohemian}Bohemian, \il{Moravian}Moravian, \il{Croatian}Croatian, \il{Serbian}Serbian or Rascian, \il{Windic}Wendic, \il{Dalmatian}Dalmatian, \il{Russian}Russian and \il{Polish!Quasi-half Polish}quasi-half Polish” \citep[380]{grellman_statistische_1795}. Only together could these disparate Slavic communities pose as a majority.

The confusion surrounding \ia{Herkel, Jan}Herkel’s \il{Hungarian}Hungarian usage can also be clarified by placing Herkel’s pamphlet in its political context. Herkel’s pamphlet belongs to a popular genre of polemical writing in the early nineteenth century kingdom of Hungary. Indeed, the lively debate over Hungary’s administrative language (or languages) ultimately became so heated that several authors invoked the military metaphor of \is{Language battle}a “language battle [\textit{Sprachkampf} or \textit{Sprachenkampf}]” (\citeauthor{thomasek_sprachkampf_1841} \citeyear{thomasek_sprachkampf_1841}; \citeauthor{bekesy_nyelvbeke_1843} \citeyear{bekesy_nyelvbeke_1843}; \citeauthor{roth_sprachkampf_1842} \citeyear{roth_sprachkampf_1842}, \citeyear{roth_sprachkampf_1847}; \citeauthor{stur_sprachenkampf_1843} \citeyear[1070--1092; 1077--1078, 1088--1090; 1800--1802]{stur_sprachenkampf_1843}). Slavs from northern Hungary, lacking any administrative unit to serve as a focus for political activism, particularly emphasized linguistic rights in their political tracts (\cite[552--558]{kollar_etwas_1821}; \cite{hoitsy_sollen_1833, hoitsy_apologie_1843}; \cite{suhajda_magyarismus_1834}; \cite{stur_beschwerden_1843, stur_neunzehnte_1845}; \cite{hodza_slowak_1848}).

Literary historian Ján \citet{ormis_o_1973} collectively described such polemics \linebreak{}against \is{Magyarization}Magyarization as “Slovak national defences”, but many of the texts that Ormis depicted as “Slovak” national defences did not articulate Slovak particularist nationalism, but rather a \is{Pan-Slavism!Linguistic Pan-Slavism}linguistic Pan-Slavism. Several explicitly refer to “Slavs” in the title. Consider Jan \citeauthor{kollar_etwas_1821}’s “Something about the \is{Magyarization}Magyarization of the Slavs [\textit{Slaven}] of Hungary” (\citeyear[552--558]{kollar_etwas_1821}), \citeauthor{stur_beschwerden_1843}’s “Complaints and Accusations of the Slav [\textit{Slaven}] of Hungary” (\citeyear{stur_beschwerden_1843}), Ján \citeauthor{caplovic_slawismus_1842}’s “Slavism and Pseudomagyarism” (\citeyear{caplovic_slawismus_1842}), and Samuel \citeauthor{hoitsy_apologie_1843}’s “Apology of Hungarian Slavism” (\citeyear{hoitsy_apologie_1843}).\footnote{\citet[169--176, 515--594, 595--665]{ormis_o_1973} gives these titles as \textit{Něco o pomaďarčovaní Slovanov v Uhorsku}, \textit{Sťažnosti a žaloby Slovanov v Uhorsku na protizákoné prechmaty Maďarov}, \textit{Slovanstvo a psuedomaďarstvo}, and \textit{Apologie uhorského Slovanstva}.} \citeauthor{ormis_o_1973}’ 800-page study appeared in \citeyear{ormis_o_1973}, and thus was presumably written during the aftermath of the 1968 Soviet invasion. During the era of so-called “normalization”, perhaps Ormis, or his publisher, found it prudent to transform nineteenth-century \is{Pan-Slavism}Pan-Slavs into good Slovaks? \citep[207--211]{taborsky_czechoslovakias_1973}. In happier times, however, scholars should acknowledge that when \ia{Kollár, Jan}Kollár and \ia{Štúr, Ľudovít}Štúr wrote about \textit{Slaven}, Čaplovič and Hoitsy about \textit{Slawismus}, and Herkel about \textit{Tótok}, they were writing about “Slavs”.

Herkel circulated his pamphlet widely. He also sent a copy to \ia{Palkovič, Juraj}Juraj Palkovič (1769--1850), professor at the Lutheran gymnasium in the town now known as Bratislava, but then called Pozsony, Pressburg, or Prešporok/Prešporek. He won the support of archbishop \ia{Rudnay, Alexander}Alexander Rudnay (1760--1831) \citep[16]{macho_jazyk_2002}, a prominent Slavic prelate who would become a cardinal in 1828. Though both Palkovič and Rudnay came from Slovak northern Hungary, neither articulated Slovak particularist sentiments. On the title pages of his numerous published works, \citeauthor{palkovic_bohmisch-deutsch-lateinisches_18201821} variously described himself as \il{Slavic!Bohemian-Slavic}a “professor of the Bohemian-Sla\-vic language” (\citeyear{palkovic_bohmisch-deutsch-lateinisches_18201821, palkovic_bestreitung_1830}), or a professor of \textit{slovenský} (\citeyear{palkovic_wytah_1808, palkovic_tatranka_1832}), an ambiguous adjective arguably translatable as \il{Slovak}\il{Slavic}either “Slovak” or “Slavic”. Rudnay, meanwhile, famously declared in one of his sermons \textit{Slavus sum, et si in \il{Latin}cathedra Petri forem: Slavus ero!} [‘I am a Slav, and if I should sit in Peter’s chair, I will remain a Slav!’] (\citetalias{anon_alexander_1863}, \cite[122]{anon_alexander_1863}; \cite[156]{precechtel_ceskoslovansky_1872}).

In the late 1820s, following the publication of \textit{Elementa Universalis Linguae Slavicae} and the distribution of his pamphlet, \ia{Herkel, Jan}Herkel participated regularly in Slavic public life in the Hungarian capital. He financially supported \il{Slavic}Slavic literary works funded through subscriptions, including \citeauthor{safarik_geschichte_1826}’s \citeyear{safarik_geschichte_1826} influential \textit{Geschichte der slawischen Sprache und Literatur: Nach allen Mundarten} [‘History of the Slavic language and literature, in all \is{Dialect}dialects’] (\citeauthor{safarik_geschichte_1826} \citeyear[520]{safarik_geschichte_1826}; see also \cite[9--10]{matovcik_vzajomna_1965}), and \citeauthor{kollar_rozprawy_1830}’s \citeyear{kollar_rozprawy_1830} booklet investigating the origins of the ethnonym “Slav” (\citeyear[no page numbers]{kollar_rozprawy_1830}), both published by Buda University press. Kollár also credited \ia{Herkel, Jan}Herkel with contributing “several songs” to his \citeyear{kollar_narodinie_1835} \textit{Národinié zpiewanky čili pjsně swětské Slowákůw w Uhrách} [‘National Songbook, or Secular Songs of the Slovaks in Hungary’] (\citeyear[505]{kollar_narodinie_1835}). In 1831, Herkel participated in \ia{Hamuljak, Martin}Hamuljak’s unsuccessful attempt to found a newspaper called \textit{Budínskí Priatel} [‘The Buda Friend’] (\cite[117]{pisut_dejiny_1960}; \cite[123]{vyvijalova_snahy_1960}; \cite[63]{ruttkay_dejiny_1999}). In 1842, he signed a petition requesting the appointment of a censor specifically for \il{Slovak}Slovak books \citep[11]{matovcik_poznamky_2002}.

Significantly, \ia{Herkel, Jan}Herkel’s Slavic philanthropy extended beyond “Slovak” particularist circles. Starting in 1825, he financially supported the scholarly journal \textit{Serbske Lětopisi} [‘Serbian Chronicle’] both when it was initially published by Buda University press (\citetalias{anon_gg_1825}, \cite[entry for \textit{Budimъ}]{anon_gg_1825}), and when it was subsequently published by the newly-established Matica Serbska (\citetalias{anon_gg_1826}, \cite[entry for \textit{Peshta}]{anon_gg_1826}). Herkel also supported the literary efforts of various Serbian writers in Buda or Pest, including Jovan Pačić (1771--1849) \citep[entry for \textit{Budimъ}]{pacic_dodatak_1827}, and Dositej Obradović (1739--1811) \citep[entry for \textit{Peshta}]{obradovic_imena_1826}. Nor did Herkel confine his Slavic philanthropy to the Hungarian capital. From 1827 to 1833, Herkel subscribed to the \textit{Časopis Českého Musea}, a scholarly journal published by the Bohemian museum in Prague (\citetalias{anon_gmena_1827}, \cite[no page numbers]{anon_gmena_1827}; \citetalias{anon_gmena_1829}, \cite[135]{anon_gmena_1829}; \citetalias{anon_poznamenanj_1830}, \cite[488]{anon_poznamenanj_1830}; \citetalias{anon_pp_1832}, \cite[491]{anon_pp_1832}; \citetalias{anon_gmena_1833}, \cite[no page numbers]{anon_gmena_1833}). He also appears on the subscriber list for \il{Serbian}Serbian-Cyrillic literary journal \textit{Danica} [‘Morning Star’], which Vuk Karadžić (1787--1864) published in Vienna from 1826 to 1829 (\cite[123]{anon_imena_1826}; \cite[159]{anon_imena_1827}; \cite[no page numbers]{anon_imena_1828}; \cite[no page numbers]{anon_imena_1829}). Though Herkel apparently confined his patronage to the Habsburg lands, he supported the literary efforts not just of Slovaks, but of Serbs and Czechs, his fellow Slavs.

\enlargethispage{\baselineskip}

Herkel typically appears on such subscriber lists as a “lawyer in Pest”, and he at least once used his legal expertise to defend Slavic interests. In 1828, four Lutheran peasants from Lajoskomárom parish, in Fejér county, approached Her\-kel for help. Nearly twenty years later, Herkel shared his recollections with Ľudovít \citeauthor{stur_neunzehnte_1845} (1815--1856), who used \ia{Herkel, Jan}Herkel’s recollections in his own “national defence”, \textit{Das neunzehnte Jahrhundert und der Magyarismus} [‘The Nineteenth Century and Magyarism’] (\citeyear{stur_neunzehnte_1845}), a polemic against the Hungarian government’s assimilationist policies. Štúr apparently transposed \ia{Herkel, Jan}Herkel’s first-person dictation into a third person narrative.\footnote{\citet[25]{stur_neunzehnte_1845} wrote, for example, that “the names of the plaintiffs were Bartosch and Wrabec, he could not recall the names of the other two”.} Herkel was nearly sixty years old when he spoke to Štúr, but if his recollections wandered, Štúr edited them into a coherent narrative condemning Magyar injustice.

\begin{quote}
    Some years ago (it was in 1828) four peasants from Lajoskomárom came with a petition from their village notary […] They, the Lajoskomároers, after notifying church authorities, had elected a minister with the general consent of the parish and had even gathered the requisite paperwork, but instead another pastor whom they had not elected came to their parish […] who either would not or could not speak to them. This development greatly disturbed them, some of them understood a little \il{Hungarian}Magyar, but the others, and the womenfolk, only understood their mother language, they had always had church services in their mother tongue, but now they had gone a long time without any church service. For poor farmers, religion is the only comfort in this world, and now they were forced to travel to a distant locality for this consolation to have service on holy days, since the pastor who had suddenly arrived in their village either would not or could not speak to them. (\cite[24--25]{stur_neunzehnte_1845}; see also \cite[589--603]{kis_emlekezesei_1890})
\end{quote}

\noindent{When the peasants complained to the church authorities, they were thrown into prison, whipped, and told to abandon their “repulsive Slavic language”. This outrage not only became a stock grievance in Slovak national defences (\cite[10]{hoitsy_sollen_1833}; \citetalias{anon_recurs_1843}, \cite[199]{anon_recurs_1843}; \cite[10]{hodza_slowak_1848}; \cite[211]{vorbis_evangelisch-lutherische_1861}, \citeyear[487]{vorbis_lutherischen_1862}; \cite[4]{gerometta_altere_1876}), but even attracted the condemnation of foreign travelers \citep[116]{mackenzie_across_1862}.}

\ia{Herkel, Jan}Herkel told Štúr that he initially found the story hard to believe, but, once persuaded that the peasants had done nothing to provoke such treatment, agreed to help. Shared nationality and love of a shared language overcame confessional difference between the Catholic Herkel and Protestant peasants. Herkel eventually won a royal resolution granting the peasants the right to a chaplain who knew the \il{Slavic}Slavic language [\textit{der slavischen Sprache kundig}]. Local officials in Fejér county, however, still tried to punish the peasants for disturbing the peace and disrespecting Hungary’s “national language”, adding that “nothing good comes from language confusion and the country’s happiness depended on unity of language”. County officials also tried to censure Herkel for his involvement with the case \citep[26--27]{stur_neunzehnte_1845}.

Herkel thus participated in Slavic national life through a wide variety of public acts. He co-wrote a “national defence” in 1826, and in 1845 provided material for Štúr’s. He financially supported several literary initiatives, and personally wrote a short story. He used his legal expertise to help the Lajoskomárom peasants, who presumably could not pay handsomely. Apart from these political activities, however, Herkel also participated in Slavic debates about \is{Linguistic reform}linguistic reform. Before turning to \textit{Elementa Universalis Linguae Slavicae}, his most important work, let us consider the literary and linguistic context which Herkel’s proposals addressed.

\section{\textit{Zora} as a window on Slavic literature in Herkel’s Hungary}

During his later middle age, Herkel took an active interest in the literary cultivation of his native \il{Slavic}Slavic. In 1834, he became a founding member of a literary society variously known to its contemporaries as the \textit{Spolok milowňíkow reči a literatúry slowenskég}, the \textit{Cultorum Linguae et Literaturae Slavicae Unio} or the \textit{Societas cultorum literaturae Slavicae}, but remembered in modern \is{Orthography}orthography as the \textit{Spolok milovníkov reči a literatúry slovenskej} \citep[209]{urhegyi_almanach_1984}. \ia{Hamuljak, Martin}Hamuljak was the Spolok’s founder and \textit{unionis director}. The treasurer, legal professor \ia{Ottmayer, Anton}Anton Ottmayer (1796--?), was, like Herkel and Hamuljak, a Slovak Catholic who had studied law at Buda University; he had completed his studies the year after \ia{Herkel, Jan}Herkel. Another prominent member of the Spolok, Lutheran pastor \ia{Kollár, Jan}Jan Kollár (1793--1852), an important community leader, noted poet, and active polemicist, was the most famous Slav living in the Hungarian capital \citep[68--79]{kacirek_community_2016}. Lutheran participation again illustrates the pan-confessional quality of Herkel and Hamuljak’s literary aspirations.

The Spolok’s most visible product was the literary almanac \textit{Zora} [‘Dawn’], initially inspired by the example of the \il{Hungarian}Hungarian-language literary almanac \textit{Aurora}, first published by Trattner in 1822 \citep{kisfaludy_aurora_1822}. Four volumes appeared between 1835 to 1840. \textit{Zora} had a print run of 500, though the relatively high price (2--3 Gulden) meant that many copies went unsold \citep[212, 226--227]{urhegyi_almanach_1984}. Herkel’s only contribution to \textit{Zora} was \textit{Pramény}, the possibly autobiographical description of the Orava region discussed above. Nevertheless, as an ardent Slav and Hamuljak’s friend, \ia{Herkel, Jan}Herkel apparently provided additional help behind the scenes. In 1835, for example, \ia{Kollár, Jan}Kollár sent Herkel a brief letter praising the almanac; \ia{Kollár, Jan}Kollár’s letter was probably intended as a thank-you note \citep[144]{ambrus_listy_1991}.\footnote{“Letter 104 (Kollár to Herkel) 12 August 1835”, in: \citet[144]{ambrus_listy_1991}.}

\textit{Zora}, like many \il{Slavic}Slavic publications of its era, lacked \is{Orthography}orthographic consistency, and might be taken as representative of orthographic conditions in Hungary’s Slavic literature in the early nineteenth century. Authors of different confessions represented different literary traditions. Even within a given tradition, however, additional diversity existed. Since the diversity of literary conventions inspired Herkel’s \textit{Elementa Universalis Linguae Slavicae}, a survey of \textit{Zora}’s \is{Orthography}orthography provides insight into Herkel’s literary activity. A detailed analysis of \textit{Zora} provides a microcosm of the linguistic challenges that Slavic literati faced in the early nineteenth century.

Catholic contributors to \textit{Zora}, including \ia{Herkel, Jan}Herkel and the priest-poet \ia{Hollý, Ján}Ján Hollý (1785--1849), generally followed the \is{Orthography}orthographic conventions set down in Anton \linebreak{} \citeauthor{bernolak_grammatica_1790}’s \citeyear{bernolak_grammatica_1790} \il{Latin}Latin-language grammar, which also appeared in \il{German}German translation in \citeyear{bernolak_schlowakische_1817}.\footnote{See also \citet{bernolak_dissertatio_1787} and his posthumously-published, multi-volume dictionary (\citeyear{bernolak_slowar_18251827}).} Bernolák (1762--1813), a Catholic priest, was born in Slanica (Szlanica), not far from \ia{Herkel, Jan}Herkel’s home town. Bernolák’s orthography, variously called \il{Bernoláčtina| see {Bernolákovčina}}\il{Bernolákovčina}\is{Orthography!Bernolákovčina tradition}\textit{Bernolákovčina} or \textit{Bernoláčtina}, inspired a voluminous yet mostly confessional literature dominated by prayer guides, catechisms, and sermons (\cite{kotvan_bibliografia_1957}; see also \cite[85--88]{maxwell_choosing_2009}).

Lutheran contributors to \textit{Zora}, such the physician \ia{Sucháň, Martin}Martin Sucháň (1792--1841) or \ia{Godra, Michal}Michal Godra (1801--1874), belonged to a different orthographic tradition ultimately derived from the Králice Bible, often \il{Bibličtina}\is{Orthography!Bibličtina tradition}called \textit{Bibličtina}. The Králice Bible, composed in the eponymous Moravian town and published in Halle and Prague between 1579 and 1593, also enjoyed several reprintings in Bratislava around the turn of the nineteenth century \citep[175--193]{pisna_druhy_2013}. Hungary’s Lutheran Slavs published liturgical works in this tradition, but also produced a broader range of poetry and \textit{belles lettres} than their Catholic counterparts.

Publications from the \il{Bernolákovčina}\is{Orthography!Bernolákovčina tradition}\textit{Bernolákovčina} and \il{Bibličtina}\is{Orthography!Bibličtina tradition}\textit{Bibličtina} traditions are easily distinguished by their orthographies. Texts from the Lutheran \il{Bibličtina}\is{Orthography!Bibličtina tradition}\textit{Bibličtina} tradition use several letters which \ia{Bernolák, Anton}Bernolák rejected, most notably \{ě, ř, ů\}. When examining \il{Slovak}a “Slovak” text published in \ia{Herkel, Jan}Herkel’s time, therefore, the presence or absence of the letters \{ě, ř, ů\} provides some indication of the author’s religion.

Other \is{Orthography}orthographic controversies visible in \textit{Zora}, however, have no confessional significance. Sucháň and Godra were both Lutherans, and both used the letters \{ě, ř, ů\} (\cite[157--166]{szuhany_odplata_1835}; \cite[268--281]{godra_basne_1835}), but where Sucháň used the letters \{j, w\}, Godra preferred \{í, v\}. \ia{Hollý, Ján}Hollý and \ia{Herkel, Jan}Herkel were both Catholic, both eschewed \{ě, ř, ů\}, and thus both belonged to the \il{Bernolákovčina}\is{Orthography!Bernolákovčina tradition}\textit{Bernolákovčina} tradition (\cite[5--70]{holly_selanki_1835}; \cite[209--215]{herkel_prameny_1836}), but Hollý used \{w\} where Herkel used \{v\}. Such differences reflect a generational shift. Both \il{Slovak!Modern Slovak}modern Slovak and \il{Czech!Modern Czech}modern Czech use \{v\} in places where \citeauthor{dolezal_grammatica_1746}’s \citeyear{dolezal_grammatica_1746} grammar and \citeauthor{bernolak_grammatica_1790}’s \citeyear{bernolak_grammatica_1790} grammar used the letter \{w\}, or more accurately the letter \{{\Blackletter w}\}, printed in a blackletter typeface. The shift from blackletter type to Latin type, and from \{w/{\Blackletter w}\} to \{v\}, no longer arouses passions in either Slovakia or Czechia. The letters \{ě, ř, ů\}, by contrast, remain important \is{Shibboleth}shibboleths: all three appear in \il{Czech!Modern Czech}modern standard Czech and are absent in \il{Slovak!Modern Slovak}modern standard Slovak. They have ceased to signify the confessional difference between Catholic \il{Bernolákovčina}\is{Orthography!Bernolákovčina tradition}\textit{Bernolákovčina} and Lutheran \il{Bibličtina}\is{Orthography!Bibličtina tradition}\textit{Bibličtina}; they now signify the national difference between Slovaks and Czechs.

In terms of the \is{Shibboleth}shibboleths discussed above, \ia{Herkel, Jan}Herkel qualifies as part of the Berno\-lákovčina tradition because he eschewed \{ě, ř, ů\}. He also sided with the younger generation by using \{v\} instead of \{w\}. Nevertheless, Herkel’s own Slavic diverged significantly from \ia{Bernolák, Anton}Bernolák’s rules, at least to judge by the only text he ever published in \il{Slavic}Slavic, the aforementioned short story \textit{Pramény}. Katarína \citet[102]{habovstiakova_almanach_1970}, in a detailed linguistic analysis of \textit{Zora} contributors, concluded that “Herkel deviated from \il{Bernolákovčina}\is{Orthography!Bernolákovčina tradition}Bernolákovčina”, while Josef \citet[197]{vavro_jan_1961} characterized the story as “a sort of mixture of \ia{Herkel, Jan}Herkel’s native \is{Dialect}dialect, \il{Bernolákovčina}\is{Orthography!Bernolákovčina tradition}Berno\-lákovčina, and \il{Czech}Czech”. Emília \citet[222]{urhegyi_almanach_1984}, meanwhile, detected so many regionalisms that she argued that the story was “not in \il{Bernolákovčina}\is{Orthography!Bernolákovčina tradition}Bernoláčtina at all, but instead written according to the linguistic usage of his hometown (Upper Orava)”.

\ia{Herkel, Jan}Herkel’s short story did not establish a new literary standard. Nothing justifies Fráňo \citeauthor{ruttkay_dejiny_1999}’s (\citeyear[65]{ruttkay_dejiny_1999}) assertion that Herkel “returned to philological problems in 1836 in the almanac Zora […] in which in the article ‘Premena’ he attempted to create an individual \il{Slovak}Slovak written language different from \ia{Bernolák, Anton}Bernolák’s linguistic norms”. Instead, \ia{Herkel, Jan}Herkel’s personal \is{Idiosyncrasy}idiosyncrasies suggest that the literary standards circulating in northern Hungary were only weakly established, even in the minds of ardent patriot intellectuals.

\textit{Zora}’s \is{Orthography}mixed orthography attracted the criticism of contemporaries. Writing in the \textit{Časopis Českého museum}, Bohemian poet and translator Josef \citeauthor{chmelensky_literatura_1836} (1800--1839) scathingly reviewed the first two volumes as follows:

\begin{quote}
    Is \textit{Zora} written in \il{Czech}Czech? – It isn’t. – \il{Slovak}Slovak? – Not at all. – \il{Polish}Polish, \il{Croatian}Croatian or maybe \il{Russian}Russian? – not at all. – Which language [\textit{řeč}] is it? – Please don’t ask. We in Bohemia don’t know, but it is not written in \il{Czech}Czech or in any other \il{Slavic}Slavic language [\textit{gazyk}] that we know. It is a mixture of various gentlemen who would, I suspect, like to deafen us; gentlemen who perhaps do not know what they want, who with their unnatural love harm their mother language [\textit{řeč}] more than they could with their worst hatred; gentlemen, who have no native language [\textit{řeč}], grammar or \is{Orthography}orthography. (\cite[208]{chmelensky_literatura_1836}; see also \cite[144]{matovcik_martin_1971})
\end{quote}

\noindent{When citing individual poems or stories, \citet[211]{chmelensky_literatura_1836} signaled his \is{Orthography}orthograph\-ic disapproval by appending “(sic)” to non-standard spellings as often as seven times per page. While Chmelenský’s haughtiness surely reflects his individual pomposity, Slavs from northern Hungary could be forgiven if they experienced it as Czech arrogance.}

Nevertheless, numerous scholars, following \ia{Chmelenský, Josef}Chmelenský, have tried to analyze \textit{Zora}’s \is{Orthography!Orthographic differences}orthographic and linguistic diversity through the analytical \il{Slovak}\il{Czech}categories “Slovak” and “Czech”, without concern that these contemporary ethnonyms may mislead when applied to a text from the early nineteenth century. Such scholars routinely classify \il{Bernolákovčina}\is{Orthography!Bernolákovčina tradition}\textit{Bernolákovčina} as a form \il{Slovak}of “Slovak”, and essentially interpret \il{Bibličtina}\is{Orthography!Bibličtina tradition}\textit{Bibličtina} texts \il{Czech}as “Czech”. Slovak historian Dušan \citet[103]{kovac_dejiny_1999}, for example, wrote that \textit{Zora} “published contributions in \il{Czech}Czech and \il{Bernolákovčina}\is{Orthography!Bernolákovčina tradition}Bernolákovčina”. Mária \citet[53]{vyvijalova_spolok_1970} described \textit{Zora} as a platform for “cultural collaboration between \ia{Bernolák, Anton}Bernolák’s supporters and supporters of \il{Czech}Czech”. Augustín \citet[18]{matovcik_vzajomna_1965} thought that \ia{Hamuljak, Martin}Hamuljak accepted contributions in the \il{Czech}Czech and \il{Bernolákovčina}\is{Orthography!Bernolákovčina tradition}Berno\-lákovčina orthographies”. Other scholars characterized the \il{Bibličtina}\is{Orthography!Bibličtina tradition}\textit{Bibličtina} tradition \il{Czech!Biblical Czech| see {Bibličtina}}\il{Bibličtina}as “Biblical Czech”, signaling its archaic quality but still associating it with the Czech ethnonym. Milan \citet[124]{cechvala_michal_1970}, for example, wrote of \textit{Zora} that “the \is{Intelligentsia!Lutheran intelligentsia}Lutheran intelligentsia wrote in \il{Bibličtina}Biblical Czech, and a substantial part of the \is{Intelligentsia!Catholic intelligentsia}Catholic intelligentsia used \il{Bernolákovčina}\is{Orthography!Bernolákovčina tradition}Bernolákovčina”. Even though \citet[224]{urhegyi_almanach_1984} analyzed \textit{Zora}’s \is{Orthography}orthography in terms of “the dual literary languages, \il{Bibličtina}\is{Orthography!Bibličtina tradition}Bibličtina and \il{Bernolákovčina}\is{Orthography!Bernolákovčina tradition}Bernoláčtina”, she still glossed \il{Bibličtina}\textit{Bibličtina} as “the \il{Bibličtina}Czech of the Králice Bible” (\citeyear[221]{urhegyi_almanach_1984}).

Treating \il{Bibličtina}\is{Orthography!Bibličtina tradition}\textit{Bibličtina} \il{Czech}as “Czech” implies that it is \il{Slovak}not “Slovak”, which within the context of Slovak national historiography implicitly denationalizes Slavic Hungary’s influential \is{Intelligentsia!Lutheran intelligentsia}Lutheran intelligentsia. Indeed, some modern Slovak scholarship explicitly refuses to acknowledge \il{Bibličtina}\is{Orthography!Bibličtina tradition}\textit{Bibličtina} contributions as part of \il{Slovak}Slovak literature. Diaspora Slovak literary historian Peter \citet[67]{petro_history_1995}, for example, adduced \textit{Zora} as evidence that “Catholics continued to publish in \ia{Bernolák, Anton}Bernolák’s \linebreak{}\il{Slovak}Slovak […] since they did not suffer from \il{Czech}the ‘Czech complex’”. Stanislav \citeauthor{kirschbaum_z_2010} (\citeyear[49]{kirschbaum_z_2010}, \citeyear[48]{kirschbaum_historical_2014}), focusing on the Spolok rather than \textit{Zora}, claimed that its function was “to support the \il{Slovak}Slovak language \is{Codification}codified by Anton Berno\-lák”, and though he admitted that the Spolok published works from “those who preferred to use literary \il{Czech}Czech”, he insisted that “supporters of the latter left the association in 1835”. The 1840 volume of \textit{Zora}, however, includes several works from the \il{Bibličtina}\is{Orthography!Bibličtina tradition}\textit{Bibličtina} tradition, complete with \{ě, ř, ů\}, not only from Kollár but also from the comparatively obscure Lutheran poet Ludovit Želo (1809--1873) (\cite[7--10, 275--80]{kollar_puwod_1840}; \cite[23--24]{zelo_slawy_1840}). Mária \citet[388]{zsilak_budai_2017}, finally, wrote that \textit{Zora} demonstrated “the flowering of \il{Bernolákovčina}\is{Orthography!Bernolákovčina tradition}Bernolákovčina literature”, evidently ignoring the \il{Bibličtina}\is{Orthography!Bibličtina tradition}\textit{Bibličtina} contributions entirely.

Why scholars would seek to excise \il{Bibličtina}\is{Orthography!Bibličtina tradition}\textit{Bibličtina} literary works from the Slovak literary heritage is unclear. Perhaps some ardently Catholic scholars hope to depict their religious traditions as more authentically national than those of their Lutheran rivals? If so, it may be worth remembering that Slavic Lutherans in Hungary had been cultivating, \is{Codification}codifying and debating Slavic grammatical and literary conventions since the Reformation (\cite{masnicius_zprawa_1696}; \cite{dolezal_grammatica_1746}; \cite{palkovic_bohmisch-deutsch-lateinisches_18201821}). The \il{Bibličtina}\is{Orthography!Bibličtina tradition}\textit{Bibličtina} tradition could also claim \citeauthor{kollar_slawy_1824}’s poetry, including the literary sensation \textit{Sláwy dcera} [‘The daughter of Sláwa’], an epic poem first published in \citeyear{kollar_slawy_1824}, and later expanded in 1832. In \ia{Herkel, Jan}Herkel’s day, \il{Bibličtina}\is{Orthography!Bibličtina tradition}\textit{Bibličtina} could boast a much longer and more impressive literary tradition in Slavic Hungary than the relatively upstart and predominantly confessional literature written in \il{Bernolákovčina}\is{Orthography!Bernolákovčina tradition}\textit{Bernolákovčina}.

Though orthographic \is{Shibboleth}shibboleths \{ě, ř, ů\} easily sort individual \textit{Zora} contributors into the \il{Bernolákovčina}\is{Orthography!Bernolákovčina tradition}\textit{Bernolákovčina} or \il{Bibličtina}\is{Orthography!Bibličtina tradition}\textit{Bibličtina} traditions, a more nuanced analysis problematizes a \is{Slovak/Czech binary}binary dichotomy. \citet[222]{urhegyi_almanach_1984} found that some contributions to Zora “are not in any literary language, but in the \is{Dialect}dialect of upper Orava county”, specifically characterizing various \is{Dialect}dialectical elements in \ia{Herkel, Jan}Herkel’s contribution as “very strange” (\citeyear[222]{urhegyi_almanach_1984}). By denigrating non-standard texts as \is{Dialect}dialectical and “strange”, Űrhegyi, following \ia{Chmelenský, Josef}Chmelenský, reveals a somewhat unrealistic expectation that Slovak literati should have mastered the fine conventions of a \is{Codification}literary codification without the benefit of extensive schooling in its peculiarities.

In practice, however, few texts from the early nineteenth century, whether from the \il{Bibličtina}\is{Orthography!Bibličtina tradition}\textit{Bibličtina} or \il{Bernolákovčina}\is{Orthography!Bernolákovčina tradition}\textit{Bernolákovčina} traditions, conformed to any literary standard, much less to literary \il{Slovak}Slovak or \il{Czech}literary Czech as subsequently \is{Codification}codified. As evidence that the \is{Orthography!Orthographic differences}orthographic differences between the \il{Bibličtina}\is{Orthography!Bibličtina tradition}\textit{Bibličtina} and \il{Bernolákovčina}\is{Orthography!Bernolákovčina tradition}\textit{Bernolá\-kovčina} traditions do not correspond to the standard national \is{Codification}codifications of subsequent centuries, consider a pair of proverbs. The top two rows of \hyperref[tab:Table 1.1]{Table 1.1} show how they appeared in Pavel \citeauthor{dolezal_grammatica_1746}’s \citeyear{dolezal_grammatica_1746} \il{Bibličtina}\is{Orthography!Bibličtina tradition}\textit{Bibličtina} grammar and in \citeauthor{bernolak_grammatica_1790}’s \citeyear{bernolak_grammatica_1790} grammar. The bottom two rows show the same proverb in twen\-tieth-century standard \il{Czech}Czech and \il{Slovak}Slovak \is{Orthography}orthographies.\footnote{Documenting the second proverb \il{Czech!Modern Czech}in “modern Czech” poses difficulties; most twentieth-century versions read \textit{kdo mlčí, souhlasí}. The version here comes from an 1867 translation of a \il{Polish}Polish short story. The other “modern” versions are from the twentieth century. \citeauthor{dolezal_grammatica_1746} (\citeyear[286]{dolezal_grammatica_1746} (\textit{wľas}), 279 (\textit{mlčj})); \citeauthor{bernolak_grammatica_1790} (\citeyear[311]{bernolak_grammatica_1790} (\textit{Wlaſ}), 297 (\textit{mlčí})); \citeauthor{hanusem_clovekoslovi_1867} (\citeyear[167]{hanusem_clovekoslovi_1867} (\textit{vlas})); \citeauthor{flajshans_ceska_1911a} (\citeyear[770]{flajshans_ceska_1911b} (\textit{vlas}), \citeyear[967]{flajshans_ceska_1911a} (\textit{mlčí})); \citeauthor{melichercik_slovenske_1953} (\citeyear[258]{melichercik_slovenske_1953} (\textit{vlas}), 90 (\textit{mlčí})).}

\newpage

\begin{table}
    \centering
    \small
    \caption*{Table 1.1: Two proverbs as they appear in different \is{Orthography}orthographic conventions.}
    \label{tab:Table 1.1}
    \begin{tabular}{c c c}
        \lsptoprule
         & “Each hair bristles differently” & “Who stays silent, consents” \\
         \midrule
        \vspace*{1mm} \citeauthor{dolezal_grammatica_1746} (\citeyear{dolezal_grammatica_1746}\il{Bibličtina}\is{Orthography!Bibličtina tradition}) & \multirow{4}{*}{\includegraphics[width=4cm]{figures/maxwell_1-4.png}} & \multirow{4}{*}{\includegraphics[width=4cm]{figures/maxwell_5-8.png}} \\
        \vspace*{0.8mm} \citeauthor{bernolak_grammatica_1790} (\citeyear{bernolak_grammatica_1790}\il{Bernolákovčina}\is{Orthography!Bernolákovčina tradition}) & & \\
        \vspace*{0.8mm} \il{Czech!Modern Czech}Modern Czech & & \\
         \il{Slovak!Modern Slovak}Modern Slovak & & \\
        \lspbottomrule
        \end{tabular}
\end{table}

All four \is{Orthography}orthographies differ, but readers can also see for themselves that all four versions also resemble each other closely. The differences between the eighteenth-century versions, furthermore, do not always foreshadow those of the subsequent \il{Slovak}Slovak/\il{Czech}Czech \is{Slovak/Czech binary}binary. Doležal’s usage indeed resembles \il{Czech!Modern Czech}modern Czech and differs from \ia{Bernolák, Anton}Bernolák and \il{Slovak!Modern Slovak}modern Slovak by using the letters \{ř, ů\} and the reflexive \textit{se} instead of \textit{sa} (or {\Blackletter ſa}). However, \ia{Doležal, Pavel}Doležal also resembles Bernolák in using blackletter type and \{g, w\} in place of modern \{j, v\}. \il{Slovak!Modern Slovak}Modern Slovak, furthermore, has some unique features not present in Bernolák: the letter \{ô\}, the short vowel at the end of the word \textit{priznáva}, and the unvoiced medial consonant in the word \textit{kto}. The blackletter typeface, furthermore, makes the eighteenth-century versions resemble each other. If we classify Doležal’s \textit{Grammatica Sla\-vico-Bohemica} as “Czech” and Bernolák’s \textit{Gramatica Slavica} as “Slovak”, then the differences between “Slovak” and “Czech” seem less striking than the differences within \il{Slovak}the “Slovak” versions and within \il{Czech}the “Czech” versions.

Nevertheless, modern scholars find it hard to escape the analytical \il{Slovak}categories “Slovak” \il{Czech}and “Czech”, even when they transcend the \il{Bernolákovčina}\is{Orthography!Bernolákovčina tradition}\il{Bibličtina}\is{Orthography!Bibličtina tradition}\il{Bernolákovčina}\textit{Bernolákovčina}/\textit{Bibličtina} dichotomy and accept works in the \il{Bibličtina}\is{Orthography!Bibličtina tradition}\textit{Bibličtina} tradition as part \il{Slovak}of “Slovak” literature. Konstantin Lifanov, for example, judged both \il{Bernolákovčina}\is{Orthography!Bernolákovčina tradition}\textit{Bernolákovčina} and \il{Bibličtina}\is{Orthography!Bibličtina tradition}\textit{Bibličtina} contributions to \textit{Zora} as fundamentally similar, and thus as fundamentally \il{Slovak}Slovak:

\begin{quote}
    [T]hanks to the penetration of Pan-Slovak elements into texts written “in \il{Czech}Czech”, the strengthening of the Central \il{Slovak}Slovak elements in Berno\-lákovčina, and the penetration of Central Slovak elements into \il{Czech}both “Czech” \is{Orthography!Bernolákovčina tradition}and Ber\-nolákovčina, these texts acquire many common features. \citep[41]{lifanov_almanach_2010}
\end{quote}

\noindent \citet[95]{habovstiakova_almanach_1970}, by contrast, emphasized overall diversity. She depicted each Catholic contributor as using his own individual “variant of \il{Bernolákovčina}\is{Orthography!Bernolákovčina tradition}Bernolákov-\linebreak{}čina”, and while she characterized several \il{Bibličtina}\is{Orthography!Bibličtina tradition}\textit{Bibličtina} contributions \il{Czechoslavic}as “Czecho-\linebreak{}slavic” (\citeyear[105]{habovstiakova_almanach_1970}), she described \ia{Kollár, Jan}Kollár’s contributions both \il{Slovak!Old Slovak}as “Old Slovak” and as coming “from the tradition of \is{Orthography}Czech orthography” (\citeyear[103]{habovstiakova_almanach_1970}). Whether the linguistic features of \textit{Zora} contributors are fundamentally similar or fundamentally different, of course, is a matter of opinion: Lifanov chose to emphasize similarity, and Habovštiaková chose to emphasize \is{Orthography!Orthographic differences}difference. Yet while both \ia{Lifanov, Konstantin}Lifanov and Habovštiaková accepted that \textit{Zora}’s \is{Orthography!Orthographic differences}orthographic diversity transcends a neat \il{Slovak}\il{Czech}\is{Slovak/Czech binary}Slovak/Czech binary, they both nevertheless invoked that binary by describing \il{Bibličtina}\is{Orthography!Bibličtina tradition}\textit{Bibličtina} \il{Czech}as “Czech”.

In practice, \is{Orthography!Orthographic differences}orthographic differences in the 1830s did not correspond to subsequent national categories. To classify early nineteenth-century texts in terms of those categories is therefore anachronistic. Habovštiaková’s \il{Czechoslavic}label “Czechoslavic” probably sheds more light into \textit{Zora} contributors than the \il{Slovak}\il{Czech}labels “Slovak” or “Czech”, precisely because the latter categories have subsequently become so important. Scholars could, perhaps, even better avoid anachronism with a description \il{North Hungarian Slavic}like “North Hungarian Slavic”.

The \is{Orthography!Orthographic differences}orthographic diversity of \textit{Zora} not only problematizes the analytical dichotomy \il{Slovak}\il{Czech}between “Slovak” and “Czech”, it also calls into question the utility of the category “Slovak”. The category “Slovak” is deeply entrenched in modern scholarly thinking, since for more than a century social, economic, legal and cultural institutions have posited “Slovak” as a distinct ethno-linguistic category. Twenty-first century observers may struggle to dispense with the notion of “Slovak” when considering that region on the \is{Dialect!Slavic dialect continuum}Slavic dialect continuum which now corresponds to the Slovak Republic, and which in Herkel’s lifetime formed the northern counties of the Kingdom of Hungary. The importance and success of Slovak institutions, however, underscores the transformative differences between \ia{Herkel, Jan}Herkel’s day and the twenty-first century. An overview of such transformations may prove instructive.

The category “Slovak” was not always so well-established. In \citeyear{bauer_nationalitatenfrage_1907}, Austrian Social Democrat Otto Bauer described the Slovaks as a “nation without history” \citep[188]{bauer_nationalitatenfrage_1907}. The phrase alluded to the lack of independent statehood, implying that Slovaks had no distinct political history, at least, not since the short-lived medieval Kingdom of Great Moravia, which had collapsed more than a thousand years previously.\footnote{The final collapse of Great Moravia cannot be dated precisely, but probably occurred between 903 and 904. See \citet[69]{spinei_great_2003}; \citet[69]{kouril_magyars_2019}.} Since 1907, however, Slovakia has not only acquired a political history, but a history that has undergone several phases. After 1918, the government of interwar Czechoslovakia established Slovakia as a distinct administrative unit, a step which may be more important than statehood in establishing distinct “languagehood” \citep[40]{maxwell_taxonomies_2015}. The years between 1939 and 1945 witnessed a new Slovak state, characterized by historian Marián Mark \citet[xxiii]{stolarik_introduction_2010} as “a semi-independent Slovak republic, backed by Germany”, and somewhat less charitably by Stefan \citet[131]{auer_liberal_2004} as “a Nazi puppet state”. Communist Czechoslovakia declared itself a federal state in 1969, granting further weight to Slovakia as an administrative unit \citep[444--467]{kirschbaum_federalism_1977}. In 1992, when federal Czechoslovakia dissolved, a renewed Slovak republic not only gained political independence, but became, in the memorable phrase of Rogers \citeauthor{brubaker_national_1995} (\citeyear[107--132]{brubaker_national_1995}, \citeyear[411--437]{brubaker_nationalizing_1996}, \citeyear[1785--1814]{brubaker_nationalizing_2011}), a “nationalizing state”.

Many of these “Slovak” administrative structures have bestowed an official status to a unique literary standard, usually characterized \il{Slovak}as “the Slovak language”, which uses several distinctive letters that do not occur in \il{Czech}literary Czech, namely \{ä, ľ, ĺ, ô, ŕ\}. The First Czechoslovak Republic admittedly postulated a \il{Czechoslovak}single “Czechoslovak language” encompassing both a “Czech version [\textit{české znění}]” and a somewhat disadvantaged “Slovak version [\textit{slovenské znění}]” \citep[20--22]{horacek_jazykove_1928}. Interwar Czechoslovakia nevertheless introduced the visually distinct “Slovak version” into government administration, courts, and schools. In 1938, the First Slovak Republic elevated this literary standard from a “version” of \il{Czechoslovak}Cze\-choslovak to “the exclusive state language in the Slovak Republic [\textit{vylučným statnym jayzkom v slovenskej republiky}]” \citep[209--210]{janosik_na_1938}. The 1968 Czechoslovak constitution further declared that the “the \il{Czech}Czech language and the \il{Slovak}Slovak language enjoy” equality under the law (\textit{Ústavní zákon ze dne 27. října 1968}, \citeyear[§6(1), 382]{noauthor_ustava_1968}). Under the 1992 Slovak constitution, \il{Slovak}literary Slovak enjoys the status of “state language [\textit{štátnym jazykom}]” (\textit{Ústava Slovenskej republiky}, \citeyear[§6(1)]{noauthor_ustava_1992}).

The governments that bestowed official status on a \il{Slovak}distinctively “Slovak” literary standard have also created institutions to \is{Codification}codify and promote it. A scholarly journal devoted to the cultivation of the Slovak language, \textit{Slovenská reč}, began publishing in 1932. The Linguistic Institute of the Slovak Academy of Sciences, founded on 1 April 1943, has lent the journal its prestige since September 1950, when it took over the journal’s publication.\footnote{On the shift in ownership, cf. \citetalias{anon_notitle_19491950}, \citet[192]{anon_notitle_19491950}; \citetalias{anon_notitle_19501951}, \citet[32]{anon_notitle_19501951}.} Since 1953, the Slovak Academy of Sciences also boasts an institute for the study of Slovak literature; its journal, \textit{Slovenská literatúra}, first appeared in 1954 (\citetalias{anon_uvodom_1954}, \cite[3--4]{anon_uvodom_1954}).

Elite institutions devoted to \is{Codification}\is{Standardization| see {Codification}}standardized \il{Slovak}Slovak have helped establish a mass education system that trains children to read and write according to its conventions. Indeed, institutional efforts to separate literary \il{Slovak}Slovak from literary \il{Czech}Czech began under the Habsburgs: \ia{Hattala, Martin}Hattala’s comparative grammar of Slovak and Czech was used a school textbook as early as 1857, though not in very many schools \citep[507]{lindner_cechisches_1873}. In the second half of the nineteenth century, leading \il{Slovak}Slovak literary figures had to take great pains to write in conformity to the newly \is{Codification}codified standard: \ia{Hurban, Svetozár Miloslav}Svetozár Miloslav Hurban, who wrote under the pen-name Svetozár Hurban-Vajanský, meticulously corrected his own poetry from draft to draft; one interwar Slovak educator characterized Hurban-Vajanský’s small refinements \is{Dialect}as “correcting the dialectical”.\footnote{Sample changes: \textit{Dvatsať päť} > \textit{Dvadsaťpät} [‘twenty-five’], \textit{Chéf} > \textit{Šéf} [‘boss’], \textit{prísnokárný} > \textit{prísnokárny} [‘strict, stern’], \textit{ubierám} > \textit{uberám} [‘I harvest’], \textit{hladievam} > \textit{hľadievam} [‘I’m looking’], \textit{módných} > \textit{módnych} [‘fashionable’]. See \citet[20]{krusinsky_prispevky_1928}.} By the end of the Czechoslovak era, however, all Slovak children underwent nine years of compulsory education using a \il{Slovak}Slovak standard, \is{Codification}codified in textbooks, as the medium of instruction \citep[104, 106]{kopp_east_1992}.\footnote{Figures as of 1984.} The Slovak literary standard itself also became a compulsory subject in schools \citep[32]{pokrivcakova_clil_2013}. In 2013, to give a sample year, the Slovak educational system boasted 2,716 state schools, supplemented by 154 private or church schools, educating around 153,000 pupils annually \citep[46]{santiago_oecd_2016}. Such institutions have created mass literacy in a unique literary standard associated with the \il{Slovak}glottonym “Slovak”.

State institutions have also sponsored and promoted the concept of a \il{Slovak}distinct “Slovak language” outside of the school system. The independent Slovak republic employs state power to promote its preferred literary standard. Act 270/1995 “on the State Language of the Slovak Republic”, for instance, declared “the \il{Slovak}Slovak language” to be “the most important attribute of the Slovak nation’s specificity and the most precious value of its cultural heritage, as well as an expression of sovereignty of the Slovak Republic (see the preamble to \textit{Zákon č. 270/1995 Z. z.} \citeyear[1999--2002]{noauthor_zakon_1995})”.\footnote{Also available from \textit{Zbierka zákonov}: \textit{Zákon č. 270/1995 Z. z.} (\citeyear[1999--2002]{noauthor_zakon_1995}), \textit{Zákon č. 184/1999 {Z}. z.} (\citeyear[1418--1419]{noauthor_zakon_1999}), \textit{Zákon č. 318/2009 {Z}. z.} (\citeyear[2362--2367]{noauthor_zakon_2009}), \textit{Zákon č. 35/2011 {Z}. z.} (\citeyear[388--389]{noauthor_zakon_2011}).} The law initially required \il{Czech}Czech films shown in Slovakia to have \il{Slovak}Slovak subtitles, though protests from film distributors and public ridicule forced the Slovak government to back down \citep[109]{fisher_political_2006}. At the time of writing, however, legislation still insists that “an audiovisual work in another language intended for minors below the age of 12 that is transmitted by broadcasting must be dubbed into the state language” (\textit{Zákon č. 270/1995 Z. z.} \citeyear[§5(2)]{noauthor_zakon_1995}). \il{Slovak}Slovak and \il{Czech}Czech appear as separate languages on the labels of consumer products, such as cereal boxes, vitamin tablets, shampoo bottles, and so forth (\cite[83]{nabelkova_slovencina_1999}, \citeyear[32]{nabelkova_slovencina_2008}).

Institutional support has influenced popular attitudes. \il{Czech}Czech books are regularly translated into \il{Slovak}Slovak \citep[95]{nabelkova_sucasne_2003}. At least two separate services offer machine translation from \il{Czech}Czech to \il{Slovak}Slovak \citep[92--93]{kubon_comparison_2014}. Twenty-first century Slovaks admittedly vary in how strongly they experience their linguistic distinctiveness in relationship to \il{Czech}Czech. Slovak sociolinguist Mira \citet[90]{nabelkova_sucasne_2003} posits Slovak-Czech “interlinguality [\textit{medzijazykovosť}]”, \linebreak{}since a “wide range of contact with Czech (watching TV and movies, reading books and magazines) still remains in the Slovak environment after 1993” \linebreak{} \citep[66]{nabelkova_case_2014}. Nevertheless, a sociolinguistic survey as far back as 1971 found that 71\% of Slovaks viewed \il{Slovak}Slovak and \il{Czech}Czech as separate languages, and only 23\% as “two different literary forms of the same language” \citep[24]{salzmann_sociolinguistic_1971}. Decades of Slovak statehood can only have heightened the sense of Slovak linguistic distinctiveness.

Outside Slovakia, furthermore, a scholarly consensus recognizes a distinct \il{Slovak}\linebreak{}“Slovak language” alongside other Slavic languages. A comparison of encyclopedia entries, for example, shows that “since the Second World War […] the Slovak category has enjoyed a nearly universal support” \citep[37]{maxwell_taxonomies_2015}. Heinz Kloss influenced many sociolinguists with his concept of an\is{Ausbausprache} \textit{Ausbausprache}, or “language by development”. When \citeauthor{kloss_abstandsprachen_1967} (\citeyear[32]{kloss_abstandsprachen_1967}, cf. \citeyear[311]{kloss_abstandsprachen_1976}) first introduced these terms to Anglophone scholarship, he chose the specific example of \il{Slovak}Slovak in relation to \il{Czech}Czech to illustrate the concept.\footnote{A subsequent \il{English}English article using the same graphic gave \il{Czech}\il{Slovak}Czech/Slovak, \il{Danish}\il{Swedish}Danish/Swedish, and \il{Bulgarian}\il{Macedonian}Bulgarian/Macedonian as three equivalent examples; see \citet[160]{kloss_abstand_1993}.} \il{Slovak}The “Slovak language” was the \is{Ausbausprache}original \textit{Ausbausprache}.

In \ia{Herkel, Jan}Herkel’s day, however, none of these developments had yet taken place. While the language of state administration had become an object of heated political contestation in \ia{Herkel, Jan}Herkel’s Hungary, the primary options were \il{Latin}Latin and \il{Hungarian}Hungarian. The vast majority of Slovak children did not attend schools; those few that did studied \il{Latin}Latin, \il{German}German or \il{Hungarian}Hungarian. Only a handful of people studied any Slavic literary standard, and those that did were confessionally divided between the \il{Bernolákovčina}\is{Orthography!Bernolákovčina tradition}\textit{Bernolákovčina} and \il{Bibličtina}\is{Orthography!Bibličtina tradition}\textit{Bibličtina} traditions.

During \ia{Herkel, Jan}Herkel’s lifetime, finally, most Slavs in northern Hungary were wholly illiterate. Samuel \citeauthor{czambel_slovensky_1890}’s \is{Orthography}orthographic handbook, which guided educators in the First Czechoslovak Republic, may have first been printed in \citeyear{czambel_slovensky_1890}, but the number of Slovaks who mastered its conventions in 1918 was demographically insignificant \citep[150]{maxwell_choosing_2009}. While Habsburg census returns suggest that the Slovak literacy rate had reached 50.1\% in 1900 \citep[146, 166]{kuzmin_vyvoj_1981}, Habsburg census-takers asked respondents only whether they could sign their name, not whether they mastered the finer \is{Orthography}orthographic distinctions of a particular literary standard. When Czechoslovakia was founded, only a handful of intellectuals could have confidently judged between \textit{ten se přiznáwá}, \textit{ten sa priznáwá}, \textit{ten se přiznává}, and \textit{ten sa priznáva}. The establishment of \il{Slovak}Slovak as a distinct language, in short, did not occur until decades after \ia{Herkel, Jan}Herkel’s death.

Conscious effort, then, is required for twenty-first century readers to imagine the linguistic situation in Slavic northern Hungary in the early nineteenth century. Illiterate peasants would presumably have distinguished the spoken variety of their particular village from local varieties spoken in distant locale. But how would illiterate “Slovak” peasants have understood the linguistic differences at the Hungarian-Moravian frontier? To have understood that difference as a discontinuity \il{Slovak}\il{Czech}between “Slovak” and “Czech”, those peasants would have had to believe that a certain region within the \is{Dialect!Slavic dialect continuum}Slavic dialect continuum had a special status as “Slovak” within the Slavic world, and to judge the internal diversity of that “Slovak” region as less significant than the diversity between adjacent parts of the \is{Dialect!Slavic dialect continuum}Slavic dialect continuum across the frontier with Moravia. Such an understanding seems unlikely.

Indeed, strong evidence suggests that in \ia{Herkel, Jan}Herkel’s day even educated Slovaks had not yet developed the concept of \il{Slovak}a “Slovak language”. The leading savants of Slavic north Hungary, even those active in the \is{Codification}codification of a unique literary standard for the use of north Hungarian Slavs, instead imagined \il{Slavic}a “Slavic language”. They believed that Slavs in northern Hungary spoke one and the same language as the inhabitants of Russia, Poland, Bohemia, Carinthia, Croatia, Serbia, Bulgaria and so forth. Nineteenth-century Slavic savants acknowledged differences between the Slavic spoken in St. Petersburg and the Slavic spoken in Prague, just as twenty-first century Slovaks acknowledge differences between what is spoken in Prešov and what is spoken in Trnava. Nevertheless, they imagined such differences as \is{Dialect}merely “dialectal”.

\section{The idea of the Slavic language}

Evidence that Hungarian Slavs imagined a single \il{Slavic}Slavic language is abundant, yet sometimes ambiguous. We have noted above the difficulty of interpreting \ia{Herkel, Jan}Herkel’s use of the word \textit{Tót}. Similar problems arise when reading \il{Slavic}Slavic texts. Taking the contemporary \il{Slovak}Slovak terms as a reference point, the adjective \textit{slovanský} [‘Slavic’] and the adjective \textit{slovenský} [‘Slovak’] both derive from the same \il{Proto-Slavic}proto-Slavic root, which Max \citet[664--665]{vasmer_russisches_1958} has reconstructed as *\textit{slo\-věninъ}. The distinction between \textit{slovanský} and \textit{slovenský}, while firmly estab-\linebreak{}lished by the First World War, was not yet fully developed in \ia{Herkel, Jan}Herkel’s day.

That Hungarian Slavs used descendants of *\textit{slověninъ} to mean both ‘Slavic’ and ‘Slovak’ is easily documented, since geographic clues sometimes remove any doubt as to the intended meaning. In an \citeyear{nosak_slowenka_1842} poem by Bohuslav \citeauthor{nosak_slowenka_1842} (1818--1877), for example, the (nominative feminine singular) adjective \il{Slavic}\textit{slowenská} clearly means ‘Slavic’:

\begin{multicols}{2}
\noindent \hspace*{5mm} \textit{I skály Kaukasa} \\ 
\hspace*{5mm} \textit{Slawii se kořj} \\ 
\hspace*{5mm} \textit{Šjře slowenská řeč} \\ \columnbreak
\hspace*{5mm} \textit{Gak dennice zořj} \\
And the stones of the Caucasus \\
Burn with glory; \\
The broad \il{Slavic}\textit{slowenská} speech \\
Shines like the morning star. \smallskip \\
\citep[164]{nosak_slowenka_1842}
\end{multicols}

\noindent By contrast, Lutheran theologian and patriotic writer Karol \citeauthor{kuzmany_pjsen_1835} (1806--1866) probably used the (nominative masculine singular) adjective \textit{slowenský} to \il{Slovak}mean ‘Slovak’.

\begin{multicols}{2}
\noindent \hspace*{5mm} \textit{Po horách, po dolách,} \\
\hspace*{10mm} \textit{Letj zpěw slowenský:} \\
\hspace*{5mm} \textit{Nože len užime} \\ \columnbreak
\hspace*{10mm} \textit{Ten wěk náš mládenský!} \\
In hills and in valleys, \\
\hspace*{5mm} Soars the \il{Slovak}\textit{slowenský} song, \\
Let us then enjoy \\
\hspace*{5mm} The age of our youth! \smallskip \\
\citep[167]{kuzmany_pjsen_1835}
\end{multicols}

\noindent \ia{Kuzmány, Karol}Kuzmány’s subsequent stanzas refer to the Tatra mountains, the Turiec (Turóc) and Liptov (Liptó) regions, and the towns of Trenčín and Zvolen; his imagined geography thus evokes a specifically Slovak ethnoterritory. That said, \ia{Nosák, Bohuslav}Nosák’s stanza more persuasively documents \is{Pan-Slavism}Pan-Slavism than \ia{Kuzmány, Karol}Kuzmány’s poem demonstrates Slovak particularism: \ia{Kuzmány, Karol}Kuzmány might theoretically have extolled the Ta\-tras as a part of Slavdom, Zvolen as a Slavic town, and so forth.

In several important works by “Slovak” savants, geographic clues explicitly specify a \is{Pan-Slavism}Pan-Slavic ethnoterritory. Verse 257 in the expanded 1832 edition of \citeauthor{kollar_slawy_1832}’s famous poem \textit{Sláwy dcera}, for example, posits the following national homeland:

\begin{small}
\begin{multicols}{2}
\noindent \hspace*{5mm} \textit{Od Athose k Trigle, k Pomořanům,} \\
\hspace*{5mm} \textit{od Psjho k poli Kosowu,} \\
\hspace*{5mm} \textit{ode Carigradu k Petrowu,} \\ \columnbreak
\hspace*{5mm} \textit{od Ladoǧy dole k Astrachanům;} \\
From Athos to Triglav, and Pomerania, \\
from Pskov to Kosovo field, \\
from Constaninople to St. Petersburg, \\
from Lagoda down to the Astrachanese; \\

\end{multicols}
\end{small}

\begin{small}
\begin{multicols}{2}
\noindent \hspace*{5mm} \textit{od Kozáků ku Dubrowničanům,} \\
\hspace*{5mm} \textit{od Blatona k Baltu, Ozowu,} \\
\hspace*{5mm} \textit{ode Prahy k Moskwě, Kyowu,} \\ \columnbreak
\hspace*{5mm} \textit{od Kamčatky až tam ku Japanům,} \\
from the Cossacks to the Dubrovnikers, \\
from Balaton to the Baltic, to Azov, \\
from Prague to Moscow, Kiev \\
From Kamchatka there to the Japanese. \smallskip \\
\citep[verse 257 (no page numbers)]{kollar_slawy_1832}
\end{multicols}
\end{small}

\noindent This territory, which \ia{Kollár, Jan}Kollár described as “All-Slavia [\textit{Wšesláwia}]”, exceeds the wildest daydreams of \is{Irredentism!Slovak irredentism}Slovak irredentism.\footnote{On extremist Slovak claims to territory currently governed by Austria, Poland or Czechia, see \citet[88, 92]{mares_extreme_2009}.} Clearly, \ia{Kollár, Jan}Kollár imagined a \is{Pan-Slavism}Pan-Slavic ethnoterritory, not a Slovak ethnoterritory.

Formal linguistic works from \ia{Herkel, Jan}Herkel’s era prove equally explicit in their Slavism, even if subsequent scholars tend to categorize them in terms of subsequent linguistic categories. Since \citeauthor{dolezal_grammatica_1746}’s 1746 grammar “used many \il{Slovak}Slovak words and forms”, for example, Eugen \citet[265]{jona_pavel_1978} suggested that “Doležal’s book can be understood as the first \il{Slovak}Slovak grammar”. Some Slovak diaspora scholars are even more strident: Josef \citet[101]{kirschbaum_slovak_1975} described Matej Bel’s preface to Doležal’s grammar as an “introduction on the beauty of the \il{Slovak}Slovak language”, while Josef \citet[22]{mikus_pride_1973} claimed Bel had “exalted the \il{Slovak}Slovak language”. \citet[ix (no page numbers)]{bel_praefatio_1746} himself, however, posited an unambiguously \is{Pan-Slavism}Pan-Slavic geography, claiming that the \textit{lingua Slauica} was spoken in “Istria, Dalmatia, Croatia, Bosnia, Bohemia, Silesia, Lusatia, Poland, Lithuania, Prussia, Scandinavia and Russia”.

In the \citeyear{bernolak_grammatica_1790} \textit{Grammatica Slavica}, \citeauthor{bernolak_grammatica_1790} posited an imagined geography identical to \ia{Doležal, Pavel}Doležal’s. According to \citet[v]{bernolak_grammatica_1790}, the language he described was “used by the inhabitants of Istria, Dalmatia, Croatia, Bosnia, Bohemia, Silesia, Lusatia, Poland, Lithuania, Prussia, and Scandinavia, and widely spread in Russia”, and “differing only in \is{Dialect}dialects”.\footnote{On the phrase “to differ only in \is{Dialect}dialects”, see \citet[95--109]{van_hal_differing_2017}.} \citet[3, 7, 36, 248, 269]{bernolak_grammatica_1790} described his native north-Hungarian variety \il{Slavic!Pannonian-Slavic}as “Pannonian-Slavic [\textit{Pannonico-Slavica}]”, spoken by “Pannonian Slavs [\textit{Pannonios Slavos}]”, who supposedly spoke the \linebreak{}“most magnificent” and “genuinely Slavic \is{Idiom}idiom” (\citeyear[iv]{bernolak_grammatica_1790}). Bernolák’s translator \ia{Bresťansky, Andrej}Andrej Bresťansky articulated an equally explicit Slavism with different geographic clues: Bernolák’s grammar, according to Bresťansky, described the language spoken in “Hungary, Bohemia, Moravia, Silesia, Poland, Slavonia, Croatia, Dalmatia, Serbia, Bosnia, Bulgaria, Moldova, Wallachia, Ukraine, Lithuania, and the great Russian Empire”, then adding that the language also used in “Asiatic Turkey, through Anatolia to Armenia and Persia” \citep[i-ii (no page numbers)]{bernolak_schlowakische_1817}. Such geographic clues have, however, not prevented scholars from repeatedly proclaiming Bernolák “the first \is{Codification}codifier of the \il{Slovak}Slovak language” (\cite[61]{sebik_strucne_1940}; \cite{kirschbaum_anton_1964}). Richard \citet[401]{auty_linguistic_1958} even wrote that \il{Slovak}Bernolák’s “Slovak grammar […] was firmly based on the concept of a separate Slovak nation”. Bernolák’s imagined geography, however, suggests otherwise.

Perhaps the most unambiguous evidence of a \is{Pan-Slavism!Linguistic Pan-Slavism}Pan-Slavic linguistic concept comes from the work of \ia{Šafařík, Pavel}Pavel Josef Šafařík (1795--1861), an influential linguist, ethnologist and antiquarian who was born near Rožňava and eventually settled in Prague. (Šafařík’s surname, which modern Slovak scholarship gives as \textit{Šafárik}, again illustrates the \is{Orthography}orthographic uncertainty of the early nineteenth century: he published, as Robert \citet[215]{pynsent_questions_1994} observed, under \il{Hungarian!Semi-Hungarian}the “semi-Hungarian name \textit{Safáry}, \il{German}German \textit{Schaffarik} and \il{Czech}Czech \textit{Šaffařík}”.) In an \citeyear{safarik_slowansky_1842} ethnographic study, \citeauthor{safarik_slowansky_1842} proposed a seven-layer hierarchy of linguistic classification: human speech, described with the word \textit{howor}, was divided into various languages, denoted with the word \textit{jazyk}, the standard \il{Slavic}Slavic word for “language”. According to \citet[3]{safarik_slowansky_1842}, a \textit{jazyk} could be subdivided into \textit{mluvy}, a \textit{mluwa} into \textit{řeči}, a \textit{řeč} into \textit{nářečí}, the (singular) \textit{nářečí} into (plural) \textit{podřečí}, and a \textit{podřečí} into various \textit{různořečí}. \citet[5--6]{safarik_slowansky_1842} provided no criteria for assigning a particular variety to any particular rung of the taxonomy, but \il{Slovak!Hungarian Slovak}classified \textit{uherskoslowenské} [‘Hungarian Slovak’] as one of two \textit{nářeči} of the \il{Czech}Czech \textit{řeč}, part of the western \textit{mluva} of the \il{Slavic}Slavic \textit{jazyk}. For Šafařík, therefore, Slovak did not have the status of a \textit{jazyk} [‘language’]: it was a \textit{nářeč}, a subcategory of a subcategory of a subcategory of a \textit{jazyk} (\cite[738--739]{maxwell_objectivefacts_2023}).

Štúr, though remembered as \is{Codification}the “codifier of \il{Slovak}written Slovak” (\cite[197--213]{jona_ludovit_1956}; \cite[265--275]{zigo_sturova_2005}; \cite[21--34]{durovic_kodifikacia_2007}), also propounded a \is{Pan-Slavism!Linguistic Pan-Slavism}Pan-Slavic linguistic concept, rather than believe in a \il{Slovak}particularist “Slovak language”. In 1846, after consulting with both Catholic and Lutheran savants, \citeauthor{stur_nauka_1846} published a new grammar, \textit{Nauka reči slovenskej} [‘Handbook of Slovak Speech’] \citep{stur_nauka_1846}, and an important pamphlet justifying the new \is{Codification}codification: \textit{Nárečja slovenskuo alebo potreba písaňje v tomto nárečí} [‘The Slovak Dialect, or the Need to Write in this \is{Dialect}Dialect’, hereafter \textit{Nárečja slovenskuo}] \citep[51]{stur_narecja_1846}. As these titles suggest, Štúr variously imagined \il{Slovak}Slovak not as a \textit{jazyk}, but as a \textit{reč} or a \textit{nárečja}. Indeed, he invoked several of Šafařík’s other terminological categories when justifying his new \is{Codification}codification: \citeauthor{stur_nauka_1846}’s grammar (\citeyear[vii]{stur_nauka_1846}) declared that Slovaks have their “own \textit{nárečja}, which is not just a \textit{rozličnorečja} of \il{Czech}Czech”. Though Ján \citet[3]{dorula_slovencina_2011} wrote that Štúr had “scientifically formulized the essential relationship between \il{Slovak}Slovak and \il{Czech}Czech as two similar, yet distinct \il{Slavic}Slavic languages”, and treated as factual “Štúr’s \ia{Štúr, Ľudovít}scientific finding that \il{Slovak}Slovak is a separate \il{Slavic}Slavic language” (\citeyear[5]{dorula_slovencina_2011}), \ia{Štúr, Ľudovít}Štúr actually subsumed \il{Slovak}Slovak within a greater \il{Slavic}Slavic language. Indeed, insofar as he accepted \ia{Šafařík, Pavel}Šafařík’s terminology, Štúr variously saw the Slovak \textit{reč} as a subcategory of a subcategory of the \il{Slavic}Slavic language, or the \il{Slovak}Slovak \textit{nárečja} as a subcategory of a subcategory of a subcategory of the \il{Slavic}Slavic language (\cite[739--740]{maxwell_objectivefacts_2023}).

If subsequent scholars have struggled to realize that grammarians such as \ia{Doležal, Pavel}Doležal, \ia{Bernolák, Anton}Bernolák, and \ia{Štúr, Ľudovít}Štúr imagined a \il{Slavic}Slavic language of which \il{Slovak}Slovak was merely \is{Dialect}a “dialect” (or other more subordinate subcategory), the blame may lie partly with twentieth-century theories about how to distinguish “languages” \is{Dialect}\linebreak{}from “dialects”. Several scholars assume some sort of developmental process \linebreak{} \is{Dialect}transforms “dialects” into “languages”, or alternatively into “standard languages” (\cite{lodge_french_1993}; \cite{nielsen_dialect_2005}; \cite[13--34]{van_marle_dialect_2014}). Einar \citet[933]{haugen_dialect_1966}, for example, \is{Codification}treated “codification” as one of the “crucial features in taking the step \is{Dialect}from ‘dialect’ to ‘language’, from vernacular to standard”. \ia{Haugen, Einar}Haugen specifically proposed a four-stage developmental model, though other models have been suggested (\cite[28--33]{ferguson_language_1968}; \cite{hroch_social_1994}; \citeyear[67--96]{hroch_comparative_2007}). Linguists accustomed to thinking about \is{Codification}codification as the defining criterion for successfully claiming the prestigious status of “language” may be surprised or confused by \is{Codification}codifiers who contentedly assign the status \is{Dialect}of “dialect” to the object of their \is{Codification}codification efforts. Nevertheless, Slavic grammarians, philologists, dictionary-compilers, poets and savants repeatedly proclaimed the existence of \il{Slavic}a “Slavic language” with multiple written \is{Codification}codifications.

The grammatical traditions of Slavic northern Hungary generally challenge contemporary linguists’ assumptions about the nature of language \is{Codification}standardization. One recent study by René Appel and Pieter Muysken described the process of “language \is{Codification}codification” as follows:

\begin{quote}
    The central problem in \is{Codification}codification, of course, is homogeneity. For example, \is{Codification}codification of the grammar of a language is not simply writing down the grammatical rules of language, but generally means that one of two or more rules from different \is{Dialect}dialects will have to be chosen as the “standard” one. \citep[51--52]{appel_language_1987}
\end{quote}

\noindent By contrast, \ia{Doležal, Pavel}Doležal, \ia{Bernolák, Anton}Bernolák, and \ia{Štúr, Ľudovít}Štúr believed that a single “language” could encompass multiple literary traditions, even multiple grammatical \is{Codification}standardizations. Hungarian Slavs understood that Slavs from other regions were proud of their particularities: they imagined \il{Russian}Russian as part of a common \il{Slavic}Slavic language, for example, but did not expect Russians to abandon Cyrillic. They described their preferred \is{Declension}declensions or \is{Conjugation}conjugations while acknowledging and respecting alternatives from other corners of the Slavic world. Even within the confines of Slavic north Hungary, Slavic grammarians showed a remarkable tolerance for heterogeneity. \citeauthor{bernolak_dissertatio_1787}, for example, was presumably thinking of the Protestant \il{Bibličtina}\is{Orthography!Bibličtina tradition}\textit{Bibličtina} tradition when he wrote in \citeyear{bernolak_dissertatio_1787} that he “leaves fully to his own will he who wishes to write in the \il{Czech}Czech fashion” \citep[22--23]{bernolak_dissertatio_1787}.

As linguistic nationalism gained momentum in the early nineteenth century, however, Slavic savants who celebrated grammatical and lexical heterogeneity increasingly found \is{Orthography!Orthographic differences}orthographic diversity problematic. Grammarians and philologists took for granted the essential unity of the Slavic world; their linguistic patriotism also led them to celebrate local grammatical variations, regional vocabulary, and the like. The \is{Orthography!Orthographic differences}diversity of spelling, by contrast, troubled them.

The patriotic fantasies of Slavic savants focused above all on \is{Orthography!Orthographic unity}orthographic unity. On 6 December 1789, for example, the respected philologist \ia{Dobrovský, Josef}Josef Dobrov\-ský (1753--1829), based in Prague, complained about \il{Dalmatian}\is{Orthography}Dalmatian orthography in a letter to \ia{Ribay, Juraj}Juraj Ribay, a (1754--1812), a Hungarian savant writing in the \il{Bibličtina}\is{Orthography!Bibličtina tradition}\textit{Bibličtina} tradition. \ia{Dobrovský, Josef}Dobrovský specifically lamented that Dalmatian priest and diplomat \ia{Komulović, Aleksandar}Aleksandar Komulović had written \textit{charf} where Dobrovský would have pre-\linebreak{}ferred \textit{krew} or \textit{karw} [‘blood’]. Slavic literary progress, Dobrovský thought, would be easier “if only all the Slavs had our own, truly good \is{Orthography}orthography” (\cite[150]{patera_korrespondence_1913}; cf. \cite[217]{agnew_origins_1993}). Observe that \ia{Dobrovský, Josef}Dobrovský’s desire for a common \is{Orthography}orthography extended not merely from Prague to northern Hungary, the territory of the future Czechoslovakia; it also extended south to the Adriatic coast.

Jernej \citeauthor{kopitar_grammatik_1808} (1780--1844), a Carniolan-born savant working in Vienna, first as librarian at the Imperial Court Library and later as the imperial censor for \il{Slavic}Slavic and \il{Ancient Greek| see {Greek}}\il{Greek}Greek books, similarly advocated a \is{Orthography!Orthographic unity}uniform Slavic orthography in his \citeyear{kopitar_grammatik_1808} “Grammar of the \il{Carniolan}\il{Carinthian}\il{Styrian}Slavic language in Carniola, Carinthia and Styria”. Kopitar unfavorably compared \is{Orthography!Orthographic differences}Slavic orthographic diversity to the relative homogeneity in other European languages:

\begin{quote}
    The Germans all have one and the same writing system, and so too the French, the English, the Italians, but the Slavs have one in Carniola, another in Dalmatia, a third in Croatia, in Bohemia a fourth, in Poland a fifth, and in Lusatia a sixth; what’s more in Dalmatia alone Dellabella for example writes in one way, Voltiggi another, and still others in still other ways!! \citep[xxv-xxvi]{kopitar_grammatik_1808}
\end{quote}

\noindent After a brief historical overview of Slavic alphabets, he lamented that “words become unrecognizable, words that are not only the same but even words that are pronounced the same” \citep[xxvi]{kopitar_grammatik_1808}. The word for ‘six’, to give one of Kopitar’s examples, was variously spelled \textit{шесть}, \textit{ſesſt}, \textit{ſceſt}, \textit{sheſt}, \textit{ṡeſt}, \textit{szećś}, {\Blackletter ſcheſż}. The pronunciation varied slightly, but the \is{Orthography!Orthographic differences}orthography magnified small differences, concealing a fundamental similarity.

In his grammar, therefore, Kopitar (\citeyear[xxi]{kopitar_grammatik_1808}) urged Slavs to follow the example of Ancient Greeks: “all of their tribes wrote in their own \is{Dialect}dialect, as with us, but all tribes used one and the same alphabet, one and the same \is{Orthography}orthography!” Kopitar used the \il{Greek}Greek example to express a striking tolerance for lexical diversity: “Just give us a \is{Orthography!Orthographic unity}uniform alphabet! Where the Athenian wrote ϑεος [sic], because that’s how he spoke, the Spartan wrote σιος [sic] because that’s how he spoke. If one would just simplify the writing, it is in general a necessary tool, so everybody should be able to use it easily!”\citep[136--137]{jagic_pisma_1895}.\footnote{“Letter no. 13 (Kopitar to Dobrovský) \ia{Dobrovský, Josef}20 April 1810”, in: \citet[136--137]{jagic_pisma_1895}.} Elsewhere, Kopitar invoked Slavic history: in his 1810 “Patriotic fantasies of a Slav”, \citeauthor{kopitar_patriotische_1810} recalled that “in the 9\textsuperscript{th} century Cyrill’s \is{Dialect}dialect was well on the way to become the common written language [\textit{Schriftsprache}] of all Slavs”. He implicitly urged his readers to follow that example: “write like the Greeks, all with one alphabet, and not according to a dozen contradictory writing systems” \citep[92]{kopitar_patriotische_1810}.

In a lengthy correspondence, \ia{Kopitar, Jernej}Kopitar and \ia{Dobrovský, Josef}Dobrovský pondered \is{Orthography!Orthographic reform}orthographic reform at length. Kopitar sent Dobrovský at least seven different alphabets,\footnote{“Letter no. 1 (Kopitar to Dobrovský), 30 March 1808”, in: \citeauthor{jagic_pisma_1895} (\citeyear[6--7]{jagic_pisma_1895} (\textit{Alphabetum Venedicum})); “Letter no. 8 (Kopitar to Dobrovský) 1/5 February 1810”, in: (\citeyear[103--104]{jagic_pisma_1895} (\textit{Alphabetum Tzervianum}), 105 (\textit{Alphabetum Hieronymianum}), 106 (\textit{Alphabetum Latinoslavum})); “Letter no. 11 (Kopitar to Dobrovský) 26 March 1810”, in: (\citeyear[130]{jagic_pisma_1895} (\ia{Vodnik, Valentin}Valentin Vodnik’s alphabet)); “Letter no. 33 (Kopitar to Dobrovský) 27 October 1811”, in: (\citeyear[220]{jagic_pisma_1895} (\textit{versionem Agendorum Wirtembergicorum})); “Letter no. 61 (Kopitar to Dobrovský) 22 April 1813”, in: (\citeyear[332--333]{jagic_pisma_1895} (\textit{\il{Carniolan}Carn. \il{Croatian}Croat. Dalm. Novum})).} frequently making \is{Orthography}orthographic suggestions of his own \citep[160--161, 163; 278; 291]{jagic_pisma_1895}.\footnote{“Letter no. 16 (Kopitar to Dobrovský) 8 August 1810”, in: \citet[160--161, 163]{jagic_pisma_1895}); “Letter no. 45 (\ia{Kopitar, Jernej}Kopitar to \ia{Dobrovský, Josef}Dobrovský) 2 August 1812”, in: (\citeyear[278]{jagic_pisma_1895}); “Letter no. 48 (Kopitar to Dobrovský) 19 October 1812”, in: (\citeyear[291]{jagic_pisma_1895}).} In an 1812 letter, Kopitar longed for “the \is{Orthography!Orthographic unity}unification of all Slavs with one alphabet, and then the beautifully eternal consequence that every peasant who has learned his ABCs would be able to write \is{Orthography}orthographically” (\citeyear[251--252]{jagic_pisma_1895}).\footnote{“Letter no. 39 (Kopitar to \ia{Dobrovský, Josef}Dobrovský) 28 March 1812”, in: \citet[251--252]{jagic_pisma_1895}.} In another letter proclaiming “his love for the beautiful \il{Slavic}Slavic language” (\citeyear[29]{jagic_pisma_1895}).\footnote{“Letter no. 5 (Kopitar to Dobrovský), 6 February 1809”, in: \citeauthor{jagic_pisma_1895} (\citeyear[29]{jagic_pisma_1895} (\textit{Liebe})).} Kopitar momentarily conceded that the division between the Cyrillic and Latin alphabets was unbridgeable, yet still wished “that the half of the Slavs writing in Latin had the same alphabet!” (\citeyear[33, 41--42]{jagic_pisma_1895}).\footnote{“Letter no. 5 (Kopitar to Dobrovský), 6 February 1809”, in: \citeauthor{jagic_pisma_1895} (\citeyear[33]{jagic_pisma_1895} (\textit{Kroatische Dialekt}), 41 (\textit{Krainische Dialekt}), 42 (\textit{Alphabet})).} Awareness of how strongly individual Slavs clung to their \is{Orthography}orthographic peculiarities, \citet[92]{kopitar_patriotische_1810} even longed for a “wise despotism” that would impose one particular standard, any standard, and thus “compel fools to be wise”.

At times, Kopitar seems to have cast the influential and respected Dobrovský \ia{Dobrovský, Josef}in the role of wise despot. In May 1810, he urged Dobrovský to write “a \il{Slavic}Slavic Lord’s Prayer in all \is{Dialect}dialects, but in only one \is{Orthography}orthography” \citep[148]{jagic_pisma_1895}.\footnote{“Letter no. 14 (\ia{Kopitar, Jernej}Kopitar to \ia{Dobrovský, Josef}Dobrovský) 15/17 May 1810”, in: \citet[148]{jagic_pisma_1895}.} That same October, he begged the Bohemian savant: “make us an \is{Orthography!Orthographic unity}uniform orthography, with a simple alphabet, so that we can read each other, and also that the foreigner does not recoil in horror before \il{Polish}Polish and \il{Croatian}Croatian consonant clusters!” (\citeyear[179]{jagic_pisma_1895}).\footnote{“Letter no. 23 (Kopitar to Dobrovský) 20 October 1810”, in: \citet[179]{jagic_pisma_1895}.} Dobrovský, however, refused to don the mantle of \is{Orthography}orthographic despotism. In a 6 March 1810 letter to Kopitar, he recalled an unpleasant interaction with South-Slav grammarians \ia{Lanosović, Marijan}Marijan Lanosović (1742--1812) and \ia{Stulić, Joakim}Joakim Stulić (1730--1817):

\begin{quote}
    In Vienna, \ia{Lanosović, Marijan}Lanossovich and \ia{Stulić, Joakim}Stulli once came to me and wanted to make me the deciding judge over their \is{Orthography!Orthographic differences}orthographic differences. May God protect me from having to make such decisions. Stulli flew into a rage at the slightest contradiction, and I was happy when the two of them left my room. \citep[108]{jagic_pisma_1895}\footnote{“Letter no. 9 (Dobrovský to Kopitar) 9 March 1810”, in: \citet[108]{jagic_pisma_1895}.}
\end{quote}

\noindent While \ia{Dobrovský, Josef}Dobrovský eventually made some \is{Orthography}orthographic suggestions, he characterized them as \textit{rude et impolitum} (\citeyear[230]{jagic_pisma_1895}).\footnote{“Letter no. 34 (Dobrovský to \ia{Kopitar, Jernej}Kopitar) 20 November 1811”, in: \citet[230]{jagic_pisma_1895}.} In the end, \ia{Dobrovský, Josef}Dobrovský never shared Kopitar’s optimism that Slavic \is{Orthography!Orthographic differences}orthographic differences could be overcome: “we are not even agreed on the ABC” (\citeyear[172]{jagic_pisma_1895}).\footnote{“Letter no. 19 (Dobrovský to Kopitar), 20 October 1810”, in: \citet[172]{jagic_pisma_1895}.} Other Slavic scholars, however, dared where Dobrovský demurred.

\citeauthor{safarik_geschichte_1826}’s \citeyear{safarik_geschichte_1826} \textit{History of the \il{Slavic}Slavic Language and Literature in all its \is{Dialect}Dialects}, published the same year as \ia{Herkel, Jan}Herkel’s \textit{Elementa}, took particular pains when describing \is{Orthography!Orthographic disunity}Slavic orthographic disunity. Šafařík began by attributing the division between Latin and Cyrillic letters to confessional differences. Within that part of the Slavic world that had adopted the Latin alphabet, however,

\begin{quote}
    this adoption took place among tribes […] who were politically divided, and not in \is{Reciprocity}reciprocal exchange with each other, and thus with diverse or even contradictory combinations of Latin letters (e.g. \il{Polish}Polish \textit{cz} instead of ч, \il{Croatian}Croatian c instead of ц, \il{Polish}Polish \textit{sz} instead of ш), which means that these tribes cannot read each other’s books. \citep[64--65]{safarik_geschichte_1826}
\end{quote}

\noindent Since “the Latin alphabet has fewer symbols than the Slavic language needs”, furthermore, Slavs had to “amalgamate several letters to depict a third sound, completely different from the sounds of the individual letters”. He consoled himself somewhat with the thought \is{Orthography}that “the orthography of the Italians, Germans, French and English, etc.” shared some similar difficulties. Nevertheless, echoing a similar passage from \ia{Kopitar, Jernej}Kopitar, cited above, Šafařík lamented that

\begin{quote}
    all these, despite the clumsiness and awkwardness of combination at least have one and the same writing system; while the Slavs, as previously noted, have in Carinthia one writing system, in Dalmatia another, in Croatia a third, in Bohemia a fourth, in Poland a fifth, and in Lusatia a sixth. And on top of that: even in Dalmatia e.g. Dellabella writes one way, Voltiggi another, and others differently again; in \il{Windic}Windic we find the same by Bohorizh and P. Parcus; even the Slavonians needlessly mix \il{Croatian}Croatian letters into their normally \il{Dalmatian}Dalmatian orthography in their catechisms and other school books; the Sorbian Wends in \il{Sorbian!Upper Sorbian}Upper and \il{Sorbian!Lower Sorbian}Lower Lusatia diverge from each other in various small things and what would one have to say about the \il{Czech}Czechs and \il{Polish}Poles, if one compared the writing systems of Kochanowski, Gornicki, Januszowski, Dmochowski, Kopczynski, and many others on the one hand, and the orthography of Hus, Weleslawín, the Bohemian brothers, \ia{Dobrovský, Josef}Dobrovský, Tomas, Hromadko and many others! \citep[65--66]{safarik_geschichte_1826}
\end{quote}

\noindent \citet[66]{safarik_geschichte_1826} concluded that such diversity “annoys every friend of the Slavs, frightens away foreigners otherwise eager to learn, and is unhappily the greatest self-inflicted obstacle to the united progress of the Latin half” of the Slavic world.

\section{Pan-Slavism as orthographic reform}

Such was the intellectual atmosphere in which Herkel wrote his \textit{Elementa Universalis Linguae Slavicae}. \ia{Herkel, Jan}Herkel, like his predecessors and contemporaries, assumed that all Slavs spoke a single \il{Slavic}Slavic language. Like most of his predecessors and contemporaries, he characterized the differences between \il{Russian}Russian, \il{Polish}Polish, \il{Serbian}Serbian and so forth as those \is{Dialect}between “dialects” which “differed more or less with strange vocabulary, even though the original expressions are still present in all \is{Dialect}dialects” \citep[17]{herkel_elementa_1826}. Herkel provided grammatical information about the various \is{Dialect}dialects, providing various declination charts for \is{Declension!Noun declension}nouns and \is{Declension!Adjective declension}adjectives. Like \ia{Bernolák, Anton}Bernolák, furthermore, he devoted disproportionate attention to his native \is{Dialect}dialect, and like Bernolák characterized that dialect \il{Pannonian}as “Pannonian”.

\ia{Herkel, Jan}Herkel nevertheless proposed a \il{Slavic}\is{Orthography!Orthographic unity}universal Slavic orthography that transcended any \is{Dialect}particular “dialect”. If “the only impediment to the literature of the Slavic nations was diversity of letters for writing, in other words \is{Orthography}orthography”, then the obvious solution was to devise a common alphabet meant for all Slavs. \ia{Herkel, Jan}Herkel copied most of his letters from the Latin alphabet, though he saw no need for Latin \{q\} = \{k\}, \{x\} = \{ks\}, \{y\} = \{i\}. However, he adopted three letters from Cyrillic: \{ч, ш, x\}. He rejected Cyrillic \{ж\} because it was “too different from European letters”, proposing \{ƶ\} in its place (\citeyear[8]{herkel_elementa_1826}). The resulting basic alphabet has 27 basic letters: \{a, b, c, ч, d, e, f, g, h, x, i, j, y, k, l, m, n, o, p, r, s, ш, t, u, v, z, ƶ\} (\citeyear[11]{herkel_elementa_1826}).

He also provided a conversion table with Grazhdanka Cyrillic, which he called “Russian”, even though one of the letters, \{ћ\}, was not used in Russia, but only in the Balkans. Herkel’s transliteration table proposed digraphs for five Cyrillic letters, \{щ, я, ћ, ѣ, ю\} (\citeyear[164]{herkel_elementa_1826}) (see \hyperref[fig:Figure 1.1]{Figure 1.1}).

\begin{figure}
    \centering
    \caption*{Figure 1.1: Herkel’s basic alphabet with Cyrillic conversion table}
    \label{fig:Figure 1.1}
    \includegraphics[scale=0.6]{figures/maxwell_9.png}
\end{figure}

\ia{Herkel, Jan}Herkel’s alphabet distinguishes itself from both the \il{Bernolákovčina}\is{Orthography!Bernolákovčina tradition}\textit{Bernolákovčina} and \il{Bibličtina}\is{Orthography!Bibličtina tradition}\textit{Bibličtina} traditions in its lack of diacritical marks. According to Robert \citet[327--328]{auty_orthographical_1968}, several of Herkel’s Catholic contemporaries shared this distaste for dia\-critics: \ia{Linde, Samuel Gottlieb}Samuel Linde (1771--1847), author of a six-volume \il{Polish}Polish dictionary, supposedly disliked them; \ia{Kopitar, Jernej}Kopitar saw them as “a Hussite institution which the ‘Bohemian Hussites and Lutherans’ were trying to impose on true Catholics”. In his private correspondence, Kopitar repeatedly denigrated them as “fly excrement [\textit{Fliegendreck} (see \cite[96, 99, 107]{sakcinski_arkiv_1875}),\footnote{“Letter no. 17 (Kopitar to Kristianović) 4 May 1838”, in: \citet[96]{sakcinski_arkiv_1875}, “Letter no. 18 (Kopitar to Kristianović) 8 June 1838”, in: (\citeyear[99]{sakcinski_arkiv_1875}), “Letter no. 23 (Kopitar to Kristianović) 13 October 1840”, in: (\citeyear[107]{sakcinski_arkiv_1875}).} \textit{Fliegenschmisse} (see \citeyear[104, 105]{sakcinski_arkiv_1875})\footnote{“Letter no. 21 (Kopitar to Kristianović) 25 January 1839”, in: \citet[104, 105]{sakcinski_arkiv_1875}.}]”. Perhaps Herkel’s Catholic background explains why he too saw diacritics as a problem. Catholic feeling, however, hardly explains \ia{Herkel, Jan}Herkel’s willingness to borrow letters from Cyrillic.

\ia{Herkel, Jan}Herkel was not the first reformer of \is{Orthography}Slavic orthography to combine the Latin and Cyrillic alphabets. In 1810, \ia{Kopitar, Jernej}Kopitar had also proposed a Latin alphabet supplemented with \{ш, ч\} \citep[163]{jagic_pisma_1895}.\footnote{“Letter no. 16 (Kopitar to Dobrovský) \ia{Dobrovský, Josef}8 August 1810”, in: \citet[163]{jagic_pisma_1895}.} If \citeauthor{jagic_pisma_1895}’s transcriptions can be trusted, furthermore, \ia{Kopitar, Jernej}Kopitar in his correspondence with \ia{Dobrovský, Josef}Dobrovský sometimes mixed Cyrillic and Latin letters in individual words, e.g. the place-name \textit{шtajerſsko} [‘Styria’] (\citeyear[232]{jagic_pisma_1895}),\footnote{“Letter no. 35 (Kopitar to Dobrovský) 25 November 1811”, in: \citet[232]{jagic_pisma_1895}.} and the surname \textit{Жupàn} (\citeyear[159, 161]{jagic_pisma_1895}).\footnote{“Letter no. 16 (Kopitar to Dobrovský) 8 August 1810”, in: \citet[159, 161]{jagic_pisma_1895}.} Carniolan priest \ia{Vodnik, Valentin}Valentin Vodnik (1758--1819) had also supplemented an alphabet consisting most\-ly of Latin letters with five Cyrillic letters: \{з, ж, ч, ш, щ\} (\citeyear[130]{jagic_pisma_1895}).\footnote{“Letter no. 11 (Kopitar to Dobrovský) 26 March 1810”, in: \citet[130]{jagic_pisma_1895}.} When asked to comment on \ia{Vodnik, Valentin}Vodnik’s alphabet, Dobrovský initially approved, in one sentence specifically accepting \{ч, ж, x\} “even though the Bohemians do not want to give up č”. Dobrovský’s approval, however, proved fickle: later in the same paragraph he proposed \{q\} in place of \{ч\} so that “it does not look too much different from Latin”, “\textit{ſſ} (ss)” in place of \{ш\}, and “ſſq” for \{щ\}, since “in this way the alphabet would be all \il{Latin}Latin, \textit{mutatis mutandis}” (\citeyear[132]{jagic_pisma_1895}).\footnote{“Letter no. 12 (Dobrovský to Kopitar) 2 April 1810”, in: \citet[132]{jagic_pisma_1895}. Kopitar asked in reply “what do Bohemians have against ш? It has the same relationship to m that u does to n!” See “Letter no. 13 (Kopitar to Dobrovský) 20 April 1810”, in: \citet[142]{jagic_pisma_1895}.} In subsequent letters, Dobrovský proposed \{ç, ſ\} for \{ч, ш\} (\citeyear[187]{jagic_pisma_1895}),\footnote{“Letter no. 25 (Dobrovský to Kopitar) 30 January 1811”, in: \citet[187]{jagic_pisma_1895}.} \{q\} for \{ч\}, and “x or χ” for \{ж\} (\citeyear[259]{jagic_pisma_1895}).\footnote{“Letter no. 41 (Dobrovský to Kopitar) 3 May 1812”, in: \citet[259]{jagic_pisma_1895}.} Dobrovský also proposed the word “\textit{ч}loviek” [‘human being’, cf. \il{Old Church Slavonic}OCS {\mono чловѣкъ}, \il{Russian}Russian \textit{человек}, \il{Czech}Czech \textit{člověk}] (\citeyear[245]{jagic_pisma_1895}).\footnote{“Letter no. 38 (Dobrovský to Kopitar) 22 February 1812”, in: \citet[245]{jagic_pisma_1895}.}

Dobrovský \ia{Dobrovský, Josef}may have also foreshadowed \ia{Herkel, Jan}Herkel’s \{ƶ\}, though the typefaces available to \ia{Jagić, Vatroslav}Vatroslav Jagić when publishing Dobrovský’s correspondence make analysis difficult. In a 6 March 1810 letter, as \citeauthor{jagic_pisma_1895} reprinted it in 1895, Dobrovský wrote: “for ж I suggest z (with a strong line down the center) because of its similarity with z, or alternatively ʐ with a cedilla, because not everybody is comfortable with the dot in ž” (\citeyear[108]{jagic_pisma_1895}).\footnote{“Letter no. 9 (Dobrovský to Kopitar) 6 March 1810”, in: \citet[108]{jagic_pisma_1895}.} I have not been able to examine Dobrovský’s original manuscript, but the “strong line” seems to suggest \{ƶ\}, much as the “dot” suggests \{ż\} rather than \{ž\}.

Awareness of \il{Slavic}Slavic phonetic diversity limited \ia{Herkel, Jan}Herkel’s \is{Orthography!Orthographic unity}orthographic uniformity: Herkel understood that some varieties of Slavic require unique sounds not found in the rest of the Slavic world. Since, for example, \il{Polish}Polish has two nasal vowels, \ia{Herkel, Jan}Herkel acknowledged the need for the uniquely \il{Polish}Polish letters \{ą\} and \{ę\}, though he saw no need for \il{Polish}Polish \{ł\}. He also accepted the \il{Czech}Czech \{ř\} \citep[129]{herkel_elementa_1826}, and indeed recommended it to Poles, apparently from a distaste of digraphs: “the Bohemians have already eliminated z with r, […] we would not be discouraged if the same would happen with the Poles” (\citeyear[12]{herkel_elementa_1826}). When transliterating from \il{Polish}Polish, however, Herkel simply replaced \{rz\} with \{r\} (\citeyear[135]{herkel_elementa_1826}).

\ia{Herkel, Jan}Herkel’s grammar acknowledged grammatical diversity. \textit{Elementa Universalis Linguae Slavicae} provided a series of noun \is{Declension!Noun declension}declension tables, and each associated with a \is{Dialect}particular “dialect”. For example, Herkel provided masculine noun \is{Declension!Noun declension}declensions from seven Slavic varieties, copied from leading linguists of his time. Herkel specifically transcribed \textit{pannonica dialectus} from \citet[27--28]{bernolak_grammatica_1790},\footnote{\citet[43--44]{herkel_elementa_1826}; cf. \citet[72--73, 206--207]{herkel_jan_2009}.} the \textit{dialectus bohemica} from \citet[165]{dobrovsky_lehrgebaude_1819},\footnote{\citet[41--42]{herkel_elementa_1826}; cf. \citet[68--71, 206]{herkel_jan_2009}.} and \il{Slovene}Slovene (\textit{Vindi}) from \citet[232]{kopitar_grammatik_1808}.\footnote{\citet[50]{herkel_elementa_1826}; cf. \citet[78--79, 207]{herkel_jan_2009}.} He transliterated \il{Russian}Russian from Antonín \citeauthor{puchmayer_lehrgebaude_1820}’s \citeyear{puchmayer_lehrgebaude_1820} \textit{Lehrgebäude der russischen Sprache} (\citeyear[fold-out between 204--205]{puchmayer_lehrgebaude_1820}).\footnote{\citet[34--35]{herkel_elementa_1826}; cf. \citet[62--63, 206]{herkel_jan_2009}.} \il{Serbian}Serbian from Vuk \citeauthor{karadzic_srpski_1818}’s \citeyear{karadzic_srpski_1818} \textit{Srpski rječnik} (\citeyear[xxxvii]{karadzic_srpski_1818}),\footnote{\citet[51]{herkel_elementa_1826}; cf. \citet[80--81, 207]{herkel_jan_2009}.} and \il{Polish}Polish (\textit{dialectus polo\-nica}) from Jerzy \citeauthor{bandtkie_polnische_1824}’s \citeyear{bandtkie_polnische_1824} grammar (\citeyear[47]{bandtkie_polnische_1824}).\footnote{\citet[38, 90]{herkel_elementa_1826}; cf. \citet[66--67, 206]{herkel_jan_2009}.} He also provided \is{Declension}declensions in \il{Old Church Slavonic}the \textit{vetus dialectus} [‘old dialect’], taken from \citeauthor{dobrovsky_institutiones_1822}’s \citeyear{dobrovsky_institutiones_1822} \textit{Institutiones linguae Slavicae dialecti veteris} (\citeyear[466, 486]{dobrovsky_institutiones_1822}).\footnote{\citet[29--30]{herkel_elementa_1826}; cf. \citet[56--59, 205]{herkel_jan_2009}.} Herkel’s Cyrillic transliterations were not always faithful. For example, he softened hard vowels (e.g. \citeauthor{dobrovsky_institutiones_1822}’s \il{Old Church Slavonic}Old Slavonic {\mono сын} > \textit{sin} instead of \textit{syn} [‘son’] (\citeyear[466]{dobrovsky_institutiones_1822}); \citeauthor{puchmayer_lehrgebaude_1820}’s \il{Russian}Russian \textit{столы̀} > \textit{stoli}, instead of \textit{stoly} [‘tables’] (\citeyear[34]{puchmayer_lehrgebaude_1820}).\footnote{\citet[29, 35]{herkel_elementa_1826}; cf. \citet[206]{herkel_jan_2009}.} The important point, however, is that Herkel’s various \is{Declension}declensions do not agree with each other. According to Herkel, the masculine dative singular \is{Declension}declension is \{-u\} in \il{Serbian}Serbian, \{-ovi\} in \il{Polish}Polish, \{-u\} or \{-ovi\} \il{Pannonian}in “Pannonian”; \{-u\}, \{-ovi\} or \{-i\} in \il{Slovene}Slovene, \{-u\} or \{-i\} in \il{Czech}Czech, \{-u\} or \{-iu\} in \il{Russian}Russian, and \{-u\}, \{-ovi\}, \{-evi\}, or \{-iu\} in \il{Old Church Slavonic}Old Slavonic.

Herkel’s transcribed \is{Declension}declensions differ from those of his sources. Herkel tur\-ned \ia{Dobrovský, Josef}Dobrovský’s \il{Old Church Slavonic}Old Slavonic \{-{\mono ови}\}, \ia{Bandtkie, Jerzy Samuel}Bandtkie’s \il{Polish}Polish declination \{-owi\}, \ia{Bernolák, Anton}Berno\-lák’s \il{Pannonian}Pannonian \{-{\Blackletter owi}\}, and \ia{Kopitar, Jernej}Kopitar’s \il{Carniolan}Carniolan \{-òvi\} into \{-ovi\}. He turned Dobrovský’s \il{Old Church Slavonic}Old Slavonic \{-{\mono ꙋ}\}, \ia{Puchmajer, Antonín Jaroslav}Puchmajer’s \il{Russian}Russian \{-у̀\}, Dobrovský’s \il{Czech}Czech \{-{\Blackletter u}\}, Bernolák’s Pannonian \{-{\Blackletter u}\}, \ia{Karadžić, Vuk}Karadžić’s \il{Serbian}Serbian \{-у\}, and Kopitar’s \il{Slovene}Slovene \{-u, -ù\} into \{-u\}. Herkel’s proposed grammar thus minimized \is{Orthography!Orthographic differences}orthographic diversity (see \hyperref[tab:Table 1.2]{Table 1.2}).

\begin{table}
    \centering
    \footnotesize
    \caption*{Table 1.2: Masculine singular dative noun endings: Herkel (above), Herkel’s sources (below).}
    \label{tab:Table 1.2}
    \begin{tabular}{ c c c c c c c }
    \lsptoprule
    \il{Serbian}Serbian & \il{Russian}Russian & \il{Polish}Polish & \il{Pannonian}Pannon. & \il{Czech}Czech & \il{Slovene}Slovene & \il{Old Church Slavonic}OCS \\
    \midrule
    -u & -u, -iu & -ovi & -u, -ovi & -u, -i & -u, -ovi, -i & -u, -ovi, -evi, -iu \\
    -y & -у̀, -ю̀ & -owi & -{\Blackletter u}, -{\Blackletter owi} & -{\Blackletter u}, -{\Blackletter i} & -u, -u, -ù, -òvi, -i & -{\mono ꙋ}, -{\mono ови}, -{\mono еви}, -{\mono ю} \\
    \lspbottomrule
    \end{tabular}
\end{table}

\ia{Herkel, Jan}Herkel also endorsed some particular \is{Declension}declensions. In general, he tried to differentiate between the \is{Dialect}merely “dialectal” forms from other forms characterized as “genuinely Slavic” or as “the genuine Slavic”. Even here, however, he tolerated diversity. Herkel’s proposed masculine singular dative ending, for example, was \{-u, -vi\}, illustrated in part by \is{Declension!Noun declension}declining the word \textit{sin} [‘son’] into the dative as “\textit{sinu}, or \textit{sinovi} etc.”, forms which, Herkel claimed, agree “with both the usage and the \is{Genius}genius of the language, and without any exception”. He similarly acknowledged three locative neuter forms, \{-ax, -ix, -ox\}, illustrated with the sample words \textit{sercax}, \textit{sercih}, \textit{sercox} (from \textit{serce} [‘heart’]). This sample word, finally, illustrates \citeauthor{herkel_elementa_1826}’s surprising tolerance of lexical diversity: elsewhere in \textit{Elementa}, he spelled it \textit{srdce} (\citeyear[25, 71]{herkel_elementa_1826}). Though Herkel tried to stigmatize forms that were “merely one-sided, \is{Dialect}dialectal, and by all means not based on sound logic”, he acknowledged diversity of grammatical endings.

\ia{Herkel, Jan}Herkel then provided sample texts in the \il{Russian}Russian, \il{Little Russian}Little Russian [\textit{maloruska}], \il{Polish}Polish, \il{Bohemian}Bohemian, \il{Serbian}Serbian, and \il{Pannonian}\is{Dialect}Pannonian “dialects”. He prepared them by trans\-literating from a two-volume collection of “Slavonic songs” gathered by František Čelakovský (1799--1852).\footnote{\citeauthor{celakovsky_slowanske_1822} (\citeyear[90--92]{celakovsky_slowanske_1825} (\textit{russica}), \citeyear[112--114]{celakovsky_slowanske_1825} (\textit{maloruska}), \citeyear[150--151]{celakovsky_slowanske_1822} (\textit{polonica}), \citeyear[176]{celakovsky_slowanske_1822} (\textit{serbska}), \citeyear[12--14]{celakovsky_slowanske_1822} (\textit{bohemica})). On Herkel’s sources, see \citet[219]{herkel_jan_2009}; \citet[25--26]{matovcik_prispevok_1964}.} As authentic examples of Slavic writing, Čelakovský’s songbook leaves something to be desired; \citet[90, 113]{celakovsky_slowanske_1825} presented both \il{Russian}Russian and \il{Ukrainian}Ukrainian songs in the Latin \is{Orthography}orthography then current in Pra\-gue, complete with the letter \{ř\}. Herkel’s transliterations in turn differ from Čelakovský’s texts, though the \is{Orthography!Orthographic differences}orthographic differences between Čelakovský and Herkel remain comparable to those between \il{Bibličtina}\is{Orthography!Bibličtina tradition}\textit{Bibličtina} and \il{Bernolákovčina}\is{Orthography!Bernolákovčina tradition}\textit{Berno\-lákovčina}. Herkel’s sample \il{Polish}Polish text ultimately derived from an 1820 poem by Polish diplomat and man of letters Julian Ursyn Niemcewicz (1758--1841). A comparison illustrates the \is{Orthography!Orthographic differences}orthographic differences between \citeauthor{niemcewicz_duma_1820}’s \il{Polish}Polish as printed in 1820 (see \hyperref[tab:Table 1.3]{Table 1.3}), and \il{Polish}Polish printed six years later in Herkel’s \is{Orthography}orthography (see \hyperref[tab:Table 1.4]{Table 1.4}) (\cite[201]{niemcewicz_duma_1820}; cf. \cite[134]{herkel_elementa_1826}).

\enlargethispage{\baselineskip}

\begin{table}
    \centering
    \small
    \caption*{Table 1.3: Niemcewicz’s original (1820)}
    \label{tab:Table 1.3}
    \begin{tabular}[t]{l l}
        \lsptoprule
        \vspace*{-2.3mm}& \\
        \multirow{5}{*}{\includegraphics[scale=0.54,trim=0 0 0 10mm]{figures/maxwell_10.png}} & Już mgła na morskiéy opadła powodzi,\\
        & Już żałość padła w serce woiownika,\\
        & Z sinego morza, mgła sztara nieschodzi,\\ 
        & Ni żałość y serce mołodca nie znika.\\
        \lspbottomrule
    \end{tabular}
\end{table}

\begin{table}
    \centering
    \small
    \caption*{Table 1.4: \ia{Herkel, Jan}Herkel’s revised \is{Orthography}orthography (1826)}
    \label{tab:Table 1.4}
    \begin{tabular}[t]{l l}
        \lsptoprule
        \multirow{5}{*}{\includegraphics[scale=0.555,trim=0 0 0 6mm]{figures/maxwell_11.png}} & Juƶ mgla na morskiei opadla provodzi\\
        & Juƶ ƶalosć padla v serce vojovnika.\\
        & Ze sinego mora, mgla sara nie sxodzi,\\ 
        & Ni ƶalosć ze serca molodca nieznika.\\
        \lspbottomrule
    \end{tabular}
\end{table}

Note that \ia{Herkel, Jan}Herkel’s text acknowledged \il{Polish}Polish distinctiveness in grammatical \is{Declension}declensions, in vocabulary, and even in \is{Orthography}orthography. The letter \{ć\}, furthermore, illustrates Herkel’s awareness of phonetic diversity: he acknowledged that \il{Polish}Polish needed a special letter to depict a unique sound not used in other Slavic varieties.

Scholars have not always realized the extent to which \ia{Herkel, Jan}Herkel’s \is{Orthography}orthography accepted lexical and grammatical diversity. When Maria \citet[167]{dernalowicz_slavophilie_2002} wrote \ia{Herkel, Jan}that “Herkel proposed the formation of one single grammar, one single written language and one alphabet for all the Slavs”, she was right only about the single alphabet: Herkel envisioned a multiplicity of grammatical rules, as exemplified in the diversity of noun \is{Declension!Noun declension}declensions. Herkel also provided examples of six different writing systems. Mieczysław \citet[39]{basaj_projektach_1985} rightly emphasized the alphabet when describing Herkel’s work as “full project for a general \il{Slavic}Slavic language and alphabet [\textit{projektem języka i alfabetu ogólnosłowiańskiego}]”, yet implied that Herkel’s ambitions also extended to the “language”, whatever the term might mean in this context. Robert \citet[327]{auty_orthographical_1968} more strikingly adduced Herkel as evidence that “proposals for a \is{Pan-Slavism}\il{Slavic!Pan-Slavic}pan-Slavonic alphabet were sometimes […] specifically linked to proposals for a single pan-Slavonic language”. Auty’s term “language” apparently implies something more than a shared alphabet.

The ambiguity of the term “language” extends to more technical linguistic jargon. Helmut \citet[441]{slapnicka_sprache_1974}, for example, spoke not of unmodified “language”, but of \ia{Herkel, Jan}Herkel’s “utopian plan to create one \il{Slavic}Slavic written language”. Other anglophone scholars have claimed that Herkel “proposed a common Slav literary language” (\cite[254]{kohn_panslavism_1953}; \cite[36]{kirschbaum_pan-slavism_1966}); that he “promoted a single Slavic written language” \citep[116]{koch_slavism_1962}, that he “presented the concept of a joint Slavonic literary language” \citep[315]{lazari_idei_2009}, that he “advocated a common Slav literary language” \citep[21]{roucek_soviet_1953}, or that he “advocated a common literary language for all Slavic nationalities” \citep[93]{grebmeier_origins_1951}. Scholars writing in \il{German}German concur, variously describing Herkel’s goal as a “common written language [\textit{gemeinsame Schriftsprache}]” (\cite[94]{hantsch_panslawismus_1965}; \cite[84]{picht_m_1969}; \cite[103]{meyer_wiederbelebung_2014}), or as “a common Slavic literary language [\textit{eine gemeinsame slavische Literatursprache}]” \citep[110]{biedermann_polen_1967}.

Slavic scholars describe Herkel’s plans with equivalent \il{Slavic}Slavic phrases. Herkel has supposedly promoted “one written language [\textit{jeden spisovný jazyk}]” (\cite[191]{rosenbaum_o_1954}; \cite[131]{vyvijalova_snahy_1960}), “a single Slavic literary language [\textit{jednolity słowiański język literacki}]” \citep[254]{chlebowczyk_procesy_1975}, “a single literary language [\textit{единый литературный язык}] based on all current Slavic languages” \citep[118]{zlidnev_formirovanie_1977}, a “common Slavic literary language [\textit{skupen slovanski literarni jezik}]” \citep[33]{zajc_panslavizem_2009}, a “common book language [\textit{skupni knjižni jezik}] \citep[69]{dolgan_vzajemni_1995}, a “universal written language [\textit{univerzální spisovný jazyk}] which transcends particularism and brings about all-Slavic literary unity” \citep[376]{sefcik_buzassyova_2010}, or “an artificial common written language for the Slavs [\textit{umelý spoločný spisovný jazyk Slovanov}]” \citep[105]{butvin_problematika_1978}. Herkel’s translator \citet[42]{buzassyova_gramaticke_2002} thought he wanted “the artificial creation of a common cultural language [\textit{kultúrny jazyk spoločný}]”. Other Slavic scholars, dispensing with the adjectives “written”, “literary”, and “cultural”, simply proclaim that Herkel wanted to create a “general Slavic language [\textit{język ogólnosłowiański}]” \citep[162]{baziur_slowacka_2016}, “a common Slavic language [\textit{wspólni język słowiański}]”, \citep[67]{kola_slowianofilstwo_2004}, a “common language [\textit{об\-щего языка}] for all the Slavs” \citep[4]{pavlenko_panslavizm_2016}, or a “common Slavic language [\textit{Общеславянский язык}]” \citep[18]{kulikovskij_ocherk_1885}.

Such descriptions are not entirely mistaken. \ia{Herkel, Jan}Herkel’s \is{Orthography}orthographic proposals were linguistic. They concerned “language”, as opposed to, say, political frontiers or legal reform; they also specifically addressed written language, as opposed to spoken language. Nevertheless, phrases translatable as “written language”, such as \textit{Literatursprache}, \textit{Schriftsprache}, \textit{spisovný jazyk}, \textit{knjižni jezik}, \textit{język literacki}, imply something more extensive than a new \is{Orthography}orthography. They connote a set of prescriptive rules defining “correct” writing, which in turn implies prescriptive grammatical rules, normative pronunciation, and \is{Codification}standardized vocabulary. While \ia{Herkel, Jan}Herkel endorsed certain \is{Declension}declensions and criticized others, he did not proscribe uniform rules, he at most suggested the use of common forms. He noted diversity of pronunciation without passing judgement. He rarely discussed vocabulary.

Nevertheless, several scholars, perhaps misled by ambiguity in the secondary literature, apparently believe that \ia{Herkel, Jan}Herkel provided a full prescriptive grammar. Scholars have variously claimed that Herkel tried to “produce a universal \il{Slavic}Slavic grammar” \citep[108]{haraksim_slovak_2011}, that he “sketched out the grammar of a \is{Pan-Slavism}\il{Slavic!Pan-Slavic}pan-Slavonic language” \citep[197]{auty_internal_1967}, that he “set forth the grammar of the all-Slavic language” \citep[50]{smirnov_istoricheskij_2000}, that he “tried to give grammatical rules of a common Slavic language” \citep[27]{churkina_vopros_1998}, or that he “demanded the creation of a unified Slavic grammar” \citep[177]{kun_classification_1982}. Inna \citet[118]{leshchilovskaja_idei_1977} claimed that \ia{Herkel, Jan}Herkel had not only justified “the idea of a common Slavic language, but also developed its rules”. Endre \citet[76]{arato_slavic_1976}, finally, wrote that Herkel had “summed up the rules of the non-existing unified \il{Slavic}Slavic language, which were to be accepted by all Slavic peoples” (see also \cite[260]{arato_plan_1971}). In fact, when Herkel discussed grammatical rules, he emphasized diversity. Furthermore, he described what he believed was existing usage, drawing on existing literature. He did not attempt to impose his own fancies by fiat, he supported whichever forms that already enjoyed the broadest usage, to the best of his admittedly imperfect knowledge.

In light of such misunderstandings, therefore, let us emphasize that Herkel’s book prescribed no standard vocabulary, presented its preferred \is{Declension}declensions and \is{Conjugation}conjugations as suggestions, and took a strong normative stance only on questions of \is{Orthography}orthography. Minimizing \is{Orthography!Orthographic differences}orthographic difference, \ia{Herkel, Jan}Herkel hoped, would facilitate the exchange of ideas between different parts of the Slavic world. If Slavs could more easily read each other’s books, they could learn from each other, better appreciate one another’s ideas, share in each other’s accomplishments. He famously advocated “\textit{unity in literature} among all Slavs, which is the true \is{Pan-Slavism}Pan-Slavism” \citep[4]{herkel_elementa_1826}. This reference to “literature” may be somewhat misleading, since Herkel expressed no particular interest in \textit{belles lettres}. He probably used the term to invoke writing, as opposed to spoken conversation. While the unity he advocated touched on grammatical issues, it remained primarily \is{Orthography!Orthographic unity}orthographic.

\ia{Herkel, Jan}Herkel’s \is{Orthography}orthographic system attracted no adherents. His conventions were never adopted by any authors or publishers, nor were they ever taught in any classrooms. \citet[332]{auty_orthographical_1968} thought that the “enthusiasm for Slavonic unity which marked the national revivals of the early nineteenth century was not powerful enough to impose a single \is{Orthography}orthography (let alone a single language) on all the Slavs”. Herkel’s enthusiasm, certainly, did not suffice to impose anything on anybody.

\ia{Herkel, Jan}Herkel’s work nevertheless attracted some interest among his contemporaries. In an undated letter, Dobrovský \ia{Dobrovský, Josef}wrote with some excitement that \ia{Herkel, Jan}Herkel’s universal grammar had been published, promising that copies would soon be distributed \citep[682]{jagic_pisma_1895}.\footnote{“Letter 16” (undated, but received 6 April 1827), in: \citet[682]{jagic_pisma_1895}.} The book was reviewed in \textit{Rozmaitości}, the cultural supplement to a a \il{Polish}Polish‐language newspaper in L’viv (Lwów) (\citetalias{anon_z_1827}, \linebreak{}\cite[180]{anon_z_1827}). Adam \citeauthor{dragosavljevic_postanak_1840} (1800--1862), a Serbian pedagogue living in Hungary, cited it in an \citeyear{dragosavljevic_postanak_1840} tract about \il{Serbian}\is{Orthography!Orthographic reform}Serbian orthographic reform (\citeyear[91]{dragosavljevic_postanak_1840}). \ia{Bernolák, Anton}Bernolák’s nephew, Andrej \citeauthor{radlinsky_sobranie_1852} (1817--1879), an important Catholic dignitary and language reformer, also alluded to \ia{Herkel, Jan}Herkel in \citeyear{radlinsky_sobranie_1852}, particularly noting the Latin base with Cyrillic letters \citep[xiv]{radlinsky_sobranie_1852}.

\ia{Kollár, Jan}Kollár was particularly effusive about \ia{Herkel, Jan}Herkel’s work. For the expanded 1832 edition of \textit{Slawy dcera}, \citeauthor{kollar_slawy_1832} added a short verse praising Herkel (\citeyear[verse 462]{kollar_slawy_1832}). That same year, a primer explaining his poem to student readers credited \ia{Herkel, Jan}Herkel with having written \textit{Elementa grammaticae linguae slavicae universalis} [sic] \citep[349]{kollar_wyklad_1832}. On 15 January 1842, \ia{Kollár, Jan}Kollár again praised Herkel’s effort in a letter to \ia{Hamuljak, Martin}Hamuljak (\citetalias{anon_hamuljakova_1863}, \cite{anon_hamuljakova_1863}). Most dramatically, \citeauthor{kollar_uber_1837} quoted Herkel’s definition of \is{Pan-Slavism}Pan-Slavism in an \citeyear{kollar_uber_1837} tract on “Literary \is{Reciprocity!Literary reciprocity}Reciprocity” (\citeyear[88--89]{kollar_uber_1837}).\footnote{See also the \il{English}English edition \citet[115--116]{kollar_reciprocity_2009}.} \ia{Kollár, Jan}Kollár, the famous poet, particularly emphasized literary exchange: he wanted Slavs of all “tribes” to read literary works in all \is{Dialect}Slavic “dialects”. To facilitate this goal, he called for comparative grammars and dictionaries (\citeyear[126]{kollar_uber_1837}, cf. \citeyear[133]{kollar_reciprocity_2009}), university chairs (\citeyear[122]{kollar_uber_1837}, cf. \citeyear[131]{kollar_reciprocity_2009}), and most colorfully a “general trans-dialectal literary magazine, in which every new \il{Slavic}Slavic work will be shown and reviewed in the \is{Dialect}dialect in which it was written” (\citeyear[123]{kollar_uber_1837}, cf. \citeyear[132]{kollar_reciprocity_2009}). \citeauthor{kollar_uber_1837} (\citeyear[127--128]{kollar_uber_1837}, cf. \citeyear[134]{kollar_reciprocity_2009}) also advocated “an \is{Orthography!Orthographic unity}uniform and philosophic orthography, based on the \is{Spirit| see {Genius}}\is{Genius!Genius of the Slavic language}spirit of the \il{Slavic}Slavic language, which all Slavs can use, at least for those that use the same letters, the Latin and the Cyrillic”, a reform almost identical to what \ia{Herkel, Jan}Herkel had proposed, even if \ia{Kollár, Jan}Kollár himself did not at that time endorse Herkel’s \is{Orthography}orthography, or indeed any particular \is{Orthography}orthography at all. \citeauthor{kollar_uber_1837} (\citeyear[128]{kollar_uber_1837}, cf. \citeyear[134]{kollar_reciprocity_2009}) confined himself to the hope that Slavs would develop “orthography resting on the \is{Genius!Genius of the Slavic language}spirit of the Slavic language, or at least, for all those who use the same letters, whether Latin or Cyrillic. Neither \il{Hungarian}Magyar, nor \il{Italian}Italian, nor the \il{German}German language should have any influence on \is{Orthography}\il{Slavic}Slavic orthography”.

Other Slavists, however, were less impressed with \ia{Herkel, Jan}Herkel’s proposals. \ia{Šafařík, Pavel}Šafařík, who had gathered funds to support Herkel’s work\ia{Hamuljak, Martin} \citep[68, 75]{matovcik_listy_1965},\footnote{“Letter no. 18 (Pavel Šafařík to Martin Hamuljak), 27 July 1826”, “Letter no. 21 (Pavel Šafařík to Martin Hamuljak), 14 September 1826”, in: \citet[68, 75]{matovcik_listy_1965}.} and then impatiently awaited its publication in the final months of 1826 (\citeyear[85, 95]{matovcik_listy_1965}),\footnote{“Letter 24 (Pavel Šafařík to Martin Hamuljak) 28 October 1826”, “Letter 28 (Pavel Šafařík to Martin Hamuljak) 17 December 1826”, in: \citet[85, 95]{matovcik_listy_1965}.} expressed a bitter disappointment. On 13 July 1827, he declared it “\textit{crazy} – such must be the judgement of every understanding Catholic” (\citeyear[164]{matovcik_listy_1965}).\footnote{“Letter 66 (Pavel Šafařík to Martin Hamuljak) 13 July 1827”, in: \citet[164]{matovcik_listy_1965}.} By 26 September his judgement was less emotional: “Too bad! The author would have made better use of his time and effort if he had given us a \textit{comparative} grammar of the main \is{Dialect}dialects [\textit{Hauptdialecte}], instead he teaches \il{Slavic}us \textit{an entirely new Slavic Language} [\textit{Sprache}], half of which consists of new symbols which he has proposed. Honor to this \textit{pium desiderium} [‘pious wish’]” \citep[310]{francev_korespondence_1927}.\footnote{“5. \ia{Šafařík, Pavel}Šafařík Koeppenovi, 26 September 1826”, in: \citet[310]{francev_korespondence_1927}.} \ia{Hattala, Martin}Martin Hattala (1821--1903), in a 31 July 1871 lecture to the Bohemian Academy of Sciences, also reacted dismissively. \citeauthor{hattala_grammatica_1850}, a noted linguist best known for \is{Codification}codifying a distinctly \il{Slovak}Slovak literary standard (\citeyear{hattala_grammatica_1850, hattala_mluvnica_1865}), somewhat inconsistently declared both that Herkel’s \is{Orthography}orthography was an “extremely arbitrary mingling of \is{Dialect}Slavic dialects” \citep[51]{anon_sitzung_1872}, and that Herkel had based “his would-be \is{Pan-Slavism}Pan-Slavic gibberish” on his “mother tongue [\textit{Muttersprache}]” (\citeyear[52]{anon_sitzung_1872}).

Contemporary linguists echo Hattala’s opinion. Anna-Maria \citet[104]{meyer_wiederbelebung_2014} wrote of \ia{Herkel, Jan}Herkel’s plan that “overall \il{Slovak}Slovak elements predominate, which is unsurprising, since that was the author’s mother tongue”. Tadeusz \citet[80]{lewaszkiewicz_miedzy_2014} thought Herkel had “used grammatical and lexical elements of various \il{Slavic}Slavic languages, but the foundation of his common \il{Slavic}Slavic language was only \il{Slovak}Slovak”. To the best of my knowledge, no scholars have viewed \ia{Herkel, Jan}Herkel’s plans as disproportionately favoring the usage of the Orava region: scholars view the disproportionate influence of Herkel’s native variety as the influence \il{Slovak}of “Slovak”. Such descriptions perhaps reflect the current consensus that there is \il{Slovak}a “Slovak language”, while the variety of the Orava region does not enjoy such recognition.

The subsequent marginalization of \ia{Herkel, Jan}Herkel’s thought, however, derives less from the establishment of \il{Slovak}a “Slovak language” than from the dis-establishment of \il{Slavic}the “Slavic language”. Slavists and linguists now assign the status of “language” to subdivisions of \il{Slavic}Slavic, rather than to the Slavic world as a whole. Scholars have variously posited the “Slavic language family” \citep[99]{milewski_introduction_1973}, the “Slavonic language family” \citep[5, see also 3]{corbett_introduction_1993}, “the Slavic language group” contained within “the Indo-European family” \citep[76]{lipson_russian_1981}, “the Slavic language group” (with “three subfamilies”) \citep[415]{brown_slavic_2006}, “the Slavic phylum” \citep[77]{stolz_chapter_2009}, or even “the Slavonic stock of languages” \citep[73]{fowler_english_1859}. Such disparate formulae reflect the ongoing lack of terminological consensus about how to label a collection of related “languages” (\cite[321]{swadesh_perspectives_1954}; \cite[153]{wurm_new_1975}; \cite[391]{mcgregor_linguistics_2015}). Nevertheless, subsequent scholars have apparently agreed to cease regarding \il{Slavic}Slavic as a single “language”.

Indeed, the idea of \il{Slavic}a “Slavic language” has so completely lost its adherents that contemporary scholars not only disagree with Herkel and the Slavic language reformers of his generation, but sometimes struggle to understand their motives. How, for example, can modern scholars acknowledge \ia{Herkel, Jan}Herkel’s reforms as “language \is{Language planning}planning” if they refuse to acknowledge Slavic as a “language”? Scholars do not expect a “language family”, for example, to require a \is{Orthography}common orthography. Such considerations might explain some of the hostility that scholars have retroactively directed at \ia{Herkel, Jan}Herkel. Endre \citet[76]{arato_slavic_1976}, for instance, wrote that \ia{Herkel, Jan}Herkel had “compiled an \il{Slavic}all-Slavic grammar, with the irrealistic [sic] thought of promoting the cause of the uniform Slav literary language by this. In his grammar he summed up the rules of the non-existing unified \il{Slavic}Slavic language, which were to be accepted by all Slavic peoples”. Perhaps Arató’s disapproval reflects the vigor with which he disbelieved in the notion of a single “Slavic language”.

\ia{Herkel, Jan}Herkel’s work thus had little impact on subsequent Slavic studies. His taxonomy of the Slavic world no longer enjoys any support. As a spelling reformer, Herkel had no detectable impact. Perhaps the most enduring legacy of \ia{Herkel, Jan}Herkel’s \is{Orthography}orthographic proposals, therefore, was the word he coined to champion \is{Pan-Slavism}them: “Pan-Slavism.”

\section{Pan-Slavism: The history of a watchword}

The \is{Pan-Slavism}word “Pan-Slavism” plays a marginal role in \ia{Herkel, Jan}Herkel’s thinking. As \ia{Van Rooy, Raf}Raf Van Rooy argues \hyperref[ch:Van_Rooy]{in this volume}, Herkel’s thought owes more to the concept of \is{Genius!Genius of the Slavic language}the “genius” of the Slavic language. The \is{Pan-Slavism}term “Pan-Slavism” nevertheless went on to have a dramatic and surprising career, not least because it quickly acquired meanings quite different from \ia{Herkel, Jan}Herkel’s original coinage.

When \ia{Herkel, Jan}Herkel defined “the true \is{Pan-Slavism}Pan-Slavism” as “unity in literature”, he neglected many potential forms of “unity”. He did not seek to reconcile the confessional differences between Slavs of different denominations: he ignored religion. He did not seek to improve the legal status of disenfranchised peasants: he ignored the injustice of serfdom, and social inequality generally. Nor did he seek common citizenship by redrawing political frontiers: he ignored statehood. His analysis was exclusively linguistic, and his proposals mostly \is{Orthography}orthographic. \ia{Herkel, Jan}Herkel’s contemporaries, however, quickly adopted his word to describe \is{Irredentism!Slavic irredentism}Slavic irredentism. The revanchist meaning of the \is{Pan-Slavism}word “Pan-Slavism”, though developed mostly by non-Slavs hostile to Slavic aspirations, subsequently became the dominant meaning of the word.

Starting in the 1840s, articles \is{Pan-Slavism}denouncing “Pan-Slavism” began appearing in the \il{Hungarian}Hungarian-language press. \is{Pan-Slavism}Pan-Slavism proved a particularly popular bugbear of the journal \textit{Pesti Hirlap} [‘Pest Gazette’], which the influential \ia{Kossuth, Lajos}Lajos Kossuth (1802--1894) began publishing on 1 January 1841 shortly after his release from prison. \textit{Pesti Hirlap} launched with the financial support of sixty subscribers, but quickly became the most important journal in Hungary: one study estimated that it “reached an estimated 100,000 readers […] out of a total of 136,000 enfranchised nobles and an estimated million literate people in Hungary” \citep[146]{judson_habsburg_2016}, another that “its readership constituted about one-fourth of the estimated 200,000 Hungarians who read newspapers at the time” \citep[199]{sugar_history_1990}. Whatever the true readership figures, \textit{Pesti Hirlap} influenced an unprecedented audience in the Kingdom of Hungary.

On 2 October 1842, \textit{Pesti Hirlap} published an editorial by Ferenc \citet[702--703]{pulszky__1842} that discussed both \ia{Herkel, Jan}Herkel’s grammar and \ia{Kollár, Jan}Kollár’s tract on Slavic \is{Reciprocity!Slavic reciprocity}reciprocity. \ia{Pulszky, Ferenc}Pulszky engaged most directly with \ia{Kollár, Jan}Kollár. Firstly, he saw a “slight contradiction” in a call for “literary unity” that acknowledged linguistic diversity. \ia{Kollár, Jan}Kollár’s desire to replace loanwords with words of Slavic origin does not actually contradict his acceptance of distinct \il{Russian}Russian, \il{Polish}Polish and \il{Czech}Czech literary standards. \ia{Pulszky, Ferenc}Pulszky, however, pretended there was a contradiction in order to ridicule:

\begin{quote}
    the author believes that the amalgamation of all Slav \is{Dialect}dialects into one main literary language is a phantom and protests against it in the name of \is{Reciprocity}reciprocity, yet also hopes from the same \is{Reciprocity}reciprocity that “the general \textit{overcoming of words} from foreign languages, the adoption of genuine and purely Slavic forms, and therefore the approaching of the ideal of a \is{Pan-Slavism}\il{Slavic!Pan-Slavic}\il{Slavic!Pan-Slavonic| see {Pan-Slavic}}Pan-Slavonic language”. \citep[702--703]{pulszky__1842}
\end{quote}

\noindent \citeauthor{pulszky__1842} then told an anecdote about a Gypsy selling a donkey. Asked if a donkey was pregnant, the Gypsy answered no. When the customer turned to leave, the Gypsy insisted that the donkey was pregnant after all. Called out on his contradiction, the Gypsy declared: “if I want, it’s pregnant, and if I want, it’s not” (\citeyear[703]{pulszky__1842}; citing \cite[10, 126]{kollar_uber_1837}, cf. \citeyear[76, 133]{kollar_reciprocity_2009}).

\ia{Pulszky, Ferenc}Pulszky drew attention to this supposed contradiction to depict \ia{Kollár, Jan}Kollár’s tract as fundamentally dishonest. He ascribed a sinister subtext to Kollár’s emphasis on the linguistic and literary: \is{Pan-Slavism}Pan-Slavs, \citeauthor{pulszky__1842} (\citeyear[703]{pulszky__1842}; citing \cite[89]{kollar_uber_1837}, cf. \cite[116]{kollar_reciprocity_2009}) acknowledged, proclaimed purely literary objectives; as shown by \ia{Herkel, Jan}Herkel’s definition of “unity in literature”. But \citet[703]{pulszky__1842} argued that cunning Pan-Slavs secretly nurtured political ambitions, writing of Kollár that “for the \is{Pan-Slavism}author ‘Pan-Slavic’ is everything which concerns and \il{Hungarian}interests [\textit{illet és érdekel}] the \il{German}Slavs (\textit{alle Slawen betreffend und umfassened}), and it is hard to believe that all Slavs are interested and concerned only with literature”. \ia{Pulszky, Ferenc}Pulszky cited \ia{Kollár, Jan}Kollár quite out of context. The full passage, from a chapter explaining “how far should this \is{Reciprocity}reciprocity extend?”, urges that

\begin{quote}
    every educated Slav should have at least a \textit{grammatical-lexical} knowledge of the \is{Dialect}dialects spoken by his brothers. This means knowing the meaning of the words unique to each dialect, their forms, \is{Declension}declensions and \is{Conjugation}conjugations, and the extent to which they deviate from the other \is{Dialect}sister dialects. We do not believe that every Slav must be able to speak all \is{Dialect}Slavic dialects, to say nothing of being able to write books in them. We say only that he should understand the speech of all fellow Slavs, and be able to read every book. Since individual means are not sufficient, we also do not demand that every Slav should buy all books and periodicals appearing in all \is{Dialect}dialects, but only that which is in its way relevant, classic and \is{Pan-Slavism}Pan-Slavic in its content (i.e. concerning and encompassing all Slavs). (\cite[14]{kollar_uber_1837}, cf. \citeyear[78]{kollar_reciprocity_2009})
\end{quote}

\noindent \citet[704]{pulszky__1842} read this paragraph about vocabulary, \is{Declension}declensions, \is{Conjugation}conjugations, and the choice of reading material, and saw a threat to Hungary’s territorial integrity so pressing as to warrant double parenthetical exclamation points: “After all this, I do not think I could find a person in the whole width of Hungary who would dare to claim that \is{Pan-Slavism}Pan-Slavism is nothing but unity in literature (!!)”.

Another editorial in \textit{Pesti Hirlap}, published anonymously in July 1844, characterized \ia{Kollár, Jan}Kollár’s epic poem \textit{Sláwy dcera} as “a literary work composed in the \is{Genius!Genius of the Slavic language}spirit of \is{Pan-Slavism}Pan-Slavism”. The author admitted that the poem made no political claims, but argued nevertheless that literary works “are not actions, but ideas that hold the seeds of future actions, the seeds of a carefully prepared future”. It predicted that if Slavic \is{Reciprocity!Slavic reciprocity}reciprocity succeeded, then “fragments of the Slavic will merge in the civic sense, since the inner life cannot remain an abstract notion, but with time will manifest itself concretely, and since only one purely Slavic throne that is powerful and strong, the Slavic nationality will embrace this throne” (\citetalias{anon_orszaggyules_1844}, \cite[479]{anon_orszaggyules_1844}).

More frequently, however, \textit{Pesti Hirlap} denounced the \is{Pan-Slavism}Pan-Slav danger without bothering to engage with any actual Pan-Slav thinkers. The journal warned its readers that schools in northern Hungary were rife with Slavic propaganda, and that students in Prešov (Eperjes) broke windows and insulted their teachers because of their enthusiasm \is{Pan-Slavism}for “Pan-Slavism” (\citetalias{anon_videki_1841}, \cite[437]{anon_videki_1841}). It characterized Pan-Slavism as “a nest of wasps” whose full extent “had not yet come to light” (\citetalias{anon_megyei_1842}, \cite[640]{anon_megyei_1842}). It called on the government to investigate Pan-Slavism as “an element in the bosom of the nation” which did not move “in a friendly direction” (\citetalias{anon_magyarorszag_1842-1}, \cite[689]{anon_magyarorszag_1842-1}), and because the “Pan-Slav element” opposed “a free nation fighting for the unity of the Magyar homeland” (\citetalias{anon_magyar_1843}, \cite[38]{anon_magyar_1843}). It criticized “the \is{Pan-Slavism}Pan-Slav direction inherent in the teachings of the Moscow school of history”, according to which “Attila was a Russian Tsar” and “the Magyars originally Russians” (\citetalias{anon_magyar_1841}, \cite[98]{anon_magyar_1841}). When the \textit{Allgemeine Augsburger Zeitung} published a “national defence” by Ľudovít \citet{stur_sprachenkampf_1843}, \textit{Pesti Hirlap} responded with an article denouncing “Pan-Slav lies” as “brazen suspicion, false slander, and unfounded lies” \citep[89]{sores_ismet_1844}. Indeed, Pan-Slavs served a bogey even when \textit{Pesti Hirlap} argued for ethnic tolerance: an 1843 article arguing for Jewish legal emancipation declared of Jews that “the people is not alien to Magyar nationality, like the \is{Pan-Slavism}Pan-Slavs and Illyrians” (\citetalias{anon_lxxv_1843}, \cite[704]{anon_lxxv_1843}).

While \textit{Pesti Hirlap} took an exceptional interest in denouncing \is{Pan-Slavism}Pan-Slavism, other Magyar newspapers published similar articles (\citetalias{anon__1840-1}, \cite[365--367]{anon__1840-1}; \citetalias{anon__1840}, \cite[405--406]{anon__1840}; \citetalias{anon_szabolcs_1841}, \cite[81--82]{anon_szabolcs_1841}; \citetalias{anon_az_1842}, \cite[261, 265--266, 271--272]{anon_az_1842}; \citetalias{anon_magyarorszag_1842}, \cite[1]{anon_magyarorszag_1842}; \citetalias{anon_orosz-es_1842}, \cite[186]{anon_orosz-es_1842}). During the 1848 Revolution, one Lutheran newspaper even declared that “Pan-Slavism is not just treason against the homeland, it is killing the homeland” \citep[1]{melczer_czafolat_1848}. Hysteria in the Magyar press reflected the attitude of leading politicians. In a letter of 13 September 1842, count György Andrássy, writing to a Bohemian aristocrat, denounced “the devotees of Russia, the apostles of \is{Pan-Slavism}Pan-Slavism” for their “hatred of Hungary” \citep[27]{thun_stellung_1843}. Baron Miklós \citet[166]{wesselenyi_szozat_1843}, conflating “Russian-Slavic propaganda” and “revolutionary Slavic propaganda” (\citeyear[52--53]{wesselenyi_szozat_1843}), denounced “the idea of a gigantic Slavic republic, or federal monarchy, or smaller independent states” \citep[116]{wesselenyi_stimme_1844}, which had supposedly “become the idol to which millions sacrifice their sighs, but are also prepared to sacrifice their blood” (\citeyear[37]{wesselenyi_stimme_1844}).

The suspicion, fear, and hysteria that characterized Magyar perceptions of Hungary’s Slavs quickly spread to Germany. An 1843 article in Leipzig’s \textit{Die Grenzboten} identified three different types of \is{Pan-Slavism}Pan-Slavism: one based in Moscow seeking to overthrow the Russian dynasty, a “Polish-Russian Pan-Slavism” seeking “the freedom of Poland and with it that of all Slavdom, under a Polish-Russian scepter”, and a Czech \is{Pan-Slavism}Pan-Slavism confusingly described as “a unity of ideas for Slavdom, which is also a physical unity, i.e. through the commonality of a spiri\-tual-political main tendency” (\citetalias{anon_slawischen_1843}, \cite[1486--1487]{anon_slawischen_1843}). Most German periodicals, however, imagined only a revanchist \is{Pan-Slavism!Political Pan-Slavism}political Pan-Slavism. An 1842 essay published in the literary supplement to Leipzig’s \textit{Allgemeine Presse Zeitung} linked “the dangers of Pan-Slavism” to “the relationship of the Slavs to the Russian government and the political importance of current aspirations of the Slavs” (\citetalias{anon_literatur_1842}, \cite[931]{anon_literatur_1842}). In 1843, the literary supplement to Munich’s \textit{Allgemeine Zeitung} conflated “Pan-Slavs and Russophiles”, describing Hungary’s Slavic movement as “coquetry for Russia” \citep[1089]{lukacs_sprachenkampf_1843}. In 1844, Stuttgart’s \textit{Deutsche Vierteljahrs-Schrift} proclaimed that “Slavdom has devised the slogan of\is{Pan-Slavism} Pan-Slavism as the signal for the unification of all Slavic peoples into a great Slavic empire” (\citetalias{anon_stellung_1844}, \cite[121]{anon_stellung_1844}). That same year, \textit{Wiener Zeitung} imagined \is{Pan-Slavism}Pan-Slavism as a “net” in which Greece had been ensnared (\citetalias{anon_griechenland_1844}, \cite[227]{anon_griechenland_1844}); two years later Vienna’s \textit{Illustrierte Zeitung} warned that \is{Pan-Slavism}Pan-Slavism “can set the world in flames” (\citetalias{anon_ueber_1846}, \cite[186]{anon_ueber_1846}).

Slavs active in Hungarian public life took offence at such characterizations, which they typically dismissed as slanders. Zagreb’s \textit{Agramer Zeitung} wrote \is{Pan-Slavism}that “Pan-Slavism has no relevance to the unification of Kingdoms, so I consider it unnecessary to lose a single word over it” (\citetalias{anon_vortag_1844}, \cite[368]{anon_vortag_1844}), and in a subsequent article hopefully proclaimed that “slanders about Russian, \is{Pan-Slavism}Pan-Slavic tendencies […] no longer deceive anybody” (\citetalias{anon_kroatien_1845}, \cite[471]{anon_kroatien_1845}). In \citeyear{kulmer_speech_1843}, Croatian Baron Fran\-jo (Ferencz) \citeauthor{kulmer_speech_1843} (1806--1853), speaking in the Hungarian parliament, complained that the \il{Hungarian}Hungarian press accused anybody of Slavic descent of “Slavism, \is{Pan-Slavism}Pan-Slavism, \is{Illyrianism}Illyrianism, and God knows what other isms have been thought up” \citep[163]{kulmer_speech_1843}. \citet[40]{hoitsy_apologie_1843} attacked \textit{Pesti Hirlap} as the voice of the “the ultra-Magyar party” and accused it of fomenting civic unrest:

\begin{quote}
    Brother is ready to fight against brother, even the son against the father, one hates and suspects the other, simply because the one intends to call himself a “Magyar”, while the other wants to remain a “Hungarian”, even though both know very well that they wish in their hearts for the welfare of their father’s beloved country. \citep[6]{hoitsy_apologie_1843}\footnote{For other references to \textit{Pesti Hirlap}, see \citet[20, 41, 63, 77--79, 82, 93]{hoitsy_apologie_1843}.}
\end{quote}

\newpage

\noindent Noting that accusations of Pan-Slavism could prevent Slavic youth from attending university, \citet[25]{stur_beschwerden_1843} concluded that “the rights of Slavs in Hungary as such were being denied”.\footnote{On \ia{Kossuth, Lajos}Kossuth see also \citet[28--30]{stur_beschwerden_1843}.}

Several of these Slavic national defenses explicitly \is{Pan-Slavism!Political Pan-Slavism}distinguished “political Pan-Slavism” from something literary and thus implicitly \is{Pan-Slavism!Apolitical Pan-Slavism}apolitical. \citet[97]{hoitsy_apologie_1843} wrote that “there are friends of the \is{Pan-Slavism!Literary Pan-Slavism}literary Pan-Slavism in Hungary, but this is a world apart from the political sort”, insisting that “the \is{Pan-Slavism!Political Pan-Slavism}political Pan-Slavism has no friends among us” (\citeyear[99]{hoitsy_apologie_1843}). Nobleman-author Jonáš \citeauthor{zaborsky_predmluva_1851} (1812--1876) similarly disavowed “any civic union of All-Slavia [\textit{Všeslávie}]” while hoping for “literary \is{Reciprocity!Literary reciprocity}reciprocity to take root between Slavs” (\citeyear[ii]{zaborsky_predmluva_1851}). Jan \citet[83]{tenora_vyznam_1885} \is{Pan-Slavism!Political Pan-Slavism}contrasted “political Pan-Slavism” with “spiritual or ecclesiastical Pan-Slavism”. The prolific journalist Daniel \citet[5, 7]{lichard_rozhowor_1861} also \is{Pan-Slavism}defended “Pan-Slavism” and “literary \is{Reciprocity!Literary reciprocity}reciprocity” while attacking “political \is{Pan-Slavism!Political Pan-Slavism}Pan-Slavism”. Such efforts at terminological differentiation, however, did nothing to calm Magyar hysteria.

When the nineteenth-century Anglophone reading public first began discus\-sing \is{Pan-Slavism}Pan-Slavism, they mostly adopted the usage of Hungarian aristocrats. An article by Count László \citeauthor{teleki_hungary_1849}, published in \il{English}English translation in \citeyear{teleki_hungary_1849}, attributed the movement “which has been designated Pan-Slavism” to \ia{Kollár, Jan}Kollár. Initially, \linebreak{}\citet[19]{teleki_hungary_1849} wrote, Pan-Slavism was “an intellectual communion between the scattered nations and tribes of the race, and to establish a literary \is{Reciprocity!Literary reciprocity}reciprocity amongst all the Sclavonic nations. Later, it acquired a political complexion, in which boundless aspirations were breathed of Sclavonian empire”. An \citeyear{anon_reviews_1849} summary of \ia{Kossuth, Lajos}Kossuth’s life in London’s \textit{Athenaeum} criticized \is{Magyarization}Magyarization: “the notion of the Ministry was that it could make all the Hungarians one united people by \textit{Magyarizing} them”, but when describing the resulting “hate and bitterness in nearly all the Slavonic inhabitants of Hungary” claimed that Slavs used \is{Magyarization}Magyarization “as a pretext to conceal their plans inimical to liberty” (\citetalias{anon_reviews_1849}, \cite[855]{anon_reviews_1849}). The memoirs of Therese \citeauthor{pulszky_memoirs_1850}, \ia{Pulszky, Ferenc}Ferenc Pulszky’s Viennese-born wife, published in \il{English}English translation in \citeyear{pulszky_memoirs_1850}, claimed that “Russian machinations, and Polish fantasies […] rapidly spread the idea of ‘Panslavism’ (the political union of all Sclavonians)” \citep[149]{pulszky_memoirs_1850}.

Anglophone observers, perhaps less frightened by Slavic aspirations, sometimes acknowledged dueling definitions of \is{Pan-Slavism}Pan-Slavism. An 1850 review of the continental press, published in London, pondered the future of Pan-Slavism as follows:

\begin{quote}
    Shall it be a \is{Pan-Slavism!Political Pan-Slavism}political Panslavism or united empire of all the Slavonic nations under one flag – say that of Russia, either as Russia now is, or as she may soon be? Or shall it rather be a \is{Pan-Slavism!Literary Pan-Slavism}literary and intellectual Panslavism, based on a political distribution of the whole Slavonic mass into four groups of states, corresponding to the four great centres now existing – a Russian group, a Polish group, a Tchekhish, a Bohemian group, and an Illyrian or Graeco-Slavonian group. All this is mysterious to us; time alone can reveal it. (\citetalias{anon_eastern_1850}, \cite[244]{anon_eastern_1850}; reprinted in \citetalias{anon_eastern_1850-1}, \cite[343]{anon_eastern_1850-1})
\end{quote}

\noindent Therese \citeauthor{robinson_historical_1850} (\citeyear[86]{robinson_historical_1850}), née von Jakob, a German-American linguist familiar with several Slavic literatures, defined \is{Pan-Slavism}Pan-Slavism as “the close connection or union of all the Slavic races among themselves”. When discussing Polish author \ia{Mickiewicz, Adam}Adam Mickiewicz (1798--1855), \citet[294]{robinson_historical_1850} spoke \is{Pan-Slavism}of “Panslavism spiritualized and idealized”. Yet when discussing Russian historian Nikolay Gerasimovich Ustryalov (1805--1870), she equated “the principles of Panslavism” with the tendency “to represent Russia as the central point of the Slavic race” (\citeyear[89]{robinson_historical_1850}).

Various Russian nationalists indeed adopted the \is{Pan-Slavism}word “Pan-Slavism” to describe imperial \is{Russian expansionism}expansionism, as an extensive literature has elsewhere described (\cite{petrovich_emergence_1956}; \cite{fadner_seventy_1962}). A few representative passages from the naturalist, ethnographer, and ideologue \ia{Danilevskij, Nikolaj}Nikolaj Danilevskij (1822--1885) illustrate \is{Pan-Slavism}Pan-Slavism as a theme in Russian thought. Danilevskij, whom Andrzej \citeauthor{walicki_rosyjska_1973} (\citeyear{walicki_rosyjska_1973}; cited from \citeyear[291]{walicki_history_1979}) characterized as “the theorist of \is{Pan-Slavism}Pan-Slavism”, and whose 1869 \textit{Rossija i Evropa} [‘Russia and Europe’] Walicki described as “the first [!] and probably only systematic exposition of \is{Pan-Slavism}Panslavism”, complained that his fellow Russians

\begin{quote}
    shy away from accusations of \is{Pan-Slavism}pan-Slavism, as if an honest Russian man, who understands the meaning and knowledge of the words he pronounces, could ever not be \is{Pan-Slavism}pan-Slavic, that is, would not strive with all his soul to overthrow every yoke from his Slavic brethren, to unite them into one whole. \citep[311]{danilievskij_rossija_1871}
\end{quote}

\noindent The united whole \citet[387]{danilievskij_rossija_1871} wanted would find political expression: he advocated an “All-Slavic federation [\textit{всеславянская федерацiя}]”.

Though \ia{Danilevskij, Nikolaj}Danilevskij \is{Pan-Slavism!Political Pan-Slavism}\is{Pan-Slavism!Linguistic Pan-Slavism}Pan-Slavism had both a political and a linguistic dimension, it owed nothing to \ia{Herkel, Jan}Herkel. Future Russian conquests at the expense of Austria and Turkey, Danilevskij hoped, would

\begin{quote}
    spread knowledge of the \il{Russian}Russian language in Slavic lands after their liberation and a political union in Russia, where friendship will spread in place of hostility; will not friendship undoubtedly increase greatly when the Slavs are given a fraternal helping hand to win their freedom and affirm our common greatness, glory and prosperity? \citep[458]{danilievskij_rossija_1871}
\end{quote}

\noindent \ia{Danilevskij, Nikolaj}Danilevskij anticipated neither diversity of grammar, diversity of vocabulary, nor unique letters to represent the phonological peculiarities of particular Slavic \is{Dialect}dialects. Instead, he simply expected Slavs in the Habsburg and Ottoman lands to adopt \il{Russian}Russian. \ia{Danilevskij, Nikolaj}Danilevskij \is{Pan-Slavism}Pan-Slavism, in short, was indeed indistinguishable from Russian \is{Russian expansionism}expansionism.

Some non-Russian Slavs also proposed political forms \is{Pan-Slavism!Political Pan-Slavism}of “Pan-Slavism”. In \citeyear{krasinski_panslavism_1848}, to give one final example, Lutheran Polish exile Walerjan \citeauthor{krasinski_panslavism_1848}, writing in \il{English}English, used the \is{Pan-Slavism}slogan “Panslavism” to advocate “the voluntary union of Russia and Poland, under the same sovereign” \citep[88--89]{krasinski_panslavism_1848}. He hoped to create “a Slavonic empire sufficiently strong to exercise a decided preponderance over the rest of the continent”, and specifically capable of regaining Polish provinces under German control (\citeyear[88--89]{krasinski_panslavism_1848}). \ia{Krasiński, Walerjan}Krasiński, evidently unfamiliar with Herkel, attributed \is{Pan-Slavism}Pan-Slavism to \ia{Kollár, Jan}Kollár. He understood that \ia{Kollár, Jan}Kollár had advocated only literary objectives. He nevertheless thought literature led straight to politics:

\begin{quote}
    Was it possible that this originally purely intellectual movement, should not assume a political tendency! And was it not a natural consequence, that the different nations of the same race, striving to raise their literary significance, by uniting their separate efforts, should not arrive, by a common process of reasoning, to the idea and desire of acquiring a political importance by uniting their whole race into one powerful empire or confederation, which would insure to the Slavonians [sic] a decided preponderance over the affairs of Europe! (\cite[111--112]{krasinski_panslavism_1848}; see also \cite[101--120]{maxwell_walerjan_2008})
\end{quote}

\noindent Much as \textit{Pesti Hirlap} had predicted, \ia{Krasiński, Walerjan}Krasiński treated literary \is{Pan-Slavism!Literary Pan-Slavism}Pan-Slavism as the precursor of \is{Pan-Slavism!Political Pan-Slavism}political agitation.

\largerpage
By curious coincidence, several modern nationalism theorists also interpret linguistic nationalism as the precursor of political agitation. Several models of non-state-based nationalism posit generalizable \is{Stage theory}stages through which nationalism develops. An influential model proposed by Miroslav \citet[26]{hroch_social_1985}, for example, treated “scholarly interest” as the first \is{Stage theory}stage in a process of mobilization culminating in a “mass national movement”. \is{Stage theory}Stage theories encourage scholars of nationalism to view grammatical \is{Codification}codification, linguistic \is{Codification}standardization, dictionary compilation, and other literary activities as important not in their own right, but merely as the groundwork for something political, implicitly treated as more important. \is{Stage theory}Stage theories encourage the study of literary activism as something that foreshadows political activity. Interpreted in light of such theories, therefore, \ia{Herkel, Jan}Herkel’s \is{Pan-Slavism!Literary Pan-Slavism}literary Pan-Slavism deserves study only insofar as it anticipates \is{Irredentism!Slavic irredentism}Slavic irredentism and Russian \is{Russian expansionism}expansionism.

While \is{Stage theory}stage theories facilitate comparative study and have much to recommend them \citep{maxwell_comparative_2012}, scholars nevertheless err if they treat any \is{Stage theory}stage theory as an unbreakable law. Literary and linguistic activism sometimes anticipates political movements, but not all linguistic or literary activism leads to political agitation and state formation. \is{Stage theory}Stage theories, in other words, must leave room for contingency: activism for a particular national concept may fail, or be supplanted by some other rival concept. \ia{Herkel, Jan}Herkel’s activism for \il{Slavic}the “Slavic language”, as a case in point, did not lead to a \is{Pan-Slavism}Pan-Slavic state. During the First World War, \ia{Herkel, Jan}Herkel’s successors instead promoted a \is{Czechoslovakism}Czechoslovakism that led to the Czechoslovak republic (\cite{locher_nationale_1931}; \cite{maxwell_choosing_2009}). Subsequent generations promoting Slovak particularist nationalist politics, furthermore, successfully founded an independent Slovak state. The ultimate success of Slovak particularist nationalism, however, does not retroactively invalidate the \is{Pan-Slavism}Pan-Slavism of \ia{Herkel, Jan}Herkel’s era.

In general, scholars considering the early phases of Slavic nationalism have not paid enough attention to contingency. The resulting teleological narratives have greatly impeded the study of \is{Pan-Slavism}Pan-Slavism, since scholars wrongly presume literary activists ultimate pursue political ambitions. Hugo \citet[23]{hantsch_pan-slavism_1965}, for example, conceded that \is{Pan-Slavism!Literary Pan-Slavism}Pan-Slavism originally “had no political, but only a literary, meaning”, but still argued that since “Pan-Slavism could reach its goal only if the Austro-Hungarian monarchy fell to pieces […] the actions of Pan-Slavs, therefore, had to be hostile to the monarchy” (\citeyear[25]{hantsch_pan-slavism_1965}). Nothing in \ia{Herkel, Jan}Herkel or \ia{Kollár, Jan}Kollár supports such a conclusion. \ia{Hantsch, Hugo}Hantsch’s error, however, seems more comprehensible in light of \is{Stage theory}stage theories of nationalism. If all literary initiatives actually did lead to political activism, then \is{Pan-Slavism!Literary Pan-Slavism}literary Pan-Slavism would indeed by necessity foreshadow something hostile to the monarchy.

\largerpage
Several factors, then, have conspired to conceal \is{Pan-Slavism!Literary Pan-Slavism}literary Pan-Slavism from the historian’s view. Some scholars anachronistically impose modern Slovak nationalism onto the nineteenth century. Other scholars are interested primarily in high-politics and thus neglect linguistic politics. Still others have taken more interest in Russia than in the \is{Intelligentsia!Slavic intelligentsia}Slavic intelligentsia in northern Hungary, and presume that Russian Pan-Slavs speak for all \is{Pan-Slavism}Pan-Slavs. Perhaps the \is{Slavophobia}Slavophobia of Hungarians, Germans or others has eclipsed the memory of \ia{Herkel, Jan}Herkel and \ia{Kollár, Jan}Kollár. Perhaps other factors are at play? Whatever the cause, the effect is clear: scholars have generally ignored \ia{Herkel, Jan}Herkel’s \is{Orthography!Orthographic reform}orthographic reform scheme, and indeed the complex politics of Slavic language \is{Codification}codification in the early nineteenth century.

Recent reference works, for example, define \is{Pan-Slavism!Political Pan-Slavism}Pan-Slavism as “the movement of aspiration for the union of all Slavs or Slavonic peoples in one political organization” or as “the principle or advocacy of political unification for the Slavic peoples” (\cite[1265]{simpson_compact_1991}; \cite[312]{atkin_wiley-blackwell_2011}). The \textit{Encyclopedia of the United Nations} even claims that “the Czech writer \ia{Herkel, Jan}J. Herkel in 1826” used it “with reference to the aspirations of the Slav peoples for unification” \citep[1762]{osmanczyk_encylopedia_2003}. Another encyclopedia edited by Javier Pérez de Cuéllar, a former Secretary-General of the United Nations, declared that

\begin{quote}
    \is{Pan-Slavism}Pan-Slavism, the call to unite all the peoples of eastern Europe speaking Slav languages, was one of the many powerful and ultimately destructive linguistic national forces which swept Europe in the 19\textsuperscript{th} century. Its ambitions could only be realized by force, as Stalin demonstrated in 1945 when he united Eastern Europe under Soviet control. \citep[128]{cuellar_world_1999}
\end{quote}

\noindent Even specialist studies by Slavic authors accept the revanchist understanding of the term. In \citeyear{lucewicz_ideja_2015}, Ludmiła \citeauthor{lucewicz_ideja_2015} rightfully acknowledged the diversity of Slavic thought, differentiating “the \is{Pan-Slavism}concepts ‘Pan-Slavism’, ‘All-Slavism’, ‘Slavic \is{Reciprocity!Slavic reciprocity}reciprocity’, ‘Slavic unity’, ‘Slavic brotherhood’, etc. There neither was nor is a single interpretation of these concepts”. Łucewicz also acknowledged different ideas of what “unity” between the Slavs might entail. Nevertheless, \citet[69]{lucewicz_ideja_2015} still assumed that unity would take a political form: “1) Some saw in it the possibility of preserving political and/or cultural ties with Russia; 2) others, by contrast, sought to unite Slavic peoples in opposition to Russia.” Modern scholars have thus almost entirely adopted the revanchist and “political” understanding of \is{Pan-Slavism!Political Pan-Slavism}Pan-Slavism.

The current terminological consensus often confuses scholars examining nine\-teenth-century \is{Pan-Slavism}Pan-Slavism. Not only Herkel and \ia{Kollár, Jan}Kollár but the majority of Habsburg Pan-Slavs repeatedly explicitly denounced the \is{Pan-Slavism!Political Pan-Slavism}high-political Pan-Slavism that recent scholarship expects to find. Even during the heady days of the 1848 Revolution, and specifically in the 16 June resolution passed at the Prague \is{Pan-Slavism}Pan-Slav Congress, Slavic patriots insisted that they sought domestic reforms rather than Russian annexation, and sought to calm German fears by explicitly \is{Pan-Slavism!Political Pan-Slavism}renouncing “political Pan-Slavism” \citep[107]{moraczewski_manifest_1848}. The \is{Pan-Slavism}Pan-Slav Congress was, of course, a “political” event, insofar as it sought various constitutional and administrative reforms. Nevertheless, scholars who insist that “Pan-Slavism” implies \is{Irredentism!Slavic irredentism}Slavic irredentism and Russian \is{Russian expansionism}expansionism would, it seems, be forced to conclude that the 1848 \is{Pan-Slavism}Pan-Slav Congress rejected \is{Pan-Slavism}Pan-Slavism.

A few scholars have indeed been so insistent that “Pan-Slavism” seeks \is{Pan-Slavism!Political Pan-Slavism}political unification with Russia that they find the relatively \is{Pan-Slavism!Apolitical Pan-Slavism}apolitical activity of actual Pan-Slavs disappointing. In \citeyear{jopson_introduction_1934}, for example, Norman \citeauthor{jopson_introduction_1934} questioned whether “there ever had been such a thing as \is{Pan-Slavism}Panslavism, in the sense of an alliance and an equality of the Slav peoples” \citep[210]{jopson_introduction_1934}. In \citeyear{batowski_poles_1948}, Henryk \citeauthor{batowski_poles_1948} doubted that “any sort of Panslavism, i.e. of a movement aiming at uniting the Slavs on a racial footing, with a front directed against other nations” (\citeyear[407]{batowski_poles_1948}), had ever played an important role in European politics. In \citeyear{horak_heritage_1963}, Stephan \citeauthor{horak_heritage_1963} even declared that “there is no such thing as \is{Pan-Slavism}Pan-Slavism, i.e., as an organic, racially binding idea” (\citeyear[140]{horak_heritage_1963}). Perhaps \is{Pan-Slavism}Pan-Slavism would reappear in the historical record if scholars would permit the word to have a meaning more closely aligned to the usage of those Slavic patriot-intellectuals who actually espoused it.

Unfortunately, even those few scholars who acknowledge or investigate a \is{Pan-Slavism!Less political Pan-Slavism}less political Pan-Slavism ignore Herkel. Katharina Krosny wrote that

\begin{quote}
    while today generally associated with Russian aspirations for hegemony, the \is{Pan-Slavism}Pan-Slavism that emerged during the \is{Romanticism}Romantic period denotes the \linebreak{}movement of the disparate Slav people of Europe toward the recognition of their common ethnic background, and their various attempts to achieve a common front against the dominant nations of Europe. \citep[849]{krosny_panslavism_2004}
\end{quote}

\noindent \ia{Krosny, Katharina}Krosny acknowledged that \is{Romanticism}Romantic Pan-Slavs eschewed political goals, emphasizing in particular that “while failing to draw up any realistic political goals, \is{Pan-Slavism}Pan-Slavists encouraged their fellow Slavs to learn the four principal Slavonic languages (\il{Czech}Czech, \il{Illyrian} Illyrian, \il{Polish}Polish and \il{Russian}Russian), which they regarded as \is{Dialect}dialects”. She even acknowledged the disproportionate Slovak contribution to early Pan-Slavism. Nevertheless, \ia{Krosny, Katharina}Krosny ignored Herkel entirely. She instead discussed \ia{Kollár, Jan}Kollár and \ia{Šafařík, Pavel}Šafařík, whom she depicted as direct successors of \ia{Herder, Johann Gottfried}Herder.

Our translation, then, seeks to establish \ia{Herkel, Jan}Herkel’s place in Slavic history. We suggest that Herkel’s \is{Linguistic reform}linguistic reform scheme deserves scholarly attention and analysis in its own right, on its own terms. History is more than wars, insurrections, and state-formation: the history of nationalism includes cultural and linguistic initiatives. Linguistics also has a history, perhaps analyzable as part of the history of science. \ia{Herkel, Jan}Herkel’s grammar, we suggest, sheds light not only on the history of linguistic thought, but on Slavic linguistic nationalism generally, since his work illustrates the once widespread belief in a \il{Slavic}single “Slavic language” and its consequences for nationalist \is{Language planning}language planning. His particular proposals, finally, show how the generation of Slavic patriots active in the immediate aftermath of the French Revolution hoped to promote national unity by reforming the national language.
