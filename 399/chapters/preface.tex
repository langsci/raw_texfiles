\addchap{\lsPrefaceTitle}

This volume presents an annotated \il{English}English translation of \textit{Elementa universalis linguae Slavicae} (\textit{Elements of a universal \il{Slavic}Slavic language}, hereafter \textit{Elementa}), an \citeyear{herkel_elementa_1826} appeal for Slavs to reform their language in the name of Slavic unity. Its author, the Slovak lawyer and amateur linguist Jan \citeauthor{herkel_elementa_1826} (Ján Herkeľ, 1786--1858), originally wrote it in \il{Latin}Latin. Ours is the first \il{English}English translation. Furthermore, it is only the second translation into a living language. Our only predecessor is classical philologist Ľudmila \citeauthor{herkel_jan_2009}, who published a \il{Slovak}Slovak translation in \citeyear{herkel_jan_2009}. Our text also provides two essays contextualizing \ia{Herkel, Jan}Herkel’s thought.

\ia{Herkel, Jan}Herkel’s \textit{Elementa} was first published at the Royal University Press in Buda, the ancient capital of the Kingdom of Hungary. The Kingdom of Hungary belonged at that time to the Habsburg Empire, a state whose \is{Ethnolinguistic diversity}ethnolinguistic diversity has since become proverbial. Nineteenth-century Hungary could also boast more \is{Ethnolinguistic diversity}ethnolinguistic diversity than can the twenty-first century Hungarian republic. Before the 1920 Treaty of Trianon partitioned the crownlands of St. Stephen between Romania, Yugoslavia, Austria, and Czechoslovakia, the Hungarian crownlands included a large Romanian population in the east, now part of Romania, German colonies throughout, but also in the west, in a strip of territory now part of Austria, and Slavic populations to the north and south, now divided between Slovenia, Croatia, Serbia, Ukraine, and Slovakia. In \ia{Herkel, Jan}Herkel’s day, Hungarians of different ethnicity often used \il{Latin}Latin as the medium for inter-ethnic communication. \il{Latin}Latin also served as the medium of state administration, jurisprudence and scholarship \citep{almasi_latin_2015}. \ia{Herkel, Jan}Herkel had learned his \il{Latin}Latin primarily as a lawyer, but was neither the first nor the last Slavic patriot to articulate national aspirations in the classical tongue.

\label{sec:Preface Latin discussion}

\ia{Herkel, Jan}Herkel’s \il{Latin}Latin, however, differed significantly from the Latin of \ia{Virgil}Virgil, or even \ia{Erasmus}Erasmus: he used a nineteenth-century Hungarian brand of \il{Latin!Hungarian Latin}Latin, which poses some difficulties for classically-trained Latinists. We do not attempt a full analysis of \ia{Herkel, Jan}Herkel’s Latin here, but a few examples illustrate some of his linguistic peculiarities, as seen from the perspective of \il{Latin!Classical Latin}classical Latin. \ia{Herkel, Jan}Herkel employed several unclassical words, such as \textit{praeinvenire} (p. 4) and \textit{seorsivus} (e.g. p. 150); hypercorrect etymological spellings, such as \textit{exmitto} (e.g. p. 10) and \textit{ethymon} (p. 151); and unexpected word meanings such as \textit{supplere} (p. 162) with the meaning of ‘to supplant’, which Lewis and Short’s classical Latin dictionary renders as ‘to fill up, to complete’ \citep{lewis_latin_1879}. \ia{Herkel, Jan}Herkel’s subject matter, furthermore, calls for technical terminology describing linguistic concepts. In short, \ia{Herkel, Jan}Herkel’s \il{Latin}Latin text does not make a smooth read.

\ia{Herkel, Jan}Herkel’s argumentation also tends to be suggestive, rather than explicit. He frequently leaves steps in his reasoning unexplained, leaving the reader to complete his line of thought. For instance, at the end of §32 (p. 149), \ia{Herkel, Jan}Herkel once more made his methodological point that the \is{Genius!Genius of the Slavic language}genius of the \il{Slavic}Slavic language should be uncovered by making a comparison of different varieties, ending with an abrupt “etc.”, as if he got tired of his own reasoning: “The same should be understood about the other words, and for that reason one should examine the \is{Genius}genius of the language by comparing \is{Dialect}dialects etc.” The abbreviation “etc.” appears no less than 280 times in our translation of the \textit{Elementa}, reflecting Herkel’s original use.

The \textit{Elementa} also contains multiple typographical errors, which, \ia{Herkel, Jan}Herkel ho\-ped, \il{Latin}“the benevolent reader will easily correct”.\footnote{P. 164: “Errata benevolus lector facile emendabit.” See for instance our footnote at \hyperref[sec:3-1-8]{Section I, §8}, where we have tried to act like benevolent readers.} The book’s 164 pages contain dozens of typos, suggesting either carelessness or haste in production. Most of these typos can easily be corrected, such as \textit{protissimum} for \textit{potissimum} on page 43, but one passage contains textual problems which cannot be confidently solved. Page 159 contains the following sentence:

\begin{quote}
    […] sic \textit{e. g. kniaz} apud alios denotat Principem, apud alios sacerdotem; combinatio ta- [sic] hujus usus est facillima; nam Slavi affines sunt Indis orientalibus et linguaa [sic], et mythologia […]
\end{quote}

\begin{quote}
In this way, for instance, \textit{kniaz} for some means ‘prince’, for others ‘priest’. This double use is very straightforward, as the Slavs are related to the Oriental Indians both by language and by mythology.
\end{quote}

The meaning of the passage is more or less clear: the double meaning of \textit{kniaz} can be explained by Slavic affinity with India, where the term \textit{kagan} denoted a prince who governed both worldly and sacred matters. But the syllable \textit{ta}- left us baffled. It appears right before a line break, but is not continued on the next line, where one reads \textit{hujus} (‘of this’, genitive of \textit{hic}). In this case, and other cases where \ia{Herkel, Jan}Herkel’s reasoning is lapidary, we were guided by our judgement of \ia{Herkel, Jan}Herkel’s intended meaning.

In preparing this translation, we have inevitably had to make judgement calls. We sought to replicate \ia{Herkel, Jan}Herkel’s original text faithfully while yet producing a fluent text accessible to \il{English}English speakers. We have normalized capitalization throughout the text: \ia{Herkel, Jan}Herkel wrote important nouns with an initial capital letter, but we have omitted irregular capital letters in our translation. We have retained the \il{Slavic}Slavic \is{Orthography}orthography of \ia{Herkel, Jan}Herkel in the examples and texts he quotes rather than standardizing them following modern norms. We render blackletter typeface fonts with bold type. On pages 27--31, the sections of \ia{Herkel, Jan}Herkel’s \textit{Elementa} are numbered incorrectly. Following Buzássyová, \ia{Buzássyová, Ľudmila}we have deleted “§. 5” so that the numbering is now consistent.

\ia{Herkel, Jan}Herkel apparently assumed his readers understood \il{Slavic}Slavic, but we do not make that assumption, and have thus provided \il{English}English translations for \il{Slavic}Slavic words or phrases. Our glosses are marked by square brackets. We have also given \ia{Herkel, Jan}Herkel’s original page numbers in square brackets. The few \il{Latin}Latin glosses that \ia{Herkel, Jan}Herkel provided himself are left unmarked.

The title page of the \textit{Elementa} \il{Latin}Latinizes Herkel’s name as “Joannes Herkel”. Modern Slovak scholars prefer to spell the Herkel’s name \il{Slovak}as “Ján Herkeľ”, with the long vowel <á> in his given name and the palatalized final <ľ> in his surname. We have, however, opted for “Jan Herkel”, using unaccented <a> and final <l> with no palatalizing diacritic. We follow \ia{Herkel, Jan}Herkel’s own precedent. As a supporter of Slavic patriotic causes, his name appears in the subscriber lists of several patriotic publications. \il{Slavic}Slavic works published in both Buda and Prague give his name as “Jan Herkel” (\citetalias{anon_gmena_1827}, \cite[no page numbers]{anon_gmena_1827}; \cite[no page numbers]{kollar_rozprawy_1830}). Only when contributing to \il{Serbian}Serbian causes did \ia{Herkel, Jan}Herkel append a palatalizing symbol: he signed his name in \il{Serbian}Serbian Cyrillic “Іоаннъ Херкелъ” \citep[no page numbers]{pacic_dodatak_1827}. The modern \il{Slovak!Modern Slovak}“Slovak” spelling inappropriately associates Herkel with twenty-first century Slovak nationalism. Since Herkel is the original \is{Pan-Slavism}Pan-Slav, we prefer to distance him from subsequent particularist nationalisms.

For similar reasons, we have refrained from adapting \ia{Herkel, Jan}Herkel’s ethnonyms (or glottonyms) to contemporary thinking. When Herkel discusses the speech of \il{Serbian}\textit{Vindi}, for example, he draws on the linguistic works of \ia{Kopitar, Jernej}Jernej Kopitar, a notable Slavic scholar who was born in Carinthia and educated partly in Ljubljana. Herkel also associates \ia{Kopitar, Jernej}Kopitar’s work with the speech of \il{Carinthian}Carinthia, \il{Carniolan}Carniola, and \il{Styrian}Styria. In short, Herkel is referring to that part of the Slavic world that now comprises the Republic of Slovenia. Some scholars, therefore, might expect us to gloss \textit{Vindi} as “Slovenes”. We have, however, consistently chosen an ethnonym (glottonym) with a shared etymology, and in this case glossed \textit{Vindi} as “Winds”. We similarly glossed Herkel’s \textit{Pannonii} as “Pannonian”, rather than “Slovaks” or “Hungaro-Slavs”.

The importance of glossing ethnonyms (glottonyms) with an etymologic translation can perhaps best be illustrated explained through examples. When comparing \textit{Bohemi} to \textit{Poloni} and \textit{Pannonii}, \ia{Herkel, Jan}Herkel apparently contrasted “Czechs” with Poles and Slovaks. Yet in other passages, Herkel juxtaposed \textit{Bohemi} with \textit{Moravi}, thus distinguishing “Bohemians” from “Moravians”. The politics of ethnonyms (glottonyms) can be subtle. We thought it a mistake to introduce our judgements about the distinction between “Bohemian” and “Czech” into Herkel’s text. Etymological translation, we reasoned, provides an essentially unmediated window into Herkel’s usage. When Herkel distinguishes \textit{Croatae} from \textit{Slavonitae} and \textit{Dalmatae}, therefore, our text distinguishes Croatians from Slavonians and Dalmatians.

We have been somewhat less puristic when translating Herkel’s grammatical terminology. Most importantly, throughout the text, for example, we have translated \textit{socialis} as “instrumental”, following current usage. We have otherwise retained here, too, as much as possible Herkel’s own terms. Herkel’s linguistic analysis rests on the work of previous grammarians. \ia{Herkel, Jan}Herkel reproduced \is{Declension}declensions and \is{Conjugation}conjugations found in the work of his predecessors, whom he described as \textit{grammatici} or \is{Dialect}\textit{dialectici}, terms we have respectively glossed as “grammarians” and “dialect grammarians”. Where possible, we have identified those sources and offered bibliographical details in footnotes. In order to ease the navigation of our \il{English}English translation, we have included in the running text Herkel’s original page numbers between square brackets.

Our translation owes a significant debt to other scholars. We frequently consulted Antal \citeauthor{bartal_glossarium_1901}’s \textit{Glossarium mediae et infimae Latinitatis regni Hungariae} (first edition in \citeyear{bartal_glossarium_1901}), a specialized dictionary of \il{Latin!Hungarian Latin}Hungarian Latin.\footnote{Like \ia{Lewis, Charlton T.}Lewis and \ia{Short, Charles}Short’s \il{Latin!Classical Latin}classical Latin dictionary, we consulted this Hungarian Latin dictionary through \textsc{BREPOL}iS’ \textit{Database of Latin Dictionaries} (<\url{https://clt.brepolis.net/dld/Dictionaries/Search}>), last accessed on 8 June 2023.} For instance, Bartal helped greatly to understand the meaning of the Latinized \il{Greek}Greek word \textit{cynosura}, which Herkel used to mean ‘norm’, not in the classical sense of the ‘dog’s tail’, an ancient reference to Ursa Minor (or in the meaning of \textit{cynosura ova}, ‘addled-eggs’). Buzássyová’s \ia{Buzássyová, Ľudmila}footnotes provided invaluable guidance about \ia{Herkel, Jan}Herkel’s predecessors, and the text of her translation clarified several obscure passages for us, even if our interpretations sometimes differ from\ia{Buzássyová, Ľudmila} Buzássyová’s. Any errors of course remain our responsibility. Like Herkel, we trust the benevolent reader will easily correct any mistakes that may have remained.

The two essays accompanying Herkel’s translation reflect the respective background and expertise of the two scholars who prepared the translation. Alexander Maxwell is a historian specializing in the emergence of nationalism in the Habsburg lands; \ia{Van Rooy, Raf}Raf Van Rooy is a Neo-Latin scholar interested in the history of linguistic thought and the interplay between language and literature. Our collaboration arose from a shared fascination with the \is{Dialect!Language/dialect dichotomy} language/dialect dichotomy. In defiance of twentieth- and twenty-first century consensus opinion about the “Slavic language family”, Herkel posited a single \il{Slavic}and \is{Unitary Slavic language}unitary “Slavic language”, and analysed \il{Russian}Russian, \il{Polish}Polish, \il{Serbian}Serbian and so forth \is{Dialect}as “dialects” of that language. Both of us, for our own reasons, found \ia{Herkel, Jan}Herkel’s linguistic taxonomy fascinating.

Maxwell provides a biography of Herkel in the context of Slavic intellectual life in the Kingdom of Hungary. He explains the emergence of \is{Pan-Slavism!Linguistic Pan-Slavism}linguistic Pan-Slavism with reference to Herkel’s predecessors and contemporaries, discusses Herkel’s proposals in the context of \il{Slavic}Slavic \is{Language planning}language planning, warns against interpreting early nineteenth-century Hungaro-Slavic thought with the analytical categories generated by twentieth and twenty-first century nationalism, and ends with a brief history of Herkel’s term \is{Pan-Slavism}\textit{Panslavismus}, which is perhaps his most enduring legacy.

\ia{Van Rooy, Raf}Van Rooy focuses instead on the concept of \is{Genius}“genius”. Herkel justified his various proposals with reference to \is{Genius!Genius of the Slavic language}the “genius of the \il{Slavic}Slavic language”, a concept located at the core of his understanding of Slavic linguistics, and impregnated by both the \is{Enlightenment}Enlightenment and \is{Romanticism}Romanticism. Van Rooy contextualizes Herkel’s place in the history of linguistic thought on the \is{Genius}“genius” concept from antiquity through the Middle Ages and early modern period to \ia{Herkel, Jan}Herkel’s transformation of it in view of his \is{Pan-Slavism}Pan-Slavist ideas.

This volume would not have been possible without the generous support of an FWO senior postdoctoral fellowship at KU Leuven (2020--2022), an MSCA-IF of the European Commission at the University of Oslo (2021--2022), and a research professorship at KU Leuven (2022--2027). Our warmest thanks go out to \ia{Seldeslachts, Herman}Herman Seldeslachts for general advice and numerous corrections throughout the translation and for pointing out certain allusions (e.g. to Augustine), to Alicja Bielak for her help with \il{Polish}Polish, and Richard Millington for insight into \il{Polish}Polish and \il{Russian}Russian. We also thank Ľudmila Buzássyová \ia{Buzássyová, Ľudmila}for generously sending us copies of her book. We would also like to express our gratitude to James McElvenny and the other members of the editorial board of the Language Science Press series “History and Philosophy of the Language Sciences” for critical remarks on earlier drafts of this volume. Finally, we are grateful to Dustin Saynisch for his careful management of the typesetting process.