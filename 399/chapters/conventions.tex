\chapter{Note on conventions}

Central European cities have different names in different languages. A few important cities, such as Vienna and Prague have \il{English}English names whose use in Anglophone scholarship is widely accepted. Most towns, however, do not, so Anglophone scholars face a choice between competing names. \ia{Herkel, Jan}Herkel’s birthplace, for example, has the \il{Hungarian}Hungarian name “Vavrecska” and the \il{Slovak}Slovak name “Vavrečka”. What to do? Central European scholars cannot agree on a convention for selecting placenames, but more annoyingly also cannot agree to disagree. Authors thus face the tedious task of justifying their conventions, lest their choices be construed as mistakes.

In this book, we have opted to prioritize contemporary names, that is, the name of the state language of the government that administers the city at the time of writing. In practice, therefore, we use \il{Slovak}Slovak names to discuss places that belonged to the Kingdom of Hungary during the nineteenth century. In the main text, however, we also mention \il{Hungarian}Hungarian names either by providing an explanation of the multiple names of important cities, or by listing \il{Hungarian}Hungarian names in parentheses. Much contemporary scholarship acknowledges names in multiple languages so as to emphasize the multilingual quality of urban life in the Habsburg domains.

In the references, however, we list only contemporary names. Our bibliography, for example, anachronistically claims that several nineteenth-century books were published in the town of “Bratislava”. even though the name “Bratislava” was adopted only with the establishment of the first Czechoslovak Republic. Hungary’s former capital is known in Hungarian as “Poszony”. The town’s predominantly German population referred in \il{German}German to “Pressburg”. Even nineteenth-century Slavs, following \il{German}German usage, called the city “Prešporok” or “Prešporek”. Our anachronistic bibliography also retroactively anticipates the 1873 unity of “Budapest” when listing books published at a time when Pest and Buda were separate towns. We accept such anachronism as a price worth paying in order to assist readers who want to search for a town on a contemporary map.