% Appendix A
\chapter{Stimuli}\label{app:AppendixA2} 

\begin{exe}
	\ex \label{ex:experimentoNEUTRALDECLlimonada_APP}
	Una camarera te pregunta qué quieres beber. 
	\glt `A waitress asks you what you want to drink.' 
	\begin{xlist}[A:]
	\exi{A:} ¿Qué te pongo? \href{https://osf.io/jpx7c/}{\faVolumeUp}
	\glt `What can I bring you?' 
	\exi{B:} \textbf{Bebo una limonada.} 
	\glt `I'm having a lemonade.' 
	\end{xlist}
\ex \label{ex:experimentoMIRDECLlimonada_APP}
	Te has pedido una caipiriña en un bar. Una amiga tuya se ha pedido una limonada. Cuando llegan las bebidas, te das cuenta que parecen ser iguales. Pruebas las dos y son iguales. Esto no te lo esperabas. 
	\glt `You've ordered a caipirinha at a bar. A friend of yours has ordered a lemonade. When the drinks arrive, you notice that they look the same. You try them both and they actually are the same. You didn't expect that.' 
	\begin{xlist}[A:]
	\exi{B:} \textbf{Tienen pinta similar. Es un poco raro, ¿no?} 
	\glt `They seem similar. That's a little weird, right?' 
	\exi{A:} Pues, da igual. El tuyo contiene alcohol. Esto no se ve. ¡Salud! \href{https://osf.io/acrdh/}{\faVolumeUp} 
	\glt `Well, no matter. Yours has alcohol in it. That can't be seen. Cheers!' 
	\exi{B:} \textbf{¡Salud!} 
	\glt `Cheers!' 
	\exi {} \ldots  
	\exi{A:} ¿Y qué tal tu caipiriña? \href{https://osf.io/a2xsu/}{\faVolumeUp}
	\glt `And how's your caipirinha?' 
	\exi{B:} \textbf{¡Es una limonada!} 
	\glt `(It) is a lemonade!' 
	\end{xlist}
\ex \label{ex:experimentoWHEXCLlimonada_APP}
	Una amiga te ofrece una limonada. Normalmente no te gusta mucho la limonada, pero esta es fantástica. Díselo a tu amiga. 
	\glt `A friend offers you a lemonade. Normally you don't like lemonade that much, but this one is fantastic. Tell your friend.' 
	\begin{xlist}[A:]
	\exi{A:} ¡Toma! Es mi nueva receta. \href{https://osf.io/bsyq3/}{\faVolumeUp}
	\glt `Here! That's my new recipe.' 
	\exi{B:} \textbf{¡Qué buena limonada!} 
	\glt `What a great lemonade!' 
	\end{xlist}
\ex \label{ex:experimentoOBVDECLlimonada_APP}
	En tu café preferido siempre bebes limonada. Llega la camarera que te conoce desde hace años y que es buena amiga tuya. Te pregunta qué quieres beber, aunque debería saberlo. Dile que bebes limonada, y hazle notar francamente que debería saberlo. 
	\glt `You always drink lemonade in your favorite cafe. The waitress arrives who has known you for years and is a good friend of yours. She asks you what you would like to drink, even though she should know. Tell her that you drink lemonade, and make her be aware openly that she should know that.' 
	\begin{xlist}[A:]
	\exi{A:} ¿Qué te pongo? \href{https://osf.io/vf4m2/}{\faVolumeUp}
	\glt `What can I bring you?' 
	\exi{B:} \textbf{Bebo una limonada.} 
	\glt `I'm having a lemonade.' 
	\end{xlist}
\ex \label{ex:experimentoOBVCONFlimonada_APP}
	Estás en una fiesta con un grupo de amigos y te encargaste de llevarlos en el coche. Una amiga te ve bebiendo algo que podría ser una caipiriña. Aunque todo el mundo sabe que no bebes ni gota cuando conduces, te pregunta si es una limonada. Dile que bebes una limonada, y hazle notar francamente que debería saberlo. 
	\glt `You're at a party with a group of friends and you are the designated driver. A friend sees you drinking something that could be a caipirinha. Even though everybody knows that you don't drink even a single drop when driving, she asks if it's a lemonade. Tell her that you're having a lemonade, and make her be aware openly that she should know that..' 
	\begin{xlist}[A:]
	\exi{A:} Es una limonada lo que tomas, ¿verdad? \href{https://osf.io/m6q4j/}{\faVolumeUp}
	\glt `You're having a lemonade, right?' 
	\exi{B:} \textbf{Bebo una limonada. (¿Qué va a ser si no?)} 
	\glt `I'm having a lemonade. (What else should it be?)' 
	\end{xlist}
\largerpage
\ex \label{ex:experimentoOBVDENlimonada_APP}
	Estás en una fiesta con un grupo de amigos y te encargaste de llevarlos en el coche. Un amigo te ve bebiendo una limonada, pero piensa que es una caipiriña, aunque todo el mundo sabe que no bebes ni gota cuando conduces. Dile que bebes una limonada, y hazle notar francamente que debería saberlo. 
	\pagebreak\glt `You're at a party with a group of friends and you are the designated driver. A friend sees you drinking a lemonade, but thinks that it's a caipirinha, even though everybody knows that you don't drink even a single drop when driving. Tell her that you're having a lemonade, and make her be aware openly that she should know that..' 
	\begin{xlist}[A:]
	\exi{A:} Es una caipiriña lo que tomas, ¿verdad? \href{https://osf.io/6te3x/}{\faVolumeUp}
	\glt `You're having a caipirinha, right?' 
	\exi{B:} \textbf{Bebo una limonada.} 
	\glt `I'm having a lemonade.' 
	\end{xlist}
\ex \label{ex:experimentoNEUTRALDECLgobierno_APP}
	Hablas con una amiga tuya que ha vivido parte de su vida en otro país y por eso no puede saber todos los detalles sobre la política de España. Ella tiene una pregunta. 
	\glt `You're talking to a friend of yours who has lived part of her life in another country and therefore cannot know all the details of Spanish politics. She has a question.' 
	\begin{xlist}[A:]
	\exi{A:} Oye, tengo una pregunta. \href{https://osf.io/whnsu/}{\faVolumeUp} 
	\glt `Listen, I have a question.' 
	\exi{B:} \textbf{Sí, dime.} 
	\glt `Yes, tell me.' 
	\exi{A:} ¿Quién es Pedro Sánchez? \href{https://osf.io/3uw7x/}{\faVolumeUp}
	\glt `Who's Pedro Sánchez?' 
	\exi{B:} \textbf{Es el presidente del gobierno.} 
	\glt `(He) is the prime minister.' 
	\end{xlist}
\ex \label{ex:experimentoMIRDECLgobierno_APP}
	Una amiga te invita a una fiesta en una fundación. Un hombre en el bar parece estar borracho. Dice que es el presidente. Tu amiga pregunta si es el presidente de la fundación, y vas al bar para hablarle y a lo mejor llamarle un taxi. Cuando llegas, te das cuenta de que es el presidente del gobierno. Esto no te lo esperabas.  
	\glt `A friend invites you to a party at a foundation. A man at the bar seems to be drunk. He says that he's the president. Your friend asks if he's the president of the foundation, and you go to the bar to talk to him and perhaps call him a cab. When you arrive, you become aware that he's the prime minister. You didn't expect that.' 
	\begin{xlist}[A:]
	\exi{A:} ¿Viste al tipo este? Parece ser el presidente de la fundación. 
	\glt `Did you see that guy? He seems to be the president of the foundation.' \href{https://osf.io/6ytnk/}{\faVolumeUp}
	\exi{B:} \textbf{Déjame ver si le puedo ayudar \ldots} 
	\glt `Let me see if I can help him \ldots'\\
	\ldots
	\exi{A:} ¿Y le hablaste? ¿De verdad es el presidente de la fundación?  
	\glt `And did you talk to him? Is he really the president of the foundation?'  \href{https://osf.io/92hp4/}{\faVolumeUp}
	\exi{B:} \textbf{¡Es el presidente del gobierno!} 
	\glt `(He) is the prime minister!'
	\end{xlist}
\ex \label{ex:experimentoWHEXCLgobierno_APP}
	Sale en las noticias un escándalo de corrupción. Le dices a una amiga: 
	\glt `There's a corruption scandal on the news. You tell a friend:' 
	\begin{xlist}[A:]
	\exi{B:} \textbf{¡Qué horror de gobierno!} 
	\glt `What a horrible government!' 
	\exi{A:} Sí, son unos sinvergüenzas. \href{https://osf.io/uz4dc/}{\faVolumeUp}
	\glt `Yes, they're shameless.' 
	\end{xlist}
\ex \label{ex:experimentoOBVDECLgobierno_APP}
	En la tele ponen una entrevista con el presidente del gobierno de España. La amiga que está mirando la tele contigo siempre se cree la más lista de todos. Ella te pregunta quién es, aunque todo el mundo lo conoce. Dile quien es y hazle sentir que debería saberlo. 
	\glt `They're showing an interview with the prime minister of Spain on TV. The friend who's watching TV with you always thinks of herself as the smartest of all. She asks you who is it, even though everybody knows him. Tell her who he is, and give her the feeling that she should know that.' 
	\begin{xlist}[A:]
	\exi{A:} ¿Quién es este hombre? \href{https://osf.io/zan4g/}{\faVolumeUp}
	\glt `Who is that man?' 
	\exi{B:} \textbf{Es el presidente del gobierno. ¿Cómo que no lo conoces?} 
	\glt `(He) is the prime minister. How come you don't know him?' 
	\end{xlist}
\ex \label{ex:experimentoOBVCONFgobierno_APP}
	En la tele ponen una entrevista con el presidente del gobierno de España. La amiga que está mirando la tele contigo siempre se cree la más lista de todos. Ella te pregunta si sabes quién es, aunque todo el mundo lo conoce.
	\glt `They're showing an interview with the prime minister of Spain on TV. The friend who's watching TV with you always thinks of herself as the smartest of all. She asks you if you know who it is, even though everybody knows him.' 
	\begin{xlist}[A:]
	\exi{A:} Sabes quién es, ¿verdad? \href{https://osf.io/qjt98/}{\faVolumeUp}
	\glt `You know who he is, right?' 
	\exi{B:} \textbf{Es el presidente del gobierno.} 
	\glt `(He) is the prime minister.' 
	\end{xlist}
\ex \label{ex:experimentoOBVDENgobierno_APP}
	En la tele ponen una entrevista con el presidente del gobierno de España. La amiga que está mirando la tele contigo siempre se cree la más lista de todos. Ella dice que no sabes quién es, aunque todo el mundo lo conoce. 
	\glt `They're showing an interview with the prime minister of Spain on TV. The friend who's watching TV with you always thinks of herself as the smartest of all. She says you don't know who it is, even though everybody knows him.' 
	\begin{xlist}[A:]
	\exi{A:} No sabes quién es ¿verdad? \href{https://osf.io/3mxry/}{\faVolumeUp}
	\glt `You don't know who he is, right?' 
	\exi{B:} \textbf{Es el presidente del gobierno. ¿Cómo no lo voy a conocer?} 
	\glt `(He) is the prime minister. How would I not know him?' 
	\end{xlist}
\ex \label{ex:experimentoNEUTRALDECLalemana_APP}
	Con una amiga estás resolviendo un crucigrama. Te pregunta de dónde viene Adidas. 
	\glt `You're solving a crossword puzzle with a friend. She asks you where Adidas is from.' 
	\begin{xlist}[A:]
	\exi{A:} ¿Oye, de dónde es Adidas? \href{https://osf.io/f6m5g/}{\faVolumeUp}
	\glt `Listen, where is Adidas from?' 
	\exi{B:} \textbf{Adidas es una empresa alemana.}
	\glt `Adidas is a German company.' 
	\end{xlist}
\ex \label{ex:experimentoMIRDECLalemana_APP}
	Con una amiga estás resolviendo un crucigrama. Buscáis una empresa alemana de automóviles con cuatro letras. Queréis poner Audi, pero no entra con el resto del crucigrama. Buscas en línea y te das cuenta de que Seat forma parte del grupo Volkswagen. Esto no te lo esperabas. 
	\glt `You're solving a crossword puzzle with a friend. You're looking for a German automotive company with four letters. You want to write down Audi, but it doesn't fit with the rest of the puzzle. You search online and notice that Seat is part of the Volkswagen Group. You didn't expect that.' 
	\begin{xlist}[A:]
	\exi{A:} Audi no entra. Y Seat no puede ser. \href{https://osf.io/nxywh/}{\faVolumeUp}
	\glt `Audi doesn't fit. And it can't be Seat.' 
	\exi{B:} \textbf{Espera, lo busco en internet.} 
	\glt `Wait, I'll check online.' 
	\exi{A:} Vale. \href{https://osf.io/bgezq/}{\faVolumeUp}
	\glt `OK.' 
	\exi{B:} \textbf{¡Seat es una empresa alemana!} 
	\glt `Seat is a German company!' 
	\end{xlist}
\ex \label{ex:experimentoEXCLalemana_APP}
	Con una amiga quieres ir a Múnich para el Oktoberfest. Cuando queréis ir al aeropuerto, ella llega vestida de trajes típicos bávaros, con sombrero y todo. 
	\glt `With a friend, you want to go to Munich for the Oktoberfest. When you want to go to the airport, she arrives dressed in typically Bavarian clothes, with a hat and all.' 
	\begin{xlist}[A:]
	\exi{A:} ¿Te gusto así? \href{https://osf.io/ghxn2/}{\faVolumeUp}
	\glt `Do I look good to you like this?' 
	\exi{B:} \textbf{¡Qué buena alemana!}  
	\glt `What a great German!' 
	\end{xlist}
\ex \label{ex:experimentoOBVASSalemana_APP}
	Sale en las noticias que viene Merkel a Madrid. Una amiga tuya siempre se cree la más lista de todos y se comporta como un verdadero sabelotodo. Te pregunta si es de Inglaterra o de Alemania, aunque todo el mundo lo sabe. Dile de dónde es y hazle sentir que debería saberlo. 
	\glt `It's in the news that Merkel is coming to Madrid. A friend of yours always thinks she's the smartest of all and behaves like a real know-it-all. She asks if Merkel's from England or Germany, even though everbody knows that. Tell her where she's from and give her the feeling that she should know that.' 
	\begin{xlist}[A:]
	\exi{A:} Oye, ¿Merkel es inglesa o alemana? \href{https://osf.io/bx2hw/}{\faVolumeUp}
	\glt `Listen, is Merkel English or German?' 
	\exi{B:} \textbf{Merkel es alemana.} 
	\glt `Merkel is German.' 
	\end{xlist}
\ex \label{ex:experimentoOBVCONFalemana_APP}
	Sale en las noticias que viene Merkel a Madrid. Una amiga tuya siempre se cree la más lista de todos y se comporta como un verdadero sabelotodo. Cuando quiere asegurarse de que Merkel es alemana, le haces sentir que debería saberlo. 
	\glt `It's in the news that Merkel is coming to Madrid. A friend of yours always thinks she's the smartest of all and behaves like a real know-it-all. When she wants to make sure that Merkel's German, you give her the feeling that she should know that.' 
	\begin{xlist}[A:]
	\exi{A:} Oye, Merkel es alemana, ¿verdad? \href{https://osf.io/juzy8/}{\faVolumeUp}
	\glt `Listen, Merkel is German, right?' 
	\exi{B:} \textbf{Merkel es alemana.} 
	\glt `Merkel is German.' 
	\end{xlist}
\ex \label{ex:experimentoOBVDENalemana_APP}
	Sale en las noticias que viene Merkel a Madrid. Una amiga tuya siempre se cree la más lista de todos y se comporta como un verdadero sabelotodo. Tu amiga piensa que Merkel es del Reino Unido, aunque todo el mundo sabe que es alemana. Hazle sentir que debería saberlo. 
	\glt `It's in the news that Merkel is coming to Madrid. A friend of yours always thinks she's the smartest of all and behaves like a real know-it-all. Your friend thinks that Merkel is from Great Britain, even though everybody knows she's German. Give her the feeling she should know that.' 
	\begin{xlist}[A:]
	\exi{A:} Merkel es inglesa, ¿sabes? \href{https://osf.io/ctxg3/}{\faVolumeUp}
	\glt `Merkel is English, you know?' 
	\exi{B:} \textbf{Merkel es alemana.} 
	\glt `Merkel is German.' 
	\end{xlist}
\ex \label{ex:experimentoNEUTRALDECLmandarin_APP}
	Con una amiga pasas el día en una finca. Os sentáis debajo de un árbol y ella te pregunta qué tipo de árbol será. 
	\glt `You spend the day with a friend on an estate. You sit down under a tree and she asks you what kind of tree it might be.' 
	\begin{xlist}[A:]
	\exi{A:} ¿Qué tipo de árbol será? \href{https://osf.io/yvxm8/}{\faVolumeUp}
	\glt `What kind of tree might it be?' 
	\exi{B:} \textbf{Es un mandarino.} 
	\glt `(It) is a tangerine tree.' 
	\end{xlist}
\ex \label{ex:experimentoMIRDECLmandarin_APP}
	Con una amiga pasas el día en una finca. Desde lejos observáis tres caballos debajo de unos árboles. Parecen buscar algo en particular, porque no se comen el pasto. Te acercas para ver qué comen, y son mandarinas. Esto no te lo esperabas. 
	\glt `You spend the day with a friend on an estate. From afar, you observe three horses under some trees. They seem to be looking for something, because they're not eating the grass. You get closer to see what they are eating, and it's tangerines. You didn't expect that.' 
	\begin{xlist}[A:]
	\exi{A:} ¿Qué estarán comiendo ellos? \href{https://osf.io/6n8ep/}{\faVolumeUp}
	\glt `What could it be that they're eating?' 
	\exi{B:} \textbf{No sé. A ver si me puedo acercar.} 
	\glt `I don't know. Let's see if I can get closer.' 
	\exi{A:} Serán unas hierbas. \href{https://osf.io/quhsx/}{\faVolumeUp}
	\glt `Probably some herbs.' 
	\exi{B:} \textbf{¡Comen mandarinas!} 
	\glt `They're eating tangerines!' 
	\end{xlist}
\ex \label{ex:experimentoEXCLmandarin_APP}
	Una amiga te hace probar las mandarinas de su jardín. En tu vida has probado mandarinas tan buenas y jugosas. 
	\glt `A friend lets you try the tangerines from her garden. You've never tasted tangerines this good and juicy.' 
	\begin{xlist}[A:]
	\exi{A:} ¡Prueba estas! A ver si te gustan. \href{https://osf.io/awczv/}{\faVolumeUp}
	\glt `Try these! Let's see if you like them.' 
	\exi{B:} \textbf{¡Qué buenas mandarinas!}  
	\glt `What tasty tangerines!' 
	\end{xlist}
\ex \label{ex:experimentoOBVASSmandarin_APP}
	Estás haciendo una visita en un jardín botánico con unos amigos. Una amiga te pregunta si las mandarinas son frutas del mandarino o si son frutas jóvenes del naranjo. Hazle sentir que debería saberlo. 
	\glt `You're visiting a botanical garden with some friends. A friend asks you if tangerines are the fruits of the tangerine tree or young fruits of the orange tree. Give her the feeling that she should know that.' 
	\begin{xlist}[A:]
	\exi{A:} ¿Las mandarinas son frutas del mandarino o son naranjas pequeñas? \href{https://osf.io/v63xk/}{\faVolumeUp}
	\glt `Are tangerines fruits of the tangerine tree or small oranges?' 
	\exi{B:} \textbf{Las mandarinas son frutas del mandarino.}  
	\glt `Tangerines are fruits of the tangerine tree.' 
	\end{xlist}
\ex \label{ex:experimentoOBVCONFmandarin_APP}
	Estás haciendo una visita en un jardín botánico con unos amigos. Una amiga quiere asegurarse que las mandarinas son frutas del mandarino. Hazle sentir que debería saberlo. 
	\glt `You're visiting a botanical garden with some friends. A friend wants to make sure that tangerines are the fruits of the tangerine tree. Give her the feeling that she should know that.' 
	\begin{xlist}[A:]
	\exi{A:} Las mandarinas son frutas del mandarino, ¿verdad? \href{https://osf.io/ay4n7/}{\faVolumeUp}
	\glt `Tangerines are fruits of the tangerine tree, right?' 
	\exi{B:} \textbf{Las mandarinas son frutas del mandarino.}  
	\glt `Tangerines are fruits of the tangerine tree.' 
	\end{xlist}
\ex \label{ex:experimentoOBVDENmandarin_APP}
	Estás haciendo una visita en un jardín botánico. Una amiga no cree que las mandarinas son frutas del mandarino. Hazle sentir que debería saberlo. 
	\glt `You're visiting a botanical garden. A friend doesn't believe that tangerines are fruits of the tangerine tree. Give her the feeling that she should know that.' 
	\begin{xlist}[A:]
	\exi{A:} Las mandarinas son naranjas pequeñas. No son frutas del mandarino, aunque suena similar. \href{https://osf.io/6bzmt/}{\faVolumeUp}
	\glt `Tangerines are small oranges. They aren't fruits of the tangerine tree, even though they sound similar.' 
	\exi{B:} \textbf{Las mandarinas son frutas del mandarino.}  
	\glt `Tangerines are fruits of the tangerine tree.' 
	\end{xlist}
\ex \label{ex:experimentoNEUTRALDECLbilbao_APP}
	Quieres ir a Bilbao. Cuando sales de la casa, te encuentras con una amiga y ella te pregunta a dónde vas. 
	\glt `You want to go to Bilbao. Upon leaving your house, you encounter a friend and she asks you where you're going.' 
	\begin{xlist}[A:]
	\exi{A:} ¿Qué tal? ¡Cuánto tiempo! \href{https://osf.io/5mf47/}{\faVolumeUp}
	\glt `How are you? It's been a while!' 
	\exi{B:} \textbf{Sí, ¿cómo estás? ¿Bien?} 
	\glt `Yes, how are you? Good?' 
	\exi{A:} Muy bien, gracias. ¿A dónde vas? \href{https://osf.io/qm5x6/}{\faVolumeUp}
	\glt `Very good, thanks. Where are you going?' 
	\exi{B:} \textbf{Voy a Bilbao.} 
	\glt `I'm going to Bilbao.' 
	\exi{A:} ¡Qué bonito! Feliz viaje entonces. \href{https://osf.io/svjb6/}{\faVolumeUp}
	\glt `How nice! Have a good trip then.' 
	\end{xlist}
\ex \label{ex:experimentoMIRDECLbilbao_APP}
	Con una amiga vas en tren a Bilbao. En el camino leéis vuestros libros y os dormís sin daros cuenta. Cuando se para el tren, os despertáis. Parecen haber pasado unos pocos minutos y tu amiga pregunta si ya habéis llegado a Burgos. Ves una señal en el andén que pone Bilbao. Esto no te lo esperabas.
	\glt `You are going with a friend to Bilbao by train. On the journey you read your books and fall asleep without noticing. When the train stops, you wake up. It seems as if a few minutes have passed and your friend asks if you have arrived at Burgos already. You see a sign on the platform that says Bilbao. You didn't expect that.' 
	\begin{xlist}[A:]
	\exi{A:} ¿Dónde estamos? ¿Es Burgos? \href{https://osf.io/574vu/}{\faVolumeUp}
	\glt `Where are we? Is it Burgos?' 
	\exi{B:} \textbf{¡Ya hemos llegado a Bilbao!} 
	\glt `We have already arrived at Bilbao!' 
	\end{xlist}
\ex \label{ex:experimentoWHEXCLbilbao_APP}
	Una amiga de Bilbao te manda una carta con fotos de tu última visita hace varios años. Sientes mucho no haber vuelto ahí desde hace tanto tiempo y la llamas por teléfono. Ella te pregunta: 
	\glt `A friend from Bilbao sends you a letter with pictures from your last visit a few years ago. You're sorry not to have been back there for such a long time and you call her on the phone. She asks you:' 
	\begin{xlist}[A:]
	\exi{A:} ¿Y te llegaron las fotos que te mandé? \href{https://osf.io/rbm56/}{\faVolumeUp}
	\glt `And did you receive the fotos I sent you?' 
	\exi{B:} \textbf{Sí, son una maravilla.} 
	\glt `Yes, they are wonderful.' 
	\exi{A:} Me alegro. \href{https://osf.io/y47r3/}{\faVolumeUp} 
	\glt `Happy to hear that.' 
	\exi{B:} \textbf{¡Qué nostalgia de Bilbao!} 
	\glt `How I miss Bilbao!' 
	\end{xlist}
\ex \label{ex:experimentoOBVDECLbilbao_APP}
	Una amiga tuya es un sabelotodo que siempre se cree la más lista de todos. Te pregunta en qué ciudad del país vasco está el Museo Guggenheim, aunque todo el mundo sabe que está en Bilbao. Hazle sentir que debería saberlo. 
	\glt `A friend of yours is a know-it-all who always thinks she is the smartest of all. She asks you in which city of the Basque country the Guggenheim Museum is located, even though everybody knows it. Give her the feeling that she should know that.' 
	\begin{xlist}[A:]
	\exi{A:} ¿En qué ciudad del país vasco está el Museo Guggenheim? ¿Vitoria-Gasteiz o Bilbao? \href{https://osf.io/h23ce/}{\faVolumeUp}
	\glt `In which city of the Basque country is the Guggenheim Museum? Vitoria-Gasteiz or Bilbao?' 
	\exi{B:} \textbf{Está en Bilbao.} 
	\glt `(It) is in Bilbao.' 
	\end{xlist}
\largerpage
\ex \label{ex:experimentoOBVCONFbilbao_APP}
	Una amiga tuya es un sabelotodo que siempre se cree la más lista de todos. Quiere asegurarse que el Museo Guggenheim está en Bilbao, aunque todo el mundo lo sabe. Hazle sentir que debería saberlo. 
	\pagebreak\glt `A friend of yours is a know-it-all who always thinks she is the smartest of all. She wants to make sure the Guggenheim Museum is located in Bilbao, even though everybody knows it. Give her the feeling that she should know that.' 
	\begin{xlist}[A:]
	\exi{A:} El Museo Guggenheim está en Bilbao, ¿verdad? \href{https://osf.io/rk4ue/}{\faVolumeUp}
	\glt `The Guggenheim Museum is in Bilbao, right?' 
	\exi{B:} \textbf{Está en Bilbao.} 
	\glt `(It) is in Bilbao.' 
	\end{xlist}
\ex \label{ex:experimentoOBVDENbilbao_APP}
	Una amiga tuya es un sabelotodo que siempre se cree la más lista de todos. Afirma que el Museo Guggenheim está en Vitoria-Gasteiz, aunque todo el mundo sabe que está en Bilbao. Hazle sentir que debería saberlo. 
	\glt `A friend of yours is a know-it-all who always thinks she is the smartest of all. She states that the Guggenheim Museum is located in Vitoria-Gasteiz, even though everybody knows it is in Bilbao. Give her the feeling that she should know that.' 
	\begin{xlist}[A:]
	\exi{A:} El Museo Guggenheim no está en Bilbao, ¿sabes? Está en la capital, Vitoria-Gasteiz. \href{https://osf.io/x3jse/}{\faVolumeUp}
	\glt `The Guggenheim Musuem is not in Bilbao, you know? (It) is in the capital, Vitoria-Gasteiz.' 
	\exi{B:} \textbf{Está en Bilbao.} 
	\glt `(It) is in Bilbao.' 
	\end{xlist}
\ex \label{ex:experimentoNEUTRALDECLvegana_APP}
	Te invitaron a una barbacoa vegana. Una amiga te llama por teléfono. 
	\glt `You have been invited to a vegan barbecue. A friend calls you by phone.' 
	\begin{xlist}[A:]
	\exi{A:} Hola, ¿qué tal? \href{https://osf.io/pgj25/}{\faVolumeUp}
	\glt `Hi, how are you?' 
	\exi{B:} \textbf{Hola querida. ¿Cómo estás?} 
	\glt `Hi dear. How are you?' 
	\exi{A:} Bien, gracias. Escucho que estás con gente. ¿Hacéis una fiesta? \href{https://osf.io/cx62t/}{\faVolumeUp}
	\glt `Good, thanks. I can hear you're with people. Are you having a party?' 
	\exi{B:} \textbf{Hacemos una barbacoa vegana.} 
	\glt `We are having a vegan barbecue.' 
	\exi{A:} ¡Qué rico! \href{https://osf.io/34n76/}{\faVolumeUp}
	\glt `How tasty!' 
	\end{xlist}
\ex \label{ex:experimentoMIRDECLvegana_APP}
	Una amiga y tú van a una barbacoa y traen hamburguesas y cerveza. Cuando llegáis, no hay ni carne ni salchichas. Cuando os preguntáis por qué es así, vuelves a leer la invitación y te das cuenta que es una barbacoa vegana.  
	\glt `You and a friend are going to a barbecue and you bring hamburgers and beer with you. When you arrive, there is neither meat nor sausages. When you ask yourselves why, you read the invitation again and notice that it is a vegan barbecue.' 
	\begin{xlist}[A:]
	\exi{A:} No veo carne en la mesa. \href{https://osf.io/hb5ty/}{\faVolumeUp}
	\glt `I don't see meat on the table.' 
	\exi{B:} \textbf{¿Estás segura de que era una barbacoa?} 
	\glt `Are you sure that it was a barbecue?' 
	\exi{A:} A ver qué dice la invitación. ¿La tienes aquí? \href{https://osf.io/dth58/}{\faVolumeUp}
	\glt `Let's see what it says on the invitation. Do you have it with you?' 
	\exi{B:} \textbf{Sí, espera.} 
	\glt `Yes, wait.' 
	\exi {} \ldots  
	\exi{B:} \textbf{¡Es una barbacoa vegana!} 
	\glt `(It) is a vegan barbecue!' 
	\end{xlist}
\ex \label{ex:experimentoWHEXCLvegana_APP}
	Una tienda en un mercado de alimentos orgánicos vende salsas. Te hacen probar varias. Todas son buenas, pero la vegana es increíble. 
	\glt `A shop in an organic food market is selling sauces. They let you taste a few. They're all good, but the vegan one is incredible.' 
	\begin{xlist}[A:]
	\exi{A:} ¿Ya probaste las salsas? ¿Te gusta alguna? \href{https://osf.io/c6qgx/}{\faVolumeUp} 
	\glt `Did you try the sauces already? Did you like one?' 
	\exi{B:} \textbf{¡Qué buena la vegana!} 
	\glt `How tasty the vegan one (is)!' 
	\end{xlist}
\ex \label{ex:experimentoOBVDECLvegana_APP}
	Una amiga tuya trabaja en una tienda vegana pero no está segura si la leche de soja es vegana porque lleva el nombre de leche. Hazle sentir que debería saberlo. 
	\glt `A friend of yours works at a vegan retailer but isn't sure if the soy milk is vegan because it's called milk. Give her the feeling that she should know that.' 
	\begin{xlist}[A:]
	\exi{A:} En mi tienda vendemos leche de soja. ¿Es vegana o es leche de verdad? \href{https://osf.io/c5f26/}{\faVolumeUp}
	\glt `At my shop we are selling soy milk. Is it vegan or is it real milk?' 
	\exi{B:} \textbf{La leche de soja es vegana. Es solamente el nombre.} 
	\glt `Soy milk is vegan. It's just the name.' 
	\end{xlist}
\ex \label{ex:experimentoOBVCONFvegana_APP}
	Una amiga tuya trabaja en una tienda vegana pero quiere asegurarse que la leche de soja es vegana porque lleva el nombre de leche. Hazle sentir que debería saberlo. 
	\glt `A friend of yours works at a vegan retailer but wants to make sure that soy milk is vegan because it's called milk. Give her the feeling that she should know that.' 
	\begin{xlist}[A:]
	\exi{A:} En mi tienda vendemos leche de soja. Se llama leche, pero es vegana, ¿no? \href{https://osf.io/4u3tc/}{\faVolumeUp} 
	\glt `At my shop we are selling soy milk. It's called milk, but it's vegan, right?' 
	\exi{B:} \textbf{La leche de soja es vegana. Es solamente el nombre.} 
	\glt `Soy milk is vegan. It's just the name.' 
	\end{xlist}
\ex \label{ex:experimentoOBVDENvegana_APP}
	Una amiga tuya trabaja en una tienda vegana pero cree que la leche de soja no es vegana porque lleva el nombre de leche. Hazle sentir que debería saberlo. 
	\glt `A friend of yours works at a vegan retailer but believes that soy milk isn't vegan because it's called milk. Give her the feeling that she should know that.' 
	\begin{xlist}[A:]
	\exi{A:} En mi tienda vendemos leche de soja. Se llama leche, entonces no puede ser vegana, ¿sabes? \href{https://osf.io/c9zym/}{\faVolumeUp} 
	\glt `At my shop we are selling soy milk. It's called milk, so it can't be vegan, you know?' 
	\exi{B:} \textbf{La leche de soja es vegana. Es solamente el nombre.} 
	\glt `Soy milk is vegan. It's just the name.'
	\end{xlist}
\end{exe}
