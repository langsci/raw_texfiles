% Chapter 1

\chapter{Introduction} % Chapter title

\label{ch:1} % For referencing the chapter elsewhere, use \chapref{ch:introduction} 

%----------------------------------------------------------------------------------------

\section{Goals}
\label{sec:1.1}

This is a book on Castilian Spanish intonation. More precisely, on prosodically marked declaratives (\ref{ex:markedexamplesDECL}), \textit{wh}-exclamatives (\ref{ex:markedexamplesEXCL}), and discourse particles such as (\ref{ex:markedexamplesPARTclaro}) and (\ref{ex:markedexamplesPARTanda}) in the Madrid variety.\footnote{A guide on how to read and listen to the examples follows in \sectref{ch:guide}. Contexts and elicitation methods for (\ref{ex:markedexamplesDECL}), (\ref{ex:markedexamplesEXCL}), (\ref{ex:unmarkedexamplesDECL}), and (\ref{ex:unmarkedexamplesEXCL}) are given in \autoref{app:AppendixA2} and explained in \chapref{ch:6}. Examples (\ref{ex:markedexamplesPARTclaro}), (\ref{ex:markedexamplesPARTanda}), (\ref{ex:unmarkedexamplesPARTclaro}), and (\ref{ex:unmarkedexamplesPARTanda}) are cited from \citet{PRESEEA.20142020} and explained in \chapref{ch:5}.}

\begin{exe}
\ex \label{ex:markedexamplesDECL} ¡Es el presidente del gobierno! 
\glt `(He/She) is the prime minister!'
	\begin{xlist}
	\ex L* HL\% \href{https://osf.io/9wfsq/}{\faVolumeUp}
	\ex L+H* L!H\% \href{https://osf.io/wk46m/}{\faVolumeUp} 
	\ex L+¡H* L\% \href{https://osf.io/ghcbu/}{\faVolumeUp} 
	\end{xlist}

\ex \label{ex:markedexamplesEXCL} ¡Qué buena limonada! L+¡H* L\% \href{https://osf.io/xcvfh/}{\faVolumeUp}
\glt `What a good lemonade!' 

\ex \label{ex:markedexamplesPARTclaro} ¡Sí, sí, claro! L*~H\% \href{https://osf.io/5z7f4/}{\faVolumeUp}
\glt `Yes, yes, sure!' 

\ex \label{ex:markedexamplesPARTanda} ¡Anda! L+¡H* L\%  \href{https://osf.io/kqvn2/}{\faVolumeUp}
\glt `Wow!'
\end{exe}

I argue that these marked forms differ from unmarked forms such as (\ref{ex:unmarkedexamplesDECL}), (\ref{ex:unmarkedexamplesEXCL}), (\ref{ex:unmarkedexamplesPARTclaro}), and (\ref{ex:unmarkedexamplesPARTanda}) in that they encode modal evaluations of the at-issue meaning.

\begin{exe}
\ex \label{ex:unmarkedexamplesDECL} Es el presidente del gobierno. L* L\% \href{https://osf.io/jbuv4/}{\faVolumeUp}
\glt `(He/She) is the prime minister.' 

\ex \label{ex:unmarkedexamplesEXCL} Qué buena limonada. L* L\% \href{https://osf.io/97bys/}{\faVolumeUp}
\glt `What a good lemonade.' 

\ex \label{ex:unmarkedexamplesPARTclaro} Sí, sí, sí, claro. L* L\% \href{https://osf.io/627hp/}{\faVolumeUp}
\glt `Yes, yes, yes, sure.' 

\ex \label{ex:unmarkedexamplesPARTanda} Anda, anda. L* L\%  \href{https://osf.io/cjbev/}{\faVolumeUp}
\glt `Whoa, whoa.' 
\end{exe}

These forms are marked in the Greenbergian sense: overtly and saliently encoded, semantically complex, relatively rare in texts, and neutralized in un\-mar\-ked contexts \citep{Haspelmath.2006againstmarkedness,Greenberg.1966}. In other words, the presence of additional tonal targets or features (intonational marking) brings about conversational moves that are pragmatically marked in the sense that they not only proffer a conversational update, but evaluate it relative to possible worlds. Two epistemic evaluations that can be shown to be encoded by intonation in Spanish are linguistically encoded surprise (or mirativity, \citealt{DeLancey.1997,DeLancey.2012,RettSturman.2020,Rett.2011,HengeveldOlbertz.2012}) and obviousness. I propose that these meanings are modal in that they evaluate propositions relative to possible worlds accessible from the set of shared assumptions between interlocutors, the Common Ground \citep{Stalnaker.1974}. The fact that this evaluation is encoded via intonation allows it to combine with a Common Ground update, which can lead to different relations between the Common Ground and what is asserted (at-issue). Mirative assertions, if they are accepted as true, will lead the interlocutors to change some of their shared assumptions, because they assert a proposition and evaluate it as incompatible with all the ways they would have envisioned the conversation to proceed. Obvious assertions, on the contrary, propose a context update and evaluate it as necessary from the perspective of the Common Ground, pointing to a lack of relevance of the previous speech act that triggered the assertion. This perspective is inherently dynamic in the sense that such modal evaluation requires a temporal shift of perspective relative to the at-issue content of an utterance \citep{FilippiDeswelle.2019}. Mirativity is past impossibility, and obviousness is past necessity, of a proposition asserted or accepted as true \citep{Reich.2018}.

An empirical investigation via a production experiment with audio-stimuli finds that mirativity and obviousness are associated with distinct intonational features under constant focus scope, with stances of (dis)agreement showing an impact on obvious declaratives. \textit{Wh}-exclamatives are found not to differ significantly in intonational marking from neutral declaratives, a finding that underlines the importance of distinguishing between the meaning of \textit{wh}-syntax and exclamative intonation. Moreover, a corpus study based on natural dialogue data shows that the intonational marking that discourse particles receive differs between particles. The mirative use of \textit{anda} `wow' is not marked with intonational configurations that have been linked to obviousness, while \textit{claro} `sure' seems prone to such marking. Qualitative investigation shows a clear link between contexts in which interlocutors negotiate expectations and the occurrence of prosodically marked particles. I therefore propose to see expectations of interlocutors and projected next dialogical steps as a source of intonational variability that cannot be reduced to the marking of a set of contextually salient alternatives, which is a standard definition of focus \citep{Rooth1992,KrifkaMusan.2012}. Rather, speakers can additionally encode whether proffered content is expected or unexpected, based on what has been accepted so far. Furthermore, I take the complex interaction between expectations and (dis)agreement under constancy of focus scope as an argument for a more complex perspective on the notion of contrast. A contrast between expectations and the proffered at-issue content, as expressed by miratives, can combine with a contrast between two stances, or disagreement. But these two levels of contrast are independent. Therefore, disagreement can also combine with obviousness or occur without any modal evaluation of the at-issue content.\footnote{This proposal, while broadly along the lines also pursued by \citet[]{Repp.2016} and \citet[]{Cruschina.2021} to break up the notion of contrast, goes beyond degrees of contrast and allows for complex contrastive discourse configurations. See \citet[]{Fliessbach.inprep.TILSM} for more details and a discussion of examples from Kogi, Kurtöp, and Turkish.}

\section{The problem}
\label{sec:1.3}

\begin{quote}\sloppy
Intonation [\ldots] refers to the use of \textit{suprasegmental} phonetic features to convey “postlexical” or \textit{sentence-level} pragmatic meanings in a \textit{linguistically structured} way. \citep[4]{Ladd2008}
\end{quote}

Intonation is defined by sentence-level meanings. But what kinds of meanings are encoded intonationally? And what does sentence-level mean? There is broad consensus in the literature that variability in the mapping between syntactic and prosodic structure can be attributed to the encoding of information structure (e.g. \cite{Selkirk.2011,Buring.2016}). The di\-cho\-to\-mic dimensions \textit{focus} vs. \textit{background} and \textit{topic} vs. \textit{comment} can divide sentences and are therefore prone to be linked to the delimitative functions of prosody in the sense of \citet[29]{Trubeckoj1939}. Once divided, parts of sentences can also be foregrounded prosodically. For English and German, the choice of prominence lending cue has been attributed to more complex interactions of Common Ground update, (dis)agreement, speaker/hearer attribution \citep{Steedman.2007,Steedman.2014}, as well as referential and lexical givenness \citep{Baumann2006,BaumannRiester2012}.

While information structure, and in particular focus-background partition, is one of the main factors made responsible for intonational variation in research on Spanish as well, the state of the art is inconsistent when it comes to tonal inventories. The Atlas Interactivo de la Entonación del Español \citep{Prieto2009-2013}, updated and contextualized in 
\citet{HualdePrieto2015}, provides \textit{nuclear configurations}\footnote{In the literature on Spanish intonation, a nuclear configuration is a combination of the most prominent pitch accent in an intonational phrase, sometimes assumed to be the last or “rightmost” one by default because of a Nuclear Stress Rule \citep{ChomskyHalle.1968}, with a boundary tone. See \chapref{ch:2} for background information on the \ac{SpToBI} notations used here.} for eighteen different sentence types. The Madrid variety is described to have seven different configurations for six types of declaratives, summed up in  \autoref{tab:intonationalcategoriesPRIETOsmall}.\footnote{\autoref{app:AppendixA} provides the full picture for Madrid Spanish, including declaratives, questions, imperatives, and vocatives.} This research tradition implicitly understands ``sen\-tence-level meanings'' not as the delimitation of sentence-parts according to the notions of \ac{IS}, but rather on the level of utterances or \acp{TCU}, which can vary in length and complexity between ``sentences, clauses, phrases, and individual words'' \citep[151]{Clayman.2013}. In the following, I will understand sentences as minimally composed of one clause in the sense of \citet[77]{BrownMiller.2013}, containing at least one inflected verb, but possibly more. This will be important in the discussion of nuclear configurations, entities that operate on the level of \acp{IP} and can map to \acp{TCU} that are not sentences.\footnote{I thank an anonymous reviewer for stressing the importance of this issue.}

\vfill
\begin{table}[H]
\begin{tabular}{lr}
 \lsptoprule
 Types of statements & \acf{NI} \\
 \midrule
 Broad focus statements & L* L\% / L$+$H* L\% \\
 Contrastive focus\slash contradiction & L* HL\%  \\
 Exclamative statements                     & L$+$¡H* L\%  \\
 Dubitative/uncertainty statements          & L$+$¡H* !H\% \\
 Statement of the obvious                     & L$+$H* L!H\%  \\
 Insistent explanation                      & H$+$L* L\%\\
\lspbottomrule
\end{tabular}
	\caption{Previous findings on Castilian Spanish declarative intonation by \citet{Prieto2010.2014.Atlas} and \citet{EstebasVilaplanaPrieto.2010} with revised notation by \citet{HualdePrieto2015}\label{tab:intonationalcategoriesPRIETOsmall}}
\end{table}
\vfill
\pagebreak

In an investigation of focus as both a syntactic and prosodic phenomenon, \citet{Gabriel2007} proposes the smaller inventory in \autoref{tab:intonationalcategoriesGABRIEL}. Here, intonational categories are not defined in terms of nuclear configurations, but rather in terms of individual functions. Tables~\ref{tab:intonationalcategoriesPRIETOsmall} and~\ref{tab:intonationalcategoriesGABRIEL} contradict each other. The L+H* pitch accent is univocally associated only with the meaning of contrastive focus in \autoref{tab:intonationalcategoriesGABRIEL}, while \autoref{tab:intonationalcategoriesPRIETOsmall} presents L* HL\% as marker of contrastive focus and/or contradiction. Another feature that sets the two overviews apart is the high phrase accent H$-$, mentioned only in \autoref{tab:intonationalcategoriesGABRIEL}. Finally, obviousness, insistence, uncertainty, and exclamation are only mentioned in \autoref{tab:intonationalcategoriesPRIETOsmall}.

\citet{Face.2001,Face2001,Face2002} differs from \citet{Gabriel2007} in analyzing L+H* pitch accent as a marker that can be used independently of, and additionally to, phrase accents and scaling differences to mark contrastive focus. Nevertheless, both authors agree in taking the presence or absence of contrastive focus as their independent variable for the investigation of intonational variability in Spanish declaratives, with additional variability interpreted as phonetic implementation \citep[175]{Gabriel2007,Face2002}. As is common in intonation research \citep[102]{Ladd.1980}, the key discrepancy between research on contrastive focus and \autoref{tab:intonationalcategoriesPRIETOsmall} are different perspectives on whether nuances such as obviousness, surprise, and insistence are independent levels of meaning.

\begin{table}
	\begin{tabularx}{\textwidth}{llX}
		\lsptoprule
		\multicolumn{3}{l}{Pitch accents}\\
		/(L$+$H)*/     & L*$+$H   & prenuclear\\
		& L$+$H*   & nuclear \\
		&       & word-finally context induced: /\_)$\omega$ \\   
		&       & IP-finally context induced: /\_)ip)IP\\   
		& L*    & free variant: /\_)ip)IP \\
		/L$+$H*/       &       & contrastive focus \\
		\midrule
		\multicolumn{3}{l}{Phrase accents} \\
		/L$-$/        &       & delimitation of (contrastive) focus domain \\
		/H$-$/        &       & delimitation of presupposed prefocal material\\
		&       & continuation in coordinate structures\\
		&       & syntactic disambiguation\\
		&       & separation of left-peripheral \textit{topic} constituents\\
		\midrule
		\multicolumn{3}{l}{Boundary tones} \\
		/L\%/       &       & closure, declarative \\
		/H\%/       &       & interrogative (yes-no questions) \\
		/\%H/       &       & facultative (high initial pitch in interrogatives) \\
		\lspbottomrule 
	\end{tabularx}
	\caption{Previous findings on Spanish intonation and information structure by \citet[201]{Gabriel2007}\label{tab:intonationalcategoriesGABRIEL}}
\end{table}

\section{Proposal and structure}
\label{sec:1.4}

How should the field deal with this discrepancy? One possibility is to exclude obviousness and exclamation from the scope of our investigation by interpreting them as a matter of ``phonetic implementation of the pitch contour, and [...] as such non-structural" \citep[24]{Gussenhoven2004}. Another possibility is to see obvious and exclamative intonation as structured markers of evaluative meanings which, if taken into account, reduce the amount of unexplained intonational variability. The present book attempts to follow the latter approach. Starting from the inventory of declaratives in \autoref{tab:intonationalcategoriesPRIETOsmall}, this book proposes a perspective on intonational meaning encoded in the prosodic marking of Spanish declaratives, exclamatives, and discourse particles that is not captured by most definitions of information structure. It combines a commitment-based model of discourse meaning \citep{FarkasBruce.2010,Rett.2021emotivemarkers,Rett.2021expressivesandmiratives} with research on modal presuppositions and conventional implicatures \citep{Potts.2007expressivedimension,BianchiBocciCruschina.2016,Reich.2018}. The main argument is that we need a “full integration of intonational meaning into dynamic and multidimensional models of meaning”\citep[371]{Prieto.2015} to be able to answer the questions surrounding intonational form.
%}

The categories in \autoref{tab:intonationalcategoriesPRIETOsmall} are closely linked to the inductive methodology used in research on pragmatics and sociolinguistics (the \href{http://prosodia.upf.edu/iari/methodology.html\#Prieto}{Interactive Atlas of Romance Intonation} mentions \cite{BlumKulka1989}, \cite{Billmyer2000} and \cite{FelixBrasdefer.2010}). They are holistic labels born out of an intuition for the subtleties of spoken discourse and the necessity to create short, vivid situations accessible to native speakers who would participate in \textit{Discourse Completion Tasks} under laboratory conditions. The methodology was developed by \citet{Prieto.2001}, further refined by \citet{Prieto2009-2013}, and has since allowed the replication of similar and comparable observations on dozens of varieties of nine Romance languages (cf. \cite{Prieto2010.2014.Atlas}), a major achievement in terms of setting up a point of departure for further, in-depth study of the categories involved.

Empirically, we investigate the declarative categories subsumed in \autoref{tab:intonationalcategoriesPRIETOsmall} in Castilian Spanish from the Comunidad de Madrid, a variety that has already been the object of previous investigations on intonation. This allows us to draw on a body of comparable literature that facilitates the theoretical discussion (e.g. \cite{EstebasVilaplanaPrieto.2008,EstebasVilaplanaPrieto.2010,ElviraGarcia.2016,TorreiraGrice.2018}).

The structure of the book can be broadly divided into a theoretical part, comprising Chapters~\ref{ch:2} and \ref{ch:3}, an empirical part (Chapters~\ref{ch:4}--\ref{ch:6}), and the conclusions in \chapref{ch:7}.  \chapref{ch:2} lays out the notational conventions and some core issues of intonational phonology, particularly those relevant to the understanding of Spanish intonation. It also provides the necessary terminological basis for \chapref{ch:3}, which is broader in scope and contains the main theoretical proposal of this book. \chapref{ch:3} first provides the reader with insights into the pragmatic meanings commonly associated with intonation and reviews the relevant literature on intonational meanings. It then discusses possible ways of modeling the meaning of marked and unmarked statements and argues for a combination of modal semantics with a dynamic model of discourse commitments and discourse evaluations. \chapref{ch:4} builds a bridge between the theoretical discussion and the empirical part of the book. It sums up the main arguments made so far, and discusses ways of empirically investigating intonational meaning with methods from Laboratory Phonology and Corpus Phonology. Chapters~\ref{ch:5} and~\ref{ch:6} present the methodology and results of two such attempts. While \chapref{ch:5} is about a corpus study on the interaction of intonation and discourse particles in Madrid Spanish, \chapref{ch:6} describes a production experiment on the intonation of epistemically marked declaratives in comparison with \textit{wh}-exclamatives. \chapref{ch:7} sums up the main results and concludes with an outlook on the next steps necessary to gain a full-fledged picture of the meanings of intonation, in Madrid Spanish and in general.

\section{A quick guide to the examples in this book}\label{ch:guide}

Before we begin our discussion, a quick guide to the examples used in this book is necessary. Many of them end with exclamation marks, question marks, or even graphemic sequences such as 〈¡¿〉 and 〈?!〉, or 〈¿¡〉 and 〈!?〉. Yet punctuation is prosodically unreliable. Exclamation marks are usually used to indicate marked intonation, but the intended intonation can often only be disambiguated by reading a sentence out loud \citep{RealAcademiaEspanola.2010}. This is not surprising, given that the number of marked intonational forms exceeds the number of graphemes available for punctuation.

We therefore need to enrich written examples with additional information to allow us to capture their intonation and to fathom the intended interpretation. Unfortunately, the \ac{IPA} is not very precise when it comes to intonation, distinguishing only between global rises, falls, upsteps, and downsteps. The \ac{SpToBI} system \citep{BeckmanETAL.2002,EstebasVilaplanaPrieto.2008,HualdePrieto2015} is the most convenient tool currently available,\footnote{A detailed introduction to \ac{SpToBI} follows in \chapref{ch:2}.} given that it reduces intonation to the tones high H, low L, upstepped high ¡H (sometimes also written HH), and downstepped high !H, which can associate with prominent syllables H* to form pitch accents, with intermediate boun\-da\-ries H$-$ to form phrase accents, or with final boundaries H\% to form boundary tones. The system gets complicated by the fact that pitch accents and boundary tones can also form bitonal contour tones such as L+H*, H+L*, LH\%, and HL\%. But most importantly, the system is no International Prosodic Alphabet, but language-specific \citep{HualdePrieto2016}. Reading a language-specific ToBI system aloud is almost impossible without having heard audio data for comparison. I therefore opted for an approach that combines the textual discussion with as much audio data as was available to me. If you are reading this book on a device with access to the web, the loudspeaker symbol after figures or examples {\small\faVolumeUp{}} refers you to web hosted audio files. Some are hosted on existing online publications, but most are made available via the Open Science Framework \citep{FosterDeardorff.2017}. The external-link symbol {\small\faExternalLink*{}} links to websites in which audio files are embedded as part of larger entries. If, on the other hand, you are reading on paper but would still like to consult an audio recording, \autoref{app:AppendixA0} presents a list of all web links not contained in the bibliography.
