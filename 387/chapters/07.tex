\chapter{Conclusions}\label{ch:7} % For referencing the chapter elsewhere, use \chapref{ch:name} 

\section{Recapitulation}\label{ch:7.1}

I started my investigation of Spanish intonation by noting that the state of the art is inconsistent, not only with regard to the expression of contrastive focus, but also with regard to the kind of meanings encoded by intonation. 
In \chapref{ch:2}, I outlined how different perspectives on prosody are shaped by assumptions about the possible meanings encoded by intonation. Investigations that take the notions of information structure and syntactic constituency as a point of departure will focus on the delimitative functions of prosody. I then argued that Castilian Spanish seems to allow for paradigmatic choices at the level of pitch accents and boundary tones that serve distinctive functions not captured by the notions of focus or contrast. In \chapref{ch:3}, I revisited much of the literature on statements of the obvious and \textit{wh}-exclamatives in Spanish. I argued for a perspective that separates syntax from prosody and attributes specific meaning to each of the two. I also argued that the term \textit{wh}-exclamative can lead to the erroneous belief that \textit{wh}-syntax would be responsible for a mirative interpretation. To remedy this, I proposed to acknowledge the mirative import of the L+¡H*~L\% nuclear configuration in Spanish \textit{wh}-exclamatives, just as in sentences with declarative syntax. I then revisited examples in which a nuance of obviousness was attributed to \textit{si-} and \textit{que}-insubordinates as well as particles such as \textit{claro}, and found that these examples showed either L* HL\% or L+H* L!H\% intonation. To capture the difference between contrasting stances (disagreement) and contrasts between expectations and propositions, I concluded \chapref{ch:3} with a commitment-based model of dialogue meaning that allows for modal non-at-issue commitments that evaluate the at-issue meaning relative to the input Common Ground of a context update.

\chapref{ch:4} posed a set of empirical questions derived from the previous discussion of the literature on Castilian Spanish intonation and formal pragmatics. \chapref{ch:5} was dedicated to an exploration of a spontaneous dialogue corpus in search of evidence for prosodic variability attributable to the negotiation of expectations about propositions. Taking a corpus based approach was partly motivated by the fact that much of the literature on Castilian Spanish intonation is based on laboratory speech. An investigation of spontaneous data was necessary to find evidence for the naturalness of supposedly marked intonational forms. Using the association measure Mutual Information, I established modal and polar meaning components of four discourse particles: \textit{hombre}, \textit{claro}, \textit{anda}, and \textit{vaya}. While \textit{claro} and \textit{anda} showed lexical collocations with meanings related to obviousness and surprise, respectively, collocations for \textit{hombre} and \textit{vaya} pointed to additional or divergent meanings, with \textit{hombre} linked to expectational realignment and \textit{vaya} linked to negative bouletic evaluation. An investigation of the prosodic form of turns containing these particles showed that some prosodic contours do not occur with some discourse particles. Notably, L*~HL\% and L+H*~LH\% occurred with \textit{hombre} and \textit{claro}, but not with \textit{anda} and \textit{vaya}. Moreover, the L+¡H*~L\% nuclear configuration is frequent with \textit{anda}, whereas \textit{vaya} occurs mostly with L*~L\% intonation. I took these results as evidence for a correlation between certain marked context updates, which are specified for relative polarity and include modal non-at-issue commitments, and intonational configurations in the variety of Spanish under investigation.

Given that observations about Castilian Spanish intonation are often based on individual examples, the next task was to check for the quantitative reproducibility of these observations.  \chapref{ch:6} therefore developed an audio-enhanced Discourse Completion Task that includes target sentences with neutral assertions, mirative assertions, \textit{wh}-exclamatives, obvious assertions, obvious con\-fir\-ma\-tions, and obvious reversals. Focus position was kept constant and utterance final in all conditions to ensure that prosodic variability would not be due to differences in focus scope. Results for neutral declaratives were consistent with the observations in both Tables~\ref{tab:intonationalcategoriesPRIETOsmall} and~\ref{tab:intonationalcategoriesGABRIEL}, yielding associations with both L+H*~L\% and L*~L\%. The L+¡H*~L\% contour was significantly associated with mirative conditions, though not with \textit{wh}-exclamatives. This result is consistent with the proposal in \sectref{ch:3.1} and should be seen as an indication that \textit{wh}-exclamative syntax is independent of exclamative intonation. The main departure from \autoref{tab:intonationalcategoriesPRIETOsmall} concerns statements of the obvious. Here, I found that obvious declaratives reacting to an unbiased alternative question are significantly associated with L*~HL\% intonation, indicating that the so-called \textit{matiz de obviedad} `obvious nuance' proposed by \citet[277--279]{EstebasVilaplanaPrieto.2008} need not combine with the rejection of a previous commitment to trigger such a contour. Instead, obvious reversals in the data are associated with the L+H*~HL\% contour described as insistent call in \autoref{tab:intonationalcategoriesPRIETO}. 

\section{Open questions and outlook}\label{ch:7.2}

The experimental setup in \chapref{ch:6} does not test all factors that can have an influence on the intonation of statements. A complete picture would need to include a cross-classification of a much larger set of provocation types (assertive, inquisitive, biased) with different response types (assertion, confirmation, reversal), different evaluative commitments (necessity and impossibility of the at-issue commitment), and different accessibility relations (epistemic, bouletic, deontic, etc.). 

What remains valid is that we need a “full integration of intonational meaning into dynamic and multidimensional models of meaning” \citep[371]{Prieto.2015} to be able to answer the questions surrounding intonational form. Results for the interplay between obviousness and relative polarity indicate that my model, while dynamic, still does not incorporate a sufficient number of dimensions. The fact that obvious assertions differ from obvious confirmations in the choice of boundary tone reminds us that not only the expectability of a commitment and (dis)agreement about a commitment, but also the epistemic gradient \citep[32]{Heritage.2012epistemicengine} between interlocutors has an impact on the nuclear intonation of Spanish utterances. Obvious assertions and obvious reversals end on a low target, whereas obvious confirmations end high. This renders their nuclear intonation similar to that of polar questions, which differ in prenuclear peak alignment.

Standard questions are defined by ``ignorance on the part of the speaker and a presumption of knowledge on the part of the addressee'' \citep[289]{Dayal.2016}. While none of the categories investigated here include ignorance on the part of the speaker, elicitation contexts for obvious confirmations may trigger a presumption of knowledge on the part of the addressee.\footnote{As opposed to the presumption that \textit{p} should be entailed from the Common Ground.} \citet[284]{Dayal.2016} assumes that such a speech act will become a ``higher order assertion, one which taps into the very obviousness of the answer''. In other words, the non-at-issue commitment might be elevated to at-issue relevance, prompting an explicit negotiation of expectations. In pursuing an account of intonational variation in declaratives, I have left much of the variability within questions unaccounted for. I hope that some of the notions and methods developed here will help future endeavors to include questions into the broader picture. Importantly, production experiments and corpus investigations should remain conscious of the \textit{Provocation-Response Nexus}, which requires us to situate every dialogical turn in a context that takes into account previous conversational steps as well as the amount of shared assumptions between interlocutors. 

A remaining puzzle concerning declaratives is the role of the L+H* L!H\% contour in Madrid Spanish. Even in a production experiment with 348 contexts eliciting obvious statements, this contour is margi\-nally too infrequent to statistically associate it with obviousness. Nevertheless, it does occur in both natural data and laboratory speech. An investigation via perception experiments would be the logical next step to address this issue. The gated perspective on prolonged prenuclear rises in obvious statements presented in \autoref{fig:obviousFACE} is another result that should be subjected to a gated perception experiment of the sort used in \citet{Face2007}, ideally in comparison with the perception of polar interrogatives. As we have seen in \sectref{ch:6.3.5}, the stimuli design for such an experiment will have to take mean sentence F$_0$ into account, pointing to the possibility that iconic strategies exploiting the Frequency Code \citep{Gussenhoven2004} help speakers disambiguate between the possible functions of prenuclear rises in Castilian Spanish. 

The investigation of intonational meaning in Spanish is often hampered by the finding that ``the linguistic code may allow for what appear to be one-to-many mappings between meaning and intonational form, so that slightly or radically different contours may express the same meaning; conversely, the same contour may also serve to express a number of different meanings'' \citep[390]{HualdePrieto2015}. Intonational form-function mappings are variable in the sense that categorically definable phonological events associate only probabilistically with specific functions, but they can also show gradient variability in the sense that a phonetic parameter is modulated gradually so as to encode gradual changes in (scalar) meaning. To make matters worse, ``often it remains unclear where to draw the line between gradual and discrete distinctions'' \citep[145]{Roettger.2017}.

What I have tried to show is that part of the apparent variability found in studies on the intonation of categories such as focus \citep{Face.2001,Face2002,Face2003,Gabriel2007} or insubordinates \citep{ElviraGarcia.2016} can be explained by a dynamic and multidimensional model of meaning in discourse. Likewise, variation in the interpretation of written \textit{wh}-exclamatives \citep{Grosz.2012} can be due to an underspecification of the prosodic dimension by punctuation. The finding in \sectref{ch:6.2} that automatic and manual annotation of a corpus of highly marked intonational forms can reach a substantial level of inter-rater agreement shows that variability in Castilian Spanish intonation does not impede investigation. And finally, the amount of (bi-)univocal associations between nuclear configurations and experimental conditions presented in \autoref{tab:adjustedstandardizedresiduals} shows that probabilistic association is far from random. Most importantly, though, we have seen that the paradigmatic choice between different nuclear configurations cannot be reduced to the presence or absence of (contrastive) focus and the question-answer dichotomy. Modal evaluative meanings such as mirativity and obviousness are associated with specific nuclear configurations and can vary independently of (dis)agreement between interlocutors. Taking these kinds of intonational meaning into account is important for understanding the paradigmatic choices at the level of both pitch accents and boundary tones in Spanish. Prosody both reflects and implements the pragmatic objectives of the interlocutors \citep[260]{MartinButragueno.2015}. The Stalnakerian idea that interlocutors strive to reduce the context set by agreeing on shared assumptions that exclude possibilities certainly captures the main driving force behind much of conversation. Yet the realm of possibilities can further be structured. Such a perspective on the notion of Common Ground leads to a view ``that also takes into account the change of epistemic and deontic stances towards propositions through time'' \citep[204]{Reich.2018}. The dimension of time was already integrated into modality through the notion of a stereotypical ordering source by \citet{Kratzer.1981}. Dynamic and commitment based perspectives on discourse meaning give it an even more central role. It's about time.
