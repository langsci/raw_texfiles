% Chapter X

\chapter{Exploring corpora: Discourse particles and intonation} % Chapter title

\label{ch:5} % For referencing the chapter elsewhere, use \chapref{ch:name} 

%----------------------------------------------------------------------------------------

This chapter aims at exploring the distribution and some relevant features of mirative and obvious assertions detectable in spontaneous speech and thereby detecting possible intervening variables for the experimental investigation of their prosody. It also intends to sharpen the hypotheses derived from the model in \sectref{ch:3.3} and from the literature by providing a basis of natural data observations. This is particularly important given that studies on natural corpus data (such as \cite{CanteroSerenaFontRotches.2007,MartinButraguenoVelazquez.2018}) have proven difficult to integrate into the laboratory speech based picture as presented by \citet{HualdePrieto2015}.

Statistically valid, quantitative phonetic and phonological comparison of high\-ly marked intonation as laid out in \sectref{ch:3.3} is close to impossible with lexically, syntactically, and contextually uncontrolled corpus data at the current state of the art. Questions (\ref{ex:questionsoverview}a,b,c,d) about nuclear contours, their relation to relative polarity, and their interplay with phrasing and exclamative syntax can therefore not be answered solely by means of a corpus study. Yet qualitative observations such as those in \sectref{ch:5.2} are important steps in developing intuitions about the contexts for such marked discourse moves. Moreover, question (\ref{ex:questionsoverview}d) about the correlation between different markers of non-at-issue commitments can, and should, be tackled via spontaneous data. While we ultimately need \textit{Laboratory Phonology} research in the sense of \citet{CohnFougeronHuffman.2012intro} to find out about the individual contribution of intonation (\chapref{ch:6}), we expect marked discourse moves to be signaled by intonation, discourse particles, and syntax in natural dialogue.\footnote{Whether co-occurrences between e.g. mirative particles, mirative intonation, and mirative syntactic structures are cases of redundancy is an empirical question. The findings presented below indicate that partial semantic overlap could be a more appropriate interpretation of such co-occurrences, at least when we are not dealing with mere repetition.}

Discourse particles share many traits with intonational markers. They are non-at-issue and can be broadly categorized as modal or discourse oriented, with many of them sensitive to anticipated context states, modality, and relative polarity \citep[47--48]{Waltereit.2006}. \citet[2033--2034]{Zimmermann.2011} goes so far as to say that ``in the absence of particles, English resorts to other grammatical means for expressing speaker and/or hearer attitudes towards a proposition [which] comprise intonation [\ldots] and sentence-final tags''.  We will see in \sectref{ch:5.1} that particles abound in marked Spanish utterances. Once we have established and understood their abundance, we can ask under which circumstances they occur with marked intonation (\sectref{ch:5.2}).

Turning the argument by Zimmermann around, we can see discourse particles as the lexical equivalent of marked intonation. This is why they are used prolifically in computer mediated discourse where prosody has to be replaced by other strategies \citep{Landone.2012}. I therefore propose to see discourse particles as an indicator of points of interest when searching for marked discourse moves, their function in natural dialogue, and their prosodic form. Most importantly for our current purpose, they allow us to explore possible caveats for the experimental investigation of marked intonation. 

Marked discourse moves in the sense of the modalized Farkas and Bruce Model (\sectref{ch:3.3}) are expected to be rare in natural dialogue. A speaker who repeatedly marks the content of her declaratives as unexpected would undermine the very modal base she exploits to do so. If everything you are willing to defend as a belief is unexpected from the Common Ground, then either the previous commitments of you and your interlocutors are false, or your assertions are. Both cases would undermine the main goal of conversation as defined in our model: to expand the Common Ground. Likewise, a speaker who repeatedly marks the content of her provocations or responses as expectable would come across as either a wiseacre or odd, because her assertions would \textit{sensu stricto} not fulfill their typical function of putting commitments up for agreement and assessment.

Rare phenomena require large corpora for a sufficient number of occurrences. Large natural dialogue corpora of Spanish that include both audio recordings and textual transcriptions are few and far between. Two such corpora are the PRESEEA Corpus \citep{PRESEEA.20142020} and the C-ORAL-ROM Corpus \citep{CrestiMoneglia.2005}. The sub-corpus PRESEEA Madrid Barrio de Salamanca (published in three volumes: \cite{CesteroManceraMolinaMartosParedesGarcia.2012,MolinaMartosParedesGarciaCesteroMancera.2014,ParedesGarciaCesteroManceraMolinaMartos.2015}) contains informal dialogues with residents of the Salamanca neighborhood.\footnote{In the following, I cite the entire corpus \citep{PRESEEA.20142020} as a shorthand for the three volumes. No other sub-corpora were used here.} After cleaning the corpus from annotations and headers, approximately 500k word tokens can be counted (approx. 33k turns). The C-ORAL-ROM Corpus for Spanish contains both public and private conversations and monologues from 410 spea\-kers, primarily from the Castilia region \citep[9]{CrestiMoneglia.2005}. Its size is appr. 200k word tokens. 

Neither corpus has fine grained prosodic annotation, only some general perceptional tagging of lengthening and prosodic breaks plus an intuitive use of exclamation and interrogation marks. Given that each corpus was transcribed by a series of transcribers which did not agree on criteria for exclamation and interrogation mark placement, these can only be used heuristically.

The PRESEEA Madrid Salamanca Corpus has several advantages compared to the C-ORAL-ROM Corpus. It has been recorded with a high level of sociolinguistic control, achieving complete gender and education-level balance and recording only residents of one neighborhood of Madrid. It is a series of interviews that follow a list of topics such as perceived levels of crime, perceived changes in the neighborhood, style of living, family, political opinions on climate change, personal experiences of danger, vacations, etc. The length and depth of the conversations, held in the interviewee's homes, allows for intimate and natural dialogue, yet maintains a high level of explicitness through its semi-formal style. The C-ORAL-ROM Corpus has been recorded non-systematically in different situations of daily personal life, which greatly reduces the accessibility of expectations, states of knowledge, and other social dynamics. I therefore use the PRESEEA Corpus as my primary source of observations, only occasionally drawing on examples from C-ORAL-ROM and other, purely textual corpora (e.g. Corpus del Español News on the Web, \citealt{Davies.20122019}) for side-notes or individual arguments.

\sectref{ch:5.1} establishes the precise meanings of the relevant discourse particles \textit{hombre}, \textit{claro}, \textit{anda}, and \textit{vaya}. Based on collocation analysis and context interpretation, I corroborate the proposal that both \textit{hombre} and \textit{claro} encode obviousness and differ in terms of their relative polarity functions. Moreover, I show that \textit{anda} and \textit{vaya} are also specified for acceptance of proffered propositions, with \textit{anda} and \textit{vaya} differing in the kind of modal evaluation of the accepted proposition.

\sectref{ch:5.2} presents the observations about intonation that can be made based on corpus examples detected via a search of semantically related discourse particles. The examples show that turns with \textit{claro} and \textit{hombre} both frequently show L* HL\% intonation in contexts where a proffered proposition is confirmed and marked as necessary from the \ac{CG}. Moreover, turns with \textit{claro} also show L+H* L!H\% intonation in contexts where the truth of a proposition deemed necessary from the \ac{CG} has been called into question. \sectref{ch:5.2} also discusses the problem of H$-$ phrasing in turns preceded by \textit{hombre} L* HL\%, which tends to violate syntactic mapping constraints in favor of a long rise to a late H$-$ before the nuclear contour. For \textit{anda} and \textit{vaya}, results are less consistent. While turns with \textit{anda} show mostly L+H* L\% or L+¡H* L\% nuclear contours, turns with \textit{vaya} almost exclusively receive L* L\% prosodic marking. \sectref{ch:5.3} draws some conclusions from these corpus-based observations and formulates the tasks for experimental investigation, which follows in \chapref{ch:6}.

\section{Functions of discourse particles: \textit{hombre}, \textit{claro}, \textit{anda}, \textit{vaya}}
\label{ch:5.1}\largerpage

As mentioned in \sectref{ch:2.3.4}, \sectref{ch:3.2} and \sectref{ch:3.3.1}, literature on statements of the obvious in Spanish makes recurrent reference to particles such as \textit{claro} and \textit{hombre}. Some of the examples we have seen in the respective sections are also preceded by \textit{pues} or a combination of several particles. In a similar fashion, mirative readings of exclamatives are often disambiguated by adding discourse particles such as \textit{anda} or \textit{vaya}, much as mirative exclamatives in the literature on English or German are often disambiguated via \textit{Wow!} or \textit{Mensch!} `man/human' \citep{Grosz.2012}. The sparse literature on \textit{anda} and \textit{vaya} hints at a tendency for these markers to indicate mirative meaning (\cite[52]{OctaviodeToledoyHuerta.20012002}, \cite[161]{BorregueroZuloaga.2015}). In the only large-scale corpus study on these particles, \citet[125]{Tanghe.2016} attributes \textit{asombro} `astonishment/wonder', \textit{sorpresa} `surprise', and \textit{incredulidad} `incredulity' to \textit{anda} in 62.2\% ($N=164$) of cases and to \textit{vaya} in 37.6\% ($N=85$) of cases. She tacitly takes these meanings to be mutually exclusive with cases of \textit{desacuerdo} `disagreement' and \textit{rechazo} `rejection'. In the model presented here, modal values are not automatically determined by relative polarity and vice versa. It is therefore necessary to explicitly test this assumption.

As discussed in \sectref{ch:3.3.2}, the \textit{Diccionario de partículas discursivas del español} pres\-ents two entries for both \textit{claro} and \textit{hombre}. For \textit{claro}, it proposes a more frequent \textit{acuerdo} `agreement' function and a less frequent \textit{sobreentendido} `obviousness' function \citep{PonsBorderia.2011}. For \textit{hombre}, \citet{BrizVillalba.2011} and \citet[32--40]{Briz.2012} distinguish two uses which differ in intonation (falling vs. rising) and communicate agreement or disagreement, respectively. The analysis is based on native-speaker intuition and the comparison of corpus examples. Such subjective evaluation of particle functions has been the main (if not only) approach pursued in research on Spanish discourse particles to date. It is seen as an ``inevitable heuristic prerequisite'' (\cite[114]{Tanghe.2016}, see also \cite[14]{GhezziMolinelli.2014intro}).

Semantic categorization is a delicate issue. Individual examples can create the illusion of a highly specific meaning by ignoring the variability of meanings in different contexts. Yet a large number of case-by-case interpretations is a method that is difficult to replicate and therefore almost incontestable. Tables~\ref{tab:frequencyparticlesPRESEEA} and \ref{tab:frequencyparticlesCORALROM} show that \textit{anda} and \textit{vaya} are less frequent by one order of magnitude than \textit{hombre} and \textit{claro}.\footnote{I could easily distinguish from non-particle matches by lack of syntactic integration. Non-particle uses of \textit{hombre} are nouns (\textit{el hombre} `the man/the human'). Non-particle uses of \textit{claro} are adverbs (\textit{tener algo claro} `be sure about sth.', \textit{claro que XP} `(it's) clear that XP'). No adjectival uses were attested. Non-particle uses of \textit{anda} `walk' and \textit{vaya} `go$_{SUBJV}$' are verbs.} Given their high frequency, \textit{claro} and \textit{hombre} allow for statistical investigation of their collocations using association measures. These serve to detect affinities a) among particles and b) between particles and other, semantically more transparent lexical items such as verbs and adverbs. In the following paragraphs, I make use of the fact that collocations, if mathematically implemented in the form of association measures (\cite[75--118]{Evert.2005}, \cite{BartschEvert.2014}), can supplement intuitions by objectively showing the importance of ``lexically and/or pragmatically constrained co-occurrences of at least two lexical items'' \citep[76]{Bartsch.2004}.\footnote{The underlying argument is that you ``should know a word by the company it keeps'' \citep[11]{Firth.1957}. This insight is of course not mine, but rather the foundation of distributional semantics \citep{Lenci.2018}. I only propose that you might get to know an intonational contour by the company it keeps, too.}

\begin{table}
	\begin{floatrow}
  \captionsetup{margin=.05\linewidth}
	\ttabbox{\begin{tabular}{lrr}	
	    \lsptoprule
	    Query  & Matches & Particles \\\midrule
		claro & 2026 & 1901 \\
		hombre & 962 & 897 \\
		anda & 71 & 69 \\
		vaya & 83 & 31 \\
		\lspbottomrule 
	\end{tabular}}
	{\caption{Number of query matches and particle tokens for expectation related particles from PRESEEA Corpus Madrid Barrio de Salamanca\label{tab:frequencyparticlesPRESEEA}}}%	
\ttabbox{\begin{tabular}{lrr}
        \lsptoprule
		Query  & Matches & Particles \\\midrule
		claro & 646 & 614 \\
		hombre & 201 & 156 \\
		anda & 58 & 56 \\
		vaya & 48 & 18 \\
		\lspbottomrule 
	\end{tabular}}
	{\caption{Number of query matches and particle tokens for expectation related particles from C-ORAL-ROM Spanish \citep{CrestiMoneglia.2005}\label{tab:frequencyparticlesCORALROM}}}
	\end{floatrow}
\end{table}

\autoref{tab:associationmeasuresMIclaroHIGH} shows the most frequent collocations of \textit{claro} in the PRESEEA Madrid Salamanca Corpus, obtained with AntConc \citep{Anthony.2018}. They are ranked by \ac{MI} as calculated over token frequencies obtained within a symmetric 7 word search window (3 left, 1 node, 3 right). \ac{MI} as used in corpus linguistics is a measure that compares the observed probability O of a co-occurrence between a \textit{node} word and its co-occurring \textit{collocate} in a corpus of size N with the expected probability E of the two words co-occurring by chance in the same corpus. AntConc uses the formulas as laid out in \citet{Stubbs.1995}, repeated in (\ref{ex:mutualinfo}) for convenience.\footnote{See \citet{ChurchHanks.1990}, \citet[35--40]{Evert.2005}, and \citet{Evert.20042010} for additional explanations.}

\begin{exe}
	\ex \label{ex:mutualinfo} 
	\begin{xlist}
	\ex $O = \frac{\text{frequency}(\text{node},\text{collocate})}{N}$

	\ex $E = \frac{\text{frequency}(\text{node})}{N} \cdot \frac{\text{frequency}(\text{collocate})}{N}$

	\ex $\text{MI}(n,c) = \log_2 \frac{O}{E} = \log_2 \cfrac{f(n,c) \cdot N}{f(n) \cdot f(c)}$
	\end{xlist}
\end{exe}

\ac{MI} cannot be used fruitfully without a frequency threshold (or a secondary ordering of significant collocations by frequency, as in \cite{Davies.20122019}), since low frequency data will receive overly high \ac{MI} scores due to sampling errors \citep{EvertKrenn.2001,EvertUhrigBartschProisl.2017}.\footnote{To explain this a little more in detail, let us have a closer look at the formula in (\ref{ex:mutualinfo}c). There are several ways for the \ac{MI} score to increase. The numerator can increase either by a larger corpus size N, or by a larger count of co-occurrences. We can assume for an increase in N to increase both $f(n)$ and $f(c)$, thereby lowering the \ac{MI} score. A larger corpus therefore should not yield unwarranted high \ac{MI} scores. A larger count of co-occurrences also increases \ac{MI}, particularly if either $f(n)$ or $f(c)$, or both, are low. This is just what we want, given that a high number of co-occurrences of infrequent forms should be less likely due to chance. Finally, a small denominator can lead to a high \ac{MI} score if there are some (possibly coincidental) instances of co-occurrence. This is the case we want to avoid by applying a frequency threshold. Every natural corpus will contain some infrequent, coincidental co-occurrences of infrequent forms. Imagine a 10k word corpus in which node \textit{a} and collocate \textit{b} co-occur 4 times ($f(a,b)=4$), with $f(a)=10$ and $f(b)=10$. The numerator would be 40k, the denominator 100, yielding $\log_2(400) = 8.64$. Now imagine the frequent collocate \textit{c} with $f(c) = 500$ and $f(a,c) = 10$. This yields $\log_2(20) = 4.32$. We see here that the low-frequency collocation receives an overly high score. Therefore, each comparison of \ac{MI} scores should be seen as a ranking between forms of similar frequency above the threshold.} \citet[24]{ChurchHanks.1990} propose a frequency threshold of 5 when calculating over a 5 word window. Ideal window sizes and frequency thresholds depend on the phenomenon under investigation. In the case of particles, the frequent succession of several particles before the beginning of the core sentence suggests a larger window size going beyond what has been called the \textit{pre-front field} \citep{Auer.1996,Schroeder.2006} and into the main sentence of the adjacent turns.\footnote{Research on German has developed a somewhat richer terminology for these syntactically non-integrated discourse related phenomena. However, \citet[72]{Wiltschko.2021} makes the claim that these positions are universally provided by the \textit{interactional structure} as part of the \textit{Interactional Spine}.} The standard setting in AntConc is a symmetric 11 word window (5L,5R), which I consider to be on the upper end of reasonable window sizes in spoken interaction. With increasing window size, the frequency threshold should also be increased. Since a lower limit for a frequency threshold is not universally defined, it needs to be related to the number of collocate types. For \textit{claro} (3L,3R), a frequency threshold of 50 reduces the number of collocate types by two orders of magnitude (from 2009 to 42), which means that the 1967 least frequent collocations have been excluded from the ranking. MI $>$ 3 is commonly seen as significant attraction between two collocates \citep[206]{Desagulier.2017}, which here includes the twelve top ranked types.\footnote{Note that the notion of significance should be taken as necessary, but not sufficient for an informed reading of the tables presented here. Above a certain frequency threshold, the ranking among collocates is more informative than absolute \ac{MI} score. This is why I present ranked tables instead of mere \ac{MI} values.} In \autoref{tab:associationmeasuresMIclaroHIGH}, I show these top twelve significant associations, plus \textit{no} and the cut-off frequency threshold.\footnote{The particle \textit{no} is shown due to its importance for polarity. The values in \autoref{tab:associationmeasuresMIclaroHIGH}, \autoref{tab:associationmeasuresMIclaro}, and \autoref{tab:associationmeasuresMIhombre} pass an additional Log-Likelihood $p<0.05$ test as implemented in AntConc.}

\begin{table}
	\begin{tabular}{ll *4{r}}
		\lsptoprule
		 &  &  & \multicolumn{3}{c}{Frequency} \\\cmidrule(lr){4-6}
		  Rank   &   Collocate        &  MI  & Total  & Left& Right\\\midrule
		1 & claro & 5.39 & 330 & 165 & 165 \\
		2 & entonces & 3.99 & 128 & 67 & 61 \\
		3 & porque & 3.88 & 210 & 126 & 84 \\
		4 & hombre & 3.85 & 54 & 32 & 22 \\
		5 & sí & 3.52 & 440 & 236 & 204 \\
		6 & eso & 3.44 & 135 & 36 & 99 \\
		7 & pero & 3.25 & 190 & 102 & 88 \\
		8 & es & 3.23 & 341 & 103 & 238 \\
		9 & todo & 3.22 & 74 & 41 & 33 \\
		10 & hay & 3.16 & 67 & 18 & 49 \\
		11 & también & 3.16 & 53 & 27 & 26 \\
		12 & como & 3.02 & 67 & 17 & 50 \\
		\ldots & \ldots & \ldots & \ldots & \ldots & \ldots \\
		21 & no & 2.72 & 430 & 221 & 209 \\
		\ldots & \ldots & \ldots & \ldots & \ldots & \ldots \\
		42 & o & 1.78 & 74 & 26 & 48 \\
		\lspbottomrule 
	\end{tabular}
	\caption{High-frequency collocations of \textit{claro} in PRESEEA Corpus Madrid Barrio de Salamanca (3 left to 3 right, threshold 50)}
	\label{tab:associationmeasuresMIclaroHIGH}
\end{table}

\autoref{tab:associationmeasuresMIclaroHIGH} contains several insights. First and foremost, repeated uses of \textit{claro} are frequent, both within one turn and in short successions of turns (\ref{ex:claroPRESEEAfuerte}).\footnote{All following examples from the PRESEEA corpus are presented without XML markup, without hesitation and laughter, with stretches of simultaneous or unintelligible conversation omitted, and with added boldface emphasis. Omission of speech is marked with \ldots, both within and between turns.} The frequency of \textit{claro} in successive turns is of some interest. If \textit{claro} in examples such as (\ref{ex:claroPRESEEAfuerte}) communicated nothing but agreement, we would have a hard time arguing that agreement is unmarked and mostly happens tacitly \citep{FarkasBruce.2010}. Yet I would argue that \textit{claro} is often used to go beyond agreement and to underline the expectability of the proposition that the interlocutors agree upon, which is a separate non-doxastic commitment and can therefore be negotiated separately, prompting such successive mutual reassurances. Such uses of \textit{claro} will correlate with specific intonational contours, which cannot be extracted from textual transcription.

\begin{exe} 
		\ex (Interview 37, \cite{PRESEEA.20142020}) \label{ex:claroPRESEEAfuerte} 
		\begin{xlist}[A:]
		\exi{A:} ¿\ldots si yo le pregunto que si se va a otra ciudad?
		\glt `\ldots if I ask you if you would go to another city?'
		\sn[] {\ldots}
		\exi{B:} hombre yo me hubiera adaptado \ldots pero / es distinto ¿no? \textit{claro} aquí es que estoy me encuentro bien es que \textit{claro} ¡es que he nacido aquí! / y \textit{claro} es muy fuerte ¡\textit{claro} lo!
		\glt `Man, I would have adapted \ldots but, it's different, right? \textit{Sure}, the thing is here I'm -- I feel good. It's that, \textit{sure}, it's that I was born here! And \textit{sure}, it's very hard, \textit{sure} it!'
		\sn[] {\ldots} 
		\exi{A:}\textit{claro claro claro} que
		\glt `\textit{sure, sure, sure} that'
		\exi{B:} ¡\textit{claro}! las raíces de aquí son 
		\glt `\textit{sure}, the roots are from here'
		\exi{A:} \textit{claro}
		\glt `\textit{sure}'
		\end{xlist}
	\end{exe}

In \autoref{tab:associationmeasuresMIclaroHIGH}, we also see that \textit{claro} is frequently used together with causal conjunctions such as \textit{porque} `because' and \textit{entonces} `then/therefore'. The standard context for these sequences are within longer turns that narrate a complex succession of events, where \textit{entonces claro}  or \textit{porque claro} mark the plausibility or expectability of a conclusion based on what has been introduced so far. On the other hand, \textit{hombre} as a collocate of \textit{claro} occurs mostly in successions of provocations and responses where stances are negotiated between interlocutors. These are the very contexts that are best covered by the Farkas and Bruce Model, and these are also the points of interest when searching for marked prosody with modal functions. 

\begin{exe}
\ex \label{ex:claroPRESEEArespeto} (Interview 44, \cite{PRESEEA.20142020})
	\begin{xlist}[A:]
	\exi{A:} ¿te molesta \ldots empiezas tratando de usted a alguien el otro te trate de tú / o al revés? 
	\glt `Does it annoy you \ldots you start addressing someone with usted and the other addresses you with tú -- or vice versa?' 
	\exi{B:}no \ldots molestar no me molesta 
	\glt `No. \ldots It doesn't really bother me.' 
	\sn[] {\ldots }
	\exi{A:}mientras sea con respeto ¿verdad? 
	\glt `As long as it's respectful, right?' 
	\exi{B:}hombre \textit{claro} / efectivamente  
	\glt `Man \textit{sure}, effectively.' 
	\end{xlist}
	
\ex \label{ex:claroPRESEEAestudiar} (Interview 23, \cite{PRESEEA.20142020})
	\begin{xlist}[A:]
	\exi{A:}\ldots ¿volverías a estudiar? 
	\glt `\ldots Would you study again?' 
	\exi{B:}¡ah / \textit{claro}! / sí sí  ¡hombre! / si volviese a repetir /  volviese a nacer otra vez / por supuesto / hubiese aprovechado \ldots 
	\glt `Ah! \textit{Sure}! Yes, yes! Man! If I'd get to repeat -- get to be born again, obviously I would have taken the opportunity. \ldots' 
	\end{xlist}

\ex \label{ex:claroPRESEEArobos} (Interview 23, \cite{PRESEEA.20142020})
	\begin{xlist}[A:]
	\exi{A:}¿tú has oído que haya ocurrido algo por aquí eeh / algún robo / alguna violación / o algo?
	\glt `Did you hear about anything going on round here eeh -- some robbery, some rape, or anything?' 
	\exi{B:}hombre robos sí / \textit{claro} / nos ha fastidiado \ldots 
	\glt `Man! Robbery yes. \textit{Sure}. It has bothered us.\ldots' 
	\end{xlist}
\end{exe}

\textit{Claro} shows a significant (MI $>$ 3) association with \textit{sí} and does not reach significance for \textit{no} (MI $<$ 3). Under the assumption that we are dealing primarily with relative polarity uses, this would be an argument for an agreement function. Since \textit{sí} and \textit{no} are ambiguous between relative and absolute polarity uses, case-by-case investigation is necessary. Inspecting contextualized individual examples gives direct access to relative polarity, modal meaning, and punctuation. Even though punctuation is not standardized in the two corpora under investigation, it gives an indication of the subjective impressions of the annotators about prosodic markedness. This could in turn give some indications about prosodic effects of particular combinations of modality and relative polarity, which then need to be corroborated by audio data (\sectref{ch:5.2}). 

I took the direct-adjacency-subset of occurrences of \textit{sí} and \textit{no} in the context of \textit{claro} to be particularly relevant for determining a possible relative polarity function of \textit{claro}. For \textit{sí}, there are 170 cases of direct adjacency,\footnote{Discounting 2 adjectival cases in constructions such as \textit{lo tengo claro, sí}.} 106 left, 57 right, and 7 cases of the sequence \textit{sí, claro, sí}. For \textit{no}, there are 152 cases of direct adjacency, 82 left, 68 right, and 2 cases of the sequence \textit{no, claro, no}.\footnote{Again discounting 2 adjectival uses.} \autoref{tab:corpusparticlesCLARO} shows the results of case-by-case investigation of the direct adjacency subsets of co-occurrences of \textit{claro} as node with either \textit{sí} or \textit{no} as collocates.

\begin{table}
	\begin{tabular}{l *{10}{r}}
		\lsptoprule
		      &                   &         &           &                \multicolumn{4}{c}{Uses in \ldots}  & \multicolumn{3}{c}{Modality} \\\cmidrule(lr){5-8}\cmidrule(lr){9-11}
		Node &  \rotatebox{90}{Collocate (1L-1R)} &  \rotatebox{90}{Matches} &  \rotatebox{90}{Particles} &  \rotatebox{90}{Provocations} &  \rotatebox{90}{Responses} &  \rotatebox{90}{- Same} &  \rotatebox{90}{- Reverse} &  \rotatebox{90}{Obvious} &  \rotatebox{90}{Mirative} &  \rotatebox{90}{Other/unclear} \\\midrule
		claro & no & 154 & 152 & 55 & 97 & 97 & 0 & 152 & 0 & 0 \\
		claro & sí & 172 & 170 & 4 & 166 & 166 & 0 & 159 & 0 & 11 \\
		Total &  & 326 & 322 & 59 & 263 & 263 & 0 & 311 & 0 & 11 \\
		\lspbottomrule & 
	\end{tabular}
	\caption{Number of provocations, (dis)agreeing responses, modalities of commitment, and exclamation marks in turns containing \textit{claro} adjacent to \textit{sí} or \textit{no} in the PRESEEA Madrid Salamanca corpus\label{tab:corpusparticlesCLARO}}
\end{table}

\autoref{tab:corpusparticlesCLARO} fully corroborates the agreement function of \textit{claro}, with virtually no exceptions among responding moves. As is to be expected from the distribution of \textit{sí} and \textit{no}, co-occurrences with \textit{claro} are mostly found in responses. The higher tendency for co-occurrences with \textit{no} in provocations is also expected, given that Spanish \textit{no} is ambiguous between relative and absolute polarity (English \textit{no} and \textit{not}).\footnote{Verum focus with preverbal \textit{sí} did not occur in the data.}

Similarly, almost all uses of \textit{claro} occurred in an assertion or confirmation in which expectability or obviousness was the most plausible reading of the context. Examples (\ref{ex:claroPRESEEAcomida}), (\ref{ex:claroPRESEEAcena}), (\ref{ex:claroPRESEEAnochebuena}), and (\ref{ex:claroPRESEEAreunis}) illustrate the consistency across speakers in the use of this particle when they want to confirm a previous biased question, and also want to indicate the necessity of this confirmation from the perspective of the previously existing \ac{CG}. Without being obligatory, a combination of \textit{sí} and \textit{claro} is the adequate response to a polar question that puts into question whether the interlocutor partakes in the most common festivity in the community: Christmas.

\begin{exe}
\ex \label{ex:claroPRESEEAcomida} (Interview 43, \citealt{PRESEEA.20142020}) 
	\begin{xlist}[A:]
	\exi{A:}¿preparáis algo o alguna comida especial o alguna \ldots? 
	\glt `You prepare something or some special food or some \ldots?' 
	\exi{B:}\textit{sí} \textit{claro}
	\glt `\textit{Yes. Sure.}' 
	\exi{C:}bueno sí / \textit{sí claro}
	\glt `Well yes \textit{yes sure}.' 
	\exi{B:}sí hombre ¿qué le gusta a fulano? \ldots 
	\glt `Yes man. What does anybody like?' 
	\end{xlist}

\ex \label{ex:claroPRESEEAcena} (Interview 39, \citealt{PRESEEA.20142020}) 
	\begin{xlist}[A:]
	\exi{A:}¿y el el día de de Navidad hacían alguna cena especial o alguna \ldots? 
	\glt `And on Christmas you did some special dinner or something \ldots?'	
	\exi{B:}\textit{sí} / \textit{claro} // bueno la cena como siempre se suele hacer en Navidad 
	\glt `\textit{Yes, sure}. Well, the dinner as it is always commonly done on Christmas' 
	\end{xlist}

\ex \label{ex:claroPRESEEAnochebuena} (Interview 49, \citealt{PRESEEA.20142020}) 
	\begin{xlist}[A:]
	\exi{A:}eeh/¿que lo celebráis con un / hay algún menú especial en Nochebuena? 
	\glt `Um / that you celebrate it with a / is there a special Christmas Eve menu?'	
	\exi{B:}pues pues / sí sí \textit{sí claro} eso por supuesto y además \ldots 
	\glt `Well well/ yes, yes, \textit{yes, sure}, obviously this and moreover \ldots' 
	\end{xlist}

\ex \label{ex:claroPRESEEAsoleishacer} (Interview 03, \citealt{PRESEEA.20142020}) 
	\begin{xlist}[A:]
	\exi{A:}\ldots ¿las celebráis en familia o / cómo? / ¿qué soléis hacer? 
	\glt `You celebrate them in family or / how? / what do you commonly do?' 
	\exi{B:}\textit{sí / claro} / las celebramos en nochebuena \ldots 
	\glt `\textit{yes, sure}, we celebrate them on Christmas Eve \ldots' 
	\end{xlist}

\ex \label{ex:claroPRESEEAreunis} (Interview 12, \citealt{PRESEEA.20142020}) 
	\begin{xlist}[A:]
	\exi{A:}\ldots ¿qué soléis hacer vosotros en navidad? ¿os reunís todos? 
	\glt `What are you doing for Christmas? Are you all coming together?'
	\exi{B:}\textit{sí ¡claro!} / pues nos juntamos mm determinados días de las navidades \ldots 
	\glt `\textit{yes, sure}! so we come together um certain days of Christmas \ldots' 
	\end{xlist}
\end{exe}


In some cases, \textit{claro} is used to reassure the interlocutor of a shared assumption about possibilities in the face of a disagreement. Apart from absolute and direct relative polarity, there seems to be a kind of indirect relative polarity that targets expectations and assumptions about possibilities. In example (\ref{ex:claroPRESEEAno}), A has established the position that immigration is a problem if some specific migrants use resources and ask for medical services. B, trying to object without explicitly pointing out the xenophobia, raises the point that immigrants also work, and A confirms this objection (positive relative polarity concerning a proffered proposition), reaffirms it as a general rule, and objects to the conversational implicature that his previous commitment `they are eating us' is inconsistent with the fact that they are working (negative relative polarity concerning a conversational implicature). After this negation via \textit{no}, the use of \textit{claro} reassures the interlocutor of a shared set of assumptions about possible worlds in which it is true that the migrants work. Note that he thereby does not commit to this world being the actual world, justifying his stance towards immigrants by restricting his agreement to possible worlds in which the one who gets medical treatment or has children is also the one who works.

\begin{exe} 
	\ex \label{ex:claroPRESEEAno} (Interview 44, \cite{PRESEEA.20142020})
	\begin{xlist}[A:]
	\exi{A:} porque nos están comiendo / y no sólo comiendo sino además exigiéndonos \ldots porque yo conozco tres o cuatro muchachas y lo primero a parir aquí / para tener hijos y que les den los papeles / y otros que si se tienen que operar del hígado otros se tienen que operar de otra cosa / o sea que vienen / \ldots 
	\glt `because they are eating us / and not only eating but also demanding from us \ldots because I know three or four girls and the first thing they do is give birth here / to have children so that they give them papers / and others if they need a liver operation others need a different operation / so they come \ldots' 
	\exi{}{\ldots}
	\exi{B:}pero por otra parte también están trabajando ¿no? \ldots 
	\glt `but on the other hand they also work, right? \ldots'
	\exi{A:} exactamente // \textit{sí sí no claro} por supuesto / el que el que está trabajando \ldots 
	\glt `exactly // \textit{yes yes no sure} obviously / the one who the one who works \ldots'
	\end{xlist}
\end{exe}

Another exception from clear-cut obvious uses are cases in which one interlocutor reminds the other of something and once the memory returns, \textit{claro} marks the acknowledgment of this fact having been in the \ac{CG} at some point. Finally, there are in total three cases of an insistent use of \textit{claro} which is not warranted by shared expectations. These may either be signs of the possibility to use \textit{claro} as a marker of certainty or with evidential, rather than expectational, meaning. To illustrate this, see (\ref{ex:claroPRESEEAyoquesabia}), in which A casually mentions that she has been robbed in her elevator. B responds with an incredulous question, to which B responds with \textit{sí claro}. Given the incredulity of A, B cannot base her use of \textit{claro} on shared expectations. While infrequent among the total number of cases, we see here the possibility for non-obvious uses of \textit{claro}.


\begin{exe} 
\ex \label{ex:claroPRESEEAyoquesabia}(Interview 48, \cite{PRESEEA.20142020})
	\begin{xlist}[A:]
	\exi{A:} \ldots pues a lo mejor te encuentras en el ascensor / y te atracan como a mí me atracaron
	\glt `\ldots so perhaps you find yourself in the elevator / and they rob you like they robbed me'
	\exi{B:}¿sí aquí en el ascensor?
	\glt `Really, here in the elevator?'
	\exi{A:} \ldots \textit{sí ¡claro!} ¡yo qué sabía! se metió un chaval en el ascensor y y y me atracó
	\glt `\textit{yes sure}! little did I know! a boy got into the elevator and and and robbed me \ldots'
	\end{xlist}
\end{exe}

Among the highly frequent collocations of \textit{claro} (\autoref{tab:associationmeasuresMIhombre}), there are no modal expressions. Yet \ac{MI} can also help us to compare between the wide range of mid-frequency collocations. The same low-frequency bias should apply to all collocations above the frequency threshold alike. When applying a (still relatively high) threshold of 9 to \textit{claro} (5L,5R), the number of collocate types is reduced by one order of magnitude (from 2823 to 248). This means that the 2575 least frequent collocations have been excluded, yet a range of mid-frequent collocations is still part of the calculation. \autoref{tab:associationmeasuresMIclaro} shows the resulting mid-frequency collocations in the PRESEEA Madrid Salamanca Corpus. We see here a corroboration of the proposal by \citet{PonsBorderia.2011} to attribute an obviousness function to \textit{claro}, given that \textit{lógico} `logical', \textit{lógicamente} `logically', \textit{por supuesto} `obviously', and \textit{evidentemente} `evidently' all indicate communicative intentions beyond agreement. All in all, I take these statistical associations as evidence for the double nature of \textit{claro}: polar and modal.\largerpage[2]

\begin{table}
	\begin{tabular}{ll *4{r}}
		\lsptoprule
		     &           &    & \multicolumn{3}{c}{Frequency}\\\cmidrule(lr){4-6}
		Rank & Collocate & MI & Total & Left& Right\\\midrule
		1 & lógico & 6.79 & 9 & 6 & 3 \\
		2 & lógicamente & 6.56 & 11 & 3 & 8 \\
		3 & distinto & 6.04 & 10 & 7 & 3 \\
		4 & claro & 5.69 & 406 & 203 & 203 \\
		5 & supuesto & 5.49 & 12 & 4 & 8 \\
		6 & evidentemente & 5.40 & 9 & 1 & 8 \\
		7 & habrá & 5.32 & 11 & 5 & 6 \\
		8 & efectivamente & 5.24 & 11 & 2 & 9 \\
		9 & fuerte & 5.06 & 11 & 9 & 2 \\
		10 & encima & 4.93 & 14 & 7 & 7 \\
		\ldots & \ldots & \ldots & \ldots & \ldots & \ldots \\
		248 & esta & 1.91 & 9 & 8 & 1 \\
		\lspbottomrule 
	\end{tabular}
	\caption{Mid-frequency collocations of \textit{claro} in PRESEEA Corpus Madrid Barrio de Salamanca (5 left to 5 right, threshold 9)\label{tab:associationmeasuresMIclaro}}
\end{table}

Turning to \textit{hombre}, the picture becomes a bit more complex. \autoref{tab:associationmeasuresMIhombre} shows the collocations of \textit{hombre} in the PRESEEA Madrid Salamanca Corpus. They are again ranked by \ac{MI} as calculated over token frequencies obtained within a symmetric 11 word search window (5 left, 1 node, 5 right).\footnote{The lower frequency of \textit{hombre} requires a larger window to reach a sufficiently large sample. All values in \autoref{tab:associationmeasuresMIhombre} again pass an additional log-likelihood $p>0.05$ test. Note further that $t$-score over a 1L-1R window, which has a more stable and quite different recall curve compared with \ac{MI} \citep[537]{EvertUhrigBartschProisl.2017}, also produces high scores for \textit{pues} (score 10.58, rank 1), \textit{sí} (score 7.94, rank 2), \textit{no} (score 7.77, rank 3), \textit{si} (score 6.31, rank 5) and \textit{claro} (score 5.47, rank 6).} The argumentative and modal nature of \textit{hombre} is clearly visible in its collocations. \textit{Creo} `I believe', \textit{si} `if/but/well' and \textit{claro} score highest in terms of \ac{MI}. For the modalizing function of \textit{hombre}, the strong association with \textit{claro} is particularly important. Instances of \textit{si} include conditionals and insubordinates, only the latter of which have expectational meaning (\cite{Schwenter.2016}, \cite{ElviraGarcia.2016}, \sectref{ch:2.3.4}). The collocations \textit{sé} `I know' and \textit{creo} indicate that epistemic and doxastic modalities are also compatible.

\begin{table}
	\begin{tabular}{ll *4{r}}
		\lsptoprule
		     &           &    & \multicolumn{3}{c}{Frequency}\\\cmidrule(lr){4-6}
		Rank & Collocate & MI & Total & Left& Right\\\midrule
		1 & creo & 4.78 & 59 & 13 & 46 \\
		2 & si & 4.52 & 111 & 22 & 89 \\
		3 & claro & 4.42 & 80 & 31 & 49 \\
		4 & también & 4.31 & 58 & 20 & 38 \\
		5 & pues & 4.17 & 216 & 112 & 104 \\
		6 & yo & 4.05 & 181 & 34 & 147 \\
		7 & ahora & 4.00 & 56 & 24 & 32 \\
		8 & sí & 3.93 & 278 & 169 & 109 \\
		9 & sé & 3.73 & 46 & 18 & 28 \\
		10 & no & 3.73 & 410 & 233 & 177 \\
		\ldots & \ldots & \ldots & \ldots & \ldots & \ldots \\
		30 & bueno & 3.02 & 48 & 35 & 13 \\
		\ldots & \ldots & \ldots & \ldots & \ldots & \ldots \\
		37 & y & 2.54 & 214 & 149 & 65 \\		
		\lspbottomrule 
	\end{tabular}
	\caption[Collocations of \textit{hombre}]{High-frequency collocations of \textit{hombre} in PRESEEA Corpus Madrid Barrio de Salamanca (5 left to 5 right, threshold 45)}
	\label{tab:associationmeasuresMIhombre}
\end{table}

Both \textit{sí} and \textit{no} are associated with \textit{hombre}. Since \textit{sí} and \textit{no} are ambiguous between relative and absolute polarity uses, case-by-case investigation was again necessary. I took the direct-adjacency-subset of occurrences of \textit{sí} and \textit{no} in the context of \textit{hombre} to be particularly relevant for determining a possible relative polarity function of \textit{hombre}. For \textit{sí}, there are 93 cases of direct adjacency, 63 left, 27 right, and 3 cases of the sequence \textit{sí, hombre, sí}. For \textit{no}, there are 108 cases of direct adjacency, 55 left, 49 right, and 4 cases of the sequence \textit{no, hombre, no}. \autoref{tab:corpusparticlesHOMBRE} shows the results of case-by-case investigation of the direct adjacency subsets of co-occurrences of \textit{hombre} as node with either \textit{sí} or \textit{no} as collocates.

\begin{table}
	\begin{tabular}{l *{10}{r}}
		\lsptoprule
		      &  & & & \multicolumn{4}{c}{Uses in \ldots}  & \multicolumn{3}{c}{Modality} \\
        \cmidrule(lr){5-8} \cmidrule(lr){9-11}
		Node &  \rotatebox{90}{Collocate (1L-1R)} &  \rotatebox{90}{Matches} &  \rotatebox{90}{Particles} &  \rotatebox{90}{Provocations} &  \rotatebox{90}{Responses} &  \rotatebox{90}{- Same} &  \rotatebox{90}{- Reverse} &  \rotatebox{90}{Obvious} &  \rotatebox{90}{Mirative} &  \rotatebox{90}{Other/unclear} \\\midrule
		hombre & no & 108 & 103 & 27 & 76 & 57 & 19 & 97 &  0 & 5 \\
		hombre & sí & 93 & 90 & 14 & 76 & 68  & 8  & 75 & 1 & 14 \\
		Total &  & 201 & 193 & 41 & 152 & 125 & 27 & 172 & 1 & 19 \\
		\lspbottomrule & 
	\end{tabular}
	\caption{Number of provocations, (dis)agreeing responses, modalities of commitment, and exclamation marks in turns containing \textit{hombre} adjacent to \textit{sí} or \textit{no} in the PRESEEA Madrid Salamanca corpus\label{tab:corpusparticlesHOMBRE}}
\end{table}
% \begin{sidewaystable}
% 	\begin{tabularx}{\textwidth}{ll|ll|llllXXX}
% 		Node & Colloc. & Matches & Par- & Provo- &  \multicolumn{3}{l}{Responses}  & \multicolumn{2}{l}{Modality} & Other/  \\
% 		 & (1L-1R) &  & ticles & cations & Total & Same & Reverse & Obvious &  Mirative & unclear \\
% 		\midrule
% 		hombre & no & 108 & 103 & 27 & 76 & 57 (full 34; & 19 (full 11;  & 97 & 0 & 5 \\
% 		& & & & & & part 23) & part 8) & & & \\
% 		\midrule
% 		hombre & sí & 93 & 90 & 14 & 76 & 68 (full 56; & 8 (full 4;  & 75 & 1 & 14 \\
% 		& & & & & & part 12) & part 4) & & & \\
% 		\midrule
% 		TOTAL &  & 201 & 193 & 41 & 152 & 125 & 27 & 172 & 1 & 19 \\
% 		\bottomrule & 
% 	\end{tabularx}
% 	\caption[Modality and relative polarity of \textit{hombre} adjacent to \textit{sí} or \textit{no}]{Number of provocations, (dis)agreeing responses, modalities of commitment, and exclamation marks in turns containing \textit{hombre} adjacent to \textit{sí} or \textit{no} in the PRESEEA Madrid Salamanca corpus.}
% 	\label{tab:corpusparticlesHOMBRE}
% \end{sidewaystable}

Much as with \textit{claro}, co-occurrences of \textit{sí} and \textit{no} with \textit{hombre} are mostly found in responses. \textit{No} co-occurs more often with \textit{hombre} in provocations than \textit{sí}. This is due to the fact that absolute polarity \textit{no} `not' is quite frequent, whereas verum focus with preverbal \textit{sí} did not occur in the data. Differently from \textit{claro}, \textit{hombre} does occur in reversals. Still, the tendency for agreement uses is very clear. A prevalence of agreement responses over reversals is a default assumption of the Farkas and Bruce model, also corroborated by empirical research \citep{BogelsKendrickLevinson.2015}. When comparing the agree-reverse distribution of \textit{no} (57 to 19) with the agree-reverse distribution of \textit{sí} (68 to 8), we actually find a significant association of \textit{no} with reversals; $\chi^2 (1, N = 152) = 5.44$, $ p = 0.02$; Cramér's V of 0.19 (small to medium effect, \cite[222]{Cohen.19882013}); adjusted standardized residuals of 2.33, $p<0.05$. This indicates that the relatively few instances of reversals tend to be reactions to assertions with positive absolute polarity, making \textit{no} a more likely candidate for reversals than \textit{sí}.

A frequent case, which I will call the expectational realignment use, is that \textit{no} marks a reversal and the following \textit{hombre} introduces an agreeing assertion at the level of expectations that underlie the provocation. This is a case that perfectly illustrates to which point the use of a seemingly expendable particle can be close to obligatory under certain pragmatic conditions. Examples (\ref{ex:hombrePRESEEAcopas}--\ref{ex:hombrePRESEEAsiempre}) illustrate the amount of consistency across speakers in the use of this particle when they want to assert a reversal of a previous biased question, yet also want to indicate the validity of the expectation underlying the bias.\largerpage[2]


\begin{exe}
\ex \label{ex:hombrePRESEEAcopas} (Interview 15, \cite{PRESEEA.20142020})
	\begin{xlist}[A:]
	\exi{A:}\ldots ¿hay otros problemas en el barrio violencia? 
	\glt `\ldots Are there other problems in the neighbourhood -- violence?' 
	\exi{B:}no // yo creo que no / \textit{hombre} Juan Bravo es una zona de copas y eso pero 
	\glt `No. I think no. \textit{Man} Juan Bravo is a nightlife area and such but.' 
	\end{xlist}
	
\ex \label{ex:hombrePRESEEAlados} (Interview 36, \cite{PRESEEA.20142020}) 
	\begin{xlist}[A:]
	\exi{A:}oye y el barrio / ¿cómo es de seguro? ¿hay delincuencia o? 
	\glt `Listen and the neighbourhood. Is it safe? Is there crime or \ldots?'
	
	\exi{B:}no 
	\glt `No.' 
	\exi{A:}¿se oyen cosas? 
	\glt `One hears things?'
	
	\exi{B:}no // no // \textit{hombre} / robos / atracos  / me imagino que como en todos lo lados 
	\glt `No. No. \textit{Man} robberies, hold-ups, I imagine just as everywhere.'
	\end{xlist}

\ex \label{ex:hombrePRESEEAnoche} (Interview 17, \cite{PRESEEA.20142020}) 
	\begin{xlist}[A:]
	\exi{A:}y ¿por aquí no se oyen así cosas de delincuencia ni? 
	\glt `And round here one doesn't hear about crime nor \ldots?' 
	\exi{B:}no / \textit{hombre} a lo mejor por la noche / eeh 
	\glt `No. \textit{Man} perhaps at night, right?' 
	\end{xlist}

\ex \label{ex:hombrePRESEEAdesalmado} (Interview 37, \cite{PRESEEA.20142020}) 
	\begin{xlist}[A:]
	\exi{A:}\ldots delincuencia por ejemplo hay? 
	\glt `\ldots is there crime for example?' 
	\exi{B:}no no aquí no este barrio es muy tranquilo ¡\textit{hombre}! no quiere decir que a uno / a cualquiera venga un desalmado y ¡no no no! / aquí inclusive 
	\glt `No, no, here this neighbourhood is very calm. \textit{Man} doesn't mean that one, anyone could be approached by a soulless person and no, no, no! Even here.' 
	\end{xlist}
	

\ex \label{ex:hombrePRESEEAsiempre} (Interview 01, \cite{PRESEEA.20142020})
	\begin{xlist}[A:]
	\exi{A:}¿hay  problemas de droga? \ldots // ¿de delincuencia? 
	\glt `Are there problems with drugs? \ldots With crime?' 
	\exi{B:}hay menos que en otros sitios 
	\glt `There's less than elsewhere.' 
	\exi{A:} ¿sí? 
	\glt `Really?' 
	\exi{B:}\textit{hombre} siempre hay en todos los lados ¿no?  
	\glt `\textit{Man} it everywhere always exists, right?' 
	\end{xlist}
\end{exe}

%%%	\begin{exe} 	
%		\ex \label{ex:hombrePRESEEAmejorado} \exi{A:}  pero yo creo que hay centros bastante buenos ahora de eso ¿no? \\
%		`But I think that there are some quite good centers of this sort now, right?' \medskip\\
%		\exi{B:} no / no lo creas no lo creas \\
%		`No, don't believe that, don't believe that.' \medskip\\
%		\exi{A:} ¿no? bueno claro es la información que tienes desde fuera \\
%		`No? Well, sure, that's what it looks like from the outside.' \medskip\\
%		\exi{B:} no \textbf{no} / ¡\textbf{hombre} ha mejorado mucho ¿eh?! no no no tch tch tch en eso te doy toda la razón // toda la razón te doy / [\ldots 16 frases \ldots] pero tch / yo tengo un sueño / de que estos chicos vivieran / libres \\
%		`No, \textbf{no}. \textbf{Sure/man} it has gotten a lot better, right?! No, no, no, tch, tch, tch, I completely agree with you on that. Completely agree with you is what I do [\ldots 16 phrases \ldots] But tch, I have a dream of these children living in freedom.' \medskip\\
%		(informant 54, \cite{PRESEEA.20142020}) \medskip\\
%	\end{exe}

Examples (\ref{ex:hombrePRESEEAcopas}--\ref{ex:hombrePRESEEAsiempre}) show that \textit{hombre} often introduces conversational moves that are difficult to classify as either provocations or responses, because the function of \textit{hombre} can be to acknowledge the fact that a reversal of proffered content is limited by a set of general, shared assumptions that generate expectations regarding the \ac{QUD} (nightlife areas are not completely safe, some hold-ups happen in the best neighborhoods, etc.). In this sense, \textit{hombre} here reassures the interlocutor(s) that a reversal concerning one proposition leaves a set of shared expectations about related propositions untouched. While somewhat similar to the much rarer phenomenon of indirect positive relative polarity with \textit{claro} exemplified in (\ref{ex:claroPRESEEAno}), this use of \textit{hombre} is special in that there is no overt provocation requesting reassurance about shared assumptions.

(\ref{ex:hombrePRESEEApaseo}) shows an example where \textit{hombre} introduces a turn that agrees with a previous non-at-issue commitment. The provocation \textit{p} asserts that A often walks home from Casa de Campo park. B's reaction does three things at the level of the discourse context: it asserts the proposition \textit{q} that the distance is quite far, tacitly accepts \textit{p}, and marks \textit{p} as unexpected. This indicates that a walk from Casa de Campo is further than B would have thought possible, based on what has been established between A and B up to the previous assertion (CG$_i$). After a few turns illustrating that A does indeed walk a lot, A returns to B's expectational non-at-issue commitment and explicitly reassures B that some of his expectations regarding B's walking-habits are still valid, stating that he would not go so far as to walk home all the way from El Escorial. This reassurance of shared expectations is introduced by \textit{hombre}.


\begin{exe} 
	\ex \label{ex:hombrePRESEEApaseo}(interview 20, \cite{PRESEEA.20142020})
	\begin{xlist}[A:]
	\exi{A:} \ldots y yo muchas veces me iba de la Casa de Campo a mi casa andando 
	\glt `\ldots and I often went from Casa de Campo to my home by foot' 
	\exi{B:}¡es un buen paseo! 
	\glt `That's quite a walk!' 
	\exi{}{\ldots }
	\exi{A:} \textit{¡hombre!} no me voy a ir desde el Escorial a mi casa andando  / eso se sería ya ¡vamos!
	\glt `\textit{Man}! I wouldn't walk home from El Escorial. That would be quite, come on!' 
	\end{xlist}
\end{exe}

(\ref{ex:hombreSHARONSTONE}) shows a typical example of a use of \textit{hombre} in an agreeing assertion. Here, \textit{hombre} is used repeatedly in an agreement, underlining the fact that the agreement is in line with some shared expectations related to male heterosexuality in the 1990s.

\begin{exe} 
	\ex \label{ex:hombreSHARONSTONE} (Interview 21, \cite{PRESEEA.20142020}) 
	\begin{xlist}[A:]
	\exi{A:} eeh ¿tienes algún tipo de? yo supongo que sí ¿no? algún // tipo de  de mujer ideal  o de 
	\glt `Eeh, do you have a sort of -- I assume you do, right? -- a sort of of ideal women, or of \ldots' 
	\exi{B:} \textit{hombre}  pues sí ¡\textit{hombre}! / ¡la mujer ideal Sharon Stone! 
	\glt `\textit{Man}, well yes, \textit{man}! The ideal woman Sharon Stone!
	\end{xlist}
\end{exe}



Summing up \autoref{tab:corpusparticlesHOMBRE}, we see that among the total of 193 occurrences of the particle \textit{hombre} directly adjacent to the polarity particles \textit{sí} and \textit{no}, only 27 are part of a reversal. Use of \textit{hombre} in provocations is more common with 41 cases, but still relatively infrequent compared with the 125 cases of confirming responses. I interpret this as a strong preference for confirmations.

The possibility of using \textit{hombre} in reversals is a crucial difference between \textit{hombre} and \textit{claro}. \textit{Claro} is pragmatically inappropriate in cases of reversal, where\-as \textit{hombre} only indicates that the disagreement does not originate in the set of shared expectations (\ref{ex:hombreclaro}). This insight seems important for research on intonation, since specific tonal configurations might also be more or less specified for relative polarity.

	
\begin{exe}
	\ex \label{ex:hombreclaro} 
	\begin{xlist}[A:]
	\exi{A:} ¿Tienes hijos?
	\glt `Do you have children?'
	\end{xlist}
	\begin{xlist}\judgewidth{\#}
		\ex	 
		\begin{xlist}[B:]
			\exi{B:}[]{\textit{Hombre}, aún no. Pero un día obviamente sí.
			\glt `\textit{Man}, not yet. But one day obviously yes.' }
		\end{xlist}
		\ex
		\begin{xlist}[B:]
			\exi{B:}[\#]{\textit{Claro}, aún no. Pero un día obviamente sí.
						 \glt `\textit{Of course}, not yet. But one day obviously yes.'
					    }
	    \end{xlist}
	\end{xlist}	
\end{exe}

Before we turn to the treatment of \textit{anda} and \textit{vaya} as markers of agreement and mirativity, some brief remarks on \textit{pues}. It has been described as a marker of \textit{new} information, given that it cannot occur without assertion (\ref{ex:puesBRIZ}).\largerpage

\begin{exe}
	\ex \label{ex:puesBRIZ} \citep{PorrocheBallesteros.2011}
	\begin{xlist}[A:]
	\exi{A:} ¿Qué impresión le daba?
	\glt `What did you think of it?' 
	\end{xlist}
	\begin{xlist} \judgewidth{\#}
		\ex \begin{xlist}[B:]
			\exi{B:}[]{\textit{Pues} me sentía con mucha ilusión.
					  \glt `\textit{Well} I was really looking forward to it.'}
	        \end{xlist}
		\ex	 \begin{xlist}[B:]
			\exi{B:}[\#]{\textit{Pues} ¿puedes repetirme la pregunta?
						 \glt `\textit{Well} could you repeat the question?'}
			 \end{xlist}
	\end{xlist}
\end{exe}

At first glance, frequent combinations such as \textit{pues claro} `well sure' (\ref{ex:claroPRESEEAnochebuena}, \ref{ex:claroPRESEEAreunis}), \textit{pues hombre} `well man' (\ref{ex:hombreSHARONSTONE}), or \textit{pues por supuesto} `well of course' seem to run counter to the idea of an obviousness commitment of the respective conversational moves. Something new should not be expected. Yet this is a misunderstanding that only arises if we fail to distinguish between the commitment itself (consisting of a proposition and its modal evaluation, in this case an expectation) and the CG update that takes place once the commitment is either asserted (\textit{at-issue}) or directly added to CG (\textit{non-at-issue}). A direct CG update as in (\ref{ex:declarativeoperatorrevised1}) is still an update and thereby ``new'' from the perspective of the input discourse context K$_i$. Only if no such update is advanced does the use of \textit{pues} become ungrammatical.\footnote{This is why, when searching for sequences of the form <\textit{¿ pues WH-PRONOUN}> in the Corpus del Español News on the Web \citep{Davies.20122019}, many instances are rhetorical questions that do proffer a context update, as in (\ref{ex:NOWcarreteras}).

\begin{exe}
		\ex \label{ex:NOWcarreteras} 
		Si los funcionarios se trasladan en aviones privados, \textit{¿pues cuándo} van a conocer el estado de las carreteras?: ¡Nunca! \\
		`If the officials move around in private jets, \textit{well when} will they get to know the state of the highways?: Never!'
\end{exe}
	
One desirable outcome of such a context-update-perspective on the particle \textit{pues} `well' is that it links it with the causal conjunction use (\textit{pues} `because/so') common in turn-internal position in longer monological sections of text and speech, which is also impossible if it does not introduce a context update.}

\textit{Anda} and \textit{vaya} seem similar to \textit{hombre} and \textit{claro} in their relative polarity function, yet different in their modal evaluative function. A closer look at them reveals some further differences. \textit{Anda} and \textit{vaya} are one order of magnitude less frequent than \textit{claro} and \textit{hombre}, both in the PRESEEA Madrid Salamanca corpus (\autoref{tab:frequencyparticlesPRESEEA}) and in C-ORAL-ROM (\autoref{tab:frequencyparticlesCORALROM}). Moreover, while \textit{hombre}, \textit{claro}, and \textit{anda} are almost exclusively used as particles, \textit{vaya} is only rarely used as a particle since the majority of uses has some sort of syntactic integration. 

There is also a key difference between \textit{anda} and \textit{vaya}, namely that \textit{vaya} is used to mark acceptance of a negatively evaluated proposition, whereas \textit{anda} does not communicate a bouletic evaluation. This raises the question if the mirative meaning of \textit{vaya}, firmly acknowledged in the literature, is part of its meaning or rather a conversational implicature. (\ref{ex:OctavioToledoyHuerta}) is cited by \citet[52]{OctaviodeToledoyHuerta.20012002} to show that \textit{vaya} is a discourse particle ``enriched with additional meanings, particularly of surprise about a situation''.\footnote{``Un marcador discursivo que se irá enriqueciendo con significados adicionales, particularmente el de sorpresa ante una situación.'' \citep[52]{OctaviodeToledoyHuerta.20012002}}

\begin{exe} 
	\ex \label{ex:OctavioToledoyHuerta} Monserrate:~De manera que dices que Ruçafa no tiene madre, sino que la muger es hija de Ruçafa, y la hija que está mala ha de traer el bollo mantecada.\\
	`So you are saying that Ruçafa doesn't have a mother, but that the woman is the daughter of Ruçafa, and the bad daughter has to bring the pound cake.'\\
	Coladilla:~Que no, sino qu'en Ruçafa está una muger mala, y ha de venir su hija a traer dos reales y el bollo mantecada para entrambos. \\
	`But no, rather that Ruçafa is a bad woman, and her daughter has to come bring two reales and the pound cake for both of them.'\\
	Monserrate:~¡Vaya! Sea como fuere; venga el bollo mantecada. \\
	`Damn! Be it as it may; let the pound cake come.'\\
	Lope de Vega, \textit{Ursón y Valentín}, 1588--1595, \textit{apud} CORDE (\citeauthor{RealAcademiaEspanola.CORDE})
\end{exe}

Following \citet{Kratzer.2012}, I take every modal to denote only one modal base. In the case of \textit{vaya}, the negative bouletic interpretation is also present in adjectival use as in \textit{vaya mierda} `damn shit' or \textit{vaya coche roto} `damn broken car'. The mirative interpretation, on the contrary, is dependent on \textit{vaya} evaluating the acceptance of a proposition or state of affairs. It therefore seems that the mirative meaning of \textit{vaya} is actually a conversational implicature of the (degree of) negative evaluation of a proposition. Further research, possibly using the semantic differential technique \citep{OsgoodETAL.1967,Kohler.2005} or the GRID technique \citep{FontaineSchererSoriano.2013}, is necessary to check if this interpretation holds in different contexts.	

%\footnote{A similar observation can be made for the difference between German \textit{Mensch} `man/human' and German \textit{Mann} `man'. While \textit{Mensch} in (\ref{ex:menschmann}a) is possible in positive evaluations, \textit{Mann} in (\ref{ex:menschmann}b) is slightly odd. 

%\begin{exe}
%		
%			\ex \label{ex:menschmann} 
%			\begin{xlist}
%				\ex	 Mensch, da ist meine verlorene Uhr! Wie schön! \\
%				`Man/human, there's my lost watch! How nice!' \\
%				\ex	 ? Mann, da ist meine verlorene Uhr! Wie schön! \\
%				`Man, there's my lost watch! How nice!' 
%			\end{xlist}	
%		
%\end{exe}

%This may serve as an argument against the idea that the modal meaning of such ``vocative'' particles is but a conversational implicature of a vocative.}

For \textit{anda} and \textit{vaya}, low frequency in the two oral corpora under investigation impedes computing collocations based on association measures. Case-by-case evaluation is therefore the most fruitful approach. \autoref{tab:corpusparticlesANDAVAYA} shows the global results for \textit{anda} and \textit{vaya}. While almost all instances of \textit{anda} are particles, only 31 out of 83 uses of \textit{vaya} are particles. Among the particle uses, \textit{anda} and \textit{vaya} occur in both provocations and responses. There is a tendency toward responses, which only reaches statistical significance for \textit{anda}; $\chi^2(1, N = 69) = 5.232,\allowbreak p= 0.02$. Yet among the responding moves, we find almost exclusively agreeing responses.

\begin{table}
\begin{tabular}{lrr}
\lsptoprule
               &  \multicolumn{2}{c}{Query}\\\cmidrule(lr){2-3}
               &  \itshape anda   & \itshape  vaya\\\midrule
Matches        &   71    &  83 \\
Particles      &   69    &  31 \\
~~~Provocations   &   25    &  14 \\
~~~Responses      &         &     \\
~~~~~~Total        &   44    &  17 \\
~~~~~~Same         &   40    &  17 \\
~~~~~~Reverse      &   4     &  0  \\
Modality       &         &     \\
~~~Obvious     &   0     &  0  \\
~~~Mirative    &   45    &  0  \\
~~~Other/unclear  &    24   &  31 \\
Excl. marks    &    25   &   8 \\
\lspbottomrule
\end{tabular}
\caption{Number of query matches, particles, provocations, (dis)agreeing responses, modalities of commitment, and exclamation marks for \textit{anda} and \textit{vaya} in the PRESEEA Madrid Salamanca corpus\label{tab:corpusparticlesANDAVAYA}}
\end{table}

Only 5 out of 25 cases of particle uses of \textit{anda} in provocations encode mirativity, all based on visual or direct evidence for something unexpected. (\ref{ex:andaPRESEEAsanjose}) is one of the rare examples for provocation miratives. This small group is consistently marked with exclamation marks. The 21 remaining uses of \textit{anda} in provocations introduce greetings, exhortatives, conclusions, and reformulations.\pagebreak

\begin{exe} 
\ex (Interview 48, \cite{PRESEEA.20142020})\label{ex:andaPRESEEAsanjose} 
	\begin{xlist}[A:]
	\exi{A:} la climatología ha cambiado mucho porque yo / me acuerdo que cuan\-do era jovencita bueno pues / cuando llegaba San José / nosotros ense\-ñá\-ba\-mos ya los trajes de / de entre tiempo 
	\glt `the climatology has changed a lot because I / remember that when I was young well so / when Saint Joseph's Day came / we already put out the light clothes'
	\exi{B:} claro
	\glt `sure'
	\exi{A:} el abrigo de entretiempo / los trajes de chaqueta de entretiempo / ¡pero \textit{anda}! hoy el día de San José / estaba nevando 
	\glt `the light coat / the light jackets / but \textit{wow}! today on Saint Joseph's Day / it was snowing'
	\end{xlist}
\end{exe}

40 out of 45 responding moves with \textit{anda} are agreements. Among them, 37 out of 40 are miratives, which suggests that mirative \textit{anda} primarily has a double function of accepting a proffered provocation and marking it as unexpected. (\ref{ex:andaPRESEEAscout}) and (\ref{ex:andaPRESEEAtraqueotomia}) give examples of this prototypical function.

\begin{exe} 
	\ex \label{ex:andaPRESEEAscout} 
	(Interview 10, \cite{PRESEEA.20142020})
	\begin{xlist}[A:]
	\exi{A:} has estado fuera me has dicho un tiempo\ldots fin de semana 
	\glt `you have been outside you said once\ldots weekend' 
	\exi{B:} este primero no / el último estuve yo soy scout \ldots 
	\glt `this first one no / the last one I have I'm a scout \ldots' 
	\exi{A:} \textit{anda} / eres scout
	\glt `\textit{wow} / you're a scout' 
	\end{xlist}
	\ex \label{ex:andaPRESEEAtraqueotomia} (Interview 42, \cite{PRESEEA.20142020}) 
	\begin{xlist}[A:]
	\exi{A:}\ldots mi padre le han operado // hace cuatro años también // de un cáncer de laringe 
	\glt `\ldots my father had an operation // four years ago as well // of a laryngeal cancer' \ldots 
	\exi{B:} y está / la traqueotomía
	\glt `and there is / the tracheotomy' 
	\exi{A:} ¡no!
	\glt `no!'
	\exi{B:} ¿no?
	\glt `no?' 
	\exi{A:} se lo / lo cogieron muy bien
	\glt `they got it out really well' 
	\exi{B:}\textit{¡anda!}
	\glt `\textit{wow}!' 
	\end{xlist}
\end{exe}

\textit{Vaya} differs from the other particles seen so far in that the majority of matches are non-particle uses (verbs). \textit{Vaya} is also different in that it is not specified for either provocation or response use. The ratios of particle uses with exclamation marks for \textit{vaya} is lower than for \textit{anda}, foreshadowing the findings in \sectref{ch:5.2.3} and \sectref{ch:5.2.4} that show L+¡H* L\% to be the nuclear contour of choice in turns containing \textit{anda} and L* L\% in turns with \textit{vaya}. (\ref{ex:vayaPRESEEAaccidente}), (\ref{ex:vayaPRESEEAsuspendia}), and (\ref{ex:vayaPRESEEAgenio}) give examples of the use of \textit{vaya} in accepting responses of previous provocations.

\begin{exe} 
\ex \label{ex:vayaPRESEEAaccidente} (Interview 34, \cite{PRESEEA.20142020}) 
	\begin{xlist}[A:]
		\exi{A:} mi padre murió en un accidente de coche 
		\glt `my father died in a car accident' 
		\exi{B:} ¿ah sí? / \textit{vaya} 
		\glt `oh really? / \textit{damn}'  
	\end{xlist}
		
\ex \label{ex:vayaPRESEEAsuspendia} (Interview 11, \cite{PRESEEA.20142020}) 
	\begin{xlist}[A:]
		\exi{A:} \ldots y nada luego pues lo que pasa es que la mayoría de la gente con la que yo iba acababa aprobando y yo suspendía 
		\glt `and so then the thing is that the majority of people I went with ended up passing (the exam) and I failed' 
		\exi{B:} ¡ah \textit{vaya}! 
		\glt `ah \textit{damn}!'  
	\end{xlist}
\ex \label{ex:vayaPRESEEAgenio} (Interview 47, \cite{PRESEEA.20142020}) 
	\begin{xlist}[A:]
		\exi{A:} pues mira mi marido ya no está pero\ldots 
		\ldots sus hijos han salido a su padre 
		\glt `so look my husband isn't with us any more but\ldots 
		\ldots his children look like him' 
		\exi{B:} ¿los tres? 
		\glt `all three?' 
		\exi{A:} los tres 
		\glt `all three'  
		\exi{B:} ¡\textit{vaya}! ¡y ninguno a ti!  
		\glt `\textit{damn}! and none like you!'  
	\end{xlist}
\end{exe}

As mentioned above, \textit{vaya} has a more negative connotation than \textit{anda}. Recent deaths of close relatives are always evaluated with \textit{vaya} as an agreement particle. Similarly, \textit{vaya} used as an adjectival modifier would usually precede nouns referring to commonly dispreferred referents such as \textit{vaya enfermedad} `damn illness', \textit{vaya palo} `damn bummer', etc. On the contrary, \textit{anda} cannot be used as an adjective. This observation may be linked with a second observation, namely that \textit{anda} as an unexpectedness marker, which is the vast majority of tokens, is used exclusively by female speakers or male speakers reporting speech of a female. This seems to indicate that male speakers in the community represented by the corpus largely abstain from conventionally implicating that they did not expect something, but rather resort to a strongly negative evaluation that then conversationally implies unexpectedness.\footnote{If a similar restriction holds for intonational marking of mirativity, this may heavily influence experimental results.} The proposal to see the mirative meaning of \textit{vaya} as derived from its negative bouletic meaning via conversational implicature might seem \textit{ad hoc}. Yet in 28 out of 31 cases, this implicature seemed present in the corpus examples. Moreover, even though conversational implicatures are usually thought of as less consistently present and more context dependent than conventional implicatures or lexical meaning, large written databases corroborate the consistency of the counterexpectational use of \textit{vaya} when used as an adjectival modifier in NP-exclamatives. \autoref{tab:corpusdelespanolVAYAnoun} shows the four most frequent significant collocations of the form \textit{vaya} + NOUN in the 7.2 billion words Corpus del Español News on the Web \citep{Davies.20122019}.\footnote{I excluded the non-nominal \textit{ir preso/presa} `to go to jail'. Note that \citet{Davies.20122019} applies a significance threshold ($\text{\ac{MI}}> 3$) and then sorts results by frequency. This is an alternative to applying a frequency threshold and then sorting by \ac{MI}, as done by AntConc.} While written and mostly monological, the database still shows the tendency for \textit{vaya} to relate to surprise. \textit{Vaya sorpresa} `damn surprise' and \textit{vaya paradoja} `damn paradox' are two uses with clear counterexpectational meaning. \textit{Vaya tela}, an idiom best translated as `wow', also shows that \textit{vaya} seems on track to include the mirative conversational implicature into the lexical meaning. Only \textit{vaya mierda} still maintains a clear bouletic evaluative function.\footnote{Future research should investigate if male speakers avoiding \textit{anda} are leading language change toward a mirative use of \textit{vaya}.}

\begin{table}
	\begin{tabular}{llrr}
		\lsptoprule
		Rank & Collocate & Frequ. & MI  \\\midrule
		1    & sorpresa & 464     & 5.05  \\
		2    & paradoja & 372     & 8.55  \\
		3    & mierda   & 312     & 6.83  \\
		4    & tela     & 300     & 7.65  \\
		\lspbottomrule 
	\end{tabular}
	\caption{Most frequent $\text{MI}>3$ noun-type collocations of \textit{vaya} in the Corpus del Español News on the Web (0 left to 1 right)\label{tab:corpusdelespanolVAYAnoun}}
\end{table}

Turning to the use of exclamation marks as added by the transcribers of the PRESEEA Madrid Salamanca corpus, we can ask whe\-ther relative polarity or modality correlates with the presence of such punctuation. \autoref{tab:corpusparticlesEXLCrelpol} shows the ratios of exclamation marked particles in provocations and (dis)agreeing responses.\footnote{For \textit{claro} and \textit{hombre}, again only the direct adjacency subsets with \textit{sí} and \textit{no} were considered to maintain a constant sample.} In general, not more than a third of particle uses are marked with exclamation marks. Since transcribers did not receive acoustic criteria for such marking, we cannot expect this to faithfully represent the amount of prosodically marked uses. Yet it is a way of getting a broad idea of the ratios of uses that were sufficiently marked prosodically so as to lead to a transcription with exclamation marks. The clearest result for exclamation marks is that responses are more often marked than provocations.

\autoref{tab:corpusparticlesEXLCmodal} shows the ratios of exclamation marked particles according to evaluative modality (mirativity, obviousness) or other, non-modal meaning. Miratives seem more prone to be marked for exclamation than obviousness uses, but both uses receive exclamation marks in a number of cases. The crucial questions regarding the intonational reality behind these exclamation marks are a) whether the particles themselves receive different intonational marking for the respective uses and b) whether sentences introduced by these particles receive a particular form of intonational marking.

\begin{table}
	\begin{tabular}{l *5{r@{}>{$/$}l}}
	\lsptoprule
          & \multicolumn{2}{c}{} & \multicolumn{8}{l}{Exclamation marks in uses marking \ldots}\\\cmidrule(lr){4-11}
		 {Particle} & \multicolumn{2}{c}{Excl. marks} & \multicolumn{2}{c}{Provoc.} & \multicolumn{2}{c}{Resp.} & \multicolumn{2}{c}{Same} & \multicolumn{2}{c}{Rev.} \\
		\midrule
		claro &  31 & 322   & 2 & 59   & 29 & 263   & 29 & 263   & 0 & 0  \\
		hombre & 38 & 193   & 7 & 41   & 31 & 152   & 25 & 125   & 6 & 27 \\
		anda   & 24 & 69    & 7 & 25   & 17 & 44    & 17 & 40    &  1 & 4  \\
		vaya   & 8 & 31     & 1 & 14   &  7 & 17    & 7 & 17     &  0 & 0 \\
		\lspbottomrule
	\end{tabular}
	\caption{Ratios of exclamation marked particles \textit{hombre}, \textit{anda}, and \textit{vaya} by provocations and (dis-)agreeing responses in the PRESEEA Madrid Salamanca corpus\label{tab:corpusparticlesEXLCrelpol}}
\end{table}


\begin{table}
	\begin{tabular}{l *4{r@{}>{$/$}l}}
	\lsptoprule
		           &  \multicolumn{2}{c}{}  & \multicolumn{6}{l}{Exclamation marks in uses marking \ldots} \\\cmidrule(lr){4-9}
		{Particle} &  \multicolumn{2}{c}{Excl. marks} &  \multicolumn{2}{c}{Obviousness} &  \multicolumn{2}{c}{Mirativity} & \multicolumn{2}{c}{Other} \\
		\midrule
		claro &   31 & 322   &  27 & 253   &  0 & 0    &  4 & 11  \\
		hombre &  38 & 193   &  35 & 172   &  1 & 1    &  2 & 19  \\
		anda   &  25 & 69    &   0 & 0     & 22 & 45   &  3 & 24   \\
		vaya   &  8 & 31     &   0 & 0     &  0 & 0    &  8 & 31  \\
	\lspbottomrule
	\end{tabular}
	\caption{Ratios of exclamation marked particles \textit{hombre}, \textit{anda}, and \textit{vaya} by modalities of commitment in the PRESEEA Madrid Salamanca corpus\label{tab:corpusparticlesEXLCmodal}}
\end{table}

In a nutshell, this exploration of discourse particle functions in the PRESEEA Madrid Salamanca corpus has paved the way for an answer to question (\ref{ex:questionsoverview}e) regarding correlations between intonation and other non-at-issue markers in Spanish. The categories developed in \sectref{ch:3.3} readily lend themselves to categorizing the functions of \textit{claro}, \textit{hombre}, \textit{anda}, and \textit{vaya}. All particles under investigation occur in provocations and responses, yet with a clear tendency toward responses. Within responses, they are also all specified for positive relative polarity, some categorically (e.g. \textit{claro}, \textit{vaya}), some gradually (e.g. \textit{hombre}, \textit{anda}).\largerpage[2]

For question (\ref{ex:questionsoverview}e) regarding correlations between intonation and other non-at-issue markers in Spanish, the important insight is that prosodic marking represented by exclamation marks is present in both obvious and mirative uses of particles. The nature of this prosodic marking needs to be investigated based on audio-files, which is the topic of \sectref{ch:5.2}.

\section{Intonation and discourse particles}\label{ch:5.2}

The particles investigated in \sectref{ch:5.1} are all two-syllable words with the typical Spanish lexical stress on the penultimate syllable. They often receive their own prosodic phrase, yet the status of the prosodic break between a particle and a following sentence can range from a simple word boundary over an a intermediate phrase boundary (marked with the minus sign $-$ in Sp\_ToBI) to an intonational phrase boundary (marked with the percent sign \% in Sp\_ToBI).\footnote{The two studies that investigate the prosodic integration of Spanish particles replace this three-way distinction of breaks with a two-way distinction ($\pm$ inclusion in the intonational phrase) (\cite[208]{CabedoNebot.2013marcadores}, \cite[138]{Tanghe.2015}).} In terms of prosodic independence, the simplest case are one-word turns. As seen in \sectref{ch:5.1}, many particles do not occupy a position preceding a full sentence. Rather, ``bare'' particles are often used in turns that do not contain inflected verbs and full sentences, precisely because the proposition under investigation is already given and accessible for anaphoric reference. Provocation uses are more likely to overtly assert the propositional content, yet, as seen in many examples in \sectref{ch:5.1}, some responses also do. When there is an overt assertion in a confirming turn, exact repetition of the entire provocation is the least economical strategy, violating the Maxim of Manner due to unnecessary prolixity \citep[46]{Grice.1975}.\footnote{In languages that confirm via partial repetition of the provocation (e.g. Portuguese), a full repetition of the provocation (including arguments) is still not economical. Repetitions of one-word provocations are of course an exception.} In some cases, such as (\ref{ex:claroPRESEEAfuerte}), additional assertions sum up previous provocations instead of confirming them. In other cases, such as (\ref{ex:claroPRESEEArobos}), they indicate that even a stronger claim than the one that has been proffered would have received confirmation. A case very similar to true repetition of a provocation is when a provocation is confirmed and then reasserted in other words, as in (\ref{ex:claroPRESEEAestudiar}). Actual repetition of a provocation in a confirmation is also possible, but seems more common with particles that are not obligatorily specified for positive relative polarity, such as \textit{anda} in (\ref{ex:andaPRESEEAscout}).

As already mentioned in \sectref{ch:3.3.2}, the particles themselves can have different prosodic realizations. We have seen that \citet{Briz.2012} distinguishes two \textit{hombre} particles, one with falling or low intonation (\autoref{fig:hombre1BRIZ}) and one with low-rise-falling intonation (\autoref{fig:hombre2BRIZ}). For \textit{claro}, we could extract one example from \citet{PonsBorderia.2011} showing a low-rise-fall intonation similar to the one reported in \citet{TorreiraGrice.2018}. We do not know if other prosodic realizations of \textit{claro} do occur. For \textit{anda} and \textit{vaya}, we do not have any ToBI analysis.\footnote{\citet{Tanghe.2015}, which investigates the prosody of \textit{anda} and \textit{vaya} among other verb-based particles, only takes into consideration mean F$_0$ values of the entire word, neglecting alignment differences.} In an attempt to evaluate the degree to which full sentences are marked prosodically so as to add intonational meaning beyond information structure, particles are a small start. As one-word phrases, they cannot have a focus-background partition. Yet they allow us to test some assumptions of the model developed in \sectref{ch:3.3} and identify points of interest for the investigation of sentences including verbs with overt arguments. Taking discourse particles as indicators for points of interest for pro\-so\-dic investigation has the advantage of dramatically reducing the amount of possible target turns. Moreover, they indicate sections of dialogue in which both discourse commitments and modal expectations are negotiated. 

The PRESEEA Madrid Salamanca Corpus was not designed for intonation research. The quality of the recordings often impedes investigation of intonation, and the spontaneous nature of interactions often leads to simultaneous speech or interruption by laughter or hesitation. Unfortunately, statistical comparison between the frequencies of intonational contours in the corpus can therefore not achieve internal validity. Factors such as simultaneous speech have a stronger influence on marked obvious uses of \textit{claro} than on unmarked agreement uses. If a provocation is seen as unnecessary and the response as expectable, the response is more likely to be uttered before the previous turn has come to an end, leading to simultaneous speech. While excluding these phonetically non-transparent cases would seem like a simple reduction of noise in the data, it would actually cause a selection bias in favor of modally unmarked utterances. Another factor that adds to this problem is the relative prevalence of laughter and hesitation in cases of obviousness, which is probably due to the face-threatening potential of obviousness in responding moves. I therefore postpone any statistical comparison between contours to the experimental investigation in \chapref{ch:6}. Instead, I attempt a qualitative exploration of the intonation of corpus examples containing \textit{claro}, \textit{hombre}, \textit{anda}, and \textit{vaya}, giving only tentative indications with regard to the prevalence of certain contours in the sample.

Qualitative ``close readings'' of individual examples, while no replacement for quantitative examination, are a useful and necessary step to illustrate the sensitivity of intonation to differences in discourse meaning. A closer look at individual examples also helps us to avoid the impression that particles have one prototypical intonational form from which speakers deviate only due to performance or frequency effects. We do expect the semantic affinity between particle meaning and intonational meaning to cause a correlation between specific nuclear contours and certain particles.\footnote{We can also expect an association between lexemes and the phonetic detail of pitch accents as shown for Germanic languages by \citet{SchweitzerETAL.2015}.} Yet such a correlation should not obscure the functional load of prosodic form. As will become clear below, all particles under investigation here allow for categorically different prosodic realizations under the right pragmatic conditions.

Before we turn to the prosodic investigation, let's recapitulate the open questions based on the state of the art and the model of discourse meaning as formulated in \sectref{ch:3.3.3}. Concerning the difference between L* L\% and L+H* L\%, both Tables~\ref{tab:intonationalcategoriesPRIETO} and \ref{tab:intonationalcategoriesGABRIEL} assume ``free'' variation.\footnote{\citet[364]{HualdePrieto2015} mention, and reject, the idea that narrow focus is responsible for the selection between L+H* L\% and L* L\%. Neither they nor any other publication I know of discusses the possibility that givenness or accessibility is the relevant criterion (see \citet{Baumann2006} for such an explanation for German).} L* HL\% is only mentioned in \autoref{tab:intonationalcategoriesPRIETO}, and associated there with either contrastive focus or contradiction. We have seen in \sectref{ch:2.3.4} and \sectref{ch:2.3.5} that this contradiction contour is frequently interpreted as obviousness, yet is supposed to be different from the obviousness contour L+H* L!H\% in \autoref{tab:intonationalcategoriesPRIETO}. To date, we do not know what factors condition the selection between L* HL\% and L+H* L!H\%. In fact, to my knowledge, the L+H* L!H\% contour has not yet been observed in spontaneous dialogue data at all.

For exclamatives, the picture in the literature is somewhat clearer. ``In words in intonational phrase-final position, exclamatory force (including correction focus) is conveyed by expansion of tonal range [...] and durational increase.'' \citep[368]{HualdePrieto2015} While durational increase is not transcribed in Sp\_ToBI, expansion of tonal range is indicated via an inverted exclamation mark. The trade-off between range expansion and durational increase remains unclear.\footnote{Moreover, it remains an open question if durational increase of lexically accented syllables in exclamations with L+H* L\% contours is interpreted differently from the lengthening of lexically accented syllables in L* L\% assertions, which are supposed to convey verum focus \citep{EscandellVidal.2011}. I leave this problem to future research.} While I could not find any intonationally explicit examples with \textit{anda} and \textit{vaya} in the literature, exclamative L+¡H* L\% intonation as indicated in \autoref{tab:intonationalcategoriesPRIETO} is what we would expect to find on the prosodically marked uses.

In \sectref{ch:5.1}, we have seen that \textit{claro}, \textit{hombre}, \textit{anda}, and \textit{vaya} occur in provocations and responses, with a clear tendency toward responses. Within responses, they are also all specified for positive relative polarity. Yet there is a difference between \textit{claro} and \textit{vaya} on the one hand, which occur only in agreeing responses, and \textit{hombre} and \textit{anda}, which occur in a small number of reversals as well. The fact that \textit{claro} is not used for reversals in our sample is helpful in ruling out the possibility that the difference in meaning between L+H* L\% and L* HL\% is purely a question of relative polarity. Some analyses in the literature actually suggest just that.

\begin{displayquote}\relax
[In varieties] where both nuclear contours [L+H* L\% and L* HL\%] are found, L* HL\% carries a greater emphatic, contradictory force. \citep[369]{HualdePrieto2015}
\end{displayquote}

Yet other publications, notably \citet[279]{EstebasVilaplanaPrieto.2008}, \citet{ElviraGarcia.2016}, and \citet{TorreiraGrice.2018}, indicate that L* HL\% is rather related to obviousness than contradiction. And in fact there are many examples of L* HL\% on \textit{claro}. As already reported in \citet{TorreiraGrice.2018}, some realizations of \textit{claro} with a low tone on the lexically accented syllable and a rise on the posttonic do not end in a low tone, but either end on a high tone or only in a small dip at the end of the rise. According to \citet[16]{TorreiraGrice.2018}, the two realizations L*~HL\% and L*~H(L)\% ``strike the attentive native listener as functionally
equivalent at the intonational level.'' To get an idea of different possible realizations of a low-rise with more or less pronounced falls at the end, we can have a closer look at some contextualized examples. 

\subsection{Turns with \textit{claro}}
\label{ch:5.2.1}

\subsubsection{L* HL\%}

\autoref{fig:claroPRESEEAmontana} from the context (\ref{ex:claroPRESEEAmontana}) is an example of \textit{claro} with an L* HL\% realization. Here, A has listed a series of seven places in the mountainous area of Asturias that he likes to visit with his family, to which B replies with the assertion that they like the mountains. A reacts with a hesitant \textit{bueno}, followed by three relative polarity particles \textit{claro sí sí} indicating not only the acceptance of the provocation as true, but also the relative expectability of this agreement. While there seems to be no ``contradictory force'' \citep[369]{HualdePrieto2015}, presupposing the expectability of a commitment conversationally implicates that the person that requested this commitment violated the first part of Grice's conversational sub-maxim of Quantity ``Make your contribution as informative as required (for the current purposes of the exchange).'' \citep[45]{Grice.1975}\footnote{This violation then forces the responding interlocutor to violate the second part of Grice's conversational sub-maxim of Quantity: ``Do not make your contribution more informative than is required''.} While such a conversational implicature is not a contradiction, it is similarly prone to be face-threatening and therefore easily confused with it. 

In (\ref{ex:claroPRESEEAmontana}), the provocation by B is responsible for the violation of this maxim. In terms of the model in \sectref{ch:3.3.3}, A does not ``contradict'' B in (\ref{ex:claroPRESEEAmontana}). Rather, we are dealing with a modally marked assertion confirmation, in which the confirmation is prosodically marked as necessary from the perspective of the input Common Ground. Since the model assumes that the goal of conversation is to increase the Common Ground, such a move is impolite or uncooperative in the sense that it indicates the lack of such an update.

\vfill
\begin{figure}[H]
	\includegraphics[width=\linewidth]{gfx/MADR_H33_049_A_claro_si_gusta_la_montana_SHORT.jpg}
	\caption[L* HL\% on \textit{claro} in context (\ref{ex:claroPRESEEAmontana})]{L* HL\% on \textit{claro} in context (\ref{ex:claroPRESEEAmontana}) \href{https://osf.io/uv86f/}{\faVolumeUp} \label{fig:claroPRESEEAmontana}}
\end{figure}
\vfill\pagebreak

\begin{exe}
	\ex \label{ex:claroPRESEEAmontana} (Interview 49, \cite{PRESEEA.20142020}) 
	\begin{xlist}[A:]
	\exi{A:} \ldots de ahí tiramos para Llanes tiramos para Arriendas / tiramos para Cangas de Onís para Covadonga / o tiramos para Tazones o 
	\glt `\ldots from there we go to Llanes we go to Arriendas / we go to Cangas de Onís to Covadonga / or we go to Tazones or'
	
	\exi{B:} os gusta la montaña 
	\glt `you like the mountains' 
	\exi{A:} bueno \textit{claro} sí sí \\
	\hspace*{3em}L* HL\% 
	\glt `well \textit{sure} yes yes' 
	\end{xlist}
\end{exe}

Another example of a L* HL\% contour is \autoref{fig:claroPRESEEAnochebuena} from the context (\ref{ex:claroPRESEEAnochebuena}), repeated for convenience in (\ref{ex:claroPRESEEAnochebuena2}). As seen in \sectref{ch:5.1}, \textit{claro} is the common way of responding to the biased question about whether or not Christmas festivities involve having something special for dinner. In (\ref{ex:claroPRESEEAnochebuena2}), the two turn-initial \textit{pues} signal hesitation and the intention to perform a context update,\footnote{See (\ref{ex:puesBRIZ}) and the respective discussion.} which then resolves into a series of four relative polarity particles \textit{sí sí sí claro} followed by an additional explicit explanation of the hesitant reaction with the adverb \textit{por supuesto} `obviously'. Again, we see how the idea of a ``contradictory force'' can arise in examples in which an obviousness contour can be understood as challenging the validity of formulating the provocation as a question, given the expectability of the answer. In terms of our pragmatic model, B does not contradict A in (\ref{ex:claroPRESEEAnochebuena2}). Rather, we are dealing with a modally marked polar question confirmation, in which the confirmation is prosodically marked as necessary from the perspective of the input Common Ground.\footnote{I will not attempt a full integration of intonationally marked obviousness into a theory of compliance with Gricean Maxims. Intonational Compliance Marking theory \citep{Westera.2017,Westera.2018} seems readily extendable in this direction.}

\begin{exe}
	\ex \label{ex:claroPRESEEAnochebuena2} (Interview 49, \cite{PRESEEA.20142020}) 
	\begin{xlist}[A:]
	 \exi{A:} eeh/¿que lo celebráis con un / hay algún menú especial en Nochebuena? 
	\glt `Um / that you celebrate with a / is there a special Christmas Eve menu?'
	
	\exi{B:} pues pues / sí sí sí \textit{claro} eso por supuesto y además \ldots \\
	\hspace*{8em}L* HL$-$ 
	\glt`Well well/ yes, yes, yes, \textit{sure}, obviously this and moreover \ldots' 
	\end{xlist}
\end{exe}

\begin{figure}
	\includegraphics[width=\linewidth]{gfx/MADR_H33_049_A_claro_si_en_Nochebuena_SHORT.jpg}
	\caption[L* HL$-$ on \textit{claro} in context (\ref{ex:claroPRESEEAnochebuena2})]{L* HL$-$ on \textit{claro} in context (\ref{ex:claroPRESEEAnochebuena2}) \href{https://osf.io/8f4sn/}{\faVolumeUp} \label{fig:claroPRESEEAnochebuena}}
\end{figure}

\autoref{fig:claroPRESEEAozono} from the context in (\ref{ex:claroPRESEEAozono}) shows that in successions of \textit{hombre} and \textit{claro}, the low-rise-fall need not occur on both.\footnote{Whether there is a (rising) pitch accent on \textit{hombre} will not be decided here.} Here, A asks B whether she believes in the environmental problems that are all over the media. After a short vocalized hesitation, B agrees using \textit{hombre claro}, followed by \textit{sí} and a confirmation-seeking tag question. The pattern of hesitation and obvious agreement is visible in all three examples of \textit{claro L* HL\%} discussed so far, which can count as a further sign that speakers hesitate to use a marked, possibly impolite form.\footnote{See \citet{KendrickTorreira.2015} for the findings that marked responses (what they call ``Dispreferred Formats'' or ``qualified'' responses) are preceded by longer breaks between turns and that very late responding actions (after breaks longer than 700 ms) are almost always dispreferred moves (e.g. negative relative polarity or face-threatening acts).}\largerpage

\begin{exe}
	\ex \label{ex:claroPRESEEAozono}(Interview 05, \cite{PRESEEA.20142020})
	\begin{xlist}[A:]
	\exi{A:} ¿y tú crees que es verdad eso del fenómeno del niño y de la niña / de la capa de ozono y todo eso?
	\glt `And do you think it's true this whole phenomenon of El niño and La niña / and of the ozone layer and such?'
	\exi{B:} mmm // \textit{¡hombre claro!} sí ¿no? \ldots \\
	\hspace*{7.5em}L* HL\%
	\glt `mmm \textit{man sure}! it is, right? \ldots'
	\end{xlist}
\end{exe}

\begin{figure}
	\includegraphics[width=\textwidth]{gfx/MADR_M11_005_A_claro_si_capa_de_ozono_SHORT.jpg}
	\caption{\textit{hombre claro} L* HL\% in context (\ref{ex:claroPRESEEAozono}) \href{https://osf.io/56v78/}{\faVolumeUp} \label{fig:claroPRESEEAozono}}
\end{figure}

According to \citet{TorreiraGrice.2018}, native speakers of Spanish should truncate the final fall in one-word examples of \textit{claro L* HL\%}. The examples presented so far do not show such tonal truncation.\footnote{Perceptually, the final falls are nevertheless very subtle, as can be appreciated by listening to the audio-files.} \autoref{fig:claroPRESEEAlipotimias} from the context in (\ref{ex:claroPRESEEAlipotimias}) shows a truncated L*~H(L)$-$ realization of \textit{claro}. Here, A assures B that fainting is not as exceptional as her husband might think, to which B agrees with a succession of markers showing that her previous assertion did not call into question the mutually shared assumption that fainting can sometimes happen.\largerpage

\begin{exe} 	
	\ex \label{ex:claroPRESEEAlipotimias} (Interview 41, \cite{PRESEEA.20142020})
	\begin{xlist}[A:]
	\exi{A:} tu marido ¿qué decía?
	\glt `Your husband, what did he say?'
	\exi{B:} lloraba amargamente \ldots
	\glt `he was weeping bitterly \ldots'
	\exi{A:} mm / pero es es son normales / las lipotimias esas ¿sabes? 
	\glt `mm / but it's it's they're normal / these faintings, you know?'
	\exi{B:} \textit{claro} / hombre que sí claro \\
	\hspace*{.5em}L*~H(L)$-$
	\glt `\textit{sure} man, yes, sure'
	\end{xlist}
\end{exe}

\begin{figure}
	\centering
	\includegraphics[width=\linewidth]{gfx/MADR_M31_041_A_claro_si_lipotimias_SHORT.jpg}
	\caption{L*~H(L)$-$ on \textit{claro} in context (\ref{ex:claroPRESEEAlipotimias}) \href{https://osf.io/5qgxy/}{\faVolumeUp} \label{fig:claroPRESEEAlipotimias}}
\end{figure}

\autoref{fig:claroPRESEEAmuyduro} from the context in (\ref{ex:claroPRESEEAmuyduro}) is a rare example of an obvious assertion confirmation with an inflected verb. A asserts that the thought of death without afterlife is hard, and B agrees with a succession of particles and the sentence \textit{duro es} `it's hard'. The first particle, \textit{sí}, is dramatically lengthened to accommodate a low-rise-fall contour, which is repeated on \textit{claro} and \textit{hombre}. Since the sentence ends in a one-syllable word, we again do not know if the nuclear contour is a final low-rise L* H\% or a truncated low-rise-fall L*~H(L)\%.

The pragmatic equivalence between truncated and non-truncated examples of the low-rise-fall indicates that the phonetic difference does not encode a meaningful distinction. This leaves phonological processes as an explanation for truncation, yet a word-level analysis as attempted in \citet{TorreiraGrice.2018} cannot account for the variability in one-word examples. Rather, the examples presented here point to the possibility that weaker prosodic boundaries between a particle and the following prosodic constituent, as well as the positioning of the lexical accent on the ultimate (tonal crowding), favor a reduced fall realization.

\begin{exe}
	\ex \label{ex:claroPRESEEAmuyduro} (Interview 41, \cite{PRESEEA.20142020})
	\begin{xlist}[A:]
	\exi{A:} es algo en lo que queremos pensar porque sólo pensar en / te mueres y / y te entierran y se acabó todo / ya no eres tú ya no hay nadie eso es muy duro ¿no? 
	\glt `It's something we like to think because to only think that / you die and / and they bury you and it's all over / you're not you anymore there's nobody anymore that's very hard, right?'
	\exi{B:} pues sí \textit{claro} hombre duro es \ldots \\
	\hspace*{3.5em}L*~H(L)$-$
	\glt `well yes \textit{sure} man it's hard \ldots'
	\end{xlist}
\end{exe}

\begin{figure}
	\includegraphics[width=\textwidth]{gfx/MADR_M31_041_A_claro_si_muy_duro.jpg}
	\caption{L*~H(L)$-$ on \textit{claro} in context (\ref{ex:claroPRESEEAmuyduro}) \href{https://osf.io/5paqg/}{\faVolumeUp} \label{fig:claroPRESEEAmuyduro}}
\end{figure}

\autoref{fig:claroPRESEEAingresos} from the context in (\ref{ex:claroPRESEEAingresos}) illustrates that L* HL\% intonation can be limited to one particle within a sequence of particles. Here, A has asked repeatedly if B can give a broad estimate of her household income, to which B has replied that she has only recently started working again. A replies with the question `your husband didn't have income this year either?' introduced by the adversative conjunction \textit{pero} `but' and marked with a high plateau intonation that I interpret as incredulity, implicating that her husband must have had income for the household to survive. B agrees with this implicature, adding a non-at-issue commitment of obviousness to her agreement to indicate the expectability of her husband having income. When comparing this sequence with \autoref{fig:claroPRESEEAmuyduro}, we see that speakers can choose freely whether to mark obviousness on one or several intermediate phrases in a turn.

\begin{exe}
\ex \label{ex:claroPRESEEAingresos} (Interview 11, \cite{PRESEEA.20142020})
	\begin{xlist}[A:]
	\exi{A:} pero / ¿tu marido no ha tenido ingresos este año tampoco?
	\glt `but / your husband didn't have income this year either?'
	\exi{B:} \textit{¡sí / claro! / mi marido sí}\\
	\hspace*{2.5em}L* HL$-$
	\glt `\textit{yes / sure ! mi husband yes}'
	 \end{xlist}
\end{exe}

\begin{figure}
	\includegraphics[width=\textwidth]{gfx/MADR_M12_011_A_claro_si_ingresos_STEREO.jpg}
	\caption{L* HL$-$ on \textit{claro} in context (\ref{ex:claroPRESEEAingresos}) \href{https://osf.io/4rknm/}{\faVolumeUp} \label{fig:claroPRESEEAingresos}}
\end{figure}

In sum, we see that \textit{claro} does often receive L* HL\% prosodic marking. Given that \textit{claro} is the only Madrid Spanish discourse particle that is obligatorily specified for positive relative polarity,\footnote{There are no particle uses of \textit{cierto} in the PRESEEA Madrid Salamanca corpus. Neither are there uses of \textit{eso/esto} as an agreement particle, a phenomenon restricted to some American varieties (e.g. Antioquia Colombia).} this rules out the possibility that the function of L* HL\% is to deny or reverse a proffered proposition. This becomes even more apparent in cases where \textit{claro} is not used for confirmation, but rather to introduce an expectable assertion as part of an explanation. \autoref{fig:claroPRESEEAnieve} from the context in (\ref{ex:claroPRESEEAnieve}) is an example of a truncated low-rise-fall on \textit{claro} used within a longer turn.\footnote{Again, prenuclear pitch accents are omitted here, partly because the signal is less clear on \textit{coger} than on \textit{agachabas}. This seems not only due to the segmental makeup, but hints at the relative prominence of the nuclear accent.} Here, this contour is mirrored at the end of the assertion, indicating that \textit{claro} is in a discourse-cataphoric relationship with the following sentence.

Such prosodic congruence is different from prosodic question-answer congruence, as investigated for example by \citet{RoettgerMahrtCole.2019}, since the first \textit{claro} L*~H(L)$-$ constitutes an anticipation of a non-at-issue commitment by the same speaker. The intonational contour is not licensed by narrow focus on either \textit{claro} or \textit{nieve}, but rather by the assumption of shared expectations about how children behave in the snow.\largerpage

\begin{exe}
	\ex \label{ex:claroPRESEEAnieve} (Interview 41, \cite{PRESEEA.20142020})
	\begin{xlist}[A:]
	\exi{A:} \ldots ¿qué te parece a ti el tiempo que estamos teniendo últimamente?
	\glt `\ldots What do you think of the weather we're having lately?'

	\exi{B:} pues hombre / \ldots~ ha evolucionado mucho el tiempo en Madrid / yo me acuerdo de pequeña que me encantaba ya por estas fechas / \ldots / nevaba / yo recuerdo unas nevadas \ldots / pero esas nevadas de Madrid que yo me acuerdo de / venir del colegio con la capa / chorreando / y mi madre / darme / de cachetes porque \textit{claro}~L*~H(L)$-$ \textit{te agachabas a coger nieve} L* HL\% / \ldots\pagebreak
	\glt `well man / \ldots~ it has changed a lot the weather in Madrid / I remember as a kid I loved it on these dates / \ldots / it snowed / I remember some snowstorms  \ldots / but these snowstorms of Madrid I remember / coming home from school with the coat / dripping / and my mother / spanking me because \textit{sure you ducked down to catch some snow} / \ldots'
	\exi{A:} claro 
	\glt `sure'
	\end{xlist}
\end{exe}

\begin{figure}
	\includegraphics[width=\textwidth]{gfx/MADR_M31_041_A_0458_0500_claro_nieve_NEW.jpg}
	\caption{L*~H(L)$-$ on \textit{claro} and L* HL\% on sentence in context (\ref{ex:claroPRESEEAnieve}) \href{https://osf.io/9r84u/}{\faVolumeUp} \label{fig:claroPRESEEAnieve}}
\end{figure}


\subsubsection{L+H* L!H\%}

Much rarer than the L* HL\% contour on \textit{claro} is the L+H* L!H\% contour. \autoref{fig:claroPRESEEAloteria} from the context in (\ref{ex:claroPRESEEAloteria}) shows the contour identified by \autoref{tab:intonationalcategoriesPRIETO} as the expression for obviousness. As in the other interviews, the speakers do not know each other very well before the interview, which is visible in that A addresses B in the formal third person singular. B has just stated that she doesn't have the money to travel, to which A reacts by stating that she wonders whether they will win the lottery this year. A thereby presupposes that B plays the lottery as well. B agrees with this statement by repeating it. Nevertheless, A now explicitly asks whether the presupposition of the previous two assertions is actually true, to which B reacts with \textit{claro} L+H* L!H\%.

Determining the difference between (\ref{ex:claroPRESEEAloteria}) and the aforementioned examples with L* HL\% from context alone can only be a first approximation, and needs to be supplemented by \textit{Laboratory Phonology} research in the sense of \citet{CohnFougeronHuffman.2012intro} (\chapref{ch:6}). Nevertheless, some contextual cues are present. In (\ref{ex:claroPRESEEAloteria}), A calls into question the presupposition of an assertion that has just been confirmed. This goes beyond asserting (or inquiring about) a proposition that is necessary from the perspective of the input Common Ground, because the proposition is part of the \ac{CG}. If B were to respond negatively to the question, this would constitute a highly marked retraction from a Discourse Commitment. \textit{Claro} L+H* L!H\% can therefore be seen as a complex case of obvious insistence, in which the speaker expresses a) a polar question confirmation, b) the necessity of this confirmation from the input \ac{CG}, and c) the insistence on a commitment.

\begin{exe}
\ex \label{ex:claroPRESEEAloteria} (Interview 42, \cite{PRESEEA.20142020})
    \begin{xlist}[A:]
    \exi{A:} bueno a ver si nos toca la lotería / este año 
	\glt `well let's see if we win the lottery / this year'
	\exi{B:} bueno a ver si nos toca 
	\glt `well let's see if we win'
	\exi{A:} y vamos ¿juega a la lotería? 
	\glt `and so do you play the lottery?'
	\exi{B:} ¡sí \textit{claro}! / hay que jugar \\
	\hspace*{.5em}L+H* L!H\% 
	\glt `yes \textit{sure}! / you have to play'
	\end{xlist}
\end{exe}\largerpage

\begin{figure}
	\includegraphics[width=\linewidth]{gfx/MADR_M31_042_A_claro_si_la_lotería_SHORT.jpg}
	\caption{L+H* L!H\% on \textit{claro} in context (\ref{ex:claroPRESEEAloteria}) \href{https://osf.io/fty73/}{\faVolumeUp} \label{fig:claroPRESEEAloteria}}
\end{figure}

While L+H* L!H\% marking on \textit{claro} is very rare, \textit{sí} before sentence-adverbial use of \textit{claro que} can be lengthened so as to accommodate a low-rise-fall-rise contour, as in \autoref{fig:claroPRESEEAazar} from the context in (\ref{ex:claroPRESEEAazar}). Here, B has stated that she doesn't play the lottery, to which A reacts by asking if B doesn't believe in chance or luck. B commits to believing in luck, only to start digressing into a lengthy explanation about her belief in destiny being predetermined. A reacts by repeating her polar question whether or not B believes in chance, to which B reacts by re-asserting her previous commitment with \textit{sí} L+H*L¡H\%.\footnote{Note that this kind of annotation is only justified by comparison with examples that allow us to separate pitch accent from boundary tone via a syllable boundary. If this was the only example we had, we could as well label it LHL¡H\% or L+H*+L ¡H\%. Penultimate stress is the default in Spanish. If we assume that the inventory of contours remains the same in phrases irrespective of the stress position of the words they contain, then phrases with penultimately stressed words in nuclear position should be the main point of comparison in intonational phonology.} This examples parallels example (\ref{ex:claroPRESEEAloteria}) in many ways. Again, A calls into question a recently established commitment, prompting B to a) confirm the polar question, b) presuppose/conventionally implicate the necessity of this confirmation from the input \ac{CG}, and c) insist on her previous commitment.\largerpage{} Note that the scaling of the final rise is much higher here, a problem already mentioned with regard to \autoref{fig:exclamation_HUALDE} from \citet[278]{Hualde.2014} and \autoref{fig:exclamation_HUALDE_PRIETO_surprise_obvious} from \citet[379]{HualdePrieto2015}. The naturally occurring examples presented here indicate an even greater variability in scaling on the final high target, with upstepped final rises greatly exceeding the pitch accents in range. Nevertheless, the context-update in terms of expectability and insistence remains the same. One way of dealing with this evidence would be to treat such scaling differences as non-categorical. Yet we would still need to explain why the scaling differences concern only the final rise, and not also the previous pitch accent. An analysis purely in terms of emphasis would predict all tonal targets equally to be scaled higher or lower with higher or lower emphasis. We come back to this issue when dealing with mirative intonation, where scaling differences affect pitch accents instead of boundary tones.{\interfootnotelinepenalty=10000\footnote{I abstain from analyzing the final boundary tone in \autoref{fig:claroPRESEEAazar} as a reduced L*~H(L)\% since there is not even an indication of a low target in this lengthened syllable and the rise continues up to the end of the word. Moreover, I abstain again from marking prenuclear pitch accents, not only because of the relatively small excursion after the first \textit{claro}, but particularly because annotating \textit{creo} with H*+L would neglect the continuous fall towards \textit{azar} L*~H\%.}}

\begin{exe}
\ex \label{ex:claroPRESEEAazar} (Interview 41, \cite{PRESEEA.20142020})
	 \begin{xlist}[A:]
    \exi{A:} ¿no crees en el azar y en?
	\glt `you don't believe in chance and in?'
	\exi{B:} no no sí sí creo hombre / pero lo que te digo que es que tenemos un destino escrito y hagas lo que hagas por mucho que hagas por traerlo / como no lo tengas no lo tienes
	\glt `no no yes yes I believe man thing is / but what I'm telling you is that we have a written destiny and for whatever you do no matter how much you do for bringing it upon you / if you don't have it you don't have it \ldots'
	\exi {} \ldots
	\exi{A:} bueno y entonces ¿no crees en el azar o sí crees en el azar? 
	\glt `well and so do you not believe in chance or do you believe in chance?'
	\exi{B:} \textit{sí \hspace*{1.5em}/ claro que creo en el azar} \\
	L+H* LH\% \hspace{7em} L* H\%
	\glt `\textit{yes of course I believe in chance}'
	\end{xlist}
\end{exe}\largerpage

\begin{figure}
	\includegraphics[width=\textwidth]{gfx/MADR_M31_041_A_7302_7304_claro_azar_NEW.jpg}
	\caption[L+H* LH\% on \textit{sí} in context (\ref{ex:claroPRESEEAazar})]{L+H* LH\% on \textit{sí} in context (\ref{ex:claroPRESEEAazar}) \href{https://osf.io/y38tf/}{\faVolumeUp} \label{fig:claroPRESEEAazar}}
\end{figure}

Turning to the question of whether L* HL\% and L+H* LH\% can be combined, natural examples show no restrictions. \autoref{fig:claroPRESEEAirya} from the context in (\ref{ex:claroPRESEEAirya}) starts with \textit{claro} L*~H(L)-, but ends with a highly lengthened one syllable \textit{ya} L+H* LH\%. Here, B asserts that she will not leave her parents' place once she starts working, to which A reacts with an incredulous acceptance that calls into question the veracity of this prediction. B goes on to reassert her commitment with \textit{sí} and to set a lower limit on the time scale for which this commitment holds. Finally, she explicitly states the reason for why her commitment is to be expected, formulating it using the depersonalized passive voice which frames it as a universal truth that one lives well at one's parents' place. Here, again, B expresses a) a polar question confirmation, b) the necessity of this confirmation from the input \ac{CG}, and c) the insistence on a commitment that has been called into question.

\begin{exe}
	\ex \label{ex:claroPRESEEAirya} (Interview 05, \cite{PRESEEA.20142020})
	\begin{xlist}[A:]
	\exi{A:} ¿y cómo te imaginas tú entonces que va a ser o qué sería tu vida // si / \ldots hubieras empezado ya a estudiar lo que quieres estudiar / \ldots?
	\glt `and how do you picture your life to be // if / if you \ldots had already started to study what you want to study / \ldots?'
	\exi{B:} pues yo creo que muy parecida / porque / me pondría a trabajar de eso // y lo demás / seguiría viviendo en casa de mis padres
	\glt `well I think very similar / because / I'd work and such // and apart from that / I'd continue living at my parents' place' \ldots
	\exi{A:} ¿sí? 
	\glt `really?'
	\exi{B:} sí // \textit{y ¡claro no me voy a ir el año que viene ya!} / vamos / ¡con lo bien que se vive en casa de los padres! 
	\glt `yes // \textit{and I'm obviously not going to leave next year already!} / come on / given how good one lives at one's parents' place!'
	\end{xlist}
\end{exe}

\begin{figure}
	\includegraphics[width=\textwidth]{gfx/MADR_M11_005_A_2014_2018_claro_ir_ya_NEW.jpg}
	\caption{L+H* LH\% in context (\ref{ex:claroPRESEEAirya}) \href{https://osf.io/tmzs7/}{\faVolumeUp} \label{fig:claroPRESEEAirya}}
\end{figure}
 
\subsubsection{L* L\%}

It is important to stress that use of marked intonational contours is far from obligatory in any pragmatic context, even ones in which commitments are challenged and expectations are negotiated. The L* L\% contour is very frequent on \textit{claro}. \autoref{fig:claroPRESEEAcabeceraLONG} from the context in (\ref{ex:claroPRESEEAcabecera}) shows an example that is similar to (\ref{ex:claroPRESEEAirya}) in that A challenges the veracity of the previous assertion. B responds with \textit{claro} and then adds an explanation, acknowledging that A was lacking information that would have allowed him to rule out the alternative answer. Apart from the falling intonation on \textit{claro}, note the L+H* L\% nuclear configuration on the explanatory addition. B asserts that he has known his physician for years, without presupposing this to be expectable.

\begin{exe}
\ex (Interview 06, \cite{PRESEEA.20142020})\label{ex:claroPRESEEAcabecera} 
	\begin{xlist}[A:]
	\exi{A:} ¿tú hay alguien en especial que trates de usted y alguien en especial que trates de tú? / por ejemplo a tú médico ¿cómo lo tratas? / ¿de tú o de usted? 
	\glt `you is there someone in particular that you address with \textit{usted} and someone you address with \textit{tú}? / for example your doctor how do you address him / with \textit{tú} or \textit{usted}?'
	\exi {} \ldots
	\exi{B:} al médico de cabecera lo trato de tú 
	\glt `the general physician I address with \textit{tú}' 
	\exi{A:} ¿sí?
	\glt `really?'
	\exi{B:} \textit{claro} / \textit{después de varios años} \\
	\hspace*{.5em}L* L\% \hspace{6.5em}L+H* L\% 
	\glt `\textit{sure / after several years}' 
	\end{xlist}
\end{exe}

\begin{figure}
	\includegraphics[width=\textwidth]{gfx/MADR_M11_006_A_claro_si_cabecera_LONG.jpg}
	\caption{L* L\% on \textit{claro} and L+H* L\% on \textit{años} in context (\ref{ex:claroPRESEEAcabecera}) \href{https://osf.io/96fnk/}{\faVolumeUp} \label{fig:claroPRESEEAcabeceraLONG}}
\end{figure}


Based on this example, I want to briefly reflect on the difference between unmarked L* L\% and marked L* HL\%. (\ref{ex:claroPRESEEAcabeceraHYPO}a) gives the actual attested sentence, (\ref{ex:claroPRESEEAcabeceraHYPO}b,c) are hypothetical alternative responses. All are complex discourse updates. In (\ref{ex:claroPRESEEAcabeceraHYPO}a), \textit{claro} confirms the polar question and prosody does not add any commitments, since L* L\% is unmarked. The second part of the turn commits B to her having addressed the physician for several years now. This is a new commitment proffered for acceptance (assertion), which receives L+H* L\%. In (\ref{ex:claroPRESEEAcabeceraHYPO}b), the L+H* L\% on this assertion stays the same, while the L* HL\% on \textit{claro} adds the commitment that the truth of the confirmed proposition would be necessary from the input \ac{CG}. While the context would have licensed such a commitment (and the lexical meaning of \textit{claro} is highly compatible with it), it would have created a conflict with the following assertion that B has been addressing her physician for several years now. This conflict arises between the conventional implicature triggered by L* HL\% (It is expectable from the input \ac{CG} that I address my general physician with \textit{tú}) and the conversational Q-implicature that arises through a new assertion directly pertinent to the expectations exploited by L* HL\% (I am being informative when I tell you that I know him for several years). At the beginning of the turn, the \ac{CG} does not contain the information about the longstanding relationship between B and her physician. Marking \textit{claro} as obvious and then asserting the reason for the expectability of \textit{claro} would violate the Maxim of Quantity \citep{Grice.1975,Horn.2010}. Finally, (\ref{ex:claroPRESEEAcabeceraHYPO}c) would commit B to the expectability of both commitments. This would again be consistent, yet a bold claim given that A does not have information about the relationship between B and her physician.\footnote{Note that only the ``artificial'' interview setup allows us to be certain about this lack of shared information. When observing conversation without information about the epistemic relation between interlocutors, intonation is only predictable if you are a mind reader \citep{Bolinger.1972}.}

\begin{exe}
	\ex \label{ex:claroPRESEEAcabeceraHYPO}
	\begin{xlist}[A:]
	\exi{A:} ¿sí?
	\glt `really?
	\end{xlist}
	\begin{xlist}
	\ex[]{%
	\begin{xlist}[A:]
	\exi{B:} claro / después de varios años \\
	\hspace*{.5em}L* L\% \hspace{6.5em}L+H* L\% 
	\glt `sure / after several years'
	\end{xlist}
	}
		
	\ex[\#]{%
	\begin{xlist}[A:]
	\exi{B:} claro / después de varios años \\
	\hspace*{.5em}L* HL\% \hspace{6.5em}L+H* L\% 
	\glt `sure / after several years'
	\end{xlist}
	}
	
	\ex[] {%
	\begin{xlist}[A:]
	\exi{B:} claro / después de varios años \\
	\hspace*{.5em}L* HL\% \hspace{7em}L* HL\% 
	\glt `sure / after several years'
	\end{xlist}
	}
	\end{xlist}
\end{exe}

Coming back to corpus examples, we find that successions between two particles can stand in a downstep relation to one another, as visible in \autoref{fig:claroPRESEEAconoces} from the context in (\ref{ex:claroPRESEEAconoces}). Here, no common assumptions have been called into question. Instead, the provocation by B mainly sums up the previous assertion of A. By using \textit{claro} L* L\%, A accepts the proffered proposition without additional modal non-at-issue commitments. Phonologically, a downstep relation between two successive instances of \textit{claro} indicates that there is no intermediate boundary between the two that would license a pitch reset.

\begin{exe}
	\ex (Interview 20, \cite{PRESEEA.20142020})  \label{ex:claroPRESEEAconoces}
	\begin{xlist}[A:]
	\exi{A:} cortar arizónicas limpiar la parcela regarlo lo otro lo otro y / luego quitar las hiervas que es que salen muchas hiervas allí ¿no? // y eso da mucho trabajo 
	\glt `cut cypresses clean the plot water this that and / then remove weeds it's that there come out many weeds, right? // and that demands a lot of work'
	\exi{B:} o sea que eso lo conoces bien 
	\glt `so this is something you know well'

	\exi{A:} \textit{claro claro} \\
	\hspace*{.5em}L* \hspace{1em}L* L\%
	\glt `\textit{sure sure}'
	\end{xlist}
\end{exe}

\begin{figure}
	\includegraphics[width=\linewidth]{gfx/MADR_H21_020_A_claro_si_lo_conoces_bien_SHORT.jpg}
	\caption{Downstep from \textit{claro} L* to \textit{claro} L* L\% in context (\ref{ex:claroPRESEEAconoces}) \href{https://osf.io/38q4m/}{\faVolumeUp} \label{fig:claroPRESEEAconoces}}
\end{figure}

\subsubsection{L+H* L\%}

L+H* L\% on \textit{claro} is rare, yet should still be noted as a possibility. To get an idea of the type of context this contour appears in, see (\ref{ex:claroPRESEEAcompenetraros}) and \autoref{fig:claroPRESEEAcompenetraros}. Here, again, B sums up a lengthy argument by A, to which A reacts with a succession of four particles. One reason I see for A to use an L+H* L\% contour instead of an L* L\% contour is the fact that B introduces his provocation by \textit{pero}, thereby framing his assertion as a reversal. Irrespective of the question whether B's assertion actually reverses anything that A stated, the use of \textit{pero} now puts A in the position of having to either insist on or retract from an commitment.\footnote{Note that this insistence differs from (\ref{ex:claroPRESEEAloteria},\ref{ex:claroPRESEEAazar},\ref{ex:claroPRESEEAirya}) in that it is not combined with an additional commitment to expectability from the \ac{CG}.} L+H* L\% on \textit{claro} can therefore be seen as marking contrastive focus on the particle denoting confirmation, which is one way for A to acknowledge the contrast and still communicate agreement.

\begin{exe}
	\ex (Interview 03, \cite{PRESEEA.20142020})  \label{ex:claroPRESEEAcompenetraros}
	\begin{xlist}[A:]
	\exi{A:} por ejemplo puede ser / un amigo mío / éramos de del barrio / \ldots // como hermanos / vamos / en cambio / no es lo mismo otro que era del barrio que he estado jugando con él al / al fútbol / muchas veces / que me he ido por ahí de juerga / sí / \ldots pero / no es lo mismo que este otro \ldots / no es la misma confianza esa / claro 
	\glt `for example say / a friend of mine / we were from the same neighborhood / \ldots // like brothers / like / in turn / it's not the same another one that was from the neighborhood I was playing soccer with him / often / I went to party there / yes \ldots but / it's not the same like the other one \ldots / it's not the same trust / sure' 
	\exi{B:} sí pero habéis llegado a / a compenetraros
	\glt `yes but you got to empathize with each other'
	\exi{A:} sí / \textit{claro} claro claro
	\glt `yes \textit{sure} sure sure' 
	\end{xlist}
\end{exe}

\begin{figure}
	\includegraphics[width=\linewidth]{gfx/MADR_H11_003_A_claro_si_compenetraros_SHORT.jpg}
	\caption[L+H* L\% on \textit{claro} in context (\ref{ex:claroPRESEEAcompenetraros})]{L+H* L\% on \textit{claro} in context (\ref{ex:claroPRESEEAcompenetraros}) \href{https://osf.io/2q7ht/}{\faVolumeUp} \label{fig:claroPRESEEAcompenetraros}}
\end{figure}


\subsection{Turns with \textit{hombre}}
\label{ch:5.2.2}

Figures~\ref{fig:hombre1BRIZ} and \ref{fig:hombre2BRIZ} from \citet{Briz.2012}, already discussed in \sectref{ch:3.3.2}, show that \textit{hombre} can have a high-falling intonation and a low-rise-falling intonation. While the low-rise-fall realization is the same L* HL\% pattern also discussed for \textit{claro} above, the high-falling realization could either be a tonal (L)+H* L\% aphe\-re\-sis of the L+H* L\% contour due to the lack of onset in the first syllable of \textit{hombre}, or it could be an H* L\% contour not mentioned in Tables~\ref{tab:intonationalcategoriesPRIETO} and~\ref{tab:intonationalcategoriesGABRIEL}. I will first lay out some forms and contexts of this high-falling contour, and then proceed to argue in favor of an analysis in terms of allophony between L+H* L\% and tonal apheresis (L)+H* L\% on \textit{hombre}. The fact that turn-initial \textit{hombre} (L)+H* L\% precedes assertions with either L* L\% or L+H* L\% is in line with the general view in the literature that L* L\% is in free variation with L+H* L\% in unmarked statements.\footnote{As laid out in \sectref{ch:3.3.3}, unmarked means here that neither an at-issue reversal nor non-at-issue commitments are marked intonationally.} I end this section with some remarks on H$-$ phrasing in turns preceded by \textit{hombre} L* HL\%, illustrating a puzzle in need of experimental investigation.

\subsubsection{(L)+H* L\%}
\label{ch:5.2.2.1}

The high-falling contour is the typical contour on a turn-initial \textit{hombre}.  \autoref{fig:hombrePRESEEAhuevo} from context (\ref{ex:hombrePRESEEAhuevo}) shows a realization of \textit{hombre} that starts high and then falls approximately 40\,Hz towards the end of the one-word intermediate phrase. The context shows that we are dealing with the expectational realignment use of \textit{hombre}, already discussed in \sectref{ch:5.1}, in which the speaker reassures the interlocutor of shared expectations after having asserted something unexpected. A explains his recipe for carbonara. When B states that he has never tried it with eggs, A states the modal necessity of carbonara containing egg. When B puts this up for discussion again, he first reaffirms his statement, only to then proceed with an expectational realignment use of \textit{hombre} and the assertion that it is possible that some people might do it with cream. The intonational contour is what \citet[32]{Briz.2012} calls the ``pseudo-agreement'' use of \textit{hombre}. From the context in (\ref{ex:hombrePRESEEAhuevo}), we can see that agreement in the sense of \citet{FarkasBruce.2010} is achieved via \textit{sí}. In contrast, \textit{hombre} in (\ref{ex:hombrePRESEEAhuevo}) can be seen as a modal concessive: it accepts a possibility that would justify B's stance (\textit{perhaps they use cream}). A then continues with a second concessive \textit{pero} that would usually introduce a contrasting commitment \citep{CouperKuhlenThompson.2000} but remains without argument.\footnote{A does not continue after \textit{pero}, abstaining from re-asserting that pasta with cream would not be a carbonara. This could be interpreted as \textit{agreeing to disagree} in terms of \citet{FarkasBruce.2010}.}\largerpage[-1]\pagebreak

\begin{figure}[p]
	\includegraphics[width=\linewidth]{gfx/MADR_H23_033_A_hombre_si_con_huevo.jpg}
	\caption[Falling intonation on \textit{hombre} in context (\ref{ex:hombrePRESEEAhuevo})]{Falling intonation on \textit{hombre} in context (\ref{ex:hombrePRESEEAhuevo}) \href{https://osf.io/9e73y/}{\faVolumeUp} \label{fig:hombrePRESEEAhuevo}}
\end{figure}\clearpage

\begin{exe}
	\ex (Interview 33, \cite{PRESEEA.20142020})\label{ex:hombrePRESEEAhuevo}
	\begin{xlist}[A:]
	\exi{A:}  ¿no los has comido nunca?  
	\glt `you never tried them?'
	\exi{B:} yo he he probado pero no así no así con yo por ejemplo no le no lo he probado con huevo 
	\glt `I have have tried but not like this not like this with I for example I didn't didn't try it with egg'
	\exi{A:} ¡ah! ¿no?
	\glt `oh! really?'
	\exi{B:} no
	\glt `no'
	\exi{A:}  pues la carbonara tiene que ser con huevo 
	\glt  `well the carbonara must be with egg'
	\exi{B:} con huevo / ¿sí? 
	\glt `with egg / really?'
	\exi{A:} sí // \textit{hombre} quizá lo hagan solo con nata y tal pero
	\glt `yes // \textit{man} perhaps they do it just with cream and such but'
	\end{xlist}
\end{exe}

\autoref{fig:hombrePRESEEAciudad} from context (\ref{ex:hombrePRESEEAciudad}) is another example of falling intonation on \textit{hombre}. Responding to an unbiased alternative question is si\-mi\-lar to responding to a polar question, with the difference being that instead of a proposition and its negation the projected set contains two distinct, yet incompatible propositions. The use of \textit{hombre} with falling intonation in (\ref{ex:hombrePRESEEAciudad}) introduces an unmarked assertion, which itself ends with an L* L\% contour.

\begin{exe}
	\ex (Interview 33, \cite{PRESEEA.20142020})\label{ex:hombrePRESEEAciudad}
	\begin{xlist}[A:]
	\exi{A:} ¿y de irte / que os iríais al campo o a una ciudad pequeña? 
	\glt `and leaving / you'd leave to the countryside or a small town?'
	\exi{B:} \textit{hombre yo intentaría una ciudad pequeña}
	\glt `\textit{man I'd try a small town}' 
	\end{xlist}
\end{exe}

\begin{figure}
	\includegraphics[width=\textwidth]{gfx/MADR_H23_033_A_2912_2916_hombre_ciudad.jpg}
	\caption{Falling intonation on \textit{hombre} in context (\ref{ex:hombrePRESEEAciudad}) \href{https://osf.io/n7g4c/}{\faVolumeUp} \label{fig:hombrePRESEEAciudad}}
\end{figure}

Comparing these examples with combinations of \textit{sí} and \textit{hombre} adds further evidence to an unmarked assertive function of falling intonation on discourse particles. \autoref{fig:hombrePRESEEAinstituto} from context (\ref{ex:hombrePRESEEAinstituto}) shows a steeply falling intonation on \textit{sí}, with \textit{hombre} receiving a flat low intonation.\pagebreak

\begin{exe}
	\ex (Interview 25, \cite{PRESEEA.20142020})\label{ex:hombrePRESEEAinstituto}
	\begin{xlist}[A:]
	\exi{A:}  el Beatriz Galindo tiene buena fama / de siempre 
	\glt `The Beatriz Galindo has a good reputation / since forever' 
	\exi{B:} \textit{sí sí hombre} / es un instituto que está bien 
	\glt `\textit{yes yes man} / it's a good institute'
	\end{xlist}
\end{exe}

\begin{figure}
	\includegraphics[width=\linewidth]{gfx/MADR_H22_025_A_hombre_si_instituto_LONG.jpg}
	\caption[Falling intonation on \textit{sí} and L* L\% on \textit{hombre} in context (\ref{ex:hombrePRESEEAinstituto})]{Falling intonation on \textit{sí} and L* L\% on \textit{hombre} in context (\ref{ex:hombrePRESEEAinstituto}) \href{https://osf.io/qbnc6/}{\faVolumeUp} \label{fig:hombrePRESEEAinstituto}}
\end{figure}

\largerpage The high tonal target remains initial when phrasing \textit{hombre} and \textit{pues} together, as in \autoref{fig:hombrePRESEEAheterogeneo} from context (\ref{ex:hombrePRESEEAheterogeneo}).\footnote{\autoref{fig:hombrePRESEEAheterogeneo} also shows a rise from \textit{pues} to the second syllable of \textit{heterogeneo}, a phenomenon that will be discussed in \sectref{ch:5.2.2.3} as well as \sectref{ch:6.3.4}. Note particularly \autoref{fig:obviousFACE}.} To get at the phonology behind the falling intonation on \textit{hombre}, we can make use of the fact that both \textit{hombre pues} and \textit{pues hombre} are possible successions of particles. \textit{Pues}, when used as a discourse particle and not as a causal conjunction, does not receive lexical stress \citep[47]{AlarcosLlorach.1994}. It therefore does not associate with tonal targets to form its own pitch accent, but can allow a complex bitonal L+H* contour associated with the following syllable to surface fully. \autoref{fig:hombrePRESEEAVizcaya} from context (\ref{ex:hombrePRESEEAVizcaya}) shows that in \textit{pues hombre}, the high tonal target still remains on the first syllable of \textit{hombre}, yet with a rise through the previous syllable [pwes]. I take this as evidence for an analysis in terms of an L+H* pitch accent, followed by either a low phrase accent or a low boundary tone. In other words, I treat cases of falling intonation on \textit{hombre} as a tonal aphe\-re\-sis (L)+H* L\%, allophonic to cases that show a rising-falling L+H* L\% intonation, such as \autoref{fig:hombrePRESEEAmadre} from context (\ref{ex:hombrePRESEEAmadre}).\largerpage

\begin{figure}
	\includegraphics[width=\textwidth]{gfx/MADR_H23_033_A_1136_1138_hombre_heterogeneo.jpg}
	\caption[Falling intonation on \textit{hombre pues} in context (\ref{ex:hombrePRESEEAheterogeneo})]{Falling intonation on \textit{hombre pues} in context (\ref{ex:hombrePRESEEAheterogeneo}) \href{https://osf.io/brn9z/}{\faVolumeUp} \label{fig:hombrePRESEEAheterogeneo}}
\end{figure}

\begin{exe}
	\ex (Interview 33, \cite{PRESEEA.20142020})\label{ex:hombrePRESEEAheterogeneo}
	\begin{xlist}[A:]
		\exi{A:}  hh muy bien / oye háblame un poco de / de tu barrio / ¿cómo es? / 
		\glt `hh alright / listen tell me a little bit about / about your neighborhood / what is it like?'
		\exi{B:} hh ts ¿mi barrio? // ts \textit{hombre pues mi barrio es muy heterogéneo}  
		\glt `hh ts my neighborhood? // ts \textit{man well my neighborhood is very heterogeneous}'
	\end{xlist}

	\ex (Interview 47, \cite{PRESEEA.20142020})\label{ex:hombrePRESEEAVizcaya}
	\begin{xlist}[A:]\sloppy
	\exi{A:}  ¿qué lugar elegirías para vivir fuera de / que no fuera en Madrid? 
	\glt `What place would you choose to live out of / that wouldn't be Madrid?'
	\exi{B:} ¡\textit{pues hombre} mira Vizcaya me gusta mucho!
	\glt `\textit{Well man} look I like Biscay a lot!'
	\end{xlist}
\end{exe}

\begin{figure}
	\includegraphics[width=\linewidth]{gfx/MADR_M32_047_A_hombre_pues_Vizcaya_SHORT.jpg}
	\caption[L+H* L$-$ on \textit{hombre} in context (\ref{ex:hombrePRESEEAVizcaya})]{L+H* L$-$ on \textit{hombre} in context (\ref{ex:hombrePRESEEAVizcaya}) \href{https://osf.io/d368j/}{\faVolumeUp} \label{fig:hombrePRESEEAVizcaya}}
\end{figure}

In \autoref{fig:hombrePRESEEAmadre} from context (\ref{ex:hombrePRESEEAmadre}), we see a fully fledged L+H* rise on the first syllable of \textit{hombre}. Given that the entire rise occurs within one syllable, tonal apheresis on (L)+H* L\% seems to be optional.\largerpage

\begin{exe}
	\ex (Interview 41, \cite{PRESEEA.20142020}) \label{ex:hombrePRESEEAmadre}
	\begin{xlist}[A:]
	\exi{A:}  lo pongo nada unos minutillos en la olla porque ahora con las ollas rápidas se hace la verdad mucho más rápido / se nos ha simplificado mucho la vida 
	\pagebreak
	\glt `I put it in the pot for just a few minutes because now with the pressure cookers it gets done much faster in truth / it has made our lives much easier'
	\exi{B:}  sí 
	\glt `yes'
	\exi{A:}  \textit{hombre mi madre tampoco lo ha tenido mal ¿no?} / pero yo recuerdo la primera lavadora que entró en mi casa \ldots 
	\glt `\textit{man my mother didn't have it all that bad, right}? but I do remember the first washing machine that came to my house \ldots'
	\end{xlist}
\end{exe}

\begin{figure}
	\includegraphics[width=\textwidth]{gfx/MADR_M31_041_A_5132_5134_hombre_mi_madre.jpg}
	\caption[L+H* L\% on \textit{hombre} in context (\ref{ex:hombrePRESEEAmadre})]{L+H* L\% on \textit{hombre} in context (\ref{ex:hombrePRESEEAmadre}) \href{https://osf.io/v97bu/}{\faVolumeUp} \label{fig:hombrePRESEEAmadre}}
\end{figure}

\subsubsection{L* HL\%}
\label{ch:5.2.2.2}

Already discussed at length for \textit{claro}, the L*~H(L)\% is also present on \textit{hombre}. \autoref{fig:hombrePRESEEAdependia} from context (\ref{ex:hombrePRESEEAdependia}) shows an instance of \textit{hombre} L* HL\%. Here, A affirms that he received part of his father's pension before and after his father's death. B asks why, to which A reacts by repeating the question instead of answering it. When B insists on the \ac{QUD}, A starts with \textit{hombre} L* HL\% and then affirms that he was dependent on his father, ending the turn with falling intonation on \textit{claro}. The use of \textit{hombre} L* HL\% here does not serve to contradict the provocation. Rather, it introduces the assertion of an expectable proposition.\largerpage

\begin{figure}[p]
	\includegraphics[width=\textwidth]{gfx/MADR_H21_020_A_2926_2927_hombre_dependia.jpg}
	\caption{L* HL\% on \textit{hombre} in context (\ref{ex:hombrePRESEEAdependia}) \href{https://osf.io/9bvg4/}{\faVolumeUp} \label{fig:hombrePRESEEAdependia}}
\end{figure}\clearpage

\begin{exe}
	\ex (Interview 20, \cite{PRESEEA.20142020})\label{ex:hombrePRESEEAdependia}
	\begin{xlist}[A:]
	\exi{A:}  instituto de la Seguridad Social / fui allí a pedir una / en el momento que murió mi padre / porque yo cobraba de la pensión de mi padre
	\glt `Social Security Institute / I went there to ask for a / the moment my father died / because I received from my father's pension'
	\exi{B:}  uhum 
	\glt `uhum'
	\exi{A:}  me dieron una / una pensión
	\glt `they gave me a pension'
	\exi{B:}  sí sí / ¿pero por qué cobrabas por la / por la pensión de tu padre? eso no no es decir 
	\glt `yes yes / but why did you receive from / from the pension of your father? that's not not to say'
	\exi{A:}  ¿por qué cobraba la pensión? 
	\glt `why did I receive the pension?' 
	\exi{B:}  hm 
	\glt `hm'
	\exi{A:}  \textit{¡hombre! dependía de él ¡claro!} 
	\glt `\textit{man! I depended on him, sure!}'
	\end{xlist}
\end{exe}

\autoref{fig:hombrePRESEEAdiferencia} from context (\ref{ex:hombrePRESEEAdiferencia}) shows that, just as with \textit{claro}, the final fall can be reduced to L*~H(L)-. Here, again, \textit{hombre} is used in an assertion confirmation, yet with the additional non-at-issue commitment to the expectability of the assertion. A knows, and has committed to the fact, that B has four children, with an age difference of sixteen years between oldest and youngest. When B shows A photos of her children, A reacts with surprise about the difference between the two, to which B responds with the use of \textit{hombre} L*~HL\%, the confirmation \textit{pues sí}, and a reminder of the age difference.

\begin{exe}
	\ex (Interview 28, \cite{PRESEEA.20142020}) \label{ex:hombrePRESEEAdiferencia}
	\begin{xlist}[A:]
	\exi{A:}  tienes cuatro / el mayor de veinticuatro \ldots ¿ocho años el pequeño? \ldots o sea que tienes una buena diferencia entre el mayor y el pequeño 
	\glt `you have four / the oldest twentyfour \ldots eight years the small one? so you have quite a difference between the oldest and the small one'
	\exi{B:}  sí sí muchísima diferencia \ldots mira este es el pequeño 
	\glt `yes yes a lot of difference \ look this is the small one' 
	\exi{A:}  ¡ay qué mono!
	\glt `oh how beautiful!'
	\exi{B:}  este es el mayor 
	\glt `that's the older one'
	\exi{A:}  jopé qué diferencia ¿eh? 
	\glt `wow what a difference, right?' 
	\exi{B:}  \textit{¡hombre! pues sí} // esta es la de veinte años 
	\glt `\textit{man! well sure} // this is the twenty year old one' 
	\end{xlist}
\end{exe}

\vfill
\begin{figure}[H]
	\includegraphics[width=\textwidth]{gfx/MADR_M22_028_A_hombre_si_diferencia.jpg}
	\caption{L* HL\% on \textit{hombre} in context (\ref{ex:hombrePRESEEAdiferencia}) \href{https://osf.io/46jmb/}{\faVolumeUp} \label{fig:hombrePRESEEAdiferencia}}
\end{figure}
\vfill\pagebreak

\subsubsection{The problem of H$-$}\label{ch:5.2.2.3}

Turns with \textit{hombre} L* HL\% are particularly prone to introducing assertive provocations confirming shared assumptions (as opposed to proffered content). Given that these assertions are provocations, they are usually sentences (not just particles) that may also show internal phrasing at the level of the intermediate phrase. Coming back to the problem outlined in \sectref{ch:2.2}, these examples illustrate the fact that some phrasing patterns require reference to discourse meaning, as opposed to syntactic mapping or eurhythmicity, in order to become the optimal candidate.  

\autoref{fig:hombrePRESEEAfaena} from context (\ref{ex:hombrePRESEEAfaena}) shows an example of such a provocation that serves to reassure the interlocutor of a shared assumption.\footnote{\autoref{fig:hombrePRESEEAfaena} is another example for the difficulty of analyzing prenuclear pitch accents in cases of prolonged prenuclear rises to H$-$. While there is no doubt about the presence of rises on \textit{amigo}/\textit{puede} and of a fall on \textit{hacer}, annotating L+H* or H+L*, respectively, would disregard the continuity of both rise and fall.} A has asked B how friends can hurt him the most. B responds that it bothers him the most if friends lie to him. In an attempt to restrict the meaning of lying to malicious mischief, he starts a provocation with \textit{hombre} L* HL$-$, then reassures A that a friend can play tricks on you without losing the quality of being a friend. The F$_0$ contour of the main sentence starts low after the intermediate phrase boundary and rises continuously until an H$-$ boundary after the inflected verb. It then falls to the low pitch accent, ending in an HL\% final boundary tone.\footnote{The syllabification in this example is phonetically [ˈfa͜i.na], not /fa.ˈe.na/ as transcribed phonologically.}

Why does the H$-$ fall on \textit{puede}? \autoref{tab:intonationalcategoriesGABRIEL} proposes four different functions for H$-$: a) delimitation of presupposed prefocal material, b) continuation in coordinate structures, c) syntactic disambiguation, and d) separation of left-peripheral \textit{topic} constituents. I argue that the H$-$ in \autoref{fig:hombrePRESEEAfaena} serves to delimit presupposed prefocal material. Inverting the (supposedly universal) sentential downtrend and letting the F$_0$ rise through the sentence up to the inflected modal verb allows the speaker to not only mark their context update as expectable from the \ac{CG} via L* HL\%, but also to presuppose the modal matrix sentence.

\begin{exe}
	\ex (Interview 13, \cite{PRESEEA.20142020}) \label{ex:hombrePRESEEAfaena}
	\begin{xlist}[A:]
	\exi{A:}  ¿qué es lo que más te duele a ti que te hagan / los que consideras amigos? 
	\glt `what hurts you the most for those you consider your friends to do to you?'
	\exi{B:}  a mí eeh ¡uf! // \ldots a mí lo que más me duele es que la gente mienta // o sea / y que / si alguien tiene algo que decirlo que decirte que te lo diga pero no que se calle / y luego vaya por detrás diciendo ``mira / tal'' no sé a mí eso es lo que más me molesta // porque / no sé / \textit{hombre / un amigo te puede hacer una faena} // y te puede doler / pero lo que me parece absurdo es callarte y ir cizañeando por detrás
	\glt `to me uh, phew! // \ldots what hurts me the most is that people lie // say / and that / if someone has something to say to say to you he should say it but not stay silent / and then go behind your back saying ``look / so and so'' I don't know to me that is what bothers me the most // because / I don't know / \textit{man / a friend may play a trick on you} // and it can hurt you / but what I find absurd is to stay silent and sow discord from behind'
	\end{xlist}
\end{exe}

\vfill
\begin{figure}[H]
	\includegraphics[width=\textwidth]{gfx/MADR_H13_013_A_2831_2834_hombre_faena.jpg}
	\caption{L* HL$-$ on \textit{hombre} and H$-$ phrasing in context (\ref{ex:hombrePRESEEAfaena}) \href{https://osf.io/ut5rc/}{\faVolumeUp} \label{fig:hombrePRESEEAfaena}}
\end{figure}
\vfill\pagebreak

\autoref{fig:hombrePRESEEAinviernos} from context (\ref{ex:hombrePRESEEAinviernos}) shows another case of H$-$ phrasing after \textit{hombre} L* HL$-$. It groups \textit{mucho}, which is part of the reduced small clause \textit{(que) los inviernos (eran) mucho mas fríos} \citep[239--294]{Stowell.1981}, together with the matrix sentence.\footnote{Many thanks to Silvio Cruschina for advice on this analysis.} Here, again, prosodic phrasing does not follow syntactic mapping constraints. Rather, the prenuclear rise extends rightward up to the point at which the L* pitch accent requires the F$_0$ to fall. Phonologically, this seems to be a case of rightward tonal alignment which is not limited by syntactic mapping (association with the edge of a prosodic constituent that would correspond to a syntactic constituent), but only by tonal crowding with the pitch accent associated with the last stressed syllable in the intonational phrase. The function of such right alignment is to mark as much of the sentence as possible as presupposed, only to then mark the answer to the current \ac{QUD} (the Table) as expectable via L* HL\%. This double strategy is difficult to capture within the model presented in \sectref{ch:3.3.3}, since it requires a distinction between presupposition and conventional implicature so that both can be present in one conversational move. Modeling this would require two levels of non-at-issue meaning, one requiring a presupposition to be in the input \ac{CG} and one adding a modal non-at-issue commitment to the output \ac{CG}. Nevertheless, an analysis of the prolonged rise to H$-$ in terms of presupposed prefocal material at least provides some sort of explanation, whereas analyses in terms of continuation or syntactic disambiguation simply fail to explain the phrasing structure.

Note that separation of left-peripheral \textit{topic} constituents can account for H$-$ phrasing as in Figures~\ref{fig:hombrePRESEEAmadre} and~\ref{fig:hombrePRESEEAciudad}, but not for the phrasing in \autoref{fig:hombrePRESEEAheterogeneo}, since \textit{mi barrio es muy$_{H-}$} is not a constituent, let alone one identifying an ``entity under which the information expressed in the comment constituent should be stored in the common ground content'' \citep[28]{KrifkaMusan.2012}.


\begin{exe}
	\ex (Interview 36, \cite{PRESEEA.20142020})\label{ex:hombrePRESEEAinviernos}
	\begin{xlist}[A:]
	\exi{A:}  oye ¿y tú has observado // que ha habido un cambio de tiempo // de tiempo 
	\glt `listen and did you observe // that there has been a change of weather // of weather'
	\exi{B:}  ¿climático? \ldots yo creo que sí 
	\glt `of climate? \ldots I think yes' 
	\exi{A:}  ¿en qué sentido? 
	\glt `in which sense?'
	\exi{B:}  pues que hace más calor cada vez // vamos / a mí me parece que cada vez o más raro o eso pero / pero ya 
	\glt `well that it's always getting hotter // say / to me it seems that always or more bizarre or that but / but so'
	\exi{A:}  ¿y cuando tú eras pequeña por ejemplo / durante ¿cómo eran los inviernos y cómo son ahora? 
	\glt `and when you were young for example / during how used to be the winters and how are they now?'
	\exi{B:}  \textit{¡hombre! yo recuerdo los inviernos mucho más fríos}
	\glt `\textit{man! I remember the winters a lot colder}' 
	\end{xlist}
\end{exe}

\begin{figure}
	\includegraphics[width=\textwidth]{gfx/MADR_M23_036_A_3829_3831_hombre_inviernos.jpg}
	\caption{L* HL$-$ on \textit{hombre} and H$-$ phrasing in context (\ref{ex:hombrePRESEEAinviernos}) \href{https://osf.io/jwtbf/}{\faVolumeUp} \label{fig:hombrePRESEEAinviernos}}
\end{figure}

In sum, these corpus examples indicate that quantitative laboratory investigation should be able to confirm question (\ref{ex:questionsoverview}d). Obvious, non-contrastive assertions not only show an L* HL\% nuclear configuration, but can also include a prolonged rise to an H$-$, phrasing subject, verb, and even (parts of) verbal complements into one intermediate phrase. We return to this point in \chapref{ch:6}, focusing our exploration of corpus examples now on turns preceded by the discourse particles \textit{anda} and \textit{vaya}.

\subsection{Turns with \textit{anda}}
\label{ch:5.2.3}

As already discussed for \textit{hombre}, the lack of a syllable onset on \textit{anda} can cause tonal apheresis on the (L)+H* pitch accent. Nevertheless, examples such as \autoref{fig:andaPRESEEAidolo} from context (\ref{ex:andaPRESEEAidolo}) show that L+H* rises can be fully realized even in one-word utterances.

\begin{exe}
	\ex (Interview 52, \cite{PRESEEA.20142020}) \label{ex:andaPRESEEAidolo}
	\begin{xlist}[A:]
	\exi{A:} mi foto con P con P D que es / uno de mis ídolos  
	\glt `my photo with P with P D which is / one of my idols'
	\exi{B:} ¡ay por favor! a ver / que no se caiga esto 
	\glt `oh please! let's see / that it doesn't fall down'
	\exi{A:} y ahí tengo otra que es A A  
	\glt `and here I have another one which is A A'
	\exi{B:}  \textit{¡anda!}
	\glt `\textit{wow!}' 
	\end{xlist}
\end{exe}

\begin{figure}
	\includegraphics[width=\linewidth]{gfx/MADR_M33_052_A_002908_002910_anda_que_es_A_A.jpg}
	\caption[L+¡H* L\% on \textit{anda} in context (\ref{ex:andaPRESEEAidolo})]{L+¡H* L\% on \textit{anda} in context (\ref{ex:andaPRESEEAidolo}) \href{https://osf.io/kqvn2/}{\faVolumeUp} \label{fig:andaPRESEEAidolo}}
\end{figure}

Lack of tonal apheresis may be due to L+¡H* L\% tonal scaling on the pitch accent. Yet the decision about a categorical upstep cannot be made without taking the average tonal range as a point of comparison. While transcription with exclamation marks can be seen as indicative of perceived tonal upstep, comparison between speakers and examples is best done under experimental conditions (\chapref{ch:6}).\footnote{Contextually controlled experiments also have the advantage of giving us complete access to ``functional or semantic criteria [which] provide a sounder basis for this determination than [\ldots] formal or phonetic ones.'' \citep[112]{Ladd.1980}} The best natural dialogue corpus evidence on scaling is a comparison between rising pitch accents by one speaker and within one turn. Here, differences in scaling cannot be attributed to differences in speaker style or speech rate. \autoref{fig:andaPRESEEAbruselas} from context (\ref{ex:andaPRESEEAbruselas}) shows a case of assertion confirmation with \textit{anda}, together with mirative intonation on \textit{a Bruselas}. While no exclamation marks are used, the question marks on \textit{¿a Bruselas?} indicate that the transcriber noticed an additional, non-assertive intonational meaning in this part of the utterance. Comparing the tonal span of the pitch accent on \textit{Bruselas} with that of the pitch accent on \textit{anda} warrants a distinction between (L)+H* L\% and L+¡H* L\%.

\begin{exe}
	\ex (Interview 34, \cite{PRESEEA.20142020}) \label{ex:andaPRESEEAbruselas}
	\exi{A:} ¿cuáles son tus planes?  
	\glt `what are your plans?' 
	\exi{B:} pues no lo sé si me quedaré en Madrid  
	\glt `well I don't know if I'm going to stay in Madrid' 
	\exi{A:} hm  
	\glt `hm' 
	\exi{B:} o me iré a Bruselas / no lo sé 
	\glt `or I'll go to Brussels / I don't know'
	\exi{A:}  \textit{ah ¿a Bruselas? anda} 
	\glt `\textit{ah to Brussels? wow}' 
\end{exe}


\begin{figure}
	\includegraphics[width=\textwidth]{gfx/MADR_M23_034_A_001202_001204_anda_Bruselas.jpg}
	\caption{(L)+H* L\% on \textit{anda} and L+¡H* L\% on PP in context (\ref{ex:andaPRESEEAbruselas}) \href{https://osf.io/pvwf2/}{\faVolumeUp} \label{fig:andaPRESEEAbruselas}}
\end{figure}

Similar to what has been found for declarative speech acts in Madrid Spanish in general, variation between L+H* and L* on \textit{anda} seems not restricted by any apparent constraint.\footnote{If this variation is actually free should be further investigated.} \autoref{fig:andaPRESEEAquinientos} from context (\ref{ex:andaPRESEEAquinientos}) shows a repetitive use of \textit{anda} L+H*~L\% that is akin to the repetitive use of English \textit{whoa}.\footnote{\textit{Whoa} is etymologically derived from the exclamation \textit{ho}, used to halt horses \citep{Skeat.1896}.} American English \textit{whoa} differs from \textit{wow} in that it has become its own strongest collocation according to the Corpus of Contemporary American English \citep{Davies.19902019}.\footnote{With \ac{MI}$=$8.24 and a total of 1020 co-occurrences, \textit{whoa} actually has only itself as a collocation, which shows that it cannot be used as a syntactically integrated part of speech.} \textit{Whoa whoa}, much as \textit{anda anda}, is used to ask the interlocutor to slow down. It doesn't mark what should be slowed down, so interlocutors need to deduce this from the context. In (\ref{ex:andaPRESEEAquinientos}), a reasonable interpretation of \textit{anda anda anda} is to reduce the pace of reduction of the set of possible worlds (the shrinking of the context set, \cite{Stalnaker.1978}), or to assert propositions that are more in line with the \ac{CG}. \autoref{fig:andaPRESEEAgracias} from context (\ref{ex:andaPRESEEAgracias}) is another example of this repetitive use. In (\ref{ex:andaPRESEEAgracias}), the conversational implicature of \textit{anda anda} is to stop the action of handing over money for participation in an interview. The contextual similarity between (\ref{ex:andaPRESEEAquinientos}) and (\ref{ex:andaPRESEEAgracias}) leaves little functional load to the prosodic difference between L+H* in \autoref{fig:andaPRESEEAquinientos} and L* in \autoref{fig:andaPRESEEAgracias}.\largerpage

\begin{figure}
	\includegraphics[width=\textwidth]{gfx/MADR_M11_006_A_010102_010103_anda_quinientos_millones.jpg}
	\caption{L+H* L\% on repeated \textit{anda} in context (\ref{ex:andaPRESEEAquinientos}) \href{https://osf.io/ptc69/}{\faVolumeUp} \label{fig:andaPRESEEAquinientos}}
\end{figure}

\begin{exe}
	\ex (Interview 06, \cite{PRESEEA.20142020})\label{ex:andaPRESEEAquinientos}
	\begin{xlist}[A:]
	\exi{A:} gente como una chica el otro día en Estados Unidos que no sé cuánto le habrá tocado / pero vamos es una burrada más de doce mil millones de pesetas  
	\glt `people like a girl in the US the other day where I don't know how much she'll have won / but hell it's a shitload more than twelve thousand million pesetas' 
	\exi{B:} una cosa impresionante  
	\glt `something extraordinary' 
	\exi{A:} yo dudo que esa chica algún día diga / ``ay qué pena / me podía haber tocado algo más'' // pero al que le tocan quinientos hoy en día // que se puede hacer // con doscientos o trescientos millones te compras una casa / pues sí // quinientos millones ya no son nada   
	\glt `I doubt that this girls will one day say / ``oh what a shame / I could have won some more'' // but who wins five-hundred today // what can you do // with two hundred or three hundred million you buy a house / well yes // five-hundred million are nothing anymore' 
	\exi{B:} \textit{anda anda anda} 
	\glt `\textit{whoa whoa whoa}'
	\end{xlist}

\ex (Interview 35, \cite{PRESEEA.20142020})\label{ex:andaPRESEEAgracias}
	\begin{xlist}[A:]
	\exi{A:}  muchas gracias por todo \hfill [entrega dinero] 
	\glt `many thanks for everything' \hfill [hands over money]
	\exi{B:}  uy hija // venga / \textit{anda anda} 
	\glt `oh child // come on / \textit{whoa whoa}' 
	\exi{A:}  sí que me has contado muchas cosas 
	\glt `{you did tell me many things}' 
	\exi{B:}  no me las des // uh pues ni la mitad 
	\glt `don't give them to me // oh well not half of it' 
	\end{xlist}
\end{exe}

\begin{figure}
	\includegraphics[width=\textwidth]{gfx/MADR_M23_035_A010145_010147_anda_muchas_gracias.jpg}
	\caption[L* L\% on repeated \textit{anda} in context (\ref{ex:andaPRESEEAgracias})]{L* L\% on repeated  \textit{anda} in context (\ref{ex:andaPRESEEAgracias}) \href{https://osf.io/cjbev/}{\faVolumeUp} \label{fig:andaPRESEEAgracias}}
\end{figure}

Complex discourse updates containing both \textit{anda} and \textit{claro} again show the limits of the model presented in \sectref{ch:3.3}. \autoref{fig:andaPRESEEAtalcual} from context (\ref{ex:andaPRESEEAtalcual}) is a case of reported dialogue between a mother and her daughter. The daughter, having recently had a spiritual awakening, is characterized as firmly knowledgeable about the Bible and therefore expected to perform well in a test on biblical texts. When her mother tells her that she expects her to have performed well, the daughter humbly marks her acceptance of such high expectations with \textit{anda} (L)+H* L$-$, only to then confirm her having lived up to these expectations with obvious \textit{claro} L+H* L!H\%. The context-update potential of \textit{claro} L+H* L!H\% can be represented as assertion confirmation with a non-at-issue update adding the modal necessity of the proposition to the \ac{CG}. Yet the fact that this does not conflict with previous \textit{anda} L+H* L$-$ can only be explained by the fact that the provocation contains a non-at-issue proposition (\textit{dirás tal cual} `you'll say it as it is') and an at-issue modal evaluation (\textit{me imagino} `I imagine'), the latter of which can be accepted and marked as surprising (I did not expect you to think this way, but I accept it) independently of the former (of course I said it as it is). The difference between confirmation of an at-issue modal matrix sentence and subsequent confirmation of the embedded, non-at-issue proposition would require an extension of the model in \sectref{ch:3.3} that I leave to future research.\largerpage

\begin{exe}
	\ex (Interview 23, \cite{PRESEEA.20142020})\label{ex:andaPRESEEAtalcual}
	\begin{xlist}[A:]
	\exi{A:} al día siguiente se lo contó a una amiga // y la regaló una biblia pequeñita // es el único regalo que mi hija ha tenido // \ldots a continuación / pues / no sé si ese mismo año / o / o al año siguiente // una profesora / eeh / mandó hacer una redacción / de su primera comunión \ldots / y entonces vino a casa y me dice ``mamá me han dicho que tengo que hacer una redacción de la primera comunión'' // digo / ``me imagino que dirás tal cual'' \textit{dice ``¡anda claro!''} // y lo hizo en sucio  
	\pagebreak\glt `the following day she told a friend // who gave her a small bible as a gift // it's the only gift my daughter has had // \ldots then / well / I don't know if this very year / or / or the following year // a teacher / aah / had her do a draft of her first communion \ldots / and so she came home and says ``mama they have told me that I have to do a draft of the first communion'' // I say // ``I imagine you'll say it as it is'' \textit{she says ``whoa sure!''} // and she did a first sketch'
	\end{xlist}
\end{exe}

\begin{figure}
	\includegraphics[width=\textwidth]{gfx/MADR_M21_023_A_010442_010443_anda_tal_cual.jpg}
	\caption[L+H* L$-$ on \textit{anda} and L+H* L!H\% on \textit{claro} in context (\ref{ex:andaPRESEEAtalcual})]{L+H* L$-$ on \textit{anda} and L+H* L!H\% on \textit{claro} in context (\ref{ex:andaPRESEEAtalcual}) \href{https://osf.io/c4zrk/}{\faVolumeUp} \label{fig:andaPRESEEAtalcual}}
\end{figure}


\subsection{Turns with \textit{vaya}}\label{ch:5.2.4}\largerpage

As discussed in \sectref{ch:5.1}, \textit{vaya} is used for assertion confirmation and negative evaluation, often in combination with a conversational implicature of unexpectedness. Accordingly, the PRESEEA Madrid Salamanca corpus contains no instances of \textit{vaya} with either L* HL\% or L+H* L!H\% intonation. Instead, we find almost exclusive use of L* L\%, with one instance of L+H* L\%. \autoref{fig:vayaPRESEEAfallo} from context (\ref{ex:vayaPRESEEAfallo}) shows a typical use of \textit{vaya} L* L\%.

\begin{exe}
	\ex (Interview 38, \cite{PRESEEA.20142020}) \label{ex:vayaPRESEEAfallo}
	\begin{xlist}[A:]
	\exi{A:}  hemos seguido tan felices hasta que ha llegado la desgracia esta  
	\glt `we went on so happy until this tragedy came' 
	\exi{B:}  ¿y hace mucho tiempo que falta o?  
	\glt `and has he been gone for long or?' 
	\exi{A:} hace dos años y un mes ha hecho ahora  
	\glt `since two years and one month it has been now' 
	\exi{B:} \textit{vaya}  
	\glt `\textit{damn}'
	\end{xlist}
\end{exe}

\begin{figure}
	\includegraphics[width=\linewidth]{gfx/MADR_H31_038_A_vaya_fallo_cárdico.jpg}
	\caption[L* L\% on \textit{vaya} in context (\ref{ex:vayaPRESEEAfallo})]{L* L\% on \textit{vaya} in context (\ref{ex:vayaPRESEEAfallo}) \href{https://osf.io/jzn9c/}{\faVolumeUp} \label{fig:vayaPRESEEAfallo}}
\end{figure}

As already discussed in \sectref{ch:5.1}, there is not a single instance of a male speaker using \textit{anda} as a marker of surprise in the PRESEEA Madrid Salamanca Corpus. \textit{Vaya} can be seen as a way of indirect communication of mirativity by conversational implicature from negative bouletic evaluation. \autoref{fig:vayaPRESEEAstauffer} from context (\ref{ex:vayaPRESEEAstauffer}) shows a use of \textit{vaya} L* L\% acknowledging the fact that the house of a famous Spanish entrepreneur was taken down and seized.\largerpage

\begin{exe}
	\ex  (Interview 25, \cite{PRESEEA.20142020})\label{ex:vayaPRESEEAstauffer}
	\begin{xlist}[A:]
	\exi{A:}  pues ahí había una casa // había una casa / grande / una casona / que / eso me parece que era luego de Gil Stauffer / estaba vallada // y la tiraron para hacer esto // y luego esto debió ser embargado por el ayuntamiento / o algo así  
	\glt `so there used to be a house // there used to be a big house / a mansion / which / I think it was then Gil Stauffer's / it was fenced / and they took it down to do this // and then it must have been seized by the municipality / or something like that'
	\exi{B:}  \textit{vaya}
	\glt `\textit{damn}' 
	\end{xlist}
\end{exe}


The one example I found of \textit{vaya} with L+H* L\% intonation is represented in \autoref{fig:vayaPRESEEAproblemas} from context (\ref{ex:vayaPRESEEAproblemas}). As is the case for Madrid Spanish intonation in general, context does not indicate a difference in context update potential between \textit{vaya} L* L\% and \textit{vaya} L+H* L\%. And while (\ref{ex:vayaPRESEEAproblemas}) is a case of transcription with exclamation marks, \autoref{fig:vayaPRESEEAsuspendia} from context (\ref{ex:vayaPRESEEAsuspendia}), repeated for convenience in (\ref{ex:vayaPRESEEAsuspendia2}), indicates that exclamation marks need not correspond to a rising pitch accent.

\begin{figure}[p]
	\includegraphics[width=\linewidth]{gfx/MADR_H22_025_A_vaya_Gil_Stauffer.jpg}
	\caption[L* L\% on \textit{vaya} in context (\ref{ex:vayaPRESEEAstauffer})]{L* L\% on \textit{vaya} in context (\ref{ex:vayaPRESEEAstauffer}) \href{https://osf.io/d862v/}{\faVolumeUp} \label{fig:vayaPRESEEAstauffer}}
\end{figure}\clearpage

\begin{exe}
	\ex (Interview 34, \cite{PRESEEA.20142020}) \label{ex:vayaPRESEEAproblemas}
	\begin{xlist}[A:]
	\exi{A:}  ¿mi madre? \ldots mide uno ochenta \ldots un problema // para su época \ldots 
	\glt `my mother? \ldots is one eighty tall \ldots a problem // for her time \ldots' 
	\exi{B:}  no encontraría pareja ni para el baile \ldots ~eso le ocasionaba un problema?
	\glt `she wouldn't even find a partner for the dance \ldots ~that caused her problems?' 
	\exi{A:}  sí 
	\glt `yes' 
	\exi{B:} \textit{¡vaya!} 
	\glt `\textit{damn!}'
	\end{xlist}
\end{exe}

\begin{figure}
	\includegraphics[width=\linewidth]{gfx/MADR_M23_034_A_vaya_ocasionaba_problema.jpg}
	\caption[L+H* L\% on \textit{vaya} in context (\ref{ex:vayaPRESEEAproblemas})]{L+H* L\% on \textit{vaya} in context (\ref{ex:vayaPRESEEAproblemas}) \href{https://osf.io/kvt7j/}{\faVolumeUp} \label{fig:vayaPRESEEAproblemas}}
\end{figure}

\begin{exe} 
	\ex  (Interview 11, \cite{PRESEEA.20142020})\label{ex:vayaPRESEEAsuspendia2} 
	\begin{xlist}[A:]
	\exi{A:} \ldots y nada luego pues lo que pasa es que la mayoría de la gente con la que yo iba acababa aprobando y yo suspendía 
	\glt `and so then the thing is that the majority of people I went with ended up passing (the exam) and I failed' 
	\exi{B:} \textit{¡ah vaya!} 
	\glt `\textit{ah damn!}' 
	\end{xlist}
\end{exe}

\vfill
\begin{figure}[H]
	\includegraphics[width=\linewidth]{gfx/MADR_M11_004_A_001245_001246_vaya_suspendia.jpg}
	\caption[L* L\% on \textit{vaya} in context (\ref{ex:vayaPRESEEAsuspendia2})]{L* L\% on \textit{vaya} in context (\ref{ex:vayaPRESEEAsuspendia2}) \href{https://osf.io/r9juv/}{\faVolumeUp} \label{fig:vayaPRESEEAsuspendia}}
\end{figure}
\vfill\pagebreak

\section{Preliminary conclusions and experimental tasks}\label{ch:5.3}\largerpage

Having used discourse particles as an indicator for marked discourse moves in the sense of \sectref{ch:3.3}, we have seen in \chapref{ch:5} that discourse particles which are specified for both a relative polarity function and a modal evaluative function occur with intonational contours that mirror these functions. While the corpus examples presented in \sectref{ch:5.2} are not sufficient to fully disentangle the individual contribution of intonation, the fact that \textit{anda} and \textit{vaya} do not occur with nuclear contours associated with obviousness in the literature, while \textit{hombre} and \textit{claro} do, supports the hypothesis that modal non-at-issue meaning has an impact on the distribution of the respective nuclear contours. \textit{Anda} L+H* LH\% or \textit{vaya} L* HL\% are unlikely combinations because a context in which the meaning of \textit{anda} and \textit{vaya} would be compatible with the intonational meaning associated with the respective contours will almost never occur.\footnote{Notwithstanding oxymoronic and ironic uses.} Moreover, the fact that \textit{anda} co-occurs with upstepped L+¡H* pitch accents, while \textit{vaya} is almost categorically associated with L* pitch accents, mirrors the difference in the modal accessibility relation in the meaning of the two particles established in \sectref{ch:5.1}. \chapref{ch:5} does not provide a final proof of the association of specific intonational forms with specific modal meanings. Yet it illustrates the usefulness of a model that combines a perspective on the negotiation of discourse commitments and \ac{CG} updates with a perspective on modal evaluative meaning. Experimental investigation needs to tackle the individual contribution of intonation on the sentence level, taking into account both ``prenuclear'' and ``nuclear'' intonation. The comparison between \textit{anda} and \textit{vaya} has shown the difficulty in distinguishing between a modal evaluation of a proposition as epistemically unexpected or bouletically unwanted. In the case of \textit{vaya}, Madrid Spanish may even be undergoing language change lexicalizing the conversational implicature of unexpectedness often associated with strongly negative evaluation.\footnote{Possibly a link in the semantic maps of epistemicity and bouletic modality \citep{Anderson.1986,Boye.2010}.} Experimental investigation needs to take this into account by avoiding contextual ambiguity between these two meanings. Similarly, experimental investigation needs to allow for sentence-internal prosodic phrasing, given that turns with \textit{hombre} L* HL\% seem to also resort to H$-$ phrasing that cannot be explained by syntactic mapping constraints. Finally, corpus examples such as (\ref{ex:andaPRESEEAtalcual}) show the importance of controlling the exact form of the provocation when investigating responding moves. The \textit{Provocation-Response Nexus} needs to be incorporated into task designs, particularly the Discourse Completion Task. \chapref{ch:6} is an attempt at incorporating these preliminary insights into one experiment.
