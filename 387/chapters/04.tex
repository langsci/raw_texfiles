% Chapter X

\chapter{Summary and methodological implications}
\label{ch:4} % For referencing the chapter elsewhere, use \chapref{ch:name} 

\section{Main arguments}
\label{ch:4.1}

Before turning to the empirical treatment of the problems presented so far, I want to sum up the arguments and reflect on their methodological implications. In \chapref{ch:2}, we have seen that the presence or absence of pitch accents (F$_{0}$ excursions on stressable syllables) depends on discourse contexts \citep[74]{OrtegaLlebariaPrieto.2011}, which I take as an argument for seeing syllables not as $\pm$stressed, but as $\pm$stressable, with phonetic correlates surfacing at higher levels of prosodic structure only under the condition that sentence prosody associates accents with stressable syllables. These accents, in turn, are partly conditioned by syntax-prosody mapping constraints, but also by illocutionary mood, information structure, and ``interactive attitudinal aspects'' \citep[156]{Fery.2017}. 

I argued that (SV)$_\phi$(O)$_\phi$ phrasing can only partly be explained on the basis of mapping rules or constraints that \textit{match} or \textit{wrap} syntactic constituents. Instead, it should be linked to the scaling implications of the proposal by \citet[110--112]{Hualde.2002} to interpret H$-$ as an indicator of givenness (or presupposed prefocal material according to \cite{Gabriel2007}, see \autoref{tab:intonationalcategoriesGABRIEL}), namely a reversal of the standardly assumed downtrend throughout the utterance up to the H$-$. The empirical challenge posed by this prenuclear global rise is further complicated by the variety of possible realizations of intermediate phrase accents, some more locally realized (continuation rise), some spreading over several syllables (sustained pitch, preboundary upstep, pitch reset) \citep{GabrielFeldhausenPeskova2011}. 

The importance of solving the role of intermediate phrase accents is underlined further by the debate surrounding prenuclear rising pitch accents in Spanish, which are said to vary according to illocutionary mood \citep{FacePrieto.2007} or according to the proximity of phrase accents and boundary tones (\cite[106]{Hualde.2002}, \cite{Gabriel2007}).

In reviewing the intonational phonemes proposed for Spanish, I noted that complex phrase accents (LH$-$, HL$-$, LHL$-$) and boundary tones (LH\%, L!H\%, HL\%, LHL\%) are predominantly linked to notions of anti-expectation/incredulity, obviousness, and insistence (Tables~\ref{tab:phraseacacentsSPANISH} and \ref{tab:boundarytonesSPANISH}). Finally, I argued that two recent investigations of Madrid Spanish intonational phonology, \citet{ElviraGarcia.2016} and \citet{TorreiraGrice.2018}, both place obviousness at the heart of their respective analyses but suffer from the lack of a model of intonational meaning that would relate it to other aspects of discourse meaning.

Taken together, in \chapref{ch:2} I have tried to show from different angles of the phonological debate about Spanish intonation the need for a clarification of the relation between sentence types such as \textit{declarative}, \textit{interrogative}, \textit{exclamative} on the one hand, and \textit{anti-expectation}, \textit{obviousness}, \textit{insistence} on the other hand. Once understood, these sen\-tence-lev\-el meanings need to be related to information structure, which operates on parts of sentences.

\chapref{ch:3} first approaches this task decompositionally, before presenting a model that recomposes these complex discourse functions in a unified fashion. Regarding exclamatives, I proposed to disentangle the contribution of \textit{wh}-ex\-clam\-a\-tive syntax (which seems to be \textit{factivity} and \textit{scalar implicature}) from the mirative component (instead of \textit{widening}, \cite{ZanuttiniPortner2003}), which seems attributable to marked intonation in many of the relevant examples. Regarding statements of the obvious, we noted that the state of the art is highly inconclusive about corresponding prosodic forms, which seems to be due to a lack of understanding of what obviousness actually is. I further noted that one of the few points of agreement in the literature on obviousness is the possibility to disambiguate prosody by means of particles such as \textit{claro}, \textit{pues}, and \textit{hombre/mujer} (\cite[19]{BeckmanETAL.2002}, \cite{Prieto2009-2013}, \cite[278]{Hualde.2014}, \cite{TorreiraGrice.2018}).

I then proposed to use a modalized version of the Farkas and Bruce model \citep{FarkasBruce.2010, Rett.2021emotivemarkers} to arrive at a more complex understanding of the interplay between illocutionary moods, relative polarities, and modal stances towards propositions. This leads to an understanding of assertions in terms of at-issue Discourse Commitments, but also to an understanding of evaluative non-at-issue Discourse Commitments as direct modal Common Ground updates. I finally argued that for mirativity and obviousness, these Discourse Commitments mark the proposition as necessary/impossible from the perspective of the input Common Ground, thereby shifting the world from which the modal base of the non-at-issue evaluation is accessible one context-update back. 

Furthermore, in responses to biased provocations, this shift reaches two updates back. This is particularly relevant for statements of the obvious, which in reversals are informative. The model therefore predicts a markedness relation with regard to relative polarity. An obvious declarative that serves as a reversal of a previous assertion (or a biased tag-question) should be more marked than an agreement. Since markedness as presented in \chapref{ch:3} is seen as a predictor of overt coding, we expect differences between the (prosodic) modal marking of reversals and confirmations. In natural dialogue, we also expect lexical markers of (dis)\-a\-gree\-ment to occur together with prosodically marked modal assessment.

Modal intonation should behave just as ``chameleonic'' \citep[34]{FintelGillies.2007} as lexical modals, with modally marked intonation partly underspecified until context provides a modal base. An empirical investigation of the prosodic expression of one type of modality should therefore check for interfering effects of another type of modality.\footnote{Moreover, I would expect turns which assert a high expectation about a proposition and conventionally imply a surprise about this proposition to be either infelicitous or ironic (\ref{ex:atissuenonatissuecontradiction}).
	
\begin{exe}
\ex[\#]{\label{ex:atissuenonatissuecontradiction}Wow! I totally expected this!}
\end{exe}}

Finally, intonation is an independent cue. Modal meaning seems to be expressed by such diverse strategies as insubordinate syntax, discourse particles, and intonation. This means that intonation should not depend on the occurrence of these other strategies. While the general prevalence of redundancy in phonology and grammar (\cite[184]{Pinker.1994}, \cite{Shannon.1948}) would have us expect intonational marking to co-occur with other markers of modal non-at-issue commitments, it should in principle also occur without them.

The pending questions for empirical investigation are summed up in (\ref{ex:questionsoverview}). In \sectref{ch:4.2}, we turn to the methodological implications of these tasks, taken on in Chapters~\ref{ch:5} and~\ref{ch:6}.

\begin{exe}
\ex Pending questions\label{ex:questionsoverview} 
\begin{xlist}\sloppy
	\ex Can the findings about nuclear contours of statements of the obvious and mirative exclamations be reproduced beyond individual examples? 
	\ex Can mirative intonation be disentangled from exclamative syntax? 
	\ex Does relative polarity affect obvious intonation? 
	\ex Is phonological phrasing affected by modal non-at-issue commitments? 
	\ex Does intonation correlate with other non-at-issue markers such as discourse particles? 
\end{xlist}
\end{exe}

\section{Methodological implications}\label{ch:4.2}

\begin{displayquote}
Researchers working on language and speech are no ``signal hunters'', but hunt for functions and meanings as reflected in the speech signal [\ldots] \\
The out-of-the-way setting of a recording booth can be conducive to out-of-the-way linguistic behaviour, in cases where the speaker lacks a real addressee or a real communicative task to perform. \citep[3, 10]{NiebuhrMichaud.2015}
\end{displayquote}

\textit{Laboratory speech}, though central to the endeavor of \textit{laboratory phonology} to go beyond an ``impressionistic transcription of a corpus of utterances'' \citep[6]{CohnFougeronHuffman.2012intro}, often provides as many advantages as disadvantages. This becomes particularly visible when dealing with the pragmatics of intonation, where functions and meanings are much more dependent on interactional settings and therefore less accessible for elicitation via visual stimuli or textual cues. The trade-off seems to be between naturalness and control, with control over lexical material, syntax, and many other aspects of speech varying greatly between elicitational and observational data (\cite[217]{KasperDahl.1991}, \cite[192]{VanrellFeldhausenAstruc.2018}).\largerpage

The model presented in \sectref{ch:3.3} is based on the idea that interlocutors negotiate commitments to stances about the way things are or ought to be. The markedness relations predicted in the model are the pitfalls speakers face when entering such negotiation. It is to be expected that their willingness and capability to avoid or bridge them will vary greatly depending on their involvement in the topic of conversation, their social relation with interlocutors, the global setting, and likely many more factors. What does this mean for data acquisition and empirical investigation? Firstly, it imposes minimum requirements on the kind of data to acquire. Provocations, responses, and relative polarity are prototypically associated with dialogues. While alterity can also be constructed in monological settings (e.g. in soliloquies), a corpus of dialogical data seems preferable for the detection of context updates. Moreover, the relation between provocations and responses goes beyond lexical and syntactical structure. Crucially, since the prosodic form of the provocation determines the associated commitments, responding moves can only be understood with access to the prosodic form of their provocations. I call this the \textit{Provocation-Response Nexus}.

This nexus is no minor issue. If a laboratory production experiment is intended, it excludes any elicitation strategy in which the prosodic form of the provocation is underspecified (e.g. written text) or varies from elicitation to elicitation (e.g. provocation by an investigator/lab technician/etc.). To date, research on Spanish intonation has not achieved this level of control. Though it may seem as if perception studies avoid this problem by using an invariable stimulus as provocation, they offer either ``silent'' (written) choices or capture only a narrow section of the range of possible reactions. Moreover, obtaining and selecting the recordings needed for forced-choice perception tasks is itself a process that requires interpretable production data and selectional criteria.

I see three solutions to this problem: Firstly, to investigate natural dialogue data to develop an idea of some key features detectable in spontaneous speech and thereby sharpen the hypotheses derived from our model and from the literature. Even though a fine-grained, comparative phonetic and phonological analysis is fraught with difficulties with such data, some intonational tendencies should become visible. This strategy is pursued in \chapref{ch:5}. A second solution is to enhance trusted and well-established experimental set-ups (such as the \textit{Discourse Completion Task}, \cite{VanrellFeldhausenAstruc.2018}) so as to allow for control of the \textit{Provocation-Response Nexus}. Experimental data allow us to control as many contextual and phonotactic variables as possible to determine the specific contribution of a) modal conversational backgrounds (expectations, desires, etc.) and b) relative polarity (agreement and disagreement). This strategy is pursued in \chapref{ch:6}.

A combination of methods requires a selection of points of interest, since not all possible combinations of discourse context, illocutionary mood, at-issue and non-at-issue Discourse Commitment can be explored at once. In the following chapters, I will concentrate on the way assertive speech acts prosodically express mirativity and obviousness under different settings of relative polarity. Importantly, I thereby exclude interrogatives and directives from the scope of investigation.

\chapref{ch:5} presents a way of obtaining these marked discourse moves in free dialogue corpora not specially designed for the purpose of investigating intonation and pragmatics. \chapref{ch:6} describes the methodology and results of a laboratory production experiment designed for the purpose of answering the questions in (\ref{ex:questionsoverview}).
