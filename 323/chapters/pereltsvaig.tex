\documentclass[output=paper]{langsci/langscibook} 
\ChapterDOI{10.5281/zenodo.5524304}
\author{Asya M. Pereltsvaig\affiliation{}}
\title{Noun phrases, big and small}
\abstract{This chapter is concerned with the size of noun phrases in languages with and, more importantly, without articles. 
    It is argued that noun phrases in such languages may -- but need not -- contain the functional projection of DP. 
    Moreover, it is shown that DPs and small nominals (i.e., noun phrases lacking the DP) not only have different internal structure, 
    but also differ in their external distribution, their semantic interpretation, their ability to move, their need for case, and more. 
    It is shown that the same cluster of properties, characteristic of Small Nominals, is attested in diverse, even typologically diverse, languages.
}

\begin{document}
\SetupAffiliations{mark style=none}
\maketitle 

%\begin{stylelsAbstract}
%Keywords: noun phrases, DP, Small Nominals, article-less languages, referentiality, case
%\end{stylelsAbstract}

\section{Introduction}

Susi Wurmbrand’s research in syntax revolves around the issue of clause size: 
How much functional structure do different types of clauses have? 
How transparent different types of clauses are? 
What cross-clausal A- and A'-dependencies do clauses with more or less functional structure allow? 
This paper takes this research agenda further by examining similar questions in connection with noun phrases of different sizes 
and cataloguing properties of noun phrases, big and small, across languages.

Since the work of \citet{Szabolcsi1983} and \citet{abney1987}, a consensus has developed in syntactic literature 
that noun phrases in languages with articles, such as English, project an additional functional layer, that of the determiner phrase (aka DP). 
\citet{Longobardi1994} further argued that not all noun phrases in languages with articles (such as Italian) are DPs. 
He related the presence/absence of the DP projection to argumenthood: arguments, he claimed, are obligatorily DPs and non-arguments lack the DP. 

While the consensus emerged for languages with articles, there remains a question as to whether languages that lack articles have the DP projection as well. 
Although some of the earliest work on the DP hypothesis was concerned with article-less languages such as Turkish \citep{Kornfilt1984}, 
some researchers later claimed in article-less languages (particularly, in Slavic ones, such as Serbo-Cro\-a\-tian, but also in non-Slavic languages) all noun phrases, 
including those in argument positions, lack the DP projection (cf. \citealt{boskovic2008,Boskovic2009,boskovic2012,BoskovicSener2014,Despic2011}, inter alia). 
Yet, others have argued that even in languages that lack articles, at least some noun phrases project a DP (as originally proposed for Turkish by \citealt{Kornfilt1984}; 
see also \citealt{LyutikovaPereltsvaig2015} and \citealt{Kornfilt2018b,Kornfilt2018a} for Turkic languages in general; 
\citealt{Progovac1998} and \citealt{Pereltsvaig2001,Pereltsvaig2006,Pereltsvaig2007} for Russian; 
\citealt{VanHofwegen2013} and \citealt{GillonArmoskaite2015} for Lithuanian; 
\citealt{Norris2018} for Estonian; 
\citealt{Erschler2019b,Erschler2019a} for Digor and Iron Ossetic and Georgian; 
inter alia). 
Moreover, contrary to \citegen{Longobardi1994} claim, it has been shown that even argumental noun phrases can be smaller than DP, 
aka Small Nominals (\citealt{Pereltsvaig2006}, \citealt{KaganPereltsvaig2011}, inter alia). 

\begin{sloppypar}
In this paper, I adopt the latter view that argumental noun phrases in article-less languages come in two sizes: 
DPs and Small Nominals. 
(Following \citealt{Pereltsvaig2006}, the latter is construed as an umbrella term for a range of noun phrases of different structural sizes, 
such as QPs, NPs etc., all of which are smaller than DP.) 
In what follows, I argue that this approach offers a unified account for several empirical phenomena that have previously been treated separately, 
and in a variety of languages. 
I rely heavily on earlier research on individual languages -- both my own and that of other scholars -- but my main goal is to catalogue the properties that discriminate between DPs and Small Nominals across languages, 
with the aim of reaching a better understanding of what it is that the DP projection does. 
The bulk of the data comes from my own work on Russian and Tatar, with additional examples from published sources on Ossetic, Georgian, and Estonian. 
In Section~\ref{peresec:2}, I review five argumental positions for which a contrast between DPs and Small Nominals would offer an elegant account: 
subjects, objects, possessors, complements of attributivizers, and complements of adpositions. 
In Section~\ref{peresec:3}, I catalogue the contrasts between DPs and Small Nominals, such as their ability to take various DP-level elements; 
being selected by various heads; 
semantic interpretation; 
ability to serve as antecedents of anaphors, to control PRO, or to trigger predicate agreement; 
need for case and ability to move to a higher position. 
Section~\ref{peresec:4} offers an analysis of why these particular properties cluster together.
\end{sloppypar}

\section{DPs and Small Nominals in argument positions}
\label{peresec:2}

It is my view that both DPs and Small Nominals may be found in several argument positions. 
One such position is that of subject: 
in \citet{Pereltsvaig2006} it is argued that a ``size''-based analysis is preferable to a positional analysis 
in terms of explaining the contrast between quantified subjects in Russian that trigger predicate agreement and those that do not trigger agreement 
(with the predicate appearing in the singular neuter form, which is the morphological default). 
The two sentences have a slight difference in meaning, which is lost in the English translation, but to which I return below.

\ea%1
    \label{pereex:1}
    Russian \il{Russian}
    \ea{%1a
        \label{pereex:1a}
        \gll V ètom fil’me igrali    [pjat’ izvestnyx aktërov].\\
        in this   film    played.\textsc{pl}   five    famous    actors\\
        \glt `Five famous actors played in this film.'
    }
    \ex{%1b
        \label{pereex:1b}
        \gll V ètom fil’me igralo     [pjat’ izvestnyx aktërov].\\
        in this   film    played.\textsc{neut} five   famous     actors\\
        \glt `Five famous actors played in this film.'
    }
    \z 
\z 


Another argument position in which both DPs and Small Nominals may appear is the (direct) object position. 
For example, in Russian objects of verbs with the cumulative aspectual prefix \textit{na-} (and with or without the reflexive \textit{{}-sja}) 
have been argued to be Small Nominals, in contrast to objects of other verbs.

\ea%2
    \label{pereex:2}
    Russian \il{Russian}
    \ea{%2a
        \label{pereex:2a}
        \gll Ja na-žarila    [kotlet].\\
        I   \textsc{asp}{}-fried  meat.patties\\
        \glt `I fried up a lot of meat patties.'
    }
    \ex{%2b
        \label{pereex:2b}
        \gll Ja po-žarila    [kotlety].\\
        I   \textsc{asp}{}-fried  meat.patties\\
        \glt `I fried up meat patties.'
    }
    \z 
\z

Another language in which a ``size''-based analysis has been applied to noun phrases in the object position is Tatar: 
\citet{LyutikovaPereltsvaig2015} argued that the so-called differential object marking is best accounted for in terms of the size of the noun phrase itself. 
They proposed that case-marked objects are DPs while unmarked objects are Small Nominals.\pagebreak

\ea%3
    \label{pereex:3}
    Tatar \il{Tatar}
    \multicolsep=.25\baselineskip
    \begin{multicols}{2}\raggedcolumns
    \ea{%3a
        \label{pereex:3a}
        \gll Marat bala-nı       čakır-dı.\\
        Marat child-\textsc{acc} invite-\textsc{pst} \\
        \glt `Marat invited the child.'
    }
    \ex{%3b 
        \label{pereex:3b}
        \gll Marat bala   čakır-dı.\\
        Marat child invite-\textsc{pst}\\
        \glt `Marat invited a child.'
    }
    \z 
    \end{multicols}
\z

A similar analysis has been proposed for Lithuanian direct objects by \citet{GillonArmoskaite2015}: 
they argue that objects of imperfective verbs may be DPs or Small Nominals whereas objects of perfective verbs are necessarily DPs.

A third type of argument position in which both DPs and Small Nominals may be found is that of possessors inside the DP. 
In particular, \citet{LyutikovaPereltsvaig2015} analyze the two so-called ezafe constructions in Tatar: 
the ezafe-2, in which the possessor is not marked for case, 
and the ezafe-3, in which the possessor is marked with the genitive case. 
Otherwise, the two ezafe constructions look exactly the same; the `3' in the gloss stands for the 3\textsuperscript{rd} person possessive agreement suffix. 
(There also exists the so-called ezafe-1, which is not relevant to the topic at hand.) 
\citet{LyutikovaPereltsvaig2015} proposed that the possessors in the two ezafe constructions differ in size: 
the possessor in ezafe-3 is a DP, whereas the possessor in ezafe-2 is a Small Nominal. 

\ea%4
    \label{pereex:4}
    Tatar \il{Tatar}
    \ea{%4a
        \label{pereex:4a}
        \gll xatın-nıŋ   kijem-e\\
        woman-\textsc{gen} clothing-3\\
        \glt `a woman’s clothing' (= clothing belonging to a woman)
    }
    \ex{%4b
        \label{pereex:4b}
        \gll xatın     kijem-e\\
        woman clothing-3 \\
        \glt `woman's clothing' (= clothing meant for women)
    }
    \z 
\z 

In Tatar, yet another syntactic environment is said to host both DPs and Small Nominals: 
the complement of the so-called attributivizers, that is markers that turn a noun phrase into an attributive modifier. 
According to \citet{LyutikovaPereltsvaig2015}, the attributivizer \textit{{}-lı} takes a Small Nominal complement 
while the attributivizer \textit{{}-gı} takes a DP complement.

\ea%5
    \label{pereex:5}
    Tatar \il{Tatar}
    \multicolsep=.25\baselineskip
    \begin{multicols}{2}\raggedcolumns
    \ea{%5a
        \label{pereex:5a}
        \gll čäčäk-le   čaška\\
        flower-\textsc{attr}   cup\\
        \glt `a cup with a flower'
    }
    \ex{%5b
        \label{pereex:5b}
        \gll šähär-e-ndä-ge   uram-nar\\
        city-3-\textsc{loc}{}-\textsc{attr}   street-\textsc{pl} \\
        \glt `streets of a city'
    }
    \z\end{multicols}
\z 


Finally, adpositions may take Small Nominals as complements. 
Such an analysis has been proposed for the Russian preposition \textit{v} ‘into’ (with the meaning of change in social status) in \citet{MitreninaPereltsvaig2019}, 
in contrast to the preposition \textit{v} that does not have the “change of status” meaning.

\ea%6
    \label{pereex:6}
    Russian \il{Russian}
    \ea{%6a
        \label{pereex:6a}
        \gll Obama izbiralsja v  [prezidenty].\\
        Obama ran           into presidents\\
        \glt `Obama ran for president.'
    }
    \ex{%6b
        \label{pereex:6b}
        \gll Terrorist vystrelil v  [prezidentov].\\
        terrorist  shot        into presidents\\
        \glt `The terrorist shot at the presidents.'
    }
    \z 
\z 

Similar analysis has been proposed for the preposition \textit{ɐd} ‘with’ in Digor Ossetic by \citet{Erschler2019b,Erschler2019a} 
and for the postposition -\textit{ian} ‘with’ in Georgian by \citet{Erschler2019b}.

In what follows, I argue that these “size”-based analyses are indeed appropriate for the various abovementioned constructions 
because the nominals in them share certain properties that distinguish DPs from Small Nominals. 
Although due to independent factors only some of these contrasting properties obtain in each construction and each language, 
the overlaps are significant enough to suggest that a common explanation should be devised for all the constructions and all the languages under consideration. 
A “size”-based account, namely, that some noun phrases in each construction are DPs and others are Small Nominals, is exactly that kind of unifying explanation.

\section{Properties of DPs vs. Small Nominals}\largerpage
\label{peresec:3}

\subsection{Selection} %3.1.

As has been mentioned in the previous section, certain heads, such as certain adpositions or attributivizers, may select exclusively Small Nominals. 
In this respect, these selecting heads are parallel to those that select clausal complements of different sizes, for example, finite CPs vs. infinitive TPs, 
as noted in Susi Wurmbrand’s work. Moreover, just as some heads may select alternate complements of different types 
(e.g., \textit{know} selecting either a DP or a CP, as in: \textit{John knows the truth} vs.~\textit{John knows that Mary left}), 
some heads may select nominal complements of different sizes: 
either a DP or a Small Nominal would do. 
Presumably, this is the case with the head selecting the subject (i.e., external argument) in Russian; 
cf. examples in \REF{pereex:1} above.

Yet, just as some heads may select specifically for this or that complement type (e.g., \textit{John} \textit{ate} \textit{a} \textit{steak} vs. \textit{*John} \textit{ate} \textit{that} \textit{Mary} \textit{left}), some heads may select a nominal complement of a specific size. In particular, some heads select specifically a Small Nominal complement. In addition to the abovementioned adpositions and attributivizers, such heads include the Russian aspectual prefix \textit{na-}, which appears in Asp°. As we shall see below, all these Small Nominals exhibit a certain cluster of properties that concern their internal structure, their meaning, and their external distribution.

\subsection{Internal structure}  %3.2.

The one property shared by Small Nominals across languages and constructions that most clearly points to a “size”-based distinction is the impossibility of including various DP-level elements (i.e., elements that are argued to be in D° or Spec-DP in languages with articles), such as pronouns, proper names, demonstratives, strong quantifiers, and (some) possessors. The examples below illustrate the prohibition against demonstratives in the various syntactic environments which involve Small Nominals: non-agreeing subjects in Russian, unmarked direct objects in Tatar, objects of verbs with the cumulative \textit{na-} in Russian, possessors in ezafe-2 in Tatar, complements of the attributivizer \textit{{}-lı} in Tatar, and complements of the adpositions in Russian, Digor and Iron Ossetic, and Georgian, which are mentioned above as selecting exclusively Small Nominal complements.\footnote{A reviewer suggests that examples parallel to \REF{pereex:6} are grammatical in Serbo-Croatian. If so, the status of the demonstrative must be different in that language: if it is not a DP-level element, the demonstrative would be compatible with the lack of syntactic agreement (i.e. default morphological agreement) on the predicate. Contrasts between Serbo-Croatian and Russian with respect to the left-branching extraction further suggest that demonstratives in the two languages may have a different status.} As can be seen from the examples below, all these noun phrases disallow demonstratives; in a similar way, they disallow pronouns, proper names, and universal quantifiers, all of which can be construed as hosted in the DP projection. In contrast, quantified subjects triggering agreement and objects of verbs with other aspectual prefixes in Russian, accusative-marked objects and complements of the attributivizer \textit{{}-gı} in Tatar, as well as complements of other markers (suffixes or adpositions) in Digor and Iron Ossetic and Georgian, all select DPs and allow demonstratives, pronouns, proper names, and universal quantifiers.

\settowidth\jamwidth{Russian}

\ea[*]{ %6, erroneiously
\label{pereex:6new}
    \gll V ètom fil’me igralo      èti   pjat’   izvestnyx aktërov.\\
    in this   film    played.\textsc{neut} these five   famous     actors\\
    \jambox*{Russian\il{Russian}}
    \glt Intended: `These five famous actors played in this film.'
}
\ex[*]{%7
\label{pereex:7}
    \gll Marat bu   bala     čakır-dı.\\
    Marat this      child   invite-\textsc{pst}   \\
    \jambox*{Tatar\il{Tatar}}
    \glt Intended: `Marat invited this child.' 
}
\ex[*]{%8
    \label{pereex:8}
    \gll Ja na-žarila    ètix       kotlet.\\
    I   \textsc{asp}{}-fried  these.textsc{gen.pl}   meat.patties.textsc{gen.pl}    \\
    \jambox*{Russian\il{Russian}}
    \glt Intended: `I fried up this whole lot of meat patties.'
}
\ex[*]{%9
    \label{pereex:9}
    \gll bu   čäčäk-le   čaška\\
    this   flower-\textsc{attr}   cup    \\
    \jambox*{Tatar\il{Tatar}}
    \glt intended: `a cup with this flower' 
}
\ex[*]{%10
    \label{pereex:10}
    \gll Obama izbiralsja v    èti   prezidenty.\\
    Obama ran           into   these   presidents    \\
    \jambox*{Russian\il{Russian}}
    \glt Intended: `Obama ran for this (role of) president.'
}
\settowidth\jamwidth{Iron Ossetic}
\ex[*]{%11
    \label{pereex:11}
    \gll ɐd   asǝ   bel\\
    with   this   spade    \\
    \jambox*{Iron Ossetic\il{Iron Ossetic}\il{Ossetic}} % TODO: verify that Ossetic is a hyperonym for Iron Ossetic
    \glt Intended: `with this spade' \citep{Erschler2019b}
}
\ex[*]{%12
    \label{pereex:12}
    \gll am-ʃav-ʣaɣl-ian-i     marxil-i\\
    that-black-dog-with-\textsc{nom}   sledge-\textsc{nom}     \\
    \jambox*{Georgian\il{Georgian}}
    \glt Intended: `a sledge with that black dog' \citep{Erschler2019b}
}
\z

In the following subsections, we shall see that internal structure of DPs vs. Small Nominals 
(that is, their ability to accommodate DP-level elements or the lack thereof) 
correlates with the meaning of these nominals, as well as with their “external” properties, such as their need for case and their ability to move. 
(Due to space limitations, I will not illustrate all properties for all constructions; 
the reader is referred to \citealt{Pereltsvaig2006} and \citealt{LyutikovaPereltsvaig2015} for additional illustrative examples.)

\subsection{Semantic interpretation and referentiality-related properties}  %3.3.

So far, we have seen that noun phrases can be divided into two clusters: 
structurally larger DPs, which allow for such elements as demonstratives, pronouns, strong quantifiers etc., 
and Small Nominals, which have no room for these elements. 
Moreover, noun phrases of different sizes (i.e., DPs and Small Nominals) may be specifically selected for as such. 
In this section, we shall see that both the semantic interpretation and several referentially-related “external” properties of a given noun phrase 
also depend on the presence/absence of the DP.

First, as was noted in \citet{Pereltsvaig2006} and \citet{LyutikovaPereltsvaig2015}, 
Small Nominals have a different semantic interpretation from their DP counterparts. 
Though the exact descriptions of these differences vary from construction to construction and from language to language, 
the generalization is that DPs have a referential interpretation (i.e., denote individuals) 
whereas Small Nominals are associated with non-referential interpretations. 
For example, in the Russian examples in \REF{pereex:1} above, the DP quantified subject in \REF{pereex:1a} denotes a plural individual 
whereas the Small Nominal quantified subject in \REF{pereex:1b} has a non-referential, group interpretation. 
Likewise, in the Tatar examples in \REF{pereex:3}, the DP object in \REF{pereex:3a} has a definite interpretation (even in the absence of a demonstrative), 
whereas the Small Nominal object in \REF{pereex:3b} has an indefinite interpretation. 
Similar interpretations are associated with Small Nominals in other constructions mentioned above: 
they have indefinite or non-referential interpretations.

\citet{Pereltsvaig2001} proposed an analysis that would shed light on this contrast in the semantic interpretation of DPs and Small Nominals: 
in DPs, the D\textsuperscript{0} introduces a referential index, which is construed as a set of phi-features responsible for the DP’s referential interpretation. 
In Small Nominals, no referential index is included, thus the only interpretation is a non-referential one. 
This analysis also allows us to tie together several other contrasts between DPs and Small Nominals:

\begin{itemize}
\item DPs can serve as antecedents of anaphors while Small Nominals cannot;
\item DPs can control PRO while Small Nominals cannot;
\item DPs can trigger predicate agreement while Small Nominals cannot.
\end{itemize}

We have already seen the third of these contrasts in the previous subsection 
(in fact, the presence of predicate agreement was treated as a mark of Small Nominal subjects in Russian). 
As for the other two contrasts, they too are most evident subject noun phrases (e.g., in Russian), for independent reasons.\largerpage

\ea%13
    \label{pereex:13}
    Russian\footnote{The neuter form in \REF{pereex:13b} receives variable judgments from speakers but the majority of speakers reject it.} 
    \il{Russian}
    \ea{%13a
        \label{pereex:13a}
        \gll Dva izvestnyx aktëra \{igrali/*igralo\}                sebja.\\
        two  famous    actors \{played.\textsc{pl}/played.\textsc{neut}\} self\\
        \glt `Two famous actors played themselves.'
    }
    \ex{ %13b
        \label{pereex:13b}
        \gll Pjat’ banditov \{pytalis’  /*pytalos’\}[PRO ubit’ Bonda].\\
        five  thugs        tried.\textsc{pl} /*tried.\textsc{neut}          to.kill Bond\\
        \glt `Five thugs tried to kill Bond.'
    }
    \z 
\z 

The reason for these contrasts can also be tied to the referential index: 
both anaphor binding and control involve co-indexation, that is matching of referential indices. 
If a given noun phrase has no referential index (which is the case for Small Nominals), it cannot be involved in anaphor binding or control.

\subsection{Case}  %3.4. /

Another contrast between DPs and Small Nominals involves case marking. 
Due to the fusional morphology in Russian, where the same morpheme marks the case as well as declension/gender and number of the noun, 
noun phrases in Russian cannot appear without any morphological case at all. 
Therefore, to see this contrast between DPs and Small Nominals most clearly we must turn to a different language: Tatar. 
As mentioned above, direct objects in that language are marked for case if they are DPs while Small Nominal objects are not marked for case. 
The same is true of the DP/Small Nominal contrast in other positions as well. 
Thus, in Tatar possessors in the ezafe-3 construction, which allow for demonstratives and other DP-level elements, 
are marked with the genitive suffix -\textit{nıŋ}, 
while possessors in the ezafe-2, which do not allow demonstratives and other DP-level elements, are not marked with the genitive suffix.

\ea%14
    \label{pereex:14}
    Tatar \il{Tatar}
    \ea{ 
        \label{pereex:14a}
        \gll (bu) xatın-nıŋ   kijem-e\\
        this  woman-\textsc{gen}   clothing-3 \\
        \glt `this woman’s clothing'
    }
    \ex{ 
        \label{pereex:14b}
        \gll (*bu) xatın   kijem-e\\
        this  woman   clothing-3\\
        \glt intended: `this woman’s clothing'
    }
    \z 
\z 

A similar pattern obtains with respect to attributivizers in Tatar: 
the complement of the attributivizer \textit{{}-gı}, which is, as mentioned above, a DP, is marked for case 
(specifically, it bears the locative case suffix \textit{\nobreakdash-dä}), 
whereas the complement of the attributivizer \textit{\nobreakdash-lı}, which is a Small Nominal, is not marked for case.

\ea%15
    \label{pereex:15}
    Tatar \il{Tatar}
    \multicolsep=.25\baselineskip
    \begin{multicols}{2}
    \ea{ 
        \gll šähär-dä-ge     uram-nar\\
        city-\textsc{loc-attr}   street-\textsc{pl}\\
        \glt `the city’s streets' 
    }
    \ex{
        \gll  čäčäk-le   čaška\\
        flower-\textsc{attr} cup\\
        \glt `a cup with a flower'
    }
    \z 
    \end{multicols}
\z 

The same pattern is also seen in Digor Ossetic, where the preposition \textit{ɐd} ‘with’ takes a case-less Small Nominal, 
whereas the preposition \textit{ɐnɐ} ‘without’ takes a case-marked DP \citep{Erschler2019a}. 

\ea%16
    \label{pereex:16}
    Iron Ossetic \il{Iron Ossetic}
        \il{Ossetic} % TODO
    \ea{
        \gll ɐd   ʃtǝr   bel-${\emptyset}$\\
        with   big   spade \\
        \glt ‘with a big spade’
    }
    \ex{
        \gll ɐnɐ     ʃtǝr   bel-ɐj\\
        without   big   spade-\textsc{abl} \\
        \glt ‘without \{a/the\} big spade’
    }
    \z
\z 
         
Thus, I follow \citet{Danon2006} in that only DPs and not Small Nominals are subject to case filter.

\subsection{Movement}  %3.5. /

The last contrast between DPs and Small Nominals to be considered here concerns the nominal’s ability to move. Again, for independent reasons, it is most evident in Tatar, where both direct objects and possessors may move only if they are DPs but not if they are Small Nominals. First, let’s consider direct objects: only those direct objects that are DPs (i.e., allow DP-level elements, are marked for case, etc.) may move to a position outside the VP (i.e., to the left of the VP-adverbs such as \textit{tiz} ‘quickly’). Objects that are Small Nominals (i.e., do not allow DP-level elements, are not marked for case, etc.) cannot appear to the left of the adverb such as \textit{tiz} ‘quickly’. Small Nominal objects must appear to the right of the adverb.

\ea%17
    \label{pereex:17}
    Tatar \il{Tatar}
    \ea[]{
        \gll Marat (bu) botka-nı   tiz     aša-dı.\\
        Marat this porridge-\textsc{acc}   quickly   eat-\textsc{pst} \\
        \glt `Marat ate \{this/the\} porridge quickly.'
    }
    \ex[*]{
        \gll Marat (*bu)   botka     tiz     aša-dı.\\
        Marat this   porridge   quickly   eat-\textsc{pst}\\
        \glt intended: `Marat ate \{this/some\} porridge quickly.'
    }
    \ex[]{
        \gll Marat tiz     (*bu)   botka     aša-dı.\\
        Marat quickly   this   porridge   eat-\textsc{pst}\\
        \glt `Marat ate some porridge quickly.' \\ \textit{not}: `Marat ate this porridge quickly.'
    }
    \z 
\z 

Similarly, possessors in the ezafe-3 construction, which are DPs, appear to the left of a modifying adjective, whereas possessors in the ezafe-2 construction, which are Small Nominals, appear to the right of the adjective.

\ea%18
    \label{pereex:18}
    Tatar \il{Tatar}
    \ea{
        \gll (*kük) (bu)   bala-lar-nıŋ   (kük)   itek-lär-e\\
        blue   this   child-\textsc{pl-gen} blue   boot-\textsc{pl}{}-3 \\
        \glt `\{these/the\} children’s blue boots'
    }
    \ex{
        \gll (kük)   bala-lar   (*kük) itek-lär-e\\
        blue   child-\textsc{pl}   blue   boot-\textsc{pl}{}-3\\
        \glt `blue children’s boots' 
    }
    \z
\z

The same contrast between possessors/genitives that appear above adjectives and those that appear below adjectives is evident also in Estonian, 
as noted in \citet{Norris2018}.

\ea%19
    \label{pereex:19}
    \gll emis-te   päevane   proteiini   tarbi-mine\\
    sow-\textsc{pl.gen}   diurnal.\textsc{nom}   protein.\textsc{gen}   consume-\textsc{nmlz.nom} \\
    \jambox{    Estonian \il{Estonian}    }
    \glt `the sows’ diurnal consumption of protein'
\z

In other words, DP objects and possessors may move (and in Tatar possessors must move) to the left of an adverb or an adjective, 
whereas Small Nominal objects and possessors must stay low.\largerpage

Another consequence of the Small Nominals’ inability to move as freely as DPs do is that only DPs can scope over other quantified expressions, 
whereas scopal possibilities of Small Nominals are restricted. 
This is illustrated below with quantified subjects in Russian: 
non-agreeing subjects, which are Small Nominals, can take only the narrow scope 
whereas agreeing subjects, which are DPs, can take either wide or narrow scope:

\ea%20
    \label{pereex:20}
    Russian \il{Russian}
    \ea{
        \gll Každyj raz   [pjat’ xirurgov] operirovalo     Bonda.\\
        every    time five   surgeons operated.\textsc{neut} Bond\\
        \glt `Every time five surgeons operated on Bond.' (unambiguous: ${\forall} > 5$)
    }
    \ex{
        \gll Každyj raz [pjat’ xirurgov] operirovali  Bonda.\\
        every   time five  surgeons  operated.\textsc{pl} Bond\\
        \glt `Every time five surgeons operated on Bond.' (ambiguous: ${\forall} > 5$ or $5 > {\forall}$)%
        \footnote{
            Speakers exhibit preferences for one or the other reading 
            (sometimes a very strong preference or even impossibility of $5 > {\forall}$ reading) 
            but this depends heavily on the word order and the ability of a particular speaker to get a non-linear scope reading.
        }
    }
    \z 
\z

The same is true of DOM in Tatar: 
unmarked objects -- unlike their accusative counterparts -- cannot take wide scope with respect to another quantified expression.

\ea%21
    \label{pereex:21}
    Tatar \il{Tatar}
    \ea{
        \gll Här   ukučı     ike  kitap ukı-dı. \\
        every student   two book read-\textsc{pst}\\
        \glt 
            ${\forall} > 2$: `For every student, there are two books that (s)he read.' \\
            *$2 > {\forall}$: `There are (certain) two books that every student read.'
    }
    \ex{
        \gll Här   ukučı     ike   kitap-nı   ukı-dı.\\
        every   student   two   book-\textsc{acc}   read-\textsc{pst}\\
        \glt 
            ${\forall} > 2$: `For every student, there are two books that (s)he read.' \\
            $2 > {\forall}$: `There are (certain) two books that every student read.'
    }
    \z 
\z 

A similar, though not exactly the same, pattern obtains in Lithuanian (\citealt{GillonArmoskaite2015}), 
except that objects of perfective verbs, which are DPs, must (rather than may) take wide scope with respect to other quantified expressions, 
whereas objects of imperfective verbs, which may be DPs or Small Nominals, can take either wide or narrow scope.

\ea%22
    \label{pereex:22}
    Lithuanian \il{Lithuanian} \citep[83]{GillonArmoskaite2015}
    \ea{
        \gll Jonas      ne-pa-suko       vairo. \\
        John.\textsc{nom.sg} \textsc{neg-pref}{}-turn.\textsc{pst.3sg}   wheel.\textsc{gen.sg}\\
        \glt `John did not turn the wheel.' \\
        \textit{not}: `John did not turn any wheel.'
    }
    \ex{
        \gll Jonas      ne-suko     vairo.\\
        John.\textsc{nom.sg} \textsc{neg}{}-turn.\textsc{pst.3sg}   wheel.\textsc{gen.sg}\\
        \glt `John did not turn the wheel.' \\
        `John did not turn any wheel.'
    }
    \z 
\z 

The differing scopal possibilities of DPs and Small Nominals are accounted for as follows: 
I assume that reverse scope obtains via movement at LF, hence only noun phrases that can move can have such non-surface scope.

\section{Proposal}
\label{peresec:4}

In this brief concluding section, I recap the proposed account of 
why these specific properties, discussed in the previous section, characterize Small Nominals across constructions and across languages. 
As alluded to above, these properties derive from the absence of a referential index, 
which I take to be introduced into a derivation by D\textsuperscript{0} (contrary to \citealt{DechaineWiltschko2002}, 
who argued for the phiP that is situated lower than D). 
In semantics, the lack of a referential reading translates into a non-referential reading. 
Moreover, the lack of a referential reading is also responsible for the inability of the nominal 
to enter into relations that involve matching referential indices, such as control (which involves matching of referential indices between a controller and a PRO), 
anaphor binding (matching of referential indices between an antecedent and an anaphor) 
or agreement (matching of referential indices between the controller of agreement and the target of agreement). 
Here, I understand a referential index as a sum total of phi-features. 

A referential index is also what makes a nominal “visible” to a Probe, thus allowing it to move. 
Moreover, a referential index is what makes a nominal subject to case filter. 
In other words, it is the D rather than the N that is in need of case. 
Furthermore, heads can select for either a D or an N (or allow for either projection as a complement). 

To recap, I side with the view that treats Ns as category- or kind-denoting, 
whereas Ds are taken to introduce reference to individuals/entities in the form of a referential index, 
which is a combination of phi-features. 
I take this division of labor between N and D to be applicable not only in languages with articles but in article-less languages as well. 
At LF, there is no difference between languages with and without article in this respect; 
the difference is purely in the morphosyntactic expression of the D°.

{\sloppy\printbibliography[heading=subbibliography,notkeyword=this]}

\end{document} 
