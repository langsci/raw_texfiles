\phantomsection
\addcontentsline{toc}{chapter}{Lösungen zu den Übungen}
\chapter*{Lösungen zu den Übungen}
\markboth{Lösungen zu den Übungen}{Lösungen zu den Übungen}

\label{sec:loesungen}

\Loesungen[]{sec:phonetik}

\Loesung{u31}

\begin{enumerate}\Lf
  \item Jubel
  \item Zahnarzt
  \item Unterweisung
  \item Chor
  \item Liebesbeweis
  \item Ehebruch
  \item Schlichter
  \item Klüngel
  \item Rumpelstilzchen
  \item Bache
  \item Sieb
  \item Glaubenskrieg
  \item bösartig
  \item Sehnsüchte
  \item versonnen
  \item Gürtel
\end{enumerate}

\Loesung{u32}

\begin{enumerate}\Lf
  \item \textipa{[P\t{aO}fg@t\t{aO}t]}
  \item \textipa{[Ko:d@ln]}
  \item \textipa{[ta:k]} 
  \item \textipa{[PUmtKi:bI\c{c}]} 
  \item \textipa{[ve:z@n]} 
  \item \textipa{[Panze:@n]}
  \item \textipa{[ve:nI\c{c}]} (\textipa{[ve:nIk]} ist dialektal)
  \item \textipa{[ky:l]} 
  \item \textipa{[f5P\t{aE}n]}
  \item \textipa{[Spy:l@]}
  \item \textipa{[tIS]}
  \item \textipa{[ve:@n]} 
  \item \textipa{[PI\c{c}]}
  \item \textipa{[le:K@]}\\
    (\textipa{[le:K5]} entspricht \textit{Lehrer})
  \item \textipa{[kV\t{a@}k]} 
\end{enumerate}

\Loesung{u33}

\begin{enumerate}\Lf
  \item \textipa{[PUnt5SlU\t{pf}]} 
  \item \textipa{[ni:z@n]}
  \item \textipa{[vIs@n]}
  \item \textipa{[zaXf5halt]}
  \item \textipa{[defini\t{ts}Jo:n]}
  \item \textipa{[f5P\t{aE}nsh\t{aO}s]}
  \item \textipa{[kl\t{aE}nI\c{c}k\t{aE}t]}
  \item \textipa{[za:n@t\t{O@}t@]}
  \item \textipa{[hu:st@nzaft]}
  \item \textipa{[Po:n@]}
  \item \textipa{[b@StImUN]}
  \item \textipa{[tu:X]}
  \item \textipa{[SUps@n]}
  \item \textipa{[b\t{E@}\c{c}@n]}
  \item \textipa{[lo:ppK\t{aE}zUN]}
\end{enumerate}

\Loesungen{sec:phonologie}

\Loesung{u41}

Hier wird jeweils nur ein Beispiel angegeben, auch wenn teilweise sehr viel mehr Minimalpaare existieren.

\begin{enumerate}\Lf
  \item Tank, Dank
  \item Bass, Bann
  \item wann, Mann
  \item Bach, bang
  \item Rang, Hang
  \item Schluss, Schluck
  \item Napf, nass
  \item Leine, Laune
  \item Kieme, Kimme
\end{enumerate}

\Loesung{u42}

Vgl.\ Abschnitt~\ref{sec:umablaut}, S.~\pageref{sec:umablaut}.

\Loesung{u43}

\begin{enumerate}\Lf
  \item In Wörtern wie \textit{Chemie} oder \textit{Chuzpe} kommen \textipa{[\c{c}]} bzw. \textipa{[X]} (also zugrundeliegendes Segment /\textipa{\c{c}}/) im Anlaut der ersten Silbe des Worts vor.
    Im Kernwortschatz ist dies keine Umgebung, in der diese Segmente vorkommen.
    Insofern kann man sagen, dass Realisierungen wie \textipa{[XU\t{ts}p@]} und \textipa{[\c{c}emi:]} im Vergleich zu den meisten Wörtern phonologisch auffällig sind.
  \item Realisierungen wie /\textipa{Semi:}/ und /\textipa{kemi:}/ setzen ganz andere zugrundeliegende Segmente an und sind bezüglich ihrer Phonotaktik unauffällig, da /\textipa{S}/ und /\textipa{k}/ auch sonst im Anlaut vorkommen.
    Es wird also im Lehnwort ein zugrundeliegendes Segment substituiert, und das Wort wird damit phonologisch systemkonformer.
  \item Dass systemkonforme phonologische Varianten von \textit{Chuzpe} nicht existieren, könnte darauf hindeuten, dass das Wort im allgemeinen seltener, stilistisch eingeschränkter oder schriftsprachlicher ist und damit nicht so gerne den Regularitäten des Kernwortschatzes untergeordnet wird.
    Zumindest spekulativ wären sonst früher oder später auch /\textipa{SU\t{ts}p@}/ und /\textipa{kU\t{ts}p@}/ zu erwarten.
\end{enumerate}

\Loesung{u44}

Bei \textit{Fenster} handelt es sich um einen Zweifelsfall.
Es kann vor oder nach dem \textipa{[s]} silbifiziert werden.

\begin{enumerate}

  \item 
  \begin{tabular}{ccccccc}
    V & \rnode{EX00V1}{} & \rnode{EX00V2}{} &   & \rnode{EX00V4}{\textipa{U}} & \rnode{EX00V5}{} & \rnode{EX00V6}{} \\
    L & \rnode{EX00A1}{} & \rnode{EX00A2}{} & \rnode{EX00F3}{\textipa{K}} & \rnode{EX00A4}{} & \rnode{EX00A5}{} & \rnode{EX00A6}{} \\
    N & \rnode{EX00N1}{} & \rnode{EX00N2}{} & \rnode{EX00N3}{} & \rnode{EX00N4}{} & \rnode{EX00N5}{\textipa{m}} & \rnode{EX00N6}{} \\
    F & \rnode{EX00F1}{\textipa{S}} & \rnode{EX00F2}{} & & \rnode{EX00F4}{} & \rnode{EX00F5}{} & \rnode{EX00F6}{} \\
    P & \rnode{EX00P1}{} & \rnode{EX00P2}{\textipa{t}} & \rnode{EX00P3}{} & \rnode{EX00P4}{} & \rnode{EX00P5}{} & \rnode{EX00P6}{\textipa{\t{pf}}} \\
  \end{tabular}
  \ncline[nodesep=3pt]{->}{EX00F1}{EX00P2}
  \ncline[nodesep=3pt]{->}{EX00P2}{EX00F3}
  \ncline[nodesep=3pt]{->}{EX00F3}{EX00V4}
  \ncline[nodesep=3pt]{->}{EX00V4}{EX00N5}
  \ncline[nodesep=3pt]{->}{EX00N5}{EX00P6}

  \item
  \begin{tabular}{ccccccccc}
    V & \rnode{EX01V1}{} &  & \rnode{EX01V3}{\textipa{I}} & \rnode{EX01V4}{} & \rnode{EX01V5}{} & \rnode{EX01V6}{\textipa{@}} & \rnode{EX01V7}{} \\
    L & \rnode{EX01A1}{} & \rnode{EX01F2}{\textipa{K}} & \rnode{EX01A3}{} & \rnode{EX01A4}{} & \rnode{EX01A5}{} & \rnode{EX01A6}{} & \rnode{EX01A7}{} \\
    N & \rnode{EX01N1}{} & \rnode{EX01N2}{} & \rnode{EX01N3}{} & \rnode{EX01N4}{\circlearound{\textipa{N}}} & \rnode{EX01N5}{} & \rnode{EX01N6}{} & \rnode{EX01N7}{\textipa{n}} \\
    F & \rnode{EX01F1}{\textipa{v}} &  & \rnode{EX01F3}{} & \rnode{EX01F4}{} & \rnode{EX01F5}{} & \rnode{EX01F6}{} & \rnode{EX01F7}{} \\
    P & \rnode{EX01P1}{} & \rnode{EX01P2}{} & \rnode{EX01P3}{} & \rnode{EX01P4}{} & \rnode{EX01P5}{} & \rnode{EX01P6}{} & \rnode{EX01P7}{} \\
  \end{tabular}
  \ncline[nodesep=3pt]{->}{EX01F1}{EX01F2}
  \ncline[nodesep=3pt]{->}{EX01F2}{EX01V3}
  \ncline[nodesep=3pt]{->}{EX01V3}{EX01N4}
  \ncline[nodesep=3pt]{->}{EX01N4}{EX01V6}
  \ncline[nodesep=3pt]{->}{EX01V6}{EX01N7}

  \item
  \begin{tabular}{ccccccccc}
    V & \rnode{EX02V1}{} & \rnode{EX02V2}{\textipa{I}} & \rnode{EX02V3}{} & \rnode{EX02V4}{} & \rnode{EX02V5}{} & \rnode{EX02V6}{} & \rnode{EX02V7}{\textipa{@}} \\
    L & \rnode{EX02A1}{} & \rnode{EX02A2}{} & \rnode{EX02A3}{} & \rnode{EX02A4}{} & \rnode{EX02A5}{} & \rnode{EX02A6}{} & \rnode{EX02A7}{} \\
    N & \rnode{EX02N1}{} & \rnode{EX02N2}{} & \rnode{EX02N3}{\textipa{N}} & \rnode{EX02N4}{} & \rnode{EX02N5}{} & \rnode{EX02N6}{} & \rnode{EX02N7}{} \\
    F & \rnode{EX02F1}{\textipa{v}} & \rnode{EX02F2}{} & \rnode{EX02F3}{} & \rnode{EX02F4}{} & \rnode{EX02F5}{} & \rnode{EX02F6}{} & \rnode{EX02F7}{} \\
    P & \rnode{EX02P1}{} & \rnode{EX02P2}{} & \rnode{EX02P3}{} & \rnode{EX02P4}{\textipa{k}} & \rnode{EX02P5}{} & \rnode{EX02P6}{\textipa{t}} & \rnode{EX02P7}{} \\
  \end{tabular}
  \ncline[nodesep=3pt]{->}{EX02F1}{EX02V2}
  \ncline[nodesep=3pt]{->}{EX02V2}{EX02N3}
  \ncline[nodesep=3pt]{->}{EX02N3}{EX02P4}
  \ncline[nodesep=3pt]{->}{EX02P6}{EX02V7}

  \item
  \begin{tabular}{cccccccccccc}
    V & \rnode{EX03V1}{} & \rnode{EX03V2}{} & \rnode{EX03V3}{\textipa{\t{a@}}} & \rnode{EX03V4}{} & \rnode{EX03V5}{} & \rnode{EX03V6}{} & \rnode{EX03V7}{} & \rnode{EX03V8}{\textipa{\t{aE}}} & \rnode{EX03V9}{} & \rnode{EX03V10}{} & \rnode{EX03V11}{\textipa{@}} \\
    L & \rnode{EX03A1}{} & \rnode{EX03A2}{} & \rnode{EX03A3}{} & \rnode{EX03A4}{} & \rnode{EX03A5}{} & \rnode{EX03A6}{} & \rnode{EX03A7}{} & \rnode{EX03A8}{} & \rnode{EX03A9}{} & \rnode{EX03A10}{} & \rnode{EX03A11}{} \\
    N & \rnode{EX03N1}{} & \rnode{EX03N2}{} & \rnode{EX03N3}{} & \rnode{EX03N4}{} & \rnode{EX03N5}{} & \rnode{EX03N6}{} & \rnode{EX03N7}{} & \rnode{EX03N8}{} & \rnode{EX03N9}{} & \rnode{EX03N10}{} & \rnode{EX03N11}{} \\
    F & \rnode{EX03F1}{} & \rnode{EX03F2}{\textipa{v} } & \rnode{EX03F3}{} & \rnode{EX03F4}{} & \rnode{EX03F5}{} & \rnode{EX03F6}{\textipa{S}} & \rnode{EX03F7}{} & \rnode{EX03F8}{} & \rnode{EX03F9}{} & \rnode{EX03F10}{\textipa{z}} & \rnode{EX03F11}{} \\
    P & \rnode{EX03P1}{\textipa{k}} & \rnode{EX03P2}{} & \rnode{EX03P3}{} & \rnode{EX03P4}{\textipa{k}} & \rnode{EX03P5}{} & \rnode{EX03P6}{} & \rnode{EX03P7}{\textipa{p}} & \rnode{EX03P8}{} & \rnode{EX03P9}{} & \rnode{EX03P10}{} & \rnode{EX03P11}{} \\
  \end{tabular}
  \ncline[nodesep=3pt]{->}{EX03P1}{EX03F2}
  \ncline[nodesep=3pt]{->}{EX03F2}{EX03V3}
  \ncline[nodesep=3pt]{->}{EX03V3}{EX03P4}
  \ncline[nodesep=3pt]{->}{EX03F6}{EX03P7}
  \ncline[nodesep=3pt]{->}{EX03P7}{EX03V8}
  \ncline[nodesep=3pt]{->}{EX03F10}{EX03V11}

  \item
  \begin{tabular}{ccccccc}
    V & \rnode{EX04V1}{} & \rnode{EX04V2}{\textipa{e:}} & \rnode{EX04V3}{} & \rnode{EX04V4}{} & \rnode{EX04V5}{\textipa{5}} & \rnode{EX04V6}{} \\
    L & \rnode{EX04A1}{\textipa{l}} & \rnode{EX04A2}{} & \rnode{EX04A3}{} & \rnode{EX04A4}{} & \rnode{EX04A5}{} & \rnode{EX04A6}{} \\
    N & \rnode{EX04N1}{} & \rnode{EX04N2}{} & \rnode{EX04N3}{} & \rnode{EX04N4}{} & \rnode{EX04N5}{} & \rnode{EX04N6}{} \\
    F & \rnode{EX04F1}{} & \rnode{EX04F2}{} & \rnode{EX04F3}{} & \rnode{EX04F4}{\textipa{z}} & \rnode{EX04F5}{} & \rnode{EX04F6}{} \\
    P & \rnode{EX04P1}{} & \rnode{EX04P2}{} & \rnode{EX04P3}{} & \rnode{EX04P4}{} & \rnode{EX04P5}{} & \rnode{EX04P6}{} \\
  \end{tabular}
  \ncline[nodesep=3pt]{->}{EX04A1}{EX04V2}
  \ncline[nodesep=3pt]{->}{EX04F4}{EX04V5}

  \item
  \begin{tabular}{cccccccccc}
    V & \rnode{EX05V1}{} & \rnode{EX05V2}{\textipa{e:}} & \rnode{EX05V3}{} & \rnode{EX05V4}{} & \rnode{EX05V5}{\textipa{@}} & \rnode{EX05V6}{} & \rnode{EX05V7}{} & \rnode{EX05V8}{\textipa{I}} & \rnode{EX05V9}{} \\
    L & \rnode{EX05A1}{\textipa{l}} & \rnode{EX05A2}{} & \rnode{EX05A3}{} & \rnode{EX05A4}{} & \rnode{EX05A5}{} & \rnode{EX05A6}{} & \rnode{EX05F7}{\textipa{K}} & \rnode{EX05A8}{} & \rnode{EX05A9}{} \\
    N & \rnode{EX05N1}{} & \rnode{EX05N2}{} & \rnode{EX05N3}{} & \rnode{EX05N4}{} & \rnode{EX05N5}{} & \rnode{EX05N6}{} & \rnode{EX05N7}{} & \rnode{EX05N8}{} & \rnode{EX05N9}{\textipa{n}} \\
    F & \rnode{EX05F1}{} & \rnode{EX05F2}{} & \rnode{EX05F3}{} & \rnode{EX05F4}{\textipa{z}} & \rnode{EX05F5}{} & \rnode{EX05F6}{} &  & \rnode{EX05F8}{} & \rnode{EX05F9}{} \\
    P & \rnode{EX05P1}{} & \rnode{EX05P2}{} & \rnode{EX05P3}{} & \rnode{EX05P4}{} & \rnode{EX05P5}{} & \rnode{EX05P6}{} & \rnode{EX05P7}{} & \rnode{EX05P8}{} & \rnode{EX05P9}{} \\
  \end{tabular}
  \ncline[nodesep=3pt]{->}{EX05A1}{EX05V2}
  \ncline[nodesep=3pt]{->}{EX05F4}{EX05V5}
  \ncline[nodesep=3pt]{->}{EX05F7}{EX05V8}
  \ncline[nodesep=3pt]{->}{EX05V8}{EX05N9}

  \item
  \begin{tabular}{ccccccccccc}
    V & \rnode{EX06V1}{} & \rnode{EX06V2}{\textipa{u:}} & \rnode{EX06V3}{} & \rnode{EX06V4}{} & \rnode{EX06V5}{\textipa{E}} & \rnode{EX06V6}{} & \rnode{EX06V7}{} & \rnode{EX06V8}{} & \rnode{EX06V9}{\textipa{I}} & \rnode{EX06V10}{} \\
    L & \rnode{EX06A1}{} & \rnode{EX06A2}{} & \rnode{EX06A3}{} & \rnode{EX06A4}{} & \rnode{EX06A5}{} & \rnode{EX06A6}{} & \rnode{EX06A7}{} & \rnode{EX06A8}{\textipa{l}} & \rnode{EX06A9}{} & \rnode{EX06A10}{} \\
    N & \rnode{EX06N1}{} & \rnode{EX06N2}{} & \rnode{EX06N3}{} & \rnode{EX06N4}{} & \rnode{EX06N5}{} & \rnode{EX06N6}{} & \rnode{EX06N7}{} & \rnode{EX06N8}{} & \rnode{EX06N9}{} & \rnode{EX06N10}{} \\
    F & \rnode{EX06F1}{} & \rnode{EX06F2}{} & \rnode{EX06F3}{} & \rnode{EX06F4}{\textipa{z}} & \rnode{EX06F5}{} & \rnode{EX06F6}{} & \rnode{EX06F7}{} & \rnode{EX06F8}{} & \rnode{EX06F9}{} & \rnode{EX06F10}{\textipa{\c{c}}} \\
    P & \rnode{EX06P1}{\textipa{\t{ts}}} & \rnode{EX06P2}{} & \rnode{EX06P3}{} & \rnode{EX06P4}{} & \rnode{EX06P5}{} & \rnode{EX06P6}{\textipa{\t{ts}}} & \rnode{EX06P7}{} & \rnode{EX06P8}{} & \rnode{EX06P9}{} & \rnode{EX06P10}{} \\
  \end{tabular}
  \ncline[nodesep=3pt]{->}{EX06P1}{EX06V2}
  \ncline[nodesep=3pt]{->}{EX06F4}{EX06V5}
  \ncline[nodesep=3pt]{->}{EX06V5}{EX06P6}
  \ncline[nodesep=3pt]{->}{EX06A8}{EX06V9}
  \ncline[nodesep=3pt]{->}{EX06V9}{EX06F10}

  \item
  \begin{tabular}{ccccccccccccc}
    V & \rnode{EX07V1}{} & \rnode{EX07V2}{\textipa{u:}} & \rnode{EX07V3}{} & \rnode{EX07V4}{} & \rnode{EX07V5}{\textipa{E}} & \rnode{EX07V6}{} & \rnode{EX07V7}{} & \rnode{EX07V8}{} & \rnode{EX07V9}{\textipa{I}} & \rnode{EX07V10}{} & \rnode{EX07V11}{} & \rnode{EX07V12}{\textipa{@}} \\
    L & \rnode{EX07A1}{} & \rnode{EX07A2}{} & \rnode{EX07A3}{} & \rnode{EX07A4}{} & \rnode{EX07A5}{} & \rnode{EX07A6}{} & \rnode{EX07A7}{} & \rnode{EX07A8}{\textipa{l}} & \rnode{EX07A9}{} & \rnode{EX07A10}{} & \rnode{EX07A11}{} & \rnode{EX07A12}{} \\
    N & \rnode{EX07N1}{} & \rnode{EX07N2}{} & \rnode{EX07N3}{} & \rnode{EX07N4}{} & \rnode{EX07N5}{} & \rnode{EX07N6}{} & \rnode{EX07N7}{} & \rnode{EX07N8}{} & \rnode{EX07N9}{} & \rnode{EX07N10}{} & \rnode{EX07N11}{} & \rnode{EX07N12}{} \\
    F & \rnode{EX07F1}{} & \rnode{EX07F2}{} & \rnode{EX07F3}{} & \rnode{EX07F4}{\textipa{z}} & \rnode{EX07F5}{} & \rnode{EX07F6}{} & \rnode{EX07F7}{} & \rnode{EX07F8}{} & \rnode{EX07F9}{} & \rnode{EX07F10}{} & \rnode{EX07F11}{\circlearound{\textipa{\c{c}}}} & \rnode{EX07F12}{} \\
    P & \rnode{EX07P1}{\textipa{\t{ts}}} & \rnode{EX07P2}{} & \rnode{EX07P3}{} & \rnode{EX07P4}{} & \rnode{EX07P5}{} & \rnode{EX07P6}{\textipa{\t{ts}}} & \rnode{EX07P7}{} & \rnode{EX07P8}{} & \rnode{EX07P9}{} & \rnode{EX07P10}{} & \rnode{EX07P11}{} & \rnode{EX07P12}{} \\
  \end{tabular}
  \ncline[nodesep=3pt]{->}{EX07P1}{EX07V2}
  \ncline[nodesep=3pt]{->}{EX07F4}{EX07V5}
  \ncline[nodesep=3pt]{->}{EX07V5}{EX07P6}
  \ncline[nodesep=3pt]{->}{EX07A8}{EX07V9}
  \ncline[nodesep=3pt]{->}{EX07V9}{EX07F11}
  \ncline[nodesep=3pt]{->}{EX07F11}{EX07V12}

  \item
  \begin{tabular}{ccccccc}
    V & \rnode{EX08V1}{} & \rnode{EX08V2}{\textipa{a}} & \rnode{EX08V3}{} & \rnode{EX08V4}{} & \rnode{EX08V5}{\textipa{5}} & \rnode{EX08V6}{} \\
    L & \rnode{EX08A1}{} & \rnode{EX08A2}{} & \rnode{EX08A3}{} & \rnode{EX08A4}{} & \rnode{EX08A5}{} & \rnode{EX08A6}{} \\
    N & \rnode{EX08N1}{} & \rnode{EX08N2}{} & \rnode{EX08N3}{} & \rnode{EX08N4}{\circlearound{\textipa{m}}} & \rnode{EX08N5}{} & \rnode{EX08N6}{} \\
    F & \rnode{EX08F1}{\textipa{h}} & \rnode{EX08F2}{} & \rnode{EX08F3}{} & \rnode{EX08F4}{} & \rnode{EX08F5}{} & \rnode{EX08F6}{} \\
    P & \rnode{EX08P1}{} & \rnode{EX08P2}{} & \rnode{EX08P3}{} & \rnode{EX08P4}{} & \rnode{EX08P5}{} & \rnode{EX08P6}{} \\
  \end{tabular}
  \ncline[nodesep=3pt]{->}{EX08F1}{EX08V2}
  \ncline[nodesep=3pt]{->}{EX08V2}{EX08N4}
  \ncline[nodesep=3pt]{->}{EX08N4}{EX08V5}

  \item
  \begin{tabular}{cccccccccc}
    V & \rnode{EX09aV1}{} & \rnode{EX09aV2}{\textipa{E}} & \rnode{EX09aV3}{} & \rnode{EX09aV4}{} & \rnode{EX09aV5}{} & \rnode{EX09aV6}{} & \rnode{EX09aV7}{\textipa{5}} & \rnode{EX09aV8}{} \\
    L & \rnode{EX09aA1}{} & \rnode{EX09aA2}{} & \rnode{EX09aA3}{} & \rnode{EX09aA4}{} & \rnode{EX09aA5}{} & \rnode{EX09aA6}{} & \rnode{EX09aA7}{} & \rnode{EX09aA8}{} \\
    N & \rnode{EX09aN1}{} & \rnode{EX09aN2}{} & \rnode{EX09aN3}{\textipa{n}} & \rnode{EX09aN4}{} & \rnode{EX09aN5}{} & \rnode{EX09aN6}{} & \rnode{EX09aN7}{} & \rnode{EX09aN8}{} \\
    F & \rnode{EX09aF1}{\textipa{f}} & \rnode{EX09aF2}{} & \rnode{EX09aF3}{} & \rnode{EX09aF4}{\textipa{s}} & \rnode{EX09aF5}{} & \rnode{EX09aF6}{} & \rnode{EX09aF7}{} & \rnode{EX09aF8}{} \\
    P & \rnode{EX09aP1}{} & \rnode{EX09aP2}{} & \rnode{EX09aP3}{} & \rnode{EX09aP4}{} & \rnode{EX09aP5}{} & \rnode{EX09aP6}{\textipa{t}} & \rnode{EX09aP7}{} & \rnode{EX09aP8}{} \\
  \end{tabular}
  \ncline[nodesep=3pt]{->}{EX09aF1}{EX09aV2}
  \ncline[nodesep=3pt]{->}{EX09aV2}{EX09aN3}
  \ncline[nodesep=3pt]{->}{EX09aN3}{EX09aF4}
  \ncline[nodesep=3pt]{->}{EX09aP6}{EX09aV7}

  \item
  \begin{tabular}{cccccccc}
    V & \rnode{EX10V0}{} & \rnode{EX10V1}{\textipa{i:}} & \rnode{EX10V2}{} & \rnode{EX10V3}{} & \rnode{EX10V4}{} & \rnode{EX10V5}{\textipa{u:}} & \rnode{EX10V6}{} \\
    L & \rnode{EX10A0}{} & \rnode{EX10A1}{} & \rnode{EX10A2}{} & \rnode{EX10A3}{} & \rnode{EX10A4}{\textipa{l}} & \rnode{EX10A5}{} & \rnode{EX10A6}{} \\
    N & \rnode{EX10N0}{} & \rnode{EX10N1}{} & \rnode{EX10N2}{} & \rnode{EX10N3}{} & \rnode{EX10N4}{} & \rnode{EX10N5}{} & \rnode{EX10N6}{} \\
    F & \rnode{EX10F0}{} & \rnode{EX10F1}{} & \rnode{EX10F2}{} & \rnode{EX10F3}{} & \rnode{EX10F4}{} & \rnode{EX10F5}{} & \rnode{EX10F6}{} \\
    P & \rnode{EX10P0}{\textipa{P}} & \rnode{EX10P1}{} & \rnode{EX10P2}{} & \rnode{EX10P3}{\textipa{g}} & \rnode{EX10P4}{} & \rnode{EX10P5}{} & \rnode{EX10P6}{} \\
  \end{tabular}
  \ncline[nodesep=3pt]{->}{EX10P0}{EX10V1}
  \ncline[nodesep=3pt]{->}{EX10P3}{EX10A4}
  \ncline[nodesep=3pt]{->}{EX10A4}{EX10V5}

  \item
  \begin{tabular}{cccccccccc}
    V & \rnode{EX11V1}{} & \rnode{EX11V2}{\textipa{O}} & \rnode{EX11V3}{} & \rnode{EX11V4}{} & \rnode{EX11V5}{} & \rnode{EX11V6}{} & \rnode{EX11V7}{\textipa{E}} & \rnode{EX11V8}{} \\
    L & \rnode{EX11A1}{} & \rnode{EX11A2}{} & \rnode{EX11A3}{} & \rnode{EX11A4}{} & \rnode{EX11A5}{} & \rnode{EX11A6}{\textipa{l}} & \rnode{EX11A7}{} & \rnode{EX11A8}{} \\
    N & \rnode{EX11N1}{} & \rnode{EX11N2}{} & \rnode{EX11N3}{\textipa{m}} & \rnode{EX11N4}{} & \rnode{EX11N5}{} & \rnode{EX11N6}{} & \rnode{EX11N7}{} & \rnode{EX11N8}{} \\
    F & \rnode{EX11F1}{} & \rnode{EX11F2}{} & \rnode{EX11F3}{} & \rnode{EX11F4}{} & \rnode{EX11F5}{} & \rnode{EX11F6}{} & \rnode{EX11F7}{} & \rnode{EX11F8}{} \\
    P & \rnode{EX11P1}{\textipa{k}} & \rnode{EX11P2}{} & \rnode{EX11P3}{} & \rnode{EX11P4}{} & \rnode{EX11P5}{\textipa{p}} & \rnode{EX11P6}{} & \rnode{EX11P7}{} & \rnode{EX11P8}{\textipa{t}} \\
  \end{tabular}
  \ncline[nodesep=3pt]{->}{EX11P1}{EX11V2}
  \ncline[nodesep=3pt]{->}{EX11V2}{EX11N3}
  \ncline[nodesep=3pt]{->}{EX11P5}{EX11A6}
  \ncline[nodesep=3pt]{->}{EX11A6}{EX11V7}
  \ncline[nodesep=3pt]{->}{EX11V7}{EX11P8}

\end{enumerate}

\Loesung{u45}

\begin{enumerate}\Lf
  \item \Akz freches
  \item \Akz Klingel
  \item \Akz Opa
  \item nach\Akz dem
  \item \Akz Auto
  \item \Akz Autoreifen
  \item Be\Akz endigung
  \item Me\Akz lone
  \item \Akz rötlich
  \item \Akz Rötlichkeit
  \item Pöbe\Akz lei
  \item respek\Akz tabel
  \item \Akz Schulentwicklungsplan\\
    oder Schulent\Akz wicklungsplan (mit Bedeutungsunterschied)
\end{enumerate}

\Loesung{u47}

Silben mit \textipa{[@]} oder \textipa{[5]} im Nukleus sind niemals Einsilbler.
Außerdem kann \textipa{[s]} nicht im Wortanlaut stehen, sondern nur \textipa{[z]}.

\Loesungen{sec:wortklassen}

\Loesung{u52}

\begin{sloppypar}

\begin{enumerate}\Lf
  \item Im Gegensatz zu \textit{gerne} können Wörter wie \textit{quitt} als Ergänzung zu sogenannten Kopulaverben (\textit{sein}, \textit{bleiben}, \textit{werden}) verwendet werden:
    (i) \Ast \textit{Wir sind gerne.} (ii) \textit{Wir sind quitt.}
  \item \textit{Gerne} kann als Antwort auf eine Ja\slash Nein-Frage verwendet werden:
    \textit{Kommst du mit? --- Gerne.}
    Mit \textit{erfreulicherweise} geht das aber nicht:
    \textit{Kommst du mit? --- \Ast Erfreulicherweise.}
  \item Beide Arten von Wörtern können als Pronomen wie eine vollständige Nomenphrase verwendet werden:
    (i) \textit{Ich gehe.}
    (ii) \textit{Das geht.}
    Die Personalpronomina (\textit{ich} usw.) können aber nicht wie ein Artikel verwendet werden:
    (iii) \textit{Der Mann geht.}
    (iv) \Ast \textit{Ich Mann geht.}
  \item Vgl.\ Abschnitt~\ref{sec:artikelpronomen}
\end{enumerate}

\end{sloppypar}

\Loesung{u53}

\begin{enumerate}\Lf
  \item Das Wort \textit{statt} kann sich mit allen Arten von Wörtern, Sätzen und Satzteilen verbinden um ein Kontrastelement zu bilden:
    (i) \textit{Sie will Blumen} [\textit{statt Böller}].
    (ii) \textit{Es ist rot} [\textit{statt grün}].
    (iii) \textit{Er geht} [\textit{statt zu laufen}]. 
  \item \textit{Außer} und \textit{bis auf} sind ähnlich wie \textit{statt} und können Ausnahmeelemente mit allen möglichen Satzteilen bilden.
    Zusätzlich haben diese Wörter die Besonderheit, dass sie manchmal selber den Dativ regieren (i) und manchmal den vom Verb regierten Kasus durchlassen wie in (ii).
    (i) \textit{Ich begrüße alle} [\textit{außer dem Präsidenten}].
    (ii) \textit{Ich begrüße alle} [\textit{außer den Präsidenten}].
  \item \textit{Wie} und \textit{als} leiten Vergleichselemente ein:
    (i) \textit{Der Baum ist größer} [\textit{als der Strauch}].
    (ii) \textit{Die Kiefer ist genauso groß} [\textit{wie die Platane}].
\end{enumerate}

\Loesung{u54}

Die Beispiele für \textit{eben} in den verschiedenen Wortklassen finden sich in (\ref{ex:ex9920}).
Die verschiedenen Unterklassen von \textit{eben} als Partikel können durch Austausch mit \textit{genau} und \textit{halt} ermittelt werden, vgl.\ (\ref{ex:ex9921}).

\begin{exe}
  \ex \label{ex:ex9920}
  \begin{xlist}
    \ex{\label{ex:ex-9920a} Die Platte ist eben. (Adjektiv, Synonym zu \textit{glatt}, \textit{plan})}
    \ex{\label{ex:ex-9920b} Eben kam der Weihnachtsmann um die Ecke.\\
    (Adverb, Synonym zu \textit{vorhin}, \textit{gerade})}
    \ex{\label{ex:ex-9920c} Dann geht es eben nicht. (Partikel, s.\,u.)}
  \end{xlist}
  \ex \label{ex:ex9921}
  \begin{xlist}
    \ex[ ]{\label{ex:ex-9921a} Und genau dieser Test hat die Studierenden so verwirrt.}
    \ex[*]{\label{ex:ex-9921b} Und halt dieser Test hat die Studierenden so verwirrt.}
    \ex[*]{\label{ex:ex-9921c} Diese Tests sind genau schwierig.}
    \ex[ ]{\label{ex:ex-9921d} Diese Tests sind halt schwierig.}
  \end{xlist}
\end{exe}

\Loesung{u55}

Die orthographischen Hinweise durch Groß/Kleinschreibung wurden hier respektiert.
Insofern kann \textit{Abseits} nur ein Substantiv und \zB kein Adverb sein.
Lautlich (aber nicht orthographisch) identisch ist mindestens auch ein Adverb: \textit{Der Stürmer steht abseits.}
Auch eine Präposition \textit{abseits} gibt es: \textit{abseits der Menge}.

Bei dem Pronomen \textit{etwas} handelt es sich um einen Problemfall.
Diskutieren Sie, warum.

\begin{enumerate}\Lf
  \item Adverb
  \item Substantiv
  \item Präposition (\textit{während des Spiels}),\\
    Komplementierer (\textit{während wir spielen})
  \item Pronomen\slash Artikel (\textit{Ich sehe etwas.}),\\
    evtl.\ Adverb (\textit{Etwas kannst du schon zur Seite rutschen.})
  \item Konjunktion (\textit{lecker, aber ungesund}),\\
    Partikel (\textit{Das geht aber nicht.})
  \item Verb
  \item Satzäquivalent
  \item Präposition (\textit{Gruppenbild mit Dame}),\\
    Partikel (\textit{Kommst du mit?})
  \item Adjektiv, Adverb
  \item Substantiv
  \item Komplementierer
  \item Präposition, Adverb
  \item Partikel
  \item Adjektiv, Adverb
  \item Pronomen
  \item Partikel
  \item Präposition (\textit{durch den Nil}),\\
    Partikel (\textit{Er ist durch.})
  \item Adjektiv
  \item Adjektiv (\textit{das gelungene Bild}), Verb (\textit{Es ist gelungen.})
  \item Adverb (\textit{Ich kann damit nichts anfangen.}),\\
    Komplementierer (\textit{Ich gebe, damit du gibst.})
  \item Partikel
  \item Adverb
  \item Verb
  \item Substantiv
  \item Partikel
  \item Adverb
\end{enumerate}


\Loesungen{sec:wortbildung}

\Loesung{u71}

Bei der Analyse wird jeweils nur eine mögliche Klammerung angegeben.
Da sich Produktivität und Transparenz nicht exakt bestimmen lassen, handelt es sich bei dieser Teilfrage um eine Diskussionsfrage, für die es keine Musterlösung gibt.

\begin{center}
  \resizebox{\textwidth}{!}{
    \begin{tabular}{llll}
      \lsptoprule
      \textbf{Analyse} & \textbf{Kopf} & \textbf{Typus} & \textbf{Ausgangswort} \\
      \midrule
      (Wesen-s.zug)-s.analyse & Analyse & Rek./Det. & simplex \\
      Einschub.öffnung & Öffnung & Rek./Det. & Suffix \textit{:ung} \\
      Ess.tisch & Tisch & Determinativk. & simplex \\
      (Räder.werk)-s.reparatur & Reparatur & Rektionsk. & Fremdsuffix \textit{:tur} ? \\
      Einschieb-e.öffnung & Öffnung & Determinativk. & Suffix \textit{:ung}\\
      Groß.rechner & Rechner & Determinativk. & Suffix \textit{:er} \\
      (Bank.note)-n.fälschung & Fälschung & Rek./Det. & Suffix \textit{:ung} \\
      ((Berg.bau).wissenschaft)-s.studium & Studium & Determinativk. & Fremdsuffix \textit{:um} ? \\
      Anschlag-s.vereitelung & Vereitelung & Rektionsk. & Suffix \textit{:ung} \\
      &&& Verbpräfix \textit{ver:} \\
      Bio.laden & Laden & Determinativk. & simplex \\
      Kind-er.garten & Garten & Determinativk. & simplex \\
      Mit.bewohner & Bewohner & Determinativk. & Suffix \textit{:er} \\
      &&& Verbpräfix \textit{be:} \\
      (Absicht-s.erklärung)-s.verlesung & Verlesung & Rektionsk. & Suffix \textit{:ung} \\
      &&& Verbpräfix \textit{ver:} \\
      Monat-s.planung & Planung & Rek./Det. & Suffix \textit{:ung} \\
      feuer.rot & rot & Determinativk. & simplex \\
      (Not.lauf).programm & Programm & Determinativk. & simplex \\
      \lspbottomrule
    \end{tabular}
  }
\end{center}

Bei \textit{Wesenszugsanalyse} und den anderen, für die Rek./Det.\ angegeben ist, gibt es jeweils eine Lesart als Determinativkompositum und eine als Rektionskompositum.
Wenn \zB \textit{Planung} als \textit{der Vorgang des Planens} gelesen wird, ist \textit{Monatsplanung} ein Rektionskompositum.
Wenn man \textit{Planung} hingegen als \textit{Plan} (\zB auf dem Papier) liest, ist es ein Determinativkompositum.
\textit{Bioladen} ist nur dann ein Kompositum, wenn \textit{Bio} als eigenständiges Wort existiert.
Darauf deuten Belege wie \textit{Besseres Bio}.%
\footnote{\raggedright{\url{http://www.erdkorn.de/}, 16.01.2011}}

\Loesung{u72}

Da sich Produktivität und Transparenz nicht exakt bestimmen lassen, handelt es sich bei dieser Teilfrage um eine Diskussionsfrage, für die es keine Musterlösung gibt.

\begin{center}
  \resizebox{\textwidth}{!}{
    \begin{tabular}{lllll}
      \lsptoprule
      \textbf{Analyse} & \textbf{Klasse Ausgangs-/Zielwort} & \textbf{Typ} & \textbf{Umlaut} \\
        \midrule
        (ver:käuf):lich & V > V > Adj & Deriv., Deriv. & ja (\char`~lich) \\
        unter:wander & V > V & Deriv. & nein \\
        alternativ:los & Subst > Adj & Deriv. & nein \\
        Lauf & V > Subst & Stammkonv. & nein \\
        auf=steig & V > V & Deriv. & nein \\
        Ge:bell & V > Subst & Deriv. & nein \\
        be:schließ & V > V & Deriv. & nein \\
        be:gegn & V > V & Deriv. & nein \\
        Röhr:chen & Subst > Subst & Deriv. & ja (Rohr) \\
         &&& nein (Röhre) \\
        Schlingern & V > Subst & Wfkonv. & nein \\
        Ge:ruder & V > Subst & Deriv. & nein \\
        Über:(zock:er) & V > Subst > Subst & Deriv., Deriv. & nein \\
        Ge:brüder & Subst > Subst & Deriv. & ja? \\
        Münd:el & Subst > Subst & Deriv. & ja? \\
        schweig:sam & V > Adj & Deriv. & nein \\
        \lspbottomrule
    \end{tabular}
  }
\end{center}

\begin{sloppypar}

Es wurden nur die Stämme analysiert und das Infinitiv-Suffix \textit{-en} weggelassen.
Es gilt wieder, dass die Einschätzung der Produktivität in vielen Fällen nicht eindeutig ist.
Bei \textit{be:gegn-en} ist aber \zB eine produktive Bildung auszuschließen, weil \textit{gegn-en} kein selbständiges Verb ist, sondern nur in Verbindungen wie \textit{be:gegn-en} und \textit{ent:gegn-en} vorkommt.
Das Problem bei \textit{Gebrüder} ist, dass es sich um ein Pluraletantum handelt.
Es gibt kein Wort \textit{\Ast Ge:bruder}, und \textit{Ge\char`~} ist hier ggf.\ eine Art unregelmäßiger Pluralbildung, womit der Status als Wortbildung fraglich wäre.
Bei \textit{Münd:el} liegt historisch eine Ableitung vor, aber wie transparent diese noch ist, ist fraglich.
Wenn die Form aber nicht einmal mehr transparent wäre, könnte man eigentlich auch nicht mehr von einer Umlautung sprechen.

\end{sloppypar}

\Loesung{u73}

Das Problem ist, dass hier größere syntaktische Strukturen ähnlich einer Konversion wie Wörter verwendet werden.
Im Fall von \textit{Mehr-als-Beliebigkeit} muss man \zB annehmen, dass die syntaktische Einheit \textit{mehr als beliebig} zugrundeliegt.
Diese ist normalerweise vom Status her kein Adjektiv bzw.\ kein Wort, so dass das Wortbildungselement \textit{-keit} eigentlich nicht angeschlossen werden kann.
In erster Näherung kann man diese Phänomene als kreativen Sprachgebrauch teilweise von der Kerngrammatik abkoppeln, oder man müsste in der Tat die Grammatik selber flexibler formulieren.

\Loesung{u74}

Hier liegen sog.\ Kurzwortbildungen vor.
Im Fall von \textit{Lok} oder \textit{Vopo} werden einfach Teile des Wortes weggekürzt.
Bei \textit{Fundi} kann man nur schwer von Wortbildung sprechen, da kein eindeutiges Ausgangswort auszumachen ist.
Mit ziemlicher Sicherheit ist es keine Bildung zu \textit{Fundamentalist\slash in}.
Bei den Koseformen in \textit{Kotti} usw.\ wird das Ausgangswort auf eine Silbe verkürzt und ein \textit{-i} angehängt, was ein grundlegend anderer Fall mit eigener Bedeutungskomponente (Verniedlichung o.\,Ä.) ist.
Bei \textit{Poldi} wird zusätzlich der Wortrest verändert (das /\textipa{l}/ von der nächsten Coda in die vorherige geholt, quasi zur \textit{Poldoski}), da wahrscheinlich \textit{Podi} keinen ausreichenden Wiedererkennungswert hätte.

\Loesungen{sec:nominalflexion}

\Loesung{u81}

Ob Adjektive pronominal oder adjektivisch flektiert sind, steht bei der Klasse in Klammern dahinter.
Es ist zu beachten, dass hier einmal \textit{Buchstabe} (schwach) und einmal \textit{Buchstaben} (gemischt) als Stamm vorkommen.

\begin{center}
  \begin{tabular}{lllll}
    \lsptoprule
    \textbf{Analyse} & \textbf{Klasse} & \textbf{\textsc{Kasus}} & \textbf{\textsc{Numerus}} & \textbf{Aufgabe 4-6} \\
    \midrule
    ein & Art & \textit{nom} & \textit{sg} & indef. \\
    zweit-es & Adj (pron.) & \textit{nom} & \textit{sg} & pron. \\
    Erbe & Subst & \textit{nom} & \textit{sg} & gem./st. \\
    d-es & Art & \textit{gen} & \textit{sg} & def. \\
    Krieg-es & Subst & \textit{gen} & \textit{sg} & st. \\
    die & Art & \textit{akk} & \textit{sg} & def. \\
    Tragen & Subst & \textit{dat} & \textit{sg} & gem. \\
    Wagen & Subst & \textit{akk} & \textit{sg} & gem. \\
    einig-em & Pron & \textit{dat} & \textit{sg} & pron. \\
    gut-en & Adj (adj.) & \textit{dat} & \textit{sg} & adj. \\
    sein-er & Art & \textit{dat} & \textit{sg} & indef. \\
    mein-er & Art & \textit{gen} & \textit{sg} & indef. \\
    Erfahrung & Subst & \textit{gen} & \textit{sg} & fem. \\
    Sache & Subst & \textit{nom} & \textit{sg} & fem. \\
    gross-em & Subst & \textit{dat} & \textit{sg} & pron. \\
    das & Pron & \textit{nom} & \textit{sg} & pron. \\
    Fall & Subst & \textit{nom} & \textit{sg} & st. \\
    d-em & Art & \textit{dat} & \textit{sg} & def. \\
    Form & Subst & \textit{dat} & \textit{sg} & fem. \\
    Buchstabe-n & Subst & \textit{gen} & \textit{sg} & schw. \\
    Mäus-e-n & Subst & \textit{dat} & \textit{pl} & fem. \\
    Sätz-e & Subst & \textit{akk} & \textit{pl} & st. \\
    Leid-s & Subst & \textit{gen} & \textit{sg} & gem. \\
    Nägel & Subst & \textit{akk} & \textit{pl} & st. \\
    entsprechend-em & Adj (pron.) & \textit{dat} & \textit{sg} & pron. \\
    technisch-em & Adj (pron.) & \textit{dat} & \textit{sg} & pron. \\
    jen-en & Pron & \textit{akk} & \textit{sg} & pron. \\
    rasche-n & Adj (adj.) & \textit{akk} & \textit{sg} & adj. \\
    Buchstaben-s & Subst & \textit{gen} & \textit{sg} & gem. \\
    \lspbottomrule
  \end{tabular}
\end{center}

\Loesung{u82}

\begin{sloppypar}
Alle Formen des Wortes lauten \textit{Kuchen}, außer dem Genitiv Singular, der \textit{Kuchen-s} lautet.
Ob der Plural hier \textit{-e} (traditionelle starke Flexion) oder \textit{-en} (traditionelle gemischte Flexion) ist, kann nicht entschieden werden, da sowohl \textit{\Ast Kuchen-e} als auch \textit{\Ast Kuchen-en} zu \textit{Kuchen} reduziert würden.
Auffällig ist außerdem, dass die Pseudo-Endung \textit{en} in Wortbildungsprozessen getilgt wird, \zB \textit{Küch:}\textit{lein}.
Letzteres geschieht aber bei allen ähnlichen Wörtern auf \textit{en} (\textit{Wäglein}, \textit{Gärtchen} usw.).
\end{sloppypar}

\Loesung{u83}

Die normativ korrekten Plurale verlangen eine Tilgung der (überwiegend) lateinischen Endungen des Nominativ Singulars \textit{-us} bzw.\ \textit{-a}, um den Stamm freizulegen, an den dann deutsche Pluralendungen treten können.
Die Pluralendung ist prinzipiell \textit{-en}, der Dativ im Plural damit nicht sichtbar.
Im Singular wird die lateinische Endung nicht gelöscht und konform zum deutschen Kasussystem im Maskulinum und Neutrum ein \textit{-s} im Genitiv suffigiert, wenn das Wort nicht sowieso auf /\textipa{s}/ endet.
Rein innerhalb der deutschen Grammatik betrachtet haben wir es hier also mit Substantiven zu tun, die einen Singular- und einen Pluralstamm haben (\textit{Organismus\slash Organism} usw.), was recht ungewöhnlich ist.
Daher kommen die selbstverständlich stilistisch stark umgangssprachlichen Ausgleichsvarianten des Plurals wie \textit{Organismusse} oder \textit{Firmas}.

\Loesung{u83a}

Es fällt auf, dass wie beschrieben das Femininum keine Genitiv-Markierung hat, also \textit{der Petunie}.
Bei den Maskulina und Neutra sind bei \textit{Tisch} und \textit{Bett} wie beschrieben jeweils \textit{des Tisch-s} und \textit{des Tisch-es} bzw.\ \textit{des Bett-s} und \textit{des Bett-es} möglich, weil die Stämme nicht auf \textit{el}, \textit{er} oder \textit{en} ausgehen.
Bei Wörtern auf \textit{lein}, Wörtern mit vokalischem Auslaut, Wörtern auf \textit{tum} und Eigennamen kann aber jeweils nur \textit{-s} verwendet werden, vgl. \textit{*Häuslein-es}, \textit{*Stroh-es}, \textit{*Brauchtum-es}, \textit{*Ischariot-es}.
Für vokalische Auslaute wird manchmal auch \textit{-es} zugelassen, womit \textit{Stroh-es} dann grammatisch wäre.
Bei Wörtern, die auf \textit{s} auslauten, kann allerdings nur \textit{-es} verwendet werden, vgl. \textit{*Hindernis-s}.

\Loesung{u84}

Für \textit{lila} und \textit{rosa} wird ein Stamm auf \textit{n} rekonstruiert, der ein normal flektierendes Adjektiv darstellt, also \textit{rosan-es}, \textit{lilan-em} usw.
Durch den Einschub des \textit{n} wird ein Zusammentreffen von Vollvokal und Schwa verhindert: \textit{\Ast rosa-es}.\\
\textit{Durch} und \textit{pleite} sind Partikeln, deren Stamm hier entweder einfach als flektierbarer Adjektivstamm verwendet wird (\textit{durch-es}), oder aber um \textit{en} erweitert wird (\textit{durchen-es}).
Dass dies passiert, liegt mit hoher Wahrscheinlichkeit an der funktionalen Nähe zu prädikativen Adjektiven.
Die attributive Funktion nebst der passenden Flexion wird also analog zu den Adjektiven ergänzt.

Alle diese Bildungen verstärken die Systematizität der Grammatik, da durch sie die Einheitlichkeit der Flexion der Adjektive erhöht wird, und weil die Klasse der Partikeln, die nur bei Kopulaverben vorkommt (\textit{durch}, \textit{pleite} usw.) mit der funktional verwandten Klasse der Adjektive verschmilzt.

\Loesung{u85}

Bei dem Paradigma von \textit{derjenige} handelt es sich im Wesentlichen um eine Kombination aus einem pronominal\slash stark flektierenden Definitartikel \textit{der} mit einem adjektival\slash schwach flektierenden Element \textit{jenig}.

\Loesungen{sec:verben}

\Loesung{u91}

Es werden jeweils die relevanten Zeitpunkte definiert (Variable mit Doppelpunkt und Beschreibung des Zeitpunkts), dann die Relationen angegeben.
\textbf{S} muss nicht definiert werden, weil es immer der Sprech-/Schreibzeitpunkt des Satzes ist.

\begin{enumerate}
  \item \textbf{E}: Niederösterreich lebt.\\
    \textbf{S\REq E} (Präsens, als Sonderfall von \textbf{E\RUn S})
  \item \textbf{E}: Sie ziehen ohne Beute ab.\\
    \textbf{E\RPr S} (Präteritum)
  \item \textbf{E\Tidx{1}}: Näf setzt sich von seinen Gegnern ab.\\
    \textbf{E\Tidx{2}}: Ein erster Angriff ist nicht erfolgreich.\\
    \textbf{E\Tidx{1}\RPr S} (Präteritum)\\
    \textbf{R\REq E\Tidx{1}} (Einführung von \textbf{R} für Plusquamperfekt)\\
    \textbf{E\Tidx{2}\RPr R\RPr S} (Plusquamperfekt)
  \item \textbf{E}: Die Pflanzzeit für Stauden beginnt.\\
    \textbf{S\RPr E} (Präsens, als Sonderfall von \textbf{E\RUn S})
  \item \textbf{E}: Der Schulrat wählt dieselbe Vorgehensweise.\\
    \textbf{S\RPr E} (Futur)
\end{enumerate}

\Loesung{u92}

In der Tabelle stehen die einzelnen Wortformen analytischer Konstruktionen in der Reihenfolge, wie sie auch im Satz vorkommen.

\begin{center}
  \resizebox{\textwidth}{!}{
    \begin{tabular}{lllll}
      \lsptoprule
      \textbf{Form} & \textbf{Bezeichnung} & \textbf{syn.\slash ana.} & \textbf{finit} & \textbf{infinit} \\
      \midrule
      heißt & Präsens & syn. & heißt & --- \\
      kamen & Präteritum & syn. & kamen & --- \\
      wird sein & Futur & ana. & wird & sein \\
      gewesen ist & Perfekt & ana. & ist & gewesen \\
      hat gezeigt & Perfekt & ana. & hat & gezeigt \\
      besteht & Präsens & syn. & besteht & --- \\
      hatte geäußert & Plusquamperfekt & ana. & hatte & geäußert \\
      hatte gesprochen & Plusquamperfekt & ana. & hatte & gesprochen \\
      ahnten & Präteritum & syn. & ahnten & --- \\
      zukommt & Präsens & syn. & zukommt & --- \\
      war gewesen & Plusquamperfekt & ana. & war & gewesen \\
      wird gewesen sein & Futurperfekt & ana. & wird & gewesen, sein \\
      gelingt & Präsens & syn. & gelingt & --- \\
      geht & Präsens & syn. & geht & --- \\
      können & Präsens & syn. & können & --- \\
      beginnt & Präsens & syn. & beginnt & --- \\
      hofft & Präsens & syn. & hofft & --- \\
      leuchten wird & Futur & ana. & wird & leuchten \\
      wird können & Futur & ana. & wird & können \\
      \lspbottomrule
    \end{tabular}
  }
\end{center}

Zu beachten ist, dass die infiniten Komplemente von Modalverben (\zB \textit{wählen} in \textit{wird wählen können}) nicht Teil der Tempuskonstruktion sind.
Außerdem liegt in \textit{war abgestellt gewesen} ein Plusquamperfekt eines sogenannten Zustandspassivs (\textit{ist abgestellt}) vor.
Auch das Partizipkomplement von \textit{sein} ist in diesem Fall nicht unbedingt Teil der Tempuskonstruktion.
Genaueres zu diesen Konstruktionen findet sich in Kapitel~\ref{sec:relationenpraedikate}.

\Loesung{u94}

\begin{enumerate}\Lf
  \item
    \begin{enumerate}\Lf
      \item \textit{pflege}: \textit{pfleg}, schwach
      \item \textit{sei}: \textit{sei}, unregelmäßig (suppletiv)
    \end{enumerate}
  \item
    \begin{enumerate}\Lf
      \item \textit{sollte}: \textit{soll}, prätertialpräsentisch
      \item \textit{umgebaut}: \textit{bau}, schwach
      \item \textit{werden}: \textit{werd}, stark
    \end{enumerate}
  \item
    \begin{enumerate}\Lf
      \item \textit{könnet}: \textit{könn}, prätertialpräsentisch
      \item \textit{wachen}: \textit{wach}, schwach
    \end{enumerate}
  \item
    \begin{enumerate}\Lf
      \item \textit{wird}: \textit{wir}, stark
      \item \textit{gewaschen}: \textit{wasch}, stark
      \item \textit{gesponnen}: \textit{sponn}, stark
      \item \textit{gedreht}: \textit{dreh}, schwach
    \end{enumerate}
  \item \textit{bäckt}: \textit{bäck}, stark (bei den meisten Sprechern schwach)
  \item
    \begin{enumerate}\Lf
      \item \textit{gibt}: \textit{gib}, stark
      \item \textit{untergestellt}: \textit{stell}, schwach
      \item \textit{werden}: \textit{werd}, stark
      \item \textit{können}: \textit{könn}, prätertialpräsentisch
    \end{enumerate}
  \item
    \begin{enumerate}\Lf
      \item \textit{drohte}: \textit{droh}, schwach
      \item \textit{töten}: \textit{töt}, schwach
      \item \textit{käuft}: \textit{käuf}, schwach, hier aber mit Umlautstufe 2.\slash 3.~Pers Sg Präs Ind
    \end{enumerate}
  \item
    \begin{enumerate}\Lf
      \item \textit{rauften}: \textit{rauf}, schwach
      \item \textit{schmiedeten}: \textit{schmiede}, schwach
    \end{enumerate}
  \item
    \begin{enumerate}\Lf
      \item \textit{schwängen}: \textit{schwäng}, stark
      \item \textit{wären}: \textit{wär}, unregelmäßig (suppletiv)
    \end{enumerate}
  \item
    \begin{enumerate}\Lf
      \item \textit{musst}: \textit{muss}, prätertialpräsentisch
      \item \textit{boxen}: \textit{box}, schwach
    \end{enumerate}
\end{enumerate}

\Loesung{u95}

Die Analysen der Affixe gemäß der Aufgabenstellung sind jeweils interlinear unter den Stämmen und Affixen angegeben.
Für die Stufe des Stamms wird hier --- geschrieben, wenn das Verb schwach ist, sonst 1 -- 4.
Bei vollständig unregelmäßigen Verben und Modalverben steht bei der Stammanalyse $+$, weil hier die Bestimmung der Stufe auch nicht im Sinn der vier Stufen der starken Verben möglich ist.

\setcounter{equation}{0}

\begin{exe}
  \ex
  \begin{xlist}
    \ex\gll pfleg -e\\
    --- Konj\\
    \ex\gll sei\\
    $+$\\
  \end{xlist}
  \ex
  \begin{xlist}
    \ex\gll soll -te\\
    $+$ Prät\\
    \ex\gll ge- bau -t\\
    Part --- Part\\
    \ex\gll werd -en\\
    1 Inf\\
  \end{xlist}
  \ex
  \begin{xlist}
    \ex\gll könn -e -t\\
    $+$ Konj {PN2 (2~Pl)}\\
    \ex\gll wach -en\\
    --- Inf\\
  \end{xlist}
  \ex
  \begin{xlist}
    \ex\gll wird\\
    2\\
    \ex\gll ge- wasch -en\\
    Part 4 Part\\
    \ex\gll ge- sponn -en\\
    Part 4 Part\\
    \ex\gll ge- dreh -t\\
    Part --- Part\\
  \end{xlist}
  \ex\gll bäck -t\\
  --- {PN1 (3~Sg)}\\
  \ex
  \begin{xlist}
    \ex\gll gib -t\\
    2 PN1(3~Sg)\\
    \ex\gll ge- stell -t\\
    Part --- Part\\
    \ex\gll werd -en\\
    1 Inf\\
    \ex\gll könn -t -e -n\\
    $+$ Prät Konj {PN2 (3~Pl)}\\
  \end{xlist}
  \ex
  \begin{xlist}
    \ex\gll droh -te\\
    --- Prät\\
    \ex\gll töt -en\\
    --- Inf\\
    \ex\gll käuf -t\\
    --- {PN1 (3~Sg)}\\
  \end{xlist}
  \ex
  \begin{xlist}
    \ex\gll rauf -te -n\\
    --- Prät {PN2 (3~Pl)}\\
    \ex\gll schmiede -te -n\\
    --- Prät {PN2 (3~Pl)}\\
  \end{xlist}
  \ex
  \begin{xlist}
    \ex\gll schwäng -e -n\\
    3+Umlaut Konj {PN2 (3~Pl)}\\
    \ex\gll wär -e -n\\
    $+$ Konj {PN2 (3~Pl)}\\
  \end{xlist}
  \ex
  \begin{xlist}
    \ex\gll muss -t\\
    $+$ {PN2 (2~Sg)}\\
    \ex\gll box -en\\
    --- Inf\\
  \end{xlist}
\end{exe}

Bei \textit{käuft} liegt eine Umlautstufe vor, obwohl es bei den meisten Sprecher\slash innen ein schwaches Verb ist.
Bei \textit{wird} ist das orthographische \textit{-d} phonologisch /\textipa{t}/.
Als zugrundeliegende und phonologisch kompatible Analyse käme daher auch \textit{wird-t} infrage, also mit dem Suffix PN1 3.~Pers Sg (\textit{-t}).
Eine alternative Analyse mit einem Stamm \textit{wir} und der Form \textit{wir-d} (\textit{-d} geschrieben für \textit{-t}) müsste den Stamm \textit{wir} annehmen.
Da dieser für die Form \textit{wir-st} evtl.\ sowieso benötigt wird, wäre das aber nicht unbedingt ein Problem.
In solchen Fällen sind synchron, also jenseits der historischen Entwicklung, die zu diesem Zustand geführt hat, eindeutig die Grenzen der systematischen Segmentierung erreicht.

\Loesung{u96}

Das Verb \textit{wissen} flektiert im Wesentlichen als Präteritalpräsens mit einer Singularstufe und einer Pluralstufe im Präsens Indikativ (\textit{weiß} und \textit{wiss}).
Das Präteritum und das Partizip geht schwach, wie bei den anderen Präteritalpräsentien, allerdings mit einer eigenen Vokalstufe \textit{wuss}.
Problematisch ist das Flexionsverhalten von \textit{wissen} nur dann, wenn die Präteritalpräsentien mit den Modalverben genau identifiziert werden sollen, weil \textit{wissen} sich semantisch anders verhält als die Modalverben, nämlich wie ein Vollverb.
Wenn es einen Infinitiv regiert, dann anders als die Modalverben den 2.~Status: \textit{Er weiß zu gefallen.}
Vgl.\ Abschnitte~\ref{sec:infinflex} und~\ref{sec:verbkomplexe}.

\Loesungen{sec:konstituentenstruktur}

\Loesung{u101}

Zu jeder Testserie wir einleitend eine Entscheidung getroffen, ob es sich um eine Konstituente handelt.

\begin{enumerate}\Lf
  \item Es handelt sich um eine Konstituente:
    \begin{itemize}\Lf
      \item \PronTest So nimmt er sich [dann] auch zurück \ldots
      \item \VfTest [Während den Spielen] nimmt er sich (so) auch zurück \ldots
      \item \KoorTest So nimmt er sich [[während den Spielen] und [nach den Spielen]] auch zurück \ldots
    \end{itemize}
  \item Es handelt sich nicht um eine Konstituente:
    \begin{itemize}\Lf
      \item \PronTest \Ast Parteichef wird (wahrscheinlich) [er].\\
        Bei dieser Anwendung des Tests wird \textit{wahrscheinlich} nicht mitpronominalisiert.
      \item \VfTest \Ast [Wahrscheinlich Sigmar Gabriel] wird Parteichef.
      \item \KoorTest Parteichef wird [[wahrscheinlich Sigmar Gabriel] oder [hoffentlich Peer Steinbrück]].
    \end{itemize}
  \item Auf Basis der Tests kann man den Status der potentiellen Konstituente nicht eindeutig bestimmen:
    \begin{itemize}\Lf
      \item \PronTest \Ast Ein Vermieter kann mittels eines Formularvertrags keine Betriebskosten für die Reinigung eines Öltanks [?].
      \item \VfTest [Auf den Mieter umlegen] kann ein Vermieter mittels eines Formularvertrags keine Betriebskosten für die Reinigung eines Öltanks.
      \item \KoorTest Ein Vermieter kann mittels eines Formularvertrags keine Betriebskosten für die Reinigung eines Öltanks [[auf den Mieter umlegen] oder [steurlich geltend machen]].
    \end{itemize}
  \item Es handelt sich um eine Konstituente:
    \begin{itemize}\Lf
      \item \PronTest Die beste Möglichkeit vergab [einer].
      \item \VfTest [Ein Gäste-Stürmer, dessen Schuss knapp am Gehäuse drüber ging] vergab die beste Möglichkeit.
      \item \KoorTest Die beste Möglichkeit vergab [[ein Gäste-Stürmer, dessen Schuss knapp am Gehäuse drüber ging] oder [ein anderer Spieler]].
    \end{itemize}
  \item Es handelt sich um eine Konstituente, aber aus Gründen, die in Kapitel~\ref{sec:saetze} besprochen werden, muss die Stellung der Pronomina (einschließlich der Fragepronomina) beim PronTest verändert werden:
    \begin{itemize}\Lf
      \item \PronTest Die vier Musiker lösen ihre Band nach dreieinhalb Jahren [deswegen] auf.
      \item \VfTest [Weil sich der Sänger musikalisch verändern will], lösen die vier Musiker ihre Band nach dreieinhalb Jahren auf.
      \item \KoorTest Die vier Musiker lösen ihre Band nach dreieinhalb Jahren auf, [[weil sich der Sänger musikalisch verändern will] und [weil sie einander nicht mehr austehen können]].
    \end{itemize}
  \item Es handelt sich um eine Konstituente:
    \begin{itemize}\Lf
      \item \PronTest In der Gemeindestube weiß man von diesen konkreten Plänen [nichts].
      \item \VfTest [Überhaupt nichts] weiß man in der Gemeindestube von diesen konkreten Plänen.
      \item \KoorTest In der Gemeindestube weiß man von diesen konkreten Plänen [[überhaupt nichts] oder [alles]].
    \end{itemize}
  \item Es handelt sich um eine Konstituente:
    \begin{itemize}\Lf
      \item \PronTest \Ast Wagas suchte eifrig nach einem dickeren Ast [darum].
      \item \VfTest [Um zu helfen] suchte Wagas eifrig nach einem dickeren Ast.
      \item \KoorTest Wagas suchte eifrig nach einem dickeren Ast, [[um zu helfen] und [um positiv aufzufallen]].
    \end{itemize}
  \item Es handelt sich nicht um eine Konstituente:
    \begin{itemize}\Lf
      \item \PronTest \Ast Wagas suchte eifrig [?] dickeren Ast, um zu helfen.
      \item \VfTest \Ast [Nach einem] suchte Wagas eifrig dickeren Ast, um zu helfen.
      \item \KoorTest \Ast Wagas suchte eifrig [[nach einem] und [unter einem]] dickeren Ast, um zu helfen.
    \end{itemize}
  \item\label{it:insel} Es handelt sich um eine Konstituente, obwohl einige Tests fehlschlagen:
    \begin{itemize}\Lf
      \item \PronTest \Ast Auch viele Beobachter sprachen von einer sterilen Debatte [?].
      \item \VfTest \Ast [Ohne spannende Passagen] sprachen auch viele Beobachter von einer sterilen Debatte.
      \item \KoorTest Auch viele Beobachter sprachen von einer sterilen Debatte [[ohne spannende Passagen] und [mit viel langweiligem Gefasel]].
    \end{itemize}
\end{enumerate}

In Beispiel \ref{it:insel} liegt das Scheitern der meisten Tests daran, dass die gesuchte Konstituente in eine NP eingebettet ist, vgl.\ Abschnitt~\ref{sec:ngr}.
Solche Konstituenten kann man nicht einfach so erfragen oder ins Vorfeld stellen.

\Loesung{u102}

Ist der VfTest erfolgreich, muss auf Satzgliedstatus der Konstituente geschlossen werden.

\begin{enumerate}\Lf
  \item \VfTest [Den Wahlabend so direkt zu verfolgen und den direkten Kontakt mit dem Wähler zu erleben], wird spannend sein.
  \item \VfTest \Ast [Mit dem Wähler] wird es spannend sein, den Wahlabend so direkt zu verfolgen und den direkten Kontakt zu erleben.
  \item \VfTest [Er] nimmt sich (so) während den Spielen auch zurück, denn die taktischen Anweisungen gibt es vorher.
  \item \VfTest [Sehr wahrscheinlich] hätten dann die 37 problemlos in Deutschland Asyl erhalten.
  \item \VfTest \Ast [Den Mieter] kann ein Vermieter mittels eines Formularvertrags keine Betriebskosten für die Reinigung eines Öltanks auf umlegen.
  \item \VfTest [Ein Gäste-Stürmer, dessen Schuss knapp am Gehäuse drüber ging], vergab die beste Möglichkeit.
  \item \VfTest [Weil sich der Sänger musikalisch verändern will], lösen die vier Musiker ihre Band nach dreieinhalb Jahren auf.
  \item \VfTest \Ast [Der Gemeindestube] weiß man in von diesen konkreten Plänen überhaupt nichts.
  \item \VfTest [Von diesen konkreten Plänen] weiß man in der Gemeindestube überhaupt nichts.
  \item \VfTest [Um zu helfen] suchte suchte Wagas eifrig nach einem dickeren Ast.
  \item \VfTest [Außer Kaffee und Kuchen] erwarteten sie dort gekühlte Getränke und Leckeres vom Grill.
  \item \VfTest [Bis auf den Pürierstab-Kollegen] grinsten oder kudderten alle.
\end{enumerate}

\Loesung{u103}

Die Konstituente \textit{über Syntax} wäre nach klassischer Auffassung kein Satzglied und sollte nicht ins Vorfeld gestellt werden können, genauso wie \textit{mit der Sahne} in (\ref{ex:syn5556a}) auf S.~\pageref{ex:syn5556a}.
Man würde erwarten, dass nur \textit{ein Buch über Syntax} ein Satzglied darstellt und ins Vorfeld gestellt werden kann.
In \citet{Dekuthy2002} werden semantische und pragmatische Faktoren beschrieben, die diese Konstruktion erlauben oder nicht erlauben.
Wenn dieser Sätze akzeptabel sind, stimmt das zum Satzglied Gesagte nicht mehr wirklich, bzw.\ der Test auf Satzgliedstatus liefert unzuverlässige Ergebnisse.

\Loesungen{sec:phrasen}

\Loesung{u111}

\begin{enumerate}
  \item \Tree[1.7]{
      & \K{NP}\B{dl}\B{d}\B{drr} \\
      \K{Art}\B{d} & \K{\textbf{N}}\B{d} && \K{PP}\TRi[-8] \\
      \K{\textit{dem}} & \K{\textit{Führungstreffer}} && \K{\textit{durch Winkler}} & \\
  }
  \item \Tree[1.7]{
      \K{NP}\B{d}\B{drr} \\
      \K{\textbf{N}}\B{d} && \K{KP}\TRi[-9] \\
    \K{\textit{Überzeugung}} && \K{\textit{dass dies ein}}\Below{\textit{Weisungstraum sei}} & \\
  }
  \item \Tree{
      &&& \K{NP}\B{dll}\B{d} \\
      & \K{AP}\TRi[-5] && \K{\textbf{N}}\B{d} \\
    & \K{\textit{trockenen,}}\Below{\textit{glatten}} && \K{\textit{Asphalt}} \\
  }
  \item \Tree{
      &&& \K{AP}\B{dll}\B{d}\B{drr} \\
      & \K{AP}\TRi[-4] && \K{(Konj)}\B{d} && \K{AP}\TRi[-6] \\
    & \K{\textit{trockenen}} && \K{(,)} && \K{\textit{glatten}} & \\
  }
  \item \Tree[1.7]{
      \K{NP}\B{d}\B{drr} \\
      \K{\textbf{N}}\B{d} && \K{KP}\TRi[-7] \\
    \K{\textit{Unsicherheit}} && \K{\textit{ob entsprechender}}\Below{\textit{Platz zur Verfügung stehe}} & \\
  }
  \item \Tree[1.7]{
      &&& \K{AP}\B{dll}\B{d} \\
      & \K{NP}\TRi[-7] && \K{\textbf{A}}\B{d} \\
   & \K{\textit{der Wiederholung}} && \K{\textit{überdrüssig}} \\
  }
\end{enumerate}

\Loesung{u112}

\begin{enumerate}
  \item \Tree{
      &&& \K{PP}\B{dll}\B{d}\B{drr} \\
      & \K{AdvP}\TRi[-8] && \K{\textbf{P}}\B{d} && \K{NP}\TRi[-4] \\
      & \K{\textit{weit}} && \K{\textit{in}} && \K{\textit{das Land}} & \\
    }
  \item \Tree{
      & \K{PP}\B{dl}\B{d}\B{drr} \\
      \K{Ptkl}\B{d} & \K{\textbf{P}}\B{d} && \K{NP}\TRi[-7] \\
      \K{\textit{sehr}} & \K{\textit{unter}} && \K{\textit{Druck}} & \\
    }
  \item \Tree{
      & \K{AdvP}\B{dl}\B{d} \\
      \K{Ptkl}\B{d} & \K{\textbf{Adv}}\B{d} \\
    \K{\textit{ganz}} & \K{\textit{vorne}} \\
    }
  \item \Tree{
      &&& \K{PP}\B{dll}\B{d}\B{drr} \\
      & \K{NP}\TRi[-3] && \K{\textbf{P}}\B{d} && \K{NP}\TRi[-2] \\
    & \K{\textit{zwei Stunden}} && \K{\textit{nach}} && \K{\textit{dem Diebstahl}} & \\
    }
  \item \Tree[3]{
      & \K{AdvP}\B{dl}\B{d} \\
      \K{Ptkl}\B{d} & \K{\textbf{Adv}}\B{d} \\
    \K{\textit{völlig}} & \K{\textit{unverständlicherweise}} \\
    }
\end{enumerate}

\Loesung{u113}

Hier folgen zwei P einander, nämlich \textit{von} und \textit{unter}.
Die Phrasenschemata erlauben dies auf keinen Fall.
Eine mögliche Analyse mit wenig Aufwand wäre es, \textit{von unter} als komplexe Präposition zu analysieren und an der Syntax gar nichts zu ändern.
Dagegen spricht, dass eine PP wie [\textit{von weit unter der Erde}] denkbar ist, bei der der Modifikator \textit{weit} zwischen die beiden Präpositionen tritt.
Evtl.\ ist es auch eine Lösung, mit starken semantischen Einschränkungen ein Schema PP=[~P~PP~] zuzulassen.

\Loesung{u114}

\begin{enumerate}
  \item
    \Tree{
      \K{KP}\B{dd}\B{drrrrrrrr} \\
      &&&&&&&& \K{VP}\B{dllllll}\B{dllll}\B{dll}\B{d} \\
      \K{\textbf{K}}\B{d} && \K{NP}\TRi[-1] && \K{PP}\TRi[-5] && \K{PP}\TRi[-7] && \K{\textbf{V}}\B{d} \\
      \K{\textit{dass}} && \K{\textit{ein griechischer}}\Below{\textit{Außenminister}} && \K{\textit{zu einem}}\Below{\textit{Besuch}} && \K{\textit{in der}}\Below{\textit{Türkei}} && \K{\textit{weilt}} & \\
    }
  \item
    \Tree{
      \K{KP}\B{dd}\B{drrrrrrrr} \\
      &&&&&&&& \K{VP}\B{dllllll}\B{dllll}\B{dll}\B{d} \\
      \K{\textbf{K}}\B{d} && \K{NP}\TRi[-3] && \K{AdvP}\TRi[-9] && \K{?}\TRi[-7] && \K{\textbf{V}}\B{d} \\
      \K{\textit{dass}} && \K{\textit{die schweren}}\Below{\textit{Fehler}} && \K{\textit{so}} && \K{\textit{nicht}}\Below{\textit{mehr}} && \K{\textit{passieren}} & \\
  }\\

  Welchen Status die Konstituente [\textit{nicht mehr}] genau hat, und ob es überhaupt eine Konstituente ist, soll hier nicht interessieren.
  Mangels Vorfeldfähigkeit kann es keine AdvP sein.
  Wenn es sich um Partikeln handeln würde, müssten wir erklären, wie zwei Partikeln zusammen eine Konstituente bilden können, denn immerhin wurde dafür kein Phrasenschema angegeben.
  \item
    \Tree{
      \K{KP}\B{dd}\B{drrrrrr} \\
      &&&&&& \K{VP}\B{dllll}\B{dll}\B{d} \\
      \K{K}\B{d} && \K{NP}\TRi[-6] && \K{PP}\TRi[-4] && \K{\textbf{V}}\TRi[-7] \\
      \K{\textit{ob}} && \K{\textit{ein}}\Below{\textit{Casino}} && \K{\textit{zum}}\Below{\textit{Gesamtkonzept}} && \K{\textit{passen}}\Below{\textit{würde}} & \\
  }
  \item
    \resizebox{0.9\textwidth}{!}{
      \Tree{
	\K{KP}\B{dd}\B{drrrrrrrrrr} \\
	&&&&&&&&&& \K{VP}\B{dllllllll}\B{dllllll}\B{dllll}\B{dll}\B{d} \\
	\K{\textbf{K}}\B{d} && \K{NP}\TRi[-8] && \K{Ptkl}\B{d} && \K{NP}\TRi[-7] && \K{PP}\TRi[-5] && \K{\textbf{V}}\TRi[-4] \\
	\K{\textit{obwohl}} && \K{\textit{sich}} && \K{\textit{auch}} && \K{\textit{diese}} && \K{\textit{über die}}\Below{\textit{Jahrzehnte}} && \K{\textit{verändert}}\Below{\textit{haben}} & \\
      }
    }
  \item
    \Tree{
      \K{KP}\B{dd}\B{drrrr} \\
      &&&& \K{VP}\B{dll}\B{d} \\
      \K{\textbf{K}}\B{d} && \K{NP}\TRi[-9] && \K{\textbf{V}}\TRi[-5] \\
      \K{\textit{falls}} && \K{\textit{sie}} && \K{\textit{ermittelt}}\Below{\textit{werden}} & \\
  }
  \item
    \Tree{
      \K{KP}\B{dd}\B{drrrrrrrr} \\
      &&&&&&&& \K{VP}\B{dllllll}\B{dllll}\B{dll}\B{d} \\
      \K{\textbf{K}}\B{d} && \K{NP}\TRi[-7] && \K{NP}\TRi[-7] && \K{NP}\TRi[-5] && \K{\textbf{V}}\B{d} \\
      \K{\textit{weil}} && \K{\textit{der}}\Below{\textit{Staat}} && \K{\textit{ihnen}} && \K{\textit{Wölfe}}\Below{\textit{und Bären}} && \K{\textit{schenkt}} & \\
  }
\end{enumerate}

\Loesung{u115}

Die Frage nach dem Status der Verben wird hier nicht gelöst.

\begin{enumerate}
  \item \Tree[1.5]{
      && \K{\textbf{V}}\B{dl}\B{ddd} \\
      & \K{\textbf{V}}\B{dl}\B{dd} \\
      \K{\textbf{V}}\B{d} \\
      \K{\textbf{V}}\B{d} & \K{\textbf{V}}\B{d} & \K{\textbf{V}}\B{d} \\
    \K{\textit{konfrontiert}}\Below{3} & \K{\textit{werden}}\Below{2} & \K{\textit{sollen}}\Below{1} \\
  }
  \item \Tree[1.5]{
      && \K{\textbf{V}}\B{dl}\B{ddd} \\
      & \K{\textbf{V}}\B{dl}\B{dd} \\
      \K{\textbf{V}}\B{d} \\
      \K{\textbf{V}}\B{d} & \K{\textbf{V}}\B{d} & \K{\textbf{V}}\B{d} \\
    \K{\textit{beruhigt}}\Below{3} & \K{\textit{haben}}\Below{2} & \K{\textit{müssen}}\Below{1} \\
  }
  \item \Tree[1.5]{
      && \K{\textbf{V}}\B{dl}\B{ddd} \\
      & \K{\textbf{V}}\B{dl}\B{dd} \\
      \K{\textbf{V}}\B{d} \\
      \K{\textbf{V}}\B{d} & \K{\textbf{V}}\B{d} & \K{\textbf{V}}\B{d} \\
    \K{\textit{geschenkt}}\Below{3} & \K{\textit{bekommen}}\Below{2} & \K{\textit{hatte}}\Below{1} \\
  }
  \item \Tree[1.5]{
      && \K{\textbf{V}}\B{dl}\B{ddd} \\
      & \K{\textbf{V}}\B{dl}\B{dd} \\
      \K{\textbf{V}}\B{d} \\
      \K{\textbf{V}}\B{d} & \K{\textbf{V}}\B{d} & \K{\textbf{V}}\B{d} \\
      \K{\textit{erlaubt}}\Below{3} & \K{\textit{werden}}\Below{2} & \K{\textit{kann}}\Below{1} \\
  }
\end{enumerate}

\Loesungen{sec:saetze}

\Loesung{u121}

\begin{enumerate}\Lf
  \item{} [Sarah isst den Kuchen alleine auf.]
    \begin{itemize}\Lf
      \item Kf: ---
      \item Vf: Sarah
      \item LSK: isst
      \item Mf: den Kuchen alleine
      \item RSK: auf.
      \item Nf: ---
    \end{itemize}
  \item{} [Man sollte den Tag genießen.]
    \begin{itemize}\Lf
      \item Kf: ---
      \item Vf: Man
      \item LSK: sollte
      \item Mf: den Tag
      \item RSK: genießen.
      \item Nf: ---
    \end{itemize}
  \item{} [Kann mal jemand das Fenster aufmachen?]
    \begin{itemize}\Lf
      \item Kf: ---
      \item Vf: ---
      \item LSK: Kann
      \item Mf: mal jemand das Fenster
      \item RSK: aufmachen.
      \item Nf: ---
    \end{itemize}
  \item Das ist das Eis, [das wir selber gemacht haben].
    \begin{itemize}\Lf
      \item Kf: ---
      \item Vf: ---
      \item LSK: das
      \item Mf: wir selber
      \item RSK: gemacht haben
      \item Nf: ---
    \end{itemize}
  \item{} [Was hat Ischariot gemalt?]
    \begin{itemize}\Lf
      \item Kf: ---
      \item Vf: Was
      \item LSK: hat
      \item Mf: Ischariot
      \item RSK: gemalt?
      \item Nf: ---
    \end{itemize}
  \item{} [Gehst du?]
    \begin{itemize}\Lf
      \item Kf: ---
      \item Vf: ---
      \item LSK: Gehst
      \item Mf: du?
      \item RSK: ---
      \item Nf: ---
    \end{itemize}
  \item{} [Geh!]
    \begin{itemize}\Lf
      \item Kf: ---
      \item Vf: ---
      \item LSK: Geh!
      \item Mf: ---
      \item RSK: ---
      \item Nf: ---
    \end{itemize}
  \item Es ist eine tolle Sommernacht, [denn der Mond scheint hell].
    \begin{itemize}\Lf
      \item Kf: denn
      \item Vf: der Mond
      \item LSK: scheint
      \item Mf: hell
      \item RSK: ---
      \item Nf: ---
    \end{itemize}
  \item{} [Den leckeren Kuchen auf dem Tisch hatte Rigmor sofort entdeckt.]
    \begin{itemize}\Lf
      \item Kf: ---
      \item Vf: Den leckeren Kuchen auf dem Tisch
      \item LSK: hatte
      \item Mf: Rigmor sofort
      \item RSK: entdeckt.
      \item Nf: ---
    \end{itemize}
  \item{} [Obwohl Liv einkaufen wollte], ist nichts im Haus.
    \begin{itemize}\Lf
      \item Kf: ---
      \item Vf: ---
      \item LSK: obwohl
      \item Mf: Liv
      \item RSK: einkaufen wollte
      \item Nf: ---
    \end{itemize}
  \item Kann man feststellen, [wer den Kuchen gegessen hat]?
    \begin{itemize}\Lf
      \item Kf: ---
      \item Vf: wer
      \item LSK: ---
      \item Mf: den Kuchen
      \item RSK: gegessen hat
      \item Nf: ---
    \end{itemize}
\end{enumerate}

\Loesung{u122}

\begin{enumerate}\Lf
  \item Dass der Kuchen gegessen wurde, bedauern alle sehr, die es erfahren haben.
    \begin{itemize}\Lf
      \item Kf: ---
      \item Vf: Dass der Kuchen gegessen wurde
      \item LSK: bedauern
      \item Mf: alle sehr,
      \item RSK: ---
      \item Nf: die es erfahren haben.
    \end{itemize}
  \item Wohin man auch blickt, kann man die Bäume kaum erkennen, denn der Schnee bedeckt alles.
    \begin{itemize}\Lf
      \item Kf: ---
      \item Vf: Wohin man auch blickt,
      \item LSK: kann
      \item Mf: man die Bäume kaum
      \item RSK: erkennen,
      \item Nf: denn der Schnee bedeckt alles.
    \end{itemize}
  \item Geht derjenige, der kommt, auch wieder?
    \begin{itemize}\Lf
      \item Kf: ---
      \item Vf: ---
      \item LSK: Geht
      \item Mf: derjenige, der kommt, auch wieder?
      \item RSK: ---
      \item Nf: ---
    \end{itemize}
  \item Die Kollegen, denen wir nichts vom Kuchen gegeben haben, schimpfen.
    \begin{itemize}\Lf
      \item Kf: ---
      \item Vf: Die Kollegen, denen wir nichts von dem Kuchen gegeben haben,
      \item LSK: schimpfen.
      \item Mf: 
      \item RSK: 
      \item Nf: 
    \end{itemize}
  \item Denn ob es Eis gibt, kann nur einer wissen, der Zugang zur Eismaschine hat.
    \begin{itemize}\Lf
      \item Kf: Denn
      \item Vf: ob es Eis gibt,
      \item LSK: kann
      \item Mf: nur einer
      \item RSK: wissen
      \item Nf: der Zugang zur Eismaschine hat.
    \end{itemize}
  \item Liv will, dass Rigmor ihr von dem Eis abgibt.
    \begin{itemize}\Lf
      \item Kf: ---
      \item Vf: Liv
      \item LSK: will,
      \item Mf: ---
      \item RSK: ---
      \item Nf: dass Rigmor ihr von dem Eis abgibt.
    \end{itemize}
\end{enumerate}

\Loesung{u123}

\begin{enumerate}
  \item Die Lösung entspricht Abbildung~\ref{fig:v2mitpartikel} auf S.~\pageref{fig:v2mitpartikel}.
  \item
    \resizebox{0.9\textwidth}{!}{
      \Tree{
	&&& \K{S}\B{dddll}\B{ddd}\B{drrrrrrrr} \\
	&&&&&&&&&&& \K{VP}\B{ddllllll}\B{ddllll}\B{d} \\
	&&&&&&&&&&& \K{\textbf{V}}\B{dll}\B{d} \\
	& \K{NP\ORii}\TRi[-8] && \K{\textbf{V\ORi}}\B{d} && \K{\Tii} && \K{NP}\TRi[-6] && \K{\textbf{V}}\B{d} && \K{\Ti} \\
	& \K{\textit{Man}} && \K{\textit{sollte}} && \K{} && \K{\textit{den Tag}} && \K{\textit{genießen}} && \K{} \\
      }
    }
  \item
    \Tree{
      \K{FS}\B{ddd}\B{drrrrrrrrr} \\
      &&&&&&&&& \K{VP}\B{ddllllllll}\B{ddllllll}\B{ddllll}\B{d} \\
      &&&&&&&&& \K{\textbf{V}}\B{dll}\B{d} \\
      \K{\textbf{V\ORi}}\B{d} & \K{Ptkl}\B{d} && \K{NP}\TRi[-6] && \K{NP}\TRi[-4] && \K{\textbf{V}}\B{d} && \K{\Ti} \\
      \K{\textit{Kann}} & \K{\textit{mal}} && \K{\textit{jemand}} && \K{\textit{das Fenster}} && \K{\textit{aufmachen}} && \K{} \\
    }
  \item \label{it:9829}
    \resizebox{0.9\textwidth}{!}{
      \Tree{
	&&& \K{S}\B{dddll}\B{ddd}\B{drrrrrrrr} \\
	&&&&&&&&&&& \K{VP}\B{ddllllll}\B{ddllll}\B{d} \\
	&&&&&&&&&&& \K{\textbf{V}}\B{dll}\B{d} \\
	& \K{NP\ORii}\TRi[-7] && \K{\textbf{V\ORi}}\B{d} && \K{NP}\TRi[-5] && \K{\Tii} && \K{\textbf{V}}\B{d} && \K{\Ti} \\
	& \K{\textit{Was}} && \K{\textit{hat}} && \K{\textit{Ischariot}} && \K{} && \K{\textit{gemalt}} && \K{} \\
      }
    }
  \item
    \Tree{
      \K{FS}\B{dd}\B{drrrr} \\
      &&&& \K{VP}\B{dll}\B{d} \\
      \K{\textbf{V\ORi}}\B{d} && \K{NP}\TRi[-7] && \K{\Ti} \\
      \K{\textit{Gehst}} && \K{\textit{du}} && \K{} \\
    }
  \item
    \resizebox{0.9\textwidth}{!}{
      \Tree{
	&&& \K{S}\B{dddll}\B{ddd}\B{drrrrrrrrrr} \\
	&&&&&&&&&&&&& \K{VP}\B{ddllllllll}\B{ddllllll}\B{ddllll}\B{d} \\
	&&&&&&&&&&&&& \K{\textbf{V}}\B{dll}\B{d} \\
	& \K{NP\ORii}\TRi[-1] && \K{\textbf{V\ORi}}\B{d} && \K{NP}\TRi[-4] && \K{\Tii} && \K{AdvP}\TRi[-5] && \K{\textbf{V}}\B{d} && \K{\Ti} \\
	& \K{\textit{Den leckeren}}\Below{\textit{Kuchen auf dem Tisch}} && \K{\textit{hatte}} && \K{\textit{Rigmor}} && \K{} && \K{\textit{sofort}} && \K{\textit{entdeckt}} && \K{} \\ 
      }
    }
\end{enumerate}

\Loesung{u124}

\begin{enumerate}
  \item\label{it:99001}
    \resizebox{0.9\textwidth}{!}{
      \Tree{
        &&& \K{S}\B{dddll}\B{ddd}\B{drrrrr}\B{drrrrrrr} \\
        &&&&&&&& \K{VP}\B{ddlll}\B{ddll}\B{ddl}\B{dd} && \K{RS\ORiii}\TRi[-4] \\
        &&&&&&&&&& \K{\textit{die \ldots haben}} &\\
        & \K{KP\ORii}\TRi[-2] && \K{\textbf{V\ORi}}\B{d} && \K{NP}\TRi[-7] & \K{\Tii} & \K{Ptkl}\B{d} & \K{\Ti} & \\
        & \K{\textit{Dass \ldots wurde}} && \K{\textit{bedauern}} && \K{\textit{alle} \Tiii} && \K{\textit{sehr}} & \\ 
      }
    }\\

    Hier liegen zwei Komplikationen vor.
    Einerseits wurde ein Komplementsatz ins Vorfeld gestellt, und es ist zunächst etwas ungewohnt, eine Konstituente als extrahiert zu betrachten, die so gut wie nie in der Basisposition (also nicht extrahiert und damit innerhalb der VP) vorkommt.
    Außerdem muss aus dem NP-Satzglied in einer zusätzlichen Operation der RS nach rechts (ins Nf) gestellt werden.
    Dafür gibt es kein Schema, aber wir unterstellen mit dieser Lösung, dass es ein erweitertes S-Schema gibt, dass optional die Bewegung einer dritten Konstituente in eine Nachfeldposition erlaubt.
    Dass der RS im Diagramm nicht auf der Grundlinie liegt, hat rein darstellungstechnische Gründe.

  \item\label{it:99002}
    \Tree{
      &&& \K{S}\B{dddll}\B{ddd}\B{drrrr} \\
      &&&&&&& \K{VP}\B{ddll}\B{dd} \\
      &&&&&&& \\
      & \K{NP\ORii}\TRi[-1] && \K{\textbf{V\ORi}}\B{d} && \K{\Tii} && \K{\Ti} \\
      & \K{\textit{Die Kollegen,}}\Below{\textit{denen \ldots haben}} && \K{\textit{schimpfen}} && \\
    }\\

    Hier liegt eigentlich nur eine recht lange NP mit RS vor, die ins Vf gestellt wurde.
    Weil es nur ein finites Verb und sonst keine Ergänzungen und Angaben gibt, bleibt das Mf und die RSK leer.
    Phrasenstrukturell gesehen haben wir es also mit einer komplett entleerten VP zu tun.
  \item
    \resizebox{0.9\textwidth}{!}{
      \Tree{
	&&& \K{S}\B{dddll}\B{ddd}\B{drrrrrr}\B{drrrrrrrr} \\
	&&&&&&&&& \K{VP}\B{ddllll}\B{ddll}\B{dd} && \K{KP\ORiii}\TRi[-1] \\
	&&&&&&&&&&& \K{\textit{dass Rigmor\ldots}}\Below{\textit{\ldots abgibt}} & \\
	& \K{NP\ORii}\TRi[-7] && \K{\textbf{V\ORi}}\B{d} && \K{\Tii} && \K{\Tiii} && \K{\Ti} & \\
	& \K{\textit{Liv}} && \K{\textit{will}} & \\
      }
    }\\

    In diesem Fall ist ein Komplementsatz ins Nf gestellt worden.
    Wir benötigen einen ähnlichen Mechanismus wie in Beispiel \ref{it:99001}, in diesem Fall um den Komplementsatz nach rechts zu bewegen.
    Wie in Beispiel \ref{it:99002} bleibt eine leere VP zurück, was in der Felderanalyse einem leeren Mf und einer leeren RSK entspricht.
\end{enumerate}

\Loesung{u125}

\begin{enumerate}
  \item
    \Tree{
      & \K{NP}\B{dddl}\B{ddd}\B{drr} \\
      &&& \K{RS}\B{dd}\B{drrrrr} \\
      &&&&&&&& \K{VP}\B{dlll}\B{dl}\B{d} \\
      \K{Art}\B{d} & \K{\textbf{N}}\B{d} && \K{NP\ORi}\TRi[-9] && \K{NP}\TRi[-9] && \K{\Ti} & \K{\textbf{V}}\B{d} \\
      \K{\textit{das}} & \K{\textit{Buch}} && \K{\textit{das}} && \K{\textit{ich}} &&& \K{\textit{lese}} & \\
    }
  \item
    \Tree{
      & \K{NP}\B{ddd}\B{drr} \\
      &&& \K{RS}\B{dd}\B{drrrr} \\
      &&&&&&& \K{VP}\B{dll}\B{dl}\B{d} \\
      & \K{\textbf{N}}\B{d} && \K{PP\ORi}\TRi[-6] && \K{NP}\TRi[-9] & \K{\Ti} & \K{\textbf{V}}\TRi[-5] \\
      & \K{\textit{Menschen}} && \K{\textit{auf die}} && \K{\textit{wir}} && \K{\textit{vertrauen}}\Below{\textit{können}} & \\
    }
  \item
    \resizebox{0.9\textwidth}{!}
    {
      \Tree{
	& \K{NP}\B{dddl}\B{ddd}\B{drrrr} \\
	&&&&& \K{RS}\B{d}\B{drrrr} \\
	&&&&& \K{NP\ORi}\B{dll}\B{d} &&&& \K{VP}\B{dll}\B{dl}\B{d} \\
	\K{Art}\B{d} & \K{\textbf{N}}\B{d} && \K{NP}\TRi[-7] && \K{\textbf{N}}\B{d} && \K{NP}\TRi[-8] & \K{\Ti} & \K{\textbf{V}}\TRi[-5] \\
	\K{\textit{die}} & \K{\textit{Komm.}} && \K{\textit{deren}} && \K{\textit{Kuchen}} && \K{\textit{wir}} && \K{\textit{gegessen}}\Below{\textit{haben}} & \\
      }
    }
\end{enumerate}

\Loesung{u126}

Wenn der Relativsatz\slash eingebettete w-Satz [\textit{wer dass kommt}] lautet, handelt es sich um eine KP, bei der das Relativ-Element über den K hinausbewegt wird.
Es wäre also in etwa eine Struktur wie in Abbildung~\ref{fig:relativsatzzmitkomp}.
Damit ist das Schema für RS in solchen Varianten nicht mehr erforderlich, und man könnte einheitlich Nebensatzbildung als V-Bewegung analysieren, der eine XP-Bewegung folgt, genau wie im unabhängigen Aussagesatz.

\begin{figure}[!htbp]
  \begin{center}
    \Tree{
      && \K{KP}\B{ddl}\B{dd}\B{drr} \\
      &&&& \K{VP}\B{dl}\B{d} \\
      & \K{NP\ORi}\TRi[-9] & \K{\textbf{K}}\B{d} & \K{\Ti} & \K{\textbf{V}}\B{d} \\
      & \K{\textit{wer}} & \K{\textit{dass}} && \K{\textit{kommt}} & \\
    }
  \end{center}
  \caption{Analyse eines Relativsatzes mit Komplementierer}
  \label{fig:relativsatzzmitkomp}
\end{figure}

\Loesung{u127}

Die Sätze zeigen ein perfektes Zusammenspiel von Syntax, Prosodie und Orthographie bzw.\ Graphematik.
Im intakten Verbkomplex stehen Verbpartikel (\textit{zurück}) und Verb (\textit{bleibt}) in einer prosodisch und syntaktisch kohärenten Position und werden dementsprechend zusammengeschrieben.
Wurde das Verb in die LSK und die Verbpartikel in das Vf bewegt, wird diese Einheit zerstört und nicht wiederhergestellt, auch wenn sich zufällig dieselbe lineare Abfolge ergibt (\textit{Zurück bleibt \ldots}).
Damit einher geht eine deutliche prosodische Grenze (sehr vereinfacht: eine Pause) zwischen der Partikel und dem Verb.

\Loesungen{sec:relationenpraedikate}

\Loesung{u131}

\begin{enumerate}\Lf
  \item \textit{Mausi}: Subjekt\\
    \textit{den Brief}: Akkusativobjek\\
    \textit{an ihre Mutter}: Präpositionalobjekt
  \item \textit{dass der Brief nicht angekommen ist}: Subjekt
  \item \textit{Der Grammatiker}: Subjekt\\
    \textit{dass die Modalverben eine gut definierbare Klasse sind}: Objektsatz
  \item \textit{Den Eisschrank zu plündern}: Subjekt
  \item \textit{Wen jemand bewundert}: Akkusativobjekt\\
    \textit{wer die Bewunderung empfindet}: Subjekt
  \item \textit{Ich}: Subjekt\\
    \textit{den Dart}: Akkusativobjekt\\
    \textit{in ein Triple-Feld}: Präpositionalobjekt
  \item \textit{die durstigen Rottweiler}: Akkusativobjekt
  \item \textit{Der immer die dummen Fragen gestellt hat}: Subjekt\\
    \textit{Matthias}: Akkusativobjekt\\
    \textit{ob das wirklich Musik sein soll}: Objektsatz 
  \item \textit{Vor dem Hund}: Präpositionalobjekt\\
    \textit{man}: Subjekt\\
    \textit{niemanden}: Akkusativobjekt
  \item \textit{Es}: Subjekt
\end{enumerate}

\Loesung{u132}

\begin{enumerate}\Lf
  \item \textit{kreischen}: unergativ
  \item \textit{schenken}: ditransitiv
  \item \textit{nützen}: unakkusatives Dativverb
  \item \textit{trocknen}: unakkusativ oder transitiv
  \item \textit{kosten}: transitiv (mit unagentiver Subjektsrolle)
  \item \textit{antworten}: präpositional dreiwertig
  \item \textit{arbeiten}: unergativ
  \item \textit{bedürfen}: Verb mit unagentivem Nominativ und Genitiv (kein eigener Name)
  \item \textit{blitzen}: unakkusativ
  \item \textit{verzeihen}: ditransitiv
  \item \textit{abtrocknen}: transitiv
  \item \textit{überlaufen}: unergativ (im Sinn von Fahnenflucht usw.) oder unakkusativ (im Sinn von übergelaufenen Töpfen usw.)
  \item \textit{fallen}: unakkusativ
  \item \textit{verschieben}: transitiv oder präpositional dreiwertig (wenn an einen Ort\slash auf einen Zeitpunkt verschoben wird)
  \item \textit{schwindeln}: Verb ohne Subjekt und mit Experiencer-Akkusativ
\end{enumerate}

\Loesung{u133}

\textit{Wiegen} ist in dieser Verwendung ein Verb mit einem unagentiven Nominativ und einem Akkusativ.
Bei Verben wie \textit{wundern} haben wir einen unagentiven Nominativ und einen Experiencer-Akkusativ.
Bei beiden ist der Akkusativ nicht weglassbar und damit valenzgebunden.
Mangels eines agentiven Subjekts können die Verben erwartungsgemäß nicht passiviert werden, und man hätte deswegen wahrscheinlich Bedenken, von \textit{transitiven Verben} zu sprechen.

\Loesung{u134}

\begin{enumerate}\Lf
  \item \textit{mir}: Nutznießer-Dativ
  \item \textit{Dem Verein}: normal regiert
  \item \textit{ihrer Mutter}: normal regiert
  \item \textit{Dem Grammatiker}: Bewertungsdativ
  \item \textit{unserem Hund}: Pertinenzdativ
  \item \textit{der Oma}: Bewertungsdativ\\
    \textit{dem Hund}: anderer Dativ
\end{enumerate}

\Loesung{u135}

Beim Pertinenzdativ wird neben dem Dativ im Rahmen der Valenzänderung immer auch die PP hinzugefügt, die das Körperteil bezeichnet.
Das ist bei diesem Dativ nicht der Fall.
Das wäre zumindest ein Argument, diese Fälle von den Pertinenzdativen zu unterscheiden.

\Loesung{u136}

Die Bewertung erfolgt hier durch * für die Sätze, bei denen der Test nicht anschlägt und von valenzgebundenen PPs ausgegangen werden muss.

\begin{enumerate}\Lf
  \item $*$ Matthias interessiert sich. Dies geschieht für elektronische Musik.
  \item $\phantom{*}$ Die Band spielt. Dies geschieht nach den \textit{Verschwundenen Pralinen}.
  \item $\phantom{*}$ Matthias spielt. Dies geschieht für eine Jazzband.
  \item $*$ Der Rottweiler bewahrte Marina. Dies geschah vor der Langeweile.
  \item $*$ Doro fragt. Dies geschieht nach den verschwundenen Pralinen.
  \item $*$ Ich bat sie. Dies geschah um einen Rat.
  \item $\phantom{*}$ Ihr Rottweiler baute sich auf. Dies geschah vor dem Schrank mit dem Hundefutter.
\end{enumerate}

\Loesung{u136a}

Das Hilfsverb \textit{werden} regiert den ersten Status, anders als \textit{haben} (Perfekt), das den dritten Status regiert.
Daher ist ein Ersatzinfinitiv in diesem Satz nicht möglich, denn das von \textit{wird} regierte \textit{müssen} könnte ja nie anders als im Infinitiv stehen.
Trotzdem liegt hier eine Oberfeldumstellung vor (132), denn die typische Abfolge wäre \textit{lesen müssen wird} (321).
Während der Ersatzinfinitiv also immer die Oberfeldumstellung nach sich zieht, gibt es auch optionale Oberfeldumstellungen ohne Ersatzinfinitv.

\Loesung{u137}

\begin{enumerate}\Lf
  \item $\phantom{*}$ dass Michelle Marina hilft, den Hund zu verstehen.
  \item $*$ dass Michelle fährt neues Hundefutter holen.
  \item $\phantom{*}$ dass Michelle verspricht, den Hund in Verwahrung zu nehmen.
  \item $*$ dass Michelle kann sehr gut mit Hunden umgehen.
  \item $*$ dass Michelle sieht den Hund spielen.
  \item $\phantom{*}$ dass Michelle versucht, den Hund gut zu erziehen.
\end{enumerate}

\Loesung{u138}

\begin{enumerate}\Lf
  \item \textit{Matthias} (Dativobjekt)
  \item \textit{Matthias} (Dativobjekt)
  \item \textit{Doro} (Akkusativobjekt)
  \item \textit{Doro} (Subjekt) oder \textit{Matthias} (Akkusativobjekt) oder beide
  \item \textit{Matthias} (Präpositionalobjekt) oder arbiträre Kontrolle
\end{enumerate}

\Loesung{u139}

Ungrammatisch sind: 1, 4, 5, 9.

\Loesungen{sec:phonschrift}

\Loesung{u141}

Die Lösung versteht sich als beispielhaft.
Die mit * gekennzeichneten Wörter sind allerdings jeweils vermutlich die einzigen (oder eines von wenigen), die alle Bedingungen für die jeweiligen Tabellenzellen erfüllen.
Die Vokale /\textipa{\o}/ und /\textipa{\oe}/ sind also leicht eingeschränkt verteilt.

\begin{center}
  \begin{tabular}{lllll}
    \lsptoprule
    & & & \textbf{/y/, /Y/} & \textbf{/\o/, /\oe/} \\ 
    \midrule
    \multirow{4}{*}{\rotatebox{90}{\textbf{kurz}}}

      & \multirow{2}{*}{\rotatebox{90}{\textbf{offen}}}
	& \textbf{einsilb.}  & \textit{\Nono}  & \textit{\Nono} \\
      && \textbf{zweisilb.}  & \textit{Mü.cke} & \textit{Lö.ffel} \\

      & \multirow{2}{*}{\rotatebox{90}{\textbf{gesch.}}}
	& \textbf{einsilb.}  & \textit{dünn}   & \textit{Löss}*  \\
      && \textbf{zweisilb.}  & \textit{Mün.del} & \textit{Mör.tel}* \\


      \midrule

      \multirow{4}{*}{\rotatebox{90}{\textbf{lang}}}

      & \multirow{2}{*}{\rotatebox{90}{\textbf{offen}}}
	& \textbf{einsilb.}  & \textit{früh}   & \textit{Bö}* \\
      && \textbf{zweisilb.}  & \textit{Müh.le} & \textit{Möh.re} \\

      & \multirow{2}{*}{\rotatebox{90}{\textbf{gesch.}}}
	& \textbf{einsilb.}  & \textit{wüst}  & \textit{schön} \\
      && \textbf{zweisilb.}  & (\textit{pflüg.te}) & (\textit{föhn.te}) \\

    \lspbottomrule
  \end{tabular}
\end{center}

\Loesung{u142} Diphthonge sind prinzipiell lang.
Damit stellt sich die Frage nach Dehnungs- und Schärfungsschreibungen bei ihnen nie.

\Loesung{u143} Der glottale Plosiv ist in den zugrundeliegenden Formen nicht vorhanden, und die Buchstaben verschriften zugrundeliegende Segmente.
Daher ist es plausibel, dass der glottale Plosiv nicht verschriftet wird.

\Loesung{u144} 

Die Dehnungsschreibungen sind hier in (~), die Schärfungsschreibungen in [~] gesetzt.
Bei \textit{ha}[\textit{tt}]\textit{e} liegt ein etwas anderer Fall vor, als im Buch primär besprochene wurde, weil das Silbengelenk mit der Morphgrenze zusammenfällt und damit nicht innerhalb eines Simplex liegt.

\begin{sloppypar}
\begin{enumerate}\Lf
  \item Auf dem W(oh)nungsmarkt ist Entspa[nn]ung eingek(eh)rt.
  \item Der König von Schweden ha[tt]e angeblich Kontakte zur Unterwelt.
  \item Eine Leseprobe endete in einer wüsten Schlägerei.
  \item Unter einer einstweiligen Verfügung ka[nn] sich Ischariot nichts vorste[ll]en.
  \item Mit M(öh)ren ka[nn] Vane[ss]a ihr Pferd glü[ck]lich machen.
  \item Sie fragen sich jetzt sicher, wer die Sta[ll]pflege überni[mm]t.
  \item Pa[ss]en S(ie) beim Einsteigen auf (Ih)r Kn(ie) auf. 
\end{enumerate}
\end{sloppypar}

Alle Dehnungsschreibungen bis auf \textit{Sie}, \textit{ihr} und \textit{Knie} sind hier optional.
Es könnte also \textit{Wonungsmarkt}, \textit{eingekert} und \textit{Mören} geschrieben werden.
Wenn alle Dehnungsschreibungen obligatorisch wären, ergäbe sich \zB folgendes Bild.

\begin{enumerate}\Lf
  \item Auf dem Wohnungsmarkt ist Entspannung eingekehrt.
  \item Der Köhnig von Schweeden hatte angehblich Kontakte zur Unterwelt.
  \item Eine Leeseprohbe endete in einer wühsten Schlähgerei.
  \item Unter einer einstweiligen Verfühgung kann sich Ischariot nichts vorstellen.
  \item Mit Möhren kann Vanessa ihr Pferd glücklich machen.
  \item Sie frahgen sich jetzt sicher, wer die Stallpfleege übernimmt.
  \item Passen Sie beim Einsteigen auf Ihr Knie auf. 
\end{enumerate}

\Loesung{u145} Weil das Prinzip der Onset-Maximierung gilt, wird ein einzelner vorhandener Konsonant immer in den Onset der zweiten Silbe gezwungen.
Wenn es mehr als einen Konsonanten gibt, kann zwar auch die Coda der Erstsilbe besetzt werden, aber nur nachdem die Coda der Zweitsilbe bereits gefüllt ist.

\Loesung{u146}

\begin{enumerate}\Lf
  \item \textit{ch} steht nicht in initialer Position, \textit{th} kommt sonst nur an der Silbengrenze komplexer Wörter (\textit{Buntheit}) vor.
  \item Die Schreibung an sich ist eigentlich nicht besonders auffällig.
    Sie könnte ein Simplex /\textipa{gEnK@}/ verschriften.
    Das \textit{g} kodiert hier aber das Segment /\textipa{Z}/, das je nach Auffassung nicht zum Kern gehört.
    Außerdem kodiert das \textit{en} bei vielen Sprechern einen nasalen Vokal, den es im Kern ebenfalls nicht gibt.
    Auch wenn die Schreibung nicht auffällig aussieht, sind die Korrespondenzen der Buchstaben zu den Segmenten hier dem Kern fremd.
  \item Es wird die Realisierung \textipa{[gonoK\o:]} angenommen.
    Das \textit{rr} wäre im Kern eine Silbengelenkschreibung, ist es hier aber nicht.
    Das \textit{h} hat hier keine Funktion nach den besprochenen Regularitäten.
  \item Je nach Aussprache verschriften hier \textit{en} und \textit{ant} wieder nasale Vokale.
    Die Situation ist also ähnlich wie bei \textit{Genre}.
  \item Die Folge \textit{ou} korrespondiert hier zu /\textipa{u:}/.
    Außerdem wäre \textit{w} der angemessene Buchstabe für /\textipa{v}/.
    Wenn \textipa{[su:v@n\t{i5}]} realisiert wird, ist auch die Korrespondenz des \textit{s} im Wortanlaut nicht konform zum Kern, weil im Wortanlaut \textit{s} immer einen stimmhaften Frikativ verschriftet.
  \item Statt \textit{sch} steht \textit{sh}.
  \item Wie in \textit{chthonisch} ist \textit{th} auffällig.
  \item Der Buchstabe \textit{y} kommt im Kern nicht vor.
\end{enumerate}

\Loesungen{sec:andereschrift}

\Loesung{u151} Diese Wörter haben alle (ggf.\ aus verschiedenen Gründen) keine Formen, in denen die Schärfungsschreibung tatsächlich als Silbengelenkschreibung auftritt.
Wörter wie \textit{dann} und \textit{wenn} flektieren überhaupt nicht, und \textit{Mett} usw.\ haben keinen Plural (*\textit{Mette}), so dass keine trochäischen Flexionsformen anzutreffen sind.
Interessanterweise gibt es im Bereich der Stoffsubstantive (\textit{Butter}, \textit{Öl}, \textit{Wasser}, \textit{Wein}), die aus semantischen Gründen nur sehr schlecht einen Plural bilden können, kaum kurzvokalische Einsilbler.

\Loesung{u153} In diesem Lösungstext wurden Kommas eingefügt.
Hinter jedem Komma steht in eckigen Klammern seine Funktion [\textit{Koord}] für Koordination, [\textit{Inf}] für Infinitiv, [\textit{NS}] für Nebensatz, [\textit{Par}] für Parenthese.
Kommas, die das Ende einer Struktur anzeigen, sind mit [\textit{Ende}] markiert, und bei den Nebensätzen ist nach \textit{NS} die Art des Nebensatzes angegeben: \textit{Adv} für Adverbialsätze, \textit{Komp} für Komplementsätze und \textit{Rel} für Relativsätze.

\vspace{0.5cm}

\begin{sloppypar}
\begin{quote}
Eine molekulare Ratsche oder auch Brownsche Ratsche ist eine Nanomaschine,[NS Rel] die aus brownscher Molekularbewegung (also aus Wärme) gerichtete Bewegung erzeugt.
Dies kann nur funktionieren,[NS Adv] wenn von außen Energie in das System gebracht wird.
Solche Systeme werden in der Literatur meistens Brownsche Motoren (siehe Literatur/Links) genannt.
Eine molekulare Ratsche ohne von außen zugeführter Energie wäre ein Perpetuum Mobile zweiter Art und funktioniert somit nicht.
Der Physiker Richard Feynman zeigte in einem Gedankenexperiment 1962 als erster,[NS Komp] wie eine molekulare Ratsche prinzipiell aussehen könnte,[Koord] und erklärte,[NS Komp] warum sie nicht funktioniert.

Eine molekulare Ratsche besteht aus einem Flügelrad und einer Ratsche mit Sperrzahn.
Die gesamte Maschine muss sehr klein sein (wenige Mikrometer),[NS Adv] damit die Stöße des umgebenden Gases keinen nennenswerten Einfluss auf sie haben.
Die Funktionsweise ist denkbar einfach:
Ein Gasteilchen,[NS Rel] das das Flügelrad beispielsweise so trifft,[Par] wie durch den grünen Pfeil markiert,[Ende:Par] bewirkt ein Drehmoment,[NS Rel] das sich über die Achse auf die Ratsche überträgt und diese eine Stellung weiterdrehen kann.
Ein Teilchen,[NS Rel] das wie durch den roten Pfeil markiert auftrifft,[Ende:NS Rel] bewirkt keine Drehung,[NS Adv] da der Sperrzahn die Ratsche blockiert.
Die molekulare Ratsche sollte also aus Wärmeenergie eine gerichtete Bewegung erzeugen,[NS Rel] was aber nach dem zweiten Hauptsatz der Thermodynamik nicht möglich ist.

Der Sperrzahn funktioniert nur,[NS Adv] wenn er mit einer Feder gegen die Ratsche gedrückt wird.
Auch er unterliegt dem Bombardement der Brownschen Molekularbewegung.
Wird er durch diese ausgelenkt,[NS Adv] beginnt er auf die Ratsche zu schlagen,[NS Rel] was zu einem Nettodrehmoment entgegen der zuvor angenommen Drehrichtung führt.
Die Wahrscheinlichkeit für die Auslenkung des Sperrzahns,[NS Rel] die groß genug ist,[Inf] um eine Ratschenposition zu überspringen,[Ende:Inf] ist $exp(-\Delta E/k_BT)$,[NS Adv] wobei $\Delta E$ die Energie,[NS Rel] die benötigt wird,[Inf] um die Feder des Sperrzahns auszulenken,[Ende:Inf] ist,[Koord] $T$ die Temperatur und $k_B$ die Boltzmann-Konstante.
Die Drehung über das Flügelrad muss aber auch die Feder spannen,[Inf] um in die nächste Position der Ratsche zu gelangen,[Koord unabhängiger Sätze] das heißt,[NS Komp] die Wahrscheinlichkeit ist ebenfalls $exp(-\Delta E/k_BT)$.
Folglich dreht sich die Ratsche im Mittel nicht.

Anders sieht es aus,[Ende:NS Adv] wenn ein Temperaturunterschied zwischen Flügelscheibe und Ratsche vorliegt.
Ist die Umgebung des Flügelrades wärmer als die der Ratsche,[NS Adv] dreht sich die molekulare Ratsche wie zuvor angenommen.
Ist die Umgebung der Ratsche wärmer,[Ende:NS Adv] dreht sich die Maschine in die entgegengesetzte Richtung.
\end{quote}
\end{sloppypar}

In den letzten beiden Sätzen sind die Adverbialsätze vorangestellt, und es gibt daher nur eine Markierung ihres rechten Randes.

\Loesung{u154} Die Wörter sind Konjunktionen.
\textit{Aber} hat im Grunde syntaktisch dasselbe Verhalten wie \textit{und} (\textit{spät, aber rechtzeitig}).
Bei \textit{sondern} muss das erste Konjunkt irgendwie negiert sein (\textit{nicht sauber, sondern rein}), aber ansonsten verbindet es auch einfach zwei syntaktisch gleichartige Konstituenten.
Das Komma markiert also am ehesten eine Koordination, tritt aber zusammen mit einer Konjunktion auf.
Sonst ist diese doppelte Markierung der Koordinationsstelle nur möglich und üblich, wenn syntaktisch unabhängige Sätze mit Konjunktionen verbunden werden.
Es sind bestimmte semantische Klassen, \zB die adversativen (also gegenüberstellenden) Konjunktionen, die das zusätzliche Komma angeblich verlangen.
Das ist eigentlich ungewöhnlich, da diese Regel damit wohl die einzige semantisch motivierte Regel für syntaktische Zeichen im Deutschen ist.

