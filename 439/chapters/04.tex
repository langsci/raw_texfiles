\chapter{Conclusions and outlook}\label{chapter4}
In the preceding chapters, I have surveyed the different functions that expressions for the phasal polarity concept \textsc{still} have in the languages of the world. I defined this concept as combining the notion of \isi{persistence} with the evocation of an alternative \isi{discontinuation} (\Cref{sectionDefinition}), and based my survey on a global \isi{sample} of 76 languages from 45 distinct phyla (\Cref{sectionSample}).

In \Cref{chapter2} I cast the spotlight on uses with a time-related core of meaning. I started with an examination of a set of uses that lie on the borderline between the concept of \textsc{still} and functional extensions (\Cref{sectionFringes}). This included an in-depth discussion of the use of \textsc{still} expressions in contexts of scalar\is{scale} variables (e.g. \lq still … left\rq{}, \lq still only …\rq), the alleged status of which as a distinct function had been the subject of a heated debate during the first wave of research on phasal polarity; see below. I then turned to uses of \textsc{still} expressions that infringe on the territory of other phasal polarity concepts (\Cref{sectionOtherPhPConcepts}), namely the signalling of \textsc{not yet} without a negator\is{negation}\is{not yet} and the phenomenon of interrogative \lq{}yet\rq{}. Afterwards I surveyed functions that lie outside the realm of phasal polarity altogether. These encompass various kinds of adverbial modifications (\Cref{sectionAdverbial}), such as future-oriented \lq eventually\rq{},\is{prospective} iterative/restitutive \lq again\rq{},\is{repetition} or the construal of a situation as precedential or preliminary \lq first, for now\rq{},\is{precedence} to give just a few examples. Lastly, I examined several functions of \textsc{still} expressions in which they modify a temporal frame setter (\Cref{sectionConnectiveFrameSetters}), for instance scalar\is{scale} additive uses specialised for temporal foci (i.e. \lq as late as\rq{}, \lq{}as early as\rq{}, \lq{}as far removed as\rq{}) or the marking of simultaneous duration \lq{}while\rq{}.\is{simultaneity}

As can be gathered, \textsc{still} expressions cover a broad spectrum of time-related functions. Several of these may even appear contrary at first glance, such as \lq first, for now\rq{ }vs. \lq eventually\rq{}.\is{prospective} In some instances, this fact can be traced back to each function in a pair sharing a common denominator with one rather than the other meaning component of \textsc{still}. Thus, the \lq first, for now\rq{ }use and the phasal polarity concept both evoke a contrast between the time under discussion and a possible subsequent change (\Cref{sectionFirst}). The \lq eventually\rq{ }use,\is{prospective} on the other hand, profiles the notion of continuity with preceding developments (\Cref{sectionProspective}; also see below). In other cases, such seemingly opposite functions have their roots in different domains of operation. For instance, \textsc{still} expressions marking \lq as late as\rq{ }go back to equating the time under discussion with an advanced runtime\is{scale} of the situation depicted in the clause (\Cref{sectionTimeScalar}). A structurally similar function, on the other hand, often suggests earliness by stressing the \isi{persistence} of the temporal frame itself and hence the non-attainment of later times (\Cref{sectionTemporalFrameTT}).

In terms of frequency and distribution, some time-related functions are limited to individual sample expressions and languages, such as \ili{Kekchí} \textit{toj} as temporal \lq until\rq{ }(\Cref{sectionTimeScalarRestrictive}). Many others are found across linguistic and/or geographic boundaries, although their recurrent nature has often gone unnoticed. For instance, the recruitment of \textsc{still} expressions in near past \lq have just Verb-ed\rq{} constructions is attested for unrelated languages from distant parts of the world (\Cref{sectionRemotenessPast}),\is{remoteness} but has only received scant attention thus far, always limited to descriptions of individual cases. Which functions are the most common is somewhat dependent on where to draw the line.\is{scale|(} As alluded to above, the first wave of research on phasal polarity saw an intense debate about whether the use of the relevant expressions in the contexts of scalar variables (e.g. \lq still … left\rq{}, \lq still only …\rq) is to be understood as an instantiation of a distinct scalar function. If the question is answered positively, then this would constitute the singlemost common functional extension in my sample. However, based on my examination of the sample data, and elaborating on arguments brought forward by \citeauthor{vanderAuwera1991BeyondDuality} (\citeyear{vanderAuwera1991BeyondDuality}, \citeyear{vanderAuwera1993}), \textcite{Garrido1992}, and \textcite[ch. 3.1]{MosegaardHansen2008}, I reasoned that there is no principled need to stipulate a distinct scalar function (\Cref{sectionScalar}).\is{scale|)} Against this backdrop, \Cref{TableConclusionsTemporal} lists the three most common temporal-aspectual uses in my sample. Given their prominent status, it is worthwhile briefly summarising a few key points as they relate to each of them.


\begin{table}[t]
\caption{Most common temporal and aspectual uses \label{TableConclusionsTemporal}}
\small
\begin{tabular}{l ccc l}
	\lsptoprule
	         &   Macro- & \\
	Function &  {areas} & \multicolumn{1}{l}{Languages} &  \multicolumn{1}{l}{Expressions} & Discussed in\\\midrule
	\textsc{not} \textsc{yet} without \isi{negation} & 6 & 22 & 22 & \Cref{sectionNotYet}\\
	Repetition\is{repetition} & 6 & 16 & 17 & \Cref{sectionIterative}\\
	Prospective\is{prospective} \lq eventually\rq{} & 4 & 14 & 14 & \Cref{sectionProspective}\\
	\lspbottomrule
\end{tabular}
\end{table}

\is{not yet|(} 
The first use in \Cref{TableConclusionsTemporal}, the marking \textsc{not yet} in the absence of a negator,\is{negation} is doubtlessly motivated by the fact that \textsc{still} and \textsc{not yet} are \lq\lq exactly the same … retrospectively continuative and prospectively geared towards possible change" \parencite[40]{vanderAuwera1998}, the sole difference lying in polarity. On a more fine-grained level, this use comes in several overarching types (also see \cite{PersohnNotYet}). The first type is characterised by contexts lacking an overt predicate. This often involves disjunction, for example in polar questions that follow an \lq (already) or still > (already) or not yet\rq{ }pattern. Here, the only meaningful interpretation of the  \textsc{still} expression is in relation to the negative member of a pair of propositions. In other cases, context can be said to strongly favour a negative\is{negation} reading.\is{infinitive|(} In a second type of \textsc{still} as \textsc{not yet}, an expression goes together with a less-than-finite and/or dependent predicate. This finds a straightforward explanation in a conventionalised inference from a persistently\is{persistence} pending situation to a continuing negative\is{negation} state (as suggested before me by \cite{Gueldemann1996}, \citeyear[129–130]{Gueldemann1998}; \cite[148]{Nurse2008}; \cite{VeselinovaDevos2021}).\is{infinitive|)}\is{actionality|(} In yet a third type, polarity depends on the situation's internal temporal structure. Here, a motivated explanation can be found in the combinatory possibilities and/or possible readings of different actional classes with the notion of persistence,\is{persistence} in conjunction with particularities of the respective tense-aspect\is{tense}\is{aspect} systems.\is{actionality|)}\is{syntax|(} In a fourth type, an expression's function as affirmative \textsc{still} or negative \textsc{not yet} is determined by its position in the clause. This pattern is rare and only attested in Alor-Pantar, where comparative evidence points towards the erosion of an embracing pattern of \isi{negation} as its source.\is{syntax|)} Lastly, \textsc{still} as \textsc{not yet} without a negator was shown to play a role in the history of \ili{Thai} \textit{yaŋ} as interrogative \lq{}yet\rq{ }(\Cref{sectionInterrogativeYet}), and it was also observed in temporal clauses\is{temporal clause} of \isi{precedence} (\Cref{sectionBefore}).\is{not yet|)}

Moving on to repetition-related uses,\is{repetition|(} these share an obvious common denominator with phasal polarity \textsc{still} in the notions of continuity and identity, the main difference here lying in the presence vs. absence of temporal gaps (\cite[108–109]{vanBaar1997}; \cite{SchultzeBerndt2002}; \cite{TovenaDonazzan2008}; among many others). Given these similarities, it comes as no surprise that diachronic links in either direction are attested. Furthermore, the combination with wide-scope negation,\is{negation} yielding \textsc{no longer}, is likely to have played a role in the diachrony of at least some of the relevant expressions. On a more fine-grained level the sample data point to a greater affinity between phasal polarity \textsc{still} and iteration (e.g. \lq call again\rq{}) than restitution (e.g. \lq close the door again\rq{}), counter to what is claimed in \citeauthor{vanBaar1997}'s (\citeyear[108–109]{vanBaar1997}) seminal study of phasal polarity.\is{repetition|)}

\is{necessity|(}\is{possibility|(}\is{prospective|(}\is{modality|(}
The thirdmost frequent time-related function is that of future-oriented \lq eventually\rq{}, as in \ili{German} \textit{Ich mach' das noch} \lq I'll do it yet\rq{}. I argued that a motivated explanation for this use is best sought in diachrony. Elaborating on a proposal first made by \textcite{Abraham1977}, I suggested that it goes back to phasal polarity \textsc{still} in contexts of prospective or modal propositions. These \lq can/must still do\rq{ }combinations entail that a situation of the relevant type possibly/necessarily comes into existence before the end of the time under discussion, thereby opening the gates to a reanalysis of the \textsc{still} expression's contribution as \lq eventually, before it's too late\rq{}.\is{necessity|)}\is{possibility|)}\is{prospective|)}\is{modality|)} Speaking of pathways of change, several of the historical scenarios put forward throughout \Cref{chapter2} remain to be tested against diachronic data, where available. One such proposal, which owes much to \citeauthor{Mustajoki1988}'s (\citeyear{Mustajoki1988}) synchronic discussion of Russian \textit{ešcë}, is that the \lq as far removed as\rq{ }function in Slavic and Modern Hebrew is rooted in a series of analogical context expansions that have their starting point in the marking of a persistent\is{persistence} time frame (\Cref{sectionTimeScalar}).\is{scale}

In \Cref{chapter3} I turned to non-temporal uses, which display a functional coverage no less diverse than that of their time-related counterparts. I began my examination with a discussion of \textsc{still} expressions as markers of an entity's marginal\is{marginality} position on a \isi{scale} or within a graded category (\Cref{sectionMarginality}; also see below). I then turned to functions pertaining to \isi{focus} quantification, first as markers of additivity\is{additive} and related notions (\Cref{sectionAdditiveMain}) and then as \isi{restrictive}  \lq only\rq{}, \lq at least\rq{ }operators (\Cref{sectionRestrictiveUebergeordnet}). One key finding here is the observation that the coexpression of phasal polarity \textsc{still} and exclusive\is{restrictive} \lq only\rq{ }is widespread across the macro-areas of Australia and Papunesia, more so than has been observed previously (\Cref{sectionExclusive}). Other functions discussed in \Cref{chapter3} include a host of modal\is{modality} and strongly pragmaticised meanings (\Cref{sectionBroadlyModal}), such as uses pertaining to concessivity\is{concessive} (\lq although\rq{}, \lq nonetheless\rq{}) or ones relating to specific types of speech acts, to give a few examples. Just as in the realm of tempo-aspectuality, the various non-temporal functions are sometimes near-mirror images of one another, as in the case of \isi{additive} vs. \isi{restrictive} \isi{focus} quantification. Also in line with a theme observed in \Cref{chapter2}, several of the non-temporal uses are highly idiosyncratic or attested for pairs of only two sample expressions. This is particularly true of speech-act specific functions, such as \ili{German} \textit{noch} in \lq remind me again\rq{ }questions (\Cref{sectionInterrogativeOther}). Many others uses, though, are common, with the three most frequent being listed in \Cref{TableConclusionsNonTemporal}. Again, a brief discussion of each of them is in order.

\begin{table}
\caption{Most common non-temporal uses. \emph{Note}: Values in parenthesis exclude borderline cases of \textsc{still} expressions and tentative categorisations.\label{TableConclusionsNonTemporal}}
\small
\begin{tabular}{l c *2{c@{~}c} l}
	\lsptoprule
	         &  Macro-  & \\
	Function &  {areas} & \multicolumn{2}{c}{Languages} &  \multicolumn{2}{c}{Expressions} & Discussed in\\\midrule
	Additive\is{additive} & 6 && \phantom{(}28\phantom{)} && \phantom{(}29\phantom{)} & \Cref{sectionAdditive}\\
	Marginality\is{marginality} & 5 & 19 & (18) & 21 &(20) & \Cref{sectionMarginality}\\
	Concessive\is{concessive} apodoses & 5 & 19 & (15) & 17 &(13) & \Cref{sectionProspective}\\
	\lspbottomrule
\end{tabular}
\end{table}

\is{marginality|(}
The secondmost common non-temporal function is that of marking marginal membership on a \isi{scale} or within a graded category (e.g. \lq a penguin is still a bird\rq{}). While this use involves a fairly simple metonymical transfer from times to other scales, it cannot be fully equated with phasal polarity. Thus, the two functions differ in important semantic characteristics, namely the orientation of the \isi{scale} and the reality status of the subsequent discontinuation;\is{discontinuation} these points of divergence are also sometimes reflected in grammar. What is more, the marginality function is not universal, as negative evidence from three sample expressions shows. Lastly, marginality lies  at the heart of several other functions, for instance \ili{French} \textit{encore} and \ili{Spanish} \textit{todavía} as \lq at least\rq{ }type markers in counterfactual \isi{conditional} constructions (\Cref{sectionScalarRestrictive}).\is{marginality|)}

The third use in \Cref{TableConclusionsNonTemporal} is that of marking the apodosis clause in a \isi{concessive} construction, as in \textit{He studied hard and still he failed the exam}. This function instantiates a well-known trend for the wider field of expressions for continuity or coexistence to be co-opted as markers of \isi{concessive} relationships (\cite{Koenig1985}, \citeyear{KoenigConcessives}). More specifically, it is safe to assume that the recruitment of \textsc{still} expressions to this function is rooted in the alignment of a position on a \isi{scale} that is far removed from the origo (e.g. an advanced runtime, a marginal\is{marginality} position) with less accessible possible worlds\is{modality} (\cite{EderlyCurco2016}; \cite{Michaelis1993}; and many others). This process may be aided by factors such as the presence of negation,\is{negation} and it can receive additional motivation from focus-related\is{focus} functions of the same expressions. Lastly, with several sample expressions the marking of \isi{concessive} apodoses has given rise to a holophrastic interjective use (\Cref{sectionConcessiveInterjections}),\is{interjection} and it may also lie at the heart of Mandarin Chinese\il{Chinese, Mandarin} \textit{hái} as a counter-expectational\is{expectations} marker (\Cref{sectionCounterExpectation}).

To bring this book to a conclusion, by examining the different functions attested in my sample of \textsc{still} expressions, I hope to have laid the cornerstone for more cross-linguistic work on an area that is characterised by intriguing degrees of both convergence and diversity, and which had previously been approached predominantly from a language-specific angle.
