\setchapterpreamble{\dictum[Will Rogers]{I’m not a real movie star. I’ve \textit{still} got the same wife I started out with twenty-eight years ago.}}
\chapter{Introduction}
\addtocontents{toc}{\protect\enlargethispage{2\baselineskip}}
\section{Introductory remarks}
It is a well-known fact that linguistic expressions, especially ones pertaining to the realm of grammar, are often polyfunctional. That is to say, they are characterised by a cluster of meanings that are conceptually related, but which, at the same time, can be differentiated from one another on the basis of structural and/or meaning-based criteria. Examining the cross-linguistic patterns of these different uses and functions allows insights into the question of how human language organises meaning (e.g. \cite{Evans2010}).

A semanto-pragmatic domain that tends to be characterised by a high degree of polyfunctionality is that of \textsc{phasal polarity}, an umbrella term coined by \textcite{vanBaar1997} for the notions of \textsc{still}, \textsc{already},\is{already} \textsc{not yet},\is{not yet} and \textsc{no longer},\is{no longer} as well as for the expressions denoting them. Throughout this book, I represent the abstract phasal polarity concepts in small caps. Not only does phasal polarity lie at the intersection of polarity and aspectuality\is{aspect} in the broadest sense, but the functions of individual expressions frequently extend into a wide array of other linguistic domains. In this book, which is aimed at semanticists, typological researchers, and descriptive grammarians alike, I survey the polyfunctionality patterns of linguistic items whose meaning includes the concept of \textsc{still} in a sample of the world's languages.\il{German|(} To give just a little preview of such functional extensions, consider the excerpt in (\ref{exBookIntroNoch}), which features German \textit{noch} in three distinct uses.

\begin{exe}
	\ex \label{exBookIntroNoch}
	\begin{xlist}	
	\exi{}German
	\ex \textit{Alle inländischen wohlmeinenden Anträge hat sie ausgeschlagen, \textbf{noch} neulich musste ich den gescheiten und tüchtigen Melchior Böhni heimschicken,}\\
	 \lq She has refused all well-meaning offers by domestic businesses, \textbf{just} the other day she made me send home the clever and capable Melchior Böhni,\rq{}\label{exBookIntroNochA}
 \ex  \textit{der \textbf{noch} große Geschäfte machen wird,}\\
  \lq who will \textbf{eventually} run a very successful enterprise.\rq{}\label{exBookIntroNochB}
  \ex  \textit{und sie hat ihn \textbf{noch} schrecklich verhöhnt, weil er nur ein rötliches Backenbärtchen trägt und aus einem silbernen Döschen schnupft!}\\
	 \lq And she \textbf{also} dreadfully mocked him, because of his little red beard and because he sniffs his tabacco from a little silver box!'
	 \\(Keller, \textit{Kleider machen Leute})\label{exBookIntroNochC}
	\end{xlist}
\end{exe}

In (\ref{exBookIntroNochA}) \textit{noch} serves as a scalar\is{scale} \isi{focus} particle that combines with the temporal adverb \textit{neulich} \lq the other day' and relates its denotation to alternative, earlier times. The same function is found, for instance, with \ili{French} \textit{encore} in (\ref{exBookIntroTimeScalar}). In (\ref{exBookIntroNochB}), on the other hand, \textit{noch} itself has a function akin to that of a temporal adverb, signalling that some development is on its way that will culminate in the type of event described by \textit{große Geschäfte machen} \lq run a very successful enterprise\rq{}.\is{prospective} One expression that has the same function is \textit{ɓà} in the Mbumic language \ili{Mundang} of Cameroon and Chad; see (\ref{exBookIntroProspective}).

\begin{exe}
	\ex \ili{French}\label{exBookIntroTimeScalar}\\
	\gll \textbf{Encore} \textbf{hier}, j'-ai vu le film {La Chute}….\\
	still yesterday 1\textsc{sg}-have.1\textsc{sg} see.\textsc{ptcp} \textsc{def}.\textsc{sg}.\textsc{m} movie(\textsc{m}) Downfall\\
	\glt \lq \textbf{Just yesterday}, I watched the movie \textit{Downfall}…'
	\\(found online, glosses added)%\footnote{\url{https://www.saintbrice95.fr/a-la-une/actualites/actualites-2019/lacteur-michel-bouquet-sous-la-direction-du-saint-bricien-ulysse-di-gregorio-1113.html} (02 May, 2022).}
		\ex\label{exBookIntroProspective}\ili{Mundang}\is{prospective}\\
		\gll ʔà fʊ̄ō \textbf{ɓà}.\\
		\textsc{subj}.3:\textsc{ipfv} think.\textsc{pot} still\\
		\glt \lq Il pensera un jour. [He will think \textbf{one} \textbf{day}.]' \parencite[382]{Elders2000}
\end{exe}

\is{additive|(}Lastly,  in (\ref{exBookIntroNochC}) \textit{noch} depicts the act of mocking as an additional event that forms part of the same overarching scene. An expression that shares this function with \textit{noch} is \textit{təmna} in the Samoyedic language Tundra Nenets,\il{Nenets, Tundra} as shown in (\ref{exBookIntroAdditiveTundraNenets}).\il{German|)}

\begin{exe}
		\ex Tundra Nenets\il{Nenets, Tundra}\label{exBookIntroAdditiveTundraNenets}\\
		\gll Nʼeᵒka-nʼi sʼitᵒ nʼada-wa-h tʼaxᵒmna mənʼᵒ \textbf{təmna} tedə-mtᵒ taᵒ-dəm-cʼᵒ.\\
		elder\_brother-\textsc{gen}.1\textsc{sg} 2\textsc{sg}.\textsc{acc} help-\textsc{ipfv}.\textsc{nmlz}-\textsc{gen} beside 1\textsc{sg} still reindeer-\textsc{fut.poss}:\textsc{acc}:2\textsc{sg} give-1\textsc{sg}-\textsc{pst}\\
		\glt \lq In addition to my brother helping you, I \textbf{also} gave you a reindeer.' \parencite[371]{Nikolaeva2014}
\end{exe}\is{additive|)}

\il{Spanish|(}
With many of the expressions I discuss in this book, such cross-linguistically recurring functions are paired with more idiosyncratic ones. For instance, in (\ref{exSpanishAcceptableLimits}) Spanish \textit{todavía} signals that the situation described in the preceding complement clause falls within socially acceptable limits. While this use may appear unrelated to the meanings discussed so far at first sight, in \Cref{sectionMarginality} I discuss how all steps leading from \textsc{still} to \lq within acceptable limits\rq{ }remain visible in the present-day language.

\begin{exe}
	\ex\label{exSpanishAcceptableLimits}
	Spanish\\
	\gll Que un gran artista ten-ga eso-s humo-s, \textbf{todavía}, pero él es un simple aprendiz.\\
	\textsc{comp} a great artist have-\textsc{sbjv}.3\textsc{sg} \textsc{dem}.\textsc{m}-\textsc{pl} fume(\textsc{m})-\textsc{pl} still but \textsc{3sg.m} \textsc{cop}.3\textsc{sg} \textsc{indef}.\textsc{sg}.\textsc{m} simple apprentice(\textsc{m})\\
	\glt \lq A great artist having such an attitude, \textbf{that} \textbf{would} \textbf{be} \textbf{acceptable}, but heʼs just an apprentice.\rq{ }(\cite[207]{Bosque2016}, glosses added)
\end{exe}
\il{Spanish|)}

The remainder of this introductory chapter is structured as follows: in \Cref{sectionGoalsLimitations}, I lay out the goals and limitations of this book in more detail.  In \Cref{sectionPreviousResearch}, I briefly address previous research on \textsc{still} expressions. In \Cref{sectionDefinition}, I provide my definition of the subject matter. Afterwards, in \Cref{sectionDataCollectionAnalysis}, I discuss the processes of data collection and analyses. This is followed by a summary of some theoretical preliminaries in \Cref{sectionKeyConcepts}. Lastly, \Cref{sectionStructureBook} is an overview of the structure of the present volume.

\section{Goals and limitations}
\label{sectionGoalsLimitations}
The main aim of this book is to survey and explain the different functions that expressions for the concept of \textsc{still} have, based on a sample of the world's languages. I thereby hope to establish a stepping stone for future cross-linguistically minded research on the matter, and to equip descriptive grammarians with a reference work that can hopefully guide them through disentangling some of the intricate patterns of use they may encounter in their languages of study. In order to achieve these goals, I opted for a comparatively small \isi{sample} of 76 languages, paired with a high-resolution approach that takes into account as broad an array of data sources as feasible within the bounds of a research project (\Cref{sectionDataCollectionAnalysis}). To increase transparency and to encourage follow-up works, this book features an extensive set of appendices in the form of commented data sheets for each language and each marker included in the sample (see \Cref{sectionAboutAppendices} for discussion). Against this backdrop, the more specific questions that I address throughout \Crefrange{chapter2}{chapter4} are as follows:\medskip

\begin{enumerate}[label=(\roman*)]
	\item Which functions, especially recurrent ones, can be found?\label{question1}
	\item What motivates the coexpression of a given function and \textsc{still}?\label{question2}
	\item How do these functions relate to one another?\label{question3}
	\item Do certain functions correlate with a specific morphosyntactic status?	\label{question4}
	\item Are there any observable areal patterns?\label{question5}
\end{enumerate}

To briefly elaborate on these questions, \ref{question1} entails the problem of how to draw the line between two functions or uses, which is a point I discuss in \Cref{sectionDistinctiveness}. As implied in \ref{question2}, I take a usage-based approach to linguistic variation, aiming to understand \lq\lq why it is at least possible and at best natural that this particular form–meaning correspondence should exist in a given language\rq\rq{ }(\cite[217]{Goldberg2006}). I return to this matter in \Cref{sectionSemasiologicalChange}. The question of motivation, in turn, is directly tied to \ref{question3}, the issue of how different functions of a single item relate to each other. Building on the approach taken in semantic map studies (e.g. \cite{GeorgakopulosPolis2008}; \cite{Haspelmath2003}), I looked not only for semantic similarities, but also for implicational relationships. To give a fairly straightforward example, consider Spanish\il{Spanish|(} \textit{todavía} as a pro-predicate in (\ref{exSpanishAcceptableLimits}), repeated below.

\begin{exe}
	\exr{exSpanishAcceptableLimits}
	Spanish\\
	\gll Que un gran artista ten-ga eso-s humo-s, \textbf{todavía}, pero él es un simple aprendiz.\\
	\textsc{comp} a great artist have-\textsc{sbjv}.3\textsc{sg} \textsc{dem}.\textsc{m}-\textsc{pl} fume(\textsc{m})-\textsc{pl} still but \textsc{3sg.m} \textsc{cop}.3\textsc{sg} \textsc{indef}.\textsc{sg}.\textsc{m} simple apprentice(\textsc{m})\\
	\glt \lq A great artist having such an attitude, \textbf{that} \textbf{would} \textbf{be} \textbf{acceptable}, but heʼs just an apprentice.\rq{ }(\cite[207]{Bosque2016}, glosses added)
\end{exe}

From both the meaning and patterns of usage of cases like (\ref{exSpanishAcceptableLimits}), it becomes clear that this function is mediated by \textit{todavía} as a marker of \isi{marginality} (i.e. \lq{}still be in the realm of a given category\rq{}) plus the ellipsis of an evaluative predicate (\cite{Bosque2016}; \cite{Deloor2012}). This is also in line with the fact that the only other marker in the sample that has a comparable function, Serbian\hyp Bosnian\hyp Croatian\il{Serbian}\il{Croatian}\il{Bosnian} \textit{još}, likewise has the \isi{marginality} function.\il{Spanish|)} Where feasible, I also tested predictions about implicational relationships that are found in the literature, be it as explicit statements, or merely implied. For instance, in his groundbreaking study on phasal polarity, \textcite{vanBaar1997} observes that \textsc{still} expressions in isolation are recurrently used in the function of their internally negated\is{negation} counterpart \textsc{not yet}.\is{not yet}  This is illustrated in the \ili{Tashelhyit} (Afro-Asiatic > Berber) example (\ref{exNotYetGoals}). \Textcite[294–295]{vanBaar1997} then goes on to suggest the two-part universal in (\ref{exVanBaarUniversalNotYetinto}). 

\begin{exe}
	\ex \ili{Tashelhyit}\label{exNotYetGoals}\is{not yet}\\
	\gll T-šši-t yad lfḍur-nnek? – \textbf{Sul}.\\
	2\textsc{sg}-eat.\textsc{pfv}-2\textsc{sg} already lunch-\textsc{poss}.2\textsc{sg} {} still\\
	\glt \lq Did you eat your lunch already? -- \textbf{Not yet}.' \parencite[342]{Fanego2021}
	
	\ex \citeauthor{vanBaar1997}'s (\citeyear[295]{vanBaar1997}) \textsc{still}-as-\textsc{not yet} universal\label{exVanBaarUniversalNotYetinto}\is{not yet} 
	\begin{xlist}
		\ex\label{exVanBaarUniversalNotYet1intro}
		If a \textsc{still} expression is used as an isolated expression, it is invariably used for the expression of \textsc{not yet}.
		\ex\label{exVanBaarUniversalNotYet2intro}
		Examples are only found in those languages which have an internal \isi{negation} of the relevant \textsc{still} expression.
	\end{xlist}
\end{exe}

\is{not yet|(} 
As I discuss in \Cref{sectionNotYet}, neither of the components of (\ref{exVanBaarUniversalNotYetinto}) hold up against data from a larger sample.\footnote{For more extensive discussion, see \textcite{PersohnNotYet}, on which \Cref{sectionNotYet} is based.} Counterexamples to (\ref{exVanBaarUniversalNotYet1intro}) include \ili{English} \textit{still} and Modern Hebrew\il{Hebrew, Modern} \textit{ʕadayin} as \isi{concessive} interjections.\is{interjection}  As for (\ref{exVanBaarUniversalNotYet2intro}), the relevant function is also found with expressions that do not combine with \isi{negation} to signal \textsc{not yet}. These include \mbox{\textit{kya}-} and \mbox{\textit{syá}-} in the Bantu languages \ili{Ruuli} and Bende,\il{Bende} respectively (\cite{Abe2015}; \cite{MolochievaEtAl2021}) and \mbox{(\textit{y})\textit{edung}} in the Timor-Alor-Pantar language \ili{Blagar} (e.g. \cite{Steinhauer1995}). The former signal \textsc{not yet} in combination with an infinitival\is{infinitive} complement, whereas in the \ili{Blagar} case polarity is a function of the adverb's position in the clause.\is{syntax} What does, however, remain valid as an implicational universal is that the \textsc{not yet} use without \isi{negation} is only found with items that are also involved in the expression of this negative concept in more saturated sentential environments. This use also serves as an illustration of the parameter of morphosyntactic status \ref{question4}: unsurprisingly, such holophrastic usages are primarily attested with free morphemes and bound roots, such as \ili{Tashelhyit} \textit{sul} or \ili{Blagar} \mbox{(\textit{y})\textit{edung}}. They are, however, also occasionally attested with prefixal markers, which then require a copula root to fill the lexical slot. This is the case, for instance, with \ili{Bende} \mbox{\textit{syá}-} and its \ili{Ruuli} cognate \mbox{\textit{kya}-}.\is{not yet|)} Lastly, to give an example of areal patterns \ref{question5}, in \Cref{sectionExclusive} I discuss the lexical coexpression of \textsc{still} and an exclusive\is{restrictive} function \lq only'. Judging from my sample, this type of polysemy is widespread in Australia and Papunesia, but rare to non-existent elsewhere in the world.

Having laid out the questions guiding this study, I now turn to some of its limitations. First and foremost, the survey I develop in \Cref{chapter2,chapter3} is confined by bibliographical bias and the overall availability of data. Secondly, this book is intended as a conversation starter, rather than aiming to be the last word on the subject matter. As such, it inevitably leaves many loose ends open, which will hopefully be picked up in future research. In addition, and as I have only implied up to this point, I do not aim to postulate abstract \textit{Gesamtbedeutungen} intended to cover all uses of a given expression. What is more, while my initial aim was to model the attested polyfunctionality with semantic maps, it quickly became apparent that such an approach is severely constrained by the often fragmentary data. In addition, and due to the broad array of attested functions, it would ultimately only lead to a crowded and tightly interconnected, hence uninformative, map. I instead opted to keep some of the key tenets of semantic maps, such as their approach to differentiating between individual uses (\Cref{sectionDistinctiveness}) and the search for implicational relationships. Where deemed helpful, I address and illustrate partial networks of a given semantic field and/or of a specific expression.

\section{Previous research}
\label{sectionPreviousResearch}
The functional extensions of \textsc{still} expressions, or of phasal polarity in general, have received little attention from cross-linguistic scholarship. Thus, many typological works explicitly exclude additional functions, or only address them as side notes (e.g. \cite{vanderAuwera1998}; \cite{vanBaar1997}; \cite{VeselinovEtAl}). While many insightful and detailed descriptions of specific markers and systems exist, they are normally limited to individual languages. Such works include, among others, the descriptions of phasal polarity systems on the African continent found in \textcite{Kramer2021a}, \textcite{Bosque2016} on \ili{Spanish} \textit{todavía}, \textcite{Huang2007} on \textit{nahan} in the Austronesian language Saisiyat, \citeauthor{MosegaardHansen2002} (\citeyear{MosegaardHansen2002}, \citeyear{MosegaardHansen2008}) on several \ili{French} expressions, \textcite{Kockelman2020} on Kekchí Maya\il{Kekchí} \textit{toj}, and \textcite{Shetter1966} on \ili{German} \textit{noch}. Another group of works, informed by different theoretical traditions, aims to establish abstract core meanings of individual items. Such approaches include \textcite{Beck2020} on \ili{German} \textit{noch}, \textcite{Greenberg2012} on Modern Hebrew\il{Hebrew, Modern} \textit{ʕod}, \textcite{McConvell1983} on \mbox{=\textit{rni}} in the Pama-Nyungan language Gurindji,\il{Gurindji} and \textcite{JingSchmidtGries2009} on several Mandarin Chinese\il{Chinese, Mandarin} markers, including the \textsc{still} expression \textit{hái}. Those existing cross-linguistic studies that do address the wider functional scope of \textsc{still} expressions tend to focus on a small subset of uses and are typically highly constrained in geographic and/or genealogical coverage. For instance, \textcite{vanBaar1991} and \textcite{SchultzeBerndt2002} observe the recurrent coexpression of \textsc{still} and exclusive\is{restrictive} meanings  in Australia and Papunesia (\Cref{sectionExclusive}) and \textcite{Gueldemann1998} and \textcite{VeselinovaDevos2021} discuss a few additional functions of so-called \lq\lq persistives" in Bantu.\is{persistence}

To the best of my knowledge, the two major exceptions to these general tendencies are \textcite{Zhang2017} and \textcite{Panova2021}; the latter only became available at a late stage of preparing this monograph. \textcite{Zhang2017} is primarily concerned with expressions of repetition,\is{repetition} but his sample includes several \textsc{still} expressions. \textcite{Panova2021} contains many important insights, but her discussion of polyfunctionality is mostly based on what is found in the descriptive sections of grammars, as opposed to the higher-resolution approach taken in the present work (\Cref{sectionDataCollectionAnalysis}). What is more, both studies are based on a broader definition of the subject matter, which does not include a \isi{prospective} component. This definition is the topic of \Cref{sectionDefinition}.

\section{Definition of the subject matter}
\label{sectionDefinition}
Throughout this book, I understand a linguistic item to be a \textsc{still} expression if and only if it conforms to the definition in (\ref{Definition}).\footnote{I hasten to point out that a strict functional delimitation does not run counter the goal of examining polyfunctionality. Instead, it is necessary to define the criteria for inclusion of a given expression, to then examine which additional functions it exhibits.} This definition, which purposely takes as framework-neutral a stance as possible, is derived from a survey of previous studies on phasal polarity, to which I return shortly. Note that the definition consists of two components, with the formal component relying on the functional one.

\begin{exe}
	\ex 
	\label{Definition}
	\begin{xlist}
		\exi{}Definition of a \textsc{still} expression
		\ex\label{CCfunctional}  
	Functional component:\\The concept of \textsc{still} signifies the persistence of an affirmative-polarity situation at topic time.\is{topic time} At the same time, it construes the situation's subsequent \isi{discontinuation} as a valid alternative scenario.
		\ex\label{CCformal} 
	Formal component:\\A \textsc{still} expression is a linguistic form  that signals the concept of \textsc{still}.
	\end{xlist}
\end{exe}

In what follows, I offer additional discussion of both components of my definition and then address the question of its practical application. Afterwards, I briefly turn to the issue of more than one \textsc{still} expression in a single language.

\subsection{The functional component: Discussion}
\label{secFunctionalDiscussion}
It seems appropriate to begin the discussion of the definition in (\ref{Definition}) with its functional core. This, in turn, can be broken down into three components: persistence, polarity and prospection.\is{prospective}

\subsubsection{Persistence}\is{persistence|(}
It is safe to say that my understanding of \textsc{still} as a signal of persistence is fairly uncontroversial. More precisely, most authors agree that the relevant expressions trigger an existential presupposition that the situation depicted in the clause obtained during an interval that precedes and abuts the time under discussion.\footnote{See \textcite{Abraham1980}, \textcite[38–42]{vanderAuwera1998}, \textcite{Doherty1973}, \textcite{Horn1970}, \textcite{Ippolito2004}, \textcite{Koenig1977}, \textcite[173–176]{Loebner1989}, \textcite{Mittwoch1993}, \textcite{Morrisey1973}, \textcite{Muller1975}, among others.} In contributing an existential presupposition, \textsc{still} expressions thus share a semantic feature with \isi{additive} \isi{focus} quantifiers (\Cref{sectionQuantificationScales}), which is a point that proves to be of relevance in several recurrent functional extensions (\Cref{sectionEventSequencing,sectionProspective,sectionAdditive}). \is{aspect|(}The notion of persistence is furthermore intrinsically linked to the domain of aspect in the broadest sense. In the dimension of aspectual operators, expressions for \textsc{still} naturally combine with viewpoints that are fully contained within a situation, such as imperfectives\is{imperfective} and resultatives,\is{resultative} as opposed to perfectives\is{perfective} and anteriors.\is{anterior} Concerning the actional\is{actionality} dimension of aspect, \textsc{still} is only compatible with situation types that involve some extended state or process (\cite[151–153]{vanBaar1997}; \cite[134]{Dahl1985}; \cite{Kratzer2000}; \cite{Loebner1989}; \cite{NedjalkovJaxontov1988}; among others). These (in)compatibilities constitute a recurring theme throughout \Cref{chapter2,chapter3}.\is{persistence|)}\is{aspect|)}

\subsubsection{Affirmative polarity}\is{not yet|(}\is{no longer|(} My definition restricts the notion of \isi{persistence} to affirmative polarity. It is well-known that \textsc{still} items frequently combine with \isi{negation} to yield expressions for the two negative phasal polarity concepts \textsc{not yet} and \textsc{no longer}\is{no longer} (e.g. \cite{vanderAuwera1998};  \cite{vanBaar1997};  \cite{Loebner1989}). If \isi{negation} falls within the scope of the \textsc{still} expression (\lq\lq inner negation") this yields \textsc{not yet}.\footnote{I use the terms \lq\lq inner" and \lq\lq outer" \isi{negation} solely in reference to scope. This may or may not be reflected in surface syntax.\is{syntax}} Examples include \ili{German} \textit{noch nicht} or \ili{Spanish} \textit{todavía}/\textit{aún} \textit{no}, both \lq still \textsc{neg}'. Negation\is{negation} taking scope over the \textsc{still} expression  (\lq\lq outer negation"), on the other hand, yields \textsc{no longer},\is{no longer} as is the case, for instance, with \textit{oc} in the Uto-Aztecan language Classical Nahuatl\il{Nahuatl, Classical} and with \mbox{\textit{sa}-}/\mbox{\textit{se}-} in the Bantu language Xhosa. In the appendices I follow \citeauthor{vanBaar1997}'s (\citeyear{vanBaar1997}) terminology and speak of the parameter of \lq\lq polarity sensitivity". Crucially, the inclusion of affirmative polarity in my definition allows me to exclude items  that form part of expressions for the negative\is{negation} concepts \textsc{not yet} or \textsc{no longer},\is{no longer} but which do not serve as independent \textsc{still} expressions. A prime example of the latter is \ili{English} \textit{yet}. Originally a \textsc{still} expression, and continuing to share many functional extensions with such items, in present-day standard varieties it meets my definition only in a handful of relic cases, such as in \textit{It is early yet} \parencite{KoenigTraugott1982}.\is{not yet|)}\is{no longer|)}

\subsubsection{An alternative course of events}\is{discontinuation|(}\is{modality|(}
The third ingredient to my functional definition, the invocation of an alternative discontinuation scenario, requires some more explanation. To begin with, it has repeatedly been observed that  items like \ili{English} \textit{still} and its counterparts in other languages do not combine well with situations that are inalterable under normal circumstances.\footnote{This observation is found in \textcite{CranePersohn2021}, \textcite[145]{Donazzan2008}, \textcite{Doherty1973}, \textcite[ch. 4]{MosegaardHansen2008}, \textcite{Jenny2001}, \textcite{Kockelman2020}, \textcite{Michaelis1993}, \textcite{NedjalkovJaxontov1988}, \textcite[§30.8h–j]{RAEGramatica}, \textcite[391]{Weber1989}, and many other works.} Thus, examples like (\ref{exInalterableDE}–\ref{exInalterableFR}) are decisively odd.\footnote{Example (\ref{exInalterableDE}) is acceptable in a \isi{marginality} reading  \lq counts as comparatively tall\rq{ }(\Cref{sectionMarginality}).}

\begin{exe}
\ex[]{\label{exInalterableDE}\ili{German}}
\sn[?]{
\gll Elisabeth ist noch groß.\\
Elisabeth \textsc{cop}.3\textsc{sg} still tall.\\
\glt \lq Elisabeth is still tall.' \parencite[154]{Doherty1973}}

\ex[]{\ili{English}\label{exInalterableEN}}
\sn[?]{\textit{Uncle Harry is still dead.} \parencite[202]{Michaelis1993}}

\ex[]{\ili{French}\label{exInalterableFR}}
\sn[?]{
\gll Le Pape est encore célibataire.\\
\textsc{def}.\textsc{sg}.\textsc{m} pope(\textsc{m}) \textsc{cop}.3\textsc{sg} still unmarried\\
\glt \lq The pope is still unmarried.' \parencite[118]{MosegaardHansen2008}
}
\end{exe}

Instances like these suggest that the concept of \textsc{still} not only involves continuation relative to a prior time span, but also the possibility of discontinuation at an interval following the time under discussion. This subsequent termination is, however, not entailed, as it can be cancelled without contradiction; see (\ref{exGermanKingConstantine}).

\begin{exe}
\ex \label{exGermanKingConstantine}\ili{German}\\
\gll König Konstantin leb-t noch i-m Exil \textbf{und} \textbf{das} \textbf{wird} \textbf{wohl} \textbf{auch} \textbf{so} \textbf{bleib}-\textbf{en}.\\
king Constantine live-3\textsc{sg} still in-\textsc{def}.\textsc{dat}.\textsc{sg}.\textsc{n} exile(\textsc{n}) and that \textsc{fut}.\textsc{aux}:3\textsc{sg} apparently also thus remain-\textsc{inf}\\
\glt \lq King Constantine is still living in exile \textbf{and that is the way it will always be}.' (\cite[176]{Koenig1977}; glosses added)
\end{exe}

Given that the subsequent discontinuation is defeasible, many researchers relegate it to a derived inference (\cite{Abraham1980}; \cite{Klein2018}; \cite{Koenig1977}; \cite{Muller1975}, among others) and explain the oddity of cases like (\ref{exInalterableDE}–\ref{exInalterableFR}) by assuming that it is pointless to overtly describe a situation as persistent\is{persistence} if it cannot change anyway. Authors such as \citeauthor{vanderAuwera1993} (\citeyear{vanderAuwera1991BeyondDuality}, \citeyear{vanderAuwera1993}, \citeyear[38–42]{vanderAuwera1998}),  \textcite[ch. 2]{vanBaar1997}, \textcite{Michaelis1993}, or \textcite[ch. 3]{Vandeweghe1992} take a different stance. In their view, the concept of \textsc{still} intrinsically involves a perspective on time that is \lq\lq retrospectively continuative and prospectively\is{prospective} geared towards possible change" \parencite[40]{vanderAuwera1998} and where this change \lq\lq figures in the discourse, and/or in the mind of the Speaker, of the Addressee, or both, as a serious alternative of the factual situation described in the sentence" \parencite[41]{vanBaar1997}. Approaching the issue from this angle, a speaker who utters \textit{King Constantine is still living in exile} (\ref{exGermanKingConstantine}) not only signals that the monarch remains overseas. They also communicate that they have considered a possible course of events in which he has returned, or will return, from exile. As \textcite[1]{Vandeweghe1992} puts it, by using a phasal expression \lq\lq the state-of-affairs … is vested with a dynamism\rq\rq.\footnote{In the original Dutch \lq\lq de standen van zaken … krijgen een plaats in een dynamiek\rq\rq{}.} Given that the alternative development is not actually asserted, there is no contradiction in explicitly rejecting it. Note that this notion of an alternative course of event does not necessarily relate to hearer/speaker expectations;\is{expectations} I return to this point in \Cref{sectionExpectation} below.

As I see it, there are several benefits to including the evocation of an alternative scenario in my definition.  Firstly, it provides a communicative embedding for the concept of \textsc{still}. As a correlate, it allows for a straightforward functional explanation for the anormality of examples like (\ref{exInalterableDE}–\ref{exInalterableFR}). Thus, it is strange to say \textit{Uncle Harris is still dead} or \textit{Elizabeth is still tall}, because scenarios in which the subject is \textit{no longer}\is{no longer} dead or tall are at odds with everyday experience. It is only in very specific discourse worlds that such statements make sense, for instance in the context of the 1989 Disney movie \textit{Honey, I Shrunk the Kids}. In the same vein, a development in which the pope ceases to be single clashes with cultural expectations \parencite[117–118]{MosegaardHansen2008}. Secondly, making an alternative course-of-events part of my definition allows me to capture one of the otherwise hard-to-grasp differences between \textsc{still} expressions and continuative expressions of the type \textit{continue} Verb-\textit{ing},\il{Spanish|(} Spanish \textit{seguir} plus gerund, and the like. For instance, the Royal Spanish Academy's comprehensive grammar makes the following observation:

\begin{quote}
Constructions with \textit{todavía} differ from those formed via \lq\lq \textit{seguir} [\lq continue\rq] + gerund" … in that the former more clearly suggest that the described situation may change in the near future … The presence of \textit{todavía} introduces a stronger expectation that the phase following the one described in the predicate can, or should, come into existence. \parencite[§30.8i]{RAEGramatica}\footnote{In the original Spanish \lq\lq Se diferencian las construcciones con \textit{todavía} de las formadas con «\textit{seguir} + gerundio» … en que las primeras sugieren más claramente que las segundas que la situación descrita puede alterarse en un futuro próximo  … La presencia de \textit{todavía} induce en mayor medida la expectativa de que la fase posterior a la designada por el predicado puede o debe obtenerse.\rq\rq{}}
\end{quote}
\il{Spanish|)}

Such a distinction is far from a peculiarity of the European context. For instance, in discussing two specific expressions in the Ubangian language Sango,\il{Sango} \textcite[114]{NassensteinPasch2021} remark that \lq\lq [u]nlike \textit{de}, which has a clear \isi{prospective} sense, \textit{ngba} is purely continuative". In a comparable fashion, \textcite[137]{Connell2013} states that \textit{degeq} in Austronesian \ili{Mateq} \lq\lq implies the consistency of an action or state", but does not involve a possible or expected change. What is more, by having an alternative scenario figure in my definition, I can exclude expressions for concepts that are only broadly related to \textsc{still}, such as items or phrases meaning \lq until now' or  \lq even now\rq{}.\footnote{Expressions of this type can, however, constitute the diachronic precursor for \textsc{still} expressions (\cite{vanderAuwera1998}; \cite{vanBaar1997}). For instance, \ili{German} \textit{noch} goes back to  a root \lq now' plus a suffix \lq also' \parencite[s.v. \textit{noch}]{DWDS} and \ili{Spanish} \textit{aún} has its origins in Latin \textit{ad huc} \lq until here, until now' (\cite{vanderAuwera1998}; \cite[s.v. \textit{aún}]{RAEDictionary}). Similarly, \ili{Wardaman} (Yangmanic) \textit{gayawun} can transparently be segmented into \textit{gaya} \lq today, now' and \mbox{-\textit{gun}}/\mbox{-\textit{wun}} \lq pertaining to' \parencite[323]{Merlan1994}, with this original meaning having become bleached; see \appref{appendixWardaman}.}  That said, the two key ingredients of my functional definition (\isi{persistence} and an alternative course of events) need not be equally prominent in all relevant attestation. For instance, in (\ref{exNeutralScenario}) it is Peter's persistent\is{persistence} location that is most salient. In (\ref{exTwoComponentsSalience2}), on the other hand, the component of a subsequent change is given special salience, due to the fronted position of \ili{German} \textit{noch} (cf. \cite{Beck2020}; \cite{Koenig1977}, \citeyear[140]{Koenig1991}).

\begin{exe}
	\ex
	\begin{xlist}
	\sn Context: Peter is expected to take a plane that leaves London at 4 pm. It is now 3 pm.
	
	\ex \ili{English}\\
	John: (\textit{Yes, I know}). \textit{Peter is \textbf{still} in London}. \parencite[32]{vanBaar1997}\label{exNeutralScenario}

	\ex \label{exTwoComponentsSalience2}
	\ili{German}\\
	\gll \textbf{Noch} ist Peter in London.\\
	still is P. in L.\\
	\glt \lq (As for now) Peter is still in London (e.g. if you want to talk to him, don't wait any longer).'  (personal knowledge)
	\end{xlist}
\end{exe}
\is{discontinuation|)}
 
\subsubsection{A note on expectations}\label{sectionExpectation}\is{expectations|(}
Before turning to the formal side of my definition, a word on expectations is required. While I do define \textsc{still} with reference to an alternative course of events, my comparative concept purposely makes no reference to speaker/hearer expectations. With regard to the latter, two scenarios can be distinguished. In the first scenario, the \isi{persistence} of the situation described is in line with speaker- and/or hearer expectations; \textcite[ch. 2]{vanBaar1997} terms this the \lq\lq neutral scenario". Example (\ref{exNeutralScenario}) above is such a case. In the second scenario the situation persists\is{persistence} at an unexpectedly late point in time; this is illustrated in (\ref{exLateScenario}). 
\begin{exe}
\ex \label{exLateScenario}
Context: In order to solve an urgent business matter, Peter was expected to take a flight departing London at 4 pm. It is now 6 pm and John is surprised to find out that Peter will only leave at 7 pm.\\
John: (\textit{Damn!}) \textit{Peter is \textbf{still} in London}. \parencite[33]{vanBaar1997}
\end{exe}

The unexpectedly late scenario is often described as evoking an alternative timeline in which events unfold according  to expectations (\cite[38–42]{vanderAuwera1998}; \cite[ch. 2]{vanBaar1997}; \cite{Michaelis1993}). That is, it can be understood as involving not only a contrast between two different courses of events, but also between two possible worlds that differ in their reality status. Accordingly, \textcite{vanBaar1997} speaks of the \lq\lq simultaneously counterfactual" scenario.\is{simultaneity} Figure \ref{FigLateScenario} illustrates this interpretation for (\ref{exLateScenario}). Importantly, cross-linguistic research has shown that sensitivity to expectations is an axis of significant variation across \textsc{still} expressions. Recent discussion of this issue can be found in \textcite{vanderAuwera2021} and \citeauthor{Kramer2017} (\citeyear{Kramer2017}, \citeyear*{Kramer2021b}). A single expression may be compatible with both scenarios, as is the case with \ili{English} \textit{still}. Alternatively, a language may make use one of the following strategies to specifically signal the unexpectedly late scenario \parencite[134]{vanBaar1997}:

\begin{itemize}
\item a dedicated expression, e.g. Classical Nahuatl\il{Nahuatl, Classical} \textit{nozan}.
\item combinations of \textsc{still} expressions, e.g. \ili{Georgian} \textit{ʒer} + \textit{isev}, or \textit{ʒer} + \textit{kʼidev}.
\item the use of additional coding, as in \ili{German} \textit{immer noch} lit. \lq always still\rq{}.
\end{itemize}

 
In the appendices, I follow \textcite{Kramer2017} in coding the compatibility with one and/or the other scenario under the label of \lq\lq pragmaticity".\is{expectations|)}\is{modality|)}

\setlength{\MinimumWidth}{\widthof{expectedx}}
\begin{figure}[hbt]\is{utterance time}
	\caption{Graphic illustration of (\ref{exLateScenario}), based on \textcite[33]{vanBaar1997}\label{FigLateScenario}}
	\begin{tikzpicture}[node distance = 0pt]
		\node[mynode, minimum width=7*\hoehe, text width=7*\hoehe, fill=cyan, very near start] (Sit1){P. in London};
		\node[mynode,  minimum width=2*\hoehe, text width=2*\hoehe, right= of Sit1] (Flug1){\small{\faPlane}};
		\draw[-, thick, label distance=0] (Sit1.north east) to ($(Sit1.south east)+(0,-\hoehe)$) node [below] {7 pm};
		\draw[-latex]  (Sit1.south west) to  (Flug1.south east) node [right] {t\textsubscript{factual}};	
		\node[mynode, minimum width=4*\hoehe, text width=4*\hoehe,below=4*\hoehe of Sit1.west, anchor=west, fill=cyan] (Sit2) {P. in L.};
		\node[mynode,  minimum width=2*\hoehe, text width=2*\hoehe, right= of Sit2] (Flug2){\small{\faPlane}};
		\draw[-, thick, label distance=0] (Sit2.north east) to ($(Sit2.south east)+(0,-\hoehe)$) node [below] {4 pm};
		\draw[-latex]  (Sit2.south west) to   ($(Flug2.south east)+(3*\hoehe,0)$) node [right] {t\textsubscript{expected}};
		\draw[-,densely dashed, label distance=0] ($(Sit1.north east)+(-\hoehe,\hoehe)$) node [above] {utterance} to ($(Sit2.south east)+(2*\hoehe,-\hoehe)$)  node [below] {6 pm};
	\end{tikzpicture}	
\end{figure}

\subsection{The formal component: Discussion}\label{secFormalDiscussion}
Having discussed the functional component of my definition, I can briefly turn to its formal component, which is repeated below. 

\begin{exe}
	\exr{CCformal}
	Formal component:\\A \textsc{still} expression is a linguistic form that signals the concept of \textsc{still}.
\end{exe}

This formal part of my definition closely follows \textcite{vanBaar1997}, who defines phasal polarity expressions as \lq\lq specific formal means of referring to the semantics of Phasal Polarity\rq\rq{ }\parencite[48]{vanBaar1997}. On a more fine-grained level, the definition purposely makes no reference to morphology or syntax.\is{syntax} While in well-studied European languages the functional concept of \textsc{still} is typically expressed by independent small words \parencite{vanderAuwera1998}, restricting the subject matter to this type of expression would severely impede cross-linguistic comparison; see \textcite{Kramer2017, Kramer2021b} for recent discussions. To give just two examples, in the Bantu language Manda,\il{Manda} \textsc{still} is expressed by an auxiliary-like element \mbox{(\textit{a})\textit{kona}}, as shown in (\ref{exMandaAux}). In the Sino-Tibetan language Northern Qiang,\il{Qiang, Northern} on the other hand, the concept is expressed by a prefix \mbox{\textit{tɕ}V-} on the predicate; see (\ref{exQiangPrefix}). In my discussion of cross-linguistic variation, as well as in the appendices, I follow \textcite{Kramer2017} in speaking of \lq\lq wordhood" as a shorthand for an expression's morphosyntactic status.

\begin{exe}
	\ex\label{exMandaAux}\ili{Manda}\\
	\gll W-\textbf{akona} w-i-henga lihengu?\\
	\textsc{subj}.2\textsc{sg}-still \textsc{subj}.2\textsc{sg}-\textsc{prs}-work \textsc{ncl5}.work\\
	\glt \lq Are you still working?\rq{ }(Rasmus Bernander, p.c.)
	
	\ex \label{exQiangPrefix} Northern Qiang\il{Qiang, Northern} \\
	\gll Theː \textbf{tɕɑ}-n.\\
	3\textsc{sg} still-sleep\\
	\glt \lq S/he is still sleeping.\rq{ }\parencite[228]{LaPollaHuang2003}
\end{exe}

\subsection{Identifying \textsc{still} expressions}
\label{sectionIdentifying}
As I discuss in \Cref{sectionSample}, I first and foremost chose the sample languages based on where I was able to identify, with reasonable confidence, expressions that i) conform to my definition, and ii) for which there were good indications of other functions. This raises the question of how I applied this definition, which seems the more relevant given the fact that I primarily rely on third-party data. In the most straightforward cases, I had access to dedicated studies on phasal polarity, such as \textcite{vanBaar1997}, \textcite{vanderAuwera1998}, \textcite{Vandeweghe1992} or the contributions in \textcite{Kramer2021a}, whose definitions matched mine. To give an example, \textcite[113]{NassensteinPasch2021} describe the \ili{Sango} verb \textit{de} as \lq\lq meaning \lq continue to have some quality until quality changes{'}". This is opposed to another \ili{Sango} marker, \textit{ngba}, for which they remark that \lq\lq [u]nlike \textit{de}, which has a clear \isi{prospective} sense, \textit{ngba} is purely continuative" \parencite[114]{NassensteinPasch2021}. Thus, I included \ili{Sango} \textit{de}, but not \textit{ngba} in my sample. Where no such studies were available, my point of departure were descriptive grammars and, to a lesser degree, dictionaries. In some cases, these materials openly addressed all elements of my definition. \citeauthor{Guillaume2008}'s (\citeyear{Guillaume2008}) description of the\il{Cavineña|(} Cavineña (Pano-Tacanan) marker \mbox{=\textit{jari}} is such a case:

\begin{quote}
In past \isi{tense} settings, =\textit{jari} means that the state/event has begun before and was still holding at the story time but does not hold at the present time anymore … In present \isi{tense} setting, =\textit{jari} means that the state/event has begun in the past and will still hold true in the future (although not forever). \parencite[660–661]{Guillaume2008}
\end{quote}\il{Cavineña|)}

\il{Trió|(}In other cases, finding the relevant pieces of information required a bit more detective work. For instance, Trió (Cariban) \mbox{=\textit{nkërë}} is listed by \textcite[15]{Letschert1998} under the Portuguese lemma \textit{ainda} \lq still' and is termed a \lq\lq persistive"\is{persistence} by \textcite{Carlin2004}. \textcite[468]{Meira1999} states that \mbox{=\textit{nkërë}} \lq\lq is used to indicate continuation, coming quite close to the English \lq still{'}". That this marker also evokes an alternative scenario becomes particularly evident in its behaviour with nominal predicates: in order for these to combine with persistive \mbox{=\textit{nkërë}},\is{persistence} they must be augmented by a suffix \mbox{-\textit{me}}. This suffix signals a manifest, but not intrinsic quality, typically a transient state \parencite[123–124, 130]{Carlin2004}. What is more, 

\begin{quote}
	when the temporary state is seen from the point of view of an extended period rather than as a bounded unit, ... someone who has left that state but is talking about the period when that state existed, the [derived noun] is always followed by … \textit{nkërë} \lq still'. \parencite[130]{Carlin2004}
\end{quote}

These pieces of information, when taken together, give evidence that Trió \mbox{=\textit{nkërë}} conforms to my definition, in that it signals both \isi{persistence} and evokes a \isi{discontinuation} scenario.\il{Trió|)} In yet other cases, the available examples, especially if they came with textual embedding, gave good indications as to whether or not certain markers glossed as \lq still', \lq yet', \lq persistive\rq{} and the like were to be included in my sample. Thus, as \textcite[182–183]{Loebner1989} points out, the alternative scenario evoked by \textsc{still} leads to prototypical usage patterns in which the two scenarios are implicitly or explicitly juxtaposed.\il{Movima|(} Applied to a concrete case, consider (\ref{exIdentifyingMovima}) from the Bolivian isolate Movima.

\begin{exe}
	\ex Movima\label{exIdentifyingMovima}\\
	Context: recollecting a conversation between the narrator's mother and grandfather. \\
	\gll N-asko \textbf{dira} pawaneɬ-wa=ʼne kaj sin̍lototo=ʼne mereʼ.\\
\textsc{obl}-\textsc{pro} still hear-\textsc{nmlz}=3\textsc{f} \textsc{neg} deaf.\textsc{nmlz}=3\textsc{f} big\\
	\glt \lq That was when she [narrator's mother] could still hear, she wasn't very deaf then.\rq{ }\parencite{MovimaCorpus}
\end{exe}

In (\ref{exIdentifyingMovima}), the item \mbox{\textit{diːra}(\textit{n})} serves, to all appearances, a dual purpose. On the one hand it links the mother's existing sense of hearing to the same capacity at an earlier stage in her life;\is{persistence} at the same time, it anticipates the onset of deafness that is also hinted at in the next clause. I have taken examples like (\ref{exIdentifyingMovima}) as an indication that  Movima \mbox{\textit{diːra}(\textit{n})} conforms to my definition.\il{Movima|)} In the appendices, I discuss the evidence for including a given expression in my sample under the list item \lq\lq specialisation", following the term used by \textcite{vanBaar1997}.

\subsection{Allolexy}\label{sectionAllolexy}
Up to this point, I have spoken of the inclusion of a language or expression in my sample somewhat interchangeably. However, it is not uncommon for a single language to possess more than one expression that matches my definition, i.e. several allolexes. To mention just a few instances, Classical Nahuatl\il{Nahuatl, Classical} has \textit{oc} and \textit{nozan}, in Modern Hebrew\il{Hebrew, Modern} one finds \textit{ʕod} and \textit{ʕadayin}, \ili{Spanish} has \textit{todavía} and \textit{aún}, and \ili{Swahili} has \textit{bado} and \textit{ngali}. In such cases, the different expressions may contrast with each other along several axes. For instance, in \ili{Swahili} the Omani Arabic\il{Arabic, Omani} loan \textit{bado} is much more frequent than the inherited \textit{ngali}. In Classical Nahuatl\il{Nahuatl, Classical} \textit{nozan} is a dedicated expression for the unexpectedly\is{expectations} late scenario of \textsc{still} \parencite[76]{vanBaar1997}. What is more, \textit{nozan} appears to only have this function, whereas \textit{oc} is widely polyfunctional. Functional range is also what distinguishes the two pairs of \textsc{still} expressions in Modern Hebrew\il{Hebrew, Modern} and Spanish:\il{Spanish} in each case, both have multiple functions, several of which are unique to one member of the pair.

\section{Data and analysis}
\label{sectionDataCollectionAnalysis}
\subsection{The sample}\label{sectionSample}\is{sample|(}
The survey I develop in \Cref{chapter2,chapter3} is based on a world-wide convenience sample of 76 languages from 45 distinct phyla.\footnote{Throughout this study, both genetic affiliations and macro-areas are based on glottolog \parencite{glottolog}.} \Cref{tabSampleNumbers} gives an overview of the number of languages and phyla for each of the six macro-areas; note that Austronesian and Afro-Asiatic are represented in two macro-areas each. The languages themselves are listed in \Cref{tableSample} and their approximate locations are plotted onto a map in \Cref{figureMap}.

\begin{table}[htb]
	\caption{Geographic breakdown of sample\label{tabSampleNumbers}}
	\begin{tabularx}{0.5\textwidth}{ld{1.0}d{1.0}}
		\lsptoprule	
		Macro-area & \multicolumn{1}{l}{Languages} & \multicolumn{1}{l}{Phyla} \\	
		\midrule
		Africa &  19 & 7 \\
		 Australia & 7 & 6 \\
		Eurasia & 16 &  9\\ 
		North America & 9 &8 \\ 
		Papunesia & 18 & 10\\ 
		South America & 7 & 7\\ 
		\lspbottomrule
	\end{tabularx}
\end{table}

I chose the sample languages primarily on the basis of the available resources that allowed me to confidently identify \textsc{still} expressions for which there were also good indications of additional functions. As phasal polarity is typically not given much attention in descriptive materials, my sample is heavily restricted by bibliographical bias. I nonetheless aimed for some degree of representativeness. As a very rough metric to this aim, I allotted each macro-area a share of the sample that approximates its relative contribution to the world's languages, based on the numbers in the appendix to \textcite{HammarstromDonohue2014}. That said, languages from Australia are overrepresented, as I opted to include more languages based on the existing literature on \textsc{still}–\lq only\rq{ }coexpression\is{restrictive} (\Cref{sectionExclusive}). I also kept an eye on genealogical variety and geographic spread within each macro-area. However, several of the larger and often better described phyla are represented with more than one language each; this is especially the case for Atlantic-Congo, Austronesian, and Indo-European.

\begin{table}
\caption{The sample languages \label{tableSample}}\small
\begin{tabularx}{\textwidth}{lQ}
\lsptoprule
Africa & Afro-Asiatic: Tashelhyit, Tunisian Arabic; Austronesian: Plateau Malagasy; Atlantic-Congo: Adamawa Fulfulde, Bende, Chuwabu, Ewe, Manda, Mundang, Nyangbo, Ruuli, Sango, Southern Ndebele, Swahili, Xhosa; Central Sudanic: Kaba; Katla-Tima: Tima; Mande:  Bambara; Nilotic: Barabayiiga-Gisamjanga Datooga\\

Australia & Bunaban:  Gooniyandi; Gunwinyguan: Wubuy; Mirndi: Jaminjung-Ngaliwurru; Pama-Nyungan: Gurindji, Martuthunira; Tangkic:  Kayardild; Yangmanic: Wardaman\\

Eurasia &  Afro-Asiatic: Hebrew (Modern); Indo-European: English, French, German, Serbian\hyp Croatian\hyp Bosnian, Spanish; Nakh-Dagestanian: Lezgian; Sino-Tibetan: Japhug, Hills Karbi, Mandarin Chinese, Northern Qiang; Tai-Kadai: Thai; Tungusic:  Udihe; Uralic: Tundra Nenets; Yeniseian: Ket; Yukaghir: Southern Yukaghir\\

North America & Athabaskan-Eyak-Tlingit: Slave; Eskimo-Aleut: Kalaallisut; Cochimi-Yuman: Maricopa; Mayan: Kekchí; Muskogean: Creek; Tsimshian: Gitxsan/Nisga'a; Siouan: Osage; Uto-Aztecan: Classical Nahuatl, Western Shoshoni\\

Papunesia & Anim: Coastal Marind; Austronesian: Acehnese, Chamorro, Lewotobi, Mateq, Paiwan, Rapanui, Saisiyat; Maybrat-Karon: Maybrat; Ndu: Iatmul; North Halmahera: Ternate-Tidore; Nuclear Torricelli: Bukiyip; Nuclear  Trans New Guinea: Kewa, Ma Manda, Western Dani; Timor-Alor-Pantar: Blagar; West Bomberai: Kalamang; Yam: Komnzo\\

South America & Arawan: Culina; Cariban: Trió; Isolate: Movima; Lengua-Mascoy: Southern Lengua; Pano-Tacanan: Cavineña; Quechuan: Huallaga-Huánuco Quechua; Macro-Je: Xavánte\\
	\lspbottomrule

\end{tabularx}
\end{table}

\begin{figure}
	\includegraphics[width=\textwidth]{figures/map.pdf}
	\caption{Map of the sample languages}
	\label{figureMap}
\end{figure}
\is{sample|)}

\subsection{Data sources}
\label{sectionDataSources}
In \Cref{sectionIdentifying} I hinted at specialised studies, descriptive grammars, and dictionaries as my primary resource for identifying \textsc{still} expressions. For each sample expression I aimed to mine as many data sources as feasible, so that I could obtain a high-resolution picture. Given the state of documentation of the world's languages, the extent of available materials naturally varied greatly. For well-studied and wide-spread languages such as German,\il{German} English,\il{English} \ili{French} or Spanish,\il{Spanish} I could rely on an abundance of papers and monographs pertinent to the subject matter and even perform the occasional internet search for specific patterns of usage. For some less described languages, extensive corpora or text collections allowed me to gain many insights. This includes cases like Kalamang,\il{Kalamang} Paiwan,\il{Paiwan} Ruuli,\il{Ruuli} and Movima,\il{Movima} to name just a few. In yet other cases, I had to make do with whatever examples I could find in the descriptive materials. Where possible, I also consulted expert linguists, who provided invaluable help with hard-to-interpret examples and difficult glosses. In a few cases, where my contacts included native speaker linguists, I was also able to elicit additional data. An overview of the data sources used is given in each language appendix.

\subsection{Data analysis}\label{sectionDataAnalysis}
Given the exploratory nature of the present study, I opted for a low-tech and qualitative approach to data analysis. Once I had established a sufficiently large sample, I created preliminary tabulations of the distinct uses I could identify based on both formal and semantic-pragmatic criteria (\Cref{sectionDistinctiveness}). Subsequently, I performed several recursive cycles of analysis in which I compared the relevant attestations across expressions, while at the same time incorporating additional data sources, until a clearer picture of the sets of recurrent uses, their characteristics and the criteria distinguishing them from each other emerged. To give an illustration of this approach, several sample expressions form part of collocations that denote a near past event (\Cref{sectionRemotenessPast}).\is{remoteness|(} This is illustrated in (\ref{exDataAnalysisTunisian}) for Tunisian Arabic \textit{māzāl}, here surfacing as \textit{māzil}.\il{Arabic, Tunisian}

\begin{exe}
	\ex Tunisian Arabic\il{Arabic, Tunisian}\label{exDataAnalysisTunisian}\\
	\gll Kun-t ānā \textbf{māzil}-\textbf{t} \textbf{kīf} \textbf{bdī}-\textbf{t} n-umġud̠̣ fī ṭaṛf il-lḥam had̠āya u-ẓaṛṣt-i ṛā-hi tnaṭṛ-it tanṭīṛa waḥd-a.\\
	\textsc{cop}.\textsc{pfv}-1\textsc{sg} 1\textsc{sg} still-1\textsc{sg} when begin.\textsc{pfv}-1\textsc{sg} 1\textsc{sg}-chew.\textsc{ipfv} in piece \textsc{def}-meat(\textsc{m}) \textsc{prox}.\textsc{sg}.\textsc{m} and-molar(\textsc{f})-\textsc{poss}.1\textsc{sg} \textsc{prestt}-3\textsc{sg}.\textsc{f} slip\_out.\textsc{pfv}-3\textsc{sg}.\textsc{f} slip\_out.\textsc{nmlz}(\textsc{f}) one-\textsc{sg}.\textsc{f}\\
	\glt \lq Ich hatte eben erst damit begonnen, auf dem Stück Fleisch herumzubeißen, da flog auch schon mein (Backen-)Zahn im hohen Bogen. [\textbf{I had only just begun} chewing on the piece of meat, when all of a sudden my molar tooth came flying out.]ʼ (\cite[651]{Singer1984}, cited in \cite{FischerEtAlTunisian})	
\end{exe}\is{remoteness|)}

A comparison of the available contextualised data points across the relevant languages showed that they consistently form part of the discursive background and, as in (\ref{exDataAnalysisTunisian}), they recurrently involve a \lq\lq past in the past\rq\rq{}. From this I could conclude that the relevant constructions do not denote \isi{tense} in the deictic sense of the term and that they involve a\is{aspect|(} \isi{topic time} that is fully contained in the situation's post-time (see \Cref{sectionTenseAspect} on notions of \isi{tense} and aspect). Both points, in turn, are consistent with the available descriptions of the verbal paradigms involved.\is{aspect|)}

Once the set of distinct uses and their semanto-pragmatic characteristics were established, I moved on to looking for areal patterns and implicational relationships and to developing the functional explanations given throughout \Cref{chapter2,chapter3}.

\section{Theoretical preliminaries}
\label{sectionKeyConcepts}
Before moving on to an overview of the structure of this book, it is necessary to address a few basic assumptions that guide my survey, as well as to define some key terms and concepts. It is safe to say that my understanding of the vast majority of them is rather uncontroversial in the functionalist realm of the language sciences, which is why I keep the following discussions fairly short.

\subsection{Polyfunctionality and the notion of \lq\lq function" or \lq\lq use\rq\rq}
\label{sectionDistinctiveness}
Throughout this book, I assume that multifaceted linguistic items, like the \textsc{still} expressions investigated here, are best understood as forming clusters of distinct functions or uses, which are linked together by a network of family likeness (e.g. \cite[ch. 4]{Haspelmath1997}; \cite{Janda2015}), as opposed to them having a single and highly abstract \textit{Gesamtbedeutung}. With that in mind, the baseline for all questions that guide the present study is that of which distinct  uses or functions can be identified in my sample (\Cref{sectionGoalsLimitations}). This raises the issue of what I understand by the terms \textsc{use} and \textsc{function} in the first place and when to establish a plurality of them. In what follows, I address both of these points.

To begin with, throughout this book I use the terms \textit{function} and \textit{use }interchangeably. Following the approach used in classical semantic maps (see \cite{GeorgakopulosPolis2008}; \cite{Haspelmath2003} for overviews), I furthermore make use of these labels in a wide sense that, to a large degree, conflates the distinction between semanticised meanings and those that arise in specific constructional and/or pragmatic contexts. Also in line with the aforementioned approach, I take two uses or functions to be distinct if a pair of expressions differs with respect to them, be it within a single language or cross-linguistically.\is{repetition|(} To give a fairly straightforward example, several sample expressions, including \ili{French} \textit{encore} and \il{Nahuatl, Classical|(}Classical Nahuatl \textit{oc}, have an iterative use. In both cases, such an \lq again\rq{ }reading can be accompanied by an event quantifier \lq one time', as illustrated in (\ref{exGoalsLimitationsIterativeIncrementCN}) for \textit{oc}. In descriptive grammars, the addition of such an item is usually described as stressing the iterative reading and/or as making it explicit vis-à-vis other contextually available interpretations (e.g. \cite[535]{BatchelorChebliSaadi2011}; \cite[1265]{Launey1986}). That is, iterative uses of \textit{encore} and \textit{oc} with or without a \lq one time\rq{ }element are described as mere variants of a single function.

\begin{exe}
	\ex Classical Nahuatl\label{exGoalsLimitationsIterativeIncrementCN}\\
	\gll Auh quēmman \textbf{oc} \textbf{cē}-\textbf{ppa} ti-tla-cuā-z?\\
	and when still one-time \textsc{subj}.2\textsc{sg}-\textsc{obj}.\textsc{indef}:\textsc{non}.\textsc{human}-eat-\textsc{prosp}\\
	\glt \lq Y a qué hora has de comer otra vez? [And when will you eat \textbf{again}?]' (\cite[505]{Carochi1645}, glosses added)
\end{exe}\il{Nahuatl, Classical|)}

However, comparable collocations are also attested on the basis of expressions such as \ili{German} \textit{noch} in (\ref{exGoalsLimitationsIterativeIncrementDE}). Crucially, \textit{noch} on its own cannot signal repetition. This by itself is a strong indicator that the repetition reading of \lq still one time\rq{ }phrases is, from a comparative point of view, not to be considered a mere variant of a more general iterative use. What is more, all relevant expressions in my sample have broadly \is{additive|(}additive uses which motivate the iterative reading with an event quantifier (\lq still one time' > \lq one more time' > \lq again'). This interpretation, in turn, finds independent evidence in the existence of iterative collocations featuring other markers of additivity, such as \ili{English} \textit{once more} or \ili{Spanish} \textit{otra vez} lit. \lq another time\rq{}.\is{additive|)}

\begin{exe}
	\ex \ili{German}\label{exGoalsLimitationsIterativeIncrementDE}\\
	\gll Opposition gewinn-t Wahl in Istanbul: İmamoğlu macht-'s \textbf{noch} \textbf{ein}-\textbf{mal}.\\
	opposition win-3\textsc{sg} election.\textsc{acc}.\textsc{sg} in I. I. make.3\textsc{sg}-3\textsc{sg}.\textsc{acc}.\textsc{n} still \textsc{indef}-time\\
	\glt \lq Opposition wins elections in Istanbul: İmamoğlu does it \textbf{again}.'
	\\(found online, glosses added)%\footnote{\url{https://taz.de/Opposition-gewinnt-Wahl-in-Istanbul/!5605032/} (07 April, 2022).}
\end{exe} 
\is{repetition|)} 

That said, it is important to keep in mind that the distinct uses I discuss in this book are, first and foremost, comparative constructs in the form of landmarks in a continuous conceptual space. What is more, the fact that two expressions share a common use does not necessarily mean that they are entirely congruent in this respect. For instance, in \Cref{sectionConcessiveConsequent} I discuss the use of \textsc{still} expressions as signals of a \isi{concessive} relationship between two propositions, which is a very common functional extension. Despite the shared common denominator of concessivity, the exact semanto-pragmatic import of the individual expressions differs considerably. To give just one example, though both \ili{English} \textit{still} and \ili{Spanish} \textit{todavía} can mark a clause as containing a concession,\is{concessive} only the former is felicitous in the context of (\ref{exUsesConcessives}).

\begin{exe}
	\ex\label{exUsesConcessives}\is{concessive}
	\begin{xlist}
	\sn[]{Context: Harry beats his dog.}
	\ex[]{\ili{English}\\
	\textit{\textbf{Still}, he is a nice guy.}}
	\ex[]{\ili{Spanish}}
	\sn[\#]{\gll \textbf{Todavía} es un buen tipo.\\
	still \textsc{cop}.3\textsc{sg} \textsc{indef}.\textsc{sg}.\textsc{m} good.\textsc{m} guy(\textsc{m})\\
	\glt(\cite[36]{EderlyCurco2016}, glosses added)}
	\end{xlist}
\end{exe}

\subsection{Motivation and semasiological change}
\label{sectionSemasiologicalChange}
As I pointed out in \Cref{sectionGoalsLimitations}, one of the questions that guide the present study is that of the \textsc{motivation} underlying the coexpression of a given function and the phasal polarity concept of \textsc{still}. With respect to this question, I take a usage-based approach, according to which 

\begin{quote}
[m]otivation aims to explain why it is at least possible and at best natural that this particular form–meaning correspondence should exist in a given language. Motivation is distinct from prediction: recognizing the motivation for a construction does not entail that the construction \textit{must} exist in that language or in any language. It simply explains why the construction \lq\lq{}makes sense\rq\rq{ }or is natural ... Functional and historical generalizations count as explanations (\cite[217]{Goldberg2006}, emphasis in the original)
\end{quote}

Functional generalisations include processes such as metonymy, whereas historical generalisations include attested usage patterns over time and established directions of functional change. With respect to the latter, I follow \textcite{Traugott1989} and subsequent works (e.g. \cite{Traugott1999}; \cite{TraugottDasher2002}) in assuming that there are three overarching tendencies, which are mono-directional and bring about a shift in semiotic status.\footnote{See \textcite{Diewald2011} on functional change and semiotic status.} These tendencies, summarised in \Cref{figureTendenciesSemanticChange}, are internalisation, proceduralisation,\is{proceduralisation} and change towards the expressive.\is{expressivity} A classic example of the first tendency is the internalisation of a perceived external association of \lq crude person\rq{ }with \lq farmer\rq{ }to the core meaning of \ili{English} \textit{boor}. In the present study, this tendency manifests itself, for instance, in the \isi{marginality} use of \textsc{still} expressions (\Cref{sectionMarginality}). An illustration of the tendency for \isi{proceduralisation} is the development of \ili{English} \textit{while} to a temporal \isi{connective} out of its original meaning \lq time span\rq{}. I discuss similar developments of \textsc{still} expressions in \Cref{sectionSimultaneity}.\is{simultaneity} As for the third and final tendency, the change towards the expressive,\is{expressivity} a textbook example is found in the extension from deontic modals to epistemic ones.\is{modality} Cases from the present study include the development of \textsc{still} expressions to markers of \isi{concessive} consequent clauses, and from there onward to concessive-expressive interjections\is{interjection} (\Cref{sectionConcessiveInterjections}).

\begin{figure}[hbt]
	\caption{Tendencies of semantic change based on \textcite[34–35]{Traugott1989}. Arrows depict possible feeding relationships.}
	\label{figureTendenciesSemanticChange}
	\begin{tikzpicture}
   		\node [rectangle, text width=.97\linewidth, 
            draw=black, text centered,fill=none, anchor=north west]
            (TendencyIII) {\strut\emph{III. Change towards the expressive}\\  Meanings become increasingly based in the speaker’s subjective\is{subjectivity} belief state\slash attitude toward the proposition.\is{expressivity}\strut};      
	    \node [rectangle, text width=0.485\linewidth-\hoehe,  anchor=north west, draw=black, text centered, fill=none, above=\hoehe of TendencyIII.north west, anchor=south west]
            (TendencyI) {\strut\emph{I. Internalisation}\\ \strut{}Meanings based in the external situation\strut{}\\ >  \strut{}meanings based in the internal situation.\strut};
     	 \node [rectangle,text width=0.485\linewidth-\hoehe,text centered,
            anchor=north east,
            draw=black,fill=none, above=\hoehe of TendencyIII.north east, anchor=south east]
            (TendencyII) {\strut\emph{II. Proceduralisation}\\  \strut{}Meanings based in the external or internal situation\strut{}\\>\strut{ }meanings based in the textual\is{textuality}\is{proceduralisation} and metalinguistic situation.\strut};
	   \draw [-latex, very thick] (TendencyI) to (TendencyII);
	   \draw[-latex, very thick] (TendencyI.south) -- ($(TendencyI.south)+(0,-\hoehe)$);
   		\draw[-latex, very thick] (TendencyII.south) -- ($(TendencyII.south)+(0,-\hoehe)$);
	\end{tikzpicture}         
\end{figure}

As the preceding discussion implies, at various points in \Cref{chapter2,chapter3} I opt for a diachronic perspective, in that I try to trace back the diachronic constellations and processes that lead to individual expressions getting equipped with specific functions. Crucially, \lq\lq if the internal semantic structure of a lexical category consists in clustered and overlapping readings\rq\rq, as I take these structures to be, \lq\lq{}then meaning changes …  are likely to originate in several older meanings simultaneously\rq\rq{ }\parencite[227]{MosegaardHansen2008}. This point becomes relevant, for instance, in my discussion of the \lq as far removed as\rq{ }use of \textsc{still} expressions in \Cref{sectionTimeScalar}.

\subsection{All things temporal}
\label{sectionTenseAspect}\is{tense|(}\is{aspect|(}\is{topic time|(}\is{utterance time|(}
When it comes to tense, aspect, and related notions, I mostly rely on the time-relational approach of \textcite{Klein1994}. The key to this framework lies in the assumption of three privileged time spans and a set of possible relationships between them. The three time spans are \textsc{utterance time}, \textsc{situation time} and \textsc{topic time}. With utterance time being a straightforward affair, situation time encompasses various types of actional content,\is{actionality} whereas topic time is an elaboration of \citeauthor{Reichenbach1947}'s (\citeyear{Reichenbach1947}) \textit{reference time} and represents \lq\lq
the time span to which the speaker’s claim on this occasion is confined\rq\rq{ }\parencite[4]{Klein1994}. In what follows, I sketch out how temporal and aspectual meaning arises in this framework, and I also briefly address the additional notion of temporal regions.

\subsubsection{Tense and aspect}\label{sectionTenseAspect2}
In \citeauthor{Klein1994}'s framework, the function of \textsc{tense} lies in relating topic time to utterance time. With the present tense, utterance time is included in topic time, whereas in the past tense topic time precedes utterance time, and the other way around for the future tense. This is in contrast to \textsc{aspectual} \textsc{operators} which establish different kinds of relationships between topic time and situation time. Unless explicitly stated otherwise, I assume that aspectual operators have their \lq\lq canonical" \parencite{Polancec2021} functions. Thus, with the\is{imperfective|(}\is{perfective|(} \textsc{imperfective} viewpoint, the interval of topic time is fully contained within the time of the situation, whereas with the \textsc{perfective} aspect it falls partially in the situation's post-time. \Cref{figureAspects} is a graphic schematisation of these two primary viewpoints.\il{Spanish|(} For a more hands-on example, consider (\ref{exAspectSpanish}), which features a verb in the Spanish past perfective inflection. By means of the past tense, topic time is narrowed down to one that precedes utterance time. The perfective aspect, on the other hand, establishes a topic time around the right edge of the \lq eating an apple\rq{ }situation, thus yielding a reading of closure.\is{utterance time|)}

\begin{figure}[hbt]
	\centering
	\begin{subfigure}[b]{0.4\linewidth}
	\centering
		\begin{tikzpicture}[node distance = 0pt]
			\node[mynode, text width=\breit, fill=cyan, very near start] (Sit1){Situation};
			\draw[-, densely dashed] ($(Sit1.north east)+ (-0.2pt-\hoehe,0) $) to ($(Sit1.south east)+ (-0.2pt-\hoehe,-0.5*\hoehe) $);
			\draw[-, densely dashed] ($(Sit1.north east)+ (-2*\hoehe-0.2pt,0) $) to ($(Sit1.south east)+ (-2*\hoehe-0.2pt,-0.5*\hoehe) $);
			\node (TT) [below, align=center, label distance=0] at ($(Sit1.south east)+(0.2pt-1.5*\hoehe,-0.5*\hoehe)$) {Topic\\time};		
			\draw[-latex, line width=0.5pt]  (Sit1.south west) to  ($(Sit1.south east)+(4pt,0)$) node [right] {t};
		\end{tikzpicture}
		\subcaption{Imperfective}
	\end{subfigure}
	\begin{subfigure}[b]{0.4\linewidth} 
		\centering
		\begin{tikzpicture}[node distance = 0pt]
			\node[mynode, text width=\schmal, fill=cyan, very near start] (Sit2){Situation};
			\draw[-, densely dashed] ($(Sit2.north east)+ (-0.2pt-0.5*\hoehe,0) $) to ($(Sit2.south east)+ (-0.2pt-0.5*\hoehe,-0.5*\hoehe) $);
			\draw[-, densely dashed] ($(Sit2.north east)+ (0.5*\hoehe-0.2pt,0) $) to ($(Sit2.south east)+ (0.5*\hoehe-0.2pt,-0.5*\hoehe) $);
			\draw[-, thick]($(Sit2.north east)+(-0.4pt,0)$) to ($(Sit2.south east) + (-0.4pt,0) $);
			\node (TT) [below, align=center, label distance=0] at ($(Sit2.south east)+(0.2pt,-0.5*\hoehe)$) {Topic\\time};
			\draw[-latex, line width=0.5pt]  (Sit2.south west) to  ($(Sit2.south east)+(\hoehe+1pt,0)$) node [right] {t};
		\end{tikzpicture}
		\subcaption{Perfective}
	\end{subfigure}
		\caption{Major aspectual viewpoints, based on \textcite{Klein1994}}\label{figureAspects}
\end{figure}


\begin{exe}
	\ex Spanish\label{exAspectSpanish}\\
	\gll María com-ió una manzana.\\
	M. eat-\textsc{pst}.\textsc{pfv}.3\textsc{sg} \textsc{indef}.\textsc{sg}.\textsc{f} apple(\textsc{f})\\
	\glt \lq Maria ate an apple.' (personal knowledge)
\end{exe}\il{Spanish|)}\is{tense|)} 

\is{anterior|(}A third type of viewpoint that is relevant to the present study is the \textsc{anterior}, also commonly referred to as the \textsc{perfect}. In \citeauthor{Klein1994}'s understanding, this viewpoint denotes that topic time is fully contained in a situation's post-time. At various points throughout the main body of this book, I make reference to an additional analysis in which the anterior viewpoint denotes a concept of its own, namely an ongoing state brought about by the termination of a situation (e.g. \cite[234–235]{Parsons1990}; \cite[106–109]{Smith1997}). \Cref{figureAnterior} illustrates these two understandings.

\begin{figure}[htb]
		\begin{subfigure}[b]{0.4\linewidth}
			\begin{tikzpicture}[node distance = 0pt]
			\node[mynode, text width=\superschmal, fill=cyan, very near start] (Sit3){Situation};
			\draw[-, densely dashed] ($(Sit3.north east)+ (-0.2pt+1*\hoehe,0) $) to ($(Sit3.south east)+ (-0.2pt+1*\hoehe,-0.5*\hoehe) $);
			\draw[-, densely dashed] ($(Sit3.north east)+ (2*\hoehe-0.2pt,0) $) to ($(Sit3.south east)+ (2*\hoehe-0.2pt,-0.5*\hoehe) $) node [below, xshift=-0.5*\hoehe, align=center, label distance=0]{Topic\\time};
			\draw[-, thick]($(Sit3.north east)+(-0.2pt,0)$) to ($(Sit3.south east) + (-0.2pt,0) $);
			\draw[-latex, thick, line width=0.5pt]  (Sit3.south west) to  ($(Sit3.south east)+(\schmal+4pt,0)$) node [right] {t};
		\end{tikzpicture}
		\subcaption{Time-relational}
	\end{subfigure}
	\begin{subfigure}[b]{0.4\linewidth}
		\begin{tikzpicture}[node distance = 0pt]
			\node[mynode, text width=\superschmal, fill=none, very near start] (situation){Situation};
\node[mynode, anchor=west, fill=cyan, text width=\schmal, anchor=west, align=right] at (situation.east) (region){State} ;

\coordinate (Mitte) at ($(situation.north east)!0.5!(situation.south east)$);
\draw[-latex] ($(Mitte)+(-0.375*\hoehe,0)$) to ($(Mitte)+(0.375*\hoehe,0)$);

			\draw[-, thick] ($(region.north west)+(-0.2pt,0)$) to ($(region.south west)+(-0.2pt,0)$);
			\draw[-, densely dashed] ($(situation.north east)+ (1*\hoehe-2pt,0) $) to ($(situation.south east)+ (1*\hoehe-2pt,-0.5*\hoehe) $);
			\draw[-, densely dashed] ($(situation.north east)+ (2*\hoehe-2pt,0) $) to ($(situation.south east)+ (2*\hoehe-2pt,-0.5*\hoehe) $) node [below, xshift=-0.5*\hoehe, align=center, label distance=0]{Topic\\time};
			\draw[-latex, line width=0.5pt]  (situation.south west) to  ($(region.south east)+(4pt,0)$) node [right] {t};			
		\end{tikzpicture}
		\subcaption{Stative}
	\end{subfigure}	
	\caption{Two analyses of the anterior viewpoint\label{figureAnterior}}
\end{figure}

Lastly, it is important to keep in mind that what is labelled a \lq\lq perfect(ive)" or the like in the descriptive tradition of a given language may correspond to more than one actual viewpoint. For instance, from \citeauthor{Singer1984}'s (\citeyear[298–300]{Singer1984}) description of the \lq\lq Perfektum" in Tunisian Arabic\il{Arabic, Tunisian} it becomes clear that this verbal paradigm has both perfective and anterior readings. At various points throughout this book I use the term \textsc{bounded viewpoint} as a handy cover term for the perfective and anterior viewpoints, as well as for instances of non-progressive\is{progressive} future situations.\is{topic time|)}\is{imperfective|)}\is{perfective|)}\is{anterior|)}

\subsubsection{Actionality}\is{actionality|(}\is{telicity|(} As far as situation time is concerned, I mostly rely on \citeauthor{Vendler1957}'s (\citeyear{Vendler1957}) familiar classification of predicates into four classes, based on the three binary parameters [±durative] [±dynamic] and [±telic]. Thus, according to \citeauthor{Vendler1957}, \textsc{states} like the denotation of \ili{English} \textit{to know} are durative (they extend in time), stative (they require no energy input) and atelic (they are not directed towards an inherent endpoint). \textsc{activities}, such as \textit{to run}, share [+durative] and [-telic] with states, but differ from states in being dynamic. Vendlerian \textsc{accomplishments} such as \textit{to build a house}, in turn, are distinguished from activities in being [+telic]. Lastly, \textsc{achievements} like \textit{to die} are considered to lack a significant duration, but to be dynamic and telic. In addition to Vendler's four classes, I follow \textcite{Smith1997} in assuming a group of \textsc{semelfactives}, which are punctual and dynamic but atelic occurrences such as \textit{to blink}. At certain points, primarily in the discussion of data from Bantu languages, I also adopt the notion of \textsc{inchoative verbs}. Simplifying slightly, these are predicates that encode a state plus a point or process of entrance into it; see \textcite{CranePersohn2019} and references therein for more discussion.\is{actionality|)}\is{telicity|)}\is{aspect|)}

\subsubsection{Temporal regions} An additional concept from \citeauthor{Klein1994}'s framework which will be relevant, among other things, in my discussion of degrees of temporal \isi{remoteness} (\Cref{sectionRemoteness}), is that of a \textsc{temporal} \textsc{region}. In broad strokes, the latter is the characteristic and contextually varying environment surrounding any given time span. For instance, the notion of \lq now\rq{ }can be understood as \lq\lq in the region of, but not before" \isi{utterance time} \parencite[156]{Klein1994}.

\subsection{Focus, focus quantifiers, and scales}\is{focus|(}
\label{sectionQuantificationScales}
Throughout various parts of this book, especially in \Cref{chapter3}, I discuss uses of \textsc{still} expressions as focus-sensitive quantifiers and in functions that involve scales\is{scale} of some sort. In what follows, I briefly lay out my understanding of these concepts.

\subsubsection{Focus and focus quantifiers}Following established practice, I assume that linguistic \textsc{focus} serves to relate the denotation of a given constituent to alternative meanings of the same type (\cite{Gast2012}; \cite{GastvanderAuwera2011}; \cite[ch. 1]{Hole2004}, \citeyear{Hole2006}; \cite[ch. 3]{Koenig1991}; \cite[ch. 1]{Rooth1985}, \citeyear{Rooth1992}, \citeyear{Rooth1996}; among others). Focus-sensitive operators, in turn, quantify over the resulting set of \textsc{context propositions} \parencite{Kay1990}, which are those propositions that differ from the given \textsc{text proposition} in the denotation of the constituent containing the focus. Within this realm, I distinguish between two primary types of operators.\is{additive|(} \textsc{additive} operators denote positive, existential quantification. That is, their use requires that the common ground contain at least one alternative denotation that can yield a true proposition; see (\ref{exBasicConceptsAdditive1}). Note that in the case of (the perhaps slightly misnomed) scalar\is{scale} additives, to which I turn below, the existential reading is a mere default assumption. As far as \textsc{still} expressions are concerned, I discuss a variety of additive functions in \Cref{sectionAdditive}.

\begin{exe}
	\ex\label{exBasicConceptsAdditive1}
	 \textit{Tom}\textsubscript{\textsc{foc}} \textit{attended the meeting}, \textbf{\textit{too}}.\\
	 \rightarrow { } At least one more person attended the meeting.
\end{exe} 

\is{restrictive|(}In contrast to additives, \textsc{restrictive} focus quantifiers denote negated\is{negation} quantification. The most familiar type of these are \textsc{exclusive} operators such as \ili{English} \textit{only}. These denote negated existential quantification, that is to say,  none of the context propositions is possibly true. Example (\ref{exBasicConceptsRestrictive1}) is an illustration.

\begin{exe}
	\ex\label{exBasicConceptsRestrictive1}
	\textbf{\textit{Only}} \textit{Oscar}\textsubscript{\textsc{foc}} \textit{attended the meeting}.\\
	\rightarrow{ }No one else attended the meeting.
\end{exe}
\largerpage[-1]

Exclusive operators are not the only type of restrictive focus quantifiers. Items from this larger class may also express negated\is{negation} universal quantification (\cite{Gast2012}; \cite{Hole2004}, \citeyear{Hole2006}, among others), which is to say, not all context propositions hold true. An example of such an operator is \ili{English} \textit{at least} in its evaluative and epistemic senses \parencite{Gast2012}, the first of which is illustrated in (\ref{exBasicConceptsRestrictive2}). For lack of an established term, I refer to this type of marker by the mnemonic label of \textsc{at least}-\textsc{type} operators. In \Cref{sectionTimeScalarRestrictive,sectionRestrictiveUebergeordnet} I discuss various restrictive functions of \textsc{still} expressions. A schematic summary of the typology of focus quantifiers just outlined is given in \Cref{figureFocusQuantifiers}.

\begin{exe}
	\ex\label{exBasicConceptsRestrictive2}
	\textit{He may not always meet all my needs, but} \textbf{\textit{at least}} {[}\textit{he doesn’t chase after other women or beat me up}]\textsubscript{\textsc{foc}}{}\textsubscript{\textsc{}}. (Forst, \textit{Experiencing father's embrace})\\
	\rightarrow{ }He has some, but not the maximal degree of husband qualities.
\end{exe}

\begin{figure}[hbt]
	\caption{Typology of focus quantifiers\label{figureFocusQuantifiers}}
	\begin{tikzpicture}
\tikzset{level distance=32mm,sibling distance=0pt}
\tikzset{grow'=right}
\tikzset{edge from parent/.style= 
    {thick, draw,
        edge from parent fork right}}
\tikzset{every tree node/.style={anchor=base west}}
        \tikzset{execute at begin node=\strut}
\Tree 
[. {Focus quantifiers} 
[.{Additive (\exists)}
]
[. {Restrictive}
[.{Exclusive (\neg\exists)} ]
[.{\textit{At least}-type (\neg\forall)} ]
] 
]
	\end{tikzpicture}
\end{figure}\is{additive|)}\is{restrictive|)}

\subsubsection{Scales and scalar operators}\label{sectionScales}\is{scale|(}\largerpage[-1]
 Having addressed the core tenets of focus and focus quantification, I now turn to \textsc{scales}. These are, in short, ordered sets of alternative denotations (e.g. \cite{Fauconnier1975}; \cite{Israel2001}, \citeyear[ch. 3]{Israel2011}; \cite[128–138]{Jacobs1983}). Their ordering relation may be strictly logical or mediated by \lq\lq general and contingent pragmatic knowledge about how the world normally seems to work\rq\rq{ }\parencite[53]{Israel2011}. Crucially, such an ordering based on (potentially pragmatically mediated) entailments means that any given scale has a single, unequivocal direction: lower ranking denotations yield fewer entailments and vice versa. However, as \textcite[17]{Israel2011} highlights, \lq\lq for every canonical scale there exists a corresponding inverted scale\rq\rq, such as scales of plausiblity vs. \textit{im}plausibility, scales of centrality vs. \textit{de}centrality, and so on. This point proves to be relevant in my discussion of the \textsc{still}-as-\isi{marginality} use in \Cref{sectionMarginality}. 

\textsc{scalar} \textsc{focus} \textsc{quantifiers} provide a ranking of propositions based on \textsc{scalar models} or propositional schemas, which are ordered by the degree of inferences they license in regard to a question under discussion (\cite{Gast2012}; \cite{GastvanderAuwera2011}, \citeyear{GastvanderAuwera2013}; \cite{Israel2001}; \cite[ch. 3]{Israel2011}; \cite{Kay1990}). Consider example (\ref{exBasicConceptsRestrictive2}). While the three accessible propositions themselves do not contain any scalar elements, they license inferences about different degrees of marital quality, as illustrated in (\ref{exBasicConceptsRestrictive2Implications}). 
\begin{exe}
	\ex\label{exBasicConceptsRestrictive2Implications}
	Question under discussion: How good of a husband is he?
	\begin{tabbing}	
		He doesn't chase after other women { }\= \rightarrow { } \= He's the perfect husband. { } \=\kill
		\emph{Propositional content}  \> \> \emph{Inference}\\
		He meets all my needs. \> \rightarrow \> He's the perfect husband.\> \tikzmark{husbandstart}\\
		He doesn't chase after other women.\> \rightarrow \> He's an average husband.\\
		He doesn't beat me up. \> \rightarrow \>  He has minimal qualities.\> \tikzmark{husbandend}\\
		(cf. \cite[113]{Gast2012})
		\begin{tikzpicture}[overlay,remember picture]
			\draw[-latex] ([yshift=\baselineskip] pic cs:husbandstart)  --   (pic cs:husbandend);
			\node[rotate=-90, above] at ($([yshift=\baselineskip] pic cs:husbandstart)!0.5! (pic cs:husbandend)$){entails};
		\end{tikzpicture}	
	\end{tabbing}
\end{exe}

\is{additive|(}Perhaps the best known scalar focus operators are the so-called \textsc{scalar additives}. These signal that the text proposition ranks higher in a propositional schema, i.e. is more informative, than all other propositions under consideration (\cite{GastvanderAuwera2011}, \citeyear{GastvanderAuwera2013}; \cite{Kay1990}). Example (\ref{exBasicConceptsScalarAdditiveCars}) is an illustration. As schematised in (\ref{exBasicConceptsScalarAdditiveCarsIllustration}), the text proposition \lq he commands the winds and the water\rq{ }licenses more inferences than all other accessible propositions, hence it outranks them. Importantly, and unlike for additive operators in the stricter sense, an existential reading (i.e. that the less informative propositions hold true as well) is only a default assumption which is derived from the scalar notion and arises under normal conversational conditions (\cite{Fauconnier1975}; \cite{GastvanderAuwera2011}; \cite{Schwartz2005}). This becomes particularly visible in an example like (\ref{exBasicConceptsScalarAdditivesObey2}), where one of the text propositions is negated\is{negation} without resulting in contradiction.
\begin{exe}
	\ex \ili{English}\label{exBasicConceptsScalarAdditiveCars}\\
	\textit{Who is this? He commands} \textbf{\textit{even}} [\textit{the winds and the water}]\textsubscript{\textsc{foc}}\textit{, and they obey him} (Luke 8:25, \textit{New International Version)}
	\ex \label{exBasicConceptsScalarAdditiveCarsIllustration}
	Question under discussion: How powerful is this man?
	\begin{tabbing}
		The winds and the water obey him. { }\= \rightarrow { } \= He has normal powers. { } \=\kill
		\emph{Propositional content}  \> \> \emph{Inference}\\
		The winds and the water obey him. \> \rightarrow \> He is very powerful.\> \tikzmark{start}\\
		His children obey him.   \> \rightarrow \> He has normal powers.\\
		His dogs obey him.  \> \rightarrow \> He has little power.\> \tikzmark{end}\\
		(cf. \cite[10]{GastvanderAuwera2011})
	\end{tabbing}
	\ex\textit{Can it be possible that his children do not obey him, given that \textbf{even}} [\textit{the
winds and the water}]\textsubscript{\textsc{foc}} \textit{do}?\label{exBasicConceptsScalarAdditivesObey2}
	\begin{tikzpicture}[overlay,remember picture]
		\draw[-latex] ([yshift=\baselineskip] pic cs:start)  --   (pic cs:end);
		\node[rotate=-90, above] at ($([yshift=\baselineskip] pic cs:start)!0.5! (pic cs:end)$){entails};
  \end{tikzpicture}	
\end{exe}

Lastly, scalar additives are usually said to come in two primary flavours (\cite{GastvanderAuwera2011}, \citeyear{GastvanderAuwera2013}). With \textsc{beyond} operators like \ili{German} \textit{sogar} or Italian \textit{perfino}, the degree of some salient property corresponding to the focus denotation correlates positively with the rank of the entire proposition. With \textsc{beneath} operators, such as \ili{German} \textit{auch nur} or \ili{English} \textit{so much as}, on the other hand, a high-ranking proposition is aligned with a low focus value. Correspondingly, the two types of scalar additives are felicitous in different environments. Yet a third type, \textsc{universal} scalar additives, can have both readings, depending on the context. \ili{English} \textit{even} is such a case. Thus, while in (\ref{exBasicConceptsScalarAdditiveCars}) above a high degree on a scale of control goes together with a highly informative proposition, under certain conditions, this alignment can be reversed, as illustrated in (\ref{exUniversalSAO}). I address scalar additive uses of \textsc{still} expressions in \Cref{sectionTimeScalar,sectionScalarAdditive}.

\begin{exe}
	\ex \ili{English}\label{exUniversalSAO}\\
	\textit{I refuse to believe that Bill \textbf{even} slapped\textup{\textsubscript{\textsc{foc}}} that man}.
	\begin{tabbing}
	\textsc{beneath} reading:x\=\kill
	\textsc{beneath} reading:\>\lq … let alone inflicted more harm on him\rq{}\\
	\textsc{beyond} reading: \>\lq … not only insulted but also …\rq{}\\
	(cf. \cite{GastvanderAuwera2011})
	\end{tabbing}
\end{exe}
\is{scale|)}\is{focus|)}\is{additive|)}

\section{The structure of this book}
\label{sectionStructureBook}
\subsection{The main body}
The wide array of functions attested for the sample expressions, many of which are tightly interconnected, proposes a serious challenge for their presentation in a linear format. For the sake of convenience, I have made a rough division in my survey between time-related uses (\Cref{chapter2}) and those uses that are primarily non-temporal (\Cref{chapter3}). Due to the nature of the subject matter this division is an imperfect one. For instance, in \Cref{sectionThusFarOnly} I discuss a use that combines \isi{restrictive} and temporal notions (\lq thus far only\rq{}) and in  \Cref{sectionFurtherTo}  I address the so-called \lq\lq further-to\rq\rq{ }\parencite{Klein2018} use, which involves both additivity\is{additive} and phasal notions (\lq do in addition to, and before moving on\rq{}). Within each of the two main chapters I have grouped together different sets of uses on a primarily semanto-pragmatic basis and lay the focus on those uses attested for at least two sample languages and/or expressions. The discussion of each use, is, for the most part, based on the templatic structure outlined in \Cref{figureTemplaticStructure}. At some points, however, I slightly deviate from this template, primarily by splitting the \lq\lq{}closer look\rq\rq{ }and \lq\lq{}discussion\rq\rq{ }components into several sub-parts each or by merging them into one. Lastly, \Cref{chapter4} contains a brief summary of my findings.

\begin{figure}
\caption{Templatic structure used in \Cref{chapter2,chapter3}}\label{figureTemplaticStructure}
\begin{PersohnBox}
	\begin{description}
	\item[Introduction:] A brief definition, together with one or more illustrations and a delimitation from superficially similar and/or related functions.
	\item[Distribution in the sample:] An overview of the expressions that have the relevant use, plus a summary of observable areal patterns and the question of morphosyntactic status. Where applicable, this includes a first glance at relevant collocational patterns or distinguishable subtypes of a use.
	\item[A closer look:] An in-depth descriptive exposition of the semanto-pragmatic characteristics and of recurrent usage patterns, accompanied by ample illustrations.
	\item[Discussion:] An examination of the conceptual and implicational relationships of the function under discussion to phasal polarity \textsc{still} and to other relevant functions. This may include a diachronic perspective.
	\end{description}
\end{PersohnBox}
\end{figure}

\subsection{About the appendices}
\label{sectionAboutAppendices}
The main body of this book is accompanied by an extensive set of appendices in the form of data sheets with additional comments and discussion. These appendices give an overview of the individual expressions found in each of the sample languages, in contrast to the discussion in \Cref{chapter2,chapter3}, which is based on individual sets of uses across expressions and languages. I have opted to include these appendices to not only increase transparency, but also to build a resource that allows an easier access to the often widely dispersed data, thereby hopefully encouraging subsequent research~-- be it by building on my interpretation of the data or by developing alternative analyses. The appendices are sorted by macro-area, followed by languages and expressions. For each item, they indicate the grounds on which it was classified as an exponent of \textsc{still} and give background information, such as questions of formal variation or known etymologies (where applicable). For each function of a given marker, bibliographic references are given and the key findings therein are summarised. Where my own analysis of the data differs from existing descriptions or expands on them, these points are given explicit discussion. The same applies to analyses of the more tentative type. The overview of each function is furthermore accompanied with illustration from a wide array of sources. As I discuss in \Cref{sectionDataAnalysis}, many of the most valuable examples were found dispersed throughout grammars, or deep within text collections and corpora, and in quite a few cases they lacked segmentation and glosses. To allow for an easier interpretation of the data, I added context information to examples from actual discourse, together with glosses and morpheme-by-morpheme segmentation. For more information about the templatic structure of the appendices, the reader is referred to \Cref{Appendix}.
