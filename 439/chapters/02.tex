\setchapterpreamble{\dictum[Rubén Darío: ¡Torres de Dios! ¡Poetas!]{
La mágica Esperanza anuncia el día\\
en que sobre la roca de armonía\\
expirará la pérfida sirena.\\
¡Esperad, esperemos \textit{todavía}!\footnotemark}}
\chapter{Temporal and aspectual uses}\label{chapter2}
\footnotetext{\lq Magic Hope foretells the day when on the rock of harmony the perfidious siren shall expire. Wait, let us \textit{still} wait!\rq}
\section{Introduction}
In this chapter, I survey uses of \textsc{still} expressions that are primarily time-related. I start out by reviewing a set of uses that lie on the fringes of phasal polarity \textsc{still} in \Cref{sectionFringes} and then turn to true functional extensions in \Crefrange{sectionOtherPhPConcepts}{sectionConnectiveFrameSetters}. In \Cref{sectionOtherPhPConcepts} I discuss functions that relate to other phasal polarity concepts, and in \Cref{sectionAdverbial} I address uses that can be subsumed under a broad label of other time-related \lq\lq adverbial\rq\rq{ }functions. These include, among other things, uses pertaining to repetition,\is{repetition} the signalling of \lq always, all the time\rq{}, or the marking of degrees of temporal remoteness.\is{remoteness} Lastly, in \Cref{sectionConnectiveFrameSetters} I turn to uses in the realm of temporal connectives\is{connective} and the modification of temporal adverbials.

\section{Uses on the fringes of \lq still\rq{}}
\addtocontents{toc}{\protect\setcounter{tocdepth}{2}}
\label{sectionFringes}
\subsection{Introduction}
In the following subsections, I discuss several uses that lie on the fringes of \textsc{still}. By this I mean patterns of use for which there is not a clear-cut answer as to whether they can be subsumed under the phasal polarity function or constitute genuine extensions into other domains. More specifically, in \Cref{sectionScalar}, I examine the use of \textsc{still} expressions in the context of monotone changes along some scale.\is{scale} In \Cref{sectionNochImmer}, I discuss \ili{German} \textit{noch} in an idiomatic collocational pattern with the meaning of \lq thus far always, thus far everyone\rq{}. In \Cref{sectionContinuedIteration}, I turn to a use in which \textsc{still} expressions signal continued iteration \lq yet again\rq{}. Lastly, in \Cref{sectionSame} I address functions that pertain to identity.\is{identity}

Before moving on to an examination of these fringe uses, a few words on some cases that I do not discuss here. To begin with, for the \textsc{still}-as-\isi{marginality} use (\lq a penguin still counts as a bird\rq{}), I refer the reader to \Cref{sectionMarginality}.  Secondly, about a handful of expressions in my sample are used as modifiers of a temporal adverbial instead of the main predication (\lq it is still T, when e occurs\rq, \cite[202]{Loebner1989}). I reserve my discussion of these cases for \Cref{sectionTemporalFrameTT}. Another set of sample expressions can signal a combination of temporal inclusion and a \isi{restrictive} meaning element (\lq thus far only\rq{}). I discuss those in the context of \isi{restrictive} operators, in \Cref{sectionThusFarOnly}. In addition, there are several semi-transparent patterns of use that I address in the context of interactional functions, in \Cref{sectionBroadlyModal}.\il{Spanish|(} Lastly, for the most part I do not offer a discussion of \textsc{still} expressions as phrase-level adverbial modifiers, such as in (\ref{exFringesIntroSpanishIncipiente}). I take such instances to involve phasal polarity in the context of language-specific, but generalised syntactic\is{syntax} configurations or processes.\footnote{See, for instance, \textcite[13.16l]{RAEGramatica} on Spanish, or \textcite[2091–2096]{ZifonunEtAl1997} on German.\il{German}}

\begin{exe}
	\ex Spanish\label{exFringesIntroSpanishIncipiente}\\
	\gll Es ciert-o que exist-e un gran nivel de frustración por \textup{[}\textbf{la} \textup{[}\textbf{todavía} \textbf{limitad}-\textbf{a}\textup{]\textsubscript{\textsc{adjp}}} \textbf{participación}\textup{]\textsubscript{\textsc{np}}} en el proceso polític-o.\\
\textsc{cop}.3\textsc{sg} certain-\textsc{m} \textsc{comp} exist-3\textsc{sg} \textsc{indef}.\textsc{sg}.\textsc{m} great level(\textsc{m}) of frustration for \phantom{[}\textsc{def}.\textsc{sg}.\textsc{f} \phantom{[}still limited-\textsc{f} participation(\textsc{f}) in \textsc{def}.\textsc{sg}.\textsc{m} process(\textsc{m}) political-\textsc{m}\\
	\glt \lq It is true that there is a high level of frustration over \textbf{the} \textbf{still limited} \textbf{participation} in the political process.\rq{ }(CORPES XXI,  glosses added)
\end{exe}\il{Spanish|)}

\subsection{\lq Still\rq{ }in contexts of scalar variables}\label{sectionScalar}\is{scale|(}
\subsubsection{Introduction (part 1)}
In this subsection I discuss the use of \textsc{still} expressions in contexts involving a (possible or factual) development on a scale. By this, I mean contexts that feature an ordered set of alternatives that encompasses more than a polar opposition between \isi{persistence} and discontinuations.\is{discontinuation}\il{Spanish|(} Examples (\ref{exScalarIntroSpanish}, \ref{exScalarIntroTunisian}) are illustrations.\il{Arabic, Tunisian} In (\ref{exScalarIntroSpanish}) there is a decrease in fleet size, whereas (\ref{exScalarIntroTunisian}) involves a limited advancement on a scale of age.


\begin{exe}
		\ex Spanish \label{exScalarIntroSpanish}\\
		\gll La Armada ten-ía 337 vehículo-s y se desprend-ió de tan solo nueve. \textbf{Todavía} \textbf{tiene} \textbf{una} \textbf{flota} \textbf{de} \textbf{328}.\\
\textsc{def}.\textsc{sg}.\textsc{f} navy(\textsc{f}) have-\textsc{pst}.\textsc{ipfv}.3\textsc{sg} 337 vehicle-\textsc{pl} and \textsc{refl}.3 get\_rid\_off-\textsc{pst}.\textsc{pfv}.3\textsc{sg} of so only nine still have.3\textsc{sg} \textsc{indef}.\textsc{sg}.\textsc{f} fleet(\textsc{f}) of 328\\
	\glt \lq The navy had 337 vehicles and got rid of only nine. \textbf{It still has a fleet of 328}.\rq{ }(CORPES XXI,  glosses added)

	\ex Tunisian Arabic\il{Arabic, Tunisian}\label{exScalarIntroTunisian}\\
	\gll Humma māzāl-u mā-bilġ-ū-š \textbf{māzāl} \textbf{ʕumᵊṛ}-\textbf{hum} \textbf{tisʕ} \textbf{ᵊsnīn}.\\
	3\textsc{pl}.\textsc{m} still-3\textsc{pl}.\textsc{m} \textsc{neg}-reach\_puberty.\textsc{pfv}-3\textsc{pl}.\textsc{m}-\textsc{neg} still(.3\textsc{sg}.\textsc{m}) age(\textsc{m})-\textsc{poss}.3\textsc{pl}.\textsc{m} nine year.\textsc{pl}\\
	\glt \lq Sie sind noch nicht in der Pubertät, sie sind erst ungefähr neun Jahre alt. [They haven’t reached puberty yet, they are \textbf{still only about nine years old}.]\rq{ }(\cite[220–221]{RittBenmimoun2011}, glosses by \cite{FischerEtAlTunisian})	
\end{exe}

Whereas (\ref{exScalarIntroSpanish}, \ref{exScalarIntroTunisian}) feature numerical scales,\il{Spanish|)}\il{Quechua, Huallaga-Huánuco|(} example (\ref{exScalarIntroQuechua}) involves a development on a degree scale, namely a reduction in the intensity of precipitation. Lastly, the variable in question may itself be the extension of a time span. For instance, what is at stake in (\ref{exScalarIntroTunisianDays}) is the extent of the stretch between time of utterance\is{utterance time} and the feast.
 
\begin{exe}
	\ex Huallaga-Huánuco Quechua\label{exScalarIntroQuechua}\\
	\gll Manana sinchi	tamya-r-pis \textbf{tamya}-\textbf{ga}	\textbf{poga}-\textbf{pa}-\textbf{yca}-\textbf{n}-\textbf{rä}-\textbf{mi}.\\
	no\_longer strong rain-\textsc{adv}.\textsc{ss}-also rain-\textsc{top} drizzle-\textsc{appl}-\textsc{ipfv}-3-still-\textsc{evid}\\
	\glt \lq Although it is no longer raining hard, \textbf{a few drops are still falling}.\rq
	\\ (\cite[420]{WeberEtAl2008}, glosses added)
	\ex Tunisian Arabic\il{Arabic, Tunisian}\label{exScalarIntroTunisianDays}\\
	\gll \textbf{Māzāl} \textbf{yūm}-\textbf{ēn}  l-l-ʕīd.\\
	still day-\textsc{du}  to-\textsc{def}-feast\\
	\glt \lq Il reste encore deux jours avant la fête. [\textbf{Two days remain} before the feast.]\rq{ }(\cite[1736]{MarcaisGuiga19581961}, glosses by \cite{FischerEtAlTunisian})	
\end{exe}\il{Quechua, Huallaga-Huánuco|)}


\subsubsection{Introduction (part 2)} 
The use of phasal polarity expressions in scalar contexts led to an extensive debate during the first wave of research interest in these items (\cite{vanderAuwera1991BeyondDuality}, \citeyear{vanderAuwera1993}; \cite{Garrido1992}; \cite{Loebner1989}, \citeyear{Loebner1999}; \cite{Mittwoch1993}). As this dispute has direct implications for my examination of the sample data, a brief excursion is in order. Much of the just-mentioned controversy centres around peculiarities of German,\il{German|(} but is said to have theoretical implications of cross-linguistic validity. The bone of contention lies, for all intents and purposes, in the interpretation of two sets of empirical facts. The first set pertains to issues of compatibility. Thus, in the context of monotone functions, German \textit{noch} only combines with decreases, as in (\ref{exScalarGermanBooks}) and in (\ref{exScalarIntroGermanWenigeMeter}) below.

\begin{exe}
	\ex German\label{exScalarGermanBooks}\\
	 Context: I had ten copies of a book.\\
	\gll Ich hab' \textbf{noch} \textbf{fünf} \textup{(}übrig\textup{)}. \\
	1\textsc{sg} have.1\textsc{sg} still five \phantom{(}left\\
	\glt \lq I still have five (left).' (personal knowledge)
\end{exe}

Unlike an expression such as Tunisian Arabic \textit{māzāl} in (\ref{exScalarIntroTunisian}) above,\il{Arabic, Tunisian} or \ili{English} \textit{still} (in collocations like \textit{still only}, \textit{still no more than}), German \textit{noch} is generally infelicitous when a limited increase is at stake and the alternatives are not polar opposites.\is{restrictive|(} These contexts are instead covered by an entirely distinct item, the restrictive marker \textit{erst} \lq no more than\rq{},  lit. \lq{}first, erstwhile\rq{}, as illustrated in (\ref{exScalarIntroGermanErst}).

\begin{exe}
	\ex German\label{exScalarIntroGermanErst}\\
	\gll Er ist \textbf{erst} \textbf{zwanzig} \textbf{Jahr}-\textbf{e} alt.\\
	3\textsc{sg}.\textsc{m} \textsc{cop}.3\textsc{sg} first twenty year-\textsc{nom}.\textsc{pl} old\\
	\glt \lq He is (still) only twenty years old.\rq{ }(\cite[s.v. \textit{erst}]{DWDS},  glosses added)
\end{exe}\is{restrictive|)}

\is{syntax|(}
Secondly, when a scalar variable is at issue German \textit{noch} displays the characteristic syntactic behaviour of a \isi{focus} particle, in that it can become a syntactic sister to the constituent containing the focus.\is{focus} For instance, in (\ref{exScalarIntroGermanWenigeMeter}) \textit{noch} is an adjunct to the noun phrase \textit{wenige Meter} \lq a few meters\rq{}, as is evident from the fact that they occupy the forefield position of a V2 clause together. This behaviour is shared, among the relevant expressions, with \textit{erst} and with the other phasal polarity items, as shown for the \textsc{already}\is{already} expression \textit{schon} in (\ref{exScalarIntroGermanSpenden}).\pagebreak

\begin{exe}
	\ex German
	\begin{xlist}
		\ex\label{exScalarIntroGermanWenigeMeter}
\gll \textup{[}\textbf{Noch} \textup{[}\textbf{wenig}-\textbf{e} \textbf{Meter}\textup{]]\textsubscript{\textsc{np}}} war-en es bis zu-r Staffel-übergabe…\\
	\phantom{[}still \phantom{[}few-\textsc{nom}.\textsc{pl} meter.\textsc{pl} \textsc{cop}.\textsc{pst}-3\textsc{pl} 3\textsc{sg}.\textsc{n} until to-\textsc{def}.\textsc{dat}.\textsc{sg}.\textsc{f} relay-handover(\textsc{f})\\
	\glt \lq \textbf{A few meters were left} until the passing of the baton…\rq
	\ex\label{exScalarIntroGermanSpenden}\is{already}
	\gll
	\textup{[}\textbf{Schon} \textup{[}\textbf{acht} \textbf{Mal}\textup{]]\textsubscript{\textsc{advp}}}  hat Christoph … eine Spende-n-aktion zu-m Geburt-s-tag ge-start-et.\\
	already \phantom{[}eight time have.3\textsc{sg} C. {} \textsc{indef}.\textsc{acc}.\textsc{sg}.\textsc{f} donation-\textsc{lnk}-campaign(\textsc{f}) to-\textsc{def}.\textsc{dat}.\textsc{sg}.\textsc{m} birth-\textsc{lnk}-day(\textsc{m}) \textsc{ptcp}-start-\textsc{ptcp}\\
	\glt \lq [It's] \textbf{already} [been] \textbf{eight} \textbf{times} [that] Christoph … has initiated a donation campaign on his birthday.\rq{ }(found online, glosses added)%\footnote{\url{https://www.betterplace.org/c/spenden-sammeln/als-privatperson/geburtstag} and \url{https://www.wz.de/nrw/kreis- viersen/viersen/ein- schueler- staffellauf- fuer- mehr- miteinander_aid-30775241} (26 January, 2023).}
		\end{xlist}
\end{exe}

The interpretation of these characteristics goes hand in hand with the acceptance or rejection of \citeauthor{Loebner1989}'s (\citeyear{Loebner1989}, \citeyear{Loebner1999}) \lq\lq duality hypothesis\rq\rq{}. Without going too deep down the rabbit hole,\footnote{See \textcite[ch 3.1.]{MosegaardHansen2008} for a more concise summary.} this hypothesis claims that sets of operators manifest as neatly organised paradigms in the form of Aristotelian squares of oppositions. Given that in contexts of scalar increases and affirmative polarity the \textsc{already}\is{already} expression \textit{schon} patterns with \isi{restrictive} \textit{erst} rather than with the \textsc{still} expression \textit{noch}, a distinct scalar paradigm is posited for German. \textit{Noch} in contexts like (\ref{exScalarIntroGermanWenigeMeter}), on the other hand, is said by \textcite{Loebner1999} to form part of yet a third paradigm, together with \textit{nur noch} \lq only as little as … anymore (lit. only still)\rq{} occupying the second positive corner. While this point relates, first and foremost, to the question of how systems of operators are conceived of, it is tightly interwoven with a particular semantic analysis. Thus, under the duality hypothesis it is assumed that the separate scalar paradigms also have a distinct meaning structure, namely that their members are operators that i) associate with a lower constituent and ii) vest this constituent with a predetermined temporal dynamic in the form of a presupposition about different values at adjacent times. In other words, in an example like (\ref{exScalarGermanBooks}), repeated below, \textit{noch} is said to not function as a sentence adverb. Instead, it is said to relate the denotation of \textit{fünf} \lq five\rq{ }to higher values at an earlier time (\approx{}\lq I have the now reduced number of five\rq{}).
\pagebreak

\begin{exe}
	\exr{exScalarGermanBooks} German\\
	 Context: I had ten copies of a book.\\
	\gll Ich hab' \textbf{noch} \textbf{fünf} \textup{(}übrig\textup{)}. \\
	1\textsc{sg} have.1\textsc{sg} still five \phantom{(}left\\
	\glt \lq I still have five (left).' (personal knowledge)
\end{exe}

Opponents of \citeauthor{Loebner1989}'s hypothesis, on the other hand, understand the contribution of \textit{noch} in the context of scalar variables as the same that it has in binary contexts. Seen from this point of view, example (\ref{exScalarGermanBooks}) denotes my persistent\is{persistence} possession of five books vis-à-vis the possible \isi{discontinuation} of this state-of-affairs. The reading of a decrease over time then goes back to the combined effects of phasal polarity, scalar semantics and pragmatic calculus.\footnote{I purposely avoid the terms \textit{focus}\is{focus}  and \textit{scope} here, as they are sometimes used in confusing ways in the literature. In line with \textcite{Loebner1989}, I assume that the relevant expressions have sentential scope in both cases and under both analyses. What is debated is the size of the constituent within the expression's scope and for which alternatives are considered. While it is tempting to portray the latter issue as a question of broad vs. narrow focus,\is{focus}  phasal polarity itself is not associated with \isi{focus} in the same way that operators like \textit{also}, \textit{even}, or \textit{only} are (\cite{Beck2020}; \cite{Klein2018}), unless we assume that the (propositional content of the) entire sentence may be in focus; the latter stance is taken, for instance, in \textcite{Krifka2000}.} To come straight to the point, below I advocate for the latter stance. Nonetheless, a separate examination of scalar contexts allows for valuable insights into parameters of variation across expressions and languages as well as into derived usage patterns.\il{German|)}

To conclude, a cross-linguistic examination of \textsc{still} expressions in scalar contexts must consider the parameter of syntactic distinctiveness and whether a given item is compatible with decrease and/or increase contexts. As a derived question, such an examination needs to address the issue of whether there is any reason to assume that the semantic import of \textsc{still} expressions in contexts of scalar variables differs in any way from their contribution to run-of-the-mill polar contexts.\is{syntax|)}

\subsubsection{Distribution in the sample}\largerpage
Against the backdrop of the questions just outlined, \Cref{tableScalar} lists the sample expressions for which the data include scalar contexts and it indicates whether the expressions in question show a distinct syntactic\is{syntax} behaviour there. \Cref{tableScalar} also indicates whether a given expression is attested with decreases and/or increases along some scale. For a numeric breakdown of the latter parameter, see \Cref{tableScalarQuant} below. Note that \Cref{tableScalar,tableScalarQuant} do not give information as to whether increase contexts generally require the addition of a \isi{restrictive} \lq only\rq{ }operator, which is a question I discuss separately further down the line. 
 
As a brief summary, in the sample data scalar contexts are attested for 35 expressions from 33 languages (37 expressions and 35 languages if problematic cases are included), covering all six macro-areas. If I were to consider the use in scalar contexts as featuring a distinct function, these numbers would make it both the most frequent and the geographically most widespread function in my sample. However, as I argue below, there is no principled reason for such an assumption, at least on semantic grounds. On a more fine-grained level, the sample data reveal a marked asymmetry in the types of attestations. Thus, decrease contexts are attested for more than twice as many expressions as increase contexts (see \Cref{tableScalarQuant}). Lastly, a distinct syntactic\is{syntax} behaviour is found with only two expressions, namely the case of \ili{German} \textit{noch} discussed above and Serbian\hyp Croatian\hyp Bosnian \textit{još}.\il{Serbian}\il{Croatian}\il{Bosnian}

In what follows, I first take another look at the question of syntactic\is{syntax} distinctiveness. I then discuss the distribution of decrease vs. increase contexts. Subsequently, I examine each of the two types of contexts in some more depth; this includes a discussion of usage patterns that are derived from and/or border on \textsc{still} in scalar contexts. Lastly, I offer a brief summary and conclusion.

\begin{table}[p]
	\caption{Scalar contexts. \emph{Notes}: *: Only one clear-cut example in the data. †: Only one example, which may be subsumed under the \lq thus far only\rq{ }use (\Cref{sectionThusFarOnly}). ‡: Incompatible in their unaugmented forms. §: Borderline case of a \textsc{still} expression or tentative inclusion.}\label{tableScalar}
	\footnotesize
		\begin{tabularx}{\textwidth}{llllccc}
			\lsptoprule
			M.-area & Language & Expr. & Appx. & \rot{Decreases} & \rot{Increases}  & \rot{\parbox[b]{\widthof{Divergent}}{Divergent\\ syntax\is{syntax}}}\\
			\midrule
			Africa & \ili{Manda} & (\textit{a})\textit{kona} & \ref{appendixMandaScalar} & y & \phantom{*}y* & n\\
			& Plateau Malagasy\il{Malagasy, Plateau} & \textit{mbola} & \ref{appendixMalagasyScalar} & y & n & n\\
			& \ili{Ruuli} & \textit{kya}- & \ref{appendixRuuliScalar} & y & \phantom{\textsuperscript{†}}(y)\textsuperscript{†}  & n\\
			& Southern Ndebele & \textit{sa}- & \ref{appendixSouthernNdebeleScalar} & y & y & n\\
			& \ili{Swahili} & \textit{bado} & \ref{appendixSwahiliScalar} & y & y &n\\
			& \ili{Tashelhyit} & \textit{sul} & \ref{appendixTashelhyitScalar} & \phantom{*}y* & \phantom{*}y* & n\\			
			& Tunisian Arabic\il{Arabic, Tunisian} & \textit{māzāl} & \ref{appendixTunisianMazalScalar} & y & y & n\\
			& \ili{Xhosa} & \textit{sa}- & \ref{appendixXhosaScalar} &  y & y & n\\
			Australia & \ili{Kayardild} & =(\textit{i})\textit{da} & \ref{appendixKayardildScalar} &  \phantom{*}y* & n & n\\
			& \ili{Wardaman} & \textit{gayawun} & \ref{appendixWardamanScalar} &  \phantom{*}y* & n & n\\
			Eurasia& \ili{English} & \textit{still} & \ref{appendixEnglishScalar} & y & y & n\\
			&\ili{French} & \textit{encore} & \ref{appendixFrenchEncoreScalar} & y & y & n\\
			&\ili{German} & \textit{noch} & \ref{appendixGermanScalar} & y & \phantom{\textsuperscript{‡}}(incompat.)\textsuperscript{‡} & y\\
			& Hebrew (Modern)\il{Hebrew, Modern} & \textit{ʕadayin} & \ref{appendixHebrewAdayinScalar} & y & y & n\\
			& & \textit{ʕod} & \ref{appendixHebrewAdayinScalar} & y & y & n\\
			&Mandarin Chinese\il{Chinese, Mandarin} & \textit{hái} & \ref{appendixMandarinScalar} & y & incompat. & n\\
			&Serb.-Croat.-Bosn. & \textit{još} & \ref{appendixBCMSScalar} & y & \phantom{\textsuperscript{‡}}(incompat.)\textsuperscript{‡} &y\\
			& \ili{Spanish} & \textit{aún} & \ref{appendixSpanishAunScalar} & y &y &n\\
			&& \textit{todavía} & \ref{appendixSpanishTodaviaScalar}&y &y &n\\
			& \ili{Thai} & \textit{yaŋ} & \ref{appendixThaiScalar} & y & y & n\\
			& Tundra Nenets\il{Nenets, Tundra} & \textit{təmna} & \ref{appendixTundraNenetsScalar} & y & n & n\\
			& \ili{Udihe} & \textit{xai}(\textit{si}) & \ref{appendixUdiheScalar} & y & n & n\\
			N. America & Classical Nahuatl\il{Nahuatl, Classical} & \textit{oc} & \ref{appendixClassicalNahuatlScalar} & y & n &n\\
			& \ili{Creek} & (\textit{i})\textit{monk}\textsuperscript{§} & \ref{appendixCreekScalar} & y & n & n\\
			& \ili{Kekchí} & \textit{toj} & \ref{appendixKekchiScalar} & y & n & n\\
			Papunesia & \ili{Acehnese} & \textit{mantöng} & \ref{appendixAcehneseScalar} & y & n & n\\
			& \ili{Chamorro} & \textit{ha'} & \ref{appendixChamorroScalar} & y & n & n\\
			& Coastal Marind\il{Marind, Coastal} & \textit{ndom} & \ref{appendixCoastalMarindScalar} & y & n & n\\
			& \ili{Kalamang} & \textit{tok} & \ref{appendixKalamangScalar} & y & n &n\\
			& \ili{Komnzo} & \textit{komnzo} & \ref{appendixKomnzoScalar} & y & n & n\\
			&  \ili{Mateq} & \textit{bayu}\textsuperscript{§} & \ref{appendixMateqStill} & n & y & n\\
			& \ili{Paiwan} & =\textit{anan} & \ref{appendixPaiwanScalar} & y & y & n\\
			& \ili{Saisiyat} & \textit{nahan} & \ref{appendixSaisiyatScalar} &  \phantom{*}y* & n &n\\
			& Ternate-Tidore\il{Ternate}\il{Tidore}  & \textit{moju} & \ref{appendixTernateScalar} &  y & \phantom{*}y*  & n\\
			S. America & \ili{Cavineña} & =\textit{jari} & \ref{appendixCavinenaScalar} & \phantom{*}y* & n & n\\
			& H.-H. Quechua\il{Quechua, Huallaga-Huánuco} & -\textit{raq} & \ref{appendixQuechuaScalar} & y & y & n\\
			& Southern Lengua\il{Lengua, Southern} & \textit{makham} & \ref{appendixEnxetSurScalar} & y & n & n\\
			\lspbottomrule	
		\end{tabularx}
\end{table}

\subsubsection{A closer look: Scalar uses as structurally distinct}\is{focus|(}\is{syntax|(}\largerpage
As I indicated initially, one of the points that the debate around the status of phasal polarity expressions in scalar contexts is based on is the  fact that the relevant \ili{German} items show a distinct syntactic behaviour there. That is, they act like focus particles, being able to form a single constituent with a phrase containing the focus. This is shown in (\ref{exScalarIntroGermanWenigeMeter}), where \textit{noch} and the noun phrase \lq few meters\rq{} together occupy the initial position of a V2 clause.

\begin{exe}
		\exr{exScalarIntroGermanWenigeMeter} \ili{German}\\
\gll \textup{[}\textbf{Noch} \textup{[}\textbf{wenig}-\textbf{e} \textbf{Meter}\textup{]]\textsubscript{\textsc{np}}} war-en es bis zu-r Staffel-übergabe…\\
	\phantom{[}still \phantom{[}few-\textsc{nom}.\textsc{pl} meter.\textsc{pl} \textsc{cop}.\textsc{pst}-3\textsc{pl} 3\textsc{sg}.\textsc{n} until to-\textsc{def}.\textsc{dat}.\textsc{sg}.\textsc{f} relay-handover(\textsc{f})\\
	\glt \lq Only a few meters were left until the passing of the baton…\rq
	\\(found online, glosses added)
\end{exe} 

The only other expression in my sample that displays this type of distinct behaviour is Serbian\hyp Croatian\hyp Bosnian\il{Serbian}\il{Croatian}\il{Bosnian} \textit{još}. The examples in (\ref{exScalarStructuralSerbian}) illustrate this item as moving through the sentence together with its sister constituent \textit{pet knigja} \lq five books\rq{}. As a further diagnostic of constituency, note how \textit{još pet knigja} in (\ref{exScalarStructuralSerbianB}) precedes the second-position clitic complex \textit{mi je}.
\begin{exe}
	 \ex 	\label{exScalarStructuralSerbian}

	\begin{xlist}	 
	\exi{}Serbian-Croatian-Bosnian\il{Serbian}\il{Croatian}\il{Bosnian}
	 \exi{}Context: I had ten books and I've given away some of them.
		\ex  
		\gll Osta-lo mi je \textup{[}\textbf{još} \textup{[}\textbf{pet} \textbf{knjiga}\textup{]]\textsubscript{\textsc{np}}}.\\
  	 remain.\textsc{pfv}.\textsc{ptcp}-\textsc{sg}.\textsc{n} 1\textsc{sg}.\textsc{dat} \textsc{cop}.3\textsc{sg} \phantom{[}still \phantom{[}five book.\textsc{gen}.\textsc{sg}\\
	 \glt  \lq I still have five books left.\rq{}
		
	\ex\label{exScalarStructuralSerbianB}\il{Serbian}\il{Croatian}\il{Bosnian}
	\gll \textup{[}\textbf{Još} \textup{[}\textbf{pet} \textbf{knjiga}\textup{]]}\textsubscript{\textsc{np}} mi je osta-lo.\\
	   \phantom{[}still \phantom{[}five  book.\textsc{gen}.\textsc{pl} 1\textsc{sg}.\textsc{dat} \textsc{cop}.3\textsc{sg} remain.\textsc{pfv}.\textsc{ptcp}-\textsc{sg}.\textsc{n}\\
	 \glt \lq What I have still left is five books.\rq{ }(Stefan Savič, p.c)
	 \end{xlist}
\end{exe}

On purely structural grounds, one may therefore consider the cases of \ili{German} \textit{noch} and Serbian\hyp Croatian\hyp Bosnian\il{Serbian}\il{Croatian}\il{Bosnian} \textit{još} as indicators of a distinct scalar use. As I discuss below, there is, however, no need to stipulate a distinct semantics that differs from phasal polarity \textsc{still}. Lastly, note that, in several cases in the sample, what appears to be comparable syntactic behaviour at first glance turns out to be a mere superficial similarity upon closer examination. For instance, the \ili{Chamorro} expression \textit{ha'} is an enclitic (though written separately from its host) that attaches to its focus, which in phasal polarity function is the main predicate. The two attestations of scalar contexts in the data, which include the example in (\ref{exScalarStructuralChamorro}), both involve non-verbal predication without an overt copula. That is, any possible distinction in the associated constituent of \textit{ha'} is masked from the outset.

\begin{exe}
	\ex \ili{Chamorro}\label{exScalarStructuralChamorro}\\
\gll \textbf{Bu}\sim{}\textbf{bula} \textbf{ha'} \textbf{tinanum} ti hu tungu' na manmamakannu'.\\
	\textsc{cont}\sim{}many \textsc{still} plant \textsc{neg} 1\textsc{sg}.\textsc{rl} know \textsc{rel} \textsc{pl}.edible\\
	\glt \lq \textbf{There are still many plants} I did not know are edible. (lit. they are still many, the plants …)'
	\parencite[325]{Chung2020}
\end{exe}\is{focus|)}\is{syntax|)}

\subsubsection{A closer look: Distribution of decrease vs. increase contexts}
Moving on to a closer examination of different contexts, \Cref{tableScalarQuant} gives a numeric breakdown of the attestations of scalar decreases vs. increases in the sample data. As can be gathered, decrease contexts are attested for roughly over twice as many expressions and languages as contexts involving a limited increase. What is more, the sample data point towards the implicational near-universal in (\ref{exUniversalScalar}), the statistical nature of the latter being due to \il{Mateq|(}Mateq \textit{bayu}. In the available data, this item is only attested in increase contexts. That said, its inclusion in the sample is somewhat tentative, in that \textit{bayu} seems to lexicalise a combination of phasal polarity \textsc{still} and \isi{restrictive} \lq only\rq{}, which would explain its comparatively unusual distribution. I discuss this expression in more depth in \Cref{sectionThusFarOnly}.\il{Mateq|)}

\begin{table}
	\caption{Quantitative breakdown of scalar contexts. \emph{Note}: Counts in parentheses include borderline cases.\label{tableScalarQuant}}
	\begin{tabular}{l rl rl}
			\lsptoprule
			& \multicolumn{2}{c}{Expressions} & \multicolumn{2}{c}{Languages}\\
			\midrule
			Overall & 35 & (37) & 33 & (35)\\
			\midrule
			Decrease contexts & 35 & (36) & 33 & (34)\\ 
			Increase contexts & 15 & (20) & 13 & (17)\\	
			\lspbottomrule
		\end{tabular}
\end{table}


\begin{exe}
	\ex If a \textsc{still} expression is compatible with contexts of a scalar increase, it is more likely than not to also be compatible with contexts of a decrease.\label{exUniversalScalar}
\end{exe}

Briefly setting aside the cases of \ili{German} \textit{noch}, Mandarin Chinese\il{Chinese, Mandarin} \textit{hái} and Serbian\hyp Croatian\hyp Bosnian\il{Serbian}\il{Croatian}\il{Bosnian} \textit{još}, it appears that the marked asymmetry observed in \Cref{tableScalarQuant} boils down to two main factors. First, and as I pointed out initially, I only considered those instances that feature a non-binary set of alternatives, given that it is only in these cases that the question of a distinct scalar function arises in the first place. This inherently favours decrease contexts. Consider example (\ref{exScalarAcehnese}).\il{Acehnese} Accessible alternatives to the text proposition \lq there is much [sugar] at home\rq{ }are \lq there is little at home\rq{ }and \lq there is none at home\rq{}. Both would constitute valid rhetorical arguments for purchasing sugar, and the two of them can be subsumed under the evoked \isi{discontinuation} scenario (\lq there isn't much anymore\rq{}).

\begin{exe}
	\ex \ili{Acehnese}\label{exScalarAcehnese}\\
	\gll Hana lōn-bloe saka sabab \textbf{mantöng} \textbf{le} \textbf{di} \textbf{rumoh}.\\
	\textsc{neg} 1\textsc{sg}-buy sugar because still many in house\\
	\glt \lq I am not buying any sugar because there is \textbf{still much at home}.'
	\\\parencite[175]{Asyik1987}
\end{exe}

In many examples that could be read as involving an increase, on the other hand, the existence of some entity is presupposed, such that a zero value is not even under consideration. Paired with a predicate that forms part of an antonym pair, such as small vs. big in (\ref{exScalarRuuliPortion}), this results in numerous attestations and, due to the often limited data, entire expressions that I had to exclude from the counts.\il{Ruuli}

\begin{exe}
	\ex \ili{Ruuli}\label{exScalarRuuliPortion}\\
	\textit{OBukama, abantu abali omu Bukama abasing\textup{[}a\textup{]} obwingi tibasomere.}\\
	\lq{}The cultural institution, most of people in the cultural institution are not educated.\rq{}\\
	\gll Aka-tundu aka-som-ere \textbf{ka}-\textbf{kya}-\textbf{li} \textbf{ka}-\textbf{tono}.\\
	\textsc{ncl}12-piece \textsc{rel}.\textsc{subj}.\textsc{ncl}12-read-\textsc{pfv} \textsc{subj}.\textsc{ncl}12-still-\textsc{cop} \textsc{ncl}12-small\\
	\glt \lq The portion that is educated \textbf{is still small}.\rq{ }(\cite{RuuliCorpus}, glosses added)
\end{exe}

A second, related reason for the predominance of decrease contexts lies in discursive considerations. In many instances of a decrease along some scale, what stands in the communicative foreground is \isi{persistence} as such, rather than the precise value in question. For instance, in (\ref{exScalarDecreaseEnglishPounds}) the main message is \lq we hadn't spent all our money\rq{}, which invites the use of \textit{still}. Where the monotonic relationship is positive, on the other hand, it is often a particularly low value that constitutes the key takeaway.\is{restrictive|(} In these cases, using a restrictive marker does the job just as well, especially if the variable in question is inherently time-dependent, as it is in (\ref{exScalarOnlyYears}). This option may even be preferred for reasons of brevity, namely if the use of a \textsc{still} expression would necessarily or preferably go along with using an additional restrictive marker anyway.

\begin{exe}
	\ex 
	\begin{xlist}	
	\exi{}\ili{English}
	\ex 
	\textit{We \textbf{still} \textbf{had} \textbf{a} \textbf{few} \textbf{pounds} in our pockets, so we very quickly decided, why not!} (Cassidy, \textit{Indifferently}) 	\label{exScalarDecreaseEnglishPounds}
	
	\ex\il{English}
	\textit{Harry's parents were \textbf{only 21 years old} when they were killed by Voldemort.} (found online)\label{exScalarOnlyYears}%\footnote{\url{https://www.reddit.com/r/harrypotter/comments/stdrb7/harrys_parents_were_only_21_years_old_when_they/} (09 March, 2023).}
	\end{xlist}
\end{exe}\is{restrictive|)}

\il{German|(}The last point provides a direct link to the case of German. As I pointed out above, part of the debate about the alleged special status of scalar contexts is due to the general incompatibility of German \textit{noch} with increases,\is{restrictive|(} whether it occurs alone or as part of a \lq still only\rq{ }collocation. Increase contexts are instead covered by a separate item, the scalar restrictive marker \textit{erst}. In the sample languages, a parallel situation obtains in Serbian\hyp Croatian\hyp Bosnian,\il{Serbian}\il{Croatian}\il{Bosnian} where increase contexts require \textit{tek} \lq no more than\rq{ }instead of the \textsc{still} expression \textit{još}. Similarly, in Mandarin Chinese\il{Chinese, Mandarin} \textit{cái} \lq only\rq{ }is the marker of choice with increase functions. I am aware of no other sample expressions that are subject to a similar restriction. What is more, it is noteworthy that in German and Serbian\hyp Croatian\hyp Bosnian\il{Serbian}\il{Croatian}\il{Bosnian} this division of labour only affects the unaugmented forms \textit{noch} and \textit{još}. Their emphatic versions (i.e. dedicated to unexpectedly\is{expectations} late scenarios) of the form \lq always still/still always\rq{ }are amply attested in increase contexts, where they take scope over a restrictive operator, as in (\ref{exScalarImmerNochNur}, \ref{exScalarImmerNochSerbian}). This is likely because there are no evaluative counterparts to \textit{erst} and \textit{tek} in the same fashion that \textit{noch} and \textit{još} have \textit{immer noch}/\textit{noch immer} and \textit{još} \mbox{\textit{uv}(\textit{ij})\textit{ek}}.

\begin{exe}
	\ex German\label{exScalarImmerNochNur}\\
	\gll 2012 ist der Durchbruch der E-Books. Sie \textbf{mach}-\textbf{en} \textbf{zwar} \textbf{immer} \textbf{noch} \textbf{nur} \textbf{fünf} \textbf{Prozent} \textbf{des} \textbf{Umsatz}-\textbf{es} aus\textbf{,} ihr Anteil steig-t aber rapide.\\
	2012 \textsc{cop}.3\textsc{sg} \textsc{def}.\textsc{nom}.\textsc{sg}.\textsc{m} breakthrough(\textsc{m}) \textsc{def}.\textsc{gen}.\textsc{pl} e-books 3\textsc{pl} make-3\textsc{pl} though always still only five percent \textsc{def}.\textsc{gen}.\textsc{sg}.\textsc{m} revenue(\textsc{m})-\textsc{gen} out \textsc{poss}.3\textsc{pl}:\textsc{sg}.\textsc{m} share(\textsc{m}) rise-3\textsc{sg} however rapidly\\
	\glt \lq 2012 will be the breakthrough year for e-books. \textbf{Though they still only account for five percent of revenue}, their share is rising rapidly.\rq{}
	\\(found online, glosses added)%\footnote{\url{https://www.abendblatt.de/kultur-live/article109757622/Schwarzenegger-auf-Buchmesse-Ein-totaler-Auftritt.html} (08 March, 2023).}
	\il{German|)}
	\ex Serbian-Croatian-Bosnian\il{Serbian}\il{Croatian}\il{Bosnian}\label{exScalarImmerNochSerbian}\\
	\gll Svi će vam reći – naći ćete to na web stranici, ali kod nas \textbf{još} \textbf{uvijek} \textbf{samo} \textbf{tri} \textbf{posto} \textbf{stanovništva} ima pristup Internetu.\\
	everyone will.3\textsc{sg} 2\textsc{sg}.\textsc{dat} say.\textsc{pfv}.\textsc{inf} { } find.\textsc{pfv}.\textsc{inf} will.2\textsc{pl} \textsc{dem}.\textsc{nom}.\textsc{sg}.\textsc{n} on web page.\textsc{loc}.\textsc{sg} but near 1\textsc{pl}.\textsc{gen} still always only three percent population.\textsc{gen}.\textsc{sg} have.3\textsc{sg} access.\textsc{acc}.\textsc{sg} internet.\textsc{dat}.\textsc{sg}\\
	\glt \lq Everyone will tell you \lq\lq{}You’ll find it [the information you need] on the website\rq\rq{}, but where we live, \textbf{still only three percent of the population} has access to the internet.\rq{ }(found online, glosses added)%\footnote{\url{https://www.slobodnaevropa.org/a/823402.html} (08 March, 2023).}
\end{exe}\is{restrictive|)}

In what follows, I describe and examine both types of contexts separately and in some more depth. This discussion includes a look at some derived usage patterns.

\subsubsection{A closer look: Scalar decreases}\il{Spanish|(} The Spanish example (\ref{exScalarIntroSpanish}), repeated below, is an illustration of a prototypical decrease context.\pagebreak

\begin{exe}
	\exr{exScalarIntroSpanish} Spanish\\
	\gll La Armada ten-ía 337 vehículo-s y se desprend-ió de tan solo nueve. \textbf{Todavía} \textbf{tiene} \textbf{una} \textbf{flota} \textbf{de} \textbf{328}.\\
\textsc{def}.\textsc{sg}.\textsc{f} navy(\textsc{f}) have-\textsc{pst}.\textsc{ipfv}.3\textsc{sg} 337 vehicle-\textsc{pl} and \textsc{refl}.3 get\_rid\_off-\textsc{pst}.\textsc{pfv}.3\textsc{sg} of so only nine still have.3\textsc{sg} \textsc{indef}.\textsc{sg}.\textsc{f} fleet(\textsc{f}) of 328\\
	\glt \lq The navy had 337 vehicles and got rid off only nine. \textbf{It still has a fleet of 328}.\rq{ }(CORPES XXI,  glosses added)
\end{exe}\il{Spanish|)}

The notion of a decrement is often made explicit, or given extra emphasis, by the use of an additional expression meaning \lq remain, left\rq{},\il{Kalamang|(} as in (\ref{exScalarDecreaseKalamangBallsack}, \ref{exScalarDecreaseTundraNenets}) and in (\ref{exScalarDecreaseSerbianJoSSamo}) below.

\begin{exe}
	\ex Kalamang\label{exScalarDecreaseKalamangBallsack}\\
	Context: A giant has been killed.\\
	\gll Mu he din=at uw=i koyet mu he di=sara karuar keitko na na na na na na na mindi bo \textbf{tinggal} \textbf{elkin}-\textbf{un}=\textbf{a} \textbf{tok}.\\
	3\textsc{pl} already fire=\textsc{obj} kindle=\textsc{lnk} finish 3\textsc{pl} already \textsc{caus}=ascend smoke\_dry above consume consume consume consume consume consume consume like\_that go remain ballsack-\textsc{poss}.3=\textsc{foc} still\\
	\glt \lq After kindling the fire, they put him up the drying rack, ate and ate and ate until \textbf{only his ballsack was still} [\textbf{there}].'
	\\(\cite{Visser2021b},  glosses added)\il{Kalamang|)}

	\ex Tundra Nenets\il{Nenets, Tundra}\label{exScalarDecreaseTundraNenets}\\
	\gll \textbf{Təmna} \textbf{n′ax°r} \textbf{yúq} \textbf{n′encawey°} \textbf{xayi} m′inc′°maq yax°-naq.\\
	still three ten stretch\_of\_river remain.3\textsc{sg} journey:\textsc{poss}.1\textsc{pl} place:\textsc{dat}.\textsc{sg}-\textsc{poss}.1\textsc{pl}\\
	\glt \lq{}Noch sind auf dem Fluss, den wir daherfahren, dreissig Strecken übrig. [There \textbf{still remain} \textbf{thirty} \textbf{stretches} of river on our journey.]\rq
	\\(\cite[319]{Lehtisalo1956},  glosses added) 
\end{exe}

\is{restrictive|(}
\is{focus|(}\il{German|(}German \textit{noch} and Serbian-Croatian-Bosnian\il{Serbian}\il{Croatian}\il{Bosnian} \textit{još} are frequently found in combination with a restrictive marker to signal a great degree of reduction. \is{syntax|(}In syntactic terms, these collocations, which are illustrated in (\ref{exScalarDecreaseGermanNurNoch}, \ref{exScalarDecreaseSerbianJoSSamo}), behave like single, complex focus particles.\is{focus|)}

\begin{exe}	
	\ex German\label{exScalarDecreaseGermanNurNoch}\\
	\gll Heute können \textbf{nur} \textbf{noch} \textbf{10\%} \textbf{all}-\textbf{er} \textbf{Sami} von der Rentier-zucht und vo-m Fisch-fang alleine leb-en.\\
	today can.3\textsc{pl} only still 10\% all-\textsc{gen}.\textsc{pl} S. of \textsc{def}.\textsc{dat}.\textsc{sg}.\textsc{f} raindeer-breeding(\textsc{f}) and of-\textsc{def}.\textsc{dat}.\textsc{sg}.\textsc{m} fish-catch(\textsc{m}) alone live-\textsc{inf}\\
	\glt \lq These days, \textbf{no more than} \textbf{10\% of all Sami people} can \textbf{still} make a living of reindeer breeding and fishing alone.\rq{ }(Heyne, \textit{…Nur noch bis dahinten!}, glosses added)

	\ex Serbian-Croatian-Bosnian\il{Serbian}\il{Croatian}\il{Bosnian}\label{exScalarDecreaseSerbianJoSSamo}\\
	\gll Njemu ostaje \textbf{samo} \textbf{još} \textbf{jedno}: što pre pobeći odavde.\\
	 3\textsc{sg}.\textsc{dat}.\textsc{m} remain.\textsc{ipfv}.3\textsc{sg} only still one.\textsc{acc}.\textsc{sg}.\textsc{n}  as before flee.\textsc{pfv}.\textsc{inf} from\_here\\
	 \glt \lq Für ihn blieb nur noch das eine zu tun: so schnell wie möglich zu fliehen. [There was \textbf{only one option left} for him: to get away as soon as possible.]' (\cite[248]{KontrastiveGrammatik}, glosses added)
\end{exe}

German additionally has the fixed expression \textit{kaum noch} \lq hardly anymore\rq{}, lit. \lq{}hardly still\rq{}, for a significant reduction in frequency or intensity; see (\ref{exScalarDecreaseGermanKaumNoch}).

\begin{exe}
	\ex German\label{exScalarDecreaseGermanKaumNoch}\\
	\gll Er trank \textbf{kaum} \textbf{noch}.\\
	3\textsc{sg}.\textsc{m} drink.\textsc{pst}.3\textsc{sg} hardly still\\
	\glt \lq He hardly (ever) took a drink anymore.\rq{}
	\\(\cite[172]{KoenigEtAl1993}, glosses added)
\end{exe}\il{German|)}\is{syntax|)}

\il{Komnzo|(}Though different in terms of surface structure, a comparable \lq only still\rq{ }combination is found in the Komnzo example (\ref{exScalarDecreaseKomnzo}). In the sample data, similar attestations are found for \ili{Udihe} \mbox{\textit{xai}(\textit{si})} and \ili{Kalamang} \textit{tok}. Other languages, however, make use of the mirror image, so to speak, in the form of \lq already only\rq{}.\il{Spanish|(} Example (\ref{exScalarDecreaseYaSolo}) illustrates this for Spanish.\footnote{\Textcite{vanderAuwera1993} reports the same for \ili{Hungarian} (not in my sample).}
\begin{exe}
	 \ex Komnzo\label{exScalarDecreaseKomnzo}\\
	 Context: A girl has been attacked and eaten by a crocodile.\\
	 \gll \textbf{Ebar}=\textbf{nzo} \textbf{komnzo} zwarärm.\\
	head=just still 3\textsc{sg}.\textsc{f}:\textsc{pst}:\textsc{dur}:be\\
	\glt \lq Just her head was still there.' \parencite{Doehler2020}\il{Komnzo|)}
 
	\ex Spanish\label{exScalarDecreaseYaSolo}\\
	\gll Pedro \textbf{ya} \textbf{tiene} \textbf{solo} cien \textup{[}libro-s\textup{]}.\\
	P. already have.3\textsc{sg} only hundred \phantom{[}book-\textsc{pl}\\
	\glt \lq Pedro \textbf{already has only} a hundred [books].\rq{ }\parencite[384]{Garrido1992}
\end{exe}\il{Spanish|)}\is{focus|)}\is{restrictive|)}

\subsubsection{Scalar decreases: Borderline cases and derived uses}
Before moving on to a discussion of the semantic underpinnings of \textsc{still} expressions in decrease contexts, it is worthwhile addressing a few usage patterns that build on them. The first case involves Plateau Malagasy\il{Malagasy, Plateau} \textit{mbola}. While, in the data I consulted, this expression is not attested in combination with scalar predicates, it features in the  fixed collocations \textit{mbola aiza} \lq still where\rq{ }and \textit{mbola rahoviana} \lq still when\rq{ }that refer to large remaining quantities of distance and time, respectively. Example (\ref{exScalarDecreaseMalagasy}) illustrates \textit{mbola aiza}. These collocations bear some resemblance to the idiomatic \il{German|(}German expression \textit{noch hin sein} \lq be a long way off\rq{}, lit. \lq{}still be thither\rq{}. The latter, illustrated in (\ref{exScalarDecreaseNochHin}), also implies a large remaining quantity (of time), even if none is explicitly stated.\footnote{The opposite perspective is also available in German: \textit{Das ist schon her} \lq A long time has passed since (lit. that's already hither)\rq{}.}

\begin{exe}
	\ex Plateau Malagasy\il{Malagasy, Plateau}\label{exScalarDecreaseMalagasy}\\
	\gll \textbf{Mbola} \textbf{aiza} i<za>ny.\\
	still where \textsc{dem}.\textsc{sg}<\textsc{invis}>\\
	\glt \lq{}C'est encore loin, il c'en faut encore de beaucoup, il y en a encore pour longtemps (a marcher, a progresser). [\textbf{It’s still far away}, it’s still a long way to go, it’s still a long time walking or moving.]'
	\\(\cite[398]{Dez1980}, glosses added)

	\ex German\label{exScalarDecreaseNochHin}\\
	\gll Bis Weihnachten \textbf{ist} \textbf{noch} \textbf{hin}, aber wir ess-en Keks-e i-m Sommer auch gerne.\\
	until Christmas \textsc{cop}.3\textsc{sg} still thither but 1\textsc{pl} eat-1\textsc{pl} cookie-\textsc{pl} in-\textsc{def}.\textsc{dat}.\textsc{sg}.\textsc{m} summer(\textsc{m}) also gladly\\
	\glt \lq Christmas \textbf{is still a long way off,} but we also enjoy cookies during summer time.\rq{ }(found online, glosses added)%\footnote{\url{https://www.chefkoch.de/rezepte/2289401365066405/Haselnussplaetzchen.html} (13 February, 2023).}
\end{exe}\il{German|)}

In fact, usages pertaining to remaining stretches of time constitute a recurring pattern in the data. Thus, \ili{Manda} (\mbox{\textit{a})\textit{kona}} forms part of a conventionalised construction in which it pairs with an adjectival predicate consisting of  \textit{chokópi} \lq little\rq{ }marked for one of the locative noun classes. This construction signals proximity\is{remoteness} of a future situation (\appref{appendixMandaSoon}), as illustrated in (\ref{exScalarMandaSoon}). Its meaning obviously builds on a decrease function \lq there is little [time] still [left] > soon\rq{}. It furthermore finds a partial parallel in Swahili,\il{Swahili} the local lingua franca; see the fixed phrase in (\ref{exScalarExtensionsSwahili}). Notwithstanding its semantic transparency, \textcite{Bernander2021} points out that the \ili{Manda} construction shows first signs of grammaticalisation.\is{grammaticalisation} Thus, it is not only restricted to expletive-like locative subjects, but also invariably follows the foregrounded predicate, occupying what in a mono-clausal structure corresponds to the typical adverb position. 

\begin{exe}
	\ex \ili{Manda}\label{exScalarMandaSoon}\\
	\gll Ya-u-sóv-i \textbf{p}-\textbf{ákóna} \textbf{pa}-\textbf{chokópi}.\\
	\textsc{fut}-\textsc{subj}.\textsc{ncl}14-be\_lost-\textsc{fut} \textsc{subj}.\textsc{ncl}16(\textsc{loc})-still \textsc{ncl}16(\textsc{loc})-little\\
	\glt \lq It will \textbf{soon} be lost (the flour) (lit. it will be lost, it is still a little).\rq{ }\parencite[58]{Bernander2021}
	\ex \ili{Swahili}\label{exScalarExtensionsSwahili}\\
	\gll Bado kidogo.\\
	still a\_little\\
	\glt \lq Pas de sitôt encore, pas de suite, un peu plus tarde [Not quite yet, not immediately, a little later]\rq{ }(\cite[85]{Sacleux19391941},  glosses added)
\end{exe}
	
Outside of East Africa, a case that is comparable in meaning is found in Tundra Nenets;\il{Nenets, Tundra} see (\ref{exScalarTundraNenetsWenig}).\footnote{See \textcite[133–134]{Nikolaeva2014} on comparative\is{comparison} \mbox{-\textit{rka}} as a minimiser.} In syntactic\is{syntax} terms, the collocation \mbox{\textit{təmna}-\textit{rka}} is partially irregular, in that it constitutes one of only two instances in which the \textsc{still} expression \textit{təmna} can serve as the main predicate of a sentence.

\begin{exe}
	\ex Tundra Nenets\il{Nenets, Tundra}\label{exScalarTundraNenetsWenig}\\
	\gll \textbf{Təmna}-rka.\\
	still-\textsc{cmpr}\\
	\glt \lq Still a bit, not quite yet.\rq{ } (\cite[458]{Lehtisalo1956}; \cite[624]{Tereshchenko2008})
\end{exe}

\il{Nahuatl, Classical|(}The duration of a time span, no matter whether short or long, also lies at the heart of two biclausal structures in Classical Nahuatl (\appref{appendixClassicalNahuatlTimeSpanBefore}). The first of these is schematically illustrated in (\ref{exTimeBeforePatternA}) and exemplified in (\ref{exTimeBeforePatternAex}). An illustration of the second structure, again both in schematised form and with an attested examples, is given in (\ref{exTimeSpanBeforeB}).\largerpage[-1]

\begin{exe}
	\ex \label{exTimeSpanBeforeA}
	\begin{xlist}
		\exi{}Classical Nahuatl
		\ex\label{exTimeBeforePatternA}
		 \gll \textup{[}oc \textit{p}\textup{]} \textup{[}in (ìcuāc) \textit{q}\textup{]}\\
		\phantom{[}still \textit{p} \phantom{[}\textsc{det} \phantom{(}then/when \textit{q} \\
		\glt \lq it is still \textit{p} when \textit{q} (i.e. it is still a certain amount of time until \textit{q})'\\(see \cite[1268]{Launey1986})
	
	\ex\label{exTimeBeforePatternAex}
		\gll \textup{[}\textbf{In}-\textbf{in} \textbf{ca} \textbf{oc} \textbf{huècauh}\textup{]} \textup{[}in mo-chīhua-tīuh\textup{]}.\\
		\phantom{[}\textsc{det}-\textsc{prox} \textsc{pred} still long\_time \phantom{[}\textsc{det} \textsc{subj}.3:\textsc{refl}.3-do-go.\textsc{ipfv}\\
		\glt \lq Ceci se produira dans longtemps. [It will happen \textbf{in} \textbf{a} \textbf{long} \textbf{time}.]' (\cite[1268]{Launey1986},  glosses added)	
		\end{xlist}
	\ex \label{exTimeSpanBeforeB}
	\begin{xlist}
		\exi{}Classical Nahuatl
		\ex\label{exTimeBeforePatternB}
		\gll \textup{[}oc \textit{p}\textup{]} \textup{[}(in) \textit{q}-z\textup{/}-quiuh\textup{]}\\
		\phantom{[}still \textit{p} \phantom{[(}\textsc{det} \textit{q}-{\textsc{prosp}/-\textsc{come}.\textsc{ipfv}}\\
		\glt \lq It is still \textit{p} that \textit{q} is going to happen/is approaching (i.e. it is still a certain amount of time with \textit{q} being anticipated).\rq{}\\(see \cite[1268]{Launey1986})
	
		\ex\label{exTimeBeforePatternBex}	
		\gll \textup{[}\textup{[}\textbf{Oc} \textbf{yuh} \textbf{macuil}-\textbf{ilhuitl}\textup{]} \textup{[}àci-quiuh in to-tlàtò-ca-uh\textup{]}\textup{]}, in ō-tech-tlalhuì-quê.\\
	\phantom{[}\phantom{[}still thus five-day \phantom{[}arrive-\textsc{come}.\textsc{ipfv} \textsc{det} \textsc{poss}.1\textsc{pl}-king-\textsc{lnk}-\textsc{poss} \textsc{det}  \textsc{aug}-\textsc{subj}.3:\textsc{obj}.1\textsc{pl}-warn-\textsc{pst}.\textsc{pfv}:\textsc{pl}\\
		\glt \lq Cinco días antes que llegara el Virrey nos preuinieron. [\textbf{Five days before} the viceroy’s arrival they warned us.]'
		(\cite[501]{Carochi1645},  glosses added)
	\end{xlist}
\end{exe}	

Similar to the \ili{Manda} case discussed above, the two Classical Nahuatl constructions are fully transparent in their meaning. That they build on a decrease function also becomes evident from the fact that the \textsc{already}\is{already} expression \textit{ye} is used in a parallel fashion, namely to signal the time span that has elapsed since a given situation (i.e. an increase of time); see (\ref{exTimeSpanBeforeAlready}). What is more, \lq still a certain amount of time\rq{ }is attested with other sample expressions as well. See, for instance, Tunisian Arabic \textit{māzāl} in (\ref{exScalarIntroTunisianDays}) above.\il{Arabic, Tunisian}

\begin{exe}
	\ex\label{exTimeSpanBeforeAlready}Classical Nahuatl\\
	\gll  \textup{[}\textbf{Ye} \textbf{onxi}-\textbf{huitl}\textup{]} \textup{[}in àcān ni-quīça\textup{]}.\\
	\phantom{[}already two-year \phantom{[}\textsc{det} nowhere \textsc{subj}.1\textsc{sg}-go\_out\\
\glt \lq Dos años ha que no salgo a ninguna parte. [\textbf{It's been two years} that I haven't gone out at all.]' (\cite[503]{Carochi1645},  glosses added)
\end{exe}

What is striking, however, is the apparently high degree of conventionalisation of the Classical Nahuatl constructions as two of the default means of indicating the extension of a time span before some situation. What is more, in this, but in no other use, the \textsc{still} expression \textit{oc} and the \textsc{already}\is{already} item \textit{ye} can modify the same predicate, which yields a double perspective on the temporal separation between two points in time. Example (\ref{exTimeSpanBefore3}) is an illustration. Here, \textit{oc} signals the duration of the span from ancient times forward to utterance time,\is{utterance time} while \textit{ye} projects backwards, indicating the time that has passed since \parencite[1269]{Launey1986}.

\begin{exe}
	\ex Classical Nahuatl \label{exTimeSpanBefore3}\is{already}\\
	\gll \textup{[}\textup{[}In \textbf{oc} \textbf{ye} huècauh\textup{]}, \textup{[}in \textbf{oc} \textbf{ye} nēpa\textup{]}, \textup{[}in \textbf{oc} \textbf{ye} nechca\textup{]}, \textup{[}in oc īm-pan huehuētquê\textup{]} …\textup{]}\\
	\phantom{[}\phantom{[}\textsc{det} still already long\_time \phantom{[}\textsc{det} still already there \phantom{[}\textsc{det} still already over\_there \phantom{[}\textsc{det} still \textsc{poss}.3\textsc{pl}-\textsc{loc} old\_person.\textsc{pl}\\
	\glt\lq \textbf{Long ago in the past}, during the time of the ancients [i.e. when it was still a long time from there (to now) and already a long time (ago)]…' \parencite[369]{LauneyMackay2011}
\end{exe}\il{Nahuatl, Classical|)}

Lastly, note that Modern Hebrew\il{Hebrew, Modern} \textit{ʕod} and Northern Qiang\il{Qiang, Northern} \mbox{\textit{tɕe}-} feature in collocational patterns \lq still a bit, still a moment\rq{ }that signal proximity to some development (\lq almost, at the point of\rq{}). While these constructions bear some similarity to the ones discussed up to this point, they appear to involve an additive build-up of intervals \lq a little more and …\rq{ }rather than a decrease function. I therefore discuss them separately in \Cref{sectionNearAttainment}.

\subsubsection{Discussion: Scalar decreases}\largerpage
As I laid out initially, there has been an extensive debate as to whether phasal polarity expressions make a different semantic contribution in scalar contexts than in \lq\lq{}regular\rq\rq{ }polar contexts. In what follows, I address this question in regard to \textsc{still} expressions in decrease contexts. Building on the arguments brought forward by \citeauthor{vanderAuwera1991BeyondDuality} (\citeyear{vanderAuwera1991BeyondDuality}, \citeyear{vanderAuwera1993}), \textcite{Garrido1992} and \textcite[ch. 3.1]{MosegaardHansen2008}, I reason that there is no principled need for stipulating a distinct scalar function of the items in question, at least not in terms of core meaning.

To begin with, the basic fact that \textsc{still} expressions are used in decrease contexts is far from surprising, given the nature of scales. Thus, any item on a given scale entails the values corresponding to all lower ranks and is, by the same token, logically compatible with any and all higher ranks (see \Cref{sectionQuantificationScales}). For instance, in (\ref{exScalarDecreaseEnglishPounds}), repeated below, carrying any greater amount of money entails having a few pounds in the pockets, so that there is no logical contradiction in depicting the latter as a persistent\is{persistence} state of affairs.

\begin{exe}[(14a)]
	\exr{exScalarDecreaseEnglishPounds}\ili{English}\\
	\textit{We \textbf{still} \textbf{had} \textbf{a} \textbf{few} \textbf{pounds} in our pockets, so we very quickly decided, why not!} (Cassidy, \textit{Indifferently}) 
\end{exe}

At the same time, it is well known that,  by generalised conversational implicature, the use of a scalar predicate conveys that the value under discussion is the highest one that can be truthfully applied. Importantly, the concept of \textsc{still}, as defined in \Cref{secFunctionalDiscussion}, not only involves persistence,\is{persistence} but also a \isi{prospective} component in the form of a conceivable \isi{discontinuation} scenario. In the absence of a \isi{restrictive} operator or contextual enrichment to the same effect (to be discussed below), this \isi{negation} of the text proposition entails the falsity of all propositions containing higher values. Putting the two pieces together, an example like (\ref{exScalarDecreaseEnglishPounds}) suggests that the development, if any, can only be a downward movement. In other words, the decrease reading can easily be arrived at through the interplay of phasal polarity, scalar semantics and straightforward pragmatic calculus. 

\il{German|(}\is{focus|(}
What remains to be accounted for, however, are the peculiarities of German \textit{noch} addressed initially. To briefly recapitulate, this expression is only felicitous in decrease contexts, at least in its unaugmented form. Limited increases, on the other hand, are covered by a dedicated item \textit{erst} lit. \lq{}first, erstwhile\rq{}. What is more, like the other German phasal polarity expressions and \isi{restrictive} \textit{erst}, \textit{noch} shows the characteristic syntactic\is{syntax} behaviour of a focus particle when it is used in scalar contexts. In the context of \citeauthor{Loebner1989}'s (\citeyear{Loebner1989}, \citeyear{Loebner1999}) duality hypothesis, the sum of these properties is taken as an indication for a distinct semantics. Thus, \textit{noch} in an example like (\ref{exScalarIntroGermanWenigeMeter}), repeated below, is analysed as associating with a lower constituent, the NP \textit{wenige Meter}, and as vesting its denotation with a pre-specified temporal dynamic in the form of a decrease (\approx{}\lq it was the now reduced distance of a few meters…\rq). This argument could be extended verbatim to Serbian-Bosnian-Croatian \il{Serbian}\il{Croatian}\il{Bosnian}\textit{još}, given the parallels in both syntax and distribution discussed above.

\begin{exe}
\exr{exScalarIntroGermanWenigeMeter}German\\
\gll \textup{[}\textbf{Noch} \textup{[}\textbf{wenig}-\textbf{e} \textbf{Meter}\textup{]\textsubscript{}]\textsubscript{\textsc{np}}} war-en es bis zu-r Staffel-übergabe…\\
	\phantom{[}still \phantom{[}few-\textsc{nom}.\textsc{pl} meter.\textsc{pl} \textsc{cop}.\textsc{pst}-3\textsc{pl} 3\textsc{sg}.\textsc{n} until to-\textsc{def}.\textsc{dat}.\textsc{sg}.\textsc{f} relay-handover(\textsc{f})\\
	\glt \lq \textbf{A few meters were left} until the passing of the baton…\rq{}\\
	(found online, glosses added)
\end{exe}

However, even if one accepts the (not unreasonable) assumption that the syntactic\is{syntax} behaviour of German \textit{noch} reflects a more narrow semantic operandum, there is still no principled reason to assume a fundamentally different meaning. The same calculus described above can be applied, such that in (\ref{exScalarIntroGermanWenigeMeter}) a proposition along the lines of \lq{}it was a distance of still a few meters …\rq{ }suggests \lq there was a remaining distance of a few meters …\rq{}. Put differently, the semantic import of \textit{noch} itself in (\ref{exScalarIntroGermanWenigeMeter}) can be understood as fundamentally the same it has as a modifier of a depictive predicate in (\ref{exScalarGermanSecondaryPredicate}).\is{syntax}

\begin{exe}
	\ex 
	German\label{exScalarGermanSecondaryPredicate}\\
	\gll \textbf{Noch} \textbf{benommen} erhob sie sich…\\
	still dazed raise.\textsc{pst}.3\textsc{sg} 3\textsc{sg}.\textsc{f} \textsc{refl}\\
	\glt \lq Still dazed, she got up…\rq{ }(personal knowledge)
\end{exe}\is{focus|)}

Lastly, the fact that German \textit{noch} is only used in decrease contexts need not be attributed to this item itself. As pointed out before me by \textcite{Garrido1992}, it finds an equally satisfactory explanation in the existence of an item dedicated to depicting limited increases, whose high degree of entrenchment can be said to pre-emptively block \textit{noch} from increase contexts.\footnote{See \textcite[ch. 5]{Goldberg2019} for a usage-based theory of statistical pre-emption.}  Intra-systemic support for this interpretation comes from the fact that its evaluative forms \textit{immer noch}/\textit{noch immer} lit. \lq{}still always/always still\rq{} can take scope over restrictives\is{restrictive} and then be used in increase contexts, as in (\ref{exScalarImmerNochNur}) above. Presumably, this is because there is no equally evaluative counterpart to \textit{erst} in the same way that \textit{noch} has \textit{immer noch}/\textit{noch immer}. These points carry over verbatim to Serbian\hyp Croatian\hyp Bosnian\il{Serbian}\il{Croatian}\il{Bosnian} \textit{još}; see (\ref{exScalarImmerNochSerbian}) above for an illustration.\largerpage

Further support for the interpretation just outlined comes from German's close relative \ili{Dutch} (not in my sample). Thus, the \ili{Dutch} \textsc{still} expression \textit{nog} is perfectly compatible with a \isi{restrictive} marker in its scope and increase contexts. In all other respects, this item shares the same functional range as its German cognate, which makes it a questionable move to assign different sets of meanings to the two expressions. Crucially, while \ili{Dutch} possesses two dedicated scalar restrictives,\is{restrictive} their use is far less generalised than German \textit{erst}. Thus, in most varieties of Dutch,\il{Dutch} the cognate form \textit{eerst} is restricted to the written medium \parencite[72]{Vandeweghe1992}. The second marker, \textit{pas}, is only beginning to replace the \textit{nog}-plus-restrictive\is{restrictive} collocations, in that it is mostly interchangeable with the latter, but preferred with inherently time-functional variables such as age \parencite[110]{Vandeweghe1992}. Lastly, a case similar to German and  Serbian\hyp Croatian\hyp Bosnian\il{Serbian}\il{Croatian}\il{Bosnian} \textit{još} can be made for Mandarin Chinese,\il{Chinese, Mandarin} where \isi{restrictive} \textit{cái} is used instead of the \textsc{still} expression \textit{hái} in increase contexts. Though \textit{cái} is not inherently scalar in all of its uses (see \cite[ch. 4.1]{Hole2004} for discussion), it is strongly associated with scalar contexts and covers much of the same ground as German \textit{erst} (e.g. \cite{Lai1999}).

In a nutshell, I see no principled reason to assume that the semantics of \textsc{still} expressions in decrease contexts is any different from that of contexts in which a binary opposition is at stake. Instead, its interpretation can be arrived at compositionally, through the interplay of phasal polarity, scalar semantics and pragmatic considerations. The restriction of German \textit{noch}, Serbian\hyp Croatian\hyp Bosnian\il{Serbian}\il{Croatian}\il{Bosnian} \textit{još}, and Mandarin Chinese\il{Chinese, Mandarin} \textit{hái} to decrease context is best understood as a function of the options available in each system and their respective entrenchment, rather than being due to a distinct meaning.\il{German|)}

\subsubsection{A closer look: Scalar increases}
Having discussed \textsc{still} expressions in the context of decreases, I now turn to the opposite, namely limited increases along some scale. The Tunisian Arabic example (\ref{exScalarIntroTunisian}), repeated below, is an illustration.\il{Arabic, Tunisian}

\begin{exe}
	\exr{exScalarIntroTunisian} Tunisian Arabic\il{Arabic, Tunisian}\\
	\gll W-xallā-hū-l-hum w-humma māzāl-u mā-bilġ-ū-š \textbf{māzāl} \textbf{ʕumᵊṛ}-\textbf{hum} \textbf{tisʕ} \textbf{ᵊsnīn}.\\
	and-leave\_behind.\textsc{pfv}.3\textsc{sg}.\textsc{m}-3\textsc{sg}.\textsc{m}-to-3\textsc{pl}.\textsc{m} and-3\textsc{pl}.\textsc{m} still-3\textsc{pl}.\textsc{m} \textsc{neg}-reach\_puberty.\textsc{pfv}-3\textsc{pl}.\textsc{m}-\textsc{neg} still(.3\textsc{sg}.\textsc{m}) age(\textsc{m})-\textsc{poss}.3\textsc{pl}.\textsc{m} nine year.\textsc{pl}\\
	\glt \lq Er hinterließ sie ihnen, aber sie sind noch nicht in der Pubertät, sie sind erst ungefähr neun Jahre alt. [He (father) bequeathed it (a wall) to them (the orphans), but they haven’t reached puberty yet, they are \textbf{still only about nine years old}.]\rq{ }(\cite[220–221]{RittBenmimoun2011}, glosses by \cite{FischerEtAlTunisian})
\end{exe}

In (\ref{exScalarIntroTunisian}) it is the \textsc{still} expression \textit{māzāl} alone that modifies the scalar predicate \textit{tisʕ} \textit{ᵊsnīn} \lq nine years\rq{}. A similar case can be seen in the Ternate-Tidore example (\ref{exScalarIncreaseTernate}).\il{Ternate}\il{Tidore} This is in contrast to \ili{French} \textit{encore}, which requires the predicate to be modified by \textit{ne … que} \lq no more than\rq{} in such instances, as illustrated in (\ref{exScalarIncreaseFrench}).

\begin{exe}
	\ex Ternate-Tidore\il{Ternate}\il{Tidore} \label{exScalarIncreaseTernate}\\
	 Context: An angelic being and mother of several children wants to return to the heavens. On her first attempt of leaving, the youngest child cried and she returned to comfort it. Now she is attempting to leave for the second time.\\
	\gll \textbf{Konora} \textbf{ine} \textbf{moju} ngofa kage reke, yali mina uci tora yali.\\
	middle upwards still child be\_shocked cry again 3\textsc{sg}.\textsc{f} descend downwards again\\
	\glt \lq Half way up again (lit. [when she was] \textbf{still} [\textbf{only}] \textbf{half way up}), the child was frightened [and] cried, she came down again.' (\cite[375]{vanStaden2000})
		\ex \ili{French}\label{exScalarIncreaseFrench}\\
	\gll Un an après sa promulgation, la loi pour la croissance \textbf{n}-\textbf{'a} \textbf{encore} \textbf{que} \textbf{des} \textbf{effets} \textbf{limités}.\\
	\textsc{indef}.\textsc{sg}.\textsc{m} year(\textsc{m}) after \textsc{poss}.3\textsc{sg}.\textsc{f}:\textsc{sg} promulgation(\textsc{f}) \textsc{def}.\textsc{sg}.\textsc{f} law(\textsc{f}) for \textsc{def}.\textsc{sg}.\textsc{f} growth(\textsc{f}) \textsc{neg}-have.3\textsc{sg} still than \textsc{indef}.\textsc{pl} effect(\textsc{m}).\textsc{pl} limited.\textsc{pl}.\textsc{m}\\
	\glt \lq A year after its promulgation the law for [economic] growth \textbf{still shows only limited effects}.\rq{ }(found online, glosses added)%\footnote{\url{https://www.lesechos.fr/2016/08/la-loi-macron-un-symbole-plus-quune-revolution-234679} (02 March, 2023).}
\end{exe}

\is{restrictive|(}The requirement for an overt restrictive marker constitutes a parameter of significant variation across expressions and, by extension, across languages. Out of a total of 17 expressions (21 if including problematic cases) that are attested in increase contexts, this prerequisite appears to be found with only four items, namely \ili{English} \textit{still}, \ili{French} \textit{encore}, as well as \ili{German} \textit{noch} and Serbian\hyp Croatian\hyp Bosnian\il{Serbian}\il{Croatian}\il{Bosnian} \textit{još} (in their emphatic variants \textit{immer noch}/\textit{noch immer} and \textit{još uvek}\sim{}\textit{još} \textit{uvijek}).\footnote{Outside of my sample, \textcite{vanderAuwera1993} and \textcite{Vandeweghe1992} report \lq still only\rq{ }collocations for \ili{Hungarian} (\textit{még csak}) and \ili{Dutch} (\textit{nog maar}, \textit{nog slechts}).} With the limitations of my sample data in mind, this suggests that the need to overtly specify \lq still only\rq{ }is far from the norm in the world's languages. What is more, even with those expressions that normally require such an addition, exceptions may be found. Thus, as \citeauthor{Ippolito2004} (\citeyear{Ippolito2004}, \citeyear{Ippolito2007}) points out, some speakers of \ili{English} allow for bare \textit{still} in the context of a limited progress of time; see (\ref{exScalarIncreaseStillEnglishTime}). Which contexts prefer an overt restrictive marker, or even require one, in the case of each expression is a question to be addressed in future research.

\begin{exe}
	\ex\ili{English}\label{exScalarIncreaseStillEnglishTime}
	\begin{xlist}
		\exi{A:}\textit{Eat? \textbf{It's} \textbf{still} \textbf{10} \textbf{am}!}
		\exi{B:} \textit{So what? I deserve a good lunch!}
		\exi{A:} \textit{Lunch is for noon.}
		\exi{B:} \textit{Whatever. }(online example, cited in \cite[2 fn2]{Ippolito2007})
	\end{xlist}
\end{exe}

Somewhat comparable to what I discussed above for decrease contexts and additional \lq{}remain\rq{ }expressions, items that do not require an overt restrictive marker may nonetheless be combined with such an operator to make the direction of change more salient. For instance,\il{Spanish|(} Spanish \textit{todavía} follows the \ili{French} pattern in (\ref{exScalarIncreaseSpanishBooks}), but can also be used on its own in the relevant contexts, as in (\ref{exScalarIncreaseTrinidad}) below.

\begin{exe}
	\ex Spanish\label{exScalarIncreaseSpanishBooks}\\
	\gll Pedro \textbf{todavía} \textbf{tiene} \textbf{solo} cien libro-s.\\
	P. still have.3\textsc{sg} only hundred book-\textsc{pl}\\
	\glt \lq Pedro \textbf{still only has} a hundred books.' (\cite[383]{Garrido1992},  glosses added)
\end{exe}\il{Spanish|)}

\il{Mateq|(}Against this backdrop, two expressions require a bit of discussion. The first is Mateq \textit{bayu}, illustrated in (\ref{exScalarIncreaseMateq}). The inclusion of this item in my sample is tentative. As I lay out in more detail in \Cref{sectionThusFarOnly}, \textit{bayu} appears to lexicalise phasal polarity \textsc{still} together with a restrictive operator, a combination from which its use in scalar increase contexts falls out naturally.

\begin{exe}
	\ex Mateq\label{exScalarIncreaseMateq}\\
	\gll \textbf{Bayu} \textbf{aroq}=\textbf{nq} panèi nyidoq.\\
	still beginning=\textsc{poss}.3 clever speak\\
	\glt \lq He'd only just begun to speak well (lit. it is \textbf{still only his beginning} …).' \parencite[138]{Connell2013}	
\end{exe}

\il{Ruuli|(}\is{perfective|(}
The second item worth discussing is Ruuli \mbox{\textit{kya}-}. The one clear-cut example of an increase with non-exhaustive alternatives in the sample data is (\ref{exScalarIncreaseRuuli}), which features the perfective \is{aspect|(} inflection. In Ruuli, as in two other Bantu languages in my sample, the \textsc{still}-plus-perfective frame itself has developed a \lq thus far only\rq{ }use (\Cref{sectionThusFarOnly}). Roughly comparable to the case of Mateq \textit{bayu}, this function is compatible with an ordered set of alternatives, but does not require one. That is to say, (\ref{exScalarIncreaseRuuli}) could be subsumed under the \lq thus far only\rq{ }use of \mbox{\textit{kya}-}.\il{Mateq|)}

\begin{exe}
	\ex Ruuli\label{exScalarIncreaseRuuli}\\
		\gll Eirai budi ni-ba-kya-li ku-leeta bi-ni eby-a nasare, aka-ana \textbf{ka}-\textbf{kya}-\textbf{zw}-\textbf{ire}=\textbf{mbe} \textbf{oku} \textbf{ma}-\textbf{beere} nti ka-ab-e ka-tandika oku-soma.\\
	in\_the\_past previously when-\textsc{subj}.\textsc{ncl}2-still-\textsc{cop} \textsc{ncl}15(\textsc{inf})-bring \textsc{ncl}8-\textsc{prox} \textsc{ncl}9-\textsc{assoc} nursery \textsc{ncl}12-child \textsc{subj}.\textsc{ncl}12-still-abandon-\textsc{pfv}=\textsc{foc} \textsc{ncl}17(\textsc{loc}) \textsc{ncl}6-breast \textsc{comp} \textsc{subj}.\textsc{ncl}12-go-\textsc{sbjv} \textsc{subj}.\textsc{ncl}12-start \textsc{ncl}15(\textsc{inf})-read\\
	\glt \lq Long ago, before they brought these nurseries, [when] a child \textbf{has only stopped breastfeeding}, it would go and start studying.'
	\\(\cite{RuuliCorpus},  glosses added)
\end{exe}
\il{Ruuli|)}\is{aspect|)} \is{perfective|)}

\is{anterior|(}Example (\ref{exScalarIncreaseRuuli}) brings me to the last point. In contexts of a potential increase one sometimes encounters attestations with an anterior viewpoint. Example (\ref{exScalarIncreaseEnglishCat}) is another illustration.

\begin{exe}
	\ex \ili{English}\label{exScalarIncreaseEnglishCat}\\
	\textit{Harry \textbf{has} \textbf{still} \textbf{only} \textbf{fed} the cat \textup{(}he has not performed greater kindness\textup{)}.} \parencite[203]{Michaelis1993}
\end{exe}

Importantly, all comparable instances that I am aware of feature some sort of restrictive operator. The latter may be overt, as in (\ref{exScalarIncreaseEnglishCat}), or be built into the expression (or its collocation with a specific inflectional paradigm) itself, as in the \ili{Mateq} and Bantu cases. \is{aspect|(}Whatever its status, the restrictive operator can be said to conciliate the anterior aspect with the concept of \textsc{still}. If one assumes that the anterior viewpoint denotes a state, then having fed the cat is invariable (e.g. \cite[234]{Parsons1990}) and therefore incompatible with \textsc{still} (\Cref{secFunctionalDiscussion}), but having \textit{only fed} it can be subject to change \parencite{Michaelis1993}.\is{aspect|)}\is{anterior|)}

\subsubsection{Discussion: Scalar increases}
I now turn to the semantics and pragmatics underlying the use of \textsc{still} expressions in increase contexts. It is opportune to start this discussion by taking another look at those cases involving an additional restrictive marker, as in (\ref{exScalarIncreaseFrench2}), which is a condensed version of (\ref{exScalarIncreaseFrench}) above. Here, the predicate \lq have limited effects\rq{ }lies in the scope of restrictive \textit{ne} … \textit{que} \lq no more than\rq{}.\il{French}

\begin{exe}
	\ex \ili{French}\label{exScalarIncreaseFrench2}\\
	\gll La loi \textbf{n}-\textbf{'a} \textbf{encore} \textbf{que} des effets limités.\\
	\textsc{def}.\textsc{sg}.\textsc{f} law(\textsc{f}) \textsc{neg}-have.3\textsc{sg} still than \textsc{indef}.\textsc{pl} effect(\textsc{m}).\textsc{pl} limited.\textsc{pl}.\textsc{m}\\
	\glt \lq The law \textbf{still} \textbf{shows} \textbf{only} limited effects.\rq{}
\end{exe}

\is{focus|(}
The semantics at play in instances like (\ref{exScalarIncreaseFrench2}) are, of course, fully compositional. By means of the restrictive operator the truth of all propositions containing values higher than the focus denotation are negated.\is{negation} In (\ref{exScalarIncreaseFrench2}) such alternatives include, for instance \lq shows noteworthy effects\rq{ }or \lq changes everything\rq{}.\il{French} Crucially, in the evoked \isi{discontinuation} scenario, the negative component of the restrictive operator falls under the scope of negation itself. The two negations\is{negation} cancelling each other out, this yields the requirement for a truthfully applicable higher value (\cite{Garrido1992}; \cite[112]{Vandeweghe1992}). In other words, any possible development must involve an upward movement on the scale.\il{Spanish|(} The same logic can be applied to instances like (\ref{exScalarIntroTunisian}, \ref{exScalarIncreaseTernate}) above or (\ref{exScalarIncreaseTrinidad}), which do not feature an overt restrictive marker. Here, the question under discussion, such as the cumulative impact of the Covid19 pandemic in (\ref{exScalarIncreaseTrinidad}), only allows for a maximal reading of the respective scalar value.\footnote{\textcite[385 fn36]{Garrido1992} speaks of a \lq\lq cognitive frame … that includes the increase direction of the process.\rq\rq{}} That is to say, these cases can be understood as involving a contextually enriched semantic representation.

\begin{exe}
	\ex Spanish\label{exScalarIncreaseTrinidad}\\
	\textit{En las áreas francófonas, anglófonas y territorios ligados a los Paises Bajos del Caribe es Trinidad y Tobago uno de los países más afectados,}\\
	\lq Within the francophone, anglophone and Dutch territories of the Carribean, Trinidad and Tobago is one of the countries hit the hardest [by the Covid19 pandemic].\rq{}
	\exi{}\gll pero con \textbf{todavía} \textbf{unos} \textbf{modest}-\textbf{o}-\textbf{s} \textbf{51} \textbf{caso}-\textbf{s} y sin muerte-s.\\
	 but with still \textsc{indef}.\textsc{pl}.\textsc{m} modest-\textsc{m}-\textsc{pl} 51 case(\textsc{m})-\textsc{pl} and without death-\textsc{pl}\\
	\glt \lq but with \textbf{still only some modest 51 cases} and no casualties.\rq{}\\(found online, glosses added)%\footnote{\url{https://es-us.noticias.yahoo.com/estrictas-medidas-puerto-rico-covid-182536565.html} (23 January, 2023).}
\end{exe}

Support for this interpretation comes from the fact that the more straightforward decrease reading seems to be the default interpretation of the relevant items, unless there are contextual indications to the contrary (see \cite{Garrido1992} on Spanish).\il{Spanish|)} Whether or not a language allows for such contextual enrichment  appears to be an arbitrary, and perhaps gradual, choice. Crucially, however, there is again no need for assuming a fundamentally distinct semantics of the \textsc{still} expressions themselves.\is{restrictive|)}

\subsubsection{Discussion: A brief conclusion}\il{German|(} 
Given the extent of the preceding exposition and discussion, a short summary is called for. Based on the sample data, I have argued that scalar contexts do not involve a distinct use, at least in semantic terms. Instead, they can be explained through a combination of phasal polarity, scalar semantics and pragmatic considerations. In some cases, intra-systemic constellations play an additional role in the interpretation and combinatory possibilities. From a purely structural point of view, it could, however, be argued that the cases of German \textit{noch} and Serbian\hyp Croatian\hyp Bosnian\il{Serbian}\il{Croatian}\il{Bosnian} \textit{još} are distinct from the garden-variety of phasal polarity. In addition, my discussion has illustrated several usage patterns that build on phasal polarity plus scalar variables, and I have highlighted parameters of variation across expressions and languages, such as the requirement (or absence thereof) of an additional \isi{restrictive} \lq only\rq{ }marker in contexts of a limited increase.\is{scale|)}

\subsection{\lq{}Thus far always/everyone\rq}\label{sectionNochImmer}
\subsubsection{Introduction}
One item in my sample, German \textit{noch}, forms part of an idiomatic and schematic construction that signals the \isi{persistence} of a generic rule \lq thus far always\rq{} (\appref{appendixGermanNochJeder}). As far as its formal make-up is concerned, this construction consists of \textit{noch} as a focus particle, the associated constituent of which contains a universal quantifier, and whose host clause features either the periphrastic 
\isi{anterior} or the synthetic past tense.\is{tense} Example (\ref{exGermanNochImmer1}) is an illustration.\footnote{This construction also features famously in the Cologne dialect saying \textit{Et hätt noch emmer joot jejange} \lq Thus far it has always worked out (i.e. don't worry, everything will be just fine).\rq{}}

\begin{exe}
	\ex German\label{exGermanNochImmer1}\\
	\gll Wie \textbf{jed}-\textbf{es} \textbf{Jahr} \textbf{noch} war die Zahl der Fremd-en allmählich bis Ostern ge-wachs-en.\\
	as every-\textsc{nom}.\textsc{sg}.\textsc{n} year(\textsc{n}) still \textsc{cop}.\textsc{pst}.3\textsc{sg} \textsc{def}.\textsc{nom}.\textsc{sg}.\textsc{f} number(\textsc{f}) \textsc{def}.\textsc{gen}.\textsc{pl} stranger-\textsc{gen}.\textsc{pl} slowly until Easter \textsc{ptcp}-grow-\textsc{ptcp}\\
	\glt \lq As \textbf{every year thus far}, the number of visitors had been steadily increasing towards Easter.\rq{ }(von Ompleda, \textit{Margret und Ossana},  glosses added)
\end{exe}

\subsubsection{A closer look} 
As example (\ref{exGermanNochImmer2}) shows, the universal quantifier may fall under the scope of an \lq almost\rq{ }type of expression. 

\begin{exe}
	\ex German\label{exGermanNochImmer2}\\
	\gll Ich war \textbf{noch} \textbf{beinahe} \textbf{jed}-\textbf{es} \textbf{Jahr} an der Fachtagung und ich hab-e immer gross-e Vor-freude.\\
	1\textsc{sg} \textsc{cop}.\textsc{pst}.1\textsc{sg} still nearly every-\textsc{nom}.\textsc{sg}.\textsc{n} year(\textsc{n}) at \textsc{def}.\textsc{dat}.\textsc{sg}.\textsc{f} symposium(\textsc{f}) and 1\textsc{sg} have-1\textsc{sg} always big-\textsc{acc}.\textsc{sg}.\textsc{f} pre-joy(\textsc{f})\\
	\glt \lq I have participated in the congress \textbf{nearly every year so far} and I always look forward to it.\rq{ }(found online, glosses added)%\footnote{\url{https://www.furrerevents.ch/fachtagung-muetter-und-vaeterberatung} (24 February, 2023).}
\end{exe}

Lastly, unlike what (\ref{exGermanNochImmer1}, \ref{exGermanNochImmer2}) may suggest, the focus need not have a temporal meaning. In (\ref{exGermanNochImmer3}) it is a masculine form of \textit{jeder} as generic \lq everyone\rq{}.
	
\begin{exe}
	\ex German\label{exGermanNochImmer3}\\
	\gll Sterb-en kann nicht so schwer sein. Das hat \textbf{noch} \textbf{jed}-\textbf{er} ge-schaff-t.\\
	die-\textsc{inf}(\textsc{n}) can.3\textsc{sg} \textsc{neg} so difficult \textsc{cop}.\textsc{inf} 3\textsc{sg}.\textsc{acc}.\textsc{n} have.3\textsc{sg} still every-\textsc{nom}.\textsc{sg}.\textsc{m} \textsc{ptcp}-achieve-\textsc{ptcp}\\
	\glt \lq Dying can’t be that difficult. \textbf{So} \textbf{far} \textbf{everyone} has achieved it.\rq{}
	\\(found online, glosses added)%\footnote{\url{https://www.zeit.de/kultur/2018-03/sterben-thomas-macho-philosoph-gesellschaft-tod} (22 February, 2023, glosses added)}	
\end{exe}

\subsubsection{Discussion} 
Perhaps the most straightforward explanation for the collocation in (\ref{exGermanNochImmer1}–\ref{exGermanNochImmer3}) is that it involves the coercion of an experiential state into a transient property of the subject. This semantic shift, in turn, is aided by focus on the universal quantifier as a signal that the common ground contains less encompassing alternatives. In other words, while the fact that some situation has occurred regularly is normally an invariable state-of-affairs and therefore incompatible with \textsc{still} (\Cref{secFunctionalDiscussion}), it is here portrayed as a state that might come to an end. In more applied terms, it is conceivable that the construction draws on analogy from two sources. The first are \lq so far, never ever\rq{ }statements, as in (\ref{exGermanNochImmer4}), where \textit{noch} features as a compositional part for the expression of \textsc{not yet}.\is{not yet} These are semantically unproblematic and find cross-linguistic parallels, for instance in the\il{Spanish|(} Spanish example (\ref{exGermanNochImmerSpanish}).

\begin{exe}
	\ex German\label{exGermanNochImmer4}\is{not yet}\\
	\gll Das hat \textbf{noch} \textbf{niemand} ge-schaff-t.\\
	3\textsc{sg}.\textsc{acc}.\textsc{n} have.3\textsc{sg} still nobody \textsc{ptcp}-achieve-\textsc{ptcp}\\
	\glt \lq \textbf{As of yet}, \textbf{nobody} has ever achieved that.\rq{ }(personal knowledge)
	\ex Spanish\label{exGermanNochImmerSpanish}\\
	\gll Gan-ar … donde \textbf{todavía} \textbf{no} lo ha conseg-uido \textbf{nadie}\\
	win-\textsc{inf} {} where still \textsc{neg} 3\textsc{sg}.\textsc{acc}.\textsc{m} have.3\textsc{sg} achieve-\textsc{ptcp} nobody\\
	\glt \lq To win … where \textbf{nobody has ever yet} …\rq{}
	\\(found online, glosses added)%\footnote{\url{https://www.marca.com/2010/10/16/futbol/1adivision/1287246090.html} (23 February, 2023,  glosses added).}
\end{exe}\il{Spanish|)}

\is{scale|(}
A second source for analogy involves the \textsc{already}\is{already} expression \textit{schon}. This item forms part of a fixed phrase \textit{schon immer} \lq since ever\rq{}, lit. \lq{}already always\rq{ }(\cite[203–204]{KoenigEtAl1993}; \cite[469]{MetrichFaucher2009}); see (\ref{exGermanSchonImmer}). This complex adverbial phrase is most likely derived from \textit{schon} in the context of a scalar increase, more specifically, an increase in the duration of a time-span, as in (\ref{exGermanSchon2h}).

\begin{exe}
	\ex 
	\begin{xlist}
		\exi{} German	
		\ex \label{exGermanSchonImmer}
		\gll Das ist \textbf{schon} \textbf{immer} so \textup{(}gewesen\textup{)}.\\
		3\textsc{sg}.\textsc{n} \textsc{cop}.3\textsc{sg} already always thus \phantom{(}\textsc{cop}.\textsc{ptcp}\\
		\glt \lq It's been like this \textbf{since ever}.\rq{}
		
		\ex \label{exGermanSchon2h}\is{already}
		\gll Ich wart-e \textbf{schon} \textup{(}\textbf{seit}\textup{)} \textbf{zwei} \textbf{Stund}-\textbf{en}.\\
		1\textsc{sg} wait-1\textsc{sg} already \phantom{(}since two hour-\textsc{pl}\\
		\glt \lq I've been waiting \textbf{for} \textbf{two} \textbf{hours} (\textbf{already}).\rq{ }(personal knowledge)
	\end{xlist}
\end{exe}\is{scale|)}\il{German|)}\is{focus|)}

\subsection{Continued iteration/restitution}\is{repetition|(}
\label{sectionContinuedIteration}
\subsubsection{Introduction}
In this subsection, I briefly turn to the use of \textsc{still} expressions to signal the continuation of a series of repetitions (\lq yet again').\il{Martuthunira|(} Example (\ref{exContinuedIterationIntro}) is an illustration. Here, the Martuthunira adverb \textit{ngartil} marks the recurrence of the people's chasing the boomerang, and the \textsc{still} expression \textit{waruul} highlights the \isi{persistence} of a series of repetions.

\begin{exe}
	\ex Martuthunira\label{exContinuedIterationIntro}\\
	Context: A man has thrown a boomerang several times, and each time a group of people has tried to catch it.\\
	\gll Ngartil thawu-lalha. \textbf{Ngartil} \textbf{waruul}-purtu ngunhu-ngara yanga-lwala.\\
	again send-\textsc{pst} again still-\textsc{comp} that.\textsc{nom}-\textsc{pl} chase-\textsc{purp}:\textsc{ds}\\
	\glt \lq Again he sent it, and \textbf{yet again} they chased it.' \parencite[294–295]{Dench1994}
\end{exe}

Note that instances like (\ref{exContinuedIterationIntro})\il{Martuthunira|)} differ from the iterative and restitutive uses which I discuss in \Cref{sectionIterative}, in that in the latter instances it is the \textsc{still} expression itself that contributes the repetitive event construal.

\subsubsection{Distribution in the sample and a closer look}
\begin{table}
\caption{Continued iteration/restitution. †: Inclusion is tentative.\label{tableContinuedIteration}}
	\begin{tabular}{llll}
	\lsptoprule
	Macro-Area & Language & Expression & Appendix\\
	\midrule
	Africa & Plateau Malagasy\il{Malagasy, Plateau} & \textit{mbola}\textsuperscript{†} & \ref{appendixMalagasyContinuedIteration}\\
	Australia & \ili{Gooniyandi} & \mbox{=\textit{nyali}}\textsuperscript{†} & \ref{appendixGooniyandiIterative}\\
	& \ili{Martuthunira} & \textit{waruu}(\textit{l}) & \ref{appendixMartuthuniraContinuedIteration}\\
	Papunesia & \ili{Paiwan} & =\textit{anan} & \ref{appendixPaiwanIterative}\\
	South America & \ili{Trió} & =\textit{nkërë} & \ref{appendixTrioContinuedIteration}\\
	\lspbottomrule
	\end{tabular}
\end{table}

\Cref{tableContinuedIteration} lists the five expressions in my sample that have the \lq yet again\rq{ }use. This overview is, however, rather tentative. On the one hand, there might be more relevant cases. Thus, in \Cref{sectionIterative} I point out that \lq again\rq{ }uses of \textsc{still} expressions are often accompanied by another item indicating repetition, and it is possible that some of those collocations involve the notion of continued iteration/restitution. On the other hand, with two expressions in \Cref{tableContinuedIteration}, their inclusion is far from settled.\il{Malagasy, Plateau|(} The first of these is Plateau Malagasy \textit{mbola}, which
\begin{quote}
	can accompany a predicate modified by the ambimodal marker \textit{indray}, a marker of iteration. Its presence introduces an additional nuance of meaning that is difficult to render in French. \parencite[537]{Dez1980}\footnote{In the original French, \lq\lq{}peut accompagner un prédicat déterminé par le marqueur ambimodal \textit{indray}, marqueur du renouvellement de l'action. Sa présence introduit une nuance de sens complémentaire malaisée à rendre en français.\rq\rq{}}
\end{quote}

I found only a single translated example of \textit{mbola} plus \textit{indray} in the literature, which is given in (\ref{exContinuedIterationMalagasy}). If I understand \citeauthor{Dez1980}'s (\citeyear{Dez1980}) parenthetical remarks correctly, \textit{mbola} here adds the notion of \lq once more\rq{ }to the iterative meaning contributed by \textit{indray}, i.e. it is raining for at least the third time in a row.

\begin{exe}
\ex Plateau Malagasy\label{exContinuedIterationMalagasy}\\
\gll \textup{(}\textbf{Mbola}\textup{)} avy \textbf{indray} ny orana.\\
	\phantom{(}still come again \textsc{det} rain\\
	\glt \lq Il pleut encore une fois, la pluie tombe encore (une nouvelle fois).' [It's raining \textbf{once} \textbf{again}, it is raining \textbf{again} (\textbf{once more}).] 	\\(\cite[537]{Dez1980},  glosses added)
\end{exe}\il{Malagasy, Plateau|)}

\il{Gooniyandi|(}The second unclear case involves Gooniyandi \mbox{=\textit{nyali}}. This expression is attested with \textit{ngambiddi} \lq again' as its host, but the function of this collocation is not always clear. In some cases, it is related to \mbox{=\textit{nyali}} in another function, namely as \lq always\rq{}, with the two items combining to signal \lq\lq daily iterative occurrences of processes such as shaving" \parencite[464]{McGregor1990}. In other cases, the contribution of this collocation is much harder to determine. For instance, what is at stake in the narrative excerpt in (\ref{exContinuedIterationGooniyandi1}) is the third occurrence of people catching a crocodile. This suggests that \mbox{=\textit{nyali}} here serves to indicate that the iteration of successful hunts (signalled by \textit{ngambiddi}) did not come to its conceivable end. However, based on the limited data, it is hard to make any definite conclusions. \textcite[244]{SchultzeBerndt2002} reports personal communication from William McGregor that \textit{ngambiddi}=\mbox{\textit{nyali}} is idiomatic and might be undergoing lexicalisation.
 
\begin{exe}
\ex \label{exContinuedIterationGooniyandi1}
Context: Hunters have pulled in one crocodile and killed it, then pulled in a second one and also killed it.\\
	\gll Yaanya / \textbf{ngambiddi}=\textbf{nyali} ridd-widi / gard-biddini\\
	other {} again=still pull-3\textsc{pl}:catch {} hit-3\textsc{pl}:hit\\
	\glt \lq Then they pulled \textbf{yet another} in and killed it.' \parencite[575]{McGregor1990}
\end{exe}\il{Gooniyandi|)}

\subsubsection{Discussion}
On a conceptual level, the combination of \textsc{still} and repetition is straightforward: the notions of \isi{persistence} and a conceivable \isi{discontinuation} here apply not to an individual situation, but to an overarching series of recurrences. At the same time, the continued iteration/restitution use is far from universally attested in my sample and it is clearly unavailable with expressions such as \ili{Swahili} \textit{bado};  see (\ref{exContinuedIterationSwahili}). Due to this arbitrary nature, it is best understood as a function of its own, distinct from phasal polarity \textit{sensu stricto}.

\begin{exe}
	\ex[]{\ili{Swahili}\label{exContinuedIterationSwahili}}
	\exi{}[\#]{\gll A-li-i-fanya \textbf{bado} \textbf{tena}.\\
	\textsc{subj}.\textsc{ncl}1-\textsc{pst}-\textsc{obj}.\textsc{ncl}9-do still again\\	
	\glt (intended: \lq He did it yet again.\rq{ }) (Ponsiano Kanijo, p.c.)}
\end{exe}
\is{repetition|)}

\subsection{Sameness}\label{sectionSame}\is{identity|(}
\il{Creek|(}\il{Udihe|(}
\subsubsection{Introduction} In this subsection, I briefly address uses pertaining to a notion of identity, or \lq the same\rq{}. Such functions have been described for two sample expressions, Creek \mbox{(\textit{i})\textit{monk}} and Udihe \textit{xai}(\textit{si}) (\appsref{appendixCreekSame},  \ref{appendixUdiheSame}). Example (\ref{exSameCreek1}) illustrates \mbox{(\textit{i})\textit{monk}}. Here, this item signals that the dances remain unchanged across times, and despite the Muskogee Nation's involuntary relocation.

\begin{exe}
	\ex Creek \label{exSameCreek1}\\
Context: According to customs they would have fasts and \\
	 \gll opánka oːc-ít  foll-atí-n oːm-âːt \textbf{imóŋk}-\textbf{os}-\textbf{in} foll-atíː-s.\\
dance have-\textsc{ss} go\_about.\textsc{pl}-\textsc{happen}-\textsc{ds} \textsc{cop}-\textsc{ref} still.\textsc{nmlz}-\textsc{dim}-\textsc{non}.\textsc{subj} go\_about.\textsc{pl}-\textsc{dist.pst}-\textsc{ind}\\
	\glt \lq the \textbf{same} dances they used to have (lit. if they happened to have a dance, they would do it in the same way [as before]).\rq{}
	\\(\cite[57–58]{HaasHill2014},  glosses added)
\end{exe}

\subsubsection{A closer look and discussion} 
In (\ref{exSameCreek1}), as in all other relevant attestations of \mbox{(\textit{i})\textit{monk}}, the notion of identity pertains to some quality across times. Admittedly, \mbox{(\textit{i})\textit{monk}} is a borderline case of a \textsc{still} expression and might be understood as a marker of stasis or permanence (see \appref{appendixCreekStill}). One way or another, there is an obvious close relationship between notions pertaining to \isi{persistence} and the use I discuss here. A likely link between the two functions can be found in an attestation like (\ref{exSameCreekTurtle}), where the temporal continuation\is{persistence} of a state is paired with zero anaphora. 

\begin{exe}
	\ex Creek\\
	 Context: Turtleʼs wife has splattered blood in his eyes.\label{exSameCreekTurtle}\\
	\gll Itóɬwa 	ak-caˑt-ak-átiˑ-t ôˑm-it hayyôˑm-eys \textbf{imóŋka}-\textbf{t} \textbf{ôn}-\textbf{t} \textbf{ôˑm}-\textbf{iˑ}-\textbf{s}.\\
3.eye \textsc{loc}-red-\textsc{pl}-\textsc{rem}.\textsc{pst}-\textsc{ss} \textsc{cop}-\textsc{ss} be\_now.\textsc{res}-even still.\textsc{nmlz}-\textsc{subj} \textsc{cop}-\textsc{ss} \textsc{cop}-\textsc{dur}-\textsc{ind}\\
	\glt \lq His eyes turned red, and \textbf{they are the same way even now} (lit. … and even now \textbf{it still is}).\rq{ }(\cite[442]{HaasHill2014}, glosses added)
\end{exe}

The case of Udihe differs from the Creek one in two respects.\il{Creek|)} Firstly, it involves a collocational pattern consisting of the \textsc{still} expression \mbox{\textit{xai}(\textit{si})} plus an item from a set of anaphoric pronouns, which are characterised by initial \mbox{(\textit{u})\textit{t}-}. Secondly, this construction is attested not only as an exponent of \isi{persistence} over time, such as in (\ref{exUdiheSameHunting}), but also in the context of identity with abstract referents, as in (\ref{exSameHappened}).

\begin{exe}
	\ex 
	\begin{xlist}
		\exi{}Udihe
		\ex 	\label{exUdiheSameHunting}
	 \gll Bi abuga-i ei zulie-ni xuliː bede \textbf{xaisi} \textbf{ute} \textbf{bede} xuliː-ni ba:-za ge-tigi-ni.\\
1\textsc{sg} father-1\textsc{sg} this before-3\textsc{sg} go:\textsc{prs}.\textsc{ptcp}:\textsc{ss} like still that like go-3\textsc{sg} place-\textsc{n} surface-\textsc{lat}-3\textsc{sg}\\
	\glt \lq My father still goes hunting \textbf{in the same way} as he used to.\rq{}
	\\\parencite[398–399]{NikolaevaTolskaya2001}
	
		\ex Context: A man's wife has run away to escape an evil spirit. Now he has met who appears to be another woman, and she has told him her story of escaping from an evil spirit.\label{exSameHappened}\\
		\gll Merge g(une)-ini \lq\lq{}Bi mamasa-i \textbf{xai} \textbf{ute} bi-s’e.\rq\rq\\
	hero say-3\textsc{sg} \phantom{\lq\lq}1\textsc{sg} wife-1\textsc{sg} still that \textsc{cop}-\textsc{pfv}\\ 
		\glt \lq The man said, \lq\lq{}\textbf{The same} happened to my wife.{\rq\rq }\rq{}
	\\\parencite[76–77]{NikolaevaEtAl2003}
	\end{xlist}
\end{exe}

Similar to the \ili{Creek} case, however, many of the Udihe attestations allow for a transparently compositional reading. For instance, the identity of manner in (\ref{exUdiheSameHunting}) most likely goes back to the notion of \isi{persistence} provided by \mbox{\textit{xai}(\textit{si})}, together with \textit{ute} anaphorically pointing to the past way of hunting \lq his prior way of going, he still goes like that\rq{}. In other cases, a reading of identity can be linked to a combination of anaphora plus additive and/or iterative uses of \mbox{\textit{xai}(\textit{si})}. Thus, example (\ref{exSameHappened}) features the recurrence of an aforementioned situation \lq that also happened …\rq{}. I assume that cases like (\ref{exSameUdiheDay}) can be subsumed here, as well (\lq also on that day\rq{}). Whether mediated by persistence,\is{persistence} additivity\is{additive} or iteration,\is{repetition} a common denominator lies in the accumulation of sub-parts of one overarching state-of-affairs.

\begin{exe}
	\ex Udihe \label{exSameUdiheDay}\\
	\gll Bi-de \textbf{xai} \textbf{utelinie} \textbf{neŋi}-\textbf{ni} ŋeneː-mi.\\
	1\textsc{sg}-\textsc{foc} still then day-3\textsc{sg} go.\textsc{pst}-1\textsc{sg}\\
	\glt \lq I left on \textbf{the very same day}.\rq{ }\parencite[440–441]{NikolaevaTolskaya2001}
\end{exe}\il{Udihe|)}
\is{identity|)}

\section{Uses relating to other phasal polarity concepts}
\label{sectionOtherPhPConcepts}
In this section, I discuss uses that pertain to, or infringe on, other phasal polarity concepts. More specifically, in \Cref{sectionNotYet} I lay out how \textsc{still} expressions may serve to signal \textsc{not yet}\is{not yet} without an overt negator\is{negation} in certain environments. In \Cref{sectionAlready} I take a look at the phenomenon of \isi{interrogative} \lq yet\rq{ }as a link between \textsc{still} and the second affirmative concept, \textsc{already}.\is{already}
\subsection{\lq{}Still\rq{ }as \lq{}not yet\rq}\label{sectionNotYet}\is{negation|(}\is{not yet|(}
\subsubsection{Introduction}
Example (\ref{exNotYetIntro}) illustrates how the \ili{Swahili} \textsc{still} expression \textit{bado} may serve to signal the negative concept \textsc{not yet} in a response to a polar question without any negator.\footnote{This subsection is a condensed version of \textcite{PersohnNotYet}.} Cases like this thus constitute a manifestation of \lq\lq negatives without negators" \parencite{Miestamo2010}.

\begin{exe}
	\ex \ili{Swahili}\label{exNotYetIntro}\\
	\gll A-me-kuja? – \textbf{Bado}.\\
	\textsc{subj}.\textsc{ncl}1-\textsc{ant}-come {} still\\
	\glt \lq Ist er gekommen? -- Noch nicht. [Has he come? -- Not yet.]\rq{} (\cite[15]{Schadeberg1990}, glosses added)
\end{exe}

While examples like (\ref{exNotYetIntro}) may be striking from a European perspective, one should keep in mind that \textsc{still} and \textsc{not yet} are, 
aside from polarity, \lq\lq exactly the same … retrospectively continuative and prospectively\is{prospective} geared towards possible change" \parencite[40]{vanderAuwera1998}.

\subsubsection{Distribution and types}
\Cref{tableNotYetContexts} lists the 22 sample languages and expressions which have a use as markers of \textsc{not yet} without negation. As can be gathered, this use is found in all six macro-areas. On a more fine-grained level, the sample data indicate four types of environment in which this use is found. These are:

\begin{enumerate}[label=(\roman*)]
	\item In the absence of an overt predicate.
	\item With less-than-finite or dependent predicates.\is{infinitive}
	\item With specific actional types.\is{actionality}
	\item In a specific position relative to the predicate.\label{typeNotYet4}\is{syntax}
\end{enumerate}

Of these contexts, \textsc{not yet} in the absence of an overt predicate is by far the most common, as it is attested in twenty languages and all six macro-areas. The rarest type appears to be the one in which polarity is a function of the position that the expression occupies within the clause;\is{syntax} in my sample, this type is attested only in the \ili{Blagar} language of eastern Indonesia. Lastly, \Cref{tableNotYetContexts} also indicates whether each expression participates in the signalling of \textsc{not yet} together with an overt negator, which is a point I return to below. In what follows, I discuss each of the four types of \textsc{still} as \textsc{not yet} separately.

\vfill
\begin{table}[H]
 \captionsetup{justification=centering}	
	\caption{\lq still\rq{ }as \lq not yet\rq}\label{tableNotYetContexts}
		\footnotesize
		\begin{tabular}{llllccccc}
		\lsptoprule
		&		 &	&	&   &   \multicolumn{4}{c}{Without \textsc{neg}}\\\cmidrule(lr){6-9}
		Macro-area&Language&Expression & Appendix & \textsc{neg} & \rot{Empty predicate} & \rot{Less than finite}  & \rot{Actional class} & \rot{Position}\\
		\midrule
		Africa & Adamawa Fulfulde\il{Fulfulde, Adamawa}  & \textit{tawon} & \ref{appendixAdamawaNotYet} & y & y & n & n & n\\
		&			\ili{Bende} & \textit{syá}-& \ref{appendixBendeNotYet} & n &  y & y & n & n \\
		& \ili{Manda} & (\textit{a})\textit{kona} & \ref{appendixMandaNotYet} & \phantom{\textsuperscript{a}}(y)\footnote{Marginal or described as a rare occurrence; see discussion below.} & y & y & n & n\\
		& \ili{Ruuli} & \textit{kya}- & \ref{appendixRuuliNotYet} & n & y & y& n & n\\
		&\ili{Nyangbo} & \textit{ka}- & \ref{appendixNyangboNotYet} & n & n & n & y & n\\
		& \ili{Sango} & \textit{de} & \ref{appendixSangoNotYet} & y & y & n & n & n\\
		& \ili{Swahili} & \textit{bado} & \ref{appendixSwahiliNotYet} & y & y & y & n & n\\
		& \ili{Tashelhyit} & \textit{sul} & \ref{appendixTashelhyitNotYet} & y & y & n & n & n\\
		& \ili{Tima} & \textit{bʌʌr} & \ref{appendixTimaNotYet} & y & y & n & n & n\\
		& Tunisian Arabic\il{Arabic, Tunisian} & \textit{māzāl} & \ref{appendixTunisianNotYet} & y & y & n & n & n\\
			Australia & \ili{Wardaman} & \textit{gayawun} & \ref{appendixWardamanNotYet} & y &y & n & n & n\\
		Eurasia & \ili{Japhug}& \textit{pɤjkʰu}\footnote{Only in combination with the attenuating \textsc{sfp} \textit{je}.} & \ref{appendixJaphugNotYet} & y & y& n& n & n\\
		& \ili{Thai} & \textit{yaŋ} & \ref{appendixThaiNotYet} & y & y & n & n & n\\
		& Tundra Nenets\il{Nenets, Tundra} & \textit{təmna} & \ref{appendixTundraNenetsNotYet} & y & y & n & n & n \\
		North America & \ili{Kalaallisut} & \textit{suli} & \ref{appendixGreenlandicNotYet} & y & y & n &n &n\\
		Papunesia & \ili{Blagar} & (\textit{y})\textit{edung} & \ref{appendixBlagarNotYet} & (n) & y &n &n & y\\
		& \ili{Bukiyip} & \textit{wotak} & \ref{appendixBukiyipNotYet} & y &y & n & n & n\\
		& \ili{Iatmul} & \textit{wata} & \ref{appendixIatmulNotYet} & y & y & n &n & n \\
		& \ili{Kalamang} & \textit{tok} & \ref{appendixKalamangNotYet} & y & y & n & n & n\\		
		& \ili{Lewotobi} & \textit{morә̃} & \ref{appendixLewotobiNotYet} & y & n & n &y & n\\
		& Western Dani\il{Dani, Western} & \textit{awo} & \ref{appendixWesternDaniNotYet} & y & y & n & n & n\\
		South America & \ili{Xavánte} & (\textit{za})\textit{hadu} & \ref{appendixXavanteNotYet}  & y & y & n & n & n\\
		\lspbottomrule
		\end{tabular}
\end{table}
\vfill
\pagebreak

\subsubsection{A closer look and discussion: Absence of an overt predicate} 
Beginning with the most common type of \textsc{still} as \textsc{not yet},\footnote{This type is also  amply attested outside of my sample. Examples on the African continent include cognates of Tunisian Arabic\il{Arabic, Tunisian} \textit{māzāl} across the Maghreb\il{Arabic, Maghrebian} (\cite[196]{Caubet1993}; \cite[430]{Pereira2010}; \cite[197–198]{Tapiero1978}), \textit{ʕad} in the Berber variety \ili{Tarafit} \parencite{Fanego2021}, and \textit{tùkùna} in the Chadic language \ili{Hausa} \parencite[211]{Newman2007}. Instances from Eurasia include \textit{lɪssa} and \textit{baʕd} in Levantine Arabic\il{Arabic, Levantine} \parencite[189–192]{MacNeillHoyt2010} and the \ili{Lao} cognate of \ili{Thai} \textit{yaŋ}, \textit{ñang} \parencite[207–208]{Enfield2007}. In Australia, the pattern is attested, \textit{inter alia}, for \ili{Pitjantjatjara} (Pama-Nyungan) \textit{kuwaripa} \parencite[186–187]{EckertHudson1988}.} namely those instances characterised by the absence of an overt predicate, several recurrent environments stand out.\footnote{As far as I am aware of, the earliest observation that these environments are unified by the lack of an overt predicate is found in \citeauthor{MarcaisGuiga19581961}'s (\citeyear{MarcaisGuiga19581961}) glossary of the Takrouna variety of Tunisian Arabic.\il{Arabic, Tunisian} In a discussion of the expression \textit{māzāl} they observe that \lq\lq dans le sens de \lq pas encore\rq{ }le complexe ne peut être suivi ni d'un attribut, ni d'une proposition en asyndète\rq\rq{ }[With the meaning of \lq not yet\rq, the complex cannot be followed by an attribute nor by a proposition in asyndesis] \parencite[1740]{MarcaisGuiga19581961}.} The most common ones are negative answers to polar questions, such as in the \ili{Swahili} example (\ref{exNotYetIntro}) above, or in the \ili{Xavánte} one in (\ref{exNotYetXavante}).

\begin{exe}
	\ex \ili{Xavánte}\label{exNotYetXavante}\\
	\gll E ma tô a-sa? – \textbf{Hadu}.\\
	\textsc{q} \textsc{ant} \textsc{rl} 2-eat {} still\\
	\glt \lq As-tu mangé? -- (Pas) encore. [Have you eaten? -- (\textbf{Not}) \textbf{yet}.]\rq{ }
	\parencite[107]{Estevam2011}
\end{exe}

In all cases that I am aware of, the question being answered pertains to whether a certain situation has been attained. In other words, the interrogative opens a choice between affirmative \textsc{already}\is{already} (which may be implicit) and negative \textsc{not yet}. The only meaningful interpretation of a stand-alone \textsc{still} expression here is in relation to a negative proposition, as \textsc{not yet} is not only equivalent to the inner negation of \textsc{still} (i.e. \mbox{\textsc{still} \neg\textit{p}}), but also constitutes the wide-scope negation of \textsc{already}\is{already} (i.e. \mbox{\neg\textsc{already} \textit{p}}); see e.g. \textcite{Loebner1989}. What is more, all relevant sample expressions also serve as signals of \textsc{not yet} in more complete clauses. This allows for an interpretation of the stand-alone items as elliptical versions of such more saturated clause patterns.

Closely related to the negative replies in (\ref{exNotYetIntro}, \ref{exNotYetXavante}) are polar questions that follow a disjunctive pattern \lq (already) …  or still\rq{ }in the meaning of \lq (already) … or not yet\rq{}. An example is given in (\ref{exNotYetQuestion}). Again, the only possible interpretation here is one in which the \textsc{still} expression takes scope over an implied negator. What is more, in four out of the six languages in question (Bukiyip,\il{Bukiyip} Kalamang,\il{Kalamang} \ili{Thai} and Tunisian Arabic),\il{Arabic, Tunisian} polar questions generally tend to follow a \lq \textit{p} or \textsc{neg}\rq{ }pattern, and the constructions in question can be understood to instantiate this pattern.

\begin{exe}
		\ex  Tunisian Arabic\il{Arabic, Tunisian}\label{exNotYetQuestion}\\
		\gll Flān ža \textbf{walla} \textbf{māzāl}?\\
		so\_and\_so come.\textsc{pfv}.3\textsc{sg}.\textsc{m} or still\\
		\glt \lq Ist er schon da? [Has he come \textbf{yet} (lit. has he come \textbf{or still})?]'
		\\(\cite[650]{Singer1984},  glosses added)
\end{exe}

Disjunction is also at play in cases like (\ref{exNotYetContrastive2}).\il{Kalamang|(} A variation on this theme is found in (\ref{exNotYetContrastiveVariation}), where the situation in question and its attainment for some of the addressees, but not for others, can be retrieved from context.

\begin{exe}
		\ex \ili{Swahili}\label{exNotYetContrastive2}\\
		\gll Kwa sasa ma-ji ya-me-toka m-to-ni (Malulumo) na ku-fika Mgera \textbf{lakini} \textbf{vi}-\textbf{jiji} \textbf{v}-\textbf{ingine} \textbf{bado}.\\
		for now \textsc{ncl}6-water \textsc{subj}.\textsc{ncl}6-\textsc{ant}-leave \textsc{ncl3}-river-\textsc{loc} \phantom{(}M. and \textsc{ncl}15(\textsc{inf})-arrive M. but \textsc{ncl}8-village \textsc{ncl}8-other still\\
		\glt \lq As for now, the water has come from the river (Malulumo) and reached Mgera, but \textbf{no} \textbf{other villages yet}.’ (Helsinki Corpus of Swahili 2.0)

\ex\label{exNotYetContrastiveVariation}
	Kalamang\\	
	Context: People are taking turns at repairing a roof.\\
	\gll Ma he gerket \lq\lq{}naman=a \textbf{tok}?” An \textbf{tok$\sim$tok} an=a ming yuonin=in.\\
	3\textsc{sg} already ask \phantom{\lq\lq}who=\textsc{foc} still 1\textsc{sg} still$\sim$\textsc{redupl} 1\textsc{sg}=\textsc{foc} oil rub=\textsc{neg}\\
	\glt \lq He asked \lq\lq Who \textbf{hasn’t ye}t [taken their turn]?" I \textbf{hadn’t yet}, I didn’t rub oil.' (\cite{Visser2021b},  glosses added)
\end{exe}\il{Kalamang|)}

\ili{Sango} \textit{de}, Tunisian Arabic \textit{māzāl},\il{Arabic, Tunisian} and Western Dani\il{Dani, Western}  \textit{awo} also have a negation-less \textsc{not yet} use in a clause pattern consisting of a nominal subject plus \textsc{still} expression. In the case of Tunisian Arabic\il{Arabic, Tunisian} \textit{māzāl}, this is restricted to \textit{waqt} \lq time' and other nominals referring to time spans, such as \textit{ṣēf} \lq summer\rq{} or \textit{taʕlīm} \textit{fažᵊr} \lq{}first light of dawn\rq{}. The few data points for Sango \textit{de} and Western Dani\il{Dani, Western} \textit{awo}, which include (\ref{exNotYetDaniRiver}, \ref{exNotYetSangoCassava}), suggest that the nominal must be associated with a telic\is{telicity} process and/or reaching a certain point in time, such as the rainy season or plantation work. If this interpretation is correct, it links these cases to the \isi{telicity}-driven alternation with \ili{Lewotobi} \textit{morә̃} that I discuss below. Thus, no matter whether the hearer assumes a negated predicate to be elided, or an affirmative one like \lq the river is still swelling\rq{ }or \lq the river is yet to swell\rq{}, the situation's characteristic endpoint has not yet been reached at the time under discussion.

\begin{exe}
	\ex\label{exNotYetDaniRiver}
	Western Dani\il{Dani, Western}\\
	\gll Yi \textbf{awo}.\\
	river still\\
	\glt \lq The river has not yet been swelled (by the rains).' \parencite[440]{Barclay2008}
	
	\ex\label{exNotYetSangoCassava}
	\ili{Sango}\\
	\gll Yaka ti gozo	 a-\textbf{de}.\\
	plantation	of cassava \textsc{subj}-still\\
	\glt \lq The [work on the] cassava plantation is not yet done / not yet finished.' (original translation: \lq… will/must be continued/finished')
	\\\parencite[115]{NassensteinPasch2021}
\end{exe}

Lastly, \textsc{still} as \textsc{not yet} without a predicate is recurrently found in exclamations,\is{exclamation|(} where it often goes together with pragmatic nuances like \lq wait', \lq hold on', and the like; see the \ili{Wardaman} example (\ref{exNotYetWardaman}). Similar cases involve the notions of not being ready,\footnote{The same type of nuances is found with full-fledged \textsc{not yet} expressions, e.g.  \ili{Nungon} (Nuclear Trans New Guinea) \textit{awe} (cf. \cite[308, 449]{Sarvasy2017}) or \textit{waluu} in the Pama-Nyungan language \ili{Yuwaalaraay} \parencite[259]{Glacon2014}.} as in the Tundra Nenets\il{Nenets, Tundra} example (\ref{exNotYetTundraNenets}). Note that this use of Tundra Nenets \textit{təmna} is slightly irregular: this marker is usually invariable, and such exclamations constitute one of only two contexts in which \textit{təmna} can carry person-number indexes and \isi{tense} marking. As in the patterns discussed before, it is likely the underspecification in polarity brought about by elliptical contexts that allows for negative readings to arise and become conventionalised.

\begin{exe}
	\ex \ili{Wardaman}\label{exNotYetWardaman}
	\begin{xlist}
		\exi{A:}\gll Ngayin.gun-yonga-rri.\\
		3\textsc{non}.\textsc{sg}>1\textsc{du}.\textsc{incl}-farewell-\textsc{pst}\\
		\glt \lq They've said good-bye to the two of us.'
		\exi{B:}\gll \textbf{Gayawun} ngawun-yongi-we!\\
		still 1\textsc{sg}>3\textsc{non}.\textsc{sg}-farewell-\textsc{fut}\\
		\glt \lq \textbf{Hang on}! I have to say goodbye to them!' \parencite[323]{Merlan1994}
	\end{xlist}
	\is{exclamation|)}
	\ex Tundra Nenets\il{Nenets, Tundra}\label{exNotYetTundraNenets}\\
	\gll\textbf{Təmna}-dᵒm-c'ᵒ\\
	still-1\textsc{sg}-\textsc{pst}\\
	\glt \lq I wasnʼt ready yet.ʼ \parencite[187]{Nikolaeva2014}
\end{exe}

Unsurprisingly, there appears to be a correlation between the \textsc{still}-as-\textsc{not yet} use in the absence of a predicate and the wordhood of the expressions involved. Nearly all cases involve independent grammatical words or bound roots. The only cases of affixes in my sample are \ili{Bende} \mbox{\textit{syá}-} and \ili{Ruuli} \mbox{\textit{kya}-}. In these instances, the lexical slot is filled by a copula verb, as illustrated for \ili{Ruuli} \mbox{\textit{kya}-} in (\ref{exNotYetRuuli}).

\begin{exe}
	\ex \ili{Ruuli}\label{exNotYetRuuli}
	\begin{xlist}
		\exi{A:} \gll Ati ga-bba-nga-ku obu-janjabi?\\
		now \textsc{subj}.\textsc{ncl}6-\textsc{cop}-\textsc{hab}-\textsc{ncl}17(\textsc{loc}) \textsc{ncl}14-treatment\\
		\glt \lq Did the eyes have treatment?'
		\exi{B:}\gll Ati ba-n-dongos-ere li-nu, \textbf{li}-\textbf{ni} \textbf{li}-\textbf{kya}-\textbf{li}.\\
		now \textsc{subj}.\textsc{ncl}2-\textsc{obj}.1\textsc{sg}-operate-\textsc{pfv} \textsc{ncl}5-\textsc{prox} \textsc{ncl}5-\textsc{prox} \textsc{subj}.\textsc{ncl}5-still-\textsc{cop}\\
		\glt \lq They operated me this one, \textbf{this one}, \textbf{not yet}.\rq{}
		\\\parencite{RuuliCorpus}
	\end{xlist}
\end{exe}

\Textcite{vanBaar1997}, the first author to discuss some of the relevant instances from a cross-linguistic angle, postulates the two-part implicational universal in (\ref{exVanBaarUniversalNotYet}).

\begin{exe}
	\ex \label{exVanBaarUniversalNotYet}
	\begin{xlist}
		\exi{}\citeauthor{vanBaar1997}'s (\citeyear[295]{vanBaar1997}) \textsc{still}-as-\textsc{not yet} universal
		\ex\label{exVanBaarUniversalNotYet1}
		If a \textsc{still} expression is used as an isolated expression, it is invariably used for the expression of \textsc{not yet}.
		\ex\label{exVanBaarUniversalNotYet2}
		Examples are only found in those languages which have an internal negation of the relevant \textsc{still} expression.
	\end{xlist}
\end{exe}

As can be gathered; \citeauthor{vanBaar1997} only concerns himself with \textsc{still} expressions as pro-sentences. According to (\ref{exVanBaarUniversalNotYet1}), these invariably yield \textsc{not yet}. However, counterexamples are readily found in the sample. For instance, \ili{English} \textit{still} and Modern Hebrew\il{Hebrew, Modern} \textit{ʕadayin} have conventionalised functions as \isi{concessive} interjections\is{interjection} (\Cref{sectionConcessiveInterjections}).\footnote{To \citeauthor{vanBaar1997}'s credit, possibly he only had phasal uses in mind.}  As for the second part of the universal, (\ref{exVanBaarUniversalNotYet2}) predicts that isolated \textsc{still} as \textsc{not yet} should only be found with markers that are also involved in signalling \textsc{not yet} by taking scope over a negator. This is also contradicted by several cases in my sample. In the two Bantu languages \ili{Bende} and \ili{Ruuli} \textsc{not yet} is expressed via \textsc{still} plus infinitive,\is{infinitive} not negation. With \ili{Blagar} \mbox{(\textit{y})\textit{edung}}, the combination with negation is marginal, at best, and it is the position\is{syntax} relative to the predicate that determines the item's interpretation as \textsc{still} or \textsc{not yet} in full clauses. Pending further cross-linguistic work, (\ref{exVanBaarUniversalNotYet2}) may best be understood as a tendency rather than an absolute universal. The strongest absolute implicational universal supported by my sample data is the one in (\ref{exNotYetMyUniversal}). 

\begin{exe}
	\ex In \textsc{still} as \textsc{not yet} expressions, the absence of an overt predicate (encompassing \lq\lq isolated\rq\rq{ }uses) is only found with expressions  that~– in one way or another~-- also participate in the formation of \textsc{not yet} with overt predicates.\label{exNotYetMyUniversal}
\end{exe}

\subsubsection{A closer look and discussion: Less than finite predicates}\is{infinitive|(}
The second type of \textsc{still} as \textsc{not yet}, the collocation with a less-than-finite complement, is illustrated in (\ref{exNotYetManda}) for \ili{Manda} \mbox{(\textit{a})\textit{kona}}.

\begin{exe}
	\ex\ili{Manda}\label{exNotYetManda}\\
	\gll N-\textbf{ákóna} \textbf{ku}-líma ngʼʊ́nda w-ángu.\\
	\textsc{subj}.1\textsc{sg}-still \textsc{ncl}15(\textsc{inf})-cultivate \textsc{ncl}3.plot \textsc{ncl}3-\textsc{poss}.1\textsc{sg}\\
	\glt \lq I have \textbf{not} cultivated my plot \textbf{yet}.' \parencite[45]{Bernander2021}
\end{exe}

Other sample expressions that occur in such collocations are \ili{Swahili} \textit{bado}, \ili{Bende} \mbox{\textit{sya}-}, and \ili{Ruuli} \mbox{\textit{kya}-}; being prefixes, the latter two require a copula verb in these instances. More generally speaking, \textsc{still} plus infinitive is a recurrent \textsc{not yet} device in Bantu (\cite[148]{Nurse2008}; \cite{VeselinovaDevos2021}). As pointed out before me by \citeauthor{Gueldemann1996} (\citeyear{Gueldemann1996}, \citeyear[129–130]{Gueldemann1998}), \textcite[148]{Nurse2008} and \textcite{VeselinovaDevos2021}, the conventionalisation of this function involves only a very short conceptual leap: the \isi{persistence} of a negative situation is a straightforward inference from a persistently unactualised one like \lq still to cultivate my plot\rq{}. The same explanation probably applies to variations on this theme outside of my sample. Thus, in another Bantu language, Kagulu,\il{Kagulu} the item \textit{ng'hali}\sim\textit{ngh'ati} signals \textsc{not yet} with infinitival complements, as well as with verbs in the subjunctive \isi{mood} inflection \parencite[146]{Petzell2008}. Across Bantu, this less-than-finite verbal paradigm comes close to what \textcite[326]{Timberlake2007} terms \lq\lq an all-purpose mood used to express a range of less-than-completely real modality\is{modality}".\footnote{See \textcite[83–84]{RoseEtAl2002} for an overview.} Similarly, \textcite{deAnguloFreeland1930} report that \textit{nám} in the Palaihnihan language \ili{Achumawi} signals \textsc{not yet} when it is combined with the subordinate\is{subordination} form of the verb. These additional data points suggest that \textsc{still} as \textsc{not yet} with less than finite and/or dependent predicates may be more widespread than what the sample data suggest.\is{infinitive|)}

\subsubsection{A closer look and discussion: Actional characteristics}\is{actionality|(}\is{telicity|(}\is{aspect|(} 
With two items in my sample, polarity is a function of the predicate's actional characteristics (the situation's internal make up; see \Cref{sectionTenseAspect}).\il{Lewotobi|(} The first is Lewotobi \textit{morә̃}. This expression signals \textsc{still} with atelic predicates, as shown in (\ref{exNotYetLewotobi1}). With telic predicates, on the other hand, \textit{morә̃} marks \textsc{not yet}; see (\ref{exNotYetLewotobi2}). Predicates that allow for both construals are ambiguous without further context; this is illustrated in (\ref{exNotYetLewotobi3}).

\begin{exe}
	\ex
	\begin{xlist}
	\exi{}Lewotobi
	\ex\label{exNotYetLewotobi1}
	 \gll Go mahasiswa \textbf{morә̃} di.\\
	 1\textsc{sg} student still \textsc{dm}:excuse\\
	 \glt \lq I am \textbf{still} a student (so I am not married).'

	\ex\label{exNotYetLewotobi2}
	\gll Go kriә̃ waha \textbf{morә̃}.\\
	1\textsc{sg} work finish still\\
	\glt \lq I \textbf{haven’t} finished working \textbf{yet}.'

	\ex\label{exNotYetLewotobi3}
	\gll Go kә̃ \textbf{morә̃}.\\
	1\textsc{sg} eat.1\textsc{sg} still\\
	\glt \lq I am \textbf{still} eating / I \textbf{haven’t} eaten [a meal] \textbf{yet}.\rq
	\\\parencite[415–416, 434]{Nagaya2012}
	\end{xlist}
\end{exe}

The two functions of \textit{morә̃} can be conciliated by assuming that this expression emphasises the \isi{prospective} component of \textsc{still}. With telic predicates, this change corresponds to their defining point of culmination.

In diachronic terms, there are several reasons to assume that \textit{morә̃} started out as signalling \textsc{still}. First, \textit{morә̃} also marks \textsc{not yet} in conjunction with negated predicates, which can be resolved compositionally as \textsc{still} \neg{}\textit{p}. Secondly,  its apparent cognate \mbox{\textit{muri}'} in closely related \ili{Lamaholot} is an additive and iterative marker (\cite{NishiyamaKelen2007}; \cite{Pampus2010}), two functions that are notionally close to the concept of \textsc{still}.  What is more, a diachronic shift from \textsc{still} to \textsc{not yet} in the context of telic predicates is highly motivated from a semantic point of view. This requires some elaboration. Based on \citeauthor{Vendler1957}'s (\citeyear{Vendler1957}) famous classification, one can distinguish between two major types of telic predicates. These are the telic processes known as \lq\lq accomplishments\rq\rq{ }and near-instantaneous changes or \lq\lq{}achievements\rq\rq.\is{topic time|(} With accomplishment predicates, a construal of a persistent\is{persistence} process such as \lq s/he's still building a house\rq{ }entails that the point of culmination has not  yet been reached at the time under discussion. The gap to \textsc{not yet} is even narrower with achievements, as these cannot persist\is{persistence} in time. Though this clash could be resolved via a construal in which the \textsc{still} expression takes wide scope over a \isi{prospective} or modal\is{modality} operator (something akin to \lq s/he'll arrive yet\rq{}), this would also entail that the telos has not been attained at topic time.\is{topic time|)} In a nutshell, there is an affinity between telic predicates and \textsc{not yet}. A conceivable catalyst for this development just outlined can be found in the fact that Lewotobi has neither grammatical \isi{tense} nor grammaticalised\is{grammaticalisation}  aspect distinctions,\footnote{Lewotobi does have a marker termed \lq\lq perfective"\is{perfective} by \textcite{Nagaya2012} but the description and available examples suggest that this is rather a iamative, i.e. a specific type of \textsc{already}\is{already} expression (see \cite{Olsson2013}). What is more, it occupies the same position in the clause as \textit{morә̃}.\is{syntax} Both facts suggest that the two items are in complementary distribution.} thereby allowing for the same bare predicate to be construed with an \isi{imperfective} viewpoint or with a perfective\is{perfective}/anterior\is{anterior} one, as in (\ref{exNotYetLewotobi3}).\is{telicity|)}\il{Lewotobi|)}

\il{Nyangbo|(} 
The second sample expression that displays a polarity alternation based on the predicate's actional class is \ili{Nyangbo} \mbox{\textit{ka}-}. With this prefix, the dividing line runs between stative and dynamic predicates. In the default form of the verb, which lacks overt marking of aspect, \mbox{\textit{ka}-} plus stative predicate yields \textsc{still}; see (\ref{exNotYetNyangbo1}). Dynamic predicates, on the other hand, go together with \textsc{not yet} and a \isi{perfective} viewpoint, as in (\ref{exNotYetNyangbo2}). In order for dynamic predicates to be construed as persistent processes,\is{persistence} a \isi{progressive}\is{aspect} marker needs to be added; see (\ref{exNotYetNyangbo3}).

\begin{exe}
	\ex
	\begin{xlist}
		\exi{}\ili{Nyangbo}
		\ex\label{exNotYetNyangbo1}
		Context: Three friends had agreed to meet at the roadside to catch a bus. Two of them met, waited for a bit and then one of them called the third friend on his cell phone. He reports:\\
		\gll A-\textbf{ka}-lɛ́ bɔ-pã́-m.\\
		\textsc{subj}.3\textsc{sg}-still-be\_at \textsc{ncl}-house-inside\\
		\glt \lq He is \textbf{still} at home.’

		\ex\label{exNotYetNyangbo2}
		\gll A-\textbf{ka}-ta̅kɛ̅ sikã́.\\
		\textsc{subj}.3\textsc{sg}-still-pick money.\\
		\glt \lq He \textbf{hasn’t} collected the money \textbf{yet}.'

		\ex\label{exNotYetNyangbo3}
		\gll A-\textbf{ka}-á-ta̅kɛ̅ sikã́.\is{persistence}\\
		\textsc{subj}.3\textsc{sg}-still-\textsc{prog}-pick money.\\
		\glt \lq He is \textbf{still} collecting the money.' 
		\\(\cite[46]{Essegbey2012}; James Essegbey, p.c.)
	\end{xlist}
\end{exe}

Similar to the case of \ili{Lewotobi} \textit{morә̃}, the two readings can be conciliated by assuming that \mbox{\textit{ka}-} places special emphasis on the \isi{prospective} view towards the situation's end. That stative verbs are the odd ones out can be linked to a peculiarity of the language's aspect system. Thus, in Nyangbo, the default form of the verb is what West Africanists term a \lq\lq factative". Factatives have an ongoing state reading with stative verbs,\footnote{Or those labelled as \lq\lq stative"; see \textcite[140]{Ameka2008} for a problematisation of this term with reference to Ewe,\il{Ewe} as well as \textcite{CranePersohn2019} for a discussion of similar facts across Bantu.} but signal a situation that has passed with all other predicates \parencite[346–347]{Welmers1973}. This split in temporal-aspectual interpretation is preserved when \mbox{\textit{ka}-} is added.\il{Nyangbo|)}\is{actionality|)}\is{aspect|)}

\subsubsection{A closer look and discussion: Linear position}\is{syntax|(}\il{Blagar|(} 
The fourth and last flavour of \textsc{still} as \textsc{not yet} is found with \ili{Blagar} \mbox{(\textit{y})\textit{edung}}. This item marks \textsc{still} when it occupies the pre-predicate position of the clause, as in (\ref{exNotYetBlagar1}). When it occurs after the predicate, on the other hand, it serves as an expression of \textsc{not yet}, as shown in (\ref{exNotYetBlagar2}).
\begin{exe}
	\ex\label{exNotYetBlagar}
	\begin{xlist}
		\exi{}Blagar
		\ex\label{exNotYetBlagar1}
		\gll {Qangu veng qangu} na \textbf{jedung} kiki.\\
		at\_that\_time \textsc{subj}.1\textsc{sg} still small\\
		\glt \lq At that time I was \textbf{still} a little child.'
		\\(\cite[195]{SteinhauerGomang2016}, glosses added)

	\ex\label{exNotYetBlagar2}
	\gll N-iva guru \textbf{jeduŋ}.\\ 
	\textsc{poss}.1\textsc{sg}-mother teacher still\\
	\glt \lq My mother \textbf{isn’t} a teacher \textbf{yet}.' \parencite[165]{SteinhauerBlagar}
	\end{xlist}
\end{exe}

To all appearences, this behaviour of \mbox{(\textit{y})\textit{edung}} is shared by the relevant markers in two of the neighbouring and closely related Alor-Pantar languages, \ili{Nedebang} and Teiwa.\il{Teiwa} In all likelihood, this is related to the fact that the post-predicate position is the locus of clausal negation in these languages. In diachronic terms, a comparison with the third neighbour, \ili{Reta} \parencite{Willemsen2020} suggests that this pattern goes back to a pattern of embracing negation [\textsc{neg} \textit{p} \textsc{neg}]. This pattern was analogously applied to the signalling of \textsc{not yet} via [\textsc{neg} \textit{p} \textsc{still}]. The dropping of the initial negator ultimately led to clause-final \mbox{(\textit{y})\textit{edung}} as the sole exponent of \textsc{not yet}.
\il{Blagar|)}\is{negation|)}\is{syntax|)}

\subsection{\lq{}Still\rq{ }and \lq{}already\rq{}, or interrogative \lq yet\rq{}}\label{sectionAlready}\label{sectionInterrogativeYet}\is{interrogative|(}
\subsubsection{Introduction}\largerpage[2]
In this subsection, I discuss the link between \textsc{still} expressions and the second affirmative phasal polarity concept, \textsc{already}.\is{already} More specifically, I focus on a phenomenon known as interrogative \lq{}yet\rq{} (\cite{vanderAuwera1993}, \citeyear[92–98]{vanderAuwera1998}). The latter consists in using the non-negative\is{negation} component of a polymorphemic \textsc{not yet} expression in indirect and/or direct questions. Illustrations are provided in (\ref{exInterrogativeYetEnglish}) for the phenomenon's namesake, \ili{English} \textit{yet}.

\begin{exe}
	\ex  \label{exInterrogativeYetEnglish}
	\begin{xlist}
		\exi{}\ili{English}
		\ex Indirect question\\
	\textit{I doubt that Fred has arrived \textbf{yet}.}
	
		\ex Direct question\\
		\textit{Has Fred arrived \textbf{yet}?}
	\end{xlist}
\end{exe}

While \ili{English} \textit{yet} has ceased to be a bona fide \textsc{still} expression, interrogative \lq yet\rq{} is also attested for cases that continue to serve as such. Example (\ref{exFrenchInterrogative}) is an illustration featuring \ili{French} \textit{encore}.\largerpage[2]

\begin{exe}
	\ex \label{exFrenchInterrogative} \ili{French}\\
	Context: Speaking of a house that is only heated intermittently.\\
	\gll Je me demande s'-il fait \textbf{encore} \textbf{assez} \textbf{chaud} dans la maison.\\
1\textsc{sg} \textsc{refl}.1\textsc{sg} ask.1\textsc{sg} if-3\textsc{sg}.\textsc{m} make.3\textsc{sg} still sufficient heat in \textsc{def}.\textsc{sg}.\textsc{f} house(\textsc{f})\\
	\glt \lq I wonder whether the house is \textbf{sufficiently warm yet}.'
	\\(\cite[178]{Martin1980}, glosses added)
\end{exe}

In (\ref{exInterrogativeYetEnglish},\il{English} \ref{exFrenchInterrogative}), the meaning of \textit{yet} and \textit{encore} approaches that of \lq already\rq{}. \il{German|(}Unsurprisingly, some language make use of \textsc{already}\is{already} expressions in the same contexts, as illustrated in (\ref{exInterrogativeYetGerman}) for the German counterpart to (\ref{exFrenchInterrogative}). What is more, below I discuss how the use of a \textsc{still} expression as interrogative \lq yet\rq{ }may ultimately lead to its reanalysis as an \textsc{already} marker.\is{already}

\begin{exe}
	\ex German\label{exInterrogativeYetGerman}\\
	\gll Ich frag-e mich, ob es \textbf{i}-\textbf{m} \textbf{Haus} \textbf{schon} \textbf{warm} \textbf{genug} \textit{ist}.\\
	1\textsc{sg} ask-1\textsc{sg} \textsc{refl}.1\textsc{sg} whether 3\textsc{sg}.\textsc{n} in-\textsc{def}.\textsc{dat}.\textsc{sg}.\textsc{n} house(\textsc{n}) already warm enough \textsc{cop}.3\textsc{sg}\\
	\glt \lq I wonder whether the house \textbf{is sufficiently warm ye}t.\rq
	\\(personal knowledge)
\end{exe}\il{German|)}

\subsubsection{Distribution in the sample (and beyond)}
\Cref{tableInterrogativeYet} gives an overview of the known attestations of the phenomenon, based on the sample data and previous studies, with the sample languages highlighted in grey.\footnote{For languages outside the sample, see \citeauthor{vanderAuwera1993} (\citeyear{vanderAuwera1993}, \citeyear{vanderAuwera1998}), \textcite[193–194]{vanBaar1997}, \textcite{Vaelikangas1982}, and references therein. \Textcite[193]{vanBaar1997} additionally lists \ili{Hausa} \textit{har yànzu}. This marker should not be considered a phasal polarity expression, but is a transparent combination of temporal deixis and inclusion \lq up to and including now' \parencite{Ziegelmeyer2021}. Note that I did not actively look for interrogative \lq yet\rq{ }with items that do not synchronically serve as a \textsc{still} expression. It is therefore possible that the phenomenon is found in more of my sample languages.} For each marker, \Cref{tableInterrogativeYet} also indicates whether it synchronically functions as a \textsc{still} expression and whether the interrogative \lq yet\rq{ }function is available in indirect and direct questions. As can be gathered, the phenomenon has primarily been reported for languages from Europe, though not exclusively so. In my sample, cases from outside of Europe involve \ili{Thai} \textit{yaŋ} and \ili{Slave} \textit{k'ála}. On a more fine-grained level, it is noteworthy that the use in direct questions entails that in indirect ones, as previously observed for Europe by \textcite{vanderAuwera1998}.\largerpage[2]

\begin{table}[H]
	\caption{Interrogative \lq yet\rq{}. \emph{Notes}: *: \textcite{Haberland1991} reports inter-speaker differences. †: Listed by \textcite{vanderAuwera1998} only with reservation. ‡: With limitations; see discussion below.}
	\label{tableInterrogativeYet}
		\footnotesize
		\begin{tabularx}{\textwidth}{llllccc}
			\lsptoprule
			&			&				&			&	& Indirect & Direct\\
			Macro-area & Language & Expression & Appendix & \textsc{still} & QNs & QNs\\
				\midrule
			
			Africa &\ili{Ewe} & \textit{hade} &  n/a & n & y & y \\
			& \ili{Hausa} & \textit{tùkùn}(\textit{a}) & n/a  & n &? & y \\
			Eurasia & \ili{Breton}  & \textit{c'hoazh} & n/a  & y & y & n \\
			& \ili{Danish} & \textit{endnu} & n/a  & y & y & \phantom{*}(y)*\\
			 & \ili{Dutch} &  \textit{nog} & n/a  & y & y & n\\
				 & \ili{English} & \textit{yet} & n/a  & n & y & y\\
			 & \ili{Faroese} &  \textit{enn} & n/a  & y & y & n \\
			 & \ili{Finnish}& \textit{vielä} & n/a  & y & y & y \\
			 & \cellcolor{lightgray}\ili{French} &\cellcolor{lightgray}\textit{encore} &  \cellcolor{lightgray}\ref{appendixFrenchEncoreInterrogativeYet} &  \cellcolor{lightgray}y & \cellcolor{lightgray}y & \cellcolor{lightgray}n\\
			 & \ili{Icelandic} & \textit{enn} & n/a & y & y & n\\
			 & \ili{Irish} & \textit{fós} &  n/a & y& y & y\\
			 &  &  \textit{go fóill} & n/a  & y & y & y\\
			 & \ili{Karaim} & \textit{hanuz}\textsuperscript{†} & n/a & y & (y) & (y)\\ 
			 &  \cellcolor{lightgray}\ili{Lezgian} &  \cellcolor{lightgray}\textit{hele}  & \cellcolor{lightgray}\ref{appendixLezgianHeleInterrogativeYet} &  \cellcolor{lightgray}\phantom{\textsuperscript{‡}}(y)\textsuperscript{‡} & \cellcolor{lightgray}y &  \cellcolor{lightgray}y\\
			 & \ili{Norwegian} & \textit{ennå} & n/a & y &y & \phantom{*}(y)*\\ 
			 & \ili{Sami} & \textit{vel} & n/a & y & y & y\\
					& \cellcolor{lightgray}\ili{Spanish} & \cellcolor{lightgray}\textit{aún} &  \cellcolor{lightgray}\ref{appendixSpanishAunInterrogativeYet} &\cellcolor{lightgray}y &  \cellcolor{lightgray}(y)&  \cellcolor{lightgray}n\\
			&\cellcolor{lightgray}&\cellcolor{lightgray}\textit{todavía} & \cellcolor{lightgray}\ref{appendixSpanishTodaviaInterrogativeYet} & \cellcolor{lightgray}y &\cellcolor{lightgray}(y) & \cellcolor{lightgray}n\\
			& \ili{Swedish} & \textit{än} & n/a & y   & y & y\\
			&  \cellcolor{lightgray}\ili{Thai} &\cellcolor{lightgray}\textit{yaŋ} &  \cellcolor{lightgray}\ref{appendixThaiNotYet} & \cellcolor{lightgray}y  &  \cellcolor{lightgray}y &  \cellcolor{lightgray}y\\
			& \ili{Welsh} &  \textit{eto} & n/a  &  n & y & y\\
			North America  & \cellcolor{lightgray}\ili{Slave} & \cellcolor{lightgray}\textit{k'ála}  & \cellcolor{lightgray}\ref{appendixSlaveInterrogativeYet} & \cellcolor{lightgray}y  &\cellcolor{lightgray}y &\cellcolor{lightgray}y\\	
			& \ili{Navajo} & \textit{t'ah}(-\textit{dii}) &n/a & y & y & ?\\
			Papunesia & \ili{Abun} & \textit{tó} &n/a & y&y  & y\\
		\lspbottomrule
	\end{tabularx}
\end{table}

In what follows, I first discuss a diachronic pathway that has been proposed by \textcite[92–96]{vanderAuwera1998}. I then turn to the case of \ili{Thai} \textit{yaŋ}, the function of which as interrogative \lq yet\rq{ }is closely interwoven with its use as \textsc{not yet} without a negator\is{negation} (see \Cref{sectionNotYet}).


\subsubsection{A closer look and discussion: The subordination scenario}\is{subordination|(}
As I pointed out initially, interrogative \lq yet\rq{ }is a phenomenon that is primarily related to \textsc{not yet}. However, the vast majority of known instances involve cases where this negative\is{negation} concept is expressed via a \textsc{still} expression taking scope over a negator; this includes \ili{English} \textit{yet}, which used to be a marker of \isi{persistence} before it became superseded in this function by \textit{still} (\cite[146–148]{Koenig1991}; \cite{KoenigTraugott1982}). Against this backdrop \textcite{vanderAuwera1998} observes an implicational universal in his sample of European languages, namely that interrogative \lq yet\rq{ }in direct questions unilaterally entails the same function in indirect questions. Based on this observation, he suggests a chain-wise development that starts in marking \textsc{not yet} via \textsc{still} plus negation,\is{negation} specifically in subordinate contexts. Subsequently, negative raising creates a syntactic\is{syntax} separation between the two components of the complex \textsc{not yet} expression, which leads to the \textsc{still} expression getting associated with non-assertive contexts. This structural\is{syntax} pattern is then carried over to indirect questions that are biased towards the negative. The now narrow gap to direct questions finds an additional bridge in indirect inquiries of the type \lq I would like to know, whether…'. This scenario is schematically illustrated in \Cref{figureSubordinationScenario}.

\il{Spanish|(}In my sample, the two Spanish expressions \textit{aún} and \textit{todavía} have progressed only to the stage of negatively oriented indirect questions.\is{negation} Both items are illustrated in (\ref{exInterrogativeYetSpanish}).

\begin{exe}
	\ex \label{exInterrogativeYetSpanish}
	\begin{xlist}
		\exi{}Spanish
		\ex 
		\gll \textbf{Dud}-\textbf{o} \textbf{que} \textbf{h}-\textbf{aya} \textbf{nac}-\textbf{ido} \textbf{todavía} el que ten-ga los suficiente-s cojon-es para hac-er-lo.\\
	doubt-1\textsc{sg} \textsc{subord} have-\textsc{sbjv}.3\textsc{sg} be\_born-\textsc{ptcp} still \textsc{def}.\textsc{sg}.\textsc{m} \textsc{subord} have-\textsc{sbjv}.3\textsc{sg} \textsc{def}.\textsc{pl}.\textsc{m} enough-\textsc{pl} testicle(\textsc{m})-\textsc{pl} for do-\textsc{inf}-3\textsc{sg}.\textsc{acc}.\textsc{m}\\
	\glt \lq\textbf{I doubt that} the person with big enough balls to do so \textbf{has been born} \textbf{yet}.' (Montaño, \textit{Andanzas}, cited in \cite[§80.8m]{RAEGramatica}, glosses added)
		\ex
		\gll Cré-an-me, no hay político … superior … en este siglo, y \textbf{dud}-\textbf{o} \textbf{que} \textbf{h}-\textbf{aya} \textbf{nac}-\textbf{ido} \textbf{aún}.\\
	believe-\textsc{sbjv}.3\textsc{pl}-1\textsc{sg}.\textsc{obj} \textsc{neg} \textsc{exist} politician {} superior {} in \textsc{prox}.\textsc{sg}.\textsc{m} century(\textsc{m}) and doubt-1\textsc{sg} \textsc{subord} have-\textsc{sbjv}.3\textsc{sg} be\_born-\textsc{ptcp} still\\
	\glt \lq Believe me, there is (currently) no better politician … in this century … and \textbf{I doubt that one has been born yet}.\rq{ }(found online, glosses added)
	\end{xlist}
\end{exe}\il{Spanish|)}

\setlength{\MinimumWidth}{\widthof{\textit{I'd like to know \textbf{whether} Fred has arrived \textbf{yet}}}+2em}
\setlength{\MinimumWidthB}{\widthof{\small{Interrogative \lq yet\rq}}}
\begin{figure}[h]
\caption{\Citeauthor{vanderAuwera1998}'s (\citeyear{vanderAuwera1998}) subordination scenario. Arrows depict feeding relationships.\label{figureSubordinationScenario}}
	\begin{tikzpicture}[node distance = 0pt]
		\node[rounded corners, align=center, draw=black, minimum width=\MinimumWidth] (Stage1) {\strut Subordinate context\\\strut\textit{I think that Fred has\textbf{n't} arrived \textbf{yet}.}};
		
		\node[rounded corners, align=center, draw=black, minimum width=\MinimumWidth, below=1em of Stage1] (Stage2) {\strut Negative raising\\\strut\textit{I \textbf{don't} think that Fred has arrived \textbf{yet}.}};

	\node[rounded corners, align=center, draw=black, fill=lightgray, minimum width=\MinimumWidth, below=1em of Stage2] (Stage3) {\strut Negative indirect question\\\strut\textit{I \textbf{doubt} that Fred has arrived \textbf{yet}.}};		
	
		\node[rounded corners, align=center, draw=black, fill=lightgray,  minimum width=\MinimumWidth, below=1em of Stage3] (Stage4) {\strut Indirect inquiry\\\strut\textit{I'd like to know \textbf{whether} Fred has arrived \textbf{yet}.}};		
		
		\node[rounded corners, align=center, fill=lightgray,  draw=black, minimum width=\MinimumWidth, below=1em of Stage4] (Stage5) {\strut Direct question \\\strut\textit{Has Fred arrived \textbf{yet}?}};				
		
   \draw [-latex, very thick] (Stage1.south) to (Stage2.north);
   \draw [-latex, very thick] (Stage2.south) to (Stage3.north);
   \draw [-latex, very thick] (Stage3.south) to (Stage4.north);
   \draw [-latex, very thick] (Stage4.south) to (Stage5.north);
\end{tikzpicture}
\end{figure}

\il{Lezgian|(}
On the other hand of the spectrum, having taken the development even further than the scenario in \Cref{figureSubordinationScenario}, lies Lezgian \textit{hele}. The comparative evidence on this marker strongly points towards \textsc{still} as its original function (\cite{vanderAuwera1993}, \citeyear[126–127]{vanderAuwera1998}). In addition, \textit{hele} combines with \isi{negation} to signal \textsc{not yet}, as shown in (\ref{exInterrogativeYetLezgian1}). Furthermore, it is attested in negative indirect and direct question contexts; see (\ref{exInterrogativeYetLezgian2}, \ref{exInterrogativeYetLezgian3}).\pagebreak

\begin{exe}
	\ex 
	\begin{xlist}
		\exi{}Lezgian
		\ex \label{exInterrogativeYetLezgian1}
		\gll Am naq' nisini-qʰ wacʼa-l fe-na, \textbf{hele} \textbf{xta}-\textbf{nwa}-\textbf{č}.\\
		3\textsc{sg}.\textsc{abs} yesterday noon-at river-to go-\textsc{aor} still return-\textsc{ant}-\textsc{neg}\\
		\glt \lq He went to the river yesterday at noon, and he \textbf{hasn't returned yet}.\rq
			\ex \label{exInterrogativeYetLezgian2}
		\gll Zun Jusuf Derbentd-aj \textbf{hele} qʰfe-nwa-j-da-l \textbf{šaklu} \textbf{že}-\textbf{zwa}.\\
		1\textsc{sg}.\textsc{abs} J. Derbentl-out\_of still leave-\textsc{ant}-\textsc{ptcp}-\textsc{nmlz}-on doubting \textsc{cop}-\textsc{ipfv}\\
		\glt \lq I \textbf{doubt whether} Jusuf has left Derbent \textbf{yet}.\rq
		\ex \label{exInterrogativeYetLezgian3}
		\gll Jusuf Derbentd-aj \textbf{hele} qʰfe-na-ni?\\
		J. Derbent-out\_of still leave-\textsc{aor}-\textsc{q}\\
		\glt \lq Has Jusuf left Derbent \textbf{yet}?' \parencite[84, 88]{Haspelmath1991}
	\end{xlist}
\end{exe}

What stands out about \textit{hele} is that its original function as a \textsc{still} expression is presently only realised when it reinforces another item, the affix \mbox{-(C)\textit{ma}}, as in (\ref{exInterrogativeYetLezgian4}). Its primary affirmative function in the present-day language pertains to \textsc{already}\is{already}; see (\ref{exInterrogativeYetLezgian5}). Crucially, the contribution of \textit{hele} to marking \mbox{\textsc{not yet}} is compatible with both concepts, in that \mbox{\textsc{not yet} \textit{p}} is tantamount to \mbox{\textsc{still} \neg\textit{p}} and to \mbox{\neg\textsc{already} \textit{p}},\is{already} with the sole difference lying in the scope of the negator.\is{negation} Given this logical equivalence, as well as the fact hat interrogative \lq yet\rq{ }is notionally very close to \textsc{already},\is{already} the most parsimonious explanation is that the latter use has led to the reanalysis of \textit{hele} as an \textsc{already} expression (\cite{vanderAuwera1998}).

\begin{exe}
	\ex 
	\begin{xlist}
	\exi{}Lezgian
		\ex\label{exInterrogativeYetLezgian4}
		\gll Am fadlaj pensijadi-z eq̄ečʼ-na kʼan-zawa-j-di ja, amma am \textbf{hele} \textbf{kʼwalaxa}-\textbf{l} \textbf{xü}-\textbf{zma}.\\
		3\textsc{sg}.\textsc{abs} long pension-\textsc{dat} go\_out-\textsc{aor} must-\textsc{ipfv}-\textsc{ptcp}-\textsc{nmlz} \textsc{cop} but 3\textsc{sg}.\textsc{abs} still work-on keep-\textsc{ipfv}.still\\
		\glt \lq He ought to have retired long ago, but \textbf{they're still keeping him at work}.\rq{ }\parencite[88]{Haspelmath1991}
	\ex\label{exInterrogativeYetLezgian5}\is{already}
	\gll Jusuf Derbentd-aj \textbf{hele} qʰfe-na.\\
	J. D.-out\_of still/already leave-\textsc{aor}\\
	\glt Jusuf has \textbf{already} left Derbent.\rq{ }\parencite[210]{Haspelmath1993}
	\end{xlist}
\end{exe}
\il{Lezgian|)}

\il{Slave|(}Outside of western Eurasia, a development comparable to \citeauthor{vanderAuwera1998}'s (\citeyear{vanderAuwera1998}) scenario appears to have occurred with Slave \textit{kʼála}.  Examples (\ref{exInterrogativeYetSlave2}, \ref{exInterrogativeYetSlave3}) illustrate this item as interrogative \lq yet\rq{ }in direct and indirect questions, respectively.
\begin{exe}
	\ex 
	\begin{xlist}
		\exi{}Slave
		\ex \label{exInterrogativeYetSlave2}
		\gll Sú \textbf{kʼála} shéteni̜?\\
		\textsc{q} still \textsc{subj}.2\textsc{sg}.eat\\
		\glt \lq Did you eat yet?\rq{ }\parencite[346]{Rice1989}

		\ex \label{exInterrogativeYetSlave3}
		\gll Sú \textbf{kʼála} ˀelá ráyéhdí hi̜si̜.\\
		\textsc{q} still boat \textsc{subj}.3:bought uncertainty\\
		\glt \lq I wonder if s/he bought a boat yet.\rq{ }\parencite[421]{Rice1989}
	\end{xlist}
\end{exe}

It is conceivable that the use of \textit{kʼála} as interrogative \lq{}yet\rq{ }was mediated by a surface scope ambiguity in examples like (\ref{exInterrogativeYetSlave4}). Here, this item is linearly separated from the negator\is{negation} \textit{yíle} and could potentially be parsed as either a preposed part of the matrix clause, or as belonging to the complement clause. Alternatively, it might be that the relevant instances represent an older \lq up to now\rq{ }meaning.

\begin{exe}
		\ex Slave\label{exInterrogativeYetSlave4}\\
		\gll \textbf{Kʼála} rídené̜wé gu kodisho̜ \textbf{yíle}.\\
		still \textsc{subj}.3:arrived whether \textsc{subj}.1\textsc{sg}:know \textsc{neg}\\
		\glt \lq I didnʼt know if it came (by air) yet.' \parencite[1250]{Rice1989}
\end{exe}
\il{Slave|)}

\il{French|(}
The last point brings me to the case of French \textit{encore}. In the present-day language, \textit{encore} as interrogative \lq yet\rq{ }is restricted to indirect questions, such as in (\ref{exFrenchInterrogative}) above. In older lects,\il{French, Old} however, it was also found in direct questions; see (\ref{exInterrogativeYetoldFrench}).

\begin{exe}
	\ex Old French,\il{French, Old} mid-12\textsuperscript{th} century\label{exInterrogativeYetoldFrench}\\
	\gll Est, va, \textbf{encore} toz mes charroiz entr-ez?\\
	\textsc{cop}.3\textsc{sg} \textsc{dm} still all.\textsc{pl} \textsc{poss}.1\textsc{sg}:\textsc{pl} car.\textsc{pl} enter-\textsc{ptcp}.\textsc{pl}\\
	\glt \lq Say, have all my carts entered yet?\rq{ }(\cite[144]{MosegaardHansen2008}, glosses added)
\end{exe}

As \textcite[144]{MosegaardHansen2008} points out, attestations like (\ref{exInterrogativeYetoldFrench}) most likely constituted a relic usage that goes back to the origins of \textit{encore} in a Latin phrase \lq until now, thus far'. This begs the question of where the present-day function in indirect questions comes from. Possibly, it constitutes the renovation of a once lost function, thereby conforming to \citeauthor{vanderAuwera1998}'s scenario. However, it is equally conceivable that negative raising, as in (\ref{exInterrogativeYetFrenchSubordinate}), did not strictly feed into indirect questions, but instead provided motivation for the retention of a function that already existed as an archaism. Future corpus-driven work will surely shed a light on this issue.

\begin{exe}
	\ex French\label{exInterrogativeYetFrenchSubordinate}\\
	\gll \textbf{Mais} \textbf{je} \textbf{ne} \textbf{vois} pas qu'-il ait \textbf{encore} fait son choix d’-une façon positive et irrevocable.\\
	but 1\textsc{sg} \textsc{neg} see.1\textsc{sg} \textsc{neg} \textsc{comp}-3\textsc{sg}.\textsc{m} have.\textsc{sbjv}.3\textsc{sg} still make.\textsc{ptcp} \textsc{poss}.3\textsc{sg}:\textsc{sg}.\textsc{m} choice(\textsc{m}) of-\textsc{indef}.\textsc{sg}.\textsc{f} manner(\textsc{f}) positive.\textsc{f} and irrevocable.\textsc{f}\\
	\glt \lq \textbf{But I don’t see} that he has \textbf{yet} made his choice in a positive and irrevocable way.' (\cite[169]{Martin1980}, glosses added)
\end{exe}
\il{French|)}\is{subordination|)}

\subsubsection{A closer look and discussion: Thai \textit{yaŋ}}\il{Thai|(}
In the preceding discussion I  hinted at the fact that there might be more ways in which \textsc{still} expressions can start to function as interrogative \lq yet\rq{}, and in \Cref{sectionNotYet} I lay out how \textsc{still} expressions frequently serve as markers of \textsc{not yet} in certain contexts lacking an overt predicate.\is{negation} A common subtype of the latter use is found in polar questions that follow a disjunctive pattern \lq already \textit{p} or still\rq{}. Here, an implied negator yields \lq … or not yet'. One marker used in this fashion is Thai \textit{yaŋ}, as shown in (\ref{exInterrogativeYetThai1}).

\begin{exe}
	\ex Thai\label{exInterrogativeYetThai1}\is{negation}\\
	\gll Thíŋ		còtmǎay	pai	tûu	praisanii	\textbf{lɛ́ɛw}		\textbf{rɯ̌ɯ}	\textbf{yaŋ}?\\
	throw letter go box mail already or still\\
	\glt \lq Did (you) put the letter into the mail-box \textbf{or not yet}?'
	\\(\cite[500]{Koelver1991},  glosses added)
\end{exe}

In present-day Thai, this disjunctive question tag is undergoing erosion: in the intermediate stages either \textit{lɛ́ɛw} \lq already' or \textit{rɯ̌ɯ} \lq or' are dropped; see (\ref{exInterrogativeYetThai2}, \ref{exInterrogativeYetThai3}). 

 \begin{exe}
	\ex\is{negation}
	\label{exInterrogativeYetThai}
	\begin{xlist}
		\exi{}Thai
		\ex\textit{Lɛ́ɛw} \lq already' omitted\label{exInterrogativeYetThai2} \\
		\gll khun	hâi	ʔaahǎan	mǎa	\textbf{rɯ̌ɯ}	\textbf{yaŋ}?\\
		2 give food dog or still\\
		\glt \lq Have you fed the dog \textbf{yet}?' \parencite[80]{Smyth2002}

		\ex
		\textit{Rɯ̌ɯ} \lq or' omitted	\label{exInterrogativeYetThai3}\\
		\gll Rîak	rót	thɛ́ksîi	\textbf{lɛ́ɛw}		\textbf{yaŋ}?\\
	call car taxi already still\\
	\glt \lq Did you call the taxi [\textbf{yet}]?' (found online)%\footnote{\url{http://www.thai-language.com/id/590018} (11 March 2021).}
	\end{xlist}
\end{exe}

 In the final stage of erosion, both \textit{lɛ́ɛw} and \textit{rɯ̌ɯ} are omitted, leaving \textit{yaŋ} as interrogative \lq yet\rq. This can be seen in (\ref{exInterrogativeYetThaiFinalStage}).
	
\begin{exe}
	\ex \label{exInterrogativeYetThaiFinalStage}
	\begin{xlist}
		\exi{}Thai

		\ex
		\gll Hǐw		\textbf{yaŋ}?\\
	 hungry still\\
	 \glt \lq Are you hungry \textbf{yet}?\rq{ }(found online)%\footnote{\url{http://www.thai-language.com/id/246231} (10 June, 2021).}

	\ex
	\gll Chán	sǒŋsǎi	wâa		khun	kin	khâaw	\textbf{yaŋ}.\\
	1\textsc{sg} wonder \textsc{comp} 2\textsc{sg} eat rice still\\
	\glt \lq I wonder whether you have eaten \textbf{yet}.\rq{ }(Chingduang Yurayong, p.c.)
	\end{xlist}
\end{exe}

\is{subordination|(}Not only are all steps from \textsc{not yet} to interrogative \lq yet\rq{ }directly observable, but there is an additional piece of evidence that \textit{yaŋ} did not follow \citeauthor{vanderAuwera1998}'s (\citeyear{vanderAuwera1998}) subordination scenario. Thus, Thai does not make use of negative\is{negation} raising with \textit{yaŋ} (Chingduang Yurayong, p.c.); recall that this is the first step in \Cref{figureSubordinationScenario}. Instead, when \isi{negation} is marked on the matrix clause, the \textsc{already}\is{already} item \textit{lɛ́ɛw} figures in the lower clause, as in (\ref{exInterrogativeYetThai6}). 

\begin{exe}
 	 \ex Thai\label{exInterrogativeYetThai6}\\
	 \gll Chán	mâi	khít	wâa		kháu	maa		\textup{(}thɯ̌ŋ\textup{)}	\textbf{lɛ́ɛw}.\\
	 1\textsc{sg} \textsc{neg} think \textsc{comp} 3 come \phantom{(}arrive already\\
	 \glt \lq I don't think he has arrived yet.' (Chingduang Yurayong, p.c.)
\end{exe}
\is{interrogative|)}\is{not yet|)}\il{Thai|)}\is{subordination|)}

\section{Broadly adverbial temporal-aspectual functions}\label{sectionAdverbial}
\addtocontents{toc}{\protect\setcounter{tocdepth}{3}}
\subsection{Introduction}
In this section, I survey time-related function that can be subsumed under a loosely defined label of \textit{adverbial} notions. In \Cref{sectionRepetition} I discuss uses pertaining to repetition\is{repetition} and in \Cref{sectionAlways} I survey \textsc{still} expressions as markers of \lq always\rq{ }and related notions. In \Cref{sectionProspective} I turn to a future-oriented \lq eventually\rq{} function.\is{prospective} In \Cref{sectionTaxis} I address uses pertaining to the broader realm of taxis or the relative ordering between situations. Lastly, in \Cref{sectionRemoteness} I discuss functions pertaining to degrees of temporal remoteness.\is{remoteness}

\subsection{Repetition}\is{repetition|(}
\label{sectionRepetition}
In this subsection I discuss uses pertaining to the domain of repetition, which I understand as encompassing two primary types of event construals. The first type, \textsc{iteration}, involves the complete repetition of a prior situation. A \textsc{restitutive} use, on the other hand, involves the restoration of a prior state, thereby implying a preceding process that yielded an opposing state  (\cite{FabriciusHansen1980}, \citeyear{FabriciusHansen2001}; \cite{KampRossdeutscher1994}; \cite{Waelchli2006}; among others). Both are shown in (\ref{exIterativeSectionIntro}), using \ili{English} \textit{again} for ease of illustration.

\begin{exe}
	\ex \label{exIterativeSectionIntro}
	\begin{xlist}
		\exi{}\ili{English}
		\ex Iterative\label{exIterativeSectionIntroIterative}\\
		\textit{Ines rang the bell once. Then she rang it \textbf{again}.}
		\ex Restitutive\label{exIterativeSectionIntroRestitutive}\\
		\textit{A sudden gust of wind blew the door shut. Rick opened it \textbf{again}.}
	\end{xlist}
\end{exe}

Iteration and restitution may be signalled by \textsc{still} expressions or by larger collocations containing one. In \Cref{sectionIterative} I discuss those cases in which repetition is conveyed by a \textsc{still} expression on its own. In \Cref{sectionIterativeViaIncrement}, I turn to those instances where an iterative or restitutive construal involves the collocation with an event quantifier \lq still one time > one more time > again\rq{}. Note that I discuss cases in which a \textsc{still} expression adds the notion of continuation\is{persistence} to an existing repetitive construal separately in \Cref{sectionContinuedIteration}.

\subsubsection{Repetition (plain  \lq still\rq{ }expression)}
\label{sectionIterative}
\subsubsubsection{Introduction} 
The \ili{Bambara} examples in (\ref{exIterativeBambaraRepeated}) illustrate repetition uses of a \textsc{still} expression. In (\ref{exIterativeBambaraRepeatedIt}), \textit{túguni} marks iteration in the form of a second occurrence of the fish speaking, whereas in (\ref{exIterativeBambaraRepeatedRest}) it signals that letting the chicken free corresponds to the restitution of a prior state.

\begin{exe}
	\ex \label{exIterativeBambaraRepeated}
	\begin{xlist}
		\exi{}\ili{Bambara}
		\ex Context: A fish spoke and was then cooked.\label{exIterativeBambaraRepeatedIt}\\		
		\gll À	y’ í kánto \textbf{túguni}.\\
		3\textsc{sg} \textsc{pfv}.\textsc{tr} \textsc{refl} speak still\\
		\glt \lq The fish spoke \textbf{again}.'

		\ex 
		\gll À yé syɛ̀ mìnɛ kʼ à bìla \textbf{túgun}.\label{exIterativeBambaraRepeatedRest}\il{Bambara}\\
		3\textsc{sg} \textsc{pfv}.\textsc{tr} chicken catch \textsc{inf} 3\textsc{sg} let still\\
		\glt \lq He caught the chicken and let it free \textbf{again}.'
		\\\parencite[118]{DombrowskyHahn2020}
		\end{xlist}
\end{exe}

\subsubsubsection{Distribution in the sample}
\Cref{tableIterative} lists the expressions in my sample that can mark repetition. Note that, in addition to these, \ili{Gurindji} \mbox{=\textit{rni}} forms part of a complex enclitic \mbox{=\textit{rningan}} with repetition uses (\appref{appendixGurindjiIterative}). \Cref{tableIterative} also indicates which of the two major types of uses (iterative vs. restitutive) are attested, as well as grammatical or lexical restrictions, where applicable. As can be gathered, a repetition-related use is attested for seventeen expressions from sixteen distinct languages across all six macro-areas, making it one of the most geographically wide-spread functions in my sample. This impression is also supported by an abundance of candidates not included in my sample.\footnote{For Africa, these include \textit{nánu} and \textit{lisusu} in the Bantu language \ili{Língala} \parencite{NassensteinPasch2021}, and  \textit{həma} in Chadic \ili{Lamang} \parencite[259–260]{Wolff2015}. In Australia,  \mbox{-\textit{anteye}} in the Pama-Nyungan language Eastern Arrernte\il{Arrernte, Eastern} \parencite[351]{Wilkins1989} is among the candidates. In Eurasia, repetition functions are found with \textit{nog} in southern varieties of \ili{Dutch} (\cite{vanderAuwera1993}; \cite[42 fn3]{vanBaar1997}), its cognate \textit{nokh} in \ili{Yiddish} \parencite{vanderAuwera1991Yiddish}, as well as with the \ili{Italian} cognate of \ili{French} \textit{encore}, \textit{ancora} \parencite{TovenaDonazzan2008}. The CLICS\textsuperscript{3} database \parencite{CLICS3} lists, among others, \ili{Adyghe} (Abkhaz-Adyge) \textit{d͡ʒərəj}, Northern Kurdish\il{Kurdish, Northern} (Iranian) \textit{disa}, and \textit{taxʲiăt}, \textit{taxʲiŋ} in Halh Mongolian.\il{Mongolian, Halh} Candidates in North America include Central Alaskan Yupik\il{Yupik, Central Alaskan} (Eskimo-Aleut) \textit{cali} (\cite[55, 66]{Amos2003}; \cite[1535]{Miyaoka2012}), \ili{Choctaw} (Muskogean) \textit{mo̠ma} \parencite[316–317]{Broadwell2006}, Plains Cree\il{Cree, Plains} (Algic) \textit{kêyâpic} \parencite[s.v. \textit{kêyâpic}]{Wolvengray2001}, and the \lq\lq repetitive prefix\rq\rq{ }in the Iroquian language \ili{Seneca} \parencite[38–39]{Chafe2015}. In Papunesia, examples from Formosan (Austronesian) are \mbox{-\textit{nʔa}} in \ili{Tsou} and \mbox{-\textit{aŋ}} in \ili{Atayal} and \ili{Bunun} (examples throughout \cite{ZeitounEtal2010}). Outside of Formosan, one finds, among others, \ili{Ida'an} \textit{masong} \parencite{Goudswaard2005}. Examples from South America include the cognates of Southern Lengua\il{Lengua, Southern} \textit{makham} across Lengua-Mascoy (e.g. \cite [79]{UnruhKalisch1999}; \cite[317]{UnruhEtAl2003}).} What is more, repetition uses are found with markers of various shapes and forms (free morphemes, affixes, clitics). In what follows, I first take a closer look at the distinction between iteration and restitution, then examine grammatical and/or lexical restrictions, to ultimately discuss the notional similarities and diachronic links between phasal polarity \textsc{still} and repetitive event construals.


\begin{table}
	\caption{Iterative and restitutive uses\label{tableIterative}}	
	\footnotesize
		\begin{tabular}{llllccl}
		\lsptoprule
		M.-Area & Language & Expr. & Appx. & \rot{Iterative} & \rot{Restitutive} & Restriction(s)\\
		\midrule
		Africa & \ili{Bambara} & \textit{túguni} & \ref{appendixBambaraTuguniIterative} & y & y  & Perfective\is{perfective} \isi{aspect} plus\\
			&&&&&& dynamic predicate\\

		&							& \textit{bìlen} & \ref{appendixBambaraBilenIterative} & y & y & Perfective \isi{aspect} plus\\
			&&&&&& dynamic predicate,\\
			&&&&&&locative copula\\
		&	B.-G. Datooga\il{Datooga, Barabayiiga-Gisamjanga} & \textit{údu-} &  \ref{appendixDatoogaIterative} & y & ? & Telic predicate in future\\
		&&&&&& or prohibitives\is{prohibitive}\is{command}\\
		& \ili{Ewe} & \textit{ga}- & \ref{appendixEweIterative} & y & y \\
		& \ili{Kaba} & \textit{bbáy} &  \ref{appendixKabaIterative} & y & y\\
		& S. Ndebele\il{Ndebele, Southern} & \textit{sa}- & \ref{appendixSouthernNdebeleIterative} &y & n & Perfective \isi{aspect} plus\\
	&&&&&&	non-inchoative predicate\\
		Australia & \ili{Gooniyandi} & =\textit{nyali} & \ref{appendixGooniyandiIterative} & y & y \\							
		& \ili{Wubuy} & -\textit{wugij} & \ref{appendixWubuyIterative} & \phantom{\textsuperscript{\itshape a}}y\footnote{Only one example in the data.} & y & Punctual \isi{aspect}\\
		Eurasia & \ili{French} & \textit{encore} & \ref{appendixFrenchEncoreIterative} &y & n \\
		& \ili{Ket} & \textit{hāj} &  \ref{appendixKetIterative} & y & n &\\ 
		& Mand. Chinese\il{Chinese, Mandarin} & \textit{hái} & \ref{appendixMandarinIterative} & y & n & Future reference \\
		& S. Yukaghir\il{Yukaghir, Southern} & \textit{āj} & \ref{appendixSouthernYukaghirIterative}& y & y \\
		& \ili{Udihe} & \textit{xai}(\textit{si}) & \ref{appendixUdiheIterative} & y & y \\
		N. America & Cl. Nahuatl\il{Nahuatl, Classical} & \textit{oc} & \ref{appendixClassicalNahuatlIterative}& y & n\\
		&		 \ili{Paiwan} & =\textit{anan} &  \ref{appendixPaiwanIterative} &y & y \\
		&		\ili{Saisiyat} & \textit{nahan} & \ref{appendixSaisiyatIterative} & y & y \\
		S. America & S. Lengua\il{Lengua, Southern} & \textit{makham} & \ref{appendixEnxetSurIterative}& y & y \\
		\lspbottomrule
		\end{tabular}
\end{table}	

\subsubsubsection{A closer look: Iteration vs. restitution}
It is convient to start my discussion with a brief look at the distribution of the two primary types of repetition, iteration vs. restitution. As can be gathered from \Cref{tableIterative}, with seventeen expressions the iterative use is slightly more common than the restitutive one, which is attested for eleven expressions (twelve if counting unclear cases). What is more, there are several expressions in the sample that only have the iterative function (Classical Nahuatl\il{Nahuatl, Classical} \textit{oc}, \ili{French} \textit{encore}, Mandarin Chinese\il{Chinese, Mandarin} \textit{hái}, and Southern Ndebele\il{Ndebele, Southern} \mbox{\textit{sa}-}), but none that exclusively have the restitutive one. The only expression that comes close to representing the latter type is \ili{Wubuy} \mbox{-\textit{wugij}}, where the restitutive use is common, wheareas the iterative one is rare, with only one attestation in \citeauthor{Heath1982}'s (\citeyear{Heath1982}) extensive text collection. The data thus point to the implicational universal in (\ref{exUniversalIterative}). Note that this contradicts \textcite[108–109]{vanBaar1997}, who stipulates a closer proximity between restitution and \textsc{still}.

\begin{exe}
	\ex\label{exUniversalIterative}
	If a \textsc{still} expression has a restitutive use, it also has an iterative use.
\end{exe}
\pagebreak

\is{syntax|(}\il{Gooniyandi|(}On a more fine-grained level, in Gooniyandi the two relevant uses of \mbox{ =\textit{nyali}} are typically distinguished via the expression's syntactic host. Thus, iterative \mbox{=\textit{nyali}} normally attaches to the part of the verbal predicate that denotes the repeated act, whereas the host of restitutive \mbox{=\textit{ngali}} is often the constituent depicting the state or location that is restored \parencite[460–461]{McGregor1990}. This is illustrated in (\ref{exIterativeGooniyandi}). An essentially parallel situation is found with the complex \ili{Gurindji} clitic \mbox{=\textit{rningan}}, which includes the \textsc{still} expression \mbox{=\textit{rni}} as part of its composition (\appref{appendixGurindjiIterative}). This distinction is furthermore reminiscent of what is found with \ili{German} \textit{wieder} \lq again', where, in certain cases, word order has a disambiguating function \parencite{vonStechow1996}.

\begin{exe}
	\ex \label{exIterativeGooniyandi}
	\begin{xlist}
		\exi{}Gooniyandi
		\ex Iterative\label{exIterativeGooniyandi1}\\
		\gll Jamoondoo milanggiddinyayi mila=\textbf{nyali} yawinggiddinyayi.\\
		other\_day I\_saw\_you\_two see=still I\_will\_extend\_you\_two\\
		\glt \lq I saw you two the other day, and I'll see you again.\rq{}
	
		\ex Restitutive\label{exIterativeGooniyandi2}\\
		\gll Niyi barnbindi ngiwawoo=\textbf{nyali}.\\
		he he\_returned south=still\\
		\glt \lq He returned south again.\rq{ }\parencite[460]{McGregor1990}
	\end{xlist}
\end{exe}\is{syntax|)}

Lastly, Gooniyandi \mbox{=\textit{nyali}} is also attested in a closely related responsive sense. Thus, in (\ref{exIterativeGooniyandi3}) this item highlights that the speaker returns another man's call.

\begin{exe}
	\ex Gooniyandi\label{exIterativeGooniyandi3}\\
	\gll Yoowarni-ngga baami-ngadda briyandi \textbf{baa}=\textbf{nyali} limi-nhi.\\
	one-\textsc{erg} he\_called-to\_me in\_return call=still I\_did-to\_him\\
	\glt \lq One (man) called to me, and in turn I \textbf{called} \textbf{back} \textbf{to} him.'
	\\\parencite[461]{McGregor1990}
\end{exe}\il{Gooniyandi|)}

\subsubsubsection{A closer look: Combinatory differences and restrictions}
\is{actionality|(}As I discuss below, phasal polarity \textsc{still} and \lq again\rq{ }have a common semantic denominator in the continuation\is{persistence} or accumulation of time intervals, with the main difference lying in the question of homogeneity. This, in turn, has implications as far as combinatory potentials are concerned.\is{aspect|(}\is{perfective|(}\is{anterior|(} Thus, the intermittent nature of iteration and restitution allows for such uses to combine with aspectual categories that would be incompatible with the notion of \textsc{still}. In the dimension of aspectual operators, this includes bounded viewpoints (e.g. perfectives and anteriors), and when it comes to the internal temporal make-up of the predicate, this affects (near-)spontaneous situations, i.e. achievements and semelfactives. Example (\ref{exIterativePFVFrench}) is an illustration of \ili{French} \textit{encore} in its iterative use together with the analytical anterior, which doubles as a perfective past in the spoken language. Example (\ref{exIterativePFVFrenchKolyma}) shows Southern Yukaghir\il{Yukaghir, Southern} \textit{āj} together with the perfective aspect and an achievement predicate.

\begin{exe}
	\ex\label{exIterativePFVFrench}
	\ili{French}\\
	\gll Caroline a \textbf{encore} téléphoné.\\
	C. have.3\textsc{sg} still call.\textsc{ptcp}\\
	\glt \lq Caroline has phoned \textbf{again}.\rq
	\\(\cite[148]{MosegaardHansen2008},  glosses added)
	
	\ex\label{exIterativePFVFrenchKolyma}
	 Southern Yukaghir\il{Yukaghir, Southern}\\
	 Context: A man has brought the master of the earth a barrel full of spirit in exchange for fur. Now he has come a second time.\\
	\gll D’e tude boc’ka-gele \textbf{aj} køud-ej-m.\\
	\textsc{dm} 3\textsc{sg}.\textsc{gen} barrel-\textsc{acc} still take\_away-\textsc{pfv}-\textsc{tr}.3\textsc{sg}\\
	\glt \lq He brought his barrel \textbf{again}.' \parencite[text 25]{NikolaevaMayer2004}
\end{exe}
\is{perfective|)}\is{anterior|)}

On the flipside of things, the differences in compatibilities just outlined offer a motivated explanation for several of the morphosyntactic\is{syntax} and lexical restrictions listed in \Cref{tableIterative}. While these may appear rather dissimilar at first glance, they turn out to be variations over the common theme in (\ref{exIterativeRestrictions}).

\begin{exe}

	\ex\label{exIterativeRestrictions}
	When \lq again\rq{ }uses of \textsc{still} expressions are subject to morphosyntactic\is{syntax} and/or lexical restrictions, they are barred from those constellations in which they compete with phasal polarity \textsc{still} (setting aside third types of uses).
\end{exe}

\il{Wubuy|(}Thus, in Wubuy, iterative and restitutive \mbox{-\textit{wugij}} are restricted to what \textcite{Heath1984} terms the \lq\lq punctual" aspect. This combination is illustrated in (\ref{exIterativeWubuy}).\is{continuous|(} In broad strokes, the Wubuy punctual–continuous distinction is primarily an actional one. The punctual aspect is the paradigm of choice with achievements and semelfactives. When it combines with states or processes, these are converted into achievements; see \textcite[337–341]{Heath1984} for more discussion. In essence then, readings of repetition are only found with non-durative situations, which cannot persist\is{persistence} in time.\is{continuous|)}

\begin{exe}	
\ex	Wubuy\\
	Context: A python has devoured two boys. A magician has killed the python and has taken out her guts containing the boys.\label{exIterativeWubuy}\\
\gll 
Ni=lhan\textsuperscript{g}ad̠bi-nʸ \textbf{wani}=\textbf{ya}-\textbf{nʸ}-\textbf{bugij} mana-n\textsuperscript{g}udan man-uba-ma-yun\textsuperscript{g} wan=ya-nʸ yuːguni, wani=ya-nʸ.\\
3\textsc{sg}.\textsc{m}=emerge-\textsc{pst}.\textsc{punct} 3\textsc{sg}.\textsc{m}>3\textsc{pl}=give-\textsc{pst}.\textsc{punct}-still \textsc{ncl}\textsubscript{MANA}-guts \textsc{ncl}\textsubscript{MANA}-\textsc{anaph}-\textsc{ncl}\textsubscript{MANA}-\textsc{abs} 3\textsc{sg}.\textsc{m}>3\textsc{pl}=give-\textsc{pst}.\textsc{punct} \textsc{dist}  3\textsc{sg}.\textsc{m}>3\textsc{pl}=give-\textsc{pst}.\textsc{punct}\\
\glt \lq He came out. \textbf{He gave} those guts (containing the two boys) [\textbf{back}] \textbf{to them} [people].' \parencite[23]{Heath1980}
\end{exe}\il{Wubuy|)}

\is{perfective|(}In the case of Southern Ndebele\il{Ndebele, Southern} \mbox{\textit{sa}-}, illustrated in (\ref{exIterativeSouthernNdebele}), the iterative use is restricted to the perfective aspect inflection. The same is true, with one minor exception, of the iterative and restitutive uses of \ili{Bambara} \textit{túguni} and \textit{bìlen}. An additional lexical constraint shared by both languages is that the uses in question are only found with predicates labelled \lq\lq inchoative" (Southern Ndebele) or \lq\lq stative" (\ili{Bambara}) in the respective descriptive traditions. Terminological differences aside, the predicates excluded are those that have an ongoing state reading in the perfective aspect and which would hence allow for a construal of a persistent\is{persistence} state.\footnote{See \textcite{HewsonBambara} and \textcite{CranePersohn2021} for a more extensive discussion of the interaction between aspect and actionality in these languages.}

\begin{exe}
\ex Southern Ndebele\il{Ndebele, Southern}\label{exIterativeSouthernNdebele}\\
\gll Idrayara i-\textbf{sa}-rhuny-ez-e irhembe (godu).\\
	\textsc{ncl}9.dryer \textsc{subj}.\textsc{ncl}9-still-shrink-\textsc{caus}-\textsc{pfv} \textsc{ncl}9.shirt (again)\\
	\glt \lq The dryer shrank a shirt again (it happened another time).'
	\\\parencite[244]{CranePersohn2021}
\end{exe}

Similar restrictions are found with \ili{Udihe} \mbox{\textit{xai}(\textit{si})} and \ili{Saisiyat} \mbox{=\textit{nahan}}, albeit as strong tendencies rather than as hard-and-fast rules. In a related manner, an iterative reading of \ili{French} \textit{encore} is disprefered in the constellation of an \isi{imperfective} viewpoint\is{aspect} plus atelic predicate \parencite[155 fn19]{MosegaardHansen2008}.\is{perfective|)} \is{tense|(} In Barabayiiga-Gisamjanga Datooga,\il{Datooga, Barabayiiga-Gisamjanga} the \textsc{still} expression \mbox{\textit{údu}-} brings with it a neutralisation of most tense-aspect distinctions to one of future vs. non-future tense. Both the aspectual viewpoint and the type of event construal are determined by the \isi{telicity} of the predicate. Atelic predicates go along with an \isi{imperfective} viewpoint and yield \textsc{still}. Telic predicates, on the other hand, go along with a \isi{perfective} or \isi{anterior} viewpoint. In the future tense and with prohibitives,\is{prohibitive}\is{command} this yields a reading of repetition \parencite{Mitchell2021}.  Broadly comparable, Mandarin Chinese\il{Chinese, Mandarin} \textit{hái} only has an iterative use in reference to future states-of-affairs, as in (\ref{exIterativeMandarin}).\is{actionality|)}

\begin{exe}
	\ex\label{exIterativeMandarin}Mandarin Chinese\il{Chinese, Mandarin}\\
	\gll  Míngtiān	\textbf{hái}	huì	xiàyǔ	ma?\\
	Tomorrow still will rain \textsc{q}\\
	\glt \lq Will it rain again tomorrow?' \parencite[61]{HuangShi2016}
\end{exe}\is{tense|)}\is{aspect|)}

\subsubsubsection{A closer look: Superficial redundancies}
It is noteworthy that several sample expressions in their function as \lq again' tend to be accompanied by other items that signal repetition. In the sample, such cases are attested for Barabayiiga-Gisamjanga Datooga\il{Datooga, Barabayiiga-Gisamjanga} \mbox{\textit{údu}-}, \ili{Paiwan} \mbox{=\textit{anan}}, Southern Ndebele\il{Ndebele, Southern} \mbox{\textit{sa}-}, Southern Lengua\il{Lengua, Southern} \textit{makham}, and \ili{Udihe} \mbox{\textit{xai}(\textit{si})}. Superficial redundancies of this kind are a recurrent theme with functional extensions of \textsc{still} expressions, see, for instance \Cref{sectionProspective,sectionAdditive}. In the case of 
Southern Ndebele\il{Ndebele, Southern} \mbox{\textit{sa}-} and \ili{Paiwan} \mbox{=\textit{anan}} this gives emphasis to the recurrent nature of the event in question; in the case of \mbox{=\textit{anan}}, a reading of a continued\is{persistence} series of repetitions is also available (\Cref{sectionContinuedIteration}).

\subsubsubsection{Discussion: Conceptual similarities}
The widespread coexpression of phasal polarity \textsc{still} and iteration/restitution is not surprising if one considers the conceptual similarities between the two notions. In broad strokes, they both involve continuation,\is{persistence} or the accumulation of time intervals. Where they differ is primarily in homogeneity: \textsc{still} involves uninterrupted continuation, whereas iteration and restitution feature intervening temporal gaps (\cite{vanderAuwera1991BeyondDuality}, \citeyear{vanderAuwera1993}; \cite[108–109]{vanBaar1997}; \cite{JingSchmidtGries2009}; \cite{MosegaardHansen2002}, \citeyear[155]{MosegaardHansen2008}; \cite{TovenaDonazzan2008}; \cite[351]{Wilkins1989}; among others). In addition, \textsc{still} and \lq again' also share the notion of identity,\is{identity} which is relevant for the Australian markers in question.

\subsubsubsection{Discussion: A diachronic perspective}
Against the backdrop of the similarities just outlined, developments from \textsc{still} to \lq again\rq{ }as well as the other way around, are attested. In what follows, I briefly summarise what is known about the history of some of the sample expressions. I sketch out a few bridging contexts and point to intra-systemic constellations that may play a role in the relevant developments, without any claim to comprehensiveness.

The development  of an \lq again\rq{ }sense out of an originally phasal polarity one is well-established for \ili{French} \textit{encore}.\is{modality|(}  As \textcite[156]{MosegaardHansen2008} discusses, a bridging context can be found in instances like (\ref{exIterativeOldFrench}).\il{French, Old} This example can be interpreted as involving the possible continuation\is{persistence} of an overarching series \lq may still help me' or as the occurrence of another event of the same type \lq may help me again'. A very similar instance, albeit featuring \isi{necessity} rather than possibility,\is{possibility} is found in the \il{Nahuatl, Classical|(}Classical Nahuatl example (\ref{exRepetitionNahuatlForgive}), where the iterative interpretation is furthermore strengthened through the presence of \textit{quēmman} \lq at times\rq{}.

\begin{exe}
	\ex Old French,\il{French, Old} ca. 1176–1184 \label{exIterativeOldFrench}\is{persistence}\\
	\gll Dius m'-a bien aidié {dusc'a ore}, // Si \textbf{me} \textbf{puet} \textbf{bien} \textbf{aidier} \textbf{encore}.\\
	God 1\textsc{sg}.\textsc{acc}-have.3\textsc{sg} well help.\textsc{ptcp} {until now} {} then 1\textsc{sg}.\textsc{acc} can.3\textsc{sg} well help.\textsc{inf} still\\
	\glt \lq God has helped me so far, so he \textbf{may} \textbf{well} \textbf{help} \textbf{me} \textbf{still} \textbf{/} \textbf{again}.'
	\\(\textit{Eracle}, cited in \cite[156]{MosegaardHansen2008},  glosses added)
	\ex Classical Nahuatl\is{persistence}\label{exRepetitionNahuatlForgive}\\
		\gll Zā \textbf{oc} \textbf{quēmman} \textbf{mo}-\textbf{cuā}-\textbf{zquê} in ī-nāmic.\\
	only stilll at\_times \textsc{subj}.3:\textsc{refl}-eat-\textsc{prosp} \textsc{det} \textsc{poss}.3\textsc{sg}-spouse\\
	\glt \lq Il faut que son mari (et elle) s'accouplent (\lq\lq se mangent") encore de temps en temps. [It is necessary that her husband (and her) \textbf{mate} (\textbf{lit. eat each other}) \textbf{again from time to time}.]' (\cite[1265]{Launey1986}, glosses added)\il{Nahuatl, Classical|)}
\end{exe}\is{modality|)} 

Another item that most likely started out as an exponent of \textsc{still} and subsequently acquired an \lq again\rq{ }function is Southern Ndebele\il{Ndebele, Southern} \mbox{\textit{sa}-}. Cognates of this expression are found as phasal polarity expressions across the entire Nguni branch of Southern Bantu, but in most instances without repetition-related functions (e.g. \cite{CranePersohn2021}; \cite[338–345]{PoulosMsimang1998}; \cite[192–195]{SavicThesis}). The same overall direction of change could also hold true for Barabayiiga-Gisamjanga-Datooga\il{Datooga, Barabayiiga-Gisamjanga} \mbox{\textit{údu}-}. While the exact pathways of change are unknown, it is noteworthy that \mbox{\textit{sa}-} and \mbox{\textit{údu}-} are both also involved in the expression of \textsc{no longer},\is{no longer} and there is a well-known notional overlap between \isi{discontinuation} and disrepetition (e.g. \cite[92–93]{vanBaar1997}). This is particularly salient in \isi{irrealis} contexts, or the negative\is{negation} mirror image of (\ref{exIterativeOldFrench}): \lq I won't do it again' is not all too different from \lq I won't do it anymore'. Conceivably, this has played at least some role in the development of these markers. If I am on the right track here, then the restriction of B.-G. Datooga\il{Datooga, Barabayiiga-Gisamjanga} \mbox{\textit{údu}-} as \lq again\rq{ }to the future \isi{tense} and prohibitives\is{prohibitive}\is{command} could be a reflection of this function's history.

The inverse direction, from \lq again' to \textsc{still} can be firmly established for at least two markers in the sample, namely Mandarin Chinese \il{Chinese, Mandarin}\textit{hái} and \ili{Ewe} \mbox{\textit{ga}-}, both of which have their etymons in motion verbs \lq return, go back\rq{}. Similarly, the comparative data on Hebrew\il{Hebrew, Modern} \textit{ʕod} are indicative of an original meaning pertaining to repetition or circularity (\cite[s.v. \RL{עוּד}]{BrownEtAl}), though this sense has become lost in the present-day variety.\is{actionality|(}\is{telicity|(} In terms of actual usage, a  plausible link is found in those instances in which a snapshot of the same atelic situation is given across different times. For instance, the Middle Chinese\il{Chinese, Middle} example (\ref{exIterativeMiddleChinese}) can be understood as either involving a reiteration of a similar life one generation after another, or as life as such remaining the same, with little to no difference in communicative import. This ambiguity is further enhanced by the presence of \textit{ran} \lq same'.\footnote{The combination of \textit{hái} with \textit{fu} \lq again' in (\ref{exIterativeMiddleChinese}) is common at this stage of the language, and is due to \textit{hái} < \textit{huan} \lq return\rq{ }acquiring the iterative reading through the frequent collocation with this marker \parencite{Yeh1998}.}

\begin{exe}
	\ex Middle Chinese,\il{Chinese, Middle}  8\textsuperscript{th}/9\textsuperscript{th} century\label{exIterativeMiddleChinese}\\
	\gll Zi sun ri yang chang, \textbf{shi} \textbf{shi} \textbf{hai} \textbf{fu} \textbf{ran}.\\
	son grandchild day and grow generation generation still again same\\
	\glt \lq The children are growing day by day; \textbf{life is still the same}, \textbf{generation by generation}.' \parencite[247]{Yeh1998}
\end{exe}

In a related fashion, \textcite{Waelchli2006} observes that iterative markers can acquire a continuative\is{persistence} interpretation in the context of \lq\lq atelic (and therefore not strictly countable) activities where it does not make sense to apply the repetitive concepts \lq for a second/third time’" \parencite[76]{Waelchli2006}. He illustrates this with the \ili{Vietnamese} example (\ref{exIterativeVietnamese}). Here, the second quote from Moses constitutes a seamless continuation\is{persistence} of a speech event that is recollected in two pieces.

\begin{exe}
	\ex \ili{Vietnamese}\label{exIterativeVietnamese}\\
	\gll Vì Môise có nói: Hãy tôn-kính cha mẹ ngươi; \textbf{lại} \textbf{nói}: Ai rủa-sả cha mẹ, thì phải bị giết.\\
	for Moses have say \textsc{imp} honour-honour father mother pupil again say who curse-charge father mother then shall suffer die\\
	\glt \lq For Moses said \lq\lq Honor your father and mother", \textbf{and} \lq\lq Anyone who curses their father or mother is to be put to death."' \\(Mark 7: 10, cited in \cite[76]{Waelchli2006})
\end{exe}\is{actionality|)}\is{telicity|)}

Aside from such relatively direct pathways, the expression of the negative\is{negation} phasal polarity concept \textsc{no longer}\is{no longer} may once again play a role. Thus, it is conceivable that the logical equivalence between \textsc{no longer} \textit{p} and \mbox{\neg\textsc{still}} \textit{p} (\Cref{secFunctionalDiscussion}) can open the doors to the reanalysis of an originally disrepetitive construction as the outer \isi{negation} of a \textsc{still} expression.\footnote{The mirror image, so to speak, of this process is attested for most of Southern Slavic, where \lq more' > \textsc{already}\is{already} via the reanalysis of \mbox{\textsc{no longer}\is{no longer} \textit{p}} as \mbox{\textsc{already} \neg\textit{p}}\is{negation} (\cite[60]{vanderAuwera1998}; \cite{Buchholz1991}).} \Textcite[92–93]{vanBaar1997} suggests that this is what took place in the case of \ili{Ewe} \mbox{\textit{ga}-}, and the same process appears to be currently going on with \ili{Bambara} \textit{túguni} and \textit{bìlen}. The two \ili{Bambara} items are both involved in the expression of \textsc{no longer},\is{no longer} and, according to \citeauthor{DombrowskyHahn2020} (\citeyear{DombrowskyHahn2020}, \citeyear{DombrowskyHahn2021}), the \lq again' function in affirmative contexts is both older and more frequent.

\il{Gooniyandi|(}Last, but far from least, another indirect link may be at play in the case of the two relevant sample expressions from Australia, \ili{Wubuy} \mbox{-\textit{wugij}} and Gooniyandi \mbox{=\textit{nyali}}. Both markers also function as non-scalar exclusive operators \lq only' (\Cref{sectionExclusive}). The same is true of \ili{Gurindji} \mbox{=\textit{rni}}, which forms part of the enclitic \mbox{=\textit{rningan}}. As \textcite{SchultzeBerndt2002} points out, exclusive and repetitive uses share the notion of identity\is{identity} \lq just that (type of) situation, just that original state (not any other)'. At least in principle, it is conceivable that the affirmative phasal polarity function and the one as markers of repetition arise independently out of a prior exclusive function. I leave this question open for further cross-linguistic research.\il{Gooniyandi|)}

\subsubsection{Repetition via increment}
\label{sectionIterativeViaIncrement}
\subsubsubsection{Introduction}\largerpage
In this subsection, I turn to iterative and restitutive uses that involve the collocation of a \textsc{still} expression with an event quantifier. \il{Nahuatl, Classical|(}The Classical Nahuatl examples in (\ref{exIterativeSectionIntroClassicalNahuatlRepeated}) are illustrations. In (\ref{exIterativeSectionIntroClassicalNahuatlRepeatedIterative}), \textit{oc} plus \textit{cē}-\textit{ppa} \lq one-time' signal a second eating event. In (\ref{exIterativeSectionIntroClassicalNahuatlRepeatedRestitutive}), the same collocation marks the restoration of a bone's original location.
\begin{exe}
	\ex  \label{exIterativeSectionIntroClassicalNahuatlRepeated}
	\begin{xlist}
	\exi{}Classical Nahuatl
	\ex
	\gll Auh quēmman \textbf{oc} \textbf{cē}-\textbf{ppa} ti-tla-cuā-z?\label{exIterativeSectionIntroClassicalNahuatlRepeatedIterative}\\
	and when still one-time \textsc{subj}.2\textsc{sg}-\textsc{obj}.\textsc{indef}.\textsc{non}.\textsc{human}-eat-\textsc{prosp}\\
	\glt \lq Y a qué hora has de comer otra vez? [And when will you eat \textbf{again}]?'

	\ex
\gll In tlā cē chico-petōni huel tēcocô, auh nō huel tecocô \textbf{in} \textbf{ic} \textbf{oc} \textbf{ce}-\textbf{ppa} \textbf{ī}-\textbf{ye}-\textbf{yān} \textbf{mo}-\textbf{zalao}.\label{exIterativeSectionIntroClassicalNahuatlRepeatedRestitutive}\\
\textsc{det} if one sideways-dislocate \textsc{intens} painful and also \textsc{intens} painful \textsc{det} thus still one-time \textsc{poss}.3\textsc{sg}-\textsc{loc}.\textsc{cop}-customary\_place \textsc{refl}.3-put\_together\\
\glt \lq Si se desconcierta vno, y se sale a vn lado, duele mucho, como tambien duele mucho, quando se buelue a su lugar. [When one [of our bones] dislocates it hurts a lot, and it also hurts \textbf{when} \textbf{it} \textbf{moves} \textbf{back} \textbf{into} \textbf{its} \textbf{place}.]' (\cite[498, 505]{Carochi1645},  glosses added)
	\end{xlist}
\end{exe}

While the examples in (\ref{exIterativeSectionIntroClassicalNahuatlRepeated})\il{Nahuatl, Classical|)} feature a nominal constituent as the second part of the collocation, the same use is also attested with denumeral verbs \lq do time(s)\rq{},\il{Saisiyat|(} as in the Saisiyat example (\ref{exIterativeIncrementSaisiyat}).\footnote{For a discussion of the structure of denumeral verbs in Saisiyat, see \citeauthor{ZeitounEtal2010} (\citeyear[575–577]{ZeitounEtal2010}, \citeyear[522]{ZeitounEtal2015}).}

\begin{exe}
	\ex Saisiyat\label{exIterativeIncrementSaisiyat}\\
		\gll Yao \rq{}am=mari’ ka lapwar boay \rq{}a-k<m>ai:, \rq{}okay kay-hoero:, \rq{}oka\rq{}=ila=o \textbf{mon}-\textbf{ha}-\textbf{l} \textbf{naehan}, k<om>ay-hoero:=ila mari\rq{}=ila ka boay noka lapwar.\\
1\textsc{sg}.\textsc{nom} \textsc{prog}=take \textsc{acc} guava fruit \textsc{prog}-hook<\textsc{agt}.\textsc{foc}> \textsc{neg}:\textsc{lnk} hook-succeed \textsc{neg}=\textsc{cmpl}=\textsc{conj} \textsc{agt}.\textsc{foc}:do\_times-one-times still hook<\textsc{agt}.\textsc{foc}>-succeed=\textsc{cmpl} take=\textsc{cmpl} \textsc{acc} fruit \textsc{gen} guava\\
\glt \lq I was trying to gather guavas but I could not hook them; I tried \textbf{again} and succeeded in catching guavas.'  \parencite[524]{ZeitounEtal2015}
\end{exe}\il{Saisiyat|)}

\subsubsubsection{Distribution in the sample}
\Cref{tableIterativeViaIncrement} lists the expressions in the sample that have repetition uses in combination with an event quantifier. \Cref{tableIterativeViaIncrement} also indicates whether these phasal polarity items can have a comparable function on their own. As can be gathered, the uses I discuss here are attested for ten expressions. Crucially, only four of these also have similar functions outside the relevant collocations. At least from a cross-linguistic perspective, this means that the two ways of marking repetitions should be kept separate, unlike what is commonly assumed in descriptive grammars of individual languages such as \ili{French} (e.g. \cite[535]{BatchelorChebliSaadi2011}). In terms of geographic distribution, repetition uses involving an event quantifier are attested primarily from Eurasia, which suggests some degree of areality to the phenomenon.\footnote{The same functions are attested for several cognates of my sample expressions, for instance \ili{Dutch} \textit{nog} \parencite[145]{Koenig1991} and cognates of Serbian\hyp Croatian\hyp Bosnian\il{Serbian}\il{Croatian}\il{Bosnian} \textit{još} across Slavic (\cite[s.v. \textit{ešte}]{SSSJ}  \cite[s.v. \textit{ještě}]{SSJC}; \cite[s.v. \textit{ešte}]{KSS4}; \cite{Mustajoki1988}). They are also found with \ili{Hungarian} \textit{még} \parencite[s.v. \textit{még}]{BarcziOrszagh1992} and \ili{Finnish} \textit{vielä} \parencite[145]{Koenig1991}.}  Lastly, except for Northern Qiang \mbox{\textit{tɕe}-}\il{Qiang, Northern} all relevant cases involve independent grammatical words, though this may merely be an indirect effect of the geographic distribution.

\begin{table}
	\caption{Repetition via increment. *: Only one example in the data.\label{tableIterativeViaIncrement}}	
		\footnotesize
		\begin{tabular}{lllclll}
			\lsptoprule
			&  & & & \multicolumn{3}{c}{With event quantifier}\\\cmidrule(lr){5-7}
			M.-Area & Language & Expr. & Alone & Collocate && Appendix  \\\midrule
			Eurasia & \ili{French} & \textit{encore} & y &\textsc{num} \textit{fois} & \lq time(s)\rq{}& \ref{appendixFrenchIterativeIncrement} \\
			& \ili{German} & \textit{noch} & n & (\textit{ein})\textit{mal} & \lq(one)time\rq{}& \ref{appendixGermanIterativeViaIncrement}\\
				&  &&& -\textit{mal}-\textit{s} & \lq -time-\textsc{adv}' &\ref{appendixGermanIterativeViaIncrement} \\
			& Hebrew (Mod.)\il{Hebrew, Modern} &  \textit{ʕod} & n& \textit{paʕam} & \lq time'& \ref{appendixHebrewOdIterativeIncrement}\\
		& \ili{Ket} & \textit{hāj} & y & \textit{biks'a} & \lq different'& \ref{appendixKetIterativeIncrement} \\
		& &		&  &  \textit{s'in} & \lq once' & \ref{appendixKetIterativeIncrement}\\
		& Northern Qiang\il{Qiang, Northern} & \textit{tɕe}- & n & \textit{a}-\textit{ʂ}/\textit{thən} & \lq one-time'  & \ref{appendixNorthernQiangIterativeIncrement}\\
		& Serb.-Croat.-Bosn. & \textit{još} & n & \textit{jednom} & \lq once' & \ref{appendixBCMSIterativeIncrement}  \\
		& Tundra Nenets \il{Nenets, Tundra}& \textit{təmna}& n & \textit{ŋopoy\textsuperscript{o}} & \lq once'* & \ref{appendixTundraNenetsIterativeIncrement}\\
		N. America & Class. Nahuatl\il{Nahuatl, Classical} & \textit{oc} & y & \textit{cē}-\textit{ppa} & \lq one-time' & \ref{appendixClassicalNahuatlIterativeIncrement} \\
		&	\ili{Kalaallisut} & \textit{suli} & n & \multicolumn{2}{l}{denumeral verb} & \ref{appendixGreenlandicIterativeIncrement} \\
		 Papunesia & \ili{Saisiyat} & \textit{nahan} & y & \multicolumn{2}{l}{denumeral verb} & \ref{appendixSaiyiatIterativeIncrement} \\
		\lspbottomrule
		\end{tabular}
\end{table}

\subsubsubsection{A closer look}
I have already indicated that a repetitive event construal in combination with an event quantifier does not entail that a \textsc{still} expression has a similar function on its own. What is more, there are no indications that the relevant collocations are subject to the same lexical and/or grammatical restrictions that bare \lq again' uses of \textsc{still} expressions are (\Cref{sectionIterative}). A noteworthy parallel can, however, be observed in terms of the iterative–restitutive distinction. \Cref{tableIterativeViaIncrement2} indicates which of the two construals is available for each collocation. As can be seen upon examination, all collocations in question have an iterative use, whereas the restitutive one is attested for only four collocations and three distinct expressions. This distribution can be stated as the absolute universal in (\ref{exUniversalIterativeViaIncrement}), which is fully parallel to the one given in (\ref{exUniversalIterative}) above.

\begin{exe}
	\ex If a \textsc{still} expression has a restitutive-via-increment use, it also has an iterative-via-increment use.\label{exUniversalIterativeViaIncrement}
\end{exe}

\begin{table}[ht]
\caption{Iterative-via-increment vs. restitutive-via-increment. *: With qualifications; see discussion below. \label{tableIterativeViaIncrement2}}
	\footnotesize
	\begin{tabular}{lllllcc}
		\lsptoprule
		Macro-area & Language & Expr. & Collocate & & \rot{Iterative} & \rot{Restitutive}.  \\
		\midrule
Eurasia &\ili{French} & \textit{encore} & \textsc{num} \textit{fois} & \lq time(s)' & y & n \\
		 &\ili{German} & \textit{noch} & (\textit{ein})\textit{mal} & \lq (one)time' & y & \phantom{*}y*\\
		  &&	& -\textit{mal}-\textit{s} & \lq -time-\textsc{adv}' & y & \phantom{*}y* \\
&			 Hebrew (Modern)\il{Hebrew, Modern} &  \textit{ʕod} & \textit{paʕam} & \lq time' &  y & n \\		
&	 \ili{Ket} & \textit{hāj} & \textit{biks'a} & \lq different' & y & y \\
	&		&  & \textit{s'in} & \lq once' & y & n \\
	 &Northern Qiang\il{Qiang, Northern} & \textit{tɕe}- & \textit{a}-\textit{ʂ}/\textit{a}-\textit{thən} & \lq one-time' & y & n\\
	& Serbian-Croatian-Bosnian\il{Serbian}\il{Croatian}\il{Bosnian} & \textit{još} & \textit{jednom} & \lq once' &  y & n  \\
	 &Tundra Nenets & \textit{təmna}& \textit{ŋopoy\textsuperscript{o}}& \lq once' & y & n \\
North America & Classical Nahuatl\il{Nahuatl, Classical} & \textit{oc} & \textit{cē}-\textit{ppa} & \lq one-time' & y & y \\
&			\ili{Kalaallisut} & \textit{suli} &  \multicolumn{2}{l}{denumeral verb}& y & n \\
Papunesia& \ili{Saisiyat} & \textit{nahan} &  \multicolumn{2}{l}{denumeral verb}& y & n \\
	\lspbottomrule
	\end{tabular}
\end{table}	

Interestingly, though, the sample data suggest that the availability of the less common restitutive function is largely independent of whether the \textsc{still} expression in question can also signal the restoration of a state on its own.\il{Nahuatl, Classical|(} Thus, Classical Nahuatl \textit{oc} \textit{cē}-\textit{ppa} \lq{}still one-time\rq{ }is unequivocally attested in both readings, as shown in (\ref{exIterativeSectionIntroClassicalNahuatlRepeated}) above. Bare \textit{oc}, on the other hand, appears to only have the iterative use.\il{Nahuatl, Classical|)} A comparable situation might hold true of \ili{Ket} \textit{hāj} and the collocation \textit{hāj} \textit{biks'a} \lq still different/again\rq{}, but the data are much more limited in this case.\il{German|(} Lastly, the case of German \textit{noch einmal} and its variants \textit{nochmal}, \textit{nochmals} requires some discussion. Aside from an uncontroversial iterative reading, these forms are occasionally described as having a restitutive use as well (\cite[s.v. \textit{noch einmal}, \textit{nochmal}, \textit{nochmals}]{Duden}; \cite[105]{Nederstigt2003}; \cite{Shetter1966}). A typical example is given in (\ref{exIterativeViaIncrementGermanRestitutive}).\pagebreak

\begin{exe}
	\ex German\label{exIterativeViaIncrementGermanRestitutive}\\
	\gll Sie bestell-ten eine Flasche Gruaud Larose bei ihr, \textbf{die} {\textbf{Hans} \textbf{Castorp}} \textbf{noch} \textbf{ein}-\textbf{mal} \textbf{fort}-\textbf{schick}-\textbf{te}, um sie besser temperier-en zu lass-en.\\
	3\textsc{pl} order-\textsc{pst}.3\textsc{pl} \textsc{indef}.\textsc{acc}.\textsc{sg}.\textsc{f} bottle(\textsc{f}) G. L. at 3\textsc{sg}.\textsc{dat}.\textsc{f} \textsc{rel}.\textsc{acc}.\textsc{sg}.\textsc{f} {H. C.} still one-time away-send-\textsc{pst}.3\textsc{sg} to 3\textsc{sg}.\textsc{acc}.\textsc{f} better temper-\textsc{inf} to let-\textsc{inf}\\
	\glt \lq They ordered a bottle of Gruaud Larose from her, \textbf{which} \textbf{Hans} \textbf{Castorp} \textbf{sent} \textbf{back} to have it brought to drinking temperature.' (Mann, \textit{Der Zauberberg}; cited in \cite[61]{Shetter1966},  glosses added)
\end{exe}

In (\ref{exIterativeViaIncrementGermanRestitutive}) the restoration of the bottle's original location is markedly transitory. As is made explicit in the subsequent clause, the bottle is to be brought to the table once more after it has reached the desired temperature. My personal judgement suggests that the same applies to similar German examples in the literature. This is probably due to other functions of the items involved. Thus, \textit{noch} also has a \lq\lq further-to" use (\Cref{sectionFurtherTo}), which can be paraphrased roughly as \lq do one more thing (before moving on)\rq{}.\is{precedence} The second part of the collocation, (\textit{ein})\textit{mal}, also serves as an indefinite temporal adverb and as a hedge (e.g. \cite[s.v. \textit{mal}]{Duden}). Putting these two pieces together, example (\ref{exIterativeViaIncrementGermanRestitutive}) can be read as \lq … which H. C. then sent away for some time (before ultimately consuming it)'. In other words, such instances of \textit{noch} (\textit{ein})\textit{mal}/\textit{nochmals} appear to not be purely restitutive, which qualifies the cross-linguistic availability of this type of use even further.\il{German|)}

\subsubsubsection{Discussion}\is{additive|(} 
I now turn to the question of what motivates the \textsc{still}-plus-event quantifier collocations. As the title of this subsection already suggests, what lies at the heart of them is additivity, in that \lq still one time' via \lq one more time' yields \lq again'. This interpretation is, of course, in line with adverbial phrases such as \ili{English} \textit{once more} or \ili{Spanish} \textit{otra vez} lit. \lq{}(an)other time'. And, indeed, out of the ten relevant sample expressions, nine have an additive function (\Cref{sectionAdditive}).\il{Kalaallisut|(} The one possible exception is Kalaallisut \textit{suli}, illustrated in (\ref{exIterativeIncrementGreenlandic}).

\begin{exe}
	\ex Kalaallisut\\
	 Context: Whitey has been told to take a closer look at a mare and her colt. He has circled them once and not noticed anything.\label{exIterativeIncrementGreenlandic}\\
	\gll \lq\lq Aju-quti-qar-nir-su-q taku-sinnaa-nngi-la-ra" Whitey uaqr-pu-q \textbf{suli} \textbf{ataasi}-\textbf{iar}-lu-ni histi arnaviaq pi-ara-a=lu kajalla-riar-llu-gut.\\
	\phantom{\lq\lq}be\_bad-cause-have-wonder-\textsc{ptcp}-3\textsc{sg} see-be\_able-\textsc{neg}-\textsc{ind}-1\textsc{sg}>3\textsc{sg} W. say-\textsc{ind}-3\textsc{sg} still one-do\_times-\textsc{ptcp}-3\textsc{sg} horse female do-little-3\textsc{sg}>3\textsc{sg}=and circle-after-\textsc{ptcp}-3\textsc{pl}\\
	\glt \lq {\lq\lq}I can’t see anything that’s wrong with him", said Whitey, after circling the mare and her colt \textbf{one} \textbf{more} \textbf{time}.\rq{ }\parencite[Hesti piaraq tappiitsoq]{BittnerTexts}
\end{exe}

The available data are not very conclusive when it comes to an additive function of \textit{suli}. However, \textcite[90]{FortescueEtAl1984} list \lq more' as a sense of Proto-Eskimo \mbox{*\textit{cu}(\textit{na})\textit{li}} and generalised additive functions are found, for instance with the cognate expression \textit{cali} in Central Alaskan Yupik\il{Yupik, Central Alaskan} \parencite{Miyaoka2012}. All this suggests that the Kalaallisut attestation in (\ref{exIterativeIncrementGreenlandic}) also builds on an additive function of \textit{suli}, if perhaps only in diachronic terms.\il{Kalaallisut|)}\is{repetition|)}\is{additive|)}


\subsection{\lq{}Always, all the time\rq}\label{sectionAlways}\is{always|(}\is{distributive|(}\is{frequentative|(}
\addtocontents{toc}{\protect\setcounter{tocdepth}{2}}
\subsubsection{Introduction}In this subsection I address uses involving universal temporal quantification (\lq always') and closely related notions, such as quantification over the entirety of some specific time span (\lq the whole time'), or frequentative (\lq all the time') and distributive 
(\lq every time\rq{}) event construals. Example (\ref{exAlwaysChuwabu}) is an illustration from \ili{Chuwabu}. 
\begin{exe}
	\ex \ili{Chuwabu}\label{exAlwaysChuwabu}\\
	Context: A certain man’s wife constantly falls sick.\\
	\gll 	\textbf{sabw’} \textbf{eelá} \textbf{w}-\textbf{aá}-\textbf{kála} \textbf{mu}-\textbf{reddá}=\textbf{ví} ábále, ábáálé éenâ á-á-ni-mu-nyapwaaríya.\\
	because \textsc{comp} \textsc{subj}.\textsc{ncl}1-\textsc{pst}.\textsc{ipfv}-\textsc{cop} \textsc{ncl}1-sick.\textsc{pl}=still \textsc{dist}.\textsc{ncl}2 \textsc{ncl}2.sister \textsc{ncl}2.other \textsc{subj}.\textsc{ncl}2-\textsc{pst}-\textsc{ipfv}-\textsc{obj}.\textsc{ncl}1-despise\\
	\glt \lq \textbf{because she was always sick}, those ones, the other sisters, despised her.' \parencite[608]{Guerois2015}
\end{exe}

\subsubsection{Distribution in the sample}\largerpage
\Cref{tableAlways} lists the expressions in my sample that have the \lq always, all the time\rq{ }use. As can be gathered, this function is attested for seven languages and expressions from all macro-areas minus South America. It is unclear whether the absence of examples from the latter continent is a true areal effect or merely an artefact of the data.

\begin{table}
	\small
	\caption{\lq Always, all the time'}
	\label{tableAlways}
		\begin{tabularx}{\textwidth}{llllQ}
		\lsptoprule
		M.-Area & Language & Expression & Appendix & Restriction\\\midrule
		Africa & \ili{Chuwabu} & =\textit{vi} & \ref{appendixChuwabuAlways} \\
		Australia & Jaminjung\il{Jaminjung}\il{Ngaliwurru} & =(C)\textit{ung}\footnote{Only one example in the data.} & \ref{exAppendixJaminjungAlways} & With secondary predicates \\
		& \ili{Gooniyandi} & =\textit{nyali}	& \ref{appendixGooniyandiAlways} &	 With secondary predicates, with \textit{ngambiddi} \lq again'\\
		& \ili{Gurindji} & =\textit{rni} & \ref{appendixGurindjiAlways}\\
		Eurasia & 	\ili{French} & \textit{toujours} & \ref{appendixFrenchToujoursAlways} & Borderline \textsc{still} expression\\
		N. America & \ili{Osage} & \textit{šó̜} & \ref{appendixOsageAlways} & Reduplicated forms plus \textit{ðe} \lq \textsc{prox}\rq{}\\
		Papunesia & \ili{Rapanui} & \textit{nō} &\ref{appendixRapanuiAlways}\\
		\lspbottomrule
		\end{tabularx}
\end{table}

Unfortunately, my sample obscures a well-established tendency in Eurasia. Thus, the case of \ili{French} \textit{toujours} finds a synchronic parallel in cases like \ili{Italian} \textit{sèmpre} \parencite[619]{GDLI}. What is more, \ili{English} \textit{still} and \ili{Spanish} \textit{todavía} both historically had an \lq always\rq{ }sense. In addition, many European languages make the the unexpectedly\is{expectations} late scenario of \textsc{still} explicit by augmenting a \textsc{still} expression with an \lq always\rq{ }item. In my sample, this is the case for \ili{German} \textit{noch} > \textit{immer noch}/\textit{noch immer} and for Serbian\hyp Croatian\hyp Bosnian\il{Serbian}\il{Croatian}\il{Bosnian} \textit{još} > \textit{još} \mbox{\textit{uv}(\textit{ij})\textit{ek}}. These instances, together with similar cases involving a generic universal quantifier (\lq all\rq{}), cover a contingent area in central, western and eastern Europe \parencite[85–90]{vanderAuwera1998}. In what follows, I take a closer look at the European context. Afterwards, I turn to the cases from African, Australian, and Papunesia cases and lastly to \ili{Osage} \textit{šó̜}. 

\subsubsection{A closer look and discussion: The European cases} 
To develop a better understanding of the relevant European expressions, it is worthwhile taking a diachronic perspective. In all cases that I am aware of, \lq always' and related notions historically precede phasal polarity \textsc{still}. What is more, such notions entail \isi{persistence} \parencite[149]{MosegaardHansen2008}, in that an interval leading up from an earlier time to the speaker's now\is{utterance time} (or another salient evaluation time) is a subset of any larger interval centred around it. With this in mind, valuable insights can be gained from the well-documented cases of \ili{French} \textit{toujours} and \il{Spanish|(}Spanish \textit{todavía}. \il{French, Old}\il{Spanish, Old}In \citeauthor{MosegaardHansen2008}'s (\citeyear{MosegaardHansen2008}) diachronic corpus study, the first attestation of \textit{toujours} as a marker of \isi{persistence} is found in (\ref{exAlwaysOldFrench}). Judging from the surrounding text, a distributive reading might remain latent, namely that the man in question is found with the same strength at each attempt of removing him. Similarly, \textcite{Bosque2016} observes that there are instances of \textit{todavía} in Old Spanish that can be understood both ways, as in (\ref{exAlwaysOldSpanish1}).\il{Spanish, Old}

\begin{exe}
\ex Old French,\il{French, Old} 13\textsuperscript{th} century\label{exAlwaysOldFrench}\is{persistence}\\
\textit{Si le troevent de tele force et de tele vistece que il ne cuident mie que il soit hons terriens …Si s’esmaient mout, car il voient que il nel pueent remuer de place,}\\
\lq Thus, they find him to have such strength and such speed that they do not believe that he is an earthly man …  So they become very fearful, for they see that they cannot remove him from his place'
\exi{}
\gll ainz le troevent {\textbf{tor} \textbf{jorz}} \textbf{d’}-\textbf{autel} \textbf{force} \textbf{come} \textbf{a}-\textbf{u} \textbf{comencement}.\\
\textsc{dem} 3\textsc{sg}.\textsc{m}.\textsc{acc} find.3\textsc{pl} always/still of-similar.\textsc{m} strength(\textsc{m}) like at-\textsc{def}.\textsc{sg}.\textsc{m} start.\textsc{m}\\
\glt \lq but find that \textbf{he still has as much strength as in the beginning}.'
\\(\textit{La Queste del Saint Graal}, cited in \cite[149]{MosegaardHansen2008},  glosses added) 

\ex Old Spanish,\il{Spanish, Old} 13\textsuperscript{th} century\label{exAlwaysOldSpanish1}\is{persistence}\\
\gll Por ende nos \textbf{d}-\textbf{amos} \textbf{gracias} \textbf{a} \textbf{Dios} {\textbf{toda} \textbf{uia}}.\\
for \textsc{dem} 1\textsc{pl} give-1\textsc{pl} thanks to God always/still\\
\glt \lq Therefore, we \textbf{always} \textbf{thank} \textbf{God} / \textbf{keep} \textbf{thanking} \textbf{God}.' (\textit{El Nuevo Testamento según el manuscrito escurialense I-j-6}, cited in \cite[209]{Bosque2016},  glosses added)
\end{exe}

Present-day \ili{French} \textit{toujours} remains a borderline case of a \textsc{still} expression in that it does not necessarily feature a \isi{prospective} component. No such qualification applies to Spanish \textit{todavía}.\il{Spanish, Middle}  For several centuries after developing its persistive\is{persistence} function, \textit{todavía} tended to be used together with expressions that defeasibly implicate a point of \isi{discontinuation} (\cite{Bosque2016}; \cite{Morera1999}). For instance, in (\ref{exAlwaysMiddleSpanish}) it is accompanied by \textit{fasta que se viene} \lq until he comes'. In the same vein, \textit{todavía} often went together with \textit{aún}, which was already established as \textsc{still} expression at that stage of the language; see (\ref{exAlwaysSpanish18th}). As \textcite{Bosque2016} argues, through a strong association with such collocations the evocation of an alternative scenario became a conventionalised component of \textit{todavía}'s own meaning, thereby transforming it into a \textit{bona fide} exponent of \textsc{still}.\il{Spanish, Middle} 

\begin{exe}
	\ex Middle Spanish,\il{Spanish, Middle} 15\textsuperscript{th} century\label{exAlwaysMiddleSpanish}\\
	\gll Quando el roque oy-e la boz d-el canto va-se contra alla \& el cantador cant-a \textbf{todavía} \textbf{fasta} \textbf{que} \textbf{se} \textbf{viene} \textbf{el} \textbf{roque}.\\
when \textsc{def}.\textsc{sg}.\textsc{m} roque(\textsc{m}) hear-3\textsc{sg} \textsc{def}.\textsc{sg}.\textsc{f} voice(\textsc{f}) of-\textsc{def}.\textsc{sg}.\textsc{m} song(\textsc{m}) go.\textsc{3sg}-\textsc{refl}.3 against there {} \textsc{def}.\textsc{sg}.\textsc{m} singer(\textsc{m}) sing-3\textsc{sg} still until \textsc{subord} \textsc{refl}.3 come.3\textsc{sg} \textsc{def}.\textsc{sg}.\textsc{m} roque(\textsc{m})\\
	\glt \lq When the roque [type of mythical beast] hears the voice of the song, he sets off in its direction, and the singer \textbf{keeps} \textbf{singing} \textbf{until} \textbf{the} \textbf{roque} \textbf{comes}.' (\textit{Libro de astrología}, cited in \cite[211]{Bosque2016},  glosses added)

	\ex Spanish, 18\textsuperscript{th} century\label{exAlwaysSpanish18th}\\
	\gll Pues \textbf{aun} \textbf{todavia} \textbf{se} \textbf{ven} en España much-o-s con esta propiedad.\\
	well still still \textsc{refl}.3 see.3\textsc{pl} in Spain many-\textsc{m}-\textsc{pl} with \textsc{prox}.\textsc{sg}.\textsc{f} property(\textsc{f})\\
	\glt \lq \textbf{You} \textbf{still} \textbf{see} many of that kind in Spain.\rq{ }(de Ulloa, \textit{Viaje al reino de Perú}, cited in \cite[213]{Bosque2016},  glosses added)
\end{exe}\il{Spanish|)}

In fact, the same development may be currently going on with \ili{French} \textit{toujours}. This item is often used jointly with the full-fledged \textsc{still} expression \textit{encore} (e.g. \cite[105]{MosegaardHansen2008}), as in (\ref{exAlwaysEncoreEtToujours}). This could ultimately lead to \textit{toujours} acquiring a \isi{prospective} meaning component, as well.

\begin{exe}
	\ex \ili{French}\label{exAlwaysEncoreEtToujours}\\
	\gll \textbf{Je} \textbf{suis} \textbf{encore} \textbf{et} \textbf{toujours} / après tant et tant d’-années / cet enfant qui tire sur une ficelle / à la poursuite d-u vent.\\
	1\textsc{sg} \textsc{cop}.1\textsc{sg} still and still/always {} after so\_many and so\_many of-year.\textsc{pl} {} \textsc{prox} child \textsc{rel} pull.3\textsc{sg} on \textsc{def}.\textsc{sg}.\textsc{f} twine(\textsc{f}) {} to \textsc{def}.\textsc{sg}.\textsc{f} pursuit(\textsc{f}) of-\textsc{def}.\textsc{sg}.\textsc{m} wind(\textsc{m})\\
	\glt \lq \textbf{I am still and ever} / after so many many years / that same child pulling a
string / in pursuit of the wind.\rq{ }(Soupault, \textit{Cerf-volant}, cited in \cite[105]{MosegaardHansen2008}, glosses added)
\end{exe}

\subsubsection{A closer look: The southern hemisphere cases}\il{Chuwabu|(} 
While \lq always, all the time' entails persistence,\is{persistence} this implication does not hold the other way around. Consequently, I am aware of no expression that unequivocally went from \textsc{still} to \lq always', at least not in a straight path. \textcite{Guerois2021} does suggest that such a development might have taken place with \mbox{=\textit{vi}} in the Bantu language \ili{Chuwabu}, but this requires the stipulation of an \lq\lq interpretive augmentation … \lq I am still walking' may be equivalent to saying \lq I am always walking'" \parencite[166]{Guerois2021} and which I find hard to follow. On the other hand, \mbox{=\textit{vi}}\il{Chuwabu} shares a common denominator with the Australian and Papunesian cases in \Cref{tableAlways} (Jaminjung-Ngaliwurru\il{Jaminjung}\il{Ngaliwurru} \mbox{(C)=\textit{ung}}, \ili{Gooniyandi} \mbox{=\textit{nyali}}, \ili{Gurindji} \mbox{=\textit{rni}}, \ili{Rapanui} \textit{nō}). All five expressions also serve as exclusive operators \lq only\rq{ }(\Cref{sectionExclusive}) and there is an overlap between \lq only\rq{ }and \lq always, all the time\rq{ }in cases where

\begin{quote}
the states envisaged … are mutually exclusive, i.e. they cannot occur simultaneously,\is{simultaneity} if two states were to occur during a period, they would occur at different times, and therefore neither of the states would occur uninterruptedly \lq all the time'. \parencite[28]{McConvell1983}
\end{quote}

\il{Gurindji|(}For instance, what is at stake in (\ref{exAlwaysGurindji}) is the addressee's communicative behaviour vis-à-vis the speakers. If understood as an exclusive operator, Gurindji \mbox{=\textit{rni}} signals that all contextually accessible propositions that differ in the denotation of \textit{pawu} \lq ignore [us]\rq{} are false. This set encompasses at minimum \lq listen (to us)', which is mentioned in the immediate discourse environment. If all the addressee does is to ignore the speakers, then he never engages in any other relevant behaviour, and vice versa.

\begin{exe}
	\ex Gurindji\label{exAlwaysGurindji}\\
	\gll Kula=n pura nya-ngku; ngu=n=ngantipa \textbf{pawu}=\textbf{rni} \textbf{pa}-\textbf{nana}.\\
	\textsc{neg}=\textsc{subj}.2\textsc{sg} hear see-\textsc{fut} \textsc{aux}=\textsc{subj}.\textsc{2sg}=\textsc{obj}.1\textsc{pl}.\textsc{excl} ignore=still hit-\textsc{prs}\\
	\glt \lq You can’t listen; you \textbf{always} \textbf{just} \textbf{ignore} us.' \parencite[20]{McConvell1983}
\end{exe}
\il{Gurindji|)}

\il{Jaminjung|(}\il{Ngaliwurru|(}With Jamjinjung-Ngaliwarru \mbox{=(C)\textit{ung}} the \lq always, all the time\rq{ }use is limited to secondary predicates, as in (\ref{exAlwaysJaminjungSecondaryPred}). With one minor exception, the same restriction applies to \ili{Gooniyandi} \mbox{=\textit{nyali}}. Both markers have a whole range of meanings when adjoined to a secondary predicate, all of which are compatible with their exclusive function (\cite[463–465]{McGregor1990}; \cite{SchultzeBerndt2002}).

\begin{exe}
	\ex Jamjinjung-Ngaliwarru\label{exAlwaysJaminjungSecondaryPred}\\
	\gll Ngarrgina=malang gujarding digirrij ga-jgany \textbf{bidimab}-\textbf{nyunga}=\textbf{wung} ngayug=gung nga-ngangarna-nyi mangarra nganja\sim nganjany.\\
	\textsc{poss}.1\textsc{sg}=\textsc{given} mother die 3\textsc{sg}-go.\textsc{pst} feed:\textsc{tr}-from=still 1\textsc{sg}=still 1\textsc{sg}>3\textsc{sg}-give.\textsc{redupl}-\textsc{ipfv} plant\_food \textsc{redupl}\sim what\\
	\glt \lq My mother passed away (\textbf{always}) \textbf{having} \textbf{been} \textbf{cared} for (lit. \lq\lq fed"), me I used to give her food (and) things.' \parencite[235]{SchultzeBerndt2002}
\end{exe}\il{Jaminjung|)}\il{Ngaliwurru|)}

In a nutshell, these data points suggest that the \lq always, all the time' use of the African, Australian and Papunesian sample expressions is not directly related to the same items as exponents of \textsc{still}. Instead, it  seems more plausible that the nexus between these functions lies in their additional function as exclusive operators. This interpretation, in turn, finds external support from \mbox{=\textit{gon}} in the Finisterre-Huon language Nungon,\il{Nungon} which is a polyfacetic exclusive marker that can also signal \lq all the time\rq{ }\parencite[417]{Sarvasy2017}, but which appears to not function as an exponent of phasal polarity.
\il{Chuwabu|)}

\subsubsection{A closer look: The case of Osage \textit{šó̜}}\il{Osage|(} 
There is one expression in \Cref{tableAlways} that I have not discussed yet, namely Osage \textit{šó̜}. This item signals \lq always\rq{ }in various complex forms, all of which involve its reduplication plus suffixation of the proximal demonstrative \textit{ðe}; see (\ref{exAlwaysOsage}). As in this example, the initial segment of \textit{ðe} is often weakened to a glide.

\begin{exe}
	\ex Osage\label{exAlwaysOsage}\\
	\gll \textbf{Šó̜ó̜}\sim{}\textbf{šó̜}-\textbf{we} nanió̜pa ðaašoé hta apa-i.\\
	\textsc{redupl}\sim{}still-\textsc{prox} pipe 3\textsc{sg}.\textsc{a}:smoke \textsc{fut} 3.\textsc{cont}-\textsc{decl}\\
	\glt \lq He will always smoke.\rq{ }\parencite[328]{Quintero2004}
\end{exe}

Aside from signalling phasal polarity and \lq always\rq{}, Osage \textit{šó̜} also functions as a marker of simultaneous duration\is{simultaneity} (\lq while\rq), which is a cross-linguistic recurrent use of \textsc{still} expressions; see \Cref{sectionSimultaneity}. In fact, this is the only other use in which \textit{šó̜} can merge with the demonstrative \textit{ðe} and where the latter morpheme's initial segment tends to undergo lenition (\cite[444]{Quintero2004}, \citeyear[209]{QuinteroDictionary}).\is{syntax|(} What is more, the two functions share a key syntactic feature. Thus, \textit{šó̜} as a marker of phasal polarity invariably stands between the predicate and the following continuative auxiliaries, as in (\ref{exAlwaysOsage2}), but the \lq always\rq{ }forms can precede the verb complex and thereby occupy the same position as a temporal clause.\is{temporal clause}\is{subordination} These morphological and syntactic characteristics unanimously point to the \lq always\rq{ }use of the reduplicated \mbox{\textit{šó̜}-\textit{we}} forms as building on the simultaneity function. Presumably, the underlying semantic pathway is along the lines of \lq during that time' > \lq during all times', which would be fully in line with the meaning of reduplication in Osage (see \cite[87, 444]{Quintero2004}).\is{simultaneity}

\begin{exe}
	\ex Osage\label{exAlwaysOsage2}\\
	\gll Híi ðáalí̜ \textbf{šó̜} ðaašé.\\
	tooth good still 2\textsc{sg}.\textsc{cont}\\
	\glt \lq Your teeth are still good.' (\cite[208]{QuinteroDictionary},  glosses added)
\end{exe}
\il{Osage|)} \is{syntax|)}\is{always|)}
\is{distributive|)}\is{frequentative|)}

\subsection{Prospective \lq eventually\rq}\label{sectionProspective}\is{prospective|(}
\subsubsection[tocentry={}]{Introduction}
In this subsection, I discuss a use in which a \textsc{still} expression signals that \lq\lq there is a certain development under way which (finally) leads to an event of the kind stated\rq\rq{ }(\cite[199–200]{Loebner1989}).\footnote{Similar characterisations are given by \textcite[173]{Ferrand1903}, \textcite[142]{Koenig1991}, \textcite{KoenigTraugott1982}, \textcite[617]{MetrichFaucher2009}, \textcite{Shetter1966} and \textcite{Vaelikangas1982}, among others.} Example (\ref{exProspectiveIntro}) is an illustration. Here, Plateau Malagasy\il{Malagasy, Plateau} \textit{mbola} signals that the predicted situation of the type \lq he comes\rq{ }will take place before the end of some contextually salient larger time span.
 
\begin{exe}
	\ex Plateau Malagasy\il{Malagasy, Plateau}\label{exProspectiveIntro}\\
	\gll \textbf{Mbola} ho avy.\\
	still \textsc{fut} come\\
	\glt \lq Il doit venir dans un délai plus ou moins long. [He should come sooner or later.]' (\cite[173]{Ferrand1903}, glosses added)
\end{exe}

\subsubsection[tocentry={}]{Distribution in the sample}
\Cref{tableEventually} lists the languages and expressions for which the prospective \lq eventually\rq{ }function is attested. As can be gathered, fourteen expressions and languages have this use. These stem primarily from Africa and Eurasia, plus one instance each from Papunesia and South America.\footnote{Outside of my sample, the function is found with \ili{Hungarian} \textit{még} \parencite[s.v. \textit{még}]{BarcziOrszagh1992} and Finish \textit{vielä} \parencite[141]{Koenig1991}, as well as with cognates of several sample expressions, such as \ili{Dutch} \textit{nog} (\cite{Vandeweghe1984}; \cite[142]{Koenig1991}), the congeners of Serbian\hyp Croatian\hyp Bosnian\il{Serbian}\il{Croatian}\il{Bosnian} \textit{još} across Slavic languages (e.g. \cite{Bogacki1989}; \cite[142]{Koenig1991}) and the cognates of \ili{Kaba} \textit{bbáy} in other Sara languages (e.g. \cite[244]{Palayer1989}, \citeyear[167]{Palayer1992}; examples throughout \cite{Keegan2014}; \cite{Thayer1978}; \cite{Vandame1963}).} With the limitations of the sample data in mind, this suggests some degree of areality to the phenomenon. 

\begin{table}[bt]
	\caption{Prospective \lq eventually'}\label{tableEventually}
	\begin{tabular}{llll}
		\lsptoprule
		Macro-area&Language&Expression & Appendix\\\midrule
		Africa & \ili{Kaba} & \textit{bbáy} & \ref{appendixKabaProspective}\\
		 & \ili{Mundang} & \textit{ɓà} & \ref{appendixMundangProspective} \\
	 	 & Plateau Malagasy\il{Malagasy, Plateau} & \textit{mbola} &  \ref{appendixMalagasyProspective}\\
  		 & Southern Ndebele\il{Ndebele, Southern} & \textit{sa}- & \ref{appendixSouthernNdebeleProspective}\\
		 & \ili{Tashelhyit} & \textit{sul} & \ref{appendixTashelhyitProspective} \\
		 & \ili{Tima} & \textit{bʌʌr} & \ref{appendixTimaProspective}\\
		 & \ili{Xhosa} & \textit{sa}- & \ref{appendixXhosaProspective}\\
		Eurasia & \ili{French} & \textit{encore} &  \ref{appendixFrenchEncoreProspective}  \\
		& \ili{German} & \textit{noch} & \ref{appendixGermanProspective} \\
		& Hebrew (Modern)\il{Hebrew, Modern} & \textit{ʕod}  & \ref{appendixHebrewOdProspective}\\
		& Serbian-Croatian-Bosnian\il{Serbian}\il{Croatian}\il{Bosnian} & \textit{još} &\ref{appendixBCMSProspective}  \\
		& \ili{Spanish} & \textit{todavía} & \ref{appendixSpanishTodaviaEventually}\\
		Papunesia & \ili{Saisiyat} & \textit{nahan} & \ref{appendixSaisiyatProspective}\\
		South America & Southern Lengua & \textit{makham} & \ref{appendixEnxetSurProspective}\\
		\lspbottomrule	
	\end{tabular}
\end{table}	

\subsubsection[tocentry={}]{A closer look}\is{tense|(}
It comes as little surprise that the marking of prospective \lq eventually\rq{ }is most commonly attested in combination with future tenses and prospective aspects,\is{aspect} as in the initial example (\ref{exProspectiveIntro}). However, the same function also occurs with other tense-aspect paradigms that can point forward in time,\il{Spanish|(} like the Spanish present-as-future in (\ref{exProspectiveSpanishFuturate}). Similarly, one finds the occasional instance of other non-actualised contexts,\il{Saisiyat|(} such as the imperative in (\ref{exProspectiveSaisiyat}), or the purpose clause in (\ref{exProspectiveGermanNeg}) below.

\begin{exe}
	\ex Spanish\label{exProspectiveSpanishFuturate}\\
	\gll ¿A que \textbf{todavía} \textbf{termin}-\textbf{o}?\\
	\phantom{¿}to \textsc{subord} still finish-1\textsc{sg}\\
	\glt \lq I bet you that I \textbf{will finish} [\textbf{yet}].\rq{}
	\\(\cite[382 fn 30]{Garrido1992}, glosses added)
	\il{Spanish|)}
	\ex Saisiyat\label{exProspectiveSaisiyat}\\
	\gll Maʼan ka-obaang-an no<m>obaang \rq{}okik lalʼoz. \textbf{Rimaʼ} \textbf{baeiw} \textbf{naehan}!\\
	1\textsc{sg}.\textsc{gen} \textsc{rl}-draw-\textsc{loc}.\textsc{foc} \textsc{inst}<\textsc{agt}.\textsc{foc}>draw \textsc{neg}.\textsc{lnk}.\textsc{stat} enough go.\textsc{imp} buy.\textsc{imp} still\\
	\glt \lq I do not have enough paper and pens; \textbf{go and buy some later}!\rq
	\\\parencite[506]{ZeitounEtal2015}
\end{exe}\il{Saisiyat|)}

\is{mood|(}\ili{Mundang} \textit{ɓà} is somewhat of an outlier in that its \lq eventually\rq{ }function has been described as being restricted to the optative and potential mood inflections \parencite[382]{Elders2000}; example (\ref{exProspectiveMundang}) is an illustration. This is despite Mundang having dedicated future tense and prospective \isi{aspect} paradigms. 
\begin{exe}
	\ex\label{exProspectiveMundang}\ili{Mundang}\\
	\gll ʔà fʊ̄ō \textbf{ɓà}.\\
\textsc{subj}.3:\textsc{ipfv} think.\textsc{pot} still\\
	\glt \lq Il pensera un jour. [He will think \textbf{one} \textbf{day}.]' \parencite[382]{Elders2000}
\end{exe}\is{tense|)}\is{mood|)}

\is{actionality|(}
When it comes to the situation's internal make-up, \lq eventually\rq{ }is less selective than phasal polarity \textsc{still}, because the situation in question need not actually persist\is{persistence} in time. Examples (\ref{exProspectiveMalagasy}, \ref{exProspectiveBCMS}) illustrate this use in combination with an achievement predicate.

\begin{exe}
	\ex Plateau Malagasy\il{Malagasy, Plateau}\label{exProspectiveMalagasy}\\
	\gll \textbf{Mbola} ho avy.\\
	still \textsc{fut} come\\
	\glt \lq Il doit venir dans un délai plus ou moins long. [He should come \textbf{sooner or later}.]' (\cite[173]{Ferrand1903}, glosses added)
	
	\ex Serbian-Croatian-Bosnian\il{Serbian}\il{Croatian}\il{Bosnian}\label{exProspectiveBCMS}\\
	Context: At a sports event. Our team is lying behind.\\
	\gll Pobedi-ćemo mi \textbf{još}.\\
	win.\textsc{pfv}-\textsc{fut}.1\textsc{pl} 1\textsc{pl} still\\
	\glt \lq We'll win \textbf{yet}.\rq{ }(Stefan Savić, p.c.)
\end{exe}
\is{actionality|)}

\il{Hebrew, Modern|(}The two functions may also differ when it comes to their interaction with negation.\is{negation} Thus, when Modern Hebrew \textit{ʕod} in its phasal polarity function is paired with a negated predicate, this predictably yields \textsc{not yet}.\is{not yet} In the \lq eventually\rq{ }use, on the other hand, this constellation gives rise to a prediction of a negative state \parencite{FrancezOd}. This is shown in (\ref{exProspectiveHebrew}).

\begin{exe}
	\ex Modern Hebrew \label{exProspectiveHebrew}\is{negation}\\
	Context: About failures to deal with sexual harassment in organised sport. The writer is being sarcastic.\\
		\gll Im ze yi-mašex kaxa, b-a-sof \textbf{ʕod} \textbf{lo} \textbf{ti}-\textbf{hye} \textbf{la}-\textbf{hem} \textbf{brera} ela le-manot  iša l-a-tafqid.\\
	if \textsc{prox}.\textsc{sg}.\textsc{m} 3\textsc{sg}.\textsc{m}-continue.\textsc{fut} thus at-\textsc{def}-end still \textsc{neg} 3\textsc{sg}.\textsc{f}-\textsc{cop}.\textsc{fut} to-3\textsc{pl}.\textsc{m} choice(\textsc{f}) except to-appoint woman to-\textsc{def}-position\\
	\glt \lq If it goes on like this, in the end  \textbf{they will have no choice} but to appoint a woman to the position.\rq{ }(online example, cited in \cite{FrancezOd})
\end{exe}\il{Hebrew, Modern|)}

\il{Xhosa|(}
In (\ref{exProspectiveHebrew}) \isi{negation} falls under the scope of \textit{ʕod}. When, on the other hand, negation takes wide scope, this denies any future occurrence of the type described by the predicate \parencite{Loebner1989}. This is illustrated in (\ref{exProspectiveGermanNeg}, \ref{exProspectiveXhosa}).

\begin{exe}
	\ex \ili{German} \label{exProspectiveGermanNeg}\\
	Context: About a rescue exercise.\is{negation}\\
	\gll Damit der Darsteller des Unfall-opfer-s bei-m Wart-en … \textbf{nicht} \textbf{noch} \textbf{krank} \textbf{wird}, bekomm-t er eine Wärme-folie mit auf den Weg.	\\
	\textsc{purp} \textsc{def}.\textsc{nom}.\textsc{sg}.\textsc{m} impersonator(\textsc{m}) \textsc{def}.\textsc{gen}.\textsc{sg}.\textsc{n} accident-victim(\textsc{n})-\textsc{gen} at-\textsc{def}.\textsc{dat}.\textsc{sg}.\textsc{n} wait-\textsc{inf}(\textsc{n}) {}
	 \textsc{neg} still ill become.3\textsc{sg} get-3\textsc{sg} 3\textsc{sg}.\textsc{m} \textsc{indef}.\textsc{acc}.\textsc{sg}.\textsc{f} warmth-foil(\textsc{f}) with on \textsc{def}.\textsc{acc}.\textsc{sg}.\textsc{m} way(\textsc{m})\\
	\glt \lq The actor impersonating the victim is given a thermal foil, so he \textbf{doesn't end up sick} while waiting.\rq{ }(found online, glosses added)%\footnote{\url{https://www.donaukurier.de/archiv/alle-sechs-unfallopfer-bestens-versorgt-4210654} (22 March, 2023).}
	
	\ex Xhosa\label{exProspectiveXhosa}\is{negation}\\
	\gll Um-khosi \textbf{a}-\textbf{wu}-\textbf{sa}-\textbf{y}-\textbf{i} \textbf{k}-oyis-wa.\\
	\textsc{ncl}3-army \textsc{neg}-\textsc{subj}.\textsc{ncl}3-still-go-\textsc{neg} \textsc{ncl}15(\textsc{inf})-defeat-\textsc{pass}\\
	\glt \lq The army \textbf{will never} be defeated.\rq{ }(\cite[132]{McLaren1936}, glosses added)
\end{exe}
\il{Xhosa|)}

In terms of its larger sentential embedding, it is not uncommon for the use I describe here to be accompanied by another expression with a similar meaning. For instance, in (\ref{exProspectiveHebrew}) it goes together with \textit{basof} \lq in the end\rq{}. Closely related, with \ili{German} \textit{noch} the \lq eventually\rq{ }reading can be enforced through items like \textit{schon} \lq no doubt\lq{}, lit. \lq{}already\rq{ }in the same clause; see (\ref{exProspectiveGerman1}).

\begin{exe}
	\ex \ili{German}\label{exProspectiveGerman1}\\
	\gll Die Männer sind eben stark verunsichert, aber die werd-en-'s \textbf{schon} \textbf{noch} lern-en.\\
	\textsc{def}.\textsc{nom}.\textsc{pl} man.\textsc{pl} \textsc{cop}.3\textsc{pl} just strongly unnerved but 3\textsc{pl} \textsc{fut}.\textsc{aux}-3\textsc{pl}-3\textsc{sg}.\textsc{acc}.\textsc{n} already still learn-\textsc{inf}\\
	\glt \lq The men are just very unsettled, but \textbf{no doubt eventually} they'll learn it.\rq{ }(\cite[122]{Thurmair1989}, glosses added)
\end{exe}

\subsubsection[tocentry={}]{Discussion}\is{topic time|(} 
Having taken a closer look at the prospective \lq eventually\rq{ }use, the question arises what motivates its recurrent co-encoding with phasal polarity \textsc{still}. As pointed out before me by \textcite{Loebner1989}, \textcite[142–143]{Koenig1991} and \textcite{FrancezOd}, the two functions share a core of \lq\lq antiterminativity\rq\rq{ }\parencite{Vandeweghe1984}. That is to say, just as the phasal polarity concept signals that the conceivable end of a situation has not been reached at the time under discussion, the use I discuss here marks that a situation will take place \lq\lq before all is over\rq\rq{ }\parencite[s.v. \textit{yet}]{OED2022}. \Cref{figureEventually} is a graphic comparison, using a \isi{perfective} viewpoint\is{aspect} for ease of illustration.
 
\begin{figure}
	\centering
	\begin{subfigure}[b]{0.48\linewidth}\is{perfective}
		\centering
		\begin{tikzpicture}[node distance = 0pt]
			\node[mynode, text width=\breit, fill=cyan, very near start] (spain){Situation};
			\node[mynode, fill=none, text width=2*\hoehe, right= of spain] (france){\small{$\Diamond$}\neg Sit.};	
			\draw[-, densely dashed] ($(spain.north east)+ (0,0) $) to ($(spain.south east)+ (0,-0.5*\hoehe) $);
			\draw[-, densely dashed] ($(spain.north east)+ (-1*\hoehe,0) $) to ($(spain.south east)+ (-1*\hoehe,-0.5*\hoehe) $) node [below, align=center, label distance=0, xshift=0.5*\hoehe] {\strut{}Topic\\time\strut};
			\draw[-latex, line width=0.5pt]  (spain.south west) to  ($(france.south east)+(1ex,0)$) node [right] {t};
			\draw[-latex, line width=0.5pt]  (spain.south west) to  ($(spain.south east)+(-\hoehe,0)$);
			\node[align=left,label distance=0, anchor=north west] at ($(spain.south west)+(0,-0.5*\hoehe)$) {\strut{}Prior\\runtime\strut};		
		\end{tikzpicture}
		\subcaption{\textsc{still}}
	\end{subfigure}
	\begin{subfigure}[b]{0.48\linewidth}
		\centering

			\begin{tikzpicture}[node distance = 0pt]
				\node[mynode, text width=7*\hoehe, fill=cyan, very near start, fill opacity=0.5, text opacity=1] (spain){Larger phase};
			\node[mynode, fill=cyan, text width=1.5*\hoehe, anchor=east, xshift=-0.5*\hoehe] (france)
		at (spain.east)	{Sit.};	
			\draw[-, densely dashed] ($(spain.north east)+ (-0.2pt,0) $) to ($(spain.south east)+ (-0.2pt,-0.5*\hoehe) $);
			\draw[-, densely dashed] ($(spain.north east)+ (-1*\hoehe+2pt,0) $) to ($(spain.south east)+ (-1*\hoehe+2pt,-0.5*\hoehe) $) node 	[below, align=center, label distance=0, xshift=0.5*\hoehe] {\strut{}Topic\\time\strut};
			\draw[thick] (france.north east) to (france.south east);
			\draw[-] (france.north west) to (france.south west);
			\draw[-latex, line width=0.5pt]  (spain.south west) to  ($(spain.south east)+(3ex,0)$) node [right] {t};
\draw[-latex, line width=0.5pt]  (spain.south west) to (france.south west);
\node[align=left,label distance=0, anchor=north west] at ($(spain.south west)+(0,-0.5*\hoehe)$) {\strut{}Prior course\\of events\strut};			
		\end{tikzpicture}
		\subcaption{Prospective \lq{}eventually\rq{ }(with \textsc{pfv})}
	\end{subfigure}
		\caption{Schematic illustration of prospective \lq eventually\rq{}	\label{figureEventually}}
\end{figure}\is{topic time|)}

As far as a more material nexus between the two functions is concerned, I cannot disregard the possibility that the two functions may be linked together by a common source, similar to how the \textsc{still} expression \textit{baʕd} in Levantine Arabic\il{Arabic, Levantine} is from a lexeme \lq behind, after\rq{ }\parencite{TaineCheikh2016}. However, for around half of the items in \Cref{tableEventually}, the known etymologies and/or the comparative evidence strongly point to a primacy of phasal polarity \textsc{still}. Where diachronic investigations are available, they indicate that prospective \lq eventually\rq{ }can arise very early on. Thus, \textcite[146–147]{MosegaardHansen2008} observes that relevant attestations of \ili{French} \textit{encore} are found just a century after the same item as an exponent of \textsc{still}. An example is given in (\ref{exProspectiveOldFrench}).\il{French, Old} Similarly, in the case of \ili{English} \textit{yet}, which once used to be a full-fledged \textsc{still} expression, the \lq eventually\rq{ }function is documented since early stages of the Old English\il{English, Old} period \parencite{KoenigTraugott1982}; see (\ref{exProspectiveOldEnglish}). Assuming the two cases are representative, they suggest some readily available contexts or constellations that trigger the development of the prospective function.

\begin{exe}
	\ex Old French,\il{French, Old} ca. 1150\label{exProspectiveOldFrench}\\
	\gll La vielle dist: \lq\lq Or entend-ez // et que ce est si devinn-ez; // \textbf{encor} \textbf{vous} \textbf{fera} \textbf{touz} \textbf{iriez}"\\
	\textsc{def}.\textsc{sg}.\textsc{f} old.\textsc{f} say.\textsc{pst}.\textsc{pfv}.3\textsc{sg} \phantom{\lq\lq}now listen-2\textsc{pl} {} and \textsc{comp} \textsc{prox}.\textsc{sg}.\textsc{m} \textsc{cop}.3\textsc{sg} then guess-2.\textsc{pl} {} still 2\textsc{pl}.\textsc{acc} make.\textsc{fut}.3\textsc{sg} all.\textsc{pl} angry.\textsc{pl}\\
	\glt \lq The old woman says \lq\lq Now listen // and then guess what it is // \textbf{it’ll drive you all mad yet!}{"}\rq{} (\textit{Le roman de Thèbes}, cited in \cite[146]{MosegaardHansen2008},  glosses added)
	
	\ex Old English\il{English, Old} (ca. 885)\label{exProspectiveOldEnglish}\\
	\gll \textbf{Giet} \textbf{cymð} se micla \& se mæra \& se egeslica God-es dæg.\\
yet come.3\textsc{sg} \textsc{dem}.\textsc{prox}.\textsc{sg}.\textsc{m} much.\textsc{nom}.\textsc{sg}.\textsc{m} {} \textsc{dem}.\textsc{prox}.\textsc{sg}.\textsc{m} famous.\textsc{nom}.\textsc{sg}.\textsc{m}  {} \textsc{dem}.\textsc{prox}.\textsc{sg}.\textsc{m} terrible.\textsc{nom}.\textsc{sg}.\textsc{m} God-\textsc{gen} day(\textsc{m}).\textsc{nom}.\textsc{sg}\\
	\glt \lq \textbf{There} \textbf{will} \textbf{come} \textbf{(}\textbf{lit.} \textbf{still}/\textbf{yet} \textbf{comes}\textbf{)} the great and famous and terrible day of God.\rq{ }(King Ælfred, \textit{Cura Pastoralis}, cited in \cite[s.v. \textit{yet}]{OED2022},  glosses added)
\end{exe}

Against this backdrop, the pertinent literature has three interpretations on offer. In what follows, I briefly summarise each of them and discuss how they hold up against the sample data. I then conclude that the most likely source of the \lq eventually\rq{ }use is found in contexts where a \textsc{still} expression takes scope over a modal\is{modality} or \isi{prospective} operator.

\is{additive|(}According to the first proposal, most prominently made by \textcite{KoenigTraugott1982}, phasal polarity \textsc{still} and prospective \lq eventually\rq{ }are connected only indirectly.\il{German|(} What supposedly links them together are additive functions of the same expressions, such as the one illustrated for German \textit{noch} in (\ref{exProspectiveGermanBier}). This idea is based on the observation that additive uses of \textsc{still} expressions tend to  go together with notions of \lq\lq adding up to a larger whole\rq\rq{ }\parencite[146]{Koenig1991}; see \Cref{sectionAdditive} for more discussion of this observation.

\begin{exe}
	\ex German\label{exProspectiveGermanBier}\\
	Context: After having consumed other drinks.\\
	\gll Ich trink-e (auch) \textbf{noch} ein Bier.\\
	1\textsc{sg} drink-1\textsc{sg} \phantom{(}also still \textsc{indef}.\textsc{acc}.\textsc{sg}.\textsc{m} 	beer(\textsc{n})\\
	\glt \lq I will have a beer, too.\rq{ }(\cite[143]{Koenig1991}, glosses added)
\end{exe}\il{German|)}

As far as the sample expressions are concerned, additivity as the key could yield a motivated explanation as to why in Modern Hebrew\il{Hebrew, Modern} the \lq eventually\rq{ }use is found with \textit{ʕod}, which has a range of additive functions, but not with its near-synonym \textit{ʕadayin}, which lacks them.\footnote{This begs the question of relative chronology, though.} What speaks against this interpretation are the cases of \ili{Mundang} \textit{ɓà} and \mbox{\textit{sa}-} in Southern Ndebele\il{Ndebele, Southern} and Xhosa,\il{Xhosa} which are not attested in a general additive function.\footnote{\ili{Mundang} \textit{ɓà} does have a \isi{sequencing} \lq and then\rq{ }use that involves temporal additivity,\is{additive} but the data suggest that the latter goes back to the use I discuss here, rather than the other way around (\Cref{sectionEventSequencing}).} That is to say, additive functions are not a prerequisite for the \lq{}eventually\rq{ }use. In \citeauthor{KoenigTraugott1982}'s defence though, it is conceivable that they do provide some degree of supplementary motivation in certain cases. For instance, in the \ili{Spanish} example (\ref{exProspectiveSpanish}) below, there is a lingering notion of \lq (if it weren’t bad enough as is) on top of that …\rq{}.\is{additive|)}

\is{persistence|(}\is{modality|(} 
The second approach has it that prospective \lq eventually\rq{ }is essentially no different from phasal polarity \textsc{still}. Instead, it just involves a wider scope, such that \lq\lq T[opic] T[ime]\is{topic time} … is … not the unspecified future time … but rather the time of utterance.\is{utterance time} It is then not the as-yet-unrealised SoA itself that remains in force during an extended interval, but the prediction of its eventual realisation.\rq\rq{ }\parencite[147]{MosegaardHansen2008}. At first glance, this is an elegant interpretation insofar as it requires only a simple, well-established mechanism and little change in meaning. What is more, it is tempting to understand the case of \ili{Mundang} \textit{ɓà} as an argument in favour of it.\is{mood|(} As I laid out above, this item has the relevant use only in combination with the potential and optative mood inflections, which suggests a fully compositional analysis. However, the relevant attestations do not necessarily have  any modal meaning components other than those inherent to any future prediction:

\begin{quote}
the construction signals an event that is not actualised at the time of speech\is{utterance time} but that will be actualised at some time in the future. A statement with the tardative {[}marker \textit{ɓà}{]} and a verb in the non-actualised forms translates into a future … in Cameroonian French it is often translated as \textit{après} \lq later\rq{}. \parencite[382]{Elders2000}\footnote{In the original French, \lq\lq{}la construction indique un événement qui n’est pas actualisé au moment d’énonciation mais qui sera actualisé dans le futur après une période de temps. Un énoncé avec le Tardatif et un verbe non-actualisé se traduit par un future … se traduit souvent en français camerounais par \lq{}après\rq{}{\rq\rq}.}
\end{quote}\il{Mundang}\is{mood|)}

What is more, an analysis involving a persistent prediction is hard to conciliate with the discursive embedding of many examples from other languages. For instance,\il{Spanish|(} in the context of (\ref{exProspectiveSpanish}) it is hard to see how a prediction of accusation previously figures in the common ground.\il{Saisiyat|(} Likewise, in (\ref{exProspectiveSaisiyat}), repeated below, it would be odd to assume a prior obligation to go and buy pens.

\begin{exe}
	\ex Spanish\label{exProspectiveSpanish}\\
	 Context: There's an unconscious girl with bloodstains on her face and neck.\\
	 \gll Tienes que llam-ar a la doctor-a … no puedes carg-ar con la responsabilidad de que se muera. \textbf{Todavía} \textbf{dir}-\textbf{án} \textbf{que} \textbf{eres} \textbf{cómplice}.\\
	have.2\textsc{sg} \textsc{subord} call-\textsc{inf} \textsc{acc} \textsc{def}.\textsc{sg}.\textsc{f} doctor-\textsc{f} {}
	\textsc{neg} can.2\textsc{sg} load-\textsc{inf} with \textsc{def}.\textsc{sg}.\textsc{f} responsibility(\textsc{f}) of \textsc{subord} \textsc{refl}.3 die.\textsc{sbjv}.3\textsc{sg} still say.\textsc{fut}-3\textsc{pl} \textsc{subord} \textsc{cop}.2\textsc{sg} accomplice\\
	\glt \lq You have to call the doctor … You can't take responsibility for her dying. \textbf{They'll end up saying you're an accomplice}.\rq{ }(CORPES XXI, glosses added)\il{Spanish|)}
	\exr{exProspectiveSaisiyat} Saisiyat\\
	\gll Maʼan ka-obaang-an no<m>obaang \rq{}okik lalʼoz. \textbf{Rimaʼ} \textbf{baeiw} \textbf{naehan}!\\
	1\textsc{sg}.\textsc{gen} \textsc{rl}-draw-\textsc{loc}.\textsc{foc} \textsc{inst}<\textsc{agt}.\textsc{foc}>draw \textsc{neg}.\textsc{lnk}.\textsc{stat} enough go.\textsc{imp} buy.\textsc{imp} still\\
	\glt \lq I do not have enough paper and pens; \textbf{go and buy some later}!\rq
	\\\parencite[506]{ZeitounEtal2015}
\end{exe}\il{Saisiyat|)}

In the same vein, \textcite[141–150]{Koenig1991} discusses several subtle, yet crucial, differences between \ili{English} \textit{still} and prospective \textit{yet} in the context of modal operators (recall that \textit{yet} started out as a \textsc{still} expression). Consider the examples in (\ref{exProspectiveEnglish}). \textit{Still} in (\ref{exProspectiveEnglish1}) yields a persistent obligation regarding a specific situation, and therefore allows for a referential reading of the object NP. With \textit{yet} in (\ref{exProspectiveEnglish2}), on the other hand, the \isi{necessity} pertains to any event of the type \lq meet a generous Scotsman\rq{ }that may occur in the future. If the \lq eventually\rq{ }reading of \textit{yet} were due to the adverb taking wide scope, one would expect the two sentences to share the same interpretation.\il{English}

\begin{exe}
	\ex \ili{English}\label{exProspectiveEnglish}
	\begin{xlist}
		\ex \textit{\textbf{I still have to meet} a generous Scotsman} (\textit{namely Bill Stewart}).\label{exProspectiveEnglish1}
		\ex \textit{\textbf{I have yet to meet} a generous Scotsman} (\textit{before I believe that such people exist}).\label{exProspectiveEnglish2}
\parencite[149–150]{Koenig1991}
\end{xlist}
\end{exe}

\is{necessity|(}\is{possibility|(}
In sum, the wide-scope proposal cannot be maintained as a synchronic analysis. It does, however, contain an important pointer as to how prospective \lq eventually\rq{ }can arise in the first place (as suggested before me by \cite{Abraham1977}). Thus, a motivated source for this use can be found in contexts in which phasal polarity \textsc{still} associates with (i.e. has scope over) a modal or prospective proposition. Not only are such combinations likely to arise in discourse. Crucially, the persistent, though perhaps not ever-lasting, prediction of some situation unilaterally entails that a situation of that type (possibly or necessarily) becomes a historic reality before the end of the same time span. This can easily open the doors to structural ambiguity and ultimately to a reanalysis of the contribution of the phasal polarity expression as one akin to a temporal frame adverbial indicating a time before the end of some salient larger interval (i.e. before it is too late). And, in fact, the historical record of \ili{French} has the example (\ref{exProspectiveOldFrench2}) on offer, which predates the more clear-cut instances of prospective \lq eventually\rq{ }by several decades \parencite[146 fn 12]{MosegaardHansen2008}.\il{French, Old}

\begin{exe}
		\ex Old French,\il{French, Old}\il{French} ca. 1080\label{exProspectiveOldFrench2}\\
		Context: About converting to Christianism.\\
		\gll Charles respunt: \lq\lq{}\textbf{Uncore} \textbf{purrat} \textbf{guar}-\textbf{ir}.\rq\rq\\
		C. reply.3\textsc{sg} \phantom{\lq\lq}still can.\textsc{fut}.3\textsc{sg} save-\textsc{inf}\\
		\glt \lq Charles replies, \lq\lq \textbf{He may yet}\textbf{/still be saved}.{\rq\rq}\rq{ }(\textit{La chanson de Roland}, cited in \cite[146 fn 12]{MosegaardHansen2008}, glosses added)
\end{exe} 

\il{German|(}In the same vein, \textcite{Shetter1966} points out that similarly ambiguous instances are frequently encountered in present-day German. This is illustrated in (\ref{exProspectiveGermanBridge}), featuring the \textsc{still} expression \textit{noch} and a periphrastic necessity construction. To all appearances, a comparable case is found in the \ili{Kaba} example (\ref{exProspectiveKaba}). As far as the speaker's communicative goals are concerned, the difference between the two construals is subtle, at best.

\begin{exe}
	\ex German \label{exProspectiveGermanBridge}\\
	\gll Trotz der schön-en selbst-gezogen-en Perinette- und Grand-Richard-Äpfel, \textbf{die} \textbf{noch} \textbf{zu} \textbf{prüf}-\textbf{en} \textbf{waren}, a-m Nach-mittag war ich davon-geritten.\\
	despite \textsc{def}.\textsc{gen}.\textsc{pl} beautiful-\textsc{gen}.\textsc{pl} self-grown-\textsc{pl} P. and G.-R.-apple.\textsc{pl} \textsc{rel}.\textsc{nom}.\textsc{pl} still to check-\textsc{inf} \textsc{cop}.\textsc{pst}.3\textsc{pl} at-\textsc{def}.\textsc{dat}.\textsc{sg}.\textsc{m} after-noon(\textsc{m}) \textsc{cop}.\textsc{pst}.1\textsc{sg} 1\textsc{sg} off-ride.\textsc{ptcp}\\
	\glt \lq Despite the fine home-grown Perinette and Grand Richard apples \textbf{that were still to be tasted/to be tasted at some point}, come afternoon I had set off.' (Storm, \textit{Der Schimmelreiter},  glosses added)
	\il{German|)}
	\ex \ili{Kaba}\label{exProspectiveKaba}\\
	\gll Àkàá kə̀sə nàrɛ̀ làá \textbf{nàyn} \textbf{tà} \textbf{kàrə} \textbf{ń}-\textbf{tél} \textbf{né} \textbf{d}-\textbf{ár}-\textbf{ɛ́} \textbf{bbay}.\\
	but leftover money \textsc{gen}.3\textsc{sg} stay \textsc{purp} \textsc{caus} \textsc{subj}.3\textsc{sg}-return \textsc{anaph} \textsc{subj}.3\textsc{pl}-\textsc{dat}.\textsc{aux}-\textsc{obj}.3\textsc{sg} still\\
	\glt \lq But his change \textbf{remained to be returned to him yet}.' \parencite[356]{Moser2004}
\end{exe}

While (\ref{exProspectiveOldFrench2}, \ref{exProspectiveGermanBridge}) feature auxiliary verb constructions and (\ref{exProspectiveKaba}) involves a purpose clause, there is no reason to assume that the same reinterpretation cannot apply to  more strongly grammaticalised\is{grammaticalisation}  contexts, and the above-mentioned restrictions of \ili{Mundang} \textit{ɓà} to certain modal\is{mood} paradigms may simply be a reflection of this function's history.\is{necessity|)}\is{possibility|)}\is{prospective|)}\is{persistence|)} \is{modality|)}

\subsection{Uses relating to sequentiality}\label{sectionTaxis}
\addtocontents{toc}{\protect\setcounter{tocdepth}{3}}
\subsubsection{\lq{}First, for now\rq}\label{sectionFirst}\is{precedence|(}
\subsubsubsection{Introduction} In this subsection, I discuss a use of \textsc{still} expressions that marks precedence. This can be in a strictly sequential sense, with the situation depicted in the clause constituting the initial step in a series;\il{Kalamang|(} an illustration is given in (\ref{exFirstKalamang}), where Kalamang \textit{tok} depicts the prayer as the first act of the day. Alternatively, the notion of precedence can pertain to a preliminary situation that may be superseded later. This is shown in (\ref{exFirstJaphug}), where \ili{Japhug} \textit{pɤjkʰu} contrasts the time for which the prompt to take a break is valid with a subsequent time span at which work will be resumed.

\begin{exe}
	 \ex Kalamang\label{exFirstKalamang}\\
	  Context: The protagonists have decided to set out for a trip.\\
	\gll Go yuol=ta me, wa me o, hari sabtu <me tok>, \textbf{mu} \textbf{tok} \textbf{doa} \textbf{salamar}=\textbf{at} \textbf{paruo}, fibir-un.\\
	condition day=\textsc{non}.\textsc{fin} \textsc{top}  \textsc{prox} \textsc{top} \textsc{emph} day saturday \phantom{<}\textsc{top} still, 3\textsc{pl} still prayer good\_wish=\textsc{obj} do/make fibre-\textsc{poss}.3\\
	\glt \lq The next day, Saturday, they first, \textbf{they first did the good wish prayer}, for their fibre boat.' \parencite{Visser2021b}\il{Kalamang|)}

	\ex \ili{Japhug}\label{exFirstJaphug}\\
	\gll \textbf{Nɤʑo} \textbf{pɤjkhu} \textbf{tɯ}-\textbf{rzɯɣ} \textbf{tɤ}-\textbf{nɯna} tɕe tɕetha rɤma-tɕi.\\
	2\textsc{sg} still one-section \textsc{imp}-rest \textsc{lnk} later work-1\textsc{du}\\
	\glt \lq {\cn 你暂时休息 一下,等一会我们再工作} [\textbf{Take a break for now} and we’ll work again later].' (\cite[361]{Jacques2016}, glosses added)
\end{exe}

The use I discuss here bears some similarity to the \lq\lq{}further-to\rq\rq{ }use I address in \Cref{sectionFurtherTo}, and which likewise suggests a contrast between the time under discussion and a subsequent interval. The key difference is that the latter function marks an event as an addition to some larger, ongoing course of events. Thus,\il{German|(} in (\ref{exFirstGermanFurtherTo}) the act of taking a shower is depicted as belonging to the phase of the day defined by soccer practice, while also signalling that it lies just before the transition to dinner time. Lastly, note that 
I discuss the marking of precedence in subordinate\is{subordination} contexts \lq before \textit{p}, \textit{q}\rq{ }separately in \Cref{sectionBefore}.

\begin{exe}
	\ex German\label{exFirstGermanFurtherTo}\\
	 Context: I have just come home from soccer practice. It is fairly late.\\
	\gll \textbf{Ich} \textbf{dusch'} \textbf{noch}. Dann gibt-'s Abend-essen.\\
	1\textsc{sg} take\_shower.1\textsc{sg} still then \textsc{exist}.3\textsc{sg}-3\textsc{sg}.\textsc{n} evening-meal(\textsc{n})\\
	\glt \lq \textbf{I'm just taking a quick shower}. Dinner will be just after.' \parencite[16]{Beck2019}
\end{exe}\il{German|)}

\subsubsubsection{Distribution in the sample}
\Cref{tableFirst} lists the ten expressions in my sample for which the \lq first, for now\rq{ }use is attested.\footnote{Outside of my sample, this use has been described for \textit{nánu} in the Bantu language \ili{Língala} \parencite{NassensteinPasch2021}, as well as for \mbox{\textit{ci}-} in another Bantu language, \ili{Luvale} \parencite[130–131]{Horton1949}, and its \ili{Holoholo} cognate \mbox{\textit{ki}-} \parencite[103–104]{Coupez1955}. In Papunesian, it is attested for \textit{ngabak} in the Austronesian language \ili{Dakaka} \parencite[222–223]{vonPrince2015} and for \ili{Yeri} (Nuclear Torricelli) \textit{kua} \parencite[477–479]{Wilson2017}. A candidate from South America is \ili{Tariana} (Arawakan) \textit{daka} (examples throughout \cite{Aikhenvald2003}).} As can be gathered, it is found in all macro-areas minus Australia and with independent grammatical words as well as with clitics and affixes. Note that, in addition to the items listed, \ili{Wardaman} \textit{gaya}-\textit{wun} and Western Shoshoni\il{Shoshoni, Western} \textit{eki}-\textit{sen} transparently go back to \lq{}now/today-pertaining\_to\rq{ }and \lq now-only\rq{}, respectively.\largerpage

\begin{table}[H]
	\caption{\lq For now, for a while, first\rq{}\label{tableFirst}}
		\begin{tabular}{llll}
		\lsptoprule
		Macro-area & Language & Expression & Appendix\\
		\midrule
		Africa & Adamawa Fulfulde\il{Fulfulde, Adamawa} & \textit{tawon} & \ref{appendixAdamawaFirst}\\
		& \ili{Bende} & \textit{sya}- & \ref{appendixBendeFirst}\\
		Eurasia & \ili{Japhug} & \textit{pɤjkʰu} & \ref{appendixJaphugFirst}\\
		North America & Classical Nahuatl\il{Nahuatl, Classical} & \textit{oc} & \ref{appendixClassicalNahuatlFirst}\\
		Papunesia & \ili{Kalamang} & \textit{tok} & \ref{appendixKalamangFirst}\\
		& \ili{Paiwan} & =\textit{anan} & \ref{appendixPaiwanFirst}\\
		& \ili{Saisiyat} & \textit{nahan} & \ref{appendixSaisiyatFirst}\\
		South America & \ili{Cavineña} & =\textit{jari} & \ref{appendixCavinenaFirst}\\
		& \ili{Culina} & -\textit{kha} & \ref{appendixKulinaFirst}\\
		& Huallaga-Huánuco Quechua\il{Quechua, Huallaga-Huánuco} & -\textit{raq} & \ref{appendixQuechuaFirstMain}\\
		\lspbottomrule
		\end{tabular}
\end{table}

\subsubsubsection{A closer look}\il{Quechua, Huallaga-Huánuco|(} On closer examination of the \lq first, for now\rq{ }use, there is a noteworthy gray area between this function and phasal polarity \textsc{still}, as evidenced by examples like (\ref{exFirstQuechua}). Here, the state in question is obtained before \isi{topic time} (the time under discussion) and the speaker intends for it to remain the same way, albeit not endlessly so, as becomes evident from the discourse context.
\begin{exe}
	\ex Huallaga-Huánuco Quechua\label{exFirstQuechua}\\
	\gll Ama\textup{(}-raq\textup{)} aywa-y-raq-chu. \textbf{Ka}-\textbf{ku}-\textbf{yka}:-\textbf{shun}-\textbf{raq}.\\
	\textsc{neg}-still go-\textsc{fut}.2-still-\textsc{neg} \textsc{cop}-\textsc{refl}-\textsc{ipfv}-\textsc{fut}:1\textsc{pl}.\textsc{incl}-still\\
	\glt \lq Donʼt go yet. \textbf{Letʼs be yet} (\textbf{awhile} \textbf{here} \textbf{together}).' \parencite[388]{Weber1989}
\end{exe}\il{Quechua, Huallaga-Huánuco|)}

\is{command|(}In fact, as-yet unactualised states-of-affairs, be it ones that entail the \isi{persistence} of a pre-existing condition or ones corresponding to an entirely new situation, are a recurrent theme. Thus, with \ili{Cavineña} \mbox{=\textit{jari}} the relevant use is primarily found in imperative-hortative constructions and with predictions of transient future situations. Both types of contexts are illustrated in (\ref{exFirstCavineña1}).

\begin{exe}
	\ex \label{exFirstCavineña1}
	\begin{xlist}
		\exi{}\ili{Cavineña}
		\ex\label{exFirstCavineñaRest}
		\gll \textbf{Pisu}-\textbf{kwe}=\textbf{jari}=\textbf{shana} \textbf{juye}=\textbf{ekatse}! Pa-kanajara ekatse!\\
	untie-\textsc{imp}.\textsc{sg}=still=\textsc{pity} ox=\textsc{du} \textsc{juss}-rest 3\textsc{du}\\
		\glt \lq \textbf{Untie the oxen (\textsc{du}) for a while}, the poor animals (who are suffering so much pulling the cart)! Let them (\textsc{du}) rest!'
		\\\parencite[655]{Guillaume2008}	
\is{command|)}
		\ex
		\gll \textbf{Tasi} \textbf{ju}-\textbf{ya}=\textbf{jari} metajudya=ishu.\\
	drive\_taxi:1\textsc{sg} \textsc{cop}-\textsc{ipfv}=still tomorrow=\textsc{purp}\\
	\glt \lq\textbf{I will drive my} (\textbf{motorcycle}) \textbf{taxi a little bit} for (me to have money) tomorrow.' \parencite[662]{Guillaume2008}
	\end{xlist}
\end{exe}

A third type of context in which \mbox{=\textit{jari}} can have the \lq first, for now\rq{ }reading is the turn- or conversation closer in (\ref{exFirstCavinena2}).\il{Cavineña} The latter appears to be a cross-linguistically recurring environment, and similar attestations are found with Classical Nahuatl\il{Nahuatl, Classical} \textit{oc} and with \ili{Culina} \mbox{-\textit{kha}}.

\begin{exe}
	\ex \ili{Cavineña}\label{exFirstCavinena2}\\
	\gll Jadya=kamadya=\textbf{jari}!\\
	thus=only=still\\
	\glt \lq That's all \textbf{for now}! (but there will be another meeting or story)' (\cite[662]{Guillaume2008}; Antoine Guillaume, p.c.)		
\end{exe}

In the case of \ili{Bende} \mbox{\textit{syá}-} the use in question is restricted to the formally unmarked \isi{imperfective} present. It is noteworthy, however, that this collocation always involves a modal\is{modality} component in the form of an exhortation; see (\ref{exBendeFirst}).\il{Paiwan|(} Relatedly, in Paiwan the \lq first, for now\rq{ }use of \mbox{=\textit{anan}} often blends into an announcement of what the speaker is about to do, or that it is someone's turn to perform an act (\appref{appendixPaiwanAnnouncement}). This is illustrated in (\ref{exFirstPaiwan1}).

\begin{exe}
	\ex \ili{Bende}\label{exBendeFirst}\\
	\gll Tu-\textbf{syá}-tehǎ.\\
	\textsc{subj}.1\textsc{pl}-still-love\\
	\glt \lq Let us love first.\rq{ }\parencite{Abe2016}	

	\ex Paiwan \label{exFirstPaiwan1}\\
	 Context: Crab and Monkey are trying to burn one another. The fire did no harm to Crab.\\
	\gll \textbf{Sa} \textbf{tisun}-\textbf{ay}=\textbf{anan}, qali-an i ḍai$\sim$ḍail.\\
	and 2\textsc{sg}.\textsc{nom}-\textsc{sbjv}=\textsc{still} friend-\textsc{nmlz} \textsc{prep} \textsc{redupl}$\sim$monkey\\
	\glt [Crab:] \lq \textbf{Now it’s your turn}, friend monkey.'
	\\\parencite[197]{EarlyWhitehorn2003}
\end{exe}\il{Paiwan|)}

\il{Saisiyat|(}With \ili{Kalamang} \textit{tok} and Saisiyat \textit{nahan}, on the other hand, \lq first, for now\rq{ }has given rise to a \isi{politeness} function in directive speech acts; see (\ref{exFirstSaisiyatPolite}). As I discuss in some more detail in \Cref{sectionPoliteness}, this is a clear instance of intersubjectivisation,\is{subjectivity}\is{expressivity} based on reducing the threat to the addressee's negative face and acknowledging the imposition that is being made.

\begin{exe}
	\ex Saisiyat\label{exFirstSaisiyatPolite}\is{politeness}\\
	\gll Si\rq{}ael \textbf{nahaen}.\\
	eat.\textsc{imp} still\\
	\glt \lq Chi gė dongxi (zài zŏu) ba! / Come have a bite (\textbf{before you leave})!\rq{ }\parencite[120]{Huang2008}.
\end{exe}\il{Saisiyat|)}

\is{actionality|(}\is{aspect|(}
Given that the situation depicted in the predicate need neither be actualised nor persistent,\is{persistence} the \lq first, for now\rq{ }use is less selective than phasal polarity \textsc{still} when it comes to aspect and actionality. \il{Culina|(}For instance, in (\ref{exFirstCulinaGive}), it occurs together with an achievement predicate \lq give\rq{}, signalling that the transfer of possession is only temporary. To give another example, in (\ref{exFirstSaisiyat}) below this function goes together with a \isi{perfective} viewpoint.

\begin{exe}
	\ex Culina\label{exFirstCulinaGive}\\
	\gll O-kha koshiro tia-za \textbf{da} \textbf{o}-\textbf{to}-\textbf{na}-\textbf{kha}-\textbf{ni} \textbf{towi}.\\
	1\textsc{sg}-\textsc{poss} knife 2-\textsc{loc} give 1\textsc{sg}-\textsc{it}-\textsc{aux}-still-\textsc{decl}.\textsc{f} \textsc{fut}\\
	\glt \lq Emprestarei minha faca para você. [\textbf{I'll lend} you my knife.]\rq{}
	\\\parencite[185]{Tiss2004}\footnote{As \textcite[185]{Tiss2004} discusses, the translation as \textit{emprestar} \lq lend\rq{ }is due to the collocation of \textit{da} \lq give\rq{ }and \mbox{-\textit{kha}}. Without this suffix, the predicate would be understood as denoting a permanent transfer of possession.}
\end{exe}
\is{actionality|)}\il{Culina|)}\is{aspect|)}

In terms of its larger sentential embedding, this use, like many functional extensions of \textsc{still} expressions, is recurrently attested with other items that have a similar meaning. For instance, in the \il{Nahuatl, Classical|(}Classical Nahuatl example (\ref{exFirstClassicalNahuatl}) \textit{oc} goes together with \textit{achto} \lq first\rq{}.\il{Saisiyat|(} In (\ref{exFirstSaisiyat}), the predicate modified by Saisiyat \textit{nahan} itself denotes a transitory event \lq pass by\rq{}.

\begin{exe}
	\ex Classical Nahuatl\label{exFirstClassicalNahuatl2}\\
	\gll In ìcuāc ti-tla-chpāna-z-nequi, \textbf{oc} \textbf{yê} \textbf{achto} \textbf{in} \textbf{ti}-\textbf{tl}-\textbf{àhuachī}-\textbf{z}.\\
	\textsc{det} when \textsc{subj}.2\textsc{sg}-\textsc{obj}.\textsc{indef}.\textsc{non}.\textsc{human}-sweep-\textsc{prosp}-want still actually first \textsc{det} \textsc{subj}.2\textsc{sg}-\textsc{obj}.\textsc{indef}.\textsc{non}.\textsc{human}-irrigate-\textsc{prosp}\\
	\glt \lq Quando quieras barrer, primero has de regar. [When you want to sweep, \textbf{you first have to apply water}.]\rq{ }(\cite[502]{Carochi1645}, glosses added)\il{Nahuatl, Classical|)}
	
	\ex Saisiyat\label{exFirstSaisiyat}\\
	\gll \textbf{Yako} \textbf{kal}-\textbf{'aish} \textbf{kala} '\textbf{okay} \textbf{naehan}, ma-'ngel=ila.\\
	1\textsc{sg}.\textsc{nom} pass-in\_passing \textsc{loc}.\textsc{pl} O. still \textsc{stat}-slow=\textsc{compl}\\
	\glt \lq \textbf{I stopped by Okayʼs home for a while} and was late (for the meeting).' \parencite[561]{ZeitounEtal2015}
\end{exe}\il{Saisiyat|)}

\il{Culina|(}Lastly, Culina \mbox{-\textit{kha}}, first seen in (\ref{exFirstCulinaGive}) above, is an interesting case. To all appearances, even in its function as a marker of phasal polarity it tends to evoke a notion of precedence. Thus, \textcite[183]{Tiss2004} describes its semantic import in such sentences as \lq\lq before something else happens, the situation in question still continues\is{persistence} for a limited time\rq\rq{}.\footnote{In the original Portuguese, \lq\lq  antes de algo novo acontecer, a situação em questão ainda continua por um período limitado\rq\rq{}.} Example (\ref{exFirstCulina}) is an illustration. As I discuss in \Cref{sectionThusFarOnly}, in this characteristic Culina \mbox{-\textit{kha}} is very similar to another sample expression,\il{Mateq|(} Mateq \textit{bayu}. The main difference is that \textit{bayu} is not attested in a purely relational sense. I return to this point below.\il{Mateq|)}

\begin{exe}
	\ex Culina\label{exFirstCulina}\\
	\gll Zohe papeo wa wa  \textbf{na}-\textbf{kha}-\textbf{wi}.\\
	Z. paper call call 3:\textsc{aux}-still-\textsc{decl}.\textsc{m}\\
	\glt \lq Zohe ainda está lendo (antes de logo fazer outras coisas). [Zohe \textbf{is still} reading (\textbf{before later} doing something else).\rq{ }\parencite[184]{Tiss2004} 
\end{exe}
\il{Culina|)}

\subsubsubsection{Discussion} 
I now turn to a discussion of the conceptual and usage-based ties between phasal polarity \textsc{still} and the \lq first, for now\rq{ }use. In general terms, both functions share a sequential component in that they evoke a contrast between a situation (or a state brought about by one) and a subsequent phase at which it may have ceased.\footnote{A very similar observation is made by \textcite[1264]{Launey1986}.} \Cref{figureFirstForNow} is a graphic comparison, using an \isi{imperfective} viewpoint\is{aspect} for ease of comparison. Given this similarity plus the lack of etymological or historical data for most of the relevant expressions, a development in either direction seems plausible.

\begin{figure}[bth]\is{imperfective}
	\centering
	\begin{subfigure}[htb]{0.45\linewidth}\is{topic time}
		\centering
		\begin{tikzpicture}[node distance = 0pt]
			\node[mynode, text width=\breit, fill=cyan, very near start] (spain){Situation};
			\node[mynode, fill=none, text width=\superschmal, right= of spain] (france){(\small{$\Diamond$}\neg{ }Sit.)};
	\node[mynode, text width=1.5*\hoehe, fill=none, anchor=east] at (spain.east) (TT) {};			
			\draw[-, densely dashed] ($(TT.north east)+ (-0.2pt,0) $) to ($(TT.south east)+ (-0.2pt,-0.5*\hoehe) $);
			\draw[-, densely dashed] ($(TT.north west)+ (0.2pt,0)$) to ($(TT.south west)+ (0.2pt,-0.5*\hoehe)$);
			\node [align=center, label distance=0, anchor=north] at ($(TT.south)+(-0,-0.5*\hoehe)$) {Topic\\time};
			\draw[-latex, line width=0.5pt]  (spain.south west) to  ($(france.south east)+(1ex,0)$) node [right] {t};
		\end{tikzpicture}
	\subcaption{\textsc{still}}
	\end{subfigure}
\begin{subfigure}[htb]{0.45\linewidth}
		\centering
		\begin{tikzpicture}[node distance = 0pt]
			\node[mynode, text width=1.5*\hoehe, fill=cyan, very near start] (spain){Sit.\textsubscript{1}};
 			\node[mynode, fill=none, text width=\superschmal, right= of spain] (france){(Sit.\textsubscript{2})};
			\draw[-, densely dashed] ($(spain.north east)+ (-0.2pt,0) $) to ($(spain.south east)+ (-0.2pt,-0.5*\hoehe) $);
			\draw[-, densely dashed] ($(spain.north west)+ (0.2pt,0)$) to ($(spain.south west)+ (0.2pt,-0.5*\hoehe)$);
			
			\node [align=center, label distance=0, anchor=north] at ($(spain.south)+(-0,-0.5*\hoehe)$) {Topic\\time};
			\draw[-latex, line width=0.5pt]  (spain.south west) to  ($(france.south east)+(1ex,0)$) node [right] {t};
		\end{tikzpicture}
		\subcaption{First, for now (\textsc{ipfv})}
	\end{subfigure}
	\caption{Schematic illustration of \lq first, for now\rq{ }(with an imperfective viewpoint)\label{figureFirstForNow}}
\end{figure}

\il{Bende|(} 
The one sample expression for which a primacy of phasal polarity can be firmly established is \ili{Bende} \mbox{\textit{syá}-}. Thus, cognates of this prefix are widespread across Eastern Bantu as markers of phasal polarity, but usually without an additional precedence function (\cite{Abe2015}, \citeyear{Abe2016}). \is{modality|(}Earlier I pointed out that the \ili{Bende} case always contributes an exhortative force, as seen in (\ref{exBendeFirst}) above. Example (\ref{exBendeFirst2}) is another illustration.

\begin{exe}
	\ex \ili{Bende} \label{exBendeFirst2}\\
	\gll Tu-\textbf{syá}-kola mú-límó.\\
	\textsc{subj}.1\textsc{pl}-still-do \textsc{ncl}3-work\\
	\glt i.\phantom{i} \lq We are still working.'\\
	ii. \lq \textbf{Let us work} \textbf{first}.' \parencite[29]{Abe2015}
\end{exe}

This element of \isi{necessity} suggests that the precedence function of  \mbox{\textit{syá}-} goes back to statements about persistent\is{persistence} intentions or obligations, together with the conventionalisation of an erstwhile relevance implicature. For instance, in a case like (\ref{exBendeFirst2}) an assertion that there is work to be done could suggest that once this task has been ticked off, the group can turn their attention to whatever the addressee has in mind. Support for this interpretation comes from a comparable configuration in \ili{Holoholo} (not in my sample) that features the cognate prefix \textit{ki}-. This construction, illustrated in (\ref{exFirstHoloholo}) marks a \lq\lq 
process that must take place before another one\rq\rq{ }\parencite[104]{Coupez1955}.\footnote{In the original French, \lq\lq procès qui doit se dérouler avant un autre\rq\rq{}.} This scenario is very similar to one of the pathways I suggest in \Cref{sectionFurtherTo} for the \lq\lq{}further-to\rq\rq{ }use of \ili{German} \textit{noch} and Hills Karbi\il{Karbi, Hills} \textit{-\textit{làng}}, except that what is profiled here is the contrast with subsequent situations, rather than cohesion with preceding ones.

\begin{exe}
	\ex \ili{Holoholo}\is{persistence}\label{exFirstHoloholo}\\
	\gll Tw-a-\textbf{ki}-lólâ.\\
	\textsc{subj}.1\textsc{pl}-\textsc{prs}-still-look\\
	\glt \lq Il faut que nous regardions d\rq{}abord. [We \textbf{have to} look \textbf{first}.]\rq{}
	\\(\cite[103]{Coupez1955}, glosses added)
\end{exe}
\il{Bende|)}\is{modality|)} 

The sample data suggest another pathway from phasal polarity \textsc{still} to \lq first, for now\rq{}. As I pointed out initially, the two functions overlap in contexts that feature an exhortation to the effect that an already existing situation be preserved, such as in (\ref{exFirstQuechua}) above, or in (\ref{exFirstCavinenaRetain}).\il{Kalamang|(} Closely related are attestations like (\ref{exFirstKalamang2}), where it is the semantics of the predicate \lq leave the tips\rq{ }that includes an element of causation. What is more, this predicate by itself entails the seamless continuation\is{persistence} of a prior state. The contribution of the additional \textsc{still} expression with its requirement for an alterable situation,  can then, via the maxim of quantity, be taken to indicate \lq for the time being\rq{}.\largerpage[2.25]

\begin{exe}
	\ex \ili{Cavineña}\label{exFirstCavinenaRetain}\is{persistence}\\
	\gll E-ra=tu ani-sha-ya=\textbf{jari}.\\
	1\textsc{sg}-\textsc{erg}=3\textsc{sg} sit-\textsc{caus}-\textsc{ipfv}=still\\
	\glt \lq I will retain him (lit. make him sit) \textbf{some more time}.'
	\\\parencite[289]{Guillaume2008}
	\ex Kalamang\label{exFirstKalamang2}\is{persistence}\\
	Context: The narrator was constructing a canoe.\\
	\gll An se ewun=at kies ah \textbf{tim}=\textbf{at} \textbf{an} \textbf{tok} \textbf{mamun}.\\
	1\textsc{sg} already trunk=\textsc{obj} carve \textsc{interj} edge=\textsc{obj} 1\textsc{sg} still leave\\
	\glt \lq I carved (cut off) the base, \textbf{I} \textbf{still} [\textbf{for the time being}] \textbf{left the tips}.\rq{ }\parencite{Visser2021b}	
\end{exe}

\il{Nahuatl, Classical|(}From instances like (\ref{exFirstCavinenaRetain}, \ref{exFirstKalamang2}) it is only a very narrow gap to an example like (\ref{exFirstClassicalNahuatl}),\il{Kalamang|)} where a new preliminary state is established. In other words, there is a continuum running from \lq let/make still be\rq{ }to \lq let/make be for now\rq{ }and ultimately \lq do/be for now\rq{}.

\begin{exe}
	\ex Classical Nahuatl\label{exFirstClassicalNahuatl}\\
	\gll \textbf{Oc} \textbf{ni}-\textbf{c}-\textbf{cahua} \textbf{in}, quin çā-tēpan ni-c-tzonquīxtī-z in ī-tlàtōllo, in oncān ī-monecyan.\\
	still \textsc{subj}.1\textsc{sg}-\textsc{obj}.3\textsc{sg}-leave \textsc{det} then only-momentarily \textsc{subj}.1\textsc{sg}-\textsc{obj}.3\textsc{sg}-finish-\textsc{prosp} \textsc{det} \textsc{poss}.3\textsc{sg}-speech \textsc{det} there \textsc{poss}.3\textsc{sg}-proper\_place/time\\
	\glt \lq Dexolo aqui por agora, después acabaré de tratar dello en su lugar. [\textbf{I’ll leave it at this for now}, I’ll finish talking about it afterwards, in due time.]\rq{ }(\cite[500]{Carochi1645}, glosses added)
\end{exe}\il{Nahuatl, Classical|)}

\il{Quechua, Huallaga-Huánuco|(}Turning to the reverse direction, a development from marking precedence to phasal polarity \textsc{still} has been proposed by \textcite[90-91]{vanBaar1997} for \mbox{-\textit{raq}} and its equivalents across Quechuan. This seems plausible for several reasons.\is{restrictive|(} To begin with, it would provide for a motivated and direct link to the temporal restrictive function \lq no earlier than\rq{ }of Huallaga-Huánuco Quechua \mbox{-\textit{raq}} and its congeners, illustrated in (\ref{exFirstQuechuaRSTR}), namely as a separate extension rooted in \lq first \textit{p}, (then) \textit{q}\rq{}. 

\begin{exe}
\ex Huallaga-Huánuco Quechua\label{exFirstQuechuaRSTR}\\
	\gll \textbf{Ñaka}-\textbf{y}-\textbf{ta}-\textbf{raq} tari-sha.\\
	take\_long\_time-\textsc{inf}-\textsc{adv}-still find-\textsc{ant}.3\\
	\glt \lq He found it \textbf{only} \textbf{after} he had searched \textbf{a} \textbf{good} \textbf{while}.\rq{ }\parencite[132]{Weber1993}
\end{exe}\il{Quechua, Huallaga-Huánuco|)}
	
Secondly, the transparent origin of Western Shoshoni\il{Shoshoni, Western} \textit{ekisen} in \lq right now\rq{ }and that of \ili{Wardaman} \textit{gayawun} in \lq pertaining to now\rq{ }point in the same evolutionary direction.\footnote{Outside of my sample \textcite[75]{vanderAuwera1998} lists \ili{Lithuanian} \textit{dar} < \textit{dabar} \lq now\rq{ }and \textcite[94]{vanBaar1997} indicates that \textit{napy} in the Siberian isolate \ili{Nivkh} could go back to a deictic element \textit{na} and \mbox{\textit{by}} \lq only\rq.}\il{Culina|(} Lastly, the data on Culina \mbox{-\textit{kha}} suggest that this expression is currently in the transitional stage. As I discussed earlier, even when marking persistence,\is{persistence} \mbox{-\textit{kha}} appears to often evoke notions of restrictive \lq only\rq{ }or of relational \lq before …\rq{}. This was seen in (\ref{exFirstCulina}) above; example (\ref{exFirstCulina2}) is another illustration.\pagebreak
	
\begin{exe}	
	\ex Culina\label{exFirstCulina2}
		\begin{xlist}
			\exi{A:} \textit{Ami wadani?}\\
			\lq A mãe está dormindo? [Is mother sleeping?]\rq{}
			\exi{B:}
			\gll Nowe ra-ni \textbf{hapi} \textbf{na}-\textbf{kha}-\textbf{ni}.\\
			3:not\_be \textsc{aux}-\textsc{decl}.\textsc{f} bathe 3:\textsc{aux}-still-\textsc{decl}.\textsc{f}\\
			\glt \lq Não (está), ela ainda está tomando banho (antes de logo aparecer).' [No she isn’t, \textbf{she's still bathing} (\textbf{before showing up after}).]
			\\\parencite[184]{Tiss2004}
		\end{xlist}
\end{exe}\il{Culina|)}\is{restrictive|)}

Presumably, in this scenario the acquisition of a persistive\is{persistence} meaning component goes back to the recurrent use of an expression in contexts that involve pre-established states or processes together with an \isi{imperfective} viewpoint.\is{aspect} \il{Quechua, Huallaga-Huánuco|(}Again, a conceivable bridge can be found in instances where a preliminary continuation\is{persistence} of a situation is at stake, such as in (\ref{exFirstQuechua}), repeated below. Another type of context that could have played a role is illustrated in (\ref{exFirstQuechuaMote}). Here, the at-issue content corresponds to the mid-point in a development. This not only allows for a contrast with a later stage at which the process has become completed. Crucially, it also entails a prior process leading from \lq raw\rq{ }to \lq half cooked\rq{ }(for more discussion of phasal polarity and scalar\is{scale} predicates, see \Cref{sectionScalar}).

\begin{exe}
	\exr{exFirstQuechua} Huallaga-Huánuco Quechua\\
	\gll Ama\textup{(}-raq\textup{)} aywa-y-raq-chu. \textbf{Ka}-\textbf{ku}-\textbf{yka}:-\textbf{shun}-\textbf{raq}.\\
	\textsc{neg}-still go-\textsc{fut}.2-still-\textsc{neg} \textsc{cop}-\textsc{refl}-\textsc{ipfv}-\textsc{fut}:1\textsc{pl}.\textsc{incl}-still\\
	\glt \lq Donʼt go yet. \textbf{Letʼs be yet} (\textbf{awhile} \textbf{here} \textbf{together}).' \parencite[388]{Weber1989}

	\ex \label{exFirstQuechuaMote}
	\gll Mute-ga \textbf{capru}-\textbf{lla}-\textbf{rä}-\textbf{mi}. Mana-mi pata-n-rä-chu.\\
	mote-\textsc{top} half\_cooked-only-still-\textsc{evid} \textsc{neg}-\textsc{evid} burst-3-still-\textsc{neg}\\
	\glt \lq El mote está todavía a medio cocinar.  Todavía no se revienta / The muti \textbf{is still half cooked}. It has not yet cracked open.\rq{ }(\cite[121]{WeberEtAl2008}, glosses added)	
\end{exe}

Lastly, examples (\ref{exFirstQuechua}, \ref{exFirstQuechuaMote}) also contain an adjacent, negated\is{negation} clause and thereby illustrate another point. Thus, it is likely that the development from \lq first, for now\rq{ }to marking phasal polarity \textsc{still} went together with signalling \textsc{not yet}\is{not yet} via \lq not for now, not at first\rq{}. This is an independently attested strategy, an example being \ili{Bambara} \textit{fɔ́lɔ} in (\ref{exFirstBambara}). Such a configuration can easily be reinterpreted as \textsc{still} \mbox{\neg{}\textit{p}}\is{negation} and most expressions in \Cref{tableFirst} participate in signalling \textsc{not yet} together with a negator (the exceptions being \ili{Bende} \mbox{\textit{sya}-} and Classical Nahuatl\il{Nahuatl, Classical} \textit{oc}).\il{Quechua, Huallaga-Huánuco|)}
\begin{exe}
	\ex \label{exFirstBambara}
	\begin{xlist}		
		\exi{}\ili{Bambara}
		\ex\label{exFirstBambara2}
		\gll Nà yàn \textbf{fɔ́lɔ}!\\
		come here first\\
		\glt \lq Viens ici d’abord! [Come here \textbf{first}/\textbf{for} \textbf{the} \textbf{moment}!]\rq{ }
		 
		\ex\label{exFirstBambara3}\il{Bambara}\is{not yet}
		\gll Ù \textbf{má} nà \textbf{fɔ́lɔ}.\\
		3\textsc{pl} \textsc{neg}.\textsc{pfv} come first\\
		\glt \lq They have \textbf{not} come \textbf{yet}.\rq{ }\parencite[105]{DombrowskyHahn2020}
	\end{xlist}
\end{exe}
\is{precedence|)}

\subsubsection{Sequencing \lq and then\rq{}}\label{sectionEventSequencing}\is{sequencing|(}
\subsubsubsection{Introduction}
In this subsection, I discuss a use in which a \textsc{still} expression marks a subsequent event in a series \lq and then, afterwards\rq{}. The \ili{Bukiyip} example in (\ref{exEventSequencingBukiyipIntro}) is an illustration. Here, \textit{wotak} signals that the situation depicted in its host clause \lq look at the betel nut\rq{ }is intended to occur after, and is tied to, the one depicted in the preceding clause.

\begin{exe}
	\ex \ili{Bukiyip}\label{exEventSequencingBukiyipIntro}\\
	Context: A boy has climbed up a betel nut tree to look for betel nuts. Trying to split one open, he got stuck by a wasp. A bystander shouts.\\
	\gll Kw-autu-i anab \textbf{wotak} \textbf{i}-\textbf{túl}-\textbf{úb}, \textbf{bú}-\textbf{b}.\\
	\textsc{imp}-throw\_down-\textsc{ven} \textsc{indef}.\textsc{ncl}1.\textsc{sg} still \textsc{subj}.1\textsc{sg}:\textsc{irr}-see-\textsc{obj}.\textsc{ncl}1.\textsc{sg} betel\_nut-\textsc{ncl}1.\textsc{sg}\\
	\glt \lq Throw down one betel nut \textbf{and I’ll} \textbf{[then] take a look at it}.'
	\\(\cite[233]{ConradWigoga1991}, glosses added)
\end{exe}

Note that the sequencing use differs from the additive one I discuss in \Cref{sectionAdditive} in the temporal ordering it establishes. For instance, example (\ref{exEventSequencingChinese}) features Mandarin Chinese\il{Chinese, Mandarin} \textit{hái} as \lq in addition, also\rq{}, but does not entail that ironing the tablecloth follows the chores described in the preceding clauses. Lastly, the function I address here also differs from the use of  \textsc{still} expressions as coordinators\is{coordination} (\Cref{sectionCoordination}), in that it does not establish a syntactic\is{syntax} relationship between clauses.\largerpage[2.25]

\begin{exe}
	\ex Mandarin Chinese\il{Chinese, Mandarin}\label{exEventSequencingChinese}\\
		\gll Zhāngsān dǎsǎo le fángzi, zuò le dàngāo \textbf{hái} \textbf{yùn} \textbf{le} \textbf{zhuōbù}.\\
	Z. sweep \textsc{pfv} house do \textsc{pfv} cake still iron \textsc{pfv} tablecloth\\
	\glt \lq Zhangsan a balayé la pièce, a fait un gâteau et aussi/même repassé la nappe. [Zhangsan swept the room, baked a cake and also ironed the tablecloth.]\rq{ }\parencite[113]{Donazzan2008}
\end{exe}

\subsubsubsection{Distribution in the sample}\il{Ngambay|(}\il{Laka|(}
\Cref{tableEventSequencing} lists the four expressions in my sample that have the sequencing use. As can be gathered, they stem from unrelated languages and three distinct macro-areas. Except for the case of Kekchí,\il{Kekchí} where the sequencing function is marked by a polymorphemic collocation, all instances involve small, grammatical words. Lastly, the inclusion of \ili{Kaba} \textit{bbáy} here is somewhat tentative. Although \textcite[310, 407]{MoserDingatoloum2007} list \textit{ensuite} \lq afterwards' and \textit{puis} \lq then' as senses of this expression, there are no clear-cut instances of a sequencing use in the data I consulted. However, the existence of this function has been established beyond doubt for the cognate item \textit{ɓə́y}/\textit{ɓí} in the other two Western Sara languages, Laka and Ngambay (\cite[130]{Keegan2014}; \cite[108, 114]{Thayer1978}; \cite[119, 138]{Vandame1963}). I therefore resort to illustrations from these varieties.\footnote{In addition; \citeauthor{Huang2007} (\citeyear{Huang2007}, \citeyear[114–116]{Huang2008}) suggests that \ili{Saisiyat} \textit{nahan} possesses a sequencing function. Both his description and the few available examples are, however, compatible with the additive use that the same expression has.}

\begin{table}
	\caption{Event sequencing\label{tableEventSequencing}}
		\fittable{\begin{tabular}{lllll}
			\lsptoprule
			Macro-area & Language & Expression & Collocate & Appendix\\
			\midrule
			Africa & \ili{Mundang} & \textit{ɓà} &  & \ref{appendixMundangEventSequencing}\\
			& \ili{Kaba} & \textit{bbáy}\footnote{With qualifications; see discussion.} & & \ref{appendixKabaSequencing}\\
			North America & \ili{Kekchí} & \textit{toj} & \textit{toja’}\sim{}\textit{toje’}\sim{}\textit{tojo’} \textit{naq} &
			\ref{appendixKekchiSequencing}\\
		& & & \lq still.\textsc{subord} \textsc{subord}\rq{}\\
		Papunesia & \ili{Bukiyip} & \textit{wotak} &  & \ref{appendixBukiyipSequencing}\\
		\lspbottomrule
		\end{tabular}}
\end{table}

\subsubsubsection{A closer look} A first illustration of the sequencing use has been given in (\ref{exEventSequencingBukiyipIntro}) above. As in that example, this function typically involves the juxtaposition of two clauses, and the situations depicted in the two clauses often form a pair of preparation and culmination. In fact, the Laka and Ngambay cognates of \ili{Kaba} \textit{bbáy} commonly figure in a fixed pattern [[p \textit{ɓá}] [q \textit{ɓə́y}\textup{/}\textit{ɓí}]] \lq \textit{p} first, \textit{q} still\rq{}, i.e. \lq{}first \textit{p} then \textit{q}\rq{}. This is illustrated in (\ref{exEventSequencingLaka}, \ref{exEventSequencingNgambay}).

\begin{exe}
		\ex Laka\label{exEventSequencingLaka}\\
		\gll \sout{Í}-tɔ́l bāngàw \textbf{ɓá} \textbf{d}-\textbf{{ɔ́ bɨ̄}} \textbf{ɓí}.\\
		\textsc{subj}.\textsc{3pl}-peel sweet\_potato first \textsc{subj}.\textsc{3pl}-fry still\\
		\glt \lq On pèle la patate douce avand de la frire. [One peels the sweet potato \textbf{first} \textbf{and then fries it}.]\rq{ }(\cite[123]{Keegan2014}, glosses added)
		\ex Ngambay\label{exEventSequencingNgambay}\\
		\gll K-únd-á sár\sim{}sár \textbf{ɓá} \textbf{d}-\textbf{ḭ́ā̰}-\textbf{á} \textbf{ɓə́y}.\\
	\textsc{subj}.3\textsc{pl}-beat-\textsc{obj}.3\textsc{sg} \textsc{redupl}\sim{}long\_time first \textsc{subj}.3\textsc{pl}-let-\textsc{obj}.3\textsc{sg} still\\
		\glt \lq Ils le frappaient pendant longtemps, puis ils l'ont  laissé. [They were hitting him for a long time, \textbf{then they left him}.]\rq{}
	\\(\cite[259]{Keegan2014}, glosses added)
\end{exe}

While such bi-clausal patterns are certainly a common theme with the event sequencing use, they do not constitute a hard-and-fast rule. For instance, in the \ili{Bukiyip} example (\ref{exEventSequencingBukiyip2}) the initial \lq fixing the road\rq{ }event is elaborated upon in a sequence of clauses before turning to its culmination.

\begin{exe}
	\ex \ili{Bukiyip}\label{exEventSequencingBukiyip2}\\
	 \gll Énech ch-ú-lib wichap,\\
	 \textsc{indef}.\textsc{ncl}8.\textsc{pl} \textsc{subj}.\textsc{ncl}8.\textsc{pl}-\textsc{irr}-cut grass\\
	 \glt \lq Some of them will cut grass,\rq{}
	 \sn  \textit{énech, chútúk dagubés. Énech chúblo lowas, énech chúlak éménab, inap chúne-stretimu yah étúh.}\\
	 \lq  some will take out bamboo roots, some will cut trees, some will smooth out the ground and they will continue until they have fixed up the road.\rq{}
	 \sn
	 \gll \textbf{Bai} \textbf{wotak} \textbf{ch}-\textbf{ú}-\textbf{tanomoli} \textbf{gen}.\\
	\textsc{fut} still \textsc{subj}.\textsc{ncl}8.\textsc{pl}-\textsc{irr}-return:\textsc{ven} again\\
	\glt \lq And then they will return.' (\cite[126]{ConradWigoga1991}, glosses added)
\end{exe}

\il{Kekchí|(}\is{subordination|(}In Kekchí, the sequential use is often understood as marking an immediate or sudden development, as in (\ref{exEventSequencingKekchi}). In structural terms, it is noteworthy that it involves a form \textit{toja\rq{}} (or its regional variants \textit{toje\rq}, \textit{tojo\rq}), which is an irregular merger of the \textsc{still} expression \textit{toj} and the subordinator \textit{naq}, and which also forms an integral part of a near past construction (\Cref{sectionRemotenessPast}).\is{remoteness} I return to this point below.\pagebreak

\begin{exe}
	\ex Kekchí\label{exEventSequencingKekchi}\\
	\gll Naq x-e'-raq-e' chi x-b'anunk-il, \textbf{toja'} \textbf{naq} \textbf{x}-\textbf{e'}-\textbf{ok} \textbf{chi} \textbf{x}-\textbf{k'at}-\textbf{b'al}.\\
	\textsc{subord} \textsc{pfv}-3\textsc{pl}-finish-\textsc{pass} \textsc{comp} \textsc{poss}.3\textsc{sg}-do-\textsc{nmlz} still.\textsc{subord} \textsc{subord} \textsc{pfv}-3\textsc{pl}-start \textsc{comp} \textsc{pfv}.3\textsc{sg}-burn-\textsc{nmlz}\\
	\glt \lq When they finished doing that, \textbf{then} (\textbf{immediately}) \textbf{they} \textbf{began} \textbf{to} \textbf{burn} \textbf{it}.' \parencite[468]{Kockelman2020}
\end{exe}\il{Kekchí|)}\is{subordination|)}

In the case of \ili{Mundang} \textit{ɓà} all relevant attestations in \citeauthor{Elders2000}'s (\citeyear{Elders2000}) grammar involve future situations, as in (\ref{exEventSequencingMundang1}). This is another point that I return to below.

\begin{exe}
	\ex \ili{Mundang}\label{exEventSequencingMundang1}\\
		\gll Sóó mó cḭ̄ḭ̄ ɓè, \textbf{nà} \textbf{pʊ̀ò}-\textbf{n} \textbf{mə̀ngwáá} \textbf{ɓà}.\\
millet \textsc{sit} germinate \textsc{ant} 1\textsc{pl}.\textsc{incl} work-\textsc{ipfv} first\_plowing still\\
\glt \lq Quand le mil aura germé, nous ferons le premier labour. [Once the millet has germinated, \textbf{we will do the first plowing}.]\rq{ }\parencite[383]{Elders2000}
\end{exe}

Lastly, given that the use I discuss here does not involve \isi{persistence} of one and the same situation, it does not require a viewpoint\is{aspect} that is fully contained in it, nor need the situation necessarily extend in time. On the contrary, this function is commonly found in conjunction with a \isi{perfective} viewpoint,\is{aspect} as well as with achievement predicates,\is{actionality} such as in (\ref{exEventSequencingNgambay}–\ref{exEventSequencingKekchi}).

\subsubsubsection{Discussion} 
The sequencing function bears an obvious similarity to the concept of \textsc{still} in that it involves the accumulation of temporal intervals. Thus, comparable to how the concept of \textsc{still} involves an advanced runtime of the situation depicted in the clause, the use I discuss here involves the accumulation of sub-parts of one contiguous and overarching situation. This is schematically illustrated in \Cref{figureSequencing}, using a \isi{perfective} viewpoint\is{aspect} for ease of illustration.
	
\begin{figure}\is{perfective}
	\centering
	\begin{subfigure}[b]{0.48\linewidth}\is{topic time}
		\centering

		\begin{tikzpicture}[node distance = 0pt]
			\node[mynode, text width=\breit, fill=cyan, very near start] (spain){Situation};
			\node[mynode, fill=none, text width=2*\hoehe, right= of spain] (france){\small{$\Diamond$}\neg Sit.};	
			\draw[-, densely dashed] ($(spain.north east)+ (0,0) $) to ($(spain.south east)+ (0,-0.5*\hoehe) $);
			\draw[-, densely dashed] ($(spain.north east)+ (-1*\hoehe,0) $) to ($(spain.south east)+ (-1*\hoehe,-0.5*\hoehe) $) node [below, align=center, label distance=0, xshift=0.5*\hoehe] {\strut{}Topic\\time\strut};
			\draw[-latex, line width=0.5pt]  (spain.south west) to  ($(france.south east)+(1ex,0)$) node [right] {t};
			\draw[-latex, line width=0.5pt]  (spain.south west) to  ($(spain.south east)+(-\hoehe,0)$);
			\node[align=left,label distance=0, anchor=north west] at ($(spain.south west)+(0,-0.5*\hoehe)$) {\strut{}Prior\\runtime\strut};		
		\end{tikzpicture}
		\subcaption{\textsc{still}}
	\end{subfigure}
	\begin{subfigure}[b]{0.48\linewidth}
		\centering

			\begin{tikzpicture}[node distance = 0pt]

				\node[mynode, fill=cyan, text width=2*\hoehe, very near start] (spain) {Sit.\textsubscript{1}};
						
			\node[mynode, fill=cyan, text width=2*\hoehe, anchor=west] (france)
		at (spain.east)	{Sit.\textsubscript{2}};	
			\draw[-, densely dashed] ($(france.north east)+ (0.5*\hoehe-0.2pt,0) $) to ($(france.south east)+ (-0.2pt+0.5*\hoehe,-0.5*\hoehe) $);
			\draw[-, densely dashed] ($(france.north east)+ (-0.5*\hoehe+0.2pt,0) $) to ($(france.south east)+ (0.2pt-0.5*\hoehe,-0.5*\hoehe) $)	node 	[below, align=center, label distance=0, xshift=0.5*\hoehe] {\strut{}Topic\\time\strut};
			\draw[thick] (france.north east) to (france.south east);
			\draw[-] (france.north west) to (france.south west);
			\draw[-latex, line width=0.5pt]  (spain.south west) to  ($(france.south east)+(\hoehe+3ex,0)$) node [right] {t};
		\draw[-latex, line width=0.5pt]  (spain.south west) to  ($(france.south west)+(-0.2pt,0)$);
\node[align=left,label distance=0, anchor=north west] at ($(spain.south west)+(0,-0.5*\hoehe)$) {\strut{}Prior\\event\strut};			
		\end{tikzpicture}
		\subcaption{Sequencing (with \textsc{pfv})}		
	\end{subfigure}
	\caption{Graphic illustration of sequencing use\label{figureSequencing}}
\end{figure}

Unfortunately,  the history of most items in \Cref{tableEventSequencing} remains obscure, except for \ili{Kekchí} \textit{toj}. Therefore I can only make educated guesses concerning more concrete links to the same expressions as markers of phasal polarity. To begin with, there is the distinct possibility that the two functions constitute separate developments out of a common ancestor, insofar as relevant sources figure in the known etymologies of other \textsc{still} expressions. For instance, Hebrew\il{Hebrew, Modern} \textit{ʕadayin} is most likely from \ili{Aramaic} \textit{ʔedayin} \rq{}then, thereupon\rq{ }(\cite{TsirkinSadan2019} and references therein) and, outside of my sample, Levantine Arabic\il{Arabic, Levantine} \textit{baʕd} has its lexical source in \lq behind, after\rq{ }\parencite{TaineCheikh2016}. Another conceivable nexus lies in an iterative/restitutive function, as there is a known connection between \lq again\rq{ }and the marking of sequential events (e.g. \cite{MoyseFaurie2012}; \cite{Zhang2017}). This \isi{repetition} function is attested for \ili{Kaba} \textit{bbáy} and its Laka and Ngambay cognates (\cite[426]{Moser2004}; \cite[118–119]{Vandame1963}; examples throughout \cite{Keegan2014}). In this scenario, an example like (\ref{exSequencingNgambay2}) could constitute a critical context, in that it involves both a restitution of the subject's location and what can be read as the preparation and culmination stages of one overarching event of cyclical motion.

\begin{exe}
	\ex Ngambay\label{exSequencingNgambay2}\\
	\gll M-āo ɓá m-ā \textbf{ɗè} \textbf{ɓéi}.\\
	\textsc{subj}.1\textsc{sg}-go first \textsc{subj}.1\textsc{sg}-\textsc{fut} come still\\
	\glt \lq Je pars et je reviendrai. [I'm going and \textbf{I'll come back}.]\rq{}\\(\cite[119]{Vandame1963}, glosses added)
\end{exe}

\is{prospective|(}Featuring a future event, example (\ref{exSequencingNgambay2}) also illustrates yet another possible source for the sequencing use. Thus, the relevant \ili{Kaba}/Laka/Ngambay expressions, as well as \ili{Mundang} \textit{ɓà}, also have a prospective \lq eventually\rq{ }function. This use shares a key ingredient with the one I discuss here: it marks the continuity\is{persistence} of an overarching course of events that culminates in the situation depicted in the clause. As I discuss in \Cref{sectionProspective}, the prospective function finds a motivated source in phasal polarity \textsc{still} plus contexts of a persistent\is{persistence} prediction, where scope ambiguity can lead to a reanalysis of \lq can/must still …\rq{ }as \lq can/must … eventually\rq{}. In other words, there is a conceivable cline running from \textsc{still} via prospective \lq eventually\rq{ }to the sequencing function, and throughout this cline there is stable semantic core of temporal advancement. This would yield a motivated explanation for the fact that all relevant examples of \ili{Mundang} \textit{ɓà} feature unactualised, pending situations. Although the two functions can be differentiated synchronically on the basis of their inflectional compatibilities (the \lq eventually\rq{ }use, but not the sequencing one requires either the potentialis or the optative mood\is{mood}), there are attestations like (\ref{exSequencingMundang2}) which are compatible with both.

\begin{exe}
	\ex \ili{Mundang}\label{exSequencingMundang2}\\
	\gll Mō tə́ráŋ tə̀kɪ́ɪ́ hʊ̰́ó̰ pɪ̄ɪ̄ kō \textbf{mō} \textbf{tɪ́ŋ} \textbf{zə̀}-\textbf{n}-\textbf{yā} \textbf{ɓà}.\\
	\textsc{subj}.2\textsc{sg} dilute.\textsc{opt} gruel \textsc{dem} first \textsc{conj} \textsc{subj}.2\textsc{sg}.\textsc{opt} start.\textsc{opt} drink-\textsc{inf}-\textsc{poss}.3\textsc{sg} still\\
	\glt \lq Dilue d’abord cette bouillie, avant que tu commences à la boire.' [Dilute this gruel first, and \textbf{then you start drinking it.]} \parencite[384]{Elders2000}
\end{exe}
\il{Ngambay|)}\il{Laka|)}\is{prospective|)}

\il{Kekchí|(}\is{subordination|(}Lastly, a somewhat different pathway appears to be at play in the case of Kekchí \textit{toj}. Earlier I pointed out that the sequencing construction in this language involves \textit{toja’}\sim\textit{toje’}\sim\textit{tojo’}, an irregular merger of the \textsc{still} expression \textit{toj} and the subordinator \textit{naq}. Its second constituent part is another, uneroded instance of \textit{naq}. This was seen in (\ref{exEventSequencingKekchi}), repeated below. Example (\ref{exEventSequencingKekchi2}) is another illustration.

\begin{exe}
	\exr{exEventSequencingKekchi}Kekchí\\
	\gll Naq x-e'-raq-e' chi x-b'anunk-il, \textbf{toja'} \textbf{naq} \textbf{x}-\textbf{e'}-\textbf{ok} \textbf{chi} \textbf{x}-\textbf{k'at}-\textbf{b'al}.\\
	\textsc{subord} \textsc{pfv}-3\textsc{pl}-finish-\textsc{pass} \textsc{comp} \textsc{poss}.3\textsc{sg}-do-\textsc{nmlz} still.\textsc{subord} \textsc{subord} \textsc{pfv}-3\textsc{pl}-start \textsc{comp} \textsc{pfv}.3\textsc{sg}-burn-\textsc{nmlz}\\
	\glt \lq When they finished doing that, \textbf{then} (\textbf{immediately}) \textbf{they began to burn it}.' \parencite[468]{Kockelman2020}

	\ex\label{exEventSequencingKekchi2}
	\gll Xb’eenwa t-oo-wa’aq \textbf{tojo’} \textbf{naq} \textbf{t}-\textbf{oo}-\textbf{xik}.\\
	first  \textsc{prosp}-1\textsc{pl}--eat still.\textsc{subord} \textsc{subord} \textsc{prosp}-1\textsc{pl}-leave\\
	\glt \lq Primero comeremos y luego nos vamos. [We'll eat first, \textbf{then we'll leave}.]\rq{ }(\cite[193]{VocabularioKechi2004}, glosses added)
\end{exe}

\textit{Toja’} and its dialectal variants also form an integral part of a near past construction, which is illustrated in (\ref{exEventSequencingKekchi3}). As I discuss in \Cref{sectionRemotenessPast}, one possible interpretation of the latter construction involves \textit{toj} in yet another function, namely as a temporal \isi{restrictive} \lq no earlier than\rq{}, i.e. \lq (it is) not until now/then (when) …\rq{}. A sequencing function can easily be derived from the same meaning. In this case, the link to phasal polarity \textsc{still} is a fairly indirect one. As \textcite{Kockelman2020} points out, across all these uses, however, there is again a stable core in the form of temporal progress leading up to a certain point in time.\pagebreak

\begin{exe}
	\ex Kekchí\label{exEventSequencingKekchi3}\\
	\gll \textbf{Toje'} x-c’ulun arin Cobán.\\
	still.\textsc{subord} \textsc{pfv}.3\textsc{sg}-arrive here C.\\
	\glt \lq Hace poco que vino aquí a Cobán. [He \textbf{just recently} came here to Cobán.]' (\cite[202]{EachusCarlson1980}, glosses added)
\end{exe}\il{Kekchí|)}\is{sequencing|)}\is{subordination|)}

\subsection{Degrees of temporal remoteness}\is{remoteness|(}
\label{sectionRemoteness}
In this subsection, I discuss uses of \textsc{still} expressions that pertain to the degrees of temporal remoteness of a past or future situation. The examples in (\ref{exRemotenessPastTunisianArabic1},\il{Arabic, Tunisian} \ref{exRemotenessWesternShoshoni}) are illustrations. In (\ref{exRemotenessPastTunisianArabic1}), Tunisian Arabic \textit{māzāl}\il{Arabic, Tunisian} combines with an adverbial clause\is{subordination|(}\is{temporal clause|(} introduced by \textit{kīf} \lq when/how' that features a verb in the \isi{perfective} \isi{aspect} paradigm. This collocation signals an immediate past situation. In (\ref{exRemotenessWesternShoshoni}), Western Shoshoni \textit{ekisen} narrows down the future time frame to one in the vicinity of the speaker's now.\il{Shoshoni, Western}

\begin{exe}
	\ex Tunisian Arabic\il{Arabic, Tunisian}\label{exRemotenessPastTunisianArabic1}\\
	\gll Znīxā uxt-i \textbf{māzāl-t} \textbf{kīf} \textbf{xd̠ā-t}.\\
	Z. sister-\textsc{poss}.1\textsc{sg} still-3\textsc{sg}.\textsc{f} when/how take\_spouse.\textsc{pfv}-3\textsc{sg}.\textsc{f}\\
	\glt \lq Meine Schwester Zulayxa hat eben erst geheiratet. [My sister Zulayxa \textbf{just recently got married}, lit. my sister Zulayxa is still when/how she married.]' (\cite[651]{Singer1984},  glosses added)
    \is{subordination|)}\is{temporal clause|)}
	
	\ex Western Shoshoni\il{Shoshoni, Western} \label{exRemotenessWesternShoshoni}\\
	\gll \textbf{Ekisen} tahma-to'i-han-\textbf{to'i}.\\
	still spring-emerge-\textsc{res}-\textsc{fut}\\
	\glt \lq \textbf{Pretty soon} it’s going to be spring.\rq{ }(\cite[150]{CrumDayley1993}, glosses by \cite[68]{McLaughlin2012})
\end{exe}

As in (\ref{exRemotenessPastTunisianArabic1}, \ref{exRemotenessWesternShoshoni}), the relevant readings all arise in specific collocational contexts, which I lay out in more detail below. Note that in I do not consider transparently compositional cases in which a \textsc{still} expression has scope over a stative predicate that expresses a degree of temporal distance, as in (\ref{exRemotenessEnglish}).

\begin{exe}
	\ex \ili{English}\label{exRemotenessEnglish}\\
	\textit{\textbf{It is still recent that} the Gators found out they are taking on Arkansas and Texas A\&{}M this season.} (found online)%\footnote{\url{https://www.wruf.com/headlines/2020/08/18/gators-keep-prepping-some-have-yet-to-report/} (08 May, 2022).}
\end{exe}

\il{Maybrat|(}That said, a borderline case is found in Maybrat. In this language,  the combination of the \textsc{still} expression \textit{fares} and \textit{tna} \lq new, recently' is a transparently compositional way of marking a near past situation; see (\ref{exRemotenessMaybrat}). However, this collocation has become strongly conventionalised and, what is more, the originally bipartite structure appears to have merged to a single clause.\is{syntax} Lastly, for a discussion of \textsc{still} expressions as focus-sensitive\is{focus} scalar\is{scale} operators \lq as late/early as, as far removed as\rq{}, see \Cref{sectionTimeScalar}.
 
\begin{exe}	
	\ex Maybrat\label{exRemotenessMaybrat}\\
	\gll Au ma-ama \textbf{tna} \textbf{fares} iye.\\
	3\textsc{u} 3\textsc{u}-come new still too\\
	\glt \lq She also came \textbf{very} \textbf{recently}. (lit. She came, it is still recent too.)'\\\parencite[164]{Dol2007}
\end{exe}\il{Maybrat|)}

In what follows, I first examine the predominant case in the sample, namely the marking of temporal proximity in the past (\Cref{sectionRemotenessPast}). I then move on to a discussion of future-oriented uses (\Cref{sectionRemotenessfuture}).

\subsubsection{Past proximity}
\label{sectionRemotenessPast}
\subsubsubsection{Introduction}\is{subordination|(}\is{temporal clause|(} The Tunisian Arabic example (\ref{exRemotenessPastTunisianArabic1}),\il{Arabic, Tunisian} repeated below, illustrates a past degrees-of-remoteness construction centred around a \textsc{still} expression. Here, \textit{māzāl} combines with an adverbial clause that contains a verb in the \isi{perfective} \isi{aspect} paradigm, together signalling that the situation in question took place just recently.

\begin{exe}
	\exr{exRemotenessPastTunisianArabic1} Tunisian Arabic\il{Arabic, Tunisian}\\
	\gll Znīxā uxt-i \textbf{māzāl}-\textbf{t} \textbf{kīf} \textbf{xd̠ā-t}.\\
	Z. sister-\textsc{poss}.1\textsc{sg} still-3\textsc{sg}.\textsc{f} when/how take\_spouse.\textsc{pfv}-3\textsc{sg}.\textsc{f}\\
	\glt \lq Meine Schwester Zulayxa hat eben erst geheiratet [My sister Zulayxa \textbf{just recently got married}, lit. my sister Zulayxa is still when/how she married].' (\cite[651]{Singer1984},  glosses added)
\end{exe}\is{subordination|)}\is{temporal clause|)}

Like the Tunisian \textit{māzāl} \mbox{\textit{kī}(\textit{f})} construction,\il{Arabic, Tunisian} all relevant instances in the sample data involve a near past. This constitutes a crucial difference to the marking of degrees of temporal remoteness in the future;  see \Cref{sectionRemotenessfuture}.

\subsubsubsection{Distribution in the sample}\largerpage[2]
\Cref{tableRemotenessPast} lists the seven expressions in my sample that are involved in signalling past proximity. As can be gathered, this function is attested in genetically unrelated languages from Africa, North America, and Papunesia. In terms of their composition, two cases (featuring Tunisian Arabic\il{Arabic, Tunisian} \textit{māzāl} and \ili{Kekchí} \textit{toj}) involve syntactically\is{syntax} complex constructions.

\begin{table}
\caption{Near past collocations}
\label{tableRemotenessPast}
\small
\begin{tabularx}{\textwidth}{lllQl}
\lsptoprule
 Macro-area & Language & Exp. & Collocate & Appx. \\
\midrule
Africa & B.-G. Datooga\il{Datooga, Barabayiiga-Gisamjanga} & \textit{údu}- & Non-future \isi{tense} plus telic predicate&  \ref{appendixDatoogaNearPast}\\
	   & \ili{Bende} & \textit{syá}- & Perfective \isi{aspect}\is{perfective} & \ref{appendixBendeNearPast}\\
	   & Tunisian Arabic\il{Arabic, Tunisian} & \textit{māzāl} &  \textit{Kī}(\textit{f})\is{subordination} \lq when\rq{ }plus perfective \isi{aspect} or participle& \ref{appendixTunisianArabicNearPast}\\
North America & \ili{Gitxsan}/\ili{Nisga'a} & \textit{k̠'ay} & Achievement predicate or \textit{hlis} \lq finish' & \ref{appendixGitsxanNearPast}\\
			  & \ili{Maricopa} & -\textit{haay} & Perfective \isi{aspect} or achievement\is{actionality} predicate& \ref{appendixMaricopaNearPast}\\
			  & \ili{Kekchí} & \textit{toj} & \textit{Naq} \lq\textsc{subord}'\is{subordination} plus perfective aspect& \ref{appendixKekchiImmediatePast}\\
Papunesia & Western Dani\il{Dani, Western}  & \textit{awo} & Intermediate past or non\hyp finite medial clause &  \ref{appendixWesternDaniNearPast}\\
\lspbottomrule
\end{tabularx}
\end{table}

\is{tense|(}\is{aspect|(}
In the following discussion, I first examine the aspectual and temporal characteristics of the near past use in some more detail, to then address the question of its functional motivation and lastly sketch out conceivable diachronic pathways.

\subsubsubsection{A closer look: The viewpoint}\is{anterior|(}\largerpage[2.25]
Like the Tunisian Arabic\il{Arabic, Tunisian} example (\ref{exRemotenessPastTunisianArabic1}), all attestations of the near past use appear to involve an anterior viewpoint, that is to say, a \isi{topic time} which is fully contained in a situation's aftermath (\Cref{sectionTenseAspect}). Against this backdrop, it is worthwhile taking a brief look at the collocates listed in \Cref{tableRemotenessPast} and their position in the respective tense-aspect systems to lay the ground for the following discussion. As this summary will show, all cases involve verbal paradigms that have an anterior viewpoint as one of their readings. In some cases, the aspectual viewpoint is determined by the predicate's telicity,\is{actionality}\is{telicity} or boundedness on the lexical/phrasal level (\Cref{sectionTenseAspect2}).{\interfootnotelinepenalty=10000\footnote{Outside of my sample, \ili{Ngemba} (Bantoid) \textit{wí} constitutes what, at first sight, appears to be a counterexample. This auxiliary-like marker, which often translates as \lq still\rq{ }takes (as far as inflected verbs are concerned) only \isi{imperfective} complements, yet it can have a \lq have just Verb-ed\rq{ }reading. However, as \textcite[220]{Mekamgoum2021} states, \textit{wí} \lq\lq is not semantically dedicated to phasal polarity as it originally conceptualises the notion of \lq just\rq{ }or \lq{}venir juste de\rq{ }in French; it refers to a non-continuative event that takes place a short time prior to the reference time\rq\rq{}. In other words, the relevant readings are not determined by the aspectual inflection of \textit{wí}'s syntactic\is{syntax} complement, but by the viewpoint of the larger clause.\is{aspect} In broad strokes, an \isi{imperfective} viewpoint yields persistence,\is{persistence} whereas \lq have/had just Verb-ed\rq{ }arises in non-imperfective contexts.}}

\is{perfective|(}\is{anterior|(}Thus, in Bende,\il{Bende} Kekchí,\il{Kekchí} Maricopa,\il{Maricopa} and Tunisian Arabic,\il{Arabic, Tunisian} near past construals mainly involve with what is termed the \lq\lq perfect(ive)" in the respective descriptive traditions. For \ili{Bende}, Kekchí,\il{Kekchí} and Tunisian Arabic,\il{Arabic, Tunisian} the available grammatical descriptions indicate that these inflections have an anterior viewpoint among their functions (\cite[92–95]{Kockelman2010}, \citeyear{Kockelman2020}; \cite[s.v. \textit{F10 Sí-bhendé, Sí-tóngwé}]{NurseData}; \cite[298–300]{Singer1984}). \ili{Maricopa} is harder to judge, but the examples throughout \textcite{Gordon1986} do not preclude an anterior viewpoint. In Tunisian Arabic,\il{Arabic, Tunisian} the near past use is also attested with certain predicative participles, which likewise allow for anterior readings (\cite[293]{RittBenmimoun2014}; \cite[303–308]{Singer1984}). The case of \ili{Gitxsan}/\ili{Nisga'a} \textit{k̠'ay} plus \textit{hlis} \lq finish' can be subsumed here as well.

In Western Dani,\il{Dani, Western} the near past use of \textit{awo} is attested in two contexts. The first of these consists of verbs in the intermediate past inflection, as in (\ref{exRemotenessWesternDani1}). The exact range of viewpoints that this verbal paradigm has is unclear, but it stands in opposition to specifically \isi{continuous}\is{progressive} and \isi{habitual} constructions \parencite[ch. 5]{Barclay2008}; I address the question of its status as a tense below.

\begin{exe}
	\ex Western Dani\il{Dani, Western} \label{exRemotenessWesternDani1}\\
	\gll \textbf{Ndi} \textbf{awo} \textbf{aret} \textbf{k}-\textbf{inom} \textbf{nogo} \textbf{yo}-\textbf{gotak} kenok, roti noo-rak meek o.\\
	and still \textsc{intens} 2\textsc{pl}-with sleep do-\textsc{interm}.\textsc{pst}:\textsc{subj}.2\textsc{pl} if bread eat-\textsc{intentive} cannot \textsc{dm}\\
	\glt \lq \textbf{And if you have just slept with them} (women), you must not eat the bread.' \parencite[175]{Barclay2008}
\end{exe}

\is{actionality|(}\is{telicity|(}
The second Western Dani\il{Dani, Western}  context consists of an uninflected medial clause in a clause chain. There is only one such instance in my data; see (\ref{exRemotenessWesternDaniMedial}). This example features a telic predicate with an anterior viewpoint in the past, i.e. a pluperfect reading. Without further data, it is not clear whether \textit{awo} could also yield a near past in the same syntactic\is{syntax} environment plus an atelic predicate, or whether this combination would invariably yield phasal polarity \textsc{still}. Other examples of medial clauses (without \textit{awo}) throughout \citeauthor{Barclay2008}'s (\citeyear{Barclay2008}) grammar suggest that there is at minimum a correlation between telicity and viewpoint.

\begin{exe}
\ex Western Dani\il{Dani, Western} \label{exRemotenessWesternDaniMedial}\\
\gll It in-eebe awo tawe paga iniklom no-mba-kwi, \textbf{kwe} \textbf{ogonggelo} \textbf{awo} \textbf{imbirak} \textbf{lambun}-\textbf{ggo} \textbf{logonet} ogonggelo kun-ik ee-ke menggi kwak, inik ee'-na-kwi … eer-eegwaarak nogo n-iniki aber-ak wona-ge agarik o.\\
3\textsc{pl} \textsc{poss}.3\textsc{pl}-body still unmarried at enjoyment \textsc{obj}.1\textsc{sg}-think-\textsc{pl} woman husband still with.\textsc{du} join-\textsc{pl} \textsc{sim}.\textsc{ss} husband join-\textsc{adj} do-\textsc{sg} \textsc{hab}.3\textsc{sg} like heart do-\textsc{obj}.1\textsc{sg}-\textsc{subj}.\textsc{pl} {} do-\textsc{rem}.\textsc{pst} \textsc{anaph} \textsc{poss}.1\textsc{sg}-heart think-\textsc{adj} \textsc{cop}-\textsc{sg} \textsc{hab}.1\textsc{sg} \textsc{cont}\\
\glt \lq When they were still unmarried (young) they loved me, \textbf{while the woman had just been united to a husband}, like she (a woman) usually unites with a husband, they loved me … Concerning all those things they did long ago, I am always remembering them.' (Peter Barclay, p.c.)
\end{exe}

This brings me to those cases where the aspectual viewpoint is directly linked to the telicity of the predicate. Thus, in Barabayiiga-Gisamjanga Datooga,\il{Datooga, Barabayiiga-Gisamjanga} the \textsc{still} expression \mbox{\textit{údu}-} leads to a neutralisation of most tense-aspect oppositions to future vs. non-future. In this configuration, atelic predicates with affirmative polarity predictably bring about an \isi{imperfective} viewpoint plus \textsc{still}, whereas telic predicates in the affirmative non-future yield the near past construal (telic predicates in the future yield \lq again').\footnote{As far as can be judged from the examples in \textcite{Kiessling2000} and \textcite[177–179]{Rottland1982}, telic predicates without \mbox{\textit{údu}-} can have habitual,\is{habitual} progressive\is{progressive} and bounded past interpretations,\is{perfective}\is{anterior} depending on context.} Example (\ref{exRemotenessDatooga}), featuring the adverb \textit{qámnà} \lq now', suggests that the viewpoint in this case is anterior.

\begin{exe}
\ex Barabayiiga-Gisamjanga Datooga\il{Datooga, Barabayiiga-Gisamjanga}\label{exRemotenessDatooga}\\
\gll G-á-bày èa míi dá-yíi hà \textbf{g}-\textbf{út}-\textbf{tà}-\textbf{yíi} \textbf{qámnà}.\\
\textsc{aff}-\textsc{subj}.3-be\_early \textsc{conj} \textsc{neg}.\textsc{cop} \textsc{subj}.1\textsc{sg}-hear \textsc{dm} \textsc{aff}-still-\textsc{subj}.1\textsc{sg}-hear now\\
\glt \lq I've never heard this; \textbf{I've just heard it now}.' \parencite[431]{Mitchell2021}
\end{exe}

Variations on the same theme can be found in two of the relevant North American languages, \ili{Maricopa} and Gitxsan/Nisga'a.\il{Gitxsan}\il{Nisga'a}  In Maricopa,\il{Maricopa} the per\-fec\-tive--im\-per\-fec\-tive opposition in realis \isi{mood} only applies to durative predicates. With punctual predicates, realis \isi{mood} without overt aspect inflection invariably yields a bounded viewpoint \parencite[102–103]{Gordon1986}, which is one of the contexts in which the near past reading arises. In the same vein, achievement predicates in \ili{Gitxsan}\slash\ili{Nisga'a}  per default go along with a non-imperfective viewpoint \parencite{JohansdottirMatthewson2007} and nothing precludes the assumption that this includes an anterior perspective.\is{actionality|)}\is{telicity|)}\is{perfective|)}\is{anterior|)}

\subsubsubsection{A closer look: No true past tense}
The question of the aspectual viewpoint leads me to another shared characteristic across near past uses. In (\ref{exRemotenessWesternDaniMedial}) above, the collocation of Western Dani\il{Dani, Western} \textit{awo} plus a medial clause does not relate to utterance time,\is{utterance time} but involves a past-in-the-past reading  \lq …had just been…\rq{}. The past tense itself, in the sense of an anchoring of \isi{topic time} before utterance time\is{utterance time} (\Cref{sectionTenseAspect2}), is established by the following verb in the distant past inflection. Similar cases are attested for the relevant constructions in Gitxsan/Nisga'a,\il{Gitxsan}\il{Nisga'a} Kekchí,\il{Kekchí} and Tunisian Arabic,\il{Arabic, Tunisian} as shown in  (\ref{exRemoteNessGitsxanRunning}–\ref{exTunisianArabicRemotenessTooth}). These examples also illustrate another point, which relates back to the question of an \isi{anterior} viewpoint: where I have data from narrative discourse, the near-past use is exclusively found in background clauses.

\begin{exe}
	\ex \ili{Gitxsan}/\ili{Nisga'a}\label{exRemoteNessGitsxanRunning}\\
	\gll \textbf{K̠'ay} \textbf{hlis} \textbf{bax̠}=\textbf{hl} \textbf{gimxdi}-\textbf{ʼy} win ʼwitxw haʼw-iʼy kyʼoots.\\
	still finish run=\textsc{conn} sister-\textsc{poss}.1\textsc{sg} \textsc{subord} arrive go\_home-1\textsc{sg} yesterday\\
	\glt \lq \textbf{My sister had just finished running} when I came home yesterday.' \parencite[69]{Aonuki2021}
	
 	\ex \ili{Kekchí}\label{exRemotenessKekchiJustMarried}\\
	Context: About the time frame of a scary event.\\
	\gll Toj maak'a'-q qa-kok'al, \textbf{toja'} \textbf{k}-\textbf{oo}-\textbf{sumlaak}.\\
	still \textsc{neg}.\textsc{exist}-\textsc{non}.\textsc{specific} \textsc{poss}.1\textsc{pl-}children still.\textsc{subord} \textsc{pfv}.\textsc{evid}-1\textsc{pl}-marry\\
	\glt \lq We still had no children. \textbf{We had just married}.' \parencite[468]{Kockelman2020}

\ex Tunisian Arabic\il{Arabic, Tunisian} \label{exTunisianArabicRemotenessTooth}\\
	\gll Kun-t ānā \textbf{māzil}-\textbf{t} \textbf{kīf} \textbf{bdī}-\textbf{t} n-umġud̠̣ fī ṭaṛf il-lḥam had̠āya u-ẓaṛṣt-i ṛā-hi tnaṭṛ-it tanṭīṛa waḥd-a.\\
	\textsc{cop}.\textsc{pfv}-1\textsc{sg} 1\textsc{sg} still-1\textsc{sg} when begin.\textsc{pfv}-1\textsc{sg} 1\textsc{sg}-chew.\textsc{ipfv} in piece \textsc{def}-meat(\textsc{m}) \textsc{prox}.\textsc{sg}.\textsc{m} and-molar(\textsc{f})-\textsc{poss}.1\textsc{sg} \textsc{prestt}-3\textsc{sg}.\textsc{f} slip\_out.\textsc{pfv}-3\textsc{sg}.\textsc{f} slip\_out.\textsc{nmlz}(\textsc{f}) one-\textsc{sg}.\textsc{f}\\
	\glt \lq Ich hatte eben erst damit begonnen, auf dem Stück Fleisch herumzubeißen, da flog auch schon mein (Backen-)Zahn im hohen Bogen. [\textbf{I had only just begun} chewing on the piece of meat, when all of a sudden my molar tooth came flying out.]ʼ (\cite[651]{Singer1984}, cited in \cite{FischerEtAlTunisian})
\end{exe}

The observations just made are in line with the fact that the relevant verbal paradigms are not inherently tensed. While the case of B.-G. Datooga \mbox{\textit{údu-}} involves the non-future tense,\il{Datooga, Barabayiiga-Gisamjanga} the only contribution of the latter is that \isi{topic time} is no later than utterance time.\is{utterance time} The one possible exception to the rule is the Western Dani\il{Dani, Western}  \lq\lq intermediate past" inflection illustrated in (\ref{exRemotenessWesternDani1}) above. This form primarily stands in opposition to a distant past that appears to be more strongly dissociated, to use a term from \textcite{BotneKershner2008}.  It is possible that the Western Dani\il{Dani, Western} intermediate past is either purely aspectual (an \isi{anterior} aspect), or that it establishes an ordering between time of utterance\is{utterance time} and the time of the situation. The latter is the analysis made by \textcite{CableGikuyu} for certain \isi{remoteness} inflections in the Bantu language \ili{Gikuyu}, a language not in my sample.\is{tense|)}\is{aspect|)}

\subsubsubsection{Discussion}
I now turn to the question of functional motivation and possible diachronic pathways. Speaking in the most general terms, phasal polarity \textsc{still} and the near past use share a common denominator in the notion of continuity.\is{persistence} Thus, the former highlights the connection between an earlier runtime of an ongoing situation and topic time,\is{topic time} while contrasting it with its possible \isi{discontinuation} (\Cref{secFunctionalDiscussion}). Similarly, the near past use can be understood as depicting a terminated situation as being contiguous with utterance time\is{utterance time} (or some other salient evaluation time), as opposed to a more detached past occurrence.

As far as diachrony is concerned, it is likely that the use of the relevant expressions as exponents of \textsc{still} predates their past function; I address the possible exception of \ili{Kekchí} \textit{toj} below. The strongest support for this assumption comes from Tunisian Arabic \textit{māzāl}\il{Arabic, Tunisian} and \ili{Bende} \mbox{\textit{syá}-}. \textit{Māzāl} transparently goes back to a verbal paraphrase \lq has not ceased',\il{Arabic, Tunisian} which is found as a \textsc{still} expression across several other varieties of Arabic, but without having the near past function there (\cite{FischerEtAlTunisian} and references therein). Similarly, cognates of \ili{Bende} \mbox{\textit{syá}-} serve as phasal expressions across Eastern Bantu, but typically do not have the near past use either (\cite{Abe2015}, \citeyear{Abe2016}). Evidence is slightly weaker for \ili{Maricopa} \mbox{-\textit{haay}}, in that cognates of this marker across River Yuman consistently serve as phasal polarity markers, though \ili{Quechan} \mbox{-\textit{xay}} also has the the additional near past use (\cite[284]{Halpern1946}; \cite[64]{Munro1974}; \cite[312]{MunroEtAl1992}). In the same vein, the cognates of Western Dani\il{Dani, Western} \textit{awo} in its closest relatives appear to uniformly function as persistive\is{persistence} expressions (\cite[60]{Burung2007}; \cite[195]{ZoellnerRiesberg2017}; examples throughout \cite{Burung2017} and \cite{Fahner1979}). Lastly, evidence of the more indirect type comes from Gitxsan/Nisga'a.\il{Gitxsan}\il{Nisga'a} In this language, polar questions containing \textit{k̠’ay} in near past function are only felicitous if the speaker knows that a situation of the type depicted in the clause has indeed taken place. That is, what is at issue is the situation's recency; see (\ref{exRemotenessGitsxanQ}). As \textcite{Aonuki2021} points out, this peculiarity can be understood as a carry-over from the prior time presupposition that forms part of the semantics of \textsc{still}. In other words, it mirrors how \lq are you still sleeping?' presupposes \lq you were sleeping'. 

\begin{exe}
		\ex[]{\ili{Gitxsan}/\ili{Nisga'a}\label{exRemotenessGitsxanQ}}
		\sn[\tiny{\ding{51}}]{Context: Mary’s dog had recently run away. You run into her and see that she has her dog, looking excited and happy.}
		\sn[\#]{Context: … You don’t know if she found it yet.}
		\sn[]{
		\gll \textbf{Ḵ’ay}=t ’\textbf{wa}=s Mary=hl us-t=aa?\\
		still=3 find=\textsc{conn} M=\textsc{assoc} dog-\textsc{poss}.3=\textsc{q}\\
		\glt \lq Did Mary just find her dog?\rq{ }\parencite[71]{Aonuki2021}}
\end{exe}

\is{aspect|(}\is{anterior|(}\is{topic time|(}Against this backdrop, there are at least two conceivable pathways leading from \textsc{still} to a near past construal. Recall from \Cref{secFunctionalDiscussion} that the phasal polarity concept requires a pre-existing situation that i) has been going on through to the time under discussion and which ii) can potentially come to an end. The near past use, on the other hand, seems to always involve an anterior viewpoint. That is, topic time is fully contained in the situation's post-time (\Cref{sectionTenseAspect2}), which is, furthermore, normally an invariable state-of-affairs.\footnote{\lq\lq{}If Mary eats lunch, then there is a state that holds forever after: the state of Mary’s having eaten lunch\rq\rq{ } \parencite[234]{Parsons1990}.}\is{persistence|(} One way of reconciling these two elements is by projecting the notion of persistence away from the situation itself onto the impermanent length of time subjectively\is{subjectivity} characterised by the situation's prior occurrence, such as the post-wedding phase in (\ref{exRemotenessPastTunisianArabic1}, \ref{exRemotenessKekchiJustMarried}). In slightly more technical terms, the \textsc{still} expression acquires the function of a temporal frame adverb, in that it restricts the time under discussion to one within the situation's temporal region (\lq\lq{}the … varying environment which each time span has \lq{}around itself\rq {\rq\rq{}}; \cite[122]{Klein1994}).\footnote{Alternatively, and to the same effect, one could assume that anterior aspect denotes a state brought about by the closure of a situation and that the latter is coerced into a transient property of the S/A argument.}  A schematic illustration of this mapping is given in \Cref{figureNearPast1}. An argument in favour of this scenario comes form Tunisian Arabic,\il{Arabic, Tunisian} where the near past construction involves the \textsc{still} expression \textit{māzāl} as a predicator that takes a temporal clause\is{temporal clause}\is{subordination} as its complement  \lq \textsc{subj} is still when s/he Verb-ed\rq{}, that is to say \lq{}… is still in the time of having Verb-ed'. Assuming continuity\is{persistence} of the temporal region would also explain the use of an event time adverbial in (\ref{exRemotenessTATwoWeeks}). Admittedly, however, this is the only example of \textsc{still}-as-near past with such an adverbial that I am aware of.
 
\setlength{\MinimumWidth}{\widthof{Sit.x}}
\setlength{\MinimumWidthB}{\widthof{Regionx}}
\begin{figure}
	\centering
	\begin{subfigure}[b]{0.48\linewidth}
		\centering
		\begin{tikzpicture}[node distance = 0pt]
			\node[mynode, text width=4.5*\hoehe, fill=cyan, very near start] (spain){Situation};
			\node[mynode, fill=none, text width=\schmal, right= of spain] (france){(\small{$\Diamond$}\neg{ }Sit.)};
			\draw[-, densely dashed] ($(spain.north east)+ (-0.2pt,0) $) to ($(spain.south east)+ (-0.2pt,-0.5*\hoehe) $);
			\draw[-, densely dashed] ($(spain.north east)+ (-1*\hoehe,0) $) to ($(spain.south east)+ (-1*\hoehe,-0.5*\hoehe) $) node [below, align=center, label distance=0, xshift=0.5*\hoehe+0.2pt] {Topic\\time};
			\draw[-latex, line width=0.5pt]  (spain.south west) to  ($(france.south east)+(1ex,0)$) node [right] {t};
		\end{tikzpicture}
	\subcaption{\textsc{still}}
	\end{subfigure}
	\begin{subfigure}[b]{0.48\linewidth}
		\centering
		\begin{tikzpicture}[node distance = 0pt]
			\node[mynode,  anchor=west, fill=cyan, text width=\hoehe,draw=none] (B)  {Sit.}; 
				\node[mynode, text width=3.5*\hoehe, fill=cyan, text opacity=1, fill opacity=.5, anchor=west] at (B.east) (D){Region};
				\draw[-] (B.south east) to (B.north east);
					\node[mynode, fill=white, text width=\schmal,anchor=west] at (D.east) (E){(\small{$\Diamond$}\neg{ }Region)};		
			\draw[-, densely dashed] ($(D.north east)+(-1*\hoehe+0.2pt,0)$) to ($(D.south east)+ (-1*\hoehe+0.2pt,-0.5*\hoehe) $);
			\draw[-, densely dashed] ($(D.north east)+ (-0.2pt,0) $) to ($(D.south east)+ (-0.2pt,-0.5*\hoehe) $);
			\node (TT) [below, align=center, label distance=0] at ($(D.south east)+(-0.5*\hoehe,-0.5*\hoehe) + (0.2pt,0)$) {Topic\\time};
			\draw[-latex, line width=0.5pt]  (B.south west) to  ($(E.south east)+(1ex,0)$) node [right] {t};

		\end{tikzpicture}	
		\subcaption{Near past use}
	\end{subfigure}
		\caption{Schematic illustration of the near past use based on the situation's temporal region
	\label{figureNearPast1}}
\end{figure}

\begin{exe}
	\ex Tunisian Arabic\il{Arabic, Tunisian}\label{exRemotenessTATwoWeeks}\\
	\gll Famma āna \textbf{māzāl} \textbf{kī} \textbf{tʕallam}-\textbf{t} \textbf{il}-\textbf{žimᵊʕtīn} \textbf{it}-\textbf{tāli} kilmit gaḥḥāf. Mā-kun-t-š n-aʕṛaf-ha.\\
	\textsc{exist} 1\textsc{sg} \textsc{still} when learn.\textsc{pfv}-1\textsc{sg} \textsc{def}-week.\textsc{du} to-back word(\textsc{f}) scrounger \textsc{neg}-\textsc{cop}.\textsc{pfv}-1\textsc{sg}-\textsc{neg} 1\textsc{sg}-know.\textsc{ipfv}-3\textsc{sg}.\textsc{f}\\
	\glt \lq For example, \textbf{I only learned} the word \textit{gaḥḥāf} \lq scrounger' \textbf{two weeks ago}. I didn't know it before.' (TuniCo, cited in \cite{FischerEtAlTunisian})
\end{exe}\is{persistence|)}

A different pathway has been proposed by \textcite{Aonuki2021} for Gitxsan/Nisga'a.\il{Gitxsan}\il{Nisga'a}  In her scenario, what is shared between the concept of \textsc{still} and the near past use is the presupposition that the situation's prior runtime abuts topic time. Given that the anterior viewpoint features a perspective fully contained in the situation's post-time, this means that the entire runtime of the situation immediately precedes topic time. Put differently, the near past use goes back to depicting a situation as (if it had been) unfolding until the time under discussion \lq have been Verb-ing until now/then'. This proposal would also be in line with \citeauthor{Abe2015}'s (\citeyear{Abe2015}) translation of several of the relevant \ili{Bende} examples as \lq have just finished'. In slightly more technical terms, this pathway establishes a relationship of adjacency between topic time and the time of the situation; \Cref{figureNearPast2} is a graphic illustration.

\begin{figure}[bth]
	\centering
	\begin{subfigure}[b]{0.48\linewidth}
		\centering
		\begin{tikzpicture}[node distance = 0pt]
			\node[mynode, text width=4.5*\hoehe, fill=cyan, very near start] (spain){Situation};
			\node[mynode, fill=none, text width=\superschmal, right= of spain] (france){(\small{$\Diamond$}\neg{ }Sit.)};
				
				\draw[-latex, line width=0.5pt]  ($(spain.south east)+(-2*\hoehe,0)$) to  ($(spain.south east)+(-1*\hoehe+0.2pt,0)$);			
		
			\draw[-, densely dashed] ($(spain.north east)+ (-0.2pt,0) $) to ($(spain.south east)+ (-0.2pt,-0.5*\hoehe) $);
			\draw[-, densely dashed] ($(spain.north east)+ (-\hoehe-0.2pt,0) $) to ($(spain.south east)+ (-\hoehe-0.2pt,-0.5*\hoehe) $);
	
				\draw[-, densely dashed] ($(spain.north east)+ (-\hoehe-0.2pt,0)$) to ($(spain.south east)+ (-\hoehe-0.2pt,-0.5*\hoehe) + (0.2pt,0)$);

			\node (TT) [below, align=center, label distance=0] at ($(spain.south east)+(-0.5*\hoehe,-0.5*\hoehe) + (0.2pt,0)$) {\strut{}Topic\\time\strut};
			
				\draw[-, densely dashed] ($(spain.north west)+(0.2pt,0)$) to ($(spain.south west)+ (0.2pt,-0.5*\hoehe) $);

\node[align=left,label distance=0, anchor=north west] at ($(spain.south west)+(0,-0.5*\hoehe)$) {\strut{}Prior\\runtime\strut};	
	
			
			\draw[-latex, line width=0.5pt]  (spain.south west) to  ($(france.south east)+(3ex,0)$) node [right] {t};
		\end{tikzpicture}
		\subcaption{\textsc{still}}
	\end{subfigure}
	\begin{subfigure}[b]{0.48\linewidth}
		\centering

		\begin{tikzpicture}[node distance = 0pt]
			\node[mynode, text width=4.5*\hoehe, fill=cyan, very near start] (situation){Situation};

				\draw[-latex, line width=0.5pt]  ($(spain.south east)+(-1*\hoehe,0)$) to  ($(spain.south east)+(0,0)$);
			\draw[-, thick]($(situation.north east)+(-0.4pt,0)$) to ($(situation.south east) + (-0.4pt,-0.5*\hoehe) $);
			\draw[-latex, line width=0.5pt]  (situation.south west) to  ($(situation.south east)+(\hoehe+3ex,0)$) node [right] {t};
			\draw[-, densely dashed] ($(situation.north west)+(0.2pt,0)$) to ($(situation.south west)+ (0.2pt,-0.5*\hoehe) $);
			
			
			\draw[-, densely dashed] ($(situation.north east)+ (\hoehe,0)$) to ($(situation.south east)+ (\hoehe,-0.5*\hoehe) + (0.2pt,0)$) node [below, align=center, label distance=0, xshift=-0.5*\hoehe+0.2pt] {\strut{}Topic\\time\strut};

		\node[align=left,label distance=0, anchor=north west] at ($(spain.south west)+(0,-0.5*\hoehe)$) {\strut{}Total\\runtime\strut};		
		\end{tikzpicture}	
		\subcaption{Near past use}
	\end{subfigure}
	\caption{Schematic illustration of the near past use based on an abutting runtime\label{figureNearPast2}}
\end{figure}

While the two pathways differ in the underlying mechanisms, they both ultimately yield a construal of a subjectively\is{subjectivity} near, terminated situation. What is more, there is significant convergence between the two scenarios: in many contexts a topic time that falls immediately after a situation's runtime will also be fully included in the situation's temporal region.\is{topic time|)}\is{anterior|)}\il{Kekchí|(} In fact, both scenarios may have played some role in the case of Kekchí \textit{toj}.\is{subordination|(} To briefly recapitulate, in this language the near past use involves a construction that consists of the \textsc{still} expression \textit{toj} in clause-initial\is{syntax} position plus a subordinate clause containing a verb in the \isi{perfective} \isi{aspect} inflection. This was seen in (\ref{exRemotenessKekchiJustMarried}) above, and (\ref{exRemotenessKekchi1}) is another illustration. Note that in this use, the sequence of \textit{toj} and the subordinator \textit{naq} is normally reduced to \textit{toja\rq}\sim\textit{toje\rq}\sim\textit{tojo\rq}.

\begin{exe}
	\ex Kekchí\label{exRemotenessKekchi1}\\
	\gll \textbf{Toje'} \textbf{x}-\textbf{c’ulun} arin Cobán.\\
	still.\textsc{subord} \textsc{pfv}.3\textsc{sg}-arrive here C.\\
	\glt \lq Hace poco poco que vino aquí a Cobán. [He \textbf{just recently} \textbf{came} here to Cobán.]' (\cite[202]{EachusCarlson1980}, glosses added)\\
\end{exe}\is{aspect|)}

In its composition, the Kekchí near past construction thus bears a striking structural similarity to the Tunisian Arabic\il{Arabic, Tunisian} pattern \textit{māzāl} \mbox{\textit{kī}\textit{f}} \lq (is) still when …\rq{}. This interpretation finds support in another Mayan language, Tzeltal.\il{Tzeltal} Here, the cognate form \mbox{=\textit{to}} participates in a very similar construction, the only major difference being that deranking is achieved through a non-finite\is{infinitive} form of the verb instead of clausal subordination \parencite[653–656]{Polian2013}; see (\ref{exRemotenessTzeltalA}) for illustration. What is more, \ili{Tzeltal} possesses an analogous construction that is based on the \textsc{already} \is{already}expression \textit{ix} and which signals a near future. The latter pattern, illustrated in (\ref{exRemotenessTzeltalB}), could easily go back to \lq S/A is already in the region of Verb-ing\rq{}.

\begin{exe}
	\ex 
	\begin{xlist}
		\exi{}\ili{Tzeltal}
		\ex \gll Lok\rq{}-el-on=\textbf{to}.\label{exRemotenessTzeltalA}\\
		leave-\textsc{inf}-1=still\\
		\glt \lq Acabo de salir. [I \textbf{just} left.]\rq{}
		
		\ex\il{Tzeltal}\is{already}
		\gll Poxtay-el-on=\textbf{ix}.\label{exRemotenessTzeltalB}\\
		cure-\textsc{inf}-1=already\\
		\glt \lq Estoy a punto de que me curen. [I'm \textbf{about to} be cured.]\rq{}
		\\\parencite[654]{Polian2013}
	\end{xlist}
\end{exe}

At the same time, Kekchí \textit{toj} differs from the expressions discussed up to this point in that it also has a use as a temporal \is{restrictive|(}restrictive marker \lq no earlier than\rq{}. This function surfaces, \textit{inter alia}, in the context of the \isi{perfective} \isi{aspect} inflection (\Cref{sectionTimeScalarRestrictive}). Crucially, the restrictive use of \textit{toj} often blends into a positional or sequential\is{sequencing} reading, such that the situation depicted in its host clause takes place right after the state-of-affairs denoted in its complement has come into existence. This is shown in (\ref{exRemotenessKekchiWhen}).

\begin{exe}
\ex Kekchí\label{exRemotenessKekchiWhen}\\
	\gll Nek-e'-xik sa' li tz'oleb'al \textbf{toj} \textbf{wan} \textbf{r}-\textbf{e} \textbf{waqib'} \textbf{chihab'}.\\
	\textsc{prs}-3\textsc{pl}-go \textsc{prep} \textsc{det} school still \textsc{exist}.3\textsc{sg} 3\textsc{sg}-\textsc{dat} six year\\
	\glt \lq They go to school \textbf{when} \textbf{they} \textbf{are} \textbf{six} \textbf{years} \textbf{old}.' \parencite[466]{Kockelman2020}
\end{exe}

\is{topic time|(}As pointed out by \textcite{Kockelman2020}, the near past function of \textit{toj} could be derived from the restrictive use in conjunction with a contextually retrieved topic time instead of an overt complement \lq (it is) not until now/then (that) S/A has Verb-ed\rq{}. This would align the Kekchí case with \citeauthor{Aonuki2021}'s proposal: if a given time is the earliest that a situation can be depicted as completely over, then the total runtime of the situation immediately precedes this interval.\is{topic time|)} It also yields a motivated explanation for the occasional attestation of the type illustrated in (\ref{exRemotenessKekchiCough}) throughout \citeauthor{SamJuarezEtAl2003}'s (\citeyear{SamJuarezEtAl2003}) dictionary. Here, \textit{toja\rq} (or one of its variants) goes together with an \isi{imperfective} present, marking the onset of a new situation.

\begin{exe}
	\ex Kekchí\label{exRemotenessKekchiCough}\\
	\gll \textbf{Toja\rq} yo chi nume\rq{}k in-\rq{}oj.\\
	still.\textsc{subord} do.\textsc{prs}.3\textsc{sg} \textsc{comp} pass \textsc{poss}.1\textsc{sg}-cough\\
	\glt \lq Hasta ahora se me está quitando el catarro. [\textbf{Only now} my cold is passing.]\rq{ }(\cite[231]{SamJuarezEtAl2003}, glosses added)
\end{exe}

While deriving the Kekchí near past construction from the restrictive function requires the stipulation of temporal zero anaphora, this would not come out of thin air. Thus, the restrictive use can be traced back to yet another function, namely limitative \lq until\rq{},\is{limitative} which also constitutes the most likely source for \textit{toj} as an exponent of \textsc{still}. Crucially, this development requires the same anaphoric mechanism \lq until now/then, \textit{p}\rq{} > (at topic time) still \textit{p}\rq{} and is also tied to a fixed, clause-initial position.\is{syntax} In addition, examples similar to the one in (\ref{exRemotenessKekchiScared}) may have played a role. Here, \textit{toja\rq} plus \isi{perfective} inflection patterns together with the two preceding clauses in depicting the setting of a new narrative scene. It is conceivable that, at an earlier stage, such constellations were compatible with two readings, one as adverbial modifiers based on the restrictive use  \lq (not until) when thirteen days had passed …\rq{} and an alternative parsing involving two equally ranked assertions \lq thirteen days had just passed (and) …\rq{}.

\begin{exe}
	\ex Kekchí\label{exRemotenessKekchiScared}\\
	Context: Lord B\rq{}alamq\rq{}e had announced he will be back in thirteen days. The preceding scene evolved around certain objects at the house.\\
	\textit{Kach’inaq ix ch’ool aj eechal kab’l. Naxuwak.}\\
	\lq The owner of the house is timid (her heart is small). She is scared.\rq{}
	
	\exi{}
	\gll \textbf{Toja\rq} \textbf{ki}-\textbf{num}-\textbf{e\rq} \textbf{ox}-\textbf{kuja} \textbf{kutan} ki-r-il na-nach\rq{}ok qaawa\rq{} b\rq{}alamaq\rq{}e.\\
	still.\textsc{subord} \textsc{pfv}.\textsc{evid}:3-pass-\textsc{pass} three-ten day \textsc{pfv}.\textsc{evid}:3.\textsc{u}-3.\textsc{a}-see \textsc{prs}.3-near \textsc{hon} B.\\
	\glt \lq \textbf{When thirteen days had passed}, she saw Lord B\rq{}alamq\rq{}e approaching.\rq{ }\parencite[230]{Kockelman2010}
\end{exe}\is{subordination|)}

The relationships and similarities between the various uses of \textit{toj} just outlined are summarised in \Cref{figureKekchiNearPst}. More comparative work on Mayan is necessary to disentangle the mutual influence that each of the functions of this expression and its cognates had on each other throughout their history.

\setlength{\MinimumHeight}{1.2cm}
\setlength{\MinimumWidth}{\widthof{xTemp. limitat.x}}
\setlength{\MinimumWidthB}{\widthof{xcompl.-takingx}}
\begin{figure}\is{syntax}
	\caption{Hypothetical, partial diachronic network for Kekchí \textit{toj}		
	\label{figureKekchiNearPst}}
	\begin{tikzpicture}[node distance=1.5em]
		\node [rectangle, rounded corners, minimum height=\MinimumHeight,text width=\MinimumWidth, draw=black, text centered] (until) {\strut{}Temp. limit.\\\lq{}until\rq\strut};
		\node [rectangle, rounded corners, minimum height=\MinimumHeight,text width=\MinimumWidth, draw=black, text centered, right=of until]  (tscalar) {\strut{}Temp. excl.\\\lq{}not until\rq\strut}; 
	  	\node [rectangle, rounded corners, minimum height=\MinimumHeight,text width=\MinimumWidth, draw=black, text centered, below=of until] (still) {\textsc{still}}; 
		\node [rectangle, rounded corners, minimum height=\MinimumHeight,text width=\MinimumWidth, draw=black, text centered, below=of tscalar] (pst) {\strut\textit{toja'}\\near past\strut};
		\draw [-latex, very thick] (until) to (still);
		\draw [-latex, very thick] (until) to (tscalar);
		\draw [-latex, very thick, densely dotted] (still) to (pst);
		\draw [-latex, very thick, densely dotted] (tscalar) to (pst);
		\draw [ thick, decorate,decoration={brace,amplitude=5pt,raise=1ex}]
 (until.north west)   -- (tscalar.north east)
 node[midway,yshift=4ex]{Complement-taking, mobile};
 		 \draw [thick, decorate,decoration={brace, mirror, amplitude=5pt,raise=1ex}]
 (still.south west)   -- (pst.south east)
 node[midway,yshift=-4ex]{Anaphoric, clause-initial};
	\end{tikzpicture} 
\end{figure}
\il{Kekchí|)}\is{restrictive|)}

\subsubsection{Degrees of remoteness in the future}
\label{sectionRemotenessfuture}
\subsubsubsection{Introduction} 
\il{Shoshoni, Western|(}Having discussed \textsc{still} expressions as markers of temporal proximity in the past, I now turn to uses that are related to degrees of temporal remoteness in the future. These are found with two sample expressions, \ili{Chuwabu} \mbox{=\textit{vi}} and 
Western Shoshoni \textit{ekisen} (\appsref{appendixChuwabuDistantFuture}, \ref{appendixWesternShoshoniNearFuture}). In the \ili{Chuwabu} case, this yields a near future, whereas in the Western Shoshoni case it marks a distant time to come. In what follows, I examine both cases separately.

\subsubsubsection{A closer look and discussion: Western Shoshoni \is{tense|(}\textit{ekisen}} In Western Shoshoni, the \textsc{still} expression \textit{ekisen} can combine with the future tense to narrow down the time frame to a proximal one. Example (\ref{exRemotenessWesternShoshoniRepeated}) is an illustration.

\begin{exe}
	\ex Western Shoshoni \label{exRemotenessWesternShoshoniRepeated}\\
	\gll \textbf{Ekisen} tahma-to'i-han-\textbf{to'i}.\\
	still spring-emerge-\textsc{rslt}-\textsc{fut}\\
	\glt \lq \textbf{Pretty soon} it’s going to be spring.'
	\\(\cite[150]{CrumDayley1993}; glosses by \cite[68]{McLaughlin2012})
\end{exe}\is{tense|)}

This near future reading is not directly related to \textit{ekisen} as an exponent of \textsc{still}, but is a relict of the expression's history. Thus, \textit{ekisen} can be segmented into \textit{eki} \lq now' and an exclusive\is{restrictive} marker \mbox{–\textit{seN}}. In other words, its etymological meaning is \lq now, to the exclusion of other times'. From this, a reading of proximity falls out in a straightforward manner: \textit{ekisen} functions as a time frame adverbial, restricting the time of the anticipated situation to one in the general vicinity of utterance time.\is{utterance time} \textit{Ekisen} in (\ref{exRemotenessWesternShoshoniRepeated}) is thus no different from the transparent \textit{yaningi}\mbox{=\textit{nyali}} in the \ili{Gooniyandi} example (\ref{exRemotenessGooniyandi}).

\begin{exe}
	\ex \ili{Gooniyandi}\label{exRemotenessGooniyandi}\\
	\gll \textbf{Yaningi}=\textbf{nyali} thithiwalimi\\
	now=still/only I\_am\_going\\
	\glt \lq I'm going \textbf{right} \textbf{now}.' \parencite[466]{McGregor1990}
\end{exe}

This interpretation finds support in the available data on Western Shoshoni's close relative Panamint.\il{Panamint} In this language, the cognate item \textit{üküsü} serves as a fully compositional, emphatic version of \textit{ükü} \lq now', with the expected effect in combination with a future \isi{tense} verb. Crucially, however, \ili{Panamint} \textit{üküsü} has not developed into a \textsc{still} expression (cf. \cite[369]{Dayley1989Dictionary}, \citeyear*[300]{Dayley1989Grammar}).\il{Shoshoni, Western|)}

\subsubsubsection{A closer look and discussion: Chuwabu \mbox{=\textit{vi}}}\il{Chuwabu|(}\is{tense|(} The last case to be addressed is that of Chuwabu \mbox{=\textit{vi}}. According to \textcite{Guerois2021}, when this clitic is combined with a future tense verb, it suggests a later occurrence than the same verb without the enclitic. Consequently, it fares better in combination with distal event time adverbials like \lq next week', such as in (\ref{exRemotenessChuwabu}), than with more proximal ones like \lq today' or \lq tomorrow'. Based on the available data, it is not clear what motivates this peculiarity. Possibly, this is actually a case of an \lq eventually' reading\is{prospective} (\Cref{sectionProspective}). Alternatively, it might be in some way related to \mbox{=\textit{vi}} in its additional function as an exclusive\is{restrictive} operator \lq only\rq{ }(\Cref{sectionExclusive}), placing emphasis on the future tense.

\begin{exe}
	\ex\label{exRemotenessChuwabu}
	\gll Ddi-\textbf{neel}-óó-gulá=\textbf{vi} má-fúgi sumaán' ééjw' éé-n̩-d-a=wo\\
	\textsc{subj}.1\textsc{sg}-\textsc{fut}.\textsc{aux}-\textsc{ncl}15(\textsc{inf})-buy=still \textsc{ncl}6-banana \textsc{ncl}9.week \textsc{dem}.\textsc{ncl}7/9 \textsc{subj}.\textsc{ncl}7/9-\textsc{ipfv}-come-\textsc{rel}=\textsc{ncl}16(\textsc{loc})\\
	\glt \lq I will buy bananas next week.' \parencite[191]{Guerois2021}
\end{exe}
\il{Chuwabu|)}\is{remoteness|)}\is{tense|)}

\section{Temporal connectives and modifiers}\label{sectionConnectiveFrameSetters}\is{connective|(}
\subsection{Introduction}
In the following subsections I discuss various kinds of uses within the larger domain of temporal connectors and modifiers. I first survey instances in which a \textsc{still} expression serves as a modifier of temporal adverbials (\Cref{sectionTemporalFrameSubconstituent}). Such uses include, among other things, functions as time-scalar additive operators \lq as early/late as, as far removed as\rq{}.\is{scale} Subsequently, I discuss uses pertaining to the modification of temporal clauses (\Cref{sectionTemporalSubordination}),\is{subordination}\is{temporal clause} for instance \textsc{still} expressions as markers of simultaneous duration.\is{simultaneity}

\subsection{Modifications of positional temporal frame expressions}\label{sectionTemporalFrameSubconstituent}
\largerpage
In this subsection, I discuss several uses in which a \textsc{still} expression modifies a positional temporal adverbial. This is typically an adverb in the strict sense, but it can also be another constituent\is{syntax|(} in the same function. While this constituent may be an embedded clause,\is{temporal clause}\is{subordination} it need not be, which is a key difference to the clause-modifying uses I discuss in \Cref{sectionTemporalSubordination}.\is{syntax|)} In the first type of function as an adverbial modifier, \textsc{still} expressions serve as time-scalar\is{scale} \isi{restrictive} operators \lq not until\rq{} (\Cref{sectionTimeScalarRestrictive}). In another type (\Cref{sectionTemporalFrameTT}), the relevant expressions combine with a temporal positional adverbial and signal \lq\lq that the event in question occurs while the time specification is still valid" (\cite[202]{Loebner1989}). In yet a third type, a \textsc{still} item serves as a focus-sensitive\is{focus}\is{scale} scalar operator that relates a time frame to alternative times, (\lq as early/late/far removed as\rq{}), making it a type of scalar\is{scale} additive operator (\Cref{sectionTimeScalar}). In addition to these recurring functions, one item, \il{Kekchí|(}Kekchí \textit{toj}, also serves as limitative \lq until\rq{}.\is{limitative} I discuss this in the context of the \isi{restrictive} use, to which it is likely related on the diachronic plane (\Cref{sectionTimeScalarRestrictive}).\il{Kekchí|)}

\subsubsection{Temporal restrictive (and limitative)}\is{restrictive|(}
\label{sectionTimeScalarRestrictive}\is{scale|(}\is{focus|(}\largerpage
\subsubsubsection{Introduction}\il{Quechua, Huallaga-Huánuco|(}In this subsection, I address a use in which a \textsc{still} expression serves as a scalar restrictive operator whose focus is a temporal constituent.\is{syntax} In my sample, such a \lq not until\rq{ }function is found with two expressions, namely Huallaga-Huánuco Quechua \mbox{-\textit{raq}} and \ili{Kekchí} \textit{toj} (\appsref{AppendixQuechuaFirstSubordinate}, \ref{appendixKekchiRestrictive}). In what follows, I illustrate and discuss each case separately.

\subsubsubsection{A closer look and discussion: H.-H. Quechua \mbox{-\textit{raq}}} The examples in  (\ref{exTimeScalarRestictiveQuechuaIntro}) illustrate Huallaga-Huánuco Quechua \mbox{-\textit{raq}}. In (\ref{exTimeScalarRestictiveQuechuaIntro1}) this item modifies the temporal clause\is{temporal clause}\is{subordination} \lq when the harpist and violinist play for them' and signals that all alternative times under consideration (all earlier times) would lead to a false proposition. This fact is also pointed out explicitly in the following sentence. The same function can be seen in (\ref{exTimeScalarRestictiveQuechuaIntro2}), but with a nominal complement.

\begin{exe}
	\ex\label{exTimeScalarRestictiveQuechuaIntro}
	\begin{xlist}
		\exi{}Huallaga-Huánuco Quechua\is{temporal clause}\is{subordination}
		\ex \label{exTimeScalarRestictiveQuechuaIntro1}
		\gll 	… dansa-n arpista bigulista tuka-pa-pti-n-\textbf{raq}. Mana tuka-pti-n-qa mana dansa-n-chu.\\
	{} dance-3 harpist violinist play-\textsc{ben}-\textsc{subord}.\textsc{adv}-3-still \textsc{neg} play-\textsc{adv}-\textsc{poss}.3-\textsc{top} \textsc{neg} dance-3-\textsc{neg}\\
	\glt \lq … they dance \textbf{when and not until} the harpist and violinist play for them (and not before). If they do not play, they do not dance.'

		\ex \label{exTimeScalarRestictiveQuechuaIntro2}
		\gll \textbf{Ishkay} \textbf{killa}-\textbf{raq} haru-shka:.\\
		two month-still step-\textsc{ant}.1\\
		\glt \lq \textbf{It was two months before} I stepped on it (a disjointed ankle).'
		\\\parencite[387]{Weber1989}
	\end{xlist}
\end{exe}

The temporal restrictive function of \mbox{-\textit{raq}} forms the basis of several idiomatic phrases. Among these, \textit{chay}-\textit{raq}-\textit{shi} \lq that-still-\textsc{evid}\rq{},  i.e. \lq{}it is only then\rq{ }stands out in coming close to \lq finally\rq{} and thereby to a marker of the unexpectedly\is{expectations} late scenario of \textsc{already}.\is{already}\footnote{See \textcite{Kramer2017} and references therein on the three scenarios of \textsc{already}.\is{already}} This fixed phrase is illustrated in (\ref{exQuechuaFinally}).

\begin{exe} 
	\ex Huallaga-Huánuco Quechua\label{exQuechuaFinally}\\
	Context: An old man has been suspecting that his wife cheats on him.\\
	\gll … isha-n qaqa-sha. Awkin-na-shi ollqo-yka-n ruru-lla-pa-qa. \textbf{Chay}-\textbf{raq}-\textbf{shi} awkin ollqo-yka-n.\\
	{} two-3 be\_parallel-\textsc{ptcp} old\_man-already-\textsc{evid} be\_angry-\textsc{pfv}-3 inside-just-\textsc{gen}-\textsc{top} that-still-\textsc{evid} old\_man be\_angry-\textsc{pfv}-3\\
	\glt \lq The two of them are together (the old man’s wife and her lover). At that, the old man becomes angry, but just inside. \textbf{Finally}, \textbf{only} \textbf{then}, did the old man become angry.' \parencite[379]{Weber1989}
\end{exe}

The most likely link between \mbox{-\textit{raq}} as a temporal restrictive operator and the same item as a marker of phasal polarity is found in yet a third function of this item, namely as  \lq first, for now\rq{}. I discuss this question in some more detail in \Cref{sectionScalarRestrictive}.\il{Quechua, Huallaga-Huánuco|)}

\subsubsubsection{A closer look and discussion: Kekchí \textit{toj}}\il{Kekchí|(} Moving on to Kekchí, example (\ref{exTimeScalarRestrictiveKekchiIntro}) illustrates the \textsc{still} expression \textit{toj} in its function as a time-scalar restrictive marker; for an example of the same function with a clausal argument,\is{subordination}\is{temporal clause} see (\ref{exTimeScalarRestrictiveKekchi2}) below.

\begin{exe}	
	\ex Kekchí\label{exTimeScalarRestrictiveKekchiIntro}\\
	\gll \textbf{Toj} ewu t-in-xik.\\
	still afternoon \textsc{prosp}-1\textsc{sg}-go\\
	\glt \lq I'll go [no earlier than] in the afternoon.\rq{ }\parencite[464]{Kockelman2020}
\end{exe}

\is{topic time|(}Often-times, the restrictive use of \textit{toj} is understood as establishing a topic time immediately after the coming-into-existence of the state-of-affairs depicted in its complement. This is shown in (\ref{exTimeScalarRestrictiveKekchi2}).

\begin{exe}
\ex Kekchí\label{exTimeScalarRestrictiveKekchi2}\\
	\gll Nek-e'-xik sa' li tz'oleb'al \textbf{toj} \textbf{wan} \textbf{r}-\textbf{e} \textbf{waqib'} \textbf{chihab'}.\\
	\textsc{prs}-3\textsc{pl}-go \textsc{prep} \textsc{det} school still \textsc{exist}.3\textsc{sg} 3\textsc{sg}-\textsc{dat} six year\\
	\glt \lq They go to school \textbf{when} \textbf{they} \textbf{are} \textbf{six} \textbf{years} \textbf{old}.' \parencite[466]{Kockelman2020}
\end{exe}\is{topic time|)}
\is{limitative|(}

Unlike Huallaga-Huánuco Quechia \mbox{-\textit{raq}}, Kekchí \textit{toj} also functions as a temporal limitative marker \lq until\rq{ }(\appref{appendixKekchiUntil}). For the most parts, there is a clear division of labour between these two functions.\footnote{It is safe to assume that \textit{toj} as temporal \lq until\rq{ }constitutes the source of its function as a phasal polarity expression. Presumably, this involved zero anaphora, which is a recurrent theme across several uses of this marker. Outside of my sample, a similar situation appears to hold true of \textit{gardi} in the Australian language \ili{Bardi} (Nyulnyulan); see \textcite[649–651]{Bowern2012}.} Thus, restrictive \textit{toj} is primarily found in environments of affirmative polarity and a non-continuous\is{continuous} viewpoint,\is{aspect} as in (\ref{exTimeScalarRestrictiveKekchiIntro}, \ref{exTimeScalarRestrictiveKekchi2}). The limitative function, on the other hand, is found in two types of contexts, comparable to \ili{English} \textit{until} (see \cite{Lakoff1969}). The first are environments of affirmative polarity and in which the duration of a situation is at stake, typically with a \isi{continuous} viewpoint,\is{aspect} as in (\ref{exLimitativeKekchi1}). The second type are contexts of clause-mate negation;\is{negation} see (\ref{exLimitativeKekchi2}).

\begin{exe}
	\ex	
	\begin{xlist}
		\exi{}Kekchí

		\ex
		\gll \textbf{Toj} \textbf{maak’a’}-\textbf{q} \textbf{chik} \textbf{in}-\textbf{k’as} t-in-k’anjelaq.\label{exLimitativeKekchi1}\\
	still \textsc{neg}.\textsc{exist}-\textsc{non}.\textsc{specific} more \textsc{poss}.1\textsc{sg}-debt \textsc{prosp}-1\textsc{sg}-work\\
	\glt \lq \textbf{Until I no longer have debt} I will work.\rq{}

		\ex
		\gll \textbf{Ink'a'} nek-e'-xik sa' li tz'oleb'al \textbf{toj} \textbf{wan} \textbf{r}-\textbf{e} \textbf{waqib'} \textbf{chihab'}.\label{exLimitativeKekchi2}\\
	\textsc{neg} \textsc{prs}-3\textsc{pl}-go \textsc{prep} \textsc{det} school still \textsc{exist}.3\textsc{sg} 3\textsc{sg}-\textsc{dat} six year\\
	\glt \lq They \textbf{do not} go to school \textbf{until} \textbf{they} \textbf{are six years old}.'\\\parencite[466, 480]{Kockelman2020}
	\end{xlist}
\end{exe}

Crucially, limitative \lq until\rq{ }within the scope of a negator\is{negation} can be understood as equivalent to a restrictive operator; see, for instance, \textcite[159–162]{Koenig1991} on \ili{English} \textit{not until}, as well as \textcite{deSwartEtAl2022} for a discussion of different compositional analyses of this and comparable items. This fact suggests the temporal restrictive function of \textit{toj} goes back to the association of limitative \lq until\rq{ }with negated\is{negation} predicates, from where it would have spread to affirmative contexts with a non-continuous\is{continuous} viewpoint.\is{aspect}\footnote{A similar case is found in Mesoamerican varieties of Spanish,\il{Spanish} where \textit{hasta} \lq until' with temporal expressions can mean \lq not until' \parencite[s.v. \textit{hasta}]{RAEDictionary}. This suggests an areal dimension to the phenomenon.} Functional motivation for this scenario can be found in usage patterns. As \textcite{Kockelman2020} points out, in continuous discourse both \lq until\rq{ }and restrictive \lq not until\rq{ }tend to go along with a constellation of two states-of-affairs, one that lasts until the time denoted by \textit{toj}'s complement, and a second situation that takes place no earlier than that. With this overlap in mind, in-between contexts are readily found in the data. The first are elliptical utterances like the one highlighted in (\ref{exTimeScalarRestictiveKekchiWife}). In such a context, what would have originally been the elision of a negative\is{negation} predicate \lq there is no speaking…\rq{ }could easily be reinterpreted as featuring an affirmative one \lq  only tomorrow there is…\rq{}.
\pagebreak

\begin{exe}
	\ex Kekchí\label{exTimeScalarRestictiveKekchiWife}
	\begin{xlist}
		\exi{A:}\textit{Maak'a' li aatinak hoon rik'in laawixaqil?}\\
		\lq There is no speaking with your wife today?\rq
		
		\exi{B:}\textit{Ink'a'.}\\
		\lq No.'
		
		\exi{A:}\gll \textbf{Toj} \textbf{kab'ej}?\\
		still tomorrow\\
		\glt \lq (\textbf{Not}) \textbf{until tomorrow}?'
		
		\exi{B:}\textit{Eq'ela kab'ej.}\\
		\glt \lq Early tomorrow.' \parencite[465]{Kockelman2020}
	\end{xlist}
\end{exe}

\is{syntax|(}A similar case, albeit from narrative discourse, is shown in (\ref{exTimeScalarRestrictiveKekchiBridge}). Here, the initial clause again establishes a negative state.\is{negation} In addition, the clause following the one in question depicts a subsequent event that yields a change to the contrary. Crucially, the underlying temporal and \isi{causal} relationships are the same no matter whether \textit{toj} and its clausal complement\is{temporal clause}\is{subordination} are understood as continuations of the first sentence and the situation depicted therein (\lq  … wasn't pleased until he placed the rat…\rq{}) or if they are read as establishing a new unit (\lq only when he placed the rat … it improved\rq{}).

\begin{exe}
	\ex Kekchí\label{exTimeScalarRestrictiveKekchiBridge}\\
	\gll Maa-min ki-hu[u]lak chu r-u qaawa\rq{} b\rq{}alamq\rq{}e, \textbf{toj} \textbf{ki}-\textbf{x}-\textbf{k\rq{}e} \textbf{li} \textbf{ch\rq{}o} \textbf{chi} \textup{(}\textbf{x}\textup{)}-\textbf{sa}\rq{} \textbf{li} \textbf{po}, \textbf{ut} \textbf{li} \textbf{ch}\rq{}\textbf{o} \textbf{aran} \textbf{ki}-\textbf{chu\rq{}uk}, jo\rq{}ka\rq{}in ki-usa.\\
		\textsc{neg}-\textsc{emph} \textsc{pfv}.\textsc{evid}.3\textsc{sg}-arrive \textsc{prep} \textsc{poss}.3\textsc{sg}-face \textsc{hon} B. still \textsc{pfv}.\textsc{evid}:3\textsc{sg}.\textsc{p}-3\textsc{sg}.\textsc{a}-give \textsc{det} rat \textsc{prep} \textsc{poss}.3\textsc{sg}-inside \textsc{det} moon \textsc{conj} \textsc{det} rat there \textsc{pfv}.\textsc{evid}:3\textsc{sg}-urinate thus \textsc{pfv}.\textsc{evid}:3\textsc{sg}-improve\\
\glt \lq In no way was Lord B’alamq’e pleased, \textbf{but when} [\textbf{lit. (not) until}] \textbf{he placed the rat inside the moon}, \textbf{and the rat peed there}, then it improved.\rq{ }\parencite[236]{Kockelman2010}
\end{exe}
\is{limitative|)}\is{syntax|)}

\is{sequencing|(}Lastly, \textit{toj} as restrictive \lq not until\rq{ }might play some role in the Kekchí near past construction (\Cref{sectionRemotenessPast})\is{remoteness} and a formally related event sequencing construction (\Cref{sectionEventSequencing}).
\is{scale|)}\is{focus|)}\il{Kekchí|)}\is{restrictive|)}\is{sequencing|)}

\subsubsection{Persistent time frame}\is{persistence|(}
\label{sectionTemporalFrameTT}
\subsubsubsection{Introduction}
In this subsection I discuss a use in which a \textsc{still} expression syntactically\is{syntax} and semantically associates with a temporal frame adverbial and signals \lq\lq that the event in question occurs while the time specification is still valid: \lq it is still T, when e occurs'" \parencite[202]{Loebner1989}.\footnote{See \textcite[200]{Vandeweghe1992} for a similar characterisation.}\il{German|(} Example (\ref{exTemporalFrameSectionIntroRep}) is an illustration. Here, German \textit{noch} signals that the event of leaving for Boston takes place before the end of the time span described in its sister constituent \textit{am Abend meiner Ankunft} \lq on the evening of my arrival\rq{}.

\begin{exe}
	\ex\label{exTemporalFrameSectionIntroRep} German\\
	Context: About a trip to the USA. The speaker did not want to be stuck in New York.\\
	\gll So bin ich \textbf{noch} \textbf{a}-\textbf{m} \textbf{Abend} \textbf{meiner} \textbf{Ankunft} … nach Boston ge-fahr-en.\\
thus \textsc{cop}.1\textsc{sg} 1\textsc{sg} still at-\textsc{def}.\textsc{dat}.\textsc{sg}.\textsc{m} evening(\textsc{m}) \textsc{poss}.1\textsc{sg}:\textsc{gen}.\textsc{sg}.\textsc{f} arrival(\textsc{f}) {} to B. \textsc{ptcp}-drive-\textsc{ptcp}\\
	\glt \lq Thus, \textbf{the very evening of my arrival} I went to Boston.\rq{ }(\textit{Rheinischer Merkur}, cited in \cite[57]{Shetter1966},  glosses added)
\end{exe}

The meaning of the associated constituent need not be strictly temporal. Example (\ref{exTemporalFrameLocative}) illustrates space-to-time metonymy. Lastly, as far as syntax\is{syntax|(} is concerned, the expression's complement may also be an embedded clause;\is{temporal clause}\is{subordination} see (\ref{exTemporalFrameTTKuendigung}, \ref{exContinuativeTTGermanBefreien}) below for illustrations.

\begin{exe}
	\ex\label{exTemporalFrameLocative}German\\
	\gll Er wurde \textbf{noch} \textbf{a}-\textbf{m} \textbf{Unfall}-\textbf{ort} operier-t.\\
3\textsc{sg}.\textsc{m} become.\textsc{pst}.3\textsc{sg} still at-\textsc{def}.\textsc{dat}.\textsc{sg}.\textsc{m} accident-place(\textsc{m}) operate-\textsc{ptcp}\\
	\glt \lq He was operated \textbf{right} \textbf{at} \textbf{the} \textbf{scene} \textbf{of} \textbf{the} \textbf{accident}.'
\\(\cite[s.v. \textit{noch}]{Duden},  glosses added)
\end{exe}\il{German|)}\is{syntax|)}

\subsubsubsection{Distribution in the sample (and beyond)}
\Cref{tableContinuativeTT} lists the markers in my sample for which the persistent time frame use is attested. As can be gathered, it is found with four expressions from four distinct languages, all  from western Eurasia. Outside of my sample, this use is also attested for \ili{Dutch} \textit{nog}, many cognates of Serbian\hyp Croatian\hyp Bosnian\il{Serbian}\il{Croatian}\il{Bosnian} \textit{još}, \ili{Danish} \textit{endnu}, \ili{Finnish} \textit{vielä}, and \ili{Hungarian} \textit{még}.\footnote{See \textcite{Rombouts1979} and \textcite[ch. 11.3]{Vandeweghe1992} on \ili{Dutch} \textit{nog}. For the Slavic cases, see i.a. \textcite[s.v. \textit{ešte}]{SSSJ}, \textcite[s.v. \textit{jeszcze}]{PWN}, \textcite[s.v. \textit{šè}]{SSKJ}, \textcite[s.v. \textit{ješte}]{SSJC}, \textcite{Komarek1979} and \textcite{Mustajoki1988}. Information on \ili{Danish} \textit{endnu} and \ili{Hungarian} \textit{még} comes from \textcite[s.v. \textit{endnu}]{DDO} and \textcite[s.v. \textit{még}]{BarcziOrszagh1992}, respectively. \ili{Finnish} \textit{vielä} is discussed in \textcite{Mustajoki1988}.} Further research pending, this suggests an areal pattern of Central, Northern and Eastern Europe plus, perhaps, the Caucasus. The case of Hebrew\il{Hebrew, Modern} \textit{ʕod} can most likely be explained through influence from West Germanic and Slavic.

\begin{table}
\caption{Persistent time frame use\label{tableContinuativeTT}}
\begin{tabular}{llll}
\lsptoprule
	Macro-area & Language & Expression & Appendix\\\midrule
    Eurasia & \ili{German} & \textit{noch} & \ref{appendixGermanContinuativeTT} \\
	& Hebrew (Modern)\il{Hebrew, Modern} & \textit{ʕod} & \ref{appendixHebrewOdContinuativeTT} \\
	& \ili{Lezgian} & \textit{hele}\footnote{Only one example in the data.} & \ref{appendixLezgianContinuative}\\
	& Serbian-Croatian-Bosnian\il{Serbian}\il{Croatian}\il{Bosnian} & \textit{još} & \ref{appendixBCMSContinuativeTT}\\
\lspbottomrule
\end{tabular}
\end{table}

\il{Wubuy|(}Note that \Cref{tableContinuativeTT} does not include the occasional tokens from Australia and Papunesia, such as in the Wubuy example (\ref{exContinuativeTTWubuy}), all of which involve the items in question as markers of an emphatic assertion of \isi{identity} (see \Cref{sectionExclusive}).\is{restrictive}

\begin{exe}
	\ex Wubuy\\
 Context: About preparing mangrove fruits. The fruits have been roasted.\label{exContinuativeTTWubuy}\\
	\gll N\textsuperscript{g}a ad̠aba wirima:diːni, ad̠aba aːguguwuy wirimaːralhalwulhan\textsuperscript{g}i, wirimaːralhalwulhan\textsuperscript{g}i \textbf{yimbaj}-\textbf{bugij}, ad̠aba \textbf{yimbaj}-\textbf{bugij} wirimadhurmaː.\\
and\_then then they\_took\_it\_out.\textsc{cont} then to\_water they\_soaked\_it\_all.\textsc{cont} they\_soaked\_it\_all.\textsc{cont} today-still then today-still they\_crushed\_it.\textsc{cont}\\
	\glt \lq Then they (people) took them out of the oven and put them in fresh water to soak. \textbf{On the same day} (i.e. a few hours later) they began to grind the fruits.' \parencite[423]{Heath1980}
\end{exe}\il{Wubuy|)}

\il{Udihe|(}Similarly, \Cref{tableContinuativeTT} does not include Udihe \mbox{\textit{xai}(\textit{si})} in attestations like (\ref{exContinuativeTTUdihe}). As in this example, all relevant instances involve the collocation with an anaphoric demonstrative of the \mbox{(\textit{u})\textit{t}-} series. This pattern has developed a more widely productive \lq same\rq{ }function, which I assume to be at play here. For a discussion of this use, see  \Cref{sectionSame}.

\begin{exe}
	\ex Udihe\label{exContinuativeTTUdihe}\\
	\textit{Ni: lä baza gele lali:nzi budei, zeude ei diga ilama neŋini, uti ni:we mene e:tigi mene ŋenezeŋei} <\textit{mene ŋenezeŋei}>\\
	\lq Many people in the taiga die from hunger. If a human doesn’t eat anything for three days, the tiger directs this person to where it has to go.'
	\exi{}
	\gll \textbf{Xai} \textbf{uti} \textbf{dogbo}-\textbf{ni} uti xokto-tigi-ni ŋene-wen’e uti ni:-we xebu-ini.\\
still that night-\textsc{poss}.3\textsc{sg} this road-\textsc{lat}-\textsc{poss}.3\textsc{sg} go-\textsc{caus}.\textsc{pfv} this man-\textsc{acc} take-3\textsc{sg}\\
\glt \lq It will direct him to the road \textbf{the very same night} and lead this person.' \parencite[The tiger for Udihe people]{NikolaevaEtAl2019}
\end{exe}\il{Udihe|)}

\subsubsubsection{A closer look: Meaning and appropriateness conditions} I now turn to a closer examination of the meaning of the persistent time-frame use and the conditions for its felicitous employment. This will help to delineate it further from the superficially similar time-scalar\is{scale} additive uses that I discuss in \Cref{sectionTimeScalar} (\lq as early/late as, as far removed as\rq{}).\is{topic time|(} As I hinted above, the use I discuss here signals the continuity of a time frame in essentially the same way that phasal polarity \textsc{still} does for the main predication of a clause. This parallel is graphically illustrated in \Cref{figurePersistenTimeFrame}, using a \isi{perfective} vantage point\is{aspect} for ease of illustration.

\setlength{\MinimumWidth}{\widthof{Sit.x}}
\begin{figure}[hbt]\is{perfective}

	\centering
	\begin{subfigure}[b]{0.48\linewidth}
		\centering

		\begin{tikzpicture}[node distance = 0pt]
			\node[mynode, text width=\breit, fill=cyan, very near start] (spain){Situation};
			\node[mynode, fill=none, minimum width=\superschmal, text width=\superschmal, right= of spain] (france){(\small{$\Diamond$}\neg{ }Sit.)};	
			\draw[-, densely dashed] ($(spain.north east)+ (-0.2pt,0) $) to ($(spain.south east)+ (-0.2pt,-0.5*\hoehe) $);
			\draw[-, densely dashed] ($(spain.north east)+ (-1*\hoehe-0.2pt,0)$) to ($(spain.south east)+ (-1*\hoehe-0.2pt,-0.5*\hoehe) + (0.2pt,0)$) node [below, align=center, label distance=0, xshift=0.5*\hoehe] {Topic\\time};
			\draw[-latex, line width=0.5pt]  (spain.south west) to  ($(france.south east)+(1ex,0)$) node [right] {t};
		\end{tikzpicture}
		\subcaption{\textsc{still}}
	\end{subfigure}
	\begin{subfigure}[b]{0.48\linewidth}
		\centering
			\begin{tikzpicture}[node distance = 0pt]
			\node[mynode, minimum width=\breit, text width=\breit, fill=cyan, text opacity=1, fill opacity=.5, very near start] (A){Frame};
				
			\node[mynode,  anchor=east, fill=cyan, text width=\MinimumWidth,draw=none, xshift=-0.5*\hoehe] (D) at (A.east) {Sit.};
			\draw[thick] (D.south east) to (D.north east);
			\draw[-] (D.south west) to (D.north west);
			\node[mynode, fill=white, minimum width=\schmal, text width=\schmal, right= of A] (C){(\small{$\Diamond$}\neg{ }Frame)};		
			\draw[-, densely dashed] ($(A.north east)+(0-0.2pt,0)$) to ($(A.south east)+ (0,-0.5*\hoehe) $);
			\draw[-, densely dashed] ($(A.north east)+ (-1*\hoehe-0.2pt,0) $) to ($(A.south east)+ (-1*\hoehe-0.2pt,-0.5*\hoehe) $) node [below, align=center, label distance=0, xshift=0.5*\hoehe] {Topic\\time};
			\draw[-latex, line width=0.5pt]  (A.south west) to  ($(C.south east)+(1ex,0)$) node [right] {t};
		\end{tikzpicture}
	\subcaption{Persistent time frame (with \textsc{pfv})}
	\end{subfigure}
	\caption{Schematic illustration of persistent time-frame use\label{figurePersistenTimeFrame}}
\end{figure}

Consequently, the persistent time frame use requires a contextually salient topic time for which the persistence of the time span in question can be evaluated (cf. \cite[266]{Boguslavsky1996}).\footnote{Also see \textcite{Greenberg2008} for a discussion of \textsc{still} and established topic times.}\il{German|(} For instance, in (\ref{exTemporalFrameTTKuendigung}) topic time is narrowed down through the discourse topic of how to behave when having received a termination notice plus world knowledge that dismissals in Germany do not normally have immediate effect. This setting is anaphorically referenced by the associated constituent of \textit{noch}, the clause\is{temporal clause}\is{subordination} \textit{während man arbeitet} \lq while you're employed' and it is stressed that the affected reader should pay a visit to the job centre before the end of this period.

\begin{exe}
	\ex German \label{exTemporalFrameTTKuendigung}\\
	Context: On how an employee should behave when your work contract has been terminated.\\
	\gll Also, \textbf{noch} \textbf{während} \textbf{man} \textbf{arbeit}-\textbf{et} und sobald man weiß, wann der letzt-e Arbeit-s-tag sein soll: zu-r Arbeitsargentur geh-en.\\
	so, still while \textsc{impr} work-3\textsc{sg} and once \textsc{impr} know.3\textsc{sg} when \textsc{def}.\textsc{nom}.\textsc{sg}.\textsc{m} last-\textsc{nom}.\textsc{sg}.\textsc{m} work-\textsc{lnk}-day(\textsc{m}) \textsc{cop}.\textsc{inf} should.3\textsc{sg} to-\textsc{def}.\textsc{dat}.\textsc{sg}.\textsc{f} job\_centre(\textsc{f}) go-\textsc{inf}\\
	\glt \lq So, \textbf{while} [\textbf{it is still the time that}] \textbf{you're} \textbf{employed} and as soon as you know when your last day at work is scheduled: visit the job centre.\rq
	\\(found online, glosses added)
\end{exe}\il{German|)}

\il{Serbian|(}\il{Croatian|(}\il{Bosnian|(}
To give another example,  in (\ref{exTemporalFrameMedSchool}) the preceding text establishes an interval around the right temporal edge of receiving the good news. It is then highlighted that narrative time\is{textuality} does not progress further than this.\largerpage

\begin{exe}
	\ex[]{Serbian-Croatian-Bosnian\il{Serbian}\il{Croatian}\il{Bosnian}\label{exTemporalFrameMedSchool}\\
	Context: A student with the ardent desire to become a doctor did not get accepted into medical school at the first try. He applied again, and the good news that this time he was accepted in Berlin got to him while he was travelling.}
	\exi{}[]{\gll \textbf{Još} \textbf{sledećeg} \textbf{dana} je otputova-o za Berlin jer je semestar već počinja-o.\\
 	 still next.\textsc{gen}.\textsc{m} day(\textsc{m}).\textsc{gen} \textsc{cop}.3\textsc{sg} leave.\textsc{pfv}.\textsc{ptcp}-\textsc{sg}.\textsc{m} for Berlin because \textsc{cop}.3\textsc{sg} semester(\textsc{m}).\textsc{nom}.\textsc{sg} already begin.\textsc{ipfv}.\textsc{ptcp}-\textsc{sg}.\textsc{m}\\
	 \glt  \lq \textbf{The very next day} he set off for Berlin, because the semester was about to begin.\rq{ }(found online, glosses added)}
\end{exe}

The denotation of \textit{sledećeg dana} \lq the next day\rq{ }in (\ref{exTemporalFrameMedSchool}) can be said to be included in the interval adjacent to the decisive event, and exchanging it for something more removed like \textit{tri dana kasnije} \lq three days later' would render the sentence infelicitous, as shown in (\ref{exTemporalFrameMedSchool3Days}). The same holds true for the corresponding \il{German|(}German and Modern Hebrew\il{Hebrew, Modern} translations in (\ref{exTemporalFrameMedSchool3DaysGerman}, \ref{exTemporalFrameMedSchool3DaysHebrew}).

\begin{exe}
	\ex
	\begin{xlist}
	\sn[]{Context: same as (\ref{exTemporalFrameMedSchool})}
	\ex[]{\label{exTemporalFrameMedSchool3Days}Serbian-Croatian-Bosnian}
	\exi{}[\#]{\gll  \textbf{Još} \textbf{tri} \textbf{dana} \textbf{kasnije} je otputova-o.\\
	 { }still three day.\textsc{gen}.\textsc{gen} after \textsc{cop}.3\textsc{sg} leave.\textsc{pfv}.\textsc{ptcp}-\textsc{sg}.\textsc{m}\\}
	 
	\ex[]{\label{exTemporalFrameMedSchool3DaysGerman}German}
	\exi{}[\#]{\gll \textbf{Noch} \textbf{drei} \textbf{Tag}-\textbf{e} \textbf{danach} brach er auf.\\
	still three day-\textsc{pl} thereafter break.\textsc{pst}.3\textsc{sg} 3\textsc{sg}.\textsc{m} off\\}\il{German|)}
	
	\ex[]{\label{exTemporalFrameMedSchool3DaysHebrew}Modern Hebrew\il{Hebrew, Modern}}
	\exi{}[\#]{\gll \textbf{ʕod} \textbf{le}-\textbf{aħar} \textbf{šlo\rq{}ša} \textbf{yamim} hu ʕazav.\\
	still to-after three day.\textsc{pl}  3\textsc{sg}.\textsc{m} leave.\textsc{pst}.3\textsc{sg}.\textsc{m}\\
	 \glt (intended: \lq … before the end of the third day he set off.\rq{}) 
	\\(personal knowledge; Itamar Francez and Stefan Savić, p.c.)}
	\end{xlist}
\end{exe}

\il{German|(}\il{Hebrew, Modern|(}Likewise, the examples in (\ref{exTemporalUebermorgen}) are unusual, \lq\lq unless it is clear that for independent reasons reference is implicitly to the day after tomorrow" (\cite[202]{Loebner1989}; also see \cite[266]{Boguslavsky1996}).

\begin{exe}
	\ex\label{exTemporalUebermorgen}
	\begin{xlist}
		\ex[]{German}
		\exi{}[?]{\gll Sie komm-t \textbf{noch} \textbf{übermorgen}.\\
		3\textsc{sg}.\textsc{f} come-3\textsc{sg} still day\_after\_tomorrow\\}\il{German|)}

		\ex[]{Modern Hebrew}
		\exi{}[?]{\gll Hi t-agiaʕ \textbf{ʕod} \textbf{moħratayim}.\\
		3\textsc{sg}.\textsc{f} 3\textsc{sg}.\textsc{f}-arrive.\textsc{fut} still day\_after\_tomorrow\\}
		\ex[]{Serbian-Croatian-Bosnian}
		\exi{}[?]{\gll Ona će stići \textbf{još} \textbf{prekosutra}.\\
	3\textsc{sg}.\textsc{f} \textsc{aux}.\textsc{fut}:3\textsc{sg} arrive.\textsc{pfv}.\textsc{inf} still day\_after\_tomorrow\\
		\glt \lq She'll arrive the \textbf{very} \textbf{day} \textbf{after} \textbf{tomorrow}.\rq{ }(\cite[202]{Loebner1989}; Stefan Savić and Itamar Francez, p.c.)}
	\end{xlist}
\end{exe}
\il{Hebrew, Modern|)}
\il{Serbian|)}\il{Croatian|)}\il{Bosnian|)}

The requirement for an established and salient topic time constitutes a key difference to the time-scalar additive uses that I discuss in \Cref{sectionTimeScalar}, where topic time is an open variable. The second critical difference is more subtle. Thus, while the persistent time frame use often gives rise to a scalar\is{scale} notion of earliness (e.g. \cite[s.v. \textit{noch}]{Duden}; \cite[185]{Helbig1994}; \cite{Loebner1989}), this is best understood as a derived inference, at least on the synchronic plane. Like phasal polarity \textsc{still}, this use evokes an alternative course of events at which the interval in question has ceased to be valid, hence later times \parencite[200]{Vandeweghe1992}. To give just one example, in (\ref{exTemporalFrameMedSchool}) the protagonist might have waited a day or two until heading off to Berlin. Such a consideration of later alternatives can be enforced by the temporal constituent receiving \isi{focus} (\cite{Beck2016}, \citeyear{Beck2020}). That relative earliness is not part of the conventional meaning becomes apparent when comparing an example like (\ref{exTemporalFrameMedSchool}), repeated below, to a version where the \textsc{still} expression has been exchanged for an \textsc{already}\is{already} expression functioning as a temporal \isi{focus} particle.\il{Serbian|(}\il{Croatian|(}\il{Bosnian|(} As shown in (\ref{exTemporalFrameMedSchoolAlreadySerbian}) for Serbian\hyp Croatian\hyp Bosnian \textit{još}, this substitution introduces an overtly scalar\is{scale} notion that is not inherent to the original text; see \textcite{Mustajoki1988} and \textcite{Rombouts1979} for similar observations.

\begin{exe}
	\exr{exTemporalFrameMedSchool}Serbian-Croatian-Bosnian\\
	\gll \textbf{Još} \textbf{sledećeg} \textbf{dana} je otputova-o za Berlin.\\
 	 still next.\textsc{gen}.\textsc{m} day(\textsc{m}).\textsc{gen} \textsc{cop}.3\textsc{sg} leave.\textsc{pfv}.\textsc{ptcp}-\textsc{sg}.\textsc{m} for Berlin\\
	 \glt  \lq \textbf{The very next day} he set off for Berlin.\rq{ }(found online, glosses added)
	 
	 \ex\label{exTemporalFrameMedSchoolAlreadySerbian}
	\gll \textbf{Več} \textbf{sledećeg} \textbf{dana} je otputova-o za Berlin.\\
	already next.\textsc{gen}.\textsc{m} day(\textsc{m}).\textsc{gen} \textsc{cop}.3\textsc{sg} leave.\textsc{pfv}.\textsc{ptcp}-\textsc{sg}.\textsc{m} for Berlin\\
	\glt \lq \textbf{As early as the next day} he left for Berlin.' (Stefan Savić, p.c.)
\end{exe}\il{Serbian|)}\il{Croatian|)}\il{Bosnian|)}
\is{topic time|)}
\il{German|(}
What is more, and as shown in (\ref{exTemporalFrameGermanExpectation}) for German \textit{noch}, a reading of an unexpected early timing is defeasible in just the same way that a reading of unexpected lateness is defeasible with phasal polarity \textsc{still}.\is{expectations}

\begin{exe}
	\ex German\label{exTemporalFrameGermanExpectation}\\
	\gll \textbf{Noch} \textbf{a}-\textbf{m} \textbf{Vor}-\textbf{mittag} ist Lydia, wie von all-en erwart-et, ab-ge-reis-t.\\
	still at-\textsc{def}.\textsc{dat}.\textsc{sg}.\textsc{m} pre-noon(\textsc{m}) \textsc{cop}.3\textsc{sg} L. as from all-\textsc{dat}.\textsc{pl} expect-\textsc{ptcp} off-\textsc{ptcp}-travel-\textsc{ptcp}\\
	\glt \lq Lydia left [\textbf{while it was still}] \textbf{in the morning}, as everyone expected.\rq
	\\\parencite[27 fn10]{Beck2020}
\end{exe}\il{German|)}

\is{aspect|(}
Lastly, given that the notion of persistence pertains to the time frame rather than to the situation depicted in the host clause, the persistent time-frame use neither requires a durative situation, nor an aspectual perspective that is fully contained within it. In fact, this function normally goes together with a \isi{perfective} or \isi{anterior} viewpoint,\is{aspect} as in (\ref{exTemporalFrameSectionIntroRep}, \ref{exTemporalFrameLocative}) and in (\ref{exTemporalFrameTTKuendigung}–\ref{exContinuativeTTLezgian}).\is{aspect|)}

\subsubsubsection{A closer look: A recurring usage pattern}\il{German|(}\is{precedence|(}\textcite{Shetter1966} observes that German \textit{noch} in the persistent time frame function is recurrently found together with expressions of precedence.\footnote{Also see \textcite{Rombouts1979} on the \ili{Dutch} cognate \textit{nog}.} In narrative texts, this is often a temporal clause\is{temporal clause}\is{subordination} that depicts a non-occurrence, as in (\ref{exContinuativeTTGermanBefreien}). This is, to all appearances, also at play in the \ili{Lezgian} example (\ref{exContinuativeTTLezgian}). This pattern is clearly motivated by the fact that the \isi{discontinuation} of the time-frame, and hence the eventual actualisation of the event in question, are not entailed, but merely constitute a conceivable alternative scenario (\Cref{secFunctionalDiscussion}). As the free translation \lq even before' in (\ref{exContinuativeTTGermanBefreien}, \ref{exContinuativeTTLezgian}) indicates, the relevant instances also tend to give rise to the now familiar scalar\is{scale} inference. 

\begin{exe}
		\ex German\label{exContinuativeTTGermanBefreien}\is{temporal clause}\is{subordination}\\
		\gll
		Aber \textbf{noch} \textbf{ehe} \textbf{sie} \textbf{die} \textbf{Tür} \textbf{hinter} \textbf{sich} \textbf{schließ}-\textbf{en} \textbf{konnte}, hat-te sich jener mit letzt-er Kraft von dem Griff des Brandenburger-s los-gerissen.\\
but still before 3\textsc{sg}.\textsc{f} \textsc{def}.\textsc{acc}.\textsc{sg}.\textsc{f} door(\textsc{f}) behind \textsc{refl}.3 close-\textsc{inf} can.\textsc{pst}.3\textsc{sg} have-\textsc{pst}.3\textsc{sg} \textsc{refl}.3 \textsc{dist}:\textsc{nom}.\textsc{sg}.\textsc{m} with last-\textsc{dat}.\textsc{sg}.\textsc{f} force(\textsc{f}) from \textsc{def}.\textsc{dat}.\textsc{sg}.\textsc{m} grip(\textsc{m}) \textsc{def}.\textsc{gen}.\textsc{sg}.\textsc{m} inhabitant\_of\_Brandenburg(\textsc{m})-\textsc{gen} loose-tear.\textsc{ptcp}\\
		\glt \lq But \textbf{before she even had the chance to close the door behind herself}, that man had freed himself from the grip of the man from Brandenburg.'\\(von le Fort, \textit{Die Verfemte}; cited in \cite[57]{Shetter1966},  glosses added)

	\ex\label{exContinuativeTTLezgian}\ili{Lezgian}\\
	\gll \textbf{Hele} \textbf{zun} \textbf{akwa}-\textbf{daldi}, am qʰfe-na.\\
	still 1\textsc{sg}.\textsc{abs} see-\textsc{cvb}:before 3\textsc{sg}.\textsc{abs} leave-\textsc{aor}\\
	\glt \lq \textbf{Even before she saw me}, she left. / \textbf{Noch ehe sie mich sah}, ging sie weg.' \parencite[90]{Haspelmath1991}
\end{exe}\il{German|)}
	
For \ili{Russian} \textit{ešcë}, which is not included in my sample, \textcite{Mustajoki1988} points out that such combinations are not licensed. The same appears to be true of its Serbian\hyp Croatian\hyp Bosnian\il{Serbian}\il{Croatian}\il{Bosnian} cognate \textit{još}, as well as of Modern Hebrew \textit{ʕod}.\il{Hebrew, Modern} In both cases, this is probably due to pre-emptive blocking by a time-scalar\is{scale} additive use \lq as far removed as\rq{ }of the same markers. These are the subject of \Cref{sectionTimeScalar}.\is{precedence|)}

\subsubsubsection{Discussion}
As the preceding exposition has shown, there is a very close resemblance between the persistent time frame use and phasal polarity \textsc{still}. The key difference is that the notions of persistence and a subsequent \isi{discontinuation} do not apply to the situation depicted in the clause, but to the time-frame denoted in the expression's complement. This was schematically illustrated in \Cref{figurePersistenTimeFrame} above. The question of the exact conditions leading to the emergence of the persistent time frame use is a task for future diachronic corpus research.\il{German|(} The key most likely lies in the type of adverbial modification found in examples like (\ref{exFringesIntroGermanAfrika}). The main difference is that with the use I discuss in this subsection a property of time is at stake, whereas (\ref{exFringesIntroGermanAfrika}) features a secondary predication about the subject.

\begin{exe}
	\ex German\label{exFringesIntroGermanAfrika}\\
	Context: After a longer trip to Africa, the narrator has decided to become a wildlife ranger. Now she has returned to Germany and doubts are creeping in.\\
	\gll \textbf{Noch} \textbf{in} \textbf{Afrika} erschien mir der Gedanke wenig-er abwegig … \textbf{Zurück} \textbf{in} \textbf{den} \textbf{eigen}-\textbf{en} \textbf{vier} \textbf{Wänd}-\textbf{en} kling-t diese Idee jetzt aber verrückt.\\
	still in Africa appear.\textsc{pst}.3\textsc{sg} 1\textsc{sg}.\textsc{dat} \textsc{def}.\textsc{nom}.\textsc{sg}.\textsc{m} thought(\textsc{m}) little-\textsc{cmpr} devious {} back in \textsc{def}.\textsc{dat}.\textsc{pl} own-\textsc{dat}.\textsc{pl} four wall-\textsc{dat}.\textsc{pl} sound-3\textsc{sg} \textsc{prox}:\textsc{nom}.\textsc{sg}.\textsc{f} idea(\textsc{f}) now but crazy\\
	\glt \lq \textbf{Still} [\textbf{being in}] \textbf{Africa} the thought seemed less absurd to me …, [\textbf{being}] \textbf{back} \textbf{at} \textbf{home}, the idea seems crazy.\rq{ }	(Neitzel, \textit{Frühstück mit Elefanten}, glosses added)
\end{exe}\is{persistence|)}\il{German|)}

\subsubsection{Time-scalar additive}\is{scale|(}\is{focus|(}
\label{sectionTimeScalar}
\subsubsubsection{Introduction} In this subsection, I discuss \textsc{still} expressions as a type of focus-sensitive scalar operator that relates the denotation of a temporal frame expression to lower-ranking alternatives (see \Cref{sectionQuantificationScales} for general discussion). Example (\ref{exTimeScalarIntro}) illustrates such a time-specific scalar additive function. Here, \ili{French} \textit{encore} modifies the prepositional phrase \textit{en 2003} and relates its denotation to a set of earlier, alternative topic times.\is{topic time}

\begin{exe}
	\ex\label{exTimeScalarIntro}\ili{French}\\
	 \gll \textbf{Encore} \textbf{en} \textbf{2003}, Vincent aurait voté oui à l'-europe.\\
	still in 2003 V. have.\textsc{cond}.3\textsc{sg} vote.\textsc{ptcp} yes to \textsc{def}.\textsc{sg}-Europe\\
	\glt \lq \textbf{As late as 2003}, Vincent would have voted yes to Europe.' (\cite[160–161]{MosegaardHansen2008}, glosses added)
\end{exe}

While in (\ref{exTimeScalarIntro}) the alternatives under consideration are earlier times, one can also find the opposite, namely the focus being related to later times. Example (\ref{exTimeScalarIntroLezgian}) is an illustration.

\begin{exe}
	\ex \ili{Lezgian}\label{exTimeScalarIntroLezgian}\\
	\gll Jusuf \textbf{hele} \textbf{naq'} ata-na.\\
	Jusuf still yesterday come-\textsc{aor}\\
	\glt \lq Jusuf came \textbf{as early as yesterday}.' \parencite[85]{Haspelmath1991}
\end{exe}

Despite the differences in the relative ordering of times, examples (\ref{exTimeScalarIntro}) and (\ref{exTimeScalarIntroLezgian}) both involve a scale of time proper, or calendric values. \textsc{still} expressions in the relevant function can also operate in a scalar model of temporal distance. These instances consistently feature less distant alternatives \lq as far removed as\rq{}, such as in (\ref{exTimeScalarEinstein}). This example also illustrates another point: the associated constituent need not be a temporal adverbial phrase, but location in time can be mediated by metonymy. Lastly, while (\ref{exTimeScalarIntro}–\ref{exTimeScalarEinstein}) feature nominal and prepositional phrases as complements, the same function is also attested with an embedded clause as the modificandum;\is{subordination}\is{temporal clause}\is{syntax} see (\ref{exTimeScalarSerbianEvenBefore}) below for an illustration.

\begin{exe}
	\ex Serbian-Croatian-Bosnian\il{Serbian}\il{Croatian}\il{Bosnian}\label{exTimeScalarEinstein}\\
	\gll No, iako ih je \textbf{još} \textbf{Einstein} najavi-o fizičari dosad nisu uspje-li otkriti postojanje gravitacijskih valova. \\
but although 3\textsc{pl}.\textsc{acc} \textsc{cop}.3\textsc{sg} still E. announce.\textsc{pfv}.\textsc{ptcp}-\textsc{sg}.\textsc{m} physicist.\textsc{nom}.\textsc{pl} thus\_far \textsc{neg}.\textsc{cop}.3\textsc{pl} succeed.\textsc{pfv}.\textsc{ptcp}-\textsc{pl}.\textsc{m} uncover.\textsc{pfv}.\textsc{inf} existence.\textsc{acc}.\textsc{sg} gravitational.\textsc{gen}.\textsc{pl} wave.\textsc{gen}.\textsc{pl}\\
\glt \lq But even though \textbf{(someone as far removed as) Einstein} postulated them, until now physicists have failed to detect the existence of gravitational waves.\rq{ }(found online, glosses added)%\footnote{\url{https://www.24sata.hr/tech/otkrili-su-gravitacijske-valove-konacni-dokaz-za-einsteina-460394} (27 April, 2022).}
\end{exe}

\subsubsubsection{Distribution in the sample}
\Cref{tableTimeScalar} lists the six expressions in my sample for which time-scalar additive functions are attested. In terms of their geographic distribution, it is striking that all cases come from western Eurasia.\footnote{The same functions are also found with several cognates of my sample expressions, such as \ili{Dutch} \textit{nog} (e.g. \cite[ch. 11.3]{Vandeweghe1992}) or cognates of Serbian\hyp Croatian\hyp Bosnian\il{Serbian}\il{Croatian}\il{Bosnian} \textit{još} across Slavic (e.g. \cite{Boguslavsky1996}; \cite[s.v. \textit{jeszcze}]{PWN}; \cite[s.v. \textit{ще}]{CYM11}; \cite{Komarek1979}; \cite{Mustajoki1988}).} As is to be expected from this part of the world (see \cite{vanderAuwera1998}), they all involve independent grammatical words. When it comes to the specific semantic types, all relevant expressions except Modern Hebrew\il{Hebrew, Modern} \textit{ʕod} have a use that involves a scale of time proper. Of these five cases, four exclusively relate to earlier alternatives, making them a type of \textsc{beyond} operator (\Cref{sectionQuantificationScales}). One item, Lezgian,\il{Lezgian} \textit{hele} can relate the denotation of its focus to earlier as well as to later alternatives. This means that no expression in my sample constitutes a time-scalar \textsc{beneath} operator, i.e. one that exclusively refers to earlier alternatives. One item, Serbian\hyp Croatian\hyp Bosnian\il{Serbian}\il{Croatian}\il{Bosnian} \textit{još}, can operate both on a scale of calendric values, as well as on one of temporal distance. In either use \textit{još} presupposes lower alternatives (earlier times and less removed times, respectively). Lastly, Modern Hebrew\il{Hebrew, Modern} \textit{ʕod} only has the \lq as far removed as\rq{} use.

\begin{table}
\caption{Time-scalar additive uses\label{tableTimeScalar}}
\begin{tabularx}{\textwidth}{lllQl}
\lsptoprule
	M.-area & Language & Expr. & Type  & Appendix\\\midrule
	Eurasia & \ili{French} & \textit{encore} & \lq as late as\rq{}& \ref{appendixFrenchEncoreTimeScalar}\\
	& \ili{German} & \textit{noch} & \lq as late as\rq{} & \ref{appendixGermanTimeScalar}\\
	& Hebrew (Modern)\il{Hebrew, Modern} & \textit{ʕod} & \lq as far removed as\rq{} & \ref{appendixHebrewOdTimeScalar}\\
	& \ili{Lezgian} & \textit{hele}\footnote{\ili{Lezgian} \textit{hele} also serves as an \textsc{already}\is{already} expression; see discussion below.} & \lq as late as\rq{}, \lq as early as\rq{}& \ref{appendixLezgianTimeScalar}\\
	&Serb.-Croat.-Bosn.\il{Serbian}\il{Croatian}\il{Bosnian} & \textit{još} & \lq as late as\rq{}, \lq as far removed as\rq{}& \ref{appendixBCMSTimeScalar}\\
	& \ili{Spanish} & \textit{todavía} & \lq as late as\rq{} & 	\ref{appendixSpanishTodaviaTimeScalar}\\
\lspbottomrule
\end{tabularx}
\end{table}

\il{Spanish|(}To the list in \Cref{tableTimeScalar} one could add Spanish \textit{aún}. Example (\ref{exTimeScalarSpanishAun1}) illustrates this expression in an attestation that is both structurally and semantically comparable to my initial examples. As in (\ref{exTimeScalarSpanishAun1}), such instances of \textit{aún} most commonly involve earlier relata, although this expression is also occasionally attested with later alternatives (\appref{appendixSpanishAunTimeScalar}). Crucially, however, \textit{aún} has a generalised use as a scalar additive operator \lq even\rq{}. Instances like (\ref{exTimeScalarSpanishAun1}) can be understood as instantiation of this function which just so happen to involve time-related foci (conceivably, though, they receive additional motivation from the pathway that I discuss below).

\begin{exe}
	\ex Context: From a text about the origins of jewellery. The preceding sentences discusses talismans in pre-history\label{exTimeScalarSpanishAun1}.\\
	\gll y \textbf{aún} \textbf{en} \textbf{la} \textbf{Edad} \textbf{Medi}-\textbf{a} … a-l uso de ciert-a-s piedra-s precios-a-s se le otorg-aba divers-a-s propiedade-s.\\
	and still in \textsc{def}.\textsc{sg}.\textsc{f} age(\textsc{f}) medium-\textsc{f} {} to-\textsc{def}.\textsc{sg}.\textsc{m} use(\textsc{m}) of certain-\textsc{f}-\textsc{pl} stone(\textsc{f})-\textsc{pl} precious-\textsc{f}-\textsc{pl} \textsc{refl}.3 3\textsc{sg}.\textsc{dat}.\textsc{m} assign-\textsc{pst}.\textsc{ipfv}.3\textsc{sg} diverse-\textsc{f}-\textsc{pl} property(\textsc{f})-\textsc{pl}\\
	\glt \lq \textbf{and as late as in medieval times} … the use of certain precious stones was (still) associated with a number of properties.\rq{} (found online, glosses added)%\footnote{\url{https://vestuarioescenico.wordpress.com/2014/01/31/historia-del-boton/} (16 February, 2023).} 
\end{exe}\il{Spanish|)}

Lastly, most of the relevant sample languages also have \textsc{already}\is{already} expressions with a time-scalar additive function. These expressions appear to invariably operate on a scale of calendric values, relating the focus to later times (\lq as early as'). This is illustrated in (\ref{exTimeScalarAlready}) for \ili{French} \textit{déjà}. The outlier is found in the more complex case of \ili{Lezgian}, where the expression \textit{hele} itself has nearly completely transformed from a \textsc{still} expression to an \textsc{already} expression.\is{already}

\begin{exe}
	\ex \ili{French}\label{exTimeScalarAlready}\\
	\gll \textbf{Déjà} \textbf{à} \textbf{l’}-\textbf{âge} \textbf{de} \textbf{sept} \textbf{an}-\textbf{s}, elle lisait le latin sans problème-s.\\
	already at \textsc{def}.\textsc{sg}-age of seven year-\textsc{pl} 3\textsc{sg}.\textsc{f} read.\textsc{pst}.\textsc{ipfv}.3\textsc{sg} \textsc{def}.\textsc{sg}.\textsc{m} Latin without problem-\textsc{pl}\\
	\glt \lq \textbf{Already at the age of seven}, she read Latin without any problems.' (\cite[162]{MosegaardHansen2008},  glosses added)
\end{exe}

In what follows, I examine each of the time-scalar uses in more detail. I begin with the majority type \lq as late as\rq{} and then address the \lq as early as\rq{ }use of \ili{Lezgian} \textit{hele}. After that, I turn to the distance-based use \lq as far removed as\rq{}.

\subsubsubsection{A closer look: The `as late as' use} 
As I pointed out above, all relevant expressions that have a use involving a scale of times proper  allow for an \lq as late as\rq{ }reading. Example (\ref{exTimeScalarIntro}), repeated below, is an illustration. Here, the focus value of 2003 is set in relation to earlier alternatives from the common ground.\is{remoteness|(} Involving past times, these earlier alternatives are further removed from the temporal origo,\is{utterance time} which allows for a derived reading of a comparatively recent time.

\begin{exe}
	\exr{exTimeScalarIntro}\ili{French}\\
	 \gll \textbf{Encore} \textbf{en} \textbf{2003}, Vincent aurait voté oui à l'-europe.\\
	still in 2003 V. have.\textsc{cond} vote.\textsc{ptcp} yes to \textsc{def}.\textsc{sg}-Europe\\
	\glt \lq \textbf{As late as 2003}, Vincent would have voted yes to Europe.' (\cite[160–161]{MosegaardHansen2008},  glosses added)
\end{exe}

In future contexts, on the other hand, it is the earlier alternatives that are more proximal. With future reference, \lq as late as\rq{ }operators can thus be read as referring to a relatively distant time,\il{German|(} as in (\ref{exTimeScalarFutureGerman}). This yields an overlap in interpretation with the \lq as far removed\rq{ }type of use.

\begin{exe}
	\ex German\label{exTimeScalarFutureGerman}\\
	\gll \textbf{Noch} \textbf{in} \textbf{zehn} \textbf{Jahr}-\textbf{en} werd-en wir die Früchte dieser Entscheidung genieß-en.\\
	still in ten year-\textsc{dat}.\textsc{pl} \textsc{fut}.\textsc{aux}-1\textsc{pl} 1\textsc{pl} \textsc{def}.\textsc{acc}.\textsc{pl} fruit.\textsc{pl} \textsc{prox}:\textsc{gen}.\textsc{sg}.\textsc{f} decision(\textsc{f}) enjoy-\textsc{inf}\\
	\glt \lq \textbf{Even in ten years time} we will still be enjoying the fruits of this decision.' \parencite[182]{Koenig1979}
\end{exe}
\il{German|)}\is{remoteness|)}

\il{French|(}
To gain a better understanding of the functional motivation behind this use, it is worthwhile taking a diachronic perspective. As  \textcite[161–162]{MosegaardHansen2008} and \textcite{Shetter1966} have pointed out before me, a bridge from phasal polarity \textsc{still} to \lq as late as\rq{ }can be found in examples of the type in (\ref{exTimeScalarBridging}). Here, the temporal adverbial refers to a point in time and the viewpoint\is{aspect} is imperfective.\is{imperfective} In communicative terms, there is little difference between \textit{encore} in (\ref{exTimeScalarBridging}) serving as an exponent of \textsc{still} or as time-scalar additive \lq as late as\rq{ }(with \textit{ex situ} focus). In either case, Vincent's pro-European stance is normally understood as persisting\is{persistence|(} from an earlier time, but not necessarily beyond the year 2003. A similar case can be found in the recurrent attestations of \textsc{still} expressions together with adverbs like \lq today' or \lq now\rq{},\il{Spanish|(} as in (\ref{exTimeScalarSpanishStillToday}).


\begin{exe}
	\ex\label{exTimeScalarBridging}\ili{French}\\
	\gll Vincent aurait \textbf{encore} voté oui à l’Europe \textbf{en} \textbf{2003}.\\
	V. have.\textsc{cond} still vote.\textsc{ptcp} yes to \textsc{def}.\textsc{f}.\textsc{sg}-Europe in 2003\\
	\glt \lq Vincent would \textbf{still} have voted yes to Europe \textbf{in} \textbf{2003}.’
	\\(\cite[160–161]{MosegaardHansen2008},  glosses added)

	\ex Spanish\label{exTimeScalarSpanishStillToday}\\
	\gll \textbf{Hoy} \textbf{todavía} es bastante popular la noción de que el dolor es un ingrediente necesari-o para el desarrollo de un carácter sólid-o en la persona.\\
today still \textsc{cop}.3\textsc{sg} rather popular \textsc{def}.\textsc{sg}.\textsc{f} notion(\textsc{f}) of \textsc{subord} \textsc{def}.\textsc{sg}.\textsc{m} pain(\textsc{m}) \textsc{cop}.3\textsc{sg} \textsc{indef}.\textsc{sg}.\textsc{m} ingredient(\textsc{m}) necessary-\textsc{m} for \textsc{def}.\textsc{sg}.\textsc{m} development(\textsc{m}) of \textsc{indef}.\textsc{sg}.\textsc{m} character(\textsc{m}) solid-\textsc{m} in \textsc{indef}.\textsc{sg}.\textsc{f} person(\textsc{f})\\
	\glt \lq \textbf{Today} the idea that pain is a necessary ingredient for developing a robust character is \textbf{still} rather popular / \textbf{Even} \textbf{today} the idea …'\\(CORPES XXI,  glosses added)
  \end{exe}
\il{French|)}\il{Spanish|)}

\is{topic time|(}In other words, contexts like the ones in (\ref{exTimeScalarBridging}, \ref{exTimeScalarSpanishStillToday}) allow for a reanalysis of a \textsc{still} expression as a time-specific scalar additive operator, thereby instantiating the well-known tendency of linguistic change towards procedural\is{proceduralisation} and more subjective\is{subjectivity} meanings (\Cref{sectionSemasiologicalChange}). In semantic terms, this reinterpretation involves a reversal of dependencies. Thus, with phasal polarity \textsc{still} topic time is given and the question that is addressed pertains to the situation's polarity. In the \lq as late as\rq{ }use, on the other hand, topic time is the open variable \parencite{Krifka2000}. What is preserved, however, is the nature of the scale and its orientation. Thus, \textsc{still} presupposes the unfolding of a situation from an earlier time through to topic time, but not necessarily any further. In the same vein, with \lq as late as' the denotation of focus is the highest among the set of calendric values under consideration \parencite[161]{MosegaardHansen2008}; \Cref{figureTimeScalarLate} is a graphic comparison.

\begin{figure}[hbt]
	\centering
	\begin{subfigure}[b]{0.48\linewidth}
		\centering
		\begin{tikzpicture}[node distance = 0pt]
			\node[mynode, text width=\superbreit, fill=cyan, very near start] (spain){Situation};
			\node[mynode, fill=none, text width=\superschmal, right= of spain] (france){(\small{$\Diamond$}\neg{ }Sit.)};	
			\draw[-, densely dashed] ($(spain.north east)+ (-0.2pt,0) $) to ($(spain.south east)+ (-0.2pt,-0.5*\hoehe) $);
			\draw[-, densely dashed] ($(spain.north east)+ (-\hoehe-0.2pt,0) $) to ($(spain.south east)+ (-\hoehe-0.2pt,-0.5*\hoehe) $);
				\draw[-, densely dashed] ($(spain.north east)+ (-\hoehe-0.2pt,0)$) to ($(spain.south east)+ (-\hoehe-0.2pt,-0.5*\hoehe) + (0.2pt,0)$) node 	[below, align=center, label distance=0, xshift=0.5*\hoehe] {\strut{}Topic\\time\strut};
				\draw[-, densely dashed] ($(spain.north west)+(0.2pt,0)$) to ($(spain.south west)+ (0.2pt,-0.5*\hoehe) $);
	\node[below, align=left,label distance=0, anchor=north west] at ($(spain.south west)+(0,-0.5*\hoehe)$) {\strut{}Prior\\runtime\strut};
		
			\draw[-latex, line width=0.5pt]  (spain.south west) to  ($(france.south east)+(1ex,0)$) node [right] {t};
		\end{tikzpicture}
	\subcaption{\textsc{still}}	
	\end{subfigure}
	\begin{subfigure}[b]{0.48\linewidth}
		\centering

			\begin{tikzpicture}[node distance = 0pt]
			\node[mynode, text width=\superbreit, fill=cyan, very near start] (spain){Under consid.};
			\node[mynode, pattern=north west lines, fill=none, text width=\superschmal, right= of spain] (france){};	
		
			\draw[-, densely dashed] ($(spain.north east)+ (-0.2pt,0) $) to ($(spain.south east)+ (-0.2pt,-0.5*\hoehe) $);
			\draw[-, densely dashed] ($(spain.north east)+ (-\hoehe-0.2pt,0) $) to ($(spain.south east)+ (-\hoehe-0.2pt,-0.5*\hoehe) $)	node 	[below, align=center, label distance=0, xshift=0.5*\hoehe] {\strut{}Focus\\value\strut};
				\draw[-, densely dashed] ($(spain.north west)+(0.2pt,0)$) to ($(spain.south west)+ (0.2pt,-0.5*\hoehe) $);
	\node[below, align=left,label distance=0, anchor=north west] at ($(spain.south west)+(0,-0.5*\hoehe)$) {\strut{}Alternative\\values\strut};
		
			\draw[-latex, line width=0.5pt]  (spain.south west) to  ($(france.south east)+(1ex,0)$) node [right] {t};
		\end{tikzpicture}
	\subcaption{\lq as late as\rq{}}
	\end{subfigure}
	\caption{Schematic illustration of time-scalar additive use with earlier alternatives\label{figureTimeScalarLate}}
\end{figure}\is{persistence|)}\is{topic time|)}

Although contexts like the ones in (\ref{exTimeScalarBridging}, \ref{exTimeScalarSpanishStillToday}) offer a motivated explanation for the emergence of the \lq as late as' use, there are important aspects of meaning in favour of considering it a synchronic function in its own right (\textit{pace} \cite[162]{MosegaardHansen2008}).\il{German|(} To begin with, \textcite{Beck2020} observes that German \textit{noch} in \lq as late as\rq{ }function plus \isi{imperfective} viewpoint\is{aspect} does not require a prior runtime of the situation in question. Thus, in (\ref{exTimeScalarGermanCondo}) the speakers did not live in Danbury before the focus value of 1997, yet the sentence does not give rise to any contradiction. This is because what is entailed through the use of \textit{noch} here is merely that the year 1997 yields a more informative answer to the question under discussion than all earlier alternatives. While this stands in sharp contrast to the prior runtime presupposition triggered by \textsc{still}, it is in line with cross-linguistic findings on scalar additive operators, namely that the truth of the pragmatically weaker context propositions is merely a default assumption (\cite{GastvanderAuwera2011}; \cite{Schwartz2005}).

\begin{exe}
\ex German\\
Context: We had a Condo in Danbury between March and November 1997.\label{exTimeScalarGermanCondo}
	\begin{xlist}
	\exi{A:} \textit{Wie lange haben wir eigentlich in Mt. Kisco gewohnt?}\\
\lq For how long did we live in Mt. Kisco?'
	\exi{B:} \textit{So lang kann das nicht gewesen sein}.\\
\lq It can't have been that long.'
	\exi{}\gll \textbf{Noch} \textbf{1997} hab-en wir ja in Danbury ge-wohn-t.\\
still 1997 have-1\textsc{pl} 1\textsc{pl} \textsc{dm} in D. \textsc{ptcp}-reside-\textsc{ptcp}\\
	\glt \lq \textbf{As recently as 1997} we lived/were living in Danbury.'
	\\\parencite[30 fn13]{Beck2020}
	\end{xlist}
\end{exe}	

Another difference is found in the fact that the \lq as late as\rq{} use is perfectly compatible with a \isi{perfective} vantage point,\is{aspect} whereas \textsc{still} is not. This is illustrated in (\ref{extimeScalarPFVFrench}); also see (\ref{extimeScalarPFVGerman}) below.\il{German|)}

\begin{exe}
	
	\ex \ili{French} \label{extimeScalarPFVFrench}\\
	\gll \textbf{Encore} \textbf{hier}, \textbf{j'}-\textbf{ai} \textbf{vu} \textbf{le} \textbf{film} {\textbf{La} \textbf{Chute}}, sur la mort d’-Hitler, que je n’-avais pas pu voir quand il est sorti.\\
	still yesterday 1\textsc{sg}-have.1\textsc{sg} see.\textsc{ptcp} \textsc{def}.\textsc{sg}.\textsc{m} movie(\textsc{m}) Downfall about \textsc{def}.\textsc{sg}.\textsc{f} death(\textsc{f}) of-H. \textsc{subord} 1\textsc{sg} \textsc{neg}-have.\textsc{pst}.\textsc{ipfv}.1\textsc{sg} \textsc{neg} can.\textsc{ptcp} see.\textsc{inf} when 3\textsc{sg}.\textsc{m} \textsc{cop}.3\textsc{sg} leave.\textsc{ptcp}\\
	\glt \lq \textbf{Just yesterday}, \textbf{I watched the movie \textit{Downfall}}, about the death of Hitler, which I hadn't been able to watch when it came out.'
	\\(found online, glosses added)%\footnote{\url{https://www.saintbrice95.fr/a-la-une/actualites/actualites-2019/lacteur-michel-bouquet-sous-la-direction-du-saint-bricien-ulysse-di-gregorio-1113.html} (02 May, 2022).}
\end{exe}	

\is{syntax|(}These differences in meaning furthermore have syntactic repercussions. As is typical of focus-sensitive operators (e.g. \cite[ch. 2]{Koenig1991}), all expressions in question can form a single constituent with their focus, with which they move through the clause.\il{German|(} For instance, \textit{noch gestern} in (\ref{extimeScalarPFVGerman}) occupies the so-called \lq\lq forefield position" of the German clause, which can only host a single constituent. Being a syntactic sister to the focus, \ili{French} \textit{encore} is also attested within more complex syntactic configurations, such as presentative \textit{voilà} and its argument in (\ref{exTimeScalarFrenchVoila}).

\enlargethispage{\baselineskip}
\begin{exe}
	\ex German\label{extimeScalarPFVGerman}\\
	Context: About perpetual conflicts in the Aegean islands.\\
	\gll \textbf{Noch} \textbf{letzt}-\textbf{e} \textbf{Woche} kam es … zu … heftig-en Auseinandersetzung-en zwischen der Polizei und den Insel-bewohner-n.\\
	still last-\textsc{nom}.\textsc{sg}.\textsc{f} week(\textsc{f}) come.\textsc{pst}.3\textsc{sg} 3\textsc{sg}.\textsc{n} {} to {} severe-\textsc{dat}.\textsc{pl} conflict-\textsc{pl} between \textsc{def}.\textsc{dat}.\textsc{sg}.\textsc{f} police(\textsc{f}) and \textsc{def}.\textsc{dat}.\textsc{pl} island-inhabitant-\textsc{dat}.\textsc{pl}\\
	\glt \lq \textbf{As late as last week}, altercations between the police and the islanders occurred.\rq{ }(found online, glosses added)%\footnote{\url{https://www.nzz.ch/international/die-harte-linie-an-der-grenze-ist-in-griechenland-populaer-ld.1544107} (28 November, 2022).}
    \il{German|)}
 
	\ex \ili{French}\label{exTimeScalarFrenchVoila} \\
	 \gll \textbf{Voilà} \textbf{encore} \textbf{quelques} \textbf{mois}, l'-idée eût paru aussi provocatrice qu’-interdire la cigarette.\\
	\textsc{prstt} still a\_few month.\textsc{pl} \textsc{def}.\textsc{sg}-idea have.\textsc{pst}.\textsc{sbjv}.3\textsc{sg} appear.\textsc{ptcp} as provocative.\textsc{f} as-forbid.\textsc{inf} \textsc{def}.\textsc{sg}.\textsc{f} cigarette(\textsc{f})\\
	\glt \lq \textbf{Only a few months ago} the idea would have been as provocative as banning cigarettes.' (\textit{Puest-France}, cited in \cite[258]{Fuchs1993})	
\end{exe}\is{syntax|)}	

To summarize, the time-scalar additive \lq as late as' function is a diachronic extension of phasal polarity \textsc{still} in the context of temporal frame adverbials. From there, it inherits the temporal scale and its orientation. In synchronic terms, this use does, however, differ from its precursor in showing the characteristic syntactic\is{syntax} and meaningful traits of scalar focus operators.

\subsubsubsection{A closer look: The `as early as' use}\il{Lezgian|(}\is{topic time|(} 
One expression in the sample, namely Lezgian \textit{hele}, not only has the \lq as late as\rq{ }use just discussed, but also one in which it relates the focus to later alternatives. The examples in (\ref{exTimeScalarLezgian}) are illustrations. In (\ref{exTimeScalarLezgian1}) the calendric value corresponding to \textit{cʼerid} \textit{lahaj} \lq seventeenth century\rq{ }is related to earlier alternatives, whereas in (\ref{exTimeScalarLezgian2}) the denotation of \textit{naq'} \lq yesterday\rq{ }is compared to later times.\pagebreak

\begin{exe}
	\ex \label{exTimeScalarLezgian}
	\begin{xlist}
		\exi{}Lezgian
		\ex \label{exTimeScalarLezgian1}
		 \gll \textbf{Hele} \textbf{cʼerid} \textbf{lahaj} \textbf{asir.d}-\textbf{a} fikir-zawa-j x̂i, …\\
		still seventeen \textsc{ord} century-\textsc{iness} think-\textsc{ipfv}-\textsc{pst} \textsc{dm}\\
		\glt \lq \textbf{As late as}/\textbf{even in the seventeenth} century people (still) believed that …' \parencite[90]{Haspelmath1991}
	
	\ex \label{exTimeScalarLezgian2}
	\gll Am mus ata-na? – \textbf{Hele} \textbf{naq'}.\\
	3\textsc{sg}.\textsc{abs} when come-\textsc{aor} {} still yesterday\\
\glt \lq When did she come? -- \textbf{As early as yesterday.}\rq{}
\\\parencite[85]{Haspelmath1991}
	\end{xlist}
\end{exe}

If disambiguation is called for, it appears that Lezgian speakers can resort to using a \isi{restrictive} operator \lq no earlier than\rq{ }to specify a comparatively late topic time, as in (\ref{exTimeScalarLezgian3}).

\begin{exe}
	\ex Lezgian\label{exTimeScalarLezgian3}\\
	\gll \textbf{Anžax}\textup{/}\textbf{tek} \textbf{q'we} \textbf{jiq̄a}-\textbf{n} \textbf{wilik} am saǧ tir.\\
		only/only two day-\textsc{gen} before 3\textsc{sg}.\textsc{abs} healthy \textsc{cop}.\textsc{pst}\\
		\glt \lq \textbf{As recently as/only two days ago} she was healthy.\rq{}
		\\\parencite[90]{Haspelmath1991}
\end{exe}\is{topic time|)}

Unlike the case of \ili{Spanish} \textit{aún} that I briefly addressed above, the underspecification of \textit{hele} cannot be traced back to a more general scalar additive function. This begs the question of what lies at the heart of this peculiarity. Again, taking a diachronic perspective allows for valuable insights in the form of two possible strains of motivation. First, though \textit{hele} started out as a \textsc{still} expression, this function has become highly restricted in the present-day language, where the item primarily serves as a marker of \textsc{already};\is{already} I discuss this pathway of change in \Cref{sectionInterrogativeYet}. It is not hard to see how the uses with earlier vs. later alternatives might be a reflection of these two functions. In other words, the \lq as early as\rq{ }use in (\ref{exTimeScalarLezgian2}) could be understood as no different from the time-scalar use of a more canonical  \textsc{already}\is{already} expression, such as \il{Spanish|(}Spanish \textit{ya} in (\ref{extimeScalaralreadySpanish}).\footnote{This also seems to be the view taken by \citeauthor{Haspelmath1991} (\citeyear{Haspelmath1991}, \citeyear{Haspelmath1993}), who discusses the \lq as early as\rq{ }use{ }under the heading of \lq already\rq{}.\is{already}}

\begin{exe}
	\ex Spanish\label{extimeScalaralreadySpanish}\is{already}\\
	\gll No es nada nuev-o. \textbf{Ya} \textbf{hace} \textbf{año}-\textbf{s} … pod-ía-mos encontr-ar fals-a-s Nintendo de 8 bit-s y consola-s Sega-s fabric-ad-a-s en Taiwan.\\
	\textsc{neg} \textsc{cop}.3\textsc{sg} nothing new-\textsc{m} already ago year-\textsc{pl} {} can-\textsc{pst}.\textsc{ipfv}-1\textsc{pl} find-\textsc{inf} false-\textsc{f}-\textsc{pl} N. of 8 bit-\textsc{pl} and console(\textsc{f})-\textsc{pl} S.-\textsc{pl} fabricate-\textsc{ptcp}-\textsc{f}-\textsc{pl} in T.\\
	\glt \lq It's nothing new. \textbf{Already years ago} … we'd find fake 8 bit Nintendo and Sega consoles built in Taiwan.\rq{ }(CORPES XXI, glosses added)
\end{exe}\il{Spanish|)}

\is{persistence|(}\is{topic time|(}
Another source of the \lq as early as\rq{ }use might be the persistent time frame use that I discuss in \Cref{sectionTemporalFrameTT}. In this function, the expression modifies a temporal frame adverbial and signals the persistence of an established time frame. This often gives rise to a scalar inference of earliness. For instance, in (\ref{exContinuativeTTLezgian2}) an alternative course of events would have it that the third person argument leaves only after taking notice of the speaker (when it is \textit{no longer} \lq before she [possibly] saw me\rq{}). It is conceivable that such an erstwhile inference be conventionalised, thereby giving rise to an \lq as early as\rq{ }use; I discuss  a partially overlapping scenario below. Quite possibly though, the shift of \textit{hele} from \textsc{still} to \textsc{already}\is{already} and the earliness inference of the persistent time frame use conspired towards one and the same outcome.

\begin{exe}
	\ex\label{exContinuativeTTLezgian2}Lezgian\\
	\gll \textbf{Hele} \textbf{zun} \textbf{akwa}-\textbf{daldi}, am qʰfe-na.\\
	still 1\textsc{sg}.\textsc{abs} see-\textsc{cvb}:before 3\textsc{sg}.\textsc{abs} leave-\textsc{aor}\\
	\glt \lq \textbf{Even before she saw me}, she left. / \textbf{Noch ehe sie mich sah}, ging sie weg.' \parencite[90]{Haspelmath1991}
\end{exe}\is{persistence|)}\il{Lezgian|)}\is{topic time|)}

\subsubsubsection{A closer look: The `as far removed as' use}\il{Hebrew, Modern|(}\il{Serbian|(}\il{Croatian|(}\il{Bosnian|(}As I indicated initially, \textsc{still} expressions as time-scalar additives do not always operate on a scale of calendric values. Instead, they can also involve degrees of distance from the temporal origo. In my sample, this use is found with Modern Hebrew \textit{ʕod} and Serbian\hyp Croatian\hyp Bosnian \textit{još}. Example (\ref{exTimeScalarSerbianDriver}) illustrates \textit{još}. Here, the denotation of \textit{prije 15 hodina} \lq 15 years prior\rq{ }is related to less removed times, thereby emphasising the culprit driver's refractoriness. 

\begin{exe}
	\ex Serbian-Bosnian-Croatian\label{exTimeScalarSerbianDriver}\\
	\gll Obavljenim nadzorom utvrđen-o je da vozač upravlja automobilom iako mu je {vozačka dozvola} oduzet-a \textbf{još} \textbf{prije} \textbf{15} \textbf{godina}.\\
	carry\_out.\textsc{pass}.\textsc{inst}.\textsc{sg}.\textsc{m} supervision(\textsc{m}).\textsc{ins}.\textsc{sg} strengthen.\textsc{pfv}:\textsc{pass}.\textsc{ptcp}-\textsc{sg}.\textsc{n} \textsc{cop}.3\textsc{sg} \textsc{comp} driver.\textsc{nom}.\textsc{sg} drive.\textsc{ipfv}.3\textsc{sg} car.\textsc{inst}.\textsc{sg} although 3\textsc{sg}.\textsc{dat}.\textsc{m} \textsc{cop}.3\textsc{sg} driver's\_licence(\textsc{f}).\textsc{nom}.\textsc{sg}  take\_away.\textsc{pfv}:\textsc{pass}.\textsc{ptcp}-\textsc{sg}.\textsc{f} still before 15 year.\textsc{gen}.\textsc{pl}\\
	\glt \lq Through the investigation it was established that the driver was steering the car although his driver's licence had been revoked \textbf{as far back as} \textbf{15} \textbf{years} \textbf{ago}.' (found online, glosses added)%\footnote{\url{http://www.radio-maestral.hr/pula/vozio-pijan-a-vozacka-mu-je-oduzeta-jos-prije-15-godina/} (09 Februrary, 2023).}
\end{exe}

\textit{Još} is, to a certain extent, ambiguous, as it allows for \lq as late as\rq{ }construals, as well. With Modern Hebrew \textit{ʕod}, on the other hand, \lq as far removed as\rq{ }is the only possible relevant interpretation. This is shown in (\ref{exTimeScalarHebrewBoth}).

\begin{exe}
	\ex \label{exTimeScalarHebrewBoth}
	\begin{xlist}
		\exi{}[]{Modern Hebrew}
		\ex[\#]{
		\gll \textbf{ʕod} \textbf{ke}-\textbf{adam} \textbf{zaken} hu išen harbe.\\
		still as-man old 3\textsc{sg}.\textsc{m} smoke.\textsc{pst}.3\textsc{sg}.\textsc{m} a\_lot\\
		\glt (intended: \lq Even/as late as as an old man he (still) smoked a lot.\rq)}
		\ex[]{\gll \textbf{ʕod} \textbf{ke}-\textbf{adam} \textbf{tsaʕir} hu išen harbe.\label{exTimeScalarHebrewYoungMan}\\
		still as-man young.\textsc{m} 3\textsc{sg}.\textsc{m} smoke.\textsc{pst}.3\textsc{sg}.\textsc{m} a\_lot\\
		\glt \lq All the way back as a young man he (already) smoked a lot.\rq
		\\(Yael Greenberg, p.c.)}
	\end{xlist}
\end{exe}

On the basis of (\ref{exTimeScalarSerbianDriver}, \ref{exTimeScalarHebrewBoth}), one may be tempted to suspect that the use in question is actually an \lq as early as\rq{ }one, similar to what I discussed above for \ili{Lezgian} \textit{hele} and for \textsc{already} expressions.\is{already} That this is not the case becomes clear once more data are taken into account.\footnote{This following discussion owes much to \citeauthor{Boguslavsky1996}'s (\citeyear{Boguslavsky1996})  and  \citeauthor{Mustajoki1988}'s (\citeyear{Mustajoki1988}) exposition of the parallel facts found with \ili{Russian} \textit{eščë}, a cognate of Serbian\hyp Croatian\hyp Bosnian \textit{još}.} To begin with, there is a marked difference in meaning between the two types of operators, as illustrated in the pairs of examples in  (\ref{exTimeScalarComparisonStill}, \ref{exTimeScalarComparisonAlready}). While in both cases the situation in question manifests itself at a comparatively early time, the \textsc{still} expressions in (\ref{exTimeScalarComparisonStill}) contribute a notion of temporal dissociation that is not found with the \textsc{already} expressions in (\ref{exTimeScalarComparisonAlready}).\is{already}

\begin{exe}
		\ex\label{exTimeScalarComparisonStill}\is{already}
	\begin{xlist}
		\ex Modern Hebrew\\
		\gll 
		\textbf{ʕod} \textbf{be}-\textbf{gel} \textbf{šeš} \textbf{yadʕ}-\textbf{a} li-qroʕ latinit.\\
still at-age six know.\textsc{pst}-3\textsc{sg}.\textsc{f} to-read Latin\\
		
		\ex Serbian-Croatian-Bosnian\\
		\gll \textbf{Još} \textbf{sa} \textbf{šest} \textbf{godina} zna-la je čitati latinski.\\
still at six year.\textsc{gen}.\textsc{pl} know.\textsc{ipfv}.\textsc{ptcp}-\textsc{sg}.\textsc{f} \textsc{cop}.3\textsc{sg} read.\textsc{ipfv}.\textsc{inf} Latin.\textsc{acc}.\textsc{sg}\\
	\glt \lq \textbf{All the way back at the age of six} she could (already) read Latin.\rq
	\\(Jurica Polančec and Yael Greenberg, p.c.)
	\end{xlist}
	
		\ex\label{exTimeScalarComparisonAlready}\is{already}
	\begin{xlist}
		\ex Modern Hebrew\\
		\gll \textbf{Kvar} \textbf{be}-\textbf{gel} \textbf{šeš} \textbf{yadʕ}-\textbf{a} li-qroʕ latinit.\\ 
already at-age six can.\textsc{pst}-3\textbf{sg}.\textsc{f} to-read Latin\\
		\ex Serbian-Croatian-Bosnian\\
		\gll \textbf{Več} \textbf{sa} \textbf{šest} \textbf{godina} zna-la je čitati latinski.\\
still at six year.\textsc{gen}.\textsc{pl} know.\textsc{ipfv}.\textsc{ptcp}-\textsc{sg}.\textsc{f} \textsc{cop}.3\textsc{sg} read.\textsc{ipfv}.\textsc{inf} Latin.\textsc{acc}.\textsc{sg}\\
	\glt \lq \textbf{As early as at the age of six} she could (already) read Latin.\rq
	\\(Jurica Polančec and Yael Greenberg, p.c.)
	\end{xlist}
\end{exe}

These differences become even more pronounced in contextualised attestations where the two sets of expressions cannot be freely swapped, such as in (\ref{exTimeScalarNuclei}, \ref{exTimeScalarPlagiarism}). What is at stake in these examples is a forward leap on the temporal axis, and by means of the \textsc{already}\is{already} expressions \textit{kvar} and \textit{već} the amount of time elapsed is related to greater alternatives. This, in turn, clashes with the \lq all the way back\rq{ }notion that the \textsc{still} expressions \textit{ʕod} and \textit{još} would provide.  

\begin{exe}
	\ex Modern Hebrew\label{exTimeScalarNuclei}\is{already}\\
	 Context: About a fast-acting biochemical agent.\\
	\gll \textbf{Kvar} (\#\textbf{ʕod})  \textbf{axarey} \textbf{10} \textbf{daq}-\textbf{ot} \textbf{mi}-\textbf{hosafa}-\textbf{t} \textbf{ha}-\textbf{ħomer} hay-u garʕin-im me-ħuts l-a-ta’-im.\\
	already \phantom{\#(}still after 10 minute-\textsc{pl} from-addition-\textsc{cs} \textsc{def}-substance \textsc{cop}.\textsc{pst}-3\textsc{pl} nucleus-\textsc{pl} from-outside to-\textsc{def}-cell-\textsc{pl}\\
	\glt \lq \textbf{Already} \textbf{ten minutes after the addition of the chemical}, some nuclei are found outside the cells.\rq{ }(Yael Greenberg, p.c.)
	
	\ex Serbian-Croatian-Bosnian\label{exTimeScalarPlagiarism}\\
	Context: About a Croatian politician that graduated in 2011.\is{already}\\
	\gll Međutim, \textbf{već} \textup{(\#}\textbf{još}\textup{)} \textbf{2012.} \textbf{godine} se naša-o u skupini od 18 policijskih dužnosnika koje se sumnjiči-lo za plagir-anje diplom-sk-og.\\
	however already \phantom{(\#}still 2012 year.\textsc{gen}.\textsc{sg} \textsc{refl} find.\textsc{pfv}.\textsc{ptcp}-\textsc{sg}.\textsc{m} at group.\textsc{loc}.\textsc{sg} from 18 constabulary.\textsc{gen}.\textsc{pl} official.\textsc{gen}.\textsc{pl} \textsc{rel}.\textsc{acc}.\textsc{pl}.\textsc{n} \textsc{refl} suspect.\textsc{ipfv}.\textsc{ptcp}-\textsc{pl}.\textsc{n} for plagiarise-\textsc{nmlz} diploma-\textsc{adj}-\textsc{gen}.\textsc{sg}\\
	\glt \lq However, \textbf{already in 2012}, he found himself in a group of 18 police officials who were suspected of plagiarising their diploma.\rq{ }(found online and Jurica Polančec, p.c.)%\footnote{\url{https://www.index.hr/vijesti/clanak/barisic-nije-jedini-pogledajte-i-ostale-poznate-hrvatske-plagijatore/943153.aspx} (24 November, 2022).}
\end{exe}

\is{tense|(}\is{topic time|(}
The examples I have discussed up to this point all feature past tense contexts. Consistent with the observations there, in the future topic time is construed as more remote\is{remoteness} than the alternatives under consideration; see (\ref{exTimeScalarRemoteFutureHebrew}, \ref{exTimeScalarRemoteFutureSerbian}). This leads to an overlap in interpretation with the \lq as late as\rq{ }type of use discussed above.

\begin{exe}
	\ex Modern Hebrew\label{exTimeScalarRemoteFutureHebrew}\\
	\gll \textbf{ʕod} \textbf{ha}-\textbf{nin}-\textbf{im} \textbf{šel}-\textbf{i} ye-šalm-u et ha-ħovot ha-ʼele.\\
	still \textsc{def}-great\_grandchild-\textsc{pl} \textsc{poss}-1\textsc{sg} 3-pay.\textsc{fut}-\textsc{pl} \textsc{acc} \textsc{def}-obligation.\textsc{pl} \textsc{def}-\textsc{prox}.\textsc{pl}\\
	\glt \lq \textbf{Even} (\textbf{someone as far removed as}) \textbf{my} \textbf{grandchildren} will (still) be paying off these debts.\rq{ }(Yael Greenberg, p.c.)
	
	\ex Serbian-Croatian-Bosnian\label{exTimeScalarRemoteFutureSerbian}\\
	\gll \textbf{Još} \textbf{za} \textbf{sto} \textbf{godina} ljudi će govoriti o ovome.\\
	still at hundred year.\textsc{gen}.\textsc{pl} people will.3\textsc{sg} speak.\textsc{ipfv}.\textsc{inf} on \textsc{prox}.\textsc{loc}.\textsc{sg}.\textsc{m}\\
	\glt \lq \textbf{Even} (\textbf{as far removed as}) \textbf{in a hundred years} people will (still) speak about this.\rq{ }(Stefan Savić, p.c.)
\end{exe}\is{tense|)}\is{topic time|)}

To make sense of \textsc{still} expressions signalling \lq as far removed as\rq{}, it is worthwhile once again taking a diachronic perspective. With a few augmentations,  \citeauthor{Mustajoki1988}'s (\citeyear{Mustajoki1988}) insightful observations on \ili{Russian} \textit{eščë} can be easily translated into a motivated historical scenario.\is{persistence|(} Thus, \citeauthor{Mustajoki1988} remarks that different contextual manifestations of the \lq as far removed as\rq{ }use show varying degrees of conceptual similarity to the persistent time frame use (\Cref{sectionTemporalFrameTT}).\is{topic time|(} To briefly recapitulate, in this use, which is attested for both relevant items, a \textsc{still} expression combines with a constituent referring to an established time span and signals \lq\lq that the event in question occurs while the time specification is still valid: \lq it is still T, when e occurs'" \parencite[202]{Loebner1989}. Example (\ref{exTimeScalarSerbianContinuative}) is an illustration.

 \begin{exe}
	\ex Serbian-Croatian-Bosnian\label{exTimeScalarSerbianContinuative}\\
	Context: There was an accident involving a lorry and a cyclist.\\
	\gll Žrtva je umr-la \textbf{još} \textbf{na} \textbf{licu} \textbf{mesta}.\\
	victim(\textsc{f}) \textsc{cop}.3\textsc{sg} die.\textsc{ptcp}-\textsc{sg}.\textsc{f} still at face.\textsc{loc}.\textsc{sg} place.\textsc{gen}.\textsc{sg}\\
	\glt \lq The victim died \textbf{right at (lit. still at) the scene of the accident}.'
	\\(Stefan Savič, p.c.) 
\end{exe}

Taking this function as a point of departure, one may assume a gradual process of context expansion, accompanied by a relaxation of appropriateness conditions and instantiating the tendency for meaning to become more subjective\is{subjectivity} (\Cref{sectionSemasiologicalChange}). In a first step, cases like (\ref{exTimeScalarSerbianContinuative}) would have provided a model for the use of the relevant items together with foci referring to stages in life, phases in a process, and the like, such as in (\ref{exTimeScalarHebrewYoungMan}), repeated below, and in (\ref{exTimeScalarSerbianWomb}). Quite possibly, this was facilitated by cases in which the original requirement for an established, salient topic time was satisfied via accomodation.\footnote{See \textcite{SchwenterWaltereit2010} on presupposition accommodation in functional change.}

\begin{exe}[(225b)]
		\exr{exTimeScalarHebrewYoungMan}
	Modern Hebrew\\		
		\gll \textbf{ʕod} \textbf{ke}-\textbf{adam} \textbf{tsaʕir} hu išen harbe.\\
		still as-man young.\textsc{m} 3\textsc{sg}.\textsc{m} smoke.\textsc{pst}.3\textsc{sg}.\textsc{m} a\_lot\\
		\glt \lq All the way \textbf{back as a young man} he (already) smoked a lot.\rq
		\\(Yael Greenberg, p.c.)
	\ex Serbian-Croatian-Bosnian \label{exTimeScalarSerbianWomb}\\
	\gll Jezik se uči \textbf{još} \textbf{u} \textbf{maternici}, pokaza-la je studija finskog Sveučilišta Helsinki.\\
	language \textsc{refl} learn.\textsc{ipfv}.3\textsc{sg} still at womb.\textsc{loc} show.\textsc{ptcp}-\textsc{sg}.\textsc{f} \textsc{cop}.3\textsc{sg} study(\textsc{f}) Finnish university.\textsc{gen}.\textsc{sg} H.\\
	\glt \lq Language is (already) learned \textbf{(as far back as) in the womb}, according to a Finnish study at the University of Helsinki.\rq{ }
	\\(found online, glosses added).%\footnote{\url{https://www.24sata.hr/lifestyle/pazite-sto-pricate-bebe-slusaju-i-pamte-rijeci-jos-u-maternici-329923} (27 April, 2022).}
\end{exe}

\is{precedence|(}\il{German|(}
A similar transfer would have taken cases like the one illustrated in (\ref{exTimeScalarGermanPrecedence}) for German \textit{noch}, which pairs the persistence of an established time frame with the relational notion of precedence, as its model (this seems not be possible in present-day Modern Hebrew and Serbian\hyp Croatian\hyp Bosnian). The outcome would have been purely scalar instances like (\ref{exTimeScalarHebrewEvenBefore}, \ref{exTimeScalarSerbianEvenBefore}), which establish a new topic time. 

\begin{exe}
	\ex German\label{exTimeScalarGermanPrecedence}\\
\gll Die Maske-n-pflicht … könnte … \textbf{noch} \textbf{vor} \textbf{Ostern} ge-locker-t oder ganz abgeschafft werd-en.\\
\textsc{def}.\textsc{nom}.\textsc{sg}.\textsc{f} mask-\textsc{pl}-requirement(\textsc{f}) {} can.\textsc{cond}.3\textsc{sg} {} still before Easter \textsc{ptcp}-loosen-\textsc{ptcp} or completely abandon.\textsc{ptcp} become-\textsc{inf}\\
\glt The requirement to wear a [face] mask … might be loosened or abandoned altogether … \textbf{even} [lit. while it is still] \textbf{before} \textbf{Easter}.'\\(found online, glosses added)%\footnote{\url{https://www.mallorcazeitung.es/gesundheit/2022/04/04/maskenpflicht-innenraumen-spanien-ostern-64628568.html} (06 April 2022).}
\il{German|)}

	\ex Modern Hebrew\label{exTimeScalarHebrewEvenBefore}\\
	Context: Commenting on a blog post stating that the first gas-powered buses entered operation in 2019.\\
	\gll Aval nir’a l-i še-otobus-im še-mufʕal-im ʕal gaz nixnes-u be-{Beitar Ilit} \textbf{ʕod} \textbf{lifney} \textbf{2019}.\\
	but seems to-1\textsc{sg} \textsc{subord}-autobus(\textsc{m})-\textsc{pl} \textsc{subord}-operate.\textsc{pass}-\textsc{pl}.\textsc{m} by gas enter.\textsc{pst}-3\textsc{pl} at-{B. I.} still before 2019\\
	\glt \lq But it seems to me that gas-powered buses entered operation in Beitar Ilit \textbf{even} \textbf{before} \textbf{2019}.\rq{ }(found online, glosses added)%\footnote{\url{http://israelbikebus.blogspot.com/2019/12/blog-post.html} (03 November, 2022).} 
		
	\ex Serbian-Croatian-Bosnian\label{exTimeScalarSerbianEvenBefore}\\
	\gll Bog nas je, \textbf{još} \textbf{prije} \textbf{nego} \textbf{što} \textbf{je} \textbf{stvori}-\textbf{o} \textbf{svijet}, izabra-o da u Kristu budemo sveti i bez nedostatka u njegovim očima.\\
God 1\textsc{pl}.\textsc{acc} \textsc{cop}.3\textsc{sg} still before than as \textsc{cop}.3\textsc{sg} create.\textsc{ipfv}.\textsc{ptcp}-\textsc{sg}.\textsc{m} world.\textsc{acc}.\textsc{sg} choose.\textsc{pfv}.\textsc{ptcp}-\textsc{sg}.\textsc{m} \textsc{comp} at Christ \textsc{cop}.1\textsc{pl} holy.\textsc{nom}.\textsc{pl}.\textsc{m} and without shortage.\textsc{gen}.\textsc{sg} at \textsc{poss}.3\textsc{sg}.\textsc{m}:\textsc{loc}.\textsc{pl} eye.\textsc{loc}.\textsc{pl}\\
	\glt \lq For he chose us in him \textbf{(even) before the creation of the world} to be holy and blameless in his sight.\rq{ }(Eph. 1:4, \textit{Knjiga o Kristu},  glosses added)	
\end{exe}
\is{precedence|)}

Before moving on to the last stages of this scenario, it is worthwhile to take a brief pause and to highlight a few key components. First, the persistent time frame use, which I assume to be the starting point, often gives rise to a scalar inference of earliness. For instance, in the accident example (\ref{exTimeScalarSerbianContinuative}) above, it could be imagined that the victim only dies at a later stage, such as after being loaded into an ambulance, or after being brought to hospital. Secondly, the illustrations up to this point all retain what is arguably the essence of phasal polarity, namely the segmentation of a time line into two sequential intervals of contrasting polarity. The first phase includes topic time (< \lq while it was still …\rq{}) and the second phase includes \isi{utterance time} or some other salient evaluation time. In other words, an example like (\ref{exTimeScalarSerbianEvenBefore}) can be read as invoking a contrast between the \lq then\rq{ }before the creation of the world and the \lq now\rq{ }at which it exists, hence a subjective\is{subjectivity} element of temporal distance. The same holds true, \textit{mutatis mutandis}, of (\ref{exTimeScalarHebrewYoungMan}, \ref{exTimeScalarSerbianWomb}, \ref{exTimeScalarHebrewEvenBefore}). A schematic illustration is given in \Cref{figureAsEarlyAsCreation}.
 
\setlength{\MinimumWidth}{\widthof{Choicex}}
\begin{figure}[hbt]\is{utterance time}
	\caption{Schematic illustration of (\ref{exTimeScalarSerbianEvenBefore}) \label{figureAsEarlyAsCreation}}
	\centering
			\begin{tikzpicture}[node distance = 0pt]
			\node[mynode, text width=1.5*\breit, fill=cyan, text opacity=1, fill opacity=.5, very near start] (A){Pre-creation};
				
			\node[mynode,  anchor=east, fill=cyan, text width=\MinimumWidth,draw=none, xshift=-0.5*\hoehe] (D) at (A.east) {Choice};
			\draw[thick] (D.south east) to (D.north east);
			\draw[-] (D.south west) to (D.north west);
			\node[mynode, fill=white, text width=\superbreit+\hoehe,align=center, right= of A] (C){(Post-creation)};		
			\draw[-, densely dashed] ($(A.south east)+(-0.2pt,0)$) to ($(A.north east)+ (-0.2pt,0.5*\hoehe) $);
			\draw[-, densely dashed] ($(A.south east)+ (-1*\hoehe-0.2pt,0) $) to ($(A.north east)+ (-1*\hoehe-0.2pt,0.5*\hoehe) $) node [above, align=center, label distance=0, xshift=0.5*\hoehe] {\strut{}Topic\\time\strut};
			\draw[-latex, line width=0.5pt]  (A.south west) to  ($(C.south east)+(1ex,0)$) node [right] {t};
			\draw[-, densely dashed] ($(C.south east)+ (-0.5*\hoehe-0.2pt,0) $) to ($(C.north east)+ (-0.5*\hoehe-0.2pt,0.5*\hoehe) $) node [above, align=center, anchor=south, label distance=0] {\strut{}Utter.\\time\strut};			 	\draw [dashed, decorate, decoration={brace,mirror,amplitude=0.75*\hoehe,raise=0}] (C.south west)   -- ($(C.south east)+(0,0)$); 
  \coordinate (X) at ($(C.south west)!0.5!(C.south east)+(0,-0.5*\hoehe)$)	;
 \node[anchor=north, below= of X]{\strut{}Later alternatives};	
		\end{tikzpicture}
\end{figure}\is{topic time|)}
  
\il{German|(}  
Support for the interpretation just outlined comes from German. Though the German \textsc{still} expression \textit{noch} does not generally have the \lq as far removed as\rq{ }use, it can acquire a strikingly similar reading in a very specific set of contexts, namely when the temporal origins of a persistent state-of-affair are at stake (\appref{appendixGermanSourcePersistentState}).\footnote{According to \textcite{Mustajoki1988}, the same usage is attested for \ili{Finnish} \textit{vielä} (not in my sample).} This is illustrated in (\ref{exTimeScalarGermanStaabsoffiziere}), which features the perhaps prototypical case, the verb–particle collocation \textit{stammen}…\textit{aus} \lq stem from\rq{}.  In an example like this, \textit{noch} arguably fulfils a twofold function \parencite{Shetter1966}. On the one hand it does its usual job of providing the main proposition with phasal polarity (\lq I had and continue to have a shyness…\rq{}). At the same time, it can be read as associating with the temporal predication about the source of the state (\lq …from when I was still in the army\rq{}). The union of these two interpretations yields a reading of a remnant object from a bygone era.

\begin{exe}
	\ex German\label{exTimeScalarGermanStaabsoffiziere}\\
	\textit{Ich habe eine natürliche Scheu vor Polizisten und Stabsoffizieren.}\\
	 \lq I have a natural shyness of policemen and staff officers.\rq
	 \exi{} \gll Das \textbf{stamm}-\textbf{t} \textbf{noch} \textbf{aus} \textbf{meiner} \textbf{Militär}-\textbf{zeit}.\\
	 3\textsc{sg}.\textsc{n} stem-3\textsc{sg} still from \textsc{poss}.1\textsc{sg}:\textsc{dat}.\textsc{sg}.\textsc{f} military-time(\textsc{f})\\
	\glt \lq \textbf{Goes back} (all the way) \textbf{to my army days}.' (Remarque, \textit{Drei Kameraden}, cited in \cite[466]{Shetter1966},  glosses added)	
\end{exe}
\il{German|)}

Returning to the overarching development, after the initial analogical acquisition of new contexts, the \lq as far removed as\rq{ }reading would eventually be applied to other time specifications, such as in (\ref{exTimeScalarHebrewFreeGuy}), and to cases where the associated constituent refers to people, as in (\ref{exTimeScalarEinstein}) above. As \textcite{Mustajoki1988} points out, in such cases an interpretation as \lq when it is/was still …\rq{ }requires quite a stretch of the imagination and the corresponding periphrases are hardly adequate.

\enlargethispage{\baselineskip}
\begin{exe}
	\ex Modern Hebrew\label{exTimeScalarHebrewFreeGuy}\\
	\gll Lemaʕase gam \lq{}Le-šaħrer et Guy\rq{} haya amur la-tset šana še-ʕavr-a ke-še-ha-treyler ha-rišon yatsa \textbf{ʕod} \textbf{be}-\textbf{2019}(!)\\
	actually also to-free \textsc{acc} guy \textsc{cop}.\textsc{pst}.3\textsc{sg}.\textsc{m} say:\textsc{ptcp}.\textsc{pass}.\textsc{sg}.\textsc{m} to-come\_out year(\textsc{f}) \textsc{subord}-pass.\textsc{pst}-\textsc{sg}.\textsc{f} as-\textsc{subord}-\textsc{def}-trailer \textsc{def}-first come\_out.\textsc{pst}.3\textsc{sg}.\textsc{m} still at-2019\\
	\glt \lq Actually, [the movie] \textit{Free Guy}, was also supposed to be released last year, whereas the first trailer came out \textbf{all the way back in 2019}(!)\rq{}
	\\ (found online, glosses added)%\footnote{\url{https://www.geekster.co.il/entertainment/freeguy} (03 November, 2022).}
\end{exe}

\setlength{\MinimumWidthB}{0.45\textwidth}
\setlength{\MinimumWidth}{2\MinimumWidthB+2em}
\setlength{\MinimumWidthC}{\widthof{Dissociation}}
\begin{figure}[bth]
	\caption{Hypothesised development 
	of the \lq as far removed as\rq{ }use \label{figureRemoved}}
	\begin{tikzpicture}

	\node[minimum width=\MinimumWidth] (a) {Persistent time frame\strut};
	\node[anchor=north east, minimum width=\MinimumWidthB] (c) at (a.south east) {\strut{}\lq still before Easter\rq{ }(\ref{exTimeScalarGermanPrecedence})};
  \node[anchor=north west, minimum width=\MinimumWidthB] (b) at (a.south west) {\strut{}\lq still at the scene …\rq{ }(\ref{exTimeScalarSerbianContinuative})};
        \node[inner sep=0pt,draw,rounded corners,fit=(a)(b)(c)] (A) {}; 
        \draw [thin, color=lightgray] (b.north west) to (c.north east);
	
	\node[draw=none, minimum width=\MinimumWidthB, below=1em of A.south west, anchor=north west, align=center] (c1) {\strut{}Stages in process, life, etc.};
	\node[anchor=north west, minimum width=\MinimumWidthB, align=center] (c2) at (c1.south west) {\strut{}\lq way back, as a young man\rq{ }(\ref{exTimeScalarHebrewYoungMan})\\ \lq way back in the womb\rq{ }(\ref{exTimeScalarSerbianWomb})\strut{}};
	 \node[inner sep=0pt,draw,rounded corners,fit=(c1)(c2)] (C) {}; 
        \draw [thin, color=lightgray] (c2.north west) to (c2.north east);
        
        	\node[draw=none, minimum width=\MinimumWidthB, below=1em of A.south east, anchor=north east, align=center] (d1) {\strut{}Scalar precedence};
	\node[anchor=north west, minimum width=\MinimumWidthB, align=center] (d2) at (d1.south west) {\strut{}\lq even before 2019\rq{ }(\ref{exTimeScalarHebrewEvenBefore})\\ \lq even before creation\rq{ }(\ref{exTimeScalarSerbianEvenBefore})\strut};
	 \node[inner sep=0pt,draw,rounded corners,fit=(d1)(d2)] (D) {}; 
        \draw [thin, color=lightgray] (d2.north west) to (d2.north east);
	

		\node [draw=none, minimum width=\MinimumWidth, below=1em of C.south west, anchor=north west, align=center] (e1) {\strut{}Other time specifications};
	\node [anchor=north, align=center] (e2) at (e1.south) {\strut{}\lq way back in 2019\rq{ }(\ref{exTimeScalarHebrewFreeGuy})\\\strut{}\lq someone as removed as Einstein\rq{ }(\ref{exTimeScalarEinstein})};
 \node[inner sep=0pt,draw,rounded corners,fit=(e1)(e2)] (E) {}; 
        \draw [thin, color=lightgray] (e1.south west) to (e1.south east);
	\draw[-latex, very thick] (b.south) to (C.north);
	\draw[-latex, very thick] (c.south) to (D.north);
	\draw[-latex, very thick] (C.south) to ($(C.south)+(0,-1em)$);
		\draw[-latex, very thick] (D.south) to ($(D.south)+(0,-1em)$);
   		\end{tikzpicture}
\end{figure}

In a nutshell, the use of \textsc{still} expressions as scalar \lq as far removed as\rq{ }operators can be understood as a functional extension that does not directly build on phasal polarity. Instead, it is mediated by the persistent time frame use and involves a gradual and schematic context expansion, accompanied by the conventionalisation of subjective\is{subjectivity} inferences and the relaxation of appropriateness conditions. Testing the empirical validity of this scenario, schematically outlined in \Cref{figureRemoved}, is an open task for future diachronic corpus studies.\is{focus|(}\is{scale|)}\is{persistence|)}\il{Hebrew, Modern|)}\il{Serbian|)}\il{Croatian|)}\il{Bosnian|)}

\subsection{Modifications of temporal clauses}\is{temporal clause|(}\is{subordination|(}
\label{sectionTemporalSubordination}
In this subsection, I discuss a set of uses that involve the modification of a temporal clause. I understand \textit{temporal clause} in a loose sense here, as covering subordination in the strictest sense, embedding through nominalisation, as well as frame-setting medial clauses of a clause chain and constructions that \textit{prima facie} involve the mere juxtaposition of two independent clauses (\lq\lq asyndetic hypotaxis" in the terms of \citeauthor{Gueldemann1998} \citeyear{Gueldemann1998}). This set of uses encompasses \textsc{still} expressions as markers of simultaneous\is{simultaneity} duration (\Cref{sectionSimultaneity}) and as parts of constructions that signal coextensive\is{coextensiveness} duration \lq as long as\rq{ } (\Cref{sectionAsLongAs}). Lastly, I briefly address the use of \textsc{still} expressions to signal precedence,\is{precedence} which, strictly speaking, relates to \textsc{not yet},\is{not yet} rather than to \textsc{still} (\Cref{sectionBefore}).

\subsubsection{Simultaneous duration}\label{sectionSimultaneity}\is{simultaneity|(}
\subsubsubsection{Introduction} 
Several sample expressions have a use as markers of simultaneous duration, understood here as a signal that the situation expressed in the matrix clause partially or fully overlaps with the one depicted in the temporal clause (cf. \cite[84]{Kortmann1997}).\il{Nahuatl, Classical|(} Example (\ref{exSimultaneousNahuatl1}) is an illustration from Classical Nahuatl. Here, the \textsc{still} expression \textit{oc} figures in a temporal clause introduced by the determiner \textit{in} and specifies the temporal overlap between the two situations at hand. As in this instance, the Classical Nahuatl case often, but not always, features an additional level of embedding through \textit{ic} \lq when\rq{}, i.e. lit. \lq when it is still the time that …\rq{}.

\begin{exe}
	\ex Classical Nahuatl\label{exSimultaneousNahuatl1}\\
	\gll Mācamo xi-còcoch-ti-ye-cān \textbf{in} \textbf{oc} \textbf{ic} \textbf{n}-\textbf{on}-\textbf{tē}-\textbf{machtia}.\\
	\textsc{proh} \textsc{sbjv}-doze-\textsc{lnk}-stay-\textsc{pl} \textsc{det} still when \textsc{subj}.1\textsc{sg}-\textsc{it}-\textsc{obj}.\textsc{indef}.\textsc{human}-teach\\
	\glt \lq Don’t be dozing off \textbf{while I’m teaching}.'
	\\(\cite[366]{LauneyMackay2011},  glosses added) 
\end{exe}\il{Nahuatl, Classical|)}

Note that I discuss the related, but semantically more specific, notion of coextensive\is{coextensiveness} duration separately in \Cref{sectionAsLongAs}.

\subsubsubsection{Distribution in the sample}
\Cref{tableWhile} lists the four sample expressions that can serve as markers of simultaneous duration use, be it in conjunction with other embedding strategies or in a \textit{prima facie} paratactic construction. As can be gathered, this function is found in four sample languages from North America. However, I rush to point out that this use is far from restricted to this continent. Outside of my sample, \textcite{Gueldemann1998} reports several relevant cases from the Bantu languages of Africa. In what follows, I first summarise \citeauthor{Gueldemann1998}'s observations, as they allow for insights into the North American cases as well. I then return to the sample expressions.

\begin{table}
\caption{Simultaneous duration\label{tableWhile}}
\begin{tabularx}{\textwidth}{llllQ}
\lsptoprule
	Macro-area & Language & Expr. & Appx & Additional notes\\\midrule
	N. America & Classical Nahuatl\il{Nahuatl, Classical} & \textit{oc} &  \ref{appendixClassicalNahuatlWhile} & Often as \textit{oc ic} lit. \lq{}(is) still when'\\
	& \ili{Creek} & (\textit{i})\textit{mônk}\footnote{Borderline case of a \textsc{still} expression.} & \ref{appendixCreekWhile} & In medial clauses\\
	& \ili{Maricopa} & -\textit{haay} &  \ref{appendixMaricopaWhile} & With inessive -\textit{ly}\\
	& \ili{Osage} & \textit{šó̜} &\ref{appendixOsageWhile}\\
\lspbottomrule
\end{tabularx}
\end{table}

\subsubsubsection{Interlude: Güldemann's observations on Bantu} 
In comparing the reflexes of Proto-Bantu \mbox{*\textit{ki}-} and \mbox{*\textit{ka}-} across the eastern branch of the family, \textcite{Gueldemann1998} observes a grammaticalisation\is{grammaticalisation}  cline: the expressions in question start out as \lq\lq persistives\rq\rq{ }\is{persistence}in main clauses and then spread to dependent clauses. Here, they first keep their original meaning and then undergo a shift to markers of simultaneity, often-times ending up as the only overt indicator of the dependency relation. Subsequently, the expressions may lose its productivity in main clauses, thereby ceasing to be phasal polarity items altogether. This chain of developments is schematised in \Cref{exWhileTG}, and two aspects are worth elaborating upon. For one, the use of the relevant expressions as markers of \textsc{still} (or a closely related notion) is diachronically prior. What is more, a semantic change occurs between the second and third stage, which leads to a functional mismatch between main and temporal clauses. This situation, which closely resembles the North American cases in my sample, is illustrated for \ili{Shona} \mbox{\textit{ci}-} in (\ref{exWhileShona}). In (\ref{exWhileShona1}) this prefix forms part of the main clause predicate and serves as a phasal polarity expression. In (\ref{exWhileShona2}), on the other hand,  \mbox{\textit{ci}-} occurs on the dependent predicate and signals simultaneous duration.\pagebreak

\begin{exe}
	\ex \label{exWhileShona}
	\begin{xlist}
		\exi{}\ili{Shona}
		\ex\label{exWhileShona1}
		\gll Ndi-\textbf{ci}-ri ku-tora.\\
		\textsc{subj}.1\textsc{sg}-still-\textsc{cop} \textsc{ncl}15(\textsc{inf})-talk\\
		\glt \lq I am \textbf{still} taking.'
		
		\ex\label{exWhileShona2}\il{Shona}
		\gll A-ka-reʋa so-mu-nhu \textbf{a}-\textbf{ci}-\textbf{rw}-\textbf{ira} \textbf{ru}-\textbf{penyu} \textbf{rw}-\textbf{ake}.\\
		\textsc{subj}.1\textsc{sg}-\textsc{rem}.\textsc{pst}-speak like-\textsc{ncl}1-person \textsc{subj}.\textsc{ncl}1-still-fight-\textsc{appl} \textsc{ncl}11-life \textsc{ncl}11-\textsc{poss}.\textsc{ncl}1\\
		\glt \lq He spoke like a person \textbf{fighting for his life}.'
		\\(\cite[259, 297]{Fortune1955},  glosses added)
	\end{xlist}
\end{exe}

\setlength{\MinimumWidth}{\widthof{\textsc{still}x}}
\setlength{\MinimumWidthB}{\widthof{dependentx}}
\tikzset{SimultanBox/.style={rectangle split, rectangle split parts=2,rectangle split part align=center, text width=\MinimumWidth, align=center, draw=none}}
\begin{figure}[tb]
	\caption{Schematic illustration of \citeauthor{Gueldemann1998} (\citeyear{Gueldemann1998})'s observations on Bantu persistives\is{persistence}\label{exWhileTG}}
	\begin{tikzpicture}[node distance = 0pt]
\node[SimultanBox, align=right, text width=10ex, draw=none] (N) {\strut main \nodepart{two} \strut dependent};	
\node[SimultanBox, right= of N] (A) {\strut\textsc{still} \nodepart{two} \strut \textemdash};		
\node[SimultanBox, right=of A] (B) {\strut \textsc{still} \nodepart{two} \strut \textsc{still} };	
\node[SimultanBox, right=of B, fill=cyan] (C) {\strut \textsc{still} \nodepart{two} \strut \textsc{sim}};	
\node[SimultanBox, right=of C] (D) {\strut \textemdash  \nodepart{two} \strut \textsc{sim}};	
\node[fit=(N)(D), rounded corners, draw, inner sep=0] {};
\draw[thin] (N.north east) to (N.south east);
\draw[thin] (A.north east) to (A.south east);
\draw[thin] (B.north east) to (B.south east);
\draw[thin] (C.north east) to (C.south east);
\draw[-latex] ($(N.south west)+(0,-\baselineskip)$)  -- ($(D.south east)+(0,-\baselineskip)$) node [midway, fill=white]  {evolution};
	\end{tikzpicture}
\end{figure}

\textcite{Gueldemann1998} traces this functional change back to the frequent use of the relevant expressions as part of the discursive background. This leads to semantic bleaching and their reinterpretation as markers of the dependency relationship itself, especially in those cases lacking any other overt signal of subordination, as in (\ref{exWhileShona2}). The assumed directionality is, of course, in line with the well-established tendency for meaning change towards the procedural (\Cref{sectionSemasiologicalChange}). Conceivably, bridging contexts are found in cases like (\ref{exWhileOsage2}). Here, the persistent\is{persistence} nature of the fleeting state is only of indirect relevance, in that it provides the bounds for what is primarily a relationship of simultaneous duration.\il{Osage}

\begin{exe}
	\ex \ili{Osage}\label{exWhileOsage2}\\
	\gll \textbf{Wakˀó} \textbf{ðáali̜} \textbf{šoo} \textbf{ðai̜še} wáščuɣe ðáali̜.\\
	woman good still 2\textsc{sg}.\textsc{cont} 2\textsc{sg}:get\_married(\textsc{f}) good\\
	\glt \lq You ought to get married \textbf{while you're} (\textbf{still}) \textbf{a pretty woman}.' (\cite[208]{QuinteroDictionary},  glosses added)
\end{exe}

In the same  vein, where the sample data includes material from narrative discourse, there is a recurrent pattern \lq \textit{p} … when still \textit{p}, \textit{q}\rq{}.\footnote{As observed before me, for instance, by \textcite[344]{Kieviet2017} for \ili{Rapanui} \textit{nō} and by \textcite[575, 619]{Barclay2008} for Western Dani\il{Dani, Western} \textit{awo}.}\il{Rapanui|(} Example (\ref{exWhileRapaNui}) is an illustration. Arguably, in such cases the main contribution of the \textsc{still} expression lies in fostering textual\is{textuality} cohesion, which could open the doors to a functional reanalysis.\is{proceduralisation}

\begin{exe}
	\ex Rapanui\label{exWhileRapaNui}\\
	Context: Two people have fled from the rain and are sitting in a cave.\\
	\gll I nonoho era, he papaŋahaʼa ʼi te haʼuru. \textbf{E} \textbf{haʼuru} \textbf{nō} \textbf{ʼā}, he tuʼu atu hoko rua hakaʼou nuʼu mai te puhi iŋa mo te evinio.\\
	\textsc{pfv} stay.\textsc{pl} \textsc{dist} \textsc{neutral} heavy.\textsc{pl} at \textsc{art} sleep \textsc{ipfv} sleep still \textsc{cont} \textsc{neutral} arrive away \textsc{num}:human two again people from \textsc{art} fish\_at\_night \textsc{nmlz} for \textsc{art} lent\\
\glt \lq While they stayed there, they fell asleep. \textbf{While they were} [\textbf{still}] \textbf{sleeping}, two other people arrived, who had been fishing at night for Lent.ʼ \parencite[588–589]{Kieviet2017}
\end{exe}\il{Rapanui|)}

\subsubsubsection{A closer look: The sample expressions}\il{Maricopa|(} 
I now return to a closer examination of the relevant sample expressions. To begin with, Maricopa \mbox{-\textit{haay}} has the simultaneity function in collocation with inessive \mbox{-\textit{ly}}. This is illustrated in (\ref{exWhileMaricopa1}). It is noteworthy that the same construction yields an ordering of \isi{precedence} when it is paired with clausemate negation;\is{negation} see (\ref{exWhileMaricopa2}). This doubtlessly goes back to \lq while still \mbox{\neg\textit{p}}, \textit{q} \equiv{ }while not yet \textit{p}, \textit{q}\rq{ }(\Cref{sectionBefore}). In other words, there is a clear trace of an original phasal polarity function, and \textcite{Gueldemann1998} observes a similar case in the Bantu language Bemba.\il{Bemba}\largerpage

\begin{exe}
	\ex \label{exWhileMaricopa}
	\begin{xlist}
	\exi{}Maricopa
	\ex\label{exWhileMaricopa1}
	\gll \textbf{M}-\textbf{nak}-\textbf{k} \textbf{m}\textbf{-uuvaa}-\textbf{haay}-\textbf{ly} dany nym-k-ev-k.\\
	2-sit-\textsc{ss} 2-\textsc{loc}.\textsc{cop}-still-in \textsc{dem} \textsc{dem}.\textsc{assoc}-\textsc{imp}-work-\textsc{rl}\\
	\glt \lq \textbf{While you are sitting there}, work on this.'
		\ex\label{exWhileMaricopa2}\is{precedence}
		\gll \textbf{Aly}-\textbf{m}-\textbf{yem}-\textbf{m}-\textbf{haay}-\textbf{ly} ny-yuu-ksh.\\
		\textsc{neg}-2-go-\textsc{neg}-still-in 1>2-see-\textsc{pfv}.1\\
		\glt \lq \textbf{Before you left}, I saw you.' \parencite[270–271]{Gordon1986}
	\end{xlist}
\end{exe}
\il{Maricopa|)}
\il{Osage|(}
Osage \textit{šó̜} can serve as a simultaneity marker in a pattern that, on the surface, consists of the juxtaposition of two main clauses; see (\ref{exWhileOsage1}). Crucially, adverbial subordination in Osage is consistently marked at the right edge of the clause \parencite[444]{Quintero2004}, except for these \textit{šó̜} clauses. That is, Osage displays a striking similarity to the \ili{Shona} case in (\ref{exWhileShona}) above, albeit with the structural parallels being found in \isi{syntax} rather than on the word-internal level.

\begin{exe}
	\ex Osage\label{exWhileOsage1}\\
	\gll \textbf{Á}\textbf{wa}-\textbf{hkik}-\textbf{ie} \textbf{šó̜} akxa-i wihtáeži̜ ə̜kxa má̜zeíe hí-ð-ap-e.\\
\textsc{pvb}:1\textsc{sg}.\textsc{a}-\textsc{recp}-speak still 3.\textsc{cont}-\textsc{decl} sister \textsc{subj} phone\_call  \textsc{prev}:arrive\_there-\textsc{caus}-\textsc{pl}-\textsc{decl}\\
\glt \lq I was talking [to someone] when my younger sister called (lit. \textbf{While I was having a conversation} …).' \parencite[445]{Quintero2004}
\end{exe}
\il{Osage|)}

\il{Creek|(}Creek (\textit{i})\textit{mônk} has been described as signalling simultaneous duration in certain parts of a clause chain (\cite[404]{Martin2011}; \cite[25]{MartinMcKaneMauldin2000}).\footnote{(\textit{i})\textit{mônk} is a borderline case of a \textsc{still} expression and might be considered a marker of stasis or permanence. This does, however, not affect the simultaneity function.} In the materials I consulted, all relevant attestations of affirmative polarity are compatible with a reading of a persistence.\is{persistence} For instance, in (\ref{exWhileCreek1}) the background situation modified by (\textit{i})\textit{mônk} already holds true before the occurrence of the foregrounded one. It is perhaps a question of how prominent one or the other reading is in Creek. Like the \ili{Maricopa} case discussed above, clausemate \isi{negation} yields a \isi{precedence} reading.\largerpage

\begin{exe}
	\ex Creek\label{exWhileCreek1}\\
	Context: A boy has been turned into a snake. A man is watching him.\\
	\gll A:y-ít ma óywa atĭ:ⁿk-os-a:n ił-hôył-in … ak-somêyk-in \textbf{hóyɬ-}\textbf{i:} \textbf{mônk}-\textbf{in}.\\
go.\textsc{sg}-\textsc{ss} that water up\_to.\textsc{emph}-\textsc{dim}-\textsc{ref}.\textsc{ds} go\_and-stand.\textsc{sg}:\textsc{res}-\textsc{ds} {} \textsc{loc}:water-disappear.\textsc{sg}:\textsc{pfv}-\textsc{ds} stand.\textsc{sg}-\textsc{dur} \textsc{still}-\textsc{ds}\\
\glt \lq He went to the water's edge and stood [and saw] … \textbf{and as he stood there} watching [the snake] went under again.'
\\(\cite[138]{HaasHill2014},  glosses added)
\end{exe}
\il{Creek|)}

\il{Nahuatl, Classical|(}\is{causal|(}Lastly, with Classical Nahuatl, illustrated in (\ref{exSimultaneousNahuatl1}) above, the development has gone one step further and has given rise to the signalling of causal relations; see (\ref{exSimultaneousNahuatl2}). As \textcite[1269]{Launey1986} points out, this is clearly a case of the well-known extension from time to causality (e.g. \cite[425]{KutevaEtAl2019}).

\begin{exe}
	\ex Classical Nahuatl\label{exSimultaneousNahuatl2}\\
		\gll \textbf{In} \textbf{oc} \textbf{ic} \textbf{ti}-\textbf{tēpil}-\textbf{tzin} … xi-m-ìmat-cā-nemi.\\
	\textsc{det} still when \textsc{subj}.2\textsc{sg}-offspring-\textsc{hon} {} \textsc{sbjv}-\textsc{refl}-be\_wise-\textsc{lnk}-live\\
		\glt \lq Pues que eres bien nacido … viue con cordura. [\textbf{Since you are noble}, live wisely.]' (\cite[503]{Carochi1645},  glosses added)
\end{exe}\il{Nahuatl, Classical|)}\is{causal|)}

\subsubsection{Coextensiveness}\is{coextensiveness|(}
\label{sectionAsLongAs}

\subsubsubsection{Introduction}\il{Hebrew, Modern|(}\is{modality|(}
Two expressions in my sample, Modern Hebrew \textit{ʕod} and Plateau Malagasy\il{Malagasy, Plateau} \textit{mbola} (\appsref{appendixHebrewOdAsLongAs}, \ref{appendixMalagasyAsLongAs}), occur in collocations that signal simultaneous coextension, or that the matrix clause proposition holds true during the entire duration of the situation depicted in the background clause (cf. \cite[84]{Kortmann1997}). With Modern Hebrew \textit{ʕod}, this use is found in clauses introduced by the universal quantifier \textit{kol}, as shown in (\ref{exAsLongAsHebrew}). In Plateau Malagasy,\il{Malagasy, Plateau} the \lq as long as' reading occurs in clauses governed by the temporal and \isi{conditional} subordinator \textit{raha} \lq if/when\rq{}; see (\ref{exAsLongAsMalagasy}).

\begin{exe}
	\ex Modern Hebrew\label{exAsLongAsHebrew}\\
	\gll \textbf{Kol} \textbf{ʕod} (ti-hye) qayem-et ha-šiṭa ha-zot, ze yi-mašex.\\
	all still \phantom{(}3\textsc{sg}.\textsc{f}-\textsc{cop}.\textsc{fut} exist-3\textsc{sg}.\textsc{f} \textsc{def}-system(\textsc{f}) \textsc{def}-\textsc{prox}.\textsc{sg}.\textsc{f} \textsc{prox}.\textsc{sg}.\textsc{m} 3\textsc{sg}.\textsc{m}-continue.\textsc{fut}\\
	\glt \lq \textbf{As long as} this system exists, it'll go on.'
	\\(\cite[548]{Glinert1989}, glosses added)
	
	\ex Plateau Malagasy\il{Malagasy, Plateau}\label{exAsLongAsMalagasy}\\
	\gll Ma-toki-a \textbf{raha} \textbf{mbola} velona aina.\\
	\textsc{agt}.\textsc{foc}-confident-\textsc{imp} if/when still alive breath\\
	\glt \lq Tant qu'il y a de la vie, il y a de l’espoir. [\textbf{As long as} there's life, there's hope.]' (\cite[157]{MalagasyPhd}, glosses added)	
\end{exe}

\il{German|(}Note that these instances differ from ones like the German example (\ref{exAsLongAsGerman}), where the notion of coextension is due to a separate marker, \textit{solange}, and the \textsc{still} expression \textit{noch} does its usual job of relating two phases to each other.

\begin{exe}
	\ex German\label{exAsLongAsGerman}\\
	\gll Eröffne-t wurde der Tanz, \textbf{solange} \textbf{meine} \textbf{Mutter} \textbf{noch} \textbf{leb}-\textbf{te} … \\
	open-\textsc{ptcp} become.\textsc{pst}.3\textsc{sg} \textsc{def}.\textsc{sg}.\textsc{m} dance(\textsc{m}) as\_long\_as \textsc{poss}.1\textsc{sg}:\textsc{nom}.\textsc{sg}.\textsc{f} mother(\textsc{f}) still live-\textsc{pst}.3\textsc{sg}\\
	\glt \lq The dance would be opened, \textbf{for as long as my mother was still alive}…\rq{ }(\cite[621]{MetrichFaucher2009}, glosses added)
\end{exe}\il{German|)}

\subsubsubsection{Discussion} 
The semantics at play in the Hebrew and Malagasy\il{Malagasy, Plateau} cases is straightforward: at all times and, by metonymy, in all possible worlds, in which the subordinate proposition remains true, the same applies to the matrix clause one.\is{simultaneity|)}\is{coextensiveness|)}\il{Hebrew, Modern|)}\is{modality|)} 

\subsubsection{A note on precedence}\is{precedence|(}
\label{sectionBefore}
\subsubsubsection{Introduction}\is{infinitive|(}
In this subsection, I briefly discuss cases in which a \textsc{still} expression forms part of a common or default strategy to form temporal clauses of precedence (\lq before \textit{p}, \textit{q}\rq{}). Example (\ref{exBeforeRuuli}) is an example from Ruuli,\il{Ruuli} featuring \mbox{\textit{kya}-} in collocation with the copula \textit{li} and an infinitival complement. Note that I do not discuss constructions of the type \lq (it is still) a certain amount of time, then \textit{q}\rq{} here; for a discussion of these, see \Cref{sectionScalar}.

\begin{exe}
	\ex \ili{Ruuli}\label{exBeforeRuuli}\\
	\gll \textbf{Nga} \textbf{ebi}-\textbf{dima} \textbf{bi}-\textbf{kya}-\textbf{li} \textbf{kw}-\textbf{iza} tu-a-lum-isya-nga bu-sika bu-ni.\\
	\textsc{conj} \textsc{ncl}8-hoe \textsc{subj}.\textsc{ncl}8-still-\textsc{cop} \textsc{ncl}15(\textsc{inf})-come \textsc{subj}.1\textsc{pl}-\textsc{pst}-dig-\textsc{caus}-\textsc{hab} \textsc{ncl}14-small\_hoe \textsc{ncl}14-\textsc{prox}\\
	\glt \lq \textbf{Before the hoes arrived} (lit. when hoes did not yet arrive), we dug with these small hoes.' \parencite[81]{MolochievaEtAl2021}
\end{exe}

\subsubsubsection{Distribution in the sample}
The twelve expressions in my sample which have the precedence use, in one collocation or another, are listed in \Cref{tableBefore}. As can be observed, this function is attested in unrelated languages from all macro-areas minus Australia, and with expressions of different morphological or syntactic statuses.\is{syntax}

\begin{table}
\caption{\lq Before'\label{tableBefore}}
\begin{tabular}{llll}
	\lsptoprule
	Macro-Area & Language & Expression & Collocate\\\midrule
	Africa & \ili{Manda} & (\textit{a})\textit{kóna} & \textsc{inf}\\
	& Plateau Malagasy\il{Malagasy, Plateau} & \textit{mbola} & \textsc{neg} \\
	& \ili{Ruuli} & \textit{kya}- & \textsc{cop} (+\textsc{inf})\\
	& \ili{Tima} & \textit{bʌ̀ʌ̀r} & \textsc{neg}\\
	Eurasia & Northern Qiang\il{Qiang, Northern} & \textit{tɕe}- & \textsc{neg}\\
	North America & \ili{Creek} & (\textit{i})\textit{mônk}\footnote{Borderline case of a \textsc{still} expression.} & \textsc{neg}\\
	& \ili{Kalaallisut} & \textit{suli} & \textsc{neg}\\
	& \ili{Maricopa} & -\textit{haay} & \textsc{neg}\\
	Papunesia & Western Dani\il{Dani, Western}  & \textit{awo} & \textsc{neg}\\
	South America & \ili{Cavineña} & =\textit{jari} & \textsc{neg}\\
	& Huallaga-Huánuco Quechua\il{Quechua, Huallaga-Huánuco} & -\textit{raq} & \textsc{neg}\\
	& \ili{Trió} & =\textit{nkërë} & \textsc{neg}\\
	\lspbottomrule
\end{tabular}
\end{table}

\is{negation|(}\is{not yet|(}
\subsubsubsection{A closer look and discussion} 
As can be gathered from the literal translation of the \ili{Ruuli} example (\ref{exBeforeRuuli}), the precedence reading here arises compositionally, through the adverbial subordinator \textit{nga} \lq when, while\rq{ }plus the negative concept \textsc{not yet}. The latter concept is, in this case, signalled through the combination of \mbox{\textit{kya}-} with the copula and an infinitive; I discuss such uses of \textsc{still} as \textsc{not yet} without a negator in \Cref{sectionNotYet}.\is{infinitive|)} In fact, all relevant instances in the sample can be traced back to the items in question forming part of a \textsc{not yet} expression of varying shapes and forms,\footnote{A possible counterexample from outside the sample is found in \ili{Turkana} (Nilotic) \textit{ròkò}, see \textcite[360]{Dimmendaal1983}. Judging from the other examples of this expression throughout \citeauthor{Dimmendaal1983}'s grammar, I would expect \textit{ròkò} to go together with a copula if it were to serve as a marker of \textsc{still} in the relevant cases. That is, the instances in question appear to be based on a pattern in which a \textsc{still} expression serves as a marker of \textsc{not yet} in the absence of an overt complement; see \Cref{sectionNotYet} on the latter.} together with a semantically unspecific clause combining device.\footnote{That is, we are not dealing with \lq\lq{}expletive negation\rq\rq{ }\parencite{JinKoenig2021}, as in \ili{German} 
\textit{Bevor du nicht aufgeräumt hast, darfst du nicht spielen} \lq You aren't allowed to play before tidying up (lit. … before you haven't tidied up).'} Where the majority of them differ from the \ili{Ruuli} one is that they are built around clausemate negation. Examples (\ref{exBeforeNorthernQiang},\il{Qiang, Northern} \ref{exBeforeCreek}) illustrate this more transparent pattern for a nominalised temporal clause in Northern Qiang and a medial clause in \ili{Creek}.

\begin{exe}
	\ex Northern Qiang\label{exBeforeNorthernQiang}\il{Qiang, Northern}\\
	\gll Nəs qɑ \textbf{ma}-\textbf{tɕi}-\textbf{kə}-\textbf{tç}, theː qɑ səimi de-l.\\
	yesterday 1\textsc{sg} \textsc{neg}-still-go-\textsc{gen} 3\textsc{sg} 1\textsc{sg} fruit \textsc{dir}-give.\\
	\glt \lq Yesterday \textbf{before I left}, s/he gave me (a package of) fruit.'\\\parencite[241]		{LaPollaHuang2003}
	\ex \ili{Creek}\label{exBeforeCreek}\\
	\gll \textbf{Lítk}-\textbf{iko}-\textbf{ː} \textbf{mônk}-\textbf{in} is-ás.\\
	run.\textsc{sg}-\textsc{neg}-\textsc{dur} still-\textsc{ds} catch-\textsc{imp}\\
	\glt \lq Catch it \textbf{before} \textbf{it} \textbf{runs} (lit. while it has not [yet] run, catch it).\rq{}
	\\\parencite[404]{Martin2011}
\end{exe}

The use of \textsc{not yet} constructions as signals of precedence is a strategy attested beyond the confines of my sample (\cite[336–340]{OlguinMartinez2022}; \cite{VeselinovaDevos2021}) and the underlying temporal relations are straightforward. When \isi{topic time} is narrowed down to a time at which the subordinate situation does not yet hold true, it is restricted to a time before the possible manifestation of said situation; the same observation has been made before me by \textcite[271]{Gordon1986} and \textcite[388]{Weber1989}, among others. In a nutshell, what appears to be a \textsc{still}-as-\lq before\rq{ }use at first glance turns out to be an artefact of the expressions in question also being involved in marking the negative concept \textsc{not yet}.
\is{negation|)}\is{not yet|)}\is{precedence|)}\is{temporal clause|)}\is{connective|)}\is{subordination|}
\addtocontents{toc}{\protect\setcounter{tocdepth}{2}}
