\appendix
\addtocontents{toc}{\protect\setcounter{tocdepth}{1}}

\counterwithin{xnumi}{chapter} % the chngcntr way (preferred)
\counterwithin{equation}{chapter} % reset example counter every section
\setlist[itemize]{leftmargin=*, nosep}
\togglefalse{autoexewidth}
\exewidth{(W.5555)}
\nogltOffset 

\chapter{Introduction}\label{Appendix}
The following appendices contain a large part of the data that \Cref{chapter2,chapter3} are based on, and their inclusion is meant to increase transparency as well as to establish a reference resource for subsequent research on the subject matter (see \Cref{sectionAboutAppendices} for more discussion). Unlike the main body of the book, the appendices are sorted by macro-area > language > individual expressions > attested functions.

Many of the individual language sections contain short introductory remarks that list out data sources other than descriptive materials (e.g. corpora or text collections), give background information on language structure that I deemed helpful for the interpretation of the examples, and/or acknowledge the help of individual scholars who provided personal communication, provided glosses, etc. For each individual expression, the appendices start of with a \lq\lq General information\rq\rq{ }subsection, which addresses the expression's woordhood (morphosyntactic status) and, in many instances, also contains notes on formal variation, syntactic behaviour, and/or etymology. Subsequently, the expressionʼs categorisation as an exponent of phasal polarity is summarised. This discussion follows the templatic structure given in \Cref{figureTemplaticStructureAppendix}.\largerpage[3]

\begin{figure}[H]
	\caption{Templatic structure of initial overviews. List items in parenthesis are optional.}\label{figureTemplaticStructureAppendix}
\begin{PersohnBox}
\begin{description}\itemsep=.3\itemsep
		\item [References] ~
		\item [Specialisation:] the evidence that the expression conforms to my definition (\Cref{sectionDefinition}). Where the data are less than conclusive, an expression's tentative inclusion is made explicit.
		\item [Polarity sensitivity:] whether the expression also participates in signalling \textsc{not yet} and/or \textsc{no longer}
		\item [Pragmaticity:] the expression's compatibility with the neutral and unexpectedly late scenarios of \textsc{still}.
		\item [(Syntax):] notes on the expression's linear position and/or constituency.
		\item [(Further notes):] possible usage restrictions, (in)compatibilities with certain inflections, the possibility to occur in elliptical utterances, etc.
\end{description}
\end{PersohnBox}
\vskip-.25\baselineskip
\end{figure}

These background expositions are followed by (sub)sections for each of the attested functions, in order of their discussion in \Cref{chapter2,chapter3}. In each instance, bibliographic references are given and the key findings therein are summarised. Where my own analyses of the data differ from existing descriptions, expand on them, or where analyses are of the tentative kind, this is again made explicit. The overviews are furthermore accompanied with illustrations from a wide array of sources. To allow for an easier interpretation of the data, I added context information to examples from actual discourse, together with glosses and, in the vast majority of instances, morpheme-by-morpheme segmentation.
