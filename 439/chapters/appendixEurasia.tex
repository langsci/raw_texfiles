\chapter{Eurasia}
\label{appendixEurasia}
\section{English (eng, stan1293)}\il{English|(} 
	\label{appendixEnglish}
	

\subsection{still}

\subsubsection{General information}
\begin{itemize}
	\item Wordhood: free morpheme.
	\item Etymology: < Middle English\il{English, Middle} \textit{stille} \lq without motion, without sound', via \lq always, ever, etc.\rq{}
\end{itemize}


\subsubsection{As a \lq{}still\rq{ }expression}
\begin{itemize}
	\item \Citeauthor{vanderAuwera1991BeyondDuality} (\citeyear{vanderAuwera1991BeyondDuality}, \citeyear{vanderAuwera1993}, \citeyear{vanderAuwera1998}), \textcite{vanBaar1997}, \citeauthor{Ippolito2004} (\citeyear{Ippolito2004}, \citeyear{Ippolito2007}), \citeauthor{Koenig1977} (\citeyear{Koenig1977}, \citeyear[ch. 7]{Koenig1991}), \textcite{KoenigTraugott1982}, \textcite{Krifka2000}, \textcite{Lewis2019}, \textcite{Michaelis1993} and \textcite{Mittwoch1993}, among many others.
	\item Specialisation: both the requirement for an abutting runtime and the incompatibility with inalterable states have been pointed out repeatedly in the literature (e.g. \cite{Beck2020}; \cite{Ippolito2007}; \cite{Koenig1991}; \cite{Michaelis1993}).
	\item Polarity sensitivity: inner negation yields \textsc{not yet} (unexpectedly late scenario).
	\item Pragmaticity: compatible with both scenarios.
\end{itemize}

\begin{exe}
	\ex \textit{When I first came to London, Piccadilly \textbf{still} had its goat.}
	\\(Lucas, \textit{Wanderer in London}, cited in \cite[s.v. \textit{still}]{OED2022})
	\ex \textit{The night guards had already arrived, but a few clerks were \textbf{still} hanging around…} (Pratchett, \textit{Making money})
	
	\ex\label{exAppendixEnglish3}	
	\textit{Here is a summary of today’s changes as we move into Phase 3 of the routemap out of lockdown in Scotland. \textbf{Please still be cautious} and follow the guidelines to allow us all to progress through the dates and phases.} (found online)\footnote{\url{https://twitter.com/jasonleitch/status/1281218133748047882} (25 March, 2023).}
\end{exe}

\subsubsection{Uses on the fringes of \lq{}still\rq{}}
\paragraph{Scalar contexts}\label{appendixEnglishScalar}
\begin{itemize}
	\item \citeauthor{Ippolito2004} (\citeyear{Ippolito2004}, \citeyear{Ippolito2007}) and \textcite{Michaelis1993}.
	\item \textit{Still} is encountered with monotone changes along a scale. This encompasses both decreases (\ref{exAppendixEnglishScalar1}, \ref{exAppendixEnglishScalar2}) and limited increases (\ref{exAppendixEnglishScalar3}, \ref{exAppendixEnglishScalar4}). In the latter case, an additional restrictive operator, such as \textit{only} in (\ref{exAppendixEnglishScalar3}) or \textit{no more than} in (\ref{exAppendixEnglishScalar4}), is required. The exception is a \lq{}no later than\rq{ }use of bare \textit{still} (\ref{exAppendixEnglishScalar5}); \citeauthor{Ippolito2004} (\citeyear{Ippolito2004}, \citeyear{Ippolito2007})  reports that this is not acceptable to all speakers.
	\item Note the compatibility of \textit{still} plus restrictive with the anterior aspect in examples like (\ref{exAppendixEnglishScalar4}). As \textcite{Michaelis1993} points out, if we assume that such experiential uses denote a state, then scalar \textit{only} converts the latter into a transient one (with future increases as an open possibility), thereby reconciliating it with the concept of \textsc{still}.	
\end{itemize}

\begin{exe}

	\ex\label{exAppendixEnglishScalar1}
	\textit{We \textbf{still} had a few pounds in our pockets, so we very quickly decided, why not!} (Cassidy, \textit{Indifferently})
	
	\ex\label{exAppendixEnglishScalar2}
	\textit{Milwaukee well into the 1970s \textbf{still} had many homes that were not connected to city waters and sewer lines.} (Guenther, \textit{The orphan that danced at the white house})
	
	\ex\label{exAppendixEnglishScalar3}
	\textit{Harry has \textbf{still} only fed the cat.} (He has not performed greater kindness.) \parencite[203]{Michaelis1993}
		
	\ex\label{exAppendixEnglishScalar4}
\textit{Heathrow said its own resources are sufficient to cope with about 85\% of the traffic seen in 2019, which is roughly in line with current demand. About 1,300 people have been hired in the past six months and the number of security personnel is about the same as it was pre-pandemic. Airline ground handlers, by contrast, \textbf{still} have no more than 70\% of pre-Covid resources available, Holland-Kaye said.} (found online)\footnote{\url{https://www.bloomberg.com/news/articles/2022-07-26/heathrow-says-passenger-cap-to-stay-until-airlines-boost-hiring} (15 March, 2023).}
	
	\ex\label{exAppendixEnglishScalar5}
	\begin{xlist}
		\exi{A:}\textit{ Eat? It's \textbf{still} 10 am!}
		\exi{B:} \textit{So what? I deserve a good lunch!}
		\exi{A:} \textit{Lunch is for noon.}
		\exi{B:} \textit{Whatever. }(Online example, cited in \cite[2 fn2]{Ippolito2007})
	\end{xlist}

\end{exe}

\subsubsection{Broadly adverbial temporal-aspectual functions}
\paragraph{\# Prospective \lq eventually'}
\begin{itemize}
	\item \textcite[ch. 7]{Koenig1991} discusses how \textit{still} does not allow for an \lq eventually' use.
\end{itemize}

\subsubsection{Additive and related functions}
\paragraph{Additive}\label{appendixEnglishAdditive}
\begin{itemize}
	\item \textcite[145]{Koenig1991}, \textcite[198–201]{Ranger2018} and \textcite[s.v. \textit{still}]{OED2022}.
	\item This use appears to be restricted to listing environments plus additions of the same kind.
	 Thus, the OED defines it as \lq\lq [i]n addition; after the apparent ending of a series; yet\rq\rq{ }\parencite[ s.v. still]{OED2022}.
\end{itemize}

\begin{exe}
	\sloppy
	\ex \textit{Some say … others say … \textbf{still} others say …} \parencite[145]{Koenig1991}
	\ex Context: After having discussed two creative periods of the same writer.\\
	\textit{\textbf{Still} a third type of fiction appeared in the 1950s with Mrs. Miller's successful stories of Christmas at the homes of famous men.} (found online)\footnote{\url{https://www.ncpedia.org/biography/miller-helen} (22 May, 2023).}
\end{exe}

\paragraph{\#Further-to}
\begin{itemize}
	\item\textcite{Beck2020} explicitly points out that \textit{still} does not have the further-to use.
\end{itemize}

\paragraph{Comparisons of inequality}\label{appendixEnglishComparisons}
\begin{itemize}
	\item \textcite{Ippolito2007} \textcite[145]{Koenig1991}, \textcite{Lewis2019}, \textcite[198–201]{Ranger2018} and \textcite[s.v. \textit{still}]{OED2022}.
	\item This use is rather restricted vis-à-vis scalar additive \textit{even} in the same function. \textcite{Ranger2018} points out that it is primarily found in an incrementally organised discourse, ex. (\ref{exAppendixEnglishComparisons3}) being a prototypical case.
	\item According to \textcite{Lewis2019}, this function developed out of a now obsolete distributive \lq always, ever\rq{ }sense, ex. (\ref{exAppendixEnglishComparisons4}) being the earliest attestation.
	\item Syntax: forms a constituent with its focus; note how it intervenes between the article and its complement in (\ref{exAppendixEnglishComparisons1}).
\end{itemize}

\begin{exe}
	\ex \textit{A \textbf{still} greater offer came from the Dean.} \parencite[23 fn37]{Ippolito2007}\label{exAppendixEnglishComparisons1}
	\ex\textit{A university novel is a tricky thing; an Oxford novel \textbf{still} trickier.} (\textit{The Guardian}, cited in \cite[129]{Lewis2019})
		\ex\label{exAppendixEnglishComparisons3}
		 \textit{For example, we would expect to find a very high proportion of cognate words in British and American English but a much lower percentage if we compare English and German and \textbf{still} lower if we compare English and Russian.} (BCN, cited in \cite[200]{Ranger2018})
\ex Early Modern English,\il{English, Early Modern} 16\textsuperscript{th} century\label{exAppendixEnglishComparisons4}\\
	\textit{spoonefull by spoonefull: bitterer and bitterer 	\textbf{still}} \parencite[135]{Lewis2019}
\end{exe}


\subsubsection{Marginality}
\label{appendixEnglishMarginal}
\begin{itemize}
	\item \textcite{Beck2020}, \citeauthor{Koenig1977} (\citeyear{Koenig1977}, \citeyear[151–155]{Koenig1991}), \textcite{Ippolito2007}, \textcite{Michaelis1993} and \textcite[201–203]{Ranger2018}.
	\item \textit{Still} is compatible with readings of marginality.
	\item Syntax: other than phasal polarity \textit{still}, its marginality cousin cannot occur in clause-final position.
\end{itemize}
	
\begin{exe}
	\ex \textit{I can \textbf{still} beat Paul. Peter is too good for me.} \parencite[185]{Koenig1977}
	\ex Context: Commenting on a murder sentence.\\
	\textit{In a very real sense, they} [\textit{culprits}] \textit{got what they deserved even though friends of the victims might argue they \textbf{still} got off lightly.} (found online)\footnote{\url{https://lcnme.com/opinion/letters/a-palatable-sadness/} (14 February, 2023).}
	\ex \textit{I think South East Asia \textbf{still} counts as Asia.} (\textit{The Guardian}, cited in \cite[129]{Lewis2019}.)
\end{exe}

\subsubsection{Broadly modal and interactional functions}
\paragraph{Concessive apodoses}
\label{appendixEnglishConcessiveConsequent}
\begin{itemize}
	\item \textcite{Bell2010}, \textcite{Ippolito2007}. \textcite{KoenigTraugott1982}, \textcite{Lewis2019},  \textcite{Michaelis1993}, \textcite[s.v. still]{OED2022} and \textcite{Ranger2018}, among others.
	\item  It has repeatedly been pointed out that, broadly speaking, concessive \textit{still} underlines continued discourse cohesiveness (vis-à-vis concessive markers like \textit{yet} or \textit{nevertheless}).
	\item According to \textcite{KoenigTraugott1982} and  \textcite[s.v. still]{OED2022} the concessive function arose several centuries later than the phasal polarity function.
	\item Syntax: unlike phasal polarity \textit{still}, concessive \textit{still} can occupy the clause-initial position (\ref{exAppendixEnglishConcessive3}), where it is prosodically detached; \textcite{Ranger2018} terms this \lq\lq conclusive \textit{still}\rq\rq.
\end{itemize}

\begin{exe}
	\ex \textit{We told Bill not to come, but he \textbf{still} showed up. }\parencite[193]{Michaelis1993}.
	\ex\textit{Even though he studied all night, Larry \textbf{still} failed the test.}\rq{ }\parencite[209]{Michaelis1993}.
	\ex
	\label{exAppendixEnglishConcessive3}
	… \textit{It is always the dregs of the population who show their patriotism by this sort of behaviour. \textbf{Still}, it is refreshing to see someone taking some sort of action. }\parencite[139]{Lewis2019}
\end{exe}
	
\paragraph{Concessive-evaluative \textit{but still}}\label{exAppendixEnglishButStill}
\begin{itemize}
	\item \textcite{Lewis2019} and \textcite[215–218]{Ranger2018}.
	\item Form: in the collocation \textit{but still}.
	\item Syntax: at the right periphery.
	\item In this use, \textit{but still} serves as a comment on an unfavourable situation just expressed, signalling that its effect can be disregarded.
\end{itemize}	

\begin{exe}
	\ex \textit{I don't know what they've done to it to make it spread \textbf{but still}.}
	\\(BNC, cited in \cite[129]{Lewis2019})
\end{exe}
	
\paragraph{Concessive interjection}
\label{appendixEnglishConcessiveInterjection}
\begin{itemize}
	\item \textcite{Lewis2019} and \textcite[218–220]{Ranger2018}.
	\item Form: as a stand-alone fragment/holophrase.
	\item Through this use, the speaker expresses their evaluation of a previously mentioned unfavourable situation as less serious than it might appear at first glance.
	\item As \textcite{Lewis2019} shows, this use is a recent development and	is rooted in the truncation of a specific discourse pattern involving concessive \textit{still} (\appref{appendixEnglishConcessiveConsequent}) in initial position. In this pattern, illustrated in (\ref{exAppendixEnglishConcessive3}) above,
	\begin{quote}
	an undesirable, negatively-evaluated event is conceded and a less adverse mitigating event is then put forward, often containing an explicit speaker stance. Initial-position concessive still constructions become correlated with positive affect on the part of the speaker. \parencite[138]{Lewis2019}
	\end{quote}	
\end{itemize}

\begin{exe}

	\ex 
	\begin{xlist}
	\exi{A:}\textit{ D'ya think she'll go back to work Kevin?}
	\exi{B:} \textit{No .. I don't think she will to be honest with you.}
	\exi{A:} <sigh>
	\exi{B:}\textit{ \textbf{Still}.}
	\exi{A:}\textit{ I don't blame her.} (BNC KBC, cited in \cite[129]{Lewis2019})
	\end{xlist}
	
	\ex
	\begin{xlist}
		\exi{A:} \textit{What a shame they've missed their walk together.}
		\exi{B:} \textit{Yes … \textbf{still} …. you just going back now are you?}
		\\(BNC, cited in \cite[129]{Lewis2019})
	\end{xlist}
\end{exe}
\il{English|)} 
 
\section{French (fra, stan1290)}\il{French|(}
\label{appendixFrench}

\subsection{Introductory remarks}
I am indebted to Maj-Britt Hansen for discussing French data with me, and to Guillaume Jacques for providing additional examples.

\subsection{encore}
\label{appendixFrenchEncore}

\subsubsection{General information}
\begin{itemize}
	\item Wordhood: free morpheme.
	\item Etymology: from a Latin adverbial \lq until now, thus far.'
\end{itemize}


\subsubsection{As a \lq{}still\rq{ }expression}
\begin{itemize}
	\item \Citeauthor{vanderAuwera1991BeyondDuality} (\citeyear{vanderAuwera1991BeyondDuality}, \citeyear{vanderAuwera1998}), \textcite{Borillo1984},
	\citeauthor{Fuchs1988} (\citeyear{Fuchs1988}, \citeyear{Fuchs1993}), \citeauthor{MosegaardHansen2002} (\citeyear{MosegaardHansen2002}; \citeyear[135, 144–148]{MosegaardHansen2008}), \textcite{HoepelmanRohrer1980}, \textcite{Koenig1977}, \textcite{Martin1980}, \citeauthor{Muller1975} (\citeyear{Muller1975}, \citeyear{Muller1991}), \textcite{Nef1981} and 
	\citeauthor{VictorriFuchs1992} (\citeyear{VictorriFuchs1992}, \citeyear{VictorriFuchs1996}), among others.
	\item Specialisation: the abundant descriptions, especially \textcite{vanderAuwera1998}, \citeauthor{Muller1975} (\citeyear{Muller1975}, \citeyear{Muller1991}) and \textcite{MosegaardHansen2008}, are in line with my definition. \citeauthor{MosegaardHansen2002} (\citeyear{MosegaardHansen2002}, \citeyear[145]{MosegaardHansen2008}), \textcite{HoepelmanRohrer1980}, \textcite{Martin1980} and \citeauthor{Muller1975} (\citeyear{Muller1975}, \citeyear{Muller1991}) point out the incompatibility with inalterable states. It is also repeatedly stated that \textit{encore} much more strongly suggests a future discontinuation (cf. ex. \ref{exAppendixFrenchEncore1}) than \textit{toujours} (\cite{Fuchs1988}; \cite[149–150]{MosegaardHansen2008}; \cite{Muller1991}; among others).
	\item Pragmaticity: compatible with both scenarios; however, for the unexpectedly late scenario, \textit{toujours} is typically used.
	\item Polarity sensitivity: inner negation yields \textsc{not yet}.
\end{itemize}
	
\begin{exe}
	\ex Context: From a French translation of \textit{Snow White}. Snow White has grown up and is turning into a rival to the queen's beauty. When the queen asks her mirror who the most beautiful of all women is, it replies:\label{exAppendixFrenchEncore1}\\
	\gll Reine, tu es \textbf{encore} la plus belle.\\
	queen 2\textsc{sg} \textsc{cop}.2\textsc{sg} still \textsc{def}.\textsc{sg}.\textsc{f} most beautiful.\textsc{f}\\
	\glt \lq My queen, you are [still] as yet the fairest of them all.' (\cite[149–150]{MosegaardHansen2008}, glosses added)
	\ex
	\gll Mon frère est marié, ma sœur est divorcée et moi, je suis \textbf{encore} célibataire.\\
	\textsc{poss}.1\textsc{sg}:\textsc{sg}.\textsc{m} brother(\textsc{m}) \textsc{cop}.3\textsc{sg} marry.\textsc{ptcp}.\textsc{m} \textsc{poss}.1\textsc{sg}:\textsc{sg}.\textsc{f} sister(\textsc{f}) \textsc{cop}.3\textsc{sg} divorce.\textsc{ptcp}.\textsc{f} and 1\textsc{sg} 1\textsc{sg} \textsc{cop}.1\textsc{sg} still unmarried\\
	\glt \lq My brother's married, my sister's divorced, and I'm still single.'
	\\(\cite[162]{LangPerez2004}, glosses added)
	\ex 
	\gll Si en 1977 il était \textbf{encore} ministre, c'-est qu'-il l'-était déjà avant 1977.\\
	if in 1977 3\textsc{sg}.\textsc{m} \textsc{cop}.\textsc{pst}.\textsc{ipfv}.3\textsc{sg} still ministre \textsc{prox}.\textsc{sg}.\textsc{m}-\textsc{cop}.3\textsc{sg} \textsc{comp}-3\textsc{sg}.\textsc{m} 3\textsc{sg}.\textsc{acc}.\textsc{m}-\textsc{cop}.\textsc{pst}.\textsc{ipfv}.3\textsc{sg} already before 1977\\
	\glt \lq If in 1977 he was still ministre, then he also was before 1977.'
	\\(\cite[167]{Martin1980}, glosses added)
\end{exe}

\subsubsection{Uses related to other phasal concepts}
\paragraph{Interrogative \lq yet\rq}
\label{appendixFrenchEncoreInterrogativeYet}
\begin{itemize}
	\item \Textcite{vanderAuwera1998}, \textcite[144]{MosegaardHansen2008}, \textcite{Martin1980}, \textcite{Muller1991} and \textcite{Vaelikangas1982}.
		\item \textcite[144]{MosegaardHansen2008} points out that in Old French,\il{French, Old} \textit{encore} as interrogative \lq yet\rq{ }is found in direct questions (\ref{exAppendixFrenchEncoreInterrogativeOld}). As she indicates, this is a relict of the item's etymological meaning \lq thus far.'
		\item In the present-day language, this use is marginal and is limited to indirect questions, as in (\ref{exAppendixFrenchEncoreInterrogativeYet1}, \ref{exAppendixFrenchEncoreInterrogativeYet2}). Ex. (\ref{exAppendixFrenchEncoreInterrogativeSubordinate}) illustrates \textsc{not yet} plus negative raising.
\end{itemize}

\begin{exe}
	\ex\label{exAppendixFrenchEncoreInterrogativeYet1}
	Context: Speaking of a house that is only heated intermittently.\\
	\gll Je me demande s'-il fait \textbf{encore} assez chaud dans la maison.\\
	1\textsc{sg} \textsc{refl}.1\textsc{sg} ask.1\textsc{sg} if-3\textsc{sg}.\textsc{m} make.3\textsc{sg} still sufficient heat in \textsc{def}.\textsc{sg}.\textsc{f} house(\textsc{f})\\
	\glt \lq I wonder whether the house is sufficiently warm yet.'
	\\(\cite[178]{Martin1980}, glosses added)

	\ex\label{exAppendixFrenchEncoreInterrogativeYet2}
	\gll Je ne sais pas si elle est \textbf{encore} sortie \textup{(=}déjà sortie\textup{)}.\\
	1\textsc{sg} \textsc{neg} know.1\textsc{sg} \textsc{neg} if 3\textsc{sg}.\textsc{f} \textsc{cop}.3\textsc{sg} still leave.\textsc{ptcp}.\textsc{f} \phantom{(=}already leave.\textsc{ptcp}.\textsc{f}\\
	\glt \lq I don't know if she has left yet (=already left).' (\cite[224]{Muller1991}, glosses added)

	\ex\label{exAppendixFrenchEncoreInterrogativeOld}
	Old French,\il{French, Old} mid-12\textsuperscript{th} century\\
	\gll Est, va, \textbf{encore} toz mes charroiz entr-ez?\\
	\textsc{cop}.3\textsc{sg} \textsc{dm} still all.\textsc{pl} \textsc{poss}.1\textsc{sg}:\textsc{pl} car.\textsc{pl} enter-\textsc{ptcp}.\textsc{pl}\\
	\glt \lq Say, have all my carts entered yet?\rq{ }(\cite[144]{MosegaardHansen2008}, glosses added)

	\ex\label{exAppendixFrenchEncoreInterrogativeSubordinate}
	\gll Mais je ne vois pas qu-’il ait \textbf{encore} fait son choix d’-une façon positive et irrevocable.\\
	but 1\textsc{sg} \textsc{neg} see.1\textsc{sg} \textsc{neg} \textsc{comp}-3\textsc{sg}.\textsc{m} have.\textsc{sbjv}.3\textsc{sg} still make.\textsc{ptcp} \textsc{poss}.3\textsc{sg}:\textsc{sg}.\textsc{m} choice(\textsc{m}) of-\textsc{indef}.\textsc{sg}.\textsc{f} manner(\textsc{f}) positive.\textsc{f} and irrecovable.\textsc{f}\\
	\glt \lq But I don’t see that he has yet made his choice in a positive and irrevocable way.' (\cite[169]{Martin1980}, glosses added)

\end{exe}

\subsubsection{Uses on the fringes of \lq{}still\rq{}}

\paragraph{Scalar contexts}
\label{appendixFrenchEncoreScalar}
\begin{itemize}
	\item \textcite{Fuchs1988}, \textcite[106–108]{MosegaardHansen2008}, \textcite{Vaelikangas1982} and \textcite[78]{VictorriFuchs1996}.
	\item Examples (\ref{exAppendixFrenchEncoreScalar1}, \ref{exAppendixFrenchEncoreScalar2}) illustrate decreases. For a limited increase French resorts to the collocation with restrictive \textit{ne … que} \lq only\rq{}, as in (\ref{exAppendixFrenchEncoreScalar3}, \ref{exAppendixFrenchEncoreScalar4}); note that the latter is not the internal negation of \textit{encore} (which would be \textit{ne … pas encore}).
\end{itemize}
\largerpage
\begin{exe}
	\ex \label{exAppendixFrenchEncoreScalar1}
	\gll Il reste \textbf{encore} dix million-s de billet-s disponible-s pour cet été.\\
	3\textsc{sg}.\textsc{m} remain.3\textsc{sg} still ten million-\textsc{pl} of ticket-\textsc{pl} available-\textsc{pl} for \textsc{prox}.\textsc{sg}.\textsc{m} summer(\textsc{m})\\
	\glt \lq There are still ten million tickets available for this summer.\rq{}
	\\(found online, glosses added)\footnote{\url{https://www.lechorepublicain.fr/dreux-28100/actualites/il-reste-encore-cinq-places-au-centre-de-loisirs_1152634/} (23 February, 2023).}

	\ex \label{exAppendixFrenchEncoreScalar2}
	\gll Au cours de l’-interview, Lowenthal révèle qu-’il a \textbf{encore} un peu de travail à faire sur la suite.\\
	at:\textsc{def} run of \textsc{def}-interview L. reveal.3\textsc{sg} \textsc{comp}-3\textsc{sg}.\textsc{m} have.3\textsc{sg} still \textsc{indef}.\textsc{sg}.\textsc{m} little(\textsc{m}) of work to do.\textsc{inf} on \textsc{def}.\textsc{sg}.\textsc{f} sequel(\textsc{f})\\
	\glt \lq During the interview, Lowenthal revealed that he still has a little work to do on the sequel.\rq{ }(found online, glosses added)\footnote{\url{https://testeurjoe.fr/yuri-lowenthal-a-encore-un-peu-de-travail-a-faire-sur-spider-man-2/} (25 January, 2023).}
	
	\ex \label{exAppendixFrenchEncoreScalar3}
	\gll Pierre n'-a \textbf{encore} que cinq livres.\\
	P. \textsc{neg}-have.3\textsc{sg} still than five book.\textsc{pl}\\
	\glt \lq Pierre still only has five books.\rq{ }(\cite[108]{MosegaardHansen2008}, glosses added)
	
	\ex \label{exAppendixFrenchEncoreScalar4}
	\gll Un an après sa promulgation, la loi pour la croissance n-'a \textbf{encore} que des effets limités.\\
	\textsc{indef}.\textsc{sg}.\textsc{m} year(\textsc{m}) after \textsc{poss}.3\textsc{sg}.\textsc{f}:\textsc{sg} promulgation \textsc{def}.\textsc{sg}.\textsc{f} law(\textsc{f}) for \textsc{def}.\textsc{sg}.\textsc{f} growth(\textsc{f}) \textsc{neg}-have.3\textsc{sg} still than \textsc{indef}.\textsc{pl} effect(\textsc{m}).\textsc{pl} limited.\textsc{pl}.\textsc{m}\\
	\glt \lq A year after its promulgation the law for [economic] growth still shows only limited results.\rq{ }(found online, glosses added)\footnote{\url{https://www.lesechos.fr/2016/08/la-loi-macron-un-symbole-plus-quune-revolution-234679} (02 March, 2023).}
\end{exe}


\subsubsection{Broadly adverbial temporal-aspectual functions}
\largerpage
\paragraph{Iterative}
\label{appendixFrenchEncoreIterative}
\begin{itemize}
	\item  \textcite{Borillo1984}, \citeauthor{Fuchs1988} (\citeyear{Fuchs1988}, \citeyear{Fuchs1993}), \citeauthor{MosegaardHansen2002} (\citeyear{MosegaardHansen2002}; \citeyear[155–156]{MosegaardHansen2008}),  \textcite{HoepelmanRohrer1980}, \textcite{Martin1980}, \citeauthor{Muller1975} (\citeyear{Muller1975}, \citeyear{Muller1991}),  \textcite{Nef1981},  \textcite{TovenaDonazzan2008} and \citeauthor{VictorriFuchs1992} (\citeyear{VictorriFuchs1992}, \citeyear{VictorriFuchs1996}), among others.
 	\item There are no restrictions regarding aspectual operators and/or actional classes; \textcite[155 fn19]{MosegaardHansen2008}, however, points out that with atelic predicates (plus those aspectual inflections that are compatible with \textsc{still}) a phasal polarity reading is preferred, unless context suggests otherwise.
	\item Iterative \textit{encore} is also attested in the absence of an overt predicative (\ref{exAppendixFrenchEncore3}). 
	\item \textcite[155–156]{MosegaardHansen2008} shows that the iterative function is attested from the earliest attestations on, albeit infrequently and rarely unambiguously. As she discusses, a case like (\ref{exFrenchEncoreIterativeOld}) constitutes a likely bridging context from \textsc{still} to iteration: this example can be interpreted as either depicting a persistent series of situations, or as evoking an (expected) repetition.
\end{itemize}
\begin{exe}
	\ex
	\gll Marie a \textbf{encore} déclamé le poème.\\
	M. have.3\textsc{sg} still declaim.\textsc{ptcp} \textsc{def}.\textsc{sg}.\textsc{m} poem(\textsc{m})\\
	\glt \lq Marie declaimed the poem again.\rq{ }(\cite[9]{TovenaDonazzan2008}, glosses added)

	\ex
	\gll Tiens, Anne a \textbf{encore} les cheveux roux. Hier, elle était blonde.\\
	\textsc{interj} A. have.3\textsc{sg} still \textsc{def}.\textsc{pl} hair(\textsc{m}).\textsc{pl} ginger.\textsc{pl}.\textsc{m} yesterday 3\textsc{sg}.\textsc{f} \textsc{cop}.\textsc{pst}.\textsc{ipfv}.3\textsc{sg} blond.\textsc{sg}.\textsc{f}\\
	\glt \lq Oh, Anne is a redhead again. Yesterday, she was blonde.' (\cite[155]{MosegaardHansen2008}, glosses added)
	
	\ex\label{exAppendixFrenchEncore3}
	\gll Merci, \textbf{encore}.\\
	thanks still\\
	\glt \lq Thanks, again.' \parencite[226]{LangPerez2004}
	
	\ex Old French,\il{French, Old} ca. 1176–1184\label{exFrenchEncoreIterativeOld}\\
	\gll Dius m'-a bien aidié {dusc'a ore}, // Si me puet bien aidier \textbf{encore}.\\
	God 1\textsc{sg}.\textsc{acc}-have.3\textsc{sg} well help.\textsc{ptcp} {until now} {} then 1\textsc{sg}.\textsc{acc} can.3\textsc{sg} well help.\textsc{inf} still\\
	\glt \lq God has helped me so far, so he may well help me still/again.'
	\\(\textit{Eracle}, cited in \cite[156]{MosegaardHansen2008}, glosses added)
\end{exe}
\paragraph{Iterative via increment}
\label{appendixFrenchIterativeIncrement}
\begin{itemize}
	\item \textcite[s.v. \textit{encore}]{Dicctionnaire}, \textcite{Borillo1984}, \textcite[535]{BatchelorChebliSaadi2011}, \citeauthor{Koenig1977} (\citeyear{Koenig1977}, \citeyear[145]{Koenig1991}), \textcite{Muller1991}, \textcite{Nef1981} and \textcite[487–488]{Price2013}.
	\item Form: in conjunction with an NP consisting of a numeral and the event quantifier \textit{fois} \lq time(s)'.
\end{itemize}
\begin{exe}
	\ex \gll La petite Canadienne a battu le record \textbf{encore} \textbf{une} \textbf{fois}.\\
	\textsc{def}.\textsc{sg}.\textsc{f} small.\textsc{f} Canadian.\textsc{f} have.3\textsc{sg} beat.\textsc{ptcp} \textsc{def}.\textsc{sg}.\textsc{m} record(\textsc{m}) still one.\textsc{f} time(\textsc{f})\\
	\glt \lq The little Canadian girl/woman has broken the record again.' (\cite[535]{BatchelorChebliSaadi2011}, glosses added)

	\ex \gll Marie est \textbf{encore} alléé \textbf{trois} \textbf{fois} au cinema.\\
	M. \textsc{cop}.3\textsc{sg} still go.\textsc{ptcp}.\textsc{f} three time.\textsc{pl} to:\textsc{def}.\textsc{sg}.\textsc{m} cinema(\textsc{m})\\
	\glt \lq Marie has gone to the cinema three more times.\rq{ }(\cite[43]{Donazzan2008}, glosses added)
\end{exe}

\paragraph{Prospective \lq eventually\rq{}}
\label{appendixFrenchEncoreProspective}
	\begin{itemize}
	\item \textcite[146–147]{MosegaardHansen2008}, \textcite{HoepelmanRohrer1980}, \textcite[142]{Koenig1991} and \textcite{Vaelikangas1982}.
	\item \textcite[146–147]{MosegaardHansen2008} points out that this use is attested from relatively early on, roughly one century after the first clear-cut cases of \textit{encore} as a phasal polarity marker (\ref{exAppendixFrenchEncoreProspectiveOld1}).
Ambiguous instances involving modals are attested even earlier (\ref{exAppendixFrenchEncoreProspectiveOld2}).
	\item \textcite{Vaelikangas1982} reports that (\ref{exAppendixFrenchEncoreProspective}) was not accepted by his language assistants, so there might be some variation.
	\end{itemize}
	
\begin{exe}
	\ex Context: At a sports match; the opposite team is in the lead.\\
	\gll On va \textbf{encore} gagner.\\
	\textsc{impr}/1\textsc{pl} go.3\textsc{sg} still win.\textsc{inf}\\
	\glt \lq We'll win yet!' (\cite[142]{Koenig1991})

	\ex\label{exAppendixFrenchEncoreProspective}
	\gll Jean viendra \textbf{encore}.\\
	J. come.\textsc{fut}.3\textsc{sg} still\\
	\glt \lq Jean will come yet.' (\cite[128]{HoepelmanRohrer1980}, glosses added)
	
	\ex Old French,\il{French, Old} ca. 1150\label{exAppendixFrenchEncoreProspectiveOld1}\\
	\gll La vielle dist: \lq\lq Or entend-ez // et que ce est si devinn-ez; // \textbf{encor} vous fera touz iriez"\\
	\textsc{def}.\textsc{sg}.\textsc{f} old.\textsc{f} say.\textsc{pst}.\textsc{pfv}.3\textsc{sg} \phantom{\lq\lq}now listen-2\textsc{pl} {} and \textsc{comp} \textsc{prox}.\textsc{sg}.\textsc{m} \textsc{cop}.3\textsc{sg} then guess-2.\textsc{pl} {} still 2\textsc{pl}.\textsc{acc} make.\textsc{fut}.3\textsc{sg} all.\textsc{pl} angry.\textsc{pl}\\
	\glt \lq The old woman says \lq\lq Now listen // and then guess what it is // it’ll drive you all mad yet!" (\textit{Le roman de Thèbes}, cited in \cite[146]{MosegaardHansen2008}, glosses added)

		\ex Old French,\il{French, Old} ca. 1080\label{exAppendixFrenchEncoreProspectiveOld2}\\
		\lq\lq …Enz en voz bainz que Deus pur vos i fist, // La vuldrat il chrestïens devenir.\rq\rq\\
		\lq {\lq\lq}… There, in the baths that God made for you, // He will become a Christian.{\rq\rq}\rq
		\sn
		\gll Charles respunt: \lq\lq{}\textbf{Uncore} purrat guarir.\rq\rq\\
		C. reply.3\textsc{sg} \phantom{\lq\lq}still can.\textsc{fut}.3\textsc{sg} save.\textsc{inf}\\
		\glt \lq Charles replies, \lq\lq He may yet/still be saved.{\rq\rq}\rq{ }(\textit{La chanson de Roland}, cited in \cite[146 fn 12]{MosegaardHansen2008}, glosses added)
\end{exe}



\subsubsection{Temporal connectives and frame setters}
\paragraph{Time-scalar additive (\lq as late as\rq)}
\label{appendixFrenchEncoreTimeScalar}
\begin{itemize}
	\item \textcite{Borillo1984}, \textcite{Fuchs1993}, \textcite[161–162]{MosegaardHansen2008} , \textcite{Nef1981} and \citeauthor{VictorriFuchs1992} (\citeyear{VictorriFuchs1992}, \citeyear{VictorriFuchs1996}).
	\item Time-scalar additive \textit{encore} invariably relates the focus to earlier alternatives on a scale of time proper (\lq as late as').
	\item As \textcite[162]{MosegaardHansen2008} points out, in cases where the adverbial denotes a point in time and the viewpoint is imperfective, such as in (\ref{exAppendixFrenchEncoreTimeScalar1}), the time-scalar additive use and the phasal polarity function often overlap.
	\item The earliest attestation of time-scalar additive \textit{encore} shows up approx. 150 years later than phasal polarity \textit{encore}.
\item Ex. (\ref{exAppendixFrenchEncoreTimeScalar5}) illustrates \lq as late as\rq{ }with a perfective viewpoint
		\item Syntax: \textit{encore} plus temporal expression can form a single constituent (see \cite{Fuchs1993} for an extensive discussion). Apart from bare \textsc{adv} \textit{encore} / \textit{encore} \textsc{adv}, this function is found intervening between presentative \textit{voilá} and its argument (\ref{exAppendixFrenchEncoreTimeScalar4}).
\end{itemize}

\begin{exe}
	\ex\label{exAppendixFrenchEncoreTimeScalar1}
	 \gll \textbf{Encore} en 2003 / En 2003 \textbf{encore}, Vincent aurait voté oui à l'-europe.\\
	still in 2003 {} in 2003 still, V. have.\textsc{cond}3.\textsc{sg} vote.\textsc{ptcp} yes to \textsc{def}.\textsc{sg}-Europe\\
	\glt \lq As late as 2003, Vincent would have voted yes to Europe.\rq{ }(\cite[160–161]{MosegaardHansen2008}, glosses added)
	
	\ex\label{exAppendixFrenchEncoreTimeScalar2}
	\gll \textbf{Encore} à midi, je parlais de ta mère avec Émile.\\
	still at noon 1\textsc{sg} speak.\textsc{pst}.\textsc{ipfv}.1\textsc{sg} of \textsc{poss}.2\textsc{sg}:\textsc{sg}.\textsc{f} mother(\textsc{f}) with E.\\
	\glt \lq As recently as this noon, I was talking about your mother with Émile.' (M. Aymé, \textit{Uranus}, cited in \cite[259]{Fuchs1993}, glosses added)
	

	\ex\label{exAppendixFrenchEncoreTimeScalar4}
	 \gll Voila \textbf{encore} quelques mois, l'-idée eût paru aussi provocatrice qu’-interdire la cigarette.\\
	\textsc{prstt} still a\_few month.\textsc{pl} \textsc{def}.\textsc{sg}-idea(\textsc{f}) have.\textsc{pst}.\textsc{sbjv}.3\textsc{sg} appear.\textsc{ptcp} as provocative.\textsc{f} as-forbid.\textsc{inf} \textsc{def}.\textsc{sg}.\textsc{f} cigarette(\textsc{f})\\
	\glt \lq Only a few months ago the idea would have been as provocative as banning cigarettes.' (\textit{Puest-France}, cited in \cite[258]{Fuchs1993})
	
	\ex\label{exAppendixFrenchEncoreTimeScalar5}
\gll \textbf{Encore} \textbf{hier}, j'-ai vu le film {La Chute}, sur la mort d’-Hitler, que je n’-avais pas pu voir quand il est sorti.\\
	still yesterday 1\textsc{sg}-have.1\textsc{sg} see.\textsc{ptcp} \textsc{def}.\textsc{sg}.\textsc{m} movie(\textsc{m}) Downfall about \textsc{def}.\textsc{sg}.\textsc{f} death(\textsc{f}) of-H. \textsc{subord} 1\textsc{sg} \textsc{neg}-have.\textsc{pst}.\textsc{ipfv}.1\textsc{sg} \textsc{neg} can.\textsc{ptcp} see.\textsc{inf} when 3\textsc{sg}.\textsc{m} \textsc{cop}.3\textsc{sg} leave.\textsc{ptcp}.\textsc{sg}.\textsc{m}\\
	\glt \lq Just yesterday, I watched the movie \textit{Downfall}, about the death of Hitler, which I hadn't been able to watch when it came out.\rq{ }(found online, glosses added)\footnote{\url{https://www.saintbrice95.fr/a-la-une/actualites/actualites-2019/lacteur-michel-bouquet-sous-la-direction-du-saint-bricien-ulysse-di-gregorio-1113.html} (02 May, 2022).}
\end{exe}

\subsubsection{Marginality}\label{appendixFrenchEncoreMarginal}
\begin{itemize}
	\item \textcite{Deloor2012}, \citeauthor{MosegaardHansen2002} (\citeyear{MosegaardHansen2002}, \citeyear[175–181]{MosegaardHansen2008}), \textcite{Koenig1977}, \textcite{Muller1991}, \textcite{Nef1981} and \citeauthor{VictorriFuchs1992} (\citeyear{VictorriFuchs1992}, \citeyear{VictorriFuchs1996}).
	\item  This includes derogatory uses in comparisons like (\ref{exAppendixFrenchEncoreMarginal3}). \textcite[175]{MosegaardHansen2008} describes (\ref{exAppendixFrenchEncoreMarginal1}, \ref{exAppendixFrenchEncoreMarginal2}) as evoking a concessive nuance, namely that things are better than they might appear. In line with this observation, she observes that marginality \textit{encore} is (near-)exclusively found with predicates that are evaluated positively.
	\item \textcite[180]{MosegaardHansen2008} and \textcite{Muller1991} point out that the marginality use is incompatible with prototypical members of a category (\#\textit{Un moineau, c’est encore un oiseau} \lq A sparrow is still a bird') and geographically central members of a territory (\#\textit{Paris, c’est encore la France} \lq Paris is still France').
	\item \textcite[173]{MosegaardHansen2008} notes that isolated (elliptical) \textit{encore} cannot have a marginality reading.
\end{itemize}

\begin{exe}
	\ex\label{exAppendixFrenchEncoreMarginal1}
	\begin{xlist}
		\exi{A:}\textit{Je suis bien embêté. Ma tante a légué la plus grande partie de sa fortune à un refuge animalier, et je n’aurai que 10.000 euros}…\\
		\lq I’m really annoyed. My aunt has left the better part of her fortune to an animal shelter, and only 10,000 euros to me…'
		\exi{B:}\gll Ben, 10.000 euros, c'-est \textbf{encore} une somme.\\
	well, 10,000 euros \textsc{prox}.\textsc{sg}.\textsc{m}-\textsc{cop}.3\textsc{sg} still \textsc{indef}.\textsc{sg}.\textsc{f} sum(\textsc{f})\\
		\glt \lq Well, 10,000 euros is still a decent sum.' (\cite[172]{MosegaardHansen2008}, glosses added)
\end{xlist}
	
	\ex\label{exAppendixFrenchEncoreMarginal2}
	\begin{xlist}
		\exi{A:}\textit{A vingt ans, Solange a eu le prix Molière; à trente ans, elle a été virée de la Comédie-Française; et à quarante ans, elle ne joue plus que dans des théâtres de province.}\\
		\lq At twenty, Solange received the Molière award; at thirty, she was fired from the Comédie-Française; and now that she’s forty, she only appears in small-town playhouses.'
		\exi{B:}\gll Enfin, vu qu-{il y a} 80\% de chômage parmi les comédiens, c’-est \textbf{encore} fabuleux!\\
		finally see.\textsc{ptcp} \textsc{comp}-\textsc{exist} 80\% of unemployment under \textsc{def}.\textsc{pl} actor.\textsc{pl} \textsc{prox}.\textsc{sg}.\textsc{m}-\textsc{cop}.3\textsc{sg} still fabulous.\textsc{m}\\
		\glt \lq Well, given that there’s 80\% unemployment among actors, that’s still fantastic!' (\cite[176]{MosegaardHansen2008}, glosses added)
	\end{xlist}
	
	\ex\label{exAppendixFrenchEncoreMarginal3}
	\gll De tous tes copains, c'-est \textbf{encore} Benjamin le plus beau. \textup{(}Mais franchement, ils sont tous plutôt moche-s. \textup{/} \textup{?}Et ils resemblement tous à des mannequin-s\textup{)}.\\
	of all.\textsc{pl}.\textsc{m} \textsc{poss}.2\textsc{sg}:\textsc{pl} friend(\textsc{m}).\textsc{pl} 3\textsc{sg}.\textsc{m}-\textsc{cop}.3\textsc{sg} still B. \textsc{def}.\textsc{sg}.\textsc{m} most pretty.\textsc{sg}.\textsc{m} \phantom{(}but frankly, 3\textsc{pl}.\textsc{m} \textsc{cop}.3\textsc{pl} all.\textsc{pl}.\textsc{m} rather ugly-\textsc{pl} {} \phantom{(}and 3\textsc{pl}.\textsc{m} resemble.3\textsc{pl} all.\textsc{pl}.\textsc{m} to of.\textsc{pl} model-\textsc{pl}\\
	\glt \lq Of all your friends, Benjamin is still the best looking. (But frankly, they are all rather ugly /? And they all lool like models.)' (\cite[37]{MosegaardHansen2008}, glosses added)
	
		
	\ex\label{exAppendixFrenchEncoreMarginal4}
	\gll Menton, c'-est \textbf{encore} la France.\\
	M. \textsc{prox}.\textsc{sg}.\textsc{m}-\textsc{cop}.3\textsc{sg} still \textsc{def}.\textsc{sg}.\textsc{f} F.\\
	\glt \lq Menton [a nondescript border town] is still in France.' (\cite[180]{MosegaardHansen2008}, glosses added)
	
	\ex\label{exAppendixFrenchEncoreMarginal5}
	\gll Un pingouin, c’-est \textbf{encore} un oiseau.\\
	\textsc{indef}.\textsc{sg}.\textsc{m} penguin(\textsc{m}) \textsc{prox}.\textsc{sg}.\textsc{m}-\textsc{cop}.3\textsc{sg} still \textsc{indef}.\textsc{sg}.\textsc{m} bird(\textsc{m})\\
	\glt \lq A penguin is still some kind of bird.\rq{ }(\cite[180]{MosegaardHansen2008}, glosses added)
	
\end{exe}
\subsubsection{Additive and related functions}

\paragraph{Additive}
\label{appendixFrenchEncoreAdditive}
\begin{itemize}
\item \textcite[s.v. \textit{encore}]{Dicctionnaire}, \textcite{Borillo1984}, \citeauthor{MosegaardHansen2002} (\citeyear{MosegaardHansen2002}; \citeyear[156–158, 162–168]{MosegaardHansen2008}), \textcite[104–105, 140–142]{Noelke1983} and \citeauthor{VictorriFuchs1992} (\citeyear{VictorriFuchs1992}, \citeyear{VictorriFuchs1996}), among others.
	\item This includes the collocations \textit{non seulement} … \textit{mais} (\textit{encore}) \lq not only … but also' and the disjunctive connective \textit{ou encore}, which marks the last element in a series of non-exclusive alternatives (\ref{exAppendixFrenchEncoreAdditiveLastInSeries}).
	\item In diachronic terms, \textcite[158]{MosegaardHansen2008} observes that \textit{encore} as additive \lq in addition\rq{ }shows up from very early on, even before iterative uses. The first attestation is given in (\ref{exAppendixFrenchEncoreAdditiveOld}). As \textcite[158]{MosegaardHansen2008} discusses, 
	\begin{quote}it is likely to be motivated directly by the cumulative element of meaning that inheres in phasal encore, such that – in spite of the telicity of the verb \textit{ocire}– [it] may be loosely understood as meaning that the Saracen … remains in a \lq killing-mode'. \end{quote}
\textit{Encore} as an additive with alternatives of the same kind \lq another\rq{}, shows up considerably later, ex. (\ref{exAppendixOldFrenchRide}) featuring a likely bridging context. Another bridge may be found in examples ambiguous between \lq another\rq{ }and an iterative reading (\ref{exAppendixOldFrenchRespite}). 
	\item As repeatedly observed in the literature, there is a strong association between additive \textit{encore} and an incrementally unfolding discourse. However, as \textcite[158]{MosegaardHansen2008} highlights, this does not equal a sequential ordering of events. Thus \textit{puis} \lq then\rq{ }in (\ref{exAppendixFrenchAdditiveTShirt}) can be understood in reference to the order of mentioning.
	\item Syntax: when additivity pertains to another entity/quantity of the same type, \textit{encore}	is usually a syntactic sister to the focus.
\end{itemize}
\begin{exe}

	\ex \label{exAppendixFrenchAdditiveTShirt}
	\gll Aline a acheté deux pull-s, une mini-jupe et un caleçon. Et puis, elle a \textbf{encore} acheté un t-shirt.\\
	A. have.3\textsc{sg} buy.\textsc{ptcp} two sweater-\textsc{pl} \textsc{indef}.\textsc{sg}.\textsc{f} mini-skirt(\textsc{f}) and \textsc{indef}.\textsc{sg}.\textsc{m} leggins(\textsc{m}) and then 3\textsc{sg}.\textsc{f} have.3\textsc{sg} still buy.\textsc{ptcp} \textsc{indef}.\textsc{sg}.\textsc{m} t-shirt(\textsc{m})\\
	\glt \lq Aline bought two sweaters, a mini-skirt, and a pair of leggings. And then, she also bought a t-shirt.\rq{ }(\cite[157]{MosegaardHansen2008}, glosses added)

	\ex\label{exAppendixFrenchEncoreAdditiveLastInSeries}
	 \gll Appelle, Pierre, Jean \textbf{ou} \textbf{encore} Paul.\\
	call.\textsc{imp} P. J. or still P.\\
	\glt \lq Call Pierre, Jean, or (even) Paul.' (\cite[40]{Borillo1984}, glosses added)
		
	\ex
	\gll Marie a lu \textbf{encore} un livre.\\
	M. have.3\textsc{sg} read.\textsc{ptcp} still \textsc{indef}.\textsc{sg}.\textsc{m} book(\textsc{m})\\
	\glt \lq Mary read one more book.' (\cite[9]{TovenaDonazzan2008}, glosses added)
	
	\ex Old French,\il{French, Old} ca. 1080\label{exAppendixFrenchEncoreAdditiveOld}\\
\textit{Aprof li ad sa bronie desclose, // El cors li met tute l’enseigne bloie, // Que mort l’abat en une halte roche. //} \\
	\lq Afterwards, he tears open his coat of mail, he drives the entire blue ensign into his body, killing him on a tall rock.'
	\exi{}\gll Sun cumpaignun Gerers ocit \textbf{uncore}, // E Berenger e guiun de Seint Antonie.\\
	\textsc{poss}.3\textsc{sg}.\textsc{m}:\textsc{sg} companion G. kill.3\textsc{sg} still {} and B. and G. of S.-A.\\
	\glt \lq He also kills his companion Gérier, and Berenger and Gui de Saint-Antoine.' (\textit{La chanson de Roland}, cited in \cite[158]{MosegaardHansen2008}, glosses added)	
	
	
		\ex Old French,\il{French, Old} 13\textsuperscript{th} century\label{exAppendixOldFrenchRide}\\
	\textit{Et se vous baés orendroit a cevauchier vers la Petite Bretaingne, ceste autre voie vous i menra tout droit}.\\
	\lq And if you wish now to ride to Brittany, this other road will take you straight there.\rq{}
	\exi{}
	\gll  Et se vous volés \textbf{encore} une piece cevauchier pour veoir les merveilleuses aventures d-u roiaume de Logres, avoec moi poés cevauchier...\\
	and if 2\textsc{pl} want.2\textsc{pl} still \textsc{indef}.\textsc{sg}.\textsc{f} bit(\textsc{f}) ride.\textsc{inf} for see.\textsc{inf} \textsc{indef}.\textsc{pl}.\textsc{f} marvellous.\textsc{pl}.\textsc{f} adventure(\textsc{f}).\textsc{pl} of-\textsc{def}.\textsc{sg}.\textsc{m} realm(\textsc{m}) of L. with 1\textsc{sg}.\textsc{obj} can.2\textsc{pl} ride.\textsc{inf}\\
	\glt \lq And if you \textbf{still} \textbf{want} \textbf{to} \textbf{ride} \textbf{a} \textbf{bit} with me/if you \textbf{want} \textbf{to} \textbf{ride} \textbf{another} \textbf{bit} with me to see the marvelous adventures of the kingdom of Logres, you can ride with me...\rq{ }(\textit{Tristan en prose}, cited in \cite[163]{MosegaardHansen2008}, glosses added)
	
	\ex Old French,\il{French, Old} early 13\textsuperscript{th} century\label{exAppendixOldFrenchRespite}\\
	Context: The barons have given the emperor a respite. Now the date has passed.\\
	\gll Et li empereres redemanda \textbf{encore} un respit, et on li donna.\\
	and \textsc{def}.\textsc{sg}.\textsc{m} emperor(\textsc{m}) ask\_again.\textsc{pst}.\textsc{pfv}.3\textsc{sg} still \textsc{indef}.\textsc{sg}.\textsc{m} respite(\textsc{m}) and \textsc{impr} 3\textsc{sg}.\textsc{dat} give.\textsc{pst}.\textsc{pfv}.3\textsc{sg}\\
	\glt \lq And the emperor \textbf{asked for a respite again}/\textbf{asked for another respite}, and it was given him.\rq{ }(de Clari, \textit{La conqueste de Constantinople}, cited in \cite[163–164]{MosegaardHansen2008}, glosses added)
\end{exe}

\paragraph{Comparisons of inequality}\label{appendixFrenchEncoreComparisons}
\begin{itemize}
	\item \textcite[s.v. \textit{encore}]{Dicctionnaire}, \textcite{Borillo1984}, \citeauthor{MosegaardHansen2002} (\citeyear{MosegaardHansen2002}, \citeyear[164–168]{MosegaardHansen2008}), \textcite{Muller1991}, \textcite[487–488]{Price2013} and \citeauthor{VictorriFuchs1992} (\citeyear{VictorriFuchs1992}, \citeyear{VictorriFuchs1996}).
	\item Comparisons of inequality in French are expressed via \textit{plus} \lq more' (except for certain suppletive comparative forms), and the standard of comparison, if overtly mentioned, is introduced by \textit{que} \lq than'. Addition of \textit{encore} yields \lq even more\rq{}.
		\item \textit{Encore} modifies not only comparisons \textit{sensu stricto}, but also combines with degree achievements  (\ref{exAppendixFrenchEncoreComparativeDegreeAchievement}) and other verbs that express a comparison of degrees, such as \textit{preferer} \lq prefer' in (\ref{exAppendixFrenchEncoreComparativeVerb}).
		\item \textcite[166–167]{MosegaardHansen2008} points out several differences between \textit{encore} in this use vis-à-vis \textit{toujours} in the same contexts. First, \textit{encore}, but not \textit{toujours}, suggests that the degree held by the comparee is far on the upper end of the relevant scale. Secondly, \textit{encore} can be used to compare the same entity across two points in time, as in (\ref{exAppendixFrenchEncoreComparativeTwoPoints}), whereas \textit{toujours} implies that an increase in degree has been observed at least once before. 
	\item Syntax: forms a constituent with the degree expression. For instance, in (\ref{exappendixFrenchEncoreComparisons1}) it occurs not in its usual post-auxiliary position, but immediately precedes \textit{mieux} \lq better.\rq{}.
\end{itemize}
\begin{exe}
	\ex \label{exappendixFrenchEncoreComparisons1}
	\gll Le premier roman de Duschnock a eu beaucoup de succès. Le deuxième s’-est vendu \textbf{encore} mieux.\\
	\textsc{def}.\textsc{sg}.\textsc{m} first.\textsc{sg}.\textsc{m} novel(\textsc{m}) of D. have.3\textsc{sg} have.\textsc{ptcp} much of success \textsc{def}.\textsc{sg}.\textsc{m} second.\textsc{m} \textsc{refl}.3-\textsc{cop}.3\textsc{sg} sell.\textsc{ptcp}.\textsc{sg}.\textsc{m} still better\\
	\glt \lq Duschnock’s first novel was a great success. The second one sold even better.’ (\cite[164]{MosegaardHansen2008}, glosses added)
	
	\ex\label{exAppendixFrenchEncoreComparativeTwoPoints}
	\gll Je n’-apprécie pas la nouvelle copine de Hugues: elle m’-a semblé \textbf{encore} plus bête hier soir que la première fois que je l’-avais vue.\\
	1\textsc{sg} \textsc{neg}-appreciate.1\textsc{sg} \textsc{neg} \textsc{def}.\textsc{sg}.\textsc{f} new.\textsc{f} girlfriend of H. 3\textsc{sg}.\textsc{f} 1\textsc{sg}.\textsc{dat}-have.3\textsc{sg} seem.\textsc{ptcp} still more stupid yesterday evening than \textsc{def}.\textsc{sg}.\textsc{f} first.\textsc{f} time \textsc{rel} 1\textsc{sg} 3\textsc{sg}.\textsc{obj}.\textsc{f}-have.\textsc{pst}.\textsc{ipfv}.1\textsc{sg} see.\textsc{ptcp}.\textsc{sg}.\textsc{f}\\
	\glt \lq I don’t like Hugues’ new girlfriend. She seemed even more stupid to me last night than the first time I met her.\rq{ }(\cite[164]{MosegaardHansen2008}, glosses added)

	\ex \label{exAppendixFrenchEncoreComparativeDegreeAchievement}
	\gll Cet accident ralentit \textbf{encore} la circulation.\\
	\textsc{prox}.\textsc{sg}.\textsc{m} accident(\textsc{m}) slow\_down.3\textsc{sg} still \textsc{def}.\textsc{sg}.\textsc{f} traffic(\textsc{f})\\
	\glt \lq This accident slows down traffic even further.\rq{ }(\cite[40]{Borillo1984}, glosses added)

	\ex \label{exAppendixFrenchEncoreComparativeVerb}
	\gll Je lui préfère \textbf{encore} son frère.\\
	1\textsc{sg} 3\textsc{sg}.\textsc{dat} prefer.1\textsc{sg} still \textsc{poss}.3\textsc{sg}.\textsc{m}:\textsc{sg} brother(\textsc{m})\\
	\glt \phantom{i}i. \lq I still prefer his brother.'\\
	ii. \lq I like his brother even more (than him).\rq{ }(\cite[145–146]{VictorriFuchs1992}, glosses added)
\end{exe}

\subsubsection{Restrictive (non)-temporal}
\paragraph{Scalar restrictive \lq if at least': \textit{encore si}/\textit{si encore}}
\label{appendixFrenchEncoreIfOnly}
\begin{itemize}
	\item \textcite[s.v. \textit{encore}]{Dicctionnaire}, \textcite{Deloor2012} and \citeauthor{VictorriFuchs1992} (\citeyear{VictorriFuchs1992}, \citeyear{VictorriFuchs1996}).
	\item Form: in the collocations \textit{encore si} and \textit{si encore}, introducing a counterfactual protasis.
	\item  \textcite[83–84]{VictorriFuchs1996} point out that this use can be paraphrased with \textit{encore} in its marginality use  (\appref{appendixFrenchEncoreMarginal}) in the apodosis, e.g. (\ref{exAppendixFrenchEncoreIfOnly1}) as (\ref{exAppendixFrenchEncoreIfOnly1Mod}). In line with this observation, they note that \textit{encore} and \textit{si} are sometimes separated (\ref{exAppendixFrenchEncoreIfOnly3}).
	\item The apodosis is often ellided, as in (\ref{exAppendixFrenchEncoreIfOnly4})
\end{itemize}
\begin{exe}
	\ex\label{exAppendixFrenchEncoreIfOnly1}
	\gll \textbf{Si} \textbf{encore} \textbf{/} \textbf{encore} \textbf{si} tu te ten-ais droit, tu aurais quelque chance de ne pas te faire montrer du doigt.\\
	if still {} still if 2\textsc{sg} \textsc{refl}.2\textsc{sg} keep-\textsc{pst}.\textsc{ipfv}.2\textsc{sg} straight 2\textsc{sg} have.\textsc{cond}.2\textsc{sg} some chance of \textsc{neg} \textsc{neg} \textsc{obj}.2\textsc{sg} make.\textsc{inf} show of:\textsc{def}.\textsc{sg}.\textsc{m} finger(\textsc{m})\\
	\glt \lq If you at least stood straight, you'd have some chance of not being pointed out.'  (\cite[83]{VictorriFuchs1996}, glosses added)
	
		\ex\label{exAppendixFrenchEncoreIfOnly2}
	\gll \textbf{Si} \textbf{encore} il était beau, je pourrais sortir avec lui.\\
	if still 3\textsc{sg}.\textsc{m} \textsc{cop}.\textsc{pst}.\textsc{ipfv}.3\textsc{sg} attractive.\textsc{m} 1\textsc{sg} can.\textsc{cond}.1\textsc{sg} go\_out.\textsc{inf} with 3\textsc{sg}.\textsc{dat}\\
	\glt \lq If he were at least good looking, then I could go out with him.'
	\\(\cite{Deloor2012}, glosses added)
	
	\ex\label{exAppendixFrenchEncoreIfOnly3}
	\gll \textbf{Encore}, si c'-était six heure-s et demie, \textbf{vous} \textbf{pourriez} \textbf{décider} \textbf{de} \textbf{vous} \textbf{lever}.\\
	still if \textsc{prox}.\textsc{sg}.\textsc{m}-\textsc{cop}.\textsc{pst}.\textsc{ipfv}.3\textsc{sg} six hour-\textsc{pl} and half (\textsc{f}) 2\textsc{pl}
 can.\textsc{cond}.2\textsc{pl} decide.\textsc{inf} of \textsc{refl}.2\textsc{pl} get\_up.\textsc{inf}\\
	\glt \lq  If it was half past six, at least then you would decide to get up.'
	\\(\cite[83]{VictorriFuchs1996}, glosses added)	
	
	\ex\label{exAppendixFrenchEncoreIfOnly4}
	\gll \textbf{Si} \textbf{encore} il faisait un effort!\\
	if still 3\textsc{sg}.\textsc{m} make.\textsc{pst}.\textsc{ipfv}.3\textsc{sg} \textsc{indef}.\textsc{sg}.\textsc{m} effort(\textsc{m})\\
	\glt \lq If at least he made an effort!\rq{ }(\cite[83]{VictorriFuchs1996}, glosses added)	
	
	\ex\label{exAppendixFrenchEncoreIfOnly1Mod}
	\gll Si tu te ten-ais droit, \textbf{tu} \textbf{aurais} \textbf{encore} \textbf{quelque} \textbf{chance} de ne pas te faire montrer du doigt.\\
if 2\textsc{sg} \textsc{refl}.2\textsc{sg} keep-\textsc{pst}.\textsc{ipfv}.2\textsc{sg} straight 2\textsc{sg} have.\textsc{cond}.2\textsc{sg} still some chance of \textsc{neg} \textsc{neg} \textsc{obj}.2\textsc{sg} make.\textsc{inf} show of:\textsc{def}.\textsc{sg}.\textsc{m} finger(\textsc{m})\\
	\glt \lq If you stood straight, \textbf{you'd still have some chance} of not being pointed out.' (\cite[83]{VictorriFuchs1996}, glosses added)
\end{exe}

\subsubsection{Broadly modal and interactional functions}\label{appendixFrenchEncoreQ}
\paragraph{Follow-up questions}
\begin{itemize}
	\item \citeauthor{MosegaardHansen2002} (\citeyear{MosegaardHansen2002}, \citeyear[213–214]{MosegaardHansen2008}), \textcite[105]{Noelke1983} and \textcite{Vaelikangas1982}.
	\item In this function, which is an extension of the iterative use, \textit{encore} operates on the speech-act level by highlighting that a similar question has been asked before and it signals that ideally it would not be necessary to ask again. This can yield an attenuating reading, as in (\ref{exAppendixFrenchEncoreQuestion1}). Frequently, however, this use conveys annoyance, as in  (\ref{exAppendixFrenchEncoreQuestion2}).
	\item Syntax: in questions, dislocated to the right.
\end{itemize}
\begin{exe}
	\ex\label{exAppendixFrenchEncoreQuestion1}
	\gll Quel est votre nom, \textbf{encore}?\\
	which \textsc{cop} \textsc{poss}.2\textsc{pl}:\textsc{sg} name, still\\
	\glt \lq What was your name, again?'
(\cite[46]{MosegaardHansen2002}, glosses added)
	
	\ex\label{exAppendixFrenchEncoreQuestion2}
	\begin{xlist}
		\exi{A:}\textit{Seigneur Aristote, peut-on savoir ce qui vous met si fort en colère?}\\
		\lq My Lord Aristote, may one know what makes you so angry?'
		
		\exi{B:}\textit{Un sujet le plus juste au monde.}\\
		\lq A subject that is as reasonable as can be.'
	
		\exi{A:} \gll Et quoi, \textbf{encore}?\\
		and what still\\
		\glt \lq And what IS that?' (Molière, \textit{Le mariage forcé}, cited in \cite[214]{MosegaardHansen2008}, glosses added)
	\end{xlist}
\end{exe}

\paragraph{Concessive protases: \textit{encore que}}\largerpage
\label{appendixFrenchEncoreConcessiveAntecedent}
\begin{itemize}
	\item \textcite[s.v. \textit{encore}]{Dicctionnaire},  \citeauthor{MosegaardHansen2002} (\citeyear{MosegaardHansen2002}, \citeyear[192–197]{MosegaardHansen2008}), \textcite{Morel1996}, and \citeauthor{VictorriFuchs1992} (\citeyear{VictorriFuchs1992}, \citeyear{VictorriFuchs1996}).
	\item Form: in collocation with the subordinator \textit{que}. The clause it introduces may take either the indicative or the subjunctive mood, depending on the givenness of the situation. Typically, the \textit{encore que} protasis follows the apodosis. While it can also precede, \textcite[24]{Morel1996} and \textcite[195]{MosegaardHansen2008} point out that this is rare in contemporary French and has an archaic feel to it.
	\item These (postposed) \textit{encore} \textit{que} concessives are rectificational and work on the speech-act level: the speaker indicates that another assertion may not have been justified, because it could be invalidated by the condition introduced by \textit{encore que}. 
	\item In prosodic terms, \textit{encore que} tends to be set off, both from the preceding clause, and from the clause it governs.
	\item \citeauthor{MosegaardHansen2002} (\citeyear{MosegaardHansen2002}, \citeyear[193–197]{MosegaardHansen2008}) traces this collocation back to the 16th century.
\end{itemize}
\begin{exe}
	\ex
	\gll Max aura une très bonne note, \textbf{encore} \textbf{que} son prof ne l'-aime guère.\\
	M. have.\textsc{fut}.3\textsc{sg} \textsc{indef}.\textsc{sg}.\textsc{f} very good.\textsc{f} grade(\textsc{f}) still \textsc{subord} \textsc{poss}.3\textsc{sg}:\textsc{sg}.\textsc{m} teacher(\textsc{m}) \textsc{neg} 3\textsc{sg}.\textsc{acc}-love.3\textsc{sg} much\\
	\glt \lq Max will get a very good grade. Even so, his teacher doesn’t like him very much.\rq{ }(\cite[195]{MosegaardHansen2008})
	
	\ex
	\begin{xlist}
		\exi{A:}\textit{Cette solidarité, est-ce une tendance montante ou déjà une survivance en crise?}\\
		\lq Is this show of solidarity a growing trend or already an unstable relic from the past?
		\exi{B:}
		\gll Difficile de le savoir. \textbf{Encore} \textbf{que} l’-exemple américain [...] doit nous inciter à la prudence sur les bons sentiments.\\
		difficult of 3\textsc{sg}.\textsc{acc}.\textsc{m} know still \textsc{comp} \textsc{def}.\textsc{sg}-example(\textsc{m}) American.\textsc{m} {} must.3\textsc{sg} 1\textsc{pl}.\textsc{acc} incite.\textsc{inf} to \textsc{def}.\textsc{sg}.\textsc{f} caution(\textsc{f}) about \textsc{def}.\textsc{pl} good.\textsc{pl}.\textsc{m} feeling(\textsc{m}).\textsc{pl}\\
		\glt \lq Difficult to say. Still, the example of the United States should encourage us to be sceptical about finer feelings.’ (\textit{Nouvel Observateur}, cited in \cite[192]{MosegaardHansen2008}, glosses added)
	\end{xlist}
\end{exe}

\paragraph{Concessive apodoses (i): \textit{encore} plus \lq\lq clitic inversion"}
\label{appendixFrenchEncoreConcessiveConsequent1}
\begin{itemize}
	\item \textcite[s.v. \textit{encore}]{Dicctionnaire} and \citeauthor{MosegaardHansen2002} (\citeyear{MosegaardHansen2002}, \citeyear[193–197]{MosegaardHansen2008}, \citeyear[336–337]{MosegaardHansen2016}).
	\item Form/syntax: in clause-initial position, with subject clitics surfacing in an \lq\lq inverted" (diachronically older, but preserved in formal registers and with certain conjunctional adverbials) order. In the vast majority of cases, the verb is \textit{falloir} \lq be necessary', and \textit{encore faut-il} \lq nonetheless it is necessary' (\ref{exAppendixFrenchEncoreConcessiveInversion1}) comes close to a frozen collocation (\cite{MosegaardHansen2002},  \citeyear[193]{MosegaardHansen2008}).

	\item In terms of its pragmatic contribution, \textcite[196]{MosegaardHansen2008} describes this structure as
	\begin{quote} an essentially additive marker, which indicates that the preceding discourse has not exhausted the topic, and that there are thus still things to be said before the first sub-maxim of Quantity (\lq\lq Say as much as is required\rq\rq) can be considered to have been observed.\end{quote}
\end{itemize}

\begin{exe}	\ex\label{exAppendixFrenchEncoreConcessiveInversion1}
	\begin{xlist}
	\exi{} \textit{\lq\lq Une enquête menée par une psychologue auprès de ces familles montre que les enfants ont tout à y gagner”, préciset-il.}\\
	\lq {\lq\lq}Research carried out by a female psychologist on these families shows that the children have everything to gain from it", he specifies.\rq
	\exi{} \gll Mais \textbf{encore} faut-il pouvoir s'-entendre entre parents, sans craindre que l'-autre ne mette fin, sans crier gare, à l'-accord officieux.\\
	but still be\_necessary.3\textsc{sg}-3\textsc{sg}.\textsc{m} can.\textsc{inf} \textsc{refl}.3-understand.\textsc{inf} between parent-\textsc{pl} without be\_afraid.\textsc{inf} \textsc{comp} \textsc{def}-other \textsc{neg} put.\textsc{sbjv}.3\textsc{sg} end without shout.\textsc{inf} warning to \textsc{def}.\textsc{sg}-agreement(\textsc{m}) unofficial.\textsc{m}\\
	\glt \lq Still, the parents do have to be able to get along, without being afraid that the other will suddenly put an unexpected stop to their unofficial agreement.' (\textit{Marie-Claire}, cited in \cite[193–194]{MosegaardHansen2008}, glosses added)
	\end{xlist}
	
	\ex\label{exAppendixFrenchEncoreConcessiveInversion2}
	\begin{xlist}
	\exi{}...[\textit{I]}\textit{l n’est pas un seul pays important pour les intérêts américains où la CIA \textendash{ }ou l’une des douze autres agences de renseignements} ... \textit{n’ait mené récemment une opération d’envergure}.\\
	\lq ...There is not a single country of importance to American interests where the CIA – or one of the twelve other intelligence agencies [...] hasn’t recently carried out a large-scale operation.'
	\exi{}\gll \textbf{Encore} ne connaît-on que celles, fiascos ou succès, que les officiels ont bien voulu divulguer à la presse américaine.\\
	still \textsc{neg} know.3\textsc{sg}-\textsc{impr}/1\textsc{sg} \textsc{comp} \textsc{dist}.\textsc{pl}.\textsc{f} fiasco.\textsc{pl} or success.\textsc{pl} \textsc{rel} \textsc{def}.\textsc{pl} official.\textsc{pl} have.3\textsc{pl} well want.\textsc{ptcp} divulge.\textsc{inf} to \textsc{def}.\textsc{sg}.\textsc{f} press(\textsc{f}) American.\textsc{f}\\
	\glt \lq And even then we only know of those, failures or successes, which officials have been willing to divulge to the American press.' (\textit{Nouvel Observateur}, cited in \cite[194]{MosegaardHansen2008}, glosses added)
	\end{xlist}
\end{exe}	
	
\paragraph{Concessive apodoses (ii): \textit{et encore}}
\label{appendixFrenchEncoreConcessiveConsequent2}
\begin{itemize}
	\item \citeauthor{MosegaardHansen2002} (\citeyear{MosegaardHansen2002}, \citeyear[193–197]{MosegaardHansen2008}) and \textcite{VictorriFuchs1996}.
	\item Form/syntax: in collocation with connective \textit{et} \lq and'. These collocations stand in clause-initial position and tend to be set prosodically separated from both the preceding protasis, and the rest of its host clause.
	\item \textcite[195]{MosegaardHansen2008} describes the contribution of this function as an \lq\lq \lq indirect' form of concession, signaling that one or more inferences that might have been drawn on the basis of the preceding discourse are invalidated". A by and large similar characterisation is given by \textcite{VictorriFuchs1996}.
\end{itemize}
\begin{exe}
	\ex
	\gll Max aura une très bonne note. \textbf{Et} \textbf{encore} son prof ne l'-aime guère.\\
	M. have.\textsc{fut}.3\textsc{sg} \textsc{indef}.\textsc{sg}.\textsc{f} very good.\textsc{f} grade(\textsc{f}) and still \textsc{poss}.3\textsc{sg}:\textsc{sg}.\textsc{m} teacher(\textsc{m}) \textsc{neg} 3\textsc{sg}.\textsc{acc}-love.3\textsc{sg} much\\
	\glt \lq Max will get a very good grade. Even so, his teacher doesn’t like him very much.\rq{ }(\cite[195]{MosegaardHansen2008}, glosses added)

	\ex
	...[\textit{L}]\textit{es responsables de la stratégie des grands opérateurs nous annoncent 19 à 20 millions d’abonnés pour décembre de cette année. Et 60\% des Français, en comptant les nourrissons, équipés en 2002.}\\
	\lq ...The marketing directors of the large telecommunications services tell us they are expecting 19 to 20 million subscribers by December of this year. And that 60\% of the inhabitants of France, including infants, will be equipped by 2002.'
	\exi{}\gll \textbf{Et} \textbf{encore}, assure Yves Goblet, responsable de la stratégie et du développement chez Bouygues Télécom, notre rythme de croissance, certes soutenu, reste modeste par rapport à d’-autres pays européens, comme l’-Italie ou l’-Espagne.\\
	and still assure.3\textsc{sg} Y. G., person\_in\_charge(\textsc{m}) of \textsc{def}.\textsc{sg}.\textsc{f} strategy(\textsc{f}) and of:\textsc{def}.\textsc{sg}.\textsc{m} development(\textsc{m}) at B. T. \textsc{poss}.1\textsc{pl}:\textsc{sg} rythm of growth(\textsc{f}) certainly stable.\textsc{f} rest.3\textsc{sg} modest for relation to of-other.\textsc{pl} country(\textsc{m}).\textsc{pl} European.\textsc{pl}.\textsc{m} like \textsc{def}.\textsc{sg}-Italy or \textsc{def}.\textsc{sg}-Spain\\
	\glt \lq Even so, Yves Goblet, head of marketing and development at Bouygues Télécom, assures us, our growth rate, while certainly stable, is modest in comparison to that of other European countries, such as Italy or Spain.' (\textit{Nouvel Observateur}, cited in \cite[193]{MosegaardHansen2008}, glosses added)
\end{exe}

\paragraph{Concessive interjections: \textit{encore que}, \textit{et encore}}\label{appendixFrenchEncoreConcessiveInterjections}
\begin{itemize}
	\sloppy
	\item  \textcite[s.v. \textit{encore}]{Dicctionnaire}, \textcite{Deloor2012}, \textcite[192–193]{MosegaardHansen2008} and \textcite{VictorriFuchs1996}.
	\item These are holophrastic uses of the concessive collocations \textit{encore que} (\appref{appendixFrenchEncoreConcessiveAntecedent}; ex. \ref{exAppendixFrenchEncoreConcessiveInterjection1}) and  \textit{et encore} (\appref{appendixFrenchEncoreConcessiveConsequent2}; ex. \ref{exAppendixFrenchEncoreConcessiveInterjection2}). Like their full-fledged counterparts, they serve the pragmatic functions of indicating that the speaker might not have been justified in making the preceding statement (\textit{encore que}) and of countering possible inferences (\textit{et encore}).
	\item Without doubt, this is (partially) motivated by the fact that the \textit{encore que} and \textit{et encore} occur in clause-initial position and tend to set off in prosodic terms.
\end{itemize}
\largerpage
\begin{exe}
	\ex\label{exAppendixFrenchEncoreConcessiveInterjection1}
		\begin{xlist}
		\exi{A:}\textit{Le bush, le désert australien, tu aimes?}\\
		 \lq The bush, the Australian desert, you like it?'
		\exi{B:}\textit{Connais pas.}\\
	 \lq Don't know it.'
		\exi{A:}\gll Alors, documente-toi très vite. Seul le bush australien est assez profond pour fuir une femme qui veut un enfant de toi. \textbf{Et} \textbf{encore} …\\
		\textsc{interj} inform.2\textsc{sg}-\textsc{refl}.2\textsc{sg} very quick only \textsc{def}.\textsc{sg}.\textsc{m} bush Australian.\textsc{m} \textsc{cop}.3\textsc{sg} sufficiently deep.\textsc{m} for escape.\textsc{inf} \textsc{indef}.\textsc{sg}.\textsc{f} woman(\textsc{f}) \textsc{rel} want \textsc{indef}.\textsc{sg}.\textsc{m} child(\textsc{m}) of 2\textsc{sg} and still\\
		\glt \lq Well, find out about it as fast as you can. Only the Australian bush is a deep enough hiding-place from a woman who wants a child by you. And even so...' (Pennac, \textit{Monsieur Malaussène}, cited in \cite[193]{MosegaardHansen2008}, glosses added)
\end{xlist}
	\ex\label{exAppendixFrenchEncoreConcessiveInterjection2}
		\textit{La fébrilité qui régnait en fin de semaine dernière rue des Italiens pourrait laisser croire à un proche dénouement de l’affaire.}\\
		\lq The feverishness reigning at the end of last week at the court house in rue des Italiens might lead one to expect that a solution to the matter was imminent.'
		\exi{}\gll \textbf{Encore} \textit{que}! Car depuis l’-origine règne dans ce dossier un climat de manipulation et de désinformation.\\
		still \textsc{comp} because after \textsc{def}.\textsc{sg}-origin reign.3\textsc{sg} in \textsc{prox}.\textsc{sg}.\textsc{m} case(\textsc{m}) \textsc{indef}.\textsc{sg}.\textsc{m} climate(\textsc{m}) of manipulation and of desinformation\\
		\pagebreak
		\glt \lq Not necessarily! For this case has from the very beginning been characterised by a climate of manipulation and misinformation.' (\textit{Nouvel Observateur}, cited in \cite[192–193]{MosegaardHansen2008}, glosses added)
\end{exe}

\paragraph{Performative \textit{encore une fois}}\label{appendixFrenchEncoreIterativeIncrement}
\begin{itemize}
	\item \textcite{Borillo1984}.
	\item The iterative collocation \textit{encore une fois} (\ref{appendixFrenchIterativeIncrement}) can be used as a booster in directive speech acts.
\end{itemize}

\begin{exe}
	\ex
	\gll \textbf{Encore} \textbf{une} \textbf{fois}, rest-ez calm-es.\\
	still \textsc{indef}.\textsc{sg}.\textsc{f} time(\textsc{f}) stay-\textsc{imp}.\textsc{pl} quiet-\textsc{pl}\\
	\glt \lq(I'm saying it) again, be quiet!\rq{ }\parencite[51 fn11]{Borillo1984}
\end{exe}
\largerpage
\subsection{toujours}
\subsubsection{General information}
\begin{itemize}
		\item Wordhood: independent grammatical word.
		\item Etymology: < \lq every day\rq{}
\end{itemize}

\subsubsection{As a (borderline) \textsc{still expression}}
\label{appendixFrenchToujours}
\begin{itemize}
	\item \textcite[s.v. \textit{toujours}]{Dicctionnaire}, \textcite{vanderAuwera1998}, \textcite{Fuchs1988}, \citeauthor{MosegaardHansen2002} (\citeyear{MosegaardHansen2002}, \citeyear[148–150]{MosegaardHansen2008}), \citeauthor{Muller1975} (\citeyear{Muller1975}, \citeyear{Muller1991}) and \textcite{Vaelikangas1982}, among many others.
	 \item Specialisation: while some authors (e.g. \cite{vanderAuwera1998}) treat \textit{toujours} as a full-fledged \textsc{still} expression that is dedicated to the unexpectedly late scenarios, others (e.g. \cite[148–150]{MosegaardHansen2008}; \cite{Fuchs1988}) argue that it does not evoke a subsequent discontinuation scenario. I therefore treat \textit{toujours} as a borderline case of a \textsc{still} expression.
	 \item Polarity sensitivity: inner negation yields (a notion close to) \textsc{not yet}.
	 \item Pragmaticity: if taken to be a \textsc{still} expression, \textit{toujours} would be specialised for the unexpectedly late scenario.
	 \item Further note: \textcite[149]{MosegaardHansen2008} points out that \textit{toujours} as a marker of persistence is first attested from the 13\textsuperscript{th} century on. It is likely that her first attestation, given in (\ref{exAppendixFrenchToujours2}), still involves a frequentative notion \lq at each attempt…\rq{}.
\end{itemize}
\begin{exe}
	\ex 
	\gll Max et Sylvie ont divorcé {il y a} dix an-s, mais ils se détestent \textbf{touiours} autant.\\
	M. and S. have.3\textsc{pl} divorce.\textsc{ptcp} \textsc{exist} ten year-\textsc{pl} but 3\textsc{pl} \textsc{refl}.3 hate.3\textsc{pl} still so\_much\\
	\glt \lq Max and Sylvie divorced ten years ago, but they still hate each other as much as ever.\rq{ }(\cite[24]{MosegaardHansen2002}, glosses added)
	
	\ex Old French,\il{French, Old} 13\textsuperscript{th} century\label{exAppendixFrenchToujours2}\\
	\textit{Si le troevent de tele force et de tele vistece que il ne cuident mie que il soit hons terriens: car il n’a home ou monde qui la moitié poïst soffrir que il a soffert. Si s’esmaient mout, car il voient que il nel pueent remuer de place,}\\
\lq Thus, they find him to have such strength and such speed that they do not believe that he is an earthly man: for there is no man in the world who could endure half of what he has endured. So they become very fearful, for they see that they cannot remove him from his place'
\exi{}
\gll ainz le troevent {\textbf{tor} \textbf{jorz}} d\rq{}-autel force come a-u comencement.\\
\textsc{dem} 3\textsc{sg}.\textsc{acc}.\textsc{m} find.3\textsc{pl} always/still of-similar.\textsc{m} strength(\textsc{m}) like at-\textsc{def}.\textsc{sg}.\textsc{m} start(\textsc{m})\\
\glt \lq but find that he still has as much strength as in the beginning.\rq{ }(\textit{La Queste del Saint Graal}, cited in \cite[149]{MosegaardHansen2008}, glosses added) 

\end{exe}


\subsubsection{Uses on the fringes of \lq{}still\rq{}}
\paragraph{Scalar contexts}
\begin{itemize}
	\item \textcite{Muller1991}
	\item In contexts involving a scale, \textit{toujours} is usually understood as marking the absence of any conceivable change.
	
\end{itemize}

\begin{exe}

	\ex 
	\gll Il lance \textbf{toujours} le javelot à 60 mètre-s.\\
	3\textsc{sg}.\textsc{m} throw.3\textsc{sg} still \textsc{def}.\textsc{sg}.\textsc{m} spear(\textsc{m}) to 60 metre-\textsc{pl}\\
	\glt \lq He still (as ever) throws the spear 60 metres.\rq{ }(\cite[230]{Muller1991}, glosses added)

\end{exe}
\pagebreak
\subsubsection{Broadly adverbial temporal-aspectual functions}
\paragraph{Always}\label{appendixFrenchToujoursAlways}
\begin{itemize}
	\item \textcite[s.v. \textit{toujours}]{Dicctionnaire}, \textcite{Fuchs1988}, \textcite[136–139]{MosegaardHansen2008} and \citeauthor{Muller1975} (\citeyear{Muller1975}, \citeyear{Muller1991}).
	\item A distributive reading (\ref{exAppendixFrenchToujoursAlways2}) can be subsumed under this use.
	\item \textit{Toujours} as a marker of persistence and as \lq always\rq{ }differ in their negation patterns. Thus, the inner negation of the persistive sense is expressed in an embracing fashion, whereas that of the \lq always\rq{ }use involves lexical suppletion through \textit{jamais} \lq (n)ever\rq{}. 
\end{itemize}


\begin{exe}
	\ex 
	\gll A l-’époque où Hugo était en thèse, il était \textbf{toujours} déprimé.\\
	at \textsc{def}-epoch when H. \textsc{cop}.\textsc{pst}.\textsc{ipfv}.3\textsc{sg} in thesis 3\textsc{sg}.\textsc{m} \textsc{cop}.\textsc{pst}.\textsc{ipfv}.3\textsc{sg} always depressed\\
	\glt \lq Back when Hugo was doing his Ph.D., he was always depressed.\rq{ }(\cite[138]{MosegaardHansen2008}, glosses added)

	\ex\label{exAppendixFrenchToujoursAlways2}
	\gll Les Hollandais sont \textbf{toujours} très grand-s.\\
	\textsc{def}.\textsc{pl}.\textsc{m} Dutch.\textsc{pl} \textsc{cop}.3\textsc{pl} always very big-\textsc{pl}.\textsc{m}\\
	\glt \lq The Dutch are always very tall.\rq{ }(\cite[138]{MosegaardHansen2008}, glosses added)
\end{exe}

\subsubsection{\# Marginality / \lq one way or another, in any case\rq{}}\label{appendixFrenchToujoursMarginal}
\begin{itemize}
	\item \textcite[s.v. \textit{toujours}]{Dicctionnaire}; \textcite{Fuchs1988}, \textcite{MosegaardHansen2002}, \citeyear[178–183]{MosegaardHansen2002}) and \textcite{Muller1991}.
	\item \textit{Toujours} does not have a marginality use. Instead, and in line with its etymology, it only has a use that is argumentatively neutral, adding a notion of \lq one way or another, either way\rq{}. 
	\item \textcite[178]{MosegaardHansen2008} points out that, with scalar predicates, this use is only compatible with items ranking low, hence the oddity of (\ref{exAppendixFrenchToujoursMarginality3}).
	\item According to \textcite[179]{MosegaardHansen2008}, this use only shows up around four centuries after the persistive use of \textit{toujours}.
\end{itemize}
\begin{exe}	
	\ex
	\begin{xlist}
		\exi{A:} \textit{Où est-elle, cette preuve?}\\
		\lq Where is this proof?\rq{}
		\exi{B:}\gll Pas dans ma poche, \textbf{toujours}!\\
		\textsc{neg} at \textsc{poss}.1\textsc{sg}:\textsc{sg}.\textsc{f} bag(\textsc{f}) still\\
	\glt \lq Not right here, in any case.\rq{ }(\cite[25]{MosegaardHansen2002}, glosses added)
	\end{xlist}
	
	\ex
	\begin{xlist}
		\exi{A:} \textit{On a fait une collecte parmi les parents afin de pouvoir rénover l’aire de jeux de l’école, et on n’a eu que 1.000 euros.}\\
	\lq We took up a collection among the parents in order to renovate the school’s playground, and we only got 1,000 euros.'
	\exi{B:}\gll Hm! Enfin, c’-est \textbf{toujours} de l\rq{}-argent.\\
	\textsc{interj} anyway  \textsc{prox}.\textsc{sg}.\textsc{m}-\textsc{cop}.3\textsc{sg} still of \textsc{def}.\textsc{sg}-money\\
	\glt \lq Hm! Well, it's always money.' (\cite[172]{MosegaardHansen2008}, glosses added)
	\end{xlist}
	
	\ex
	\gll Un pingouin, c\rq{}-est \textbf{toujours} un oiseau.\\
	\textsc{def}.\textsc{sg}.\textsc{m} penguin(\textsc{m}) \textsc{prox}-\textsc{cop}.3\textsc{sg} always \textsc{indef}.\textsc{sg}.\textsc{m} bird(\textsc{m})\\
	\glt \lq A penguin is always some kind of a bird.\rq{ }(\cite[180]{MosegaardHansen2008}, glosses added)

	\ex\label{exAppendixFrenchToujoursMarginality3}
	\begin{xlist}
		\exi{A:}[]{\textit{Solange n’aura peut-être pas le prix Molière, mais elle a quand même joué dans une vraie pièce, dans un vrai théâtre.}\\
		\lq Solange may not get the Molière prize, but she did after all act in a real play, in a real theater.\rq{}}
		\exi{B:}[?]{\gll Oui, c’-est \textbf{toujours} beaucoup!\\
		yes \textsc{prox}.\textsc{sg}-\textsc{cop}.3\textsc{sg} always much\\
		\glt (intended: \lq Yes, it’s always a big thing!\rq{}) (\cite[178]{MosegaardHansen2008}, glosses added)}
	\end{xlist}
\end{exe}


\subsubsection{Additive and related uses}
\paragraph{Comparisons of inequality}\label{appendixFrenchToujoursComparisons}
\begin{itemize}
	\sloppy
	\item \textcite[164–168]{MosegaardHansen2008}.
	\item Concerning the structure of comparative constructions in French, see \appref{appendixFrenchEncoreComparisons}. Addition of \textit{toujours} yields \lq even more\rq{}. This includes the use with degree verbs, as in (\ref{exApppendixFrenchToujoursComparisons2}).
	\item \textcite[166–167]{MosegaardHansen2008} points out that, unlike \textit{encore} (\appref{appendixFrenchEncoreComparisons}), \textit{toujours} cannot be used in comparisons of inequality that lack a temporal development.
	\item According to \textcite[167]{MosegaardHansen2008} unambiguous instances of \textit{toujours} in comparisons of inequality  are attested about four centuries after the development of the persistive function. She points to  examples like (\ref{exApppendixFrenchToujoursComparisons3}) as possible bridging contexts.
	\item Syntax: forms a constituent with the degree phrase; for instance, in (\ref{exApppendixFrenchToujoursComparisons1}) it occurs between the preposition \textit{avec} \lq with\rq{ }and its complement.
\end{itemize}
\begin{exe}
	\ex\label{exApppendixFrenchToujoursComparisons1}
	\gll Elle me regardait avec \textbf{toujours} plus d'-inquiétude.\\
	3\textsc{sg}.\textsc{f} 1\textsc{sg}.\textsc{obj} look.\textsc{pst}.\textsc{ipfv}.3\textsc{sg} with still more of-concern\\
	\glt \lq She looked at me with ever more disquiet.\rq{ }(\cite[164]{MosegaardHansen2008}, glosses added)
	
	\ex\label{exApppendixFrenchToujoursComparisons2}
	\gll Avec la mondialisation, on peut craindre que les différences entre les ethnie-s s’-accentue  \textbf{toujours} dans les années à venir.\\
	with \textsc{def}.\textsc{sg}.\textsc{f} globalisation(\textsc{f}) \textsc{impr}/1\textsc{pl} can.3\textsc{sg} fear.\textsc{inf} \textsc{subord} \textsc{def}.\textsc{pl} difference.\textsc{pl} between \textsc{def}.\textsc{pl} ethnic\_group-\textsc{pl} \textsc{refl}.3-intensify.3\textsc{pl} still at \textsc{def}.\textsc{pl} year.\textsc{pl} to come.\textsc{inf}\\
	\glt \lq With globalisation, we may fear that the differences between ethnic groups will still increase in the years to come.\rq{ }(\cite[165]{MosegaardHansen2008}, glosses added)
	
	\ex Middle French,\il{French, Middle} 17\textsuperscript{th} century\label{exApppendixFrenchToujoursComparisons3}\\
	\textit{Mais si l’ esperance est esteinte, // pourquoy desir , t’ efforces-tu // de faire une plus grande atteinte? // C’est que tu nays de la vertu, //}\\
	\lq But if hope is extinguished, // why, desire, do you endeavor // to reach higher? // It is because you are born of virtue //\rq{}
	
	\sn\gll 	et comme elle est \textbf{toujours} plus forte, // et sans faveur-s et sans appas, // quoy que l’-esperance soit morte, // desir, pourtant tu ne meurs pas.\\
	and as 3\textsc{sg}.\textsc{f} \textsc{cop}.3\textsc{sg} still more strong {} and without favour-\textsc{pl} and without attraction.\textsc{pl} {} what \textsc{rel} \textsc{def}.\textsc{sg}-hope \textsc{cop}.\textsc{sbjv}.3\textsc{sg} dead {} desire.\textsc{imp} nontheless 2\textsc{sg} \textsc{neg} die.2\textsc{sg} \textsc{neg}\\
	\glt \lq and as it [is always stronger / grows ever stronger], // both without favors and without attractions, // even though hope is dead, // desire, nevertheless you do not die.\rq{ }(d'Urfé, \textit{L’astrée}, cited in \cite[167]{MosegaardHansen2008}, glosses added)
\end{exe}

\subsubsection{Broadly modal and interactional uses}

\paragraph{Concessive apodoses: \textit{toujour est-il que}}\label{appendixFrenchToujoursConcessive}
\begin{itemize}
	\item \textcite{Fuchs1988}, \textcite[199–201]{MosegaardHansen2008} and  \textcite{Morel1996}.
	\item Form: in the fixed expression \textit{toujours est}-\textit{il que} \lq still \textsc{cop}.3\textsc{sg}-3\textsc{sg}.\textsc{m} \textsc{subord}\rq{ }\lq still it's the case that\rq{}. That this phrase is petrified in the present-day language becomes evident in the fact that the predicate in the clause it governs can only surface in the present indicative and can neither be negated, nor can the clause form a question.
	\item This fixed expression can either mark a return from a digression, as in (\ref{exAppendixFrenchToujoursEstilQue1}), or a weak form of concession in which the speaker remits to only stating the facts, without taking a stance; see (\ref{exAppendixFrenchToujoursEstilQue2}).
\end{itemize}
\begin{exe}
	\ex
	\textit{Il est possible que Jean réusisse brillamment à son examen}.\label{exAppendixFrenchToujoursEstilQue1}\\
	\lq It’s possible that Jean will pass his exam with flying colors.\rq
	\exi{}\gll \textbf{Toujours} \textbf{est}-\textbf{il} \textbf{que} son prof ne l’aime guère.\\
	 still \textsc{cop}.3\textsc{sg}-3\textsc{sg}.\textsc{m} \textsc{subord} \textsc{poss}.3\textsc{sg}:\textsc{sg}.\textsc{m} teacher(\textsc{m}) \textsc{neg} 3\textsc{sg}.\textsc{acc}-love.3\textsc{sg} much\\
	 \glt \lq It’s possible that Jean will pass his exam with flying colors. Be that as it may, his teacher doesn’t like him very much.\rq{ }(\cite[199]{MosegaardHansen2008}, glosses added)
	
	\ex\label{exAppendixFrenchToujoursEstilQue2}
	\begin{xlist}
		\exi{A:}\textit{Ton ami Fernand ne me plaît pas du tout: il est trop arrogant.}\\
		\lq I don’t care for your friend Fernand at all: he’s too arrogant.\rq{}
		\exi{B:}\gll  Comme tu veux. \textbf{Toujours} \textbf{est}-\textbf{il} \textbf{qu}-’il est beau mec.\\
		as 2\textsc{sg} want.2\textsc{sg} still \textsc{cop}.3\textsc{sg}-3\textsc{sg}.\textsc{m} \textsc{subord}-3\textsc{sg}.\textsc{m} \textsc{cop}.3\textsc{sg} attractive.\textsc{sg}.\textsc{m} guy(\textsc{m})\\
		\glt \lq Fine. He’s good-looking, though.\rq{ }(\cite[200]{MosegaardHansen2008}, glosses added)
	\end{xlist}
\end{exe}

\paragraph{Speech-act \lq in any case\rq{}}
\begin{itemize}
	\item \textcite[201–203]{MosegaardHansen2008}.
	\item In this function, \textit{toujours} has an \lq in any case\rq{ }reading similar to the one discussed in \appref{appendixFrenchToujoursMarginal}, but functioning on the speech-act level.
	\item According to \textcite[201–202]{MosegaardHansen2008}, this use is first attested around five centuries after the persistive function. She points out that it \lq\lq{}probably arose as a type of afterthought, whereby the speaker could comment on the status of her own speech act as valid independently of the contents of previous discourse.\rq\rq{}
	\item Syntax and prosody: always right-detached.
\end{itemize}

\begin{exe}
	\sloppy
	\ex \textit{Raboliot reconnut Sarcelotte à son parler; il fut heureux d’entendre, après longtemps, son nasillement cordial et gai. – Il y a une pièce que je t’attends,
dit Sarcelotte. On peut causer? Rien qu’ à l’ accent du camarade, Raboliot devina tout de suite. Un tressaillement le parcourut, le chauffa de la nuque aux talons. Il demanda, un peu anxieux:}\\
	\lq Raboliot recognised Sarcelotte by the way he talked; he was happy to hear his hearty and cheerful nasal twang after so long {\lq\lq}I’ve been waiting for you for quite a while\rq\rq{}, said Sarcelotte. \lq\lq{}Can we talk?\rq\rq{} Just by the tone of his friend’s voice, Raboliot guessed it immediately. He shuddered, became hot from head to toe. A bit anxiously, he asked,\rq{}
	\sn
	\gll Tu n’-es pas allé chez moi, \textbf{toujours}? \\
	2\textsc{sg} \textsc{neg}-\textsc{cop}.2\textsc{sg} \textsc{neg} go.\textsc{ptcp}.\textsc{sg}.\textsc{m} place\_of 1\textsc{sg}.\textsc{obj} still\\
	\glt \lq {\lq\lq}You didn’t go to my house, now, did you?{\rq\rq{}}\rq{ }(Genevoix, \textit{Raboliot}, cited in \cite[202]{MosegaardHansen2008}, glosses added)
\end{exe}
\il{French|)}

\section{German (deu, stan1295)}\il{German|(}
\label{appendixGerman}


\subsection{noch}

\subsubsection{General information}
\begin{itemize}
	\item Wordhood: free morpheme.
	\item Etymology: < Proto-Indo-European *\textit{nū̌}-\textit{ku̯e} \lq now-also'.
	\item Further note: not to be confused with disjunctive \textit{noch} \lq nor', which has a different etymology.
\end{itemize}


\subsubsection{As a \lq{}still\rq{ }expression}
\label{appendixGermanStill}
\begin{itemize}
	\item \citeauthor{Abraham1977} (\citeyear{Abraham1977}, \citeyear{Abraham1980}), \citeauthor{vanderAuwera1991BeyondDuality} (\citeyear{vanderAuwera1991BeyondDuality}, \citeyear{vanderAuwera1993}, \citeyear{vanderAuwera1998}),  \citeauthor{Beck2016} (\citeyear{Beck2016}, \citeyear{Beck2019} \citeyear{Beck2020}), \textcite{Doherty1973}, \textcite[185–187]{Helbig1994}, \textcite{HoepelmanRohrer1981} and \textcite{Loebner1989}, among many others.
	
\item Specialisation: several descriptions, such as those by \textcite{Abraham1977}, \citeauthor{Beck2016} (\citeyear{Beck2016}, \citeyear{Beck2019}, \citeyear{Beck2020}), \textcite{Doherty1973} and \textcite{Klein2018}, explicitly address the two components of my definition.
\item Polarity sensitivity: inner negation yields \textsc{not yet}.
\item Pragmaticity: usually  augmented by \textit{immer} > \textit{immer noch}/\textit{noch immer} for the unexpectedly late scenario.
\item Syntax: only phasal polarity \textit{noch} can (on its own) occur in the forefield of the sentence. There it strongly suggests an imminent change (\ref{exAppendixGerman3}).
\item Further note: can be used as an elliptical single-word utterance. Like \textit{noch} in forefield position, this is only available for the phasal polarity function and emphasises the possibility of an imminent discontinuation (\ref{exAppendixGerman4}). Ex (\ref{exAppendixGerman5}) illustrates the use in an imperative.
\end{itemize}
\begin{exe}
	\ex 
	\gll Nachmittag-s war \textbf{noch} alles gut und schön, und jetzt bin ich ein verloren-er Mensch und muss mich tot-schieß-en.\\
afternoon-\textsc{adv} \textsc{cop}.3\textsc{sg} still everything good and beautiful and now \textsc{cop}.1\textsc{sg} 1\textsc{sg} \textsc{indef}.\textsc{nom}.\textsc{sg}.\textsc{m} lost-\textsc{nom}.\textsc{sg}.\textsc{m} human(\textsc{m}) and must.1\textsc{sg} \textsc{refl}.1\textsc{sg} dead-shoot-\textsc{inf}\\
\glt \lq In the afternoon everything was still fine, and now I'm a lost person and have to shoot myself.' (Schnitzler, \textit{Leutnant Gustl}, cited in \cite[45]{Shetter1966}, glosses added)

	\ex
	\gll Hier schein-t \textbf{noch} die Sonne und dort regne-t es schon.\\
	here shine-3\textsc{sg} still \textsc{def}.\textsc{nom}.\textsc{sg}.\textsc{f} sun(\textsc{f}) and there rain-3\textsc{sg} 3\textsc{sg}.\textsc{n} already\\
	\glt \lq Here the sun is still shining and over there it's raining already.'
	\\(\cite[169]{Doherty1973}, glosses added)

	\ex\label{exAppendixGerman3}
	\gll \textbf{Noch} ist Peter in London\\
	still \textsc{cop}.3\textsc{sg} P. in L.\\
	\glt \lq As for now, Peter is still in London (e.g. if you want to talk to him, don't wait any longer).'  (personal knowledge)

	\ex\label{exAppendixGerman4}
	\gll Markus ist der best-e deutsch-e Tennis-spieler. – \textbf{Noch}.\\
	M. \textsc{cop}.3\textsc{sg} \textsc{def}.\textsc{nom}.\textsc{sg}.\textsc{m} best-\textsc{nom}.\textsc{sg}.\textsc{m} German-\textsc{nom}.\textsc{sg}.\textsc{m} tennis-player(\textsc{m}) {} still\\
	\glt \lq Markus is the best German tennis player. -- Still.' (Strongly suggests: this may soon no longer be the case). (\cite[287]{Klein2018}, glosses added)
	
	\ex\label{exAppendixGerman5}
	\gll Wenn ihr das Geld nicht sofort brauch-t, dann \textbf{lass}-\textbf{t} \textbf{es} \textbf{noch} \textbf{lieg}-\textbf{en}.\\
	if/when 2\textsc{pl} \textsc{def}.\textsc{acc}.\textsc{sg}.\textsc{n} money(\textsc{n}) \textsc{neg} immediately need-2\textsc{pl} then let.\textsc{imp}-\textsc{pl} 3\textsc{sg}.\textsc{acc}.\textsc{n} still lay-\textsc{inf}\\
	\glt \lq If you don't need the money right away, then leave it untouched.\rq{}
	\\(found online, glosses added)\footnote{\url{https://www.finanzfrage.net/g/frage/bargeld-ungueltig-zum-umtausch-nach-schweden-verschicken} (30 March, 2023).}

\end{exe}
\largerpage[-1]\pagebreak
\subsubsection{Uses on the fringes of \lq{}still\rq{}}
\paragraph{Scalar contexts}\label{appendixGermanScalar}
\begin{itemize}
	\item \Citeauthor{vanderAuwera1991BeyondDuality} (\citeyear{vanderAuwera1991BeyondDuality}, \citeyear{vanderAuwera1993}), \textcite{Beck2020}, \textcite[s.v. \textit{noch}]{DWDS}, \textcite[s.v. \textit{noch}]{Duden}, \textcite[176–177]{KoenigEtAl1993} \citeauthor{Loebner1989} (\citeyear{Loebner1989}, \citeyear{Loebner1999}), \textcite[620, 630]{MetrichFaucher2009},  \textcite{Shetter1966} and \textcite{Vaelikangas1982}.
	\item The question of whether \textit{noch} in scalar contexts is to be separated from its phasal polarity function has sparked intense discussion in the literature, especially \citeauthor{vanderAuwera1991BeyondDuality} (\citeyear{vanderAuwera1991BeyondDuality}, \citeyear{vanderAuwera1993}) vs. \citeauthor{Loebner1989} (\citeyear{Loebner1989}, \citeyear{Loebner1999}). A crucial element to this debate is that, with narrow focus on a scalar change, \textit{noch} can only have a reading of a decrease (\ref{appendixGermanScalar1}, \ref{appendixGermanScalar2}). It does not allow for a restrictive \lq still only, no more than' reading (be it on its own or in combination with an \lq only\rq{ }marker) which is conveyed by a separate lexical item \textit{erst} \lq no more than\rq{}, lit. \lq{}first, erstwhile'. However, this restriction applies only to \lq\lq{}bare\rq\rq{ }\textit{noch}. Its augmented, emphatic (unexpectedly late) variants \textit{immer noch}/\textit{noch immer} are amply attested with scope over a restrictive marker in increase contexts; see (\ref{appendixGermanScalar3}, \ref{appendixGermanScalar4}).
	
	\item The fixed expression \textit{noch} … \textit{hin sein} \lq still be … off\rq{ }is sometimes encountered with the scalar predicate left implied, as in (\ref{appendixGermanScalar3}).
	\item	  The collocation (complex particle) \textit{nur noch} \lq only still' conveys the idea of \lq as little as … left', as in (\ref{appendixGermanScalar6}).  Similarly \textit{kaum noch} \lq hardly still\rq{ }conveys a high degree of reduction of intensity or frequency (\ref{appendixGermanScalar7}).
	\item Syntax: scalar \textit{noch} can form a single constituent with the focus. Note, for instance, \textit{noch} and \textit{wenige Meter} \lq a few meters\rq{ }together occupying the initial position of a V2 clause in (\ref{appendixGermanScalar2})
\end{itemize}
\begin{exe}
	\ex\label{appendixGermanScalar1}
	 Context: I had ten copies of a book, but I gave away five.\\
	\gll Ich hab' \textbf{noch} fünf \textup{(}übrig\textup{)}. \\
	1\textsc{sg} have.1\textsc{sg} still five \phantom{(}left\\
	\glt \lq I still have five (left).' (personal knowledge)
	\ex\label{appendixGermanScalar2}
	\gll \textbf{Noch} wenig-e Meter war-en es bis zu-r Staffel-übergabe.\\
	still few-\textsc{nom}.\textsc{pl} meter.\textsc{pl} \textsc{cop}.\textsc{pst}-3\textsc{pl} 3\textsc{sg}.\textsc{n} until to-\textsc{def}.\textsc{dat}.\textsc{sg}.\textsc{f} relay-handover(\textsc{f})\\
	\glt \lq A few meters were left until the passing of the baton.\rq{ }(found online, glosses added)\footnote{\url{https://www.wz.de/nrw/kreis-viersen/viersen/ein-schueler-staffellauf-fuer-mehr-miteinander_aid-30775241} (24 January, 2023).}
	
	\ex\label{appendixGermanScalar3}
		\gll Bis Weihnachten \textbf{ist} \textbf{noch} \textbf{hin}, aber wir ess-en Keks-e i-m Sommer auch gerne.\\
	until christmas \textsc{cop}.3\textsc{sg} still thither but 1\textsc{pl} eat-1\textsc{pl} cookie-\textsc{pl} in-\textsc{def}.\textsc{dat}.\textsc{sg}.\textsc{m} summer(\textsc{m}) also gladly\\
	\glt \lq Christmas is still a long way off, but we also enjoy cookies during summer time.\rq{ }(found online, glosses added)\footnote{\url{https://www.chefkoch.de/rezepte/2289401365066405/Haselnussplaetzchen.html} (13 February, 2023).}

	\ex\label{appendixGermanScalar4}
	\textit{Helge Malchow, Verlagschef von Kiepenheuer und Witsch, glaubt:}\\
	\lq Helge Malchow, publishing director at Kiepenheuer und Witsch, believes:\rq
	\exi{}\gll 2012 ist der Durchbruch der E-Books. Sie mach-en zwar \textbf{immer} \textbf{noch} nur fünf Prozent des Umsatz-es aus, ihr Anteil steig-t aber rapide.\\
	2012 \textsc{cop}.3\textsc{sg} \textsc{def}.\textsc{nom}.\textsc{sg}.\textsc{m} breakthrough(\textsc{m}) \textsc{def}.\textsc{gen}.\textsc{pl} e-books 3\textsc{pl} make-3\textsc{pl} though always still only five percent \textsc{def}.\textsc{gen}.\textsc{sg}.\textsc{m} revenue(\textsc{m})-\textsc{gen} out \textsc{poss}.3\textsc{pl}:\textsc{nom}.\textsc{sg}.\textsc{m} share(\textsc{m}) rise-3\textsc{sg} however rapidly\\
	\glt 2012 will be the breakthrough year for e-books. Though they still only account for five percent of revenue, their share is rising rapidly.\rq{ }(found online, glosses added)\footnote{\url{https://www.abendblatt.de/kultur-live/article109757622/Schwarzenegger-auf-Buchmesse-Ein-totaler-Auftritt.html} (08 March, 2023).}
	
	\ex\label{appendixGermanScalar5}
	Context: About a freely available software for contact tracking.\\
	\gll Trotzdem nutz-t \textbf{immer} \textbf{noch} {gerade mal} ein Drittel der deutsch-en Gesundheitsämter die Software. Der Rest greif-t auf Excel-Tabelle-n oder eigen-e Lösung-en zurück.\\
	nonetheless use-3\textsc{sg} always still merely one third \textsc{def}.\textsc{gen}.\textsc{pl} german-\textsc{gen}.\textsc{pl} health\_department.\textsc{pl} \textsc{def}.\textsc{acc}.\textsc{sg}.\textsc{f} software(\textsc{f}) \textsc{def}.\textsc{nom}.\textsc{sg}.\textsc{m} rest(\textsc{m}) grasp-3\textsc{sg} on Excel-table-\textsc{pl} or own-\textsc{acc}.\textsc{pl} solution-\textsc{pl} back\\
	\glt \lq Nonetheless so far no more than a third of all German health departments employ the software. The remainder resorts to Excel sheets or to proprietary solutions.\rq{ }(found online, glosses added)\footnote{\url{https://netzpolitik.org/2021/kontaktverfolgung-in-den-semesterferien/} (08 March, 2023).}
	

	\ex\label{appendixGermanScalar6}
	\gll Heute können \textbf{nur} \textbf{noch} 10\% all-er Sami von der Rentier-zucht und vo-m Fisch-fang alleine leb-en.\\
	today can.3\textsc{pl} only still 10\% all-\textsc{gen}.\textsc{pl} S. of \textsc{def}.\textsc{dat}.\textsc{sg}.\textsc{f} raindeer-breeding(\textsc{f}) and of-\textsc{def}.\textsc{dat}.\textsc{sg}.\textsc{m} fish-catch(\textsc{m}) alone live-\textsc{inf}\\
	\glt \lq These days, no more than 10\% of all Sami people can still make a living of reindeer breeding and fishing alone.\rq{ }(Heyne, \textit{…Nur noch bis dahinten!}, glosses added)

	\ex\label{appendixGermanScalar7}
	\gll Er trank kaum \textbf{noch}.\\
	3\textsc{sg}.\textsc{m} drink.\textsc{pst}.3\textsc{sg} hardly still\\
	\glt \lq He hardly (ever) took a drink anymore.\rq{ }(\cite[172]{KoenigEtAl1993}, glosses added)
\end{exe}

\paragraph{\lq Thus far always/every(one)\rq{}}\label{appendixGermanNochJeder}
\begin{itemize}
	\item \textcite[172–173]{KoenigEtAl1993} and \textcite[628]{MetrichFaucher2009}.
	\item Form: in collocation with a focus that includes or consists of a universal quantifier (incl. \textit{immer} \lq always\rq{}) and a predicate in the analytical anterior or the past tense.
	\item This collocation expresses a generic rule that has been true until utterance time (or another evaluation time) \lq thus far always, thus far everyone\rq{}.
	\item The meaning of this collocation can be explained through coercion of a generic or experiential state into an alterable one, which, in turn, is facilitated by focus on the universal quantifier. In terms of usage, it most likely goes back to analogy with \lq not ever (yet)\rq{ }statements, in which \textit{noch} forms part of the expression for \textsc{not yet} (\ref{exAppendixGermanNochJeder4}), as well as with the collocation \textit{schon immer} \lq since always\rq{}, lit. \lq{}already always\rq{}). The latter, in turn, builds on the \textsc{already} expression \textit{schon} in scalar contexts.
	\item Syntax: forms a constituent with its focus, either preceding or following it.
\end{itemize}

\begin{exe}
	\ex 
	\gll Wie \textbf{jed}-\textbf{es} \textbf{Jahr} \textbf{noch} war die Zahl der Fremd-en allmählich bis Ostern ge-wachs-en.\\
	as every-\textsc{nom}.\textsc{sg}.\textsc{n} year(\textsc{n}) still \textsc{cop}.\textsc{pst}.3\textsc{sg} \textsc{def}.\textsc{nom}.\textsc{sg}.\textsc{f} number(\textsc{f}) \textsc{def}.\textsc{gen}.\textsc{pl} stranger-\textsc{gen}.\textsc{pl} slowly until Easter \textsc{ptcp}-grow-\textsc{ptcp}\\
	\glt \lq As every year thus far, the number of visitors had been steadily increasing towards Easter.\rq{ }(von Ompleda, \textit{Margret und Ossana}, glosses added)

	\ex 
	\gll Ich war \textbf{noch} beinahe jedes Jahr an der Fachtagung und ich hab-e immer gross-e Vor-freude.\\
	1\textsc{sg} \textsc{cop}.\textsc{pst}.1\textsc{sg} still nearly every-\textsc{nom}.\textsc{sg}.\textsc{n} year(\textsc{n}) at \textsc{def}.\textsc{dat}.\textsc{sg}.\textsc{f} symposium(\textsc{f}) and 1\textsc{sg} have-1\textsc{sg} always big-\textsc{acc}.\textsc{sg}.\textsc{f} pre-joy(\textsc{f})\\
	\glt \lq I have participated in the congress nearly every year so far and I always look forward to it.\rq{ }(found online, glosses added)\footnote{\url{https://www.furrerevents.ch/fachtagung-muetter-und-vaeterberatung} (24 February, 2023).}
	
	\ex
	\gll Sterb-en kann nicht so schwer sein. Das hat \textbf{noch} \textbf{jeder} ge-schaff-t.\\
	die-\textsc{inf} can.3\textsc{sg} \textsc{neg} so difficult \textsc{cop}.\textsc{inf} 3\textsc{sg}.\textsc{acc}.\textsc{n} have still every-\textsc{nom}.\textsc{sg}.\textsc{m} \textsc{ptcp}-achieve-\textsc{ptcp}\\
	\glt \lq Dying can’t be that difficult. So far everyone has achieved it.\rq{ }(found online, glosses added)\footnote{\url{https://www.zeit.de/kultur/2018-03/sterben-thomas-macho-philosoph-gesellschaft-tod} (22 February, 2023).}
	
	\ex\label{exAppendixGermanNochJeder4}
	\gll Das hat \textbf{noch} \textbf{niemand} ge-schaff-t.\\
	3\textsc{sg}.\textsc{acc}.\textsc{n} have.3\textsc{sg} still nobody \textsc{ptcp}-achieve-\textsc{ptcp}\\
	\glt \lq As of yet, nobody has ever achieved that.\rq{ }(personal knowledge)
	
	\ex\label{exAppendixGermanNochJeder5}
	\gll Das ist \textbf{schon} \textbf{immer} so \textup{(}gewesen\textup{)}.\\
		3\textsc{sg}.\textsc{n} \textsc{cop}.3\textsc{sg} already always thus \phantom{(}\textsc{cop}.\textsc{ptcp}\\
	\glt \lq It's been like this since ever.\rq{ }(personal knowledge)
	\end{exe}

\subsubsection{Broadly adverbial temporal-aspectual functions}

\paragraph{Iterative (and restitutive) via increment}
\label{appendixGermanIterativeViaIncrement}
\begin{itemize}
	\item 
	\textcite[s.v. \textit{noch einmal}, \textit{nochmal}, \textit{nochmals}]{Duden},
	\textcite[s.v. \textit{nochmals}]{DWDS}, \textcite[176]{KoenigEtAl1993}, \textcite[627]{MetrichFaucher2009}, \textcite[27, 105]{Nederstigt2003}, \textcite{Umbach2012}, \textcite{Sauerland2009} and \textcite{Shetter1966}.
	\item Form: in collocation with the event quantifier \textit{mal} \lq time(s)\rq{}; stylistic variants with and without the indefinite article. Additionally, there is a form \textit{nochmals}, featuring the adverbial derivation \mbox{-\textit{s}} (\ref{exAppendixGermanIterativeViaIncrement3}, \ref{exAppendixGermanIterativeViaIncrement4}).	 Lastly, this collocation also underlies the compound adjective \textit{noch}-\textit{mal}-\textit{ig} \lq still-time-\textsc{adj}' \lq repeated'.
		\item Aside from iteration (\ref{exAppendixGermanIterativeViaIncrement1}– \ref{exAppendixGermanIterativeViaIncrement3}), this use also has been described as having restitutive readings in examples like (\ref{exAppendixGermanIterativeViaIncrement4}, \ref{exAppendixGermanIterativeViaIncrement5}). 	My own judgement suggests that these and similar examples evoke a transitory restitution of a state. This is particularly salient in (\ref{exAppendixGermanIterativeViaIncrement5}), which could be continued \textit{bevor sie ihn dann tranken} \lq before eventually drinking it'. Similarly, (\ref{exAppendixGermanIterativeViaIncrement4}) could be continued  \textit{bevor sie dann losfuhr} \lq before she set off'. This suggests that the restitutive readings simultaneously evoke the \lq\lq further-to" use of \textit{noch} (\appref{appendixGermanFurtherTo}), in which case (\textit{ein})\textit{mal} serves not as an event quantifier but as \lq for a moment, just briefly'.
\end{itemize}

\begin{exe}
	\ex\label{exAppendixGermanIterativeViaIncrement1}
	\gll Lass uns \textbf{noch} \textbf{mal} klingel-n.\\
	let 1\textsc{pl}.\textsc{acc} still time ring-\textsc{inf}\\
	\glt \lq Let's ring the bell again.\rq{ }(\cite[s.v. \textit{nochmal}]{Duden}, glosses added)

	\ex\label{exAppendixGermanIterativeViaIncrement2}
	Context: A newspaper headline about the 2019 Istanbul municipal elections.\\
	\gll Opposition gewinn-t Wahl in Istanbul: İmamoğlu macht-'s \textbf{noch} \textbf{ein}-\textbf{mal}\\
	opposition win-3\textsc{sg} election.\textsc{acc}.\textsc{sg} in Istanbul İmamoğlu make.3\textsc{sg}-3\textsc{sg}.\textsc{acc}.\textsc{n} still \textsc{indef}-time\\
	\glt \lq Opposition wins elections in Istanbul: İmamoğlu does it again.\rq{ }(found online, glosses added)\footnote{\url{https://taz.de/Opposition-gewinnt-Wahl-in-Istanbul/!5605032/}; (07 April, 2022).}
	
	\ex Context: The subject has intended standing up, but his legs failed him.\label{exAppendixGermanIterativeViaIncrement3}
	\exi{}\gll Und er versuch-te \textbf{noch}-\textbf{mal}-\textbf{s}, auf die Bein-e zu komm-en, und da er sich gehörig zusammen-nahm, so ging es.\\
	and 3\textsc{sg}.\textsc{m} try-\textsc{pst}.3\textsc{sg} still-time-\textsc{adv} on \textsc{def}.\textsc{acc}.\textsc{pl} leg-\textsc{acc}.\textsc{pl} to come-\textsc{inf} and because 3\textsc{sg}.\textsc{m} \textsc{refl}.3 properly together-take.\textsc{pst}.3\textsc{sg} so go.\textsc{pst}.3\textsc{sg} 3\textsc{sg}.\textsc{n}\\
	\glt \lq And he tried once more to get up, and as he pulled himself together, he managed to.' (Mann, \textit{Der Zauberberg}, glosses added)

	\ex\label{exAppendixGermanIterativeViaIncrement4}
	\gll Sie hatte den Motor schon angelassen, da stieg sie \textbf{noch}-\textbf{mal}-\textbf{s} aus.\\
3\textsc{sg}.\textsc{f} have.\textsc{pst}.3\textsc{sg} \textsc{def}.\textsc{acc}.\textsc{sg}.\textsc{m} motor(\textsc{m}) already start\_up.\textsc{ptcp} there climb.\textsc{pst}.3\textsc{sg} 3\textsc{sg}.\textsc{f} still-time-\textsc{adv} out\\
\glt \lq She had already started the engine, when she climbed out of the car again.' (\cite[s.v. \textit{nochmals}]{Duden}, glosses added)

	\ex\label{exAppendixGermanIterativeViaIncrement5}
\gll Sie bestell-ten eine Flasche Gruaud Larose bei ihr, die Hans Castorp \textbf{noch} \textbf{ein}-\textbf{mal} fort-schick-te. um sie besser temperier-en zu lass-en.\\
3\textsc{pl} order-\textsc{pst}.3\textsc{pl} \textsc{indef}.\textsc{acc}.\textsc{sg}.\textsc{f} bottle(\textsc{f}) G. L. at 3\textsc{sg}.\textsc{dat}.\textsc{f} \textsc{rel}.\textsc{acc}.\textsc{sg}.\textsc{f} H. C. still one-time away-send-\textsc{pst}.3\textsc{sg} to 3\textsc{sg}.\textsc{acc}.\textsc{f} better temper-\textsc{inf} to let-\textsc{inf}\\
\glt \lq They ordered a bottle of Gruaud Larose from her, which Hans Castorp sent back to have it brought to drinking temperature.' (Mann, \textit{Der Zauberberg}; cited in \cite[61]{Shetter1966}, glosses added)
\end{exe}

\paragraph{Prospective \lq eventually\rq{}}
\label{appendixGermanProspective}
\begin{itemize}
	\sloppy
	\item \citeauthor{Abraham1977}, (\citeyear{Abraham1977}, \citeyear{Abraham1980}, \textcite{vanderAuwera1993}, \textcite[s.v. \textit{noch}]{DWDS},  \textcite{Doherty1973}, \textcite[s.v. \textit{noch}]{Duden}, \textcite[187]{Helbig1994}, \textcite{HoepelmanRohrer1981}, \citeauthor{Koenig1977} (\citeyear{Koenig1977}, \citeyear[140–143]{Koenig1991}), \textcite[173]{KoenigEtAl1993}, \textcite{Loebner1989}, \textcite[621–622]{MetrichFaucher2009}, \textcite{Shetter1966} and \textcite{Vaelikangas1982}; also see \textcite{Vandeweghe1984} on the \ili{Dutch} cognate \textit{nog}.	
	\item This reading is forced when \textit{noch} combines with emphatic \textit{schon} (\lq no doubt', lit. \lq already') and adverbials like \textit{schließlich} \lq eventually' or \textit{eines Tages} \lq one day'.
	\item This reading is available with telic (\ref{exAppendixGermanProspective1}, \ref{exAppendixGermanProspective2}) and atelic predicates (\ref{exAppendixGermanProspective3}).
	\item As \textcite{Shetter1966} points out, in combination with modals (\ref{exAppendixGermanProspective4}) and the copula\hyp plus\hyp infinitive constructions (\ref{exAppendixGermanProspective5}) there is often ambiguity between a persistent possibility/necessity and prospective \lq eventually\rq{}.
	\item Ex. (\ref{exAppendixGermanProspective6}) illustrates the combination with a negated predicate.
\end{itemize}

\begin{exe}
	\ex\label{exAppendixGermanProspective1}
	\gll Er wird sich \textbf{noch} zu Tod-e arbeit-en.\\
	3\textsc{sg}.\textsc{m} \textsc{fut}.\textsc{aux}:3\textsc{sg} \textsc{refl}.3 still to death-\textsc{dat} work-\textsc{inf}\\
	\glt \glt \lq He will end up working himself to death.\rq{ }(\cite[141]{Koenig1991}, glosses added)
		
	\ex\label{exAppendixGermanProspective2}
	\gll Wir mach-en \textbf{noch} einen gut-en Kricket-spieler aus ihm.\\
	1\textsc{pl} make-1\textsc{pl} still \textsc{indef}.\textsc{acc}.\textsc{sg}.\textsc{m} good-\textsc{acc}.\textsc{sg}.\textsc{m} cricket-player(\textsc{m}) of 3\textsc{sg}.\textsc{dat}.\textsc{m}\\
	\glt \lq Weʼll make a good cricketer of him yet.\rq{ }(\cite[197]{Koenig1977}, glosses added)
	
	\ex\label{exAppendixGermanProspective3}
	\textit{Carlo berührte ihn am Arm. \lq\lq Still, komm jetzt hinunter!" Geronimo schwieg und gehorchte dem Bruder.}\\
	\lq Carlo touched his arm. \lq\lq That's enough, come downstairs already!" Geronimo kept quiet and obeyed his brother.\rq
	\exi{}\gll Aber auf den Stufe-n sag-te er: \lq\lq Wir red-en \textbf{noch}, wir red-en \textbf{noch}!"\\
	but on \textsc{def}.\textsc{dat}.\textsc{pl} stair-\textsc{pl} say-\textsc{pst}.3\textsc{sg} 3\textsc{sg}.\textsc{m} \phantom{\lq\lq}1\textsc{pl} talk-1\textsc{pl} still 1\textsc{pl} talk-1\textsc{pl} still!"\\
	\glt \lq But on the stairs he said \lq\lq We'll talk yet, we'll talk yet!" (i.e. this is not over)' (Schnitzler, \textit{Der blinde Geronimo}, cited in \cite[48–49]{Shetter1966}, glosses added)

	\ex\label{exAppendixGermanProspective4}
	\gll Du mein-st, dass er durch-s Moor die Seinen \textbf{noch} erreich-en kann?\\
	2\textsc{sg} think-2\textsc{sg} \textsc{comp} 3\textsc{sg}.\textsc{m} through-\textsc{def}.\textsc{acc}.\textsc{sg}.\textsc{n} moor(\textsc{n}) \textsc{def}.\textsc{acc}.\textsc{pl} \textsc{poss}.3\textsc{sg}.\textsc{m}:\textsc{acc}.\textsc{pl} still reach-\textsc{inf} can.3\textsc{sg}\\
	\glt \lq You think that going via the moor he can still get to his people / … he can get to his people eventually?' (von le Fort, \textit{Anna Elisabeth von Golzow}, cited in \cite[49]{Shetter1966}, glosses added)

	\ex\label{exAppendixGermanProspective5}	
	\gll Trotz der schön-en selbst-gezogen-en Perinette- und Grand-Richard-Äpfel, die \textbf{noch} zu prüf-en war-en, a-m Nach-mittag war ich davon-geritten.\\
	despite \textsc{def}.\textsc{gen}.\textsc{pl} beautiful-\textsc{gen}.\textsc{pl} self-grown-\textsc{pl} P. and G.-R.-apple.\textsc{pl} \textsc{rel}.\textsc{nom}.\textsc{pl} still to check-\textsc{inf} \textsc{cop}.\textsc{pst}-3\textsc{pl} at-\textsc{def}.\textsc{dat}.\textsc{sg}.\textsc{m} after-noon(\textsc{m}) \textsc{cop}.\textsc{pst}.1\textsc{sg} 1\textsc{sg} off-ride.\textsc{ptcp}\\
	\glt \lq Despite the fine home-grown Perinette and Grand Richard apples that were still to be tasted / to be tasted at some point, come afternoon I had set off.' (Storm, \textit{Der Schimmelreiter}, glosses added)
	
	\ex  Context: About a rescue exercise.\label{exAppendixGermanProspective6}\\
	\gll Damit der Darsteller des Unfall-opfer-s bei-m Wart-en … \textbf{nicht} \textbf{noch} \textbf{krank} \textbf{wird}, bekomm-t er eine Wärme-folie mit auf den Weg.	\\
	\textsc{purp} \textsc{def}.\textsc{nom}.\textsc{sg}.\textsc{m} impersonator(\textsc{m}) \textsc{def}.\textsc{gen}.\textsc{sg}.\textsc{n} accident-victim(\textsc{n})-\textsc{gen} at-\textsc{def}.\textsc{dat}.\textsc{sg}.\textsc{n} wait-\textsc{inf}(\textsc{n}) {}
	 \textsc{neg} still ill become.3\textsc{sg} get-3\textsc{sg} 3\textsc{sg}.\textsc{m} \textsc{indef}.\textsc{acc}.\textsc{sg}.\textsc{f} warmth-foil(\textsc{f}) with on \textsc{def}.\textsc{acc}.\textsc{sg}.\textsc{m} way(\textsc{m})\\
	\glt \lq The actor impersonating the victim is given a thermal foil, so he doesn't end up sick while waiting.\rq{ }(found online, glosses added)\footnote{\url{https://www.donaukurier.de/archiv/alle-sechs-unfallopfer-bestens-versorgt-4210654} (22 March, 2023).}
\end{exe}



\subsubsection{Temporal connectives and frame setters}
\paragraph{Persistent time frame}\label{appendixGermanContinuativeTT}
\begin{itemize}
	\item \textcite[s.v. \textit{noch}]{DWDS}, \citeauthor{Beck2016} (\citeyear{Beck2016},  \citeyear{Beck2019}, \citeyear{Beck2020}), \textcite[s.v. \textit{noch}]{Duden}, \textcite[185–186]{Helbig1994}, \textcite{Koenig1977}, \textcite[177]{KoenigEtAl1993}, \textcite{Loebner1989}, \textcite[623–624]{MetrichFaucher2009}, \textcite{Shetter1966} and \textcite{Vaelikangas1982}. 
	\item As \textcite{Loebner1989} points out, this use can be paraphrased as \lq it is still … when the situation occurs', and is restricted to time specifications that include an established or salient topic time. The latter can be mediated by place-to-time metonymy (\ref{exAppendixGermanContinuativeTimeFrame4}).
	\item This use often gives the impression of relative earliness. Accordingly, the \textit{Duden} dictionary \parencite[s.v. \textit{noch}]{Duden} describes it as  \lq\lq … den Umständen nach früh [early for the circumstances given]". A similar description is given by \textcite[185]{Helbig1994}. As \textcite{Beck2020} shows, this meaning component is defeasible. This is illustrated in (\ref{exAppendixGermanContinuativeTimeFrame6}).
	\item This use is also frequent in combination with expressions of precedence, as in (\ref{exAppendixGermanContinuativeTimeFrame5}). This can be explained by the fact that \textit{noch} in this function generally evokes an idea of precedence, e.g. \lq before the end of the negotiations' in (\ref{appendixGermanContinuativeTT1}), \lq before the end of the day' in (\ref{appendixGermanContinuativeTT2}), or \lq before death' in (\ref{appendixGermanContinuativeTT3}). As \textcite{Shetter1966} points out, in this case \textit{bevor}/\textit{vor}/\textit{ehe} is stressed, which clearly differentiates \textit{noch vor Ostern} \lq before Easter' in (\ref{exAppendixGermanContinuativeTimeFrame5}) from \textit{noch vor hundert Jahren} \lq as recently as a hundred years ago' in (\ref{exAppendixGermanTimeScalar2}).
	\item Syntax: forms a constituent with the temporal expression, either preceding or following it; note, for instance, their co-occurrence in the forefield position of a V2 clause in (\ref{exAppendixGermanContinuativeTimeFrame6}).
\end{itemize}

\begin{exe}
\ex\label{appendixGermanContinuativeTT1}
	 Context: A 1989 article newspaper headline and a subsequent paraphrase.
		\exi{}\gll SPD befürcht-et, daß der Bundestag \textbf{noch} während der Gespräch-e der Stationierung zustim-en soll.\\
		SPD fear-3\textsc{sg} \textsc{comp} \textsc{def}.\textsc{nom}.\textsc{sg}.\textsc{m} German\_parliament(\textsc{m}) still during \textsc{def}.\textsc{gen}.\textsc{pl} talk-\textsc{gen}.\textsc{pl} \textsc{def}.\textsc{dat}.\textsc{sg}.\textsc{f} deployment(\textsc{f}) agree\_on-\textsc{inf} should.3\textsc{sg} \\
	\glt \lq The SPD fears that the German parliament will vote in favour of deployment while the talks are still going on.'
			\exi{}\textit{Kohl nehme damit eilfertig die ihm zugeschobene Rolle an, die Nachrüstung einzuleiten, \textbf{während}  \textbf{die}  \textbf{Großmächte}  \textbf{noch} \textbf{am}  \textbf{Verhandlungsstisch}  \textbf{säßen}, erklärte der stellvertretende SPD-Fraktions-vorsitzende Horst Ehmke}.\\
 \lq The representative of the SPD parliament group, Horst Ehmke, declared that, in so doing, Kohl had too hastily accepted the role he was forced into, launching rearmament while the super powers were still having negotiations.' (\textit{Süddeutsche Zeitung}, cited in \cite[211, fn 12]{Loebner1989})

\ex\label{appendixGermanContinuativeTT2}
\gll Deshalb hab-e ich damals den Stossseufzer ausgestossen, ich woll-e gern nach Amerika, aber nicht nach New York. So bin ich \textbf{noch} a-m Abend meiner  Ankunft wenigstens nach Boston ge-fahr-en.\\
that\_is\_why have-1\textsc{sg} 1\textsc{sg} back\_then \textsc{def}.\textsc{acc}.\textsc{sg}.\textsc{m} deep\_sigh(\textsc{m})
eject.\textsc{ptcp} 1\textsc{sg} want-\textsc{sbjv}.1\textsc{sg} gladly to America but \textsc{neg} to N. Y. thus \textsc{cop}.1\textsc{sg} 1\textsc{sg} still at-\textsc{def}.\textsc{dat}.\textsc{sg}.\textsc{m} evening(\textsc{m}) \textsc{poss}.1\textsc{sg}:\textsc{gen}.\textsc{sg}.\textsc{f} arrival(\textsc{f}) at\_least to B. \textsc{ptcp}-drive-\textsc{ptcp}\\
\glt \lq That is why back then I complained that, while I did want to go to America, I didn't want to be stuck in New York. Thus, the very day of my arrival I went to Boston, at least.' (\textit{Rheinischer Merkur}, cited in \cite[57]{Shetter1966}, glosses added)

\ex\label{appendixGermanContinuativeTT3}
	Context: On how an employee should behave when your work contract has been terminated.\\
	\gll Also, \textbf{noch} während man arbeitet und sobald man weiß, wann der letzt-e Arbeit-s-tag sein soll: zu-r Arbeitsargentur geh-en.\\
	so, still while \textsc{impr} work-3\textsc{sg} and once \textsc{impr} know.3\textsc{sg} when \textsc{def}.\textsc{nom}.\textsc{sg}.\textsc{m} last-\textsc{nom}.\textsc{sg}.\textsc{m} work-\textsc{lnk}-day(\textsc{m}) \textsc{cop}.\textsc{inf} should.3\textsc{sg} to-\textsc{def}.\textsc{dat}.\textsc{sg}.\textsc{f} job\_centre(\textsc{f}) go-\textsc{inf}\\
	\glt \lq So, while [it is still the time that] you're employed and as soon as you know when your last day at work is scheduled: visit the job centre.\rq{ }(found online, glosses added)\footnote{\url{https://www.verdi.de/service/fragen-antworten/++co++3f41ec44-a247-11e0-42ae-00093d114afd} (26 April, 2022).}

\ex\label{exAppendixGermanContinuativeTimeFrame4}
\gll Er wurde \textbf{noch} a-m Unfall-ort operier-t.\\
3\textsc{sg}.\textsc{m} become.\textsc{pst}.3\textsc{sg} still at-\textsc{def}.\textsc{dat}.\textsc{sg}.\textsc{m} accident-place(\textsc{m}) operate-\textsc{ptcp}\\
\glt \lq He was operated right at (before leaving) the scene of the accident.'  (\cite[s.v. \textit{noch}]{Duden}, glosses added)

\ex\label{exAppendixGermanContinuativeTimeFrame5}
\gll Die Maske-n-pflicht  … könnte … \textbf{noch} vor Ostern ge-locker-t oder ganz abgeschafft werd-en.\\
\textsc{def}.\textsc{nom}.\textsc{sg}.\textsc{f} mask-\textsc{pl}-requirement(\textsc{f}) {} can.\textsc{cond}.3\textsc{sg} {} still before Easter \textsc{ptcp}-loosen-\textsc{ptcp} or completely abandon.\textsc{ptcp} become-\textsc{inf}\\
\glt The requirement to wear a [face] mask be loosened or abandoned altogether before Easter.' (found online, glosses added)\footnote{\url{https://www.mallorcazeitung.es/gesundheit/2022/04/04/maskenpflicht-innenraumen-spanien-ostern-64628568.html} (06 April 2022).}

\ex\label{exAppendixGermanContinuativeTimeFrame6}
\gll \textbf{Noch} a-m Vor-mittag ist Lydia, wie von allen erwarte-t, abgereist.\\
still at-\textsc{def}.\textsc{dat}.\textsc{sg}.\textsc{m} pre-noon(\textsc{m}) \textsc{cop}.3\textsc{sg} L. as from all.\textsc{dat} expect-\textsc{ptcp} leave.\textsc{ptcp}\\
\glt \lq Lydia left when it was still morning, as had been expected by everyone.' \parencite[27 fn10]{Beck2020}
\end{exe}

\paragraph{Time-scalar additive (\lq as late as\rq)}\label{appendixGermanTimeScalar}
\begin{itemize}
	\item \Textcite{vanderAuwera1993}, \citeauthor{Beck2016} (\citeyear{Beck2016}, \citeyear{Beck2019}, \citeyear{Beck2020}), \textcite[s.v. \textit{noch}]{DWDS}, \textcite[s.v. \textit{noch}]{Duden}, \textcite{HoepelmanRohrer1981}, \textcite{Klein2018}, \citeauthor{Koenig1977} (\citeyear{Koenig1977}, \citeyear{Koenig1979}, \citeyear[ch. 7] {Koenig1991}),  \textcite[177]{KoenigEtAl1993}, \textcite[623–624]{MetrichFaucher2009} and \textcite{Shetter1966}.
	\item In this function, \textit{noch} consistently relates the focus to lower alternatives on a scale of time proper (\lq as late as\rq{}). Times may be mediated by metonymy (\ref{exAppendixGermanTimeScalar4}, \ref{exAppendixGermanTimeScalar5}). \textcite{Shetter1966} and \textcite{Koenig1979} point out that with reference to the past and present, this normally gives rise to a reading of recency. As \textcite{Koenig1979} points out, with future reference the opposite occurs, as in (\ref{exAppendixGermanTimeScalar6}).
	\item Similar to \textcite{MosegaardHansen2008} on \ili{French} \textit{encore} (\appref{appendixFrenchEncoreTimeScalar}), \textcite{Shetter1966} observes a functional overlap between this use and \textsc{still} when the adverbial in question denotes a point in time. However, as \textcite{Beck2020} points out, even with an imperfective vantage point, time-scalar additive \textit{noch} does not require that the situation held before; see (\ref{exAppendixGermanTimeScalar3}). What is more, time-scalar additive \textit{noch} is perfectly compatible with a perfective viewpoint (\ref{exAppendixGermanTimeScalar1}).
	\item Syntax: can form constituent with its focus. Note, for instance, them occupying the forefield of V2 clauses together in (\ref{exAppendixGermanTimeScalar1}–\ref{exAppendixGermanTimeScalar6}).
\end{itemize}
\largerpage[-1]\pagebreak

\begin{exe}
	\ex\label{exAppendixGermanTimeScalar1}
	Context: About perpetual conflicts in the Aegean islands.\\
	\gll \textbf{Noch} letzt-e Woche kam es … zu … heftig-en Auseinandersetzung-en zwischen der Polizei und den Insel-bewohner-n.\\
	still last-\textsc{nom}.\textsc{sg}.\textsc{f} week(\textsc{f}) come.\textsc{pst}.3\textsc{sg} 3\textsc{sg}.\textsc{n} {} to {} severe-\textsc{dat}.\textsc{pl} conflict-\textsc{pl} between \textsc{def}.\textsc{dat}.\textsc{sg}.\textsc{f} police(\textsc{f}) and \textsc{def}.\textsc{dat}.\textsc{pl} island-inhabitant-\textsc{dat}.\textsc{pl}\\
	\glt \lq As late as last week, altercations between the police and the islanders occurred.\rq{ }(found online, glosses added)\footnote{\url{https://www.nzz.ch/international/die-harte-linie-an-der-grenze-ist-in-griechenland-populaer-ld.1544107} (28 November, 2022).}
	
	\ex\label{exAppendixGermanTimeScalar2}
	\gll \textbf{Noch} vor hundert Jahr-en leb-ten annähernd drei Viertel der Bevökerung in der Land-wirtschaft.\\
	still before hundred year-\textsc{dat}.\textsc{pl} live-\textsc{pst}.3\textsc{pl} nearly three quarter \textsc{def}.\textsc{gen}.\textsc{sg}.\textsc{f} population(\textsc{f}) in \textsc{def}.\textsc{dat}.\textsc{sg}.\textsc{f} land-economy(\textsc{f})\\
	\glt \lq As recently as a hundred years ago, nearly three quarters of the population lived on farms.' (\textit{Rheinischer Merkur}, cited in \cite[51]{Shetter1966}, glosses added)
	
	\ex Context: we had a Condo in Danbury between March and November 1997.\label{exAppendixGermanTimeScalar3}
	\begin{xlist}
	\exi{A:} \textit{Wie lange haben wir eigentlich in Mt. Kisco gewohnt?}\\
\lq For how long did we live in Mt. Kisco?'
	\exi{B:} \textit{So lang kann das nicht gewesen sein.}\\
\lq It can't have been that long.'
	\exi{}\gll \textbf{Noch} 1997 hab-en wir ja in Danbury ge-wohn-t.\\
still 1997 have-1\textsc{pl} 1\textsc{pl} \textsc{dm} in D. \textsc{ptcp}-reside-\textsc{ptcp}\\
	\glt \lq As recently as 1997 we lived/were living in Danbury.\rq{ }\parencite[30 fn13]{Beck2020}
	\end{xlist}
	
	\ex\label{exAppendixGermanTimeScalar4}
	\gll \textbf{Noch} in Köln lief der Motor einwand-frei.\\
	still in Cologne run.\textsc{pst}.3\textsc{sg} \textsc{def}.\textsc{nom}.\textsc{sg}.\textsc{m} Motor(\textsc{m}) objection-free\\
	\glt \lq As recently as [when we were] in Cologne the Motor was running without any problems.\rq{ }(\cite[s.v. \textit{noch}]{Duden}, glosses added)
	\largerpage
	\ex\label{exAppendixGermanTimeScalar5}
	\gll \textbf{Noch} Lessing und Adelung schreib-en \textit{hamtückisch}.\\
	still L. and A. write-3\textsc{pl} hamtückisch\\
	\glt \lq (Authors as recent as) Lessing and Adelung use the spelling \textit{hamtückisch}.' (Kluge, \textit{Etymologisches Wörterbuch}, cited in \cite[52]{Shetter1966}, glosses added) 

	\ex\label{exAppendixGermanTimeScalar6}
	\gll \textbf{Noch} in zehn Jahr-en werd-en wir die Früchte dieser Entscheidung genieß-en.\\
	still in ten year-\textsc{dat}.\textsc{pl} \textsc{fut}.\textsc{aux}-1\textsc{pl} 1\textsc{pl} \textsc{def}.\textsc{acc}.\textsc{pl} fruit.\textsc{pl} \textsc{prox}:\textsc{gen}.\textsc{sg}.\textsc{f} decision(\textsc{f}) enjoy-\textsc{inf}\\
	\glt \lq Even in ten years time we will still be enjoying the fruits of this decision.' \parencite[182]{Koenig1979}
\end{exe}


\paragraph{\textit{Noch} and the source of persistent states}\label{appendixGermanSourcePersistentState}
\begin{itemize}
	\item \textcite{Mustajoki1988} and \textcite{Shetter1966}.
	\item There is a recurrent usage pattern that consists of \textit{noch} in combination with an expression indicating the source of some state, as in (\ref{exAppendixGermanSecondaryTemporalUses4}– \ref{exAppendixGermanSecondaryTemporalUses6}). As observed by \textcite{Shetter1966}, \textit{noch} can be said to serve a dual function here. On the one hand, it signals the persistence of the state in question, while at the same time it associates with the temporal predication about its source (\lq from while … still …\rq{}). Together this yields a reading strikingly similar to the \lq as far removed as\rq{ }type of time-scalar additive (\Cref{sectionTimeScalar}).
	\item Occasionally, in this pattern \textit{noch} is attested as a syntactic sister to the adpositional phrase, as in (\ref{exAppendixGermanSecondaryTemporalUses6}). This type of adverbial modification on the phrase level is, however, not restricted to \textit{noch}, nor to the larger set of phasal polarity expressions (see \cite[2091–2096]{ZifonunEtAl1997}).	
\end{itemize}
\largerpage[2.25]
\begin{exe}	\ex\label{exAppendixGermanSecondaryTemporalUses4}
	\textit{Ich habe eine natürliche Scheu vor Polizisten und Staabsoffizieren.}\\
	 \lq I have a natural shyness of policemen and staff officers.\rq
	 \exi{} \gll Das stamm-t \textbf{noch} aus meiner Militärzeit.\\
	 3\textsc{sg}.\textsc{n} stem-3\textsc{sg} still from \textsc{poss}.1\textsc{sg}:\textsc{dat}.\textsc{sg}.\textsc{f} military-time(\textsc{f})\\
	\glt \lq Goes back (all the way) to my army days.' (Remarque, \textit{Drei Kameraden}, cited in \cite[466]{Shetter1966}, glosses added)	
	\ex\label{exAppendixGermanSecondaryTemporalUses5}
	\gll Er wusste ja \textbf{noch} vo-m erst-en Tag-e seines neu-en Leben-s her, dass der Vater ihm gegenüber nur die grösst-e Strenge für angebracht hielt.\\
	3\textsc{sg}.\textsc{m} know.\textsc{pst}.3\textsc{sg} yes still from-\textsc{def}.\textsc{dat}.\textsc{sg}.\textsc{m} first-\textsc{dat}.\textsc{sg}.\textsc{sg}.\textsc{m} day(\textsc{m})-\textsc{dat} \textsc{poss}.3\textsc{sg}.\textsc{m}:\textsc{gen}.\textsc{sg}.\textsc{n} new-\textsc{gen}.\textsc{sg}.\textsc{n} life(\textsc{n})-\textsc{gen} hither \textsc{comp} \textsc{def}.\textsc{nom}.\textsc{sg}.\textsc{m} father(\textsc{m}) 3\textsc{sg}.\textsc{dat}.\textsc{m} in\_front\_of only \textsc{def}.\textsc{acc}.\textsc{sg}.\textsc{f} great.\textsc{sup}-\textsc{acc}.\textsc{sg}.\textsc{f} strictness(\textsc{f}) for appropriate consider.\textsc{pst}.3\textsc{sg}\\
	\glt \lq For he still knew from the first day of his new life that, as far as he was concerned, his father believed only the strictest measures to be appropriate.\rq{ }(Kafka, \textit{Die Verwandlung}, cited in \cite[47]{Shetter1966}, glosses added)	

	\ex\label{exAppendixGermanSecondaryTemporalUses6}
Context: From the review of a Dutch harbour-cum-campsite.\\
	\textit{In erster Linie ist es ein kleiner Hafen, für kleinere Segel- und Motorschiffe.}\\
	\lq It's primarily a small harbour for smaller sailing and motor boats.\rq
	\exi{}\gll \textbf{Noch} aus meiner Kind-heit kenn-e ich den Hafen, wir hab-en in den Jahr-en 1980 – 1984 viel Zeit mit meinen Elter-n hier verbracht.\\
	still from \textsc{poss}.1\textsc{sg}:\textsc{dat}.\textsc{sg}.\textsc{f} child-hood(\textsc{f}) know-1\textsc{sg} 1\textsc{sg} \textsc{def}.\textsc{acc}.\textsc{sg}.\textsc{n} harbour(\textsc{n}) 1\textsc{pl} have-1\textsc{pl} in \textsc{def}.\textsc{dat}.\textsc{pl} year-\textsc{dat}.\textsc{pl} 1980 {} 1984 much time with \textsc{poss}.1\textsc{sg}:\textsc{dat}.\textsc{pl} parent-\textsc{pl} here spend.\textsc{ptcp}\\
	\glt \lq Since way back in my childhood I've known this harbour, my parents and me spent a lot of time here between 1980 and 1984.\rq{ }(found online, glosses added)\footnote{\url{https://happycamping.info/camping-de-koevoet-bekommt-besuch/} (08 January, 2023).}
\end{exe}


\subsubsection{Marginality}\largerpage[2]
\label{appendixGermanMarginal}
\begin{itemize}
	\sloppy
	\item \citeauthor{Beck2016} (\citeyear{Beck2016}, \citeyear{Beck2020}), \textcite[s.v. \textit{noch}]{DWDS}, \textcite[s.v. \textit{noch}]{Duden}, \textcite{Eckardt2006}, \textcite{HoepelmanRohrer1981}, \textcite{Klein2018}, \citeauthor{Koenig1977} (\citeyear{Koenig1977}, \citeyear[151–155]{Koenig1991}), \textcite[178–179]{KoenigEtAl1993} \citeauthor{Loebner1989} (\citeyear{Loebner1989}, \citeyear{Loebner1999}), 	\textcite{Shetter1966} and \textcite{Umbach2009}.
	\item In comparisons, such as in (\ref{exAppendixGermanMarginal4}), this can lead to derogative readings.
	\item Marginal \textit{noch} often combines with expressions like \textit{eben} \lq just, precisely', (\textit{so}) \textit{gerade} \lq just barely', as in (\ref{exAppendixGermanMarginal5}).
	\item Syntax: cannot stand in forefield position together with the entity in question, indicating they do not form one constituent (cf. \cite[151]{Koenig1991}).
\end{itemize}
\begin{exe}
	\ex\label{exAppendixGermanMarginal1}
	Context: About working at a certain employer. \\
	\gll Die Raum-situation ist nicht ideal, ich find-e Groß-raum-büro-s nicht so gut wie klein-ere Büro-s; es ist aber \textbf{noch} in Ordnung.\\
	\textsc{def}.\textsc{nom}.\textsc{sg}.\textsc{f} space-situation(\textsc{f}) \textsc{cop}.3\textsc{sg} \textsc{neg} ideal 1\textsc{sg} find-1\textsc{sg} big-space-office-\textsc{pl} \textsc{neg} so good like small-\textsc{cmpr}.\textsc{acc}.\textsc{pl} office-\textsc{pl} 3\textsc{sg}.\textsc{n} \textsc{cop}.3\textsc{sg} however still in order\\
	\glt \lq The working situation is not ideal, I don't like open-plan office spaces as much as smaller offices; but it's still OK.' (found online, glosses added)\footnote{\url{https://www.kununu.com/de/smartray/bewertung/9cdecae8-9a66-476f-8d5e-7ea1b05f2b43} (06 April 2022).}

	\ex\label{exAppendixGermanMarginal2}
	Context: About removing furniture from the room.\\
	\gll Nun, den Kasten konnte Gregor i-m Not-fall \textbf{noch} entbehr-en, aber schon der Schreib-tisch muss-te bleib-en.\\
	now \textsc{def}.\textsc{acc}.\textsc{sg}.\textsc{m} chest\_of\_drawers(\textsc{m}) can.\textsc{pst}.3\textsc{sg} G. in-\textsc{def}.\textsc{dat}.\textsc{sg}.\textsc{m} emergency-case(\textsc{m}) still do\_without-\textsc{inf} but already \textsc{def}.\textsc{nom}.\textsc{sg}.\textsc{m} write-table(\textsc{m}) must-\textsc{pst}.3\textsc{sg} stay-\textsc{inf}\\
	\glt \lq Now, Gregor could still do without the chest of drawers, if necessary, but the writing desk had to stay.' (Kafka, \textit{Die Verwandlung}, cited in \cite[546]{Shetter1966}, glosses added)
	
	\ex\label{exAppendixGermanMarginal3}
Context: About the effects of a torrential rainfall.\\
	\gll I-m Vergleich zu den Bild-ern aus ander-en Teil-en der Republik hat es Kamen \textbf{noch} glimpflich getroffen.\\
	in-\textsc{def}-\textsc{dat}.\textsc{sg}.\textsc{m} comparison(\textsc{m}) to \textsc{def}.\textsc{dat}.\textsc{pl} image-\textsc{dat}.\textsc{pl} from other-\textsc{dat}.\textsc{pl} part-\textsc{dat}.\textsc{pl} \textsc{def}.\textsc{gen}.\textsc{sg}.\textsc{f} republic(\textsc{f}) have 3\textsc{sg}.\textsc{n} K. still mildly strike.\textsc{ptcp}\\
	\glt \lq In comparison to what you can see in images from other parts of the country, [the town of] Kamen was hit relatively mildly (lit. … still mildly).\rq{ }(found online, glosses added)\footnote{\url{https://www.stadt-kamen.de/leben-und-mehr/aktuelles/aktuelle-themen/3067-unwetter-beschaeftigt-kamener-feuerwehr-bis-in-den-fruehen-morgen} (13 February 2023).}

	\ex\label{exAppendixGermanMarginal4}
	\gll Paul ist \textbf{noch} der intelligent-est-e \textup{(}von der Familie\textup{)}.\\
	P. \textsc{cop}.3\textsc{sg} still \textsc{def}.\textsc{nom}.\textsc{sg}.\textsc{m} intelligent-\textsc{sup}-\textsc{nom}.\textsc{sg}.\textsc{m} \phantom{(}from \textsc{def}.\textsc{gen}.\textsc{sg}.\textsc{f} family(\textsc{f})\\
	\glt \lq(They are all pretty stupid. But) Paul is still the most intelligent of the family.\rq{ }\cite[190]{Koenig1977}, glosses added)

	\ex\label{exAppendixGermanMarginal5}
	\gll Osnabrück liegt \textup{(}gerade\textup{)} \textbf{noch} in Niedersachsen.\\
	O. lie.3\textsc{sg} \phantom{(}just still in Lower\_Saxony\\
	\glt \lq Osnabrück is still Lower Saxony (i.e.  it is a marginal case of being in the Lower Saxony territory).' (\cite[1843]{Umbach2012}, glosses added)
\end{exe}
\pagebreak
\subsubsection{Additive and related functions}
\subsubsubsection{Additive}\label{appendixGermanAdditive}
\begin{itemize}
	\item \citeauthor{Beck2016} (\citeyear{Beck2016}, \citeyear{Beck2020}), 	\textcite[s.v. \textit{noch}]{DWDS}, \textcite{Doherty1973}, \textcite[s.v. \textit{noch}]{Duden}, \textcite{Eckardt2006}, \textcite[186]{Helbig1994}, \textcite{HoepelmanRohrer1981}, \textcite{Klein2018}, \citeauthor{Koenig1977} (\citeyear{Koenig1977}, \citeyear[140–146]{Koenig1991}), \textcite[174–176, 177]{KoenigEtAl1993}, \textcite[626–630]{MetrichFaucher2009}, \textcite[ch. 2.2]{Nederstigt2003},
	\citeauthor{Umbach2009} (\citeyear{Umbach2009}, \citeyear{Umbach2012})  and \textcite{Shetter1966}.
			\item Stressed \textit{noch} indicates addition of an entity of the same kind (\ref{exAppendixGermanAdditive1}).
		\item It has been repeatedly pointed out (\cite{Eckardt2006}; \cite{Grubic2018}; \cite{Koenig1991}; \cite{Nederstigt2003}; \cite{Umbach2012}) that additive \textit{noch} is associated with incremental discourse (sequential events, shifts in topic situations, and the like) vis-à-vis the parallelisms invoked by additive \textit{auch}, and that the two items are consequently not always freely exchangeable. However, they can co-occur (\textit{auch noch}); see (\ref{exAppendixGermanAdditive5}). What is more, \textit{noch} does not mark temporal sequentiality, as highlighted by \textcite{Umbach2012}. For instance, \textit{dann} \lq then\rq{ } in (\ref{exAppendixGermanAdditive2}) can either refer to the order of consumption or the order of mentioning.
	\item In narrative discourse, the incremental pattern often goes together with the last in a series of events. For instance, in (\ref{exAppendixGermanFurtherTo6}) \textit{noch} anticipates the subsequent closure of the relevant section of the novel. This brings such instances markedly close to the further-to use (\appref{appendixGermanFurtherTo}).
	\item \textcite{Nederstigt2003} and \textcite{Shetter1966} point out that examples like (\ref{exAppendixGermanAdditive4}) are ambiguous between a phasal polarity and additive interpretation.
\end{itemize}
\largerpage
\begin{exe}
	\ex\label{exAppendixGermanAdditive1}
	\gll Ich trink-e \textbf{\textup{ˈ}noch} ein Bier.\\
	1\textsc{sg} drink-1\textsc{sg} still \textsc{indef}.\textsc{acc}.\textsc{sg}.\textsc{n} beer(\textsc{n})\\
	\glt \lq I will have another beer.' (\cite[143]{Koenig1991}, glosses added)

	\ex\label{exAppendixGermanAdditive2}
	\gll Otto hat ein Bier getrunken. Dann hat er \textbf{noch} einen Schnaps getrunken.\\
	O. have.3\textsc{sg} \textsc{indef}.\textsc{acc}.\textsc{sg}.\textsc{n} beer(\textsc{n}) drink.\textsc{ptcp} then have.3\textsc{sg} 3\textsc{sg}.\textsc{m} still \textsc{indef}.\textsc{acc}.\textsc{sg}.\textsc{m} schnaps drink.\textsc{ptcp}\\
	\glt \lq Otto had a beer. Then he had a schnaps in addition.\rq{ }(\cite[1850]{Umbach2012}, glosses added)
	\ex\label{exAppendixGermanAdditive4}

	Context: The speaker is assembling a wooden toy plane.\\
	\gll Also und {ach so}, und dann brauch-e ich noch eine Siebener-leiste.\\
	well and \textsc{interj} and then need-1\textsc{sg} 1\textsc{sg} still \textsc{indef}.\textsc{acc}.\textsc{sg}.\textsc{f} seven\_piece-bracket(\textsc{f})\\
	\glt \lq Well and oh yes, and then I also need/still need a 7-hole piece.\rq{ }(\cite[104]{Nederstigt2003}, glosses added)

	\ex\label{exAppendixGermanAdditive5}
	Context: In 2014, Max visited his parents for Christmas.\\
	\gll Das Jahr danach hat er auch \textbf{noch} die Eltern seiner Freundin besuch-t.\\
	\textsc{def}.\textsc{acc}.\textsc{sg} year(\textsc{n}) thereafter have.3\textsc{sg} 3\textsc{sg}.\textsc{m} also still \textsc{def}.\textsc{pl}.\textsc{acc} parents \textsc{poss}.3\textsc{sg}.\textsc{m}:\textsc{gen}.\textsc{sg}.\textsc{f} girlfriend(\textsc{f}) visit-\textsc{ptcp}\\
	\glt \lq The next year, in addition, he visited the parents of his girlfriend.\rq{ }\parencite[528]{Grubic2018}
	
		\ex\label{exAppendixGermanFurtherTo6}
	\textit{Da gab ihm der Vater von hinten einen jetzt wahrhaftig erlösenden starken Stoß, und er flog, heftig blutend, weit in sein Zimmer hinein.}\\
	\lq Then his father gave him a strong and liberating push from behind, and he scurried, bleeding heavily, far into his room.'
	\exi{}\gll Die Tür wurde \textbf{noch} mit dem Stock zugeschlagen, dann wurde es endlich still.\\
	\textsc{def}.\textsc{nom}.\textsc{sg}.\textsc{f} door(\textsc{f}) become.\textsc{pst}.3\textsc{sg} still with \textsc{def}.\textsc{dat}.\textsc{sg}.\textsc{m} cane(\textsc{m}) slam\_shut.\textsc{ptcp} then become.3\textsc{sg} 3\textsc{sg}.\textsc{n} finally quiet\\
	\glt \lq{}The door was slammed shut with the cane, and finally all was quiet.'
	
	\exi{}
	\textit{Erst in der Abend-dämmerung erwachte Gregor aus seinem schweren ohnmachtähnlichen Schlaf}.\\
\lq It was not until the dawn of evening that Gregor awoke from a deep and swoon-like sleep.' (Kafka, \textit{Die Verwandlung}, glosses added)
\end{exe}

\subsubsubsection{Further-to}\label{appendixGermanFurtherTo}
\begin{itemize}
	\sloppy
	\item \textcite{Beck2019}, \textcite[s.v. \textit{noch}]{Duden}, \textcite{Klein2018}, \textcite[176]{KoenigEtAl1993}, \textcite{Nederstigt2003}, \textcite{Shetter1966}, \textcite{Umbach2012} and \textcite{Vaelikangas1982}; also see \textcite{Vandeweghe1984} on the \ili{Dutch} cognate \textit{nog}.
	\item Further-to readings of \textit{noch} differ from purely additive ones in that they allow for an easy accommodation in out-of-the-blue utterances. However, many authors do not make this distinction. 
	\item This reading can be enforced through items like \textit{doch}, a discourse marker that indicates a revision of previous information or assumptions (see \cite{RojasEsponda} and references therein), as in (\ref{exAppendixGermanFurtherTo4}).
\end{itemize}
\begin{exe}
	\ex \gll Früher ist-'s mir immer sonderbar vorgekommen, dass die Leut', die verurteil-t sind, in der Früh \textbf{noch} ihren Kaffee trink-en und ihr Zigarr-l rauch-en.\\
in\_the\_past \textsc{cop}.3\textsc{sg}-3\textsc{sg}.\textsc{n} 1\textsc{sg}.\textsc{dat} always strange appear.\textsc{ptcp} \textsc{comp} \textsc{def}.\textsc{nom}.\textsc{pl} people \textsc{rel}.\textsc{nom}.\textsc{pl} condemn-\textsc{ptcp} \textsc{cop}.3\textsc{pl} in \textsc{def}.\textsc{dat}.\textsc{sg}.\textsc{f} morning(\textsc{f}) still \textsc{poss}.3\textsc{pl}:\textsc{acc}.\textsc{sg}.\textsc{m} coffee(\textsc{m}) drink-3\textsc{pl} and \textsc{poss}.3\textsc{pl}:\textsc{acc}.\textsc{sg}.\textsc{n} cigarre-\textsc{dim}(\textsc{n}) smoke-3\textsc{pl}\\
\glt \lq I used to find it strange that people sentenced [to death] would have coffee and a cigar in the morning [before getting executed].' (Schnitzler, \textit{Leutnant Gustl}; cited in \cite[58]{Shetter1966}, glosses added)

	\ex Context: I have just come home from soccer practice.\\
	\gll Ich dusch' \textbf{noch}. Dann gibt-'s Abend-essen.\\
	1\textsc{sg} take\_shower.1\textsc{sg} still then \textsc{exist}.3\textsc{sg}-3\textsc{sg}.\textsc{n} evening-meal(\textsc{n})\\
	\glt \lq I'm just taking a quick shower. Dinner will be just after.' \parencite[16]{Beck2019}
	
	\ex \textit{Die erste Halbzeit verlief torlos,}\\
	\lq{} the first half-time passed by without any goals,\rq{}\\
	\gll Aber in der 2. Halb-zeit schoss Bayern München \textbf{noch} zwei Tor-e.\\
	but in \textsc{def}.\textsc{dat}.\textsc{sg}.\textsc{f} 2. half-time(\textsc{f}) shoot.\textsc{pst}.3\textsc{sg} Bayern Munich still two goal-\textsc{pl}\\
	\glt \lq but in the second half-time Bayern Munich (eventually) did score twice.' (\cite[127]{HoepelmanRohrer1980}, glosses added)
	
		\ex\label{exAppendixGermanFurtherTo4}
		\gll Mein Buch ist doch \textbf{noch} an-ge-komm-en. Heute war der Post-bote wohl extra spät.\\
	\textsc{poss}.1\textsc{sg}:\textsc{nom}.\textsc{sg}.\textsc{n} book(\textsc{n}) \textsc{cop}.3\textsc{sg} \textsc{dm} still at-\textsc{ptcp}-come-\textsc{ptcp} today \textsc{cop}.\textsc{pst}.3\textsc{sg} postman apparently extra late\\
	\glt \lq My book did finally arrive. Seems the postman was just particularly late today.\rq{ }(found online, glosses added)\footnote{\url{https://www.lovelybooks.de/autor/Teresa-Kirchengast/Schwarze-Schafe-2412244804-w/leserunde/2617636226/2617642413/} (26 June, 2023).}
\end{exe}



\subsubsubsection{Scalar additive}\label{appendixGermanScalarAdditive}
\largerpage
\begin{itemize}
	\item \textcite[s.v. \textit{noch}]{DWDS}, \textcite[ch. 7]{Koenig1991} and \textcite{PersohnSchonNoch}.
	\item As I discuss in \textcite{PersohnSchonNoch}, \textit{noch} as a scalar additive works on a scalar model of sufficiency. It requires a high focus value, i.e. it constitutes a \textsc{beyond} operator in  \citeauthor{GastvanderAuwera2011}'s (\citeyear{GastvanderAuwera2011}) typology. However, this value has to be rung on a negatively defined scale, such as the degrees of \textit{in}credulity in (\ref{appendixGermanScalarAdditive3}).
	\item The denotation of the associated constituent is often undesirable (\ref{appendixGermanScalarAdditive2}, \ref{appendixGermanScalarAdditive3}).
	\item Syntax: forms a constituent with its focus.
\end{itemize}

\begin{exe}
	\ex\label{appendixGermanScalarAdditive1}
	Context: There are no more innocent things.\\
	\gll \textbf{Noch} \textup{[}\textbf{der} \textbf{Baum} \textbf{der} \textbf{blüh}-\textbf{t}\textup{]\textsubscript{\textsc{foc}}} lüg-t,\\
	still \phantom{[}\textsc{def}.\textsc{nom}.\textsc{sg}.\textsc{m} tree(\textsc{m}) \textsc{rel}:\textsc{nom}.\textsc{sg}.\textsc{m} blossom-3\textsc{sg} lie-3\textsc{sg}\\
	\glt \lq Even the blossoming tree lies\rq{}

	\sn 
\textit{in dem Augenblick, in welchem man sein Blühen ohne den Schatten des Entsetzens wahrnimmt.}\\	
	\lq the moment its bloom is perceived without the shadow of terror.'
	\\(Adorno, \textit{Minima Moralia}, glosses added)
	
	\ex\label{appendixGermanScalarAdditive2}
	\gll …einem Vater gleich, der \textbf{noch} den Sohn, der ihn mißhandel-t hat, an sein Herz zieht…\\
	\phantom{…}\textsc{indef}.\textsc{dat}.\textsc{sg}.\textsc{m} father(\textsc{m}) like \textsc{rel}:\textsc{nom}.\textsc{sg}.\textsc{m} still \textsc{def}.\textsc{acc}.\textsc{sg}.\textsc{m} son(\textsc{m}) \textsc{rel}:\textsc{nom}.\textsc{sg}.\textsc{m} 3\textsc{sg}.\textsc{acc}.\textsc{m} abuse-\textsc{ptcp} have.3\textsc{sg} at \textsc{poss}.3\textsc{sg}.\textsc{m}:\textsc{acc}.\textsc{sg}.\textsc{n} heart(\textsc{n}) pull.3\textsc{sg}\\
	\glt \lq …like a father that embraces even the son who has abused him…' 	(von le Fort, \textit{Der Papst aus dem Ghetto}, cited in \cite[s.v. \textit{noch}]{DWDS}, glosses added)
	
	\ex
	
\gll Die einen feier-n schließlich die dreist-e Selbst-ermächtigung von Trump und Co. und red-en sich beharrlich ein, dass \textbf{noch} der dümmste Tweet die Wahrheit sp[r]ech-e.\label{appendixGermanScalarAdditive3}\\
\textsc{def}.\textsc{nom}.\textsc{pl} \textsc{indef}.\textsc{nom}.\textsc{pl} celebrate-\textsc{pl} finally \textsc{def}.\textsc{acc}.\textsc{sg}.\textsc{f} bold-\textsc{acc}.\textsc{sg}.\textsc{f} self-empowerment(\textsc{f}) of T. and company and talk-3\textsc{pl} \textsc{refl}.3 persistently in \textsc{comp} still \textsc{def}.\textsc{nom}.\textsc{sg}.\textsc{m} stupid.\textsc{sup}:\textsc{nom}.\textsc{sg}.\textsc{m} tweet(\textsc{m}) \textsc{def}.\textsc{nom}.\textsc{sg}.\textsc{m} truth(\textsc{f}) speak-\textsc{sbjv}.3\textsc{sg}\\
\glt \lq Lastly, some celebrate Trump and company's bold self-authorisation and talk themselves into believing that even the most stupid tweet speaks the truth.' (found online, glosses added)\footnote{\url{https://taz.de/Zeitalter-der-Desinformation/!5693636&SuchRahmen=Print/} (20 June, 2022).} \end{exe}

\subsubsubsection{Comparisons of inequality}\label{appendixGermanComparisons}
\largerpage
\begin{itemize}
	\item	\textcite[s.v. \textit{noch}]{DWDS}, \textcite[s.v. \textit{noch}]{Duden}, \textcite[187]{Helbig1994}, \textcite{Klein2018}, \citeauthor{Koenig1977} (\citeyear{Koenig1977}, \citeyear[145]{Koenig1991}), \textcite[174]{KoenigEtAl1993}, \textcite[631–633]{MetrichFaucher2009}, \citeauthor{Umbach2009} (\citeyear{Umbach2009}, \citeyear{Umbach2012}) and \textcite{Shetter1966}.
	\item \textit{Noch} adds the scalar additive notion of \lq even\rq{ }to comparisons of inequality.
	\item  \textcite{Umbach2009} points out that stress may either fall on \textit{noch} or on the property-denoting predicate, thus linking \textit{noch} in comparisons to its additive function (\appref{appendixGermanAdditive}); a similar observation is found in \textcite[631]{MetrichFaucher2009}.
\item This use extends to predicates with the meaning of \lq surpass' (\ref{exAppendixGermanComparative3}, \ref{exAppendixGermanComparative4}) and degree achievements derived from comparative adjectives (\ref{exAppendixGermanComparative5}).
\item Syntax: forms a constituent with its focus.
\end{itemize}
\largerpage
\begin{exe}
	\ex
	\gll Der Betätigung-s-bereich für Physiotherapeut-en ist i-m letzt-en Jahrzehnt größer geworden; er könnte aber durchaus \textbf{noch} größer sein.\\
\textsc{def}.\textsc{nom}.\textsc{sg}.\textsc{m} activity-\textsc{lnk}-range(\textsc{m}) for physical\_therapist-\textsc{pl} \textsc{cop}.3\textsc{sg} in-\textsc{def}.\textsc{dat}.\textsc{sg}.\textsc{n} last-\textsc{dat}.\textsc{sg}.\textsc{n} decade(\textsc{n}) bigger become.\textsc{ptcp} 3\textsc{sg}.\textsc{m} can.\textsc{cond}.3\textsc{sg} but definitely still bigger \textsc{cop}.\textsc{inf}\\
	\glt \lq The range of activities of physiotherapists increased in the last decade, but it could be still larger.' (online example, cited in \cite[544]{Umbach2009})

	\ex\textit{Konsumenten bekommen derzeit richtig Druck: Die hohe Inflation frisst ihr Erspartes auf. Und die Lieferketten-Probleme führen zu Lieferstaus.}\\
\lq Consumers are under pressure: the high inflation rates eat up their savings. And supply chain problems lead to limited supplies.\rq
	\exi{}\gll Das mach-t War-en \textbf{noch} teurer.\\
3\textsc{sg}.\textsc{n} make-3\textsc{sg} good-\textsc{pl} still expensive.\textsc{cmpr}\\
	\glt \lq This leads to prices becoming even higher.\rq{ }(found online, glosses added)\footnote{\url{https://www.focus.de/finanzen/news/nicht-nur-elektroartikel-inflation-schlaegt-zu-diese-waren-werden-jetzt-schnell-teurer_id_48057964.html} (24 March, 2023).}
	\ex\label{exAppendixGermanComparative3}
	\gll Die römisch-e Kunst geht in der Personifikation der Natur-erscheinung-en \textbf{noch} über die griechisch-e hinaus.\\
	\textsc{def}.\textsc{nom}.\textsc{sg}.\textsc{f} roman-\textsc{nom}.\textsc{sg}.\textsc{f} art(\textsc{f}) go.3\textsc{sg} in \textsc{def}.\textsc{dat}.\textsc{sg}.\textsc{n} personification(\textsc{n}) \textsc{def}.\textsc{gen}.\textsc{pl} nature-appearence-\textsc{pl} still over \textsc{def}.\textsc{acc}.\textsc{sg}.\textsc{f} Greek-\textsc{acc}.\textsc{sg}.\textsc{f} beyond\\
	\glt \lq In its personification of natural phenomena, Roman art even goes beyond its Greek counterpart.' (Struck, \textit{Bedeutungslehre}, cited in \cite[62]{Shetter1966}, glosses added)
	
\ex\label{exAppendixGermanComparative4}
	 \gll Das Englisch-e weist ein Reichtum an kurz-en, … einsilbig-en Wörtern auf, der wohl nur i-m Chinesisch-en \textbf{noch} übertroffen wird.\\
	\textsc{def}.\textsc{nom}.\textsc{sg}.\textsc{n} English-\textsc{nom}.\textsc{sg}.\textsc{n} show.3\textsc{sg} \textsc{indef}.\textsc{acc}.\textsc{sg}.\textsc{m} wealth(\textsc{m}) at short-\textsc{dat}.\textsc{pl} {} monosyllabic-\textsc{dat}.\textsc{pl} word.\textsc{acc}.\textsc{pl} out \textsc{rel}.\textsc{acc}.\textsc{sg}.\textsc{m} perhaps only in-\textsc{def}.\textsc{dat}.\textsc{sg}.\textsc{n} Chinese-\textsc{dat}.\textsc{sg}.\textsc{n} still surpass.\textsc{ptcp} become.3\textsc{sg}\\
	\glt \lq The English language possesses a wealth of short, monosylabic words, that is surpassed perhaps only in Chinese.' (Leisei, \textit{Das heutige Englisch}, cited in \cite[62]{Shetter1966}, glosses added)

\ex\label{exAppendixGermanComparative5}
\textit{RTL II setzte auf eine neue Folge von «Armes Deutschland», die diesmal nicht vom ursprünglichen Produzenten Good Times, sondern von Odeon kam. 1,64 Millionen Menschen ab drei Jahren schauten zu, das entsprach sehr guten 9,4 Prozent bei den Umworbenen.}\\
\lq [The TV channel] RTL II opted for new episodes of the show \lq\lq Armes Deutschland", created not by the original production company Good Times, but by [the company] Odeon. 1.64 million viewers above the age of three years watched, yielding a very good 9.4 percent of the target audience.'
\exi{}\gll Eine alt-e Folge der Reihe ab 22.15 Uhr verbesser-te das Ergebnis \textbf{noch} auf fantastisch-e 9,9 Prozent.\\
\textsc{indef}.\textsc{nom}.\textsc{sg}.\textsc{f} old-\textsc{nom}.\textsc{sg}.\textsc{f} episode(\textsc{f}) \textsc{def}.\textsc{gen}.\textsc{sg}.\textsc{f} series(\textsc{f}) from 22.15 o'clock improve-\textsc{pst}.3\textsc{sg} \textsc{def}.\textsc{acc}.\textsc{sg}.\textsc{n} result(\textsc{n}) still up fantastic-\textsc{acc}.\textsc{pl} 9.9 percent\\
\glt \lq{}An old episode of the show aired at 10:15 p.m. improved the result even further, to a fantastic 9.9 percent.' (found online, glosses added)\footnote{\url{https://www.fernsehserien.de/armes-deutschland-stempeln-oder-abrackern/folgen/9x01-folge-48-1440341}, (11 April, 2022).}
\end{exe}

\subsubsubsection{Non-temporal connective \lq what is more\rq{}: \textit{mehr noch}}\label{appendixGermanWhatIsMore}
\begin{itemize}
	\item \textcite[631–632]{MetrichFaucher2009}.
	\item Form: in the fixed collocation \textit{mehr noch}.
	\item In this function, \textit{mehr noch} serves as an argumentative connective \lq what is more\rq{}. This use is obviously derived from \textit{noch} in comparisons of inequality (\appref{appendixGermanComparisons}), with a straightforward mapping from the propositional to the textual domain.
\end{itemize}

\begin{exe}
	\ex \textit{Europa hat nicht die Macht, selbständige Weltpolitik zu treiben, weder Frankreich noch England und auch nicht ein Bund aller europäischen Staaten}.\\
	\lq Europe does not have the necessary power to autonomously persue global politics, neither France nor England nor a union of all European states.\rq{}
	\exi{}\gll  \textbf{Mehr} \textbf{noch}: das Fort-besteh-en Europa-s ist nur dank Amerika möglich.\\
	more still \textsc{def}.\textsc{nom}.\textsc{sg}.\textsc{n} further-exist-\textsc{inf} Europe-\textsc{gen} \textsc{cop}.3\textsc{sg} only thanks\_to America possible\\
	\glt \lq What is more: the continued existence of Europe is only possible thanks to America.\rq{ }(\cite[631]{MetrichFaucher2009}, glosses added)
\end{exe}
\begin{exe}
	\ex \textit{Zehntausend, hunderttausend spezifische Eigengerüche hatte er gesammelt und hielt sie zu seiner Verfügung, so deutlich, so beliebig, daß er sich nicht nur ihrer erinnerte, wenn er sie wiederroch, sondern daß er sie tatsächlich roch, wenn er sich ihrer wiedererinnerte};\\
	\lq He had collected tens of thousand, hundreds of thousand smells and kept them available to himself, so clearly, so vividly, that he not only remembered them when he smelled them for a second time, but actually smelled them when he merely remembered them.\rq{}
	\exi{} \gll Ja, \textbf{mehr} \textbf{noch}, daß er sie sogar in seiner bloß-en Phantasie untereinander neu zu kombinier-en verstand.\\
	yes more still \textsc{comp} 3\textsc{sg}.\textsc{m} \textsc{3pl}.\textsc{acc} even in \textsc{poss}.3\textsc{sg}.\textsc{m}:\textsc{dat}.\textsc{sg}.\textsc{f} mere-\textsc{dat}.\textsc{sg}.\textsc{f} imagination(\textsc{f)} among\_each\_other new to combine-\textsc{inf} know.\textsc{pst}.3\textsc{sg}\\
	\glt \lq What is more, he was even capable of combining them together into new smells using his mere imagination.\rq{ }(Süskind, \textit{Das Perfum}, glosses added)
\end{exe}


\subsubsection{Broadly modal and interactional functions}
\largerpage
\subsubsubsection{Universal concessive conditionals: \textit{noch so}}\label{appendixGermanNochSo}
\begin{itemize}
	\item \textcite[s.v. \textit{noch}]{DWDS}, \textcite[s.v. \textit{noch}]{Duden}, \textcite{HaspelmathKoenig1998}, \textcite[186–187]{Helbig1994}, \textcite[174]{KoenigEtAl1993}, \textcite[634–636]{MetrichFaucher2009} and \textcite{Shetter1966}.
	\item Form: in collocation with \textit{so} \lq so, thus, that much.'
	\item In this function \textit{noch} is stressed, which links it to the additive use with increments of the same kind (\appref{appendixGermanAdditive}). A link to additivity has also been proposed by \textcite{HaspelmathKoenig1998}, \textcite{MetrichFaucher2009} and \textcite{Shetter1966}. The second item of the collocation, \textit{so} has among its functions that of signalling a high, potentially infinite degree of some property (e.g. \cite[s.v. \textit{so}]{DWDS}). That is, the universal quantification effect can be understood as going back to \lq with any additional degree conceivable\rq{}. In diachronic terms, according to \textcite{DWDS}, this collocation started out as \textit{noch einmal so} \lq by an additional same degree\rq{}.

\end{itemize}
\begin{exe}
	\ex\label{exAppendixGermanNochSo1}
	\gll Du kann-st \textbf{noch} \textbf{so} \textup{(}sehr\textup{)} bitt-en, es wird dir nichts nütz-en.\\
	2\textsc{sg} can-2\textsc{sg} still so \phantom{(}very ask-\textsc{inf} 3\textsc{sg}.\textsc{n} \textsc{fut}.\textsc{aux}:3\textsc{sg} 2\textsc{sg}.\textsc{dat} nothing be\_of\_use-\textsc{inf}\\
	\glt \lq You can beg as much as you like, it won't help.\rq{ }(\cite[s.v. \textit{noch}]{Duden}, glosses added)

	\ex\label{exAppendixGermanNochSo2}
	\gll Er nutz-te jeden \textbf{noch} \textbf{so} klein-en Vorteil aus.\\
	3\textsc{sg} use-\textsc{pst}.3\textsc{sg} every.\textsc{acc}.\textsc{sg}.\textsc{m} still so small-\textsc{acc}.\textsc{sg}.\textsc{m} advantage(\textsc{m}) out\\
	\glt \lq He took advantage of every opportunity, no matter how small.' (\cite[s.v. \textit{noch}]{DWDS}, glosses added)

	\ex\label{exAppendixGermanNochSo3}
	 \gll Sie bekam nur strahlend-e, glücklich-e Aug-en, und damit sag-te sie mehr als mit \textbf{noch} \textbf{so} viel-en Wort-en.\\
	3\textsc{sg}.\textsc{f} got.3\textsc{sg} only shining-\textsc{acc}.\textsc{pl} happy-\textsc{acc}.\textsc{pl} eye-\textsc{pl} and thereby say-\textsc{pst}.3\textsc{sg} 3\textsc{sg}.\textsc{f} more than with still so many-\textsc{dat}.\textsc{pl} word-\textsc{dat}.\textsc{pl}\\
	\glt \lq Her eyes became only radiant and happy, and thereby she said more than any amount of words could tell.' (Remarque, \textit{Drei Kameraden}, cited in \cite[62]{Shetter1966}, glosses added)
\end{exe}

\subsubsubsection{\lq Remind me again\rq{ }questions: noch/noch (ein)mal}
\label{appendixGermanRemindMe}
\begin{itemize}
	\item \textcite[s.v. \textit{noch}]{DWDS}, \textcite{Iwasaki1977}, \textcite[176]{KoenigEtAl1993}, \textcite[634]{MetrichFaucher2009} and \textcite{Sauerland2009}.
	\item Form: both in the iterative-via-addition (\appref{appendixGermanIterativeViaIncrement}) collocation \textit{noch} (\textit{ein})\textit{mal} (\ref{exAppendixGermanRemindMe2}) and as plain \textit{noch} (\ref{exAppendixGermanRemindMe1}). The latter is particularly common in monological question.
	\item For plain \textit{noch}, \textcite{Iwasaki1977} discusses how this use shows traces of phasal polarity in that it signals the speaker's belief that they should still be able to recall the information in question, although they no longer do.
\end{itemize}

\begin{exe}
	\ex\label{exAppendixGermanRemindMe1}
	Context: A waiter has forgotten the orders of each person.\\
	\gll Was hat \textbf{noch}-\textbf{mal} jeder bestell-t?\\
	what have.3\textsc{sg} still-time everyone order-\textsc{ptcp}\\
	\glt \lq Remind me again, what did each of you order?' \parencite[63]{Sauerland2009}	
	
	\ex\label{exAppendixGermanRemindMe2}
	\textit{…dann pflegte der Onkel das Lied von den heimatlosen Matrosen durchs Haus zu schmettern. Er hatte es einst auf seinen Seereisen gelernt.}\\
	\lq … the uncle then used to belt out the song of the homeless seafarers. He had learned it a long time ago during his sea voyages.\rq\\
	\gll Wie hieß \textit{noch} die zweit-e Strofe?\\
	how be\_called.\textsc{pst}.3\textsc{sg} still \textsc{def}.\textsc{nom}.\textsc{sg}.\textsc{f} second-\textsc{nom}.\textsc{sg}.\textsc{f} verse(\textsc{f})\\
	\glt \lq [Asking himself:] How did the second verse go?\rq{ }(von der Vring, \textit{Spur im Hafen}, cited in \cite[63]{Iwasaki1977}, glosses added) 
	

\end{exe}


\subsubsubsection{Lamenting exclamations}\label{appendixGermanExclamations}
\begin{itemize}
	\item \textcite[s.v. \textit{noch}]{Duden} and \textcite[633–634]{MetrichFaucher2009}.
	\item Form: in exclamations with (declarative) V2 order and a contrastive (i.e. focussed) topic.
	\item In this use, \textit{noch} adds emphasis and emotive notions such as melancholy or regret about the perceived loss of some quality. \textcite[s.v. \textit{noch}]{Duden} points out that this use also invites agreement on side of the hearer. As observed by \textcite{MetrichFaucher2009}, the phasal polarity origins of this use are clearly observable in some cases, e.g. in (\ref{exAppendixGermanNochZeiten1}), whereas a shift towards emotive functions dominates in others, such as in (\ref{exAppendixGermanNochZeiten2}, \ref{exAppendixGermanNochZeiten3}). However, even such instances latently evoke the marginality use of \textit{noch} (e.g. a last outpost of reliability of quality in ex. \ref{exAppendixGermanNochZeiten2}).
\end{itemize}

\begin{exe}

	\ex\label{exAppendixGermanNochZeiten1}
	\gll Das war-en \textbf{noch} Zeit-en!\\
	3\textsc{sg}.\textsc{n} \textsc{cop}.\textsc{pst}-3\textsc{pl} still time-\textsc{pl}\\
	\glt \lq Those were the days!\rq{ }(\cite[633]{MetrichFaucher2009}, glosses added)
	
\ex\label{exAppendixGermanNochZeiten2}
		\gll Auf ihn kann man sich \textbf{noch} verlass-en!\\
	on 3\textsc{sg}.\textsc{acc}.\textsc{m} can.3\textsc{sg} \textsc{impr} \textsc{refl}.3 still rely-\textsc{inf}\\
	\glt \lq At least he is reliable (if no one else)!\rq{ }(\cite[633]{MetrichFaucher2009}, glosses added)
	
\ex\label{exAppendixGermanNochZeiten3}
	\gll Das nenn-e ich \textbf{noch} Qualität!\\
	3\textsc{sg}.\textsc{n} call-1\textsc{sg} 1\textsc{sg} still quality\\
	\glt \lq See, that's (still) real quality!\rq{ }(\cite[633]{MetrichFaucher2009}, glosses added)
\end{exe}

\subsubsubsection{Booster: \textit{noch} (\textit{ein})\textit{mal}, \textit{nochmals}}
\label{appendixGermanIterativeDirectives}
\begin{itemize}
	\item \textcite[s.v. \textit{noch}]{DWDS}, \textcite[s.v. \textit{nochmal}, \textit{nochmals}, \textit{noch einmal}]{Duden} and \textcite[627–628]{MetrichFaucher2009}.
	\item The iterative collocation \textit{noch} (\textit{ein})\textit{mal} and its variant \textit{nochmals} (\appref{appendixGermanIterativeViaIncrement}) are used as boosters in directive speech acts, exclamations and curses.
	 \item A bridging context can be found in cases like (\ref{exAppendixGermanIterativeDirective3}), where the notion of repetition is transferred to the speech-act level, thereby reinforcing the command.
\end{itemize}

\begin{exe}

	\ex \label{exAppendixGermanIterativeDirective1}
	\gll Zieh-en Sie doch den Bauch ein, Hergott \textbf{noch} \textbf{mal}!\\
	pull-3\textsc{pl} 3\textsc{pl}(\textsc{hon}) \textsc{dm} \textsc{def}.\textsc{acc}.\textsc{sg}.\textsc{m} belly(\textsc{m}) in Dear\_Lord still time\\
	\glt \lq For God's sake, pull your belly in.\rq{ }(\cite[627]{MetrichFaucher2009}, glosses added)
	
	\ex \label{exAppendixGermanIterativeDirective2}
	\gll Jed-er Mensch ... überhaupt: jed-e Kreatur ... man muß doch an etwas glaub-en in der Welt, verdammt \textbf{noch}-\textbf{mal}!\\
	every-\textsc{nom}.\textsc{sg}.\textsc{m} human(\textsc{m}) {} generally every-\textsc{nom}.\textsc{sg}.\textsc{f} creature(\textsc{f)} {} \textsc{impr} must.3\textsc{sg} \textsc{dm} at something believe-\textsc{inf} in \textsc{def}.\textsc{dat}.\textsc{sg}.\textsc{f} world(\textsc{f}) damnit still-time\\
	\glt \lq Every person, more generally, every creature … you need something to believe in, Goddammit!\rq{ }(Sparschuh, \textit{der Zimmerspringbrunnen}, glosses added)
	
	\ex \label{exAppendixGermanIterativeDirective3}
	\gll \textbf{Noch}-\textbf{mal}: du soll-st deinen Müll nicht einfach in die Landschaft werf-en!\\
	still-time 2\textsc{sg} should-2\textsc{sg} \textsc{poss}.2\textsc{sg}:\textsc{acc}.\textsc{sg}.\textsc{m} trash(\textsc{m}) \textsc{neg} simply in \textsc{def}.\textsc{acc}.\textsc{sg}.\textsc{f} landscape(\textsc{f}) throw-\textsc{inf}\\
	\glt \lq [I'm telling you] again: don't just throw your trash into nature!' (personal knowledge)
\end{exe}
\il{German|)}

\section{Hebrew (Modern) (heb, hebr1245)}\il{Hebrew, Modern|(}
\label{appendixHebrew}

\subsection{Introductory remarks}
I am indebted to Jael Greenberg and Itamar Francez for long discussions of Hebrew data and for providing numerous examples, and to Jens G. Fischer for helping with glosses and transliteration.

\subsection{ʕadayin}
\label{appendixHebrewAdayin}

\subsubsection{General information}
	\begin{itemize}
		\item Form: variously transcribed as \textit{ʕadayin}, \textit{ʼadayin}, \textit{adayin}; עדין/עדיין in Hebrew script.
		\item Wordhood: free morpheme.
		\item Etymology: likely < \ili{Aramaic} \textit{ʔedayin} \lq then, thereupon\rq{}
	\end{itemize}


\subsubsection{As a \lq{}still\rq{ }expression}
\begin{itemize}
	\item \textcite{Tobin1985} and \textcite{TsirkinSadan2019}.
	\item Specialisation: the descriptions by \citeauthor{Tobin1985} and \citeauthor{TsirkinSadan2019} meet my definition.
	\item Polarity sensitivity: inner negation yields \textsc{not yet}.
	\item Pragmaticity: the data allow no conclusions.
	\item Syntax: typically preceding the predicate.
\end{itemize}

\begin{exe}
	\ex \gll Izevel nahag-a ke\rq{}ilu hi \textbf{ʕadayin} paħad-a.\\
	I. act.\textsc{pst}-3\textsc{sg}.\textsc{f} as\_if 3\textsc{sg}.\textsc{f} still afraid.\textsc{pst}-3\textsc{sg}.\textsc{f}\\
	\glt \lq Jezebel acted as if she were still afraid.\rq{ }(\cite[136]{Glinert1989}, glosses added)

	\ex \gll Hu yošev \textbf{ʕadayin} b-a-kele.\\
	3\textsc{sg}.\textsc{m} sit.\textsc{sg}.\textsc{m} still at-\textsc{def}-prison\\
	\glt \lq He is still sitting in prison.' (\cite[358]{AronsonBerman1978}, glosses added)
\end{exe}

\subsubsection{Uses on the fringes of \lq{}still\rq{}}
\paragraph{Scalar contexts}\label{appendixHebrewAdayinScalar}
\begin{itemize}
	\item \textit{ʕadayin} is attested in scalar contexts. These encompass decreases, as in (\ref{appendixHebrewAdayinScalar1}, \ref{appendixHebrewAdayinScalar2}) and limited increases, as in (\ref{appendixHebrewAdayinScalar3}–\ref{appendixHebrewAdayinScalar5}). Note that in the presence of an unequivocally scalar item, as in (\ref{appendixHebrewAdayinScalar5}), an \lq only\rq{ }marker is not required.
\end{itemize}

\begin{exe}
	\ex\label{appendixHebrewAdayinScalar1}
	\gll Šney	sug-ey	ha-par-ot	\textbf{ʕadayin}	nimtsa\rq{}-ot	kan.\\
	two	kind-\textsc{cs}.\textsc{pl} \textsc{def}-cow(\textsc{f})-\textsc{pl} still  find.\textsc{pass}-\textsc{pl}.\textsc{f} here\\
	\glt \lq Two types of cows still exist here.\rq{ }\parencite[95]{Glinert1989}

	\ex\label{appendixHebrewAdayinScalar2}
	\gll Lo yitaxen še-be-2015 yeš \textbf{ʕadayin} 100 elef iš ha-ħay-im llo taštit ħašmal ve-mayim.\\
	\textsc{neg} possible \textsc{subord}-at-2015 \textsc{exist} still 100 thousand person(\textsc{m}) \textsc{rel}-live-\textsc{pl}.\textsc{m} without infrastructure.\textsc{cs} electricity and-water.\\
	\glt \lq It can't be that in 2015 there are still 100,00 people living without electricity and water supplies.\rq{ }(found online, glosses added)\footnote{\url{https://13news.co.il/item/news/politics/ntr-1118164/} (01 March, 2023).}


	\ex Context: About the challenges of quitting alcohol.\label{appendixHebrewAdayinScalar3}\\
	\gll \textbf{ʕadayin} yeš l-i raq kama ħodš-ey pikaħon.\\
	still \textsc{exist} to-1\textsc{sg} only few month-\textsc{cs}.\textsc{pl} sobriety\\
	\glt \lq I still only have a few months of sobriety.\rq
	\exi{}\textit{Aval lamadeti harbe al hitroʕaʕut be-fikaħon be-mahalax ha-ħodašim ha-axaronim, hine kama tipim še-asuyim la\rq{}azor.}\\
		\lq But I've learned a lot about socializing sober during the past few months, so here are some pieces of advice that might help.\rq{ }
		\\(found online, glosses added)\footnote{\url{https://www.cobylapidot.co.il/post/\%D7\%90\%D7\%99\%D7\%9A-\%D7\%94\%D7\%97\%D7\%99\%D7\%99\%D7\%9D-\%D7\%94\%D7\%97\%D7\%91\%D7\%A8\%D7\%AA\%D7\%99\%D7\%99\%D7\%9D-\%D7\%A9\%D7\%9C\%D7\%9A-\%D7\%9E\%D7\%A9\%D7\%AA\%D7\%A0\%D7\%99\%D7\%9D-\%D7\%90\%D7\%9D-\%D7\%90\%D7\%AA\%D7\%94-\%D7\%9E\%D7\%95\%D7\%95\%D7\%AA\%D7\%A8-\%D7\%A2\%D7\%9C-\%D7\%94\%D7\%A9\%D7\%AA\%D7\%99\%D7\%99\%D7\%94} (01 March, 2023).}

	\ex Context: About technological advances.	\label{appendixHebrewAdayinScalar4}\\
	\gll Ani ħošev še-anaħnu \textbf{ʕadayin} roʼ-im raq et ktse ha-qarħon b-a-txum ha-ze.\\
	1\textsc{sg} think.\textsc{sg}.\textsc{m} \textsc{subord}-1\textsc{pl} still see-\textsc{pl}.\textsc{m} only \textsc{acc} tip.\textsc{cs} \textsc{def}-iceberg at-\textsc{def}-area(\textsc{m}) \textsc{def}-\textsc{prox}.\textsc{sg}.\textsc{m}\\
	\glt \lq I think we are still only seeing the tip of the iceberg in this domain.\rq{ }(found online, glosses added)\footnote{\url{https://www.ynet.co.il/articles/0,7340,L-4727971,00.html} (01 March, 2023).}

	\ex\label{appendixHebrewAdayinScalar5}
	 Context: We need to see at least two lines on the screen.\\
	 \gll \textbf{ʕadayin} roʼ-im qav eħad.\\
	 still see-\textsc{pl}.\textsc{m} line one\\
	 \glt \lq We're still seeing just one line.\rq{ }(Itamar Francez, p.c.)
\end{exe}


\subsubsection{Broadly adverbial temporal-aspectual functions}
\paragraph{\# \lq eventually', prospective}

\begin{itemize}
	\item \textcite[141]{Koenig1991} discusses that \textit{od} (\ref{appendixHebrewOdProspective}), but not \textit{ʕadayin} has the prospective \lq eventually\rq{ }function. 
\end{itemize}

\begin{exe}
	\ex Context: You have to catch a flight.\\
	\gll Te-maher! \#Ata \textbf{ʕadayin} te-axer.\\
	2\textsc{sg}.\textsc{m}-hurry\_up.\textsc{fut} \phantom{\#}2\textsc{sg}.\textsc{m} still 2\textsc{sg}.\textsc{m}-be\_late.\textsc{fut}\\
	\glt intended: \lq Hurry up, you’ll end up late.' (Yael Greenberg, p.c.)
	
	\ex Context: Uttering a threat.\\
	\exi{}[\#]{\gll Ata \textbf{ʕadayin} ti-r’e ma yi-qre le-xa.\\
	2\textsc{sg}.\textsc{m} still 2\textsc{sg}.\textsc{m}-see.\textsc{fut} what 3\textsc{sg}.\textsc{m}-happen.\textsc{fut} to-2\textsc{sg}.\textsc{m}\\
	\glt intended: \lq You’ll see yet what you get out of it.' (Yael Greenberg, p.c.)}
\end{exe}

\subsubsection{Marginality}
\label{appendixHebrewAdayinMarginal}
\begin{itemize}
	\item \textcite{TsirkinSadan2019}.
\end{itemize}

\begin{exe}

	\ex\label{exAppendixHebrewAdayinMarginal1}
	\begin{xlist}
		\exi{A:} I am annoyed because my rich aunt has left the better part of her fortune to an animal shelter, and only 10.000€ of inheritance go to me.
		\exi{B:}\gll 10.000 Euros \textbf{ʕadayin} sxum yafe.\\
		10.000 Euros still sum(\textsc{m}) good.\textsc{m}\\
		\glt \lq 10.000 Euro is still a good sum.' (Yael Greenberg, p.c.)
	\end{xlist}
	
	\ex\label{exAppendixHebrewAdayinMarginal2}
	Context: Talking about tennis skills.\\
	\gll Et Paul ani \textbf{ʕadayin} yaxol le-natseaħ.\\
	\textsc{acc} P. 1\textsc{sg} still can.\textsc{sg}.\textsc{m} to-beat\\
	\glt \lq Paul I can still beat (but other players might be too good for me).'
	\\(Yael Greenberg, p.c.)
	
	\ex\label{exAppendixHebrewAdayinMarginal3}
	\gll Ha\rq{}im ha-rama ha-nemux-a be-yoter ha-muter-et hi \textbf{ʕadayin} betox mirvaħ ha-tkina ha-eyropit li-vtiħut?\\
	\textsc{q} \textsc{def}-level(\textsc{f}) \textsc{def}-low-\textsc{sg}.\textsc{f} at-more \textsc{def}-allowed-\textsc{sg}.\textsc{f} 3\textsc{sg}.\textsc{f} still inside space.\textsc{cs} \textsc{def}-standardisation \textsc{def}-European for-safety?\\
	\glt \lq Is the lowest level allowed still within the European standard for safety?' (Knesset protocol, cited in \cite[104 fn 21]{TsirkinSadan2019})
	
	\ex\label{exAppendixHebrewAdayinMarginal4}
	Context: Viewed from Tel Aviv.\\
	\gll Heifa hi \textbf{ʕadayin} be-Yisrael, aval Tsur hi kvar be-Levanon.\\
	H. 3\textsc{sg}.\textsc{f} still at-Israel but Tyros 3\textsc{sg}.\textsc{f} already at-Lebanon\\
	\glt \lq Haifa is still in Israel, but Tyros is already in Lebanon.\rq{ }(Yael Greenberg, p.c.)
\end{exe}

\subsubsection{Broadly modal and interactional functions}

\paragraph{Concessive apodoses}\label{appendixHebrewAdayinConcessiveConsequent}
\begin{itemize}
	\item \textcite{TsirkinSadan2019}.
	\item Note the perfective viewpoint in (\ref{exAppendixHebrewAdayinConcessiveQ3}), which would be incompatible with the notion of \textsc{still}, as well as the negation within the scope of \textit{ʕadayin}.
	\item Syntax and prosody: can occur in clause-initial position, as in (\ref{exAppendixHebrewAdayinConcessiveQ1}), together with comma-intonation.
\end{itemize}

\begin{exe}
	\ex\label{exAppendixHebrewAdayinConcessiveQ1}
	\gll Laqaħ-ta šir ve-tirgam-ta -- beʕetsem ʕasi-ta yetsira ħadaš-a. Kazo še-nišʕen-et ʕal yetsira aħer-et, aval \textbf{ʕadayin} – sug šel yetsira ħadaš-a.\\
	take.\textsc{pst}-2\textsc{sg}.\textsc{m} poem and-translate.\textsc{pst}-2\textsc{sg}.\textsc{m} {} in\_fact make.\textsc{pst}-2\textsc{sg}.\textsc{m} creation(\textsc{f}) new-\textsc{f} like\_this \textsc{subord}-lean-\textsc{sg}.\textsc{f} on creation(\textsc{f}) other-\textsc{f} but still {} sort of creation(\textsc{f}) new-\textsc{f}\\
	\glt \lq You took a poem and you translated it -- in fact you've created a new poem. One that leans on another poem, but still -- a sort of a new poem.' \parencite[100]{TsirkinSadan2019}
	
	\ex\label{exAppendixHebrewAdayinConcessiveQ2}
	\textit{Im at rotsa ħofeš al teviʼi yeladim … Agav, gam ba-hanaqa yordim be-miškal aval ha-ħaze mištane.}\\
	\lq If you want freedom, don’t have children … By the way, you’ll also lose weight when breastfeeding, but your breasts change.\rq{}\\
	\gll \textbf{ʕadayin}, lo keday le-vater al ha-ħavaya ha-zo.\\
	still \textsc{neg} worthwhile to-give\_up on \textsc{def}-experience(\textsc{f}) \textsc{def}-\textsc{prox}.\textsc{sg}.\textsc{f}\\
	\glt \lq Still, you shouldn’t give up on this experience.\rq{ }(found online, glosses added)\footnote{\url{https://www.facebook.com/LaishaMagazine/photos/10154667930505836} (02 March, 2023).} 
	
	\ex\label{exAppendixHebrewAdayinConcessiveQ3}
	\gll Šm-a šel Livni ʕazar le-mifleget ha-ʕavoda le-matsev et ʕatsm-a b-a-toda\rq{}a ke-alternativa. Ze \textbf{ʕadayin} lo hafax ot-a le-alternativa.\\
	name(\textsc{m})-\textsc{poss}.3\textsc{sg}.\textsc{f} \textsc{subord} L. help.\textsc{pst}.3\textsc{sg}.\textsc{m} to-party(\textsc{f}) \textsc{def}-Labour to-present \textsc{acc} itself-\textsc{f} in-\textsc{def}-consciousness as-alternative \textsc{prox}.\textsc{sg}.\textsc{m} still \textsc{neg} turn\_over.\textsc{pst}.3\textsc{sg}.\textsc{m} \textsc{acc}-3\textsc{sg}.\textsc{f} to-alternative\\
	\glt \lq (Tzipi) Livni's name helped the Labor Party to present itself in the public opinion as an alternative. It still did not turn it into one.' (\textit{Yedi ot Aħoronot}, cited in \cite[100]{TsirkinSadan2019})
\end{exe}

\paragraph{Concessive interjection}
\label{appendixHebrewConcessiveInterjection}
\begin{itemize}
	\item \textit{ʕadayin} can be used in isolation, as a concessive interjection.
	\item The fact that \textit{ʕadayin} as a concessive conjunction typically occurs in initial position and is prosodically seperated from the rest of the clause (\appref{appendixHebrewAdayinConcessiveConsequent}) indicates a development parallel to that of \ili{English} \textit{still} (\appref{appendixEnglishConcessiveInterjection}).
\end{itemize}

\begin{exe}
	\ex
		\begin{xlist}
		\exi{A:}\textit{Rina lo neħmada elay.}\\
					\lq Rina is not nice to me.'
	
		\exi{B:}  \textit{Aval hi omeret lexa šalom kol boqer.}\\
		\lq But she greets you \lq\lq Hello" every morning.\rq

		\exi{A:}\textit{ \textbf{ʕadayin}…}\\
		\lq Still…' (Yael Greenberg, p.c.)
	\end{xlist}
	\ex
	\begin{xlist}
		\exi{A:} \textit{Anaħnu holxim le-aħer la-hofaʕa!}\\
		\lq We are going to be late to the show.'

		\exi{B:} \textit{Lo batuaħ. ʕaxšav ha-ramzor hofex le-yaroq.}\\
		\lq That's not certain. The traffic light is turning green now.'
	
		\exi{A:} \gll \textbf{ʕadayin…}\\
		still\\
		\glt \lq Still…' (Yael Greenberg, p.c.)
	\end{xlist}

\end{exe}

\subsection{ʕod}
\label{appendixHebrewOd}

\subsubsection{General information}
\largerpage

\begin{itemize}
	\item Form: also transcribed as \textit{od}, \textit{\lq{}\textit{od}}; עוֹד in Hebrew script.
	\item Wordhood: free morpheme. Usually invariable, but in formal registers it can take subject suffixes in the affirmative present tense, as in (\ref{exAppendixHebrewOd2}).
	\item Etymology: attested in essentially the same functions in Biblical Hebrew.\il{Hebrew, Biblical} Cognates in other Semitic varieties have iterative and restitutive functions, as well as meanings pertaining to circular motion and being accustomed (\cite[s.v. \RL{עוּד}]{BrownEtAl}).
\end{itemize}


\subsubsection{As a \lq{}still\rq{ }expression}

\begin{itemize}
	\sloppy
	\item \citeauthor{Glinert1976} (\citeyear{Glinert1976}, \citeyear[239, 532]{Glinert1989}), \textcite{Greenberg2012}, \textcite[45]{Schwarzwald2001}, \textcite{Thomas2018}, \textcite{Tobin1985} and \textcite{TsirkinSadan2019}.
	\item Specialisation: the various descriptions, when taken together, indicate that this marker is comparable to English \textit{still} or German \textit{noch}.
	\item Polarity sensitivity: inner negation yields \textsc{not yet}, outer negation yields \textsc{no longer}.
	\item Pragmaticity: the data allow no conclusions.
	\item Syntax: fixed, preceding the predicate.
\end{itemize}

\begin{exe}
	\ex
	\gll Be-šeš va-ħetsi Dani \textbf{ʕod} yašan.\\
	at-six and-half Danny still sleep.\textsc{pst}.3\textsc{sg}.\textsc{m}\\
	\glt \lq At half past six Danny was still asleep.' \parencite[151]{Greenberg2012}
	
	\ex\label{exAppendixHebrewOd2}
	\gll Sara \textbf{ʕod}-ena zoxer-et.\\
	S. still-3\textsc{sg}.\textsc{f} rember-3\textsc{sg}.\textsc{f}\\
	\glt \lq Sara still remembers.\rq{ }(\cite[532]{Glinert1989}, glosse added)
\end{exe}

\subsubsection{Uses on the fringes of \lq{}still\rq{}}
\paragraph{Scalar contexts}\label{appendixHebrewOdScalar}
\begin{itemize}
	\item \textit{ʕod} is attested in scalar contexts, encompassing both decreases  (\ref{appendixHebrewOdScalar1}, \ref{appendixHebrewOdScalar2}) and limited increases (\ref{appendixHebrewOdScalar3}–\ref{appendixHebrewOdScalar5}). Note the lack of need for an \lq only\rq{ }marker in (\ref{appendixHebrewOdScalar4}, \ref{appendixHebrewOdScalar5}). 
\end{itemize}

\begin{exe}
	\ex Context: From a mountain biking guide.\label{appendixHebrewOdScalar1}\\
\textit{Im kvar ktsat laħuts lanu ve rotsim limšox la-rexev az mumlats limšox ʕim ha-simun švilim ha-yaroq ve-ha-mitpatel…}\\
	\lq If we are already a bit exhausted and want to return to the car, then it is recommended to take  the green and winding trail …\rq{}
	\exi{}
	\gll Im \textbf{ʕod} yeš la-nu ktsat koaħ ve-lo sevaʕ-nu me-ha-nuf-im ha-adir-im kabir-im en sofiy-im še-mi-saviv la-nu  … mumlats be-yoter la-ʕalot l-a-rexes … \\
	if still \textsc{exist} to-1\textsc{pl} a\_little energy and-\textsc{neg} become\_satisfied.\textsc{pst}-1\textsc{pl} from-\textsc{def}-landscape-\textsc{pl}.\textsc{m} \textsc{def}-mighty-\textsc{pl}.\textsc{m} tremendous-\textsc{pl}.\textsc{m} \textsc{neg}.\textsc{exist} ending-\textsc{pl}.\textsc{m} \textsc{subord}-from-surrounding  to-1\textsc{pl} {} recommended at-more to-go\_up to-\textsc{def}-ridge\\
	\glt \lq If we still have a little energy [left], and we haven't had enough of the tremendous, endless views around us, then it's highly recommended to ride up the ridge …\rq{ }(found online, glosses added)\footnote{\url{https://bike.co.il/\%D7\%A7\%D7\%A8\%D7\%9F-\%D7\%90\%D7\%9C-\%D7\%97\%D7\%92\%D7\%A8-\%D7\%9E\%D7\%A1\%D7\%9C\%D7\%95\%D7\%9C-\%D7\%98\%D7\%99\%D7\%95\%D7\%9C-\%D7\%91\%D7\%A6\%D7\%A4\%D7\%95\%D7\%9F-\%D7\%9E\%D7\%93\%D7\%91\%D7\%A8-\%D7\%99\%D7\%94\%D7\%95\%D7\%93\%D7\%94/} (01 March, 2023).}

	\ex \label{appendixHebrewOdScalar2}
	\gll Le-mazali aħarey šney hamburger-im ha-yom, \textbf{ʕod} nišʼar harbe basar medamem b-a-sakit meħake le-šemen ve-maħavat.\\
to-luck after two burger-\textsc{pl}.\textsc{m} \textsc{def}-day, still remain.\textsc{sg}.\textsc{m} much meat(\textsc{m}) bleed.\textsc{sg}.\textsc{m} at-\textsc{def}-bag await.\textsc{sg}.\textsc{m} to-grease and-frying\_pan\\
\glt \lq Luckily, after two burgers today, there's still a lot of bloody meat left in the package, awaiting oil and a pan.\rq{ }(found online, glosses added)\footnote{\url{https://www.facebook.com/groups/619669744790558/posts/1234771469947046/} (01 March, 2023).}

	\ex \label{appendixHebrewOdScalar3}
	\gll ʕadayin muqdam la-daʕat et ha-tšuv-ot l-a-še\rq{}el-ot ha-mitbaqš-ot, šeken karegaʕ anaħnu \textbf{ʕod} ro\rq{}-im raq et netun-ey ha-rivʕon ha-revi\rq{}i šel ha-šana še-ʕavr-a.\\
	still early to-know \textsc{acc} \textsc{def}-answer(\textsc{f})-\textsc{pl}
to-\textsc{def}-question(\textsc{f})-\textsc{pl} \textsc{def}-asked-\textsc{pl}.\textsc{f} because right\_now 1\textsc{pl} still see-\textsc{pl}.\textsc{m} only \textsc{acc} data-\textsc{cs}.\textsc{pl} \textsc{def}-quarter \textsc{def}-fourth of \textsc{def}-year(\textsc{f}) \textsc{subord}-pass.\textsc{pst}-3\textsc{sg}.\textsc{f}\\
\glt \lq Itʼs still early to know the answers to the questions asked, since currently we are only seeing the data for the fourth quarter of last year.\rq{ }(found online, glosses added)\footnote{\url{https://www.globes.co.il/news/article.aspx?did=1000301071} (20 March, 2023).}

	\ex \label{appendixHebrewOdScalar4}
	\begin{xlist}
		\exi{A:} \textit{Ba-ha-tħala ro\rq{}im al ha-masax qav eħad, aval atem amurim kvar lir\rq{}ot šney qavim.}\\
		\lq In the begging you see one line on the screen, but you are supposed to already see two lines.\rq
		\exi{B:}
		\gll Anaħnu \textbf{ʕod} ro\rq{}-im eħad.\\
		1\textsc{pl} still see-\textsc{pl}.\textsc{m} one\\
		\glt \lq We still see (only) one.\rq{ }(Itamar Francez, p.c.)
\end{xlist}

	\ex \label{appendixHebrewOdScalar5}
	\begin{xlist}
		\exi{A:} \textit{Ha-bat šelxa kvar bat 15 naxon?}\\
		\lq Your daughter is 15 already?\rq{}
		\exi{B:} \gll Lo, hi \textbf{ʕod} bat 12.\\
		\textsc{neg} 3\textsc{sg}.\textsc{f} still of\_age(\textsc{f}) 12\\
		\glt \lq No, sheʼs (still) only 15.\rq{ }(Itamar Francez, p.c.)
	\end{xlist}
\end{exe}

\subsubsection{Broadly adverbial temporal-aspectual functions}
\paragraph{Prospective \lq eventually'}\label{appendixHebrewOdProspective}
\begin{itemize}
	\item \textcite{FrancezOd}, \textcite[142]{Koenig1991} and \textcite{Tobin1985}.
	\item As \textcite{FrancezOd} points out, in this function, \textit{ʕod} can modify a negated predicate without yielding \textsc{not yet}, as in (\ref{appendixHebrewOdEventually4}).
	\item  \textcite{FrancezOd} observes that this use is already found in Biblical Hebrew,\il{Hebrew, Biblical} as in  (\ref{exAppendixProspectiveHebrewBible}).
	\item Syntax: in pre-predicate position.
\end{itemize}
\begin{exe}
	\ex \gll Al tafsiq le-hitamen! B-a-sof \textbf{ʕod} te-natseaħ ot-i.\\
	\textsc{proh} 2\textsc{sg}.\textsc{m}.stop.\textsc{fut} to-practice.\textsc{inf} at-\textsc{def}-end still 2\textsc{sg}.\textsc{m}-beat.\textsc{fut} \textsc{acc}-1\textsc{sg}\\
	\glt \lq Don't stop practising! In the end, you'll yet beat me.\rq{ }\parencite{FrancezOd}	
	
	\ex Context: About failures to deal with sexual harassment in organised sport. The writer is being sarcastic. \label{appendixHebrewOdEventually3}\\
	\gll Im ze yi-mašex kaxa, b-a-sof \textbf{ʕod} lo ti-hye la-hem brera ela le-manot  iša l-a-tafqid.\\
	if \textsc{prox}.\textsc{sg}.\textsc{m} 3\textsc{sg}.\textsc{m}-continue.\textsc{fut} thus at-\textsc{def}-end still \textsc{neg} 3\textsc{sg}.\textsc{f}-\textsc{cop}.\textsc{fut} to-3\textsc{pl}.\textsc{m} choice(\textsc{f}) except to-appoint woman to-\textsc{def}-position\\
	\glt \lq If it goes on like this, in the end they will have no choice but to appoint a woman to the position.\rq{ }(online example, cited in \cite{FrancezOd})
	
	\ex 
	\gll Amar-ti l-o lihyot be-šeqeṭ, hu \textbf{ʕod} haya meʕir et ha-banot.	 \label{appendixHebrewOdEventually4}\\
	say.\textsc{pst}-1\textsc{sg} to-3\textsc{sg}.\textsc{m} \textsc{cop}.\textsc{inf} in-quiet 3\textsc{sg}.\textsc{m} still \textsc{cop}.\textsc{pst}.3\textsc{sg}.\textsc{m} wake.\textsc{sg}.\textsc{m} \textsc{acc} \textsc{def}-girl.\textsc{pl}\\
	\glt \lq I told him to be quiet, he would have woken up the girls.\rq{ }\parencite{FrancezOd}
	
	\ex Biblical Hebrew\il{Hebrew, Biblical}\label{exAppendixProspectiveHebrewBible}\\
	\gll \textbf{ʕod} evne-x ve-nivne-t betula-t Yisrael.\\
	still build.\textsc{ipfv}.1\textsc{sg}-2\textsc{sg}.\textsc{f} and-build.\textsc{pass}-2\textsc{sg}.\textsc{f} maiden-\textsc{cs} Israel\\
	\glt \lq \textbf{Yet will I rebuild you} and you will be built, O Virgin Israel.\rq{ }(Jeremiah 31: 3, cited in \cite{FrancezOd})
\end{exe}

\paragraph{Iterative via increment}
\label{appendixHebrewOdIterativeIncrement}
\begin{itemize}
	\item Form: in combination with the event quantifier \textit{paʕam}.
	\item This function extends to \lq\lq remind me again" questions (\ref{exAppendixHebrewOdIterativeIncrement3}).
	\item Yael Greenberg (p.c.) informs me that \textit{ʕod paʕam} does not have a restitutive reading.
\end{itemize}	

\begin{exe}
	\ex
	\gll Fisfas-ti et ha-rakevet \textbf{ʕod} \textbf{paʕam}.\\
	miss.\textsc{pst}-1\textsc{sg} \textsc{acc} \textsc{def}-train still time\\
	\glt \lq I've missed the train again.' (Yael Greenberg, p.c.)
	
	\ex
	\gll Bo'-u ne-nagen et ha-šir \textbf{ʕod} \textbf{paʕam}.\\
	come.\textsc{imp}-\textsc{pl} 1\textsc{pl}-play.\textsc{fut} \textsc{acc} \textsc{def}-song still time\\
	\glt \lq Let's play the song again.' (Yael Greenberg, p.c.)
	
	\ex\label{exAppendixHebrewOdIterativeIncrement3}
	\gll Ma haya \textbf{ʕod} \textbf{paʕam} ha-šem šel-o?\\
	what \textsc{cop}.\textsc{pst}.3\textsc{sg}.\textsc{m} still time \textsc{def}-name \textsc{poss}-3\textsc{sg}.\textsc{m}\\
	\glt \lq What was his name, again?' (Yael Greenberg, p.c.)
\end{exe}

\subsubsection{Temporal connectives and frame setters}
\paragraph{Persistent time frame}
\label{appendixHebrewOdContinuativeTT}
\begin{itemize}
	\item \textit{ʕod} has the persistent time frame use (\lq still \textit{t} when\rq{}).
	\item Syntax: forms a constituent with the temporal expression, preceding it.
\end{itemize}
\largerpage

\begin{exe}
	\ex{\textit{ʕerev eħad šamaʕti ʕal Dr. Gil Šaxar Yosef me-ħavera še-sipra še-šamaʕ hartsaʼa šelo ve ha-hexela laʕaqov aħarav}.\\
\lq One evening I heard about Dr. Gil Shahar Yosef from a friend who said that she heard a lecture of his and had started following him.\rq{}\\
\gll \textbf{ʕod} be-ot-o ʕerev ħipas-ti ʕala-v ħomer u-lemoħorat kvar raʼi-ti et ha-hartsaʼa še-šin-ta et ħaya-y.\\
still at-same-\textsc{sg}.\textsc{m} evening(\textsc{m}) look\_for.\textsc{pst}-1\textsc{sg} on-\textsc{3sg}.\textsc{m} material and-next\_day already see.\textsc{pst}-\textsc{1sg} \textsc{acc} \textsc{def}-lecture(\textsc{f}) \textsc{subord}-change.\textsc{pst}-\textsc{3sg}.\textsc{f} \textsc{acc} life-\textsc{poss}.\textsc{1sg}\\
\pagebreak
\glt \lq The very same evening I looked for material about him and the next day I already saw the lecture that changed my life.\rq{ }(found online, glosses added)\footnote{\url{https://talifinkelshtein.co.il/about/} (02 March, 2023).}
}

	\ex{Context: I am worried that a certain parcel will arrive too late. Customer service reassures me:\\
	\gll Ze ya-giaʕ \textbf{ʕod} ha-yom.\\
	\textsc{prox}.\textsc{sg}.\textsc{m} 3\textsc{sg}.\textsc{m}-arrive.\textsc{fut} still \textsc{def}-day\\
	\glt \lq It'll arrive this very day / before the end of the day.\rq{ }(Yael Greenberg, p.c.)}
	
	\ex{\label{exAppendixHebrewContinuativeTT3}
	\gll \textbf{ʕod} lemoħorat lexida-t-o šel {Adolf Eichmann} heħliṭ ha-ʕolam ha-Ze la-tet ʕadifut ʕelyon-a le-kisuy ha-mišpaṭ. \\
	still next\_day capture-\textsc{cs}-\textsc{poss}.3\textsc{sg}.\textsc{m} of {A. E.} decide.\textsc{pst}.3\textsc{sg}.\textsc{m} \textsc{def}-world(\textsc{m}) \textsc{def}-\textsc{prox}.\textsc{sg}.\textsc{m} to-give  priority(\textsc{f}) upper-\textsc{f} to-coverage.\textsc{cs} \textsc{def}-trial\\
	\glt \lq The very next day after Eichmann was captured, [the newspaper] \textit{This World} decided to give top priority to covering the trial.\rq{ }(found online, glosses added)\footnote{\url{https://thisworld.online/1961/1244} (20 March, 2023).}}
	
	\ex{Context: A student with the ardent desire to become a doctor did not get accepted into medical school at the first try. He applied again, and the good news that this time he was accepted for admission in Berlin got to him while he was travelling. (see ex. \ref{exAppendixSerbocroatianTemporalFrameAdverbial1})\label{exAppendixHebrewContinuativeTTMedSchool}}
	\exi{}\gll \textbf{ʕod} lemoħorat hu ʕazav le-Berlin.\\
	still next\_day 3\textsc{sg}.\textsc{m} leave.\textsc{pst}.3\textsc{sg}.\textsc{m} to-B.\\
	\glt The very next day, he set off for Berlin.\rq{ }(Itamar Francez, p.c.)
\end{exe}

\paragraph{Time-scalar additive \lq as far removed as\rq{}}\label{appendixHebrewOdTimeScalar}
\largerpage
\begin{itemize}
	\item \textit{ʕod} has a time-scalar additive use, in which it relates the focus to lower alternatives on a scale of temporal distance (\lq as far removed as\rq{}).
	\item Unlike markers such as Serbian-Croatian-Bosnian \textit{još} (\ref{appendixBCMSTimeScalar}), \textit{ʕod} does not have an additional reading in which it operates on a time scale proper, relating the focus to earlier alternatives (\lq as late as\rq{}); see esp. (\ref{exAppendixHebrewOdTimeScalar1}).
	\item That we are not dealing with a reading of \lq as early as\rq{ }becomes clearest in examples like (\ref{exAppendixHebrewOdTimeScalar5}, \ref{exAppendixHebrewOdTimeScalar6}), where developments in time are at stake and the \textsc{already} expression \textit{kvar} in the maning of \lq as early as\rq{ }is felicitous, but \textit{ʕod} is not.
\end{itemize}

\begin{exe}
	\ex\label{exAppendixHebrewOdTimeScalar1}
	\begin{xlist}
		\ex[\#]{
		\gll \textbf{ʕod} ke-adam zaken hu išen harbe.\\
		still as-man old 3\textsc{sg}.\textsc{m} smoke.\textsc{pst}.3\textsc{sg}.\textsc{m} a\_lot\\
		\glt (intended: \lq Even as an old man he (still) smoked a lot\rq)}
		\ex[ ]{\gll \textbf{ʕod} ke-adam tsaʕir hu išen harbe.\\
		still as-man young.\textsc{m} 3\textsc{sg}.\textsc{m} smoke.\textsc{pst}.3\textsc{sg}.\textsc{m} a\_lot\\
		\glt \lq All the way back as a young man he (already) smoked a lot.\rq{}
		\\(Yael Greenberg, p.c.)}
	\end{xlist}
	
	\ex Context: Commenting on a blog post stating that the first gas-powered buses entered operation in 2019.\\
	\gll Aval nir’a l-i še-otobus-im še-mufʕal-im ʕal gaz nixnes-u be-{Beitar Ilit} \textbf{ʕod} lifney 2019.\\
	but seems to-1\textsc{sg} \textsc{subord}-autobus(\textsc{m})-\textsc{pl} \textsc{subord}-operate.\textsc{pass}-\textsc{pl}.\textsc{m} by gas enter.\textsc{pst}-3\textsc{pl} at-{B. I.} still before 2019\\
	\glt \lq But it seems to me that gas-powered buses entered operation in Beitar Ilit even before 2019.\rq{ }(found online, glosses added)\footnote{\url{http://israelbikebus.blogspot.com/2019/12/blog-post.html} (03 November, 2022).}

	\ex \gll Lemaʕase gam \lq{}Le-šaħrer et Guy\rq{} haya amur la-tset šana še-ʕavr-a ke-še-ha-treyler ha-rišon yatsa \textbf{ʕod} be-2019(!)\\
	actually also \phantom{\lq}to-free \textsc{acc} guy \textsc{cop}.\textsc{pst}.3\textsc{sg}.\textsc{m} say:\textsc{ptcp}.\textsc{pass}.\textsc{sg}.\textsc{m} to-come\_out year(\textsc{f}) \textsc{subord}-pass.\textsc{pst}-\textsc{sg}.\textsc{f} as-\textsc{subord}-\textsc{def}-trailer \textsc{def}-first come\_out.\textsc{pst}.3\textsc{sg}.\textsc{m} still at-2019\\
	\glt \lq Actually, [the movie] \textit{Free Guy} was also supposed to be released last year, whereas the first trailer came out way back in 2019(!)\rq{ }(found online, glosses added)\footnote{\url{https://www.geekster.co.il/entertainment/freeguy} (03 November, 2022).}
	\largerpage
	\ex
	\gll \textbf{ʕod} {Ben Guryon} heħliṭ ... še-hu rotse li-grom le-kax še-kol ha-yehud-im še-magiʕ-im le-medina-t Yisrael y-uxl-u li-ħyot yaħad b-a-medina.\\
	still {B. G.} decide.\textsc{pst}.3\textsc{sg}.\textsc{m} {} \textsc{subord}-3\textsc{sg}.\textsc{m} want.\textsc{sg}.\textsc{m} to-cause to-thus \textsc{subord}-all \textsc{def}-jew-\textsc{pl}.\textsc{m} \textsc{subord}-arrive-\textsc{pl}.\textsc{m} to-state-\textsc{cs} I. 3-be\_able.\textsc{fut}-\textsc{pl} to-live together at-\textsc{def}-state\\
	\glt \lq Already Ben Guryon decided that he wanted to create a situation where all Jews arriving in the state of Israel could live together in this state.\rq{ }(found online, glosses added){\interfootnotelinepenalty=10000\footnote{\url{https://mishmar.org.il/\%D7\%A1\%D7\%92\%D7\%9F-\%D7\%94\%D7\%A9\%D7\%A8-\%D7\%90\%D7\%9C\%D7\%99-\%D7\%91\%D7\%9F-\%D7\%93\%D7\%94\%D7\%9F-\%D7\%9C\%D7\%A8\%D7\%A4\%D7\%95\%D7\%A8\%D7\%9E\%D7\%99\%D7\%9D-\%D7\%91\%D7\%90\%D7\%A8\%D7\%94\%D7\%91-\%D7\%A8\%D7\%A6\%D7\%99\%D7\%AA/} (03 November, 2022).}}
	\ex\label{exAppendixHebrewOdTimeScalar5}
	 Context: About a fast-acting biochemical agent.\\
	\gll \textbf{Kvar} (\#\textbf{ʕod})  aħarey 10 daq-ot mi-hosafa-t ha-ħomer hay-u garʕin-im me-ħuts l-a-ta’-im.\\
	already \phantom{\#(}still after 10 minute-\textsc{pl} from-addition-\textsc{cs} \textsc{def}-substance \textsc{cop}.\textsc{pst}-3\textsc{pl} nucleus-\textsc{pl} from-outside to-\textsc{def}-cell-\textsc{pl}\\
	\glt \lq Already ten minutes after the addition of the chemical, some nuclei are found outside the cells.\rq{ }(Yael Greenberg, p.c.)
	
	\ex\label{exAppendixHebrewOdTimeScalar6}
	\gll Kvar (\#\textbf{ʕod})  be-gil 18 hi qibl-a ʕavoda b-a-universiṭa.\\
	already \phantom{(\#}still at-age 18 3\textsc{sg}.\textsc{f} get.\textsc{pst}-3\textsc{sg}.\textsc{f} job at-\textsc{def}-university\\
	\glt \lq Already at the age of 18 she got a job at university.\rq{ }(Yael Greenberg, p.c.)
\end{exe}

\paragraph{Coextensiveness: \textit{kol ʕod} \lq as long as'}
\label{appendixHebrewOdAsLongAs}
\begin{itemize}
	\item \textcite[548]{Glinert1989} and \textcite[37]{Schwarzwald2001}.
	\item Form: in collocation with the universal quantifier \textit{kol}.
	\item This collocation introduces a temporal clause of extensiveness (\lq as long as').
\end{itemize}	
\begin{exe}
\ex
	\gll \textbf{Kol} \textbf{ʕod} (\textbf{ti}-\textbf{hye}) qayem-et ha-šiṭa ha-zot, ze yi-mašex.\\
	all still \phantom{(}3\textsc{sg}.\textsc{f}-\textsc{cop}.\textsc{fut} exist-3\textsc{sg}.\textsc{f} \textsc{def}-system(\textsc{f}) \textsc{def}-\textsc{prox}.\textsc{sg}.\textsc{f} \textsc{prox}.\textsc{sg}.\textsc{m} 3\textsc{sg}.\textsc{m}-continue.\textsc{fut}\\
	\glt \lq As long as this system exists, it'll go on.\rq{ }(\cite[548]{Glinert1989}, glosses added)
	
	\ex Context: About medical companies profiting from the Covid-19 pandemic.\\
	\gll \textbf{Kol} \textbf{ʕod} ha-Qorona kan hen y-amšix-u li-dpoq qupa.\\
all still \textsc{def}-Corona here 3\textsc{pl}.\textsc{f} 3-continue.\textsc{fut}-\textsc{pl} to-beat till\\
	\glt \lq As long as Corona is here, they'll keep knocking the box office.'\\(found online, glosses added)\footnote{\url{https://www.themarker.com/markets/.premium-1.10489466} (17 May, 2022).}
\end{exe}
	
\subsubsection{Marginality}
\label{appendixHebrewOdMarginal}
\begin{itemize}
	\item \textit{ʕod} is, in principle, compatible with readings of marginality. 
	\item However, \textit{ʕadayin} (\ref{appendixHebrewAdayinMarginal}) appears to fare better here; thus the oddness of (\ref{exAppendixHebrewOdMarginal2}, \ref{exAppendixHebrewOdMarginal4}). This might be due the strong association of \textit{ʕod} with additivity.
\end{itemize}

\begin{exe}
	\ex\label{exAppendixHebrewOdMarginal1}
	Context: Talking about tennis skills.\\
	\gll Et Paul ani \textbf{ʕod} yaxol le-natseaħ.\\
	\textsc{acc} P. 1\textsc{sg} still can.\textsc{sg}.\textsc{m} to-beat\\
	\glt \lq Paul I can still beat (i.e. other players are too good, but Paul is beatable).' (Yael Greenberg, p.c.)

	\ex\label{exAppendixHebrewOdMarginal2}
	Context: I am annoyed because my rich aunt has left the better part of her fortune to an animal shelter, and only 10,000€ of inheritance go to me.
	\exi{}[?]{\gll 10,000 Euros ze \textbf{ʕod} sxum yafe.\\
	10,000 Euro \textsc{prox}.\textsc{sg}.\textsc{m} still amount(\textsc{m}) beautiful.\textsc{m}\\
	\glt \lq{}10,00 Euros is still a decent sum.' (Yael Greenberg, p.c.)\\
	NB: Possible if \textit{ʕod} is strongly destressed.}
	
	\ex\label{exAppendixHebrewOdMarginal3}
	Context: Viewed from Tel Aviv.\\
	\gll Heifa hi \textbf{ʕod} be-Yisrael, aval Tsur hi kvar be-Levanon.\\
	Haifa 3\textsc{sg}.\textsc{f} still at-Israel but Tyros 3\textsc{sg}.\textsc{f} already at-Lebanon\\
	\glt \lq Haifa is still in Israel, but Tyros is already in Lebanon.\rq{ }(Yael Greenberg, p.c.)
	
	\ex\label{exAppendixHebrewOdMarginal4}
	Context: Talking about political view.
	\exi{}[?]{\gll Paul hu \textbf{ʕod} matun, leʕumat Mark, še-hu qitsoni.\\
	P. 3\textsc{sg}.\textsc{m} still moderate.\textsc{m} but M. \textsc{rel}-3\textsc{sg}.\textsc{m} radical.\textsc{m}\\
	\glt \lq Paul is still moderate, but Mark is radical.' (Yael Greenberg, p.c.)}	
\end{exe}	

\subsubsection{Additive and related functions}
\paragraph{Additive}
\label{appendixHebrewOdAdditive}
\begin{itemize}
	\sloppy
	\item \citeauthor{Glinert1976} (\citeyear{Glinert1976}; \citeyear[78, 87, 226]{Glinert1989}), \textcite{Greenberg2012}, \textcite{Thomas2018} and \textcite{Tobin1985}.
	\item When addition involves more of the same type, \textit{ʕod} is NP-internal (\ref{exAppendixHebrewOdAdditive2}, \ref{exAppendixHebrewOdAdditive3}), and can even stand in for a quantified indefinite NP \lq more\rq{ }(\ref{exAppendixHebrewOdAdditive4}). The syntactic rules governing the position (preposed or postposed) of \textit{ʕod} as a phrasal adjunct are complex; see \textcite{Glinert1976}. Relatedly, \textit{ʕod} in post-predicative position can mark that an event develops further \lq Verb some more' (\ref{exAppendixHebrewOdAdditive5}).
	\item Postponed \textit{ʕod} can also mark an additional, particularly strong argument, typically with concessive overtones (\ref{exAppendixHebrewOdAdditive6}).
\end{itemize}

\pagebreak
\begin{exe}
	\ex\label{exAppendixHebrewOdAdditive1}
	Context: I recently arrived in Israel. I report a conversation I had with a local.\\
	\gll Ve-hu \textbf{ʕod} amar še-ani medaber ʕivrit ṭov me'od.\\
	and-3\textsc{sg}.\textsc{m} still say.\textsc{pst}.3\textsc{sg}.\textsc{m} \textsc{subord}-1\textsc{sg} speak.\textsc{sg}.\textsc{m} Hebrew good very\\
	\glt \lq He also said that I speak Hebrew very well.' (Yael Greenberg, p.c.)

	\ex\label{exAppendixHebrewOdAdditive2}
	\gll Ha-mesiba ha-zo hay-ta ason! Hayi-ti tsarix le-tapel be-oto zman be-6 yelad-im box-im ve-\textbf{ʕod} 4 rav-im ve-tsorħ-im!\\
	\textsc{def}-party(\textsc{f}) \textsc{def}-\textsc{prox}.\textsc{sg}.\textsc{f} \textsc{cop}.\textsc{pst}-3\textsc{sg}.\textsc{f} desaster \textsc{cop}.\textsc{pst}-1\textsc{sg} must to-deal\_with in-same time in-6 kid-\textsc{pl}.\textsc{m} cry-\textsc{pl}.\textsc{m} and-still 4 fight-\textsc{pl}.\textsc{m} and-scream-\textsc{pl}.\textsc{m}\\
	\glt \lq That party was a disaster! I had to deal at the same time with 6 children crying and 4 more fighting and screaming!' \parencite[132]{Greenberg2012}
	
	\ex\label{exAppendixHebrewOdAdditive3}
	\gll Ma ata ta-ʕase ʕim \textbf{ʕod} neyar?\\
	what 2\textsc{sg}.\textsc{m} 2\textsc{sg}.\textsc{m}-do.\textsc{fut} with still paper\\
	\glt \lq What will you do with more paper?\rq{ }\parencite[249]{Glinert1976}
	
	\ex\label{exAppendixHebrewOdAdditive4}
	\gll Etmol axal-ti 3 tapuz-im. Ha-yom axal-ti \textbf{ʕod} (tapuz-im)\\
	yesterday eat.\textsc{pst}-1\textsc{sg} 3 orange-\textsc{pl}.\textsc{m} \textsc{def}-day eat.\textsc{pst}-1\textsc{sg} still \phantom{(}orange-\textsc{pl}.\textsc{m}\\
	\glt \lq Yesterday I ate 3 oranges. Today I ate some more (oranges).
	\rq{ }\parencite[127]{Greenberg2012}
		
	\ex\label{exAppendixHebrewOdAdditive5}
	\gll Ba-boqer Rina rats-a ktsat. B-a-tsoharayim hi rats-a \textbf{ʕod}.\\
	at-morning Rina run.\textsc{pst}-3\textsc{sg}.\textsc{f} a\_bit at-\textsc{def}-noon 3\textsc{sg}.\textsc{f} run.\textsc{pst}-3\textsc{sg}.\textsc{f} still\\
	\glt \lq In the morning Rina ran a bit. At noon she ran some more.\rq{ }\parencite[127]{Greenberg2012}
	
	\ex\label{exAppendixHebrewOdAdditive6}
	Context: He did something terriby immoral.\\
	\gll Aval ha-ben\rq{}adam dati \textbf{ʕod}.\\
	but \textsc{def}-guy(\textsc{m}) religious.\textsc{sg}.\textsc{m} still\\
	\glt \lq But the fellow’s religious, what’s more!\rq{ }(\cite[537]{Glinert1989} and Itamar Francez, p.c.)
\end{exe}
	
\paragraph{\# Further-to}
\begin{itemize}
	\item \textit{ʕod} does not have \lq\lq further-to" readings (Yael Greenberg, p.c.).
\end{itemize}
		
\paragraph{Comparisons of inequality }\label{appendixHebrewOdComparisons}
\begin{itemize}
	\sloppy
	\item \citeauthor{Glinert1976} (\citeyear{Glinert1976}, \citeyear[220]{Glinert1989})
and \textcite{MiashkurGreenberg}.
	\item Note that comparisons of inequality in Modern Hebrew are formed via \textit{yoter} \lq more'\slash \textit{paħot} \lq less'. The standard of comparison, if present, is introduced by \textit{mi} or \textit{me\rq{}asher} \lq from' \parencite[216–217]{Glinert1989}. \textit{ʕod} contributes scalar additive \lq even\rq{ }to the comparison.
	\item Syntax: precedes and forms a constituent with \textit{yoter}; note their moving through the clause together in (\ref{appendixHebrewOdComparisons2}).
\end{itemize}	
\begin{exe}
	\ex
	\gll Bil \textbf{ʕod} yoter gavoah mi-Jon.\\
	Bill still more tall from-John\\
	\glt \lq Bill is even taller than John.' \parencite[1]{MiashkurGreenberg}
	
	\ex\label{appendixHebrewOdComparisons2}
	\gll Ze \textbf{ʕod} yoter meguħax. \textup{/} Ze meguħax \textbf{ʕod} yoter.\\
	\textsc{prox}.\textsc{sg}.\textsc{m} still more ridicoulous {} \textsc{prox}.\textsc{sg}.\textsc{m} ridicoulous still more\\
	\glt \lq This is even more ridiculous\rq{ }\parencite[251–252]{Glinert1976}
		
	
	\ex
	\gll Qibl-u me\rq{}a mig-im ve-\textbf{ʕod} yoter mi-kax tank-im.\\
	receive.\textsc{pst}-3\textsc{pl} hundred mig-\textsc{pl}.\textsc{m} and-still more from-thus tank-\textsc{pl}.\textsc{m}\\
	\glt \lq They received one hundres Migs, and even more than that number of tanks.' (\cite[59]{Glinert1989}, glosses added)	
	
	\ex
	\gll Ze \textbf{ʕod} yoter madhim ot-i.\\
	\textsc{dem}.\textsc{m} still more amaze.\textsc{sg}.\textsc{m} \textsc{acc}-1\textsc{sg}\\
	\glt \lq It appals me still more.' (\cite[220]{Glinert1989}, glosses added)	
\end{exe}
	
\paragraph{Conjunctional adverb}\label{appendixHebrewConjunctional}
\begin{itemize}
	\item \textcite[537]{Glinert1989}.
	\item Syntax: clause-initially.
	\item In this function, \textit{ʕod} can occur alone, as in (\ref{exAppendixHebrewOdConjunctional1}), a use that \textcite[537]{Glinert1989} describes as belonging to a formal register. It also occurs in the collocation \textit{ma od} \lq what still', i.e. what is more' (\appref{appendixHebrewWhatIsMore}).
\end{itemize}

\begin{exe}
\ex\label{exAppendixHebrewOdConjunctional1}
	\gll \textbf{ʕod} moser katav-enu ki…\\
	still report.3\textsc{sg}.\textsc{m} reporter-\textsc{poss}.1\textsc{pl} \textsc{comp}\\
	\glt \lq Our reporter further reports that…' (\cite[537]{Glinert1989}, glosses added)
\end{exe}


\paragraph{Conjunctional adverbial: \textit{ma od} \lq what is more\rq{}}\label{appendixHebrewWhatIsMore}
\begin{itemize}
	\item \textcite[537]{Glinert1989}.
	\item Form: in the collocation \textit{ma od} \lq what is more\rq{}, lit. \lq{}what still\rq{}.
	\item This collocation is clearly based on \textit{ʕod} in general additive function (\appref{appendixHebrewOdAdditive}), i.e. it is entirely parallel to \ili{English} \textit{what is more}.
	\item Syntax: clause-initially.
\end{itemize}

\begin{exe}
\ex\label{exAppendixHebrewOdConjunctional2}
	\gll Hay-ta te\rq{}una, \textbf{ma} \textbf{ʕod} še-ha-nehag-im šavt-u.\\
	\textsc{cop}.\textsc{pst}-3\textsc{sg}.\textsc{f} accident(\textsc{f}) what still \textsc{subord}-\textsc{def}-driver-\textsc{pl}.\textsc{m} strike.\textsc{pst}-3\textsc{pl}\\
	\glt \lq There was an accident, and what's more, the drivers were striking.' (\cite[267]{Glinert1989}, glosses added)	
\end{exe}



\subsubsection{Broadly modal and interactional uses}
\paragraph{Exclamation \textit{ve}-\textit{ʕod} \textit{ex}! \lq And how!\rq{}}\label{appendixHebrewOdAndHow}
\begin{itemize}
	\item \citeauthor{Glinert1976} (\citeyear{Glinert1976}, \citeyear[282]{Glinert1989}).
	\item Form: in the fixed exclamation \textit{Ve}-\textit{ʕod} \textit{ex!} \lq and-still how\rq{ }\lq and how\rq{}.
	\item This is most likely based on additive \textit{ʕod} (\appref{appendixHebrewOdAdditive}) and is strikingly similar to Serbian-Croatian-Bosnian \textit{još} \textit{kako}, \textit{još} \textit{koliko}, both \lq and how\rq{}, lit. \lq{}still how, still how much\rq{ }(\appref{appendixBCMSSpecificational}).
\end{itemize}
\il{Hebrew, Modern|)}

\section{Hills Karbi (mjw, karb1241)}\il{Karbi, Hills|(}
\label{appendixHillsKarbi}

\subsection{Introductory remarks}
Apart from descriptive materials, I searched \citeauthor{KonnerthTisso2018}'s (\citeyear{KonnerthTisso2018}) text collection. My understanding of Hills Karbi has furthermore profited from discussions with Linda Konnerth.

\subsection{-làng}

\subsubsection{General information}
\begin{itemize}
	\item Form: mid-toned \mbox{-\textit{lāng}} when following irrealis \mbox{-\textit{jí}}.
	\item Wordhood: bound morpheme (verb suffix).
\end{itemize}


\subsubsection{As a \lq{}still\rq{ }expression}
\begin{itemize}
	\item \textcite[103]{Gruessner1978}, \textcite[222–223, 299]{Konnerth2014} and \textcite[424]{Taro2010}.
	\item Specialisation: attestations like (\ref{exAppendixHillsKarbi1}–\ref{exAppendixHillsKarbi4}) give evidence that this marker conforms to my definition. For instance, in (\ref{exAppendixHillsKarbi1}) \mbox{-\textit{làng}} evokes an alternative scenario in which the father's children are no longer small, and he could therefore leave them home alone.
	\item Pragmaticity: the data allow no conclusions.
	\item Polarity sensitivity: inner negation yields \textsc{not yet}.
\end{itemize}
\begin{exe}
	\ex\label{exAppendixHillsKarbi1}
	 Context: The mother of two children has died. The father is worried that he cannot go work in the field and leave them home alone.\\
	\gll E ne-oso-màr=tā bī-hek$\sim$hāk-\textbf{làng}.\\
	\textsc{dm} 1.\textsc{excl}-child-\textsc{pl}=also be\_small-small$\sim$\textsc{pl}-still\\
	\glt \lq O, my children are still so small.' \parencite[159]{KonnerthTisso2018}
	
	\ex\label{exAppendixHillsKarbi3}
	\gll A-hotón a-béléng mamát-\textbf{làng}=ma, ma a-ki-mī cho-lóng-lo=ma?\\
	\textsc{poss}-bamboo\_basket \textsc{poss}-strainer self-still=\textsc{q} \textsc{q} \textsc{poss}-\textsc{nmlz}-be\_new \textsc{auto}.\textsc{ben}-get-\textsc{rl}=\textsc{q}\\
	\glt \lq Hast du noch die Körbe und Siebe, oder hast du neue bekommen? [Do you still have those [same] baskets and strainers, or did you buy yourself new ones?]' (\cite[129]{Gruessner1978}; glosses by \cite[528]{Konnerth2014})
	
	\ex\label{exAppendixHillsKarbi4}
	\gll Nè-li wàng-wē-té nàng mék jáng-dàk-bōm-ji-\textbf{làng} a-pót-ló.\\
	1\textsc{excl}-\textsc{hon} come-\textsc{neg}-\textsc{cond} 2 eye fall-spread\_out-\textsc{cont}-\textsc{irr}-still should\\
	\glt \lq Wenn ich nicht gekommen wäre, würdest du noch weiter schlafen. [If I hadn't come here, you'd still be asleep.]' (\cite[138]{Gruessner1978}, glosses added)
\end{exe}

\subsubsection{Broadly adverbial temporal-aspectual functions}
\paragraph{\# Prospective \lq eventually\rq}
\begin{itemize}
	\item \mbox{-\textit{làng}} does not have the prospective \lq eventually\rq{ }use (Linda Konnerth, p.c.).
\end{itemize}

\pagebreak
\subsubsection{Additive and related functions}
\paragraph{Additive}\label{appendixKarbiAdditive}
\begin{itemize}
	\item \textcite[335–336]{Konnerth2014}
	\item This function is attested with \mbox{-\textit{làng}} alone, as in (\ref{exAppendixKarbiAdditive1}), as well as in combination with additive \mbox{=\textit{tā}}, where it gives emphasis to the notion of inclusion (\ref{exAppendixKarbiAdditive2}). Note how the additive function in (\ref{exAppendixKarbiAdditive2}) also becomes clear from the fact that despite negation \mbox{-\textit{làng}} is not interpreted as \textsc{not yet}. Also note the two occurences of \mbox{-\textit{làng}} in this example, in line with a common pattern in the usage of \mbox{=\textit{tā}} (see \cite[344]{Konnerth2014} on the latter).
	\item To express addition of an entity of the same type (\lq another'), \mbox{-\textit{làng}} is used in combination with a numeral (\ref{exAppendixKarbiAdditive3}) and ~– optionally~– \textit{nón} \lq now' (\ref{exAppendixKarbiAdditive4}) and/or iterative \mbox{-\textit{thū}}  (\ref{exAppendixKarbiAdditive5}).
\end{itemize}

\begin{exe}
 	\ex Context: A father whose wife has died is desperate.\label{exAppendixKarbiAdditive1}\\
	\gll Sì a-oso-màr a-phān che-arjū-lò ò pēi a-tūm tē ko-pù-jí-\textbf{lāng}=ma.\\
	therefore \textsc{poss}-child-\textsc{pl} \textsc{poss}-\textsc{non}.\textsc{subj} \textsc{refl}/\textsc{recp}-ask-\textsc{rl} \textsc{voc} mother \textsc{poss}-\textsc{pl} if what-like\_this-\textsc{irr}-still-\textsc{q}\\
	\glt \lq Therefore, he asked his children, \lq\lq O mothers, so then, what else could we do?"' \parencite[159–169]{KonnerthTisso2018}
 
	\ex\label{exAppendixKarbiAdditive2}
	\gll Nang-phì aphān theklōng-lē án=tā kalī-\textbf{làng} akò nàng=tā ningjé-thù-sèr-\textbf{làng}\\
	2-grandmother \textsc{non}.\textsc{subj} see-\textsc{neg} all=also \textsc{neg}.\textsc{cop}-still on\_the\_other\_hand 2=also speak-again-unexpectedly(uttering)-still\\
	\glt \lq Not only will you not see your grandmother, but also you speak like this, though you shouldn't.' \parencite[232]{KonnerthTisso2018}

	\ex\label{exAppendixKarbiAdditive3}
	Context: A tiger is collecting people as sacrifices. He has caught several already, locked them in a cage and put them in the house of the village head tiger.\\
	\gll Bí-dàm-lò te e-jōn náng-jí-\textbf{lāng}.\\
	keep-go-\textsc{rl} and\_then one-\textsc{clf}:animal need-\textsc{irr}-still\\
	\glt \lq He had gone and put them there, and then, one more is needed.ʼ \parencite[637]{Konnerth2014}
	
	\pagebreak
	\ex\label{exAppendixKarbiAdditive4}
	Context: A boy has encountered a tiger. Using a container with a mirror, he outsmarts the tiger, making it think he has already caught one tiger.\\
	\gll Húladāk ingtòng ke-bèng=tā dō-lò mm. Nòn e-jōn náng-jí-\textbf{lāng} nè=tā mm.\\
	there big\_bamboo\_basket \textsc{nmlz}-lock=also exist-\textsc{rl} \textsc{aff} now one-\textsc{clf}:animal need-\textsc{irr}-still 1.\textsc{excl}=also \textsc{aff}\\
	\glt \lq There in the bamboo basket, I have (a tiger). I also need one more.' \parencite[195]{KonnerthTisso2018}
	
	\ex\label{exAppendixKarbiAdditive5}
	\gll Isī a-lám dō-thū-\textbf{làng}.\\
	one \textsc{poss}-matter exist-again-still\\
	\glt \lq There is still one other thing.' \parencite[336]{Konnerth2014}
\end{exe}


\paragraph{Further-to}\label{appendixHillsKarbiFurtherTo}
\begin{itemize}
	\item \textcite[299–300]{Konnerth2014}.
	\item Ex. (\ref{exAppendixKarbiFurtherTo3}) is perhaps best considered a pure additive use, but comes close to a further-to one. Similarly, ex. (\ref{exAppendixKarbiFurtherTo4}) is likely to be ambiguous between a phasal polarity reading \lq still need to\rq{ }and a further-to one \lq need to (before moving on)\rq{}.
\end{itemize}

\begin{exe}
	\ex Context: There is a plan to go to the market.\label{exAppendixKarbiFurtherTo1}\\
	\gll Rí chersām-dām-\textbf{làng}.\\
	hand wash-go-still\\
	\glt \lq Iʼm just gonna go wash my hands real quick (and then we can go).' \parencite[299–300]{Konnerth2014}
	
	\ex Context: Two children who have been raised by tigers want to live with their biological parents.\\
	\gll Mh ne-pei ne-po hadak do apot ne-pei ne-po ch-arju-dam-lang pu amatsi halaso a-teke along ako che-dam-lo .\\
	\textsc{interj} \textsc{poss}.1\textsc{excl}-mother \textsc{poss}.1\textsc{excl}-father there stay because \textsc{poss}.1\textsc{excl}-mother \textsc{poss}.1\textsc{excl}-father \textsc{refl}/\textsc{recp}-ask-go-still \textsc{quot} because that \textsc{poss}-tiger \textsc{loc} again \textsc{refl}/\textsc{recp}-go-\textsc{rl}\\
	\glt \lq Because my mother and father [i.e. the tiger parents] are there, let’s still go and ask our parents, and then they went to the tigers.\rq{ }\parencite[154–155]{KonnerthTisso2018}
	
	\pagebreak
	\ex\label{exAppendixKarbiFurtherTo3}
	Context: From the beginning of the recollection of a trip.\\
	\textit{…là elilitūm ajirpò alànglì Yu’éspensi kevàng Kavòn Kavòn aphāntā chepōnlò}\\
	\lq … this friend of ours, he who has come from the US, Kavon, Kavon we also took along with us.'
	\exi{}\gll sì ladāk=pen dàm-lò Dimápúr vùr-pōn sá jùn-pōn-\textbf{làng}. \\
	therefore here=from go-\textsc{rl} D. drop\_in-in\_passing tea drink-in\_passing-still\\
	\glt \lq And then, from here we went, we stopped by in Dimapur [about a quarter of the way in] and just had tea.'
	\exi{}
	\textit{lasì bají sirkē-bāk apòrpe=si puthōt dàmthūlò Kohìmàán továr kēkò.}\\
	\lq At nine o’clock, from about that time, we again went, up to Kohima, the road is winding a lot.' \parencite[361–362]{KonnerthTisso2018}
	
		\ex\label{exAppendixKarbiFurtherTo4}
	Context: Two children, adopted by tigers, want to live with their biological father, a king.\\
	\gll Nè ne-pēi ne-pō aphān che-arjū-dām-\textbf{làng}.\\
	1\textsc{excl} \textsc{poss}.1\textsc{excl}-mother \textsc{poss}.1\textsc{excl}-father \textsc{non}.\textsc{subj} \textsc{refl}/\textsc{recp}-ask-go-still\\
	\glt \lq We still need to ask our mother and father.\rq{ }\parencite[154]{KonnerthTisso2018}
\end{exe}
\il{Karbi, Hills|)}

\section{Japhug (jya, japh1234)}\il{Japhug|(}

\subsection{Introductory remarks}
I am indebted to Guillaumes Jacques for discussing Japhug data with me and for eliciting additional data.

\subsection{pɤjkʰu}
\subsubsection{General information}
\begin{itemize}
	\item Wordhood: independent grammatical word.
	\item Syntax: relatively fixed (pre-verbal position, may be preceded or followed by arguments).
\end{itemize}

\subsubsection{As a \textsc{still} expression}
\begin{itemize}
	\item \citeauthor{Jacques2016} (\citeyear[219]{Jacques2016}, \citeyear[1200–1201]{Jacques2021}).
	\item Specialisation: the discussion by \citeauthor{Jacques2016} (\citeyear[219]{Jacques2016}, \citeyear[1201]{Jacques2021}), together with examples like (\ref{exAppendixJaphug1}, \ref{exAppendixJaphug2}) give evidence that this marker conforms to my definition. For instance, in (\ref{exAppendixJaphug1}) \textit{{pɤjkʰu}} not only signals the continuation of a prior state, but also alludes to its future discontinuation. Further, albeit indirect, evidence comes from its use as \textsc{not yet} without negation (\appref{appendixJaphugNotYet}).
	\item Pragmaticity: ex. (\ref{exAppendixJaphug2}) suggests that \textit{pɤjkʰu} is not only compatible with the neutral scenario, but also the unexpectedly late one.
	\item Polarity sensitivity: inner negation yields \textsc{not yet}.
\end{itemize}
\begin{exe}
	\ex\label{exAppendixJaphug1}
	\gll Cʰa-a ɕi mɤ-cʰa-a mɤ-xsi, \textbf{pɤjkʰu} xtɕi-a ɕti.\\
	can-1\textsc{sg} \textsc{q} \textsc{neg}-can-1\textsc{sg} \textsc{neg}-\textsc{impr}.know still be\_small-1\textsc{sg} \textsc{cop}.\textsc{aff}\\
	\glt \lq I don’t know whether I will be able to do it, I am still young.\rq{ }\parencite[1172]{Jacques2021}

	\ex\label{exAppendixJaphug2}
	\gll Iʑo a-mu nɯ tʰamtʰam kɯrcɤsqaptɯɣ tʰɯ-azɣɯt ŋu. \textbf{pɤjkʰu} ji-paʁ pjɯ-nge cʰa.\\
	1\textsc{pl} \textsc{poss}.1\textsc{sg}-mother \textsc{dem} now eighty\_one \textsc{aor}-arrive \textsc{cop} still \textsc{poss}.1\textsc{pl}-pig \textsc{ipfv}-feed can\\
	\glt \lq Our mother is now eighty-one years old, she can still feed our pigs.' \parencite[1200]{Jacques2021}
\end{exe}

\subsubsection{Uses related to other phasal polarity concepts}
\paragraph{Not yet / wait a bit}\label{appendixJaphugNotYet}
\begin{itemize}
	\item \citeauthor{Jacques2016} (\citeyear[219]{Jacques2016}, \citeyear[1200–1201]{Jacques2021}).
	\item Form: in combination with the sentence-final particle \textit{je}. The latter is otherwise used to attenuate commands and prohibitives, as well as with phatic expressions such as \lq goodbye', \lq good night' or \lq take care' (see \cite[458–459]{Jacques2021}). The two usually fuse to phonologically irregular \textit{pɤkʰije}.
	\item This signals a notion of \lq Wait!\rq{}.
\end{itemize}

\begin{exe}
	\ex \gll \textbf{Pɤkʰije}!\\
	still.\textsc{sfp}:attenuation\\
	\glt \lq Wait!' \parencite[453]{Jacques2021}
	
	\ex
	\gll \textbf{Pɤkʰije} tɕe <guan> ma-kɤ-tɯ-βze je!\\
	still.\textsc{sfp}:attenuation \textsc{lnk} turn\_off \textsc{neg}-\textsc{imp}-2-make \textsc{sfp}:attenuation\\
	\glt \lq Wait, don’t hang up [your phone].' \parencite[458]{Jacques2021}
\end{exe}

\subsubsection{Broadly adverbial temporal-aspectual functions}
\paragraph{First, for now}\label{appendixJaphugFirst}
\begin{itemize}
	\item \textit{Pɤjkʰu} is repeatedly attested with a meaning along the lines of \lq first, for now'.
	\item In (\ref{exAppendixJaphugfirst4}) \textit{pɤjkʰu} in this function combines with a verb \textit{mɤku} \lq be first'.
\end{itemize}

\begin{exe}
	\ex
	\gll \textbf{Pɤjkhu} nɤʑo thɯ-ɣi ra.\\
	still 2\textsc{sg} downstream-come be\_needed\\
	\glt \lq {\cn 暂时先要你一个人 下来.} [For now, I want you to come down alone.]' (\cite[219]{Jacques2016}, glosses added)
	
	\ex
	\gll Nɤʑo \textbf{pɤjkhu} tɯ-rzɯɣ tɤ-nɯna tɕe tɕetha rɤma-tɕi.\\
	2\textsc{sg} still one-section \textsc{imp}-rest \textsc{lnk} later work-1\textsc{du}\\
	\glt \lq {\cn 你暂时休息 一下,等一会我们再工作.} [Take a break for now and we’ll work again later.]' (\cite[361]{Jacques2016}, glosses added)

	\ex Context: A father suspects his son to have drunk alcohol and is waiting for him at the door of the house.\\
	\gll Jɤ-ɣi tɕe, \textbf{pɤjkʰu} tu-ta-nɤ-mnɤm.\\
	\textsc{imp}-come \textsc{lnk} still \textsc{ipfv}:up-1>2-\textsc{tropative}-have\_a\_smell\\
	\glt \lq Come (here), I will smell you [first] (to see if you have had alcohol).'\footnote{The Japhug \lq\lq tropative" is a transitivizing device, with a meaning along the lines of \lq find/consider to be …', see \textcite[868–870]{Jacques2021}.} (\cite[872]{Jacques2021}; Guillaumes Jacques, p.c.)

	\ex\label{exAppendixJaphugfirst4}
	Context: About an animal that is not to be executed for the next months.\\
	\gll Iʑo kɯ-mɤku \textbf{pɤjkʰu}, ɯ-ɕɣa kɯ-mtɕoʁ nɯ ci ɲɯ́-wɣ-pʰɯt.\\
	1\textsc{pl} \textsc{ptcp}-be\_first still \textsc{poss}.3\textsc{sg}-tooth \textsc{ptcp}-be\_sharp \textsc{dem} a\_little \textsc{ipfv}-\textsc{inv}-take\_out\\
	\glt \lq [For now] let us first take out its sharp teeth [so as to prevent it from biting].' (\cite[602]{Jacques2021}; Guillaumes Jacques, p.c.)
\end{exe}


\subsubsection{\# Marginality}\label{appendixJaphugMarginal}
\begin{itemize}
	\item \textit{Pɤjkhu} seems not to allow for a marginality construal; see (\ref{exAppendixJaphugMarginal1}, \ref{exAppendixJaphugMarginal2}).
\end{itemize}

\begin{exe}
	\ex Context: Talking about skills in a sport\label{exAppendixJaphugMarginal1}\\
	\gll \textbf{Pɤjkʰu} pjɯ-ɕɯ-nŋam-a cʰa-a.\\
	still \textsc{ipfv}-\textsc{caus}-be\_defeated-1\textsc{sg} can-1\textsc{sg}\\
	\glt \lq Currently (but not necessarily later), I can still beat him.ʼ\\
	not: \lq He still falls within the range of those I can beat.ʼ (Guillaume Jacques, p.c.)
	
	\ex\label{exAppendixJaphugMarginal2}
	\gll Maoxian nɯ kɯrɯ sɤtɕha maʁ ri, Lixian nɯ \textbf{pjɤkʰu} kɯrɯ sɤtɕha kɤ-rtsi ŋu.\\
	M. \textsc{dem} Tibetan area \textsc{neg}.\textsc{cop} but L. \textsc{dem} still Tibetan area \textsc{inf}-count \textsc{cop}\\
	\glt \lq Maoxian is not a Tibetan area, but Lixian still counts as a Tibetan area (that might change in the future).'\\
	not: \lq Maoxian is not a Tibetan area, but Lixian is still (i.e. qualifies as a marginal member) of a Tibetan area.' (Guillaume Jacques, p.c.)
\end{exe}
\il{Japhug|)}

\section{Ket (ket, ket1243)}\il{Ket|(}
\label{appendixKet}
\subsection{Introductory remarks}
I am indebted to Andrey Nefedov, Stefan Georg, and Heinrich Werner for discussing Ket data with me and for helping with glosses.

\subsection{hāj}
\subsubsection{General information}
\begin{itemize}
	\item Form: often reduced to \textit{āj} or \textit{hɨ̄}; there is also a variant \mbox{\textit{hās}(\textit{ja})}.
	\item Wordhood: independent grammatical word.
	\item Syntax: judging from the available data, \textit{hāj} generally precedes its focus, which in the phasal polarity function is the main predicate.
\end{itemize}


\subsubsection{As a \lq{}still\rq{ }expression}
\begin{itemize}
	\item \textcite[45]{Donner1955}, \textcite[144]{Georg2007}, \textcite[177]{KotorovaNefedov2015}, \textcite[41]{Vajda2004}, and \citeauthor{Werner1997} (\citeyear[71, 145]{Werner1997}, \citeyear[292]{Werner2002}); additional discussion throughout \textcite{vanBaar1997}.
	\item Specialisation: \textcite{vanBaar1997} identifies this marker as one that is in line with my definition. 
	\item Pragmaticity: compatible with both scenarios \parencite[76–77]{vanBaar1997}.
	\item Polarity sensitivity: inner negation yields \textsc{not yet}.
\end{itemize}

\begin{exe}
	\ex
	\gll Ít-iŋ \textbf{hɨ} ad-a-den-qaka áàŋ ūl d-a-b-dob.\\
	tooth-\textsc{pl} still hurt:\textsc{subj}.3.\textsc{n}-\textsc{prs}-go-when hot water \textsc{subj}.1-\textsc{prs}-\textsc{obj}.3.\textsc{n}-drink\\
	\glt \lq When the teeth still hurt, Iʼm drinking hot water.\rq{ }(\cite[90]{Grishina1979}, cited in \cite[193]{Nefedov2015})

	\ex
	\gll Bu \textbf{haj} kis'ɛŋ.\\
	3\textsc{sg} still is\_here\\
	\glt \lq He is still here.' \parencite[306]{vanBaar1997}
\end{exe}

\subsubsection{Broadly adverbial temporal-aspectual functions}

\paragraph{Iterative}
\label{appendixKetIterative}
\begin{itemize}
	\item \textcite[45]{Donner1955}, \textcite[311]{Georg2007}, \textcite[177]{KotorovaNefedov2015}, \textcite[96]{Nefedov2015} and \textcite[292]{Werner2002}.
	\item There appear to be no restrictions concerning aspectual viewpoints.
	\item No clear-cut cases of restitution are attested in the data.
\end{itemize}
\begin{exe}
	\ex 
	\gll Ād énqoŋ kəˀj d-il-aq, \textbf{hāj} di-sel-q-ej.\\
	1\textsc{sg} today hunt \textsc{subj}.1-\textsc{pst}-go still \textsc{subj}.1-raindeer-\textsc{pst}-kill\\
	\glt \lq Today I went hunting, again I killed a reindeer.\rq{ }(\cite[113]{Krejnovic1969}, cited in \cite[230]{Georg2007}).

	\ex\label{appendixKetIterative2}
	\gll  \textbf{Haj} d-iˑ-m-bɛsʲ.\\
	still \textsc{subj}.3-here-\textsc{pst}-move\\
	\glt \lq (He) came again.' (\cite[292]{Werner2002}; glosses by \cite[96]{Nefedov2015})
\end{exe}

\paragraph{Iterative and restitutive via increment}
\label{appendixKetIterativeIncrement}
\begin{itemize}
\item \textcite[177]{KotorovaNefedov2015} and \citeauthor{Werner1997} (\citeyear[145, 381]{Werner1997}; \citeyear[292]{Werner2002}); additional discussion in \textcite[306]{vanBaar1997}.
\item This function occurs in two collocations. The first is \textit{haj} plus \textit{biks'a} \lq different, again' (\ref{exAppendixKetIterativeIncrement1}, \ref{exAppendixKetIterativeIncrement2}). Ex. (\ref{exAppendixKetIterativeIncrement2}) is clearly restitutive. The second collocation is \textit{haj} plus \textit{s'in'} \lq once\rq{ }(\ref{exAppendixKetIterativeIncrement3}).
\end{itemize}

\begin{exe}
	\ex\label{exAppendixKetIterativeIncrement1}
	\gll \textbf{Haj} \textbf{bíksʼa} aˑnʼiŋ-aŋ-g-ɔ́Rɔn.\\
	still different/again play-\textsc{subj}.3\textsc{pl}-\textsc{det}-\textsc{pst}-become\\
	\glt \lq Sie begannen wieder zu spielen. [They began playing again.]'
	\\(\cite[292]{Werner2002})
	
	\ex\label{exAppendixKetIterativeIncrement2}
	Context: One brother has broken the other one (Tuta) into pieces.
	\exi{}\gll Bū ít-ò-l-am tútà \textbf{haj} \textbf{bíksà} sēn d-éèt-a\\
	3\textsc{sg} sense-\textsc{subj}.3\textsc{m}-\textsc{pst}-take T. still different/again one \textsc{subj}.3\textsc{m}-alive-\textsc{dur}.event\_extends\\
	\glt \lq He knew Tuta would come back to life once again.' \parencite[94]{Vajda2004}
	
	\ex\label{exAppendixKetIterativeIncrement3}
	\gll \textbf{Haj} \textbf{sʼinʼ} t-á-nʼgi!\\
	still once \textsc{det}-\textsc{th}-\textsc{imp}-say\\
	\glt \lq Sag (es) noch einmal! [Say it again!]\rq{ }(\cite[292]{Werner2002}, glosses added)
\end{exe}

\subsubsection{Additive and related functions}
\paragraph{Additive}\label{appendixKetAdditive}
\begin{itemize}
	\sloppy
	\item \textcite[311]{Georg2007}, \textcite[177]{KotorovaNefedov2015}, \textcite[96]{Nefedov2015} and \textcite[292]{Werner2002}; additional discussion is found in \textcite[306–307]{vanBaar1997}.
	\item This function extends into equative comparisons (\ref{exAppendixKetAlso4}).
\end{itemize}

\begin{exe}
	\ex
	\gll Āt \textbf{haj} kʌnɛsʲ-ket.\\
	1\textsc{sg} still	light-person\\
	\glt \lq Ich bin auch ein Mensch dieser Welt. [I am also a man of this world.]\rq{ }(\cite[292]{Werner2002}; glosses by \cite[96]{Nefedov2015})
	
	\ex\label{exAppendixKetAlso4}
	\gll Bilʲa d-i:-n-bεsʲ, tɔʔn \textbf{hāj} du-γ-a-daq.\\
	like \textsc{subj}.3\textsc\textsc{sg}-here-\textsc{pst}-move thus still \textsc{subj}.3\textsc{m}-\textsc{lnk}-\textsc{th}-live\\
	\glt \lq как приехал, так и [тоже] живёт. [He lives the way he came, lit. the way he came, that way he also lived.]\rq{ }(\cite[177]{KotorovaNefedov2015}, glosses added)
\end{exe}

\paragraph{Scalar additive(?)}\label{appendixKetScalarAdditive}
\begin{itemize}
	\item There is only one candidate for this function in the data consulted. Andrey Nefedov (p.c.) indicates that this could be a calque on \ili{Russian} \textit{i}.
\end{itemize}

\begin{exe}
	\ex \gll Bu-ŋ-s-ɔʁɔ-dāsʲ, bū kɛˀt \textbf{hāj} du-ɣa-jɛj.\\
	\textsc{subj}.3-\textsc{th}-\textsc{prs}-search\_for-when 3\textsc{sg} person still \textsc{subj}.3-\textsc{obj}.3\textsc{m}-kill\\
	\glt \lq When he looks, he can even kill a man.' \parencite[173]{Nefedov2015}
\end{exe}


\paragraph{Comparisons of inequality}\label{appendixKetComparisons}
\begin{itemize}
	\item \textcite[138]{Georg2007}, \textcite[177]{KotorovaNefedov2015} and \textcite[124]{Werner1997}.
	\textit{Hāj} (or a variant form) adds the notion of \lq even\rq{ }to comparisons of inequality.
\end{itemize}
\begin{exe}
	\ex
	\gll \textbf{Ha˙s’} s’ul’em-s’.\\
	still red-\textsc{nmlz}\\
	\glt \lq Noch röter [even more red].' (\cite[124]{Werner1997}, glosses added)
	
	\ex
	\gll \textbf{Hāj} qà da-éjs-a-ʁɔt ba:t-daŋa .\\
	still big/very \textsc{subj}.3\textsc{sg}.\textsc{f}-up-\textsc{th}-\textsc{root} old\_man-\textsc{dat}\\
	\glt \lq ещё сильнее ругается на старика [She scolded the old man even more].'\footnote{-\textit{ʁɔt} is a semantically near-empty root.} (\cite[177]{KotorovaNefedov2015}, glosses added)
\end{exe}

\paragraph{Constituent coordination}\label{appendixKetCoordination}
\begin{itemize}
	\item \textcite[311]{Georg2007}, \textcite[96–100, 108]{Nefedov2015}, \textcite[85]{Vajda2004} and \citeauthor{Werner1997} (\citeyear[318–323]{Werner1997}, \citeyear[292]{Werner2002}); additional discussion in \textcite[306]{vanBaar1997}.
	\item In this function, \textit{hāj} can coordinate nouns (\ref{exAppendixKetCoordinator1}), adjectives  (\ref{exAppendixKetCoordinator2}), adverbs (\ref{exAppendixKetCoordinator3}), verbs (\ref{exAppendixKetCoordinator4}), and parallel clauses (\ref{exAppendixKetCoordinator5}).
	\item \textcite[96]{Nefedov2015} suggests that this is, diachronically speaking, an extension of the additive function. This is in line with typological findings on the sources of coordinators (see \cite{Forker2016}; \cite[58–59]{KutevaEtAl2019}, and references therein).
\end{itemize}
\begin{exe}
	\ex\label{exAppendixKetCoordinator1}
	\gll ə̄tn, assanɔ dɛˀŋ \textbf{haj} isqɔ dɛˀŋ,	haj kiˀ dʌˀq di-b-bɛt-in\\
	1\textsc{pl} hunt.\textsc{nmlz} people still fish.\textsc{nmlz} people still new live.\textsc{nmlz} \textsc{subj}.1-\textsc{obj}.3\textsc{n}-make-\textsc{pl}\\
	\glt \lq We, hunters and fishermen, also build a new life.' \parencite[97]{Nefedov2015}

	\ex\label{exAppendixKetCoordinator2}
	\gll Hʌna \textbf{haj} qē-ŋ dɨlʲgat škɔla-di-ŋa ɔŋ-ɔ-tn\\
	small still big-\textsc{pl} children school-\textsc{n}-\textsc{dat} \textsc{subj}.3\textsc{pl}-\textsc{prs}-go\\
	\glt \lq Kleine und große Kinder gehen in die Schule. [Small and big children go to school.]\rq{ }(\cite[321]{Werner1997}; glosses by \cite[97]{Nefedov2015})
 
	\ex\label{exAppendixKetCoordinator3}
	\gll Būŋ aqta \textbf{haj} dʌqtɛ t-lʲɔvɛr-a-vɛt-in\\
	3\textsc{pl} good still	fast \textsc{subj}.3-work-\textsc{prs}-\textsc{iter}-\textsc{pl}\\
	\glt \lq Sie arbeiten gut und schnell. [They work well and fast.]\rq{ }(\cite[321]{Werner1997}; glosses by \cite[97]{Nefedov2015})
 
 	\ex\label{exAppendixKetCoordinator4}
 	\gll Dɨlʲ duɣ-a-ɣ-ɔ-ʁɔn \textbf{hāj} qɔʁ-a-ʁ-ɔ-n\\
	child shout.\textsc{nmlz}-\textsc{subj}.3\textsc{m}-\textsc{th}-\textsc{pst}-become still cry.\textsc{nmlz}-\textsc{subj}.3\textsc{m}-\textsc{pst}-become\\
	\glt \lq The child began shouting and (began) crying.'  \parencite[97]{Nefedov2015}
	
	 	\ex\label{exAppendixKetCoordinator5}
	Context: Two brothers are being served fatty meat by a witch.\\
	\gll Éɾȕla ánùn-tu-ɾu bū òn īs bə̄n d-b-īl-[a]. A Tútà-da-ŋa ánùn bənsàŋ bū ísqàl-s óvɨ̀lde \textbf{haj} bɨ́ldè ba d-b-īl-[a].\\
	E. mind-\textsc{adj}-\textsc{pred}.\textsc{m} 3\textsc{sg}.\textsc{m} much meat \textsc{neg} \textsc{subj}.3\textsc{m}-\textsc{obj}.3\textsc{n}-\textsc{pst}-eat but T.-\textsc{m}-\textsc{dat}  mind not.be.\textsc{prs} 3\textsc{sg}.\textsc{m} greedy-\textsc{nom} was still everything customarily \textsc{subj}.3\textsc{m}-\textsc{obj}.3\textsc{n}-\textsc{pst}-eat\\
	\glt \lq Erula was smart and didn't eat much of the meat. But Tuta was stupid. He was greedy and would always eat it all up.\rq{ }\parencite[93]{Vajda2004}
\end{exe}
	


\subsubsection{Broadly modal and interactional uses}
\paragraph{Discourse marker in questions}\label{appendixKetQuestions}
\begin{itemize}
	\item \textcite{Belimov1976} and \textcite[317]{Werner1997}.
	\item This function is restricted to wh-questions. \textcite{Belimov1976} describes this (and other markers) as supplying an emotive character to interrogatives. Similarly, \textcite{Werner1997} translates it into \ili{German} as \textit{denn}, a marker commonly described as highlighting a follow-up question (e.g. \cite{Wegener2001}). This is in line with the one contextualised example in the data (\ref{exAppendixKetQuestion1}) being a follow-up question after an unsatisfactory initial response.
	\item In all examples, \textit{hā̄j} precede the predicate as with, among other things, the iterative use (\appref{appendixKetIterative}). An extension of the latter function to the speech-act level would be in line with what has been described for \ili{French} \textit{encore} (\appref{appendixFrenchEncoreQ}).
\end{itemize}

\begin{exe}
\ex\label{exAppendixKetQuestion1}
	\begin{xlist}
	\sloppy
	\exi{A:} \textit{Bilʼaŋsän\rq{} diˑmbesʼin? Dɨ l\rq{}gat u škɔladiŋal\rq{} diˑmbes\rq{}in?}\\
	\lq Wer ist gekommen? Ob es Kinder sind, die aus der Schule gekommen sind? [Who has come? Will it be children, coming from school?]\rq
	\exi{B:} \textit{Bəŋ.}\\
	\lq Nein. [No.]\rq
	\exi{A:}
	\gll Anεt \textbf{haj} d-iˑ-m-bes\rq{}?\\
	who still \textsc{subj}.3-here-\textsc{pst}-move\\
	\glt \lq Wer ist denn gekommen? [Who is it then, who just came?]\rq{ }
	\exi{A:} \textit{Kaˀt hiˑɣ diˑmbes}\rq{}.\\
	\lq Ein alter Mann ist gekommen [An old man came.]\rq{ }(\cite[366–367]{Werner1997}, glosses added)
	\end{xlist}
	
	\ex
	\gll Baːm dativij: \lq\lq Baːt \textbf{haj} birɔ?"\\
	old\_woman she\_asked \phantom{\rq\rq}old\_man still where\\
	\glt \lq старуха спрашивает: а где же старый? [The old woman asked: \lq\lq But where's the old man?"]'
	(\cite[25]{Belimov1976}; \cite[319]{Werner1997}, glosses added)
\end{exe}
\il{Ket|)}

\section{Lezgian (lezg1247, lez)}\label{appendixLezgian}\il{Lezgian|(}
\subsection{Introductory remarks}
Lezgian has two (sets of) \textsc{still} expressions: the first group is constituted by \textit{ama} \lq still:\textsc{loc}.\textsc{cop}' and the verb suffix \mbox{-C(\textit{a})\textit{ma}}, which goes back to periphrastic constructions involving \textit{ama}. In the data consulted, there are no indications of additional functions for \textit{ama} and \mbox{-C(\textit{a})\textit{ma}}. Both forms ultimately go back to \textit{mad}, which forms part of the expression of \textsc{no longer}, and has a range of additional functions, including iterative and additive ones. In the data consulted, \textit{mad} is not attested as an affirmative phasal polarity expression. Lastly, there is \textit{hele}(-\textit{lig}). The latter is primarily an \textsc{already} expression in the present-day language, and only serves as an exponent of \textsc{still} in combination with \mbox{-C\textit{ma}}.

\subsection{hele(lig)}
\label{appendixLezgianHele}

\subsubsection{General information}
\begin{itemize}
	\item Form: transcribed as гьеле in Cyrillic. There is a free variant with the Turkic suffix \mbox{-\textit{lig}}. 
	\item Wordhood: free morpheme.
	\item Etymology: ultimately from Arabic \textit{hāl}, in all likelihood mediated via Turkic.
	\item Further note: as discussed below, this item only marginally serves to mark \textsc{still} in the present-day language; it is primarily an \textsc{already} expression (the latter function, however, seems not to be attested for the long form \textit{helelig}). As discussed by \citeauthor{vanderAuwera1993} (\citeyear{vanderAuwera1993},  \citeyear{vanderAuwera1998}) it is safe to assume that the concept of \textsc{still} was once the primary denotatum of \textit{hele}(\textit{lig}); see \Cref{sectionInterrogativeYet} for a general discussion of the diachronic chain between \textsc{still}, \textsc{not yet} > \textsc{already}.
\end{itemize}


\subsubsection{As a \lq{}still\rq{ }expression}
\begin{itemize}
	\item \citeauthor{Haspelmath1991} (\citeyear{Haspelmath1991}, \citeyear[145, 210]{Haspelmath1993}) and \textcite[102]{TalibovGadziev1966};  further discussion in \citeauthor{vanderAuwera1993} (\citeyear{vanderAuwera1993},  \citeyear{vanderAuwera1998})  and \textcite[195–197]{vanBaar1997}.
	\item Specialisation: that this marker once functioned as an expression of \textsc{still} is only indirectly retrievable, namely from its extension into \textsc{already} via (\textsc{not yet} > ) interrogative \lq yet\rq{ }(\appref{appendixLezgianHeleInterrogativeYet}), and by comparison with cognates across neighbouring languages.
	\item Pragmaticity: \citeauthor{Haspelmath1991} (\citeyear{Haspelmath1991}, \citeyear[210]{Haspelmath1993}) describes \textit{hele}(\textit{lig}) as giving emphasis to \mbox{-C\textit{ma}}; this might indicate that this combination serves to indicate the unexpectedly late scenario.
	\item Polarity sensitivity: inner negation yields \textsc{not yet}; the vast majority of examples involve bare \textit{hele}, without \mbox{-\textit{lig}}.
	\item Further note: as noted above, this expression does not (anymore) serve as \textsc{still} in its own right.
\end{itemize}

\subsubsection{Uses related to other phasal polarity concepts}
\paragraph{Interrogative \lq yet\rq{ }(and \textsc{already})}
\label{appendixLezgianHeleInterrogativeYet}
\begin{itemize}
	\item \textcite{Haspelmath1991}; further discussion in \citeauthor{vanderAuwera1993} (\citeyear{vanderAuwera1993},  \citeyear{vanderAuwera1998})  and \textcite[195–197]{vanBaar1997}.
	\item Form: in this function, the suffix \mbox{-\textit{lig}} appears to not occur.
	\item This function is found with indirect (\ref{exAppendixLezgianHeleInterrogativeYet1}) and direct (\ref{exAppendixLezgianHeleInterrogativeYet2}) questions.
	\item Ex. (\ref{exAppendixLezgianHeleInterrogativeYet3}) illsutrates \textit{hele} as an \textsc{already} expression.
\end{itemize}
\begin{exe}
	\ex\label{exAppendixLezgianHeleInterrogativeYet1}
	\gll Zun Jusuf Derbentd-aj \textbf{hele} qʰfe-nwa-j-da-l šaklu že-zwa.\\
	1\textsc{sg}.\textsc{abs} J. Derbentl-out\_of still leave-\textsc{ant}-\textsc{ptcp}-\textsc{nmlz}-on doubting \textsc{cop}-\textsc{ipfv}\\
	\glt \lq I doubt whether Jusuf has left Derbent yet.' \parencite[84]{Haspelmath1991}

	\ex\label{exAppendixLezgianHeleInterrogativeYet2}
	\gll Jusuf Derbentd-aj \textbf{hele} qʰfe-na-ni?\\
	J. Derbent-out\_of still leave-\textsc{aor}-\textsc{q}\\
	\glt \lq Has Jusuf left Derbent yet?' \parencite[84]{Haspelmath1991}
	
	\ex\label{exAppendixLezgianHeleInterrogativeYet3}
	\gll Jusuf Derbentd-aj \textbf{hele} qʰfe-na.\\
	J. Derbendt-out\_of still leave-\textsc{aor}\\
	\glt \lq Jusuf has already left Derbent.\rq{ }\parencite[210]{Haspelmath1993}
\end{exe}

\subsubsection{Temporal connectives and frame setters}
\paragraph{Persistent time frame}
\label{appendixLezgianContinuative}
\begin{itemize}
	\item \textcite{Haspelmath1991}.
	\item There is only one clear example in the data.
\end{itemize}

\begin{exe}
	\ex\label{exAppendixLezgianContinuative}
	\gll \textbf{Hele} zun akwa-daldi, am qʰfe-na.\\
	still 1\textsc{sg}.\textsc{abs} see-\textsc{cvb}:before 3\textsc{sg}.\textsc{abs} leave-\textsc{aor}\\
	\glt \lq Even before she saw me, she left. / Noch ehe sie mich sah, ging sie weg.' \parencite[90]{Haspelmath1991}
\end{exe}

\paragraph{Time-scalar (\lq as early/late as\rq)}
\label{appendixLezgianTimeScalar}
\begin{itemize}
	\item \citeauthor{Haspelmath1991} (\citeyear{Haspelmath1991}, \citeyear[240]{Haspelmath1993}).
	\item \textit{Hele} functions as a time-scalar additive marker that operates on a scale of time proper and is compatible both with earlier alternatives (\ref{exAppendixLezgianTimeScalar1}) and later ones (\ref{exAppendixLezgianTimeScalar2}-\ref{exAppendixLezgianTimeScalar4}). That is, it can be understood as underspecified for a relative rank on the time scale.
	\item In all likelihood, the availability of  the \lq as early as\rq{ }use is at least partially due to the fact that this item has developed from a \textsc{still} expression into an \textsc{already} expression. At the same time, scalar inferences from the persistent time frame use (\appref{appendixLezgianContinuative}) might also have played a role; thus compare (\ref{exAppendixLezgianContinuative}) above to (\ref{exAppendixLezgianTimeScalar4}).
	\item Syntax: \textcite{Haspelmath1991} points out that \textit{hele} in this use behaves like a typical Lezgian focus particle, occurring together with the focus in (\ref{exAppendixLezgianTimeScalar3}), where it stands in opposition to \textit{anžax}/\textit{tek} \lq only\rq{}.
\end{itemize}

\begin{exe}
	\ex\label{exAppendixLezgianTimeScalar1}
	\gll \textbf{Hele} cʼerid lahaj asir.d-a fikir-zawa-j x̂i, …\\
	still seventeen \textsc{ord} century-\textsc{iness} think-\textsc{ipfv}-\textsc{pst} \textsc{dm}\\
	\glt \lq As late as/even in the seventeenth century people (still) believed that … / Noch im siebzehnten Jahrhundert hat man gedacht dass…' \parencite[90]{Haspelmath1991}
	
	\ex\label{exAppendixLezgianTimeScalar2}
\gll Jusuf \textbf{hele} naq' ata-na.\\
Jusuf already yesterday come-\textsc{aor}\\
\glt \lq Jusuf came as early as yesterday / Jusuf ist schon gestern gekommen.' \parencite[85]{Haspelmath1991}

	\ex\label{exAppendixLezgianTimeScalar3}
\gll Am mus ata-na? – \textbf{hele} naq'\\
3\textsc{sg}.\textsc{abs} when come-\textsc{aor} {} still yesterday\\
\glt \lq When did she come? -- As early as yesterday / Schon gestern.' \parencite[85]{Haspelmath1991}

	\ex\label{exAppendixLezgianTimeScalar4}
\gll \textbf{Hele} Oktjabrdi-n inq̄ilab že-daldi Stʼal Sulejman wič ustad satirik jaz q̄alur-na-j.\\
still october-\textsc{gen} revolution \textsc{cop}-\textsc{cvb}:before S. S. self master satirist as show-\textsc{aor}-\textsc{pst}\\
\glt \lq Even before the October Revolution (happened) Sulejman St'al had shown himself as a master satirist. / Schon bevor die Oktoberrevolution geschah, hatte Sulejman St'al sich als Meistersatiriker gezeigt.' \parencite[86]{Haspelmath1991}
\end{exe}
\il{Lezgian|)}

\section{Mandarin Chinese (cmn, mand1245)}
\il{Chinese, Mandarin|(}
\label{appendixMandarin}

\subsection{Introductory remarks}
I am indebted to Zhuang Chen for long discussions of Mandarin data with me, and for providing additional examples. Further thanks go to Dan Ke for helping standardize tone annotations. Note that several functions of \textit{hái} involve the combination with copula \textit{shì}. This type of \lq\lq compounds" involving \textit{shì} (originally introducing an embedded predicate) occurs with several adverbials in Mandarin; see \textcite[311–312]{Wiedenhof2015}.


\subsection{hái}
\label{appendixMandarinHai}
\subsubsection{General information}
\begin{itemize}
	\item Form: {\cn 还} in Chinese characters.
	\item Wordhood: independent grammatical word.
	\item Syntax: fixed position, immediately preceding the predicate.
	\item Etymology: < \textit{huan} \lq go/come back'.
\end{itemize}


\subsubsection{As a \lq{}still\rq{ }expression}
\begin{itemize}
\item 	\textcite[ch. 2.3]{Alleton1972}, \textcite[106–117]{Chu1998}, \textcite[122–127, 142–147]{Donazzan2008}, \textcite{JingSchmidtGries2009}, \textcite[334, 345]{LiThompson1981}, \textcite{Liu2000}, \textcite{Lu2019}, \textcite{Paris1988},  \textcite{Yang2017}, \textcite{Yeh1998}, and \textcite{ZhangLing2016}, among many others.
		\item Form: may combine with copula \textit{shì}, yielding \textit{háishì}, as in (\ref{exAppendixMandarin4}).
	\item Specialisation: the numerous descriptions, when taken together, clearly show that this marker conforms to my definition; \textcite[145]{Donazzan2008} and \textcite[265]{Yeh1998} explicitly discuss the incompatibility of \textit{hái} with inalterable states.
	\item Pragmaticity: compatible with both scenarios. According to \textcite{Paris1988}, the unexpectedly late scenario is marked by sentence-final particle \textit{ne}. \textcite{Chu1998} and \textcite{Liu2000}, on the other hand, state that it is made salient by stress on \textit{hái}.
	\item Polarity sensitivity: inner negation yields \textsc{not yet}.
\end{itemize}

\begin{exe}
	\ex 
	\gll Qùnián fēijī chūshì zài qǐfēi de bàn ge zhōngtóu yǐnèi \textbf{hái} yǒu diànhuà, yǐhòu jìu méi liánluò le.\\
	last\_year airplane have\_accident at take\_off \textsc{assoc} half \textsc{clf} hour within still \textsc{exist} telephone after then \textsc{neg} contact \textsc{sfp}\\
	\glt \lq In last year's plane accident, for an half hour after takeoff there was still telephone contact, then (the changed situation was that) there was no more contact.' \parencite[259]{LiThompson1981}
		\ex

\gll Nǐ 	dōu	èrshí	háojǐ	de	rén	le,	zěnme		\textbf{hái}	zài	chī	fùmǔ?\\
	2\textsc{sg} all twenty several \textsc{assoc} person \textsc{pfv} how still \textsc{prog} eat parents\\
	\glt \lq You are over twenty now. How can you still live off your parents?\rq{ }\parencite[84]{Li2016}
		
	\ex \label{exAppendixMandarin4}
	\gll Nǐ \textbf{hái}-\textbf{shi} zhēn de tài xiǎo le.\\
	2\textsc{sg} still-\textsc{cop} real \textsc{assoc} too young \textsc{sfp}\\
	\glt \lq You are really much too young still.\rq{ }(\cite[339]{Wiedenhof2015}, glosses added)
\end{exe}


\subsubsection{Uses on the fringes of \lq{}still\rq{}}
\paragraph{Scalar contexts}\label{appendixMandarinScalar}
\begin{itemize}
	\item \textcite[ch. 3.2]{Alleton1972}, \textcite{Lai1999}, \textcite[312]{Wiedenhof2015}.
	\item \textit{Hái} is compatible with contexts of decreases over time (\ref{exAppendixMandarinDecrement1}, \ref{exAppendixMandarinDecrement2}). On the other hand, \textit{hái} appears to be odd in contexts of increase functions, where
another item, \textit{cái} is used instead	(\ref{exAppendixMandarinIncrement1}, \ref{exAppendixMandarinIncrement2}). The latter is a restrictive \lq only\rq{ }marker which, though not inherently scalar in all of its uses (see \cite[ch 4.1]{Hole2004} for discussion), is strongly associated with scalar contexts (e.g. \cite{Lai1999}).

\end{itemize}
\begin{exe}
	\ex\label{exAppendixMandarinDecrement1}
	\gll Nǐ	\textbf{hái}	shèng		duōshǎo	qián?\\
	2\textsc{sg} still remain how\_much money\\
	\glt \lq How much money is left with you?' \parencite[219]{Shi2016}
	
	\ex\label{exAppendixMandarinDecrement2}
	\gll Xiànzài \textbf{hái} yǒu duōshao rén, hái yǒu liǎng sān wàn zài lǐmian?\\
	now still \textsc{exist} how\_much person still \textsc{exist} pair three 10,000 in  inside\\
	\glt \lq How many people are there still, some twenty or thirty thousand are still in prison?' \parencite[279]{Wiedenhof2015}

	\ex\label{exAppendixMandarinIncrement1}
	Context: We are supposed to be sent a total of 5 books each.
	\begin{xlist}
		\exi{A:} I've received four.
		\exi{B:} I've received four, too.
		\exi{C:}	\gll Wǒ	\textbf{cái}	shōu-dào	sān	běn.\\
		1\textsc{sg} only receive-arrive three \textsc{clf}\\
		\glt \lq I've only received three so far.\rq{ }(Zhuang Chen, p.c.)
	\end{xlist}

	\ex\label{exAppendixMandarinIncrement2}
	\gll Zhè	gè sài-jì,	wǒmén duì	\textbf{cái}	wán	le	yì	chǎng	bǐsài.\\
	\textsc{prox} \textsc{clf} match-season 1\textsc{pl} team only play \textsc{pfv} one \textsc{clf} match\\
	\glt \lq So far, our team has only played one game this season.\rq{ }(Zhuang Chen, p.c.)
\end{exe}


\subsubsection{Uses related to other phasal polarity concepts}
\paragraph{\# Interrogative \lq yet\rq{}}
\begin{itemize}
	\item \textit{Hái} does not have uses as interrogative yet, neither in direct, nor in indirect questions (Zhuang Chen, p.c.).
\end{itemize}

\subsubsection{Broadly adverbial temporal-aspectual functions}
\paragraph{Iterative}
\label{appendixMandarinIterative}
\begin{itemize}
	\item \textcite[ch. 3.2]{Alleton1972}, \textcite[122–127]{Donazzan2008}, \textcite{JingSchmidtGries2009}, \textcite{Liu2000}, \textcite{Paris1988}, \textcite{Yeh1998} and \textcite{Zhang2017}.
	\item This function is restricted to future time reference. As a correlate, it often cooccurs with modals such as \textit{yao} \lq want' or \textit{hui} \lq will', as in (\ref{exAppendixMandarinIterative1}, \ref{exAppendixMandarinIterative2}).
\end{itemize}
	
\begin{exe}
	\ex\label{exAppendixMandarinIterative1}
	\gll Míngtiān	\textbf{hái}	huì	xiàyǔ	ma?\\
	Tomorrow still will rain \textsc{q}\\
	\glt \lq Will it rain again tomorrow?' \parencite[61]{HuangShi2016}

	\ex\label{exAppendixMandarinIterative2}
	\gll Lǎowáng	zuótiān	qù	yóuyǒng,	jīntián	\textbf{hái}	yào	qù.\\
	Laowang yesterday go swim today still want go\\
	\glt \lq Laowang went swimming yesterday, and he will go again today.\rq{ }\parencite[55]{Liu2000}
	
	\ex\label{exAppendixMandarinIterative3}
	\gll Míngtiān	\textbf{hái}	chī	miàotiáo,	wǒ	kě	shòubùliǎo	le.\\
	tomorrow still eat noodles 1\textsc{sg} but stand \textsc{sfp}\\
	\glt \lq If (we) eat noodles again tomorrow, I won’t stand it.\rq{ }(\cite[210]{Ma2000}, cited in \cite[34]{JingSchmidtGries2009})
\end{exe}

\subsubsection{Marginality}
\label{appendixMandarinMarginal}
\begin{itemize}
	\item \textcite[ch. 3.2]{Alleton1972}, \textcite[345]{BiqHuang2016}, \textcite[124, 206–207]{Donazzan2008}, \textcite{JingSchmidtGries2009}, \textcite[335]{LiThompson1981}, \textcite{Lu2019}, \textcite{Yang2017}, \textcite{Yeh1998}, \textcite{Zhang2017} and \textcite{ZhangLing2016}.
	\item This includes derogatory comparisons like (\ref{exAppendixMandarinMarginal4}).
	\item Readings of marginality are often accompanied by \textit{suan} \lq consider', i.e. \lq can still be considered to be', as in (\ref{exAppendixMandarinMarginal3}).
	\item Several authors point out that marginal readings of \textit{hái} are incompatible with situations that are usually evaluated negatively: \textit{hai gan-jing} \lq still clean', but \#\textit{hai zang} \lq still dirty' (\cite[264]{Yeh1998}; \cite[56]{Lu2019}, and references therein).
	\item According to \textcite{Yeh1998}, the marginality use shows up eight centuries later than \textit{hái} as \textsc{still}.
	\item As pointed out by \textcite{Paris1988}, marginality readings also appear to be responsible for \textit{hái} in \lq\lq make-do" contexts such as (\ref{exAppendixMandarinMarginal6}). Similarly, the marginality use is likely to motivate the optional employment of \textit{hái} in the collocation \textit{yǔq́i} … \textit{bù rú} \lq rather than A … would (still) be better B', which signals that B is the less bad of the two options, as in (\ref{exAppendixMandarinMarginal7}).
\end{itemize}
\largerpage
\begin{exe}
	\ex\label{exAppendixMandarinMarginal1}
	\begin{xlist}
		\exi{A:}
		\textit{Nǐ	huì	bú	huì	pà	lái	bù	jí?}\\
		\lq Are you afraid that we won’t be there on time?'
		
		\exi{B:}\textit{Huì a.}\\
		\lq Yeah, I am.'
		
		\exi{} …
		
		\exi{A:}
		\gll Huh. Méi guānxī	la.	Wǒ	júede	dào	nà	biān,	yīnggāi	\textbf{hái}	hǎo	a\\
		\textsc{interj} neg relation \textsc{sfp} 1\textsc{sg} think arrive that place should still ok \textsc{sfp}\\
		\glt \lq It’s ok. I think when we arrive there, we should get there on time (lit. … it should still be OK).\rq{ }\parencite[54–56]{Lu2019}
	\end{xlist} 
	
	\ex\label{exAppendixMandarinMarginal2}
	Context: Talking about tennis skills.\\
	\gll Paul, wǒ	\textbf{hái}	dǎ de guò.\\
	P. 1\textsc{sg} still beat \textsc{assoc} pass\\
	\glt \lq Paul I can still beat (but other players are too good for me).\rq{ }(Zhuang Chen, p.c.)
	
	\ex\label{exAppendixMandarinMarginal3}
	%\gll Jintian \textbf{hai} suan liangkuai.\\
	\gll Jīntiān		\textbf{hái}	suàn	liángkuài.\\
	today still consider cool\\
	\glt \lq Today can [still] be considered to be relatively cool.'
	\parencite[67]{Liu2000}
	
	\ex\label{exAppendixMandarinMarginal4}
	\gll Zhèdào	dòufu	bù	zěnme		hǎochī,	zhè	\textbf{hái}	shì	zhèjiā	diàn	zuì	hǎo	de cài	le.\\
	\textsc{prox}-\textsc{clf} tofu \textsc{neg} that tasty \textsc{prox} still \textsc{cop} \textsc{prox}-\textsc{clf} store most good \textsc{assoc} dish \textsc{sfp}.\\
	\glt \lq This tofu dish is not that tasty, and this is already the best dish of the restaurant.' \parencite[60]{Liu2000}
	
	\ex\label{exAppendixMandarinMarginal6}
	\gll Bié	shuō	le,	\textbf{hái}-shì	kuài	zǒu	ba.\\
	\textsc{proh} talk \textsc{pfv} still-\textsc{cop} quickly leave \textsc{sfp}\\
	\glt \lq Don't talk anymore; we had better leave quickly. [i.e. leaving quickly does not solve things, but it's the best we can do right now]\rq{ }\parencite[48]{Liu2000}

	\ex\label{exAppendixMandarinMarginal7}
	\gll Yǔqí	zài	jiē	shàng	xiánguàng,	\textbf{hái}	bù	rú	qù	dǎ	lánqiú.\\
	rather\_than \textsc{cop}.\textsc{loc} street at wander still \textsc{neg} as\_if go play basketball\\
	\glt \lq It is [still] better to play basketball than to wander aimlessly on the street.' \parencite[28]{HuangShi2016}
\end{exe}

\subsubsection{Additive and related functions}
\paragraph{Additive}
\label{exAppendixMandarinAdditive}
\begin{itemize}
	\item \textcite[ch. 2.3]{Alleton1972}, \textcite[344]{BiqHuang2016}, \textcite{Chen2018}, \textcite[106–117]{Chu1998}, \textcite[111–113]{Donazzan2008}, \textcite[344]{HuangShi2016}, \textcite[334–335]{LiThompson1981}, \textcite{Liu2000}, \textcite{Lu2019}, \textcite{Paris1988}, \textcite[88–89]{RossShengMa2014}, \textcite{Yeh1998} and \textcite{ZhangLing2016}, among others.
	\item The focus is the predicate, excluding the subject. That is, additive \textit{hái} cannot be used in contexts of the type \lq A Verb-ed and B, too, Verb-ed'.
	\item \textcite{ZhangLing2016} point out that \textit{hái} as an additive marker is associated with an incremental build-up of discourse, similar to \ili{German} \textit{noch} (\appref{appendixGermanAdditive}). Relatedly, \textcite{JingSchmidtGries2009}, based on \textcite{Ma2000}, point out that additive \textit{hái} requires the relevant situations to fall within the same time span: additive \textit{hái} is infelicitous in contexts like \lq last year \mbox{s/he} did this and this year \mbox{s/he} additionally did that', unless there is a single overarching time span (e.g. \lq in recent years years \mbox{s/he} has been busy…').	
	\item A scalar additive reading (\lq even') only arises with longer enumerations, i.e. it is a contextual inference (\ref{exAppendixMandarinAdditive4}).
	\item In additive function, \textit{hái} often co-occurs with existential \textit{yǒu}, yielding \textit{hái} \textit{yǒu} (\ref{exAppendixMandarinHaiYou}). This is, in fact, a common collocation; according to \textcite[60]{Lu2019} it accounts for more than twenty percent of all tokens of \textit{hái} in the Academia Sinica Balanced Corpus of Modern Mandarin.
	\item According to \textcite{Yeh1998}, this function developed shortly after the iterative one (\appref{appendixMandarinIterative}) and about a century before \textit{hái} as a phasal polarity expression.
	\end{itemize}
\begin{exe}

	\ex
	\gll Tā	chúle	jiāshū,		\textbf{hái}	zuò	yánjiū.\\
	3\textsc{sg} apart\_from teach still do research\\
	\glt \lq En dehors de l'enseignement, il fait de la recherche. [Apart from teaching, he also does research.]\rq{ }\parencite[273]{Paris1988}
	
		\ex
	\gll (Shěnme?	Chàng lě	yí	gè) \textbf{hái} chàng		yí	gè?\\
	\phantom{(}what sing \textsc{pfv} one \textsc{clf} still sing one \textsc{clf}\\
	\glt \lq (What, I sang a song and) I should sing one more?' (\cite[127]{Shen2006}, cited by \cite[56]{JingSchmidtGries2009})
	
	\ex\label{exAppendixMandarinHaiYou}
	\gll Qù Lúndūn, Bālí, \textbf{hái} yǒu Luómǎ.\\
	go London Bali still \textsc{exist} Rome\\
	\glt \lq We are going to London, Bali, and Rome as well.' (\cite[312]{Wiedenhof2015}, glosses added)
	
	\ex\label{exAppendixMandarinAdditive4}
	\gll Zhāngsān dǎsǎo le fángzi, zuò le dàngāo \textbf{hái} yùn le zhuōbù.\\
	Z. sweep \textsc{pfv} house do \textsc{pfv} cake still iron \textsc{pfv} tablecloth\\
	\glt \lq Zhangsan a balayé la pièce, a fait un gâteau et aussi/même repassé la nappe. [Zhangsan swept the room, baked a cake and also/even ironed the tablecloth.]\rq{ }\parencite[113]{Donazzan2008}
\end{exe}

\paragraph{Comparisons of inequality}\label{appendixMandarinComparisons}
\begin{itemize}
	\item \textcite[ch. 2.3]{Alleton1972}, \textcite{Chen2018}, \textcite[193–206]{Donazzan2008}, \textcite{JingSchmidtGries2009}, \textcite{Liu2000}, \textcite{Lu2019}, \textcite{Paris1988}, \textcite{ParisShi2016}, \textcite[187, 335]{Wiedenhof2015}, \textcite{Yang2017}, \textcite{Yeh1998}, \textcite{Zhang2017}, and \textcite{ZhangLing2016}.
	\item Form: in this function \textit{hái} does not combine with copula \textit{shì} (see below on a different collocation, \textit{hái yào}). While examples of \textit{háishì} in comparatives are found (e.g. \cite[326]{BiqHuang2016}; \cite[335]{Wiedenhof2015}), in these cases \textit{háishì} does not contribute to the comparative construction, but provides a counter-expectational and/or concessive reading (Zhuang Chen, p.c.).
	\item \textit{Hái} in comparisons of inequality occurs in two contexts:
	\begin{itemize}
		\item In comparisons of inequality with an overt standard (\ref{exAppendixMandarinComparison1}–\ref{exAppendixMandarinComparison2}). Note that Mandarin Chinese makes use of an exceed-comparative \parencite{Stassen2013} featuring the comparative marker \textit{bǐ}. \textit{Hái} surfaces in the clause featuring the predicate that is compared and adds the notion of \lq even\rq{}.
		\item In the absence of an overtly expressed standard of comparison introduced by \textit{bǐ}, \textit{hái} combines with \textit{yào} \lq will' to give a reading of \lq even more' (\ref{exAppendixMandarinComparison3}).
		\item There are several examples of \textit{hái} of the type in (\ref{exAppendixMandarinComparison4}) that are discussed under the heading of comparisons, but which are best interpreted as involving a combination of the marginality reading of \textit{hái} (\ref{appendixMandarinMarginal}) in combination with an implied comparison. This interpretation is supported by the fact that these cases are compatible with expressions for low degrees such as \textit{yīdǐan} \lq a little' but not with \textit{hěn duō} / \textit{dé duō} \lq very muchʼ (see \cite[66]{Liu2000}).
	\end{itemize}
\end{itemize}
\begin{exe}
	\ex\label{exAppendixMandarinComparison1}
	\gll Yīngbàng	bǐ	méijīn	\textbf{hái}	dà	ma.\\
	pound than dollar still big \textsc{sfp}\\
	\glt \lq{}Pounds have an even higher value than dollars.' \parencite[62]{Lu2019}
	
	\ex\label{exAppendixMandarinComparison2}
	\gll Nǐ	jí	a?	Wǒ	bí	nǐ	\textbf{hái}	jí.\\
	2\textsc{sg} anxious \textsc{sfp} 1\textsc{sg} than 2\textsc{sg} still anxious\\
	\glt \lq You are anxious? I’m even more anxious than you.' (\cite{LiuEtAl2001}, cited by \cite[53]{JingSchmidtGries2009})

	\ex\label{exAppendixMandarinComparison3}
	\gll John	\textbf{hái}	yào	gāo.\\
	J. still will big\\
	\glt \lq John is even taller (than some other person retrievable from context).\rq{ } (Zhuang Chen, p.c.)
	
	\ex\label{exAppendixMandarinComparison4}
	\gll Zhè	fángjiān	\textbf{hái}	gānjìng	yìdiǎn.\\
	this room still clean a\_little\\
	\glt \lq This room is a little cleaner.' \parencite[66]{Liu2000}\\
	More literal translation: \lq (Comparatively speaking), this room is still clean.'
\end{exe}


\paragraph{Conjunctional adverb}\label{appendixMandarinConjunctional}
\begin{itemize}
	\item \textcite[89]{RossShengMa2014}.
	\item In this function, \textit{hái} appears in the collocation \textit{hái yǒu} (see \appref{exAppendixMandarinAdditive} on the latter), giving evidence that this is an extension of the additive function.
	\item Syntax: sentence-initial position.
\end{itemize}
\largerpage
\begin{exe}
	\ex
	\gll Tā de nán péngyou hěn héqi. \textbf{Hái} \textbf{yǒu}, tā hěn shuài!\\
	3\textsc{sg} \textsc{assoc} man partner very friendly still \textsc{exist} 3\textsc{sg} very attractive\\
	\glt \lq Her boyfriend is very friendly. In addition, he is really cute!\rq{ }(\cite[89]{RossShengMa2014}, glosses added)

	\ex
	\gll Rùzhù qián yào fù yājīn. \textbf{Hái} \textbf{yǒu}, bù néng dài chǒngwù.\\
	check\_in before should pay deposit still \textsc{exist} \textsc{neg} able carry pet\\
	\glt \lq You need to pay the deposit before using the room. Also, you're not allowed to have pets here.' (found online, glosses added)\footnote{\url{https://resources.allsetlearning.com/chinese/grammar/Expressing_\%22in_addition\%22_with_\%22haiyou\%22} (18 October, 2021).}
\end{exe}

\paragraph{Disjunctive constituent coordination}\label{appendixMandarinCoordination}
\begin{itemize}
	\item \textcite[11–1121]{Donazzan2008}, \textcite[27]{HuangShi2016}, \textcite{Lu2019}, \textcite[339]{Wiedenhof2015} and \textcite[433–434]{ZhanBai2016}.
	\item Form: normally occurs in combination with copula \textit{shì}, hence \textit{háishì}. Only in very informal speech does \textit{hái} alone serve this function \parencite[39]{Wiedenhof2015}; see (\ref{exAppendixMandarinDisjunctive4}).
	\item This function typically occurs in interrogatives (including indirect questions). However, it is also attested within alternative concessive conditionals (\ref{exAppendixMandarinDisjunctive2}).
	\item \textcite{Lu2019} proposes that this function goes back to a pattern [[(\textit{shì}) p \textit{hái}] [\textit{shì} q]], with additive \textit{hái} linking two propositions, copula \textit{shì} serving as a subordinator (\lq it is the case that \textit{q}') and the disjunctive reading arising as a contextual inference. This pattern then underwent structural reanalysis to [p \textit{háishì} q]. Building on \citeauthor{Lu2019}'s proposal, it can be assumed that the structural reanalysis was accompanied by a conventionalisation of the disjunctive function.
\end{itemize}
\begin{exe}
	\ex Context: From an interview on a radio show. A listener has called in and asked the artists whether further productions from her are to be expected.\\
%	\gll Ni shi zhi wutaiju \textbf{hai}-\textbf{shi} zhi changpian?\\
\gll Nǐ	shì	zhǐ	wútáijù,	\textbf{hái}-\textbf{shì}	zhǐ	chàngpiān?\\
	2\textsc{sg} \textsc{cop} refer stage\_play still-\textsc{cop} refer album\\
	\glt \lq Are you referring to (participation in) stage plays or  [referring to] (releasing) albums?' \parencite[63–64]{Lu2019}

	\ex
	\gll Nǐ zuì xǐhuān lüchá \textbf{hái-shi} huāchá?\\
	2\textsc{sg} most like green\_tea still-\textsc{cop} jasmine\_tea\\
	\glt \lq Tu préfères le thé vert ou bien le thé au jasmin? [Do you prefer green tea or Jasmine tea?]' \parencite[112]{Donazzan2008}
	
	\largerpage[-1]
	\pagebreak
	\ex\label{exAppendixMandarinDisjunctive2}
	%\gll Wu\textsuperscript{2}lun\textsuperscript{4} shi\textsuperscript{4} zai\textsuperscript{4} men\textsuperscript{2}zhen\textsuperscript{3} zai\textsuperscript{4} dian\textsuperscript{4}ti\textsuperscript{1} \textbf{hai\textsuperscript{2}-shi\textsuperscript{4}} zai\textsuperscript{4} bing\textsuperscript{4}fang\textsuperscript{2} dou\textsuperscript{1} mei\textsuperscript{2}you\textsuperscript{3} ren\textsuperscript{2} yin\textsuperscript{1} ta\textsuperscript{1} de\textsuperscript{0} guai\textsuperscript{4}bing\textsuperscript{4} er\textsuperscript{2} qi\textsuperscript{2}shi\textsuperscript{4} ta\textsuperscript{1}.\\
	\gll Wúlùn		shì	zài	ménzhěn,	zài	diàntī,	\textbf{hái}-\textbf{shì}	zài	bìngfáng, dōu	méiyǒu	rén	yīn	tā	de	guàibìng	ér	qíshì	tā.\\
	no\_matter \textsc{cop} at outpatient\_service at elevator still-\textsc{cop} at ward all \textsc{neg} people because 3\textsc{sg} \textsc{assoc} strange\_illness thus discriminate 3\textsc{sg}\\
	\glt \lq No matter if it was in the outpatient services, elevators, or wards, no one discriminated against him because of his rare disease.' \parencite[399]{LinSun2016}
	
	\ex\label{exAppendixMandarinDisjunctive4}
	\gll Nǐ de \textbf{hái} wǒ de?\\
	1\textsc{sg} \textsc{assoc} still 2\textsc{sg} \textsc{assoc}\\
	\glt \lq Mine or yours?\rq{ }(\cite[339]{Wiedenhof2015}, glosses added)
\end{exe}
	



\subsubsection{Broadly modal and interactional functions}
\paragraph{Concessive apodoses}
\label{appendixMandarinConcessiveConsequent}
\begin{itemize}
	\item \textcite[111]{Donazzan2008}, \textcite{Huang2007}, \textcite{JingSchmidtGries2009}, \textcite[637–638]{LiThompson1981}, \textcite{Lu2019}, \textcite{Paris1988}, \textcite{Yeh1998} and \textcite[338]{Wiedenhof2015}. 
	\item Form: in this function, \textit{hái} often combines with copula \textit{shì}, yielding \textit{háishì}, as in (\ref{exAppendixMandarinConcessive3})
\end{itemize}
\begin{exe}
	\ex
	\gll Zhème hǎo de shèr, nǐ \textbf{hái} gěi wàng le!\\
	such good \textsc{assoc} event 2\textsc{sg} still give forget \textsc{pfv}\\
	\glt \lq Such a special occasion, and still you forgot about it!\rq{ }\parencite[149]{Wiedenhof2015}

	\ex\label{exAppendixMandarinConcessive3}
	\gll Suīrán zìjǐ de érzi jiéhūn le, tā \textbf{hái}-\textbf{shi} bu guǎn.\\
	although own \textsc{assoc} son marry \textsc{pfv} 3\textsc{sg} still-\textsc{cop} \textsc{neg} interest\\
	\glt \lq Bien que son propre fils se marie, ̧ca ne l’intéresse (quand mˆeme) pas.' [Although her own son got married, it doesnʼt interest her].\rq{ }\parencite[112]{Donazzan2008}
	
	\ex\label{exAppendixMandarinConcessive4}
	\gll Jíshǐ	tā	gēn	wǒ	shuō	le,	wǒ	\textbf{hái}-\textbf{shì}	bú	dà	xiāngxìn tā	de huà.\\
	even\_if 3\textsc{sg} with 1\textsc{sg} say \textsc{pfv} 1\textsc{sg} still-\textsc{cop} \textsc{neg} big believe 3\textsc{sg} \textsc{assoc} word\\
	\glt \lq Même si'il en parlait avec moi, jes ne le croirais quand même pas beaucomp. [Even if he talked to me about it, I still wouldnʼt believe him much.]\rq{ }\parencite[275]{Paris1988}
\end{exe}

\paragraph{Counter-expectation}\label{appendixMandarinModal}
\begin{itemize}
	\item \textcite[ch. 2.3]{Alleton1972}, \textcite{BiqHuang2016}, \textcite{Liu2000} and \textcite{Yeh1998}.
	\item In this function \textit{hái} \lq\lq indicate[s] the speaker's incredulousness … there is a gap between the situation described in the clause and the assumption or expectation held by him or other people." \parencite[345]{BiqHuang2016}. Its meaning thus strongly points towards \textit{hái} in concessive apodoses (\appref{appendixMandarinConcessiveConsequent}) as the source of this use.
	\item All examples in the data consulted feature \lq\lq bare\rq\rq{ }\textit{hái} without \textit{shì}. \textcite{Alleton1972} describes this use as usually being presented with the intonation of a (rhetorical) question and that it can be paraphrased faithfully by using \textit{nándào} \lq by any chance, isn't it possible that?\rq{}.
	\item Many examples feature additional evaluative material: \textit{zhēn} \lq really' in (\ref{exAppendixMandarinCounterExpectation1}, \ref{exAppendixMandarinCounterExpectation3}, \ref{exAppendixMandarinCounterExpectation5}), \textit{chū	wǒ yìliào} \lq beyond my expectation' in (\ref{exAppendixMandarinCounterExpectation3}). Similarly, sentence\hyp final \textit{a} in (\ref{exAppendixMandarinCounterExpectation5}) has been described as often conveying a notion of surprise \parencite{HuangShi2016}
\end{itemize}

\begin{exe}
	\ex\label{exAppendixMandarinCounterExpectation1}
		\gll Zhēn	ké	yóu	qián,	\textbf{hái}	bù	zǎo	jiāo	le?\\
	really indeed \textsc{exist} money still \textsc{neg} early hand\_over \textsc{pfv}\\
	\glt \lq Si vraiment nous avions de l'argent, comment n'aurions-nous pas payé plus tôt? [If we really had the money, don't you think we'd have paid earlier?]\rq{ }(\cite[16]{Alleton1972}, glosses added)

	\ex\label{exAppendixMandarinCounterExpectation3}
	\gll Wǒ	lǎogōng	de	hóngshāo	yú	\textbf{hái}	zhēn	chū	wǒ yìliào	zhīwài	de 	hǎo.\\
	1\textsc{sg} husband \textsc{assoc} braise\_in\_soy\_sauce fish still really beyond 1\textsc{sg} expectation out\_of \textsc{assoc} good\\
	\glt \lq To my surprise, the braised fish cooked by my husband was exceptionally good!' \parencite[345]{BiqHuang2016}

	\ex\label{exAppendixMandarinCounterExpectation5}
	Context: A is speaking about his diet when living in the Netherlands.\\
	\begin{xlist}
		\exi{A:}… \textit{xiànzài wǒ fēicháng xǐhuān chī qìsi}.\\
		\lq … now I am extremly fond of cheese.'
		\exi{B:} \gll Nà nǐ \textbf{hái} xíguàn de zhēn kuài a!\\
		that 2\textsc{sg} still habit \textsc{assoc} really real \textsc{sfp}:surprise\\
		\glt \lq Well, in that case, you did get used to it really fast!\rq{ }\parencite[110]{Wiedenhof2015}\footnote{Initial \textit{na} contributes a notion along the lines of \lq in that case, if that is so …' \parencite[110]{Wiedenhof2015}}
	\end{xlist}
\end{exe}
\il{Chinese, Mandarin|)}

\section{Northern Qiang (cng, nort2722)}\il{Qiang, Northern|(}
\label{appendixQiang}
\subsection{Introductory remarks}
In addition to the autochthonous Northern Qiang \textsc{still} expression \mbox{\textit{tɕe}-}, the texts in \textcite{LaPollaHuang2003} feature several tokens of \textit{χaiʂə} \lq still is' < Mandarin Chinese \textit{háishì}. Judging from a remark in \textcite[222]{LaPollaHuang2003}, this item primarily serves as a filler. I have thus not included it in my sample.

\subsection{tɕe-}

\subsubsection{General information}
\begin{itemize}
	\item Form: the vowel segment is subject to regressive vowel harmony.
	\item Wordhood: bound morpheme (a prefix that normally occurs on the verb, although in certain contexts it can attach to adjectives).
\end{itemize}


\subsubsection{As a \lq{}still\rq{ }expression}
\begin{itemize}
	\item \textcite[169]{LaPollaHuang2003} and \textcite{Huang2005}.
	\item Specialisation: examples like (\ref{exAppendixQiang1}–\ref{exAppendixQiang3}) give a fairly good indication that this marker conforms to my definition.
	\item Pragmaticity: appears to be compatible with both scenarios (tentative conclusion). (\ref{exAppendixQiang1}, \ref{exAppendixQiang2}) are prime candidates for the unexpectedly late scenario.
	\item Polarity sensitivity: inner negation yields \textsc{not yet}. With negated nominalisation, this serves to construe temporal clauses of precedence \parencite[241]{LaPollaHuang2003}.
\end{itemize}
\largerpage[1.5]
\begin{exe}
	\ex\label{exAppendixQiang1}
	Context: An orangutan wants to eat an orphaned boy.\\
	\gll Dʑy-leː-tɑ dzeke i-ɕtɕi-kəi-tɕu, dʑy-leː i-ɕtɕi χa-lɑ-hɑ thɑ ʑi-kui-ɲiɑu, χa-lɑ-hɑ japə-leː-tɑ ə-tʂə-ŋiɑufu sɑq-phi-keː ə-ʂə-kui-tɕu, tu χaiʂə tɕəu-lɑ ə-ʁə ɕtɕɑq \textbf{tɕo}-lu-kui-ʂə.\\
	door-\textsc{def}.\textsc{clf}-\textsc{loc} \textsc{ideoph} \textsc{dir}-push-\textsc{narr}-\textsc{sfp} door-\textsc{def}.\textsc{clf} \textsc{dir}-push needle-\textsc{def}.one-\textsc{pl} there exist-\textsc{narr}-\textsc{sfp} needle-\textsc{def}.one-\textsc{pl} hand-\textsc{def}.\textsc{clf}-\textsc{loc} \textsc{dir}-stab-\textsc{lnk} blood-flow-\textsc{indef}.\textsc{clf} \textsc{dir}-put-\textsc{narr}-\textsc{sfp} result still\_is(<Chinese) home-\textsc{loc} \textsc{dir}-go heart still-come-\textsc{narr}-\textsc{lnk}\\
	\glt \lq When he pushed the door with a creak, the needle was there, and as soon as the needle pricked his hand, his hand was all covered with blood but he still wanted to go in the room.' \parencite[264, 268]{LaPollaHuang2003}

	\ex\label{exAppendixQiang2}
	\gll Fɑ-tsa-qəi bɑ-hɑŋu̥ə̥lu, ə-lə \textbf{tɕɑ}-nɑ-wa.\\
	clothes-this:one-\textsc{clf} old-although \textsc{dir}-look still-good-very\\
	\glt \lq Although this piece of clothing is old, it still looks very good.\rq{ }\parencite[245]{LaPollaHuang2003}

	\ex\label{exAppendixQiang3}
	\gll Nə-dʐə-m theː \textbf{tɕɑ}-n.\\
	sleep-able-\textsc{nom} 3\textsc{sg} still-sleep\\
	\glt \lq S/he who likes to sleep late is still sleeping.\rq{ }\parencite[228]{LaPollaHuang2003}
\end{exe}

\subsubsection{Broadly adverbial temporal-aspectual functions}

\paragraph{Iterative via increment}
\label{appendixNorthernQiangIterativeIncrement}
\begin{itemize}
	\item \textcite[170]{LaPollaHuang2003}.
	\item There appear to be two different expressions,  \mbox{\textit{ɑ}-\textit{ʂ}} and \mbox{\textit{a}-\textit{thən}}, both glossed as \lq one-time' by \textcite{LaPollaHuang2003}, that yield this function in combination with \mbox{\textit{tɕe}-}.
\end{itemize}

\begin{exe}
	\ex 
	Context: Trying on shoes.\\
	\gll Tse-si xsusu-ɲɑʐguə-χɑu-ɲɑ-pan, ʔũ \textbf{ɑ-ʂ} i-\textbf{tɕi}-tsi-n.\\
	this-pair thirty-\textsc{com}-nine-size-\textsc{com}-half 2\textsc{sg} one-time \textsc{dir}-still-see-2\textsc{sg}\\
	\glt \lq This pair (is a) size 39 ½, you try once again.\rq{ }\parencite[170]{LaPollaHuang2003}
	
	\ex
	Context: Two thirsty men have encountered water. They have drunk a little and encountered a stone slab cover.\\
	\gll ɦala ɕa-laː-ji tə-me-qe-kəi, ha, \textbf{a-thən} sə-\textbf{tɕi}-tɕ-ʂɑm \textbf{a-thən} sə-\textbf{tɕi}-tɕ-kəi, ɦa-la-χui-tu, sə-tɕ, ʂəpɑn tə-qe-ɲiɑufu\\
	\textsc{interj} small-\textsc{def}.\textsc{cl}-\textsc{except} \textsc{dir}-\textsc{neg}-lift-\textsc{narr} \textsc{experiential} one-time \textsc{dir}-still-drink-\textsc{lnk} one-time \textsc{dir}-still-drink-\textsc{narr} \textsc{intens}-\textsc{def}.one-time-\textsc{lnk} \textsc{dir}-drink stone\_slab \textsc{dir}-turn\_over-\textsc{lnk}\\
	\glt \lq They could still only lift it [stone slab] a little. The two drank mouthful of water again and they turned over the stone slab.\rq{ }\parencite[311–312, 327]{LaPollaHuang2003}
\end{exe}

\paragraph{First (?)}\label{appendixNorthernQiangFirst}
\begin{itemize}
	\item  \textcite[194]{LaPollaHuang2003}.
	\item In this function, \mbox{\textit{tɕe}-} occurs with the copula \textit{ŋuə} and a nominalised verb, which is a collocation denoting obligation (see \cite[190–194]{LaPollaHuang2003}).
	\item \textcite[194]{LaPollaHuang2003} state that without the \textsc{still} expression, \lq\lq the form would express an action that had been agreed upon or set beforehand". Based on the available examples, it is hard to interpret the function of this construction. Possibly, it indicates what remains to be done prior to some other action, thus straddling the boundary between precedence (\Cref{sectionFirst}) and a \lq\lq{}further-to\rq\rq{ }use (\Cref{sectionFurtherTo}). Both are attested in the wider vicinity 
(\appsref{appendixJaphugFirst}, \ref{appendixHillsKarbiFurtherTo}).
	
	\item Example (\ref{exAppendixQiangFirst3}) shows that this reading is also available under negation.
\end{itemize}


\begin{exe}
	\ex
	\gll Qɑ kə-s \textbf{tɕa}-ŋu̥ə̥.\\
	1\textsc{sg} go-\textsc{nmlz} still-\textsc{cop}\\
	\glt \lq I (still) must go.' \parencite[194]{LaPollaHuang2003}
	
	\ex
	\gll Theː ləɣz zdə-s \textbf{tɕa}-ŋuə.\\
	3\textsc{sg} book read/study-\textsc{nmlz} still-\textsc{cop}\\
	\glt \lq I (still) must study.' \parencite[194]{LaPollaHuang2003}

	\ex\label{exAppendixQiangFirst3}
	\gll Theː ləɣz zdə-s ma-\textbf{tɕa}-ŋuə.\\
	3\textsc{sg} book read/study-\textsc{nmlz} \textsc{neg}-still-\textsc{cop}\\
	\glt \lq I (still) don't need to study.' \parencite[194]{LaPollaHuang2003}
\end{exe}



\subsubsection{Marginality}
\label{appendixNorthernQiangMarginal}
\begin{itemize}
	\item \textcite[213–214]{LaPollaHuang2003}.
	\item Judging from the few available examples, this gives a reading along the lines of \lq relatively', as is also often found with Mandarin Chinese \textit{hái} (\ref{appendixMandarinMarginal}).
	\item Note that marginality survives under negation, as in (\ref{exAppendixQiangMarginal2}).
\end{itemize}
\begin{exe}
	\ex \gll \textbf{Tɕɑ}-wa.\\
	still-big\\
	\glt \lq Relatively big.' \parencite[214]{LaPollaHuang2003}
	
	\ex
	\gll \textbf{Tɕɑ}-bɑstɑ.\\
	still-slow/late\\
	\glt \lq Relatively slow/late.' \parencite[214]{LaPollaHuang2003}
	
	\ex\label{exAppendixQiangMarginal2}
	\gll Mɑ-\textbf{tɕa}-χtʂa.\\
	\textsc{neg}-still-small\\
	\glt \lq Not so small.'  \parencite[214]{LaPollaHuang2003}
\end{exe}

\subsubsection{Additive and related functions}
\paragraph{Additive}\label{appendixQiangAdditive}
\begin{itemize}
	\item  \textcite{Huang2005} and \textcite[169–170]{LaPollaHuang2003}.
	\item This use encompasses a \lq do some more\rq{ }reading in the prospective aspect inflection (\ref{exAppendixQiangcontinue1}).
\end{itemize}

\begin{exe}
	\ex 
	\gll ʁzə-pies ɑ-fəʴ hɑ-\textbf{tɕi}-ŋu̥ə̥\\
	fish-meat one-portion \textsc{dir}-still-\textsc{cop}\\
	\glt \lq Also bring (give me) a portion of fish.' \parencite[170]{LaPollaHuang2003}
	
	\ex
	\gll Theː ʐdʐytɑː ɦɑ-qɑ me-tɕhi, peitɕin-lɑ dɑ-\textbf{tɕə}-qɑ-kəi.\\
	3\textsc{sg} Chengdu.\textsc{loc} \textsc{dir}-go \textsc{neg}-want Beijing-\textsc{loc} \textsc{dir}-still-go-\textsc{evid}\\
	\glt \lq It seems he not only went to Chengdu, he also went to Beijing.\rq{ }\parencite[209]{LaPollaHuang2003}

	\ex\label{exAppendixNorthernQiangIncrement1}
	\gll A-zə ə-\textbf{tɕi}-z.\\
	one-\textsc{clf} \textsc{dir}-still-eat\\
	\glt \lq Eat some more!\rq{ }\parencite{Huang2005}

	\ex\label{exAppendixNorthernQiangIncrement2}
	\gll ʔile ɑ-za ɑ-\textbf{tɕi}-tɕə-i.\\
	2\textsc{pl} one-\textsc{clf} \textsc{dir}-still-wait-2\textsc{pl}\\
	\glt \lq Donʼt go now (lit. Wait a while longer).' \parencite[170]{LaPollaHuang2003}
	
	\ex\label{exAppendixQiangcontinue1}
	\gll Qɑ \textbf{tɕɑ}-naː.\\
	1\textsc{sg} still-sleep.\textsc{prosp}\\
	\glt \lq Iʼm still going to sleep (I want to sleep some more).\rq{ }\parencite[169]{LaPollaHuang2003}
\end{exe}


\paragraph{Comparisons of inequality}\label{appendixQiangComparisons}
\begin{itemize}
	\item \textcite[213]{LaPollaHuang2003}.
	\item Comparisons of inequality in Northern Qiang follow a pattern [NP\textsubscript{comparee} \mbox{NP\textsubscript{standard}-\textit{s}} \textsc{pred}]. Addition of \mbox{\textit{tɕe}-} yields a great difference in degree (\lq much more\rq{}). It seems that an \lq even more\rq{ }reading is a contextual inference in contexts like (\ref{exAppendixQiangComparison2}), where the standard is explicitly described as having the property in question.
	\item As with Mandarin Chinese \textit{hái} (\appref{appendixMandarinComparisons}), there is also a reading of a relative degree of comparisons, which likely involves the marginality use (\appref{appendixNorthernQiangMarginal}) plus comparison, i.e. \lq compared to … still counts as\rq{}.
\end{itemize}

\begin{exe}
	\ex\label{exAppendixQiangComparison1}
	\gll Qɑ theː-s \textbf{tɕe}-fia.\\
	1\textsc{sg} 3\textsc{sg}-than still-white:1\textsc{sg}\\
	\glt \lq I am lighter (in color) than him (a lot lighter).\rq{ }\parencite[88]{LaPollaHuang2003}
	
	\ex\label{exAppendixQiangComparison2}
	\gll Pəs məpɑ wa, təp-ɲi tsə-s \textbf{tɕɑ}-məpɑː lu.\\
	today cold very tomorrow-\textsc{adv} this-than still-cold.\textsc{prosp} will\\
	\glt \lq Today is very cold; tomorrow is going to be even colder than this.'  \parencite[161]{LaPollaHuang2003}
\end{exe}

\subsubsection{Broadly modal and interactional uses}
\paragraph{Almost}\label{appendixNorthernQiangAlmost}
\begin{itemize}
	\item \textcite[219–220]{LaPollaHuang2003}.
	\item This function obtains in a combination of two clauses, the first of which contains a construction \mbox{\textit{a}-\textit{zə}} \mbox{\textit{tɕa}-\textit{ŋuə}-\textit{ʂə}} \lq one-\textsc{clf} still-\textsc{cop}-\textsc{lnk}.\textsc{cf}'.
	\item This use signals that an event or state nearly came about. As pointed out by \textcite[220]{LaPollaHuang2003}, in this construction the clause linker \mbox{-\textit{ʂə}} contributes a counterfactual meaning, whereas \mbox{\textit{tɕe}-} in all likelihood serves as an additive (\lq had it been a little more'). That is, we find the same function as in (\ref{exAppendixNorthernQiangIncrement1}, \ref{exAppendixNorthernQiangIncrement2}) above.
	\item Only one example is found in \citeauthor{LaPollaHuang2003}'s (\citeyear{LaPollaHuang2003}) grammar.
\end{itemize}
\begin{exe}
	\ex
	\gll \textbf{ɑ-zə} \textbf{tɕɑ-ŋuaː-ʂə}, qɑ i-pə-l mɑ-lə-jya.\\
	one-\textsc{clf} still-\textsc{cop}.\textsc{prosp}-\textsc{lnk} 1\textsc{sg} \textsc{dir}-arrive-come \textsc{neg}-able-\textsc{asp}:1\textsc{sg}\\
	\glt \lq I almost couldnʼt return.' \parencite[219]{LaPollaHuang2003}
\end{exe}
\il{Qiang, Northern|)}

\section{Serbian-Croatian-Bosnian (hbs, sout 1528)}\il{Serbian|(}\il{Croatian|(}\il{Bosnian|(}
\label{appendixBCMS}
\subsection{Introductory remarks}
I am indebted to Jurica Polančec, Marijana Kresić Vukosav, and Stefan Savić for discussing Serbian-Croatian-Bosnian data with me and for providing many additional examples.
\subsection{još}

\subsubsection{General information}
\begin{itemize}
	\item Form: \textit{јoш} in Cyrillic script.
	\item Wordhood: independent grammatical word, invariable.
	\item Etymology: < Proto-Slavic \mbox{*\textit{ěsče}} \lq still, yet' with the initial segment going back to \mbox{*\textit{i}} \lq and' \parencite[146]{Derksen2008}.
\end{itemize}


\subsubsection{As a \lq{}still\rq{ }expression}
\begin{itemize}
	\item \textcite[161]{Alexander2006}, \textcite{Buchholz1991}, \textcite[739]{KontrastiveGrammatik}, \textcite[69–70]{Hammond2005}, \textcite{Prajnkovic2018}, \textcite{Tekavcic1989} and \textcite[s.v. \textit{još}]{HJP}.
		\item Specialisation: identified as a marker that is in line with my definition by \textcite{vanderAuwera1998}.
	\item Polarity sensitivity: inner negation yields \textsc{not yet}.
	\item Pragmaticity: augmented by \textit{uvek}/\textit{uvijek} \lq always' for the unexpectedly late scenario.
	\end{itemize}
\begin{exe}
	\ex
	\gll \textbf{Još} si mlad-a.\\
	still \textsc{cop}.2\textsc{sg} young-\textsc{nom}.\textsc{sg}.\textsc{f}\\
	\glt \lq Du bist noch jung. [You are still young.]' (\cite[26]{Buchholz1991}, glosses added)
	
	\ex
	\gll Ujutro \textbf{još} mogu raditi ali uveče nikako.\\
	in\_the\_morning still can.\textsc{ipfv}.1\textsc{sg} work.\textsc{ipfv}.\textsc{inf} but in\_the\_evening by\_no\_means\\
	\glt \lq Am Morgen kann ich noch arbeiten, aber am Abend überhaupt nicht. [In the morning I'm still capable of working, but in early evening, not at all.]' (\cite[889]{KontrastiveGrammatik}, glosses added)
\end{exe}

\subsubsection{Uses on the fringes of \lq{}still\rq{}}

\paragraph{Scalar contexts}\label{appendixBCMSScalar}
\begin{itemize}
	\item \textcite{Prajnkovic2018}.
	\item \textit{Još} itself is only used for decreases, whereas for limited increases (\lq still only')  scalar restrictive \textit{tek} is used (\ref{exappendixBCMSScalar3}). However, its emphatic (unexpectedly late) counterpart \textit{još} \mbox{\textit{uv}(\textit{ij})\textit{ek}} can combine with a restrictive marker to signal \lq still only\rq{ }(\ref{exappendixBCMSScalar5}).
	\item Also note the complex focus particle \textit{samo još} \lq only still' for \lq only as little as … left' in (\ref{exappendixBCMSScalar2}).
	\item Syntax: can form a single constituent with the focus. As shown in (\ref{exappendixBCMSScalar1}) they move through the clause together and can appear before second-position clitics.
\end{itemize}
\pagebreak
\begin{exe}
	 \ex Context: I had ten books and I've given away some of them.\label{exappendixBCMSScalar1}\\
	 \gll Osta-lo mi \textup{(}ih\textup{)} je \textbf{još} pet knjiga. / \textup{[}\textbf{Još} pet knjiga\textup{]} mi je osta-lo.\\
	 remain.\textsc{pfv}.\textsc{ptcp}-\textsc{sg}.\textsc{n} 1\textsc{sg}.\textsc{dat} \phantom{(}3\textsc{pl}.\textsc{gen} \textsc{cop}.3\textsc{sg} still five book.\textsc{gen}.\textsc{sg} {} \phantom{[}still five book.\textsc{gen}.\textsc{pl} 1\textsc{sg}.\textsc{dat} \textsc{cop}.3\textsc{sg} remain.\textsc{pfv}.\textsc{ptcp}-\textsc{sg}.\textsc{n}\\
	 \glt \lq I still have five books left. / What I have left are still five books.\rq{}
	 \\(Stefan Savič, p.c)

	 \ex\label{exappendixBCMSScalar2}
	\gll Njemu ostaje samo \textbf{još} jedno: što pre pobeći odavde.\\
	 3\textsc{sg}.\textsc{dat}.\textsc{m} remain.\textsc{ipfv}.3\textsc{sg} only still one.\textsc{acc}.\textsc{sg}.\textsc{n}  as before flee.\textsc{pfv}.\textsc{inf} from\_here\\
	 \glt \lq Für ihn blieb nur noch das eine zu tun: so schnell wie möglich zu fliehen. [There was only one option left for him: to get away as soon as possible.]' (\cite[248]{KontrastiveGrammatik}, glosses added)

  \ex\label{exappendixBCMSScalar3}
	 Context: We are supposed to be sent several books.
	 \exi{A:} I have got five.
	 \exi{B:} Me, too.
	 \exi{C:}\gll A ja sam dobi-o sam \textbf{tek} \textup{(\#{}}\textbf{još}\textup{)} tri.\\
	 but 1\textsc{sg} \textsc{cop}.1\textsc{sg} receive.\textsc{ptcp}-\textsc{sg}.\textsc{m} \textsc{cop}.1\textsc{sg} only \phantom{(\#{}}still three\\
	 \glt \lq I have (still) only received three.\rq{ }(Stefan Savič, p.c)

  \ex\label{exappendixBCMSScalar4}
	\gll Svi će vam reći – naći ćete to na web stranici, ali kod nas \textbf{još} \textbf{uvijek} \textbf{samo} \textbf{tri} \textbf{posto} \textbf{stanovništva} ima pristup Internetu.\\
	everyone will.3\textsc{sg} 2\textsc{sg}.\textsc{dat} say.\textsc{pfv}.\textsc{inf} { } find.\textsc{pfv}.\textsc{inf} will.2\textsc{pl} \textsc{dem}.\textsc{nom}.\textsc{sg}.\textsc{n} on web page.\textsc{loc}.\textsc{sg} but near 1\textsc{pl}.\textsc{gen} still always only three percent population.\textsc{gen}.\textsc{sg} have.3\textsc{sg} access.\textsc{acc}.\textsc{sg} internet.\textsc{dat}.\textsc{sg}\\
	\glt \lq Everyone will tell you \lq\lq{}You’ll find it [the information you need] on the website\rq\rq{}, but where we live, still only three percent of the population has access to the internet.\rq{ }(found online, glosses added)\footnote{\url{https://www.slobodnaevropa.org/a/823402.html} (08 March, 2023).}
	
	\pagebreak
	\ex
	Context: About increased funding for kindergartens.\label{exappendixBCMSScalar5}\\
	\gll Tu su \textbf{još} \textbf{uvijek} \textbf{samo} četiri grad-a koj-a su roditelje u potpunosti oslobodil-a plaća-nja - Belišće, Umag, Vrlika i Obrovac.\\
	here \textsc{cop}.3\textsc{sg} still always only four city-\textsc{pl} \textsc{rel}-\textsc{pl} \textsc{cop}.3\textsc{sg} parent.\textsc{acc}.\textsc{pl} at completion.\textsc{loc}.\textsc{sg} liberate.\textsc{pfv}.\textsc{ptcp}-\textsc{pl} pay-\textsc{nmlz} {} B. U. V. and O.\\
	\glt \lq \textbf{There are still only four cities} that have completely exempted parents from paying - Belišće, Umag, Vrlika and Obrovac.\rq{ }(found online, glosses added)\footnote{\url{https://grad-vinkovci.hr/hr/objave/default/za-vrtice-najvise-izdvajaju-mali-gradovi-rekorder-2018-su-vinkovci} (08 March, 2023).}
\end{exe} 



\subsubsection{Broadly adverbial temporal-aspectual functions}
\paragraph{Iterative via increment}
\label{appendixBCMSIterativeIncrement}
\begin{itemize}
	\item Form: in collocation with the event quantifier \textit{jednom} \lq once, one time'.
	\item Ex. (\ref{exappendixBCMSIterativeAdditive}) shows that a restitutive use is not available.
\end{itemize}

\begin{exe}
	\ex \gll
\textbf{Još} \textbf{jednom} da te vidim // još ti jednu boru stvorim // \textbf{još} ti \textbf{jednom} slomim srce // pa ti kažem da te volim\\
still once \textsc{comp} 2\textsc{sg}.\textsc{acc} see.\textsc{ipfv}.1\textsc{sg} {} still 2\textsc{sg}.\textsc{acc} one.\textsc{acc}.\textsc{sg}.\textsc{f} wrinkle(\textsc{f}).\textsc{acc}.\textsc{sg} create.\textsc{pfv}.1\textsc{sg} {} still 2\textsc{sg}.\textsc{dat} once break.\textsc{pfv}.1\textsc{sg} heart.\textsc{acc},\textsc{sg} {} then 2\textsc{sg}.\textsc{dat} say.\textsc{pfv}.1\textsc{sg} \textsc{comp} 2\textsc{sg}.\textsc{acc} love.\textsc{ipfv}.1\textsc{sg}\\
\glt \lq If only I could see you one more time // lf only I could create one more wrinkle [on your face]  // If only I could break your heart once more // and then tell you I love you.' (from the song \textit{Još jednom} by the Croatian rock band \textit{Silente})

	\ex Context: We're in a band and practicing for a concert.\\
	\gll Ajmo odsvirati \textbf{još} \textbf{jednom}.\\
	\textsc{hort} play.\textsc{inf} still once\\
	\glt \lq Let's play it one more time.' (Jurica Polančec, p.c.)
	
	\pagebreak
	\ex Context: I opened the door but a gust of wind has closed it. I ask you:\label{exappendixBCMSIterativeAdditive}\\
	\gll Je l' možeš opet\textup{/}ponovo\textup{/}\textup{\#}\textbf{još} \textbf{jednom} da otvoriš vrata?\\
	2\textsc{sg} \textsc{q} can.2\textsc{sg}  again/again/still once \textsc{comp} open.\textsc{pfv}.2\textsc{sg} door.\textsc{acc}.\textsc{sg}\\
	\glt \lq Could you close the door again?\rq{ }(Stefan Savič, p.c.)
\end{exe}

\paragraph{Prospective \lq eventually\rq{}}
\label{appendixBCMSProspective}
\begin{itemize}
	\item \textcite[142]{Koenig1991} and \textcite{Prajnkovic2018}.
	\item Negation denies the existence of any future situation of the type depicted in the predicate; see (\ref{exAppendixBCMSProspective4}).
\end{itemize}
\begin{exe}
	\ex
	\gll \textbf{Još} ćeš ti videti ko je od nas dvojice upravu.\\
	still will.2\textsc{sg} 2\textsc{sg} see.\textsc{inf} who \textsc{cop}.3\textsc{sg} of 1\textsc{pl}.\textsc{gen} at right\\
	\glt \lq Du wirst noch sehen, wer von uns beiden Recht hat. [You'll see yet which one of us two is right.]' (\cite[244]{JDahl1988})

	\ex Context: At a sports event. Our team is lying behind.\\
	\gll Pobedi-ćemo mi \textbf{još}.\\
	win.\textsc{pfv}-\textsc{fut}.1\textsc{pl} 1\textsc{pl} still\\
	\glt \lq We'll win yet.\rq{ }(Stefan Savić, p.c.)

	\ex Context: I've told you about a party I'll be hosting.\\
	\gll Javiću ti \textbf{još}.\\
	inform.\textsc{pfv}.\textsc{fut}.1\textsc{sg} 1\textsc{sg}.\textsc{dat} still\\
	\glt \lq I'll follow up with more information.' (Stefan Savič, p.c)
	
	\ex\label{exAppendixBCMSProspective4}
	\gll …da se \textbf{ne} \textbf{bi} \textbf{još} i razboleo.\\
	\phantom{…}\textsc{comp} \textsc{refl} \textsc{neg} \textsc{cop}.2\textsc{sg} still and fall\_ill.\textsc{pfv}.2\textsc{sg}\\
	\glt \lq … so that you won't end up ill.\rq{ }(Stefan Savić, p.c.)
\end{exe}



\subsubsection{Temporal connectives and frame setters}

\paragraph{Persistent time frame}
\label{appendixBCMSContinuativeTT}
\begin{itemize}
	\item \textcite{Buchholz1991}.
	\item In this function, \textit{još} associates with a temporal frame adverbial (\ref{exAppendixSerbocroatianTemporalFrameAdverbial1}– \ref{exAppendixSerbocroatianTemporalFrameAdverbial3})	 or a related expression (\ref{exAppendixSerbocroatianTemporalFrameAdverbial4}), giving a reading of \lq while (it is) still …'.
	\item Syntax: forms a constituent with the expression it associates with; note how
\textit{još danas} in (\ref{exAppendixSerbocroatianTemporalFrameAdverbial3}) precedes the second-position future auxiliary.
\end{itemize}

\begin{exe}
	\ex\label{exAppendixSerbocroatianTemporalFrameAdverbial1}
Context: A student with the ardent desire to become a doctor did not get accepted into medical school at the first try. He applied again, and the good news that this time he was accepted for admission in Berlin got to him while he was travelling.\\
	\gll \textbf{Još} sledećeg dana je otputova-o za Berlin jer je semestar već počinja-o.\\
 	 still next.\textsc{gen}.\textsc{m} day(\textsc{m}).\textsc{gen} \textsc{cop}.3\textsc{sg} leave.\textsc{pfv}.\textsc{ptcp}-\textsc{sg}.\textsc{m} for Berlin because \textsc{cop}.3\textsc{sg} semester(\textsc{m}).\textsc{nom}.\textsc{sg} already begin.\textsc{ipfv}.\textsc{ptcp}-\textsc{sg}.\textsc{m}\\
	 \glt  \lq The very next day he set off for Berlin, because the semester was about to begin.' (found online, glosses added)\footnote{\url{https://arhiva.vesti-online.com/Vesti/Zanimljivosti/63714/Pozarevljanin-buduci-vrhunski-berlinski-lekar} (08 February, 2023).}

	\ex\label{exAppendixSerbocroatianTemporalFrameAdverbial2}	
	\gll Austrijsko Ministarstvo zdravlja izradi-lo je zakon … i on će \textbf{još} ove nedelje biti upućen na razmatranje.\\
Austrian.\textsc{nom}.\textsc{sg}.\textsc{n} ministery(\textsc{n}).\textsc{nom}.\textsc{sg} health.\textsc{gen}.\textsc{sg} make.\textsc{pfv}:\textsc{ptcp}-\textsc{sg}.\textsc{n} \textsc{cop}.3\textsc{sg} law(\textsc{m}).\textsc{acc}.\textsc{sg} {} and 3\textsc{sg}.\textsc{m} will.3\textsc{sg} still \textsc{prox}.\textsc{gen}.\textsc{sg}.\textsc{f} week(\textsc{f}).\textsc{gen}.\textsc{sg} \textsc{cop}.\textsc{inf} send:\textsc{ptcp}.\textsc{pass}:\textsc{sg}.\textsc{m} to consider.\textsc{nmlz}\\
	\glt \lq The Austrian Ministry of Health has drafted a law … and it will be sent for consideration this very week.'
	(found online, glosses added)\footnote{\url{https://24sedam.rs/svet/vesti/96437/izraden-nacrt-zakona-o-obaveznoj-vakcinaciji-odredena-i-visina-kazni/vest} (24 March, 2022).}

	\ex\label{exAppendixSerbocroatianTemporalFrameAdverbial3}
	\gll \textbf{Još} danas ću otići kod njega.\\
	still today will.1\textsc{sg} leave.\textsc{pfv}.\textsc{inf} at 3\textsc{sg}.\textsc{acc}.\textsc{m}\\
	\glt \lq Noch heute werde ich zu ihm gehen. [This very day / before the end of the day, I'll go to his place.]\rq{ }(\cite[31]{Buchholz1991}, glosses added)
	
	\ex\label{exAppendixSerbocroatianTemporalFrameAdverbial4}
	Context: There was an accident involving a lorry and a cyclist.\\
	\gll Žrtva je umr-la \textbf{još} na licu mesta.\\
	victim(\textsc{f}) \textsc{cop}.3\textsc{sg} die.\textsc{ptcp}-\textsc{sg}.\textsc{f} still at face.\textsc{loc}.\textsc{sg} place.\textsc{gen}.\textsc{sg}\\
	\glt \lq The victim died right at the scene of the accident.' (Stefan Savič, p.c.)
\end{exe}

\paragraph{Time-scalar (\lq as late as/as far removed as\rq{})}
\label{appendixBCMSTimeScalar}
\begin{itemize}
	\sloppy
	\item \textcite{Buchholz1991}, \textcite[s.v. \textit{još}]{HJP} and \textcite{Prajnkovic2018}.
	\item Time-scalar additive \textit{još} is attested both with a scale of time proper \lq as late as\rq{ }(\ref{exappendixBCMSTimeScalar1}, \ref{exappendixBCMSTimeScalar2}) and with one of temporal distance \lq as far removed as\rq{ }(\ref{exappendixBCMSTimeScalar4}–\ref{exappendixBCMSTimeScalar7}). The common denominator lies in the fact that the alternatives are constistently lower values (earlier times/less removed times). The two functions overlap in future contexts like (\ref{exappendixBCMSTimeScalar3}).
	\item  The data suggest that readings of \lq as far removed as, as early as\rq{ }arise in the same contexts described by \textcite{Mustajoki1988} for the \ili{Russian} cognate \mbox{\textit{eščë}}. These are, in broad strokes:
	\begin{itemize}
		\item With specifications of a point in time (\ref{exappendixBCMSTimeScalar3}).
		\item With reference to steps in a process, stages in life, etc. (\ref{exappendixBCMSTimeScalar4}, \ref{exappendixBCMSTimeScalar5}).
		\item With specifications \lq even before' (\ref{exappendixBCMSTimeScalar6}).
		\item With designations for people and personal names, typically deceased or otherwise inaccessible (\ref{exappendixBCMSTimeScalar7}).
	\end{itemize}
	\item That the \lq as far removed as\rq{ }reading is not one of earliness (\lq as early as\rq{}) becomes visible by a direct comparison with the \textsc{already} expression \textit{več} in examples like (\ref{exappendixBCMSTimeScalar9}) and, even more prominently, in (\ref{exappendixBCMSTimeScalar10}), where a comparatively minor forward leap on the temporal axis is at stake and \textit{još} is infelicitous.
	\item Syntax: forms a constituent with its associated expression.
\end{itemize}
\begin{exe}
	\ex\label{exappendixBCMSTimeScalar1}
	\gll \textbf{Još} 1967, su se na {Novom Zelandu} pabovi zatvara-li u 18:00.\\
	still 1967 \textsc{cop}.3\textsc{sg} \textsc{refl} at {New Zealand.\textsc{loc}} pub(\textsc{m}).\textsc{nom}.\textsc{pl} close:\textsc{ptcp}-\textsc{pl}.\textsc{m} at 6pm\\
	\glt \lq As late as 1967, the pubs in New Zealand would close at 6pm.'
	\\(Stefan Savič, p.c.)

	\ex\label{exappendixBCMSTimeScalar2}
	\gll Ko bi \textbf{još} pre godinu dana pomisl-io da ćemo danas imati efikasne vakcine protiv {virusa korona} i da ćemo toliko napredovati sa vakcinacijom?!\\
	who.\textsc{nom} would still before year.\textsc{acc}.\textsc{sg} day.\textsc{gen}.\textsc{sg} think.\textsc{pfv}:\textsc{ptcp}-\textsc{sg}.\textsc{m} \textsc{comp}  will.1\textsc{pl} today have.\textsc{ipfv}.\textsc{inf} efficient.\textsc{gen}.\textsc{sg}.\textsc{f} vaccine(\textsc{f}).\textsc{gen}.\textsc{sg} against {corona virus} and \textsc{comp} will.1\textsc{pl} so\_much progress.\textsc{inf} with vaccination.\textsc{dat}.\textsc{sg}\\
	\glt \lq Who would have thought, just a year ago, that today we would have effective vaccinations against the corona virus and make so much progress with vaccination?' (found online, glosses added)\footnote{\url{https://belgrad.diplo.de/rs-sr/aktuelles/-/2473212} (24 March, 2022).}
	
	\pagebreak
	\ex\label{exappendixBCMSTimeScalar3}
	\gll \textbf{Još} \textbf{za} \textbf{sto} \textbf{godina} ljudi će govoriti o ovome.\\
	still at hundred year.\textsc{gen}.\textsc{pl} people will.3\textsc{sg} speak.\textsc{ipfv}.\textsc{inf} on \textsc{prox}.\textsc{loc}.\textsc{sg}.\textsc{m}\\
	\glt \lq Even in a hundred years people will (still) speak about this.\rq{ }(Stefan Savić, p.c.)
	
		\ex\label{exappendixBCMSTimeScalar4}
	\gll Obavljenim nadzorom utvrđen-o je da vozač upravlja automobilom iako mu je {vozačka dozvola} oduzet-a \textbf{još} prije 15 godina.\\
	carry\_out.\textsc{pass}.\textsc{ins}.\textsc{sg}.\textsc{m} supervision(\textsc{m}).\textsc{ins}.\textsc{sg} strengthen:\textsc{pfv.}\textsc{pass}.\textsc{ptcp}-\textsc{sg}.\textsc{n} \textsc{cop}.3\textsc{sg} \textsc{comp} driver.\textsc{nom}.\textsc{sg} drive.\textsc{ipfv}.3\textsc{sg} car.\textsc{ins}.\textsc{sg} although 3\textsc{sg}.\textsc{dat}.\textsc{m} \textsc{cop}.3\textsc{sg} driver's\_licence(\textsc{f}).\textsc{nom}.\textsc{sg}  take\_away.\textsc{pfv}:\textsc{pass}.\textsc{ptcp}-\textsc{sg}.\textsc{f} still before 15 year.\textsc{gen}.\textsc{pl}\\
\glt \lq Through the investigation it was established that the driver was steering the car although his driver's licence had been revoked as far back as 15 years ago.' (found online, glosses added)\footnote{\url{http://www.radio-maestral.hr/pula/vozio-pijan-a-vozacka-mu-je-oduzeta-jos-prije-15-godina/} (09 Februrary, 2023).}
	
	\ex\label{exappendixBCMSTimeScalar5}
	\gll \textbf{Još} u mladosti bio je ozbiljan.\\
	still at youth.\textsc{loc} \textsc{cop}.\textsc{ptcp}.\textsc{sg}.\textsc{m} \textsc{cop}.3\textsc{sg} serious.\textsc{nom}.\textsc{sg}.\textsc{m}\\
	\glt \lq He has been serious since (as far back as) in his youth.\rq{ }(\cite[s.v. \textit{još}]{HJP}, glosses added)
	
	
	\ex\label{exappendixBCMSTimeScalar6}
	\gll Jezik se uči \textbf{još} u maternici, pokaza-la je studija finskog Sveučilišta Helsinki.\\
	language \textsc{refl} learn.\textsc{ipfv}.3\textsc{sg} still at womb.\textsc{loc} show.\textsc{ptcp}-\textsc{sg}.\textsc{f} \textsc{cop}.3\textsc{sg} study(\textsc{f}) finnish university.\textsc{gen}.\textsc{sg} H.\\
	\glt \lq Language is (already) learned as far back as in the womb, according to a Finnish study at the University of Helsinki.\rq{ }(found online, glosses added)\footnote{\url{https://www.24sata.hr/lifestyle/pazite-sto-pricate-bebe-slusaju-i-pamte-rijeci-jos-u-maternici-329923} (27 April, 2022).}
	
	\pagebreak
	\ex\label{exappendixBCMSTimeScalar7}
	\gll
	Bog nas je, \textbf{još} \textbf{prije} \textbf{nego} \textbf{što} \textbf{je} \textbf{stvori}-\textbf{o} \textbf{svijet}, izabra-o da u Kristu budemo sveti i bez nedostatka u njegovim očima.\\
	God 1\textsc{pl}.\textsc{acc} \textsc{cop}.3\textsc{sg} still before than as \textsc{cop}.3\textsc{sg} create.\textsc{ipfv}.\textsc{ptcp}-\textsc{sg}.\textsc{m} world.\textsc{acc}.\textsc{sg} choose.\textsc{pfv}.\textsc{ptcp}-\textsc{sg}.\textsc{m} \textsc{comp} at Christ \textsc{cop}.1\textsc{pl} holy.\textsc{nom}.\textsc{pl}.\textsc{m} and without shortage.\textsc{gen}.\textsc{sg} at \textsc{poss}.3\textsc{sg}.\textsc{m}:\textsc{loc}.\textsc{pl} eye.\textsc{loc}.\textsc{pl}\\
	\glt \lq For he chose us in him \textbf{(already) before the creation of the world} to be holy and blameless in his sight.\rq{ }(Eph.1:4, \textit{Knijga o Kristu}, glosses added)
	
	\ex\label{exappendixBCMSTimeScalar8}
		\gll No, iako ih je \textbf{još} Einstein najavi-o fizičari dosad nisu uspje-li otkriti postojanje gravitacijskih valova. \\
but although 3\textsc{pl}.\textsc{acc} \textsc{cop}.3\textsc{sg} still E. announce.\textsc{pfv}.\textsc{ptcp}-\textsc{sg}.\textsc{m} physicist.\textsc{nom}.\textsc{pl} thus\_far \textsc{neg}.\textsc{cop}.3\textsc{pl} succeed.\textsc{pfv}.\textsc{ptcp}-\textsc{pl}.\textsc{m} uncover.\textsc{pfv}.\textsc{inf} existence.\textsc{acc}.\textsc{sg} gravitational.\textsc{gen}.\textsc{pl} wave.\textsc{gen}.\textsc{pl}\\
\glt \lq But even though someone as early as/as far removed as Einstein postulated them, until now physicists have failed to detect the existence of gravitational waves.' (found online, glosses added)\footnote{\url{https://www.24sata.hr/tech/otkrili-su-gravitacijske-valove-konacni-dokaz-za-einsteina-460394} (27 April 2022).} 


	\ex\label{exappendixBCMSTimeScalar9}
	\begin{xlist}
		\ex \gll \textbf{Još} sa šest godina zna-la je čitati latinski.\\
		still at six year.\textsc{gen}.\textsc{pl} know.\textsc{ipfv}.\textsc{ptcp}-\textsc{sg}.\textsc{f} \textsc{cop}.3\textsc{sg} 	read.\textsc{ipfv}.\textsc{inf} Latin.\textsc{acc}.\textsc{sg}\\
		\glt \lq All the way back at the age of six she could (already) read Latin.\rq
	
	\ex \gll \textbf{Več} sa šest godina zna-la je čitati latinski.\\
		already at six year.\textsc{gen}.\textsc{pl} know.\textsc{ipfv}.\textsc{ptcp}-\textsc{sg}.\textsc{f} \textsc{cop}.3\textsc{sg} 	read.\textsc{ipfv}.\textsc{inf} Latin.\textsc{acc}.\textsc{sg}\\
		\glt \lq As early as at the age of six she could (already) read Latin.\rq{ }
		\\(Jurica Polančec, p.c.)
	\end{xlist}

	\ex Context: About a politician who graduaded in 2011. \label{exappendixBCMSTimeScalar10}\\
	\gll Međutim, \textbf{već} \textup{(\#}\textbf{još}\textup{)} \textbf{2012.} \textbf{godine} se naša-o u skupini od 18 policijskih dužnosnika koje se sumnjiči-lo za plagir-anje diplom-sk-og…\\
	however already \phantom{(\#}still 2012 year.\textsc{gen}.\textsc{sg} \textsc{refl} find.\textsc{pfv}.\textsc{ptcp}-\textsc{sg}.\textsc{m} at group.\textsc{loc}.\textsc{sg} from 18 constabulary.\textsc{gen}.\textsc{pl} official.\textsc{gen}.\textsc{pl} \textsc{rel}.\textsc{acc}.\textsc{pl}.\textsc{n} \textsc{refl} suspect.\textsc{ipfv}.\textsc{ptcp}-\textsc{pl}.\textsc{n} for plagiarize-\textsc{nmlz} diploma-\textsc{adj}-\textsc{gen}.\textsc{sg}\\
	\glt \lq However, already in 2012, he found himself in a group of 18 police officials who were suspected of plagiarizing their diploma….\rq{ }(found online and Jurica Polnačec, p.c.)\footnote{\url{https://www.index.hr/vijesti/clanak/barisic-nije-jedini-pogledajte-i-ostale-poznate-hrvatske-plagijatore/943153.aspx} (24 November, 2022).}
\end{exe}

\subsubsection{Marginality}
\label{appendixBCMSMarginal}
\begin{itemize}
	\item \textcite[s.v. \textit{još}]{HJP} and \textcite{Prajnkovic2018}.
	\item \textit{Još} is compatible with readings of marginality.  This includes derogatory comparisons like (\ref{exappendixBCMSMarginal3}).
\end{itemize}
\begin{exe}
	\ex\label{exappendixBCMSMarginal1}
	Context: Talking about tennis skills.\\
	\gll Mark-a \textbf{još} \textup{(}i\textup{)}  mogu da pobedim, ali Tom je bolji od mene.\\
	M.-\textsc{acc} still (also) can.\textsc{ipfv}.1\textsc{sg} \textsc{comp} beat.\textsc{pfv}.1\textsc{sg} but T. \textsc{cop} better than 1\textsc{sg}.\textsc{acc}\\
	\glt \lq Mark I can still beat, but Tom is better than me.' (Stefan Savič, p.c.)
	
		\ex Context: weʻre in Croatia, taking a roadtrip East and discussing which country Osijek belongs to.	\label{exappendixBCMSMarginal2}\\
	\gll Osijek je \textbf{još} Hrvatska.\\
	O. \textsc{cop}.3\textsc{sg} still Croatia\\
	\glt \lq Osijek is still in Croatia.' (Jurica Polančec, p.c.) 

	\ex\label{exappendixBCMSMarginal3}
	\gll Ivo je \textbf{još} \textup{(}i\textup{)} najpametniji u porodici (da vidiš kakvi se tek osta-li).\\
	I. \textsc{cop}.3\textsc{sg} still (also) smartest.\textsc{acc}.\textsc{sg}.\textsc{m} at family.\textsc{loc}.\textsc{sg} \textsc{comp} see.2\textsc{sg} what\_kind\_of \textsc{refl} first/only remain.\textsc{ipfv}.\textsc{ptcp}-\textsc{m}.\textsc{pl}\\
	\glt \lq Ivo is still the smartest of his family (just look at the rest of the bunch).' (Stefan Savič, p.c.)
	
	\ex
	\gll Ti si \textbf{još} jeftino proša-o.\\
	2\textsc{sg} \textsc{cop}.2\textsc{sg} still cheaply pass.\textsc{pfv}.\textsc{ptcp}-\textsc{sg}.\textsc{m}\\
	\glt \lq You still got off cheaply.\rq{ }(\cite[72]{Prajnkovic2018}, glosses added)

\end{exe}

\paragraph{Marginality, \lq within limits': \textit{još}-\textit{još}}\label{appendixBCMSJosJos}
\begin{itemize}
	\item \textcite[s.v. \textit{još}]{HJP}.
	\item Form: in the reduplicated form \textit{jošjoš}, typically together with the neuter anaphoric demonstrative \textit{to} as the subject.
	\item This use signals
	\begin{quote}
… da se dopušta kao mogućnost, da se na što može pristati ili da je pod nekim okolnostima prihvatljivo (slijedi ili se podrazumijeva nešto drugačije ili suprotno) {[}…what is permissible, can be agreed to, or is acceptable in a given circumstance (following or implying something else or to the contrary){]}. \parencite[s.v. \textit{još}]{HJP}
\end{quote}
	\item This use was not known to all native-speaker linguists consulted by me, and might be a regionalism. The same use is, however, described for the cognate form \textit{ešte}\textit{ešte} in \ili{Slovak} (\cite[s.v. \textit{ešte}]{SSSJ}; \cite[s.v. \textit{ešte}]{KSS4}).
\end{itemize}
\begin{exe}
	\ex
	\gll Platiti tolike novce, to \textbf{još}\sim\textbf{još} (ali za lošu stvar nikada).\\
	pay.\textsc{pfv}.\textsc{inf} so\_much.\textsc{gen}.\textsc{sg}.\textsc{f} money(\textsc{f}).\textsc{acc}.\textsc{sg} \textsc{dem}.\textsc{n} still\sim still but for bad.\textsc{acc}.\textsc{sg}.\textsc{f} thing(\textsc{f}).\textsc{acc}.\textsc{sg} never\\
\glt \lq To spend so much money, \textbf{well ok}  (but [I'd] never [spend it] on [such] a bad thing).\rq{ }(\cite[s.v. \textit{još}]{HJP}, glosses added)
\end{exe}

\subsubsection{Additive and related functions}
\paragraph{Additive}\label{appendixBCMSAdditive}
\begin{itemize}
	\sloppy
	\item \textcite{Buchholz1991}, \textcite[755]{KontrastiveGrammatik}, \textcite{Prajnkovic2018}, \textcite{Tekavcic1989} and \textcite[s.v. \textit{još}]{HJP}.
	\item When an increment of the same type is added, \textit{još} itself can \lq\lq stand in\rq\rq{ }for an indefinite quantifier \lq more\rq{ }(\ref{exappendixBCMSAdditive5}).
	\item Additive \textit{još} may co-occur with other additive markers, such as \textit{i} in (\ref{exappendixBCMSAdditive2}).
	\item In the pattern illustrated in (\ref{exappendixBCMSAdditive6}), the additive use can yield intersubjective and concessive overtones.
	\item Syntax: can form a constituent with its focus.
\end{itemize}
\begin{exe}
	\ex
	\gll {Pored toga što} piše pjesme, čime se \textbf{još} bavi?\\
	besides write.\textsc{ipfv}.3\textsc{sg} poem.\textsc{acc}.\textsc{pl} what \textsc{refl} still engage\_in.\textsc{ipfv}.3\textsc{sg}\\
	\glt \lq What else does she do beside writing poems?\rq{ }(\cite[246]{Alexander2006}, glosses added)
	
	\ex\label{exappendixBCMSAdditive2}
	 \gll Pametna je devojka, a sad je \textbf{još} i ljuta.\\
	smart.\textsc{nom}.\textsc{sg}.\textsc{f} \textsc{cop}.3\textsc{sg} girl(\textsc{f}).\textsc{nom}.\textsc{sg} and now \textsc{cop}.3\textsc{sg} still also angry.\textsc{nom}.\textsc{sg}.\textsc{f}\\
	\glt \lq She's a smart girl, and now an angry one, too.\rq{ }(found online, glosses added)\footnote{\url{https://glosbe.com/hr/sr/pametan} (24 March, 2022).}
	
\ex
	\gll Popi-o je \textbf{još} jednu čašu vina.\\
	drink.\textsc{ptcp}-\textsc{sg}.\textsc{m} \textsc{cop}.3\textsc{sg} still one.\textsc{acc}.\textsc{sg}.\textsc{f} glass(\textsc{f}).\textsc{acc}.\textsc{sg} wine.\textsc{gen}.\textsc{sg}\\
	\glt \lq Er hat noch ein Glas Wein (aus)getrunken [He drank another glass of wine].’ (\cite[32]{Buchholz1991})

	\ex\label{exappendixBCMSAdditive5}
	\textit{Kad su prodali svoje čudesne ljekarije bolje nego što su mislili}\\
	\lq When their miracle potion sold better than they had thought,\rq{}
	\gll otišl-i su kući po \textbf{još}.\\
	leave.\textsc{ptcp}-\textsc{pl}.\textsc{m} \textsc{cop}.3\textsc{pl} home.\textsc{dat}.\textsc{sg} for still\\
	\glt \lq they went home for more.\rq{ }(\cite[s.v. \textit{još}]{HJP}, glosses added)
	
	\ex\label{exappendixBCMSAdditive6}
	\begin{xlist}
    \exi{A:} \textit{Slabo ti on poznaje propise.}\\
	\lq He hardly knows the rules.\rq{}
	\exi{B:}\gll \textbf{Još} je pravnik!\\
	still \textsc{cop}.3\textsc{sg} lawyer.\textsc{nom}.\textsc{sg}\\
	\glt \lq \textbf{And} he's a lawyer (i.e. he of all people should know)!\rq{ }(\cite[s.v. \textit{još}]{HJP}, glosses added)		
    \end{xlist}
\end{exe}


\paragraph{Comparisons of inequality}\label{appendixBCMSComparisons}
\begin{itemize}
	\item \textcite[180]{Alexander2006}, \textcite{Buchholz1991}, \textcite{Prajnkovic2018} and \textcite[s.v. \textit{još}]{HJP}
	\item Note that comparatives in Serbian-Bosnian-Croatian involve a derived or suppletive comparative form of the predicate, and the standard NP is introduced by either \textit{od} or \textit{nego} \lq than' (see \cite[173–180]{Alexander2006}). The use of \textit{još} adds the notion of \lq even\rq{}.
	\item Syntax: can form a constituent with its focus.
\end{itemize}
\begin{exe}
	\ex
	\gll Jučer ih je bilo mal-o, a danas ih ima \textbf{još} manje.\\
	yesterday 3\textsc{pl}.\textsc{gen} \textsc{cop}.3\textsc{sg} \textsc{cop}:\textsc{ptcp}.\textsc{sg}.\textsc{n} few-\textsc{sg}.\textsc{n} and today 3\textsc{pl}.\textsc{gen} \textsc{exist}.\textsc{ipfv}.3\textsc{sg} still less\\
	\glt \lq Yesterday there were not many, and today there are even fewer.' (\cite[181]{Alexander2006}, glosses added)

	\ex
	\gll Želim \textbf{još} više dobrog života.\\
	want.\textsc{ipfv}.1\textsc{sg} still more good.\textsc{gen}.\textsc{sg}.\textsc{m}.\textsc{def} life(\textsc{m}).\textsc{gen}.\textsc{sg}\\
	\glt \lq I want even more of the good life.' (\cite[56]{Alexander2006}, glosses added)

\ex \gll Taj mu je roman \textbf{još} zanimljiviji od prethodnoga.\\
\textsc{prox}.\textsc{nom}.\textsc{sg}.\textsc{m} 3\textsc{sg}.\textsc{dat}.\textsc{m} \textsc{cop}.3\textsc{sg} novel(\textsc{m}).\textsc{nom}.\textsc{sg} still interesting.\textsc{cmpr}:\textsc{nom}.\textsc{sg}.\textsc{m} than previous.\textsc{gen}.\textsc{sg}.\textsc{m}.\textsc{def}\\
\glt \lq This novel is even more interesting for him than the previous one.' (\cite[72]{Prajnkovic2018}, glosses added)
\end{exe}

\subsubsection{Broadly modal and interactional uses}\largerpage
\paragraph{Exclamation  \textit{još} \textit{kako}, \textit{još} \textit{koliko} \lq And how'}\label{appendixBCMSSpecificational}
\begin{itemize}
	\item \textcite[262]{JDahl1988}, \textcite{Prajnkovic2018}, and \textcite[s.v. \textit{još}]{HJP}.
	\item Form: this function obtains in collocation with \textit{kako} \lq how'
	(\ref{exappendixBCMSJosKako1}, \ref{exappendixBCMSJosKako2}) and \textit{kaliko} \lq how much'.
	\item This collocation is used in conversational dialogue. It has an affirmative, reinforcing and anaphoric function \lq and how (much)' that is a straightforward extension of the additive use of \textit{još} (\appref{appendixBCMSAdditive}) in collaboration with the cumulative component of \textsc{still}; note the variant forms \textit{i te kako}, \textit{i te kaliko} \lq and also how (much)\rq{}.
\end{itemize}

\begin{exe}
	\ex\label{exappendixBCMSJosKako1}
	\begin{xlist}
		\exi{A:}\gll Jesi li dobro naspava-o?\\
		\textsc{cop}.2\textsc{sg} \textsc{q} well sleep.\textsc{pfv}.\textsc{ptcp}-\textsc{sg}.\textsc{m}\\
		\glt \lq Did you sleep well?'
		\exi{B:} \gll \textbf{Još} \textbf{kako}.\\
		still how\\
		\glt \lq And how I did!' (\cite[72]{Prajnkovic2018}, glosses added)
	\end{xlist}

	\ex\label{exappendixBCMSJosKako2}
	\begin{xlist}
		\exi{A:} \gll Ne spominji je već.\\
		\textsc{neg} mention.\textsc{ipfv}.\textsc{imp} 3\textsc{sg}.\textsc{acc}.\textsc{f} already\\
		\glt \lq Don't mention her anymore!'
		\exi{B:} \gll \textbf{Još} \textbf{kako} ću je spominjati.\\
		still how will.1\textsc{sg} 3\textsc{sg}.\textsc{acc}.\textsc{f} mention.\textsc{ipfv}.\textsc{inf}\\
		\glt \lq And how I will mention her!' (HrWac 2.2, glosses added)
	\end{xlist}
	
	\ex\label{exappendixBCMSJosKaliko}
	\begin{xlist}
	\exi{A:}\gll Ja sam njih upozorava-o.\\
	1\textsc{sg} \textsc{cop}.1\textsc{sg} 3\textsc{pl}.\textsc{acc} warn.\textsc{ptcp}-\textsc{sg}.\textsc{m}\\
	\glt \lq I warned them.'
	\exi{B:}\gll \textbf{Još} \textbf{kaliko}.\\
	still how\_much\\
	\glt \lq Very much so.' (\cite[s.v. \textit{još}]{HJP}, glosses added)
	\end{xlist}
\end{exe}
\il{Serbian|)}\il{Croatian|)}\il{Bosnian|)}

\section{Southern Yukaghir (yux, sout2750)}\il{Yukaghir, Southern|(}

\subsection{Introductory remarks}
\begin{sloppypar}
Apart from descriptive materials, I searched \citeauthor{YukaghirTexts}'s (\citeyear{YukaghirTexts}) text collection and the Southern Yukaghir online documentation \parencite{NikolaevaMayer2004}. I adapted examples from the latter to the practical orthography used in \textcite{Maslova2003}.
\end{sloppypar}

\subsection{ajī / āj}

\subsubsection{General information}
\begin{itemize}
	\item Form: two variants, \textit{ajī} and \textit{āj}. Both forms can serve as phasal polarity expressions, but differ slightly in their additional functions.
	\item Wordhood: independent grammatical word.
\end{itemize}


\subsubsection{As a \lq{}still\rq{ }expression}
\begin{itemize}
	\item \textcite[531–532]{Maslova2003}.
	\item Form:  both as \textit{ajī} and \textit{āj}.
	\item Specialisation: textual attestations like (\ref{exAppendixKolyma1}–\ref{exAppendixKolyma3}) give evidence that this marker conforms to my definition. For instance, in (\ref{exAppendixKolyma1}) \textit{ajī} not only indicates the persistence of a prior state in which there were few Russians and Yakut on Yukaghir land, but it also evokes a contrast with the time of speech, where the situation has changed. 
	
	\item Pragmaticity: compatible with both scenarios (tentative conclusion). Example (\ref{exAppendixKolyma3}) is a prime candidate for the unexpectedly late scenario. It is yet unclear, if additional marking, such as the intensifier in (\ref{exAppendixKolyma3}), is compulsory for indicating the unexpectedly late scenario.
	\item Polarity sensitivity: inner negation yields \textsc{not yet}.
	\item Syntax: preverbal position.
\end{itemize}

\begin{exe}
	\ex\label{exAppendixKolyma1}
	Context: The Yukaghir and Even understood each other and would exchange goods.\\
	\gll Lučī-pe jaqa-pe tiŋ lebie-ge \textbf{ajī} ča-pe-de parā-ge…\\
	Russian-\textsc{pl} Yakut-\textsc{pl} this earth-\textsc{loc} still be\_few-\textsc{pl}-\textsc{poss}:\textsc{attr} time-\textsc{loc}\\
	\glt \lq At the time when there were still few Russians and Yakuts here…' (\cite[138, 157]{YukaghirTexts}; \citeyear[425]{Maslova2003})

	\ex Context: About the days of the speaker’s great grandfather.\\
	\gll Irki-d’e qojl numø uj-ā-ge parā-pe-de-ge ta-ŋ pulut \textbf{aj} ō-ŋō-l’el-te-j.\\
	one-\textsc{freq} God house do-\textsc{inch}-\textsc{ds} time-\textsc{pl}-\textsc{poss}.3-\textsc{loc} that-\textsc{attr} old\_man still child-\textsc{vblz}-\textsc{evid}-\textsc{fut}-3\textsc{sg}\\
	\glt \lq Once when the old man was still young, they started building a church.' \parencite[Text 36]{NikolaevaMayer2004}
	
	\largerpage
	\ex\label{exAppendixKolyma3}
	Context: About a well-regarded old man.\\
	\gll Bosj’e lig-mu-lle sto n’emolhil-ŋōt gude-din lʼe-de-ge \textbf{āji} eg-užu-j-bed-ek tud-idʼie.\\
	at\_all old-\textsc{inch}-\textsc{ss}.\textsc{pfv} hundred year-\textsc{transformative} become-\textsc{ptcp} \textsc{cop}-3-\textsc{ds} still walk-\textsc{iter}-\textsc{attr}-\textsc{rel}.\textsc{nmlz}-\textsc{pred} 3\textsc{sg}-\textsc{intens}\\
	\glt \lq As a very old man, a hundred years old, he still walked alone unaided (lit. … is one that still walks around …).'\footnote{See \textcite[126–128]{Maslova2003} on the Southern Yukaghir \lq\lq transformative" \mbox{-\textit{ŋōt}}.} \parencite[141, 158]{YukaghirTexts}
	\end{exe}
	
\subsubsection{Broadly adverbial temporal-aspectual functions}
\paragraph{Iterative and restitutive}
\label{appendixSouthernYukaghirIterative}
\begin{itemize}
	\item \textcite[528–529]{Maslova2003}.
	\item Form: in this function, the marker invariably occurs as \textit{āj}.
	\item Both iterative (\ref{appendixSouthernYukaghirIterative1}, \ref{appendixSouthernYukaghirIterative2}) and restitutive uses (\ref{appendixSouthernYukaghirRestitutive1}, \ref{appendixSouthernYukaghirRestitutive2}) are attested.
	\item An example like (\ref{exAppendixKolymaAdditiveIterative}) can be read as either iterative or additive.
	\item Syntax: preverbal position.
\end{itemize}
\begin{exe}

	\ex\label{appendixSouthernYukaghirIterative1}
	\gll Pulut, pulut, \textbf{āj} kimdan’e-jek.\\
	old\_man old\_man still lie-\textsc{intr}:2\textsc{sg}\\
	\glt \lq Old man, old man, you are cheating again.' \parencite[529]{Maslova2003}

	\ex\label{appendixSouthernYukaghirIterative2}
	 Context: A man has brought the master of the earth a barrel full of spirit in exchange for fur. Now he has come a second time.
	\exi{} \gll D’e tude boc’ka-gele \textbf{aj} køud-ej-m.\\
	\textsc{dm} 3\textsc{sg}.\textsc{gen} barrel-\textsc{acc} still take\_away-\textsc{pfv}-\textsc{tr}.3\textsc{sg}\\
	\glt \lq He brought his barrel again.' \parencite[text 25]{NikolaevaMayer2004}

	\ex\label{exAppendixKolymaAdditiveIterative}
	\gll D\rq{}e taŋ jeklie \textbf{āj} ekr-īl\rq{}i sobenn\rq{}i \textbf{āj} ejr-īl\rq{}i.\\
	\textsc{dm} that behind still walk-\textsc{intr}.1\textsc{pl} today still walk-\textsc{intr}.1\textsc{pl}\\
	\glt \lq Well, we walked beforehand, too, and we walked again today.\rq{ }\parencite[529]{Maslova2003}

	\ex Context: The narrator and his companions have gone to look after an old woman who lives alone.\label{appendixSouthernYukaghirRestitutive1}
	\exi{}\gll Jaqa-īli, jaqa-j-lu-ke los’il n’imē-l’el, los’il \textbf{aj} pēde-t-i.\\
	reach-1\textsc{pl} reach-\textsc{pfv}-1|2-\textsc{ds} fire be\_extinguished-\textsc{evid} fire still burn-\textsc{tr}-\textsc{tr}.1\textsc{pl}\\
	\glt \lq We arrived. When we arrived, we saw that the fire had gone down, so we lit the fire again.' \parencite[Text 31]{NikolaevaMayer2004}
	\ex Context: The narrator has woken up on the floor.\label{appendixSouthernYukaghirRestitutive2}
	\exi{}\gll Tāt mo-dʼe joŋžō-dʼe-nʼe-t urun-get met cʼircʼe-g-ej-lʼel-dʼe. \textbf{Aj} abud-ā-je.\\
	then say-1\textsc{sg} blanket-\textsc{propr}-\textsc{ss}.\textsc{ipfv} bed-\textsc{abl} 1\textsc{sg} jump-\textsc{iter}-\textsc{pfv}-\textsc{evid}-1\textsc{sg} still lie-\textsc{inch}-1\textsc{sg}\\
	\glt \lq I thought that I had jumped down from the bed in my sleep. I lay down again.' \parencite[Text 42]{NikolaevaMayer2004}
\end{exe}

\subsubsection{Additive and related uses}
\paragraph{Additive}\label{appendixKolymaAdditive}
\begin{itemize}
	\item \textcite[528–535]{Maslova2003}.
	\item Form and syntax: dependent on the focus of the expression.
	\begin{itemize}
		\item The variant \textit{āj} marks additive \lq too, also\rq{ } (\ref{exAppendixKolymaAdditiveAlso1}-\ref{exAppendixKolymaAdditiveAlso3}). It normally occurs in the preverbal position, but, under certain conditions, it may follow an NP that constitutes its focus, without preceding the verb.  It may be used across two adjacent clauses in a pattern reminiscent of bisyndetic coordination (\ref{exAppendixKolymaAdditiveAlso3}); also see (\ref{exAppendixKolymaAdditiveIterative}) above.
		\item The variant \textit{ajī} occurs with transitive predicates, the P-argument being the focus and the alternative denotations belonging to the same type, i.e. \lq another, more\rq{ }(\ref{exAppendixKolymaIncrement1}, \ref{exAppendixKolymaIncrement2}). It is a phrasal adjunct to the constituent containing the focus.
	\end{itemize}	
\end{itemize}

\begin{exe}
	\ex\label{exAppendixKolymaAdditiveAlso1}
	Context: People have gone gathering berries.\\
	\gll Met emdʼe juk-ō-j, lebejdī-le \textbf{āj} šaqalʼe-š-nu-m.\\
	1\textsc{sg} sibling small-\textsc{stat}-3 berries-\textsc{acc} still gather-\textsc{caus}-\textsc{ipfv}-\textsc{tr}.3\\
	\glt \lq Even though my younger brother was small, he was gathering berries, too.' \parencite[146, 160]{YukaghirTexts}
	
	\ex\label{exAppendixKolymaAdditiveAlso2}
	\gll Ōžī el-jūke lʼe-t-i kindʼe podʼerqo \textbf{āj} lʼe-t-i.\\
	water \textsc{neg}-far \textsc{cop}-\textsc{fut}-\textsc{intr}.3\textsc{sg} moon light still \textsc{cop}-\textsc{fut}-\textsc{intr}-3\textsc{sg}\\
	\glt \lq The water will not be far, and there will be moonlight, too.\rq{ }\parencite[530]{Maslova2003}
	
	\ex\label{exAppendixKolymaAdditiveAlso3}
	\gll Tude-l \textbf{āj} met-ke-t joule-dʼā-j met \textbf{āj} joule-sʼ.\\
	3\textsc{sg}-\textsc{nom} still 1\textsc{sg}-\textsc{loc}-\textsc{abl} ask-\textsc{mid}-\textsc{intr}.3\textsc{sg} 1\textsc{sg} still ask-\textsc{tr}.1\textsc{sg}\\
	\glt \lq He asked questions of me, and I of him.\rq{ }\parencite[142, 159]{YukaghirTexts}

	\ex\label{exAppendixKolymaIncrement1}
	\gll Met āj	čumuc̄-ie-je, \textbf{ajī} ningō ī-de-j.\\
	1\textsc{sg} still fish-\textsc{inch}-\textsc{intr}.1\textsc{sg} still many get\_caught-\textsc{caus}-1\textsc{pl}.\textsc{tr}\\
	\glt \lq I began to fish, too, we have caught much more.\rq{ }\parencite[532]{Maslova2003}
	
	\ex\label{exAppendixKolymaIncrement2}
	\gll \textbf{Ajī} lebejdī-k šaqalʼe-šu-l.\\
	still berries-\textsc{pred} gather-\textsc{caus}-1\textsc{pl}\\
	\glt \lq (Then) we gathered more berries.\rq{ }\parencite[532]{Maslova2003}
\end{exe}

\paragraph{Comparisons of inequality}\label{appendixKolymaComparisons}
\begin{itemize}
	\item \textcite[364]{Maslova2003}.
	\item Form: both \textit{āj} and \textit{ajī} occur in this function.
	\item Southern Yukaghir uses a from-comparative \parencite{Stassen2013}, in which the standard of comparison~-- if it is made explicit~-- is marked with the ablative case, as in \ref{exAppendixKolymaComparison1}. There is no overt marking of comparison on the predicate \parencite[364]{Maslova2003}. The use of  \textit{āj}/\textit{ajī} adds the scalar additive notion of \lq even\rq{}.
	\item Syntax: appears to be a syntactic sister to the focus.
\end{itemize}

\begin{exe}
	\ex\label{exAppendixKolymaComparison1}
	\gll Tudel mit-ket \textbf{āj} omosʼ modo-j.\\
	3\textsc{sg} 1\textsc{pl}-\textsc{abl} still well sit-\textsc{intr}.3\textsc{sg}\\
	\glt \lq He lives even better than we do.' \parencite[364]{Maslova2003}

	\ex 
	Context: The narrator and his companions have heard that the moon is full of candies and are eager to get there. After hearing another story about the nice things to find there, their desire has increased further.\\
	\gll  Mit \textbf{ajī} cʼom-ō-ʁete erʼdē-īli cʼugō-n, tamun kindʼe-ge jaqa-īli monu-t.\\
	1\textsc{pl} still big-\textsc{vblz}-\textsc{adv} wish-1\textsc{pl} fast-\textsc{adv} that moon-\textsc{loc} reach-1\textsc{pl} say-\textsc{ss}.\textsc{ipfv}\\
	\glt \lq We wanted even more to go quickly to the moon.' \parencite[Text 50]{NikolaevaMayer2004}
	
	\ex
	Context: In order to prepare Perch for a dangerous trip, other fish have provided him with an iron coat. That has proven not enough armour.\\
	\gll \textbf{Ajī} omosʼ adi ā-gi nado.\\
	still well strongly make-\textsc{poss}.3 necessary\\
	\glt \lq It is necessary to strengthen him even more!' \parencite[565, 573]{Maslova2003}
	

\end{exe}

\il{Spanish|(}
\section{Spanish (spa, stan1288)}
\subsection{Introductory remarks}
Apart from descriptive materials, I consulted the \textit{Corpus del Español del Siglo XXI} (CORPES XXI) (\url{https://www.rae.es/banco-de-datos/corpes-xxi}). I am furthermore indebted to  Kristian Roncero for discussing Spanish data with me.

\subsection{aún}
\subsubsection{General information}
\begin{itemize}
	\item Form: usually written as \textit{aun} when unstressed.
	\item Wordhood: independent grammatical word, invariable; some authors (e.g. \cite{Trujillo1990}) treat the unstressed form as a proclitic.
	\item Etymology: from a Latin phrase \lq until today.'
\end{itemize}
\il{Yukaghir, Southern|)}

\subsubsection{As a \lq{}still\rq{ }expression}
\begin{itemize}
	\item \Textcite{vanderAuwera1998}, \textcite{Garrido1993}, \textcite{OlivaresSorpena2001}, \citeauthor{RAEGramatica} (\citeyear[§30.8r]{RAEGramatica}, \citeyear[s.v. \textit{aun}]{RAEDictionary}) and \textcite{Trujillo1990}, among others.
	\item Specialisation: identified as a marker that is in line with my definition by \textcite{vanderAuwera1998}; the description by \textcite{RAEGramatica} directly addresses both meaning components.
	\item Polarity sensitivity: inner negation yields \textsc{not yet}.
	\item Pragmaticity: compatible with both scenarios.
	\item Further note: \textit{aún} is attested as an elliptical one-word answer (\ref{appendixSpanishAun3}).
\end{itemize}
\begin{exe}
	\ex
	\gll Aquí empieza la verdader-a historia d-el Ojo. En aquel tiempo \textbf{aún} viv-ía en París y su-s foto-s iba-n a ilustr-ar un texto de un conoc-id-o escritor francés que se hab-ía  especializ-ado en el sub-mundo de la prostitución.\\
	here begin.3\textsc{sg} \textsc{def}.\textsc{sg}.\textsc{f} true-\textsc{f} story(\textsc{f}) of-El Ojo in \textsc{dist}.\textsc{sg}.\textsc{m} time(\textsc{m}) still live-\textsc{pst}.\textsc{ipfv}.3\textsc{sg} in Paris and \textsc{poss}.3-\textsc{pl} foto-\textsc{pl} go.\textsc{pst}.\textsc{ipfv}-3\textsc{pl} to illustrate-\textsc{inf} \textsc{indef}.\textsc{sg}.\textsc{m} text(\textsc{m}) of \textsc{indef}.\textsc{sg}.\textsc{m} know-\textsc{ptcp}-\textsc{m} writer(\textsc{m}) french \textsc{subord} \textsc{refl}.3 have-\textsc{pst}.\textsc{ipfv}.3\textsc{sg} specialize-\textsc{ptcp} en \textsc{def}.\textsc{sg}.\textsc{m} under-world(\textsc{m}) of \textsc{def}.\textsc{sg}.\textsc{f} prostitution(\textsc{m})\\
	\glt \lq Here the true story of El Ojo begins. In those days he still lived in Paris and his photos would illustrate a text by a well-known French author that had specialised in the underworld of prostitution.' (CORPES XXI, glosses added)
	\ex 
	\gll Cálma-te ahora. Deja que el tiempo pas-e, \textbf{aún} es muy pronto.\\
calm\_down.\textsc{imp}-\textsc{refl}.2\textsc{sg} now let.\textsc{imp} \textsc{subord} \textsc{def}.\textsc{sg}.\textsc{m} time(\textsc{m}) pass-\textsc{sbjv}.3\textsc{sg} still \textsc{cop}.3\textsc{sg} very early\\
	\glt \lq Calm down. Let time pass, it's still very early.\rq{ }(CORPES XXI, glosses added)

	\ex\label{appendixSpanishAun3}
	\textit{-Y aún estás esperando, ¿no? -dijo Jordi.}\\
	\glt \lq{\lq\lq}And you are still waiting, no?" Jordi said.
	\exi{}\textit{Abdul dio un manotazo en la mesa.}\\
	\glt \lq Abdul smacked his hand on the table.'
	\exi{}\gll -\textbf{Aún} -grit-ó riendo-. \textbf{Aún}.\\
	\phantom{-}still\phantom{-} shout-\textsc{pst}.\textsc{pfv}.3\textsc{sg} laugh.\textsc{ptcp} still\\
\glt \lq {\lq\lq}[I] Still [am]" he cried, laughing. \lq\lq [I] Still [am].{\rq\rq}\rq{ }(CORPES XXI, glosses added)
\end{exe}


\subsubsection{Uses on the fringes of \lq{}still\rq{}}
\paragraph{Scalar contexts}
\label{appendixSpanishAunScalar}
\begin{itemize}
	\item \textit{Aún} is compatible with scalar contexts, both involving decreases (\ref{exAppendixSpanishAunScalar1}, \ref{exAppendixSpanishAunScalar2}) and limited increases (\ref{exAppendixSpanishAunScalar3}, \ref{exAppendixSpanishAunScalar4}). To disambiguate between the two directions of change, additional expressions can be used, such as \textit{quedar} \lq remain\rq{ }in (\ref{exAppendixSpanishAunScalar2}), or \textit{sólo} \lq only\rq{ }in (\ref{exAppendixSpanishAunScalar3}). Note, however, the absence of an \lq only\rq{ }marker in (\ref{exAppendixSpanishAunScalar4}, \ref{exAppendixSpanishAunScalar5}).
	\item In this use, \textit{aún} also combines with time expressions, yielding \lq no later than\rq{}, as in  (\ref{exAppendixSpanishAunScalar5}).
\end{itemize}
\begin{exe}
	\ex\label{exAppendixSpanishAunScalar1}
	\gll En América Latin-a -inclu-id-a Venezuela- \textbf{aún} exist-e un alt-o porcentaje de persona-s que cocin-an con leña.\\
	in America(\textsc{f}) latin-\textsc{f} \phantom{-}include-\textsc{ptcp}-\textsc{f} V. still exist-3\textsc{sg} \textsc{def}.\textsc{sg}.\textsc{m} high-\textsc{m} percentage(\textsc{m}) of person-\textsc{pl} \textsc{subord} cook-3\textsc{pl} with firewood\\
	\glt In Latin America, including Venezuela, there is still a large percentage of people that cook with firewood.' (CORPES XXI, glosses added)

	\ex\label{exAppendixSpanishAunScalar2}
	\gll \textbf{Aun} qued-an, {sin embargo}, dos mes-es de campaña y cualquier cosa puede suced-er.\\
	still remain-3\textsc{pl} {nonetheless} two month-\textsc{pl} of campaign and any thing can.3\textsc{sg} happen-\textsc{inf}\\
	\glt \lq There are, however, two months still left to the campaign, and anything can happen.\rq{ }(CORPES XXI, glosses added)

	\ex\label{exAppendixSpanishAunScalar3}
	Context: From the announcement of a book about Soviet Gulags.
	\exi{}\gll  …la primer-a historia pormenoriz-ad-a y solvente de un mundo \textbf{aún} sólo parcialmente conoc-id-o en tod-o su horror.\\
	\phantom{…}\textsc{def}.\textsc{sg}.\textsc{f} first-\textsc{f} history(\textsc{f}) describe\_in\_detail-\textsc{ptcp}-\textsc{f} and dependable of \textsc{indef}.\textsc{sg}.\textsc{m} world(\textsc{m}) still only partially know-\textsc{ptcp}-\textsc{m} in all-\textsc{m} \textsc{poss}.3 horror(\textsc{m})\\
	\glt \lq …the first detailed and reliable history of a world that is still only partially known in all its horror.' (CORPES XXI, glosses added)
	\pagebreak
	\ex\label{exAppendixSpanishAunScalar4}
	\gll En 1992 una bebé con \textbf{aún} unos poc-o-s mes-es de vida fue llev-ad-a desde Paraguay a los Estado-s Unid-o-s.\\
	in 1992 \textsc{indef}.\textsc{sg}.\textsc{f} baby with still \textsc{indef}.\textsc{pl}.\textsc{m} few-\textsc{m}-\textsc{pl} month(\textsc{m})-\textsc{pl} of life \textsc{cop}.\textsc{pst}.\textsc{pfv}.3\textsc{sg} carry-\textsc{ptcp}-\textsc{f} from P. to \textsc{def}.\textsc{pl}.\textsc{m} state(\textsc{m})-\textsc{pl} united-\textsc{m}-\textsc{pl}\\
	\glt \lq In 1992, a (female) baby that was still only a few years old was taken from Paraguay to the United States.\rq{ }(found online, glosses added)\footnote{\url{https://www.abc.com.py/tag/mariana-grala/} (23 January, 2023).}
	
	\ex\label{exAppendixSpanishAunScalar5}
	Context: The negatives from a hotel review.\\
	\gll Que \textbf{aun} era-n las 8:30am y la persona encarg-ad-a d-el desayuno dij-o que ya no saca-ría más huevo.\\
	\textsc{comp} still \textsc{cop}.\textsc{pst}.\textsc{ipfv}-3\textsc{pl} \textsc{def}.\textsc{pl}.\textsc{f} 8:30am and \textsc{def}.\textsc{sg}.\textsc{f} person(\textsc{f}) assign-\textsc{ptcp}-\textsc{f} of-\textsc{def}.\textsc{sg}.\textsc{m} breakfast(\textsc{m}) say.\textsc{pst}.\textsc{pfv}-3\textsc{sg} \textsc{comp} already \textsc{neg} take\_out-\textsc{cond}.3\textsc{sg} more egg\\
	\glt \lq (That) it was only 8:30am and the person in charge of breakfast said they wouldn't bring out egg (dishes) anymore.\rq{ }(found online, glosses added)\footnote{\url{https://planetofhotels.com/es/estados-unidos/phoenix/la-quinta-wyndham-phoenix-chandler} (23 January, 2023).}
\end{exe}

\subsubsection{Uses relating to other phasal polarity concepts}
\paragraph{Interrogative \lq yet\rq}
\label{appendixSpanishAunInterrogativeYet}
\begin{itemize}
	\item Like \textit{todavía} (\ref{appendixSpanishTodaviaInterrogativeYet}), \textit{aún} is attested as \textsc{not yet} with negative raising (\ref{exAppendixSpanishAunInterrogativeYet2}) and, closely related, in complement clauses of attitude verbs with an inherently negative component (\ref{exAppendixSpanishAunInterrogativeYet1}).
\end{itemize}
\begin{exe}
	\ex\label{exAppendixSpanishAunInterrogativeYet2}
	 \gll No cre-o que el tren h-aya lleg-ado \textbf{aún}.\\
	\textsc{neg} believe-1\textsc{sg} \textsc{subord} \textsc{def}.\textsc{sg}.\textsc{m} train(\textsc{m}) have-\textsc{sbjv}.3\textsc{sg} arrive-\textsc{ptcp} still\\
	\glt \lq I don't think the train has arrived yet.' (Kristian Roncero, p.c.)

	\ex \label{exAppendixSpanishAunInterrogativeYet1}
	\gll Cré-an-me, no hay político en la Argentina superior a CFK en este siglo, y dud-o que haya nac-ido \textbf{aún}.\\
	believe-\textsc{sbjv}.3\textsc{pl}-1\textsc{sg}.\textsc{obj} \textsc{neg} \textsc{exist} politician in \textsc{def}.\textsc{sg}.\textsc{f} A. superior to CFK in \textsc{prox}.\textsc{sg}.\textsc{m} century(\textsc{m}) and doubt-1\textsc{sg} \textsc{subord} have.\textsc{sbjv}.3\textsc{sg} be\_born-\textsc{ptcp} still\\
	\glt \lq Believe me, there is (currently) no better politician in this century's Argentina than CFK, and I doubt that one has been born yet.\rq{ }(found online, glosses added)\footnote{\url{https://www.facebook.com/profile/100063285907281/search/?q=nacido aun} (28 February, 2023).}
\end{exe}




\subsubsection{Temporal connectives and frame setters}
\paragraph{Time-scalar additive (\lq as late as, as early as\rq{})}\label{appendixSpanishAunTimeScalar}
\begin{itemize}
	\item \textcite{Trujillo1990}, 
	\item In this function, \textit{\textit{aún}} combines with a time frame adverbial and operates on a scale of time proper. More commonly, the alternatives are lower times (\lq as late as\rq{}), as in (\ref{appendixSpanishAunTimeScalar1}, \ref{appendixSpanishAunTimeScalar2}). But there is also the occasional case in which the alternatives rank higher (\lq as early as\rq{}), such as in (\ref{appendixSpanishAunTimeScalar3}).
	\item This function could be subsumed under the more general scalar additive function of \textit{aún} (\appref{appendixSpanishAunScalarAdditive}), which allows for both lower and higher alternatives. The \lq as late as\rq{ }use, however, finds additional motivation in the development described in \Cref{sectionTimeScalar}.
	\item Syntax: can form a single constituent with its focus.
\end{itemize}

\begin{exe}

	\ex\label{appendixSpanishAunTimeScalar1}
	Context: From a text about the origins of jewellery. The preceding sentences discusses talismans in pre-history.\\
	\gll y \textbf{aún} \textbf{en} \textbf{la} \textbf{Edad} \textbf{Medi}-\textbf{a}, dentro de la Europa cristianiz-ad-a, a-l uso de ciert-a-s piedra-s precios-a-s se le otorg-aba divers-a-s propiedade-s:\\
	and still in \textsc{def}.\textsc{sg}.\textsc{f} age(\textsc{f}) medium-\textsc{f} inside of \textsc{def}.\textsc{sg}.\textsc{f} Europe(\textsc{f}) christianize-\textsc{ptcp}-\textsc{f} to-\textsc{def}.\textsc{sg}.\textsc{m} use(\textsc{m}) of certain-\textsc{f}-\textsc{pl} stone(\textsc{f})-\textsc{pl} precious-\textsc{f}-\textsc{pl} \textsc{refl}.3 3\textsc{sg}.\textsc{dat}.\textsc{m} assign-\textsc{pst}.\textsc{ipfv}.3\textsc{sg} diverse-\textsc{f}-\textsc{pl} property(\textsc{f})-\textsc{pl}\\
	\glt \lq \textbf{and as late as in medieval times}, within christianised Europe, the use of certain precious stones was (still) associated with a number of properties.\rq{ }(found online, glosses added)\footnote{\url{https://vestuarioescenico.wordpress.com/2014/01/31/historia-del-boton/} (16 February, 2023).}
	\largerpage[-1]\pagebreak
	\ex\label{appendixSpanishAunTimeScalar2}
	Context: About the erruption of a volcano several days ago. Researchers have confirmed that …\\
	\gll \textbf{aún} ayer se present-aba descenso de materiale-s y una nube de polvo y ceniza que se mov-ió con el viento hasta uno-s cuatro kilómetro-s d-el cráter.\\
still yesterday \textsc{refl}.3 present-\textsc{pst}.\textsc{ipfv}.3\textsc{sg} decent of material-\textsc{pl} and \textsc{indef}.\textsc{sg}.\textsc{f} cloud(\textsc{f}) of dust and ash \textsc{subord} \textsc{refl}.3 move-\textsc{pst}.\textsc{pfv}.3\textsc{sg} with \textsc{def}.\textsc{sg}.\textsc{m} wind(\textsc{m}) until \textsc{indef}.\textsc{m}-\textsc{pl} four kilometre(\textsc{m})-\textsc{pl} of-\textsc{def}.\textsc{sg}.\textsc{m} crater(\textsc{m})\\
	\glt \lq even yesterday there was (still) matter descending and a cloud of dust and ashes that moved with the wind until around four kilometres from the crater.\rq{ }(CORPES XXI, glosses added)
	
	\ex Context: From the description of a medical trial. If three attempts of a procedure failed…\label{appendixSpanishAunTimeScalar3}\\
	\gll … o se juzg-ó peligroso realiz-ar-lo \textbf{aun} en el primer intento, se decid-ió suspend-er el procedimiento.\\
	{} or \textsc{refl}.3 judge-\textsc{pst}.\textsc{pfv}.3\textsc{sg} dangerous carry\_out-\textsc{inf}-\textsc{3}\textsc{sg}.\textsc{acc}.\textsc{m} still in \textsc{def}.\textsc{sg}.\textsc{m} first attempt \textsc{refl}.3 decide-\textsc{pst}.\textsc{pfv}.3\textsc{sg} cancel-\textsc{inf} \textsc{def}.\textsc{sg}.\textsc{m} procedure(\textsc{m)}\\
	\glt \lq … or if even/as early as at the first attempt this was judged to be dangerous, the procedure was abandoned.\rq{ }(found online, glosses added)\footnote{\url{http://www.scielo.org.ar/scielo.php?script=sci_arttext&pid=S2250-639X2015000400003} (20 February, 2023).}
\end{exe}

\subsubsection{Marginality}
\label{appendixSpanishAunMarginal}
\begin{itemize}
	\item \textit{Aún} is compatible with construals of marginality. 
\end{itemize}
\begin{exe}

	\ex\label{exAppendixSpanishAunMarginal1}
	\gll Con Juan \textbf{aún} me atrev-o (pero con Pedro ya no).\\
	with J. still \textsc{refl}.1\textsc{sg} dare-1\textsc{sg} \phantom{(}but wih P. already \textsc{neg}\\
	\glt \lq I can still argue with John (but not with Peter).' (Kristian Roncero, p.c.)
	\largerpage[-1]\pagebreak
	\ex\label{exAppendixSpanishAunMarginal2}
	Context: Complaining about overly worried parents-in-law.\\
\gll Y que lo hag-an mi-s padr-es \textbf{aún} lo aguant-o pero me jod-e que unos señor-es que no son nada mio-s a-l final, se tom-en esa preocupación como si fuera su hij-a.\\
and \textsc{subord} 3\textsc{sg}.\textsc{n} do.\textsc{sbjv}-3\textsc{pl} \textsc{poss}.1\textsc{sg}-\textsc{pl} parent-\textsc{pl} still 3\textsc{sg}.\textsc{m}.\textsc{acc} bear-1\textsc{sg} but 1\textsc{sg}.\textsc{obj} fuck-3\textsc{sg} \textsc{subord} \textsc{indef}.\textsc{pl}.\textsc{m} mister-\textsc{pl} \textsc{subord} \textsc{neg} \textsc{cop}.3\textsc{pl} nothing \textsc{poss}.1\textsc{sg}-\textsc{pl} at-\textsc{def}.\textsc{sg}.\textsc{m} end(\textsc{m}) \textsc{refl}.3 take-\textsc{sbjv}.3\textsc{pl} \textsc{dem}.\textsc{sg}.\textsc{f} worry(\textsc{f}) like if \textsc{cop}.\textsc{pst}.\textsc{sbjv}.1\textsc{sg} \textsc{poss}.3 child-\textsc{f}\\
\glt \lq I can still stand my parents doing that, but it fucks me up when some people who aren't related to me at all worry about me as if I were their daughter.' (found online, glosses added)\footnote{\url{https://weloversize.com/topic/situacion-rara-con-mis-suegros/} (30 March, 2022).}

	\ex\label{exAppendixSpanishAunMarginal3}
	\gll Los coche-s compact-o-s \textbf{aún} son segur-o-s en la autopista; los utilitario-s ya no lo son tanto.\\
	\textsc{def}.\textsc{pl}.\textsc{m} car(\textsc{m})-\textsc{pl} compact-\textsc{m}-\textsc{pl} still \textsc{cop}.3\textsc{pl} safe-\textsc{m}-\textsc{pl} in \textsc{def}.\textsc{sg}.\textsc{f} highway(\textsc{f}) \textsc{def}.\textsc{pl}.\textsc{m} subcompact\_car(\textsc{m})-\textsc{pl} already \textsc{neg} 3\textsc{sg}.\textsc{n} \textsc{cop}.3\textsc{sg} that\_much\\
	\glt \lq Compact cars are still safe on the highway; subcompact cars, not that much.' (Kristian Roncero, p.c.)

	\ex\label{exAppendixSpanishAunMarginal4}
	\gll Irún \textbf{aún} es España y Hendaya ya es Francia.\\
	I. still \textsc{cop}.3\textsc{sg} Spain and H. already \textsc{cop}.3\textsc{sg} France\\
	\glt \lq Irún is still in Spain and Hendaya is in France already.\rq{ }(Kristian Roncero, p.c.)
\end{exe}


\subsubsection{Additive and related functions}
\paragraph{Additive}\label{appendixSpanishAunAdditive}\largerpage[2]
\begin{itemize}
	\item \textcite{Garrido1993} and \textcite[§§30.8r, 40.8b]{RAEGramatica}.
\end{itemize}

\begin{exe}
	\ex
	\gll El español \textbf{aún} permanec-ió unos instante-s muy quiet-o-s.\\
	\textsc{def}.\textsc{sg}.\textsc{m} spaniard(\textsc{m}) still remain-\textsc{pst}.\textsc{pfv}.3\textsc{sg} \textsc{indef}.\textsc{pl}.\textsc{m} moment(\textsc{m})-\textsc{pl} very quiet-\textsc{m}-\textsc{pl}\\
	\glt \lq The Spaniard remained very quiet for a few more moments.' (Vázquez-Figueroa, \textit{Caribes}, cited in \cite[§30.8r]{RAEGramatica}, glosses added).
	
	\ex \textit{«¿Indecente?», preguntó ella. «No veo por qué iba a ser indecente desnudarse delante de un hombre para servirle de modelo», añadió.}\\
	\lq {\lq\lq}Indecent?\rq\rq{ }she asked. \lq\lq I don't see how it would be indecent to undress in front of a man to serve him as a model\rq\rq, she added.\rq
	\exi{}\gll  Y a continuación, \textbf{aún} dij-o: «{Por supuesto}, si Indalecio me lo pidiese, yo no ten-dría inconveniente. En el arte, como él sab-e muy bien, no hay pecado»\\
	and to continuation still  say.\textsc{pst}.\textsc{pfv}-3\textsc{sg} \phantom{«}{of course}, if I. 1\textsc{sg}.\textsc{obj} 3\textsc{sg}.\textsc{acc}.\textsc{m} ask\_for.\textsc{pst}.\textsc{sbjv}.3\textsc{sg} 1\textsc{sg} \textsc{neg} have-\textsc{cond}.1\textsc{sg} inconvenience in \textsc{def}.\textsc{sg}.\textsc{m} art(\textsc{m}) like 3\textsc{sg}.\textsc{m} know-3\textsc{sg} very well \textsc{neg} \textsc{exist} sin\\
	\glt \lq And then she also said \lq\lq Of course, if Indalecio asked me for it, I wouldn't have a problem. In art, as he knows very well, there is no sin."' (CORPES XXI, glosses added)
\end{exe}

\paragraph{Scalar additive}
\label{appendixSpanishAunScalarAdditive}
\begin{itemize}
	\item \textcite{Elvira2005}, \textcite[54–55]{FuentesRodriguez2018}, \textcite{Garrido1993}, \textcite{GastvanderAuwera2011}, \textcite{Cid1999}, \citeauthor{RAEGramatica} (\citeyear[§§30.8r, 40.8b]{RAEGramatica}; \citeyear[s.v. \textit{aun}]{RAEDictionary}) and \textcite{Trujillo1990}.
	\item Form: negated via \textit{ni} \lq neither, nor' (\ref{exAppendixSpanishAunScalarAdditive4}).
	\item As \textcite{GastvanderAuwera2011} show, \textit{aun} serves as a \lq\lq universal" scalar additive, i.e. both particularly high focus values (\ref{exAppendixSpanishAunScalarAdditive1}) and particularly low ones (\ref{exAppendixSpanishAunScalarAdditive3}, \ref{exAppendixSpanishAunScalarAdditive4}) can be vested with a high degree of pragmatic strength.
	\item According to \textcite{Trujillo1990}, this use goes back to the conventionalisation of an erstwhile scalar inference in examples like (\ref{exAppendixSpanishAunScalarAdditive4}).\il{Spanish, Old} 
	\item Syntax: can form a constituent with its focus.
\end{itemize}

\begin{exe}
	\ex\label{exAppendixSpanishAunScalarAdditive1}
	\gll ¿Quién es éste, que mand-a \textbf{aun} a los viento-s y a-l agua, y le obedec-en?\\
	\phantom{¿}who \textsc{cop}.3\textsc{sg} \textsc{prox}.\textsc{sg}.\textsc{m} \textsc{subord} command-3\textsc{sg} still \textsc{acc} \textsc{def}.\textsc{pl}.\textsc{m} wind(\textsc{m})-\textsc{pl} and \textsc{acc}-\textsc{def}.\textsc{sg}.\textsc{m} water and 3\textsc{sg}.\textsc{dat}.\textsc{m} obey-3\textsc{pl}\\
	\glt \lq Who is this? He commands even the winds and the water, and they obey him.' (Luke 8:25, \textit{La biblia al día}, cited in \cite[3]{GastvanderAuwera2011})
	
	\ex\label{exAppendixSpanishAunScalarAdditive2}
	\gll ... d-a vergüenza \textbf{aun} mencion-ar lo que los desobediente-s hac-en en secreto.\\
	{} give-3\textsc{sg} shame still mention-\textsc{inf} 3\textsc{sg}.\textsc{n} \textsc{subord} \textsc{def}.\textsc{pl}.\textsc{m} disobedient-\textsc{pl} do-3\textsc{pl} in secret\\
	\glt \lq It is shameful to even mention what the disobedient do in secret.' (Eph. 5:12, \textit{La biblia al día}, cited in \cite[2]{GastvanderAuwera2011})
	
	\ex\label{exAppendixSpanishAunScalarAdditive3}
	\gll No ten-go yo tanto, ni \textbf{aun} la mitad.\\
\textsc{neg} have-1\textsc{sg} 1\textsc{sg} that\_much nor still \textsc{def}.\textsc{sg}.\textsc{f} half(\textsc{f})\\
\glt \lq I don't have that much, not even half as much.\rq{ }(\cite[s.v. \textit{aun}]{RAEDictionary}, glosses added)

	\ex Old Spanish,\il{Spanish, Old}  11\textsuperscript{th}/12\textsuperscript{th} century\label{exAppendixSpanishAunScalarAdditive4}\\
	\textit{Aquel que gela diesse sopiesse una palabra // Que perderie los aueres e mas los oios de la cara //}\\
	\lq Let anyone who might give it to him know, that he would lose his possessions and the eyes of his head\rq{}\\
	\gll  E \textbf{aun} demas los cuerpo-s e las alma-s.\\
	and still  other \textsc{def}.\textsc{pl}.\textsc{m} body(\textsc{m})-\textsc{pl} and \textsc{def}.\textsc{pl}.\textsc{f} soul(\textsc{f})-\textsc{pl}\\
	\glt \lq And, furthermore, also/even his body and soul.\rq{ }(\textit{Cantar de mio Cid}, cited in \cite[79]{Trujillo1990}, glosses added)	
	
\end{exe}

\paragraph{Scalar additive \lq so much as': \textit{aunque sea}}
\label{appendixSpanishAunqueSea}
\begin{itemize}
	\item \textcite{GastvanderAuwera2011} and \textcite[§47.12q]{RAEGramatica}.
	\item Form: in the collocation \textit{aun}-\textit{que} \textit{sea} \lq still-\textsc{subord} \textsc{cop}.\textsc{sbjv}.3\textsc{sg}', i.e. lit. \lq even if it were\rq{}.
	\item In \citeauthor{GastvanderAuwera2011}'s (\citeyear{GastvanderAuwera2011}) terminology, this is a \textsc{beneath} operator (\lq so much as\rq{}) that is restricted to non-negative contexts.
		\item As \textcite{GastvanderAuwera2011} point out, this collocation probably goes back to a parenthetical scalar concessive conditional involving \textit{aunque} plus subjunctive mood (\appref{appendixSpanishAunque}) and the elision of the first occurence of the focus; that is, ex. (\ref{exAppendixSpanishAunqueSea1}) goes back to \textit{Si dices \sout{algo}, aunque sea una palabra}… \lq if you say something, even if it's just one word…'. \textcite[§47.12q]{RAEGramatica} point out that in European Spanish \textit{aunque sea} is often used in a postposed, prosodically set-off manner, as in (\ref{exAppendixSpanishAunqueSea2}). This supports \citeauthor{GastvanderAuwera2011}'s interpretation.
\end{itemize}

\begin{exe}
	\ex\label{exAppendixSpanishAunqueSea1}
	\gll Si dices \textbf{aun}-\textbf{que} \textbf{sea} una palabra, v-as a ten-er problema-s.\\
	if say.2\textsc{sg} still-\textsc{subord} \textsc{cop}.\textsc{sbjv}.3\textsc{sg} \textsc{indef}.\textsc{sg}.\textsc{f} word(\textsc{f}) go-2\textsc{sg} to have-\textsc{inf} problem-\textsc{pl}\\
	\glt \lq If you say even one word, you’ll get into trouble.\rq{ }\parencite[356]{GastvanderAuwera2011}
	
	\ex\label{exAppendixSpanishAunqueSea2}
	\begin{xlist}
		\exi{A:} \textit{Demasiadas preguntas... demasiadas preguntas...}
		\\ \lq Too many questions… too many questions\rq{}
		\exi{B:} \gll Contesta \textbf{aun}-\textbf{que} \textbf{sea} un-a.\\
		answer.\textsc{imp} still-\textsc{subord} \textsc{cop}.\textsc{sbjv}.3\textsc{sg} one-\textsc{f}\\
		\glt \lq Answer so much as one of them.\rq{ }(CORPES XXI, glosses added)
	\end{xlist} 
	
	\ex\label{exAppendixSpanishAunqueSea3}
	\gll Da-me una galleta, \textbf{aun}-\textbf{que} \textbf{sea}, que ten-go hambre.\\
	give.\textsc{imp}-\textsc{obj}.1\textsc{sg} \textsc{indef}.\textsc{sg}.\textsc{f} cookie(\textsc{f}) still-\textsc{subord} \textsc{cop}.\textsc{sbjv}.3\textsc{sg} \textsc{subord} have-1\textsc{sg} hunger\\
	\glt \lq Give me so much as a cookie, I'm hungry.\rq{ }(\cite[§47.12q]{RAEGramatica}, glosses added)
\end{exe}

\paragraph{Comparisons of inequality}
\label{appendixSpanishAunComparatives}
\begin{itemize}
	\item \textcite[565]{FuentesRodriguez2018}, \textcite{Cid1999} and \citeauthor{RAEGramatica} (\citeyear[§30.8r]{RAEGramatica}, \citeyear[s.v. \textit{aun}]{RAEDictionary}).
	\item Note that comparisons of inequality in Spanish are formed via \textit{más} \lq more' / \textit{menos} \lq less' (with the exception of some adjectives that have suppletive comparative forms). The standard of comparison, if overtly mentioned, is  introduced by subordinator \textit{que} \parencite[ch. 45]{RAEGramatica} 	
	\item \textit{Aún} in comparisons of inequality yields the scalar additive notion \lq even more\rq{}. This extends to the modification of degree achievements (\ref{exAppendixSpanishAunComparisons3}).
	\item Syntax: forms a constituent with its focus.
\end{itemize}
\begin{exe}
	\ex \gll Me dol-ía su belleza. Era \textbf{aún} más guap-a de lo que la hab-ía imagin-ado.\\
	1\textsc{sg}-\textsc{obj} hurt-\textsc{pst}.\textsc{ipfv}.1\textsc{sg} \textsc{poss}.3 beauty
	\textsc{cop}.\textsc{pst}.\textsc{ipfv}.1\textsc{sg} still more attractive-\textsc{f} of 3\textsc{sg}.\textsc{n} \textsc{subord} 3\textsc{sg}.\textsc{acc}.\textsc{f} have-\textsc{pst}.\textsc{ipfv}.1\textsc{sg} imagine-\textsc{ptcp}\\
	\glt \lq Her beauty hurt me. She was even prettier than what I imagined her to be.' (CORPES XXI, glosses added)

	\ex \textit{El virus está en constante mutación, y éste podría acumular suficientes cambios genéticos como para volverse contagioso de persona a persona. Hasta ahora, los casos humanos han derivado de aves.}\\
	\lq The virus is constantly mutating, and this could lead to the accumulation of so many genetic changes that it becomes transmissible between people. As of yet, the cases of human infection have come from birds.'
	\exi{}\gll 	La otr-a posibilidad, \textbf{aún} peor, sería un cambio repentin-o caus-ad-o por la combinación d-el virus con el de influenza human-a en el cuerpo de una persona.\\
	\textsc{def}.\textsc{sg}.\textsc{f} other-\textsc{f} possibility(\textsc{f}) still worse \textsc{cop}.\textsc{cond}.3\textsc{sg} \textsc{indef}.\textsc{sg}.\textsc{m} change(\textsc{m}) sudden-\textsc{m} cause-\textsc{ptcp}-\textsc{m} for \textsc{def}.\textsc{sg}.\textsc{f} combination(\textsc{f}) of-\textsc{def}.\textsc{sg}.\textsc{m} virus(\textsc{m}) with 3\textsc{sg}.\textsc{m} of flue(\textsc{f}) human-\textsc{f} in \textsc{def}.\textsc{sg}.\textsc{m} body(\textsc{m}) of \textsc{indef}.\textsc{sg}.\textsc{f} person(\textsc{f})\\
	\glt \lq The other possibility, even worse, would be a sudden change caused by the merging of the virus with the human influenza virus in a person's body.' (CORPES XXI, glosses added)
	
	\ex\label{exAppendixSpanishAunComparisons3}
	\textit{Aunque simplista y exagerada, la tesis del conflicto de las generaciones multiplicado por el poder "apalancador" de Internet, brinda nuevos bríos y encarnaciones a las hipótesis de Margaret Mead (1980)…}\\
	\lq Despite being simplistic and exaggerated, the thesis of a generational conflict that is multiplied by the \lq\lq leveraging" power of the internet reaches a new level in the hypothesis of Margaret Mead (1980)…'
	\gll Para complejiz-ar \textbf{aún} más este ya de por sí confus-o panorama, vé-an-se los aporte-s de Régis Debray (2001) …\\
	for complicate-\textsc{inf} still more \textsc{prox}.\textsc{sg}.\textsc{m} already of for yes confuse-\textsc{m} panorama(\textsc{m}) see-\textsc{sbjv}.3\textsc{pl}-\textsc{refl}.3 \textsc{def}.\textsc{pl}.\textsc{m} contribution(\textsc{m})-\textsc{pl} of R. D. \\
	\glt \lq To complicate this in and by itself confusing panorama even more, see the contributions by Regis Debray (2001)…' (CORPES XXI, glosses added)
\end{exe}



\paragraph{Conjunctional adverbial: \textit{más aún}/\textit{aún más} \lq what is more'}
\label{appendixSpanishAunConjunctionalAunMas}
\begin{itemize}
	\item \textcite[s.v. más aún]{DPDE} and \textcite[346]{FuentesRodriguez2018}.
	\item Form: in collocation with \textit{más} \lq more'.
	\item This is a straightforward extension of the use in comparatives (\ref{appendixSpanishAunComparatives}) to the textual domain and parallel to \textit{todavía más}/\textit{todavía más} in the same function (\ref{appendixSpanishTodaviaConjunctionalTodaviaMas}).
	\item Syntax: clause-initial position.
\end{itemize}
\largerpage[2]

\begin{exe}
	\ex \gll En La Habana hay también gente pobre. \textbf{Más} \textbf{aún}, hay esclavo-s.\\
	en \textsc{def}.\textsc{sg}.\textsc{f} H. \textsc{exist} also people poor more still \textsc{exist} slave-\textsc{pl}\\
\glt \lq There are also poor people in Havana. What is more, there are slaves.' (CORPES XXI, glosses added)
	
	\ex \textit{Para ser imparciales con el Dictador hay que añadir que Correa no sólo no le dio ninguna satisfacción con respecto a sus reclamos...}\\
	\lq To be impartial towards the dictator, it has to be added that Correa not only did not satisfy his demands…'
	\exi{}\gll \textbf{Aún} \textbf{más}: Pretend-ió engañ-ar-le con respecto a las arma-s promet-id-a-s.\\
	still more pretend-\textsc{pst}.\textsc{pfv}.3\textsc{sg} deceive-\textsc{inf}-3\textsc{sg}.\textsc{dat}.\textsc{m} with respect to \textsc{def}.\textsc{pl}.\textsc{f} weapon(\textsc{f})-\textsc{pl} promise-\textsc{ptcp}-\textsc{f}-\textsc{pl}\\
	\glt \lq What is more, he tried to deceive him in regard to the arms that had been promised.' (CORPES XXI, glosses added)
\end{exe}

\subsubsection{Broadly modal and interactional uses}
\paragraph{Concessive protases (i): \textit{aun} + gerund}
\label{appendixSpanishAunGerundConcessive}
\begin{itemize}
	\sloppy
	\item \textcite{FernandezLagunilla1999}, \textcite{PerezSaldanyaVincent2014}, \textcite[§27.5i]{RAEGramatica} and \textcite{Trujillo1990}.
	\item Form: in combination with gerunds.
	\item This collocation serves to introduce the protasis of ordinary concessives (\ref{exAppendixSpanishAunGerundConcessive1}, \ref{exAppendixSpanishAunGerundConcessive2}), as well as of scalar concessive conditionals (\ref{exAppendixSpanishAunGerundConcessive3}, \ref{exAppendixSpanishAunGerundConcessive4}).
	\item As pointed out by \textcite[§27.5i]{RAEGramatica}, this use is motivated by \textit{aún} as a scalar additive (\appref{appendixSpanishAunScalarAdditive}) as well as by the fact that gerunds in peripheral position of the clause by themselves often allow for a concessive reading (cf. \cite[§27.5g]{RAEGramatica}).
\end{itemize}
\begin{exe}
	\ex\label{exAppendixSpanishAunGerundConcessive1}
	\gll Lo que más impresion-a de esas página-s es que, \textbf{aun} \textbf{trat}-\textbf{ándo}-\textbf{se} de una declaración de amor, la palabra amor no aparec-e nunca.\\
		3\textsc{sg}.\textsc{n} \textsc{subord} more/most impress-3\textsc{sg} of \textsc{dem}.\textsc{pl}.\textsc{f} page(\textsc{f})-\textsc{pl} \textsc{cop}.3\textsc{sg} \textsc{subord} still constitute-\textsc{ptcp}-\textsc{refl}.3 of \textsc{indef}.\textsc{sg}.\textsc{f} declaration(\textsc{f}) of love \textsc{def}.\textsc{sg}.\textsc{f} word love \textsc{neg} appear-3\textsc{sg} never\\
	\glt \lq What’s most impressive about those pages is that, despite being a declaration of love, the word love never appears.' (Martínez, \textit{Santa Evita}, cited in \cite[§27.5i]{RAEGramatica}, glosses added)
		\ex\label{exAppendixSpanishAunGerundConcessive2}
	\gll El doctor Amoedo no hab-ía gan-ado un real en su vida, \textbf{aun} \textbf{siendo} un gran médico.\\
	\textsc{def}.\textsc{sg}.\textsc{m} doctor(\textsc{m}) A. \textsc{neg} have-\textsc{pst}.\textsc{ipfv}.3\textsc{sg} win-\textsc{ptcp} \textsc{indef}.\textsc{sg}.\textsc{m} Real in \textsc{poss}.3 life still \textsc{cop}.\textsc{ptcp} \textsc{indef}.\textsc{sg}.\textsc{m} great doctor(\textsc{m})\\
	\glt \lq Doctor Amoedo hadn’t earned a single real in his life, despite being a great physician.' (Torrente Ballester, \textit{La saga/fuga de J.B.} cited in \cite[§27.5i]{RAEGramatica}, glosses added)
	\ex\label{exAppendixSpanishAunGerundConcessive3}
	\gll \textbf{Aun} \textbf{pag}-\textbf{ando} una fortuna no te la vend-er-án.\\
	still pay-\textsc{ptcp} \textsc{indef}.\textsc{sg}.\textsc{f} fortune(\textsc{f}) \textsc{neg} 2\textsc{sg}.\textsc{obj} 3\textsc{sg}.\textsc{acc}.\textsc{f} sell-\textsc{fut}-3\textsc{pl}\\
	\glt \lq Even if you paid them a fortune, they won't sell it to you.\rq{ }(\cite[40]{FerrariEtAl2011}, glosses added)
		\ex\label{exAppendixSpanishAunGerundConcessive4}
	\gll Edmundo admit-ía para su-s adentro-s que, \textbf{aun} \textbf{cont}-\textbf{ando} con lo tópico y lo cursi, la tarde hab-ía sido agradable.\\
	E. admit-\textsc{pst}.\textsc{ipfv}.3\textsc{sg} for \textsc{poss}.3\textsc{sg}-\textsc{pl} inside-\textsc{pl} \textsc{subord} still count-\textsc{ptcp} with \textsc{def}.\textsc{sg}.\textsc{n} stereotypical and \textsc{def}.\textsc{sg}.\textsc{n} corny \textsc{def}.\textsc{sg}.\textsc{f} afternoon(\textsc{f}) have-\textsc{pst}.\textsc{ipfv}.3\textsc{sg} \textsc{cop}.\textsc{ptcp} pleasent\\
	\glt \lq Edmundo admitted to himself that, even taking into consideration the commonplaces and corniness, the afternoon had been pleasant.'
	(Gopegui, \textit{Lo real} cited in \cite[§27.5i]{RAEGramatica}, glosses added)
\end{exe}

\paragraph{Concessive protases (ii): \textit{aun cuando}}
\label{appendixSpanishAunCuando}
\begin{itemize}
	\item \textcite{Elvira2005}, \textcite{FlamencoGarcia1999}, \citeauthor{RAEGramatica} (\citeyear[§47.12g]{RAEGramatica}, \citeyear[s.v. \textit{aun}]{RAEDictionary}) and \textcite{Trujillo1990}.
	\item  Form: in combination with a subordinate clause introduced by \textit{cuando} \lq when' (a collocation that can also retain a strictly temporal meaning).
	\item This construction governs the same systematic alternations in mood as \textit{aunque} (\appref{appendixSpanishAunque}). Thus, the subjunctive mood can result in protasis of a scalar concessive conditional, as in (\ref{exAppendixSpanishAunAunCuando3}, \ref{exAppendixSpanishAunAunCuando4}). According to \textcite[§47.12.g]{RAEGramatica}, this is the more frequent case.
	\item This use belongs to a high/formal register.
	\item This use is transparently derived from \textit{aún} as a scalar additive operator (\appref{appendixSpanishAunScalarAdditive}), i.e. \lq even when/if'.
	\item According to \textcite{Elvira2005}, this construction started out as marking scalar concessive conditionals.
\end{itemize}
\pagebreak
\begin{exe}
	\ex\label{exAppendixSpanishAunAunCuando1}
	\gll  \textbf{Aun} \textbf{cuando} las orientacion-es polític-a-s de promoción d-el diálogo social se origin-an en 1990, lo que se puede observ-ar a-l efectu-ar un diagnóstico preliminar de su deven-ir es que éste ha sido un proceso marc-ado por la escas-a sistematicidad de los esfuerzo-s.\\
	still when \textsc{def}.\textsc{pl}.\textsc{f} orientation(\textsc{f})-\textsc{pl} political-\textsc{f}-\textsc{sg} of promotion of-\textsc{def}.\textsc{sg}.\textsc{m} dialogue(\textsc{m}) social \textsc{refl}.3 originate-3\textsc{pl} in 1999 3\textsc{sg}.\textsc{n} \textsc{subord} \textsc{refl}.3 can.3\textsc{sg} observe-\textsc{inf} at-\textsc{def}.\textsc{sg}.\textsc{m} carry\_out-\textsc{inf} \textsc{indef}.\textsc{sg}.\textsc{m} diagnostic(\textsc{m}) preliminary of \textsc{poss}.3 become-\textsc{inf} \textsc{cop}.3\textsc{sg} \textsc{subord} \textsc{prox}.\textsc{sg}.\textsc{m} have.3\textsc{sg} \textsc{cop}.\textsc{ptcp} \textsc{indef}.\textsc{sg}.\textsc{m} process(\textsc{m}) mark-\textsc{ptcp}.\textsc{sg}.\textsc{m} for \textsc{def}.\textsc{sg}.\textsc{f} scarce-\textsc{f} systematicity(\textsc{f}) of \textsc{def}.\textsc{m}.\textsc{pl} effort-\textsc{pl}\\
	\glt \lq Even though the political ideas of promoting social dialogue go back to 1990, what can be observed when carrying out a preliminary study of their development is that it has been a process characterised by a lack of systematic efforts.' (CORPES XXI, glosses added)

	\ex\label{exAppendixSpanishAunAunCuando2}
	 \gll \textbf{Aun} \textbf{cuando} me lo recomend-aron, no le-í el prospecto.\\
	still when 1\textsc{sg}.\textsc{obj} 3\textsc{sg}.\textsc{acc}.\textsc{m} recommend-\textsc{pst}.\textsc{pfv}.3\textsc{sg} \textsc{neg} read-\textsc{pst}.\textsc{pfv}.1\textsc{sg} \textsc{def}.\textsc{sg}.\textsc{m} leaflet(\textsc{m})\\
	\glt \lq Although they recommended it to me, I didn't read the leaflet.' (\cite[§47.12.g]{RAEGramatica}, glosses added)

	\ex\label{exAppendixSpanishAunAunCuando3}
	\gll Va a seg-uir adelante, \textbf{aun} \textbf{cuando} ten-ga que lleg-ar él solo con tod-o-s los tripulant-es colg-ad-o-s de los palo-s.\\
	go.3\textsc{sg} to continue-\textsc{inf} forward still when have-\textsc{sbjv}.3\textsc{sg} \textsc{subord} arrive-\textsc{inf} 3\textsc{sg}.\textsc{m} only with all-\textsc{m}-\textsc{pl} \textsc{def}.\textsc{pl}.\textsc{m} sailor-\textsc{pl} hang-\textsc{ptcp}-\textsc{m}-\textsc{pl} of \textsc{def}.\textsc{pl}.\textsc{m} mast-\textsc{pl}\\
\glt \lq He'll move forward, even if he has to arrive alone, with all the sailors hanged on the masts.'
 (Roa Bastos, \textit{Vigilia del Almirante}, cited in \cite[§47.12g]{RAEGramatica}, glosses added)
 
	\ex\label{exAppendixSpanishAunAunCuando4}
	\gll Por otr-a parte, el incremento de nuevo en los últim-o-s año-s, \textbf{aun} \textbf{cuando} no sea a-l nivel de los año-s 90, oblig-a a reflexion-ar y produc-ir cambio-s en las estrategia-s de intervención actual-es.\\
	for other-\textsc{f} side(\textsc{f}) \textsc{def}.\textsc{sg}.\textsc{m} increase(\textsc{m}) of new in \textsc{def}.\textsc{pl}.\textsc{m} last-\textsc{m}-\textsc{pl} year(\textsc{m})-\textsc{pl} still when \textsc{neg} \textsc{cop}.\textsc{sbjv}.3\textsc{sg} to-\textsc{def}.\textsc{sg}.\textsc{m} level(\textsc{m}) of \textsc{def}.\textsc{pl}.\textsc{m} year-\textsc{pl} 90 oblige-3\textsc{sg} to reflect-\textsc{inf} and produce-\textsc{inf} change-\textsc{pl} in \textsc{def}.\textsc{pl}.\textsc{f} strategy(\textsc{f})-\textsc{pl} of intervention current-\textsc{pl}\\
	\glt \lq On the other hand, the returning increase during the last years, even if not to the level of the 90s, obliges us to reflect and make changes to the current intervention strategies.' (CORPES XXI, glosses added)
\end{exe}

\paragraph{Concessive protases (iii): \textit{aunque}}
\label{appendixSpanishAunque}
\begin{itemize}
	\item \textcite{Cid1999}, \textcite{Elvira2005}, \textcite{FlamencoGarcia1999}, \textcite{PerezSaldanyaVincent2014}, \citeauthor{RAEGramatica} (\citeyear[§§47.2,  47.12–47.13]{RAEGramatica}; \citeyear[s.v. \textit{aunque}]{RAEDictionary}) and \textcite{Trujillo1990}.
	\item Form: in collocation with the subordinator \textit{que} > \textit{aunque}.
	\item With indicative verbs, the \textit{aunque} clause serves as the protasis of an ordinary concessive construction (\ref{exAppendixSpanishAunqueConcessive1}–\ref{exAppendixSpanishAunqueConcessive3}). With verbs in the subjunctive mood, two interpretations are possible. First, the clause can serve as the protasis of a scalar concessive conditional, as in (\ref{exAppendixSpanishAunqueScalarConcessiveConditional2}– \ref{exAppendixSpanishAunqueScalarConcessiveConditional3}) and the first interpretation of (\ref{exAppendixSpanishAunqueScalarConcessiveConditional1}). Alternatively, the clause can be interpreted as the protasis of an ordinary concessive, with the proposition contained in it constituting given information that is commented upon; this is the second interpretation of (\ref{exAppendixSpanishAunqueScalarConcessiveConditional1}).
	\item As pointed out repeatedly in the literature, this function can be traced back to \textit{aún} as a scalar additive (\appref{appendixSpanishAunScalarAdditive}), with \textit{que} introducing a clausal complement. As \textcite{Elvira2005} and \textcite{PerezSaldanyaVincent2014} show, its function in scalar concessive conditionals is older than its non-conditional counterpart.
\end{itemize}

\begin{exe}
	\ex\label{exAppendixSpanishAunqueConcessive1}
	\gll \textbf{Aun}-\textbf{que} llueve salg-o.\\
	still-\textsc{subord} rain.3\textsc{sg} go\_out-1\textsc{sg}\\
	\glt \lq Although it is raining, I'm going out.' \parencite[589]{HaspelmathKoenig1998}

	\ex\label{exAppendixSpanishAunqueConcessive2}
	\gll \textbf{Aun}-\textbf{que} viv-e en esta ciudad {desde hace} treinta año-s, mantiene el mism-o apartamento que alquil-ó a-l lleg-ar.\\
	still-\textsc{subord} live-3\textsc{sg} in \textsc{prox}.\textsc{sg}.\textsc{f} city(\textsc{f}) since thirty year-\textsc{pl} keep.3\textsc{sg} \textsc{def}.\textsc{sg}.\textsc{m} same-\textsc{m} apartment(\textsc{m}) \textsc{subord} rent-3\textsc{sg} at-\textsc{def}.\textsc{sg}.\textsc{m} arrive-\textsc{inf}\\
	\glt \lq Although s/he has been living in this city for thirty years, s/he still lives in the same apartment s/he rented when s/he arrived.' (\cite[§47.12a]{RAEGramatica}, glosses added)

	\ex\label{exAppendixSpanishAunqueConcessive3}
	\gll \textbf{Aun}-\textbf{que} est-aba muy cansad-a por el viaje, impart-ió una conferencia magnífic-a.\\
	still-\textsc{subord} \textsc{cop}-\textsc{pst}.\textsc{ipfv}.3\textsc{sg} very tired-\textsc{f} for \textsc{def}.\textsc{sg}.\textsc{m} trip(\textsc{m}) confer-\textsc{pst}.\textsc{pfv}.3\textsc{sg} \textsc{indef}.\textsc{sg}.\textsc{f} talk(\textsc{f}) magnificent-\textsc{f}\\
	\glt \lq Although she was very tired from the journey, she gave a magnificient talk.' (\cite[§47.12c]{RAEGramatica}, glosses added)

	\ex\label{exAppendixSpanishAunqueScalarConcessiveConditional1}
	\gll \textbf{Aun}-\textbf{que} llueva sal-go.\\
	still-\textsc{subord} rain.\textsc{sbjv}.3\textsc{sg} go\_out-1\textsc{sg}\\
	\glt \phantom{i}i. \lq Even if it is raining, I am going out.' (concessive conditional)\\
	ii. \lq Even though it's raining [given information], I am going out.' (ordinary concessive) (personal knowledge)
	
	\ex\label{exAppendixSpanishAunqueScalarConcessiveConditional2}
	\gll \textbf{Aun}-\textbf{que} te qued-es sin dorm-ir, h-as de prepar-ar bien este examen.\\
still-\textsc{subord} \textsc{refl}.2\textsc{sg} remain-\textsc{sbjv}.2\textsc{sg} without sleep-\textsc{inf}	 have-2\textsc{sg} of prepare-\textsc{inf} well \textsc{prox}.\textsc{sg}.\textsc{m} exam(\textsc{m})\\
\glt \lq Even if [it means that] you don't get any sleep, you have to prepare well for this exam.' \cite[§47.12e]{RAEGramatica}, glosses added)
	
	\ex\label{exAppendixSpanishAunqueScalarConcessiveConditional3}
	\gll Lo invit-ar-é, \textbf{aun}-\textbf{que} solo sea por cortesía.\\
	3\textsc{sg}.\textsc{acc}.\textsc{m} invite-\textsc{fut}-1\textsc{sg} still-\textsc{subord} only \textsc{cop}.\textsc{sbjv}.3\textsc{sg} for courtesy\\
	\glt \lq I'll invite him, even if it's only out of courtesy.' (\cite[§47.12p]{RAEGramatica}, glosses added)
\end{exe}

\paragraph{Concessive protases (iv): \textit{aun si}}
\label{appendixSpanishConcessiveAunSi}
\begin{itemize}
	\item \textcite{Elvira2005} and \textcite[§47.2o]{RAEGramatica}.
	\item Form: in collocation with \textit{si} \lq if'.
	\item This is another transparent use involving the scalar additive function of \textit{aún} (\appref{appendixSpanishAunScalarAdditive}), i.e. \lq even if'. Judging from the description, this collocation always introduces scalar concessive conditionals; the latter are, according to \textcite{Elvira2005}, also what is found in the earliest attestations. Consequently, past tense predicates (which also feature in counterfactual conditionals) take the subjunctive mood, as in (\ref{exAppendixSpanishAunSi2}). Present tense predicates, however, surface in the indicative mood, due to collocational restrictions of \textit{si} \lq if\rq{ }(\ref{exAppendixSpanishAunSi2}).

\end{itemize}
\begin{exe}
	\ex\label{exAppendixSpanishAunSi1}
	\gll \textbf{Aun} \textbf{si} no me otorg-an el crédito, ampliar-é la casa.\\
	still if \textsc{neg} 1\textsc{sg}.\textsc{obj} grant-3\textsc{pl} \textsc{def}.\textsc{sg}.\textsc{m} credit(\textsc{m}) enlarge.\textsc{fut}-1\textsc{sg} \textsc{def}.\textsc{sg}.\textsc{f} house(\textsc{f})\\
	\glt \lq Even if they don't grant me the credit, I will enlarge the house.' (\cite[§47.2o]{RAEGramatica}, glosses added)
	\ex\label{exAppendixSpanishAunSi2}
	 \gll  \textbf{Aun} \textbf{si} result-ara cierto que Tadeo fue un fanático de la masturbación … sería una arbitrariedad hac-er analogía-s entre su vida y su obra a la luz de un mer-o accidente biográfic-o.\\
	 still if result-\textsc{pst}.\textsc{sbjv}.3\textsc{sg} true \textsc{subord} T. \textsc{cop}.\textsc{pst}.\textsc{pfv}.3\textsc{sg} \textsc{indef}.\textsc{sg}.\textsc{m} fan(\textsc{m}) of \textsc{def}.\textsc{sg}.\textsc{f} masturbation(\textsc{f}) {} \textsc{cop}.\textsc{cond}.3\textsc{sg} \textsc{indef}.\textsc{sg}.\textsc{f} arbitrariness(\textsc{f}) make-\textsc{inf} analogy-\textsc{pl} between \textsc{poss}.3 life and \textsc{poss}.3 work at \textsc{def}.\textsc{sg}.\textsc{f} light of \textsc{indef}.\textsc{sg}.\textsc{m} mere-\textsc{m} accident(\textsc{m}) biographic-\textsc{m}\\	 
	 \glt \lq Even if it turned out to be that Tadeo was a fan of masturbation … it would be arbitrary to draw analogies between his life and his works based on a mere biographical coincidence.\rq{ }(CORPES XXI, glosses added)
\end{exe}

\paragraph{Universal concessive conditional protases: \textit{aunque más}}\label{appendixSpanishAunAunqueMas}
\begin{itemize}
	\item \textcite[s.v. \textit{aunque}]{RAEDictionary}.
	\item Form: in collocation with the subordinator \textit{que} and \textit{más} \lq more, most'.
	\item This use is clearly derived from \textit{aunque} governing the protases of concessives and concessive conditionals (\appref{appendixSpanishAunque}): \lq even if it is/were the most …\rq{ }> \lq no matter how much\rq{}. What is at play here is that the conditional relationship is asserted to hold under the most adverse circumstance, thus (all things equal) across the entire set of conceivable circumstances.

\end{itemize}
\begin{exe}
	\ex
	\gll Pero \textbf{aun}-\textbf{que} \textbf{más} tend-imos la vista, ni poblado, ni persona, ni senda, ni camino descubr-imos.\\
	but still-\textsc{subord} more extend-\textsc{pst}.\textsc{pfv}.1\textsc{pl} \textsc{def}.\textsc{sg}.\textsc{f} view(\textsc{f}) nor settlement nor persona nor path nor way discover-\textsc{pst}.\textsc{pfv}.1\textsc{sg}\\
	\glt \lq But no matter how far we looked, we could find neither a settlement, nor a person, path, or way.' (\cite[s.v. \textit{aunque}]{RAEDictionary}, glosses added)
	
	\ex
	\gll A vec-es lo mejor es alej-ar-se \textbf{aun}-\textbf{que} \textbf{más} te duela.\\
	at time-\textsc{pl} \textsc{def}.\textsc{sg}.\textsc{n} best \textsc{cop}.3\textsc{sg} withdraw-\textsc{inf}-\textsc{refl}.3 still-\textsc{subord} more/most \textsc{obj}.2\textsc{sg} hurt.\textsc{sbjv}.3\textsc{sg}\\
	\glt \lq Sometimes the best thing to do is to distance yourself, no matter how much it might hurt.\rq{ }(found online, glosses added)\footnote{\url{https://twitter.com/Xavi30856894/status/1137061141375606784} (10 November, 2022).}
\end{exe}

\paragraph{Concessive apodoses (i)}
\label{appendixSpanishAunConcessiveConsequent}
\begin{itemize}
	\item \textcite{Cid1999}, \textcite{OlivaresSorpena2001} and \textcite[s.v. \textit{aun}]{RAEDictionary}.
	\item \textit{Aún} can be used to mark a concessive apodosis.
\end{itemize}
\begin{exe}
	\ex 
	\gll Tiene cuanto quiere, y, \textbf{aún}, se quej-a.\\
	have.3\textsc{sg} how\_much want.3\textsc{sg} and still \textsc{refl} complain-3\textsc{sg}\\
	\glt \lq S/he has all s/he wants, and still s/he complains.' (\cite[103]{Cid1999}, glosses added)

	\ex
	\gll Era quien más espacio ten-ía y \textbf{aún} protest-ó.\\
	\textsc{cop}.\textsc{pst}.\textsc{ipfv}.3\textsc{sg} who most space have-\textsc{pst}.\textsc{ipfv}.3\textsc{sg} and still protest-\textsc{pst}.\textsc{pfv}.3\textsc{sg}\\
	\glt \lq S/he was the one who had the most space, and yet s/he complained.' (\cite[s.v. \textit{aun}]{RAEDictionary}, glosses added)
\end{exe}

\paragraph{Concessive apodoses (ii): \textit{aun así}}
\label{appendixSpanishAunConcessiveAunAsi}
\begin{itemize}
	\item \textcite[s.v. \textit{aun así}]{DPDE}, \textcite[55]{FuentesRodriguez2018} and \textcite[47.16q]{RAEGramatica}.
	\item Form: in collocation with \textit{así} \lq so'. \textcite[s.v. \textit{aun así}]{DPDE} point out the prosodic separation from the rest of the clause, indicated by a comma in (\ref{exAppendixSpaunAunConcessiveAunAsi2}).
	\item This collocation is clearly motivated by \textit{aún} as a scalar additive (\appref{appendixSpanishAunScalarAdditive}), transparently yielding \lq even so.'
\end{itemize}
\largerpage
\begin{exe}
	\ex\label{exAppendixSpaunAunConcessiveAunAsi1}
	\gll Los muchacho-s se mir-aron entre ellos, los not-é escasamente atra-íd-o-s por el ofrecimiento, \textbf{aun} \textbf{así} se levant-aron de su lugar y se acerc-aron hacia mí.\\
	\textsc{def}.\textsc{pl}.\textsc{m} boy(\textsc{m})-\textsc{pl} \textsc{refl}.3 look-\textsc{pst}.\textsc{pfv}.3\textsc{pl} between 3\textsc{pl}.\textsc{m} 3\textsc{pl}.\textsc{acc}.\textsc{m} note-\textsc{pst}.\textsc{pfv}.1\textsc{sg} hardly atract-\textsc{ptcp}-\textsc{m}-\textsc{sg} for \textsc{def}.\textsc{sg}.\textsc{m} offering(\textsc{m}) still so \textsc{refl}-3 get\_up-\textsc{pst}.\textsc{pfv}.3\textsc{pl} of \textsc{poss}.3 place and \textsc{refl}.3 approach-\textsc{pst}.\textsc{pfv}.3\textsc{pl} towards 1\textsc{sg}\\
	\glt \lq The boys looked at each other, I saw them being hardly attracted by the offering, even so, they got up and moved closer towards me.'
	\\(CORPES XXI, glosses added)
	
	\ex\label{exAppendixSpaunAunConcessiveAunAsi2}
	\gll El fácil acceso a este recurso tecnológic-o facili-ta su uso en las práctica-s pedagógic-a-s. \textbf{Aun} \textbf{así}, no es suficiente.\\
	\textsc{def}.\textsc{sg}.\textsc{m} easy access(\textsc{m}) to \textsc{prox}.\textsc{sg}.\textsc{m} resource(\textsc{m}) tecnological-\textsc{m} facilitate-3\textsc{sg} \textsc{poss}.3 use in \textsc{def}.\textsc{pl}.\textsc{f} practice(\textsc{f})-\textsc{pl} pedagogical-\textsc{f}-\textsc{pl} still so \textsc{neg} \textsc{cop}.3\textsc{sg} sufficient\\
	\glt \lq Easy access to this technological resource facilitates its use in pedagogical practices. Nonetheless, it is not sufficient.'
	\\(CORPES XXI, glosses added)
\end{exe}

\paragraph{Concessive interjection: \textit{aun así}}
\label{appendixSpanishAunAsinterjection}
\begin{itemize}
	\item The collocation \lq aun así' \lq still so' > \lq even so' (\appref{appendixSpanishAunConcessiveAunAsi}) is attested as a holophrase. Doubtlessly, this is facilitated by the collocation's initial position and prosodic separation.
\end{itemize}
\largerpage
\begin{exe}
	\ex
	\begin{xlist}
		\exi{A:} \textit{Si fuese tan amable de ayudarme a ubicar.}\\
		\lq If you could be so friendly and help me know where I am.'
		\exi{B:}
		\textit{¿El lugar? No le asigne ninguna importancia.}\\
		\lq The location? Don't give it any importance.'
		\exi{A:}\gll \textbf{Aun} \textbf{así}.\\
		still so\\
		\glt \lq Still!' (CORPES XXI, glosses added)
	\end{xlist}
	
	\ex —\textit{Oiga, ¿el aprendiz no es un poco joven para este oficio?} \textit{Las verdades de la vida no conocen edad, hermana —ofreció Fermín.} \textit{La monja me sonrió dulcemente, asintiendo. No había desconfianza en aquella mirada, sólo tristeza.}\\
	\lq Listen, isn't the apprentice a little young for this kind of work? [The nun asked] \lq\lq The truths of life know no age, sister" said Fermín. The nun nodded and smiled at me sweetly. There was no suspicion in that look, only sadness.'
	\exi{}\gll —\textbf{Aun} \textbf{así}. —murmur-ó.\\
	\phantom{—}still so \phantom{—}mumble-\textsc{pst}.\textsc{pfv}.3\textsc{sg}\\
	\glt \lq {\lq\lq}Even so", she mumbled.' (CORPES XXI, glosses added)
\end{exe}

\subsection{todavía}
\label{appendixSpanishTodavia}
\subsubsection{General information}
\begin{itemize}
	\item Wordhood: independent grammatical word, invariable.
	\item Etymology: from a late Latin phrase \lq all (the) way', via \lq one way or another' and \lq always'.
\end{itemize}


\subsubsection{As a \lq{}still\rq{ }expression}
\label{appendixSpanishTodaviaStill}
\begin{itemize}
	\item \Textcite{vanderAuwera1998}, \textcite{Bosque2016}, \textcite{EderlyCurco2016},	
		\citeauthor{Garrido1991} (\citeyear{Garrido1991}, \citeyear{Garrido1992}, \citeyear{Garrido1993}), \textcite{Morera1999}, \textcite[§§24.4.m, 30.8f–m]{RAEGramatica} and \textcite{Trujillo1990}, among many others.
	\item Specialisation: several descriptions (e.g. \cite{Bosque2016}; \cite{EderlyCurco2016}; \cite{Garrido1991}; \cite{RAEGramatica}) explicitly address the two components of my definition.
	\item Polarity sensitivity: inner negation yields \textsc{not yet}.
	\item Pragmaticity: compatible with both scenarios.
	\item Further note: \textit{todavía} can be used as an elliptical utterance (\ref{exAppendixSpanishTodavia3}).
\end{itemize}
\begin{exe}
	\ex
	\gll \textbf{Todavía} record-aba a Carax bes-ando a Penélope Aldaya en el caserón de la avenida d-el Tibidabo.\\
	still remember-\textsc{pst}.\textsc{ipfv}.3\textsc{sg} \textsc{acc} C. kiss-\textsc{ptcp} \textsc{acc} P. A. in \textsc{def}.\textsc{sg}.\textsc{m} 	shanty(\textsc{m}) of \textsc{def}.\textsc{sg}.\textsc{f} avenue(\textsc{f}) of-\textsc{def}.\textsc{sg}.\textsc{m} T.\\
	\glt \lq He still remembers Caraxos kissing Penélope Aldaya in the shanty at Tibidabo Avenue.' (Ruíz Zafón, \textit{La sombra del viento}, cited in \cite[§30.8.f]{RAEGramatica}, glosses added)

\ex
\gll ¿Y ella te quiere \textbf{todavía}? —pregunt-ó con la picardía de un juez de instrucción.\\
\phantom{¿}and 3\textsc{sg}.\textsc{f} 2\textsc{sg}.\textsc{obj} love.3\textsc{sg} still  \phantom{—}ask-\textsc{pst}.\textsc{pfv}.3\textsc{sg} with \textsc{def}.\textsc{sg}.\textsc{f} malice(\textsc{f}) of \textsc{indef}.\textsc{sg}.\textsc{m} judge(\textsc{m}) of instruction\\
\glt \lq\lq And she still loves you?" s/he asked with roughishness of an investigative judge.' (Peréz Galdos, \textit{Fortunata y Jacinta}, cited in \cite[§30.8.f]{RAEGramatica}, glosses added)

	\ex\label{exAppendixSpanishTodavia3}
	\begin{xlist}
		\exi{A:}\textit{Este será el regalo de Juan para cuando se case con María.}\\
		\lq This will be Juan’s present for his marriage with Maria.'
		\exi{B:} \textit{Pero si Juan está enamorado de Ana.}\\
		\glt \lq But he is in love with Ana.'
		\exi{A:} \gll \textbf{Todavía}.\\
		still\\
		\glt \lq [For now he] still [is].' (\cite[10]{EderlyCurco2016})
	\end{xlist}
\end{exe}

\subsubsection{Uses related to other phasal polarity concepts}
\paragraph{Interrogative \lq yet\rq}
\label{appendixSpanishTodaviaInterrogativeYet}
\begin{itemize}
	\item \textcite[§30.8m]{RAEGramatica}.
	\item This function only obtains in complement clauses of attitude verbs with an inherent negative component.	 Ex. (\ref{exAppendixSpanishTodaviaInterrogative1}) illustrates this for \textit{dudar} \lq doubt'.
	\item Ex. (\ref{exAppendixSpanishTodaviaInterrogative2}) illustrates the expression of \textsc{not yet} in the context of negative raising.
\end{itemize}

\begin{exe}

	\ex\label{exAppendixSpanishTodaviaInterrogative1}
	\gll Dud-o que h-aya nac-ido \textbf{todavía} el que ten-ga los suficiente-s cojon-es para hac-er-lo.\\
	doubt-1\textsc{sg} \textsc{comp} have-\textsc{sbjv}.3\textsc{sg} be\_born-\textsc{ptcp} still \textsc{def}.\textsc{sg}.\textsc{m} \textsc{comp} have-\textsc{sbjv}.3\textsc{sg} \textsc{def}.\textsc{pl}.\textsc{m} enough-\textsc{pl} testicle(\textsc{m})-\textsc{pl} for do-\textsc{inf}-3\textsc{sg}.\textsc{acc}.\textsc{m}\\
	\glt \lq I doubt that the person with big enough balls to do so has been born yet.' (Montaño, Andanzas, cited in \cite[§80.8m]{RAEGramatica}, glosses added).

	\ex\label{exAppendixSpanishTodaviaInterrogative2}
	\gll No ten-go noticias de que lo ha-yan logr-ado \textbf{todavía}.\\
	\textsc{neg} have-1\textsc{sg} news of \textsc{subord} 3\textsc{sg}.\textsc{acc}.\textsc{m} have-\textsc{sbjv}.3\textsc{sg} achieve-\textsc{ptcp} still\\
	\glt \lq I don't have any news that they have achieved it yet.\rq{ }
	\\(\cite[30.8m]{RAEGramatica}; glossed)
\end{exe}

\subsubsection{Uses on the fringes of \lq{}still\rq{}}
\paragraph{Scalar contexts}
\label{appendixSpanishTodaviaScalar}
\begin{itemize}
	\item \citeauthor{Garrido1992} (\citeyear{Garrido1992}, \citeyear{Garrido1993}).
 \item \textit{Todavía} is compatible with scalar contexts, both in the form of decreases  (\ref{exAppendixSpanishTodaviaScalar1}, \ref{exAppendixSpanishTodaviaScalar2}) and increases (\ref{exAppendixSpanishTodaviaScalar3}– \ref{exAppendixSpanishTodaviaScalar5}). To disambiguate, \textit{todavía} can be combined with expressions like \textit{quedar} \lq be left\rq{} or \textit{sólo} \lq only\rq{}, as in (\ref{exAppendixSpanishTodaviaScalar2}, \ref{exAppendixSpanishTodaviaScalar3}). Note, however, that a restrictive marker is not categorically necessary in contexts of a limited increase, as evidenced by examples like (\ref{exAppendixSpanishTodaviaScalar4}, \ref{exAppendixSpanishTodaviaScalar5}). 
\end{itemize}

\begin{exe}
	\ex\label{exAppendixSpanishTodaviaScalar1}

\gll La Armada ten-ía 337 vehículo-s y se desprendió de tan solo nueve. \textbf{Todavía} tiene una flota de 328.\\
\textsc{def}.\textsc{sg}.\textsc{f} navy(\textsc{f}) have-\textsc{pst}.\textsc{ipfv}.3\textsc{sg} 337 vehicle-\textsc{pl} and \textsc{refl}.3 get\_rid\_off.\textsc{pst}.\textsc{pfv}.3\textsc{sg} of so only nine still have.3\textsc{sg} \textsc{indef}.\textsc{sg}.\textsc{f} fleet(\textsc{f}) of 328\\
\glt \lq The navy had 337 vehicles and got rid of only nine. It still has a fleet of 328.\rq{ }(CORPES XXI, glosses added)
	
	\ex\label{exAppendixSpanishTodaviaScalar2}
	\gll No me preocup-a, porque \textbf{todavía} me qued-an dos año-s de contrato.\\
	\textsc{neg} 1\textsc{sg}.\textsc{obj} worry-3\textsc{sg} because still 1\textsc{sg}.\textsc{obj} remain-3\textsc{pl} two year-\textsc{pl} of contract\\
	\glt \lq Iʼm not worried about it, because I still have two years to my contract.ʼ (CORPES XXI, glosses added)
	
	\ex\label{exAppendixSpanishTodaviaScalar3}
	\gll Pedro \textbf{todavía} tiene solo cien libro-s.\\
	P. still have.3\textsc{sg} only hundred book-\textsc{pl}\\
	\glt \lq Pedro still only has a hundred books.' (\cite[383]{Garrido1992}, glosses added)

	\ex\label{exAppendixSpanishTodaviaScalar4}
	\textit{En las áreas francófonas, anglófonasy territorios ligados a los Paises Bajos del Caribe es Trinidad y Tobago uno de los países más afectados,}\\
	\lq Within the francophone, anglophone and Dutch territories of the Caribbean, Trinidad and Tobago is one of the countries hit the hardest [by the Covid19 pandemic]\rq{}
			
	\exi{}\gll pero con \textbf{todavía} unos modest-o-s 51 caso-s y sin muerte-s.\\
	 but with still \textsc{indef}.\textsc{pl}.\textsc{m} modest-\textsc{m}-\textsc{pl} 51 case(\textsc{m})-\textsc{pl} and without death-\textsc{pl}\\
	\glt \lq but with still only some modest 51 cases and no casualties.\rq{ }(found online, glosses added)\footnote{\url{https://es-us.noticias.yahoo.com/estrictas-medidas-puerto-rico-covid-182536565.html} (23 January, 2023).}
	
	\ex\label{exAppendixSpanishTodaviaScalar5}
	\gll No hay que d-ar-se tant-a prisa. \textbf{Todavía} son las cuatro.\\
	\textsc{neg} \textsc{exist} \textsc{subord} give-\textsc{inf}-\textsc{refl}.3 that\_much-\textsc{f} rush(\textsc{f}) still \textsc{cop}.3\textsc{pl} \textsc{def}.\textsc{sg}.\textsc{f} four\\
	\glt \lq No need to rush. It's only four o'clock.\rq{ }(\cite[220]{Bosque2016}, glosses added)
\end{exe}
\largerpage[-1]\pagebreak
\subsubsection{Broadly adverbial temporal-aspectual functions}
\paragraph{Prospective \lq eventually\rq{}}\label{appendixSpanishTodaviaEventually}
\begin{itemize}
	\item \textcite{Bosque2016}, \citeauthor{Garrido1991} (\citeyear{Garrido1991}, \citeyear{Garrido1992}, \citeyear{Garrido1993}), \textcite{UrdialesCampos1973} and \textcite{Martinez1996}.
\end{itemize}

\begin{exe}
	\ex\label{appendixSpanishTodaviaEventually1}
	\gll ¿A que \textbf{todavía} termin-o?\\
	\phantom{¿}to \textsc{subord} still finish-1\textsc{sg}\\
	\glt \lq I bet you that I will finish [yet].\rq{ }(\cite[382 fn 30]{Garrido1992}, glosses added)
			
	\ex Context: There's an unconscious girl with bloodstains on her face and neck.\label{appendixSpanishTodaviaEventually2}\\
	\gll Tienes que llam-ar a la doctor-a …	 no puedes carg-ar con la responsabilidad de que se muera. \textbf{Todavía} dir-án que eres cómplice.\\
	have.2\textsc{sg} \textsc{subord} call-\textsc{inf} \textsc{acc} \textsc{def}.\textsc{sg}.\textsc{f} doctor-\textsc{f} {}
	\textsc{neg} can.2\textsc{sg} load-\textsc{inf} with \textsc{def}.\textsc{sg}.\textsc{f} responsibility(\textsc{f}) of \textsc{subord} \textsc{refl}.3 die.\textsc{sbjv}.3\textsc{sg} still say.\textsc{fut}-3\textsc{pl} \textsc{subord} \textsc{cop}.2\textsc{sg} accomplice\\
	\glt \lq You have to call the doctor … You can't take responsibility for her dying. They'll end up saying you're an accomplice.\rq{ }(CORPES XXI, glosses added)
	
		\ex\label{appendixSpanishTodaviaEventually3}
	\gll No le habl-es, que \textbf{todavía} pens-ar-á que le gust-as.\\
	\textsc{neg} 3\textsc{sg}.\textsc{dat} talk-\textsc{sbjv}.2\textsc{sg} \textsc{subord} still think-\textsc{fut}-3\textsc{sg} \textsc{subord} 3\textsc{sg}.\textsc{dat} please-2\textsc{sg}\\
	\glt \lq Don't talk to him, he'll end up thinking you like him.\rq{ }(Kristian Roncero, p.c.)

	\ex\label{appendixSpanishTodaviaEventually4}
	\gll \textbf{Todavía} vamos a gan-ar.\\
	still go.1\textsc{pl} to win-\textsc{inf}\\
	\glt \lq We're still going to win.\rq{ }(\cite[370 fn18]{Garrido1992}, glosses added)
\end{exe}

\subsubsection{Temporal connectives and frame setters}
\paragraph{Time-scalar (\lq as late as\rq{})}
\label{appendixSpanishTodaviaTimeScalar}
\begin{itemize}
	\item \textcite{Bosque2016}, \textcite{EderlyCurco2016}, \textcite{Garrido1993}, \textcite{Martinez1996} and \textcite{Trujillo1990}.
	\item In this function, \textit{todavía} combines with a time frame adverbial, giving a reading along the lines of \lq as late as'.
	\item Syntax: can form a constituent with its focus.
\end{itemize}
\begin{exe}
	\ex Context: About the Trinidad neighbourhood in Asunción. The neighbourhood has been described as the city's port of entry.
\gll {Sin embargo}, \textbf{todavía} en la primer-a mitad d-el siglo XX ir a Trinidad era para el asunceno como viaj-ar a-l interior, y ningún medio de transporte era más apropiad-o que el tren.\\
	nonetheless still in \textsc{def}.\textsc{sg}.\textsc{f} first-\textsc{f} half(\textsc{f}) of-\textsc{def}.\textsc{sg}.\textsc{m} century(\textsc{m}) 20 go.\textsc{inf} to T. \textsc{cop}.\textsc{pst}.\textsc{ipfv}.3\textsc{sg} for \textsc{def}.\textsc{sg}.\textsc{m} person\_from\_Asunción like travel-\textsc{inf} to-\textsc{def}.\textsc{sg}.\textsc{m} interior(\textsc{m}) and none.\textsc{m} medium(\textsc{m}) of transport \textsc{cop}.\textsc{pst}.\textsc{ipfv}.3\textsc{sg} more appropriate-\textsc{m} \textsc{subord} \textsc{def}.\textsc{sg}.\textsc{m} train(\textsc{m})\\
	\glt \lq Nonetheless, even during the first half of the 20th century, for a denizen of Asunción travelling to Trinidad was like a trip to the interior, and no means of transport was more appropriate than the train.' (CORPES XXI, glosses added)
	
	\ex Context: About a certain graphic novel.\\
	\gll ¡Me muero por le-er-lo! \textbf{Todavía} ayer lo v-i en la biblio y me dij-e, para la próxim-a te vienes con-migo!\\
	\textsc{refl}.1\textsc{sg} die.1\textsc{sg} for read-\textsc{inf}-3\textsc{sg}.\textsc{acc}.\textsc{m} still yesterday 3\textsc{sg}.\textsc{acc}.\textsc{m} see-\textsc{pst}.\textsc{pfv}.1\textsc{sg} in \textsc{def}.\textsc{sg}.\textsc{f} library(\textsc{f}) and \textsc{refl}.1\textsc{sg} say.\textsc{pst}.\textsc{pfv}-1\textsc{sg} for \textsc{def}.\textsc{sg}.\textsc{f} next-\textsc{f} \textsc{refl}.2\textsc{sg} come.2\textsc{sg} with-1\textsc{sg}\\
	\glt \lq I'm dying to read it! Just yesterday I saw it in the library and I said to myself \lq\lq Next time, you're coming with me!"' (found online, glosses added)\footnote{\url{https://cafelou.wordpress.com/2017/09/26/pyongyang-la-novela-grafica-de-guy-delisle-con-la-que-viajamos-a-corea-del-norte/} (29 March, 2022).}
\end{exe}	

\subsubsection{Marginality}
\label{appendixSpanishTodaviaMarginal}
\begin{itemize}
	\item \textcite{Bosque2016}, \textcite{Deloor2012}, \textcite{EderlyCurco2016}, \citeauthor{Garrido1991} (\citeyear{Garrido1991}, \citeyear{Garrido1992}, \citeyear{Garrido1993}), \textcite[§30.8ñ]{RAEGramatica} and \textcite{Trujillo1990}.
	\item \textcite{Deloor2012} points out that \textit{todavía} as a signal of marginality is infelicitous with central members of a conceptual category: \#\textit{Paris todavía es Francia} \lq Paris is still in France'.
\end{itemize}
\begin{exe}
	\ex\label{exAppendixSpanishTodaviaMarginal1}
	\gll A Juan \textbf{todavía} lo aguant-o, pero a Pedro no.\\
	\textsc{acc} J. still 3\textsc{sg}.\textsc{acc}.\textsc{m} tolerate-1\textsc{sg} but \textsc{acc} P. \textsc{neg}\\
	\glt \lq I still stand Juan, but not Pedro.\rq{ }(\cite[3]{EderlyCurco2016}, glosses added)

	\ex\label{exAppendixSpanishTodaviaMarginal2}
	Context: Commenting on a person getting seriously injured.\\
	\gll Nadie lo mand-a a meter-se en propiedad privad-a. \textbf{Todavía} la sac-ó barat-a, un par de tiro-s hubies-en hecho falta también.\\
	nobody 3\textsc{sg}.\textsc{acc}.\textsc{m} force-3\textsc{sg} to stick-\textsc{refl}.3 in property(\textsc{f}) private-\textsc{f} still 3\textsc{sg}.\textsc{acc}.\textsc{f} take\_out-\textsc{pst}.\textsc{pfv}.3\textsc{sg} cheap-\textsc{f} \textsc{indef}.\textsc{sg}.\textsc{m} couple of gunshot-\textsc{pl} have.\textsc{pst}.\textsc{sbjv}-3\textsc{pl} make.\textsc{ptcp} miss also\\
\glt \lq Nobody forces him to get onto private property. He still got off cheaply, a couple of gunshots would have been well deserved.\rq{ }(found online, glosses added)\footnote{\url{https://twitter.com/gerodutto/status/1621103017750851586} (14 February, 2023).}

	\ex\label{exAppendixSpanishTodaviaMarginal3}
	\gll Los coche-s compact-o-s \textbf{todavía} son segur-o-s en la autopista; los utilitario-s ya no lo son tanto.\\
	\textsc{def}.\textsc{pl}.\textsc{m} car(\textsc{m})-\textsc{pl} compact-\textsc{m}-\textsc{pl} still \textsc{cop}.3\textsc{pl} safe-\textsc{m}-\textsc{pl} in \textsc{def}.\textsc{sg}.\textsc{f} highway(\textsc{f}) \textsc{def}.\textsc{pl}.\textsc{m} subcompact\_car(\textsc{m})-\textsc{pl} already \textsc{neg} 3\textsc{sg}.\textsc{n} \textsc{cop}.3\textsc{sg} that\_much\\
	\glt \lq Compact cars are still safe on the highway; subcompact cars, not that much.' (\cite[221]{Bosque2016}, glosses added)

	\ex\label{exAppendixSpanishTodaviaMarginal4}
	\gll Irún \textbf{todavía} es España y Hendaya ya es Francia.\\
	I. still \textsc{cop}.3\textsc{sg} Spain and H. already \textsc{cop}.3\textsc{sg} France\\
	\glt \lq{}Irún is still in Spain and Hendaya is in France already.'
	\\(\cite[58]{Garrido1991}, glosses added)
\end{exe}

\paragraph{Marginality, \lq with acceptable limits'}
\label{appendixSpanishTodaviaAcceptableLimits}
\begin{itemize}
	\item \textcite{Bosque2016} and \textcite{Deloor2012}
	\item In this function, illustrated in (\ref{exAppendixSpanishTodaviaAcceptableLimits1}, \ref{exAppendixSpanishTodaviaAcceptableLimits2}), \textit{todavía} depicts a state-of-affairs as a borderline case of what is socially acceptable. All examples in the literature feature the juxtaposition between a factual, unacceptable state-of-affairs and a hypothetical, marginally acceptable one.
	\item As pointed out by \textcite{Bosque2016} and \textcite{Deloor2012}, this is an extension of the marginality function of \textit{todavía} (\appref{appendixSpanishTodaviaMarginal}). In all likelihood, this goes back to the elipsis of an evaluative predicate. Cases like (\ref{exAppendixSpanishTodaviaAcceptableLimits3}), where a clausal complement separates \textit{todavía} from the predicate, could have facilitated this deletion.
\end{itemize}

\begin{exe}
	\ex\label{exAppendixSpanishTodaviaAcceptableLimits1}	
	\gll En el coche, en la mesa, en la cama, \textbf{todavía}… pero tienes 5 minuto-s para sac-ar ese televisor d-el baño.\\
	in \textsc{def}.\textsc{sg}.\textsc{m} car(\textsc{m}) in \textsc{def}.\textsc{sg}.\textsc{f} table(\textsc{f}) in \textsc{def}.\textsc{sg}.\textsc{f} bed(\textsc{f}) still but have.2\textsc{sg} 5 minute-\textsc{pl} for remove-\textsc{inf} \textsc{dem}.\textsc{sg}.\textsc{m} television(\textsc{m}) of-\textsc{def}.\textsc{sg}.\textsc{m} bathroom(\textsc{m})\\
	\glt \lq In the car, on the table, in bed, [that would] still [be acceptable]… but you have five minutes to move that TV out of the bathroom.' (\cite{Deloor2012}, glosses added)
	
	\ex\label{exAppendixSpanishTodaviaAcceptableLimits2}
	\gll Que un gran artista ten-ga eso-s humo-s, \textbf{todavía}, pero él es un simple aprendiz.\\
	\textsc{subord} \textsc{indef}.\textsc{sg}.\textsc{m} great artist(\textsc{m}) have-\textsc{sbjv}.3\textsc{sg} \textsc{dem}.\textsc{m}-\textsc{pl} fume(\textsc{m})-\textsc{pl} still but \textsc{3sg.m} \textsc{cop}.3\textsc{sg} \textsc{indef}.\textsc{sg}.\textsc{m} simple apprentice(\textsc{m})\\
	\glt \lq A great artist having such an attitude, that would be acceptable, but heʼs just an apprentice.\rq{ }(\cite[207]{Bosque2016}, glosses added)
	
	\ex\label{exAppendixSpanishTodaviaAcceptableLimits3}
	\gll
	\textbf{Todavía} \textup{[}que eso ocurr-ies-e en público\textup{]} \textbf{pod}-\textbf{ría} \textbf{pas}-\textbf{ar}, ¡pero en privado!\\
	still \phantom{[}\textsc{subord} \textsc{dem}.\textsc{sg}.\textsc{n} happen-\textsc{pst}.\textsc{sbjv}-3\textsc{sg} in public can-\textsc{cond}.3\textsc{sg} happen-\textsc{inf} \phantom{¡}but in private\\
	\glt \lq Such a thing occurring in public, that could happen, but in private?!'
	(Goméz de la Serna, \textit{Automoribundia}, cited in \cite[222]{Bosque2016}, glosses added)
\end{exe}

\subsubsection{Additive and related functions}
\paragraph{Additive}
\label{appendixSpanishTodaviaAdditive}
\begin{itemize}
	\item \textcite{Bosque2016}, \textcite{Garrido1993}, \textcite[§§30.8k–p, 40.8l–m]{RAEGramatica} and \textcite{Trujillo1990}.
	\item A scalar inference may arise in certain contexts (\ref{exAppendixSpanishTodaviaAdditive5}); in fact, in some American varieties of Spanish \textit{todavía} serves as a scalar additive.
	\item In narrative contexts, this often involves a last event in a series, particularly with predicates of the \lq manage to, find time to\rq{ }kind (\ref{exAppendixSpanishTodaviaAdditive7}).
	\item \textcite{Bosque2016} and \textcite{RAEGramatica} point out that examples like (\ref{exAppendixSpanishTodaviaAdditive6}) are ambiguous between a phasal polarity reading and an additive one.
\end{itemize}

\begin{exe}
%	\ex\label{exAppendixSpanishTodaviaAdditive1}
%	\gll \textbf{Todavía} viv-ió en Sevilla dos año-s. / Vivi-ó en Sevilla \textbf{todavía} dos año-s.\\
%	still live-\textsc{pst}.\textsc{pfv.3\textsc{sg}} in S. two year-\textsc{pl} {} live-\textsc{pst}.\textsc{pfv.3\textsc{sg}} in S. still two year-\textsc{pl}\\
%	\glt \lq S/he lived in Seville for two more years.' (\cite[§30.8k]{RAEGramatica}, glosses added)
	
	\ex\label{exAppendixSpanishTodaviaAdditive2}
	\gll \textbf{Todavía} ca-yó un-o de los seis, llam-ad-o {Francisco Herrán};\\	
	still fall-\textsc{pst}.\textsc{pfv}.3\textsc{sg} one-\textsc{m} of \textsc{def}.\textsc{pl}.\textsc{m} six call-\textsc{ptcp}-\textsc{m} {F. H.}\\
	\glt \lq Another one of the six, called Francisco Herrán, fell;\rq{}
	\exi{}  \textit{y los demás, todos muy heridos, volviéronse a su pueblo}.\\
	\lq and the others, all of them severely injured, returned to their village.\rq{ }(CORPES XXI, glosses added)

	\ex\label{exAppendixSpanishTodaviaAdditive4}	
	\textit{Tuvo la deferencia de acompañarme hasta la puerta, y mientras bajaba yo las escaleras,}
	\\ \lq He had the politeness of accompanying me to the door, and while I was climbing down the stairs\rq{}
	
	\gll me recomend-ó \textbf{todavía} que le dijera si {de veras} nos gust-aba ese jerez.\\
	1\textsc{sg}.\textsc{obj} recommend-\textsc{pst}.\textsc{pfv}.3\textsc{sg} still \textsc{subord} 3\textsc{sg}.\textsc{m}.\textsc{dat} say.\textsc{pst}.\textsc{sbjv}.1\textsc{sg} if actually 1\textsc{pl}.\textsc{obj} like-\textsc{pst}.\textsc{ipfv}.3\textsc{sg} \textsc{dem}.\textsc{sg}.\textsc{m} sherry(\textsc{m})\\
	\glt \lq  he also recommended that I tell him if we really liked that sherry.' (Ayala, \textit{El fondo del vaso}; cited in \cite[215]{Bosque2016}, glosses added)

	\ex\label{exAppendixSpanishTodaviaAdditive5}		
	\gll Después de todo lo que hab-ía com-ido \textbf{todavía} pid-ió postre.\\
	after of everything 3\textsc{sg}.\textsc{n} \textsc{rel} have-\textsc{pst}.\textsc{ipfv}.3\textsc{sg} eat-\textsc{ptcp} still order-\textsc{pst}.\textsc{pfv}.3\textsc{sg} dessert\\
	\glt \lq After everything s/he had eaten, s/he also/even ordered dessert.\rq{}
	\\(\cite[26]{Garrido1993}, glosses added)

	
	\ex\label{exAppendixSpanishTodaviaAdditive6}	
	\gll Vamos a casa —dij-o—. \textbf{Todavía} quiero cont-ar-te una cosa.\\
	go.1\textsc{pl} to home \phantom{–}say.\textsc{pst}.\textsc{pfv}-3\textsc{sg} still want.1\textsc{sg} tell-\textsc{inf}-2\textsc{sg}.\textsc{obj} \textsc{indef}.\textsc{sg}.\textsc{f} thing(\textsc{f})\\
	\glt \lq {\lq\lq}Let’s go home" s/he said. -- \lq\lq I want to tell you one more thing / I still want to tell you something."' (Grandes, \textit{Malena es un nombre tango}, cited in \cite[§30.8k]{RAEGramatica}, glosses added)
	
		\ex\label{exAppendixSpanishTodaviaAdditive7}
		Context: About a former professional athlete.\\
	\textit{La boda fue un éxito …  se quedaron a vivir en Barcelona…}\\
	\lq The wedding was a success … they settled in Barcelona….\rq{}

	\exi{}\gll \textbf{Todavía} le dio tiempo a gan-ar una medalla de bronce en Sydney 2000,\\
	still 3\textsc{sg}.\textsc{acc}.\textsc{m} give.\textsc{pst}.\textsc{pfv}.3\textsc{sg} time to win-\textsc{inf} \textsc{indef}.\textsc{sg}.\textsc{f} medal(\textsc{f}) if bronce en S. 2000\\
	 \glt \lq He still found time to win a bronze medal at the Sidney 2000 olympics.\rq{}
	 
	\exi{} \textit{… Esa medalla precipitó su retirada a los 32 años.}\\
	 \lq This medal precipitated his retirement at 32 years of age.\rq{}
	 \\(CORPES XXI, glosses added)
\end{exe}


\paragraph{Comparisons of inequality}
\label{appendixSpanishTodaviaComparisons}
\begin{itemize}
	\sloppy
	\item \textcite{Bosque2016}, \textcite{Cid1999}, \textcite{EderlyCurco2016}, \textcite{Garrido1993}, \textcite{Martinez1996}, \textcite{Morera1999} and \textcite[§30.8q–r]{RAEGramatica}.
	\item Note that comparisons of inequality in Spanish are formed via \textit{más} \lq more'\slash\textit{menos} \lq less' (with the exception of some adjectives with a suppletive comparative form). The standard of comparison, if overtly mentioned, is  introduced by subordinator \textit{que} \parencite[ch. 45]{RAEGramatica}. \textit{Todavía} contributes the notion of \lq even more\rq{}.
	\item \textcite{Bosque2016} notes the functional overlap between \textit{todavía} as \textsc{still} and the comparative usage function in the context of degree achievements (\ref{appendixSpanishTodaviaComparisons4}).
\end{itemize}
\begin{exe}
	\ex \gll De día Marketa era \textbf{todavía} más guap-a.\\
	of day M. \textsc{cop}.\textsc{pst}.\textsc{ipfv}.3\textsc{sg} still more attractive-\textsc{f}\\
	\glt \lq By day Marketa was even more attractive.' (CORPES XXI)
	
	\ex \gll El servicio de autobus-es es \textbf{todavía} peor que el d-el metro.\\
	\textsc{def}.\textsc{sg}.\textsc{m} service(\textsc{m}) of bus-\textsc{pl} \textsc{cop}.3\textsc{sg} still worse \textsc{subord} 3\textsc{sg}.\textsc{m} of-\textsc{def}.\textsc{sg}.\textsc{m} metro(\textsc{m})\\
	\glt \lq The bus service is even worse than the metroʼs.\rq{ }(\cite[36]{EderlyCurco2016}, glosses added)
	
	\ex\label{appendixSpanishTodaviaComparisons3}
	\gll Está muy alt-a, pero \textbf{todavía} crec-erá.\\
	\textsc{cop}.3\textsc{sg} very tall-\textsc{f} but still grow-\textsc{fut}.3\textsc{sg}\\
	\glt i.\phantom{i}\lq She's very tall, but she'll grow even more.\rq
	\\ii. \lq{}She's very tall but she'll still be growing.\rq{ }(\cite[214]{Bosque2016}, glosses added)
	
		\ex\label{appendixSpanishTodaviaComparisons4}
		\gll El placer aument-ó \textbf{todavía} más.\\
		\textsc{def}.\textsc{sg}.\textsc{m} pleasure(\textsc{m}) increase-\textsc{pst}.\textsc{pfv}.3\textsc{sg} still more\\
		\glt \lq Pleasure increased even more.' (CORPES XXI, glosses added)
\end{exe}

\paragraph{Conjunctional adverbial: \textit{todavía más}/\textit{más todavía} \lq what is more'}
\label{appendixSpanishTodaviaConjunctionalTodaviaMas}
\begin{itemize}
	\item \textcite[346]{FuentesRodriguez2018}.
	\item Form: in collocation with \textit{más} \lq more'.
	\item This is a straightforward extension of the use in comparisons of inequality (\appref{appendixSpanishTodaviaComparisons}) to the textual domain.
	\item Syntax: clause-initial position.
\end{itemize}
\begin{exe}
	\ex 
	\textit{Se ha descrito que para desarrolar adicción, las mujeres requieren de beber, fumar o de utilizar drogas legales por menos tiempo y en menor cantidad.}\\
	\lq It has been described that, in order to develop an addiction, women need to drink, smoke or use legal drugs for a shorter time and and in a lower quanty.'
	\exi{} \gll  \textbf{Todavía} \textbf{más} cuando las persona-s que sufr-en dependencia a-l alcohol y otr-a-s droga-s recurr-en a ayuda profesional o de grupo-s de autoayuda, la severidad de adicción … es generalmente mayor entre las mujer-es.\\
	still more when \textsc{def}.\textsc{pl}.\textsc{f} person(\textsc{f})-\textsc{pl} \textsc{subord} suffer-3\textsc{pl} addiction to-\textsc{def}.\textsc{sg}.\textsc{m} alcohol(\textsc{m}) and other-\textsc{f}-\textsc{pl} drug(\textsc{f})-\textsc{pl} resort\_to-3\textsc{pl} to help professional or of group-\textsc{pl} of self\_help \textsc{def}.\textsc{sg}.\textsc{f} severity(\textsc{f}) of addiction {} \textsc{cop}.3\textsc{sg} generaly bigger between \textsc{def}.\textsc{pl}.\textsc{f} woman(\textsc{f})-\textsc{pl}\\
	\glt \lq What is more, when people suffering from addiction to alcohol or other drugs make use of professional help or self-help groups, the addiction is usually more severe in women.\rq{ }(\cite[346]{FuentesRodriguez2018}, glosses added)

	\ex \textit{Según Pessoa, lo que caracteriz al genio literario es la inadaptación a su medio.}\\
	\lq According to Pessoa, what characterises literary genius is its nonconformity with its medium.'
	\exi{}\gll  \textbf{Más} \textbf{todavía}: la fama literari-a de hoy excluy-e el éxito en el porvenir …\\
	more still \textsc{def}.\textsc{sg}.\textsc{f} fame(\textsc{f}) literary-\textsc{f} of today exclude-3\textsc{sg} \textsc{def}.\textsc{sg}.\textsc{m} success(\textsc{m}) in \textsc{def}.\textsc{sg}.\textsc{m} future(\textsc{m})…\\
	\glt \lq What is more: today's literary fame impedes future success.\rq{ }(CORPES XXI, glosses added)
\end{exe}


\subsubsection{Restrictive (non-temporal)}
\paragraph{Scalar restrictive \textit{todavía si} \lq if at least'}
\label{appendixSpanishTodaviaIfAtLeast}
\begin{itemize}
	\item \textcite{Bosque2016}, \textcite{Deloor2012} and \textcite[s.v. \textit{todavía}]{RAEDictionary}.
	\item Form: in combination with \textit{si} \lq if' introducing the protasis of a counterfactual conditional.
	\item In this function, \textit{todavía} indicates that the condition depicted in the protasis constitutes a minimal requirement.	This reading is very close to the \lq within acceptable limits' use (\appref{appendixSpanishTodaviaAcceptableLimits}). Like the latter, it can be considered an extension of a construal of marginality (\appref{appendixSpanishTodaviaMarginal}).
	\item Ellipsis of the apodosis is common (\ref{exAppendixSpanishTodaviaIfOnly4}).
\end{itemize}
\begin{exe}
	\ex\label{exAppendixSpanishTodaviaIfOnly1}
	\gll ¿Para qué ahorra-s?; \textbf{todavía} \textup{[}\textbf{si} tuvier-as hijo-s\textup{]} est-aría justificado.\\
	for what save\_money-2\textsc{sg} still \phantom{[}if have.\textsc{pst}.\textsc{sbjv}-2\textsc{sg} child-\textsc{pl} \textsc{cop}-\textsc{cond}.3\textsc{sg} justified\\
	\glt \lq What are you saving money for? If you at least had kids, then it would make sense.' (\cite[s.v. todavía]{RAEDictionary}, glosses added)
	
	\ex\label{exAppendixSpanishTodaviaIfOnly2}
\textit{	Toda la eternidad de Dios quemándoos en el infierno, con el  calor, el fuego, y sin poder beber agua ni nada... por pensar en ese calvo.}\\
	\lq God's whole eternity burning in hell, with all the heat, fire, and without being able to drink water or anything… (all that) for thinking about that bold guy.'
	\exi{}\gll \textbf{Todavía} \textbf{si} fuera el padre Felipe, lo comprend-ería. Pero 
anda que el huertero...\\
still if \textsc{cop}.\textsc{pst}.\textsc{sbjv}.3\textsc{sg} \textsc{def}.\textsc{sg}.\textsc{m} father(\textsc{m}) F. 3\textsc{sg}.\textsc{acc}.\textsc{m} understand-\textsc{cond}.1\textsc{sg} but \textsc{dm} \textsc{subord} \textsc{def}.\textsc{sg}.\textsc{m} gardener(\textsc{m})\\
\glt \lq If it at least were father Felipe, I'd understand. But the gardener…' (CORPES XXI, glosses added)

	\ex\label{exAppendixSpanishTodaviaIfOnly4}
	 \gll … \textbf{todavía} \textbf{si} te pag-as-en, pero, ya ve-s, veinte duros por artículo, una miseria.\\
	{} still if 2\textsc{sg}.\textsc{obj} pay-\textsc{pst}.\textsc{sbjv}-3\textsc{pl} but already see-2\textsc{sg} twenty nickles for article, \textsc{indef}.\textsc{sg}.\textsc{f} pittance(\textsc{f})\\
	\glt \lq If at least they paid you, but you can see, twenty nickles for an article, a mere pittance.ʼ (Delibes, \textit{Cinco horas con Mario}, cited in \cite[222]{Bosque2016}, glosses added)
\end{exe}

\subsubsection{Broadly modal and interactional uses}
\paragraph{Concessive protases: \textit{todavía que}}
\label{appendixSpanishTodaviaConcessiveAntecedent}
\begin{itemize}
	\sloppy
	\item \textcite{Bosque2016}, \textcite{Cid1999}, \textcite{Morera1999} and \textcite[§30.8o]{RAEGramatica}.	
	\item Form: together with a clause introduced by the subordinator \textit{que}.
	\item In present-day Spanish, this use is restricted to the Americas.	
	\item According to \textcite{Morera1999} and \textcite[§30.8o]{RAEGramatica}, \textit{todavía que} can often be paraphrased as \textit{encima de que} p \lq on top of \textit{p}\rq{}, which suggests an additive component and parallels what is found with \textit{todavía} as a marker of concessive apodoses (\appref{appendixSpanishTodaviaConsessiveConsequent}).
	\item \textcite{Bosque2016} sees an additional link between this use and the scalar element of the time-scalar additive \lq as late as\rq{ }function (\appref{appendixSpanishTodaviaTimeScalar}). 
\end{itemize}
\begin{exe}
	\ex
	\gll Mi herman-o es un entrometid-o, bien infantil, \textbf{todavía} \textbf{que} le prest-é mi computadora, sac-ó unas foto-s de una fiesta con mi-s amig-o-s y and-a de chismoso con mi-s papá-s.\\
	\textsc{poss}.1\textsc{sg} sibling-\textsc{m} \textsc{cop}.3\textsc{sg}	\textsc{indef}.\textsc{sg}.\textsc{m} nosey-\textsc{m} well childish still \textsc{subord} 3\textsc{sg}.\textsc{dat}.\textsc{m} lend-\textsc{pst}.\textsc{pfv}.1\textsc{sg} \textsc{poss}.1\textsc{sg} computer remove-\textsc{pst}.\textsc{pfv}.3\textsc{sg} \textsc{indef}.\textsc{pl}.\textsc{f} photo(\textsc{f})-\textsc{pl} of \textsc{indef}.\textsc{sg}.\textsc{f} party(\textsc{f}) with \textsc{poss}.1\textsc{sg}-\textsc{pl} friend-\textsc{m}-\textsc{pl} and walk-3\textsc{sg} of gossip with \textsc{poss}.1\textsc{sg}-\textsc{pl} parent-\textsc{pl}\\
	\glt \lq My brother is so nosey and immature, even when I [was so kind and] lent him my computer [instead of being grateful] he dug up some photos from a party with my friends and goes around gossiping with my parents.' (Internet example, cited in \cite[206]{Bosque2016}, glosses added)

	\ex
	\gll ¡No sea desagradecid-a! \textbf{¡Todavía} \textbf{que} la dej-amos entrar, contest-a con sarcasmo-s!\\
	\phantom{!}\textsc{neg} \textsc{cop}.\textsc{sbjv}.3\textsc{sg} ungrateful-\textsc{f} still \textsc{subord} 3\textsc{sg}.\textsc{acc}.\textsc{f} let-1\textsc{pl} enter.\textsc{inf} reply-3\textsc{sg} with sarcasm-\textsc{pl}\\
	\glt \lq She shouldn't be ungrateful! Even though we let her in / As if it weren't sufficient that we let her in, she responds with ridicule.' (Gotbeter, \textit{La prudencia}; cited in \cite[206]{Bosque2016}, glosses added)
\end{exe}
\pagebreak
\paragraph{Concessive apodoses}
\label{appendixSpanishTodaviaConsessiveConsequent}
\begin{itemize}
	\item \textcite{Bosque2016}, \textcite{Cid1999}, \textcite{Deloor2012}, \textcite{EderlyCurco2016}, \textcite{Garrido1993}, \textcite{Morera1999} and \textcite[§30.8ñ]{RAEGramatica}.
	\item This concessive function has been described as often involving an additive notion \parencite[§30.8ñ]{RAEGramatica}, which is particularly salient in cases like (\ref{exAppendixSpanishTodavíaConcessiveConsequent1}, \ref{exAppendixSpanishTodavíaConcessiveConsequent2}) and which corresponds to what is found with its protasis marking counterpart \textit{todavía que} (\appref{appendixSpanishTodaviaConcessiveAntecedent}).	As a correlate, \textcite{EderlyCurco2016} point out that concessive \textit{todavía} is not congruent with concessive \textit{still}, i.e. \#\textit{Trataron de ayudarle, pero todavía murió} \lq They tried to help him, but he still died.'
	\item \textcite{Bosque2016} speculates that the presence of the universal quantifier \textit{toda} provides additional motivation for its use, given its presence in concessive expressions like \textit{con todo} lit. \lq{}with all', \textit{de todas formas}, \textit{de todas maneras} \lq anyway'. 
\end{itemize}
\begin{exe}
	\ex\label{exAppendixSpanishTodavíaConcessiveConsequent1}
	\gll Y \textbf{todavía} tuv-iste el descaro de enviar-me una tarjeta en la que me dec-ías \lq\lq Compañer-a: Cuba es extraordinari-a”.\\
	and still have.\textsc{pst}.\textsc{pfv}-2\textsc{sg} \textsc{def}.\textsc{sg}.\textsc{m} impudence(\textsc{m}) of send.\textsc{inf}-1\textsc{sg}.\textsc{obj} \textsc{indef}.\textsc{sg}.\textsc{f} card(\textsc{f}) en \textsc{def}.\textsc{sg}.\textsc{f} \textsc{subord} 1\textsc{sg}.\textsc{obj} say-\textsc{pst}.\textsc{ipfv}.2\textsc{sg} \phantom{\lq\lq}companion-\textsc{f} Cuba \textsc{cop}.3\textsc{sg} extraordinary-\textsc{f}\\
	\glt \lq And yet/on top of that you were so imprudent as to send me a card telling me \lq\lq Comrade: Cuba is extraordinario!"' (Espinosa, \textit{Testigos de Jesús}, cited in  \cite[§30.8ñ]{RAEGramatica}, glosses added)

	\ex\label{exAppendixSpanishTodavíaConcessiveConsequent2}
	\gll Los obrer-o-s solo sab-en hac-er huelga-s y pon-er petardo-s, ¡y \textbf{todavía} pretend-en que se les d-é la razón!\\
	\textsc{def}.\textsc{pl}.\textsc{m} worker-\textsc{m}-\textsc{pl} only know-3\textsc{pl} do-\textsc{inf} strike-\textsc{pl} and put-\textsc{inf} firecracker-\textsc{pl} \phantom{¡}and still pretend-3\textsc{pl} \textsc{subord} \textsc{refl}.3 3\textsc{pl}.\textsc{dat} give-\textsc{sbjv}.3\textsc{sg} \textsc{def}.\textsc{sg}.\textsc{f} reason(\textsc{f})\\
	\glt The workers only know how to go on strike and light up firecrackers, and yet/on top of that they want people to agree with them!' (Mendoza, \textit{La verdad sobre el caso Savolta}; cited in \cite[§30.8ñ]{RAEGramatica}, glosses added)

	\ex 
	\gll Lo ayud-é ayer y \textbf{todavía} me reclam-a.\\
	3\textsc{sg}.\textsc{acc}.\textsc{m} help-\textsc{pst}.\textsc{pfv}.3\textsc{sg} yesterday and still 1\textsc{sg}.\textsc{obj} claim-3\textsc{sg}\\
	\glt \lq I helped him yesterday and nevertheless he complains.\rq{ }(\cite[36]{EderlyCurco2016}, glosses added)
\end{exe}
\il{Spanish|)}

	
\section{Thai (tha, thai1261)}\il{Thai|(}
\label{appendixThai}
\subsection{Introductory remarks}
\begin{sloppypar}
My understanding of the Thai data has greatly profited from discussion with Chingduang Yurayong, who also provided additional examples, helped with glosses, and aided in standardizing the transliteration. The Thai \textsc{still} expression \textit{yaŋ} often co-occurs with the post-predicate auxiliary \textit{yúu}, for simplicity here glossed as \textsc{cont} \lq continuous', as in (\ref{exAppendixThai2}). This marker, originally a locative copula, has a general function of signalling that a situation is transient (see \cite{Jenny2001}). In many of its extended functions, \textit{yaŋ} combines with a morpheme \textit{kɔ̂ɔ}, for present purposes glossed as \textsc{dm} \lq discourse marker'. This is a notoriously versatile item which covers many functions broadly related to linkage, such as additive \lq also', clause linkage (\lq so, and then'), or hesitation (\lq well…'); see  \textcite[170–177]{IwasakiIngkaphirom2005} . In addition, \textit{kɔ̂ɔ} also serves as a main clause marker conveying a notion of relevance or evaluation \parencite[253–254]{IwasakiIngkaphirom2005}. For an attempt at unveiling its etymology see \textcite{Burusphat2004}.
\end{sloppypar}

\subsection{yaŋ}

\subsubsection{General information}

\begin{itemize}
	\item Form: found in various transliterations (\textit{yaŋ}, \textit{yang}, \textit{ya:ng},  \textit{yung}, and \textit{jaŋ}).
	\item Wordhood: independent grammatical word, invariable.
	\item Syntax: fixed position, immediately after the subject and before all elements belonging to the predicate.
	\item Etymology: originally a locative copula.
\end{itemize}	

	
\subsubsection{As a \lq{}still\rq{ }expression}
	\label{appendixThaiStill}
		\begin{itemize}
			\item \Textcite{vanBaar1997}, \textcite[100]{CampbellShaweevongs1982}, \textcite[95–97, 105–106]{HigbieThinsan2002}, \textcite[153, 156]{IwasakiIngkaphirom2005}, \textcite{Jenny2001}, \textcite[81, 182]{Noss1964} and \textcite[103, 139–140]{Smyth2002}.
			\item Specialisation: the descriptive materials show that this marker conforms to my definition; see especially \textcite[118]{Jenny2001}, who explicitly mentions the imcompatibility of \textit{yaŋ} with inalterable states. The expression is also identified by \textcite{vanBaar1997} as one that is in line with my definition.
			\item Pragmaticity: according to \textcite[76]{vanBaar1997} \textit{yaŋ} by itself is restricted to the neutral scenario, whereas in the unexpectedly late scenario \textit{yaŋ} is combined with \textit{talɔ̀ɔt} \lq throughout, completely'. The data considered for the present study, however, suggest that \textit{yaŋ} by itself is compatible with both scenarios. For instance, (\ref{exAppendixThai3}) looks like a prototypical case of the late scenario, albeit bordering on a concessive reading.
	\item Polarity sensitivity: inner negation yields \textsc{not yet}.
	\item Further note: according to \textcite{Jenny2001}, \textit{yaŋ} is incompatible with imperatives and prohibitives.
	\end{itemize}

\begin{exe}
	\ex\label{exAppendixThai1}
	\gll Rau yáay maa yùu kruŋthêep tâŋtɛ̀ɛ chán \textbf{yaŋ} dèk.\\
	1\textsc{pl} move come live Bangkok since 1\textsc{sg} still child\\
	\glt \lq We moved (here) to Bangkok when I was still a child.\rq{ }(\cite[62]{Smyth2002}, glosses added)

	\ex\label{exAppendixThai2}
	\gll Tháŋ		bâan		tɛ̀ɛláʔ		lǎŋ	\textbf{yaŋ}	plùuksâaŋ	yùu hàaŋ		kan		mâi		ʔɛɛʔàt	yátyîat	mɯ̌an	pàtcubanníi.\\
	all house every \textsc{clf} still build \textsc{cont} far \textsc{recp} \textsc{neg} crowded tightly alike at\_present\\
	\glt \lq Each and every home was [still] built separate from its neighbour, not crowded together like [they are] today.' (found online)\footnote{\url{http://www.thai-language.com/id/225350} (09 June, 2021).}

	\ex\label{exAppendixThai3}
	\gll Nâaplɛ̀ɛk	thîi		tháŋ		nɛɛn		lɛ́ʔ	kháu		lɤ̂ɤk		kan	lɛ́ɛw tɛ̀ɛ		\textbf{yaŋ}		pen		phɯ̂an	nai	féesbúk.\\
	strange	\textsc{comp}	all		N.		and	3	separate	\textsc{recp}	already but		still		\textsc{cop}		friend	in	Facebook\\
	\glt \lq It is strange that Naen and her ex had divorced, but they were still \lq\lq friends" on Facebook.' (found online)\footnote{\url{http://www.thai-language.com/id/244919} (10 June, 2021).}
\end{exe}

\paragraph{Scalar contexts}\label{appendixThaiScalar}
\begin{itemize}
	\item \textcite{Zhang2017}
	\item \textit{Yaŋ} is attested in scalar contexts of decreases over time (\ref{exAppendixThaiScalar1}, \ref{exAppendixThaiScalar2}) and limited increases (\ref{exAppendixThaiScalar3}, \ref{exAppendixThaiScalar4}). Note the absence of an overt \lq only\rq{ }marker in ( \ref{exAppendixThaiScalar4}).
\end{itemize}
\begin{exe}
	\ex\label{exAppendixThaiScalar1}
	 \gll \textbf{Yaŋ}	mii	phǒnlamái	lɯ̌a		yùu	bon	tôn	sák	lûuk	sɔ̌ɔŋ	lûuk.\\
	still	exist	fruit		remain	\textsc{cont}	on tree	just	\textsc{clf}	two	\textsc{clf}\\
	\glt \lq There are still about one or two pieces of fruit left on the tree.\rq{ }\parencite[21]{Zhang2017}

		
		\ex\label{exAppendixThaiScalar2}
	\gll Khun	khruu		wian	klàaw	chomchɤɤy	wâa		dii	mâak lɛ́ɛw	bɔ̀ɔk	wâa		\textbf{yaŋ}	lɯ̌a		thífaam	pen	khon		sùttháay hâi	ʔɔ̀ɔk	maa	ʔàan	riaŋkhwaam.\\
	\textsc{hon} teacher W. speak praise \textsc{comp} godo very already tell \textsc{comp} still remain {T.} \textsc{cop} person last \textsc{caus} out come read essay\\
	\glt \lq Teacher Wien praised her saying that she did a very good job and said that there was only one student remaining, Tifam, to give his presentation.\rq{ }(found online)\footnote{\url{http://www.thai-language.com/id/246826} (27 February, 2023).}

		
	\ex\label{exAppendixThaiScalar3}
	Context: We’re supposed to be sent 5 books.
	\begin{xlist}
		\exi{A:} I've got four.
		\exi{B:} Me, too.
		\exi{C:} \gll Phǒm	\textbf{yaŋ}	dâi	khɛ̂ɛ	sǎam	lêm	tɛ̀ɛ	ʔìik	sɔ̌ɔŋ	lêm	khoŋ		kamlaŋ	maa.\\
		1\textsc{sg}	still	get	only	three \textsc{clf}	but	more two	\textsc{clf}	maybe	\textsc{prog}		come\\
	\glt \lq I still got only three, but the other two are probably coming.\rq{ }(Chingduang Yurayong, p.c.)
	\end{xlist}
	
	\ex\label{exAppendixThaiScalar4}
	\begin{xlist}
		\exi{A:}\gll Lûuk	khun	ʔaayúʔ	sìphâa	pii	châi	mái?\\
		child 2\textsc{sg} age fifteen year yes \textsc{neg}\\
		\glt \lq Your child is fifteen years old, right?\rq{}
		\exi{B:} \gll Mâi	châi,	\textbf{yaŋ}	ʔaayúʔ	sìpsìi		pii	jùu.\\
		\textsc{neg} yes, still age fourteen year \textsc{cont}\\
		\glt \lq No, she's still only fourteen years old.\rq{ }(Chinduang Yurayong, p.c.)
	\end{xlist}
\end{exe}


\subsubsection{Uses related to other phasal polarity concepts}
\paragraph{Not yet (and interrogative \lq yet\rq)}
\label{appendixThaiNotYet}
	\begin{itemize}
		\item  \Textcite[62]{vanBaar1997}, \textcite[97–100]{CampbellShaweevongs1982}, \textcite[60, 93, 98]{HigbieThinsan2002}, \textcite[284–286]{IwasakiIngkaphirom2005}, \textcite{Jenny2001}, \textcite[81, 84, 182–183]{Noss1964}, and \textcite[113, 150, 153, 157–159]{Smyth2002}.
		\item \textit{Yaŋ} as \textsc{not yet} without negation is attested in the following two contexts, both (originally) characterised by the absence of an overt predicate:
		\begin{itemize}
			\item In polar questions that follow a pattern \textit{lɛ́ɛw rɯ̌ɯ yaŋ} \lq already or still > \lq already or not yet\rq{ }(\ref{exAppendixThaiNotYet1}). This can be understood as an instantiation of a broader \lq or \textsc{neg}' pattern of question formation (e.g. \cite[283–284]{IwasakiIngkaphirom2005}; \cite[59–60]{HigbieThinsan2002}). This question tag is undergoing erosion: \textit{lɛ́ɛw} \lq already' or \textit{rɯ̌ɯ} \lq or' can be dropped (\ref{exAppendixThaiNotYet3}, \ref{exAppendixThaiNotYet4}). What is more, omission of both is attested as well, leaving \textit{yaŋ} as interrogative \lq yet' in direct (\ref{exAppendixThaiNotYet5}) and indirect questions (\ref{exAppendixThaiNotYet6}). Note that Thai does not make use of negative raising with \textit{yaŋ} (Chingduang Yurayong, p.c.); instead \textit{lɛ́ɛw} \lq already' is used in the subordinate clause (\ref{exAppendixThaiNotYet7}).
					\item As a negative response (\ref{exAppendixThaiNotYet4}).
		\end{itemize}
	\end{itemize}

\begin{exe}
	\ex \label{exAppendixThaiNotYet1} 
	\gll Thíŋ		còtmǎay	pai	tûu	praisanii	lɛ́ɛw		rɯ̌ɯ	\textbf{yaŋ}?\\
	throw letter go box mail already or still\\
	\glt \lq Did (you) put the letter into the mail-box or not yet?\rq{ }(\cite[500]{Koelver1991}, glosses added)

	\ex\label{exAppendixThaiNotYet3}
	\gll Khun	hâi	ʔaahǎan	mǎa	\textbf{rɯ̌ɯ}	\textbf{yaŋ}?\\
	2 give food dog or still\\
	\glt \lq Have you fed the dog yet?' \parencite[80]{Smyth2002}

	\ex\label{exAppendixThaiNotYet4}
	\begin{xlist}
		\exi{A:} \gll Rîak	rót	thɛ́ksîi	\textbf{lɛ́ɛw}		\textbf{yaŋ}?\\
		call car taxi already still\\
		\glt \lq Did you call the taxi [yet]?

		\exi{B:} \gll \textbf{Yaŋ}	khà.\\
		still \textsc{dm}:politeness.\textsc{f}\\
		\glt \lq No, not yet.' (found online)\footnote{\url{http://www.thai-language.com/id/590018} (11 March, 2021).}
	\end{xlist}


	\ex
	\label{exAppendixThaiNotYet5}
	\gll Hǐw		\textbf{yaŋ}?\\
	 hungry still\\
	 \glt \lq Are you hungry yet?' (found online)\footnote{\url{http://www.thai-language.com/id/246231} (10 June, 2021).}
	 
 	\ex
	\label{exAppendixThaiNotYet6}
	\gll Chán	sǒŋsǎi	wâa		khun	kin	khâaw	\textup{(}lɛ́ɛw		rɯ̌ɯ\textup{)}		\textbf{yaŋ}.\\
	1\textsc{sg} wonder \textsc{comp} 2 eat rice \phantom{(}already or still\\
	\glt \lq I wonder whether you have eaten yet.' (Chingduang Yurayong, p.c.)
	
	\ex
	\label{exAppendixThaiNotYet7}
	\gll Chán	mâi	khít	wâa		kháu	maa		(thɯ̌ŋ)	lɛ́ɛw.\\
	1\textsc{sg} \textsc{neg} think \textsc{comp} 3 come \phantom{(}arrive already\\
	\glt \lq I don't think he has arrived yet.' (Chingduang Yurayong, p.c.)
\end{exe}


\pagebreak
\subsubsection{Marginality}
\label{appendixThaiMarginal}
\begin{itemize}
	\item \textit{Yaŋ} is compatible with a marginality use.
	\item As with many functions of \textit{yaŋ}, in this use it tends to be accompanied by \textit{kɔ̂ɔ}, as in (\ref{exAppendixThaiMarginal1}, \ref{exAppendixThaiMarginal3}).  Example (\ref{exAppendixThaiMarginal2}) shows that this appears not to be compulsory.
\end{itemize}

\begin{exe}
	\ex\label{exAppendixThaiMarginal1}
	 Context: Talking about tennis skills.\\
	\gll Thɔm		chán		\textbf{yaŋ}		chaná		dâi.	/ Chán		\textbf{yaŋ}		chaná	thɔm			dâi.\\
	Tom 1\textsc{sg} still win can {} 1\textsc{sg} still win Tom can\\	
	\glt \lq Tom I can still beat / I can still beat Tom (but other players are too good for me).' (Chingduang Yurayong, p.c.)

	\ex\label{exAppendixThaiMarginal2}
	\gll Mɛ́kdonɔ̂l		thîinân	phɛɛŋ		kwàa		tɛ̀ɛ	kɔ̂ɔ	\textbf{yaŋ}	thùuk		yùu.\\
McDonald’s	there		expensive	exceed	but	\textsc{dn}	still	cheap		\textsc{cont}\\
	\glt \lq McDonald\rq{}s is more expensive over there, but it\rq{}s still cheap.\rq{ }(\cite[106]{HigbieThinsan2002}, glosses added)

	\ex\label{exAppendixThaiMarginal3}
	\gll Chiaŋraay		kɔ̂ɔ	yaŋ	pen	praʔthêet	thai	yùu.\\
{Chiang Rai} \textsc{dm} \textbf{still} be country Thai \textsc{cont}\\
	\glt \lq Chiang Rai [city in the north] is still Thailand (as opposed to other places across the border).' (Chingduang Yurayong, p.c.)
\end{exe}

\subsubsection{Additive and related functions}

\paragraph{Additive}\label{appendixThaiAdditive}
\begin{itemize}
	\item \textcite[122]{Smyth2002} and \textcite[147]{HigbieThinsan2002}.
\end{itemize}
	
\begin{exe}
	\ex \gll Mâi	phiaŋ		tɛ̀ɛ	tɔ̂ŋ	pàʔtisèet	ʔaamítsǐncâaŋ	rɯ̌ɯ	sǐnbon thúk		chanít	hàak	tɛ̀ɛ	\textbf{yaŋ}	mâi	klua	thîi		càʔ phíphâaksǎa		tàtsǐn		phûu		lamɤ̂ɤt	ʔamnâat	sǎan.\\
			\textsc{neg} only but must deny bribery or bribe every type if but still \textsc{neg} fear \textsc{comp} \textsc{irr} deliver\_judgement judge \textsc{nmlz} break\_law authority court\\
	\glt \lq Not only must (we) refuse bribery, graft, or payoffs of any kind, but (we) must [also] not be afraid of judging those who are contemptuous of (our) courts.' (found online)\footnote{\url{http://www.thai-language.com/id/216487} (01 March, 2021).}
			
	\ex
	\gll \textbf{Yaŋ}	mii	bùkkhon	nai	prawàtsàat	thai	sɯ̂ŋ	dâi	thùuk	banthɯ́k	ʔau	wái…\\
		still exist person in history	Thai \textsc{rel}	 get pass record take keep\\
	\glt \lq There are also individuals in Thai history who are noteworthy …' (found online)\footnote{\url{http://www.thai-language.com/id/214624} (27 February, 2021).}
\end{exe}


\paragraph{Scalar additive}\label{appendixThaiScalarAdditive}
\begin{itemize}
	\item \textcite[240]{HigbieThinsan2002} and \textcite[182]{Noss1964}.
	\item It appears that addition of \textit{kɔ̂ɔ}, as in (\ref{exAppendixThaiScalarAdditive3}) gives a strong emphasis to the scalar reading (Chingdong Yurayong, p.c.)
	\item Several examples, including (\ref{exAppendixThaiScalarAdditive1}) feature an additional additive marker \textit{dûay}.
\end{itemize}

\begin{exe}

	\ex\label{exAppendixThaiScalarAdditive1}
\gll Phǒm	chûay	kháu	talɔ̀ɔt.		baaŋkhráŋ	\textbf{yaŋ}	hâi	ŋɤn		kháu	dûay.\\
1\textsc{sg} help 3 all\_the\_time sometimes still give money 3 also\\
	\glt \lq I help him all the time. Sometimes I even give him money.' (\cite[240]{HigbieThinsan2002}, glosses added)


	\ex\label{exAppendixThaiScalarAdditive2}
	 \gll Fǎafɛ̀ɛt	khûu	nîi	mɯ̌an	kan		píap. Baaŋkhráŋ	phɔ̂ɔ		mɛ̂ɛ		\textbf{yaŋ}		yɛ̂ɛk		mâi		ʔɔ̀ɔk.\\
twin		pair	\textsc{prox}	alike	\textsc{recp} \textsc{ideoph}	sometimes father mother still seperate \textsc{neg} out\\
	\glt \lq The two twins look exactly alike; sometimes even their parents canʼt tell them apart.' (found online)\footnote{\url{http://www.thai-language.com/id/247337} (11 March, 2021).}

	\ex\label{exAppendixThaiScalarAdditive3}
	\gll Mâi	ʔàʔ.	mâi	khɔ̂y	sabaay	tɔɔn		níi yàawâatɛ̀ɛ	kin	lɤɤy		mɛ́ɛ		dɤɤn	kɔ̂ɔ	\textbf{yaŋ}	mâi	wǎi.\\
	\textsc{neg} \textsc{sfp} \textsc{neg} little {fine} moment \textsc{prox} not\_only eat at\_all though walk \textsc{dm} still \textsc{neg} manage\\
	\glt \lq No (thanks). I\rq{}m not feeling very well. Not only can\rq{}t I eat, but I can\rq{}t even walk.' (found online)\footnote{\url{http://www.thai-language.com/id/230604} (10 June, 2021).}
\end{exe}

\pagebreak
\paragraph{Comparisons of inequality}\label{appendixThaiComparisons}
\begin{itemize}
	\sloppy
	\item \textcite[240]{HigbieThinsan2002} and \textcite[21]{Zhang2017}.
	\item Note that Thai makes use of an exceed-comparative construction \parencite{Stassen2013}, where the marker \textit{kwàa} follows the predicate and precedes the standard of comparison.
	\item Addition of \textit{yaŋ} indicates \lq even more\rq{}.
	\item There appears to be a preference for \textit{yaŋ} to be used with predicates denoting the lower end of a scale (Chingduang Yurayong, p.c.), which is reflected in the examples found in the literature: \textit{cháa} \lq slow' in (\ref{exAppendixThaiComparison1}, \ref{exAppendixThaiComparison2}) and \textit{nǎaw} \lq cold' in (\ref{exAppendixThaiComparison3}).
\end{itemize}

\begin{exe}
	\ex\label{exAppendixThaiComparison1}
	\gll Kháu	\textbf{yaŋ}	phûut		cháa	kwàa.\\
3 still speak slow exceed\\
	\glt \lq He speaks even more slowly (than someone else).' (\cite[190]{Noss1964}, glosses added)

	\ex\label{exAppendixThaiComparison2}
	\gll Phǒm	wîŋ	cháa	lɛ́ɛw		\textup{(}tɛ̀ɛ\textup{)}	kháu	\textbf{yaŋ}	cháa	kwàa		phǒm		ìik.\\
1\textsc{sg} run slow already \phantom{(}but 3 still slow exceed 1\textsc{sg} more\\
	\glt \lq I run slowly, but he runs even more slowly than I do.' (\cite[240]{HigbieThinsan2002}, glosses added)

	\ex\label{exAppendixThaiComparison3}
	 \gll Thaaŋ		tawanʔɔ̀ɔk	kɔ̂ɔ	yaŋ		nǎaw		kwa..\\
way	east	\textsc{dm} still	cold exceed\\
	\glt \lq It is still colder on the east side'. \parencite[21]{Zhang2017}
\end{exe}

\subsubsection{Broadly modal and interactional functions}

\paragraph{Concessive apodoses}
\label{appendixThaiConcessiveConsequent}
\begin{itemize}
	\item \textcite[107–108]{HigbieThinsan2002}, \textcite{Jenny2001}, and \textcite[182]{Noss1964}.
	\item In this function, \textit{yaŋ} is often preceded by the discourse marker \textit{dûay}, as in (\ref{exAppendixThaiConcessive2}) and~-- optionally~-- in (\ref{exAppendixThaiConcessive3}).
\end{itemize}

\begin{exe}
	\ex\label{exAppendixThaiConcessive1}
 \gll Mɛ́ɛtɛ̀ɛ	yâatphîinɔ́ɔŋ	khɔ̌ɔŋ		kháu	kháu	\textbf{yaŋ}	mâi	chûay	lɤɤy nápprasǎaʔarai		kàp	phɯ̂an	thîi	mâi	khɤɤy		sanìt yàaŋ	khun..\\
even relatives of 3 3 still \textsc{neg} help at\_all what\_can\_you\_expect with friend \textsc{rel} \textsc{neg} ever close\_together type 2\\
\glt \lq He wonʼt even help his flesh and blood, let alone someone like you who is not even his close friend (lit. even though they are his kin he still does not help them...).' (found online)\footnote{\url{http://www.thai-language.com/id/216252} (01 March, 2021).}

	\ex\label{exAppendixThaiConcessive2}
	\gll Kháu	mâi	mii	ŋɤn		mâak		tɛ̀ɛ	kháu	kɔ̂ɔ	\textbf{yaŋ}	hâi	phǒm..\\
3 \textsc{neg} have money much but 3 \textsc{dm} still give 1\textsc{sg}\\
	\glt \lq She didn’t have much money, but she gave me some anyway.\rq{ }(\cite[107]{HigbieThinsan2002}, glosses added)

	\ex\label{exAppendixThaiConcessive3}
	\gll Tháŋtháŋthîi	fǒn	tòk		tɛ̀ɛ	rau	\textup{(}kɔ̂ɔ\textup{)}	\textbf{yaŋ}	pai..\\
even\_though rain heavy but 1\textsc{pl} \textsc{dm} still go\\
	\glt \lq Although it's raining, we're still going.' (\cite[121]{Smyth2002}, glosses added)
\end{exe}
\il{Thai|)}

\section{Tundra Nenets (yrk, nene1249)}\il{Nenets, Tundra|(}
\label{appendixTundraNenets}
\subsection{Introductory remarks}
Apart from descriptive materials, I searched \citeauthor{NikolaevaEtAl2019}'s (\citeyear{NikolaevaEtAl2019}) online text collection. I am furtheremore indebted to Irina Nikolaeva and Tapani Salminen for discussing Tundra Nenets data with me and for helping with glosses.

\subsection{təmna}

\subsubsection{General information}
\begin{itemize}
	\sloppy
	\item Form: also transcribed as \textit{ta̯mnɒ} and, in Cyrillic, \textit{тaмнa}. 
	\item Wordhood: independent grammatical word; mostly invariable, but takes person\hyp number and tense marking in its \lq not ready yet' function (\appref{appendixTundraNenetsNotYet}).
	\item Etymology: related to the pronominal stem \textit{tə}, which also has a temporal meaning \lq then\rq{}, plus prolative \mbox{-\textit{mna}}, i.e. \lq through then\rq{ }(\cite[144]{Janhunen1977}; \cite[199]{Salminen1998}).
\end{itemize}

\largerpage
\subsubsection{As a \lq{}still\rq{ }expression}
\begin{itemize}
	\item \textcite[144]{Janhunen1977}, \textcite[458]{Lehtisalo1956}, \textcite[186]{Nikolaeva2014} and \textcite[623]{Tereshchenko2008}.
	\item Specialisation: the descriptions and dictionary entries, when taken together, give a good indication that this item corresponds to English \textit{still}, German \textit{noch} and Russian \textit{eščë}. This also becomes evident in examples like (\ref{exAppendixTundraNenets1}–\ref{exAppendixTundraNenets3}). For instance, in (\ref{exAppendixTundraNenets1}) \textit{təmna} restricts the time frame to one in which the speaker and companions continue to be children, as opposed to being grown up.
	\item Polarity sensitivity: inner negation yields \textsc{not yet}.
	\item Pragmaticity: the data allow no conclusions.
	\item Syntax: in phasal polarity function, \textit{təmna} typically precedes the predicate and object, but follows the subject NP.
\end{itemize}
\begin{exe}
	\ex\label{exAppendixTundraNenets1}
	Context: From the opening paragraph of an autobiographical account.\\
	\gll Tí pæ°rt′a° \textbf{təmna} n′ada-ŋko-s′°ti-waq əc′ekewaq æ-b°q-naqa.\\
	reindeer:\textsc{acc}.\textsc{pl} do:\textsc{nmlz}.\textsc{ipfv}:\textsc{dat}.\textsc{pl} still help-\textsc{intr}-\textsc{hab}-1\textsc{pl} child:\textsc{poss}.1\textsc{pl} \textsc{cop}-\textsc{cond}-1\textsc{pl}.\textsc{emph}\\
	\glt \lq We used to give help to reindeer herders when we were still children.' \parencite[Life story]{NikolaevaEtAl2019}

	\ex\label{exAppendixTundraNenets2}
	Context: A boy has been killed.\\
	\gll Lobeku-h n'eb'a ma: \lq\lq Ŋyamc'ida \textbf{təmna} səwa-q. Xad°r'ih yil'e-bt'e-° xorta-nakewᵒ."\\
	L-\textsc{gen} mother say \phantom{\lq\lq}flesh.\textsc{pl}.3\textsc{sg} still good-3\textsc{pl} of\_course.only live-\textsc{caus}-\textsc{mod}.\textsc{cvb} try-\textsc{evid}.1\textsc{sg}>\textsc{sg}\\
	\glt \lq Lobeku's mother said \lq\lq His muscles are still good. I might try and revive him."' \parencite[453]{Nikolaeva2014}
	
	\ex\label{exAppendixTundraNenets3}
	
	\gll Pidᵒkəyeda	\textbf{təmna} pəxᵒncʻə-yəᵒ.\\
	3\textsc{sg}:\textsc{pejorative}:\textsc{dim} still be\_without\_root-\textsc{pejorative}\\
	\glt \lq Poor thing, he is still small (lit. without roots).' (\cite[138]{Nikolaeva2014}; Irina Nikolaeva, p.c.)
\end{exe}
	
\subsubsection{Uses on the fringes of \lq{}still\rq{}}
\largerpage
\paragraph{Scalar contexts}\label{appendixTundraNenetsScalar}
\begin{itemize}
	\item At least the tokens in (\ref{exAppendixTundraNenetsDecrement1}, \ref{exAppendixTundraNenetsDecrement2}) involve scalar contexts in the form of decreases over time.
	\item Note that both examples stem from \textcite{Lehtisalo1956}, who uses a phonetic transcription. These have been adopted to the phonological transcription used in \textcite{Nikolaeva2014} with the help of Tapani Salminen.
	\item Also see \appref{appendixTundraNenetsNotYet} and (\ref{exAppendixtundraNenetsNotYet3}) on \lq still-\textsc{cmpr}' \lq not quite yet'.
\end{itemize}

\begin{exe}
	\ex\label{exAppendixTundraNenetsDecrement1}
	\gll Yiqm′aq xor°koəd° təmna s′əy°nə°, \textbf{təmna} oka, \textbf{təmna} s′id′a yal′a yamp°h tæw°ŋkuq.\\
	water/alcohol:\textsc{poss}.1\textsc{pl} keg.\textsc{abl} still flow\_freely.3\textsc{sg} still much still two day.\textsc{gen} long.\textsc{gen} reach.\textsc{refl}.3\textsc{sg}\\
	\glt \lq Der Branntwein fliesst noch stark aus dem Fass, es ist noch viel da, es genügt noch für zwei Tage. [The brandy is still flowing strong, there is still a lot, it’s still enough for two days.]\rq{ }(\cite[435]{Lehtisalo1956}, glosses added)

	\ex\label{exAppendixTundraNenetsDecrement2}
	\gll \textbf{Təmna} n′ax°r yúq n′encawey° xayi m′inc′°maq yax°-naq.\\
	still three ten stretch\_of\_river remain.3\textsc{sg} journey:\textsc{poss}.1\textsc{pl} place:\textsc{dat}.\textsc{sg}-\textsc{poss}.1\textsc{pl}\\
	\glt \lq{}Noch sind auf dem Fluss, den wir daherfahren, dreissig Strecken übrig. [There still remain thirty stretches of river on our journey.]\rq
	\\(\cite[319]{Lehtisalo1956}, glosses added)
\end{exe}	
	
	
\subsubsection{Uses related to other phasal polarity concepts}
\label{appendixTundraNenetsNotYet}
\paragraph{Not (ready) yet}
\begin{itemize}
	\item \textcite[458]{Lehtisalo1956}, \textcite[187]{Nikolaeva2014} and \textcite[624]{Tereshchenko2008}.
	\item This function occurs in the absence of a propositional complement and conveys the notion of \lq not ready yet' (\ref{exAppendixtundraNenetsNotYet1}, \ref{exAppendixtundraNenetsNotYet2}). In this use, \textit{təmna} can take tense and person-number marking (\ref{exAppendixtundraNenetsNotYet1}).
	\item There is also a collocation \textit{təmna}-\textit{rka} \lq still-\textsc{cmpr}' \lq not quite yet', see (\ref{exAppendixtundraNenetsNotYet3}). This is probably best understood as scalar, with comparative \mbox{-\textit{rka}} contributing a notion of \lq a bit' (see \cite[133–134]{Nikolaeva2014} on this function of the comparative suffix); that is, \lq still a bit', which is also the translation given by \textcite[458]{Lehtisalo1956}. In structural terms, however, this is the only other case in which \textit{təmna} can serve as a main predicate.
\end{itemize}
\begin{exe}
	\ex\label{exAppendixtundraNenetsNotYet1}
	\gll \textbf{Təmna}-dᵒm-cʼᵒ.\\
	still-1\textsc{sg}-\textsc{pst}\\
	\glt \lq I wasnʼt ready yet.ʼ \parencite[187]{Nikolaeva2014}
	
	\ex\label{exAppendixtundraNenetsNotYet2}
	\gll \textbf{Təmna} ŋǣ-wiᵒ.\\
	still \textsc{cop}-\textsc{evid}\\
	\glt \lq He wasnʼt ready, apparently.ʼ \parencite[187]{Nikolaeva2014}
		
	\ex\label{exAppendixtundraNenetsNotYet3}
	Context: Lobeku's mather has revived a dead man.\\
	\gll Lobeku-h n'eb'a ma: \lq\lq \textbf{Təmna}-\textbf{rka}. Ŋarka tu-h yad'o-xəna yūsʼidaᵒ-ya. Pūna xərᵒ-ta ŋamtʼo-ᵒ xaqmᵒ-cu-q."\\
		L.-\textsc{gen} mother say \phantom{\lq\lq}still-\textsc{cmpr} big fire-\textsc{gen} heat-\textsc{loc} lie-\textsc{juss} after \textsc{refl}-3\textsc{sg} sit-\textsc{mod}.\textsc{cvb} fall-\textsc{mod}-\textsc{refl}.3\textsc{sg}\\
	\glt \lq Lobeku's mother said \lq\lq It's not over yet. Let him lie in the heat of the big fire. Later he will sit up himself."\rq{ }\parencite[453–454]{Nikolaeva2014}
\end{exe}

\pagebreak
\subsubsection{Broadly adverbial temporal-aspectual functions}
\paragraph{Iterative via increment}
\label{appendixTundraNenetsIterativeIncrement}
\begin{itemize}
	\item There is only one example of this use in the data.
\end{itemize}
\begin{exe}
	\ex\
	 \gll Xərwa-bᵒ-ta \textbf{təmna} \textbf{ŋo}-\textbf{poyᵒ} me-wa-h tū-tᵒ-nakedᵒm.\\
	want-\textsc{cond}-3\textsc{sg} still one-\textsc{moderative} \textsc{cop}-\textsc{ipfv}.\textsc{nmlz}-\textsc{gen} come-\textsc{fut}-\textsc{mod}.1\textsc{sg}\\
	\glt \lq Perhaps I will come again if he wants.' \parencite[100]{Nikolaeva2014}\footnote{See \textcite[135]{Nikolaeva2014} on the \lq\lq moderative" suffix. Suffice it to say that \textit{ŋopoy\textsuperscript{o}} here serves as an event quantifier similar to \ili{Russian} \textit{raz} (Irina Nikolaeva, p.c.).}
\end{exe}


\subsubsection{Additive and related functions}
\paragraph{Additive}\label{appendixTundraNenetsAdditive}
	\begin{itemize}
	\item \textcite[186]{Nikolaeva2014} and \textcite[623]{Tereshchenko2008}.
	\item In (\ref{exAppendixTundraNenetsAlso3}) \textit{təmna} is optional, as the focus suffix \mbox{-\textit{xərt}} by itself contributes an additive meaning (see \cite[128–129]{Nikolaeva2014}).
	\item Syntax: as \lq another\rq{ }(i.e. additions of the same kind), \textit{təmna} stands NP\hyp internally.
	\end{itemize}
	
	\begin{exe}
		\ex
		\gll \textbf{Təmna} ŋoka yabto-m xada-wenᵒ.\\
		still many goose-\textsc{acc} kill-\textsc{evid}.2\textsc{sg}\\
		\glt \lq In addition, you have apparently killed many geese.' (Labanauskas 1995: 44, cited in \cite[186]{Nikolaeva2014})
		
		\ex
		\gll Nʼeᵒka-nʼi sʼitᵒ nʼada-wa-h tʼaxᵒmna mənʼᵒ \textbf{təmna} tedə-mtᵒ taᵒ-dəm-cʼᵒ.\\
		elder\_brother-\textsc{gen}.1\textsc{sg} 2\textsc{sg}.\textsc{acc} help-\textsc{ipfv}.\textsc{nmlz} beside 1\textsc{sg} still reindeer-\textsc{fut.poss}:\textsc{acc}:2\textsc{sg} give-1\textsc{sg}-\textsc{pst}\\
		\glt \lq In addition to my brother helping you, I also gave you a reindeer.' \parencite[371]{Nikolaeva2014}
 
 	\ex\label{exAppendixTundraNenetsAlso3}
	 \gll Tʼonʼa-m nʼəqmaᵒ, pūna \textbf{(təmna)} noxa-xərtᵒ-m nʼəqmaᵒ.\\
	fox-\textsc{acc} catch then \phantom{\lq}still polar\_fox-\textsc{foc}-\textsc{acc} catch\\
	\glt \lq He caught a fox and then a polar fox, too.' \parencite[129]{Nikolaeva2014}


			\ex
		\gll Tʹiki° sʹedar° Sʹay°tan° sʹedam nʹubʹeqŋa. Ŋaw°nanta nʹeney° wadaw°na Xæbʹidʹa sʹedasʹ°. … \textbf{Tǝmna} ŋob xæbʹidʹa sʹeda tǝnʹa°.\\
		it mountain:\textsc{poss}.2\textsc{sg} Shaytan mountain.\textsc{acc} call:2\textsc{sg} long\_ago:3\textsc{sg} true language.\textsc{prol} holy mountain:\textsc{pst}:3\textsc{sg} {} still one holy mountain exist:3\textsc{sg}\\
		\glt \lq This mountain is called Shaytan. In the past it was called Holy Mountain in the Nenets language. There is another holy mountain.'
\parencite[The holy mountains]{NikolaevaEtAl2019}
		
		\ex 
		\gll \textbf{Təmna} ngob ya-h xora-m xadaᵒ\\
		still one place-\textsc{gen}	 bull-\textsc{acc} kill\\
		\glt \lq He killed another mammoth.' (Labanauskas 1995: 18, cited in \cite[186]{Nikolaeva2014})
\end{exe}

\paragraph{Comparisons of inequality}\label{appendixTundraNenetsComparisons}
\begin{itemize}
	\item \textcite[458]{Lehtisalo1956} and \textcite[623]{Tereshchenko2008}.
	\item Tundra Nenets uses a from-comparative in which the standard of comparison is marked in the ablative case. Comparative marking on the predicate itself is optional \parencite[174–175]{Nikolaeva2014}.
	\item Addition of \textit{təmna} yields \lq even more\rq{}. This is attested with comparisons in the strict sense (\ref{exAppendixTundraNenetsComparative1}). It is also found with \lq\lq degree achievements"  \parencite{Dowty1979}, as in the combination of motion verb and adverbial in (\ref{exAppendixTundraNenetsComparative2}).
\end{itemize}
	
\begin{exe}
	\ex\label{exAppendixTundraNenetsComparative1}
	\glll \textbf{Тамна} саваркавна\\
	təmna səwa-mpoyᵒh\\
	still good-\textsc{moderative}\\
	\glt \lq Eщё лýчше [Even better]\rq{ }(\cite[623]{Tereshchenko2008}, glosses added)
		
\ex\label{exAppendixTundraNenetsComparative2}
	Context: A man is following his brother's tracks.\\
	\gll Xurka-r’i yil’ebc’əye xada-bə-sʼᵒtə-wiᵒ yaxᵒ-sʼətᵒ-wiᵒ \textbf{təmna} nʼerᵒnʼah xǣ-sʼᵒtə-wiᵒ.\\
	which-only wild\_reindeer:\textsc{acc}.\textsc{pl} kill-\textsc{ipfv}-\textsc{hab}-\textsc{evid} skin-\textsc{hab}-\textsc{evid} still forth go-\textsc{hab}-\textsc{evid}\\
	\glt \lq It seemed the older brother had been killing and skinning all sorts of wild reindeer, and then moving on even further.ʼ \parencite[440]{Nikolaeva2014} 
\end{exe}
\il{Nenets, Tundra|)}

\pagebreak
\section{Udihe (ude, udih1248)}\il{Udihe|(}
\label{appendixUdihe}
\subsection{Introductory remarks}
Apart from the grammars listed below, I consulted \citeauthor{NikolaevaEtAl2002}'s  (\citeyear{NikolaevaEtAl2002}, \citeyear{NikolaevaEtAl2003}, \citeyear{NikolaevaEtAl2019}) text collections. Note that the relevant Udihe marker, \mbox{\textit{xai}(\textit{si})}, is a very frequent item in texts, with the exact contribution of many instances remaining unclear. My categorisation of those uses not discussed in the descriptive materials is therefore but a first approximation.


\subsection{xai(si)}

\subsubsection{General information}
\begin{itemize}
	\item Form: two freely exchangeable variants, \textit{xai} and \textit{xaisi}.
	\item Wordhood: free morpheme.
	\item Etymology: from Mandarin Chinese \mbox{\textit{hái}(-\textit{shì}}).
\end{itemize}


\subsubsection{As a \lq{}still\rq{ }expression}
\begin{itemize}
	\item \textcite[20, 438–440]{NikolaevaTolskaya2001} and \textcite[78]{Schneider1936}.
	\item Specialisation: textual attestations like (\ref{exAppendixUdihe1}–\ref{exAppendixUdihe3}) give evidence that this expression conforms to my definition. For instance, in (\ref{exAppendixUdihe1}) \textit{xaisi} not only indicates that the dog continued to be young, but it also evokes an alternative scenario in which she is no longer young and would consequently be harnessed.
	\item Pragmaticity: compatible with both scenarios; (\ref{exAppendixUdihe3}) is a prime candidate for an unexpectedly late continuation.  
	\item Polarity sensitivity: inner negation yields \textsc{not yet}.
\end{itemize}
\begin{exe}
	\ex\label{exAppendixUdihe1}
	Context: The narrator, together with her family and dog, is preparing to end a hunting trip and go home.\\
	\gll In’ei-we e-u alu, pal’ma, mene beje-zi uŋkäla-u, pal’ma ic’a bi-se bueni \textbf{xaisi}.\\
	dog-\textsc{acc} \textsc{neg}-1\textsc{pl}.\textsc{excl} harness P. \textsc{refl} self-\textsc{ins} fetch-1\textsc{pl}.\textsc{excl} P. small \textsc{cop}-\textsc{pfv} 3\textsc{sg} still\\
	\glt \lq We didn’t harness our dog Palma, who was still young.\rq{ }\parencite[A tame roe cub named Wasya]{NikolaevaEtAl2019}
	\pagebreak
	\ex\label{exAppendixUdihe2}
	Context: An old humpback has ordered a fairy to cook and serve him food in a silver bowl. She refused, and he has conjured a torrential rain.
	\gll \lq\lq Wadi-wene-je ele, ñaŋmu-gi-wene-je!" \lq\lq Ñaŋmu," guŋ-ki-ni eitene, ñaŋmu-gi-e-ni. Ña ña eme-gi-e-ni boxo-s’o zugdi-tigi-ni. E-si-ni-de olokto, mene nixe-zeŋe-i \textbf{xai} nixe-ini eitene.\\
	\phantom{\lq\lq}stop-\textsc{caus}-\textsc{imp} enough clear-\textsc{iter}-\textsc{caus}-\textsc{imp} \phantom{\lq\lq{}}clear.\textsc{imp} say-\textsc{pst}-3\textsc{sg} now clear-\textsc{iter}-\textsc{pst}-3\textsc{sg} again again come-\textsc{iter}-\textsc{pst}-3\textsc{sg} humpback-\textsc{dim} house-\textsc{lat}-\textsc{poss}.3\textsc{sg} \textsc{neg}-\textsc{pst}-3\textsc{sg} cook \textsc{refl} make-\textsc{fut}.\textsc{ptcp}-\textsc{refl} still make-3\textsc{sg} now\\
	\glt \lq [The fairy said:] \lq\lq{}Stop it, enough, make (the sky) clear up again!\rq\rq{ }\lq\lq{}Let it be clear!\rq\rq{ }he said, and it cleared up. The humpback entered the house again. (The fairy) was not cooking, she kept doing her work.\rq{ }\parencite[35, 37]{NikolaevaEtAl2002} 

	\ex\label{exAppendixUdihe3}
	Context: About a hunting trip. Ilya was told to stay behind while his wife and son set out to hunt.\\
	\gll Ili-enti neŋi ilä \textbf{xaisi} ŋua-ini, ila-ma ili-enti neŋi zube neŋi t'ei ŋua-ini, bu ili-enti neŋi \lq\lq si ono ŋua-i?".\\
	three-\textsc{ord} day I. still sleep-3\textsc{sg} three-\textsc{acc} three-\textsc{ord} day two day whole sleep-3\textsc{sg} we three-\textsc{ord} day \phantom{\lq\lq}2\textsc{sg} how sleep-2\textsc{sg}\\
	\glt \lq On the third day Ilya was still sleeping, he had slept for two days. On the third day we came back and asked: \lq\lq Why are you sleeping?{\rq\rq}\rq{ }\parencite[A hunting trip]{NikolaevaEtAl2019}
\end{exe}

\subsubsection{Uses on the fringes of \lq{}still\rq{}}
\paragraph{Scalar contexts}\label{appendixUdiheScalar}
\begin{itemize}
	\item At minimum the tokens in (\ref{exAppendixUdiheDecrement1}, \ref{exAppendixUdiheDecrement2}) involve scalar contexts in the form of decreases over time.
\end{itemize}

\begin{exe}

	\ex\label{exAppendixUdiheDecrement1}
	Context: A couple had stored some dried meat, wrapped with birch tree bark.\\
	\gll Uta-digi uti=de \textbf{xai} wac'a esi-gi-e-ni, tuŋa=as adi=es xeke, teu uti talu-zi kapta-si-e-ni.\\
	this-\textsc{abl} this=\textsc{foc} still little become-\textsc{iter}-\textsc{pst}-3\textsc{sg} five=or how\_many=or bundle all this birch\_bark-\textsc{ins} wrap-\textsc{ipfv}-\textsc{pst}-3\textsc{sg}\\
	\glt \lq Only a little was left, five wraps or so, she had wrapped it all.' \parencite[An old woman and her tiger cub]{NikolaevaEtAl2019}

	\ex\label{exAppendixUdiheDecrement2}
	Context: A girl has found a dog’s head.\\
	\gll In’ei \textbf{xai} omo dili=mei xai inixi.\\
	dog still one head=only still alive\\
	\glt \lq Although there was only a head left from the dog, it was nevertheless alive.' \parencite[The fairy and the ten bald spirits]{NikolaevaEtAl2019}

\end{exe}


\paragraph{Sameness}\label{appendixUdiheSame}
\begin{itemize}
	\item \textcite[440–441]{NikolaevaTolskaya2001}.
	\item This function occurs in combination with distal anaphoric pronouns of the shape \mbox{(\textit{u})\textit{t}-} and expresses identity of the referent.	
	\item In many textual attestations, the referent of the anaphoric pronoun is a situation, which, in turn, either continues or is repeated with different participants, thereby providing a link to the phasal polarity, additive, and iterative functions of \mbox{\textit{xai}(\textit{si})}. For instance, a more literal translation of (\ref{exAppendixUdiheSame2}) is \lq the way he used to walk, he still walks that way' and  (\ref{exAppendixUdiheSame3}) could be paraphrased as \lq … that also happened to my wife'.
\end{itemize}

\begin{exe}
	\ex \textit{Ni: lä bazagele lali:nzi budei, zeude ei diga ilama neŋini, uti ni:we mene e:tigi mene ŋenezeŋei} <\textit{mene ŋenezeŋei}>\\
	\lq Many people in the taiga die from hunger. If a human doesn’t eat anything for three days, the tiger directs this person to where it has to go.'
	\exi{}
	\gll \textbf{xai} \textbf{uti} dogbo-ni uti xokto-tigi-ni ŋene-wen’e uti ni:-we xebu-ini.\\
still that night-\textsc{poss}.3\textsc{sg} this road-\textsc{lat}-\textsc{poss}.3\textsc{sg} go-\textsc{caus}.\textsc{pfv} this man-\textsc{acc} take-3\textsc{sg}\\
\glt \lq It will direct him to the road the very same night and lead this person.' \parencite[The tiger for Udihe people]{NikolaevaEtAl2019}

	\ex\label{exAppendixUdiheSame2}
	\gll Bi abuga-i ei zulie-ni xuliː bede \textbf{xaisi} ute bede xuliː-ni ba:-za ge-tigi-ni.\\
1\textsc{sg} father-1\textsc{sg} this before-3\textsc{sg} go:\textsc{prs}.\textsc{ptcp}:\textsc{ss} like still that like go-3\textsc{sg} place-\textsc{n} surface-\textsc{lat}-3\textsc{sg}\\
\glt \lq My father still goes hunting in the same way as he used to.\rq{ }\parencite[398–399]{NikolaevaTolskaya2001}


	\ex\label{exAppendixUdiheSame3}
	Context: A man's wife had run away to escape an evil spirit. Now he has met what appears to be another woman, and she has told him her story of escaping from an evil spirit. \\
	\gll  Merge (g)une-ini, \lq\lq Bi mamasa-i \textbf{xai} \textbf{ute} bi-s’e."\\
	hero say-3\textsc{sg} \phantom{\lq\lq}1\textsc{sg} wife-\textsc{poss}.1\textsc{sg} still that \textsc{cop}-\textsc{pfv}\\
	\glt \lq The man said \lq\lq The same happened to my wife.{\rq\rq}\rq{ }\parencite[76–77]{NikolaevaEtAl2003}
\end{exe}


\subsubsection{Broadly adverbial temporal-aspectual functions}
\paragraph{Iterative and restitutive}
\label{appendixUdiheIterative}
\begin{itemize}
	\item \textcite[439, 444–445]{NikolaevaTolskaya2001} and \textcite[78]{Schneider1936}.
	\item Iterative \mbox{\textit{xai}(\textit{si})} often occurs in conjunction with the additive/iterative marker \textit{ña} and/or the iterative verb suffix \mbox{-\textit{gi}}, as in (\ref{exAppendixUdiheIterative3}).
	\item	According to \textcite[439]{NikolaevaTolskaya2001}, iterative \mbox{\textit{xai}(\textit{si})} has \lq\lq the whole sentence in its scope" and involves \lq\lq the complete repetition of a situation". There are, however, several instances in which an event is repeated, but with a different object, as in (\ref{exAppendixUdiheIterative5}). In these cases, the line between the iterative use and the additive function (\appref{appendixUdiheAdditive}) becomes blurred.
\item Several instances in the data, including (\ref{exAppendixUdiheRestitutive1}, \ref{exAppendixUdiheRestitutive2}), involve a restitutive reading. It is noteworthy that the majority of them additionally feature the iterative suffix \mbox{-\textit{gi}}, which by itself can mark restitution; see (\ref{exAppendixUdiheRestitutive2}).
	\item According to \textcite[439]{NikolaevaTolskaya2001}, iterative/restitutive \mbox{\textit{xai}(\textit{si})} always goes along with a bounded viewpoint. Ex. (\ref{exAppendixUdiheIterative4}) indicates that this is a strong tendency rather than a fixed rule.
	\item Syntax: mobile, but typically close to the predicate, often preceding it.
\end{itemize}

\begin{exe}
	\ex\label{exAppendixUdiheIterative1}
	Context: A hero has been killed once, and nearly died on another occasion. A shaman warns him.\\
	\gll Merge ogoko si ogoko ŋene-mi ŋene-mi ogoko jeu=de maŋga-wa-ni \textbf{xai} b’a-zaŋa-i.\\
	hero \textsc{refrain} 2\textsc{sg} \textsc{refrain} go-\textsc{inf} go-\textsc{inf} \textsc{refrain} what=\textsc{foc} trouble-\textsc{acc}-\textsc{poss}.3\textsc{sg} still find-\textsc{fut}-2\textsc{sg}\\
	\glt \lq Hero, when you walk you will have troubles again.' \parencite[Sisam Zauli and the hero]{NikolaevaEtAl2019}

	\ex\label{exAppendixUdiheIterative2}
	Context: One sister from a group of seven has stopped talking and laughing. The other sisters have attempted several times to get her to speak, to no avail. Now they've made a big fire in order to play and laugh.\\
	\gll Toː ilaː-ti neŋu-ne-ni, ute dian'e-isiː-ni=de exi-ni=de \textbf{xai} jeu=de e-i diana, tuːtuː biː.\\
	fire kindle.\textsc{pst}-3\textsc{pl} younger\_sibling-\textsc{pl}-\textsc{poss}.3\textsc{sg} that say.\textsc{pfv}-\textsc{pfv}.\textsc{cvb}-3\textsc{sg}=\textsc{foc} older\_sister-3\textsc{sg}=\textsc{foc} still what=\textsc{foc} \textsc{neg}-\textsc{prs}.\textsc{ptcp} say silent.\textsc{iter} \textsc{cop}:\textsc{prs}.\textsc{ptcp}\\
	\glt \lq The younger sisters made fire but the elder sister didn’t say anything again, she was silent.' \parencite[The seven sisters]{NikolaevaEtAl2019}

	\ex\label{exAppendixUdiheIterative3}
	\gll Ge, \textbf{ña} biː-mie, \textbf{xai} geːnzi eː-\textbf{giː}-li.\\
	\textsc{interj} again \textsc{cop}-\textsc{inf} still pregnant become-\textsc{iter}-3\textsc{sg}\\
	\glt \lq Some time later she got pregnant again.' \parencite[106–107]{NikolaevaEtAl2003} 	
	
\ex\label{exAppendixUdiheIterative4}
	Context: A hero has knocked on a girl’s door. He has sung, urging her to open, but she didn’t respond. So he sat down. Now he has tried a second time.
	\exi{}\gll Opjat’ teu-teu, teu-teu bi-si-ni, \textbf{xai} te:-ni:, alasi-e-ni ñentile-i-we-ni.\\
	again all-all all-all \textsc{cop}-\textsc{pst}-3\textsc{sg} still sit.\textsc{pst}-3\textsc{sg} wait-\textsc{pst}-3\textsc{sg} open-\textsc{prs}.\textsc{ptcp}-\textsc{acc}-3\textsc{sg}\\
	\glt \lq She didn’t answer again, she kept quiet, so he was sitting again and waiting for her to open.\rq{ }\parencite[When Yegdige ate an evil spirit]{NikolaevaEtAl2019}

	\ex\label{exAppendixUdiheIterative5}
	Context: A tame tiger has previously brought an old woman a dead roe.\\
	\gll Ñä bi-mie ei ba:-ixi ŋen'e \textbf{xai} nakta-wa w'a:-si: gazi-e-ni, imo: xai cu:-ñieñie=de.\\
	again \textsc{cop}-\textsc{inf} \textsc{evid} outside-\textsc{lat} go.\textsc{pfv} still boar-\textsc{acc} kill.\textsc{pfv}-\textsc{pst}.\textsc{ptcp}.\textsc{ss} bring-\textsc{pst}-3\textsc{sg} fat still \textsc{ideoph}:through=\textsc{foc}\\
	\glt \lq After some time she went out again: and again the tiger had brought a boar, a very fat one.\rq{ }\parencite[An old woman and her tiger cub]{NikolaevaEtAl2019}
	
	\ex Context: A girl is married to a crow. He leaves  the house in the morning and comes back late in the evening.\label{exAppendixUdiheRestitutive1}\\
 \gll Ŋua-gi-xi-ni, neme-gi-si:-ni, \textbf{xai} anči-gde ŋene:-ni.\\
	sleep-\textsc{iter}-\textsc{cvb}.\textsc{pfv}-3\textsc{sg}  cover-\textsc{iter}-\textsc{pfv}.\textsc{cvb}-3\textsc{sg}  still no-\textsc{foc}  go.\textsc{pst}-3\textsc{sg}\\
	\glt \lq When he fell asleep, she covered him, then he went away again.\rq{ }\parencite[169, 171]{NikolaevaEtAl2003} 	
	
	\ex\label{exAppendixUdiheRestitutive2}
	Context: A boy and a girl were up a tree. They have climbed down to steal some food.\\
\gll  Moː-mo uniŋa-zi eːxi ai-le-ni tulo-ndoː-ti gampa-wa \textbf{xaisi} tukti-gi-e-ti moː-tigi.\\
	tree-\textsc{adj} spoon-\textsc{acc} frog butt-\textsc{loc}-\textsc{poss}.3\textsc{sg} smear-\textsc{semelfactive}-\textsc{pst}-3\textsc{pl} thick\_porridge-\textsc{acc} still climb-\textsc{iter}-\textsc{pst}-3\textsc{pl} tree-\textsc{lat}\\
	\glt \lq They smeared porridge on the frog’s behind with a wooden spoon and climbed the tree again.'  \parencite[169, 171]{NikolaevaEtAl2002}
\end{exe}

\subsubsection{Additive and related functions}
\paragraph{Additive}\label{appendixUdiheAdditive}
\begin{itemize}
	\item \textcite[438]{NikolaevaTolskaya2001} and \textcite[78]{Schneider1936}. 
	\item This function extends to equative comparisons, as in (\ref{exAppendixUdiheAdditiveEquative}).
\end{itemize}
\begin{exe}
	\ex 
	Context: The opening of a narrative.\\
	\gll Bi-mie 	omo ... bi-mie, Kanda mafa  bi-si-ni. 	Mamaka-ŋi … s(i)te-n(i)-de \textbf{xai} bi-si-ni\\
	\textsc{cop}-\textsc{inf} one {} \textsc{cop}-\textsc{inf}  K. old\_man \textsc{cop}-\textsc{pst}-3\textsc{sg}  old\_woman-\textsc{poss} {} child-\textsc{poss}.3\textsc{sg}-\textsc{foc} still \textsc{cop}-\textsc{pst}-3\textsc{sg}\\
	\glt \lq (Once upon a time) there lived an old man, Kanda. He had a wife, and they also had a son.' \parencite[58, 63]{NikolaevaEtAl2002}
	
	\ex
	Context: Three brothers have promised a fairy that one of them will marry her. Now they try to decide which one.\\
	\gll Sagdi aga-ti gune:-ni \lq\lq{}Bi e-zeŋe-i zawa, bi sagdin-dima bi-mi." Uti gagda-ni xegi-le bi: neŋu-ni \textbf{xaisi} e-ini ča:la.\\
	old brother-\textsc{poss}.3\textsc{pl} say.\textsc{pst}-3\textsc{sg} \phantom{\lq\lq}1\textsc{sg} \textsc{neg}-\textsc{fut}-1\textsc{sg} take 1\textsc{sg} old-\textsc{adj} \textsc{cop}-1\textsc{sg} this second-3\textsc{sg} under-\textsc{loc} \textsc{cop}:\textsc{prs}.\textsc{hab} younger\_sibling-\textsc{poss}.3\textsc{sg} still \textsc{neg}-3\textsc{sg} want\\
	\glt \lq The second brother didn’t want her either.' \parencite[The alder tree girl]{NikolaevaEtAl2019}

	\ex Context: About the events from a certain story.\\
	\gll Uta-la armija nemica wali-ŋie-ni, uteli maŋmu-ziga-tigi xauntasi-e-mi ute. Mafasa maŋmu-ziga \textbf{xai} sa:-du, teu sa-iti uta-wa ceze bi-si-ni, ceze, \textbf{xai} sa-i. maŋmu mafasa {staryj} utempi-utempi teluŋu sa-i-we je-i-we, \textbf{xai} bueti sa-iti, ceze-we {govorit}, ceze bi-si-ni.\\
	that-\textsc{loc} army German fight-\textsc{ipfv}.\textsc{cvb}-3\textsc{sg} then Nanai-\textsc{pl}-\textsc{lat} ask-\textsc{pst}-1\textsc{sg} that old\_man Nanai-\textsc{pl} still know:\textsc{pst}.\textsc{ptcp}-3\textsc{pl} all know-3\textsc{sg} that-\textsc{acc} true \textsc{cop}-\textsc{pst}-3\textsc{sg} true still know-\textsc{prs}.\textsc{ptcp} Nanai old\_man {old} such.\textsc{redupl} tale know-2\textsc{sg}-\textsc{acc} do\_what-2\textsc{sg}-\textsc{acc} still they know-3\textsc{pl} true-\textsc{acc} {says} true \textsc{cop}-\textsc{pst}-3\textsc{sg}\\
	\glt \lq During the war time I asked the Nanai people about that. The old Nanais also know that, everybody knows that it’s true, they know too. An old Nanai knew exactly the same story, they also know it and say that it is true.' \parencite[An old woman and her tiger cub]{NikolaevaEtAl2019}
	
	\ex \label{exAppendixUdiheAdditiveEquative}
	\gll Ei moː \textbf{xaisi} gugda-laŋki-ni tauxi moː-digi.\\
	this tree still high-\textsc{adj}-\textsc{poss}.3\textsc{sg} that tree-\textsc{abl}\\
	\glt \lq This tree is as high as that one.' \parencite[187]{NikolaevaTolskaya2001}
\end{exe}
\largerpage[2]
\paragraph{Comparisons of inequality}\label{appendixUdiheComparisons}
\begin{itemize}
	\item The two attestations in (\ref{exAppendixUdiheComparison1}, \ref{exAppendixUdiheComparison2}) indicate that \mbox{\textit{xai}(\textit{si})} can be used in comparisons of inequality. Note that Udihe makes use of a locational comparative structure \parencite{Stassen2013}: the standard of comparison, if explicitly mentioned, is marked with the ablative case, and there is no overt marking of the comparison on the predicate \parencite[398–399]{NikolaevaTolskaya2001}.
	\item The use of \mbox{\textit{xai}(\textit{si})} appears to add the scalar additive notion of \lq even\rq{}.
\end{itemize}
\begin{exe}
	\ex\label{exAppendixUdiheComparison1}
	\gll Ñʼädiga waː-la bi nuan-digi \textbf{xai} waː-la bi-mi.\\
	N. kill-\textsc{nmlz} 1\textsc{sg} 3\textsc{sg}-\textsc{abl} still kill-\textsc{nmlz} \textsc{cop}-1\textsc{sg}\\
	\glt \lq Nadiga is lucky (in hunting) and I am more lucky than him.\rq{ }\parencite[440]{NikolaevaTolskaya2001}

	\ex\label{exAppendixUdiheComparison2}
	Context: Moose and Frog are competing in a race. They have been running and Moose, unaware of Frog outwitting him, thinks that Frog is ahead of him.\\
	\gll \textbf{Xaisi} beje-zi bele tukä-li-nie, jeː sokco-mie, bugdi kölomie=de tukä-nie.\\
	still fast-\textsc{ins} fast run-\textsc{inch}-3\textsc{sg} antlers sticking\_out? legs ?=\textsc{foc} run-3\textsc{sg}\\
	\glt \lq The Moose ran [even] faster, his antlers and legs stick out.\rq{ }\parencite[The moose and the frog]{NikolaevaEtAl2019}\footnote{The glosses for \textit{sokcomie} and \textit{kölomie} are marked as uncertain, as their exact meaning is unclear. As \textcite{NikolaevaEtAl2019} explain, they serve to mock the Moose.}
\end{exe} 

\paragraph{Switch/contrastive topic}\label{appendixUdiheTopic}
\begin{itemize}
	\item A few tokens of \mbox{\textit{xai}(\textit{si})} in texts appear to involve a switch in topic/contrastive topics. For instance, in (\ref{exAppendixUdiheTopic1}), \textit{xaisi} appears to indicate a switch in topic from the tiger's role in culture to a description of its physical appearance.
	\item While this function is not described in \textcite{NikolaevaTolskaya2001}, cross-linguistically this is a common functional extension of markers with additive functions \parencite{Forker2016} 
\end{itemize}

\begin{exe}

	\ex\label{exAppendixUdiheTopic1}
	 Context: About tigers. They are considered to be God’s animal.\\
	\gll Ute bi:, iŋakta-na-ni \textbf{xaisi} oño-ni, kegdeje-ni iŋakta-na-ni bagdi:-ni zülie-li.\\
	this \textsc{cop}:\textsc{prs}.\textsc{ptcp} fur-\textsc{designative}-\textsc{poss}.3\textsc{sg} still draw.\textsc{pst}-3\textsc{sg} striped-3\textsc{sg} fur-\textsc{designative}-\textsc{poss}.3\textsc{sg} grow-3\textsc{sg} striped-\textsc{adj}\\
	\glt \lq That’s how it is. Its fur all grows motley and stripy.'\footnote{See \textcite[126–127]{NikolaevaTolskaya2001} on the \lq\lq designative" or \lq\lq destinative" case marker \mbox{-\textit{na}}.}  \parencite[The tiger for the Udihe people]{NikolaevaEtAl2019}

	
	\ex Context: A fairy has been given a task by an old woman. She has completed it and is coming back.\\
	\gll Läta tukäma-gi: gune, \textbf{xai} mamaka isi: je-we=ke nixe-mi.\\
	fast run-\textsc{iter}:\textsc{ptcp}.\textsc{nmlz} say still old\_woman look:\textsc{prs}.\textsc{ptcp} what-\textsc{acc}=\textsc{indef} do-\textsc{inf}\\
	\glt \lq The girl ran back quickly, and the old woman was watching what she was doing.'  \parencite[The fairy and the ten bald spirits]{NikolaevaEtAl2019}

\end{exe}

\paragraph{Conjunctional adverb}\label{appendixUdiheConjunctional}
\largerpage
\begin{itemize}
	\item There are several textual tokens of \mbox{\textit{xai}(\textit{si})} that appear to involve a conjunctional uses, an interpretation that is supported by their clause-initial position; see \textcite{Forker2016} on conjunctional extensions of markers with additive functions.
\end{itemize}

\begin{exe}
	\ex Context: A man has gotten hold of a horse that defecates bread.\\
	\gll Uta gazi-e-k, ba:-la xeke:-k, \textbf{xai} mamasa-ti(gi) diaŋ-ki-ni, \lq\lq{}Ge, mamasa, si uti mui xegie-le soktu-je c’aligi mei seudine-we sagdi-zie-de mui xegie-le-n(i).\\
	that bring-\textsc{pst}-\textsc{intens} place-\textsc{loc} tie.\textsc{pst}-\textsc{intens} still wife-\textsc{lat} say-\textsc{pst}-3\textsc{sg} \phantom{\lq\lq}\textsc{interj} wife 2\textsc{sg} that horse under-\textsc{loc} spread-\textsc{imp} white only kerchief-\textsc{acc} big-\textsc{ins}-\textsc{foc} horse under-\textsc{loc}-\textsc{poss}.3\textsc{sg}\\
	\glt \lq He brought it, tethered it outside, and [then] said to his wife, \lq\lq Hey, wife, spread out a large white kerchief under the horse."' \parencite[92–93, 99]{NikolaevaEtAl2003}

	\ex
	\gll Adi-me=ke aŋa-ni bi:, \textbf{xai} mamaka bele ele bude-li'e, ele maŋga.\\
	how\_many-\textsc{acc}=\textsc{indef} year-3\textsc{sg} \textsc{cop}.\textsc{ptcp} still old\_woman even\_more soon die-\textsc{inch}.\textsc{pfv} soon hard\\
	\glt Several years passed and the old woman was about to die, she felt very ill.\rq{ }\parencite[An old woman and her tiger cub]{NikolaevaEtAl2019}
	
	\ex
	\gll Si \textup{[}si\textup{]} jeu=de jai-ni ede-ili tuŋči:, \textbf{xai} geje dieli-zeŋe-fi \textup{[}geje dieli-zeŋe-fi\textup{]}.\\
	2\textsc{sg} \phantom{[}2\textsc{sg} what=\textsc{foc} noise-3\textsc{sg} start-3\textsc{sg} jump\_on.\textsc{imp} still together fly-\textsc{fut}-1\textsc{pl} \phantom{[}together fly-\textsc{fut}-1\textsc{pl}\\
	\glt \lq If some noise starts, jump on me and we’ll fly together.\rq{ }\parencite[Yegdige in a silk gown]{NikolaevaEtAl2019}
\end{exe}

\paragraph{Specificational adverb}\label{appendixUdiheSpecificational}
\begin{itemize}
	\item Closely related to the conjunctional uses, there are a few instances of clause-initial \mbox{\textit{xai}(\textit{si})}, where the marker has wide scope and introduces additional, specifying information similar to \ili{German} \textit{und zwar} (see \cite{OneaVolodina2011} on this and related markers).
\end{itemize}

\begin{exe}
	\ex
	\gll Omo gugda gugda we: xo:-lo-ni, uti we:-tigi go:, \textbf{xai} zube ila neŋi-ni i:ne-zeŋe-i, site-i meŋde eme-i, utala-da aŋi xu:lu bagdi:-ni.\\
	one high high hill top-\textsc{loc}-\textsc{poss}.3\textsc{sg} that hill-\textsc{lat} far still two three day-\textsc{poss}.3\textsc{sg} reach-\textsc{fut}-\textsc{imp} child-\textsc{ref} with come-\textsc{imp} that-\textsc{loc}-\textsc{foc} \textsc{indef} gourd live-3\textsc{sg}\\ 
	\glt \lq … a very high mountain, which is far away, [namely] two or three days of walking. A gourd grows there.' \parencite[84, 89]{NikolaevaEtAl2003}

\ex
	Context: A hero has shot at an iron bird.\\
	\gll Tada-ni=dele piktige ŋen’e, \textbf{xai} zokpo-ni culi=de.\\
	arrow-\textsc{poss}.3\textsc{sg}=\textsc{foc} right go.\textsc{pfv} still throat-\textsc{poss}.3\textsc{sg} directly=\textsc{foc}\\
	\glt \lq The arrow had struck it exactly, [that is] straight into the throat.' \parencite[Sisam Zauli and the hero]{NikolaevaEtAl2019}
\end{exe}



\subsubsection{Broadly modal and interactional functions}
\paragraph{Concessive apodoses}
\label{appendixUdiheConcessiveConsequent}
\begin{itemize}
	\item \textcite[440]{NikolaevaTolskaya2001}.
	\item Note the bounded viewpoint in (\ref{exAppendixUdiheConcessive1}) and the negated predicate in (\ref{exAppendixUdiheConcessive2}).
		\item Several tokens of the concessive function, including (\ref{exAppendixUdiheConcessive6}), involve alternative concessive conditionals. Example (\ref{exAppendixUdiheConcessive7}) suggests that this might be additionally motivated by the additive function of \mbox{\textit{xai}(\textit{si})}.
\end{itemize}
	
\begin{exe}
	\ex\label{exAppendixUdiheConcessive1}
	\gll Nua-ni e-zeŋe ŋene bi-si-ni \textbf{xai}(\textbf{si}) ŋeneː-ni.\\
	he-3\textsc{sg} \textsc{neg}-\textsc{fut}.\textsc{ptcp} go \textsc{cop}-\textsc{pst}.\textsc{ptcp}-3\textsc{sg} still go.\textsc{pst}-3\textsc{sg}\\
	\glt \lq Although he didn't have to go, he still went.\rq{ }\parencite[440]{NikolaevaTolskaya2001}

	\ex\label{exAppendixUdiheConcessive2}
	Context: A man has encountered a tiger. It has bitten him.\\
	\gll Iteme-l'e, bueni \textbf{xai} e-si-ni tiŋme, ili-me bi:, mäusa-ni=de anči, i:-ni=de anči, kusige-i ga:-gi-mie, kusige-le:-ni tig(ra) aŋi uti kuti-we.\\
	bite-\textsc{pfv} 3\textsc{sg} still \textsc{neg}-\textsc{pst}-3\textsc{sg} fall stand-\textsc{inf} \textsc{cop}.\textsc{prs}.\textsc{ptcp} gun-\textsc{poss}.3\textsc{sg}=\textsc{foc} no what-\textsc{poss}.3\textsc{sg}=\textsc{foc} no knife-\textsc{refl} take-\textsc{iter}-\textsc{inf} knife-\textsc{pst}-3\textsc{sg} tiger \textsc{indef} this tiger-\textsc{acc}\\
	\glt \lq It bit him, but [nonetheless] the old man didn’t fall, he was standing like that, he didn’t have a gun, nothing, but he took out a knife and stabbed the tiger with it.' 
	\parencite[An old woman and her tiger cub]{NikolaevaEtAl2019}
	
	\ex\label{exAppendixUdiheConcessive3}
	Context: A snake has moved in with an old couple. It demands they find a human wife for it.\\
	\gll
	Mafasa gun'e \lq\lq I-du gele-i si mamasa-na-i? Kuliga, si kuliga, ni: ono ŋene-ze sin-tigi kuliga-tigi? Ni mafala-za sin-tigi?" – \lq\lq \textbf{Xaisi} ŋene, e-li: ge:-ne sina-wa wa-zaŋa-i.”\\
	old\_man say.\textsc{pfv} \phantom{\lq\lq}what-\textsc{dat} ask-2\textsc{sg} 2\textsc{sg} wife-\textsc{designative}-\textsc{refl} snake 2\textsc{sg} snake human how go-\textsc{sbjv} 2\textsc{sg}-\textsc{lat} snake-\textsc{lat} who marry-\textsc{sbjv} 2\textsc{sg}-\textsc{lat} {} \phantom{\lq\lq}still go.\textsc{imp} \textsc{neg}-\textsc{cond}.2\textsc{sg} bring-\textsc{dir} 2\textsc{sg}-\textsc{acc} kill-\textsc{fut}-1\textsc{sg}\\
	\glt \lq The old man said: \lq\lq Why are you asking for a wife for yourself? You are a snake, how can a human marry you, a snake? Who will marry you?" – \lq\lq All the same, go, if you don’t bring me a wife, I’ll kill you.{\rq\rq}\rq{ }\parencite[Zabdala, an extraordinary snake]{NikolaevaEtAl2019}
	
	\ex\label{exAppendixUdiheConcessive6}
	Context: A man has eaten from tiger’s kill, but left two legs. He says to the tiger:\\
	\gll Ei zuːbe bugdi-we sin-du ne-gi-e-mi, ali=de goː=ko daː=ka bi-mi \textbf{xai} o-lo eme-giː, a-wa dig’a si.\\
	this two leg-\textsc{acc} 2\textsc{sg}-\textsc{dat} put-\textsc{iter}-\textsc{pst}-1\textsc{sg} when=\textsc{foc} far=\textsc{indef} near=\textsc{indef} \textsc{cop}-\textsc{inf} still here-\textsc{loc} come-\textsc{iter}.\textsc{imp} that-\textsc{acc} eat.\textsc{imp} 2\textsc{sg}\\
	\glt \lq I put these two legs aside for you. Whether you walk far away or close to here, [it doesn't matter, BP] come here and eat them.\rq{ }\parencite[The tiger for Udihe people]{NikolaevaEtAl2019}	
	
	\ex\label{exAppendixUdiheConcessive7}
	Context: Once upon a time there was a Chinese tsar. He buried people alive.\\
	\gll Ñuŋu-za seː i:ne-wene-mie bude-isiː-ni, buge-ini, e-siː-ni, bude, \textbf{xai} buge-ini.\\
	six-ten year come-\textsc{caus}-\textsc{inf} die-\textsc{pfv}.\textsc{cvb}-3\textsc{sg} bury-3\textsc{sg} \textsc{neg}-\textsc{pst}-3\textsc{sg} die still bury-3\textsc{sg}\\
	\glt \lq When a person became sixty years old, he buried him, no matter whether he was dead or not. (lit: … he died, he buried him, he did not die, he still=also buried him).' \parencite[18–20]{NikolaevaEtAl2003}
\end{exe}\il{Udihe|)}
