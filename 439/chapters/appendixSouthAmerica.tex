\chapter{South America}
\label{appendixSouthAmerica}
\largerpage
\section{Cavineña (cav, cavi1250)}
\il{Cavineña|(}
\subsection{Introductory remarks}
My understanding of the Cavineña data has profited greatly from discussion with Antoine Guillaume, who also helped with several glosses. Apart from descriptive materials, I consulted the text collections by \citeauthor{CampLiccardi1973} (\citeyear{CampLiccardi1973}, \citeyear{CampLiccardi1989}) and \citeauthor{TaboMayo1977} (\citeyear{TaboMayo1977}, \citeyear{TaboMayo1978}).

\subsection{=jari}
\subsubsection{General information}
\begin{itemize}
	\item Wordhood: bound morpheme (enclitic).
	\item Syntax: attaches to verbal predicates, copula complements and secondary predicates.
	\item Etymology: unknown, but strikingly similar to \textit{jara} \lq lie (\textsc{intr})'.
\end{itemize}

\subsubsection{As a \lq{}still\rq{ }expression}
\begin{itemize}
	\item \textcite[67, 246]{CampLiccardi1989} and \textcite[660–662]{Guillaume2008}.
	\item Specialisation: the description by \citeauthor{Guillaume2008} meets my definition.
	\item Pragmaticity: compatible with both scenarios (tentative conclusion).  Ex. (\ref{exAppendixCavinena2}) is a good candidate for the unexpectedly late scenario. It is unclear if the late scenario requires additional marking, e.g. emphatic \mbox{=\textit{di}} and/or focus \mbox{=\textit{dya}}.
	\item Polarity sensitivity: inner negation yields \textsc{not yet}. This can be used in temporal clauses to signal precedence  \parencite[278]{CampLiccardi1989}.
	\item Further note: ex. (\ref{exAppendixCavinena4}) illustrates the use in a jussive context, which may alternatively be subsumed under the \lq first, for now\rq{ }use (\appref{appendixCavinenaFirst}).
\end{itemize}

\begin{exe}
	\ex\label{exAppendixCavinena1}
	\gll Tumepatya=tuna-ja=tu ani-kware=\textbf{jari} ududu ewikani=ju.\\
	at\_that\_time=3\textsc{pl}=\textsc{dat}=3\textsc{sg} sit-\textsc{rem}.\textsc{pst}=still feather nose=\textsc{loc}\\
	\glt \lq At that time (when I visited the Pacahuara people) they were still wearing feathers in their noses. (but they don’t anymore)ʼ. \parencite[590]{Guillaume2008}
	
	\ex\label{exAppendixCavinena2}
	 \gll Masa=dya=di nei mara=\textbf{jari}.\\
	seemingly=\textsc{foc}=\textsc{emph} rain time=still\\
	\glt \lq (How come there is still so much water on the paths!) It is as if it were still the rainy season.' \parencite[636]{Guillaume2008}

	\ex\label{exAppendixCavinena3}
Context: Victor has been hit by a horse. The wound has healed.\\
\gll Victor=ja ani-ya=\textbf{jari} baji-da=que cahuayu=cuana=tsehue.\\
V.=\textsc{dat}|\textsc{gen} sit-\textsc{ipfv}=still scared-\textsc{adj}=\textsc{rel} horse=\textsc{pl}=\textsc{assoc}\\
\glt \lq[P]ero Victor todavía tiene miedo a los caballos. [But Victor is still afraid of horses].\rq{ }(\cite[6, 9]{CampLiccardi1973}; glosses added)

	\ex\label{exAppendixCavinena4}
	\gll E-ra=tu ani-sha-ya=\textbf{jari}.\\
	1\textsc{sg}-\textsc{erg}=3\textsc{sg} sit-\textsc{caus}-\textsc{ipfv}=still\\
	\glt \lq I will retain him (lit. make him sit) some more time.\rq{ }\parencite[289]{Guillaume2008}
\end{exe}

\subsubsection{Uses on the fringes of \lq{}still\rq{}}

\paragraph{Scalar contexts}\label{appendixCavinenaScalar}
\begin{itemize}
	\item One attestation in the data can be counted as involving a scalar context (a decrease on a scale of aliveness)
\end{itemize}

\begin{exe}
	\ex
	\gll Chacha\sim{}chacha=piji=\textbf{jari} ju-kware tume=ke ura.\\ 
	alive\sim{}\textsc{redupl}=\textsc{dim}=still \textsc{cop}-\textsc{rem}.\textsc{pst} there=\textsc{lnk} hour\\
\glt \lq It (the deer that I had shot) was still a little bit alive at that time (so I had to shoot at it again).\rq{ }\parencite[388]{Guillaume2008}

\end{exe}


\subsubsection{Broadly adverbial temporal-aspectual functions}
\paragraph{First, for now, for a while}\label{appendixCavinenaFirst}
\begin{itemize}
	\item \textcite[662–663]{Guillaume2008}.
	\item A \lq for while, for now\rq{ }reading obtains in directive speech acts (\ref{exAppendixCavinenaForAWhile1}, \ref{exAppendixCavinenaForAWhile2}) and with future time settings, as in (\ref{exAppendixCavinenaForAWhile3}). It also obtains in performative \lq that's all\rq{ }(\ref{exAppendixCavinenaForAWhile4}).
\end{itemize}

\begin{exe}
	\ex\label{exAppendixCavinenaForAWhile1}
	\gll Pisu-kwe=\textbf{jari}=shana juye=ekatse! Pa-kanajara ekatse!\\
	untie-\textsc{imp}.\textsc{sg}=still=\textsc{pity} ox=\textsc{du} \textsc{juss}-rest 3\textsc{du}\\
	\glt \lq Untie the oxen (\textsc{du}) for a while, the poor animals (who are suffering so much pulling the cart)! Let them  (\textsc{du}) rest!' \parencite[655]{Guillaume2008}
		
	\ex\label{exAppendixCavinenaForAWhile2}
	\gll Aama! Radio=ju pa-kwa=\textbf{jari}!\\
	\textsc{neg}.\textsc{exist} radio=\textsc{loc} \textsc{hort}.\textsc{sg}-go=still\\
	\glt \lq No (I can't come right now)! I'm going to the radio (house) for a while! (but don't worry, I'll come back later.)' \parencite[662]{Guillaume2008}
	
	\ex\label{exAppendixCavinenaForAWhile3}
	\gll Tasi ju-ya=\textbf{jari} metajudya=ishu.\\
	drive\_taxi:1\textsc{sg} \textsc{cop}-\textsc{ipfv}=still tomorrow=\textsc{purp}\\
	\glt \lq I will drive my (motorcycle) taxi a little bit for (me to have money) tomorrow.' \parencite[662]{Guillaume2008}

	\ex\label{exAppendixCavinenaForAWhile4}
	\gll Jadya=kamadya=\textbf{jari}!\\
	thus=only=still\\
	\glt \lq That's all for now! (but there will be another meeting or story)' (\cite[662]{Guillaume2008}; Antoine Guillaume, p.c.)
\end{exe}
\il{Cavineña|)}

\section{Culina (cul, culi1244)}\il{Culina|(}
\label{appendixCulina}
\subsection{Introductory remarks}
I am indebted to Jim and Cindy Boyer for discussing Culina data with me, and to Stefan Dienst for helping with glosses. Apart from descriptive materials, I searched \citeauthor{Boyer1990}'s (\citeyear{Boyer1990}, \citeyear{Boyer2001}) and \citeauthor{Boyer1999} (\citeyear{Boyer1999})'s text collections.  Culina appears to have two \textsc{still} expressions: \textit{paha}/\textit{pahi} (only in the Juruá variety?) and \mbox{-\textit{kha}}. Only for the latter do the data indicate additional functions. Lastly, note that \textcite{Tiss2004} describes the verb suffixes -\textit{i}/-\textit{ni} as imperfective, whereas \textcite{Dienst2014} treats them as declarative markers. In the glosses, I follow \citeauthor{Dienst2014}'s (\citeyear{Dienst2014}) analysis.

\subsection{-kha}

\subsubsection{General information}
\begin{itemize}
	\item Form: also transcribed as \mbox{-\textit{kka}} and  \mbox{-\textit{cca}}.
	\item Wordhood: bound morpheme (verb suffix).
\end{itemize}

\subsubsection{As a \lq{}still\rq{ }expression}\label{appendixCulinaStill}
\begin{itemize}
	\sloppy
	\item \textcite{Boyer2020}, \textcite[126]{Dienst2014} and \textcite[183]{Tiss2004}.
	\item Specialisation: the descriptions of this marker, especially \textcite[183]{Tiss2004}, show that it conforms to my definition (see \appref{appendixKulinaFirst} below for further discussion). This is reflected in examples like (\ref{exAppendixKulina1}–\ref{exAppendixKulina3}). For instance, in (\ref{exAppendixKulina1}) \mbox{-\textit{kha}} (transcribed as \mbox{-\textit{cca}}), to all apparences, not only construes the use of earthen pots and shells as a continuation of a prior habit, but also points to its later discontinuation. According to \textcite{Tiss2004}, \mbox{-\textit{kha}} not only marks persistance and evokes a possible discontinuation scenario, but the latter tends to be represented by a follow-up situation \lq\lq antes de algo novo acontecer, a situação em questão ainda continua por um período limitado" [before something else happens, the situation in question still continues for a limited time] \parencite[183]{Tiss2004}. This is reminiscent of the \lq still only\rq{ }meaning of Mate \textit{bayu} (\appref{appendixMateqStill})
		
	\item Pragmaticity: the data allow no conclusions.
	\item Polarity sensitivity: inner negation yields \textsc{not yet}
	\item Further note: \mbox{-\textit{kha}} can be combined with another \textsc{still} expression \textit{paha}/\textit{pahi}.
\end{itemize}
\begin{exe}
	\ex\label{exAppendixKulina1}
	Context: The opening of a short expository text about food preparation.\\
	\gll Maittaccadsama madija idi-deni=pa po-cca-deni bani ppeppe=pi dsipa=dsa ppe ppe qui-na-\textbf{cca}-hi na-de. Idi-deni=pa po-cca-deni coidse=pi daro=dsa dse dse qui-na-\textbf{cca}-hi na-de.\\
	formerly Kulina ancestor-\textsc{non}.\textsc{sg}=\textsc{top}.\textsc{m} 3\textsc{m}-\textsc{assoc}-\textsc{non}.\textsc{sg} game pot=\textsc{top}.\textsc{f} earthen\_pot=\textsc{loc} cook cook \textsc{non}.\textsc{sg}-\textsc{aux}-still-\textsc{decl}.\textsc{m} \textsc{aux}-\textsc{pst} ancestor-\textsc{non}.\textsc{sg}=\textsc{top}.\textsc{m} 3\textsc{m}-\textsc{assoc}-\textsc{non}.\textsc{sg} spoon=\textsc{top}.\textsc{f} shellfish=\textsc{ins} drink drink \textsc{non}.\textsc{sg}-\textsc{aux}-still-\textsc{decl}.\textsc{m} \textsc{aux}-\textsc{pst}\\
	\glt \lq Hace muchos años nuestros antepasados cocinaban su carne en ollas de cerámica. Ellos todavía usaban conchas en vez de cucharas. [A long time ago, our ancestors [still] used to cook their meat in earthen pots. They still used shells as spoons.]\rq{ }(\cite[49]{Boyer1999}, glosses added)

	\ex\label{exAppendixKulina2}
	\begin{xlist}
			\exi{A:} \textit{Ami wadani?}\\
			\lq A mãe está dormindo? [Is mother sleeping?]\rq{}
			\exi{B:}
			\gll Nowe ra-ni hapi na-\textbf{kha}-ni.\\
			3:not\_be \textsc{aux}-\textsc{decl}.\textsc{f} bathe 3:\textsc{aux}-still-\textsc{decl}.\textsc{f}\\
			\glt \lq Não (está), ela ainda está tomando banho (antes de logo aparecer). [Is mother sleeping? -- No she isn’t, she's still bathing (before showing up after).]\rq{ }\parencite[184]{Tiss2004}
		\end{xlist}
	
	\ex\label{exAppendixKulina3}
	\gll Zohe papeo wa wa na-\textbf{kha}-wi.\\
	Z. paper call call 3:\textsc{aux}-still-\textsc{decl}.\textsc{m}\\
	\glt \lq Zohe ainda está lendo (anted de logo fazer outras coisas). [Zohe is still reading (before later doing something else).]\rq{ }\parencite[184]{Tiss2004} 
\end{exe}


\subsubsection{Broadly adverbial temporal-aspectual functions}
\paragraph{First, for now, for a while}
\label{appendixKulinaFirst}
\begin{itemize}
	\item \textcite[183–184]{Tiss2004} and \textcite[34]{MonserratSilva1986}.
	\item As \textcite[184]{Tiss2004} discusses, without \mbox{-\textit{kha}} (\ref{exAppendixKulinaForNow1}) would be understood as involving a permanent transfer of possession.
	\item A token like (\ref{exAppendixCulinaFirst2}) is hard to judge: Neither are there contextual indications that persistence plays a role, nor is there an established past topic time. It thus seems that  \mbox{-\textit{kha}} (i.e. <\textit{cca}>) serves to highlight the polarity contrast and mark precedence (\lq we used to live here back then/for a while then, but no longer do'). 
\end{itemize}
\begin{exe}
	\ex\label{exAppendixCulinaFirst1}
	 Context: After a hunting trip, a father and his child have reached their home estate.\\
	\gll Nadsa=pi: majonana passa o-ca-na-\textbf{cca}-na---  o-na-jaro. Tahide ti-cca-ni-po-na--- o-na-jaro.\\
	then=\textsc{top}.\textsc{f} sugar\_cane chew \textsc{subj}.1\textsc{sg}-\textsc{obj}-\textsc{aux}-still-\textsc{imm}.\textsc{fut}  \textsc{subj}.1\textsc{sg}-say-\textsc{narr}.\textsc{f}  ahead \textsc{subj}.2-go--back-first-\textsc{imm}.\textsc{fut} \textsc{subj}.1\textsc{sg}-say-\textsc{narr}.\textsc{f}\\
	\glt [The child said] \lq Voy a chupar caña de azucar todavia – le dije. – Tu puedes ir de frente a la casa.' [I said to him \lq\lq I'll first (lit. still) chew some sugar cane, you can go straight home.\rq\rq]\rq{ }(\cite[92–93]{Boyer1990}; glosses added)
	
	\ex\label{exAppendixKulinaForNow1}
	\gll O-kha koshiro tia-za da o-to-na-\textbf{kha}-ni towi.\\
	1\textsc{sg}-\textsc{poss} knife 2-\textsc{loc} give 1\textsc{sg}-\textsc{it}-\textsc{aux}-still-\textsc{decl}.\textsc{f} \textsc{fut}\\
	\glt \lq Emprestarei minha faca para você. [I'll lend you my knife.]\rq{ }\parencite[185]{Tiss2004}

	\ex\label{exAppendixKulinaForNow2}
	 Context: The ending of a story.\\
	\gll Najaro huapima. Epeji-\textbf{cca}-ni.\\
	that.\textsc{f} all end-still-\textsc{decl}.\textsc{f} \\
	\glt \lq Esto es todo por ahora. He terminado. [That's all for now. I'm done.]\rq{ }(\cite[67]{AgnewAdams1992}; glosses added)	
	
	\ex\label{exAppendixCulinaFirst2}
	Context: The story's protagonists have come by a certain location.\\
	\gll Nadsa=pi: Aji=dsa i-que-je-na-\textbf{cca}-de.\\
	then=\textsc{top}.\textsc{f} \textsc{dem}.\textsc{f}=\textsc{loc} \textsc{subj}:1\textsc{non}.\textsc{sg}-\textsc{non}.\textsc{sg}-\textsc{cop}-\textsc{non}.\textsc{sg}-still-\textsc{pst}\\
	\glt \lq Yo dije: aquí es donde vivíamos antes. [I said: this is where we used to live.]\rq{ }(\cite[88–89]{Boyer1990}; glosses added)
\end{exe}\il{Culina|)}


\section{Huallaga Huánuco Quechua (qub, hual1241)}\il{Quechua, Huallaga-Huánuco|(}
\label{appendixQuechua}

\subsection{-raq}
\subsubsection{General information}

\begin{itemize}
	\item Form: also transcribed as \mbox{-\textit{raj}} and as \mbox{-\textit{rä}} (with the diaeresis signalling vowel length), the latter occuring in contexts that trigger ellision of the coda segment. In addition, there is a free variant \mbox{-\textit{ran}}, usually via assimilation from a preceding suffix ending in /n/ .
	\item Wordhood: a bound morpheme that can occur on various parts of speech.
	\item Syntax: \mbox{-\textit{raq}} attaches to its associate; in its function as a phasal polarity marker the host is typically the main predicate.
	\item Etymology: unclear, but \textcite{vanBaar1997} suggests \lq for now, first\rq{ }(\appref{appendixQuechuaFirstMain}) as the original function.
\end{itemize}

\subsubsection{As a \lq{}still\rq{ }expression}
\begin{itemize}
	\item \textcite[75–76, 343, 386–392]{Weber1989} and \textcite[640]{WeberEtAl2008}.
	\item Specialisation: \citeauthor{Weber1989}'s description, especially (\citeyear[391]{Weber1989}) meets my definition. \citeauthor{Weber1989} furthermore notes the incompatibility with inalterable states. Its specialisation also becomes evident in prototypical contexts like (\ref{exAppendixQuechua1}, \ref{exAppendixQuechua2}).
	\item Pragmaticity: appears to be compatible with both scenarios.
	\item Polarity sensitivity: inner negation yields \textsc{not yet}. In temporal clauses, this commonly serves as a signal of precedence. 
	\item Further note: ex. (\ref{exAppendixQuechua4}) illustrates the use in a hortative; this could alternatively be subsumed under the \lq first, for now\rq{ }use (\appref{appendixQuechuaFirstMain}).
\end{itemize}
\begin{exe}
	\ex\label{exAppendixQuechua1}
	\gll Gam-pa wamra-yque-ga tacsha-lla-\textbf{rä}-mi. Noga-pa-ga jatu-n-na.\\
	2\textsc{sg}-\textsc{gen} child-\textsc{poss}.2-\textsc{top} small-only-still-\textsc{evid} 1\textsc{sg}-\textsc{gen}-\textsc{top} big-3-already\\
		\glt \lq Tu hijo todavía es pequeño. El mío ya es grande / Your child is still small. Mine is now big.\rq{ }(\cite[522]{WeberEtAl2008}, glosses added)

	\ex\label{exAppendixQuechua2}
	\gll Mana-mi wañu-n-rä-chu. Cawa-yca-n-\textbf{rä}-mi.\\
	\textsc{neg}-\textsc{evid} die-3-still-\textsc{neg} live-\textsc{ipfv}-3-still-\textsc{evid}\\
	\glt \lq Todavía no muere. Todavía está vivo. / He has not died yet. He is still alive.\rq{ }(\cite[131]{WeberEtAl2008}, glosses added)
	
	\ex
	\gll Waquin wamra-ga quimsa wata-yoj ca-shpa-n-pis \lq\lq{}Chichi-na-ycä mama\rq\rq{} ni-n-rä-mi.\\
	others child-\textsc{top} three year-have \textsc{cop}-\textsc{subord}.\textsc{ss}-3-also \phantom{\lq\lq}breast-want-\textsc{ipfv}.1\textsc{sg} mother say-3-still-\textsc{evid}\\
	\glt \lq Otros niños, aunque ya tienen tres años dicen todavía: \lq\lq{}Querio teta, mamá.\rq\rq{ }/ Some children, even though they are three years old, are still saying \lq\lq{}Mother, I want to nurse.{\rq\rq{}}\rq{ }(\cite[148]{WeberEtAl2008}, glosses added)

	\ex\label{exAppendixQuechua4}
	\gll Ama\textup{(}-raq\textup{)} aywa-y-raq-chu. Ka-ku-yka:-shun-\textbf{raq}.\\
	\textsc{neg}-still go-\textsc{fut}.2-still-\textsc{neg} \textsc{cop}-\textsc{refl}-\textsc{ipfv}-\textsc{fut}:1\textsc{pl}.\textsc{incl}-still\\
	\glt \lq Donʼt go yet. Letʼs be yet (awhile here together).' \parencite[388]{Weber1989}
\end{exe}


\subsubsection{Uses on the fringes of \lq{}still\rq{}}
\paragraph{Scalar contexts}\label{appendixQuechuaScalar}
\begin{itemize}
	\item -\textit{raq} is attested in scalar contexts. These include decreases along some scale (\ref{exAppendixQuechuaScalar1}, \ref{exAppendixQuechuaScalar2}), as well as limited limited increases, as in (\ref{exAppendixQuechuaScalar3}). 
\end{itemize}

\begin{exe}
	\ex\label{exAppendixQuechuaScalar1}
	\gll Inca	uysha-pa	ishcay	wagra-n-ta		roguri-ptë ishcay-\textbf{rä}-mi 	wagra-n 	quëra-sha.\\
	inca sheep-\textsc{gen} two horn-\textsc{poss}.3-\textsc{obj} cut\_through-\textsc{adv}.\textsc{ds}:1>3 two-still-\textsc{evid} horn-\textsc{poss}.3 remain-\textsc{ant}.3\\
	\glt \lq Cuando le corté dos de los cuernos del carnero inca, todavía le quedaron dos. /
When I cut off two of the inca ramʼs horns there were still two.\rq{ }(\cite[252]{WeberEtAl2008}, glosses added)

	\ex\label{exAppendixQuechuaScalar2}
	\gll Manana sinchi	tamya-r-pis tamya-ga	poga-pa-yca-n-\textbf{rä}-mi.\\
	no\_longer strong rain-\textsc{adv}.\textsc{ss}-also rain-\textsc{top} drizzle-\textsc{appl}-\textsc{ipfv}-3-still-\textsc{evid}\\
	\glt \lq Aunque ya no llueve fuerte todaviá están cayendo gotas. / Although it is no longer raining hard, a few drops are still falling.\rq{ 
	\\}(\cite[420]{WeberEtAl2008}, glosses added)
	
	\ex\label{exAppendixQuechuaScalar3}
	\gll Iti wamra mana-rä-mi puri-n-rä-chu. Läta-cu-yca-n-lla-\textbf{rä}-mi.\\
	suckling child \textsc{neg}-still-\textsc{evid} walk-3-still-\textsc{neg} crawl-\textsc{refl}-\textsc{ipfv}-3-only-still-\textsc{evid}\\
	\glt \lq El niño todavía no camina. Todavía está gateando. The infant does not yet walk. It is still [only] crawling.\rq{ }(\cite[298]{WeberEtAl2008}, glosses added)
\end{exe}


\subsubsection{Broadly adverbial temporal-aspectual functions}
\paragraph{First, for now}
\label{appendixQuechuaFirstMain}
\begin{itemize}
	\item \textcite[388]{Weber1989}.
	\item \Textcite[90–91]{vanBaar1997}, based on descriptions of several Quechuan languages, suggests that this is the  diachronically original meaning of \mbox{-\textit{raq}}. This would also provide a direct bridge to the temporal restrictive function (\appref{AppendixQuechuaFirstSubordinate}).
\end{itemize}

\begin{exe}
	\ex
	\gll Ese nuwal-wan tiñi-rku-r atapa-nchiː-\textbf{raq}.\\
	that walnut-\textsc{com} dye-then-\textsc{adv} form\_skein-1\textsc{pl}.\textsc{incl}-still\\
	\glt \lq Dying it with that walnut, we first form a skein.' \parencite[388]{Weber1989}
	
	\ex
	\gll Maː pay-ta-\textbf{raq} tapu-y.\\
	\textsc{hort} 3-\textsc{obj}-still ask-\textsc{fut}.2\\
	\glt \lq Ask him first (before you ask me/do it).' \parencite[390]{Weber1989}

	\ex
	\gll Hina ka-shun-\textbf{ran} ishkay kimsa killa-kama.\\
	like\_that \textsc{cop}-\textsc{fut}:1\textsc{pl}.\textsc{incl}-still two three month-\textsc{lim}\\
	\glt \lq Letʼs be like that until two or three months have passed (and then we can go back to doing it).' \parencite[388]{Weber1989}
\end{exe}

\subsubsection{Temporal connectives and frame setters}
\paragraph{Temporal restrictive}
\label{AppendixQuechuaFirstSubordinate}
\begin{itemize}
	\item \textcite[386–387]{Weber1989}.
	\item In this function, the focus of \mbox{-\textit{raq}} can be an adverbial (\ref{exAppendixQuechuaFirstSubordinate1}, \ref{exAppendixQuechuaFirstSubordinate2}) or a temporal clause in the same function (\ref{exAppendixQuechuaFirstSubordinate4}, \ref{exAppendixQuechuaFirstSubordinate5}) . 
	\item This uses also underlies several fixed expressions, including \parencite[65–66, 390]{Weber1989}:
	\begin{itemize}

		\item \textit{Qepa}-\textit{n}-\textit{ta}-\textit{raq} \lq back-\textsc{poss}.3-\textsc{adv}-still' > \lq later on'
		\item \textit{Naka}-\textit{y}-\textit{raq} \lq take\_long\_time-\textsc{inf}-\textsc{adv}-still' > \lq yet a while later'
		\item \textit{Chay}-\textit{chaw}-\textit{raq} \lq that-\textsc{loc}-still' > \lq not until that point in timeʼ
		\item \textit{Chay}-\textit{raq}-\textit{shi} \lq that-still-\textsc{evid}' > \lq right then, it was not until then'
		\item \textit{Chay}-\textit{lla}-\textit{raq} \lq that-only-still' > \lq momentarily, very recently'
	\end{itemize}
	\item Note that \textit{chay}-\textit{raq}-\textit{shi} \lq that-still-\textsc{evid}'  \lq right then, it was not until then' in (\ref{exAppendixQuechuaFirstSubordinate6}) can be understood as \lq finally', thus bleeding into \textsc{already}-territory.
\end{itemize}

\begin{exe}
	\ex\label{exAppendixQuechuaFirstSubordinate1}
	\gll Ishkay killa-\textbf{raq} haru-shka:.\\
	two month-still step-\textsc{ant}.1\\
	\glt \lq It was two months before I stepped on it (a disjointed ankle).\rq{ }\parencite[387]{Weber1989}

	\ex\label{exAppendixQuechuaFirstSubordinate2}
	\gll Allcha-ka-sha killa-ta-\textbf{raq}.\\
	fix-\textsc{pass}-\textsc{ant}.3 month-\textsc{obj}-still\\
	\glt \lq He got well after a month (and not before).'\footnote{See \textcite[183]{Weber1989} on the object suffix \mbox{-\textit{ta}} as a marker of time elapsing.} \parencite[387]{Weber1989}

	\ex\label{exAppendixQuechuaFirstSubordinate4}
	\gll Ñaka-y-ta-\textbf{raq} tari-sha.\\
	take\_long\_time-\textsc{inf}-\textsc{adv}-still find-\textsc{ant}.3\\
	\glt \lq He found it only after he had searched a good while.\rq{ }\parencite[132]{Weber1993}
	
	\ex\label{exAppendixQuechuaFirstSubordinate5}
	\gll 	… dansa-n arpista bigulista tuka-pa-pti-n-\textbf{raq}. Mana tuka-pti-n-qa mana dansa-n-chu.\\
	{} dance-3 harpist violinist play-\textsc{ben}-\textsc{subord}.\textsc{adv}-\textsc{poss}.3-still \textsc{neg} play-\textsc{adv}-3-\textsc{top} \textsc{neg} dance-3-\textsc{neg}\\
	\glt \lq … they dance when (and not until) the harpist and violinist play for them (and not before). If they do not play, they do not dance.'  \parencite[387]{Weber1989}

	\ex\label{exAppendixQuechuaFirstSubordinate6}
	Context: An old man has been suspecting that his wife cheats on him.\\
	\gll …ishka-n qaqa-sha. Awkin-na-shi ollqo-yka-n ruru-lla-pa-qa. Chay-\textbf{raq}-shi awkin ollqo-yka-n.\\
	\phantom{…}two-3 be\_parallel-\textsc{ptcp} old\_man-already-\textsc{evid} be\_angry-\textsc{pfv}-3 inside-just-\textsc{gen}-\textsc{top} that-still-\textsc{evid} old\_man be\_angry-\textsc{pfv}-3\\
	\glt \lq The two of them are together (the old man’s wife and her lover). At that, the old man becomes angry, but just inside. Finally, only then, did the old man become angry.\rq{ }\parencite[379]{Weber1989}
\end{exe}
\pagebreak
\subsubsection{Restrictive (non-temporal)}
\paragraph{Scalar restrictive}
\label{appendixQuechuaExtraordinary}
\begin{itemize}
	\item \textcite[389]{Weber1989}.
	\item In this function \mbox{-\textit{raq}} indicates that situation in question \lq\lq was an extreme measure, i.e., not carried out to an ordinary degree or applied to the ordinary ob­jects" \parencite[389]{Weber1989}. That is, the negated context propositions are all lower-ranking (lower degrees, etc.).
	\item Cases like (\ref{exAppendixQuechuaExtreme3}, \ref{exAppendixQuechuaExtreme4}) appear to constitute a bridge between this function and the \lq not until' function (\appref{AppendixQuechuaFirstSubordinate}): they involve both time spans (waiting for the decision, possibly preceded by consulting with lower authorities / working for a considerable amount of time) as well as degrees  (the highest authority, working to such a degree that it induces suffering).
\end{itemize}

\begin{exe}
	\ex
	\gll Sasa-ta-\textbf{raq}-shi hichqa-yku-n awkin-qa.\\
	difficult-\textsc{obj}-still-\textsc{evid} strike-impact-3 old\_man-\textsc{top}\\
	\glt \lq With difficulty, the old man strikes the match (\textit{sasataraqshi} implies that it was only with considerable difficulty that the old man was able to manage striking a match).'  \parencite[389]{Weber1989}
	
	\ex\label{exAppendixQuechuaExtreme2}
	Context: A husband has been informed that his wife is having an affair with another man.\\
	\gll Lulla-ku-nki-chari \textbf{kiki:}-\textbf{raq}-\textbf{mi} warmi-:-ta watqa-yku-shaq. Rika-yku-shaq.\\
	lie-\textsc{refl}-2-surely self:\textsc{poss}.1-still-\textsc{evid} wife-\textsc{poss}.1-\textsc{obj} spy-\textsc{pfv}-\textsc{fut}.1 see-\textsc{pfv}-\textsc{fut}.1\\
	\glt \lq You must be lying! I myself will spy on my wife. I will see. (implies that nothing will determine the truth short of the speakerʼs spying on his wife)' \parencite[132, 389]{Weber1989}
	
	\ex\label{exAppendixQuechuaExtreme3}
	\gll Hatun awturidaa-chaw-\textbf{raq}-mi musya-ka:-shun kapital-chaw-\textbf{raq}-mi.\\
	big authority-\textsc{loc}-still-\textsc{evid} know-\textsc{pass}-\textsc{fut}:1\textsc{pl}.\textsc{incl} capital-\textsc{loc}-still-\textsc{evid}\\
	\glt \lq We will find out only at the higher authority, in the capital (and we will not find out any sooner).' \parencite[389]{Weber1989}
	\largerpage[2]
	\ex\label{exAppendixQuechuaExtreme4}
	\gll … wayu-chi-na-yki-paq ñaka-r-\textbf{raq}-mi aru-nki.\\
	{} produce-\textsc{caus}-\textsc{subord}.\textsc{nmlz}-\textsc{poss}.2-\textsc{purp} suffer-\textsc{adv}-still-\textsc{evid} work-2\\
	\glt \lq … in order to cause it to produce you will have to work even to the point of suffering (i.e. you will not be able to make it produce without working to the point of suffering).' \parencite[389]{Weber1989}
\end{exe}


\subsubsection{Broadly modal and interactional uses}
\paragraph{Concessive(-like) apodoses}\label{appendixQuechuaConcessive}
\begin{itemize}
	\item \textcite[389–390]{Weber1989}.
	\item With first person subjects plus the future tense \mbox{-\textit{raq}} serves to indicate that the speaker maintains a certain plan. This often goes together with a concessive notion that the plan persists in spite of what the addressee may assume.
\end{itemize}

\begin{exe}
	\ex
	\gll Ura-shaq-\textbf{raq}.\\
	do-\textsc{fut}.1-still\\
	\glt \lq I will yet do it / I still intend to do it (despite your thinking that I wonʼt).' \parencite[389]{Weber1989}
	
	\ex
	\gll Ichan-qa yapyaː-ta-\textbf{raq} usha-ku-ri-shaq.\\
	perhaps-\textsc{top} field:\textsc{poss}.1-\textsc{obj}-still finish-\textsc{refl}-\textsc{fut}.1\\
	\glt \lq Perhaps I will finish (plowing) my field (before you take one of my oxen).' \parencite[389]{Weber1989}
\end{exe}

\paragraph{Dubitative: -\textit{chu}-\textit{raq}}
\label{appendixQuechuaDubitative}
\begin{itemize}
	\item \textcite[326–327]{Weber1989}.
	\item Form: this function obtains in combination with the question marker \mbox{-\textit{chu}}.
	\item In this use, \mbox{-\textit{chu}-\textit{raq}} serves to express doubt about the host constituent.
	\item The relation to other uses of \mbox{-\textit{raq}} remains unclear; it might be motivated by cases like (\ref{exAppendixQuechuaExtreme2}) above.
\end{itemize}

\begin{exe}
	\ex
	\gll Qam-pa surti-ki noqa-pa surtiː-naw-chu-raq o mas piyur-\textbf{chu}-\textbf{raq}.\\
	2-\textsc{gen} fate-\textsc{poss}.2 1-\textsc{gen} fate:\textsc{poss}.1-like-\textsc{q}-still or more worse-\textsc{q}-still\\
	\glt \lq Is your fate perhaps like mine, or perhaps worse?' \parencite[327]{Weber1989}
	
	\ex
	\gll Kanan hunaq tamya-nqa-\textbf{chu}-\textbf{raq}?\\
	now day rain-\textsc{fut}.3-\textsc{q}-still\\
	\glt \lq Will it perhaps rain today?' \parencite[327]{Weber1989}
\end{exe}

\paragraph{Rhetorical question, despair: \textit{kanan}-\textit{raq}-\textit{chi} \textit{kannan}-\textit{lla}-\textit{qa}}
\begin{itemize}
	\item \textcite[446–447]{Weber1989}.
	\item Form: this function obtains in a fixed construction \textit{kanan}-\textit{raq}-\textit{chi} \textit{kanan}-\textit{lla}-\textit{qa} \lq now-still-\textsc{evid} now-only-\textsc{top} \lq Oh dear, what now!'. 
	\item This use might be related to the scalar restrictive function (\ref{appendixQuechuaExtraordinary}).
\end{itemize}

\begin{exe}
	\ex
	\gll Kanan-raq-chi kanan-lla-qa ima-ta-raq ni-man-qa duyñu-n?\\
	now-still-\textsc{evid} now-only-\textsc{top} what-\textsc{obj}-\textsc{q} say-1>3-\textsc{top} owner-3\\
	\glt \lq Oh dear! What now! What will its owner say to me?\rq{ }\parencite[447]{Weber1989}
\end{exe}


\paragraph{\textit{chay}-\textit{raq} N-\textit{lla}-\textit{qa}: \lq it sure is N!'}
\begin{itemize}
	\item Discused by \textcite[446]{Weber1989}.
	\item Form: this function obtains in an construction \textit{chay}-\textit{raq} N-\textit{lla}-\textit{qa} \lq that-still N-only-\textsc{top}' where N may be a nominalised verb.
	\item \textcite[446]{Weber1989} translates this as \lq it sure is … !'.
	\item This use is probably be related to the scalar restrictive function (\appref{appendixQuechuaExtraordinary}).
\end{itemize}

\begin{exe}
	\ex
	\gll Chay-\textbf{raq} aka-y-lla-q.\\
	that-still be\_hot-\textsc{inf}-only-\textsc{top}\\
	\glt \lq It sure is hot!' \parencite[446]{Weber1989}
\end{exe}
\il{Quechua, Huallaga-Huánuco|)}

\section{Movima (mzp, movi1243)}\il{Movima|(}
\subsection{Introductory remarks}
I am indebted to Katharina Haude, for sharing unpublished Movima data with me, for lengthy discussions of them, and for helping with many tricky glosses. In addition to descriptive materials, I consulted Haude's unpublished corpus \parencite{MovimaCorpus}.

\subsection{diːra(n)}
\subsubsection{General information}
\begin{itemize}
	\item Form: free variation between \textit{diːran} and \textit{diːra}.
	\item Wordhood: free morpheme.
	\item Syntax: relatively mobile, but typically occurring before the predicate.
\end{itemize}

\subsubsection{As a \lq{}still\rq{ }expression}
\begin{itemize}
	\item \textcite[520–521]{Haude2006} and \textcite[31, 82]{JudyJudy1962}.
	\item Specialisation: examples like (\ref{exAppendixMovima1}–\ref{exAppendixMovima3}) give evidence that this adverb conforms to my definition. For instance in (\ref{exAppendixMovima1}) the mother's hearing abilities are both linked to a prior state and contrasted with the opposite situation at a later time.
	\item Pragmaticity: compatible with both scenarios. Whether the unexpectedly late scenario requires additional marking is to be determined in future research.
	\item Polarity sensitivity: inner negation yields \textsc{not yet}.
\end{itemize}

\begin{exe}
	\ex\label{exAppendixMovima1}
	Context: Recollecting a conversation between the narrator's mother and grandfather. \\
	\gll N-asko \textbf{dira} pawaneɬ-wa=ʼne kaj sin̍lototo=ʼne mereʼ.\\
\textsc{obl}-\textsc{pro} still hear-\textsc{nmlz}=3\textsc{f} \textsc{neg} deaf.\textsc{nmlz}=3\textsc{f} big\\
	\glt \lq That was when she [narrator's mother] could still hear, she wasn't very deaf then.\rq{ }\parencite{MovimaCorpus}

	\ex\label{exAppendixMovima2} Context: From an autobiographical account.\\
	\gll Choń inła la' n-os \textbf{di:ra} tenapanłe:-wa os iloni:-wa joy-cheł nosde:, ban jayna ney jayna kał joy-wa, tenapanłe:-wa as iloni:-wa\\
\textsc{hab} 1\textsc{sg} before \textsc{obl}-\textsc{n} still be\_able-\textsc{nmzl} \textsc{n} walk-\textsc{nmlz} go-\textsc{refl} there but already here already \textsc{neg}:1 go-\textsc{nmlz} be\_able-\textsc{nmlz} \textsc{n} walk-\textsc{nmlz}\\
	\glt \lq Y yo, cuando todavía podía andar, iba ahí. Pero ya no voy. No puedo caminar. [And I, when I could still walk, I used to go there. But I don't go anymore. I can't walk (anymore).]\rq{ }\parencite{MovimaCorpus}
	
	\ex\label{exAppendixMovima3}
	\gll Diːra ay moː to:ro:-ɬe, \textbf{diːra} ay tochiʼ.\\
still \textsc{prox}.\textsc{n} \textsc{neg} bull-\textsc{nmlz}.\textsc{n} still \textsc{prox}.\textsc{n} small\\
	\glt \lq This one isnʼt a bull yet, itʼs still small.' \parencite[473]{Haude2006}
\end{exe}

\subsubsection{Restrictive (non-temporal)}
\paragraph{Scalar restrictive \lq at least'}
\label{appendixMovimaAtLeast}
\begin{itemize}
	\item \textcite[521]{Haude2006}.
	\item In this function \mbox{\textit{diraː}(\textit{n})} signals a non-maximal value on a scale of subjective evaluation, i.e. it it involves negated universal quantification, comparable to English evaluative \textit{at least} (see \cite{Kay1992}; \cite{Gast2012}). For example, in (\ref{exAppendixMovimaAtLeast1}) it conveys a positive evaluation of the fact that the deceased received a Christian funeral. While more desirable alternatives are available in the common ground (e.g. a less gruesome death), there is a also a less preferable one (no dignified burial). Ex. (\ref{exAppendixMovimaAtLeast4}) shows that this reading survives under negation.	
	\item In (\ref{exAppendixMovimaAtLeast5}) the context plus the combination with the \textsc{already}-marker \textit{jayna} gives rise to a temporal restrictive reading (\lq no earlier then that, then at least/at last'). 
\end{itemize}
\largerpage
\begin{exe}
	\ex\label{exAppendixMovimaAtLeast1}
	Context: A woman who had the magic power of transforming into a jaguar has been caught and her jaguar hide has been burned, leading to her gruesome death.\\
	\gll Jayna \textbf{diran} oso' wulako='is.\\
	already still \textsc{dem}.\textsc{n}.\textsc{pst} bury=3\textsc{pl}\\
	\glt \lq Ya siquiera la enterraron (le dieron una sepultura cristiana porque era humana). [They at least buried her (they had a Christian funeral for her, because she was human).]\rq{ }\parencite{MovimaCorpus}
	
	\ex\label{exAppendixMovimaAtLeast2}
	Context: About how a place has changed over the course of the years.\\
	\gll Jayna ney \textbf{di:ra}, \textbf{di:ra} ay jayna de:deye, jayna ay dede:ye is to:wa neyru jema'a.\\
	already here still still \textsc{prox}.\textsc{n} already see.\textsc{impr} already here see.\textsc{impr} \textsc{pl} path here also\\
	\glt \lq Ya aquí siquiera, siquiera ya se ve esto, ya siquiera se ve un camino aquí. [Now here you can at least see … now you can at least see a path.]\rq{ }\parencite{MovimaCorpus}
		
	\ex\label{exAppendixMovimaAtLeast3}
	\gll Bo a’ko \textbf{di:ra} jey-na koi ka:na.\\
	because \textsc{n} still depart-\textsc{dr} \textsc{n} eat.\textsc{dr}\\
	\glt \lq Because that is where my food comes from [if not from anywhere else].' \parencite[521]{Haude2006}	
	
	\ex\label{exAppendixMovimaAtLeast4}
	Context: The speaker's sister put yuca flour into a broken container, and it has gone bad.\\
	\gll Kas \textbf{di:ran} rey ja' chok-a-kwanɬe-wa=sne n-os di'  lora-nkwa … kabo rey di' n-is mariːko di' plastiko.\\
	\textsc{neg} still again just cover-\textsc{dr}-mouth-\textsc{nmlz}=\textsc{f} \textsc{obl}-\textsc{n} perhaps leaf-\textsc{n} {} or  again perhaps \textsc{obl}-\textsc{pl} bag \textsc{rel} plastic\\
	\glt \lq  She did not even cover it at least with a leaf or a plastic bag.\rq{ }\parencite[530]{Haude2006}
	
	\ex\label{exAppendixMovimaAtLeast5}
	Context: The speaker has been bothered by ants. She has smeared kerosine on a tree trunk\\
	\gll \textbf{Di:ra} jayna jay'asłe … jayłe jayna \textbf{di:ra} iso'o.\\
	still already flee {} then already still \textsc{dem}.\textsc{pst}.\textsc{pl}\\
	\glt \lq Un poco se fueron … ya por lo menos. [They fled … then eventually, at least.]\rq{ }\parencite{MovimaCorpus}
\end{exe}
\il{Movima|)}

\section{Southern Lengua (enx, sout2989)}\il{Lengua, Southern|(}
\label{appendixEnxetSur}

\subsection{Introductory remarks}
I am indebted to John Elliot for sharing and discussing Southern Lengua (a.k.a Enxet Sur) data with me, and for helping with tricky glosses.

\subsection{makham}
\subsubsection{General information}
\begin{itemize}
	\item Wordhood: free morpheme.
\end{itemize}

\subsubsection{As a \lq{}still\rq{ }expression}
\begin{itemize}
	\item \textcite[483–486]{Elliot2021}.
	\item Specialisation: examples like (\ref{exAppendixEnxetSur1}–\ref{exAppendixEnxetSur3}) give a fairly good indication that this marker conforms to my definition. For instance, in (\ref{exAppendixEnxetSur1}) \textit{makham} not only signals that the men were young, but it also highlights that they were not yet grown-up (i.e. no longer youth).
	\item Pragmaticity: compatible with both scenarios (tentative conclusion); it is unclear if the unexpectedly late scenario receives additional marking.
	\item Polarity sensitivity: inner negation yields \textsc{not yet}.
		\item Syntax: fixed position, following the predicate, preceding dependent nominals.
	\item Further note: ex. (\ref{exAppendixEnxetSur4}) illustrates an instance that borders on a \lq same\rq{ }use.
\end{itemize}

\begin{exe}
	\ex\label{exAppendixEnxetSur1}
	Context: About Enxet youths taken during war to do labour for the Paraguayan military.\\
	\gll Eyke apk-el-mak-po \textbf{makham} joven, kel-an-e ámay, kel-m-agkya’a pala.\\
	\textsc{ass} \textsc{m}-\textsc{dist}-have.\textsc{m}.\textsc{mid}-\textsc{nmlz}.\textsc{ipfv} still youth \textsc{dist}-make-\textsc{nmlz}.\textsc{pfv} road \textsc{dist}-have-go\_around:\textsc{nmlz}.\textsc{ipfv} shovel\\
	\glt \lq The young men were taken, to prepare the roads, to carry the shovels.' \parencite[525]{Elliot2021}
	
	\ex
	Context: From a text about the founding of a new settlement. Time has gone by.\\
	\gll Pero tás-ek=eyke \textbf{makham} neg-h-eykha.\\
	but good-\textsc{decl}-\textsc{ass} still 1\textsc{pl}-sit-go\_around:\textsc{nmlz}.\textsc{pfv}\\
	\glt \lq But our lives are still good.' \parencite[771]{Elliot2021}

	\ex\label{exAppendixEnxetSur3}
	Context: A boy has fought off several demon-like monsters. He has only briefly caught his breath, then another one emerged.\\
	\gll Eyke apk-ennap-ekp-o \textbf{makham}=ma’a.\\
	\textsc{ass} \textsc{m}-kill\_many-\textsc{m}-\textsc{nmlz}.\textsc{ipfv} still=\textsc{dem}\\
	\glt \lq But he fought still.' \parencite[759]{Elliot2021}
	
	\ex\label{exAppendixEnxetSur4}
	\gll Wa’ keso nelán-tepak-xa, keso ámay’, ámay \textbf{makham}.\\
	look \textsc{prox} 1\textsc{pl}.\textsc{ptcp}.\textsc{dist}-emerge-\textsc{nmlz}.\textsc{obl} \textsc{prox} road road still\\
	\glt \lq Look, this is where we came out earlier, this road, this is the same road (lit. …this is still the road).\rq{ }\parencite[266]{Elliot2021}
\end{exe}

\subsubsection{Uses on the fringes of \lq{}still\rq{}}

\paragraph{Scalar contexts}\label{appendixEnxetSurScalar}
\begin{itemize}
	\item \textit{Makham} is attested in contexts of decreases along a scale.
\end{itemize}

\begin{exe}
	\ex Context: While taking honey combs out of a tree trunk, speaker sees more inside.\\
	\gll Wa’ k-emexch-e’=nak \textbf{makham}.\\ 
	so \textsc{f}-lack-\textsc{decl}=\textsc{evid} still\\
	\glt \lq Look, there’s still more.\rq{ }\parencite[325]{Elliot2021}\\

	\ex Context: About a larger pond that has mostly dried up.\\
	\gll Neg-wet’-ak \textbf{makham} k-etsék yegmen=se’e.\\
	1\textsc{pl}-see-\textsc{decl} still \textsc{f}-few water=\textsc{prox}\\
	\glt \lq We still see a little water here.\rq{ }(\cite[562]{Elliot2021} and John Elliot, p.c.)
\end{exe}

\subsubsection{Broadly adverbial temporal and aspectual functions}


\paragraph{Iterative and restitutive}
\label{appendixEnxetSurIterative}
\begin{itemize}
	\item \textcite[484]{Elliot2021}.
	\item Examples (\ref{exAppendixEnxetSurIterative1}–\ref{exAppendixEnxetSurIterative3}) illustrate iterative uses. Restitutive ones are illustrated in (\ref{exAppendixEnxetSurRestitutive1}–\ref{exAppendixEnxetSurRestitutive3}).
	\item Both uses are often, but not always, accompanied by the iterative/restitutive verb suffix \mbox{-\textit{akx}} (termed \lq\lq duplicative" in \cite{Elliot2021}), e.g. in 	(\ref{exAppendixEnxetSurIterative3}, \ref{exAppendixEnxetSurRestitutive3}).
	\item An example like (\ref{exAppendixEnxetSurIterative4}) can be understood as lying at the intersection of iterativity and additivity: as there is no indexation of the p-argument, \textit{apchaqha} kill can be intepreted as either intransitive (\lq he killed again') or as involving an implied indefinite patient argument (\lq he killed another one') (John Elliot, p.c.). 
	\item Syntax: fixed, post-predicate position.
\end{itemize}

\begin{exe}
	\ex\label{exAppendixEnxetSurIterative1}
	Context: A preacher speaking, at the beginning of a sermon.\\
	\gll
	Bueno, keso axto’o=naj=se’e nél-wet-axk-o \textbf{makham} como siempre.\\
	good this morning=\textsc{evid}=\textsc{prox} 1\textsc{pl}:\textsc{dist}-see-\textsc{mid}-\textsc{nmlz}.\textsc{ipfv} still as always\\
	\glt \lq Well, this morning, we see each other again as always.\rq{ }(John Elliot, p.c.)

	\ex\label{exAppendixEnxetSurIterative2}
	Context: A boy has fought off several demon-like creatures. Another one has come.\\
	\gll Apk-ennáp-eykpek=axta \textbf{makham} ek-w-om-oho apch-aqh-a.\\
	\textsc{m}-kill\_many-\textsc{mid}.\textsc{m}:\textsc{decl}=\textsc{pst} still \textsc{f}-arrive-on\_arrival-\textsc{intens}:\textsc{nmlz}.\textsc{ipfv} \textsc{m-}kill-\textsc{nmlz}.\textsc{pfv}\\
	\glt \lq He fought again until he killed it.' \parencite[758]{Elliot2021}

	\ex\label{exAppendixEnxetSurIterative3}
	\gll Exchek ka-m-háp-awo keleyke aktek sek-han-ma=exchek, né-m-hápey-ás-eyk-ekx-oho=sa \textbf{makham}.\\
	\textsc{rec}.\textsc{pst} \textsc{f}.\textsc{irr}-\textsc{vblz}-soft-\textsc{ins}.\textsc{decl} beans \textsc{poss}.\textsc{f} 1\textsc{sg}-cook-\textsc{nmlz}.\textsc{pfv}=\textsc{rec}.\textsc{pst} 1\textsc{pl}-\textsc{vblz}-soft-\textsc{caus}-\textsc{temp}.\textsc{indef}-\textsc{intens}=\textsc{fut} still\\
	\glt \lq The beans that I have cooked were not soft, they have to be softened again.'\footnote{See \textcite[169–175]{Elliot2021} on the \lq\lq temporal indefinite" marker.}  \parencite[447]{Elliot2021}
		
	\ex\label{exAppendixEnxetSurIterative4}
	Context: A boy has fought off a demon. Another demon has appeared.\\
	\gll Natámen apch-aqh-a \textbf{makham} w-okm-ek=axta a-anet apk-ennap-ma tén.\\
	then \textsc{m}-kill-\textsc{nmlz}.\textsc{ipfv} still \textsc{f}.arrive-\textsc{terminative}-\textsc{decl}=\textsc{pst} \textsc{f}.\textsc{stat}-two \textsc{m}-kill\_many-\textsc{nmlz}.\textsc{pfv} then\\
	\glt \lq Then he killed it [again], coming to two of them he had battled, and…' \parencite[757]{Elliot2021}

	\ex\label{exAppendixEnxetSurRestitutive1}
	\gll Wa-haxyegk-egk-es-ekx-ak=sa’ sey-ánt-e nápakha escuela, tén=sa' e-yxy-ok \textbf{makham}.\\
	1\textsc{sg}-circle-\textsc{complexive}-\textsc{caus}/\textsc{appl}-\textsc{iter}-\textsc{nmlz}.\textsc{pot}=\textsc{fut} 1\textsc{sg}-mount-\textsc{nmlz}.\textsc{pfv} side.\textsc{m} school then=\textsc{fut} 1\textsc{sg}.\textsc{irr}-return-\textsc{intens}:\textsc{nmlz}.\textsc{pot} still\\
	\glt \lq I’m going to do donuts around the school in my car, then I’ll come back again.'\footnote{See \textcite[500–506]{Elliot2021} on the \lq\lq complexive" suffix \mbox{-\textit{ey}}.} \parencite[734]{Elliot2021}
	
	\ex\label{exAppendixEnxetSurRestitutive2}
	\gll E-s-an-tag-kas-ek \textbf{makham}=ma’a ap-t-eyak.\\
	\textsc{m}.\textsc{irr}-carry-\textsc{complexive}-\textsc{ven}-\textsc{caus}-\textsc{pot}.\textsc{nom} still=\textsc{dem} \textsc{m}.\textsc{ptcp}--eat-\textsc{complexive}.\textsc{nom}.\textsc{pfv}\\
	\glt \lq They would bring the bread back.' \parencite[180]{Elliot2021}

	\ex\label{exAppendixEnxetSurRestitutive3}
	\gll Mass-eg-ke’=axta=eykhe sek-m-o-wán-a a-pawak-s-ek piano, ey-e-wagk-ekx-eyk=eyke \textbf{makham} sek-páw-áss-o.\\
	\textsc{f}:diminish-\textsc{complexive}-\textsc{decl}=\textsc{pst}=\textsc{frust} 1\textsc{sg}-\textsc{temp}.\textsc{indef}-\textsc{vblz}-able-\textsc{nmlz}.\textsc{pfv} 1\textsc{sg}.\textsc{irr}-noise-\textsc{caus}/\textsc{appl}-\textsc{nmlz}.\textsc{pot} piano 1\textsc{sg}-\textsc{vblz}-able-\textsc{iter}-\textsc{decl}=\textsc{asrt} still 1\textsc{sg}-noise-\textsc{caus}/\textsc{appl}-\textsc{nmlz}.\textsc{pfv}\\
	\glt \lq I lost my ability to play piano, but I have regained it.\rq{ }\parencite[346–347]{Elliot2021}
\end{exe}

\paragraph{Prospective \lq eventually\rq{}}\label{appendixEnxetSurProspective}
\begin{itemize}
	\item At minimum the two tokens in (\ref{exAppendixEnxetSurProspective1}, \ref{exAppendixEnxetSurProspective2}), both from the same text, feature a prospective \lq{}eventually\rq{ }use.
\end{itemize}

\begin{exe}
	\ex\label{exAppendixEnxetSurProspective1}
	Context: About the prophecies made by a pastor when the speaker was young.\\
	\gll Chaxa, chaxa ap-xeyen-ma=axta, pastor enles=xa, é-t-ak=sa' \textbf{makham} énxet'ák keso eg-mogye'=nak=se'e.\\
	that that \textsc{m}-show-\textsc{nmlz}.\textsc{pfv}=\textsc{pst} pastor English=\textsc{dem} \textsc{m}.\textsc{irr}-see-\textsc{pot}=\textsc{fut} still man.\textsc{pl} this \textsc{poss}.1\textsc{pl}-before=\textsc{evid}=\textsc{prox}\\
	\glt \lq That, that’s what he told us, that English pastor, The Enxet will see these things [yet] in the future before us.' (John Elliot, p.c.)

	\ex Context: About the prophecies made by a pastor when the speaker was young.\\
	\label{exAppendixEnxetSurProspective2}
	\gll Natamén makham, ap-xén-chek=axta anhan=ma'a, yaqsa, ap-xeyen-ma=axta, bueno ap-xén-chek e-tneh-ek, kelwesse'e énxet'ák enles, ek-wok-moho \textbf{makham} sónegwanxa.\\
	then still \textsc{m}-show-\textsc{decl}=\textsc{pst} and=\textsc{dem} what \textsc{m}-show-\textsc{nmlz}.\textsc{pfv}=\textsc{pst} so \textsc{m}-show-\textsc{decl} \textsc{m}.\textsc{irr}-be/say-\textsc{pot} rich\_person.\textsc{pl} man.\textsc{pl} English \textsc{f}-arrive-\textsc{terminative}.\textsc{intens}:\textsc{nmlz}.\textsc{pfv} still nowadays\\
	\glt \lq So then, he told us that... what did he tell us? well, the English man said the Enxet would become rich people, when we arrived in the present time.' (John Elliot, p.c.)
\end{exe}


\subsubsection{Additive and related uses}
\paragraph{Additive}\label{appendixEnxetSurAdditive}
\begin{itemize}
	\item \textcite[570]{Elliot2021}.
	\item This use is repeatedly found in conjunction with \textit{pók}/\textit{mók} \lq other\rq{ }(\ref{exAppendixEnxetSurAdditive3}).
\end{itemize}
\begin{exe}
	\ex Context: The speaker is finding medicinal plants in the forest.\\
	\gll Pánaqte keso, pánaqte \textbf{makham}=se'e.\\
	medicine this medicine still=\textsc{prox}\\
	\glt \lq This is a kind of medicine, this is another kind of medicine.' 
	\\(John Elliot, p.c.)
	
	\ex	Context: About the many English missionaries at a certain mission.\\
	\gll Ekeso e-etche'=axta, Pwege=axta e-etche natámen makham, ekeso Pegwe enles \textbf{makham}.\\
	this \textsc{poss}.\textsc{f}-child=\textsc{pst} P.=\textsc{pst}  \textsc{poss}.\textsc{f}-child after.\textsc{f} still this P. English still\\
	\glt \lq This one, he was her son, he was Pegwe's son, then still, this Pegwe, she was another English person.' \parencite[267–268]{Elliot2021}
	\pagebreak
	\ex\label{exAppendixEnxetSurAdditive3}
	Context: A group of Enxet sneak up on a group of Sanapaná and ambush them, shooting with arrows.\\
	\gll Pelakkasek=hek=ñat ap-makt-ákp-o xama, ap-makt-ákpek=hek=ñat nahan pók, pók \textbf{makham} lap-makt-ak énxet.\\
	suddenly=\textsc{evid}=\textsc{rem}.\textsc{pst} \textsc{m}-shoot-\textsc{mid}.\textsc{m}-\textsc{nmlz}.\textsc{ipfv} one \textsc{m}-shoot-\textsc{mid}.\textsc{m}.\textsc{decl}=\textsc{evid}=\textsc{rem}.\textsc{pst} and \textsc{m}.other \textsc{m}.other still \textsc{m}-shoot-\textsc{decl} Enxet\\
	\glt \lq All of a sudden one was shot, and another one was shot, the Enxet shot yet another.' (\cite{LopezRamirez1996}; John Elliot, p.c.)
	\end{exe}

\paragraph{Conjunctional adverb}\label{appendixEnxetSurConjunctional}
\begin{itemize}
	\item \textit{Makham} recurrently occurs following the connective \textit{natámen} \lq then', particularly if the subsequent clauses contain the marker.
\end{itemize}

\begin{exe}
	\ex
	\gll \textbf{Natamén} \textbf{makham} ap-tamh-aha makham Kennaqte Appeywa Tásek Amya’a.\\
	then still \textsc{m}-work-go\_around.\textsc{decl} still K. A. good story\\
	\glt \lq Then later, Kennaqte Appeywa spoke again from the Bible.\rq{ }(John Elliot, p.c.)

	\ex
	Context: The speaker is listing out names.\\
	\textit{Yetneyk axta anhan apkelwányam, apwesey axta Monte,}
	\\ \lq And here was an old man, this name was Monte,\rq{}

	\gll \textbf{natámen} \textbf{makham} may-'á-segk-ok apwesey, wánxa.\\
	then still \textsc{neg}-1\textsc{sg}.\textsc{irr}-know-\textsc{complexive}-\textsc{decl} \textsc{ptcp}.\textsc{m}-called only\\
	\glt \lq And then, I forgot their names, that's it.' (John Elliot, p.c.)
\end{exe}
\il{Lengua, Southern|)}

\pagebreak
\section{Trió (tri, trio1238)}\il{Trió|(}
\label{appendixTrio}

\subsection{Introductory remarks}
My understanding of Trió has greatly profited from discussion with Sérgio Meira, who also shared unpublished data with me.

\subsection{=nkërë}

\subsubsection{General information}
\begin{itemize}
	\item Form: transcribed as \textit{nkörö} in \textcite{Letschert1998}.
	\item Wordhood: bound morpheme, an enclitic that takes constituents of various syntactic classes as its host.
	\item Syntax: \mbox{=\textit{nkërë}} attaches to its focus; with the main predicate as its host, it serves as a sentence adverb.
\end{itemize}

\subsubsection{As a \lq{}still\rq{ }expression}
\begin{itemize}
	\sloppy
	\item \textcite[130, 453–454, 504]{Carlin2004}, \textcite[15]{Letschert1998} and \textcite[468–469]{Meira1999}.
	\item Specialisation: the descriptions by \textcite{Carlin2004} and \textcite{Meira1999}, when taken together, clearly identify this marker as \textsc{still} expression. The description by \textcite[453–454]{Carlin2004} includes the notion of an alternative scenario (but see below on the question of pragmaticity). This is corroborated by the fact that in order for nouns to host phasal polarity \mbox{=\textit{nkërë}}, they need to first be augmented by the \lq\lq attributivizer" \parencite{Meira1999} or \lq\lq facsimile" \parencite{Carlin2004} suffix \mbox{-\textit{me}}, which denotes a manifest, but not intrinsic quality, typically a transient state \parencite[123–124, 130]{Carlin2004}. What is more, when a state construed via \mbox{Noun-\textit{me}} is described from the perspective of a later point in time, where the state no longer holds, use of \mbox{=\textit{nkërë}} is compulsory \parencite[130]{Carlin2004}. 
	\item Pragmaticity: \textcite[450]{Carlin2004} describes \mbox{=\textit{nkërë}} as counter-expectational. This is, however, hard to conciliate with examples like (\ref{exAppendixTrio1}) and, more generally, the obligatoriness of \mbox{=\textit{nkërë}} with past transient states like childhood. Example (\ref{exAppendixTrio3}) suggests that an unexpected late scenario is construed by additional use of a counter-expectational clitic (see \cite[461–462]{Meira1999} and \cite[451–452]{Carlin2004} on the latter).
	\item Polarity sensitivity: inner negation yields \textsc{not yet}. In temporal clauses, this is the default way of signalling precedence \parencite[504]{Carlin2004}.
	\item Syntax: as a \textsc{still} expression, \mbox{=\textit{nkërë}} attaches to the main predicate.
\end{itemize}

\begin{exe}
	\ex\label{exAppendixTrio1}
	Context: From the opening of an autobiographical narrative.\\
	\gll Pena pijukuku=me=\textbf{nkërë}  ahtao, susu=pë, manko akëërë. Këpëewa, irëe-n-ai ji-wame, pijukuku=me=\textbf{nkërë} ji-w-eh-topo-npë=n-ai ji-wame.\\
	long\_ago baby-\textsc{attr}=still while breast-with \textsc{poss}.1:mother with but 3.\textsc{anaph}-3\textsc{a}-\textsc{cop} 1-don't\_know baby-\textsc{attr}=still 1-\textsc{a}-\textsc{cop}-\textsc{circumstance}.\textsc{nmlz}-\textsc{pst}=3\textsc{a}-\textsc{cop} 1-don't\_know\\
	\glt \lq Long ago, when I was still a baby, at the breast, with my mother. But that I do not know. I do not know what I was like when I was [still] a baby.' \parencite[606]{Meira1999}
		
	\ex
	\gll Pëera w-ah-këne=\textbf{nkërë}.\\
	stupid 1>3-\textsc{cop}-\textsc{rem}.\textsc{pst}=still\\
	\glt \lq I still used to be stupid (unknowledgeable).'  \parencite[295]{Carlin2004}
	
	\ex\label{exAppendixTrio3}
	Context: A mother has been told that her son died. Much to her surprise, he is alive and has come to visit her.\\
	\gll Waa=w-eh-to w-ekanïpï, tëërë=\textbf{nkërë}=hkaarë manan, tïí-ka-e.\\
	\textsc{neg}.\textsc{exist}=\textsc{a}-\textsc{cop}-\textsc{nmlz}.\textsc{circumstance}  1>3-think:\textsc{imm.pst} \textsc{exist}=still=\textsc{counter\_expectation}  2:\textsc{cop}:\textsc{prs}   \textsc{rem}.\textsc{pst}-say-\textsc{rem}.\textsc{pst}\\
	\glt \lq {\lq\lq}I thought you were dead (lit. I thought your not being), but you turn out to be still there (= alive)!" (she) said.' (Sérgio Meira, p.c.)
\end{exe}

\subsubsection{Uses on the fringes of \lq{}still\rq{}}
\paragraph{Continued iteration (and continued restitutive)}
\label{appendixTrioContinuedIteration}
\begin{itemize}
	\item \textcite[468–469]{Meira1999}
	\item This function obtains in combination with the \lq\lq repetition" \parencite{Meira1999} or \lq\lq cyclical" \parencite{Carlin2004} marker \mbox{=\textit{pa}}. Often, these two clitics occur on the same host, hence \mbox{=\textit{nkërë}=\textit{pa}}, as in (\ref{exAppendixTrioContinuedIteration1}). This is, however, not always the case, as \mbox{=\textit{pa}} gravitates toward sentence-initial elements (\cite[453]{Meira1999}; \cite[430]{Carlin2004}), as in (\ref{exAppendixTrioContinuedIteration2}, \ref{exAppendixTrioContinuedIteration3}) .
	\item This function covers both continued iteration sensu stricto (\ref{exAppendixTrioContinuedIteration1}, \ref{exAppendixTrioContinuedIteration2}), as well as continued restitutions (\ref{exAppendixTrioContinuedIteration3}).
\end{itemize}

\begin{exe}	
	\ex\label{exAppendixTrioContinuedIteration1}
	\gll Eeke mi-ponopïï-je=\textbf{nkërë}=\textbf{pa}?\\
	how 2\textsc{a}-tell-\textsc{mod}=still=\textsc{iter}\\
	\glt \lq How could you tell this again?' \parencite[317]{Meira1999}\footnote{\mbox{-\textit{je}}, here for the sake of simplicity glosses as \textsc{mod} \lq modal' is a type of mirative marker, indicating surprise and disbelief (\cite[298]{Carlin2004}; \cite[316–317]{Meira1999}). In this instance, it serves to contribute a negative evaluation of the addressee's repeated retelling (Sérgio Meira, p.c.).}
	
			
	\ex\label{exAppendixTrioContinuedIteration2}
	\gll
	Tëin=ken=\textbf{pa} kaikui t-ëpoo-se wïtoto=ja=\textbf{nkërë}.\\
	once=\textsc{cont}=\textsc{iter} jaguar \textsc{rem}.\textsc{pst}-meet-\textsc{rem}.\textsc{pst} person=\textsc{agt}=still\\
	\glt \lq The person met Jaguar still once more (i.e. after several previous meetings).' \parencite[469]{Meira1999}
	
	\ex\label{exAppendixTrioContinuedIteration3}
	\gll Irë=mao=\textbf{pa} tï-w-ëe-se=\textbf{nkërë} irënehka wïi t-ënee-se ii-ja.\\
	3.\textsc{anaph}=then=\textsc{iter} \textsc{rem}.\textsc{pst}-\textsc{subj}-come-\textsc{rem}.\textsc{pst}=still at\_last cassava \textsc{rem}.\textsc{pst}-bring-\textsc{rem}.\textsc{pst} 3-\textsc{agt}\\
	\glt \lq Then he came back [once again]; at last, he had brought cassava.'
	\\(\cite[448]{Meira1999}; Sérgio Meira, p.c.) 
\end{exe}

\subsubsection{Marginality}
\label{appendixTrioMarginal}
\begin{itemize}
	\item I have only two examples of this use.
	\item Syntax: as with \mbox{=\textit{nkërë}} as \textsc{still}, in this function it attaches to the predicate.
\end{itemize}

\begin{exe}
	
	\ex\label{exAppendixTrioMarginal1}
	\gll Kure=\textbf{nkërë}.\\
	good=still\\
	\glt \lq Good enough i.e. could be better, but is still good)\rq{ }(Sérgio Meira, p.c.)
	
	\ex\label{exAppendixTrioMarginal2}
	Context: About a fish the protagonist of the story has caught.\\
	\gll Akï=hpe eka, eka, nërë, pejo-pisi apo=pohpa mono=\textbf{nkërë}, arawe apo.\\
	what=\textsc{indef} \textsc{poss}.3:name \textsc{poss}.3:name 3.\textsc{anaph} fish\_sp-\textsc{dim} like=\textsc{emph} big\_one=still cockroach\_sp like\\
	\glt \lq What's-it-called, its name, its name, that one just like a little \textit{pejo} fish, a little big, like a (big) cockroach.' (i.e. it still counts as big, despite being similar to a small pejo fish) (Sérgio Meira, p.c.)
\end{exe}

\pagebreak\largerpage[2]
\subsubsection{Additive and related uses}
\paragraph{Additive}\label{appendixTrioAdditive}
\begin{itemize}
	\item \textcite[237, 242]{Carlin2004} and \textcite[450–451]{Meira1999}.
	\item Sometimes the host of \mbox{=\textit{nkërë}} as \lq also, too\rq{ }is the clausal connective \mbox{\textit{se}(\textit{h})\textit{ke}(\textit{n})} \lq also, likewise\rq{}. This collocation seems to stress the notions of continuity and parallelism (\ref{exAppendixTrioSekeNkere}).
\end{itemize}
\begin{exe}
	\ex\label{exAppendixTrioAlso1}
	Context: At a store.\\
	\gll Tëërë=\textbf{nkërë} …?\\
	\textsc{exist}=still\\
	\glt \lq Do you also have … (i.e. in addition to what I've already bought)?' (Sérgio Meira, p.c.)
		
	\ex\label{appendixTrioAlso2}
	\gll A-tï=\textbf{nkërë}-hpe  m-ene?\\
	\textsc{q}-\textsc{inanimate}=still=\textsc{indef} 2>3-see:\textsc{imm}.\textsc{pst}\\
	\glt \lq What on earth else did you see?ʼ \parencite[233]{Carlin2004}

	\ex\label{appendixTrioIncrement1}
	Context: Giving a list of family members present on a specific occasion.\\
	\gll Eemi-rï, i-mama-rï eemi-rï=\textbf{nkërë} tëri-me,	pirë-me, i-papa	marë.\\
	\textsc{poss}.3:daughter-\textsc{poss} \textsc{poss}.3-mother-\textsc{poss} \textsc{poss}.3:daughter-\textsc{poss}=still three-\textsc{attr} four-\textsc{attr} \textsc{poss}.3-father also\\
	\glt \lq (They were) his daughter, his mother, another daughter (lit: still [a] daughter), three of them, four of them, his father too.\rq{ }\parencite[454]{Carlin2004}	
	
	\ex\label{appendixTrioIncrement4}
	Context: From a picture elicitation task.\\
	\gll Ma akoron=n-ai tëërë=\textbf{nkërë}=n-ai, atï=rë=\textbf{nkërë}, ...  kahku.\\
	\textsc{top} other=3-\textsc{cop} exist=still=3-\textsc{cop} something=\textsc{emph}=still {} wheelbarrow\\
	\glt \lq Well, another thing is there, there's still one (more), still something else, … a wheelbarrow.' (Sérgio Meira, p.c.)	

	\ex\label{exAppendixTrioSekeNkere}
	\gll Kure	menu=tao=ken	t-ee-se			nërë,	Waruku,	i-nmuku-pisi
	\textbf{seke}=\textbf{nkërë} t-ee-see.\\
	pretty paint=\textsc{loc}=\textsc{cont} \textsc{rem}.\textsc{pst}-\textsc{cop}-\textsc{rem}.\textsc{pst} 3.\textsc{anaph}.\textsc{anim} W. 3-son.\textsc{poss}-\textsc{dim} also=still \textsc{rem}.\textsc{pst}-\textsc{cop}-\textsc{rem}.\textsc{pst}\\
	\glt \lq Waruku was beautifully painted, and so was her little son.\rq{ }\parencite[451]{Meira1999}	
\end{exe}\il{Trió|)}

\section{Xavánte (xav, xava1240)}
\il{Xavánte|(}
\label{appendixXvante}
\subsection{(za)hadu}

\subsubsection{General information}
\begin{itemize}
	\item Form: free variation \textit{zahadu} \sim{ }\textit{hadu}.
	\item Wordhood: free morpheme.
	\item Syntax: fixed, preceding the predicate.
\end{itemize}

\subsubsection{As a \lq{}still\rq{ }expression}
\begin{itemize}
	\item \citeauthor{Lachnitt1987} (\citeyear[7]{Lachnitt1987}, \citeyear[167]{Lachnitt1988}), \textcite[107]{Estevam2011} and \textcite[123, 134]{McLeod2004}.
	\item Specialisation: examples like (\ref{exAppendixXavante1}, \ref{exAppendixXavante2}), which involve a contrast with a later point in time, give a fairly good indication that the marker conforms to my definition; additional, albeit indirect evidence, comes from its function as \textsc{not yet} (\appref{appendixXavanteNotYet}).
	\item Polarity sensitivity: inner negation yields \textsc{not yet}.
	\item Pragmaticity: the data allow no conclusions.
\end{itemize}

\begin{exe}
	\ex\label{exAppendixXavante1}
	\gll Ãhã wede hã awaʔawi \textbf{zahadu} te za, taza hã awẽmhã wa za wa-siwi me ni.\\
	\textsc{dem} tree \textsc{emph} today still \textsc{nom} 3.stand \textsc{conj} \textsc{emph} tomorrow \textsc{ego} \textsc{prosp} 1-\textsc{coll} 3.cut\_down \textsc{indef}\\
	\glt \lq Cet arbre est encore debout aujourd'hui, mais demain nous allons l'abattre. [This tree is still standing today, but tomorrow we're going to cut it down.]\rq{ }\parencite[107]{Estevam2011}

	\ex Context: From an autobiographic text.\label{exAppendixXavante2}\\
	\textit{Awaʔawi na hã wa oto ĩĩhöjbaprédub na wa ĩĩhöjmana. Tamémhã azarutu wamhã wa oto dame ĩĩmro},\\
	\glt \lq Aujourdʼhui je suis vielle. Alors quand jʼétais jeune fille je me suis mariée [Today I am old. So, when I was a young girl I got married.]\rq{}
	\exi{}
	\gll ĩĩ-ãma ĩ-wahu hã \textbf{hadu} suru re di.\\	
	1\textsc{sg}-\textsc{pvb} \textsc{nmlz}-pass\_year \textsc{emph} still 3.be\_small \textsc{dm}:respect \textsc{impr}\\
	\glt \lq je n'avais pas encore l'âge (lit. les années écoulées par rapport à moi étaient encore petites) [I wasn't of age yet (lit. the years that had passed by me were still little).]\rq{ }\parencite[503]{Estevam2011}
\end{exe}

\subsubsection{Uses relating to other phasal polarity concepts}
\paragraph{Not yet}\label{appendixXavanteNotYet}
\begin{itemize}
	\item \citeauthor{Lachnitt1987} (\citeyear[47]{Lachnitt1987}, \citeyear[167]{Lachnitt1988}), \textcite[107]{Estevam2011} and \textcite[123, 209]{McLeod2004}.
	\item This usage is attested as a a stand-alone negative response to a polar question (\ref{exAppendixXavanteNotYet1}) and as an interjection \lq Wait!\rq{ }(\ref{exAppendixXavanteNotYet2}).
\end{itemize}

\begin{exe}
	\ex\label{exAppendixXavanteNotYet1}
	\gll E ma tô a-sa? – \textbf{Hadu}.\\
	\textsc{q} \textsc{ant} \textsc{rl} 2-eat {} still\\
	\glt \lq As-tu mangé? -- (Pas) encore. [Have you eaten? -- (Not) yet.]\rq
	\parencite[107]{Estevam2011}

	\ex\label{exAppendixXavanteNotYet2}
	\begin{xlist}
		\exi{A:}\gll We ĩ̠-ma a na.\\
		\textsc{ven} 1\textsc{sg}-\textsc{dat} give \textsc{sbjv}\\
		\glt \lq Me dá. [Give me.]\rq
		
		\exi{B:}
		\gll \textbf{Zahadu}! Wa za ai-ma ti'-a tô.\\
		still \textsc{ego} \textsc{prosp} 2-\textsc{dat} 3-give \textsc{rl}\\
		\glt \lq Pacência! Vou te dar certamente. [Just wait. I'll definitely give (it to) you.]\rq{ }(\cite[123]{McLeod2004}, glosses added)
	\end{xlist}
\end{exe}
\il{Xavánte|)}
