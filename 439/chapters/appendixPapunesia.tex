\chapter{Papunesia}
\label{appendixPapunesia}

\section{Acehnese (ace, achi1257)}\label{appendixAcehnese}
\il{Acehnese|(} 
\subsection{mantöng}
\subsubsection{General information}
\begin{itemize}
	\item Wordhood: independent grammatical word, invariable.
	\item Form: also transcribed as \textit{mantong} and \textit{mantòng}.
\end{itemize}

\subsubsection{As a \lq{}still\rq{ }expression}
\label{appendixAcehneseStill}
\begin{itemize}
	\item  \textcite[112, 150]{DaudDurie1985}, \textcite[30]{DjajadiningratVol2}, \textcite[224]{Durie1985}, and \textcite[175]{Kreemer1931}; additional discussion in \citeauthor{vanBaar1991} (\citeyear{vanBaar1991}, \citeyear[110–111]{vanBaar1997}).
	\item Specialisation: the entries/translations in 
the literature, as well as examples like (\ref{exAppendixAcehneseStill2}–\ref{exAppendixAcehneseStill4}), give evidence that \textit{mantöng} meets my definition. Note for instance the prototypical contrast between two phases in (\ref{exAppendixAcehneseStill2}). Further, albeit indirect, evidence for the specialisation of this marker comes from the robustly attested restrictive–still polysemy (\Cref{sectionExclusive}). 
	 \item Pragmaticity: seems to be compatible with both scenarios. Ex. (\ref{exAppendixAcehneseStill3}) is a prime candidate for the unexpectedly late scenario.
	\item Polarity sensitivity: no examples featuring negation in the data. There are several distinct \textsc{not yet} expressions and \textsc{no longer} is expressed via \textsc{neg} \textit{lê} \lq more'.
	\item Syntax: in phasal polarity function, \textit{mantöng} can either precede or follow the predicate.
	\item Further note: \textit{mantöng} can serve as an (elliptical) pro-sentence, see (\ref{exAppendixAcehneseStill4}).
\end{itemize}

\begin{exe}
	\ex\label{exAppendixAcehneseStill2}
	\gll Dilèë \textbf{mantòng} lè-lò, djinòë ka teutab akaj.\\
	earlier still restless now already stationary reason\\
	\glt \lq Vroeger was hij nog ongedurig, nu is hij rustig geworden. [Back in the days he still used to be restless, now he has become settled.]' (\cite[1046]{DjajadiningratVol2}, glosses added)

	\ex\label{exAppendixAcehneseStill3}
	\gll {Gòb njan} landjoet that oemoe, ka sireutōïh thōn \textbf{mantòng} teuga.\\
	3 long very lifetime already hundred year still strong\\
	\glt \lq Wat een lang leven heeft die man, hij is al honderd jaar en is nog krachtig. [What a long life that man has, he is a hundred years old already and he is still strong.]\rq{ }(\cite[882]{DjajadiningratVol1}, glosses added)

	\ex\label{exAppendixAcehneseStill4}
	\gll Mantòng teukoe di kantō jōh dròë=neu wòë boenòë? – \textbf{Mantòng}.\\
	still T. at office when self=2 return earlier\_today {} still\\
	\glt \lq Was de teukoe nog op het kantoor, toen u zooeven naar huis ging? -- (ja hij was er) nog.' [Was the Teukoe still in his office when you just left?  -- (Yes he) still (was).]\rq{ }(\cite[30]{DjajadiningratVol2}, glosses added)
\end{exe}

\subsubsection{Uses on the fringes of \lq{}still\rq{}}
\paragraph{Scalar contexts}\label{appendixAcehneseScalar}
\begin{itemize}
	\item \textit{Mantöng} is attested in scalar contexts involving a decrease over time.
	\item That this use relates to \textsc{still} (rather than to the exclusive function) becomes visible in (\ref{exAppendixAcehneseDecrement2}), where \textit{mantöng} precedes the predicate.
\end{itemize}

\begin{exe}
	\ex\label{exAppendixAcehneseDecrement1}
	 \gll Le that ceuneurōh \textbf{mantöng}\\
	many very frying still\\
	\glt \lq There is still a large quantity of things to be fried.'
	\parencite[57]{Asyik1987}

	\ex \label{exAppendixAcehneseDecrement2}
	\gll Hana lōn-bloe saka sabab \textbf{mantöng} le di rumoh\\
	\textsc{neg} 1\textsc{sg}-buy sugar because still many in house\\
	\glt \lq I am not buying any sugar because there is still much at home.\rq{ }\parencite[175]{Asyik1987}
\end{exe}

\subsubsection{Restrictive (non-temporal)}
\paragraph{(Non-scalar) exclusive}\label{appendixAcehneseRestrictive}
\begin{itemize}
	\sloppy
	\item \textcite[223–225]{Durie1985}, \textcite[30]{DjajadiningratVol2}, \textcite[175]{Kreemer1931} and \textcite[32, 63]{Hurgronje1900}; additional discussion in \citeauthor{vanBaar1991} (\citeyear{vanBaar1991}, \citeyear[110–111]{vanBaar1997}).
	\item As is typical of exclusive markers, this is best considered a cluster of functions. These include restricting the reference of a nominal or noun phrase to a single entity (\ref{exAppendixAcehneseRestrictive1}, \ref{exAppendixAcehneseRestrictive2}) and yielding \lq alone' in combination with \textit{droe} \lq self' (\ref{exAppendixAcehneseRestrictive3}). 
	\item \Textcite{vanBaar1991}, based on the two relevant tokens in \textcite{Durie1985}, tentatively suggests that the restrictive function only occurs with associates lower than the main predicate, hence the pseudo-cleft construction in (\ref{exAppendixAcehneseRestrictive1}). This is contradicted by examples like (\ref{exAppendixAcehneseRestrictive4}, \ref{exAppendixAcehneseRestrictive5}), where restrictive \textit{mantöng} associates with the predicate.
	\item Syntax: \textit{mantöng} always follows the focus.
\end{itemize}

\begin{exe}	\ex\label{exAppendixAcehneseRestrictive1}
	\gll Lôn \textbf{mantöng} nyang=sakêt.\\
	1\textsc{sg} still \textsc{rel}=sick\\
	\glt \lq There’s only me who is sick.ʼ (i.e. only I am sick) \parencite[229]{Durie1985}	
	\ex\label{exAppendixAcehneseRestrictive2}
	\gll Ta-peu-loempang séb meung malam njòë \textbf{mantòng.}\\
	2-\textsc{caus}-shelter enough only night \textsc{prox} still\\
	\glt \lq Geef mij enkel en alleen maar voor dezen nacht een onderkomen. [Give me shelter for just this night.]\rq{ }(\cite[280]{Kreemer1931}, glosses added)

	\ex\label{exAppendixAcehneseRestrictive3}
	\gll Lōn keumeung dja\textsuperscript{ʔ} keu-dròë \textbf{mantòng}.\\
	1\textsc{sg} want go by-self still\\
	\glt \lq Ik zal alleen gaan. [I will go alone.]\rq{ }(\cite[175]{Kreemer1931}, glosses added)
	
	\ex\label{exAppendixAcehneseRestrictive4}
	\gll Pakòn ta-tagoeën boe meuntah\sim meuntah? – Lōn klakla \textbf{mantòng}, areuta dròë=neu meu-kla that boenòë.\\
	why 2-cook rice raw\sim\textsc{redupl} {} 1\textsc{sg} prepare\_in\_hurry still possession self=2 \textsc{mid}-starving very earlier\_today\\
	\glt \lq Waarom heb je de rijst niet (goed) gaar gekookt?  -- Ik heb het in de haast gedaan, omdat u zoeven zooʼn honger had. [Why didn't you cook the rice until it's completely done? -- I [just] prepared it in a hurry, because you were so hungry a moment ago.]\rq{ }(\cite[764–765]{DjajadiningratVol1}, glosses added)

	\ex\label{exAppendixAcehneseRestrictive5}
	\gll Han lōn blòë njòë, lōn meu-teumeung \textbf{mantòng}.\\
	\textsc{neg} 1\textsc{sg} buy \textsc{prox} 1\textsc{sg} \textsc{mid}-find still\\
	\glt \lq Ik heb dit niet gekocht, ik heb het zoo maar verkregen (b.v. gevonden, cadeau gekregen). [I didnʼt buy this, I just got it (e.g. found it, received it as a gift).]\rq{ }(\cite[30]{DjajadiningratVol2}, glosses added)
\end{exe}
\il{Acehnese|)}

\section{Blagar (beu, blag1240)}
\il{Blagar|(} 
\subsection{yedung}
\subsubsection{General information}
\begin{itemize}
	\item Form: also transcribed as (\textit{j})\textit{edung}, \textit{jeduŋ}. Presence vs. absence of the initial segment is a matter of dialect variation.
	\item Wordhood: independent grammatical word, invariable.
	\item Syntax: fixed, but dependent on polarity. Preceding the predicate as \textsc{still}, following it as \textsc{not yet}. 
\end{itemize}

\subsubsection{As a \lq{}still\rq{ }expression}
\begin{itemize}
	\sloppy
	\item \citeauthor{Steinhauer1995} (\citeyear[281]{Steinhauer1995}, \citeyear[153]{SteinhauerBukalabang}) and \textcite[195, 295]{SteinhauerGomang2016}.
	\item Specialisation: examples like (\ref{exAppendixBlagar1}–\ref{exAppendixBlagar3}) give a good indication for the specialisation of this marker. Further, albeit indirect, evidence comes from its function as \textsc{not yet} (\appref{appendixBlagarNotYet}).
	\item Pragmaticity: the available data allow no conclusion.
	\item Polarity sensitivity: not attested with negation; \textsc{not yet} is expressed via \textit{yedung} in post-predicate position (\appref{appendixBlagarNotYet}). In the related neighbouring language Teiwa \textit{yed}, which appears to share the same pattern, the relevant marker is in complementary distribution with the standard negator, but does combine with the prohibitive and inherently negative verbs (judging from examples throughout \cite{Klamer2010}), i.e. a highly restricted form of inner negation.
	\item Syntax: precedes the predicate.
\end{itemize}

\begin{exe}
	\ex\label{exAppendixBlagar1}
	\gll Lokoe lahatala \textbf{jedung} ping veng ani.\\
	\textsc{interj} God still \textsc{obj}.1\textsc{pl}.\textsc{incl} about remember\\
	\glt \lq Thank God, God still remembers us (idiom, said when God's flag, the rainbow appears).' (\cite[207]{SteinhauerGomang2016}, glosses added)

	\ex\label{exAppendixBlagar2}
	\gll {Qangu veng qangu} na \textbf{jedung} kiki.\\
	at\_that\_time \textsc{subj}.1\textsc{sg} still small\\
	\glt \lq At that time I was still a little child.' (\cite[195]{SteinhauerGomang2016}, glosses added)
		
	\ex\label{exAppendixBlagar3}
	\gll Sa ge eda \textbf{edung} gi xarani weng nang gu mulal.\\
	person \textsc{poss}.3\textsc{sg} \textsc{pst} still \textsc{poss}.3\textsc{pl} nut about \textsc{pl} \textsc{dem} play\\
	\glt \lq Those who just now still had their kenari nuts, they play.'
	\\ \parencite[185]{SteinhauerBukalabang}
\end{exe}

\subsubsection{Uses related to other phasal polarity concepts}
\paragraph{Not yet}
\label{appendixBlagarNotYet}
\begin{itemize}
	\item  \citeauthor{Steinhauer1995} (\citeyear[281]{Steinhauer1995}, \citeyear[174]{SteinhauerBlagar}, \citeyear[153]{SteinhauerBukalabang}) and \textcite[175, 285]{SteinhauerGomang2016}.
	\item This function is attested in two contexts:
	\begin{itemize}
		\item As a negative reply to a polar question; this is shown in (\ref{exAppendixBlagarNotYet1}).
		\item When occupying the post-predicate position, as in (\ref{exAppendixBlagarNotYet2}, \ref{exAppendixBlagarNotYet3}). The same complementary distribution is found in two of the three neighbouring Alor-Pantar languages: with \ili{Nedebang} \textit{yadda} (\cite[84–85]{Schapper2014Nedebang}) and \ili{Teiwa} \textit{yed}; the latter can also embrace the predicate (judging from the examples throughout \cite{Klamer2010}). This pattern is closely related to the fact that the post-predicate position is the locus for clausal negation (see \cite[165]{SteinhauerBlagar}; \cite[273–275]{Klamer2010}; \cite[83–84]{Schapper2014Nedebang}).\il{Teiwa}\il{Nedebang} In Blagar's third neighbour Reta,\il{Reta} \textit{doo} also serves as \textsc{still} in the pre-predicate position. \textsc{not yet} is expressed via embracing negation, involving a clause-initial negator and \textit{doo} in post-predicate position. The clause-initial negator is, however, often dropped \parencite[212]{Willemsen2020}. In diachronic terms, this most likely constitutes an intermediate stage.
	\end{itemize}
	\item (\textit{y})\textit{edung} as \textsc{not yet} also combines with additive \mbox{=\textit{di}}, yielding \lq not ever' (\ref{exAppendixBlagarNotYet4}).
\end{itemize}

\begin{exe}
	\ex\label{exAppendixBlagarNotYet1}
	\gll Ana na nat ʔila? – \textbf{Jeduŋ.}\\
	\textsc{subj}.2\textsc{sg} thing eat go {} still\\
	\glt \lq Have you eaten already? -- Not yet.' \parencite[215]{SteinhauerBlagar}
	
	\ex\label{exAppendixBlagarNotYet2}
	\gll N-iva guru \textbf{jeduŋ}.\\ 
	\textsc{poss}.1\textsc{sg}-mother teacher still\\
	\glt \lq My mother isn’t a teacher yet.' \parencite[165]{SteinhauerBlagar}
	
	\ex\label{exAppendixBlagarNotYet3}
	\gll Ning lamar \textbf{edung}, ning nag na gi.\\
	\textsc{subj}.1\textsc{pl}.\textsc{excl} walk still \textsc{subj}.1\textsc{pl}.\textsc{excl} something eat first\\
	\glt \lq We did not go yet, we ate first.\rq{ }\parencite[154]{SteinhauerBukalabang}
	
	\ex\label{exAppendixBlagarNotYet4}
	\gll Qana pura por tang hoqa=\textbf{di} \textbf{jedung}.\\
	\textsc{subj}.3\textsc{sg} P. land on come=also still\\
	\glt \lq He has never yet come to Pura.' (\cite[175]{SteinhauerGomang2016}, glosses added)
\end{exe}
\il{Blagar|)} 

\section{Bukiyip (ape, buki1249)}
\label{appendixBukiyip}\il{Bukiyip|(} 

\subsection{Introductory remarks}
Bukiyip has an extensive gender/noun class system that is indexed on various parts of speech. In the examples, I gloss the corresponding markers as \textsc{ncl} \lq noun class' plus Arabic numeral, following the numbering employed in \textcite{ConradWigoga1991}.
\largerpage[2.25]

\subsection{wotak}
\subsubsection{General information}
\begin{itemize}
	\item Form: \textcite{Conrad1998}  and \textcite{ConradWigoga1991} have \textit{wotak}, \textcite{Gerstner1963} has \mbox{\textit{wota}(\textit{g})}.
	\item Wordhood: independent grammatical word, invariable.
	\item Syntax: fixed position, following the subject NP and preceding the predicate.
\end{itemize}

\subsubsection{As a \lq{}still\rq{ }expression}
\begin{itemize}
	\item \textcite[41]{ConradWigoga1991} and \textcite[28, 31, 33]{Gerstner1963}.
	\item Specialisation: evident in examples like (\ref{exAppendixBukiyip1}–\ref{exAppendixBukiyip3}). For instance, in (\ref{exAppendixBukiyip1}) \textit{wotak} construes the background situation as one in which Joseph and Mary continue to be in Bethlehem and also evokes a perspective towards their future return to Nazareth. Additional, albeit indirect evidence comes from its use as \textsc{not yet} without negation (\appref{appendixBukiyipNotYet}).
	\item Pragmaticity: appears to be  compatible with both scenarios.
	\item Polarity sensitivity: inner negation yields \textsc{not yet}. Outer negation via \textit{kob}(\textit{w})\textit{i} \lq \textsc{neg}.\textsc{irr}' yields \textsc{no longer}.
\end{itemize}

\begin{exe}
	\ex\label{exAppendixBukiyip1}
	Context: Luke 2:6. Joseph and Mary have travelled from their home in Nazareth to Bethlehem for a census. They were not planning to stay there.\\
	\gll Douk \textbf{wotak} sh-a-pe Bethlehem aria nyiato-b b-a-tagur um Maria um k-u-banuh.\\
	now still \textsc{subj}.\textsc{ncl}8.\textsc{pl}-\textsc{rl}-\textsc{cop} B. and time-\textsc{ncl}1.\textsc{sg} \textsc{subj}.\textsc{ncl}1.\textsc{sg}-\textsc{rl}-arrive for M. for \textsc{subj}.\textsc{ncl}4.\textsc{sg}-\textsc{irr}-give\_birth\\
	\glt \lq Now when they were still at Bethlehem then the time came for Mary to give birth (lit:…were still at Bethlehem and then …).' \parencite[63]{Conrad1998}\footnote{\citeauthor{Conrad1998} gloss \textit{p}(\textit{w})\textit{e} as \lq remain' here, but mostly as \lq be' elsewhere. \textcite[33]{ConradWigoga1991} and the examples throughout their grammar show that this verb is in fact a locative and existential copula.}
		
	\ex\label{exAppendixBukiyip2}
	Context: People have worked away for several years and want to return to the village.\\
	\gll \textbf{Wotak} m-a-kli m-u-nak umu m-u-ne senis ume yohleguh ch-a-húl jah. Ch-a-lhwas apak-i-ch tuag-omi.\\
	still \textsc{subj}.1\textsc{pl}-\textsc{rl}-say/want \textsc{subj}.1\textsc{pl}-\textsc{irr}-go for \textsc{subj}.1\textsc{pl}-\textsc{irr}-do change for years \textsc{subj}.\textsc{ncl}8.\textsc{pl}-\textsc{rl}-take cargo \textsc{subj}.\textsc{ncl}8.\textsc{pl}-\textsc{rl}-run\_away 1\textsc{pl}-\textsc{poss}-\textsc{ncl}8.\textsc{pl} European-those\\
	\glt \lq When we had not yet changed work crews to replace those who had finished their yearly contract (lit. when we were still wanting to go and change) the Europeans took their cargo and ran away.\rq{} (\cite[205]{ConradWigoga1991}, glosses added)
	
	\ex\label{exAppendixBukiyip3}
	\gll Oli wo m-u-gabwe-yegas e. \textbf{Wotak} ny-a-pwe.\\
and \textsc{neg} \textsc{subj}.1\textsc{pl}-\textsc{irr}-fix-\textsc{obj} \textsc{neg} still \textsc{subj}.\textsc{ncl}8.\textsc{sg}-\textsc{rl}-\textsc{cop}\\
	\glt \lq And we haven't made peace yet -- the trouble is still with us.\rq{ }\parencite[186]{ConradWigoga1991}
\end{exe}


\subsubsection{Uses related to other phasal polarity concepts}
\paragraph{Not yet}\label{appendixBukiyipNotYet}
\begin{itemize}
	\item \textcite[28]{Conrad1998} and \textcite[41, 43, 98–99, 167]{ConradWigoga1991}.
	\item This function is attested in the absence of an overt predicate. More specifically,
	\begin{itemize}
		\item In disjunctive questions following a pattern \lq \textit{p} or still > \textit{p} or not yet\rq{ }(\ref{exAppendixBukiyipNotYet1}). These can be understood as instantiating a more general pattern of question formation \lq \textit{p} or \textsc{neg}\rq{ }(see \cite[4]{ConradWigoga1991}).		
		\item As a negative one-word reply to a polar question (\ref{exAppendixBukiyipNotYet2}).
		\item As a pro-sentence, adding to a negative proposition (\ref{exAppendixBukiyipNotYet3}).
	\end{itemize}
	\item There are two other attestations of \textit{wotak} in \citeauthor{ConradWigoga1991}'s (\citeyear{ConradWigoga1991}) grammar that are translated as \lq not yet', but where the negative appears to be an artefact of translation. The first case is (\ref{exAppendixBukiyip2}) above. The second case, involves \textit{anapu wotak malhwasia}, translated as \lq some of us still hadn't come' \parencite[205, ex. 737]{ConradWigoga1991}. Composition and context both suggest that this is literally \lq some of us were still running hither.'
\end{itemize}

\begin{exe}
	\ex\label{exAppendixBukiyipNotYet1}
	\gll Núgawikw yopu-kw o \textbf{wotak}?\\
	daughter:\textsc{ncl}4.\textsc{sg} healed-\textsc{ncl}4.\textsc{sg} or still\\
	\glt \lq Is the daughter healed, or not yet?' \parencite[99]{ConradWigoga1991}
	\pagebreak
	\ex\label{exAppendixBukiyipNotYet2}
	 Context: Someone has asked if the food is done.\\
	\gll \textbf{Wotak}.\\
	still\\
	\glt \lq Not yet.' \parencite[146–147]{Fortune1942}

	\ex\label{exAppendixBukiyipNotYet3}
	\gll M-a-túlúgún. Wak. \textbf{Wotak}.\\
	\textsc{subj}.1\textsc{pl}-\textsc{rl}-look\_for no still\\
	\glt \lq We looked for (the cows), but couldn’t find them, not yet.\rq{ }\parencite[167]{ConradWigoga1991}
\end{exe}

\subsubsection{Broadly adverbial temporal-aspectual uses}
\paragraph{Event sequencing \lq{}and then\rq}
\label{appendixBukiyipSequencing}
\begin{itemize}
	\item There are several instances of \textit{wotak} in \textcite[167]{ConradWigoga1991} that involve event sequencing (\lq and then').
	\item With many tokens in question, there is an element of culmination involved, in the form of a condition or preparatory event that facilitates the situation introduced by \textit{wotak}. 
\end{itemize}

\begin{exe}
	\ex\label{exAppendixBukiyipSequencing1}
	\gll Y-é-nak y-a-bih y-a-pwe, wak, ulku-m m-o-lú m-o-lali, \textbf{wotak} y-a-ltow-i. Y-a-kih-i.\\
	\textsc{subj}.1\textsc{sg}-\textsc{rl}-go \textsc{subj}.1\textsc{sg}-\textsc{rl}-go\_down \textsc{subj}.1\textsc{sg}-\textsc{rl}-\textsc{cop} no heart-\textsc{ncl}5.\textsc{sg} \textsc{subj}.\textsc{ncl}5.\textsc{sg}-\textsc{rl}-think \textsc{subj}.\textsc{ncl}5.\textsc{sg}-\textsc{rl}-think.\textsc{ven} still \textsc{subj}.1\textsc{sg}-\textsc{rl}-go\_up-\textsc{ven} \textsc{subj}.1\textsc{sg}-\textsc{rl}-arrive-\textsc{ven} \\
	\glt \lq I went, went down, and stayed there, and was not satisfied. I thought about returning and waited and then I came back up.' (\cite[168]{ConradWigoga1991}, glosses added)


	\ex\label{exAppendixBukiyipSequencing2}
	Context: A boy has climbed up a betel nut tree to look for betel nuts. Trying to split one open, he got stuck by a wasp. A bystander shouts.\\
	\gll Kw-autu-i anab \textbf{wotak} i-túl-úb, bú-b.\\
	\textsc{imp}-throw\_down-\textsc{ven} \textsc{indef}.\textsc{ncl}1.\textsc{sg} still \textsc{subj}.1\textsc{sg}:\textsc{irr}-see-\textsc{obj}.\textsc{ncl}1.\textsc{sg} betel\_nut-\textsc{ncl}1.\textsc{sg}\\
	\glt \lq Throw down one betel nut and I’ll [then] take a look at it.'\\ (\cite[233]{ConradWigoga1991}, glosses added)
	\pagebreak
	\ex\label{exAppendixBukiyipSequencing3}
	 \gll Énech ch-ú-lib wichap, énech, ch-ú-túk dagubés. Énech ch-ú-blo lowas, énech ch-ú-lak éménab, inap ch-ú-ne-stretimu yah étúh. Bai \textbf{wotak} ch-ú-tanomoli gen.\\
	\textsc{indef}.\textsc{ncl}8.\textsc{pl} \textsc{subj}.\textsc{ncl}8.\textsc{pl}-\textsc{irr}-cut grass \textsc{indef}.\textsc{ncl}8.\textsc{pl} \textsc{subj}.\textsc{ncl}8.\textsc{pl}-\textsc{irr}-take\_out bamboo\_roots \textsc{indef}.\textsc{ncl}8.\textsc{pl} \textsc{subj}.\textsc{ncl}8.\textsc{pl}-\textsc{irr}-cut trees \textsc{indef}.\textsc{ncl}8.\textsc{pl} \textsc{subj}.\textsc{ncl}8.\textsc{pl}-\textsc{irr}-build ground until \textsc{subj}.\textsc{ncl}8.\textsc{pl}-\textsc{irr}-do-fix\_up.\textsc{appl} road:\textsc{ncl}13.\textsc{sg} only:\textsc{ncl}13.\textsc{sg} \textsc{fut} still \textsc{subj}.\textsc{ncl}8.\textsc{pl}-\textsc{irr}-return:\textsc{ven} again\\
	\glt \lq Some of them will cut grass, some will take out bamboo roots, some will cut trees, some will smooth out the ground and they will continue until they have fixed up the road, and then they will return.' (\cite[126]{ConradWigoga1991}, glosses added)
\end{exe} 
\il{Bukiyip|)} 

\section{Chamorro (cha, cham1213)}
\il{Chamorro|(}
\subsection{Introductory remarks}
Chamorro has two candidates for \textsc{still} expressions: \textit{trabiha} and \textit{ha'}. Only for the latter there are indications for additional functions. Note that I gloss what \textcite{Chung2020} terms the \lq\lq progressive\rq\rq{ }aspect form as \textsc{cont} for \lq continuative\rq{}, as it is compatible with stative predicates.

\subsection{ha'}
\subsubsection{General information}
\begin{itemize}
	\item Wordhood: bound morpheme, enclitic (but separated from its host in spelling).
	\item Syntax: follows (i.e. hosted by) its focus.
\end{itemize}

\subsubsection{As a \lq{}still\rq{ }expression}
\begin{itemize}
	\item \textcite[513]{Chung2020}.
	\item Specialisation: in absence of contextualised data, examples like (\ref{exAppendixChamorro1}–\ref{exAppendixChamorro3}) give a fairly good indication that this marker conforms to my definition. For instance, (\ref{exAppendixChamorro1}) appears to not only evoke the continuation of speakers' being alive, but also contrast it with others who are deceased. Further, albeit indirect, evidence comes from the robustly attested restrictive-\textsc{still} polysemy and the semantic parallels between these functions (see \Cref{sectionExclusive}).
	\item Polarity sensitivity: in the data, there is only one example of \textit{ha'} as \textsc{still} in combination with negation, yielding \textsc{not yet} \parencite[603]{Chung2020}. \textsc{not yet} seems to be more commonly expressed via the inner negation of \textit{trabiha}.
	\item Pragmaticity: the data allow no conclusions.
\end{itemize}

\begin{exe}
	\ex \label{exAppendixChamorro1}
	\gll Hita ni manlåla'la' \textbf{ha'} para ta abiba.\\
	1\textsc{pl}.\textsc{incl} \textsc{rel}.\textsc{def} \textsc{pl}.\textsc{rl}:alive.\textsc{cont} still \textsc{fut} 1\textsc{pl}.\textsc{incl} encourage\\
	\glt \lq We (incl.) who are still living are going to encourage them.\rq{ }\parencite[167–168]{Chung2020}

	\ex \label{exAppendixChamorro2}
	\gll Mungnga hit manburuka mientras ki ma\sim{}maigu' \textbf{ha}' i neni.\\
\textsc{proh} 1\textsc{pl}.\textsc{incl} \textsc{pl}.make\_noise while \textsc{prep} \textsc{cont}\sim{}sleep still \textsc{def} baby\\
	\glt \lq Let's (incl.) not make noise while the baby is still sleeping.\rq{ }\parencite[344]{Chung2020}
	
	\ex \label{exAppendixChamorro3}
	\gll Hu ha\sim{}hassu \textbf{ha'} atyu na tiempu anai påpatgun yu'.\\
	1\textsc{sg}.\textsc{rl} \textsc{cont}\sim{}remember still \textsc{dist} \textsc{lnk} time \textsc{subord} child.\textsc{cont} 1\textsc{sg}\\
	\glt \lq I still remember the time when I was a child.' \parencite[18]{Chung2020}
\end{exe}

\subsubsection{Uses on the fringes of \lq{}still\rq{}}
\paragraph{Scalar contexts}
\label{appendixChamorroScalar}
\begin{itemize}
	\item There are a few examples of \textit{ha'} translated as \lq still' in \citeauthor{Chung2020} (\citeyear{Chung2020})'s grammar that involve scalar contexts of decreases over time.
	\item In all cases, \textit{ha'} follows (i.e. is attached to) the quantifier, which is also the main predicate (see \cite[324–328]{Chung2020} on quantifiers as predicates).
\end{itemize}

\begin{exe}
	\ex
\gll Bu\sim{}bula \textbf{ha'} tinanum ti hu tungu' na manmamakannu'.\\
	\textsc{cont}\sim{}many \textsc{still} plant \textsc{neg} 1\textsc{sg}.\textsc{rl} know \textsc{rel} \textsc{pl}.edible\\
	\glt \lq There are still many plants I did not know are edible.\rq{ }\parencite[325]{Chung2020}

	\ex
	\gll  Bu\sim{}bula  \textbf{ha’} tetehnan tano’-hu ti hu bebendi.\\
	\textsc{cont}\sim{}many \textsc{still} remaining.\textsc{lnk} land-1\textsc{sg} \textsc{neg} 1\textsc{sg} sell.\textsc{cont}\\
	\glt \lq There's still much of my remaining land that I'm not selling.\rq{ }\parencite[382]{Chung2020}
\end{exe}

\subsubsection{Restrictive (non-temporal)}
\paragraph{(Non-scalar) exclusive}
\label{appendixChamorroRestrictive}
\begin{itemize}
	\item \textcite[513–516]{Chung2020} and \textcite[216–217]{Topping1973}.
	\item As is typical of exclusive markers, this is a cluster of functions. These include restricting the reference of the predicate (\ref{exAppendixChamorroRestrictive1}, \ref{exAppendixChamorroRestrictive2}), noun phrases (\ref{exAppendixChamorroRestrictive3}), prepositional phrases (\ref{exAppendixChamorroRestrictive4}), or subordinators (\ref{exAppendixChamorroRestrictive5}).
\end{itemize}

\begin{exe}	\ex\label{exAppendixChamorroRestrictive1}
	\gll Siña u ma\sim{}maigu' \textbf{ha'}.\\
	can 3\textsc{sg}.\textsc{irr} \textsc{cont}\sim{}sleep still\\
	\glt \lq She can just be sleeping.' \parencite[39]{Chung2020}	

\ex\label{exAppendixChamorroRestrictive2}
	\gll Na’galilik \textbf{ha’} håfa i listu para na’-ta.\\
	 make\_roll still any \textsc{def} ready for food-1\textsc{pl}.\textsc{incl}\\
	 \glt \lq Just stir over whatever is ready for our (incl.) food.\rq{ }\parencite[201]{Chung2020}

\ex\label{exAppendixChamorroRestrictive3}
	\gll Fåtta un puntu \textbf{ha’} para u gånna i tes.\\
absent one point still \textsc{fut} 3\textsc{sg}.\textsc{irr} win \textsc{def} test\\
	\glt \lq He needed only one point (lit. only one point was missing) to pass the test.’ \parencite[514]{Chung2020}

\ex\label{exAppendixChamorroRestrictive4}
	\gll Para hami \textbf{ha}' esti na inetnun.\\
	for 1\textsc{pl}.\textsc{excl} still \textsc{prox} \textsc{lnk} group\\
	\glt \lq This gathering is only for us (excl.)' \parencite[514]{Chung2020}
		
\ex\label{exAppendixChamorroRestrictive5}
	\gll Pues gigun \textbf{ha'} måttu atyu gi hinasson-ña malågu.\\
	then as\_soon\_as still arrive \textsc{dist} \textsc{lcl} thought-3\textsc{sg} run\\
	\glt \lq Then just as soon as that thought came to his mind, he ran away.' \parencite[516]{Chung2020}
\end{exe}

\subsubsection{Broadly modal and interactional uses}
\paragraph{Concessive apodoses}
\label{appendixChamorroConcessiveConsequent}
\begin{itemize}
	\item \textit{Ha'} is repeatedly attested in the apodoses of concessive constructions.
	\item Syntax: in all relevant examples, \textit{ha'} attaches to the predicate and its arguments.
\end{itemize}

\begin{exe}
	\ex
	\gll Ni taimanu chinago’-mu, {bai hu} hanågui håo \textbf{ha’}.\\
	not any\_degree far-2\textsc{sg} {1\textsc{sg}.\textsc{rl}} go\_to 2\textsc{sg} still\\
	\glt \lq No matter how far you are, I will still come to you.\rq{ }\parencite[204]{Chung2020}

	\ex
	\gll Maseha un fa’håhafa håo, ya\sim{}ya-hu håo \textbf{ha'}.\\
	although 2\textsc{sg} make\_into\_what.\textsc{cont} 2\textsc{sg} \textsc{cont}\sim{}like-1\textsc{sg}.\textsc{rl} 2\textsc{sg} still\\
	\glt \lq Even though you're making yourself into something else, I'll still like you.' \parencite[335]{Chung2020}

	\ex
	\gll Si Julie para u guput \textbf{ha'} gi Sabalu maseha malålangu.\\
	\textsc{unmarked} J. \textsc{fut} 3\textsc{sg}.\textsc{irr} party still \textsc{lcl} Saturday even\_though \textsc{sg}.sick.\textsc{cont}\\
	\glt \lq Julie will still have her party on Saturday even though she’s sick.’ \parencite[398]{Chung2020}
\end{exe}
\il{Chamorro|)}

\section{Coastal Marind (mrz, hali1245)}\il{Marind, Coastal|(}
\label{appendixCoastalMarind}

\subsection{Introductory remarks}
Apart from descriptive materials, I also consulted the text collections by \citeauthor{Olsson2021a} (\citeyear*{Olsson2021a}, \citeyear*{Olsson2021b}). My understanding of the data has furthermore greatly profited from discussions with Bruno Olsson, who also helped with several glosses.

Note that Roman numerals in the glosses indicate gender, following the conventions in \textcite{Olsson2017}. In addition, Coastal Marind has a complex system of verb orientation marking. In broad strokes, this is a focus/actor marking system reminiscent of the Austronesian type (see \cite[ch. 10]{Olsson2017}). Lastly, several examples feature the \lq\lq contessive" prefix \textit{ap}-. This is a marker that is common with posture and motion verbs, often \lq\lq expresses[ing] placement or movement on a surface off the ground" \parencite[480]{Olsson2017}. It is, however, also lexicalised with other verbs, and its exact contribution is often hard to pin down; see \textcite[480–485]{Olsson2017}.

\subsection{ndom}
\subsubsection{General information}
\begin{itemize}
	\item Wordhood: free morpheme.
	\item Syntax: invariably in pre-verbal position.
	\item Further note: \textit{ndom} requires the \lq\lq neutral\rq\rq{ }orientation of the verb. Since this can be considered the unmarked case, I do not indicate it in the glosses. 
	\item Etymology: there is a homophonous adjective \textit{ndom} \lq bad', but it is unclear whether this is a chance resemblance or not.
\end{itemize}

\subsubsection{As a \lq{}still\rq{ }expression}
\begin{itemize}
	\item \textcite[302, 525–526]{Olsson2017}.
	\item Specialisation: that \textit{ndom} conforms to my definition is evident in examples like (\ref{exAppendixCoastalMarind1}–\ref{exAppendixCoastalMarind3}). For instance, in (\ref{exAppendixCoastalMarind1}), the women continue to be busy fishing, which is contrasted with the discontinuation of that activity at the boy's next visit.
	\item Pragmaticity:  seems compatible with both scenarios. Ex. (\ref{exAppendixCoastalMarind3}) features the unexpectedly late scenario.
	\item Polarity sensitivity: there are no attestations of \textit{ndom} with negation. This is in line with \citeauthor{Olsson2017}'s (\citeyear[525]{Olsson2017}) observation that there are no expressions for any other phasal polarity concepts.
\end{itemize}

\begin{exe}
	\ex\label{exAppendixCoastalMarind1}
	 Context: A boy in disguise has gone to look at women who are fishing.\\
	\gll Mahai a-d-ind-het epe \textbf{ndom} da-n-sanak-at. Tanama a-d-ind-i-het epe oso m-a-p-y-avasig.\\
	first \textsc{subord}-\textsc{pst}.\textsc{dur}:3\textsc{sg}.\textsc{a}-\textsc{dir}-be\_moving \textsc{dist} still \textsc{pst}.\textsc{dur}-3\textsc{pl}.\textsc{a}-search\_for\_fish-\textsc{stat} again \textsc{subord}-\textsc{pst}.\textsc{dur}:3\textsc{sg}.\textsc{a}-\textsc{dir}-\textsc{iter}-be\_moving \textsc{dist} beginning \textsc{obj}-3\textsc{sg}.\textsc{a}-\textsc{contessive}-2$|$3\textsc{pl}.\textsc{u}-go\_up\_from\_water\\
	\glt \lq The first time he went, they were still fishing. When he went there again, they began to go up from the beach.' \parencite[20]{Olsson2021a}\\

	\ex\label{exAppendixCoastalMarind2}
	Context: About the olden days.\\
	\gll Imimil nahe \textbf{ndom} d-a-ya-hwala.\\
	grown\_up grandparents still \textsc{pst}.\textsc{dur}-3\textsc{sg}.\textsc{a}-2|3\textsc{pl}.\textsc{u}-be\\
	\glt \lq When I was growing up the old-timers were still alive.'
	\\(\cite{Olsson2015}, glosses added)

	\ex\label{exAppendixCoastalMarind3}
	Context: The speaker is complaining. He has worked his way up in life and made a career in town. His younger siblings, however, remain in the village and continue to need financial support.\\
	\gll Nok a mandin menda-gheghay yogh, papes-patul nd-ak-ap-hyadih. namagha \textbf{ndom} k-a, sekolah sa-d-n-a, kuliah-la sa-d-n-a na-hwala. Namagha pegawai menda-b-n-in, \textbf{ndom} k-ak-e-hwaghib.\\
	1\textsc{sg} \textsc{top} long\_ago \textsc{ant}:1\textsc{sg}.\textsc{a}-look\_after 2\textsc{pl} small-boy \textsc{loc}-1.\textsc{a}-\textsc{contessive}-2|3\textsc{pl}.\textsc{u}.see now still \textsc{prs}-3\textsc{sg}.\textsc{a}:\textsc{cop}:\textsc{non}.\textsc{pst} school only-\textsc{pst}.\textsc{dur}-3\textsc{sg}.\textsc{a}:1.\textsc{u}-\textsc{aux} (go\_to)university-\textsc{stat} only-\textsc{pst}.\textsc{dur}-3\textsc{sg}.\textsc{a}:1.\textsc{u}-\textsc{aux} 1.\textsc{u}-be now official \textsc{ant}-\textsc{emph}:3\textsc{sg}.\textsc{a}-1.\textsc{u}-become still \textsc{prs}-1.\textsc{a}-2|3\textsc{pl}.\textsc{dat}-put\_away:III.\textsc{u}\\
	\glt \lq I've been caring for you a long time, I was look after you as a small boy. Now I still do, I went to school, I went to university. Now I'm a civil servant, I'm still taking care of things for you.' (\cite{Olsson2015}, glosses added)
\end{exe}

\subsubsection{Uses on the fringes of \lq{}still\rq{}}
\paragraph{Scalar contexts}\label{appendixCoastalMarindScalar}
\begin{itemize}
	\item \textit{Ndom} is attested with decreases over time.
\end{itemize}

\begin{exe}
	\ex \label{exAppendixCoastalMarindDecrement1}
	\gll Bensin sam \textbf{ndom} k-a-nam.\\
	gasoline big still \textsc{prs}-3\textsc{sg}.\textsc{a}-1.\textsc{gen}\\
	\glt \lq I still have a lot of gasoline.' \parencite[172]{Olsson2017}
	
	\ex Context: Before the arrival of the Dutch, there used to be many small hamlets.\\
	\textit{Epe tendanap-ghahway ehe milah ehe.}
	\\ \lq They gathered the small hamlets into villages.\rq{}
	\\ \gll Isi \textbf{ndom} d-a-ya-hwala hyakod ti. hyakod ti \textbf{ndom} d-a-ya-hwala.\\
other still \textsc{pst}.\textsc{dur}-3\textsc{sg}.\textsc{a}-2|3.\textsc{pl}.\textsc{u}-be one with one with still \textsc{pst}.\textsc{dur}-3\textsc{sg}.\textsc{a}-2|3.\textsc{pl}.\textsc{u}-be\\
\glt \lq There were a few still left. A few were still left.\rq{ }(\cite{Olsson2015}, glosses added)
\end{exe}

\subsubsection{Additive and related uses}
\paragraph{Additive}\label{appendixCoastalMarindAdditive}
\begin{itemize}
	\item \textcite[115, 525–526]{Olsson2017}.
	\item This use predominantly occurs in motion contexts (going along or bringing along), where it bleeds into the comitative domain (\ref{exAppendixCoastalMarindAlso1}, \ref{exAppendixCoastalMarindAlso2}). Note that in these uses, \textit{ndom} is a common one-word answer (\ref{exAppendixCoastalMarindAlso2}).
	\item Other than suggested by \textcite[525]{Olsson2017}, additive \textit{ndom} is not completely restricted to motion contexts; this is shown in (\ref{exAppendixCoastalMarindAlso3}).
\end{itemize}

\begin{exe}
	\ex\label{exAppendixCoastalMarindAlso1}
	\gll Namagha nok mend-am-b-euma<n>ah, Iwoni \textbf{ndom}, patul d-a-ola, imimil-patul menda-b w-in.\\
	now 1 1.\textsc{a}-\textsc{a}-\textsc{appl}-go<1.\textsc{u}> I. still boy \textsc{dur}-3\textsc{sg}.\textsc{a}-\textsc{cop}:3\textsc{sg}.\textsc{u} grown\_up-boy \textsc{ant}:3\textsc{sg}.\textsc{a}-be 3\textsc{sg}.\textsc{u}-become\\
	\glt \lq Now we already left, Iwoni too, he was a boy, a big boy already.'
	\\(\cite{Olsson2015}, glosses added)

\ex\label{exAppendixCoastalMarindAlso2}
	\begin{xlist}
		\exi{A:} \gll E, epe anem epe \textbf{ndom} d-a-ghet ay?\\
		\textsc{interj} \textsc{dist} man \textsc{dist} still \textsc{pst}.\textsc{dur}-3\textsc{sg}.\textsc{a}-be\_moving \textsc{q}\\
		\glt \lq Oh, so he also went?'
		\exi{B:} \gll \textbf{Ndom}.\\
		still\\
		\glt \lq He too!' (\cite{Olsson2015}, glosses added; also see \cite[115]{Olsson2017})
	\end{xlist}
	
	\ex\label{exAppendixCoastalMarindAlso3}
	Context: The previous day, people set up troughs and prepared sago.\\
	\gll Kwemek ipe tanama ipe, \textbf{ndom} na-kahahib ipe.\\
	morning \textsc{dist} again \textsc{dist} still 3\textsc{pl}.\textsc{a}-fasten:IV.\textsc{u} \textsc{dist}\\
	\glt \lq The day after, they also prepared the troughs.\rq{ }(\cite{Olsson2015}, glosses added)
\end{exe}

\paragraph{The collocation \textit{ inah ndom}}
\begin{itemize}
	\item There are five attestations of a collocation \textit{inah ndom} \lq still two' in \citeauthor{Olsson2015}'s (\citeyear{Olsson2015}) corpus that cannot be explained by the phasal polarity or additive functions. Two of these are given in (\ref{exAppendixCoastalMarindInahNdom1}, \ref{exAppendixCoastalMarindInahNdom2}).
	\item The contribution of this collocation is not entirely clear. Three of the attestations, including (\ref{exAppendixCoastalMarindInahNdom1}), involve information given in an afterthought-like manner.
	\item Bruno Olsson (p.c.) suggests a conceptual link to additive uses as in (\ref{exAppendixCoastalMarindAlso1}, \ref{exAppendixCoastalMarindAlso2}) above. Under this interpretation, in (\ref{exAppendixCoastalMarindInahNdom1}) the speaker adds \textit{inah ndom} to stress \lq [in fact], two of them [not just one]'. Similarly, the addition of \textit{Iwoni ndom} in (\ref{exAppendixCoastalMarindAlso1}) emphasises \lq [not just them, in fact], Iwoni also [went]'.
\end{itemize}

\begin{exe}
	\ex \label{exAppendixCoastalMarindInahNdom1}
	Context: Preparing gifts for the children's teachers.\\
	\gll Ihe guru nanggo ma-n-de-og epe sep epe, \textbf{inah} \textbf{ndom}.\\
	\textsc{prox} teacher for \textsc{obj}-1.\textsc{a}-1\textsc{pl}-give \textsc{dist} leaf\_oven \textsc{dist} two still\\
	\glt \lq We made the sep [leaf ovens] for the teachers, two of them.' 	\\(\cite{Olsson2015}, glosses added)

	\ex \label{exAppendixCoastalMarindInahNdom2}
	Context: The speaker has been very hungry.\\
	\gll Onggat \textbf{inah} \textbf{ndom} ma-n-od-ghi hyahyak-onggat, epe t-e-nakap-isik.\\
	coconut two still \textsc{obj}-1.\textsc{a}-\textsc{dur}-eat split-coconut  \textsc{dist} 1.\textsc{dat}-before-\textsc{contessive}-become\_full \\
		\glt \lq I ate two coconuts, ripe ones, until I was full.\rq{ }(\cite{Olsson2015}, glosses added)\\
\end{exe}
\il{Marind, Coastal|)}

\section{Iatmul (ian, iatm1242)}
\label{appendixIatmul}\il{Iatmul|(}
\subsection{Introductory remarks}
Apart from descriptive materials, I consulted \citeauthor{Jandraschek2008}'s (\citeyear{Jandraschek2008}) text collection. My understanding of the data was furthermore greatly aided by discussions with Gerd Jandraschek, who also helped with glosses.

\subsection{wata}

\subsubsection{General information}
\begin{itemize}
	\item Wordhood: independent grammatical word, invariable.
	\item Syntax: invariably precedes the predicate.
\end{itemize}

\subsubsection{As a \lq{}still\rq{ }expression}
\begin{itemize}
	\item \citeauthor{Jandraschek2007} (\citeyear[24]{Jandraschek2007}; \citeyear[218, 550]{Jandraschek2012}).
	\item Specialisation: textual examples give evidence that \textit{wata} conforms to my definition. For instance, in (\ref{exAppendixIatmul3}) the first instance of \textit{wata} serves to frame the situational background as a continuation of the ongoing fight. At the same time, it anticipates the soon-to-occur end of the battle. Likewise, the second token in (\ref{exAppendixIatmul3}) can be interpreted as evoking an (expectable) alternative scenario in which the grandfather took notice of the ongoing battle and stopped working. Further, albeit indirect, evidence comes from its uses as \textsc{not yet} in the absence of negation (\appref{appendixIatmulNotYet}).
\item Pragmaticity: appears to be compatible with both scenarios.
\item Polarity sensitivity: inner negation yields \textsc{not yet}.
\end{itemize}
\pagebreak
\begin{exe}
	\ex \label{exAppendixIatmulq}
 	Context: Recollecting the arrival of the German colonisers. The Iatmul tried to fight them off with spears.\\
	 \gll \textbf{Wata} yi-li’-di yi-ka li’-ka yi-ka agiyabak kan walaga-kak bun-kak Kabadi’mali-kak di’ wan waliya-li’-j-a-vaak ana vi’-di’ di’ \textbf{wata} vaala viya-li’-ka-di’.\\
 still go-\textsc{ipfv}-3\textsc{pl} go-\textsc{dep} stay-\textsc{non}.\textsc{fin} go-\textsc{non}.\textsc{fin} that's\_all \textsc{prox.sg.m} grand\_grand\_father-\textsc{dat} ???-\textsc{dat} K.-\textsc{dat} 3\textsc{sg.m} \textsc{dem.anaph:sg.m} fight-\textsc{ipfv}-3\textsc{pl}-\textsc{subord}-\textsc{nmlz} \textsc{neg} see-3\textsc{sg.m} 3\textsc{sg.m} still canoe hit-\textsc{ipfv}-\textsc{prs}-3\textsc{sg.m}\\
	\glt \lq They continued to throw spears, my grand-grand-father Kabadumali, he did not see them fighting, he was still working on a canoe.' (He would then be hit by a bullet, leading the Iatmul people to abandon their resistance.) (\cite[54]{Jandraschek2008}, glosses added)

	\ex \label{exAppendixIatmul2}
	 \gll \textbf{Wata} ki'-li'-ka-li'.\\
	still eat-\textsc{ipfv}-\textsc{prs}-3\textsc{sg.f}\\
	\glt \lq She is still eating.' \parencite[218]{Jandraschek2012}

	\ex \label{exAppendixIatmul3}
	 \gll Kiʼkiʼda \textbf{wata} alaku yi-li'-ka-di'.\\
	food still hot go-\textsc{ipfv}-\textsc{prs}-3\textsc{sg.m}\\
	\glt \lq The food is still hot.' \parencite[295]{Jandraschek2012}
\end{exe}

\subsubsection{Uses related to other phasal polarity concepts}
\paragraph{Not yet} \label{appendixIatmulNotYet}
\begin{itemize}
	\item \citeauthor{Jandraschek2007} (\citeyear[24]{Jandraschek2007}; \citeyear[218, 550]{Jandraschek2012}).
	\item In this function \textit{wata} serves as a pro-sentence or interjection.
	\item This use covers a broad spectrum of nuances: \lq not yet', \lq wait one moment', \lq it doesn't end here', etc. (\cite[550]{Jandraschek2012}; Gerd Jandraschek, p.c.)
\end{itemize}

\begin{exe}
	\ex
	\gll \textbf{Wata}!\\
	still\\
	\glt \lq Wait a second!' \parencite[218]{Jandraschek2012}

	\ex
	\gll
	Wa-di’ \textbf{wata} \textbf{wata} wa-laa valaya-di’ nya’ik wupmâ a-li’ valaya-di’ wan di’n-a	nyaan taba.\\
say-3\textsc{sg.m} still still	 say-\textsc{consec} exit-3\textsc{sg.m} father \textsc{dem.anaph.adv} \textsc{imp}-stay exit-3\textsc{sg.m} \textsc{dem}.\textsc{anaph}:\textsc{sg}.\textsc{m} 3\textsc{sg.m}-\textsc{gen} child already\\
\glt \lq His son said, \lq\lq Wait, wait, father, stay there!", and he went out.' (\cite[54]{Jandraschek2008}, glosses added)
	\pagebreak
	\ex Context: About a rite of passage. Long spears were thrown at a target.\\
	\gll Sukku-a  nyaan-kak \textbf{wata} kan nyaan wata ana yi-kiya-di’, wa-a li’-di.\\
	miss-\textsc{subord}	child-\textsc{dat} still \textsc{prox.sg.m} child still \textsc{neg} go-\textsc{irr}-3\textsc{sg.m} say-\textsc{dem.fin} stay-3\textsc{sg.m}\\
	\glt \lq About someone who missed they would say that he will not go yet.' (lit. …about someone who missed  [they would say] not yet …)' (\cite[31]{Jandraschek2008}, glosses added)
\end{exe}
\il{Iatmul|)}

\section{Kalamang (kgv, kara1499)}\il{Kalamang|(}
\label{appendixKalamang}

\subsection{Introductory remarks}
Apart from descriptive materials, I consulted \citeauthor{Visser2021b}'s (\citeyear*{Visser2021b}) corpus. My understanding of the data has furtheremore greatly profited from discussion with Eline Visser, who also helped with some tricky glosses.
\subsection{tok}

\subsubsection{General information}
\begin{itemize}
	\ex Wordhood: free morpheme.
	\ex Syntax: fixed position,  following the subject NP, preceding object NPs and the predicate. There are a few instances in \citeauthor{Visser2021b}'s (\citeyear*{Visser2021b}) corpus in which \textit{tok} embraces the subject, as in (\ref{exAppendixKalamang1}) below. In all likelihood, these are false starts (Eline Visser, p.c.). 
\end{itemize}

\subsubsection{As a \lq{}still\rq{ }expression}
\begin{itemize}
		\item \citeauthor{VisserKalamangDictionary} (\citeyear*{VisserKalamangDictionary}, \citeyear[354–358]{Visser2022}).
		\item Specialisation: examples like (\ref{exAppendixKalamang1}–\ref{exAppendixKalamang3}) clearly identify \textit{tok} as a \textsc{still} expression. For instance, in (\ref{exAppendixKalamang1}) the logging company not only continued to exist at topic time but this state is also contrasted with its discontinuation at the time of speech. Further, albeit indirect evidence comes from \textit{tok}'s use as \textsc{not yet} without a negator (\appref{appendixKalamangNotYet}).
		\item Pragmaticity: appears to be compatible with both scenarios. The vast majority of tokens in \citeauthor{Visser2021a} (\citeyear*{Visser2021a}, \citeyear*{Visser2021b}) belong to the neutral scenario. If (and which) additional marking is used to make the unexpectedly late scenario explicit is an open question.
		\item Polarity sensitivity: inner negation yields \textsc{not yet}.
		\item Further notes: a token like (\ref{exAppendixKalamang4}) might also be considered an instance of the \lq first, for now\rq{ }use (\appref{appendixKalamangFirst}). \textit{Tok} can serve as an elliptical pro\hyp sentence; see (\ref{exAppendixKalamang5}).
\end{itemize}

\begin{exe}
	\ex\label{exAppendixKalamang1}
	Context: The narrator and his friends have speared a fish. There used to be a logging company (now defunct) in the village.\\
	\gll Kan waktu ma me <tok-> perusahan \textbf{tok} metko mambon. Mu sor opa kon=a mu toni tabai-nggi kosalir=et\\
		\textsc{interj} time 3\textsc{sg} \textsc{top} still company still there exist 3\textsc{pl} fish \textsc{dem}.\textsc{anaph} one=\textsc{foc} 3\textsc{pl} say tobacco-\textsc{inst} change=\textsc{irr}\\
	\glt \lq At that time, the company was still there. They saw the fish, they said they wanted to exchange (it) with tobacco.\rq{ }(\cite{Visser2021b}, glosses added) 
	
	\ex\label{exAppendixKalamang2}
	Context: The opening line of an expository text about times past.\\
	\gll An-dain, waktu an \textbf{tok} cicaun, an me an kanggeir-an hanya.\\
		1\textsc{sg}-alone time 1\textsc{sg} still be\_small 1\textsc{sg} \textsc{top} 1\textsc{sg} play-1\textsc{sg} only\\
	\glt \lq As for me, when I was [still] young, I played like this.\rq{ }(\cite{Visser2021b}, glosses added)
		
	\ex\label{exAppendixKalamang3}
	Context: Planning a trip to the city to attend a training program. For the program to take place, funds have to be secured first.\\
	\gll An pitis=at gerket=ta ba mu toni pitis tok\sim{}tok saerak mu toni mu \textbf{tok} pitis=at komer=et.\\
		1\textsc{sg} money=\textsc{obj} ask=\textsc{non}.\textsc{fin} but 3\textsc{pl} say money still\sim\textsc{redupl} \textsc{neg}.\textsc{exist} 3\textsc{pl} say 3\textsc{pl} still money=\textsc{obj} see=\textsc{irr}\\
	\glt \lq I ask for money but they say there is no money yet, they're still looking for money.' (\cite{Visser2021b}, glosses added)

	\ex\label{exAppendixKalamang4}
	Context: The narrator was constructing a canoe.\\
	\gll An se ewun=at kies ah \textbf{tim}=\textbf{at} \textbf{an} \textbf{tok} \textbf{mamun}.\\
	1\textsc{sg} already trunk=\textsc{obj} carve \textsc{interj} edge=\textsc{obj} 1\textsc{sg} still leave\\
	\glt \lq I carved (cut off) the base, I still [for the time being] left the tips.\rq{ }\parencite{Visser2021b}
	
	\ex \label{exAppendixKalamang5}
	\gll Ka tok sekola? – \textbf{Tok}.\\
	2\textsc{sg} still go\_to\_school {} still\\
	\glt \lq Do you still go to school? -- Yes [I still go to school].\rq{ }\parencite[355]{Visser2022}
\end{exe}

\subsubsection{Uses on the fringes of \lq{}still\rq{}}
\paragraph{Scalar contexts}\label{appendixKalamangScalar}
\begin{itemize}
	\item Several tokens in the data, including (\ref{exAppendixKalamangDecrement1}, \ref{exAppendixKalamangDecrement2}) involve scalar contexts, more precisely monotone decreases.
	\item Note	that (\ref{exAppendixKalamangDecrement1}) involves \textit{kodak} \lq just one\rq{} < \textit{kon}-\textit{tak} \lq one-just\rq{ }and in (\ref{exAppendixKalamangDecrement2}) what is left of the giant is marked by the focus clitic \mbox{=\textit{a}}. That is, in both cases the remaining value clearly constitutes the sentence focus.
\end{itemize}

\begin{exe}
	\ex\label{exAppendixKalamangDecrement1}
	Context: Several pouches of tobacco have been smoked.\\
	\gll Pas kosom bo koyet pas kodak \textbf{tok}.\\
	exactly smoke go finish exactly just\_one still\\
	\glt \lq When they finished smoking [there was] still one [pouch].\rq{ }\parencite[9]{Visser2021a}

	\ex\label{exAppendixKalamangDecrement2}
	Context: A giant has been killed.\\
	\gll Mu he din=at uw=i koyet mu he di=sara karuar keitko na na na na na na na mindi bo tinggal elkin-un=a \textbf{tok}.\\
	3\textsc{pl} already fire=\textsc{obj} kindle=\textsc{lnk} finish 3\textsc{pl} already \textsc{caus}=ascend smoke\_dry above consume consume consume consume consume consume consume like\_that go remain ballsack-\textsc{poss}.3=\textsc{foc} still\\
	\glt \lq After kindling the fire, they put him up the drying rack, ate and ate and ate until only his ballsack was still [there].\rq{ }(\cite{Visser2021b}, glosses added)
\end{exe}



\subsubsection{Uses related to other phasal polarity concepts}
\paragraph{Not yet}\label{appendixKalamangNotYet}
\begin{itemize}
	\item \citeauthor{VisserKalamangDictionary} (\citeyear*{VisserKalamangDictionary}, \citeyear*[355–356]{Visser2022}).
	\item This function occurs in the absence of an overt predicate, namely 
	\begin{itemize}
		\item In polar questions following a pattern \lq already \textit{p} or still > already \textit{p} or not yet', as in (\ref{exAppendixKalamangNotYet1}). This can be understood as a specific instantiation of a broader pattern of question formation \lq \textit{p} or \textsc{neg}' (see \cite[301–302]{Visser2022}).
		\item In replies to questions and statements featuring \textit{se}/\textit{he} \lq already', where \textit{tok} is the only possible negative answer. This is illustrated in (\ref{exAppendixKalamangNotYet2}, \ref{exAppendixKalamangNotYet3}). Note that if the question contains \textit{tok} as \textsc{still}, \textit{tok} as a one-word answer is also interpreted as \textsc{still}; see (\ref{exAppendixKalamang4}) above.
		\item As holophrase. In these cases, \textit{tok} is often reduplicated; see (\ref{exAppendixKalamangNotYet4}).
		\pagebreak
		\item In contexts involving contrastive sets; this is illustrated in (\ref{exAppendixKalamangNotYet5}). In this use, \textit{tok} is also often reduplicated, as in the second instance in (\ref{exAppendixKalamangNotYet5}).
	\end{itemize}
\end{itemize}
	
\begin{exe}
	\ex \label{exAppendixKalamangNotYet1}
	\gll Se koluk ye \textbf{tok}?\\
	already find or still\\
	\glt \lq Did you find it yet or not?' \parencite{Visser2021b}

	\ex\label{exAppendixKalamangNotYet2}
	\gll Ka he namgon? – \textbf{Tok}.\\
	2\textsc{sg} already married(\textsc{f}) {} still\\
	\glt \lq Are you married? -- Not yet.' \parencite[356]{Visser2022}
		
	\ex\label{exAppendixKalamangNotYet3}
	Context: A black-haired monkey is trapped in a cage and waiting for his body to turn white.\\
		\gll An=at kahatmei eran-an se iren. – O kusukusu toni \textbf{tok} nakal-ca tok kuskap=ta ime.\\
		1\textsc{sg}=\textsc{obj} open.\textsc{imp} body-\textsc{poss}.1\textsc{sg} already white {} \textsc{emph} cuscus say still head-\textsc{poss}.2\textsc{sg} still black=\textsc{non}.\textsc{fin} \textsc{dist}\\
		\glt \lq [Monkey:] \lq\lq Release me, my body is white [already]." -- The cuscus said \lq\lq Not yet, your head is still black."' (\cite[222]{Visser2021a}, \citeyear[356]{Visser2022})

	\ex\label{exAppendixKalamangNotYet4}
	Context: The addressee wants to start weaving a basket.\\
	\gll \textbf{Tok\sim{}tok} \textbf{tok} pi tok masarur=et.\\
	\textsc{redupl}\sim{}still still 1\textsc{pl}.\textsc{incl} still tear=\textsc{irr}\\
	\glt \lq Not yet, we're still ripping (the pandanus leaves, to prepare the material with which we will weave the basket).' (\cite{Visser2021b})

	\ex \label{exAppendixKalamangNotYet5}
	Context: People are taking turns in repairing a roof.\\
	\gll Ma he gerket \lq\lq{}naman=a \textbf{tok}?\rq\rq{} An \textbf{tok\sim tok} an=a ming yuonin=in.\\
3\textsc{sg} already ask \phantom{\lq\lq}who=\textsc{foc} still 1\textsc{sg} still\sim \textsc{redupl} 1\textsc{sg}=\textsc{foc} oil rub=\textsc{neg}\\
\glt \lq He asked \lq\lq{}Who hasn’t yet [taken their turn]?\rq\rq{ }I hadn’t yet, I didn’t rub oil.' (\cite{Visser2021b}, glosses added)
\end{exe}

\subsubsection{Broadly adverbial temporal-aspectual functions}
\paragraph{First, for now}\label{appendixKalamangFirst}
\begin{itemize} 
	\item  \citeauthor{VisserKalamangDictionary} (\citeyear*{VisserKalamangDictionary}, \citeyear[354]{Visser2022}).
	\item In some instances of this function, \textit{tok} combines with other expressions for \lq first'. For instance in (\ref{exAppendixKalamangFirstBorani}) it is followed by a verb \lq be first, be in front', and in (\ref{exAppendixKalamangFirstPertama}) it is preceded by the adverb \textit{perama} \lq first' (a loan from Malay).
\end{itemize}

\begin{exe}
	\ex 
	Context: Nyong’s mother has gone landwards.\\
	\gll Kan Nyong emun	 \textbf{tok} bo=ta opa me … Nyong	esun	tok bot=nin\\
	\textsc{interj} N. mother:\textsc{poss}.3\textsc{sg} still go=\textsc{non}.\textsc{fin} \textsc{dem}.\textsc{anaph} \textsc{top} {} N. father:\textsc{poss}.3\textsc{sg} still go=\textsc{neg}\\
	\glt \lq Right, Nyong's mother went first, Nyong's father didnʼt yet go.' (i.e. for now only Nyong’s mother went, but if circumstances change, his father might join her) (\cite[220]{Visser2022}; Eline Visser, p.c.)


	\ex\label{exappendixKalamangFirst1}
	 Context: The protagonists have decided to set out for a trip.\\
	\gll Go yuol=ta me, wa me o, hari sabtu me \textbf{tok}, mu \textbf{tok} doa salamar=at paruo, fibir-un.\\
	condition day=\textsc{non}.\textsc{fin} \textsc{top}  \textsc{prox} \textsc{top} \textsc{emph} day saturday \textsc{top} still, 3\textsc{pl} still prayer good\_wish=\textsc{obj} do/make fibre-\textsc{poss}.3\\
	\glt \lq The next day, Saturday, they first, they first did the good wish prayer, for their fibre boat.' \parencite{Visser2021b}

	\ex\label{exAppendixKalamangFirstBorani}
	Context: About nutmeg.\\
	\gll Bunga-un me harga-un main me tang-un=at lebe. Jadi kalo bisa pi \textbf{tok} \textbf{boran=i} bunga-un=at parein\\
	flower-\textsc{poss}.3 \textsc{top} price-\textsc{poss}.3 \textsc{poss}.3\textsc{sg} \textsc{top} seed-\textsc{poss}.3=\textsc{obj} exceed so if can 1\textsc{pl}.\textsc{incl} still be\_first=\textsc{lnk} flower-\textsc{poss}.3\textsc{sg}=\textsc{obj} sell\\
	\glt \lq As for its flowers, its price is higher than [that for] its seeds. So if [we] can, we sell the flowers first.' \parencite[34–35]{Visser2021a}
	
	\ex\label{exAppendixKalamangFirstPertama}
	Context: The opening line of an expository text about cultivating nutmeg.\\
	\gll Sayang \textbf{pertama} \textbf{tok} go=at masir=et\\
	nutmeg first still place=\textsc{obj} weed=\textsc{irr}\\
	\glt \lq Nutmeg: first, weed a place.' \parencite[31]{Visser2021a}
\end{exe}


\subsubsection{Marginality}\label{appendixKalamangMarginal}
\begin{itemize}
	\item There are only two clear-cut example in the data. In ex. (\ref{exAppendixKalamangMarginal1}) it overlaps with a phasal polarity reading (the tide/sea level being a function of time).
\end{itemize}
\largerpage
\begin{exe}
	\ex\label{exAppendixKalamangMarginal1}
	Context: A boat trip around an island.\\
	\gll Mindi warkin laur warkin kararak=tauna o {get me} tiri osew=ar=a pareir=et. Wa me \textbf{tok} bisa.\\
	like\_that tide rising tide dry=so \textsc{emph} {if not} sail beach=\textsc{obj}=\textsc{foc} follow=\textsc{irr} \textsc{prox} \textsc{top} still can\\
	\glt \lq Like that the tide is low, if not we’d sail following the beaches. This is still OK (lit: this, [we] still can).' \parencite{Visser2021b}
	
	\ex\label{exAppendixKalamangMarginal2}
	Context: About small sinkers on a fishing net.\\
	\gll {Kacan panjang}=barak \textbf{tok} temun weinun.\\
	long\_bean=too/even still big too\\
	\glt \lq Long beans are even big (compared to the sinkers).' (i.e. even long beans still qualify as big when compared to these sinkers)
	\\(\cite{Visser2021b}, glosses added)
\end{exe}
	
\subsubsection{Broadly modal and interactional uses}
\paragraph{Polite commands}\label{appendixKalamangPoliteness}
\begin{itemize}
	\item In directive speech acts, \textit{tok} can serve as a hedge (Eline Visser, p.c.), which clearly goes back to its \lq first, for now\rq{ }use (\appref{appendixKalamangFirst}).
\end{itemize}

\begin{exe}
	\ex\label{exAppendixKalamangFirstImperative}
	\gll Ma he min ma he mat nawarar ka \textbf{tok} parar=te.\\
	3\textsc{sg} already sleep 3\textsc{sg} already 3\textsc{sg}.\textsc{obj} wake\_up 3\textsc{sg} still get\_up=\textsc{non}.\textsc{fin}\\
	\glt \lq He slept, [I] woke him up, \lq\lq You get up! (before you do anything else)"' (\cite{Visser2021a}; Eline Visser, p.c.)
\end{exe}
\il{Kalamang|)}
 
\section{Kewa (kjs/kew, east2516/west2599)}
\label{appendixKewa}\il{Kewa|(}
\subsection{Introductory remarks}
The data encompass East Kewa (kjs, east2516) and West Kewa (kew, west2599). According to \textcite{Franklin1968} these are mutually intelligible varieties of one language.

\subsection{pa}
\subsubsection{General information}
\begin{itemize}
	\item Wordhood: unclear. Either a free morpheme or a proclitic.
	\item Syntax: the data suggest that \textit{pa} precedes the associated constituent.
	\item Etymology: Kewa also has two verbs \textit{pa} \lq go', \lq do' as well as a \lq\lq completive" aspect marker \textit{pa}, and a disjunctive coordinator \textit{pa} \lq or'. From the available data, it is not clear if i) they are true homphones and ii) they are related to each other.
\end{itemize}

\subsubsection{As a \lq{}still\rq{ }expression}
\begin{itemize}
	\item Specialisation: \textit{pa} is primarily a restrictive marker (\appref{appendixKewaRestrictive}). Its use as a \textsc{still} expression is not explicitly discussed in the literature, although examples with the translation \lq still' abound. I haven taken cases like (\ref{exAppendixKewa1}–\ref{exAppendixKewa3}) as evidence that the phasal polarity concept is among the denotata of \textit{pa}. For instance, in (\ref{exAppendixKewa1}), the continuation of Kodopea being alive is contrasted with the death (hence, alternative scenario) of other relatives of the speaker. Further, albeit indirect, evidence comes from the cross-linguistically robust restrictive–\textsc{still} polysemy and the clear semantic parallels between the two functions (see \Cref{sectionExclusive}).
	\item Pragmaticity: the available data does not allow any definite conclusions. (\ref{exAppendixKewa2}) suggests that the unexpectedly late scenario can be made explicit by additional material such as \textit{abi pege} \lq even now'.
	\item Polarity sensitivity: no negated examples of phasal polarity \textit{pa} as \textsc{still} in the data.
\end{itemize}

\begin{exe}
	\ex \label{exAppendixKewa1}
	Context: About clan history.\\
	\gll Paga Waimi-lopo-re koma-pe. Kodopea-re \textbf{pa} pi-a. Ee, Oge-re komi-sa-yaa.\\
	P. W.-\textsc{du}-\textsc{top} die-\textsc{imm.pst}:3\textsc{du} K.-\textsc{top} still sit-\textsc{prs}:3\textsc{sg} Yes, O.-\textsc{top} die-\textsc{dist.pst}:3\textsc{sg}-\textsc{evid}\\
	\glt \lq Paga and Waimi died. Kodopea is still alive. Yes, Oge was reported to have died.' \parencite[345]{Yarapea2006}

	\ex \label{exAppendixKewa2}
	Context: About a debt from long ago.\\
	\gll Lobe Kabisimi-lopo-na mena gawa abi pege \textbf{pa} a-ya\\
L. K.-\textsc{du}-\textsc{gen} pig cow now even still stand-\textsc{prs}:3\textsc{sg}\\
	\glt \lq Even now Lobe and Kabisimi’s cow debt is still there.\rq{ }\parencite[336]{Yarapea2006}

	\ex \label{exAppendixKewa3}
	\gll Go naaki-ri adu \textbf{pa} na la-aya\\
	\textsc{dem} boy-\textsc{top} breast still eat stand-\textsc{prs}:3\textsc{sg}\\
	\glt \lq That boy is still breast-fed (lit. still eats breast).\rq{ }(\cite[7]{Franklin2007}, glosses added)\footnote{Kewa \lq\lq continuative" aspect describes intermittently recurrent situations; see \textcite[243–244]{Yarapea2006}.}
\end{exe}
\pagebreak
\subsubsection{Restrictive (non-temporal)}
\paragraph{(Non-scalar) exclusive}\label{appendixKewaRestrictive}
\begin{itemize}
	\item \citeauthor{Franklin1971} (\citeyear[69]{Franklin1971}, \citeyear[42]{Franklin2007}) and \textcite[82]{Yarapea2006}.
	\item This is an umbrella term for a range of conceptually related functions, in all of which \textit{pa}'s focus is the main predicate of the clause. These functions include narrowing down the reference of the predicate (\lq do nothing but'), as in (\ref{exAppendixKewaRestrictive1}, \ref{exAppendixKewaRestrictive2}), depicting an act as effortless (\ref{exAppendixKewaRestrictive3}), or doing something without reason (\ref{exAppendixKewaRestrictive4}). With many attestations, the exact contribution of \textit{pa} is hard to determine.
\end{itemize}

\begin{exe}
	\ex \label{exAppendixKewaRestrictive1}
	\gll \textbf{Pa} piru aa-lua koe le sa pi\\
still stay stand.\textsc{dur}-\textsc{fut}:1\textsc{sg} bad thing put sit.\textsc{prs}:1\textsc{sg}\\
	\glt \lq (If) I don’t say something (lit: just stay) I have put something valueless.' \parencite[311–312]{Yarapea2006}

	\ex \label{exAppendixKewaRestrictive2}
	\gll Oro kóko na-re-a pare \textbf{pa} ogépú kegaapú pe-a\\
really cold \textsc{neg}-emit-\textsc{prs}:3\textsc{sg}  but still little hot do-\textsc{prs}:3\textsc{sg}\\
	\glt \lq It is not really cold but (rather) just a little bit hot.' \parencite[116]{Franklin1971}

	\ex \label{exAppendixKewaRestrictive3}
	Context: About raising pigs.\\
	\gll Sapi adaa-ai \textbf{pa} maa ne-a robo-re ora adaa-ai popa a-ya\\
sweet\_potato big-\textsc{nom} still take eat-\textsc{prs}:3\textsc{sg} when-\textsc{top} really big-\textsc{nom} come stand-\textsc{prs}:3\textsc{sg}\\
	\glt \lq When it takes a sweet potato which is a big one and eats it (without much effort), it really becomes a big one.' \parencite[286]{Yarapea2006}

	\ex \label{exAppendixKewaRestrictive4}
	\gll Nipú kíri \textbf{pa}-rupa ta\\
	he laugh still-like \textsc{prog}:3\textsc{sg}\\
	\glt \lq He is just laughing (without reason).' \parencite[34]{Franklin1971}
\end{exe}
\il{Kewa|)}

\largerpage
\section{Komnzo (tci, wara1294)}\il{Komnzo|(}
\label{appendixKomnzo}

\subsection{Introductory remarks}
Apart from descriptive materials, I searched \citeauthor{Doehler2020}'s (\citeyear{Doehler2020}) corpus. In addition, my understanding of the data has greatly profited from discussion with Christian Döhler. Note that the \textsc{still} expression in this language is identical to the term used to refer to the language itself. This is due to the latter being shortened from \textit{komnzo zokwasi} \lq just speech' (\cite[1]{Doehler2018}).

\subsection{komnzo}
\subsubsection{General information}
\begin{itemize}
	\item Form: there is also a reduced, enclitic form \mbox{=\textit{nzo}} \parencite[169–170]{Doehler2018}. Only one attestation of this enclitic form translates as \lq still' in the Komnzo text corpus \parencite{Doehler2020}. I therefore do not treat it as a separate \textsc{still} expression.
	\item Wordhood: independent grammatical word.
	\item Syntax: relatively movile, but typically in preverbal position.
\end{itemize}

\subsubsection{As a \lq{}still\rq{ }expression}
\begin{itemize}
	\item \textcite[125–126]{Doehler2018}.
	\item Specialisation: \textit{komnzo} is primarily a multi-faceted restrictive marker. Evidence for the phasal polarity concept of \textsc{still} as one of its denotata comes from examples like (\ref{exAppendixKomnzo1}–\ref{exAppendixKomnzo3}). The signalling of persistence in unison with the evocation of an alternative scenario becomes particularly clear in cases like (\ref{exAppendixKomnzo1}, \ref{exAppendixKomnzo2}). Additional, indirect evidence comes from the robustly attested restrictive–\textsc{still} polysemy (\Cref{sectionExclusive}) and the semantic parallels between the two functions.
	\item Pragmaticity: judging from the Komnzo text corpus \parencite{Doehler2020}, \textit{komnzo} is compatible with both scenarios.
	\item Polarity sensitivity: no attestations with scope over a negator.
\end{itemize}

\begin{exe}
	\ex\label{exAppendixKomnzo1}
	\gll Fi z zebnaf o \textbf{komnzo} y-rugr?\\
	3.\textsc{abs} already 3\textsc{sg}:wake\_up or still 3\textsc{sg.m}-sleep\\
	\glt \lq Did he wake up already or is he still sleeping?' \parencite[7]{Doehler2018}

	\ex\label{exAppendixKomnzo2}
	Context: As a punishment for murder, a woman has been sentenced to being burned alive. \\
	\gll Wati nagawa ŋabrigwa si=r. \textbf{Komnzo} rä o z kwarsir mni=n.\\
	then N. 2$|$3:\textsc{pst}:\textsc{ipfv}:return eye=\textsc{purp} still 3\textsc{sg}.\textsc{f}:\textsc{non}.\textsc{pst}:\textsc{ipfv}:be or already 2$|$3\textsc{sg}:\textsc{rec.pst}:\textsc{ipfv}:burn fire=\textsc{loc}\\
	\glt \lq Then Nagawa [the woman's husband] returned to check: was she still alive or did she burn in the fire?' (\cite[126]{Doehler2018}, \citeyear{Doehler2020})

	\ex\label{exAppendixKomnzo3} 
	\gll Emoth fäth nima ämnzr ote=n. \textbf{Komnzo} zena bobo rä ane kar we nä fof rä trikasi kar fof.\\
	girl \textsc{dim} like\_this 2|3\textsc{pl}:\textsc{non}.\textsc{pst}:\textsc{ipfv}:sit O.=\textsc{loc} still today \textsc{dem} 3\textsc{sg}.\textsc{f}:\textsc{non}.\textsc{pst}:\textsc{ipfv}:be \textsc{dem} village also \textsc{indef} \textsc{emph} 3\textsc{sg}.\textsc{f}:\textsc{non}.\textsc{pst}:\textsc{ipfv}:be story place \textsc{emph}\\
	\glt \lq There were the girls staying in Ote. It is still there now. Another place with another story of its own.' \parencite{Doehler2020}
\end{exe}

\subsubsection{Uses on the fringes of \lq{}still\rq{}}
\paragraph{Scalar contexts}\label{appendixKomnzoScalar}
\begin{itemize}
	\item Some examples in the corpus can be read involving as \textsc{still} plus scalar contexts.
	\item While cases like (\ref{exAppendixKomnzoDecrementHead}) are relatively clear and involve a decrease over time, others are harder to judge. Thus, (\ref{exAppendixKomnzoDecrementPalmWine}) could also be read as \textit{komnzo} contributing contrastive focus (\lq there are plenty all right\rq{}).
\end{itemize}

\begin{exe}

	\ex \label{exAppendixKomnzoDecrementHead}
	 Context: A girl has been attacked and eaten by a crocodile.\\
	 \gll Ebar=nzo \textbf{komnzo} zwarärm fäsi=nzo wänatha\\
	head=just still 3\textsc{sg}.\textsc{f}:\textsc{pst}:\textsc{dur}:be shame=just \textsc{sg}>3\textsc{sg}.\textsc{f}:\textsc{pst}:\textsc{ipfv}:eat\\ 
	\glt \lq [Just] her head was still there, it ate her private parts.' \parencite{Doehler2020}

	\ex \label{exAppendixKomnzoDecrementPalmWine}
	\gll \lq\lq Eh ngthé bana! sgeru komnzo e-mi-thgr?" \lq\lq Ah, segeru \textbf{komnzo} y–rn"\\
	\phantom{\lq\lq}\textsc{interj} brother poor palm\_wine still 2|3\textsc{non.sg}-hang-\textsc{stat} \phantom{\lq\lq}\textsc{interj} palm\_wine still 3\textsc{sg}.\textsc{m}-\textsc{cop.du}\\
	\glt \lq {\lq\lq}Hey brother, are the palm wine (containers) still hanging?" \lq\lq Yes, there are still plenty.{\rq\rq}\rq{ }\parencite[220]{Doehler2018}\footnote{The notion of \lq plenty' arises from the combination of dual and singular number; see \textcite[219]{Doehler2018}.}
	
\end{exe}

\subsubsection{Restrictive (non-temporal)}
\paragraph{(Non-scalar) exclusive}
\label{appendixKomnzoRestrictive}

\begin{itemize}
	\item \textcite[125–126]{Doehler2018}.
	\item As is common with such markers, this is a cluster of uses or functions, with the associated constituent being the predicate. For lower constituents a reduced and enclitic form =\textit{nzo} is used (see \cite[169–170]{Doehler2018}). The common denominator is the exclusion of (mostly implicit) alternatives; see (\ref{exAppendixKomnzoRestrictive1}–\ref{exAppendixKomnzoRestrictive4}). 
	\item The restrictive use is by far the most frequent use in the text corpus.
\end{itemize}

\begin{exe}
	\ex \label{exAppendixKomnzoRestrictive1}
	Context: Headhunters are raiding a village. A woman is hiding.\\
	\gll Watik, zan wogan=é \textbf{komnzo} zfnagwrmth.\\
then fight man=\textsc{erg}:\textsc{non}.\textsc{sg} still 2$|$\textsc{pl}>3\textsc{sg.f}:\textsc{pst}:\textsc{dur}:miss\\
	\glt \lq The headhunters just passed by without taking notice.' \parencite{Doehler2020}

	\ex \label{exAppendixKomnzoRestrictive2}
	\gll Mni kwarsirm mni \textbf{komnzo} zöfthé zethkäfa.\\
fire \textsc{sg}:\textsc{pst}:\textsc{dur}:burn fire still before \textsc{sg}:\textsc{pst}:\textsc{pfv}:start\\
	\glt \lq It was the fire burning. The fire which has just started to burn.\rq{ }\parencite[372]{Doehler2018}

	\ex \label{exAppendixKomnzoRestrictive3}
	Context: About a dance. Some lead the dance, others follow.\\
	\gll Eso fi \textbf{komnzo} enmithagra wath=r.\\
thanks 3.\textsc{abs} still 2|3\textsc{pl}:\textsc{pst}:\textsc{stat}:\textsc{ven}:hang dance=\textsc{purp}\\
	\glt \lq That's right, they were just tagging along.' \parencite{Doehler2020}

	\ex \label{exAppendixKomnzoRestrictive4}
	Context: About a certain moaning ritual.\\
	\gll Keke mane fthé kwanafrmth, fthé ausi fäth thwän-thor, ŋarey-é. \textbf{Komnzo} nima tbraw.\\
	\textsc{neg} which when 2|3\textsc{pl}:\textsc{pst}:\textsc{dur}:talk when old\_woman \textsc{dim} 2|3\textsc{pl}:\textsc{iter}:\textsc{ven}-arrive woman-\textsc{abs}:\textsc{non}.\textsc{sg} still like\_this \textsc{interj}:quiet\\
	\glt \lq Nobody talks at this time, until the women returned, many women. Just like this! Dead silent. Just like this! Dead silent.' \parencite{Doehler2020}
\end{exe}
\il{Komnzo|)}

\section{Lewotobi (lwt, lewo1244)}\il{Lewotobi|(}
\subsection{morә̃}

\subsubsection{General information}
\begin{itemize}
	\item Wordhood: independent grammatical word, invariable.
	\item Syntax: fixed position, following the predicate (and negation).
	\item Etymology: unknown; in closely related Lamaholot, the cognate \textit{muri\rq} is an additive and iterative marker (\cite{NishiyamaKelen2007}; \cite{Pampus2010}).
\end{itemize}

\subsubsection{As a \lq{}still\rq{ }expression}
\begin{itemize}
	\item \textcite[414–416]{Nagaya2012}.
	\item Specialisation: in absence of textual attestations, examples like (\ref{exAppendixLewotobi1}–\ref{exAppendixLewotobi3}) give a reasonably solid indication that this marker confines my definition. For instance, in (\ref{exAppendixLewotobi1}) \mbox{\textit{morә̃}} not only signals that the speaker continues to be a student, but also hints at the future discontinuation, which is when the question about marital status will become relevant. In (\ref{exAppendixLewotobi2}), to all appearances, the addressee's persistent presence is in contrast to the speaker's expectation of the opposite. Note that \mbox{\textit{morә̃}} only serves as \textsc{still} with atelic predicates, whereas the combination with telic predicates yields \textsc{not yet} (\appref{appendixLewotobiNotYet}).	
	\item Pragmaticity: appears to be compatible with both scenarios (tentative conclusion). Ex. (\ref{exAppendixLewotobi2}) is a prime candidate for the unexpectedly late scenario.
	\item Polarity sensitivity: \textit{morә̃} itself serves as \textsc{not yet} with telic predicates. There are also a few examples of \textit{morә̃} with scope over negation in \citeauthor{Nagaya2012}'s (\citeyear{Nagaya2012}) grammar.
	\end{itemize}

\begin{exe}
	\ex\label{exAppendixLewotobi1}
	 Context: Used as an answer to an inquiry about marital status. Students are not supposed to marry yet.\\
	 \gll Go mahasiswa \textbf{morә̃} di.\\
	 1\textsc{sg} student still \textsc{dm}:excuse\\
	 \glt \lq I am still a student (so I am not married).' \parencite[434]{Nagaya2012}

	\ex\label{exAppendixLewotobi2}
	\gll Ona=e mo pi  \textbf{morә̃} ta?\\
	wow=hey 2\textsc{sg} \textsc{prox} still \textsc{q}\\
	\glt \lq Wow, hey, are you still here!?'  \parencite[206]{Nagaya2012}

	\ex\label{exAppendixLewotobi3}
	\gll Neku go səga pi, ra kriә̃ \textbf{morә̃}.\\
	ago   1\textsc{sg} arrive \textsc{prox} 3\textsc{pl} work still\\
	\glt \lq When I arrived here a while ago, they were still working.\rq{ }\parencite[450]{Nagaya2012}
\end{exe}

\subsubsection{Uses related to other phasal polarity concepts}
\paragraph{Not yet}\label{appendixLewotobiNotYet}
\begin{itemize}
	\item \textcite[415–416]{Nagaya2012}.
	\item This function occurs with telic predicates (\ref{exAppendixLewotobiNotYet1}, \ref{exAppendixLewotobiNotYet2}), as opposed to atelic ones, with which \textit{morә̃} serves as \textsc{still}. With predicates that allow for both a telic and an atelic construal, this results in ambiguity (\ref{exAppendixLewotobiNotYet3}, \ref{exAppendixLewotobiNotYet4}).
\end{itemize}

\begin{exe}
	\ex\label{exAppendixLewotobiNotYet1}
	\gll Ale hәkә oto \textbf{morә̃}.\\
	A. stop car still\\
	\glt \lq Ale hasn’t stopped the car yet.'  \parencite[415]{Nagaya2012}

	\ex\label{exAppendixLewotobiNotYet2}
	\gll Go kriә̃ waha \textbf{morә̃}.\\
	1\textsc{sg} work finish still\\
	\glt \lq I haven’t finished working yet.' \parencite[415]{Nagaya2012}
	
	\ex\label{exAppendixLewotobiNotYet3}
	\gll Go kә̃ \textbf{morә̃}.\\
	1\textsc{sg} eat.1\textsc{sg} still\\
	\glt \lq I am still eating / I haven’t eaten yet.'  \parencite[416]{Nagaya2012}

	\ex\label{exAppendixLewotobiNotYet4}
	\gll  Go kwukә̃ \textbf{morә̃}.\\
	1\textsc{sg} drunk still\\
	\glt \lq I am still drunk (the speaker has a hangover) / I am not drunk yet (the speaker wants to drink more).' \parencite[416]{Nagaya2012}
\end{exe}
\il{Lewotobi|)}

\section{Ma Manda (skc, sauk1252)}\il{Ma Manda|(}
\label{appendixMaManda}

\subsection{-gût}

\subsubsection{General information}
\begin{itemize}
	\sloppy
	\item Wordhood: a bound morpheme that attaches to roots of several syntactic classes. \textcite{Pennington2016} describes it as a suffix; its low host sensitivity could be taken as an argument to consider it an enclitic.
	\item Syntax: attaches to its focus. With a predicate as the host, it functions as a sentence adverb.
\end{itemize}

\subsubsection{As a \lq{}still\rq{ }expression}
\label{appendixMaMandaStill}
\begin{itemize}
	\item \textcite[125–126, 166–167, 202]{Pennington2016}.
	\item Specialisation: -\textit{gût} is primarily a versatile restrictive marker. My assessment that the marker also serves as a specialised \textsc{still} expression is based on a combination of the discussion in \textcite{Pennington2016} and examples like (\ref{exAppendixMaManda1}, \ref{exAppendixMaManda2}). For instance, in (\ref{exAppendixMaManda1}) the ascending slope is construed as continuing, as opposed to the protagonist's initial assumption that the mountain would plateau at cloud level. Further, indirect, evidence comes from the cross-linguistic robustness of restrictive-\textsc{still} polysemy and the semantic parallels between the two functions (\Cref{sectionExclusive}).
	\item Pragmaticity: seems to be compatible with both scenarios.
	\item Polarity sensitivity: combination with negation yields \textsc{not yet}; -\textit{gût} also forms part of the fixed expression \textit{kogût} \lq not yet'.
	\item Further note: Ex. (\ref{exAppendixMaManda3}, \ref{exAppendixMaManda4}) superficially resemble the persistent time frame use (\Cref{sectionTemporalFrameTT}). However, in structural terms, these are most likely adverbial clauses (see \cite[445–450]{Pennington2016} on non-verbal predication in Ma Manda). Ex. (\ref{exAppendixMaManda5}) illustrates the use in a hortative.
\end{itemize}

\begin{exe}
	\ex \label{exAppendixMaManda1}
Context: The subject is going up a mountain. He thought that the peak coincided with the clouds and now he is moving beyond that point.\\
	\gll I kame mun kun aatûku-go-k-\textbf{gût} kunt.\\
this.\textsc{anaph} ground partial up.\textsc{dist} remain-\textsc{rem.pst}-\textsc{subj}.3\textsc{sg}-still \textsc{dist}\\
	\glt \lq This part of the land still kept going up (lit. still remained partially up) [where clouds had previously blocked].'
\parencite[578]{Pennington2016}

	\ex Context: A girl is thirsty and has been told that there is no water nearby. \label{exAppendixMaManda2}\\
	\gll Ta-ng idi welû udu mi na=la n-e-la-k-\textbf{gût} taa-ka maangût-ta a-go-k.\\
do-\textsc{ds} this.\textsc{anaph} daughter:\textsc{poss}.3\textsc{sg} that.\textsc{anaph} water eat=\textsc{ben} \textsc{obj}.\textsc{1sg}-bite-\textsc{prs}-\textsc{subj}.3\textsc{sg}-still say-\textsc{ss} sit-\textsc{ss} \textsc{cop}-\textsc{rem.pst}-\textsc{subj}.3\textsc{sg}\\
	\glt \lq And, his daughter, kept thinking about being thirsty (lit. said \lq\lq consuming water still bites me") and was sitting up.\rq{ }\parencite[596]{Pennington2016}

	\ex\label{exAppendixMaManda3}
	Context: At night, one man has set out to steal a pineapple and is followed by another man.\\
	\gll Tandon=ta-\textbf{gût} nolû ban wa=lû bulûnap sako-be-k=ka ta-ka kosaan kesuwang-go-k.\\
	night=\textsc{ben}-still brother:\textsc{poss}.3\textsc{sg} \textsc{indef} that=\textsc{nom} pineapple hold:\textsc{obj.}3\textsc{sg}-\textsc{irr.sg}-\textsc{subj}.3\textsc{sg}=\textsc{ben} do-\textsc{ss} side reach\_for-\textsc{rem.pst}-\textsc{subj}.3\textsc{sg}\\
	\glt \lq Still in the night, the other brother reached out on the side to grab the pineapple.' \parencite[566]{Pennington2016} 
	\pagebreak
	\ex\label{exAppendixMaManda4}
	Context: The narrator and her companions have been moving around.\\
	\gll Ba-ka mo, kagang ba-ngkadopm-ûnggû-m, tafala-\textbf{gût}.\\
	come-\textsc{ss} already village come-arrive-\textsc{rem}.\textsc{pst}-\textsc{subj}.1\textsc{pl} afternoon-still\\
	\glt \lq I came, and then we arrived at the village, while it was still afternoon.’ \parencite[549]{Pennington2016}
	
	\ex\label{exAppendixMaManda5}
	\gll Ku-de-m ku-de-m-\textbf{gût} wa nûng-ka=ta ba-gû-mok.\\
go-\textsc{irr.du}-1\textsc{non.sg} go-\textsc{irr.du}-\textsc{subj}.1\textsc{non.sg}-still that tell-\textsc{ss}=do come-\textsc{rem.pst}-\textsc{subj}2$|$3\textsc{du}\\
	\glt \lq Keep telling (her) \lq\lq Let’s go! Let’s keep going!" they came.\rq{ }\parencite[166]{Pennington2016}
\end{exe}


\subsubsection{Restrictive (non-temporal)}
\paragraph{(Non-scalar exclusive)}
\label{appendixMaMandaRestrictive}
\begin{itemize}
	\item \textcite[151–157, 164–167]{Pennington2016}
	\item As with many exclusive markers, this is a cluster of functions. These include restricting (\lq only') the reference of nominals and demonstrative constituents (\ref{exAppendixMaMandaRestrictive1}, \ref{exAppendixMaMandaRestrictive2}) as well as the reference of the predicate (\ref{exAppendixMaMandaRestrictive3}). It is also used for emphasis on identity of place (\ref{exAppendixMaMandaRestrictive4}) or time (ex. \ref{exAppendixMaMandaRestrictive5}; also see exx \ref{exAppendixMaManda3}, \ref{exAppendixMaManda4} above) and to derive adverbs from stems of other syntactic classes (\ref{exAppendixMaMandaRestrictive6}).
\end{itemize}
\begin{exe}
	\ex \label{exAppendixMaMandaRestrictive1}
	\gll Nak Yase ba-ka mani wa-\textbf{nggût} naa-mû-la-k-ngang.\\
	1\textsc{sg} Y. come-\textsc{ss} money that-still \textsc{obj}.1\textsc{sg}-give-\textsc{prs}-\textsc{subj}.3\textsc{sg}-\textsc{hab}\\
	\glt \lq Yase comes and gives me that very money.' \parencite[165]{Pennington2016}

	\ex \label{exAppendixMaMandaRestrictive2}
	\gll Taawaa-\textbf{gût} wa=lû bûsenang ku-ngkadopm-ûngka…\\
	ridge-still that=\textsc{nom} jungle go-arrive-\textsc{ss}\\
	\glt \lq Just along the ridge they went into the jungle…' \parencite[165]{Pennington2016}
	
	\ex \label{exAppendixMaMandaRestrictive3}
	\gll Elang-\textbf{gût} met ku-taa-t.\\
	lie-still later go-\textsc{fut}-\textsc{subj}.1\textsc{sg}\\
	\glt \lq Just kidding I'll go later.'  \parencite[165]{Pennington2016}
	
	\ex \label{exAppendixMaMandaRestrictive4}
	\gll Nûnûng ya-\textbf{nggût} taa-waa-m.\\
	1\textsc{pl}.\textsc{emph} here-still say-\textsc{prs}-\textsc{subj}.1\textsc{pl}\\
	\glt \lq Just us are speaking right here.'   \parencite[152]{Pennington2016}
	
	\ex \label{exAppendixMaMandaRestrictive5}
	\gll Wa=long-\textbf{ngût} tritoin=tû kekng taa-go-k…\\
	that=\textsc{all}-still T=\textsc{nom} call say-\textsc{rem}.\textsc{pst}-\textsc{subj}.3\textsc{sg}\\
	\glt \lq At that [very] moment, Tritoin called out, …' \parencite[165]{Pennington2016}
	
		\ex \label{exAppendixMaMandaRestrictive6}
	\gll Wa-sû=lû mumung kaalin-\textbf{gût} dom gaai-go-k.\\
	that-\textsc{lnk}=\textsc{nom} loincloth good-still \textsc{neg} fasten-\textsc{rem.pst}-\textsc{subj.}3\textsc{sg}\\
	\glt \lq He didn’t fast his loincloth well.' \parencite[164]{Pennington2016}
\end{exe}
\il{Ma Manda|)}

\section{Mateq (xem, kemb1249)}\il{Mateq|(}
\label{appendixMateq}

\subsection{Introductory remarks}
Mateq has two markers that qualify as candidates for \textsc{still} expressions: \textit{mege} and \textit{buyu}. For \textit{mege}, only two tokens are found in \citeauthor{Connell2013}'s (\citeyear{Connell2013}) grammar. I therefore did not include it in my sample.

\subsection{bayu}
\subsubsection{General information}
\begin{itemize}
	\item Wordhood: independent grammatical word, invariable.
	\item Syntax: fixed position, immediately before the predicate.
\end{itemize}

\subsubsection{As a \textsc{still} expression (plus restrictive notions)}
\label{appendixMateqStill}
\begin{itemize}
	\item \textcite[137–138]{Connell2013}
	\item Specialisation: inclusion is somewhat more tentative than that of most other expressions in the sample. \textcite[137]{Connell2013} describes \textit{bayu} marker as denoting \lq\lq a type of imperfective aspect". Connell applies the same label to \textit{mege} \lq still' and \textit{degeq} \lq constantly, keep Verb-ing', which indicates that he uses \textit{imperfective} as a synonym for continuity/persistence. 	As for the invocation of an alternative scenario, \textit{bayu} not only implies discontinuation, but also a subsequent contingent development. This is particularly salient in (\ref{exAppendixMateq1}, \ref{exAppendixMateq2}) and distinguishes \textit{bayu} from \textit{mege}.

	\item Pragmaticity: the available data allow no conclusion.
	\item Polarity sensitivity: inner negation yields \textsc{not yet}.
\end{itemize}

\begin{exe}
	\ex \label{exAppendixMateq1}
	\gll \textbf{Bayu} kurak\sim kurak tubiq.\\
	still bubble\sim\textsc{redupl} rice\\
	\glt \lq The rice was [still] bubbling (implies an expectation that once the rice has bubbled it will be scooped out and eaten).\rq{ }\parencite[137–138]{Connell2013}
	
	\ex \label{exAppendixMateq2}
	\gll  Mateq tuet=n, okoq \textbf{bayu} man.\\
	soon first=\textsc{adv} 1\textsc{sg} still eat\\
	\glt \lq Just a moment, I'm still eating.' \parencite[148]{Connell2013}

	\ex \label{exAppendixMateq3}
	\gll Oya babu téq matéqéh \textbf{bayu} oji.\\
	mother mouse \textsc{prox} just\_before still go\_to\_forest\\
	\glt \lq Mother Mouse was still out in the forest.\rq{ }\parencite[138]{Connell2013}
\end{exe}

\subsubsection{Uses on the fringes of \lq{}still\rq{}}

\paragraph{Scalar contexts}\label{appendixMateqThusFarOnly}
\begin{itemize}
	\item \textit{Bayu} is compatible with scalar predicates. All attestations in the data feature limited increases.  This includes two attestations involving an anterior viewpoint in combination with \textit{sidah} \lq once\rq{ }(\ref{exAppendixMateqRestrictive2}, \ref{exAppendixMateqRestrictive3}). Note that \citeauthor{Connell2013} translates (\ref{exAppendixMateqRestrictive1}) with the English present perfect, but the Mateq sentence features a nominal predicate.
	\item Assuming that my general interpretation of \textit{bayu} as a \lq still only\rq{ }expression is correct, these can be subsumed under the latter, more general function.
\end{itemize}

\begin{exe}
	\ex \label{exAppendixMateqRestrictive1}
	\gll \textbf{Bayu} aroq=nq panèi nyidoq.\\
	still beginning=\textsc{poss}.3 clever speak\\
	\glt \lq He'd only just begun to speak well (lit. it is still only his beginning …).' \parencite[138]{Connell2013}	

	\ex\label{exAppendixMateqRestrictive2}
	\gll Okoq \textbf{bayu} sidah=ng koq mamuh.\\
	1\textsc{sg} still once=\textsc{adv} 1\textsc{sg} bathe\\
	\glt \lq I've only bathed once.'  \parencite[150]{Connell2013}
	
	\ex\label{exAppendixMateqRestrictive3}
	\gll Okoq \textbf{bayu} sidah téq.\\
	1\textsc{sg} only once this\\
	\glt \lq I've never done this before (lit. I've only done this once, i.e. now).' \parencite[150]{Connell2013}
\end{exe}
\il{Mateq|)}

\section{Maybrat (ayz, maib1239)}\il{Maybrat|(}
\label{appendixMaybrat}
\subsection{Introductory remarks}
My understanding of the Maybrat data has greatly profited from discussion with Philomena Dol. Maybrat seems to have two \textsc{still} expressions: \textit{fares} \lq still' and \textit{etu} \lq still be'. Only \textit{fares} is (close to) polyfunctional.

\subsection{fares}

\subsubsection{General information}
\begin{itemize}
	\item Wordhood: free morpheme.
	\item Syntax: invariably in clause-final position.
\end{itemize}

\subsubsection{As a \lq{}still\rq{ }expression}
\begin{itemize}
	\item \textcite[160–161, 173]{Dol2007}; additional discussion in \textcite[55]{vanBaar1997}.
	\item Wordhood: free morpheme.
	\item Specialisation: examples like (\ref{exAppendixMaybrat1}–\ref{exAppendixMaybrat3}), when taken together, give a reasonably good indication that \textit{fares} conforms to my definition. For instance, in (\ref{exAppendixMaybrat1}), the process of studying is construed as extending over time, and contrasted with its later discontinuation. \textit{Fares} is also taken to be a genuine \textsc{still} expression by \textcite{vanBaar1997}.
	\item Pragmaticity: the data allow no conclusion.
	\item Polarity sensitivity: inner negation yields \textsc{not yet}, outer negation \textsc{no longer}.
	\item Further note: \textit{fares} often combines with \textit{etu} \lq still be', as in (\ref{exAppendixMaybrat3}).
\end{itemize}

\begin{exe}
	\ex\label{exAppendixMaybrat1}
	\gll Tuo sia t-ao iwai farkor \textbf{fares}.\\
	1\textsc{sg} with 1\textsc{sg}-sibling:\textsc{ss} earlier study still\\
	\glt \lq I with my brother, formerly we still studied.' \parencite[160]{Dol2007}
	
	\ex\label{exAppendixMaybrat2}
	\gll Ku ro m-ait po-iit kiyam \textbf{fares}.\\
	child \textsc{rel} 3.\textsc{u}-eat thing-eat.\textsc{pl} ill still\\
	\glt \lq The child that eats food is still ill.' \parencite[150]{Dol2007}

	\ex\label{exAppendixMaybrat3}
	\gll Paulince m-haf m-etu \textbf{fares}.\\
		P. 3.\textsc{u}-pregnant 3.\textsc{u}-still\_be still\\
	\glt \lq Paulince is still pregnant.' \parencite[160]{Dol2007}
\end{exe}

\subsubsection{Broadly adverbial temporal-aspectual functions}
\paragraph{Near past}\label{appendixMaybratImmediatePast}
\begin{itemize}
	\item \textcite[161, 184]{Dol2007}.
	\item This function obtains in the fixed, and common, sequence \textit{tna} \lq new, recently' plus \textit{fares}. That is, we are dealing with a transparent construction \lq [it is] still recent [that]'. This interpretation is in line with the fact that Maybrat does not make use of an overt copula (see \cite[147–148]{Dol2007}). To all appearances, this originally biclausal structure (\lq \textit{p}, it is still recent\rq{}) has undergone clause union.
	\item Note that Maybrat has neither grammatical tense nor aspect (see \cite{Dahl2011} for discussion). All three examples of \textit{tna fares} in \citeauthor{Dol2007}'s grammar feature telic predicates (or telic construals) and a bounded viewpoint.
\end{itemize}

\begin{exe}
	\ex
	\gll M-ros \textbf{tna} \textbf{fares}.\\
	3.\textsc{u}-stand recently still\\
	\glt \lq She got up only recently.' \parencite[161]{Dol2007}
	
	\ex
	\gll M-ape m-aku \textbf{tna} \textbf{fares}.\\
	3.\textsc{u}-give\_birth 3.\textsc{u}-small recently still\\
	\glt \lq She gave birth to small ones only recently.' \parencite[161]{Dol2007}
\end{exe}
\il{Maybrat|)}
	
\section{Paiwan (pwn, paiw1248)}\il{Paiwan|(}
\label{appendixPaiwan}

\subsection{Introductory remarks}
Apart from descriptive materials, I consulted \citeauthor{EarlyWhitehorn2003}'s (\citeyear{EarlyWhitehorn2003}) text collection. I am furthermore indebted to Wei-chen Huang and Drungdrung a Tjakivusung for discussing Paiwan data with me.

\subsection{=anan}
\subsubsection{General information}
\begin{itemize}
	\item Wordhood: bound morpheme (enclitic).
	\item Syntax: attaches to the predicate.
\end{itemize}

\subsubsection{As a \lq{}still\rq{ }expression}
\begin{itemize}
	\item \textcite[228]{Chang2006}, \textcite[574]{EarlyWhitehorn2003}, \citeauthor{Egli1990} (\citeyear[164]{Egli1990}; \citeyear[29–30, 573]{Egli2002}), \textcite[13, 58, 98, 445, 478]{Ferrell1982}, \textcite[15]{Huang2012} and \textcite{Wu2006}.
	\item Specialisation: textual examples like (\ref{exAppendixPaiwan1}–\ref{exAppendixPaiwan4}) give evidence that this item conforms to my definition. For instance, in (\ref{exAppendixPaiwan1}) \mbox{=\textit{anan}} signals that young deer continue to be considered \lq\lq sprouts" (i.e. not game) for a certain amount of time, while at the same time evoking a prospective view towards the point at which this status is no longer upheld.
	\item Pragmaticity: judging from the texts in \textcite{EarlyWhitehorn2003}, \mbox{=\textit{anan}} appears to be compatible with both scenarios; ex. (\ref{exAppendixPaiwan4}) is a good candidate for the unexpectedly late scenario.
	\item Polarity sensitivity: inner negation yields \textsc{not yet}.
	\item Further note: ex. (\ref{exAppendixPaiwan5}) illustrates the use in an imperative.
\end{itemize}

\begin{exe}
	\ex\label{exAppendixPaiwan1}
	Context: About young deer.\\
	\gll A-zua m-amaw tua udjudju=\textbf{anan}; sa-ka ini-ka tj-en a dj<em>ameq nu k<em>uang itjen. A-zua ini=anan a san satsemel-i. \\
	\textsc{nom}-that \textsc{agt}.\textsc{foc}-same \textsc{obl} sprout=still and-after \textsc{neg}-after there-\textsc{pat}.\textsc{foc} \textsc{lnk} catch<\textsc{agt}.\textsc{foc}> when firearm<\textsc{agt}.\textsc{foc}> 1\textsc{pl}.\textsc{incl}.\textsc{nom} \textsc{nom}-that \textsc{neg}=still \textsc{lnk} make wild\_game-\textsc{pat}.\textsc{foc}\\
	\glt \lq They were still like soft sprouts; and we didnʼt catch them when we were shooting. We didnʼt then consider them as wild game [yet].'
	\exi{} \textit{Nu tsuatsuay anga saka namasan venan anga sa puquvaɬ anga sa putsuarquɬquɬan anga tsaɬeqiɬ anga a kula vaik anga sema gadu a djemavats. Avan anga zazua tja qaɬuen anga, sa tja kuangi anga}.\\
	\lq After a long time when they had become deer, with hair and hooves and hard feet, they walked up to the mountains. After that we hunted and shot them.' \parencite[170–171]{EarlyWhitehorn2003}
	
 	\ex\label{exAppendixPaiwan3}
	Context: From the opening lines of a personal narrative.\\
	\gll Ka i gaku=\textbf{anan} aken m-atsay ti kama. Qau ti kaka i-tua kuaping. Qau ti kina i-zua aɬak a lumamad.	 I-zua ku kaka ka\sim{}keḍi-an=\textbf{anan} a ma-ḍusa, ini=anan ka ma-kaya ma-sengseng.\\
	when \textsc{loc} school=still 1\textsc{sg}.\textsc{nom}  \textsc{agt}.\textsc{foc}-die  \textsc{nom}.\textsc{sg} father so \textsc{nom}.\textsc{sg} sibling \textsc{loc}-\textsc{obl} soldier so \textsc{nom}.\textsc{sg} mother \textsc{loc}-that child \textsc{lnk} infant \textsc{loc}-that 1\textsc{sg}.\textsc{gen} sibling \textsc{redupl}\sim{}small-\textsc{nmlz}=still \textsc{lnk} \textsc{num}-two \textsc{neg}=still when \textsc{acaus}-able \textsc{acaus}-work\\
\glt \lq When I was still at school my father died. My elder brother was in the army. My mother had very young children. My two siblings were still children, not yet able to work.' \parencite[365–366]{EarlyWhitehorn2003}

	 \ex\label{exAppendixPaiwan4}
	Context: A mother thought that her son had died in battle.\\
	\gll M-angtjez i umaq. Q<em>au\sim{}qaung a kina. \lq\lq Maya q<em>aung kina; i-zua=\textbf{anan} aken. Ma-kuda ɬa uqaɬay."\\
	\textsc{agt}.\textsc{foc}-come \textsc{loc} house \textsc{redupl}<\textsc{agt}.\textsc{foc}>\sim{}cry \textsc{nom} mother \phantom{\lq\lq}\textsc{proh} cry<\textsc{agt}.\textsc{foc}> mother \textsc{loc}-that=still 1\textsc{sg}.\textsc{nom} \textsc{acaus}-do\_what really male\\
	\glt \lq When he got home, his mother was crying. \lq\lq Don’t cry, mother, I’m still alive. It’s nothing to a male like me."' \parencite[206]{EarlyWhitehorn2003}

	 \ex\label{exAppendixPaiwan5}
	\gll Kan-u=\textbf{anan}.\\
	eat-\textsc{imp}=still\\
	\glt \lq Continue to eat!\rq{ }\parencite[229]{Chang2006}
\end{exe}

\subsubsection{Uses on the fringes of \lq{}still\rq{}}
\paragraph{Scalar contexts}\label{appendixPaiwanScalar}
\begin{itemize}
	\item =\textit{anan} is attested in scalar contexts.
	\item These encompass decreases, as in (\ref{exAppendixPaiwanScalar1}, \ref{exAppendixPaiwanScalar2}) and contexts involving increases or the absence thereof, as in (\ref{exAppendixPaiwanScalar3}, \ref{exAppendixPaiwanScalar4}). Note that the latter do not require an additional \lq only\rq{ }marker.	
	 \item Example (\ref{exAppendixPaiwanScalar5}), which at first sight appears to involve (the absence of) an increase is more likely to have a binary reading (the presence or absence of the moon).
\end{itemize}

\begin{exe}
	\ex\label{exAppendixPaiwanScalar1}
	\gll Amin=anga=men a mateɬu a mareka kaka a i-zua=\textbf{anan}.\\
	be\_only=\textsc{cmpl}=1\textsc{pl}.\textsc{excl}.\textsc{nom} \textsc{nom} three(\textsc{human}) \textsc{lnk} \textsc{pl} sibling \textsc{lnk} \textsc{loc}-that=still\\
	\glt \lq We are only three siblings who are still (in this world).\rq{ }\parencite[190]{Chang2006}

	\ex \label{exAppendixPaiwanScalar2}
	\gll Na=tapuluq=angata  a       ku=laqulj.        tucu, alu=\textbf{anan}  a       ku=laqulj.\\
\textsc{pst}=ten=true  \textsc{nom} 1\textsc{sg}.\textsc{gen}=book  now eight=still  \textsc{nom} 1\textsc{sg}.\textsc{gen}=book\\
	\glt \lq I had ten books, now I still have eight (left).\rq{ }(Wei-chen Huang, p.c.)

	\ex \label{exAppendixPaiwanScalar3}
	\gll  Matjelu        a       ku=aljak tucu. ljaku ka      kaicavilj, macidilj=\textbf{anan}    a       ku=aljak.\\
three(\textsc{human}) \textsc{nom} 1\textsc{sg}.\textsc{gen}=child  now but  when last.year one(\textsc{human})=still \textsc{nom} 1\textsc{sg}.\textsc{gen}=child\\
\glt \lq I have three children now, but last year I still had only one.\rq{ }(Wei-chen Huang, p.c.)

	\ex \label{exAppendixPaiwanScalar4}
	\gll Kaicavilj, i       ka  na=ma-lje-nguaq; a       p<in>a-veli-an,        na=kudraw             a       ku=paisu          a   sinipazulju. Tucu a   cavilj, ma-lje-nguaq a        p<in>a-veli-an.         ljaku kudraw=\textbf{anan}        a       sinipazulju.\\
last\_year \textsc{neg} ka \textsc{pst}=\textsc{acaus}-go\_toward-good \textsc{nom} \textsc{caus}<\textsc{pat}.\textsc{foc}:\textsc{pfv}>-buy-\textsc{nmlz} \textsc{pst}=ten\_thousand \textsc{nom} 1\textsc{sg}.\textsc{gen}=money \textsc{lnk} bonus now \textsc{lnk} year \textsc{acaus}-go\_toward-good \textsc{nom} \textsc{caus}<\textsc{pat}.\textsc{foc}:\textsc{pfv}>-buy-\textsc{nmlz} but ten\_thousand=still \textsc{nom} bonus\\
		\glt \lq Our business was bad last year, and we had a \$10,000 bonus share. This year, our business is good, but our bonus share is still only \$10,000.\rq{ }(Wei-chen Huang, p.c.)

	\ex\label{exAppendixPaiwanScalar5}
	Context: From a mythical story about a time in which there were two suns but no moon in the sky (Wen-chei Huang, p.c.)\\
	\gll Amin=\textbf{anan} a qadau.\\
	be\_only=still \textsc{lnk} sun\\
	\glt \lq Damals gab es noch bloss die Sonnen. (Mythenthema) [Back then there were still only the suns. (topic of a myth)]\rq{ }(\cite[29]{Egli2002}, glosses added)
\end{exe}

\subsubsection{Broadly adverbial temporal-aspectual functions}

\paragraph{Iterative (and continued iteration) and restitutive}
\label{appendixPaiwanIterative}
\begin{itemize}
	\item \textcite[228, 295]{Chang2006} and \textcite[29, 613]{Egli2002}.
	\item Iterativity is illustrated in (\ref{exAppendixPaiwanIterative1}–\ref{exAppendixPaiwanIterative3}), restitution in (\ref{appendixPaiwanRestitutive1}, \ref{appendixPaiwanRestitutive2}).
	\item In these functions, \mbox{=\textit{anan}} is repeatedly attested in combination with the additive/iterative marker \textit{uta}. This collocation can indicate a continued iteration. Thus, in (\ref{exAppendixPaiwanIterative3}), it marks the third occurrence of the same situation. It can also give emphasis to the iteration or restored state (\ref{exAppendixPaiwanIterative4}).
\end{itemize}
\pagebreak
\begin{exe}
	\ex\label{exAppendixPaiwanIterative1}Context: A child’s grandmother has told him to go look at what they sowed, but nothing had sprouted yet. She has told him to be patient.\\
	\gll Tsua\sim tsuay anga a uta pa-gaɬu \lq\lq sa-u ki-qenetj-i=\textbf{anan}," aya-in ni sa Vuluvulung.\\
\textsc{redupl}\sim long\_time \textsc{cmpl} \textsc{lnk} also \textsc{caus}-slow \phantom{\lq\lq}go-\textsc{imp} do-see-\textsc{imp}=still say-\textsc{pat}.\textsc{foc} \textsc{gen}.\textsc{sg} \textsc{hon} V.\\
\glt \lq After quite a time more the child was told by Vuluvulung [grandmother]: \lq\lq{}Go and have another look.{\rq\rq}\rq{ }\parencite[145]{EarlyWhitehorn2003}


	\ex\label{exAppendixPaiwanIterative2}
	 Context: A thief and another man are hiding under the sleeping platform of a house. Moving about, they have bumped heads once.\\
	\gll Manu i-zu-anga tsaḍ a na-ki-qiɬa ma-tsa\sim{}tsagtsag=\textbf{anan} a qulu.\\
	then \textsc{loc}-that-\textsc{cmpl} bandit \textsc{lnk} \textsc{pst}-do-hide \textsc{acaus}-\textsc{redupl}\sim{}bump=still \textsc{nom} head\\
	\glt \lq The thief was hidden there, and they bumped heads again.\rq{ }\parencite[91]{EarlyWhitehorn2003}

	\ex\label{exAppendixPaiwanIterative3}
	\gll Qau nu q<em>a\sim{}qivu ti sa Kidadaw, “aa,” aya a-zua  tsaqi. Nu  qa\sim{}qivu-in uta, “aa,” aya a-zua tsaqi. Qa\sim{}qivu-en=\textbf{anan} \textbf{uta}.\\
	so when \textsc{redupl}\sim{}<\textsc{agt}.\textsc{foc}>-speak \textsc{nom}.\textsc{sg} \textsc{hon} K. \phantom{\lq\lq}ah say \textsc{nom}-that dung when \textsc{redupl}\sim{}speak-\textsc{pat}.\textsc{foc} also \phantom{\lq\lq}ah say \textsc{nom}-that dung \textsc{redupl}\sim{}speak-\textsc{pat}.\textsc{foc}=still also\\
	\glt \lq When Kidadaw called, the faeces said \lq\lq Ah". When he called again, it said \lq\lq Ah". He called once more.' \parencite[70]{EarlyWhitehorn2003}
	
	\ex\label{exAppendixPaiwanIterative4}
	Context: From a farewell message to John Whitehorn.
	\exi{}\gll Qau a-i-tsu nu tja-viliɬ anga ki-tjen a me-tsevu\sim{}tsevung=\textbf{anan} \textbf{uta}.\\
	So be\_thus-\textsc{loc}-this when more-behind \textsc{cmpl} will-1\textsc{pl}.\textsc{incl} \textsc{lnk} \textsc{agt}.\textsc{foc}-\textsc{redupl}\sim{}meet=still also\\
	\glt \lq And in the future we will [surely] all meet again.\rq{ }\parencite[482]{EarlyWhitehorn2003}
	
	\ex\label{appendixPaiwanRestitutive1}
	\gll Me-v’alut=\textbf{anan}\\
	\textsc{agt}.\textsc{foc}-live=still\\
	\glt \lq Er ist wieder aufgelebt. [He came back to life.]\rq{ }(\cite[29]{Egli2002}, glosses added)
	\pagebreak
	\ex\label{appendixPaiwanRestitutive2}
			 Context: From a farewell message to John Whitehorn.
	\exi{}\gll Uła tjen a laḍuq=anan a tja nasi, uła tjen a na-se-tsevung=anan tjai sinsi, uła tjen a na-ki-dja\sim{}djalan=\textbf{anan} a ki-samuła tua pa-tjara Tsemas, ay\sim{}aya men uta a ki-qau\sim{}qaung tua Tsemas.\\
	so\_that 1\textsc{pl}.\textsc{incl}.\textsc{nom} \textsc{lnk} long=\textsc{still} \textsc{lnk} 1\textsc{pl}.\textsc{incl}.\textsc{gen} breath so\_that 1\textsc{pl}.\textsc{incl}.\textsc{nom} \textsc{lnk} \textsc{pst}-happen\_suddenly-meet=still \textsc{obl} teacher so\_that 1\textsc{pl}.\textsc{incl}.\textsc{nom} \textsc{lnk} \textsc{pst}-do-\textsc{redupl}\sim{}road=still \textsc{lnk} do-urgent \textsc{obl} \textsc{caus}-surely God \textsc{redupl}\sim{}say 1\textsc{pl}.\textsc{excl}.\textsc{nom} also \textsc{lnk} do-\textsc{redupl}\sim{}cry \textsc{obl} God\\
	\glt \lq May our lives [still] be long, so that we’ll meet the teacher again, so that we can join in [again] working for God, is what we say in prayer to God.' \parencite[482–483]{EarlyWhitehorn2003}
\end{exe}


\paragraph{First, for now}
\label{appendixPaiwanFirst}
\begin{itemize}
	\item \textcite[228, 295]{Chang2006} and \textcite[164]{Egli1990}.
	\item This reading might also be involved in the collocation \mbox{\textit{gaɬu}=\textit{anan}} \lq wait a bit!'.
	\item For \textit{nahan} in close-by Saisiyat, which has a very similar functional range to Paiwan \mbox{=\textit{anan}} (\appref{appendixSaisiyat}) \textcite{Huang2008} stipulates that the \lq first'-function goes back to the additive usage in a discursive pattern similar to omnisyndetic coordination (i.e. both the clause describing an initial event and those describing subsequent events feature the marker in question). No such discourse pattern is, however, attested for Paiwan \mbox{=\textit{anan}} in \citeauthor{EarlyWhitehorn2003}'s (\citeyear{EarlyWhitehorn2003}) collection of one hundred texts. Also note that \citeauthor{Huang2008}'s (\citeyear[116]{Huang2008}) example includes a verb glossed as \lq do first'.
\end{itemize}
\largerpage

\begin{exe}
	\ex Context: From the opening of a narrative. Once upon a time there were some girls and men weeding. They took it in turns to weed.\\
	\gll  Ki-ɬavar-an a mare-qali: \lq\lq ti-mun-ay=\textbf{anan} a mare-ḍava a m-asik," aya-in.\\
	 do-speak-\textsc{loc}.\textsc{foc} \textsc{lnk} pair-friend \phantom{\lq\lq{}}\textsc{nom}-2\textsc{pl}-will=still \textsc{nom} pair-female\_friend \textsc{lnk} \textsc{agt}.\textsc{foc}-weed say-\textsc{pat}.\textsc{foc}\\
	\glt \lq  The men said: \lq\lq You girls weed first."' \parencite[108]{EarlyWhitehorn2003}

	\ex
	Context: Monkey and ant-eater have decided to try who can stay under water the longest.\\
	\gll \lq\lq Nu maitazua tiaken-ay=\textbf{anan}", aya a ḍail.\\
	\phantom{\lq\lq}when like\_that 1\textsc{sg}.\textsc{nom}-will=still say \textsc{nom} monkey\\
	\glt \lq {\lq\lq}All right; I’ll go first", said the monkey.\rq{ }\parencite[250]{EarlyWhitehorn2003}

	\ex Context: A group of mice has set out to kill a cat.\\
	\gll Ka tjału q<in>ałiv-an tiamadju, se<m>kez=\textbf{anan} a pa-kełang tua uri pa-ka-zu-an-an. \lq\lq ari," ka aya-in na-zua kama\sim{}kama niamadju vaik a s<em>a teku a mapuɬat a ki-pa-gaɬu.\\
	when reach roof<\textsc{pat}.\textsc{foc}:\textsc{pfv}>-\textsc{nmlz} 3\textsc{pl}.\textsc{nom} rest<\textsc{agt}.\textsc{foc}>=still \textsc{lnk} \textsc{caus}-know \textsc{obl} will \textsc{caus}-in\_passing-that-\textsc{nmlz}-\textsc{nmlz} \phantom{\lq\lq}go\_on when say-\textsc{pat}.\textsc{foc} \textsc{gen}-that \textsc{redupl}\sim{}father 3\textsc{pl}.\textsc{gen} leave \textsc{lnk} go<\textsc{agt}.\textsc{foc}> down \textsc{lnk} all \textsc{lnk} do-\textsc{caus}-slow\\
	\glt \lq When they got to the roof, they [first] paused for instructions about the route. When their leader said: \lq\lq Let’s go," they all went below slowly.' \parencite[386–387]{EarlyWhitehorn2003}
\end{exe}

\paragraph{Announcements}\label{appendixPaiwanAnnouncement}

\begin{itemize}
	\sloppy
	\item There are numerous instances of \mbox{=\textit{anan}} in \citeauthor{EarlyWhitehorn2003}'s (\citeyear{EarlyWhitehorn2003}) text collection that involve the subject announcing what they are about to do (\ref{exAppendixPaiwanAnnounce1}) or that it is another participant's turn to do something (\ref{exAppendixPaiwanAnnounce3}).
	\item This cannot always be separated from the \lq first, for now use\rq{ } (\appref{appendixPaiwanFirst}) that it is most likely based on. For instance, ex. (\ref{exAppendixPaiwanAnnounce4}) can be read both as an announcement of the speaker's plan, as well as a statement of what needs to be done before confronting the enemies.	
\end{itemize}

\largerpage[2.25]
\begin{exe}
	\ex\label{exAppendixPaiwanAnnounce1}
	 Context: Parents to one of their children.\\
	\gll \lq\lq Ka i-umaq-an su kaka. Vaik=\textbf{anan} amen a s<em>a va\sim{}vua. Kim-an anga tua tja vaqu sa tsugtsug-an anga\rq\rq, aya-in ti Kuɬeɬuɬeɬu.\\
	\phantom{\lq\lq}when \textsc{loc}-house-\textsc{loc}.\textsc{foc} 2\textsc{sg}.\textsc{gen} sibling leave=still 1\textsc{pl}.\textsc{excl}.\textsc{nom} \textsc{lnk} go<\textsc{agt}.\textsc{foc}> \textsc{redupl}\sim{}field search-\textsc{loc}.\textsc{foc} \textsc{cmpl} \textsc{obl} 1\textsc{pl}.\textsc{incl}.\textsc{gen} millet and bump-\textsc{loc}.\textsc{foc} \textsc{cmpl} say-\textsc{pat}.\textsc{foc} \textsc{nom}.\textsc{sg} K.\\
	\glt \lq \lq\lq Stay at home with your sibling. We are just going to the fields. Find some of our millet and pound it," they said to Kulelulelu.' \parencite[210]{EarlyWhitehorn2003}

	\ex\label{exAppendixPaiwanAnnounce3}
	 Context: Crab and Monkey are trying to burn one another. The fire did no harm to Crab.\\
	\gll Sa tisun-ay=\textbf{anan}, qali-an i ḍai\sim{}ḍail.\\
	and 2\textsc{sg}.\textsc{nom}-\textsc{sbjv}=\textsc{still} friend-\textsc{nmlz} \textsc{prep} \textsc{redupl}\sim{}monkey\\
	\glt [Crab:] \lq Now it’s your turn, friend monkey.\rq{ }\parencite[197]{EarlyWhitehorn2003}

	\ex\label{exAppendixPaiwanAnnounce4}
	Context: The speaker sees people from another village approaching.
	\gll  Ai anga tsa vuluq sa kuang, a ma-tu ka-ɬa-ɬudjeɬudj-an a ma-tu ngudju\sim{}ngudjuɬ-an. Gaɬu=\textbf{anan}; ku si-kes-an=\textbf{anan} tua puk sa ku qaɬi<djela>djelav-i.\\
	\textsc{interj} \textsc{cmpl} this spear this firearm \textsc{lnk} \textsc{acaus}-alike main-class-thorn\_plant-\textsc{nmlz} \textsc{lnk} \textsc{acaus}-alike \textsc{redupl}\sim{}stump-\textsc{nmlz} slow=still 1\textsc{sg}.\textsc{gen} \textsc{inst}.\textsc{foc}-food-\textsc{loc}.\textsc{foc}=still \textsc{obl} bean\_sp and 1\textsc{sg}.\textsc{gen} undercooked<\textsc{redupl}>-\textsc{pat}.\textsc{foc}\\
	\glt \lq Oh dear! Their spears and guns are like thorn-grass and high tree stumps. Wait a bit, I’ll cook some tree-beans for them, and leave them undercooked.' (He then proceeds to eat the undercooked beans and defeats the enemies by breaking a wind so strong that it blasts them away.)\rq{ }\parencite[229]{EarlyWhitehorn2003}
\end{exe}

\subsubsection{Additive and related uses}

\paragraph{Additive}\label{appendixPaiwanAdditive}
\begin{itemize}
	\item \textcite[164]{Egli1990}
	\item In this function, \mbox{=\textit{anan}} recurrently combines with another additive marker \textit{uta}, as in (\ref{exAppendixTaiwanAlso2}).
\end{itemize}

\begin{exe}
	\ex
	Context: A boy’s father had been killed in a fight. The boy is asking about his whereabouts.\\
	\gll ɬakua ma-rekutj a kina a tj<em>umaɬ tu q<in>tsi nua se Tjuavudas; ma-rekutj tu vaik=\textbf{anan} a ki-kuang tua se Tjuavudas.\\
	but \textsc{acaus}-fear \textsc{lnk} mother \textsc{lnk} discuss<\textsc{agt}.\textsc{foc}> \textsc{comp} fight<\textsc{pat}.\textsc{foc}:\textsc{pfv}> by person\_of T. \textsc{acaus}-fear \textsc{obl} leave=still \textsc{lnk} do-firearm \textsc{obl} person\_of T.\\
	\glt \lq But his mother was afraid to tell him that his father had been killed by the Tjuavudas people; she was afraid that he too would go and fight the Tjuavudas people.' \parencite[55]{EarlyWhitehorn2003}

	\ex\label{exAppendixTaiwanAlso2}
	Context: About wedding traditions.\\
	\gll A-i-tsu a si-pu-tseke\sim{}tsekeł nua ma-ma-zangił-an ka-djunang-an pana qału\sim{}qałup-an djilung a ka-ułay-an qata kuang tjakit, i-zua=\textbf{anan} uta a zuma vadis.\\
	be\_thus-\textsc{loc}-this \textsc{lnk} \textsc{inst}.\textsc{foc}-have-\textsc{redupl}\sim{}spouse of \textsc{acaus}-\textsc{acaus}-chief-\textsc{nmlz} main-earth-\textsc{nmlz} river \textsc{redupl}\sim{}hunt-\textsc{nmlz} jar \textsc{lnk} main-piece-\textsc{nmlz} bead firearm knife \textsc{loc}-that=still also \textsc{nom} other chief’s\_meat\\
	\glt \lq The bride-prices among chieftains include land, rivers, hunting grounds, precious pots, beads, guns and swords; and there is [also] other tribute to be paid as well.' \parencite[416]{EarlyWhitehorn2003}

	\ex
	\gll Kan-u=\textbf{anan} pagaɬu.\\
	eat-\textsc{imp}=still few\\
	\glt \lq Eat a little more.' (\cite[115]{Ferrell1982}, glosses added)
\end{exe}

\paragraph{Comparisons of inequality}\label{appendixPaiwanComparisons}
\begin{itemize}
	\item \textcite[164]{Egli1990}.
	\item Comparisons of inequality in Paiwan are formed via the prefix \mbox{\textit{tja}-} \lq more' on the predicate, with the standard of comparison –if present– marked for oblique case (see \cite[149, 167]{Egli1990}; \cite[26]{Ferrell1982}; \cite[219]{Huang2012}).
	\item Addition of \mbox{=\textit{anan}} yields \lq even more\rq{}.
\end{itemize}

\begin{exe}
	\ex
	\gll tja-pu-pitsur=\textbf{anan}\\
	more-have-strength=still\\
	\glt \lq Noch stärker. [Even stronger.]\rq{ }(\cite[164]{Egli1990}, glosses added)
	
	\ex 
	Context: At a wedding dinner, a man has been eating a lot. His wife was ashamed of him and tied a string around his arm, telling him that he can only eat some more, once she tugs the string.\\
	\gll Nu tsuay anga ini tsiḍtsiḍ-i. \lq\lq{}Aku ini anga su tsiḍ\sim{}tsiḍtsiḍ-i?\rq\rq{} aya a uqaɬay. Manu tjemala tideq sa pa-qa\sim{}quɬuts a vatu. Qau ma-tsiḍtsiḍ a lima. Tja-djaɬaw=\textbf{anan} a dj<em>amay.\\
	when long\_time \textsc{cmpl} \textsc{neg} tug-\textsc{pat}.\textsc{foc} \phantom{\lq\lq}why \textsc{neg} \textsc{cmpl} 2\textsc{sg}.\textsc{nom} \textsc{redupl}\sim{}tug-\textsc{pat}.\textsc{foc} say \textsc{nom} male then enter space and \textsc{caus}-\textsc{redupl}\sim{}tussle \textsc{nom} dog so \textsc{acaus}-tug \textsc{nom} hand more-quickly=still \textsc{lnk} side\_dish<\textsc{agt}.\textsc{foc}>\\
	\glt \lq For a long time there was no tug. \lq\lq Why haven’t you tugged yet?" said the man. Then some dogs came into the space between them and started a tussle; and so his arm got tugged. He went at the side dishes even more crazily.' \parencite[46–47]{EarlyWhitehorn2003}
\end{exe}
\il{Paiwan|)}

\section{Rapanui (rap, rapa1244)}\label{appendixRapanui}\il{Rapanui|(}
\subsection{Introductory remarks}
My understanding of Rapanui is based mostly on \citeauthor{Kieviet2017}'s (\citeyear{Kieviet2017}) grammar, which is corpus-based and also incorporates findings from many preceding descriptions. Several of the examples below feature the \lq\lq neutral aspect" marker \textit{he}; see \textcite[316–320]{Kieviet2017} for a discussion of this item.

\subsection{nō}
\subsubsection{General information}
\begin{itemize}
	\item Form: also transcribed as \textit{no}.
	\item Wordhood: independent grammatical word, invariable.
	\item Syntax: mobile, typically following its associated constituent.
\end{itemize}

\subsubsection{As a \lq{}still\rq{ }expression}
\begin{itemize}
	\item \textcite[160]{duFeu1996}, \textcite[132,272]{Fuentes1960} and \textcite[115, 343–344, 347]{Kieviet2017}.
	\item Specialisation: the descriptions of \textit{nō} clearly identify the concept of \textsc{still} as one of its functions; see especially \textcite[115]{Kieviet2017}. Further, albeit indirect, evidence comes from the robustly attested restrictive-\textsc{still} polysemy and the semantic parallels between these functions (\Cref{sectionExclusive}).
	\item Pragmaticity: compatible with both scenarios \parencite[344, 347]{Kieviet2017}.
	\item Polarity sensitvity: not attested as a phasal polarity expression plus negation. This is in line with \textsc{not yet} being expressed by negation plus (\textit{h})\textit{ia} \lq yet' \parencite[509]{Kieviet2017}, and \textsc{no longer} by negation plus \textit{haka'ou} \lq again' \parencite[183]{Kieviet2017}.
	 \item Further note: \textcite[322, 344]{Kieviet2017} observes that \textit{nō} often features in texts as a cohesive device, indicating the continuity of a situation described earlier, and which serves as the background for an event; see (\ref{exAppendixRapaNui3}).
\end{itemize}

\begin{exe}
	\ex\label{exAppendixRapaNui1}
	\gll E mata \textbf{nō} ʼana hoʼi te miro era i hore mai era.\\
	\textsc{ipfv} unripe still \textsc{cont} indeed \textsc{art} wood \textsc{dist} \textsc{pfv} cut hither \textsc{dist}\\
	\glt \lq The wood that has been cut is still green.'  \parencite[326]{Kieviet2017}
	
	\ex\label{exAppendixRapaNui2}
	\gll E ʼitiʼiti \textbf{nō} ʼā a koe; kai ʼite ʼana e tahi meʼe o te via taŋata.\\
\textsc{ipfv} small.\textsc{redupl} still \textsc{cont} \textsc{art} 2\textsc{sg} \textsc{neg}.\textsc{pfv} know \textsc{cont} \textsc{num} one thing of \textsc{art} life person\\
	\glt \lq You are (still) little; you donʼt know anything about human life (yet).ʼ \parencite[501]{Kieviet2017}
	
	\ex\label{exAppendixRapaNui3}
	Context: Two people have fled from the rain and are sitting in a cave.\\
	\gll I nonoho era, he papaŋahaʼa ʼi te haʼuru. E haʼuru \textbf{nō} ʼā, he tuʼu atu hoko rua hakaʼou nuʼu mai te puhi iŋa mo te evinio.\\
	\textsc{pfv} stay.\textsc{pl} \textsc{dist} \textsc{neutral} heavy.\textsc{pl} at \textsc{art} sleep \textsc{ipfv} sleep still \textsc{cont} \textsc{neutral} arrive away \textsc{num}:human two again people from \textsc{art} fish\_at\_night \textsc{nmlz} for \textsc{art} lent\\
\glt \lq While they stayed there, they fell asleep. While they were [still] sleeping, two other people arrived, who had been fishing at night for Lent.ʼ \parencite[588–589]{Kieviet2017}
\end{exe}

\subsubsection{Broadly adverbial temporal-aspectual functions}

\paragraph{Always}\label{appendixRapanuiAlways}
\begin{itemize}
	\item \textcite[132, 272]{Fuentes1960}.
	\item This use seems common with, but not entirely restricted to, a periphrastic habitual aspect construction, as in (\ref{exAppendixRapaNuiAlways2}, \ref{exAppendixRapaNuiAlways3}).
\end{itemize}

\begin{exe}
		\ex\label{exAppendixRapaNuiAlways1}
		\gll É manáu \textbf{no} mai.\\
		\textsc{ipfv} mind still hither\\
		\glt \lq Me acordaré siempre. [I will always remember.]\rq{ }(\cite[103]{Fuentes1960}, glosses added)
	
		\ex\label{exAppendixRapaNuiAlways2}
		\gll 'I rā noho iŋa he tu'u \textbf{nō} mai te aŋa o te nu'u pa'ari ki tō'oku koro u'i.\\
		at \textsc{dist} stay \textsc{nmlz} \textsc{pred} arrive still hither \textsc{art} do of \textsc{art} people adult to \textsc{poss}.1\textsc{sg} Dad look\\
		\glt \lq At that time the old people always came to see my father.\rq{ }\parencite[91]{Kieviet2017}

		\ex\label{exAppendixRapaNuiAlways3}
		\gll He kai toke \textbf{nō} mai o te taŋata te aŋa.\\
		\textsc{pred} food steal still hither of \textsc{art} man \textsc{art} do\\
		\glt \lq Stealing the peopleʼs food was what she did all the time.\rq{ }\parencite[263]{Kieviet2017}
\end{exe}
\largerpage[2]
\subsubsection{Restrictive (non-temporal)}
\paragraph{(Non-scalar) exclusive}
\label{appendixRapanuiRestrictive}
\begin{itemize}
	\item \textcite[37, 78, 123–124]{duFeu1996}, \textcite[134, 272]{Fuentes1960} and \textcite[174, 266–268, 343–345]{Kieviet2017}.
	\item As with many exclusive markers, this is a cluster of related functions. These include restricting the reference of nominal constituents (\ref{exAppendixRapaNuiRestrictive1}) and excluding alternative denotations for  the predicate (\ref{exAppendixRapaNuiRestrictive2}). Clearly derived is the use of \textit{nō}  to mark an act as performed \lq\lq[w]ithout further ado, without thinking, without taking other considerations into account" (\cite[343]{Kieviet2017}); see (\ref{exAppendixRapaNuiRestrictive4}).
	\item Restrictive \textit{nō} also forms part of a construction \textit{te} Noun \textit{nō} \lq \textsc{art} Noun only' that invariably stands in sentence-initial position and introduces an exception to an (implicit or explicit) generalisation; this is illustrated in (\ref{exAppendixRapaNuiRestrictive5}). This construction, in turn, instantiates a pattern, by which
	
	\begin{quote} in initial subject NPs, \textit{nō} indicates that the set referred to by the noun phrase has only one entity, viz. the one described in the rest of the sentence. The sentence can be paraphrased as: ‘There is only one [NP], and that is [rest of sentence]\rq{ }\parencite[266]{Kieviet2017}
	\end{quote}
	
\end{itemize}
\begin{exe}
	\ex\label{exAppendixRapaNuiRestrictive1}
	\gll 'I te pō \textbf{nō} te ika nei ana hī.\\
	at \textsc{art} night still \textsc{art} fish \textsc{prox} \textsc{irr} fish\\
	\glt \lq Only at night this fish can be fished.\rq{ }\parencite[267]{Kieviet2017}

	\ex\label{exAppendixRapaNuiRestrictive2}
	\gll ʼIna a Tiare kai mate; ko rerehu \textbf{nō} ʼā.\\
	\textsc{neg} \textsc{art} T. \textsc{neg}.\textsc{pfv} die \textsc{ant} faint still \textsc{cont}\\
	\glt \lq Tiare was not dead; she had just fainted.' \parencite[343]{Kieviet2017}
	
	\ex\label{exAppendixRapaNuiRestrictive4}
	\gll ¿Kai haʼamā koe i toʼo \textbf{nō} koe i te mauka mo taʼo i taʼa ʼumu?\\
	\phantom{¿}\textsc{neg}.\textsc{pfv} ashamed 2\textsc{sg} \textsc{pfv} take still 2\textsc{sg} \textsc{acc} \textsc{art} grass for cook \textsc{acc} \textsc{poss}:2\textsc{sg} earth\_oven\\
	\glt \lq Werenʼt you ashamed, that you just took the grass (as fuel) to cook your earth oven (without asking, even though the grass was mine)?' \parencite[343]{Kieviet2017}
	
	\ex\label{exAppendixRapaNuiRestrictive5}
	Context: He used to drink.\\
	\gll Te riva \textbf{nō}, e taʼero era, ʼina he tiŋaʼi i tāʼana huaʼai.\\
	\textsc{art} good still \textsc{ipfv} drunk \textsc{dist} \textsc{neg} \textsc{neutral} strike \textsc{acc} \textsc{poss}.3\textsc{sg} family\\
	\glt \lq Fortunately (=the good thing was), when he was drunk, he did not beat his family.' \parencite[268]{Kieviet2017}
\end{exe}	
\largerpage[2]
\subsubsection{Broadly modal and interactional uses}
\paragraph{Concessive protases}\label{appendixRapaNuiConcessiveAntecedent}
\begin{itemize}
	\item \textcite[59]{duFeu1996}, and \textcite[570]{Kieviet2017}.
	\item This function obtains in the fixed expression \textit{nōatu}, which could either be segmented as \textit{nō} plus \textit{atu} \lq away' or be a loan from Tahitian (see \cite[570 fn34]{Kieviet2017}), plus nominalised verb.
	\item As (\ref{exAppendixRapaNuiConcessiveP3}) shows, this construction is not restricted to concessives.
\end{itemize}

\begin{exe}
	\ex\label{exAppendixRapaNuiConcessiveP1}
	 \gll\textbf{Nōatu} te paŋahaʼa, te mahana te mahane e hāpī ena ʼi ira.\\
	never\_mind \textsc{art} heavy \textsc{art} day \textsc{art} day \textsc{ipfv} teach \textsc{dem} at \textsc{anaph}\\
	\glt \lq Even though itʼs heavy, they teach there day after day.\rq{ }\parencite[570]{Kieviet2017}

	\ex\label{exAppendixRapaNuiConcessiveP2}
	 \gll Pura oho au ki a ia mo uʼi pauró te tapati, \textbf{noatu} te roa o te kona hare.\\
	 \textsc{hab} go 1\textsc{sg} to \textsc{cont} 3\textsc{sg} \textsc{ben} see every \textsc{art} week never\_mind \textsc{art} long of \textsc{art} place house\\
	 \glt \lq I visit him regularly every week even though he lives far away.\rq{ }\parencite[59]{duFeu1996}

	\ex\label{exAppendixRapaNuiConcessiveP3}
	\gll\textbf{Nōatu} tōʼona ture mai.\\
	 never\_mind \textsc{poss}.3\textsc{sg} scold hither\\
	 \glt \lq Don't mind his scolding.' \parencite[305]{Kieviet2017}
\end{exe}

\paragraph{Concessive apodoses (i)}
\label{appendixRapaNuiConcessiveConsequent1}
\begin{itemize}
	\item \textcite[343]{Kieviet2017}.
	\item \textit{Nō} can mark the apodoses of concessive constructions.
\end{itemize}

\begin{exe}
	\ex \label{exAppendixRapaNuiConcessiveQ1}
	Context: Nowadays there are all kinds of things to take care of children.\\
	\gll  … e māuiui \textbf{nō} ʼana te ŋā poki.\\
	{} \textsc{ipfv} sick still \textsc{cont} \textsc{art} \textsc{pl} child\\
	\glt \lq But even so, children get sick.' \parencite[343]{Kieviet2017}

	\ex \label{exAppendixRapaNuiConcessiveQ2}
	\gll Ka rahi atu tāʼaku poki, e hāpaʼo \textbf{nō} e au ʼā.\\
	\textsc{contiguous} many away \textsc{poss}.1\textsc{sg} child \textsc{ipfv} care\_for still \textsc{agt} 1\textsc{sg} self\\
	\glt \lq Even if I have many children, I will still take care of them myself.' \parencite[344]{Kieviet2017}\footnote{The concessive conditional \lq even if' arises through the combination of \lq\lq contiguous" \textit{ka} and directional \textit{atu}; see \textcite[569–570]{Kieviet2017}.}
\end{exe}

\paragraph{Concessive apodoses (ii)}
\label{appendixRapaNuiConcessiveConsequent2}
\begin{itemize}
	\item  \textcite[570]{Kieviet2017}.
	\item This function obtains with the clause-initial collocation \textit{te me'e nō} \lq \textsc{art} thing only', which is an instantiation of the contrastive \textit{te Noun nō} construction discussed in \ref{appendixRapanuiRestrictive}.
\end{itemize}	
\begin{exe}
	\ex \label{exAppendixRapaNuiConcessiveQ3}
	\gll ʼApa te toe a au he mate; \textbf{te} \textbf{meʼe} \textbf{nō} īʼ a au e ora nō ʼā.\\
	half \textsc{art} remain \textsc{art} 1\textsc{sg} \textsc{neutral} die \textsc{art} thing still here/now \textsc{art}  1\textsc{sg} \textsc{ipfv} live still \textsc{cont}\\
	\glt \lq I almost died; even so, I am [still] alive.' \parencite[570]{Kieviet2017}.
		
	\ex \label{exAppendixRapaNuiConcessiveQ4}
	Context: His boat was like the other ones.\\
	\gll \textbf{Te} \textbf{meʼe} \textbf{nō}, ʼi ruŋa i tū vaka era ōʼona e ai rō ʼā e tahi pēʼue, e rua miro ʼi te kaokao o te vaka.\\
\textsc{art} thing still at above at \textsc{dem} boat \textsc{dist} \textsc{poss}.3\textsc{sg} \textsc{ipfv} exist \textsc{emph} \textsc{cont} \textsc{num} one mat \textsc{num} two wood at \textsc{art} side.\textsc{redupl} of \textsc{art} boat\\
\glt \lq However, in his boat there was a rug, and two poles on the sides of the boat.' \parencite[268]{Kieviet2017}
\end{exe}
\il{Rapanui|)}

\section{Saisiyat (sais1237, xsy)}\il{Saisiyat|(}
\label{appendixSaisiyat}
\subsection{Introductory remarks}
My understanding of the Saisiyat data has greatly profited from discussion with Elizabeth Zeitoun and Lalo a tahesh kaybabaw. Saisiyat has two (candidates for) \textsc{still} expressions: \textit{haisiya} and \textit{nahan}. Only the latter displays polyfunctionality. Note that forward slashes in examples from \citeauthor{Huang2007} (\citeyear{Huang2007}, \citeyear{Huang2008}) indicate intonation units.

\subsection{nahan}
\paragraph{General information}
\begin{itemize}
	\item Form: also transcribed as \textit{naehan}, \textit{naehaen}.
	\item Wordhood: independent grammatical word.
	\item Syntax: relatively mobile, but never occurs to the left of the predicate.
\end{itemize}

\subsubsection{As a \lq{}still\rq{ }expression}
\begin{itemize}
	\item \citeauthor{Huang2007} (\citeyear{Huang2007}, \citeyear[109–110]{Huang2008}) and \textcite[154–155]{ZeitounEtal2015}.
	\item Specialisation: there are few contextualised attestations of \textit{nahan} as \textsc{still} in the references. However, examples like (\ref{exAppendixSaisiyat1}–\ref{exAppendixSaisiyat3}) and (\ref{exAppendixSaisiyatDecrement1}) below, when taken together, give a fairly strong indication that this marker conforms to my definition. For instance, in (\ref{exAppendixSaisiyat1}) \textit{nahan} not only indicates that the child continues to be small, but it also evokes a prospective perspective towards a time when that is no longer the case and the child can be made swim. 
	\item Pragmaticity: the data do not allow for conclusions.
	\item Polarity sensitivity: there are very few examples of phasal polarity \textit{nahan} together with negation. The available examples (e.g. \cite[118]{Huang2008}; \cite[529]{ZeitounEtal2015}) indicate that this combination yields \textsc{not yet}. The latter concept is more commonly expressed by a separate item \textit{}\rq{}i\rq{}ini\rq{}. \textsc{no longer} is marked by negation plus completive \mbox{=\textit{ila}}.
\end{itemize}

\begin{exe}
	\ex\label{exAppendixSaisiyat1}
	\gll Hini korkoring 'ol'ol'an \textbf{naehan}, 'izip pa-p-lalangoy!\\
	this child small still \textsc{neg}.\textsc{lnk} \textsc{caus}-\textsc{dynamic}-swim\\
	\glt \lq This child is still small, so don't make him swim!\rq{ }\parencite[265]{ZeitounEtal2015}

	\ex\label{exAppendixSaisiyat2}
	\gll Hini malat ka-sh-p<in>asha-ha-l ka somay, kayzaeh \textbf{naehan} kay='in-'otoeh.\\
	this knife \textsc{real}-\textsc{pat}.\textsc{foc}-pierce<\textsc{pfv}>one-times \textsc{acc} bear good still \textsc{neg}.\textsc{lnk}=\textsc{stat}-break\\

	\glt \lq This knife was used to kill a bear once but it is still good and not yet broken.' \parencite[574]{ZeitounEtal2015}

	\ex\label{exAppendixSaisiyat3}
	Context: From a rendition of the Pear Story. A man is up in a tree, while boys steal some of the fruit he has harvested.\\
	\gll Isaza tatini' rima' r<om>okrok \textbf{nahan} babaw ka boway.\\
	that old\_man \textsc{agt}.\textsc{foc}:go pick<\textsc{agt}.\textsc{foc}> still above \textsc{acc} fruit\\
	\glt \lq Lăorén hái zài shùshàng zhai shŭigŭo / The old man was still up in the tree to pick fruits.\rq{ }(\cite[589]{Huang2007}, \citeyear[110]{Huang2008})
\end{exe}

\subsubsection{Uses on the fringes of \lq{}still\rq{}}
\paragraph{Scalar contexts}\label{appendixSaisiyatScalar}
\begin{itemize}
	\item At minimum the attestation in (\ref{exAppendixSaisiyatDecrement1}) involves a scalar context, in the form of a decrease over time.
\end{itemize}

\begin{exe}
	\ex \label{exAppendixSaisiyatDecrement1}
	\gll Yako pash-raya' ka raromaeh, kita'-en 'akoy \textbf{naehan}. Rima' pas-'izaeh-en naehan la'oz=ila kin-'oehoep\\
	1\textsc{sg}.\textsc{nom} chop-above \textsc{acc} bamboo see-\textsc{pat}.\textsc{foc} many still go make-again-\textsc{pat}.\textsc{foc} still enough=\textsc{cmpl} \textsc{intens}=dense\\
	\glt \lq I chopped bamboo, but there was still a lot. I went to do it again, and it did not look so dense.' \parencite[586]{ZeitounEtal2015}
\end{exe}

\subsubsection{Broadly adverbial temporal-aspectual functions}
\paragraph{Iterative and Restitutive}
\label{appendixSaisiyatIterative}
\begin{itemize}
	\item \citeauthor{Huang2007} (\citeyear{Huang2007}, \citeyear[106–107, 101–110]{Huang2008}) and \textcite[155]{ZeitounEtal2015}. 
	\item According to \citeauthor{Huang2007} (\citeyear{Huang2007}, \citeyear{Huang2008}) this use typically, but not exclusively, obtains with telic predicates and/or in combination with the completive marker =\textit{ila}.
	\item Examples (\ref{appendixSaisiyatIterative1}, \ref{appendixSaisiyatIterative2}) illustrate iterative uses.
	\item Concerning restitutive uses, with most examples in question it is not entirely clear whether they have an iterative and/or restitutive reading.  \textcite[586]{Huang2007} considers (\ref{exAppendixSaisiyatRestitutive1}) to be iterative rather than restitutive, on the ground that \lq\lq there is a felt discontinuous phase between the men's twice of being youth". However, this example features the restoration of an initial state (being young) that has been undone by a process to the opposite (growing old). The clearest case of restitution is found in  (\ref{exAppendixSaisiyatRestitutive2}).
\end{itemize}

\begin{exe}
	\ex\label{appendixSaisiyatIterative1}
	\gll Yako ma-ngoip r<om>a'oe: ka 'io' \textbf{naehan}.\\
	1\textsc{sg}.\textsc{nom}	\textsc{agt}.\textsc{foc}-forget drink<\textsc{agt}.\textsc{foc}> \textsc{acc} medicine still\\
	\glt \lq I forgot to take my medicine again.' \parencite[206]{Wang2018}
	
	\ex\label{appendixSaisiyatIterative2}
	 \gll Yako hayza' ila min-osa' raroemoe'an ko'hael 'am rima' \textbf{nahaen}.\\
	1\textsc{sg}.\textsc{nom} \textsc{exist} \textsc{cmpl} \textsc{agt}.\textsc{foc}-go Xiangtian\_lake next\_year \textsc{fut} go:\textsc{agt}.\textsc{foc} still\\
	\glt \lq Wŏ yĭqían qù-gùo Xìangtianhú, míngnían hái yào qù / I have been to Xiantian Lake before. I am going there again next year.' \parencite[101]{Huang2008}

 	\ex\label{exAppendixSaisiyatRestitutive1}
	Context: It is Saisiyat believe that their ancestors, when grown old, would molt and then turn young again.\\
	\gll Soː m-olaw kitaʼ-en maʼ ʼalʼalak ila \textbf{nahan}.\\
	if \textsc{agt}.\textsc{foc}-molt see-\textsc{pat}.\textsc{foc} also young \textsc{cmpl} still\\
	\glt \lq They molted and looked young again.' (\cite[587]{Huang2007}, \citeyear[106–107]{Huang2008})
	
		\ex\label{exAppendixSaisiyatRestitutive2}
	 Context: About preserved bamboo shoots.\\
	 \gll So: 'a-s<m>i’ael	ila /  senge-en \textbf{nahan} ray ralom\\
	 if \textsc{fut}-eat<\textsc{agt}.\textsc{foc}> \textsc{cmpl} {} soak-\textsc{pat}.\textsc{foc} still \textsc{loc} water\\
	 \glt \lq If (one) feels like eating (it) [bamboo shoots], soak them again [i.e. rehydrate them] in water.' \parencite[588]{Huang2007}
\end{exe}

\paragraph{Iterative via increment}
\label{appendixSaiyiatIterativeIncrement}
\begin{itemize}
	\sloppy
	\item This function obtains in combination with complex de-numeral verbs; see \citeauthor{ZeitounEtal2010} (\citeyear[575–577]{ZeitounEtal2010}, \citeyear[522]{ZeitounEtal2015}) on the latter.
\end{itemize}

\begin{exe}
	\ex\label{exAppendixSaisiyatIterativeViaIncrement1}
	\gll Yako k<om>oe'ha-l nanaw t<om>awbon=o 'okik hoepay. 'Am=\textbf{mon}-\textbf{ha}-\textbf{l} \textbf{naehan}\\
	1\textsc{sg}.\textsc{nom} pound<\textsc{agt}.\textsc{foc}>one-do\_times only pound<\textsc{agt}.\textsc{foc}>=\textsc{conj} \textsc{neg}:\textsc{lnk}:\textsc{stat} tired \textsc{prog}=\textsc{agt}.\textsc{foc}:do\_times-one-times still\\
	\glt \lq I pound glutinous cake once but I was not tired. I did it once more.' \parencite[530]{ZeitounEtal2015}
	
	\ex\label{exAppendixSaisiyatIterativeViaIncrement2}
	\gll Yao \rq{}am=mari’ ka lapwar boay \rq{}a-k<m>ai:, \rq{}okay kay-hoero:, \rq{}oka\rq{}=ila=o \textbf{mon}-\textbf{ha}-\textbf{l} \textbf{naehan}, k<om>ay-hoero:=ila mari\rq{}=ila ka boay noka lapwar.\\
1\textsc{sg}.\textsc{nom} \textsc{prog}=take \textsc{acc} guava fruit \textsc{prog}-hook<\textsc{agt}.\textsc{foc}> \textsc{neg}:\textsc{lnk} hook-succeed \textsc{neg}=\textsc{cmpl}=\textsc{conj} \textsc{agt}.\textsc{foc}:do\_times-one-times still hook<\textsc{agt}.\textsc{foc}>succeed=\textsc{cmpl} take=\textsc{cmpl} \textsc{acc} fruit \textsc{gen} guava\\
\glt \lq I was trying to gather guavas but I could not hook them; I tried again and succeeded in catching guavas.'  \parencite[524]{ZeitounEtal2015}

\ex\label{exAppendixSaisiyatIterativeViaIncrement3}
	\begin{xlist}
		\exi{A:} 
	\textit{Nimon tinawbon kinopilazan ila?}\\
	\lq  How many times did you pound the glutinous cake?\rq{}
		\exi{B:}
		\gll mita' k<in>o-too-l-an=ila 'am \textbf{k}<\textbf{in}>\textbf{o}-\textbf{posha}-\textbf{l}-\textbf{an} \textbf{naehan} 'am=s<om>I’ael=ila, saboeh k<in>o-aseb-an\\
		1\textsc{pl}.\textsc{gen} pound<\textsc{pfv}>-three-do\_times-\textsc{loc}.\textsc{foc}=\textsc{cmpl} \textsc{fut} pound<\textsc{pfv}>-two-do\_times-\textsc{pat}.\textsc{foc} still \textsc{irr}=eat<\textsc{agt}.\textsc{foc}>=\textsc{cmpl} all pound<\textsc{pfv}>five-do\_times-\textsc{loc}.\textsc{foc}\\
		\glt \lq We pounded it three times. We have to pound it two more times, and then we will eat it. In all (we need to) pound it five times.' \parencite[530]{ZeitounEtal2015}
	\end{xlist}
\end{exe}

\paragraph{First, for now, for a while}\label{appendixSaisiyatFirst}
\begin{itemize}
	\item \citeauthor{Huang2007} (\citeyear{Huang2007}, \citeyear[116–120]{Huang2008}) and \textcite[154]{ZeitounEtal2015}.
	\item \textcite{Huang2008} stipulates that a reading of precedence goes back the additive (\appref{appendixSaisiyatAdditive}) use of \textit{nahan} (\appref{appendixSaisiyatAdditive}) and a pattern akin to omnisyndetic coordination, as in (\ref{exAppendixSaisiyatFirst4}). Note that this example also includes a verb \lq do first'; furthermore, no such discourse pattern is attested in \citeauthor{EarlyWhitehorn2003}'s (\citeyear{EarlyWhitehorn2003}) extensive collections of texts in nearby \ili{Paiwan} (\ref{appendixPaiwan}), where the same coexpression is found.
\end{itemize}
	
\begin{exe}
	\ex 
	\gll Yako 'a-s<m>i'ael ka boriː pas'i'is-in \textbf{naehan}=o shae'en-en=ila\\
	1\textsc{sg}.\textsc{nom} \textsc{prog}-eat<\textsc{agt}.\textsc{foc} \textsc{acc} meat chew-\textsc{pat}.\textsc{foc} still=\textsc{conj} swallow-\textsc{agt}.\textsc{foc}=\textsc{cmpl}\\
	\glt \lq When I am eating meat, I (first) chew it and then I swallow it.\rq{ }\parencite[546]{ZeitounEtal2015}
	
	\ex\label{exAppendixSaisiyatFirst4}
	\gll Yao minSalaʼ baiw ka taumoʼ \textbf{naehan} baiw ila ka lapuwar \textbf{naehan} baiw ila ka zozoʼ \textbf{naehan}.\\
	1\textsc{sg}.\textsc{nom} \textsc{agt}.\textsc{foc}:do\_first \textsc{agt}.\textsc{foc}:buy \textsc{acc} banana still \textsc{agt}.\textsc{foc}:buy \textsc{cmpl} \textsc{acc} guava still \textsc{agt}.\textsc{foc}:buy \textsc{cmpl} \textsc{acc} plum still\\
	\glt \lq I first bought bananas, and then bought guavas, and then bought plums.' \parencite[116]{Huang2008} 

	\ex\label{exAppendixSaisiyatWhile2}
	\gll Ma’an pa-tishko-\rq{}aish-in hi baki’ parain kosha: \lq\lq Yami 'am='oka’=ila=’i wai', 'am=ma-shangay=ila \textbf{naehan}."\\
	1\textsc{sg}.\textsc{nom} \textsc{caus}-say-in\_passing-\textsc{pat}.\textsc{foc} \textsc{acc} grandfather P. say \phantom{\lq\lq}1\textsc{pl}.\textsc{excl} \textsc{irr}=\textsc{neg}=\textsc{cmpl}=\textsc{lnk}	 come \textsc{irr}=\textsc{agt}.\textsc{foc}-rest=\textsc{cmpl} still\\
	\glt \lq I asked (him/her) to tell Grandfather Parain that we would not be coming (as) we want to rest [for a while].' (\cite[262]{ZeitounEtal2015}; Elisabeth Zeitoun, p.c.)
	
	\ex\label{exAppendixSaisiyatWhile3}
	\gll 'Obay ki kizaw lasiwazay \textbf{nahaen}.\\
	O. \textsc{com} K. separate still\\
	\glt \lq Obay and Kizaw separate on a temporary basis.' \parencite[231]{Huang2008} 

	\ex\label{exAppendixSaisiyatWhile4}
	\gll Yako kal-'aish kala 'okay \textbf{naehan}, ma-'ngel=ila.\\
	1\textsc{sg}.\textsc{nom} pass-in\_passing \textsc{loc}.\textsc{pl} O. still \textsc{stat}-slow=\textsc{cmpl}\\
	\glt \lq I stopped by Okayʼs home for a while and was late (for the meeting).' \parencite[561]{ZeitounEtal2015}
\end{exe}	

\paragraph{Prospective \lq eventually\rq{}}\label{appendixSaisiyatProspective}
\begin{itemize}
	\item \textcite[154]{ZeitounEtal2015}.
\end{itemize}
\begin{exe}
	\ex\label{exAppendixSaisiyatLater1}
	\gll Ma’an noe-h<m>iwa' 'oka'=ila='i-k somaom	shi-hiwa’. 'okay	paloso=ila, ka-hirhir-in \textbf{naehan}, kayzaeh paloso:.\\
	1\textsc{sg}.\textsc{gen} \textsc{inst}.\textsc{nmlz}-saw<\textsc{agt}.\textsc{foc}> \textsc{neg}=\textsc{cmpl}=\textsc{lnk}-\textsc{stat}	sharpened \textsc{inst}.\textsc{foc}-saw \textsc{neg}.\textsc{lnk} cut\_in\_pieces=\textsc{cmpl} \textsc{irr}-whet-\textsc{pat}.\textsc{foc} still can cut\_in\_pieces\\
	\glt \lq My saw is not sharpened, so when I use it to saw, I can’t cut (things) in pieces (properly). I will whet it later so that it (can be used) to cut (things) in pieces.' \parencite[506]{ZeitounEtal2015}
	
	\ex\label{exAppendixSaisiyatLater2}
	\gll Maʼan ka-obaang-an no<m>obaang \lq{}okik lalʼoz. Rimaʼ baeiw \textbf{naehan}!.\\
	1\textsc{sg}.\textsc{gen} \textsc{rl}-draw-\textsc{loc}.\textsc{foc} \textsc{inst}<\textsc{agt}.\textsc{foc}>draw \textsc{neg}.\textsc{lnk}.\textsc{stat} enough go.\textsc{imp} buy.\textsc{imp} still\\
	\glt \lq I do not have enough paper and pens; go and buy some later!\rq{ }\parencite[506]{ZeitounEtal2015}.

	\ex\label{exAppendixSaisiyatLater3}
	\gll Yako m-ia-ralom=a=tomal, \rq{}am=rima\rq{} mil-ralom \textbf{naehan}.\\
	1\textsc{sg}.\textsc{nom} \textsc{agt}.\textsc{foc}-want-water=\textsc{lnk}=very \textsc{irr}=go \textsc{agt}.\textsc{foc}-drink\_water still\\
	\glt \lq I am very thirsty and I will go and drink water in a while.\rq{ }\parencite[504]{ZeitounEtal2015}
\end{exe}
\pagebreak
\subsubsection{Marginality}\label{appendixSaisiyatMarginality}
\begin{itemize}
	\item \textcite[110]{Huang2008}.
	\item Only one example of a marginality use is found in the data.
\end{itemize}
\begin{exe}
	\ex 
	\gll \rq{}ima hasa\rq{}  h<om>ayap kabkabahae\rq{} koSa\rq{}-en kabkabahae\rq{} \textbf{nahaen}.\\
\textsc{prog} unable fly<\textsc{agt}.\textsc{foc}> bird say-\textsc{pat}.\textsc{foc} bird still\\
\glt \lq Búhùi fei dė nĭao háishì jìaozùo nĭao. / Birds that cannot fly are still called birds.\rq{ }\parencite[110]{Huang2008}
\end{exe}


\subsubsection{Additive and related uses}
\paragraph{Additive}\label{appendixSaisiyatAdditive}
\begin{itemize}
	\item \citeauthor{Huang2007} (\citeyear{Huang2007}, \citeyear[108–113]{Huang2008}).
	\item Also see (\ref{exAppendixSaisiyatFirst4}) above.
	\item In cases like (\ref{exAppendixSaisiyatIncrement3}) there is an overlap with the iterative function: \lq I see another boy coming \sim {} again I see a boy coming'.
\end{itemize}
\begin{exe}
	\ex
	\gll Sia sh<om>bet ka ma\rq{}iaeh=o \rq{}<om>angang \textbf{naehan} ka ma\rq{}iaeh.\\
	3\textsc{sg}.\textsc{nom} beat<\textsc{agt}.\textsc{foc}> \textsc{acc} person=\textsc{conj} scold<\textsc{agt}.\textsc{foc}> still \textsc{acc} person\\
	\glt \lq He beat and [also] scolded people.' \parencite[16]{Wang2018}

	\ex Context: Speaking about different types of deer.\\
	\gll KoSa\rq{}-en ... kasakiray bangol ka=wa\rq{}ae\rq{} ... \rq{}aehae\rq{} \textbf{nahan} sinkano'on.\\
	say-\textsc{pfv} {} field forest \textsc{nom}=deer {} one still what\\
	\glt \lq (There is one kind) called field deer … There is one more watchamacalit.' \parencite[590]{Huang2007}
	
	\ex
	\gll Moyo kaysa\rq{}an m-wai\rq{} \rq{}oka\rq{}=ila=o kari\rq{}ael hayzaː \textbf{naehan} \rq{}a-m-wai\rq{} rini, 'am=k<om>iim \rq{}iakin.\\
	2\textsc{pl}.\textsc{nom} today \textsc{agt}.\textsc{foc}-come \textsc{neg}=\textsc{cmpl}=\textsc{conj} day\_after\_tomorrow have still \textsc{irr}-\textsc{agt}.\textsc{foc}-come here \textsc{irr}=look\_for<\textsc{agt}.\textsc{foc} 1\textsc{sg}.\textsc{acc}\\
	\glt \lq You are coming today and the day after tomorrow, someone else will come here to look for me (lit. … the day after tomorrow not, there is someone else [there is still] …).' \parencite[174]{ZeitounEtal2015}
	
	\ex\label{exAppendixSaisiyatIncrement3}
	Context: From a rendition of the pear story. A boy with a goat has passed by. Now a boy on a bicycle is passing by the same location.\\
	\gll O: rima\rq{} ila hiza / kita’-en m-wa:i\rq{} ila \textbf{naehan} / \rq{}aehae\rq{} kaːː / kamo\rq{}alay / kamamanraːan / \rq{}ima papama\rq{} rayːː / kapapama\rq{}anːː\\
	\textsc{dm} \textsc{agt}.\textsc{foc}:go \textsc{cmpl} there {} see-\textsc{pat}.\textsc{foc} \textsc{agt}.\textsc{foc}-come \textsc{cmpl} still  {} one \textsc{nom} {} young\_man {} man {} \textsc{prog} \textsc{agt}.\textsc{foc}:ride \textsc{loc} {} vehicle\\
	\glt \lq Tamėn zŏu lė. Yòu kàndào lìngyigė nánháizi lái lė. Ta qízhė jĭaotàche. / (Off) they went. (Then I) see another boy coming; he was riding a bike.’ \parencite[108–109]{Huang2008}
\end{exe}

\subsubsection{Broadly modal and interactional uses}
\paragraph{Polite commands}\label{appendixSaisiyatPoliteness}
\begin{itemize}
	\item \textcite[119–120]{Huang2008}.
	\item In directive speech acts, \textit{nahan} can serve to contribute politeness.
	\item This can clearly be traced back to the \lq first, for now\rq{ }use (\appref{appendixSaisiyatFirst}). \textcite{Huang2008} suggests that it relies on explicitly depicting the invitation or command as a prelude, thereby acknowledging that the addressee has other things to do.
\end{itemize}

\begin{exe}
	\ex
	\gll Si\rq{}ael \textbf{nahaen}.\\
	eat.\textsc{imp} still\\
	\glt \lq Chi gė dongxi (zài zŏu) ba! / Come have a bite (before you leave)!\rq{ }\parencite[120]{Huang2008}
\end{exe}
\il{Saisiyat|)}

\section{Ternate-Tidore (tft/tvo, tern1247/tido1248)}
\il{Ternate|(}\il{Tidore|(} 
\label{appendixTernateTidore}

\subsection{Introductory remarks}
The data encompass Ternate (tft, tern1247) and Tidore (tvo, tido1248). These two are mutually intelligible and can arguably be considered as two dialects of one language (\cite{Voorhoeve1987}, \citeyear{Voorhoeve1988}). Judging from the data, the relevant expression (\textit{moju}) has a similar, if not even identical, set of functions in the two varieties. Ternate-Tidore's second  \textsc{still} expression \textit{masi}, a loan from Malay, is attested with only five tokens in the data consulted. Therefore I did not include it in my sample. Last, but far from least, I am indebted to Miriam van Staden for discussing some of the Ternate-Tidore data with me. 

\subsection{moju}

\subsubsection{General information}
\begin{itemize}
	\item Wordhood: free morpheme.
	\item Syntax: fixed position, following the predicate and its arguments.
\end{itemize}

\subsubsection{As a \lq{}still\rq{ }expression}
\label{appendixTernateTidoreStill}
\begin{itemize}
	\item \textcite[92–93]{Hayami2001}, \textcite[32]{PikkertEtal1994} and \textcite[145, 228, 242–243]{vanStaden2000}.
	\item Specialisation: the descriptions, especially \textcite[92]{Hayami2001}, clearly indicate that this item conforms to my definition. This also becomes evident in examples like (\ref{exAppendixTernate1}–\ref{exAppendixTernate4}). For instance, in (\ref{exAppendixTernate1}), \textit{moju} not only construes the husband's absence as continuing, but also anticipates his later return.
	\item Pragmaticity: the data allow no conclusion.
	\item Polarity sensitivity: no attestations of \textit{moju} plus negation in the data. \textsc{not yet} is expressed by a dedicated item, \textit{hang}. \textsc{no longer} is marked by \mbox{\textit{re}(\textit{w})\textit{a}}. \textit{Moju} is, however, attested in combination with \textit{hang} \lq not yet'.
	\item Further notes: \textit{moju} is commonly used in questions that follow a \lq still or no longer' pattern, where it is also the only possible affirmative reply (\ref{exAppendixTernate4}). \textit{Moju} is furthermore occasionally attested in combination with the Malay loan \textit{masi} \lq still'.
\end{itemize}

\begin{exe}
	\ex\label{exAppendixTernate1}
	Context: A man's wife, an angelic creature, is returning to the heavens while her husband is at sea. When he returns, she is gone.\\
	\gll  Ge\sim ge ma-raa Jafar Sadik=tai toma ngolo=tai \textbf{moju}, yau nyao=tai \textbf{moju}. Wako isa yang.\\
	\textsc{redupl}\sim there \textsc{poss}.3-husband J. S.=seaward \textsc{loc} sea=seaward still fish fish=seaward still return landwards not\_yet.\\
	\glt \lq Thus, her husband Jafar Sadik was still at sea, still fishing for fish at sea. He had not yet returned.' \parencite[375]{vanStaden2000}

	\ex\label{exAppendixTernate2}
	 Context: About circumcision and children’s sense of shame.\\
	\gll Gata sema ona=kann lamo ena rasa mae ua? Ona lamo lau ua rasa mae ua, tege. Kalau ona kini \textbf{moju}=kan, ona maha-waro ua waje.\\
	Like be 3\textsc{pl}=\textsc{emph} large 3\textsc{non}.\textsc{human} feel shame \textsc{neg} 3\textsc{pl} large very  \textsc{neg} feel ashamed  \textsc{neg} manner.there if 3\textsc{pl} small still=\textsc{emph} 3\textsc{pl} wait-know \textsc{neg} say\\
	\glt \lq Like when they are big, they will feel ashamed, right? They who are not big do not feel ashamed, like that. When they are still young, they don’t know, so they say.' \parencite[478]{vanStaden2000}

	\ex\label{exAppendixTernate3}
	Context: About nutmeg. The base of a young fruit is white, that of a ripe one is dark.\\
	\gll Gee po-isa gee mai himo hang gee rata\sim rata isa gee fere hang si-moi. Barang ena ma-koa ma-, ma-tano budo \textbf{moju} si.\\
	That next\_to-landward that but old not\_yet that \textsc{redupl}\sim align landward that climb not\_yet \textsc{caus}-finish because 3\textsc{non}.\textsc{human}	\textsc{poss}.3-what \textsc{poss}.3 \textsc{poss}.3-base\_of\_fruit white still first\\
	\glt \lq The one landward side is not yet old, those aligned landward ones are all not yet harvested. Because their, what was that, the base of fruit is still white.' \parencite[257]{Hayami2001}
	
	\ex\label{exAppendixTernate4}
	\gll Sema nyao \textbf{moju} bolo rewa? – \textbf{Moju}.\\
	\textsc{cop} fish still or no\_longer {} still\\
	\glt \lq Is there [still] any fish left? -- [Yes, there] still [is].' \parencite[243]{vanStaden2000}
\end{exe}

\subsubsection{Uses on the fringes of \lq{}still\rq{}}
\paragraph{Scalar contexts}\label{appendixTernateScalar}
\begin{itemize}
	\item A few instances of \textit{moju} in the data consulted involve scalar context. These include the decrement use in (\ref{exAppendixTernateDecrement}) and the \lq still only' use in (\ref{exAppendixTernateStillOnly}); note how the latter features no overt \lq only\rq{ }operator.
\end{itemize}

\begin{exe}
	\ex\label{exAppendixTernateDecrement}
	\gll Ma-nyiha dofu \textbf{moju} gee.\\
	\textsc{poss}.3-leftover much still that\\
	\glt \lq There is still a lot left.' \parencite[93]{Hayami2001}
	
	\ex
	\label{exAppendixTernateStillOnly}
	 Context: An angelic being and mother of several children wants to return to the heavens. On her first attempt of leaving, the youngest child cried and she returned to comfort it. Now she is attempting to leave for the second time.\\
	\gll Konora ine \textbf{moju} ngofa kage reke, yali mina uci tora yali.\\
	middle upwards still child be\_shocked cry again 3\textsc{sg}.\textsc{f} descend downwards again\\
	\glt \lq Half way up again (lit. [when she was] still [only] half way up), the child was frightened [and] cried, she came down again.' (\cite[375]{vanStaden2000})
\end{exe}

\subsubsection{Additive and related uses}

\paragraph{Additive}\label{appendixTernateAdditive}
\begin{itemize}
	\item Several attestations of \textit{moju} involve an additive use.
\end{itemize}
\begin{exe}
	\ex \gll Kai fo=tike ri-ngofa. Sema ngofa rai=m fo=tike koa \textbf{moju}?\\
	marry 1\textsc{pl}.\textsc{incl}=look\_for \textsc{poss}.1\textsc{sg}-child own child already=\textsc{attention} 1\textsc{pl}.\textsc{incl}-look\_for what still?\\
	\glt \lq We marry and expect children. We already have children and what more do we want?' \parencite[92]{Hayami2001}
	
	\ex Context: Discussing who is seen on a particular photograph.\\
	\begin{xlist}
		\exi{A:} \textit{Min sulo foto mina Dahlan se ngofangofa ifa.}\\
		\lq She mustn’t order a picture to be made of her Dahlan and other children.'
		
		\exi{B:} \textit{E?}\\
		\lq Hey?'
		
		\exi{A:}\gll Ma oe, Si \textbf{moju} gua.\\
		but yeah S. still there.\textsc{neg}\\
		\glt \lq But yeah Si also, no?'
		
		\exi{C:}\textit{Oe foto keluarga.}\\
		\lq Yeah a picture of the whole family.' \parencite[503–504]{vanStaden2000}	
		\end{xlist}

	\ex Context: A man is attempting to win his wife back. Her father (a heavenly king) has set up a near-impossible task for him proof his worthiness.\\
	\gll Jou kolano=re wahe \lq\lq Ah, ngona=ge aku yali tapi duga rimoi nde ua, rimoi \textbf{moju}. Ngoto so-g-uci ngana rimoi \textbf{moju}."\\
	lord king=here say \phantom{\lq\lq}\textsc{interj} 2\textsc{sg}=there may again but only one 3\textsc{non}.\textsc{human}:there \textsc{neg} one still 1\textsc{sg} \textsc{caus}-\textsc{nmlz}-descend 2\textsc{sg} one still\\
	\glt \lq The king said: {\lq\lq}Ah, you managed this one  [task], but this is not the only one, there is another one. I give you one more".' \parencite[390]{vanStaden2000}
\end{exe}
\il{Ternate|)}\il{Tidore|)} 

\section{Western Dani (dnw, west2594)}\il{Dani, Western|(}
\label{appendixWesternDani}
\subsection{Introductory remarks}
My understanding of Western Dani has greatly profited from discussion with Peter Barclay.

\subsection{awo}
\subsubsection{General information}
\begin{itemize}
	\item Wordhood: independent grammatical word, invariable.
	\item Syntax: fixed, immediately preedecing the predicate.
\end{itemize}

\subsubsection{As a \lq{}still\rq{ }expression}
\begin{itemize}
	\item \textcite[304, 440–441]{Barclay2008}.
	\item Specialisation: that \textit{awo} conforms to my definition is supported by examples like (\ref{exAppendixWesternDani1}–\ref{exAppendixWesternDani3}). For instance, (\ref{exAppendixWesternDani1}) not only involves a seamless continuation of David's looks, but also evokes an alternative scenario, as Goliath expects to be confronted by an opponent of more mature appearance. Further, albeit indirect, evidence comes from \textit{awo}'s uses as \textsc{not yet} in the absence of negation (\appref{appendixWesternDaniNotYet}).
	\item Pragmaticity: hard to judge; most examples seem to involve the neutral scenario.
	\item Polarity sensitivity: inner negation (often via the negative \lq\lq intentive mood") yields \textsc{not yet}. This is also used as a signal of precedence.
	\item Further note: \textit{awo} often features in cohesive clauses of the type \lq while still V-ing…', as in (\ref{exAppendixWesternDani2}, \ref{exAppendixWesternDani3});  also see \textcite[575, 619]{Barclay2008}.
\end{itemize}
\largerpage[2]

\begin{exe}
	\ex \label{exAppendixWesternDani1} Context: David faces the Philistine Goliath (1 Samuel 17)\\
	\gll Aap Pilitin mendek nogo nen {ena koon}-ogo Ndawut pek-ka-ge nagagerik at \textbf{awo} tawe etenggen teretak eebe abu kigirikwe ka-ge nagagerik, meek o-mbar-eegarek\\
	man Philistine kind \textsc{anaph} from focussed-and David checked-\textsc{obj}.3\textsc{sg}-\textsc{subj.sg} \textsc{consec:dist.pst:}3\textsc{sg}  3\textsc{sg} still young face thin \textsc{poss}.3\textsc{sg}:body \textsc{intens} handsome saw:\textsc{obj}.3\textsc{sg}-\textsc{subj.sg} \textsc{consec:dist.pst:}3\textsc{sg} despised \textsc{obj}.3\textsc{sg}-think-\textsc{dist.pst}:\textsc{subj}.3\textsc{sg}\\
	\glt \lq The Philistine looked David over and saw that he was still young, thin and very handsome and he despised him.' \parencite[94]{Barclay2008}

	\ex \label{exAppendixWesternDani2}
	\gll \textbf{Awo} yo-ge kagak enegen ogut a-gagerak nogo lek a-ge nagagirik enegen {pagak yer}-eegerak\\
	still talking-\textsc{subj.sg} 3\textsc{sg}.\textsc{sim}:\textsc{ds} \textsc{poss}.3\textsc{sg}:eyes blind become-\textsc{dist.pst}:\textsc{subj}.3\textsc{sg} \textsc{anaph} \textsc{neg} become-\textsc{subj.sg} \textsc{consec:dist.pst:}3\textsc{sg} \textsc{poss}.3\textsc{sg}:eyes  open.\textsc{pass}-\textsc{dist.pst}:\textsc{subj}.3\textsc{sg}\\
	\glt \lq While he was still speaking his eyes which had become blind were healed and he could see.' \parencite[148]{Barclay2008}

	\ex \label{exAppendixWesternDani3}
	\gll Nin-ombo Abarakam o Karan na-ga'lek logonet, \textbf{awo} o Metopotamiya wona-ge me at Ala mondok tiyan-ak menggerak nogo kaa-wak nduk wa-gagerak.\\
\textsc{poss}.1\textsc{pl}-ancestor Abraham \textsc{loc} Canaan went-\textsc{neg} \textsc{sim}:\textsc{ss} still \textsc{loc} Mesopotamia was-\textsc{subj}.\textsc{sg} while 3\textsc{sg} God \textsc{intens} high-\textsc{adj} \textsc{prs.hab}:\textsc{subj}.3\textsc{sg} \textsc{anaph} see.\textsc{obj}.3\textsc{sg}-\textsc{intentive}.\textsc{fut}:\textsc{subj}.3\textsc{sg} \textsc{purp} come-\textsc{dist.pst}:\textsc{subj}.3\textsc{sg}\\
	\glt \lq Before our ancestor Abraham went to Canaan and while he was still in Mesopotamia, the God of glory came in visible form.' \parencite[619]{Barclay2008}
\end{exe}
\largerpage
\subsubsection{Uses related to other phasal polarity concepts}
\paragraph{Not yet}\label{appendixWesternDaniNotYet}
\begin{itemize}
\item \textcite[192, 440–441]{Barclay2008}.
\item This function is found in two types of contexts, both lacking an overt predicate.
	\begin{itemize}
		\item In a clause pattern featuring a nominal subject plus \textit{awo}. This is rare, and only three examples are found in \citeauthor{Barclay2008}'s (\citeyear{Barclay2008}) grammar: one with a plain noun in (\ref{exAppendixWesternDaniNotYet1}) and two with a complex noun phrase in (\ref{exAppendixWesternDaniNotYet2}, \ref{exAppendixWesternDaniNotYet3}). The latter instances both include the affirmative form of the \lq\lq intentive mood", a verbal paradigm \lq\lq often used in contexts where the action has not yet occurred … and when the course of action to be followed in certain circumstances is outlined" \parencite[236]{Barclay2008} and which can be used in the formation of complex noun phrases \parencite[238]{Barclay2008}. The lack of an overt predicate is visible in the fact that \textit{awo} in all other instances invariably precedes the predicate (i.e. in the case of atransitive predicates such as \lq it is still a river' we would expect the inverse order \textit{awo yi}). The examples suggest that the noun phrase must be associated with a characteristic periodic development (river > \lq swell', harvest season > \lq come', time > \lq come'). 

\item As an interjection. This use carries additional pragmatic flavours: \lq wait', \lq not now, maybe later' (Peter Barclay, p.c.), as in (\ref{exAppendixWesternDaniNotYet3}).
	\end{itemize}
\end{itemize}

\begin{exe}
	\ex\label{exAppendixWesternDaniNotYet1}
	\gll Yi \textbf{awo}.\\
	river still\\
	\glt \lq The river has not yet been swelled (by the rains).' \parencite[440]{Barclay2008}

	\ex\label{exAppendixWesternDaniNotYet2}
	\gll Anggen mban-iyak e-yom \textbf{awo}	 me imbirak lagan-gge lago-wak nduk tep-p-inip o.\\
fruit pick-\textsc{intentive} \textsc{poss}.3\textsc{sg}-time still while both grow-\textsc{sg} \textsc{cont}-\textsc{imp.fut}:\textsc{subj}.3\textsc{sg} \textsc{purp} let-\textsc{obj}.3\textsc{sg}-\textsc{imp.pl} \textsc{dm}\\
	\glt \lq While the harvest has not yet come (lit. while the harvest time, still), let them both grow together.'  \parencite[441]{Barclay2008}

	\ex\label{exAppendixWesternDaniNotYet3}
	\gll Nin-oor-iyak eyom \textbf{awo} kagak, aa’nduk nin-oor-i wa-gandak ya?\\
\textsc{obj}.1\textsc{pl}-hit-\textsc{intentive} time still 3\textsc{sg}.\textsc{sim}.\textsc{ds} preceding \textsc{obj}.1\textsc{pl}-kill-\textsc{purp} come-\textsc{interm}.\textsc{pst}.3\textsc{sg} \textsc{q}\\
	\glt \lq Have you come to attack us before the right time? (lit. Have you come early, when the time to hit us, still)'\footnote{\textit{Aa’nduk}, here glossed as \lq preceding' is glossed varyingly as \lq first, before, ahead' throughout \citeauthor{Barclay2008}'s (\citeyear{Barclay2008}) grammar. Judging from the examples, it has a general meaning of precedence.} \parencite[475]{Barclay2008}

	\ex\label{exAppendixWesternDaniNotYet4}
	\gll Nin-ogoba \textbf{awo}!\\
	\textsc{poss}.1\textsc{pl}-father still\\
	\glt \lq Our father, no!' \parencite[441]{Barclay2008}
\end{exe}
\largerpage
\subsubsection{Broadly adverbial temporal-aspectual functions}
\paragraph{Near past}\label{appendixWesternDaniNearPast}
\begin{itemize}
	\item \textcite[440]{Barclay2008}.
	\item This function is found in combination with a past tense verb. The two examples in \citeauthor{Barclay2008}'s (\citeyear{Barclay2008}) grammar both feature the intermediate past  (\ref{exAppendixWesternDaniImmediatePast1}, \ref{exAppendixWesternDaniImmediatePast2}) and a perfective or anterior viewpoint.
	
	Judging from \textcite[253–263]{Barclay2008}, the intermediate past is the least semantically specific of Western Dani's three degrees of remoteness in the past. As pointed out by \textcite[440]{Barclay2008} the immediate past and the near past use of \textit{awo} overlap to some degree. The remote past is probably not compatible with a near past construal due to its \lq\lq dissociated" \parencite{BotneKershner2008} nature. It is not entirely clear whether the intermediate past is inherently perfective; in any case, it stands in paradigmatic opposition to specifically continuous and habitual constructions. 
	
	\item The third example (\ref{exAppendixWesternDaniImmediatePast3}) features a minimally inflected medial clause in a clause chain, which receives its past interpretation from the final clause (cf. \cite[615–617]{Barclay2008}). The resulting reading is a near past-in-the-past. Judging from similar medial clauses throughout \citeauthor{Barclay2008}'s, it appears that the bounded viewpoint is an effect of the telic predicate (cf. the occurence of \textit{awo} with the atelic nominal predicate \textit{tawe} \lq unmarried', yielding \textsc{still}, one clause earlier).
	
	\item According to \textcite[440]{Barclay2008}, this function is also found with deverbal adjectives, which often have resultative readings (see \cite[101–105]{Barclay2008}). The only example I am aware of (\ref{exAppendixWesternDaniImmediatePast4}) can be interpreted as \lq\lq plain" \textsc{still} with a nominal predicate modified by \textit{ngget} \lq new', i.e \lq still of the newborn kind'; cf. \mbox{\textit{nggweendo}}  \mbox{\textit{ngget}}  \mbox{\textit{mendek}}  \mbox{\textit{ogobakkigir}-\textit{ik}} \lq cart new kind made-\textsc{adj}' \lq newly made cart' \parencite[369]{Barclay2008}.
\end{itemize}

\begin{exe}
	\ex\label{exAppendixWesternDaniImmediatePast1}
	\gll At \textbf{awo} wa-gaarak\\
	3\textsc{sg} still come-\textsc{interm.pst}:\textsc{subj}.3\textsc{sg}\\
	\glt \lq He has just come.' \parencite[440]{Barclay2008}

	\ex\label{exAppendixWesternDaniImmediatePast2}
	\gll Ndi \textbf{awo} aret k-inom nogo yo-gotak kenok, roti noo-rak meek o.\\
	and, still \textsc{intens} 2\textsc{pl}-with sleep do-\textsc{interm.pst}:\textsc{subj}.2\textsc{pl} if bread eat-\textsc{intentive} cannot \textsc{dm}\\
	\glt \lq And if you have just slept with them (women), you must not eat the bread.' 	\parencite[175]{Barclay2008}

	\ex\label{exAppendixWesternDaniImmediatePast3}
	\gll It in-eebe awo tawe paga iniklom no-mba-kwi, \{kwe ogonggelo \textbf{awo} imbirak lambun-ggo logonet\} ogonggelo kun-ik ee-ke menggi kwak, inik ee'-na-kwi … eer-eegwaarak nogo n-iniki aber-ak wona-ge agarik o.\\
	3\textsc{pl} \textsc{poss}.3\textsc{pl}-body still unmarried at enjoyment \textsc{obj}.1\textsc{sg}-think-\textsc{pl} woman husband still with.\textsc{du} join-\textsc{pl} \textsc{sim}.\textsc{ss} husband join-\textsc{adj} do-\textsc{sg} \textsc{hab}.3\textsc{sg} like heart do-\textsc{obj}.1\textsc{sg}-\textsc{subj}.\textsc{pl} {} do-\textsc{rem}.\textsc{pst} \textsc{anaph} \textsc{poss}.1\textsc{sg}-heart think-\textsc{adj} \textsc{cop}-\textsc{sg} \textsc{hab}.1\textsc{sg} \textsc{cont}\\
	\glt \lq When they were still unmarried (young) they loved me, while the woman had just been united to a husband, like she (a woman) usually unites with a husband, they loved me … Concerning all those things they did long ago, I am always remembering them.' (Peter Barclay, p.c.)

	\ex\label{exAppendixWesternDaniImmediatePast4}
	\gll Elege ndar-ak iigak tahun mbere	eer-eegwaarak mendek inom, ando weyonggwe tahun ambiret eer-eegwaarak \textbf{awo} \textbf{ngget} \textbf{ndar}-\textbf{ak} mendek inom, abok aret in-oor-eegwarek.\\
	child born-\textsc{adj} \textsc{sim}:3\textsc{pl}:\textsc{ds} years two \textsc{cop}-\textsc{dist.pst}:\textsc{subj}.3\textsc{pl} kind and some under year one \textsc{cop}-\textsc{dist.pst}:\textsc{subj}.3\textsc{pl} still new born-\textsc{adj} kind and all \textsc{intens} \textsc{obj}.3\textsc{pl}-kill-\textsc{dist.pst}:\textsc{subj}.3\textsc{pl}\\
	\glt \lq They killed all children two years old and those younger than one year who were just born.' \parencite[440]{Barclay2008}
\end{exe}
\il{Dani, Western|)}
