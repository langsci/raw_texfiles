\chapter{Australia}
\label{appendixAustralia}

\section{Gooniyandi (gni, goon1238)}\il{Gooniyandi|(}
\label{appendixGooniyandi}

\subsection{Introductory remarks}
A note on the structure of the Gooniyandi verb phrase: in broad strokes, verbal predicates in Gooniyandi typically consist of an uninflected element belonging to the open class of lexical (co)verbs, followed by an inflected element from the small and closed class of \lq\lq classifier" verbs; see \textcite[190–227]{McGregor1990} for more information. Note that slashes within Gooniyandi examples indicate the boundaries of intonational units, as marked by \textcite{McGregor1990}.

\subsection{=nyali}
\subsubsection{General information}
\begin{itemize}
	\item Wordhood: bound morpheme; an enclitic that can attach to various parts of speech.
	\item Syntax: typically attaches to its associate (but see below on host restrictions). The marker never attaches to the inflected part of the verb phrase.
\end{itemize}

\subsubsection{As a \lq{}still\rq{} expression}
\label{appendixGooniyandiStill}
\begin{itemize}
	\item \textcite[463–464]{McGregor1990}; additional discussion in \textcite{SchultzeBerndt2002}.
	\item Specialisation: \mbox{=\textit{nyali}} is primarily a multifaceted restrictive marker. There are few contextualised examples of \mbox{=\textit{nyali}} as \textsc{still} in \citeauthor{McGregor1990}'s (\citeyear{McGregor1990}) grammar. Cases like (\ref{exAppendixGooniyandi1}–\ref{exAppendixGooniyandi4}), however, give quite a solid indication that this item conforms to my definition. Further, albeit indirect, evidence, comes from the robust \textsc{still}–restrictive polysemy (see \Cref{sectionExclusive}) and the semantic parallels between the two functions. 
	\item Pragmaticity: the available data do not allow for any conclusions.
	\item Polarity sensitivity: inner negation yields \textsc{not yet}.
	\item Further note: the host of \mbox{=\textit{nyali}} as \textsc{still} is never a verbal constituent. Often \textit{wamba} \lq later, after that point' (essentially a marker of persistence; see \cite[511–512]{McGregor1990}) serves as the host. In (\ref{exAppendixGooniyandi2}) the marker is cliticised to a nominal predicate. In (\ref{exAppendixGooniyandi4}) \mbox{=\textit{nyali}} has scope over a secondary predicate.
\end{itemize}
\begin{exe}
	\ex\label{exAppendixGooniyandi1} 
	\gll Wamba-\textbf{nyali} marla-ya goorijga yaningi=nyali balayawi.\\
	later=still hand-\textsc{loc} he\_holds\_it now=still he\_will\_send\_it\\
	\glt \lq He’s still got (the letter) in his hand, (but) he’ll send it directly.' \parencite[509–510]{McGregor1990}
	
	\ex\label{exAppendixGooniyandi2} 
	\gll Yoowarni boolga / marlami boolga / yanoonggoo=\textbf{nyali} yoowooloo.\\
	One old\_man {} not old\_man {} young=still man\\
	\glt \lq There was an old man, not really an old man, he was still young.' \parencite[315]{McGregor1990}
	
		\ex 
	\gll Jiginya biddangi mangaddi wardgiri \textbf{giraddiri} \textbf{wamba}-\textbf{nyali}.\\
	child their not he:walks he:crawls later-still\\
	\glt \lq Their child doesn’t walk; he still crawls.\rq{ }\parencite[464]{McGregor1990}
	
	\ex\label{exAppendixGooniyandi4}
	\gll Boolba-ngaddi=\textbf{nyali} mooyoo bagiri.\\
	things-\textsc{com}=still sleep he\_lies\\
	\glt \lq He sleeps still dressed.'  \parencite[355]{McGregor1990}
\end{exe}

\subsubsection{Broadly adverbial temporal-aspectual functions}
\paragraph{Always}
\label{appendixGooniyandiAlways}
\begin{itemize}
	\item \textcite[464]{McGregor1990}; additional discussion is found in \textcite{SchultzeBerndt2002}.
	\item This function is primarily found with \mbox{=\textit{nyali}} operating on a secondary predicate, as in (\ref{exAppendixGooniyandiAlways1}, \ref{exAppendixGooniyandiAlways2}). Sometimes, albeit rarely, \mbox{=\textit{nyali}} as \lq always' takes \textit{ngambiddi} \lq again' as its host, together \lq\lq indicating daily iterative occurrences of processes such as shaving" \parencite[464]{McGregor1990}; see (\ref{exAppendixGooniyandiAlways3}).
\end{itemize}
\begin{exe}
	\ex\label{exAppendixGooniyandiAlways1}
	\gll Mangaddi gilbawayingi marlami=\textbf{nyali} barnbidi.\\
	\textsc{neg} he\_was\_finding\_it without=still he\_returned\\
	\glt \lq He would never find (anything); he’d always return empty handed.' \parencite[464]{McGregor1990}

	\ex\label{exAppendixGooniyandiAlways2}
	\gll Jimandi=\textbf{nyali} bariwindi mangaddi dagooddawani.\\
	good=still he\_climbed \textsc{neg} he\_fell\_in\\
	\glt \lq He climbed well all the way; he didn't fall.' \parencite[465]{McGregor1990}

	\ex\label{exAppendixGooniyandiAlways3}
\gll \textbf{Ngambiddi}=\textbf{nyali} thiddajga.\\
again=still he\_defecates\\
	\glt \lq He defecates daily.' \parencite[464]{McGregor1990}
\end{exe}

\paragraph{Iterative and restitutive}
\label{appendixGooniyandiIterative}
\begin{itemize}
	\sloppy
	\item \textcite[460–462]{McGregor1990}; additional discussion in \textcite{SchultzeBerndt2002}.
	\item Iterative (\ref{exAppendixGooniyandiIterative1}–\ref{exAppendixGooniyandiIterative3}) and restitutive uses (\ref{exAppendixGooniyandiRestitutive2}) are available. Examples (\ref{exAppendixGooniyandiRestitutive3}, \ref{exAppendixGooniyandiRestitutive4}) illustrate responsive uses, which are notionally closely related to the restitutive ones.
	\item In iterative function, \mbox{=\textit{nyali}} typically attaches to the lexical (i.e. uninflected) verb, thus preceding the classifying verb, as in (\ref{exAppendixGooniyandiIterative1}).
	\item Iterative \mbox{=\textit{nyali}} is also found in a collocation with \textit{ngambiddi} \lq again, next time' (\ref{exAppendixGooniyandiIterative3}, \ref{exAppendixGooniyandiIterative4}). While the data do not allow for any conclusions, (\ref{exAppendixGooniyandiIterative4}) suggests that we might be dealing with continued iteration (\lq yet again') and/or list continuation (\lq yet another'). \textcite[244]{SchultzeBerndt2002} reports personal communication from William McGregor that this collocation is idiomatic and might be undergoing lexicalisation.
	\item In its restitutive function \mbox{=\textit{ngali}} often cliticises to the constituent denoting the former state or location, as in (\ref{exAppendixGooniyandiRestitutive2}).

\end{itemize}
\begin{exe}
	\ex\label{exAppendixGooniyandiIterative1}
	 \gll Jamoondoo milanggiddinyayi mila=\textbf{nyali} yawinggiddinyayi.\\
	other\_day I\_saw\_you\_two see=still I\_will\_extend\_you\_two\\
	\glt \lq I saw you two the other day, and I’ll see you again later.\rq{ }\parencite[460]{McGregor1990}
	
	\ex\label{exAppendixGooniyandiIterative3}
	\gll Nganyi nyagginboowoo \textbf{ngambiddi}=\textbf{nyali}.\\
	I he\_will\_spear\_me again=still\\
	\glt \lq I might be speared again (not necessarily by the same person).' \parencite[462]{McGregor1990}
	
	\ex\label{exAppendixGooniyandiIterative4}
	Context: Hunters have pulled in one crocodile and killed it, then pulled in a second one and also killed it.\\
	\gll Yaanya / \textbf{ngambiddi}=\textbf{nyali} ridd-widi / gard-biddini\\
	other {} again=still pull-3\textsc{pl}:catch {} hit-3\textsc{pl}:hit\\
	\glt \lq Then they pulled yet another in and killed it.' \parencite[575]{McGregor1990}

	\ex\label{exAppendixGooniyandiRestitutive2}
	\gll Niyi barnbindi ngiwawoo=\textbf{nyali}.\\
	he he\_returned south=still\\
	\glt \lq He returned south again.' \parencite[460]{McGregor1990}

	\ex \label{exAppendixGooniyandiRestitutive3}
	\gll Yoowarni-ngga baami-ngadda briyandi baa=\textbf{nyali} limi-nhi.\\
	one-\textsc{erg} he\_called-to\_me in\_return call=still I\_did-to\_him\\
	\glt \lq One (man) called to me, and in turn I called back to him.\rq{ }\parencite[461]{McGregor1990}

	\ex \label{exAppendixGooniyandiRestitutive4}
	\gll Gardjayooni briyandi gard=\textbf{nyali} yoonirni.\\
	he\_might\_have\_hit\_him in\_turn hit=still he\_might\_have\_hit\_him\\
	\glt \lq Had he\textsubscript{1} hit him\textsubscript{2}, he\textsubscript{2} would have hit him\textsubscript{1} back in revenge.' \parencite[461]{McGregor1990}
\end{exe}

\subsubsection{Restrictive (non-temporal)}
\paragraph{(Non-scalar) exclusive}
\label{appendixGooniyandiRestrictive}
\begin{itemize}
	\item \textcite[462–467]{McGregor1990}; additional discussion in \textcite{SchultzeBerndt2002}.
	\item As is typically of restrictives, this is rather a cluster of functions. These include restricting the reference of nominals and pronouns (\lq just N'), as in  (\ref{exAppendixGooniyandiRestrictive1}, \ref{exAppendixGooniyandiRestrictive2}), signalling identity of place (\ref{exAppendixGooniyandiRestrictive3}), and intensification (\ref{exAppendixGooniyandiRestrictive5}).
\end{itemize}

\begin{exe}
	\ex\label{exAppendixGooniyandiRestrictive1}
	\gll Ngarloodoo-ngga birrangi=\textbf{nyali} riddwiddarniddi thaanoonggoo.\\
	three-\textsc{erg} them=still they\_pulled\_themselves up\\
	\glt \lq The three pulled themselves up by themselves.\rq{ }\parencite[172–173]{McGregor1990}
	
	\ex\label{exAppendixGooniyandiRestrictive2}
	\gll Nganyi=\textbf{nyali} milaana mangaddi ngooddoo milaa\\
	me=still he\_sees\_me \textsc{neg} that he\_sees\_him\\
	\glt \lq He’s still looking at me [or: only looking at me], not at him.' (\cite[464]{McGregor1990}; also see \cite[254]{SchultzeBerndt2002})
	
	\ex\label{exAppendixGooniyandiRestrictive3}
	\gll Mangaddi yoodgoowali ngirnda marla-ya=\textbf{nyali} goorijgila\\
	\textsc{neg} I\_was\_putting\_it\_down this hand-\textsc{loc}=still I\_hold\_it\\
	\glt \lq It haven’t put it down; it’s right here in my hand / still in my hand.' \parencite[465]{McGregor1990}

	\ex\label{exAppendixGooniyandiRestrictive5}
	\gll Moongaya=\textbf{nyali} gijloondi.\\
	morning=still I\_got\_up\\
	\glt \lq I got up early in the morning.' \parencite[466]{McGregor1990}
\end{exe}

\subsubsection{Additive and related functions}
\paragraph{Additive}\label{appendixGooniyandiAdditive}
\begin{itemize}
	\item \textcite[460–462]{McGregor1990}; additional discussion in \textcite{SchultzeBerndt2002}.
	\item Some of the examples grouped here, including (\ref{exAppendixGooniyandiAlso4}), are discussed by \textcite{McGregor1990} under the heading of iteration, as these involve the addition of an identical act, but with switched participants.
	\item =\textit{nyali} is also repeatedly attested in conjunction with \textit{yaanya} \lq other\rq{ }(\ref{appendixGooniyandiAdditive4}).
\end{itemize}

\begin{exe}
	\ex \gll Moongayayoo bala=\textbf{nyali} ya-wili\\
	morning call=still \textsc{sbjv}-I\_will\_catch\_it\\
	\glt \lq Tomorrow I’ll send a letter also.' \parencite[461]{McGregor1990}

	\ex Context: Dogs sleep in holes.\\
	\gll Thiddoo gooddgoo-ya=\textbf{nyali} bagiri.\\
	kangaroo hole-\textsc{loc}=still he\_lives\\
	\glt \lq Kangaroos live in holes too.'  \parencite[462]{McGregor1990}

	\ex \label{exAppendixGooniyandiAlso4}
	\gll Birdi dijga yaanya birdi dij=\textbf{nyali}-a.\\
	leg he\_breaks\_it other leg break=still-he\_extends\_to\_it\\
	\glt \lq He breaks one leg (of a frog), then the other [as well].\rq{ }\parencite[461]{McGregor1990}

	\ex \label{appendixGooniyandiAdditive4}
	\gll Yaanya gamba-ngarna=\textbf{nyali} giwili.\\
	other water-dweller=still water\_goanna\\
	\glt \lq Another water creature is the water goanna.' \parencite[232]{McGregor1990}
\end{exe}
\il{Gooniyandi|)}

\section{Gurindji (gue, guri1247)}\il{Gurindji|(}
\label{appendixGurindji}

\subsection{Introductory remarks}
I am indebted to Felicity Meakins for making the Gurindji dictionary \parencite{MeakinsEtAl2013} available to me. Similar to the other languages of northern Australia (e.g. \cite{Bowern2014}), verbal predicates in Gurindji often consist of an uninflected lexical (co)verb, combined with an inflected element from the small, closed class of inflecting verbs; see \textcite[442–481]{MeakinsMcConvell2021} for extensive discussion.

\subsection{=rni}
\subsubsection{General information}
\begin{itemize}
	\sloppy
	\item Form: following consonants, an epenthetic syllable \textit{pa}/\textit{wa} is added, yielding \mbox{=\textit{parni}}/\mbox{=\textit{warni}}. In addition, there is reduplicated \mbox{=\textit{rnirni}} and a complex form \mbox{=\textit{rningan}}.
	\item Wordhood: bound morpheme, an enclitic that attaches to various parts of speech, but not to the inflected part of the verbal predicate (but see \appref{appendixGurindjiIterative}–\ref{appendixGurindjiAdditive}  on \mbox{=\textit{rningan}}).
 \end{itemize}

\subsubsection{As a \lq{}still\rq{} expression}
\begin{itemize}
	\item \textcite{McConvell1983} and \textcite[594–595]{MeakinsMcConvell2021}; additional discussion in \citeauthor{vanBaar1991} (\citeyear{vanBaar1991}, \citeyear[112–113]{vanBaar1997}) and \textcite{SchultzeBerndt2002}.
	\item Specialisation: \mbox{=\textit{rni}} is primarily a multi-faceted restrictive marker. The discussion by \textcite{McConvell1983} gives a fairly good indication that it also encodes the function of \textsc{still}. Further, albeit indirect, evidence, comes from the robust still–restrictive polysemy (\Cref{sectionExclusive}) and the semantic parallels between the two functions.
	\item Pragmaticity: the data allow no conclusion.
	\item Polarity sensitivity: inner negation yields \textsc{not yet}.
	\item Further note: in (\ref{exAppendixGurindji4}) \mbox{=\textit{rni}} has scope over a secondary predicate. Ex. (\ref{exAppendixGurindji5}, \ref{exAppendixGurindji6}) illustrate the use in an imperative and jussive, respectively. \textcite[17]{McConvell1983} points out that (\ref{exAppendixGurindji6}) could be read as also invoking the restrictive use of \mbox{=\textit{rni}} (\appref{appendixGurindjiRestrictive}).
	\end{itemize}
\begin{exe}
	\ex
	\gll Lawara=rni marlumarlu=\textbf{rni}\\
	nothing=still crippled=still\\
	\glt \lq No (improvement) yet, (he is) still crippled.' \parencite[20]{McConvell1983}
	
	\ex \gll Jakiliny tampang=\textbf{parni} karri-nyana.\\
	moon dead=still be-\textsc{prs}\\
	\glt \lq The moon is still dead (i.e. it is new moon).' \parencite[20]{McConvell1983}
	
	\ex\label{exAppendixGurindji4}
	\gll Ngu=rna=rla kurrurij-ja jalngak waninya rarraj-ja=\textbf{rni}.\\
	\textsc{aux}=\textsc{subj}.1.\textsc{excl}=\textsc{3}.\textsc{obl} truck-\textsc{loc} mount fall.\textsc{pst} run-\textsc{loc}=still\\
	\glt \lq I got on the truck while it was still moving.\rq{ }\parencite[595]{MeakinsMcConvell2021}
	
	\pagebreak
	\ex\label{exAppendixGurindji5}	
	 Context: The addressee is climbing up a tree.\\
	\gll Partaj ya-nta kankula=\textbf{rni}!\\
	climb go-\textsc{imp} up=still\\
	\glt \lq {\lq\lq}Climb further up!\rq\rq{ }[he said].' \parencite[691]{MeakinsMcConvell2021}
	
	\ex\label{exAppendixGurindji6}	
	Context: After a car breakdown.\\
	\gll Ya-nan-ku=li jamana=\textbf{rni} warrij.\\
	go-\textsc{prog}-\textsc{fut}=\textsc{subj}.1\textsc{du}.\textsc{incl} foot=still away\\
	\glt \lq Let’s keep going (just) on foot.’ \parencite[17]{McConvell1983}
	
\end{exe}

\subsubsection{Uses on the fringes of \lq{}still\rq{}}
\subsubsection{Broadly adverbial temporal-aspectual uses}


\paragraph{Iterative and restitutive =\textit{rningan}}
\label{appendixGurindjiIterative}
\begin{itemize}
	\item \textcite{McConvell1983}, \textcite[597–598]{MeakinsMcConvell2021} and \textcite[s.v. \textit{rningan}]{MeakinsEtAl2013}; additional discussion ii \textcite{SchultzeBerndt2002}.
	\item Form: in this function, as well as the related functions below, we find a complex enclitic \mbox{=\textit{rningan}}; it is unclear what the source of the /\textit{ngan}/ part is. As with \mbox{=\textit{rni}}, following consonants.
	\item Iterative (\ref{exAppendixGurindjiIterative1}–\ref{exAppendixGurindjiIterative3}) and restitutive  (\ref{exAppendixGurindjiRestitutive1}, \ref{exAppendixGurindjiRestitutive2}) uses are attested. 
	\item Iterative \mbox{=\textit{rningan}} can take the inflected verbs as its host, as in (\ref{exAppendixGurindjiIterative1}), whereas restitutive \mbox{=\textit{rningan}} always attaches to the uninflected lexical verb.
\end{itemize}
\begin{exe}
	\ex \label{exAppendixGurindjiIterative1}
	\gll Karu-ngku wumara=ma jakarr ma-ni=\textbf{rningan}.\\
	child-\textsc{erg} money=\textsc{top} cover get-\textsc{pst}=rningan\\
	\glt \lq A child covered up the money again.' (the act of covering up is repeated) \parencite[6]{McConvell1983}

	\ex
	\gll Ngu=rla rarrarraj ya-ni nyila=ma ngu=rla wartpaj=\textbf{parningan} ma-ni lanti-kari-la, ngu=rla pu-nya kaa-rni-yin-tu.\\
	\textsc{aux}=3.\textsc{obl} run.\textsc{redupl} go-\textsc{pst} that=\textsc{top} \textsc{aux}=3.\textsc{obl} throw\_spear=rningan do-\textsc{pst} hip-other-\textsc{loc} \textsc{aux}=3.\textsc{obl} pierce-\textsc{pst} east-up-\textsc{abl}-\textsc{erg}\\
	\glt \lq He went running over to him, threw again and got him bang in the eastern hip from the towards the river.' \parencite[299]{MeakinsMcConvell2021}
	
	\ex \label{exAppendixGurindjiIterative3}
	\gll Limpal=\textbf{warningan} ngu=rnalu ma-rni.\\
	mens\_business=\textsc{still} \textsc{aux}=\textsc{subj}.1\textsc{pl}.\textsc{incl} talk-\textsc{pst}\\
	\glt \lq We talked about important men's business again.' \parencite[597]{MeakinsMcConvell2021}

	\ex \label{exAppendixGurindjiRestitutive1}
	\gll Karu-ngku wumara=ma jakarr=\textbf{warningan} ma-ni.\\
	child-\textsc{erg} money=\textsc{top} cover=rningan get-\textsc{pst}\\
	\glt \lq A child covered up the money again.' (the state of being covered is restored) \parencite[7]{McConvell1983}
	
	\ex \label{exAppendixGurindjiRestitutive2}
	\gll Ngu=rna jutuk=parningan yuwa-ni.\\
	\textsc{aux}=\textsc{subj}.1\textsc{sg} straighten=\textbf{rningan} put-\textsc{pst}\\
	\glt \lq I got on the right track again.' \parencite[597]{MeakinsMcConvell2021}
\end{exe}

\paragraph{Always, all the time}
\label{appendixGurindjiAlways}
\begin{itemize}
	\sloppy
	\item \textcite{McConvell1983}, \textcite[594–595]{MeakinsMcConvell2021} and \textcite[s.v. \textit{rni}]{MeakinsEtAl2013}; additional discussion in \citeauthor{vanBaar1991} (\citeyear{vanBaar1991}, \citeyear[112–113]{vanBaar1997}) and \textcite{SchultzeBerndt2002}.
\end{itemize}
\begin{exe}
	
	\ex 
	\gll Lawarrkap=\textbf{parni} wanyja-ni ngu=yina.\\
	dodge=still leave-\textsc{pst} \textsc{aux}=3\textsc{pl}\\
	\glt \lq He continually dodged them.' \parencite[594]{MeakinsMcConvell2021}
	
	\ex 
	\gll Kula=n pura nya-ngku; ngu=n=ngantipa pawu=\textbf{rni} pa-nana.\\
	\textsc{neg}=\textsc{subj}.2\textsc{sg} hear see-\textsc{fut} \textsc{aux}=\textsc{subj}.\textsc{2sg}=\textsc{obj}.1\textsc{pl}.\textsc{excl} ignore=still hit-\textsc{prs}\\
	\glt \lq You can’t listen; you always just ignore us.' \parencite[20]{McConvell1983}
	
	\ex 
	\gll Yamak=\textbf{parni} ka-ngka.\\
	slow=still take-\textsc{imp}\\
	\glt \lq Drive slowly (all the time).' \parencite[21]{McConvell1983}
\end{exe}


\subsubsection{Additive and related functions}
\paragraph{Additive =\textit{rningan}}
\label{appendixGurindjiAdditive}
\begin{itemize}
	\item \textcite{McConvell1983}, \textcite[597–598]{MeakinsMcConvell2021} and \textcite[s.v. \textit{rningan}]{MeakinsEtAl2013}; additional discussion is found in \textcite{SchultzeBerndt2002}.
	\item Form: see \ref{appendixGurindjiIterative} above.
	\item Syntax: in this use, \mbox{=\textit{rningan}} can attach to the inflected verb, but also to constituents below the predicate, depending on focus (and disambiguation).
\end{itemize}
\begin{exe}
	\ex
	\gll Parntawurru=wayin ngu=rna pa-ni=\textbf{rningan}.\\
	back=including \textsc{aux}=\textsc{subj}.1\textsc{sg} hit-\textsc{pst}=rningan\\
	\glt \lq I hit it on the back, too.' \parencite[8]{McConvell1983}

	\ex Context: Someone else has fallen.\\
	\gll Karu=ma ngaja wani-nyana=\textbf{rningan}.\\
	child=\textsc{top} \textsc{admonitive} fall-\textsc{prs}=rningan\\
	\glt \lq The child might fall too.' \parencite[8]{McConvell1983}
		
	\ex
	\gll Ngamayi-lu=ma=ngantipa=wula ka-nya kuya-ny-ja=\textbf{rningan}, juluj-juluj.\\
	mother-\textsc{erg}=\textsc{top}=\textsc{non}.\textsc{subj}:1\textsc{pl}.\textsc{excl}=\textsc{subj}.3\textsc{du} take-\textsc{pst} thus-\textsc{nmlz}-\textsc{loc}=rningan carry\_under\_arm.\textsc{redupl}\\
	\glt \lq Our mothers used to carry us in this sort of coolamon too.\rq{ }\parencite[597]{MeakinsMcConvell2021}

	\ex
	\gll Karu-ngku=\textbf{rningan} wumara=ma jakarr ma-ni.\\
	child-\textsc{erg}=rningan money=\textsc{top} cover get-\textsc{pst}\\
	\glt \lq A child, too, covered up the money.' \parencite[7]{McConvell1983}
\end{exe}

\subsubsection{Restrictive (non-temporal)}
\paragraph{(Non-scalar) exclusive}
\label{appendixGurindjiRestrictive}
\begin{itemize}
	\sloppy
	\item \textcite{McConvell1983}, \textcite[587–596]{MeakinsMcConvell2021} and \textcite[s.v. \textit{rni}]{MeakinsEtAl2013}; additional discussion in \citeauthor{vanBaar1991} (\citeyear{vanBaar1991}, \citeyear[112–113]{vanBaar1997}) and \textcite{SchultzeBerndt2002}.
	\item As is typical of restrictive markers, this is a cluster of functions. These include the defining \lq only' use with nominals  (\ref{exAppendixGurindjiRestrictive1}) as well as with the main predicate as the associate of the marker (\ref{exAppendixGurindjiRestrictive2}, \ref{exAppendixGurindjiRestrictive3}). It is also used, among other things, to indicate that an act is performed all by oneself (\ref{exAppendixGurindjiRestrictive4}), for emphasis on identity of location (\ref{exAppendixGurindjiRestrictive5}), and for intensification (\ref{exAppendixGurindjiRestrictive6}).
	\item Restrictive \mbox{=\textit{rni}} can be reduplicated, as in (\ref{exAppendixGurindjiRestrictive3}). The reduplicated form exclusively has the \lq only' function, thus allowing disambiguation from the competing phasal polarity use.
\end{itemize}
\begin{exe}
	\ex\label{exAppendixGurindjiRestrictive1}
	\gll Ngayi-ny=\textbf{parni} ngu=yi nyila=ma kujingka=ma.\\
	1\textsc{sg}-\textsc{dat}=still \textsc{aux}=\textsc{obj}.1\textsc{sg} \textsc{dem}=\textsc{top} song\_cycle=\textsc{top}\\
	\glt \lq That song cycle belongs to me only.' \parencite[17]{McConvell1983}
	
	\ex\label{exAppendixGurindjiRestrictive2}
	\gll Juurl\sim jurrl=\textbf{warni} ma-nyja=wula=nyunu.\\
	apologise\sim \textsc{redupl}=still talk-\textsc{imp}=\textsc{subj}.3\textsc{du}=\textsc{refl}/\textsc{recp}\\
	\glt \lq You two just apologise to each other!\rq{ }\parencite[588]{MeakinsMcConvell2021}
	
	\ex\label{exAppendixGurindjiRestrictive3}
	\gll Ngu=rna karrap=\textbf{parnirni} nya-nya kurti=rni ngu=rna pirrkap=ma ma-nku.\\
	\textsc{aux}=\textsc{subj}.1\textsc{sg} look=still.\textsc{redupl} see-\textsc{pst} later=still \textsc{aux}=\textsc{subj}.1\textsc{sg} make=\textsc{top} get-\textsc{fut}\\
	\glt \lq I have only looked at it (the car); I’ll do the repairs later.\rq{ }\parencite[23]{McConvell1983}
	
	\ex\label{exAppendixGurindjiRestrictive4}
	\gll Jartaj ma-ni ngu=rla=nyanta nyanuny-ja=\textbf{rni} kuya.\\
	hit\_with\_boomerang do-\textsc{pst} \textsc{aux}=3\textsc{obl}=3\textsc{obl} 3\textsc{sg}.\textsc{dat}-\textsc{loc}=still thus\\
	\glt \lq He aimed at him all by himself.\rq{ }\parencite[592]{MeakinsMcConvell2021}
	
	\ex\label{exAppendixGurindjiRestrictive5}
	\gll Murrkun-kari=ma nyawa=ma karrinya, karnti-ka=\textbf{rni}.\\
	three-other=\textsc{top} this=\textsc{top} \textsc{cop}.\textsc{pst} tree-\textsc{loc}=still\\
	\glt \lq Another three were right there in the tree.\rq{ }\parencite[593]{MeakinsMcConvell2021} 
	
	\ex\label{exAppendixGurindjiRestrictive6}
	\gll kaput=\textbf{parni}\\
	morning=still\\
	\glt \lq{}early in the morning'  \parencite[20]{McConvell1983} 
\end{exe}
\il{Gurindji|)}


\section{Jaminjung-Ngaliwurru (djd, djam1255)}
\label{appendixJaminjung}\il{Jaminjung|(}\il{Ngaliwurru|(}

\subsection{Introductory remarks}
I am indebted to Eva-Schultze Berndt for discussing Jaminjung-Ngaliwurru data with me, and for providing additional examples. A note on the structure of the verb phrase: simplifying slightly, verbal predicates in  Jaminjung-Ngaliwurru typically consist of an uninflected element belonging to a large open class of uninflecting \lq\lq coverbs" plus an inflected member of the small and closed class of verbs proper. See \textcite[ch. 3]{SchultzeBerndt2000} for more information.

\subsection{=(C)ung}

\subsubsection{General information}
\begin{itemize}
	\item Wordhood: bound morpheme, an enclitic.
	\item Syntax: in the case of complex predicates, the uninflected (co)verb, but not the inflected verb can serve as the host.
 \end{itemize}

\subsubsection{As a \lq{}still\rq{} expression}
\begin{itemize}
	\item \citeauthor{SchultzeBerndt2000} (\citeyear[140]{SchultzeBerndt2000}, \citeyear{SchultzeBerndt2002}).
	\item Specialisation: \mbox{=(\textit{C})\textit{ung}} is primarily a multifaceted restrictive marker. That phasal polarity \textsc{still} is amongst its denotata becomes evident in examples like (\ref{exAppendixJaminjung1}–\ref{exAppendixJaminjung4}). For instance, in (\ref{exAppendixJaminjung1}) the fact that the rock remains covered with water is in contrast with the assumption that, at this particular point in time, this state should no longer hold. Further, albeit indirect, evidence comes from the robustly attested restrictive–\textsc{still} polysemy and the semantic overlap between the two functions (see \Cref{sectionExclusive}).
	\item Pragmaticity: examples like (\ref{exAppendixJaminjung1}, \ref{exAppendixJaminjung2}) suggest that \mbox{=(\textit{C})\textit{ung}} is compatible not only with the neutral scenario, but also with the unexpectedly late one.
	\item Polarity sensitivity: \mbox{=(\textit{C})\textit{ung}} as \textsc{still} is not attested in combination with negation (Eva Schultze-Berndt, p.c.).
	\item Syntax: the host of phasal polarity \mbox{=(\textit{C})\textit{ung}} is relatively free, but subject to the general restriction against inflected verbs.
\end{itemize}
\begin{exe}
	\ex\label{exAppendixJaminjung1}
	Context: Watching a video of a dreaming site which was partially submitted in water. This would be expected in the wet season, but not necessarily at the time of the recording.\\
	\gll Bad\sim bad=\textbf{ung} ga-yu=di wagurra?\\
	\textsc{redupl}\sim cover=still 3\textsc{sg}-\textsc{cop}.\textsc{prs}=\textsc{foc} rock\\
	\glt \lq The rock is still covered?' \parencite[234]{SchultzeBerndt2002}

	\ex\label{exAppendixJaminjung2}
	Context: Speaking about a particular person who is expected to arrive.\\
	\gll Gurrany bul ga-ruma-ny, yina ga-yu=\textbf{wung}=yunyag.\\
	\textsc{neg} appear 3\textsc{sg}-come-\textsc{pst} \textsc{dist} 3\textsc{sg}-\textsc{cop}.\textsc{prs}=still=\textsc{obl}.1\textsc{pl}.\textsc{incl}\\
	\glt \lq She hasn't appeared, she is still over there on us three.'
	\\(Eva Schultze-Berndt, p.c.)\footnote{Technically, \mbox{=\textit{yunyag}} is the \lq\lq unit augmented" form (speaker and addressee, plus one more person), hence \lq us three'.}
	
	\ex\label{exAppendixJaminjung3}
	Context: Retelling of a video showing a family member who is fishing. She has caught a fish and is trying to pull out the hook.\\
	\gll Gara=biya, ngayin-ni=\textbf{wung} wilb gan-ardgiya-ny.\\
	no=now meat/animal-\textsc{erg}=still buck 3\textsc{sg}>3\textsc{sg}-throw-\textsc{pst}\\
	\glt \lq But no (she didn't succeed), the animal still jumped.'
	\\(Eva Schultze-Berndt, p.c.)
	\pagebreak
	\ex\label{exAppendixJaminjung4}
	Context: From a bush trip narrative.
\exi{}\gll {Tongue hanging} gugu-wu, gugu yagbali-g=\textbf{gung} ga-gba=yirrag.\\
{tongue hanging(<Kriol)} water-\textsc{dat} water place-\textsc{loc}=still/only 3\textsc{sg}-\textsc{cop}=1\textsc{pl}.\textsc{excl}.\textsc{dat}\\
\glt \lq (His) tongue hanging out for water, (since) our water was still (back) in the camp / right in the camp.' (Eva Schultze-Berndt, p.c.)
\end{exe}

\subsubsection{Broadly adverbial temporal-aspectual functions}
\paragraph{Always, all the time}
\label{exAppendixJaminjungAlways}
\begin{itemize}
	\item \textcite{SchultzeBerndt2002}.
	\item This function obtains with secondary predicates as the host constituent.
\end{itemize}
\begin{exe}
	\ex
	\gll Ngarrgina=malang gujarding digirrij ga-jgany bidimab-nyunga=\textbf{wung} ngayug=gung nga-ngangarna-nyi mangarra nganja\sim nganjany.\\
	\textsc{poss}.1\textsc{sg}=\textsc{given} mother die 3\textsc{sg}-go.\textsc{pst} feed:\textsc{tr}-from=still 1\textsc{sg}=still 1\textsc{sg}>3\textsc{sg}-give.\textsc{redupl}-\textsc{ipfv} plant\_food \textsc{redupl}\sim what\\
	\glt \lq My mother passed away (always) having been cared for (lit. \lq\lq fed"), me I used to give her food (and) things' \parencite[235]{SchultzeBerndt2002}
\end{exe}

\subsubsection{Restrictive (non-temporal)}
\paragraph{(Non-scalar) exclusive}
\begin{itemize}
	\item \textcite[76–78]{Cleverly1968} and \citeauthor{SchultzeBerndt2000} (\citeyear[104]{SchultzeBerndt2000}, \citeyear{SchultzeBerndt2002}).
	\item As is typical of restrictive markers, this is rather a cluster of related functions. These include emphasis on the identity of the referent of pronouns (\ref{exAppendixJaminjungRestrictive1}) and of manner expressions (\ref{exAppendixJaminjungRestrictive2}). It is also frequent with expressions of location and time, as in (\ref{exAppendixJaminjungRestrictive3}–\ref{exAppendixJaminjungRestrictive4}).
\end{itemize}
\largerpage[2]
\begin{exe}
	\ex\label{exAppendixJaminjungRestrictive1}
	\gll Ngayug=\textbf{gung} nga-ruma-ny, wurlug.\\
	1\textsc{sg}=still 1\textsc{sg}-come-\textsc{pst} alone\\
	\glt \lq I came just (by) myself, alone.' \parencite[236]{SchultzeBerndt2002}
	
	\ex\label{exAppendixJaminjungRestrictive2}
	Context: About a person affected by alcoholism.\\
	\gll Maja gan-unggu-m darlarlab=\textbf{bung} ga-ngga warlnginy.\\
	do\_like\_that 3\textsc{sg}>3\textsc{sg}-say/do-\textsc{prs} shiver=still 3\textsc{sg}-go.\textsc{prs} walk\\
	\glt \lq He goes like that, shakingly [i.e. in no other way] he walks.' \parencite[235]{SchultzeBerndt2002}
	
	\ex\label{exAppendixJaminjungRestrictive3}
	Context: About water during wet season floodings.\\
	\gll Gurrany=biyang walg ga-jga-ny, ga-gba nginyju-ni=\textbf{wung} danggad-gi=\textbf{wung}.\\
	\textsc{neg}=now open 3\textsc{sg}-go-\textsc{pst} 3\textsc{sg}-\textsc{cop}.\textsc{pst} \textsc{dem}-\textsc{loc}=still junction-\textsc{loc}=still\\
	\glt \lq It didn’t go out, it stayed right here, right at the junction.' \parencite[238]{SchultzeBerndt2002}

	\ex\label{exAppendixJaminjungRestrictive4}
	\gll Na-ruma-ny jaru, larrman-gi=\textbf{wung} na-ram.\\
	2\textsc{sg}-come-\textsc{pst} in\_same\_way dry-\textsc{loc}=still 2\textsc{sg}-come.\textsc{prs}\\
	\glt \lq You came in the same way (i.e. at the same time), right in the dry (season) you come.' \parencite[238]{SchultzeBerndt2002}
\end{exe}
\il{Jaminjung|)}\il{Ngaliwurru|)}

\section{Kayardild (gyd, kaya1319)}\il{Kayardild|(}
\label{appendixKayardild}

\subsection{Introductory remarks}
Most of the Kayardild examples below feature verbs in the \lq\lq actual" form. This is, in essence, the default inflection for past, present, and immediate future situations and which contrasts with forms that carry more specific temporal-aspectual and modal meanings; see \textcite[256–257]{Evans1995} for discussion.

\subsection{=(i)da}
\subsubsection{General information}

\begin{itemize}
	\item Wordhood: bound morpheme, an enclitic that takes words of various syntactic classes as its host.
	\item Etymology: < \textit{niida} \lq same'.
\end{itemize}

\subsubsection{As a \lq{}still\rq{} expression}
\begin{itemize}
	\item \textcite[389–392]{Evans1995} and \textcite[181–183]{Round2009}; additional discussion in \textcite{SchultzeBerndt2002}.
	\item Specialisation: originally a multifaceted restrictive marker, this item also serves as a \textsc{still} expression when attached to the main predicate or a secondary predicate. Its function as a \textsc{still} expression becomes evident in examples like (\ref{exAppendixKayardild1}–\ref{exAppendixKayardild3}). For instance, in (\ref{exAppendixKayardild2}), the protagonist continues to be alive, against his assailants' assumption to the contrary.
	\item Pragmaticity: the available data do not allow for any conclusions. \textcite[392]{Evans1995} notes that \lq\lq persistence may be emphasized" by the verb \textit{wirdi}-\textit{ja} \lq stay, remain' as the host of \mbox{=(\textit{i})\textit{da}} (also see \cite[32]{Evans1995}); this could be an indication that the unexpectedly late scenario receives additional marking.
	\item Polarity sensitivity: inner negation yields \textsc{not yet}. However, few examples with negation (including semantically negative items like \lq nothing\rq{}) are attested in the data consulted (e.g \cite[183 ex. 3.66, 720 ex. A.27f]{Round2009}).
	\item Further notes: in (\ref{exAppendixKayardild3}) \mbox{=(\textit{i})\textit{da}} marks a secondary predicate. Examples (\ref{exAppendixKayardild4}, \ref{exAppendixKayardild5}) illustrate a jussive and prospective context, respectively.
\end{itemize}
\begin{exe}
	\ex\label{exAppendixKayardild1}
	\gll Ngada warirr, marndi-i-n=\textbf{id}.\\
	1\textsc{sg}.\textsc{nom} nothing.\textsc{nom} cadge-\textsc{mod}-\textsc{nmlz}=still\\
	\glt \lq I’ve got nothing, I’m still cadged out.' \parencite[391]{Evans1995}
	
	\ex\label{exAppendixKayardild2} 
	Context:  After fighting his way across Bentinck Island through a shower of spears, Kajurku disappears into the sea. Some time later, his assailants sight a campfire on Sweers Island.\\
	\gll (Kajurku) birjin=\textbf{ida} kinaa-j.\\
	\phantom{(}Kajurku alive=still declare-\textsc{actual}\\
	\glt \lq Kajurku was showing he was still alive.' \parencite[327]{Evans1995}

	\ex\label{exAppendixKayardild3}
	\gll Kala-a-n-marri=\textbf{da} mardala-a-j.\\
	cut-\textsc{mod}-\textsc{nmlz}-without=still paint-\textsc{mod}-\textsc{actual}\\
	\glt \lq (The initiates) were painted while still uncircumcised (before being circumcised).' \parencite[391]{Evans1995}
	
	\ex\label{exAppendixKayardild4}
	\gll Wirdi-jinja=\textbf{da} dathin-a dukurduku bintha.\\
	stay-\textsc{hort}=still that-\textsc{nom} moist.\textsc{nom} foreskin.\textsc{nom}\\
	\glt \lq Let those moist foreskins wait a while yet (before burying them).\rq{ }\parencite[264]{Evans1995}
	
	\ex\label{exAppendixKayardild5}
	\gll Ngada ngaka-thuu=\textbf{d}.\\
	1\textsc{sg}.\textsc{nom} wait-\textsc{pot}=still\\
	\glt \lq I'll have to wait a long time yet.\rq{ }\parencite[392]{Evans1995}	
\end{exe}


\subsubsection{Uses on the fringes of \lq{}still\rq{}}

\paragraph{Scalar contexts}\label{appendixKayardildScalar}
\begin{itemize}
	\ex One example in the data features a scalar decrease context.
\end{itemize}

\begin{exe}
	\ex
	\gll Barri kuliya\sim{}kuliya-n mutha-yarrath=\textbf{id}.\\
	just fill\sim{}\textsc{redupl}-\textsc{proh} many-other=still\\
	\glt \lq Just don't give me too much food, there's plenty yet to feed.\rq{ }
	\parencite[384]{Evans1995}
\end{exe}


\subsubsection{Restrictive (non-temporal)}
\paragraph{(Non-scalar) exclusive}
\label{appendixKayardildRestrictive}
\begin{itemize}
	\item \textcite[389–392]{Evans1995} and \textcite[173–174, 181–183, 202–203]{Round2009}; additional discussion in \textcite{SchultzeBerndt2002}.
	\item As is typical of non-scalar exclusive markers, this is rather a cluster of functions that include signalling sameness/identity (\ref{exAppendixKayardildRestrictive1}), restriction and/or emphasis with expressions of time and space (\ref{exAppendixKayardildRestrictive2}, \ref{exAppendixKayardildRestrictive3}).
\end{itemize}
\begin{exe}
	\ex \label{exAppendixKayardildRestrictive1}
	\gll Dathin-a kiyarrng-ka dangka-a bi-rr=\textbf{ida} dankga-a barruntha-ri nga-ku-lu-wan-jir kamburi-jir? – Bi-rr=\textbf{ida} dangka-a.\\
	that-\textsc{nom} two-\textsc{nom} person-\textsc{nom} 3-\textsc{du}=still person-\textsc{nom} yesterday-\textsc{all} 1-\textsc{incl}-\textsc{pl}-\textsc{poss}-\textsc{all} speak-\textsc{all} {} 3-\textsc{du}=still person-\textsc{nom}\\
	\glt \lq Are they the two same men who came to talk to us yesterday? -- The same two men.' \parencite[390]{Evans1995}
	
	\ex \label{exAppendixKayardildRestrictive2}
	\gll Ngarii-ja=\textbf{da} narra-nguni-ya kala-th.\\
	before-\textsc{actual}=still shell\_knife-\textsc{ins}-\textsc{loc} cut-\textsc{actual}\\ 
	\glt \lq Way back in the old days we used to cut things with shell knives.' \parencite[392]{Evans1995}

	\ex \label{exAppendixKayardildRestrictive3}
	\gll Walmathi=\textbf{da} walmathi bath-in-d, burldi\sim burldi-ja warra-j, burldi-ja birrk-i.\\
high=still high west-from-\textsc{nom} roll\sim\textsc{redupl}-\textsc{actual} go-\textsc{actual} roll-\textsc{actual} string-\textsc{loc}\\
\glt \lq High up, moving from the west, she (Kaarku, the Seagull Being) came along, rolling string as she went.' \parencite[309]{Evans1995}
\end{exe}
\il{Kayardild|)}

\section{Martuthunira (vma, mart1255)}\il{Martuthunira|(}
\label{appendixMartuthunira}

\subsection{Introductory remarks}
Martuthunira has two candidates for \textsc{still} expressions: \textit{waruu}(\textit{l}) and \textit{parilha}; the latter seems to be related to a verb \lq keep doing, keep trying'. It is only for \textit{waruu}(\textit{l}) that there is clear evidence of additional functions. Note that several of the examples feature the marker -\textit{rru} \lq now' (sometimes preceded by an epenthetic element -\textit{wa}). This is a very frequent item in Martuthunira texts and 

\begin{quote}
is used to foreground the item to which it is attached … [a]t the same time, the clitic serves to define a kind of narrative present, a statement that what has already been said can be now taken as established, and that the narrative will build from this point. \parencite[184]{Dench1994}.
\end{quote}

\subsection{waruu(l)}
\subsubsection{General information}
\begin{itemize}
	\item Form: free variants differ in the presence or absence of final /l/; see \textcite[130–131]{Dench1994} for discussion.
	\item Wordhood: free morpheme.
\end{itemize}

\subsubsection{As a \lq{}still\rq{} expression}
\begin{itemize}
	\item \textcite[130–131]{Dench1994}; additional discussion is found in \textcite{SchultzeBerndt2002}.
	\item Specialisation: the phasal polarity function becomes evident in examples like (\ref{exappendixMartuthinura1}–\ref{exappendixMartuthinura3}). For instance, in (\ref{exappendixMartuthinura1}) the cloth continues to have powder on it, while at the same time it is suggested that at some later point, the powder would no longer be there.
	\item Pragmaticity: compatible with both scenarios (tentative conclusion). Example (\ref{exappendixMartuthinura2}) looks like a good candidate for the unexpectedly late scenario, with the uninterrupted flight of the boomerang continuing to the astonishment of its chasers.
	\item Polarity sensitivity: combination with negation yields \textsc{not yet}.
	\item Further note: \textit{waruu}(\textit{l}) also forms part of a derived verb \textit{waruulwa} \lq continue to be unable to do'. Ex. (\ref{exappendixMartuthinura4}) illustrates the use in an elliptical imperative.
\end{itemize}\largerpage
\begin{exe}
	\ex\label{exappendixMartuthinura1}
	\gll Ngunhaa kanyja-rnu nhawani-ma-lwayara, thurlwa-nnguli-wayara, parrapari-marnu. Ngunhu wanti-nguru powder-marta \textbf{waruul}, wanti-lha kuwarri thurlwa-rnu. Wanthala parrapari?\\
	that.\textsc{nom} keep-\textsc{pass}.\textsc{pfv} thing-\textsc{caus}-\textsc{hab} pull-\textsc{pass}-\textsc{hab} rifle-\textsc{assoc} that.\textsc{nom} lie-\textsc{prs} powder-\textsc{proprietive} still lie-\textsc{pst} now pull--\textsc{pass}.\textsc{pfv}  where rifle\\
	\glt \lq That one was being kept, [the thing that] makes it what's-its-name, the one that gets pulled through, for a rifle. That cloth still has powder on it as if it had just been pulled through. But where's the rifle?'  \parencite[147]{Dench1994}\footnote{See \textcite[84–87]{Dench1994} on \lq\lq proprietive" \mbox{-\textit{marta}}.}

	\ex\label{exappendixMartuthinura2}
	Context: People are trying to catch a boomerang, chasing and hitting it.\\
	\gll Piyuwa ngunhu, puni-nyila ngunhu \textbf{waruul}. Thampa-rru jirli wurnta-nngu-rra yartapalyu. Yanga-lalha ngurnaa. Ngunhaa puni-nyila nyingkurlu \textbf{waruu}, ngulangu pungka-lu. Karti-ngka manku-ngu-layi-warnu. \\
	finish that.\textsc{nom} go-\textsc{prs}:\textsc{rel} that.\textsc{nom} still almost-now arm cut-\textsc{pass}-\textsc{sim} others chase-\textsc{pst} that.\textsc{acc} that.\textsc{nom} go-\textsc{prs}:\textsc{rel} in\_front still there fall-\textsc{purp}:\textsc{ss} side-\textsc{loc} grab-\textsc{pass}-\textsc{fut}-\textsc{emph}\\
	 \glt \lq But there's no chance, it's still travelling. Some others almost get their arms cut off. They chased it and that boomerang is still in front and falls right there, and is picked up once more.\rq{ }\parencite[293–294]{Dench1994}

	\ex \label{exappendixMartuthinura3}
	 \gll Nhiyu warnan parnta-rnuru-rru warnu ngaliwa-a. Muthu-npa-layi-rru. Nhiyu ngapala-ma-rnu-rru warnan-tu. Nhiyu parnta-rnuru \textbf{waruu}. Wantharni-npa-layi-lwa parnta-rninyji, wayil waya-a yirla?\\
	this.\textsc{nom} rain rain-\textsc{prs}-now \textsc{ass} 1\textsc{pl}.\textsc{incl}-\textsc{acc} cold-\textsc{inch}-\textsc{fut}-now this.\textsc{nom} mud-\textsc{caus}-\textsc{pass}.\textsc{pfv}-now rain-\textsc{inst} this.\textsc{nom} rain-\textsc{prs} still how-\textsc{inch}-\textsc{fut}-\textsc{identification} rain-\textsc{fut} maybe night-\textsc{acc} only\\
	\glt \lq It's raining on us now. Now it's getting cold. It's getting muddy now from the rain. It's still raining. What's the rain going to do, it might go until tonight?\rq{}\footnote{See \textcite[182–183]{Dench1994} on the Martuthunira \lq\lq{}identification\rq\rq{ }clitic} \parencite[184]{Dench1994}
	
	\ex \label{exappendixMartuthinura4}
	 Context: You have found a hole with a goanna in it, and dug the hole open.\\
	\gll Wanthanha-rru kana murla-a kanangkalwa-lalha? \textbf{Waruul}-warru, murla-a manku-rrawaara ngurnaa.\\
	which-now \textsc{q} meat-\textsc{acc} make\_clear-\textsc{pst} still-now meat-\textsc{acc} grab-\textsc{seq} that.\textsc{acc}\\
	\glt \lq What now that you've uncovered the meat? Keep going, grab hold of that meat.' \parencite[268]{Dench1994}
\end{exe}
\pagebreak
\subsubsection{Uses on the fringes of \lq{}still\rq{}}
\paragraph{Continued iteration}
\label{appendixMartuthuniraContinuedIteration}
\begin{itemize}
	\item Several istances of \textit{waruu}(\textit{l}) in \citeauthor{Dench1994}'s (\citeyear{Dench1994}) grammar feature the sequence \textit{ngartil waruu} \lq still again'. In one instance (\ref{exAppendixMartuthuniraIterative5}) the order is reversed.
	\item The data indicate that this collocation signals continued repetition (\lq yet again'), perhaps emphasizing the recurrence:
	\begin{itemize}
		\item Out of a total of eight tokens, in five cases context and/or translation clearly indicate continued repetition of the same type of event; see (\ref{exAppendixMartuthuniraIterative1}, \ref{exAppendixMartuthuniraIterative2}).
		\item In (\ref{exAppendixMartuthuniraIterative3}) the sub-events are not completely identical: the act of throwing occurs for a third time, albeit only for the second time with this specific boomerang. Similarly, in (\ref{exAppendixMartuthuniraIterative4}) \textit{ngartil waruul} refers to a series of events (bringing home goods for others), with the roles of the participants switching between the individual occurrences.
		\item Without further context, ex. (\ref{exAppendixMartuthuniraIterative5}) is unclear: I can only assume that the addressee has been dissatisfied more than once before.
	\end{itemize}
\end{itemize}
\begin{exe}
	\ex\label{exAppendixMartuthuniraIterative1}
	\gll Wiruwanti yirla karlwa-marri-layi, \textbf{ngartil} \textbf{waruul} mungka-yarri-layi ngurnu tharnta-a.\\
	morning only get\_up-\textsc{collective}-\textsc{fut} again still eat-\textsc{collective}-\textsc{fut} that.\textsc{acc} euro-\textsc{acc}\\
	\glt \lq In the morning we'll get up together, and we'll still have another feed of that euro [a type of kangaroo].' \parencite[153]{Dench1994}

	\ex\label{exAppendixMartuthuniraIterative2}
	Context: A man has thrown a boomerang several times, and each time a group of people has tried to catch it.\\
	\gll Ngartil thawu-lalha. \textbf{Ngartil} \textbf{waruul}-purtu ngunhu-ngara yanga-lwala.\\
	again send-\textsc{pst} again still-\textsc{comp} that.\textsc{nom}-\textsc{pl} chase-\textsc{purp}:\textsc{ds}\\
	\glt \lq Again he send it, and yet again they chased it.\rq{ }\parencite[294–295]{Dench1994}

	\ex\label{exAppendixMartuthuniraIterative3}	
	Context: A man has chiselled two boomerangs. He has tried out both, and both work fine, to the astonishment of observers.\\
	\gll Patha-rralha ngurnaa \textbf{ngartil} \textbf{waruu}, ngunhu-ngara nhawu-rra.\\
	throw-\textsc{pst} that.\textsc{acc} again still that.\textsc{nom}-\textsc{pl} see-\textsc{sim}\\ 
	\glt \lq So he threw it again and they watched.' \parencite[292]{Dench1994}

	\ex\label{exAppendixMartuthuniraIterative4}
	Context: The addressee has gone to town and brought with him goods for others. They have done the same for him on a different occasion.\\
	\gll Ngartil wii nhuwana puni-rra thawun-mulyarra,  \textbf{ngartil} \textbf{waruul} ngayu yungku-layi nhuwana-a warnmalyi-i.\\
	again if 2\textsc{pl}.\textsc{nom} go-\textsc{sim} town-\textsc{all} again still 1\textsc{sg}.\textsc{nom} give-\textsc{fut} 2\textsc{pl}-\textsc{acc} money-\textsc{acc}\\
	\glt \lq If you go to town again, I'll give you money (yet again).\rq{ }\parencite[247, 275]{Dench1994}

	\ex\label{exAppendixMartuthuniraIterative5}
	\gll Ngayu wangka-layi mir.ta-rru yinka-rninyji wirra-a ngartil yarna-rniyangu. Ngayu kuntirri-nguru-rru. Nhartu-npa-lha kuntirri-nguru? \textbf{Waruul-}warru \textbf{ngartil} yarna-nnguli-yirri kartungku.\\
	1\textsc{sg}.\textsc{nom} say-\textsc{fut} not-now chisel-\textsc{fut} boomerang-\textsc{acc} again dissatisfied-\textsc{pass}.lest 1\textsc{sg}.\textsc{nom} give\_up-\textsc{prs}-now what-\textsc{inch}-\textsc{pst} give\_up-\textsc{prs} still-now again dissatisfied-\textsc{pass}-lest 2\textsc{sg}.\textsc{inst}\\ 	\glt \lq I'll say that I won't chisel a boomerang again in case [he's] dissatisfied with me. I'm giving up now. Why am I giving up? Lest you be dissatisfied with me again.'  \parencite[249]{Dench1994}
\end{exe}

\subsubsection{Restrictive (non-temporal)}
\paragraph{(Non-scalar) exclusive}
\label{appendixMartuthuniraRestrictive}
\begin{itemize}
	\item \textcite[130–131]{Dench1994}; additional discussion is found in \textcite{SchultzeBerndt2002}.
	\item As is typical of restrictive markers, this is rather a cluster of related functions. These include restricting the reference of the predicate (\ref{exAppendixMartuthinuraRestrictive1}), marking an emphatic assertion (\ref{exAppendixMartuthinuraRestrictive2}), and signalling identity of location (\ref{exAppendixMartuthinuraRestrictive3}). It is also frequently attested with manner expressions (\ref{exAppendixMartuthinuraRestrictive4}) and often occurs together with \textit{wuraal} \lq all right' and \textit{purrkuru} \lq true', as in (\ref{exappendixMartuthinura2}) above.
\end{itemize}
\begin{exe}
		\ex\label{exAppendixMartuthinuraRestrictive1}
		\gll Jalya \textbf{waruul}. Kartu wartirra-a wiru kanyara yirla.\\
		rubbish still 2\textsc{sg}.\textsc{nom} woman-\textsc{acc} liking man only.\\
		\glt \lq You're good for nothing. The only thing you're interested in is women (lit. you're [just] rubbish …).\rq{ }\parencite[286]{Dench1994}
		
			\ex\label{exAppendixMartuthinuraRestrictive2}
	\gll Ngawu! Panyu \textbf{waruul}-warru yimpala. Punyjarti warnu pala kartu. \\
	yes good still-now like\_that generous \textsc{assert}  \textsc{prestt} 2\textsc{sg}.\textsc{nom}\\
	\glt \lq Yes! That's very good. You're certainly generous to do that.\rq{ }\parencite[275]{Dench1994}
	
		\ex\label{exAppendixMartuthinuraRestrictive3}
		\gll Yawarru \textbf{waruu}, Kawuyu-wini pularna-lwa, wanthala Jinpingayinu-wini.\\
		west still K.-near 3\textsc{pl}-\textsc{identification} somewhere J.-near\\
		\glt \lq{}Still [right there] in the west, they were near Kawuyu (Mount Nicholson), somewhere near Jinpingayinu (Peter Creek).'\\
		\parencite[81]{Dench1994}
		
		\ex\label{exAppendixMartuthinuraRestrictive4}
		\gll Ngunhaa malumalu-npa-waarru jarruru-u \textbf{waruul}.\\
		that.\textsc{nom} dark-\textsc{inch}-\textsc{purp}:\textsc{ss}-now slow still\\
		\glt \lq That will make everything go dark slowly.' \parencite[254]{Dench1994}
\end{exe}
\il{Martuthunira|)}

\section{Wardaman (wrr, ward1246)}\il{Wardaman|(}
\label{appendixWardaman}

\subsection{gayawun}

\subsubsection{General information}
\begin{itemize}
	\item Form: often occurs with the \lq\lq article suffix" \parencite{Merlan1994} \mbox{-\textit{bi}}, an element that is best understood as a non-scalar restrictive marker \parencite{SchultzeBerndt2002}.
	\item Wordhood: free morpheme.
	\item Etymology: transparently < \textit{gaya} \lq today, now' plus \mbox{-\textit{wun}}/\mbox{-\textit{gun}} \lq pertaining to'.
\end{itemize}

\subsubsection{As a \lq{}still\rq{} expression}
\begin{itemize}
	\item \textcite[164, 323]{Merlan1994}.
	\item Specialisation: examples like (\ref{exAppendixWardaman1}–\ref{exAppendixWardaman3}), as well as (\ref{exAppendixWardamanDecrement}) strongly suggest that \textit{gayawun} conforms to my definition. For instance, in (\ref{exAppendixWardaman3}) the subject's continuous sleep is contrasted with the expected polar opposite (having gotten up). Despite its transparent composition \lq relating to today, now', \textit{gayawun} is compatible with past tense contexts, as in (\ref{exAppendixWardaman3}), which indicates that its original meaning has become bleached. Further, albeit indirect, evidence comes from its use as \textsc{not yet} (\appref{appendixWardamanNotYet}).
	\item Pragmaticity: \textit{gayawun} seems to be compatible with both scenarios of \textsc{still}.
	\item Polarity sensitivity: inner negation yields \textsc{not yet}.
\end{itemize}
\largerpage
\begin{exe}
	\ex \label{exAppendixWardaman1}
	\gll \textbf{Gayawun}-bi yibi wongo ngegba-rri yum-nyi yibam deb nu-bu.\\
	still-\textsc{art} live.\textsc{abs} \textsc{neg} die.3\textsc{sg}-\textsc{pst} stick-\textsc{inst} head.\textsc{abs} hit 2\textsc{non}.\textsc{sg}-\textsc{aux}\\
	\glt \lq He’s still alive, he hasn’t died, hit him on the head with a stick.' \parencite[164]{Merlan1994}

	\ex \label{exAppendixWardaman2}
	\gll Yirr-ya \textbf{gayawun} ngo-yongi-we.\\
	1\textsc{non}.\textsc{sg}.\textsc{incl}-go.\textsc{prs} still 1\textsc{sg}>3\textsc{sg}-farewell-\textsc{fut}\\
	\glt \lq Let's go, I still want to say good-bye to him.' \parencite[164]{Merlan1994}
	
	\ex \label{exAppendixWardaman3}
	\gll Nga-gaju-rri ngayugu nurrug-bulu nurr-gurrgba-rri nganunya \textbf{gayawun}-bi.\\
	1\textsc{sg}-rise-\textsc{pst} 1\textsc{sg}.\textsc{abs} 2\textsc{non}.\textsc{sg}-\textsc{pl}-\textsc{abs} 2\textsc{non}.\textsc{sg}-sleep-\textsc{pst} 1\textsc{sg}.from still-\textsc{art}\\
	\glt \lq Me, I’d [already] gotten up, as for you, you were still sleeping on me.' \parencite[323]{Merlan1994}
\end{exe}

\subsubsection{Uses on the fringes of \lq{}still\rq{}}
\paragraph{Scalar contexts}\label{appendixWardamanScalar}
\begin{itemize}
	\item One item in the data involves a context of a decrease over time.
\end{itemize}
\begin{exe}
	\ex Context: White settlers have shot many of the Wardaman.\label{exAppendixWardamanDecrement}\\
	\gll Wunggun-bu-ndi lege\sim lege \textbf{gayawun} na.\\
	3\textsc{sg}>3\textsc{non}.\textsc{sg}-hit-\textsc{pst} \textsc{redupl}\sim one.\textsc{abs} still now\\
	\glt \lq He shot them, [there were] just a few now (lit. a few still [left] now).\rq{ }\parencite[398]{Merlan1994}
\end{exe}

\subsubsection{Uses pertaining to other phasal polarity concepts}
\paragraph{Not yet}\label{appendixWardamanNotYet}
\begin{itemize}
	\item \textcite[164, 323]{Merlan1994}.
	\item In this function, \textit{gayawun}(\textit{bi}) serves as a pro-sentence/interjection and typically features notions like {\lq\lq}\lq wait!, hang on!', or the like" \parencite[323]{Merlan1994}.
\end{itemize}
\largerpage[2.25]
\begin{exe}
	\ex
	\begin{xlist}
		\exi{A:}\gll Ngayin.gun-yonga-rri.\\
		3\textsc{non}.\textsc{sg}>1\textsc{du}.\textsc{incl}-farewell-\textsc{pst}\\
		\glt \lq They've said good-bye to the two of us.'
		\exi{B:}\gll \textbf{Gayawun} ngawun-yongi-we!\\
		still 1\textsc{sg}>3\textsc{non}.\textsc{sg}-farewell-\textsc{fut}\\
		\glt \lq Hang on! I have to say goodbye to them!' \parencite[323]{Merlan1994}
	\end{xlist}
	
	\ex Context: Pregnant Wardaman women are afraid of visual penetration.\\
	\gll Maybe nana wurren ye-we-yen yigle warda-gan \textbf{gayawun}-bi.\\
	might\_be that.\textsc{abs} child.\textsc{abs} 3-3\textsc{sg}-\textsc{pot} rotten \textsc{q}.\textsc{emph}-\textsc{indef} still-\textsc{art}\\
	\glt \lq Might be the child will be born dead [because of men looking at the mother], I don't know, wait and see.' \parencite[431]{Merlan1994}
\end{exe}
\il{Wardaman|)}

\section{Wubuy (nuy, nung1290)}\label{appendixWubuy}\il{Wubuy|(}
\subsection{Introductory remarks}
Apart from descriptive materials, I searched \citeauthor{Heath1980}'s (\citeyear{Heath1980}) text collection. A note on the Wubuy aspect system: simplifying slightly, in the past and future Wubuy distinguishes between what \textcite{Heath1984} terms \lq\lq punctual" and \lq\lq continuous" aspects (glossed as and \textsc{punct} and \textsc{cont} in the examples below). This is not an opposition of aspectual operators, but functions in the actional dimension (\lq\lq lexical aspect"): The punctual aspect naturally combines with achievement and semelfactive predicates, whereas continuous is the aspect of choice for states and processes. The latter, can, however, also be construed in the punctual aspect, thereby converting them into achievements that denote either a near-spontaneous event or the entrance into a state/process. The interested reader is directed to \textcite[337–341]{Heath1984} for further discussion. Lastly, Wubuy has an elaborate noun-class system. I gloss the relevant forms as \textsc{ncl} for \lq noun class', together with a subscript of the labels used in  \textcite{Heath1984}. That is, \textsc{ncl}\textsubscript{ANA} stands for \lq noun class referred to by the shorthand ANA'.

\subsection{-wugij}
\subsubsection{General information}
\begin{itemize}
	\item Wordhood: a bound morpheme that straddles the boundary between suffix and enclitic and which combines with hosts of various syntactic classes. \textcite{Heath1984} terms this type of marker \lq\lq postposition". 
	\item Syntax: -\textit{wugij} generally attaches to its focus; with the main predicate as its host, it can be interpreted as a sentence adverb.
\end{itemize}

\subsubsection{As a \lq{}still\rq{} expression}
\begin{itemize}
	\item \citeauthor{Heath1982} (\citeyear[238, 303]{Heath1982}; \citeyear[217, 320, 335, 447–448]{Heath1984}); additional discussion in \citeauthor{vanBaar1991} (\citeyear{vanBaar1991}, \citeyear[111]{vanBaar1997}) and \textcite{SchultzeBerndt2002}.
	\item Specialisation: -\textit{wugij} is primarily a multifaceted restrictive marker. When attached to (parts of) the main predicate or a secondary predicate, \textsc{still} is among one of its functions. This becomes evident in examples like (\ref{exAppendixWubuy1}–\ref{exAppendixWubuy3}). For instance, in (\ref{exAppendixWubuy1}), the woman continues to see the protagonist, a situation that no longer holds soon after. Further, albeit indirect evidence comes from the robustness of the \textsc{still}-restrictive polysemy and the semantic parallels between these functions (\Cref{sectionExclusive}).
	\item Polarity sensitivity: uncommon in negative contexts \parencite[448]{Heath1984}; where attested, this yields \textsc{not yet}.
	\item Pragmaticity: judging from \citeauthor{Heath1980}'s (\citeyear{Heath1980}) text collection, \mbox{-\textit{wugij}} appears to be compatible with both scenarios.
	\item Further note: There is a strong tendency for \mbox{-\textit{wugij}} as \textsc{still} to occur with demonstratives, verbs, and predicate nominals as the host constituent, whereas the restrictive function (\appref{appendixWubuyRestrictive}) occurs mainly with nominals in non-predicative use \parencite[217, 447]{Heath1984}. Other than suggested by \citeauthor{vanBaar1991} (\citeyear{vanBaar1991}, \citeyear[111]{vanBaar1997}), the association between function and syntactic class/function of the host is a probabilistic one rather than an absolute rule. Counterexamples include (\ref{appendixWubuyRestrictive3}). Ex. (\ref{exAppendixWubuyKeep}) illustrates the use in an imperative.
	
\end{itemize}
\begin{exe}
	\ex\label{exAppendixWubuy1}
	 Context: The protagonist is performing for a woman.\\
	\gll Nigaːjbajmiri niːbad̠iːni, oːbani daji niwann\textsuperscript{g}awann\textsuperscript{g}aː daji, wair̠i an\textsuperscript{g}unani, n\textsuperscript{g}ununani-\textbf{wugij} niyamanʸ, jal̠g! niyal̠dhinʸ.\\
	he\_himself he\_tapped\_sticks\_for\_himself.\textsc{cont} that.\textsc{ncl}\textsubscript{ANA} there he\_danced.\textsc{cont} there not he\_saw\_her she\_saw\_him.\textsc{cont}-still he\_did\_that.\textsc{punct} go\_past he\_went\_past.\textsc{punct}\\
	\glt \lq He was tapping the sticks together by himself. He danced there. He could not see her, but she could still see him. Then he went past (out of sight).' \parencite[173–174]{Heath1980}
	
	\ex\label{exAppendixWubuy2}
	Context: King Brown snake and Water Python have both set out from the land of their respective moiety.\\
	\gll Ad̠aba niyan\textsuperscript{g}i, yin\textsuperscript{g}ga muga nan\textsuperscript{g}udalhardharg niːnjamanʸ, wulhalmandhaːyun\textsuperscript{g} anubani nigawi-\textbf{wugij} analhaːl … n\textsuperscript{g}iga:yun\textsuperscript{g} anuwagaːla ad̠aba n\textsuperscript{g}iyan\textsuperscript{g}i arwiyaj, ya:ji yin\textsuperscript{g}a wulhalyirija-\textbf{wugij} anubani n\textsuperscript{g}iyamanʸ.\\
	then he\_went.\textsc{cont} nearly indeed King\_Brown he\_thought.\textsc{punct} it\_was\_Mandhaːyun\textsuperscript{g}\_country that it\_was\_his-still country {} as\_for\_her from\_there then  she\_went.\textsc{cont} upward here nearly it\_was\_Yirija\_country-still that she\_did\_that.\textsc{punct}\\
	\glt \lq Then King Brown $[$snake$]$ went along. He thought that he was still in the territory of the Mandhaːyun\textsuperscript{g} moiety, that it was still his (country) … As for her, she came up (toward the inland hillls). She was still in Yirija moiety territory.' \parencite[150–151]{Heath1980}
	
	\ex\label{exAppendixWubuy3}
	Context: Tortoise has been collecting mussels for a long time. Bandicoot has gone up towards tortoise.\\
	\gll Nan\textsuperscript{g}aː ad̠aba wuguru aːgambaː ad̠aba, manawan̠gurag wuguru, wuguru waːd̠almaːran\textsuperscript{g} wuguwuni, bagu n\textsuperscript{g}a wuːbuburi aːguguruj-\textbf{bugij} wini-man\textsuperscript{g}aman\textsuperscript{g}i wurugu wurugurij\\
it\_burned\_it.\textsc{cont} then it it\_cooked\_it\_in\_oven.\textsc{cont} then bandicoot it it tortoise permanent there and\_then it.\textsc{ncl}\textsubscript{\textsc{wara}}\_sat.\textsc{cont} in\_water-still it.\textsc{ncl}\textsubscript{WARA}\_got\_it.\textsc{ncl}\textsubscript{WARA} later slow\\
\glt \lq It (bandicoot) cooked them (mussel), it cooked them in a stone oven then, bandicoot. On the other hand, tortoise was still there in the water (collecting mussels). It was slowly getting them.\rq{ }\parencite[202]{Heath1980} 

	\ex \label{exAppendixWubuyKeep}
	Context: The narrator and his companions have encountered a buffalo.\\
	\gll N\textsuperscript{g}ayardinʸ, \lq\lq numbura:ran\textsuperscript{g}gana-\textbf{wugij}"\\
	I\_started\_it(motor).\textsc{punct} \phantom{\lq\lq}you(pl)\_look.\textsc{cont}-still\\
	\glt \lq I started the engine (of the vehicle). \lq\lq You all keep looking,\rq\rq (I shouted to the others).' \parencite[512]{Heath1980}	
\end{exe}

\subsubsection{Broadly adverbial temporal-aspectual functions}
\paragraph{(Iterative and) restitutive}
\label{appendixWubuyIterative}
\begin{itemize}
	\item \textcite[447–448]{Heath1984}.
	\item \textcite{Heath1984} only discusses the restitutive use, which is illustrated in (\ref{appendixWubuyRestitutive1}, \ref{appendixWubuyRestitutive2}), and is restricted to the \lq\lq punctual" aspect. There is only one good candidate for an iterative reading in the data consulted (\ref{exAppendixWubuyIterative}). This example borders on the restitutive (i.e. \lq threw the spear back at it'), but the discourse context suggests that what is salient is the repeated act of throwing. Like the restitutive uses of -\textit{wugij}, (\ref{exAppendixWubuyIterative}) features the \lq\lq punctual" aspect.
\end{itemize}	
\begin{exe}
	\ex \label{exAppendixWubuyIterative}
	Context: A boy has been throwing a spear at a bush.\\
	\gll n\textsuperscript{g}i-ga bagu n\textsuperscript{g}i=julubi-ʼ-nʸ a-n\textsuperscript{g}aːl̠i-duj a-n\textsuperscript{g}aːl̠i niwu-walwara=r̠a-m-\textbf{bugij} bagu\\
	\textsc{pro.}\textsc{f}.\textsc{sg} there 3\textsc{sg}.\textsc{f}=put\_in-\textsc{refl}-\textsc{pst}.\textsc{punct} \textsc{ncl}\textsubscript{ANA}-bush\_with\_berries-\textsc{loc} \textsc{ncl}\textsubscript{ANA}-bush 3\textsc{sg}.\textsc{m}>\textsc{ncl}\textsubscript{ANA}-shrub=spear-\textsc{pst}.\textsc{punct}-still there\\
	\glt \lq She (Emu) jumped in and hid right there in (that same) bush. He threw the spear at it again.' \parencite[30]{Heath1980}
	\ex Context: A python has devoured two boys. A magician has killed the python and has taken out her guts containing the boys.\label{appendixWubuyRestitutive1}
	\exi{}\gll 
	Ni=lhan\textsuperscript{g}ad̠bi-nʸ wani=ya-nʸ-\textbf{bugij} mana-n\textsuperscript{g}udan man-uba-ma-yun\textsuperscript{g} wani=ya-nʸ yuːguni, wani=ya-nʸ.\\
	3\textsc{sg}.\textsc{m}=emerge-\textsc{pst.punct} 3\textsc{sg}.\textsc{m}>3\textsc{pl}=give-\textsc{pst}.\textsc{punct}-still \textsc{ncl}\textsubscript{MANA}-guts \textsc{ncl}\textsubscript{MANA}-\textsc{anaph}-\textsc{ncl}\textsubscript{MANA}-\textsc{abs} 3\textsc{sg}.\textsc{m}>3\textsc{pl}=give-\textsc{pst}.\textsc{punct} \textsc{dist}  3\textsc{sg}.\textsc{m}>3\textsc{pl}=give-\textsc{pst}.\textsc{punct}\\
	\glt \lq He came out. He gave those guts (containing the two boys) [back] to them [people].' \parencite[23]{Heath1980}

	\ex Context: Frilled Lizard had been hiding in a clump of trees. A swarm of flies has flushed him.\label{appendixWubuyRestitutive2}
	\exi{} \gll Niyaran\textsuperscript{g}ganʸ dagi niman̠burdi, nil̠al̠aginʸ-\textbf{bugij} mari, dhan\textsuperscript{g}gid̠! yaːji nuyud̠urdhinʸ.\\
	he\_looked.\textsc{punct} he\_is\_there he\_crouched.\textsc{cont}  he\_got\_up.\textsc{punct}-still and \textsc{ideoph}:chop here he\_chopped\_his\_nose.\textsc{punct}\\
	\glt \lq He (man) looked, and there he (Frilled Lizard) was crouching. He had gotten [back] up (after being disturbed by the flies), and he (Wyriyambi) chopped at him (with an axe) across the nose.\rq{ }\parencite[113]{Heath1980}
\end{exe}

\subsubsection{Restrictive (non-temporal)}
\paragraph{(Non-scalar) exclusive}
\label{appendixWubuyRestrictive}
\begin{itemize}
	\item \citeauthor{Heath1982} (\citeyear[238, 298]{Heath1982}; \citeyear[170, 217, 335, 447–448]{Heath1984}); additional discussion in \citeauthor{vanBaar1991} (\citeyear{vanBaar1991}, \citeyear[111]{vanBaar1997}) and \textcite{SchultzeBerndt2002}.
	\item As is common with restrictive markers, this is really a cluster of related functions. These include (but are not limited to) restricting the reference (\lq just, nothing but') of noun phrases (\ref{appendixWubuyRestrictive1}, \ref{appendixWubuyRestrictive2}) and predicates (\ref{appendixWubuyRestrictive3}), and emphasis on the identity of place or time frame, as in (\ref{appendixWubuySameDay}, \ref{appendixWubuyRightThere}). The latter two examples could alternatively be interpreted as involving a persistent time frame use (\lq while it is still'; see \Cref{sectionTemporalFrameTT}).
\item There is a strong tendency for the restrictive function to occur with nominals in non-predicative use \parencite[447]{Heath1984}.
\end{itemize}
\begin{exe}
	\ex\label{appendixWubuyRestrictive1} 
	\gll Mari yaːjiːli n\textsuperscript{g}unulhumuwuldhan\textsuperscript{g}i n\textsuperscript{g}unubalhunʸ ad̠aba, ad̠aba n\textsuperscript{g}unun\textsuperscript{g}uni n\textsuperscript{g}unun\textsuperscript{g}unun\textsuperscript{g}uni ad̠aba wulam-\textbf{bugij} maːramunʸmulhi, wulam-\textbf{bugij} wulam-\textbf{bugij}\\
	and from\_here she\_severed\_him\_at\_the\_waist.\textsc{cont} she\_cut\_him\_up.\textsc{punct} then then she\_ate\_him.\textsc{cont} she\_ate\_him.\textsc{cont} then blood-still it\_lay.\textsc{cont} blood-still blood-still\\
	\glt \lq Then she cut through him at the waist, along here. She cut him up (with her long bill) and ate him. There was nothing but blood lying there.' \parencite[70]{Heath1980}
	
	\ex\label{appendixWubuyRestrictive2} 
	Context: How people used to prepare cycads.\\
	\gll Maːwad̠awad̠anmaː bagaraag man\textsuperscript{g}ubagal̠an\textsuperscript{g}-\textbf{bugij}, manamamuwaj bagaraag, wirimawal̠gaː an̠ugawuy wirimabud̠dhan\textsuperscript{g}i.\\
	cycads\_became\_strong.\textsc{cont} cycad\_nut eyes-still named.\textsc{ncl}\textsubscript{MANA} cycad\_nut they\_pounded\_it.\textsc{punct} to\_stone they\_cooked\_it.\textsc{cont}\\ 
\glt \lq The cycads became firm. Cycad nuts, just the ʻeyesʼ (nuts), named \textit{bagaraag}. They pounded them on a stone and cooked them in ashes.' \parencite[416]{Heath1980}
	
	\ex\label{appendixWubuyRestrictive3}
	Context: from an expository text on how tree bark was used to wrap and bury the dead in.\\
	\gll Anajan\textsuperscript{g}awili, wuguraːyun\textsuperscript{g} waːr̠i anud̠an\textsuperscript{g}ag, wiːl̠al̠aːdijgaː-\textbf{wugij}, awumurnʸjiː, wiːl̠al̠a:di-jgaːːːː.\\
	large\_tree\_sp as\_for\_it.\textsc{ncl}\textsubscript{ANA} not wood.\textsc{ncl}\textsubscript{ANA} they\_skinned\_it.\textsc{cont}-still like\_humpy they\_skinned\_it.\textsc{cont}\\ 
	\glt \lq Jangawili tree. Not the wood part. They just took the bark (\lq skin') off, like a humpy (bark shelter, from bark of stringybark tree).' \parencite[265]{Heath1980}
	\pagebreak
	\ex Context: About preparing mangrove fruits. The fruits have been roasted.\label{appendixWubuySameDay}\\
	\exi{}\gll N\textsuperscript{g}a ad̠aba wirima:diːni, ad̠aba aːguguwuy wirimaːralhalwulhan\textsuperscript{g}i, wirimaːralhalwulhan\textsuperscript{g}i yimbaj-\textbf{bugij}, ad̠aba yimbaj-\textbf{bugij} wirimadhurmaː.\\
and\_then then they\_took\_it\_out.\textsc{cont} then to\_water they\_soaked\_it\_all.\textsc{cont} they\_soaked\_it\_all.\textsc{cont} today-still then today-still they\_crushed\_it.\textsc{cont}\\
	\glt \lq Then they (people) took them out of the oven and put them in fresh water to soak. On the same day (i.e. a few hours later) they began to grind the fruits.' \parencite[423]{Heath1980}

	\ex\label{appendixWubuyRightThere} 
	Context: About a circumcision ritual. After performing the circumcision dance, one boy after the other would get circumcised.\\
	\gll Wuːyamayamaː, wuruwan\textsuperscript{g}irimiraːdhu wurumal̠mal̠an\textsuperscript{g}imiraːdhu, wurugu wuruyay bagu bagu-\textbf{wugij}.\\
	they\_did\_that.\textsc{cont} from\_them\_staying\_up\_at\_night.\textsc{cont} from\_them\_dancing\_circumsision\_dance later they\_slept.\textsc{cont} there there-still\\
	\glt \lq They did that after staying up all night dancing the circumcision dance. Then they finally went to sleep and slept [right] there (at the circumcision ground).' \parencite[273]{Heath1980}
\end{exe}\il{Wubuy|)}
