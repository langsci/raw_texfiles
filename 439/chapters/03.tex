\setchapterpreamble{\dictum[Theodor Adorno: Minima Moralia]{\textit{Noch} der armseligste Mensch ist fähig, die Schwächen des bedeutendsten, \textit{noch} der dümmste, die Denkfehler des klügsten zu erkennen.\footnotemark}}
\chapter{Non-temporal uses}
\footnotetext{\lq{}Even the basest of humans is \textit{still} capable of detecting the weaknesses of those most eminent, even the most stupid person can \textit{still} recognise the fallacies committed by the smartest.\rq}
\label{chapter3}
\section{Introduction}
In this chapter, I discuss uses that are primarily non-temporal. More specifically, in \Cref{sectionMarginality} I address \textsc{still} expressions in \isi{marginality} function (e.g. \lq a penguin still counts as a bird\rq{}). This is followed by an overview of uses pertaining to, or closely related to, the realm of \isi{additive} \isi{focus} quantification in \Cref{sectionAdditiveMain}. In \Cref{sectionRestrictiveUebergeordnet} I survey uses of \textsc{still} expressions as \isi{restrictive} \isi{focus} quantifiers. Lastly, in \Cref{sectionBroadlyModal} I discuss uses that are modal\is{modality} (in the widest sense of the term) and/or interactional.

\section{Marginality}
\label{sectionMarginality}\is{marginality|(}\is{scale|(}
\subsection[tocentry={}]{Introduction}
In this section, I discuss a use that has come to be known under the label of \lq\lq marginality" (\cite{Ippolito2007}; \cite{Koenig1977}; \cite{Michaelis1993}, among others). In this function, a \textsc{still} expression combines with a scalar predicate or an expression for a graded category and, as the label \textit{marginality} suggests, it signals a rank that lies in proximity to a threshold. The examples in  (\ref{exMarginalSectionIntro1}, \ref{exMarginalSectionIntro2}) are illustrations. In (\ref{exMarginalSectionIntro1}) Juan is portrayed as borderline bearable. People ranking even lower in the speaker's esteem, such as Pedro, fall outside the relevant region on a scale of sympathy.\il{German|(} In (\ref{exMarginalSectionIntro2}) \textit{noch} highlights that Aachen lies within German territory, albeit in proximity to the border. Not much further West, one crosses into Belgium. Such comparisons\is{comparison} between applicable and non-applicable arguments (i.e. contrastive topics\is{topic}) are a recurrent theme with the marginality use.\footnote{\textcite[ch. 7.2]{MosegaardHansen2008}, in dealing with data from \ili{French}, makes a distinction between two types of marginality uses. In the first type, which she terms \lq\lq scalar", two scales are aligned with each other. This is found in (\ref{exMarginalSectionIntro1}), where the a ranking of subjects is aligned with different degrees of sympathy. In the second type, which she terms \lq\lq categorizing", a binary question of inclusion vs. exclusion in a given category is at stake, typically in a copula clause. This second type is instantiated in (\ref{exMarginalSectionIntro2}). As \textcite[181]{MosegaardHansen2008} points out herself \lq\lq [o]ne may legitimately wonder whether there is any reason to distinguish the scalar and the categorizing use at the semantic level". Against this background, and given my often limited third-party data, I do not follow this distinction in my discussion.}

\begin{exe}
	\ex\label{exMarginalSectionIntro1}\ili{Spanish}\\
	\gll \textbf{A} \textbf{Juan} \textbf{todavía} \textbf{lo} \textbf{aguant}-\textbf{o}, pero a Pedro no.\\
	\textsc{acc} J. still 3\textsc{sg}.\textsc{acc}.\textsc{m} tolerate-1\textsc{sg} but \textsc{acc} P. \textsc{neg}\\
	\glt \lq \textbf{I still stand Juan}, but not Pedro.' (\cite[3]{EderlyCurco2016}, glosses added)

	\ex\label{exMarginalSectionIntro2}German\\
	\gll \textbf{Aachen} \textbf{lieg}-\textbf{t} \textbf{noch} \textbf{in} \textbf{Deutschland}, Lüttich ist schon Belgien.\\
	A. lie-3\textsc{sg} still in Germany, Liège \textsc{cop}.3\textsc{sg} already Belgium\\
	\glt \lq \textbf{Aachen is still in Germany}, Liège is in Belgium already.'
	\\(personal knowledge)\il{German|)}
\end{exe}

Note that marginality is not conventionally associated with \isi{focus} (\cite[276]{Grosz2012}; \cite[151–152]{Koenig1991}), which distinguishes it from the \isi{additive} and \isi{restrictive} operator uses that I discuss in \Cref{sectionAdditive,sectionRestrictiveUebergeordnet}, but aligns it with phasal polarity.

\subsection[tocentry={}]{Distribution in the sample}
\Cref{tableMarginal} lists the expressions for which the marginality use is found in the sample data. As can be gathered, it is attested for 21 expressions from 19 distinct languages, and for all macro-areas except Australia.\footnote{The marginality use is also attested for several cognates of expressions in my sample, e.g. \ili{Dutch} \textit{nog} (\cite{Rombouts1979}; \cite{Vandeweghe1984}), \ili{Catalan} \textit{encara}  \parencite{PerezSaldanyaSalvador1995}, or the various cognates of Serbian-Croatian-Bosnian\il{Serbian}\il{Croatian}\il{Bosnian} \textit{još} across Slavic (\cite{Bogacki1993}; \cite[s.v. \textit{ešte}]{SSSJ}; \cite[s.v. \textit{jeszcze}]{PWN}; \cite[s.v. \textit{ješte}]{SSJC};  \cite[s.v. \textit{ще}]{CYM11}; \cite[s.v. \textit{ešte}]{KSS4}; \cite{Komarek1979}; \cite{Mustajoki1988}). In addition, it is found with \ili{Hungarian} \textit{még} \parencite{CsirmazSlade2020}.}  They are, however, relatively rare in Papunesia compared to the number of sample languages. I cannot disregard the possibility that this scarcity of attestations is simply a function of the available data. At the same time, many \textsc{still} expressions in Australia and Papunesia also serve as non-scalar exclusive operators \lq just, only' (\Cref{sectionExclusive}). It is conceivable that this \isi{restrictive} function blocks the development of marginality readings.

\begin{table}
\caption{\textsc{still} Marginality use. \emph{Notes}: *: Only one example in the data. †: With reservations; see discussion below. ‡: Borderline case of a \textsc{still} expression.\label{tableMarginal}}
\begin{tabular}{lllll}
	\lsptoprule
	Macro-Area & Language & Expression &  Appendix\\
	\midrule
	Africa & \ili{Manda} & (\textit{a})\textit{kona} & \ref{appendixMandaMarginal}\\	
	& Plateau Malagasy\il{Malagasy, Plateau} & \textit{mbola}* &	\ref{appendixMalagassyMarginal}\\
	& Southern Ndebele\il{Ndebele, Southern} & \textit{sa}- & \ref{appendixSouthernNdebeleMarginal}\\
	& \ili{Swahili} & \textit{bado}  &   \ref{appendixSwahiliMarginal}\\
	& \ili{Tashelhyit} & \textit{sul} &  \ref{appendixTashelhyitMarginal}\\
	& \ili{Xhosa} &  \textit{sa}-  &  \ref{appendixXhosaMarginal}\\
	Eurasia & \ili{English} & \textit{still} & \ref{appendixEnglishMarginal}\\
	& \ili{French} & \textit{encore} & \ref{appendixFrenchEncoreMarginal}\\	
	& 		 \ili{German} & \textit{noch}  & \ref{appendixGermanMarginal}\\
	& Hebrew (Modern)\il{Hebrew, Modern} & \textit{ʕadayin}& \ref{appendixHebrewAdayinMarginal}\\
	& & \textit{ʕod}\textsuperscript{†} & \ref{appendixHebrewOdMarginal}\\
	& Mandarin Chinese\il{Chinese, Mandarin}& \textit{hái} &\ref{appendixMandarinMarginal}\\
	& Northern Qiang\il{Qiang, Northern} & \textit{tɕe}- &  \ref{appendixNorthernQiangMarginal}\\
	& Serbian-Croatian-Bosnian\il{Serbian}\il{Croatian}\il{Bosnian} & \textit{još} &   \ref{appendixBCMSMarginal}\\
	& \ili{Spanish} & \textit{aún} &  \ref{appendixSpanishAunMarginal}\\
	& & \textit{todavía} &  \ref{appendixSpanishTodaviaMarginal}\\
	& \ili{Thai} & \textit{yaŋ} &  \ref{appendixThaiMarginal}\\
	North America & \ili{Creek} & (\textit{i})\textit{monk}\textsuperscript{‡} &  \ref{appendixCreekMarginal}\\
	Papunesia & \ili{Kalamang} & \textit{tok} &  \ref{appendixKalamangMarginal}\\
	& \ili{Saisiyat} & \textit{nahan}* & \ref{appendixSaisiyatMarginality}\\
	South America & \ili{Trió} & =\textit{nkërë} & \ref{appendixTrioMarginal}\\
	\lspbottomrule
\end{tabular}
\end{table}

\subsection[tocentry={}]{A closer look: Similarities and differences to phasal polarity}\is{persistence|(} I now turn to a closer look at the marginality use. In what follows, I first focus on the similarities to, and differences from, its phasal polarity cousin. Afterwards I discuss some more specific usage patterns and functions derived from marginality. To begin with, there is an obvious conceptual relationship between marginality and phasal polarity. \is{topic time|(}Thus, just as through \textsc{still} a situation is depicted as extending through to topic time and towards its possible end, the marginality construal involves the continuation of a scale (or a specific region thereon) through to the value or entity under discussion and towards its end (\cite{MosegaardHansen2002}, \citeyear[ch. 7]{MosegaardHansen2008}; \cite{Koenig1977}, \citeyear[ch. 7.4]{Koenig1991}; \cite{Liu2000}; \cite{Loebner1989}; \cite{Michaelis1993}; \cite{Yeh1998}; among others). Correspondingly, marginality \lq\lq \textit{still} establishes a perspective where the P scale has been under discussion", whereas its counterpart \lq\lq \textit{already}\is{already} P establishes a perspective where the ¬P to P transition is salient" \parencite[25]{Ippolito2007}.\il{German|(} For instance, an example like (\ref{exMarginalSectionIntro2}) above is felicitous in a context where locations within Germany have been established as the topic\is{topic} of discussion. \Cref{figureMarginality} is a graphic illustration.	


\setlength{\MinimumWidth}{\widthof{Aachen}}
\begin{figure}[hbt]
	\begin{subfigure}[t]{0.48\linewidth}
		\centering
		\begin{tikzpicture}[node distance = 0pt]
			\node[mynode, text width=\schmal, fill=cyan, very near start] (spain){Situation};
			\node[mynode, fill=none, text width=\schmal-1ex,  right= of spain] (france){(\small{$\Diamond$}\neg{ }Sit.)};
			\draw[-, densely dashed] (spain.north west) to ($(spain.north west) + (0,-\hoehe*2) $) node [below, label distance=0, align=center] {\strut{}Onset\\\strut};
			\draw[-, densely dashed] ($(spain.north east)+(-0.5*\hoehe,0)$) to ($(spain.north east) + (-0.5*\hoehe,-\hoehe*2) $);		
			\draw[-, densely dashed] (spain.north east) to ($(spain.north east) + (0,-\hoehe*2) $) node [below, align=center, label distance=0, xshift=-0.25*\hoehe] {\strut{}Topic\\time\strut{}};
			\draw[-latex, line width=0.5pt]  (spain.south west) to  ($(france.south east)+(1ex,0)$) node [right] {t};
		\end{tikzpicture}
	\subcaption{\textsc{still}}	
	\end{subfigure}	
	\begin{subfigure}[t]{0.48\linewidth}
		\centering

		\begin{tikzpicture}[node distance = 0pt]
			\node[mynode, text width=\schmal, fill=cyan, very near start] (spain){Germany};
			\node[mynode, text width=\schmal, right= of spain] (france){(Belgium)};
			\draw[-] (spain.south east) to (spain.north east);
			\draw[-, densely dashed, label distance=0] (spain.north west) to ($(spain.north west) + (0,-\hoehe*2) $) node [below, label distance=0, align=center] {\strut{}Centre\\\strut};
			\draw[-,thick] (spain.north east) to ($(spain.north east) + (0,-\hoehe*2)$)  node [below, align=left, text width=\MinimumWidth, label distance=0, anchor=north west] {\strut{}Border\\\strut};
			\draw[-latex, line width=0.5pt]  (spain.south west) to  (france.south east) node [right] {d};
						\draw[-, densely dashed] ($(spain.north east)+(-0.5*\hoehe,0)$) to ($(spain.north east) + (-0.5*\hoehe,-\hoehe*2) $) node [below left, label distance=0, align=right] {\strut{}Aachen\\\strut};
		\end{tikzpicture}
	\subcaption{Marginality (ex. \ref{exMarginalSectionIntro2})\label{figureMarginalityB}}	
	\end{subfigure}
	\caption{Illustration of the marginality use\label{figureMarginality}}
\end{figure}\is{persistence|)}\il{German|)}\is{topic time|)}

Despite this striking similarity, there are distinctive differences between the two uses. The first and obvious one pertains to the temporal vs. non-temporal nature of the scales in question. The second, and more subtle, point of divergence is the reality status of the threshold,\is{discontinuation|(} which is indicated by the dashed vs. solid rightmost vertical line in \Cref{figureMarginality}. Whereas the appropriateness conditions of \textsc{still} only require that a discontinuation scenario be entertainable, marginality

\begin{quote}
can be said to conventionalize … the presupposition of expected\is{expectations} transition: the speaker’s assertion that an entity bears some scalar property is informative only in so far as the entity’s location … is subject to debate. The equivocal nature of the entity’s membership … arises from it being situated at or near a transition point \parencite[238]{Michaelis1993}
\end{quote}

Consequently, the marginality use is not available with clear-cut cases, such as central members of a category (\cite{Deloor2012}; \cite[181]{MosegaardHansen2008}; \cite{Muller1991}; \cite{Rombouts1979}; among others).\il{German|(} Thus, (\ref{exMarginalGoettingen}) is markedly odd, given that Göttingen lies slap bang in the centre of Germany.

\begin{exe}
	\ex[]{\label{exMarginalGoettingen}German}
		\exi{}[?]{\gll Göttingen lieg-t \textbf{noch} in Deutschland.\\
		G. lie-3\textsc{sg} still in Germany\\
		\glt \lq Göttingen is still in Germany.' (personal knowledge)}
\end{exe}\is{discontinuation|)}\il{German|)}

\is{topic time|(}The third difference between marginality and phasal polarity \textsc{still} lies in the orientation of the scale. Recall that \textsc{still} presupposes an earlier runtime of the same situation and addresses the question of whether it has ceased at topic time (the time under discussion). Both the earlier temporal relatum and topic time can be understood as positive calendric values. Scales such as the geographic regions, however, end at their most decentral members (those furthest removed, most unrepresentative, etc.). In other words, the city of Aachen in (\ref{exMarginalSectionIntro2}) is an applicable argument for the predicate \lq be in Germany', but a location-wise highly atypical one. Correspondingly, \textcite[9]{Umbach2009} posits an ordering of \lq\lq inverse prototypicality" and \textcite[175]{MosegaardHansen2008} speaks of the addressee being \lq\lq instructed to scan the relevant scale of values counter-directionally." This is illustrated in \Cref{figureMarginalityMap}.

\begin{figure}[hbt]
	\setlength{\fboxsep}{0pt}
	\setlength{\fboxrule}{0.5pt}
 	\frame{\includegraphics[width=0.5\textwidth]{figures/Aachen.eps}}
	\caption{Graphic illustration of (\ref{exMarginalSectionIntro2})\label{figureMarginalityMap}}
\end{figure}\is{topic time|)}

These differences in meaning, in turn, can have grammatical correlates. For instance, though the marginality use, just like its phasal polarity ancestor, naturally combines with an \isi{imperfective} viewpoint,\is{aspect} it is occasionally attested with a \isi{perfective} one; see (\ref{exMarginalityGotOffLightly}). In line with the above observations on (\ref{exMarginalSectionIntro1}–\ref{exMarginalGoettingen}), what is at issue in such instances is not the event depicted in the main predicate, but a degree variable and the question of its location inside or outside the bounds of some salient scale. 

\begin{exe}
	\ex Serbian-Croatian-Bosnian\il{Serbian}\il{Croatian}\il{Bosnian} \label{exMarginalityGotOffLightly}\\
	\gll Ti si \textbf{još} jeftino proša-o.\\
	2\textsc{sg} \textsc{cop}.2\textsc{sg} still cheaply pass.\textsc{pfv}.\textsc{ptcp}-\textsc{sg}.\textsc{m}\\
	\glt \lq You still got off cheaply.\rq{ }(\cite[72]{Prajnkovic2018}, glosses added)
\end{exe}

\is{syntax|(}When it comes to syntactic placement, \ili{English} \textit{still} in clause-final position can only have a phasal polarity interpretation, not the marginality one \parencite{Michaelis1993}. \il{German|(}Similary, German \textit{noch} in the forefield position of a V2 clause is always a phasal polarity item, and cannot have a marginality reading, as shown in (\ref{exMarginalGermanForefield}). The same holds true for isolated \textit{noch}; see (\ref{exMarginalGermanIsolated}). In the same vein, \ili{French} \textit{encore} does not have a marginality reading when used in elliptical utterance \parencite[173]{MosegaardHansen2008}.

\begin{exe}
	\ex 
	\begin{xlist}
		\exi{}German
		\ex \label{exMarginalGermanForefield}
		\gll \textbf{Noch} ist Paul gemäßigt.\\
		still \textsc{cop}.3\textsc{sg} P. moderate\\
		\glt \lq So far Paul is still moderate.'\\
		not: \lq Paul is still within the range of people counting as moderate.' (\cite[55]{Koenig1991}, glosses added)
	\ex \label{exMarginalGermanIsolated}
	\gll Paul ist gemäßigt. – \textbf{Noch}.\\
	P. \textsc{cop}.3\textsc{sg} moderate {} Still\\
	\glt \lq Paul is moderate. -- For now [he still] is.'\\
	not: \lq Paul is moderate. -- Barely so.' (personal knowledge)
\end{xlist}
\end{exe}\il{German|)}\is{syntax|)}

\subsection[tocentry={}]{A closer look: Some more usage patterns} Having established the key similarities and differences between phasal polarity \textsc{still} and its marginality counterpart, I now turn to some more specific usage patterns. \il{German|(}In the case of German \textit{noch}, the borderline status of the topical\is{topic} argument is often made explicit by additional expressions like \textit{eben} \lq just, precisely\rq{ }or (\textit{so}) \textit{gerade} \lq (just) barely\rq{}, as in (\ref{exMarginalOsna}).

\begin{exe}
	\ex German\label{exMarginalOsna}\\
	\gll Osnabrück liegt \textup{(}\textbf{gerade}\textup{)} \textbf{noch} in Niedersachsen.\\
	O. lie.3\textsc{sg} \phantom{(}just still in Lower\_Saxony\\
	\glt \lq Osnabrück is still Lower Saxony (i.e.  it is a marginal case of being in the Lower Saxony territory).' (\cite[1843]{Umbach2012}, glosses added)
\end{exe}\il{German|)}

\il{French|(}
For French \textit{encore} it has been observed that its marginality use is exclusively (or nearly so) found with predicates that involve a positive evaluation \parencite[176]{MosegaardHansen2008}. For instance, in (\ref{exMarginalFrenchDecentSum}) \textit{encore une somme} \lq still a decent sum' indicates that the inheritance is not the fortune expected by speaker A, but falls within that portion of the scale that they should be happy about. A similar finding has been reported for Mandarin Chinese \textit{hái} (\cite{Yeh1998}; \cite{Lu2019}; and references therein).\il{Chinese, Mandarin}

\begin{exe}
\ex French\label{exMarginalFrenchDecentSum}
	\begin{xlist}
	\exi{A:}\textit{Je suis bien embêté. Ma tante a légué la plus grande partie de sa fortune à un refuge animalier, et je n’aurai que 10.000 euros…}\\
	\lq I’m really annoyed. My aunt has left the better part of her fortune to an animal shelter, and only 10,000 euros to me…'
	\exi{B:}\gll Ben, 10.000 euros, c'-est \textbf{encore} \textbf{une} \textbf{somme}.\\
	well, 10,000 euros \textsc{prox}.\textsc{sg}.\textsc{m}-\textsc{cop}.3\textsc{sg} still \textsc{indef}.\textsc{sg}.\textsc{f} sum(\textsc{f})\\
	\glt \lq Well, 10,000 euros is \textbf{still} \textbf{a} \textbf{decent} \textbf{sum}.'
	\\(\cite[172]{MosegaardHansen2008}, glosses added)
	\end{xlist}
	\end{exe}
\il{French|)}

Given that better exemplars need to be conceivable, the marginality use of \textsc{still} expressions lends itself to downgrading and can even yield downright derogatory overtones. These are particularly salient in the context of direct comparisons,\is{comparison} even when the property in question is, in itself, desirable (\cite{MosegaardHansen2002}; \cite{Koenig1977}).\il{German|(} This is illustrated in (\ref{exMarginalGermanComparison}, \ref{exMarginalMandarinComparison}) for German \textit{noch} and Mandarin Chinese \textit{hái}.\il{Chinese, Mandarin} In (\ref{exMarginalGermanComparison}), Paul is depicted as a marginal case of an intelligent person. The invocation of a specific standard through \textit{von der Familie} \lq of the family' suggests that under normal circumstances he would fall below the threshold value. The conclusion is that he is outright dumb. His family, which, as a function of the superlative, ranks below him, fares even worse. A similar conclusion is brought about in (\ref{exMarginalMandarinComparison}), where the dish under discussion is explicitly described as unappetising in the immediate discourse context.\il{Chinese, Mandarin}

\begin{exe}
	\ex German\is{comparison}\label{exMarginalGermanComparison}\\
	\gll \textbf{Paul} \textbf{ist} \textbf{noch} \textbf{der} \textbf{intelligent}-\textbf{est}-\textbf{e} von der Familie.\\
	P. \textsc{cop}.3\textsc{sg} still \textsc{def}.\textsc{nom}.\textsc{sg}.\textsc{m} intelligent-\textsc{sup}-\textsc{nom}.\textsc{sg}.\textsc{m} from \textsc{def}.\textsc{gen}.\textsc{sg}.\textsc{f} family(\textsc{f})\\
	\glt \lq (They are all pretty stupid. But) \textbf{Paul} \textbf{is} \textbf{still} \textbf{the} \textbf{most} \textbf{intelligent} of the family.'  (\cite[190]{Koenig1977}, glosses added)\il{German|)}

	\ex Mandarin Chinese\is{comparison}\il{Chinese, Mandarin}\label{exMarginalMandarinComparison}\\
	\gll Zhè-dào	dòufu	bù	zěnme		hǎochī,	zhè	\textbf{hái}	\textbf{shì}	\textbf{zhè}-\textbf{jiā}	\textbf{diàn}	\textbf{zuì}	\textbf{hǎo}	\textbf{de} \textbf{cài}	\textbf{le}.\\
	\textsc{prox}-\textsc{clf} tofu \textsc{neg} that tasty \textsc{prox} still \textsc{cop} \textsc{prox}-\textsc{clf} store most good \textsc{assoc} dish \textsc{sfp}.\\
	\glt \lq This tofu dish is not that tasty, \textbf{and this is already the best dish of the restaurant}.' \parencite[60]{Liu2000}
\end{exe}

In the sample data, examples comparable to (\ref{exMarginalGermanComparison}, \ref{exMarginalMandarinComparison}) are attested for \ili{French} \textit{encore} and Serbian-Croatian-Bosnian\il{Serbian}\il{Croatian}\il{Bosnian} \textit{još}. Attestations involving comparisons,\is{comparison} but not necessarily with derrogatory overtones, are found for \ili{Kalamang} \textit{tok} and Trió \mbox{=\textit{nkërë}}. Cases like (\ref{exMarginalMandarinComparison}) probably also motivates the optional use of Mandarin Chinese\il{Chinese, Mandarin} \textit{hái} in make-do contexts like (\ref{exMarginalExtensionsMandarin1}) and in a disjunctive clause pattern \textit{yǔq́i} … (\textit{hái}) \textit{bù rú} \lq rather than … would (still) be better' that is illustrated in (\ref{exMarginalExtensionsMandarin2}). Both types involve an implicit or explicit comparison and the proposition in the scope of \textit{hái} is evaluated as somewhat better than the comparee, at least in the context of the specific local standard.\il{Chinese, Mandarin}\footnote{A similar suggestion, albeit only for make-do contexts, has been made by \textcite{Paris1988}.}

\begin{exe}
	\ex \label{exMarginalExtensionsMandarin}
	\begin{xlist}
		\exi{}Mandarin Chinese\is{comparison}\il{Chinese, Mandarin}
		\ex\label{exMarginalExtensionsMandarin1}
		\gll Bié	shuō	le,	\textbf{hái}-\textbf{shì}	\textbf{kuài}	\textbf{zǒu}	\textbf{ba}.\\
	\textsc{proh} talk \textsc{pfv} still-\textsc{cop} quickly leave \textsc{sfp}\\
	\glt \lq Don't talk anymore; \textbf{we had better leave quickly}. [i.e. leaving quickly does not solve things, but it's the best we can do right now]'
	\\\parencite[48]{Liu2000}
	
		\ex\label{exMarginalExtensionsMandarin2}\il{Chinese, Mandarin}\is{comparison}
	\gll \textbf{Yǔqí}	zài	jiē	shàng	xiánguàng,	\textbf{hái}	\textbf{bù}	\textbf{rú}	qù	dǎ	lánqiú.\\
	rather\_than \textsc{cop}.\textsc{loc} street at wander still \textsc{neg} as\_if go play basketball\\
	\glt \lq \textbf{It is} [\textbf{still}] \textbf{better} to play basketball than to wander aimlessly on the street.' \parencite[28]{HuangShi2016}
\end{xlist}
\end{exe}

\il{Spanish|(}Evaluation and indirect comparison\is{comparison} are also at play in the use of Spanish \textit{todavía} as a pro-predicate signalling that a state-of-affairs is barely within social norms (\appref{appendixSpanishTodaviaAcceptableLimits}). This is illustrated in (\ref{exMarginalExtensionsSpanishLimits1}). Like this attestation, all examples in the literature feature hypothetical states-of-affairs and serve as a comment on a factual, but unacceptable situation. As pointed out before me by \textcite{Bosque2016} and \textcite{Deloor2012}, this function of \textit{todavía} doubtlessly goes back to a marginality use plus ellipsis of an evaluative predicate. That is, example (\ref{exMarginalExtensionsSpanishLimits1}) is modelled after cases like (\ref{exMarginalExtensionsSpanishLimits2}).

\begin{exe}
	\ex 
	\begin{xlist}
	\exi{}Spanish
	\ex\label{exMarginalExtensionsSpanishLimits1}
	\gll En el coche, en la mesa, en la cama, \textbf{todavía}… pero tienes 5 minuto-s para sac-ar ese televisor d-el baño.\\
	in \textsc{def}.\textsc{sg}.\textsc{m} car(\textsc{m}) in \textsc{def}.\textsc{sg}.\textsc{f} table(\textsc{f}) in \textsc{def}.\textsc{sg}.\textsc{f} bed(\textsc{f}) still but have.2\textsc{sg} 5 minute-\textsc{pl} for remove-\textsc{inf} \textsc{dem}.\textsc{sg}.\textsc{m} television(\textsc{m}) of-\textsc{def}.\textsc{sg}.\textsc{m} bathroom(\textsc{m})\\
	\glt \lq In the car, on the table, in bed,	 [\textbf{that} \textbf{would}] \textbf{still} [\textbf{be} \textbf{acceptable}]… but you have five minutes to move that TV out of the bathroom.' (\cite{Deloor2012}, glosses added)

	\ex\label{exMarginalExtensionsSpanishLimits2}
	\gll \textbf{Todavía} que eso ocurr-ies-e en público \textbf{pod}-\textbf{ría} \textbf{pas}-\textbf{ar}, ¡pero en privado!\\
	still \textsc{subord} \textsc{dem}.\textsc{sg}.\textsc{n} happen-\textsc{pst}.\textsc{sbjv}-3\textsc{sg} in public can-\textsc{cond}.3\textsc{sg} happen-\textsc{inf} \phantom{¡}but in private\\
	\glt \lq Such a thing occurring in public, \textbf{that} \textbf{could} \textbf{happen}, but in private!?' (Goméz de la Serna, \textit{Automoribundia}, cited in \cite[222]{Bosque2016}, glosses added)
	\end{xlist}
\end{exe}\il{Spanish|)}

A broadly comparable pattern is found in Serbian-Croatian-Bosnian\il{Serbian}\il{Croatian}\il{Bosnian} and involves the reduplicated form \textit{jošjoš} (\appref{appendixBCMSJosJos}). Typically, the anaphoric neuter demonstrative \textit{to} stands in as the subject, as in (\ref{exMarginalExtensionsJosJos}). This usage signals that a state-of-affair is somewhat acceptable in a given circumstance, but not in another one. Not all native-speaker linguists consulted by me knew of this specific function, so it might be a regionalism. Interestingly, however, outside of my sample it finds a parallel in the \ili{Slovak} cognate \textit{ešteešte}, as shown in (\ref{exMarginalExtensionsSlovak}).

\begin{exe}
	\ex\label{exMarginalExtensionsJosJos}
	Serbian-Bosnian-Croatian\il{Serbian}\il{Croatian}\il{Bosnian}\\
	\gll Platiti tolike novce, to \textbf{još}\sim\textbf{još} (ali za lošu stvar nikada).\\
	pay.\textsc{pfv}.\textsc{inf} so\_much.\textsc{gen}.\textsc{sg}.\textsc{f} money(\textsc{f}).\textsc{acc}.\textsc{sg} \textsc{dem}.\textsc{n}.\textsc{sg} still\sim still \phantom{(}but for bad.\textsc{acc}.\textsc{sg}.\textsc{f} thing(\textsc{f}).\textsc{acc}.\textsc{sg} never\\
	\glt \lq To spend so much money, \textbf{well ok}  (but [I'd] never [spend it] on [such] a bad thing).\rq{ }(\cite[s.v. \textit{još}]{HJP}, glosses added)

	\ex \ili{Slovak}\label{exMarginalExtensionsSlovak}\\
	\gll V lete bolo \textbf{ešte}\sim\textbf{ešte}, ale v zime!\\
	at summer.\textsc{loc} \textsc{cop}.\textsc{ptcp}.\textsc{n}.\textsc{sg} still\sim still but at winter.\textsc{loc}\\
	\glt \lq In summer it was \textbf{so}-\textbf{so}, but in winter!' (\cite[s.v. \textit{ešte}]{SSSJ}, glosses added)
\end{exe}

\il{Spanish|(}The marginality use of \ili{French} \textit{encore} and Spanish \textit{todavía} has led them to serve as \textit{at least}-type \is{restrictive|(} operators in counterfactual conditionals.\is{conditional}  Example (\ref{exMarginalSpanishConditional}) is an illustration; for more discussion see \Cref{sectionScalarRestrictive}.

\begin{exe}
	\ex Spanish\label{exMarginalSpanishConditional}\is{conditional}\\
	\gll ¿Para qué ahorr-as?; \textbf{todavía} \textbf{si} \textbf{tuvier}-\textbf{as} \textbf{hijo}-\textbf{s} esta-ría justificado.\\
	\phantom{¿}for what save\_money-2\textsc{sg} still if have.\textsc{pst}.\textsc{sbjv}-2\textsc{sg} child-\textsc{pl} \textsc{cop}-\textsc{cond}.3\textsc{sg} justified\\
	\glt \lq What are you saving money for? \textbf{If you at least had kids}, then it would make sense.' (\cite[s.v. \textit{todavía}]{RAEDictionary}, glosses added)
\end{exe}\il{Spanish|)}\is{restrictive|)}

\il{German|(}Lastly, in \textcite{PersohnSchonNoch}\is{additive|(} I argue that marginality also feeds into German \textit{noch} as a peculiar type of scalar additive \lq even\rq{ }in cases like (\ref{exMarginalGermanNochScalarAdditive}); also see \Cref{sectionScalarAdditive}.

\begin{exe}
	\ex German\label{exMarginalGermanNochScalarAdditive}\\
	Context: There are no more innocent things.\\
	\gll \textbf{Noch} \textup{[}\textbf{der} \textbf{Baum} \textbf{der} \textbf{blüh}-\textbf{t}\textup{]\textsubscript{\textsc{foc}}} lüg-t,\\
	still \phantom{[}\textsc{def}.\textsc{nom}.\textsc{sg}.\textsc{m} tree(\textsc{m}) \textsc{rel}:\textsc{nom}.\textsc{sg}.\textsc{m} blossom-3\textsc{sg} lie-3\textsc{sg}\\
	\glt \lq Even the blossoming tree lies\rq{}

	\sn 
\textit{in dem Augenblick, in welchem man sein Blühen ohne den Schatten des Entsetzens wahrnimmt.}\\	
	\lq the moment its bloom is perceived without the shadow of terror.'
	\\(Adorno, \textit{Minima Moralia}, glosses added)
\end{exe}\il{German|)}\is{additive|)}

\subsection[tocentry={}]{Discussion} 
As I pointed out above, there is a noteworthy conceptual similarity between phasal polarity \textsc{still} on the one hand and its marginality counterpart on the other. In fact, the latter can be understood as metonymically derived from the former (\cite{MosegaardHansen2008}; \cite{Krifka2000}; \cite{Loebner1989}; among others), at least in diachronic terms. This direction of change is, of course, in line with the well-known tendency of the subjectivisation\is{subjectivity} of meaning (\Cref{sectionSemasiologicalChange}) and a projection from times to other scales corresponds to \citeauthor{Himmelmann2004}'s (\citeyear{Himmelmann2004}) \lq\lq semanto-pragmatic context expansion" that is characteristic of grammaticalisation\is{grammaticalisation}/pragmaticisation processes.\il{Kalamang|(} As far as actual usage is concerned, bridging contexts can be readily found in examples like (\ref{exMarginalBridgingOldFrench}, \ref{exMarginalBridgingKalamang}).

\begin{exe}
	\ex Old French,\il{French, Old} 13\textsuperscript{th}/14\textsuperscript{th} century\label{exMarginalBridgingOldFrench}\\
	Context: After having being caught in a brothel, a knight had to give up his horse and armor and leave camp. The king has been asked if the horse could be given to a poor nobleman.
	\exi{}\gll Et le roy me respondi que ceste priere n’-estoit pas resonnable, que le cheval valoit \textbf{encore} \textbf{IIIIxx} \textbf{livre}-\textbf{s}.\\
and \textsc{def}.\textsc{sg}.\textsc{m} king(\textsc{m}) 1\textsc{sg}.\textsc{obj} answer.\textsc{pst}.\textsc{pfv}.3\textsc{sg} \textsc{subord} \textsc{prox}.\textsc{sg}.\textsc{f} request(\textsc{f}) \textsc{neg}-\textsc{cop}.\textsc{pst}.\textsc{ipfv}.3\textsc{sg} \textsc{neg} reasonable \textsc{subord} \textsc{def}.\textsc{sg}.\textsc{m} horse(\textsc{m}) be\_worth.\textsc{pst}.\textsc{ipfv}.3\textsc{sg} still eigthy pound-\textsc{pl}\\
\glt \lq And the King answered that this request was unreasonable, for \textbf{the horse was worth 80 pounds}, \textbf{after all} / \textbf{was} \textbf{still} \textbf{worth} \textbf{80} \textbf{pounds}.' (de Joinville, \textit{La vie de Saint Louis}, cited in \cite[177–178]{MosegaardHansen2008}, glosses added)

	\ex Kalamang\label{exMarginalBridgingKalamang}\\
	Context: A boat trip around an island.\\
	\gll Mindi warkin laur warkin kararak=tauna o {get me} tiri osew=ar=a pareir=et. Wa me  \textbf{tok} \textbf{bisa}.\\
	like\_that tide rising tide dry=so \textsc{emph} {if not} sail beach=\textsc{obj}=\textsc{foc} follow=\textsc{irr} \textsc{prox} \textsc{top} still can\\
	\glt \lq Like that the tide is low, if not we’d sail following the beaches.  This \textbf{is} \textbf{still} \textbf{OK} (lit: this, [we] \textbf{still} \textbf{can}).' \parencite{Visser2021b}
\end{exe}

For (\ref{exMarginalBridgingOldFrench}), \textcite[177]{MosegaardHansen2008} points out that a phasal polarity reading is possible, but less salient, because nowhere in the preceding texts has the persistent\is{persistence} or remaining value of the horse, let alone that of any other item, been discussed.\il{French, Old} In the Kalamang example (\ref{exMarginalBridgingKalamang}), a marginality reading and the phasal polarity one are intertwined, in that the sea levels are a function of the tide, hence of time.\il{Kalamang|)} A similar case is found in the one relevant example I have from Plateau Malagasy,\il{Malagasy, Plateau} given in (\ref{exMarginalMalagasy}). The context and \citeauthor{Dez1980}'s (\citeyear{Dez1980}) parenthetical remarks suggest that a reading of marginality is at least latent, though this is also a function of decay over time.

\begin{exe}
	\ex\label{exMarginalMalagasy}Plateau Malagasy\il{Malagasy, Plateau}\\
	Context: About a house with a ruined roof.\\
	\gll \textbf{Mbola} trano io.\\
	still house \textsc{dem}.\textsc{sg}\\
	\glt \lq Ceci sert encore de maison (parce que, par exemple, le toit, quoique endommagé, peut encore servir d'abri). [This still serves as a house (because, for example, the roof, although damaged, can still serve as a shelter).]' (\cite[128]{Dez1980}, glosses added)
\end{exe}

As observed before me by \textcite{Rombouts1979} and \textcite[77]{VictorriFuchs1996}, an overlap between the two functions is also found in some instances where, to quote \textcite[203]{Loebner1989}, \lq\lq the time scale and the reference point thereon are replaced by a scale of objects which are located in time and hence (indirectly) temporally ordered\rq\rq{}.\il{German|(} For instance, what is at stake in (\ref{ex2aryTemporalGoethe}) Goethe's style during the production of various successive works. At the same time, the character of the first entity under discussion, \textit{Jägers Abendlied}, is depicted as a borderline case that can be categorised as part of the earlier period, but whose tone leans towards the later one.

\begin{exe}
	\ex German\label{ex2aryTemporalGoethe}\\
	 Context: About different poems by Goethe.\\
	\gll Wenn \lq\lq Jägers Abendlied\rq\rq{} \textbf{noch} \textbf{zwischen} \textbf{Trotz} \textbf{und} \textbf{Ergebung} \textbf{schwank}-t, ist hier der neu-e mild-e Ton entschieden.\\
	when/if \phantom{\lq\lq}J. A. still between defiance and submission fluctuate-3\textsc{sg} \textsc{cop}.3\textsc{sg} here \textsc{def}.\textsc{nom}.\textsc{sg}.\textsc{m} new-\textsc{nom}.\textsc{sg}.\textsc{m} mild-\textsc{nom}.\textsc{sg}.\textsc{m} tone(\textsc{m}) decide.\textsc{ptcp}\\
	\glt \lq While Jägers Abendlied \textbf{still fluctuates between defiance and submission}, in this poem the new mild tone has been settled upon.\rq{ }(Staiger, \textit{Goethe}, cited in \cite[52]{Shetter1966}, glosses added)
\end{exe}\il{German|)}

On a more abstract level, when membership in a graded category is at stake, the \lq\lq mental category scan" that the hearer is instructed to perform \lq\lq can be assumed to occupy a certain (if small) amount of time" \parencite[181]{MosegaardHansen2008}.\il{German|(} Taking (\ref{exMarginalSectionIntro2}), repeated below, as an example, by the time someone scanning their mental map of Germany from centre to periphery has localised Aachen, their virtual self may be felt to have spent an advanced amount of time in the process. In the same vein, the hard-wired threshold that goes along with marginality, and which I discussed above, can be understood as the conventionalisation of an erstwhile implicature \parencite{Michaelis1993}.

\begin{exe}
	\exr{exMarginalSectionIntro2}German\\
	\gll \textbf{Aachen} \textbf{lieg}-\textbf{t} \textbf{noch} \textbf{in} \textbf{Deutschland}, Lüttich ist schon Belgien.\\
	A. lie-3\textsc{sg} still in Germany, Liège \textsc{cop}.3\textsc{sg} already Belgium\\
	\glt \lq \textbf{Aachen is still in Germany}, Liège is in Belgium already.'
	\\(personal knowledge)
\end{exe}\il{German|)}

Further support for the assumed direction of change comes from diachronic data. First of all, for about half of the expressions in \Cref{tableMarginal} a primacy of phasal polarity can be deduced from the known etymologies. What is more, diachronic studies of \ili{French} \textit{encore} and Mandarin Chinese\il{Chinese, Mandarin} \textit{hái} have shown that unequivocal instances of the marginality use only show up centuries after phasal polarity functions of the same items (\cite[177, 181]{MosegaardHansen2008}; \cite{Yeh1998}). Lastly, for three expressions in my sample the data indicate that the marginality use is unavailable. These are \ili{Japhug} \textit{pɤjkʰu}, \ili{Tima} \textit{bʌʌr}, and Tunisian Arabic \textit{māzāl} (\appsref{appendixJaphugMarginal}, \ref{appendixTimaMarginal}, \ref{appendixTunisianMarginal}).\il{Arabic, Tunisian} The examples in (\ref{exMarginalJaphug}) illustrate this for Japhug \textit{pɤjkhu}.

\begin{exe}
	\ex   \label{exMarginalJaphug}
	\begin{xlist}
		\exi{}\ili{Japhug}
		\ex Context: Talking about skills in a sport.
		\exi{}\gll \textbf{Pɤjkʰu} pjɯ-ɕɯ-nŋam-a cʰa-a.\\
		still \textsc{ipfv}-\textsc{caus}-be\_defeated-1\textsc{sg} can-1\textsc{sg}\\
		\glt \lq Currently (but not necessarily later), I can still beat him.ʼ\\
		not: \lq He still falls within the range of those I can beat.ʼ
		\\(Guillaume Jacques, p.c.)
	
		\ex\il{Japhug}
		\gll Maoxian nɯ kɯrɯ sɤtɕha maʁ ri, Lixian nɯ \textbf{pjɤkʰu} kɯrɯ sɤtɕha kɤ-rtsi ŋu.\\
		M. \textsc{dem} Tibetan area \textsc{neg}.\textsc{cop} but L. \textsc{dem} still Tibetan area \textsc{inf}-count \textsc{cop}\\
		\glt \lq Maoxian is not a Tibetan area, but Lixian still counts as a Tibetan area (that might change in the future).'\\
		not: \lq … Lixian is still [i.e. classifies as a marginal member] of a Tibetan area.' (Guillaume Jacques, p.c.)
	\end{xlist}
\end{exe}

\il{French|(}
In addition, Modern Hebrew\il{Hebrew, Modern} \textit{ʕod} appears to allow for the marginality use, but fares less well in these contexts than its near-synonym \textit{ʕadayin}. This is likely due to \textit{ʕod} being strongly associated with event-based additivity.\is{additive} \ili{French} \textit{toujours} does not have a marginality use either, but gives a reading of \lq always, invariably' in these contexts; see (\ref{exMarginalToujours}). Admittedly, \textit{toujours} is a borderline case of a \textsc{still} expression in the first place and is probably best considered a marker of stasis (\appref{appendixFrenchToujours}).

\begin{exe}
\sloppy
	\ex French \label{exMarginalToujours}
	\exi{A:} \textit{On a fait une collecte parmi les parents afin de pouvoir rénover l’aire de jeux de l’école, et on n’a eu que 1.000 euros.}\\
	\lq We took up a collection among the parents in order to renovate the school’s playground, and we only got 1,000 euros.'
	\exi{B:}\gll Hm! Enfin, c’-est \textbf{toujours} \textbf{de} \textbf{l’}-\textbf{argent}.\\
	\textsc{interj} anyway  \textsc{prox}.\textsc{sg}.\textsc{m}-\textsc{cop}.3\textsc{sg} still of \textsc{def}.\textsc{sg}-money\\
	\glt \lq Hm! Well, \textbf{it's} \textbf{always} \textbf{money}.' (\cite[172]{MosegaardHansen2008}, glosses added)
\end{exe}
\il{French|)}\is{marginality|)}\is{scale|)}

\section{Additive and related functions}\label{sectionAdditiveMain}\is{additive|(}
\subsection{Introduction}\is{focus|(} 
In this section, I turn to the use of \textsc{still} expressions as additive markers, as well as to several functions which \textcite{Forker2016} observes to be common extensions of such operators. In \Cref{sectionAdditive} I discuss additive uses in the stricter sense \lq also, in addition, another\rq{}. Subsequently, in \Cref{sectionFurtherTo} I discuss the \lq\lq{}further-to\rq\rq{ }\parencite{Klein2018} use, which combines additivity and phasal notions. In  \Cref{sectionScalarAdditive} I turn to scalar\is{scale} additive \lq even\rq{}. In \Cref{sectionComparisons} I then examine uses as degree modifiers in comparisons\is{comparison} of inequality, before turning to uses as coordinators\is{coordination} \lq and, or\rq{ }in \Cref{sectionCoordination}. In \Cref{sectionConjunctionalMain} I discuss \textsc{still} expressions as conjunctional\is{conjunction} adverbs. This is followed by a brief reflection on some of \citeauthor{Forker2016}'s (\citeyear{Forker2016}) observations on other additive operators as compared to \textsc{still}-as-additive expressions in \Cref{sectionAdditiveConclusion}. Lastly, in \Cref{sectionAdditiveRemnantUses} I discuss a few language specific remnant uses that have some loose conceptual similarity to additivity.

\subsection{(Plain) additive}\label{sectionAdditive}
\subsubsection{Introduction} As laid out in \Cref{sectionQuantificationScales}, I understand additive focus quantification in the stricter sense of the term as a signal that the communicative common ground contains at least one alternative to the focus denotation that may yield a true proposition. Against this backdrop, the Tundra Nenets\il{Nenets, Tundra} examples in (\ref{exAdditiveIntro}) are illustrations. In (\ref{exAdditiveIntroA}) the \textsc{still} expression \textit{təmna} marks the transfer of a reindeer as an addition to the brother's act of giving help. In (\ref{exAdditiveIntroB}) \textit{təmna} sets the denotation of the object noun phrase \textit{ngob} \textit{yah} \textit{xoram} \lq a mammoth\rq{ }in relation to an additional entity of the same kind.

\begin{exe}
	\ex \label{exAdditiveIntro}
	\begin{xlist}
		\exi{}Tundra Nenets\il{Nenets, Tundra}
		\ex\label{exAdditiveIntroA}
		\gll Nʼeᵒka-nʼi sʼitᵒ nʼada-wa-h tʼaxᵒmna mənʼᵒ \textbf{təmna} \textbf{tedə}-\textbf{mtᵒ} \textbf{taᵒ}-\textbf{dəm-cʼᵒ}.\\
		elder\_brother-\textsc{gen}.1\textsc{sg} 2\textsc{sg}.\textsc{acc} help-\textsc{ipfv}.\textsc{nmlz}-\textsc{gen} beside 1\textsc{sg} still reindeer-\textsc{fut.poss}:\textsc{acc}:2\textsc{sg} give-1\textsc{sg}-\textsc{pst}\\
		\glt \lq In addition to my brother helping you, \textbf{I also gave you a reindeer}.' \parencite[371]{Nikolaeva2014}

		\ex\label{exAdditiveIntroB}
		\gll \textbf{Təmna} \textbf{ngob} ya-h xora-m xadaᵒ\\
		still one place-\textsc{gen}	 bull-\textsc{acc} kill\\
		\glt \lq He killed \textbf{another} mammoth.' (\cite[18]{Labanauskas1995}, cited in \cite[186]{Nikolaeva2014})
	\end{xlist}
\end{exe} 


\begin{table}
\caption{Additive uses. \emph{Notes}: *: Weak indications, but clearly attested for the \ili{Ngambay} cognate form. †: Only one example in the data. ‡: Only in list continuations.\label{tableAdditive}}
\begin{tabular}{llll}
	\lsptoprule
	Macro-area & Language & Expression & Appendix\\
	\midrule
	Africa & \ili{Ewe} & \textit{ga}- & \ref{appendixEweAdditive}\\
	& \ili{Kaba} & \textit{bbáy}* & \ref{appendixKabaAdditive}\\
	& Plateau Malagasy\il{Malagasy, Plateau} & \textit{mbola}\textsuperscript{†} & \ref{appendixMalagasyAdditive}\\
	& \ili{Swahili} & \textit{bado} & \ref{appendixSwahiliAdditive}\\
	& \ili{Tashelhyit} & \textit{sul} & \ref{appendixTashelhyitAdditive}\\
	& Tunisian Arabic\il{Arabic, Tunisian} & \textit{māzāl} & \ref{appendixTunisianArabic}\\
	Australia\textsuperscript{‡} & \ili{Gooniyandi} & =\textit{nyali} & \ref{appendixGooniyandiAdditive}\\
	Eurasia & \ili{English} & \textit{still}\textsuperscript{‡} & \ref{appendixEnglishAdditive}\\
	& \ili{French} &  \textit{encore} & \ref{appendixFrenchEncoreAdditive}\\
	& \ili{German} & \textit{noch} & \ref{appendixGermanAdditive}\\
	& Hebrew (Modern)\il{Hebrew, Modern} & \textit{ʕod} & \ref{appendixHebrewOdAdditive}\\
	& Hills Karbi\il{Karbi, Hills} & -\textit{làng} & \ref{appendixKarbiAdditive}\\
	& \ili{Ket} & \textit{hāj} & \ref{appendixKetAdditive}\\
	& Mandarin Chinese\il{Chinese, Mandarin} & \textit{hái} & \ref{exAppendixMandarinAdditive}\\
	& Northern Qiang\il{Qiang, Northern} & \textit{tɕe}- & \ref{appendixQiangAdditive}\\
	& Serbian-Croatian-Bosnian\il{Serbian}\il{Croatian}\il{Bosnian} & \textit{još} & \ref{appendixBCMSAdditive}\\
	& Southern Yukaghir\il{Yukaghir, Southern} & \textit{ajī}/\textit{āj} & \ref{appendixKolymaAdditive}\\
	& \ili{Spanish} & \textit{aún} & \ref{appendixSpanishAunAdditive}\\
	& & \textit{todavía} & \ref{appendixSpanishTodaviaAdditive}\\
	& \ili{Thai} & \textit{yaŋ} & \ref{appendixThaiAdditive}\\
	& Tundra Nenets\il{Nenets, Tundra} & \textit{təmna} & \ref{appendixTundraNenetsAdditive}\\
	& \ili{Udihe} & \textit{xai}(\textit{si}) & \ref{appendixUdiheAdditive}\\
	North America & Classical Nahuatl\il{Nahuatl, Classical} & \textit{oc} & \ref{appendixClassicalNahuatlAdditive}\\
	Papunesia & Coastal Marind\il{Marind, Coastal} & \textit{ndom} & \ref{appendixCoastalMarindAdditive}\\
	& \ili{Paiwan} & =\textit{anan} & \ref{appendixPaiwanAdditive}\\
	& \ili{Saisiyat} & \textit{nahan} & \ref{appendixSaisiyatAdditive}\\
	& Ternate-Tidore\il{Ternate}\il{Tidore} & \textit{moju} & \ref{appendixTernateAdditive}\\
	South America & Southern Lengua\il{Lengua, Southern} & \textit{makham} & \ref{appendixEnxetSurAdditive}\\
	& \ili{Trió} & =\textit{nkërë} & \ref{appendixTrioAdditive}\\
	\lspbottomrule
\end{tabular}
\end{table}


\subsubsection{Distribution in the sample}
\Cref{tableAdditive} lists the expressions in my sample that have an additive function. As can be gathered, such uses are attested for 29 expressions from 28 distinct languages, which makes them the single-most common functional extensions in my sample. To these instances one could add \ili{Gurindji} =\textit{rni}, which forms part of the complex clitic \mbox{=\textit{rningan}} with additive functions (\appref{appendixGurindjiAdditive}). In geographic terms, all six macro-areas are covered, though there is a noticeable distributional bias towards Eurasia and, to a lesser degree, Africa. In terms of the wordhood parameter, the additive use is found with expressions of all types.\footnote{Outside of my sample, an additive use is found with several cognates of sample expressions, such as \ili{Dutch} \textit{nog} \parencite{Vandeweghe1984}, \ili{Italian} \textit{ancora} (e.g. \cite{Tovena1994}; \cite{Vegnaduzzo2000}), or cognates of Serbian-Croatian-Bosnian\il{Serbian}\il{Croatian}\il{Bosnian} \textit{još} across Slavic (e.g. \cite{Bogacki1989}; \cite[s.v. \textit{ešte}]{SSSJ}; \cite[s.v. \textit{ještě}]{SSJC}; \cite[s.v. \textit{ešte}]{KSS4}; \cite{Mustajoki1988}). It is also found, for example, with \ili{Albanian} \textit{akoma} and Modern Greek\il{Greek, Modern} \textit{akome} \parencite{Buchholz1991}, \ili{Danish} \textit{endnu} \parencite[s.v. \textit{endnu}]{DDO} and \ili{Hungarian} \textit{még} (e.g. \cite{CsirmazSlade2020}; \cite{ZhangLing2016}).}

In what follows, I first take a glance at a few examples that are ambiguous between a phasal polarity sense and an additive interpretation. I then turn to some characteristic usage patterns, focussing primarily on a comparison to other additive markers. I then briefly touch on usages involving a \textsc{still} expression in additive function together with other additive markers. Afterwards, I turn to some similarities between additivity and event repetitions, to finally address the question of functional motivation and possible diachronic pathways.


\subsubsection{Additivity and phasal polarity} Whereas in cases like the initial examples (\ref{exAdditiveIntroA}, \ref{exAdditiveIntroB}) an additive use can be clearly differentiated from a phasal polarity one,\footnote{Note, for instance, that both examples feature a \isi{perfective} viewpoint\is{aspect} plus achievement predicate,\is{actionality} which would be incompatible with the concept of \textsc{still} (\Cref{secFunctionalDiscussion}), as well as the additional clue through \textit{tʼaxᵒmna} \lq besides\rq{ }in (\ref{exAdditiveIntroA}).} this distinction is not always that clear-cut. As has been repeatedly observed in the literature, it is not uncommon to encounter attestations that can be read both ways.\footnote{See \textcite{Borillo1984}, \textcite{Bosque2016}, \textcite{Doherty1973}, \textcite[162–164]{MosegaardHansen2008}, \textcite[104–105]{Nederstigt2003}, \textcite[§30.8k]{RAEGramatica}, \textcite{Shetter1966}, \textcite{VictorriFuchs1992}, among others.}\il{German|(} For instance, example (\ref{exAdditiveGermanSiebener}) can be understood as either depicting the need for a missing piece, or as involving an addition to the growing list of parts being used.

\begin{exe}
	\ex German\label{exAdditiveGermanSiebener}\\
	Context: The speaker is assembling a wooden toy plane.\\
	\gll Also und {ach so}, und dann brauch-e ich \textbf{noch} \textbf{eine} \textbf{Siebener}-\textbf{leiste}.\\
	well and \textsc{interj} and then need-1\textsc{sg} 1\textsc{sg} still \textsc{indef}.\textsc{acc}.\textsc{sg}.\textsc{f} seven\_piece-bracket(\textsc{f})\\
	\glt \lq Well and oh yes, and then I \textbf{also need}/\textbf{still need} \textbf{a 7-hole piece}.\rq{ }(\cite[104]{Nederstigt2003}, glosses added)
\end{exe}

This \lq\lq glass half empty/glass half full\rq\rq{ }situation is particularly pronounced when the persistent\is{persistence} duration of a situation itself is at stake. Thus, the question in (\ref{exAdditiveWieLangeNoch}) is either about the amount of time left to a state-of-affairs, or an inquiry about its supplementary duration. Both, of course, ultimately amount to the same thing.

\begin{exe}
	\ex German\label{exAdditiveWieLangeNoch}\\
	\gll Wie lange wird denn das \textbf{noch} dauer-n?\\
	how long \textsc{fut}.\textsc{aux}:3\textsc{sg} \textsc{dm} 3\textsc{sg}.\textsc{n} still last-\textsc{inf}\\
	\glt \lq For how long is this still going to take? / How much longer will this take?\rq{ }(Schnitzler, \textit{Leutnant Gustl}, cited in \cite[60]{Shetter1966}, glosses added)
\end{exe}

I examine some more examples of this type below. Before that, it is worthwhile addressing a few of the semanto-pragmatic characteristics of \textsc{still} expressions in additive function.

\subsubsection{A closer look: \textsc{still} expressions as additives and incremental discourse} It has repeatedly been observed that the additive use of expressions such as \ili{French} \textit{encore}, \textit{noch}, Mandarin Chinese \textit{hái}\il{Chinese, Mandarin} or, outside of my sample, \ili{Dutch} \textit{nog} and \ili{Hungarian} \textit{még}, is strongly associated with an incremental progress in discourse and a cumulative notion of \lq\lq adding up to a larger whole\rq\rq{ }(\cite[146]{Koenig1991}).\footnote{Also see \textcite{Borillo1984}, \textcite{Eckardt2006}, \textcite{Grubic2018}, \textcite[156–158]{MosegaardHansen2008}, \textcite[100–107]{Nederstigt2003}, \textcite[141]{Noelke1983}, \textcite{Umbach2012}, \textcite{Vandeweghe1984}, \textcite{ZhangLing2016}, among others.} This characteristic is commonly understood as the textual\is{textuality} counterpart to how phasal polarity \textsc{still} portraits a situation as progressing from an earlier time through to \isi{topic time} and towards its conceivable end. It manifests itself in a preference for contexts of stepwise elaborations on an established topic,\is{topic|(} sequential events, and the like. For instance, German \textit{noch} is the default marker in questions that request further elaboration on a topic, as in (\ref{exAdditiveIncrementalDiscourseGerman1}). The other common additive marker, \textit{auch} \lq also\rq{}, on the other hand, is odd here, and would only be acceptable in a type of rhetorical question \parencite{Umbach2012}.

\begin{exe}
	\ex German\label{exAdditiveIncrementalDiscourseGerman1}
	\begin{xlist}
		\exi{A:} \textit{Gestern auf der Party habe ich Otto getroffen.}\\
		\lq Yesterday at the party I met Otto.\rq{}
		
		\exi{B:} \gll Wer war \textbf{noch} \textup{(\textbf{?}}\textbf{auch}\textup{)} da?\\
		who \textsc{cop}.3\textsc{sg}.\textsc{pst} still \phantom{?(}also there\\
		\glt \lq Who else was there?\rq{ }(\cite[1857]{Umbach2012}, glosses added)
	\end{xlist}
\end{exe}

Conversely, \textit{noch} is infelicitous in (\ref{exAdditiveIncrementalDiscourseGerman2}), which draws up a parallelism between two subjects and thereby a switch in discourse topic.

\begin{exe}
	\ex German\label{exAdditiveIncrementalDiscourseGerman2}\\
	Context: Speaker A dominates the conversation with reports about himself.
	\begin{xlist}
		\exi{A:} \textit{…und letztes Jahr war ich in Paris.}\\
		\lq{}…and last year, I was in Paris.\rq{}
		\exi{B:}
		\gll Mein-e Tante Agathe war \textbf{auch} \textup{(\textbf{\#}}\textbf{noch}\textup{)} in Paris!\\
		\textsc{poss}.1\textsc{sg}-\textsc{nom}.\textsc{sg}.\textsc{f} aunt(\textsc{f}) A. \textsc{cop}.\textsc{pst}.3\textsc{sg} also \phantom{(\#}still in P.\\
		\glt \lq My aunt Agathe was also in Paris!\rq{ }(\cite[82]{Eckardt2006}, glosses added)
	\end{xlist}		
\end{exe}\il{German|)}

In a related fashion, the additive use of Mandarin Chinese \textit{hái} has been described as requiring events to fall into a single time span \parencite{JingSchmidtGries2009}.\il{Chinese, Mandarin} Thus, example (\ref{exAdditiveMandarin}) is infelicitous, unless the discourse topic provides for a single, overarching setting (e.g. \lq in recent years he has been busy travelling\rq{}).

\begin{exe}
	\ex[]{Mandarin Chinese\label{exAdditiveMandarin}}\il{Chinese, Mandarin}
	\exi{}[\#]{\gll Tā qùnián qù le Yīngguó, jīnnián \textbf{hái} \textbf{qù} \textbf{le} \textbf{Fàgúo}.\\
	3\textsc{sg} last\_year go \textsc{pfv} England this\_year still go \textsc{pfv} France\\
	\glt (intended: \lq He went to England last year, \textbf{and this year he went to France}.\rq{}) \parencite[39]{JingSchmidtGries2009}}
\end{exe}\is{topic|)}

\il{French|(}
Where the additive use is associated with such incremental discourse management, its focus is often the last item in a list, as in (\ref{exAdditiveListeGerman}). In French this has become conventionalised in the form of the disjunctive \isi{connective} \textit{ou encore} \lq or still\rq{ }which marks the last-mentioned element in a series of non-exclusive alternatives; see (\ref{exAdditiveOuEncore}).

\begin{exe}
	\ex German\label{exAdditiveListeGerman}\\
	\gll Sie reich-te dem Händler glänz-end-e schwarz-e Vanille-stang-en, Zimt-rind-en, eine Handvoll Muskatnüsse, Lakritzen-gebild-e … \textbf{und} \textbf{was} \textbf{weiß} \textbf{ich} \textbf{noch}.\\
	3\textsc{sg}.\textsc{f} pass-\textsc{pst}.3\textsc{sg} \textsc{def}.\textsc{dat}.\textsc{sg}.\textsc{m} merchant(\textsc{m}) shine-\textsc{ptcp}-\textsc{acc}.\textsc{pl} black-\textsc{acc}.\textsc{pl} vanilla-stick-\textsc{pl} cinnamon-bark-\textsc{pl} \textsc{indef}.\textsc{acc}.\textsc{sg}.\textsc{f} handful(\textsc{f}) nutmeg.\textsc{acc}.\textsc{pl} liquorice-formation-\textsc{acc}.\textsc{pl} {} and what know.1\textsc{sg} 1\textsc{sg} still\\
	\glt \lq She handed the merchant shiny black vanilla sticks, cinnamon barks, a handful of nutmegs, liquorice \textbf{and} \textbf{what-have-you} \textbf{else}.\rq{}
	\\(\cite[627]{MetrichFaucher2009}, glosses added)

 
	\ex French\label{exAdditiveOuEncore}\\
	 \gll Appelle, Pierre, Jean \textbf{ou} \textbf{encore} Paul.\\
	call.\textsc{imp} P. J. or still P.\\
	\glt \lq Call Pierre, Jean, \textbf{or} (\textbf{even}) Paul.' (\cite[40]{Borillo1984}, glosses added)
\end{exe}

Containing alternatives to be considered at the same time, example (\ref{exAdditiveOuEncore}) illustrates another important point: an expression's preference for an incrementally progressing discourse is not to be equated with the marking of temporal advancement.\il{French|)} To give another example, the use of \textit{noch} in (\ref{exAdditiveSchnaps}) does not commit the speaker to any order in which Otto consumed his drinks. Instead, the anaphoric relatum of \textit{dann} \lq then\rq{ }can equally well be the order of mentioning (\cite{Umbach2012}; also see \cite[158]{MosegaardHansen2008} for a very similar case).

\begin{exe}
	\ex German\label{exAdditiveSchnaps}\\
	\gll Otto hat ein Bier getrunken. Dann hat er \textbf{noch} \textbf{einen} \textbf{Schnaps} getrunken.\\
	O. have.3\textsc{sg} \textsc{indef}.\textsc{acc}.\textsc{sg}.\textsc{n} beer(\textsc{n}) drink.\textsc{ptcp} then have.3\textsc{sg} 3\textsc{sg}.\textsc{m} still \textsc{indef}.\textsc{acc}.\textsc{sg}.\textsc{m} schnaps(\textsc{m}) drink.\textsc{ptcp}\\
	\glt \lq Otto had a beer. Then he \textbf{had} \textbf{a} \textbf{schnaps} \textbf{in addition}.\rq{}
	\\(\cite[1850]{Umbach2012}, glosses added)
\end{exe}\il{German|)}

In narrative discourse,\is{textuality} where the order of discourse time and that of events is normally aligned, the counterpart to a final list item consists in a recurring pattern by which \lq\lq an \lq in-between act' or \lq in-between happening\rq{} is slid in before the concluding event\rq\rq{ }\parencite[130]{Vandeweghe1984}.\footnote{In the original Dutch \lq\lq …wordt vóór die \lq afsluitende' handeling nog een \lq tussenhandeling' of \lq tussengebeuren' ingeschoven\rq\rq{}.}\il{Spanish|(} This is illustrated in (\ref{exAdditiveSpanishDioTiempo}), where Spanish \textit{todavía} goes together with the last event in the former athlete's series of personal and professional victories. This brings an example like (\ref{exAdditiveSpanishDioTiempo}) markedly close to the \lq\lq further-to\rq\rq{ }use I discuss in \Cref{sectionFurtherTo}. The key difference is that the latter use requires much less of a textual\is{textuality} embedding (it allows for easy accommodation).

\begin{exe}
		\ex Spanish\label{exAdditiveSpanishDioTiempo}\\
		Context: About a former professional athlete. He got married to a member of the royal family, and the wedding is described a \lq\lq success\rq\rq{}.\\
	\gll \textbf{Todavía} \textbf{le} \textbf{dio} \textbf{tiempo} a gan-ar una medalla de bronce en Sydney 2000…\\
	still 3\textsc{sg}.\textsc{acc}.\textsc{m} give.\textsc{pst}.\textsc{pfv}.3\textsc{sg} time to win-\textsc{inf} \textsc{indef}.\textsc{sg}.\textsc{f} medal(\textsc{f}) if bronce en S. 2000\\
	 \glt \lq \textbf{He still found time} to win a bronze medal at the Sidney 2000 olympics.\rq{}
	 
	\exi{} \textit{Esa medalla precipitó su retirada a los 32 años.}\\
	 \lq This medal precipitated his retirement at 32 years of age.\rq{}
	 \\(CORPES XXI, glosses added)
\end{exe}\il{Spanish|)}

\subsubsection{A closer look: Some correlates of incremental discourse} The incremental or cumulative notions found on the level of discourse organisation can give rise to various expressive\is{expressivity} flavours of meaning. To begin with, in the context of longer lists they can invite in a scalar\is{scale} additive inference \lq even\rq{}, as in (\ref{exAdditiveMandarinInference}).\il{Chinese, Mandarin}

\begin{exe}
	\ex Mandarin Chinese\il{Chinese, Mandarin}\label{exAdditiveMandarinInference}\\
	\gll Zhāngsān dǎsǎo le fángzi, zuò le dàngāo \textbf{hái} \textbf{yùn} \textbf{le} \textbf{zhuōbù}.\\
	Z. sweep \textsc{pfv} house do \textsc{pfv} cake still iron \textsc{pfv} tablecloth\\
	\glt \lq Zhangsan a balayé la pièce, a fait un gâteau et aussi/même repassé la nappe. [Zhangsan swept the room, baked a cake and \textbf{also/even ironed the tablecloth}.]\rq{ }\parencite[113]{Donazzan2008}
\end{exe}

\il{Spanish|(}A scalar\is{scale} reading can also arise in examples like (\ref{exAdditiveSpanishInference}), where the preceding clause suggests an already noteworthy degree of consumption. In the case of Spanish \textit{todavía}, this furthermore tends to shade over into \isi{concessive} reading \lq despite all that\rq{ }(see \Cref{sectionConcessiveConsequent}).
	
\begin{exe}	
	\ex Spanish\label{exAdditiveSpanishInference}\\
	\gll Después de todo lo que hab-ía com-ido \textbf{todavía} \textbf{pid}-\textbf{ió} \textbf{postre}.\\
	after of everything 3\textsc{sg}.\textsc{n} \textsc{rel} have-\textsc{pst}.\textsc{ipfv}.3\textsc{sg} eat-\textsc{ptcp} still order-\textsc{pst}.\textsc{pfv}.3\textsc{sg} desert\\
	\glt \lq After everything s/he had eaten, \textbf{s/he also/even ordered desert}.\rq{}
	\\(\cite[26]{Garrido1993}, glosses added)
\end{exe}\il{Spanish|)}

\il{Hebrew, Modern|(}Concessive\is{concessive} overtones are also found with Modern Hebrew \textit{ʕod} and Serbian-Croatian-Bosnian\il{Serbian}\il{Croatian}\il{Bosnian} \textit{još}. More specifically, they arise in a pattern by which the two expressions mark the introduction of an additional argument to a conclusion that runs counter to shared assumptions. This is illustrated in (\ref{exAdditiveHebrewReligious}, \ref{exAdditiveBCMSLawyer}). A similar notion of \lq (not only that), but what is more…\rq{}, albeit without \isi{concessive} flavours, also lies at the heart of \textit{ʕod} and \textit{još} as parts of \lq and how!\rq{ }formulae (\Cref{sectionAndHow}).

\begin{exe}
	\ex Modern Hebrew\label{exAdditiveHebrewReligious}\\
	Context: He did something terribly immoral.\\
	\gll Aval ha-ben\rq{}adam dati \textbf{ʕod}.\\
	but \textsc{def}-guy(\textsc{m}) religious.\textsc{sg}.\textsc{m} still\\
	\glt \lq But the fellow’s religious, \textbf{what’s more}!\rq{ }(\cite[537]{Glinert1989} and Itamar Francez, p.c.)
	\il{Hebrew, Modern|)}
	\ex Serbian-Croatian-Bosnian\il{Serbian}\il{Croatian}\il{Bosnian}\label{exAdditiveBCMSLawyer}
	\begin{xlist}
		\exi{A:} \textit{Slabo ti on poznaje propise.}\\
		\lq He hardly knows the rules.\rq{}
		\exi{B:}\gll \textbf{Još} je pravnik!\\
		still \textsc{cop}.3\textsc{sg} lawyer.\textsc{nom}.\textsc{sg}\\
		\glt \lq \textbf{And} he's a lawyer [i.e. he of all people should know]!\rq{}\\(\cite[s.v. \textit{još}]{HJP}, glosses added)		
	\end{xlist}
\end{exe}

Another variation on the expressive\is{expressivity} theme is found with \ili{Swahili} \textit{bado}, the additive function of which, illustrated in (\ref{exAdditiveSwahili}), tends to contribute more emphasis than the common additive marker \textit{pia} (Ponsiano Kanijo, p.c.). Lastly, comparable notions lie at the heart of the fixed Tunisian Arabic exclamation in (\ref{exAdditiveTunisian1}).\is{exclamation|(}\il{Arabic, Tunisian} As evidenced by the presence of the bound \isi{interrogative} marker \mbox{-\textit{ši}}, this idiomatic expression goes back to rhetorical questions of the type in (\ref{exAdditiveTunisian2}), which feature the \textsc{still} expression \textit{māzāl} in an additive function.\il{Arabic, Tunisian}

\begin{exe}
	\ex \ili{Swahili} \label{exAdditiveSwahili}\\
	\gll Fulani ni mw-ongo \textbf{na} \textbf{bado} \textbf{ni} \textbf{mw}-\textbf{izi}.\\
	so\_and\_so(\textsc{ncl}1) \textsc{cop} \textsc{ncl}1-liar and still \textsc{cop} \textsc{ncl}1-thief\\
	\glt \lq Un tel est menteur, et de plus il est voleur. [So-and-so is a liar, \textbf{and on top of that s/he's a thief}].’ (\cite[85]{Sacleux19391941}, glosses added)

	\ex 
	\begin{xlist}
		\exi{}Tunisian Arabic\il{Arabic, Tunisian}
		\ex\label{exAdditiveTunisian1}
		\gll \textbf{Māzāl}-ši tawwa!\\
		still-\textsc{q} now\\
		\glt \lq That's enough!\rq{ }\parencite{FischerEtAlTunisian}	
		
		\ex\label{exAdditiveTunisian2}
		\gll Āš \textbf{māzāl}  ṭṛah?\\
		what	still \textsc{hort}\\
		\glt \lq Na, was denn noch! [What else do you want!]\rq
		\\(\cite[650]{Singer1984}, glosses added)
	\end{xlist}	
\end{exe}\is{exclamation|)}

\subsubsection{A closer look: Non-incremental discourse}\is{topic|(} While an association between an additive use of \textsc{still} expressions and incremental or cumulative notions is certainly cross-linguistically recurrent, it is important to point out that this is far from a universal pattern. For instance, Southern Yukaghir\il{Yukaghir, Southern} \textit{āj} in (\ref{exAdditiveParallelismKolyma}) is clearly used to draw a parallel between two different topical arguments. In other words, it is found in the exact type of context that was shown to be infelicitous for the use of \ili{German} \textit{noch} in (\ref{exAdditiveIncrementalDiscourseGerman2}) above. 

\begin{exe}
	\ex Southern Yukaghir\il{Yukaghir, Southern}\label{exAdditiveParallelismKolyma}\\
	Context: People have gone gathering berries.\\
	\gll Met emdʼe juk-ō-j, lebejdī-le \textbf{āj} \textbf{šaqalʼe}-\textbf{š}-\textbf{nu}-\textbf{m}.\\
	1\textsc{sg} sibling small-\textsc{stat}-3 berries-\textsc{acc} still gather-\textsc{caus}-\textsc{ipfv}-\textsc{tr}.3\\
	\glt \lq Even though my younger brother was small, \textbf{he was gathering berries, too}.' \parencite[146, 160]{YukaghirTexts}
\end{exe}

The notion of a similitude is even more salient in (\ref{exAdditiveParallelismKolyma2}), where  \textit{āj} furthermore occurs across two adjacent clauses, in a pattern broadly reminiscent of bisyndetic coordination.\is{coordination}

\begin{exe}
	\ex Southern Yukaghir\il{Yukaghir, Southern}\label{exAdditiveParallelismKolyma2}\\
	\gll Tude-l \textbf{āj} met-ke-t joule-dʼā-j \textbf{met} \textbf{āj} \textbf{joule}-\textbf{sʼ}.\\
	3\textsc{sg}-\textsc{nom} still 1\textsc{sg}-\textsc{loc}-\textsc{abl} ask-\textsc{mid}-\textsc{intr}.3\textsc{sg} 1\textsc{sg} still ask-\textsc{tr}.1\textsc{sg}\\
	\glt \lq He asked questions of me, \textbf{and I of him}.\rq{ }\parencite[142, 159]{YukaghirTexts}
\end{exe}\is{topic|)}

Apart from Southern Yukaghir\il{Yukaghir, Southern} \textit{āj}, clear-cut attestations of such analogous additions feature in the data on Coastal Marind\il{Marind, Coastal} \textit{ndom}, \ili{Gooniyandi} \mbox{=\textit{nyali}}, \ili{Ket} \textit{hāj}, and \ili{Udihe} \mbox{\textit{xai}(\textit{si})}. With \ili{Ket} \textit{hāj} and Udihe \mbox{\textit{xai}(\textit{si})} this extends to equative comparisons.\is{comparison} This is shown in (\ref{exAdditiveEquative}) for the former of the two. In the case of Udihe,\il{Udihe} such comparisons furthermore play a role  in a recurring collocation pattern that consists of \mbox{\textit{xai}(\textit{si})} plus a set of anaphoric pronouns and which signals \isi{identity} \lq that, too > the same\rq{ }(\Cref{sectionSame}). Lastly, analogies marked by  \mbox{\textit{xai}(\textit{si})} also appear to lie at the root of its recurrent attestation in the apodoses of alternative \isi{concessive} conditionals\is{conditional} \lq no matter whether …\rq{}. This is illustrated in (\ref{exAdditiveUdiheConcessive}) and discussed in more detail in \Cref{sectionConcessiveConsequent}. 

\begin{exe}
	\ex \ili{Ket}\label{exAdditiveEquative}\is{comparison}\\
	\gll Bilʲa d-i:-n-bεsʲ, tɔʔn \textbf{hāj} \textbf{du}-\textbf{γ}-\textbf{a}-\textbf{daq}.\\
	like \textsc{subj}.3\textsc{sg}-here-\textsc{pst}-move thus still \textsc{subj}.3\textsc{m}-\textsc{lnk}-\textsc{th}-live\\
	\glt \lq как приехал, так и [тоже] живёт. [He lives the way he came, lit. the way he came, that way \textbf{he also lived}.]\rq{ }(\cite[177]{KotorovaNefedov2015}, glosses added)
	\ex \ili{Udihe}\label{exAdditiveUdiheConcessive}\is{concessive}\is{conditional}\\
	Context: Once upon a time there was a Chinese tsar. He buried people alive.\\
	\gll Ñuŋu-za seː i:ne-wene-mie bude-isiː-ni, buge-ini, e-siː-ni, bude, \textbf{xai} \textbf{buge}-\textbf{ini}.\\
	six-ten year come-\textsc{caus}-\textsc{inf} die-\textsc{pfv}.\textsc{cvb}-3\textsc{sg} bury-3\textsc{sg} \textsc{neg}-\textsc{pst}-3\textsc{sg} die still bury-3\textsc{sg}\\
	\glt \lq When a person became sixty years old, he buried him, no matter whether he was dead or not. (lit. … he died, he buried him, he did not die, \textbf{he still}\textbf{/}\textbf{also} \textbf{buried} \textbf{him}.)' \parencite[18–20]{NikolaevaEtAl2003}
\end{exe}	
	
\il{Marind, Coastal|(}Lastly, a very different case is found in Costal Marind, where the additive use of \textit{ndom} is strongly associated with motion contexts. In these environments, it comes close to a comitative marker, in that it signals that some entity goes along or is brought along. This is shown in (\ref{exAdditiveMarind}).\largerpage

\begin{exe}
	\ex Coastal Marind\label{exAdditiveMarind}\\
	\gll Namagha nok mend-am-b-euma<n>ah, \textbf{Iwoni} \textbf{ndom}, patul d-a-ola, imimil-patul menda-b w-in.\\
	now 1 1.\textsc{a}-\textsc{a}-\textsc{appl}-go<1.\textsc{u}> I. still boy \textsc{dur}-3\textsc{sg}.\textsc{a}-\textsc{cop}:3\textsc{sg}.\textsc{u} grown\_up-boy \textsc{ant}:3\textsc{sg}.\textsc{a}-be 3\textsc{sg}.\textsc{u}-become\\
	\glt \lq Now we already left, \textbf{Iwoni too}, he was a boy, a big boy already.'
	\\(\cite{Olsson2015},  glosses added)	
\end{exe}\il{Marind, Coastal|)}

\subsubsection{A closer look: Additions of the same kind} Issues of discourse structure are not the only possible area of divergence between \textsc{still} expressions in additive function and the garden-variety of additive markers. Another difference can be found in the fact (e.g. \cite[146]{Koenig1991}) that many of the relevant expressions allow for the focus denotation and the contextual alternatives to be of the same kind. This was shown in (\ref{exAdditiveIntroB}), repeated below. Example (\ref{exAdditiveFrenchBook}) is another illustration.

\begin{exe}[(22b)]
		\exr{exAdditiveIntroB} Tundra Nenets\il{Nenets, Tundra}\\
		\gll \textbf{Təmna} \textbf{ngob} ya-h xora-m xadaᵒ\\
		still one place-\textsc{gen}	 bull-\textsc{acc} kill\\
		\glt \lq He killed \textbf{another} mammoth.' (\cite[18]{Labanauskas1995}, cited in \cite[186]{Nikolaeva2014})
		
		\ex \ili{French}\label{exAdditiveFrenchBook}\\
		\gll Marie a lu \textbf{encore} \textbf{un} \textbf{livre}.\\
	M. have.3\textsc{sg} read.\textsc{ptcp} still \textsc{indef}.\textsc{sg}.\textsc{m} book(\textsc{m})\\
	\glt \lq Mary read \textbf{one more book}.' (\cite[9]{TovenaDonazzan2008}, glosses added)
\end{exe}

\il{German|(}In some cases, the \lq in addition, also\rq{ }and \lq another, more\rq{ }readings are distinguished formally.\is{prosody|(} With German \textit{noch}, this distinction is primarily prosodic. When more of the same kind is added, this particle normally receives focal stress, as in (\ref{exAdditiveTypesGermanA}), whereas stress otherwise tends to fall on the constituent denoting the additional item; see (\ref{exAdditiveTypesGermanB}).

\begin{exe}
	\ex 
	\begin{xlist}
		\exi{}German
		\ex\label{exAdditiveTypesGermanA}
		\gll Ich trink-e \textbf{ˈnoch} ein Bier.\\
	1\textsc{sg} drink-1\textsc{sg} \phantom{ˈ}still \textsc{indef}.\textsc{acc}.\textsc{sg}.\textsc{n} beer(\textsc{n})\\
	\glt \lq I will have \textbf{another} beer.' 

	\ex\label{exAdditiveTypesGermanB}
	\gll Ich trink-e \textbf{noch} ein ˈBier.\\
	1\textsc{sg} drink-1\textsc{sg} still \textsc{indef}.\textsc{acc}.\textsc{sg}.\textsc{n} \phantom{ˈ}beer(\textsc{n})\\
	\glt \lq I will have a beer (\textbf{in addition to whatever} I had before).\rq{}
	\\(\cite[143]{Koenig1991}, personal knowledge)
	\end{xlist}
\end{exe}\il{German|)}\is{prosody|)}

\is{syntax|(}In other cases, the distinction is marked purely structurally. Thus, Tundra Nenets\il{Nenets, Tundra} \textit{təmna} is a phrasal adjunct to the constituent containing the focus only when signalling \lq another\rq{}, but not as \lq in addition, also\rq{}. The same appears to be the case with Classical Nahuatl \textit{oc},\il{Nahuatl, Classical} Modern Hebrew \textit{ʕod},\il{Hebrew, Modern} and, as a strong tendency rather than a hard-and-fast rule, with \ili{French} \textit{encore}. With Southern Yukaghir\il{Yukaghir, Southern} \textit{āj}/\textit{ajī} the two readings are distinguished in syntactic terms as well as in the morphological domain. Thus, the variant \textit{āj} marks \lq in addition\rq{ }and normally occurs in the pre-verbal position, as shown in (\ref{exAdditiveTypesYukaghirA}). The variant \textit{ajī}, on the other hand, is used when \isi{identity} of kind is at stake. It can only take the P-argument of transitive verbs as its focus and is a syntactic sister to the latter; see (\ref{exAdditiveTypesYukaghirB}). 

\begin{exe}
	\ex 
	\begin{xlist}
		\exi{}Southern Yukaghir\il{Yukaghir, Southern}
		\ex\label{exAdditiveTypesYukaghirA}
		\gll Ōžī el-jūke lʼe-t-i kindʼe podʼerqo \textbf{āj} lʼe-t-i.\\
	water \textsc{neg}-far \textsc{cop}-\textsc{fut}-\textsc{intr}.3\textsc{sg} moon light still \textsc{cop}-\textsc{fut}-\textsc{intr}.3\textsc{sg}\\
	\glt \lq The water will not be far, and there will be moonlight, \textbf{too}.'

	\ex\label{exAdditiveTypesYukaghirB}\il{Yukaghir, Southern}
\gll Met āj	čumuc̄-ie-je, \textup{[}\textbf{ajī} \textbf{ningō}\textup{]\textsubscript{\textsc{np}}} ī-de-j.\\
	1\textsc{sg} still fish-\textsc{inch}-\textsc{intr}.1\textsc{sg} \phantom{[}still many get\_caught-\textsc{caus}-1\textsc{pl}.\textsc{tr}\\
	\glt \lq I began to fish, too, we have caught \textbf{much more}.'
	\\\parencite[530, 532]{Maslova2003}
		\end{xlist}
\end{exe}

With several sample expressions, such an \lq another, more\rq{ }reading has given rise to a repetitive event construal when paired with an event quantifier (\lq another time > again\rq{}).\il{German|(} Example (\ref{exAdditiveIterativeTimes}) is an illustration. I discuss this pattern in more detail in \Cref{sectionIterativeViaIncrement}.

\begin{exe}
	\ex German\label{exAdditiveIterativeTimes}\\
	\gll Opposition gewinn-t Wahl in Istanbul: İmamoğlu macht-'s \textbf{noch} \textbf{ein}-\textbf{mal}\\
	opposition win-3\textsc{sg} election.\textsc{acc}.\textsc{sg} in Istanbul İmamoğlu make.3\textsc{sg}-3\textsc{sg}.\textsc{acc}.\textsc{n} still \textsc{indef}-time\\
	\glt \lq Opposition wins elections in Istanbul: İmamoğlu does it \textbf{again}.' 
	\\(found online, glosses added)%\footnote{\url{https://taz.de/Opposition-gewinnt-Wahl-in-Istanbul/!5605032/}; (07 April, 2022).}
\end{exe}\il{German|)}\is{syntax|)}

\il{Nahuatl, Classical|(}With Classical Nahuatl \textit{oc}, the notion of \lq another\rq{ }has brought about a function in which it marks the existence of a distinct entity of the same kind \lq other, different\rq{}. This is shown in (\ref{exAdditiveCNOther}) and discussed in \Cref{sectionAdditiveRemnantUses}. Closely related to both senses, \textit{oc} also commonly participates in marking the comparative\is{comparison} degree of a predicate (\appref{appendixClassicalNahuatlComparisons}), as in (\ref{exAdditiveCNCompDegree}).

\begin{exe}
	\ex 
	\begin{xlist}
		\exi{}Classical Nahuatl 
		\ex\label{exAdditiveCNOther}	
		\gll Nicān câ ce tepētl: cān câ in \textbf{oc}-\textbf{cē}?\\
		here \textsc{loc}.\textsc{cop} one mountain where \textsc{loc}.\textsc{cop} \textsc{det} still-one\\
		\glt \lq One mountain's here. Where's the \textbf{other} \textbf{one}?'
		\\(\cite[66]{LauneyMackay2011}, glosses added)
	
		\ex\label{exAdditiveCNCompDegree}	
		\gll \textbf{Oc} \textbf{achi} ni-chicāhuac in àmo mach yuhqui tèhuātl.\is{comparison} \\
	still rather/a\_bit \textsc{subj}.1\textsc{sg}-strong \textsc{det} \textsc{neg} indeed thus 2\textsc{sg}\\
	\glt \lq Mas fuerte soy que tu. [I'm stronger than you, lit. I have \textbf{another}/\textbf{different} \textbf{bit} of strength that you don't.]\rq{}
	\\(\cite[491]{Carochi1645}, glosses added)
	\end{xlist}
\end{exe}\il{Nahuatl, Classical|)}

\il{Hebrew, Modern|(}Also closely related to additivity plus \isi{identity} of kind, Modern Hebrew \textit{ʕod} and Serbian-Croatian-Bosnian\il{Serbian}\il{Croatian}\il{Bosnian} \textit{još} can serve as indefinite quantifiers \lq some more\rq{}; see (\ref{exAdditiveTypesHebrewMore1}, \ref{exAdditiveTypesBCMSmore}).

\begin{exe}
	\ex Modern Hebrew\label{exAdditiveTypesHebrewMore1}\\
	\gll Etmol axal-ti 3 tapuz-im. Ha-yom axal-ti \textbf{ʕod} \textup{(}tapuz-im\textup{)}.\\
	yesterday eat.\textsc{pst}-1\textsc{sg} 3 orange-\textsc{pl}.\textsc{m} \textsc{def}-day eat.\textsc{pst}-1\textsc{sg} still \phantom{(}orange-\textsc{pl}.\textsc{m}\\
	\glt \lq Yesterday I ate 3 oranges. Today I ate \textbf{some more} (oranges).
	\\\parencite[127]{Greenberg2012}

	\ex Serbian-Croatian-Bosnian\il{Serbian}\il{Croatian}\il{Bosnian}\label{exAdditiveTypesBCMSmore}\\
	\textit{Kad su prodali svoje čudesne ljekarije bolje nego što su mislili}\\
	\lq When their miracle potion sold better than they had thought,\rq{}
	\exi{}\gll otišl-i su kući \textbf{po} \textbf{još}.\\
	leave.\textsc{ptcp}-\textsc{pl}.\textsc{m} \textsc{cop}.3\textsc{pl} home.\textsc{dat}.\textsc{sg} for still\\
	\glt \lq they went home \textbf{for} \textbf{more}.\rq{ }(\cite[s.v. \textit{još}]{HJP}, glosses added)
\end{exe}

Modern Hebrew \textit{ʕod} in post-predicate position can also quantify over an intransitive verbal predicate, yielding \lq do some more\rq{}, as shown in (\ref{exAdditiveTypesHebrewMore2}). The same is found with Northern Qiang \mbox{\textit{tɕe}-},\il{Qiang, Northern} where this specific usage is restricted to the \isi{prospective} \isi{aspect} inflection; see (\ref{exAdditiveyTypesQiangMore}).

\begin{exe}
	\ex Modern Hebrew\label{exAdditiveTypesHebrewMore2}\\
	\gll Ba-boqer Rina rats-a ktsat. B-a-tsoharayim hi rats-a \textbf{ʕod}.\\
	at-morning Rina run.\textsc{pst}-3\textsc{sg}.\textsc{f} a\_bit at-\textsc{def}-noon 3\textsc{sg}.\textsc{f} run.\textsc{pst}-3\textsc{sg}.\textsc{f} still\\
	\glt \lq In the morning Rina ran a bit. At noon she ran \textbf{some more}.'
	\\\parencite[127]{Greenberg2012}\il{Hebrew, Modern|)}
	\pagebreak	
    \ex Northern Qiang\il{Qiang, Northern}\label{exAdditiveyTypesQiangMore}\\
	\gll Qɑ \textbf{tɕɑ}-naː.\\
	1\textsc{sg} still-sleep.\textsc{prosp}\\
	\glt \lq Iʼm still going to sleep (I want to sleep \textbf{some} \textbf{more}).'
	\\\parencite[169]{LaPollaHuang2003}
\end{exe}

Lastly,  the case of \ili{English} \textit{still} is related to both additions of the same kind and the incremental discourse strategies discusses above, in that this expression only has an additive use in list continuations of the type \lq … another, still another\rq{}, as in (\ref{exAdditiveEnglish}).

\begin{exe}
	\ex \ili{English}\label{exAdditiveEnglish}\\
Context: After having discussed two creative periods of the same writer.\\
	\textit{\textbf{Still} \textbf{a} \textbf{third} \textbf{type} of fiction appeared in the 1950s with Mrs. Miller's successful stories of Christmas at the homes of famous men.} (found online)%\footnote{\url{https://www.ncpedia.org/biography/miller-helen} (22 May, 2023).}
\end{exe}

\subsubsection{A closer look: Together with other additive markers} My discussion up to this point has focussed primarily on differences to other additive expressions. However, and as with many functional extensions of \textsc{still} expressions, the use I discuss here is repeatedly attested together with other linguistic items carrying a similar meaning. For instance, in (\ref{exAdditiveEwe}) \ili{Ewe} \mbox{\textit{ga}-} goes together with additive \textit{hã̂}, the two expressions sharing the same focus. Example (\ref{exAdditiveBCMS2}) from Serbian-Croatian-Bosnian\il{Serbian}\il{Croatian}\il{Bosnian} is another illustration.

\begin{exe}
	\ex \ili{Ewe}\label{exAdditiveEwe}\\
	\gll É-nye hatí-nyè, \textbf{ga}-\textbf{nyé} \textbf{xɔ̃́}-\textbf{nyè} \textbf{hã̂}.\\
3\textsc{sg}-\textsc{cop} colleague-1\textsc{sg} still-\textsc{cop} friend-1\textsc{sg} also\\
	\glt \lq Il est mon camarade mais aussi mon ami. [He's my colleague, \textbf{and also my friend}.]' (\cite[153]{Westermann1905}, glosses added)

	\ex Serbian-Croatian-Bosnian\il{Serbian}\il{Croatian}\il{Bosnian}\label{exAdditiveBCMS2}\\
		\gll Pametna je devojka, a sad je \textbf{još} \textbf{i} \textbf{ljuta}.\\
	smart.\textsc{nom}.\textsc{sg}.\textsc{f} \textsc{cop}.3\textsc{sg} girl(\textsc{f}).\textsc{nom}.\textsc{sg} and now \textsc{cop}.3\textsc{sg} still also angry.\textsc{nom}.\textsc{sg}.\textsc{f}\\
	\glt \lq She's a smart girl, and now \textbf{an angry one}, \textbf{too}.'
	\\(found online, glosses added)%\footnote{\url{https://glosbe.com/hr/sr/pametan} (24 March, 2022).}
\end{exe}

\il{Trió|(}Trió \mbox{=\textit{nkërë}} in additive function is sometimes encountered with the clausal \isi{connective} \mbox{\textit{se}(\textit{h})\textit{ke}(\textit{n})} \lq also, likewise\rq{ }as its host. Judging from the available examples, this usage stresses the notions of continuity and parallelism, and it straddles the boundary to constituent coordination.\is{coordination}

\begin{exe}
	\ex Trió\label{exAdditiveTrio}\\
	\gll Kure	menu=tao=ken	t-ee-se			nërë,	Waruku,	i-nmuku-pisi
	\textbf{seke}=\textbf{nkërë} t-ee-see.\\
	pretty paint=\textsc{loc}=\textsc{cont} \textsc{rem}.\textsc{pst}-\textsc{cop}-\textsc{rem}.\textsc{pst} 3.\textsc{anaph}.\textsc{anim} W. 3-son.\textsc{poss}-\textsc{dim} also=still \textsc{rem}.\textsc{pst}-\textsc{cop}-\textsc{rem}.\textsc{pst}\\
	\glt \lq Waruku was beautifully painted, \textbf{and so }was her little son.\rq{}
	\\\parencite[451]{Meira1999}	
\end{exe}\il{Trió|)}

The same superficial redundancy is also attested with the \lq another, more\rq{ }reading of \textsc{still} expressions.\il{Lengua, Southern|(} For example, in (\ref{exAdditiveEnxetRedundant}) Southern Lengua \textit{makham} goes together with \textit{pók} \lq another\rq{}. More research on usage patterns is needed to disentangle the individual and joint contribution of each item in such instances.

\begin{exe}
	\ex Southern Lengua\label{exAdditiveEnxetRedundant}\\
	\gll Ap-tép-ekx-eyk=axta \textbf{makham} \textbf{pók} wesse\rq{} neptámen=axta.\\
	\textsc{m}-emerge-\textsc{iter}-\textsc{decl}=\textsc{pst} still other.\textsc{m} leader(\textsc{m}) after.\textsc{m}=\textsc{pst}\\
	\glt \lq \textbf{Another} leader rose up after him.\rq{ }\parencite[462]{Elliot2021}
\end{exe}\il{Lengua, Southern|)}

\subsubsection{A closer look: Additivity and repetitions} Earlier I pointed out that an additive use of a \textsc{still} expression, together with an event quantifier \lq time(s)\rq{}, can give rise to a repetitive event construal \lq again\rq{}. This is not the only similarity between the two domains, as has been repeatedly observed in the pertinent literature, both in regards to \textsc{still} expressions and to items which do not have this phasal polarity function.\footnote{See, for instance, \textcite{Donazzan2008}, \textcite{Evans1995Mayali}, \textcite[162–163]{MosegaardHansen2008}, \textcite[108–113]{Huang2008}, \textcite{Lichtenberk1991}, \textcite{Liu2000}, \textcite[528–532]{Maslova2003}, \textcite[461]{McGregor1990},  \textcite{TovenaDonazzan2008}, \textcite{Yeh1998} and \textcite{Zhang2017}.} In broad strokes, additivity often goes along with the recurrence of a similar situation across different times and/or sets of participants, similar to how \lq again\rq{ }involves the iteration or restoration of a previous state-of-affairs.\footnote{Out of the twenty-nine expressions in \Cref{tableAdditive}, ten have an iterative and/or restitutive use. These are Classical Nahuatl\il{Nahuatl, Classical} \textit{oc}, \ili{Ewe} \mbox{\textit{ga}-}, \ili{French} \textit{encore}, \ili{Gooniyandi} \mbox{=\textit{nyali}}, \ili{Kaba} \textit{bbáy}, \ili{Ket} \textit{hāj}, Mandarin Chinese\il{Chinese, Mandarin} \textit{hái,} \ili{Paiwan} \mbox{=\textit{anan}}, \ili{Saisiyat} \textit{nahan}, and Southern Lengua\il{Lengua, Southern} \textit{makham}.} To give an example, in (\ref{exAdditiveKolyma3}) an identical act is performed by the same participants on two different days, thus making an additive and iterative interpretation virtually indistinguishable.\il{Yukaghir, Southern}

\begin{exe}
	\ex Southern Yukaghir\il{Yukaghir, Southern}\label{exAdditiveKolyma3}\\
	\gll D\rq{}e taŋ jeklie āj ekr-īl\rq{}i \textbf{sobenn\rq{}i} \textbf{āj} \textbf{ejr-īl\rq{}i}.\\
	\textsc{dm} that behind still walk-\textsc{intr}.1\textsc{pl} today still walk-\textsc{intr}.1\textsc{pl}\\
	\glt \lq Well, we walked beforehand, too, and \textbf{we walked again today}.\rq{}
	\\\parencite[529]{Maslova2003}
\end{exe}

\il{Gurindji|(}In Gurindji, the notional link between the two functions is reflected in morphology. Thus, the \textsc{still} expression \mbox{=\textit{rni}} forms part of a complex clitic \mbox{=\textit{rningan}}, the source of the second syllable of which is obscure. Importantly, \mbox{=\textit{rningan}} is specialised for marking repetitions and additivity. This is illustrated in (\ref{exAdditiveGurindji}).

\begin{exe}
	\ex \label{exAdditiveGurindji}
	\begin{xlist}
		\exi{}Gurindji
		\ex
		\gll Karu-ngku wumara=ma jakarr ma-ni=\textbf{rni}-\textbf{ngan}.\\
	child-\textsc{erg} money=\textsc{top} cover get-\textsc{pst}=still-?\\
	\glt \lq A child covered up the money \textbf{again}.\rq{}
	\ex Context: Someone else has fallen.\label{exAdditiveGurindjiB}\\
	\gll Karu=ma ngaja wani-nyana=\textbf{rni}-\textbf{ngan}.\\
	child=\textsc{top} \textsc{admonitive} fall-\textsc{prs}=still-?\\
	\glt \lq The child might fall \textbf{too}.' \parencite[6, 9]{McConvell1983}
	\end{xlist}
\end{exe}


While examples (\ref{exAdditiveParallelismKolyma2}, \ref{exAdditiveGurindjiB}) feature \lq in addition, also\rq{ }readings,\il{Gurindji|)} the same conceptual overlap can be observed between iterative event construals and the \lq another, more\rq{ }type of additivity.\il{Saisiyat|(} For instance, the Saisiyat example (\ref{exAdditiveSaisiyat}) can be read as iterative \lq see some boy coming again\rq{}, or as involving the addition of an entity of the same kind \lq see another boy coming\rq{}, with little to no difference in communicative import \parencite[108–109]{Huang2008}.

\begin{exe}
	\ex Saisiyat\label{exAdditiveSaisiyat}\\
	Context: From a rendition of the pear story. A boy with a goat has passed by. Now a boy on a bicycle is passing by the same location.\\
	\gll O: rima\rq{} ila hiza / kita’-en m-wa:i\rq{} ila \textbf{naehan} / \textbf{\rq{}aehae\rq}{} \textbf{kaːː} / \textbf{kamo\rq{}alay} / \textbf{kamamanraːan} / \rq{}ima papama\rq{} rayːː / kapapama\rq{}anːː\\
	\textsc{dm} \textsc{agt}.\textsc{foc}:go \textsc{cmpl} there {} see-\textsc{pat}.\textsc{foc} \textsc{agt}.\textsc{foc}-come \textsc{cmpl} still  {} one \textsc{nom} {} young\_man {} man {} \textsc{prog} \textsc{agt}.\textsc{foc}:ride \textsc{loc} {} vehicle\\
	\glt \lq Tamėn zŏu lė. Yòu kàndào lìngyigė nánháizi lái lė. Ta qízhė jĭaotàche. / (Off) they went. (Then I) see \textbf{another boy} coming; he was riding a bike.’ \parencite[108–109]{Huang2008}
\end{exe}\il{Saisiyat|)} 

\il{Lengua, Southern|(}A very similar case can be made for Southern Lengua \textit{makham} in (\ref{exAdditiveEnxetSur}). Here, the predicate \textit{apchaqha} can be interpreted as intransitive, with the \textsc{still} expression contributing iterativity \lq he killed again'. Alternatively, it can be read as involving an implied indefinite patient argument of the same kind \lq he killed another one\rq{ }(John Elliot, p.c.). I discuss more examples of this kind below.

\begin{exe}
	\ex Southern Lengua\label{exAdditiveEnxetSur}\\
	Context: A boy has fought off a demon. Another demon has appeared.\\
	\gll Natámen \textbf{apch}-\textbf{aqh}-\textbf{a} \textbf{makham} w-okm-ek=axta a-anet apk-ennap-ma tén.\\
	then \textsc{m}-kill-\textsc{nmlz}.\textsc{ipfv} still \textsc{f}.arrive-\textsc{terminative}-\textsc{decl}=\textsc{pst} \textsc{f}.\textsc{stat}-two \textsc{m}-kill\_many-\textsc{nmlz}.\textsc{pfv} then\\
	\glt \lq Then he killed it [\textbf{again}], coming to two of them he had battled, and…' \parencite[757]{Elliot2021}
\end{exe}\il{Lengua, Southern|)}

\il{Karbi, Hills|(}Lastly, in the case of Hills Karbi \mbox{-\textit{làng}} the connection between additivity and repetition is reflected in co-occurrence patterns. Thus, the \textsc{still} expression \mbox{-\textit{làng}} as \lq another, more\rq{ }often goes together with the repetitive suffix \mbox{-\textit{thū}}, as in (\ref{exAdditiveKarbi}).

\begin{exe}
	\ex Hills Karbi\label{exAdditiveKarbi}\\
	\gll Isī a-lám dō-\textbf{thū}-\textbf{làng}.\\
	one \textsc{poss}-matter exist-again-still\\
	\glt \lq There is still one \textbf{other} thing.' \parencite[336]{Konnerth2014}
\end{exe}\il{Karbi, Hills|)}

\subsubsection{Discussion}
I now turn to a discussion of the motivation underlying the coexpression of additivity and \textsc{still}. In the most abstract terms, both functions share a common denominator in the inclusion of entities, or existential quantification (\cite{vanderAuwera1993}; \cite{Doherty1973}; \cite{Liu2000}; \cite{Koenig1977}; \cite[531]{Maslova2003}; \cite[141]{Noelke1983}; among others). In other words, using an additive operator requires that the common ground contain at least one valid alternative to the focus denotation, similar to how the phasal polarity concept contributes the presupposition that its propositional argument held true during an interval immediately preceding topic time.\is{topic time} As I hinted above, a very similar parallel holds between additive and  \lq again\rq{ }uses of the same expressions. With that in mind, several links are conceivable and/or attested. In what follows, I briefly sketch out some of them and how they apply to the sample languages, without any claim to comprehensiveness.

A direct nexus between phasal polarity \textsc{still} and the additive use can be assumed, based on etymological and/or diachronic data, for \ili{French} \textit{encore}, \ili{German} \textit{noch}, Serbian-Croatian-Bosnian \il{Serbian}\il{Croatian}\il{Bosnian}\textit{još}, Tundra Nenets\il{Nenets, Tundra} \textit{təmna}, Tunisian Arabic \textit{māzāl},\il{Arabic, Tunisian} and, albeit more tentatively so, \ili{Thai} \textit{yaŋ}. As I pointed out above, there are recurrent contexts in which the two functions overlap, which suggests that similar instances may have served as diachronic bridges.\il{German|(} For instance, example (\ref{exAdditiveGermanSiebener}), repeated below, can be read as signalling the persistent\is{persistence} need for a certain piece, or as involving an addition to the list of parts used. A very similar case is found in the\il{Spanish|(} Spanish example (\ref{exAdditiveSpanishInformes}).

\begin{exe}
	\exr{exAdditiveGermanSiebener} German\is{persistence}\\
	Context: The speaker is assembling a wooden toy plane.\\
	\gll Also und {ach so}, und dann \textbf{brauch}-\textbf{e} \textbf{ich} \textbf{noch} eine Siebener-leiste.\\
	well and \textsc{interj} and then need-1\textsc{sg} 1\textsc{sg} still \textsc{indef}.\textsc{acc}.\textsc{sg}.\textsc{f} seven\_piece-bracket(\textsc{f})\\
	\glt \lq Well and oh yes, and then I \textbf{also need}/\textbf{still need} a 7-hole piece.\rq{ }(\cite[104]{Nederstigt2003}, glosses added)\il{German|)}

	\ex Spanish\label{exAdditiveSpanishInformes}\\
	\gll \textbf{Todavía} ten-go que entreg-ar dos informe-s.\\
	still have-1\textsc{sg} \textsc{subord} hand\_in-\textsc{inf} two report-\textsc{pl}\\
	\glt \lq I \textbf{still} have to hand in two reports / I still have to hand in two \textbf{more} reports.\rq{ }(\cite[214]{Bosque2016}, glosses added)
\end{exe}\il{Spanish|)}

As I also indicated earlier, such ambiguities are particularly pronounced when the persistent\is{persistence} duration of a situation is at stake,\il{German|(} such as in (\ref{exAdditiveWieLangeNoch}), repeated below. A comparable instance is found in the Old French\il{French, Old} example (\ref{exAdditiveOldFrenchRide}), except that time is here mediated by stretches of a path.

\begin{exe}
	\exr{exAdditiveWieLangeNoch} German\is{persistence}\\
	\gll \textbf{Wie} \textbf{lange} wird denn das \textbf{noch} dauer-n?\\
	how long \textsc{fut}.\textsc{aux}:3\textsc{sg} \textsc{dm} 3\textsc{sg}.\textsc{n} still last-\textsc{inf}\\
	\glt \lq For \textbf{how} \textbf{long} is this \textbf{still} going to last? / \textbf{How much longer} will this take?\rq{ }(Schnitzler, \textit{Leutnant Gustl}, cited in \cite[60]{Shetter1966}, glosses added)
    \il{German|)}
 
	\ex Old French,\il{French, Old} 13\textsuperscript{th} century\label{exAdditiveOldFrenchRide}\is{persistence}\\
	\textit{Et se vous baés orendroit a cevauchier vers la Petite Bretaingne, ceste autre voie vous i menra tout droit}.\\
	\lq And if you wish now to ride to Brittany, this other road will take you straight there.\rq{}
	\exi{}
	\gll  Et se vous volés \textbf{encore} \textbf{une} \textbf{piece} \textbf{cevauchier} pour veoir les merveilleuses aventures d-u roiaume de Logres, avoec moi poés cevauchier...\\
	and if 2\textsc{pl} want.2\textsc{pl} still \textsc{indef}.\textsc{sg}.\textsc{f} bit(\textsc{f}) ride.\textsc{inf} for see.\textsc{inf} \textsc{indef}.\textsc{pl}.\textsc{f} marvellous.\textsc{pl}.\textsc{f} adventure(\textsc{f}).\textsc{pl} of-\textsc{def}.\textsc{sg}.\textsc{m} realm(\textsc{m}) of L. with 1\textsc{sg}.\textsc{obj} can.2\textsc{pl} ride.\textsc{inf}\\
	\glt \lq And if you \textbf{still} \textbf{want} \textbf{to} \textbf{ride} \textbf{a} \textbf{bit} with me/if you \textbf{want} \textbf{to} \textbf{ride} \textbf{another} \textbf{bit} with me to see the marvelous adventures of the kingdom of Logres, you can ride with me...\rq{ }(\textit{Tristan en prose}, cited in \cite[163]{MosegaardHansen2008}, glosses added)
\end{exe}

\il{French|(}
While examples (\ref{exAdditiveGermanSiebener}, \ref{exAdditiveWieLangeNoch}) and (\ref{exAdditiveSpanishInformes}, \ref{exAdditiveOldFrenchRide}) involve contextual ambiguity, a more direct, metonymical transfer from the propositional to the textual\is{textuality}\is{proceduralisation} domain is also attested. Thus, in the oldest clear-cut attestation of French \textit{encore} as \lq in addition, also\rq{}, given in (\ref{exAddidiveOldFrenchKill}),\il{French, Old} the notions of \isi{persistence} and a possible \isi{discontinuation} are applied to a succession of related events instead of a single situation. Or, as \textcite[158]{MosegaardHansen2008} puts it, \lq\lq in spite of the \isi{telicity} of the verb \textit{ocire} [\lq kill\rq{}]\rq\rq, this example \lq\lq may be loosely understood as meaning that … the subject ... remains in a \lq{}killing-mode\rq{}{\rq\rq{}.\is{actionality} Lastly, against the background of these examples, the preference for markers such as \textit{noch} or \textit{encore} for an incremental discourse organisation is surely a synchronic reflection of the origins of their additive function.

\begin{exe}
	\ex Old French,\il{French, Old} ca. 1080\label{exAddidiveOldFrenchKill}\is{persistence}\\
\textit{Aprof li ad sa bronie desclose, // El cors li met tute l’enseigne bloie, // Que mort l’abat en une halte roche. //} \\
	\lq Afterwards, he tears open his coat of mail, He drives the entire blue ensign into his body, Killing him on a tall rock.'
	\exi{}\gll Sun cumpaignun Gerers \textbf{ocit} \textbf{uncore}, // E Berenger e guiun de Seint Antonie.\\
	\textsc{poss}.3\textsc{sg}.\textsc{m}:\textsc{sg} companion G. kill.3\textsc{sg} still {} and B. and G. of S.-A.\\
	\glt \lq He \textbf{also kills} his companion Gérier, and Berenger and Gui de Saint-Antoine.' (\textit{La chanson de Roland}, cited in \cite[158]{MosegaardHansen2008}, glosses added)
\end{exe}
\il{French|)}

As just hinted at, an extension from phasal polarity to marking additivity involves the well-known tendency for meaning to become increasingly based in the textual\is{textuality}\is{proceduralisation} domain (\Cref{sectionSemasiologicalChange}). The same is true of an extension from an iterative/restitutive function to additivity. The primacy of repetition can be firmly established for Mandarin Chinese\il{Chinese, Mandarin} \textit{hái}, which has its lexical source in a motion verb \textit{huan} \lq go back\rq{}. As laid out by \textcite{Yeh1998}, this item first developed into a restitutive marker, then acquired an iterative sense, and only centuries later came to signal \textsc{still}. Crucially, the additive sense arose before the phasal polarity one. The same is likely true for \ili{Ewe} \mbox{\textit{ga}-} which has a similar lexical source. Against this backdrop, \textcite{Yeh1998} points to cases like the Middle Chinese\il{Chinese, Middle}  example (\ref{exAdditiveMiddleChinese}) as historical bridges from iterativity to additivity. Under a more conservative reading, this example would have been understood as involving two similar and sequentially ordered events, but with a different topical\is{topic} argument. At the same time, the host clause of \textit{hái} can be understood as introducing an additional reason for the same conclusion as the first clause, namely the difficulty of the trip, and the presence of \textit{ji} \lq as well as\rq{ }likewise yields a bias towards an additive interpretation.

\begin{exe}
	\ex Middle Chinese\il{Chinese, Middle}\il{Chinese, Mandarin}\label{exAdditiveMiddleChinese} 6\textsuperscript{th}/7\textsuperscript{th} century\\
	\gll Ji shang qian li mu, \textbf{hai} jing jiu zhe hun.\\
	as\_well\_as hurt thousand mile eye still frighten nine turn soul\\
	\glt \lq Not just the view of the long trail makes you sad, the winding turns \textbf{also} frighten you.\rq{ }\parencite[245]{Yeh1998}
\end{exe}

Whereas (\ref{exAdditiveMiddleChinese}) involves a pathway from \lq again\rq{ }to \lq also, in addition\rq{}, a similar convergence in meaning can be found for additivity plus \isi{identity} of kind.\il{Saisiyat|(} For instance, in the context of (\ref{exAdditiveSaisiyat}), repeated below, 
an iterative interpretation \lq again I see a boy coming\rq{ }and an additive reading \lq I see another boy\rq{ }overlap considerably, with little to no difference in communicative terms.

\begin{exe}
	\exr{exAdditiveSaisiyat} Saisiyat\\
	Context: From a rendition of the pear story. A boy with a goat has passed by. Now a boy on a bicycle is passing by the same location.\\
	\gll O: rima\rq{} ila hiza / kita’-en m-wa:i\rq{} ila \textbf{naehan} / \textbf{\rq{}aehae\rq}{} \textbf{kaːː} / \textbf{kamo\rq{}alay} / \textbf{kamamanraːan} / \rq{}ima papama\rq{} rayːː / kapapama\rq{}anːː\\
	\textsc{dm} \textsc{agt}.\textsc{foc}:go \textsc{cmpl} there {} see-\textsc{pat}.\textsc{foc} \textsc{agt}.\textsc{foc}-come \textsc{cmpl} still  {} one \textsc{nom} {} young\_man {} man {} \textsc{prog} \textsc{agt}.\textsc{foc}:ride \textsc{loc} {} vehicle\\
	\glt \lq Tamėn zŏu lė. Yòu kàndào lìngyigė nánháizi lái lė. Ta qízhė jĭaotàche. / (Off) they went. (Then I) see \textbf{another boy} coming; he was riding a bike.’ \parencite[108–109]{Huang2008}
\end{exe}\il{Saisiyat|)} 

\il{French|(}
The Old French\il{French, Old} example (\ref{exAdditiveOldFrenchRespite}) is another illustration. At the time of this attestation, the expression \textit{encore} had already acquired an iterative use, whereas the \lq another, more\rq{ }function was yet to be established. As \textcite[163]{MosegaardHansen2008} points out, attestations like  (\ref{exAdditiveOldFrenchRespite}) figure prominently among the critical contexts in that development.

\begin{exe}
	\ex Old French,\il{French, Old} early 13\textsuperscript{th} century\label{exAdditiveOldFrenchRespite}\\
	Context: The barons have given the emperor a respite. Now the date has passed.\\
	\gll Et li empereres \textbf{redemanda} \textbf{encore} \textbf{un} \textbf{respit}, et on li donna.\\
	and \textsc{def}.\textsc{sg}.\textsc{m} emperor(\textsc{m}) ask\_again.\textsc{pst}.\textsc{pfv}.3\textsc{sg} still \textsc{indef}.\textsc{sg}.\textsc{m} respite(\textsc{m}) and \textsc{impr} 3\textsc{sg}.\textsc{dat} give.\textsc{pst}.\textsc{pfv}.3\textsc{sg}\\
	\glt \lq And the emperor \textbf{asked for a respite again}/\textbf{asked for another respite}, and it was given him.\rq{ }(de Clari, \textit{La conqueste de Constantinople}, cited in \cite[163–164]{MosegaardHansen2008}, glosses added)
\end{exe}
\il{French|)}

Lastly, it is conceivable that the function of a given expression as an exponent of \textsc{still} may go back to an element with an incremental meaning, via \lq Verb more > still\rq{}. Thus, Northern Qiang \mbox{\textit{tɕe}-}\il{Qiang, Northern} is suspiciously similar to the superlative marker \mbox{\textit{tɕi}-}, both in its segmental make-up and in the linear position it occupies on the predicate, except that the superlative marker does not undergo vowel harmony. This is shown in (\ref{exAdditiveQiang}). In the same vein, \textcite[88]{HeineEtAl1984} indicate that in certain varieties of Southern \ili{Kikongo} (Bantu, not in my sample) a verb with the meaning of \lq add, put more\rq{ }has grammaticalised\is{grammaticalisation} into a \textsc{still} expression.

\begin{exe}
	\ex \label{exAdditiveQiang}
	\begin{xlist}
		\exi{}Northern Qiang\il{Qiang, Northern}
		\ex
		\gll Nə-dʐə-m theː \textbf{tɕɑ}-n.\\
	sleep-able-\textsc{nom} 3\textsc{sg} still-sleep\\
	\glt \lq S/he who likes to sleep late is still sleeping.'

	\ex\il{Qiang, Northern}
	\gll \textbf{tɕi}-χtʂa\\
	most-small\\
	\glt \lq smallest\rq{ }\parencite[214, 228]{LaPollaHuang2003}
	\end{xlist}
\end{exe}

\subsection{Further-to}\label{sectionFurtherTo}\is{precedence|(}
\subsubsection{Introduction}\il{German|(}
In this subsection, I discuss what \textcite{Klein2018} calls the \lq\lq further-to\rq\rq{ }use.\footnote{\textcite{Nederstigt2003} speaks of \lq\lq additive before a turning point\rq\rq{ }and \textcite{Vandeweghe1984} subsumes this use under \lq\lq antiterminativeit … met externe relevantiecriterium\rq\rq{ }[antiterminativity with an external criterium of relevance].} In this function a \textsc{still} expression marks a combination of additive and phasal notions. This is illustrated in (\ref{exFurtherToIntro1}), where German \textit{noch} depicts taking a shower as an additional act within the current frame of experience, which is defined by the preceding soccer practice. At the same time, using \textit{noch} hints at the subsequent transition to a new section of the day, which is made explicit in the following clause.

\begin{exe}
	\ex German\label{exFurtherToIntro1}\\
	Context: I have just come home from soccer practice. It is fairly late.\\
	\gll \textbf{Ich} \textbf{dusch'} \textbf{noch}. Dann gibt-'s Abend-essen.\\
	1\textsc{sg} take\_shower.1\textsc{sg} still then \textsc{exist}.3\textsc{sg}-3\textsc{sg}.\textsc{n} evening-meal(\textsc{n})\\
	\glt \lq \textbf{I'm just taking a quick shower}. Dinner will be just after.' \parencite[16]{Beck2019}
\end{exe}

\il{Karbi, Hills|(}
The further-to use differs from the more general additive one I discuss in \Cref{sectionAdditive} in that the relevant statements \lq\lq can be uttered out of the blue and allow for accommodation of the additional item\rq\rq{ }(\cite[1846]{Umbach2012}; also see \cite{Klein2018}; \cite[101–102]{Nederstigt2003}; \cite{Vandeweghe1984}). Thus, example (\ref{exFurtherToIntro1}) could be produced right upon walking through the door and being met by hungry faces, without the need for any further discursive embedding. Example (\ref{exFurtherToKarbi1}) from Hills Karbi displays a parallel case.

\begin{exe}
	\ex Hills Karbi\\
	Context: There is a plan to go to the market.\label{exFurtherToKarbi1}\\
	\gll Rí chersām-dām-\textbf{làng}.\\
	hand wash-go-still\\
	\glt \lq I'm \textbf{just} gonna go wash my hands real quick (\textbf{and then} we can go).' \parencite[299–300]{Konnerth2014}
\end{exe}


\subsubsection{Distribution in the sample}
The further-to use is attested for the two sample languages and expressions illustrated in (\ref{exFurtherToIntro1}, \ref{exFurtherToKarbi1}), German \textit{noch} and Hills Karbi \mbox{-\textit{làng}} (\appsref{appendixGermanFurtherTo}, \ref{appendixHillsKarbiFurtherTo}).\footnote{Outside of the sample, the further-to use has been described for \ili{Dutch} \textit{nog} \parencite{Vandeweghe1984} and for \ili{Hungarian} \textit{még} \parencite{CsirmazSlade2020}.} That said, in \Cref{sectionFirst} I discuss a use of \textsc{still} expressions as markers of precedence \lq first, for now\rq{}. And as has been seen in (\ref{exFurtherToIntro1}, \ref{exFurtherToKarbi1}), the further-to use also tends to invite a contrast with subsequent times. With this in mind, upon examination of more data some of the items discussed in \Cref{sectionFirst} may turn out to have the further-to use rather than a purely relational one.

\subsubsection{A closer look and discussion}\is{tense|(}\is{aspect|(} I now turn to a closer look at the further-to use and a discussion of its motivation. While examples (\ref{exFurtherToIntro1}) and (\ref{exFurtherToKarbi1}) both feature \isi{prospective} statements, this use is also found in other temporal-aspectual configurations. Example (\ref{exFurtherToGerman2}) is an illustration featuring the present tense in a generic reading. For an example in the past, see (\ref{exFurtherToHalbzeit}) below. In the case of German \textit{noch}, one could subsume the future \lq eventually\rq{ }use I discuss in \Cref{sectionProspective} here, as well.\is{prospective}

\begin{exe}
	\ex German\label{exFurtherToGerman2}\\
	\textit{Früher ist's mir immer sonderbar vorgekommen, dass die Leut', die verurteilt sind,}
	\\\lq I used to find it strange that people that have been sentenced [to death]\rq
	\exi{}\gll  in der Früh \textbf{noch} \textbf{ihren} \textbf{Kaffee}  \textbf{trink}-\textbf{en} \textbf{und} \textbf{ihr} \textbf{Zigarr}-\textbf{l} \textbf{rauch}-\textbf{en}.\\		
	in \textsc{def}.\textsc{dat}.\textsc{sg}.\textsc{f} morning(\textsc{f}) still \textsc{poss}.3\textsc{pl}:\textsc{acc}.\textsc{sg}.\textsc{m} coffee(\textsc{m}) drink-3\textsc{pl} and \textsc{poss}.3\textsc{pl}:\textsc{acc}.\textsc{sg}.\textsc{n} cigarre-\textsc{dim}(\textsc{n}) smoke-3\textsc{pl}\\
\glt \lq \textbf{would have coffee and a cigar in the morning} [before getting executed].' (Schnitzler, \textit{Leutnant Gustl}, cited in \cite[58]{Shetter1966}, glosses added)
\end{exe}\is{tense|)}\is{aspect|)}

As has been repeatedly been observed in the literature on German and \ili{Dutch} (e.g. \cite{Klein2018}; \cite{Nederstigt2003}; \cite{Shetter1966}; \cite{Vandeweghe1984}), there is a noteworthy similarity between the further-to use and phasal polarity \textsc{still}, in that both functions involve a segmentation of the time-line based on \isi{persistence} vs. a possible discontinuation.\is{discontinuation} The main difference is that in the case of \textsc{still} this perspective is directly based on the situation described in its propositional argument, whereas in the further-to use the \lq it's not over yet\rq{ }meaning applies to a larger frame of experience, or a series of contiguous events. \Cref{figureFurtherTo} is a graphic illustration, based on (\ref{exFurtherToIntro1}).

\begin{figure}[htb]
	\centering
	\begin{subfigure}[b]{0.48\linewidth}\is{topic time}
		\centering
		\begin{tikzpicture}[node distance = 0pt]
			\node[mynode, text width=4.5*\hoehe, fill=cyan, very near start] (spain){Situation};
			\node[mynode, fill=none, text width=1.5*\hoehe, right= of spain] (france){(\small{$\Diamond$}\neg Sit.)};	
			\draw[-, densely dashed] ($(spain.north east)+ (0,0) $) to ($(spain.south east)+ (0,-0.5*\hoehe) $);
			\draw[-, densely dashed] ($(spain.north east)+ (-1*\hoehe,0) $) to ($(spain.south east)+ (-1*\hoehe,-0.5*\hoehe) $) node [below, align=center, label distance=0, xshift=0.5*\hoehe] {\strut{}Topic\\time\strut};
			\draw[-latex, line width=0.5pt]  (spain.south west) to  ($(france.south east)+(3ex,0)$) node [right] {t};
			\node[below, align=left,label distance=0, anchor=north west] at ($(spain.south west)+(0,-0.5*\hoehe)$) {\strut{}Prior\\runtime\strut};
		\end{tikzpicture}
	\subcaption{\textsc{still}}	
	\end{subfigure}
\begin{subfigure}[b]{0.48\linewidth}
		\centering
		\begin{tikzpicture}[node distance = 0pt]
				\node[mynode, text width=\superbreit+0.5*\hoehe, fill=cyan, very near start, fill opacity=0.5, text opacity=1] (spain){Soccer time};
%								\node[mynode, text width=\superbreit, fill=cyan, very near start, fill opacity=0.5, text opacity=1] (spain){Soccer time};
			\node[mynode, fill=cyan, minimum width=\hoehe, anchor=east] (france)
		at (spain.east)	{\small{\faShower}};	
			\node[mynode, fill=none, text width=1.5*\hoehe, right= of spain] (series){Dinner};	
			\draw[-, densely dashed] ($(france.north west)+ (-0.2pt,0) $) to ($(france.south west)+ (-0.2pt,-0.5*\hoehe) $);
			\draw[-, densely dashed] ($(france.north west)+ (-1*\hoehe,0) $) to ($(france.south west)+ (-1*\hoehe,-0.5*\hoehe) $) node 	[below, align=center, label distance=0, xshift=0.5*\hoehe] {\strut{}Topic\\time\strut};
			\draw[-] (france.north east) to (france.south east);

			\draw[-latex, line width=0.5pt]  (spain.south west) to  ($(series.south east)+(3ex,0)$) node [right] {t};
		\node[below, align=left,label distance=0, anchor=north west] at ($(spain.south west)+(0,-0.5*\hoehe)$) {\strut{}Prior course\\of events\strut};
		\end{tikzpicture}
		\subcaption{Example (\ref{exFurtherToIntro1})}
	\end{subfigure}	
	\caption{Schematic illustration of the further-to use
	\label{figureFurtherTo}}	
\end{figure}

In more applied terms, there are at least two functions and usage patterns that likely motivate the further-to use. First, both German \textit{noch} and Hills Karbi \mbox{-\textit{làng}} have a general additive use. As I discuss in \Cref{sectionAdditive}, there is a cross-linguistic tendency for this functional extension to preserve the \lq\lq antiterminative\rq\rq{ }\parencite{Vandeweghe1984} core of phasal polarity \textsc{still}. In the narrative genre of texts, this tendency often manifests itself as the insertion of one last event before a contextually given turning point. An illustration of German \textit{noch} in such an embedding is given in (\ref{exFurtherToKafka}), where the slamming shut of the door forestalls the end of the narrative episode.

\begin{exe}
	\ex German\label{exFurtherToKafka}\\
	\textit{Da gab ihm der Vater von hinten einen jetzt wahrhaftig erlösenden starken Stoß, und er flog, heftig blutend, weit in sein Zimmer hinein.}\\
	\lq Then his father gave him a strong and liberating push from behind, and he scurried, bleeding heavily, far into his room.'
	\exi{}\gll \textbf{Die} \textbf{Tür} \textbf{wurde} \textbf{noch} \textbf{mit} \textbf{dem} \textbf{Stock} \textbf{zugeschlagen}, dann wurde es endlich still.\\
	\textsc{def}.\textsc{nom}.\textsc{sg}.\textsc{f} door(\textsc{f}) become.\textsc{pst}.3\textsc{sg} still with \textsc{def}.\textsc{dat}.\textsc{sg}.\textsc{m} cane(\textsc{m}) slam\_shut.\textsc{ptcp} then become.3\textsc{sg} 3\textsc{sg}.\textsc{n} finally quiet\\
	\glt \lq \textbf{The door was slammed shut with the cane}, and finally all was quiet.'
	\exi{}\textit{Erst in der Abenddämmerung erwachte Gregor aus seinem schweren ohnmachtähnlichen Schlaf}.\\
\lq It was not until the dawn of evening that Gregor awoke from a deep and swoon-like sleep.' (Kafka, \textit{Die Verwandlung}, glosses added)
\end{exe}

A similar case is shown in (\ref{exFurtherToKarbi2}) for Hills Karbi. Here, \mbox{-\textit{làng}} marks the quick tea stop as one last step in the initial stage of the trip, after which the speaker and company set out for the main part of their journey.\pagebreak

\begin{exe}
	\ex Hills Karbi\label{exFurtherToKarbi2}\\
	Context: From the beginning of the recollection of a trip.\\
	\textit{…là elilitūm ajirpò alànglì Yu’éspensi kevàng Kavòn Kavòn aphāntā chepōnlò}\\
	\lq … this friend of ours, he who has come from the US, Kavon, Kavon we also took along with us.'
	\exi{}\gll sì ladāk=pen dàm-lò Dimápúr vùr-pōn \textbf{sá} \textbf{jùn}-\textbf{pōn}-\textbf{làng}. \\
	therefore here=from go-\textsc{rl} D. drop\_in-in\_passing tea drink-in\_passing-still\\
	\glt \lq And then, from here we went, we stopped by in Dimapur [about a quarter of the way in] \textbf{and just had tea}.'
	\exi{}
	\textit{lasì bají sirkē-bāk apòrpe=si puthōt dàmthūlò Kohìmàán továr kēkò.}\\
	\lq At nine o’clock, from about that time, we again went, up to Kohima, the road is winding a lot.' \parencite[361–362]{KonnerthTisso2018}
\end{exe}

There is a fluent transition between the pattern in (\ref{exFurtherToKafka}, \ref{exFurtherToKarbi2}) and the further-to use in (\ref{exFurtherToHalbzeit}). Here, the preceding clause depicts the lack of any relevant occurrence and the subsequent addition of an eventive predicate therefore requires accommodation as the proverbial \lq\lq one last thing\rq\rq{}.

\begin{exe}
	\ex German\label{exFurtherToHalbzeit}\\
	 \textit{Die erste Halbzeit verlief torlos,}\\
	\lq the first half-time passed by without any goals,\rq{}\\
	\gll Aber in der 2. Halb-zeit \textbf{schoss} \textbf{Bayern} \textbf{München} \textbf{noch} \textbf{zwei} \textbf{Tor}-\textbf{e}.\\
	but in \textsc{def}.\textsc{dat}.\textsc{sg}.\textsc{f} 2nd half-time(\textsc{f}) shoot.\textsc{pst}.3\textsc{sg} Bayern Munich still two goal-\textsc{pl}\\
	\glt \lq but in the second half-time \textbf{Bayern} \textbf{Munich} (\textbf{eventually}) \textbf{did} \textbf{score} \textbf{twice}.' (\cite[127]{HoepelmanRohrer1980}, glosses added)
\end{exe}

As I hinted earlier, German \textit{noch}, but not Hills Karbi \mbox{-\textit{làng}}, has an \lq eventually\rq{ }function in future-oriented contexts like (\ref{exFurtherToKricket}).\is{prospective} This latter use shares a common denominator with the use I discuss here in the element of \lq\lq{}before all is over\rq\rq{ }\parencite[s.v. \textit{yet}]{OED2022}.\is{modality|(} As I argue in \Cref{sectionProspective},\is{prospective} the most likely source for the \lq will eventually\rq{ }function lies in contexts in which a \textsc{still} expression associates with a modal proposition \lq can/must still …\rq{}, giving rise to a reanalysis as \lq can/must … yet (before it's over)\rq{}. A similar case can be made for the further-to use. In other words, it is conceivable that an instance like (\ref{exFurtherToKarbiBridge}) is ambiguous between a reading of a persistent\is{persistence} necessity and an interpretation involving the need to perform one more task before transitioning into a new phase of life.

\begin{exe}
	\ex German\label{exFurtherToKricket}\is{prospective}\is{persistence}\\
	\gll Wir mach-en \textbf{noch} einen gut-en Kricket-spieler aus ihm.\\
	1\textsc{pl} make-1\textsc{pl} still \textsc{indef}.\textsc{acc}.\textsc{sg}.\textsc{m} good-\textsc{acc}.\textsc{sg}.\textsc{m} cricket-player(\textsc{m}) of 3\textsc{sg}.\textsc{dat}.\textsc{m}\\
	\glt \lq Weʼll make a good cricketer of him \textbf{yet}.\rq{ }(\cite[197]{Koenig1977}, glosses added)

	\ex Hills Karbi\label{exFurtherToKarbiBridge}\is{persistence}\\
	Context: Two children, adopted by tigers, want to live with their biological father, a king.\\
	\gll Nè ne-pēi ne-pō aphān \textbf{che}-\textbf{arjū}-\textbf{dām}-\textbf{làng}.\\
	1\textsc{excl} \textsc{poss}.1\textsc{excl}-mother \textsc{poss}.1\textsc{excl}-father \textsc{non}.\textsc{subj} \textsc{refl}/\textsc{recp}-ask-go-still\\
	\glt \lq \textbf{We still need to ask} our mother and father (for permission to stay with the king).’ \parencite[154]{KonnerthTisso2018}
\end{exe}\is{modality|)}

It is probably a combination of the patterns just outlined that led to the emergence of the further-to use in the history of German \textit{noch}, Hills Karbi \mbox{-\textit{làng}} and other items with this function.\is{precedence|)}\il{German|)}\il{Karbi, Hills|)}

\subsection{Scalar additive}\is{scale|(}
\label{sectionScalarAdditive}
\subsubsection{Introduction} In this section, I discuss scalar additive uses of \textsc{still} expressions. As I lay out in more detail in \Cref{sectionScales}, scalar additive operators signal that the focus denotation yields a more informative proposition than all alternatives under consideration. Example (\ref{exScalarAdditivesIntro1}) is an illustration. \il{Spanish|(}Here, Spanish \textit{aún} indicates that the denotation of the focus \lq the winds and the water\rq{ }yields a stronger statement about the subject's powers than all accessible alternatives.

\begin{exe}
	\ex Spanish\label{exScalarAdditivesIntro1}\\
	\gll ¿Quién es éste, que mand-a \textbf{aun} \textup{[}\textbf{a} \textbf{los} \textbf{viento}-\textbf{s} \textbf{y} \textbf{a}-\textbf{l} \textbf{agua}\textup{]\textsubscript{\textsc{foc}}} y le obedec-en?\\
	\phantom{¿}who \textsc{cop}.3\textsc{sg} \textsc{prox}.\textsc{sg}.\textsc{m} \textsc{subord} command-3\textsc{sg} still \phantom{[}\textsc{acc} \textsc{def}.\textsc{pl}.\textsc{m} wind(\textsc{m})-\textsc{pl} and \textsc{acc}-\textsc{def}.\textsc{sg} water and 3\textsc{sg}.\textsc{dat}.\textsc{m} obey-3\textsc{pl}\\
	\glt \lq Who is this? He commands \textbf{even the winds and the water}, and they obey him.' (Luke 8:25, \textit{La biblia al día}, cited in \cite[3]{GastvanderAuwera2011})
\end{exe}

In (\ref{exScalarAdditivesIntro1}) there is a positive correlation between a scale of difficulty and the strength of the overall proposition. In (\ref{exScalarAdditivesIntro2}), on the other hand, this relationship is inversed. Thus, merely mentioning the disobendient's secret acts is less involved than observing or performing them. Yet it yields a more informative answer to the question of how despicable these acts are.

\begin{exe}
	\ex Spanish\label{exScalarAdditivesIntro2}\\
	\gll ... d-a vergüenza \textbf{aun} \textup{[}\textbf{mencion}-\textbf{ar}\textup{]\textsubscript{\textsc{foc}}} lo que los desobediente-s hac-en en secreto.\\
	{} give-3\textsc{sg} shame still \phantom{[}mention-\textsc{inf} 3\textsc{sg}.\textsc{n} \textsc{subord} \textsc{def}.\textsc{pl}.\textsc{m} disobedient-\textsc{pl} do-3\textsc{pl} in secret\\
	\glt \lq It is shameful \textbf{to even mention} what the disobedient do in secret.'
	\\(Eph. 5:12, \textit{La biblia al día}, cited in \cite[2]{GastvanderAuwera2011})
\end{exe}

Scalar additives can be sensitive to the direction that inferences run. Following \citeauthor{GastvanderAuwera2011} (\citeyear{GastvanderAuwera2011}, \citeyear{GastvanderAuwera2013}), I refer to markers that are only used in contexts like (\ref{exScalarAdditivesIntro1}) as \textsc{beyond} operators, and to those that require contexts like (\ref{exScalarAdditivesIntro2}) as \textsc{beneath} operators. Items like Spanish\il{Spanish|)} \textit{aún} or \ili{English} \textit{even}, which are compatible with both types of context, are termed \textsc{universal} scalar additives. Note that I discuss scalar additive uses that are specialised for temporal foci (\lq as early/late as, as far removed as\rq{}) separately in \Cref{sectionTimeScalar}. Lastly, for a discussion of \lq even more\rq{ }in comparisons of inequality,\is{comparison} see \Cref{sectionComparisons}.

\subsubsection{Distribution in the sample}
\Cref{tableScalarAdditive} lists the expressions in the sample that are attested as scalar additive operators. On examination, several things stand out. First, with only four expressions this use is strikingly rare, given the high number of expressions attested in a plain additive use (\Cref{sectionAdditive}).  Secondly, all instances stem from Eurasia and, as this geographic distribution already suggests, they involve invariant small words. One item, \il{Spanish|(}Spanish \textit{aún}, not only serves as a scalar additive by itself, but also forms part of a fixed polymorphemic collocation. The latter is not only a structural outlier, but also the only \textsc{beneath} operator in \Cref{tableScalarAdditive}, whereas all other instances involve either \textsc{beyond} operators or a \textsc{universal} scalar additive sense.

\begin{table}
	\caption{Scalar additive uses\label{tableScalarAdditive}}
	\small
	\begin{tabular}{llllll}
		\lsptoprule
		Macro-area & Language & Expression & Collocation & Type & Appendix\\
		\midrule
		
		Eurasia & \ili{German} & \textit{noch} & n/a & \textsc{beyond} & \ref{appendixGermanScalarAdditive}\\
		& \ili{Ket} & \textit{hāj}\footnote{Inclusion is tentative: only one example in the data.} & n/a &  \textsc{beyond}(?) & \ref{appendixKetScalarAdditive}\\
		& Spanish\il{Spanish|)} & \textit{aún} & n/a & \textsc{universal} & \ref{appendixSpanishAunScalarAdditive} \\
		& & \textit{aún} & \textit{aun}\textit{que sea} & \textsc{beneath} & \ref{appendixSpanishAunqueSea}\\

		& & & \lq even if it were\rq{}\\
		& \ili{Thai} & \textit{yaŋ} & n/a & \textsc{beyond} & \ref{appendixThaiScalarAdditive}\\
		\lspbottomrule		
		\end{tabular}
\end{table}

\subsubsection{A closer look} I now turn to a closer look at the scalar additive use. To begin with, the inclusion of \ili{Ket} \textit{hāj} as a scalar additive is somewhat tentative. There is only one relevant example in the data, given in (\ref{exScalarAdditiveKet}), and it is possible that the scalar reading is a contextual inference of what is essentially an \lq also\rq{ }use.

\begin{exe}
	\ex \ili{Ket}\label{exScalarAdditiveKet}\\
	\gll Bu-ŋ-s-ɔʁɔ-dāsʲ, bū \textbf{kɛˀt} \textbf{hāj} \textbf{du}-\textbf{ɣa}-\textbf{jɛj}.\\
	\textsc{subj}.3-\textsc{th}-\textsc{prs}-search\_for-when 3\textsc{sg} person still \textsc{subj}.3-\textsc{obj}.3\textsc{m}-kill\\
	\glt \lq When he looks, he can \textbf{even kill a man}.' \parencite[173]{Nefedov2015}
\end{exe}

\il{Spanish|(}\is{subordination|(}Spanish \textit{aún} as \lq even\rq{}, which was seen in (\ref{exScalarAdditivesIntro1}, \ref{exScalarAdditivesIntro2}), underlies a multitude of \isi{concessive} constructions, which I examine in more detail in \Cref{sectionConcessive}. For instance, in (\ref{exScalarAdditiveAunque}) \textit{aún} occurs in collocation with the subordinator \textit{que} and the subjunctive mood,\is{mood} together marking the apodosis of a \isi{concessive} \isi{conditional} sentence \lq even if\rq{}. Below I discuss how such concessive conditionals, in turn, lie at the heart of the second Spanish item in \Cref{tableScalarAdditive}, the morphologically complex \textsc{beneath} operator \textit{aunque sea} \lq so much as\rq{}, lit. \lq{}even if it were\rq{}, which is illustrated in (\ref{exScalarAdditiveAunqueSea1}).

\begin{exe}
	\ex Spanish\label{exScalarAdditiveAunque}\is{concessive}\\
	\gll \textbf{Aun}-\textbf{que} te qued-es sin dorm-ir, h-as de prepar-ar bien este examen.\\
	still-\textsc{subord} \textsc{refl}.2\textsc{sg} remain-\textsc{sbjv}.2\textsc{sg} without sleep-\textsc{inf} have-2\textsc{sg} of prepare-\textsc{inf} well \textsc{prox}.\textsc{sg}.\textsc{m} exam(\textsc{m})\\
	\glt \lq \textbf{Even} \textbf{if} [it means that] you don't get any sleep you have to prepare well for this exam.' (\cite[§47.12e]{RAEGramatica}, glosses added)
	\ex Spanish \label{exScalarAdditiveAunqueSea1}
	\begin{xlist}
		\exi{A:} \textit{Demasiadas preguntas... demasiadas preguntas...}
		\\ \lq Too many questions… too many questions…\rq{}
		\exi{B:} \gll Contesta \textbf{aun}-\textbf{que} \textbf{sea} un-a.\\
		answer.\textsc{imp} still-\textsc{subord} \textsc{cop}.\textsc{sbjv}.3\textsc{sg} one-\textsc{f}\\
		\glt \lq Answer \textbf{at least} one of them.\rq{ }(CORPES XXI, glosses added)
	\end{xlist} 
\end{exe}\il{Spanish|)}\is{subordination|)}

\il{German|(}
German \textit{noch} as a scalar additive is illustrated in (\ref{exScalarAdditiveGerman1}). As I discuss in more detail in \textcite{PersohnSchonNoch}, \textit{noch} constitutes a \textsc{beyond} operator that evokes a scalar model of degrees of sufficiency. As an additional twist, it requires a negatively defined scale. For instance, \textit{noch} in (\ref{exScalarAdditiveGerman1}) highlights that something as far removed from the standards of reason as the most stupid social media post (i.e. a high degree of \textit{in}credibility) is judged to be plausible enough by the former president's supporters.

\begin{exe}
	\ex German\label{exScalarAdditiveGerman1}\\
	\textit{Die einen feiern schließlich die dreiste Selbstermächtigung von Trump und Co. und reden sich beharrlich ein,}\\
	\lq Lastly, some celebrate Trump and company's bold self-authorisation and talk themselves into believing\rq{}\\
	\gll dass \textbf{noch} \textup{[}\textbf{der} \textbf{dümmst}-\textbf{e} \textbf{Tweet}\textup{]\textsubscript{\textsc{foc}}} die Wahrheit sp[r]ech-e.\\
\textsc{comp} still \phantom{[}\textsc{def}.\textsc{nom}.\textsc{sg}.\textsc{m} stupid.\textsc{sup}-\textsc{nom}.\textsc{sg}.\textsc{m} tweet(\textsc{m}) \textsc{def}.\textsc{nom}.\textsc{sg}.\textsc{m} truth(\textsc{f}) speak-\textsc{sbjv}.3\textsc{sg}\\
\glt \lq that \textbf{even the most stupid tweet} speaks the truth.'
\\(found online, glosses added)%\footnote{\url{https://taz.de/Zeitalter-der-Desinformation/!5693636&SuchRahmen=Print/} (20 June, 2022).} 
\end{exe}

This stylistically somewhat marked use of \textit{noch} finds its mirror image in the the much more common case of the \textsc{already\is{already}} expression \textit{schon} as a \textsc{beneath} operator. This is illustrated in (\ref{exScalarAdditiveGerman2}), where the focus denotation \lq one cup per day\rq{ }is related to a possibly higher intake of coffee.
\enlargethispage{\baselineskip}
\begin{exe}
	\ex German\label{exScalarAdditiveGerman2}\\
	\gll \textbf{Schon} \textup{[}\textbf{ein}-\textbf{e} \textbf{Tasse} \textbf{a}-\textbf{m} \textbf{Tag}\textup{]\textsubscript{\textsc{foc}}} senk-t dauerhaft den Blut-druck.\\
	already \phantom{[}one-\textsc{nom}.\textsc{sg}.\textsc{f} cup(\textsc{f}) at-\textsc{def}.\textsc{dat}.\textsc{sg}.\textsc{m} day(\textsc{m}) \phantom{[}lower-3\textsc{sg} permanently \textsc{def}.\textsc{acc}.\textsc{sg}.\textsc{m} blood-pressure(\textsc{m})\\
	\glt \lq \textbf{So much as one cup} [of coffee] \textbf{per day} results in a permanently lower blood pressure.\rq{ }(found online, glosses added)%\footnote{\url{https://www.aponet.de/artikel/kaffee-senkt-den-blutdruck-11322} (06 December, 2022).}
\end{exe}\il{German|)}

\is{syntax|(}Lastly, most of the expressions in \Cref{tableScalarAdditive} tend to sister the constituent containing the focus, as is common of focus-sensitive operators. The exception is \ili{Thai} \textit{yaŋ}, illustrated in (\ref{exScalarAdditiveThai}), which invariably follows the subject NPs and precedes all elements belonging to the predicate.

\begin{exe}
	\ex \ili{Thai}\label{exScalarAdditiveThai}\\
	\gll Fǎafɛ̀ɛt	khûu	nîi	mɯ̌an	kan		píap. Baaŋkhráŋ	\textbf{phɔ̂ɔ}		\textbf{mɛ̂ɛ}		\textbf{yaŋ}		yɛ̂ɛk		mâi		ʔɔ̀ɔk.\\
twin		pair	\textsc{prox}	alike	\textsc{recp} \textsc{ideoph}	sometimes father mother still seperate \textsc{neg} out\\
	\glt \lq The two twins look exactly alike; sometimes \textbf{even their parents} canʼt tell them apart.' (found online)%\footnote{\url{http://www.thai-language.com/id/247337} (11 March, 2021).}
\end{exe}\is{syntax|)}

\subsubsection{Discussion} To make sense of why \textsc{still} expressions may serve as scalar additive operators, it is worthwhile approaching the topic from a diachronic angle. To anticipate some of the following discussion, in all instances the link to phasal polarity appears to be a fairly indirect one that is mediated by other functions.

Earlier I suggested that an \lq even\rq{ }reading of \ili{Ket} \textit{hāj} may be a contextual inference,  and in \Cref{sectionAdditive} I discuss how \textsc{still} expressions used as plain additives can, given the right context, trigger a scalar inference. According to \textcite{Trujillo1990}, the conventionalisation of such an erstwhile pragmatic interpretation is what lead to the scalar additive function of Spanish\il{Spanish|(} \textit{aún}. With this in mind, an example like (\ref{exAscalarAdditiveOldSpanish}) constitutes a likely diachronic bridge from \lq in addition\rq{ }to \lq even\rq{}.\il{Spanish, Old} Here, the discourse context evokes a scale of affectedness on which the focus denotation \lq his body and soul\rq{ }ranks higher than the alternative \lq his possessions and eyes\rq{ }mentioned in the preceding clause. Most likely, a similar scenario can be posited for \ili{Thai} \textit{yaŋ}, seen in (\ref{exScalarAdditiveThai}) above. 

\begin{exe}
	\ex Old Spanish,\il{Spanish, Old} 11\textsuperscript{th}/12\textsuperscript{th} century\label{exAscalarAdditiveOldSpanish}\\
	\textit{Aquel que gela diesse sopiesse una palabra // Que perderie los aueres e mas los oios de la cara //}\\
	\lq Let anyone who might give it to him know, that he would lose his possessions and the eyes of his head\rq{}\\
	\gll  E \textbf{aun} demas los cuerpo-s e las alma-s.\\
	and still  other \textsc{def}.\textsc{pl}.\textsc{m} body(\textsc{m})-\textsc{pl} and \textsc{def}.\textsc{pl}.\textsc{f} soul(\textsc{f})-\textsc{pl}\\
	\glt \lq And, furthermore, \textbf{also/even} his body and soul.\rq{ }(\textit{Cantar de mio Cid}, cited in \cite[79]{Trujillo1990}, glosses added)
\end{exe}

Importantly, in (\ref{exAscalarAdditiveOldSpanish}) the focus denotation corresponds to a high degree on the relevant scale, whereas in the present-day variety \textit{aún} is also compatible with particularly low-ranking foci. This was seen in (\ref{exScalarAdditivesIntro2}), repeated below. This distribution suggests that, once the scalar additive use had become established, it underwent context expansion and a concomitant generalisation of meaning.

\begin{exe}
	\exr{exScalarAdditivesIntro2} Spanish\\
	\gll ... d-a vergüenza \textbf{aun} \textup{[}\textbf{mencion}-\textbf{ar}\textup{]\textsubscript{\textsc{foc}}} lo que los desobediente-s hac-en en secreto.\\
	{} give-3\textsc{sg} shame still \phantom{[}mention-\textsc{inf} 3\textsc{sg}.\textsc{n} \textsc{subord} \textsc{def}.\textsc{pl}.\textsc{m} disobedient-\textsc{pl} do-3\textsc{pl} in secret\\
	\glt \lq It is shameful \textbf{to even mention} what the disobedient do in secret.'
	\\(Eph. 5:12, \textit{La biblia al día}, cited in \cite[2]{GastvanderAuwera2011})
\end{exe}

Presumably, an intermediate stage in this process involved the use of the expression in environments of clause-mate negation;\is{negation} see \textcite{GastvanderAuwera2011} for general discussion. An example of such an instance is given in (\ref{exScalarAdditiveNiAun}).\footnote{Note how \textit{aún} as a scalar additive is negated via \textit{ni} \lq nor\rq{}, whereas the outer negation of \textsc{still}, i.e. the expression of \textsc{no longer},\is{no longer} is \textit{ya no} \lq already \textsc{neg}\rq{}. Asides from the clear differences in meaning, this gives further evidence for the separation of the two functions.} Subsequently, \textit{aún} would have expanded to non-negative scale reversal contexts like (\ref{exScalarAdditivesIntro2}). Validating this scenario is an open task for diachronic corpus work.\largerpage[2.15]

\begin{exe}
	\ex Spanish\label{exScalarAdditiveNiAun}\is{negation}\\
	\gll No ten-go yo tanto, \textbf{ni} \textbf{aun} la mitad.\\
	\textsc{neg} have-1\textsc{sg} 1\textsc{sg} that\_much nor still \textsc{def}.\textsc{sg}.\textsc{f} half(\textsc{f})\\
	\glt \lq I don't have that much, \textbf{not even} half as much.\rq{}
	\\(\cite[s.v. \textit{aun}]{RAEDictionary}, glosses added)
\end{exe}\il{Spanish|)}

Scalar inferences are not the only conceivable source of a scalar additive function. In \Cref{sectionTimeScalar} I discuss a use of \textsc{still} expressions that closely resembles a scalar additive operator, except for being restricted to scales of time.\il{German|(} This is illustrated in (\ref{exScalarAdditiveAegan}), where German \textit{noch} highlights that the common ground contains earlier (i.e. lower) alternatives.

\begin{exe}
	\ex German\label{exScalarAdditiveAegan}\\
	Context: About perpetual conflicts in the Aegean islands.\\
	\gll \textbf{Noch} \textup{[}\textbf{letzt}-\textbf{e} \textbf{Woche}\textup{]\textsubscript{\textsc{foc}}} kam es … zu … heftig-en Auseinandersetzung-en zwischen der Polizei und den Insel-bewohner-n.\\
	still last-\textsc{nom}.\textsc{sg}.\textsc{f} week(\textsc{f}) come.\textsc{pst}.3\textsc{sg} 3\textsc{sg}.\textsc{n} {} to {} severe-\textsc{dat}.\textsc{pl} conflict-\textsc{pl} between \textsc{def}.\textsc{dat}.\textsc{sg}.\textsc{f} police(\textsc{f}) and \textsc{def}.\textsc{dat}.\textsc{pl} island-inhabitant-\textsc{dat}.\textsc{pl}\\
	\glt \lq \textbf{As late as last week}, altercations between the police and the islanders occurred.\rq{ }(found online, glosses added)%\footnote{\url{https://www.nzz.ch/international/die-harte-linie-an-der-grenze-ist-in-griechenland-populaer-ld.1544107} (28 November, 2022).}
\end{exe}

While the focus in (\ref{exScalarAdditiveAegan}) is strictly temporal, the conceptual gap to a more general scalar additive operator is narrowed considerably in examples like (\ref{exScalarAdditiveHoffnung}), where time is mediated by metonymy.

\begin{exe}
	\ex German\label{exScalarAdditiveHoffnung}\\
	Context: Hope accompanies man through life.\\
	\gll \textbf{Noch} \textup{[}\textbf{a}-\textbf{m} \textbf{Grab}-\textbf{e}\textup{]\textsubscript{\textsc{foc}}} pflanz-t er die Hoffnung auf.\\
	still \phantom{[}at-\textsc{def}.\textsc{dat}.\textsc{sg}.\textsc{n} grave(\textsc{n})-\textsc{dat} plant-3\textsc{sg} 3\textsc{sg}.\textsc{m} \textsc{def}.\textsc{acc}.\textsc{sg}.\textsc{f} hope(\textsc{f}) up\\
	\glt \lq{}\textbf{Even/as late as} at his grave, he plants hope.\rq{}
	\\(Schiller, \textit{Hoffnung}, glosses added)
\end{exe}

In \textcite{PersohnSchonNoch} I argue that instances like (\ref{exScalarAdditiveHoffnung}) constitute the source for \textit{noch} as a \textsc{beyond} operator in cases like  (\ref{exScalarAdditiveGerman1}), repeated in condensed form in (\ref{exScalarAdditiveGerman3}). The additional peculiarities of \textit{noch} as a scalar additive that I described above (a scalar model of sufficiency and the requirement for a negatively defined scale) find a motivated explanation in the simultaneous evocation of the \isi{marginality} use of \textit{noch}. To briefly recapitulate from \Cref{sectionMarginality}, the \isi{marginality} functions addresses the question of whether a specific entity is included within the bounds of some scale or category (hence the model of sufficiency) and brings along an antonymic ordering relationship (degrees of \textit{im}plausiblity, \textit{de}centrality, etc.).

\begin{exe}
	\ex German\label{exScalarAdditiveGerman3}\\
	Context: They talk themselves into believing\\
	\gll dass \textbf{noch} \textup{[}\textbf{der} \textbf{dümmst}-\textbf{e} \textbf{Tweet}\textup{]\textsubscript{\textsc{foc}}} die Wahrheit sprech-e.\\
\textsc{comp} still \phantom{[}\textsc{def}.\textsc{nom}.\textsc{sg}.\textsc{m} stupid.\textsc{sup}-\textsc{nom}.\textsc{sg}.\textsc{m} tweet(\textsc{m}) \textsc{def}.\textsc{nom}.\textsc{sg}.\textsc{m} truth(\textsc{f}) speak-\textsc{sbjv}.3\textsc{sg}\\
\glt \lq that \textbf{even the most stupid tweet} speaks the truth.'
\end{exe}\il{German|)}

\il{Spanish|(}Lastly, an even more indirect relationship to phasal polarity \textsc{still} can be established for the one \textsc{beneath} operator in \Cref{tableScalarAdditive}, the complex Spanish item \textit{aunque sea}. This collocation was seen in (\ref{exScalarAdditiveAunqueSea1}) above. Example (\ref{exScalarAdditiveAunqueSea2}) is another illustration.

\begin{exe}
	\ex Spanish\label{exScalarAdditiveAunqueSea2}\\
	\gll Da-me \textbf{una} \textbf{galleta}, \textbf{aun}-\textbf{que} \textbf{sea}, que ten-go hambre.\\
	give.\textsc{imp}-\textsc{obj}.1\textsc{sg} \textsc{indef}.\textsc{sg}.\textsc{f} cookie(\textsc{f}) still-\textsc{subord} \textsc{cop}.\textsc{sbjv}.3\textsc{sg} \textsc{subord} have-1\textsc{sg} hunger\\
	\glt \lq Give me \textbf{so much as a cookie}, I'm hungry.\rq{}\\(\cite[§47.12q]{RAEGramatica}, glosses added)
\end{exe}

The composition of \textit{aunque sea} instantiates a productive pattern of marking the protasis of \isi{concessive} conditionals\is{conditional} \lq even if\rq{}. This pattern, in turn, builds on \textit{aún} in its scalar additive use. With that in mind, \textcite{GastvanderAuwera2011} suggests that  \textit{aunque sea} started out as a parenthetically inserted protasis with the literal meaning of \lq even if it were\rq{}, followed by the deletion of a first pronominal occurrence of the focus. This scenario is illustrated in (\ref{exScalarAdditiveAunqueSea3}). It finds support in the fact that, at least in European Spanish,\is{prosody|(} \textit{aunque sea} often follows its focus and is prosodically set of from it \parencite[§47.12q]{RAEGramatica}, as is the case in (\ref{exScalarAdditiveAunqueSea2}). 

\begin{exe}
	\ex Spanish\label{exScalarAdditiveAunqueSea3}\\
	\gll Si dices \sout{\textbf{algo}}, \textbf{aun}-\textbf{que} \textbf{sea} \textbf{una} \textbf{palabra}, v-as a ten-er problema-s.\\
	if say.2\textsc{sg} something still-\textsc{subord} \textsc{cop}.\textsc{sbjv}.3\textsc{sg} \textsc{indef}.\textsc{sg}.\textsc{f} word(\textsc{f}) go-2\textsc{sg} to have-\textsc{inf} problem-\textsc{pl}\\
	\glt \lq If you say \textbf{\sout{something}}, \textbf{even if it were a word}, you’ll get into trouble.'
	\\\parencite[356]{GastvanderAuwera2011}
\end{exe}\il{Spanish|)}\is{prosody|)}

\subsection{Comparisons of inequality}\label{sectionComparisons}\is{comparison|(} 
\subsubsection{Introduction} In this subsection, I address uses pertaining to the modification of comparisons of inequality. \il{German|)}The example in (\ref{exComparisonsIntroGerman}) is an illustration. Here German \textit{noch} contributes the scalar additive notion of \lq even\rq{ }to the comparison, the latter being signalled by the comparative form of the adjective \textit{größer} \lq bigger\rq{}.

\begin{exe}
	\ex German\label{exComparisonsIntroGerman}\\
	\gll Adam ist größer als Chris. Aber Berta ist \textbf{noch} \textbf{größer} (als Adam).\\
	A. \textsc{cop}.3\textsc{sg} big.\textsc{cmpr} than C. but B. \textsc{cop}.3\textsc{sg} still big.\textsc{cmpr} \phantom{(}than A.\\
	\glt \lq Adam is taller than Chris. But Berta is \textbf{even taller} (than Adam).\rq{}
	\\(\cite[9]{Umbach2009}, glosses added)
\end{exe}

In (\ref{exComparisonsIntroGerman}) the \textsc{still} expression serves to relate the comparison in its host clause to the one expressed in the preceding clause.\il{German|)} In (\ref{exComparisonsIntroQiang}),\il{Qiang, Northern} on the other hand, Northern Qiang \mbox{\textit{tɕe}-} serves as an intensifier, marking a particularly high degree of difference between the third person standard and the first person comparee.

\begin{exe}
	\ex Northern Qiang\il{Qiang, Northern}\label{exComparisonsIntroQiang}\\
	\gll Qɑ theː-s \textbf{tɕe}-\textbf{fia}.\\
	1\textsc{sg} 3\textsc{sg}-than still-white:1\textsc{sg}\\
	\glt \lq I am lighter (in color) than him (\textbf{a lot lighter}).'
	\\\parencite[88]{LaPollaHuang2003}
\end{exe}

\subsubsection{Distribution in the sample}
\Cref{tableComparisons} lists the expressions in my sample that have uses related to the modification of comparisons of inequality. In addition, \Cref{tableComparisons} indicates the type of contribution each expression makes in these constructions.\footnote{In addition, Classical Nahuatl\il{Nahuatl, Classical} \textit{oc} is commonly found in collocations like \textit{oc} \textit{achi} \lq still a bit > another bit\rq{ }to overtly mark the comparative degree (\appref{appendixClassicalNahuatlComparisons}).} As can be gathered, the relevant functions are attested for 17 items from 15 languages, and the distribution is strikingly biased towards the Eurasian macro-area. What is more, such uses are attested for 13 out of 16 Eurasian sample languages.\footnote{This use is also found with several cognates of my sample expressions, such as \ili{Dutch} \textit{nog} (e.g. \cite[124]{Vandeweghe1984}), \ili{Italian} \textit{ancora} (e.g. \cite{Tovena1994}; \cite{Vegnaduzzo2000}), or cognates of Serbian-Croatian-Bosnian\il{Serbian}\il{Croatian}\il{Bosnian} \textit{još} across Slavic (e.g. \cite[s.v. \textit{jeszcze}]{PWN}; \cite[s.v. \textit{ještě}]{SSJC}; \cite[s.v. \textit{ще}]{CYM11}; \cite[s.v. \textit{šè}]{SSKJ}; \cite[s.v. \textit{ešte}]{KSS4}). It is also found, among other markers, with \ili{Hungarian} \textit{még} \parencite{CsirmazSlade2020}.} Even with the limitations of the sample data in mind, this strongly points to a primarily areal phenomenon. In line with this geographic distribution, comparative functions are primarily found with independent grammatical words. As for the type of contribution, the scalar additive \lq even more\rq{ }type clearly predominates, with the intensifying \lq much more\rq{ }type being found with only one sample expression, which is Northern Qiang \mbox{\textit{tɕe}-}. In what follows, I examine the two types of uses separately.

\begin{table}
\caption{Functions pertaining to comparisons of inequality\label{tableComparisons}}
\small
\begin{tabularx}{\textwidth}{lllQl}
	\lsptoprule
	Macro-Area & Language & Expr. & Type & Appendix\\\midrule
	Eurasia & \ili{English} & \textit{still} & \lq even more\rq{} & \ref{appendixEnglishComparisons}\\
	& \ili{French} & \textit{encore} & \lq even more\rq{} & \ref{appendixFrenchEncoreComparisons}\\
	& & \textit{toujours}\footnote{Borderline case of a \textsc{still} expression.} & \lq even more\rq{} & \ref{appendixFrenchToujoursComparisons}\\
	& \ili{German} & \textit{noch} & \lq even more\rq{} &  \ref{appendixGermanComparisons}\\
	& Hebrew (Modern)\il{Hebrew, Modern} & \textit{ʕod} & \lq even more\rq &\ref{appendixHebrewOdComparisons}\\
	& \ili{Ket} & \textit{hāj} & \lq even more\rq{} & \ref{appendixKetComparisons}\\
	& Mandarin Chinese\il{Chinese, Mandarin} & \textit{hái} & \lq even more\rq{} &  \ref{appendixMandarinComparisons}\\
	& Northern Qiang\il{Qiang, Northern} & \textit{tɕe}- &\lq even more\rq{}(?), \lq much more\rq{}&  \ref{appendixQiangComparisons}\\
	&	Serbian-Croatian-Bosnian\il{Serbian}\il{Croatian}\il{Bosnian} & \textit{još} & \lq even more\rq{}& \ref{appendixBCMSComparisons}\\
	& Southern Yukaghir\il{Yukaghir, Southern} & \textit{ajī}/\textit{āj} & \lq even more\rq{} & \ref{appendixKolymaComparisons}\\
	& \ili{Spanish} & \textit{aún} & \lq even more\rq & \ref{appendixSpanishAunComparatives}\\
	& & \textit{todavía} & \lq even more\rq &  \ref{appendixSpanishTodaviaComparisons}\\
	& \ili{Thai} & \textit{yaŋ} & \lq even more\rq{} & \ref{appendixThaiComparisons}\\
	& Tundra Nenets\il{Nenets, Tundra} & \textit{təmna} & \lq even more\rq & \ref{appendixTundraNenetsComparisons}\\
	& \ili{Udihe} & \textit{xai}(\textit{si}) & \lq even more\rq{} & \ref{appendixUdiheComparisons}\\
	N. America& \ili{Kalaallisut} & \textit{suli} & \lq even more\rq{} & \ref{appendixKalaallisutComparatives}\\
	Papunesia & \ili{Paiwan} & =\textit{anan} & \lq even more\rq{} & \ref{appendixPaiwanComparisons}\\
	\lspbottomrule
\end{tabularx}
\end{table}

\subsubsection{A closer look: \lq even more\rq{}}\il{German|(} I now turn to a closer look at  \textsc{still} as \lq even more\rq{}. This type is illustrated in (\ref{exComparisonsIntroGerman}), repeated below. Here, German \textit{noch} serves to relate the comparison made in its host clause to the one expressed in the preceding clause, and it signals that the former yields a more informative proposition (that the comparee has a higher degree of the property in question).

\begin{exe}
	\exr{exComparisonsIntroGerman}German\\
	\gll Adam ist größer als Chris. Aber Berta ist \textbf{noch} \textbf{größer} (als Adam).\\
	A. \textsc{cop}.3\textsc{sg} big.\textsc{cmpr} than C. but B. \textsc{cop}.3\textsc{sg} still big.\textsc{cmpr} \phantom{(}than A.\\
	\glt \lq Adam is taller than Chris. But Berta is \textbf{even} \textbf{taller} (than Adam).\rq{}
	\\(\cite[9]{Umbach2009}, glosses added)
\end{exe}

\pagebreak
Whereas in (\ref{exComparisonsIntroGerman}) the second comparison is overtly manifest in the immediate discourse context, it is often left implied, as in (\ref{exComparisonsKolyma},\il{Yukaghir, Southern} \ref{exComparisonsGermanVerbesserte}). Example (\ref{exComparisonsGermanVerbesserte}) also illustrates another point: the use in question is not restricted to comparative construction \textit{sensu stricto}, but is recurrently commonly attested with verbal predicates that express a change in degree (a.k.a. \lq\lq degree achievements\rq\rq).

\begin{exe}
	\ex Southern Yukaghir\il{Yukaghir, Southern}\label{exComparisonsKolyma}\\
	\gll Tudel mit-ket \textbf{āj} \textbf{omosʼ} modo-j.\\
	3\textsc{sg} 1\textsc{pl}-\textsc{abl} still well sit-\textsc{intr}.3\textsc{sg}\\
	\glt \lq He lives \textbf{even better} than we do.' \parencite[364]{Maslova2003}
	
	\ex German\label{exComparisonsGermanVerbesserte}\\
	Context: About a TV production. The first episode discussed was watched by 9.4 percent of the target audience.\\
	\gll Eine alt-e Folge … \textbf{verbesser}-\textbf{te} \textbf{das} \textbf{Ergebnis} \textbf{noch} auf fantastisch-e 9,9 Prozent.\\
\textsc{indef}.\textsc{nom}.\textsc{sg}.\textsc{f} old-\textsc{nom}.\textsc{sg}.\textsc{f} episode(\textsc{f}) {} improve-\textsc{pst}.3\textsc{sg} \textsc{def}.\textsc{acc}.\textsc{sg}.\textsc{n} result(\textsc{n}) still up fantastic-\textsc{acc}.\textsc{pl} 9.9 percent\\
\glt \lq{}An old episode … \textbf{improved the result even further}, to a fantastic 9.9 percent.' (found online, glosses added)%\footnote{\url{https://www.fernsehserien.de/armes-deutschland-stempeln-oder-abrackern/folgen/9x01-folge-48-1440341}, (11 April, 2022).}
\end{exe}\il{German|)}

On a more fine-grained level, several of the items in \Cref{tableComparisons} are worth a brief discussion. Thus, \ili{English} \textit{still} in comparisons of inequality is far less frequent than \textit{even}. According to \textcite[198–201]{Ranger2018}, \textit{still} is primarily found in discourse contexts that feature an incremental build-up and an implied standard, a prototypical instance being (\ref{exComparisonsEnglish}). This is reminiscent of how additive \textit{still} is restricted to list environments (\appref{appendixEnglishAdditive}). 

\begin{exe}
	\ex \ili{English}\label{exComparisonsEnglish}\\
	\textit{For example, we would expect to find a very high proportion of cognate words in British and American English but a much lower percentage if we compare English and German and \textbf{still} \textbf{lower} if we compare English and Russian.} (BCN, cited in \cite[200]{Ranger2018})
\end{exe}

\il{French|(}
In a related fashion, in the case of French \textit{toujours} (which is a borderline case of a \textsc{still} expression and perhaps best treated as a marker of stasis), temporal notions persist.\is{persistence} Thus, as \textcite[166–167]{MosegaardHansen2008} observes, the felicitous use of \textit{toujours} in comparisons requires that at least one prior change along the scale in question has been observed. In other words, the implied standard of comparison in an example like (\ref{exComparisonsToujours}) cannot be a different entity at the same point in time.

\begin{exe}
	\ex French\label{exComparisonsToujours}\\
	\gll Elle me regardait avec \textbf{toujours} \textbf{plus} \textbf{d'}-\textbf{inquiétude}.\\
	3\textsc{sg}.\textsc{f} 1\textsc{sg}.\textsc{obj} look.\textsc{pst}.\textsc{ipfv}.3\textsc{sg} with still more of-concern\\
	\glt \lq She looked at me with \textbf{ever more disquiet}.\rq{}
	\\(\cite[164]{MosegaardHansen2008}, glosses added)
\end{exe}
\il{French|)}

With \ili{Thai} \textit{yaŋ}, there appears to be a preference for predicates denoting the lower end of a scale (Chingduang Yurayong, p.c.). This is reflected by the relevant examples in the literature featuring items like \textit{cha} \lq slow' in (\ref{exComparisonsThai}), or \textit{naaw} \lq cold\rq{}.

\begin{exe}
	\ex \ili{Thai}\label{exComparisonsThai}\\
	\gll Phǒm	wîŋ	cháa	lɛ́ɛw		\textup{(}tɛ̀ɛ\textup{)}	kháu	\textbf{yaŋ}	\textbf{cháa}	\textbf{kwàa}		\textbf{phǒm}		\textbf{ìik}.\\
1\textsc{sg} run slow already \phantom{(}but 3 still slow exceed 1\textsc{sg} more\\
	\glt \lq I run slowly, but he runs even more slowly than I do.' (\cite[240]{HigbieThinsan2002}, glosses added)
\end{exe}

\is{connective|(}
Lastly, with \ili{German} \textit{noch} and \il{Spanish|(}Spanish \textit{aún} and \textit{todavía}, the use in comparisons of inequality, together with a \lq more\rq{ }expression, has given rise to a discourse connective \lq what is more\rq{ }(\Cref{sectionWhatIsMore}). This is illustrated in (\ref{exComparisonsSpanishTodaviaMas}) for Spanish \textit{todavía} plus \textit{más}.

\begin{exe}
	\ex Spanish\label{exComparisonsSpanishTodaviaMas}\\
	\textit{Según Pessoa, lo que caracteriza al genio literario es la inadaptación a su medio.}\\
	\lq According to Pessoa, what characterises literary genius is its nonconformity with its medium.'
	\exi{}\gll  \textbf{Más} \textbf{todavía}: la fama literari-a de hoy excluy-e el éxito en el porvenir …\\
	more still \textsc{def}.\textsc{sg}.\textsc{f} fame(\textsc{f}) literary-\textsc{f} of today exclude-3\textsc{sg} \textsc{def}.\textsc{sg}.\textsc{m} success(\textsc{m}) in \textsc{def}.\textsc{sg}.\textsc{m} future(\textsc{m})…\\
	\glt \lq \textbf{What is more}: today's literary fame impedes future success.'
	\\(CORPES XXI, glosses added)
\end{exe}\is{connective|)}\il{Spanish|)}

\il{French|(}\is{syntax|(}
In syntactic terms, many of the expressions in \Cref{tableComparisons} are phrasal sisters to the degree predicate, as is characteristic of focus-sensitive operators. For instance, in (\ref{exComparisonsGreaterOffer}) it can be observed that \ili{English} \textit{still} intervenes between the indefinite article and the following noun phrase. Simiarly, in (\ref{exComparisonsFrenchSyntax}) French \textit{encore} immediately precedes \textit{mieux} \lq better\rq{}, rather than occurring in its usual post-auxiliary position. Similar cases are attested for \ili{German} \textit{noch}, French \textit{toujours}, Serbian-Croatian-Bosnian\il{Serbian}\il{Croatian}\il{Bosnian} \textit{još}, \ili{Spanish} \textit{aún} and \textit{todavía}, and Southern Yukaghir \textit{ajī}/\textit{āj}.\il{Yukaghir, Southern} It is conceivable that upon examination of more data this list becomes even longer.

\begin{exe}[(100)]
	\ex \ili{English}\label{exComparisonsGreaterOffer}\\
	\textit{\textbf{A} \textbf{still} \textbf{greater} \textbf{offer} came from the Dean.} \parencite[23 fn37]{Ippolito2007}
	
	\ex French\label{exComparisonsFrenchSyntax}\\
	\gll Le premier roman de Duschnock a eu beaucoup de succès. Le deuxième s’-est vendu \textbf{encore} \textbf{mieux}.\\
	\textsc{def}.\textsc{sg}.\textsc{m} first.\textsc{sg}.\textsc{m} novel(\textsc{m}) of D. have.3\textsc{sg} have.\textsc{ptcp} much of success \textsc{def}.\textsc{sg}.\textsc{m} second.\textsc{m} \textsc{refl}.3-\textsc{cop}.3\textsc{sg} sell.\textsc{ptcp}.\textsc{sg}.\textsc{m} still better\\
	\glt \lq Duschnock’s first novel was a great success. The second one sold \textbf{even better}.’ (\cite[164]{MosegaardHansen2008}, glosses added)
\end{exe}
\il{French|)}

\il{Hebrew, Modern|(}
 Modern Hebrew differs slightly, in that it is the degree modifier that forms a constituent with the \textsc{still} expression \textit{ʕod}, as can be seen in (\ref{exComparisonsHebrew}), where they move through the clause together.

\begin{exe}
	\ex Modern Hebrew\label{exComparisonsHebrew}\\
	\gll Ze \textbf{ʕod} \textbf{yoter} meguħax. \textup{/} Ze meguħax \textbf{ʕod} \textbf{yoter}.\\
	\textsc{prox}.\textsc{sg}.\textsc{m} still more ridicoulous {} \textsc{prox}.\textsc{sg}.\textsc{m} ridicoulous still more\\
	\glt \lq This is \textbf{even more} ridiculous.\rq{ }\parencite[251–252]{Glinert1976}
\end{exe}\il{Hebrew, Modern|)}\is{syntax|)}

\subsubsection{Discussion: \lq even more\rq{}}
There is an obvious metonymic relationship between phasal polarity \textsc{still} and the \lq even more\rq{ }use in comparisons of inequality, as observed before me by \textcite[165]{MosegaardHansen2008}, \textcite{JingSchmidtGries2009}, \textcite[200]{Ranger2018} and \textcite{Shetter1966}, among others. Thus, \textsc{still} presupposes that a situation not only obtains at topic time,\is{topic time} but has progressed beyond its onset. Similarly, the \lq even more\rq{ }use triggers the presupposition that the comparee has some degree of the property in question (as compared to the relevant norm or a contextually retrievable standard of comparison) and that, as a function of the comparison, this degree exceeds that of another entity. \Cref{figureComparisons} is a schematic illustration of these parallels.

\begin{figure}
	\centering
	\begin{subfigure}[b]{0.48\linewidth}\is{topic time}
		\centering
		\begin{tikzpicture}[node distance = 0pt]
			\node[mynode, text width=\superbreit+\hoehe, fill=cyan, very near start] (spain){Situation};
			\node[mynode, fill=none, minimum width=\hoehe, right= of spain] (france){};
			\draw[-, densely dashed] ($(spain.north west)+ (0,0) $) to ($(spain.south west)+ (0,-0.5*\hoehe) $);
	\node (onset) [below, align=center, label distance=0] at ($(spain.south west)+(0,-0.5*\hoehe)$) {\strut{}Onset\\\strut};	
			\draw[-, densely dashed] ($(spain.north east)+ (0,0) $) to ($(spain.south east)+ (0,-0.5*\hoehe) $);
			\draw[-, densely dashed] ($(spain.north east)+ (-\hoehe,0) $) to ($(spain.south east)+ (-\hoehe,-0.5*\hoehe) $);
			\node (TT) [below, align=center, label distance=0] at ($(spain.south east)+(-0.5*\hoehe,-0.5*\hoehe)$) {\strut{}Topic\\time\strut};
			\draw[-latex, line width=0.5pt]  (spain.south west) to  ($(france.south east)+(1ex,0)$) node [right] {t};
			
		\node [below, align=center, label distance=0] at ($(spain.south west)!0.5!(spain.south east)+(0,-0.5*\hoehe)$) {\strut{}Prior\\runtime\strut{}};			

\draw[-to, line width=0.5pt] (spain.south west) to  ($(spain.south west)!0.5!(spain.south east)$);
		\end{tikzpicture}
	\subcaption{\textsc{still}}	
	\end{subfigure}
	\begin{subfigure}[b]{0.48\linewidth}
		\centering
			\begin{tikzpicture}[node distance = 0pt]
			\node[mynode, text width=\superbreit+\hoehe, fill=cyan, very near start] (spain){Property};
			\node[mynode, fill=none, minimum width=\hoehe, right= of spain] (france){};
			\draw[-, densely dashed] ($(spain.north west)+ (0,0) $) to ($(spain.south west)+ (0,-0.5*\hoehe) $);
	\node [below, align=center, label distance=0] at ($(spain.south west)+(0,-0.5*\hoehe)$) {Std.\textsubscript{1}\strut\\\strut};			
		\draw[-, densely dashed] ($(spain.south west)!0.5!(spain.south east)$) to ($(spain.south west)!0.5!(spain.south east)+(0,-0.5*\hoehe)$) ;
	
		\node [below, align=center, label distance=0] at ($(spain.south west)!0.5!(spain.south east)+(0,-0.5*\hoehe)$) {Std.\textsubscript{2}\strut\\\strut};	
			\draw[-, densely dashed] ($(spain.north east)+ (0,0) $) to ($(spain.south east)+ (0,-0.5*\hoehe) $);

			\node [below, align=center, label distance=0] at ($(spain.south east)+(0,-0.5*\hoehe)$) {Comparee\strut\\\strut};
			\draw[-latex, line width=0.5pt]  (spain.south west) to  ($(france.south east)+(1ex,0)$) node [right] {d};
		\end{tikzpicture}
	\subcaption{\lq{}Even more\rq{}}	
	\end{subfigure}
	\caption{Schematic comparison of \textsc{still} and comparative \lq even more\rq{}	\label{figureComparisons}}
\end{figure}

In terms of the function's emergence, the vast majority of expressions in \Cref{tableComparisons} have other additive uses that can be said to motivate their use in comparisons of inequality. However, this is not universally the case, the clearest exception being \ili{French} \textit{toujours}. What is more, additive functions are far from necessary for the \lq even more\rq{ }use to arise.  To begin with, phasal polarity \textsc{still} and the use in question are virtually indistinguishable in cases like (\ref{exComparisonFrenchIncrease}, \ref{exComparisonsSpanishGrow}),
as observed by authors such as \textcite{Bosque2016}, \textcite[165]{MosegaardHansen2008} and \textcite[76]{VictorriFuchs1996}, among others.\il{French|(}\il{Spanish|(} Thus, in (\ref{exComparisonFrenchIncrease}, \ref{exComparisonsSpanishGrow}), the \textsc{still} expressions \textit{encore} and \textit{todavía} take scope over a verbal predicate that denotes a gradual change along a scale. As the phasal polarity use presupposes a prior runtime of the same situation, and therefore a previous advancement on the relevant scale, this unavoidably yields a dual comparison. In other words, such combinations make for a perfect bridge from \textsc{still} to \lq even more\rq{}.

\begin{exe}
	\ex French\label{exComparisonFrenchIncrease}\\
	\gll Avec la mondialisation, on peut craindre que les différences entre les ethnie-s \textbf{s’}-\textbf{accentue}  \textbf{encore} dans les années à venir.\\
	with \textsc{def}.\textsc{sg}.\textsc{f} globalisation.\textsc{inf}(\textsc{f}) \textsc{impr}/1\textsc{pl} can.3\textsc{sg} fear.\textsc{inf} \textsc{subord} \textsc{def}.\textsc{pl} difference.\textsc{pl} between \textsc{def}.\textsc{pl} ethnic\_group-\textsc{pl} \textsc{refl}.3-intensify.3\textsc{pl} still at \textsc{def}.\textsc{pl} year.\textsc{pl} to come.\textsc{inf}\\
	\glt \lq With globalisation, we may fear that the differences between ethnic groups \textbf{will still increase} in the years to come.\rq{ }(\cite[165]{MosegaardHansen2008}, glosses added)
 
	\ex Spanish\label{exComparisonsSpanishGrow}\\
	\gll Está muy alt-a, pero \textbf{todavía} \textbf{crec}-\textbf{erá}.\\
	\textsc{cop}.3\textsc{sg} very tall-\textsc{f} but still grow-\textsc{fut}.3\textsc{sg}\\
	\glt i.\phantom{i} \lq She's very tall, but \textbf{she'll grow even more}.\rq{}\\
	ii. \lq She's very tall, but \textbf{she'll still be growing}.\rq{}
	\\(\cite[214]{Bosque2016}, glosses added)
\end{exe}\il{Spanish|)}

Another, and somewhat related, bridging context can be found in an instance like (\ref{exComparisonsOldFrench}).\il{French, Old} Under an etymologically more conservative reading, this example features a persistent\is{persistence} intent do advance a process. As a function of the comparative adjective \textit{plus} \lq more\rq{}, the latter is depicted as having already yielded an advanced degree on some scale. As in (\ref{exComparisonFrenchIncrease}, \ref{exComparisonsSpanishGrow}), the sum of these elements entails the notion of \lq even more\rq{}. 

\begin{exe}
	\ex Old French,\il{French, Old} 13\textsuperscript{th} century\label{exComparisonsOldFrench}\\
	\textit{Sire, dist il, vostre merci. // A ceste damoisele ci // Vous pri c’ox pardounés vostre ire; //}
	\\ \lq Sire, he says, Your Grace. I ask you that you will renounce on your anger against this maiden;\rq{}
	\sn
	\gll Et si vous veul \textbf{encor} \textbf{plus} \textbf{dire},\\
	and indeed 2\textsc{pl}.\textsc{obj} want.1\textsc{sg} still more say.\textsc{inf}\\
	\glt \lq And, indeed, I \textbf{still want to say something more}/\textbf{want to say even more}.\rq{ }(\textit{L’ Atre périlleux}, cited in \cite[167]{MosegaardHansen2008}, glosses added)
\end{exe}

\is{always|(}\is{distributive|(}Lastly, a link of the more indirect kind can be established for French \textit{toujours} and \ili{English} \textit{still}. \textit{Toujours}, asides from being a marker of persistence,\is{persistence} also has a diachronically earlier \lq always\rq{ }use. As \textcite[167]{MosegaardHansen2008} discusses, the \lq even more\rq{ }use can be traced back to the \lq always\rq{ }function. Thus, some of the earliest relevant attestations, like the one in (\ref{exComparisonsToujoursMiddleFrench}) are compatible with both readings.\il{French, Middle}  The same case can be made for \ili{English} \textit{still}, which also once possessed an \lq always\rq{ }sense. According to \textcite{Lewis2019}, the example in (\ref{exComparisonsEME}), which features both distributivity (i.e. the recurrence of a similar situation across multiple times) and the comparative degree of an adjective, constitutes the first relevant attestation.  

\begin{exe}
	\ex Middle French,\il{French, Middle}  17\textsuperscript{th} century\label{exComparisonsToujoursMiddleFrench}\\
	\textit{Mais si l’ esperance est esteinte, // pourquoy desir , t’ efforces-tu // de faire une plus grande atteinte? // C’est que tu nays de la vertu, //}\\
	\lq But if hope is extinguished, // why, desire, do you endeavor // to reach higher? // It is because you are born of virtue //\rq{}
	
	\sn\gll et comme elle \textbf{est} \textbf{toujours} \textbf{plus} \textbf{forte}, // et sans faveur-s et sans appas, // quoy que l’-esperance soit morte, // desir, pourtant tu ne meurs pas.\\
	and as 3\textsc{sg}.\textsc{f} \textsc{cop}.3\textsc{sg} still more strong {} and without favour-\textsc{pl} and without attraction.\textsc{pl} {} what \textsc{rel} \textsc{def}.\textsc{sg}-hope \textsc{cop}.\textsc{sbjv}.3\textsc{sg} dead {} desire.\textsc{imp} nontheless 2\textsc{sg} \textsc{neg} die.2\textsc{sg} \textsc{neg}\\
	\glt \lq and as it [\textbf{is always stronger / grows ever stronger}], // both without favors and without attractions, // even though hope is dead, // desire, nevertheless you do not die.\rq{ }(d'Urfé, \textit{L’astrée}, cited in \cite[167]{MosegaardHansen2008}, glosses added)
    \il{French|)}

	\ex Early Modern English, 16\textsuperscript{th}\il{English, Early Modern}\il{English} century\label{exComparisonsEME}\\
	\textit{spoonefull by spoonefull: \textbf{bitterer} \textbf{and} \textbf{bitterer} \textbf{still}} \parencite[135]{Lewis2019}
\end{exe}\is{distributive|)}\is{always|)}

\subsubsection{A closer look and discussion: \lq much more\rq{}}\il{Qiang, Northern|(} Unlike the majority of expressions in \Cref{tableComparisons}, Northern Qiang \mbox{\textit{tɕe}-} can signal a high degree of difference between the comparee and the standard of comparison. Example (\ref{exComparisonsIntroQiang}), repeated below, is an illustration.

\begin{exe}
	\exr{exComparisonsIntroQiang} Northern Qiang\\
	\gll Qɑ theː-s \textbf{tɕe}-\textbf{fia}.\\
	1\textsc{sg} 3\textsc{sg}-than still-white:1\textsc{sg}\\
	\glt \lq I am lighter (in color) than him (\textbf{a lot lighter}).'
	\\\parencite[88]{LaPollaHuang2003}
\end{exe}

Whether \mbox{\textit{tɕe}-} can also have the \lq even more\rq{ }use is less clear. It is possible that in (\ref{exComparisonsQiangEven}) such a reading is a contextual inference, due to the property in question being predicated of the standard of comparison in the preceding clause.

\begin{exe}
		\ex Northern Qiang\label{exComparisonsQiangEven}\\
		\gll Pəs məpɑ wa, təp-ɲi tsə-s \textbf{tɕɑ}-\textbf{məpɑː} \textbf{lu}.\\
	today cold very tomorrow-\textsc{adv} this-than still-cold.\textsc{prosp} will\\
	\glt \lq Today is very cold; tomorrow \textbf{is going to be even colder} than this.\rq{}
	\\\parencite[161]{LaPollaHuang2003}
\end{exe}

In addition, \mbox{\textit{tɕe}-} is strinkingly similar to the superlative marker \mbox{\textit{tɕi}-}, which furthermore occupies the same position in the predicate. This suggest some etymological relationship between the two items. The one synchronic formal difference is that the superlative prefix, illustrated in (\ref{exComparisonsQiangSuperlative}), does not undergo vowel harmony.

\begin{exe}
	\ex Northern Qiang\label{exComparisonsQiangSuperlative}\\
	\gll \textbf{tɕi}-χtʂa\\
	most-small\\
	\glt \lq smallest\rq{ }\parencite[214]{LaPollaHuang2003}
\end{exe}

Whereas the \lq even more\rq{ }use examined above involves a metonymic projection from temporal intervals to degrees, in the use I discuss here the parallels to phasal polarity pertain to the differential function. In other words, the \lq much more\rq{ }use signals that the difference between the comparee and the standard goes beyond the minimum necessary for faithfully using the comparative construction, similar to how phasal polarity \textsc{still} contributes the presupposition that the situation depicted in the clause has progressed beyond its onset. These similarities are schematised in \Cref{figureComparisons2}.

\begin{figure}
	\centering
	\begin{subfigure}[b]{0.48\linewidth}\is{topic time}
		\centering
		\begin{tikzpicture}[node distance = 0pt]
			\node[mynode, text width=\schmal, fill=cyan, very near start] (spain){Situation};
			\node[mynode, fill=none, minimum width=\hoehe, right= of spain] (france){};
			\draw[-, densely dashed] ($(spain.north west)+ (0,0) $) to ($(spain.south west)+ (0,-0.5*\hoehe) $);
	\node (onset) [below, align=center, label distance=0] at ($(spain.south west)+(0,-0.5*\hoehe)$) {\strut{}Onset\\\strut};			
			
			\draw[-, densely dashed] ($(spain.north east)+ (0,0) $) to ($(spain.south east)+ (0,-0.5*\hoehe) $);
			\draw[-, densely dashed] ($(spain.north east)+ (-\hoehe,0) $) to ($(spain.south east)+ (-\hoehe,-0.5*\hoehe) $);
			\node (TT) [below, align=center, label distance=0] at ($(spain.south east)+(-0.5*\hoehe,-0.5*\hoehe)$) {\strut{}Topic\\time\strut};
			\draw[-latex, line width=0.5pt]  (spain.south west) to  ($(france.south east)+(1ex,0)$) node [right] {t};	
\draw[-to, line width=0.5pt] (spain.south west) to  ($(spain.south west)!0.5!(spain.south east)$);
		\end{tikzpicture}
	\subcaption{\textsc{still}}	
	\end{subfigure}
	\begin{subfigure}[b]{0.48\linewidth}
		\centering
			\begin{tikzpicture}[node distance = 0pt]
				\node[mynode, minimum width=2*\hoehe] (min){};				\draw[-, densely dashed] (min.north west) to ($(min.south west)+ (0,-0.5*\hoehe) $);
				\node [below,  label distance=0, align=center] at ($(min.south west)+(0,-0.5*\hoehe)$) {Std.\strut\\\strut};	
				\node[mynode, text width=\schmal, fill=cyan, very near start, right=of min] (spain){Inequality};
				\draw[-, densely dashed] (spain.north west) to ($(spain.south west)+ (0,-0.5*\hoehe) $);
			\node [below,  label distance=0, align=left, anchor=north west] at ($(spain.south west)+(0,-0.5*\hoehe)$) {\strut{}$\Delta$\textsubscript{min}};	
			\node[mynode, fill=none, minimum width=\hoehe, right= of spain] (france){};
	\draw[-to, line width=0.5pt] (spain.south west) to  ($(spain.south west)!0.5!(spain.south east)$);
	\draw[-, densely dashed] ($(spain.north east)+ (0,0) $) to ($(spain.south east)+ (0,-0.5*\hoehe) $);
	\node [below, align=center, label distance=0] at ($(spain.south east)+(0,-0.5*\hoehe)$) {Comparee\strut\\\strut};
	\draw[-latex, line width=0.5pt]  (min.south west) to  ($(france.south east)+(1ex,0)$) node [right] {d};
		\end{tikzpicture}
	\subcaption{\lq{}Much more\rq{}}	
	\end{subfigure}
	\caption{Schematic comparison of \textsc{still} and comparative \lq much more\rq{}		\label{figureComparisons2}}
\end{figure}
\is{scale|)}\is{comparison|)}\il{Qiang, Northern|)}

\subsection{Constituent coordination}\label{sectionCoordination}\is{conjunction|(}\is{coordination|(}
\subsubsection{Introduction}\il{Ket|(}
Two expressions in my sample have uses as constituent coordinators. These are Ket \textit{hāj} and Mandarin Chinese\il{Chinese, Mandarin} \textit{hái} (\appsref{appendixKetCoordination}, \ref{appendixMandarinCoordination}). In what follows, I briefly illustrate and discuss each of the two items separately.

\subsubsection{A closer look and discussion: Ket \textit{hāj}}\is{syntax|(} The Ket expression \textit{hāj} serves as a conjunctive coordinator and is attested as linking constituents of various types. This is illustrated for a pair of attributive adjectives in (\ref{exCoordinationKet1}), and for two clauses in (\ref{exCoordinationKet2}). As pointed out by \textcite[96]{Nefedov2015}, the coordinative function of \textit{hāj} is clearly a diachronic extension of \textit{hāj} in additive function, following the well-known path from \lq also\rq{ }to \lq and\rq{ }(see e.g. \cite{Mithun1988}).

\begin{exe}
	\ex 
	\begin{xlist}
		\exi{}Ket
		\ex\label{exCoordinationKet1}	
		\gll \textbf{Hʌna} \textbf{haj} \textbf{qē}-\textbf{ŋ} dɨlʲgat škɔla-di-ŋa ɔŋ-ɔ-tn\\
		small still big-\textsc{pl} children school-\textsc{n}-\textsc{dat} \textsc{subj}.3\textsc{pl}-\textsc{prs}-go\\
	\glt \lq Kleine und große Kinder gehen in die Schule. [\textbf{Small} \textbf{and} \textbf{big} children go to school.]\rq{ }(\cite[321]{Werner1997}, glosses by \cite[97]{Nefedov2015})
		
	\ex\label{exCoordinationKet2}		
	Context: Two brothers are being served fatty meat by a witch.\\
	\textit{Éɾȕla ánùntuɾu bū òn īs bə̄n dbīl[a].}
	\\ \lq Erula was smart and didn't eat much of the meat.\rq{}

	\gll A Tútà-da-ŋa ánùn bənsàŋ \textbf{bū} \textbf{ísqàl}-\textbf{s} \textbf{óvɨ̀lde} \textbf{haj} \textbf{bɨ́ldè} \textbf{ba} \textbf{d}-\textbf{b}-\textbf{īl}-[a\textbf{].}\\
	but T.-\textsc{m}-\textsc{dat}  mind not.be.\textsc{prs} 3\textsc{sg}.\textsc{m} greedy-\textsc{nom} was still everything customarily \textsc{subj}.3\textsc{m}-\textsc{obj}.3\textsc{n}-\textsc{pst}-eat\\
	\glt \lq But Tuta was stupid. \textbf{He was greedy and would always eat it all up}.'
	\\\parencite[93]{Vajda2004}	
	\end{xlist}
\end{exe}

\subsubsection{A closer look and discussion: Mandarin Chinese \textit{hái}-\textit{shì}}\il{Chinese, Mandarin|(}
Unlike Ket \textit{hāj},\il{Ket|)} Mandarin Chinese \textit{hái} signals disjunctive coordination, normally in collocation with copula \textit{shì}. Example (\ref{exCoordinationMandarin1}) is an illustration. As in this example, this function is predominantly found in interrogatives.\is{interrogative}

\begin{exe}
	\ex Mandarin Chinese\label{exCoordinationMandarin1}\\
	\gll Nǐ zuì xǐhuān lüchá \textbf{hái-shi} \textbf{huāchá}?\\
	2\textsc{sg} most like green\_tea still-\textsc{cop} jasmine\_tea\\
	\glt \lq Tu préfères le thé vert ou bien le thé au jasmin? [Do you prefer green tea \textbf{or Jasmine tea}?]' \parencite[112]{Donazzan2008}
\end{exe}

\textcite{Lu2019} suggests that the disjunctive function of \textit{hái}-\textit{shì} goes back to the additive function of \textit{hái} and a sentence pattern [[p] [\textit{hái} [\textit{shì} q]]]. In this pattern, copula \textit{shì} originally served as a predicator, and the disjunctive reading started out as a contextual inference from \lq \textit{p}, in addition \textit{q}\rq{}. The conventionalisation of this inference was accompanied by a structural reanalysis to [[p] [\textit{háishì} q]]. According to \textcite[339]{Wiedenhof2015}, in very informal registers \textit{shì} can get dropped; see (\ref{exCoordinationMandarin3}). Given its limited diastratic distribution, this likely constitutes a subsequent innovation. 

\begin{exe}
	\ex Mandarin Chinese\label{exCoordinationMandarin3}\\
	\gll Nǐ de \textbf{hái} wǒ de?\\
	1\textsc{sg} \textsc{assoc} still 2\textsc{sg} \textsc{assoc}\\
	\glt \lq Mine or yours?\rq{ }(\cite[339]{Wiedenhof2015}, glosses added)
\end{exe}
\il{Chinese, Mandarin|)}\is{coordination|)}\is{syntax|)}

\subsection{Conjunctional adverb}\label{sectionConjunctionalMain}
\addtocontents{toc}{\protect\setcounter{tocdepth}{3}}
In this subsection I turn to uses of \textsc{still} expressions and collocations containing them as conjunctional adverbs. I first turn to a more general type of conjunctional use (\Cref{sectionConjunctional}) and then address the more specific case of argumentative \lq what is more\rq{ }collocations (\Cref{sectionWhatIsMore}). Note that I discuss the somewhat related type of \lq And how!\rq{ }collocations separately in the context of exclamative functions (\Cref{sectionAndHow}).\is{exclamation}

\subsubsection{General conjunctional adverb}\label{sectionConjunctional}

\subsubsubsection{Introduction}
Example (\ref{exConjunctionalIntroUdihe}) illustrates the use of \ili{Udihe} \mbox{\textit{xai}(\textit{si})} as a conjunctional adverb \lq and (then)\rq{}.

\begin{exe}
	\ex \ili{Udihe}\label{exConjunctionalIntroUdihe}\\
	\gll Si \textup{[}si\textup{]} jeu=de jai-ni ede-ili tuŋči:, \textbf{xai} \textbf{geje} \textbf{dieli}-\textbf{zeŋe}-\textbf{fi} \textup{[}geje dieli-zeŋe-fi\textup{]}.\\
	2\textsc{sg} \phantom{[}2\textsc{sg} what=\textsc{foc} noise-3\textsc{sg} start-3\textsc{sg} jump\_on.\textsc{imp} still together fly-\textsc{fut}-1\textsc{pl} \phantom{[}together fly-\textsc{fut}-1\textsc{pl}\\
	\glt \lq If some noise starts, jump on me \textbf{and} \textbf{we’ll} \textbf{fly} \textbf{together}.\rq{}\\\parencite[Yegdige in a silk gown]{NikolaevaEtAl2019}
\end{exe}

This conjunctional use differs from 
coordinative\is{coordination} function with clausal scope that I discuss in \Cref{sectionCoordination}, in that \lq\lq it does not establish a syntactic\is{syntax} relationship of clauses as equal parts of another constituent, the complex sentence.\rq\rq{ }\parencite[81]{Forker2016}.

\subsubsubsection{Distribution in the sample}
\Cref{tableConjunctional} lists the four expressions in my sample that have a general conjunctional function. As can be gathered, this use is primarily found in Eurasia. Two of the four cases involve fixed collocation, and in one of these two instances the collocate itself is a \isi{connective} \lq then, after\rq{}.

\begin{table}
	\caption{Conjunctional adverb use\label{tableConjunctional}}
	\small
	\begin{tabular}{llll@{~}ll}
		\lsptoprule
		Macro-area & Language & Expression & \multicolumn{2}{l}{Collocate} & Appendix\\
		\midrule
		Eurasia & Hebrew (Modern)\il{Hebrew, Modern} & \textit{ʕod} & n/a && \ref{appendixHebrewConjunctional}\\
		& Mandarin Chinese\il{Chinese, Mandarin} & \textit{hái} & \textit{yǒu} & \lq \textsc{exist}\rq{} & \ref{appendixMandarinConjunctional}\\
		& \ili{Udihe} & \textit{xai}(\textit{si}) & n/a && \ref{appendixUdiheConjunctional}\\
		South America & Southern Lengua\il{Lengua, Southern} & \textit{makham} & \textit{natamén} & \lq then\rq{} & \ref{appendixEnxetSurConjunctional}\\
		\lspbottomrule
	\end{tabular}
\end{table}

\subsubsubsection{A closer look} I now turn to a closer look at the use of \textsc{still} expressions as conjunctional adverbs.\il{Hebrew, Modern|(}
In Modern Hebrew, this use of \textit{ʕod}, illustrated in (\ref{exConjunctionalHebrew}), has been described as being characteristic of more formal registers.

\begin{exe}
	\ex Modern Hebrew\label{exConjunctionalHebrew}\\
		\gll \textbf{ʕod} moser katav-enu ki…\\
	still report.3\textsc{sg}.\textsc{m} reporter-\textsc{poss}.1\textsc{pl} \textsc{comp}\\
	\glt \lq Our reporter \textbf{further} reports that…' (\cite[537]{Glinert1989}, glosses added)
\end{exe}\il{Hebrew, Modern|)}

\ili{Udihe} \mbox{\textit{xai}(\textit{si})}, illustrated as a conjunctional adverb in (\ref{exConjunctionalIntroUdihe}) above, is also repeatedly attested in what appears to be the notionally very close function of a specificational marker \lq that is to say, namely\rq{ }(\appref{appendixUdiheSpecificational}). This is illustrated in (\ref{exConjunctionalUdiheSpecificational}).

\begin{exe}
	\ex \ili{Udihe}\label{exConjunctionalUdiheSpecificational}\\
	Context: A hero has shot at an iron bird.\\
	\gll Tada-ni=dele piktige ŋen’e, \textbf{xai} \textbf{zokpo}-\textbf{ni} \textbf{culi}=\textbf{de}.\\
	arrow-\textsc{poss}.3\textsc{sg}=\textsc{foc} right go.\textsc{pfv} still throat-\textsc{poss}.3\textsc{sg} directly=\textsc{foc}\\
	\glt \lq The arrow had struck it exactly, [\textbf{that is}] \textbf{straight into the throat}.' \parencite[Sisam Zauli and the hero]{NikolaevaEtAl2019}
\end{exe}

\il{Chinese, Mandarin|(}
Turning to the cases that involve fixed collocations, Mandarin Chinese \textit{hái} has the conjunctional function in collocation with existential \textit{yǒu}; see (\ref{exConjunctionalMandarin1}). This collocation is also very common with \textit{hái} in plain additive use, as illustrated in (\ref{exConjunctionalMandarin2}).

\begin{exe}
	\ex
	\begin{xlist}
		\exi{}Mandarin Chinese
		\ex\label{exConjunctionalMandarin1}
	\gll Rùzhù qián yào fù yājīn. \textbf{Hái} \textbf{yǒu}, bù néng dài chǒngwù.\\
	check\_in before should pay deposit still \textsc{exist} \textsc{neg} able carry pet\\
	\glt \lq You need to pay the deposit before using the room. \textbf{Also}, you're not allowed to have pets here.' (found online, glosses added)%\footnote{\url{https://resources.allsetlearning.com/chinese/grammar/Expressing_\%22in_addition\%22_with_\%22haiyou\%22} (18 October, 2021).}
	
	\ex\label{exConjunctionalMandarin2}
	\gll Qù Lúndūn, Bālí, \textbf{hái} \textbf{yǒu} \textbf{Luómǎ}.\\
	go London Bali still \textsc{exist} Rome\\
	\glt \lq We are going to London, Bali, \textbf{and Rome as well}.\rq
	\\(\cite[312]{Wiedenhof2015}, glosses added)
	\end{xlist}
\end{exe}
\il{Chinese, Mandarin|)}

\il{Lengua, Southern|(}Lastly, with Southern Lengua \textit{makham}, the relevant function is found in combination with \textit{natamén} \lq then, after\rq{}. John Elliot (p.c.) reports that this is particularly common if the following clauses also contains \textit{makham}, as in (\ref{exConjunctionalEnxetSur}).

\begin{exe}
	\ex Southern Lengua\label{exConjunctionalEnxetSur}\\
	\gll \textbf{Natamén} \textbf{makham} ap-tamh-aha makham Kennaqte Appeywa Tásek Amya’a.\\
	then still \textsc{m}-work-go\_around.\textsc{decl} still K. A. good story\\
	\glt \lq \textbf{Then} \textbf{later}, Kennaqte Appeywa spoke again from the Bible.' \\(John Elliot, p.c.)
\end{exe}\il{Lengua, Southern|)}

\subsubsubsection{Discussion} The conjunctional function just described clearly goes back to additive uses of the same expressions, rather than being directly related to phasal polarity \textsc{still}. Not only is an additive function attested for all expressions in \Cref{tableConjunctional}, but such an extension is common of additive markers in general \parencite{Forker2016}. Example (\ref{exConjunctionalTurkish}) is an illustration featuring \ili{Turkish} (not in my sample) \mbox{=\textit{de}}.

\begin{exe}
	\ex \ili{Turkish}\label{exConjunctionalTurkish}\\
	Context: We will buy armchairs.\\
	\gll On-lar-a çiçek-li basma örtü-ler dik-er-im ben. Bir=\textbf{de} kabul gün-ümüz ol-ur.\\
	3-\textsc{pl}-\textsc{dat} flower-\textsc{adj} print cover-\textsc{pl} sew-\textsc{aor}-1\textsc{sg} 1\textsc{sg} one=also admission day-\textsc{poss}.1\textsc{pl} be(come)-\textsc{aor}\\
	\glt \lq I will make covers for them in floral print. \textbf{And} we will have an \lq\lq{}at home day{\rq\rq}.\rq{ }\parencite{Kerslake1996}
\end{exe}

Additional support for this interpretation comes from the fact that in Mandarin Chinese the conjunctional use occurs in the collocation \textit{hái yǒu},\il{Chinese, Mandarin} which is a common occurrence with \textit{hái} as \lq{}also\rq{}. In semasiological terms, this development constitutes a fairly straightforward instantiation of the tendency for meanings to become increasingly procedural (\Cref{sectionSemasiologicalChange}).\is{conjunction|)}

\subsubsection{\lq What is more\rq{}}\label{sectionWhatIsMore}\is{connective|(}
\subsubsubsection{Introduction} In this subsection, I briefly turn to collocations that involve \textsc{still} expressions and which serve as clause-initial, argumentative connectives \lq what is more\rq{}.\il{Spanish|(} Examples (\ref{exWhatIsMoreSpanish}, \ref{exWhatIsMoreHebrew}) are illustrations.

\begin{exe}
	\ex Spanish\label{exWhatIsMoreSpanish}\\
	\gll En La Habana hay también gente pobre. \textbf{Más} \textbf{aún}, hay esclavo-s.\\
		en \textsc{def}.\textsc{sg}.\textsc{f} H.(\textsc{f}) \textsc{exist} also people poor more still \textsc{exist} slave-\textsc{pl}\\
	\glt \lq There are also poor people in Havana. \textbf{What is more}, there are slaves.' (CORPES XXI, glosses added)
 
	\ex Modern Hebrew\il{Hebrew, Modern}\label{exWhatIsMoreHebrew}\\
	\gll Hay-ta te\rq{}una, \textbf{ma} \textbf{ʕod} še-ha-nehag-im šavt-u.\\
	\textsc{cop}.\textsc{pst}-3\textsc{sg}.\textsc{f} accident(\textsc{f}) what still \textsc{subord}-\textsc{def}-driver-\textsc{pl}.\textsc{m} strike.\textsc{pst}-3\textsc{pl}\\
	\glt \lq There was an accident, \textbf{what's more}, the drivers were striking.'
	\\(\cite[267]{Glinert1989}, glosses added)	
\end{exe}

\subsubsubsection{Distribution and types}
\Cref{tableWhatIsMore} lists the attested \lq what is more\rq{ }collocations in my sample. As can be gathered, they are attested for four expressions from three languages, all stemming from western Eurasia. In Spanish and German,\il{German} these collocations involve \textsc{still} expressions together with a quantifier \lq more\rq{}, as in (\ref{exWhatIsMoreSpanish}). The Modern Hebrew\il{Hebrew, Modern} case, on the other hand, features a WH-element \lq what\rq{}; see (\ref{exWhatIsMoreHebrew}) above.

\begin{table}[htb]
	\caption{Connective collocations \lq what is more\rq\label{tableWhatIsMore}}
	\small
	\fittable{\begin{tabular}{lllll}
		\lsptoprule
		Macro-Area & Language & Expression & Collocation & Appendix\\
		\midrule
		Eurasia & German & \textit{noch} & \textit{mehr noch} \lq still more\rq{} & \ref{appendixGermanWhatIsMore}\\
		& Hebrew (Modern)\il{Hebrew, Modern} & \textit{ʕod} & \textit{ma} \textit{ʕod} \lq what still\rq{} & \ref{appendixHebrewWhatIsMore}\\
		& Spanish & \textit{aún} & \textit{aún} \textit{más} / \textit{más aún}& \ref{appendixSpanishAunConjunctionalAunMas}\\
		& & &	 \lq still more / more still\rq{}\\
		& & \textit{todavía} & \textit{todavía} \textit{más} / \textit{más} \textit{todavía} & \ref{appendixSpanishTodaviaConjunctionalTodaviaMas}\\
		& & & \lq{}still more / more still\rq{}\\
		\lspbottomrule
	\end{tabular}}
\end{table}

\subsubsubsection{Discussion}\largerpage[-2]
In terms of their motivation and origins, all collocations in \Cref{tableWhatIsMore} are fairly transparent. The \ili{German} and Spanish cases, which involve a comparative \lq more\rq{ }marker are clearly based on the respective collocations in comparison of inequality \lq even more\rq{ }(\Cref{sectionComparisons}).\is{comparison}} In the case of Hebrew \il{Hebrew, Modern}\textit{ma ʕod}, on the other hand, it is the \textsc{still} expression \textit{ʕod} itself that contributes the quantificational notion (\Cref{sectionAdditive}). That is, despite superficial differences, \textit{ma ʕod} is no different from \ili{English} \textit{what is more}.\il{Hebrew, Modern} What both types have in common is that they involve additivity of some sort and a straightforward transfer from the propositional to the textual\is{textuality}\is{proceduralisation} domain.\is{connective|)}\il{Spanish|)}

\subsection{A brief reflection on additive and related uses}\label{sectionAdditiveConclusion}\largerpage[-1]
In a ground-laying typological study, \textcite{Forker2016} examines the functional range of additive markers and the implicational relationships between their individual uses. Comparing her findings to those from my sample reveals some striking areas of divergence, even with the limitations of the data in mind. Thus, whereas around four out of five additive markers examined by \citeauthor{Forker2016} can also serve as scalar\is{scale} additives, in my sample this is the case for only roughly one out of seven items. A similar discrepancy obtains for the constituent \isi{coordination} function. Likewise, the  use of an additive expression as a conjunctional adverb use is about twice as frequent in \citeauthor{Forker2016}'s sample as compared to mine; see \Cref{tableAddForker}.

\begin{table}
\caption{Relative frequency of additive-related functions compared to \citeauthor{Forker2016}'s (\citeyear{Forker2016}) findings\label{tableAddForker}}
\begin{tabular}{l rr}
	\lsptoprule
	Function & \multicolumn{1}{c}{Forker} & \multicolumn{1}{c}{My sample}\\\midrule
	Scalar\is{scale} additive & 81\% & 14\%\\
	Constituent \isi{coordination} & 60\% & 7\%\\
	Conjunctional adverb\footnote{Excluding \lq what is more\rq{}} & 29\% & 14\%\\
	\lspbottomrule
\end{tabular}
\end{table}

\is{topic|(}
What is more, nearly two thirds of the additive expressions in \citeauthor{Forker2016}'s (\citeyear{Forker2016}) are also used as switch topic markers, This function is illustrated in (\ref{exAdditiveTopicGawwada}), where \ili{Ale-Gawwada} (Afro-Asiatic > Cushitic) \mbox{=\textit{kka}} marks the reintroduction of Lion as the story's protagonist.

\begin{exe}
	\ex \ili{Ale-Gawwada}\label{exAdditiveTopicGawwada}\\
	Context: Lion had summoned all animals.\\
	\gll Ye=okaay-ú=ppa / \textup{[}\textbf{karma}-\textbf{o}\textup{]\textsubscript{\textsc{foc}}}=\textbf{kka} saˤa-k-o kat-a ˀɨ=ˀˀassap-aɗ-i.…\\
	\textsc{neg}.3=come-\textsc{pfv}.\textsc{neg}.3=\textsc{lnk} {} lion-\textsc{m}=also heart-\textsc{sg}-\textsc{m} down-out \textsc{def}=think-\textsc{mid}-\textsc{pfv}.3.\textsc{m}\\
	\glt \lq (The Monkey only) did not come; therefore, \textbf{the Lion} thought in his heart…\rq{ }\parencite[332]{Tosco2010}
\end{exe}

The only marker in my sample for which there are some indications of a topic-related use is \ili{Udihe} \mbox{\textit{xai}(\textit{si})} (\appref{appendixUdiheTopic}). Example (\ref{exAdditiveTopicUdihe}) is an illustration. Here, \mbox{\textit{xai}(\textit{si})} appears to emphasise the switch from the tiger's role in culture to a description of its outward appearance. That said, \mbox{\textit{xai}(\textit{si})} is a very frequent item in texts, and its individual contribution in each attestation is far from fully understood. Furthermore, according to \textcite{Forker2016}, there is an implicational relationship between the switch topic function and the use of the same items as conjunctional adverbs, in that all expressions that have the latter use also have the former. This is clearly not the case in my sample.

\begin{exe}
	\ex \ili{Udihe}\label{exAdditiveTopicUdihe}\\
	 Context: About tigers. They are considered to be God’s animal.\\
	\gll Ute bi:, \textbf{iŋakta}-\textbf{na}-\textbf{ni} \textbf{xaisi} oño-ni, kegdeje-ni iŋakta-na-ni bagdi:-ni zülie-li.\\
	this \textsc{cop}:\textsc{prs}.\textsc{ptcp} fur-\textsc{designative}-\textsc{poss}.3\textsc{sg} still draw.\textsc{pst}-3\textsc{sg} striped-3\textsc{sg} fur-\textsc{designative}-\textsc{poss}.3\textsc{sg} grow-3\textsc{sg} striped-\textsc{adj}\\
	\glt \lq That’s how it is. \textbf{Its} \textbf{fur} all grows motley and stripy.' \parencite[The tiger for the Udihe people]{NikolaevaEtAl2019}\footnote{See \textcite[126–127]{NikolaevaTolskaya2001} on the \lq\lq designative" or \lq\lq destinative" case marker \mbox{-\textit{na}}.}
\end{exe}\is{topic|)}

All these points suggest that \textsc{still} expressions as additive markers develop along very different lines from run-of-the-mill additives. I can only speculate at this point, but it seems plausible that, at least to some extent, these differences are a function of the differences in discourse patterns that I discuss in \Cref{sectionAdditive}.
\is{focus|)} 

\subsection{Other functions adjacent to additivity}\label{sectionAdditiveRemnantUses}\il{Nahuatl, Classical|(}
In this subsection, I briefly turn to two uses that are loosely related to additivity in that they involve the inclusion of other entities. More specifically, I address Classical Nahuatl \textit{oc} as \lq other, different one\rq{ }and \ili{Kekchí} \textit{toj} as spatial \lq until\rq{}.

\subsubsection{Classical Nahuatl \textit{oc} as \lq other, different\rq{}} Example (\ref{exOther1}) illustrates Classical Nahuatl \textit{oc} in combination with a quantifier, together referring to a different entity of the same  type as a contextually given one (\appref{appendixClassicalNahuatlOther}). 
\begin{exe}
	\ex Classical Nahuatl\label{exOther1}\\
	\gll Huel \textbf{oc} \textbf{cen}-\textbf{tlamantli} in ic ni-c-mati-ya in mo-tēnyo.\\
	\textsc{intens} still one-thing \textsc{det} thus \textsc{subj}.1\textsc{sg}-\textsc{obj}.3\textsc{sg}-know-\textsc{pst}.\textsc{ipfv} \textsc{det} \textsc{poss}.2\textsc{sg}-fame.\textsc{poss}\\
	\glt \lq Tout autre est lʼopinion que jʼavais de ta renommée. [Quite \textbf{a different one} is the opinion I had of your reputation.]\rq
	\\(\cite[1266]{Launey1986}, glosses added)
\end{exe}

\textit{Oc} also has a use as an additive marker, and in this function it can combine with a quantifier phrase to signal the addition of a different entity of the same kind, as illustrated in (\ref{exOther2}).

\begin{exe}
	\ex\label{exOther2}
	\gll Is-ka' iwaan \textbf{ok} \textbf{sen}-tlamantli.\\
	here-\textsc{pred} and still one-thing\\
	\glt \lq And here is \textbf{yet another} thing.' \parencite[41]{Langacker1977}
\end{exe}

As observed before me by \textcite[1266]{Launey1986}, there is an obvious similarity between \textit{oc} in examples like (\ref{exOther2}) and the use I discuss here, a proximity that is also reflected in items like \ili{English} \textit{another} or \ili{Spanish} \textit{otro}. The two functions are especially close in cases like (\ref{exOther3}), which involves additivity as well as choice from an established set in the form of commonly agreed-upon mortal sins. Such examples likely constitute a bridging context from additivity to alterativity, the link to phasal polarity thus being an indirect one.

\begin{exe}
	\ex Classical Nahuatl\label{exOther3}\\
	\textit{Monequi anquìtōzquè in quēzquipa ōantlāhuānquè, in quēzqui-pa ō anquìtlacòquê in īmissàtzin Totēcuiyo}
	\glt \lq Que digais quantas vezes os aueis emborrachado, quantas aueis dexado de oyr Missa, [You must say how many times you got drunk, how many times you failed to go to mass]\rq
	\exi{}\gll  çācè quēcīzqui-pa in ō ī-pan an-huetz-quê in \textbf{oc}-\textbf{cequi} \textbf{tē}-\textbf{mictiā}-\textbf{ni} \textbf{tlàtlacōlli}.\\
	finally how\_many\_each-time \textsc{det} \textsc{aug} \textsc{poss}.3\textsc{sg}-top \textsc{subj}.2\textsc{pl}-fall-\textsc{pst}.\textsc{pfv}.\textsc{pl} \textsc{det} still-some \textsc{obj}.\textsc{indef}.\textsc{human}-kill-\textsc{agt}.\textsc{nmlz} sin\\
	\glt \lq y finalmente quantas vezes aueis caido en otroc pecados mortales. [and how many times you have fallen into \textbf{other} \textbf{mortal} \textbf{sins}.]\rq{}
	\\(\cite[117]{Carochi1645}, glosses added)
\end{exe}\il{Nahuatl, Classical|)}

\subsubsection{Kekchí \textit{toj} as spatial \lq until\rq{}}\is{limitative|(}\il{Kekchí|(} Example (\ref{exSpatialLimitative1}) illustrates Kekchí \textit{toj} as \lq until\rq{ }together with a spatial complement (\appref{appendixKekchiSpatialLimit}).

\begin{exe}
	\ex Kekchí\label{exSpatialLimitative1}
	\begin{xlist}
		\exi{A:}\textit{B'ar naxik li manguera?}\\
		\lq Where does the hose go?\rq{}
		\exi{B:}\gll  Ay ink'a' n-in-naw, mare arin \textbf{toj} \textbf{najt} \textbf{chi}-\textbf{r}-\textbf{ix} \textbf{li} \textbf{tzuul}.\\
	\textsc{interj} \textsc{neg} \textsc{prs}.3\textsc{sg}.\textsc{p}-1\textsc{sg}.\textsc{a}-know perhaps here still far \textsc{prep}-\textsc{poss}.3\textsc{sg}-back \textsc{det} mountain\\
	\glt \lq Ay, I don't know, perhaps (from) here \textbf{until far over the hill}.\rq{ }\parencite[464]{Kockelman2020}
	\end{xlist}
\end{exe}

The same \lq until\rq{ }reading of \textit{toj} is also available with temporal complements (\Cref{sectionTimeScalarRestrictive}), as illustrated in (\ref{exSpatialLimitative2}). It is safe to assume that this latter use, in conjunction with temporal zero anaphora, also constitutes the source of \textit{toj} as an exponent of \textsc{still} and thereby the link between the spatial limiative use and phasal polarity \textsc{still}.\largerpage

\begin{exe}
	\ex Kekchí\label{exSpatialLimitative2}\\
	\gll Chalen sa' x-kach'inal \textbf{toj} \textbf{anaqwan} maa-jun-wa x-kala.\\
	from \textsc{prep} \textsc{poss}.3-youth until today \textsc{neg}-one-time \textsc{pfv}-3.get\_drunk\\
		\glt \lq From his youth \textbf{until today}, not once has he gotten drunk.\rq
		\\(\cite[197]{EachusCarlson1980}, glosses by \cite[463]{Kockelman2020})
\end{exe}\is{limitative|)}\il{Kekchí|)}
\is{additive|)}

\section{Restrictive}\is{focus|(}\is{restrictive|(}
\addtocontents{toc}{\protect\setcounter{tocdepth}{2}}
\label{sectionRestrictiveUebergeordnet}
\subsection{Introduction}
In this section, I discuss restrictive uses of \textsc{still} expressions. As laid out in  \Cref{sectionQuantificationScales}, I understand restrictive operators as denoting negative\is{negation} quantification over a set of context propositions that differ from the text proposition in the denotation of the focus. Their quantificational force may be existential, in which case I speak of \textit{exclusive} operators. If they denote universal quantification, on the other hand, I speak of \textit{at least}-type restrictives. As a second parameter of variation, restrictive operators may or may not require the alternative denotations to form an ordered set (a scale).\is{scale}

In what follows, I first discuss cases in which a \textsc{still} expression serves as a non-scalar exclusive operator, which is a coexpression pattern primarily found in the Australian and Papunesian macro-areas (\Cref{sectionExclusive}). I then turn to a use that combines temporal inclusion with the exclusion of context propositions \lq thus far only\rq{}, again without the requirement of scalar alternatives. Lastly, in \Cref{sectionScalarRestrictive}, I discuss scalar restrictive uses. Note that I do not discuss a specifically temporal restrictive function \lq no earlier than\rq{ }here. For examinations of the latter use, see \Cref{sectionTimeScalarRestrictive}.

\subsection{Non-scalar exclusive}
\label{sectionExclusive}
\subsubsection{Introduction}\il{Kewa|(}
The examples in (\ref{exRestrictive1}) illustrate the colexification of phasal polarity \textsc{still} and a non-scalar exclusive function for Kewa \textit{pa}. Thus, in (\ref{exKewaRestrictive1}) this expression signals the \isi{persistence} of the boy being breastfed, whereas in  (\ref{exKewaRestrictive2}) it serves to exclude all alternatives to the proposition \lq I stay'.

\begin{exe}
\ex  \label{exRestrictive1}
	\begin{xlist}
		\exi{}Kewa
		\ex\label{exKewaRestrictive1}
		 \gll Go naaki-ri \textbf{adu} \textbf{pa} \textbf{na} \textbf{la}-\textbf{aya}.\\
	\textsc{dem} boy-\textsc{top} breast still eat stand-\textsc{prs}:3\textsc{sg}\\
		\glt \lq That boy \textbf{is} \textbf{still} \textbf{breastfed} (lit. still eats breast).'
		\\(\cite[7]{Franklin2007}, glosses added)	
	
		\ex\label{exKewaRestrictive2}
		\gll \textbf{Pa} \textbf{piru} \textbf{aa}-\textbf{lua} koe le sa pi.\\
	still stay stand.\textsc{dur}-\textsc{fut}:1\textsc{sg} bad thing put sit.\textsc{prs}:1\textsc{sg}\\
		\glt \lq (If) \textbf{I} don’t say something (lit. \textbf{just stay}) I have put something valueless.' \parencite[311–312]{Yarapea2006}
	\end{xlist}
\end{exe}
\il{Kewa|)}
\subsubsection{Distribution in the sample}
\Cref{tableRestrictive} lists the markers in my sample that function as non-scalar exclusives. The sample data suggest that \textsc{still}–exclusive coexpression is a widespread phenomenon in Australia and Papunesia.\footnote{An additional Australian candidate from outside the sample is Jaru/Wanyjirra\il{Jaru}\il{Wanyjirra} (Pama-Nyungan) \mbox{=\textit{lu}} (\cite[506–512]{Senge2015}; \cite[210]{Tsunoda1981}). In another Pama-Nyungan language, Eastern Arrernte,\il{Arrernte, Eastern} \textit{anteye} \lq still' is formally related to restrictive \textit{ante} \parencite[350–351]{Wilkins1989}. Some further candidates in Papunesia include \textit{kəp} in the Ndu language \ili{Manambu} \parencite[100, 567]{Aikhenvald2008}, and \textit{nawe} in the Sepik language \ili{Mehek} \parencite[479]{Hatfield2016}. Austronesian candidates from Papunesia include  \textit{nanga} in \ili{Nalik} \parencite[201–202]{Volker1994}, \mbox{=\textit{te}} in \ili{Pohnpeian} \parencite[301–302]{Rehg1981}, and \textit{boko} in \ili{Tolai} (examples throughout \cite{Mosel1984}).}   As can be gathered from \Cref{tableRestrictive}, outside of these two Macro-areas, it appears to be rare; the only exception in my sample is \mbox{=\textit{vi}} in the Bantu language Chuwabu.\il{Chuwabu}  That said, it is possible that sampling bias plays some role. Thus, several of the sample languages were included due to them being mentioned in \citeauthor{vanBaar1991} (\citeyear{vanBaar1991}, \citeyear{vanBaar1997}) and \textcite{SchultzeBerndt2002}, whose discussions focus on Australia and Papunesia. In terms of their wordhood, markers serving as both \textsc{still} and non-scalar exclusives are attested as bound morphemes (clitics or suffixes) and as independent grammatical words, suggesting no correlation between morphosyntactic status and function. Lastly, note that other \textsc{still} expressions may go together with an exclusive marker. Thus, \ili{Ewe} \mbox{\textit{ga}-} is often combined with exclusive \textit{ko} or its reduplicate \textit{kookooko}. Similarly, \ili{Wardaman} \textit{gayawun} frequently goes together with \mbox{-\textit{bi}}, a marker that can be analysed as exclusive (see \cite{SchultzeBerndt2002}).

\begin{table}
\caption{Coexpression of exclusive \lq only\rq{}\label{tableRestrictive}}
\begin{tabular}{llll}
\lsptoprule
	Macro-Area & Language & Expression & Appendix\\\midrule
	Africa & \ili{Chuwabu} & =\textit{vi} & \ref{appendixChuwabuRestrictive}\\
	Australia & \ili{Gooniyandi} & =\textit{nyali} & \ref{appendixGooniyandiRestrictive}\\
	 & \ili{Gurindji} & =\textit{rni} & \ref{appendixGurindjiRestrictive}\\
	 & \ili{Kayardild} & =(\textit{i})\textit{da} & \ref{appendixKayardildRestrictive}\\
	 & \ili{Martuthunira} & \textit{waruu}(\textit{l}) & \ref{appendixMartuthuniraRestrictive}\\
	 & \ili{Wubuy} & -\textit{wugij} & \ref{appendixWubuyRestrictive} \\
	Papunesia & \ili{Acehnese} & \textit{mantöng} 	&\ref{appendixAcehneseRestrictive} \\
	 & \ili{Chamorro} & \textit{ha'} & \ref{appendixChamorroRestrictive}\\
	 & \ili{Kewa} & \textit{pa} & \ref{appendixKewaRestrictive} \\
	 & \ili{Komnzo} & \textit{komnzo} & \ref{appendixKomnzoRestrictive}\\
	 & \ili{Ma Manda} & -\textit{gût} & \ref{appendixMaMandaRestrictive}\\
	 & \ili{Rapanui} & \textit{nō} & \ref{appendixRapanuiRestrictive}\\
\lspbottomrule
\end{tabular}
\end{table}


\subsubsection{A closer look} 
I now turn to a closer examination of the coexpression of \textsc{still} and \lq{}only\rq{}. To begin with, the operators under discussion here do not require a ranking of the alternatives under consideration. This, however, does not prevent them from combining with scalar expressions and/or contexts evoking a scale.\il{Rapanui|(} For instance, in (\ref{exRapaNuiRestrictive}) the asserted event  \lq she fainted' can be understood as lying lower on a contextually evoked scale of affectedness than the negated\is{negation} proposition \lq she died/is dead\rq{ }contained in the immediately preceding clause.

\begin{exe}
	\ex Rapa Nui \label{exRapaNuiRestrictive}\\
	\gll 'Ina a Tiare kai mate; \textbf{ko} \textbf{rerehu} \textbf{nō} '\textbf{ā}.\\
	\textsc{neg} \textsc{art} T. \textsc{neg}.\textsc{pfv} die \textsc{ant} faint still/only \textsc{cont}\\
	\glt \lq Tiare was not dead; she \textbf{had just fainted}.' \parencite[343]{Kieviet2017}
\end{exe}

Although I discuss all expressions in \Cref{tableRestrictive} under a single heading, in many instances the exclusive function can be understood as forming a cluster of conceptually related uses. For instance, \ili{Kewa} \textit{pa} in (\ref{exKewaRestrictive3}) serves to depict the pig's eating of a sweet potato as requiring no effort, and Rapanui \textit{nō} in (\ref{exRapanuiRestrictive2}) signals that an act is performed \lq\lq{}[w]ithout further ado, without thinking, without taking other considerations into account" \parencite[343]{Kieviet2017}. With hosts other than a verbal predicate, a recurrent function is that of intensification or amplification, as in \il{Gurindji|(}(\ref{exGurindjiRestrictiveIntensification}). For an extensive discussion of the various functions of the relevant Australian expressions, I refer the reader to \textcite{SchultzeBerndt2002}.

\begin{exe}
	\ex \ili{Kewa}\label{exKewaRestrictive3}\\
	\gll Sapi adaa-ai \textbf{pa} \textbf{maa} \textbf{ne}-\textbf{a} robo-re ora adaa-ai popa a-ya.\\
	sweet\_potato big-\textsc{nom} still take eat-\textsc{prs}:3\textsc{sg} when-\textsc{top} really big-\textsc{nom} come stand-\textsc{prs}:3\textsc{sg}\\
	\glt \lq When it [pig] takes a sweet potato which is a big one and \textbf{eats it} (\textbf{without much effort}), it really becomes a big one.' \parencite[286]{Yarapea2006}
	
	\ex Rapanui\label{exRapanuiRestrictive2}\\
	\gll ¿Kai haʼamā koe \textbf{i} \textbf{toʼo} \textbf{nō} \textbf{koe} \textbf{i} \textbf{te} \textbf{mauka} mo taʼo i taʼa ʼumu?\\
	\phantom{¿}\textsc{neg}.\textsc{pfv} ashamed 2\textsc{sg} \textsc{pfv} take still 2\textsc{sg} \textsc{acc} \textsc{art} grass for cook \textsc{acc} \textsc{poss}:2\textsc{sg} earth\_oven\\
	\glt \lq Werenʼt you ashamed, that \textbf{you just took the grass} (as fuel) to cook your earth oven (without asking, even though the grass was mine)?' \parencite[343]{Kieviet2017}\il{Rapanui|)}
	
	\ex Gurindji\label{exGurindjiRestrictiveIntensification}\\
	\gll kaput=\textbf{parni}\\
	morning=still\\
	\glt \lq early in the morning\rq{ }\parencite[19]{McConvell1983}
\end{exe}\il{Gurindji|)}

In some instances, the contribution of a specific expressions is ambiguous, being compatible with both, a reading of phasal polarity \textsc{still} and the exclusive function.\il{Gooniyandi|(} For instance, in (\ref{exRestrictiveGooniyandi2}) Gooniyandi \mbox{=\textit{nyali}} occurs on the locative predicate \textit{ngirnda marlaya} \lq in this hand' and, as \textcite[465–466]{McGregor1990} points out, the exact interpretation is dependent on the discourse context. An exclusive reading arises if the question under discussion pertains to the \lq where' of the object being held, whereas a phasal polarity reading answers the question of whether its location has changed.\il{Jaminjung|(}\il{Ngaliwurru|(} A similar instance is found in the Jaminjung-Ngaliwurru example (\ref{exRestrictiveJaminjung}), where the difference between the two interpretations is subtle, at best. I return to this issue below.

\begin{exe}
	\ex Gooniyandi\label{exRestrictiveGooniyandi2}\\
	\gll Mangaddi yoodgoowali ngirnda \textbf{marla}-\textbf{ya}=\textbf{nyali} goorijgila.\\
	\textsc{neg} I:was:putting:it:down this hand-\textsc{loc}=still I:hold:it\\
	\glt \lq It haven’t put it down; it’s \textbf{right here in my hand}/\textbf{still in my hand}.'  \parencite[465–466]{McGregor1990}\il{Gooniyandi|)}
	
	\ex Jaminjung-Ngaliwurru\label{exRestrictiveJaminjung}\\
	Context: From a bush trip narrative.\\
	\gll {Tongue hanging} gugu-wu, gugu \textbf{yagbali}-\textbf{g}=\textbf{gung} ga-gba=yirrag.\\
	{tongue hanging(<Kriol)} water-\textsc{dat} water place-\textsc{loc}=still 3\textsc{sg}-\textsc{cop}=1\textsc{pl}.\textsc{excl}.\textsc{dat}\\
	\glt \lq (His) tongue hanging out for water, (since) our \textbf{water was still} (\textbf{back}) \textbf{in} \textbf{the} \textbf{camp}/\textbf{right} \textbf{in} \textbf{the} \textbf{camp}.' (Eva Schultze-Berndt, p.c.)
\end{exe}\il{Jaminjung|)}\il{Ngaliwurru|)}

\il{Gurindji|(}Gurindji has morphological means of disambiguating the two functions. A reduplicated form \mbox{=\textit{rnirni}} can only serve as an exclusive operator, never as a phasal polarity expression. This is shown in (\ref{exRestrictiveDisambiguation}). 
\is{syntax|(}With \ili{Acehnese} \textit{mantöng}, on the other hand, the distinction between the two functions is reflected in  syntax. While \textit{mantöng} as \lq only\rq{ }always stands to the right of its focus, \textit{mantöng} in phasal polarity function can either precede or follow the predicate; see (\ref{exRestrictiveAcehnese}).

\begin{exe}
	\ex Gurindji\label{exRestrictiveDisambiguation}\\
	\gll Ngu=rna \textbf{karrap}=\textbf{parnirni} nya-ya kurti=rni ngu=rna pirrkap=ma ma-nku.\\
	\textsc{aux}=\textsc{subj}.1\textsc{sg} look=still.\textsc{redupl} see-\textsc{pst} later=still \textsc{aux}=\textsc{subj}.1\textsc{sg} make=\textsc{top} get-\textsc{fut}\\
	\glt \lq I have \textbf{only looked} at it (the car); I’ll do the repairs later.'
	\\\parencite[23]{McConvell1983}\il{Gurindji|)}
 
	\ex \ili{Acehnese}\label{exRestrictiveAcehnese}\\
	\gll Jih sakêt \textbf{mantöng} \textup{/} Jih \textbf{mantöng} sakêt.\\
	3 sick still {} 3 still sick\\
	\glt \lq She is still sick.\rq{ }\parencite[224]{Durie1985}
\end{exe}\is{syntax|)}

\subsubsection{Discussion}\is{discontinuation|(}
At first sight, it may appear strange that an exclusive function and phasal polarity \textsc{still} are recurrently expressed by one and the same item. After all, in the familiar European languages, the functional extensions of \textsc{still} expressions typically lie in the realm of additivity.\is{additive} However, recall from \Cref{secFunctionalDiscussion} that my definition of \textsc{still} includes the evocation of a discontinuation scenario that \lq\lq figures in the discourse, and/or in the mind of the Speaker, of the Addressee, or both, as a serious alternative of the factual situation described in the sentence" \parencite[41]{vanBaar1997}. Similarly, non-scalar exclusive operators serve to exclude alternative propositions from the common ground that differ in the value of the marker's associated constituent. When the expression's focus is a durative situation that is depicted as ongoing and the alternative(s) under consideration amount(s) to the discontinuation of said situation, the two functions coincide. Similar observations, albeit in slightly different terms, have been made before me by \textcite[336]{Doehler2018}, \textcite{McConvell1983} and \textcite{SchultzeBerndt2002}.\il{Gooniyandi|(} This overlap  becomes particularly salient in examples like (\ref{exRestrictiveGooniyandi2}),\il{Komnzo|(} repeated below, or in (\ref{exRestrictiveKomnzo}). As I pointed out above, in (\ref{exRestrictiveGooniyandi2})  the salience of one reading over the other depends on the exact question under discussion, with both being mutually compatible. In (\ref{exRestrictiveKomnzo}) the denotation of the predicate \textit{rä} \lq she is (alive)' constitutes the obvious continuation\is{persistence} of a prior state and the explicitly mentioned alternative \textit{z} \textit{kwarsis} \textit{mnin} \lq she burned in the fire' amounts to its discontinuation.

\begin{exe}
	\exr{exRestrictiveGooniyandi2} Gooniyandi\\
	\gll Mangaddi yoodgoowali ngirnda \textbf{marla}-\textbf{ya}=\textbf{nyali} goorijgila.\\
	\textsc{neg} I:was:putting:it:down this hand-\textsc{loc}=still I:hold:it\\
	\glt \lq It haven’t put it down; it’s \textbf{right here in my hand} / \textbf{still in my hand}.\rq{ }\parencite[465–466]{McGregor1990}
    \il{Gooniyandi|)}
 
	\ex Komnzo\\
	Context: As a punishment for a murder, a woman has been sentenced to being burned alive.\label{exRestrictiveKomnzo}\\
	\gll Wati nagawa ŋabrigwa si=r. \textbf{Komnzo} \textbf{rä} o z kwa-rsir mni=n.\\
	then N. 2$|$3:\textsc{pst.ipfv}:return eye=\textsc{purp} still 3\textsc{sg.f}:\textsc{non.pst.ipfv}:\textsc{cop} or already 2$|$3\textsc{sg}:\textsc{rec}.\textsc{pst}.\textsc{ipfv}-burn fire=\textsc{loc}\\
	\glt \lq Then Nagawa $[$the woman's husband$]$ returned to check: was she \textbf{still alive} or did she burn in the fire?' (\cite[126]{Doehler2018}; \cite{Doehler2020})
\end{exe}\il{Komnzo|)}

\il{Gooniyandi|(}A similar argument can be made for an example like (\ref{exRestrictiveGooniyandi3}), where the meaning of the predicate \textit{giraddiri} \lq s/he crawls\rq{ }constitutes an intermediate step in an overarching sequence, which is contrasted with a denied subsequent stage.

\begin{exe}
	\ex Gooniyandi\label{exRestrictiveGooniyandi3}\\
	\gll Jiginya biddangi mangaddi wardgiri \textbf{giraddiri} \textbf{wamba}=\textbf{nyali}.\\
	child their not he:walks he:crawls later=still\\
	\glt \lq Their child doesn’t walk; he still crawls.\rq{ }\parencite[464]{McGregor1990}
\end{exe}\is{discontinuation|)}

The associated constituent of \mbox{=\textit{nyali}}\il{Gooniyandi} and \textit{komnzo}\il{Komnzo} in  (\ref{exRestrictiveGooniyandi2}, \ref{exRestrictiveKomnzo}, \ref{exRestrictiveGooniyandi3}) is the main predicate. Unsurprisingly, all relevant sample expressions can modify a predicate. That said, the same functional convergence can also arise in cases where the expression's focus is a lower constituent, as long as the resulting context propositions can be construed as yielding a new situation. For instance, in (\ref{exRestrictiveGooniyandi1}) Gooniyandi \mbox{=\textit{nyali}} occurs on the 1\textsc{sg} pronoun \textit{nganyi}, which is the patient argument of \textit{milaana} \lq he sees me'. The sentence can nonetheless be interpreted both ways, because \lq\lq in order to assert that someone is looking just at one person and not another, the event of \lq looking' must have continued\is{persistence} for some time" \parencite[254]{SchultzeBerndt2002}. 

\begin{exe}

	\ex Gooniyandi\label{exRestrictiveGooniyandi1}\\
	\gll Nganyi=\textbf{nyali} milaana mangaddi ngooddoo milaa.\\
	1\textsc{sg}=still he:sees:me \textsc{neg} that he:sees:him\\
	\glt \lq He’s still looking at me, not at him.' \parencite[464]{McGregor1990}
\end{exe}\il{Gooniyandi|)}

\textcite[412–421]{Sarvasy2017} gives one of the most detailed descriptions of an exclusive marker in Papunesia, namely of \mbox{=\textit{gon}} in the Nuclear Trans New Guinea language Nungon.\il{Nungon} Interestingly, she observes that the exclusive function often goes along with a broadly temporal one: \lq\lq \mbox{=\textit{gon}} has exclusive/restrictive and/or durative semantics: it can be translated as \lq only, just,' and also tends to indicate that the exclusivity lasts for some time." \parencite[412]{Sarvasy2017}. While phasal polarity \textsc{still} does not appear to form part of the denotata of \ili{Nungon} \mbox{=\textit{gon}}, if a similar durative notion were found with other non-scalar exclusives of the area, it would narrow the conceptual gap between the two functions even further. This last point brings me to the question of diachronicity. In her discussion of several Australian languages, \textcite{SchultzeBerndt2002} convincingly argues that the non-scalar exclusive function is older than the same markers serving as expressions for \textsc{still}. The exclusive function, in turn, in all likelihood goes back to the \lq\lq emphatic assertion of \isi{identity}" \parencite[120]{Koenig1991}. A clear case in point is \ili{Kayardild} \mbox{=(\textit{i})\textit{da}} < \textit{niida} \lq same' \parencite[389–390]{Evans1995}. The fact that a reduplicated form in \ili{Gurindji} only has the exclusive function likewise points to its primacy, and it is conceivable that a similar development from non-scalar exclusive to \textsc{still}  also accounts for at least some of the Papunesian cases. \ili{Chuwabu} \mbox{=\textit{vi}}, on the other hand, could be a reduced form of the \isi{additive} marker \textit{viina} \lq also, as well' \parencite{Guerois2021}, which is also involved in the formation of \textsc{no longer}.\is{no longer} Possibly then, the diachronic direction for this expression runs (\textsc{no longer} >) \textsc{still} > \lq only'.

\subsection{Thus far only}\label{sectionThusFarOnly}
\il{Xhosa|(}\is{scale|(}
\subsubsection{Introduction} In this subsection, I discuss a use in which a \textsc{still} expression signals a combination of temporal inclusion \lq thus far\rq{ }and negated\is{negation} existential quantification \lq only\rq{}. The \ili{Xhosa} example (\ref{exThusFarOnlyXhosa}) is an illustration. Here, the collocation of the \textsc{still} expression \mbox{\textit{sa}-} and the \isi{perfective} \isi{aspect} inflection signals that the speaker remains in a transitory state of having eaten nothing but mango.

\begin{exe}
	\ex \ili{Xhosa}\label{exThusFarOnlyXhosa}\\
	\gll Ndi-\textbf{sa}-ty-\textbf{e} imengo.\\
	\textsc{subj}.1\textsc{sg}-still-eat-\textsc{pfv} \textsc{ncl}9.mango\\
	\glt \lq (So far) I've only eaten mango (e.g. out of more foods that are on offer).'
	\\(my field notes)
\end{exe} 

The use I discuss here differs from the \textsc{still}-exclusive coexpression addressed in \Cref{sectionExclusive} in that temporal notions are always present. It is also distinct from the use of \textsc{still} expressions in scalar contexts (\Cref{sectionScalar}) in that it does not require the set of alternatives to be ordered. For instance, example (\ref{exThusFarOnlyXhosa}) presupposes no ranking of food items nor any particular order in which they would be consumed.

\subsubsection{Distribution in the sample} 
\Cref{tableThusFarOnly} lists the five expressions in my sample that have the \lq thus far only\rq{ }use. As can be gathered, these expressions stem from three macro-areas and come in two primary flavours. In the first type, which is found with three Bantu languages in the sample, the relevant reading arises only in collocation with the \isi{perfective} \isi{aspect} inflection. \il{Mateq|(}No such collocational restriction is found, on the other hand, with Mateq \textit{bayu} and \il{Culina|(}Culina \textit{-\textit{kha}}. In what follows, I first address the last two items and then turn to the Bantu cases.

\begin{table}
\caption{\lq Thus far only' use\label{tableThusFarOnly}}
\begin{tabular}{lllll}
	\lsptoprule
	Macro-Area & Language & Expression & Collocation & Appendix\\\midrule
	Africa & \ili{Ruuli} & \textit{kya}- & With \textsc{pfv}\is{aspect} & \ref{appendixRuuliRestrictive}\\
		   & Southern Ndebele\il{Ndebele, Southern} & \textit{sa}- & With \textsc{pfv} &\ref{appendixSouthernNdebeleRestrictive}\\
		   & Xhosa & \textit{sa}- & With \textsc{pfv} &\ref{appendixXhosaRestrictive}\\
	Papunesia & Mateq & \textit{bayu}\footnote{Inclusion in the sample is tentative.} & n/a & \ref{appendixMateqStill}, \\
	          &       &                                                               &     & \ref{appendixMateqThusFarOnly}\\
	S. America & Culina & -\textit{kha} & n/a & \ref{appendixCulinaStill} \\
	\lspbottomrule
\end{tabular}
\end{table}
\il{Xhosa|)}

\subsubsection{A closer look: Mateq \textit{bayu} and Culina -\textit{kha}} 
The examples in (\ref{exStillOnlyMateq1}, \ref{exStillOnlyCulina1}) illustrate Mateq \textit{bayu} and Culina \mbox{-\textit{kha}}. In both cases, the expressions signal not only the \isi{persistence} of one situation, but also evoke a \isi{discontinuation} in the form of a follow-up event.

\begin{exe}
		\ex Mateq\label{exStillOnlyMateq1}\\
		\gll \textbf{Bayu} kurak\sim kurak tubiq.\\
		still bubble\sim\textsc{redupl} rice\\
		\glt \lq The rice was [still] bubbling (implies an expectation that once the rice has bubbled it will be scooped out and eaten).\rq{ }\parencite[137–138]{Connell2013}
		
		\ex Culina\label{exStillOnlyCulina1}
		\begin{xlist}
			\exi{A:} \textit{Ami wadani?}\\
			\lq A mãe está dormindo? [Is mother sleeping?]\rq{}
			\exi{B:}
			\gll Nowe ra-ni \textbf{hapi} \textbf{na}-\textbf{kha}-\textbf{ni}.\\
			3:not\_be \textsc{aux}-\textsc{decl}.\textsc{f} bathe 3:\textsc{aux}-still-\textsc{decl}.\textsc{f}\\
			\glt \lq Não (está), ela ainda está tomando banho (antes de logo aparecer). [No she isn’t, \textbf{she's still bathing} (\textbf{before showing up after}.)]\rq{ }\parencite[184]{Tiss2004}
		\end{xlist}
\end{exe}	

As for Culina \mbox{-\textit{kha}}, \textcite[183]{Tiss2004} describes its function in examples like (\ref{exStillOnlyCulina1}) as \lq\lq before something else happens, the situation in question still continues\is{persistence} for a limited time\rq\rq{}.\footnote{In the original Portuguese: \lq\lq antes de algo novo acontecer, a situação em questão ainda continua por um período limitado\rq\rq{}.} While the descriptions by \textcite{Boyer2020} and \textcite[126]{Dienst2014} make no reference to a subsequent event, they both translate the marker as \lq still\rq{}.
When taken together, these descriptions give a relatively solid indication that \isi{persistence} forms part of the relevant sense of \mbox{-\textit{kha}}. The inclusion of Mateq \textit{bayu} in the sample is more tentative. According to \textcite[137]{Connell2013}, this expression \lq\lq indicates a type of \isi{imperfective} aspect.\is{aspect} More specifically, ... an action which temporally precedes\is{precedence} another anticipated action or event.\rq\rq{} \textcite{Connell2013} also applies the label \lq\lq imperfective" to \textit{mege} \lq still' and \textit{degeq} \lq constantly, keep doing'. I take this as an indication that he uses it as a synonym for persistence,\is{persistence} an interpretation that is supported by his gloss \lq still.only'. This would also explain why there are two attestations in \citeauthor{Connell2013}'s grammar that feature \textit{bayu} together with the event quantifier \textit{sidah} \lq once' and an \isi{anterior} viewpoint;\is{aspect} one of these examples is given in (\ref{exMateqStillOnlyOnce}). As I discuss in \Cref{sectionScalar}, in a language like \ili{English} this type of usage can be arrived at compositionally by a \textsc{still} expression taking scope over a restrictive operator; see (\ref{exThusFarOnlyEnglish}).

\begin{exe}
	\ex Mateq\label{exMateqStillOnlyOnce}\is{persistence}\\
	\gll Okoq \textbf{bayu} \textbf{sidah} téq.\\
	1\textsc{sg} still once this\\
	\glt \lq I've never done this before (lit. I've only done this once (i.e. now)).'  \parencite[150]{Connell2013}
	\ex \ili{English}\label{exThusFarOnlyEnglish}\\
	
	\textit{The Seasiders have \textbf{still only had one} game televised this season.}
	\\(found online)%\footnote{\url{https://www.blackpoolgazette.co.uk/sport/football/blackpool- fc/blackpool-overlooked-by-sky-sports-again-as-fans-bemoan-disparity- in-tv-revenue-3551200} (24 February, 2023).}
\end{exe}

Assuming that the above interpretations of the two markers are adequate, Mateq \textit{bayu} and Culina \mbox{-\textit{kha}} constitute enriched \textsc{still} expressions which require a \isi{discontinuation} scenario that corresponds to a follow-up situation.\il{Culina|)}\il{Mateq|)}

\subsubsection{A closer look: The Bantu cases}\il{Xhosa|(}\il{Ndebele, Southern|(}\il{Ruuli|(}\is{aspect|(}\is{perfective|(} I now turn to Ruuli \mbox{\textit{kya}-} and Southern Ndebele and Xhosa \mbox{\textit{sa}-}. With these expressions a \lq thus far only\rq{ }reading requires the perfective aspect inflection. This is shown in the Xhosa illustration in (\ref{exThusFarOnlyXhosa}) above and in the Ruuli example (\ref{exThusFarOnlyRuuli1}). In (\ref{exThusFarOnlyRuuli1}), the addition of restrictive \textit{ereere} appears to emphasise the restrictive reading.

\begin{exe}
	\ex Ruuli\label{exThusFarOnlyRuuli1}\\
	\gll Yee	n-\textbf{kya}-byal-\textbf{ire} ba-ala 	ba-ereere\\
	yes	\textsc{subj}.1\textsc{sg}-still-give\_birth\_to-\textsc{pfv}	\textsc{ncl}2-girl \textsc{ncl}2-only\\
	\glt \lq Yes, I have so far given birth only to girls.\rq{ }\parencite[87]{MolochievaEtAl2021}
\end{exe}

\is{actionality|(}
In light of the inflectional restriction, a brief excursion about the interplay between grammatical aspect and the lexical make-up of situations in Bantu is in order. A characteristic feature shared by most languages from this family is that many ongoing states are depicted via the combination  of so-called \lq\lq inchoative" verbs together with what is commonly referred to as the \lq\lq perfective" or \lq\lq perfect\rq\rq{ }inflection; see  \textcite{CranePersohn2019} for an overview. Example (\ref{exBantuState}) illustrates this with the Xhosa verb \textit{dana} \lq be(come) disappointed'. Crucially, in the stative reading the perfective aspect inflection is compatible with the \textsc{still} expression \mbox{\textit{sa}-}, as shown in (\ref{exBantuStateStill}).

\begin{exe}
	\ex 	 
	\begin{xlist}
		\exi{}Xhosa
		\ex\label{exBantuState}
		 \gll U-Sipho u-dan-ile.\\
		\textsc{ncl}1a-S. \textsc{subj}.\textsc{ncl}1-be(come)\_disappointed-\textsc{pfv}\\
		\glt \lq Sipho is disappointed.' 
		\ex\label{exBantuStateStill}
		\gll U-Sipho u-\textbf{sa}-dan-\textbf{ile}\\
		\textsc{ncl}1a-S. \textsc{subj}.\textsc{ncl}1-still-be(come)\_disappointed-\textsc{pfv}\\
		\glt \lq Sipho is still disappointed.' (my field notes)
	\end{xlist}
\end{exe}

Given the central role the perfective aspect paradigms play in the depiction of stative situations, it comes as no surprise that such a construal can sometimes be coerced and/or has become idiomatic with verbs that do not \textit{per} \textit{se} lexicalise a resultant state. For instance, Southern Ndebele \textit{akha} \lq build' allows for a stative construal as \lq live, inhabit' in the perfective aspect, a collocation that is compatible with the \textsc{still} expression \mbox{\textit{sa}-}; see (\ref{exSouthernNdebeleCoercion}).

\begin{exe}
	\ex Southern Ndebele \label{exSouthernNdebeleCoercion}\\
	\gll \textbf{U}-\textbf{s}-\textbf{akh}-\textbf{e} phambi kw-esi-tolo.\\
	\textsc{subj}.\textsc{ncl}1-still-build-\textsc{pfv} front \textsc{loc}-of.\textsc{ncl}9-store\\
	\glt \lq S/he is \textbf{still living} in front of the store (lit. … has still built …).' \\
	cf. \#{}\textit{Usakhe indlu}. (intended: \lq S/he has still built a house.')\\\parencite[275]{CranePersohn2021}
\end{exe}

\subsubsection{Discussion: The Bantu cases} Against the backdrop of the facts just laid out, \textcite{CranePersohn2021} suggest that the \lq thus far only' reading in Bantu also has its root in coercion. A class of predicates that readily offer themselves to this process are those that describe telic\is{telicity} situations which form one among several steps in a series, such as laying eggs \Rightarrow{ }brooding \Rightarrow{ }raising chicks in (\ref{exSouthernNdebeleLayEggs}). With these predicates, once the situation depicted in the clause culminates, the subject argument can easily be construed as being in an interim state brought about by one event and preceding a second one.

\begin{exe}
	\ex Southern Ndebele\label{exSouthernNdebeleLayEggs}\\
	 \gll Ikukhu \textbf{i}-\textbf{sa}-\textbf{bekele}.\\
	 \textsc{ncl}9.chicken \textsc{subj}.\textsc{ncl}9-still-lay\_eggs.\textsc{pfv}\\
	 \glt \lq The chicken has (\textbf{only}) \textbf{laid} \textbf{these} \textbf{eggs} (i.e. it has not yet started brooding).' \parencite[275]{CranePersohn2021}
\end{exe} \is{actionality|)}

While \textcite{CranePersohn2021} only deal with Southern Ndebele and Xhosa, the Ruuli example (\ref{exThusFarOnlyRuuli2}) suggests a convergent evolution. Here, the weaning of a child is an intermediate stage in a natural development sequence.\footnote{The first instance of \textit{kya}- in (\ref{exThusFarOnlyRuuli2}) forms part of a \textsc{not yet}\is{not yet} construction (\lq were still to bring > hadn't yet brought\rq{}); see \Cref{sectionNotYet}. \is{subordination}In subordinate contexts this construction serves to mark precedence\is{precedence} (\lq when they hadn't brought yet > before they brought\rq{}); see \Cref{sectionBefore}.}

\begin{exe}
	\ex Ruuli\label{exThusFarOnlyRuuli2}\\
	\gll Eirai budi ni-ba-kya-li ku-leeta bi-ni eby-a nasare, aka-ana \textbf{ka}-\textbf{kya}-\textbf{zw}-\textbf{ire}=\textbf{mbe} \textbf{oku} \textbf{ma}-\textbf{beere} nti ka-ab-e ka-tandika oku-soma.\\
	in\_the\_past previously when-\textsc{subj}.\textsc{ncl}2-still-\textsc{cop} \textsc{ncl}15(\textsc{inf})-bring \textsc{ncl}8-\textsc{prox} \textsc{ncl}9-\textsc{assoc} nursery \textsc{ncl}12-child \textsc{subj}.\textsc{ncl}12-still-abandon-\textsc{pfv}=\textsc{foc} \textsc{ncl}17(\textsc{loc}) \textsc{ncl}6-breast \textsc{comp} \textsc{subj}.\textsc{ncl}12-go-\textsc{sbjv} \textsc{subj}.\textsc{ncl}12-start \textsc{ncl}15(\textsc{inf})-read\\
	\glt \lq Long ago, before they brought these nurseries, [when] a child \textbf{has only stopped breastfeeding}, it would go and start studying.'
	\\(\cite{RuuliCorpus}, glosses added)
\end{exe}

Crucially though, as examples like (\ref{exThusFarOnlyXhosa}, \ref{exThusFarOnlyRuuli1}) above show, in the Bantu languages this meaning has become more general to the extent that an inherent ranking or fixed sequence (other than the mere possibility of events occurring after one another) is no longer a prerequisite, similar to the cases of \ili{Culina} \mbox{-\textit{kha}} and \ili{Mateq} \textit{bayu}.\il{Xhosa|)}\il{Ndebele, Southern|(}\il{Ruuli|)}\is{aspect|)}\is{perfective|)}

\subsection{Scalar restrictive}
\label{sectionScalarRestrictive}
\subsubsection{Introduction}
\il{Quechua, Huallaga-Huánuco|(}In this section, I discuss the use of \textsc{still} expressions as scalar restrictive operators. Example (\ref{exScalarRestrictiveQuechuaExtraordinary1}) is an illustration. Here, Huallaga-Huánuco Quechua \mbox{-\textit{raq}} serves the function of \lq\lq indicat[ing] that the action/event of the clause was an extreme measure, i.e., not carried out to an ordinary degree or applied to the ordinary ob­jects." \parencite[389]{Weber1989}.

\begin{exe}
	\ex Huallaga-Huánuco Quechua \label{exScalarRestrictiveQuechuaExtraordinary1}\\
		\gll \textbf{Sasa}-\textbf{ta}-\textbf{raq}-shi hichqa-yku-n awkin-qa.\\
	difficult-\textsc{obj}-still-\textsc{evid} strike-\textsc{pfv}-3 old\_man-\textsc{top}\\
	\glt \lq With difficulty, the old man strikes the match (\textit{sasataraqshi} implies that it was \textbf{only} \textbf{with} \textbf{considerable} \textbf{difficulty} that the old man was able to manage striking a match).' \parencite[389]{Weber1989}
\end{exe}

\subsubsection{Distribution in the sample}

\Cref{tableScalarRestrictive} lists the four expressions in my sample which have a scalar restrictive use. As can be gathered, these are found in two geographically adjacent Romance languages, and in two unrelated languages from central/western South America. One out of the four expressions, Huallaga-Huánuco Quechua \mbox{-\textit{raq}}, is an exclusive operator, i.e. it signals that none of the context propositions possibly holds true. The other three are \textit{at least}-type restrictives. In the \ili{French} and \ili{Spanish} cases, this function is restricted to the protases of counterfactual conditionals.\is{conditional}

\begin{table}
	\caption{Scalar restrictive functions\label{tableScalarRestrictive}}
	\small
		\fittable{\begin{tabular}{llllll}
			\lsptoprule	
			M.-area & Language & Expr. & Constraint & Type & Appx.\\
			\midrule
			Eurasia & \ili{French} & \textit{encore} & Counterfact. cond.\is{conditional} & \textit{At least} type & \ref{appendixFrenchEncoreIfOnly}\\
			& \ili{Spanish} & \textit{todavía} &  Counterfact. cond. &\textit{At least} type{}&  \ref{appendixSpanishTodaviaIfAtLeast}\\
			S. America & H.-H. Quechua & -\textit{raq} & n/a & Exclusive & \ref{appendixQuechuaExtraordinary}\\
			& \ili{Movima} & \textit{diːra}(\textit{n}) & n/a &   \textit{At least} type & \ref{appendixMovimaAtLeast}\\
			\lspbottomrule
		\end{tabular}}
\end{table}

In what follows, I first take a closer look at the case of \mbox{-\textit{raq}} and then move on to \ili{Movima}  \mbox{\textit{diːra}(\textit{n})}. Lastly, I discuss the case of \ili{French} \textit{encore} and \ili{Spanish} \textit{todavía} in conjunction with counterfactual protases.

\subsubsection{A closer look and discussion: H.-H. Quechua \mbox{-\textit{raq}}} The one scalar exclusive expression in my sample, Huallaga-Huánuco Quechua, was illustrated in (\ref{exScalarRestrictiveQuechuaExtraordinary1}) above. Another example is given in (\ref{exScalarRestrictiveQuechuaExtraordinary2}).

\begin{exe}
	\ex Huallaga-Huánuco Quechua\label{exScalarRestrictiveQuechuaExtraordinary2}\\
	Context: A husband has been informed that his wife is having an affair with another man.\\
	\gll Lulla-ku-nki-chari \textbf{kiki:}-\textbf{raq}-\textbf{mi} warmi-:-ta watqa-yku-shaq. Rika-yku-shaq.\\
	lie-\textsc{refl}-2-surely self:\textsc{poss}.1-still-\textsc{evid} wife-\textsc{poss}.1-\textsc{obj} spy-\textsc{pfv}-\textsc{fut}.1 see-\textsc{pfv}-\textsc{fut}.1\\
	\glt \lq You must be lying! \textbf{I myself} will spy on my wife. I will see. (implies that \textbf{nothing} will determine the truth \textbf{short} \textbf{of} the speakerʼs spying on his wife)' \parencite[132, 389]{Weber1989}
\end{exe}

As can be seen in (\ref{exScalarRestrictiveQuechuaExtraordinary1}, \ref{exScalarRestrictiveQuechuaExtraordinary2}), with \mbox{-\textit{raq}} as a scalar restrictive, the focus denotation ranks higher on some salient scale than the alternatives under consideration. Thus, in (\ref{exScalarRestrictiveQuechuaExtraordinary1}) contextual alternatives include \lq with ease' and \lq with an average effort', and in (\ref{exScalarRestrictiveQuechuaExtraordinary2}) the discourse environment suggests second-hand information as a value ranking lower on an epistemic scale. The scalar restrictive function of \mbox{-\textit{raq}} appears to also underlie several idiomatic uses, such as the construction illustrated in (\ref{exScalarRestrictiveQuechuaSureIs}), where it is conceivable that \lq it sure is hot' goes back to \lq nothing less than this is truly hot!'. Cases like (\ref{exScalarRestrictiveQuechuaExtraordinary2}) might motivate the dubitative collocation \mbox{-\textit{chu}}\mbox{-\textit{raq}} \lq \textsc{q}-still', illustrated in (\ref{exScalarRestrictiveQuechuaDubitative}).\pagebreak

\begin{exe}
	\ex 
	\begin{xlist}
		\exi{} Huallaga-Húanuco Quechua
		\ex \label{exScalarRestrictiveQuechuaSureIs}
		\gll \textbf{Chay}-\textbf{raq} aka-y-lla-q.\\
		that-still be\_hot-\textsc{inf}-only-\textsc{top}\\
		\glt \lq It sure is hot!'
	
		\ex \label{exScalarRestrictiveQuechuaDubitative}
		\gll Kanan hunaq tamya-nqa-\textbf{chu}-\textbf{raq}?\\
		now day rain-\textsc{fut}.3-\textsc{q}-still\\
		\glt \lq Will it perhaps rain today?'
		\\\parencite[327, 446]{Weber1989}
	\end{xlist}
\end{exe}

In addition, \mbox{-\textit{raq}} also serves as a scalar exclusive operator with temporal frame expressions (\Cref{sectionTimeScalarRestrictive}). Like the cases in (\ref{exScalarRestrictiveQuechuaExtraordinary1}, \ref{exScalarRestrictiveQuechuaExtraordinary2}), time-scalar restrictive \mbox{-\textit{raq}} requires lower alternatives in the form of earlier times. What is more, some examples in \citeauthor{Weber1989}'s grammar are compatible with both readings. For instance, example (\ref{exScalarRestrictiveQuechuaExtraordinary3}) involves an earliest time (once a decision has been made, possibly preceded by consulting with other authorities) as well as a specific degree (the highest authority).

\begin{exe}
	\ex Huallaga-Huánuco Quechua\label{exScalarRestrictiveQuechuaExtraordinary3}\\
	\gll \textbf{Hatun} \textbf{awturidaa}-\textbf{chaw}-\textbf{raq}-mi musya-ka:-shun kapital-chaw-\textbf{raq}-mi.\\
	big authority-\textsc{loc}-still-\textsc{evid} know-\textsc{pass}-\textsc{fut}:1\textsc{pl}.\textsc{incl} capital-\textsc{loc}-still-\textsc{evid}\\
	\glt \lq We will find out \textbf{only} \textbf{at} \textbf{the} \textbf{higher} \textbf{authority}, \textbf{in} \textbf{the} \textbf{capital} (and we will not find out any sooner).' \parencite[389]{Weber1989}	
\end{exe} 

The shared requirement that the common ground contain lower alternatives sets the time-scalar and general scalar restrictive functions of \mbox{-\textit{raq}} apart from a third use, namely as \lq for now, first' (\Cref{sectionFirst}). The latter is illustrated in (\ref{exScalarRestrictiveQuechuaFirst}), where the act of forming a skein is related to more advanced steps in a series.

\begin{exe}
	\ex Huallaga-Huánuco Quechua\label{exScalarRestrictiveQuechuaFirst}\\
	\gll Ese nuwal-wan tiñi-rku-r \textbf{atapa}-\textbf{nchiː}-\textbf{raq}.\\
	that walnut-\textsc{com} dye-then-\textsc{adv} form\_skein-1\textsc{pl}.\textsc{incl}-still\\
	\glt \lq Dying it with that walnut, \textbf{we} \textbf{first} \textbf{form} \textbf{a} \textbf{skein}.' \parencite[388]{Weber1989}
\end{exe}	

\is{precedence|(} 
On the diachronic plane, \textcite[90–91]{vanBaar1997} reports an examination of several Quechuan languages and suggests that the \lq for now, first\rq{ }use constitutes the historically original function of \mbox{-\textit{raq}} and its cognates. In this scenario, the  time-scalar restrictive \lq not until\rq{ }function of the same items could easily go back to a reanalysis of \lq first \textit{p}, \textit{q}' as \lq not until \textit{p}, \textit{q}'. The more general restrictive function I discuss here, which shares the scale reversal with the temporal restrictive one, would then have been brought about by bridging contexts like (\ref{exScalarRestrictiveQuechuaExtraordinary3}). Assuming that \citeauthor{vanBaar1997} is on the right track, then \mbox{-\textit{raq}} as an exponent of \textsc{still} would be a separate offspring. As I discuss in some more detail in \Cref{sectionFirst}, it is likely that the development from \lq first, for now\rq{ }to marking \isi{persistence} was preceded, or catalysed, by the expression of \mbox{\textsc{not yet}\is{not yet} \textit{p}} via \lq \mbox{\neg{}\textit{p}} for now\rq{}, which allows for an alternative parsing as \textsc{still} \mbox{\neg{}\textit{p}}. \Cref{figureRestrictiveQuechua} is a graphic illustration of this assumed chain of developments. More comparative work on Quechuan is needed to see whether it holds up against further scrutiny.
\setlength{\MinimumHeight}{1.5cm}
\setlength{\MinimumWidth}{\widthof{\lq no less thanx\rq{}}}
\setlength{\MinimumWidthB}{\widthof{(with negx.)}}
\begin{figure}
	\caption{Hypothetical, partial diachronic network for H.-H. Quechua \mbox{-\textit{raq}\is{not yet}}		
	\label{figureRestrictiveQuechua}}
	\begin{tikzpicture}[node distance=1em]
		\node [rectangle, rounded corners, minimum height=\MinimumHeight,text width=\MinimumWidth, draw=black, text centered] (first) {\strut{}\lq first,\\for now\rq\strut};
		  \node [rectangle, rounded corners, minimum height=\MinimumHeight, text width=\MinimumWidth, draw=black, text centered, below= of first] (notyet)
            {\strut{}\textsc{not yet}\\(with neg.\strut{})};    
	   	\node [rectangle, rounded corners, minimum height=\MinimumHeight,text width=\MinimumWidth, draw=black, text centered, right=of notyet] (still) {\textsc{still}}; 
		\node [rectangle, rounded corners, minimum height=\MinimumHeight,text width=\MinimumWidth, draw=black, text centered, right=of first]    (tscalar) {\strut{}Temp. excl.\\\strut{} \lq not until\rq}; 
	\node [rectangle, rounded corners, minimum height=\MinimumHeight,
            text width=\MinimumWidth, draw=black, text centered, right= of tscalar]    
           (scalar) {\strut{}Scalar excl.\\\strut{}\lq no less than'};      
		\draw [-latex, very thick] (first) to (notyet);
		\draw [-latex, very thick] (notyet) to (still);
		\draw [-latex, very thick, densely dotted, color=gray] (first.south east) to (still.north west);
		\draw [-latex, very thick]  (first) to (tscalar);
		\draw [-latex, very thick] (tscalar) to (scalar);
	\end{tikzpicture}  
\end{figure}
\is{precedence|)}\il{Quechua, Huallaga-Huánuco|)} 

\subsubsection{A closer look and discussion: Movima \mbox{\textit{di:ra}(\textit{n})}}\il{Movima|(}\is{modality|(} 
Moving over to the \textit{at least}-type restrictives in my sample, example (\ref{exScalarRestrictiveMovima1}) illustrates Movima \mbox{\textit{di:ra}(\textit{n})}. Like the evaluative use of \ili{English} \textit{at least} (\cite{Kay1992}; \cite{Gast2012}), \mbox{\textit{di:ra}(\textit{n})} invokes a bouletic scale. Thus, in (\ref{exScalarRestrictiveMovima1}) seeing a path is subjectively\is{subjectivity} evaluated as a positive development, though it does not yield the maximum degree of contentment.

\begin{exe}
	\ex Movima\label{exScalarRestrictiveMovima1}\\
	Context: About how a place has changed over the course of the years.\\
	\gll Jayna ney \textbf{di:ra}, \textbf{di:ra} ay jayna de:deye, jayna ay dede:ye is to:wa neyru jema'a.\\
	already here still still \textsc{prox}.\textsc{n} already see.\textsc{impr} already here see.\textsc{impr} \textsc{pl} path here also\\
	\glt \lq Ya aquí siquiera, siquiera ya se ve esto, ya siquiera se ve un camino aquí.' [Now here you can \textbf{at} \textbf{least} see … now you can \textbf{at} \textbf{least} see a path.]\rq{ }\parencite{MovimaCorpus}
\end{exe}

Example (\ref{exScalarRestrictiveMovima2}) is another illustration. Again, the common ground provides for propositions with higher-ranking contextual implications, corresponding to a less gruesome death, but also a lower ranking one in which the late witch receives no dignity whatsoever. The lower-ranking propositions are implied, for all rhetoric purposes, whereas the higher-ranking ones are false (cf. \cite{Gast2012}). 
		
\begin{exe}
	\ex Movima\label{exScalarRestrictiveMovima2}\\
	Context: A woman who had the magic power of transforming into a jaguar has been caught and her jaguar hide has been burned, leading to her gruesome death.\\
	\gll Jayna \textbf{diran} oso' wulako='is.\\
	already still \textsc{dem}.\textsc{n}.\textsc{pst} bury=3\textsc{pl}\\
	\glt \lq Ya siquiera la enterraron (le dieron una sepultura cristiana porque era humana). [They \textbf{at least} buried her (they had a Christian funeral for her, because she was human).]\rq{ }\parencite{MovimaCorpus}
\end{exe}	

There is a striking conceptual similarity between \mbox{\textit{di:ra}(\textit{n})} as an exponent of \textsc{still} and its \lq at least' function. Thus, the phasal polarity function involves a situation that has progressed beyond its onset, but which has not reached its conceivable end. Likewise, \mbox{\textit{di:ra}(\textit{n})} as a scalar restrictive marker involves a degree of evaluation that lies above the minimum, but below the conceivable maximum. The main differences lie in the nature of the scales and the status of the cut-off point. \Cref{figureScalarRestrictiveMovima} is a graphic illustration of these parallels.\is{marginality|(} There is also a striking similarity to the \textsc{still}-as-\isi{marginality} use (\Cref{sectionMarginality}), except for the inverse ordering relation that goes along with the latter.

\begin{figure}[htb]
	\centering
	\begin{subfigure}[b]{0.48\linewidth}\is{topic time}
		\centering
		\begin{tikzpicture}[node distance = 0pt]
			\node[mynode, text width=\schmal, fill=cyan, very near start] (spain){Situation};
			\node[mynode, fill=none, text width=\superschmal,  right= of spain]
 (france){(\small{$\Diamond$} \neg{}Sit.)};
			\draw[-, densely dashed, label distance=0] (spain.north west) to ($(spain.north west) + (0,-\hoehe*2) $) node [below] {Onset};
			\draw[-, densely dashed] ($(spain.north east)+(-0.2pt,0)$) to ($(spain.north east) + (-0.2pt,-\hoehe*2) $);
		\draw[-, densely dashed] ($(spain.north east)+(-1*\hoehe+2pt,0)$) to ($(spain.north east)+(-1*\hoehe+2pt,0)+ (0,-\hoehe*2) $) node [below, xshift=0.5*\hoehe,align=center] {Topic\\time};			
			\draw[-latex, line width=0.5pt]  (spain.south west) to  ($(france.south east)+(1ex,0)$) node [right] {t};
		\end{tikzpicture}
	\subcaption{\textsc{still}}	
	\end{subfigure}
	\begin{subfigure}[b]{0.48\linewidth}
		\centering

		\begin{tikzpicture}[node distance = 0pt]
			\node[mynode, text width=\schmal, fill=cyan, very near start] (spain){True};
			\node[mynode,  text width=\superschmal, right= of spain] (france){False};
			\draw[-, thick, label distance=0] (spain.north west) to ($(spain.north west) + (0,-\hoehe*2) $) node [below] {Min.};
			\draw[-, densely dashed] (spain.north east) to ($(spain.north east) + (0,-\hoehe*2) $)
		node [below, align=center] {Focus\\value};				
			\draw[-latex, line width=0.5pt]  (spain.south west) to  (france.south east) node [right] {eval.};
		\end{tikzpicture}	
	\subcaption{\textit{at least}-type restrictive\label{figureScalarRestrictiveMovima2}}	
	\end{subfigure}
	\caption{Graphic illustration of Movima \textit{di:ra}(\textit{n})\label{figureScalarRestrictiveMovima}	}
\end{figure}\il{Movima|)}

\subsubsection{A closer look and discussion: French \textit{encore} and Spanish \textit{todavía}}\il{French|(}\il{Spanish|(}\is{conditional|(} The topic of marginality brings me to the remaining two expressions in \Cref{tableScalarRestrictive}, French \textit{encore} and Spanish \textit{todavía}. These serve as \textit{at least}-type restrictives whose focus is the protasis of a counterfactual \isi{conditional} construction, where they signal a minimal requirement. This is illustrated in (\ref{exMarginalExtensionsSpanishConditional1}) for Spanish \textit{todavía}, where having children is depicted as (close to) the lowest conceivable degree of responsibilities that would justify saving up money. French \textit{encore} is shown in (\ref{exMarginalExtensionsFrenchConditional1}), where it is signalled that standing straight is the least one has to do to have so much as a chance of not being pointed out; note the two possible orders \textit{encore si}/\textit{si encore}.

\begin{exe}
	\ex
	 Spanish\label{exMarginalExtensionsSpanishConditional1}\\
	\gll ¿Para qué ahorr-as?; \textbf{todavía} \textbf{si} \textbf{tuvier}-\textbf{as} \textbf{hijo}-\textbf{s} esta-ría justificado.\\
	\phantom{¿}for what save\_money-2\textsc{sg} still if have.\textsc{pst}.\textsc{sbjv}-2\textsc{sg} child-\textsc{pl} \textsc{cop}-\textsc{cond}.3\textsc{sg} justified\\
	\glt \lq What are you saving money for? \textbf{If you at least had kids}, then it would make sense.' (\cite[s.v. \textit{todavía}]{RAEDictionary}, glosses added)

	\ex French\label{exMarginalExtensionsFrenchConditional1}\\
	\gll \textbf{Si} \textbf{encore} \textup{\textbf{/}} \textbf{encore} \textbf{si} \textbf{tu} \textbf{te} \textbf{ten}-\textbf{ais} \textbf{droit}, tu aurais quelque chance de ne pas te faire montrer du doigt.\\
	if still {} still if 2\textsc{sg} \textsc{refl}.2\textsc{sg} keep-\textsc{pst}.\textsc{ipfv}.2\textsc{sg} straight 2\textsc{sg} have.\textsc{cond}.2\textsc{sg} some chance of \textsc{neg} \textsc{neg} \textsc{obj}.2\textsc{sg} make.\textsc{inf} show.\textsc{inf} of:\textsc{def}.\textsc{sg}.\textsc{m} finger(\textsc{m})\\
	\glt \lq \textbf{If you at least stood straight}, you'd have some chance of not being pointed out.'  (\cite[83]{VictorriFuchs1996}, glosses added)
\end{exe}

Both cases can be traced back to \textit{encore} and \textit{todavía} as markers of marginality. To briefly summarise from \Cref{sectionMarginality}, the marginality use of \textsc{still} expressions addresses the question of whether or not a given entity falls within the bounds of a specific scale, such as the Spanish territory in (\ref{exScalarRestrictiveIrunFrench}, \ref{exScalarRestrictiveIrunSpanish}). As in these examples, marginality often invites a contrast between the topical\is{topic} argument and entities that fall outside the relevant scale.

\begin{exe}
	\ex\label{exScalarRestrictiveIrun}
		\begin{xlist}
	 \ex French\label{exScalarRestrictiveIrunFrench}\\
	\gll \textbf{Irun}, \textbf{c'}-\textbf{est} \textbf{encore} l'-Espagne (et Hendaye, c'-est déjà la France).\\
	I. \textsc{prox}.\textsc{sg}.\textsc{m}-\textsc{cop}.3\textsc{sg} still \textsc{def}.\textsc{sg}-Spain \phantom{(}and H. \textsc{prox}.\textsc{sg}.\textsc{m}-\textsc{cop}.3\textsc{sg} already \textsc{def}.\textsc{sg}.\textsc{f} France(\textsc{f})\\
	\pagebreak
	\ex\label{exScalarRestrictiveIrunSpanish}Spanish\\
	\gll \textbf{Irún} \textbf{todavía} \textbf{es} \textbf{España} (y Hendaya ya es Francia).\\
	I. still \textsc{cop}.3\textsc{sg} Spain  \phantom{(}and H. already \textsc{cop}.3\textsc{sg} France\\
	\glt both: \lq\textbf{Irún is still in Spain} (and Hendaya is in France already).'\\ (\cite[58]{Garrido1991}, personal knowledge)
	\end{xlist}
\end{exe}

With this in mind, \textcite[83–84]{VictorriFuchs1996} point out that the initial French example (\ref{exMarginalExtensionsFrenchConditional1}) can be paraphrased with \textit{encore} as a marker of marginality inside the apodosis clause; see (\ref{exMarginalExtensionsFrenchConditional1mod}). It then anaphorically refers to the proposition expressed in the antecedent clause, ranking it on a scale of efforts and evoking a contrast with alternative propositions. The same holds true for the Spanish example (\ref{exMarginalExtensionsSpanishConditional1}), as shown in (\ref{exMarginalExtensionsSpanishConditional1mod}).

\begin{exe}

	\ex French\label{exMarginalExtensionsFrenchConditional1mod}\\
\gll Si tu te ten-ais droit, \textbf{tu} \textbf{aurais} \textbf{encore} \textbf{quelque} \textbf{chance} de ne pas te faire montrer du doigt.\\
if 2\textsc{sg} \textsc{refl}.2\textsc{sg} keep-\textsc{pst}.\textsc{ipfv}.2\textsc{sg} straight 2\textsc{sg} have.\textsc{cond}.2\textsc{sg} still some chance of \textsc{neg} \textsc{neg} \textsc{obj}.2\textsc{sg} make.\textsc{inf} show of:\textsc{def}.\textsc{sg}.\textsc{m} finger(\textsc{m})\\
	\glt \lq If you stood straight, \textbf{then you'd still have some chance} of not being pointed out.' (\cite[83]{VictorriFuchs1996}, glosses added)
	
\ex Spanish\label{exMarginalExtensionsSpanishConditional1mod}\\
\gll ¿Para qué ahorr-as?; si tuvier-as hijo-s \textbf{todavía} \textbf{est}-\textbf{aría} \textbf{justificado}.\\
	\phantom{¿}for what save\_money-2\textsc{sg} if have.\textsc{pst}.\textsc{sbjv}-2\textsc{sg} child-\textsc{pl} still \textsc{cop}-\textsc{cond}.3\textsc{sg} justified\\
	\glt \lq What are you saving money for? If you had kids \textbf{it would still make sense}.' (personal knowledge)
\end{exe}

\is{syntax|(}What is more, French \textit{encore} is sometimes prosodically\is{prosody} set off from \textit{si} \lq if'. This is likely to have syntactic correlates, such that (\ref{exMarginalExtensionsFrenchSeparation}) can be parsed with \textit{encore} inside the apodosis clause, while the protasis constitutes a parenthetical insertion. In the same vein, the Spanish example (\ref{exMarginalExtensionsSpanishConditional1}), repeated below, could also receive a reading in which \textit{todavía} forms part of the apodosis clause \lq it would still make sense\rq{}, from which it is separated by a parenthetical protasis \lq if you had kids\rq{}.

\begin{exe}
\ex French\label{exMarginalExtensionsFrenchSeparation}\\
\gll \textbf{Encore}, si c'-était six heure-s et demie, \textbf{vous} \textbf{pourriez} \textbf{décider} \textbf{de} \textbf{vous} \textbf{lever}.\\
still if \textsc{prox}.\textsc{sg}.\textsc{m}-\textsc{cop}.\textsc{pst}.\textsc{ipfv}.3\textsc{sg} six hour-\textsc{pl} and half(\textsc{f}) 2\textsc{pl}
 can.\textsc{cond}.2\textsc{pl} decide.\textsc{inf} of \textsc{refl}.2\textsc{pl} get\_up.\textsc{inf}\\
\glt \lq If it was half past six, then you'd still decide to get up.'
\\(\cite[83]{VictorriFuchs1996}, glosses added)

\exr{exMarginalExtensionsSpanishConditional1} Spanish\\
	\gll ¿Para qué ahorr-as?; \textbf{todavía} si tuvier-as hijo-s \textbf{esta}-\textbf{ría} \textbf{justificado}.\\
	\phantom{¿}for what save\_money-2\textsc{sg} still if have.\textsc{pst}.\textsc{sbjv}-2\textsc{sg} child-\textsc{pl} \textsc{cop}-\textsc{cond}.3\textsc{sg} justified\\
	\glt \lq What are you saving money for? If you at least had kids, then it would make sense.' (\cite[s.v. \textit{todavía}]{RAEDictionary}, glosses added)
\end{exe}

In other words, and as far as their syntax is concerned, Spanish \textit{todavía si} and French \textit{encore si}/\textit{si encore} \lq if at least' can be understood as a result of rebracketing, as illustrated in (\ref{exMarginalExtensionsRebracketing}). In this scenario, the alternative order \textit{si encore} in French constitutes a subsequent development, which is in line with what is found with other uses of \textit{encore} as a focus-sensitive operator (see e.g. \Cref{sectionTimeScalar}).

\begin{exe}
	\ex\label{exMarginalExtensionsRebracketing}
	[\textit{encore}/\textit{todavía} [\textit{si} p] q] > [\textit{encore}/\textit{todavía} \textit{si} p] [q]
\end{exe}
\is{syntax|)}

In semanto-pragmatic terms, the development just outlined involves just a minor conceptual leap, namely the conventionalisation of meaning components whose seeds are already planted in the marginality use. Thus, in the geographic examples (\ref{exScalarRestrictiveIrunFrench}, \ref{exScalarRestrictiveIrunSpanish}) the town of Irún is compared\is{comparison} both to less marginal exemplars and to locations that fall entirely outside of the Spanish territory. In the same vein, in the counterfactual conditionals the possible world depicted in the protasis is compared\is{comparison} to other hypothetical cases that are more likely to yield a true conditional relationship (e.g. making a greater effort, having greater responsibilites, etc.) as well as to the actual world, which does not. What is more, the marginality use goes along with an inverted scale, such as degrees of geographic \textit{de}centrality in (\ref{exScalarRestrictiveIrunFrench}, \ref{exScalarRestrictiveIrunSpanish}). Similarly, conditional protases are a well-known scale-reversal context (e.g. \cite[33–37]{Haspelmath1997}), which makes the two a perfect match. A graphic comparison is given in \Cref{figureScalarRestrictiveConditionalComparisonB}. Note that \Cref{figureScalarRestrictiveConditionalComparisonB} differs from the \ili{Movima} illustration in \Cref{figureScalarRestrictiveMovima2} above in that with \ili{Movima} \textit{di:ra}(\textit{n}) the excluded values lie higher on a positively defined scale than the focus value. 

\begin{figure}[H]
	\begin{subfigure}[t]{0.48\linewidth}

		\begin{tikzpicture}[node distance = 0pt]
			\node[mynode, text width=\breit, fill=cyan, very near start] (spain){Spain};
			\node[mynode, text width=\schmal, right= of spain] (france){(France)};
			\draw[-] (spain.south east) to (spain.north east);
			\draw[-, thick, label distance=0] (spain.north west) to ($(spain.north west) + (0,-\hoehe*2) $) node [below, align=center, label distance=0] {\strut{}Centre\\\strut};
\draw[-,thick] (spain.north east) to ($(spain.north east) + (0,-\hoehe*2)$) node [below right, align=left, label distance=0] {\strut{}Border\\\strut};
			\draw[-latex, line width=0.5pt]  (spain.south west) to  (france.south east);
						\draw[-, densely dashed] ($(spain.north east)+(-0.1,0)$) to ($(spain.north east) + (-0.1,-\hoehe*2) $)
		node [below left, label distance=0, align=right] {\strut{}Irún\\\strut};
		\end{tikzpicture}
		\subcaption{Illustration of (\ref{exScalarRestrictiveIrun})}	
	\end{subfigure}
	\begin{subfigure}[t]{0.48\linewidth}
			\begin{tikzpicture}[node distance = 0pt]
				\node[mynode, text width=\breit, fill=cyan, very near start] (spain){Chance};
				\node[mynode, text width=\schmal, right= of spain] (france){(No chance)};
					\draw[-, thick, label distance=0] (spain.north west) to ($(spain.north west) + (0,-\hoehe*2) $) node [below, align=center, label distance=0] {\strut{}Max.\\effort\strut};
\draw[-,thick] (spain.north east) to ($(spain.north east) + (0,-\hoehe*2)$)  node [below right, align=left, label distance=0] {\strut{}Threshold\\\strut};
			\draw[-latex, line width=0.5pt]  (spain.south west) to  (france.south east);
						\draw[-, densely dashed] ($(spain.north east)+(-0.1,0)$) to ($(spain.north east) + (-0.1,-\hoehe*2) $)
		node [below left, label distance=0, align=right] {\strut{}Standing\\straight\strut};
		\end{tikzpicture}	
	\subcaption{Illustration of (\ref{exMarginalExtensionsFrenchConditional1mod})}	
	\end{subfigure}
	\caption{Graphic illustration of (\ref{exScalarRestrictiveIrun}) and  (\ref{exMarginalExtensionsFrenchConditional1mod})\label{figureScalarRestrictiveConditionalComparisonB}}
\end{figure}
\is{marginality|)}

\is{exclamation|(}
Lastly, the examples I have discussed so far all featured an overt apodosis. However, the latter can be left implied, thereby giving way to counterfactual wishes and exclamations like the ones in (\ref{exMarginalExtensionsFrenchConditional3}, \ref{exMarginalExtensionsSpanishConditional3}). 

\begin{exe}
	\ex French\label{exMarginalExtensionsFrenchConditional3}\\
\gll \textbf{Si} \textbf{encore} il faisait un effort!\\
if still 3\textsc{sg}.\textsc{m} make.\textsc{pst}.\textsc{ipfv}.3\textsc{sg} \textsc{indef}.\textsc{sg}.\textsc{m} effort(\textsc{m})\\
\glt \lq \textbf{If at least} he made an effort!' (\cite[83]{VictorriFuchs1996}, glosses added)	

	\ex Spanish\label{exMarginalExtensionsSpanishConditional3}\\
		 \gll … \textbf{todavía} \textbf{si} \textbf{te} \textbf{pag}-\textbf{as}-\textbf{en}, pero, ya ve-s, veinte duros por artículo, una miseria.\\
	{} still if 2\textsc{sg}.\textsc{obj} pay-\textsc{pst}.\textsc{sbjv}-3\textsc{pl} but already see-2\textsc{sg} twenty nickles for article, \textsc{indef}.\textsc{sg}.\textsc{f} pittance(\textsc{f})\\
	\glt \lq \textbf{If at least they paid you}, but you can see, twenty nickles for an article, a mere pittance.ʼ (Delibes, \textit{Cinco horas con Mario}, cited in \cite[222]{Bosque2016}, glosses added)
\end{exe}
\il{French|)}\il{Spanish|)}\is{exclamation|)}\is{scale|)}\is{conditional|)}\is{focus|)} \is{restrictive|)}\is{modality|)}

\section{Broadly modal and interactional uses}
\addtocontents{toc}{\protect\setcounter{tocdepth}{3}}
\label{sectionBroadlyModal}
\subsection{Introduction}
In this section, I address different uses that are either modal\is{modality} (in the widest possible sense of the term) or which are predominantly pragmatic and associated with specific types of speech acts. In \Cref{sectionConcessive} I discuss different functions pertaining to, or related to, concessivity.\is{concessive}  I then briefly turn to some other, more idiosyncratic connective functions in \Cref{sectionOtherConnective}.\is{connective} Afterwards, I turn to uses in the realm of non-actualisation of a predicted event in \Cref{sectionNearAttainment}. Lastly, in \Cref{sectionInterrogative,sectionExclamatory} I address functions pertaining to specific types of questions and exclamations,\is{exclamation} respectively.\is{interrogative}
\subsection{Concessive and related functions}\label{sectionConcessive}\is{connective|(}\is{concessive|(} 
In this subsection I survey the uses that \textsc{still} expressions have in concessive constructions and in notionally related functions. I understand the term \textsc{concessive} as an umbrella for complex clause patterns that involve two propositions which are evaluated against a background assumption that they are generally incompatible (e.g. \cite{KoenigConcessives}). It is well-known that \textsc{still} expressions are a common device in such constructions; see (\ref{exConcessiveEnglishIntro}).

\begin{exe}
	\ex \ili{English}\label{exConcessiveEnglishIntro}\\
	\textit{Peter studied hard for the exam. He \textbf{still} failed.}
\end{exe}

In (\ref{exConcessiveEnglishIntro}), \textit{still} forms part of the second half of the concessive construction, namely the clause depicting a situation that holds true despite opposing circumstances. The relevant expressions may also figure in the clause that depicts this circumstance itself,\il{Spanish|(} as shown in in (\ref{exConcessiveSpanishIntro1}) for the collocation of Spanish \textit{aún} plus the subordinator\is{subordination} \textit{que}.

\begin{exe}
	\ex Spanish\label{exConcessiveSpanishIntro1}\\
	\gll \textbf{Aun}-\textbf{que} Pedro estud-ió mucho, fall-ó el examen.\\
	still-\textsc{subord} P. study-\textsc{pst}.\textsc{pfv}.3\textsc{sg} much fail-\textsc{pst}.\textsc{pfv}.3\textsc{sg} \textsc{def}.\textsc{sg}.\textsc{m} exam(\textsc{m})\\
	\glt \lq \textbf{Although} Pedro studied hard, he failed the exam.'\\(personal knowledge)
\end{exe}

I follow \textcite[ch. 17]{Croft2022} in referring to the clause depicting the unfavourable circumstance as the \textsc{protasis} and to the clause that describes the unexpected\is{expectations} outcome as the \textsc{apodosis}. This terminology has the advantage of readily covering not only concessives proper, but also \textsc{concessive conditionals}. As the name suggests, the latter are constructions that combine the characteristics of concessive and \isi{conditional} ones (e.g. \cite{Bossuyt2023}; \cite{HaspelmathKoenig1998}). Example (\ref{exConcessiveSpanishIntro3}) is an illustration, featuring a concessive conditional of the scalar\is{scale} kind; I address the various subtypes of concessive conditionals at the appropriate places below.

\begin{exe}
	\ex Spanish\label{exConcessiveSpanishIntro3}\is{scale}\is{conditional}\\
	\gll \textbf{Aun}-\textbf{que} te qued-es sin dorm-ir, h-as de prepar-ar bien este examen.\\
	still-\textsc{subord} \textsc{refl}.2\textsc{sg} remain-\textsc{sbjv}.2\textsc{sg} without sleep-\textsc{inf} have-2\textsc{sg} of prepare-\textsc{inf} well \textsc{prox}.\textsc{sg}.\textsc{m} exam(\textsc{m})\\
	\glt \lq \textbf{Even} \textbf{if} [it means that]  you don't get any sleep, you have to prepare well for this exam.' (\cite[§47.12e]{RAEGramatica}, glosses added)
\end{exe}

Lastly, my choice of designations does not imply that the two clauses are always found in the order protasis–apodosis. Thus, in (\ref{exConcessiveSpanishIntro2}) the protasis follows the apodosis.

\begin{exe}
	\ex Spanish\label{exConcessiveSpanishIntro2}\\
	\gll Despert-ó d-el coma … con un marcad-o acento australian-o, \textbf{aun}-\textbf{que} \textbf{jamás} \textbf{ha} \textbf{est}-\textbf{ado} \textbf{en} \textbf{ese} \textbf{continente}\\
wake\_up-\textsc{pst}.\textsc{pfv}.3\textsc{sg} of-\textsc{def}.\textsc{sg}.\textsc{m} coma(\textsc{m}) {} with \textsc{indef}.\textsc{sg}.\textsc{m} marked-\textsc{m} accent(\textsc{m}) australian-\textsc{m} still-\textsc{subord} never have.3\textsc{sg} \textsc{cop}-\textsc{ptcp} in \textsc{dem}.\textsc{sg}.\textsc{m} continent(\textsc{m})\\
	\glt \lq He woke up from the coma … with a marked Australian accent,  \textbf{although he has never been to that continent}.' (\textit{Semana}, cited in \cite[47.12t]{RAEGramatica}, glosses added)
\end{exe}

In what follows, I first survey the use of \textsc{still} expressions within the protases of concessive constructions (\Cref{sectionConcessiveAntecedent}) and then turn to the marking of apodoses (\Cref{sectionConcessiveConsequent}). Afterwards, I discuss derived, stand-alone uses as concessive interjections\is{interjection} (\Cref{sectionConcessiveInterjections}) and a likely related counter-expectation\is{expectations} function of Mandarin Chinese\il{Chinese, Mandarin} \textit{hái} (\Cref{sectionCounterExpectation}).
\subsubsection{Concessive protases}
\label{sectionConcessiveAntecedent}
\subsubsubsection{Introduction}
Example (\ref{exConcessiveAntecedentIntroSpanish}) illustrates a \textsc{still} expression as a marker of a concessive protasis. Here  Spanish \textit{aún}, in combination with a gerund, signals that a declaration of love is an unusual embedding for the state-of-affairs depicted in the following apodosis, the absence of the word \textit{love}. As in this case, all relevant instances in the sample involve multi-morphemic collocations.

\begin{exe}
	\ex Spanish\label{exConcessiveAntecedentIntroSpanish}\\
	\gll Lo que más impresion-a de esas página-s es que, \textbf{aun} \textbf{trat}-\textbf{ándo}-\textbf{se} \textbf{de} \textbf{una} \textbf{declaración} \textbf{de} \textbf{amor}, la palabra amor no aparec-e nunca.\\
		3\textsc{sg}.\textsc{n} \textsc{subord} more/most impress-3\textsc{sg} of \textsc{dem}.\textsc{pl}.\textsc{f} page(\textsc{f})-\textsc{pl} \textsc{cop}.3\textsc{sg} \textsc{subord} still constitute-\textsc{ptcp}-\textsc{refl}.3 of \textsc{indef}.\textsc{sg}.\textsc{f} declaration(\textsc{f}) of love \textsc{def}.\textsc{sg}.\textsc{f} word(\textsc{f}) love \textsc{neg} appear-3\textsc{sg} never\\
	\glt \lq What’s most impressive about those pages is that, \textbf{despite being a declaration of love}, the word \textit{love} never appears.' (Martínez, \textit{Santa Evita}, cited in \cite[§27.5i]{RAEGramatica}, glosses added)
\end{exe}\il{Spanish|)}

\subsubsubsection{Distribution in the sample}
\Cref{tableConcessiveAntecedent} lists the expressions and collocations in my sample that mark concessive protases. On examination, several patterns emerge. In terms of areal distribution, the data suggest that \textsc{still} expressions serving as (part of) markers of concessive protases are predominantly a western Eurasian phenomenon, the sole exception in my sample being Rapanui.\il{Rapanui} \is{syntax|(}\is{subordination|(}In syntactic terms, the majority of collocations in question feature some kind of deranking, at least \textit{prima facie}, be it subordination (\ili{French} \textit{encore que}, \ili{Spanish} \textit{aun cuando}, \textit{aunque}, \textit{aun si}, \textit{todavía que}) or the nominalisation of a predicate (\ili{Rapanui} \textit{nōatu te} and the \ili{Spanish} \textit{aún}-plus-gerund construction).\is{syntax|)}\is{subordination|)} Lastly, nearly all \textsc{still} expressions involved can also participate in marking concessive apodoses. The exception to both generalisations is \il{German|(}German \textit{noch so}. This collocation is also an outlier in that it is semantically much more specific, featuring only in universal concessive conditionals\is{conditional} \lq no matter how much\rq{}. In what follows, I first examine deranked\is{subordination} protases in some more depth and then briefly turn to the case of German \textit{noch so}.\il{German|)}

\begin{table}[hbt]
\caption{Concessive protases marking\label{tableConcessiveAntecedent}}
\small
	\fittable{\begin{tabular}{lllllll}
		\lsptoprule	
		M.-area & Language & Expression & Collocation & & Appendix\\
		\midrule
		Eurasia & \ili{French} & \textit{encore} & \textit{encore} \textit{que} & \lq still \textsc{subord}\rq & \ref{appendixFrenchEncoreConcessiveAntecedent}\\
& \ili{German} & \textit{noch} & \textit{noch }\textit{so} & \lq still so (much)\rq{} & \ref{appendixGermanNochSo}\\
		& \ili{Spanish} & \textit{aún} & with gerund && \ref{appendixSpanishAunGerundConcessive}\\
		& & \textit{aún} & \textit{aun} \textit{cuando} & \lq still when' & \ref{appendixSpanishAunCuando}\\
		& & \textit{aún} & \textit{aun}-\textit{que}  &\lq still-\textsc{subord}' & \ref{appendixSpanishAunque}\\
		& & \textit{aún} & \textit{aun} \textit{si} & \lq still if' & \ref{appendixSpanishConcessiveAunSi}\\
& & \textit{todavía} & \textit{todavía} \textit{que}& \lq still \textsc{subord}\rq{} &  \ref{appendixSpanishTodaviaConcessiveAntecedent}\\
		Papunesia & \ili{Rapanui} & \textit{nō} & \textit{nōatu} \textit{te} & \lq still-away(?)\textsc{art}\rq{}
		& \ref{appendixRapaNuiConcessiveAntecedent}\\
		\lspbottomrule
	\end{tabular}}
\end{table}

\subsubsubsection{A closer look: Deranked protases}\is{subordination|(}
It is well known that concessive constructions can convey a variety of pragmatic nuances (e.g. \cite{KoenigConcessives}), being able to operate in different conceptual domains and to express various degrees of concession (see \cite{Bell2010} on the latter). That is to say, despite the common denominator of marking an obstructing circumstance, the exact meaning contributions and usage patterns of the collocations in \Cref{tableConcessiveAntecedent} are not necessarily congruent. What is more, some expressions and collocations are attested only in concessives proper, while others are restricted to concessive conditionals,\is{conditional} mostly of the scalar kind \lq even if\rq{}.\is{scale} Yet a third group can mark both.

\il{French|(} 
The protases clauses governed by French \textit{encore que} typically express concession on the speech-act level, signalling that the speaker may not have been justified in making a preceding statement (\cite[194–195]{MosegaardHansen2008}; \cite[10–12]{Morel1996}). As is normal for such rectificational concessives \parencite{KoenigConcessives}, they follow the apodosis. Both points are illustrated in in (\ref{exConcessiveAntecedentEncoreQue}). Here, the speaker restricts their previous statement, signalling that, despite their initial answer \lq hard to say\rq{}, there is indeed something definite to be stated about the subject matter. Marking a type of afterthought, \textit{encore que} is often prosodically\is{prosody} set off from the preceding clause and from the one it introduces. This, in turn, has given rise to a holophrastic, stand-alone use (\Cref{sectionConcessiveInterjections}).\is{interjection} 

\begin{exe}
\ex French\label{exConcessiveAntecedentEncoreQue}
	\begin{xlist}
		\exi{A:}\textit{Cette solidarité, est-ce une tendance montante ou déjà une survivance en crise?}\\
		\lq Is this show of solidarity a growing trend or already an unstable relic from the past?\rq
		\exi{B:}
		\gll \textup{[}Difficile de le savoir.\textup{]\textsubscript{\textsc{apodosis}}} \textup{[}\textbf{Encore} \textbf{que} l’-exemple américain … doit nous inciter à la prudence sur les bons sentiments.\textup{]\textsubscript{\textsc{protasis}}}\\
		\phantom{[}difficult of 3\textsc{sg}.\textsc{acc}.\textsc{m} know \phantom{[}still \textsc{comp} \textsc{def}.\textsc{sg}-example(\textsc{m}) American.\textsc{m} {} must.3\textsc{sg} 1\textsc{pl}.\textsc{acc} incite.\textsc{inf} to \textsc{def}.\textsc{sg}.\textsc{f} caution(\textsc{f}) about \textsc{def}.\textsc{pl} good.\textsc{pl}.\textsc{m} feeling(\textsc{m}).\textsc{pl}\\
		\glt \lq Difficult to say. \textbf{Still,} the example of the United States should encourage us to be sceptical about finer feelings.’ (\textit{Nouvel Observateur}, cited in \cite[192]{MosegaardHansen2008}, glosses added)
	\end{xlist}
\end{exe}\il{French|)} 

\il{Spanish|(}\is{additive|(}The Spanish collocation \textit{todavía que}, which in the present-day language is restricted to American varieties, has been described as typically contributing an additive nuance. Thus, according to \textcite{Morera1999} and \textcite[§30.8o]{RAEGramatica}, it can often be paraphrased as \textit{encima de que} p \lq on top of \textit{p}\rq{}. This fully mirrors how \textit{todavía} as a marker of a concessive apodosis clause tends to feature a notion of \lq \textit{p}, on top of that \textit{q}\rq{}. Example (\ref{exConcessiveConsequentTodaviaQue}) is an illustration.

\begin{exe}
	\ex Spanish\label{exConcessiveConsequentTodaviaQue}\\
	Context: She shouldn't be ungrateful.
	\exi{}\gll ¡\textbf{Todavía} \textbf{que} la dej-amos entr-ar, contest-a con sarcasmo-s!\\
		\phantom{¡}still \textsc{subord} 3\textsc{sg}.\textsc{acc}.\textsc{f} let-1\textsc{pl} enter-\textsc{inf} reply-3\textsc{sg} with sarcasm-\textsc{pl}\\
	\glt \lq \textbf{Even though/as if it weren't sufficient that} we let her in, she responds with ridicule.' (Gotbeter, \textit{La prudencia}; cited in \cite[206]{Bosque2016}, glosses added)
\end{exe}\is{additive|)}

\il{French|(}\is{scale|(}\is{mood|(}
As far as the discourse status of the proposition is concerned, with French \textit{encore que} and Spanish \textit{aun cuando}, \textit{aunque} this is tied to the choice of mood. The indicative marks an assertion, as in (\ref{exConcessiveAntecedentEncoreQue}) above, whereas the subjunctive signals given information that is commented upon.\il{French|)} In Spanish, the latter configuration can also yield a scalar concessive conditional;\is{conditional} see (\ref{exConcessiveAntecedentSpanishAunqueAlternation}). Both types of readings are also available with the Spanish \textit{aún}-plus-gerund construction (which does not feature mood inflection), whereas the collocation \textit{aun si} appears to always mark scalar concessive conditionals.\footnote{The question of mood selection is slightly more complex in the case of \textit{aun si}, as \textit{si} \lq if\rq{ }brings with it a constraint against the present subjunctive \parencite[§47.8d]{RAEGramatica}.} 

\begin{exe}
	\ex Spanish\is{conditional}\label{exConcessiveAntecedentSpanishAunqueAlternation}\\
\gll \textbf{Aun}-\textbf{que} \textbf{llueva} sal-go.\\
	still-\textsc{subord} rain.\textsc{sbjv}.3\textsc{sg} go\_out-1\textsc{sg}\\
	\glt i.\phantom{i} \lq Even if it is raining, I am going out.'\\
	ii. \lq Even though it's raining [given information], I am going out.'
	\\(personal knowledge)
\end{exe}

Scalar concessive protases governed by \textit{aunque} lie at the heart of the complex \is{focus|(} particle \textit{aunque sea} \lq so much as\rq{}, lit. \lq{}even if it were\rq{ }(\Cref{sectionScalarAdditive}). Example (\ref{exConcessiveAntecedentSpanishAunqueSea}) is an illustration of this expression.

\begin{exe}
	\ex Spanish\label{exConcessiveAntecedentSpanishAunqueSea}\\
	\gll Si dices \textbf{aun}-\textbf{que} \textbf{sea} \textbf{una} \textbf{palabra}, v-as a ten-er problema-s.\\
	if say.2\textsc{sg} still-\textsc{subord} \textsc{cop}.\textsc{sbjv}.3\textsc{sg} \textsc{indef}.\textsc{sg}.\textsc{f} word(\textsc{f}) go-2\textsc{sg} to have-\textsc{inf} problem-\textsc{pl}\\
	\glt \lq If you say \textbf{even one word} (lit. if you say [something], even if it were a word), you’ll get into trouble.' \parencite[356]{GastvanderAuwera2011}
\end{exe}\is{scale|)}\is{focus|)}

Lastly, Spanish \textit{aunque} can combine with \textit{más} \lq more, most\rq{} to mark universal concessive conditionals\is{conditional} \lq no matter how much\rq{ }(\appref{appendixSpanishAunAunqueMas}). This is illustrated in (\ref{exConcessiveAntecedentsAunqueMas}). The semantics at play is fairly straightforward: from the assertion that the conditional relationship holds under the most adverse circumstance (the contribution of \textit{más}), it normally follows that it holds across the entire set of conceivable circumstances.

\begin{exe}
	\ex Spanish\label{exConcessiveAntecedentsAunqueMas}\is{conditional}\\
	\gll A vec-es lo mejor es alej-ar-se \textbf{aun}-\textbf{que} \textbf{más} \textbf{te} \textbf{duela}.\\
at time-\textsc{pl} \textsc{def}.\textsc{sg}.\textsc{n} best \textsc{cop}.3\textsc{sg} withdraw-\textsc{inf}-\textsc{refl}.3 still-\textsc{subord} more/most \textsc{obj}.2\textsc{sg} hurt.\textsc{sbjv}.3\textsc{sg}\\
	\glt \lq Sometimes the best thing to do is to distance yourself, \textbf{no matter how much it might hurt}.\rq{ }(found online, glosses added).
\end{exe}\is{mood|)}\largerpage[-1]

\subsubsubsection{Discussion: Deranked protases} I now turn to the question of what motivates the recruitment of \textsc{still} expressions in deranked concessive protases, and this issue is best approached from a diachronic angle. To anticipate some of the following discussion, in all cases the link to phasal polarity appears to be an indirect one, mediated by other functions of the same items, primarily ones pertaining to additivity.\is{additive|(} This is, of course, a manifestation of a more far-reaching tendency for additives to be employed in the marking of concessive protases (\cite{Forker2016}; \cite{Koenig1985}).

\is{scale|(}Thus, Spanish \textit{aún} is not only a phasal polarity expression, but also serves as a scalar additive marker \lq even\rq{ }(\Cref{sectionScalarAdditive}). As has been repeatedly observed in the literature, this function lies without doubt at the heart of the relevant collocations (e.g. \cite{Elvira2005}; \cite{PerezSaldanyaVincent2014}; \cite{RAEGramatica}).\footnote{In the case of \textit{aún} plus gerund, additional motivation is found in the fact that gerunds in peripheral position by themselves allow for a concessive reading \parencite[§27.5g]{RAEGramatica}.} This is most obvious in the case of scalar concessive conditionals\is{conditional} like the one in (\ref{exConcessiveAntecedentsAunqueSCC}). Here \textit{aún} as \lq even\rq{} signals that its \isi{focus} denotation, the proposition contained in the subordinate clause, yields a particularly informative statement \lq even \textit{p} being the case, \textit{q}\rq{}. This licenses the inference that the conditional relationship also holds true under less antagonistic circumstances (cf. \cite[80]{Koenig1991}).

\begin{exe}
	\ex Spanish\is{conditional}\label{exConcessiveAntecedentsAunqueSCC}\\
	\gll \textbf{Aun}-\textbf{que} \textbf{te} \textbf{qued}-\textbf{es} \textbf{sin} \textbf{dorm}-\textbf{ir}, h-as de prepar-ar bien este examen.\\
still-\textsc{subord} \textsc{refl}.2\textsc{sg} remain-\textsc{sbjv}.2\textsc{sg} without sleep-\textsc{inf} have-2\textsc{sg} of prepare-\textsc{inf} well \textsc{prox}.\textsc{sg}.\textsc{m} exam(\textsc{m})\\
	\glt \lq \textbf{Even if} [it means that] \textbf{you don't get any sleep}, you have to prepare well for this exam.' (\cite[§47.12e]{RAEGramatica}, glosses added)
\end{exe}

Such concessive conditionals\is{conditional} based on \textit{aún} as \lq even\rq{ }not only allow for a compositional analysis, but they also correspond to the earliest respective attestations (\cite{Elvira2005}; \cite{PerezSaldanyaVincent2014}). From there, one can assume a gradual extension to marking ordinary concessives. This is a cross-linguistically common development (\cite[82]{Koenig1991}, \citeyear{KoenigConcessives}) and motivated by the fact that the boundary between the two types is fluid, as was seen for \textit{aunque} in (\ref{exConcessiveAntecedentSpanishAunqueAlternation}) above:
\begin{quote}
In many, and perhaps all, languages, concessive conditionals\is{conditional} with \isi{focus} particles can be used in a factual sense, i.e., in exactly the same way as genuine concessive clauses e.g., \ili{English} \textit{Even if he IS my brother, I am not going to give him any more money}. \parencite[822]{KoenigConcessives}
\end{quote}

\il{French|(}
By and large, the case of the Spanish collocations with \textit{aún} is reflected in French \textit{encore que}.\il{Spanish|)}  While \textcite{MosegaardHansen2008} takes this collocation to be directly related to \textsc{still}, several points speak against this assumption, at least as far as its ultimate origins are concerned. First, if \isi{persistence} were to lie at the heart of \textit{encore que}, one would expect it to have an anaphoric function and hence to mark an apodosis, in the same way that \textsc{still} involves an anaphorically retrieved \isi{topic time} (i.e. \lq \textit{p}, [at that time] still \textit{q} > \lq \textit{p}, nonetheless \textit{q}\rq{}). But \textit{encore que} clauses are semantically protases, as they depict an opposing circumstance, notwithstanding their present-day preference for introducing afterthoughts. Secondly, a precursor of this construction, albeit without the subordinator \textit{que}, is attested in early documents \parencite[197]{MosegaardHansen2008}.\il{French, Old}\is{mood|(} Crucially, these attestations feature the subjunctive mood and mark scalar concessive conditionals,\is{conditional} as in (\ref{exConcessiveAntecedentOldFrench}). 

\begin{exe}
	\ex Old French,\is{conditional}\il{French, Old} ca. 12\textsuperscript{th} century\label{exConcessiveAntecedentOldFrench}\\
	\gll \textbf{encore} \textbf{ait} \textbf{ele} \textbf{en} \textbf{son} \textbf{tresor} // \textbf{mil} \textbf{mars} \textbf{d’}-\textbf{argent} \textbf{et} \textbf{mil} \textbf{mars} \textbf{d’}-\textbf{or}, // si est povre n’-{i a} celi // por qu’-ele ait avarisse en li\\
	still have.\textsc{sbjv}.3\textsc{sg} 3\textsc{sg}.\textsc{f} in \textsc{poss}.3\textsc{sg}:\textsc{sg} vault {} thousand mark.\textsc{pl} of-silver and thousand mark.\textsc{pl} of-gold { } if \textsc{cop}.3\textsc{sg} poor \textsc{neg}-\textsc{exist} \textsc{rel} {} for \textsc{subord}-3\textsc{sg}.\textsc{f} have.\textsc{sbjv}.3\textsc{sg} avarice in 3\textsc{sg}\\
	\glt \lq \textbf{and even if she has a fortune} // \textbf{of a thousand marks of silver and a thousand marks of gold}, // she is poor there is no-one [who isn’t] // provided she has avarice within her.\rq{ }(Gautier d'Arras, \textit{Eracle}, cited in \cite[197]{MosegaardHansen2008}, glosses added)
\end{exe}\is{mood|)}

While French \textit{encore} does not function as a full-fledged scalar additive in the way that \ili{Spanish} \textit{aún} does, it has a comparable additive function \lq as late as\rq{ }in collocation with temporal foci (\Cref{sectionTimeScalar}). The latter use is attested from roughly the same historical period as the construction in (\ref{exConcessiveAntecedentOldFrench}), and it takes only a short conceptual leap from \lq as late as\rq{ }to \lq even if\rq{}, namely a mapping from times to possible worlds.\is{modality} What is more, an origin in scalar concessive conditionals\is{conditional} would also be in line with the history of its \ili{Catalan} (not in my sample) cognate \textit{encara} \textit{que} (see \cite{PerezSaldanyaSalvador1995}). In a nutshell, then, all these points suggest that French \textit{encore que} started out as marking the protases of scalar concessive conditionals,\is{conditional} from where it spread to the protases of concessives proper. These would then ultimately become increasingly used in a postponed manner. To \citeauthor{MosegaardHansen2008}'s credit, it is possible that the very last step of the scenario just outlined was motivated by the presence of the \textsc{still} expression \textit{encore}, in the sense that

\begin{quote}
with \lq\lq encore que \textit{p}\rq\rq{} the speaker retrospectively … suggest[s] that she is not sure that the border between a negative conclusion \sim{}r (which would be the expected\is{expectations} …) and the positive conclusion r … has actually been crossed after all. \parencite[212]{MosegaardHansen2008}
\end{quote}
\il{French|)} \is{scale|)}

\il{Spanish|(}Additivity also seems to play a role in the case of Spanish \textit{todavía que} \lq still \textsc{subord}\rq{}, which was illustrated in (\ref{exConcessiveConsequentTodaviaQue}) above. As discussed in the context of that example, \textit{todavía que} has been described as often bringing about a notion of \lq on top of \textit{p}, as if it weren't enough that \textit{p}\rq{}. In this contribution, it fully mirrors \textit{todavía} as a marker of concessive apodoses, the main difference being one of cataphora in the case of \textit{todavía que} protases, as opposed to the anaphoric \textit{todavía} apodoses \parencite{Bosque2016}.\il{Spanish|)}

\il{Rapanui|(}Lastly, the one exception to additivity\is{additive|)} as playing a major role is the Rapanui \textit{nōatu}-plus-nominalisation construction. The composition of this collocation, illustrated in (\ref{exConcessiveAntecedentRapanui1}), is not fully understood. It might involve the \textsc{still} expression \textit{nō}, which also serves as an exclusive\is{restrictive} marker \lq just, only\rq{ }(\Cref{sectionExclusive}), together with directional \mbox{-\textit{atu}} \lq away\rq{}. The directional morpheme also forms part of another concessive construction \textit{ka atu} \lq even if\rq{} \parencite[569–570]{Kieviet2017} and it is possible that \textit{nōatu} is based on a meaning along the lines of \lq just move away from \textit{p}\rq{ }>\lq never mind \textit{p}\rq{}. Alternatively, \textit{nōatu} might be a non-compositional loan from \ili{Tahitian} \parencite[570 fn34]{Kieviet2017}. No matter what its history may be, it is clear that the concessive use is just a special case of a much broader adnominal \lq never mind\rq{ }function; see (\ref{exConcessiveAntecedentRapanui2}).

\begin{exe}
	\ex \label{exConcessiveAntecedentRapanui}
	\begin{xlist}
		\exi{}Rapanui
		\ex \label{exConcessiveAntecedentRapanui1}
		 \gll Pura oho au ki a ia mo uʼi pauró te tapati, \textbf{noatu} \textbf{te} \textbf{roa} \textbf{o} \textbf{te} \textbf{kona} \textbf{hare}\\
		 \textsc{hab} go 1\textsc{sg} to \textsc{cont} 3\textsc{sg} \textsc{ben} see every \textsc{art} week nevermind \textsc{art} long of \textsc{art} place house\\
		 \glt \lq I visit him regularly every week \textbf{even} \textbf{though} \textbf{he} \textbf{lives} \textbf{far} \textbf{away}.' \parencite[59]{duFeu1996}
		 
		 \ex \label{exConcessiveAntecedentRapanui2}
		\gll\textbf{Nōatu} tōʼona ture mai.\\
		never\_mind \textsc{poss}.3\textsc{sg} scold hither\\
		\glt \lq\textbf{Don't mind} his scolding.' \parencite[305]{Kieviet2017}	 
	\end{xlist}
\end{exe}\il{Rapanui|)}\is{subordination|)}

\il{German|(}\is{focus|(}
\subsubsubsection{A closer look and discussion: German \textit{noch so}}
Up to this point, I have set aside the German collocation \textit{noch so}. In its concessive function, this complex item is restricted to universal concessive conditionals\is{conditional} \lq no matter how much\rq{}, as illustrated in (\ref{exConcessiveAntecedentsNochSo1}).
 
\begin{exe}
	\ex German\label{exConcessiveAntecedentsNochSo1}\is{conditional}\\
	\gll \textbf{Du} \textbf{kann}-\textbf{st} \textbf{noch} \textbf{so} \textup{(}\textbf{sehr}\textup{)} \textbf{bitt}-\textbf{en}, es wird dir nichts nütz-en.\\
	2\textsc{sg} can-2\textsc{sg} still so \phantom{(}very ask-\textsc{inf} 3\textsc{sg}.\textsc{n} will.3\textsc{sg} 2\textsc{sg}.\textsc{dat} nothing be\_of\_use-\textsc{inf}\\
	\glt \lq \textbf{You can beg as much as you like}, it won't help.'
	\\(\cite[s.v. \textit{noch}]{Duden}, glosses added)
\end{exe}

\is{additive|(}\textit{Noch so} is clearly based on \textit{noch} in additive function (\cite{HaspelmathKoenig1998}; \cite[634]{MetrichFaucher2009}; \cite{Shetter1966}).\is{prosody|(} This is reflected in the fact that it receives focal stress here, in the same way it does when used as an additive with entities of the same kind \lq another\rq{ }(\Cref{sectionAdditive}).\is{prosody|)} According to the DWDS \parencite[s.v. \textit{noch}]{DWDS}, the collocation started out as \textit{noch ein-mal so} \lq still one-time so, i.e. by another identical degree\rq{}. In the present-day language, the item \textit{so} also serves to signal a particularly high degree of some property (e.g. \cite[s.v. \textit{so}]{DWDS}), such that the \textit{noch so} can be understood along the lines of  \lq with any additional degree conceivable\rq{ }(\cite[634]{MetrichFaucher2009}; \cite{Shetter1966}). Outside of my sample, a rough parallel can be found in the \ili{Turkish} construction illustrated in (\ref{exConcessiveAntecedentsTurkish}).

\begin{exe}
	\ex \ili{Turkish}\label{exConcessiveAntecedentsTurkish}\\
	\gll Bu konu-da \textbf{ne} \textbf{kadar} \textbf{iyi} \textbf{bir} \textbf{kitap} yaz-sa-n=\textbf{da} meşhur ol-a-ma-yacak-sın.\\
	\textsc{prox} subject-\textsc{loc} what much good one book write-\textsc{cond}-2\textsc{sg}=also famous become-can-\textsc{neg}-\textsc{fut}-2\textsc{sg}\\
	\glt \lq \textbf{However} \textbf{good} \textbf{a} \textbf{book} you write on this subject, you won't become famous.\rq{ }(\cite[435]{GoekselKerslake2005}, glosses added)
\end{exe}\is{additive|)}

Lastly, concessive \textit{noch so} is also found as a focus quantifier outside of concessive constructions in the stricter sense, as in (\ref{exConcessiveAntecedentsNochSo2}).

\begin{exe}
	\ex German\label{exConcessiveAntecedentsNochSo2}\\
 \gll Sie bekam nur strahlende, glückliche Aug-en, und damit sag-te sie mehr als \textbf{mit} \textbf{noch} \textbf{so} \textbf{vielen} \textbf{Wort}-\textbf{en}.\\
	3\textsc{sg}.\textsc{f} got.3\textsc{sg} only shining.\textsc{acc}.\textsc{pl} happy.\textsc{acc}.\textsc{pl} eye-\textsc{pl} and thereby say-\textsc{pst}.3\textsc{sg} 3\textsc{sg}.\textsc{f} more than with still so many.\textsc{dat}.\textsc{pl} word-\textsc{dat}.\textsc{pl}\\
	\glt \lq Her eyes became only radiant and happy, and thereby she said more than \textbf{any amount of words} [could possibly tell].' (Remarque, \textit{Drei Kameraden}, cited in \cite[62]{Shetter1966}, glosses added)
\end{exe}
\il{German|)}\is{focus|)}

\subsubsection{Concessive apodoses}
\label{sectionConcessiveConsequent}
\subsubsubsection{Introduction}
In this subsection I discuss \textsc{still} expressions as markers of concessive apodoses. Example (\ref{exConcessiveConsequentIntro}) is an illustration from Mandarin Chinese.\il{Chinese, Mandarin} Here, \textit{hái} highlights the presumed incompatibility between the two propositions \lq (it is) such a special occasion\rq{ }and \lq you forgot about it\rq{}.

\begin{exe}
	\ex Mandarin Chinese\il{Chinese, Mandarin}\label{exConcessiveConsequentIntro}\\
	\gll Zhème hǎo de shèr, \textbf{nǐ} \textbf{hái} \textbf{gěi} \textbf{wàng} \textbf{le}!\\
	such good \textsc{assoc} event 2\textsc{sg} still give forget \textsc{pfv}\\
	\glt \lq Such a special occasion, \textbf{and still you forgot about it}!'
	\\\parencite[149]{Wiedenhof2015}
\end{exe}

\subsubsubsection{Distribution in the sample}
\Cref{tableConcessiveConsequent} lists the expressions and collocations in my sample that occur as markers of concessive apodoses. As can be gathered, \textsc{still} expressions in this function are a common occurrence. Including borderline and tentative cases, the marking of concessive apodoses is attested for 19 expressions in 17 languages. In addition one could list the case of \ili{German} \textit{dennoch} \lq{}nonetheless\rq{}, which historically goes back to \textit{denn}-\textit{noch} \lq then-still\rq{}.\footnote{Marking concessive apodoses is also attested for cognates of my sample expressions, including the \ili{Catalan} cognate of \ili{French} \textit{encore}, \textit{encara} \parencite{PerezSaldanyaSalvador1995}
and \ili{Lao} \textit{ñang}, cognate with \ili{Thai} \textit{yaŋ} \parencite[208–209]{Enfield2007}.} In geographic terms, the marking of concessive apodoses is attested for all macro-areas minus Australia,\footnote{For an Australian candidate from outside my sample, see \textcite[649–651]{Bowern2012} on \ili{Bardi} (Nyulnyulan) \textit{gardi}.} which makes this one of the geographically most widespread coexpression patterns in my sample. It is noteworthy that the vast majority of cases involve independent grammatical words or bound roots, rather than affixes or clitics. In some cases from western Eurasia, the marking of concessive apodoses clauses plus ellipsis has also given rise to concessive-evaluative interjections;\is{interjection}  I discuss those instances separately in \Cref{sectionConcessiveInterjections}.

In what follows, I first take a closer look at some nuanced differences in meaning and use between the various expressions and collocations in question. I then address some structural differences to phasal polarity functions of the same items. Lastly, I turn to the question of the motivation underlying this coexpression pattern.

\begin{table}
\caption{Marking concessive apodoses. \emph{Notes}: *: Only one example in the data. †: Borderline cases of \textsc{still} expressions. ‡: On the borderline between \textsc{still} and concession; see discussion below. §: Only attested in elicited data, not in the extensive text collections.\label{tableConcessiveConsequent}}
\fittable{\begin{tabular}{lllll}
	\lsptoprule	
	M.-area & Language & Expr. & Collocation & Appx.\\
	\midrule
	Africa & \ili{Chuwabu} & =\textit{vi}* & w/ future \isi{tense}
	 &  \ref{appendixChuwabuConcessiveConsequent}\\
	&	\ili{Swahili} & \textit{bado} &   &  \ref{appendixSwahiliConcessiveConsequent}\\
	& \ili{Tashelhyit} & \textit{sul} &    &   \ref{appendixTashelhyitConcessiveConsequent}	\\
	& Tunisian Arabic\il{Arabic, Tunisian} & \textit{bāqi}\textsuperscript{†} & & \ref{appendixtunisianBaqiConcessive}\\
	
	Eurasia & \ili{English} & \textit{still} &   & \ref{appendixEnglishConcessiveConsequent}\\
	& & \textit{still} & \textit{but still} & \ref{exAppendixEnglishButStill}\\
	& \ili{French} & \textit{encore} & w/ clitic inversion & \ref{appendixFrenchEncoreConcessiveConsequent1}\\
	& &\textit{encore} & \textit{et encore} \lq and still' & \ref{appendixFrenchEncoreConcessiveConsequent2}\\
&	& \textit{toujours}\textsuperscript{†} & \textit{toujours est-il que} & \ref{appendixFrenchToujoursConcessive}\\
& & &	\lq it's still \textsc{subord}\rq \\
	& Hebrew (Modern)\il{Hebrew, Modern}  & \textit{ʕadayin} &    &  \ref{appendixHebrewAdayinConcessiveConsequent}\\
	& Mandarin Chinese\il{Chinese, Mandarin} & \textit{hái} &    & \ref{appendixMandarinConcessiveConsequent}\\
	& \ili{Spanish} & \textit{aún} &   & \ref{appendixSpanishAunConcessiveConsequent} \\
	& & \textit{aún} &  \textit{aun} \textit{así} \lq even so\rq{} & \ref{appendixSpanishAunConcessiveAunAsi}\\
	& & \textit{todavía} &     & \ref{appendixSpanishTodaviaConsessiveConsequent}\\
	& \ili{Thai} & \textit{yaŋ} &   &  \ref{appendixThaiConcessiveConsequent}\\
	& \ili{Udihe} & \textit{xai}(\textit{si}) &    & \ref{appendixUdiheConcessiveConsequent}\\
	North America & Classical Nahuatl\il{Nahuatl, Classical} & \textit{oc}&  \textit{ic oc}/\textit{oc ic}\textsuperscript{‡}  & 	\ref{appendixClassicalNahuatlConcessiveConsequent}\\
	& & &\lq thereby still' \\
	&	\ili{Creek} & (\textit{i})\textit{monk}\textsuperscript{§} &  &  \ref{appendixCreekConcessiveConsequent}\\
	& \ili{Kekchí} & \textit{toj} &  &  \ref{appendixKekchiConcessiveConsequent}\\
	Papunesia & \ili{Chamorro} & \textit{ha'}    & & \ref{appendixChamorroConcessiveConsequent}\\
	& \ili{Rapanui} & \textit{nō} &  & \ref{appendixRapaNuiConcessiveConsequent1}\\
	&	&	\textit{nō}& \textit{te me'e nō} & \ref{appendixRapaNuiConcessiveConsequent2}\\
	& & & \lq \textsc{art} thing still'\\
	South America & H.-H. Quechua\il{Quechua, Huallaga-Huánuco} & -\textit{raq}\textsuperscript{‡} & 1\textsc{sg} plus \textsc{fut}  & \ref{appendixQuechuaConcessive}\\
	\lspbottomrule
\end{tabular}}
\end{table}


\subsubsubsection{A closer look: Flavours of concession}
\begin{sloppypar}
As I pointed out in \Cref{sectionConcessiveAntecedent}, concession comes in many flavours and strengths, which means that the expressions and collocations in \Cref{tableConcessiveConsequent} are not necessarily congruent in their meaning and use.\il{French|(} Thus, for the French \textit{encore}-plus-inversion construction,\footnote{\textit{Inversion} here refers to a diachronically older order of pronominal clitics that is found in formal registers and is triggered by certain conjunctional adverbials; see \textcite[336–337]{MosegaardHansen2016}.} \textcite[196]{MosegaardHansen2008} remarks that it signals that \lq\lq the preceding discourse has not exhausted the topic\rq\rq{ }and that \lq\lq{}the matter under discussion is not as simple as it appears\rq\rq{}. This is illustrated in (\ref{exConcessiveConsequentEncoreFautIl}).
\end{sloppypar}
 
\begin{exe}
	\ex French\label{exConcessiveConsequentEncoreFautIl}\\
	\textit{\lq\lq Une enquête menée par une psychologue auprès de ces familles montre que les enfants ont tout à y gagner\rq\rq, préciset-il.}\\
	\lq {\lq\lq}Research carried out by a female psychologist on these families shows that the children have everything to gain from it", he specifies.\rq
	\exi{} \gll Mais \textbf{encore} \textbf{faut}-\textbf{il} \textbf{pouvoir} \textbf{s'}-\textbf{entendre} \textbf{entre} \textbf{parent}-\textbf{s}, sans craindre que l'-autre ne mette fin, {sans crier gare}, à l'-accord officieux.\\
but still be\_necessary.3\textsc{sg}-3\textsc{sg}.\textsc{m} can.\textsc{inf} \textsc{refl}.3-understand.\textsc{inf} between parent-\textsc{pl} without be\_afraid.\textsc{inf} \textsc{comp} \textsc{def}-other \textsc{neg} put.\textsc{sbjv}.3\textsc{sg} end {without warning} to \textsc{def}.\textsc{sg}-agreement(\textsc{m}) unofficial.\textsc{m}\\
	\glt \lq \textbf{Still}, \textbf{the parents do have to be able to get along}, without being afraid that the other will suddenly put an unexpected stop to their unofficial agreement.' (\textit{Marie-Claire}, cited in \cite[193–194]{MosegaardHansen2008}, glosses added)
\end{exe}

The French collocation \textit{et encore} \lq and still\rq{}, on the other hand, serves to cancel potential inferences \parencite[195]{MosegaardHansen2008}. This is shown in (\ref{exConcessiveConsequentEtEncore}). Here, the first sentence might lead the addressee to conclude that Max is liked by his teacher, given that he is apparently a good student. This conclusion is ruled out in the second sentence.

\begin{exe}
	\ex French\label{exConcessiveConsequentEtEncore}\\
	\gll Max aura une très bonne note. \textbf{Et} \textbf{encore} son prof ne l'-aime guère.\\
	M. have.\textsc{fut}.3\textsc{sg} \textsc{indef}.\textsc{sg}.\textsc{f} very good.\textsc{f} grade(\textsc{f}) and still \textsc{poss}.3\textsc{sg}:\textsc{sg}.\textsc{m} teacher(\textsc{m}) \textsc{neg} 3\textsc{sg}.\textsc{acc}-love.3\textsc{sg} much\\
	\glt \lq Max will get a very good grade. \textbf{Even so}, his teacher doesn’t like him very much.' (\cite[195]{MosegaardHansen2008}, glosses added)
\end{exe}

The third French marker in \Cref{tableConcessiveConsequent}, the fixed expression \textit{toujours est-il que} \lq still it's the case that\rq{}, marks a weak form of concession, in which the speaker only states the facts, without taking a stance \parencite[199–201]{MosegaardHansen2008}; see (\ref{exConcessiveConsequentToujours}). Note that \textit{toujours} is a borderline case of a \textsc{still} expression though, perhaps better considered a marker of stasis.\pagebreak

\begin{exe}
	\ex French\label{exConcessiveConsequentToujours}
	\begin{xlist}
		\exi{A:}\textit{Ton ami Fernand ne me plaît pas du tout: il est trop arrogant.}\\
		\lq I don’t care for your friend Fernand at all: he’s too arrogant.\rq{}
		\exi{B:}\gll  Comme tu veux. \textbf{Toujours} \textbf{est}-\textbf{il} qu-\rq{}il est beau mec.\\
		as 2\textsc{sg} want.2\textsc{sg} still \textsc{cop}.3\textsc{sg}-3\textsc{sg}.\textsc{m} \textsc{subord}-3\textsc{sg}.\textsc{m} \textsc{cop}.3\textsc{sg} attractive.\textsc{sg}.\textsc{m} guy(\textsc{m})\\
		\glt \lq Fine. He’s good-looking, \textbf{though}.\rq{ }(\cite[199]{MosegaardHansen2008}, glosses added)
	\end{xlist}
\end{exe} \il{French|)} 

Tunisian Arabic \textit{bāqi},\il{Arabic, Tunisian} likewise a borderline case of a \textsc{still} expression, is only attested in the apodoses of certain concessive conditionals.\is{conditional} This is illustrated in (\ref{exConcessiveConsequentBaqi}) for an alternative concessive conditional.

\begin{exe}
	\ex Tunisian Arabic\is{conditional}\il{Arabic, Tunisian}\label{exConcessiveConsequentBaqi}\\
	\gll T-ikbir wulla t-usġur \textbf{il}-\textbf{lifʕa} \textbf{muxwf}-\textbf{a} \textbf{bāqi}.\\
	3\textsc{sg}.\textsc{f}-be\_big.\textsc{ipfv} or 3\textsc{sg}.\textsc{f}-be\_small.\textsc{ipfv} \textsc{def}-viper(\textsc{f}) dangerous-\textsc{sg}.\textsc{f} still\\
	\glt \lq Qu’elle se trouvé à être grande ou petite, la vipère est toujours dangereuse. [Big or small, \textbf{a viper is dangerous no matter what}.]\rq{ }(\cite[366]{MarcaisGuiga19581961}, glosses by \cite{FischerEtAlTunisian})
\end{exe}\largerpage

\ili{English} \textit{still} in concessive use is commonly described as stressing coherence, whereas competing markers like \textit{yet} and \textit{nonetheless} invoke a more strongly contrastive notion (\cite{Bell2010}; \cite{KoenigTraugott1982}; \cite[ch. 5]{Ranger2018}; among others). This is shown in (\ref{exConcessiveConsequentEnglish}).

\begin{exe}
	\ex \label{exConcessiveConsequentEnglish}
	\begin{xlist}
		\exi{}\ili{English}
		\ex Context: Universality has been the topic of discussion.\\
		\textit{The death of man is unique. \textbf{Still} it is universal.}
		\ex Context: Universality is a new, contrasting fact.\\
		 \textit{The death of man is unique. \textbf{Yet} it is universal.} \\\parencite[175]{KoenigTraugott1982}
	\end{xlist}
\end{exe}

In the collocation \textit{but still}, this expression can occur at the right periphery of the sentence, without governing a second clause. In structural terms, this brings this fixed phrase close to the concessive interjections\is{interjection} I discuss in \Cref{sectionConcessiveInterjections}. As far as its discourse function is concerned, \textcite[130]{Lewis2019} describes \textit{but still} as providing \lq\lq a retrospective speaker comment signalling that the unfavourable situation just expressed is bearable or can be disregarded\rq\rq. Example (\ref{exConcessiveConsequentButStill}) is an illustration.

\begin{exe}
	\ex \ili{English}\label{exConcessiveConsequentButStill}\\
	\textit{I don't know what they've done to it to make it spread \textbf{but still}.}
	\\(BNC, cited in \cite[129]{Lewis2019})
\end{exe}

\il{Spanish|(}
\is{additive|(}Spanish \textit{todavía} as a concessive marker, on the other hand, often goes along with an additive notion along the lines of \lq on top of that\rq{ }\parencite[§30.8ñ]{RAEGramatica}; see (\ref{exConcessiveConsequentSpanish1}). This nuance probably lies at the heart of the oddity that \textcite{EderlyCurco2016} observe for cases like (\ref{exConcessiveConsequentSpanish2}), where the two propositions are in conflict with each other and hence do not readily allow for an additive interpretation. The additive element of concessive \textit{todavía} is furthermore mirrored in its cataphoric, protasis-introducing counterpart \textit{todavía que} (\Cref{sectionConcessiveAntecedent}).

\begin{exe}
	\ex 
	\begin{xlist}
		\sn[]{Spanish}
		\ex[]{\label{exConcessiveConsequentSpanish1}
		\gll Los obrer-o-s solo sab-en hac-er huelga-s y pon-er petardo-s, ¡y \textbf{todavía} pretend-en que se les d-é la razón!\\
		\textsc{def}.\textsc{pl}.\textsc{m} worker-\textsc{m}-\textsc{pl} only know-3\textsc{pl} do-\textsc{inf} strike-\textsc{pl} and put-\textsc{inf} firecracker-\textsc{pl} \phantom{¡}and still pretend-3\textsc{pl} \textsc{subord} \textsc{refl}.3 3\textsc{pl}.\textsc{dat} give-\textsc{sbjv}.3\textsc{sg} \textsc{def}.\textsc{sg}.\textsc{f} reason(\textsc{f})\\
		\glt The workers only know how to go on strike and light up firecrackers, and \textbf{yet}\textbf{/}\textbf{on top of that} they want people to agree with them!' (Mendoza, \textit{La verdad sobre el caso Savolta}; cited in \cite[§30.8ñ]{RAEGramatica}, glosses added)}
		\ex[\#]{\label{exConcessiveConsequentSpanish2}
		\gll Sí, Harry golpe-a a su perro. \textbf{Todavía} es un buen tipo..\\
		yes H. beat-3\textsc{sg} \textsc{acc} \textsc{poss}.3\textsc{sg} dog still \textsc{cop}.3\textsc{sg} \textsc{indef}.\textsc{sg}.\textsc{m} good.\textsc{m} guy(\textsc{m})\\
		\glt (intended: \lq Yes, Harry beats his dog. \textbf{Still}, he is a nice guy.\rq{})\\
		 (\cite[36]{EderlyCurco2016}, glosses added)}
	\end{xlist}
\end{exe}
\il{Spanish|)}\is{additive|)}

\is{modality|(}Other than the specimen discussed so far, two cases in my sample appear to lie on the intersection between marking \isi{persistence} and concession.\il{Nahuatl, Classical|(} The first one involves Classical Nahuatl \textit{oc}, more specifically, clauses introduced by \textit{ic oc}/\textit{oc ic} \lq thereby still/still thereby\rq{} plus a verb in the \isi{prospective} \isi{aspect} inflection or a verb in the generic present; the latter case is illustrated in (\ref{exConcessiveConsequentNahuatl}). Judging from \citeauthor{Launey1986}'s (\citeyear[1264]{Launey1986}) description, this clause pattern expresses a persistent\is{persistence} \isi{possibility} despite unfavourable circumstances. 
\il{Quechua, Huallaga-Huánuco|(}\is{tense|(}The second fringe case involves Huallaga-Huánuco Quechua \mbox{-\textit{raq}} plus a future tense verb with a first person subject; see (\ref{exConcessiveConsequentQuechua}). This collocation is commonly used to signal that the speaker maintains a certain plan, despite what the addressee might assume  (thus bordering on the \isi{prospective} \lq eventually\rq{ }use I discuss in \Cref{sectionProspective}). 

\begin{exe}
	\ex Classical Nahuatl\label{exConcessiveConsequentNahuatl}\is{persistence}\\
	\gll \textbf{Ic} \textbf{oc} palēhuī-lo in pil-huâ.\\
	thereby still help-\textsc{pass} \textsc{det} child-\textsc{poss}.\textsc{nmlz}\\
	\glt \lq Par ce procédé la mère est encore soulagée. [By this procedure the mother is still relieved.]' (\cite[1265]{Launey1986}, glosses added)\il{Nahuatl, Classical|)}
 
	\ex Huallaga-Huánuco Quechua\is{persistence}\label{exConcessiveConsequentQuechua}\\
	\gll Ura-shaq-\textbf{raq}.\\
	do-\textsc{fut}.1-still\\
	\glt \lq I will yet do it / I still intend to do it (despite your thinking that I wonʼt).' \parencite[389]{Weber1989}
\end{exe}\il{Quechua, Huallaga-Huánuco|)}\is{tense|)}\is{modality|)}

\subsubsubsection{A closer look: Structural and combinatory characteristics}\is{actionality|(} Having discussed some nuances in meaning, I now turn to a few more abstract characteristics that set the use in concessive apodoses apart from its phasal polarity ancestor.\is{aspect|(} The first difference pertains to aspectual compatibilities. Given that the situation in question need not persist\is{persistence} in time, the use in the focus of this section is perfectly compatible with a \isi{perfective} viewpoint\is{aspect} and/or achievement predicates. For Mandarin Chinese,\il{Chinese, Mandarin} this was seen in  (\ref{exConcessiveConsequentIntro}) above. Example (\ref{concessiveConsequentTashelhyit}) is another illustration, featuring \ili{Tashelhyit} \textit{sul}.

\begin{exe}
	\ex \ili{Tashelhyit}\label{concessiveConsequentTashelhyit}\\
	Context: Fox has tricked a man into giving him food and then rushed off. The man, however, was holding Fox’s tail, which has torn off.\\
	\gll T-frḥ-t nit \textbf{sul} \textbf{t}-\textbf{fl}-\textbf{t} \textbf{d} \textbf{abakku} \textbf{nnk} \textbf{γ} \textbf{ufus} \textbf{inu}!\\
	2\textsc{sg}-be\_happy.\textsc{pfv}-2\textsc{sg} indeed still 2\textsc{sg}-leave.\textsc{pfv}-2\textsc{sg} \textsc{ven} tail \textsc{poss}.2\textsc{sg} at hand \textsc{poss}.1\textsc{sg}\\
	\glt \lq You may be happy now, \textbf{but you left your tail in my hand}!'\\ (\cite[82]{Stroomer2001}, glosses added)
\end{exe}\is{actionality|)}\is{aspect|)}

\il{Kekchí|(}Relatedly, Kekchí \textit{toj} in phasal polarity function does not normally combine with a negated\is{negation} predicate, whereas it does when it is used as a concessive marker; see (\ref{concessiveConsequentKekchi}).

\begin{exe}
	\ex\label{concessiveConsequentKekchi}
	Context: Moon's father has asked Thunder to kill Moon and Sun.\\
	\gll \textbf{Toj} a'an \textbf{ink'a'} ki-r-aj.\\
	still 3\textsc{sg} \textsc{neg} \textsc{pfv}.\textsc{evid}:3\textsc{sg}.\textsc{p}-3\textsc{sg}.\textsc{a}-desire\\
	\glt \lq He \textbf{still didn't} want to kill them (despite his brother's wishes).' \parencite[462–463]{Kockelman2020}
\end{exe}\il{Kekchí|)}

\is{syntax|(}Distinctive differences may also be found on the syntactic level. Thus, the \ili{French} \textit{encore}-plus-inversion construction features a marked order of pronominal elements that is not available in the phasal polarity function. In \ili{English}, clause-initial \textit{still} can only serve as a concessive marker, and a parallel situation obtains with Modern Hebrew\il{Hebrew, Modern|(} \textit{ʔadayin}, illustrated in (\ref{exConcessiveConsequentHebrew}). Presumably, in both cases this is a syntactic reflection of the wider scope of the concessive use.
 
\begin{exe}
	\ex Modern Hebrew\label{exConcessiveConsequentHebrew}\\
	\textit{Im at rotsa ħofeš al teviʼi yeladim … Agav, gam ba-hanaqa yordim be-miškal aval ha-ħaze mištane.}\\
	\lq If you want freedom, don’t have children … By the way, you’ll also lose weight when breastfeeding, but your breasts change.\rq{}\\
	\gll \textbf{ʕadayin}, lo keday le-vater al ha-ħavaya ha-zo.\\
	still \textsc{neg} worthwhile to-give\_up on \textsc{def}-experience(\textsc{f}) \textsc{def}-\textsc{prox}.\textsc{sg}.\textsc{f}\\
	\glt \lq \textbf{Still,} you shouldn’t give up on this experience.\rq{}
	\\(found online, glosses added)%\footnote%{\url{https://www.facebook.com/LaishaMagazine/photos/10154667930505836} (02 March, 2023).}
\end{exe}\il{Hebrew, Modern|)}\is{syntax|)}

\subsubsubsection{Discussion: From \textsc{still} to marking concessive apodoses} 
\begin{sloppypar}
In diachronic terms, it is safe to assume that the concessive functions of all expressions and collocations in \Cref{tableConcessiveConsequent} are secondary, given the well-established tendencies for meanings to become increasingly textual\is{textuality}\is{proceduralisation} and expressive\is{expressivity} (\Cref{sectionSemasiologicalChange}). In most cases the concessive uses can be linked directly to phasal notions. Not only is the broader domain of expressions for continuity\is{persistence} or coexistence is a well-known source of concessive markers (\cite{Koenig1985}, \citeyear{KoenigConcessives}), but the historic primacy of \textsc{still} is perceivable in many etymologies and diachronic corpus studies. All of this raises the question of what motivates a development from signalling phasal polarity \textsc{still} to marking concessive apodoses. In broad strokes, it can be said that \lq\lq [p]ersistence of a given state of affairs is continuity. Persistence\is{persistence} of a given state of affairs in a context containing adverse factors that would normally render such \isi{persistence} unlikely amounts to concession\rq\rq{ }(\cite[13]{EderlyCurco2016}).\footnote{For very similar statements see \textcite{JingSchmidtGries2009}, \textcite{Michaelis1993}, \textcite{Kockelman2020} and \textcite{Yeh1998}, among others. For Modern Hebrew \textit{ʕadayin},\il{Hebrew, Modern} \textcite{TsirkinSadan2019} additionally stipulates influence from \ili{English}. In light of the widespread attestation of \textsc{still} expressions as markers of a concessive relationship, this seems superfluous.}
\end{sloppypar}

In slightly more hands-on terms, authors such as \textcite{Deloor2012}, \textcite{EderlyCurco2016}, \textcite{KoenigTraugott1982}, \textcite{Lewis2019}, and \textcite{PerezSaldanyaSalvador1995} point to cases like (\ref{exConcessiveConsequentMidnight1}) as bridging contexts. Here, the preceding clause depicts a time that is so advanced that the situation's persistence can be reasonably questioned.\footnote{Often, this is linked to the unexpectedly\is{expectations} late scenario of \textsc{still}. My sample data is inconclusive here: as far as can be judged, most expressions in \Cref{tableConcessiveConsequent} are, in principle, compatible with the unexpectedly\is{expectations} late scenario of \textsc{still}. The major exception seems to be Classical Nahuatl\il{Nahuatl, Classical} \textit{oc}. The issue is further complicated by the question of relative chronology: a marker's compatibility at the time that the concessive functions arose need not be identical to those found on the synchronic plane.} This allows for a concessive inference about the simultaneous\is{simultaneity} incompatibility of the two states-of-affairs.

\begin{exe}
	\ex \ili{English}\\
	\textit{It is midnight and he is \textbf{still} working}.\label{exConcessiveConsequentMidnight1} \parencite[276]{Koenig1985}
\end{exe}

\is{marginality|(}\is{scale|(}
Works such as \textcite{Ippolito2004}, \textcite{Koenig1977}, \textcite{Michaelis1993}, on the other hand, stress the similarity between concession and the marginality use of \textsc{still} expressions. To briefly recapitulate from \Cref{sectionMarginality}, the marginality use involves a metonymical transfer from times onto other scales\is{scale} and addresses the question of whether a given entity is included within the bounds of that scale.\is{scale|(}\il{Spanish|(} Example (\ref{exConcessiveConsequentMarginal}) is an illustration. By using \textit{todavía} here, the speaker signals that Juan does not receive the greatest degree of sympathy from them, but unlike Pedro, he is somewhat likeable. The marginality use can thus lead to concessive overtones  (cf. \cite[174–175]{MosegaardHansen2008}) along the lines of \lq far from the best case, but it does count\rq{}.\footnote{The marginality uses is attested for seven of the expressions in \Cref{tableConcessiveConsequent}, namely \ili{English} \textit{still}, \ili{French} \textit{encore}, Mandarin Chinese \textit{hái},\il{Chinese, Mandarin} Modern Hebrew\il{Hebrew, Modern} \textit{ʔadayin}, \ili{Swahili} \textit{bado}, \ili{Tashelhyit} \textit{sul}, and \ili{Thai} \textit{yaŋ}.}

\begin{exe}
	\ex Spanish\label{exConcessiveConsequentMarginal}\\
	\gll A Juan \textbf{todavía} lo aguant-o, pero a Pedro no.\\
	\textsc{acc} J. still 3\textsc{sg}.\textsc{m}.\textsc{acc} tolerate-1\textsc{sg} but \textsc{acc} P. \textsc{neg}\\
	\glt \lq I still stand Juan, but not Pedro.\rq{ }(\cite[3]{EderlyCurco2016}, glosses added)
\end{exe}\il{Spanish|)}

\is{modality|(}Though the two approaches differ in their details, they ultimately agree that the recruitment of \textsc{still} expressions as markers of concessive apodoses involves the conventionalisation of an erstwhile inference rooted in aligning possible worlds with some scale: \lq\lq worlds are arrayed with respect to the degree to which they favor the outcome in question; the least adverse (or most favorable) world is nearest the origin\rq\rq{ }\parencite[219]{Michaelis1993}.\is{marginality|)}\is{scale|)}

In addition to these general bridging contexts, \textcite{Koenig1985} points out two specific constellations that may catalyse the emergence and conventionalisation of a concessive inference. The first environment is clausemate negation,\is{negation} given that its felicitous use requires the corresponding affirmative (i.e. the more expected\is{expectations} outcome) to be contained in the communicative common ground (e.g. \cite{Givon1978}). The second context are conditionals,\is{conditional} which by themselves  involve a mapping of propositions onto possible worlds. Both are at play in (\ref{exConcessiveConsequentBillPays}).

\begin{exe}
	\ex \ili{English}\is{conditional}\is{negation}\label{exConcessiveConsequentBillPays}\\
	\textit{Even if Bill pays me £200, I'm \textbf{still} \textbf{not} going to do it.} \parencite[194]{Koenig1977}
\end{exe}

\is{tense|(}Coincidentally, (\ref{exConcessiveConsequentBillPays}) also features the future tense. According to \textcite[141]{vanBaar1991}, this is another common environment in which \textsc{still} expressions come to serve as concessive markers, presumably because of the modal element inherent to predictions. As I have discussed above, it is the future tense inflection to which concessive readings of Huallaga-Huánuco Quechua\il{Quechua, Huallaga-Huánuco} \mbox{-\textit{raq}} are restricted. The same is true of \ili{Chuwabu} \mbox{=\textit{vi}} (but see below). Relatedly, with Classical Nahuatl\il{Nahuatl, Classical} \textit{oc}, a concessive interpretation is found in the \isi{prospective} aspect,\is{aspect} and in certain generic contexts, apparently always in conjunction with modal nuances about persistent\is{persistence} possibilities.\is{possibility}\is{tense|)}\is{modality|)}

\subsubsubsection{Discussion: Other strands of motivation}
\il{French|(} Up to this point I have focussed on direct links between concessive apodoses and phasal polarity \textsc{still} (or its close \isi{marginality} cousin). In some instances, motivation for the concessive function can be found in other functions of the same items. \is{additive|(}Thus, based on a diachronic corpus study, \textcite[197–199]{MosegaardHansen2008} discusses how the two relevant French constructions find additional motivation in the additive function of \textit{encore} (\Cref{sectionAdditive}). She specifically points to examples like (\ref{exConcessiveConsequentEncoreInversionOldFrench}).\il{French, Old} Here, the negated\is{negation} additive can be re-interpreted as a form of concession, countering a possible inference that someone who does not act as a judge may instead act as a counsel.

\begin{exe}
	\ex Old French,\il{French, Old} 13\textsuperscript{th} century\label{exConcessiveConsequentEncoreInversionOldFrench}\\
	\textit{Car nus ne doit estre en nule querele juges et avocas, et, se li ples n’estoit pas devant li mes devant autre seigneur, mes toutes voies li ples pourroit venir par devant li por reson de ressort},\\
	\lq For no man may be both judge and counsel in any dispute, and, if the case is not tried before him, but before some other lord, but might nevertheless come before him later in the event of an appeal,\rq{}\\
	\gll \textbf{encore} \textbf{ne} \textbf{doit} \textbf{il} \textbf{pas} \textbf{estre} \textbf{avocas}.\\
	still \textsc{neg} must.3\textsc{sg} 3\textsc{sg}.\textsc{m} \textsc{neg} \textsc{cop}.\textsc{inf} counsel\\
	\glt\lq\textbf{he cannot act as counsel}, \textbf{either}\textbf{/}\textbf{he still cannot act as counsel}.\rq{}\\
(de Beaumanoir, \textit{Coutumes de Beauvaisis}, cited in \cite[197–198]{MosegaardHansen2008}, glosses added)
\end{exe}
\il{French|)}

\il{Spanish|(}As I hinted above, additivity appears to also play a role in the use of Spanish \textit{todavía} as a concessive apodosis marker.\footnote{\textcite{Bosque2016} speculates that further motivation might be found in the fact that \textit{todavía} involves the universal quantifier \textit{todo}/\textit{toda}, which also figures in concessive expressions such as \textit{con todo} lit. \lq{}with everything\rq{}.} The two notions are, of course, intimately connected, in that they both invoke a sense of \lq it doesn't end here\rq{}.\il{Spanish|)} Additivity likewise seems to be involved in some attestations of \ili{Udihe} \mbox{\textit{xai}(\textit{si})}. Thus, in the texts I consulted, this expression not only figures in concessives proper, but it is also repeatedly attested as part of the apodoses of alternative concessive conditionals.\is{conditional} These are a type of concessive conditionals that feature a disjunctive set of protases propositions (see \cite{HaspelmathKoenig1998}). Example (\ref{exConcessiveConsequentUdihe1}) is an illustration.

\begin{exe}
	\ex \ili{Udihe}\is{conditional}\label{exConcessiveConsequentUdihe1}\\
	Context: A man has eaten from a tiger’s kill, but left two legs. He says to the tiger:\\
	\gll Ei zuːbe bugdi-we sin-du ne-gi-e-mi, ali=de goː=ko daː=ka bi-mi \textbf{xai} \textbf{o}-\textbf{lo} \textbf{eme}-\textbf{giː}, \textbf{a}-\textbf{wa} \textbf{dig’a} \textbf{si}.\\
	this two leg-\textsc{acc} 2\textsc{sg}-\textsc{dat} put-\textsc{iter}-\textsc{pst}-1\textsc{sg} when=\textsc{foc} far=\textsc{indef} near=\textsc{indef} \textsc{cop}-\textsc{inf} still here-\textsc{loc} come-\textsc{iter}.\textsc{imp} that-\textsc{acc} eat.\textsc{imp} 2\textsc{sg}\\
	\glt \lq I put these two legs aside for you. Whether you walk far away or close to here, \textbf{[just the same]} \textbf{come} \textbf{here} \textbf{and eat them}.\rq
	\\\parencite[The tiger for Udihe people]{NikolaevaEtAl2019}	
\end{exe} 

Example (\ref{exConcessiveConsequentUdihe2}), which can be read as additive or as marking concession, suggests that additive functions of \mbox{\textit{xai}(\textit{si})} at least partially motivate its use in alternative concessive conditionals.\is{conditional}

\begin{exe}
	\ex \ili{Udihe}\label{exConcessiveConsequentUdihe2}\is{conditional}\\
	Context: Once upon a time there was a Chinese tsar. He buried people alive.\\
	\gll Ñuŋu-za seː i:ne-wene-mie bude-isiː-ni, buge-ini, e-siː-ni, bude, \textbf{xai} \textbf{buge}-\textbf{ini}.\\
	six-ten year come-\textsc{caus}-\textsc{inf} die-\textsc{pfv}.\textsc{cvb}-3\textsc{sg} bury-3\textsc{sg} \textsc{neg}-\textsc{pst}-3\textsc{sg} die still bury-3\textsc{sg}\\
	\glt When a person became sixty years old, he buried him, no matter whether he was dead or not. (lit. … he died, he buried him, he did not die, \textbf{he still}\textbf{/}\textbf{also} \textbf{buried} \textbf{him}).' \parencite[18–20]{NikolaevaEtAl2003}
\end{exe}

\is{scale|(}\il{Spanish|(}
Additivity, but of the scalar kind, doubtlessly lies at the heart of the Spanish collocation \textit{aun así}. Example (\ref{exConcessiveConcesquentAunAsi}) is an illustration; see \Cref{sectionScalarAdditive} on \textit{aún} as scalar additive \lq{}even\rq{}. That is, despite superficial differences, this collocation is fully parallel to \ili{English} \textit{even so}.

\begin{exe}
	\ex Spanish \label{exConcessiveConcesquentAunAsi}\\
	\gll El fácil acceso a este recurso tecnológic-o facili-ta su uso en las práctica-s pedagógic-a-s. \textbf{Aun} \textbf{así}, \textbf{no} \textbf{es} \textbf{suficiente}.\\
	\textsc{def}.\textsc{sg}.\textsc{m} easy access(\textsc{m}) to \textsc{prox}.\textsc{sg}.\textsc{m} resource(\textsc{m}) tecnological-\textsc{m} facilitate-3\textsc{sg} \textsc{poss}.3 use in \textsc{def}.\textsc{f}.\textsc{pl} practice(\textsc{f})-\textsc{pl} pedagogical-\textsc{f}-\textsc{pl} still so \textsc{neg} \textsc{cop}.3\textsc{sg} sufficient\\
	\glt Easy access to this technological resource facilitates its use in pedagogical practices. \textbf{Nonetheless}, \textbf{it} \textbf{is} \textbf{not} \textbf{sufficient}.'
		\\(CORPES XXI, glosses added)		
\end{exe}\is{scale|)}\il{Spanish|)}

\il{Chuwabu|(} 
Not only additive functions can play into the development of \textsc{still} expressions as markers of concession.\is{additive|)} Thus, three of the relevant items lexicalise an exclusive\is{restrictive|(} sense \lq only\rq{ } (\Cref{sectionExclusive}). These are \ili{Chamorro} \textit{ha'}, \ili{Chuwabu} \mbox{=\textit{vi}}, and \ili{Rapanui} \textit{nō}. Example (\ref{exConcessiveConsequentChamorro}) illustrates Chamorro \textit{ha'}.

\begin{exe}
	\ex \ili{Chamorro}\label{exConcessiveConsequentChamorro}\\
	\gll Maseha un fa’håhafa håo, \textbf{ya}\sim\textbf{ya}-\textbf{hu} \textbf{håo} \textbf{ha'}.\\
	although 2\textsc{sg} make\_into\_what.\textsc{cont} 2\textsc{sg} \textsc{cont}\sim{}like-1\textsc{sg}.\textsc{rl} 2\textsc{sg} still\\
	\glt \lq Even though you're making yourself into something else, \textbf{I'll still like you}.' \parencite[335]{Chung2020}
\end{exe}

It is conceivable that a concessive reading can be derived from the exclusive function, namely by rendering alternative propositions which are more in line with the assumed incompatibilities false. Put differently, in examples like (\ref{exConcessiveConsequentChamorro}) there is one and only one outcome, no matter the circumstances. In lieu of data on the relative chronology of the three expressions' individual functions, it might be that their restrictive and phasal polarity functions conspired towards the same outcome.\il{Chamorro} \il{Rapanui|(}Lastly, an exclusive function lies without doubt at the heart of the Rapanui concessive phrase \textit{te me'e nō} \lq \textsc{art} thing still/only\rq{}, which is illustrated in (\ref{exConcessiveConsequentRapanui1}). This complex marker instantiates a productive construction \textit{te} N \textit{nō} \lq \textsc{art} N still/only\rq{ }that signals an exception to some circumstance (\cite[266–268, 570]{Kieviet2017}); see (\ref{exConcessiveConsequentRapanui2}) for an illustration.

\begin{exe}
	\ex 
	\begin{xlist}
		\exi{}Rapanui
		\ex Context: His boat was like the other ones.\label{exConcessiveConsequentRapanui1}\\
		\gll \textbf{Te} \textbf{meʼe} \textbf{nō}, ʼi ruŋa i tū vaka era ōʼona e ai rō ʼā e tahi pēʼue, e rua miro ʼi te kaokao o te vaka.\\
	\textsc{art} thing still at above at \textsc{dem} boat \textsc{dist} \textsc{poss}.3\textsc{sg} \textsc{ipfv} exist 	\textsc{emph} \textsc{cont} \textsc{num} one mat \textsc{num} two wood at \textsc{art} side.\textsc{redupl} of \textsc{art} boat\\
	\glt \lq \textbf{However}, in his boat there was a rug, and two poles on the sides of the boat.' \parencite[268]{Kieviet2017}
	\ex \label{exConcessiveConsequentRapanui2}
	Context: He used to drink.\\
		\gll \textbf{Te} \textbf{riva} \textbf{nō}, e taʼero era, ʼina he tiŋaʼi i tāʼana huaʼai.\\
	\textsc{art} good still \textsc{ipfv} drunk \textsc{dist} \textsc{neg} \textsc{neutral} strike \textsc{acc} \textsc{poss}.3\textsc{sg} family\\
		\glt \lq \textbf{Fortunately} (=the good thing was), when he was drunk, he did not beat his family.' \parencite[268]{Kieviet2017}
	\end{xlist}
\end{exe}
\il{Chuwabu|)}\il{Rapanui|)}\is{restrictive|)}

\subsubsection{Concessive interjections}\is{interjection|(} 
\label{sectionConcessiveInterjections}
\subsubsubsection{Introduction} In \Cref{sectionConcessiveAntecedent,sectionConcessiveConsequent} I discuss concessive functions that involve complete clauses.\is{syntax} In addition, \textsc{still} expressions, or collocations involving these, are also attested as concessive holophrases, as in the \ili{English} example (\ref{exConcessiveInterjectionEnglish}).

\begin{exe}
	\ex \ili{English}\label{exConcessiveInterjectionEnglish}
	\begin{xlist}
		\exi{A:} \textit{D'ya think she'll go back to work Kevin?}
		\exi{B:} \textit{No .. I don't think she will to be honest with you.}
		\exi{A:} <sigh>
		\exi{B:} \textit{\textbf{Still}.}
		\exi{A:} \textit{I don't blame her.} (BNC KBC, cited in \cite[129]{Lewis2019})
	\end{xlist}
\end{exe}

Of course, such elliptical concessives are by no means limited to \textsc{still} expressions.\il{German|(} To give just one example, the very same use is found with German \textit{trotzdem} \lq regardless\rq{}, illustrated in (\ref{exConcessiveInterjectionGerman}).

\begin{exe}
	\ex German\label{exConcessiveInterjectionGerman}
	\begin{xlist}
	\exi{A:}\textit{ Deine Schrift ist schwer zu lesen}.\\
		\lq Your handwriting is difficult to read.\rq
		\exi{B:} \textit{Was? Ich habe mir die allergrößte Mühe gegeben!}\\
		\lq What? I made the greatest effort possible!\rq{}		
		\exi{A:} \gll \textbf{Trotzdem}.\\
		regardless\\
		\glt \lq Regardless.\rq{ }(Funke, \textit{Tintenherz})
	\end{xlist}
\end{exe}\il{German|)}

\subsubsubsection{Distribution in the sample}
\Cref{tableConcessiveInterjection} lists the four expressions in my sample that are involved in a holophrastic concessive use. It is noteworthy that all attested cases stem from western Eurasia.\footnote{Outside of my sample, a parallel case is found in \ili{Catalan} \textit{i encara}, a cognate of \ili{French} \textit{et encore}; see \textcite{PerezSaldanyaSalvador1995}.} Unsurprisingly, each of the expressions or phrases in question also has concessive functions in more saturated environments. From these, the interjective usages are clearly derived, at least in diachronic terms (see e.g. \cite{Lewis2019} on \ili{English} \textit{still}). With the exception of \ili{French} \textit{encore que}, all cases go back to the marking of the apodosis in a concessive construction. In what follows, I examine this predominant type first, and then turn to \textit{encore que}.

\begin{table}[H]
	\caption{Concessive interjections\label{tableConcessiveInterjection}}
	\footnotesize	
	\begin{tabularx}{\textwidth}{llllll}
		\lsptoprule
		M.-Area & Language	& Expr. & Collocation	&& Appendix\\
		\midrule
		Eurasia &	\ili{English}	 &	\textit{still} & &	&	\ref{appendixEnglishConcessiveInterjection}\\
		&	\ili{French} & \textit{encore}& \textit{encore que} &\lq still \textsc{subord}\is{subordination} > although\rq{}&\ref{appendixFrenchEncoreConcessiveInterjections}\\
		&	&	& \textit{et encore}& \lq and still\rq{} & \ref{appendixFrenchEncoreConcessiveInterjections}\\
		& Hebrew (Mod.)\il{Hebrew, Modern} & \textit{ʕadayin}&	& & \ref{appendixHebrewConcessiveInterjection}\\
		&	\ili{Spanish} & \textit{aún}	& \textit{aun así} & \lq still so > even so\rq{} & \ref{appendixSpanishAunAsinterjection}\\
		\lspbottomrule
	\end{tabularx}
\end{table}

\subsubsubsection{A closer look and discussion: Interjections from concessive apodoses}\largerpage[2] 
\is{prosody|(}\is{syntax|(}As I have just pointed out, the predominant case in my sample is an interjective use that derived from the marking of concessive apodoses. More specifically, this use can be traced back to the truncation of a pattern in which the expressions (or collocations) in question occupy the clause-initial position and are prosodically separated from the preceding clause as well as from the rest of their host clause.\footnote{For the relevant facts, see \textcite[s.v. \textit{aun así}]{DPDE}, \textcite[193]{MosegaardHansen2008}, \textcite{Lewis2019}, \textcite[205–206]{Ranger2018}, \textcite{TsirkinSadan2019} and \textcite[86–87]{VictorriFuchs1996}.} This pattern is illustrated in (\ref{exConcessiveInterjectionEnglishInitial}) for \ili{English} \textit{still}. A comparable example involving Modern Hebrew\il{Hebrew, Modern} \textit{ʕadayin} is found in (\ref{exConcessiveConsequentHebrew}) above.

\begin{exe}
\ex \ili{English}\label{exConcessiveInterjectionEnglishInitial}\\
… \textit{It is always the dregs of the population who show their patriotism by this sort of behaviour. \textbf{Still}, it is refreshing to see someone taking some sort of action}. \parencite[139]{Lewis2019}
\end{exe}\is{prosody|)}\is{syntax|)}

\is{additive|(}\is{scale|(}\il{Spanish|(}In the case of Spanish \textit{aun así}, the use in question involves \textit{aún} as a scalar additive operator (\Cref{sectionScalarAdditive}), i.e. \lq still so > even so\rq{}. For an illustration of this collocation in a full-fledged concessive construction, see (\ref{exConcessiveConcesquentAunAsi}) above. Example (\ref{exConcessiveInterjectionSpanish}) shows \textit{aun así} as a holophrase.

\begin{exe}
	\ex Spanish
	\label{exConcessiveInterjectionSpanish}
	\begin{xlist}
		\exi{A:} \textit{Si fuese tan amable de ayudarme a ubicar.}\\
		\lq If you could be so friendly and help me know where I am.'
		\exi{B:}
		\textit{¿El lugar? No le asigne ninguna importancia.}\\
		\lq The location? Don't give it any importance.'
		\exi{A:} \gll \textbf{Aun} \textbf{así}.\\
		still/even so\\
		\glt \lq \textbf{Still}!' (CORPES XXI)
	\end{xlist}
\end{exe}\is{scale|)}\il{Spanish|)}\is{additive|)}

\il{French|(}
\subsubsubsection{A closer look and discussion: An interjection from a postponed protasis} Other than the cases I have discussed up to this point, the stand-alone use of French \textit{encore que} goes back to ellipsis within a concessive apodosis, rather than within a protasis. However, although the relevant \textit{encore que} clauses are semantically protases (i.e. they express an unfavourable condition), they are of the rectificational type and thus follow their apodoses.\is{prosody|(} What is more, \textit{encore que} tends to form a prosodic phrase of its own (\cite[193]{MosegaardHansen2008}; \cite[10–11]{Morel1996}; \cite[85]{VictorriFuchs1996}). That is, we find the same syntactic\is{syntax} and prosodic conditions that facilitate truncation. Example (\ref{exConcessiveInterjectionFrench2}) illustrates the interjective usage of \textit{encore que}.

\begin{exe}
	\ex French\label{exConcessiveInterjectionFrench2}\\
	\textit{La fébrilité qui régnait en fin de semaine dernière rue des Italiens pourrait laisser croire à un proche dénouement de l’affaire.}\\
	\lq The feverishness reigning at the end of last week at the court house in rue des Italiens might lead one to expect that a solution to the matter was imminent.'
	\exi{}\gll \textbf{Encore} \textbf{que}! Car depuis l’-origine règne dans ce dossier un climat de manipulation et de désinformation.\\
	still \textsc{comp} because after \textsc{def}.\textsc{sg}-origin reign.3\textsc{sg} in \textsc{prox}.\textsc{m} case(\textsc{m}) \textsc{indef}.\textsc{sg}.\textsc{m} climate(\textsc{m}) of manipulation and of desinformation\\
	\glt \lq \textbf{Not} \textbf{necessarily}! For this case has from the very beginning been characterised by a climate of manipulation and misinformation.' (\textit{Nouvel Observateur}, cited in \cite[192–193]{MosegaardHansen2008}, glosses added)
\end{exe}
\il{French|)}\is{interjection|)}\is{prosody|)}

\subsubsection{Counter-expectation}\label{sectionCounterExpectation}\il{Chinese, Mandarin|(}\is{expectations|(}\largerpage
\subsubsubsection{Introduction}
One expression in the sample, Mandarin Chinese \textit{hái} (\appref{appendixMandarinModal}), has a use in which it signals \lq\lq the speaker's incredulousness\rq\rq{ }and that \lq\lq{}there is a gap between the situation described in the clause and the assumption or expectation held by him or other people.\rq\rq{ }\parencite[345]{BiqHuang2016}. This is illustrated in (\ref{exCounterExpectation1}).

\begin{exe}
	\ex Mandarin Chinese\label{exCounterExpectation1}\\
	Context: Speaker A is speaking about his diet in the Netherlands.
	\begin{xlist}
		\exi{A:}… \textit{xiànzài wǒ fēicháng xǐhuān chī qìsi}.\\
		\lq … now I am extremly fond of cheese.'
		\exi{B:} \gll Nà nǐ \textbf{hái} xíguàn de zhēn kuài a!\\
		that 2\textsc{sg} still habit \textsc{assoc} really real \textsc{sfp}\\
		\glt \lq Well, in that case, you did get used to it really fast!'
		\\\parencite[110]{Wiedenhof2015}
	\end{xlist}
\end{exe}

\subsubsubsection{A closer look and discussion} 
The counter-expectational use of \textit{hái} often goes together with other evaluative material. Thus, example (\ref{exCounterExpectation1}) also features \textit{zhēn} \lq really\rq{}, which is a common occurrence in this use, and the sentence-final particle \textit{a} has been described as often contributing notions of surprise, as well \parencite{HuangShi2016}. Example (\ref{exCounterExpectation2}) is another illustration featuring \textit{zhēn}, plus \textit{kě} \lq indeed\rq{ }for additional emphasis.

\begin{exe}
	\ex\label{exCounterExpectation2}
	\gll Zhēn	ké	yóu	qián,	\textbf{hái}	bù	zǎo	jiāo	le?\\
	really indeed \textsc{exist} money still \textsc{neg} early hand\_over \textsc{pfv}\\
	\glt \lq Si vraiment nous avions de l'argent, comment n'aurions-nous pas payé plus tôt? [If we really had the money, \textbf{don't you think} we'd have paid earlier?]\rq{ }(\cite[16]{Alleton1972}, glosses added)
\end{exe}

Example (\ref{exCounterExpectation2}) illustrates another point: according to \textcite[ch. 2.3]{Alleton1972}, this use of \textit{hái} is often presented as a rhetorical question and can therefore be paraphrased via \textit{nándào} \lq by any chance, isn't it possible that?\rq{}. While the exact pathway leading to the emergence of the counter-expectational function is unclear and deserving of a dedicated diachronic corpus study, its meaning contribution strongly points to \textit{hái} in concessive apodoses (\Cref{sectionConcessiveConsequent}) as its origin.\footnote{This also seems to be the implicit assumption held by \textcite{Alleton1972}, who subsumes the concessive function under the counter-expectational one.} The well-known tendencies for semasiological change (\Cref{sectionSemasiologicalChange}) suggest that this development involved the strengthening of the expressive\is{expressivity} contribution, i.e. the conflict between the situation depicted in the clause and the assumptions held by the speaker and/or hearer, to the detriment of the inter-propositional meaning element defining of the concessive use.
\il{Chinese, Mandarin|)}\is{expectations|)}\is{concessive|)} 

\subsection{Other connective functions}\label{sectionOtherConnective}
In this subsection, I briefly turn to two other connective functions attested in my sample, namely \il{Kekchí|(}Kekchí \textit{toj} as a marker of exceptive protases \lq unless\rq{ }(\appref{appendixKekchiUnless}) and Classical Nahuatl\il{Nahuatl, Classical} \textit{oc} as a \isi{causal} connective (\appref{appendixClassicalNahuatlCausal}).

\subsubsection{Kekchí \textit{toj} and protases of exceptive conditionals}\is{conditional|(}\is{modality|(}
Example  (\ref{exUnlessKekchi1}) illustrates the use of Kekchí \textit{toj} as a marker of the protasis of an exceptive conditional \lq unless\rq{}.

\begin{exe}
	\ex Kekchí\\
	 Context: A hummingbird explaining why he does not want to give away his feathers.\label{exUnlessKekchi1}\\
	\gll T-in-kaamq rah (x)-b'aan ke \textbf{toj} \textbf{t}-\textbf{in}-\textbf{b'at}-\textbf{e'q} \textbf{sa'} \textbf{x}-\textbf{noq'al} \textbf{inup}.\\
	\textsc{prosp}-1\textsc{sg}-die \textsc{cf} \textsc{poss}.3-because cold still \textsc{prosp}-1\textsc{sg}-wrap-\textsc{pass} \textsc{prep} \textsc{poss}.3-thread ceiba\\
	\glt \lq I will die of the cold \textbf{unless I am wrapped in the bark of a ceiba tree}.' \parencite[467]{Kockelman2020}
\end{exe}

\is{limitative|(}
This \lq unless\rq{ }function doubtlessly goes back to \textit{toj} as a temporal delimiter \lq{}until\rq{}, illustrated in (\ref{exUnlessKekchi2}), with a mapping from times to possible worlds. In other words, the situation described in the apodosis obtains until the preventing condition is met, and defeasibly no longer after \parencite{Kockelman2020}. The relationship to \textit{toj} as a phasal polarity expression is thus an indirect one.

\begin{exe}
		\ex \label{exUnlessKekchi2}
		\gll \textbf{Toj} \textbf{maak’a’}-\textbf{q} \textbf{chik} \textbf{in}-\textbf{k’as} t-in-k’anjelaq.\\
		still \textsc{neg}.\textsc{exist}-\textsc{non}.\textsc{specific} more \textsc{poss}.1\textsc{sg}-debt \textsc{fut}-1\textsc{sg}-work\\
		\glt \lq \textbf{Until I no longer have debt} I will work.\rq{ }\parencite[480]{Kockelman2020}
\end{exe}
\is{conditional|)}\is{limitative|)}\il{Kekchí|)}\is{modality|)}

\subsubsection{Causal connective}\il{Nahuatl, Classical|(}\is{causal|(}
Classical Nahuatl \textit{oc} forms part of a causal connective. This function is found in conjunction with the determiner \textit{in} as a signal of \isi{subordination} plus the expression \textit{ic} \lq when\rq{}. Example (\ref{exCausalNahuatl}) is an illustration.
 
\begin{exe}
	\ex Classical Nahuatl\label{exCausalNahuatl}\\
	\gll \textbf{In} \textbf{oc} \textbf{ic} \textbf{ti}-\textbf{tēpiltzin} … xi-m-ìmat-cā-nemi.\\
	\textsc{det} still when \textsc{subj}.2\textsc{sg}-offspring {} \textsc{sbjv}-\textsc{refl}-be\_wise-\textsc{lnk}-live\\
	\glt \lq Pues que eres bien nacido … viue con cordura. [\textbf{Since you are noble}, live wisely.]' (\cite[503]{Carochi1645}, glosses added)
\end{exe}

As \textcite[1269]{Launey1986} points out, the causal function is clearly based on \textit{oc} as a marker of simultaneous duration,\is{simultaneity} a function illustrated in (\ref{exCausalNahuatlSim}) and discussed in \Cref{sectionSimultaneity}. It is therefore only indirectly related to \textit{oc} as a phasal polarity expression. The development of the causal function clearly involves the well-established pathway from time to causality (e.g. \cite[425]{KutevaEtAl2019}). Nonetheless, it is remarkable, given that causal functions appear to be more commonly associated with \textsc{already}\is{already} expressions, such as \il{Spanish|(}Spanish \textit{ya} in (\ref{exCausalSpanish}).
\begin{exe}
	\ex Classical Nahuatl\label{exCausalNahuatlSim}\is{simultaneity}\\
	\gll Mācamo xi-còcoch-ti-ye-cān \textbf{in} \textbf{oc} \textbf{ic} \textbf{n}-\textbf{on}-\textbf{tē}-\textbf{machtia}.\\
	\textsc{proh} \textsc{sbjv}-doze-\textsc{lnk}-stay-\textsc{pl} \textsc{det} still when \textsc{subj}.1\textsc{sg}-\textsc{it}-\textsc{obj}.\textsc{indef}.\textsc{human}-teach\\
	\glt \lq Don’t be dozing off \textbf{while I’m teaching}.'
	\\(\cite[366]{LauneyMackay2011}, glosses added)
	\pagebreak
	\ex Spanish\label{exCausalSpanish}\\
	\gll Pod-emos dar por conclu-id-a la reunión, \textbf{ya} \textbf{que} no hay más cuestion-es pendient-es.\\
	can-1\textsc{pl} give.\textsc{inf} for conclude-\textsc{ptcp}-\textsc{f} \textsc{indef}.\textsc{sg}.\textsc{f} meeting(\textsc{f}) already \textsc{subord} \textsc{neg} \textsc{exist} more issue-\textsc{pl} pending-\textsc{pl}\\
	\glt \lq{}We can close the meeting, \textbf{given that} there are no more issues to be discussed.\rq{ }(\cite[§46.3a]{RAEGramatica}, glosses added)
\end{exe}\il{Spanish|)}
\is{connective|)}\il{Nahuatl, Classical|)}\is{causal|)}

\subsection{Uses pertaining to non-realisations}\label{sectionNearAttainment}\is{irrealis|(}
\subsubsection[tocentry={}]{Introduction}\il{Qiang, Northern|(}
Two sample expressions,  Northern Qiang \mbox{\textit{tɕe}-} and Barabayiiga-Gisamjanga Datooga \mbox{\textit{údu}-},\il{Datooga, Barabayiiga-Gisamjanga} have uses that pertain to the broader domain of non-realisation (\appsref{appendixNorthernQiangAlmost}, \ref{appendixDatoogaNonHappening}). In what follows, I discuss each case separately.

\subsubsection[tocentry={}]{A closer look and discussion:  Northern Qiang \mbox{\textit{tɕe}-} and \lq almost\rq{}} The Northern Qiang expression \mbox{\textit{tɕe}-} forms part of a collocational pattern that signals the near-attainment of some state-of-affairs \lq almost\rq{}. This bi-clausal construction is illustrated in (\ref{exAlmostQiang}).

\begin{exe}
	\ex Northern Qiang\label{exAlmostQiang}\\
	\gll \textbf{ɑ}-\textbf{zə} \textbf{tɕɑ}-\textbf{ŋuaː}-\textbf{ʂə}, qɑ i-pə-l mɑ-lə-jya.\\
	one-\textsc{clf} still-\textsc{cop}.\textsc{prosp}-\textsc{lnk}.\textsc{cf} 1\textsc{sg} \textsc{dir}-arrive-come \textsc{neg}-able-\textsc{asp}:1\textsc{sg}\\
	\glt \lq I \textbf{almost} couldnʼt return.' \parencite[219]{LaPollaHuang2003}
\end{exe}

As discussed by \textcite[220]{LaPollaHuang2003}, the pattern in (\ref{exAlmostQiang}) is transparent to the largest degree. Thus, the linker \mbox{-\textit{ʂə}} contributes the counterfactual meaning,\is{additive|(} whereas the contribution of \mbox{\textit{tɕe}-} lies in its additive function (\Cref{sectionAdditive}). In other words, the \lq almost\rq{ }reading goes back to \lq had it been a little more (like that)\rq{}. In this, the Northern Qiang construction is somewhat similar to the Modern Hebrew\il{Hebrew, Modern|(} collocation \textit{ʕod meʕat} \lq still a little\rq{ }illustrated in (\ref{exAlmostHebrew}), which also signals temporal proximity, albeit without the modal\is{modality} component, and where the phrase-internal position gives evidence that it goes back to the additive use of \textit{ʕod}.\pagebreak

\begin{exe}
	\ex Modern Hebrew\label{exAlmostHebrew}\\
	\gll ɦašáv-ti ki \textbf{ʕod} \textbf{meʕat} ta-ʕazov et ha-ħéder, kol-kax hayu pane-ha mabiʼot tsaar u-keʼev.\\
	think.\textsc{pst}-1\textsc{sg} \textsc{comp} still a\_little 2\textsc{sg}.\textsc{f}-leave.\textsc{fut} \textsc{acc} \textsc{def}-room all-thus \textsc{exist}.\textsc{pst}.\textsc{pl} face-\textsc{poss}.3\textsc{sg}.\textsc{f} express.\textsc{pl}.\textsc{f} anguish and-pain\\
	\glt \lq I thought that \textbf{soon} she would leave the room, so much did her face express her anguish and pain.\rq{ }(\cite[550]{Glinert1989}, glosses added)
\end{exe}
\il{Qiang, Northern|)}\il{Hebrew, Modern|)}\is{additive|)}\is{irrealis|)}

\subsubsection[tocentry={}]{A closer look and discussion: Datooga \mbox{\textit{údu}-} as \lq not actually\rq{}}\il{Datooga, Barabayiiga-Gisamjanga|(} The prefix \mbox{\textit{údu}-} in Barabayiiga-Gisamjanga Datooga can mark the non\hyp actualisation of an anticipated past event when combined with negation.\is{negation} This is illustrated in (\ref{exNonHappening}).

\begin{exe}
	\ex Barabayiiga-Gisamjanga Datooga\label{exNonHappening}\\
	 \gll [N-ì-]néek-íid àbà híji áa \textbf{m}-\textbf{údú}-\textbf{qwáa}-\textbf{hìidu}\\
\textsc{ant}-3-close-\textsc{inch} \textsc{loc} here \textsc{conj} \textsc{neg}-still-\textsc{subj}.3-arrive.\textsc{caus.ven}\\
	\glt \lq He approached here \textbf{but he didn't bring it}.' \parencite[431]{Mitchell2021}
\end{exe}

\is{actionality|(}\is{telicity|(}
This non-happening use is restricted to telic predicates in the formally unmarked non-future tense.\is{tense} With atelic predicates, the combination of \mbox{\textit{údu}-} and negation\is{negation} predictably yields the negative phasal polarity concept \textsc{no longer}\is{no longer}.\footnote{Telic predicates and affirmative polarity yield a near past, see \Cref{sectionRemotenessPast}.\is{remoteness}} The most perspicuous explanation for the use I discuss here is hence a conventionalised mapping from times to possible worlds, in that the factual world no longer aligns with that in which the situation in question was envisaged.\il{Datooga, Barabayiiga-Gisamjanga|)}\is{actionality|)}\is{telicity|)}\is{no longer}

\subsection{Interrogative functions}\label{sectionInterrogative}\is{interrogative|(}
In this subsection, I survey a few pragmaticised and mostly language-specific uses that relate to interrogatives. In \Cref{sectionFollowUpQuestions} I discuss \ili{French} \textit{encore} and \ili{Ket} \textit{hāj} as markers of follow-up inquiries. In \Cref{sectionInterrogativeOther} I turn to the case of certain rhetorical questions featuring \ili{Mundang} \textit{ɓà}, as well as to \ili{German} \textit{noch} in \lq remind me again\rq{ }utterances.

\subsubsection{Follow-up questions}\label{sectionFollowUpQuestions}
\subsubsubsection{Introduction}
\il{French|(}\il{Ket|(}
Two expressions in the sample, \ili{French} \textit{encore} and Ket \textit{hāj}, are used in questions to mark a request for more information
(\appsref{appendixFrenchEncoreQ}, \ref{appendixKetQuestions}). Example (\ref{exQuestionsFrench}) illustrates French \textit{encore}. As in this example, \textit{encore} is always dislocated to the right and often conveys a notion of annoyance at the need to ask again.

\begin{exe}
	\ex French\label{exQuestionsFrench}
	\begin{xlist}
		\exi{A:}\textit{Seigneur Aristote, peut-on savoir ce qui vous met si fort en colère?}\\
		\lq My Lord Aristote, may one know what makes you so angry?'
		\exi{B:}\textit{Un sujet le plus juste au monde.}\\
		\lq A subject that is as reasonable as can be.'
		\exi{A:} \gll Et quoi, \textbf{encore}?\\
		and what still\\
		\glt \lq And what IS that?' (Molière, \textit{Le mariage forcé}, cited in \cite[214]{MosegaardHansen2008}, glosses added)
	\end{xlist}
\end{exe}

\subsubsubsection{A closer look and discussion} As \citeauthor{MosegaardHansen2002} (\citeyear{MosegaardHansen2002}, \citeyear[214]{MosegaardHansen2008}) and \textcite{Vaelikangas1982} point out, right-dislocated \textit{encore} is based on the same item in iterative function (\Cref{sectionIterative}) and involves a straightforward transfer from the propositional to the speech-act domain: \lq\lq it suggests that the question ought to be unnecessary … [and] focuses attention on the fact that a similar type of question has been asked at least once before\rq\rq{ }\parencite[214]{MosegaardHansen2008}. In other words, this is a textbook example of the tendency for meaning to become increasingly expressive\is{expressivity} (\Cref{sectionSemasiologicalChange}). 

The case of Ket \textit{hāj} appears to be mostly parallel to the French one, except that this expression maintains its usual pre-predicate position.\il{French|)} While there are no in-depth descriptions of \textit{hāj}'s contribution to interrogatives, the item also has an iterative use, and \textcite{Werner1997} consistently translates the relevant examples into \ili{German} using \textit{denn}, a marker commonly described as highlighting a follow-up question (e.g. \cite{Wegener2001}). This interpretation is in line with the fact that the one contextualised example in the data marks a \isi{repetition} of nearly the same question after an unsatisfactory first answer; see (\ref{exQuestionsKet}).\pagebreak

\begin{exe}
	\ex Ket\label{exQuestionsKet}
	\begin{xlist}
		\exi{A:} \textit{Bil\rq{}aŋsän\rq{} diˑmbes\rq{}in? Dɨ l\rq{}gat u škɔladiŋal\rq{} diˑmbes\rq{}in?}\\
	\lq Wer ist gekommen? Ob es Kinder sind, die aus der Schule gekommen sind? [Who has come? Will it be children, coming from school?]\rq
	\exi{B:} \textit{Bəŋ.}\\
	\lq Nein. [No.]\rq
	\exi{A:}
	\gll Anεt \textbf{haj} d-iˑ-m-bes\rq{}?\\
	who still \textsc{subj}.3-here-\textsc{pst}-move\\
	\glt \lq Wer ist denn gekommen? [Who is it \textbf{then}, who just came?]\rq{ }
	\exi{A:} \textit{Kaˀt hiˑɣ iˑmbes\rq{}}.\\
	\lq Ein alter Mann ist gekommen. [An old man came.]\rq{}
	\\(\cite[366–367]{Werner1997}, glosses added)
	\end{xlist}
\end{exe}
\il{Ket|)}

\subsubsection{Other interrogative functions}\label{sectionInterrogativeOther}
\addtocontents{toc}{\protect\setcounter{tocdepth}{4}}
In this subsection, I briefly discuss certain rhetorical questions in \ili{Mundang} as well as the conventionalised use of \ili{German} \textit{noch} in \lq remind me again\rq{ }questions. 

\subsubsubsection{Rhetorical \lq still\rq{ }questions or Mundang \textit{ɓàā}}\ili{Mundang} \textit{ɓà} can be used in conjunction with the interrogative marker \textit{nè} in a type of rhetorical question that signals that the addressee should not continue with whatever they are doing (\appref{appendixMundangInterrogative}). In this function, the two items can merge into a portmanteau morpheme \textit{ɓàā}, as shown in (\ref{exMundangInterrogative}).

\begin{exe}
	\ex \ili{Mundang}\label{exMundangInterrogative}\\
	\gll Mò dɔ̀ŋ yɛ́ɓ \textbf{ɓàā}?\\
	\textsc{subj}.2\textsc{sg} do.\textsc{\textsc{ipfv}} work still.\textsc{q}\\
	\glt \lq Travailles-tu encore? ([Are you \textbf{still} working?] le locuteur sait que l’interlocuteur est en train de travailler et il lui demande s’il continuera avec le travail [speaker knows that the addressee is working and \textbf{asks whether he will continue}])' \parencite[485]{Elders2000}
\end{exe}

Of course, the meaning of \ili{Mundang} \textit{ɓàā} is mostly compositional, and it involves an obvious intersubjective\is{subjectivity}\is{expressivity} inference. The reasons I follow \textcite[484–285]{Elders2000} in considering it a separate use are the formal irregularity and the apparent high degree of conventionalisation of the deontic component.

\subsubsubsection{\lq Remind me again\rq{} questions and German \textit{noch}}\il{German|(}The German \textsc{still} expression \textit{noch} frequently marks inquiries about information that was once known, but which the speaker has since forgotten or which is at least no longer immediately accessible to their mind (\appref{appendixGermanRemindMe}). In such questions, \textit{noch} can figure in the iterative collocation \textit{noch} (\textit{ein})\textit{mal} (\Cref{sectionIterativeViaIncrement}), as in (\ref{exRemindMe1}). Alternatively, it surfaces in its bare form, which is particularly common in monological questions like (\ref{exRemindMe2}).

\begin{exe}
	\ex 
		\begin{xlist}
			\exi{}German
			\ex
		Context: A waiter has forgotten the orders of each person.\label{exRemindMe1}\\
		\gll Was hat \textbf{noch}-\textbf{mal} jeder bestell-t?\\
		what have.3\textsc{sg} still-time everyone order-\textsc{ptcp}\\
		\glt \lq \textbf{Remind me again}, what did each of you order?' \parencite[63]{Sauerland2009}		
		
		\ex\label{exRemindMe2}
			\textit{…dann pflegte der Onkel das Lied von den heimatlosen Matrosen durchs Haus zu schmettern. Er hatte es einst auf seinen Seereisen gelernt.}\\
		\lq … the uncle then used to belt out the song of the homeless seafarers. He had learned it a long time ago during his sea voyages.\rq\\
		\exi{}\gll Wie hieß \textbf{noch} die zweit-e Strofe?\\
		how be\_called.\textsc{pst}.3\textsc{sg} still \textsc{def}.\textsc{nom}.\textsc{sg}.\textsc{f} second-\textsc{nom}.\textsc{sg}.\textsc{f} verse(\textsc{f})\\
		\glt \lq [Asking himself:] How did the second verse go \textbf{again}?\rq{ }(von der Vring, \textit{Spur im Hafen}, cited in \cite[63]{Iwasaki1977}, glosses added) 
	\end{xlist}
\end{exe}

The use of a collocation with an iterative meaning in (\ref{exRemindMe1}) is fairly unproblematic, and it is well-known to occur with other iterative markers such as \ili{Swedish} \textit{igen} in (\ref{exRemindMeAgainSwedish}). What does, however, appear to be cross-linguistically uncommon is the use of bare \textit{noch} in cases like (\ref{exRemindMe2}), given that this expression does not generally have a repetition-related function.\is{repetition} As \textcite{Iwasaki1977} discusses, the contribution of \textit{noch} in such instances intuitively shows traces of phasal polarity, albeit in regard to the speaker's belief state rather than the proposition itself. In other words, \textit{noch} can be understood as signalling the speaker's conviction that they should continue to be able to recall the information in question, although they no longer do. In fact, as \citeauthor{Iwasaki1977} also points out, this is a manifestation of a common theme with temporal-aspectual\is{aspect} adverbs in German. For instance, \textit{gleich} \lq immediately\rq{ }has a very similar use of signalling impatience or frustration at not being immediately able to recollect a piece of information; see (\ref{exRemindMeAgainGleich}).

\begin{exe}
	\ex \ili{Swedish}\label{exRemindMeAgainSwedish}\\
	\gll Vat hette han \textbf{igen}?\\
	what be\_called.\textsc{pst} 3\textsc{sg}.\textsc{m} again\\
	\glt \lq What was his name, \textbf{again}?\rq{ }(\cite[399]{Vaelikangas1982}, glosses added)	

	\ex German\label{exRemindMeAgainGleich}\\
	\gll Was hat er \textbf{gleich} ge-sag-t?\\
	what have.3\textsc{sg} 3\textsc{sg}.\textsc{m} immediately \textsc{ptcp}-say-\textsc{ptcp}\\
	\glt \lq \textbf{Remind me quickly}, what was it he said?\rq{ }(\cite[s.v. \textit{gleich}]{Duden}, glosses added)
\end{exe}\is{interrogative|)}\il{German|)}

\subsection{Exclamatory and directive functions}\label{sectionExclamatory}\is{exclamation|(}
\addtocontents{toc}{\protect\setcounter{tocdepth}{3}}
In this subsection, I survey a set of functions that relate to exclamations and commands.\is{command} Similar to what can be seen for interrogative uses in \Cref{sectionInterrogative}, all of these are either highly language-specific or found with no more than two sample expressions each. In \Cref{sectionAndHow} I discuss Serbian-Croatian-Bosnian\il{Serbian}\il{Croatian}\il{Bosnian} \textit{još} and Modern Hebrew\il{Hebrew, Modern} \textit{ʕod} as integral parts of \lq and how!\rq{ }formulae. In \Cref{sectionPoliteness} I turn to the use of \ili{Kalamang} \textit{tok} and \ili{Saisiyat} \textit{nahan} as markers of polite commands.\is{politeness} In \Cref{sectionExclamativeOther}, I discuss a set of remnant uses,\il{German|(} which involve German \textit{noch} in lamenting exclamations, the same item in certain collocations that serve as boosters,\is{booster}  as well as the case of \ili{Ewe} \mbox{\textit{ga}-}, which is a compulsory ingredient of negative\is{negation} commands.\is{command}\is{prohibitive} A common theme with all these cases, except for the German lamenting exclamations, is that they are only indirectly related to the same items as exponents of phasal polarity \textsc{still}.\il{German|)}

Lastly, note that there are related uses that I address in more specific places. Thus, in \Cref{sectionConcessiveInterjections} I discuss uses as \isi{concessive}  interjections,\is{interjection} and for a discussion of  \ili{Spanish} \textit{todavía} and \ili{French} \textit{encore} in counterfactual wishes and exclamations the reader is directed to \Cref{sectionScalarRestrictive}.

\subsubsection{\lq And how!\rq{}}\label{sectionAndHow}\il{Hebrew, Modern}
\subsubsubsection{Introduction} Two expressions in the sample, Modern Hebrew \textit{ʕod} and Serbian\hyp Croatian\hyp Bosnian\il{Serbian}\il{Croatian}\il{Bosnian} \textit{još}, are attested in fixed or semi-fixed collocations that serve as affirmative, reinforcing responses \lq and how!, very much so!\rq{ } (\appsref{appendixHebrewOdAndHow}, \ref{appendixBCMSSpecificational}). Example (\ref{exAndHowBCMS}) is an illustration featuring Serbian-Croatian-Bosnian \textit{još}.\pagebreak

\begin{exe}
		\ex Serbian-Croatian-Bosnian\il{Serbian}\il{Croatian}\il{Bosnian}\label{exAndHowBCMS}
		\begin{xlist}
		\exi{A:} \gll Ne spominji je već.\\
		\textsc{neg} mention.\textsc{ipfv}.\textsc{imp} 3\textsc{sg}.\textsc{acc}.\textsc{f} already\\
		\glt \lq Don't mention her anymore!'
		\exi{B:} \gll \textbf{Još} \textbf{kako} ću je spominjati.\\
		still how will.1\textsc{sg} 3\textsc{sg}.\textsc{acc}.\textsc{f} mention.\textsc{ipfv}.\textsc{inf}\\
		\glt \lq \textbf{And how} I will mention her!' (HrWac 2.2, glosses added)
		\end{xlist}
\end{exe}

\subsubsubsection{A closer look and discussion}
While in  (\ref{exAndHowBCMS}) Serbian-Croatian-Bosnian\il{Serbian}\il{Croatian}\il{Bosnian} \textit{još} goes together with \textit{kako} \lq how\rq{}, the second constituent of the construction may also be the quantificational wh word \textit{kaliko} \lq how much\rq{}, as shown in (\ref{exSpecificationalSerbian2}). This example illustrates another point: these collocations are commonly found in holophrastic or elliptical utterances.

\begin{exe}
	\ex Serbian-Croatian-Bosnian\il{Serbian}\il{Croatian}\il{Bosnian}\label{exSpecificationalSerbian2}
	\begin{xlist}
	\exi{A:}\gll Ja sam njih upozorava-o.\\
	1\textsc{sg} \textsc{cop}.1\textsc{sg} 3\textsc{pl}.\textsc{acc} warn.\textsc{ptcp}-\textsc{sg}.\textsc{m}\\
	\glt \lq I warned them.'
	\exi{B:}\gll \textbf{Još} \textbf{kaliko}.\\
	still how\_much\\
	\glt \lq \textbf{Very much so!}' (\cite[s.v. \textit{još}]{HJP}, glosses added)
	\end{xlist}
\end{exe}

A very similar instance is found in Modern Hebrew,\il{Hebrew, Modern|(} in form of the collocation shown in (\ref{exSpecificationalHebrew}) and listed by \citeauthor{Glinert1976} (\citeyear{Glinert1976}, \citeyear[282]{Glinert1989}), unfortunately without much additional description.\is{additive|(} In terms of their origins, these collocations clearly build on \textit{još} and \textit{ʕod} in additive function (\Cref{sectionAdditive}) and involve a transfer from the textual\is{textuality} onto the intersubjective\is{subjectivity}\is{expressivity} dimension of meaning. This interpretation finds support in the fact that Serbian-Croatian-Bosnian possesses\il{Serbian}\il{Croatian}\il{Bosnian} synonymous forms \textit{i te kako}/\textit{i te koliko} lit. \lq and also how (much)\rq{}.

\begin{exe}
	\ex Modern Hebrew\label{exSpecificationalHebrew}\\
	\gll Ve-\textbf{ʕod} ex!\\
	and-still how\\
	\glt \lq And how!\rq{ }\parencite[282]{Glinert1989})
\end{exe}
\il{Hebrew, Modern|)}\is{additive|)}

\subsubsection{Polite commands}\label{sectionPoliteness}\is{command|(}\is{politeness|(}
\subsubsubsection{Introduction}\il{Saisiyat|(}
\il{Kalamang|(}Two expressions in the sample, Kalamang \textit{tok} and Saisiyat \textit{nahan} (\appsref{appendixKalamangPoliteness}, \ref{appendixSaisiyatPoliteness}), have a function of mitigating commands. The examples in (\ref{exPolitenessKalamang}, \ref{exPolitenessSaisiyat}) are illustrations.

\begin{exe}
	\ex Kalamang\label{exPolitenessKalamang}\\
	\gll Ma he min ma he mat nawarar ka \textbf{tok} \textbf{parar}=\textbf{te}.\\
	3\textsc{sg} already sleep 3\textsc{sg} already 3\textsc{sg}.\textsc{obj} wake\_up 3\textsc{sg} still get\_up=\textsc{non}.\textsc{fin}\\
	\glt \lq He slept, [I] woke him up, \lq\lq \textbf{You get up!} (before you do anything else)"' (\cite{Visser2021a}; Eline Visser, p.c.)
	
	\ex Saisiyat\label{exPolitenessSaisiyat}\\
	\gll Si\rq{}ael \textbf{nahaen}.\\
	eat.\textsc{imp} still\\
	\glt \lq Chi gė dongxi (zài zŏu) ba! / Come have a bite (before you leave)!\rq{ }\parencite[120]{Huang2008}
\end{exe}

\subsubsubsection{A closer look and discussion}
As hinted at in the free translations of (\ref{exPolitenessKalamang}, \ref{exPolitenessSaisiyat}), this use can be traced back to the \lq first, for now\rq{ }function of both items (\cite[119–120]{Huang2008}; Eline Visser, p.c.), which I discuss in \Cref{sectionFirst}. Like the uses addressed in the preceding subsections, it clearly instantiates the tendency of meaning to become more intersubjective,\is{subjectivity}\is{expressivity} in this instance functioning in the realm of negative politeness (see \cite{BrownLevinson1987}). More specifically, it can be explained by recourse to two mutually compatible pathways and politeness strategies. In the first of these, the demand on the addressee and the threat to their negative face are reduced by asking for an action that takes place only for a limited amount of time. The second scenario, suggested by \textcite[119–120]{Huang2008} for Saisiyat \textit{nahan}, stresses the sequential component of the \lq first, for now\rq{ }use. By placing the demand in a succession of events, the speaker acknowledges that the addressee has other things to do and thereby effectively plays a variation on \citeauthor{BrownLevinson1987}'s (\citeyear{BrownLevinson1987}) strategy to \lq\lq admit the impingement\rq\rq{}.\is{command|)}\is{politeness|)}\il{Kalamang|)}\il{Saisiyat|)}
\pagebreak

\subsubsection{Other exclamatory/directive functions}\label{sectionExclamativeOther}
\addtocontents{toc}{\protect\setcounter{tocdepth}{4}}
In this subsection I briefly turn to some item-specific exclamatory and directive functions.\il{German|(} I first discuss the use of German \textit{noch} in lamenting exclamations. Afterwards I turn to the same item in certain collocations that serve as pragmatic boosters.\is{booster} Lastly I address the compulsory use of \ili{Ewe} \mbox{\textit{ga}-} in prohibitives.\is{prohibitive}\is{command}

\subsubsubsection{Lamenting exclamations with German \textit{noch}}
German \textit{noch} has a strongly conventionalised use in exclamations of the type illustrated in (\ref{exGermanWarenNochZeiten}). As in this example, these exclamations invariable feature the verb-second order characteristic of declarative clauses (\appref{appendixGermanExclamations}). On the meaning side of things, they feature contrastive topics\is{topic} and convey emotive notions such as lament or woefulness (e.g. that the good times are over). At the same time, they serve to invite agreement from the hearer \parencite[s.v. \textit{noch}]{Duden}.

\begin{exe}
	\ex German\label{exGermanWarenNochZeiten}\\
	\gll Das war-en \textbf{noch} Zeit-en!\\
	3\textsc{sg}.\textsc{n} \textsc{cop}.\textsc{pst}-3\textsc{pl} still time-\textsc{pl}\\
	\glt \lq Those were the days!\rq{ }(\cite[633]{MetrichFaucher2009}, glosses added)
\end{exe}

Expressive notions notwithstanding, the contrast between \isi{persistence} of an earlier state and its subsequent \isi{discontinuation} remains visible in (\ref{exGermanWarenNochZeiten}). In cases like (\ref{exGermanWarenNochZeitenVerlassen}), on the other hand, the (inter-)subjective\is{subjectivity}\is{expressivity} meaning dominates \parencite[633]{MetrichFaucher2009}. However, such instances do provide hints of the \isi{marginality} sense (\Cref{sectionMarginality}), such as \lq him\rq{ }being the last outpost of reliability.

\begin{exe}
	\ex German\label{exGermanWarenNochZeitenVerlassen}\is{persistence}\\
	\gll Auf ihn kann man sich \textbf{noch} verlass-en!\\
	on 3\textsc{sg}.\textsc{acc}.\textsc{m} can.3\textsc{sg} \textsc{impr} \textsc{refl}.3 still rely-\textsc{inf}\\
	\glt \lq At least \textit{he} is reliable (if no one else)!\rq{ }(\cite[633]{MetrichFaucher2009}, glosses added)
\end{exe}

\subsubsubsection{German \textit{nochmal}: From repetition to booster}\is{repetition|(}\is{booster|(}
German \textit{noch}, in the iterative collocations \textit{nochmal} (and variants; \Cref{sectionIterativeViaIncrement}), can serve as a booster in directive speech acts, exclamations of frustration, and the like (\appref{appendixGermanIterativeDirectives}). Example (\ref{exBooster1}) is an illustration.

\begin{exe}
	\ex German\label{exBooster1}\\
	\gll Jed-er Mensch ... überhaupt: jed-e Kreatur ... man muß doch an etwas glaub-en in der Welt, \textbf{verdammt} \textbf{noch}-\textbf{mal}!\\
	every-\textsc{nom}.\textsc{sg}.\textsc{m} human(\textsc{m}) {} generally every-\textsc{nom}.\textsc{sg}.\textsc{f} creature(\textsc{f)} {} \textsc{impr} must.3\textsc{sg} \textsc{dm} at something believe-\textsc{inf} in \textsc{def}.\textsc{dat}.\textsc{sg}.\textsc{f} world(\textsc{f}) damnit still-time\\
	\glt \lq Every person, more generally, every creature … you need something to believe in, \textbf{Goddammit}!\rq{ }(Sparschuh, \textit{der Zimmerspringbrunnen}, glosses added)
\end{exe}

\il{French|(}
In all likelihood, this function is mediated by an application of the collocation's iterative meaning to the speech-act level, as in (\ref{exBooster2}). The latter usage pattern is also attested for the equivalent French collocation \textit{encore une fois}, as in (\ref{exBooster3}), where it, however, does not appear to have generalised to the same degree as in German.

\begin{exe}
	\ex German\label{exBooster2}\\
	\gll \textbf{Noch}-\textbf{mal}: du soll-st deinen Müll nicht einfach in die Landschaft werf-en!\\
	still-time 2\textsc{sg} should-2\textsc{sg} \textsc{poss}.2\textsc{sg}:\textsc{acc}.\textsc{sg}.\textsc{m} trash(\textsc{m}) \textsc{neg} simply in \textsc{def}.\textsc{acc}.\textsc{sg}.\textsc{f} landscape(\textsc{f}) throw-\textsc{inf}\\
	\glt \lq [\textbf{I'm telling you}] \textbf{again}: don't just throw your trash into nature!' (personal knowledge)
	\ex French\label{exBooster3}\\
	\gll \textbf{Encore} \textbf{une} \textbf{fois}, rest-ez calm-es.\\
	still \textsc{indef}.\textsc{sg}.\textsc{f} time(\textsc{f}) stay-\textsc{imp}.\textsc{pl} quiet-\textsc{pl}\\
	\glt (\textbf{I'm saying it}) \textbf{again}, be quiet! \parencite[51 fn11]{Borillo1984}
\end{exe}
\il{French|)}\is{exclamation|)}\is{repetition|)}\is{booster|)}\il{German|)}

\subsubsubsection{Ewe \textit{ga}- and prohibitives}\is{command|(}\is{prohibitive|(}\is{negation|(}The \ili{Ewe} \textsc{still} expression \mbox{\textit{ga}-} is compulsory in negative commands (\appref{appendixEweProhibitives}). This is illustrated in (\ref{exProhibitive1}).

\begin{exe}
	\ex \ili{Ewe}\label{exProhibitive1}\\
	\gll Me-\textbf{ga}-dzró nú \textit{o}.\\
	\textsc{neg}.2\textsc{sg}-still-desire thing \textsc{neg}\\
	\glt \lq Do not crave for things.\rq{ }\parencite[358]{Ameka1991}
\end{exe}

As pointed out before me by \textcite[53]{Ameka1991} and \textcite[67 fn1]{Westermann1907}, the prohibitive use is clearly motivated by \mbox{\textit{ga}-} as iterative \lq again\rq{ } (\Cref{sectionIterative}), as well as by the same item forming part of the expression for \textsc{no longer}\is{no longer} (which ultimately goes back to marking disrepetition).\is{repetition} It is safe to assume that its employment in prohibitives started out with control predicates and commands to either not repeat an action, or to stop one, and that a high type frequency ultimately led to semantic bleaching.\il{Ewe}\is{command|)}\is{prohibitive|)}\is{negation|)}
