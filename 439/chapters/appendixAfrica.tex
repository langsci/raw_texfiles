\chapter{Africa}

\section{Adama Fulfulde (adam1253, fub)}\il{Fulfulde, Adamawa|(} 
\label{appendixAdamawa}
\subsection{tawon}

\subsubsection{General information}
\begin{itemize}
	\item Form: there is a free variant \textit{tawan}; in spoken registers, often reduced to \textit{taw}.
	\item Wordhood: free morpheme.
	\item Syntax: fixed, clause-final position.
	\item Etymology: in all likelihood < \ili{Kanuri} (Saharan) \textit{dùwonyì} \lq first, before, still'.
\end{itemize}


\subsubsection{As a  \lq still\rq{ }expression}
\begin{itemize}
	\item  \textcite[54, 387, 435]{Kingenheben1963}, \textcite{Kramer2021Adamawa} and \textcite[s.v. \textit{yet}]{deWolf1995}.
	\item Specialisation: described in line with my definition by \textcite{Kramer2021Adamawa}.
	\item Pragmaticity: dependent on the specific variety; in the spoken variety that \citeauthor{Kramer2021Adamawa} terms \lq\lq Adamawa Fula Communis", it is restricted to the unexpectedly late scenario.
	\item Polarity sensitivity: inner negation yields \textsc{not yet}.
\end{itemize}

\begin{exe}
	\ex
	Context: A young man was asked whether he is the owner of a shop.\\
	\gll Kay mi derkeejo \textbf{tawon}.\\
	no 1\textsc{sg} young\_man still\\
	\glt \lq No, I am still a young man.' \parencite[250]{Kramer2021Adamawa}

	\ex
	\gll Min mbiiɓe ma yaake en ɗonno haa Misra \textbf{tawon}.\\
	1\textsc{pl}.\textsc{excl} tell.\textsc{pl} \textsc{obj}.2\textsc{sg} \textsc{conj} 1\textsc{pl}.\textsc{incl} \textsc{loc}.\textsc{ant} at Egypt still\\
	\glt \lq We told you (this), when we were still in Egypt.'
	\\(Genesis 29: 9, cited in \cite[250]{Kramer2021Adamawa})
	
	\ex
	\gll O yiɗi yah-go booko, ammaa o peeto \textbf{tawon}.\\
	\textsc{ncl}1 want.\textsc{pfv} go-\textsc{inf} school but \textsc{ncl}1 small still\\
	\glt \lq He wants to go to school but he is still (too) small.\rq{ }\parencite[250]{Kramer2021Adamawa}
\end{exe}

\subsubsection{Uses related toother phasal polarity concepts}
\paragraph{Not yet}\label{appendixAdamawaNotYet}
\begin{itemize}
	\item In the data, this is only attested as a negative reply to polar questions.
\end{itemize}

\begin{exe}
	\ex
	\gll On 'don yaha ngesa hande naa? – A'a, \textbf{tawon}. Sey jango.\\
	\textsc{ncl}24 \textsc{loc} go field today \textsc{q} {} no still until tomorrow\\
	\glt \lq Are you (pl.) going to the field today? -- No, [not] until tomorrow.' (\cite[122]{PelletierSkinner1981}, glosses added)

	\ex
	\gll A danyi yaa-de? – \textbf{Tawon}!\\
	2\textsc{sg} succeed.\textsc{pfv} go-\textsc{inf} {} still\\
	\glt \lq Did you manage to leave? -- Not yet!\rq{ }(\cite[s.v. \textit{yet}]{deWolf1995}, glosses added)
\end{exe}


\subsubsection{Broadly adverbial temporal-aspectual uses}

\paragraph{First, for now}\label{appendixAdamawaFirst}
\begin{itemize}
	\item \textcite[54, 387, 435, 452]{Kingenheben1963}, \textcite{Kramer2021Adamawa}, \textcite[63]{Taylor1953}, \textcite[177]{Labatut1982} and \textcite[s.v. \textit{first}]{deWolf1995}.
\end{itemize}

\begin{exe}
	\ex
	Context: Jacob has prepared a meal. Esau has returned hungry from his field and has asked for some of the food. Jacob answers:\\
	\gll Soorranam daraja afaaku ma hannde nden \textbf{tawon}.\\
	sell.\textsc{dat}:\textsc{obj}.1\textsc{sg} privilege firstbornness \textsc{poss}.2\textsc{sg} today \textsc{dem} still\\
	\glt \lq Sell me your birthright today first (i.e. then I will give you food).' (Genesis 25: 31; cited in \cite[250]{Kramer2021Adamawa})

	\ex
	\gll Sey to o wari \textbf{tawan}.\\
	until when/if \textsc{ncl}1 come.\textsc{pfv} still\\
	\glt \lq He must come first.'\footnote{\textit{Sey to} is a collocation meaning \lq only if, under the condition that', see \textcite[s.v. \textit{condition}]{deWolf1995}.} (\cite[s.v. \textit{first}]{deWolf1995}, glosses added)
	
	\ex Context: A response to a greeting.\\
	\gll Jam ni \textbf{tawon}.\\
	fine thus still\\
	\glt \lq Fine for the moment.\rq{ }(\cite[187–188]{PelletierSkinner1981}, glosses added)
\end{exe}
\il{Fulfulde, Adamawa|)}

\section{Bambara (bam, bamb1269)}
\label{appendixBambara}\il{Bambara|(} 

\subsection{Introductory remarks} I am indebted to Klaudia Dombrowsky\hyp Hahn for discussing Bambara data with me, and for helping with tricky glosses. According to \citeauthor{DombrowskyHahn2020} (\citeyear{DombrowskyHahn2020}, \citeyear{DombrowskyHahn2021}), Bambara has four expressions for the concept of \textsc{still}. Only two of them, \textit{túguni} and \textit{bìlen}, appear to have additional functions. Lastly, note that Bambara has a typical West African \lq\lq factative" \parencite[346–347]{Welmers1973} system, i.e. those verbs typically labelled as being \lq\lq stative" have an ongoing state reading in the perfective aspect  as opposed to the reading of a bygone situation found with dynamic verbs (see \cite{HewsonBambara}).

\subsection{túgúni}

\subsubsection{General information}
\begin{itemize}
	\item Form: there is free variation \textit{túguni}$\sim$\textit{túgun}$\sim$\textit{tún}.
	 \item Wordhood: free morpheme.
	 \item Syntax: fixed, clause-final position.
	 \item Etymology: originally an iterative expression.
\end{itemize}


\subsubsection{As a  \lq still\rq{ }expression}
\begin{itemize}
	\item \citeauthor{Dumestre2003} (\citeyear[327]{Dumestre2003}, \citeyear[1003]{Dumestre2011}) and \citeauthor{DombrowskyHahn2020} (\citeyear{DombrowskyHahn2020}, \citeyear{DombrowskyHahn2021}).
	\item Specialisation: \textcite{DombrowskyHahn2021} identifies this marker as one that is in line with my definition. 
	\item Pragmaticity: restriced to the neutral scenario.
	\item Polarity sensitivity: outer negation yields \textsc{no longer}.
	\item Further note: rarely used as a \textsc{still} expression.
\end{itemize}

\begin{exe}
	\ex\label{exAppendixBambara1}
	\gll Nkà ń tʼ à dɔ́n lʼ à fɔ́ à bɛ́ bálo lá \textbf{tún}.\\
	but 1\textsc{sg} \textsc{neg}.\textsc{ipfv} 3\textsc{sg} know \textsc{inf} 3\textsc{sg} say 3\textsc{sg} \textsc{cop} life at still\\
	\glt \lq But I don't know if she is still alive.' \parencite[14]{DombrowskyHahn2021}
	
	\ex\label{exAppendixBambara2}
	\gll Ò tɛ́ bìlen hábada kà né Silimakan Ardo tó bálo lá \textbf{túgun}.\\
	\textsc{dem} \textsc{ipfv}.\textsc{neg} still never \textsc{inf} 1\textsc{sg} S. A. remain life at still\\
	\glt \lq Cela ne se fera plus tant que moi Silamakan je serai vivan. [It will not ever happen again for as long as I, Silimakan, remain alive.]\rq{ }(\cite[327]{Dumestre2003}, glosses added)
\end{exe}

\subsubsection{Broadly adverbial temporal-aspectual uses}
\paragraph{Iterative and restitutive}
\label{appendixBambaraTuguniIterative}
\begin{itemize}
	\item \citeauthor{Dumestre2003} (\citeyear[327]{Dumestre2003}, \citeyear[1003]{Dumestre2011}) and \textcite{DombrowskyHahn2020}.
	\item This function is restricted to the perfective aspect with dynamic predicates.
	\item Both iterative (\ref{exAppendixBambaraIterative1}–\ref{exAppendixBambaraIterative3}) and restitutive uses (\ref{exAppendixBambaraRestitutive1}, \ref{exAppendixBambaraRestitutive2}) are available.
	\item Examples (\ref{exAppendixBambaraIterative2}, \ref{exAppendixBambaraRestitutive2}) illustrates that the iterative and restitutive readings survive under negation.
	\item In (\ref{exAppendixBambaraIterative3}) \textit{túgun} is translated as \lq aussi [also]\rq{ }by \textcite{Dumestre2011}. However, \textit{túguni} does not appear to have an additive function outside this context, and a more literal translation would be \lq You spoke again, saying' (Klaudia Dombrowsky\hyp Hahn, p.c.). Note that \textit{kó} is a defective verb, hence the absence of a perfective marker.
\end{itemize}

\begin{exe}
	\ex\label{exAppendixBambaraIterative1}
	Context: A fish spoke and was then cooked.
	\exi{}\gll À	y’ í kánto \textbf{túguni}.\\
	3\textsc{sg} \textsc{pfv}.\textsc{tr} \textsc{refl} speak	 still\\
	\glt \lq The fish spoke again.' \parencite[118]{DombrowskyHahn2020}

	\ex
	\label{exAppendixBambaraIterative2}
	Context: A jinn came to a village each rainy season, but then…\\
	\gll Jínɛ-kɛ túnun-na. Kàbini ó kɛ́-ra, mɔ̀gɔ sí má jínɛ yé \textbf{túgun}.\\
	jinn-man get\_lost-\textsc{pfv}.\textsc{intr} since \textsc{anaph} happen-\textsc{pfv}.\textsc{intr} person no \textsc{neg}.\textsc{pfv} jinn see still\\
	\glt \lq It disappeared. Since this happened, nobody saw the jinn again.' \parencite[118]{DombrowskyHahn2020}

	\ex
	\label{exAppendixBambaraIterative3}
	\gll I kó \textbf{túgun} kó.\\
	2\textsc{sg} say/\textsc{quot} still say/\textsc{quot}\\
	\glt \lq Tu dis aussi que … [You also said …]\rq{ }(\cite[1003]{Dumestre2011}, glosses added)

	\ex\label{exAppendixBambaraRestitutive1}
	\gll À yé syɛ̀ mìnɛ kʼ à bìla \textbf{túgun}.\\
	3\textsc{sg} \textsc{pfv}.\textsc{tr} chicken catch \textsc{inf} 3\textsc{sg} let still\\
	\glt \lq He caught the chicken and let it free again.\rq{ }\parencite[118]{DombrowskyHahn2020}

	\ex\label{exAppendixBambaraRestitutive2}
	Context: One person is about to smash a cane on another one’s head. A third person has appeared and prevents this from happening.\\
	\gll Í bílo fìla kɔ́rɔtàlen bɛ́ cógo mín à ní bére tɛ́ jìgin \textbf{túgun} dɛ́.\\
	2\textsc{sg} arm two raise.\textsc{part} \textsc{cop} way \textsc{rel} 3\textsc{sg} and stick \textsc{neg}.\textsc{ipfv} lower still \textsc{intens}\\
	\glt \lq The way you hold your two arms, you will not lower the one holding the cane again.' (\cite[100]{Dumestre1979}, cited in \cite[118]{DombrowskyHahn2020})
\end{exe}

\subsection{bìlen}
\subsubsection{General information}
\begin{itemize}
	\item Form: there is free variation \textit{bìlen}$\sim$\textit{blèn}.
	\item Wordhood: free morpheme.
	\item Syntax: either in post-subject or clause-final position.
	\item Etymology: originally an iterative expression.
\end{itemize}


\subsubsection{As a  \lq still\rq{ }expression}
\label{exAppendixBambaraBilenStill}
\begin{itemize}
	\item \textcite[311]{Dumestre2003}, \citeauthor{DombrowskyHahn2020} (\citeyear{DombrowskyHahn2020}, \citeyear{DombrowskyHahn2021}) and \textcite[123]{Vydrine2015}.
	\item Specialisation: described in line with my definition by \textcite{DombrowskyHahn2021}.
	\item Pragmaticity: \textit{bìlen} on its own is only compatible with the neutral scenario; for the unexpectedly late scenario it needs to be combined with another \textsc{still} expression, \textit{háli bì}.
	\item Polarity sensitivity: outer negation yields \textsc{no longer}.
	\item Further notes: as a \textsc{still} expression, \textit{bìlen} appears to be restricted to the dialect of Segu. Furthermore, this seems to be an infrequent use. \textcite[311]{Dumestre2003} notes \lq\lq l'usage comme interjection … avec le sense de \lq à cette heure-ci', \lq encore maintenant' (avec une nuance de surprise et de reproche) [the usage as an interjection … with the sense of \lq at this time, even now' (with a surprised and reproachful nuance)]\rq\rq.
\end{itemize}

\begin{exe}
	\ex
	\gll Ń bɛ́ ò báara lá \textbf{bìlen}.\\
	1\textsc{sg} \textsc{cop} \textsc{anaph} work at still\\
	\glt \lq I am still doing that work.ʼ \parencite[120]{DombrowskyHahn2020}
\end{exe}

\subsubsection{Broadly adverbial temporal-aspectual uses}
\paragraph{Iterative and restitutive}
\label{appendixBambaraBilenIterative}
\begin{itemize}
	\item \textcite{DombrowskyHahn2020} and \textcite[123]{Vydrine2015}.
	\item Both iterative (\ref{exAppendixBambaraBilenIterative1}) and restitutive uses (\ref{exAppendixBambaraBilenRestitutive2}) are available; without further discourse context, ex. (\ref{exAppendixBambaraBilenIterative2}) is ambiguous between the two.
		\item According to \textcite{DombrowskyHahn2020}, iterative and restitutive readings are restricted to the perfective aspect with dynamic predicates. Example (\ref{exAppendixBambaraBilenIterative2}) suggests that they are also available with the locative copula.
	\item Like the use of \textit{bìlen} as \textsc{still}, the repetition function is described as infrequent.
\end{itemize}
\begin{exe}
	\ex
	\label{exAppendixBambaraBilenIterative1}
	\gll À \textbf{bìlén} bòli-lá kà nʼ à fɔ́ à bámusò yé.\\
	3\textsc{sg} still run-\textsc{pfv}.\textsc{intr} \textsc{inf} come 3\textsc{sg} say 3\textsc{sg} mother to\\
	\glt \lq She ran again to tell her mother about this.\rq{ }(\cite[123]{Vydrine2015}, glosses added)

	\ex
	\label{exAppendixBambaraBilenIterative2}
	\gll Í bɛ́ yǎn \textbf{bìlén}.\\
	2\textsc{sg} \textsc{cop} here still\\
	\glt \lq Here you are again.' (\cite[123]{Vydrine2015}, glosses added)

	\ex \label{exAppendixBambaraBilenRestitutive2}
	Context: Elephant said a forbidden word and immediately fell to the ground.\\
	\gll Sàma má wúli \textbf{bìlen}.\\
	elephant \textsc{neg}.\textsc{pfv} get\_up still\\
	\glt \lq Elephant did not get up again.'  \parencite[121]{DombrowskyHahn2020}
\end{exe}

\subsubsection{Other functions}
\paragraph{Emphatic negation}
\begin{itemize}
	\item \textcite[123]{Vydrine2015}.
	\item There is only one example of this use in the data consulted.
	\item It is unclear if this is a homophonous marker, and if not, how this function relates to the other functions of \textit{bìlen}.
\end{itemize}

\begin{exe}
	\ex
	\gll  Ù má sɔ̌n kà dòn sánsarà kɔ́nɔ \textbf{bìlén}.\\
	3\textsc{pl} \textsc{neg}.\textsc{pfv} agree \textsc{inf} enter cage inside still(?)\\
	\glt \lq They definitely refused to enter the cage.\rq{ }(\cite[123]{Vydrine2015}, glosses added)
\end{exe}

\paragraph{Negative hypotheticals}
\begin{itemize}
	\item \textcite[311]{Dumestre2003} and \textcite[123]{Vydrine2015}.
	\item In this function, which is described as archaic, \textit{bìlen} marks negative hypothetical conditionals (\lq if not'). It is unclear whether this is the same marker as phasal polarity \textit{bìlen} \lq still', and if so, how it relates to the other functions of this item.
	\item Syntax: invariably in post-subject position.
\end{itemize}

\begin{exe}
	\ex
	\gll Í \textbf{bìlén} máà dòn fána, í bɛ́ tèrejuguya sáyà kɛ́.\\
	2\textsc{sg} still(?) \textsc{neg}.\textsc{pfv} enter also 2\textsc{sg} \textsc{ipfv} fate\_of\_sb\_attracting\_evil death do\\
	\glt \lq If you don't enter there, you will die ignominiously.\rq{ }(\cite[123]{Vydrine2015}, glosses added)
	
	\ex
	\gll Í bìlén bámanankaǹ mɛ́n…\\
	2\textsc{sg} still(?) Bamana\_language understand\\
	\glt \lq If you don't understand the Bamana language…\rq{ }(\cite[123]{Vydrine2015}, glosses added)
\end{exe}
\il{Bambara|)}

\section{Barabayiiga-Gisamjanga Datooga (tcc, dato1239)}\il{Datooga, Barabayiiga-Gisamjanga|(}
\label{appendixDatooga}
\subsection{Introductory remarks}
The data encompass the Barabayiiga-Gisamjanga varieties of North-Central Datooga (glottolog: nort3277).

\subsection{údu-}
\subsubsection{General information}
\begin{itemize}
	\item Wordhood: bound morpheme (prefix).
	\item Form: There is a free variant \textit{dúú-}.
	\item Etymology: unclear, but in all likelihood derived from a former biverbal construction (cf. \cite[428]{Mitchell2021}).
	\item Further note: in the context of \mbox{\textit{údu}-}, tense-aspect inflection is reduced to a simple opposition of future vs. non-future, the latter being formally unmarked.
\end{itemize}


\subsubsection{As a  \lq still\rq{ }expression}
\begin{itemize}
\item \citeauthor{Rottland1982} (\citeyear[174, 179]{Rottland1982}; \citeyear[226]{Rottland1983}) and \textcite{Mitchell2021}.
\item Specialisation: the description by \textcite[426]{Mitchell2021} meets my definition.
\item Polarity sensitivity: outer negation yields \textsc{no longer}.
\item Pragmaticity: compatible with both scenarios.
\item Further note: \mbox{\textit{údu}-} as \textsc{still} is restricted to atelic predicates.
\end{itemize}

\largerpage
\begin{exe}
	\ex\label{exAppendixDatooga1}
	Context: About in-law avoidance. Women of child-bearing age strictly obey linguistic taboos, but their importance diminishes with age.\\
	\gll Gídá g-\textbf{ùdù}-gw-á-jíil-êan íi-lási qêawùn-gá gwéargwèe-da éa mùy\\
thing \textsc{aff}-still-\textsc{aff}-\textsc{subj}.3-reproduce-\textsc{ven} \textsc{cond}:\textsc{subj}.2\textsc{sg}-cut.\textsc{appl} name-\textsc{pl} elder-\textsc{sg} \textsc{cop} bad.\textsc{sg}\\
	\glt \lq For someone who is still having children, if you say the name of your father-in-law, itʼs bad.' \parencite[132]{Mitchell2015}
	
	\ex
	\begin{xlist}
				\exi{A:} \textit{Géaléabu qwêenga íiyá níihíidí gáwíischòoda?}\\
		\lq Iʼm back from firewood, dear; Have you finished milking?'

		\exi{B:} \gll G-\textbf{út}-tá-gáw-ìi-s-chì.\\
		\textsc{aff}-still-\textsc{subj}.1\textsc{sg}-milk-\textsc{plur}-\textsc{appl}-\textsc{ap}\\ 
		\glt \lq Iʼm still milking' \parencite[436]{Mitchell2021}
\end{xlist}
\end{exe}

\subsubsection{Broadly adverbial temporal-aspectual uses}
\paragraph{Iterative (and restitutive?)}\label{appendixDatoogaIterative}
\begin{itemize}
	\item \textcite{Mitchell2021}.
	\item This function is restricted to telic predicates in the future tense (both affirmative and negated; ex. \ref{exAppendixDatoogaIterative1}, \ref{exAppendixDatoogaIterative2})  and to prohibitives (\ref{exAppendixDatoogaIterative3}).
	\item \citeauthor{Mitchell2021} describes (\ref{exAppendixDatoogaIterative1}) as iterative: \lq\lq  there will be a repeated instance of a prior fire-lighting event" (\citeyear[429]{Mitchell2021}). Similarly, she paraphrases the negative future example (\ref{exAppendixDatoogaIterative2}) as \lq\lq will not repeat/occur again" \parencite[429]{Mitchell2021}. In absence of further discourse context I cannot exclude a restitutive reading, i.e. \lq relight the fire', \lq come back'.
\end{itemize}

\begin{exe}
	\ex\label{exAppendixDatoogaIterative1}
	\gll G-ày-g-\textbf{úd}-êe-gwèarsíina gìchá?\\
\textsc{aff}-\textsc{fut}-\textsc{aff}-still-\textsc{subj}.1\textsc{pl}-light\_fire.\textsc{appl} again\\ 
	\glt \lq Shall we light the fire again?' \parencite[429]{Mitchell2021}\footnote{\textit{gìchá} \lq again' seems to emphasize the repetition.}

	\ex\label{exAppendixDatoogaIterative2}
	\gll M-ày-g-\textbf{údú}-gá-bíigu.\\
\textsc{neg}-\textsc{fut}-\textsc{aff}-still-\textsc{subj.}3-return.\textsc{ven}\\
	\glt \lq They [things of the past] wonʼt return again.' \parencite[429]{Mitchell2021}

	\ex\label{exAppendixDatoogaIterative3} Context: Scolding a child.\\
	\gll Ádá g-\textbf{ùt}-tá-yíi ... Ádá g-\textbf{ùd}-óo-sínyì níi n-ì-hít.\\
\textsc{proh} \textsc{aff}-still-\textsc{subj}.1\textsc{sg}-hear {} \textsc{proh} \textsc{aff}-still-\textsc{subj}.2\textsc{pl}-do \textsc{prox} \textsc{ant}-\textsc{subj}.3-come.\textsc{ven}\\
	\glt \lq Don't let me hear this again; don't do this again; it's done.\rq{ }\parencite[429–430]{Mitchell2021}
\end{exe}

\paragraph{Near past}
\label{appendixDatoogaNearPast}
\begin{itemize}
	\item \textcite{Mitchell2021}.
	\item This function is restricted to telic predicates in affirmative non-future forms. With predicates that can be construed as either telic or atelic, such as \lq eat' in (\ref{exAppendixDatoogaNearPast3}), both the immediate past interpretation as well as the phasal polarity one are available.
\end{itemize}

\begin{exe}
	\ex \gll G-\textbf{út}-tá-hídù.\\
	\textsc{aff}-still-\textsc{subj}.1\textsc{sg}-arrive.\textsc{ven}\\
	\glt \lq  Iʼve just arrived.' \parencite[430]{Mitchell2021}

	\ex
	\gll G-á-bày èa míi dá-yíi hà g-\textbf{út}-tà-yíi qámnà.\\
\textsc{aff}-\textsc{subj}.3-be\_early \textsc{conj} \textsc{neg}.\textsc{cop} \textsc{subj}.1\textsc{sg}-hear \textsc{dm} \textsc{aff}-still-\textsc{subj}.1\textsc{sg}-hear now\\
	\glt \lq I've never heard this; I've just heard it now.' \parencite[431]{Mitchell2021}

	\ex\label{exAppendixDatoogaNearPast3}
	\gll G-\textbf{út}-téa-ág-ìi-s-chì.\\
	\textsc{aff}-still-\textsc{subj}.1\textsc{sg}-eat-\textsc{plur}-\textsc{appl}-\textsc{antip}\\
	\glt i.\phantom{i} \lq I've just eaten [something].'\\
	ii. \lq I'm still eating.' \parencite[430]{Mitchell2021}
\end{exe}

\subsubsection{Broadly modal and interactional uses}
\paragraph{Non-happening}\label{appendixDatoogaNonHappening}
\begin{itemize}
	\item \textcite{Mitchell2021}.
	\item This function is restricted to telic predicates in negated non-future forms.
	\item This likely goes back to \textsc{no longer}, via a mapping from times to possible worlds (a predicted course of events ceases to be true).
\end{itemize}

\begin{exe}
	\ex \gll [N-ì-]néek-íid àbà híji áa m-\textbf{údú}-qwáa-hìidu\\
\textsc{ant}-3-close-\textsc{inch} \textsc{loc} here \textsc{conj} \textsc{neg}-still-\textsc{subj}.3-arrive.\textsc{caus.ven}\\
	\glt \lq He approached here but he didn't bring it.' \parencite[431]{Mitchell2021}

	\ex\gll M-\textbf{údú}-qà-m\\
	\textsc{neg}-still-\textsc{subj}.3-die\\
	\glt \lq S/he didn’t actually die.' \parencite[432]{Mitchell2021}
\end{exe}
\il{Datooga, Barabayiiga-Gisamjanga|)}

\section{Bende (bdp, bend1258)}\il{Bende|(}
\label{appendixBende}
\subsection{Introductory remarks}
Bende has a fairly typical Narrow Bantu noun class system. I have glossed the individual classes as \textsc{ncl} \lq noun class' plus Arabic numeral, following the common Bleek-Meinhof system.

\subsection{syá-}
\subsubsection{General information}
\begin{itemize}
	\item Wordhood: bound morpheme (verb prefix).	
	\item Form: there is a free variant \mbox{\textit{sí}-} in some functions.
\end{itemize}


\subsubsection{As a  \lq still\rq{ }expression}
\begin{itemize}
	\item  \citeauthor{Abe2015} (\citeyear{Abe2015}, \citeyear{Abe2016}).
	\item Form: also in the variant \textit{sí}-.
	\item Specialisation: the description by \citeauthor{Abe2015} meets my definition.
	\item Polarity sensitivity: does not appear to combine with negation.
	\item Pragmaticity: no conclusions possible, based on the available data.
	\item Further notes: \textit{syá}- as an exponent of \textsc{still} frequently occurs with copula \textit{lɪ} plus an inflected predicate (\ref{exAppendixBende2}). With inchoative lexical verbs, \textit{syá}- is compatible with the perfective aspect inflection, yielding a persistent state (\ref{exAppendixBende3}).
\end{itemize}

\begin{exe}
	\ex\label{exAppendixBende1}
	\gll Li-\textbf{syá}-lɪ́=ko\\
	\textsc{subj}.\textsc{ncl}5-still-\textsc{cop}=\textsc{ncl}17(\textsc{loc})\\
	\glt \lq (The sun) is still there. (a greeting in daytime)\rq{ }\parencite[31]{Abe2015}
	
	\ex\label{exAppendixBende2}
	\gll Tu-\textbf{syá}-lɪ́ tu-lɪ́kú-lyǎ.\\
	\textsc{subj}.1\textsc{pl}-still-\textsc{cop} \textsc{subj}.1\textsc{pl}-\textsc{prs}-eat\\
	\glt \lq We are still eating.' \parencite[29]{Abe2015}

	\ex\label{exAppendixBende3}
	\gll Tu-\textbf{syá}-nyúnk-ílé\\
	\textsc{subj}.1\textsc{pl}-still-stink-\textsc{pfv}\\
	\glt \lq We still stink.' \parencite{Abe2016}
\end{exe}

\subsubsection{Usages relating toother phasal polarity concepts}
\paragraph{Not yet}
\label{appendixBendeNotYet}
\begin{itemize}
	\item  \citeauthor{Abe2015} (\citeyear{Abe2015}, \citeyear{Abe2016}).
	\item This function obtains in combination with copula \textit{lɪ}, either together with an infinitival complement (\ref{exAppendixBendeNotYet1}, \ref{exAppendixBendeNotYet2}), or no complement at all. The latter cases include negative one-word responses to polar questions, as in the second token in (\ref{exAppendixBendeNotYet3}) as well as polar questions biased towards a negative response; see the first token in (\ref{exAppendixBendeNotYet3}).
\end{itemize}

\begin{exe}
	\ex\label{exAppendixBendeNotYet1}
	\gll Tu-\textbf{syá}-lɪ́ kú-ɣúla.\\
	\textsc{subj}.1\textsc{pl}-still-\textsc{cop} \textsc{ncl15}(\textsc{inf})-buy\\
	\glt \lq We haven't bought yet.' \parencite[26]{Abe2015}

	\ex\label{exAppendixBendeNotYet2}
	\gll Tu-\textbf{syá}-lɪ́ kú-kolá mú-límo.\\
	\textsc{subj}.1\textsc{pl}-still-\textsc{cop} \textsc{ncl15}(\textsc{inf})-do \textsc{ncl}3-work\\
	\glt \lq We haven't worked yet.' \parencite[26]{Abe2015}

	\ex\label{exAppendixBendeNotYet3}
	\gll U-\textbf{syá}-lɪ́? – N-\textbf{syá}-lɪ́.\\
	\textsc{subj}.2\textsc{sg}-still-\textsc{cop} {} \textsc{subj}.1\textsc{sg}-still-\textsc{cop}\\
	\glt \lq Haven't you (cooked) yet? -- Not yet.' \parencite[26]{Abe2015}
\end{exe}

\subsubsection{Broadly adverbial temporal-aspectual uses}

\paragraph{First}\label{appendixBendeFirst}
\begin{itemize}
	\item  \citeauthor{Abe2015} (\citeyear{Abe2015}, \citeyear{Abe2016}).
	\item  This function obtains only in the (formally unmarked) imperfective present. It is noteworthy, however, that it has a directive force.
	\item \textcite{Abe2016} points out that this function is compatible with all actional classes, including inchoative verbs (which would require the perfective aspect inflection to be compatible with \textsc{still}).
\end{itemize}

\begin{exe}
	\ex \gll Tu-\textbf{sí}-kola / tu-\textbf{syá}-kola mú-límó\\
	\textsc{subj}.1\textsc{pl}-still-do {} \textsc{subj}.1\textsc{pl}-still-do \textsc{ncl}3-work\\
	\glt i.\phantom{i} \lq We are still working'\\
	ii. \lq Let us work first.' \parencite[29]{Abe2015}
	
	\ex \gll Tu-\textbf{syá}-tehǎ.\\
	\textsc{subj}.1\textsc{pl}-still-love\\
	\glt \lq Let us love first.\rq{ }\parencite{Abe2016}
\end{exe}

\paragraph{Near past}
\label{appendixBendeNearPast}
\begin{itemize}
	\item  \citeauthor{Abe2015} (\citeyear{Abe2015}, \citeyear{Abe2016}).
	\item Form: also in the variant \textit{sí}-. This use is restricted to the perfective aspect\slash anterior inflection.
\end{itemize}
\begin{exe}
	\ex 
	\gll Tu-\textbf{syá}-kos-ílé / tu-\textbf{sí}-kos-ílé mú-límo.\\
	\textsc{subj}.1\textsc{pl}-still-do-\textsc{pfv} {} \textsc{subj}.1\textsc{pl}-still-do-\textsc{pfv} \textsc{ncl}3-work\\
	\glt \lq We have finished working just now.’ \parencite[25]{Abe2015}

	\ex
	\gll Tu-\textbf{syá}-teék-ílé\\
	\textsc{subj}.1\textsc{pl}-still-cook-\textsc{pfv}\\
	\glt \lq We have just cooked\rq{ }\parencite{Abe2016}
	
	\ex
	\gll Tu-\textbf{syá}-lí tu-nyaágh-ílé\\
	\textsc{subj}.1\textsc{pl}-still-\textsc{cop} \textsc{subj}.1\textsc{pl}-bathe-\textsc{pfv}\\
	\glt \lq We have just bathed.\rq{ }\parencite{Abe2016}
\end{exe}
\il{Bende|)}

\section{Chuwabu (chw, chuw1238)}\il{Chuwabu|(} 
\label{appendixChuwabo}
\subsection{Introductory remarks}
Chuwabu has a fairly typical Narrow Bantu noun class system. I have glossed the individual classes as \textsc{ncl} \lq noun class' plus Arabic numeral, following the common Bleek-Meinhof system.

\subsection{=vi}
\subsubsection{General information}
\begin{itemize}
	\item Wordhood: bound morpheme (enclitic).
	\item Syntax: attaches to its focus; as a \textsc{still} expression, the host is the predicate.
	\item Etymology: unknown, but similar to \textit{viina} \lq also', which is involved in the expression of \textsc{no longer}.
\end{itemize}


\subsubsection{As a  \lq still\rq{ }expression}\largerpage[2]
\begin{itemize}
	\item \citeauthor{Guerois2015} (\citeyear[310–311]{Guerois2015}, \citeyear{Guerois2021}).
	\item Specialisation: the discussion by \textcite{Guerois2021} gives a decent indication that this marker conforms to my definition. Further, albeit indirect, evidence comes from the robustly attested \textsc{still}–restrictive polysemy (\Cref{sectionExclusive}).
	\item Pragmaticity: compatible with both scenarios.
	\item Polarity sensitivity: inner negation yields \textsc{not yet}.
\end{itemize}
\begin{exe}
	\ex
	Context: Ddoolrinddo has been killed by her sister. She has turned into a singing flower. People hear her singing. Some time later:\\
	\gll Ddóólríndd' óó-now-ííbá=\textbf{vi}.\\
	D. \textsc{subj}.\textsc{ncl}1-\textsc{ipfv}-sing=still\\
	\glt \lq Ddoolrinddo is still singing.' \parencite[588]{Guerois2015}
	
	\ex
	\gll Vátî vaa-rîbá {(na vánó)} mu-nó-lába=\textbf{vi}?\\
	\textsc{ncl}16.ground \textsc{subj}.\textsc{ncl}16:\textsc{pfv}-be\_dark \phantom{(}{until now} subj.2\textsc{sg}-\textsc{ipfv}-work=still\\
	\glt \lq The night has come and you are still working?\rq{ }\parencite[168]{Guerois2021}
\end{exe}

\subsubsection{Broadly adverbial temporal-aspectual uses}
\paragraph{Always, all the time}
\label{appendixChuwabuAlways}
\begin{itemize}
	\item \textcite{Guerois2021}.
	\item =\textit{vi} is attested with a range of \lq always, all the time, constantly\rq{ }uses.
\end{itemize}

\begin{exe}
	\ex
	\gll Mu-kwél-e bábááni dd-a-gon-ágá ddi-ǹ-kál' óó-mu-rohá=\textbf{vi}.\\
	\textsc{ncl}18(\textsc{loc})-die-\textsc{appl}-\textsc{pfv}.\textsc{rel} my\_father(\textsc{ncl}1) \textsc{subj}.\textsc{ncl}1-\textsc{subord}-sleep-\textsc{hab} \textsc{subj}.\textsc{ncl}1-\textsc{ipfv}-\textsc{cop} \textsc{ncl}15(\textsc{inf})-\textsc{obj}.\textsc{ncl}1-dream=still\\
	\glt \lq Since my father died, while sleeping, I keep dreaming of him.\rq{ }\parencite[218]{Guerois2015}

	\ex 
	Context: A certain man’s wife constantly falls sick.\\
	\gll eǹtãwu vēlévalʼ aápale m-oon-él-iiyé wíî íyééne ó-n-óbúléla vaddíddí ka-á-vódh' oottidda mabas' aápále va-tákúlú sabw’ eelá w-aá-kála mu-reddá=\textbf{ví} ábále, ábáálé éenâ á-á-ni-mu-nyapwaaríya.\\
	then \textsc{dist}.\textsc{ncl}16(\textsc{loc}) \textsc{dist}.\textsc{ncl}16(\textsc{loc}) \textsc{subj}.\textsc{ncl}18(\textsc{loc})-see-\textsc{appl}-\textsc{pfv}.\textsc{rel}:3\textsc{sg} \textsc{comp} 3\textsc{sg} \textsc{subj}.\textsc{ncl}1-\textsc{ipfv}-get\_sick much \textsc{neg}.\textsc{subj}.\textsc{ncl}1-\textsc{pst}.\textsc{ipfv}-be\_able \textsc{ncl}15(\textsc{inf}).hold.\textsc{pl} \textsc{ncl}6.work \textsc{dist}.\textsc{ncl}6 \textsc{ncl}16(\textsc{loc})-house(\textsc{ncl}9) because \textsc{comp} \textsc{subj}.\textsc{ncl}1-\textsc{pst}.\textsc{ipfv}-\textsc{cop} \textsc{ncl}1-sick.\textsc{pl}=still \textsc{dist}.\textsc{ncl}2 \textsc{ncl}2.sister \textsc{ncl}2.other \textsc{subj}.\textsc{ncl}2-\textsc{pst}-\textsc{ipfv}-\textsc{obj}.\textsc{ncl}1-despise\\
	\glt \lq Then, when she saw that she was always getting ill, (that) she was not able to clean the house she was always sick, those ones, the other sisters, despised her.' \parencite[608]{Guerois2015}
\end{exe}
\pagebreak

\paragraph{Distant future}
\label{appendixChuwabuDistantFuture}
\begin{itemize}
	\item \textcite{Guerois2021}.
	\item According to \textcite{Guerois2021} the use of \mbox{=\textit{vi}} in the future tense can suggest a later occurrence than the use of the \lq\lq bare" future. Thus, if  \mbox{=\textit{vi}}  were to be removed in (\ref{exAppendixChuwaboFuture}), time frame adverbials such as \lq today' or \lq tomorrow' would be more suitable than  \textit{sumaán' ééjw' één̩dhawo} \lq next week'
\end{itemize}

\begin{exe}
	\ex\label{exAppendixChuwaboFuture}
	\gll Ddi-neel-óó-gulá=\textbf{vi} má-fúgi sumaán' ééjw' éé-n̩-d-a=wo\\
	\textsc{subj}.1\textsc{sg}-\textsc{fut}.\textsc{aux}-\textsc{ncl}15(\textsc{inf})-buy \textsc{ncl}6-banana \textsc{ncl}9.week \textsc{dem}.\textsc{ncl}7/9 \textsc{subj}.\textsc{ncl}7/9-\textsc{ipfv}-come-\textsc{rel}=\textsc{ncl}16(\textsc{loc})\\
	\glt \lq I will buy bananas next week.' \parencite[191]{Guerois2021}
\end{exe}

\subsubsection{Restrictive (non-temporal)}
\paragraph{(Non-scalar) exclusive}
\label{appendixChuwabuRestrictive}
\begin{itemize}
	\item \citeauthor{Guerois2015} (\citeyear[310–312]{Guerois2015}, \citeyear{Guerois2021}).
	\item In this function, \mbox{=\textit{vi}} can take verbs (\ref{exAppendixchuwaboRestrictive1}), nominals (\ref{exAppendixchuwaboRestrictive2}), and adverbials (\ref{exAppendixchuwaboRestrictive1}) as its host.
\end{itemize}
\begin{exe}
	\ex
	\label{exAppendixchuwaboRestrictive1}
	\gll Mw-aap-él-íiyé, kurúmáanj’ oó-n-ósógóra ó-n-ódhówá=\textbf{vi}, íyééné oó-n-ó-ḿ-fwará=\textbf{vi}\\
	\textsc{subj}.\textsc{ncl}18(\textsc{loc})-\textsc{appl}-\textsc{pfv}.\textsc{rel}:3\textsc{sg} bee.sp(\textsc{ncl}1) \textsc{subj}.\textsc{ncl}1-\textsc{ipfv}-go\_on \textsc{subj}.\textsc{ncl}1-\textsc{ipfv}-go=still 3\textsc{sg} \textsc{subj}.\textsc{ncl}1-\textsc{ipfv}-\textsc{obj}.\textsc{ncl}1-follow=still\\
	\glt \lq Now she plucked it, the bee.sp is going on, he going, she following him.' \parencite[311]{Guerois2015}

	\ex
	\label{exAppendixchuwaboRestrictive2}
	\gll Nikúrábedha o-fíy' óó-fíy-iléé=ye o-m̩-fwany-ilé baáŕku=\textbf{vi}.\\
	Dugong \textsc{ncl}15(\textsc{inf})-arrive \textsc{ncl}15(\textsc{inf})-arrive-\textsc{pfv}.\textsc{rel}=3\textsc{sg} \textsc{subj}.\textsc{ncl}1-\textsc{obj}.\textsc{ncl}1-meet-\textsc{pfv} boat(\textsc{ncl}1)=still\\
	\glt Mr. Dugong, hardly had he arrived, found only the boat.\rq{ }\parencite[167]{Guerois2021}
	
	\ex
	\label{exAppendixchuwaboRestrictive3}
	\gll Ddi-ḿ̩-fúná ddi-tagîh-é vañgónó=\textbf{vi} ésíle dhi-íw-il=îîmi.\\
	\textsc{subj}.1\textsc{sg}-\textsc{ipfv}-want \textsc{subj}.1\textsc{sg}-repeat-\textsc{sbjv} a\_little=still \textsc{dist}.\textsc{ncl}8/10 \textsc{subj}.\textsc{ncl}8/10-hear-\textsc{pfv}.\textsc{rel}=1\textsc{sg}\\
	\glt \lq I want to talk just a little bit about what I heard.\rq{ }\parencite[167–168]{Guerois2021}
\end{exe}

\subsubsection{Broadly modal and interactional functions}
\paragraph{Concessive apodoses}
\label{appendixChuwabuConcessiveConsequent}
\begin{itemize}
	\item \textcite{Guerois2021}.
	\item This function only obtains in the future tense. There is only one example in the available data, which involves the apodosis of an alternative concessive conditional. The concessive reading might go back to \mbox{=\textit{vi}} as \textsc{still}, but could also be motivated by its restrictive function (\appref{appendixChuwabuRestrictive}), i.e. by emphasising the identity of outcome in both possible scenarios.
\end{itemize}

\begin{exe}
	\ex
	\gll Oo-kálá vénévá ku-kál-lé=vo íyéén' ó-neel' óó-kwa=\textbf{vi}.\\
	\textsc{subj}.2\textsc{sg}.\textsc{pfv}-\textsc{cop} \textsc{dem}.\textsc{ncl16}(\textsc{loc}) \textsc{neg}.\textsc{subj}.2\textsc{sg}-\textsc{cop}-\textsc{pfv}=\textsc{ncl}16(\textsc{loc}) 2\textsc{sg} \textsc{subj}.\textsc{ncl}1-\textsc{fut}.\textsc{aux} \textsc{ncl}15(\textsc{inf})-die=still\\
	\glt \lq Whether or not you are here (lit. you are here, you are not here), he is still going to die.' \parencite[190]{Guerois2021}
\end{exe}
\il{Chuwabu|)}

\section{Ewe (ewe, ewee1241)}\il{Ewe|(}
\subsection{ga-}
\subsubsection{General information}
\begin{itemize}
	\item Wordhood: bound morpheme (prefix).
	\item Etymology <\textit{gbɔ}\sim\textit{gba} \lq come back, return'.
\end{itemize}


\subsubsection{As a  \lq still\rq{ }expression}\label{appendixEweStill}
\begin{itemize}
	\item \textcite{Ameka2008}, \textcite[468]{Rongier2015} and \textcite[152]{Westermann1905}; further discussion throughout \textcite{vanBaar1997}.
		\item Specialisation: identified as a \textsc{still} expression by \textcite{vanBaar1997}
\item Polarity sensitivity: outer negation yields \textsc{not yet}.
\item Pragmaticity and further note: \textit{ga}- as \textsc{still} often co-occurs with \textit{ko} \lq just' (\ref{exAppendixEwe1}). According to \textcite[76]{vanBaar1997} \textit{ko}, or its reduplicate \textit{kokooko} (\ref{exAppendixEwe3}, \ref{exAppendixEwe4}), is compulsory in the unexpectedly late scenario.
\item Further note: all examples of \mbox{\textit{ga}-} as \textsc{still} in the literature involve a copula.
\end{itemize}
\begin{exe}
	\ex\label{exAppendixEwe1}
	\gll É-\textbf{ga}-le aha no-m ko.\\
	3\textsc{sg}-still-\textsc{loc}.\textsc{cop} alcohol drink-\textsc{prog} only\\
	\glt \lq He is still drinking alcohol.' (\cite{Ameka2018}, cited in \cite[8]{Kramer2021b})

	\ex\label{exAppendixEwe2}
	\gll Peter \textbf{ga}-le London.\\
	P. still-\textsc{loc}.\textsc{cop} L.\\
	\glt \lq Peter is still in London.' \parencite[24]{vanBaar1997}

	\ex\label{exAppendixEwe3}
	\gll Égbe lá, é-fé ŋ́kɔ́ \textbf{ga}-li kokoo\sim{}ko.\\
	today \textsc{def} 3\textsc{sg}-\textsc{poss} name still-\textsc{loc}.\textsc{cop}.3\textsc{sg} \textsc{redupl}\sim{}just\\
	\glt \lq Today, his name still exists.' \parencite[142]{Ameka2008}

	\ex\label{exAppendixEwe4}
	\gll Fo, è-\textbf{ga}-le dɔme-dzo-e dó-ḿ ɖé ŋú-nye kokoo\sim{}ko a?\\
	elder\_brother 2\textsc{sg}-still-\textsc{loc}.\textsc{cop} stomach-fire-\textsc{dim} wear-\textsc{prog} at side-1\textsc{sg} \textsc{redupl}\sim{}just \textsc{q}\\
	\glt \lq Dear, are you still annoyed with me?' \parencite[582]{Ameka1991}
\end{exe}

\subsubsection{Broadly adverbial temporal-aspectual uses}
\paragraph{Iterative and restitutive}
\label{appendixEweIterative}
\begin{itemize}
	\item \citeauthor{Ameka1991} (\citeyear[49]{Ameka1991}, \citeyear{Ameka2008}), \textcite[468]{Rongier2015} and \citeauthor{Westermann1905} (\citeyear[152]{Westermann1905}, \citeyear[73]{Westermann1907}).
	\item Both an iterative (\ref{appendixEweIterative1}-\ref{appendixEweIterative3}) and a restitutive (\ref{appendixEweRestitutive1}, \ref{appendixEweRestitutive2}) use are available.
	\item This function can optionally be reinforced through \textit{áké} \lq again', as in (\ref{appendixEweIterative1}).
\end{itemize}
\begin{exe}
	\ex\label{appendixEweIterative1}
	\gll Me-\textbf{ga}-vá yi (áké).\\
	\textsc{sg}-still-come go (again)\\
	\glt \lq I have passed again.' \parencite[142]{Ameka2008}
	
	\ex\label{appendixEweIterative2}
	\gll Kofi \textbf{ga}-le avi fa-ḿ.\\
	K. still-\textsc{loc}.\textsc{cop} cry cry-\textsc{prog}\\
	\glt \lq Kofi is crying again.' \parencite[49]{Ameka1991}	
	
	\ex\label{appendixEweIterative3}
	\gll Dzilá-wó á-\textbf{ga}-ɸo nu ná wó vi lá kpɔ́.\\
	parent-\textsc{pl}	\textsc{irr}-still-strike mouth to 3\textsc{pl} child \textsc{def} \textsc{pfv}\\
	\glt \lq The parents will (try to) speak again to their child.'\\ \parencite[146]{Ameka1991}
	
	\ex\label{appendixEweRestitutive1}
	\gll Ékemá súbɔ́lá-wó \textbf{ga}-kɔ́-nɛ yi-a núɖuxɔme\\
	then servant-\textsc{pl} still-carry-\textsc{hab}.3\textsc{sg} go-\textsc{hab} dining\_room\\
	\glt \lq Then the servants carry him back to the dining room.\rq{ }\parencite[142]{Ameka2008}

	\ex\label{appendixEweRestitutive2}
	\gll Hé ɗe nè-\textbf{ga}-trɔ́ gbɔ-na loo?\\
	\textsc{interj} \textsc{q} 2\textsc{sg}-still-turn come-\textsc{hab} \textsc{q}\\
	\glt \lq Hey! Are you coming back or what?' \parencite[483]{Ameka1991}
\end{exe}

\subsubsection{Additive and related functions}
\paragraph{Additive}\label{appendixEweAdditive}
\begin{itemize}
	\item \textcite[468]{Rongier2015} and \textcite[153]{Westermann1905}.
	\item As with many expressions in my sample, several examples involve other additive markers (\ref{appendixEweAdditive2},\ref{appendixEweAdditive3})
\end{itemize}
\begin{exe}
	\ex
	\gll É-\textbf{ga}-fiá abé dɔsrɔ́ví la xɔxɔ ɖé aɸetɔ́-wó ɸé hame ené\\
	3\textsc{sg}-still-show as\_if apprentice \textsc{def} receive into master-\textsc{pl} \textsc{poss} group as\\
	\glt \lq It also shows as if the apprentice has been accepted into the group of masters.' \parencite[63]{Ameka1991}

	\ex\label{appendixEweAdditive2}
	\gll Wó-\textbf{ga}-ɖe kúkú ná máwú bé wò-a ná bé śemanú ná-xɔ ɗaseɗi-gbalē lá hã.\\
	3\textsc{pl}-still-take\_off hat to  God \textsc{comp} 3\textsc{sg}-\textsc{irr} cause \textsc{comp} S. \textsc{sbjv}-get certificate \textsc{def} too\\
	\glt \lq They also begged God to grant that Semanu should receive a certificate.' \parencite[612]{Ameka1991}

	\ex\label{appendixEweAdditive3}
	\gll É-nye hatí-nyè, \textbf{ga}-nyé xɔ̃́-nyè hã̂.\\
3\textsc{sg}-\textsc{cop} colleague-1\textsc{sg} still-\textsc{cop} friend-1\textsc{sg} also\\
	\glt \lq Il est mon camarade mais aussi mon ami. [He's my colleague, and also my friend.]' (\cite[153]{Westermann1905}, glosses added)
\end{exe}

\subsubsection{Broadly modal and interactional functions}
\paragraph{Prohibitives}\label{appendixEweProhibitives}
\begin{itemize}
	\item \textcite[53]{Ameka1991} and \textcite[67]{Westermann1907}.
	\item \textit{Ga}- is a requisite part of prohibities.
	\item As both \citeauthor{Ameka1991} and \cite{Westermann1907} point out, this is motivated by \mbox{\textit{ga}-} in its iterative function (\appref{appendixEweIterative}) and the marking of \textsc{no longer}, which builds on the latter.
\end{itemize}
\begin{exe}
	\ex 
	\gll Me-\textbf{ga}-dzró nú \textit{o}.\\
	\textsc{neg}.2\textsc{sg}-still-desire thing \textsc{neg}\\
	\glt \lq Do not crave for things.\rq{ }\parencite[358]{Ameka1991}
	
	\ex 
	\gll Me-\textbf{ga}-yi o!\\
	\textsc{neg}.2\textsc{sg}-go \textsc{neg}\\
	\glt \lq{}N'y va pas! [Don't go!]\rq{ }(\cite[468]{Rongier2015}, glosses added)
\end{exe}
\il{Ewe|)}

\section{Kaba (sbz, sara1319)}\label{appendixKaba}
\il{Kaba|(}
\subsection{bbáy}
\subsubsection{General information}
\begin{itemize}
	\item Form: also transcribed as \textit{ɓə́y}.
	\item Wordhood: independent grammatical word, invariable.
	\item Syntax: fixed, clause-final position.
\end{itemize}


\subsubsection{As a  \lq still\rq{ }expression}
\begin{itemize}
	\item \textcite[41]{Keegan2014}, \textcite[423–425]{Moser2004} and \textcite[10]{MoserDingatoloum2007}.
	\item Specialisation: examples (\ref{exAppendixKaba1}–\ref{exAppendixKaba4}) give a good indication that \textit{bbáy} conforms to my definition. For instance, in (\ref{exAppendixKaba1}) it not only triggers the presupposition that Enjamgotoje was at home before, but also introduces a dynamic perspective towards his return (i.e. him no longer being at home). Similarly, in (\ref{exAppendixKaba2}) \textit{bbáy} evokes an alternative scenario: if the drum were no longer with the corpse, there would be no reason to complain.	
	\item Pragmaticity: the available data allow no conclusions.
	\item Polarity sensitivity: inner negation yields \textsc{not yet}.
	\item Further note: ex. (\ref{exAppendixKaba4}) illustrates \textit{bbáy} under a verb of manipulation.
\end{itemize}
\begin{exe}
	\ex\label{exAppendixKaba1}
	Context: The opening lines of a narrative.\\
	\gll Esú dé kə̀ Enjàmgòtóje d-îya kùla, ngà kùla lè Enjàmgòtóje lé ùbba éngɔru. Ngà kàké Enjàmgòtóje nàyn bbè-é \textbf{bbáy} à Esúje ddèe ke ngɔru ɔ̀ru éngɔrú lé kúla lè Enjàmgòtóje lé ùlà kùlà té là-á.\\
E. 3\textsc{pl} \textsc{com} E. \textsc{subj}.3\textsc{pl}-hide trap then trap of E. that catch pigeon then when E. stay village-\textsc{loc} still then E. come \textsc{adv} quickly take\_away pigeon that trap of E. that place trap \textsc{loc} of-3\textsc{sg}\\
\glt \lq Esu and Enjamgotoje made traps and Enjamgotoje's trap caught a pigeon and when he [Enjamgotoje's] was still at home, Esu came quickly and removed the bird from Enjamgotoje's trap and put it into his trap.' (Ejamgoteje then returns to the traps, and the two start quarreling about the bird) \parencite[435]{Moser2004}

	\ex\label{exAppendixKaba2}
	Context: The protagonist has borrowed a group of girls his drum so that they could use it in a funeral ceremony. When he comes to take it back, he finds the skin of the drum perforated. He complains of his hardship.\\
	\gll Dàlé-m nàyn tà ndúm yò té \textbf{bbáy}.\\
	drum-\textsc{poss}.1\textsc{sg} stay mouth rotten corpse \textsc{loc} still\\
	\glt \lq My drum is still in the mouth of the rotten corpse.' \parencite[446]{Moser2004}
	
	\ex\label{exAppendixKaba3}
	\gll Bbó náyn tò pòr-ó \textbf{bbáy} à n-únga wa bìr-í.\\
	if sauce \textsc{cop} fire-\textsc{loc} still then \textsc{subj}.3\textsc{sg}-pour millet mortar-\textsc{loc}\\
	\glt \lq If the sauce is still on the fire, she pours the millet into the mortar.'  \parencite[296]{Moser2004}
	
	\ex\label{exAppendixKaba4}
	\gll  E-ndìkí {tà kàre} ǹ-tɔ́n tam \textbf{bbay}?\\
	2\textsc{sg}-want \textsc{comp} 1\textsc{pl}-chew talk still\\
	\glt \lq Do you want that we talk still?\rq{ }\parencite[381]{Moser2004}
\end{exe}

\subsubsection{Broadly adverbial temporal-aspectual uses}
\paragraph{Iterative and restitutive}
\label{appendixKabaIterative}
\begin{itemize}
	\item \textcite[426]{Moser2004}.
	\item Iterative readings are illustrated in (\ref{exAppendixKabaIterative1}, \ref{exAppendixKabaIterative2}). Examples (\ref{exAppendixKabaRestitutive1}, \ref{exAppendixKabaRestitutive2}), both of which describe the same situation, are clear cases of a restitutive reading. Example (\ref{exAppendixKabaRestitutive3}) also appears to be restitutive. Note that it includes a verb \textit{tél} \lq return', which might be (partially) responsible for this interpretation.
	\item The iterative use is also apparent in the reduplicated ideophone \textit{bbáybbáy} \lq repeated action\rq{ }\parencite[10, 314, 357]{MoserDingatoloum2007}.
\end{itemize}

\begin{exe}
	\ex\label{exAppendixKabaIterative1}
	\gll
	J-à kòo nàa \textbf{bbáy} á ma!\\
	\textsc{subj}.1\textsc{pl}-\textsc{fut} see \textsc{recp} still \textsc{fut} \textsc{dm}:wish\\
	\glt \lq We shall see each other again (I hope).\rq{ }(\cite[126]{MoserDingatoloum2007}, glosses added)

	\ex\label{exAppendixKabaIterative2}
	Context: On a previous occasion, Esu has sung a terrifying song to his child. His wife is hiding and spying on him.\\
	\exi{}\gll Esú ùnn kùtù tà kɔ̀sù pa là-á ke kára bàán lé \textbf{bbay}.\\
	E. seize buttock \textsc{purp} sing song \textsc{gen}-3\textsc{sg} \textsc{adv} one like \textsc{det} still\\
	\glt \lq Esu began again to to sing his unique song.' \parencite[450]{Moser2004}
	
	\ex\label{exAppendixKabaRestitutive1}
	Context: The story's protagonist has met a woman who was collecting salt and has given her a bird.\\
	\gll Dèné lé tàa yèel lé íla pòr-ó à àw tà mbɔ́n kàte \textbf{bbáy}.\\
	woman that take bird that throw fire-\textsc{loc} and go \textsc{purp} collect salt still\\
	\glt \lq The woman took Esuʼs bird and placed it on the fire and went to collect salt again.' (i.e. she leaves the scene to resume collecting salt) \parencite[436]{Moser2004}
	
	\ex\label{exAppendixKabaRestitutive2}
	Context: The bird that Esu has given the woman appears to have burned in the fire. The woman is speaking.\\
	\gll Bɔ̀i lé ay! Yèel lá m-il-ɛ́ pòr-ó à m-aw tà mbɔ́n kàte lèé-m \textbf{bbáy} à yèel lé ngàá nòto nèénn sé ma!\\
	father that \textsc{dm} bird \textsc{foc} \textsc{subj}.1\textsc{sg}-place-\textsc{obj}.3\textsc{sg} fire-\textsc{loc} and \textsc{subj}.1\textsc{sg}-go \textsc{purp} collect salt \textsc{gen}-1\textsc{sg} still and bird that now burn here see that\\
	\glt \lq My god. Look! The bird that I put into the fire and I went to gather my salt again, this bird is now completely reduced to charcoal, as you see.ʼ \parencite[437]{Moser2004}	
	
	\ex\label{exAppendixKabaRestitutive3}
	\gll Babe kàrə tà dodo tél in \textbf{bbáy}.\\
	\textsc{seq} \textsc{caus}	\textsc{purp} fight return rise still\\
	\glt \lq Before it will also cause the fight to start again.' \parencite[425]{Moser2004}	
\end{exe}

\paragraph{Prospective \lq eventually\rq}
\label{appendixKabaProspective}
\begin{itemize}
	\item In (\ref{exAppendixKabaPerfectiveFuture1}–\ref{exAppendixKabaPerfectiveFuture3}) \textit{bbáy} refers to eventualities that are yet to happen.
	\item While (\ref{exAppendixKabaPerfectiveFuture1}, \ref{exAppendixKabaPerfectiveFuture2}) both involve proverbs, ex. (\ref{exAppendixKabaPerfectiveFuture3}) and the available data on cognates of \textit{bbáy} in other Sara languages indicate that this use is more productive. See, for instance, the examples throughout \textcite{Keegan2014}, \textcite{Thayer1978} and \textcite{Vandame1963} on \textit{ɓə́y}\sim{}\textit{ɓí} in Kaba's closest relatives \ili{Ngambay} and Laka,\il{Laka} as well as \citeauthor{Palayer1989} (\citeyear[244]{Palayer1989}, \citeyear[167]{Palayer1992}) on \ili{Sar} \textit{báy}.
\end{itemize}

\begin{exe}
	\ex\label{exAppendixKabaPerfectiveFuture1}
	\gll M-á dda \textbf{bbáy} á étɛ̀n ùnda né káre té.\\
	\textsc{subj}.1\textsc{sg}-\textsc{fut} do still \textsc{foc} t.o.\_insect 	\textsc{subj}.2\textsc{sg}.place \textsc{anaph} stalk \textsc{loc}\\
	\glt {\lq\lq}\lq I will do it later" causes the insect to live on a millet stalk (proverb: to be a boaster is to create problems).\rq{ }\cite[50]{MoserDingatoloum2007}, glosses added)

	\ex\label{exAppendixKabaPerfectiveFuture2}
	\gll M-á dda \textbf{bbáy} á ùjà wàlà.\\
	\textsc{subj}.1\textsc{sg}-\textsc{fut} do still \textsc{foc} \textsc{subj}.2\textsc{sg}.harvest weed\\
\glt \lq He who says \lq\lq Iʼll do it" will harvest weeds (proverb: he who puts things off for tomorrow will find troubles on the way).\rq{ }(\cite[10]{MoserDingatoloum2007}, glosses added)

	\ex\label{exAppendixKabaPerfectiveFuture3}
	\gll Àkàá kə̀sə nàrɛ̀ làá nàyn tà kàrə ń-tél né d-ár-ɛ́ \textbf{bbay}.\\
	but leftover money \textsc{gen}.3\textsc{sg} stay \textsc{purp} \textsc{caus} \textsc{subj}.3\textsc{sg}-return \textsc{anaph} \textsc{subj}.3\textsc{pl}-\textsc{dat}-\textsc{obj}.3\textsc{sg} still\\
	\glt \lq But his change remained to be returned to him yet.\rq{ }\parencite[356]{Moser2004}
\end{exe}

\paragraph{Sequencing \lq and then, afterwards\rq{} (?)}\label{appendixKabaSequencing}
\begin{itemize}
	\item \textcite[310, 407]{MoserDingatoloum2007}.
	\item \textcite{MoserDingatoloum2007} list \textit{ensuite} \lq afterwards' and \textit{puis} \lq then\rq{ }as senses of this item.  This could refer to the prospective \lq eventually\rq{ }use of \textit{bbáy} (\Cref{sectionProspective}, \appref{appendixKabaProspective}), but may also be indicative of a sequencing function. While there are no clear-cut instances of the latter in the data consulted, such a use is undoubtedly found with the cognate cognate expression \textit{ɓə́y}\sim{}\textit{ɓí} in Kaba's very close relatives \ili{Ngambay} and \ili{Laka} (\ref{exKabaThen1}–\ref{exKabaThen3}). An example like (\ref{exKabaThen3}), featuring the future tense, could be read as both \lq and then\rq{ }or\lq eventually\rq{}.
\end{itemize}
\begin{exe}
	\ex \ili{Ngambay}\label{exKabaThen1}\\
	\gll K-únd-á sár\sim{}sár ɓá d-ḭ́ā̰-á \textbf{ɓə́y}.\\
	\textsc{subj}.3\textsc{pl}-beat-\textsc{obj}.3\textsc{sg} \textsc{redupl}\sim{}long\_time first \textsc{subj}.3\textsc{pl}-let-\textsc{obj}.3\textsc{sg} still\\
	\glt \lq Ils le frappaient pendant longtemps, puis ils l'ont  laissé. [They were hitting him for a long time, then they left him.]\rq{ }(\cite[259]{Keegan2014}, glosses added)
	
	\ex \ili{Laka}\label{exKabaThen2}\\
	\gll ɨ́-tɔ́l bāngàw ɓá d-{ɔ́ bɨ̄} \textbf{ɓí}.\\
	\textsc{subj}.\textsc{3pl}-peel sweet\_potato first \textsc{subj}.\textsc{3pl}-fry still\\
	\glt \lq On pèle la patate douce avand de la frire. [One peels the sweet potato first and then fries it.]\rq{ }(\cite[123]{Keegan2014}, glosses added)
	\pagebreak
	\ex \ili{Laka}\label{exKabaThen3}\\
	\gll Iyā̰-m̄ ā-m̄ m-ə́jī njḛ́ ɓá m-á pà kàr̄-ī \textbf{ɓí}.\\
	let-\textsc{obj}.1\textsc{sg} give-\textsc{obj}.1\textsc{sg} \textsc{subj}.1\textsc{sg}-think little first \textsc{subj}.1\textsc{sg}-\textsc{fut} say give-\textsc{obj}.1\textsc{pl} still\\
	\glt \lq Laisse-moi réfléchir un peu dʼabord et après je te dirai. [Let me think a bit first and then Iʼll tell you.]\rq{ }(\cite[142]{Keegan2014}, glosses added)
\end{exe}

\subsubsection{Additive and related functions}
\paragraph{Additive}\label{appendixKabaAdditive}
\begin{itemize}
	\item \textcite[425]{Moser2004}. 
	\item  \textcite[425]{Moser2004} lists \lq more' as one of the senses of \textit{bbáy}. The only clear-cut example of this is (\ref{exAppendixKabaIncrement1}), which is hard to delineate from an iterative reading (\lq give us water again, please'). However, additive uses are clearly attested for the cognate expression \textit{ɓə́y}\sim{}\textit{ɓí} in Kaba's very close relative \ili{Ngambay} (\ref{exAppendixKabaIncrement2}).
\end{itemize}

\begin{exe}
	\ex\label{exAppendixKabaIncrement1}
	\gll Ar-i-je-jé màann \textbf{bbáy}.\\
	\textsc{subj}.2.give-\textsc{mod}-\textsc{pl}-\textsc{obj}.1\textsc{pl} water still\\
	\glt \lq Give us some more water please.\rq{ }(\cite[60]{MoserDingatoloum2007}, glosses added)
	
	\ex \ili{Ngambay} \label{exAppendixKabaIncrement2}\\
	\gll Mbáiuàlá á kào sə-ḿ – Mā \textbf{ɓéi.}\\
	M. \textsc{fut}.3 go with-1\textsc{sg} {} 1\textsc{sg} still\\
	\glt \lq Mbaywala ira avec moi. -- Moi aussi! [Mbaywala will go with you -- Me, too!]\rq{ }(\cite[119]{Vandame1963}, glosses added)
\end{exe}
\il{Kaba|)}

\section{Manda (mgs, mand1423)}\label{appendixManda}\il{Manda|(}
\subsection{Introductory remarks}
I am indebted to Rasmus Bernander for discussing Manda data with me, and for eliciting additional examples. Note that Manda has a typical Narrow Bantu noun class system. Noun class prefixes in the examples are glossed as \textsc{ncl} for \lq noun class', together with an Arabic numeral that follows the common Bleek-Meinhof system of referring to Bantu noun classes.

\subsection{(a)kona}
\subsubsection{General information}
\begin{itemize}
	\item Form: there is variation (partially morphophonemic, partially free) between forms with and without the initial vowel segment.
	\item Wordhood: intermediate, an auxiliary-like element that requires subject marking.
\end{itemize}
	

\subsubsection{As a  \lq still\rq{ }expression}
\begin{itemize}
	\item \citeauthor{Bernander2017} (\citeyear[259–265]{Bernander2017}, \citeyear{Bernander2021})
	\item Specialisation: described in line with my definition; see especially \textcite[51]{Bernander2021}.
	\item Pragmaticity: compatible with both scenarios.
	\item Polarity sensitivity: inner negation yields \textsc{not yet}, but this is uncommon, the normal choice being \mbox{(\textit{a})\textit{kona}} plus infinitive (\appref{appendixMandaNotYet}). \textsc{no longer} is expressed via negation plus \textit{kávɪ́lɪ} \lq again'. 
	\item Further note: compatible with the perfective aspect in a stative reading (\ref{exAppendixManda2}).
\end{itemize}

\begin{exe}
	\ex
	\gll N-\textbf{ákóna} ku-Songéa, kóma ni-páng-í' ku-séléla lu-kúmbi lu-chókópi.\\
	\textsc{subj}.1\textsc{sg}-still \textsc{ncl}17(\textsc{loc})-S. but \textsc{subj}.1\textsc{sg}-plan-\textsc{pfv} \textsc{ncl}15(\textsc{inf})-descend \textsc{ncl}11-time \textsc{ncl}11-soon\\
	\glt \lq I’m still in Songea, but I have planned to descend (to the lake litoral) in soon time.' \parencite[261]{Bernander2017}

	\ex\label{exAppendixManda2}
	Context:  My God, I have waited for Michael for half an hour now.\\
	\gll Á-\textbf{kóna} á-v-íli ku-ndɪndɪ́ma.\\
	\textsc{subj}.\textsc{ncl}1-still \textsc{subj}.\textsc{ncl}1-be(come)-\textsc{pfv} \textsc{ncl}17(\textsc{loc})-\textsc{ncl}9.toilet\\
	\glt \lq He is still (sitting) on the toilet?!\rq{ }\parencite[54]{Bernander2021}
	
	\ex\label{exAppendixManda3}
	\gll W-\textbf{akona} w-i-henga lihengu? – Ena, n-\textbf{akona}.\\
	\textsc{subj}.2\textsc{sg}-still \textsc{subj}.2\textsc{sg}-\textsc{prs}-work \textsc{ncl5}.work {} yes \textsc{subj}.1\textsc{sg}-still\\
	\glt \lq Are you still working? -- Yes, I still am.' (Rasmus Bernander, p.c.)
\end{exe}

\subsubsection{Uses on the fringes of \lq still\rq{}}

\paragraph{Scalar contexts}\label{appendixMandaScalar}
\begin{itemize}
	\item \mbox{(\textit{A})\textit{kona}} is attested in scalar contexts, both of a decrease (\ref{appendixMandaScalar1}, \ref{appendixMandaScalar2}) and a limited increase (\ref{appendixMandaScalar3}). Note how (\ref{appendixMandaScalar3}) features no overt \lq only\rq{ }marker.
	\item Also see \appref{appendixMandaSoon} for a use that is clearly based on a scalar decrease.
\end{itemize}

\begin{exe}

	\ex\label{appendixMandaScalar1}
	Context: We're at a bar. I suggest we get another round, but you've run out of money.\\
	\gll Kotokai ku-holalela n-\textbf{akona} nina sende/silingi elufu kumi ya-ni-ku-hemel-il-ayi.\\
\textsc{proh} \textsc{ncl}15(\textsc{inf})-think\_about \textsc{subj}.1\textsc{sg}-still \textsc{subj}.1\textsc{sg}-\textsc{com} money/shillings thousand ten \textsc{fut}-\textsc{subj}.1\textsc{sg}-\textsc{obj}.2\textsc{sg}-buy-\textsc{appl}-\textsc{fut}\\
	\glt Don't worry, I still got ten thousand shillings, I'll invite you.\rq{}
	\\(Rasmus Bernander, p.c.)
	
	\ex\label{appendixMandaScalar2}
	\gll Y-\textbf{akona} my-ese yi-debe ku-hika li-gono l-angu l-a ku-hogol-eka.\\
	\textsc{subj}.\textsc{ncl}4-still \textsc{ncl}4-month \textsc{ncl}4-few \textsc{ncl}15(\textsc{inf})-arrive \textsc{ncl}5-day \textsc{ncl}5-\textsc{poss}.1\textsc{sg} \textsc{ncl}5-\textsc{assoc} \textsc{ncl}15-give\_birth-\textsc{acaus}\\
	\glt \lq{}There's still a few months (left) until my birthday.\rq{}
	\\(Rasmus Bernander, p.c.)
	
	\ex Context: We're supposed to be sent a total of five books.\label{appendixMandaScalar3}
	\begin{xlist}
		\exi{A:} \textit{Nipatili ncheche.}\\
		\lq I've gotten four so far.\rq
		\exi{B:} \textit{Nene nani nivili nafu ncheche.}\\
		\lq Me too, I've got four.\rq{}
		\exi{C:}
		\gll Nene n-\textbf{akona} ni-v-ili na-fu fi-datu.\\
		1\textsc{sg} \textsc{subj}.1\textsc{sg}-still \textsc{subj}.1\textsc{sg}-be(come)-\textsc{pfv} with-\textsc{dem}.\textsc{ncl}8 \textsc{ncl}8-three\\
		\glt \lq Me, I've still only got three.\rq{ }(Rasmus Bernander, p.c.)
	\end{xlist}
\end{exe}

\paragraph{Scalar contexts, soon: (\textit{a})\textit{kona} \textsc{loc}-\textit{chokópi}}\label{appendixMandaSoon}
\begin{itemize}
	\item \textcite{Bernander2021}).
	\item This function occurs in collocation with \textit{chokópi} \lq little' marked for a locative class (usually \textsc{ncl} 16) plus the corresponding subject agreement.
	\item This is clearly an extension of a scalar use (i.e. \lq little [time] still [left] > \lq soon'); also cf. \ili{Swahili} \textit{bado kidogo} \lq still a bit, not quite yet [lit. still a little]'.
	\item Syntax: always follows the foregrounded verb, thus occupying (what in a single clause would be) the typical adverbial position. As \textcite{Bernander2021} points out, together with the restriction to locative marking, this can be taken as an indication for the construction's petrification and reanalysis as an adverbial. 
\end{itemize}

\begin{exe}
	\ex
	\gll Ya-u-sóv-i \textbf{p}-\textbf{ákóna} \textbf{pa}-\textbf{chokópi}.\\
	\textsc{fut}-\textsc{subj}.\textsc{ncl}14-be\_lost-\textsc{fut} \textsc{subj}.\textsc{ncl}16(\textsc{loc})-still \textsc{ncl}16(\textsc{loc})-little\\
	\glt \lq It will \textbf{soon} be lost (the flour) (lit. it will be lost, it is still a little\rq{}).\rq{ }\parencite[58]{Bernander2021}
\end{exe}


\subsubsection{Uses related toother phasal polarity-concepts}
\paragraph{Not yet}
\label{appendixMandaNotYet}
\begin{itemize}
	\item \citeauthor{Bernander2017} (\citeyear[262–265]{Bernander2017}, \citeyear{Bernander2021})
	\item This function occurs in the following contexts:
	\begin{itemize}
		\item With an infinitival complement, as in (\ref{exAppendixMandaNotYet1}, \ref{exAppendixMandaNotYet2}); this \textsc{not yet} construction can also mark a temporal clause of precedence (\lq not yet \textit{p}, \textit{q}\rq{} \equiv{ }\lq before \textit{p}, \textit{q}\rq). 
		\item In the absence of an overt predicate. This encompasses questions following an \lq already or still' > \lq already or not yet' pattern (\ref{exAppendixMandaNotYet3}) and negative one-word replies  (\ref{exAppendixMandaNotYet4}). Note that \mbox{(\textit{a})\textit{kona}} can also serve as an elliptical affirmative answer, as in (\ref{exAppendixManda3}) above. Closely related to disjunctive questions is the use as a pro-predicate in contexts suggesting a polarity contrast; see (\ref{exAppendixMandaNotYet5}).
	\end{itemize}	
\end{itemize}
\begin{exe}
	\ex\label{exAppendixMandaNotYet1}
	Context: My God, I have waited for Michael for half an hour now.\\
	\gll Á-\textbf{kóna} \textbf{ku}-píta ku-ndɪndɪ́ma.\\
	\textsc{subj}.\textsc{ncl}1-still \textsc{ncl}15(\textsc{inf})-leave \textsc{ncl}17(\textsc{loc})-\textsc{ncl}9.toilet\\
	\glt \lq Has he not yet left the toilet?!' \parencite[54]{Bernander2021}
	
	\ex\label{exAppendixMandaNotYet2}
	\gll N-\textbf{ákóna} \textbf{ku}-líma ngʼʊ́nda w-ángu.\\
	\textsc{subj}.1\textsc{sg}-still \textsc{ncl}15(\textsc{inf})-cultivate \textsc{ncl}3.plot \textsc{ncl}3-\textsc{poss}.1\textsc{sg}\\
	\glt \lq I have not cultivated my plot yet.' \parencite[45]{Bernander2021}
		
	\ex\label{exAppendixMandaNotYet3}
	\gll W-i-hótó‘ ku-gésa ku-wóna kítá' ngʼ-káti gú-p-ílí au w-\textbf{ákóna}.\\
	\textsc{subj}.2\textsc{sg}-\textsc{prs}-can \textsc{ncl}15(\textsc{inf})-try \textsc{ncl}15(\textsc{inf})-see \textsc{comp} \textsc{ncl}3-bread \textsc{subj}.\textsc{ncl}3-be\_baked-\textsc{pfv} or \textsc{subj}.\textsc{ncl3}-still\\
	\glt \lq You can try to see whether the bread has been baked or not yet.' \parencite[265]{Bernander2017}
		
	\ex\label{exAppendixMandaNotYet4}	
	Context: Can I take the phone?\\
	\gll Y-\textbf{ákóna}.\\
	\textsc{subj}.\textsc{ncl9}-still\\
	\glt \lq Not yet (it is still charging).' \parencite[265]{Bernander2017}
	
	\ex\label{exAppendixMandaNotYet5}
	\gll Kwa hɪnu u-meme u-hiki pa-Nkomang'ombi ni ku-Lwɪlu koma fi-jiji f-ɪngɪ fy-\textbf{akona}.\\
	for now \textsc{ncl}14-electricity \textsc{subj}.\textsc{ncl}14-arrive.\textsc{pfv} \textsc{ncl}16(\textsc{loc})-N. \textsc{com} \textsc{ncl}.17(\textsc{loc})-L. but \textsc{ncl}8-village \textsc{ncl}8-other \textsc{subj}.\textsc{ncl}8-still\\
	\glt \lq So far electricity has come to (the villages of) Nkomgangʼombi and Lwilu, but to other villages, not yet.ʼ (Rasmus Bernander, p.c.)
\end{exe}

\subsubsection{Marginality}\label{appendixMandaMarginal}
\begin{itemize}
	\item (\textit{A})\textit{kona} is compatible with a range of marginality readings.
\end{itemize}

\begin{exe}
	\ex\label{exAppendixMandaMarginal1}
	Context: About a swimming contest.\\
	\gll 
	Rehema n-\textbf{akona} ni-ku-m-pɪta, kangi Pili a-nya ma-kakala ku-pɪta nene.\\
	R. \textsc{subj}.1\textsc{sg}-still \textsc{subj}.1\textsc{sg}-\textsc{prs}-\textsc{obj}.\textsc{ncl}1-pass but P. \textsc{subj}.\textsc{ncl}1-have \textsc{ncl}6-strength \textsc{ncl}15(\textsc{inf})-pass 1\textsc{sg}\\
	\glt \lq Rehema I can still beat, but Pili is too strong for me.'
	\\(Rasmus Bernander, p.c.)
	
	\ex\label{exAppendixMandaMarginal2}
	\gll Ludeva y-\textbf{akona} ku-Pangwa, Nkomang'ombi yi-mali ku-vya ku-Manda.\\
	L. \textsc{subj}.\textsc{ncl}9-still \textsc{ncl}17(\textsc{loc})-P. N. \textsc{subj}.\textsc{ncl}9-already \textsc{ncl}15(\textsc{inf})-be(come) \textsc{ncl}17(\textsc{loc})-M.\\
	\glt \lq Ludeva [village] is still in Pangwa territory, Nkokmang’ombi is already Manda land.' (Rasmus Bernander, p.c.)
\end{exe}
\il{Manda|)}

\section{Mundang (mua, mund1325)}\il{Mundang|(}
\label{appendixMundang}

\subsection{ɓà}
\subsubsection{General information}
\begin{itemize}
	\item Wordhood: free morpheme.
	\item Form: can merge with interrogative \textit{nè} yielding \textit{ɓàā} (\appref{appendixMundangInterrogative}).
	\item Syntax: invariably in clause-final position.
\end{itemize}

\subsubsection{As a \lq{}still\rq{ }expression}
\begin{itemize}
	\item \textcite[332, 379–381, 457]{Elders2000}.
	\item Specialisation: \citeauthor{Elders2000}'s description meets my definition.
	\item Pragmaticity: According to \textcite[379]{Elders2000}, \textit{ɓà} invariably construes a situation as unexpectedly late (hence his label \lq\lq tardatif"). It is, however, not entirely clear whether this is supported by the data, or rather a theory-driven conclusion. \citeauthor{Elders2000} references \textcite{Schadeberg1990}, who considers counter-expectation to be an inherent component of \textsc{still}~-- see \cite{vanderAuwera2021} for a recent criticism. Most French translations of examples feature both \textit{encore} and \textit{toujours}. This fact and examples like (\ref{exAppendixMundang2}) suggest that \textit{bà} is compatible with the neutral scenario, as well.
	\item Polarity sensitivity: inner negation yields \textsc{not yet}.
	\item Further note: \textit{ɓà} is compatible with the anterior aspect when the latter is combined with the so-called \lq\lq convictif" \textit{ròò}, a type of modal marker with an intersubjective component (see \cite[434–436]{Elders2000}). This collocation gives a reading of a persistent result state, as in (\ref{exAppendixMundang3}).
\end{itemize}	
\begin{exe}
	\ex \label{exAppendixMundang1}
	\gll Bwàm tə̀-n \textbf{ɓà.}\\
	rain fall\_from\_sky-\textsc{ipfv} still\\
	\glt \lq Il pleut toujours/Il pleut encore. [It is still raining.]\rq{ }\parencite[380]{Elders2000}

	\ex \label{exAppendixMundang2}
	\gll Mò yɛ́l \textbf{ɓà}.\\
	\textsc{subj}.2\textsc{sg} child still\\
	\glt \lq Tu es encore enfant. [You are still a child.]' \parencite[380]{Elders2000}

	\ex \label{exAppendixMundang3}
	\gll Kàl ɓè ròò \textbf{ɓà}.\\
	leave \textsc{ant} \textsc{mod} still\\
	\glt \lq Il est parti pour l’instant./Il est toujours absent.  [He is still gone.] (Son retour immédiat proche est attendu. [His immediate return is expected.])' \parencite[380]{Elders2000}
\end{exe}

	
\subsubsection{Broadly adverbial temporal-aspectual uses}
\paragraph{Prospective \lq eventually\rq{}} \label{appendixMundangProspective}
\begin{itemize}
	\item \textcite[382]{Elders2000}.
	\item This function obtains with the optative and potential mood inflections of the verb, but apparently not with the dedicated future forms (which denote more certain intentions or predictions; see \cite[371–373]{Elders2000}). However, judging from \citeauthor{Elders2000}'s description, these collocations do not yield necessarily yield modal notions (e.g. of a persistent possibility).
\end{itemize}
\begin{exe}
	\ex \label{exAppendixMundangFuture1}
	\gll Mè kó-mō kə̀ pɪ́l \textbf{ɓà}.\\
	\textsc{subj}.1\textsc{sg} see-\textsc{obj}.2\textsc{sg}:\textsc{pot} \textsc{prep} ahead still\\
	\glt \lq Je te verrai plus tard! [I will see you later!]' \parencite[382]{Elders2000}

	\ex
	\gll ʔà fʊ̄ō \textbf{ɓà}.\\
	\textsc{subj}.3:\textsc{ipfv} think.\textsc{pot} still\\ 
	\glt \lq Il pensera un jour. [He will think one day.]' \parencite[382]{Elders2000}

	\ex 
	\gll Mō dán gɪ̀ \textbf{ɓà.}\\
	\textsc{subj}.2\textsc{sg}:\textsc{opt} enter.\textsc{opt} come.\textsc{ipfv} still\\
	\glt \lq Tu entreras plus tard... (on te dis d’attendre et après tu as le droit d’entrer)' [You will enter later (we tell you to wait and then you have the right to enter).]'  \parencite[382]{Elders2000}
\end{exe}

\paragraph{Sequencing \lq{}and then\rq}\label{appendixMundangEventSequencing}
\begin{itemize}
	\item \textcite[382–383]{Elders2000}.
	\item This function obtains in a bi-clausal pattern. The first clause, which may be a subordinate clause (\ref{exAppendixMundangaApodosis1}, \ref{exAppendixMundangaApodosis2}) or a coordinate clause (\ref{exAppendixMundangaApodosis3}), describes a prerequisite for another situation to obtain. The latter is depicted in the second clause, which features  \textit{ɓà}. The second clause (or a clause it governs) often additionally includes a form of \textit{tɪ̀ŋ} \lq start\rq{}, as in (\ref{exAppendixMundangaApodosis2}).
	\item All examples in \citeauthor{Elders2000}'s grammar feature non-actualised situations (future events or commands). This fact, and the notion of a pending event suggest a link to the prospective \lq eventually\rq{ }function (\appref{appendixMundangProspective}).
\end{itemize}\largerpage

\begin{exe}
	\ex\label{exAppendixMundangaApodosis1}
	\gll Sóó mó cḭ̄ḭ̄ ɓè, nà pʊ̀ò-n mə̀ngwáá \textbf{ɓà}.\\
millet \textsc{sit} germinate \textsc{ant}, 1\textsc{pl}.\textsc{incl} work-\textsc{ipfv} first\_plowing still\\
\glt \lq Quand le mil aura germé, nous ferons le premier labour. [Once the millet has germinated, we will do the first plowing.]\rq{ }\parencite[383]{Elders2000}

	\ex\label{exAppendixMundangaApodosis2}
	\gll Lɪ́l mō dɔ̀ŋ ɓè, ʔà gā ʔḭ̀ḛ̀-rā súl má nə́ nə̄nɪ̄, ká tɪ̀ŋ hɪ̀-n-yā \textbf{ɓà}.\\
	evening sit do \textsc{ant} \textsc{subj}.3.\textsc{ipfv} \textsc{fut} look.\textsc{ipfv}-\textsc{pl} termite\_mound only with eyes for start.\textsc{ipfv} clear-\textsc{nmlz}-\textsc{poss}.3\textsc{sg} still\\
	\glt \lq Quand il fera soir, ils iront voir la termitière en vie pour commencer à débroussailler les alentours. [When it is evening they will go to see the living termite mound to start clearing the surrounding area.]\rq{ }\parencite[383]{Elders2000}
	
	\ex\label{exAppendixMundangaApodosis3}
	\gll Mō tə́ráŋ tə̀kɪ́ɪ́ hʊ̰́ó̰ pɪ̄ɪ̄ kō mō tɪ̀ŋ zə̀-n-yā \textbf{ɓà}.\\
	\textsc{subj}.2\textsc{sg} dilute.\textsc{opt} gruel \textsc{dem} first \textsc{conj} \textsc{subj}.2\textsc{sg}.\textsc{opt} start.\textsc{opt} drink-\textsc{inf}-\textsc{poss}.3\textsc{sg} still\\
	\glt \lq Dilue d’abord cette bouillie, avant que tu commences à la boire.' [Dilute this gruel first, before you start drinking it.] \parencite[384]{Elders2000}
\end{exe}

\subsubsection{Broadly modal and interactional functions}
\paragraph{Interrogatives with \textit{nè}}\label{appendixMundangInterrogative}
\begin{itemize}
	\item \textcite[484–485]{Elders2000}.
	\item This function obtains in combination with interrogative \textit{nè}. The two markers can optionally merge into a portmanteau-morpheme \textit{ɓàā}, as in (\ref{exAppendixMundangInterrogative1}).
	\item This is a  fairly transparent use, an indirect way of signalling that the addressee should not continue with whatever they are doing.
	\item Only one example (\ref{exAppendixMundangInterrogative1}) is given in \citeauthor{Elders2000}'s grammar.
\end{itemize}
\begin{exe}
	\ex \label{exAppendixMundangInterrogative1}
	\gll Mò dɔ̀ŋ yɛ́ɓ \textbf{ɓàā}?\\
	\textsc{subj}.2\textsc{sg} do.\textsc{ipfv} work still.\textsc{q}\\
	\glt \lq Travailles-tu encore? [Are you still working?] (Le locuteur sait que l’interlocuteur est en train de travailler et il lui demande s’il continuera avec le travail. [The speaker knows that the addressee is working and asks whether he will continue.])' \parencite[485]{Elders2000}
\end{exe}
\il{Mundang|)}

\section{Nyangbo (nyb, nyan1302)}\label{appendixNyangbo}
\il{Nyangbo|(} 
\subsection{Introductory remarks}
\begin{sloppypar}
I am indebted to James Essegbey for sharing and discussing unpublished Nyangbo data with me, as well as for helping with several glosses. Note that Nyangbo has a so-called \lq\lq factative" system typical of West African languages, in which the formally unmarked paradigm of the verb has a reading of a past event with dynamic verbs and a present state reading with those verbs commonly deemed to be stative. 
\end{sloppypar}

\subsection{ka-}
\subsubsection{General information}
\begin{itemize}
	\item Form: subject to vowel harmony.
	\item Wordhood: bound morpheme (verb prefix).
\end{itemize}


\subsubsection{As a  \lq still\rq{ }expression}
\begin{itemize}
	\item \citeauthor{Essegbey2012} (\citeyear{Essegbey2012}, \citeyear[159–161]{Essegbey2019}).
	\item Specialisation: examples like (\ref{exAppendixNyangbo1}–\ref{exAppendixNyangbo3}) give a fairly good indication that this marker conforms to my definition. For instance, in (\ref{exAppendixNyangbo1}) the third's friend persistent state of being at home is framed against the assumed alternative, namely that he is on his way to meeting the other friends (i.e. no longer at home). Further, albeit indirect, evidence for the specialisation of \mbox{\textit{ka}-} come from its uses as \textsc{not yet} (\appref{appendixNyangboNotYet}).
	\item Pragmaticity: compatible with both scenarios (tentative conclusion).
	\item Polarity sensitivity: outer negation yields \textsc{no longer}.
	\item Restriction: in the absence of overt aspect marking, \mbox{\textit{ka}-} only serves as \textsc{still} with stative verbs, whereas the combination with dynamic verbs yields \textsc{not yet}. In order to get an affirmative reading with dynamic verbs, \mbox{\textit{ka}-} needs to be combined with the progressive aspect; thus compare (\ref{exAppendixNyangbo3}) to (\ref{exAppendixNyangboNotYet3}) below.
\end{itemize}
\begin{exe}
	\ex\label{exAppendixNyangbo1}
	Context: Three friends had agreed to meet at the roadside to catch a bus. Two of them met, waited for a bit and then one of them called the third friend on his cell phone. He reports:\\
	\gll A-\textbf{ka}-lɛ́ bɔ-pã́-m.\\
	\textsc{subj}.3\textsc{sg}-still-be\_at \textsc{ncl}-house-inside\\
	\glt \lq He is still at home.’ \parencite[46]{Essegbey2012}

	\ex\label{exAppendixNyangbo2}
	Context: A man had been waiting for his girlfriend for a long time and got angry. When she eventually showed up, her intents of mollifying him showed no effect. \\
	\gll Yē-nú sɛ ɔ-mɔpɔɛ nɔ́ lɔ-\textbf{kɔ}-kpasɛ yɛ shú mɛ kokoko sɛe-be-po-é.\\
	\textsc{subj}.3\textsc{sg}-be that \textsc{ncl}-anger \textsc{def} \textsc{subj}-still-be\_contained 3\textsc{sg} skin inside by\_all\_means that-\textsc{subj}.3\textsc{sg}-\textsc{ven}-wait\_for-\textsc{obj}.3\textsc{sg}\\
	\glt \lq That is to say he was still very angry that he had come and waited for her.' \parencite[46]{Essegbey2012}

	\ex\label{exAppendixNyangbo3}
	\gll A-\textbf{ka}-á-ta̅kɛ̅ sikã́.\\
	\textsc{subj}.3\textsc{sg}-still-\textsc{prog}-pick money.\\
	\glt \lq He is still collecting the money.' (James Essegbey, p.c.)
\end{exe}

\subsubsection{Uses related toother phasal polarity-concepts}
\paragraph{Not yet}\label{appendixNyangboNotYet}
\begin{itemize}
	\item \citeauthor{Essegbey2012} (\citeyear{Essegbey2012}, \citeyear[159–161]{Essegbey2019}).
	\item This function obtains with dynamic verbs in the absence of overt aspectual marking; thus compare (\ref{exAppendixNyangboNotYet3}) to (\ref{exAppendixNyangbo3}) above.
	\item All available examples of this reading have a past or perfective interpretation (as indicated by the use of the English anterior, as well as by \cite{Essegbey2019} referring to it as the \lq\lq negative perfect").
\end{itemize}

\begin{exe}
	\ex
	\gll ɛ-\textbf{ka}-ŋa bɔ-dɔ.\\
	\textsc{subj}.1\textsc{sg}-still-eat \textsc{ncl}-thing\\
	\glt \lq I have not yet eaten.' \parencite[160]{Essegbey2019}

	\ex
	Context: About making palm wine. When no more wine comes out of the tapping hole, the palm tree is dead.\\
	\gll So otsíɛ́ tsyɔ̃ɔ̃ gɛ be-\textbf{ke}-tsí soɔ i-í-bó.\\
	therefore now right\_away \textsc{rel} \textsc{subj}.3\textsc{pl}-still-die therefore \textsc{subj}.1\textsc{sg}-\textsc{prog}-tap\_palm\_wine\\
	\glt \lq So now that they are not yet dead, I am tapping.\rq{ }(James Essegbey, p.c.)
	
	\ex\label{exAppendixNyangboNotYet3}
	\gll A-\textbf{ka}-ta̅kɛ̅ sikã́.\\
	\textsc{subj}.3\textsc{sg}-still-pick money.\\
	\glt \lq He hasn’t collected the money yet.' (James Essegbey, p.c.)
\end{exe}
\il{Nyangbo|)} 

\section{Plateau Malagasy (plt, plat1254)}\label{appendixMalagasy}\il{Malagasy, Plateau|(}
\subsection{mbola}
\subsubsection{General information}
\begin{itemize}
	\item Form: transcribed as \textit{b̃ula} in \textcite{Garvey1964}.
	\item Wordhood: free morpheme.
	\item Syntax: fixed, preceding the predicate.
\end{itemize}


\subsubsection{As a  \lq still\rq{ }expression}
\begin{itemize}
	\item \textcite[536–537]{Dez1980}, \textcite[71]{Garvey1964}, \textcite[80, 151]{Malzac1960}, \textcite{Rackowski1998}, \textcite[12, 69]{Rajaonarimanana2001} and \textcite{MalagasyPhd}.
	\item Specialisation: \citeauthor{Dez1980}'s description addresses the components of my definition.
	\item Pragmaticitiy: the data do not allow for any conclusion.
	\item Polarity sensitivity: inner negation yields \textsc{not yet}; in temporal clauses introduced by \textit{dieny} \lq while' this also serves to signal precedence (see \cite[460]{Dez1980}; \cite[12]{MalagasyDictionary}).
\end{itemize}
\begin{exe}
	\ex\label{exAppendixMalagasy1}
	\gll Raha ho \textbf{mbola} velona izy dia ho n-aha-soa olona maro tokoa.\\
	if \textsc{fut} still alive 3\textsc{sg} \textsc{top} \textsc{fut} \textsc{pst}:\textsc{agt}.\textsc{foc}-\textsc{caus}-good person many \textsc{intens}\\
	\glt \lq S\rq{}il était encore en vie, il aurait fait du bien à beaucoup de personnes. [If he were still alive, he would have done good for many people.]\rq{ }\parencite[90]{Dez1980}
	
	\ex\label{exAppendixMalagasy2}
	\gll \textbf{Mbola} antoandro ny andro.\\
	still broad\_daylight \textsc{det} day\\
	\glt \lq Il fait encore grand jour. [It's still broad daylight.]\rq{ }(\cite[151]{Malzac1960}, glosses added)

	\ex\label{exAppendixMalagasy3}
	\gll Zazavavy \textbf{mbola} tanora io ka efa dimy anaka sahady.\\
	girl still young \textsc{dem}.\textsc{sg} and already five child already\\
	\glt \lq C’est une femme encore jeune et elle a déjà cinq enfants. [She is still a young woman and she already has five children.]\rq{ }(\cite[151]{Dez1980}, glosses added)
\end{exe}

\subsubsection{Uses on the fringes of \lq still\rq{}}

\paragraph{Scalar contexts}\label{appendixMalagasyScalar}
\begin{itemize}
	\item \textcite[398–400]{Dez1980} and \textcite[153]{Malzac1960}.
	\item While in the data consulted \textit{mbola} is not attested with scalar items strictly speaking, there are fixed collocations with two wh-words, \textit{aiza} \lq where' (\ref{exAppendixMalagasyScalar1}) and \textit{rahoviana} \lq when' (\ref{exAppendixMalagasyScalar2}). These collocations refer to large remaining quantities (of distance and time).
\end{itemize}

\begin{exe}
	\ex\label{exAppendixMalagasyScalar1}
	\gll \textbf{Mbola} aiza i<za>ny.\\
	still where \textsc{dem}.\textsc{sg}<\textsc{invis}>\\
	\glt \lq C'est encore loin, il c'en faut encore de beaucoup, il y en a encore pour longtemps (a marcher, a progresser). [It’s still far away, it’s still a long way to go, it’s still a long time walking or moving.]' (\cite[398]{Dez1980}, glosses added)
	
	\ex\label{exAppendixMalagasyScalar2}
	\gll \textbf{Mbola} rahoviana no tonga izy.\\
	still when \textsc{foc} arrive 3\textsc{sg}\\
	\glt \lq Il viendra dans longtemps. [He'll arrive some long time in the future.]\rq{ } (\cite[400]{Dez1980}, glosses added)
\end{exe}

\paragraph{Continued iteration (?)}
\label{appendixMalagasyContinuedIteration}
\begin{itemize}
	\item \textcite[537]{Dez1980}.
	\item According to \textcite[537]{Dez1980} \lq\lq [m]bola peut accompagner un prédicat déterminé par le marqueur ambimodal indray, marqueur du renouvellement de  l'ac\-tion. Sa présence introduit une nuance de sens complémentaire malaisée à rendre en français." [\textit{Mbola} can accompany a predicate modified by the ambimodal marker \textit{indray}, a marker of iteration. Its presence introduces an additional nuance of meaning that is difficult to render in French]. Judging from the free translation of the sole example (\ref{exAppendixMalagasyIteration}), this nuance of meaning is one of continued iteration (\lq yet again').
\end{itemize}

\begin{exe}
	\ex
	\label{exAppendixMalagasyIteration}
	\gll (\textbf{Mbola}) avy indray ny orana.\\
	still come again \textsc{det} rain\\
	\glt \lq Il pleut encore une fois, la pluie tombe encore (une nouvelle fois). [It's raining once again, it is raining again (one more time).]\rq{ }(\cite[537]{Dez1980}, glosses added)
\end{exe}

\subsubsection{Broadly adverbial temporal-aspectual uses}
\paragraph{Prospective \lq eventually\rq{}}\label{appendixMalagasyProspective}
\begin{itemize}
	\item \textcite[537]{Dez1980} and \textcite[173]{Ferrand1903}
	\item This function occurs in combination with the future tense. \textcite[173]{Ferrand1903} describes it as signalling que \lq\lq l'action exprimée par le verbe s'accomplira dans un temps indéterminé\rq\rq{ }[that the action described in the verb will be accomplished at some indefinite time].
\end{itemize}

\begin{exe}
	\ex
	\gll \textbf{Mbola} ha-nao i<za>ny izy.\\
	still \textsc{fut}:\textsc{agt}.\textsc{foc}-do \textsc{dem}.\textsc{sg}<\textsc{invis}> 3\textsc{sg}\\
	\glt \lq Il a encore cela à faire, il fera encore cela, il lui reste encore cela à faire.  [He still has to do that, he will do that yet, it still remains for him to do.]' (\cite[537]{Dez1980}, glosses added)
	
	\ex
	\gll \textbf{Mbola} ho avy.\\
	still \textsc{fut} come\\
	\glt \lq Il doit venir dans un délai plus ou moins long. [He should come sooner or later.]' (\cite[173]{Ferrand1903}, glosses added)
	
	\ex
	\gll {Any ho any} \textbf{mbola} h-i-vidy omby aho.\\
	{later} still \textsc{fut}:\textsc{agt}.\textsc{foc}-buy ox 1\textsc{sg}\\
	\glt \lq J'espère acheter des bœufs plus tard. [I hope to buy oxen later.]\rq{ }(\cite[79]{Malzac1960}, glosses added)
\end{exe}

\subsubsection{Temporal connectives and frame setters}
\paragraph{Coexistensiveness: \textit{raha mbola} \lq as long as'}
\label{appendixMalagasyAsLongAs}
\begin{itemize}
	\item \textcite[457]{Dez1980}.
	\item This function obtains in combination with \textit{raha} \lq if, when'.
	\item This is a semi-transparent use, which signals that the situation described in the matrix clause predicate holds true in all circumstances where the condition in the subordinate clause continues to be met.
\end{itemize}

\begin{exe}
	\ex
	\gll \textbf{Raha} \textbf{mbola} mi-asa ianao dia ma-hazo vola ary raha mbola hi-asa ihany ianao dia ato no tsara.\\
	if/when still \textsc{agt}.\textsc{foc}-work 2\textsc{sg} \textsc{top} \textsc{agt}.\textsc{foc}-get silver and if/when still \textsc{fut}:\textsc{agt}.\textsc{foc}-work only 2\textsc{sg} \textsc{top} here \textsc{foc} good\\
	\glt \lq Tant que tu travailles, tu gagnes de l’argent, et si tu veux encore travaillair, c’est ici que c’est bien. [As long as you work, you earn money, and if you still want to work, this is the place to do it.]\rq{ }(\cite[457]{Dez1980}, glosses added)
	
	\ex\label{exAppendixMalagasyRahaMbola2}
	\gll Ma-toki-a \textbf{raha} \textbf{mbola} velona aina.\\
	\textsc{agt}.\textsc{foc}-confident-\textsc{imp} if/when still alive breath\\
	\glt \lq Tant qu'il y a de la vie, il y a de l’espoir. [As long as there's life, there's hope.]\rq{ }(\cite[157]{MalagasyPhd}, glosses added)
\end{exe}

\subsubsection{Marginality}
\label{appendixMalagassyMarginal}
\begin{itemize}
	\item Judging from the context and  explanation by \citeauthor{Dez1980}, ex. (\ref{exAppendixMalagasyMarginal}) indicates that the subject constitutes a marginal member of the category \lq house', although a PhP reading cannot be excluded.
\end{itemize}

\begin{exe}
	\ex\label{exAppendixMalagasyMarginal}
	Context: About a house with a ruined roof.\\
	\gll \textbf{Mbola} trano io.\\
	still house \textsc{dem}.\textsc{sg}\\
	\glt \lq Ceci sert encore de maison (parce que, par exemple, le toit, quoique endommagé, peut encore servir d'abri). [This still serves as a house (because, for example, the roof, although damaged, can still serve as a shelter).]' (\cite[128]{Dez1980}, glosses added)
\end{exe}

\subsubsection{Additive and related functions}
\paragraph{Additive}\label{appendixMalagasyAdditive}
\begin{itemize}
	\item There is only one instance of this function (\ref{exAppendixMalagasyAdditiveAlso}) in the references consulted.
\end{itemize}

\begin{exe}
	\ex\label{exAppendixMalagasyAdditiveAlso}
	\gll Efa mbola tsy mahay lesona Rakoto no \textbf{mbola} mitabataba.\\
	already still \textsc{neg} know lesson Rakoto \textsc{foc} still talkative\\
	\glt \lq Not only does he [Rakoto] not know his lessons, but he is talkative also.' \parencite[16]{Rackowski1998}
\end{exe}
\il{Malagasy, Plateau|)}

\section{Ruuli  (ruc, ruul1235)}\label{appendixRuuli}\il{Ruuli|(}
\subsection{Introductory remarks}
\begin{sloppypar}
Aside from descriptive materials, I consulted an unpublished Ruuli corpus \parencite{RuuliCorpus}. I am indebted to Alena Witzlack-Makarevich for sharing the latter with me, as well as for helping with some tricky glosses. Note that Ruuli has a typical Narrow Bantu noun class system. Noun class prefixes in the examples are glossed as \textsc{ncl} for \lq noun class', together with an Arabic numeral that follows the common Bleek-Meinhof system of referring to Bantu noun classes.
\end{sloppypar}

\subsection{kya-}
\subsubsection{General information}
\begin{itemize}
	\item Wordhood: bound morpheme (verb prefix).
\end{itemize}


\subsubsection{As a  \lq still\rq{ }expression}
\begin{itemize}
 	\item \textcite{MolochievaEtAl2021} and \textcite[63–66]{NamyaloEtAl2021}.
	\item Specialisation: the description by \citeauthor{MolochievaEtAl2021} meets my definition.
	\item Polarity sensitivity: external negation yields \textsc{no longer}.
	\item Pragmaticity: compatible with both scenarios.
	\item Further note: \mbox{\textit{kya}-} as \textsc{still} is compatible with the perfective aspect in case of inchoative verbs, as in (\ref{exAppendixruuliPFV})
\end{itemize}
\largerpage
\begin{exe}
	\ex
	\gll N-a-som-ere nga n-\textbf{kya}-li mu-to.\\
	\textsc{subj}.1\textsc{sg}-\textsc{pst}-study-\textsc{pfv} while \textsc{subj}.1\textsc{sg}-still-\textsc{cop} \textsc{ncl}1-young\\
	\glt \lq I studied while I was still young.' \parencite[75]{MolochievaEtAl2021}

	\ex
	\gll Aba-ntu ba-ingi ba-\textbf{kya}-kolesya emole oko-umboka ennyumba za-abwe.\\
	\textsc{ncl}2-person \textsc{ncl}2-many \textsc{subj}.\textsc{ncl}2-still-use reed.\textsc{ncl10} \textsc{ncl}15(\textsc{inf})-build house.\textsc{ncl}10 \textsc{ncl}10-\textsc{poss}.\textsc{ncl}2\\
	\glt \lq Many people still use reeds to construct their houses.\rq{ }\parencite[75]{MolochievaEtAl2021}

	\ex\label{exAppendixruuliPFV}
	\gll Oitaamu n'-ommaawo ma-ka ga-\textbf{kya}-ikaire.\\
	father:\textsc{poss}.2\textsc{pl} \textsc{com}-mother:\textsc{poss}.2\textsc{pl} \textsc{ncl}6-home \textsc{subj}.\textsc{ncl}6-still-be(come)\_calm.\textsc{pfv}\\
	\glt \lq Your father and mother's marriage is still stable.\rq{ }\parencite{RuuliCorpus}
\end{exe}

\subsubsection{Uses on the fringes of \lq still\rq{}}
\paragraph{Scalar contexts}
\label{appendixRuuliScalar}
\begin{itemize}
	\item \textit{Kya}- is attested in contexts of decreases over time, as in (\ref{exAppendixRuuliScalar1}).
	\item There is only clear-cut example of a limited increase in the data (\ref{exAppendixRuuliScalar2}). Given that this example features the perfective aspect inflection, it might well be subsumed under the \lq thus far only\rq{ }use (\appref{appendixRuuliRestrictive}).
\end{itemize}

\begin{exe}
	\ex\label{exAppendixRuuliScalar1}
	\gll Yete ate awo lizi w-a-mw-iyaku, w-a-mu-wa-ire emyaka enai na itaanu. Ki-ni ki-ri-wo nga wa-kya-li-wo amyaka asatu\\
how\_about but \textsc{dem}.\textsc{ncl}16(\textsc{loc}) lease \textsc{subj}.2\textsc{sg}-\textsc{pst}-\textsc{obj}.\textsc{ncl}1-remove \textsc{subj}.2\textsc{sg}-\textsc{pst}-\textsc{obj}.\textsc{ncl}1-give-\textsc{pfv} \textsc{ncl}4.year \textsc{ncl}4.four with \textsc{ncl}4.five \textsc{ncl}7-\textsc{prox} \textsc{subj}.\textsc{ncl}7-\textsc{cop}-\textsc{ncl}16(\textsc{loc}) when \textsc{subj}.\textsc{ncl}16(\textsc{loc})-still-\textsc{cop}-\textsc{ncl}16(\textsc{loc}) \textsc{ncl}6(?).year thirty\\
\glt \lq The lease you gave out was for 4-5 years. And this happened when we still have thirty years [left].\rq{} (\cite{RuuliCorpus}, glosses added)

	\ex\label{exAppendixRuuliScalar2}
	\gll Eirai budi ni-ba-kya-li ku-leeta bi-ni eby-a nasare, aka-ana ka-\textbf{kya}-zw-\textbf{ire}=mbe oku ma-beere nti ka-ab-e ka-tandika oku-soma.\\
	in\_the\_past previously when-\textsc{subj}.\textsc{ncl}2-still-\textsc{cop} \textsc{ncl}15(\textsc{inf})-bring \textsc{ncl}8-\textsc{prox} \textsc{ncl}9-\textsc{assoc} nursery \textsc{ncl}12-child \textsc{subj}.\textsc{ncl}12-still-abandon-\textsc{pfv}=\textsc{foc} \textsc{ncl}17(\textsc{loc}) \textsc{ncl}6-breast \textsc{comp} \textsc{subj}.\textsc{ncl}12-go-\textsc{sbjv} \textsc{subj}.\textsc{ncl}12-start \textsc{ncl}15(\textsc{inf})-read\\
	\glt \lq Long ago, before they brought these nurseries, [when] a child has only stopped breastfeeding, it would go and start studying.\rq{ }(\cite{RuuliCorpus}, glosses added)
\end{exe}




\subsubsection{Uses related toother phasal polarity-concepts}
\paragraph{Not yet}
\label{appendixRuuliNotYet}
\begin{itemize}
 	\item \textcite{MolochievaEtAl2021} and \textcite[63–66]{NamyaloEtAl2021}.
	\item This function occurs in two context:
	\begin{itemize}
	\item In combination with copula \textit{li} plus an infinitival complement (\ref{exAppendixRuuliNotYet1}, \ref{exAppendixRuuliNotYet2}). 
	\item In combination with copula \textit{li}, without any overt predicate (\ref{exAppendixRuuliNotYet3}, \ref{exAppendixRuuliNotYet4}) 
	\end{itemize}	
	\item In temporal clauses, these collocations also serve as a signal of precedence (\lq when not yet \textit{p}, \textit{q}' \equiv \lq before \textit{p}, \textit{q}'), as shown in (\ref{exAppendixRuuliNotYet3}).	
	\item This function might also underlie the adverb \textit{bukyali} \lq early', which can be segmented as \textit{bu}-\textit{kya}-\textit{li} \lq \textsc{subj}.\textsc{ncl14}-still-\textsc{cop}'.
\end{itemize}\largerpage

\begin{exe}
	\ex\label{exAppendixRuuliNotYet1}
	\gll Aba-sigazi ba-ni ba-\textbf{kya}-\textbf{li} \textbf{ku}-eteja kusai ki-ntu ki-ni.\\
	\textsc{ncl2}-boy \textsc{ncl}2-\textsc{prox} \textsc{subj}.\textsc{ncl}2-still-\textsc{cop} \textsc{ncl}15(\textsc{inf})-understand well \textsc{ncl}7-thing \textsc{ncl}7-\textsc{prox}\\
	\glt \lq These boys do not yet understand well this thing.\rq{ }\parencite[76]{MolochievaEtAl2021}
	
	\ex\label{exAppendixRuuliNotYet2}
	\gll Tu-\textbf{kya}-\textbf{li} \textbf{ku}-ika ku n-zala y-a maani.\\
	\textsc{subj}.1\textsc{pl}-still-\textsc{cop} \textsc{ncl}15(\textsc{inf})-arrive \textsc{ncl}17(\textsc{loc}) \textsc{ncl}9-famine \textsc{ncl}9-\textsc{assoc} strength.\textsc{ncl9}\\
	\glt \lq We have not yet reached terrible famine.' \parencite[81]{MolochievaEtAl2021}
	
	\ex\label{exAppendixRuuliNotYet3}
	\gll Naye eirai ni tu-\textbf{kya}-li aba-ana ti-tu-a-ba-bala-nga.\\
		but in\_the\_past \textsc{conj} \textsc{subj}.1\textsc{pl}-still-\textsc{cop} \textsc{ncl}2-child \textsc{neg}-\textsc{subj}.1\textsc{pl}-\textsc{pst}-\textsc{obj}.\textsc{ncl}2-count-\textsc{hab}\\
	\glt \lq But in the past, before we became (born again Christians) we used not to count children.’ (lit. But in the past, when we were not yet (born-again Christians,) we used not to count children.\rq{ }\parencite[76]{MolochievaEtAl2021}

	\ex\label{exAppendixRuuliNotYet4}
	Context: B has troubles with their eyes.\\
	\begin{xlist}
		\exi{A:}  \textit{Ati gabbangaku ob-janjabi?}\\
		 \lq Did the eyes have treatment?'
		\exi{B:}\gll Ati ba-n-dongos-ere li-nu, li-ni li-\textbf{kya}-\textbf{li}.\\
		now \textsc{subj}.\textsc{ncl}2-\textsc{obj}.1\textsc{sg}-operate-\textsc{pfv} \textsc{ncl}5-\textsc{prox} \textsc{ncl}5-\textsc{prox} \textsc{subj}.\textsc{ncl}5-still-\textsc{cop}\\
		\glt \lq They operated me this one, this one, not yet.\rq{ }(\cite{RuuliCorpus}, glosses added)
	\end{xlist}
\end{exe}

\subsubsection{Restrictive (non-temporal)}
\paragraph{Thus far only}
\label{appendixRuuliRestrictive}
\begin{itemize}
	\item \textcite{MolochievaEtAl2021}.
	\item According to \textcite[87]{MolochievaEtAl2021} this is the only reading available when \mbox{\textit{kya}-} is combined with the perfective aspect; this is, however, contradicted by examples such as (\ref{exAppendixruuliPFV}) above.
	\item The addition of restrictives in (\ref{exAppendixRuuliOnly1}, \ref{exAppendixRuuliOnly2}) seems to stress this reading.
\end{itemize}
\begin{exe}

	\ex	\label{exAppendixRuuliOnly1}
	\gll N-\textbf{kya}-li-\textbf{ire}		matooke		go-nkai.\\
	\textsc{subj}.1\textsc{sg}-still-eat-\textsc{pfv} \textsc{ncl}6.matooke \textsc{ncl}6-only\\
	\glt \lq So far I have only eaten matoke (and nothing else).\rq{ }\parencite[87]{MolochievaEtAl2021}

	\ex	\label{exAppendixRuuliOnly2}	
		\gll Yee	n-\textbf{kya}-byal-\textbf{ire} ba-ala 	ba-ereere.\\
	yes	\textsc{subj}.1\textsc{sg}-still-give\_birth\_to-\textsc{pfv}	\textsc{ncl}2-girl \textsc{ncl}2-only\\
	\glt \lq Yes, I have so far given birth only to girls.\rq{ }\parencite[87]{MolochievaEtAl2021}
\end{exe}
\il{Ruuli|)}

\section{Sango (sag, sang1328)}\label{appendixSango}\il{Sango|(}
\subsection{Introductory remarks}
I am indebted to Helma Pasch for discussing Sango data with me.

\subsection{de}
\subsubsection{General information}
\begin{itemize}
	\item Wordhood: intermediate (auxiliary-like element), but also occurs in a fixed impersonal form \textit{áde}.
\end{itemize}


\subsubsection{As a  \lq still\rq{ }expression}
\begin{itemize}
	\item \textcite[53]{Tisserant1950} and \textcite{NassensteinPasch2021}; also see \textcite[104–105]{Lekens1955} on \textit{dɛ} in Ngbandi,\il{Ngbandi} the main lexifier of modern Sango.
	\item Specialisation: the description by \citeauthor{NassensteinPasch2021} meets my definition.
	\item Pragmaticity: the available data do not allow any conclusions.
	\item Polarity sensitivity: inner negation yields \textsc{not yet}.
	\item Additional note: \textit{de} can be combined with the French loan \textit{encore} for emphasis, as in (\ref{exAppendixSango2}).
\end{itemize}
\begin{exe}
	\ex 
	\gll Lo \textbf{de} na gangu ti lo\\
	3\textsc{sg} still \textsc{prep} strength of 3\textsc{sg}\\
	\glt \lq S/he is still at the peak of his/her strength.\rq{ }\parencite[114]{NassensteinPasch2021}
	
 \ex\label{exAppendixSango2}
	\gll Lo \textbf{de} (encore) na Bangui\\
    3\textsc{sg} still \phantom{(}still \textsc{prep} B.\\
	\glt \lq S/he is still in Bangui.' \parencite[114]{NassensteinPasch2021}
 
	\ex \gll Na l’heure só mbi \textbf{de} kété\sim{}kété, babá tí mbi a-goe na {Fort Archambault}\\
	\textsc{prep} time \textsc{dem} 1\textsc{sg} still \textsc{redupl}\sim{}small father of 1\textsc{sg} \textsc{subj}-go \textsc{prep} {F. A.}\\
	\glt \lq When I was still very small, my father went to Fort Archambault.' (\cite[238]{Samarin1963}, glosses added)
\end{exe}

\subsubsection{Uses related toother phasal polarity concepts}
\paragraph{Not yet}\label{appendixSangoNotYet}
\begin{itemize}
	\item \textcite{NassensteinPasch2021}; also see \textcite[104–105]{Lekens1955}.
	\item All cases feature the absence of an overt predicate. This includes negative responses to polar questions (\ref{exAppendixSangoNotYet1}, \ref{exAppendixSangoNotYet2}), which involve the impersonal form \textit{áde}. It also subsumes a clause pattern consisting of a nominal subject plus \textit{de} and for which the available examples suggest that the nominal must be associated with a telic process, such as in (\ref{exAppendixSangoNotYet3}, \ref{exAppendixSangoNotYet4}); note that \textcite[115]{NassensteinPasch2021} describe this use as featuring affirmative polarity: {\lq\lq}It indicates the \textsc{still} phase, but with regard to a situation which is not given but expected … [t]he expectation that the given situation is going to change alludes to the \textsc{not yet} phase".
\end{itemize}

\begin{exe}
	\ex\label{exAppendixSangoNotYet1}
	\gll \textbf{Áde},	(lo	gwe	ape).\\
	still \phantom{(}3\textsc{sg}	go	\textsc{neg}\\
	\glt \lq Not yet (she has not gone yet).' \parencite[117]{NassensteinPasch2021}

	\ex\label{exAppendixSangoNotYet2}
	\begin{xlist}
		\exi{A:}\gll Ballon ni a-hunzi awe?\\
		football\_match \textsc{def} \textsc{subj}-finish already\\
		\glt \lq Is the football match over?ʼ
		\exi{B:}\gll \textbf{Áde}, \textup{(}a-hunzi ape\textup{)}.\\
		still \phantom{(}\textsc{subj}-finish \textsc{neg}\\
		\glt \lq Not yet.' \parencite[116]{NassensteinPasch2021}
	\end{xlist}

	\ex\label{exAppendixSangoNotYet3}
	\gll Yaka ti gozo	 a-\textbf{de}.\\
	plantation	of cassava \textsc{subj}-still\\
	\glt The [work on the] cassava plantation is not yet done / not yet finished.' (original translation: \lq… will/must be continued/finished.') \parencite[115]{NassensteinPasch2021}

	\ex\label{exAppendixSangoNotYet4}
	\gll (Ngoyi	ti) ngu a-\textbf{de}.\\
	season of	cassava \textsc{subj}-still\\
	\glt \lq The rainy season has not yet come / not yet finished.' (original translation: ʻ… is awaiting, has not yet begun / ended.ʼ)\rq{ }\parencite[115]{NassensteinPasch2021}
\end{exe}
\il{Sango|)}

\section{Southern Ndebele (nbl, sout3270)}\il{Ndebele, Southern|(}

\subsection{Introductory remarks}
The variety discussed here is listed as \lq\lq South Transvaal Ndebele" in glottolog, and is not to be confused with Zimbabwean Ndebele (nde, nort2795) or the other South African Ndbele variety, Sumayela Ndebele (no ISO code, sout2808). I am indebted to Petrus Mabena for discussing Southern Ndebele data with me and for providing additional examples. Note that Southern Ndebele has a fairly typical Narrow Bantu noun class system. I gloss the individual classes as \textsc{ncl} \lq noun class' plus a subscript number that follows the common Bleek-Meinhof system.

\subsection{sa-}
\subsubsection{General information}
\begin{itemize}
	\item Form: \textit{se}- /\textit{sese}- with copulatives.
	\item Wordhood: bound morpheme (prefix).
\end{itemize}


\subsubsection{As a  \lq still\rq{ }expression}
\begin{itemize}
	\item \textcite{CranePersohn2021}.
	\item Specialisation: \textcite{CranePersohn2021} describe this expression as one that is in line with my definition, and also point out its incompatibility with inalterable states.
	\item Pragmaticity: compatible with both scenarios.
	\item Polarity sensitivity: outer negation yields \textsc{no longer}.
\end{itemize}

\begin{exe}
	\ex
	\gll Ba-\textbf{sa}-funda.\\
	\textsc{subj}.\textsc{ncl}2-still-read\\
	\glt \lq They are still reading / still read.' \parencite[237]{CranePersohn2021}

	\ex
	\gll U-Sipho u-\textbf{sa}-gula.\\
	\textsc{ncl}1a-S. \textsc{subj}.\textsc{ncl}1-still-be\_ill\\
	\glt \lq Sipho is still sick.' \parencite[237]{CranePersohn2021}
\end{exe}

\subsubsection{Uses on the fringes of \lq{}still\rq{}}
\paragraph{Scalar contexts}
\label{appendixSouthernNdebeleScalar}
\begin{itemize}
	\item \textit{Sa}- is compatible with scalar contexts, both involving a direction of decrease (\ref{exAppendixSouthernNdebeleScalar1}, \ref{exAppendixSouthernNdebeleScalar2}) and increase (\ref{exAppendixSouthernNdebeleScalar3}–(\ref{exAppendixSouthernNdebeleScalar5}). The latter cases also encompass \lq no later than' readings (\ref{exAppendixSouthernNdebeleScalar5}). Note the non-obligatoriness of a restrictive operator in increase contexts (\ref{exAppendixSouthernNdebeleScalar3}, \ref{exAppendixSouthernNdebeleScalar4}).

	\item Example (\ref{exAppendixSouthernNdebeleScalar5}), which features the perfective aspect inflection, could also be subsumed under the \lq thus far only\rq{ }use (\ref{appendixSouthernNdebeleRestrictive}), which is the context in which \textcite{CranePersohn2021} discuss it.
\end{itemize}

\begin{exe}
	\ex\label{exAppendixSouthernNdebeleScalar1}
	 Context: Weʼre at a concert in a pub. I suggest go get another drink, but my friend says he has run out of money. I offer to invite him.\\
	\gll   Yewize, ngi-\textbf{sese}-ne-khulu l-ama-randa .\\
	come.\textsc{imp} \textsc{subj}.1\textsc{sg}-still.\textsc{cop}-with.\textsc{ncl}5-hundred \textsc{ncl}5-\textsc{assoc}.\textsc{ncl}6-rand\\
	\glt \lq Come, I still have 100 Rand.' (Peter Mabena, p.c.)
	
	\ex\label{exAppendixSouthernNdebeleScalar2}
	 Context: We have an apple tree in our garden and weʼve harvested a lot, but thereʼs no end in sight. So we tell our friend:\\
	\gll Na-wu-funa ama-habhula a-\textbf{sese}-ma-nengi a-sele e-m-thini.\\
	if-\textsc{subj}.2\textsc{sg}-want \textsc{ncl}6-apple \textsc{subj}.\textsc{ncl}6-still.\textsc{cop}-\textsc{ncl}6-many \textsc{subj}.\textsc{ncl6}.\textsc{subord}-remain.\textsc{pfv} \textsc{loc}-\textsc{ncl}3-tree\\
	\glt \lq If you want apples, there are still a lot left in the tree.'
	\\(Peter Mabena, p.c.)
	
	\ex\label{exAppendixSouthernNdebeleScalar3}
	 Context: You are expected to be sent eight books. You have gotten six so far, and your friend tells you she already got all eight. You reply:\\
	\gll Ngi-\textbf{sese}-nesi-thandathu \textup{(}kwaphela\textup{)}.\\
	\textsc{subj}.1\textsc{sg}-still.\textsc{cop}-with.\textsc{ncl}10-six \phantom{(}merely\\
	\glt \lq I still have (only) six.' (Peter Mabena, p.c.)
	
	\ex\label{exAppendixSouthernNdebeleScalar4}
	Context: We are planning to go to an event in the evening. I want to leave already. You reply:\\
	\gll Ku-\textbf{sese}-yi-simbi y-esi-thandathu.\\
	\textsc{subj}.\textsc{loc}-still.\textsc{cop}-\textsc{cop}.\textsc{ncl}9-bell \textsc{ncl}9-\textsc{ncl}7-six\\
	\glt \lq It's still six o' clock.' (Peter Mabena, p.c.)

	\ex\label{exAppendixSouthernNdebeleScalar5}
	\gll Um-ngami u-John, ngi-\textbf{sa}-m-bon-e kabili.\\
	\textsc{ncl}1-friend:\textsc{poss}.1\textsc{sg} \textsc{ncl}1a-J. \textsc{subj}.1\textsc{sg}-still-\textsc{obj}.\textsc{ncl}1-see-\textsc{pfv} twice\\
	\glt \lq My friend John, up to now I've (only) seen him twice.\rq{ }\parencite[243]{CranePersohn2021}		
		\end{exe}



\subsubsection{Broadly adverbial temporal-aspectual functions}
\paragraph{Iterative}\label{appendixSouthernNdebeleIterative}
\begin{itemize}
	\item \textcite{CranePersohn2021}.
	\item This function is only available in combination with the perfective aspect and with verbs that are not inchoative (do not denote a resultant state). It commonly conveys an additional notion of irritation by the recurring situation.
	\item The iterative adverb \textit{godu} is optional in (\ref{exAppendixSouthernNdebeleIterative1}, \ref{exAppendixSouthernNdebeleIterative2}); it gives emphasis to the iterative reading.
\end{itemize}

\begin{exe}
	\ex\label{exAppendixSouthernNdebeleIterative1}
	\gll U-Sipho u-\textbf{sa}-buy-ile (godu).\\
	\textsc{ncl}1a-S. \textsc{subj}.\textsc{ncl}1-still-return-\textsc{pfv} (again)\\
	\glt \lq Sipho is back (yet) again.' \parencite[244]{CranePersohn2021}
	
	\ex\label{exAppendixSouthernNdebeleIterative2}
	\gll Idrayara i-\textbf{sa}-rhuny-ez-e irhembe (godu).\\
	\textsc{ncl}9.dryer \textsc{subj}.\textsc{ncl}9-still-shrink-\textsc{caus}-\textsc{pfv} \textsc{ncl}9.shirt (again)\\
	\glt \lq The dryer shrank a shirt again (it happened another time).\rq{ }\parencite[244]{CranePersohn2021}
\end{exe}

\paragraph{Prospective \lq eventually\rq{}}\label{appendixSouthernNdebeleProspective}
\begin{itemize}
	\item Like its \ili{Xhosa} cognate (\appref{appendixXhosaProspective}), \mbox{\textit{sa}-} has a prospective \lq eventually' reading when combined with forms of \textit{ya} \lq{}go\rq{ }and \textit{za} \lq come\rq{}, including their grammaticalised forms (see \cite{CraneMabena2019} for extensive discussion of the latter).
\end{itemize}

\begin{exe}
	\ex
	\gll Se-khe w-a-ya e-si-tolo na? – Awa ngi-\textbf{sa}-zoku-ya.\\
	already-do\_once \textsc{subj}.2\textsc{sg}-\textsc{subsec}-go \textsc{loc}-\textsc{ncl}7-store \textsc{q}  {} \textsc{neg} \textsc{subj}.1\textsc{sg}-still-\textsc{fut}-go\\
	\glt \lq Have you been to the store? -- No, Iʻll go yet/later.'
	\\(Peter Mabena, p.c.)
	
	\ex Context: After a heated discussion.\\
	\gll U-\textbf{sa}-zaku-bona bonyana ngubani kithi so-ba-bili o-nemb-ile-ko.\\
	\textsc{subj}.2\textsc{sg}-still-\textsc{fut-}see \textsc{comp} \textsc{cop}.who to.1\textsc{pl} 1\textsc{pl}-\textsc{ncl}2-two  \textsc{rel}.\textsc{subj}.\textsc{ncl}1-hit-\textsc{pfv}-\textsc{rel}\\
	\glt \lq You'll see yet who of us two is right.' (Peter Mabena, p.c.)
	
	\ex Context: About someone who works way to hard.\\
	\gll (E-ku-gcineni) u-\textbf{sa}-zi-balulal-isa ngom-sebenzi.\\	
	\phantom{(}\textsc{loc}-\textsc{ncl}15(\textsc{inf})-finish.\textsc{loc} \textsc{subj}.\textsc{ncl}1-still-\textsc{refl}-kill-\textsc{caus} \textsc{cop}.\textsc{ncl}3-work\\
	\glt \lq S/he'll work himself/herself to death.' (Peter Mabena, p.c.)
\end{exe}



\subsubsection{Marginality}
\label{appendixSouthernNdebeleMarginal}
\begin{itemize}
	\item \textit{Sa}- is compatible with readings of marginality.
\end{itemize}

\begin{exe}
	\ex\label{exAppendixSouthernNdebeleMarginal1}
	\begin{xlist}
	\exi{A:} Iʼm really annoyed my aunt has left most of her fortune to an animal shelter, and only 100,000 Rand to me.
	\exi{B:}\gll i-100,000 nokho i-\textbf{sese}-si-samba es-amukeleka-ko.\\
	\textsc{ncl}9-100,00 however \textsc{subj}.\textsc{ncl}9-still.\textsc{cop}-\textsc{cop}.\textsc{ncl}7-lump\_sum \textsc{rel}.\textsc{subj}.\textsc{ncl}.7-be\_alright-\textsc{rel}\\
	\glt \lq 100,00 is still a decent sum.' (Peter Mabena, p.c.)
\end{xlist}	

	\ex\label{exAppendixSouthernNdebeleMarginal2}
	Context: Talking about tennis skills.\\
     \gll Ngi-\textbf{se}-nga-m-hlula yena u-Paul, kodwana u-Mark u-ngcono ku-na-mi.\\
     \textsc{subj}.1\textsc{sg}-still-\textsc{pot}-\textsc{obj}.\textsc{ncl}1-beat \textsc{dem}.\textsc{ncl}1 \textsc{ncl}1a-P. but \textsc{ncl}1a-M. \textsc{subj}.\textsc{ncl}1-better \textsc{loc}-with-1\textsc{sg}\\
     \glt \lq I can still beat Paul, but Peter is better than me.' (Peter Mabena, p.c.)
	
	\ex\label{exAppendixSouthernNdebeleMarginal3}
	Context: Talking about political views.\\
	\gll U-Bongani u-\textbf{sese}-nobu-ngcono.\\
	\textsc{ncl}1a-B. \textsc{subj}.\textsc{ncl}1-still.\textsc{cop}-with.\textsc{ncl}14-moderation\\
	\glt \lq Bongani is still moderate (as opposed to others, who have more radical views).' (Peter Mabena, p.c.)

	\ex\label{exAppendixSouthernNdebeleMarginal4}
	\gll I-Thohoyandou i-\textbf{sese}-nga-phasi kwe-Limpopo\\
\textsc{ncl}9-T. \textsc{subj}.\textsc{ncl}9-still.\textsc{cop}-in-land \textsc{loc}.\textsc{assoc}.\textsc{ncl}9-L.\\
	\glt \lq Thoyandou is still within Limpopo province.' (Peter Mabena, p.c.)
\end{exe}

\subsubsection{Restrictive (non-temporal)}
\paragraph{\lq{}Thus far only\rq{}}
\label{appendixSouthernNdebeleRestrictive}
\begin{itemize}
	\item \textcite{CranePersohn2021}.
	\item This reading obtains with the perfective aspect and non-inchoative verbs.
		\item \textcite{CranePersohn2021} suggest that this use goes back to aspectual coercion, originating with predicates that form part of a natural sequence of events, as in (\ref{exAppendixSouthernNdebeleThusFarOnly2}).
\end{itemize}
\begin{exe}
	\ex
	\gll Ngi-\textbf{sa}-dl-e kancani nje.\\
	\textsc{subj}.1\textsc{sg}-still-eat-\textsc{pfv} a\_little now\\
	\glt \lq I've just eaten a little portion for now.' \parencite[243]{CranePersohn2021}
		
	\ex\label{exAppendixSouthernNdebeleThusFarOnly2}
	 \gll Ikukhu i-\textbf{sa}-bekele.\\
	 \textsc{ncl}9.chicken \textsc{subj}.\textsc{ncl}9-still-lay\_eggs.\textsc{pfv}\\
	 \glt \lq The chicken has (only) laid these eggs (i.e. it has not yet started brooding).' \parencite[275]{CranePersohn2021}
\end{exe}
\il{Ndebele, Southern|)}

\section{Swahili (swh, swah1253)}
\label{appendixSwahili}\il{Swahili|(}

\subsection{Introductory remarks}
Apart from the descriptive materials below, I searched the Helsinki Corpus of Swahili 2.0 (\url{http://urn.fi/urn:nbn:fi:lb-201608301}). I am furthermore indebted to Ponsiano Kanijo, Maude Devos, and Rasmus Bernander for discussing Swahili data with me. Swahili has two \textsc{still} expressions: \textit{bado} and the much less frequent \textit{ngali}. Only for \textit{bado} do I have clear indications of additional functions. Also note that Swahili has a fairly typical Narrow Bantu noun class system. I have glossed the individual classes as \textsc{ncl} \lq noun class' plus a subscript number that follows the common Bleek-Meinhof system.
\subsection{bado}
\subsubsection{General information}
\begin{itemize}
	\item Wordhood: free morpheme.
	\item Syntax: either preceding the predicate or in clause-final position.
	\item Etymology: from Omani Arabic\il{Arabic, Omani} \textit{baʕd}-\textit{u} \lq still'.
\end{itemize}


\subsubsection{As a  \lq still\rq{ }expression}
\begin{itemize}
	\item \textcite[270–271]{Ashton1947}, \textcite[44–45, 104]{Mpiranya2015} \textcite[85]{Sacleux19391941}, \textcite{Schadeberg1990} and \textcite{TUKI2014}.
	\item Specialisation: the various descriptions of this marker, when taken together, give a fairly good indication that this marker conforms to my definition. This is corroborated by the fact that its use is odd in contexts that lack continuity (e.g. ex. \ref{exAppendixSwahili4}) and with inalterable states (e.g. ex. \ref{exAppendixSwahili5}). Further, albeit indirect, evidence comes from its use as \textsc{not yet} (\appref{appendixSwahiliNotYet}).
	\item Pragmaticity: \textcite{Schadeberg1990} describes this marker as implying an unexpectedly late continuation. This is, however, a theory-driven conclusion; see \textcite{vanderAuwera2021} for a recent criticism. My data shows that \textit{bado} is compatible with both scenarios.
	\item Polarity sensitivity: inner negation yields \textsc{not yet}
	\item Further note: \textit{bado} can serve as an elliptical affirmative answer, as in (\ref{exAppendixSwahili3}). It is also compatible with anterior \mbox{\textit{me}-} in a stative reading (\ref{exAppendixSwahili2}).
\end{itemize}
\begin{exe}
	\ex[]{\label{exAppendixSwahili1}
	\gll \textbf{Bado} ni-na-soma ma-gazeti.\\
	still \textsc{subj}.1\textsc{sg}-read \textsc{ncl}6-newspaper.\\
	\glt \lq I am still reading newspapers / I still read newspapers.\rq{ }(\cite[104]{Mpiranya2015}, glosses added)}
	
	\ex[]{\label{exAppendixSwahili2}
	\gll \textbf{Bado} a-me-lala.\\
	still \textsc{subj}.\textsc{ncl}1-\textsc{ant}-(fall)\_asleep\\
	\glt \lq S/he is still sleeping.' (\cite[270]{Ashton1947}, glosses added)}
	
	\ex[]{\label{exAppendixSwahili3}
	\gll Wewe \textbf{bado} ni mw-anafunzi? – \textbf{Bado}.\\
	2\textsc{sg} still \textsc{cop} \textsc{ncl}1-student {} still\\
	\glt \lq Are you still a student? -- Yes [I still am].' (Ponsiano Kanijo, p.c.)}
	
	\ex[\#]{\label{exAppendixSwahili4}
	\gll Leo asubuhi, Amani h-a-ku-wepo sokoni na sasa \textbf{bado} yu-po sokoni.\\
	today morning A. \textsc{neg}-\textsc{subj}.\textsc{ncl}1-\textsc{neg}.\textsc{pst}-\textsc{loc}.\textsc{cop} market and now still \textsc{subj}.\textsc{ncl}1-\textsc{loc}.\textsc{cop} market\\
	\glt (intended meaning: \lq This morning Amani was not at the market and now he still is at the market') (Ponsiano Kanijo, p.c.)}
	
	\ex[\#]{\label{exAppendixSwahili5}
	\gll Bibi y-angu \textbf{bado} ni m-kongwe.\\
	grandmother \textsc{ncl}9-\textsc{poss}.1\textsc{sg} still \textsc{cop} \textsc{ncl}1-ancient\\
	\glt (intended meaning: \lq My grandmother is still very old')
	\\(Ponsiano Kanijo, p.c.)}
\end{exe}

\subsubsection{Uses on the fringes of \lq{}still\rq}
\paragraph{Scalar contexts}
\label{appendixSwahiliScalar}
\begin{itemize}
	\item \textit{Bado} is compatible with scalar contexts, such as the decrement uses in (\ref{exAppendixSwahiliScalar1}, \ref{exAppendixSwahiliScalar2}) and restricted increase in (\ref{exAppendixSwahiliScalar3}, \ref{exAppendixSwahiliScalar4}); note the absence of a restrictive operator in the latter examples.
	\item A scalar function also underlies the common expression \textit{bado kidogo} \lq soon, not quite yet\rq{}, lit. \lq{}still a bit\rq{}.
\end{itemize}

\begin{exe}
	\ex\label{exAppendixSwahiliScalar1}
	\gll \lq\lq \textbf{Bado} maiti kama 42 zi-ko shimo-ni na ha-tu-ju-i kama kweli tu-ta-fanikiwa ku-zi-toa," mu-opoaji mw-ingine a-ka-sema.\\
	\phantom{\lq\lq}still corpse.\textsc{ncl}10 like 42 \textsc{ncl}10-\textsc{loc}.\textsc{cop} hole-\textsc{loc} \textsc{com} \textsc{neg}-\textsc{subj}.1\textsc{pl}-know-\textsc{neg} like really \textsc{subj}.1\textsc{pl}-\textsc{fut}-succeed \textsc{ncl}15(\textsc{inf})-\textsc{obj}.\textsc{ncl}10-pull \textsc{ncl}1-rescuer \textsc{ncl}1-other \textsc{subj}.\textsc{ncl}1-\textsc{narr}-say\\
	\glt \lq{}{\lq\lq}There are still about 42 corpses in the ditch and we don’t really know if we will be able to get them out", another rescuer said.\rq{ }(Helsinki Corpus of Swahili 2.0)

	\ex\label{exAppendixSwahiliScalar2}	
	\gll Simba i-me-sajili wa-chezaji 29 m-simu huu, hivyo 	\textbf{bado} i-na nafasi moja y-a ku-jaza.\\
	S.(\textsc{ncl}9) \textsc{subj}.\textsc{ncl}9-\textsc{ant}-register \textsc{ncl}2-player 29 \textsc{ncl}3-season \textsc{prox}.\textsc{ncl}9 so still \textsc{subj}.\textsc{ncl}9-\textsc{com} space(\textsc{ncl}9) one.\textsc{ncl}9 \textsc{ncl}9-assoc \textsc{ncl}15(\textsc{inf})-fill\\
	\glt \lq Simba [football team] has registered 29 players this season, so they still have one spot to fill.\rq{ }(Helsinki Corpus of Swahili 2.0, glosses added)\
	
	\ex\label{exAppendixSwahiliScalar3}
	Context: Over the course of the next weeks, the university was supposed to send us five books.\\
	\begin{xlist}
	\exi{A:}{I have gotten four so far.}
	\exi{B:}{Me, too, I have four.}
	\exi{C:}\gll Mimi \textbf{bado} ni na-vyo vi-tatu.\\
	1\textsc{sg} still \textsc{cop} \textsc{com}-\textsc{dem}.\textsc{ncl}8 \textsc{ncl}8-three\\
	\glt \lq I (still) have only three.' (Ponsiano Kanijo, p.c.)
	\end{xlist}

	\ex\label{exAppendixSwahiliScalar4}
	\gll Thomas a-li-sikia k-itanda ki-na-lia: A-li-ona kwamba zi-liku-wa bado saa tano.\\
	 T. \textsc{subj}.\textsc{ncl}1-\textsc{pst}-hear \textsc{ncl}7-bed \textsc{subj}.\textsc{ncl}7-\textsc{prs}-cry \textsc{subj}.\textsc{ncl}1-\textsc{pst}-see \textsc{comp} \textsc{subj}.\textsc{ncl}10-\textsc{pst}-\textsc{cop} still hour(\textsc{ncl}10) five\\
	 \glt \lq Thomas heard the bed creak. He saw that it was still (only) eleven o'clock.\rq{ }(Helsinki Corpus of Swahili 2.0, glosses added)\footnote{\textit{saa tano} is literally \lq five hours\rq{}, which equals 11 o'clock Swahili time.} 
\end{exe}

\subsubsection{Uses related toother phasal polarity concepts}
\paragraph{Not yet}\label{appendixSwahiliNotYet}
\begin{itemize}
	\item \textcite[85]{Sacleux19391941}, \textcite{TUKI2014} and \textcite{VeselinovaDevos2021}.
	\item This function obtains in the following context:
	\begin{itemize}
		\item In the absence of an overt predicate. This includes questions following a pattern \lq \textit{p} or still\rq{} > \lq{}\textit{p} or not yet\rq{ }(\ref{exAppendixswahiliNotYet1}, \ref{exAppendixswahiliNotYet2}), negative replies to polar questions (\ref{exAppendixswahiliNotYet3}), and uses as a pro-predicate in contrastive contexts (\ref{exAppendixswahiliNotYet4}).
		\item With an infinitival complement; see (\ref{exAppendixswahiliNotYet5}). 
	\end{itemize}
\end{itemize}

\begin{exe}
	\ex\label{exAppendixswahiliNotYet1}
	\gll I-me-rudi ku-tengenez-wa au \textbf{bado}?\\
	\textsc{subj}.\textsc{ncl}9-\textsc{ant}-return \textsc{ncl}15(\textsc{inf})-repair-\textsc{pass} or still\\
	\glt \lq Has it (fan) be fixed again or not yet?\rq{ }(\cite[392]{Ashton1947}, glosses added)
	
	\ex\label{exAppendixswahiliNotYet2}
	\gll Je, Serikali i-me-u-acha m-radi huo au \textbf{bado}?\\
	\textsc{q} government.\textsc{ncl}9 \textsc{subj}.\textsc{ncl}9-\textsc{ant}-\textsc{obj}.\textsc{ncl}3-leave \textsc{ncl}3-project \textsc{dem}.\textsc{ncl}9 or still\\
	\glt \lq Has the government abandoned the project or not yet?ʼ
	\\(Helsinki Corpus of Swahili 2.0)

	\ex\label{exAppendixswahiliNotYet3}
	\gll A-me-kuja? – \textbf{Bado}.\\
	\textsc{subj}.\textsc{ncl}1-\textsc{ant}-come {} still\\
	\glt \lq Ist er gekommen? -- Noch nicht. [Has he come? -- Not yet.]ʼ (\cite[15]{Schadeberg1990}, glosses added)
	
	\ex\label{exAppendixswahiliNotYet4}
	\gll Kwa sasa ma-ji ya-me-toka m-to-ni (Malulumo) na ku-fika Mgera lakini vi-jiji v-ingine \textbf{bado}.\\
	for now \textsc{ncl}6-water \textsc{subj}.\textsc{ncl}6-\textsc{ant}-leave \textsc{ncl3}-river-\textsc{loc} \phantom{(}M. and \textsc{ncl}15(\textsc{inf})-arrive M. but \textsc{ncl}8-village \textsc{ncl}8-other still\\
	\glt \lq As for now, the water has come from the river (Malulumo) and reached Mgera, but no other villages yet.’
	\\(Helsinki Corpus of Swahili 2.0)
	
	\ex\label{exAppendixswahiliNotYet5}
	\gll Bwana \textbf{bado} ku-ja.\\
	master still \textsc{ncl}15(\textsc{inf})-come\\
	\glt \lq Le maître n’est pas encore venu. [The master hasn't come yet.]' (\cite[85]{Sacleux19391941}, glosses added)
\end{exe}

\subsubsection{Marginality}
\label{appendixSwahiliMarginal}
\begin{itemize}
	\item \textit{Bado} has marginality readings.
\end{itemize}

\begin{exe}
	\ex\label{exAppendixSwahiliMarginal1}
	Context: Talking about tennis skills.\\
	\gll Paul, \textbf{bado} na-weza ku-m-piga (lakini Mark ni m-zuri sana kw-angu).\\
	P. still \textsc{subj}.1\textsc{sg}.\textsc{prs}-can \textsc{ncl}15(\textsc{inf})-\textsc{obj}.\textsc{ncl}1-beat \phantom{(}but M. \textsc{cop} \textsc{ncl}1-good very \textsc{loc}-\textsc{poss}.1\textsc{sg}\\
	\glt \lq Paul I can still beat (but Mark is too good for me).'
	\\(Ponsiano Kanijo, p.c.)
	
	\ex\label{exAppendixSwahiliMarginal2}
	Context: Iʼm really annoyed. My aunt has left the biggest part of her wealth to a welfare organisation and only 100,000 tsh to me.\\
	\gll 100,000 \textbf{bado} ni ki-asi ki-kubwa.\\
	100,000 still \textsc{cop} \textsc{ncl}7-ammount \textsc{ncl}7-big\\
	\glt \lq 100,000 tsh is still a huge sum.' (Ponsiano Kanijo, p.c.)
	
	\ex\label{exAppendixSwahiliMarginal3}
	Context: Discussing plans by the ruling party CCM to use the gas produced in the southern border town of Mtwara in a power plant in more central Bagamoyo.\\
	\gll Sasa ni kwa nini gesi hi-i i-si-tumik-e ku-u-tekeleza m-pango huo w-a CCM kwa ku-i-acha gesi hiyo huko$\sim$huko Mtwara na ya-le ya-na-yo-kusid-iwa ku-fany-ika huko Bagamoyo ya-fany-ik-e huko Mtwara kwa vi-le hata Mtwara \textbf{bado} ni Tanzania!?\\
	now \textsc{cop} for what gas.\textsc{ncl}9 \textsc{prox}-\textsc{ncl}9 \textsc{subj}.\textsc{ncl}9-\textsc{neg}-be\_used-\textsc{sbjv} \textsc{ncl}15(\textsc{inf})-\textsc{obj}.\textsc{ncl}3-abandon \textsc{ncl}3-plan \textsc{dem}.\textsc{ncl}3 \textsc{ncl}3-\textsc{assoc} CCM for \textsc{ncl}15(\textsc{inf})-\textsc{obj}.\textsc{ncl}9-leave gas.\textsc{ncl}9 \textsc{dem}.\textsc{ncl}9 \textsc{redupl}$\sim$\textsc{dem}.\textsc{loc} M. \textsc{com} \textsc{ncl}6-\textsc{dist} \textsc{subj}.\textsc{ncl}6-\textsc{prs}-\textsc{rel}.\textsc{ncl}6-plan-\textsc{pass} \textsc{ncl}15-do-\textsc{acaus} \textsc{dem}.\textsc{loc} B. \textsc{subj}.\textsc{ncl}6-do-\textsc{acaus}-\textsc{sbjv} \textsc{dem}.\textsc{loc} M. kwa \textsc{dist}-\textsc{ncl}8 even M. still \textsc{cop} T.\\
	\glt \lq Now, why should this gas not be used abandoning the CCM's plan and leaving it in Mtwara, so that what is planned to be done in Bagamoyo would be done in Mtwara, because even Mtwara is still Tanzania!?' (found online, glosses added)\footnote{\url{https://noordinjella.livejournal.com/38982.html} (27 February 2021).}
\end{exe}

\subsubsection{Additive and related functions}
\paragraph{Additive}\label{appendixSwahiliAdditive}
\begin{itemize}
	\item \textcite[85]{Sacleux19391941} and \textcite{Schadeberg1990}
	\item This appears to be more emphatic than \textit{na pia} \lq and also' (\ref{exAppendixSwahiliAdditive1}, \ref{exAppendixSwahiliAdditive2}), and to combinations with the root \textit{ingine} \lq (an)other', where it also provides emphasis  (\lq yet another, yet more'), as in (\ref{exAppendixSwahiliAdditive3}).
\end{itemize}
\begin{exe}
	\ex\label{exAppendixSwahiliAdditive1}
	\gll Fulani ni mw-ongo \textbf{na} \textbf{bado} ni mw-izi.\\
	so\_and\_so.\textsc{ncl}1 \textsc{cop} \textsc{ncl}1-liar and still \textsc{cop} \textsc{ncl}1-thief\\
	\glt \lq Un tel est menteur, et de plus il est voleur. [So-and-so is a liar, and on top of that s/he's a thief.]’ (\cite[85]{Sacleux19391941}, glosses added)
	
	\ex\label{exAppendixSwahiliAdditive2}
	\gll Ni-li-nunua ma-chungwa, ndizi, ma-embe, \textbf{na} \textbf{bado} nyama kidogo.\\
	\textsc{subj}.1\textsc{sg}-\textsc{pst}-buy \textsc{ncl}6-orange banana.\textsc{ncl}10 \textsc{ncl}6-mango \textsc{com} still meat.\textsc{ncl}9 a\_little\\
	\glt \lq I bought oranges, bananas, mangos, and even some meat.'
	\\(Ponsiano Kanijo, p.c.)

	\ex\label{exAppendixSwahiliAdditive3}
	\gll A-me-kuja m-tu mw-ingine \textbf{bado}.\\
	\textsc{subj}.\textsc{ncl}1-\textsc{ant}-come \textsc{ncl}1-person \textsc{ncl}1-other still\\
	\glt \lq Il est venu une autre personne encore. [Yet another person has come.]' (\cite[85]{Sacleux19391941}, glosses added)
\end{exe}



\subsubsection{Broadly modal and interactional functions}
\paragraph{Concessive apodoses}
\label{appendixSwahiliConcessiveConsequent}
\begin{itemize}
	\sloppy
	\item \textit{Bado} has concessive uses, a function that typically obtains together with commitative \textit{na}, i.e. \lq and still\rq{}, as in (\ref{exAppendixSwahiliConcessive1}, \ref{exAppendixSwahiliConcessive2}) 
	\item That this function is not a mere contextual inference of the phasal poalrity function is especially clear in cases like (\ref{exAppendixSwahiliConcessive3}), in which \textit{bado} is combined with a \lq\lq narrative" or \lq\lq consecutive" form of a telic predicate, giving a perfective and sequential reading.
	\item Impressionistically, this use is more common in formal language, where calquing from English might play a role.
\end{itemize}

\begin{exe}
	\ex\label{exAppendixSwahiliConcessive1}
	\gll Ni n-zuri na tamu kuliko u-na-vyo-weza ku-fikiria na \textbf{bado} bei y-ake ni nafuu sana.\\
	\textsc{cop} \textsc{ncl}9-good \textsc{com} tasty than \textsc{subj}.2\textsc{sg}-\textsc{prs}-\textsc{rel}.\textsc{ncl}8-can \textsc{ncl}15(\textsc{inf})-think \textsc{com} still price.\textsc{ncl}9 \textsc{ncl}9-\textsc{poss}.\textsc{sg} \textsc{cop} cheap very\\
	\glt \lq It is better and tastier than what you can imagine and yet the price is very cheap.’ (Helsinki Corpus of Swahili 2.0)
	
	\ex\label{exAppendixSwahiliConcessive2} 
	\textit{Jadili hoja isemayo kuwa vina na mizani siyo uti wala roho ya ushairi wa Kiswahili kwani katika Kichomi kanuni hizi hazikufuatwa}\\ 
	 \lq Discuss the argument that rhymes and metres are not the backbone nor the soul of Kiswahili poetry, because in the Kichomi [a book of poetry] these conventions were not followed'
	\exi{}
	\gll na \textbf{bado} ma-shairi y-a diwani hi-i ni ma-zuri sana.\\
	\textsc{com} still \textsc{ncl}6-poem \textsc{ncl}6-\textsc{assoc} anthology.\textsc{ncl}9 \textsc{prox}-\textsc{ncl}9 \textsc{cop} \textsc{ncl}6-good very\\
	\glt \lq and yet the poems in this anthology are very good.'
	\\(Helsinki Corpus of Swahili 2.0)
	
	\ex\label{exAppendixSwahiliConcessive3}	
	\gll M-heshimiwa Naibu Spika, mnamo mw-ezi w-a saba, Serikali i-li-ni-jibu ku-pit-ia hoja hi-i lakini \textbf{bado} i-ka-elekeza kwa Mw-anasheria M-kuu w-a Serikali amba-ye ndi-ye a-li-tak-iwa ku-jibu.\\
	\textsc{ncl}-honourable representative speaker approximately \textsc{ncl}3-month \textsc{ncl}3-\textsc{assoc} seven government.\textsc{ncl}9 \textsc{subj}.\textsc{ncl}9-\textsc{pst}-\textsc{obj}.1\textsc{sg}-reply \textsc{ncl}15(\textsc{inf})-pass-\textsc{appl} affair.\textsc{ncl}9 \textsc{prox}-\textsc{ncl}9 but still \textsc{subj}.\textsc{ncl}9-\textsc{narr}-direct for \textsc{ncl}1-attorney \textsc{ncl}1-big \textsc{ncl}1-\textsc{assoc} government.\textsc{ncl}9 \textsc{comp}-\textsc{ncl}1 \textsc{foc}-\textsc{ncl}1 \textsc{subj}.\textsc{ncl}1-\textsc{pst}-want-\textsc{pass} \textsc{ncl}15(\textsc{inf})-reply\\
	\glt \lq Honourable deputy speaker, around July the government replied to me about this affair but still directed it to the Attorney General, who was required to respond.' (Helsinki Corpus of Swahili 2.0)
\end{exe}
\il{Swahili|)}



\section{Tashelhyit (shi, tach1250)}\il{Tashelhyit|(}
\label{appendixTashelhyit}
Apart from descriptive materials referenced below, I searched \citeauthor{Stroomer2001}'s (\citeyear{Stroomer2001}, \citeyear{Stroomer2002}) text collections. I am furthermore indebted to Rachid Ridouane for discussing Tashelhyit data with me and for providing additional examples, as well as to Axel Fanego for helping with some tricky glosses.

\subsection{sul}
\subsubsection{General information}
\begin{itemize}
	\item Wordhood: free morpheme.
	\item Syntax: fixed, pre-predicate position.
\end{itemize}


\subsubsection{As a  \lq still\rq{ }expression}
\begin{itemize}
	\item \textcite{Fanego2021}.
	\item Specialisation: \citeauthor{Fanego2021}'s description meets my definition. Further, albeit indirect, evidence for its specialisation comes from its use as \textsc{not yet} (\appref{appendixTashelhyitNotYet}).
	\item Pragmaticity: compatible with both scenarios (tentative conclusion).
	\item Polarity sensitivity: combination with negation yields the corresponding negative concepts. With imperfective aspect, both \textsc{not yet} and \textsc{no longer} are possible outcomes, with the respective scope of the negation being determined by word order. With perfective verbs, only \textsc{not yet} is available; the latter also be expressed by negation plus \textit{ta} \lq yetʼ, or all three items (\textsc{neg} \textit{ta} \textit{sul}) combined.
	\item Further notes: the perfective inflection has an ongoing state reading with many verbs, allowing for a reading of a persistent state, as in (\ref{exAppendixTashelhyit3}).
\end{itemize}
\begin{exe}
	\ex Context: The opening of a tale.\\
	\gll Ikkattinn yat twal, lliγ a \textbf{sul} i-sawal ddunit, munn kra n lawḥaš:\\
	there\_once\_was one time when \textsc{irr} still 3\textsc{sg}.\textsc{m}-speak.\textsc{aor} world be\_together.\textsc{pfv}-3\textsc{pl}.\textsc{m} three of animals\\
	\glt \lq Once upon a time, when (all creatures in) the world (were [still] able to) speak, some animals came together:' (\cite[244–245]{Stroomer2002}, glosses added)
	
	\ex Context: A man has found a speaking tortoise. He has brought it to the king as a present, but the tortoise did not speak again. The king says: \label{exAppendixTashelhyit2}\\
	\gll Awi-yat t; bbi-yat ixf nns; an nit \textbf{sul} ili-γ, ar gig-i i-ṭṭnaẓ!\\
	take\_away.\textsc{imp}-2\textsc{pl}.\textsc{m} 3\textsc{sg}.\textsc{acc}.\textsc{m} cut.\textsc{imp}-2\textsc{pl}.\textsc{m} head \textsc{poss}.3\textsc{sg} \textsc{comp} indeed still \textsc{exist}.\textsc{aor}-1\textsc{sg} \textsc{irr} in-1\textsc{sg} 3\textsc{sg}.\textsc{m}-mock.\textsc{ipfv}\\
	\glt \lq Take him away; cut off his head; he wants to mock me, while I am still alive!' (\cite[226–227]{Stroomer2002}, glosses added)

	\ex	Context: Times are desperate. One friend hopes to find a job that puts food on his plate, no matter what the salary is. The other friend: \label{exAppendixTashelhyit3}
	\gll Max \textbf{sul} n-uḥl γ tγrad?\\
	why still 1\textsc{pl}-bother.\textsc{pfv} at salaries\\
	\glt \lq Why [still] bother about salaries?\rq{ }(\cite[164–165]{Stroomer2001}, glosses added)
	
	\ex Context: A man is flying high above the world on the back of a jinnee.\\
	\gll Fk	 yyi 				\textbf{sul}		tifiyyi,		ran	n-zayd.\\
	give.\textsc{imp} 1\textsc{sg}.\textsc{acc} still meat \textsc{fut} 1\textsc{pl}-leave.\textsc{aor}\\
	\glt \lq Give me (a piece of meat), so that we can continue flying.\rq{ }(\cite[116]{Stroomer2002}, glosses added).	
\end{exe}\largerpage[2]

\subsubsection{Uses on the fringes of \lq still\rq{}}
\paragraph{Scalar contexts}
\label{appendixTashelhyitScalar}
\begin{itemize}
	\item \textcite{Fanego2021}.
	\item \textit{Sul} is compatible with scalar contexts, both in the form of (potential) decreases (\ref{exAppendixTashelhyitScalar1}) and limited increases (\ref{exAppendixTashelhyitScalar2}). Note that (\ref{exAppendixTashelhyitScalar2}) does not feature an additional \lq only\rq{ }marker.
\end{itemize}
\begin{exe}

	\ex\label{exAppendixTashelhyitScalar1}
	\gll Llant \textbf{sul} dar-s  sddis n-tfunays, ur i-žli ḥtta yat.\\
	\textsc{cop}.\textsc{pfv}:3\textsc{pl}.\textsc{f} still at-3\textsc{sg} six of-cow \textsc{neg} 3\textsc{sg}.\textsc{m}-lose:\textsc{neg}.\textsc{pfv} even one\\
	\glt \lq He still has six cows; he didnʼt lose any.ʼ \parencite[342]{Fanego2021} 

	\ex\label{exAppendixTashelhyitScalar2}
	\gll Llan \textbf{sul} dar-s sddis n-lktub; ur ta y-ufi ḥtta yan.\\
	\textsc{cop}.\textsc{pfv}:3\textsc{sg}.\textsc{m} still at-3\textsc{sg} six of-book \textsc{neg} yet 3\textsc{sg}.\textsc{m}-find:\textsc{neg}.\textsc{pfv} even one\\
	\glt \lq He still has six books (only); he didn’t get any others.' \parencite[343]{Fanego2021}
\end{exe}

\subsubsection{Uses related toother phasal polarity-concepts}
\paragraph{Not yet}\label{appendixTashelhyitNotYet}
\begin{itemize}
	\item \textcite{Fanego2021}.
	\item This function obtains as a single-word response to a polar question. It also appears to be possible in disjunctive questions (\ref{exAppendixTashelhyitNotYet2}), but \textit{ur ta} is preferred here (Rachid Ridouane, p.c.).
\end{itemize}

\begin{exe}
	\ex\label{exAppendixTashelhyitNotYet1}
	\gll T-šši-t yad lfḍur-nnek? – \textbf{Sul}.\\
	2\textsc{sg}-eat.\textsc{pfv}-2\textsc{sg} already lunch-\textsc{poss}.2\textsc{sg} {} still\\
	\glt \lq Did you eat your lunch already? -- Not yet.' \parencite[342]{Fanego2021}
	\ex\label{exAppendixTashelhyitNotYet2}
	\gll Is-d y-uška Mark, nɣdd \textbf{sul} / ur ta?\\
	\textsc{q}-\textsc{ven} 3\textsc{sg}.\textsc{m}-arrive.\textsc{pfv} M. or still {} \textsc{neg} yet\\
	\glt \lq Has Mark arrived yet?' (Rachid, Ridouane, p.c.) 
\end{exe}


\subsubsection{Broadly adverbial temporal-aspectual functions}
\paragraph{Prospective \lq eventually\rq{}}\label{appendixTashelhyitProspective}
\begin{itemize}
	\sloppy
	\item Two tokens in the data in refer to situations that are yet to happen. While (\ref{exAppendixTashelhyitProspective1}) features the future tense construction,  (\ref{exAppendixTashelhyitProspective2}) features the irrealis\slash subjunctive \lq that he be useful\rq{}.
\end{itemize}

\begin{exe}
	\ex\label{exAppendixTashelhyitProspective1}
	Context: A horseman has encountered a greyhound, who is asking him for food.\\ 
	\gll Hati ras \textbf{sul} fll-i	t-lkm-t!\\
	\textsc{prestt} \textsc{fut} still on-1\textsc{sg} 2\textsc{sg}-arrive.\textsc{aor}-2\textsc{sg}\\
	\glt \lq Look, you will need me [yet].'\footnote{\lq Arrive on' is figurative for \lq need' (Rachid Ridouane, p.c.)}  (\cite[66–67]{Stroomer2002}, glosses added)

	\ex\label{exAppendixTashelhyitProspective2}
	Context: A king wants to kill several people. His vizir begs him to spare them, mentioning that killing them might provoke God’s wrath.\\
	\gll Ajj tn, hati-nn i-ra	ak k \textbf{sul} i-nfεa!\\
	let 3\textsc{pl}.\textsc{acc}.\textsc{m} \textsc{prestt}-\textsc{it} 3\textsc{sg}.\textsc{m}-want.\textsc{pfv} \textsc{irr} 2\textsc{sg}.\textsc{dat} still 3\textsc{sg}.\textsc{m}-be\_useful.\textsc{aor}\\
	\glt Original translation: \lq Release them, it is more useful to you!'\\
	More precise translation: \lq Release them, it'll be useful later.\rq{ }(\cite[214–215]{Stroomer2002}, glosses added)
\end{exe}

\subsubsection{Marginality}
\label{appendixTashelhyitMarginal}
\begin{itemize}
	\item \textit{Sul} allows for readings marginality.
\end{itemize}
\begin{exe}
	\label{exAppendixTahselhyitMarginal1}
	\ex Context: Talking about skills in a sport.\\
	\gll Brahim ufi-gh ad-t \textbf{sul} nru-gh.\\
B. be\_sufficient.\textsc{pfv}-1\textsc{sg} \textsc{comp}-3\textsc{sg}.\textsc{m} still beat.\textsc{aor}-1\textsc{sg}\\
	\glt \lq Brahim I can still beat.' (Rachid Ridouane, p.c.)

	\ex Context: Talking about political views.\\
	\label{exAppendixTahselhyitMarginal2}
	\gll Brahim i-ga \textbf{sul} muɛtadil.\\
	B. 3\textsc{sg}.\textsc{m}-do.\textsc{pfv} still moderate\\
	\glt \lq Brahim is still moderate (e.g. as opposed to Mark, who is a radical).' (Rachid Ridouane, p.c.)
	
	\ex Context: Viewed from the perspective of Marrakesh.\\
	\label{exAppendixTahselhyitMarginal3}
	\gll Rrachidiya, lmghrib \textbf{sul} a i-ga.\\
	Errachidia Morocco still \textsc{comp} 3\textsc{sg}.\textsc{m}-do.\textsc{pfv}\\
	\glt \lq Errachidia is still Morrocco. (as opposed to e.g. Bechar, which is across the Algerian border)' (Rachid Ridouane, p.c.)
\end{exe}

\subsubsection{Additive and related functions}
\paragraph{Additive}\label{appendixTashelhyitAdditive}
\begin{itemize}
	\item \textit{Sul} is attested in an additive function.
\end{itemize}
\begin{exe}
	\ex
	\gll T-ṣṣifḍ a-s smmus idmya n ismg d smmus idmya n talyajurt, n wurγ d smmus idmya n talyajurt n nnqwrt, t-ṣṣifḍ a-s \textbf{sul} smmus idmya n twayya.\\
	3\textsc{sg}.\textsc{f}-send.\textsc{pfv} to-3\textsc{sg} five hundreds of slave and five hundreds of brick of gold and five hundreds of brick of silver 3\textsc{sg}.\textsc{f}-send.\textsc{pfv} to-3\textsc{sg} still five hundreds of female\_slaves\\
	\glt \lq She sent five hundred slaves to him and five hundred gold bricks, five hundreds silver bricks and five hundreds female slaves.\rq{} (\cite[130]{Stroomer2001}, glosses added) 

	\ex
	Context: The king has heard of a most beautiful girl.\\
	\gll Iγ tt-id	t-iwi-t, rad ak fk-γ lmal n ddunit, \textbf{sul} g-eγ-k d luzir.\\
	if 3\textsc{sg}.\textsc{acc}.\textsc{f}-\textsc{ven} 2\textsc{sg}-bring.\textsc{aor}-2\textsc{sg} \textsc{fut} 2\textsc{sg}.\textsc{dat}.\textsc{m} give.\textsc{aor}-1\textsc{sg} money of world still make.\textsc{aor}-1\textsc{sg}-2\textsc{sg}.\textsc{acc}.\textsc{m} as vizir\\
	\glt \lq If you bring her, I give you all the money in the world and appoint you vizir too.' (\cite[137]{Stroomer2002}, glosses added) 

	\ex
	Context: Two siblings have met a mythical, speaking bird. The bird has offered to help them.\\
	\gll Ṣafi rak kwn εawn-γ, \textbf{sul} zdm-γ did-un ikššuḍn.\\
	all\_right \textsc{fut} 2\textsc{pl}.\textsc{acc}.\textsc{m} help.\textsc{aor}-1\textsc{sg} still collect\_wood.\textsc{aor}-1\textsc{sg} with-2\textsc{pl}.\textsc{m} wood.\textsc{pl}\\
	\glt \lq All right, I will help you, I will fetch firewood for you (as well).' (\cite[200]{Stroomer2001}, glosses added)

	\ex\label{exAppendixTashelhyitAdditive4}
	Context: A man is flying high above the world on the back of a jinnee.\\
	\gll Fk					yyi 				\textbf{sul}		tifiyyi,		ran	n-zayd.\\
	give.\textsc{imp} 1\textsc{sg}.\textsc{acc} still meat \textsc{fut} 1\textsc{pl}-leave\\
	\glt \lq Give me (a[nother] piece) of meat, so that we can continue flying.\rq{ }(\cite[116]{Stroomer2002}, glosses added)
\end{exe}

\subsubsection{Broadly modal and interactional functions}
\paragraph{Concessive apodoses}
\label{appendixTashelhyitConcessiveConsequent}
\begin{itemize}
	\item \textit{Sul} can mark the apodoses of concessive constructions. That this is not merely a contextual inference becomes evident by the combination with the telic predicate plus bounded viewpoint in (\ref{exAppendixtashelhyitConcessiveFox}).
\end{itemize}

\begin{exe}
	\ex
	\gll Lɛuṭla ayad macc i-xṣṣa ad \textbf{sul} i-xdm.\\
	holiday this\_is but 3\textsc{sg}.\textsc{m}-need.\textsc{pfv} \textsc{irr} still 3\textsc{sg}.\textsc{m}-work.\textsc{aor}\\
	\glt \lq It's a holiday and he still has to work.' (Rachid Ridouane, p.c.)

	\ex\label{exAppendixtashelhyitConcessiveFox}
	Context: Fox has tricked a man into giving him food and then rushed off. The man, however, was holding Fox’s tail, which has torn off.\\
	\gll T-frḥ-t nit \textbf{sul} t-fl-t d abakku nnk γ ufus inu!\\
	2\textsc{sg}-be\_happy.\textsc{pfv}-2\textsc{sg} indeed still 2\textsc{sg}-leave.\textsc{pfv}-2\textsc{sg} \textsc{ven} tail \textsc{poss}.2\textsc{sg} at hand \textsc{poss}.1\textsc{sg}\\
	\glt \lq You may be happy now, but you left your tail in my hand!' (\cite[82]{Stroomer2001}, glosses added)
\end{exe}
\il{Tashelhyit|}
\section{Tima (tms, tima1241)}\label{appendixtima}
\il{Tima|(}
\subsection{Introductory remarks}
I am indebted to Gertrud Schneider-Blum for discussing Tima data with me, for supplying glosses, and for eliciting additional examples.

\subsection{bʌʌr}
\subsubsection{General information}
\begin{itemize}
	\item Form: there is a free variant \textit{bʌr}.
	\item Wordhood: intermediate: takes a singular/plural person index.
\end{itemize}


\subsubsection{As a  \lq still\rq{ }expression}
\begin{itemize}
	\item \textcite{SchneiderBlum2013}.
	\item Specialisation: in absence of more contextualised examples, case like (\ref{exAppendixTima1}–\ref{exAppendixTima4}) give a fairly good indication that this marker conforms to my definition. For instance, (\ref{exAppendixTima1}) \mbox{\textit{bʌʌr}} not only signals that, at topic time,  Jiddu continued to be small, but also alludes to the discontinuation of this state at utterance time. Similarly, in (\ref{exAppendixTima2}) \mbox{\textit{bʌʌr}} signals that the plant is eaten before it looses its green colour. Further, albeit indirect, evidence comes from its use as \textsc{not yet} in the absence of negation (\ref{appendixTimaNotYet}).
	\item Polarity sensitivity: inner negation yields \textsc{not yet}. In subordinate contexts this yields precedence (\lq when not yet \textit{p}, \textit{q}\rq{} \equiv{ } \lq{}before \textit{p}, \textit{q}\rq{}). Outer negation yields \textsc{no longer}.
	\item Pragmaticity: the available data allow no conclusion.
	\item Further note: \textit{bʌʌr} can be used as an elliptical, affirmative one-word answer, as in (\ref{exAppendixTima4}).
\end{itemize}
\begin{exe}
	\ex\label{exAppendixTima1}
	\gll A=yu=weeŋ i-di-yʌŋ-aa tɛtɛk, Jiddu	a-\textbf{bʌʌr} a-tɛʔɛŋ.\\
	\textsc{source}=\textsc{pl}=\textsc{dem} \textsc{pl}-walk-\textsc{ven}-\textsc{inst} first J. \textsc{stat}.\textsc{sg}-still \textsc{stat}.\textsc{sg}-small\\
	\glt \lq When we came here for the first time, Jiddu was still small.\rq{ }(Gertrud Schneider-Blum, p.c.)
	
	\ex\label{exAppendixTima2}
	Context: About using a certain plant.\\
	\gll ɘ-kʌluk	ɲɪ=hwaa	ŋ=kɨ-\textbf{bʌʌr} ŋ=kɪ-hɛh.\\
	\textsc{pst}-chew.\textsc{caus} \textsc{erg}=people \textsc{inst}-\textsc{sg}-still \textsc{inst}=\textsc{sg}.\textsc{stat}-light\\
	\glt \lq People ate it when it is still green.\rq{ }(Gertrud Schneider-Blum, p.c.)	
	
	\ex\label{exAppendixTima3}
	\gll Amɛɛ mɔɔk-aa ŋ=wɔrt̪ɘmaadɘh=na ɪ-yɛmpɛrɛ, a-\textbf{bʌʌr} meye a-bayʊk.\\
	if drink-\textsc{inst} \textsc{erg}=man=\textsc{prox} \textsc{pl}-medicine \textsc{stat}.\textsc{sg}-still \textsc{opt} \textsc{stat}.\textsc{sg}-alive\\
	\glt \lq If this man had taken his medicine, he would still be alive.\rq{ }(\cite[33]{SchneiderBlum2013}, glosses added)
	
	\ex\label{exAppendixTima4}
	\gll Wɛɛn=na ɘ=kaaka=yɪ,	 caa c-idʌ ŋ=kɨ-\textbf{bʌʌr} ŋ=kɪ-yaah=ya? – A-\textbf{bʌʌr}.\\
	mother=\textsc{prox} \textsc{dir}=K.=\textsc{foc} \textsc{part} \textsc{sg}-body \textsc{inst}=\textsc{sg}-still \textsc{inst}=\textsc{sg}-unripe=\textsc{loc}.3.\textsc{q} {} \textsc{stat}.\textsc{sg}-still\\
	\glt \lq{}The mother of Kaaka, is she still sick? -- (Yes), she still is.\rq{ }(Gertrud Schneider-Blum, p.c.)
\end{exe}

\subsubsection{Uses related toother phasal polarity-concepts}
\paragraph{Not yet}
\label{appendixTimaNotYet}
\begin{itemize}
	\item \textit{Bʌʌr} is attested as a negative one-word answer \lq not yet'.
	\item Note that, depending on context, this item can also serve as an elliptical positive one-word answer; see (\ref{exAppendixTima4}) above.
\end{itemize}
\begin{exe}
	\ex
	\gll Wɛɛn=na ɘ=kaaka=yɪ, c-idʌ an-caak=t̪aŋ a-mal? – A-\textbf{bʌʌr}.\\
	mother=\textsc{prox} \textsc{dir}=K.=\textsc{selective} \textsc{sg}-body 3\textsc{pl}.\textsc{ant}-become=\textsc{loc}.3\textsc{pl}.\textsc{q} \textsc{stat}.\textsc{sg}-good=\textsc{loc}.3.\textsc{q} {} \textsc{stat}.\textsc{sg}-still\\
	\glt \lq Has Kaaka's mother recovered? -- Not yet.\rq{}\footnote{See \textcite{BeckerSchneiderBlum2020} on the selective marker in Tima.} (Gertrud Schneider-Blum, p.c.)
	
	\ex
	\gll Caa-kʌlʌʌk=a=t̪aŋ=ŋa? – A-\textbf{bʌʌr}.\\
	2\textsc{sg}.\textsc{ant}-eat.\textsc{ap}=\textsc{source}=2\textsc{sg}.\textsc{q} {} \textsc{stat}.\textsc{sg}-still\\
	\glt \lq Have you eaten yet? -- Not yet.\rq{ }(Gertrud Schneider-Blum, p.c.)
\end{exe}

\subsubsection{Broadly adverbial temporal-aspectual uses}
\paragraph{Prospective \lq eventually\rq{}}\label{appendixTimaProspective}
\begin{itemize}
	\item \textit{Bʌʌr} seems to allow a prospective \lq eventually'-reading.
	\item All examples I have include additional markers like \textit{anakɔɔ} \lq later'.
\end{itemize}

\begin{exe}
	\ex Context: About restoring a house.\\
	\gll A-bʌʌr 	kʊ=dʊŋ-kɔyɔɔ=dɔ 	ŋʊɲaŋ 	bʌttin.\\
	\textsc{stat}.\textsc{sg}-still \textsc{pot}=\textsc{fut}.1\textsc{sg}-do=1\textsc{sg} work a\_little\_later\\
	\glt \lq I will do the work yet, a little later.\rq{ }(Gertrud Schneider-Blum, p.c.)
	
	\ex Context: The addressee doesn't believe that the speaker's claim about their pending success.\\
	\gll A-\textbf{bʌʌr} kaa=daa-mɪnɛnɛ=ŋaŋ anakɔɔ.\\
	\textsc{stat}.\textsc{sg}-still \textsc{pot}.2\textsc{sg}=\textsc{fut}.2\textsc{sg}-find:\textsc{plur}=2\textsc{sg} later\\
	\glt \lq You will see yet/later.\rq{ }(Gertrud Schneider-Blum, p.c.)
\end{exe}

\subsubsection{\# Marginality}
\label{appendixTimaMarginal}
\begin{itemize}
	\sloppy
	\item \textit{Bʌʌr} appears not to be compatible with readings of marginality. Thus in (\ref{appendixTimaMarginal}),  \textit{bʌʌr} can only have a phasal polarity interpretation, not a \lq\lq categorizing" one. Similarly, speakers would not use \textit{bʌʌr} to translate \lq I can still beat Musa' in (\ref{appendixTimaMarginal2}).
\end{itemize}
\begin{exe}
	\ex\label{appendixTimaMarginal1}
	\gll Bant̪uyu a-twaarɪ=na u-sudaan=ɪ-ya, kʌwudʌ a-\textbf{bʌʌr} a=nt̪ʌ-tiin=nʌ u=sudaan=ɪ.\\
	Banthuyu \textsc{source}=outside=\textsc{prox} \textsc{dir}=Sudan=\textsc{selective}=\textsc{foc} Kawda \textsc{stat}.\textsc{sg}-still \textsc{source}=inside=\textsc{prox} \textsc{dir}=Sudan=\textsc{selective}\\
	\glt \lq Banthuyu is outside of Sudan, Kawda is still inside Sudan (but they want to be separated).'\\
	not: \lq Banthuyu is still (within the borders of) Sudan.\rq{ }(Gertrud Schneider-Blum, p.c.)

	\ex\label{appendixTimaMarginal2}
	Context: About football skills. Musa falls within the range of players I can beat, but Ali is better than me.\\
	\gll Cen-taak=da musa, ədaala kɪ=rɛɛy=da mak na l-alɪ=yaŋ.\\
	1\textsc{sg}.\textsc{ipfv}-surrpass.\textsc{ap}=1\textsc{sg} M. play \textsc{pot}.1\textsc{sg}=equal=1\textsc{sg} then with \textsc{loc}-Ali=\textsc{neg}\\
	\glt \lq I surpass Musa when playing, but I don't equal Ali.\rq{ }(Gertrud Schneider-Blum, p.c.)
\end{exe}
\il{Tima|)}

\section{Tunisian Arabic (aeb, tuni1259)}\il{Arabic, Tunisian|(} 
\subsection{Introductory remarks}
\begin{sloppypar}
Apart from descriptive materials, I consulted the text collections by \citeauthor{RittBenmimoun2005Brot} (\citeyear*{RittBenmimoun2005Brot}, \citeyear*{RittBenmimoun2005Wiener}, \citeyear{RittBenmimoun20062010}, \citeyear{RittBenmimoun2007}, \citeyear*{RittBenmimoun2011Spiele}, \citeyear*{RittBenmimoun2011}), as well as the TuniCo \parencite{TuniCo}. Note that Tunisian Arabic makes use of the non-concatenative \lq\lq root-and-pattern\rq{ }morphology that is characteristic of Semitic \parencite{PatEl2019}, hence the mentioning of discontinuous roots at several points throughout this appendix. Tunisian Arabic has two candidates for \textsc{still} expressions: \textit{māzāl} and \textit{bāqi}. While \textit{māzāl} clearly conforms to my definition, \textit{bāqi} (which is only found in northern sedentary varieties) is a borderline case, perhaps better understood as a marker of stasis.
\end{sloppypar}

\subsection{māzāl}
\label{appendixTunisianMazal}
\subsubsection{General information}
\begin{itemize}
	\sloppy
	\item Form: subject to certain changes in vowel quality and length, primarily determined by the affixes attached.
	\item Wordhood: intermediate, a bound root taking person, number, and gender (where applicable) indexes. In some varieties it is also attested, though less commonly so, as an invariable word (formally 3\textsc{sg}.\textsc{m}).
	\item Etymology: < \textit{mā}-\textit{zāl} \lq \textsc{neg}-cease.\textsc{pfv}\rq{ }\lq{}.
	\item Syntax: either in pre-predicate or in clause-final position.
\end{itemize}


\subsubsection{As a  \lq still\rq{ }expression}
\begin{itemize}
	\item \textcite[221–222]{Amor1990}, \textcite[259]{Boris1958}, \textcite{FischerEtAlTunisian}, \textcite[1735–1742]{MarcaisGuiga19581961} and \textcite[63]{Mion2013}; passing mentions in \textcite[113–114]{Jemni2011}, \textcite[160–161]{Talmoudi1980}.
	\item Specialisation: the description by \citeauthor{FischerEtAlTunisian} meets my definition and addresses both the prior time presupposition and the incompatibility with inalterable situations.
	\item Polarity sensitivity: inner negation yields \textsc{not yet}; in the Takrouna variety of \textcite{MarcaisGuiga19581961} the external negation to \textsc{no longer} is also attested.
	\item Pragmaticity: compatible with both scenarios. To give prominence to the unexpectedly late scenario, \textit{māzāl} can receive nuclear stress.
	\item Further note: example (\ref{exAppendixTunisian3}) illustrates this item with a dynamic verb in an imperative. This is attested only in older doculects; modern varieties would make use of the verb \mbox{\textit{z}-\textit{y}-\textit{d}} \lq continue\rq{}.

\end{itemize}
\begin{exe}
	\ex Context: About professional politicians who know how to play the system.\\
	\gll T-aṛqā-hum	milli	\textbf{māzāl}-u	 f-il-maktib y-aʕṛf-u d-dustūṛ	 faṣᵊl	faṣᵊl.\\
	2\textsc{sg}-find.\textsc{ipfv}-3\textsc{pl} since still-3\textsc{pl} in-\textsc{def}-primary\_school 3\textsc{pl}-know.\textsc{ipfv}-3\textsc{pl} \textsc{def}-constitution segment segment.\\
	\glt \lq Starting from when they are still in primary school, they know the constitution by heart, paragraph by paragraph.\rq{ }(TuniCo, cited in \cite{FischerEtAlTunisian})
	
	\ex Context: Complaining about low quality bread.\\
	\gll T-ākul tlāṯa	kaʕb-āt …  <f-iṣ-…> fī-fṭūṛ	iṣ-ṣbāḥ w-\textbf{māzil}-t žīʕān-a.\\
	2\textsc{sg}-eat.\textsc{ipfv} three piece-\textsc{pl} {} {} in-meal.\textsc{cs} \textsc{def}-morning and-still-2\textsc{sg} hungry-\textsc{sg}.\textsc{f}\\
	\glt \lq You eat three pieces…for breakfast and you’re still hungry.\rq{ }(TuniCo, cited in \cite{FischerEtAlTunisian})
	\ex
	\begin{xlist}
		\exi{A:}
		\textit{Āhayya, nimshīw.}\\
		\lq Shall we go?\rq
		\exi{B:}
		\gll \textbf{Māzāl} bikri.\\
		still early\\
		\glt \lq It's still early.\rq{ }(\cite[221]{Amor1990}, glosses added)
	\end{xlist}
	
	\ex\label{exAppendixTunisian3}
	\gll \textbf{Māzāl}	imši	ṭūl		ḥatta	l-il-bīr.\\
	still go.\textsc{imp} 	straight	until	to-\textsc{def}-well\\
	\glt \lq Continue, marche tout droit jusqu’au puits. [Continue, walk straight to the well.]\rq{ }(\cite[1737]{MarcaisGuiga19581961}, glosses by \cite{FischerEtAlTunisian})	
\end{exe}


\subsubsection{Uses on the fringes of \lq{}still\rq}
\paragraph{Scalar contexts}
\label{appendixTunisianMazalScalar}
\begin{itemize}
	\item \textcite{FischerEtAlTunisian}.
	\item \textit{Māzāl} is attested in scalar contexts. These include decreases over time, as in (\ref{appendixTunisianMazalScalar1}, \ref{appendixTunisianMazalScalar2}), as well as limited increases, as in (\ref{appendixTunisianMazalScalar3}). Note how the increase context  in (\ref{appendixTunisianMazalScalar3}) does not involve an overt \lq only\rq{ }marker.
\end{itemize}

\begin{exe}

	\ex\label{appendixTunisianMazalScalar1}
	\gll \textbf{Māzāl} yūm-ēn  l-l-ʕīd.\\
	still day-\textsc{du}  to-\textsc{def}-feast\\
	\glt \lq Il reste encore deux jours avant la fête. [Two days remain before the feast.]\rq{ }	(\cite[1736]{MarcaisGuiga19581961}, glosses by \cite{FischerEtAlTunisian})	
	
	\ex\label{appendixTunisianMazalScalar2}
	\gll \textbf{Māzāl}-l-i nṣayyib niʕma.\\
	still-to-1\textsc{sg} share.\textsc{dim}.\textsc{cs} food/cereal\\
	\glt \lq Il me reste encore un peu de céréales. [I still have a bit of cereal left.]\rq{ }(\cite[1736]{MarcaisGuiga19581961}, glosses added)

	\ex\label{appendixTunisianMazalScalar3}
	Context: The late father of two orphans has bequeathed a wall to them, under which there is a hidden treasure.\\
	\gll W-xallā-hū-l-hum w-humma māzāl-u mā-bilġ-ū-š	\textbf{māzāl} ʕumᵊṛ-hum tisʕ ᵊsnīn.\\
	and-leave\_behind.\textsc{pfv}.3\textsc{sg}.\textsc{m}-3\textsc{sg}.\textsc{m}-to-3\textsc{pl}.\textsc{m} and-3\textsc{pl}.\textsc{m} still-3\textsc{pl}.\textsc{m} \textsc{neg}-reach\_puberty.\textsc{pfv}-3\textsc{pl}.\textsc{m}-\textsc{neg} still age-\textsc{poss}.3\textsc{pl}.\textsc{m} nine year.\textsc{pl}\\
	\glt \lq Er hinterließ sie ihnen, aber sie sind noch nicht in der Pubertät, sie sind erst ungefähr neun Jahre alt. [He (father) bequeathed it (the wall) to them (the orphans), but they haven’t reached puberty yet, they are still only about nine years old.]\rq{ }(\cite[220–221]{RittBenmimoun2011}, glosses by \cite{FischerEtAlTunisian})
\end{exe}

\subsubsection{Uses relating toother phasal polarity concepts}
\paragraph{Not yet}\label{appendixTunisianNotYet}
\begin{itemize}
	\item \textcite{FischerEtAlTunisian}, \textcite[1736]{MarcaisGuiga19581961}, \textcite[203]{RittBenmimoun2014} and \textcite[650–651]{Singer1984}.
	\item This function obtains in the absence of an overt predicate (as first observed by \cite[1740]{MarcaisGuiga19581961}). More precisely, it is found
	\begin{itemize}
		\item As a negative response to a polar question (\ref{exAppendixTunisianNotYet1}).
		\item In disjunctive (\ref{exAppendixTunisianNotYet2}) and contrastive (\ref{exAppendixTunisianNotYet3}) contexts.
		\item With nominal subjects either denoting time spans, or  \textit{waqt} \lq time\rq{ }itself. A further complication here pertains to word order and definiteness. With a preceding subject NP, the resulting sentence is ambiguous (\ref{exAppendixTunisianNotYet4}). If the NP follows \textit{māzāl}, then an affirmative (\ref{exAppendixTunisianNotYet5}) or negative (\ref{exAppendixTunisianNotYet6}) reading is a function of definiteness.
	\end{itemize}
\end{itemize}

\begin{exe}
	\ex\label{exAppendixTunisianNotYet1}
	\gll Flān žā walla māzāl? – Lā, \textbf{māzāl}.\\
	so\_and\_so come.\textsc{pfv}.3\textsc{sg}.\textsc{m} or still {} no still\\
	\glt \lq Ist er schon da? Nein, (er ist) noch nicht (gekommen). [Has he come yet? No, not yet.]\rq{ }(\cite[650]{Singer1984}, cited in \cite{FischerEtAlTunisian})
	
	\ex\label{exAppendixTunisianNotYet2}
	\gll Il-ʕalālīš ǝḏ-ḏabbiḥ-an walla \textbf{māzāl}-an?\\
		\textsc{def}-lamb.\textsc{pl} \textsc{pass}-slaughter.\textsc{pfv}-3\textsc{pl}.\textsc{f} or still-3\textsc{pl}.\textsc{f}\\
		\glt \lq Have the lambs been slaughtered or not yet?\rq{ }(\cite[110–111]{RittBenmimoun2011}, glosses by \cite{FischerEtAlTunisian})		
		
	\ex \label{exAppendixTunisianNotYet3}
	\gll Minna šhaṛ y-ṭīb iš-šʕīṛ ām-il-qamḥ \textbf{māzāl}.\\
	from.here month 3\textsc{sg}.\textsc{m}-ripen/be\_ripe.\textsc{ipfv} \textsc{def}-barley but-\textsc{def}-wheat still\\
	\glt \lq D’ici un mois l’orge deviendra mûr, mais le blé ne le sera pas encore. [Within a month, the barley will be ripe, but the wheat won’t yet.]\rq{ }(\cite[1740]{MarcaisGuiga19581961}, glosses by \cite{FischerEtAlTunisian})

	\ex\label{exAppendixTunisianNotYet4}
	\gll Iṣ-ṣēf \textbf{māzāl}.\\
	\textsc{def}-summer still\\
	\glt i.\phantom{i} \lq Lʼ été dure encore. [It’s still summer.]\rq\\
	ii. \lq L’été n’est pas encore venu. [Summer hasn’t arrived yet.]\rq{ }(\cite[1741]{MarcaisGuiga19581961}, glosses by \cite{FischerEtAlTunisian})
		
	\ex\label{exAppendixTunisianNotYet5}
	\gll \textbf{Māzāl}-t aʕšīya.\\
	still-3\textsc{sg}.\textsc{f} afternoon/evening(\textsc{f})\\
	\glt \lq Cʼest encore le temps de après-midi. [It is still afternoon.]\rq{ }(\cite[1741]{MarcaisGuiga19581961}, glosses by \cite{FischerEtAlTunisian})
		
	\ex\label{exAppendixTunisianNotYet6}
	\gll \textbf{Māzāl}-t \textbf{il}-aʕšīya.\\
	still-3\textsc{sg}.\textsc{f} afternoon/evening(\textsc{f})\\
	\glt \lq Ce n’est pas encore le temps de après-midi. [It’s not afternoon yet.]\rq{ }(\cite[1741]{MarcaisGuiga19581961}, glosses by \cite{FischerEtAlTunisian})
\end{exe}

\pagebreak
\subsubsection{Broadly adverbial temporal-aspectual uses}
\paragraph{Near past}\label{appendixTunisianArabicNearPast}
\begin{itemize}
	\item \textcite[259]{Boris1958}, \textcite{FischerEtAlTunisian}, \textcite[1736]{MarcaisGuiga19581961}, \textcite[203]{RittBenmimoun2014} and \textcite[650–651]{Singer1984}.
	\item Form: in collocation with a clause introduced by \mbox{\textit{kī}(\textit{f})} \lq like, when\rq{ }that contains either a verb in the perfective aspect (\ref{exAppendixTunisianImmediatePast1}, \ref{exAppendixTunisianImmediatePast2}) or an active participle (\ref{exAppendixTunisianImmediatePast3}).
	\item As (\ref{exAppendixTunisianImmediatePast2}) shows, this is not a true past tense, but can also serve as a past-in-the-past.
\end{itemize}

\begin{exe}
	\ex\label{exAppendixTunisianImmediatePast1}
	\gll Znīxā uxt-i \textbf{māzāl-t} \textbf{kīf} \textbf{xd̠ā-t}.\\
	Z. sister-\textsc{poss}.1\textsc{sg} still-3\textsc{sg}.\textsc{f} when/how take\_spouse.\textsc{pfv}-3\textsc{sg}.\textsc{f}\\
	\glt \lq Meine Schwester Zulayxa hat eben erst geheiratet. [My sister Zulayxa just recently got married.]' (\cite[651]{Singer1984}, glosses added)
	
	\ex \label{exAppendixTunisianImmediatePast2}
	\gll Kun-t ānā \textbf{māzil}-\textbf{t} \textbf{kīf} \textbf{bdī}-\textbf{t} n-umġud̠̣ fī ṭaṛf il-lḥam had̠āya u-ẓaṛṣt-i ṛā-hi tnaṭṛ-it tanṭīṛa waḥda.\\
	\textsc{cop}.\textsc{pfv}-1\textsc{sg} 1\textsc{sg} still-1\textsc{sg} when begin.\textsc{pfv}-1\textsc{sg} 1\textsc{sg}-chew.\textsc{ipfv} in piece \textsc{def}-meat(\textsc{m}) \textsc{prox}.\textsc{sg}.\textsc{m} and-molar(\textsc{f})-\textsc{poss}.\textsc{sg} \textsc{prestt}-3\textsc{sg}.\textsc{f} slip\_out.\textsc{pfv}-3\textsc{sg}.\textsc{f} slip\_out.\textsc{nmlz} one.3\textsc{sg}.\textsc{f}\\
	\glt \lq Ich hatte eben erst damit begonnen, auf dem Stück Fleisch herumzubeißen, da flog auch schon mein (Backen-)Zahn im hohen Bogen. [I had only just begun chewing on the piece of meat, when all of a sudden my molar tooth came flying out.]\rq{ }(\cite[651]{Singer1984}; cited in \cite{FischerEtAlTunisian})	
	
	\ex \label{exAppendixTunisianImmediatePast3}
Context: A man has seen a mysterious woman out in the desert. He is informed that she is a ghost.\\
	\gll Magtōla b-ǝslāḥ maṛā ṣiġēra maʕnāha šābba maʕnāha \textbf{māzāl}-\textbf{at} \textbf{kīf} \textbf{ǝmʕarrsa} maʕnāha.\\
	kill:\textsc{ptcp}.\textsc{pass}:\textsc{sg}.\textsc{f} at-weapon woman(\textsc{f}) little.\textsc{sg}.\textsc{f} \textsc{dm} young.\textsc{sg}.\textsc{f}  \textsc{dm} still-3\textsc{sg}.\textsc{f} when marry.\textsc{ptcp}.\textsc{sg}.\textsc{f} \textsc{dm}\\
	\glt \lq Eine junge Frau war mit einer Waffe getötet worden, eine junge Frau, die gerade erst geheiratet hatte. [A young woman had been killed with a gun,a young woman  who had only recently got married.]\rq{ }(\cite[408–409]{RittBenmimoun2011}, cited in \cite{FischerEtAlTunisian})	
\end{exe}

\subsubsection{\# Marginality}\label{appendixTunisianMarginal}
\begin{itemize}
	\item \textcite{FischerEtAlTunisian}.
	\item \textit{Māzāl} does not allow for readings of marginality.
\end{itemize}

\begin{exe}
	\ex[]{Context: Speaking about tennis skills.}
	\exi{}[\#]{\gll Bṛāhīm	\{\textbf{māzāl}/māzil-t\} ᵊn-nažžim n-iġᵊlb-u. 	(amma	ʕAlī bāhi	baṛša) \\
	Ibrahim still/still-1\textsc{sg} \phantom{\{}1\textsc{sg}-can.\textsc{ipfv} 1\textsc{sg}-beat.\textsc{ipfv}-3\textsc{sg}.\textsc{m} \phantom{(}but Ali good a\_lot\\
	\glt (intended: \lq{} Ibrahim, I can still beat him [i.e. his skills are close to mine], but Ali is 	better than me\rq) \parencite{FischerEtAlTunisian}}
	
	\ex[\#]{\gll {Bin Gardān} 	\textbf{māzāl}-it 		fī-Tūnis,		amma	Zwāra ṛā-hi			fī-Lībya\\
	{B. G.}	still-3\textsc{sg}.\textsc{f}	in-Tunisia	but Z. \textsc{prestt}-\textsc{sg}.\textsc{f} in-Libya\\
	\glt (intended: \rq{} Ben Gardane is still in Tunisia [i.e. a marginal member of the Tunisian territory], but Zuwara is already in Libya.\rq{}) \parencite{FischerEtAlTunisian}}
\end{exe}

\subsubsection{Additive and related functions}
\paragraph{Additive}\label{appendixTunisianArabic}
\begin{itemize}
	\item \textcite{FischerEtAlTunisian}.
	\item \textit{Māzāl} is attested in additive function; note the perfective aspect in (\ref{exAppendixTunisianAdditive2}), which would be incompatible with the notion of \textsc{still}.
	\item Rhetorical questions with \textit{māzāl} in additive function, such as (\ref{exAppendixTunisianAdditive3}), clearly underlie the idiomatic exclamation in 
	 (\ref{exAppendixTunisianAdditive4}).
\end{itemize}

\begin{exe}
	\ex Context: About saints and the miracles they performed.
	\begin{xlist}
		\exi{A:}
		 \textit{Hāḏīy ʕād min...baṛša,	il-kaṛāmāt 	mawžūda yāsir	ǝmtāʕ iṣ-ṣāḷiḥīn. Ayh.	Yā wid-i yāsir	 ʕalay-y…	Ayh mā-ni	g̣utt-la ʕād.}\\
		 \lq Das ist also von... viele, es gibt viele Wunder der Heiligen. Ja. Mein Lieber,
das ist zuviel für mich. Ja, das habe ich ihm doch schon gesagt. [So that is … many, there are many miracles performed by the saints. Yes. My dear, that’s too much for me. Yes, I told him that already.]\rq
		\exi{B:} (interrupting A)\\
		\gll \textbf{Māzāl} {Sīdi Ṯāmir}\\
		still {S. T.}\\
		\glt \lq Es gibt noch Sīdi Ṯāmir. [There’s also Sidi Thamir.]ʼ (\cite[434–435]{RittBenmimoun2011}, cited in \cite{FischerEtAlTunisian})
	\end{xlist}
	
	\ex\label{exAppendixTunisianAdditive2}
	\gll \textbf{Māzāl}	wuld-it baʕd-u t̠lāṯa	krūš.\\
	still give\_birth.\textsc{pfv}-3\textsc{sg}.\textsc{f} after-3\textsc{sg}.\textsc{m} three belly.\textsc{pl}\\
	\glt \lq Elle a encore mis au monde trois enfants après lui. [She gave birth to three more children after him.]\rq{ }(\cite[1737]{MarcaisGuiga19581961}, cited in \cite{FischerEtAlTunisian})
	
	\ex\label{exAppendixTunisianAdditive3}
	\gll Āš \textbf{māzāl}  ṭṛah?\\
	what	still \textsc{hort}\\
	\glt \lq Na, was denn noch! [What else do you want!]\rq{ }(\cite[650]{Singer1984}, glosses added)

	\ex\label{exAppendixTunisianAdditive4}
	\gll \textbf{Māzāl}-ši tawwa!\\
	still-\textsc{q} now\\
	\glt \lq That's enough!\rq{ }\parencite{FischerEtAlTunisian}
\end{exe}

\subsection{bāqi}\label{appendixTunisianBaqi}
\subsubsection{General information}
\begin{itemize}
	\item Woordhood: free morpheme.
	\item Etymology: formally the \textsc{sg}.\textsc{m} form of the active participle of the root \mbox{\textit{b}-\textit{q}-\textit{y}} \lq remain\rq{}.
\end{itemize}

\subsubsection{As a (borderline) \textsc{still} expression}
\begin{itemize}
	\item \textcite[130–131]{Jemni2011}, \textcite{FischerEtAlTunisian}, \textcite[365–366]{MarcaisGuiga19581961} and \textcite[650]{Singer1984}.
	\item Specialisation: \textcite{FischerEtAlTunisian} observe that \textit{bāqi} presupposes a left\hyp abutting runtime of situation depicted in the clause and that it is infelicitous when combined with inalterable states. However, none of the data points feature clear-cut contexts that would be indicative of evoking an alternative scenario, and the translations given by \textcite{MarcaisGuiga19581961} are more indicative of a purely stative meaning, similar to what is found with French \textit{toujours} (see \appref{appendixFrenchToujours}); note for instance \lq sans qu’on en voit la fin [without an end in sight]\rq{ }in (\ref{exAppendixTunisianBaqi2}). \textit{Bāqi} thus constitutes a borderline case of a \textsc{still} expression.
	\item Polarity sensitivity: inner negation yields \textsc{not yet}.
	\item Pragmaticity: if taken to be a bona fide \textsc{still} expression, then \textit{bāqi} would be specialised for the unexpectedly late scenario.
	\item Syntax: relatively mobile.
	\item Further note: example (\ref{exAppendixTunisianBaqi3}) illustrates \textit{bāqi} with a dynamic verb in an imperative. This is attested only in older doculects; modern varieties would make use of the verb \mbox{\textit{z}-\textit{y}-\textit{d}} \lq continue\rq{}.
\end{itemize}

\begin{exe}
	\ex 
	\gll Billi ʕand-u baṛša milli mšā aʕlī-na \textbf{bāqi} y-ibʕat̠-il-na žwāb fi-l-ʕaīd.\\
	although at-3\textsc{sg}.\textsc{m} much since go.\textsc{pfv}.3\textsc{sg}.\textsc{m} away\_from-1\textsc{pl} still 3\textsc{sg}.\textsc{m}-send.\textsc{ipfv}-to-1\textsc{pl} letter in-\textsc{def}-feast\\
	\glt \lq Obwohl er uns schon lange verlassen hat, schreibt er uns doch immer noch zum Fest. [Although he left us a long time ago, he still regularly writes (a letter) to us on the Eid holiday.]\rq{ }(\cite[703]{Singer1984}; glosses by \cite{FischerEtAlTunisian})

	\ex\label{exAppendixTunisianBaqi2}
	\gll \textbf{Bāqi} ġaṛḏ-̣ik fi-r-rqād.\\
	still aim-\textsc{poss}.2\textsc{sg} in-\textsc{def}-sleep.\textsc{nmlz}\\
	\glt \lq Tu continues à avoir envie de dormir, tu as encore (sans qu’on en voit la fin)
envie de dormir. [You keep being sleepy, you are still sleepy (without any end in sight).]\rq{ }(\cite[365]{MarcaisGuiga19581961}; glosses by \cite{FischerEtAlTunisian})

	\ex \label{exAppendixTunisianBaqi3}
	 \gll \textbf{Bāqi} mīḥ l-ill-imīn.\\
	still tilt.\textsc{imp} to-\textsc{def}-right\\
	\glt \lq Continue dʼincliner toujours à droite. [Keep tilting to the right].\rq{ }(\cite[365]{MarcaisGuiga19581961}; glosses by \cite{FischerEtAlTunisian})
\end{exe}

\subsubsection{Broadly modal and interactional functions}
\paragraph{Concessive apodoses}
\label{appendixtunisianBaqiConcessive}
\begin{itemize}
	\item \textit{Bāqi} is repeatedly attested in the apodoses of alternative concessive conditionals, as in (\ref{exAppendixTunisianBaqiConcessive1}), and of universal concessive conditionals, as in (\ref{exAppendixTunisianBaqiConcessive2}).
	\item \textcite{FischerEtAlTunisian} point out the conceptual similarity to a purely persistive temporal reading: across all alternatives under discussion, the outcome remains unchanged.
\end{itemize}

\begin{exe}
	\ex \label{exAppendixTunisianBaqiConcessive1}
	\gll T-ikbir wulla t-usġur il-lifʕa muxwf-a \textbf{bāqi}.\\
	3\textsc{sg}.\textsc{f}-be\_big.\textsc{ipfv} or 3\textsc{sg}.\textsc{f}-be\_small.\textsc{ipfv} \textsc{def}-viper(\textsc{f}) dangerous-\textsc{sg}.\textsc{f} still\\
	\glt \lq Qu’elle se trouvé à être grande ou petite, la vipère est toujours dangereuse. [Big or small, a viper is dangerous no matter what.]\rq{ }(\cite[366]{MarcaisGuiga19581961}; glosses by \cite{FischerEtAlTunisian})
	
	\ex \label{exAppendixTunisianBaqiConcessive2}
	Context: The speaker is talking about his travels and the addressee keeps asking for more details.\\
	\gll Qadd-ma n-qul-l-ik baṛša \textbf{bāqi} šwayya.\\
	measure-\textsc{subord} 1\textsc{sg}-say.\textsc{ipfv}-to-2\textsc{sg} a\_lot still little\\
	\glt \lq No matter how much I tell you, you won’t be satisfied (lit. it remains little)’ (TuniCo, cited in \cite{FischerEtAlTunisian})
\end{exe}

\subsubsection{Other functions}
\paragraph{\lq{}Other than that, apart from that\rq}
\begin{itemize}
	\item \textcite{FischerEtAlTunisian} and \textcite{MarcaisGuiga19581961}.
	\item \textit{Bāqi} is repeatedly attested in uses along the lines of \lq other than that\rq{ }or\lq apart from that\rq{}. As pointed out by \textcite{FischerEtAlTunisian}, this is best understood as directly related to the lexical source \lq remainder\rq{ }rather than being mediated by \textit{bāqi} as a marker of persistence.
\end{itemize}

\begin{exe}
	\ex Context: About the design of a pearl necklace.\\
	\gll ʕmal-t hakka žǖst əkġošē b-il-akḥal hakka bȭnd. W-\textbf{bāqi} kull {dē pēġl}.\\
	make.\textsc{pfv}-1\textsc{sg} thus just crochet with-\textsc{def}-black thus band and-remainder every pearls\\
	\glt \lq I just crocheted a black band like that. Other than that, it’s all pearls.\rq{ }(TuniCo, cited in \cite{FischerEtAlTunisian})
	
	\ex \gll Il-ʕawaž illi xallā-ni tkallam-t wulla \textbf{bāqi} iš-šay mā-y-aʕnī-nī-š.\\
	\textsc{def}-crookedness(\textsc{m}) \textsc{rel} let.\textsc{pfv}.3\textsc{sg}.\textsc{m}-1\textsc{sg} speak.\textsc{pfv}-1\textsc{sg} or remainder \textsc{def}-thing \textsc{neg}-3\textsc{sg}.\textsc{m}-concern.\textsc{ipfv}-1\textsc{sg}-\textsc{neg}\\
	\glt \lq Ce qui a fait que j’ai parlé, c’est (que j’ai vu) la laideur de procédés; car, pour le
reste, la chose ne me regarde pas. 	[What made me speak was (that I saw) the ugliness of the process; other than that, I’m not concerned by the issue.]\rq{ }(\cite[366]{MarcaisGuiga19581961}; glosses by \cite{FischerEtAlTunisian})
\end{exe}\il{Arabic, Tunisian|)} 

\section{Xhosa (xho, xhos1239)}\label{appendixXhosa}
\il{Xhosa|(}
\subsection{Introductory remarks}\largerpage[2]
I am indebted to Onelisa Slater and Phumelele Lovisa for discussing Xhosa data with me and for providing additional examples. Xhosa has a typical Bantu noun class system. I have glossed the individual classes as \textsc{ncl} \lq noun class' plus a subscript number that follows the common Bleek-Meinhof system.

\subsection{sa-}
\subsubsection{General information}
\begin{itemize}
	\item Form: \textit{se}- with copulatives.
	\item Wordhood: bound morpheme (verb prefix).
\end{itemize}


\subsubsection{As a  \lq still\rq{ }expression}
\begin{itemize}
	\sloppy
	\item \textcite[126–127]{Bennie1939}, \textcite{CranePersohn2021}, \textcite[131–132, 210, 225]{McLaren1936}, \textcite[149]{GreaterDictionaryXhosa}, \textcite[315, 337]{Oosthuysen2016} and \textcite[192–195]{SavicThesis}.
	\item Specialisation:  the description by \citeauthor{CranePersohn2021} meets my definition and explicitly mentions the incompatibility with inalterable states.
	\item Pragmaticity: compatible with both scenarios.
	\item Polarity sensitivity: outer negation yields \textsc{no longer}. \textsc{not yet} is expressed by a dedicated marker \mbox{\textit{ka}-}.
	\item Further note: compatible with the perfective aspect plus inchoative verb (\ref{exAppendixXhosa2}).
\end{itemize}
\begin{exe}
	\ex
	\gll I-hashe eli-\textbf{se}-li-tsha ku-funeka li-qhel-is-w-e.\\
	\textsc{ncl}5-horse \textsc{rel}.\textsc{subj}.\textsc{ncl}5-still-\textsc{cop}.\textsc{ncl}5-young \textsc{subj}.\textsc{loc}-be\_necessary \textsc{subj}.\textsc{ncl}5-get\_used-\textsc{caus}-\textsc{pass}-\textsc{sbjv}\\
	\glt \lq A horse that is still young has to be trained.\rq{ }(\cite[167]{Oosthuysen2016}, glosses added)

	\ex\label{exAppendixXhosa2}
	\gll L-amkel-e i-thuba eli-\textbf{sa}-vulek-ile-yo o-noku-li-sebenz-isa e-ku-phemez-eni oku-ninzi.\\
	\textsc{obj}.\textsc{ncl}5-accept-\textsc{sbjv} \textsc{ncl}5-chance \textsc{rel}.\textsc{subj}.\textsc{ncl}5-still-open-\textsc{pfv}-\textsc{rel} \textsc{rel}.\textsc{subj}.2\textsc{sg}-with.\textsc{ncl}15(\textsc{inf})-\textsc{obj}.\textsc{ncl}5-work-\textsc{caus} \textsc{loc}-\textsc{ncl}15(\textsc{inf})-achieve-\textsc{loc} \textsc{ncl}15(\textsc{inf})-many\\
	\glt \lq Accept the opportunity that is still open and which you can use for many achievements.’ (AST Text Corpus, cited in \cite[243]{CranePersohn2021})

	\ex
	\gll Isi-vatho si-\textbf{se}-bomvu krwe yimbola\\
	\textsc{ncl}7-clothing \textsc{subj}.\textsc{ncl}7-still-red \textsc{ideoph}:intensly\_red \textsc{cop}.\textsc{ncl}9.red\_clay\\
	\glt \lq The clothing is still very red because of the red clay.\rq{ }(\cite[130]{DuPlessisVisser1992}, glosses added)
\end{exe}

\subsubsection{Uses on the fringes of \lq{}still\rq}
\paragraph{Scalar contexts}\largerpage
\label{appendixXhosaScalar}
\begin{itemize}
	\item \textit{Sa}- is compatible with scalar contexts, including the decremental uses in (\ref{exAppendixXhosaScalar1}, \ref{exAppendixXhosaScalar2}) and the \lq still no more than' ones in  (\ref{exAppendixXhosaScalar3}, \ref{exAppendixXhosaScalar4}). Examples (\ref{exAppendixXhosaScalar3}, \ref{exAppendixXhosaScalar4}) indicate that a restrictive \lq only\rq{ }operator is not compulsory in contexts of a limited increase.
\end{itemize}

\begin{exe}

	\ex\label{exAppendixXhosaScalar1}
	Context: We’re at a concert in a pub. I suggest to get another drink, but my friend says he ran out of money. I invite him.\\
	\gll Yiza, ndi-\textbf{se}-n-e-hundred Rand.\\
	come.\textsc{imp} \textsc{subj}.1\textsc{sg}-with-\textsc{ncl}9-hundred Rand\\
	\glt \lq Come, I still have a hundred Rand.' (Onelisa Slater, p.c.)
	
	\ex\label{exAppendixXhosaScalar2}
	Context: We have an apple tree in our garden and we’ve harvested a huge load, and there’s not end in sight. So we tell our friend:\\
	\gll Ukuba u-funa ama-apile, a-\textbf{se}-ma-nintsi awa-shiyek-ile-yo e-m-thi-ni.\\
	if \textsc{subj}.2\textsc{sg}-want \textsc{ncl}6-apple \textsc{subj}.\textsc{ncl}6-still-\textsc{cop}.\textsc{ncl}6-many \textsc{rel}.\textsc{subj.\textsc{ncl}6}-be\_left-\textsc{pfv}-\textsc{rel} \textsc{loc}-\textsc{ncl}3-tree-\textsc{loc}\\
	\glt \lq If you want apples, there are still a lot left in the tree.'
	\\(Onelisa Slater, p.c.)
	
	\ex\label{exAppendixXhosaScalar3}
	Context: You are expected to be sent eight books. You have gotten six so far, and your friend tells you she already got all eight. So you reply:\\
	\gll Ndi-\textbf{se}-n-ezin-tandathu (qha).\\
	\textsc{subj}.1\textsc{sg}-still-with-\textsc{ncl}10-six (only)\\
	\glt \lq I still have (only) six.' (Onelisa Slater, p.c.)
	
	\ex\label{exAppendixXhosaScalar4}
	Context: We’re planning to go to an event in the evening. I’ve got ants in my pants and want to leave already. You reply:\\
	\gll I-\textbf{se}-yin-tsimbi y-esi-thandatu\\
	\textsc{subj}.\textsc{ncl}9-\textsc{cop}.\textsc{ncl}9-bell \textsc{ncl}9-of.\textsc{ncl}7-six\\
	\glt \lq It's still six o' clock.' (Onelisa Slater, p.c.)
\end{exe}


\subsubsection{Broadly temporal-aspectual adverbial uses}
\paragraph{Prospective \lq eventually\rq{}}\label{appendixXhosaProspective}
\begin{itemize}
	\item \textcite[132]{McLaren1936} and \textcite[149]{GreaterDictionaryXhosa}.
	\item According to  \textcite[149]{GreaterDictionaryXhosa}, this function is restricted to verbs of directed motion, as in (\ref{exXhosaAntiterminative1}). This, however, includes grammaticalised forms of \textit{ya} \lq go' and \textit{za} \lq come'. These serve as future markers without necessarily contributing a sense of motion, as in (\ref{exXhosaAntiterminative2}, \ref{exXhosaAntiterminative3}). For an in-depth discussion of these grammaticalised forms in closely related Southern Ndebele, see \textcite{CraneMabena2019}.
	\item Negation of this use denies the future existence of any situation of the type described by the predicate (\ref{exXhosaAntiterminative4}).
\end{itemize}
\begin{exe}
	\ex\label{exXhosaAntiterminative1}
	\begin{xlist}
		\exi{A:} \textit{Sewukhe waya evenkileni}?\\
		\lq Have you been to the store?\rq{}
		
		\exi{B}\gll  Hayi, ndi-\textbf{sa}-ya.\\
		 no \textsc{subj}.1\textsc{sg}-still-go\\
		\glt \lq Have you been to the store? -- No, I am still going to go.\rq{ }(\cite[149]{GreaterDictionaryXhosa}, glosses added)
	\end{xlist}
	
	
	\ex\label{exXhosaAntiterminative2}
	Context: After a heated discussion.\\
	\gll U-\textbf{sa}-zo-bona ukuba ngom-phi ku-thi so-ba-bini o-chanek-ile-yo\\
	\textsc{subj}.2\textsc{sg}-still-come.\textsc{inf}/\textsc{fut}-see \textsc{comp} \textsc{cop}.\textsc{ncl}1-\textsc{q} \textsc{ncl}15(\textsc{inf})-say  1\textsc{pl}-\textsc{ncl}2-two \textsc{rel}.\textsc{subj}.\textsc{ncl}1-be\_correct-\textsc{pfv}-\textsc{rel}\\
	\glt \lq You'll see yet who of us two is right.' (Onelisa Slater, p.c.)
	
	\ex\label{exXhosaAntiterminative3}
	\gll U-nga-yek-i uku-zama! E-ku-gqib-el-eni u-(\textbf{sa})-zo-nd-ogqitha\\
	\textsc{subj}.2\textsc{sg}-\textsc{neg}.\textsc{sbjv}-stop-\textsc{neg} \textsc{ncl}15(\textsc{inf})-try \textsc{loc}-\textsc{ncl}15(\textsc{inf})-finish-\textsc{appl}-\textsc{loc} \textsc{subj}.\textsc{ncl}2-still-\textsc{fut}-\textsc{obj}.1\textsc{sg}-outdo\\
	\glt \lq Don't stop practising! Eventually, you'll beat me.'
	\\(Onelisa Slater, p.c.)
	
	\ex\label{exXhosaAntiterminative4}
	\gll Um-khosi a-wu-\textbf{sa}-y-i k-oyis-wa.\\
	\textsc{ncl}3-army \textsc{neg}-\textsc{subj}.\textsc{ncl}3-still-go-\textsc{neg} \textsc{ncl}15(\textsc{inf})-defeat-\textsc{pass}\\
	\glt \lq The army will never be defeated.\rq{ }(\cite[132]{McLaren1936}, glosses added)
\end{exe}

\subsubsection{Marginality}\label{appendixXhosaMarginal}
\begin{itemize}
	\item \textit{Sa}- is compatible with a range of readings of marginality.
\end{itemize}
\begin{exe}
	\exi{A:}\label{exAppendixXhosaMarginal1}I’m really annoyed. My aunt has left the better part of her fortune to an animal shelter, and only 100,000 Rand to me…
	\exi{B:}\gll  i-100,000 Rand i-\textbf{se}-sisi-xa esi-fanelek-ile-yo\\
	\textsc{ncl}9-100,000 R. \textsc{subj}.\textsc{ncl9}-\textsc{cop}.\textsc{ncl}7-portion \textsc{rel}.\textsc{subj}.\textsc{ncl}7-be\_suitable-\textsc{pfv}-\textsc{rel}\\
	\glt \lq  100,000 Rand is still a decent sum.' (Onelisa Slater, p.c.)
	
	\ex\label{exAppendixXhosaMarginal2}
	Context: Talking about tennis skills.\\
	\gll Ndi-\textbf{se}-n-oku-m-oyisa u-Paul, kodwa u-Mark yena u-ngcono ku-na-m.\\
	\textsc{subj}.1\textsc{sg}-still-with-\textsc{ncl}15(\textsc{inf})-\textsc{obj}.\textsc{ncl}1-beat \textsc{ncl}1a-P. but \textsc{ncl}1a-M. \textsc{dem}.\textsc{ncl}1 \textsc{subj}.\textsc{ncl}1-better \textsc{loc}-with-1\textsc{sg}\\
	\glt \lq I can still beat Paul, but Mark is better than me.' (Onelisa Slater, p.c.)

	\ex\label{exAppendixXhosaMarginal3}
	 Context: Speaking about political views.\\
	\gll U-Bongani u-\textbf{se}-n-obu-gncathu\\
	\textsc{ncl}1a-B. \textsc{subj}.\textsc{ncl}1-still-with-\textsc{ncl}14-moderation\\
	\glt \lq Bongani is still moderate (as opposed to others who have more radical views).' (Onelisa Slater, p.c.)
	
	\ex\label{exAppendixXhosaMarginal4}
	\gll i-Tohoyandou i-\textbf{se}-se-Mzantsi Afrika / ku-\textbf{se}-se-Mzantsi Afrika\\
	\textsc{ncl}9-T. \textsc{subj}.\textsc{ncl}9-still-\textsc{loc}-South Africa {} \textsc{subj}.\textsc{loc}-still-\textsc{loc}-South Africa\\
	\glt \lq Tohoyandou is still (in) South Africa (as opposed to other places across the border).' (Onelisa Slater, p.c.)
\end{exe}

\subsubsection{Restrictive (non-temporal)}
\paragraph{ \lq Thus far only'}
\label{appendixXhosaRestrictive}
\begin{itemize}
	\item \textcite{CranePersohn2021}.
	\item This reading is only available in combination with the perfective aspect and non-inchoative verbs.
	\item As examples like (\ref{exAppendixXhosaThusFarOnly1}, \ref{exAppendixXhosaThusFarOnly2}) show, this use does not require a scale.
	\item \textcite{CranePersohn2021} suggest that this use goes back to aspectual coercion, originating with predicates that form part of a natural sequence of events.
\end{itemize}

\begin{exe}
	\ex\label{exAppendixXhosaThusFarOnly1}
	\gll Ndi-\textbf{sa}-ty-\textbf{e} imengo.\\
	\textsc{subj}.1\textsc{sg}-still-eat-\textsc{pfv} \textsc{ncl}9.mango\\
	\glt \lq (So far) I've only eaten mango (e.g. out of more food that is on offer).' (my field notes)
	
	\ex\label{exAppendixXhosaThusFarOnly2}
	\gll Aba-ntwana ba-\textbf{sa}-dlal-ile.\\
	\textsc{ncl}2-child \textsc{subj}.\textsc{ncl}2-still-play-\textsc{pfv}\\
	\glt \lq ʻSo far the kids have played [well] (but let’s not get our hopes too high).' \parencite[243]{CranePersohn2021}
		
	\ex
	\gll Ndi-\textbf{s}-akh-e le ndlu.\\
	\textsc{subj}.1\textsc{sg}-still-build-\textsc{pfv} \textsc{prox}.\textsc{ncl}9 \textsc{ncl}9.house\\
	\glt \lq I have for now/only built this house. (but might built at least one more)' \parencite[243]{CranePersohn2021}
\end{exe}
\il{Xhosa|)}
