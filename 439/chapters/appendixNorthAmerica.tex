\chapter{North America}
\label{appendixNorthAmerica}

\section{Classical Nahuatl (nci, clas120)}\label{appendixClassicalNahuatl}\il{Nahuatl, Classical|(}
\subsection{Introductory remarks}
I am indebted to Michel Launey for discussing Nahuatl data with me and for helping with some difficult glosses. Classical Nahuatl has two \textsc{still} expressions: \textit{nozan} and \textit{oc}. Only for the latter do I have clear indications of additional functions.

\subsection{oc}
\subsubsection{General information}
\begin{itemize}
	\item Form: also transcribed as \textit{ok}, and as \mbox{\textit{oqu}-} in certain compounds.
	\item Wordhood: free morpheme, but can cliticize.
	\item Etymology: \textcite[92]{Karttunen1992} suggests that this marker is related to the numeral \textit{ōme} \lq two'. \Textcite[92]{vanBaar1997} reports personal communication from Michel Launey that this proposal is implausible for several morphophonological reasons. \Textcite{vanBaar1997} instead suggests a tentative link to the verb \textit{onoc} \lq be lying, be stretched out'.
\end{itemize}

\subsubsection{As a \textsc{still} expression}
\begin{itemize}
	\sloppy
	\item \textcite[41]{Andrews2003}, \textcite[246]{Bierhorst1985}, \textcite[501]{Carochi1645}, \textcite[175]{Karttunen1992} and \citeauthor{Launey1986} (\citeyear[1261–1622]{Launey1986}, \citeyear[64–65]{LauneyMackay2011}); further discussion throughout \textcite{vanBaar1997}.
	\item Specialisation: \textcite{vanBaar1997} identifies this marker as a \textsc{still} expression. This is supported by examples such as (\ref{exAppendixClassicalNahuatl1}–\ref{exAppendixClassicalNahuatl3}). For instance, in (\ref{exAppendixClassicalNahuatl1}) \textit{oc} not only involves the continuation of a prior habit, but also strongly suggests a contrast with people's behaviour at the time of speech.
	\item Pragmaticity: according to \textcite[76]{vanBaar1997} this expression is used for the neutral scenario; for the simultaneously counterfactual scenario it is augmented by an item \textit{nohmah}, as in (\ref{exAppendixClassicalNahuatl3}), or a different \textsc{still} expression altogether, \textit{nozan}, is used.
	\item Polarity sensitivity: outer negation yields \textsc{no longer}.
	\item Further note: ex. (\ref{exAppendixClassicalNahuatl4}) illustrates the use in an imperative.
\end{itemize}
\begin{exe}
	\ex\label{exAppendixClassicalNahuatl1}
	\gll In ye nēpa \textbf{oc} tlatlācamati-ya in mācēhual-tin.\\
	\textsc{det} already there still obey-\textsc{pst}.\textsc{ipfv} \textsc{det} commoner-\textsc{pl}\\
	\glt \lq Autrefois, les gens du peuple êtaient encore obeissants. [In the olden days, the commoners were still obedient.]\rq{ }(\cite[1262]{Launey1986}, glosses added)

	\ex\label{exAppendixClassicalNahuatl2}
	\gll Câ \textbf{oc} pil-tōn-tli, ayamo mozcalia.\\
	3\textsc{sg} still child-\textsc{dim}-\textsc{n} not\_yet be\_sensible\\
	\glt \lq Todavía es muchacho, aun no tiene juyzio. [He's still a child, he's not reasonable yet].' (\cite[501]{Carochi1645}, glosses added)

	\ex\label{exAppendixClassicalNahuatl3}
	\gll \textbf{Oc}-\textbf{nòmà} an-cochì? Cuix \textbf{oc} an-qu-ichià in tōnatiuh amo-tzontlan mo-quetza-quiuh? Cuix inmanin \textbf{oc}-\textbf{nòmà} cochī-hua?\\
	still-still \textsc{subj}.2\textsc{pl}-sleep.\textsc{pl} \textsc{q} still \textsc{subj}.2\textsc{pl}-\textsc{obj}.3\textsc{sg}-await.\textsc{pl} \textsc{det} sun \textsc{poss}.2\textsc{pl}-at\_head\_of\_bed \textsc{refl}-get\_up-\textsc{inch}.\textsc{ipfv} \textsc{q} this\_very\_time still-still sleep-\textsc{impr}\\
	\glt \lq Todavía dormis? Por ventura aguardais a que el Sol venga a dar en vuestras cabeceras? Es ésta hora de dormir? [Are you still asleep? Are you waiting for the sun to shine on your headboard? Is this the time to sleep?]ʼ (\cite[501]{Carochi1645}, glosses added)
	
	\ex\label{exAppendixClassicalNahuatl4}
	\gll Mā \textbf{oc} xi-quim-om-mo-nōchi-lī-cān, mā \textbf{oc} xi-quim-om-mo-tzàtzī-lī-lī-cān.\\
	\textsc{hort} still \textsc{sbjv}-\textsc{obj}.3\textsc{pl}-\textsc{it}-\textsc{refl}-call.-\textsc{appl}-\textsc{pl} \textsc{hort} still \textsc{sbjv}-\textsc{obj}.3\textsc{pl}-\textsc{it}-\textsc{refl}-shout-\textsc{appl}-\textsc{appl}-\textsc{pl}\\
	\glt \lq Continuez à les appeler, continuez à crier vers eux. [Keep calling them, keep shouting at them.]' (\cite[1262]{Launey1986}, glosses added)\footnote{The combination of the reflexive and applicative here serves a politeness function; see \textcite[ch. 21]{LauneyMackay2011}.}

\end{exe}

\subsubsection{Uses on the fringes of \lq{}still\rq{}}
\paragraph{Scalar contexts}\label{appendixClassicalNahuatlScalar}
\begin{itemize}
	\item \textit{Oc} is attested in contexts of scalar decreases.
	\item Also see \appref{appendixClassicalNahuatlTimeSpanBefore} for a function that clearly involves scalar decreases.
\end{itemize}

\begin{exe}
	\ex
	\gll Auh in ye on-calaqui-z tōnatiuh, in \textbf{oc} achihtōn tōnatiuh, niman ye īc huāl-tzahtzi in Ītzcuāuhtzin.\\
and \textsc{det} already \textsc{subj}.3:\textsc{it}-enter-\textsc{prosp} sun \textsc{det} still a\_bit sun right\_away already when \textsc{subj}.3:\textsc{ven}-cry\_out:\textsc{pst}.\textsc{pfv} \textsc{det} I.\\
\glt \lq And when the sun was about to set, when there was still a little sun, thereupon Itzcuauhtzin cried out.\rq{ }(\cite[609]{Andrews2003}, glosses added)

\end{exe}

\paragraph{Scalar contexts: preceding time span}
\label{appendixClassicalNahuatlTimeSpanBefore}
\begin{itemize}
	\item \textcite[501]{Carochi1645} and \citeauthor{Launey1986} (\citeyear[1268–1269]{Launey1986}, \citeyear[369]{LauneyMackay2011}); additional discussion in \textcite[308–310]{vanBaar1997}.
	\item This is a  complex clause pattern that signals a time span before a situation \lq (it is) \textit{t} before \textit{q}'. It surfaces in two types of constructions:
	\begin{itemize}
		\item A clause pattern [[\textit{in} (\textit{ìcuāc}) q] [\textit{oc} p]] \lq \textsc{det} (when/then) \textit{q}, still \textit{p}', i.e. \lq{}when \textit{p}, it is still a certain amount of time' (\ref{exAppendixClassicalNahuatlTimeSpan1}, \ref{exAppendixClassicalNahuatlTimeSpan2}). In this case, \textit{oc} can be accompanied by the \textsc{already} expression \textit{ye} to introduce a dual perspective, both from the situation forward to utterance time or another evaluation time (\textit{oc}) and from there backwards (\textit{ye}); see (\ref{exAppendixClassicalNahuatlTimeSpan3}).

	\item A clause pattern [[\textit{oc} p] [(\textit{in}) q]] \lq still \textit{p}, (\textsc{det}) \textit{q}' where the subordinate clause features either the prospective inflection or an imperfective venitive form. That is, \lq it is still a certain amount of time, \textit{q} is going to happen/is approaching'. This is illustrated in  (\ref{exAppendixClassicalNahuatlTimeSpan4}, \ref{exAppendixClassicalNahuatlTimeSpan5}).
	\end{itemize}		
	\item This function is undoubtedly an extension of a scalar use of \textit{oc}, i.e. \lq (it is still) a certain amount of time (then)'. Note how the inverse perspective (\lq [it is] already … [ago]') is expressed using the \textsc{already} expression \textit{ye} \parencite[1258]{Launey1986}.
\end{itemize}
\largerpage[2]
\begin{exe}	
	\ex\label{exAppendixClassicalNahuatlTimeSpan1}
\gll [In-in ca \textbf{oc} huècauh] [in mo-chīhua-tīuh]\\
\textsc{det}-\textsc{prox} \textsc{pred} still long\_time \textsc{def} \textsc{subj}.3:\textsc{refl}.3-do-go.\textsc{ipfv}\\
\glt \lq Ceci se produira dans longtemps [It will happen in a long time].' (\cite[1268]{Launey1986}, glosses added)

	\ex\label{exAppendixClassicalNahuatlTimeSpan2}
\gll [In ìcuāc mo-chīuh in], [\textbf{oc} iuh chicōn-xihuitl polihui-z in āltepētl chālcayōtl].\\
\textsc{det} when/then \textsc{subj}.3:\textsc{refl}.3-do-\textsc{pst}.\textsc{pfv}  \textsc{det} still thus seven-year perish-\textsc{prosp} \textsc{det} town Chalco.\textsc{poss}\\
\glt \lq Quand cela se produisit, (c'etait) sept ans avant la chute de la cité de Chalco  (\lq\lq encore comme sept ans périra"). [When it happened, (it was) seven years before the fall of the city of Chalco (\lq\lq still seven years it will fall").]\rq{ }(\cite[400]{Launey1986}, glosses added)

	\ex\label{exAppendixClassicalNahuatlTimeSpan3}
	\gll In \textbf{oc} ye huècauh, in \textbf{oc} ye nēpa, in \textbf{oc} ye nechca, in \textbf{oc} īm-pan huehuētquê,\\
	\textsc{dec} still already long\_time \textsc{det} still already there \textsc{det} still already over\_there \textsc{det} still \textsc{poss}.3\textsc{pl}-\textsc{loc} old\_person.\textsc{pl}\\
	\glt\lq Long ago in the past, during the time of the ancients [i.e. still a long time from there (to now) and already a long time (ago) …]'
	\exi{} \gll cualli ic tla-mani-ya in ī-pan t-āltepē-uh\\
	good when/thereby \textsc{obj}.\textsc{indef}:\textsc{non}.\textsc{human}-spread-\textsc{pst}.\textsc{ipfv} \textsc{det} \textsc{poss}.3\textsc{sg}-\textsc{loc} \textsc{poss}.1\textsc{pl}-town-\textsc{poss}	\\
	\glt \lq Things went (spread out) well in our city.' \parencite[369]{LauneyMackay2011}

	\ex\label{exAppendixClassicalNahuatlTimeSpan4}
	\gll \textbf{Oc} yuh \textbf{macuil}-\textbf{ilhuitl} àci-quiuh in to-tlàtò-ca-uh, in ō-tech-tlalhuì-quê.\\
	still thus five-day arrive-\textsc{come}.\textsc{ipfv} \textsc{det} \textsc{poss}.1\textsc{pl}-king-\textsc{lnk}-\textsc{poss} \textsc{det}  \textsc{aug}-\textsc{subj}.3:\textsc{obj}.1\textsc{pl}-warn-\textsc{pst}.\textsc{pfv}:\textsc{pl}\\
	\glt \lq Cinco dias antes que llegara el Virrey nos preuinieron. [Five days before the viceroy’s arrival they warned us.]' (\cite[501]{Carochi1645}, glosses added)

	\ex\label{exAppendixClassicalNahuatlTimeSpan5}
	\gll \textbf{Oc} īmōztlayōc t-àci-zquê in Pasquà, nicān ō-n-àci-co.\\
	still next\_day \textsc{subj}.1\textsc{pl}-arrive-\textsc{prosp}.\textsc{pl} \textsc{det} Easter here \textsc{aug}-\textsc{subj}.1\textsc{sg}-arrive-\textsc{pst}.\textsc{pfv}\\
	\glt \lq Vn dia antes de Pasqua llegué aqui. [One day before Easter I arrived here, lit. Still one day left until we were going to arrive at Easter, I arrived.]' (\cite[501]{Carochi1645}, glosses added)
\end{exe}

\subsubsection{Broadly adverbial temporal-aspectual functions}
\paragraph{Iterative}
\label{appendixClassicalNahuatlIterative}
\begin{itemize}
	\item  \textcite[246]{Bierhorst1985} and \parencite[1265]{Launey1986} (\citeyear[1265]{Launey1986}).
	\item Iterative uses are illustrated in (\ref{appendixClassicalNahuatlIterative1}, \ref{appendixClassicalNahuatlIterative2}).
	\item Unlike with the interative/restitutive-via-addition collocation \textit{oc cēppa} (\appref{appendixClassicalNahuatlIterativeIncrement}), there are no clear restitutive uses in the data consulted.
	\item There are no tense-aspect or modal restrictions on this function.
\end{itemize}
\begin{exe}
	\ex\label{appendixClassicalNahuatlIterative1}
	\gll In āxcān \textbf{oc} ni-mitz-tla-pòpolhuia.\\
	\textsc{det} now/today still \textsc{subj}.1\textsc{sg}-\textsc{obj}.2\textsc{sg}-\textsc{obj}.\textsc{indef}:\textsc{non}.\textsc{human}-forgive\\
	\glt \lq Pour aujourdʼhui je te pardonne encore. [For today I will forgive you once more.]' (\cite[1265]{Launey1986}, glosses added)
	\ex\label{appendixClassicalNahuatlIterative2}
	\gll Zā \textbf{oc} quēmman mo-cuā-zquê in ī-nāmic.\\
	only stilll at\_times \textsc{subj}.3:\textsc{refl}-eat-\textsc{prosp} \textsc{det} \textsc{poss}.3\textsc{sg}-spouse\\
	\glt \lq Il faut que son mari (et elle) s'accouplent (\lq\lq se mangent") encore de temps en temps. [It is necessary that her husband (and her) mate (lit. eat each other) again from time to time.]' (\cite[1265]{Launey1986}, glosses added)
\end{exe}

\paragraph{Iterative and restitutive via increment}
\label{appendixClassicalNahuatlIterativeIncrement}
\begin{itemize}
		\item \textcite[473]{Andrews2003}, \textcite[247]{Bierhorst1985}, \textcite[175]{Karttunen1992} and \textcite[1265]{Launey1986}.
	\item Form: this function occurs in collocation with \textit{cē}-\textit{ppa} \lq one-time'.
	\item Both iterative (\ref{exAppendixClassicalNahuatlIterativeIncrement1}, \ref{exAppendixClassicalNahuatlIterativeIncrement2}) and restitutive uses (\ref{exAppendixClassicalNahuatlIterativeIncrement3}–\ref{exAppendixClassicalNahuatlIterativeIncrement5})  are attested.
\end{itemize}
\begin{exe}
	\ex \label{exAppendixClassicalNahuatlIterativeIncrement1}
	\gll Auh quēmman \textbf{oc} \textbf{cē}-\textit{ppa} ti-tla-cuā-z?\\
	and when still one-time \textsc{subj}.2\textsc{sg}-\textsc{obj}.\textsc{indef}:\textsc{non}.\textsc{human}-eat-\textsc{prosp}\\
	\glt \lq Y a qué hora has de comer otra vez? [And when will you eat again]?' (\cite[505]{Carochi1645}, glosses added)

	\ex \label{exAppendixClassicalNahuatlIterativeIncrement2}
	\gll \textbf{Oc} \textbf{cē}-\textit{ppa} qu-ìtò-quê in tē$\sim$teô:\\
	still one-time \textsc{obj}.3\textsc{sg}-say-\textsc{pst}.\textsc{pfv}:\textsc{pl} \textsc{det} \textsc{pl}$\sim$god.\textsc{pl}\\
	\glt \lq De nouveau, les dieux dirent: … [Again the gods said: …]'
	\\(\cite[1265]{Launey1986}, glosses added) 

	\ex \label{exAppendixClassicalNahuatlIterativeIncrement3}
\gll In tlā cē chico-petōni huel tēcocô, auh nō huel tecocô in ic \textbf{oc} \textbf{ce}-\textbf{ppa} ī-ye-yān mo-zalao.\\
\textsc{det} if one sideways-dislocate \textsc{intens} painful and \textsc{intens} painful \textsc{det} thus when still one-time \textsc{poss}.3\textsc{sg}-\textsc{loc}.\textsc{cop}-customary\_place \textsc{refl}.3-put\_together\\
\glt \lq Si se desconcierta vno, y se sale a vn lado, duele mucho, como tambien duele mucho, quando se buelue a su lugar. [When one [of our bones] dislocates it hurts a lot, and it also hurts when it moves back into its place.]' (\cite[498]{Carochi1645}, glosses added)

	\ex\label{exAppendixClassicalNahuatlIterativeIncrement4} 
	\gll \textbf{Oc} \textbf{cē}-\textit{ppa} ti-pil-tōn-tli ti-mo-chīhua-z.\\
	still one-time \textsc{subj}.2\textsc{sg}-child-\textsc{dim}-\textsc{n} \textsc{subj}.2\textsc{sg}-\textsc{refl}-make-\textsc{prosp}\\
	\glt \lq You will become a child again.' (\cite[537]{Andrews2003}, glosses added)

	\ex\label{exAppendixClassicalNahuatlIterativeIncrement5} 
	\gll In iuh ō-c-on-ìtô in, niman ic oc cē-ppa ō-coch-tlamelāuh.\\
	\textsc{det} thus \textsc{aug}-\textsc{obj}.3\textsc{sg}-\textsc{it}-say.\textsc{pst}.\textsc{pfv} \textsc{det} right\_away then still one-time \textsc{aug}-sleep-\textsc{obj}.\textsc{indef}:\textsc{non}.\textsc{human}-straighten.\textsc{pst}.\textsc{pfv}\\
	\glt \lq Ayant dit cela, il se remit à dormir  [Having said this, he went back to sleep].' (\cite[604]{Launey1986}, glosses added)
\end{exe}

\paragraph{First, for now}\label{appendixClassicalNahuatlFirst}
\begin{itemize}
	\item \textcite[502]{Carochi1645} and \textcite[1264]{Launey1986}.
	\item This function is typically found in explanations of what the speaker is doing or announcements of what they are about to do, as in (\ref{exAppendixclassicalNahuatlFirstforNow1}–\ref{exAppendixclassicalNahuatlFirstforNow3}).

	\item The notion of precedence can be made explicit by using \textit{achto} \lq first' (or one of its variants), as in (\ref{exAppendixclassicalNahuatlFirstforNow2}).
\end{itemize}
\begin{exe}
	\ex\label{exAppendixclassicalNahuatlFirstforNow1}
	\gll Mā \textbf{oc} ni-tla-qua, ca ye huellàcà, quin teōtlac ni-mitz-yōl-cuītī-z.\\
	\textsc{hort} still \textsc{subj}.1\textsc{sg}-\textsc{obj}.\textsc{indef}:\textsc{non}.\textsc{human} \textsc{pred} already late\_in\_day momentarily/then afternoon \textsc{subj}.1\textsc{sg}-\textsc{obj}.2\textsc{sg}-heart-acknowledge\_failings-\textsc{prosp}\\
	\glt \lq Comeré primero, que ya es tarde; después te confessaré a la tarde. [I’ll first eat, it’s late already; then in the afternoon I’ll confess to you.]ʼ (\cite[502]{Carochi1645}, glosses added)

	\ex\label{exAppendixclassicalNahuatlFirstforNow2}
	\gll In ìcuāc ti-tla-chpāna-z-nequi, \textbf{oc} yê achto in ti-tl-àhuachī-z.\\
	\textsc{det} when \textsc{subj}.2\textsc{sg}-\textsc{obj}.\textsc{indef}:\textsc{non}.\textsc{human}-sweep-\textsc{prosp}-want still actually first \textsc{det} \textsc{subj}.2\textsc{sg}-\textsc{obj}.\textsc{indef}:\textsc{non}.\textsc{human}-irrigate-\textsc{prosp}\\
	\glt \lq Quando quieras barrer, primero has de regar. [When you want to sweep, you first have to apply water.]\rq{ }(\cite[502]{Carochi1645}, glosses added)

	\ex\label{exAppendixclassicalNahuatlFirstforNow3}
	\gll \textbf{Oc} ni-c-cahua in, quin çā-tēpan ni-c-tzonquīxtī-z in ī-tlàtōllo, in oncān ī-monecyan.\\
	still \textsc{subj}.1\textsc{sg}-\textsc{obj}.3\textsc{sg}-leave \textsc{det} then only-momentarily \textsc{subj}.1\textsc{sg}-\textsc{obj}.3\textsc{sg}-finish-\textsc{prosp} \textsc{det} \textsc{poss}.3\textsc{sg}-speech \textsc{det} there \textsc{poss}.3\textsc{sg}-proper\_place/time\\
	\glt \lq Dexolo aqui por agora, después acabaré de tratar dello en su lugar. [I’ll leave it at this for now, I’ll finish talking about it afterwards, in due time.]\rq{ }(\cite[500]{Carochi1645}, glosses added)

	\ex
	\gll Câ zan \textbf{oc} īxquich.\\
	3\textsc{sg} only still that\_much\\
	\glt \lq Cʼest tout pour l'instante. [That's all for now.]\rq{ }(\cite[1263]{Launey1986}, glosses added)
\end{exe}


\subsubsection{Temporal connectives and frame setters}

\paragraph{Simultaneous duration}
\label{appendixClassicalNahuatlWhile}
\begin{itemize}
	\item \textcite[452, 516, 525]{Andrews2003}, \textcite[503]{Carochi1645}, \textcite[180]{Karttunen1992}, \citeauthor{Launey1986} (\citeyear[1268–1269]{Launey1986}; \citeyear[64–65, 365–366]{LauneyMackay2011}); additional discussion in \textcite[309]{vanBaar1997}.
	\item Form: this function obtains in subordinate clauses introduced by the determiner \textit{in}, and \textit{oc} often occurs in combination with \textit{ic} \lq when', i.e. lit. \lq when it is still the time when …\rq{}.
\end{itemize}
\begin{exe}
	\ex\label{exAppendixClassicalNahuatlWhile1}
	\gll Mācamo xi-còcoch-ti-ye-cān in \textbf{oc} ic n-on-tē-machtia.\\
	\textsc{proh} \textsc{sbjv}-doze-\textsc{lnk}-stay-\textsc{pl} \textsc{det} still when \textsc{subj}.1\textsc{sg}-\textsc{it}-\textsc{obj}.\textsc{indef}:\textsc{human}-teach\\
	\glt \lq Don’t be dozing off while I’m teaching.\rq{ }(\cite[366]{LauneyMackay2011}, glosses added)
	
	\ex\label{exAppendixClassicalNahuatlWhile2}
	\gll Mā niman āxcāmpa xi-mo-nemiliz-cuepa-cān, in oc am-pāc-ti-nemî, in oc an-chicāhua-ti-nemî.\\
	\textsc{hort} right\_away thereupon \textsc{sbjv}-\textsc{refl}-way\_of\_life-turn-\textsc{pl} \textsc{det} still \textsc{subj}.2\textsc{sg}-lead\_happy\_life-\textsc{lnk}-move.\textsc{pl} \textsc{det} still \textsc{subj}.2\textsc{sg}-be\_strong-\textsc{lnk}-move.\textsc{pl}\\
	\glt \lq Conuertiros, y haced penitencia desde luego, mientras teneis tiempo, mientras estais sanos, y fuertes. [Convert and do penance right now, while you have time, while you are healthy and strong.]\rq{ }(\cite[503]{Carochi1645}, glosses added)

	\ex\label{exAppendixClassicalNahuatlWhile3}
	\gll Xo-c-on-ìcuilo, in \textbf{oc} ic n-on-yāuh tēōpan n-on-no-teō-chīhua-z.\\
	\textsc{sbjv}-\textsc{obj}.3\textsc{sg}-write \textsc{det} still when \textsc{subj}.1\textsc{sg}-\textsc{it}-go church \textsc{subj}.1\textsc{sg}-\textsc{it}-\textsc{refl}-God-make-\textsc{prosp}\\
	\glt \lq Escriue aqui esto, mientras voy a reçar a la Iglesia. [Write this down here, while I go to church to pray.]' (\cite[503]{Carochi1645}, glosses added)
\end{exe}





\subsubsection{Additive and related functions}
\paragraph{Additive}\label{appendixClassicalNahuatlAdditive}
\begin{itemize}
	\item \textcite[43–44, 318, 548–549]{Andrews2003}, \textcite[247]{Bierhorst1985}, \textcite[502]{Carochi1645}, \textcite[175–176]{Karttunen1992} and \citeauthor{Launey1986} (\citeyear[1265–1267, 1425]{Launey1986}, \citeyear[66]{LauneyMackay2011}); additional discussion in \textcite[308–309]{vanBaar1997}.
	\item \textit{Oc} as additive \lq also\rq{ }sometimes co-occurs with other additive markers, such as \textit{nō} in the last token of (\ref{exAppendixClassicalNahuatlAlso2}).
	\item Syntax: as \lq another\rq{ }(i.e. additive with type identity), \textit{oc} is a syntactic sister to the focus.
\end{itemize}
\begin{exe}
	\ex\label{exAppendixClassicalNahuatlAlso2}
	\gll Ca quēmàca, ca oc miyac no-tlàtlacōl ni-qu-ilnāmiqui: \textbf{oc} ōme-ntin cuācuahuè-quê ō-ni-quimichtec, īhuān \textbf{oc} nāuhpa ō-ni-tlāhuān, \textbf{oc} nō īzquipa ō-ni-naca-cuâ viernez-tica\\
	\textsc{pred} \textsc{aff} \textsc{pred} still many \textsc{poss}.1\textsc{sg}-sin \textsc{subj}.1\textsc{sg}-\textsc{obj}.3\textsc{sg}-remember still two-\textsc{pl} cow-\textsc{pl} \textsc{aug}-\textsc{subj}.1\textsc{sg}-steal:\textsc{pst}.\textsc{pfv} and still four\_times \textsc{aug}-\textsc{subj}.1\textsc{sg}-get\_drunk:\textsc{pst}.\textsc{pfv} still also many\_times \textsc{aug}-\textsc{subj}.1\textsc{sg}-meat-eat:\textsc{pst}.\textsc{pfv} friday-\textsc{ins}\\
	\glt \lq Oui, je me souviens encore de beaucoup de mes péchés: j’ai encore volé deux vaches, et je me suis encore enviré quatre fois, et jʼai encore autant de fois mangé de la viande le vendredi.
[Yes, I still remember a lot of my sins: I also stole two cows, and I also got drunk four times, and I also repeatedly ate meat on Fridays.]\rq{ }(\cite[1266]{Launey1986}, glosses added)

	\ex
	\gll Is-ka' iwaan \textbf{ok} sen-tlamantli.\\
	here-\textsc{pred} and still one-thing\\
	\glt \lq And here is yet another thing.' \parencite[41]{Langacker1977}

	\ex \gll \textbf{Oc} ōme.\\
	still two\\
	\glt \lq They are another two / they are two more.' (\cite[318]{Andrews2003}, glosses added)
\end{exe}




\paragraph{Comparisons of inequality}\label{appendixClassicalNahuatlComparisons}
\begin{itemize}
	\item \textcite[563–565]{Andrews2003}, \textcite[490–492]{Carochi1645} and \citeauthor{Launey1986} (\citeyear[1267]{Launey1986}, \citeyear[340]{LauneyMackay2011}).
	\item Classical Nahuatl makes use of several comparative strategies (see \cite{Stassen2013} for typological discussion of these). These are conjoined comparatives, predominantly in the form of an affirmative and a negative statement (\lq X has property, when Y does not\rq{}), exceed-type comparatives, and combinations of both.
	\item  \textit{Oc} commonly features in comparative constructions, particularly as part of the collocation \textit{oc achi} \lq still a bit, still rather\rq{}, which is used in conjoined comparatives (\ref{exAppendixClassicalNahuatlComparative1}). It seems that its contribution here is to mark the comparative degree itself \lq an additional/another somewhat (that the standard does not possess)\rq{}. This usage extends to other degree marking expressions, such as \textit{cencà} \lq much\rq{}, \textit{tāchcāuh} \lq superior thing\rq{},  \textit{tlapanahuia} \lq it exceeds\rq{} and, often, combinations of these (\ref{exAppendixClassicalNahuatlComparative2}).
In other words, this is an instantiation of the additive (\appref{appendixClassicalNahuatlAdditive}) and/or alterative (\appref{appendixClassicalNahuatlOther}) senses of \textit{oc}.
	
		\item It is not entirely clear whether \textit{oc} can also have a \lq even more\rq{ }reading. Though such a translation is attested (\ref{exAppendixClassicalNahuatlComparative3}), it could be a contextual inference going back to the standard of comparison possessing the quality in question.
\end{itemize}
\begin{exe}
	\ex\label{exAppendixClassicalNahuatlComparative1}
	\gll \textbf{Oc} achi ni-chicāhuac in àmo mach yuhqui tèhuātl.\\
	still rather/a\_bit \textsc{subj}.1\textsc{sg}-strong \textsc{det} \textsc{neg} indeed thus 2\textsc{sg}\\
	\glt \lq Mas fuerte soy que tu. [I'm stronger than you.]\rq{ }(\cite[491]{Carochi1645}, glosses added)
	

	\ex\label{exAppendixClassicalNahuatlComparative2}
	\gll \textbf{Oc} cencà tāchcāuh \textup{/} cencà \textbf{oc} tlapanahuia \textup{/} cencà \textbf{oc} hualcà inic tlaçòtli inic mahuiztic in coztic teōcuitlatl, in àmo yê tepoztli.\\
still much \textsc{subj}.3:superior {} much still \textsc{subj}.3:exceed {} much still more with \textsc{subj}.3:precious with \textsc{subj}.3:be\_esteemed \textsc{det} golden precious\_metal \textsc{det} \textsc{neg} indeed workable\_metal\\
\glt \lq Mucho más precioso, y de estima es el oro, que el hierro. [Gold is much more precious and esteemed than iron.]\rq{ }(\cite[491]{Carochi1645}, glosses added)

	\ex\label{exAppendixClassicalNahuatlComparative3}
\gll In ye tetzāhuac tōnacāyōtl, \textbf{oc} tla-panahuia qui-namacâ.\\
\textsc{det} already \textsc{subj}.3:hardened\_thing produce still 3:\textsc{obj}.\textsc{indef}:\textsc{non}.\textsc{human}-exceed \textsc{subj}.3:\textsc{obj}.3\textsc{sg}-sell.\textsc{pl}
\\ \glt \lq Quand les produits agricoles sont déjà, fermes,
ils en vendent encore plus. [When produce is ripe, it sells even more.]\rq{ }(\cite[1267]{Launey1986}, glosses added)
\end{exe}
\pagebreak
\largerpage[2]
\paragraph{\lq other, different'}\label{appendixClassicalNahuatlOther}
\begin{itemize}
	\item \textcite[318]{Andrews2003}, \textcite[247]{Bierhorst1985}, \textcite[175–176]{Karttunen1992} and \citeauthor{Launey1986} (\citeyear[1266–1267]{Launey1986}, \citeyear[66]{LauneyMackay2011}).
	\item Form: in this function, \textit{oc} combines with a quantifier, usually a form of \textit{cē}/\textit{cen} \lq one'.
	\item In this function, \textit{oc} plus its associated constituent refer to other entities of the same type, be it other members within an established set, as in (\ref{exAppendixClassicalNahuatlOther1}), or merely a different entity of the same type, as in (\ref{exAppendixClassicalNahuatlOther2}, \ref{exAppendixClassicalNahuatlOther3}).
	\item This is clearly related to the additive function of \textit{oc}, as observed before me by \textcite[1266]{Launey1986}; also note similar colexification in items like English \textit{another}, Spanish \textit{otro}, etc.
	\item Examples like (\ref{exAppendixClassicalNahuatlOther4}), which involves both additivity and a pre-established set, possibly constitute bridging contexts.
	
\end{itemize}
\begin{exe}
	\ex \label{exAppendixClassicalNahuatlOther1}
	\gll Nicān câ ce tepētl: cān câ in \textbf{oc}-cē?\\
	here \textsc{loc}.\textsc{cop} one mountain where \textsc{loc}.\textsc{cop} \textsc{det} still-one\\
	\glt \lq One mountain's here. Where's the other one?\rq{ }(\cite[66]{LauneyMackay2011}, glosses added)

	\ex \label{exAppendixClassicalNahuatlOther2}
	\gll Huel \textbf{oc} cen-tlamantli in ic ni-c-mati-ya in mo-tēnyo.\\
	\textsc{intens} still one-thing \textsc{det} thus \textsc{subj}.1\textsc{sg}-\textsc{obj}.3\textsc{sg}-know-\textsc{pst}.\textsc{ipfv} \textsc{det} \textsc{poss}.2\textsc{sg}-fame.\textsc{poss}\\
	\glt \lq Tout autre est lʼopinion que jʼavais de ta renommée. [Quite a different one is the opinion I had of your reputation.]\rq{ }(\cite[1266]{Launey1986}, glosses added)

	\ex \label{exAppendixClassicalNahuatlOther3}
	\gll \textbf{Oc} ce-cni m-itō-tiuh.\\
	still one-place \textsc{refl}-say-\textsc{prog}\\
	\glt \lq On va en parler (\lq\lq{}ça va se dire\rq\rq{}) ailleurs (\lq\lq encore dans un endroit\rq\rq{}).	[We are going to talk about it (it will be said) elsewhere.]\rq{ }(\cite[1267]{Launey1986})

	\ex \label{exAppendixClassicalNahuatlOther4}
	\gll Mo-nequi an-qu-ìtō-z-què in quēzqui-pa ō-an-tlāhuān-què, in quēzqui-pa ō-an-quìtlacò-quê in ī-missà-tzin To-tēcu-iyo, çācè quēcīzqui-pa in ō īpan an-huetz-quê in \textbf{oc}-\textbf{cequi} tē-mictiā-ni tlàtlacōlli.\\
	\textsc{refl}.3-want \textsc{subj}.2\textsc{pl}-\textsc{obj}.3\textsc{sg}-say-\textsc{prosp}-\textsc{pl} \textsc{det} how\_many-time \textsc{aug}-\textsc{subj}.2\textsc{pl}-get\_drunk-\textsc{pst}.\textsc{pfv}.\textsc{pl}  \textsc{det} how\_many-time \textsc{aug}-\textsc{subj}.2\textsc{pl}-\textsc{obj}.3\textsc{sg}-do\_wrong-\textsc{pst}.\textsc{pfv}.\textsc{pl} \textsc{det} \textsc{poss}.3\textsc{sg}-mess-\textsc{hon} \textsc{poss}.1\textsc{pl}-God-ness	 finally how\_many\_each-time \textsc{det} \textsc{aug} \textsc{poss}.3\textsc{sg}-top \textsc{subj}.2\textsc{pl}-fall-\textsc{pst}.\textsc{pfv}.\textsc{pl} \textsc{det} still-some \textsc{obj}.\textsc{indef}:\textsc{human}-kill-\textsc{a}.\textsc{nmlz} sin\\
	\glt \lq Que digais quantas vezes os aueis emborrachado, quantas aueis dexado de oyr Missa, y finalmente quantas vezes aueis caido en otros pecados mortales. [You must say how many times you got drunk, how many times you failed to go to mass and how many times you have fallen into other mortal sins.]\rq{ }(\cite[117]{Carochi1645}, glosses added)
\end{exe}

\largerpage[2.25]
\subsubsection{Broadly modal and interactional functions}
\paragraph{Concessive apodoses: \textit{ic oc} / \textit{oc ic}}
\label{appendixClassicalNahuatlConcessiveConsequent}
\begin{itemize}
	\item \textcite[1264]{Launey1986}; also see \textcite[307]{vanBaar1997}.
	\item Form: in this function, \textit{oc} combines with \textit{ic} \lq thereby' (yielding \textit{ic oc} or \textit{oc ic}) and either the present tense in a generic reading (\ref{exAppendixClassicalNahuatlConcessive1}) or the prospective aspect (\ref{exAppendixClassicalNahuatlConcessive2}, \ref{exAppendixClassicalNahuatlConcessive3}).
	\item This usage of \textit{oc} lies on the intersection of \textsc{still} proper, concession and modality, in that it expresses a persistent possibility of success despite unfavourable circumstances.
\end{itemize}
\begin{exe}
	\ex\label{exAppendixClassicalNahuatlConcessive1}
	\gll \textbf{Ic} \textbf{oc} palēhuī-lo in pil-huâ.\\
	thereby still help-\textsc{pass} \textsc{det} child-\textsc{poss}.\textsc{nmlz}\\
	\glt \lq Par ce procédé la mère est encore soulagée. [By this procedure the mother is still relieved.]' (\cite[1265]{Launey1986}, glosses added)

	\ex\label{exAppendixClassicalNahuatlConcessive2}
	\gll In tlā iuh ti-qu-ìtō-z in, cuix \textbf{oc} \textbf{ic} t-om-pàti-z?\\
	\textsc{det} if\_only thus \textsc{subj}.2\textsc{sg}-\textsc{obj}.3\textsc{sg}-speak-\textsc{prosp} \textsc{det} \textsc{q} still thereby 2\textsc{sg}-\textsc{it}-cure-\textsc{prosp}\\
	\glt \lq Si tu parles ainsi, apporteras-tu encore quelque amélioration? [If you say so, will you still  get better?]' (\cite[1265]{Launey1986}, glosses added)
	
	\ex\label{exAppendixClassicalNahuatlConcessive3}
	\gll Àço çānēn \textbf{oq}-uic nēci-z acà qualli tlācatl tēlpoca-tzin.\\
	perhaps doubt still-thereby \textsc{subj}.3\textsc{sg}.appear-\textsc{prosp} someone good person person-\textsc{hon}\\
	\glt \lq Quiçà querrà Dios, que de aqui allà se ofresca ocasion de algun moço virtuoso. [Maybe if God wills (nonetheless), a good and virtuous young man will appear (to marry my daughter).]\rq{ }(\cite[517]{Carochi1645}, glosses added)\footnote{See \textcite[1245]{Launey1986} on \textit{àzo} \textit{zā} \textit{nēn} (rendered here as \textit{àço} \textit{çānēn}) on marking a highly unlikely state-of-affairs.}
\end{exe}


\paragraph{Causal connective}
\label{appendixClassicalNahuatlCausal}
\begin{itemize}
	\item \textcite[503]{Carochi1645} and \textcite[1269]{Launey1986}, 
	\item Form: this function obtains in subordinate clauses introduced by the determiner \textit{in} and in combination with \textit{ic} \lq when'.
	\item As suggested by \textcite[1269]{Launey1986}, there is little doubt that this function is an extension of the framing function discussed in \ref{appendixClassicalNahuatlWhile} above.
\end{itemize}

\begin{exe}
	\ex
	\gll Elnantlòtze tlā zā yê ti-calaqui-cān, \textbf{oc} \textbf{ic} qui-tlapōuh-ticāuh in quilchīuhqui.\\
	Hernando if\_only only indeed \textsc{subj}.1\textsc{pl}-enter-\textsc{pl} still when \textsc{obj}.3\textsc{sg}-open-\textsc{lnk}-leave \textsc{det} gardener\\
	\glt \lq Hernando, entremos en la huerta, pues la dexó abierta el hortelano. [Hernando, let’s go in the vegetable garden, since the gardener left (the door) open.]' (\cite[503]{Carochi1645}, glosses added)
	
	\ex
	\gll In \textbf{oc} \textbf{ic} ti-tēpil-tzin … xi-m-ìmat-cā-nemi.\\
	\textsc{det} still when \textsc{subj}.2\textsc{sg}-offspring-\textsc{hon} {} \textsc{sbjv}-\textsc{refl}-be\_wise-\textsc{lnk}-live\\
	\glt \lq Pues que eres bien nacido … viue con cordura. [Since you are noble, live wisely.]' (\cite[503]{Carochi1645}, glosses added)
\end{exe}
\il{Nahuatl, Classical|)}

\section{Creek (mus, cree1270)}\il{Creek|(}
\subsection{Introductory remarks}
Apart from descriptive materials I consulted the text collections by \textcite{Gouge2004} and \textcite{HaasHill2014}. My understanding of Creek data has furthermore greatly profited from discussion with Jack Martin, who also helped with many glosses and elicited additional examples.

\subsection{(i)monk}
\subsubsection{General information}
\begin{itemize}
\sloppy
\item Form: this expression comes in two forms. The most frequent is a verb\slash auxiliary \mbox{(\textit{i})\textit{mônk}} that nearly always goes together with the durative aspect suffix \mbox{-\textit{i}} on the preceding lexical verb. Much less frequently, a nominalised form \textit{imónka} occurs, typically together with a copula, as in (\ref{exAppendixCreek2}, \ref{exAppendixCreekSame2}) below. 
\item Wordhood: intermediate (verb/auxiliary).
\item Etymology:\quad in all likelihood, this item is cognate with the \ili{Choctaw} expression \textit{mo̠ma} \lq still, be all' (Jack Martin, p.c.). In addition, Creek has an item (\textit{i})\textit{m}-\textit{mônka} with the meaning \lq nature or habit (of smth.), smth. permanent or natural' (\cite[307 fn4]{Martin2011}; \cite[25]{MartinMcKaneMauldin2000}). All this suggests that the origins of \mbox{(\textit{i})\textit{mônk}} lie in denoting the permanence and/or extended duration of a state-of-affairs.
\end{itemize}

\subsubsection{As a \lq{}still\rq{ }expression}\label{appendixCreekStill}
\begin{itemize}
	\item \textcite[306–307]{Martin2011} and \textcite[25]{MartinMcKaneMauldin2000}.
	\item Specialisation: this is a borderline case of a \textsc{still} expression. On the one hand, there are examples like (\ref{exAppendixCreek1}–\ref{exAppendixCreek3}) that point towards both main components of my definition. For instance, the discourse context of (\ref{exAppendixCreek1}) suggests that the auxiliary serves a dual purpose of indicating the persistence of an earlier tradition, as well as establishing a contrast with the time of speech, where the square ground no longer exists; also see (\ref{exAppendixCreekWhile2}, \ref{exAppendixCreekWhile4}) below. On the other hand, there are also instances of it with a more general \lq keep doing, do persistently meaning\rq{ }(see \appref{appendixCreekKeep}). Given that its origins may lie in a marker of permanence or stability, it is conceivable that \mbox{(\textit{i})\textit{monk}} is currently developing into a full-fledged phasal polarity expression. This process appears to have been completed with its \ili{Choctaw} cognate \textit{mo̠ma} (see \cite[316–317]{Broadwell2006}; \cite[120]{ChochtawDictionary}).
\item Pragmaticity: compatible with both scenarios. It is not entirely clear if the unexpectedly late scenario requires additional marking. Creek frequently makes use of the copula \textit{om} or of positional verbs to emphasize an assertion (see \cite[ch. 32]{Martin2011}) \textendash{ }it is perceivable that this strategy might also come into play in signalling the unexpectedly late scenario of \textsc{still}.
\item Polarity sensitivity: inner negation yields \textsc{not yet} (or a closely related notion).
\end{itemize}
\begin{exe}
	\ex\label{exAppendixCreek1}
	Context: From the opening paragraph of an expository text about the last game of stickball at the village square ground.\\
	\gll Kawíta im-paːskóːfa lêyk-\textbf{i:} \textbf{móŋk}-oːf, kasihtafíksiko in-hilís-haːy-atí-:s.\\
	Coweta \textsc{dat}-square\_ground sit.\textsc{sg.res}-\textsc{dur} still-when Cussetah\_Fixico \textsc{appl}-medicine-make-\textsc{dist.pst}-\textsc{ind}\\
	\glt \lq When [the town of] Coweta still had their square ground Cussetah Fixico made the medicine.' (\cite[671]{HaasHill2014}, glosses added)
	
	\pagebreak
	\ex\label{exAppendixCreek2}
	Context: About the arrival of alcohol.\\
	\gll Ohɬolopíː paːli-cahkíːp-ank-íː máːh-i wíski má:k-aːk-\textbf{i:} \textbf{imónka}-t oːm-atíː-t ôːⁿ-s. moːmêys hayyôːmat oyhomíː kéyhoːc-íː ôːn-s\\
	year ten-five-\textsc{rec.pst}-\textsc{dur} about-\textsc{dur} whisky say-\textsc{pl}-\textsc{dur} still.\textsc{nmlz}-\textsc{subj} \textsc{cop}.\textsc{dist.pst}-\textsc{ss} \textsc{cop}-\textsc{ind} however now u. call.\textsc{pass}-\textsc{dur} \textsc{cop}-\textsc{ind}\\
	\glt \lq They still used to call it \lq\lq{}whisky\rq\rq{ }about fifty years ago. Now, however, they call it uehomē [bitter water].\rq{ }(\cite[36]{HaasHill2014}, glosses added)
	
	\ex\label{exAppendixCreek3}
	\gll Ist-ocí lowǎ∙nkos-i∙ \textbf{môŋk}-i∙ i-ŋkososowá in-wá∙łho∙y-atí∙ hołkóp-ka-n tǎ∙ny-i∙ ha∙k-í∙s má∙ho∙k-ánc.\\
person-\textsc{dim} limber.\textsc{emph}-\textsc{dur} still-\textsc{dur} \textsc{poss}.3-fingernail \textsc{appl}-slice.\textsc{impr}-\textsc{rem}.\textsc{pst} steal-\textsc{nmlz}-\textsc{non}.\textsc{subj} much-\textsc{dur} become-\textsc{dur}-\textsc{ind} say.\textsc{impr}-\textsc{pst}.\textsc{ind}\\
\glt \lq If you cut the fingernails of a newborn baby [lit. a baby that is still limber], it will grow up and steal things, they used to say.\rq{ }(\cite[297]{HaasHill2014}, hlosses added)
\end{exe}

\subsubsection{Uses on the fringes of \lq{}still\rq{}}
\paragraph{Scalar contexts}
\label{appendixCreekScalar}
\begin{itemize}
	\item The two clear-cut examples of scalar contexts both involve a decrease over time.
\end{itemize}

\begin{exe}
	\ex
	Context: Two towns agreed to play four games against each other. Three games have been played.\\
	\gll A:fack-itá hámk-it ahô:sk-\textbf{i:} \textbf{mónk}-ati:-s.\\
	happy-\textsc{inf} one-\textsc{subj} left\_over.\textsc{res}-\textsc{dur} still-\textsc{rem}.\textsc{pst}-\textsc{ind}\\
	\glt \lq One game still remained.' (\cite[692]{HaasHill2014}; \cite[306]{Martin2011})
	
	\ex Context: Opposum is explaining to Rabbit how to make persimmon fruit fall off the tree.\\
	\textit{Asá a∙hálwosa∙n onápan ła∙ohhoyêyłit istimomǎlnkosit li∙tkí∙t}\\
\lq You stand up there on top of that little hill and run with all your might back downhill\rq{}
	\exi{}\gll
	ła∙-ák-ta∙skít ahopay-ós-i∙ \textbf{môŋk}-i∙-ta∙n ta∙sêykit.\\
	go\_and-\textsc{loc}-jump.\textsc{sg}.\textsc{ss} measure-\textsc{dim}-\textsc{dur} still-\textsc{dur}-\textsc{ss}:\textsc{cop}.\textsc{ref}.\textsc{ds} jump.\textsc{sg}.\textsc{pfv}-\textsc{ss}\\
	\glt \lq and jump; while you’re still a short distance away, jump.\rq{}\footnote{The segmentation of \mbox{-\textit{ta∙n}} as a contraction of \mbox{-\textit{t}} \mbox{\textit{o:m}-\textit{a:n}} \lq -\textsc{ss} \textsc{cop}-\textsc{ref}.\textsc{ds}\rq{ }is tentative.}
	\\(\cite[482]{HaasHill2014}; glosses added)
\end{exe}

\paragraph{Same}\label{appendixCreekSame}
\begin{itemize}
	\item \textcite[25]{MartinMcKaneMauldin2000}.
	\item The meaning is one of permanence over time, i.e. closely related to both \textsc{still} and to the likely etymology of \mbox{(\textit{i})\textit{monk}}. For instance, in (\ref{exAppendixCreekSame1}), the dances have not changed over time.
	\item Addition of diminutive -\textit{os} is common and seems to strengthen the notion identity, along the lines of \lq just the same' (Jack Martin, p.c); this is in line with with the functions of this suffix (see \cite[234–236]{Martin2011}).	
	\item A case like (\ref{exAppendixCreekSame2}), which pairs persistence with zero anaphora may feature a bridging context. Example (\ref{exAppendixCreekSame3}) is similarly close in that it could be read as \lq … and you will persist\rq{}.
\end{itemize}

\begin{exe}
	\ex \label{exAppendixCreekSame1}
	Context: According to customs they would have fasts and \\
	 \gll opánka oːc-ít  foll-atí-n oːm-âːt \textbf{imóŋk}-os-in foll-atíː-s\\
dance have-\textsc{ss} go\_about.\textsc{pl}-\textsc{happen}-\textsc{ds} \textsc{cop}-\textsc{ref} still.\textsc{nmlz}-\textsc{dim}-\textsc{non}.\textsc{subj} go\_about.\textsc{pl}-\textsc{dist.pst}-\textsc{ind}\\
	\glt \lq the same dances they used to have (lit. if they happened to have a dance, they would do it in the same way [as before]).\rq{ }(\cite[57–58]{HaasHill2014}, glosses added)

	\ex Context: Turtleʼs wife has splattered blood in his eyes.\label{exAppendixCreekSame2}
	\\
	\gll Itóɬwa 	ak-caːt-ak-átiː-t ôːm-it hayyôːm-eys \textbf{imóŋka}-t 	ôn-t ôːm-iː-s.\\
3.eye \textsc{loc}-red-\textsc{pl}-\textsc{rem}.\textsc{pst}-\textsc{ss} \textsc{cop}-\textsc{ss} be\_now.\textsc{res}-even still.\textsc{nmlz}-\textsc{subj} \textsc{cop}-\textsc{ss} \textsc{cop}-\textsc{dur}-\textsc{ind}\\
	\glt \lq His eyes turned red, and they are the same way even now (lit. … and even now they still are).\rq{ }(\cite[442]{HaasHill2014}, glosses added)
\pagebreak
\ex Context: The heavens and earth will perish.\label{exAppendixCreekSame3}\\
\gll momis  ceme-t \textbf{emunk}-us-et lik-etsk-en\\
but you-\textsc{subj} still-\textsc{dim}-\textsc{ss} sit.\textsc{sg}.\textsc{res}-2\textsc{sg}.\textsc{a}-\textsc{ds}\\
\glt \lq But you remain the same.' (Hebrews 1:12; Jack Martin, p.c.)
\end{exe}

\paragraph{Keep Verb-ing}\label{appendixCreekKeep}
\begin{itemize}
	\item \textcite[306–307]{Martin2011}.
	\item There are several instances of \mbox{(\textit{i})\textit{monk}} in the data that contribute a notion of \lq keep doing, do persistently\rq{}. Often, but not always, this involves a notion of iterativity (\ref{exAppendixCreekKeep2}, \ref{exAppendixCreekKeep3}). 
\end{itemize}
\begin{exe}
	
	\ex Context: A hunter has crept towards a doe and her fawns. They haven’t taken notice of him\label{exAppendixCreekKeep2}.\\
	\gll Ani-ó ayo:pk-éy-t 	o:m-éy-ka ani-ó a:y-ay-í: \textbf{mônk}-it\\
	1\textsc{sg}-also creep-1\textsc{sg}.\textsc{a}-\textsc{ss} \textsc{cop}-1\textsc{sg}-\textsc{nmlz} 1\textsc{sg}-also go.\textsc{sg}-1\textsc{sg.a}-\textsc{dur} still-\textsc{ss}\\
	\glt \lq And I kept on creeping closer and closer.\rq{ }(\cite[106]{Gouge2004}, glosses added)
	
	\ex Context: Mother skunk is angry at Tortoise for telling slander about her. She has confronted him, but he has denied it.\label{exAppendixCreekKeep3}\\
	\gll Máhk-ey-s máːk-as keyc-íː \textbf{imôŋk}-it oːm-ín …\\
	say.\textsc{pfv}-1\textsc{sg.a}-\textsc{ind} say-\textsc{imp} tell-\textsc{dur} still-\textsc{ss} \textsc{cop}-\textsc{ds}\\
\glt \lq {\lq\lq}Admit you said it\rq\rq{}, she kept saying [telling him] …\rq{ }(\cite[374]{HaasHill2014}, glosses added)
\end{exe}




\subsubsection{Temporal connectives and frame setters}
\largerpage[2]
\paragraph{Simultaneous duration}\label{appendixCreekWhile}
\begin{itemize}
	\item \textcite[404]{Martin2011} and \textcite[25]{MartinMcKaneMauldin2000}.
	\item Of the relevant tokens in the data that feature affirmative polarity, all are compatible with a reading of  \lq while still', and it might rather be a matter of how prominent the latter is vis-à-vis a reading of purely simultaneous duration.
	\begin{itemize}
		\item Several instances feature cohesive clauses restating a situation that has been described earlier (\ref{exAppendixCreekWhile1}). In one instance, it is the inception of the situation that has been introduced before (\ref{exAppendixCreekWhile2}).
		\item In other cases the situation in question can be retrieved from context, in that they all involve advice given to adolescents, as in (\ref{exAppendixCreekWhile3}).
		\item In yet other instances, (\textit{i})\textit{monk} appears to narrow down the temporal frame. For instance, in (\ref{exAppendixCreekWhile4}) it is a hunter's habit to set out at night, the implicit alternative being that he leaves later.	

		\end{itemize}
		\item In combination with internal negation this yields precedence, i.e. \lq when \textsc{not yet} \textit{p}\rq{ }> \lq before \textit{p}', as in the second instance in (\ref{exAppendixCreekWhile2}). This is a common extension of \textsc{not yet} constructions; see \Cref{sectionBefore}.
\end{itemize}

\begin{exe}
	\ex \label{exAppendixCreekWhile1}
	Context: A boy has been turned into a snake. A man is watching him.
	
\gll A:y-ít ma óywa atĭ:ⁿk-os-a:n ił-hôył-in … ak-somêyk-in hóyɬ-\textbf{i:} \textbf{mônk}-in\\
go.\textsc{sg}-\textsc{ss} that water up\_to.\textsc{emph}-\textsc{dim}-\textsc{ref}.\textsc{ds} go\_and-stand.\textsc{sg}:\textsc{res}-\textsc{ds} {} \textsc{loc}:water-disappear.\textsc{sg}:\textsc{pfv}-\textsc{ds} stand.\textsc{sg}-\textsc{dur} \textsc{still}-\textsc{ds}\\
\glt \lq He went to the water's edge and stood [and saw] … and as he stood there watching [the snake] went under again.\rq{ }(\cite[138]{HaasHill2014}, glosses added)

	\ex \label{exAppendixCreekWhile2}
	\gll Hółłi is-ín-ci:y-atí:-t ô:m-i:-s ci. Mo:m-ín itípo:y-ít sihô:k-\textbf{i} \textbf{mônk}-it hółłi ’m-iyóks-ikoː mônk-in hopoyłyahóla il-iːp-atíː-t ôm-iː-s ci.\\
	war \textsc{ins}-\textsc{appl}-enter-\textsc{rem}.\textsc{pst}-\textsc{ss} \textsc{cop}-\textsc{dur}-\textsc{ind} \textsc{dm} be\_so-\textsc{ds} fight-\textsc{ss} stand.\textsc{res}-\textsc{dur} still-\textsc{ss} war \textsc{appl}-end-\textsc{neg} still-\textsc{ds} H. die.\textsc{sg}-\textsc{mid}-\textsc{rem}.\textsc{pst}-\textsc{ss} \textsc{cop}-\textsc{dur} \textsc{dm}\\
	\glt \lq They entered the war. And while they were still fighting, before the end of the war, Hopuethlyahola died.\rq{ }(\cite[706]{HaasHill2014}, glosses added)
	
	\ex \label{exAppendixCreekWhile3}
	Context: Advising young men not to smoke tobacco.\\
	\gll Ci-manítt-\textbf{i:} \textbf{imônk}-i: ísti acǒ:ⁿl-os-i: ó:m-i: ci-háhk-i:-s\\
	2-young-\textsc{dur}	still-\textsc{dur}	 person old.\textsc{emph}-\textsc{dim}-\textsc{dur}	be\_like-\textsc{dur}	2-become.\textsc{pfv}-\textsc{dur}-\textsc{ind}\\
	\glt \lq Though you are young, you’ll become like an old man. (lit. You are (still) young / while you are (still) young… )\rq{ }(\cite[297]{HaasHill2014}, glosses added)

	\ex \label{exAppendixCreekWhile4}
	Context: A hunter would always be prepared to set out.\\
	\gll Hayaːtk-âːt yomóck-\textbf{iː} \textbf{mônk}-in aːyí-t ɬafó-tot miskíː-toː istôːm-eys\\
	dawn-\textsc{ref} dark-\textsc{dur} still-\textsc{ds} go.\textsc{sg}-\textsc{ss} winter-\textsc{foc} summer-\textsc{foc} do\_any.\textsc{res}-even\\
	\glt \lq He goes at dawn while still dark in winter, summer, whichever.\rq{ }(\cite[321]{Martin2011}; \cite[254]{HaasHill2014})
\end{exe}

\subsubsection{Marginality}
\label{appendixCreekMarginal}
\begin{itemize}
	\item (\textit{i})\textit{monk} can be used in a marginality function. 
\end{itemize}
\begin{exe}
	\ex
	Context: I am really annoyed. My aunt has left the better part of her fortune to an animal shelter and only 10,000 bucks to me. My friend replies.\\
	\gll 10,000 matî:k-a:t	hǐ:ⁿɬ-it \textbf{imónka}-t ô:-s.\\
	10,000 be\_up\_to.\textsc{res}-\textsc{ref} good.\textsc{emph}-\textsc{ss} still.\textsc{nmlz}-\textsc{subj} \textsc{cop}-\textsc{ind}\\
	\glt \lq 10,000 is still a good amount.' (Jack Martin, p.c.)
	
	\ex
	 Context: Talking about skills in some sport.\\
	\gll Mark-ta:t im-ákosl-éy-t \textbf{imónka}-t ô:-s. Mô:weys Tom ’s-an-hiɬ-í:-n ákkopa:n-í:-t ó:-s.\\
	M.-\textsc{attention} \textsc{appl}-beat-1\textsc{sg}.\textsc{a}-\textsc{ss} still.\textsc{nmlz}-\textsc{subj} \textsc{cop}-\textsc{ind} but T. \textsc{ins}-\textsc{appl}.1\textsc{sg}-good-\textsc{dur}-\textsc{ds} play-\textsc{dur}-\textsc{ss} \textsc{cop}-\textsc{ind}\\
	\glt \lq I can still beat Mark, but Tom plays better than me.\rq{ }(Jack Martin, p.c.)
	
	\ex\label{exAppendixCreekScalarCategorizing}
	Context: We’re taking a trip north and are talking about the US-Canadian border.\\
	\gll Seattle ma-t	 wacína ó:fa-n leyk-\textbf{í:} \textbf{imónk}a-t ô:-s. Mô:weys Vancouver Canada ó:fa-n leyk-ip-í:-t ô:-s.\\
	S. that-\textsc{subj} USA in-\textsc{non}.\textsc{subj} sit.\textsc{sg}-\textsc{dur} still.\textsc{nmlz}-\textsc{subj} \textsc{cop}-\textsc{ind} but V. C. in-\textsc{non}.\textsc{subj} sit.\textsc{sg}-\textsc{mid}-\textsc{dur}-\textsc{ss} \textsc{cop}-\textsc{ind}\\
	\glt \lq Seattle is still in the U.S., but Vancouver is in Canada.\rq{ }(Jack Martin, p.c.)
\end{exe}

\subsubsection{Broadly modal and interactional functions}
    \paragraph{Concessive apodoses}
\label{appendixCreekConcessiveConsequent}
\begin{itemize}
	\item The inclusion of this function is tentative, as I only have few (elicited) examples, and influence from the contact language English cannot be excluded.
\end{itemize}
\begin{exe}
	\ex
	\gll Ó:skima:h-ít	o:w-êysin,	lítk-aha:n-\textbf{í:}	\textbf{imónka}-t ô:-s.\\
	rain.really-\textsc{ss} \textsc{cop}-but run-\textsc{prosp}-\textsc{dur} still.\textsc{nmlz}-\textsc{subj} \textsc{cop}-\textsc{ind}\\
	\glt \lq It's raining heavily but he's still going for a run.' (Jack Martin, p.c.)
	
	\ex
	\gll Alí:kca-t litêyk-ícc-as keyc-ít o:w-â:n, mô:weys Jack 'timaɬka-litkitá li:tk-ít \textbf{imônk}-it ó:-s.\\
	doctor-\textsc{subj} run.\textsc{sg}.\textsc{pfv}-2\textsc{sg}.\textsc{a}-\textsc{imp} say-\textsc{ss} \textsc{cop}-\textsc{ref} but J. race run.\textsc{sg}-\textsc{ss} still.\textsc{nmlz}-\textsc{subj} \textsc{cop}-\textsc{ind}\\
	\glt \lq The doctor said to him \lq\lq Don't run", but Jack still ran in the race.'\\
	(Jack Martin, p.c.)
\end{exe}
\il{Creek|)}
   

\section{Gitxsan/Nisga'a (git, gitx1241/ncg, nisg1240)}
\il{Gitxsan|(}\il{Nisga'a|(} 
\subsection{Introductory remarks}
The data include both Gitxsan (git, gitx1241) and Nisga'a (ncg, nisg1240), both of which are closely related and mutually intelligible (e.g. \cite[21–22]{Rigsby1986}), and the relevant expression, \textit{k̠'ay}, encompasses the same functions. 

\subsection{k̠'ay}

\subsubsection{General information}
\begin{itemize}
	\item Wordhood: independent grammatical word, invariable.
	\item Syntax: fixed, in preverbal position.
\end{itemize}

\subsubsection{As a \lq{}still\rq{ }expression}
\begin{itemize}
	\item \textcite{Aonuki2021}, \textcite[445–447]{Tarpent1987} and \textcite[363–364]{Rigsby1986}.
	\item Specialisation: \textcite{Aonuki2021} addresses the presupposition of a previously ongoing situation, and  \citeauthor{Tarpent1987}'s (\citeyear{Tarpent1987}) discussion suggests the invocation of an alternative discontinuation scenario.
	\item Polarity sensitivity: not attested in combination with negation; \textsc{not yet} is expressed by a seperate item \textit{haw'en}.
	\item Pragmaticity: the data allow no conclusion.
\end{itemize}

\begin{exe}
	\ex
	\gll \textbf{K̠'ay}  hli\sim hlgutk'ihlkw-t.\\
	still \textsc{dur}\sim child-3\\
	\glt \lq S/he is still a child.ʼ \parencite[446]{Tarpent1987}

	\ex
	\gll \textbf{K̠'ay} gu\sim gwineegamks, kii huxw k'atsgwiý.\\
	still \textsc{dur}\sim cool and again land-1\textsc{sg}\\
	\glt \lq I come back (from fishing) when it is still cool.' \parencite[446]{Tarpent1987}

	\ex Context: Ten years ago Mary was in love with her husband John.\\
	\gll \textbf{Ḵ’ay}=t siip’in=s John gyuu’n=aa?\\
	still=3 like=\textsc{conn}.\textsc{pn} J. now=\textsc{q}\\
	\glt \lq Does she still love John now?’ \parencite[71]{Aonuki2021}
\end{exe}


\subsubsection{Broadly adverbial temporal-aspectual functions}
\paragraph{Near past}\label{appendixGitsxanNearPast}
\begin{itemize}
	\item \textcite{Aonuki2021}, \textcite[140–141]{Hunt1993}, and \textcite[445–447]{Tarpent1987}.
	\item This function occurs in two related contexts:
	\begin{itemize}
		\item With Vendlerian achievements (which have a bounded viewpoint by default, see \cite{JohansdottirMatthewson2007}). This is illustrated in (\ref{exGitsxanNearPast1}).
		\item With Vendlerian activities and accomplishments, when these are combined with \textit{hlis} \lq finish'. This is shown in (\ref{exGitsxanNearPast2}, \ref{exGitsxanNearPast3}).
	\end{itemize}	
	\item As discussed by \textcite{Aonuki2021}, and illustrated in  (\ref{exGitsxanNearPast4}), this reading does not involve a construal of an ongoing state.
	\item \textcite{Aonuki2021} points out that the near past is only felicitous if the context entails that the situation has been obtained; this is shown in (\ref{exGitsxanNearPast5}). She links this to the prior runtime presupposition of \textsc{still}.
\end{itemize}

\begin{exe}
	\ex\label{exGitsxanNearPast1}
	\gll \textbf{K̠'ay} gyuksxw-s t=Martin.\\
	still awake.3=\textsc{conn} \textsc{conn}.\textsc{pn}=Martin\\
	\glt \lq Martin just woke up.' \parencite[140]{Hunt1993}

	\ex\label{exGitsxanNearPast2}
	\gll \textbf{K̠'ay} hlis bax´=hl gimxdi-ʼy win ʼwitxw haʼw-iʼy kyʼoots.\\
	still finish run=\textsc{conn} sister-\textsc{poss}.1\textsc{sg} \textsc{subord} arrive go\_home-1\textsc{sg} yesterday\\
	\glt \lq My sister had just finished running when I came home yesterday.' \parencite[69]{Aonuki2021}
	
	\ex\label{exGitsxanNearPast3}
	\gll \textbf{K̠'ay} hlis=t jap=s Mary=hl gwila.\\
	still finish=3 make=\textsc{conn}.\textsc{pn} Mary=\textsc{conn} blanket\\
	\glt \lq Mary just made a blanket.'  \parencite[69]{Aonuki2021}
	
	\ex\label{exGitsxanNearPast4}
	\gll \textbf{Ḵ’ay}=t ’wa=s Mary=hl us-t ii ap hoo k’eeḵxw-t.\\
	still=3 find=\textsc{conn}.\textsc{pn} M.=\textsc{conn} dog-\textsc{poss}.3 but \textsc{decl} again run\_away-3\\
	\glt \lq Mary just found her dog, but it ran away again.’  \parencite[69]{Aonuki2021}
	
	\ex\label{exGitsxanNearPast5}
	\gll \textbf{Ḵ’ay}=t ’wa=s Mary=hl us-t=aa?\\
	still=3 find=\textsc{conn}.\textsc{pn} M=\textsc{conn} dog-\textsc{poss}.3=\textsc{q}\\
	\glt \lq Did Mary just find her dog? (felicitous if the speaker knows she has found her lost dog, not as an inquiry whether she found it or not)' \parencite[71]{Aonuki2021}
\end{exe}
\il{Gitxsan|)}\il{Nisga'a|)}

\section{Kalaallisut (kal, kala1399)}\label{appendixGreenlandic}\il{Kalaallisut|(}

\subsection{Introductory remarks}
Apart from descriptive materials, I consulted \citeauthor{BittnerTexts}'s (\citeyear{BittnerTexts}) online text collection.

\subsection{suli}
\subsubsection{General information}
\begin{itemize}
	\item Wordhood: free morpheme.
	\item Syntax: relatively free, although virtually all attestations in the data consulted precede the predicate, either immediately, or with one or more nominal arguments intervening; the only exception is (\ref{exAppendixGreenlandic3}) below.
	\item Etymology: from Proto-Eskimo \mbox{*\textit{cu}(\textit{na})\textit{li}} \lq still (more)', where \mbox{\textit{cu}(\textit{na}}) is an interrogative root \lq do what' and \textit{li} is probably a third person optative marker.
\end{itemize}

\subsubsection{As a \lq{}still\rq{ }expression}
\begin{itemize}
	\item \textcite[84]{Bergsland1955}, \textcite[128]{Bjornum2012}, \textcite[23]{Fortescue1984} and \textcite[90]{FortescueEtAl1984}; additional discussion throughout \textcite{vanBaar1997}.
	\item Specialisation: \textcite{vanBaar1997} identifies this marker as one that is in line with my definition; additional, albeit indirect, evidence comes from its use as \textsc{not yet} (\appref{appendixGreenlandicNotYet}).
	\item Polarity sensitivity: inner negation yields \textsc{not yet}. This also serves as a signal of precedence in temporal clauses (\lq while \textsc{not yet} \textit{p}' > \lq before \textit{p}'); see the numerous attestations throughout \citeauthor{BittnerTexts}'s (\citeyear{BittnerTexts}) text collection.
	\item Pragmaticity: according to \textcite[76, 103]{vanBaar1997}, in the unexpectedly late scenario \textit{suli} is augmented by marking the verb with the suffix \mbox{-\textit{juar}} \lq keep doing' (see \cite[281–282]{Fortescue1984} on the latter).
	\item Further notes: ex. (\ref{exAppendixGreenlandic4}) suggests that \textit{suli} is compatible with the anterior aspect \mbox{-\textit{sima}} under a resultative interpretation. 
\end{itemize}
\begin{exe}
	\ex
	\gll Ilianniartitsisur=mi \textbf{suli} atuarvim-miik-kallar-pa?\\
	teacher=what\_about still school--\textsc{loc}.\textsc{cop}-for\_time\_being-3\textsc{sg}.\textsc{q}\\
	\glt \lq But is the teacher still in school?' \parencite[10]{Fortescue1984}
	
	\ex
	Context: A father has run away with his baby son. He is teaching him how to dive.\\
	\gll Ulla-kut itir-lu-ni=lu irn-i \textbf{suli} sinit-tu-q annit-tar-paa sissa-mu=innaq.\\
	morning-\textsc{prolat} wake\_up-\textsc{ptcp}-3\textsc{sg}=and son-\textsc{poss}.3\textsc{sg} still sleep-\textsc{ptcp}-3\textsc{sg} take\_out-\textsc{hab}-\textsc{ind}:3\textsc{sg}>3\textsc{sg} shore-\textsc{dat}=only\\
	\glt \lq Every morning when he woke up, while his son was still asleep, he would take him out, always [down] to the shore.\rq{ }\parencite[Aataarsuup irnikasia]{BittnerTexts}
	
	\ex\label{exAppendixGreenlandic3}
	\gll Ataata-ma=mi=una niu-qqu-ga-anga; Nukappiara-u-ga-ma=lu \textbf{suli} tassa.\\
	father-\textsc{poss}.1\textsc{sg}=what\_about=this get\_off\_boat-want-\textsc{ptcp}-3\textsc{sg}>1\textsc{sg} boy-\textsc{cop}-\textsc{factual}.3\textsc{sg}.\textsc{a}-1\textsc{sg}.\textsc{p}=and still that's\_it\\
	\glt \lq Itʼs because papa wanted me to get off [your] boat. And being a boy still, that’s it.’ (From a translation of Hemingway’s \textit{The old man and the sea}, cited in \cite[370]{Bittner2005})

	\ex\label{exAppendixGreenlandic4}
	Context: A group of bears has shut their eyes.\\
	\gll Kinguni-ngaatsia-a-gut \textbf{suli} siqunngir-sima-llu-tik tusa-qa-qar-pu-t siqqu-alla-tuu-mik …\\
	after-a\_fair\_bit-3\textsc{sg}-\textsc{prol} still shut\_eyes-\textsc{ant}-\textsc{ptcp}-3\textsc{pl} hear-\textsc{relational}.\textsc{n}-have-\textsc{ind}-3\textsc{pl} bang-suddenly-\textsc{nmlz}-\textsc{sg}.\textsc{mod}\\
	\glt \lq Some time later, while they were still (sitting) with their eyes shut they heard a sudden bang.' \parencite[Silliarnamik uqaluttuaq]{BittnerTexts}
\end{exe}

\subsubsection{Uses related to other phasal polarity concepts}
\paragraph{Not yet}\label{appendixGreenlandicNotYet}
\begin{itemize}
	\item \textcite[23–24]{Fortescue1984}; additional discussion is found in \textcite[294]{vanBaar1997}.
	\item This function is only attested as a negative answer to a polar question.
\end{itemize}

\begin{exe}
	\ex
	\gll Sinippa? – \textbf{Suli} (\textit{naamik}).\\
	sleep-3\textsc{sg}.\textsc{q} {} still \phantom{(}\textsc{neg}\\
	\glt \lq Is he sleeping? -- Not yet.' \parencite[24]{Fortescue1984}
\end{exe}

\subsubsection{Broadly adverbial temporal-aspectual functions}
\paragraph{Iterative via increment}
\label{appendixGreenlandicIterativeIncrement}
\begin{itemize}
	\item One token in the data consulted (\ref{exAppendixGreenlandicIterativeIncrement}) appears to be a case of an iteration going back to additivity \lq still one time\rq{} > \lq again\rq{}. 
	\item It is noteworthy that no clear-cut cases of additive uses are attested in the data, except for comparisons of inequality (\appref{appendixKalaallisutComparatives}). However, \textcite[90]{FortescueEtAl1984} list \lq more' as a sense of Proto-Eskimo \mbox{*\textit{cu}(\textit{na})\textit{li}}, and such an additive-incremental function is attested, for instance, with Central Alaskan Yupik\il{Yupik, Central Alaskan} \textit{cali}  \parencite{Miyaoka2012}. That is, the use in (\ref{exAppendixGreenlandicIterativeIncrement}) might be a reflex of an older function that has become lost/superseded by other markers.
\end{itemize}

\begin{exe}
	\ex\label{exAppendixGreenlandicIterativeIncrement}
	Context: Whitey has been told to take a closer look at a mare and her colt. He has circled them once and not noticed anything.\\
	\gll \lq\lq Aju-quti-qar-nir-su-q taku-sinnaa-nngi-la-ra" Whitey uaqr-pu-q \textbf{suli} \textbf{ataasi}-\textbf{iar}-lu-ni histi arnaviaq pi-ara-a=lu kajalla-riar-llu-gut.\\
	be\_bad-cause-have-wonder-\textsc{ptcp}-3\textsc{sg} see-be\_able-\textsc{neg}-\textsc{ind}-1\textsc{sg}>3\textsc{sg} W. say-\textsc{ind}-3\textsc{sg} still one-do\_times-\textsc{ptcp}-3\textsc{sg} horse female do-little-3\textsc{sg}>3\textsc{sg}=and circle-after-\textsc{ptcp}-3\textsc{pl}\\
	\glt \lq \lq\lq I can’t see anything that’s wrong with him", said Whitey, after circling the mare and her colt one more time.\rq{ }\parencite[Hesti piaraq tappiitsoq]{BittnerTexts}
\end{exe}


\subsubsection{Additive and related functions}

\paragraph{Comparisons of inequality}\label{appendixKalaallisutComparatives}
\begin{itemize}
	\item \textit{Suli} is attested with comparisons of inequality, where it adds the notion of \lq even\rq{}.
	\item Note that comparisons of inequality in Kalaallisut are construed via one of several comparative markers on the predicate plus, optionally, a case-marked nominal representing the standard of comparison; see \textcite[167–170]{Fortescue1984}.
\end{itemize}

\begin{exe}
	\ex Context: When a man is overwhelmed by thoughts…\\
	\gll Taama pisoq-ar-aang-at uagut ajukkuk-uttu-reer-su-gut \textbf{suli} ajukku-nniru-ler-sar-pu-gut.\\
	thus event-have-\textsc{hab}-3\textsc{sg} 1\textsc{pl} feel\_small-exceedingly-already-\textsc{ptcp}-1\textsc{pl} still feel\_small-more-begin-\textsc{hab}-\textsc{ind}-1\textsc{pl}\\
	\glt \lq When that happens, we, who already feel exceedingly small, feel smaller still.' \parencite[A shaman’s definition of poetry]{BittnerTexts}

	\ex Context: Travelling (as a captive) through the prairie, an Indian girl is astounded by the vastness of the land.\\
	\gll Qaqqar-sui-t kiisa anngar-paat. Naarsar-suaq suurlu=mi \textbf{suli} al-li-artur-tu-q.\\
	mountain-big-\textsc{pl} finally lose\_sight\_of-\textsc{ind}:3\textsc{pl}>3\textsc{pl} plain-big.\textsc{sg} as\_if=what\_about still be\_big-more-more\_and\_more-\textsc{ptcp}-3\textsc{sg}\\
	\glt \lq In the end, they could no longer see mountains. It was as if the great prairie were growing even larger.'\footnote{The marker \mbox{(\textit{j})\textit{artur}} marks a continuous, stepwise increase \parencite[282]{Fortescue1984}. Also see \textcite{vanGeenhoven2005}, among others, for a more general discussion of such expressions.}	\parencite[Naya Nuki: Niviarsiaraq qimaasuq]{BittnerTexts}
\end{exe}
\il{Kalaallisut|)}

\section{Kekchí (kek, kekc1242)}\label{appendixKekchi}\il{Kekchí|(}
\subsection{toj}

\subsubsection{General information}
\begin{itemize}
	\item Wordhood: free morpheme.
	\item Etymology: its functional range suggest an spatial limitative \lq until\rq{ }as the diachronically original meaning.
\end{itemize}

\paragraph{As a \textsc{still} expression}
\begin{itemize}
	\item  \textcite[171, 303]{VocabularioKechi2004}, \citeauthor{Kockelman2010}, (\citeyear[105–106]{Kockelman2010}, \citeyear{Kockelman2020}) and \textcite[360]{SamJuarezEtAl2003}.
	\item Specialisation: \textcite{Kockelman2020} explicitly addresses the two components of my definition, showing that \textit{toj} involves persistence, defeasibly implies a later discontinuation and is incompatible with inalterable states.
	\item Polarity sensitivity: according to \textcite{Kockelman2020} \textit{toj} does not form part of a negative phasal polarity expression, although it can augment the \textsc{not yet} item \textit{maaji'}‚ in which case it stresses the absence of change and often appears to signal the unexpectedly late scenario. There are, however, a few scattered attestations of \textit{toj} plus negation in \citeauthor{Kockelman2010} (\citeyear{Kockelman2010}, \citeyear{Kockelman2020}) and \textcite{SamJuarezEtAl2003} that are translated as \lq todavía no' / \lq still … not'. All of these feature the negative existential \textit{maak'a'}. This suggests that, in this specific environment, \textit{toj} is involved in the expression of \textsc{not yet}.
		\item Pragmaticity: compatible with both scenarios (tentative conclusion). Example (\ref{exAppendixKekchi4}) is a prime candidate for the unexpectedly late scenario.
	\item Syntax: clause-initial position, can be preceded only by conjunctions and question markers.
	\item Further note: cannot be used as a one-word answer \parencite[459]{Kockelman2020}.
\end{itemize}

\begin{exe}
	\ex Context: About a dog waiting at the entrance of a small restaurant.\\
	\gll Ma xkoo? – Maaji' na-xik, \textbf{toj} wan.\\
	\textsc{q} 3\textsc{sg}.go.\textsc{pfv} {} not\_yet \textsc{prs}.3\textsc{sg}-go still \textsc{exist}.3\textsc{sg}\\
	\glt \lq Has it gone? -- It has not yet gone. It is still there.\rq{ }\parencite[450]{Kockelman2020}
	
	\ex
	\gll \textbf{Toj} yoo chi wa'ak naq x-in-k'ulun.\\
	still \textsc{prs}.do.3\textsc{sg} \textsc{subord} eat \textsc{comp} \textsc{pfv}-1\textsc{sg}-arrive\\
	\glt \lq He was still eating when I arrived.'   \parencite[461]{Kockelman2020}
	
	\ex\label{exAppendixKekchi4}
	\gll Nim chik li al ut \textbf{toj} na-tu’uk.\\
	big more \textsc{det} boy \textsc{conj} still \textsc{prs}.3\textsc{sg}-suckle\\
	\glt \lq El niño ya está grande y todavía mama. [The boy is big already and he still suckles.]' (\cite[172]{VocabularioKechi2004}, glosses added)
\end{exe}

\subsubsection{Uses on the fringes of \lq{}still\rq{}}\label{appendixKekchiScalar}
\begin{itemize}
	\item \textit{Toj} is attested in contexts of (potential) scalar decreases.
\end{itemize}

\begin{exe}

	\ex
	\gll \textbf{Toj} naab’al li w-aq’im wank.\\
	still much \textsc{det} \textsc{poss}.1\textsc{sg}-clean \textsc{exist}.3\textsc{sg}\\
	\glt \lq Todavía tengo mucho que limpiar. [I still have a lot of cleaning to do.]\rq{ }(\cite[30]{VocabularioKechi2004}, glosses added) 
	
	\ex 
	\gll \textbf{Toj} wan nab'al-eb' inloq'onel ink'a' n-in-ru x-kanab'-ank-il in-k'ay.\\
	still \textsc{exist}.3\textsc{sg} much-3\textsc{pl} \textsc{poss}.1\textsc{sg}-buyer \textsc{neg} \textsc{prs}-1\textsc{sg}-able \textsc{poss}.3\textsc{sg}-leave-\textsc{mid}-\textsc{nmlz} \textsc{poss}.1\textsc{sg}-sale\\
	\glt \lq Todavía tengo muchos compradores, no puedo abandonar mi venta. [I still have many customers, I can't leave my shop.]\rq{ }(\cite[196]{SamJuarezEtAl2003}, glosses added)

\end{exe}

\subsubsection{Broadly adverbial temporal-aspectual functions}
\paragraph{Near past} \label{appendixKekchiImmediatePast}
\begin{itemize}
	\item \textcite[171, 303]{VocabularioKechi2004}, \textcite[202]{EachusCarlson1980}, \textcite{Kockelman2020} and \textcite[360]{SamJuarezEtAl2003}.
	\item In this function \textit{toj} combines with the subordinator \textit{naq}. This collocation is typically reduced to \textit{toja'}\sim\textit{tojo'}\sim\textit{toje'} and tied to the clause-initial position.	The verb in the subordinate clause stands in the perfective aspect inflection.
	\item This use signals an immediate past, often translated into Spanish using the \textit{acabar de}-\textsc{inf}  \lq have just V-ed\rq{ }construction. Note the clearly anterior viewpoint and past-in-the-past semantics in (\ref{exAppendixKekchiSequencing5}).
\item \textcite{Kockelman2020} links this to the temporal restrictive function of \textit{toj} (\appref{appendixKekchiRestrictive}) with an anaphoric topic time, i.e. \lq it is not until now/then when …\rq{}, in a similar fashion to how \textit{toj} as an exponent of \textsc{still} likely goes back to its \lq until\rq{ }function (\appref{appendixKekchiUntil}) plus zero anaphora \lq until now/then\rq{}. This interpretation would be in line with the occasional token of \textit{toja\rq{}} and its variants together with the imperfective present, always with an ingressive meaning (\ref{exAppendixKekchiOnlyNow})
\end{itemize}

\begin{exe}
	\ex\label{exAppendixKekchiSequencing5}
	Context: Describing the time frame of a scary event narrated moments earlier.\\
	\gll Toj maak'a'-q qa-kok'al, \textbf{toja'} k-oo-sumlaak.\\
	still \textsc{neg}.\textsc{exist}-\textsc{non}.\textsc{specific} \textsc{poss}.1\textsc{pl-}children still.\textsc{subord} \textsc{pfv}.\textsc{evid}-1\textsc{pl}-marry\\
	\glt \lq We still had no children. We had just married.\rq{ }\parencite[468]{Kockelman2020}

	\ex\label{exAppendixKekchiSequencing6}
	\gll \textbf{Toje'} x-c’ulun arin Cobán.\\
	still.\textsc{subord} \textsc{pfv}.3\textsc{sg}-arrive here C.\\
	\glt \lq Hace poco que vino aquí a Cobán. [He just recently came here to Cobán.]' (\cite[202]{EachusCarlson1980}, glosses added)
	
	\ex\label{exAppendixKekchiOnlyNow}
		\gll \textbf{Toja\rq} yo chi nume\rq{}k in-\rq{}oj.\\
	still.\textsc{subord} do.\textsc{prs}.3\textsc{sg} \textsc{comp} pass \textsc{poss}.1\textsc{sg}-cough\\
	\glt \lq Hasta ahora se me está quitando el catarro. [\textbf{Only now} my cold is passing.]\rq{ }(\cite[231]{SamJuarezEtAl2003}, glosses added)
\end{exe}

\pagebreak
\largerpage
\paragraph{Event sequencing: \textit{toja'}/\textit{tojo'}/\textit{toje'} \textit{naq}}
\label{appendixKekchiSequencing}
\begin{itemize}
	\item \textcite[171]{VocabularioKechi2004}, \textcite[202]{EachusCarlson1980}, \textcite{Kockelman2020} and \textcite[360]{SamJuarezEtAl2003}.	
	\item Form: in a clause headed by \textit{toja'}\sim\textit{tojo'}\sim\textit{toje'} \textit{naq}, where the first item is the reduced form of \textit{toj naq} \lq still \textsc{subord}\rq{ }that is also found in the immediate past use  (\appref{appendixKekchiImmediatePast}). The seemingly redundant second instance of the subordinator \textit{naq} is likely to be facilitated by \textit{toja'}\sim\textit{tojo'}\sim\textit{toje'} no longer being perceived as compositional, in conjunction with a structural parallelism to a preceding temporal clause introduced by \textit{naq}, as in (\ref{exAppendixKekchiSequencing3}). 
	\item This use occurs in sequential contexts, where it depicts an event occurring as right at (or closely after) the time established by a preceding event. Often this brings about notions of a sudden or immediate development, as in (\ref{exAppendixKekchiSequencing2}–\ref{exAppendixKekchiSequencing4}).
	\item As \textcite{Kockelman2020} discusses, this can be linked to the temporal restrictive function of \textit{toj} (\appref{appendixKekchiRestrictive}) together with an anaphorically retrieved topic time, i.e. \lq (not until) then (is when)'.
\end{itemize}
\begin{exe}
	\ex\label{exAppendixKekchiSequencing1}
	\textit{Tacuokxi li ha' o'laju minutos.}\\
	\lq Hierva el agua qunice minutos. [Boil the water for fifteen minutes.]\rq
	\sn\gll \textbf{Toj’o} \textbf{nak} t-a-canab chi quehoc’.\\
	still.\textsc{subord} \textsc{subord} \textsc{prosp}-2\textsc{sg}-leave \textsc{comp} cool\_down\\
	\glt \lq Después déjala enfriar. [Then let it cool down.]\rq{ }(\cite[122]{EachusCarlson1980}, glosses added)
	
	\ex\label{exAppendixKekchiSequencing2}
	\gll Xb’eenwa t-oo-wa’aq \textbf{tojo’} \textbf{naq} t-oo-xik.\\
	first  \textsc{prosp}-1\textsc{pl}--eat still.\textsc{subord} \textsc{subord} \textsc{prosp}-1\textsc{pl-}leave\\
	\glt \lq Primero comeremos y luego nos vamos. [We'll eat first, then we'll leave.]\rq{ } (\cite[193]{VocabularioKechi2004}, glosses added)
	
	\ex\label{exAppendixKekchiSequencing3}
	\gll Naq x-e'-raq-e' chi x-b'anunk-il, \textbf{toja'} \textbf{naq} x-e'-ok chi x-k'at-b'al.\\
	\textsc{subord} \textsc{pfv}-3\textsc{pl}-finish-\textsc{pass} \textsc{comp} \textsc{poss}.3\textsc{sg}-do-\textsc{nmlz} still.\textsc{subord} \textsc{subord} \textsc{pfv}-3\textsc{pl}-start \textsc{comp} \textsc{pfv}.3\textsc{sg}-burn-\textsc{nmlz}\\
	\glt \lq When they finished doing that, then (immediately) they began to burn it.' \parencite[468]{Kockelman2020}
	
	\ex\label{exAppendixKekchiSequencing4}
	\gll X-c'ulun ut \textbf{tojo'} \textbf{nak} x-oc chi cua'ac.\\
	\textsc{pfv}-3\textsc{sg}.arrive \textsc{conj} still.\textsc{subord} \textsc{subord} \textsc{pfv}-3\textsc{sg}.start \textsc{comp} eat\\
	\glt \lq Vino, entonces empezó a comer. [S/he came, then s/he started to eat.]' (\cite[202]{EachusCarlson1980}, glosses added)
\end{exe}

\subsubsection{Temporal connectives and frame setters}
\paragraph{Temporal limitative}\label{appendixKekchiUntil}
\largerpage[2]
\begin{itemize}
	\item \textcite[171]{VocabularioKechi2004} and \textcite{Kockelman2020}.
	\item This function obtains with imperatives (\ref{exAppendixKekchiTemporalLimitBrowseMarket}), continuous and negated predicates (\ref{exAppendixKekchiSchoolNeg}) and, generally, contexts in which \textit{toj} takes scope over a clause featuring other phasal polarity expressions (\ref{AppendixKekchiTemporalLimitDebt}). In all other cases involving affirmative polarity, \textit{toj} is understood as \lq not until, when (after)\rq{}; see \appref{appendixKekchiRestrictive}. Elliptical examples like (\ref{exAppendixKekchiTemporalLimitWife}) can be read both ways, ultimately resulting in the same interpretation. A similar case holds in (\ref{exAppendixKekchiRatPeed}) below. 
	\item Also see \appref{appendixKekchiSpatialLimit} for the same \lq until\rq{ }reading with spatial complements.
\end{itemize}
\begin{exe}
	\ex\label{exAppendixKekchiTemporalLimitBrowseMarket}
	\gll B’eeni chaq li k’ayiil \textbf{toj} r-eetal naq t-aa-taw chaq li pix.\\
	travel.\textsc{imp} \textsc{loc} \textsc{det} market still \textsc{poss}.3-sign \textsc{subord} \textsc{prosp}-2\textsc{sg}-find \textsc{loc} \textsc{det} tomato\\
	\glt \lq Recorre todo el mercado hasta que encuentres tomate. [Browse the entire market until you find tomatoes.]' (\cite[46]{VocabularioKechi2004}, glosses added)

	\ex\label{exAppendixKekchiSchoolNeg}
	\gll Ink'a' nek-e'-xik sa' li tz'oleb'al \textbf{toj} wan r-e waqib' chihab'.\\
	\textsc{neg} \textsc{prs}-3\textsc{pl}-go \textsc{prep} \textsc{det} school still \textsc{exist}.3\textsc{sg} 3\textsc{sg}-\textsc{dat} six year\\
	\glt \lq They do not go to school until they are six years old.\rq{ }\parencite[466]{Kockelman2020}

	\ex\label{AppendixKekchiTemporalLimitDebt}
	\gll \textbf{Toj} maak’a’-q chik in-k’as t-in-k’anjelaq.\\
	still \textsc{neg}.\textsc{exist}-\textsc{non}.\textsc{specific} more \textsc{poss}.1\textsc{sg}-debt \textsc{prosp}-1\textsc{sg}-work\\
	\glt \lq Until I no longer have debt I will work.\rq{ }\parencite[480]{Kockelman2020}

	\ex\label{exAppendixKekchiTemporalLimitWife}
	\begin{xlist}
		\exi{A:}\textit{Maak'a' li aatinak hoon rik'in laawixaqil?}\\
		\lq There is no speaking with your wife today?\rq
		
		\exi{B:}\textit{Ink'a'.}\\
		\lq No.'
		
		\exi{A:}\gll \textbf{Toj} kab'ej?\\
		still tomorrow\\
		\glt \lq (Not) until tomorrow?'
		
		\exi{B:}\textit{Eq'ela kab'ej.}\\
		\glt \lq Early tomorrow.' \parencite[465]{Kockelman2020}
	\end{xlist}
\end{exe}

\paragraph{Temporal restrictive}\label{appendixKekchiRestrictive}
\begin{itemize}
	\item \textcite[171]{VocabularioKechi2004} and \textcite{Kockelman2020}.
	\item This function obtains with affirmative polarity and, usually, a non-continuous viewpoint; other combinations yield limitative \lq until\rq{ }(\appref{appendixKekchiUntil}). In actual discourse, this often bleeds into a purely positional sense \lq when, after\rq{}.
	
	\item As \textcite{Kockelman2020} discusses, there is a noticeable overlap in usage patterns between \textit{toj} as \lq until\rq{ }(\appref{appendixKekchiUntil}) plus negated predicates and the \lq not until\rq{ }function, in that both both cases often involve some situation that lasts until the time denoted by the complement of \textit{toj}, and a second situation that takes place no earlier than that. Both, the latter and the positional \lq when\rq{ }reading, can be observed in (\ref{exAppendixKekchiRatPeed}). Thus, Lord B’alamq’e's discontentment lasts until placing the rat inside the moon and it peeing there, the latter event being the trigger for a change to the better and anaphorically referenced by \textit{o\rq{}ka\rq{}in} \lq then, thus\rq{}. Seen this way, the readings of \textit{toj} as limitative \lq until\rq{ }and restrictive/positional \lq not until, when\rq{ }only differ in the (inter-)clausal relationships (i.e. whether \textit{toj} plus complement are understood as a continuation of the first sentence, or as establishing a new unit), with the temporal and causal relationships remaining stable across the two readings.
\end{itemize}

\begin{exe}

	\ex\label{exAppendixKekchiTemporalLimitAfternoon}
	\gll \textbf{Toj} ewu t-in-xik.\\
	still afternoon \textsc{prosp}-1\textsc{sg}-go\\
	\glt \lq I'll go [no earlier than] in the afternoon.'\\
	Speaker's Spanish gloss: \lq Iré por la tarde.' \parencite[464]{Kockelman2020} 
	
	\ex\label{exAppendixKekchiSchool}
	\gll Nek-e'-xik sa' li tz'oleb'al \textbf{toj} wan r-e waqib' chihab'.\\
	\textsc{prs}-3\textsc{pl}-go \textsc{prep} \textsc{det} school still \textsc{exist}.3\textsc{sg} 3\textsc{sg}-\textsc{dat} six year\\
	\glt \lq They go to school [not until] when they are six years old.\rq{ }\parencite[466]{Kockelman2020}


	\ex \label{exAppendixKekchiRatPeed}
	\gll Maa-min ki-hu[u]lak chu r-u qaawa\rq{} b\rq{}alamq\rq{}e, \textbf{toj} ki-x-k\rq{}e li ch\rq{}o chi \textup{(}x\textup{)}-sa\rq{} li po, ut li ch\rq{}o aran ki-chu\rq{}uk, jo\rq{}ka\rq{}in ki-usa.\\
		\textsc{neg}-\textsc{emph} \textsc{pfv}.\textsc{evid}.3\textsc{sg}-arrive \textsc{prep} \textsc{poss}.3\textsc{sg}-face \textsc{hon} B. still \textsc{pfv}.\textsc{evid}:3\textsc{sg}.\textsc{p}-3\textsc{sg}.\textsc{a}-give \textsc{det} rat \textsc{prep} \textsc{poss}.3\textsc{sg}-inside \textsc{det} moon \textsc{conj} \textsc{det} rat there \textsc{pfv}.\textsc{evid}:3\textsc{sg}-urinate thus \textsc{pfv}.\textsc{evid}:3\textsc{sg}-improve\\
\glt \lq In no way was Lord B’alamq’e pleased, but when (lit. not until) he placed the rat inside the moon, and the rat peed there, then it improved.\rq{ }\parencite[236]{Kockelman2010}
\end{exe}



\subsubsection{Additive and related uses}

\paragraph{Spatial limitative (\lq until\rq)}
\label{appendixKekchiSpatialLimit}
\begin{itemize}
	\item  \textcite[171]{VocabularioKechi2004}, \textcite[189]{EachusCarlson1980} and \textcite{Kockelman2020}.
	\item In this function, \textit{toj} takes a phrase denoting a point or region in space, up to which a spatial trajectory extends, as its complement.
	\item Given the well-known extension from spatial language into the temporal domain, this function may constitute the diachronic source for all other functions of \textit{toj}.
\end{itemize}
\begin{exe}
	\ex
	\begin{xlist}
		\exi{A:}\textit{B'ar naxik li manguera?}\\
		\lq Where does the hose go?\rq{}
		\exi{B:}\gll  Ay ink'a' n-in-naw, mare arin \textbf{toj} najt chi-r-ix li tzuul.\\
	\textsc{interj} \textsc{neg} \textsc{prs}.3\textsc{sg}.\textsc{p}-1\textsc{sg}.\textsc{a}-know perhaps here still far \textsc{prep}-\textsc{poss}.3\textsc{sg}-back \textsc{det} mountain\\
	\glt \lq Ay, I don't know, perhaps (from) here until far over the hill.' \parencite[464]{Kockelman2020}
	\end{xlist}

	\ex
	\gll X-oo-hulak chaq \textbf{toj} sa’ r-u’uj li k’u.\\
	\textsc{pfv}-1\textsc{pl}-reach \textsc{loc} still \textsc{prep} \textsc{poss}.3\textsc{sg}-top \textsc{det} volcano\\
	\glt \lq Hasta la cima del volcán llegamos. [We got until the top of the volcano.]' (\cite[115]{VocabularioKechi2004}, glosses added)
\end{exe}

\subsubsection{Broadly modal and interactional functions}
\paragraph{Concessive apodoses}
\label{appendixKekchiConcessiveConsequent}
\begin{itemize}
	\item \citeauthor{Kockelman2020} (\citeyear[106]{Kockelman2010}, \citeyear{Kockelman2020}).
	\item This use is described as being rather infrequent. 
	\item That this is not merely a contextual inference of a \textsc{still} use becomes clear from the fact that this reading primarily obtains with perfective predicates, as in (\ref{exappendixKekchíConcessive1}). Also note the negation within the scope of \textit{toj} in (\ref{exappendixKekchíConcessive2}). 
\end{itemize}

\begin{exe}
	\ex\label{exappendixKekchíConcessive1}
	\gll M–at–xik, m–at–xik x–in–ye, ab’anan \textbf{toj} x-'el chaq.\\
	\textsc{proh}-2\textsc{sg}-go \textsc{proh}-2\textsc{sg}-go \textsc{pfv}:3\textsc{sg}.\textsc{p}-1\textsc{sg}.\textsc{a}-say \textsc{conj} still \textsc{pfv}.3\textsc{sg}-leave \textsc{loc}\\
	\glt \lq {\lq\lq}Don't go, don't go", I said. But he still went / But he went anyway.' \parencite[461]{Kockelman2020}

	\ex\label{exappendixKekchíConcessive2}
	Context: Moon's father has asked Thunder to kill Moon and Sun.\\
	\gll \textbf{Toj} a'an ink'a' ki-r-aj.\\
	still 3\textsc{sg} \textsc{neg} \textsc{pfv}.\textsc{evid}:3\textsc{sg}.\textsc{p}-3\textsc{sg}.\textsc{a}-desire\\
	\glt \lq He still didn't want to kill them (despite his brother's wishes).' \parencite[462–463]{Kockelman2020}
\end{exe}

\paragraph{Exceptive conditionals (\lq unless\rq)}\label{appendixKekchiUnless}
\begin{itemize}
	\item \citeauthor{Kockelman2010} (\citeyear[106]{Kockelman2010}, \citeyear[467]{Kockelman2020}).
	\item As suggested by \textcite{Kockelman2020}, this is best considered an extension of temporal \lq until\rq{ } (\appref{appendixKekchiUntil}), with a mapping from times to possible worlds: the situation described in the foregrounded clause holds true until (and defeasibly no longer after) the preventing condition is met. 
\end{itemize}

\begin{exe}
	\ex Context: A hummingbird explaining why he does not want to give away its feathers.\\
	\gll T-in-kaamq rah \textup{(}x\textup{)}-b'aan ke \textbf{toj} t-in-b'at-e'q sa' x-noq'al inup.\\
	\textsc{prosp}-1\textsc{sg}-die \textsc{cf} \phantom{(}\textsc{poss}.3\textsc{sg}-because cold still \textsc{prosp}-1\textsc{sg}-wrap-\textsc{pass} \textsc{prep} \textsc{poss}.3-thread ceiba\\
	\glt \lq I will die of the cold unless I am wrapped in the bark of a ceiba tree.' \parencite[467]{Kockelman2020}
\end{exe}
\il{Kekchí|)}

\section{Maricopa (mrc, mari1440)}
\il{Maricopa|(}
\subsection{Introductory remarks}
Maricopa has two candidates for \textsc{still} expressions: \mbox{-\textit{skiit}} and \mbox{-\textit{haay}}. Only for the latter are there examples of additional functions in the data consulted.

\subsection{-haay}

\subsubsection{General information}
\begin{itemize}
	\item Wordhood: bound morpheme (verb suffix).
\end{itemize}

\subsubsection{As a \lq{}still\rq{ }expression}
\begin{itemize}
	\item \textcite[142–145]{Gordon1986}.
	\item Specialisation: examples like (\ref{exAppendixMaricopa1}, \ref{exAppendixMaricopa2}) give a fairly good indication that this marker conforms to my definition. For instance, (\ref{exAppendixMaricopa1}) evokes an alternative scenario in which J.P. is no longer small and can therefore take care of himself.
	\item Polarity sensitivity: inner negation yields \textsc{not yet}, outer negation yields \textsc{no longer}.
	\item Pragmaticity: the data allow no conclusions.
\end{itemize}

\begin{exe}

	\ex\label{exAppendixMaricopa1}
	\gll J.P.-sh nnooq-\textbf{haay}-m aany-m wi-m ntay-sh\\
	J.P.-\textsc{subj} small-still-\textsc{ds} \textsc{dem}-\textsc{assoc} do-\textsc{rl} mother-\textsc{subj}\\
	\glt \lq Because J.P. is still young, his mother takes care of him.\rq{ }\parencite[281]{Gordon1986}

	\ex\label{exAppendixMaricopa2}
	\gll Shmaa-\textbf{haay}-k v-ny-dik-m '-nchen-sh 'ayuu-m uuiiv-k.\\
	sleep-\textsc{still}-\textsc{ss} \textsc{dem}-when.1\textsc{sg}-lie-\textsc{ds} \textsc{poss}.1-older\_sibling-\textsc{subj} something-\textsc{assoc} work.\textsc{pl}-\textsc{rl}\\
	\glt \lq While I was still asleep, my brothers were working.' \parencite[269]{Gordon1986}

\end{exe}

\subsubsection{Broadly adverbial temporal-aspectual functions}
\paragraph{Near past}
\label{appendixMaricopaNearPast}
\begin{itemize}
	\item \textcite[142–143]{Gordon1986}.
	\item This is often accompanied by a form of \textit{mpis} \lq now' in the clause, as in (\ref{exAppendixMaricopaNearPast2}). However, this is by no means compulsory, as shown in (\ref{exAppendixMaricopaNearPast1}).
	\item Of the two examples in \textcite{Gordon1986}, one features a Vendlerian accomplishment plus the perfective aspect (\ref{exAppendixMaricopaNearPast1}). In Maricopa, verbs other than Vendlerian achievements (and, possibly, semelfactives) distinguish between a perfective and imperfective paradigm in the realis mood. With achievements, such as in the other relevant example (\ref{exAppendixMaricopaNearPast2}), the realis mood alone yields a bounded viewpoint (cf. \cite[102–103]{Gordon1986}).
	\item For the Quechuan cognate \mbox{-\textit{xay}} \textcite[284]{Halpern1946} lists \lq\lq no sooner than\rq\rq{ }as one of its meanings. This likely refers to the same function.
\end{itemize}
\begin{exe}
	\ex\label{exAppendixMaricopaNearPast1}
	\gll 'ayuu '-maa-\textbf{haay}-ksh.\\
	something 1-eat-still-\textsc{pfv}.1\\
	\glt \lq I just ate.' \parencite[143]{Gordon1986}

	\ex\label{exAppendixMaricopaNearPast2}
	\gll Mpis-han puu-\textbf{haay}-ksh.\\
	now-\textsc{rl} die-still-\textsc{rl}\\
	\glt \lq He just now died.' \parencite[143]{Gordon1986}
\end{exe}

\subsubsection{Temporal connectives and frame setters}
\paragraph{Simultaneous duration}\label{appendixMaricopaWhile}
\begin{itemize}
	\item \textcite[131–132, 270–271]{Gordon1986}.
	\item Form: only in collocation with inessive \mbox{-\textit{ly}}.
	\item Negation of the predicate transparently yields precedence (\lq while not yet \textit{p}, \textit{q}\rq{} > \lq before \textit{p}, \textit{q}\rq{}).
\end{itemize}

\begin{exe}
	\ex
	\gll '-ashvar-\textbf{haay}-\textbf{ly} '-nchen-sh iima-k.\\
	1-sing-still-in \textsc{poss}.1-older\_sibling-\textsc{subj} dance-\textsc{rl}\\
	\glt \lq While I sang, my older sibling danced.' \parencite[132]{Gordon1986}

	\ex
	\gll Nyaa ny-yuu-k '-uvaa-\textbf{haay}-\textbf{ly} m-kwnyminy-k.\\
	1\textsc{sg} 1>2-see-\textsc{ss} 1-\textsc{loc}.\textsc{cop}-still-in 2-different-\textsc{rl}\\
	\glt \lq As I was looking at you, you changed.' \parencite[270]{Gordon1986}

	\ex
	\gll M-nak-k m-uuvaa-\textbf{haay}-\textbf{ly} dany nym-k-ev-k.\\
	2-sit-\textsc{ss} 2-\textsc{loc}.\textsc{cop}-still-in \textsc{dem} \textsc{dem}.\textsc{assoc}-\textsc{imp}-work-\textsc{rl}\\
	\glt \lq While you are sitting there, work on this.' \parencite[270]{Gordon1986}
\end{exe}
\il{Maricopa|)}

\section{Osage (osa, osag1243)}
\il{Osage|(}
\subsection{šó̜}
\subsubsection{General information}
\begin{itemize}
	\sloppy
	\item Wordhood: independent grammatical word, but can take suffixes.
	\item Syntax:	invariably following the predicate and preceding the positional\slash continuative auxiliaries. The only other two markers occurring in the same slot of the verbal template are habitual \textit{na̜} and frequentative \textit{šta̜}.
\end{itemize}

\subsubsection{As a \lq{}still\rq{ }expression}
\begin{itemize}
	\item \citeauthor{Quintero1997} (\citeyear[305]{Quintero1997}, \citeyear[307–308, 410, 446]{Quintero2004}, \citeyear[208]{QuinteroDictionary}).
	\item Specialisation: inclusion of this marker is more tentative than that of most expressions in my sample. It is  based on examples like (\ref{exAppendixOsage1}–\ref{exAppendixOsage3}), where (\ref{exAppendixOsage1}) appears to suggests a possible discontinuation scenario.
	\item Polarity sensitivity: not attested in combination with negation.
	\item Pragmaticity: the data allow no conclusions; example (\ref{exAppendixOsage1}) might be taken as involving an unexpectedly late continuation.
\end{itemize}

\begin{exe}
	\ex\label{exAppendixOsage1}
	\gll ði̜̜i̜hó̜ kše wažáže íe wakó̜ze \textbf{šó̜} ði̜kše?\\
	your\_mother lie Osage language teach still lie\\
	\glt \lq Is your [bedridden] mother still teaching Osage?' \parencite[357]{Quintero2004}

	\ex\label{exAppendixOsage2}
	\gll Wawépaho̜ a-ní \textbf{šó̜} a̜hé.\\
	witness \textsc{pvb}:1\textsc{sg}.\textsc{a}-live still 1\textsc{sg}.\textsc{cont}\\
\glt \lq I am a witness and Iʼm still living / As a witness I live still.\rq{ }\parencite[410]{Quintero2004}

	\ex\label{exAppendixOsage3}
	\gll Híi ðáalí̜ \textbf{šó̜} ðaašé.\\
	tooth good still 2\textsc{sg}.\textsc{cont}\\
	\glt \lq Your teeth are still good.' (\cite[208]{QuinteroDictionary}, glosses added)
\end{exe}

\subsubsection{Broadly adverbial temporal-aspectual functions}
\paragraph{Always, forever, the entire time: \textit{š}(\textit{ó̜})\textit{ó̜šó̜we}, \textit{š}(\textit{ó̜})\textit{ó̜ðšéðe}}\label{appendixOsageAlways}
\begin{itemize}
	\item \citeauthor{Quintero1997} (\citeyear[305, 362–363]{Quintero1997}, \citeyear[87, 444]{Quintero2004}, \citeyear[209]{QuinteroDictionary}).
	\item Form, syntax: in reduplicated forms \textit{šó̜ó̜šó̜we}\sim\textit{šo̜ó̜šó̜we}, \textit{šó̜ó̜ðéðe}\sim\textit{šo̜ó̜ðéðe}, all of which are based based on \mbox{\textit{šó̜}-{ðe}} \lq still-\textsc{prox}' (see \appref{appendixOsageWhile} on the latter). This composition can be taken as an indication that this use is derived from the marking of simultaneous duration (\appref{appendixOsageWhile}) via \lq during that time' > \lq during all times', which is in line with the semantics of reduplication in Osage (cf. \cite[87]{Quintero2004}). What is more, unlike \textit{šó̜} in phasal polarity function, the items in question can precede the VP, as in (\ref{exAppendixOsageAlways1}, \ref{exAppendixOsageAlways2}), thereby occupying the same position as a temporal clause.
\end{itemize}
\begin{exe}
	\ex\label{exAppendixOsageAlways1}
	\gll Šó̜ó̜šó̜we nanió̜pa ðaašoé hta apa-i.\\
	always pipe 3\textsc{sg}.\textsc{a}.smoke \textsc{fut} 3\textsc{cont}-\textsc{decl}\\
	\glt \lq He will always smoke.'  \parencite[328]{Quintero2004}

	\ex\label{exAppendixOsageAlways2}
\gll \textbf{Šó̜ðéðe} brée hta mi̜kšé\\
always 1\textsc{sg}.go \textsc{pot} 1\textsc{sg}.\textsc{cont}\\
\glt \lq I'm going and I'm not coming back.'  (\cite[209]{QuinteroDictionary}, glosses added)

	\ex\label{exAppendixOsageAlways3}
	\gll ðààcháâi, wažaže íe šó̜ó̜šó̜we ma̜-ði̜ ðée hkó̜-bra.\\
	against\_all\_odds	Osage language always \textsc{pvb}-go \textsc{prox} 1\textsc{sg}.\textsc{a}:\textsc{pvb}-want\\
	\glt \lq I want this Osage language to go on forever [against all odds].' (\cite[32]{QuinteroDictionary}, glosses added)
\end{exe}

\subsubsection{Temporal connectives and frame setters}
\paragraph{Simultaneous duration}\label{appendixOsageWhile}
\begin{itemize}
	\item \citeauthor{Quintero1997} (\citeyear[305, 362–363]{Quintero1997}, \citeyear[307–308, 444–446]{Quintero2004}, \citeyear[208]{QuinteroDictionary}).
	\item This function can occur together with an overt subordinator that occupies the corresponding clause-final position, as in (\ref{exAppendixOsageWhen3}). However, judging from the examples throughout \citeauthor{Quintero1997} (\citeyear{Quintero1997}, \citeyear{Quintero2004}), this is not particularly common; instead, \textit{šó̜} is usually the only exponent of the clause's subordinate status.
	\item Form: there are variants \textit{šó̜de}\sim\textit{šó̜we}\sim\textit{šó̜e} < \mbox{\textit{šó̜}-\textit{ðe}} \lq still-\textsc{prox}'; this is shown in (\ref{exAppendixOsageWhen2}). It is not clear what differentiates these forms from plain \textit{šó̜} as \lq while'.
	\item Assuming that \textsc{still} (or a closely related notion) is the prior function of \textit{šó̜}, cases like (\ref{exAppendixOsageWhen4}) might constitute a bridging context.
\end{itemize}
\begin{exe}
	\ex
	\gll Á-wa-hkik-ie \textbf{šó̜} akxa-i wihtáeži̜ ə̜kxa má̜zeíe hí-ð-ap-e.\\
\textsc{pvb}:1\textsc{sg}.\textsc{a}-\textsc{recp}-speak still 3.\textsc{cont}-\textsc{decl} sister \textsc{subj} phone\_call  \textsc{pvb}:arrive\_there-\textsc{caus}-\textsc{pl}-\textsc{decl}\\
\glt \lq I was talking [to someone] when my younger sister called (lit. While I was having a conversation, younger sister made a phone call arrive there).' \parencite[445]{Quintero2004}

	\ex\label{exAppendixOsageWhen2}
	\gll Ó-wí-hká̜ \textbf{šó̜we} a-chí-p-e.\\
	\textsc{pvb}-1\textsc{sg}>2\textsc{sg}-help still \textsc{pvb}:3\textsc{sg}.\textsc{a}-arrive\_here-\textsc{pl}-\textsc{decl}\\
	\glt \lq I was helping you when he came in (lit. While I was helping you, he arrived here).' \parencite[444]{Quintero2004}

	\ex\label{exAppendixOsageWhen3}
	\gll Taaké \textbf{šo̜o̜} apaa, a̜ná̜-ð-api-i-áha ðíišt-a̜p-e\\
	3\textsc{pl}.\textsc{a}:fight still 3.\textsc{cont} \textsc{pvb}:1\textsc{sg}.\textsc{u}:3\textsc{pl}.\textsc{a}:see-\textsc{pl}-\textsc{imm}-when 3\textsc{pl}.\textsc{a}:finish-\textsc{pl}-\textsc{decl}\\
	\glt \lq They were fighting but they stopped when they saw me (lit. While they
were fighting no sooner did they see me than they stopped).' \parencite[444–445]{Quintero2004}
	\ex\label{exAppendixOsageWhen4}
	\gll Wakˀó ðáali̜ \textbf{šoo} ðai̜še wáščuɣe ðáali̜.\\
woman good still 2\textsc{sg}.\textsc{cont} 2\textsc{sg}:get\_married(\textsc{f}) good\\
	\glt \lq You ought to get married while you're still a pretty woman.\rq{ }(\cite[208]{QuinteroDictionary}, glosses added)
\end{exe}
\il{Osage|)}


\section{Slave (den, slav1253)}\il{Slave|(}
\subsection{kʼála}

\subsubsection{General information}
\begin{itemize}
	\item Form: there is a free variant \textit{k'ahla}.
	\item Wordhood: independent grammatical word, invariable.
	\item Syntax: either in clause-initial position (but following question markers) or preceding the verbal predicate.
\end{itemize}

\subsubsection{As a \lq{}still\rq{ }expression}
\begin{itemize}
	\sloppy
	\item \textcite[167]{SlaveyTopicalDictionary} and \citeauthor{Rice1978} (\citeyear[561]{Rice1978}, \citeyear[158]{Rice1989}).
	\item Specialisation: examples like (\ref{exAppendixSlave1}–\ref{exAppendixSlave4}) give a good indication that this marker conforms to my definition. For instance, in (\ref{exAppendixSlave1}) the husband's absence is construed as persistent, while at the same time evoking an alternative scenario in which he has returned (i.e. \textsc{no longer} is absent), and which would have led to a different course of action.
	\item Pragmaticity: appears to be compatible with both scenarios. Examples like (\ref{exAppendixSlave4}) suggest that the unexpectedly late scenario can be made explicit by expressions of the \lq even now\rq{ }type.
	\item Polarity sensitivity: inner negation yields \textsc{not yet}.
\end{itemize}

 
 \begin{exe}
 
 	\ex\label{exAppendixSlave1}
	Context: A girl has left home all on her own. She has come across a white settlement and a woman has signalled her to come in. \\
	\gll Hi̜tʼa ˀekáa bedenelí̜ \textbf{kʼála} while tʼá así̜i yegháˀedi̜dí. ˀekáa whaa goi̜ˀá le sí̜a bedenelí̜ goyánadéhtla hi̜tʼa nákinehˀi̜ dedi.\\
	now accomplished \textsc{poss}.3:husband still \textsc{subj}.3:\textsc{ipfv}:be\_gone because thing \textsc{subj}.3:\textsc{pfv}:give:\textsc{obj}.3.\textsc{obv} accomplished long area.\textsc{pfv}:\textsc{loc}.\textsc{cop} \textsc{neg} conjecture \textsc{poss}.3:husband \textsc{subj}.3\textsc{sg}:\textsc{pfv}:come\_back\_in now \textsc{subj}.3\textsc{pl}:\textsc{ipfv}:hide:\textsc{obj}.3.\textsc{obv} \textsc{subj}.3:\textsc{ipfv}:say\\
	\glt \lq Since her (the white womanʼs husband) was still away, the woman gave the girl something to eat. After a short time her husband came back in and they hid her (the girl).\rq{ }\parencite[1341, 1350]{Rice1989}
 
 	\ex\label{exAppendixSlave2}
	\gll \textbf{Kʼála} deyee tʼe kʼínaútoʼó.\\
	still \textsc{subj}.3:be\_calm while 1\textsc{pl}:\textsc{opt}:go\_around\_by\_boat\\
	\glt \lq While it is still calm, letʼs go out in the boat.' \parencite[1058]{Rice1989}
	
 	\ex\label{exAppendixSlave3}
	\gll ˀeku	ˀala	{só̙ba 	náya} ˀekú	 \textbf{kʼála} tsʼó̜dani		hehɬi̜.\\
	then	first	treaty	then		still	child		\textsc{subj}.1\textsc{sg}:was\\
	\glt \lq At the time of the treaty, I was still a child.\rq{ }\parencite[345]{Rice1989}
	
	\ex\label{exAppendixSlave4}
	\gll Hi̜du {dúyhee gahkwʼe} \textbf{kʼála} mó̙la ɬo north dágó̜tʼe keokedi̙hsho̙ íle.\\
	now this\_time during still white\_people many north how\_area\_is 3\textsc{pl}.know \textsc{neg}\\
	\glt \lq Even today, many whites donʼt know about the north.\rq{ }\parencite[346]{Rice1989}


 \end{exe}
 
\subsubsection{Uses related to other phasal polarity concepts} 
 
 \paragraph{Interrogative \lq yet\rq{}}
 \label{appendixSlaveInterrogativeYet}
 \begin{itemize}
 	\item \textcite[561]{Rice1978}. 
 	\item This function is attested in indirect (\ref{exAppendixSlaveInterrogative2}) and direct (\ref{exAppendixSlaveInterrogative3}, \ref{exAppendixSlaveInterrogative4}) questions.
	\item If not constituting a relic (an earlier \lq up to now\rq{ }meaning), it is conceivable that this phenomenon was mediated by cases like (\ref{exAppendixSlaveInterrogative5}). Here \textit{kʼála} is separated lineally from the negative marker, potentially leading to scope ambiguity (it being parsed either as part of the complement clause or as a pre-posed part of the higher clause). 
	\item I assume that \textit{kʼála} as \lq ever\rq{ }in (\ref{exAppendixSlaveInterrogative6}) can be subsumed here, as well (\lq as of yet, do you tend to go to Yellokwnife/have you gone to Yellowknife?\rq{}).
 \end{itemize}
 
 \begin{exe}

 	\ex\label{exAppendixSlaveInterrogative2}
	\gll Sú \textbf{k'ála} ˀelaá ráyéhdí hí̜sí̜.\\
	\textsc{q} still boat \textsc{subj}.3:bought uncertainty\\
	\glt \lq I wonder wonder if s/he bought a boat yet.' \parencite[421]{Rice1989}
	
 	\ex\label{exAppendixSlaveInterrogative3}
	\gll Sú \textbf{kʼála} yú kʼaŕaˀeni̜hsi ni̜ gogho̜ ˀaranetʼe?\\
	\textsc{q} still clothes \textsc{subj}.2\textsc{sg}:wash \textsc{comp} area\_of \textsc{subj}.2\textsc{sg}:finish\\
	\glt \lq Did you finish washing the clothes yet?' \parencite[1245]{Rice1989}
	
 	\ex\label{exAppendixSlaveInterrogative4}
	\gll Sú \textbf{kʼála} shéneti?\\
	\textsc{q} still \textsc{subj}.2\textsc{sg}:eat\\
	\glt \lq Did you eat yet?' \parencite[1989]{Rice1989}
	
	\ex\label{exAppendixSlaveInterrogative5}
	\gll \textbf{Kʼála} rídené̜wé gu kodisho̜ yíle.\\
	still \textsc{subj}.3:arrived \textsc{comp}/whether \textsc{subj}.1\textsc{sg}:know \textsc{neg}\\
	\glt \lq I didnʼt know if it came (by air) yet.' \parencite[1250]{Rice1989}
		
	\ex\label{exAppendixSlaveInterrogative6}
	\gll Sú	\textbf{kʼála}	só̜bakó̜é	gotsʼé̜		ˀanetʼí̜?\\
	\textsc{q} still Yellowknife	area\_to \textsc{subj}.2\textsc{sg}:go\\
	\glt \lq Do you sg. ever go to Yellowknife?\rq{ }\parencite[1133]{Rice1989}
 \end{exe}
\il{Slave|)}

\section{Western Shoshoni (shh, west2622)}
\il{Shoshoni, Western|(}
\label{appendixShoshoni}
\subsection{Introductory remarks}
I am indebted to John McLaughlin for helping tease apart examples from \textcite{CrumDayley1993} and for discussing cognates of \textit{ekisen} across Central Numic with me.

\subsection{ekisen}

\subsubsection{General information}
\begin{itemize}
	\item Form: \textcite{CrumDayley1993} have three forms that appear to be in free variation: \mbox{\textit{ekise}}\sim \mbox{\textit{ekisen}}\sim\mbox{\textit{ekisem}}. \textcite{McLaughlin2012} transcribes this marker as \mbox{\textit{ɨkisɨn}}.
	\item Wordhood: independent grammatical word, invariable.
	\item Syntax: in all examples, \textit{ekisen} precedes the predicate, typically following the subject.
	\item Etymology: < \textit{eki} \lq now' plus a suffix \mbox{-\textit{seN}}. Cognates of this marker are found across Numic languages, and are often described as adverbial, emphatic, intensifying, etc; the item's functional range suggests it is best considered a non-scalar restrictive \lq just\rq{}.
\end{itemize}

\subsubsection{As a \lq{}still\rq{ }expression}
\begin{itemize}
	\item \textcite[149]{CrumDayley1993} and \textcite[21]{McLaughlin2012}.
	\item Specialisation: ex. (\ref{exAppendixShoshoni1}–\ref{exAppendixShoshoni3}) give evidence that this marker has specialised as a \textsc{still} expression. For instance, in (\ref{exAppendixShoshoni1}) \textit{ekise} not only signals the continuation of a mythical scenario in which animals behaved like people, but also evokes a contrast with the opposite situation at the time of speech.
	\item Pragmaticity: the available data allow no conclusion.
	\item Polarity sensitivity: not attested in combination with negation. \textsc{not yet} is expressed via another expression, \textit{kaisen}.
\end{itemize}

\begin{exe}
	\ex\label{exAppendixShoshoni1}
	Context: The opening of a tale.\\
	\gll Himpaise \textbf{ekise} utihi newe niwene-ku, Itsappe ma'ai Po’naih nawaka nanaahka.\\
	long\_ago still \textsc{dem}.\textsc{du}.\textsc{acc} person speak.\textsc{pl}-\textsc{ds} coyote and mouse with\_each\_other lived\\
	\glt \lq A long time ago, when animals [still] talked ... Coyote and Mouse lived together.' \parencite[200, 202, 204]{CrumDayley1993}	

	\ex\label{exAppendixShoshoni2}
	Context: From a text about harvesting pine nuts.\\
	\gll September ma sutee \textbf{ekisem} puikaite. October-ha eke katehki-kka sutee akekkwanto’i, atee u himpeh hannihkwanto’i, aikkihte sokkattu sotee yummahkwanto’i.\\
	September in \textsc{dem}.\textsc{pl} still be\_green October-\textsc{poss} new start\_to\_be-\textsc{ds} \textsc{dem}.\textsc{pl} would\_open\_up \textsc{dem}.\textsc{pl} its thing would\_do here\_to ground\_towards they would\_fall\\
	\glt \lq During the month of September the pine cones are still green. Around the first part of October, the cones open up and the nuts start falling to the ground.' \parencite[209, 211, 214]{CrumDayley1993}
	
	\ex\label{exAppendixShoshoni3}
	\gll Charley \textbf{ekisen} nemi-kkanten soon-ti sennapin-nii ma-sea-nk-annu.\\
	C. still living-\textsc{subord}.\textsc{ss} many-\textsc{obj} aspen-\textsc{pl}.\textsc{acc} \textsc{ins}-grow-\textsc{caus}-\textsc{compl}\\
	\glt \lq When Charley was still alive, he planted many aspen trees.'
	\\(\cite[41]{CrumDayley1993}; glosses by \cite[66]{McLaughlin2012})

\end{exe}

\subsubsection{Broadly adverbial temporal-aspectual functions}
\paragraph{Near future}
\label{appendixWesternShoshoniNearFuture}
\begin{itemize}
	\item \textcite[149]{CrumDayley1993} and \textcite[21]{McLaughlin2012}.
	\item This function obtains in combination with the future tense.
	\item The source for this use lies in the original composition of \mbox{\textit{eki}-\textit{sen}}, namely \lq now-\textsc{restrictive}', i.e. \lq right now'. This interpretation is supported by the fact that in closely related \ili{Panamint} (par, pana1305) \textit{üküsü} has this near future function and serves as an emphatic version of \textit{ükü} \lq now', but does not serve as a \textsc{still} expression (cf. \cite[369]{Dayley1989Dictionary}, \citeyear*[300]{Dayley1989Grammar}).
\end{itemize}

\begin{exe}
	\ex
	\gll Sote \textbf{ekisen} pite-pite-to'i.\\
	\textsc{dem}.\textsc{sg} still arrive-finish-\textsc{fut}\\
	\glt \lq She is arriving soon.\rq{ }(\cite[103]{CrumDayley1993}; glosses by \cite[52]{McLaughlin2012})

	\ex
	\gll \textbf{Ekisen} tahma-to'i-han-to'i.\\
	still spring-emerge-\textsc{res}-\textsc{fut}\\
	\glt \lq Pretty soon it’s going to be spring.'
	\\(\cite[150]{CrumDayley1993}; glosses by \cite[68]{McLaughlin2012})

	\ex
	\gll Ne \textbf{ekisen} mi’a-kwan-to’i.\\
	1\textsc{sg} still go-\textsc{it}-\textsc{fut}\\
	\glt \lq I’m leaving pretty soon.' \parencite[95]{CrumDayley1993}
\end{exe}
\il{Shoshoni, Western|)}
