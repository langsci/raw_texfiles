\chapter{\label{ch:6}Zusammenfassung und Diskussion der Ergebnisse}

\section{\label{sec:6.0}Einleitung}

In diesem Kapitel werden die Studienergebnisse zusammenfassend diskutiert. Zunächst wird die festgestellte allgemeine positive Auswirkung der analysierten KS-Regeln erörtert. Im Anschluss wird die Auswirkung differenziert auf Regelebene (systemübergreifend) diskutiert. Auf einer tieferen Ebene wird der Effekt der einzelnen Regeln auf Systemebene (Analyse auf Regel- und Systemebene) zusammenfassend erörtert. Zum Schluss wenden wir uns dem Vergleich der vier MÜ-Ansätze (RBMÜ, SMÜ, HMÜ und NMÜ) jeweils am Beispiel des verwendeten Systems nach der Anwendung aller analysierten Regeln (d. h. regelübergreifend) zu.

\section{\label{sec:6.1}Allgemeine Auswirkung der KS-Regeln}

Die allgemeine Auswirkung der KS-Regeln auf den MÜ-Output stimmt mit den Ergebnissen früherer empirischer und theoretischer Studien (vgl. \citealt{NybergMitamura1996}; \citealt{Bernth1999}; \citealt{BernthGdaniec2001}: 208; \citealt{Drugan2013}: 98; \citealt{DrewerZiegler2014}: 196; \citealt{Wittkowsky2017}: 92) überein, bei denen festgestellt wurde, dass die KS-Anwendung den MÜ-Output aus unterschiedlichen Perspektiven verbessert. Die Ergebnisse aller angewandten Methoden bestätigten systemübergreifend die kollektive positive Wirkung aller analysierten Regeln auf den MÜ-Output:

In der Fehlerannotation sank die Fehleranzahl nach der Anwendung der Regeln signifikant. Sechs Fehlertypen nahmen nach der Anwendung der KS-Regeln in der folgenden Reihenfolge signifikant ab: (1) SM.13 „Kollokationsfehler“, (2) LX.3 „Wort ausgelassen“, (3) GR.8 „Falsches Verb“, (4) OR.2 „Großschreibung“, (5) GR.10 „Falsche Wortstellung“ und (6) GR.9 „Kongruenzfehler“. Vergleichbar mit dem Ergebnis einer vorherigen Studie (vgl. \citealt{KirchhoffEtAl2014}) war der Wortstellungsfehler (GR.10) der einzige Fehlertyp, dessen Rückgang mit dem Anstieg beider Qualitätswerte (SQ und CQ) signifikant korrelierte, obwohl der Rückgang im Fall des Wortstellungsfehlers (GR.10) nicht der höchste unter den Fehlertypen war. Im Übrigen bestand nur eine weitere Korrelation zwischen dem Rückgang des Auslassungsfehlers (LX.3) und dem Anstieg der CQ. Gleichzeitig nahmen zwei Fehlertypen signifikant zu (LX.6 „Konsistenzfehler“ und OR.1 „Zeichensetzung“), jedoch bestand keine Korrelation zwischen ihrem Anstieg und der Qualitätsveränderung.

In der Humanevaluation stiegen sowohl die Stil- als auch die Inhaltsqualität signifikant, wobei der Anstieg der Inhaltsqualität (Verständlichkeit und Genauigkeit) höher war. In der automatischen Evaluation nahmen beide AEM-Scores leicht zu. Außerdem zeigte der Spearman-Korrelationstest eine signifikante starke positive Korrelation zwischen der Differenz (nach-KS \textit{minus} vor-KS) der allgemeinen Qualität und der der AEM-Scores, was darauf hindeutet, dass der Anstieg der allgemeinen Qualität mit der Verbesserung der AEM-Scores einherging.

Die unterschiedlichen Anstiegsniveaus der Stil- und Inhaltsqualität nach der Anwendung der KS-Regeln werfen jedoch die Frage auf, in welchen Fällen sich eine deutliche Steigerung der Inhaltsqualität gegenüber der Stilqualität zeigte. Diese Frage lässt sich auf Regelebene beantworten.

\section{\label{sec:6.2}Systemübergreifende Auswirkung der KS auf Regelebene}

Für einen Überblick über die Ergebnisse auf Regelebene betrachten wir im Folgenden die einzelnen analysierten Regeln aus drei Perspektiven, nämlich unter Berücksichtigung der Fehleranzahlveränderungen, der Annotationsgruppen sowie der Qualitätsveränderungen.

\subsection{Überblick über die Fehleranzahlveränderungen der einzelnen Regeln}
Die Fehleranalyse ergab, dass die Fehleranzahl nach der Anwendung von sieben der analysierten neun Regeln sank, wobei dieser Fehleranzahlrückgang nur bei fünf von diesen Regeln signifikant war. Diese Regeln sind absteigend nach der Prozentzahl des Fehleranzahlrückgangs sortiert, wie folgt (\figref{fig:06:149}): „fvg -- Funktionsverbgefüge vermeiden“ mit einem Rückgang von 56,7~\%; „kos -- Konditionalsätze mit ‚Wenn‘ einleiten“ mit einem Rückgang von 56,5~\%; „anz -- Für zitierte Oberflächentexte gerade Anführungszeichen verwenden“ mit einem Rückgang von 46,0~\%; „per -- Konstruktionen mit ‚sein +~zu +~Infinitiv‘ vermeiden“ mit einem Rückgang von 37,2~\% und „prä -- Überflüssige Präfixe vermeiden“ mit einem Rückgang von 35,6~\%.

Bei der achten „pak -- Partizipialkonstruktionen vermeiden“ und neunten Regel „pas -- Passiv vermeiden“ stieg die Fehleranzahl nach der Regelanwendung (\figref{fig:06:149}), wobei der Anstieg nur bei der Regel „pak -- Partizipialkonstruktionen vermeiden“ (26,4~\%) signifikant war.

\begin{figure}

%\includegraphics[width=\textwidth]{figures/d3-img122.png}

%\textbf{\textit{$-$ 37,2~\%}}

%\textbf{$-$ 6,6~\%}

%\textbf{\textit{$-$ 35,6~\%}}

%\textbf{\textit{+~26,4~\%}}

%\textbf{\textit{$-$ 56,5~\%}}

%\textbf{+~30,5~\%}

%\textbf{$-$ 9,3~\%}

%\textbf{\textit{$-$ 56,7~\%}}

%\textbf{\textit{$-$ 46,0~\%}}


%\includegraphics[width=\textwidth]{figures/d3-img023.png}


%\includegraphics[width=\textwidth]{figures/d3-img123.png}
\begin{tikzpicture}
	\begin{axis}[
	ybar,
	ymin = 0,
	ymax = 140,
	axis lines* = left,
	nodes near coords,
%         nodes near coords align = vertical,
  legend style={at={(0.5,-0.1)},anchor=north,cells={align=left}},
  xtick={1,2,...,9},
  xticklabels = {anz,fvg,kos,nsp,pak,pas,per,prä,wte},
  ylabel = {Summe},
  width = \textwidth,
%  enlarge x limits = .5
  extra description/.code={
  \node at (axis cs:.5, 150)[anchor=west]{\bfitul{$-$ 46,0 \%}};
  \node at (axis cs:1.5, 100)[anchor=west]{\bfitul{$-$ 56,7 \%}};
  \node at (axis cs:2.5, 105)[anchor=west]{\bfitul{$-$ 56,5 \%}};
  \node at (axis cs:3.5, 55)[anchor=west]{$-$ 9,3 \%};
  \node at (axis cs:4.5, 150)[anchor=west]{\bfitul{$+$ 26,4 \%}};
  \node at (axis cs:5.3, 90)[anchor=west]{$+$ 30,5 \%};
  \node at (axis cs:6.5, 100)[anchor=west]{\bfitul{$-$ 37,2 \%}};
  \node at (axis cs:7.3, 60)[anchor=west]{\bfitul{$-$ 35,6 \%}};
  \node at (axis cs:8.5, 90)[anchor=west]{$-$ 6,6 \%};
  },
	]
	\addplot+[tmnlpone]
	coordinates{
	(1,137)
  (2,90)
  (3,92)
  (4,43)
  (5,110)
  (6,59)
  (7,86)
  (8,45)
  (9,76)
	};
	\addplot+[tmnlpthree]
	coordinates{
  (1,74)
  (2,39)
  (3,40)
  (4,39)
  (5,139)
  (6,77)
  (7,54)
  (8,29)
  (9,71)
	};
	\legend{Anzahl der  fehler innerh. KS vor KS,Anzahl der Fehler innerh. KS nach KS}
	\end{axis}
\end{tikzpicture}
\caption{\label{fig:06:149} Summe der Fehleranzahl vor vs. nach KS auf Regelebene}
\bspnote{\textbf{anz:} Für zitierte Oberflächentexte gerade Anführungszeichen verwenden\\
{\textbf{fvg:} Funktionsverbgefüge vermeiden}\\
{\textbf{kos:} Konditionalsätze mit ‚Wenn‘ einleiten}\\
{\textbf{nsp:} Eindeutige pronominale Bezüge verwenden}\\
\textbf{pak:} Partizipial-konstruktionen vermeiden\\
{\textbf{pas:} Passiv vermeiden}\\
{\textbf{per:} Konstruktionen mit ‚sein + zu + Infinitiv‘ vermeiden}\\
{\textbf{prä:} Überflüssige Präfixe vermeiden}\\
{\textbf{wte:} Keine Wortteile weglassen}\\
\bfitul{Signifikante Differenz vor vs. nach KS}}
\end{figure}

Eine genauere Aussage, ob diese Fehleranzahlrückgänge bzw. Fehleranzahlanstiege mit einer Qualitätsverbesserung bzw. Qualitätsverschlechterung einhergingen, konnte an dieser Stelle nicht getroffen werden. Dies bedarf einer tieferen Analyse, in der die Ergebnisse der Fehleranalyse mit denen der Humanevaluation (\tabref{tab:06:101}) sowie die Ergebnisse der Humanevaluation mit denen der automatischen Evaluation (\figref{fig:06:152}) trianguliert werden.

\subsection{Überblick über die Annotationsgruppen der einzelnen Regeln}

In der Analyse auf Annotationsgruppenebene wurden die Ergebnisse nach dem Vorhandensein bzw. Nichtvorhandensein von MÜ-Fehlern in vier Gruppen aufgeteilt: RR, FF, RF und FR.\footnote{\textrm{Die Daten wurden binär bzw. dichotom aufgeteilt (keine Fehler aufgetreten ‚0‘; Fehler aufgetreten ‚1‘), daraus wurden vier Annotationsgruppen in Bezug auf die} \textrm{\textit{KS-Stelle}} \textrm{gebildet: (1) RR: MÜ ist vor und nach der Anwendung der KS-Regel fehlerfrei; (2) FF: MÜ beinhaltet vor und nach der Anwendung der KS-Regel Fehler; (3) RF: MÜ ist nur vor der Anwendung der KS-Regel fehlerfrei; (4) FR: MÜ ist nur nach der Anwendung der KS-Regel fehlerfrei.}} Nach dieser Analyse zeigten die Regeln eine ziemlich begrenzte positive Wirkung, nämlich ausschließlich bei der FR-Gruppe (d.~h. MÜ beinhaltet vor der Regelanwendung Fehler und ist nach der Regelanwendung fehlerfrei). Diese Gruppe liegt lediglich zwischen 8~\% (Regel „Passiv vermeiden“) und 31~\% (Regel „Für zitierte Oberflächentexte gerade Anführungszeichen verwenden“), \figref{fig:06:150}.


\begin{figure}
%\includegraphics[width=\textwidth]{figures/d3-img124.png}
\begin{tikzpicture}
\tikzset{every node/.style={font=\scriptsize}};
	\begin{axis}[
	ybar,
	ymin = 0,
	ymax = 100,
	axis lines* = left,
	nodes near coords,
% nodes near coords align = vertical,
  legend style={at={(0.5,-0.1)},anchor=north,cells={align=left}},
  legend columns = {-1},
  xtick={1,2,...,9},
  xticklabels = {anz,fvg,kos,nsp,pak,pas,per,prä,wte},
  x tick label style ={yshift=-5pt},
  width = \textwidth,
  bar width = 6,
  enlarge x limits = .08,
  extra description/.code={
  \node at (axis cs:.4, 55)[anchor=west]{4,6\%};
  \node at (axis cs:.3, -1)[anchor=west]{42\%};
  \node at (axis cs:.7, 45)[anchor=west]{3,4\%};
  \node at (axis cs:.6, -3)[anchor=west]{31\%};
  \node at (axis cs:.8, 7)[anchor=west]{0,2\%};
  \node at (axis cs:.9, -1)[anchor=west]{2\%};
  \node at (axis cs:1, 36)[anchor=west]{2,9\%};
  \node at (axis cs:1.1, -3)[anchor=west]{26\%};
  \node at (axis cs:1.3, 30)[anchor=west]{2,3\%};
  \node at (axis cs:1.4, -1)[anchor=west]{21\%};
  \node at (axis cs:1.6, 40)[anchor=west]{3,2\%};
  \node at (axis cs:1.6, -3)[anchor=west]{29\%};
  \node at (axis cs:1.8, 15)[anchor=west]{2,3\%};
  \node at (axis cs:1.9, -1)[anchor=west]{8\%};
  \node at (axis cs:2, 55)[anchor=west]{4,6\%};
  \node at (axis cs:2.1, -3)[anchor=west]{42\%};
  \node at (axis cs:2.3, 30)[anchor=west]{2,3\%};
  \node at (axis cs:2.4, -1)[anchor=west]{21\%};
  \node at (axis cs:2.6, 35)[anchor=west]{2,7\%};
  \node at (axis cs:2.6, -3)[anchor=west]{24\%};
  \node at (axis cs:2.8, 11)[anchor=west]{0,6\%};
  \node at (axis cs:2.9, -1)[anchor=west]{5\%};
  \node at (axis cs:3, 65)[anchor=west]{5,6\%};
  \node at (axis cs:3.1, -3)[anchor=west]{50\%};
  \node at (axis cs:3.3, 21)[anchor=west]{1,5\%};
  \node at (axis cs:3.4, -1)[anchor=west]{13\%};
  \node at (axis cs:3.6, 26)[anchor=west]{1,9\%};
  \node at (axis cs:3.6, -3)[anchor=west]{18\%};
  \node at (axis cs:3.9, 20)[anchor=west]{1,3\%};
  \node at (axis cs:3.9, -1)[anchor=west]{12\%};
  \node at (axis cs:4, 75)[anchor=west]{6,4\%};
  \node at (axis cs:4.1, -3)[anchor=west]{58\%};
  \node at (axis cs:4.4, 55)[anchor=west]{4,6\%};
  \node at (axis cs:4.3, -1)[anchor=west]{42\%};
  \node at (axis cs:4.6, 16)[anchor=west]{1,0\%};
  \node at (axis cs:4.6, -3)[anchor=west]{9\%};
  \node at (axis cs:4.8, 31)[anchor=west]{2,4\%};
  \node at (axis cs:4.8, -1)[anchor=west]{22\%};
  \node at (axis cs:5, 38)[anchor=west]{3,1\%};
  \node at (axis cs:5.1, -3)[anchor=west]{28\%};
  \node at (axis cs:5.4, 40)[anchor=west]{3,2\%};
  \node at (axis cs:5.4, -1)[anchor=west]{29\%};
  \node at (axis cs:5.6, 15)[anchor=west]{0,9\%};
  \node at (axis cs:5.6, -3)[anchor=west]{8\%};
  \node at (axis cs:5.8, 21)[anchor=west]{1,5\%};
  \node at (axis cs:5.9, -1)[anchor=west]{13\%};
  \node at (axis cs:6.1, 65)[anchor=west]{5,5\%};
  \node at (axis cs:6.1, -3)[anchor=west]{49\%};
  \node at (axis cs:6.3, 30)[anchor=west]{2,3\%};
  \node at (axis cs:6.4, -1)[anchor=west]{21\%};
  \node at (axis cs:6.6, 38)[anchor=west]{3,1\%};
  \node at (axis cs:6.6, -3)[anchor=west]{28\%};
  \node at (axis cs:6.9, 20)[anchor=west]{1,4\%};
  \node at (axis cs:6.9, -1)[anchor=west]{13\%};
  \node at (axis cs:7, 53)[anchor=west]{4,4\%};
  \node at (axis cs:7.1, -3)[anchor=west]{39\%};
  \node at (axis cs:7.4, 25)[anchor=west]{1,8\%};
  \node at (axis cs:7.4, -1)[anchor=west]{16\%};
  \node at (axis cs:7.7, 21)[anchor=west]{1,5\%};
  \node at (axis cs:7.6, -3)[anchor=west]{13\%};
  \node at (axis cs:7.8, 10)[anchor=west]{0,4\%};
  \node at (axis cs:7.9, -1)[anchor=west]{3\%};
  \node at (axis cs:8, 86)[anchor=west]{7,5\%};
  \node at (axis cs:8.1, -3)[anchor=west]{68\%};
  \node at (axis cs:8.4, 40)[anchor=west]{3,1\%};
  \node at (axis cs:8.4, -1)[anchor=west]{28\%};
  \node at (axis cs:8.6, 27)[anchor=west]{2,0\%};
  \node at (axis cs:8.6, -3)[anchor=west]{18\%};
  \node at (axis cs:8.9, 18)[anchor=west]{1,2\%};
  \node at (axis cs:8.9, -1)[anchor=west]{11\%};
  \node at (axis cs:9, 56)[anchor=west]{4,7\%};
  \node at (axis cs:9.1, -3)[anchor=west]{43\%};
  }
	]
	\addplot+[tmnlpone]
	coordinates{
	(1,50)
  (2,25)
  (3,25)
  (4,16)
  (5,50)
  (6,35)
  (7,25)
  (8,19)
  (9,34)
	};
  \addplot+[tmnlptwo]
	coordinates{
  (1,37)
  (2,35)
  (3,29)
  (4,21)
  (5,11)
  (6,10)
  (7,33)
  (8,16)
  (9,22)
	};
	\addplot+[tmnlpthree]
	coordinates{
  (1,2)
  (2,10)
  (3,6)
  (4,14)
  (5,26)
  (6,16)
  (7,15)
  (8,4)
  (9,13)
	};
  \addplot+[tmnlpfour]
	coordinates{
  (1,31)
  (2,50)
  (3,60)
  (4,69)
  (5,33)
  (6,59)
  (7,47)
  (8,81)
  (9,51)
	};
	\legend{FF,FR,RF,RR}
	\end{axis}
\end{tikzpicture}
\caption{\label{fig:06:150} Aufteilung der Annotationsgruppen auf Regelebene}
\bspnote{N (alle Regeln) = 1080; N (pro Regel) = 120\\
Die oben angezeigten Prozentzahlen sind auf Basis des Gesamtdatensatzes aller analysierten Regeln (N = 1080) berechnet.

Die untenstehenden Prozentzahlen sind auf Regelebene (N = 120) berechnet.}
\end{figure}

In der RF-Annotationsgruppe ist die KS-Auswirkung eindeutig negativ. Die bei allen Regeln dominierenden Annotationsgruppen waren RR und FF. Da die Übersetzungen sowohl vor als auch nach der Anwendung der jeweiligen Regel fehlerfrei (RR-Gruppe) oder fehlerhaft (FF-Gruppe) waren, kann eine positive Auswirkung einer Regel nur dann gerechtfertigt sein, wenn die Qualitätswerte dieser beiden Gruppen nach der Regelanwendung stiegen. Eine Qualitätssteigerung in der RR-Gruppe würde bedeuten, dass die Qualität einer fehlerfreien MÜ nach KS höher als die einer fehlerfreien MÜ vor KS sei (z. B. durch eine stilistische Verbesserung). Ebenso würde eine Qualitätssteigerung in der FF-Gruppe bedeuten, dass beim Vergleich zweier fehlerhaften Übersetzungen vor und nach KS die Qualität der fehlerhaften MÜ nach KS höher wäre (z. B. aufgrund des Auftretens eines weniger schwerwiegenden Fehlertyps oder des Rückgangs der Fehleranzahl).

Die Triangulation der Ergebnisse der Fehlerannotation und der Humanevaluation deckte auf, wie sich jede Regel auf Annotationsgruppenebene auf die Stil- und Inhaltsqualität auswirkte (\tabref{tab:06:101}). In der Tabelle wird im Wesentlichen der Mittelwert (M) der Qualitätsveränderung (Diff. SQ, Diff. CQ und Diff. Q berechnet als nach-KS \textit{minus} vor-KS) bei jeder Annotationsgruppe sowie die Signifikanz (p) dieser Veränderung nach dem Wilcoxon-Test aufgeführt. Der grüne Hintergrund verdeutlicht die signifikanten Fälle.

\begin{sidewaystable}
\captionsetup{width=.9\textwidth}
\scriptsize
\begin{tabularx}{\textwidth}{llrrrrrrrrrrrr}
\lsptoprule

{} & & \multicolumn{3}{c}{ \textbf{FF}} & \multicolumn{3}{c}{ \textbf{FR}} & \multicolumn{3}{c}{ \textbf{RF}} & \multicolumn{3}{c}{ \textbf{RR}}\\
\cmidrule(lr){3-5}\cmidrule(lr){6-8}\cmidrule(lr){9-11}\cmidrule(lr){12-14}
{} & & \textbf{Diff. SQ}  & \textbf{Diff. CQ}  & \textbf{Diff. Q}  & \textbf{Diff. SQ}  & \textbf{Diff. CQ}  & \textbf{Diff. Q}  & \textbf{Diff. SQ}  & \textbf{Diff. CQ}  & \textbf{Diff. Q}  & \textbf{Diff. SQ}  & \textbf{Diff. CQ}  & \textbf{Diff. Q} \\
\midrule
\textbf{anz} & M & \cellcolor{lsLightGreen}0,47 & \cellcolor{lsLightGreen}0,27 & \cellcolor{lsLightGreen}0,37 & \cellcolor{lsLightGreen}0,69 & \cellcolor{lsLightGreen}0,84 & \cellcolor{lsLightGreen}0,77 & $-$ 0,38 & 0 & $-$ 0,19 & \cellcolor{lsLightGreen}0,32 & 0,05 & \cellcolor{lsLightGreen}0,18\\
& p & \cellcolor{lsLightGreen}< 0,001 & \cellcolor{lsLightGreen}0,003 & \cellcolor{lsLightGreen}< 0,001 & \cellcolor{lsLightGreen}< 0,001 & \cellcolor{lsLightGreen}0,001 & \cellcolor{lsLightGreen}< 0,001 & 1 & 1 & 1 & \cellcolor{lsLightGreen}0,001 & \textit{0,273} & \cellcolor{lsLightGreen}0,005\\
& N & \cellcolor{lsLightGreen}30 & \cellcolor{lsLightGreen}30 & \cellcolor{lsLightGreen}30 & \cellcolor{lsLightGreen}24 & \cellcolor{lsLightGreen}24 & \cellcolor{lsLightGreen}24 & & & & \cellcolor{lsLightGreen}19 & 19 & \cellcolor{lsLightGreen}19\\
& z & \cellcolor{lsLightGreen}$-$ 3,711 & \cellcolor{lsLightGreen}$-$ 2,998 & \cellcolor{lsLightGreen}$-$ 3,659 & \cellcolor{lsLightGreen}$-$ 3,950 & \cellcolor{lsLightGreen}$-$ 3,402 & \cellcolor{lsLightGreen}$-$ 3,787 & & & & \cellcolor{lsLightGreen}$-$ 3,282 & $-$ 1,096 & \cellcolor{lsLightGreen}$-$ 2,825\\
\textbf{fvg} & M & 0,14 & 0,14 & 0,14 & \cellcolor{lsLightGreen}0,85 & \cellcolor{lsLightGreen}0,75 & \cellcolor{lsLightGreen}0,80 & $-$ 0,16 & $-$ 0,50 & $-$ 0,33 & 0,15 & 0,03 & 0,09\\
& p & \textit{0,420} & \textit{0,636} & \textit{0,421} & \cellcolor{lsLightGreen}< 0,001  & \cellcolor{lsLightGreen}< 0,001 & \cellcolor{lsLightGreen}< 0,001 & \textit{0,416} &\textit{0,107} & \textit{0,058} & \textit{0,140} & \textit{0,926} & \textit{0,190}\\
& N & 17 & 17 & 17 & \cellcolor{lsLightGreen}25 & \cellcolor{lsLightGreen}25 & \cellcolor{lsLightGreen}25 & 7 & 7 & 7 & 35 & 35 & 35\\
& z & $-$ 0,807 & $-$ 0,474 & $-$ 0,805 & \cellcolor{lsLightGreen}$-$ 4,205 & \cellcolor{lsLightGreen}$-$ 3,884 & \cellcolor{lsLightGreen}$-$ 4,043 & $-$ 0,813 & $-$ 1,612 & $-$ 1,892 & $-$ 1,477 & $-$ 0,093 & $-$ 1,312\\
\textbf{kos} & M & 0,30 & 0,38 & 0,34 & \cellcolor{lsLightGreen}0,71 & \cellcolor{lsLightGreen}1,28 & \cellcolor{lsLightGreen}0,99 & $-$ 0,71 & $-$ 1,04 & $-$ 0,88 & $-$ 0,10 & 0,01 & $-$ 0,04\\
& p & \textit{0,090} & \textit{0,278} & \textit{0,157} & \cellcolor{lsLightGreen}< 0,001 & \cellcolor{lsLightGreen}< 0,001 & \cellcolor{lsLightGreen}< 0,001 & \textit{0,285} & \textit{0,285} & \textit{0,285} & \textit{0,620} & \textit{0,795} & \textit{0,229}\\
& N & 14 & 14 & 14 & \cellcolor{lsLightGreen}23 & \cellcolor{lsLightGreen}23 & \cellcolor{lsLightGreen}23 & 3 & 3 & 3 & 44 & 44 & 44\\
& z & $-$ 1,694 & $-$ 1,084 & $-$ 1,414 & \cellcolor{lsLightGreen}$-$ 4,143 & \cellcolor{lsLightGreen}$-$ 4,209 & \cellcolor{lsLightGreen}$-$ 4,201 & $-$ 1,069 & $-$ 1,069 & $-$ 1,069 & $-$ 0,222 & $-$ 0,260 & $-$ 1,204\\
\textbf{nsp} & M & $-$ 0,15 & $-$ 0,08 & $-$ 0,11 & \cellcolor{lsLightGreen}0,31 & \cellcolor{lsLightGreen}0,68 & \cellcolor{lsLightGreen}0,50 & \cellcolor{lsLightGreen}$-$ 0,50 & \cellcolor{lsLightGreen}$-$ 0,48 & \cellcolor{lsLightGreen}$-$ 0,49 & $-$ 0,05 & 0,10 & 0,02\\
& p & \textit{0,443} & \textit{0,799} & \textit{0,735} & \cellcolor{lsLightGreen}0,011 & \cellcolor{lsLightGreen}0,002 & \cellcolor{lsLightGreen}0,002 & \cellcolor{lsLightGreen}0,012 & \cellcolor{lsLightGreen}0,012 & \cellcolor{lsLightGreen}0,012 & \textit{0,220} & \textit{0,086} & \textit{0,795}\\
& N & 8 & 8 & 8 & \cellcolor{lsLightGreen}12 & \cellcolor{lsLightGreen}12 & \cellcolor{lsLightGreen}12 & \cellcolor{lsLightGreen}10 & \cellcolor{lsLightGreen}10 & \cellcolor{lsLightGreen}10 & 47 & 47 & 47\\
& z & $-$ 0,768 & $-$ 0,254 & $-$ 0,338 & \cellcolor{lsLightGreen}$-$ 2,547 & \cellcolor{lsLightGreen}$-$ 3,072 & \cellcolor{lsLightGreen}$-$ 3,065 & \cellcolor{lsLightGreen}$-$ 2,509 & \cellcolor{lsLightGreen}$-$ 2,313 & \cellcolor{lsLightGreen}$-$ 2,527 & $-$ 1,227 & $-$ 1,718 & - 0,260\\
\textbf{pak} & M & \cellcolor{lsLightGreen}$-$ 0,19 & 0,05 & $-$ 0,07 & 0,44 & \cellcolor{lsLightGreen}0,82 & 0,63 & \cellcolor{lsLightGreen}$-$ 0,73 & \cellcolor{lsLightGreen}$-$ 0,52 & \cellcolor{lsLightGreen}$-$ 0,63 & \cellcolor{lsLightGreen}$-$ 0,38 & \cellcolor{lsLightGreen}$-$ 0,13 & \cellcolor{lsLightGreen}$-$ 0,25\\
& p & \cellcolor{lsLightGreen}0,008 & \textit{0,411} & \textit{0,507} & \textit{0,107} & \cellcolor{lsLightGreen}0,021 & \textit{0,050} & \cellcolor{lsLightGreen}< 0,001 & \cellcolor{lsLightGreen}0,012 & \cellcolor{lsLightGreen}< 0,001 & \cellcolor{lsLightGreen}< 0,001 & \cellcolor{lsLightGreen}0,020 & \cellcolor{lsLightGreen}< 0,001\\
& N & \cellcolor{lsLightGreen}37 & 37 & 37 & 9 & \cellcolor{lsLightGreen}9 & 9 & \cellcolor{lsLightGreen}23 & \cellcolor{lsLightGreen}23 & \cellcolor{lsLightGreen}23 & \cellcolor{lsLightGreen}28 & \cellcolor{lsLightGreen}28 & \cellcolor{lsLightGreen}28\\
& z & \cellcolor{lsLightGreen}$-$ 2,638 & $-$ 0,823 & $-$ 0,664 & $-$ 1,612 & \cellcolor{lsLightGreen}$-$ 2,314 & $-$ 1,958 & \cellcolor{lsLightGreen}$-$ 4,114 & \cellcolor{lsLightGreen}$-$ 2,526 & \cellcolor{lsLightGreen}$-$ 4,021 & \cellcolor{lsLightGreen}$-$ 3,540 & \cellcolor{lsLightGreen}$-$ 2,322 & \cellcolor{lsLightGreen}$-$ 3,618\\
\textbf{pas} & M & \cellcolor{lsLightGreen}$-$ 0,34 &  $-$ 0,16 & \cellcolor{lsLightGreen}$-$ 0,25 & 0,54 & 0,58 & 0,56 & \cellcolor{lsLightGreen}$-$ 0,99 & \cellcolor{lsLightGreen}$-$ 1,93 & \cellcolor{lsLightGreen}$-$ 1,46 & $-$ 0,13 & 0,05 & $-$ 0,04\\
& p & \cellcolor{lsLightGreen}0,010 & \textit{0,177} & \cellcolor{lsLightGreen}0,019 & \textit{0,114} & \textit{0,080} & 0,075 & \cellcolor{lsLightGreen}0,003 & \cellcolor{lsLightGreen}0,002 & \cellcolor{lsLightGreen}0,002 & \textit{0,061} & \textit{0,427} & \textit{0,383}\\
& N & \cellcolor{lsLightGreen}25 & 25 & \cellcolor{lsLightGreen}25 & 6 & 6 & 6 & \cellcolor{lsLightGreen}12 & \cellcolor{lsLightGreen}12 & \cellcolor{lsLightGreen}12 & 40 & 40 & 40\\
& z & \cellcolor{lsLightGreen}$-$ 2,566 & $-$ 1,351 & \cellcolor{lsLightGreen}$-$ 2,345 & $-$ 1,581 & $-$ 1,753 & $-$ 1,782 & \cellcolor{lsLightGreen}$-$ 2,940 & \cellcolor{lsLightGreen}$-$ 3,062 & \cellcolor{lsLightGreen}$-$ 3,061 & $-$ 1,876 & $-$ 0,794 & $-$ 0,872\\
\textbf{per} & M & \cellcolor{lsLightGreen}0,33 & 0,27 & \cellcolor{lsLightGreen}0,30 & \cellcolor{lsLightGreen}1,29 & \cellcolor{lsLightGreen}1,59 & \cellcolor{lsLightGreen}1,44 & \cellcolor{lsLightGreen}$-$ 0,33 & \cellcolor{lsLightGreen}$-$ 0,47 & \cellcolor{lsLightGreen}$-$ 0,40 & \cellcolor{lsLightGreen}0,24 & 0,00 & \cellcolor{lsLightGreen}0,12\\
& p & \cellcolor{lsLightGreen}0,003 & \textit{0,136} & \cellcolor{lsLightGreen}0,012 & \cellcolor{lsLightGreen}< 0,001 & \cellcolor{lsLightGreen}< 0,001 & \cellcolor{lsLightGreen}< 0,001 & \cellcolor{lsLightGreen}0,016 & \cellcolor{lsLightGreen}0,007 & \cellcolor{lsLightGreen}0,001 & \cellcolor{lsLightGreen}< 0,001 & \textit{0,522} & \cellcolor{lsLightGreen}0,002\\
& N & \cellcolor{lsLightGreen}20 & 20 & \cellcolor{lsLightGreen}20 & \cellcolor{lsLightGreen}27 & \cellcolor{lsLightGreen}27 & \cellcolor{lsLightGreen}27 & \cellcolor{lsLightGreen}13 & \cellcolor{lsLightGreen}13 & \cellcolor{lsLightGreen}13 & \cellcolor{lsLightGreen}37 & 37 & \cellcolor{lsLightGreen}37\\
& z & \cellcolor{lsLightGreen}$-$ 3,021 & $-$ 1,492 & \cellcolor{lsLightGreen}$-$ 2,505 & \cellcolor{lsLightGreen}$-$ 4,463 & \cellcolor{lsLightGreen}$-$ 4,544 & \cellcolor{lsLightGreen}$-$ 4,542 & \cellcolor{lsLightGreen}$-$ 2,417 & \cellcolor{lsLightGreen}$-$ 2,680 & \cellcolor{lsLightGreen}$-$ 3,192 & \cellcolor{lsLightGreen}$-$ 3,737 &  $-$ 0,640 & \cellcolor{lsLightGreen}$-$ 3,167\\
\textbf{prä} & M & 0,10 & 0,08 & 0,09 & \cellcolor{lsLightGreen}0,65 & \cellcolor{lsLightGreen}0,78 & \cellcolor{lsLightGreen}0,71 & $-$ 0,31 & $-$ 0,88 & $-$ 0,59 &  $-$ 0,05 & $-$ 0,03 & $-$ 0,04\\
& p & \textit{0,109} & \textit{0,765} & \textit{0,439} & \cellcolor{lsLightGreen}0,002 & \cellcolor{lsLightGreen}0,001 & \cellcolor{lsLightGreen}0,001 & \textit{0,180} & \textit{0,180} & \textit{0,180} & \textit{0,148} & \textit{0,828} & \textit{0,122}\\
& N & 11 & 11 & 11 & \cellcolor{lsLightGreen}17 & \cellcolor{lsLightGreen}17 & \cellcolor{lsLightGreen}17 & 2 & 2 & 2 & 62 & 62 & 62\\
& z & $-$ 1,602 & $-$ 0,299 & $-$ 0,775 & \cellcolor{lsLightGreen}$-$ 3,133 &\cellcolor{lsLightGreen} $-$ 3,462 & \cellcolor{lsLightGreen}$-$ 3,410 & $-$ 1,342 & $-$ 1,342 & $-$ 1,342 & $-$ 1,448 & $-$ 0,217 & $-$ 1,548\\
\textbf{wte} & M & \cellcolor{lsLightGreen}$-$ 0,27 & $-$ 0,29 & \cellcolor{lsLightGreen}$-$ 0,28 & 0,20 & \cellcolor{lsLightGreen}0,52 & \cellcolor{lsLightGreen}0,36 & \cellcolor{lsLightGreen} $-$ 0,50 & \cellcolor{lsLightGreen}$-$ 1,05 & \cellcolor{lsLightGreen}$-$ 0,78 & \cellcolor{lsLightGreen}$-$ 0,35 & $-$ 0,06 & \cellcolor{lsLightGreen}$-$ 0,21\\
& p & \cellcolor{lsLightGreen}0,012 & \textit{0,177} & \cellcolor{lsLightGreen}0,018 & \textit{0,266} & \cellcolor{lsLightGreen}0,001 & \cellcolor{lsLightGreen}0,010 & \cellcolor{lsLightGreen}0,028 & \cellcolor{lsLightGreen}0,004 & \cellcolor{lsLightGreen}0,004 & \cellcolor{lsLightGreen}< 0,001 & \textit{0,078} & \cellcolor{lsLightGreen}< 0,001\\
& N & \cellcolor{lsLightGreen}24 & 24 & \cellcolor{lsLightGreen}24 & 16 & \cellcolor{lsLightGreen}16 & \cellcolor{lsLightGreen}16 & \cellcolor{lsLightGreen}12 & \cellcolor{lsLightGreen}12 & \cellcolor{lsLightGreen}12 & \cellcolor{lsLightGreen}35 & 35 & \cellcolor{lsLightGreen}35\\
& z & \cellcolor{lsLightGreen}$-$ 2,521 &  $-$ 1,351 & \cellcolor{lsLightGreen}$-$ 2,361 & $-$ 1,113 & \cellcolor{lsLightGreen}$-$ 3,221 & \cellcolor{lsLightGreen}$-$ 2,582 & \cellcolor{lsLightGreen}$-$ 2,201 & \cellcolor{lsLightGreen}$-$ 2,875 & \cellcolor{lsLightGreen}$-$ 2,904 & \cellcolor{lsLightGreen}$-$ 4,575 & $-$ 1,765 & \cellcolor{lsLightGreen}$-$ 4,175\\
\lspbottomrule
\end{tabularx}
\caption{\label{tab:06:101}Differenz der Stil- und Inhaltsqualität nach der Anwendung jeder KS-Regel auf Annotationsgruppenebene}
\bspnote{\scriptsize
SQ: Stilqualität; CQ: Inhaltsqualität; Q: Allgemeine Qualität (Mittelwert der SQ und CQ); Diff. SQ = SQ vor $-$ SQ nach KS; analog dazu sind Diff. CQ und Diff. Q.\\ \textit{insignifikante Werte p ${\geq}$ 0,05}; \txgreen{signifikante Werte p < 0,05}}
\end{sidewaystable}

Nur zwei Regeln wirkten sich positiv auf die MÜ-Qualität aus. Das lässt sich in \tabref{tab:06:101} -- wie oben erwähnt -- anhand der signifikant positiven Qualitätsdifferenzmittelwerte (positiver M bei p < 0,05) bei den beiden dominanten Gruppen RR und FF sowie der Gruppe FR erkennen. Eine signifikant positive Qualitätsdifferenz bei der Gruppe RR deutet darauf hin, dass bei zwei fehlerfreien MÜ vor und nach der KS-Anwendung die MÜ-Qualität nach KS höher bewertet wurde (z.\,B. durch eine stilistische Verbesserung). Ebenfalls zeigt ein signifikant positiver Qualitätswert bei der Gruppe FF, dass bei zwei fehlerhaften MÜ vor und nach der KS-Anwendung die MÜ-Qualität nach KS höher bewertet wurde (z.~B. aufgrund des Rückgangs der Fehleranzahl oder des Auftretens von vergleichbar weniger gravierenden Fehlern). Diese zwei Regeln sind:

Erstens -- die Regel \textit{„anz -- Für zitierte Oberflächentexte gerade Anführungszeichen verwenden“} war die einzige Regel, bei der die SQ- und CQ-Werte der FF- und RR-Gruppen nach der Implementierung der Regel signifikant anstiegen. Außerdem war die FR-Gruppe mit dem höchsten Prozentsatz (31~\%) und die RF-Gruppe mit dem niedrigsten (2~\%) vertreten. Orthografisch haben die Anführungszeichen die Aufgabe der „Kennzeichnung der Grenzen eines Einschubs innerhalb des Satzverbandes“ \citep[253]{Nerius2007}. Die Kennzeichnung der Oberflächentexte mithilfe der Anführungszeichen unterstützte demnach systemübergreifend dabei, die Oberflächentexte als spezifische Begriffe bzw. Mehrwortentitäten zu identifizieren und somit den Satz besser zu parsen. Dementsprechend hatte diese Regel einen eindeutigen positiven Einfluss auf die MÜ-Qualität.

Zweitens -- die Regel \textit{„per -- Konstruktionen mit ‚sein + zu + Infinitiv‘ vermeiden“}, bei der die SQ sich in der FF- und RR-Gruppe signifikant verbesserte, während die CQ keine signifikante Veränderung zeigte. Darüber hinaus nahmen sowohl die SQ als auch die CQ in der FR-Gruppe signifikant zu. Systemübergreifend konnte die Passiversatzkonstruktion (vor-KS) in 52~\% der Fälle (Gruppe RR und RF) fehlerfrei übersetzt werden. Die Konstruktion „sein +~zu +~Infinitiv“ ist im Deutschen eine Form des Passiversatzes für ein Passiv mit Modalverb, die ins Englische als reguläres Passiv übersetzt wird (vgl. \citealt{Teich2003}: 92; \citealt{KönigGast2012}: 161). Eine fehlerfreie Übersetzung des Imperativs (nach-KS) hingegen war in 67~\% der Fälle (Gruppe RR und FR) unproblematisch. Durch die ähnlichen Prozentsätze der richtigen Fälle in beiden Szenarien (52~\% vor-KS vs. 67~\% nach-KS) wurde die CQ nicht signifikant beeinflusst. Dennoch konnte die Verwendung des Imperativs (nach-KS) die MÜ überwiegend auf stilistischer Ebene verbessern. Ferner akzentuiert die zu 48~\% fehlerhafte MÜ vor der Regelanwendung (Gruppe FR und FF) die Schwierigkeit der MÜ dieser Passiversatzkonstruktion, die auf den kontrastiven Unterschied zwischen dem Deutschen und Englischen (vgl. \citealt{Teich2003}: 93) zurückgeführt werden kann.

Negative Auswirkungen von KS-Regeln konnten bei drei Regeln beobachtet werden. Das lässt sich in  \tabref{tab:06:101} anhand der signifikant negativen Qualitätsdifferenzmittelwerte (negativer M bei p < 0,05) bei den beiden dominanten Gruppen RR und FF erkennen. Eine signifikant negative Qualitätsdifferenz bei der Gruppe RR weist darauf hin, dass bei zwei fehlerfreien MÜ vor und nach der KS-Anwendung die MÜ\nobreakdash-Qualität nach KS niedriger bewertet wurde (z.~B. aufgrund eines schlechteren Stils). Ebenfalls zeigt eine signifikant negative Qualitätsdifferenz bei der Gruppe FF, dass bei zwei fehlerhaften MÜ vor und nach der KS-Anwendung die MÜ-Qualität nach KS niedriger bewertet wurde (z. B. aufgrund der Zunahme der Fehleranzahl oder des Auftretens von vergleichbar gravierenden Fehlern). Diese drei Regeln sind:

Erstens -- in der Regel \textit{„pak -- Partizipialkonstruktion vermeiden“} war die RF-Gruppe (22~\%) mehr als doppelt so hoch vertreten wie FR (9~\%). Die Untersuchung der FF- (42~\%) und RR-Gruppen (28~\%) zeigt, dass die SQ in FF signifikant fiel und sowohl die SQ als auch die CQ in RR signifikant abnahmen (d. h. bei einem Vergleich zweier fehlerfreien MÜ vor und nach der Regelanwendung sind die SQ und CQ der Partizipialkonstruktionen (vor-KS) höher). In der FR-Gruppe stieg nur die CQ signifikant an, während der SQ-Anstieg marginal war. Die Ergebnisse weisen auf die Komplexität der MÜ von Partizipialkonstruktionen hin (ca. die Hälfte der Fälle vor-KS beinhalteten Fehler), denn sie erschweren den Systemen das Parsen (vgl. \citealt{Reuther2003}). Gleichzeitig zeigen die Ergebnisse, dass die Ersetzung einer Partizipialkonstruktion durch einen Nebensatz mit einer überwiegend stilistischen Qualitätsverschlechterung sowie einer verringerten Verständlichkeit der MÜ verbunden war. Somit konnte die Regel die maschinelle Übersetzbarkeit nicht fördern, wie es von \citet{BernthGdaniec2001} erwartet wird.

Zweitens -- bei der Regel \textit{„pas -- Passiv vermeiden“} fiel die RF-Gruppe (13~\%) größer als die FR (8~\%) aus. Im Gegensatz zur Regel „pak“ war die RR-Gruppe (49~\%) jedoch viel höher vertreten als FF (29~\%), was zeigt, dass die Systeme in beiden Szenarien (Passiv und Aktiv) fast die Hälfte der Sätze korrekt übersetzen konnten. In der FF-Gruppe führte die Verwendung des Aktivs (nach-KS) zu einer signifikant niedrigeren SQ sowie einem insignifikanten Rückgang der CQ. Selbst in der FR-Gruppe waren die beobachteten Anstiege bei SQ und CQ nicht signifikant. In der RR-Gruppe änderten sich die Qualitätswerte bei der Verwendung des Aktivs im Vergleich zum Passiv leicht (ein insignifikanter Rückgang der SQ und eine insignifikante Zunahme der CQ). Eine genaue Betrachtung der Aufteilung der Annotationsgruppen und der Veränderung der Qualitätswerte zeigt, dass die Systeme in der Lage waren, mehr als die Hälfte beider Varianten fehlerfrei zu übersetzen. Zwei Faktoren spielen eine wesentliche Rolle bei dem beobachteten Einfluss der Regel auf die Qualität: (1) welche der beiden Varianten aus Sicht der Bewerter auf stilistischer Ebene für die Handlung erforderlich war (d.\,h. inwiefern es erforderlich war, den Leser direkt anzusprechen (Aktiv) bzw. die Handlung in den Vordergrund zu rücken (Passiv)); und (2) ob der Bewerter eher das Aktiv oder das Passiv gewohnt ist (vgl. \citealt{BaumertVerhein-Jarren2012}: 68). Nach \citet[68f.]{BaumertVerhein-Jarren2012} sind die Leser der unterschiedlichen technischen Dokumentationen die eine oder andere Formulierung gewohnt, z.~B. erwarten lesende Ingenieure Passivsätze; Leser von Zeitschriftenbeiträgen und Präsentationen hingegen das Aktiv.

Drittens -- Regel \textit{„wte -- Keine Wortteile weglassen“}: Auf orthografischer Ebene gibt es keine Regel zur Verwendung bzw. Nicht-Verwendung des Ergänzungsstriches. Entscheidet der Schreibende sich dafür, keine Wortteile wegzulassen, kann dies „zwar gegen stilistische Normen verstoßen, stellt aber keinen orthographischen Fehler dar“ \citep[191]{Nerius2007}. Bei dieser Regel stieg in der FR-Gruppe nur die CQ signifikant an. Die Regelanwendung hatte einen signifikanten negativen Einfluss auf die SQ, der auch in den Gruppen RF, FF und RR festgestellt wurde. Dies zeigte, dass die Wiederholung, die durch die Verwendung vollständiger Wörter anstelle ihrer reduzierten Formen entsteht (z. B. \textit{Start- und Endpunkt} $\to$ \textit{Startpunkt und Endpunkt}), die Verständlichkeit erhöhen konnte und gleichzeitig stilistisch nicht akzeptabel war. Dementsprechend kann der technische Redakteur je nach Textsorte entscheiden, welcher der beiden Aspekte (Verständlichkeit vs. Stil) priorisiert werden soll.

Die übrigen Regeln („fvg -- Funktionsverbgefüge vermeiden“, „kos -- Konditionalsätze mit ‚Wenn‘ einleiten“, „nsp -- Eindeutige pronominale Bezüge verwenden“ und „prä -- Überflüssige Präfixe vermeiden“) konnten auf dieser Analyseebene keinen signifikanten Einfluss auf die MÜ-Qualität zeigen. Alle Qualitätswerte dieser vier Regeln in den FF- und RR-Gruppen waren nicht signifikant.

Somit ergab die Analyse der triangulierten Daten auf Annotationsgruppenebene folgendes Ergebnis: Anders als die Ergebnisse der allgemeinen Auswirkung der KS-Regeln, die sich als positiv erwies (siehe \sectref{sec:6.1}), zeigen nur zwei der neun Regeln eine positive Auswirkung auf den MÜ-Output. Eine weitere Analyse der Qualitätsveränderungen war dementsprechend erforderlich, um sicherzustellen, ob nur zwei Regeln die Qualität verbessern konnten.

\subsection{Überblick über die Qualitätsveränderungen bei den einzelnen Regeln}

Die Analyse der Qualitätsveränderungen basierend auf den Human- und automatischen Evaluationen bestätigt weitgehend die Ergebnisse, die auf Annotationsgruppenebene (\tabref{tab:06:101}) erzielt wurden. \figref{fig:06:151} liefert einen Überblick über die Qualitätsveränderungen auf Regelebene:


\begin{figure}
%\includegraphics[width=\textwidth]{figures/d3-img123.png}
%%[Warning: Draw object ignored]

\includegraphics[width=\textwidth]{figures/d3-img017.png}
%\includegraphics[width=\textwidth]{figures/d3-img017.png}

\caption{\label{fig:06:151} Mittelwert der Qualitätsdifferenz auf Regelebene}
\bspnote{Qualitätsdifferenz = Qualitätswert nach KS \textit{minus} Qualitätswert vor KS}
\end{figure}

Die Regeln „anz -- Für zitierte Oberflächentexte gerade Anführungszeichen verwenden“ und „per -- Konstruktionen mit ‚sein +~zu +~Infinitiv‘ vermeiden“ zeigten wiederholt einen signifikanten positiven Einfluss auf die MÜ-Qualität. Die Scores von TERbase und hLEPOR verbesserten sich. Gleichzeitig wurde basierend auf der Humanevaluation ein signifikanter positiver Einfluss auf die SQ und CQ auf Regelebene festgestellt:

Bei \textit{„anz -- Für zitierte Oberflächentexte}\footnote{{{{Als} }}\textrm{Oberflächentexte}{{{ gelten „alle Texte einer Softwareoberfläche oder Texte, die sich auf einem Gerät befinden“ \citealt{tekom2013}: 117).} }}} \textit{gerade Anführungszeichen verwenden“} war der positive Einfluss auf den Stil auf die klare orthografische Darstellung der Oberflächentexte zurückzuführen. Darüber hinaus wurde durch die Verwendung der Anführungszeichen die Eignung der Übersetzung für die Intention ihres Inhalts verbessert. Die festgestellten höheren Qualitätswerte bestätigen damit, dass eine korrekte Anwendung der Anführungszeichen eine korrekte Übersetzung fördern kann, wie es von der \citep[118]{tekom2013} beabsichtigt wird. Orthografisch haben die Anführungszeichen die Aufgabe, die Grenze einer fremden Äußerung innerhalb des Satzverbandes zu kennzeichnen (\citealt{Nerius2007}: 253f.). Die Kennzeichnung der Oberflächentexte mithilfe der Anführungszeichen konnte daher die Systeme bei der Tokenisierung und dem Parsen unterstützen, was zur Verbesserung der MÜ beitrug. Gleichzeitig wurde die Aussage von \citet{McMurrey2006}, dass die Hervorhebung keine traditionelle Verwendung der Anführungszeichen im Englischen darstelle, in den Kommentaren der Bewerter thematisiert, auch wenn ihre vergebenen Qualität-Scores nach der Regelanwendung dadurch nicht negativ beeinflusst wurden. Eine Identifizierung der Oberflächentexte als feste bzw. spezifische Begriffe kann aber durch das Terminologiemanagement realisiert werden. Kunden- und produktspezifischen Begriffe werden in der Praxis als Termini eingeführt und entsprechend einer vom jeweiligen Unternehmen hinterlegten Terminologiedatenbank konsistent übersetzt (vgl. \citealt{Volk2018}). Auf diese Weise würde die Identifizierung der Termini ebenfalls ein korrektes Parsen unterstützen und somit wäre die Regelanwendung zum Zwecke der maschinellen Übersetzbarkeit nicht mehr erforderlich (Näheres dazu unter \sectref{sec:6.3}).

In Bezug auf \textit{„per -- Konstruktionen mit ‚sein +~zu +~Infinitiv‘ vermeiden“} fanden die Bewerter die Verwendung des Imperativs anstelle der Konstruktion „sein +~zu +~Infinitiv“ stilistisch besser, da der Leser mit dem Imperativ direkt angesprochen und zum Handeln angeregt wurde. In Bezug auf die CQ wurde sowohl die Genauigkeit als auch die Klarheit nach der Regelanwendung erhöht, während der Effekt auf die Klarheit höher war. In dieser Hinsicht konnte -- systemübergreifend -- der gewünschte Effekt der Regelanwendung in der Ausgangssprache, auf den die \citet[86]{tekom2013} und \citet{Congree2018} abzielen, nämlich den Leser direkt anzusprechen und damit eine schnelle und richtige Handlungsumsetzung zu fördern, in der Zielsprache beobachtet werden.

In Bezug auf die Regel \textit{„fvg -- Funktionsverbgefüge vermeiden“}, deckte die Fehlerannotation, obwohl die Analyse der Annotationsgruppen keine wesentliche Qualitätssteigerung widerspiegelte (nur die Qualität in FR stieg signifikant an, siehe \tabref{tab:06:101}), auf, dass die Verwendung des bedeutungstragenden Verbs anstelle des Funktionsverbgefüges zur semantischen und lexikalischen Verbesserung der MÜ beiträgt. Die ausdrucksschwachen Verben waren für die Systeme oft ambig. Insbesondere bei präpositionalen Funktionsverbgefügen (z.~B. ‚zur Anwendung kommen‘), Funktionsverbgefügen mit Komposita (z.~B. ‚Fleckenbehandlung durchführen‘) und Funktionsverbgefügen ohne englisches Pendant (z.~B. ‚Einstellungen vornehmen‘) unterstützte die Regel die Systeme dabei, Trans\-fer- und Parsing-Probleme zu bewältigen. So wurde die MÜ im Rahmen der Humanevaluation nach der Regelanwendung als geeigneter für die Satzintention und verständlicher bewertet. Sowohl bei der Humanevaluation als auch bei der automatischen Evaluation wurde nach der Regelanwendung eine signifikante Steigerung der MÜ-Qualität (SQ, CQ und beide AEM-Scores) festgestellt. Systemübergreifend stehen die Studienergebnisse in Einklang mit vorherigen Studien, die das Vermeiden des Funktionsverbgefüges bzw. ausdrucksschwacher Verben zur Reduzierung der Ambiguität, Vereinfachung der Satzstruktur und somit Verbesserung des Textverständnisses und der Übersetzung empfehlen (\citealt{Siegel2011}; \citealt{Congree2018}).

In Bezug auf \textit{„kos -- Konditionalsätze mit ‚Wenn‘ einleiten“} verbesserten sich SQ und CQ auf Annotationsgruppenebene nicht signifikant (siehe \tabref{tab:06:101}). Auch die automatische Evaluation (sowohl TERbase als auch hLEPOR) zeigte nach der Regelanwendung nur eine geringe Qualitätssteigerung. Das liegt daran, dass 50~\% der Sätze sowohl vor als auch nach der Regelanwendung fehlerfrei übersetzt wurden (Gruppe RR). Nach der Regelanwendung konnten die Fehler nur in 24~\% der Sätze behoben werden (Gruppe FR). Diese Verbesserung war in den Ergebnissen der Humanevaluation im Sinne einer signifikant höheren SQ und CQ nachweisbar. Das Auslassen der Konjunktion ‚Wenn‘, das grammatisch im Deutschen aber nicht im Englischen möglich ist, führte zu Problemen bei der Syntaxanalyse. Die Regelanwendung ermöglichte eine bessere Syntaxanalyse (in Übereinstimmung mit der Feststellung der Komplexität der MÜ elliptischer Konstruktionen von \citealt{Reuther2003}), daher wurden die Qualitätsbewertungen in Bezug auf Genauigkeit, Klarheit und Idiomatik erhöht.

Die Auswirkungen der Regeln „pak -- Partizipialkonstruktion vermeiden“, „wte -- Keine Wortteile weglassen“ und „pas -- Passiv vermeiden“ waren negativ:

Die Regel \textit{„pak} -- \textit{Partizipialkonstruktion vermeiden“} wurde angewendet, indem basierend auf der Partizipialkonstruktion ein Nebensatz generiert wurde. Die Humanevaluation ergab, dass die MÜ der Partizipialkonstruktion idiomatischer, orthografisch korrekter und verständlicher als die des Nebensatzes war. Dementsprechend nahmen systemübergreifend sowohl die SQ als auch die CQ nach der Regelanwendung ab, wobei nur der SQ-Rückgang signifikant war. Die automatische Evaluation bestätigte dieses Ergebnis und zeigte eine signifikante Verschlechterung der Qualitätscores von TERbase und hLEPOR. Anders als von \citet{BernthGdaniec2001} erwartet, nämlich dass diese Regel die maschinelle Übersetzbarkeit fördern würde, konnte dies im Rahmen der Studie nicht bestätigt werden. Die Ergebnisse spiegeln wider, dass die Regel die Satzkomplexität zwar reduzierte, die MÜ nach der Regelanwendung jedoch stilistisch kritisch und weniger verständlich ausfiel.

Im Fall von \textit{„wte} -- \textit{Keine Wortteile weglassen“} empfanden die Evaluatoren die MÜ aufgrund der Substantivwiederholung (anstelle der reduzierten Form) als unnatürlich. So sank die SQ signifikant, während die CQ nicht signifikant abnahm. Darüber hinaus sanken die Bewertungen beider AEMs signifikant. Systemübergreifend zeigen die Ergebnisse somit, dass die MÜ von dieser Form der Ellipsen sich verbessert hat, was einen prägnanten und natürlichen Stil fördert. So kann das Unternehmen je nach Kritikalitätsgrad des Kontexts bei dem Terminologiemanagement individuell festlegen, inwiefern ein vollständiges Ausschreiben der Wörter zwecks der Verständlichkeit bzw. Eindeutigkeit erforderlich ist. Die Regel kann entsprechend über die Spanne zwischen der Prägnanz (durch die Verwendung der abgekürzten Form, d. h. eine Ablehnung der Regel ist denkbar) und der Eindeutigkeit (durch die Verwendung vollständiger Wörter, d. h. eine Anwendung der Regel ist zwingend erforderlich) individuell angewendet werden. Je kritischer der Kontext ist (z. B. in Sicherheitsanweisungen), desto mehr kann die Eindeutigkeit priorisiert werden.

Bei \textit{„pas} -- \textit{Passiv vermeiden“} sanken nach der Regelanwendung systemübergreifend alle Qualitätsparameter (SQ, CQ und beide AEMs) signifikant. Die Regel „Passiv vermeiden“ ist eine weitverbreitete KS-Regel. Mehrere Studien argumentieren, dass das Vermeiden des Passivs die maschinelle Übersetzbarkeit verbessere, da so grammatische Parsing-Probleme umgangen werden könnten (vgl. \citealt{BernthGdaniec2001}; \citealt{Reuther2003}; \citealt{FiedererO’Brien2009}; \citealt{Siegel2013}). Nach den Ergebnissen der Humanevaluation wurde die MÜ-Qualität des Passivs höher bewertet: Die Bewerter fanden das Aktiv (nach-KS) stilistisch nicht ideal für die Satzintention. In Bezug auf die Inhaltsqualität wurde die Genauigkeit des Passivs höher eingestuft. Auf Basis dieser Ergebnisse waren die Systeme in der Lage -- im Gegensatz zu den vorherigen Studien, das Passiv inhaltlich und stilistisch mit einer hohen Qualität zu übersetzen. Das belegt einen Fortschritt bei der MÜ des Passivs und regt -- anders als früher -- dazu an, seine Verwendung in der Technikredaktion prinzipiell nicht zu verbannen, solange der Kontext und die Satzintention dies zulassen.

Parallel zu den triangulierten Ergebnissen der Fehlerannotation und der Humanevaluation (\tabref{tab:06:101}) spiegelten die AEM-Scores (sowohl TERbase als auch hLEPOR) sowie die Humanscores (von SQ und CQ) wider, dass die Regeln „nsp -- Eindeutige pronominale Bezüge verwenden“ und „prä -- Überflüssige Präfixe vermeiden“ keinen signifikanten Einfluss auf die MÜ-Qualität hatten.

\newpage
In Bezug auf \textit{„nsp -- Eindeutige pronominale Bezüge verwenden“}: Die Koreferenzauflösung\footnote{{{{Durch die Koreferenzauflösung wird die Entität, auf die sich die Koreferenz bzw. das Pronomen bezieht, identifiziert (vgl. \citealt{Ng2017}).}}}} stellte bislang für die MÜ-Systeme eine Schwierigkeit dar (vgl. \citealt{Ng2017}). Die Entscheidung, ein Pronomen zu verwenden oder es durch seine Referenz zu ersetzen, wird für gewöhnlich auf einer Fall-zu-Fall-Basis abhängig von dem Satz und der Formulierung der vorangehenden und folgenden Sätze getroffen (\citealt{BernthGdaniec2001}). So betrachteten Bernth und Gdaniec (ebd.: 187) das Vermeiden der Pronomen als einen „trade-off between MTranslatability and natural-sounding language“. Dies könnte der Grund sein, warum keine signifikante Auswirkung festgestellt werden konnte: Auf der einen Seite war die Verwendung der pronominalen Referenz von Vorteil, wenn die Identifizierung der Referenz als schwierig eingeschätzt wurde. Dies führte zu einer Erhöhung der MÜ-Klarheit, worauf \citet[137]{tekom2013} und \citet{Congree2018} ebenfalls abzielen. Auf der anderen Seite wurde die Wiederholung der Referenz in einigen Fällen auf stilistischer Ebene kritisiert. Wie \figref{fig:06:150} zeigt, waren die MÜ-Systeme in 70~\% der Fälle (Gruppe RR plus RF) in der Lage, die Koreferenzen korrekt aufzulösen. Dies stellt eine relativ hohe Erfolgsquote bei der Problematik der Koreferenzauflösung dar, wodurch ein natürlicher Stil gefördert wird. Dieser Fortschritt erlaubt es, die Regalanwendung auf die Fälle, bei denen keine Mehrdeutigkeit toleriert werden kann, z. B. kritischen Kontexte wie Sicherheitsanweisungen, einzuschränken.

In Bezug auf die Regel \textit{„prä -- Überflüssige Präfixe vermeiden“} war die RR-Annotationsgruppe sehr dominant (68~\%). Wenn die MÜ-Systeme ein Verb mit und ohne überflüssiges Präfix (z. B. ‚anbieten‘ und ‚bieten‘) korrekt übersetzen konnten, waren die Übersetzungen in beiden Fällen identisch (richtige Übersetzung: ‚\textit{offer‘}). Eine große Anzahl korrekter identischer Übersetzungen vor und nach der Regelanwendung führte entsprechend zu einer vergleichbaren Qualität. Das begründet das insignifikante Ergebnis der Qualitätsdifferenz (vor vs. nach der Regelanwendung). Gleichzeitig zeigt das Ergebnis der Gruppe FR, dass nach der Regelanwendung MÜ-Fehler in 13~\% der Fälle -- meistens bei Präfix-Verb-getrennten Fällen -- behoben wurden und die Stil- und Inhaltsqualität dieser Gruppe signifikant anstiegen. Systemübergreifend stimmt in diesem Sinne das Ergebnis mit der \citet[111]{tekom2013} sowie vorherigen Studien (\citealt{BernthGdaniec2001}; \citealt{Siegel2011}; \citealt{Siegel2013}) überein, dass diese Regel zur Förderung der maschinellen Übersetzbarkeit beiträgt.

Wie \figref{fig:06:152} zeigt, hatten die aus der Humanevaluation resultierenden sowie die in der automatischen Evaluation errechneten Qualitätsveränderungen einen vergleichbaren Verlauf. Ein solcher vergleichbarer Verlauf deutet darauf hin, dass die Ergebnisse der beiden Methoden sich gegenseitig untermauern, denn die MÜ-Qualität wurde nach den beiden Methoden auf unterschiedliche Weise untersucht und gemessen.

\begin{figure}
\includegraphics[width=\textwidth]{figures/d3-img125.png}



%\includegraphics[width=\textwidth]{figures/d3-img123.png}}\\

% & Differenz Q


%\includegraphics[width=\textwidth]{figures/d3-img121.png}
%Differenz hLEPOR

%Differenz TERbase\\
\caption{\label{fig:06:152} Differenz der allgemeinen Qualität und Differenz der AEM-Scores auf Regelebene}
\bspnote{Differenz = Szenario nach KS \textit{minus} Szenario vor KS}
\end{figure}

Darüber hinaus zeigte der Spearman-Test für die Korrelation zwischen der Differenz in der allgemeinen Qualität und der Differenz in den AEM-Scores bei den einzelnen Regeln, dass die in der Humanevaluation und automatischen Evaluation festgestellten Qualitätsveränderungen übereinstimmen bzw. sich gegenseitig bestätigen: Bei den Regeln „fvg -- Funktionsverbgefüge vermeiden“, „kos -- Konditionalsätze mit ‚Wenn‘ einleiten“ und „pas -- Passiv vermeiden“ gab es eine signifikante starke positive Korrelation (ρ > 0,5); bei den verbleibenden Regeln eine signifikante mittlere positive Korrelation (ρ > 0,3).

\section{\label{sec:6.3}Auswirkung der KS auf Regel- und MÜ-Systemebene}

Bisher zeigten die Ergebnisse auf Regelebene, dass vier Regeln einen positiven Einfluss auf die MÜ-Qualität haben („anz -- Für zitierte Oberflächentexte gerade Anführungszeichen verwenden“, „per -- Konstruktionen mit ‚sein +~zu +~Infinitiv‘ vermeiden“, „fvg -- Funktionsverbgefüge vermeiden“ und „kos -- Konditionalsätze mit ‚Wenn‘ einleiten“); und drei Regeln haben tendenziell einen negativen Einfluss auf die MÜ-Qualität („pak -- Partizipialkonstruktion vermeiden“, „pas -- Passiv vermeiden“ und „wte -- Keine Wortteile weglassen“) -- insbesondere hinsichtlich der Stilqualität. Für diese sieben Regeln wurde auf MÜ-Systemebene untersucht, welche Systeme den identifizierten Effekt aufweisen. Die Auswirkung der beiden verbleibenden Regeln „nsp -- Eindeutige pronominale Bezüge verwenden“ und „prä -- Überflüssige Präfixe vermeiden“ war nicht eindeutig. Für diese Regeln wurde auf MÜ-Systemebene genauer geprüft, ob bei einem bestimmten System signifikante Auswirkungen nachweisbar sind.

Bei der Diskussion werden die Systeme basierend auf den erzielten Ergebnissen verglichen. Die Qualitätsveränderungen werden in Zusammenhang mit den korrelierenden Fehlertypen (Ergebnisse der Fehlerannotation) bzw. den beeinflussten Qualitätskriterien (siehe [3a und 3b] in \figref{fig:4:8}) erörtert. Die Lieferung einer exakten Interpretation, was genau im Hintergrund jedes Systems zu einem bestimmten Output (bzw. zum Auftritt oder zur Aufhebung eines bestimmten Fehlers) geführt hat, erfordert eine Glas-Box-Analyse und geht somit über den Umfang dieser Studie hinaus.

\begin{table}
\footnotesize
\setlength{\tabcolsep}{2pt}
%\begin{tabularx}{\textwidth}{lllllllllllllllllll}
\begin{tabularx}{\textwidth}{llllllllllll}
\lsptoprule
%{} & \multicolumn{2}{c}{\txgreen{\textbf{-anz-}}} & \multicolumn{2}{c}{\txgreen{\textbf{-per-}}} & \multicolumn{2}{c}{\txgreen{\textbf{-fvg-}}} & \multicolumn{2}{c}{\txgreen{\textbf{-kos-}}} & \multicolumn{2}{c}{\colorbox{smRed}{\textbf{-pak-}\strut}} & \multicolumn{2}{c}{\colorbox{smRed}{\textbf{-pas-}}} & \multicolumn{2}{c}{\colorbox{smRed}{\textbf{-wte-}}} & \multicolumn{2}{c}{\colorbox{smYellow}{\textbf{-nsp-}}} & \multicolumn{2}{c}{\colorbox{smYellow}{\textbf{-prä-}}}\\
%\cmidrule(lr){2-3}\cmidrule(lr){4-5}\cmidrule(lr){6-7}\cmidrule(lr){8-9}\cmidrule(lr){10-11}\cmidrule(lr){12-13}\cmidrule(lr){14-15}\cmidrule(lr){16-17}\cmidrule(lr){18-19}
%& Anno. & HuEv & Anno. & HuEv & Anno. & HuEv & Anno. & HuEv & Anno. & HuEv & Anno. & HuEv & Anno. & HuEv & Anno. & HuEv & Anno. & HuEv\\
%\midrule
%{HMÜ} & F (- -) & SQ ++ & F (- -) & SQ ++ & F (-) & SQ + & F (- -) & SQ ++ & F + & SQ (- -) & F (- -) & SQ (-) & F + & SQ (-) & F = & SQ (-) & F (- -) & SQ +\\
%(Bing)&  & CQ ++ &  & CQ ++ &  & CQ + &  & CQ ++ &  & CQ (-) &  & CQ + &  & CQ (-) &  & CQ + &  & CQ +\\
%{NMÜ} & F (-) & SQ ++ & F = & SQ + & F (-) & SQ + & F + & SQ = & F + & SQ (- -) & F + & SQ (-) & F (-) & SQ (- -) & F ++ & SQ (- -) & F (-) & SQ (-)\\
%(Google)&  & CQ + &  & CQ (-) &  & CQ (-) &  & CQ + &  & CQ (-) &  & CQ (-) &  & CQ (-) &  & CQ (-) &  & CQ (-)\\
%{RBMÜ} & F (- -) & SQ ++ & F ++ & SQ + & F (- -) & SQ ++ & F (-) & SQ (-) & F + & SQ (- -) & F + & SQ (- -) & F (-) & SQ (- -) & F (-) & SQ + & F (-) & SQ +\\
%(Lucy)&  & CQ ++ &  & CQ (-) &  & CQ + &  & CQ (-) &  & CQ (-) &  & CQ (-) &  & CQ (-) &  & CQ ++ &  & CQ (-)\\
%{SMÜ} & F (- -) & SQ ++ & F (- -) & SQ ++ & F (- -) & SQ + & F (- -) & SQ + & F (-) & SQ (-) & F + & SQ (-) & F = & SQ (- -) & F (-) & SQ (-) & F (-) & SQ +\\
%(SDL)&  & CQ ++ &  & CQ ++ &  & CQ + &  & CQ ++ &  & CQ + &  & CQ (-) &  & CQ (-) &  & CQ + &  & CQ +\\
%{HMÜ} & F (-) & SQ ++ & F ++ & SQ ++ & F (- -) & SQ ++ & F (-) & SQ (-) & F ++ & SQ (-) & F ++ & SQ (- -) & F (-) & SQ (-) & F (-) & SQ (-) & F (-) & SQ +\\
%(Systran)&  & CQ ++ &  & CQ (-) &  & CQ + &  & CQ (-) &  & CQ ++ &  & CQ (- -) &  & CQ + &  & CQ (-) &  & CQ +\\

& & \multicolumn{2}{l}{\textbf{HMÜ}}	&	\multicolumn{2}{l}{\textbf{NMÜ}}	& \multicolumn{2}{l}{\textbf{RBMÜ}}	&	\multicolumn{2}{l}{\textbf{SMÜ}}	& 	\multicolumn{2}{l}{\textbf{HMÜ}}\\
& & \multicolumn{2}{l}{\textbf{(Bing)}}	&	\multicolumn{2}{l}{\textbf{(Google)}}	& \multicolumn{2}{l}{\textbf{(Lucy)}}	&	\multicolumn{2}{l}{\textbf{(SDL)}}	& 	\multicolumn{2}{l}{\textbf{(Systran)}}\\
\midrule
\txgreen{\textbf{-anz-}} &	Anno. &	\cellcolor{lsLightGray}F (- -)	& \cellcolor{lsLightGray} &	F (-) &	&	\cellcolor{lsLightGray}F (- -) &\cellcolor{lsLightGray} &	\cellcolor{lsLightGray}	F (- -)  &	\cellcolor{lsLightGray} &	F (-) &\\
& HuEv	&\cellcolor{lsLightGray}SQ ++ &	\cellcolor{lsLightGray}CQ ++	&\cellcolor{lsLightGray}SQ ++	&CQ +&	\cellcolor{lsLightGray}SQ ++&	\cellcolor{lsLightGray}CQ ++	&\cellcolor{lsLightGray}SQ ++	&\cellcolor{lsLightGray}CQ ++	&\cellcolor{lsLightGray}SQ ++	&\cellcolor{lsLightGray}CQ ++\\
\tablevspace
\txgreen{\textbf{-per-}} &	Anno.& 	\cellcolor{lsLightGray}F (- -) &\cellcolor{lsLightGray} &		F =	& &	\cellcolor{lsLightGray}F ++	&\cellcolor{lsLightGray}	& \cellcolor{lsLightGray}F (- -)	&\cellcolor{lsLightGray} &	\cellcolor{lsLightGray}F ++ &\cellcolor{lsLightGray}\\
& HuEv &	\cellcolor{lsLightGray}SQ ++	&\cellcolor{lsLightGray}CQ ++	&SQ +	&CQ (-)	&SQ +	&CQ (-)	&\cellcolor{lsLightGray}SQ ++	&\cellcolor{lsLightGray}CQ ++	&\cellcolor{lsLightGray}SQ ++	&CQ (-)\\
\tablevspace
\txgreen{\textbf{-fvg-}}	& Anno. &	F (-)	& &	F (-)	& &	\cellcolor{lsLightGray}F (- -)	&\cellcolor{lsLightGray} &	\cellcolor{lsLightGray}F (- -)	&	\cellcolor{lsLightGray}& \cellcolor{lsLightGray}F (- -)&\cellcolor{lsLightGray}\\
& HuEv &	SQ +	& CQ +	& SQ +	& CQ (-)	& \cellcolor{lsLightGray}SQ ++ &	CQ +	& SQ +	& CQ +	& \cellcolor{lsLightGray}SQ ++	 & CQ +\\
\tablevspace
\txgreen{\textbf{-kos-}}	& Anno.	& \cellcolor{lsLightGray}F (- -)	&\cellcolor{lsLightGray} &	F +	& &	F (-)	& &	\cellcolor{lsLightGray}F (- -)	& \cellcolor{lsLightGray}&	F (-)	&\\
& HuEv &	\cellcolor{lsLightGray}SQ ++	& \cellcolor{lsLightGray}CQ ++	& SQ =	& CQ +	& SQ (-)	& CQ (-)	& SQ +	& \cellcolor{lsLightGray}CQ ++	 & SQ (-)	& CQ (-)\\
\tablevspace
\colorbox{smRed}{\textbf{-pak-}\strut} &	Anno.& 	F +	&	& F +	& &	F +	& &	F (-)	& &	\cellcolor{lsLightGray}F ++	&\cellcolor{lsLightGray}\\
& HuEv &	\cellcolor{lsLightGray}SQ (- -) &	CQ (-) &	\cellcolor{lsLightGray}SQ (- -)	& CQ (-) &	\cellcolor{lsLightGray}SQ (- -) &	CQ (-)	& SQ (-) &	CQ +	 & SQ (-) &	\cellcolor{lsLightGray}CQ ++\\
\tablevspace
\colorbox{smRed}{\textbf{-pas-}\strut}	& Anno. &	\cellcolor{lsLightGray}F (- -) & &		F +	& &	F +	& &	F +	 & &	\cellcolor{lsLightGray}F ++ &\\
& HuEv	& SQ (-) &	CQ + &	SQ (-) &	CQ (-) &	\cellcolor{lsLightGray}SQ (- -) &	CQ (-) &	SQ (-)	& CQ (-)	& \cellcolor{lsLightGray}SQ (- -)	& \cellcolor{lsLightGray}CQ (- -)\\
\tablevspace
\colorbox{smRed}{\textbf{-wte-}\strut} &	Anno.&	F +	& &	F (-)& 	&	F (-) & &		F =	& &	F (-)&\\
& HuEv &	SQ (-)	&CQ (-)&	\cellcolor{lsLightGray}SQ (- -)	&CQ (-)	&\cellcolor{lsLightGray}SQ (- -)	&CQ (-)&	\cellcolor{lsLightGray}SQ (- -)	&CQ (-)&	SQ (-)&	CQ +\\
\tablevspace
\colorbox{smYellow}{\textbf{-nsp-}\strut}&	Anno.&F =	& &	\cellcolor{lsLightGray}F ++	&\cellcolor{lsLightGray} &	F (-)	& &	F (-) & &		F (-)&\\
& HuEv &	SQ (-)	&CQ +	&\cellcolor{lsLightGray}SQ (- -)	&CQ (-)	&SQ +	&\cellcolor{lsLightGray}CQ ++	&SQ (-)	&CQ +	&SQ (-)	&CQ (-)\\
\tablevspace
\colorbox{smYellow}{\textbf{-prä-}\strut}	& Anno.&	\cellcolor{lsLightGray}F (- -) &	&	F (-) &	&	F (-)	& &	F (-)	& &	F (-) &\\
& HuEv &	SQ +	& CQ +	& SQ (-)	& CQ (-)&	SQ +	& CQ (-)&	SQ +	& CQ +	& SQ +	& CQ +\\
\lspbottomrule
\end{tabularx}
\caption{\label{tab:06:102}Einfluss der einzelnen Regeln auf die Fehleranzahl sowie Stil- und Inhaltsqualität bei den einzelnen Systemen}
\bspnote{\txgreen{Regeln}, die auf \textit{Regelebene} einen signifikanten +~Einfluss zeigten
\\
\colorbox{smRed}{Regeln\strut}, die auf \textit{Regelebene} einen signifikanten (-) Einfluss zeigten
\\
\colorbox{smYellow}{Regeln\strut}, die auf \textit{Regelebene} keinen signifikanten Einfluss zeigten
\\
\txgray{Graue Zellen}: signifikante Veränderung
\\
F: Fehleranzahl gleich geblieben;  SQ: Stilqualität;  CQ: Inhaltsqualität; (- -) signifikanter Rückgang;  ++~signifikanter Anstieg; (-) nicht signifikanter Rückgang; +~nicht signifikanter Anstieg; = unverändert geblieben}
\end{table}

\tabref{tab:06:102} liefert eine Übersicht über die sich positiv, negativ und nicht signifikant auswirkenden Regeln zusammen mit ihren Auswirkungen auf die Fehleranzahl und die Qualitätswerte bei den einzelnen Systemen:

\subsection{Regeln mit positiver Wirkung}

Regel \textit{„anz} -- \textit{Für zitierte Oberflächentexte gerade Anführungszeichen verwenden“} war die einzige Regel, die mit einem Rückgang der Fehleranzahl sowie einer Verbesserung der SQ und CQ bei allen MÜ-Systemen verbunden war.

Das verbesserte Parsen, das durch die Kennzeichnung der Oberflächentexte mithilfe der Anführungszeichen realisiert wurde, war meist mit der Korrektur zweier Fehlertypen verbunden: Großschreibung des Oberflächentexts (OR.2) und Wortstellung (GR.10). Beim RBMÜ-System \textit{Lucy} korrelierte die Korrektur dieser beiden Fehlertypen nach der Regelanwendung stark mit der Qualitätssteigerung. Im HMÜ-System \textit{Bing} korrelierte nur die Korrektur des Wortstellungsfehlers stark mit der Qualitätsverbesserung. In den anderen Systemen konnten keine Korrelationen zwischen den Fehlertypen und der Qualität festgestellt werden.

Der Rückgang der Fehleranzahl nach der Regelanwendung war im Fall des anderen HMÜ-Systems \textit{Systran} und des NMÜ-Systems \textit{Google Translate} aus jeweils unterschiedlichen Gründen nicht signifikant. In \textit{Systran} war die Fehleranzahl sehr hoch und änderte sich kaum nach der Regelanwendung. In \textit{Google Translate} hingegen fiel die Fehleranzahl sowohl vor- als auch nach-KS sehr gering aus. Das \textit{NMÜ-System Google Translate} war in der Lage 83~\% der Sätze mit und ohne Anführungszeichen fehlerfrei zu übersetzen (gefolgt von nur 17~\% bei dem SMÜ-System \textit{SDL}). Dementsprechend verzeichnete die NMÜ die höchsten SQ- und CQ-Werte.

Der Anstieg in der SQ bei dieser Regel war der einzige signifikante Anstieg beim NMÜ-System Google Translate unter allen Regeln (siehe \tabref{tab:06:102}). Dieser Anstieg kam im Datensatz durch drei Veränderungen zustande: Die erste Veränderung beruhte darauf, dass der Oberflächentext vor KS als Terminus erkannt und großgeschrieben wurde, jedoch ohne Hervorhebung (z. B. \textit{Change device parameters function} als Übersetzung für \textit{Funktion Geräteparameter ändern}); mit der Anwendung der Regel wurde die MÜ zusätzlich mithilfe der Anführungszeichen hervorgehoben (\textit{function $"$Change device parameters$"$}). Zweites gab es Fälle, bei denen der Oberflächentext als Terminus nicht erkannt und kleingeschrieben wurde (z.~B. \textit{additional information button} als Übersetzung für \textit{Taste Zusatzinformation}); eine Erkennung des Terminus gelang nach der Regelanwendung und somit wurde er großgeschrieben und mithilfe der Anführungszeichen hervorgehoben (\textit{$"$Additional information$"$ button}). Die dritte Veränderung kam in zwei (von 24) Fällen vor, bei denen Oberflächentexte, die aus mehreren Wörtern bestanden (sog. Mehrwortentitäten), zerlegt wurden, was in Wortstellungsfehlern resultierte (z. B. \textit{Upload function \ldots from the device} als Übersetzung für \textit{Funktion Upload vom Gerät}). Durch die Regelanwendung wurde der Wortstellungsfehler behoben (\textit{$"$Upload from device$"$ function}).

Auf Basis dieser Ergebnisse, die aus der NMÜ (Stand: Ende 2016) hervorgehen, konnte die Verwendung von geraden Anführungszeichen das generische NMÜ-System Google Translate dabei unterstützen, Mehrwortentitäten korrekt zu parsen, spezifische bzw. seltene Termini korrekt zu übersetzen und somit Wortstellungsfehler zu vermeiden. Daraufhin stieg -- nach der Humanevaluation -- die Stilqualität, i.~S.~v. Eignung der Übersetzung für die Intention ihres Inhalts (\citealt{HutchinsSomers1992}: 163) nach der Regelanwendung. Problematisch dabei blieb, dass die Hervorhebung keine typische Anwendung der Anführungszeichen im Englischen ist \citep{McMurrey2006}. Dieser Einwand kam in den Kommentaren der Bewerter wiederholt vor. In dieser Phase der NMÜ-Entwicklung (2016 / 2017) thematisierten vorherige Studien die Schwäche der NMÜ bei der Übersetzung von seltenen Wörtern und Eigennamen (vgl. \citealt{LeSchuster2016}; \citealt{Köhn2017}) sowie die Problematik der Terminologieintegration in der NMÜ (vgl. \citealt{Eisold2017}; \citealt{Köhn2017}).

Bei einer wiederholten Übersetzung des Datensatzes Anfang 2020 stieg der Prozentsatz der fehlerfreien Übersetzungen (Annotationsgruppe RR) von 83~\% auf 96~\% (konkret beinhaltete nur ein Satz von 24 Sätzen einen Großschreibungsfehler bei der Übersetzung \textit{ohne} Anführungszeichen). Diese Verbesserung deutet auf einen Fortschritt bei der Übersetzung von seltenen Wörtern bzw. Eigennamen sowie bei der Identifizierung von Mehrwortentitäten (wie z.~B. der Optionsbezeichnung ‚Auswahl nach Anwendungs-Code‘) hin. Ferner wurden zuletzt einige Ansätze zur Terminologieintegration in der NMÜ entwickelt, die einen Fortschritt bei der konsistenten Übersetzung spezifischer Termini (einschließlich der Out-of-vocabulary-Fälle)\footnote{{{{„Out of vocabulary“ sind externe Wörter, die dem Modell unbekannt sind (vgl. \citealt{Eisold2017}).}}}} nach festgelegten externen Terminologielisten (bzw. Terminologiedatenbanken) und eine damit verbundene Steigerung der Gesamtübersetzungsqualität belegen (vgl. \citealt{ChatterjeeEtAl2017}; \citealt{HaslerEtAl2018}; \citealt{DinuEtAl2019}).\footnote{{{{Mehr zu diesen Studien unter \sectref{sec:3.2.4}.}}}} Weitere aktuelle Studien konnten mithilfe mehrerer Strategien einen Fortschritt bei der Übersetzung unterschiedlicher Mehrwortausdrücke (Multiword Expressions, MWEs) realisieren (vgl. \citealt{GamalloGarcia2019}; \citealt{RiktersBojar2019}; \citealt{ZaninelloBirch2020}).

\largerpage
Auf Basis dieser Entwicklung würde das Terminologiemanagement das System bei der Identifizierung und Übersetzung der spezifischen Termini und Mehrwortentitäten unterstützen. Gleichzeitig kann eine vom Unternehmen bestimmte Formatierung (z.~B. kursiv) für die Hervorhebung sorgen, um die Eignung der Übersetzung im Sinne der Textintention sicherzustellen. Mit dieser Konstellation wäre die Verwendung der geraden Anführungszeichen bei Oberflächentexten zur Unterstützung der maschinellen Übersetzbarkeit nicht mehr erforderlich.

Für die zweite Regel \textit{„fvg} -- \textit{Funktionsverbgefüge vermeiden“} war der allgemeine positive Einfluss auf den MÜ-Output auf Systemebene wie folgt:

Für das RBMÜ-System \textit{Lucy} und ein HMÜ-System (\textit{Systran}) war diese Regel besonders vorteilhaft, denn nach ihrer Anwendung sank die Fehleranzahl und die SQ verbesserte sich signifikant. In dem anderen HMÜ-System (\textit{Bing}) und dem SMÜ-System \textit{SDL} nahm die Fehleranzahl ab und die SQ und CQ nahmen zu; die Änderungen waren jedoch nicht signifikant. Die Bewerter stellten fest, dass die Verwendung des bedeutungstragenden Verbs (nach-KS) anstelle des Funktionsverbgefüges (vor-KS) die Übersetzung verständlicher und stilistisch aufmerksamkeitserregend ausfallen lässt, was in Einklang mit der Empfehlung der Regelsätze von \citet{Congree2018} und \citet{Siegel2011} zur Regelanwendung steht. Dieses Ergebnis deutet ferner darauf hin, dass der Effekt, der von der \citet[107]{tekom2013} bei der Ausgangssprache, beabsichtigt ist, nämlich den Satz konkreter und direkter zu gestalten, sich in der Zielsprache widerspiegelt. Die Analyse des \textit{NMÜ-Systems Google Translate} ergab distinkte Ergebnisse: Die Fehleranzahl war minimal; Google Translate war in der Lage, 88~\% der Sätze vor und nach der Regelanwendung fehlerfrei (Gruppe RR) zu übersetzen (gefolgt von 46~\% in Bing). Es verzeichnete die höchsten SQ und CQ unter allen Systemen sowohl vor als auch nach der Regelimplementierung.

Da nicht alle deutschen Funktionsverbgefüge ein Pendant auf Englisch haben, waren die ausdrucksschwachen Verben für die Systeme oft ambig bzw. es entstand ein Transferproblem (vgl. \citealt{BaumertVerhein-Jarren2012}: 107). Das „Vermeiden des Funktionsverbgefüges“ hat überwiegend die Satzsemantik positiv beeinflusst. Die Verwendung des bedeutungstragenden Verbs anstelle des Funktionsverbgefüges war mit der Korrektur einer Reihe semantischer Fehler, insbesondere Kollokationsfehler (SM.13), sowie lexikalischer Fehler verbunden. Lexikalische Fehler traten dann auf, wenn die Systeme das Funktionsverbgefüge wörtlich übersetzten (z. B. das Übersetzen von ‚zur Verfügung stellen‘ als ‚represent available‘ statt ‚provide‘). In \textit{Lucy} korrelierte die Korrektur der semantischen Fehler mit einem Anstieg von SQ und CQ. In \textit{Bing} und \textit{SDL} war eine Korrelation zwischen den lexikalischen Fehlern \textit{Wort ausgelassen} (LX.3) und Wort zusätzlich falsch eingefügt (LX.4) und der Qualität zu beobachten. In den anderen Systemen waren keine weiteren Korrelationen nachweisbar.

\largerpage%long distance
Wie unter \sectref{sec:4.4.2.3} diskutiert, ist die Verwendung des Funktionsverbgefüges in manchen Fällen zum Ausdrücken von bestimmten Bedeutungsnuancen oder mangels eines äquivalenten bedeutungstragenden Verbs erforderlich (\citealt{BaumertVerhein-Jarren2012}: 107). Auf Basis der Ergebnisse bietet der NMÜ-Ansatz im Vergleich zu den früheren Ansätzen in solchen Fällen eine solide Architektur zur Übersetzung des Funktionsverbgefüges. Somit ist das Vermeiden des Funktionsverbgefüges, wie es von Regelsätzen wie \citet{Congree2018} und \citet{Siegel2011} empfohlen ist, zum Zwecke der maschinellen Übersetzbarkeit nicht erforderlich.

Die Anwendung der Regel \textit{„kos} -- \textit{Konditionalsätze mit ‚Wenn‘ einleiten“} war mit einer Verringerung der Fehleranzahl in allen Systemen mit Ausnahme des \textit{NMÜ-Systems Google Translate} (aufgrund seiner geringen Fehleranzahl) verbunden. Das Einleiten von Konditionalsätzen mit ‚Wenn‘ zeigte hauptsächlich eine positive lexikalische Wirkung, die oft mit der Korrektur eines Wortstellungsfehlers verbunden war. Während deutsche Konditionalsätze mit einem Verb am Satzbeginn (d.~h. ohne die Konditionalkonjunktion ‚Wenn‘) formuliert werden können, ist dies im Englischen nicht der Fall. Daher war das Weglassen der Konjunktion (vor KS) mit zwei Fehlertypen verbunden: dem Fehlen der Konjunktion ‚Wenn‘ (LX.3) und der falschen Platzierung des Verbs des Konditionalsatzes (GR.10). Diese Fehler sanken nach der Regelanwendung und ihr Rückgang korrelierte entsprechend negativ mit der Qualitätssteigerung, und zwar eine starke Korrelation im Falle des LX.3 bzw. eine mittlere Korrelation im Falle des GR.10.\clearpage

Der MÜ-Output der einzelnen Systeme wurde unterschiedlich beeinflusst. Das zeigte wiederum, dass die Regelanwendung nicht bei allen Systemen erforderlich war. Im Folgenden eine genaue Untersuchung auf Systemebene: Der Rückgang der MÜ-Fehler war nur bei einem HMÜ-System (\textit{Bing}) und dem SMÜ-System \textit{SDL} signifikant: sowohl LX.3 als auch GR.10 bei Bing und nur LX.3 bei SDL. Infolgedessen wurde nur in diesen beiden Systemen eine signifikante Qualitätsverbesserung erzielt -- in Bing sowohl für die SQ als auch für die CQ und in SDL nur für die SQ. In \textit{Bing} und \textit{SDL} korrelierte ebenfalls die Korrektur dieser Fehler stark mit der Qualitätssteigerung. In den anderen Systemen wurden keine Korrelationen mit einem bestimmten Fehlertyp beobachtet. Nach der Regelanwendung fanden die Bewerter die MÜ genauer, verständlicher und idiomatischer. Im Gegensatz dazu sanken beide Qualitätswerte im RBMÜ-System \textit{Lucy} und im anderen HMÜ-System (\textit{Systran}) nach der Regelanwendung. Das liegt daran, dass Lucy 71~\% und Systran 58~\% vor und nach der Regelanwendung fehlerfrei übersetzen konnten (Gruppe RR). Bei den beiden Systemen wurden die Fehler nur in zwei Fällen nach der Regelanwendung behoben (Gruppe FR). Dementsprechend zeigen Lucy und Systran einen gewissen Fortschritt bei der Übersetzung dieser Art der elliptischen Konstruktion. In \textit{Google Translate} betrug der Prozentsatz fehlerfreier MÜ vor und nach der Regelanwendung (Gruppe RR) 92~\% (gefolgt von 71~\% in Lucy). Dies zeigte erneut die höchsten SQ und CQ in beiden Szenarien und weist darauf hin, dass das NMÜ-System die sprachlichen Unterschiede sowie Syntaxanalyse dieser elliptischen Konstruktion meistern konnte.

Regel \textit{„per -- Konstruktionen mit ‚sein + zu + Infinitiv‘ vermeiden“} zeigte eine generell positive Auswirkung auf den MÜ-Output:


\largerpage%long distance
In Einklang mit \citegen{Reuther2003} Beobachtung, dass die systemspezifischen Eigenschaften Einfluss auf die maschinelle Übersetzung stilistischer Phänomene wie z. B. der Passiversatzkonstruktion (sein + zu + Infinitiv) haben, wurden (vor-KS) grammatische Schwierigkeiten nur bei zwei Systemen, dem HMÜ-System \textit{Bing} und dem SMÜ-System \textit{SDL}, beobachtet. Nach der Regelanwendung sanken zwei Grammatikfehler signifikant: GR.8 falsches Verb (Zeitform, Komposition, Person) und GR.10 falsche Wortstellung. In \textit{Bing} und \textit{SDL} korrelierte die Korrektur dieser Fehlertypen stark mit einem Anstieg der Qualitätswerte. Auf der anderen Seite war die Regelanwendung mithilfe des Imperativs (anstelle der Konstruktion „sein + zu +~Infinitiv“) mit dem Auftreten des lexikalischen Additionsfehlers (LX.4) (nach KS) im RBMÜ-System \textit{Lucy} und im anderen HMÜ-System (\textit{Systran}) verbunden, da die MÜ-Systeme in einigen Fällen fälschlicherweise das Subjekt \textit{you} hinzufügten (z. B. ‚\textit{Verriegeln Sie} die Kontakte‘ $\to$ ‚\textit{You lock} the contacts‘). In \textit{Systran} korrelierte die Korrektur dieses lexikalischen Fehlers stark und signifikant mit dem Anstieg der Qualitätswerte. Das \textit{NMÜ-System Google Translate} konnte 96~\% der Sätze vor und nach der Regelanwendung fehlerfrei (Gruppe RR) übersetzen (gefolgt von nur 42~\% in Systran). Daher war die Stil- und Inhaltsqualität des MÜ-Outputs in beiden Szenarien bei dem NMÜ-System Google Translate am höchsten unter allen MÜ-Systemen (mit einer minimalen Zunahme der SQ und einer minimalen Abnahme der CQ nach KS).

\subsection{Regeln mit negativer Wirkung}
\largerpage
Die Regel \textit{„pak} -- \textit{Partizipialkonstruktion vermeiden“} wirkte sich in allen MÜ-Systemen generell negativ auf die SQ aus. Die CQ (überwiegend die Verständlichkeit) stieg nur in zwei MÜ-Systemen an: minimal im SMÜ-System \textit{SDL} und signifikant in einem HMÜ-System (\textit{Systran}). Die negative Wirkung war auch in Bezug auf die Fehleranzahl zu beobachten. Die Fehleranzahl stieg nach der Regelanwendung in allen Systemen an, außer in \textit{SDL}, wo ein leichter Rückgang festgestellt wurde. Bei \textit{Systran} fiel der Anstieg der Fehleranzahl signifikant aus.

Das \textit{NMÜ-System Google Translate} hatte keine Schwierigkeit, Partizipialkonstruktionen zu übersetzen. Ferner waren 71~\% der Übersetzungen von \textit{Google Translate} vor und nach der Regelanwendung fehlerfrei, d.~h. Annotationsgruppe RR, (gefolgt von nur 29~\% in Bing). In allen anderen Systemen war die Fehleranzahl sowohl vor als auch nach der Regelimplementierung wesentlich höher. Darüber hinaus zeigte Google Translate die höchsten Qualitätswerte in beiden Szenarien, was seinen Fortschritt beim Parsen von komplexen Strukturen wie Partizipialkonstruktionen widerspiegelt.

Mit dieser Regel waren zwei verschiedene MÜ-Fehlertypen verbunden: Erstens erschweren besonders lange Partizipialkonstruktionen die Satzstruktur und damit das Parsen, was zu Wortstellungsfehlern (GR.10) führte. Dieser Fehlertyp trat insbesondere beim SMÜ-System \textit{SDL} und HMÜ-System \textit{Bing} bei der Übersetzung von Partizipialkonstruktionen auf und wurde nach der Regelanwendung behoben. Nach der Implementierung der Regel hatten jedoch alle Systeme Probleme mit der Platzierung von Kommas in Nebensätzen, insbesondere in denen eine Unterscheidung zwischen ‚which‘ und ‚that‘ gemacht werden musste (vgl. \citealt{Swan1980}: 527 ff.).\footnote{\textrm{Man unterscheidet im Englischen zwischen restriktiven und nicht-restriktiven Relativsätzen: Im restriktiven Relativsatz wird die Bedeutung des Nomens, das beschrieben wird, begrenzt. Ohne den restriktiven Relativsatz ändert sich die Bedeutung des gesamten Satzes. Für restriktive Relativsätze verwendet man ‚that‘. Der nicht-restriktiver Relativsatz bietet lediglich zusätzliche Informationen über das Nomen und kann ohne Einfluss auf die Bedeutung entfernt werden. Für nicht-restriktive Relativsätze verwendet man ‚which‘. (vgl. \citealt{Swan1980}: 527 ff.)} } Daher stieg die Anzahl der Zeichensetzungsfehler (OR.1) bei Bing, Systran und SDL signifikant an. Im Gegensatz zum Deutschen werden im Englischen nach der Regelanwendung für den Nebensatz nicht immer Kommas benötigt. Nichtsdestotrotz ist es zu beachten, dass die Verwendung von ‚which‘ vs. ‚that‘ im Allgemeinen nicht selten problematisch ist und in der Regel kontextbezogene Informationen erfordert, um über die korrekte Verwendung entscheiden zu können (vgl. ebd.).

Bei der Regel \textit{„pas} -- \textit{Passiv vermeiden“} zeigten alle MÜ-Systeme -- mit Ausnahme von Bing -- nach der Regelanwendung einen Anstieg der Fehleranzahl. Die beiden HMÜ-Systeme lieferten gegensätzliche Ergebnisse: Bei \textit{Bing} verringerte sich die Gesamtzahl der Fehler signifikant, während sie bei \textit{Systran} erheblich zunahm. In den anderen drei Systemen war die Zunahme der Fehleranzahl nicht signifikant. \textit{Bing} und \textit{Google Translate} konnten 71~\% der Sätze sowohl im Passiv als auch im Aktiv fehlerfrei (Gruppe RR) übersetzen (gefolgt von 58~\% in Lucy).

Generell unterstreicht das Ergebnis den Fortschritt bei der MÜ des Passivs. Nach der Regelanwendung sank sowohl die SQ als auch die CQ in allen MÜ-Systemen, mit Ausnahme von \textit{Bing}, in dem die CQ leicht anstieg. In \textit{Systran} war der SQ- und CQ-Rückgang signifikant. Bei \textit{Lucy} war der Rückgang der SQ signifikant. Sowohl vor als auch nach der Implementierung der Regel lieferte \textit{Google Translate} die höchsten SQ und CQ. Die Qualitätsveränderungen korrelierten bei keinem System mit einem bestimmten Fehlertyp, denn die Regelanwendung war mit einer Zunahme verschiedener Fehlertypen bei allen Systemen verbunden. Diese Zunahme (nach KS) war allerdings bei keinem der Fehlertypen signifikant.

Regel \textit{„wte} -- \textit{Keine Wortteile weglassen“}: Der Ergänzungsstrich signalisiert die Auslassung eines Wortteils sowie die „Zusammengehörigkeit räumlich getrennter Bestandteile zusammengesetzter oder abgeleiteter koordinierter Wörter innerhalb der Wortgruppe“ \citep[191]{Nerius2007}. Gleichzeitig spielen die Satzzeichen wie in diesem Fall der Ergänzungsstrich laut \citet[2]{Reuther2003} eine wesentliche Rolle bei der MÜ: „Punctuation marks are very sensitive with respect to all applications where linguistic processing is done automatically.” Diese Regel war mit einer geringen Abnahme der Fehleranzahl bei \textit{Google Translate}, \textit{Lucy} und \textit{Systran} verbunden. Aufgrund der Unterschiede in den Rechtschreibregeln im Deutschen vs. Englischen in Bezug auf die Verwendung des Ergänzungsstriches trug die Regel zu einer verbesserten Tokenisierung bei. So wurde beobachtet, dass die Regelanwendung mit einer Verringerung des Zeichensetzungsfehlers verbunden war. Die Fehleranzahl stieg jedoch im HMÜ-System \textit{Bing} marginal an und blieb im SMÜ-System \textit{SDL} unverändert.

Trotz der unterschiedlichen Auswirkungen auf die Fehleranzahl sanken SQ und CQ in allen Systemen, mit Ausnahme von \textit{Systran}, bei dem die CQ leicht anstieg. Leichte Anstiege in der CQ kamen bei Wortkonstellationen vor, die in ihrer abgekürzten Form nicht gebräuchlich sind (z. B. ‚Wasser-, Gasrohre oder stromführende Leitungen‘ (im Vergleich zu für gewöhnlich abgekürzten Begriffen wie ‚Vor- und Nachteile‘)). Bei ungeläufigen abgekürzten Begriffen und insbesondere bei einem kritischen Inhalt, z. B. bei wichtigen Warn- oder Fehlermeldungen einer Maschine, verbesserte die Regel die Eindeutigkeit. Die SQ-Abnahme war in drei Systemen (\textit{Google Translate}, \textit{Lucy} und \textit{SDL}) signifikant, da die Bewerter die Nomenwiederholung als unnatürlich empfanden (z. B. \textit{Milch} in \textit{Sojamilch und laktosefreie Milch} nach-KS anstelle von \textit{Soja- und laktosefreie Milch} vor-KS) (siehe „Quantitätsmaxime“ von \citealt{Grice1975}: 26).\footnote{\textrm{Die Quantitätsmaxime von \citet[26]{Grice1975} bezieht sich auf die Quantität der Informationen: „Make your contribution as informative as is required (for the current purpose of the exchange) […] not more informative than is required”.} } Auch bei dieser Regel zeigte \textit{Google Translate} die geringste Fehleranzahl (75~\% der Übersetzungen waren sowohl vor als auch nach KS fehlerfrei, gefolgt von 46~\% in SDL (Gruppe RR)) und die höchsten Qualitätswerte in beiden Szenarien.

Was die Übersetzbarkeit betrifft, zeigte die Regel „\textit{Keine Wortteile weglassen}“ in Bezug auf die Fehleranzahl, die Qualitätsratings und die AEM-Scores keine signifikante Verbesserung. Obwohl die Ergebnisse der Studie von einer vorangestellten Terminologieintegration sicherlich positiv beeinflusst worden wären, konnte Google Translate auch ohne Terminologieintegration die Wortteile in den meisten Fällen korrekt übersetzen.

\subsection{Regeln ohne signifikante Auswirkung}

Die Regel \textit{„nsp -- Eindeutige pronominale Bezüge verwenden“} hatte von einem System zum anderen unterschiedliche Auswirkungen auf die MÜ-Qualität. Nur das RBMÜ-System \textit{Lucy} und das \textit{NMÜ-System Google Translate} zeigten signifikante Qualitätsveränderungen: \textit{Lucy} zeigte eine leichte Zunahme der SQ und eine signifikante Zunahme der CQ, während für \textit{Google Translate} genau das Gegenteil der Fall war -- nämlich eine signifikante Abnahme der SQ und ein leichter Rückgang der CQ. Dies könnte durch die verschiedenen Änderungen in der Fehleranzahl erklärt werden. Im Gegensatz zu den anderen Systemen konnte \textit{Google Translate} die Pronomen meist korrekt übersetzen (vor-KS), während die Verwendung einer pronominalen Referenz (nach-KS) in einigen Fällen stilistisch kritisiert wurde. Gleichzeitig waren 83~\% der Übersetzungen von \textit{Google Translate} vor und nach der Regelanwendung fehlerfrei (gefolgt von 67~\% in \textit{Lucy}). Darüber hinaus erzielte \textit{Google Translate} in beiden Szenarien die höchsten Qualität-Scores.

Die Verwendung eindeutiger pronominaler Referenzen zeigte zwei unterschiedliche Auswirkungen auf die MÜ-Systeme: In \textit{Lucy}, \textit{SDL} und \textit{Systran} war die Regelanwendung mit einer Reduzierung des semantischen Fehlers Verwechselung des Sinns (SM.11) verbunden. Dieser Fehler trat insbesondere bei der Übersetzung von Demonstrativpronomen (‚diese‘ und ‚dies‘) auf, da die MÜ-Systeme Schwierigkeiten hatten, die Referenz zu identifizieren und korrekt zu übersetzen. Dementsprechend fanden die Bewerter die Übersetzung nach der Anwendung der Regel eindeutiger. Auf der anderen Seite war die Anwendung dieser Regel mit einem Anstieg des lexikalischen Konsistenzfehlers (LX.6) in \textit{Bing} und \textit{SDL} verbunden. Um diese Regel zu implementieren, sollte ein Substantiv im Hauptsatz \textit{nicht} durch ein Pronomen im Nebensatz ersetzt werden; stattdessen sollte der Pronomenbezug verwendet werden. In einigen Fällen übersetzten die MÜ-Systeme die zweite Instanz des Substantivs anders (bzw. verwendeten Synonyme), was zu einem Konsistenzfehler (vgl. \citealt{Mertin2006}: 249) und damit zu einer verringerten Genauigkeit führte. Ein Konsistenzfehler könnte jedoch vermieden werden, wenn die verwendeten Termini in die Systeme integriert würden.\footnote{\textrm{Die Studie wurde mit generischen Black-Box-Systemen durchgeführt. Für die Auswahlkriterien der Systeme siehe \sectref{sec:4.4.1}. Für den Umgang mit den spezifischen Termini im Rahmen der Studie siehe Schritt [4] unter \sectref{sec:4.4.3.1}.} }

Regel \textit{„prä} -- \textit{Überflüssige Präfixe vermeiden“} war – im Allgemeinen – mit einer sehr geringen Fehleranzahl vor der Regelanwendung verbunden, die nach der Regelanwendung abnahm. Diese Reduktion war nur in einem HMÜ-System (\textit{Bing}) signifikant. Die SQ hat sich in allen Systemen mit Ausnahme von \textit{Google Translate} leicht verbessert. Auch die CQ stieg in \textit{Bing}, \textit{Lucy} und \textit{Systran} minimal an. Bei \textit{Google Translate} wurden 88~\% der Sätze vor und nach der Regelanwendung fehlerfrei übersetzt, d.~h. Annotatiotionsgruppe RR, (gefolgt von 71~\% bei Bing). Die Qualitätswerte in \textit{Google Translate} zeigten nach der Regelanwendung eine minimale Abnahme. Gleichzeitig waren sie sowohl vor als auch nach der Implementierung der Regel die höchsten unter allen MÜ-Systemen.

Systemspezifisch kann das Ergebnis daher nur bis zu einem gewissen Grad die Empfehlung der \citet[111]{tekom2013} sowie vorheriger Studien (\citealt{BernthGdaniec2001}; \citealt{Siegel2011}; \citealt{Siegel2013}) zur Regelanwendung im Hinblick auf die maschinelle Übersetzbarkeit bestätigen. Das Vermeiden überflüssiger Präfixe unterstützte alle Systeme mit Ausnahme des NMÜ-Systems Google Translate dabei, das Verb korrekt zu parsen. Insbesondere trennbare Verben waren oft schwer zu parsen; abhängig von der Satzstruktur stehen die Präfixe in manchen Fällen am Ende des Satzes -- weit entfernt vom Rest des Verbs. In solchen Fällen übersetzten die Systeme das Präfix zusätzlich, d. h. unabhängig vom Verb. Die Anwendung der Regel führte zur Korrektur des lexikalischen Additionsfehlers (LX.4), was wiederum die Genauigkeit der Übersetzung verbesserte.

Wie die Ergebnisse der einzelnen Regeln auf Systemebene zeigen, korrelierten die Veränderungen der Fehlertypen mit denen der Qualitätswerte bei den älteren Systemen (RBMÜ, SMÜ und HMÜ), während bei dem NMÜ-System gar keine einschlägige Korrelation bestand (\tabref{tab:06:104}).

\section{\label{sec:6.4}Regelübergreifende Auswirkung der KS auf MÜ-Systemebene}

Bei der Diskussion der Ergebnisse auf Systemebene (regelübergreifend) werden die Systeme im Hinblick auf die MÜ-Qualitätsveränderungen verglichen, dabei werden die Qualitätsveränderungen in Zusammenhang mit den korrelierenden Fehlertypen (Ergebnisse der Fehlerannotation) sowie den beeinflussten Qualitätskriterien (siehe [3a und 3b] \figref{fig:4:8}) erörtert.

Vor der Diskussion der Ergebnisse auf Systemebene sind zwei Aspekte anzumerken: Erstens, die Studie wurde mithilfe generischer Black-Box-Systeme durchgeführt (für die Auswahlkriterien der Systeme siehe \sectref{sec:4.4.1}). Dementsprechend wurden die untersuchten Systeme nicht mit Korpora trainiert, die auf das Testmaterial abgestimmt sind. Ein Training der Systeme hätte einen Einfluss auf die Ergebnisse. Vorherige Studien (vgl. \citealt{RamirezPoloHaller2005}; \citealt{AikawaEtAl2007}; \citealt{LehmannEtAl2012}) untersuchten diesen Einfluss. Bei der Arbeit mit trainierten Systemen werden erwartungsgemäß Phrasen bzw. Redewendungen, die in den Trainingsdaten vorkommen, korrekt übersetzt (\citealt{RamirezPoloHaller2005}; \citealt{AikawaEtAl2007}). Gleichermaßen sollte eine bestimmte KS-Regel in den Trainingsdaten häufig angewendet werden, so übersetzt das System Segmente, die nach dieser Regel formuliert sind, besser als nicht-Regel-konforme Segmente (vgl. \citealt{LehmannEtAl2012}). Auf dieser Grundlage beabsichtigten \citet{LehmannEtAl2012} die Untersuchung einer automatischen Auswahl der anzuwendenden Regeln basierend auf dem verwendeten Trainingskorpus. In der vorliegenden Studie wurde hingegen gezielt mit Systemen gearbeitet, die vor der Untersuchung nicht mit bestimmten Korpora trainiert wurden. Würden die Systeme in der Studie vorab trainiert, wären die Ergebnisse von den Trainingsdaten abhängig (d.~h. bessere Ergebnisse beim Kontrollierten Szenario, wenn die Korpora kontrolliert sind, und umgekehrt) (vgl. \citealt{Reuther2003}). Außerdem wäre ein angemessener Vergleich der Ergebnisse der verschiedenen Systeme nicht realisierbar, denn die Systeme, die basierend auf Trainingskorpora arbeiten, hätten je nach Kontrollgrad einen Vorteil bzw. Nachteil gegenüber dem regelbasierten System.

Ferner wurde die Problematik der Terminologieübersetzung in der Studie umgangen, indem die spezifischen Termini in den analysierten Sätzen durch geläufige Begriffe ersetzt wurden (Genaueres dazu unter \sectref{sec:4.4.3.1}, Schritt [4]). Die Terminologieintegration wäre in der RMBÜ, SMÜ und HMÜ mithilfe fachspezifischer Wörterbücher bzw. durch das Training mittels fachspezifischer Parallelkorpora möglich gewesen, jedoch befand sich die Terminologieintegration in der NMÜ zur Zeit der Durchführung der Studie noch in der experimentellen Phase (vgl. \citealt{Eisold2017}). Damit die Studie auf einer einheitlichen Basis durchgeführt wird, wurden alle Systeme in ihrem Ist-Zustand, d.~h. ohne Terminologieintegration oder Training mit domänenspezifischen Daten, verwendet.

Zweitens, jedes der analysierten Systeme hat je nach seinem Ansatz seine(n) eigene(n) Aufbau, Modelle, Abläufe, Trainingsdaten bzw. Lexika. Alle diese Komponenten beeinflussen den Output. Diese Komponenten bleiben aber bei der Analyse der Vor- und Nach-KS-Szenarien konstant. Die einzige Variable ist die Anwendung bzw. Nicht-Anwendung einer Regel. Da die Studienfrage den Fokus darauf legt, was die Anwendung der einzelnen Regeln bei den unterschiedlichen analysierten MÜ-Ansätzen bewirkt (und nicht warum bzw. wie eine Regel eine bestimmte Wirkung bei einem bestimmten MÜ-Ansatz zeigt), konnte die Frage im Rahmen einer Black-Box-Analyse beantwortet werden. Eine Untersuchung des Hintergrunds der Auswirkung (z.~B. Grund des Auftritts oder der Aufhebung eines bestimmten Fehlers) bei jedem Ansatz erfordert eine Glas-Box-Analyse und geht somit über den Umfang dieser Studie hinaus.

Auf Systemebene zeigen die Ergebnisse, dass sich das getestete NMÜ-System bei Anwendung der KS-Regeln anders verhielt. Im Allgemeinen kann beobachtet werden, dass die analysierten KS-Regeln die Leistung der RBMÜ-, SMÜ- und HMÜ-Systeme verbesserten (\tabref{tab:06:103}). Die Qualität wurde höher bewertet und die Fehleranzahl sank bei der Übersetzung der KS-Regel-konformen Ausgangssätze. Dies ist der Fall mit Ausnahme von Systran, dessen Fehleranzahl sowohl vor-KS als auch nach-KS sehr hoch war und nach-KS insignifikant stieg, mehr dazu siehe \sectref{sec:5.4.1}. Der NMÜ-Output wurde hingegen durch die Regeln nicht positiv beeinflusst; vor der KS-Anwendung fiel die Fehleranzahl geringer aus und die Qualität wurde höher bewertet.

Bezüglich der Fehleranzahl und der Fehlerannotationsgruppen zeigte das NMÜ-System die beste Leistung, unabhängig davon, ob die KS-Regeln angewendet wurden oder nicht (d. h. vor und nach KS), siehe \sectref{sec:5.4.1} und \sectref{sec:5.4.3}. Bei einigen Regeln erhöhte sich sogar die Fehleranzahl in der KS-Stelle nach der Regelanwendung, siehe \sectref{sec:5.3.4.1} (Regel 4) und \sectref{sec:5.3.5.1} (Regel 5).

\begin{table}

\begin{tabularx}{\textwidth}{lXlXlX}
\lsptoprule
 & \textbf{Fehleranzahl} & & \textbf{Stilqualität} & & \textbf{Inhaltsqualität}\\
\midrule
(1) & Bing (HMÜ) & (1) & Bing  & (1) & Bing \\
 & $-$~52,4~\% &  &  +~5,2~\%  & &  +~10,3~\% \\
 & signifikant &  &  signifikant &  &  signifikant\\
(2) & SDL (SMÜ) & (2) & SDL  & (2) & SDL \\
 &  $-$~38,0~\%  &  &  +~5,0~\% &  &  +~6,3~\% \\
 & signifikant &  & signifikant & &  signifikant\\
(3) & Lucy (RBMÜ) & (3) & Lucy  & (3) & Systran \\
 & $-$~16,1~\%  &  &  +~1,0~\% &  &  +~1,1~\% \\
 & \textit{nicht} signifikant &  &\textit{nicht} signifikant &  &  \textit{nicht} signifikant\\
(4) & Google (NMÜ)  & (4) & Systran & (4) & Lucy \\
 & +~6,1~\%  &  &  +~0,7~\%  &  &  +~0,5~\% \\
 & \textit{nicht} signifikant &  & \textit{nicht} signifikant &  &  \textit{nicht} signifikant\\
(5) & Systran (HMÜ)  & (5) & Google  & (5) & Google \\
 &  +~10,1~\%  &  &  $-$~1,7~\%  &  &  $-$~1,1~\% \\
 &  \textit{nicht} signifikant &  & signifikant &  & \textit{nicht} signifikant\\
\lspbottomrule
\end{tabularx}
\caption{\label{tab:06:103}Reihenfolge der Systeme nach dem positiven Einfluss der KS-Regeln}
\end{table}


In Bezug auf die Fehlertypen zeigt \tabref{tab:06:104} die Fehlertypen, die bei jedem MÜ-System durch die Anwendung der KS{}-Regeln negativ oder positiv signifikant beeinflusst wurden:

\begin{table}
\begin{tabularx}{\textwidth}{Xlllll}
\lsptoprule
%& OR.1 –Zeichensetzung & OR.2 –Großschreibung & LX.3 – Wort ausgelassen & LX.4 – Wort zusätzl. falsch eingefügt & LX.5 – Wort unübers. geblieben & LX.6 – Konsistenzfehler & GR.7 – Falsche Wortart & GR.8 – Falsches Verb & GR.9 – Kongruenzfehler & GR.10 – Falsche Wortstellung & SM.11 – Verwechsel. des Sinns & SM.12 – Falsche Wahl & SM.13 –Kollokationsfehler\\
%Bing (HMÜ) & + & (-) & (-) &  &  &  &  & (-) &  & (-) &  &  & \\
%SDL (SMÜ) & + & (-) & (-) & (-) &  &  &  &  & (-) & (-) &  &  & \\
%Lucy (RBMÜ) &  & (-) &  &  &  &  &  &  &  &  &  &  & (-)\\
%Systran (HMÜ) & + &  & (-) & + &  &  &  &  &  &  &  &  & (-)\\
%Google (NMÜ) &  &  &  &  &  &  &  &  &  &  &  &  & \\
& \textbf{Bing} &	\textbf{SDL} 	& \textbf{Lucy} &	\textbf{Systran} &	\textbf{Google} \\
& \textbf{(HMÜ)} &	\textbf{(SMÜ)}	&  \textbf{(RBMÜ)}& \textbf{(HMÜ)}&	\textbf{(NMÜ)}\\
\midrule
OR.1 – Zeichensetzung &	+&	+	& &	+ &\\
\tablevspace
OR.2 – Großschreibung &	($-$)&	($-$)&	($-$)& &\\
\tablevspace
LX.3 – Wort ausgelassen	& ($-$)	& ($-$)	& &	($-$)&\\
\tablevspace
LX.4 – Wort zusätzl. falsch eingefügt	& &	($-$)	& &	+	&\\
\tablevspace
LX.5 – Wort unübers. geblieben& & & & & \\
\tablevspace
LX.6 – Konsistenzfehler& & & & &\\
\tablevspace
GR.7 – Falsche Wortart	& & & & & \\
\tablevspace
GR.8 – Falsches Verb &	($-$)	& & & &\\
\tablevspace
GR.9 – Kongruenzfehler	& &	($-$)	& & &\\
\tablevspace
GR.10 – Falsche Wortstellung &	($-$) &	($-$)	& & &\\
\tablevspace
SM.11 – Verwechsel. des Sinns	& & & & &\\
\tablevspace
SM.12 – Falsche Wahl	& & & & &\\
\tablevspace
SM.13 –Kollokationsfehler	& &	&	($-$)&	($-$) &\\
\lspbottomrule
\end{tabularx}
\caption{\label{tab:06:104}Fehlertypen mit signifikanter Veränderung nach der KS-Anwendung auf MÜ-Systemebene}
\end{table}

Auf den ersten Blick erkennt man, dass es beim NMÜ-System im Gegensatz zu den anderen vier Systemen keine Fehlertypen gab, die von der KS-Anwendung signifikant beeinflusst wurden. Gefolgt vom NMÜ-System Google Translate zeigte das RBMÜ-System Lucy signifikante Änderungen in der Fehleranzahl von nur zwei Fehlertypen (OR.2 „Großschreibung“ und SM.13 „Kollokationsfehler“). Die höchste Anzahl von Fehlertypen, die von der Anwendung der Regeln signifikant betroffen waren, fand sich beim SMÜ-System SDL (6 Fehlertypen). Im Allgemeinen stehen diese Ergebnisse in Einklang mit denen einer vorherigen Studie (vgl. \citealt{LommelEtAl2014}), die zeigten, dass bei RBMÜ-Systemen semantische Fehler am häufigsten und Auslassungsfehler selten auftreten; und umgekehrt treten bei SMÜ-Systemen Auslassungs- und Wortstellungsfehler am häufigsten und semantische Fehler selten auf. Die Ergebnisse zeigen damit, dass die KS-Regeln bei der Beseitigung häufiger Fehlertypen beider Systemansätze von Nutzen waren. Schließlich verdeutlicht ein Vergleich der signifikanten Fehlertypen in den HMÜ-Systemen Bing und Systran, dass die Fehlertypen sehr unterschiedlich waren, was verschiedene Schwächen in den Architekturen beider HMÜ-Systeme widerspiegelt.

Bei dem NMÜ-System zeigt eine genauere Betrachtung der Regeln und der Fehlertypen, bei denen die Fehleranzahl zunahm, dass diese Zunahme hauptsächlich in Regel 4 „Eindeutige pronominale Bezüge verwenden“ (+4 Fehler) und Regel 5 „Partizipialkonstruktionen vermeiden“ (+6 Fehler in lediglich 4 Sätze) vorkam. In Bezug auf Regel 4 konnte das NMÜ-System die Pronomen größtenteils fehlerfrei übersetzen. Die Verwendung des Bezugsworts anstelle des Pronomens (d.~h. nach der Regelanwendung) war in einigen Fällen mit einem Kongruenzfehler (Fehlertyp GR.9) verbunden. In Bezug auf Regel 5 hatte das NMÜ-System keine Schwierigkeiten, Partizipialkonstruktionen zu übersetzen und somit zeigte es eine hohe Leistung vor der Regelanwendung. Wenn das System jedoch einen Nebensatz anstelle der Partizipialkonstruktion (d.~h. nach der Regelanwendung) übersetzte, trat der Fehlertyp OR.1 „Orthografie – Zeichensetzung“ auf, insbesondere in Fällen, in denen eine Unterscheidung zwischen den Relativpronomen ‚which‘ und ‚that‘ gemacht werden musste (vgl. \citealt{Swan1980}: 527 ff.).\footnote{{{{Man unterscheidet im Englischen zwischen restriktiven und nicht-restriktiven Relativsätzen: Im restriktiven Relativsatz wird die Bedeutung des Nomens, das beschrieben wird, begrenzt. Ohne den restriktiven Relativsatz ändert sich die Bedeutung des gesamten Satzes. Für restriktive Relativsätze verwendet man ‚that‘. Der nicht-restriktiver Relativsatz bietet lediglich zusätzliche Informationen über das Nomen und kann ohne Einfluss auf die Bedeutung entfernt werden. Für nicht-restriktive Relativsätze verwendet man ‚which‘. (vgl. \citealt{Swan1980}: 527 ff.)}}}} Dessen ungeachtet, wie die Kommentare bzw. vorgeschlagenen Übersetzungen der Humanevaluatoren zeigten, ist die Entscheidung über ‚which‘ vs. ‚that‘ in manchen Fällen nicht unkompliziert und normalerweise sind kontextbezogene Informationen für eine korrekte Verwendungsentscheidung erforderlich. Ein weiterer Fehlertyp, der im NMÜ-Output nach Anwendung der KS-Regeln anstieg, war der Fehlertyp SM.11 „Semantik – Verwechslung des Sinns“. Dieser Fehler trat jedoch in deutlich ambigen Fällen auf, wie z. B. bei der Übersetzung von ‚Wenn‘ als ‚When‘ anstelle von ‚If‘ in Regel 3 „Konditionalsätze mit ‚Wenn‘ einleiten“. Eine solche Sinnverwechslung wurde allerdings auch in den Ergebnissen der Humanevaluation beobachtet, da die Bewerter Kontextinformationen brauchten, um den Sinn in solchen Fällen zu klären, wie von den Bewertern kommentiert wurde.

Die Abnahme der Qualität nach der KS-Anwendung bei dem NMÜ-System war hauptsächlich auf die signifikant verringerte Stilqualität zurückzuführen. Die Inhaltsqualität ging zwar nach der KS-Anwendung ebenfalls zurück, jedoch war der Rückgang nicht signifikant. Wie bisherige Studien zeigen, ist die NMÜ in der Lage, typische Morphologie- und Grammatikschwierigkeiten zu bewältigen und darüber hinaus eine im Wesentlichen flüssige Übersetzung zu liefern (vgl. \citealt{BentivogliEtAl2016}; \citealt{ToralSanchez-Cartagena2017}; \citealt{VanBrusselEtAl2018}). Diese auffällige Flüssigkeit der Übersetzung zählt zu den Stärken des NMÜ-Ansatzes (vgl. \citealt{ToralSanchez-Cartagena2017}). Angesichts dieser Entwicklung ist die Berücksichtigung der Stilqualität neben der Inhaltsqualität für eine entwicklungs- bzw. forschungsstandgemäße Bewertung erforderlich, obwohl die KS im Allgemeinen die Verständlichkeit und nicht den Stil im Fokus hat. Auf Basis der erzielten Ergebnisse bot das NMÜ-System sowohl mit als auch ohne Anwendung der Regeln nicht nur eine hohe Verständlichkeit und Genauigkeit, sondern auch einen dienlichen Stil. Im Folgenden eine Diskussion der Regeln, bei denen sich die Stilqualität signifikant und die Inhaltsqualität insignifikant bei dem NMÜ-System veränderten.

Regel 1 „Für zitierte Oberflächentexte gerade Anführungszeichen verwenden“ war die einzige Regel, bei der sich die Stilqualität bei dem NMÜ-System nach der Regelanwendung aufgrund der klaren orthografischen Darstellung der Oberflächentexte signifikant verbesserte. Gleichzeitig stieg die Inhaltsqualität insignifikant. Diese Verbesserung könnte jedoch auch erreicht werden, indem (1) die Oberflächentexte durch die Formatierung (z. B. fett, kursiv usw.) hervorgehoben und (2) firmen-, produktspezifische bzw. ungebräuchliche Termini der Oberflächentexte in das System integriert würden (siehe Beispiele unter \sectref{sec:6.3} in der Analyse der Regel \textit{„anz} -- \textit{Für zitierte Oberflächentexte gerade Anführungszeichen verwenden“)}. Gleichzeitig gab es drei andere Regeln, bei denen die Stilqualität signifikant und die Inhaltsqualität insignifikant abnahmen: Regel 4 „Eindeutige pronominale Bezüge verwenden“, da die Wiederholung des Bezugsworts für die Bewerter stilistisch kritisch war; Regel 5 „Partizipialkonstruktionen vermeiden“, da die Bewerter die Übersetzung der Partizipialkonstruktion (vor KS) für idiomatischer hielten; Regel 9 „Keine Wortteile weglassen“, da die Wiederholung einiger Wortteile als unnatürlich wahrgenommen wurde. Zumal die Inhaltsqualität (Verständlichkeit und Genauigkeit) bei diesen drei Regeln vor der Regelanwendung höher ausfiel, deuten die Ergebnisse auf einen Fortschritt bei der Koreferenzauflösung (Regel 4) sowie bei der Übersetzung von Ellipsen (Regel 9) und von komplexen Strukturen wie der Partizipialkonstruktion (Regel 5) hin. Fortschritte bei der Übersetzung elliptischer Konstruktionen sowie bei der Koreferenzauflösung konnten zudem aktuelle Studien durch den Einsatz kontextfähiger NMÜ-Modelle und Strategien zur NMÜ auf Dokumentebene realisieren (vgl. \citealt{MüllerEtAl2018}; \citealt{StojanovskiFraser2018}; \citealt{VoitaEtAl2018}; \citealt{Matusov2019}; \citealt{StojanovskiFraser2019}; \citealt{VoitaEtAl2019}). Von dieser Entwicklung kann die technische Dokumentation profitieren, denn eine Anwendung dieser Regeln zwecks der maschinellen Übersetzbarkeit ist im Falle der NMÜ nicht erforderlich. Unabhängig davon können diese Regeln angewendet werden, wenn die Einheitlichkeit bzw. Verständlichkeit Vorrang hat.

Die Veränderungen der AEM-Scores vor vs. nach KS waren bei allen Systemen minimal und entsprechend nicht signifikant. Gleichzeitig zeigte der Spearman-Korrelationstest in allen MÜ-Systemen hochsignifikante mittlere oder starke positive Korrelationen zwischen den Differenzen der Scores von TERbase und hLEPOR und den Qualitätsdifferenzen (\tabref{tab:05:99} unter \sectref{sec:5.4.9}).  Diese positiven Korrelationen lassen die Ergebnisse der beiden Analysen (die Humanevaluation und die automatische Evaluation) sich gegenseitig bestätigen, denn eine Qualitätssteigerung ging mit einem verbesserten AEM-Score – und umgekehrt – einher.

\section{\label{sec:6.5}Fazit}

In diesem Kapitel wurden die Ergebnisse reflektierend zusammengefasst. In den Ergebnissen wurde eine kollektive positive Wirkung der Anwendung der analysierten KS-Regeln auf den MÜ-Output systemübergreifend festgestellt. Auf Regelebene systemübergreifend zeigten nur vier der neun untersuchten Regeln eine positive Wirkung auf den MÜ-Output. Drei Regeln zeigten hingegen einen negativen Effekt und die letzten zwei Regeln konnten keinen eindeutigen Effekt nachweisen. Betrachtet man die Regeln auf einer tieferen Ebene, nämlich auf Regel- und Systemebene, so waren die Ergebnisse unterschiedlich von einem MÜ-System zum anderen. Die Vergleichsanalyse der MÜ-Ansätze regelübergreifend am Beispiel der untersuchten Systeme zeigte, wie das NMÜ-System mit einer sehr niedrigen Fehleranzahl und sehr hohen Qualitätswerten sowohl vor als auch nach der Regelanwendung die Systeme der RBMÜ-, SMÜ- und HMÜ-Ansätze an Leistung deutlich übertraf. Während die KS-Anwendung mit einer Verbesserung der Inhalts- und Stilqualität bei den RBMÜ-, SMÜ- und HMÜ-Systemen auf unterschiedlichem Niveau einherging, fielen die Stil- und Inhaltsqualität (d.~h. die Verständlichkeit und Genauigkeit) beim NMÜ-System vor der Anwendung der Regeln höher aus. Somit liefern die Ergebnisse der untersuchten Regeln ein Indiz dafür, dass eine Anwendung der KS zu Zwecken der maschinellen Übersetzbarkeit bei den früheren MÜ-Ansätzen (RBMÜ, SMÜ und HMÜ) jedoch nicht beim NMÜ-Ansatz förderlich sein kann. Je nach Anwendungskontext bleibt es dementsprechend kritisch zu bewerten, inwiefern eine solche KS-Anwendung erforderlich ist.
