\addchap{\lsAcknowledgementTitle}


Nach Jahren intensiver Arbeit liegt sie nun vor Ihnen: meine Dissertation. Damit ist es an der Zeit, mich bei denjenigen zu bedanken, die mich in dieser herausfordernden, aber auch immens bereichernden Phase begleitet und mir die Anfertigung dieser Promotionsschrift ermöglicht haben:

An erster Stelle -- es mögen mir andere verzeihen -- muss ich meine Doktormutter nennen. Frau Prof. Dr. Silvia Hansen-Schirra war zu jeder Zeit für mich da, motivierte mich unendlich und beseitigte alle aufkommenden Hindernisse. Dank ihres unerschöpflichen Fundus an thematischen und wissenschaftlichen Hinweisen lenkten mich unsere zahlreichen Gespräche stets in neue Sphären. Ohne ihre geduldige, inspirierende und konstruktive Unterstützung wäre diese Arbeit nicht gelungen. Ich bedanke mich herzlich bei der besten Doktormutter, die man sich wünschen kann.

Der gleiche Dank gilt meinem Betreuer, Herrn Prof. Dr. Christoph Rösener, für den kritischen Diskurs und die zielführenden Diskussionen. Dank seiner langjährigen Forschungs- und Praxiserfahrung im Bereich der Kontrollierten Sprache und CLCs eröffneten mir unsere Diskussionen andere Blickwinkel. In diesem Zusammenhang auch herzlichen Dank an ihn sowie an das Institut der Gesellschaft zur Förderung der Angewandten Informationsforschung (IAI) der Universität des Saarlandes für die Forschungskooperation und die Zurverfügungstellung der Lizenz von CLAT.

Der gleiche Dank gilt meinem Betreuer, Herrn Prof. Dr. Bernd Meyer, für den intensiven und lehrreichen Austausch sowie das konstruktive Feedback, die mir halfen, Fragen aus der philosophischen Perspektive deutlicher zu sehen und zu klären. Ebenfalls bedanke ich mich vielmals bei ihm für die Gewährung großer wissenschaftlicher Freiräume für ein selbstständiges Arbeiten. Sowohl seine wissenschaftliche Betreuung als auch ständige Hilfsbereitschaft waren mir eine vielseitige Unterstützung. Herzlichen Dank.

In Zusammenhang mit der empirischen Arbeit unterstützen mich viele Menschen, ohne deren Hilfe eine solche Studie nicht durchführbar gewesen wäre:

vielen Dank an Frau Pruski und Herrn Behshad für die Bereitstellung unentbehrlicher Testdokumente und -quellen.

Ein besonders großes Dankeschön gilt Dr. Lorenz Kropfitsch, Bernd Schäfer, Sarah Signer, Patricia Graham und Kathleen Wallace für ihre herzliche und kompetente Unterstützung bei der Datenaufbereitung und den Testläufen.

Danken möchte ich außerdem den Probanden der empirischen Studie für ihre motivierte Teilnahme.

Für seine rege Aufmerksamkeit und Unterstützung bei der automatischen Evaluation bedanke ich mich herzlich bei Aaron L. F. Han von der Dublin City University.

Des Weiteren möchte ich mich bei Daniela Keller für ihre kompetente Unterstützung bei der statistischen Analyse, ihre endlose Geduld und die großartige positive Energie, die sie ausstrahlt, vielmals bedanken.

Großer Dank gebührt Ammar Suleiman für seine großartige Unterstützung bei den statistischen Tests, sein aufmerksames Zuhören und seine unglaublich freundliche und konstruktive Hilfe gerade in den stressigen Momenten.

Bei Agnieszka Surdyka möchte ich mich ferner für die mühevolle Arbeit des Korrekturlesens und vor allem für die reibungslose Kommunikation und die fröhliche Stimmung herzlich bedanken.


\begin{figure}[t]
\includegraphics[width=\textwidth]{figures/d3-img001.png}

\caption*{Salma und Adam Marzouk-Wegener (2018): Die tollste Unterstützung beim Scannen, Bibliothek des FTSK, Germersheim.}
\end{figure}

Darüber hinaus gilt mein Dank allen Verwandten, Studienkollegen und Freunden, die mich auch in schwierigen Zeiten unterstützt und immer wieder aufgeheitert haben. Dies war stets ein großer Rückhalt für mich, der als wichtiger Teil zum Erfolg meines Studiums beigetragen hat.

Tief verbunden und dankbar bin ich meiner Familie in Ägypten für ihre unglaubliche Geduld und ihr Verständnis bei der Anfertigung dieser Doktorarbeit.

Nicht minder aufreibend waren die vergangenen Jahre für meine Familie in Deutschland, die dieses Werk in allen Phasen mit jeder möglichen Unterstützung bedacht haben. Ohne Eure liebevolle Fürsorge wäre diese Arbeit nicht zu dem Werk geworden, welches sie heute ist. Für ihren unermüdlichen Beistand und ihre unendliche Geduld gilt mein voller Dank Thomas, Helga und Frank-Michael Wegener.

Ganz weit entfernt, doch mit einer ganz besonderen Bindung, sind zwei Personen, denen hier mein herauszustellender Dank gebührt; und zwar die zwei Personen, die mir den Durchhaltewillen und den Ehrgeiz für das Leben mitgaben: meine Großmutter und meine Mutter.

Zwar an letzter Stelle in dieser Aufzählung, dafür aber in meinem Herzen ganz weit vorn kommen meine tollen Kinder. Ich weiß, ihr habt in den letzten Jahren wegen mir auf viele Dinge verzichten müssen, und es tat mir oft sehr weh für euch. Liebe Salma, lieber Adam, ein riesiges Dankeschön aus~tiefstem~Herzen für euer Lächeln, das mich begleitet.

Dieses Buch widme ich Salma, Adam und Thomas.
