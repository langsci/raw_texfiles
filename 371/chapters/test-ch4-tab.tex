\documentclass[output=paper]{langscibook}

\IfFileExists{../localcommands.tex}{
  \addbibresource{../localbibliography.bib}
  % add all extra packages you need to load to this file

\usepackage{tabularx,multicol}
\usepackage{url}
\urlstyle{same}

\usepackage{listings}
\lstset{basicstyle=\ttfamily,tabsize=2,breaklines=true}

\usepackage{langsci-basic}
\usepackage{langsci-optional}
\usepackage{langsci-lgr}
\usepackage{langsci-osl}
% \usepackage{./langsci/styles/langsci-lgr}
% \usepackage{./langsci/styles/langsci-osl}
% \usepackage{langsci-gb4e}

\usepackage{tikz}
\usetikzlibrary{patterns,calc}
\pgfdeclarepatternformonly{south east lines}{\pgfqpoint{-0pt}{-0pt}}{\pgfqpoint{3pt}{3pt}}{\pgfqpoint{3pt}{3pt}}{
    \pgfsetlinewidth{0.6pt}
    \pgfpathmoveto{\pgfqpoint{0pt}{3pt}}
    \pgfpathlineto{\pgfqpoint{3pt}{0pt}}
    \pgfpathmoveto{\pgfqpoint{.2pt}{-.2pt}}
    \pgfpathlineto{\pgfqpoint{-.2pt}{.2pt}}
    \pgfpathmoveto{\pgfqpoint{3.2pt}{2.8pt}}
    \pgfpathlineto{\pgfqpoint{2.8pt}{3.2pt}}
    \pgfusepath{stroke}}
    
\usepackage{stmaryrd}
\usepackage{wasysym}
\usepackage{multirow}
\usepackage{caption}
\usepackage{subcaption}
\usepackage{mathrsfs}
\usepackage{qtree}

\usepackage{linguex}


  %pminos do not split footnotes
% \interfootnotelinepenalty=10000 %Footnote in Laporte chapters has to be split SN


%\DeclareIndexNameFormat{default}{%
%\nameparts{#1}%
%\usebibmacro{index:name}%
%{\index[names]}%
%{\namepartfamily}%
%{\namepartgiveni}%
% {}% L1
% {}% L2
%{\namepartprefix}% generates spurious space L3
%{\namepartsuffix}% generates spurious space L4
%}

%  {\DeclareIndexNameFormat{default}{%
%     \usebibmacro{index:name}{\index[names]}{#1}{#3}{#5}{#7}}}

%\DeclareIndexNameFormat{default}{%
%  \usebibmacro{index:name}{\sindex[nom]}{#1}{#3}{#5}{#7}}

%\DeclareIndexNameFormat{default}{%
%  \usebibmacro{index:name}{\sindex[person]}{#1}{#3}{#5}{#7}}
%\DeclareIndexNameFormat{default}{%
%\nameparts{#1} \usebibmacro{index:name}{\sindex[person]]}{\namepartfamily}{‌​\namepartgiven}{\nam‌​epartprefix}{\namepa‌​rtsuffix}}

%\newcommand{\smiley}{:)}

%\renewbibmacro*{index:name}[5]{%
%\usebibmacro{index:entry}{#1}%
%{\iffieldundef{usera}{}{\thefield{usera}\actualoperator}\mkbibindexname{#2}{#3}{#4}{#5}}}

% \newcommand{\noop}[1]{}

%remove for final
%\overfullrule=1mm

\newcommand{\tobi}[2]}}
\renewcommand{\S}[1]{\tobi{#1}{\textsc{*}}}

% this volume references
% puts: [this volume]
% already defined: \citetv
%\newcommand{\citepv}[1]{(\citeauthor{#1} \citeyear*{#1} [this volume])}
\newcommand{\citealtv}[1]{\citeauthor{#1} \citeyear*{#1} [this volume]}

%parentheses around example number
\newcommand{\pref}[1]{(\ref{#1})}

% in-text examples

\newcommand{\lnex}[1]{\textit{#1}} %target lang word
\newcommand{\lnlit}[1]{(lit.: `#1')} %literal reading
\newcommand{\lnlat}[1]{(#1)} % latinization
\newcommand{\lntrans}[1]{`#1'} %translation
\newcommand{\lnexl}[2]%
{\lnex{#1}{} \lnlat{#2}} % ex with latinization
\newcommand{\lnexlat}[3]{\lnex{#1}{} \lnlat{#2}{} \lntrans{#3}} % ex with latinization and tranl.

%ch01
\newcommand{\co}[1]{\mbox{\textbf{#1}}}

%ch09

\newcommand{\cyrbulg}[1]{\begin{otherlanguage*}{bulgarian}#1\end{otherlanguage*}}


%ch10
\newcommand{\nlp}{{\small NLP}}
\newcommand{\mwe}{{\small MWE}}
\newcommand{\rae}{{\small RAE}}
\newcommand{\lvc}{{\small LVC}}
\newcommand{\pos}{{\small P}o{\small S}}
%\newcommand{\todo}[1]{ \textcolor{red}{#1} }

%\renewcommand{\labelenumi}{\theenumi}
%\ainamefmt{{vv}{ll}{, ff}{, jj}} % fullname

\newcommand{\biberror}[1]{{\color{red}#1}}

\newcommand{\osenovaitem}{--~}
  %% hyphenation points for line breaks
%% Normally, automatic hyphenation in LaTeX is very good
%% If a word is mis-hyphenated, add it to this file
%%
%% add information to TeX file before \begin{document} with:
%% %% hyphenation points for line breaks
%% Normally, automatic hyphenation in LaTeX is very good
%% If a word is mis-hyphenated, add it to this file
%%
%% add information to TeX file before \begin{document} with:
%% %% hyphenation points for line breaks
%% Normally, automatic hyphenation in LaTeX is very good
%% If a word is mis-hyphenated, add it to this file
%%
%% add information to TeX file before \begin{document} with:
%% \include{localhyphenation}
\hyphenation{
    Beck-man
    Ngu-yen
    back-chan-nel
    back-chan-nels
    mo-not-o-nous
    ste-reo-typ-i-cal
}

\hyphenation{
    Beck-man
    Ngu-yen
    back-chan-nel
    back-chan-nels
    mo-not-o-nous
    ste-reo-typ-i-cal
}

\hyphenation{
    Beck-man
    Ngu-yen
    back-chan-nel
    back-chan-nels
    mo-not-o-nous
    ste-reo-typ-i-cal
}

  \togglepaper[1]%%chapternumber
}{}

\begin{document}
\noindent
[A] In der aktuellen Auflage der Leitlinien von tekom findet man zwei für die Studie relevante Gruppierungen von Regeln:

Erste Gruppierung: Regeln für übersetzungsgerechtes Schreiben

Diese Gruppierung beinhaltet die Regeln, die u.~a. für eine korrekte Verarbeitung durch Übersetzungswerkzeuge besonders relevant sind (\citealt{tekom2013}: 136f.). In der folgenden Tabelle sind alle Regeln dieser Gruppierung enthalten (linke Spalte). In den anderen Spalten wird – zusammen mit der Begründung – erwähnt, welche Regeln dieser Gruppe analysiert bzw. ausgeschlossen wurden:

Übersetzungsgerechtes Schreiben: S 102 eindeutige pronominale Bezüge verwenden
  Wurde die Regel analysiert? Ja. S 102 eindeutige pronominale Bezüge verwenden  S.

Übersetzungsgerechtes Schreibe: S 306 Aufzählungen als Listen darstellen, S 307 Satz nicht durch eine Liste unterbrechen

 Wurde die Regel analysiert? Nein, warum nicht? Die Regel bezieht sich in manchen Fällen zwar auf einen Satz (d. h. erfüllt das 1. Auswahlkriterium), jedoch geht der Satz über mehrere Zeilen. Dies würde die Humanevaluation erschweren, da die Probanden leicht von Stellen außerhalb der KS- Stelle abgelenkt werden können.
    
Satzregeln: Diese Regeln stellen die Hauptzielgruppe der Analyse dar. (ebd.: 59 ff.)

Wurde die Regel analysiert? Ja.

1. Regeln zur Vermeidung von mehrdeutigen Konstruktionen

S 102 eindeutige pronominale Bezüge verwenden

2. Regeln zur Vermeidung von unvollständigen Konstruktionen

S 201 Bedingungen als ,,Wenn'' – Sätze formulieren

S 204 keine Wortteile weglassen

Wurde die Regel analysiert? Nein, warum nicht? Die weiteren 32 Satzregeln erfüllen ein oder mehrere Auswahlkriterien nicht.

\clearpage
% tab 3, p.129
\begin{itemize}
\item[]KS-Regel Z 103b:  Für zitierte Oberflächentexte gerade Anführungszeichen ''...'' verwenden\\
Nach dieser Regel sollen Oberflächentexte, z. B. Texte in Softwareoberflächen oder Displaytexte in Geräten, in geraden Anführungszeichen stehen (tekom 2013: 117).
\begin{itemize}
    \item[]Begründung der Anwendung laut tekom: Die Anführungszeichen erhöhen die Lesbarkeit. Im Vergleich zu der Verwendung von verschiedenen Schriftarten oder Schriftgraden sind gerade Anführungszeichen optisch nicht störend. Zudem unterstützen die Anführungszeichen eine korrekte Übersetzung. (ebd.: 118)
    \item[]Begründung der Anwendung bzw. die gezielte Wirkung der Regel laut vorherigen Studien: Eine Wirkung auf die MÜ ist nachvollziehbar, denn laut Reuther (2003: 2): „Punctuation marks are very sensitive with respect to all applications where linguistic processing is done automatically.”
    \item[] Umsetzungsmuster:
        \begin{itemize}
            \item[] Vor-KS: Oberflächentext ohne Anführungszeichen
            \item[] Nach-KS: Oberflächentext angegeben in geraden Anführungszeichen
        \end{itemize}
     \item[] KS-Stelle:
        \begin{itemize}
            \item[] Vor-KS: Oberflächentext ohne Anführungszeichen
            \item[] Nach-KS: Oberflächentext mit geraden Anführungszeichen
        \end{itemize}
    \item[] Beispiele:
        \begin{itemize}
            \item[] Wählen Sie danach die Option Software automatisch installierenen
            \item[] Wählen Sie danach die Option "Software automatisch installieren"
        \end{itemize}

\end{itemize}

\end{itemize}

\end{document}