\documentclass[output=paper]{langscibook}

\IfFileExists{../localcommands.tex}{
  \addbibresource{../localbibliography.bib}
  % add all extra packages you need to load to this file

\usepackage{tabularx,multicol}
\usepackage{url}
\urlstyle{same}

\usepackage{listings}
\lstset{basicstyle=\ttfamily,tabsize=2,breaklines=true}

\usepackage{langsci-basic}
\usepackage{langsci-optional}
\usepackage{langsci-lgr}
\usepackage{langsci-osl}
% \usepackage{./langsci/styles/langsci-lgr}
% \usepackage{./langsci/styles/langsci-osl}
% \usepackage{langsci-gb4e}

\usepackage{tikz}
\usetikzlibrary{patterns,calc}
\pgfdeclarepatternformonly{south east lines}{\pgfqpoint{-0pt}{-0pt}}{\pgfqpoint{3pt}{3pt}}{\pgfqpoint{3pt}{3pt}}{
    \pgfsetlinewidth{0.6pt}
    \pgfpathmoveto{\pgfqpoint{0pt}{3pt}}
    \pgfpathlineto{\pgfqpoint{3pt}{0pt}}
    \pgfpathmoveto{\pgfqpoint{.2pt}{-.2pt}}
    \pgfpathlineto{\pgfqpoint{-.2pt}{.2pt}}
    \pgfpathmoveto{\pgfqpoint{3.2pt}{2.8pt}}
    \pgfpathlineto{\pgfqpoint{2.8pt}{3.2pt}}
    \pgfusepath{stroke}}
    
\usepackage{stmaryrd}
\usepackage{wasysym}
\usepackage{multirow}
\usepackage{caption}
\usepackage{subcaption}
\usepackage{mathrsfs}
\usepackage{qtree}

\usepackage{linguex}


  %pminos do not split footnotes
% \interfootnotelinepenalty=10000 %Footnote in Laporte chapters has to be split SN


%\DeclareIndexNameFormat{default}{%
%\nameparts{#1}%
%\usebibmacro{index:name}%
%{\index[names]}%
%{\namepartfamily}%
%{\namepartgiveni}%
% {}% L1
% {}% L2
%{\namepartprefix}% generates spurious space L3
%{\namepartsuffix}% generates spurious space L4
%}

%  {\DeclareIndexNameFormat{default}{%
%     \usebibmacro{index:name}{\index[names]}{#1}{#3}{#5}{#7}}}

%\DeclareIndexNameFormat{default}{%
%  \usebibmacro{index:name}{\sindex[nom]}{#1}{#3}{#5}{#7}}

%\DeclareIndexNameFormat{default}{%
%  \usebibmacro{index:name}{\sindex[person]}{#1}{#3}{#5}{#7}}
%\DeclareIndexNameFormat{default}{%
%\nameparts{#1} \usebibmacro{index:name}{\sindex[person]]}{\namepartfamily}{‌​\namepartgiven}{\nam‌​epartprefix}{\namepa‌​rtsuffix}}

%\newcommand{\smiley}{:)}

%\renewbibmacro*{index:name}[5]{%
%\usebibmacro{index:entry}{#1}%
%{\iffieldundef{usera}{}{\thefield{usera}\actualoperator}\mkbibindexname{#2}{#3}{#4}{#5}}}

% \newcommand{\noop}[1]{}

%remove for final
%\overfullrule=1mm

\newcommand{\tobi}[2]}}
\renewcommand{\S}[1]{\tobi{#1}{\textsc{*}}}

% this volume references
% puts: [this volume]
% already defined: \citetv
%\newcommand{\citepv}[1]{(\citeauthor{#1} \citeyear*{#1} [this volume])}
\newcommand{\citealtv}[1]{\citeauthor{#1} \citeyear*{#1} [this volume]}

%parentheses around example number
\newcommand{\pref}[1]{(\ref{#1})}

% in-text examples

\newcommand{\lnex}[1]{\textit{#1}} %target lang word
\newcommand{\lnlit}[1]{(lit.: `#1')} %literal reading
\newcommand{\lnlat}[1]{(#1)} % latinization
\newcommand{\lntrans}[1]{`#1'} %translation
\newcommand{\lnexl}[2]%
{\lnex{#1}{} \lnlat{#2}} % ex with latinization
\newcommand{\lnexlat}[3]{\lnex{#1}{} \lnlat{#2}{} \lntrans{#3}} % ex with latinization and tranl.

%ch01
\newcommand{\co}[1]{\mbox{\textbf{#1}}}

%ch09

\newcommand{\cyrbulg}[1]{\begin{otherlanguage*}{bulgarian}#1\end{otherlanguage*}}


%ch10
\newcommand{\nlp}{{\small NLP}}
\newcommand{\mwe}{{\small MWE}}
\newcommand{\rae}{{\small RAE}}
\newcommand{\lvc}{{\small LVC}}
\newcommand{\pos}{{\small P}o{\small S}}
%\newcommand{\todo}[1]{ \textcolor{red}{#1} }

%\renewcommand{\labelenumi}{\theenumi}
%\ainamefmt{{vv}{ll}{, ff}{, jj}} % fullname

\newcommand{\biberror}[1]{{\color{red}#1}}

\newcommand{\osenovaitem}{--~}
  %% hyphenation points for line breaks
%% Normally, automatic hyphenation in LaTeX is very good
%% If a word is mis-hyphenated, add it to this file
%%
%% add information to TeX file before \begin{document} with:
%% %% hyphenation points for line breaks
%% Normally, automatic hyphenation in LaTeX is very good
%% If a word is mis-hyphenated, add it to this file
%%
%% add information to TeX file before \begin{document} with:
%% %% hyphenation points for line breaks
%% Normally, automatic hyphenation in LaTeX is very good
%% If a word is mis-hyphenated, add it to this file
%%
%% add information to TeX file before \begin{document} with:
%% \include{localhyphenation}
\hyphenation{
    Beck-man
    Ngu-yen
    back-chan-nel
    back-chan-nels
    mo-not-o-nous
    ste-reo-typ-i-cal
}

\hyphenation{
    Beck-man
    Ngu-yen
    back-chan-nel
    back-chan-nels
    mo-not-o-nous
    ste-reo-typ-i-cal
}

\hyphenation{
    Beck-man
    Ngu-yen
    back-chan-nel
    back-chan-nels
    mo-not-o-nous
    ste-reo-typ-i-cal
}

  \togglepaper[1]%%chapternumber
}{}

\usepackage{soul}
\newcommand{\bspnote}[1]{\parbox{\textwidth}{\raggedright\footnotesize\noindent{#1}}}

\begin{document}

% Beispiel 2

\begin{table}
    \begin{tabularx}{\textwidth}{p{.25\textwidth}Q}
\lsptoprule
\textbf{Vor KS} & Der Hersteller \textbf{übernimmt} keine \textbf{Haftung} für Schäden, die durch nicht bestimmungsgemäßen Gebrauch entstanden sind. \\
\midrule
SMÜ SDL & \textbf{The manufacturer {\color{blue} accepts} no {\color{blue} liability} for damage caused by improper use.}\\
HMÜ Systran & The manufacturer does not {\color{red} take over} {\color{blue} liability} for damage, which resulted from not intended use.\\
RBMÜ Lucy & The manufacturer does not {\color{red} take over} {\color{blue} liability} for damages which arose through use \ul{\textit{not purpose-appropriate}}.\\
\midrule
\textbf{Nach KS} & Der Hersteller \textbf{haftet} nicht für Schäden, die durch nicht bestimmungsgemäßen Gebrauch entstanden sind.\\
\midrule
SMÜ SDL & The manufacturer {\color{blue} is} not {\color{blue} liable} for damage caused by improper use.\\
HMÜ Systran & The manufacturer {\color{blue} is} not {\color{blue} responsible} for damage, which resulted from not intended use.\\
RBMÜ Lucy & The manufacturer {\color{blue} is} not {\color{blue} liable} for damages which arose through use \ul{\textit{not purpose-appropriate}}.\\
\lspbottomrule
    \end{tabularx}
    \caption{Beispiel 2}
    \label{tabex:2}
    \bspnote{
Die KS-Stelle ist farblich dargestellt: {\color{blue} Blau} wird für die korrekten Teile der Übersetzung verwendet; {\color{red} rot} für die falschen Teile.
Fehler \textit{außerhalb} der KS-Stelle, die korrigiert wurden, sind \ul{\textit{kursiv und unterstrichen}} dargestellt.
}

\end{table}



% Beispiel 3

\begin{table}
    \begin{tabularx}{\textwidth}{p{.25\textwidth}Q}
\lsptoprule
\textbf{Vor KS} & Der Hersteller \textbf{übernimmt} keine \textbf{Haftung} für Schäden, die durch nicht bestimmungsgemäßen Gebrauch entstanden sind. \\
\midrule
SMÜ SDL & The manufacturer {\color{blue}accepts} no {\color{blue}liability} for \colorbox{lightgray}{any damage caused by improper use.}\\
HMÜ Systran & The manufacturer does not {\color{red} take over} {\color{blue} liability} for \colorbox{lightgray}{any damage caused by improper use.}\\
RBMÜ Lucy & The manufacturer does not {\color{red} take over} {\color{blue} liability} for \colorbox{lightgray}{any damage caused by improper use.}\\
\midrule
\textbf{Nach KS} & Der Hersteller \textbf{haftet} nicht für Schäden, die durch nicht bestimmungsgemäßen Gebrauch entstanden sind.\\
\midrule
SMÜ SDL & The manufacturer {\color{blue} is} not {\color{blue} liable} for \colorbox{lightgray}{any damage caused by improper use.}\\
HMÜ Systran & The manufacturer {\color{blue} is} not {\color{blue} responsible} for \colorbox{lightgray}{any damage caused by improper use.}\\
RBMÜ Lucy & The manufacturer {\color{blue} is} not {\color{blue} liable} for \colorbox{lightgray}{any damage caused by improper use}\\
\lspbottomrule
    \end{tabularx}
    \caption{Beispiel 3}
    \label{tabex:3}
    \bspnote{
\colorbox{lightgray}{Hervorgehobene Stellen} zeigen die vereinheitlichten Stellen in der MÜ auf.
}

\end{table}



% Beispiel 4

\begin{table}
    \begin{tabularx}{\textwidth}{p{.25\textwidth}Q}
\lsptoprule
\textbf{Vor KS} & Bei der Arbeit mit elektrischen Geräten \textbf{sollte} stets ein Sicherheitsstecker \textbf{verwendet werden.}\\
\midrule
SMÜ SDL

Fehlerannotation & When working with electrical \ul{\textit{equipment}} {\color{red}{should}} always be \ul{\textit{ \textbf{a safety plug}}} {\color{red}{is used}}.\\
\tablevspace
SMÜ SDL

Humanevaluation & When working with electrical \ul{\textit{devices}}, {\color{red}{should}} always be \ul{\textit{ \textbf{a safety plug}}} {\color{red}{is used}}.\\
\lspbottomrule
    \end{tabularx}
    \caption{Beispiel 4}
    \label{tabex:4}
    \bspnote{
Die KS-Stelle ist farblich dargestellt: {\color{blue} Blau} wird für die korrekten Teile der Übersetzung verwendet; {\color{red} rot} für die falschen Teile.
Fehler \textit{außerhalb} der KS-Stelle, die korrigiert wurden, sind \ul{\textit{kursiv und unterstrichen}} dargestellt; bzw.\ul{\textit{ \textbf{fett, kursiv und unterstrichen}}} bei Wortstellungsfehlern \textit{außerhalb} der KS-Stelle, die nicht korrigiert wurden.
}

\end{table}



% Beispiel 5
\begin{table}
    \begin{tabularx}{\textwidth}{p{.25\textwidth}Q}
\lsptoprule
\textbf{nach KS} & Je früher ein Fleck behandelt wird, umso größer ist die Wahrscheinlichkeit, \textbf{den Fleck} rückstandslos zu entfernen.\\
\midrule
HMÜ Systran

Fehlerannotation &  \ul{\textit{Ever in former times}} \colorbox{lightgray}{a mark\strut} is treated, \ul{\textit{all the more largely}} is the probability to remove {\color{blue}the mark} residueless.\\
\tablevspace
HMÜ Systran

Humanevaluation & \ul{\textit{The earlier}} \colorbox{lightgray}{a mark\strut\strut} is treated, \ul{\textit{the higher the possibility}} of removing {\color{blue}the mark} without leaving any residue.\\
& \\
RBMÜ Lucy

Fehlerannotation & \ul{\textit{Each formerly}} \colorbox{lightgray}{a spot\strut} is treated, \ul{\textit{the larger the probability}} is to remove {\color{blue}the spot} \ul{\textit{free of residues}}.\\
\tablevspace
RBMÜ Lucy

Humanevaluation & \ul{\textit{The earlier}} \colorbox{lightgray}{a spot\strut} is treated, \ul{\textit{the higher the possibility}} of removing {\color{blue}the spot} \ul{\textit{without leaving any residue}}.\\
\lspbottomrule
    \end{tabularx}
    \caption{Beispiel 5}
    \label{tabex:5}
    \parbox{\textwidth}{\raggedright\footnotesize\noindent
Die KS-Stelle ist farblich dargestellt: {\color{blue} Blau} wird für die korrekten Teile der Übersetzung verwendet; {\color{red} rot} für die falschen Teile.
Fehler \textit{außerhalb} der KS-Stelle, die korrigiert wurden, sind \ul{\textit{kursiv und unterstrichen}} dargestellt; bzw.\ul{\textit{ \textbf{fett, kursiv und unterstrichen}}} bei Wortstellungsfehlern \textit{außerhalb} der KS-Stelle, die nicht korrigiert wurden.
}

\end{table}


\end{document}