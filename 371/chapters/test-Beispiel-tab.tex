\documentclass[output=paper]{langscibook}

\IfFileExists{../localcommands.tex}{
  \addbibresource{../localbibliography.bib}
  \usepackage{langsci-optional}
\usepackage{langsci-gb4e}
\usepackage{langsci-lgr}

\usepackage{listings}
\lstset{basicstyle=\ttfamily,tabsize=2,breaklines=true}

%added by author
% \usepackage{tipa}
\usepackage{multirow}
\graphicspath{{figures/}}
\usepackage{langsci-branding}

  
\newcommand{\sent}{\enumsentence}
\newcommand{\sents}{\eenumsentence}
\let\citeasnoun\citet

\renewcommand{\lsCoverTitleFont}[1]{\sffamily\addfontfeatures{Scale=MatchUppercase}\fontsize{44pt}{16mm}\selectfont #1}
  
  %% hyphenation points for line breaks
%% Normally, automatic hyphenation in LaTeX is very good
%% If a word is mis-hyphenated, add it to this file
%%
%% add information to TeX file before \begin{document} with:
%% %% hyphenation points for line breaks
%% Normally, automatic hyphenation in LaTeX is very good
%% If a word is mis-hyphenated, add it to this file
%%
%% add information to TeX file before \begin{document} with:
%% %% hyphenation points for line breaks
%% Normally, automatic hyphenation in LaTeX is very good
%% If a word is mis-hyphenated, add it to this file
%%
%% add information to TeX file before \begin{document} with:
%% \include{localhyphenation}
\hyphenation{
affri-ca-te
affri-ca-tes
an-no-tated
com-ple-ments
com-po-si-tio-na-li-ty
non-com-po-si-tio-na-li-ty
Gon-zá-lez
out-side
Ri-chárd
se-man-tics
STREU-SLE
Tie-de-mann
}
\hyphenation{
affri-ca-te
affri-ca-tes
an-no-tated
com-ple-ments
com-po-si-tio-na-li-ty
non-com-po-si-tio-na-li-ty
Gon-zá-lez
out-side
Ri-chárd
se-man-tics
STREU-SLE
Tie-de-mann
}
\hyphenation{
affri-ca-te
affri-ca-tes
an-no-tated
com-ple-ments
com-po-si-tio-na-li-ty
non-com-po-si-tio-na-li-ty
Gon-zá-lez
out-side
Ri-chárd
se-man-tics
STREU-SLE
Tie-de-mann
}
  \togglepaper[1]%%chapternumber
}{}

\usepackage{soul}
\newcommand{\bspnote}[1]{\parbox{\textwidth}{\raggedright\footnotesize\noindent{#1}}}

\begin{document}

% Beispiel 2

\begin{table}
    \begin{tabularx}{\textwidth}{p{.25\textwidth}Q}
\lsptoprule
\textbf{Vor KS} & Der Hersteller \textbf{übernimmt} keine \textbf{Haftung} für Schäden, die durch nicht bestimmungsgemäßen Gebrauch entstanden sind. \\
\midrule
SMÜ SDL & \textbf{The manufacturer {\color{blue} accepts} no {\color{blue} liability} for damage caused by improper use.}\\
HMÜ Systran & The manufacturer does not {\color{red} take over} {\color{blue} liability} for damage, which resulted from not intended use.\\
RBMÜ Lucy & The manufacturer does not {\color{red} take over} {\color{blue} liability} for damages which arose through use \ul{\textit{not purpose-appropriate}}.\\
\midrule
\textbf{Nach KS} & Der Hersteller \textbf{haftet} nicht für Schäden, die durch nicht bestimmungsgemäßen Gebrauch entstanden sind.\\
\midrule
SMÜ SDL & The manufacturer {\color{blue} is} not {\color{blue} liable} for damage caused by improper use.\\
HMÜ Systran & The manufacturer {\color{blue} is} not {\color{blue} responsible} for damage, which resulted from not intended use.\\
RBMÜ Lucy & The manufacturer {\color{blue} is} not {\color{blue} liable} for damages which arose through use \ul{\textit{not purpose-appropriate}}.\\
\lspbottomrule
    \end{tabularx}
    \caption{Beispiel 2}
    \label{tabex:2}
    \bspnote{
Die KS-Stelle ist farblich dargestellt: {\color{blue} Blau} wird für die korrekten Teile der Übersetzung verwendet; {\color{red} rot} für die falschen Teile.
Fehler \textit{außerhalb} der KS-Stelle, die korrigiert wurden, sind \ul{\textit{kursiv und unterstrichen}} dargestellt.
}

\end{table}



% Beispiel 3

\begin{table}
    \begin{tabularx}{\textwidth}{p{.25\textwidth}Q}
\lsptoprule
\textbf{Vor KS} & Der Hersteller \textbf{übernimmt} keine \textbf{Haftung} für Schäden, die durch nicht bestimmungsgemäßen Gebrauch entstanden sind. \\
\midrule
SMÜ SDL & The manufacturer {\color{blue}accepts} no {\color{blue}liability} for \colorbox{lightgray}{any damage caused by improper use.}\\
HMÜ Systran & The manufacturer does not {\color{red} take over} {\color{blue} liability} for \colorbox{lightgray}{any damage caused by improper use.}\\
RBMÜ Lucy & The manufacturer does not {\color{red} take over} {\color{blue} liability} for \colorbox{lightgray}{any damage caused by improper use.}\\
\midrule
\textbf{Nach KS} & Der Hersteller \textbf{haftet} nicht für Schäden, die durch nicht bestimmungsgemäßen Gebrauch entstanden sind.\\
\midrule
SMÜ SDL & The manufacturer {\color{blue} is} not {\color{blue} liable} for \colorbox{lightgray}{any damage caused by improper use.}\\
HMÜ Systran & The manufacturer {\color{blue} is} not {\color{blue} responsible} for \colorbox{lightgray}{any damage caused by improper use.}\\
RBMÜ Lucy & The manufacturer {\color{blue} is} not {\color{blue} liable} for \colorbox{lightgray}{any damage caused by improper use}\\
\lspbottomrule
    \end{tabularx}
    \caption{Beispiel 3}
    \label{tabex:3}
    \bspnote{
\colorbox{lightgray}{Hervorgehobene Stellen} zeigen die vereinheitlichten Stellen in der MÜ auf.
}

\end{table}



% Beispiel 4

\begin{table}
    \begin{tabularx}{\textwidth}{p{.25\textwidth}Q}
\lsptoprule
\textbf{Vor KS} & Bei der Arbeit mit elektrischen Geräten \textbf{sollte} stets ein Sicherheitsstecker \textbf{verwendet werden.}\\
\midrule
SMÜ SDL

Fehlerannotation & When working with electrical \ul{\textit{equipment}} {\color{red}{should}} always be \ul{\textit{ \textbf{a safety plug}}} {\color{red}{is used}}.\\
\tablevspace
SMÜ SDL

Humanevaluation & When working with electrical \ul{\textit{devices}}, {\color{red}{should}} always be \ul{\textit{ \textbf{a safety plug}}} {\color{red}{is used}}.\\
\lspbottomrule
    \end{tabularx}
    \caption{Beispiel 4}
    \label{tabex:4}
    \bspnote{
Die KS-Stelle ist farblich dargestellt: {\color{blue} Blau} wird für die korrekten Teile der Übersetzung verwendet; {\color{red} rot} für die falschen Teile.
Fehler \textit{außerhalb} der KS-Stelle, die korrigiert wurden, sind \ul{\textit{kursiv und unterstrichen}} dargestellt; bzw.\ul{\textit{ \textbf{fett, kursiv und unterstrichen}}} bei Wortstellungsfehlern \textit{außerhalb} der KS-Stelle, die nicht korrigiert wurden.
}

\end{table}



% Beispiel 5
\begin{table}
    \begin{tabularx}{\textwidth}{p{.25\textwidth}Q}
\lsptoprule
\textbf{nach KS} & Je früher ein Fleck behandelt wird, umso größer ist die Wahrscheinlichkeit, \textbf{den Fleck} rückstandslos zu entfernen.\\
\midrule
HMÜ Systran

Fehlerannotation &  \ul{\textit{Ever in former times}} \colorbox{lightgray}{a mark\strut} is treated, \ul{\textit{all the more largely}} is the probability to remove {\color{blue}the mark} residueless.\\
\tablevspace
HMÜ Systran

Humanevaluation & \ul{\textit{The earlier}} \colorbox{lightgray}{a mark\strut\strut} is treated, \ul{\textit{the higher the possibility}} of removing {\color{blue}the mark} without leaving any residue.\\
& \\
RBMÜ Lucy

Fehlerannotation & \ul{\textit{Each formerly}} \colorbox{lightgray}{a spot\strut} is treated, \ul{\textit{the larger the probability}} is to remove {\color{blue}the spot} \ul{\textit{free of residues}}.\\
\tablevspace
RBMÜ Lucy

Humanevaluation & \ul{\textit{The earlier}} \colorbox{lightgray}{a spot\strut} is treated, \ul{\textit{the higher the possibility}} of removing {\color{blue}the spot} \ul{\textit{without leaving any residue}}.\\
\lspbottomrule
    \end{tabularx}
    \caption{Beispiel 5}
    \label{tabex:5}
    \parbox{\textwidth}{\raggedright\footnotesize\noindent
Die KS-Stelle ist farblich dargestellt: {\color{blue} Blau} wird für die korrekten Teile der Übersetzung verwendet; {\color{red} rot} für die falschen Teile.
Fehler \textit{außerhalb} der KS-Stelle, die korrigiert wurden, sind \ul{\textit{kursiv und unterstrichen}} dargestellt; bzw.\ul{\textit{ \textbf{fett, kursiv und unterstrichen}}} bei Wortstellungsfehlern \textit{außerhalb} der KS-Stelle, die nicht korrigiert wurden.
}

\end{table}


\end{document}