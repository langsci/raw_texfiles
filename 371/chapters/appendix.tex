\chapter{Testanweisungen der Humanevaluation}\label{app:1}

\section*{Thanks a lot for your interest in taking part in this study}

Before you begin, the following short description provides you with the necessary information on what to expect and what to keep in mind:

\begin{itemize}
\item You are going to rate translations of sentences extracted from \textbf{user manuals for different products}.
\begin{itemize}
\item Some sentences seem identical at the first glance, but they are not, so please read each sentence \textbf{c a r e f u l l y}.
\item If you found a sentence, for which you would prefer to have more information about its \textbf{context}, please enter that as a comment and \ul{rate the translation as best as you can}.
\end{itemize}

\item Your main task is to \textbf{rate} the content and style quality on a scale from 1 (very low quality) to 5 (very high quality) \textbf{based on the criteria mentioned in the following definitions}:

\textbf{Style quality}: The extent to which the translation \ul{sounds natural and idiomatic} in Standard Written English, is \ul{appropriate to the intention} of its content as well as is \ul{presented clearly orthographically}.

\textbf{Content quality}: The extent to which the translation \ul{reflects the information} in the source text \ul{accurately}; and the extent to which the translation is \ul{easy to understand}.

\item Please work \ul{as quickly as comfortably} possible. For statistical purposes, you need to enter the starting and ending time of each test on the first and last page, respectively.
\end{itemize}


\section*{HOW TO RATE}

\subsection*{Example 1}

%%[Warning: Draw object ignored]

%%please move the includegraphics inside the {figure} environment
%\includegraphics[width=\textwidth]{figures/d3-img126.png}
\noindent
\includegraphics[width=\textwidth]{figures/appex1-1.png}

\subsection*{Example 2}

%%[Warning: Draw object ignored]
%%[Warning: Draw object ignored]
%%[Warning: Draw object ignored]

%%please move the includegraphics inside the {figure} environment
%\includegraphics[width=\textwidth]{figures/d3-img127.png}
\includegraphics[width=\textwidth]{figures/appex2-1.png}

\subsection*{Note}

\begin{minipage}[t]{.5\textwidth}
Technically, the checkboxes can be all ticked at the same time, allowing you to select all relevant criteria.
\end{minipage}
\hfill
\begin{minipage}[t]{.45\textwidth}
\strut\vspace*{-\baselineskip}\newline\includegraphics[width=.5\textwidth]{figures/appex3-1.png}
 \end{minipage}

\medskip
\noindent
Each sentence in the test is provided separately in a table / tab in Excel below. Please scroll to the right to see all 25 tabs.

% \includegraphics[width=\textwidth]{figures/d3-img129.emf}
\noindent
\includegraphics[width=\textwidth]{figures/appex4-1.png}


\section*{How to proceed?}


\bful{Please proceed as follows:}

\begin{enumerate}
\item Tick the quality criteria that are \bful{not} satisfied, if any.
\item For the selected quality criteria, you need to enter a reason. \ul{This reason should include the following information}:
“wrong/problematic part” \textbf{+} “reason” \textbf{+} “your correction/suggestion for this part”  \textit{(just a short comment)}

\item Please score the “Style Quality” and “Content Quality” \bful{after} selecting the relevant quality criteria and commenting them.
\item When is it necessary to enter an alternative translation for the whole sentence at the bottom?

  \begin{itemize}
  \item If you just recommend replacing a certain part in the translation and agree with the rest of the translation, you do not need to enter an alternative translation.
  \item If your suggested modification (e.g. in the main clause) requires a further modification (e.g. in the subordinate clause), an alternative translation for the whole sentence is needed; otherwise, the translation might not sound idiomatic.
  \end{itemize}

\item It is possible that a certain part of the translation causes \textit{a number of different problems (e.g. 2 style problems and 1 content problem)}, reducing \textit{both style} and \textit{content quality}. In other words, you may suggest a modification (e.g. an orthographic issue), which makes the translation \textit{presented more clearly} and at the same time \textit{eases understanding its content}.
\item You may find a translation \textit{correct} with regard to the \textit{content}. However, \textit{stylistically}, it still needs to be optimized to make the translation sound \textit{natural} for an English native speaker and/or \textit{suitable} for use in an English user manual.
\item If the error is a \textit{missing word}, please tick the relevant quality criteria and enter as a reason e.g. the word “XYZ” is missing after the word “ABC”.
\end{enumerate}




\chapter{Pre- und Posttests der Humanevaluation}\label{app:2}

\section*{Pretest}

\begin{enumerate}
\item Geschlecht:
\item Herkunftsland (In welchem Land bist Du aufgewachsen?)
\item Deutschkenntnisse:

Wie hast Du die deutsche Sprache gelernt? \textit{Mehrere Antworten sind möglich:}


\begin{enumerate}
\item Ich bin zweisprachig aufgewachsen.
\item Durch Sprachkurse (wie lange war die Dauer der Kurse?)
\item In der Schule
\item Durch meinen Aufenthalt in Deutschland (wie lange war Dein Aufenthalt?)
\item Sonstiges: ………………
\end{enumerate}

\item Berufliche Übersetzungserfahrung: Hast Du berufliche Übersetzungserfahrung gesammelt? Wenn ja, wie lange und überwiegend in welchen Bereichen?
\end{enumerate}

\section*{Posttest}

\subsection*{Teil 1: Fachliche Fragen}


\begin{enumerate}
\item \textbf{Verwendung von maschinellen Übersetzungssystemen (MÜ-Systemen):}\\
\textbf{Wie gehst Du in der Regel vor, wenn Du einen technischen Text übersetzen möchtest? bzw. wie würdest Du vorgehen?}


\begin{enumerate}
\item Ich übersetze den Text zuerst mit einem MÜ-System und post-editiere die Übersetzung.
\item Ich benutze ein Übersetzungsprogramm, das mit einem MÜ-System verbunden ist. Danach post-editiere ich die Übersetzung.
\item Ich übersetze den Text und verwende ein MÜ-System, nur wenn ich es brauche (z.~B. um einzelne Wörter / Sätze zu übersetzen).
\item Ich benutze gar kein MÜ-System. Bitte kurz begründen!
\item Ich habe eine andere Strategie für die Nutzung von MÜ-Systemen. Bitte kurz beschreiben!
\end{enumerate}


\item \textbf{Hast Du Erfahrung mit der Kontrollierten Sprache (Controlled Language)?} \textit{Mehrere Antworten sind möglich:}

\begin{enumerate}
\item Gar keine Erfahrung. Ich weiß nicht, was eine Kontrollierte Sprache ist.
\item Ich habe das Thema im Rahmen meines Studiums gelernt.
\item Ich habe nach bestimmten Regeln der Kontrollierten Sprache einen Übersetzungsauftrag (oder mehrere) beruflich oder als Übersetzungsübung im Laufe des Studiums angefertigt. Falls zutreffend, waren die Regeln vom Auftraggeber / Dozenten vorgegeben oder hast Du Dich dafür entschieden?
\item Ich beachte im Allgemeinen bei meiner technischen Übersetzung die Regeln der Kontrollierten Sprache. Falls zutreffend, warum?
\item Obwohl die Regeln der Kontrollierten Sprache mir bekannt sind, bevorzuge ich danach nicht zu übersetzen. Falls zutreffend, bitte kurz begründen!
\item Sonstiges: ………
\end{enumerate}

\end{enumerate}


\subsection*{Teil 2: Feedback zur Evaluation bzw. zur Testphase}


\begin{enumerate}
\item Fandest Du die Evaluation lang / umfangreich?
\item Fandest Du die Evaluation interessant / langweilig? Falls langweilig, wann hat die Langweile eingesetzt, nach dem 10., 20., ... Test?
\item Fandest Du den Aufbau der Evaluation verständlich / zu schwer?
\item Hast Du durch die Bewertung etwas Neues gelernt?
\item War in den DE- oder EN-Sätzen irgendetwas auffällig?
\item Kannst Du im Nachhinein erraten, worum es geht?
 \end{enumerate}


\chapter{Datensatz}\label{app:3}

\section*{Quellen des analysierten Korpus}


\begin{tabularx}{\textwidth}{lXp{.3\textwidth}}

\lsptoprule

{\textbf{Ref.}} & \textbf{Bezeichnung der Quelle} & \textbf{Hersteller}\\
\midrule
 \textbf{\#1} & Verlegeanweisung und Reinigungsverfahren für $"$EASY LIFT BAHNENWARE$"$ & Halbmond Teppichwerke GmbH\\
 \tablevspace
 \textbf{\#2} & Gebrauchs- und Pflegeanleitung für keramikversiegeltes BERNDES-Kochgeschirr & Berndes~Küche GmbH\\
 \tablevspace
 \textbf{\#3} & Betriebsanleitung des Feinstzerkleinerers $"$FZK-HPC-S3$"$ & Hempe GmbH\\
 \tablevspace
 \textbf{\#4} & Handbuch der Konfigurationssoftware $"$SAUTER CASE VAV$"$ & SAUTER Controls GmbH\\
 \tablevspace
 \textbf{\#5} & Gepäckregelung – Vermissen Sie Ihr Gepäck? & Deutsche Lufthansa AG\\
 \tablevspace
 \textbf{\#6} & Bedienungsanleitung des Milchschäumers $"$CREMIO$"$ & Melitta Haushaltsprodukte GmbH \& Co. KG\\
 \tablevspace
 \textbf{\#7} & Pflege und Bedienungsanleitung des Küchenmöbels von NOLTE & Nolte Küchen GmbH \& Co. KG\\
 \tablevspace
 \textbf{\#8} & Technische Beschreibung und Bedienungsanleitung des Heimkino-Lautsprecher-Sets $"$System 10 THX Ultra 2$"$ & Teufel GmbH\\
 \tablevspace
 \textbf{\#9} & Bedienungsanleitung der elektrischen Zitruspresse $"$ZP 40$"$ & Rommelsbacher ElektroHausgeräte GmbH\\
 \tablevspace
 \textbf{\#10} & Betriebsanweisung der Haus- und Gartenpumpenserie $"$BP 3, BP 4, BP 5 und BP7$"$ & Alfred Kärcher GmbH \& Co. KG\\
\lspbottomrule
\end{tabularx}


\section*{Regel 1 -- Für zitierte Oberflächentexte gerade Anführungszeichen verwenden}


\begin{longtable}{llp{.7\textwidth}l}

\lsptoprule
%\hhline%%replace by cmidrule{~~--}
{} & \textbf{KS} & \textbf{Sätze} & \makecell[tl]{\textbf{Ref.}\\\textbf{Quelle}}\\
\midrule
{ \textbf{1}} & Vor & Das Tool bietet diese Korrekturen bei der Eingabe der Werte im Bereich \textbf{Übersicht} an. & \textbf{\#4}\\
& Nach & Das Tool bietet diese Korrekturen bei der Eingabe der Werte im Bereich \textbf{$"$Übersicht$"$}. & \\
\tablevspace
{ \textbf{2}} & Vor & Ist nur ein Gerät angeschlossen, so ist die Funktion \textbf{Punkt-zu-Punkt-Verbindung} zu wählen. & \textbf{\#4}\\
& Nach & Ist nur ein Gerät angeschlossen, so ist die Funktion \textbf{$"$Punkt-zu-Punkt-Verbindung$"$} zu wählen. & \\
\tablevspace
{ \textbf{3}} & Vor & Durch Anklicken des Buttons \textbf{Zusatzinformation} werden die Angaben über das Netzwerk eingelesen. & \textbf{\#4}\\
& Nach & Durch Anklicken des Buttons \textbf{$"$Zusatzinformation$"$} werden die Angaben über das Netzwerk eingelesen. & \\
\tablevspace
{ \textbf{4}} & Vor & Wenn eine korrekte Verbindung aufgebaut werden konnte, wird in der Statuszeile das Feld \textbf{Verbindung} grün. & \textbf{\#4}\\
& Nach & Wenn eine korrekte Verbindung aufgebaut werden konnte, wird in der Statuszeile das Feld \textbf{$"$Verbindung$"$} grün. & \\
\tablevspace
{ \textbf{5}} & Vor & Im Reiter \textbf{Kommunikation BACnet} können die notwendigen Einstellungen vorgenommen werden. & \textbf{\#4}\\
& Nach & Im Reiter \textbf{$"$Kommunikation BACnet$"$} können die notwendigen Einstellungen vorgenommen werden. & \\
\tablevspace
{ \textbf{6}} & Vor & Das Modul \textbf{ASV15} darf nur für seinen spezifizierten Einsatzzweck verwendet werden. & \textbf{\#4}\\
& Nach & Das Modul \textbf{$"$ASV15$"$} darf nur für seinen spezifizierten Einsatzzweck verwendet werden. & \\
\tablevspace
{ \textbf{7}} & Vor & Im Abschnitt \textbf{Graphikeinstellungen} können Sie Einstellungen für die angezeigten Graphiken vornehmen. & \textbf{\#4}\\
& Nach & Im Abschnitt \textbf{$"$Graphikeinstellungen$"$} können Sie Einstellungen für die angezeigten Graphiken vornehmen. & \\
\tablevspace
{ \textbf{8}} & Vor & Hierzu kann die Funktion \textbf{Upload vom Gerät} gewählt werden. & \textbf{\#4}\\
& Nach & Hierzu kann die Funktion \textbf{$"$Upload vom Gerät$"$} gewählt werden. & \\
\tablevspace
{ \textbf{9}} & Vor & Überprüfen Sie die Adresse des Ports im \textbf{Geräte-Manager}. & \textbf{\#4}\\
& Nach & Überprüfen Sie die Adresse des Ports im \textbf{$"$Geräte-Manager$"$}. & \\
\tablevspace
{ \textbf{10}} & Vor & Wählen Sie die Option \textbf{Software von einer bestimmten Liste installieren}. & \textbf{\#4}\\
& Nach & Wählen Sie die Option \textbf{$"$Software von einer bestimmten Liste installieren$"$}. & \\
\tablevspace
{ \textbf{11}} & Vor & Klicken Sie auf der Startseite auf \textbf{Gerät konfigurieren}. & \textbf{\#4}\\
& Nach & Klicken Sie auf der Startseite auf \textbf{$"$Gerät konfigurieren$"$}. & \\
\tablevspace
{ \textbf{12}} & Vor & Klicken Sie auf \textbf{Netzwerk absuchen}, um die vorhandenen Geräte im Netzwerk anzuzeigen. & \textbf{\#4}\\
& Nach & Klicken Sie auf \textbf{$"$Netzwerk absuchen$"$}, um die vorhandenen Geräte im Netzwerk anzuzeigen. & \\
\tablevspace
{ \textbf{13}} & Vor & Unter dem Reiter \textbf{Einheiten} können Sie die verwendeten Einheiten einstellen. & \textbf{\#4}\\
& Nach & Unter dem Reiter \textbf{$"$Einheiten$"$} können Sie die verwendeten Einheiten einstellen. & \\
\tablevspace
{ \textbf{14}} & Vor & Mit der Funktion \textbf{Geräteparameter ändern} können Sie die Parameter eines Gerätes ändern. & \textbf{\#4}\\
& Nach & Mit der Funktion \textbf{$"$Geräteparameter ändern$"$} können Sie die Parameter eines Gerätes ändern. & \\
\tablevspace
{ \textbf{15}} & Vor & Die Funktion \textbf{Neue Adresse} ermöglicht die manuelle Vergabe einer neuen Netzwerkadresse. & \textbf{\#4}\\
& Nach & Die Funktion \textbf{$"$Neue Adresse$"$} ermöglicht die manuelle Vergabe einer neuen Netzwerkadresse. & \\
\tablevspace
{ \textbf{16}} & Vor & Um ein neues Gerät zu konfigurieren, wählen Sie den Menüpunkt \textbf{Gerät konfigurieren} aus. & \textbf{\#4}\\
& Nach & Um ein neues Gerät zu konfigurieren, wählen Sie den Menüpunkt \textbf{$"$Gerät konfigurieren$"$} aus. & \\
\tablevspace
{ \textbf{17}} & Vor & Die Firmware-Version wird in der Infozeile in der Spalte \textbf{Firmware-Version} angezeigt. & \textbf{\#4}\\
& Nach & Die Firmware-Version wird in der Infozeile in der Spalte \textbf{$"$Firmware-Version$"$} angezeigt. & \\
\tablevspace
{ \textbf{18}} & Vor & Wenn Sie die Option \textbf{Zweites Gerät visualisieren} angewählt haben, können Sie die Parameter bestimmen. & \textbf{\#4}\\
& Nach & Wenn Sie die Option \textbf{$"$Zweites Gerät visualisieren$"$} angewählt haben, können Sie die Parameter bestimmen. & \\
\tablevspace
{ \textbf{19}} & Vor & Im Eingabefenster \textbf{Projektdaten} werden Informationen zur Lokalisierung des Reglers eingegeben. & \textbf{\#4}\\
& Nach & Im Eingabefenster \textbf{$"$Projektdaten$"$} werden Informationen zur Lokalisierung des Reglers eingegeben. & \\
\tablevspace
{ \textbf{20}} & Vor & Auf der Seite \textbf{Einstellungen} sind alle notwendigen Parameter zur Optimierung des Regelkreises zusammengefasst. & \textbf{\#4}\\
& Nach & Auf der Seite \textbf{$"$Einstellungen$"$} sind alle notwendigen Parameter zur Optimierung des Regelkreises zusammengefasst. & \\
\tablevspace
{ \textbf{21}} & Vor & Die zwei aktivierbaren Raumdruck-Sollwerte sind im Menü \textbf{Raumdruck} definiert. & \textbf{\#4}\\
& Nach & Die zwei aktivierbaren Raumdruck-Sollwerte sind im Menü \textbf{$"$Raumdruck$"$} definiert. & \\
\tablevspace
{ \textbf{22}} & Vor & Wählen Sie danach die Option \textbf{Software automatisch installieren}. & \textbf{\#4}\\
& Nach & Wählen Sie danach die Option \textbf{$"$Software automatisch installieren$"$}. & \\
\tablevspace
{ \textbf{23}} & Vor & Wählen Sie die Option \textbf{Auswahl nach Anwendungs-Code} aus. & \textbf{\#4}\\
& Nach & Wählen Sie die Option \textbf{$"$Auswahl nach Anwendungs-Code$"$} aus. & \\
\tablevspace
{ \textbf{24}} & Vor & Mit der Funktion \textbf{Anwendung neu konfigurieren}, kann dem Gerät eine neue Anwendung zugewiesen werden. & \textbf{\#4}\\
& Nach & Mit der Funktion \textbf{$"$Anwendung neu konfigurieren$"$}, kann dem Gerät eine neue Anwendung zugewiesen werden. & \\
%\hhline%%replace by cmidrule{~--~}
\lspbottomrule
\end{longtable}


\section*{Regel 2 -- Funktionsverbgefüge vermeiden}

\begin{longtable}{llp{.7\textwidth}l}

\lsptoprule
%\hhline%%replace by cmidrule{~~--}
{} & \textbf{KS} & \textbf{Sätze} & \makecell[tl]{\textbf{Ref.}\\\textbf{Quelle}}\\
\midrule
{ \textbf{1}} & Vor & Der Bediener darf erst die Maschine \textbf{in Betrieb nehmen}, wenn er die Betriebsanleitung gelesen hat. & \textbf{\#3}\\
& Nach & Der Bediener darf erst die Maschine \textbf{starten}, wenn er die Betriebsanleitung gelesen hat. & \\
\tablevspace
{ \textbf{2}} & Vor & Der Hersteller \textbf{übernimmt} keine \textbf{Haftung} für Schäden, die durch nicht bestimmungsgemäßen Gebrauch entstanden sind. & \textbf{\#8}\\
& Nach & Der Hersteller \textbf{haftet} nicht für Schäden, die durch nicht bestimmungsgemäßen Gebrauch entstanden sind. & \\
\tablevspace
{ \textbf{3}} & Vor & \textbf{Steht} die Maschine nicht \textbf{im Einsatz}, den Hauptschalter auf $"$0$"$ setzen. & \textbf{\#3}\\
& Nach & \textbf{Wird} die Maschine nicht \textbf{verwendet}, den Hauptschalter auf $"$0$"$ setzen. & \\
\tablevspace
{ \textbf{4}} & Vor & Wird diese Regel nicht beachtet, kann der Motor \textbf{Schaden nehmen}. & \textbf{\#9}\\
& Nach & Wird diese Regel nicht beachtet, kann der Motor \textbf{beschädigt werden}. & \\
\tablevspace
{ \textbf{5}} & Vor & Die Fleck\textbf{behandlung} muss so schnell wie möglich \textbf{durchgeführt} werden. & \textbf{\#1}\\
& Nach & Die Flecken müssen so schnell wie möglich \textbf{behandelt} werden. & \\
\tablevspace
{ \textbf{6}} & Vor & Der Kompaktregler \textbf{setzt sich mit} der neuen Netzwerkadresse \textbf{in Verbindung}. & \textbf{\#4}\\
& Nach & Der Kompaktregler \textbf{verbindet sich mit} der neuen Netzwerkadresse. & \\
\tablevspace
{ \textbf{7}} & Vor & Der Navigationsbaum \textbf{stellt} alle vorhandenen Seiten der Konfigurierung \textbf{zur Verfügung}. & \textbf{\#4}\\
& Nach & Der Navigationsbaum \textbf{stellt} alle vorhandenen Seiten der Konfigurierung \textbf{bereit}. & \\
\tablevspace
{ \textbf{8}} & Vor & Die vorderen Schutzgitter \textbf{bieten} den empfindlichen Lautsprechermembranen \textbf{Schutz}. & \textbf{\#8}\\
& Nach & Die vorderen Schutzgitter \textbf{schützen} die empfindlichen Lautsprechermembranen. & \\
\tablevspace
{ \textbf{9}} & Vor & Küchenmöbel\textbf{montagen} dürfen nur von geschulten Fachleuten \textbf{durchgeführt} werden. & \textbf{\#7}\\
& Nach & Küchenmöbel dürfen nur von geschulten Fachleuten \textbf{montiert} werden. & \\
\tablevspace
{ \textbf{10}} & Vor & Die \textbf{Reinigung} der Küchenmöbel sollten Sie mit einem leicht feuchten Tuch \textbf{vornehmen}. & \textbf{\#7}\\
& Nach & Sie sollten die Küchenmöbel mit einem leicht feuchten Tuch \textbf{reinigen}. & \\
\tablevspace
{ \textbf{11}} & Vor & Systemlösungen, die in öffentlichen Bereichen \textbf{zum Einsatz kommen}, müssen auf ihre Brennklasse geprüft werden. & \textbf{\#1}\\
& Nach & Systemlösungen, die in öffentlichen Bereichen \textbf{eingesetzt werden}, müssen auf ihre Brennklasse geprüft werden. & \\
\tablevspace
{ \textbf{12}} & Vor & Es \textbf{liegt in der Verantwortung} des Planers, aufeinander abgestimmte Produkte einzusetzen. & \textbf{\#1}\\
& Nach & Der Planer \textbf{ist dafür verantwortlich}, aufeinander abgestimmte Produkte einzusetzen. & \\
\tablevspace
{ \textbf{13}} & Vor & Somit kann die Fluggesellschaft nicht garantieren, dass die Gepäckregeln immer \textbf{zur Anwendung kommen}. & \textbf{\#5}\\
& Nach & Somit kann die Fluggesellschaft nicht garantieren, dass die Gepäckregeln immer \textbf{angewendet werden}. & \\
\tablevspace
{ \textbf{14}} & Vor & Die \textbf{Abwicklung} von Garantieleistungen \textbf{erfolgt} über die lokale Service-Hotline. & \textbf{\#6}\\
& Nach & Die Garantieleistungen \textbf{werden} über die lokale Service-Hotline \textbf{abgewickelt}. & \\
\tablevspace
{ \textbf{15}} & Vor & Das Gerät darf nicht \textbf{in Betrieb genommen werden}, wenn es sichtbare Schäden aufweist. & \textbf{\#9}\\
& Nach & Das Gerät darf nicht \textbf{eingeschaltet werden}, wenn es sichtbare Schäden aufweist. & \\
\tablevspace
{ \textbf{16}} & Vor & Die Höhen\textbf{verstellung} der Fronten können Sie mittels eines Schraubendrehers \textbf{vornehmen}. & \textbf{\#7}\\
& Nach & Die Höhe der Fronten können Sie mittels eines Schraubendrehers \textbf{verstellen}. & \\
\tablevspace
{ \textbf{17}} & Vor & Im oberen Abschnitt können Sie \textbf{Einstellungen} für die angezeigten Module \textbf{vornehmen}. & \textbf{\#4}\\
& Nach & Im oberen Abschnitt können Sie die angezeigten Module \textbf{einstellen}. & \\
\tablevspace
{ \textbf{18}} & Vor & Ist der Wert abweichend, kann eine \textbf{Korrektur vorgenommen werden}. & \textbf{\#4}\\
& Nach & Ist der Wert abweichend, kann er \textbf{korrigiert werden}. & \\
\tablevspace
{ \textbf{19}} & Vor & Auf der Startseite \textbf{stehen} die folgenden Funktionen zur Auswahl \textbf{zur Verfügung}. & \textbf{\#4}\\
& Nach & Auf der Startseite \textbf{sind} die folgenden Funktionen zur Auswahl \textbf{vorhanden}. & \\
\tablevspace
{ \textbf{20}} & Vor & Das Schutzmodul der Maschine darf nicht \textbf{außer Betrieb gesetzt werden}. & \textbf{\#3}\\
& Nach & Das Schutzmodul der Maschine darf nicht \textbf{abschaltet werden}. & \\
\tablevspace
{ \textbf{21}} & Vor & Die Maschine \textbf{ist} täglich 8 Stunden \textbf{in Betrieb}. & \textbf{\#10}\\
& Nach & Die Maschine \textbf{arbeitet} täglich 8 Stunden. & \\
\tablevspace
{ \textbf{22}} & Vor & Sie haben die Möglichkeit, einen \textbf{Antrag} auf Verlängerung der Frist zu \textbf{stellen}. & \textbf{\#10}\\
& Nach & Sie haben die Möglichkeit, eine Verlängerung der Frist zu \textbf{beantragen}. & \\
\tablevspace
{ \textbf{23}} & Vor & Nach Ihrer Registrierung im Programm können Sie aus den Leistungen eine \textbf{Auswahl treffen}. & \textbf{\#10}\\
& Nach & Nach Ihrer Registrierung im Programm können Sie aus den Leistungen \textbf{wählen}. & \\
\tablevspace
{ \textbf{24}} & Vor & Die Kunden \textbf{zeigen} vor allem \textbf{Interesse an} den technischen Neuentwicklungen. & \textbf{\#10}\\
& Nach & Die Kunden \textbf{interessieren sich} vor allem \textbf{für} die technischen Neuentwicklungen. & \\
%\hhline%%replace by cmidrule{~--~}
\lspbottomrule
\end{longtable}


\section*{Regel 3 -- Konditionalsätze mit ‚Wenn‘ einleiten}


\begin{longtable}{llp{.7\textwidth}l}

\lsptoprule
%\hhline%%replace by cmidrule{~~--}
{} & \textbf{KS} & \textbf{Sätze} & \makecell[tl]{\textbf{Ref.}\\\textbf{Quelle}}\\
\midrule
{ \textbf{1}} & Vor & \textbf{Ist} die Seriennummer des Gerätes bekannt, kann im Feld Seriennummer diese Nummer eingegeben werden. & \textbf{\#4}\\
& Nach & \textbf{Wenn} die Seriennummer des Gerätes bekannt \textbf{ist}, kann im Feld Seriennummer diese Nummer eingegeben werden. & \\
\tablevspace
{ \textbf{2}} & Vor & \textbf{Steht} die Maschine nicht im Einsatz, den Hauptschalter auf $"$0$"$ setzen. & \textbf{\#3}\\
& Nach & \textbf{Wenn} die Maschine nicht im Einsatz \textbf{steht}, den Hauptschalter auf $"$0$"$ setzen. & \\
\tablevspace
{ \textbf{3}} & Vor & \textbf{Schließt} der Kontaktschalter, so wird der Raumdruck-Sollwert aktiv. & \textbf{\#4}\\
& Nach & \textbf{Wenn} der Kontaktschalter \textbf{schließt}, wird der Raumdruck-Sollwert aktiv. & \\
\tablevspace
{ \textbf{4}} & Vor & \textbf{Werden} die vordefinierten Werte \textbf{verändert}, so erfolgt die Umrechnung automatisch. & \textbf{\#4}\\
& Nach & \textbf{Wenn} die vordefinierten Werte \textbf{verändert werden}, erfolgt die Umrechnung automatisch. & \\
\tablevspace
{ \textbf{5}} & Vor & \textbf{Ist} diese Zeit \textbf{erreicht}, muss das Gerät für 2 Minuten abkühlen. & \textbf{\#9}\\
& Nach & \textbf{Wenn} diese Zeit \textbf{erreicht ist}, muss das Gerät für 2 Minuten abkühlen. & \\
\tablevspace
{ \textbf{6}} & Vor & \textbf{Wird} der zweite Anschluss als Eingang \textbf{konfiguriert}, so muss der Sollwert angepasst werden. & \textbf{\#4}\\
& Nach & \textbf{Wenn} der zweite Anschluss als Eingang \textbf{konfiguriert wird}, muss der Sollwert angepasst werden. & \\
\tablevspace
{ \textbf{7}} & Vor & \textbf{Ist} die importierte Firmware-Version älter als die installierte Version, erhält der Benutzer eine Warnmeldung. & \textbf{\#4}\\
& Nach & \textbf{Wenn} die importierte Firmware-Version älter als die installierte Version \textbf{ist}, erhält der Benutzer eine Warnmeldung. & \\
\tablevspace
{ \textbf{8}} & Vor & \textbf{Ist} nur ein Gerät \textbf{angeschlossen}, so ist die Funktion PP zu wählen. & \textbf{\#4}\\
& Nach & \textbf{Wenn} nur ein Gerät \textbf{angeschlossen ist}, ist die Funktion PP zu wählen. & \\
\tablevspace
{ \textbf{9}} & Vor & \textbf{Wählt} man einen bestimmten Zeichensatz als Standardwert, so wird dieser Zeichensatz in allen Stationen verwendet. & \textbf{\#4}\\
& Nach & \textbf{Wenn} man einen bestimmten Zeichensatz als Standardwert \textbf{wählt}, wird dieser Zeichensatz in allen Stationen verwendet. & \\
\tablevspace
{ \textbf{10}} & Vor & \textbf{Steht} ein normierter Faktor zur Verfügung, kann dieser Faktor direkt in der Eingabemaske eingegeben werden. & \textbf{\#4}\\
& Nach & \textbf{Wenn} ein normierter Faktor zur Verfügung \textbf{steht}, kann dieser Faktor direkt in der Eingabemaske eingegeben werden. & \\
\tablevspace
{ \textbf{11}} & Vor & \textbf{Wird} diese Regel nicht \textbf{beachtet}, kann der Motor Schaden nehmen. & \textbf{\#9}\\
& Nach & \textbf{Wenn} diese Regel nicht \textbf{beachtet wird}, kann der Motor Schaden nehmen. & \\
\tablevspace
{ \textbf{12}} & Vor & \textbf{Ist} ein mehrstufiges Modul \textbf{parametriert}, so sind die externen Kontakte zu verriegeln. & \textbf{\#4}\\
& Nach & \textbf{Wenn} ein mehrstufiges Modul \textbf{parametriert ist}, sind die externen Kontakte zu verriegeln. & \\
\tablevspace
{ \textbf{13}} & Vor & \textbf{Sind} mehrere Geräte im Netzwerk vorhanden, so ist die Adresse des gewünschten Gerätes auszuwählen. & \textbf{\#4}\\
& Nach & \textbf{Wenn} mehrere Geräte im Netzwerk vorhanden \textbf{sind}, ist die Adresse des gewünschten Gerätes auszuwählen. & \\
\tablevspace
{ \textbf{14}} & Vor & \textbf{Erfolgt} die Zahlung nicht, kann der Anbieter Ersatz eines eventuell entstandenen Schadens verlangen. & \textbf{\#2}\\
& Nach & \textbf{Wenn} die Zahlung nicht erfolgt, kann der Anbieter Ersatz eines eventuell entstandenen Schadens verlangen. & \\
\tablevspace
{ \textbf{15}} & Vor & \textbf{Ist} der c-Faktor mit einer anderen Luftdichte \textbf{angegeben worden}, so ist diese Luftdichte einzutragen. & \textbf{\#4}\\
& Nach & \textbf{Wenn} der c-Faktor mit einer anderen Luftdichte \textbf{angegeben worden ist}, ist diese Luftdichte einzutragen. & \\
\tablevspace
{ \textbf{16}} & Vor & \textbf{Ist} das Gerät oder das Netzkabel \textbf{beschädigt}, sofort den Netzstecker herausziehen. & \textbf{\#9}\\
& Nach & \textbf{Wenn} das Gerät oder das Netzkabel \textbf{beschädigt ist}, sofort den Netzstecker herausziehen. & \\
\tablevspace
{ \textbf{17}} & Vor & \textbf{Werden} Geräte in einem anderen Land \textbf{gekauft}, werden Garantieleistungen nur in diesem Land erbracht. & \textbf{\#6}\\
& Nach & \textbf{Wenn} Geräte in einem anderen Land \textbf{gekauft werden}, werden Garantieleistungen nur in diesem Land erbracht. & \\
\tablevspace
{ \textbf{18}} & Vor & \textbf{Tritt} eine Preisänderung \textbf{ein}, so gilt der neue Preis am Tag der Lieferung. & \textbf{\#2}\\
& Nach & \textbf{Wenn} eine Preisänderung \textbf{eintritt}, gilt der neue Preis am Tag der Lieferung. & \\
\tablevspace
{ \textbf{19}} & Vor & \textbf{Veräußert} der Besteller die gelieferte, unbezahlte Ware, so tritt er dem Lieferer alle Ansprüche ab. & \textbf{\#2}\\
& Nach & \textbf{Wenn} der Besteller die gelieferte, unbezahlte Ware \textbf{veräußert}, tritt er dem Lieferer alle Ansprüche ab. & \\
\tablevspace
{ \textbf{20}} & Vor & \textbf{Wurde} die Maschine mit verdeckten Beschädigungen \textbf{angeliefert}, so verständigen Sie unverzüglich den Lieferanten. & \textbf{\#3}\\
& Nach & \textbf{Wenn} die Maschine mit verdeckten Beschädigungen \textbf{angeliefert wurde}, verständigen Sie unverzüglich den Lieferanten. & \\
\tablevspace
{ \textbf{21}} & Vor & \textbf{Liegt} die Regelabweichung innerhalb der x-Zone, so bleibt das erste Modul stehen. & \textbf{\#4}\\
& Nach & \textbf{Wenn} die Regelabweichung innerhalb der x-Zone \textbf{liegt}, bleibt das erste Modul stehen. & \\
\tablevspace
{ \textbf{22}} & Vor & \textbf{Wird} der Startbildschirm nicht \textbf{angezeigt}, war die Installation wahrscheinlich fehlerhaft. & \textbf{\#4}\\
& Nach & \textbf{Wenn} der Startbildschirm nicht \textbf{angezeigt wird}, war die Installation wahrscheinlich fehlerhaft. & \\
\tablevspace
{ \textbf{23}} & Vor & \textbf{Erscheint} eine Fehlermeldung, war die Konfigurierung wahrscheinlich unvollständig. & \textbf{\#4}\\
& Nach & \textbf{Wenn} eine Fehlermeldung \textbf{erscheint}, war die Konfigurierung wahrscheinlich unvollständig. & \\
\tablevspace
{ \textbf{24}} & Vor & \textbf{Werden} die Standardwerte der Maschine \textbf{überschritten}, führt die Überlastung zum automatischen Abschalten der Maschine. & \textbf{\#3}\\
& Nach & \textbf{Wenn} die Standardwerte der Maschine \textbf{überschritten werden}, führt die Überlastung zum automatischen Abschalten der Maschine. & \\
%\hhline%%replace by cmidrule{~--~}
\lspbottomrule
\end{longtable}

\section*{Regel 4 -- Eindeutige pronominale Bezüge verwenden}


\begin{longtable}{llp{.7\textwidth}l}

\lsptoprule
%\hhline%%replace by cmidrule{~~--}
{} & \textbf{KS} & \textbf{Sätze} & \makecell[tl]{\textbf{Ref.}\\\textbf{Quelle}}\\
\midrule
{ \textbf{1}} & Vor & Je früher ein Fleck behandelt wird, umso größer ist die Wahrscheinlichkeit, \textbf{ihn} rückstandslos zu entfernen. & \textbf{\#1}\\
& Nach & Je früher ein Fleck behandelt wird, umso größer ist die Wahrscheinlichkeit, \textbf{den Fleck} rückstandslos zu entfernen. & \\
\tablevspace
{ \textbf{2}} & Vor & Sofern auf der Oberfläche alte Klebereste anhaften, sind \textbf{diese} vollständig zu entfernen. & \textbf{\#1}\\
& Nach & Sofern auf der Oberfläche alte Klebereste anhaften, sind \textbf{diese Kleberreste} vollständig zu entfernen. & \\
\tablevspace
{ \textbf{3}} & Vor & Um die Startlinie festzustellen, kann mit Kreide \textbf{diese} markiert werden. & \textbf{\#1}\\
& Nach & Um die Startlinie festzustellen, kann mit Kreide \textbf{diese Linie} markiert werden. & \\
\tablevspace
{ \textbf{4}} & Vor & Bei einer Überhitzung bestehen keine gesundheitlichen Risiken; \textbf{diese} werden eher von verbranntem Öl oder Fett verursacht. & \textbf{\#2}\\
& Nach & Bei einer Überhitzung bestehen keine gesundheitlichen Risiken; \textbf{gesundheitliche Risiken} werden eher von verbranntem Öl oder Fett verursacht. & \\
\tablevspace
{ \textbf{5}} & Vor & Fettreste müssen vollständig abgewaschen werden, da sich \textbf{diese} ansonsten in der Pfanne einbrennen können. & \textbf{\#2}\\
& Nach & Fettreste müssen vollständig abgewaschen werden, da sich \textbf{diese Reste} ansonsten in der Pfanne einbrennen können. & \\
\tablevspace
{ \textbf{6}} & Vor & Halten Sie die Schaltschränke stets verschlossen, wenn \textbf{diese} unbeaufsichtigt sind. & \textbf{\#3}\\
& Nach & Halten Sie die Schaltschränke stets verschlossen, wenn \textbf{die Schaltschränke} unbeaufsichtigt sind. & \\
\tablevspace
{ \textbf{7}} & Vor & Nur Elektrofachkräfte dürfen Zugang zur Elektrik der Maschine haben und \textbf{diese} warten. & \textbf{\#3}\\
& Nach & Nur Elektrofachkräfte dürfen Zugang zur Elektrik der Maschine haben und \textbf{die Maschine} warten. & \\
\tablevspace
{ \textbf{8}} & Vor & Wenn Störungen an der elektrischen Energieversorgung der Maschine auftreten, ist \textbf{diese} sofort mit dem Hauptschalter auszuschalten! & \textbf{\#3}\\
& Nach & Wenn Störungen an der elektrischen Energieversorgung der Maschine auftreten, ist \textbf{die Maschine} sofort mit dem Hauptschalter auszuschalten! & \\
\tablevspace
{ \textbf{9}} & Vor & Tab 7 und Tab 8 zeigen die verfügbaren Anwendungen und \textbf{deren} unterschiedlichen Default-Parameter. & \textbf{\#4}\\
& Nach & Tab 7 und Tab 8 zeigen die verfügbaren Anwendungen und \textbf{die} unterschiedlichen Default-Parameter \textbf{dieser Anwendungen}. & \\
\tablevspace
{ \textbf{10}} & Vor & Durch Anklicken des Buttons $"$Zusatzinformation$"$ wird \textbf{sie} über das Netzwerk eingelesen. & \textbf{\#4}\\
& Nach & Durch Anklicken des Buttons $"$Zusatzinformation$"$ wird \textbf{die Zusatzinformation} über das Netzwerk eingelesen. & \\
\tablevspace
{ \textbf{11}} & Vor & Wählt man einen bestimmten Zeichensatz als Standardwert, so wird \textbf{der} in allen Stationen verwendet. & \textbf{\#4}\\
& Nach & Wählt man einen bestimmten Zeichensatz als Standardwert, so wird \textbf{dieser Zeichensatz} in allen Stationen verwendet. & \\
\tablevspace
{ \textbf{12}} & Vor & Steht ein normierter Faktor zur Verfügung, kann direkt mit \textbf{diesem} in der Eingabemaske gearbeitet werden. & \textbf{\#4}\\
& Nach & Steht ein normierter Faktor zur Verfügung, kann direkt mit \textbf{diesem Faktor} in der Eingabemaske gearbeitet werden. & \\
\tablevspace
{ \textbf{13}} & Vor & Veräußert der Besteller die gelieferte, unbezahlte Ware, so tritt \textbf{er} dem Lieferer alle Ansprüche ab. & \textbf{\#2}\\
& Nach & Veräußert der Besteller die gelieferte, unbezahlte Ware, so tritt \textbf{der Besteller} dem Lieferer alle Ansprüche ab. & \\
\tablevspace
{ \textbf{14}} & Vor & Der Bediener darf erst die Maschine in Betrieb nehmen, wenn \textbf{er} die Betriebsanleitung gelesen hat. & \textbf{\#3}\\
& Nach & Der Bediener darf erst die Maschine in Betrieb nehmen, wenn \textbf{der Bediener} die Betriebsanleitung gelesen hat. & \\
\tablevspace
{ \textbf{15}} & Vor & Wenn Sie einen Schaden erst zu Hause feststellen, melden Sie \textbf{ihn} innerhalb von sieben Tagen schriftlich. & \textbf{\#5}\\
& Nach & Wenn Sie einen Schaden erst zu Hause feststellen, melden Sie \textbf{diesen Schaden} innerhalb von sieben Tagen schriftlich. & \\
\tablevspace
{ \textbf{16}} & Vor & Um den hohen Wert der Maschine über Jahre zu erhalten, sollten Sie \textbf{sie} richtig pflegen. & \textbf{\#7}\\
& Nach & Um den hohen Wert der Maschine über Jahre zu erhalten, sollten Sie \textbf{die Maschine} richtig pflegen. & \\
\tablevspace
{ \textbf{17}} & Vor & Ziehen Sie sofort den Netzstecker und betreiben Sie das Gerät nicht, wenn \textbf{dessen} Gehäuse defekt ist. & \textbf{\#8}\\
& Nach & Ziehen Sie sofort den Netzstecker und betreiben Sie das Gerät nicht, wenn \textbf{das Gehäuse} defekt ist. & \\
\tablevspace
{ \textbf{18}} & Vor & Heben Sie die Bedienungsanleitung gut auf und übergeben Sie \textbf{sie} auch an einen möglichen Nachbesitzer. & \textbf{\#8}\\
& Nach & Heben Sie die Bedienungsanleitung gut auf und übergeben Sie \textbf{die Bedienungsanleitung} auch an einen möglichen Nachbesitzer. & \\
\tablevspace
{ \textbf{19}} & Vor & Jeder Verbraucher ist gesetzlich verpflichtet, alte Geräte bei einer Sammelstelle abzugeben, damit \textbf{sie} recycelt werden können. & \textbf{\#8}\\
& Nach & Jeder Verbraucher ist gesetzlich verpflichtet, alte Geräte bei einer Sammelstelle abzugeben, damit \textbf{diese Geräte} recycelt werden können. & \\
\tablevspace
{ \textbf{20}} & Vor & Drehen Sie den Saftbehälter im Uhrzeigersinn bis \textbf{er} hörbar einrastet. & \textbf{\#9}\\
& Nach & Drehen Sie den Saftbehälter im Uhrzeigersinn bis \textbf{der Saftbehälter} hörbar einrastet. & \\
\tablevspace
{ \textbf{21}} & Vor & Ist der c-Faktor mit einer anderen Luftdichte angegeben worden, so ist \textbf{dieser} im Feld $"$Luftdichte$"$ einzutragen. & \textbf{\#4}\\
& Nach & Ist der c-Faktor mit einer anderen Luftdichte angegeben worden, so ist \textbf{dieser Faktor} im Feld $"$Luftdichte$"$ einzutragen. & \\
\tablevspace
{ \textbf{22}} & Vor & Durch Anklicken der entsprechenden Seite wird \textbf{diese} aktiv. & \textbf{\#4}\\
& Nach & Durch Anklicken der entsprechenden Seite wird \textbf{die Seite} aktiv. & \\
\tablevspace
{ \textbf{23}} & Vor & Durch Klick auf die angezeigte Adresse kann \textbf{diese} im Menü konfiguriert werden. & \textbf{\#4}\\
& Nach & Durch Klick auf die angezeigte Adresse kann \textbf{die Adresse} im Menü konfiguriert werden. & \\
\tablevspace
{ \textbf{24}} & Vor & Überprüfen Sie die Zuweisung des Ports im Geräte-Manager und stellen Sie \textbf{diesen} ggf. um. & \textbf{\#4}\\
& Nach & Überprüfen Sie die Zuweisung des Ports im Geräte-Manager und stellen Sie \textbf{diesen Port} ggf. um. & \\
%\hhline%%replace by cmidrule{~--~}
\lspbottomrule
\end{longtable}


\section*{Regel 5 -- Partizipial-konstruktionen vermeiden}


\begin{longtable}{llp{.7\textwidth}l}

\lsptoprule
%\hhline%%replace by cmidrule{~~--}
{} & \textbf{KS} & \textbf{Sätze} & \makecell[tl]{\textbf{Ref.}\\\textbf{Quelle}}\\
\midrule
{ \textbf{1}} & Vor & Das Gerät nur an \textbf{eine vorschriftsmäßig installierte Steckdose} anschließen. & \textbf{\#9}\\
& Nach & Das Gerät nur an \textbf{eine Steckdose} anschließen, \textbf{die vorschriftsmäßig installiert ist}. & \\
\tablevspace
{ \textbf{2}} & Vor & Kunststoffverpackungen in \textbf{die dafür vorgesehenen Entsorgungsbehälter} geben. & \textbf{\#9}\\
& Nach & Kunststoffverpackungen in \textbf{die Entsorgungsbehälter} geben, \textbf{die dafür vorgesehen sind}. & \\
\tablevspace
{ \textbf{3}} & Vor & \textbf{Speziell auf diese Lautsprecher abgestimmtes Zubehör} erhalten Sie in unserem Webshop. & \textbf{\#8}\\
& Nach & \textbf{Zubehör, das speziell auf diese Lautsprecher abgestimmt ist,} erhalten Sie in unserem Webshop. & \\
\tablevspace
{ \textbf{4}} & Vor & Beachten Sie \textbf{die zusätzlich ausgehändigten Bedienungsanleitungen} der Einbaukomponenten. & \textbf{\#7}\\
& Nach & Beachten Sie \textbf{die Bedienungsanleitungen} der Einbaukomponenten, \textbf{die zusätzlich ausgehändigt wurden}. & \\
\tablevspace
{ \textbf{5}} & Vor & \textbf{Alle von uns eingesetzten Schubkästen und Einlegeböden} sind hochwertige Produkte. & \textbf{\#7}\\
& Nach & \textbf{Alle Schubkästen und Einlegeböden, die von uns eingesetzt werden,} sind hochwertige Produkte. & \\
\tablevspace
{ \textbf{6}} & Vor & Mittels \textbf{des von Ihnen ausgefüllten Formulars} wird eine Suche durchgeführt. & \textbf{\#5}\\
& Nach & Mittels \textbf{des Formulars, das von Ihnen ausgefüllt wird}, wird eine Suche durchgeführt. & \\
\tablevspace
{ \textbf{7}} & Vor & \textbf{Die für Ihren Flug erlaubte Freigepäckmenge} ist auf Ihrem Flugschein angegeben. & \textbf{\#5}\\
& Nach & \textbf{Die Freigepäckmenge, die für Ihren Flug erlaubt ist,} ist auf Ihrem Flugschein angegeben. & \\
\tablevspace
{ \textbf{8}} & Vor & Hierzu kann die Funktion $"$Upload$"$ gewählt werden, wodurch \textbf{die im Gerät gespeicherten Daten} geladen werden. & \textbf{\#4}\\
& Nach & Hierzu kann die Funktion $"$Upload$"$ gewählt werden, wodurch \textbf{die Daten, die im Gerät gespeichert sind,} geladen werden. & \\
\tablevspace
{ \textbf{9}} & Vor & Die eingegebene Netzwerkadresse bezieht sich immer auf \textbf{den in der Infozeile angezeigten Gerätetyp}. & \textbf{\#4}\\
& Nach & Die eingegebene Netzwerkadresse bezieht sich immer auf \textbf{den Gerätetyp, der in der Infozeile angezeigt ist}. & \\
\tablevspace
{ \textbf{10}} & Vor & Das Gerät verbindet sich mit \textbf{der neu gewählten Netzwerkadresse}. & \textbf{\#4}\\
& Nach & Das Gerät verbindet sich mit \textbf{der Netzwerkadresse, die neu gewählt wird}. & \\
\tablevspace
{ \textbf{11}} & Vor & Durch Eingabe \textbf{der mit einem roten Sternchen gekennzeichneten Parameter} erfolgt die minimale Konfigurierung. & \textbf{\#4}\\
& Nach & Durch Eingabe \textbf{der Parameter, die mit einem roten Sternchen gekennzeichnet sind}, erfolgt die minimale Konfigurierung. & \\
\tablevspace
{ \textbf{12}} & Vor & In den Einstellungen können Sie \textbf{die grafisch angezeigten Werte} in eine CSV-Datei speichern. & \textbf{\#4}\\
& Nach & In den Einstellungen können Sie \textbf{die Werte, die grafisch angezeigt werden,} in eine CSV-Datei speichern. & \\
\tablevspace
{ \textbf{13}} & Vor & Dank \textbf{dem im System integrierten zweiten C-Modul} werden weitere Anwendungen unterstützt. & \textbf{\#4}\\
& Nach & Dank \textbf{dem zweiten C-Modul, das im System integriert ist}, werden weitere Anwendungen unterstützt. & \\
\tablevspace
{ \textbf{14}} & Vor & Der Digitaleingang dient zur Steuerung \textbf{des am Analogeingang angelegten Sollwertes}. & \textbf{\#4}\\
& Nach & Der Digitaleingang dient zur Steuerung \textbf{des Sollwertes, der am Analogeingang angelegt ist}. & \\
\tablevspace
{ \textbf{15}} & Vor & \textbf{Die in der Betriebsanleitung angegebenen Fristen} für wiederkehrende Prüfungen sind einzuhalten. & \textbf{\#3}\\
& Nach & \textbf{Die Fristen} für wiederkehrende Prüfungen, \textbf{die in der Betriebsanleitung angegeben sind,} sind einzuhalten. & \\
\tablevspace
{ \textbf{16}} & Vor & Die Daten der Menüs werden für \textbf{die neu ausgewählte Anwendung} beibehalten. & \textbf{\#4}\\
& Nach & Die Daten der Menüs werden für \textbf{die Anwendung} beibehalten, \textbf{die neu ausgewählt worden ist}. & \\
\tablevspace
{ \textbf{17}} & Vor & \textbf{Alle darüberhinausgehenden Ansprüche} sind ausdrücklich von der Garantie ausgenommen. & \textbf{\#2}\\
& Nach & \textbf{Alle Ansprüche, die darüber hinausgehen,} sind ausdrücklich von der Garantie ausgenommen. & \\
\tablevspace
{ \textbf{18}} & Vor & Bei \textbf{fristgerecht erfolgten berechtigten Mängelrügen} ist der Lieferer zu einer kostenlosen Ersatzlieferung verpflichtet. & \textbf{\#2}\\
& Nach & Bei \textbf{berechtigten Mängelrügen, die fristgerecht erfolgen}, ist der Lieferer zu einer kostenlosen Ersatzlieferung verpflichtet. & \\
\tablevspace
{ \textbf{19}} & Vor & Erfolgt die Zahlung nicht, kann der Anbieter Ersatz \textbf{eines eventuell entstandenen Schadens} verlangen. & \textbf{\#2}\\
& Nach & Erfolgt die Zahlung nicht, kann der Anbieter Ersatz \textbf{eines Schadens} verlangen, \textbf{der eventuell entsteht}. & \\
\tablevspace
{ \textbf{20}} & Vor & Die produktspezifischen Betriebsanleitungen sind auf \textbf{der zusätzlich gelieferten CD} dokumentiert. & \textbf{\#3}\\
& Nach & Die produktspezifischen Betriebsanleitungen sind auf \textbf{der CD} dokumentiert, \textbf{die zusätzlich geliefert wurde}. & \\
\tablevspace
{ \textbf{21}} & Vor & \textbf{Die in der Betriebsanleitung enthaltenen Sicherheitshinweise} sind mit dem allgemeinen Gefahrensymbol gekennzeichnet. & \textbf{\#3}\\
& Nach & \textbf{Die Sicherheitshinweise, die in der Betriebsanleitung enthalten sind,} sind mit dem allgemeinen Gefahrensymbol gekennzeichnet. & \\
\tablevspace
{ \textbf{22}} & Vor & \textbf{Die in den Bedienungsanweisungen der eingebauten Geräte vorgeschriebenen Gebrauchsbedingungen} müssen strikt eingehalten werden. & \textbf{\#3}\\
& Nach & \textbf{Die Gebrauchsbedingungen, die in den Bedienungsanweisungen der eingebauten Geräte vorgeschrieben sind,} müssen strikt eingehalten werden. & \\
\tablevspace
{ \textbf{23}} & Vor & \textbf{Für hieraus resultierende Schäden} haftet allein der Betreiber der Maschine. & \textbf{\#3}\\
& Nach & \textbf{Für Schäden, die hieraus resultieren}, haftet allein der Betreiber der Maschine. & \\
\tablevspace
{ \textbf{24}} & Vor & \textbf{Die für die Maschine benötigten Werkzeuge} sind im Lieferumfang nicht enthalten. & \textbf{\#3}\\
& Nach & \textbf{Die Werkzeuge, die für die Maschine benötigt werden,} sind im Lieferumfang nicht enthalten. & \\
%\hhline%%replace by cmidrule{~--~}
\lspbottomrule
\end{longtable}


\section*{Regel 6 -- Passiv vermeiden}


\begin{longtable}{llp{.7\textwidth}l}

\lsptoprule
%\hhline%%replace by cmidrule{~~--}
{} & \textbf{KS} & \textbf{Sätze} & \makecell[tl]{\textbf{Ref.}\\\textbf{Quelle}}\\
\midrule
{ \textbf{1}} & Vor & Bei der Arbeit mit elektrischen Geräten \textbf{sollte} stets ein Sicherheitsstecker \textbf{verwendet werden}. & \textbf{\#3}\\
& Nach & Bei der Arbeit mit elektrischen Geräten \textbf{verwenden Sie} stets einen Sicherheitsstecker. & \\
\tablevspace
{ \textbf{2}} & Vor & Ist die Seriennummer des Gerätes bekannt, \textbf{kann} im Feld Seriennummer diese Nummer \textbf{eingegeben werden}. & \textbf{\#4}\\
& Nach & Ist die Seriennummer des Gerätes bekannt, \textbf{können Sie} im Feld Seriennummer diese Nummer \textbf{eingeben}. & \\
\tablevspace
{ \textbf{3}} & Vor & Wenn eine korrekte Verbindung \textbf{aufgebaut werden konnte}, wird in der Statuszeile das Feld Verbindung grün. & \textbf{\#4}\\
& Nach & Wenn \textbf{Sie} eine korrekte Verbindung \textbf{aufbauen konnten}, wird in der Statuszeile das Feld Verbindung grün. & \\
\tablevspace
{ \textbf{4}} & Vor & Im Reiter Kommunikation \textbf{können} die notwendigen Einstellungen zur Kommunikation über das IP-Netzwerk \textbf{vorgenommen werden}. & \textbf{\#4}\\
& Nach & Im Reiter Kommunikation \textbf{können Sie} die notwendigen Einstellungen zur Kommunikation über das IP-Netzwerk \textbf{vornehmen}. & \\
\tablevspace
{ \textbf{5}} & Vor & Reparaturen \textbf{dürfen} nur von autorisierten Fachbetrieben \textbf{durchgeführt werden}. & \textbf{\#9}\\
& Nach & Nur autorisierte Fachbetriebe \textbf{dürfen} die Reparaturen \textbf{durchführen}. & \\
\tablevspace
{ \textbf{6}} & Vor & Das Gerät stoppt, sobald der Druck auf den Presskegel \textbf{vermindert wird}. & \textbf{\#9}\\
& Nach & Das Gerät stoppt, sobald \textbf{Sie} den Druck auf den Presskegel \textbf{vermindern}. & \\
\tablevspace
{ \textbf{7}} & Vor & Flecken \textbf{sollten} so schnell wie möglich \textbf{behandelt werden}. & \textbf{\#1}\\
& Nach & Flecken \textbf{sollten Sie} so schnell wie möglich \textbf{behandeln}. & \\
\tablevspace
{ \textbf{8}} & Vor & Nach Ablauf des Spülprogramms \textbf{sollte} der Geschirrspüler nicht sofort \textbf{geöffnet werden}. & \textbf{\#7}\\
& Nach & Nach Ablauf des Spülprogramms \textbf{sollten Sie} den Geschirrspüler nicht sofort \textbf{öffnen}. & \\
\tablevspace
{ \textbf{9}} & Vor & Die akustischen Signale \textbf{können} je nach Gerät \textbf{umprogrammiert werden}. & \textbf{\#7}\\
& Nach & Die akustischen Signale \textbf{können Sie} je nach Gerät \textbf{umprogrammieren}. & \\
\tablevspace
{ \textbf{10}} & Vor & Achten Sie darauf, dass das Infrarotlicht nicht durch Gegenstände \textbf{behindert wird}. & \textbf{\#8}\\
& Nach & Achten Sie darauf, dass keine Gegenstände das Infrarotlicht \textbf{behindern}. & \\
\tablevspace
{ \textbf{11}} & Vor & Das Modul XY \textbf{darf} nur für seinen spezifizierten Einsatzzweck \textbf{verwendet werden}. & \textbf{\#4}\\
& Nach & \textbf{Sie dürfen} das Modul XY nur für seinen spezifizierten Einsatzzweck \textbf{verwenden}. & \\
\tablevspace
{ \textbf{12}} & Vor & Durch diese Öffnung \textbf{kann} der Stecker mit dem Regler \textbf{verbunden werden}. & \textbf{\#4}\\
& Nach & Durch diese Öffnung \textbf{können Sie} den Stecker mit dem Regler \textbf{verbinden}. & \\
\tablevspace
{ \textbf{13}} & Vor & Um mit einem anderen Gerätetyp zu kommunizieren, \textbf{muss} zuerst die Gerätenummer \textbf{eingegeben werden}. & \textbf{\#4}\\
& Nach & Um mit einem anderen Gerätetyp zu kommunizieren, \textbf{müssen Sie} zuerst die Gerätenummer \textbf{eingeben}. & \\
\tablevspace
{ \textbf{14}} & Vor & In dem Drop-down-Menü \textbf{kann} die gewünschte IP-Adresse \textbf{gewählt werden}. & \textbf{\#4}\\
& Nach & In dem Drop-down-Menü \textbf{können Sie} die gewünschte IP-Adresse \textbf{wählen}. & \\
\tablevspace
{ \textbf{15}} & Vor & Sinkt der Druck unter ca. 1,3 bar, \textbf{wird} die Pumpe \textbf{gestartet}. & \textbf{\#10}\\
& Nach & Sinkt der Druck unter ca. 1,3 bar, \textbf{startet} die Pumpe. & \\
\tablevspace
{ \textbf{16}} & Vor & Bei der Verwendung von nur einem Ausgang \textbf{kann} der zweite Ausgang \textbf{verschlossen werden}. & \textbf{\#10}\\
& Nach & Bei der Verwendung von nur einem Ausgang \textbf{können Sie} den zweiten Ausgang \textbf{verschließen}. & \\
\tablevspace
{ \textbf{17}} & Vor & Der EIN/AUS-Schalter \textbf{kann} komfortabel mit dem Fuß \textbf{betätigt werden}. & \textbf{\#10}\\
& Nach & \textbf{Sie können} den EIN/AUS-Schalter komfortabel mit dem Fuß \textbf{betätigen}. & \\
\tablevspace
{ \textbf{18}} & Vor & Das Programm \textbf{wird} vom Hersteller wie folgt \textbf{eingestellt}. & \textbf{\#3}\\
& Nach & Der Hersteller \textbf{stellt} das Programm wie folgt \textbf{ein}. & \\
\tablevspace
{ \textbf{19}} & Vor & Schutzvorrichtungen \textbf{dürfen} nur nach Stillstand der Maschine \textbf{entfernt werden}. & \textbf{\#3}\\
& Nach & \textbf{Sie dürfen} die Schutzvorrichtungen nur nach Stillstand der Maschine \textbf{entfernen}. & \\
\tablevspace
{ \textbf{20}} & Vor & Die Maschine \textbf{darf} nur mithilfe eines Gabelstaplers \textbf{angehoben werden}. & \textbf{\#3}\\
& Nach & \textbf{Sie dürfen} die Maschine nur mithilfe eines Gabelstaplers \textbf{anheben}. & \\
\tablevspace
{ \textbf{21}} & Vor & Die Transportösen \textbf{müssen} nach dem Transport der Maschine \textbf{demontiert werden}. & \textbf{\#3}\\
& Nach & \textbf{Sie müssen} die Transportösen nach dem Transport der Maschine \textbf{demontieren}. & \\
\tablevspace
{ \textbf{22}} & Vor & Die Laufrollen \textbf{müssen} nach dem Aufstellen der Maschine \textbf{verriegelt werden}. & \textbf{\#3}\\
& Nach & \textbf{Sie müssen} die Laufrollen nach dem Aufstellen der Maschine \textbf{verriegeln}. & \\
\tablevspace
{ \textbf{23}} & Vor & Die Konfigurierung des Moduls \textbf{kann} in eine Datei \textbf{exportiert werden}. & \textbf{\#4}\\
& Nach & \textbf{Sie können} die Konfigurierung des Moduls in eine Datei \textbf{exportieren}. & \\
\tablevspace
{ \textbf{24}} & Vor & Alle Konfigurationsdaten \textbf{können} mithilfe der Druckfunktion in eine Datei \textbf{gedruckt werden}. & \textbf{\#4}\\
& Nach & \textbf{Sie können} alle Konfigurationsdaten mithilfe der Druckfunktion in eine Datei \textbf{drucken}. & \\
%\hhline%%replace by cmidrule{~--~}
\lspbottomrule
\end{longtable}


\section*{Regel 7 -- Konstruktionen mit ‚sein + zu + Infinitiv‘ vermeiden}


\begin{longtable}{llp{.7\textwidth}l}

\lsptoprule
%\hhline%%replace by cmidrule{~~--}
{} & \textbf{KS} & \textbf{Sätze} & \makecell[tl]{\textbf{Ref.}\\\textbf{Quelle}}\\
\midrule
{ \textbf{1}} & Vor & Mängel \textbf{sind} unverzüglich \textbf{zu melden}. & \textbf{\#1}\\
& Nach & \textbf{Melden Sie} etwaige Mängel unverzüglich. & \\
\tablevspace
{ \textbf{2}} & Vor & Die Herstelleranweisungen \textbf{sind} stets \textbf{zu beachten}. & \textbf{\#1}\\
& Nach & \textbf{Beachten Sie} stets die Herstelleranweisungen. & \\
\tablevspace
{ \textbf{3}} & Vor & Festgestellte Schäden \textbf{sind} sofort \textbf{zu beheben}. & \textbf{\#3}\\
& Nach & \textbf{Beheben Sie} festgestellte Schäden sofort. & \\
\tablevspace
{ \textbf{4}} & Vor & Das Kaufdatum \textbf{ist} durch eine Kaufquittung \textbf{zu belegen}. & \textbf{\#6}\\
& Nach-KS & \textbf{Belegen Sie} das Kaufdatum durch eine Kaufquittung. & \\
\tablevspace
{ \textbf{5}} & Vor & Die Bedienungsanleitung \textbf{ist} im Falle des Verlustes \textbf{zu ersetzen}. & \textbf{\#7}\\
& Nach & \textbf{Ersetzen Sie} die Bedienungsanleitung im Falle des Verlustes. & \\
\tablevspace
{ \textbf{6}} & Vor & Ist nur ein Gerät angeschlossen, so \textbf{ist} die Funktion PP \textbf{zu wählen}. & \textbf{\#4}\\
& Nach & Ist nur ein Gerät angeschlossen, so \textbf{wählen Sie} die Funktion PP. & \\
\tablevspace
{ \textbf{7}} & Vor & Die Teppichböden \textbf{sind} entsprechend den Liefer- und Zahlungsbedingungen \textbf{zu prüfen}. & \textbf{\#1}\\
& Nach & \textbf{Prüfen Sie} die Teppichböden entsprechend den Liefer- und Zahlungsbedingungen. & \\
\tablevspace
{ \textbf{8}} & Vor & Bei Funktionsstörungen \textbf{ist} die Maschine sofort \textbf{auszuschalten}. & \textbf{\#3}\\
& Nach & Bei Funktionsstörungen \textbf{schalten Sie} die Maschine sofort \textbf{aus}. & \\
\tablevspace
{ \textbf{9}} & Vor & Das Bedienungspersonal \textbf{ist} über das Problem vor dem Reparaturbeginn \textbf{zu informieren}. & \textbf{\#3}\\
& Nach & \textbf{Informieren Sie} das Bedienungspersonal über das Problem vor dem Reparaturbeginn. & \\
\tablevspace
{ \textbf{10}} & Vor & Die Maschine \textbf{ist} gegen das Einschalten durch Unbefugte \textbf{zu sichern}. & \textbf{\#3}\\
& Nach & \textbf{Sichern Sie} die Maschine gegen das Einschalten durch Unbefugte. & \\
\tablevspace
{ \textbf{11}} & Vor & Folgende Reinigungsarbeiten der Schneidwerkzeuge \textbf{sind} täglich nach Produktionsende \textbf{durchzuführen}. & \textbf{\#3}\\
& Nach & \textbf{Führen Sie} folgende Reinigungsarbeiten der Schneidwerkzeuge täglich nach Produktionsende \textbf{durch}. & \\
\tablevspace
{ \textbf{12}} & Vor & Sofern auf der Oberfläche alte Kleberreste anhaften, \textbf{sind} diese vollständig \textbf{zu entfernen}. & \textbf{\#1}\\
& Nach & Sofern auf der Oberfläche alte Kleberreste anhaften, \textbf{entfernen Sie} diese vollständig. & \\
\tablevspace
{ \textbf{13}} & Vor & Die vorgeschriebenen Fristen für wiederkehrende Inspektionen \textbf{sind einzuhalten}. & \textbf{\#3}\\
& Nach & \textbf{Halten Sie} die vorgeschriebenen Fristen für wiederkehrende Inspektionen \textbf{ein}. & \\
\tablevspace
{ \textbf{14}} & Vor & Sämtliche Wartungsarbeiten \textbf{sind} nach Betriebsanleitung des Herstellers \textbf{auszuführen}. & \textbf{\#3}\\
& Nach & \textbf{Führen Sie} sämtliche Wartungsarbeiten nach Betriebsanleitung des Herstellers \textbf{aus}. & \\
\tablevspace
{ \textbf{15}} & Vor & Hierzu \textbf{ist} die Funktion Manueller Betrieb im Bereich Servicefunktionen \textbf{zu wählen}. & \textbf{\#4}\\
& Nach & Hierzu \textbf{wählen Sie} die Funktion Manueller Betrieb im Bereich Servicefunktionen. & \\
\tablevspace
{ \textbf{16}} & Vor & Ist ein mehrstufiges Modul parametriert, so \textbf{sind} die externen Kontakte \textbf{zu verriegeln}. & \textbf{\#4}\\
& Nach & Ist ein mehrstufiges Modul parametriert, \textbf{verriegeln Sie} die externen Kontakte. & \\
\tablevspace
{ \textbf{17}} & Vor & Um die Verbindung mit dem Regler herzustellen, \textbf{ist} die Verschlusskappe \textbf{zu öffnen}. & \textbf{\#4}\\
& Nach & Um die Verbindung mit dem Regler herzustellen, \textbf{öffnen Sie} die Verschlusskappe. & \\
\tablevspace
{ \textbf{18}} & Vor & Vor der Parametrierung \textbf{ist} der Regler \textbf{zu konfigurieren}. & \textbf{\#4}\\
& Nach & Vor der Parametrierung \textbf{konfigurieren Sie} den Regler. & \\
\tablevspace
{ \textbf{19}} & Vor & Ist der c-Faktor mit einer anderen Luftdichte angegeben worden, so \textbf{ist} diese Luftdichte \textbf{einzutragen}. & \textbf{\#4}\\
& Nach & Ist der c-Faktor mit einer anderen Luftdichte angegeben worden, \textbf{tragen Sie} diese Luftdichte \textbf{ein}. & \\
\tablevspace
{ \textbf{20}} & Vor & Nach der Parametrierung \textbf{ist} die Verbindung zwischen dem Regler und dem PC \textbf{zu trennen}. & \textbf{\#4}\\
& Nach & Nach der Parametrierung \textbf{trennen Sie} die Verbindung zwischen dem Regler und dem PC. & \\
\tablevspace
{ \textbf{21}} & Vor & Sind mehrere Geräte im Netzwerk vorhanden, so \textbf{ist} die Adresse des gewünschten Gerätes \textbf{auszuwählen}. & \textbf{\#4}\\
& Nach & Sind mehrere Geräte im Netzwerk vorhanden, \textbf{wählen Sie} die Adresse des gewünschten Gerätes \textbf{aus}. & \\
\tablevspace
{ \textbf{22}} & Vor & Um den Sollwert zu erreichen, \textbf{ist} die Konfigurierung des Anschlusses \textbf{zu berücksichtigen}. & \textbf{\#4}\\
& Nach & Um den Sollwert zu erreichen, \textbf{berücksichtigen Sie} die Konfigurierung des Anschlusses. & \\
\tablevspace
{ \textbf{23}} & Vor & Zum Anschluss an den PC \textbf{sind} die beiliegenden Kabel miteinander \textbf{zu verbinden}. & \textbf{\#4}\\
& Nach & Zum Anschluss an den PC \textbf{verbinden Sie} die beiliegenden Kabel miteinander. & \\
\tablevspace
{ \textbf{24}} & Vor & Gerät und Netzkabel \textbf{sind} von Kindern unter 8 Jahren \textbf{fernzuhalten}. & \textbf{\#6}\\
& Nach & \textbf{Halten Sie} das Gerät und das Netzkabel von Kindern unter 8 Jahren \textbf{fern}. & \\
%\hhline%%replace by cmidrule{~--~}
\lspbottomrule
\end{longtable}



\section*{Regel 8 -- Überflüssige Präfixe vermeiden}


\begin{longtable}{llp{.7\textwidth}l}

\lsptoprule
%\hhline%%replace by cmidrule{~~--}
{} & \textbf{KS} & \textbf{Sätze} & \makecell[tl]{\textbf{Ref.}\\\textbf{Quelle}}\\
\midrule
{ \textbf{1}} & Vor & Das Tool \textbf{bietet} diese Korrekturen bei der Eingabe der Werte im Bereich Übersicht \textbf{an}. & \textbf{\#4}\\
& Nach & Das Tool \textbf{bietet} diese Korrekturen bei der Eingabe der Werte im Bereich Übersicht. & \\
\tablevspace
{ \textbf{2}} & Vor & Sie können im nächsten Schritt ein Installationsverzeichnis für das Tool \textbf{auswählen}. & \textbf{\#4}\\
& Nach & Sie können im nächsten Schritt ein Installationsverzeichnis für das Tool \textbf{wählen}. & \\
\tablevspace
{ \textbf{3}} & Vor & Bevor Sie das Gerät in Betrieb nehmen, \textbf{lesen} Sie zuerst die Bedienungsanleitung aufmerksam \textbf{durch}. & \textbf{\#8}\\
& Nach & Bevor Sie das Gerät in Betrieb nehmen, \textbf{lesen} Sie zuerst die Bedienungsanleitung aufmerksam. & \\
\tablevspace
{ \textbf{4}} & Vor & \textbf{Wählt} man einen bestimmten Zeichensatz als Standardwert \textbf{aus}, wird dieser Zeichensatz in allen Stationen verwendet. & \textbf{\#4}\\
& Nach & \textbf{Wählt} man einen bestimmten Zeichensatz als Standardwert, wird dieser Zeichensatz in allen Stationen verwendet. & \\
\tablevspace
{ \textbf{5}} & Vor & \textbf{Kaufen} Sie die Geräte in einem anderen Land \textbf{ein}, werden Garantieleistungen nur in diesem Land erbracht. & \textbf{\#6}\\
& Nach & \textbf{Kaufen} Sie die Geräte in einem anderen Land, werden Garantieleistungen nur in diesem Land erbracht. & \\
\tablevspace
{ \textbf{6}} & Vor & Die Kleberreste \textbf{haften} auf der Oberfläche \textbf{an}, wenn sie nicht schnell entfernt werden. & \textbf{\#1}\\
& Nach & Die Kleberreste \textbf{haften} auf der Oberfläche, wenn sie nicht schnell entfernt werden. & \\
\tablevspace
{ \textbf{7}} & Vor & \textbf{Überprüfen} Sie die Adresse des Ports im Geräte-Manager. & \textbf{\#4}\\
& Nach & \textbf{Prüfen} Sie die Adresse des Ports im Geräte-Manager. & \\
\tablevspace
{ \textbf{8}} & Vor & Durch Anklicken des Buttons Zusatzinformation \textbf{speichert} das System die Angaben \textbf{ab}. & \textbf{\#4}\\
& Nach & Durch Anklicken des Buttons Zusatzinformation \textbf{speichert} das System die Angaben. & \\
\tablevspace
{ \textbf{9}} & Vor & Der Spediteur \textbf{liefert} die Ware zum vereinbarten Termin \textbf{an}. & \textbf{\#3}\\
& Nach & Der Spediteur \textbf{liefert} die Ware zum vereinbarten Termin. & \\
\tablevspace
{ \textbf{10}} & Vor & \textbf{Wählen} Sie die Option $"$Software von einer bestimmten Liste installieren$"$ \textbf{aus}. & \textbf{\#4}\\
& Nach & \textbf{Wählen} Sie die Option $"$Software von einer bestimmten Liste installieren$"$. & \\
\tablevspace
{ \textbf{11}} & Vor & \textbf{Schicken} Sie das Gerät originalverpackt an unsere Serviceadresse \textbf{ein}. & \textbf{\#8}\\
& Nach & \textbf{Schicken} Sie das Gerät originalverpackt an unsere Serviceadresse. & \\
\tablevspace
{ \textbf{12}} & Vor & Sie können die angezeigten Werte lokal auf der Festplatte \textbf{abspeichern}. & \textbf{\#4}\\
& Nach & Sie können die angezeigten Werte lokal auf der Festplatte \textbf{speichern}. & \\
\tablevspace
{ \textbf{13}} & Vor & Sorgen Sie dafür, dass die Quelle ein einwandfreies Signal \textbf{absendet}. & \textbf{\#8}\\
& Nach & Sorgen Sie dafür, dass die Quelle ein einwandfreies Signal \textbf{sendet}. & \\
\tablevspace
{ \textbf{14}} & Vor & \textbf{Schicken} Sie das Gerät zusammen mit dem Original-Kaufbeleg an nachstehende Adresse \textbf{zu}. & \textbf{\#9}\\
& Nach & \textbf{Schicken} Sie das Gerät zusammen mit dem Original-Kaufbeleg an nachstehende Adresse. & \\
\tablevspace
{ \textbf{15}} & Vor & Wir \textbf{senden} Ihnen innerhalb 24 Stunden einen Paketaufkleber für die kostenlose Rücksendung \textbf{zu}. & \textbf{\#9}\\
& Nach & Wir \textbf{senden} Ihnen innerhalb 24 Stunden einen Paketaufkleber für die kostenlose Rücksendung. & \\
\tablevspace
{ \textbf{16}} & Vor & \textbf{Überprüfen} Sie, ob ausreichend Wasser im Wassertank vorhanden ist. & \textbf{\#1}\\
& Nach & \textbf{Prüfen} Sie, ob ausreichend Wasser im Wassertank vorhanden ist. & \\
\tablevspace
{ \textbf{17}} & Vor & In der zweiten Phase \textbf{sendet} die Quelle ein Signal \textbf{ab}. & \textbf{\#8}\\
& Nach & In der zweiten Phase \textbf{sendet} die Quelle ein Signal. & \\
\tablevspace
{ \textbf{18}} & Vor & \textbf{Speichern} Sie die angezeigten Werte lokal auf der Festplatte \textbf{ab}. & \textbf{\#4}\\
& Nach & \textbf{Speichern} Sie die angezeigten Werte lokal auf der Festplatte. & \\
\tablevspace
{ \textbf{19}} & Vor & Falls Ihr Receiver diese Möglichkeiten nicht \textbf{anbietet}, können Sie nur die wichtigsten Einstellungen vornehmen. & \textbf{\#8}\\
& Nach & Falls Ihr Receiver diese Möglichkeiten nicht \textbf{bietet}, können Sie nur die wichtigsten Einstellungen vornehmen. & \\
\tablevspace
{ \textbf{20}} & Vor & Wenn Sie die Geräte in einem anderen Land \textbf{einkaufen}, werden Garantieleistungen nur in diesem Land erbracht. & \textbf{\#6}\\
& Nach & Wenn Sie die Geräte in einem anderen Land \textbf{kaufen}, werden Garantieleistungen nur in diesem Land erbracht. & \\
\tablevspace
{ \textbf{21}} & Vor & Bevor Sie das Gerät in Betrieb nehmen, müssen Sie zuerst die Bedienungsanleitung aufmerksam \textbf{durchlesen}. & \textbf{\#8}\\
& Nach & Bevor Sie das Gerät in Betrieb nehmen, müssen Sie zuerst die Bedienungsanleitung aufmerksam \textbf{lesen}. & \\
\tablevspace
{ \textbf{22}} & Vor & Sofern auf der Oberfläche Kleberreste \textbf{anhaften}, sind diese vollständig zu entfernen. & \textbf{\#1}\\
& Nach & Sofern auf der Oberfläche Kleberreste \textbf{haften}, sind diese vollständig zu entfernen. & \\
\tablevspace
{ \textbf{23}} & Vor & Tab 7 und Tab 8 \textbf{zeigen} die verfügbaren Anwendungen \textbf{an}. & \textbf{\#4}\\
& Nach & Tab 7 und Tab 8 \textbf{zeigen} die verfügbaren Anwendungen. & \\
\tablevspace
{ \textbf{24}} & Vor & In Tab 7 und Tab 8 \textbf{werden} die verfügbaren Anwendungen \textbf{angezeigt}. & \textbf{\#4}\\
& Nach & In Tab 7 und Tab 8 \textbf{werden} die verfügbaren Anwendungen \textbf{gezeigt}. & \\
%\hhline%%replace by cmidrule{~--~}
\lspbottomrule
\end{longtable}



\section*{Regel 9 -- Keine Wortteile weglassen}


\begin{longtable}{llp{.7\textwidth}l}

\lsptoprule
%\hhline%%replace by cmidrule{~~--}
{} & \textbf{KS} & \textbf{Sätze} & \makecell[tl]{\textbf{Ref.}\\\textbf{Quelle}}\\
\midrule
{ \textbf{1}} & Vor & \textbf{Elektro- und Gasgeräte} dürfen nur von geschulten Fachleuten installiert werden. & \textbf{\#7}\\
& Nach & \textbf{Elektrogeräte und Gasgeräte} dürfen nur von geschulten Fachleuten installiert werden. & \\
\tablevspace
{ \textbf{2}} & Vor & Innerhalb der Garantiezeit beseitigen wir alle Mängel des Gerätes, die auf \textbf{Material- oder Fabrikationsfehlern} beruhen. & \textbf{\#6}\\
& Nach & Innerhalb der Garantiezeit beseitigen wir alle Mängel des Gerätes, die auf \textbf{Materialfehlern oder Fabrikationsfehlern} beruhen. & \\
\tablevspace
{ \textbf{3}} & Vor & Sogar \textbf{Soja- und laktosefreie Milch} lassen sich mit dieser Maschine perfekt aufschäumen. & \textbf{\#6}\\
& Nach & Sogar \textbf{Sojamilch und laktosefreie Milch} lassen sich mit dieser Maschine perfekt aufschäumen. & \\
\tablevspace
{ \textbf{4}} & Vor & Das Gleiche gilt bei Nichtbeachtung der \textbf{Gebrauchs-, Pflege- und Wartungsanweisung}. & \textbf{\#6}\\
& Nach & Das Gleiche gilt bei Nichtbeachtung der \textbf{Gebrauchsanweisung, Pflegeanweisung und Wartungsanweisung}. & \\
\tablevspace
{ \textbf{5}} & Vor & Die Teppichböden sind entsprechend den \textbf{Liefer- und Zahlungsbedingungen} zu prüfen. & \textbf{\#1}\\
& Nach & Die Teppichböden sind entsprechend den \textbf{Lieferbedingungen und Zahlungsbedingungen} zu prüfen. & \\
\tablevspace
{ \textbf{6}} & Vor & Hartnäckige Flecken, wie Fettspritzer, \textbf{Lack- oder Klebstoffreste}, sind mit handelsüblichem Kunststoffreiniger eventuell zu beseitigen. & \textbf{\#7}\\
& Nach & Hartnäckige Flecken, wie Fettspritzer, \textbf{Lackreste oder Klebstoffreste}, sind mit handelsüblichem Kunststoffreiniger eventuell zu beseitigen. & \\
\tablevspace
{ \textbf{7}} & Vor & Wir raten Ihnen, die \textbf{Bedienungs- und Pflegehinweise} des Herstellers genauestens zu lesen. & \textbf{\#7}\\
& Nach & Wir raten Ihnen, die \textbf{Bedienungshinweise und Pflegehinweise} des Herstellers genauestens zu lesen. & \\
\tablevspace
{ \textbf{8}} & Vor & Dies ist nicht auf einen \textbf{Konstruktions- oder Verarbeitungsfehler} in unseren Möbeln zurückzuführen. & \textbf{\#7}\\
& Nach & Dies ist nicht auf einen \textbf{Konstruktionsfehler oder Verarbeitungsfehler} in unseren Möbeln zurückzuführen. & \\
\tablevspace
{ \textbf{9}} & Vor & Die wichtigsten Parameter der \textbf{Ein- und Ausgangskonfiguration} sind voreingestellt. & \textbf{\#4}\\
& Nach & Die wichtigsten Parameter der \textbf{Eingangskonfiguration und Ausgangskonfiguration} sind voreingestellt. & \\
\tablevspace
{ \textbf{10}} & Vor & Im Falle der Auswahl der freien Konfiguration kann der \textbf{Start- und Endpunkt} frei gewählt werden. & \textbf{\#4}\\
& Nach & Im Falle der Auswahl der freien Konfiguration können der \textbf{Startpunkt und der Endpunkt} frei gewählt werden. & \\
\tablevspace
{ \textbf{11}} & Vor & Die \textbf{Ist- und Sollwerte} des zweiten Regelkreises werden nach der Konfiguration angezeigt. & \textbf{\#4}\\
& Nach & Der \textbf{Istwert und der Sollwert} des zweiten Regelkreises werden nach der Konfiguration angezeigt. & \\
\tablevspace
{ \textbf{12}} & Vor & Schützen Sie das Gerät vor \textbf{Tropf- und Spritzwasser}. & \textbf{\#8}\\
& Nach & Schützen Sie das Gerät vor \textbf{Tropfwasser und Spritzwasser}. & \\
\tablevspace
{ \textbf{13}} & Vor & \textbf{Kalk- und Wasserflecken} beseitigen Sie mit dem vom Hersteller empfohlenen Spezialreiniger. & \textbf{\#7}\\
& Nach & \textbf{Kalkflecken und Wasserflecken} beseitigen Sie mit dem vom Hersteller empfohlenen Spezialreiniger. & \\
\tablevspace
{ \textbf{14}} & Vor & Es ist darauf zu achten, dass die \textbf{Garn- oder Mikrofaserpads} regelmäßig gewechselt werden. & \textbf{\#1}\\
& Nach & Es ist darauf zu achten, dass die \textbf{Garnpads oder Mikrofaserpads} regelmäßig gewechselt werden. & \\
\tablevspace
{ \textbf{15}} & Vor & \textbf{Kunststoffgriffe und -deckelknöpfe} werden bei Verwendung im Backofen heiß. & \textbf{\#2}\\
& Nach & \textbf{Kunststoffgriffe und Kunststoffdeckelknöpfe} werden bei Verwendung im Backofen heiß. & \\
\tablevspace
{ \textbf{16}} & Vor & Trennen Sie alle \textbf{Spannungs- und Druckquellen} von der Maschine. & \textbf{\#3}\\
& Nach & Trennen Sie alle \textbf{Spannungsquellen und Druckquellen} von der Maschine. & \\
\tablevspace
{ \textbf{17}} & Vor & Eine Bevorratung der wichtigsten \textbf{Ersatz- und Verschleißteile} ist eine wichtige Voraussetzung für ständige Einsatzbereitschaft. & \textbf{\#3}\\
& Nach & Eine Bevorratung der wichtigsten \textbf{Ersatzteile und Verschleißteile} ist eine wichtige Voraussetzung für ständige Einsatzbereitschaft. & \\
\tablevspace
{ \textbf{18}} & Vor & Wechseln der Schneidköpfe ist \textbf{prozess- und produktabhängig}. & \textbf{\#3}\\
& Nach & Wechseln der Schneidköpfe ist \textbf{prozessabhängig und produktabhängig}. & \\
\tablevspace
{ \textbf{19}} & Vor & Die \textbf{Sicherheits- und Anwendungsrichtlinien} der Reinigungsmittelhersteller sind stets einzuhalten. & \textbf{\#1}\\
& Nach & Die \textbf{Sicherheitsrichtlinien und Anwendungsrichtlinien} der Reinigungsmittelhersteller sind stets einzuhalten. & \\
\tablevspace
{ \textbf{20}} & Vor & In \textbf{Eingangs- und Übergangsbereichen} sind große Eingangsmatten fest zu installieren. & \textbf{\#1}\\
& Nach & In \textbf{Eingangsbereichen und Übergangsbereichen} sind große Eingangsmatten fest zu installieren. & \\
\tablevspace
{ \textbf{21}} & Vor & Die \textbf{Anwärm- und Entwässerungsvorgänge} sind gemäß der Betriebsanleitung zu beachten. & \textbf{\#3}\\
& Nach & Der \textbf{Anwärmvorgang und der Entwässerungsvorgang} sind gemäß der Betriebsanleitung zu beachten. & \\
\tablevspace
{ \textbf{22}} & Vor & Die \textbf{Reparatur- und Wartungsarbeiten} der Maschine sind ausschließlich von einem Industriemechaniker durchzuführen. & \textbf{\#3}\\
& Nach & Die \textbf{Reparaturarbeiten und Wartungsarbeiten} der Maschine sind ausschließlich von einem Industriemechaniker durchzuführen. & \\
\tablevspace
{ \textbf{23}} & Vor & Die Seite I/O Konfigurierung steht zur detaillierten Parametrierung der \textbf{Ein- und Ausgänge} zur Verfügung. & \textbf{\#4}\\
& Nach & Die Seite I/O Konfigurierung steht zur detaillierten Parametrierung der \textbf{Eingänge und Ausgänge} zur Verfügung. & \\
\tablevspace
{ \textbf{24}} & Vor & Prüfen Sie, ob sich \textbf{Wasser-, Gasrohre} oder stromführende Leitungen im Bohrbereich befinden. & \textbf{\#7}\\
& Nach & Prüfen Sie, ob sich \textbf{Wasserrohre, Gasrohre} oder stromführende Leitungen im Bohrbereich befinden. & \\
%\hhline%%replace by cmidrule{~--~}
\lspbottomrule
\end{longtable}
