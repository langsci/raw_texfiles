\chapter{\label{ch:7}Fazit}


 \marzoukepigram{It does not really matter what kind of ambiguity the system is up against; what matters is whether the system has the relevant data for disambiguation. (\citealt{HutchinsSomers1992}: 94)}

\section{\label{sec:7.1}Schlussfolgerungen}

Die Studie beschäftigt sich mit der Analyse bzw. dem Vergleich des MÜ-Outputs unterschiedlicher MÜ-Ansätze (RBMÜ, SMÜ, HMÜ sowie NMÜ) vor und nach der Anwendung einzelner KS-Regeln hinsichtlich der aufgetretenen Fehler, Stil- und Inhaltsqualität sowie AEM-Scores. Der Vergleich fand auf vier Ebenen (Spra\-chen\-paar-, Regel-, MÜ-System- sowie Regel- und MÜ-Systemebene) statt und zeigte Folgendes:

In Übereinstimmung mit mehreren vorherigen Studien (vgl. \citealt{NybergMitamura1996}; \citealt{Bernth1999}; \citealt{BernthGdaniec2001}: 208; \citealt{Drugan2013}: 98; \citealt{DrewerZiegler2014}: 196; \citealt{Wittkowsky2017}: 92) zeigte die Anwendung der KS-Regeln auf Sprachenpaarebene (d. h. regel- und systemübergreifend) einen signifikanten positiven Einfluss auf den MÜ-Output im Sinne einer verringerten Fehleranzahl, einer erhöhten Stil- und Inhaltsqualität sowie verbesserter Scores zweier AEMs (TERbase und hLEPOR).

Eine genauere Analyse der Auswirkungen der einzelnen Regeln (systemübergreifend) ergab, dass nur die vier Regeln „Für zitierte Oberflächentexte gerade Anführungszeichen verwenden“, „Konstruktionen mit ‚sein + zu + Infinitiv‘ vermeiden“, „Konditionalsätze mit ‚Wenn‘ einleiten“ und „Funktionsverbgefüge vermeiden“ sich positiv auf den MÜ-Output auswirkten (Rückgang der Fehleranzahl und Verbesserung der Stil- und Inhaltsqualität sowie der AEM-Scores). Diese Regeln reduzierten die Ambiguität, vereinfachten die Satzstruktur und ermöglichten damit ein besseres Parsen, was zu einer genaueren, verständlicheren, idiomatischeren und aufmerksamkeitserregenden Übersetzung beitrug. Hingegen zeigten die drei Regeln „Passiv vermeiden“, „Partizipialkonstruktionen vermeiden“ und „Keine Wortteile weglassen“ einen negativen Einfluss auf den MÜ-Output (Anstieg der Fehleranzahl und Verschlechterung der Stil- und Inhaltsqualität sowie der AEM-Scores). Die signifikante Gemeinsamkeit bei allen drei Regeln war der Rückgang der Stilqualität nach der KS-Anwendung. Die Bewerter fanden das Passiv in einigen Fällen stilistisch adäquater, die Partizipialkonstruktion idiomatischer und die Formulierung mit Wortteilen zum Teil natürlicher. Das Ergebnis reflektiert, dass diese drei Regeln auf stilistischer Ebene nachteilig sein können; dies ist prinzipiell nachvollziehbar, da die KS die Verständlichkeit und nicht den Stil im Fokus hat. Gleichzeitig weist das Ergebnis auf einen gewissen Fortschritt bei der MÜ des Passivs, der Partizipialkonstruktion und der analysierten Form der Ellipsen hin, wodurch die Nicht-Anwendung der Regeln mit einer höheren Inhaltsqualität (signifikant im Falle der Regel „Passiv vermeiden“; nicht signifikant bei den anderen zwei Regeln) und somit einer höheren Genauigkeit bzw. Verständlichkeit -- neben der besseren Stilqualität -- verbunden war.

Für die letzten zwei Regeln „Eindeutige pronominale Bezüge verwenden“ und „Überflüssige Präfixe vermeiden“ wurden keine signifikanten Auswirkungen festgestellt. Bezüglich der Regel „Eindeutige pronominale Bezüge verwenden“ zeigt die qualitative Analyse der MÜ -- in Übereinstimmung mit \citet{BernthGdaniec2001} --, dass die Entscheidung, ein Pronomen zu verwenden oder es durch seine Referenz zu ersetzen, auf einer Fall-zu-Fall-Basis getroffen wurde. So wurde eine Wiederholung der pronominalen Referenz empfohlen, nur wenn die Pronomen mehrdeutig waren, was die insignifikanten Ergebnisse dieser Regel begründet. Bei der Regel „Überflüssige Präfixe vermeiden“ waren die Ergebnisse insignifikant, da mehr als zwei Drittel der Sätze vor und nach der Regelanwendung korrekt und identisch übersetzt wurden, was zu einer vergleichbaren MÜ-Qualität führte. Trotz der insignifikanten Qualitätsveränderungen zeigen die Ergebnisse, dass die Systeme in den meisten Fällen (in 70~\% der Fälle) die Koreferenz auflösen (Regel „Eindeutige pronominale Bezüge verwenden“) und die Verben mit getrennten Präfixen (in 71~\% der Fälle) korrekt parsen (Regel „Überflüssige Präfixe vermeiden“) konnten, was auf einen Fortschritt bei zwei bekannten MÜ-Schwächen hindeutet.

Der Vergleich der Ergebnisse der früheren MÜ-Ansätze mit denen des NMÜ-Ansatzes ergab, dass die früheren MÜ-Systeme in vielen Fällen von den KS-Re\-geln zur Vermeidung unterschiedlicher MÜ-Fehler und zur Verbesserung ihrer Inhaltsqualität (Genauigkeit bzw. Verständlichkeit) sowie Stilqualität (Idiomatik der MÜ, Eignung der MÜ für die Intention des Inhalts bzw. orthografische Darstellung der MÜ) profitierten, während das NMÜ-System die meisten Sätze vor und nach der Anwendung aller Regeln fehlerfrei übersetzen konnte (in 83~\% der Fälle), was mehr als doppelt so hoch wie bei jedem anderen System war. Darüber hinaus verzeichnete das NMÜ-System in beiden Szenarien bei allen Regeln die höchsten Qualitätsratings unter allen Systemen in Bezug auf den Stil und den Inhalt (über 4,3 von 5 Punkten). Zudem war das NMÜ-System das einzige System, bei dem die Stil- und Inhaltsqualität nach der Regelanwendung zurückgingen -- mit einem signifikanten Rückgang der Stilqualität. Der signifikante Rückgang der Stilqualität wurde auf Basis der Humanevaluation mit der höheren Idiomatik und Natürlichkeit der MÜ vor der Regelanwendung begründet. Demzufolge bietet das NMÜ-System die Möglichkeit, für gewünschte Textsorten auf die Anwendung der analysierten KS-Regeln zu verzichten und entsprechend von einem natürlichen Stil zu profitieren.

\section{\label{sec:7.2}Rückblick und Ausblick}

Vorherige Studien sahen in der MÜ eine wichtige Computeranwendung der KS. Die Kernidee ist, dass eine Abstimmung der KS-Regel auf das MÜ-System im Einsatz in einer effizienteren und effektiveren MÜ resultiert. Über diese Idee waren sich mehrere häufig zitierte Arbeiten über die letzten 20 Jahre einig. Dies formulierten \citet[254f.]{NybergEtAl2003} folgendermaßen:
\begin{quote}
MT is potentially one of the most interesting computational applications of CL. If a CL and an MT system are attuned to each other, MT of texts written in that CL can be much more efficient and effective, requiring far less -- or ideally even no -- human intervention. (\citealt{NybergEtAl2003}: 254f.)
\end{quote}

Als \citet{O’Brien2003} acht KS-Regelsätze analysierte, mit dem Ziel anhand der gemeinsamen Regeln einen Kernregelsatz festzulegen, nach dem die Unternehmen arbeiten können, ohne das Rad neu erfinden zu müssen, fand sie die acht Regelsätze weitgehend individuell mit nur einer gemeinsamen Regel. Ihre Begründung dieses Ergebnisses beinhaltete:
\begin{quote}
If source text is destined to be translated by a specific MT system for specific language pairs, then the rules will reflect the inherent weaknesses of the MT system and the known transfer problems between specific language pairs. (\citealt{O’Brien2003}: 7)
\end{quote}

Mit Fokus auf der Beziehung zwischen dem Pre-Editing und dem Post-Editing führt \citet[481]{Göpferich2007b} das Argument fort -- eine Unterstützung der MÜ-Systeme mithilfe von lexikalischen und syntaktischen Regeln reduziere den Post-Editing-Aufwand:
\begin{quote}
Texte können nur dann mit geringem Nachbearbeitungsaufwand (Post-Ed\-it\-ing) wirtschaftlich maschinell übersetzt werden, wenn der in ihnen verwendete Wortschatz (Lexik) und Satzbau (Syntax) sich im Rahmen dessen bewegt, was das maschinelle Übersetzungssystem analysieren kann. Bei der Abfassung maschinell zu übersetzender Texte sind also den Wortschatz und den Satzbau beschränkende Regeln zu beachten (Pre-Editing), was ebenfalls bereits als Standardisierung bezeichnet werden kann. \citep[481]{Göpferich2007b}
\end{quote}

Bis 2017 ist die gleiche Schlussfolgerung zu finden. Sie wird von \citet{Wittkowsky2017} in ähnlichen Worten vorgebracht:
\begin{quote}
Damit Ausgangstexte dann auch maschinell gut übersetzbar sind, gilt es provokativ gesprochen eigentlich nur noch, die Regeln einzuhalten und auch das Übersetzungssystem speziell auf die Sprachverwendung abzustimmen. \citep[93]{Wittkowsky2017}
\end{quote}

Die oben zitierten Studien bezogen sich auf Systeme der RBMÜ-, SMÜ- und HMÜ-Ansätze. Die Anwendung von KS-Regeln, die gezielt auf das verwendete System abgestimmt sind, war der Weg, der eingeschlagen wurde, um die Systemschwächen auszugleichen und auf diese Weise die maschinelle Übersetzbarkeit erhöhen zu können. In der vorliegenden Studie wurden die fünf Systeme im Ist-Zustand, d. h. ohne Abstimmung zwischen den analysierten Regeln und den Systemen und ohne vorheriges Training mit Korpora, die auf das Testmaterial abgestimmt sind, untersucht.

Die Untersuchung der Auswirkungen der einzelnen Regeln auf MÜ-Sys\-tem\-ebe\-ne ergab, dass diese Auswirkungen bei den früheren MÜ-Ansätzen (RBMÜ-, SMÜ- und HMÜ-Systemen) von einem Ansatz zum anderen je nach Stärken und Schwächen des jeweiligen Systems unterschiedlich -- signifikant -- positiv und negativ waren. Es bleibt daher die Notwendigkeit bestehen, die wirksamen Regeln in jedem Implementierungskontext (Sprachenpaar, Übersetzungsrichtung, Domäne und MÜ-Ansatz bzw. -System) zu identifizieren und sie gezielt auf diesen Kontext abzustimmen.

Eine Restriktion durch die analysierten KS-Regeln konnte hingegen mithilfe des untersuchten NMÜ-Systems vermieden werden, denn das NMÜ-System Google Translate war in der Lage, die meisten Sätze vor und nach der Anwendung aller Regeln fehlerfrei und mit den höchsten Qualitätswerten unter allen Systemen -- ohne vorheriges Training mit untersuchungsrelevanten Korpora oder vorherige Abstimmung auf die analysierten Regeln -- zu übersetzen. Überdies ging die Inhaltsqualität leicht und die Stilqualität signifikant nach der Anwendung der analysierten Regeln zurück, was nach der Humanevaluation auf die höhere Idiomatik und Natürlichkeit der MÜ vor KS zurückgeführt wurde. Außerdem nahm die Fehleranzahl nach der KS-Anwendung insignifikant zu.

Die Analyse im Rahmen dieser Studie erfolgte auf Satzebene, somit beschränkt sie sich auf satzrelevante KS-Regeln. Eine Analyse in einem größeren Rahmen ist ausdrücklich erwünscht, um den Einfluss von kontextrelevanten KS-Regeln (d.~h. Regeln, die mehrere Sätze betreffen) auf den MÜ-Output zu untersuchen. Die kontextbezogene MÜ bzw. die MÜ auf Dokumentebene zählt zu den bekannten herausfordernden Zielen der MÜ (\citealt{ZhangZong2020}). Inwiefern die KS eine kontextbezogene MÜ unterstützen kann, ist eine Frage, die sich empirisch beantworten lässt. Nach dem aktuellen Stand der NMÜ-Forschung wurden bereits im Bereich der kontextbezogenen MÜ bzw. der MÜ auf Dokumentebene Fortschritte realisiert, und das sogar in der Literaturübersetzung; einer Domäne, die als besonders herausfordernd für die MÜ gilt (vgl. \citealt{ToralWay2018}; \citealt{Matusov2019}). Ferner entwickelten mehrere Studien kontextfähige NMÜ-Modelle sowie Strategien zur NMÜ auf Dokumentebene, mit denen klassische MÜ-Schwierigkeiten, wie Deixis, Ellipsen, Koreferenzauflösung, Kohärenz und lexikalische Kohäsion, bewältigt werden konnten (\citealt{MüllerEtAl2018}; \citealt{StojanovskiFraser2018}; \citealt{VoitaEtAl2018}; \citealt{StojanovskiFraser2019}; \citealt{VoitaEtAl2019}). Auf Basis der Ergebnisse der vorliegenden Studie sowie der geschilderten rapiden Entwicklungsfortschritte der NMÜ (siehe auch \sectref{sec:3.2.4}) ist es zu erwarten, dass eine Anwendung der KS zum Zwecke der maschinellen Übersetzbarkeit in naher Zukunft zunehmend in den Hintergrund gedrängt wird. Sollte dies der Fall sein, können die Unternehmen die KS gezielt für die weiteren Zwecke -- wie die Konsistenz, Lesbarkeit von umfangreichen Dokumenten und Verständlichkeit von komplexen Inhalten -- anwenden und sich auf die tatsächlich dafür erforderlichen Regeln beschränken.

Wenn der Einsatz umfangreicher KS-Regelsätze mit verringerter Regel-Usabi\-li\-ty und Autorenproduktivität \citep{Mitamura1999}, übermäßigem Intervenieren und Erschwerung des Schreibprozesses (\citealt{O’BrienRoturier2007}), Zeit- und Kostenaufwand -- trotz  der Verwendung von CL-Checkern (\citealt{Govyaerts1996}; \citealt{O’BrienRoturier2007}), verminderter Textakzeptabilität \citep{Roturier2006} und unnatürlichem Stil (\citealt{LehrndorferReuther2008}: 112f.) verbunden sein kann, ist es ersichtlich, dass \textit{eine Reduzierung des KS-Einsatzes auf das Wesentliche folgende Implikationen hätte:}

Zeit- und Kostenersparnisse: In der Praxis übersetzen die Unternehmen ihre Dokumentation in der Regel in mehrere Sprachen. Um den verschiedenen Transferproblemen entgegenwirken zu können, müssen für Zwecke der maschinellen Übersetzbarkeit für jedes Sprachenpaar die adäquaten KS-Regeln angewendet werden (vgl. \citealt{O’Brien2003}). Sollte aber ein robustes NMÜ-System dem Unternehmen eine qualitative Übersetzung in verschiedene Sprachen unter Anwendung von (deutlich) weniger KS-Regeln ermöglichen, würde dies mit Zeit- und Kostenersparnissen einhergehen.

Vereinfachung des Schreibprozesses und ggf. Schaffung eines Raums für Kreativität: Die Einhaltung der KS-Regeln erfordert in manchen Fällen eine vollständige Umformulierung der Sätze. Aus Sicht der Autoren kann die Einhaltung der KS-Regeln komplex, kreativitätshemmend und zeitaufwendig ausfallen. In der Praxis wird außerdem die technische Dokumentation von technischen Redakteuren und nicht selten auch von den Fachabteilungen bzw. Fachexperten durchgeführt, die über begrenztes linguistisches Wissen zu dem Verstehen und der Umsetzung aller Regeln verfügen (\citealt{VanderEijkEtAl1996}; \citealt{AranberriRoturier2009}). Für beide Gruppen ist die Einschränkung der angewendeten Regeln von Vorteil, denn sie würde zu einer Erhöhung der Autorenproduktivität beitragen. Technische Redakteure, die sich durch die Regeleinhaltung in ihrer Kreativität gebremst fühlen (vgl. \citealt{NybergEtAl2003}: 249; \citealt{LehrndorferReuther2008}: 112; \citealt{DrewerZiegler2014}: 209), hätten ggf. mehr Raum für Kreativität. Diesen Raum kann das Unternehmen anhand des CL-Checkers je nach Textsorte, Natur des verfassten Dokuments und Autor bzw. Abteilung steuern. Beide Effekte (Vereinfachung des Schreibprozesses und Schaffung eines Kreativitätsraums) würden im Endeffekt zur Erhöhung der Autorenmotivation beitragen.

Natürlicherer Stil und erhöhte Textakzeptabilität: Die gesteigerte Motivation und Kreativität des Autors würden sich in einem natürlichen Stil widerspiegeln. Die Konsistenz ist zwar herkömmlich in der technischen Dokumentation, gleichzeitig empfiehlt \citet[335f.]{Püschel1996} den technischen Redakteuren, „den Text so abwechslungsreich wie möglich zu machen“, denn „auch ein Stilbruch kann die Aufmerksamkeit wecken“. Wie die Ergebnisse des Einflusses der KS auf den NMÜ-Outputs zeigten, wurden MÜ, die vor und nach der KS-Anwendung fehlerfrei waren, vor der KS-Anwendung als natürlicher bewertet, was wiederum die Akzeptanz des Texts durch die Rezipienten erhöhen würde. Je nach Textsorte und Dokumentationsziel kann ein natürlicher Stil erwünscht sein, z. B. für technische Dokumentationen, die Produkte beschreiben.

Gleichzeitig ist Folgendes zu berücksichtigen: Da die deutsche KS primär die Verständlichkeit des Ausgangstexts und nicht seinen Stil zum Ziel hat, steht das Unternehmen bei der Entscheidung über die Anwendung bzw. Nicht-Anwen\-dung einer Regel angesichts der erzielten Ergebnisse vor einem gewissen Trade-off zwischen der Maximierung der Verständlichkeit des Ausgangstexts und der Schaffung von Raum für natürlichen Stil. Diese Entscheidung lässt sich fallspezifisch je nach Textsorte und -ziel treffen. Die NMÜ ebnet den Weg für einen natürlichen Stil kombiniert mit einer hohen Genauigkeit und Verständlichkeit des Zieltexts. Es bleibt dem Unternehmen überlassen, wann es diesen Weg geht und davon profitiert.

\textit{Die genannten potenziellen Vorteile der Einschränkung des KS-Einsatzes auf für vordefinierte Ziele festgelegte Regeln sowie die erzielten Ergebnisse des NMÜ-Systems hätten für den technischen Redakteur, den Übersetzer sowie für die Ausbildung von beiden folgende Implikationen:}

Je nach Unternehmen bzw. Verantwortungsaufteilung spielen die \textit{technischen Redakteure} unterschiedliche Rollen bei dem Pre-Editing bzw. Einsatz von KS-Regeln. Dazu zählen die Definition der Ziele des KS-Einsatzes, die Auswahl der dafür erforderlichen Regeln sowie die Durchführung des Pre-Editing. Darüber hinaus trifft der technische Redakteur die Entscheidung über Texte, die sich für MÜ -- mit einem Kostensparpotenzial -- eignen. Für diese Texte kann er wiederum die Textsorten differenziert behandeln und KS-Regeln, die dem Ziel des Texts dienlich sind, anwenden. Haben die Verständlichkeit bzw. die Lesbarkeit Priorität, können die entsprechenden Regeln angewendet werden. Soll der Stil priorisiert werden, zeigen die Ergebnisse der analysierten Regeln die Tendenz zu einer möglichen Deregulierung, die mit einer hohen Stil- und Inhaltsqualität der MÜ einhergeht.

Bei dem \textit{Übersetzer} kann ebenfalls zwischen Textsorten bzw. Dokumentationsarten, die für eine MÜ geeignet und anderen, die nicht geeignet sind, unterschieden werden. Bei Ersterem bleibt die MÜ als ein Tool, das nur von qualifizierten Übersetzern sinnvoll eingesetzt werden kann. Daher muss der Fortschritt dieses Tools mit einer Weiterentwicklung der Übersetzerqualifikationen einhergehen. Für die zurzeit bekannten Schwächen der NMÜ-Systeme, dass sie in manchen Fällen zwar sehr flüssigen aber ungenauen bzw. unvollständigen Output liefern, ist ein geschultes Auge gefragt. Die Post-Editing-Aufgabe muss daher tiefgehend durchgeführt werden, um diese Art von Genauigkeitsfehlern aufzudecken (vgl. \citealt{Volk2018}). Gleichzeitig darf nicht unerwähnt bleiben, dass die NMÜ-Systeme von den übersetzten Texten lernen und sich weiter verbessern. Der Verbesserungsgrad wird jedoch von einer Domäne zur anderen bzw. von einem Sprachenpaar zum anderen und somit der Umfang des Post-Editing variieren. Unerlässlich bleibt die Rolle der Übersetzer bei Texten, die weniger für die MÜ geeignet sind, nämlich Texte, die Kreativität erfordern bzw. meist nicht standardisiert sein können (z.~B. Marketing-Texte oder Kundenkorrespondenz). Für die Weiterentwicklung der NMÜ wird außerdem der Humanübersetzer benötigt: Nicht für alle Sprachenpaare existieren ausreichende bzw. qualitative Trainingsdaten. Das gilt nicht nur für seltene Sprachen, sondern sogar für Sprachen, die von vielen Völkern gesprochen werden, wie z. B. Arabisch und Russisch. Für die Erstellung dieser bilingualen Trainingsdaten sind qualitative Humanübersetzungen erforderlich.\footnote{{{{Wohlgemerkt, läuft die Weiterentwicklung der sog. „semi-supervised“ bzw. „unsupervised“ NMÜ-Modelle, die (überwiegend) mit massiven monolingualen Trainingsdaten arbeiten, auf Hochtouren (vgl. \citealt{ZhangZong2020}).}}}} Zudem werden die Fachkenntnisse und Kompetenzen der Übersetzer vermehrt zur Unterstützung von MÜ-Entwicklern bei der laufenden Optimierung der Systeme sowie von den MÜ-Forschern bei der Evaluation des MÜ-Outputs gebraucht. In der \textit{Industrie} werden die Unternehmen zunehmend ihre Übersetzungsprozesse und -workflows optimieren. Auch hierfür ist das translationstechnologische Wissen der Übersetzer bei der Gestaltung der Workflows sowie bei der Integration der Terminologiearbeit hilfreich und ausdrücklich erwünscht.

Für die \textit{Didaktik} lenken die Ergebnisse die Aufmerksamkeit auf die Notwendigkeit der Weiterentwicklung der Bildungsinhalte, um mit dem gegenwärtigen Fortschritt in den Translationstechnologien und seiner Auswirkung auf die technische Dokumentation sowie den Übersetzungsprozess mithalten zu können. Mit dem Fortschritt der NMÜ steht fest, dass die NMÜ-Systeme durch das stetig wachsende Textvolumen weiter lernen und sich verbessern werden. Es liegt daher auf der Hand, dass der Übersetzungsmarkt zunehmend von der NMÜ dominiert wird. Für diese Entwicklung müssen die technischen Redakteure und Übersetzer ausgerüstet werden. Bei beiden Zielgruppen sind neben den klassischen Aufgabenfeldern die Gestaltung, der Umgang und die Unterstützung automatisierter Übersetzungsprozesse zunehmend wesentlich für ihren Beruf. So gehören redaktions- und translationstechnisches Fachwissen sowie redaktions- und translationstechnologische Kenntnisse und Fähigkeiten neben dem Basiswissen in der technischen Redaktion und Translation zu einem marktadäquaten bzw. realitätsabbildenden Curriculum. Konkret sind Lerninhalte zu den folgenden Themenbereichen bzw. Fragestellungen zu empfehlen: Wie der Übersetzungsprozess effizient konzipiert werden kann; in welchen Szenarien Pre- bzw. Post-Editing erforderlich ist; was mit dem Pre- und Post-Editing in Zusammenhang mit der NMÜ zu beachten ist; welches Maß an Sprachkontrolle notwendig ist; in welchen Fällen die Sprachkontrolle überflüssig sein kann; wie das maschinell übersetzungsgerechte Schreiben von der Entwicklung der NMÜ beeinflusst wird; wie NMÜ-Systeme effektiv in den Unternehmensworkflow integriert werden können; wie die NMÜ-Systeme lernen bzw. trainieren und sich verbessern können. Des Weiteren bleibt die Rolle des Terminologiemanagements wesentlich für die (N)MÜ. Wie die Ergebnisse zeigten, war es die Terminologie, die dem analysierten generischen NMÜ-System in den meisten Fällen fehlte, um eine fehlerfreie MÜ zu produzieren. Daher gehören das Terminologiemanagement und seine Integration in den Übersetzungsworkflow zu einem praxisorientierten Curriculum.

\section{\label{sec:7.3}Beitrag und Einschränkungen der Studie sowie zukünftige Forschung}

%\textit{Beitrag der Studie}

Die Studie untersucht KS-Regeln der deutschen Sprache -- einer Sprache, bei der ein Mangel an empirischen Untersuchungen im Bereich der KS vorliegt. Dabei werden die Auswirkungen der einzelnen Regeln auf den MÜ-Output verschiedener MÜ-Ansätze (RBMÜ, SMÜ, HMÜ und NMÜ) analysiert. Somit deckt die Studie den jüngsten MÜ-Ansatz (NMÜ) ab und vergleicht ihn mit früheren Ansätzen im Kontext der KS. Die vorherigen Studien zu den Auswirkungen der KS auf die MÜ-Qualität beziehen sich auf die früheren MÜ-Ansätze (RBMÜ, SMÜ und HMÜ). Der aktuellste Ansatz der NMÜ wurde nach bestem Wissen noch nicht im Kontext der KS untersucht. Angesichts der hervorragenden Qualität des MÜ-Outputs neuronaler Systeme (vgl. \citealt{BentivogliEtAl2016}; \citealt{WuEtAl2016}; \citealt{CastilhoEtAl2017b}; \citealt{ToralSanchez-Cartagena2017}; \citealt{Popović2018}), ist es an der Zeit, dass die KS-Community die Auswirkung der KS-Regeln auf die MÜ wieder aufgreift und überprüft. Dies realisierte die Studie anhand eines Mixed-Methods-Triangulationsansatzes auf Basis von drei bewährten Evaluationsmethoden: Fehlerannotation, Humanevaluation und automatischer Evaluation. Die systematische Dokumentation aller Details der methodischen Vorgehensweise der einzelnen Analysen zusammen mit der Begründung sämtlicher getroffenen Entscheidungen ermöglichen eine Replikation der Studie. Anders als der bisher oft belegte allgemeine positive Effekt der KS auf die MÜ zeigt die Studie, dass nicht alle analysierten KS-Regeln zur Verbesserung der MÜ erforderlich sind. Bei den früheren MÜ-Ansätzen sind die Regeln reduzierbar und bei dem NMÜ-Ansatz überflüssig und wirken sich negativ auf den Stil aus.

Das \textit{Design der Humanevaluation} stellt einen weiteren Beitrag der vorliegenden Studie dar. Im Testdesign wurden die Qualitätsdefinitionen in die Qualitätskriterien integriert (Abschnitt [3a] und [3b] in \figref{fig:4:8} unter \sectref{sec:4.4.5.2} „Darstellung der Evaluation“), d. h. nicht nur in Form von Definitionen am Anfang des Tests zur Verfügung gestellt, wie es in Evaluationsstudien typisch ist. Somit hatten alle Teilnehmer eine direkte und einheitliche Basis für die Vergabe der Qualität-Scores, die sicherlich zu den hohen Intrarater- und Interrater-Agreements beitrug. Das Testdesign sieht eine methodeninterne Triangulation vor. Die Teilnehmer wurden aufgefordert, die zutreffenden Qualitätskriterien anzukreuzen (Abschnitt [3a] und [3b] in \figref{fig:4:8}), zu kommentieren (Abschnitt [4]) bzw. posteditieren (Abschnitt [5]) und anschließend einen Score zu vergeben (Abschnitt [2]). Diese methodeninterne Triangulation fördert die interne Konsistenz, die Zuverlässigkeit und eine genaue Interpretation der Daten. Ferner ist das Testdesign replizierbar.

\textit{Die folgenden Einschränkungen sind jedoch zu erwähnen:}

Da der Fokus überwiegend auf dem Vergleich der KS-Auswirkung auf den Systemoutput der unterschiedlichen MÜ-Ansätze und insbesondere auf einer Gegenüberstellung mit dem jüngsten Ansatz der NMÜ lag, wurden fünf MÜ-Systeme untersucht und dabei die Variablen Sprachenpaar, Übersetzungsrichtung und Domäne für eine einheitliche Vergleichsbasis \textit{konstant} gehalten. Mit diesen Dimensionen und anhand eines angemessen großen Datensatzes stand zudem der Studienumfang mit den zur Verfügung stehenden Zeit- und Finanzressourcen im Verhältnis.

Es wurde nur \textit{ein Sprachenpaar und eine Übersetzungsrichtung} untersucht, dabei lag der Fokus auf der deutschen Sprache (als Ausgangssprache), die für international agierende Unternehmen aus deutschsprachigen Ländern von großer Relevanz ist. Für diese Unternehmen stellt Englisch eine bedeutsame Zielsprache dar, in der ein großes Dokumentationsvolumen für englischsprachige EU-Länder gemäß der Maschinenrichtlinie 2006/42/EG sowie für weitere große internationale englischsprachige Märkte zur Verfügung gestellt werden müssen.

Da die \textit{technische Domäne} den Hauptanwendungsbereich für die KS darstellt und die analysierten tekom-Regeln für die technische Dokumentation bestimmt sind, wurden nur technische Texte in der Studie analysiert. Der gebildete Korpus umfasst jedoch zehn technische Dokumente für verschiedene Produkte unterschiedlicher Hersteller.

Aus zwei Gründen wurden nur \textit{neun KS-Regeln} analysiert: Erstens, es wurden nur die Regeln, die bestimmte Kriterien zur Untersuchung im Rahmen einer Black-Box-Analyse erfüllten (siehe \sectref{sec:4.4.2.1}), ausgewählt. Zweitens, es gab weitere Regeln, die die festgelegten Kriterien erfüllten, allerdings beinhaltete der Korpus nicht genug Verstöße gegen diese Regeln. Für eine ausgewogene quantitative und qualitative Analyse legte die Forscherin Wert darauf, die gleiche Anzahl an Sätzen bei allen Regeln zu untersuchen. Bei dem Einsatz der KS-Regeln wurde nur ein \textit{Umsetzungsmuster} pro Regel angewandt, um die Anzahl der unabhängigen Variablen im Rahmen zu halten. Dieser Rahmen ermöglichte in der Humanevaluation die komplette Bewertung von 1.100 Sätzen durch jeden Teilnehmer. Eine Erweiterung der Anzahl der Regeln oder der Umsetzungsmuster würde eine Erhöhung der Anzahl der Teilnehmer und eine Aufteilung der Sätze auf mehrere Teilnehmer erfordern. Der Forscherin war es jedoch ein Anliegen, dass die 1.100 Sätze komplett von allen Teilnehmern gleichermaßen bewertet werden, um einem potenziellen negativen Einfluss auf das Agreement vorzubeugen.

Die \textit{Anzahl der Ausgangssätze} war zwar nicht hoch, jedoch wurden sie von fünf verschiedenen MÜ-Systemen übersetzt. Somit war letztendlich die Anzahl der analysierten MÜ-Sätze ziemlich hoch und diese wurden von acht Teilnehmern bei der Humanevaluation bewertet. Aus Realisierbarkeitsgründen und wie das Feedback der Teilnehmer zeigte, wäre es schwierig gewesen, Teilnehmer zu finden, die bereit gewesen wären, mehr (als 1.100) Sätze zu bewerten. Schließlich wird in MÜ-Evaluationsstudien wie von Fiederer und \citet[59]{O’Brien2009} darauf hingewiesen, wie wichtig es ist, einen Kompromiss zwischen der Größe des Datensatzes und der Integrität der Ergebnisse zu finden; so begrenzten Fiederer und O’Brien die Anzahl der bewerteten Sätze pro Teilnehmer (auf 180 Sätze), um das „risk of boredom and its negative consequences“ zu vermeiden (ebd.).

Eine potenzielle Schwäche der Studie könnte in der \textit{Aufbereitung der Testsuite} liegen. Die vorgenommenen Aufbereitungsschritte waren für eine Black-Box-Analyse zur Untersuchung des Effekts jeder einzelnen Regel unerlässlich. Die Aufbereitung erfolgte nach klar definierten Kriterien, gefolgt von Schritten zur Qualitätsprüfung (siehe \sectref{sec:4.4.3}), mit dem Ziel sicherzustellen, dass die Sätze so natürlich wie möglich bleiben und nicht gezielt vereinfacht werden (wie es an der Fehleranzahl erkennbar ist). Eine Glas-Box-Analyse würde eine wesentliche Reduzierung der Aufbereitungsschritte ermöglichen, jedoch wäre sie mit den online zugänglichen untersuchten Systemen schwer realisierbar. Stattdessen wurde das Ziel der Studie mithilfe der Definition der KS-Stelle erreicht (siehe \sectref{sec:4.4.2.1}). Letztendlich lag der Fokus der Analyse primär darauf, \textit{ob} Fehler innerhalb der KS-Stelle auftraten bzw. behoben wurden und \textit{wie} die KS-Stelle qualitativ zu bewerten war; und nicht daran, \textit{warum} genau der Fehler auftrat bzw. behoben wurde.

Die Forscherin testete gezielt \textit{generische MÜ-Systeme und ohne vorheriges Training mit Korpora, die auf das Testmaterial abgestimmt sind}, um die Kapazitäten der Systeme im Ist-Zustand zu untersuchen. Hätte die Studie die Systeme vor der Untersuchung trainiert, wären die Ergebnisse von den Trainingsdaten abhängig (d. h. bessere Ergebnisse beim Kontrollierten Szenario, wenn die Korpora kontrolliert sind, und umgekehrt). Außerdem wäre ein angemessener Vergleich der Ergebnisse der verschiedenen Systeme nicht realisierbar, denn die Systeme, die basierend auf Trainingskorpora arbeiten, hätten je nach Kontrollgrad einen Vorteil bzw. Nachteil gegenüber dem regelbasierten System.

\largerpage
\textit{Dementsprechend sind folgende Forschungsaspekte erstrebenswert:}

Es wäre von großem Interesse, die Studie für weitere Sprachenpaare und Textsorten sowie mehr KS-Regeln mit unterschiedlichen Umsetzungsmustern zu replizieren.

Bei der Untersuchung weiterer Sprachenpaare ist ferner der Grad der „\textit{Configurationality}“ jeder Sprache, d. h. der Grad der Flexibilität bei der Wortstellung, zu berücksichtigen. In dieser Studie wurde das Sprachenpaar Deutsch > Englisch unter die Lupe genommen. Englisch -- im Gegensatz zum Deutschen -- zeichnet sich durch eine ziemlich starre Wortstellung aus (vgl. \citealt{Hawkins1986}: 41f.); Deutsch weist hingegen eine relativ hohe Wortstellungsfreiheit auf und ist entsprechend komplexer zu kontrollieren. Der Grad der „Configurationality“ kann eine wesentliche Rolle bei der Analyse der KS-Wirkung spielen, daher ist sein Einfluss untersuchenswert.

Zur Zeit der Durchführung der empirischen Studie dieser Arbeit (Ende 2016 – Anfang 2017) stand das NMÜ-System Google Translate als erstes NMÜ-System an der Spitze. In der Zwischenzeit wurden weitere NMÜ-Systeme entwickelt (wie z. B. das deutsche NMÜ-System DeepL)\footnote{\textrm{Das NMÜ-System DeepL findet sich unter:} \url{https://www.deepl.com/translator}} sowie ältere Systeme zu einem NMÜ-System weiterentwickelt, darunter Systran und SDL.\footnote{\textrm{Die neuronalen Systeme von Systran, SDL und Bing kamen 2017 bzw. 2018 auf den Markt:
\url{http://www.systransoft.com/systran/translation-technology/pure-neural-machine-translation};
\url{https://www.sdl.com/de/about/news-media/press /2018/sdls-neural-machine-translation-sets-new-industry-standards-with-state-of-the-art-dictionary-and-image-translation-features.html};
\url{https://www.microsoft.com/de-de/translator/blog/2018/11/14/nextgennmt} [abgerufen am 16.04.2019]}} Der NMÜ-Ansatz und seine Hybridisation mit anderen Ansätzen stellen zweifellos über die kommenden Jahre die Zukunft der MÜ dar. Vor diesem Hintergrund wäre es wertvoll, die Notwendigkeit der KS-Anwendung zu Zwecken der maschinellen Übersetzbarkeit bei mehreren NMÜ-Systemen zu untersuchen. Sollte der NMÜ-Output von mehreren Systemen bei diversen KS-Regeln vor sowie nach dem KS-Einsatz qualitativ vergleichbar sein, würde dies das Ende des KS-Einsatzes für Zwecke der maschinellen Übersetzbarkeit ankündigen.

Die Durchführung einer \textit{Glas-Box-Analyse} (anstatt einer Black-Box-Analyse) ist eine weitere erstrebenswerte Idee für zukünftige Forschung. In der Glas-Box-Analyse liegen die Systemfunktionsweise und -abläufe offen. Dies würde eine Analyse mithilfe eines natürlichen Korpus ermöglichen und somit eine genauere Verfolgung und Erklärung des Einflusses der Regeln auf die MÜ ohne bzw. mit deutlich weniger \textit{Datenaufbereitung}.

Vor dem Hintergrund, dass die KS eine Komponente des \textit{übersetzungsgerechten Schreibens} darstellt, bedarf der Bereich des übersetzungsgerechten Schreibens angesichts der erzielten Ergebnisse eines Umdenkens bzw. einer Revision. Hierbei sind empirische Untersuchungen der unterschiedlichen Regeln des maschinell übersetzungsgerechten Schreibens und ihres aktuellen Effekts auf den NMÜ-Output notwendig.

\section{\label{sec:7.4}Schlusswort: Untersuchung der Sprachkontrolle im Spiegel der MÜ}

Eine Gegenüberstellung der Ergebnisse der älteren Systeme und der des NMÜ-Systems bestätigt die Aussage von \citet[94]{HutchinsSomers1992}: „It does not really matter what kind of ambiguity the system is up against; what matters is whether the system has the relevant data for disambiguation.“ Das NMÜ-System zeigte im Gegensatz zu den älteren Systemen, dass es über die relevanten Daten bzw. die adäquate Technik zu einer erfolgreichen Disambiguierung unabhängig von der Anwendung bzw. Nicht-Anwendung der KS-Regeln verfügt. Verwendet das Unternehmen ein vergleichbares System der früheren Ansätze, kann es von der KS profitieren, (insbesondere) wenn die Regeln speziell auf die Schwächen des sich im Einsatz befindenden Systems abgestimmt sind. Vergleichbare NMÜ-Systeme in Kombination mit Terminologieintegration bzw. -management ohne Anwendung der untersuchten Regeln scheinen hingegen vielversprechend im Sinne einer hohen Stil- und Inhaltsqualität der MÜ zu sein. Die Anwendung der analysierten Regeln zum Zwecke der maschinellen Übersetzbarkeit zeigte sich als nicht erforderlich. Ferner lässt sich angesichts des sprunghaften Fortschritts im Bereich der NMÜ, der sich mittlerweile auf die kontextfähige MÜ bzw. die MÜ auf Dokumentebene erstreckt, antizipieren, dass eine KS-Anwendung zum Zwecke der maschinellen Übersetzbarkeit in naher Zukunft zunehmend in den Hintergrund gedrängt wird.

Zusammenfassend bleibt die zentrale Rolle der KS bei der Etablierung der Unternehmenssprache (Corporate Language) als einem Bestandteil der Unternehmensidentität (Corporate Identity) sowie bei der Standardisierung der technischen Dokumentation in der Ausgangssprache zum Zwecke der Lesbarkeit, Verständlichkeit und Wiederverwendbarkeit unberührt. Sofern die maschinelle Übersetzbarkeit ein Ziel ist, kann aufgrund des MÜ-Fortschritts für gewünschte Textsorten auf die KS verzichtet werden und weiterhin eine hohe -- oder sogar höhere -- Stil- und Inhaltsqualität erzielt werden.
