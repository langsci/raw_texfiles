% \newcommand*{\orcid}{}

\newcounter{Tabelle}
\newcounter{Abbildung}
\newcounter{Beispiel}

\newcommand{\bspref}[1]{Bsp.~\ref{#1}}

\newcommand{\marzoukepigram}[2]{\hfill\parbox{5cm}{
        {\itshape #1}\\
            \hfill#2}
}

\pgfplotsset
{
	marzoukBar/.style={
		ybar,
		legend style={at={(0.5,1.5)},
		anchor=north,legend columns=1},
		xtick=data,
		nodes near coords,
		axis lines*=left,
		bar width=.75cm,
		width=.75\textwidth,
		height=5cm,
		cycle list = {{draw=black,fill=black},}
	}
}

\newcommand{\smbars}[1][]{
    \begin{tikzpicture}
    \tikzset{every node/.style={font=\scriptsize}};
    \begin{axis}[ybar,
                ylabel = {Summe},
                ylabel style={yshift=-.2cm},
                enlarge x limits={.05},
                width  = \textwidth,
                height = .45\textheight,
                axis lines*=left,
                axis on top,
                bar width=8.5,
                bar shift=0pt,
                ymin = 0,
                xtick = {1,2,...,26},
                xticklabels={IN\_OR\_1.v,
                IN\_OR\_1.n,
                IN\_OR\_2.v,
                IN\_OR\_2.n,
                IN\_LX\_3.v,
                IN\_LX\_3.n,
                IN\_LX\_4.v,
                IN\_LX\_4.n,
                IN\_LX\_5.v,
                IN\_LX\_5.n,
                IN\_LX\_6.v,
                IN\_LX\_6.n,
                IN\_GR\_7.v,
                IN\_GR\_7.n,
                IN\_GR\_8.v,
                IN\_GR\_8.n,
                IN\_GR\_9.v,
                IN\_GR\_9.n,
                IN\_GR\_10.v,
                IN\_GR\_10.n,
                IN\_SM\_11.v,
                IN\_SM\_11.n,
                IN\_SM\_12.v,
                IN\_SM\_12.n,
                IN\_SM\_13.v,
                IN\_SM\_13.n},
            x tick label style={rotate=60,anchor=east,font=\scriptsize},
            nodes near coords,
            legend style={at={(0.5,-0.25)},anchor=north},
            legend columns={-1},
            #1
            ]
        \addplot +[shift={(.5,0)},tmnlpone,x filter/.code={\ifodd\coordindex\def\pgfmathresult{}\fi}] table {\datatable};
        \addplot +[shift={(-.5,0)},tmnlpthree,  x filter/.code={\ifodd\coordindex\relax\else\def\pgfmathresult{}\fi}] table {\datatable};
        \legend{vor KS,nach KS}
    \end{axis}
\end{tikzpicture}
}


\newcommand{\torte}[5][inside]{
    \begin{tikzpicture}
    \pie[radius=2,
        text={#1},
        style={ultra thin},
        color={tmnlpfour,tmnlpthree,tmnlptwo,tmnlpone},
        sum=auto,
%         text=legend,
        rotate=90]{#2/FF ,
                   #3/FR ,
                   #4/RF ,
                    #5/RR} ;
    \end{tikzpicture}
}



\newcommand{\groupedbars}{
    \begin{tikzpicture}
    \begin{axis}[ybar,
                width  = \textwidth,
                height = .45\textheight,
                axis lines*=left,
                axis on top,
                nodes near coords,
                bar width=7pt,
                enlarge x limits=0.25,
                ymin=0,
                legend cell align=left,
                xtick = data,
                symbolic x coords={FF,FR,RF,RR},
                xlabel = {Annotationsgruppe},
                ylabel = {Anzahl},
                ylabel style={yshift=-.2cm},
                cycle list name=langscicolors,
                legend style={
                    at={(0.01,1)},
                    anchor=north west,
                    column sep=1ex,
                    legend columns = 5,
                    font=\small
                    }
                    ]
        \addplot table[x=XX, y=Bing] {\datatable};
        \addplot table[x=XX, y=Google] {\datatable};
        \addplot table[x=XX, y=Lucy] {\datatable};
        \addplot table[x=XX, y=SDL] {\datatable};
        \addplot table[x=XX, y=Systran] {\datatable};
        \legend{Systran, SDL, Lucy, Google, Bing};
    \end{axis}
    \end{tikzpicture}
}


\definecolor{smGreen}{cmyk}{0.1,0,.25,0}
\definecolor{smBlue}{cmyk}{0.15,0.0125,0.0125,0}
\definecolor{smYellow}{cmyk}{0,0.125,.5,0}
\definecolor{smRed}{cmyk}{0.0125,.25,0.3,0}


% Beispiel tables
\newcommand{\bspnote}[1]{\parbox{\textwidth}{\raggedright\footnotesize\noindent{#1}}}
\newcommand{\txblue}[1]{{\color{tmnlpthree}#1}}
\newcommand{\txred}[1]{{\color{lsRed}#1}}
\newcommand{\txgray}[1]{\colorbox{lsLightGray}{#1\strut}}
\newcommand{\txgreen}[1]{\colorbox{smGreen}{#1\strut}}
\newcommand{\boxblue}[1]{\colorbox{smBlue}{#1\strut}}
\newcommand{\bful}[1]{\ul{\textbf{#1}}}
\newcommand{\itul}[1]{\ul{\textit{#1}}}
\newcommand{\bfitul}[1]{\textit{\ul{\textbf{#1}}}}


% accept/reject box, Kap. 4, S. 125ff.
\newcommand{\triplebox}[3]{\noindent\framebox[\textwidth]{\parbox[t]{.3\textwidth}{\raggedright #1}\parbox[t]{.3\textwidth}{\raggedright #2}\parbox[t]{.3\textwidth}{\raggedright #3}}}


\newcommand{\acceptbox}[2]{\noindent\framebox[\textwidth]{\parbox[t]{.45\textwidth}{\raggedright #1}\parbox[t]{.05\textwidth}{\langscicheckmark}\parbox[t]{.45\textwidth}{\raggedright #2}}}

\newcommand{\rejectbox}[2]{\noindent\framebox[\textwidth]{\parbox[t]{.45\textwidth}{\raggedright #1}\parbox[t]{.05\textwidth}{\langscicross}\parbox[t]{.45\textwidth}{\raggedright #2}}}



\renewcommand{\figref}[1]{Abbildung \ref{#1}} %no protected space ~
