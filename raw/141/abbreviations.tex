%\addchap{Abbreviations}
\addchap{Abbreviations and symbols}
%\chapter*{Abbreviations and symbols}


Morphemes throughout this study are glossed using the Leipzig Glossing Rules, with some minor additions to fit the needs of Nyakyusa morphology.\\

\begin{tabular}[h]{ll}
%\textsc{indef.fut}x\=\kill
- & segmentable morpheme boundary\\
= & clitic boundary\\
<> & infix boundary; graphemic representation\\
. & syllable boundary\\
* & ungrammatical form; reconstructed form\\
\# & contextually inadequate\\
? & questionable or only marginally acceptable\\
< & source language\\
// & phonological representation\\
{[]} & phonetic form\\
%\degree & morphological representation\\ %UNBEDINGT WIEDER REIN WENN FEHLER BEHOBEN
$\sim $& reduplication; variation between forms\\
´ & marked rise in pitch\\%\\; unpredictable stress\\
~ ̩ & syllabicity of nasal segment
\end{tabular}
\vspace{\baselineskip}


\begin{multicols}{2} 


\begin{tabular}{lp{4.5cm}} 
1\ldots 18 &noun classes\\
\textsc{1pl} &first person plural\\
\textsc{1sg}&first person singular\\
\textsc{2pl}&second person plural\\
\textsc{2sg}&second person singular\\
\textsc{adj}&deverbal adjective\\
\textsc{agnr}&agent nominalizer\\
\textsc{appl}&applicative\\
\textsc{assoc}&associative\\
\textsc{aug}&augment\\
\textsc{aux}&auxiliary verb\\
\textsc{c} & consonant segment;\\
&coda phase\\
\end{tabular}
\columnbreak

\begin{tabular}{lp{4.5cm}} 
\textsc{caus}&causative\\
\textsc{com}&comitative (\lq with'/\lq and')\\
\textsc{comp}&complementizer\\
\textsc{cmpr}&comparative\\
\textsc{cond}&conditional\\
\textsc{cop}&copula\\
\textsc{de}&German\\
\textsc{dem}&demonstrative\\
\textsc{desdtv}&desiderative\\
\textsc{dist}&distal demonstrative\\
\textsc{en}&English\\
{[\textsc{et}]} & example from elicitation\\
\textsc{fut}&future\\
\end{tabular}

\pagebreak

\begin{tabular}{lp{4.5cm}}
\textsc{fv}&final vowel\\
\textsc{g} & glide segment\\
\textsc{hort} & hortative particle\\
\textsc{imp}&imperative\\
\textsc{indef.fut}&indefinite future\\
\textsc{inf}&infinitive\\
\textsc{intens}&intensifier\\
\textsc{ints}&intensive\\
intr.&intransitive\\
\textsc{interj}&interjection\\
\textsc{ipfv}&imperfective\\
\textsc{itv}&itive\\
\textsc{loc}&locative\\
\textsc{mod.fut}&modal future\\
\textsc{n} & nasal segment;\\
&nucleus phase\\
n/a & not applicable\\
\textsc{narr}&narrative tense\\
\textsc{ncl}&noun class\\
\textsc{neg}&negation\\
\textsc{neut}&neuter (derivation)\\
\textsc{npx}&nominal prefix\\
\textsc{o} & onset phase\\
\textsc{om} & object marker\\
\textsc{part}&partitive\\
\textsc{pass}&passive\\
\textsc{pst}&past\\
\textsc{pb} &Proto-Bantu\\
p.c. &personal communication\\
\textsc{pcu} & perception, cognition,\\
& utterance\\
\textsc{pers}&persistive\\
\textsc{pfv}&perfective\\
\textsc{pl} & plural\\ %weil ja bei posessiva
\textsc{poss}&possessive\\
\textsc{prs}&present\\
\textsc{prog} & progressive aspect\\
\textsc{proh}&prohibitive\\
\textsc{ppx}&pronominal prefix\\
\end{tabular}
\columnbreak

\begin{tabular}{lp{4.5cm}}
\textsc{prox}&proximal demonstrative\\
\textsc{q}&question marker\\
\textsc{recp}&reciprocal\\
\textsc{redupl}&reduplication\\
\textsc{ref}&referential demonstrative\\
%sep	seperative (verb extension)
\textsc{s} & time of speech\\
\textsc{sg}& singular\\ %weil ja bei possessiva
\textsc{sm}&subject marker\\
\textsc{subj}&subjunctive\\
\textsc{subsec}&subsecutive\\
\textsc{swa}&Swahili\\
\textsc{tma}&tense, mood, aspect\\
tr. &transitive\\
\textsc{v} & vowel segment\\
\textsc{vb} &verb base\\
\textsc{wh} &wh-question word
\end{tabular}

 
\end{multicols} 