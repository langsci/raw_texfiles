\chapter{Defective verbs, copulae and movement grams}\label{DefectiveVerbsCopulae}
\section{Introduction}
In this chapter, a number of verbs and verbal constructions will be discussed, beginning with a description of Nyakyusa's two copula verbs (\sectref{Copulae}).\is{copula} The description will include the syntactic and semantic\is{syntax} conditions that govern copula use, copula-based existential constructions and the expression of predicative possession. This is followed by a description of the versatile defective verb \textit{tɪ} \lq say' (\sectref{defectiveti}). Lastly, two verbs of motion that have grammaticalized\is{grammaticalization} into auxiliaries\is{auxiliary} of (figurative) movement\is{motion} will be examined (\sectref{MovementGrams}).

\section{The copulae}
\subsection{Copula verbs}
\label{Copulae}\is{copula|(}
Nyakyusa has two copula verbs: defective \textit{lɪ} \lq be' and \textit{ja} \lq be(come)'. The former must be considered defective because it does not take the default final vowel and only occurs in three paradigms: a zero-marked present,\is{tense!present} the affirmative past (formed with the prefix \textit{a}-) and the negative\is{negative} past\is{tense!past} (formed with the prefixes \textit{ka}-\textit{a}-).

\begin{exe}
\ex
\begin{xlist}
\ex \textit{tʊ}-\textit{lɪ bakafu}\phantom{\textit{ka-a-}} `We are healthy.'
\ex \textit{tw}-\textit{a}-\textit{lɪ bakafu}\phantom{\textit{ka-}} `We were healthy.'
\ex \textit{tʊ}-\textit{ka}-\textit{a}-\textit{lɪ} \textit{bakafu} `We were not healthy.'
\end{xlist}
\end{exe}

The two copulae are in near complementary distribution: in all contexts other than the three mentioned above, \textit{ja} is used.\footnote{Interestingly, the distribution of the two copulae in principle corresponds to the distribution of the reflexes of \ili{Proto-Bantu} \textit{*bá} \lq dwell, be, become' and \textit{*dɪ̀} \lq be' in other languages of the Corridor, among them Ndali \citep[104]{BotneR2008}. \textit{ja} most likely stems from a verb of motion \textit{*gɪ̀} \lq go'; note the contextually triggered loss of the consonantal segmental for both the copula and the motion verb (\sectref{jaAspectualizer}).} This includes the infinitive\is{infinitive} and the \isi{negative} counterpart to the present\is{tense!present} (non-past) copula, which is formed with the \isi{negative} prefix \textit{ka}- and the final vowel -\textit{a}. In this context the consonantal segment often drops out, yielding \textit{kaa} with a long final vowel.\is{vowels!length} Note that \isi{stress} remains on \textit{kaa} and is not shifted to the new penultimate syllable.

\begin{exe}
\ex \textit{tʊkaja bakafu} $\sim$ \textit{tʊkaa bakafu} `We are not healthy.'
\end{exe}

Further, \textit{ja} is used in the present\is{tense!present} and past\is{tense!past} for generic\is{aspect!generic} statements:
\begin{exe}
\ex \gll ɪ-n-gambɪlɪ ɪ-si, boo=bʊ-no \textbf{si}-\textbf{kʊ}-\textbf{j}-\textbf{a}\\
\textsc{aug}-10-monkey \textsc{aug}-\textsc{prox}.10 \textsc{ref.14}=14-\textsc{dem} 10-\textsc{prs}-be(come)-\textsc{fv}\\
\glt `These monkeys, this is how they are!' [Thieving monkeys]
\ex \gll ba-a-kitɪk-aga ɪ-m-banda ɪɪ-nunu ɪ-n-golofu pa-katɪ paa-nyumba, ɪ-j-aa kw-ɪm-a=po ʊ-n-talɪko. lɪnga ɪɪ-nyumba nywamu \textbf{j}-\textbf{aa}-\textbf{j}-\textbf{aga} n=ɪ-m-banda i-bɪlɪ pamo i-tatʊ\\
2-\textsc{pst}-stick\_in\_ground-\textsc{ipfv} \textsc{aug}-9-\textsc{post} \textsc{aug}-good(9) \textsc{aug}-9-straight 16-middle 16-house(9), \textsc{aug}-9-\textsc{assoc} 15-stand/stop-\textsc{fv}=16 \textsc{aug}-3-beam if/when \textsc{aug}-house(9) big(9) 9-\textsc{pst}-be(come)-\textsc{ipfv} \textsc{com}=\textsc{aug}-10-post 10-two or 10-three\\
\glt `They would erect a good straight post in the middle of the house, on which lay the ridge pole.  ‎‎If the house was big, it would have two or three posts.' [Nyakyusa houses of long ago]
\end{exe}

Note that the use of \textit{ja} is obligatory for future\is{tense!future} time reference; that is, the unmarked present of \textit{lɪ} cannot normally be used as a futurate:\is{futurate}

\begin{exe}
\ex[*]{\gll kɪ-laabo a-lɪ kʊ-Tʊkʊjʊ\\
7-tomorrow 1-\textsc{cop}  17-T.\\
\glt (intended: \lq Tomorrow he will be at Tukuyu.')}
\end{exe}

There is one exception, however: copula \textit{lɪ} is licensed with reference to the future if a temporal anchor is introduced by a stressed form of the augmentless class 14 referential demonstrative \textit{bo} (\ref{exCOPliFutBo}); see \sectref{PredicativePosession} for the expression of predicative possession through the use of the copula plus the comitative \textit{na}. Likewise, a zero copula (see \sectref{CopulaUse}) is attested in this environment with reference to a future/hypothetical state-of-affairs (\ref{ex0CopFutBo}). See p.\nobreakspace\pageref{exPFVboFuture} in \sectref{PresentPerfectiveIntroduction} for a comparable case with the present perfective.\is{tense!present}\is{aspect!perfective}
\begin{exe}
\ex \label{exCOPliFutBo}
\gll lɪnga fi-kɪnd-ile ɪ-fy-ɪnja a-ma-longo ma-bɪlɪ, ɪɪ-nyumba sy-osa n-ka-aja a-ka \textbf{bo} \textbf{si}-\textbf{lɪ} n=ʊ-bʊ-meme\\
if/when 8-pass-\textsc{pfv} \textsc{aug}-8-year \textsc{aug}-6-ten 6-two \textsc{aug}-house(10) 10-all 18-12-homestead \textsc{aug}-\textsc{prox}.12 as 10-\textsc{cop} \textsc{com}=\textsc{aug}-14-electricity(<SWA)\\
\glt \lq In twenty years, all houses in this village will have electricity.' [ET]
\ex \label{ex0CopFutBo}
\gll lɪnga mu-sob-iisye, \textbf{bo} lw-ɪnʊ\\
if/when \textsc{2pl}-be\_lost-\textsc{caus.pfv} as 11-\textsc{poss.2pl}\\
\glt \lq If you lose it, grief will be yours.' [Chickens and Crow]
\end{exe}

\subsection{Copula use}\label{CopulaUse}
As described in the previous section, the choice between the two copula verbs \textit{lɪ} and \textit{ja} depends mainly on temporal reference and polarity. In the affirmative present\is{tense!present} (non-past), certain environments further license a zero copula or copulative use of the augmentless substitutives (\sectref{Demonstratives}); see \citet{StassenL2005} for a discussion of the term \textit{zero copula}.

With third person (noun class)\is{noun classes} subjects, nominal predication without any overt linking element is the common case (\ref{exZeroCupulaSG}, \ref{exZeroCupulaPL}). The predicate never carries an augment. An augmentless substitutive may be added, which seems to be related to focus (\ref{exSubstitutiveCopulative}). Note that copulative use of the substitutive also features in cleft sentences; see e.g. (\ref{exRelativeClauseParticipants2}; p.\nobreakspace\pageref{exRelativeClauseParticipants2}) and (\ref{exukutimaelijimo}; p.\nobreakspace\pageref{exukutimaelijimo}). 

\begin{exe}
\ex \begin{xlist}
\ex \label{exZeroCupulaSG}\gll ʊ-m-piki n-nywamu\\
\textsc{aug}-3-tree 3-big\\
\glt `The tree is big.' [ET]
\ex \label{exZeroCupulaPL}\gll ɪ-mi-piki mi-nywamu\\
\textsc{aug}-4-tree 4-big\\
\glt `The trees are big.' [ET]
\ex \label{exSubstitutiveCopulative} \gll ɪ-mi-piki ɪ-gɪ gyo mi-nywamu\\
\textsc{aug}-4-tree \textsc{aug}-\textsc{prox.4} \textsc{ref.4} 4-big\\
\glt \lq These trees, they are big.' [ET]
\end{xlist}
\end{exe}
Associatives and possessives are also normally used without an overt copula:
\largerpage[2]
\begin{exe}
\ex \gll gw-a ba-palamaani\\
3-\textsc{assoc} 2-neighbour\\
\glt \lq It (the tree) is the neighbours'.' [ET]
\ex \gll gw-angʊ\\
3-\textsc{poss.1sg}\\
\glt \lq It (the tree) is mine.' [ET]
\end{exe}

With certain types of predicates, however, the use of a copula verb is obligatory even with noun class\is{noun classes} subjects in the affirmative present.\is{tense!present} In some of these, an augmentless substitutive may replace the copula. First, numerals require either the copula or the augmentless substitutive:
\begin{exe}
\ex
\begin{xlist}
\ex \gll a-ba-ana ba-bɪlɪ\\
\textsc{aug}-2-child 2-two\\
\glt \lq two children' not: \lq The children are two.'
\ex \gll a-ba-ana ba-lɪ ba-bɪlɪ\\
\textsc{aug}-2-child 2-\textsc{cop} 2-two\\
\glt \lq The children are two.'
\ex \gll a-ba-ana bo ba-bɪlɪ\\
\textsc{aug}-2-child \textsc{ref.2} 2-two\\
\glt \lq The children are two.'
\end{xlist}
\end{exe}

Note that this does not hold for the quantifiers \textit{nandɪ} \lq little, few' and \textit{ingi} \lq much, many', which are treated as nominals:
\begin{exe}
\ex \gll ʊ-lw-ɪsi ʊ-lʊ lʊ-sisya. a-m-ɪɪsi ma-tiitʊ kangɪ ɪ-n-gwina \textbf{ny}-\textbf{ingi}\\
\textsc{aug}-11-river \textsc{aug}-\textsc{prox.1} 11-frightening \textsc{aug}-6-water 6-black again \textsc{aug}-10-crocodile 10-many\\
\glt \lq This river is frightening. The water is dark and the crocodiles are many.' [ET]
\ex \gll ʊ-n-tondolo mw-ingi, leelo a-ba-tondol-i \textbf{ba}-\textbf{nandɪ}\\
\textsc{aug}-3-harvest 3-many now/but \textsc{aug}-2-harvest-\textsc{agnr} 2-little\\
\glt \lq The harvest truly is great, but the labourers are few.' (Luke 10: 2)
\end{exe}

When adverbials (\ref{exCOPADV1}, \ref{exCOPADV2}) or \isi{ideophones} (\ref{exCOPIDPH1}, \ref{exCOPIDPH2}) are used predicatively, use of a copula verb is compulsory.

\begin{exe}
\ex \label{exCOPADV1}

\begin{xlist}
\ex[]{\gll ʊ-mw-ana a-lɪ nnoono\\
\textsc{aug}-1-child 1-\textsc{cop} so\_much\\
\glt `The child is too much.'}
\ex[*]{ʊmwana jo nnoono}
\ex[*]{ʊmwana nnoono}
\end{xlist}
\pagebreak

\ex \label{exCOPADV2} \begin{xlist}
\ex[]{\gll ɪ-m-bwa jɪ-lɪ kanunu\\
\textsc{aug}-9-dog 9-\textsc{cop} well\\
\glt `The dog is fine.'}

\ex[*]{ɪmbwa jo kanunu\footnotemark}
\ex[*]{ɪmbwa kanunu}
\end{xlist}

\ex \label{exCOPIDPH1} \begin{xlist}
\ex[]{\gll n-nyumba mu-lɪ kée\\
18-house(9) 18-\textsc{cop} vast\\
\glt `The house is empty.'}

\ex[*]{nnyumba mo kée}
\ex[*]{nnyumba kée}
\end{xlist}


\ex \label{exCOPIDPH2} \begin{xlist}
\ex[]{\gll ɪ-my-enda gɪ-lɪ swée\\
\textsc{aug}-4-cloth 4-\textsc{cop} intense\_white\\
\glt `The clothes are white.'}
\ex[*]{ɪmyenda gyo swée}
\ex[*]{ɪmyenda swée}
\end{xlist}
\end{exe}
\protect\footnotetext{This sentence would be acceptable with the meaning \lq This dog's name is Kanunu.'}

Locative\is{locative} predicates also require a copula verb (\ref{exCOPLocative}). This includes locative question words (\ref{exCopulaKuugu}). (\ref{exCOPLocativeSemantics}) illustrates that the \isi{locative} semantics are responsible for this rather than belonging to one of the locative noun classes.\is{noun classes}

\begin{exe}
\ex \label{exCOPLocative}
\begin{xlist}
\ex[]{\gll ʊ-mw-ana a-lɪ mu-m-piki\\
\textsc{aug}-1-child 1-\textsc{cop} 18-3-tree\\
\glt `The child is in a/the tree.'}
\ex[*]{ʊmwana jo mumpiki}
\ex[*]{ʊmwana mumpiki}
\end{xlist}
\ex \label{exCopulaKuugu}
\begin{xlist}
\ex[]{\gll ʊ-mw-ana a-lɪ kʊʊgʊ \textup{/} pooki \textup{/} mooki?\\
\textsc{aug}-1-child 1-\textsc{cop} where(17) {} where(16) {} where(18)\\
\glt `Where / Wherein is the child?}
\ex[*]{ʊmwana jo kʊʊgʊ \textup{/} pooki \textup{/} mooki?}
\ex[*]{ʊmwana kʊʊgʊ \hphantom{jo }\textup{/} pooki \textup{/} mooki?}
\end{xlist}

\ex\label{exCOPLocativeSemantics}
\begin{xlist}
\ex[]{\gll ʊ-mw-ana a-lɪ kɪfuki n=ʊ-m-piki\\
\textsc{aug}-1-child 1-\textsc{cop} near \textsc{com}=\textsc{aug}-3-tree\\
\glt `The child is near the tree.'}
\ex[*]{ʊmwana jo kɪfuki nʊmpiki}
\ex[*]{ʊmwana kɪfuki nʊmpiki}
\end{xlist}
\end{exe}

With first and second person subjects, the use of either the copula or a substitutive is obligatory for nominal predicates in the affirmative present.\is{tense!present} (\ref{exCopula2sg}--\ref{exCopulaSGNegativeExample}) illustrate this for the second person singular. With other types of predicates, the same regularities as for noun class\is{noun classes} subjects hold.

\begin{exe}
\ex[]{\label{exCopula2sg}\gll (ʊ-gwe) ʊ-lɪ n-nandɪ\\
\hphantom{(}\textsc{aug}-\textsc{2sg} \textsc{2sg}-\textsc{cop} 1-small\\
\glt \lq You are small.'}
\ex[]{\gll (ʊ-gwe) gwe n-nandɪ\\
\hphantom{(}\textsc{aug}-\textsc{2sg} \textsc{2sg} 1-small\\
\glt \lq You are small.'}
\ex[*]{\label{exCopulaSGNegativeExample}\gll ʊ-gwe n-nandɪ\\
\textsc{aug}-\textsc{2sg} 1-small\\}
\end{exe}

Lastly, the copula also forms a compulsory part of existential constructions and expressions of predicative possession, which are the topics of the following sections.

\subsection{Existential construction}\label{Existentials}
The presence or existence of an entity is expressed by a copula plus a \isi{locative} enclitic.\is{enclitic} With the copula \textit{lɪ}, the vowel segment is raised to /i/. Noun class 16 \textit{po} expresses proximity to the deictic centre or more definite locations, class 17 \textit{ko} distance from the deictic centre or general existence and class 18 \textit{mo} inside locations.
\begin{exe}
\ex \label{exExistentialaffirmiert1}
\gll \textbf{ga}-\textbf{a}-\textbf{li}=\textbf{po} a-ma-syabala, \textbf{sy}-\textbf{a}-\textbf{li}=\textbf{po} ɪ-n-jʊgʊ. \textbf{ba}-\textbf{a}-\textbf{li}=\textbf{po} baa-mwembe, \textbf{ga}-\textbf{a}-\textbf{li}=\textbf{po} a-m-ungu. fy-osa \textbf{fy}-\textbf{a}-\textbf{li}=\textbf{po} pa-ka-aja pa-n-gambɪlɪ\\
\textsc{6}-\textsc{pst}-\textsc{cop}=16 \textsc{aug}-6-groundnut 10-\textsc{pst}-\textsc{cop}=16 \textsc{aug}-10-jugo\_bean 2-\textsc{pst}-\textsc{cop}=16 2-mango 6-\textsc{pst}-\textsc{cop}=16 \textsc{aug}-6-pumpkin 8-all 8-\textsc{pst}-\textsc{cop}=16 16-12-home 16-9-monkey\\
\glt \lq There were groundnuts, there were jugo beans. There were mangoes, there were pumpkins. There was all sorts of food at Monkey's' [Monkey and Tortoise] 
\ex \label{exExistentialaffirmiert2}
\gll jɪ-kʊ-tɪ bo m-fw-ile ʊ-ne lɪnga ga-kɪnd-ile a-ma-sikʊ ma-nandɪ.\\
9-\textsc{prs}-\textsc{say} as \textsc{1sg}-die-\textsc{pfv} \textsc{aug}-\textsc{1sg} if/when 6-pass-\textsc{pfv} \textsc{aug}-6-day 6-little\\
\glt \lq It says I'm dead when few days have passed.'
\sn \gll po lɪlɪno n-gʊ-bʊʊk-aga kʊkʊtɪ na=a-ma-jolo kʊ-no ɪ-m-bʊlʊkʊtʊ jɪ-lambaleele kʊ-kʊ-jɪ-bʊʊl-a ʊkʊtɪ \lq\lq ʊ-ne n-gaalɪ \textbf{n}-\textbf{di}=\textbf{ko}, n-dɪ n=ʊ-bʊ-ʊmi''\\
then now/today \textsc{1sg}-\textsc{mod.fut}-go-\textsc{mod.fut} every \textsc{com}=\textsc{aug}-6-evening 17-\textsc{prox} \textsc{aug}-9-ear 9-lie\_down-\textsc{pfv} 17-15-9-tell-\textsc{fv} \textsc{comp} \phantom{\lq\lq}\textsc{aug}-\textsc{1sg} \textsc{1sg}-\textsc{pers} \textsc{1sg}-\textsc{cop}=17 \textsc{1sg}-\textsc{cop} \textsc{com}=\textsc{aug}-14-live\\
\glt `So now I shall go every evening when Ear has laid down, to tell it ``I'm still around, I'm alive.''{}' [Mosquito and Ear]
\end{exe}

When the copula is negated,\is{negative} non-existence or absence is expressed:
\begin{exe}
\ex \gll i-kʊ-suluk-a paa-si mu-m-piki n=ʊ-kʊ-keet-a ʊkʊtɪ kɪ-kapʊ kɪ-mo \textbf{kɪ}-\textbf{ka}-\textbf{j}-\textbf{a}=\textbf{po}\\
1-\textsc{prs}-descend-\textsc{fv} 16-below 18-3-tree \textsc{com}=\textsc{aug}-15-watch-\textsc{fv} \textsc{cmpl} 7-basket 7-one 7-\textsc{neg}-be(come)-\textsc{fv}=16\\
\glt `He climbs down the tree and sees that one basket is missing.' [Elisha Pear Story]
\ex \label{exExistentialNegated2}\gll ka-a-li=ko a-ka-aja ka-mo a-ka a-m-ɪɪsi \textbf{ga}-\textbf{ka}-\textbf{a}-\textbf{li}=\textbf{mo}\\ 
12-\textsc{pst}-\textsc{cop}=17 \textsc{aug}-12-village 12-one \textsc{aug}-\textsc{prox.12} \textsc{aug}-6-water 6-\textsc{neg}-\textsc{pst}-\textsc{cop}=18\\
\glt `There was a village in which there was no water.' [Water and toads]
\end{exe}

When \isi{locative} marking on the copula co-occurs with an overt locative noun phrase, both often reference the same locative noun class (\ref{exExistentialSameNounClasses}),\is{noun classes} but mixing of two locative classes is also found (\ref{exExistentialMixedNounClasses}).

\begin{exe}
\ex \label{exExistentialSameNounClasses} \gll a-ma-keeke \textbf{ga}-\textbf{a}-\textbf{li}=\textbf{mo} m-ingi \textbf{mw}-\textbf{ene} \textbf{mw}-\textbf{i}-\textbf{tengele} ɪ-ly-a n-ky-amba Rungwe\\
\textsc{aug}-6-type\_of\_grass 6-\textsc{pst}-\textsc{cop}=18 6-many 18-only 18-5-bush \textsc{aug}-5-\textsc{assoc} 18-7-mountain R.\\
\glt `There was a lot of a certain type of grass only in the bush on mount Rungwe.' [Nyakyusa houses of long ago]
\ex \label{exExistentialMixedNounClasses}\gll \textbf{n}-\textbf{k}-\textbf{iisʊ} kɪ-mo, \textbf{a}-\textbf{a}-\textbf{li}=\textbf{ko} ʊ-malafyale jʊ-mo. ɪ-n-gamu j-aake a-a-lɪ jo Kapyungu\\
18-7-land 7-one 1-\textsc{pst}-\textsc{cop}=17 \textsc{aug}-chief(1) 1-one \textsc{aug}-9-name 9-\textsc{poss.sg} 1-\textsc{pst}-\textsc{cop} \textsc{ref.1} K.\\
\glt `In some land there was a chief. His name was Kapyungu.' [Chief Kapyungu]
\end{exe}

Note that in examples (\ref{exExistentialaffirmiert1}--\ref{exExistentialMixedNounClasses}) the grammatical subject follows the existential copula. This is a common presentational construction. In fact, (\ref{exExistentialNegated2}, \ref{exExistentialMixedNounClasses}) are typical of the orientation sections of Nyakyusa narratives.\is{narrative}

\subsection{Expression of predicative possession}\label{PredicativePosession}
Ownership is expressed by using the copula together with an enclitic form of the comitative \textit{na} on the possessee.

\begin{exe}
\ex \gll ʊ-malafyale ʊ-jʊ \textbf{a}-\textbf{a}-\textbf{lɪ} \textbf{n}=ɪ-fi-panga fi-tatʊ\\
\textsc{aug}-chief(1) \textsc{aug}-\textsc{prox.1} 1-\textsc{pst}-\textsc{cop} \textsc{com}=\textsc{aug}-8-village 8-three\\
\glt `This chief had three villages.' [Chief Kapyungu]
\ex \label{exPredicativePosession2}\gll lɪnga \textbf{ʊ}-\textbf{ka}-\textbf{j}-\textbf{a} \textbf{n}=ʊ-bʊ-jo bw-a kʊ-gon-a=mo kʊ-gon-a muu-nyumba ɪ-sy-a kʊ-pang-a\\
if/when \textsc{2sg}-\textsc{neg}-be(come)-\textsc{fv} \textsc{com}=\textsc{aug}-14-place 14-\textsc{assoc} 15-sleep-\textsc{fv}=18 \textsc{2sg.prs}-sleep-\textsc{fv} 18-house(9) \textsc{aug}-10-\textsc{assoc} 15-rent-\textsc{fv}\\
\glt `If you do not have a place to sleep in, you sleep in a rented house.' [How to build modern houses]
\end{exe}

Typically, the noun expressing the possessee carries the augment, though use without the augment was also encountered:
\begin{exe}
\ex \gll n-ga-a na=m-bombo, n-ga-a na=heela\\
\textsc{1sg}-\textsc{neg}.be(come)-\textsc{fv} \textsc{com}=9-work \textsc{1sg}-\textsc{neg}.be(come)-\textsc{fv} \textsc{com}=money(10)\\
\glt `I don't have a job, I don't have money.' [overheard]
\end{exe}

The possessee can be referred to by a referential demonstrative without the augment. This is the case with anaphoric reference (\ref{exPossessionReferential}). The referential demonstrative can also be used cataphorically together with the overt noun phase it indexes (\ref{exPossessionBoth}).
\begin{exe}
\ex \label{exPossessionReferential}\gll kangɪ mpaka ʊ-si-keet-e taasi \textbf{ɪ}-\textbf{n}-\textbf{dalama} ɪ-si ʊ-lɪ \textbf{na}=\textbf{syo} muu-ny-ambɪ, pamopeene n=ʊ-tʊ-ndʊ ʊ-tʊ tʊ-bagiile ʊ-kʊ-kʊ-tʊʊl-a kʊ-m-bombo ɪ-jo\\
again no\_matter\_what \textsc{2sg}-10-watch-\textsc{subj} yet \textsc{aug}-10-money \textsc{aug}-\textsc{prox.10} \textsc{2sg}-\textsc{cop} \textsc{com}=\textsc{ref.10} 18-pocket(9) together \textsc{com}=\textsc{aug}-13-thing \textsc{aug}-\textsc{prox.13} 13-be\_able.\textsc{pfv} \textsc{aug}-15-\textsc{2sg}-help-\textsc{fv} 17-9-work \textsc{aug}-\textsc{ref.9}\\
\glt `Again, you should first look at the money which you have in your pocket, together with other things which can help you with this work.' [How to build modern houses]
\ex\label{exPossessionBoth} \gll lɪlɪno tʊ-ka-a \textbf{na}=\textbf{fyo} na=fi-mo \textbf{ɪ}-\textbf{fi}-\textbf{ndʊ} ɪ-fy-a kʊ-ly-a n-nyumba\\
now/today \textsc{1pl}-\textsc{neg}.be(come)-\textsc{fv} \textsc{com}=\textsc{ref.8} \textsc{com}=8-one \textsc{aug}-8-food \textsc{aug}-8-\textsc{assoc} 15-eat-\textsc{fv} 18-house(9)\\
\glt `Today we don't have anything to eat at home.' [Monkey and Tortoise]
\end{exe} 
\is{copula|)}
\section{\textit{tɪ} `say; think; do like'}\label{defectiveti} 
The verb \textit{tɪ} \lq say' must be considered defective for a number of reasons. First, its stem does not carry the final vowel -\textit{a}. Consequently, it does not change its shape in the subjunctive.\is{mood!subjunctive} In other respects, its vocalic segment, however, behaves much like the final vowel of regular verbs: in the imperfective\is{aspect!imperfective} \textit{tɪ} takes the shape \textit{tɪgɪ}, resembling the -VCV shape of the regular imperfective suffix (see \sectref{AlternationsIPFVaga}). Second, the vocalic segment is dropped when perfective\is{aspect!perfective} -\textit{ile} is suffixed, yielding \textit{tile} (not *\textit{tiile}).\is{vowels!length} Last, \textit{tɪ} does not accept any derivational suffixes.

For reasons of space and convenience, \textit{tɪ} is glossed as `say' throughout this study. However, as the following discussion will show, this versatile verb shows uses and functions that go far beyond that of a simple verb of speech. \citet{GueldemannT2000} convincingly argues that the use of \textit{tɪ} as a verb of speech across Bantu has arisen out of a more abstract cataphoric function.

Its use as a verb of speech is illustrated in (\ref{extiVerbOfSpeech}). To render speech or sound with verbs other than \textit{tɪ} itself, a form of \textit{tɪ}, either the infinitive\is{infinitive} (\ref{extiNARRINF}) or an inflected verb in a chaining construction (\ref{extiNARRNARR}), is also required.
 
\begin{exe}
\ex \label{extiVerbOfSpeech}
\gll po \textbf{ly}-\textbf{a}-\textbf{t}-\textbf{ile} \textup{\lq\lq}n-gʊ-lond-a ɪɪ-sindaano j-angʊ\textup{''}\\
then 5-\textsc{pst}-say-\textsc{pfv} \phantom{\lq\lq}\textsc{1sg}-\textsc{prs}-want-\textsc{fv} \textsc{aug}-needle(<SWA)(9) 9-\textsc{poss.1sg}\\
\glt \lq It [Crow] said \lq\lq I want my needle.''{}' [Chickens and Crow]
\ex \label{extiNARRINF} \gll a-lɪnkʊ-ba-bʊʊl-a a-ba-ndʊ \textbf{ʊ}-\textbf{kʊ}-\textbf{tɪ}\\
1-\textsc{narr}-2-tell-\textsc{fv} \textsc{aug}-2-person \textsc{aug}-15-say\\
\glt `He told the people:' [Chief Kapyungu]
\ex\label{extiNARRNARR} \gll kalʊlʊ a-lɪnkw-amul-a kʊ-m-manyaani gw-ake \textbf{a}-\textbf{lɪnkʊ}-\textbf{tɪ}\\
hare(1) 1-\textsc{narr}-answer-\textsc{fv} 17-1-friend 1-\textsc{poss.sg} 1-\textsc{narr}-say\\
\glt `Hare answered to his friend:' [Hare and Chameleon]
\end{exe}

The only attested examples in the corpus of illocutionary verbs of speech without a form of \textit{tɪ} are sections of narratives\is{narrative} that move to drama; see p.\nobreakspace\pageref{Drama} in \sectref{Drama}. Apart from speech in the strict sense, \textit{tɪ} is also used for rendering inner speech or thought:

\begin{exe}
\ex \gll ngɪmba mu-n-dumbula \textbf{i}-\textbf{kʊ}-\textbf{tɪ} \textup{\lq\lq}n-gʊ-tosiisye\textup{''}\\
behold 18-9-heart 1-\textsc{prs}-say \phantom{\lq\lq}\textsc{1sg}-\textsc{2sg}-pay\_back.\textsc{pfv}\\
\glt `But in his heart he [Skunk] is thinking ``I've paid you back.''{}' [Hare and Skunk]
\ex \gll ɪ-m-bwa jɪ-lɪnkʊ-j-a n-galɪ fiijo, \textbf{j}-\textbf{aa}-\textbf{t}-\textbf{ɪgɪ} pamo ɪ-li-paatama li-kʊ-j-a pa-kʊ-pok-a ɪɪ-nyama j-aake\\
\textsc{aug}-9-dog 9-\textsc{narr}-be(come)-\textsc{fv} 9-strict \textsc{intens} 9-\textsc{pst}-say-\textsc{ipfv} perhaps \textsc{aug}-5-cheetah 5-\textsc{prs}-be(come)-\textsc{fv} 16-15-plunder-\textsc{fv} \textsc{aug}-meat(9) 9-\textsc{poss.sg}\\
\glt `The dog became very angry, it was thinking that maybe the cheetah would seize his meat.' [Dogs laughed at each other]
\end{exe}

The verb \textit{tɪ} also serves to introduce ideophones.\is{ideophones} All cases in the data feature onomatopoeia. It is unclear if this is a restriction on the use of \textit{tɪ} or an artefact of the data at hand.

\begin{exe}
\ex\label{extiPFVPFVideoph1}
\gll ʊ-gw-a kɪ-bɪlɪ a-a-lʊ-kol-ile ʊ-lw-igi m-ma-ka ma-tupu. looli j-oope kw-a-lɪl-ile \textbf{kw}-\textbf{a}-\textbf{t}-\textbf{ile} \textup{\lq\lq}káa\textup{''}\\
\textsc{aug}-2-\textsc{assoc} 7-two 1-\textsc{pst}-11-grasp/hold-\textsc{pfv} \textsc{aug}-11-door 18-6-strength 6-sudden but 1-also 17-\textsc{pst}-cry-\textsc{pfv} 17-\textsc{pst}-say-\textsc{pfv} \phantom{\lq\lq}of\_sickle\_swinging\\
\glt `The second one grabbed the door with all his strength. But also with him, there was the sound ``káa!'' [of a sickle swinging]' [Wage of the thieves]
\end{exe}

\label{SubjunctiveTiBule}The subjunctive (\sectref{Subjunctive})\is{mood!subjunctive} of \textit{tɪ} is formed by prefixing the subject prefix,\is{subject marker} without any change in the final vowel (\ref{extiSubjunctive}). When the interrogative \textit{bʊle} `how' follows \textit{tɪ}, they optionally merge into one word, with the vocalic segment of the verb assimilating (\ref{extutubule}). The imperfective\is{aspect!imperfective} suffix in this case is attached to the right of the compound \isi{stem} and accordingly takes the shape -\textit{ege} (\ref{extutubulege}).

\begin{exe}
\ex \label{extiSubjunctive}\gll gw-itɪk-e ʊ-tɪ \textup{\lq\lq}ee, n-di=po\textup{''}\\
\textsc{2sg}-agree-\textsc{subj} \textsc{2sg}-say.\textsc{subj} \phantom{\lq\lq}yes \textsc{1sg}-\textsc{cop}=16\\
\glt \lq You shall answer ``Yes, I'm here.''{}' [Hare and Tugutu] %schoenes BSP, nicht so schlimm, dass teils fuer NARR verwendet
\ex \label{extutubule}\textit{tʊtɪ bʊle?} $\sim$ \textit{tʊtʊbʊle?}
\\ \lq What should we say/do?'
\ex \label{extutubulege}\textit{tʊtɪgɪ bʊle?} $\sim$ \textit{tʊtʊbʊlege?}
\\ \lq What should we be saying/doing?'
\end{exe}

As can be gathered from (\ref{extutubule}, \ref{extutubulege}), apart from introducing (inner) speech and sound, \textit{tɪ} can be understood to have a broader meaning of acting in a certain manner. Thus its subjunctive\is{mood!subjunctive} is also used as a prompt to imitate a certain action (\ref{exTiImitate}) (also cf. \citealt[97]{FelbergK1996}). Also note the related uses in (\ref{exTiWhat}, \ref{exukutimaelijimo}).
\begin{exe}
\ex \label{exTiImitate}
\gll \textbf{ʊ}-\textbf{tɪ} bʊ-no\\
\textsc{2sg}-say.\textsc{subj} 14-\textsc{dem}\\
\glt `Do like this!' [ET]
\ex \label{exTiWhat} \gll po jʊ-la ʊ-gw-a pa-lʊ-ʊlʊ \textbf{a}-\textbf{lɪnkʊ}-\textbf{tɪ} fi-ki, a-lɪnkʊ-kaan-a\\
then 1-\textsc{dist} \textsc{aug}-1-\textsc{assoc} 16-11-north 1-\textsc{narr}-say 8-what 1-\textsc{narr}-refuse-\textsc{fv}\\
\glt \lq The one from the north did what? He refused.' [Lake Kyungululu]
\ex \label{exukutimaelijimo}
\gll mwa=n-dugutu a-ka-a-bop-ile=po. a-a-ba-paal-ile a-ba-nine. bo a-ba a-a-ba-bɪɪk-ile \textbf{ʊ}-\textbf{kʊ}-\textbf{tɪ} maelɪ jɪ-mo, maelɪ jɪ-mo, maelɪ jɪ-mo\\
matronym=9-type\_of\_bird 1-\textsc{neg}-\textsc{pst}-run-\textsc{pfv}=\textsc{part} 1-\textsc{pst}-2-invite-\textsc{pfv} \textsc{aug}-2-companion \textsc{ref.2} \textsc{aug}-\textsc{prox.2} 1-\textsc{pst}-2-put-\textsc{pfv} \textsc{aug}-15-say mile(9)(<EN) 9-one mile(9) 9-one mile(9) 9-one\\
\glt `Mr. Tugutu did not run at all. He had gathered companions. Those are the ones he placed, like one mile, one mile, one mile.' [Hare and Tugutu]
\end{exe}

The infinitive\is{infinitive} \textit{ʊkʊtɪ} is further grammaticalized as a complementizer (\ref{exUkutiComplementizer}). It also serves to introduce clauses of purpose\is{subordinate clauses!purpose clause} and result;\is{subordinate clauses!result clause} see \sectref{SubjunctiveSubordinate}.
\begin{exe}
\ex\label{exUkutiComplementizer}
\gll ʊ-meenye ʊkʊtɪ Asia a-ka-kʊ-gan-a?\\
\textsc{2sg}-know.\textsc{pfv} \textsc{comp} A. 1-\textsc{neg}-\textsc{2sg}-love-\textsc{fv}\\
\glt `Do you know that Asia doesn't love you?' [Juma, Asia and Sambuka]
\end{exe}

Similarly, the \isi{infinitive} of \textit{tɪ} as the dependent element of the associative construction serves to introduce a clause as a nominal complement.\footnote{Following the associative particle, the augment on nouns is banned.} This is frequently used in the collocation \textit{kʊʊnongwa} (\textit{ɪ})\textit{jaa kʊtɪ} `for the reason that, because'.

\begin{exe}
\ex \label{exkunongwajaa} \gll kʊʊ-nongwa j-\textbf{aa} \textbf{kʊ}-\textbf{tɪ} Juma a-a-meenye ʊkʊtɪ Sambʊka m-manyaani gw-a Asia, po a-lɪnkw-itɪk-a ʊ-kʊ-bʊʊk-a kʊ-kʊ-many-a=ko kʊ-my-ake\\
17-issue(9) 9-\textsc{assoc} 15-say J. 1-\textsc{pst}-know.\textsc{pfv} \textsc{comp} S. 1-friend 1-\textsc{assoc} A. then 1-\textsc{narr}-agree-\textsc{fv} \textsc{aug}-15-go-\textsc{fv} 17-15-know-\textsc{fv}=17 17-4-\textsc{poss.sg}\\
\glt `Because Juma knew that Sambuka was a friend of Asia, he agreed to go and get to know her [Sambuka's] home.' [Juma, Asia and Sambuka]

\ex \gll Mfyage a-a-bʊʊk-ile kʊ-n̩-ganga ʊ-gw-a kɪ-tiitʊ, kʊ-kʊ-lond-a ʊ-n-kota ʊ-gw-\textbf{a} \textbf{kʊ}-\textbf{tɪ} ba-n̩-gan-ege fiijo a-ba-nyambala\\
M. 1-\textsc{pst}-go-\textsc{pfv} 17-1-healer \textsc{aug}-1-\textsc{assoc} 7-black 17-15-search-\textsc{fv} \textsc{aug}-3-medicine \textsc{aug}-3-\textsc{assoc} 15-say 2-1-love-\textsc{ipfv.subj} \textsc{intens} \textsc{aug}-2-man\\
\glt `Mfyage went to a witch doctor to find a medicine that would make men love her very much.' [Mfyage turns into a lion] 
\end{exe}

The verb \textit{tɪ} also serves as an auxiliary,\is{auxiliary} taking a subjunctive\is{mood!subjunctive} complement in a number of conventionalized constructions; see \sectref{ComplexConructionsWithSubjunctive}. It also forms part of the invariable \isi{evidential} of report \textit{baatɪ}.\footnote{This can doubtless be analyzed as \textit{ba-a-tɪ}. Given that this form is homophonous with the subsecutive with a noun class 2 subject (i.e. 3rd person plural used as impersonal), this might be an indication that the subsecutive configuration has developed diachronically out of a former perfective\is{aspect!perfective} or anterior,\is{aspect!anterior} thus \lq They (have) said'. Cf. also \ili{Ndali} \textit{báti}, which apparently fulfils the same function \citep[107]{BotneR2008}.} This particle serves to indicate that the source of information is hearsay. It can be used to distance oneself from what is reported, ascribing responsibility to the original source (\ref{exBaatiHare}) and is also commonly used to echo what has just been said (\ref{exBaatiDuka}, \ref{exBaatiMonkeyAndTortoise}).

\begin{exe}
\ex\label{exBaatiHare}
\gll Saliki a-lɪnkʊ-tɪ \textup{\lq\lq}ʊ-m̩-buut-ile kalʊlʊ ʊ-jʊ n-d-ile ʊ-buut-ege?\textup{''} ʊ-n-kasi gw-a Saliki a-lɪnkʊ-tɪ \textup{\lq\lq}keet-a ʊ-t-ile \textbf{baatɪ} n-heesya gw-ɪtu!? n-um̩-buut-iile ɪ-n-gʊkʊ. a-li=mo n-nyumba, a-lɪ pa-kʊ-ly-a=mo\textup{''}\\
S. 1-\textsc{narr}-say \phantom{\lq\lq}\textsc{2sg}-1-slaughter-\textsc{pfv} hare(1) \textsc{aug}-\textsc{prox.1} \textsc{1sg}-say-\textsc{pfv} \textsc{2sg}-slaughter-\textsc{ipfv.subj} \textsc{aug}-1-wife 1-\textsc{assoc} S. 1-\textsc{narr}-say \phantom{\lq\lq}look-\textsc{fv} \textsc{2sg}-say-\textsc{pfv} hearsay 1-guest 1-\textsc{poss.1pl} \textsc{1sg}-1-slaughter-\textsc{appl.pfv} \textsc{aug}-9-chicken 1-\textsc{cop}=18 18-house(9) 1-\textsc{cop} 16-15-eat-\textsc{fv}=some\\
\glt `Saliki said ``Have you slaughtered Hare, whom I told you to slaugher?'' Saliki's wife said ``Look, you said he is our guest!? I slaughtered a chicken for him. He's in the house, he's eating.''{}' [Saliki and Hare]
\ex \label{exBaatiDuka}
Context: The researcher has asked for a soda at a small shop. The friend of the shop owner is surprised by his language skills and repeats his words:\\
\gll \textbf{baatɪ} \textup{\lq\lq}n-gʊ-sʊʊm-a ɪɪ-kook\textup{''}\\
hearsay \phantom{\lq\lq}\textsc{1sg}-\textsc{prs}-beg-\textsc{fv} \textsc{aug}-C.(9)\\
\glt \lq [Quoting:] \lq\lq I'd like a Coke.''{}' [overheard]
\ex\label{exBaatiMonkeyAndTortoise}
Context: Tortoise's child has just told Monkey that Tortoise senior is sad.\\
\gll \textup{\lq\lq}fi-ki?\textup{''} \textup{\lq\lq}a-fw-ile ɗaaɗa gw-ake\textup{''} \textup{\lq\lq}n-koolel-e!\textup{''} a-lɪnkʊ-sook-a kajamba, i-kʊ-lɪl-a \textup{\lq\lq}hɪhɪhɪhɪɪ, hɪhɪhɪhɪɪ, a-fw-ile, a-fw-ile\textup{''}\\
\phantom{\lq\lq}8-what \phantom{\lq\lq}1-die-\textsc{pfv} sister(<SWA) 1-\textsc{poss.sg} \phantom{\lq\lq}1-call-\textsc{imp} 1-\textsc{narr}-leave-\textsc{fv} tortoise(1) 1-\textsc{prs}-cry-\textsc{fv} \phantom{\lq\lq}of\_crying of\_crying 1-die-\textsc{pfv} 1-die-\textsc{pfv}\\
\glt \lq [Monkey:] \lq\lq Why?'' [Tortoise's child:] \lq\lq His sister died'' [Monkey:] \lq\lq Call him!'' Tortoise came out, he is crying \lq\lq ''hihihihiii, hihihihii, she died, she died''.'
\sn \gll po mwa=n-gambɪlɪ \textup{\lq\lq}he? \textbf{baatɪ} a-fw-ile ɗaada, ee? po ndaga\textup{''}\\
then matronym=9-monkey \phantom{\lq\lq}\textsc{interj} hearsay 1-die-\textsc{pfv} sister yes then thanks\\
\glt \lq  Monkey: \lq\lq So your sister died, yes? My sympathy.''{}' [Monkey and Tortoise]
\end{exe}

Note that even in the wider discourse context of the preceding examples there is no referent of noun class 2 which the \textit{ba}- portion could cross-reference. Also note that \textit{tɪ} cannot normally be followed by another instance of itself:
\begin{exe}
\ex[*]{\gll keeta ʊ-t-ile ʊ-kʊ-tɪ n-heesya gw-ɪtʊ\\
look \textsc{2sg}-say-\textsc{pfv} \textsc{aug}-15-say 1-guest 1-\textsc{poss.1pl}\\}  
\label{exNoTwoTi}
\end{exe}
 
A homophonous form \textit{baatɪ} is also used as a call for attention (\ref{exBaatiAttention}).\footnote{Note that this parallels \ili{Swahili} \textit{ati}$\sim$\textit{eti}, which is similarly used as an evidential of report and as an interjection (\citealt[17]{MadanA1903}; \citealt[19]{MawJ2013}). It is unclear if this use of Nyakyusa \textit{baatɪ} is a result of a parallel development or if its usage has been influenced by Swahili.\il{Swahili}.}

\begin{exe}
\ex \label{exBaatiAttention}\textit{baatɪ} \lq Listen!'
\end{exe}

Another use of \textit{tɪ} is that of naming or calling people or entities. Note the \isi{object marker} in (\ref{extiOM}), which is otherwise not licensed with this verb.
\begin{exe}
\ex \gll ijolo n̩-dw-iho lw-a ba-Nyakyʊsa ba-a-lɪ na=a-ka-jɪɪlo k-a n-kiikʊlʊ ʊ-kʊ-n-tiil-a ʊ-gwise gw-a n̩-dʊme, ʊ-jʊ \textbf{tʊ}-\textbf{kʊ}-\textbf{tɪ} n-kamwana\\
old\_times 18-11-custom 11-\textsc{assoc} 2-Ny. 2-\textsc{pst}-\textsc{cop} \textsc{com}=\textsc{aug}-12-custom 12-\textsc{assoc} 1-woman \textsc{aug}-15-1-fear-\textsc{fv} \textsc{aug}-his\_father(1) 1-\textsc{assoc} 1-husband \textsc{aug}-\textsc{prox.1} \textsc{1pl}-\textsc{prs}-say 1-in\_law\\
\glt `Long ago in the tradition of the Nyakyusa people they had a custom of the woman fearing the father of her husband, whom we call Nkamwana.' [Should she save a life\ldots]

\ex \gll ky-a-li=po n=ɪ-kɪ-piki, \textbf{tʊ}-\textbf{kʊ}-\textbf{tɪ} ɪ-m-bale\\
7-\textsc{pst}-\textsc{cop}=16 \textsc{com}=\textsc{aug}-7-stump \textsc{1pl}-\textsc{prs}-say \textsc{aug}-7-type\_of\_wood\\
\glt \lq There was also a wood. We call it Mbale.' [Clothing long ago]

\ex \label{extiOM} \gll \textbf{bi}-\textbf{kʊ}-\textbf{n}-\textbf{tɪ} (jo) Mama Tuma\\ 2-\textsc{prs}-1-say \textsc{ref.1} M. T.\\\glt `They call her Mama Tuma' [ET]
\end{exe}

Lastly, \textit{tɪ} features in the conjunction \textit{kookʊtɪ} `that is to say, that means' (\ref{exkookuti}), in the universal quantifier \textit{kʊkʊtɪ} `every' (\ref{exkukuti}) and in \textit{ngatɪ} `as, like' (\ref{exngati}). 

\begin{exe}
\ex \label{exkookuti}
\gll bo ba-fik-ile pa-la ba-lɪnkʊ-sy-ag-a ɪ-n-gambɪlɪ si-tengeene m-mi-gʊnda gy-abo, si-kʊ-ly-a ɪ-fi-lombe kangɪ si-kw-ɪmb-a si-kʊ-tɪ \textup{\lq\lq}{ho! ho! ho!}\textup{''} \textbf{kookʊtɪ} \textup{\lq\lq}ee fi-nunu! ee fi-nunu! ee fi-nunu!\textup{''}\\
as 2-arrive-\textsc{pfv} 16-{dist} 2-\textsc{narr}-10-find-\textsc{fv} \textsc{aug}-10-monkey 10-live\_in\_peace.\textsc{pfv} 18-4-field 4-\textsc{poss.pl} 10-\textsc{prs}-eat-\textsc{fv} \textsc{aug}-8-maize again 10-\textsc{prs}-sing-\textsc{fv} 10-\textsc{prs}-say \phantom{\lq\lq}\textsc{interj} that\_is\_to\_say \phantom{\lq\lq}yes 8-good yes 8-good yes 8-good\\
\glt `When they arrived there, they found the monkeys looking at home in their fields, eating maize, singing and saying ``Ho! Ho! Ho!'' That is to say ``Yes, it's good! Yes, it's good! Yes, it's good!' (Thieving Monkeys)
\ex \label{exkukuti} \gll ʊ-gwe ʊ-lɪ na=a-ma-lʊndɪ lwele, \textbf{kʊkʊtɪ} kɪ-lʊndɪ k-oog-a a-m-ɪɪsi ɪ-n-dobo jɪ-mo\\
\textsc{aug}-\textsc{2sg} \textsc{2sg}-\textsc{cop} \textsc{com}=\textsc{aug}-6-leg eight every 7-leg \textsc{2sg.prs}-bath-\textsc{fv} \textsc{aug}-6-water \textsc{aug}-9-bucket 9-one\\
\glt `You have eight legs, every leg you bathe in one bucket of water.' [Hare and Spider]
\ex \label{exngati}
\gll po jʊ-la i-kw-and-a ʊ-kʊ-bin-a fiijo n=ʊ-kʊ-ʊbʊk-a ʊ-m̩-bɪlɪ \textbf{ngatɪ} lw-ifi\\
then 1-\textsc{dist} 1-\textsc{prs}-begin-\textsc{fv} \textsc{aug}-15-fall\_sick-\textsc{fv} \textsc{intens} \textsc{com}=\textsc{aug}-15-peel\_off-\textsc{fv} \textsc{aug}-3-body like 11-chameleon\\
\glt \lq Then that person begins to get very ill and his body peels like a chameleon.' [Killer woman]
\end{exe}
\section{Movement grams}\label{MovementGrams}
\is{motion|(}
In this section, two auxiliary verbs will be discussed that provide a sense of (figurative) movement:\footnote{The term \textit{movement gram} has been adopted from \citet{NicolleS2002}.} (\textit{j})\textit{a} `go' and \textit{isa} `come'. Both verbs are not only related in meaning, but also pattern together in syntactic terms.\is{syntax} As their complement, they both take an augmentless infinitive.\is{infinitive} Further, the \isi{simple present} of both verbs has undergone further \isi{grammaticalization} to a futurate,\is{futurate} a use in which the infinitive complement can take the imperfective\is{aspect!imperfective} suffix -\textit{aga}. 

\subsection{(\textit{\textit{j}})\textit{\textit{a}} `go'}\label{jaAspectualizer}
The movement verb (\textit{j})\textit{a}, which is glossed as `go' throughout this study, is  attested only as a movement gram, not as a main verb. Following the infinitive\is{infinitive} or \isi{simple present} prefix \textit{kʊ}-, only the vocalic segment is realized, yielding \textit{kwa}. This loss of the consonantal segment is shared with the copula of the same shape (\sectref{Copulae}), albeit in a different environment. Use of (\textit{j})\textit{a} construes the state-of-affairs encoded in the lexical verb against a preceding motion event (cf. \citealt[251]{WilkinsD1991}). One possible reading is that of two sequential sub-eventualities, hence \lq go (and) verb':

\begin{exe}
\ex \gll po lɪnga a-ba-bʊʊl-ile a-ba-paapi ba-ake pamo ʊ-gwise gw-a n̩-dʊmyana jʊ-la, \textbf{a}-\textbf{a}-\textbf{j}-\textbf{aga} \textbf{kʊ}-n-sʊʊm-ɪl-a kʊ-gwise gw-a n-kiikʊlʊ ʊkʊtɪ \textup{\lq\lq}ʊ-mw-anaako, n-gʊ-lond-a ʊkʊtɪ eeg-igw-ege n=ʊ-mw-anangʊ\textup{''}\\
then if/when 1-2-tell-\textsc{pfv} \textsc{aug}-2-parent 2-\textsc{poss.sg} or \textsc{aug}-his\_father(1) 1-\textsc{assoc} 1-boy 1-\textsc{dist} 1-\textsc{pst}-go-\textsc{ipfv} 15-1-beg-\textsc{appl}-\textsc{fv} 17-his\_father(1) 1-\textsc{assoc} 1-woman \textsc{comp} \phantom{\lq\lq}\textsc{aug}-1-your\_child \textsc{1sg}-\textsc{prs}-want-\textsc{fv} \textsc{comp} 1.marry-\textsc{pass}-\textsc{ipfv.subj} \textsc{com}=\textsc{aug}-1-my\_child\\
\glt \lq When he had told his parents or his father, he [father] would go to the woman's father and ask \lq\lq Your child, I want her to be married to my child.''{}' [Life and marriage long ago]

\ex Context: The researcher is on his way home in the afternoon.\\
\gll \textbf{ʊ}-\textbf{j}-\textbf{ile} \textbf{kʊ}-bomb-a?\\
\textsc{2sg}-go-\textsc{pfv} 15-work-\textsc{fv}\\
\glt \lq Did you go and work?' [overheard]
\end{exe}

In other cases, \textit{(j)a} does not introduce a change of location. This becomes clearest when it follows a form of the lexical verb \textit{bʊʊka} \lq go', as in (\ref{exJaFollowingBuuka1}, \ref{exJaPython}). Instead of introducing a second motion event, (\textit{j})\textit{a} recapitulates the goal-oriented motion expressed by preceding \textit{bʊʊka}. In (\ref{exJaFollowingBuuka1}), this involves going to the explicitly mentioned field, and in (\ref{exJaPython}) moving to the house, which is understood from the context.
\begin{exe}
\ex \label{exJaFollowingBuuka1}\gll bo ka-kɪnd-ile=po a-ka-balɪlo ka-nandɪ Pakyɪndɪ \textbf{a}-\textbf{lɪnkʊ}-\textbf{bʊʊk}-\textbf{a} kʊ-n̩-gʊnda. \textbf{a}-\textbf{lɪnkw}-\textbf{a} \textbf{kʊ}-mmw-ag-a ʊ-n-kasi n=ʊ-n-nyambala ʊ-jʊ-ngɪ mu-n̩-gʊnda mu-la\\
as 12-pass-\textsc{pfv}=\textsc{cmpr} \textsc{aug}-12-time 12-little P. 1-\textsc{narr}-go-\textsc{fv} 17-3-field 1-\textsc{narr}-go.\textsc{fv} 15-1-find-\textsc{fv} \textsc{aug}-1-wife \textsc{com}=\textsc{aug}-1-man \textsc{aug}-1-other 18-3-field 18-\textsc{dist}\\
\glt `When a short time had passed, Pakyindi went to the field. He (went and) found his wife with another man in that field.' [Sokoni and Pakyindi]
\ex \label{exJaPython}
Context: Python is hiding in a banana tree outside a house.\\
\gll po j-aa-tɪ \textup{\lq\lq}niine n-gʊ-bʊʊk-a bo a-ka-j-a=po maama jʊ-la ʊ-n-kiikʊlʊ.\textup{''} po \textbf{bo} \textbf{jɪ}-\textbf{bʊʊk}-\textbf{ile} ɪɪ-sota j-oope \textbf{j}-\textbf{aa}-\textbf{j}-\textbf{ile} \textbf{kw}-ɪmb-a\\
then 9-\textsc{subsec}-say \phantom{\lq\lq}\textsc{com.1sg} \textsc{1sg}-\textsc{prs}-go-\textsc{fv} as 1-\textsc{neg}-be(come)-\textsc{fv}=16 mother(<SWA) 1-\textsc{dist} \textsc{aug}-1-woman then as 9-go-\textsc{pfv} \textsc{aug}-python(9) 9-also 9-\textsc{pst}-go-\textsc{pfv} 15-sing-\textsc{fv}\\
\glt \lq Then it [Python] said \lq\lq Me too, I'm going when that woman isn't there.''  ‎‎When the python had gone [to the house], it sang.' [Python and woman]
\end{exe}
 
The preceding example illustrates another important point about (\textit{j})\textit{a}: this construal of a lexical state-of-affairs against the ground of a motion event is often employed in narratives\is{narrative} to trace the participants and their actions as they move through space. (\ref{exJaFollowingBuuka2}--\ref{exjafika}) are further examples of this.

\begin{exe}
\ex \label{exJaFollowingBuuka2}
Context: Hare and Skunk are staying together.\\
\gll po nsysyɪ j-oope \textbf{a}-\textbf{a}-\textbf{bʊʊk}-\textbf{ile}, \textbf{a}-\textbf{a}-\textbf{j}-\textbf{ile} \textbf{kʊ}-lond-a a-ma-ani. a-al-iis-ile na=a-ma-ani ga-la bo a-gon-ile ʊ-tʊ-lo kalʊlʊ\\
then skunk(1) 1-also 1-\textsc{pst}-go-\textsc{pfv} 1-\textsc{pst}-go-\textsc{pfv} 15-search-\textsc{fv} \textsc{aug}-6-leaf 1-\textsc{pst}-come-\textsc{pfv} \textsc{com}=\textsc{aug}-6-leaf 6-\textsc{dist} as 1-rest-\textsc{pfv} \textsc{aug}-13-sleep hare(1)\\
\glt `Skunk also went, he (went and) searched for leaves. He came with those leaves, while Hare was asleep.' [Hare and Skunk]

\ex \label{exJaNyeela}
Context: Hare is trapped in a pit.\\
\gll a-a-fum-ile na=a-ma-ka n-k-iina mu-la. a-a-nyeel-ile \textbf{a}-\textbf{a}-\textbf{j}-\textbf{ile} \textbf{kʊ}-tɪ \textup{\lq\lq}tuu!\textup{''} p-ii-sɪɪlya\\ 
1-\textsc{pst}-come\_from-\textsc{pfv} \textsc{com}=\textsc{aug}-6-force 18-7-pit 18-\textsc{dist} 1-\textsc{pst}-jump-\textsc{pfv} 1-\textsc{pst}-go-\textsc{pfv} 15-say \phantom{\lq\lq}of\_thunk 16-5-other\_side\\
\glt `He [Hare] came out of that pit with force. He jumped and made ``tuu!'' on the other side.' [Saliki and Hare]

\newpage
\ex\label{exjafika}
Context: A woman has just passed a branch-off.\\
\gll po a-lɪnkʊ-golok-a, a-lɪnkʊ-golok-a. \textbf{a}-\textbf{lɪnkw}-\textbf{a} \textbf{kʊ}-fik-a kʊ-jeng-iigwe kʊ-nunu fiijo. po \textbf{a}-\textbf{lɪnkw}-\textbf{a} \textbf{kʊ}-ba-ag-a ba-lɪndɪlɪli ba-a ka-aja ka-la. ba-lɪnkʊ-n̩-daalʊʊsy-a ba-lɪnkʊ-tɪ \textup{\lq\lq}kʊ-lond-a fi-ki?\textup{''}\\
then 1-\textsc{narr}-go\_straight-\textsc{fv} 1-\textsc{narr}-go\_straight-\textsc{fv}. 1-\textsc{narr}-go.\textsc{fv} 15-arrive-\textsc{fv} 17-build-\textsc{pass.pfv} 17-well \textsc{intens} then 1-\textsc{narr}-go.\textsc{fv} 15-2-find-\textsc{fv} 2-guard 2-\textsc{assoc} 12-village 12-\textsc{dist} 2-\textsc{narr}-1-ask-\textsc{fv} 2-\textsc{narr}-say \phantom{\lq\lq}\textsc{2sg.prs}-want-\textsc{fv} 8-what\\
\glt \lq She went straight, she went straight. She (went and) arrived at a place well built. She (went and) met the guards of that village. They asked \lq\lq What do you want?''{}' [Throw away the child]
\end{exe}

\largerpage
The simple present of (\textit{j})\textit{a} is further grammaticalized as a marker of prospective aspect,\is{aspect!prospective} which retains a possible spatial or motion reading. This is discussed in \sectref{Prospectivekwa}. Note that the movement gram (\textit{j})\textit{a} cannot express motion with purpose. For this, \textit{bʊʊka} \lq go' plus an infinitive\is{infinitive} marked for \isi{locative} class\is{noun classes} 17 or 18 has to be used; see \sectref{VerbalNounsArguments} for a discussion. Lastly, unlike its counterpart \textit{isa} (\sectref{isaAspectualizer}), the \isi{simple present} of (\textit{j})\textit{a} does not have a habitual\is{aspect!habitual} or generic\is{aspect!generic} reading:

\begin{exe}
\ex[*]{\gll kʊkʊtɪ ky-ɪnja n-gw-a kʊ-gy-ag-a ɪ-mi-kambɪlɪ kʊ-mi-gʊnda gy-ɪtʊ\\
every 7-year \textsc{1sg}-\textsc{prs}-go.\textsc{fv} 15-4-find-\textsc{fv} \textsc{aug}-6-monkey 17-4-field 4-\textsc{poss.1pl}\\
\glt (intended: \lq Every year I go and find damn monkeys in our fields.')}
\end{exe}

\subsection{\textit{isa} `come'}\label{isaAspectualizer}
The verb \textit{isa} \lq come', when used as an auxiliary,\is{auxiliary} has a figurative meaning of reaching, achieving or being led to a particular condition.

\begin{exe}

\ex \gll bo a-lɪ n=ʊ-lw-anda \textbf{iis}-\textbf{aga} \textbf{kʊ}-pon-a nalooli ʊ-mw-ana ʊ-n-kiikʊlʊ\\
as 1-\textsc{cop} \textsc{com}=\textsc{aug}-11-stomach 1.\textsc{pst}.come-\textsc{ipfv} 15-give\_birth-\textsc{fv} really \textsc{aug}-1-child \textsc{aug}-1-woman\\
\glt \lq When she was pregnant, she would eventually give birth to a girl.' [Life and marriage long ago]

\ex \label{exIsaAuxNegPRS}\gll mw-ilaamwisye ɪ-n-dagɪlo sy-angʊ. ʊ-mw-ana a-ka-bagɪl-a ʊ-kw-end-a kangɪ, \textbf{a}-\textbf{ti}-\textbf{kw}-\textbf{is}-\textbf{a} \textbf{kʊ}-job-a sikʊ kangɪ\\
\textsc{2pl}-disregard.\textsc{pfv} \textsc{aug}-10-rule 10-\textsc{poss.1sg} \textsc{aug}-1-child 1-\textsc{neg}-be\_able-\textsc{fv} \textsc{aug}-15-walk/travel-\textsc{fv} again, 1-\textsc{neg}-\textsc{prs}-come-\textsc{fv} 15-speak-\textsc{fv} ever again\\
\glt `You have disregarded my rules. The child can't walk, it'll never get to talk.' [Pregnant women]

\ex \gll po kanunu ʊ-kʊ-j-a m̩-bombi gw-abo kʊ-ka-balɪlo a-ka-a kʊ-lond-a ʊ-kʊ-kab-a ɪ-n-dalama ɪ-sy-a k-ʊʊl-ɪl-a ɪ-fi-bombelo, ɪ-fy-a \textbf{kw}-\textbf{is}-\textbf{a} \textbf{kʊ}-bomb-el-a kɪsita kʊ-lʊmbʊʊs-igw-a\\
then well \textsc{aug}-15-be(come)-\textsc{fv} 1-worker 1-\textsc{poss.pl} 17-12-time \textsc{aug}-12-\textsc{assoc} 15-want-\textsc{fv} \textsc{aug}-15-get-\textsc{fv} \textsc{aug}-10-money \textsc{aug}-10-\textsc{assoc} 15-buy-\textsc{appl}-\textsc{fv} \textsc{aug}-8-tool \textsc{aug}-8-\textsc{assoc} 15-come-\textsc{fv} 15-work-\textsc{appl}-\textsc{fv} without 15-humiliate-\textsc{pass}-\textsc{fv}\\%xxx check vowel length akaa
\glt \lq ‎‎And so it is good to be their worker for a time in which you want to get money to buy tools with, for later working with without being disparaged [lit. \ldots tools of coming to work with \ldots].' [Types of tools in the home]

\ex \gll a-ka-pango a-ka ki-kʊ-tʊ-many-isy-a ʊkʊtɪ tʊ-ng-iib-aga, \textbf{tʊ}-\textbf{ng}-\textbf{iis}-\textbf{a} \textbf{kʊ}-fw-a bo lʊʊ$\sim$lo sy-a-fw-ile ɪ-n-gambɪlɪ si-la\\
\textsc{aug}-12-story \textsc{aug}-\textsc{prox.12} 12-\textsc{prs}-\textsc{1pl}-know-\textsc{caus}-\textsc{fv} \textsc{comp} \textsc{1pl}-\textsc{neg.subj}-steal-\textsc{ipfv} \textsc{1pl}-\textsc{neg.subj}-come-\textsc{fv} 15-die-\textsc{fv} as \textsc{redupl}$\sim$\textsc{ref.11} 10-\textsc{pst}-die-\textsc{pfv} \textsc{aug}-10-monkey 10-\textsc{dist}\\
\glt `This story teaches us that we should not steal, otherwise we will die [lit. we should not come to die] just like those monkeys died.' [Thieving monkeys]
\end{exe}

In the affirmative subjunctive,\is{mood!subjunctive} a variant construction is attested in which the lexical verb is not expressed as an augment-less infinitive,\is{infinitive} but also figures in the subjunctive paradigm:

\begin{exe}
\ex \gll kangɪ ʊ-swɪl-enge=po n=ɪ-n-gʊlʊbe pa-ka-aja ʊkʊtɪ bo g-ʊʊl-iisye ɪ-n-dalama ɪ-syo \textbf{s}-\textbf{iis}-\textbf{e} \textbf{si}-\textbf{kʊ}-\textbf{tʊʊl}-\textbf{ege} ʊ-kʊ-ba-homb-a a-ba-fundi\\
again \textsc{2sg}-rear-\textsc{ipfv.subj}=16 \textsc{com}=\textsc{aug}-10-pig 16-12-homestead \textsc{comp} as \textsc{2sg}-buy-\textsc{caus.pfv} \textsc{aug}-10-money \textsc{aug}-\textsc{ref.10} 10-come-\textsc{subj} 10-\textsc{2sg}-help-\textsc{ipfv.subj} \textsc{aug}-15-2-pay-\textsc{fv} \textsc{aug}-2-workman(<SWA)\\
\glt \lq And you should be raising pigs at home so that when you have sold them, the money can be helping you to pay the workmen [lit. \ldots so that the money comes to help you \ldots].' [How to build modern houses]
\end{exe}

Note that the movement gram \textit{isa}, like its counterpart (\textit{j})\textit{a} (see \sectref{Prospectivekwa}) cannot express motion with purpose. For this, an infinitive\is{infinitive} marked for \isi{locative} class\is{noun classes} 16 or 18 has to be used; see \sectref{VerbalNounsArguments} for a discussion.

As (\ref{exIsaAuxNegPRS}) above indicates, the \isi{simple present} of \textit{isa} has a \isi{futurate} reading. Another example of this is given in (\ref{exIsaAuxFuturateMovement}). This is also the only use of \textit{isa} discussed by \citet{SchumannK1899} and \citet{EndemannC1914}. As (\ref{exIsaAuxFuturateHabGen}) illustrates, the \isi{simple present} of \textit{isa}, however, also allows for a habitual/generic reading.\is{aspect!habitual}\is{aspect!generic}

\begin{exe}
\ex \label{exIsaAuxFuturateMovement}
\gll lɪlɪno \textbf{tʊ}-\textbf{kw}-\textbf{is}-\textbf{a} \textbf{kʊ}-kin-a ʊ-m-pɪla\\
now/today \textsc{1pl}-\textsc{prs}-come-\textsc{fv} 15-play-\textsc{fv} \textsc{aug}-3-ball\\
\glt \lq Today we'll come to play football [ET]'
\ex \label{exIsaAuxFuturateHabGen} \gll kʊkʊtɪ ky-ɪnja \textbf{n}-\textbf{gw}-\textbf{is}-\textbf{a} \textbf{kʊ}-gy-ag-a ɪ-mi-gambɪlɪ m-mi-gunda gy-ɪtʊ\\
every 7-year \textsc{1sg}-\textsc{prs}-come-\textsc{fv} 15-4-find-\textsc{fv} \textsc{aug}-4-monkey 18-4-farm 4-\textsc{poss.1pl}\\
\glt \lq Every year I come to find damn monkeys in our fields.' [ET]
\end{exe}%auch nötig, weil a INF kein PRS.HAB mehr erlaubt

In the \isi{futurate} use of \textit{isa}, the infinitive complement can take the imperfective\is{aspect!imperfective} suffix \mbox{-\textit{aga}}, which yields a continuous/progressive\is{aspect!progressive} reading and can add an epistemic\is{modality} flavour (\ref{exIsaAuxImperfective1}). Imperfective\is{aspect!imperfective} \mbox{-\textit{aga}} is also used with a habitual/generic reading\is{aspect!habitual}\is{aspect!generic} (\ref{exIsaAuxImperfective2}). Lastly, this \isi{futurate} use of \textit{isa} in the \isi{simple present} has undergone further grammaticalization,\is{grammaticalization} yielding the indefinite future\is{future!indefinite future} construction (\sectref{isaFut}).

\begin{exe}
\ex \label{exIsaAuxImperfective1}
\gll i-kw-is-a kʊ-jeng-aga kʊ-la\\
1-\textsc{prs}-come-\textsc{fv} 15-build-\textsc{ipfv} 17-\textsc{dist}\\
\glt 1. \lq He will come to be building there (continuously).'\\
2. \lq He will come to build there (presumably).' [ET]
\ex \label{exIsaAuxImperfective2}
\gll bi-kw-is-a kʊ-kin-aga ʊ-m-pɪla kʊkʊtɪ ii-sikʊ\\
2-\textsc{prs}-come-\textsc{fv} 15-play-\textsc{ipfv} \textsc{aug}-3-ball every 5-day\\
\glt \lq They will come to play football every day.' [ET]
\end{exe}
\is{motion|)}
