\chapter{Mood and modal categories}
\label{MoodModal}
\section{Introduction}\is{mood|(}\is{modality|(}
In this chapter, mood and modal constructions will be described, starting with the imperative\is{mood!imperative} (\sectref{Imperative}), then the subjunctive mood (\sectref{Subjunctive}).\is{mood!subjunctive} The latter shows a broad variety of meanings and uses, as is common in Bantu languages. The treatment of the subjunctive includes a discussion of its \isi{negative} counterpart and the distal/itive\is{itive} \textit{ka}-. Following this, two modal constructions will be described, which are rather uncommon for a Bantu language, namely the desiderative\is{mood!desiderative} (\sectref{Desiderative}) and the modal future construction (\sectref{Commissive}).\is{future!modal future} Lastly, there is a section on the \isi{conditional} marker (\sectref{ngali}).

An overview of mood and modal constructions and their composition is given in \tabref{tableMoodModalConstructions}. The terminology in this chapter follows \citet{PalmerF2007}, unless indicated otherwise.

\begin{table}[htb]%h 'genau hier' t 'top of page' als zweite lösung, b als letzte
\setlength{\tabcolsep}{2pt}
\begin{tabularx}{\textwidth}{llll}
\lsptoprule
\footnotesize{Label} & \footnotesize{Shape} & \footnotesize{Example}\\
\midrule
Imperative & \textsc{vb}-a(\textit{ga}) & \textit{joba} & \lq Speak!'\\ 
Subjunctive & \textsc{sm}-\textsc{vb}-\textit{e}(\textit{ge}) & \textit{tʊjobe} & \lq we should speak' \\ 
Distal subjunctive & \textsc{sm}-\textit{ka}-\textsc{vb}-\textit{e}(\textit{ge}) & \textit{tʊkajobe} & \lq we should go speak' \\ 
Neg. subjunctive & \textsc{sm}-(\textit{lɪ})-\textit{nga}-\textsc{vb}-\textit{a}(\textit{ga}) & \textit{tʊngajoba} & \lq we should not speak' \\
Desiderative & \textsc{sm}-\textit{lɪ}-\textsc{vb}-\textit{a}(\textit{ga}) & \textit{tʊlɪjoba} & \lq we would like to speak'\\ 
Modal future & \textsc{sm}\textsubscript{2}-\textit{kʊ}-\textsc{vb}-\textit{aga} & \textit{tʊkʊjobaga} & \lq we shall speak'\\ 
Conditional & \textit{ngalɪ} verb & \textit{ngalɪ tʊkʊjoba} & \lq we would speak'\\
\lspbottomrule  
\end{tabularx}
\caption{Mood and modal constructions}\label{tableMoodModalConstructions}
\end{table}
\is{mood|)}\is{modality|)}
\section{Imperative}\label{Imperative}
\is{mood!imperative|(}Direct orders in the second person singular are expressed by the imperative, which can consist of the bare stem:\is{stem}
\begin{exe}
\ex
\begin{tabbing}
\textit{bʊʊk-a!}x\=\kill
\textit{bʊʊk-a!}\>`Go!'
\\\textit{kin-a!}\>`Play!'
\\\textit{mog-a!}\>`Dance!'
\end{tabbing}
\end{exe}
The prosody of imperatives requires any vowel in initial position of the verbal word to be short\is{vowels!length} (\ref{exImperativeVowelShort}). This holds even for vowels followed by a prenasalized plosive (\ref{ImperativeShortBeforeNC}).\footnote{No imperative of the shape /ʊNC\ldots / is attested in the data and no Nyakyusa verb features initial /u/ (\sectref{PhonologicalStructureVerbalMorphemes}).}

\begin{exe}
\ex\label{exImperativeVowelShort}
\begin{tabbing}
\textit{ʊlɪsya!}x\=\kill
\textit{igala!}\>`Open!'
\\\textit{ɪma!}\>`Stand up!'
\\\textit{ega!}\>`Take!'
\\\textit{amula!}\>`Answer!
\\\textit{okya!}\>`Roast!'
\\\textit{ʊlɪsya!}\>`Sell!'
\end{tabbing}
\ex\label{ImperativeShortBeforeNC}
\begin{tabbing}
\textit{ongelapo!}x\=[ɔ.{\ᵑ}gɛ.ˈla.pʰɔ]x\=\kill
\textit{ingɪla!}\>[i.ˈ{\ᵑ}gɪɾ.a]\>`Enter!'
\\\textit{ɪmba!}\>[ˈɪ.ᵐba]\>`Sing!' 
\\\textit{endesya}\>[ɛ.ˈⁿdɛ.sʸa]\>`Drive!'
\\\textit{anda!}\>[ˈa.ⁿda]\>`Start!'
\\\textit{ongelapo!}\>[ɔ.{\ᵑ}gɛ.ˈla.pʰɔ]\>`Continue a bit!'
\end{tabbing}
\end{exe}

Verbs in the imperative can carry an object marker.\is{object marker} With object prefixes other than the first person singular the final vowel changes to -\textit{e}. This also holds for the \isi{reflexive} object prefix.\footnote{The resulting forms can thus be considered to be hybrid forms between the imperative and the subjunctive;\is{mood!subjunctive} see \citet[17--22]{DevosVanOlmen2013} for a discussion of such forms across Bantu. Also see p.\nobreakspace\pageref{SUBJohneSM} in \sectref{Subjunctive} for a similarly hybrid imperative-like subjunctive.} When the object prefix\is{object marker} of the first person singular precedes a plosive or approximant and thus the resultant word would be monosyllabic, the prefix surfaces as a syllabic nasal. This is discussed in \sectref{SubjectConcordsParticipants}.
\begin{exe}
\ex\begin{xlist}
\ex\gll m-b-\textbf{a}=ko ʊ-n-katɪ \\
\textsc{1sg}-give-\textsc{fv}=17 \textsc{aug}-3-bread \\
\glt `Give me the bread!'
\ex \gll m-p-\textbf{e}=ko ʊ-n-katɪ \\
1-give-\textsc{imp}=17  \textsc{aug}-3-bread\\
\glt `Give him the bread!'
\ex \gll i-kom-\textbf{e}\\
\textsc{refl}-hit-\textsc{imp}\\
\glt `Hit yourself!'
\end{xlist}
\end{exe}

\label{ImperativesMonosyllabic} Likewise, monosyllabic verbs in their bare stem form cannot be used as imperatives. They have to be augmented by either a meaningless prefix \textit{i}- or carry the imperfective\is{aspect!imperfective} suffix \mbox{-\textit{aga}} (\ref{exIMPmonosyllabic}). This has also been observed by \citet[69]{SchumannK1899}. Monosyllabic verbs are, however, acceptable as imperatives when they carry an \isi{enclitic} (\ref{exIMPMonosyllabicEnclitic}).
\begin{exe}
\ex\label{exIMPmonosyllabic}
\begin{tabbing}
\textit{inwa!}x\= / \textit{nwaga!}x\=`Drink!'x\=\kill
\textit{ilwa!}\> / \textit{lwaga!}\>`Fight!'\>(not *\textit{lwa})
\\\textit{ilya!}\> / \textit{lyaga!}\>`Eat!'\>(not *\textit{lya})
\\\textit{inwa!}\> / \textit{nwaga!}\>`Drink!'\>(not *\textit{nwa})
\end{tabbing}
\ex \label{exIMPMonosyllabicEnclitic}
\begin{tabbing}
\textit{nwapo!}x\=`Drink a bit!'x\=\kill
\textit{lwapo!}\>`Fight a bit!'\\
\textit{lyapo!}\>`Eat a bit!' \\
\textit{nwapo!}\>`Drink a bit!'
 \end{tabbing}
\end{exe}

Direct orders addressed to the second person plural are expressed by the subjunctive\is{mood!subjunctive} (\sectref{Subjunctive}). The only attested cases of formal imperatives directed to the second person plural are \textit{keeta} \lq look' and \textit{isaga} \lq come', which serve as invariable discourse markers. The imperfective\is{aspect!imperfective} suffix -\textit{aga} adds a range of meanings. Depending on the context, it can express a demand for an action to be carried out habitually/regularly\is{aspect!habitual} (\ref{exImperativeHab}). Further, the imperfective imperative can express a demand to continuously perform an action, which can shade over into an urge to initialize or continue said action (\ref{exImperativeContinuousIterative}). Lastly, the imperfective\is{aspect!imperfective} suffix can also serve to mitigate the imperative appeal, yielding phrases that range from weaker commands to invitations (\ref{exImperativeWeak2}). \citet[192]{NurseD2008} notes that this mitigating function of the imperfective imperative is found across Bantu.
\begin{exe}
\ex \label{exImperativeHab}
\gll bomb-aga bwila$\sim$bwila!\\
work-\textsc{ipfv} \textsc{redupl}$\sim$always\\
\glt `Always work! [ET]
\ex \label{exImperativeContinuousIterative}
\gll job-aga!\\
speak-\textsc{ipfv}\\
\glt `Be / get / continue speaking!' [ET]
\ex \label{exImperativeWeak2}
\gll ingɪl-aga\\
enter-\textsc{ipfv}\\
\glt `Come in!' [overheard]
\end{exe} 

Note that imperatives can be used not only for verbs featuring a volitional agent,\is{semantic roles} but also for non-volitional change-of-state verbs; \textcite[320]{SeidelF2008} observes the same for \ili{Yeyi} R41.
\begin{exe}
\ex\label{exImperativeNonVolitional1} 
\gll nyop-a!\\
be(come)\_soaked-\textsc{fv}\\
\glt `Get sweating! (e.g. do the hard work yourself)' [ET]
\ex \label{exImperativeNonVolitional2} 
\gll katal-a!\\
be(come)\_tired-\textsc{fv}\\
\glt `Act tired!' [ET]
\ex \label{exImperativeNonVolitional3}
Context: The plumber is claiming an excessive price.\\
\gll hobok-a!\\
be(come)\_happy-\textsc{fv}\\
\glt `Stop exaggerating!' [overheard]
\end{exe}

While in some Bantu languages, such as \ili{Yeyi} K41 \citep{SeidelF2008} and \ili{Zulu} S42 \citep{ZiervogelDLouwJATaljaardP1981}, the subjunctive and not the imperative has to be used in a sequence of commands from the second verb on, sequences of imperatives are allowed in Nyakyusa:
\begin{exe}
\ex 
\gll bɪɪk-a ʊ-lw-igi lʊ-mo lw-ene saam-il-a bʊbʊʊ$\sim$bo\\
put-\textsc{fv} \textsc{aug}-11-door 11-one 11-only migrate-\textsc{appl}-\textsc{fv} \textsc{redupl}$\sim$\textsc{ref.14}\\
\glt `‎‎Put one door in, so you can move in.' [How to build modern houses]\label{exSeriesImperatives}
\end{exe}

There is no formal \isi{negative} counterpart to the imperative. Instead, two different strategies are available for forming negative orders (prohibitives). First, the prohibitive elements \textit{komma} or \textit{somma} followed by the \isi{infinitive} may be used (see \sectref{VerbalNounsNegation}). This is very common and bears the strongest directive force. A second strategy is the use of the negative subjunctive\is{mood!subjunctive} (\sectref{NegativeSubjunctive}).\is{mood!imperative|)}
\section{Subjunctive}\label{Subjunctive} \is{mood!subjunctive|(}
The final vowel -\textit{e} (imperfective\is{aspect!imperfective} -\textit{ege}) marks a modal category that is commonly labelled \textit{subjunctive} in the Bantuist tradition (\citealt[203f]{DokeC1935}; \citealt[83f]{RoseEtal2002}). \citet[62f]{EndemannC1914} speaks of the \textit{final} mood in his description of Nyakyusa. In the case of defective \textit{tɪ} (\sectref{defectiveti}), no change in final vowel takes place. The subjunctive in Nyakyusa gives a wide array of readings and comes close to what \citet[326]{TimberlakeA2007}, from a typological perspective, characterizes as ``an all-purpose mood used to express a range of less-than-completely real modality''.\is{modality}

In the following description, first the uses of the subjunctive as such will be outlined, distinguishing between independent and subordinate clauses. This is followed by a description of some complex verbal constructions that include a subjunctive verb and a section on the distal/itive prefix\is{itive} \textit{ka}-. Lastly, the \isi{negative} counterpart to the subjunctive paradigm will be discussed. 
\subsection{Uses of the subjunctive}
\subsubsection{Subjunctive uses in main predication}\label{SubjunctiveMainClause}
The subjunctive can be used performatively for directives, where it is considered milder than the imperative.\is{mood!imperative}
\begin{exe}
\ex
\gll ʊ-m-fis-e kʊʊ-sofu tʊ-m̩-buut-e\\
\textsc{2sg}-1-hid-\textsc{subj} 17-room(9) \textsc{1pl}-1-slaughter-\textsc{subj}\\
\glt `Hide him [Hare] in the room, we shall slaughter him.' [Saliki and Hare]
\end{exe}

\label{SUBJohneSM} When the subjunctive is used in a directive, the \isi{subject marker} of the second person singular can be omitted.\footnote{This is frequent in natural speech and also attested in the textual data. In elicitation, however, speakers were hesitant to use this form. Given the constraint on monosyllabic imperatives\is{mood!imperative} and subjunctives (see p.\nobreakspace\pageref{ImperativesMonosyllabic} in \sectref{ImperativesMonosyllabic} and p.\nobreakspace\pageref{MonosyllabicSubjunctives} in \sectref{MonosyllabicSubjunctives}), it is very probable that dropping the subject prefix\is{subject marker} is not possible with monosyllabic roots. Optionality of the subject marker also holds for \ili{Yeyi} \citep[323]{SeidelF2008}. \citet[21f]{DevosVanOlmen2013} indicate more such cases in other Bantu languages.}
\begin{exe}
\ex \gll igʊl-e ʊ-lw-igi\\
open-\textsc{subj} \textsc{aug}-11-door\\
\glt \lq Open the door!' [overheard]
\ex \gll bʊʊk-a k-ʊʊl-e ʊ-bʊ-meme\\
go-\textsc{fv} \textsc{itv}-buy-\textsc{subj} \textsc{aug}-14-electricity(<SWA)\\
\glt `Go buy [vouchers for] electricity!' [overheard]\footnotemark
\end{exe}
\protect\footnotetext{See \sectref{DistalKa} for the distal/itive prefix \textit{ka}-.}

The subjunctive is also used as a counterpart to the imperative\is{mood!imperative} for the second person plural (\ref{exSubjImp2pl}). Negative\is{negative} commands to the second person plural are formed by the use of either \textit{somma}/\textit{komma} plus the infinitive\is{infinitive} (\sectref{VerbalNounsNegation}), or by the use of the negative subjunctive (\sectref{NegativeSubjunctive}).
\begin{exe}
\ex \label{exSubjImp2pl}\gll ʊ-malafyale a-a-ba-bʊʊl-ile a-a-t-ile \textup{\lq\lq}\textbf{mu}-\textbf{bʊʊk}-\textbf{e} nuumwe \textbf{mu}-\textbf{ka}-\textbf{kol}-\textbf{e} ii-boole ɪ-ly-ʊmi \textbf{mu}-\textbf{lɪ}-\textbf{twal}-\textbf{e} kʊ-no!\textup{''}\\
\textsc{aug}-chief(1) 1-\textsc{pst}-2-tell-\textsc{pfv} 1-\textsc{pst}-say-\textsc{pfv} \phantom{\lq\lq}\textsc{2pl}-go-\textsc{subj} \textsc{com.2pl} \textsc{2pl}-\textsc{itv}-grasp/hold-\textsc{subj} 5-leopard \textsc{aug}-5-live \textsc{2pl}-5-carry-\textsc{subj} 17-\textsc{prox}\\
\sn `The chief told them ``You (pl.) too go, catch a live leopard and bring it here!''{}' [Chief Kapyungu]
%\sn used to at least keep the free translation together; might need a different solution in final typesetting
\end{exe}

The subjunctive is also used in jussives, including hortatives (directions to the first person plural):
\begin{exe}
\ex \gll ʊ-jʊ lɪnga i-kʊ-m-boni-a, \textbf{aa}-\textbf{seng}-\textbf{ege} taasi n=ɪ-n-dwanga pa-kɪ-kosi po \textbf{a}-\textbf{m}-\textbf{boni}-\textbf{ege}\\
\textsc{aug}-\textsc{prox.1} if/when 1-\textsc{prs}-\textsc{1sg}-greet-\textsc{fv} 1.\textsc{1sg}-chop-\textsc{ipfv.subj} first \textsc{com}=\textsc{aug}-9-axe 16-7-neck then 1-\textsc{1sg}-greet-\textsc{ipfv.subj}\\
\glt `Whoever greets me shall first hit me with an axe in the neck, then he shall greet me.' [Chief Kapyungu]
\ex \gll po nsyɪsyɪ a-lɪnkʊ-tɪ \textup{\lq\lq}gwe jɪ-p-iile. is-aga \textbf{tʊ}-\textbf{ly}-\textbf{ege}!\textup{̈''}\\
then skunk(1) 1-\textsc{narr}-say \phantom{\lq\lq}\textsc{2sg} 9-be(come)\_burnt-\textsc{pfv} come-\textsc{ipfv} \textsc{2pl}-eat-\textsc{ipfv.subj}\\
\glt `Then Skunk said ``You, it [meat] is done. Come, let's eat!''{}' [Hare and Skunk]
\end{exe}

A pre-initial \textit{a}=\is{proclitic} is sometimes used in directives and hortatives, which apparently provides the request with a summoning or encouraging character (\ref{SubjunctiveA1}, \ref{SubjunctiveA2}). This is not to be confused with the future proclitic \textit{aa}= (\sectref{ProcliticAa}).\is{future!future aa=}

\begin{exe}
\ex \label{SubjunctiveA1}\gll mwe a=mu-pɪlɪkɪsy-e\\
\textsc{2pl} \textsc{hort}=\textsc{2pl}-listen-\textsc{subj}\\
\glt \lq Hey, listen up!' [ET]
\ex \label{SubjunctiveA2}\gll a=tʊ-tʊʊsy-e\\
\textsc{hort}=\textsc{1pl}-rest-\textsc{subj}\\
\glt \lq Come, let's rest!' [ET]
\end{exe}

\largerpage
\label{SubjunctiveNa} A pre-initial form of comitative \textit{na}=\is{proclitic} with directives and jussives strengthens the coercive force and typically gives a reading of urgency (\ref{exnaSUBJ1pl}--\ref{exnaSUBJ2pl2}). \citet[89]{NicolleS2013b} uses the label \lq\lq emphatic subjunctive'' for the same construction in \ili{Digo} E73.
\begin{exe}
\ex \label{exnaSUBJ1pl}
\gll na=tʊ-bʊʊk-e\\
\textsc{com}=\textsc{1pl}-go-\textsc{subj}\\
\glt \lq Let's go! (urging and/or annoyed)' [ET]
\ex \label{exnaSUBJ2pl1} \gll po leelo j-aal-iis-ile ɪ-m-bwa. j-aa-t-ile \textup{\lq\lq}ʊ-mwe \textbf{na}=\textbf{mu}-\textbf{lek}-\textbf{e} ʊ-kʊ-lw-a!\textup{''}\\
then now/but 9-\textsc{pst}-come-\textsc{pfv} \textsc{aug}-9-dog 9-\textsc{pst}-say-\textsc{pfv} \phantom{\lq\lq}\textsc{aug}-\textsc{2pl} \textsc{com}=\textsc{2pl}-let-\textsc{subj} \textsc{aug}-15-fight-\textsc{fv}\\\glt `Then Dog came. He said ``You (pl.), now stop fighting!''{}' [Monkey and Tortoise]
\ex \label{exnaSUBJ2pl2}\gll a-lɪ koo=kʊʊgʊ kajamba? keet-a, n-ga-m̩-bon-a ʊ-kw-is-a kʊ-kʊ-n-geet-a. \textbf{na}=\textbf{mu}-\textbf{bʊʊk}-\textbf{e}, \textbf{mu}-\textbf{ka}-\textbf{n}-\textbf{koolel}-\textbf{e}, \textbf{mu}-\textbf{ka}-\textbf{n̩}-\textbf{dond}-\textbf{e}! iis-e a-n-geet-e\\
1-\textsc{cop} \textsc{ref.17}=where tortoise(1) look-\textsc{fv} \textsc{1sg}-\textsc{neg}-1-see-\textsc{fv} \textsc{aug}-15-come-\textsc{fv} 17-15-\textsc{1sg}-watch-\textsc{fv} \textsc{com}=\textsc{2pl}-go-\textsc{subj} \textsc{2pl}-\textsc{itv}-1-call-\textsc{subj} \textsc{2pl}-\textsc{itv}-1-search-\textsc{subj} 1.come-\textsc{subj} 1-\textsc{1sg}-watch-\textsc{subj}\\
\glt `Where is Tortoise?  ‎‎Look, I haven't seen it coming to see me.  ‎‎Go (pl.) right now, call it, find it! It must come and see me.' [Lion and Tortoise]
\end{exe}

The subjunctive is also used for expressing obligation, which ranges from a weaker notion of necessity to recommendations.
\begin{exe}
\ex \gll j-oope ʊ-n-kiikʊlʊ \textbf{a}-\textbf{j}-\textbf{ege} n=ɪ-n-dwanga ɪ-j-aa kʊ-meny-el-a n=ʊ-kʊ-tumul-ɪl-a ɪ-mi-piki\\
1-also \textsc{aug}-1-woman 1-be(come)-\textsc{ipfv.subj} \textsc{com}=\textsc{aug}-9-axe \textsc{aug}-9-\textsc{assoc} 15-chop-\textsc{appl}-\textsc{fv} \textsc{com}=\textsc{aug}-15-cut-\textsc{appl}-\textsc{fv} \textsc{aug}-4-tree\\
\glt \lq And a woman should have an axe for splitting with and cutting trees with.' [Types of tools in the home]
\ex \gll a-ka-a bʊ-bɪlɪ a-ma-pamba a-ga-a kʊ-jeng-el-a \textbf{gw}-\textbf{ijʊʊl}-\textbf{e} ʊ-kʊ-fyatʊl-a jʊ$\sim$jʊʊ-gwe n=ʊ-kʊ-kosy-a\\
\textsc{aug}-12-\textsc{assoc} \textsc{14}-two \textsc{aug}-6-brick \textsc{aug}-6-\textsc{assoc} 15-build-\textsc{appl}-\textsc{fv} \textsc{2sg}-work\_hard-\textsc{subj} \textsc{aug}-15-make\_bricks-\textsc{fv} \textsc{redupl}$\sim$1-\textsc{2sg} \textsc{com}=\textsc{aug}-15-set\_on\_fire-\textsc{fv}\\
\glt \lq ‎‎Secondly, you should try yourself to make bricks to build with and bake them.' [How to build modern houses]
\end{exe}

Conceptually close to expressing obligation, the subjunctive is found in deliberative and permissive interrogation:
\begin{exe}
\ex \gll tʊ-tɪ=bʊle na=a-ba-anɪɪtʊ!\\
\textsc{1pl}-say.\textsc{subj}=how \textsc{com}=\textsc{aug}-2-our\_child\\
\glt `‎‎What should we do with our children!' [Thieving monkeys]
\ex \gll lɪlɪno kʊʊ-many-a, fi-ki ʊ-ti-kw-amul-a bo n-gʊ-kʊ-laalʊʊsy-a? bʊle \textbf{n}-\textbf{gʊ}-\textbf{kom}-\textbf{e} n=ɪ-kɪ-buli?\\
now/today \textsc{prs}.\textsc{1sg}-know-\textsc{fv} 8-what \textsc{2sg}-\textsc{neg}-\textsc{prs}-answer-\textsc{fv} as \textsc{1sg}-\textsc{prs}-\textsc{2sg}-ask-\textsc{fv} \textsc{q} \textsc{1sg}-\textsc{2sg}-hit-\textsc{subj} \textsc{com}=\textsc{aug}-7-fist\\
\glt `Now you'll get to know me, why don't you answer when I'm asking you? Should I hit you with the fist?' [Saliki and Hare]
\end{exe}

The subjunctive is also used for volitives (\ref{exSUBJvolitive1}, \ref{exSUBJvolitive2}) and, closely related, to announce what one is about to do (\ref{exSUBJannounce}).

\begin{exe}
\ex \label{exSUBJvolitive1}
\gll kyala \textbf{a}-\textbf{kʊ}-\textbf{tʊʊl}-\textbf{e}\\
God 1-\textsc{2sg}-help-\textsc{subj}\\
\glt `May God help you.' [overheard]
\ex \label{exSUBJvolitive2}
\gll ʊ-n-nino iibiibwe fi-mo kʊ-no mw-a-jaat-aga, a-t-ile \textbf{n}-\textbf{ummw}-\textbf{ag}-\textbf{ege} papaa$\sim$pa\\
\textsc{aug}-1-your\_companion \textsc{1}.forget.\textsc{pfv} 8-one 17-\textsc{prox} \textsc{2pl}-\textsc{pst}-walk-\textsc{ipfv} 1-say-\textsc{pfv} \textsc{2sg}-1-find-\textsc{ipfv.subj} \textsc{redupl}$\sim$\textsc{prox.16}\\
\glt `Your friend has forgotten something while you were talking a walk, he said ``I want to meet him right here.''{}' [Hare and Spider]
\ex \label{exSUBJannounce}
\gll po mwa=n-gambɪlɪ a-a-tɪ ``hee. po \textbf{m}-\textbf{bʊʊk}-\textbf{e} kw-a kajamba kʊ-kʊ-mel-a ɪɪ-heela sy-angʊ''. po a-lɪnkw-end-a,  a-lɪnkw-end-a,  a-lɪnkw-end-a\\
then matronym=9-monkey 1-\textsc{subsec}-say \phantom{\lq\lq}\textsc{interj} then \textsc{1sg}-go-\textsc{subj} 17-\textsc{assoc} tortoise(1) 17-15-claim-\textsc{fv} \textsc{aug}-money(10) 10-\textsc{poss.1sg} then 1-\textsc{narr}-walk/travel-\textsc{fv} 1-\textsc{narr}-walk/travel-\textsc{fv} 1-\textsc{narr}-walk/travel-\textsc{fv}\\
\glt `So Mr. Monkey said \textup{\lq\lq}I'll go to Tortoise to claim my money\textup{''}. He walked and walked and walked.' [Monkey and Tortoise] 
\end{exe}

Following \textit{mpaka} \lq no matter what', the subjunctive expresses that the eventuality will or must be fulfilled under all conditions. Thus, in (\ref{exSubjunctiveMpaka1}) it is used in a promise, while in (\ref{exSubjunctiveMpaka2}), an excerpt from a procedural text, it marks the denoted instruction as a necessity, whereas a bare subjunctive could be understood as just one step among various others.
\begin{exe}
\ex\label{exSubjunctiveMpaka1}
\gll n-gʊ-bʊʊl-a ʊkʊtɪ \textbf{mpaka} \textbf{n}-\textbf{iis}-\textbf{e} kɪ-laabo\\
1\textsc{sg}-\textsc{prs}-tell-\textsc{fv} \textsc{comp} no\_matter\_what \textsc{1sg}-come-\textsc{subj} 7-tomorrow\\
\glt `I promise that I will come tomorrow.' [ET]
\ex\label{exSubjunctiveMpaka2} 
\gll kangɪ \textbf{mpaka} \textbf{ʊ}-\textbf{si}-\textbf{keet}-\textbf{e} taasi ɪ-n-dalama ɪ-si ʊ-lɪ na=syo muu-nyambɪ, pamopeene n=ʊ-tʊ-ndʊ ʊ-tʊ-ngɪ ʊ-tʊ tʊ-bagiile ʊ-kʊ-kʊ-tʊʊl-a kʊ-m-bombo ɪ-jo\\
again no\_matter\_what \textsc{2sg}-10-look-\textsc{subj} first \textsc{aug}-10-money \textsc{aug}-10 \textsc{2sg}-\textsc{cop} \textsc{com}=\textsc{ref}.10 18-pocket(9) together \textsc{com}=\textsc{aug}-13-thing \textsc{aug}-13-other \textsc{aug}-\textsc{prox}.13 13-be\_able.\textsc{pfv} \textsc{aug}-15-\textsc{2sg}-help-\textsc{fv} 17-9-work \textsc{aug}-\textsc{ref}.9\\
\glt `Again, you should look first at the money which you have in your pocket, together with other things which can help you in this work.' [How to build modern houses]
\end{exe}

Finally, the subjunctive is also sometimes used to elaborate on states-of-affairs construed with the past imperfective\is{tense!past}\is{aspect!imperfective} in its habitual/generic\is{aspect!habitual}\is{aspect!generic} reading (\sectref{PastImperfective}), as well as with futurates\is{futurate} (Chapter \ref{Futurates}).\footnote{\citet{CarlsonR1992} observes this kind of dual function for categories often labelled \lq subjunctive' in a number of West and East African languages. However, in the languages discussed by Carlson, these uses are commonly more generalized and fulfil the functions covered by the Nyakyusa \isi{narrative tense} (\sectref{NarrativeTense}) and \isi{subsecutive} (\sectref{Subsecutive}).} This is extremely rare in the present data, the only clear case being given in (\ref{exPSTIPFVSUBJIPFV}). The additional example in (\ref{exFuturateSUBJIPFV}) is taken from HIV prevention materials created by SIL International (orthography adapted). A few more instances are found in older text collections (\citealt{BergerP1933}; \citealt{BusseJ1942}; \citeyear{BusseJ1949}), but even there this usage seems far from obligatory.  \citet[130]{BotneR2008}, for neighbouring Ndali,\il{Ndali} lists a number of examples of subjunctive uses under the label \lq coincident future'. Two of his examples are to all appearances also continuations of past generics.\is{tense!past}\is{aspect!generic}

\begin{exe}
\ex \label{exPSTIPFVSUBJIPFV} \gll po ly-a-pɪmb-aga ii-pango po$\sim$p-oosa pa-la li-kʊ-bʊʊk-a. po ly-and-aga ʊ-kʊ-kʊb-a ii-pango lɪ-la. po a-ba-ndʊ \textbf{ba}-\textbf{mog}-\textbf{ege}\\
then 5-\textsc{pst}-carry-\textsc{ipfv} 5-type\_of\_guitar \textsc{redup}$\sim$-16-all 16-\textsc{dist} 5-\textsc{prs}-go-\textsc{fv} then 5-\textsc{pst}.begin-\textsc{ipfv} \textsc{aug}-15-beat-\textsc{fv} 5-type\_of\_guitar 5-\textsc{dist} then \textsc{aug}-2-person 2-dance-\textsc{ipfv.subj}\\
\glt \lq It (the monster) carried the guitar wherever it went. It would begin to play that guitar. People would then dance.' [Monster with guitar]
\ex \label{exFuturateSUBJIPFV} 
\gll paapo mwe ba-ana ʊ-mwe mu-lɪ ba-pɪɪna, ba-li=ko a-ba-nyambala ba-mo a-ba b-isakʊ-ba-syob-aga n=ʊ-kʊ-peefy-a\\
because \textsc{2pl} 2-child \textsc{aug}-\textsc{2pl} \textsc{2pl}-\textsc{cop} 2-orphan 2-\textsc{cop}=17 \textsc{aug}-2-man 2-one \textsc{aug}-\textsc{prox.2} 2-\textsc{indef.fut}-\textsc{2pl}-cheat-\textsc{ipfv} \textsc{com}=\textsc{aug}-15-tempt-\textsc{fv}\\
\glt \lq Because your are orphans, there are some men who might try to persuade you.'
\sn \gll \textbf{ba}-\textbf{ba}-\textbf{p}-\textbf{ege} ɪ-fi-ndʊ n=ʊ-tʊ-ndʊ ʊ-tʊ-ngɪ ʊkʊtɪ mu-logw-ege na=bo\\
2-\textsc{2pl}-give-\textsc{ipfv.subj} \textsc{aug}-8-food \textsc{com}=\textsc{aug}-13-thing \textsc{aug}-13-other \textsc{comp} \textsc{2pl}-copulate-\textsc{ipfv.subj} \textsc{com}=\textsc{ref.2}\\
\glt \lq They will give you food and presents so that you have sex with them.' [Kande's Story]\footnotemark
\protect\footnotetext{\url{https://www.nyakyusalanguage.com/sites/all/libraries/pdf.js/web/viewer.html?file=/en/file/30/force_download} (10 November, 2020). The \ili{Swahili} version of this texts, on which the translation to Nyakyusa is based, uses an infinitive-based construction instead of the subjunctive mood.}
\end{exe}


Lastly, imperfective\is{aspect!imperfective} -\textit{ege} is highly sensitive to context and yields a range of readings including general or habitual\is{aspect!habitual} (\ref{exSUBJegeGeneral}, \ref{exSUBJegeHab}) and continuous (\ref{exSUBJegeContinuous1}). It also has a mitigating function, thus marking a recommendation in (\ref{exSUBJegeRecommendation}).
\begin{exe}
\ex  \label{exSUBJegeGeneral} \gll gwe n-kiikʊlʊ \textbf{ʊ}-\textbf{n̩}-\textbf{jab}-\textbf{ɪl}-\textbf{ege} ʊ-n̩-dʊmego ɪ-kɪ-jabo kɪ-nywamu fiijo\\
\textsc{2sg} 1-woman \textsc{2sg}-1-divide-\textsc{appl}-\textsc{ipfv.subj} \textsc{aug}-1-your\_husband \textsc{aug}-7-share 7-big \textsc{intens}\\
\glt `You, woman must always [while pregnant] give your husband a good share [of the food].' [Pregnant women]
\ex \label{exSUBJegeHab} \gll ɪɪ-fubu j-aal-iitiike looli jɪ-lɪnkʊ-n-sʊʊm-a jɪ-lɪnkʊ-tɪ, ``\textbf{gw}-\textbf{is}-\textbf{enge}=\textbf{ko} kʊ-my-angʊ kʊ-kʊ-m-band-a ɪ-fi-londa\\
\textsc{aug}-hippo(9) 9-\textsc{pst}-agree.\textsc{pfv} but 9-\textsc{narr}-1-beg-\textsc{fv} 9-\textsc{narr}-say \phantom{\lq\lq}\textsc{2sg}-come-\textsc{ipfv.subj}=17 17-4-\textsc{poss.1sg} 17-15-\textsc{1sg}-heal\_wound-\textsc{fv} \textsc{aug}-8-wound\\
\glt `Hippo agreed, but asked him [Hare], \textup{\lq\lq}You should come (regularly) to my home to put hot compresses on my sores.\textup{''}' [Hare and Hippo]

\ex \label{exSUBJegeContinuous1} \gll a-ka-a kw-and-a \textbf{g}-\textbf{ʊʊl}-\textbf{ege} a-ma-lata manandɪ$\sim$ma-nandɪ, mpaka ga-fik-ɪl-e ɪ-m-balɪlo ɪ-jɪ kʊ-jɪ-lond-a\\
\textsc{aug}-12-\textsc{assoc} 15-begin-\textsc{fv} \textsc{2sg}-buy-\textsc{ipfv.subj} \textsc{aug}-6-corrugated\_iron \textsc{redupl}$\sim$6-little until 6-arrive-\textsc{appl}-\textsc{subj} \textsc{aug}-9-number \textsc{aug}-\textsc{prox.9} \textsc{2sg.prs}-9-want-\textsc{fv}\\
\glt `To start with, you should be buying corrugated iron, little by little, until you have enough.' [How to build modern houses]

\ex  \label{exSUBJegeRecommendation}
\gll n-gʊ-kʊ-bʊʊl-a mu-ndʊ ʊ-gwe, \textbf{gw}-\textbf{eg}-\textbf{ege} fy-osa ɪ-fy-ako ɪ-fi ʊ-lɪ na=fyo pamopeene n=ʊ-n-kasigo na=a-ba-anaako n=ɪ-fi-nyamaana fy-ako fy-osa.\\
\textsc{1sg}-\textsc{prs}-\textsc{2sg}-tell-\textsc{fv} 1-person \textsc{aug}-\textsc{2sg} \textsc{2sg}-take-\textsc{ipfv.subj} 8-all \textsc{aug}-8-\textsc{poss.2sg} \textsc{aug}-\textsc{prox.8} \textsc{2sg}-\textsc{cop} \textsc{com}=\textsc{ref.8} together \textsc{com}=\textsc{aug}-1-your\_wife \textsc{com}=\textsc{aug}-2-your\_child \textsc{com}=\textsc{aug}-8-animal 8-\textsc{poss.2sg} 8-all\\
\glt `I'm telling you, you should take all your belongings together with your wife and children and all your animals.'
\sn \gll \textbf{ʊ}-\textbf{bʊʊk}-\textbf{ege} kʊ-bʊ-tali komma ʊ-kʊ-buj-a kangɪ kʊ-no, ʊ-nga-ba-bʊʊl-aga na=a-ba-palamani ba-ako\\
\textsc{2sg}-go-\textsc{ipfv.subj} 17-14-long \textsc{proh} \textsc{aug}-15-return-\textsc{fv} again 17-\textsc{prox} \textsc{2sg}-\textsc{neg.subj}-2-tell-\textsc{ipfv} \textsc{com}=\textsc{aug}-2-neighbour 2-\textsc{poss.2sg}\\
\glt  \lq You should go far and never return here, you shouldn't tell not even your neighbours.' [Selfishness kills]
\end{exe}

\subsubsection{Subjunctive uses in subordinate clauses}\label{SubjunctiveSubordinate}
The subjunctive features in the complements\is{subordinate clauses!complement clause} of \isi{modality} and manipulation verbs, where it alternates with the infinitive.\is{infinitive} If the subject of the main verb is co-referential with the subject of the complement verb, the infinitive is the more common form. The subjunctive is also possible, however (\ref{exSubjunctiveModalitySameSubject}); also see 
(\ref{exNARRopeningSentence1}) on p.\nobreakspace\pageref{exNARRopeningSentence1}, (\ref{exProcliticAaQuestionHareSpider}) on p.\nobreakspace\pageref{exProcliticAaQuestionHareSpider} and
(\ref{exDistalKaSubordinateAbility}) on p.\nobreakspace\pageref{exDistalKaSubordinateAbility}. Subjunctive complements are sometimes introduced by the complementizer \textit{ʊkʊtɪ}, but this is optional in most cases. If the subjects of the two verbs differ, the subjunctive is most commonly used (\ref{exSubjunctiveModalityDifferentSubject}); this includes partly disjunctive reference (see ex. \ref{exDesiderativeHareHippo} on p.\nobreakspace\pageref{exDesiderativeHareHippo}). Alternatively, the complement verb figures as an oblique infinitival clause and its notional subject as the object of the main verb (\ref{exInfinitiveModalityDifferentSubject}).

\begin{exe}
\ex \label{exSubjunctiveModalitySameSubject}
\gll i-kʊ-lond-a \textbf{eeg}-\textbf{e} ii-peasi lɪ-mo\\
1-\textsc{prs}-want-\textsc{fv} 1.take-\textsc{subj} 5-pear(<SWA) 5-one\\
\glt \lq He wants to take one pear.' [Elisha pear story]

\largerpage
\ex \label{exSubjunctiveModalityDifferentSubject}
\gll gwe n̩-ganga n-gʊ-sʊʊm-a \textbf{ʊ}-\textbf{n}-\textbf{dʊʊl}-\textbf{e} \textbf{ʊ}-\textbf{m}-\textbf{b}-\textbf{e}=\textbf{po} ʊ-n-kota ʊ-gw-a lʊ-gano\\
\textsc{2sg} 1-healer \textsc{1sg}-\textsc{prs}-beg-\textsc{fv} \textsc{2sg}-\textsc{1sg}-help-\textsc{subj} \textsc{2sg}-\textsc{1sg}-give-\textsc{subj}=16 \textsc{aug}-3-medicine \textsc{aug}-3-\textsc{assoc} 11-love\\
\glt \lq You, witch doctor, I beg you to help me, give me a love potion.' [Mfyage turns into a lion] 

\ex \label{exInfinitiveModalityDifferentSubject}
\gll ʊ-malafyale a-ba-lagiile b-oosa \textbf{ʊ}-\textbf{kw}-\textbf{is}-\textbf{a} kʊ-my-ake\\
\textsc{aug}-chief(1) \textsc{aug}-2-order.\textsc{pfv} 2-all \textsc{aug}-15-come-\textsc{fv} 17-4-\textsc{poss.sg}\\
\glt \lq The chief ordered everybody to come to his place.' [ET]
\end{exe}

Related to the preceding examples, the subjunctive is also used in indirect orders:
\begin{exe}
\ex \gll ɪ-n-galamu jɪ-lɪnkʊ-fi-bʊʊl-a ɪ-fi-nyamaana fy-osa ʊkʊtɪ \textbf{f}-\textbf{iis}-\textbf{e} kʊ-lʊ-komaano\\
\textsc{aug}-9-lion 9-\textsc{narr}-8-tell-\textsc{fv} \textsc{aug}-8-animal 8-all \textsc{comp} 8-come-\textsc{subj} 17-11-meeting\\
\glt \lq Lion told all the animals to come to the meeting.' [Hare and Chameleon]
\end{exe}

The subjunctive further alternates with the \isi{infinitive} after predicative expressions of disapproval, approval or preference when reference is made to eventualities that are either not actualized (\ref{exSubjunctiveSubNonActualized1}, \ref{exSubjunctiveSubNonActualized2}) or are potential (\ref{exSubjunctivPotential}). These clauses are sometimes introduced by the complementizer \textit{ʊkʊtɪ}. For an example featuring the infinitive, see (\ref{exInfinitiveApproval}) on p.\nobreakspace\pageref{exInfinitiveApproval}.
 
\begin{exe}
\ex \label{exSubjunctiveSubNonActualized1} \gll kanunu \textbf{ʊ}-\textbf{pungusy}-\textbf{e}=\textbf{po} ɪ-fy-ɪma fy-ako bʊle?\\
well \textsc{2sg}-reduce-\textsc{subj}=\textsc{part} \textsc{aug}-8-thigh 8-\textsc{poss.2sg} \textsc{q}\\
\glt `Isn't it good that you should lose weight [lit. reduce a bit] from your thighs?' [Hare and Hippo]
\ex \label{exSubjunctiveSubNonActualized2} \gll kyajɪpo \textbf{tʊ}-\textbf{bʊʊk}-\textbf{ege}\\
preferable \textsc{1pl}-go-\textsc{ipfv.subj}\\
\glt \lq We'd better get going.' [ET]
\ex \label{exSubjunctivPotential}
\gll l-oope lʊ-ka-a lʊ-sumo lʊ-nunu ʊkʊtɪ ʊ-n-kiikʊlʊ \textbf{a}-\textbf{bomb}-\textbf{ege} ɪ-m-bombo n-gafu bo ʊ-n̩-dʊme a-li=ko m-ʊʊmi\\
11-also 11-\textsc{neg}.be(come)-\textsc{fv} 11-custom 11-good \textsc{comp} \textsc{aug}-1-woman 1-work-\textsc{ipfv.subj} \textsc{aug}-10-work 10-difficult as \textsc{aug}-1-husband 1-\textsc{cop}=17 1-live\\
\glt \lq ‎‎Also it is not good manners for a woman to do hard work while the husband is alive.' [Division of labour]
\end{exe}

Similarly, it is used in a variety of subordinate clauses with reference to non-actualized eventualities:
\begin{exe}
\ex \gll iijʊʊl-ege ʊ-kʊ-fi-lond-a \textbf{mpaka} \textbf{a}-\textbf{fy}-\textbf{ag}-\textbf{e}\\
1.work\_hard-\textsc{ipfv.subj} \textsc{aug}-15-8-search-\textsc{fv} until 1-8-find-\textsc{subj}\\
\glt \lq He should try hard to get them (tools) until he gets them.' [Type of tools in the home]
\ex \gll po kangɪ po a-lɪnkʊ-buj-a ʊ-kw-is-a pa-ka-aja pa-kʊ-n̩-guul-ɪl-a kajamba \textbf{ʊkʊtɪ} \textbf{iis}-\textbf{e}\\
then again then 1-\textsc{narr}-return-\textsc{fv} \textsc{aug}-15-come-\textsc{fv} 16-12-homestead 16-15-1-wait-\textsc{appl}-\textsc{fv} tortoise(1) \textsc{comp} 1.come-\textsc{subj}\\
\glt `‎‎Then he [Monkey] returned to the home and waited for Tortoise to come.' [Monkey and Tortoise]
\end{exe}

Lastly, the subjunctive is also used in purpose clauses\is{subordinate clauses!purpose clause} (\ref{exSubjunctivePurposeClause1}) and result clauses\is{subordinate clauses!result clause} (\ref{exSubjunctiveResultClause1}), which are introduced by the complementizer \textit{ʊkʊtɪ}.

\begin{exe}
\ex\label{exSubjunctivePurposeClause1}\gll popaa$\sim$po kalʊlʊ a-lɪnkʊ-bop-a fiijo \textbf{ʊkʊtɪ} \textbf{a}-\textbf{lʊ}-\textbf{kɪnd}-\textbf{e} ʊ-lw-ifi\\
\textsc{redupl}$\sim$then hare(1) 1-\textsc{narr}-run-\textsc{fv} \textsc{intens} \textsc{comp} 1-11-pass-\textsc{subj} \textsc{aug}-11-chameleon\\
\glt `Hare ran fast to beat Chameleon.' [Hare and Chameleon]
\ex\label{exSubjunctiveResultClause1}\gll ba-ka-a-fi-lek-aga panja \textbf{ʊkʊtɪ} \textbf{fi}-\textbf{nyop}-\textbf{ege} n=ɪɪ-fula\\
2-\textsc{neg}-\textsc{pst}-8-let-\textsc{ipfv} outside \textsc{comp} 8-get\_wet-\textsc{ipfv.subj} \textsc{com}=\textsc{aug}-rain(9)\\
\glt `They would not leave them [animals] outside so that they [animals] would get wet in the rain.' [Nyakyusa houses of long ago]
\end{exe}


\subsection{Complex constructions involving the subjunctive}\label{ComplexConructionsWithSubjunctive}
A number of formally biclausal constructions contain subjunctive verbs as the second element. What is common to all of them is that the first element is a form of the versatile verb \textit{tɪ} \lq  say' (\sectref{defectiveti}).

A construction featuring an inflected form of \textit{tɪ} together with a co-referential subjunctive gives projective and conative readings.\footnote{This construction is fairly common in south-eastern Bantu \citep[153]{GueldemannT1996} and has grammaticalized to future tense marking in a number of languages of Bantu zones M and N \citep{BotneR1998a}.} This is often understood as a frustrated intent:
\begin{exe}
\ex \label{exConativeFrustrated2}\gll bo \textbf{i}-\textbf{kʊ}-\textbf{tɪ} \textbf{a}-\textbf{kol}-\textbf{ege} ʊ-lw-igi a-lʊ-kab-e kw-a-t-ile \textup{\lq\lq}kóo\textup{''}. a-a-kuut-ile a-a-t-ile \textup{\lq\lq}ɪɪtaata! m-fw-ile hɪhɪhɪɪ\textup{''}\\
as 1-\textsc{prs}-say 1-grasp/hold-\textsc{ipfv.subj} \textsc{aug}-11-door 1-11-get-\textsc{subj} 17-\textsc{pst}-say-\textsc{pfv} \phantom{\lq\lq}of\_sickle\_swinging 1-\textsc{pst}-cry-\textsc{pfv} 1-\textsc{pst}-say-\textsc{pfv} \phantom{\lq\lq}oh\_father \textsc{1sg}-die-\textsc{pfv} of\_crying\\
\glt `When he tried to grab the door and get hold of it, there was the sound ``kóo!'' [of a sickle]. He cried ``Oh father! I am dead, hihihii!''{}' [Wage of the thieves]
\end{exe}

This, however, turns out to be an implicature rather than part of the meaning of the construction. This is illustrated in (\ref{exConativeImplicature1}, \ref{exConativeImplicature2}). In (\ref{exConativeImplicature1}), the referential demonstrative \textit{syo} serves as the copulative\is{copula} of the cleft sentence and refers to the information Hare has just given the woman and which has caused her to change her mind. In (\ref{exConativeImplicature2}) the intended action is carried out in the following sentence.
\begin{exe}
\ex \label{exConativeImplicature1} \gll kalʊlʊ a-a-hobwike fiijo paapo ʊ-n-kiikʊlʊ a-sambwike, Kalʊlʊ a-lɪnkʊ-tɪ \textup{\lq\lq}syo ɪ-si \textbf{n}-\textbf{d}-\textbf{ile} \textbf{n}-\textbf{gʊ}-\textbf{bʊʊl}-\textbf{e} ʊkʊtɪ ʊ-many-e\textup{”}\\
hare(1) 1-\textsc{pst}-be(come)\_happy.\textsc{pfv} \textsc{intens} because \textsc{aug}-1-woman 1-rebel.\textsc{pfv} hare(1) 1-\textsc{narr}-say \phantom{\lq\lq}\textsc{ref.10} \textsc{aug}-\textsc{prox.10} \textsc{1sg}-say-\textsc{pfv} \textsc{1sg}-\textsc{2sg}-tell-\textsc{subj} \textsc{comp} \textsc{2sg}-know-\textsc{subj} \\
\glt `Hare was very happy because the woman had changed her mind, he said ``That's what I wanted to tell you so that you know it.''{}' [Hare and Spider]
\ex \label{exConativeImplicature2} \gll po leelo a-lɪnkw-is-a ʊ-gw-a bʊ-haano [...] po j-oope \textbf{a}-\textbf{lɪnkʊ}-\textbf{tɪ} \textbf{ɪmb}-\textbf{ege} ʊ-lw-ɪmbo. j-oope a-lɪnkw-ɪmb-a a-lɪnkʊ-tɪ\\
then but 1-\textsc{narr}-come-\textsc{fv} \textsc{aug}-1-\textsc{assoc} 14-five {} then 1-also 1-\textsc{narr}-say 1.sing-\textsc{ipfv.subj} \textsc{aug}-11-song 1-also 1-\textsc{narr}-sing-\textsc{fv} 1-\textsc{narr}-say\\
\glt \lq Then the fifth (child) came. It also wanted to sing the song. It also sang:' [Children and Snake]
\end{exe}
The collocation of \textit{fikʊtɪ}, which is the \isi{simple present} of \textit{tɪ} with a noun class 8 subject, together with a subjunctive verb as its complement denotes various kinds of dynamic and deontic necessity.\is{modality} Only one token of this collocation is attested in the data (\ref{exFikutiTools}). Speakers accepted this construction in elicitation, but considered it somewhat archaic. It is frequent in older text collections. (\ref{exFikutiAlt1}, \ref{exFikutiAlt3}) illustrate this (orthography adapted).

\begin{exe}
\ex \label{exFikutiTools} \gll kʊkʊtɪ mu-ndʊ \textbf{fi}-\textbf{kʊ}-\textbf{tɪ} \textbf{a}-\textbf{j}-\textbf{ege} n=ʊ-tʊ-ndʊ t-oosa ʊ-tʊ tʊ-kʊ-lond-igw-a ʊ-kʊ-bomb-el-a ɪ-m-bombo sy-ake\\
every 1-person 8-\textsc{prs}-say 1-be(come)-\textsc{ipfv.subj} \textsc{com}=\textsc{aug}-13-thing 13-all \textsc{aug}-\textsc{prox.13} 13-\textsc{prs}-want-\textsc{pass}-\textsc{fv} \textsc{aug}-15-work-\textsc{appl}-\textsc{fv} \textsc{aug}-10-work 10-\textsc{poss.sg}\\
\glt \lq ‎It is appropriate for every person to have all the things which are needed to do their work with.' [Types of tools in the home] %deontic necessity

\ex \label{exFikutiAlt1} \gll lɪnga a-li=po ʊ-mu-ndʊ ʊ-jʊ i-kʊ-job-a na Kyala, po \textbf{fi}-\textbf{kʊ}-\textbf{tɪ} g-oope ʊ-m-piki gʊ-mo \textbf{gʊ}-\textbf{j}-\textbf{e}=\textbf{po}\\
if/when 1-\textsc{cop}=16 \textsc{aug}-1-person \textsc{aug}-\textsc{prox.1} 1-\textsc{prs}-speak-\textsc{fv} \textsc{com} god then 8-\textsc{prs}-say 3-also \textsc{aug}-3-tree 3-one 3-be(come)-\textsc{subj}=16\\
\glt \lq When there is a person speaking with God, there must also be a tree.'\\\citep[210]{BusseJ1949} %dynamic situational necessity

\ex\label{exFikutiAlt3}
\gll mwe ba-anangʊ, lɪlɪno a-pa ʊ-n-kiikʊlʊ ʊ-jʊ a-a-lond-aga ʊ-kʊ-ba-gog-a, po \textbf{fi}-\textbf{kʊ}-\textbf{tɪ} na=nuuswe \textbf{tʊ}-\textbf{n̩}-\textbf{gog}-\textbf{e}\\
\textsc{1pl} 2-my\_child now/today \textsc{aug}-\textsc{prox.16} \textsc{aug}-1-woman \textsc{aug}-\textsc{prox.1} 1-\textsc{pst}-want-\textsc{ipfv} \textsc{aug}-15-\textsc{2pl}-kill-\textsc{fv} then 8-\textsc{prs}-say \textsc{com}=\textsc{com.1pl} \textsc{1pl}-1-kill-\textsc{subj}\\
\glt `My children, now that this woman has tried to kill you, it is up to us to kill her.' \citep[143]{BergerP1933} 
\end{exe}

Lastly, the subjunctive of \textit{tɪ} itself, together with a subjunctive complement, expresses obligation.\footnote{The lack of a simple present prefix is evidence for this form not belonging to the indicative paradigms.} Interestingly, in this case the subjunctive of \textit{tɪ} can be marked for past tense (\ref{exSUBJati}).

\begin{exe}
\ex \gll \textbf{ba}-\textbf{tɪ} \textbf{b}-\textbf{iis}-\textbf{e} \textbf{m}-\textbf{ba}-\textbf{p}-\textbf{e}=\textbf{po} ɪɪ-heela\\
2-say 2-come-\textsc{subj} \textsc{1sg}-2-give-\textsc{subj}=16 \textsc{aug}-money(10)\\
\glt `They must come so that I give them the money.' [ET]
\ex\label{exSUBJati}
\gll kʊʊ-nongwa j-aa kʊ-tɪ ɪɪ-ny-iiho sy-a ba-Nyakyʊsa si-kʊ-lond-a ʊkʊtɪ \textbf{a}-\textbf{a}-\textbf{tɪ} \textbf{a}-\textbf{m̩}-\textbf{bonol}-\textbf{e} taasi po bi-kʊ-keet-an-aga n=ʊ-kʊ-ponani-a kɪsita kʊ-tiil-an-a\\
17-issue(9) 9-\textsc{assoc} 15-say \textsc{aug}-10-custom 10-\textsc{assoc} 2-Ny. 10-\textsc{prs}-want-\textsc{fv} \textsc{comp} 1-\textsc{pst}-say.\textsc{subj} 1-1-pay\_brideprice-\textsc{subj} yet then 2-\textsc{mod.fut}-look-\textsc{recp}-\textsc{mod.fut} \textsc{com}=\textsc{aug}-15-greet.\textsc{recp}-\textsc{fv} without 15-fear-\textsc{recp}-\textsc{fv}\\
\glt `‎Because the traditions of the Nyakyusa people require that he should have paid her off first, then they shall look at each other and greet each other without fearing.' [Should she save a life\ldots]
\end{exe}

\subsection{Distal/itive \textit{ka}-}\label{DistalKa}
\is{itive|(}
The subjunctive can be combined with a prefix \textit{ka}-, which is commonly labelled \textit{distal} or \textit{itive}. This morpheme \textit{ka}- is glossed as \textsc{itv} throughout this study, with \textsc{dist} being reserved for the distal demonstrative. Other common labels in the Bantuistic literature include \textit{andative} or \textit{ka movendi} \citep[242]{NurseD2008}. Note that the distal/itive does not constitute a mood of its own, but rather a deictic category that in Nyakyusa is limited to the subjunctive mood. As the name suggests, distal/itive \textit{ka}- locates the state-of-affairs away from the deictic centre (see also \citealt{BotneR1999}).\footnote{\citet[106--109]{LusekeloA2013} lists this as \lq\lq narrative ka-'' on the sole basis that in other Bantu languages a segmentally identical prefix is used for several kinds of past time reference, only to then state (p. 108) that it ``represents future tense as well in Kinyakyusa''. All examples listed by Lusekelo include subjunctive final -\textit{e}(\textit{ge}) and, although they sometimes appear in past contexts, from his own translation as well as from the distribution of the morpheme it is clear that all these cases belong to the subjunctive paradigm.} This often goes together with a sense of physical motion (\ref{exDistalFirstExamples}). Accordingly, it is commonly found after \textit{bʊʊka} \lq go (to)' (\ref{exDistalBuuka1}, \ref{exDistalBuuka2}).
\begin{exe}
\ex \label{exDistalFirstExamples}
\gll (ʊ-)\textbf{ka}-\textbf{sy}-\textbf{e} ɪ-fi-lombe!\\
(\textsc{2sg}-)\textsc{itv}-grind-\textsc{subj} \textsc{aug}-8-maize\\
\glt `Go grind maize!' [ET]
\ex \label{exDistalBuuka1}
\gll po leelo ʊ-bʊʊk-e kw-a mwa=n-gambɪlɪ \textbf{ʊ}-\textbf{ka}-\textbf{m̩}-\textbf{bʊʊl}-\textbf{e} ʊkʊtɪ ``taata i-kʊ-sʊʊm-a a-m-ungu, n=ɪ-m-bilipili na=a-ma-sɪmbɪ n=ii-seeke.''\\
then now/but \textsc{2sg}-go-\textsc{subj} 17-\textsc{assoc} matronym=9-monkey \textsc{2sg}-\textsc{itv}-1-tell-\textsc{subj} \textsc{comp} \phantom{\lq\lq}my\_father 1-\textsc{prs}-beg-\textsc{fv} \textsc{aug}-6-pumpkin \textsc{com}=\textsc{aug}-10-pepper \textsc{com}=\textsc{aug}-6-cocoyam \textsc{com}=5-sidedish\\
\glt `Go to Mr. Monkey and tell him ``Father begs for pumpkins, peppers, cocoyam and a sidedish.''{}'

\sn \gll \textbf{ʊ}-\textbf{ka}-\textbf{tɪ} ɪɪ-heela n-gʊ-twal-aga ʊ-n̩-dʊngʊ ʊ-gʊ gʊ-kw-is-a\\ 
 \textsc{2sg}-\textsc{itv}-say.\textsc{subj} \textsc{aug}-money(10) \textsc{1sg}-\textsc{mod.fut}-carry-\textsc{mod.fut} \textsc{aug}-3-week \textsc{aug}-\textsc{prox.3} 3-\textsc{prs}-come-\textsc{fv}\\
\glt \lq Say that the money I will bring next week.' [Monkey and Tortoise]
\ex \label{exDistalBuuka2}
\gll mu-bʊʊk-e mwe ba-ndʊ b-angʊ nuumwe \textbf{mu}-\textbf{ka}-\textbf{kol}-\textbf{e} ɪ-n-galamu \textbf{mu}-\textbf{ka}-\textbf{twal}-\textbf{e} ɪɪ-ny-ʊʊmi!\\
\textsc{2pl}-go-\textsc{subj} \textsc{2pl} 2-person 2-\textsc{poss.1sg} \textsc{com.2pl} \textsc{2pl}-\textsc{itv}-grasp/hold-\textsc{subj} \textsc{aug}-9-lion \textsc{2pl}-\textsc{itv}-carry-\textsc{subj} \textsc{aug}-9-live\\
\glt `You, my people, you too  go catch a lion and bring it alive!' [Chief Kapyungu]
\end{exe}

While distal/itive \textit{ka}- often goes together with a sense of physical \isi{motion} of the subject, this is not always the case. In (\ref{exDistalOtherSubject}), people plan to take a guitar string, which is made from the dead body of a child, to a witch doctor, who will then bring the child back to life. It is thus not the subject of the distal/itive-marked subjunctive (the witch doctor) that moves, but his actions happen at a place other than the deictic centre. See also (\ref{exPRSpakuProspectiveIntention}) on p.\nobreakspace\pageref{exPRSpakuProspectiveIntention} for an example of a \isi{motion} event that is displaced elsewhere. 

\begin{exe}
\ex \label{exDistalOtherSubject}
\gll baatɪ po ɪ-li-ndʊ ɪ-lɪ tʊ-lɪ-gog-e ʊkʊtɪ tw-eg-e ɪ-kɪ-sipa kɪ-la mw-i-pango, tʊ-twal-e kʊ-n̩-ganga. \textbf{a}-\textbf{ka}-\textbf{m̩}-\textbf{buj}-\textbf{ɪsy}-\textbf{e} ʊ-mw-ana\\
\textsc{interj} then \textsc{aug}-5-monster \textsc{aug}-\textsc{prox.5} \textsc{1pl}-5-kill-\textsc{subj} \textsc{comp} \textsc{1pl}-take-\textsc{subj} \textsc{aug}-7-string 7-\textsc{dist} 18-5-guitar \textsc{1pl}-carry-\textsc{subj} 17-1-healer 1-\textsc{itv}-1-return-\textsc{caus}-\textsc{subj} \textsc{aug}-1-child\\
\glt \lq Look, that monster, we should kill it so that we take that string in the guitar and bring it to the witch doctor. [So that] he'll make the child return.' [Monster with guitar]
\end{exe}

\largerpage
The following examples illustrate some more uses of \textit{ka}- in environments other than directives and requests.

\begin{exe}
\ex Volitive:\\
\gll po ɪɪ-sota bo jɪ-fum-ile n-kʊ-jaat-a jɪ-lɪnkʊ-tɪ \textup{\lq\lq}\textbf{n}-\textbf{ga}-\textbf{keet}-\textbf{e} ii-fumbɪ ly-angʊ\textup{''}\\
then \textsc{aug}-python(9) as 9-come\_from-\textsc{pfv} 18-15-walk-\textsc{fv} 9-\textsc{narr}-say \phantom{\lq\lq}\textsc{1sg}-\textsc{itv}-watch-\textsc{subj} 5-egg 5-\textsc{poss.1sg}\\
\glt \lq Python, when it had come from taking a walk, said \lq\lq I'll go look after my egg.''{}' [Python and woman]
\ex Hortative:\\
\gll is-aga tʊ-bʊʊk-e \textbf{tʊ}-\textbf{ka}-\textbf{m̩}-\textbf{bʊʊl}-\textbf{e} ʊ-n-kʊlʊmba gw-ɪtʊ gw-a fi-nyamaana fy-osa ɪɪ-sofu ʊkʊtɪ a-koolel-e ʊ-lʊ-komaano\\
come-\textsc{ipfv} \textsc{1pl}-go-\textsc{subj} \textsc{1pl}-\textsc{itv}-1-tell-\textsc{subj} \textsc{aug}-1-older 1-\textsc{poss.1pl} 1-\textsc{assoc} 8-animal 8-all \textsc{aug}-elephant(9) \textsc{comp} 1-call-\textsc{subj} \textsc{aug}-11-meeting\\
\glt `Come, let's go and tell our eldest among all animals, Elephant, that he should call a meeting.' [Hare and Chameleon]
\ex Subordinate clause, modality verb: \label{exDistalKaSubordinateAbility}\\
\gll m-bagiile ʊ-kʊ-kʊ-tol-a ʊ-gwe ʊ-kʊ-bop-a ʊ-lʊ-bɪlo, paapo kw-end-a panandɪ$\sim$panandɪ. m-bagiile ʊkʊtɪ \textbf{n}-\textbf{ga}-\textbf{fik}-\textbf{e} kʊ-bʊ-malɪɪkɪsyo n=ʊ-kʊ-gomok-a bo ʊ-kaalɪ ʊ-lɪ pala$\sim$pa-la n-gʊ-lek-ile\\
\textsc{1sg}-be\_able.\textsc{pfv} \textsc{aug}-15-\textsc{2sg}-beat-\textsc{fv} \textsc{aug}-\textsc{2sg} \textsc{aug}-15-run-\textsc{fv} \textsc{aug}-11-race because \textsc{2sg.prs}-walk/travel-\textsc{fv} \textsc{redupl}$\sim$a\_little \textsc{1sg}-be\_able.\textsc{pfv} \textsc{comp} \textsc{1sg}-\textsc{itv}-arrive-\textsc{subj} 17-14-finish \textsc{com}=\textsc{aug}-15-return-\textsc{fv} as \textsc{2sg}-\textsc{pers} \textsc{2sg}-\textsc{cop} \textsc{redupl}$\sim$16-\textsc{dist} \textsc{1sg}-\textsc{2sg}-let-\textsc{pfv}\\ 
\glt `I can beat you in running, because you walk slowly. I can [go] reach the end and return while you are still there where I left you.' [Hare and Chameleon] 
\ex Subordinate clause of purpose:\label{exDistalOhneBuuka2}\\
\gll a-a-bʊngeenie a-ma-tunda ʊkʊtɪ lʊmo \textbf{a}-\textbf{k}-\textbf{ʊʊl}-\textbf{ɪsy}-\textbf{e}\\
1-\textsc{pst}-gather.\textsc{pfv} \textsc{aug}-6-fruit \textsc{comp} maybe 1-\textsc{itv}-buy-\textsc{caus}-\textsc{subj}\\ 
\glt `He gathered the fruits to maybe go and sell them.' [Nicholaus Pear Story]
\end{exe}

The distal/itive subjunctive can be combined with desiderative\is{mood!desiderative} \textit{lɪ}-, yielding \textit{ka}-\textit{lɪ}-\textsc{vb}-\textit{e}-(\textit{ge}). This is discussed in \sectref{Desiderative}. Note that distal/itive \textit{ka}- is only found in affirmative forms. As with the bare subjunctive, it is negated by the \isi{negative} counterpart to the subjunctive.
\is{itive|)}
\subsection{Negative Subjunctive}\label{NegativeSubjunctive}
\is{negative|(}
The negative counterpart to the subjunctive consists of \textit{nga}- in the post-initial slot and the default final vowel -\textit{a} or imperfective -\textit{aga}. Contrary to the directive use of the affirmative subjunctive, the use of a \isi{subject marker} is obligatory in this construction.
\begin{exe}
\ex \textit{tʊngajoba} \lq we should not speak'
\end{exe}

The negative subjunctive prefix is also attested with a variant form \textit{ngɪ}-. This is much less frequent in the data and seems to be typical of the more northern variants\is{dialects} of Nyakyusa. With the first person singular, the combination of subject prefix\is{subject marker} and negative prefix yields \textit{ndɪnga}- (\ref{exNegSubj1SG}). \citet[34f]{SchumannK1899} and \citet[73f]{EndemannC1914} have \textit{n}-\textit{anga}-. The speakers consulted considered this typical of the variety of the lake-shore plains.\is{dialects}

\begin{exe}
\ex \label{exNegSubj1SG} \textit{ndɪngajoba} \lq I should not speak'
\end{exe}

The uses of the negative subjunctive essentially parallel those of its affirmative counterpart. Some of these are illustrated in the following examples.
\begin{exe}
\ex Prohibitive:\\
\gll \textbf{ʊ}-\textbf{nga}-\textbf{gel}-\textbf{a} ʊ-kʊ-ly-a ɪ-fi-ndʊ ɪ-fi!\\
\textsc{2sg}-\textsc{neg.subj}-try-\textsc{fv} \textsc{aug}-15-eat-\textsc{fv} \textsc{aug}-8-food \textsc{aug}-\textsc{prox.8}\\
\glt \lq Don't you dare eat that food!' [ET]
\ex Prohibitive plural:\\
\gll n-gʊ-ba-asim-a, looli \textbf{aa}=\textbf{mu}-\textbf{nga}-\textbf{sob}-\textbf{esy}-\textbf{a} ɪɪ-sindaano j-angʊ\\
\textsc{1sg}-\textsc{prs}-\textsc{2pl}-lend-\textsc{fv} but \textsc{fut}=\textsc{2pl}-\textsc{neg.subj}-get\_lost-\textsc{caus}-\textsc{fv} \textsc{aug}-needle(9)(<SWA) 9-\textsc{poss.1sg}\\
\glt \lq I'm lending it to you, but don't lose my needle.' [Chickens and crow]

\ex Indirect prohibitive:\\
\gll a-m-bʊʊl-ile ʊkʊtɪ \textbf{n}-\textbf{dɪnga}-\textbf{bomb}-\textbf{a}\\
1-\textsc{1sg}-tell-\textsc{pfv} \textsc{comp} \textsc{1sg}-\textsc{neg.subj}-work-\textsc{fv}\\
\glt \lq He told me not to work.' [ET]

\ex Negative Hortative:
\label{exNegativeHortative}\\
\gll \textbf{tʊ}-\textbf{nga}-\textbf{j}-\textbf{aga} n=ɪ-fi-nyonyo bo ɪɪ-fubu ɪ-jɪ j-aa-fw-ile kʊʊ-nongwa ɪ-j-aa fi-londa bo jɪ-kʊ-lond-a ʊ-bʊ-nunu\\
\textsc{1pl}-\textsc{neg.subj}-be(come)-\textsc{ipfv} \textsc{com}=\textsc{aug}-8-desire as \textsc{aug}-hippo(9) \textsc{aug}-\textsc{prox.9} 9-\textsc{pst}-die-\textsc{pfv} 17-issue(9) \textsc{aug}-9-\textsc{assoc} 8-wound as 9-\textsc{prs}-search-\textsc{fv} \textsc{aug}-14-beauty\\
\glt `We should not have desire like Hippo, who died because of his sores, when he was looking for beauty.' [Hare and Hippo]

\ex Negative Jussive: \label{exNegativeJussive}\\
\gll n̩-dʊme gw-angʊ, tʊ-lond-e fi-mo ɪ-fy-a kʊ-n-teg-el-a kalʊlʊ ʊkʊtɪ ii-kol-e tʊ-n̩-gog-e. \textbf{a}-\textbf{ng}-\textbf{and}-\textbf{ɪsy}-\textbf{a} ʊ-kʊ-ly-a ɪ-fi-lombe fy-ɪtʊ paapo i-kʊ-mal-a\\
1-husband 1-\textsc{poss.1sg} \textsc{1pl}-search-\textsc{subj} 8-one \textsc{aug}-8-\textsc{assoc} 15-1-trap-\textsc{appl}-\textsc{fv} hare(1) \textsc{comp} 1.\textsc{refl}-grasp/hold-\textsc{subj} \textsc{1pl}-1-kill-\textsc{subj} 1-\textsc{neg.subj}-begin-\textsc{caus}-\textsc{fv} \textsc{aug}-15-eat-\textsc{fv} \textsc{aug}-8-maize 8-\textsc{poss.1pl} because 1-\textsc{prs}-finish-\textsc{fv}\\
\glt `My husband, let's look for something to trap Hare, so that he gets caught and we kill him. He mustn't eat our maize again, because he's finishing it.' [Saliki and Hare]

\ex \label{exNegativePurposeClause}
Negative purpose clause:\\
\gll ba-lɪnkw-inogon-a ʊ-kʊ-tɪ \textup{\lq\lq}tʊ-bʊʊk-e tʊ-k-iip-e ɪ-ly-ʊndʊ ɪ-ly-a kʊ-gelek-el-a kʊ-mwanya\textup{''} \textbf{ʊkʊtɪ} \textbf{ba}-\textbf{ngɪ}-\textbf{toony}-\textbf{el}-\textbf{igw}-\textbf{aga}\\
2-\textsc{narr}-think-\textsc{fv} \textsc{aug}-15-say \phantom{\lq\lq}\textsc{1pl}-go-\textsc{subj} \textsc{1pl}-\textsc{itv}-pluck-\textsc{subj} \textsc{aug}-5-thatching\_grass \textsc{aug}-5-\textsc{assoc} 15-thatch-\textsc{appl}-\textsc{fv} 17-up \textsc{comp} 2-\textsc{neg.subj}-drip-\textsc{appl}-\textsc{pass}-\textsc{ipfv}\\
\glt \lq They thought \lq\lq We should go pluck grass for thatching the roof with'', so that they would not get wet.' [Throw away the child]
\end{exe}
Like its affirmative counterpart, the negative subjunctive is also used together with modality and manipulation verbs, where it alternates with the infinitive. Verbs with an inherently negative meaning, such as \textit{kaaniysa} \lq forbid' and \textit{sigɪla} \lq  prevent', only take the negative subjunctive:

\begin{exe}
\ex
\begin{xlist}
\ex[]{\gll a-ba-ganga ba-n-kaniisye ʊkʊtɪ a-nga-kin-a ʊ-m-pɪla\\
\textsc{aug}-2-healer 2-1-forbid.\textsc{pfv} \textsc{comp} 1-\textsc{neg.subj}-play-\textsc{fv} \textsc{aug}-3-ball\\
\glt \lq The doctors have forbidden him/her to play football.' [ET]}
\ex[*]{\gll a-ba-ganga ba-n-kaniisye ʊkʊtɪ a-kin-e ʊ-m-pɪla\\
\textsc{aug}-2-healer 2-1-forbid.\textsc{pfv} \textsc{comp} 1-play-\textsc{subj} \textsc{aug}-3-ball\\}
\end{xlist}
\end{exe}

%\pagebreak necessary because of orphan
\pagebreak
As is the case in the affirmative subjunctive, the imperfective\is{aspect!imperfective} suffix \mbox{-\textit{aga}} has a general or habitual\is{aspect!habitual} reading; see (\ref{exNegativeHortative}, \ref{exNegativePurposeClause}) above. It also has a continuous reading (\ref{exNegSubjContinuous}), as well as a mitigating one with directives (\ref{exNegSubjMitigating}).\footnote{\citet[34]{SchumannK1899}, concerning the negative subjunctive, observes that -\textit{aga} ``is very common with this form'' (translated from the original German, BP). According to \citet{NurseD1979}, with verbs of movement in the imperative as well as with its negative counterpart -\textit{aga} is obligatory; however, this could not be confirmed. \citet{MwangokaNVoorhoeveJ1960b} states that -\textit{aga} is obligatory in the negative subjunctive, but this is contradicted by the data, see i.a (\ref{exNegativeJussive}) above.}

\begin{exe}
\ex \label{exNegSubjContinuous}\gll n-um̩-bʊʊl-ile ʊkʊtɪ a-nga-nw-aga fiijo\\
\textsc{1sg}-1-tell-\textsc{pfv} \textsc{comp} 1-\textsc{neg.subj}-drink-\textsc{ipfv} \textsc{intens}\\
\glt \lq I have told him that he should not be drinking too much.' [ET]
\ex \label{exNegSubjMitigating}\gll ʊ-nga-paasy-aga\\
\textsc{2sg}-\textsc{neg.subj}-worry-\textsc{ipfv}\\
\glt \lq Don't worry!' [overheard]
\end{exe}

Lastly, a prefix \textit{lɪ}-, homophonous or identical to the desiderative\is{mood!desiderative} (\sectref{Desiderative}) is sometimes found preceding negative \mbox{\textit{nga}-} (\ref{exNegSubjLi1}, \ref{exNegSubjLi2}). The fact that it is only attested in the written material and was not spontaneously offered suggests that the employment of this prefix constitutes a case of stylistic variation.

\begin{exe}
\ex \label{exNegSubjLi1} \gll lɪnga a-agiilwe fi-mo \textbf{a}-\textbf{lɪ}-\textbf{nga}-\textbf{asim}-\textbf{aga} bwila, iijʊʊl-ege ʊ-kʊ-fi-lond-a\\
if/when 1-lack\textsc{.pfv} 8-one 1-?-\textsc{neg.subj}-borrow-\textsc{ipfv} always 1.work\_hard-\textsc{ipfv.subj} \textsc{aug}-15-8-search-\textsc{fv}\\
\glt \lq If he lacks something, he should not always borrow, he should try hard to get them.' [Types of tools in the home]
\ex \label{exNegSubjLi2} \gll ijolo n̩-dw-iho lw-a ba-Nyakyʊsa ba-a-lɪ na=a-ka-jɪɪlo k-a n-kiikʊlʊ ʊ-kʊ-n-tiil-a ʊ-gwise gw-a n̩-dʊme, ʊ-jʊ tʊ-kʊ-tɪ n-kamwana.\\
old\_times 18-11-custom 11-\textsc{assoc} 2-Ny. 2-\textsc{pst}-\textsc{cop} \textsc{com}=\textsc{aug}-12-custom 12-\textsc{assoc} 1-woman \textsc{aug}-15-1-fear-\textsc{fv} \textsc{aug}-his\_father(1) 1-\textsc{assoc} 1-husband, \textsc{aug}-\textsc{prox.1} \textsc{1pl}-\textsc{prs}-say 1-in\_law\\
\glt `Long ago in the tradition of the Nyakyusa people they had a custom of the woman fearing the father of her husband, whom we call Nkamwana.'
\sn \gll mpaka pa-la lɪnga a-m̩-bonwile ʊkʊtɪ \textbf{a}-\textbf{lɪ}-\textbf{nga}-\textbf{n}-\textbf{tiil}-\textbf{aga}\\
until 16-\textsc{dist} if/when 1-1-pay\_off.\textsc{pfv} \textsc{comp} 1-?-\textsc{neg.subj}-1-fear-\textsc{ipfv}\\
\glt \lq Until the moment that he has paid her off so that she need not fear him any more.' [Should she save a life\ldots]  
\end{exe}
\is{negative|)}\is{mood!subjunctive|)}
\section{Desiderative}
\label{Desiderative}\is{mood!desiderative|(}
The desiderative construction consists of a prefix \textit{lɪ}- and the final vowel -\textit{a}.

\begin{exe}
\ex \label{exDesiderativeEinstieg}\textit{tʊlɪjoba} `we'd like to speak'
\end{exe}
The desiderative is hardly attested at all in the text corpus. Much of the following discussion is therefore based on elicitation. As the label \textit{desiderative} and example (\ref{exDesiderativeEinstieg}) above suggest, this construction expresses a desire or preference for a state-of-affairs. Discussions of \isi{modality} in language have come to include a concept of \textit{bouletic} (also \textit{boulomaic}) modality,\is{modality} which concerns ``what is possible or necessary, given a person's desires'' \citep[2]{vonFintelK2006}, or, as \citet[12]{NuytsJ2005a} puts it, ``indicates the degree of the speaker's (or someone else's) liking or disliking of the state of affairs''. This type of modality\is{modality} to all appearances lies at the semantic core of the desiderative construction.\footnote{Interestingly, for \ili{Nyika} M23, spoken northwest of Nyakyusa, and Nyikas's western neighbour \ili{Namwanga} M22 \citeauthor{BusseJ1940} (\citeyear[70]{BusseJ1940}; \citeyear[45]{BusseJ1960}) gives a future prefix \textit{li}-. \ili{Malila} M24, according to \citet[85]{KutschLojengaC2007}, has a future prefix \textit{lɪ}(\textit{ɪ})-. Given the well-known path of \isi{grammaticalization} from desire to future (\citealt{BybeePerkinsPaglucia1994}), these prefixes might have a common source.} In this the desiderative -- like the subjunctive,\is{mood!subjunctive} with which some overlaps on the paradigmatic level are found (see below) -- is confined to states-of-affairs that are not actualized.

In declaratives sentences with a first or second person subject, the desiderative expresses the speaker's desire or preference. Thus the first person singular desiderative in (\ref{exDesiderativeDeclarative1stPerson}) denotes the speaker's preference for a future act of his/her own, while in (\ref{exDesiderativeDeclarative2ndPerson}) the speaker desires that the act be performed by the hearer, the second person singular subject. For examples of the first and second person plural, respectively, see
(\ref{exDesiderativeBerger138example1}, \ref{exDesiderativeBerger138example2}) below.
\begin{exe}
\ex\label{exDesiderativeDeclarative1stPerson} \gll n-dɪ-syal-a pa-ka-aja\\
\textsc{1sg}-\textsc{desdtv}-remain-\textsc{fv} 16-12-homestead\\
\glt `I'd rather stay at home (e.g. than join you in your activity).' [ET]
\ex\label{exDesiderativeDeclarative2ndPerson} \gll ʊ-lɪ-tem-a=po ii-bɪfu ɪ-lɪ\\
\textsc{2sg}-\textsc{desdtv}-cut-\textsc{fv}=\textsc{part} 5-banana \textsc{aug}-\textsc{prox.5}\\
\glt \lq I would like you to [i.e. please] cut off this banana.' [ET]
\end{exe}

The preceding example (\ref{exDesiderativeDeclarative2ndPerson}) represents the most common use of the desiderative, namely in polite requests. In fact, it is only in this use that the desiderative was spontaneously offered during elicitation sessions. Polite requests are also the only use attested in older text collections, as well as in a recent draft of a Bible translation. The following two examples will illustrate this (orthography adapted):\footnote{\textit{mugonile} \lq lit. you (pl.) have rested', as in (\ref{exDesiderativeBerger138example2}), sg. \textit{ʊgonile}, is the most common greeting formula among the Nyakyusa people.}

\begin{exe}
\ex \label{exDesiderativeBerger138example1}
Context: Children see guineafowls scarifying each other.\\
\gll mwe ma-kanga ʊ-mwe, \textbf{mu}-\textbf{lɪ}-\textbf{tʊ}-\textbf{tem}-\textbf{a}=\textbf{po} nuuswe\\
\textsc{2pl} 6-guineafowl \textsc{aug}-\textsc{2pl} \textsc{2pl}-\textsc{desdtv}-\textsc{1pl}-cut-\textsc{fv}=\textsc{part} \textsc{com.1pl}\\
\glt \lq You guineafowls, please cut us a bit, too.' \citep[116]{BergerP1933}

\ex \label{exDesiderativeBerger138example2}
Context: A group of children are looking for a place to spend the night. They call at a stranger's house.\\
\gll ʊ-n-kangale a-a-t-ile: \textup{\lq\lq}eena, mu-gon-ile! mwe ba-ani?\textup{''} a-ba-anike ba-a-t-ile: \textup{\lq\lq}jo ʊ-swe, tʊ-sob-ile ɪ-n-jɪla j-ɪɪtʊ, \textbf{tʊ}-\textbf{lɪ}-\textbf{gon}-\textbf{a}=\textbf{mo}\textup{''}\\
\textsc{aug}-1-old 1-\textsc{pst}-say-\textsc{pfv} \phantom{\lq\lq}yes \textsc{2pl}-rest-\textsc{pfv} \textsc{2pl} 2-who \textsc{aug}-2-young\_person 2-\textsc{pst}-say-\textsc{pfv} \phantom{\lq\lq}\textsc{ref.1} \textsc{aug}-\textsc{1pl} \textsc{1pl}-be\_lost-\textsc{pfv} \textsc{aug}-9-path 9-\textsc{poss.1pl} \textsc{1pl}-\textsc{desdtv}-rest-\textsc{fv}=18\\
\glt \lq The old woman said: \lq\lq Hello! Who are you?'' The chidren said: \lq\lq It's us, we've lost our way and would like to sleep.''{}' \citep[137]{BergerP1933}
\end{exe}

During discussions of examples such as (\ref{exDesiderativeDeclarative2ndPerson}--\ref{exDesiderativeBerger138example2}), the language assistants remarked on various occasions that a request formulated in the desiderative specifically leaves the choice to the hearer, who may accept or decline. Similarly, in hortatives the desiderative  is considered more of a suggestion (\ref{exDesiderativeHortative}) than the subjunctive,\is{mood!subjunctive} which has a stronger character of a prompt or appeal (\ref{exDesiderativeSubjunctiveHortative}).
\begin{exe}
\ex\begin{xlist}
\ex\label{exDesiderativeHortative} \gll tʊ-lɪ-ly-a=mo\\
\textsc{1pl}-\textsc{desdtv}-eat-\textsc{fv}=some\\
\glt `I'd like us to eat something.' [ET]
\ex \label{exDesiderativeSubjunctiveHortative}\gll tʊ-ly-e=mo\\
\textsc{1pl}-eat-\textsc{subj}=some\\
\glt `Let's eat something!' [ET]
\end{xlist}
\end{exe}

With third person subjects, the interpretation depends on context and co-text. With a non-agentive subject\is{semantic roles} the issuer of the modality is the speaker (\ref{exDesiderativeDeclarative3rdPersonNonAgentive}). With an agentive subject both the speaker and the subject are available as the source of the desire or preference  (\ref{exDesiderativeDeclarative3rdPersonBoth}).

\begin{exe}
\ex\label{exDesiderativeDeclarative3rdPersonNonAgentive}\gll ʊ-mw-enda ʊ-gʊ gʊ-lɪ-j-a mw-elu\\
\textsc{aug}-3-cloth \textsc{aug}-\textsc{prox.3} 3-\textsc{desdtv}-be(come)-\textsc{fv} 3-white\\
\glt `I'd prefer it if this cloth were white.' [ET]
%\ex\gll ɪ-m-bwa ɪ-jɪ jo m-biibi fiijo, jɪ-lɪ-tʊ-lʊm-a\\
%\textsc{aug}-9-dog \textsc{aug}-\textsc{prox.9} \textsc{ref.9} 9-bad \textsc{intens} 9-\textsc{desdtv}-\textsc{1pl}-bite-\textsc{fv}\\
%\glt `That dog is very bad, he wants to bite us.' [ET]
\ex\label{exDesiderativeDeclarative3rdPersonBoth} \gll ɪɪ-ng'ombe si-lɪ-jong-a\\
\textsc{aug}-cow(10) 10-\textsc{desdtv}-run\_away-\textsc{fv}\\
\glt 1. `The cows would like to escape (uttered e.g. as a warning).'\\
2. `I wish that the cows would run away (e.g. malicious thinking).' [ET]
\end{exe}

In questions with a first or second person subject, the modal assessment shifts to the hearer (see \citealt{LehmannC2012} on the role of the modal assessor). With a first person subject, this is typically understood as a request for approval, as in (\ref{exDesiderativeQuestion1SG}). Likewise in (\ref{exDesiderativeHareHippo}) -- the sole token of the desiderative in the text corpus -- Hare asks Hippo whether the latter likes or dislikes the plan of visiting girls in town. Also note the paraphrasis in the last clause. With a second person as the subject, the desiderative in questions is often understood as a request (\ref{exDesiderativeQuestion2SG}). 
\begin{exe}
\ex \label{exDesiderativeQuestion1SG} Context: parent to child.\\
\gll ka-kam-e ɪɪ-ng'ombe! \textbf{n}-\textbf{dɪ}-\textbf{kin}-\textbf{a}=\textbf{po} taasi ʊ-m-pɪla?\\
\textsc{itv}-milk-\textsc{subj} \textsc{aug}-cow(10) \textsc{1sg}-\textsc{desdtv}-play-\textsc{fv}=\textsc{part} yet \textsc{aug}-3-ball\\
\glt `Go milk the cows!' -- \lq May I first play football for a bit?' [ET]
\ex \label{exDesiderativeHareHippo} \gll gw-ɪtʊ n-ka-aja ka-la ba-a-li=ko a-ba-lɪndwana a-ba-nunu fiijo. \textbf{aa}=\textbf{tʊ}-\textbf{lɪ}-\textbf{jaat}-\textbf{a}=\textbf{ko}?\\
1-\textsc{poss.1pl} 18-12-village 12-\textsc{dist} 2-\textsc{pst}-\textsc{cop}=17 \textsc{aug}-2-girl \textsc{aug}-2-good \textsc{intens} \textsc{fut}=\textsc{1pl}-\textsc{desdtv}-walk-\textsc{fv}=17\\ 
\glt `Friend, in that town there were very beautiful girls. Should we go and visit them?' 
\sn \gll  m-ba-bʊʊl-ile ʊkʊtɪ a=n-gw-is-a n=ʊ-m-manyaani gw-angʊ. bʊle gw-igan-ile ʊkutɪ tʊ-bʊʊk-e tw-esa?\\ %erstes wort zerlegen?
\textsc{1sg}-2-tell-\textsc{pfv} \textsc{comp} \textsc{fut}=\textsc{1sg}-\textsc{prs}-come-\textsc{fv} \textsc{com}=\textsc{aug}-1-friend 1-\textsc{poss.sg} \textsc{q} \textsc{2sg}-like-\textsc{pfv} \textsc{comp} \textsc{1pl}-go-\textsc{subj} \textsc{1pl}-all\\
\glt \lq I have told them that I will come with my friend. Hey, would you like it, if we both go?' [Hare and Hippo]
% split a) because it was to long b) to avoide weird pagebreaks

\ex \label{exDesiderativeQuestion2SG}
\gll \textbf{ʊ}-\textbf{lɪ}-\textbf{m}-\textbf{b}-\textbf{a}=\textbf{ko} ɪ-fi-ndʊ?\\
\textsc{2sg}-\textsc{desdtv}-\textsc{1sg}-give-\textsc{fv}=17 \textsc{aug}-8-food\\
\glt \lq Will you give us food?' [ET]
\end{exe}

In questions with a third person subject, as with declaratives, the source of the \isi{modality} may be either the hearer or the subject:

\begin{exe}
\ex
\gll a-ba-ndʊ ba-la ba-l-ingɪl-a n-nyumba?\\
\textsc{aug}-2-person 2-\textsc{dist} 2-\textsc{desdtv}-enter-\textsc{fv} 18-house\\
\glt 1. \lq Do you wish that those people go inside?'\\
2. \lq Will those people go inside?'[ET]
\end{exe}
 
The desiderative can take the imperfective\is{aspect!imperfective} suffix -\textit{aga}. This can be used to add a continuous reading, which can shade into an emphatic one (\ref{exDesiderativeIPFV1}). It is also used to express a desire or preference for a regular/habitual\is{aspect!habitual} occurrence. (\ref{exDesiderativeIPFV2}).

\begin{exe}
\ex \label{exDesiderativeIPFV1}\gll tʊ-lɪ-bʊʊk-aga\\
\textsc{1pl}-\textsc{desdtv}-go-\textsc{ipfv}\\
\glt \lq I'd like us to get going.' [ET] %continuous, shading into emphatic

\ex \label{exDesiderativeIPFV2} \gll ʊ-lɪ-m-b-aga ɪ-fi-ndʊ kʊkʊtɪ ii-sikʊ\\
\textsc{2sg}-\textsc{desdtv}-\textsc{1sg}-give-\textsc{ipfv} \textsc{aug}-8-food every 5-day\\
\glt \lq I'd like you to give me food every day.' [ET] %hab
\end{exe}

The desiderative can also be augmented by the distal/itive prefix\is{itive} \textit{ka}- (\sectref{DistalKa}), in which case the quality of the final vowel changes to -\textit{e}, as is the case in the subjunctive.\is{mood!subjunctive}\footnote{One might take this as evidence for a circumfix \textit{ka}-\ldots-\textit{e}, as \citet{NicolleS2002} assumes for \ili{Digo} E72.}

\begin{exe}
\ex \gll ʊ-lɪ-k-iigʊl-e\\
\textsc{2sg}-\textsc{desdtv}-\textsc{itv}-open-\textsc{fv}\\
\glt `Please go open the door.' [ET]
\end{exe}

Unlike the subjunctive,\is{mood!subjunctive} the affirmative desiderative is excluded from subordinate clauses:
\begin{exe}
\ex[*]{\gll n-aalɪ-n̩-dagiile ʊkʊtɪ a-lɪ-jeng-a ɪɪ-nyumba\\
\textsc{1sg}-\textsc{pst}-1-order.\textsc{pfv} \textsc{comp} 1-\textsc{dsdtv}-build-\textsc{fv} \textsc{aug}-house(9)\\
\glt (intended: `I told him/her to please build a house.')}
\ex[*]{\gll n-gʊ-lond-a ʊkʊtɪ ʊ-l-iigʊl-a ʊ-lw-igi\\
\textsc{1sg}-\textsc{prs}-want-\textsc{fv} \textsc{comp} \textsc{2sg}-\textsc{dsdtv}-open-\textsc{fv} \textsc{aug}-11-door\\
\glt (intended: `I wish you to please open the door.')}\is{subordinate clauses!complement clause}
\ex[*]{\gll n-aalɪ-m-peele ɪ-n-dalama ʊkʊtɪ a-ly-ʊl-a ɪ-fi-ndʊ\\
\textsc{1sg}-\textsc{pst}-1-give.\textsc{pfv} \textsc{aug}-10-money \textsc{comp} 1-\textsc{dsdtv}-buy-\textsc{fv} \textsc{aug}-8-food\\
\glt (intended: `I gave him money so that he could buy food.')}\is{subordinate clauses!purpose clause}
\ex[*]{\gll kyajɪpo \textup{/} paakipo \textup{/} kanunu tʊ-lɪ-bʊʊk-a\\
better {} preferable {} well \textsc{1pl}-\textsc{desdv}-go-\textsc{fv}\\ 
\glt (intended: `It is better/preferable/good that we go.')}
\end{exe}

Lastly, the desiderative is negated\is{negative} with the negative counterpart to the subjunctive, which is discussed in (\sectref{NegativeSubjunctive}).\is{mood!desiderative|)}
\section{Modal future}\label{Commissive}\label{ModalFuture}\is{future!modal future|(}
The last modal paradigm to be discussed constitutes an interesting case of constructionalization. As the \isi{simple present} (\sectref{Present}), in the affirmative it is formed with a subject prefix\is{subject marker} from the second series (\sectref{SubjectConcords}) and a prefix \textit{kʊ}-, while the \isi{negative} consists of a subject prefix from the first series and the negative prefix \textit{ti}- preceding \textit{kʊ}-. Unlike the simple present, however, the final slot is filled by the imperfective suffix\is{aspect!imperfective} -\textit{aga} and its allomorphs; see \sectref{AlternationsIPFVaga}.

In contrast with what would be expected from the composition of this construction, it cannot have a present continuous or habitual/generic reading.\is{aspect!progressive}\is{aspect!habitual}\is{aspect!generic} Instead it expresses a future-oriented type of modality.\is{modality} The same situation is found in neighbouring \ili{Kinga} G65 and \ili{Vwanji} G66 \citep{EatonHToAppear}. The following example illustrates this:

\begin{exe}
\ex \label{exCOMMEinstiegsbeispiel}
\gll tʊ-kʊ-ly-aga ʊ-m-pʊnga\\
\textsc{1pl}-\textsc{prs}-eat-\textsc{ipfv} \textsc{aug}-1-rice\\
\glt `We shall eat rice. (e.g. announcing the next meal or a change in diet)'\\
not: `We are eating rice.'\\
not: `We eat rice.' [ET]
\end{exe}

This construction, which will be labelled \textsc{modal future} throughout this study, depicts a future state-of-affairs as a settled fact; that is, it expresses various kinds of modal necessity.

While this semantics may at first seem odd given the composition of the modal future, the apparent mismatch of form and function may be explained by taking a comparative and diachronic perspective. In various languages of the wider area, e.g. the Tanzanian variety of neighbouring \ili{Ndali} \citep{SwillaI1998}, \ili{Kisi} G67 \citep{GrayMS} and \ili{Malila} M24 (Helen Eaton,\ia{Eaton, Helen} p.c.) the imperfective suffix \mbox{-\textit{aga}} narrows down the possible readings of the simple present to an explicitly habitual\is{aspect!habitual} one. As \citet[21]{ZiegelerD2006} states, habitual\is{aspect!habitual} or generic\is{aspect!generic} aspect -- note that not all authors distinguish between these two, and while some consider habituality\is{aspect!habitual} a special case of genericity,\is{aspect!generic} others use the terms interchangeably -- is a \lq\lq prime candidate for [\ldots] categories residing on the aspect-modality interface''; see also \citet{GivonT1994}. In formal semantics, generic\is{aspect!generic} sentences are commonly understood as law-like generalizations about the \textit{most normal} cases \citep{KrifkaMetal1995}, a qualification necessary to account for the possibility of exceptions. Stating such a regularity implies a prediction that, all things being normal, the eventuality in question will continue to occur in the future. A similar observation has been made by \citet[140f]{BrintonL1988}. Assuming that at an earlier stage the situation in Nyakyusa paralleled the one found in Tanzanian Ndali,\il{Ndali} \ili{Kisi} and Malila,\il{Malila} the present-day semantics of the modal future can be understood as the semanticization of this future-oriented implicature. For a more detailed elaboration of this reconstruction see \citet{PersohnB2016}.

A verbal construction with a number of striking functional similarities is found in \ili{Yucatec Maya} \textit{he}-…-\textit{e'}. \citet{LehmannC2012} calls this a \lq\lq commissive modality''\is{modality} construction, while \citet{BohnemeyerJ2002} speaks of ``assurative''. As both these labels feature notions which rather belong to the realm of pragmatics, the more neutral, albeit vague label \textit{modal future} is preferred.

The following exposition of its uses will illustrate the meaning of the modal future. To begin with, the modal future is employed in generic contexts to depict consequences and sequences of eventualities. In (\ref{exCommissiveSequenceConsequence}), an excerpt from an expository text is given. The modal future construction is found to express determined consequences of specific behaviour in (\ref{exCommissiveSequenceConsequenceSentence1}, \ref{exCommissiveSequenceConsequenceSentence6}, \ref{exCommissiveSequenceConsequenceSentence7}). In (\ref{exCommissiveSequenceConsequenceSentence4}) it is used to depict the next step in a series of acts.

\begin{exe}
\ex Context: Discussing men who do not own tools.
\label{exCommissiveSequenceConsequence}
\begin{xlist}
\ex\label{exCommissiveSequenceConsequenceSentence1}
\gll kʊʊ-nongwa ɪ-jo lɪnga ʊ-n-nyambala a-bagiile ʊ-kʊ-tol-igw-a ʊ-kʊ-mmw-ag-a ʊ-n-kiikʊlʊ ʊ-gw-a kʊ-mmw-eg-a, a-ba-ndʊ \textbf{bi}-\textbf{kʊ}-\textbf{mmw}-\textbf{inogon}-\textbf{aga} ʊ-mu-ndʊ ʊ-jo ʊkʊtɪ m-oolo pa-kʊ-bomb-a ɪ-m-bombo\\
17-issue \textsc{aug}-\textsc{ref.9} if/when \textsc{aug}-1-man 1-be\_able.\textsc{pfv} \textsc{aug}-15-defeat-\textsc{pass}-\textsc{fv} \textsc{aug}-15-1-find-\textsc{fv} \textsc{aug}-1-woman \textsc{aug}-1-\textsc{assoc} 15-1-marry-\textsc{fv} \textsc{aug}-2-person 2-\textsc{mod.fut}-1-think-\textsc{mod.fut} \textsc{aug}-1-person \textsc{aug}-\textsc{ref.1} \textsc{comp} 1-lazy 16-15-work-\textsc{fv} \textsc{aug}-9-work\\
\glt \lq Because of this, if a man is unable to get a woman to marry, people (will) think that this person is lazy in doing work.' %consequence in generics

\ex \label{exCommissiveSequenceConsequenceSentence2}
\gll a-ba-ndʊ bo a-bo bi-kʊ-bʊʊk-a kʊ-kw-asim-a ɪ-fi-bombelo ɪ-fy-a kʊ-bomb-el-a ɪ-m-bombo bo a-b-iinaabo ba-lɪ pa-kʊ-tʊʊsy-a\\
\textsc{aug}-2-person as \textsc{aug}-\textsc{ref.2} 2-\textsc{prs}-go-\textsc{fv} 17-15-borrow-\textsc{fv} \textsc{aug}-8-tool \textsc{aug}-8-\textsc{assoc} 15-work-\textsc{appl}-\textsc{fv} \textsc{aug}-9-work as \textsc{aug}-2-their\_companion 2-\textsc{cop} 16-15-rest-\textsc{fv}\\
\glt \lq People like those go to borrow tools to do work with, when their fellows are resting.'

\ex \label{exCommissiveSequenceConsequenceSentence3}
 \gll  bo ba-m-peele ɪ-fi-bombelo, a-ka-bagɪl-a ʊ-kʊ-bomb-el-a a-ka-balɪlo a-ka-tali\\
as 2-1-give.\textsc{pfv} \textsc{aug}-8-tool 1-\textsc{neg}-be\_able-\textsc{fv} \textsc{aug}-15-work-\textsc{appl}-\textsc{fv} \textsc{aug}-12-time \textsc{aug}-12-long\\
\glt \lq When they have given him tools, he cannot work with them for a long time.'

\ex \label{exCommissiveSequenceConsequenceSentence4}
 \gll lʊmo bo a-bomb-ile=po panandɪ \textbf{kw}-\textbf{ag}-\textbf{aga} a-b-eene na=fyo b-iis-ile kʊ-kw-eg-a\\
maybe as 1-work-\textsc{pfv}=\textsc{part} a\_little \textsc{2sg.mod.fut}-find-\textsc{mod.fut} \textsc{aug}-2-owner \textsc{com}=\textsc{ref.8} 2-come-\textsc{pfv} 17-15-take-\textsc{fv}\\
\glt \lq Or when he has worked for a little while, you will find they have come to take them back.'
%sequence in generics

\ex \label{exCommissiveSequenceConsequenceSentence5}
\gll ʊ-ka-bagɪl-a ʊ-kʊ-kaan-il-a paapo fi-ka-j-a fy-ako, kʊ-gomosy-a\\
\textsc{aug}-12-be\_able-\textsc{fv} \textsc{aug}-15-refuse-\textsc{appl}-\textsc{fv} because 8-\textsc{neg}-be(come)-\textsc{fv} 8-\textsc{poss.2sg} \textsc{2sg.prs}-return.\textsc{caus}-\textsc{fv}\\
\glt \lq You cannot refuse, because they are not yours, you return them.'

\ex \label{exCommissiveSequenceConsequenceSentence6}
\gll lɪnga kʊ-kaabɪl-a ʊ-kʊ-gomosy-a \textbf{bi}-\textbf{kʊ}-\textbf{kw}-\textbf{im}-\textbf{aga} bwila\\
if/when \textsc{2sg.prs}-be\_late-\textsc{fv} \textsc{aug}-15-return.\textsc{caus}-\textsc{fv} 2-\textsc{mod.fut}-\textsc{2sg}-deprive-\textsc{mod.fut} always\\
\glt \lq If you delay in returning, they will withhold them always.'
%consequence in generics

\ex \label{exCommissiveSequenceConsequenceSentence7} \gll po \textbf{kʊ}-\textbf{kʊbɪlw}-\textbf{aga} n=ɪ-n-jala n=ʊ-kʊ-j-a n-kunwe bwila\\
then \textsc{2sg.mod.fut}-suffer-\textsc{mod.fut} \textsc{com}=\textsc{aug}-9-hunger \textsc{com}=\textsc{aug}-15-be(come)-\textsc{fv} 1-poor always\\
\glt \lq And so you will be troubled by hunger and always be poor.'
%consequence in generics
[Types of tools in the home]
\end{xlist}
\end{exe}

As was observed earlier, the modal future does not allow for a timeless reading. Likewise, it is infeliticious in contexts expressing scheduled eventualities (\ref{exCommissiveNotScheduledAffirmative}, \ref{exCommissiveNotScheduledNegative}). (\ref{exCommissiveNotScheduledAffirmative}) could, however, be said of a new train that does not yet run but is announced to leave in the afternoon. Likewise, (\ref{exCommissiveNotScheduledNegative}) is felicitous as a resolution about a new work schedule.

\begin{exe}
\ex\label{exCommissiveNotScheduledAffirmative}Context: According to schedule.\\
\gll \#ii-treni li-kʊ-sook-anga=po pa-muu-si\\
\phantom{\#}5-train(<SWA) 5-\textsc{mod.fut}-leave-\textsc{mod.fut}=16 16-3-daytime\\
\glt \makebox[\myl][l]{}(intended: \lq The train leaves in the afternoon.')
\ex\label{exCommissiveNotScheduledNegative}Context: According to contract.\\
\gll \#pa-kɪ-tatʊ tʊ-ti-kʊ-bomb-aga ɪ-m-bombo\\
\phantom{\#}16-7-three \textsc{1pl}-\textsc{neg}-\textsc{mod.fut}-work-\textsc{mod.fut} \textsc{aug}-9-work\\
\glt \makebox[\myl][l]{}(intended: \lq We do not work this Wednesday/on Wednesdays.')
\end{exe}

Closely related to the preceding examples, the modal future is very common in commissive speech acts. These are utterances which \lq\lq commit the speaker to a certain cause of events'' \citep[156]{AustinJL1962}. The following examples illustrate prototypical cases: (\ref{exCommissiveFelbergstory}) features a promise, (\ref{exCommissiveHareSpider}) an assurance and (\ref{exCommissiveHareChameleon}) an announcement. In all of these the speaker vouches that the future state-of-affairs will occur.

\begin{exe}
\ex \label{exCommissiveFelbergstory}
Context: A girl has eloped with a man. Her father has tracked them down.\\
\gll taata ʊ-ne nalooli ɪ-fy-ʊma n-gaalɪ n-ga-kab-a ɪɪ-sala ɪ-jɪ. looli \textbf{n}-\textbf{gw}-\textbf{i}-\textbf{pʊʊl}-\textbf{aga}. n-gʊ-homb-a ɪ-fy-ʊma fi-la bo ʊlʊ n-iitiike m̩-ba-ndʊ\\
father \textsc{aug}-\textsc{1sg} really \textsc{aug}-8-bride\_price \textsc{1sg}-\textsc{pers} \textsc{1sg}-\textsc{neg}-get-\textsc{fv} \textsc{aug}-hour(9) \textsc{aug}-\textsc{prox.9} but \textsc{1sg}-\textsc{mod.fut}-\textsc{refl}-thresh-\textsc{mod.fut} \textsc{1sg}-\textsc{prs}-pay-\textsc{fv} \textsc{aug}-8-bride\_price 8-\textsc{dist} as now \textsc{1sg}-agree.\textsc{pfv} 18-2-person\\
\glt `Father [honorific], I still haven't obtained the brideprice. But I'll go after it. I'm paying that brideprice, just as I've now agreed to in front of people.' [Man and his in-law]

\ex \label{exCommissiveHareSpider}
Context: Hare and Spider want to climb up a tree.\\
\gll looli kalʊlʊ a-ka-a-meenye ʊ-kʊ-kwel-a m-mwanya, a-lɪnkʊ-lʊ-bʊʊl-a ʊ-lʊ-bʊbi a-lɪnkʊ-tɪ \textup{\lq\lq}ʊ-ne n-ga-many-a ʊ-kʊ-kwel-a m-mwanya\textup{''}\\
but hare(1) 1-\textsc{neg}-\textsc{pst}-know.\textsc{pfv} \textsc{aug}-15-climb-\textsc{fv} 18-up 1-\textsc{narr}-11-tell-\textsc{fv} \textsc{aug}-11-spider 1-\textsc{narr}-say \phantom{\lq\lq}\textsc{aug}-\textsc{1sg} \textsc{1sg}-\textsc{neg}-know-\textsc{fv} \textsc{aug}-15-climb-\textsc{fv} 18-up\\
\glt `But Hare could not climb up there, he told Spider ``I can't climb up there.''{}'
\sn \gll ʊ-lʊ-bʊbi lʊ-lɪnkʊ-job-a ʊkʊtɪ \textup{\lq\lq}ʊ-nga-paasy-aga. ʊ-ne n-dɪ na=bo ʊ-bʊ-ʊsi ʊ-bʊ bʊ-kʊ-n-dwal-a ʊ-ne, mo \textbf{kw}-\textbf{end}-\textbf{anga}=\textbf{mo} nungwe\textup{''}\\
\textsc{aug}-11-spider 11-\textsc{narr}-speak-\textsc{fv} \textsc{comp} \phantom{\lq\lq}\textsc{2sg}-\textsc{neg.subj}-worry-\textsc{ipfv} \textsc{aug}-\textsc{1sg} \textsc{1sg}-\textsc{cop} \textsc{com}=\textsc{ref.14} \textsc{aug}-14-thread \textsc{aug}-\textsc{prox.14} 14-\textsc{prs}-\textsc{1sg}-carry-\textsc{fv} \textsc{aug}-\textsc{1sg} \textsc{ref.18} \textsc{2sg.mod.fut}-walk/travel-\textsc{mod.fut}=18 \textsc{com.2sg}\\
\glt Spider said ``Don't worry. I have a thread that carries me, you too will go on it.''{}' [Hare and Spider]

\ex \label{exCommissiveHareChameleon}
Context: Elephant, in his function as the eldest of animals, has called a meeting.\\
\gll ɪɪ-sofu jɪ-lɪnkʊ-tɪ \lq\lq lɪlɪno \textbf{tʊ}-\textbf{kʊ}-\textbf{ba}-\textbf{keet}-\textbf{aga} kalʊlʊ n=ʊ-lw-ifi bi-kʊ-j-a pa-kʊ-tol-an-a ʊ-lʊ-bɪlo''\\
\textsc{aug}-elephant(9) 9-\textsc{narr}-say \phantom{\lq\lq}now/today \textsc{1pl}-\textsc{mod.fut}-2-watch-\textsc{mod.fut} hare(1) \textsc{com}=\textsc{aug}-11-chameleon 2-\textsc{prs}-be(come)-\textsc{fv} 16-15-win-\textsc{recp}-\textsc{fv} \textsc{aug}-11-race\\
\glt \lq Elephant said \textup{\lq\lq}Today we shall see how Hare and Chameleon are going to compete in a race.\textup{''}' [Hare and Chameleon]
\end{exe}

The modal future is also found with a certain directive force. This use is especially common when describing the target procedure of a plan involving the hearer, as in the following example:

\begin{exe}
%target procedure
\ex
Context: Tortoise explains to his child how to make Monkey believe he is absent.\\
\gll po a-pa n-dʊʊgeele. po a-n-gw-i-sanusy-a. a-ma-lʊndɪ gi-kʊ-keet-a kʊ-mwanya. po ʊ-gwe kʊ-lond-a ii-bwe, kʊ-kol-a. kʊ-bɪɪk-a ɪ-fi-lombe pa-mwanya pa-my-angʊ.\\
then \textsc{aug}-\textsc{prox.16} \textsc{1sg}-stay.\textsc{pfv} then \textsc{fut}=\textsc{1sg}-\textsc{prs}-\textsc{refl}-alter.\textsc{caus}-\textsc{fv} \textsc{aug}-6-leg 6-\textsc{prs}-watch-\textsc{fv} 17-high then \textsc{aug}-\textsc{2sg} \textsc{2sg.prs}-search-\textsc{fv} 5-stone \textsc{2sg.prs}-grasp/hold-\textsc{fv} \textsc{2sg.prs}-put-\textsc{fv} \textsc{aug}-8-maize 16-high 16-4-\textsc{poss.1sg}\\
\glt \lq Here I stay, I'll turn myself over. The legs will look up. ‎‎You'll search for a stone and grasp it. You'll put maize on top of me.'
\sn \gll po \textbf{lʊ}-\textbf{kʊ}-\textbf{fwan}-\textbf{aga} lw-ala. po \textbf{kʊ}-\textbf{sy}-\textbf{aga}\\
 then 11-\textsc{mod.fut}-resemble-\textsc{mod.fut} 11-grindstone then \textsc{2sg.mod.fut}-grind-\textsc{mod.fut}\\
\glt \lq It'll resemble a grindstone. Then you shall grind.' [Monkey and Tortoise]
\end{exe}

Interestingly, all tokens of the modal future within interrogatives in the text corpus constitute rhetorical questions. Thus, in (\ref{exModalFutureRhetoricalQuestion1}), the question is raised as to what it is that a woman will necessarily do when she sees her father-in-law in danger of dying and whether this will involve letting him drown. The answers -- to help and not let him drown -- are implied in the co-text of this behavioural text, which criticizes the tradition of in-law avoidance. In (\ref{exModalFutureRhetoricalQuestion2}) the narrator employs a dramatic ruse by letting the trapped protagonist ask himself if his death in a pit constitutes his inevitable fate, only to let him answer to the contrary with the actions that follow.

\begin{exe}
\ex\label{exModalFutureRhetoricalQuestion1}
Context: Discussing the tradition of in-law avoidance.\\
\gll leelo lɪnga ʊ-gwise gw-a n-nyambala n̩-gaala bw-alwa kʊ-lʊ-sako ʊ-lʊ-nunu lɪnga ʊ-n-kasi gw-a mw-anaake i-kʊ-kɪnd-a pa-la ʊ-gwise gw-a n̩-dʊme i-kʊ-milw-a m-m-ɪɪsi jɪ-kʊ-j-a n-gafu kʊʊ-nongwa j-aa bʊ-gaala bw-alwa bw-ake.\\
now/but if/when \textsc{aug}-his\_father(1) 1-\textsc{assoc} 1-man 1-drink 14-alcohol 17-11-luck \textsc{aug}-11-good if/when \textsc{aug}-1-wife 1-\textsc{assoc} 1-his\_child 1-\textsc{prs}-pass-\textsc{fv} 16-\textsc{dist} \textsc{aug}-his\_father(1) 1-\textsc{assoc} 1-husband 1-\textsc{prs}-drown-\textsc{fv} 18-6-water 9-\textsc{prs}-be(come)-\textsc{fv} 9-difficult 17-issue(9) 9-\textsc{assoc} 14-drink 14-alcohol 14-\textsc{poss.sg}\\
\glt \lq But if the father of the man is a drunkard and if by chance the wife of his child passes by while the father of her husband is drowning in the water, it is difficult because of his drinking.'
\sn \gll bʊle ʊ-n-kiikʊlʊ ʊ-jo \textbf{i}-\textbf{kʊ}-\textbf{bomb}-\textbf{aga} fi-ki? [\dots] kalɪ \textbf{i}-\textbf{kʊ}-\textbf{n̩}-\textbf{dek}-\textbf{aga} a-fw-ege m-m-ɪɪsi kʊʊ-nongwa j-aa kʊ-tiil-a ʊ-kʊ-kilani-a ɪ-m-baatɪko paapo a-ka-m̩-bonol-a?\\
\textsc{q} \textsc{aug}-1-woman \textsc{aug}-\textsc{ref.1} 1-\textsc{mod.fut}-do-\textsc{mod.fut} 8-what? {} Q 1-\textsc{mod.fut}-1-let-\textsc{mod.fut} 1-die-\textsc{subj.ipfv} 18-6-water 17-issue(9) 9-\textsc{assoc} 15-fear-\textsc{fv} \textsc{aug}-15-break\_custom-\textsc{fv} \textsc{aug}-10-procedure because 1-\textsc{neg}-1-pay\_off-\textsc{fv}\\
\glt \lq What will the woman do? [\ldots] Will she really let him die in the water because of fearing to break the customs because he has not paid her off?' [Should she save a life \ldots]

\ex\label{exModalFutureRhetoricalQuestion2}
Context: Hare has fallen into a pit. He is afraid a man is waiting to kill him.\\
\gll kalʊlʊ a-aly-and-ile ʊ-kw-i-laalʊʊsy-a ʊkʊtɪ, \textup{\lq\lq}lɪlɪno ʊ-ne \textbf{n}-\textbf{gʊ}-\textbf{fw}-\textbf{aga} mu-n-k-iina mu-no? po n-ga-bagɪl-a. lɪnga jo mu-ndʊ ʊ-jʊ a-li=po pa-mwanya n-dek-e a-n-gog-ege.\textup{''}\\
hare(1) 1-\textsc{pst}-begin-\textsc{pfv} \textsc{aug}-15-\textsc{refl}-ask-\textsc{fv} \textsc{comp} \phantom{\lq\lq}now/today \textsc{aug}-\textsc{1sg} \textsc{1sg}-\textsc{mod.fut}-die-\textsc{mod.fut} 18-18-7-cave 18-\textsc{prox} then \textsc{1sg}-\textsc{neg}-be\_able-\textsc{fv} if/when \textsc{ref.1} 1-person \textsc{aug}-\textsc{prox.1} 1-\textsc{cop}=16 16-high \textsc{1sg}-let-\textsc{subj} 1-\textsc{1sg}-kill-\textsc{ipfv.subj}\\
\glt \lq Hare started to ask himself \lq\lq Am I to die now in this pit? I can't. If that's a person up there I'll let him kill me.''{}' [he goes on to jump out of the pit] [Saliki and Hare]
\end{exe}

In elicitation, the modal future was also accepted in interrogatives when asking the hearer to make a promise, as in (\ref{exCommissiveQuestionElicitation}), the interrogative counterpart to (\ref{exCommissiveFelbergstory}) above. In compliance with its semantics of a settled future, it was considered infelicitous when asking for a prediction (\ref{exCommissiveQuestionElicitationPrediction}).

\pagebreak %necessary because otherwise the context of the following example would stand alone

\begin{exe}
\ex \label{exCommissiveQuestionElicitation}Context: The hearer owes you money.\\
\gll kw-i-pʊʊl-aga?\\
\textsc{mod.fut}-\textsc{refl}-thresh-\textsc{mod.fut}\\
\glt \lq Will you [promise me to] go after it?' [ET]
\ex \label{exCommissiveQuestionElicitationPrediction} Context: A field has been devastated by monkeys. The owner has just arrived and is shocked by the sight.\\
\gll \#bʊle, i-kʊ-bomb-aga sy-a fi-ki lɪno?\\
\phantom{\#}Q 1-\textsc{mod.fut}-do-\textsc{mod.fut} 10-\textsc{assoc} 8-what now\\
\glt \makebox[\myl][l]{}(intended: \lq What will s/he do now?')
\end{exe} 
\is{future!modal future|)}
\section{Conditional \textit{ngali}}
\label{ngali}\is{conditional|(}
A conditional particle \textit{ngalɪ} serves to introduce the apodosis (consequent clause) of counterfactual conditionals.\footnote{Some information on conditional clauses is also given in \citet{LusekeloA2016}.} The protasis (antecedent) is normally introduced by \textit{lɪnga} \lq if, when' and features a past tense verb (\ref{exNgaliCounterfactual1}, \ref{exNgaliCounterfactual2}).\is{subordinate clauses!conditional protasis}\footnote{See \sectref{PastImperfectiveModal} for another means of introducing counterfactional apodoses.}

\begin{exe}
\ex\label{exNgaliCounterfactual1} \gll lɪnga n-aa-meenye \textbf{ngalɪ} n-ga-lɪm-a ɪ-ky-ɪnja ɪ-kɪ, paapo si-n-gʊfiifye fiijo ɪ-n-gambɪlɪ ɪ-si\\
if/when \textsc{1sg}-\textsc{pst}-know.\textsc{pfv} \textsc{cond} \textsc{1sg}-\textsc{neg}-farm-\textsc{fv} \textsc{aug}-7-year \textsc{aug}-\textsc{prox.7} because 10-\textsc{1sg}-cause\_trouble.\textsc{pfv} \textsc{intens} \textsc{aug}-10-monkey \textsc{aug}-\textsc{prox.10}\\
\glt `If I had known, I would not have farmed this year, because these monkeys have very much hurt me.' [Thieving monkeys]
\ex \label{exNgaliCounterfactual2}
\gll lɪnga fy-a-li=po ɪ-fi-ndʊ paa-meesa \textbf{ngalɪ} tʊ-l-iile\\
if/when 8-\textsc{pst}-\textsc{cop}=16 \textsc{aug}-8-food 16-table(<SWA) \textsc{cond} \textsc{1pl}-eat-\textsc{pfv}\\
\glt \lq If there had been food on the table, we would have eaten it.' [ET]
\end{exe}

In this use, conditional \textit{ngalɪ} can be combined with the future\is{tense!future}\is{future!future aa=} \isi{proclitic} \textit{aa}= (\sectref{ProcliticAa}). It is as yet unclear how far this changes the meaning of the clause. The following two textual examples, taken from a draft of a Bible translation and HIV prevention materials by SIL International, suggest that the addition of \mbox{\textit{aa}=} emphasizes the dissociation between the unfulfilled condition on the one hand and the divergent reality on the other.
\begin{exe}
\ex \gll
a-ba-ndʊ a-bo ba-a-fum-ile kʊ-my-ɪtʊ, looli ba-ka-a-lɪ b-iinɪɪtʊ, paapo \textbf{lɪnga} ba-a-lɪ b-iitɪki b-iinɪɪtʊ \textbf{a}=\textbf{ngalɪ} tʊ-lɪ na=bo,\\
\textsc{aug}-2-person \textsc{aug}-\textsc{ref.2} 2-\textsc{pst}-come\_from-\textsc{pfv} 17-4-\textsc{poss.1pl} but 2-\textsc{neg}-\textsc{pst}-\textsc{cop} 2-our\_companion because if 2-\textsc{pst}-\textsc{cop} 2-believer 2-our\_companion \textsc{fut}=\textsc{cond} \textsc{1pl}-\textsc{cop} \textsc{com}=\textsc{ref.2}\\
\glt \lq They went out from us, but they were not of us [lit: \ldots because if they had been of us they would be with us],'
\sn \gll fyobeene ba-a-tʊ-lek-ile, lɪnga ba-a-lɪ b-iitɪki b-iinɪɪtʊ ngalɪ ba-a-syele na=nʊʊswe. looli bo ba-a-sook-ile=po ba-a-nangiisye ʊkʊtɪ b-oosa ba-ka-a-lɪ b-iinɪɪtʊʊ\\
therefore 2-\textsc{pst}-\textsc{1pl}-let-\textsc{pfv} if 2-\textsc{pst}-\textsc{cop} 2-believer 2-our\_companion \textsc{cond} 2-\textsc{pst}-remain.\textsc{pfv} \textsc{com}=\textsc{com.1pl} but as 2-\textsc{pst}-leave-\textsc{pfv}=16 2-\textsc{pst}-show.\textsc{pfv} \textsc{comp} 2-all 2-\textsc{neg}-\textsc{pst}-\textsc{cop} 2-our\_companion\\
\glt \lq  but they went out, that they might be made manifest that they were not all of us'  [lit: \lq therefore they left us, if they had been believers like us, they would have remained with us \ldots'] (1 John 2: 19)
\ex Context: An orphan is speaking.\\
\gll n-gʊ-ba-syʊkw-a jʊʊba na taata, looli n-gʊ-sʊʊbɪl-a ʊkʊtɪ \textbf{lɪnga} ba-a-j-anga=po ʊlʊ, \textbf{a}=\textbf{ngalɪ} bi-kw-i-tuukɪfy-a ʊ-swe\\
\textsc{1sg}-\textsc{prs}-2-miss\_sadly-\textsc{fv} my\_mother(1) \textsc{com} my\_father(1) but \textsc{1sg}-\textsc{prs}-expect-\textsc{fv} \textsc{comp} if 2-\textsc{pst}-be(come)-\textsc{ipfv}=16 now \textsc{fut}=\textsc{cond} 2-\textsc{prs}-\textsc{refl}-praise.\textsc{appl}-\textsc{fv} \textsc{aug}-\textsc{1pl}\\
\glt \lq I miss Mama and Papa, but I think if they were here they would be proud of us now.' [Kande's story]\footnotemark
\protect\footnotetext{\url{https://www.nyakyusalanguage.com/sites/all/libraries/pdf.js/web/viewer.html?file=/en/file/30/force_download} (10 November, 2020).}
\end{exe}

Lastly, conditional \textit{ngalɪ} is also found outside of conditional clauses, again giving a hypothetical reading:
\begin{exe}
\ex
\gll \textbf{ngalɪ} tʊ-kʊ-ly-a (looli tʊ-kaalɪ tʊ-kʊ-ba-guul-ɪl-a a-ba-heesya)\\
\textsc{cond} 1\textsc{pl}-\textsc{prs}-eat-\textsc{fv} \phantom{(}but 1\textsc{pl}-\textsc{pers} \textsc{1pl}-\textsc{prs}-2-wait-\textsc{appl}-\textsc{fv} \textsc{aug}-2-foreigner\\
\glt \lq We would be eating (but we are still waiting for the guests).' [ET]
\ex
\gll \textbf{ngalɪ} tʊ-l-iile (looli tʊ-kaalɪ tʊ-kʊ-ba-guul-ɪl-a a-ba-heesya)\\
\textsc{cond} \textsc{1pl}-eat-\textsc{pfv}  \phantom{(}but \textsc{1pl}-\textsc{pers} \textsc{1pl}-\textsc{prs}-2-wait-\textsc{appl}-\textsc{fv} \textsc{aug}-2-foreigner\\
\glt \lq We would have eaten (but we are still waiting for the guests).' [ET]
\end{exe}
\is{conditional|)}