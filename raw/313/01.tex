\documentclass[output=paper]{langscibook}
\ChapterDOI{10.5281/zenodo.5464741}
\author{Isabelle Udry\orcid{}\affiliation{University of Fribourg, Institut de Plurilinguisme; Zurich University of Teacher Education} and Raphael Berthele\orcid{}\affiliation{University of Fribourg, Institut de Plurilinguisme} and Carina Steiner\orcid{}\affiliation{University of Bern, Center for the Study of Language and Society}}
\title[Theoretical framework of the LAPS project]
      {Language Aptitude at Primary School (LAPS): 
      Theoretical framework of the project }
\abstract{This chapter introduces the theoretical framework of the project Language Aptitude at Primary School (LAPS). We considered the impact of a range of individual difference (ID) variables and environmental factors on children’s foreign language proficiency. These variables will be discussed in turn, starting with an overview of the language aptitude construct. ID variables pertaining to general cognitive abilities are discussed next, namely intelligence, working memory (WM), creativity, field independence as cognitive style, and metalinguistic awareness. This is followed by an outline of L2 motivation and related constructs to depict the affective dispositions that underlie foreign language learning, i.e. L2 self-concepts, L2 anxiety, and locus of control. Lastly, we discuss the role of environmental factors, such as socioeconomic status, parent education, and teaching paradigm. }

\IfFileExists{../localcommands.tex}{
  \addbibresource{localbibliography.bib}
  \usepackage{langsci-optional}
\usepackage{langsci-gb4e}
\usepackage{langsci-lgr}

\usepackage{listings}
\lstset{basicstyle=\ttfamily,tabsize=2,breaklines=true}

%added by author
% \usepackage{tipa}
\usepackage{multirow}
\graphicspath{{figures/}}
\usepackage{langsci-branding}

  
\newcommand{\sent}{\enumsentence}
\newcommand{\sents}{\eenumsentence}
\let\citeasnoun\citet

\renewcommand{\lsCoverTitleFont}[1]{\sffamily\addfontfeatures{Scale=MatchUppercase}\fontsize{44pt}{16mm}\selectfont #1}
   
  %% hyphenation points for line breaks
%% Normally, automatic hyphenation in LaTeX is very good
%% If a word is mis-hyphenated, add it to this file
%%
%% add information to TeX file before \begin{document} with:
%% %% hyphenation points for line breaks
%% Normally, automatic hyphenation in LaTeX is very good
%% If a word is mis-hyphenated, add it to this file
%%
%% add information to TeX file before \begin{document} with:
%% %% hyphenation points for line breaks
%% Normally, automatic hyphenation in LaTeX is very good
%% If a word is mis-hyphenated, add it to this file
%%
%% add information to TeX file before \begin{document} with:
%% \include{localhyphenation}
\hyphenation{
affri-ca-te
affri-ca-tes
an-no-tated
com-ple-ments
com-po-si-tio-na-li-ty
non-com-po-si-tio-na-li-ty
Gon-zá-lez
out-side
Ri-chárd
se-man-tics
STREU-SLE
Tie-de-mann
}
\hyphenation{
affri-ca-te
affri-ca-tes
an-no-tated
com-ple-ments
com-po-si-tio-na-li-ty
non-com-po-si-tio-na-li-ty
Gon-zá-lez
out-side
Ri-chárd
se-man-tics
STREU-SLE
Tie-de-mann
}
\hyphenation{
affri-ca-te
affri-ca-tes
an-no-tated
com-ple-ments
com-po-si-tio-na-li-ty
non-com-po-si-tio-na-li-ty
Gon-zá-lez
out-side
Ri-chárd
se-man-tics
STREU-SLE
Tie-de-mann
} 
  \togglepaper[1]%%chapternumber
}{}

\begin{document}
\maketitle 

\section{Introduction}

The aim of the project Language Aptitude at Primary School (LAPS) was to explore the impact of a set of individual difference (ID) variables and environmental factors on young learners’ developing foreign language proficiency. Of particular interest was how language aptitude, as defined by \citet{Carroll1958}, is involved in child learning, a research topic that has only recently started to attract scholarly attention (see \sectref{sec:01:2.2}). The project was carried out in two stages. First, we investigated L2 French and L3 English proficiency cross-sectionally (LAPS I $n=174$). Second, we recorded children’s development of L2 English proficiency over 1.5 years (LAPS II, $n=637$). The children were aged 10--12 years and learnt two foreign languages in a minimal input setting with 2--3 lessons a week. More details on the study design can be found in Chapter 2, a concise summary of LAPS I and LAPS II is given in the Introduction to the volume. 

In the following, we detail the theoretical underpinnings of the ID variables and environmental factors that were considered in the LAPS project. 

\section{Language aptitude}\label{sec:01:2}
\subsection{Historical overview of language aptitude research and testing}\label{sec:01:2.1} %2.1 /

Language aptitude as a construct associated with language acquisition and learning first emerged in the United States in the late 1920s. Learning a second language (L2) as part of tertiary education was encouraged, but little time and money were allocated to foreign language classes. As a consequence, failure rates in these courses were high \citep{Spolsky1995}. Representatives of various educational institutions expressed their concerns and argued for the use of aptitude tests as a way of selecting only suitable candidates for their programs. Calls for prognostic testing became even more pronounced after World War II, when the US army reported an increased demand for staff with good language learning abilities. As a result, aptitude research aiming to develop efficient tests was encouraged and funded by the government (\citealt{StansfieldReed2004}). 

John B. Carroll was the first to conceptualize language aptitude. He administered a range of tests assessing relevant abilities for L2 learning to two Air Force groups (total $n=168$) attending a one-week intensive training course for Mandarin Chinese (\citealt{Carroll1958,Carroll1964,Carroll1958}). From a factor analysis, Carroll derived four factors associated with successful language learning, which he termed language aptitude: 

\begin{description}\sloppy
\item[Phonetic coding ability:] The most important component according to Carroll, consisting of the ability to code auditory phonetic material so that it can be recognized, identified and remembered.

\item[Grammatical sensitivity:] The ability to recognize the syntactic function of a particular word in a sentence.

\item[Inductive learning ability:] The ability to discover grammatical rules independently and without explanation.

\item[Rote learning ability:] The ability to establish associations between sound and meaning quickly and efficiently. In other words, the ability to memorize new words rapidly and a sustained capacity for retrieval.

\end{description}

\begin{sloppypar}
\citet{Skehan1998} later proposed a reduction of Carroll’s four dimensions by combining inductive ability and grammatical sensitivity into one subcomponent called \textit{linguistic ability} or \textit{language analytic ability,} while retaining the other two initial components, thus presenting a three-component model. We use this term in Chapter 10 where we discuss the stability of the language analytic aptitude component. 
\end{sloppypar}

From the assessment tools used to define the components, \citet{CarrollSapon1959} selected five tests for the Modern Language Aptitude Test (MLAT) (\tabref{tab:01:1}).

\begin{sidewaystable}
\caption{Modern Language Aptitude Test Battery (MLAT) subtests with short description and assessed components (\citealt{CarrollSapon1959}).\label{tab:01:1}}
\begin{tabularx}{\textwidth}{lQQ}
\lsptoprule
Test & Content & Components\\\midrule
{MLAT 1}

{Number Learning} & {Participants memorize numbers in an artificial language presented to them aurally. During testing, they hear number sequences and write them down.} & {Phonetic coding ability; Rote learning ability}\\
{MLAT 2}

{Phonetic Script} & {Non-words must be assigned to phonetic symbols.} & {Phonetic coding ability}\\
{MLAT 3}

{Spelling Clues} & {Multiple-choice vocabulary task with a time limit. Participants are asked to find a synonym for a word presented to them in a phonetic transcription.} & {Phonetic coding ability (Verbal Learning, high correlation on this factor, but the factor was not retained as an aptitude component due to low factor loadings)}\\
{MLAT 4}

{Words in Sentences} & {The syntactic function of a particular word in a sentence must be recognized and the word with the same function in another sentence must be identified.} & {Grammatical sensitivity}\\
{MLAT 5}

{Paired Associates} & {During a trial phase, participants memorize word pairs (artificial language/English) presented in writing. During actual testing, they must assign the English expression to the corresponding artificial word via a multiple-choice task.} & {Rote learning ability}\\
\lspbottomrule
\end{tabularx}
\end{sidewaystable}


\subsubsection{Newer test batteries}\label{sec:01:2.1.1} %2.1.1 /

Shortly after Carroll presented his work, Paul Pimsleur published the PLAB (Pimsleur Language Aptitude Battery) for adolescents from grades 7 to 12 (\citealt{Pimsleur1966}, \citealt{PimsleurQuinn1971}). Its basic structure is similar to the MLAT, but the PLAB differs in including a measure of inductive learning ability and participants’ marks from subjects other than languages. Also, Pimsleur regarded motivation as a prerequisite for L2 learning independent of aptitude and dedicated a separate section to it. The PLAB consists of six parts: 1) grade point average in academic areas other than foreign languages, 2) questionnaire on interest in learning a foreign language, 3) vocabulary (word knowledge in L1 English), 4) language analysis (ability to induce rules in an artificial language), 5) sound discrimination (ability to memorize and recognize new phonetic distinctions), and 6) sound-symbol association.

\begin{sloppypar}
Over time, a shift in research focus occurred, moving from predicting L2 achievement to explaining the underlying mechanisms of language learning. This inspired the development of new test instruments that connected more with current theories on second language acquisition (SLA) and allowed for assessing aptitude differentially in terms of learning stages or learning contexts. 
\end{sloppypar}

For example, the CANAL-FT (Cognitive Ability for Novelty in Acquisition of Language (Foreign) Test) by \citet{GrigorenkoEtAl2000} is based on a cognitive theory of knowledge acquisition (p. 392). The CANAL-F theory states that a crucial ability for foreign language acquisition is the ability to cope with novelty and ambiguity. The test therefore simulates naturalistic learning by gradually introducing participants to an artificial language. It assesses specific mechanisms relevant for language processing, including selective and accidental encoding, selective comparison, selective combination and selective transfer. It is dynamic as it allows for learning during testing. 

The MLAT identifies individuals that are likely to progress fast at the beginning of language learning. In contrast, the Hi-LAB (\citealt{DoughtyEtAl2010}, \citealt{LinckEtAl2013}) aims to predict high-level attainment in advanced stages of learning. The test includes measures of working memory, associative memory, long-term memory retrieval, implicit learning, processing speed, and auditory perceptual acuity. However, few papers have been published on the validity of the test \citep{LinckEtAl2013}, making it difficult to gauge its relevance. 

Neither the Hi-LAB nor the CANAL-FT are publicly available and information on content or administration can only be inferred from the literature; the same goes for the DLAB (Defense Language Aptitude Battery, \citealt{PetersenAlHaik1976}) and the VORD (\citealt{ParryChild1990}), two other tests mentioned in the literature which are copyrighted by the US government and only administered to its personnel \citep{Robinson2002}. On the other hand, MLAT and PLAB are commercially licensed (although the MLAT for adult learners seems currently only to be sold to government agencies).\footnote{According to information gathered from the Language Learning and Testing Foundation \url{https://lltf.net/aptitude-tests/language-aptitude-tests/modern-language-aptitude-test-2/}, last accessed on January 12, 2021.}

A freely available test is the LLAMA \citep{MearaEtAl2001}, a computer test battery developed, by Paul Meara and his team at the University of Swansea (UK). The LLAMA battery comprises four parts loosely based on the MLAT: Vocabulary learning (LLAMA B), phonemic discrimination (LLAMA D), sound-symbol correspondence (LLAMA E) and inductive ability (LLAMA F). Instructions and tests are administered with pictograms and visual stimuli. Its language-independence makes the test suitable for all participants, regardless of L1 or level of literacy. However, the LLAMA has not been standardized, a disadvantage that is emphasized by the authors themselves (see \url{http://www.lognostics.co.uk/tools/llama}{).} Nevertheless, it has been used by numerous research teams and is considered by many to be a reliable tool in aptitude research \citep{RogersEtAl2017}

\subsubsection{Critical views on aptitude testing} %2.1.2 /

The language aptitude components and the MLAT have been derived from empirical data, rather than a specific theory of foreign language learning. The construct is therefore closely linked to the test instruments that measure it. For this reason, language aptitude has been described as “a construct which is, in fact, nothing more or less than what the test measures” (\citealt{SafarKormos2008}: 4). Several inconsistencies between the MLAT subtests and the components they target, have added to the controversy over what the test actually stands for \citep{Carpenter2008}. Most notably, the various subtests cannot be assigned clearly to their corresponding aptitude component. For instance, some subtests cover more than one ability (e.g., part 1 “Number learning” assesses both phonetic coding ability and rote memory). Similarly, some components are measured by several tests (rote memory by parts 1 and 5; phonetic coding ability by parts 1, 2 and 3). On the other hand, no test was designed to tap into inductive ability, due to practical reasons of test administration \citep{Carpenter2008}. This component of language aptitude was thus only weakly assessed in part 1 “Number Learning”. The strong yet poorly specified link between Carroll’s aptitude construct and the MLAT test, makes it difficult to build a concise conceptual aptitude framework. Meta-analytical evidence by \citet{Li2016} reveals that commonly used aptitude measures demonstrate differential predictive validities, suggesting that cross-validation of test batteries is called for to determine the extent to which they tap into the same construct. Yet, large scale investigations of aptitude tests are scarce and only few comparative studies exist (for a discussion see \citealt{StansfieldReed2019}).\largerpage

The MLAT and its derivates typically rely on discrete-point testing, i.e. they focus on a particular linguistic form which is measured on an item basis. Participants are not given the opportunity to apply language within a context or show their pragmatic skills, an approach that may be more consistent with communicative teaching methods used today \citep{Singleton2017}. The relevance of MLAT-based tests for meaning-focused learning has thus been questioned on several accounts (\citealt{Krashen1981}, \citealt{Stansfield1989}, see also \citealt{Singleton2017}). This is particularly relevant for early instructed language learning and teaching, which is usually based on communication with a focus on fluency over accuracy (for the context of this study see Chapter 2, \sectref{sec:02:2.1}). Nevertheless, MLAT-derived tests have been successfully used with young learners (see \sectref{sec:01:2.2.2}) and shown explanatory power for their L2 proficiency. Also, as outlined by \citet{StansfieldReed2019}, several recent studies with adults conducted at US state institutes\footnote{US Defense Language Institute \citep{Winke2013} and the US Foreign Service Institue \citep{Ehrman1998}.} (which were reported to adhere to task-based communicative teaching) also indicate that the MLAT remains a sound predictor for L2 proficiency in these learning contexts.\largerpage

Even though considerable efforts have been made to develop new test batteries, the MLAT remains widely used in the scientific community. Other tests, such as the PLAB, LLAMA, or Hi-LAB have been modelled on it, highlighting how strongly the Carrollian take on language aptitude is still shaping the field. This may be explained by the fact that designing and validating new tools that consider SLA theories and meet the criteria for test quality is challenging. The fact that some test batteries, such as the Hi-LAB, CANAL-F, VORD, or DLAB, are withheld from the public \citep{AmeringerEtAl2019} impedes the scientific community from finding common ground in conceptualizing these measures. The MLAT has been recognized as the foundation of aptitude research. While it is less suited for validating the aptitude construct as outlined previously, its predictive value for L2 proficiency has been repeatedly demonstrated \citep{Li2016}. From this point of view, its continued use appears legitimate.

\subsubsection{New conceptions of language aptitude}\label{sec:01:2.1.3}%2.1.3 /

With a fading interest in the performance-based selection of students prevalent in the early days, explaining the role of various aptitude components for L2 learning and acquisition has become a main concern for researchers \citep{Li2019}. New models have emerged from this explanatory-interactional approach (for a concise overview see \citealt{WenEtAl2017}).\largerpage

The \textit{Macro-SLA aptitude model} by \citet{Skehan2002, Skehan2019} maps aptitude components (and corresponding aptitude sub-tests) onto stages of L2 learning. In a recent conceptualization of the model, \citet{Skehan2019} identifies three general (or macro) acquisitional stages organized around 1) handling sound (input processing and segmentation; noticing); 2) handling pattern (identifying; generalizing; and integrating patterns; handling feedback); and 3) automatizing-pro\-ce\-du\-ra\-liz\-ing (avoiding error; automatization; lexicalization). Skehan argues that aptitude components are implicated differently as L2 development progresses. Phonetic coding ability is associated with initial stages of learning when processing auditory input is crucial (handling sound). The remaining two components are more relevant at advanced stages when acquiring complex language structures is important: Language analytic ability helps to recognize and manipulate speech patterns (handling pattern), whereas memory contributes to retaining and retrieving information (automatization). The model comprehensively integrates constructs from language aptitude research and theories of SLA. Nevertheless, this integration still remains conceptual to a large extent and more empirical support is needed to validate it. 

The \textit{Aptitude complexes framework} was conceived by \citeauthor{Robinson2001} (\citeyear{Robinson2001}; see also \citealt{Robinson2002}) to be applied to instructed foreign language learning. The framework postulates aptitude clusters that consist of cognitive resources (memory, attention, basic processing speed), language-specific abilities (e.g., noticing the gap, memory for contingent speech) and do\-main-gen\-er\-al, primary cognitive abilities that support language acquisition (e.g., perceptual speed or pattern recognition). Robinson argues that individual learner characteristics reflected in these aptitude complexes are compatible with specific teaching methods. For instance, the aptitude cluster for incidental learning (via oral content) combines well with a communicative classroom setting where linguistic phenomena are mediated implicitly. The practical aim of this framework is to enhance L2 learning by matching teaching method to aptitude complex.

Other models that re-conceptualize language aptitude include the linguistic coding differences hypothesis (LCDH) by \citet{SparksGanschow1991}, which takes on the view of learning difficulties and the L1--L2 connection, the distinction between an explicit and implicit language aptitude (\citealt{Granena2012,Granena2016}) or the brain-network-based view on language aptitude in the neuro-scientific perspective (\citealt{GolestaniEtAl2011}, \citealt{ReitererEtAl2013}). Even other models are linked to the development of new test batteries and have been touched upon in \sectref{sec:01:2.1.1}: the high level language aptitude battery (Hi-LAB) model with a focus on exceptional language learners and the CANAL-F theory that highlights the ability to deal with novelty and ambiguity in language learning. 

\subsection{Aptitude differences in children}\label{sec:01:2.2} %2.2 /
\largerpage
Due to a focus on student selection for state-funded language programs, early aptitude research was mainly concerned with adults and adolescents. It was not until 1976 that Carroll and Sapon adapted their MLAT (\citealt{CarrollSapon1959}) to create the first test battery for children, the Modern Language Aptitude Test~– Elementary (MLAT-E). Still widely used today, it is designed for L1 English speakers between 9 and 12 years of age (grades 3 to 6) and consists of four subtests outlined in \tabref{tab:01:2}.

\begin{sidewaystable}
\caption{Modern Language Aptitude Test Battery – Elementary (MLAT-E) subtests with short description and assessed components (\citealt{CarrollSapon1976}).\label{tab:01:2}}
\begin{tabularx}{\textwidth}{lQQl}
\lsptoprule
{Name of the test} & {Content} & {Component} & {Based on MLAT subtest}\\\midrule
{1. Hidden Words} & {Multiple-choice vocabulary task with a time limit. Participants are asked to find the L1 English definition for a word presented to them in a phonetic transcription.} & {English vocabulary} 

{Phonetic coding ability} & {Spelling Clues}\\
{2. Matching Words}  & {Recognizing the function of a particular word in a sentence.} & {Grammatical sensitivity} & {Words in Sentences}\\
{3. Finding Rhymes} & {In a multiple-choice task, participants need to select words that rhyme.} & {Phonetic coding ability}

{(ability to hear speech sounds)} & {Not in the original MLAT}\\
{4. Number Learning} & {Participants memorize numbers in an artificial language presented to them aurally. During testing, they hear number sequences and write them down.} & Phonetic coding\footnote{As stated in the MLAT-E manual (\citealt[2]{CarrollSapon1976}) the Number Learning subtest also taps into what the authors refer to as “a special `auditory alertness' factor which would play a role in auditory comprehension of a foreign language” However, “auditory alertness” was not retained as a subcomponent of Carroll’s aptitude construct.}

{Rote memory} & {Number Learning}\\
\lspbottomrule
\end{tabularx}
\end{sidewaystable}

The lack of interest in young learners was further owed to the assumption that language aptitude accounts for L2 achievement in adults, but not children \citep{Li2018}. This claim is made with reference to the fundamental difference hypothesis (FDH, \citealt{BleyVroman1989}) and the critical period hypothesis (CPH) popularized by \citet{Lenneberg1967}. The FDH and CPH argue that children draw on implicit, domain-specific mechanisms to learn languages. Due to maturational changes, they lose access to the domain-specific abilities upon entering puberty and start to rely on domain-general abilities instead. It is further argued that language cannot be learnt fully by domain-general mechanisms, particularly in relation to grammar and phonology. As will be discussed in the next section, exceptional cases of high attainment in late starters, i.e. individuals that started learning a L2 after completion of the supposed critical period, have therefore been linked by some scholars to above-average levels of language aptitude \citep{DeKeyser2000}, particularly its verbal analysis component.

\subsubsection{Child aptitude and ultimate L2 attainment}\label{sec:01:2.2.1}  %2.2.1 /

One perspective on child aptitude is its effect on ultimate L2 attainment in adults. Based on work by \citet{JohnsonNewport1989}, \citet{DeKeyser2000} examined the role of language aptitude (along with age of arrival and years of schooling) as predictors for L2 English grammaticality judgment (GJ) accuracy among 57 Hungarian immigrants to the US. The participants were divided into groups of early ($n=15$) and late arrivals ($n=42$), as well as high aptitude ($n=15$)\footnote{The 15 participants in the high aptitude group are not identical to the $n=15$ of the early arrival group.} and average- or low-aptitude ($n=42$). Only few late arrivals reached scores within the range of early arrivals on the GJ test. Those who did all had high levels of language aptitude, operationalized as verbal analytical ability. Overall, language aptitude was not predictive of GJ accuracy. However, for late arrivals, GJ scores were significantly and positively correlated with verbal analytical ability. From this, the author concluded that language aptitude plays a role for ultimate attainment only for late starters, thus providing an explanation for those highly successful individuals that challenge the CPH.\footnote{See also \citealt{Vanhove2013} for a critical view on what counts as statistical evidence in favor or against CPH.} Similarly, in a study with 65 Chinese learners of Spanish, \citet{GranenaLong2012} found aptitude effects only for late learners whose first contact with the L2 happened between the ages of 16 and 29 years ($n=18$). Significant correlations between aptitude and pronunciation, aptitude and lexis and aptitude and knowledge of collocations were found, but not between aptitude and morphosyntax. 

\citet{AbrahamssonHyltenstam2008} provided evidence on the role of language aptitude for late starters ($n=11$) with 42 near-native L2 speakers of Swedish with L1 Spanish. But contrary to \citet{DeKeyser2000}, the authors also found aptitude effects for early starters. Yet, the authors concluded that finding a few individuals with high aptitude “does not justify a rejection of the criticial period hypothesis” (\citealt{AbrahamssonHyltenstam2008}: 503).  

Adding to the mixed findings is \citet{Granena2012}, who examined age and aptitude in relation to ultimate L2 attainment with 100 Chinese-Spanish bilinguals. She identified two types of language aptitude: One for explicit learning (termed analytic ability) and one for implicit learning (defined as sequence learning ability) and found that both affected early L2 learners’ attainment. 

\begin{sloppypar}
Several things may contribute to the inconclusiveness of these results. First, proficiency and aptitude were operationalized differently and therefore measured with different tools, making it difficult to compare findings. For instance, aptitude was assessed with the LLAMA \citep{MearaEtAl2001} by \citet{AbrahamssonHyltenstam2008}, and \citet{GranenaLong2012} while \citet{DeKeyser2000} used a subtest of language analysis from a Hungarian aptitude test (adapted from the MLAT Words in Sentences subtests by \citealt{Otto1996}). L2 proficiency was measured by an aural GJ task in \citet{DeKeyser2000}, an aural and written GJ task in \citet{AbrahamssonHyltenstam2008}, or several tests of different language domains, including pronunciation, lexis, and morphosyntax by \citet{GranenaLong2012} and \citet{Granena2012}. Furthermore, different criteria were applied to define age groups: With cut off points for early learners at 12 years (\citealt{AbrahamssonHyltenstam2008}) or 16 years \citep{DeKeyser2000}, \citet{GranenaLong2012} and \citet{Granena2012} had three groups with ages of onset between 3--6, 7--15 and 16--29. 
\end{sloppypar}

These studies were concerned with aptitude effects on ultimate L2 attainment in naturalistic contexts. Despite similar L2 learning conditions, participants may still have experienced very diverse linguistic environments including some form of formal instruction. A variety of variables, beyond aptitude or age of onset, may therefore account for ultimate achievement. As pointed out by \citet{Birdsong2014}, the DeKeyser study was built around critical period effects in relation to age of arrival and L2 proficiency. As a result, the explanatory power of education (assessed as years of schooling) was not fully explored. Reanalyzing the same data, \citet{Birdsong2014} found that years of schooling was in fact the most robust predictor of grammatical proficiency with significant correlations in all age and aptitude groups. Education and aptitude, however, did not correlate for any age group, indicating that the two variables make independent contributions.

\subsubsection{Studies with children}\label{sec:01:2.2.2} %2.2.2 /

With early instructed language learning being introduced across Europe (see Introduction, \sectref{sec:01:3}), the age factor has gained in importance on the research agenda and has led to the publication of several studies with children. They are concerned with 1) evaluating the predictive power of language aptitude (and its subcomponents) for L2 proficiency, 2) developing test batteries for young learners, 3) the stability of language aptitude, and 4) its relationship with other constructs, such as metalinguistic awareness or motivation. Some studies combined these aspects, for instance, validation studies of newly developed aptitude tests by \citet{KissNikolov2005} or \citet{SuarezVilagran2010} also investigated age-related questions. The most notable findings will be presented in the following.

First of all, it is worth pointing out that despite assumptions drawn from FDH and CPH that aptitude may be irrelevant for child learning, studies have consistently found language aptitude to be a predictor of L2 proficiency in young learners (\citealt{BialystokFroehlich1978}, \citealt{KissNikolov2005}, \citealt{Kiss2009}, \citealt{SuarezVilagran2010}, \citealt{Munoz2014}, \citealt{TellierRoehrBrackin2017}, \citealt{RoehrBrackinTellier2019}). 

For instance, \citet{TellierRoehrBrackin2017} tested 178 8- to 9-year-old English-speaking beginning learners of French on metalinguistic awareness and language aptitude (tested with a British version of the MLAT-E). Language aptitude was shown to have a significant effect on children’s progress in L2 French classes with a form-focused element. 

\citet{KissNikolov2005} developed, piloted and validated an aptitude test in Hungarian, modelled on the MENYÉT (\citealt{Otto1996}, in \citealt{KissNikolov2005}), a Hungarian adaptation of the MLAT (\citealt{CarrollSapon1959}) and the PLAB \citep{Pimsleur1966}. Their final version for young learners consists of 4 subtests (targeted aptitude component in brackets): 

\begin{enumerate}\sloppy
\item \textit{Hidden sounds:} Associating sounds with written symbols (phonetic coding);
\item \textit{Words in sentences:} Identifying semantic and syntactic functions in Hungarian sentences (grammatical sensitivity); 
\item \textit{Language analysis:} Recognizing structural patterns in an artificial language, based on part 4 of the PLAB (inductive ability); 
\item \textit{Vocabulary learning:} Paring words and phrases in an artificial language with Hungarian equivalents (rote memory). 
\end{enumerate}

\citet{KissNikolov2005} administered the aptitude test along with measures of motivation and English proficiency (listening, reading, writing) to 419 12-year-old children learning English as a foreign language. Time of exposure to English at school and in private tuition ranged considerably from 100 to 1,085 hours ($M= 343; \text{SD}= 131$). Multiple regression analysis indicated that language aptitude was the best predictor of outcomes, explaining over 20\% of the variance in L2 English proficiency. Motivation also made a significant contribution, explaining 8\% of the variance. Moreover, the authors found a weak correlation between time spent on learning and aptitude scores. From this they concluded that language aptitude in the Carrollian sense did not improve with “the amount of time used for practice and exposure” (\citealt[134]{KissNikolov2005}).

\citet{Kiss2009} adapted and piloted a version of this Hungarian test battery for 8-year olds. This was done with a practical aim in mind, i.e. selecting 26 children out of 52 for a dual Hungarian-English language program. After one year of study in the bilingual class, the children ($n=25$\footnote{One child was absent on the day of testing.}) were tested for English proficiency with a short interview. Their progress was also rated by their teachers. Achievement was related to the aptitude scores taken before they had entered the program. Most notably, the author compared the results from the 8-year-olds to those from 12-year-olds from a previous study. She found that the 12-year-olds performed much better on the vocabulary learning subtest than the younger children. \citet{Kiss2009} argued that the older children had more language learning experience and better developed strategies. Based on the idea that aptitude malleability can be evidenced by increased group averages, the author concluded that language aptitude is dynamic and shaped by language experience, at least up to the age of 12.

\begin{sloppypar}
\citet{SuarezVilagran2010} validated adaptations of the MLAT-E into Spanish (MLAT-ES, \citealt{StansfieldReed2005}) and Catalan (MLAT-EC, \citealt{SuarezVilagran2010}) with 629 Spanish-Catalan bilingual learners of English from grades 3 to 7 (aged 8,3--14,9). MLAT-ES and MLAT-EC are structured like the MLAT-E and, unlike the Hungarian version, they do not contain a test for inductive ability. There are four tasks (targeted aptitude component in brackets): 
\end{sloppypar}

\begin{enumerate}
\item \textit{Hidden words:} Ability to utilize previously learned sound-symbol associations (phonetic coding); 
\item \textit{Matching words:} Sensitivity to grammatical structures presented in the target language (grammatical sensitivity); 
\item \textit{Finding rhymes:} Ability to recognize sequences of orthographically presented speech sounds (phonetic coding); 
\item \textit{Number learning:} Memorizing numbers in an artificial language (rote memory and phonetic coding). 
\end{enumerate}

\citet{SuarezVilagran2010} measured foreign language proficiency with a multiple-choice listening test and a cloze passage in all grades. In addition, children in grades 5, 6, and 7 took a dictation test. The author found both test batteries to be valid measures for predicting L2 proficiency, although not for speaking. In terms of the subcomponents, the Hidden Words test (phonetic coding) showed the lowest correlations with proficiency across all grades, while Matching Words (grammatical sensitivity) and Finding Rhymes (ability to hear speech sounds) were significantly correlated with L2 proficiency from grades 4 to 7. The author also highlighted some age-related findings: Overall, mean scores stabilized between grades 6 and 7 \citep[349]{SuarezVilagran2010}. Grade 3 showed notable patterns in several respects: Correlations between aptitude and language proficiency for grade 3 were consistently lower than for other grades. Similar to arguments put forward by \citet{Kiss2009}, the author relates this to cognitive development, notably less developed strategies for problem-solving, for encoding and memorizing information. Also, grade 3 students scored lower on metalinguistic awareness tests than the older participants.

\subsubsection{Child aptitude and memory}\label{sec:01:2.2.3} %2.2.3 /

Moreover, \citet{SuarezVilagran2010} found that correlations between the Number Learning test (rote memory) and L2 proficiency decreased as children got older, suggesting that memory is more important for younger learners than for older ones. This finding is in line with the high importance of exemplar-based learning in child L1 acquisition as argued, e.g., by \citet{Tomasello2005}. In such a usage-based framework, variations in memory capacity are expected to be strongly associated with language learning particularly in young learners.

Investigating the relationship between language aptitude components and L2 proficiency, \citet{Munoz2014} also found slightly stronger effects for rote memory on language outcomes in 48 Spanish-Catalan bilinguals aged 10--12 years, learning L2 English. The author administered the MLAT-ES along with measures of L2 listening, reading, writing, and speaking. Her results corroborate findings, such as the ones from \citet{SuarezVilagran2010}, that “children rely on memory to a large extent”, \citep[64]{Munoz2014}. While \citet{Munoz2014} highlights children’s reliance on memory, she also emphasizes the importance of the other aptitude components; in particular the author suggests that language-analytic abilities are likely to be the key component for high achievement. 

Memory did not always yield the strongest correlations with L2 proficiency: \citet[140]{KissNikolov2005} found both memory and analytical abilities to be relevant for L2 proficiency and in \citet{RoehrBrackinTellier2019} analytic ability emerged as the strongest predictor, followed by a measure of phonetic ability.

\subsubsection{Aptitude in children with beginning literacy skills} %2.2.4 /

The studies outlined so far have found aptitude effects for children at the primary and early secondary level. Alexiou (2009; see also \citealt{MiltonAlexiou2006}) was interested in even younger children with beginning or no literacy skills. Based on the work by \citet{EsserKossling1986}, the author designed the YLAT (Young Learners Aptitude Test) for children between 5 and 7 years. It contains tasks that partially overlap with the Carrollian aptitude components. For instance, inductive ability is assessed with a task in which colors represent groups of objects (blue for flowers, white for animals, etc.) that the child must discover and systematize. Long-term memory is tested by an adapted version of the MLAT subtest Paired Associates, which contains only visual stimuli. Short-term memory, semantic integration, spatial skills, and reasoning ability (sequencing narrative elements) are also assessed by the YLAT. In a study with Greek learners of English aged 5--7 years ($n=191$), \citet{Alexiou2009} found significant correlations ranging from 0.33 to 0.65 between the different dimensions assessed by the subtests and L2 vocabulary (receptive and productive). Her findings corroborate observations that aptitude explains individual differences in child L2 learning from an early age. 

\subsubsection{Child aptitude and musical talent} %2.2.5 /

Indications for a link between musical talent and foreign language learning stem from \citet[$n=35$]{ChristinerReiterer2018} and \citet[$n=36$]{Christiner2018} who investigated pre-schoolers’ musical ability and speech imitation ability as an aspect of language aptitude. Children between 5 and 6 years of age were tested for their ability to discriminate paired musical statements, singing ability, ability to remember strings of numbers and ability to repeat Turkish, which was an unfamiliar language to them. Participants with good performance on the musicality measure also scored high on the imitation tasks and had high working memory capacity compared to participants with lower scores on the musicality test. The authors concluded that musical talent and speech imitation aptitude are related in children. 

\subsection{Aptitude stability}\label{sec:01:2.3} %2.3 /

Whether language aptitude is a stable trait or an ability susceptible to treatment is an ongoing debate in aptitude research. If aptitude was stable (and possibly also innate), success or failure at language learning would be largely predetermined. If, on the other hand, aptitude was trainable, it could be used to enhance foreign language instruction. The question also deserves attention in relation to children who are still developing mentally and physically in various ways. 

In traditional models, aptitude was assumed to be a stable trait (\citealt{Skehan1998}, \citealt{Singleton2017}). A long-term study by Skehan (1986, \citealt{SkehanDucroquet1988}) has been widely held to corroborate this view. The authors assessed the language development of children in their L1 and 13 years later in their L2. Some measures in the L1 collected between the ages of 39 months and 57 months proved to be related to measures of L2 acquisition. In particular, L1 vocabulary and early mean length of utterance were correlated with later L2 aptitude test scores. From this link between L1 acquisition indices and L2 learning, the authors concluded that an aptitude for language learning is a stable individual characteristic. In Chapter 9, the development of L1 German reading comprehension and L2 English proficiency also shows similar predictive variables. 

Early research on aptitude with its objective of selecting the apt individuals, explicitly or implicitly assumed individual differences in aptitude to be innate \parencites[122]{Carroll1964}[8]{Carroll1973}. More recent research on the genetic contribution to L2 learning seems to support the idea that substantial proportions of the variability in learning outcomes are explained by genetics (\citealt{Stromswold2001}, \citealt{RimfeldEtAl2015}, \citealt{Plomin2019}). Recent scholarly approaches in aptitude research have attempted to explain processes of SLA, rather than predict learning outcomes \citep{WenEtAl2019}. As a result, some authors now model language aptitude as an array of abilities that can potentially be developed. For instance, \citet[401]{GrigorenkoEtAl2000} refer to a form of “developing expertise rather than an entity fixed at birth”. Carroll later expressed himself neutral on the issue of aptitude stability, arguing that no empirical evidence was available to decide on the matter \citep[86]{Carroll1981}. In the same paper, \citet[84]{Carroll1981} suggested that his initial aptitude components could be modelled as “more or less enduring characteristics” and as a “current state”.

\begin{sloppypar}
Empirical studies of construct stability are scarce and generally seem to confirm its malleability (\citealt{SafarKormos2008,SuarezVilagran2010,RoehrBrackinTellier2019}). It is worth noting that, except for \citealt{RoehrBrackinTellier2019}, researchers relied on cross-sectional data to infer developmental patterns, rather than multiple indications from the same participants collected longitudinally. Moreover, the designs were based on the premise that language experience (i.e. instructed learning), leads to developments in language aptitude, especially the language analysis component (grammatical sensitivity and inductive ability). Gain scores in aptitude measures were therefore interpreted as an indication of aptitude development. However, these gain scores may be due to other cognitive changes, rather than changes in language-analytic ability. Children in particular are still evolving in terms of literacy and reasoning skills. Due to general developmental processes, young learners are expected to do better at the same aptitude test as they mature. An increase in aptitude mean scores with growing age may not be a reliable indicator for aptitude malleability. In order to ascertain if other developmental mechanisms are implicated in improved test results, these results would need to be compared to age-normed charts, such as provided for instance for standardized intelligence tests. These charts allow for classifying an individual’s score in comparison to a representative sample from the same age group. As we argue in Chapter 10, another way of investigating aptitude stability is to look for individual patterns of development in longitudinal data with several measurements for the same participants.
\end{sloppypar}

\subsection{Language aptitude and pedagogy}\label{sec:01:2.4}%2.4 /

\subsubsection{Aptitude Treatment Interaction (ATI)}\label{sec:01:2.4.1} %2.4.1 /

The description of different aptitude complexes outlined in \sectref{sec:01:2.1.3} (\citealt{Robinson2001,Robinson2002}) opens up new perspectives for researching and planning foreign language teaching: If learners have different strengths, it is to be expected that successful learning depends on the way these individual strengths can be attended to in the classroom. The assumption that matching aptitude profiles with certain teaching methods will increase learning gains is at the core of the aptitude-treatment-interaction (ATI) approach. 

Based on founding work by \citet{Snow1991}, Robinson extended the ATI framework to L2 learning and teaching. To date, ATI has explored the presumed interface between aptitude and learning environment along the lines of: 1) Implicit and explicit instruction, which both seem to be influenced by IDs in aptitude (de \citealt{Graaff1997}, \citealt{Robinson1997}, \citealt{Williams1999}); 2) deductive and inductive instruction, with current results suggesting that a deductive approach combined with extensive opportunities for production seems to benefit all learner types, regardless of aptitude profiles \citep{Erlam2005}; and 3) corrective feedback. A synthetic review by \citet{Li2017} revealed that language aptitude was moderately correlated with the effectiveness of corrective feedback ($r=0.42$), and more strongly with explicit feedback ($r=0.59$) than implicit feedback ($r=0.32$).

So far, only one study has explored the connection between aptitude profiles and instructional treatments on a large scale. \citet{Wesche1981} derived aptitude profiles for each participant from three different sources: Aptitude tests (MLAT and PLAB), L1 proficiency measures and an interview with an experienced teacher. Pairs of learners with the same profile were assigned to different instructional groups: One person was taught according to their profile, the other one according to a method that was unsuitable for their profile. The choice was between three teaching methods: The analytical approach (best suited for highly analytical students with strong L1 skills and perfectionist tendencies); the functional approach (appropriate for students with a relatively restricted command of their L1, yet with good memory capacity); and the audio-visual method (the most common way of teaching at the time of the study and best-suited for non-type-specific learners). After 55 lessons, participants who were exposed to a suitable teaching method achieved higher L2 proficiency scores and reported more pleasure in language learning than their counterparts. 

\subsubsection{Aptitude and classroom practice} %2.4.2 /

Several contributions from ATI to the foreign language classroom are worth contemplating (\citealt{Cook2001}, \citealt{Ranta2008}).

In the prognostic view, aptitude tests are used to make inferences about students’ development, a practice that reminds us of the early days of aptitude testing. By stipulating certain thresholds of scores, students can be selected or dispensed from language classes, depending on how well they reach the prescribed levels. Aptitude tests have also been used for student placement, with scores being interpreted as an indication of how well an individual will be likely to cope with foreign language instruction. Remember that aptitude tests are reliable predictors for L2 outcomes. They provide information on cognitive-linguistic aspects of the individual but say little about, for instance, motivation to learn the language. In order to fully gauge a student’s potential, it is advisable to supplement aptitude tests with assessments of motivation, general learning abilities, and careful consideration of the implications for the student’s academic future.

Findings from the exploratory-interactional approach are suited for diagnostic assessment purposes, i.e. for counselling students based on their aptitude strengths and weaknesses. For example, students with good language analytic ability could be advised to choose explicit learning. Memory-oriented students, on the other hand, could be guided towards communicative classes, since they are likely to learn through modelling (see section 3.2). However, this implies that schools can actually provide an infrastructure that accommodates these different choices. A tangible example of how aptitude clusters could be used for counselling seems to come from \citet{Doughty2013}: Students’ scores from the Hi-LAB (see 1.1.1) are visualized in a so-called aptitude profile card, which is available to learners and teachers along with advice for individual learning. Unfortunately, there is little information available on the effectiveness and exact implementation of these cards. 

\citegen{Wesche1981} intervention study discussed in \sectref{sec:01:2.4.1} is the only large-scale attempt to assign entire groups of students to a type of instruction based on their aptitude profiles for a longer period of time. Her study took place in a particular educational context with adults when learning was mainly form-focused and communicative teaching was the alternative option. Current teaching practices and learning settings differ quite considerably, especially for children. An alternative to Wesche’s approach consists in using different instructional techniques simultaneously in the same classroom, adapting continuously to individual learner requirements. For instance, if high-aptitude students benefit more from explicit corrective feedback and low-aptitude learners from implicit corrective feedback \citep{Li2017}, then both types should be used by the teacher during a lesson based on students’ needs. 

Also, drawing on language aptitude for internal differentiation regarding treatment within clusters (groups of learners, classes) assumes that there is indeed an interaction between aptitude and instruction. \citet{Erlam2005} investigated such an interaction with three teaching styles (inductive, deductive and structured input) in relation to the aptitude profiles of 60 Anglophone learners of French at secondary school. The author found that a deductive approach combined with extensive opportunities for productive output was beneficial to all learners, regardless of their aptitude profile. Her results suggest that a particular type of instruction (i.e. deductive\,+\,productive output) may diminish the influence of individual aptitude differences. It would therefore suffice to teach according to this method without providing aptitude-based differentiation. The kind of finding reported by \citet{Erlam2005} is worth pursuing as it may offer opportunities for more efficient lesson planning.

One last line of application worth mentioning is linked to the potential trainability of language aptitude, namely language analytic ability, suggested by some authors (\citealt{GrigorenkoEtAl2000}, \citealt{SafarKormos2008}). Fostering these abilities is expected to positively affect L2 learning. To date, however, the direct effects of such a training on L2 proficiency remain to be clearly ascertained empirically for primary school children.

Whereas it is uncontested that learners vary in terms of their aptitude to learn new languages, the practical consequences of this insight for the foreign language classroom are not obvious. In the previous section, we have presented some feasible suggestions which are nonetheless rarely implemented at schools today. Moreover, little is said in the literature on how to connect empirical findings from ATI to classroom practice. This may be due to several reasons. Conducting ATI research is indeed challenging given the wide range of factors that affect language learning, i.e. type of instruction, cognitive processes, IDs. Due to this complexity, ATI studies are usually carried out over short periods of time and with small samples. Because few studies have been conducted within the ATI line of research, too little is known about the interaction between aptitude and treatment. Moreover, the educational relevance of individual learning styles in general – indeed, their very existence – remains highly contested in the field of educational psychology (\citealt{RienerWillingham2010}). The results discussed in \sectref{sec:01:2.4.1} seem promising but we believe that many more similar studies would be required to make sound claims about the effects of ATI based learning settings and to counter the well-argued objections to learning style claims in the literature (\citealt{PashlerEtAl2008}, \citealt{RienerWillingham2010}). 

\section{General cognitive abilities or general learning abilities}\label{sec:01:3}

In order to explore the interplay between domain-specific and domain-general abilities, we included ID variables pertaining to what we refer to as general cognitive abilities or general learning abilities. We start by clarifying some aspects of their relationship with language aptitude. Next, we outline the constructs underlying intelligence, working memory (WM), creativity, and field independence. Finally, we discuss metalinguistic awareness and the language analysis subcomponent of aptitude which are hypothesized to be closely linked.

\subsection{General cognitive abilities and language aptitude} %3.1 /

\citet[89]{Carroll1964} described language aptitude as “a fairly specialized talent (or group of talents), relatively independent of those traits ordinarily included under ‘intelligence’”. His statement was underpinned by the observation that intelligence tests were quite unsuccessful in screening individuals for successful language learning \citep{Carroll1964}. Currently, general psychological mechanisms and processes are often highlighted as underlying language learning and acquisition. Nevertheless, aptitude test items and instructions are usually mediated by language, so the construct is at least language-related. Based on these observations, \citet{Skehan2019} has recently argued for a complementary view, suggesting that domain-general and domain-specific capacities co-exist and should be equally reflected in aptitude research. 

Recent scholarly work has often adopted a domain general perspective, investigating aptitude and intelligence (\citealt{Granena2012,Granena2013}), the role of different memory systems (declarative, procedural, \citealt{Carpenter2008}, \citealt{MorganShortEtAl2014}), or working memory as a distinct aptitude component \citep{Wen2019}. Also, new test batteries include general cognitive measures (i.e. working memory and processing speed in the Hi-LAB, \citealt{LinckEtAl2013}, see \sectref{sec:01:2.1.1}). The connection between language aptitude and general cognitive abilities is likely to remain an important research focus in the future.

\subsection{Intelligence} %3.2 /

\subsubsection{Definitions and operationalizations of intelligence} %3.2.1 /

The earliest model of intelligence goes back to Charles \citet{Spearman1904} who proposed a two-factor model with a general factor (g) plus other, more specific abilities (s). The g factor is thought of as general mental ability involving more or less complex mental activities, such as recognition, recall, speed, visual-motor abilities, motor abilities, reasoning, comprehension and hypothesis-testing activities \citep{Sattler2001}. Several other hierarchical models were derived from Spearman’s work.\footnote{For instance, Thorndike’s multifactor theory of intelligence in the late 1920s, Thurstone’s multidimensional theory of intelligence in the 1930s or Vernon’s hierarchical theory of intelligence in the 1950s \citep{Sattler2001}.} More recently, non-hierarchical models have been put forward which conceive of different forms of intelligence as existing independently and equivalently of each other. Widely known is \citegen{Gardner1983} theory of multiple intelligences, which includes social-emotional, musical, physical-kinaesthetic, interpersonal and intrapersonal forms of intelligence. \citet{Sternberg1985,Sternberg2002} theorizes three forms: Analytical, creative and practical intelligence, which are drawn on “to adapt to, shape, and select environments” \citep[15]{Sternberg2002}.

For the present study, intelligence was operationalized according to \citegen{Cattell1943} two component theory which postulates a type of fluid and crystallized intelligence. Fluid intelligence refers to a general ability to think and problem solve, largely independent of cultural influences. Fluid intelligence is considered an important prerequisite for acquiring new information and therefore learning. Cattell argues that fluid intelligence is largely fixed at birth. In contrast, crystallized intelligence consists of knowledge and skills acquired throughout life. It increases with growing experience and is thought to be influenced by culture and language ability. The two develop differently over the life span, with crystallized intelligence increasing over the years until it stagnates at some point, and fluid intelligence decreasing with age. The two are considered separate factors linked by a common overarching factor g.  Cattell’s model has left its mark on intelligence testing with the development of so-called Culture Fair Tests. These tools are designed to tap into fluid intelligence, thus cancelling out cultural differences that may affect performance. In our study we used the CFT 20-R which is a standardized version for German speaking children from the age of 8 (see also Chapter 2).

\subsubsection{Intelligence and foreign language learning} %3.2.2 /

Early studies that dealt with the relationship between intelligence and L2 acquisition reported high correlations between the two (\citealt{Spolsky1995}: 327f). In contrast, later work emphasized two independent constructs (\citealt{GardnerLambert1965}, \citealt{Skehan1986}). Recently, a more differentiated view considering the interaction between various subcomponents of aptitude and intelligence has emerged. \citet{Sasaki1996} assessed L2 English proficiency and aptitude\footnote{JLAB (Japanese Language Aptitude Battery) based on the MLAT.} as well as two measures of general intelligence (verbal and reasoning) in Japanese students. The study indicated correlations between intelligence and language analytic abilities, although phonetic coding ability and rote memory (as defined by Carroll) correlated only weakly with measures of general intelligence. 

In two studies with different samples (100 adult Chinese-Spanish bilinguals and 186 adults with different L1s), \citet{Granena2012,Granena2013} found intelligence to be associated with explicit learning. The author administered a comprehensive test battery comprising what she refers to as explicit aptitude (LLAMA B, E, F, \citealt{MearaEtAl2001}), implicit aptitude (LLAMA D and a probabilistic serial reaction time task), and intelligence (according to the author, with a test corresponding roughly to an assessment of fluid intelligence).\footnote{Spanish version of the General Ability Measure for Adults (GAMA).} Statistical analysis confirmed the presence of two distinct aptitude dimensions associated with explicit and implicit L2 learning mechanisms. General intelligence correlated strongly with the former, explicit factor.

\citet{WescheEtAl1982} concluded that aptitude (measured with the MLAT) and intelligence (Primary Mental Abilities Test PMA assessing reasoning ability, word fluency, verbal comprehension, facility with numbers, spatial visualization, and rote memory)\footnote{Assessed with the PMA Primary Mental Abilities Test.} are relatively distinct factors, but they are not independent of one another. These findings were interpreted in a hierarchical model subsuming specific abilities important to instructed language learning under a more encompassing general ability or under general intelligence as postulated in Spearman’s \textit{g} factor. 

\citet{Li2016} explored the construct validity of language aptitude in a meta-ana\-ly\-sis including 66 studies with 109 unique samples and 13,035 foreign language learners. The author found a strong correlation ($r=0.64$) between aptitude and intelligence. This may be due to similarities between measures of aptitude and intelligence. For instance, both usually include tests of L1 vocabulary and memory. The reported correlation is not strong enough to speak of an identical construct \citep{Li2016}. Nevertheless, the author argues for further examining this overlap in order to clarify construct validity. Indeed, if language aptitude is not distinguishable from abilities required in other areas of academic learning, its existence as a construct becomes redundant. 

\subsection{Working memory (WM)} %3.3 /

Working memory (WM) is associated with the ability to temporarily store and manipulate information and thus underpins our capacity for complex cognitive behavior \citep{Baddeley2003}. A widely held model for explaining language acquisition and processing is the multi-component model of WM by \citeauthor{BaddeleyHitch1974} (\citeyear{BaddeleyHitch1974}, \citealt{Baddeley2000}). It consists of the central executive that acts as an attentional control system for the flow of information. The central executive is supported by three slave systems: 1) The phonological loop (also referred to as phonological short-term memory), which processes verbal and acoustic information, 2) the visuo-spatial sketchpad, which deals with visual information and 3) the episodic buffer, which integrates and temporarily stores information from the different modalities. Several studies have corroborated the presence of Baddeley’s WM structure in children (for an overview see \citealt{BoyleEtAl2013}). 

The phonological loop contains two further subparts, 1) a short-term phonological store where memory traces of auditory information are held for a few seconds and 2) an articulatory rehearsal component that keeps information activated to prevent time-based decay. Developmental studies suggest that the phonological store is established by the age of 3 with the capacity for subvocal rehearsal emerging around the age of 7 and increasing into adolescence (\citealt{HasselhornGrube2003}). The phonological loop has been described as central to L1 vocabulary acquisition and the development of spoken language in general \citep{BaddeleyEtAl1998}. It has also been linked to L2 development in children and adults, more specifically in learning new sound patterns \citep{SpecialeEtAl2004}, L2 grammar (\citealt{FrenchOBrien2008}) and L2 oral performance (\citealt{OBrienEtAl2006}). 

The visuo-spatial sketchpad is formed by the age of 4 and no further significant developmental changes seem to occur in this subsystem between 5 and 10 years (\citealt{HasselhornGrube2003}). It has been associated in particular with learning spatial routes and faces and may be implicated in the acquisition of arithmetic skills (\citealt{GathercolePickering2000}: 179). 

The episodic buffer was later added to Baddeley’s WM model to account for language performance in individuals with impaired phonological memory \citep{Baddeley2000}. Despite deficiencies in the phonological loop, these individuals were able to perform tasks that involve processing of complex auditory and visual information, such as recalling narratives or remembering sets of playing cards dealt in a game. The episodic buffer was proposed as a possible explanation: It is hypothesized as a system which is able to integrate information from all sub-systems and from long-term memory into a unitary episodic representation \citep[417]{Baddeley2000}. The episodic buffer is conceived of as an interface between the other WM components and long-term memory.

In relation to children, there is ample evidence associating WM performance to complex cognitive abilities which are likely to influence academic achievement (\citealt{GathercolePickering2000}: 175). More specifically, the central executive has been linked to vocabulary acquisition, reading and arithmetic skills. The phonological loop is particularly related to language acquisition, i.e. long-term learning of the sound patterns of new words (\citealt{GathercolePickering2000}). WM capacity in all the components mentioned is limited and has been shown to increase throughout childhood until the individual reaches young adulthood (\citealt{HasselhornGrube2003}). 

\subsubsection{Measuring WM capacity} %3.3.1 /

Measures of WM capacity distinguish between 1) the storage and processing functions and 2) the verbal (domain specific) and non-verbal (domain general) dimension (\citealt{LinckEtAl2014}, \citealt{Wen2015}). Simple span tasks assess storage only, i.e. they are indicative of short-term memory. Simple tasks include word and digit span tests that require participants to recall increasing numbers of unrelated words or numbers (\citealt{JuffsHarrington2011}). For instance, the forward digit span is a non-verbal simple test in which increasing numbers of random digits are presented until the individual reaches maximum recall capacity. Complex span tasks assess both storage and processing, i.e they pertain to executive WM (EWM). A frequently used measure of complex WM is the Reading Span task (RST, \citealt{DanemanCarpenter1980}) in which individuals need to simultaneously read aloud and comprehend sentences and recall the final word of each sentence. The Listening Span task is equivalent to the RST and assesses auditory storage and processing. A non-verbal option for a complex task is the Operation Span test (\citealt{TurnerEngle1989}) in which sentences are replaced with simple arithmetic equations. The Backward Digit Span Task (BDS) also reduces the language load (\citealt{KormosSafar2008}). In the BDS, participants are presented with an increasing number of random digits which they have to recall in reverse order (\citealt{JuffsHarrington2011}). 

Validity and reliability of widely used measures of WM capacity (including counting span, operation span, and reading span) have been documented by \citet{ConwayEtAl2005} in a methodological review. A meta-analysis conducted by \citet[861]{LinckEtAl2014} on WM and L2 comprehension and production\footnote{The meta-analysis by \citet{LinckEtAl2014} included data from 79 samples with 3,707 participants and 748 effect sizes.} revealed that complex span tasks are more predictive of L2 outcomes than simple span tasks, indicating that EWM may be more strongly implicated in L2 use than short-term memory. According to \citet[158]{JuffsHarrington2011} the RST and the Listening Span test are particularly successful predictors of L2 learning.

\subsubsection{WM and language aptitude} %3.3.2 /

Carroll’s rote memory component stems from an associative representation of memory serving as a static and passive storage space for information. Early aptitude tests usually measured rote memory with word lists that required individuals to map unknown words to a L1 translation. The Carrollian definition and assessment of memory differs considerably from new conceptions of WM presented in the previous section. A large body of research supports the association between WM and various aspects of L2 learning,\footnote{For a discussion see e.g. \citet{DeKeyserKoeth2011}, \citet{Wen2015}, or \citet{LinckEtAl2014} for a meta-analysis on WM and L2 comprehension and production.} leading aptitude researchers therefore argue for including WM as a distinct aptitude component (\citealt{MiyakeFriedman1998}, \citealt{DeKeyserKoeth2011}, \citealt{Robinson2002}, \citealt{Skehan2019}). Aspects of WM have been integrated into new aptitude conceptions (see \sectref{sec:01:2.1.1} and \sectref{sec:01:2.1.3}), such as the Macro SLA-aptitude model \citep{Skehan2019}, the Aptitude Complexes Hypothesis \citep{Robinson2002} or the Hi-LAB framework \citep{LinckEtAl2013}. 

There is evidence that WM components relate differentially to language aptitude. A meta-analysis of 66 studies by \citet{Li2016}, found EWM to be more strongly associated with aptitude as a whole (moderate correlation of $r=0.37$) than phonological WM (PWM) with a weak correlation of $r=0.16$. \citet[828]{Li2016} therefore suggests that EWM is a “more promising aptitude component than PWM”. This hypothesis is worth exploring further, especially for young learners who were not included in the meta-analysis by \citet{Li2016}. 

\subsection{Creativity} %3.4 /

Language learning and creativity can be associated in two ways. First, current communicative teaching methods, such as the task-based approach \citep{Willis1996}, require learners to contribute their own ideas in order to cope successfully with learning activities. Creative learners may be better equipped to deal with this kind of learning environment because they are likely to generate ideas easily which leaves them with more mental resources to engage with the target language. Second, creative thinking and language learning are hypothesized to share similar cognitive processing mechanisms, leading to the assumption that creative people are also good language learners, and/or that learning languages is good for creativity \citep{Kharkhurin2012}. For these reasons, creativity has been described as an ID-variable worth exploring in L2 learning and acquisition \citep{DoernyeiRyan2015}. 

The creativity construct involves a broad range of factors, including cognitive, motivational, personality-linked, societal and procedural aspects which have all been incorporated into different theories of creativity (for an overview see \citealt{Lubart1994}). In relation to language learning ability, the focus has been narrowed down to the cognitive mechanisms underlying creative thinking. The creative cognition view (\citealt{FinkeEtAl1992}, \citealt{Cropley2006}), which was also adopted in the LAPS project, differs from the conception of creativity as a means of artistic expression. Rather, creativity is seen as a particular way of thinking that is similar to problem-solving skills. It involves two basic thought processes \citep{Guilford1950}: Divergent thinking, i.e the ability to generate many ideas and convergent thinking, i.e. the ability to pick out a suitable idea and elaborate on it.

The core mechanisms of creative thinking are the ability to successfully retrieve existing knowledge, to focus on important information and suppress the irrelevant, and to analyze and transform this information into novel ideas so that a problem or task can be solved. Individuals must be able to tolerate ambiguity when an answer is not immediately available (\citealt{Guilford1950}, \citealt{FinkeEtAl1992}). 

Similar processes are hypothesized to be involved in foreign language learning. The CANAL-F theory \citep{GrigorenkoEtAl2000} emphasizes the fact that successful language learners are able to deal well with novelty and tolerate ambiguity in the face of new and unfiltered linguistic input. These individuals can access existing knowledge easily and merge it with new information in order to fill linguistic gaps. People with these abilities are thought to be at the same time creative and good language learners. 

Studies exploring creativity and language learning are scarce and have either focused on the possibility that language learning enhances creative thinking, or that creative thinking boosts L2 proficiency. As discussed in Chapter 6, they were mainly conducted with adults or adolescents and have produced mixed findings. Our own work presented in Chapter 6 investigated the effects of creative thinking on L2 proficiency and L2 motivation. To our knowledge, the affective link between creativity and language learning has not previously been considered.

\subsection{Cognitive styles – Field independence} %3.5 /

Field independence was first mentioned in connection with language aptitude in the 1980s (\citealt{ChapelleGreen1992}). Originally, this concept was defined as a cognitive style, i.e. a preferred way of cognitively processing information \citep{WitkinEtAl2014}. Based on different tasks to assess the perception of verticality, \citet{Witkin1949} identified two conceptualizations of visual processing: Some participants relied on their surroundings as a whole (field dependent) while others perceived individual parts of an image and then reconstructed them (field independent). However, the extent to which a person is field dependent or independent is not a categorical condition but rather located along a continuum. The concept has subsequently been discussed critically by several authors (for an overview, see \citealt{EvansEtAl2013}) and investigated from different angles, namely in connection with WM \citep{MiyakeEtAl2001}, visual perception \citep{Zhang2004} or intelligence (\citealt{RichardsonTurner2000}).

As far as foreign language learning is concerned, different qualities have been associated with field independence: Field dependent learners are thought to benefit from a communicative approach, as they tend to learn from interactions and role models (for an empirical investigation see \citealt{JohnsonEtAl2000}). Field independents, on the other hand, may do well when formal aspects of language are focused on, as this caters to their affinity for analytical thinking (see e.g. results from \citealt{StansfieldHansen1983}). Overall, several studies have documented positive effects for field independence on L2 proficiency, even in communicative settings (see for instance \citealt{ChapelleRoberts1986}, \citealt{Carter1988}; or more recently \citealt{FarsiEtAl2014}, \citealt{YaghoubiEtAl2014}). These tendencies can be illustrated by \citet[59]{ChapelleGreen1992} who suggest that tests for field independence and measures of L2 proficiency usually show “at least a weak, statistically significant, positive correlation” and that field independent individuals “tend to perform better on many types of second-language tests.”

\subsection{Metalinguistic awareness} %3.6 /

Although metalinguistic awareness was not explicitly part of the initial definition by Carroll, \citet{Singleton2014} emphasizes it as being closely related to language aptitude, especially to grammatical sensitivity and inductive ability \citep{AldersonEtAl1997}, which can be subsumed under language-ana\-ly\-tic ability \citep{Skehan1998}. While there are various definitions of metalinguistic awareness, briefly stated, it can be described as the ability to “focus on linguistic form and to switch focus between form and meaning” \citep[277]{Jessner2008}. Similarly, language-ana\-ly\-tic ability involves the capacity to reflect on language form as separate from meaning, for example by reasoning analytically about language patterns to arrive at generalizations, as in the PLAB subtest Language Analysis or the MLAT subtest Words in Sentences. \citet[163]{Ranta2002} therefore argues that language-analytic ability and metalinguistic awareness are essentially “two sides of the same coin”. Or as stated by \citet[1111]{RoehrBrackinTellier2019}, language-analytic ability “is at the core of the constructs of language learning aptitude and metalinguistic awareness”. In their work with Anglophone children aged 8--9 ($n=111$), the authors examined language-analytic ability in relation to metalinguistic awareness, suggesting that both significantly predict children’s L2 proficiency, with language analysis being a stronger predictor. The aptitude component of phonetic coding has also been associated with metalinguistic awareness, namely with phonological awareness (\citealt{RoehrBrackinTellier2019}).

According to \citet{RoehrBrackinTellier2019}, the hypothesized link between language aptitude and metalinguistic awareness is substantiated by the observation that different aptitude subcomponents take on different roles in L2 learning as the individual matures. For instance, as discussed in \sectref{sec:01:2.2.3}, some findings indicate that younger children draw more strongly on memory while older children rely more on language-analytic ability. This is interpreted by \citet{RoehrBrackinTellier2019} as an indication of developing metalinguistic abilities and literacy skills. However, as has been observed for several issues discussed in this chapter, there is currently not enough empirical evidence to make sound claims, neither about the \textit{evolving memory} versus \textit{language analysis} orientation of children, nor about the relationship between various aptitude components and metalinguistic awareness. The line of work adopted by \citet{RoehrBrackinTellier2019} is therefore worth pursuing in order to clarify the relationship between the two constructs.

\section{Affective dispositions: Motivation and related constructs}\label{sec:01:4}

\begin{sloppypar}
Affective learner dispositions are among the most thoroughly researched ID variables in SLA and language learning \citep[536]{Ellis2004}. We outline motivation to learn foreign languages (henceforth L2 motivation), and other affective constructs that have been related to L2 achievement, namely L2 anxiety and L2 self-concepts. A final, personality-linked construct we discuss is locus of control.
\end{sloppypar} 

\subsection{L2 motivation} %4.1 /

Research into L2 motivation was initiated in the multilingual context of Canada during the late 1950s. Early work investigated how L2 motivation differed from other types of motivation. This resulted in the development of \citegen{GardnerLambert1965} socio-educational model of second language acquisition (\citealt{Gardner1985}: 146, \citealt{Gardner2000}). It theorizes L2 motivation as being shaped by attitudes toward an L2 speech community and the learner’s willingness to integrate into this community. Students are guided by two types of orientations: 1) Integrative orientations, which refer to the desire to learn the language in order to get in contact with and identify with members of the L2 community, and 2) instrumental orientations linked to learning the L2 for some non-linguistic goal (e.g., academic success or social recognition). The former was identified as being more important, and thus, L2 learners with an integrative orientation were expected to be more successful.

Gardner and Lambert’s theory triggered extensive research in Canada and beyond (for reviews see e.g. \citealt{Gardner1985} or \citealt{Au1988}), the results of which have been mixed. In the 90s, various scholars challenged Gardner’s concept of integrateiveness, claiming that the desire to become part of a L2 community is not fundamental for L2 motivation, but applies to specific sociocultural contexts only, such as bilingual cities in Canada, where a specific L2 community is part of the social environment (see e.g., \citealt{NoelsClement1989}, \citealt{Doernyei1990}, \citealt{MoiseEtAl1990}, \citealt{ClémentEtAl1994}). 

These critical discussions marked the beginning of a new, more interdisciplinary era which considered theories from other disciplines, such as cognitive and educational psychology. Most notably, \citeauthor{DeciRyan1985ErsterEintrag}'s (\citeyear{DeciRyan1985ErsterEintrag,DeciRyan2002}) self-de\-ter\-mi\-na\-tion theory (SDT) was extended to SLA (see e.g., \citealt{Doernyei1994}, \citealt{Dickinson1995}, \citealt{SchmidtEtAl1996}, \citealt{NoelsEtAl1999,NoelsEtAl2000}). The central construct in SDT is intrinsic motivation, which subsumes the three basic psychological needs of self-determination, competence, and interpersonal relatedness. An action is intrinsically motivated if it occurs without external pressure and because it is regarded as inherently enjoyable. Extrinsic motivation, in turn, refers to actions that are taken for secondary reasons. These two kinds of motivation are thought to be located on a continuum, where extrinsic forms of motivation can be gradually transformed to intrinsic motivation through the process of internalization (see e.g. \citealt{DeciRyan1985ErsterEintrag}). 

The recognition of SDT as a psychological framework relevant for L2 motivation research was supported by several studies (\citealt{NoelsEtAl2000}, \citealt{Noels2001}). For instance, \citet[72--74]{NoelsEtAl2000} investigated 159 English-speaking learners of French. They were able to relate different forms of intrinsic and extrinsic motivation to their counterparts in Gardner’s model, integrative and instrumental orientations.

\subsection{L2 self-concepts} %4.2 /

Global changes that affected mobility and learning contexts led to the abandonment of Gardner’s concept of integrativeness. At the same time, social and dynamic aspects of L2 motivation gained in importance. A very influential model that emerged from this trend is \citegen{Doernyei2005ErsterEintrag} L2 Motivational Self System (L2MSS), in which traditional constructs are reinterpreted in light of self-theories postulated in the 1980s.\footnote{The L2MSS is particularly based on theories of \textit{possible selves} and \textit{self-discrepancy}. The interested reader is referred to \citet{MarkusNurius1986} and \citet{Higgins1987}, respectively.} In this model, mental future projections of oneself are assumed to trigger motivational forces that guide students in their L2 learning process. Gardner’s integrativeness was reconceptualized as the “ideal L2 self”, a mental construct which essentially describes the desire to acquire L2 proficiency for personal, social and job-related reasons \citep{Doernyei2009}.

\begin{sloppypar}
In parallel to these developments, dynamic system theories (\citealt{LarsenFreeman1997}, see also e.g., \citealt{EllisLarsenFreeman2006}, \citealt{DeBotEtAl2007}, \citealt{LarsenFreemanCameron2008}, \citealt{LarsenFreeman2017}) gained popularity in L2 motivation research. These theories seemed to provide a suitable framework for capturing the complexity, multidimensionality and dynamics of motivational processes in L2 learning (\citealt{Doernyei2010ZweiterEintrag}, \citealt{Waninge2015}). However, empirical research in this area faces serious difficulties in that conventional ways of testing hypotheses using (multiple) regression models with cross-sectional or longitudinal test data are not deemed appropriate for phenomena that are hypothesized to be highly complex and intra-individually dynamic in their time-course (cf. \citealt{Doernyei2014}; for methodological considerations see e.g., \citealt{VerspoorEtAl2011} or \citealt{DoernyeiEtAl2015}).
\end{sloppypar}

\subsection{L2 anxiety} %4.3 /

Foreign language learning anxiety is defined as any negative emotional state in relation to learning and using a foreign language \citep{MacIntyre1999}. It has been closely related to L2 motivation and L2 self-concepts (for a review see \citealt{Horwitz2001}). Various studies suggest that all of these affective factors mutually influence each other and eventually contribute to success or failure in L2 learning (see e.g., \citealt{NoelsEtAl2000}, \citealt{PekrunEtAl2002}, \citealt{Stoeckli2004}, \citealt{KormosCsizer2008}, \citealt{LiuHuang2011}, \citealt{Heinzmann2013}). At the same time, there is no conclusive evidence on the direction of causality, e.g., anxiety might affect learning or be affected by poor learning abilities; in the same way, self-concepts and motivation might be affected by learning ability and learning experiences \citep{SparksEtAl2011}.

\subsection{Locus of control} %4.4 /

Locus of control has been mentioned as a personality-linked variable relating to L2 learning in the literature (\citealt{Biedron2010,Peek2016}). It describes the extent to which individuals feel in charge of what is happening to them. Locus of control is similar to the concept of self-efficacy described by \citet{Bandura1986} and \citet{Rotter1990} within the social cognitive theory framework. Self-efficacy usually refers to one’s self-confidence in particular situations, for instance academic learning, and can therefore change according to context. Locus of control is related to an individual’s general tendency to attribute responsibility for outcomes either to internal or external sources. People with internal locus of control tend to believe that they are personally responsible for an outcome. Individuals with external locus of control ascribe their achievements or failures to an external influence. Learners with internal locus of control are expected to attain higher levels of L2 proficiency as they are more likely to take responsibility for their learning. 

\section{Environmental factors}

The influence of environmental factors, such as family and language background or the role of teaching paradigms, are not the main focus of the LAPS project. However, the inquiry into what shapes foreign language learning cannot be done without considering to some extent the interaction between IDs, educational systems and social environment. A sociological view on education provides a complementary view to the psychometric perspective which bears the risk of overemphasizing the individual while neglecting the structures in which the individuals do or do not unfold their potential. This interplay is explored in Chapter 5. 

\subsection{Family background} %5.1 /

Academic development in general and language learning in particular have been shown to be consistently associated with background variables such as parents’ educational level, home literacy practices, and the family’s socioeconomic features (see e.g., \citealt{AvineriEtAl2015} for discussion and more references). In particular the acquisition of (bi-)literacy was and is the object of many studies, and the general pattern in many Western countries shows that educational systems do not consistently even out inequalities in cultural and economic resources present in children’s families (see \citealt{Farkas2018} for a recent overview and \citealt{KigelEtAl2015} for a study in the German-speaking context). Most educational sociologists, inspired by \citegen{Bourdieu1979} influential theory of different types of capital, distinguish at least between two forms of family dispositions: Economic and cultural capital. In the analyses in chapters 4 and 5 we use background variables pertaining to both economic and cultural predispositions of the learners.

\subsubsection{Socio-economic family resources} %5.1.1 /

Sociolinguists and sociologists of education have accumulated a great wealth of evidence on the systematic associations of a family’s economic wealth and language learning and using. Most of the evidence concerns first or second language learning, studies of the social conditioning of foreign language learning being relatively scarce (but see \citealt{KliemeDESIKonsortium2008} for a study that includes social information). A positive association between the socioeconomic status of a child’s family and their school performance has been documented extensively  (see \citealt{EntwisleAlexander1992} and chapter 5 for more references).

\subsubsection{Cultural and educational family resources}  %5.1.2 /

Not only parents’ economic resources, but also a family’s cultural and educational predispositions have been shown to be associated with children’s school performance. In the bi- and multilingualism literature, it is generally assumed that parents' own educational background is predictive of the school performance of children in part because of higher or lower affinities of the parents' own experience with education. Therefore, parents’ attitudes toward education and their ``habitus'' is argued to have an important impact on pupils’ school performance \citep{Gogolin1994}. Moreover, better educated parents will often also be better prepared to help and support their children in school systems in which learning highly depends on homework tasks. 

Moreover, the language repertoires of the families are important resources for additional language learning (\citealt{SchepensEtAl2016}, \citealt{SchepensEtAl2020}), not only in the obvious cases where one of the family languages is the same as a target (foreign) language in school, but also with respect to the general language of instruction and the often cited potential of multilingual children to learn additional languages more easily (as is often assumed to be the case in the multilingualism literature, e.g. \citealt{MontanariQuay2019}; but see \citealt{BertheleUdry2019} for a more critical assessment of the evidence). 

Given the prominence of such questions in educational and multilingualism research, it seems important to take into account socioeconomic and cultural factors in a thorough investigation of individual differences in language learning.

\subsection{Teaching paradigms} %5.2 /

Implementing adequate teaching approaches for young learners has been described as a major challenge in policy making \citep{GartonEtAl2011}. Compatible with the perceived global need for communicative skills in English, curricula across the globe have generally come to adopt some form of Communicative Language Teaching (CLT, \citealt{Krashen1981}, \citealt{GartonEtAl2011}). Implementing these teaching methods can be constrained by local contexts, for instance in terms of resources, cultures of learning or teacher training (\citealt{Littlewood2006}, \citealt{Baker2008}). Often, teachers have been found to respond pragmatically with adapting CLT to suit their individual situation \citep{Carless2003}.

Developing communicative language skills is also at the core of the Swiss curriculum. To meet this aim, a task-based approach to language teaching and learning (TBLT) has been adopted (\citealt{Willis1996}, \citealt{Ellis2017}). TBLT mediates language through meaningful tasks that are accomplished by using the target language. Learning takes place when students must fill linguistic knowledge gaps encountered during task completion. Language use is therefore elicited by a real communicative need. TBLT can be implemented in different ways, i.e. independent of curricular prescriptions by building solely on learner questions as they arise during task completion, or by drawing on a syllabus that is complementary to the tasks \citep{Ellis2017}. 

Swiss teaching manuals are based on TBLT and structured around units on specific topics that are introduced via authentic input. The topics are elaborated on with meaning-focused activities and complemented with elements of explicit vocabulary and grammar teaching. At the end of a unit, learners complete a task that has often a creative focus, such as writing a poem, doing a role play, or painting a picture that is described to the class. In the LAPS project, we were interested in the interplay between L2 proficiency/L2 motivation and this creative element of TBLT (Chapter 6).

\section{Summary}

In this chapter, we have presented the ID variables and environmental factors that were considered in the LAPS project. They were included in a test battery and questionnaires that were administered to the participants at the beginning of the project. The test results and measures of L2/L3 proficiency provided the basis for addressing several research questions drawn from the literature on IDs and foreign language learning. 
\begin{sloppypar}
Most notably, we explored the underlying structure of the ID variables (Chapter 3) and assessed the predictive value of each variable for L2 proficiency, proposing different models that could be used by teachers to estimate learner potential (Chapter 4). Several issues were addressed in a longitudinal perspective, namely the development of L2/L3 motivation (Chapter 8), common variables underpinning the development of L1 German and L2 English proficiency (Chapter 9) and the dynamics of child language aptitude (Chapter 10). More specific questions concerning environmental factors are addressed in Chapters 5 to 7. We investigated the impact of socioeconomic variables on L2 achievement (Chapter 5), the task-based L2/L3 classroom in relation to creativity (Chapter 6) and the question of whether living close to a French native-speaking community enhanced children’s motivation to learn the target language (Chapter 7).
\end{sloppypar}

Some of these issues, especially language aptitude, have rarely been studied with children and with large cohorts. As a result, scholarly evidence remains inconclusive and further work is welcome to advance theoretical understanding and methodological innovation in the field in general and with regard to child language aptitude in particular. We hope that our contribution from the LAPS project will add to building a theoretical and pedagogical framework and that it will encourage similar research projects. 

{\sloppy\printbibliography[heading=subbibliography,notkeyword=this]}
\end{document}
