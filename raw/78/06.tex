\chapter[Personal pronouns]{Personal pronouns}
\label{Para_6}
This chapter describes the personal \isi{pronoun} system in Papuan Malay. Generally speaking, personal pronouns are defined as “inherent referential and \isi{definite} expressions”; their main function is to signal definiteness and person-number values, whereby they allow the unambiguous identification of their referents (\citealt[26]{Helmbrecht.2004}; see also \citealt{Abbot.2006}).



This main function also applies to the Papuan Malay personal pronouns, (henceforth ``pronouns''). In addition to expressing person and number values, they also mark their referents’ definiteness; the pronouns do not mark case, clusivity, gender, or politeness.



The pronouns have the following distributional properties:



\begin{enumerate}
\item 
Substitution for \isi{noun} phrases (pronominal uses) (§\ref{Para_6.1}).

\item 
Modification with demonstratives, locatives, numerals, quantifiers, and/or relative clauses (pronominal uses) (§\ref{Para_6.1}).

\item 
Co-occurrence with \isi{noun} phrases (adnominal uses): \textsc{n/np} \textsc{pro} (§\ref{Para_6.2})\textsc{.}

\end{enumerate}

The Papuan Malay \isi{pronoun} system, presented in \tabref{Table_6.1}, distinguishes singular and plural numbers and three persons by the person and number values in what \citet[3]{Daniel.2013b} calls “an unanalyzable person-number stem”. Hence, in terms of Daniel’s (\citeyear*[3]{Daniel.2013b}) typology of personal pronouns, Papuan Malay is a ``Type 4'' language.
The \isi{pronoun} system does not mark case, clusivity, gender, or politeness. Also, the third person pronouns are unrelated to the demonstratives \textitbf{ini} ‘\textsc{d.prox}’ and \textitbf{itu} ‘\textsc{d.dist}’.\footnote{For detailed discussions of these otherwise rather common features of pronouns see the following studies: case \citep{Bhat.2007}, clusivity \citep{Filimonova.2005}, gender \citep{Siewierska.2013}, politeness \citep{Helmbrecht.2013}, third person pronouns and demonstratives \citep{Bhat.2013}.}


\begin{table}
\caption{Pro\isi{noun} system with long and short forms and token frequencies}\label{Table_6.1}

\begin{tabular}{llrrllrrlr}
\lsptoprule
 & \multicolumn{3}{l}{ Long \isi{pronoun} forms}  & & \multicolumn{3}{l}{ Short \isi{pronoun} forms}  & &  Total\\
\midrule
&  & \# & \% &  &  & \# & \% & &   \#\\
\textsc{1sg} & \textitbf{saya} &  1,014 &  23\% & & \textitbf{sa} &  3,465 &  \textstyleChUnderl{77\%} &  &  4,479\\
\textsc{2sg} & \textitbf{{}-{}-{}-} &  {}-{}-{}- &  {}-{}-{}- & &  \textitbf{ko} &  1,338 &  100\% & &   1,338\\
\textsc{3sg} & \textitbf{dia} &  1,285 &  28\% & &  \textitbf{de} &  3,347 &  \textstyleChUnderl{72\%} & &   4,632\\
\textsc{1pl} & \textitbf{kitong} &  604 &  50\% & &  \textitbf{tong} &  594 &  50\% & &   1,198\\
& \textitbf{kita} &  391 &  \textstyleChUnderl{95\%} &  & \textitbf{ta} &  11 &  5\% &   & 402\\
& \textitbf{kitorang} &  112 &  \textstyleChUnderl{77\%} & &  \textitbf{torang} &  34 &  23\% &  &  146\\
\textsc{2pl} & \textitbf{kamu} &  337 &  \textstyleChUnderl{53\%} & &  \textitbf{kam} &  300 &  47\% &  &  637\\
\textsc{3pl} & \textitbf{dorang} &  464 &  23\% & &  \textitbf{dong} &  1,526 & \textstyleChUnderl{77\%} &  &  1,990\\
\lspbottomrule
\end{tabular}
\end{table}

Each \isi{pronoun} has at least one long and one short form, except for second person singular \textitbf{ko} ‘\textsc{2sg}’. The token frequencies and percentages given in \tabref{Table_6.1} indicate clear preferences for most of the \isi{pronoun} forms (the percentages for the most frequent forms are underlined). As for the first person singular and the third person singular pronouns, the short forms are used much more often than the respective long forms: for the first person singular there is a total of 3,465 short form tokens (77\%) versus a total of 1,014 long form tokens (23\%), and for the third person singular there is a total of 3,347 short form tokens (72\%) versus a total of 1,285 long form tokens (28\%). By contrast, for the first and second person plural pronouns, the long forms are used more frequently than the corresponding short forms, that is, for the first person plural there is a total of 1,107 long form tokens (63\%) versus a total of 639 short form tokens (37\%), and for the second person plural there is a total of 337 long form tokens (53\%) versus a total of 300 short form tokens (47\%).\footnote{\label{Footnote_6.181} First person plural: Alternatively, one could argue that long \textitbf{kitong} ‘\textsc{1pl}’ and \textitbf{kitorang} ‘\textsc{1pl}’ and short \textitbf{tong} ‘\textsc{1pl}’ and \textitbf{torang} ‘\textsc{1pl}’ are not distinct forms but allomorphs. As for short \textitbf{ta} ‘\textsc{1pl}’, one could argue that, given its low token numbers, this is not a phonologically distinct form but the result of a phonetic deletion of the first syllable  (U. Tadmor p.c. 2013).\\
Second person plural: In addition, the corpus contains one token of an alternative long form, namely \textitbf{kamorang} ‘\textsc{2pl}’. Its origins are yet to be established.} These distributional distinctions are not grammatically determined. Instead they represent speaker preferences which are discussed in more detail in the following two sections.\footnote{A topic for further investigation is whether these distributional distinctions are possibly phonologically determined.}


Papuan Malay pronouns often co-occur with nouns or \isi{noun} phrases, as shown in (\ref{Example_6.1}). This chapter argues that ``\textsc{pro} \textsc{np}'' constructions in which a \isi{pronoun} precedes a \isi{noun} or \isi{noun} phrase, as in \textitbf{ko }[\textitbf{sungay ko}] ‘you, [you river]’, constitute appositional constructions, with the pronouns having pronominal function. ``\textsc{np} \textsc{pro}'' constructions in which the \isi{pronoun} follows a \isi{noun} or \isi{noun} phrase, as in \textitbf{sungay ko} ‘you river’, by contrast, are analyzed as \isi{noun} phrases with adnominally used pronouns in post-head position. To demonstrate this distinction, appositional \textsc{pro} \textsc{np} constructions and adnominal \textsc{np} \textsc{pro} constructions are discussed in some detail in §\ref{Para_6.1.6} and §\ref{Para_6.2}, respectively.

\ea
\label{Example_6.1}
\gll {\ldots} {tida} {perna} {dia} {liat,} {\bluebold{ko}} {\bluebold{sungay}} {\bluebold{ko}} {bisa} {terbuka}\\ %
  { } \textsc{neg}  once  \textsc{3sg}  see  \textsc{2sg}  river  \textsc{2sg}  be.able  be.opened\\
\gll {begini}\\
 {like.this}\\
\glt
[Seeing the ocean for the first time:] ‘[never before has he seen, what, a river that is so very big like this ocean,] never before has he seen \bluebold{you}, \bluebold{you river} can be wide like this?’ \textstyleExampleSource{[080922-010a-CvNF.0212-0213]}\footnote{Addressing a non-speech-act participant such as \textitbf{sungay} ‘river’ with second person \textitbf{ko} ‘\textsc{2sg}’ serves as a rhetorical figure of speech (for details see §\ref{Para_6.2.1.1.3}).}
\z

\largerpage

The following sections discuss the pronouns in more detail. Their pronominal uses are examined in §\ref{Para_6.1}, and their adnominal uses in §\ref{Para_6.2}. The main points of this chapter are summarized in §\ref{Para_6.3}.


\section{Pronominal uses}
\label{Para_6.1}
This section explores three major aspects with respect to the pronominal uses of the pronouns: (1) the distribution of the long and short \isi{pronoun} forms within the clause (§\ref{Para_6.1.1}), (2) their \isi{modification} (§\ref{Para_6.1.2}), and (3) their uses in different constructions, namely adnominal possessive constructions (§\ref{Para_6.1.3}), inclusory \isi{conjunction} constructions (§\ref{Para_6.1.4}), \isi{summary conjunction} constructions (§\ref{Para_6.1.5}), and appositional constructions (§\ref{Para_6.1.6}).


\subsection{Distribution of personal pronouns within the clause}
\label{Para_6.1.1}
Regarding the distribution of the long and short \isi{pronoun} forms within the clause, two topics are examined in more detail: (1) the syntactic slots that the pronouns take (§\ref{Para_6.1.1.1}), and (2) their positions within the clause (§\ref{Para_6.1.1.2}).


\subsubsection[Personal pronouns in different syntactic slots]{Personal pronouns in different syntactic slots}
\label{Para_6.1.1.1}
Both the long and the short \isi{pronoun} forms occur in all syntactic positions within the clause, as illustrated in \tabref{Table_6.2} to \tabref{Table_6.5}.\footnote{The free translations in \tabref{Table_6.2} to \tabref{Table_6.5} are taken from the glossed recorded texts. Therefore, the tenses may vary; likewise, the translations for \textitbf{dia}/\textitbf{de} ‘\textsc{3sg}’ vary.} In the corpus, all long \isi{pronoun} forms can take the subject, direct object, and oblique object slots. Only one form is unattested: in \isi{double-object} constructions \textitbf{kita} ‘\textsc{1pl}’ is unattested in a direct object slot. As for the short \isi{pronoun} forms, all of them are attested for the subject slot. For the direct object slot, however, speakers use the long  rather than the short forms much more often. This distinction in distribution is even more pronounced for the oblique object slot. As a result, not all short \isi{pronoun} forms are attested in these positions. These preferences interrelate with the distributional patterns of the pronouns within the clause, as discussed in detail in §\ref{Para_6.1.1.2}.

\largerpage

\tabref{Table_6.2} shows the uses of the pronouns in the subject slot.


\begin{table}[b]

\caption[Pronouns in the subject slot]{Pronouns in the subject slot\footnote{Documentation: Long \isi{pronoun} forms – 081006-025-CvEx.0006, 080917-003b-CvEx.0017, 080916-001-CvNP.0004, 081006-022-CvEx.0116, 080917-008-NP.0113, 080919-004-NP.0033, 081115-001a-Cv.0160, 081011-023-Cv.0296; short \isi{pronoun} forms – 080916-001-CvNP.0001, 080916-001-CvNP.0004, 081029-005-Cv.0007, 080917-008-NP.0113, 080919-004-NP.0036, 081011-022-Cv.0242, 081015-005-NP.0039.}}\label{Table_6.2}
\begin{tabularx}{\textwidth}{lll}
\lsptoprule
 \multicolumn{1}{c}{Example} & \multicolumn{1}{c}{Literal translation} &  \multicolumn{1}{c}{Free translation}\\
\midrule
\multicolumn{3}{l}{Long \isi{pronoun} forms}\\
\midrule
\textitbfUndl{saya}\textitbf{ tidor} & \textsc{1sg} sleep & ‘\textstyleChUnderl{I} slept’\\
\textitbfUndl{ko}\textitbf{ ana mama} & \textsc{2sg} child mother & ‘\textstyleChUnderl{you}’re mama’s child’\\
\textitbfUndl{dia}\textitbf{ tertawa} & \textsc{3sg} laugh & ‘\textstyleChUnderl{he} laughed’\\
\textitbfUndl{kitorang}\textitbf{ bunu dorang} & \textsc{1pl} kill \textsc{3pl} & ‘\textstyleChUnderl{we} killed them’\\
\textitbfUndl{kitong}\textitbf{ kembali dari sana} & \textsc{1pl} return from \textsc{l.dist} & ‘\textstyleChUnderl{we} returned from there’\\
\textitbfUndl{kita}\textitbf{ jalang} & \textsc{1pl} walk & ‘\textstyleChUnderl{we} walked’\\
\textitbfUndl{kamu}\textitbf{ bisa blajar} & \textsc{2pl} be.able study & ‘\textstyleChUnderl{you} can study’\\
\textitbfUndl{dorang}\textitbf{ mara} & \textsc{3pl} feel.angry(.about) & ‘\textstyleChUnderl{they} felt angry’\\
\midrule
\multicolumn{3}{l}{Short \isi{pronoun} forms}\\
\midrule
\textitbfUndl{sa}\textitbf{ bilang} & \textsc{1sg} say & ‘\textstyleChUnderl{I} said’\\
\textitbfUndl{de}\textitbf{ tertawa} & \textsc{3sg} laugh & ‘\textstyleChUnderl{he} laughed’\\
\textitbfUndl{torang}\textitbf{ berdoa} & \textsc{1pl} pray & ‘\textstyleChUnderl{we} prayed’\\
\textitbfUndl{tong}\textitbf{ jalang kaki} & \textsc{1pl} walk foot & ‘\textstyleChUnderl{we} walked on foot’\\
\textitbfUndl{ta}\textitbf{ potong babi} & \textsc{1pl} cut pig & ‘\textstyleChUnderl{we} cut up the pig’\\
\textitbfUndl{kam}\textitbf{ cari bapa} & \textsc{2pl} search father & ‘\textstyleChUnderl{you}’ll look for father’\\
\textitbfUndl{dong}\textitbf{ bilang} & \textsc{3pl} say & ‘\textstyleChUnderl{they} said’\\
\lspbottomrule
\end{tabularx}

\end{table}

\tabref{Table_6.3} and \tabref{Table_6.3a} show the uses of the pronouns in the direct object slot in monotransitive constructions. In this position only short \textitbf{ta} ‘\textsc{1pl}’ is unattested, due to the overall low token frequencies for \textitbf{kita}/\textitbf{ta} ‘\textsc{1pl}’ (see \tabref{Table_6.1}; see also Footnote \ref{Footnote_6.181}, p. \pageref{Footnote_6.181}; for details on monotransitive clauses see §\ref{Para_11.1.2}).


\begin{table}[p]

\caption[Pronouns in the direct object slot in monotransitive constructions]{Pronouns in the direct object slot in monotransitive constructions\footnote{Documentation: Long \isi{pronoun} forms – 080922-010a-CvNF.0281, 080917-007-CvHt.0005, 081025-006-Cv.0150, 081110-008-CvNP.0106, 080922-001a-CvPh.0143, 081115-001a-Cv.0169, 081010-001-Cv.0161, 081006-009-Cv.0010; short \isi{pronoun} forms – 081011-023-Cv.0167, 081014-016-Cv.0001, 081115-001a-Cv.0283, 080925-003-Cv.0221, 081025-009a-Cv.0026, 081006-009-Cv.0017.}}\label{Table_6.3}

\begin{tabularx}{\textwidth}{p{6 cm}p{6 cm}}
\lsptoprule
 \multicolumn{1}{c}{Example} &  \multicolumn{1}{c}{Free translation}\\
\midrule
\multicolumn{2}{l}{Long \isi{pronoun} forms}\\
\midrule
\textitbf{de tanya }\textitbfUndl{saya} & ‘he asked \textstyleChUnderl{me}’\\
\textsc{3pl} ask \textsc{1sg} & \\
\\[-1em]
\textitbf{nanti guru{\Tilde}guru cari }\textitbfUndl{ko} & ‘very soon the teachers will look for\\
 very.soon \textsc{rdp}{\Tilde}teacher search \textsc{2sg} & \textstyleChUnderl{you}’\\
\\[-1em]
\textitbf{sa tanya }\textitbfUndl{dia}\textitbf{ begini} & ‘I asked \textstyleChUnderl{him} like this’\\
\textsc{1sg} ask \textsc{3sg} like.this & \\
\\[-1em]
\textitbf{bapa de pukul }\textitbfUndl{kitorang}\textitbf{ di muka} & ‘father hit \textstyleChUnderl{us} in the face’\\
father \textsc{3sg} hit \textsc{1pl} at front & \\
\\[-1em]
\textitbf{dong tipu }\textitbfUndl{kitong} & ‘they cheated \textstyleChUnderl{us}’\\
\textsc{3pl} cheat \textsc{1pl} & \\
\\[-1em]
\textitbf{dong suru }\textitbfUndl{kita}\textitbf{ begitu} & ‘they order \textstyleChUnderl{us} like that’\\
\textsc{3pl} order \textsc{1pl} like.that & \\
\\[-1em]
\textitbf{sa masi tunggu }\textitbfUndl{kamu} & ‘I still wait for \textstyleChUnderl{you}’\\
\textsc{1sg} still wait \textsc{2pl} & \\
\\[-1em]
\textitbf{sa memang titip }\textitbfUndl{dorang}\textitbf{ sama tanta Defretes} & ‘I indeed left \textstyleChUnderl{them} with aunt Defretes’\\
\textsc{1sg} indeed deposit \textsc{3pl} to aunt Defretes & \\
\lspbottomrule
\end{tabularx}

\end{table}

\begin{table}[p]

\caption[Pronouns in the direct object slot in monotransitive constructions]{Pronouns in the direct object slot in monotransitive constructions continued\footnote{Documentation: Long \isi{pronoun} forms – 080922-010a-CvNF.0281, 080917-007-CvHt.0005, 081025-006-Cv.0150, 081110-008-CvNP.0106, 080922-001a-CvPh.0143, 081115-001a-Cv.0169, 081010-001-Cv.0161, 081006-009-Cv.0010; short \isi{pronoun} forms – 081011-023-Cv.0167, 081014-016-Cv.0001, 081115-001a-Cv.0283, 080925-003-Cv.0221, 081025-009a-Cv.0026, 081006-009-Cv.0017.}}\label{Table_6.3a}

\begin{tabularx}{\textwidth}{p{6 cm}p{6 cm}}
\lsptoprule
 \multicolumn{1}{c}{Example} &  \multicolumn{1}{c}{Free translation}\\
\midrule
\multicolumn{2}{l}{Short \isi{pronoun} forms}\\
\midrule
\textitbf{de pukul }\textitbfUndl{sa} & ‘he hit \textstyleChUnderl{me}’\\
\textsc{3sg} hit \textsc{1sg} & \\
\\[-1em]
\textitbf{sa tanya }\textitbfUndl{de}\textitbf{ begini} & ‘I asked \textstyleChUnderl{her} like this’\\
\textsc{1sg} ask \textsc{3sg} like.this & \\
\\[-1em]
\textitbf{bapa bawa }\textitbfUndl{torang} ke \ili{Biak} & ‘father brought \textstyleChUnderl{us} to \ili{Biak}’\\
father bring \textsc{1pl} to \ili{Biak} & \\
\\[-1em]
\textitbf{dong antar }\textitbfUndl{tong}\textitbf{ sampe muara Tor} & ‘they brought \textstyleChUnderl{us} as far as the mouth of\\
\textsc{3pl} bring \textsc{1pl} reach river.mouth \ili{Tor} & the \ili{Tor} river’\\
\\[-1em]
\textitbf{sa tunggu }\textitbfUndl{kam} & ‘I’ll await \textstyleChUnderl{you}’\\
\textsc{1sg} wait \textsc{2pl} & \\
\\[-1em]
\textitbf{sa titip }\textitbfUndl{dong}\textitbf{ sama Defretes} & ‘I left \textstyleChUnderl{them} with Defretes’\\
\textsc{1sg} deposit \textsc{3pl} to Defretes & \\
\lspbottomrule
\end{tabularx}

\end{table}

\tabref{Table_6.4} illustrates the uses of the pronouns in a direct object slot in \isi{double-object} constructions. All long forms but one are attested; the exception is \textitbf{kita} ‘\textsc{1pl}’. As for the short forms, only three are attested, namely \textitbf{sa} ‘\textsc{1sg}’, \textitbf{tong} ‘\textsc{1pl}’, and \textitbf{dong} ‘\textsc{3pl}’. (For details on \isi{double-object} constructions see §\ref{Para_11.1.3.1}.)


\begin{table}[p]

\caption[Pronouns in a direct object slot in \isi{double-object} constructions]{Pronouns in a direct object slot in \isi{double-object} constructions\footnote{Documentation: Long \isi{pronoun} forms – 081006-024-CvEx.0030, 080925-003-Cv.0209, 081108-003-JR.0002, 081025-008-Cv.0145, 080919-004-NP.0061, 081011-020-Cv.0045, 081010-001-Cv.0195; short \isi{pronoun} forms – 080922-001a-CvPh.1010, 080922-002-Cv.0127, 080922-001a-CvPh.0339, 081006-023-CvEx.0074.}}\label{Table_6.4}

\begin{tabular}{ll}
\lsptoprule
 \multicolumn{1}{c}{Example} &  \multicolumn{1}{c}{Free translation}\\
\midrule

\multicolumn{2}{l}{Long \isi{pronoun} forms}\\
\midrule
\textitbf{kasi }\textitbfUndl{saya}\textitbf{ ana satu!} & ‘give \textstyleChUnderl{me} a certain child!’\\
give \textsc{1sg} child one & \\
\\[-1em]
\textitbf{mama bisa kasi ijing }\textitbfUndl{ko} & ‘I (‘mother’) can give \textstyleChUnderl{you} permission’\\
mother be.able give permission \textsc{2sg} & \\
\\[-1em]
\textitbf{skarang dong kasi }\textitbfUndl{dia}\textitbf{ senter} & ‘now they give \textstyleChUnderl{him} a flashlight’\\
now \textsc{3pl} give \textsc{3sg} flashlight & \\
\\[-1em]
\textitbf{mace kasi nasihat }\textitbfUndl{kitorang} & ‘the woman gave \textstyleChUnderl{us} advice’\\
woman give advice\textsc{ 3pl} & \\
\\[-1em]
\textitbf{dia kasi }\textitbfUndl{kitong}\textitbf{ daging} & ‘he gave \textstyleChUnderl{us} meat’\\
\textsc{3sg} give \textsc{1pl} meat & \\
\\[-1em]
\textitbf{minta{\Tilde}minta }\textitbfUndl{kamu} \textitbf{uang?} & ‘(who) keeps asking \textstyleChUnderl{you} for money?’\\
\textsc{rdp{\Tilde}}request \textsc{2pl} money & \\
\\[-1em]
\textitbf{baru kasi }\textitbfUndl{dorang}\textitbf{ makangang} & ‘and then (you) give \textstyleChUnderl{them} food’\\
and.then give \textsc{3pl} food & \\

%\lspbottomrule
%\end{tabular}
%
%\end{table}
%\begin{table}
%
%\caption[Pronouns in a direct object slot in \isi{double-object} constructions]{Pronouns in a direct object slot in \isi{double-object} constructions continued\footnote{Documentation: Long \isi{pronoun} forms – 081006-024-CvEx.0030, 080925-003-Cv.0209, 081108-003-JR.0002, 081025-008-Cv.0145, 080919-004-NP.0061, 081011-020-Cv.0045, 081010-001-Cv.0195; short \isi{pronoun} forms – 080922-001a-CvPh.1010, 080922-002-Cv.0127, 080922-001a-CvPh.0339, 081006-023-CvEx.0074.}}\label{Table_6.4a}
%
%\begin{tabular}{ll}
%\lsptoprule
% \multicolumn{1}{c}{Example} &  \multicolumn{1}{c}{Free translation}\\
\midrule

\multicolumn{2}{l}{Short \isi{pronoun} forms}\\
\midrule
\textitbf{bli }\textitbfUndl{sa}\textitbf{ boneka!} & ‘buy \textstyleChUnderl{me} a doll!’\\
buy\textsc{ 1sg} doll & \\
\\[-1em]
\textitbf{dong kasi }\textitbfUndl{tong}\textitbf{ playangang} & ‘they’ll give \textstyleChUnderl{us} a service’\\
\textsc{3pl} give \textsc{1pl} service & \\
\\[-1em]
\textitbf{bawa }\textitbfUndl{dong}\textitbf{ pakeang} & ‘(the pastors) brought \textstyleChUnderl{them} clothes’\\
come bring \textsc{3pl} clothes & \\

\lspbottomrule
\end{tabular}

\end{table}

 \tabref{Table_6.5} and \tabref{Table_6.5a}  show the uses of the pronouns in the oblique object slot. In this position, only three short forms are unattested, namely \textitbf{sa} ‘\textsc{1sg}’, \textitbf{kam} ‘\textsc{2pl}’, \textitbf{dong} ‘\textsc{3pl}’.


\begin{table}

\caption[Pronouns in the oblique object slot - Long Pronouns Forms]{Pronouns in the oblique object slot - Long Pro\isi{noun} Forms\footnote{Documentation: Long \isi{pronoun} forms – 080917-008-NP.0004, 080922-010a-CvNF.0089, 080922-010a-CvNF.0061, 081006-024-CvEx.0021, 081110-006-Pr.0014, 081006-024-CvEx.0021, 080922-001a-CvPh.0010, 080918-001-CvNP.0050; short \isi{pronoun} forms – 080922-010a-CvNF.0209, 080922-001a-CvPh.0339, 080919-006-CvNP.0011.}}\label{Table_6.5}

\begin{tabularx}{\textwidth}{p{6.5 cm}p{5cm}}
\lsptoprule
 \multicolumn{1}{c}{Example} &  \multicolumn{1}{c}{Free translation}\\
\midrule
\textitbf{baru dia yang ceritra }\textitbfUndl{sama saya} & ‘and then (it was) him who told\\
and.then \textsc{3sg} \textsc{rel} tell to \textsc{1sg} & this story \textstyleChUnderl{to me}’\\
\\[-1em]
\textitbf{tida bisa sa kas taw }\textitbfUndl{untuk ko} & ‘it’s impossible that I inform \textstyleChUnderl{you} \\
\textsc{neg} be.able \textsc{1sg} \textsc{caus} know for \textsc{2sg} & (about this issue)’\\
\\[-1em]
\textitbf{{\ldots} yang Aris dia kasi }\textitbfUndl{sama dia itu} & ‘[Oten’s wife] (is the one) that \\
\hspace{3mm} \textsc{rel} Aris \textsc{3sg} give to \textsc{3sg} \textsc{d.dist} & Aris gave \textstyleChUnderl{to him (\textsc{emph})}’\\
\\[-1em]
\textitbf{de minta sama Ida, }\textitbfUndl{sama kitorang} & ‘he requested (a child) from Ida,\\
\textsc{3sg} request to Ida to \textsc{1pl} &  \textstyleChUnderl{from us}’\\
\\[-1em]
\textitbf{de datang kas taw }\textitbfUndl{sama kitong} & ‘he’ll come (and) inform \textstyleChUnderl{us}’\\
\textsc{3sg} come give know to \textsc{1pl} & \\
\\[-1em]
\textitbf{jadi Raymon minta }\textitbfUndl{sama kita} & ‘so Raymon requested (a child)\\
so Raymon request to \textsc{1pl} & \textstyleChUnderl{from us}’\\
\\[-1em]
\textitbf{{\ldots} kasi hadia itu }\textitbfUndl{untuk kamu itu} & ‘[immediately the government]\\
\hspace{3mm} give gift\textitbf{\textmd{\textup{ }}}\textsc{d.dist}\textitbf{\textmd{\textup{ for }}}\textsc{2pl d.dist} &  will give that gift \textstyleChUnderl{to you}’\\
\\[-1em]
\textitbf{baru de ceritra apa }\textitbfUndl{sama dorang}\textitbf{ ka} & ‘and then maybe she told some-\\
and.then \textsc{3sg} tell what to \textsc{3pl} or & thing \textstyleChUnderl{to them}’\\
\lspbottomrule
\end{tabularx}

\end{table}

\begin{table}
\caption{Pronouns in the oblique object slot - Short Pronouns Forms}\label{Table_6.5a}
\begin{tabularx}{\textwidth}{p{6.5 cm}p{5cm}}
\lsptoprule
 \multicolumn{1}{c}{Example} &  \multicolumn{1}{c}{Free translation}\\
\midrule
\textitbf{bapa-tua itu de ceritra }\textitbfUndl{sama sa}\textitbf{ begini} & ‘that uncle, he told \textstyleChUnderl{me} like this’\\
uncle \textsc{d.dist} \textsc{3sg} tell to \textsc{1sg}\textitbf{\textmd{\textup{ like.this}}} & \\
\\[-1em]
\textitbf{{\ldots} yang telpon }\textitbfUndl{sama kam}\textitbf{ dua} & ‘[very soon it’ll be uncle pastor]\\
\hspace{3mm} \textitbf{ }\textsc{rel}\textitbf{\textmd{\textup{ phone to }}}\textsc{2pl}\textitbf{\textmd{\textup{ two}}} & who’ll phone \textstyleChUnderl{you} two’\\
\\[-1em]
\textitbf{tete ini bilang }\textitbfUndl{sama dong} & ‘this grandfather spoke \textstyleChUnderl{to them}’\\
grandfather \textsc{d.prox} say to \textsc{3pl} & \\
\lspbottomrule
\end{tabularx}
\end{table}

\clearpage 
\subsubsection[Personal pronouns within the clause]{Personal pronouns within the clause}
\label{Para_6.1.1.2}
Concerning the syntactic slots that the pronouns take, the distributional distinctions between the long and short \isi{pronoun} forms interrelate with the distributional pattern of the pronouns within the clause.



The data in the corpus shows a clear preference for the ``heavy'' long \isi{pronoun} forms to occur in clause-final position, regardless of their grammatical functions. This preference does not apply to other positions. That is, in clause-initial or clause-internal position, the long and the short \isi{pronoun} forms occur, regardless of their grammatical functions and their positions vis-à-vis the predicate. This observed distributional pattern is a reflection of the cross-linguistic tendency for the clause-final position to be “the preferred site for ‘heavy’ constituents” which has to do “with processing considerations” (\citealt[179]{Butler.2003}; see also \citealt[88–114]{Hawkins.1983}).


\largerpage
So far 710 clauses with clause-final pronouns have been identified in the corpus. In 62 clauses, \textitbf{ko} ‘\textsc{2sg}’ takes the clause-final position. Given that for the second person singular \isi{pronoun} only one form exists, it is excluded from further analysis. This leaves 648 clauses with a clause-final \isi{pronoun}. In almost all clauses, it is a long \isi{pronoun} form that occurs in clause-final position (97\% – 630/648), as shown in \tabref{Table_6.6}. Only rarely a short \isi{pronoun} form occurs in this position (3\% – 18/648), and two of the short forms are not attested at all in clause-final position, namely \textitbf{tong} ‘\textsc{1pl}’ and \textitbf{ta} ‘\textsc{1pl}’.


\begin{table}
\caption{Pronouns in clause-final position}\label{Table_6.6}
\begin{tabular}{llrrllrrlr}
\lsptoprule
& \multicolumn{3}{l}{ Long \isi{pronoun} forms}  & & \multicolumn{3}{l}{ Short \isi{pronoun} forms} & &   Total\\
&  & \# & \% & &   & \# & \% & &   \#\\
\midrule

\textsc{1sg} & \textitbf{saya} &  210 &  97\% & &  \textitbf{sa} &  7 &  3\% &  &  217\\
\textsc{3sg} & \textitbf{dia} &  236 &  99\% & &  \textitbf{de} &  2 &  1\% &  &  238\\
\textsc{1pl} & \textitbf{kitorang} &  18 &  82\% &  & \textitbf{torang} &  4 &  18\% &  &  22\\
\textsc{1pl} & \textitbf{kitong} &  15 &  100\% &  & \textitbf{tong} &  {}-{}-{}- &  {}-{}-{}- &  &  15\\
\textsc{1pl} & \textitbf{kita} &  7 &  100\% &  & \textitbf{ta} &  {}-{}-{}- &  {}-{}-{}- & &   14\\
\textsc{2pl} & \textitbf{kamu} &  49 &  98\% & &  \textitbf{kam} &  1 &  2\% &  &  50\\
\textsc{3pl} & \textitbf{dorang} &  95 &  96\% & &  \textitbf{dong} &  4 &  4\% &  &  99\\
\midrule
& Total &  630 &  97\% &  &  &  18 &  3\% & & 648\\
\midrule
{\textsc{2sg}} & \textitbf{ko} &  &  &  &  &  &  &  &  62\\
\midrule
& Total &  &  &  &  &  &  &  &  710\\
\lspbottomrule
\end{tabular}
\end{table}

This tendency for the clause-final position to be the preferred site for the ``heavy'' long \isi{pronoun} forms affects the choice of the \isi{pronoun} form for the different object slots, as shown in (\ref{Example_6.2}) to (\ref{Example_6.12}).


In the examples in (\ref{Example_6.2}) to (\ref{Example_6.5}), the pronouns take the direct object slot in monotransitive clauses. When the direct object occurs in clause-internal position, both the long and the short \isi{pronoun} forms are used, as shown with long \textitbf{dia} ‘\textsc{3sg}’ in (\ref{Example_6.2}) and short \textitbf{de} ‘\textsc{3sg}’ in (\ref{Example_6.3}). When the direct object occurs in clause-final position, speakers typically take the long \isi{pronoun} form, such as \textitbf{saya} ‘\textsc{1sg}’ in (\ref{Example_6.4}). Only rarely do speakers employ a short \isi{pronoun} form in clause-final position, such as \textitbf{sa} ‘\textsc{1sg}’ in (\ref{Example_6.5}).



\begin{styleExampleTitle}
Pronouns in the direct object slot in monotransitive clauses
\end{styleExampleTitle}

\ea
\label{Example_6.2}
\gll {sa} {su} {pukul} {\bluebold{dia}} {di} {kamar}\\ %
 \textsc{1sg}  already  hit  \textsc{3sg}  at  room\\

\glt
‘I already hit \bluebold{her} in (her) room’ \textstyleExampleSource{[081115-001a-Cv.0271]}
\z

\ea
\label{Example_6.3}
\gll {sa} {tanya} {\bluebold{de}} {begini}\\ %
 \textsc{1sg}  ask  \textsc{3sg}  like.this\\

\glt
‘I asked \bluebold{her} like this’ \textstyleExampleSource{[081014-016-Cv.0001]}
\z

\ea
\label{Example_6.4}
\gll {nanti} {ko} {kejar} {\bluebold{saya}}\\ %
 very.soon  \textsc{2sg}  chase  \textsc{1sg}\\

\glt
‘in a moment you chase \bluebold{me}’ \textstyleExampleSource{[080917-004-CVHT.0001]}
\z

\ea
\label{Example_6.5}
\gll {dulu} {bole} {bapa} {gendong} {\bluebold{sa},} {skarang} {\ldots}\\ %
 first  may  father  hold  \textsc{1sg}  now  \\

\glt
[Talking to her father:] ‘in former times you (‘father’) were allowed to hold \bluebold{me}, now {\ldots}’ \textstyleExampleSource{[080922-001a-CvPh.0699]}
\z

In the examples in (\ref{Example_6.6}) to (\ref{Example_6.8}), the pronouns take a direct object slot in \isi{double-object} constructions. In this position the mentioned, distributional preferences are even more pronounced. In clause-internal position, both the long and the short \isi{pronoun} forms occur, such as long \textitbf{saya} ‘\textsc{1sg}’ in (\ref{Example_6.6}) and short \textitbf{dong} ‘\textsc{3pl}’ in (\ref{Example_6.7}). In clause-final position, by contrast, only the long \isi{pronoun} forms are attested, such as \textitbf{dorang} ‘\textsc{3pl}’ in (\ref{Example_6.8}). (Double-object constructions are discussed in detail in §\ref{Para_11.1.3.1}.)



\begin{styleExampleTitle}
Pronouns in a direct object slot in \isi{double-object} constructions
\end{styleExampleTitle}

\ea
\label{Example_6.6}
\gll {kasi} {\bluebold{saya}} {ana} {satu!}\\ %
 kasi  \textsc{1sg}  child  one\\

\glt
‘give \bluebold{me} a certain child!’ \textstyleExampleSource{[081006-024-CvEx.0030]}
\z

\ea
\label{Example_6.7}
\gll {kaka} {kirim} {\bluebold{dong}} {uang}\\ %
 oSb  send  \textsc{2pl}  money\\

\glt
‘the older sibling sent \bluebold{them} money’ \textstyleExampleSource{[080922-001a-CvPh.0860]}
\z

\ea
\label{Example_6.8}
\gll {sa} {mulay} {kasi} {nasihat} {\bluebold{dorang}}\\ %
 \textsc{1sg}  \textsc{start}  give  advice  \textsc{2pl}\\

\glt
‘I started giving \bluebold{them} advice’ \textstyleExampleSource{[081115-001a-Cv.0100]}
\z



As for pronouns in oblique object slots, again both the long and the short \isi{pronoun} forms are used, such as long \textitbf{dorang} ‘\textsc{3pl}’ in (\ref{Example_6.9}) or short \textitbf{sa} ‘\textsc{1sg}’ in (\ref{Example_6.10}). In clause-final position, however, typically the long \isi{pronoun} forms are used, such as \textitbf{dorang} ‘\textsc{3pl}’ in (\ref{Example_6.11}), while short \isi{pronoun} forms such as \textitbf{dong} ‘\textsc{3pl}’ in (\ref{Example_6.12}) are very rare.



\begin{styleExampleTitle}
Pronouns in the oblique object slot
\end{styleExampleTitle}

\ea
\label{Example_6.9}
\gll {sa} {bilang} {\bluebold{sama}} {\bluebold{dorang}} {yang} {di} {kampung}\\ %
 \textsc{1sg}  speak  to  \textsc{3pl}  \textsc{rel}  at  village\\

\glt
‘I told \bluebold{them} who are in the village’ \textstyleExampleSource{[080919-001-Cv.0157]}
\z

\ea
\label{Example_6.10}
\gll {de} {bilang} {\bluebold{sama}} {\bluebold{sa}} {begini,} {\ldots}\\ %
 \textsc{3sg}  say  to  \textsc{1sg}  like.this  \\

\glt
‘he said \bluebold{to me} like this, {\ldots}’ \textstyleExampleSource{[080917-008-NP.0163]}
\z

\ea
\label{Example_6.11}
\gll {itu} {yang} {sa} {kas} {taw} {\bluebold{sama}} {\bluebold{dorang}}\\ %
 \textsc{d.dist}  \textsc{rel}  \textsc{1sg}  already  know  to  \textsc{3pl}\\

\glt
‘that (is) what I let \bluebold{them} know’ \textstyleExampleSource{[081006-009-Cv.0010]}
\z

\ea
\label{Example_6.12}
\gll {tete} {ini} {bilang} {\bluebold{sama}} {\bluebold{dong}}\\ %
 grandfather  \textsc{d.prox}  say  to  \textsc{3pl}\\
\glt
‘this grandfather spoke \bluebold{to them}’ \textstyleExampleSource{[080919-006-CvNP.0011]}

\z


\subsection{Modification of personal pronouns}
\label{Para_6.1.2}
Pronouns are readily modified with a number of different constituents, namely demonstratives, locatives, numerals, quantifiers, prepositional phrases, and/or relative clauses, as illustrated with the examples in (\ref{Example_6.13}) to (\ref{Example_6.24}).



Proximal \isi{demonstrative} \textitbf{ini} ‘\textsc{d.prox}’ modifies long \textitbf{saya} ‘\textsc{1sg}’ in (\ref{Example_6.13}), while distal \textitbf{itu} ‘\textsc{d.dist}’ modifies \textitbf{ko} ‘\textsc{2sg}’ in example (\ref{Example_6.14}). In both examples, the demonstratives signal the speaker's psychological involvement with the events being talked about. In (\ref{Example_6.15}), distal \isi{locative} \textitbf{sana} ‘\textsc{l.dist}’ modifies short \textitbf{dong} ‘\textsc{3sg}’, designating the referent’s location relative to that of the speaker. In the corpus, pronouns are quite often modified with demonstratives, while \isi{modification} with locatives is rare. (For details on demonstratives and locatives and their different functions see \chapref{Para_6}.)



\begin{styleExampleTitle}
Modification of pronouns with demonstratives or locatives
\end{styleExampleTitle}

\ea
\label{Example_6.13}
\gll {jadi} {\bluebold{saya}} {\bluebold{ini}} {ana} {mas-kawing}\\ %
 so  \textsc{1sg}  \textsc{d.prox}  child  bride.price\\

\glt
‘so \bluebold{I (}\blueboldSmallCaps{emph}\bluebold{)} am a bride-price child’ \textstyleExampleSource{[081006-028-CvEx.0016]}
\z

\ea
\label{Example_6.14}
\gll {a,} {ko} {ke} {laut} {dulu,} {dong} {ada} {tunggu} {\bluebold{ko}} {\bluebold{itu}}\\ %
 ah!  \textsc{2sg}  to  sea  first  \textsc{3pl}  exist  wait  \textsc{2sg}  \textsc{d.dist}\\

\glt
‘ah, you (go down) to the sea first, they are waiting for \bluebold{you (}\blueboldSmallCaps{emph}\bluebold{)}!’ \textstyleExampleSource{[081015-003-Cv.0003]}
\z

\ea
\label{Example_6.15}
\gll {\bluebold{dong}} {\bluebold{sana}} {cari} {anging}\\ %
 \textsc{3pl}  \textsc{l.dist}  search  wind\\

\glt
‘\bluebold{they over there} are looking for a breeze’ \textstyleExampleSource{[081025-009b-Cv.0076]}
\z



Modification with numerals typically involves the \isi{numeral} \textitbf{dua} ‘two’, as with short \textitbf{tong} ‘\textsc{1pl}’ in (\ref{Example_6.16}), but constructions with \textitbf{tiga} ‘three’ are also found. In the corpus, \isi{modification} with quantifiers is limited to universal \textitbf{smua} ‘all’ and mid-range \textitbf{brapa} ‘several’, as shown with long \textitbf{kamu} ‘\textsc{2pl}’ in (\ref{Example_6.17}) and long \textitbf{dorang} ‘\textsc{3pl}’ in (\ref{Example_6.18}), respectively. Modification with other quantifiers is also possible, as shown with midrange \textitbf{banyak} ‘many’ in the elicited example in (\ref{Example_6.19}). The examples in (\ref{Example_6.16}) to (\ref{Example_6.19}) also demonstrate that the numerals and quantifiers always occur in post-head position. That is, they cannot occur in pre-head position as illustrated with the elicited ungrammatical constructions in (\ref{Example_6.20}) and (\ref{Example_6.21}). (In this respect pronouns differ from nouns in that \isi{noun} phrases with adnominally used numerals or quantifiers can have an \textsc{n-mod} or a \textsc{mod-n} structure, as discussed in §\ref{Para_8.3}.)



\begin{styleExampleTitle}
Modification of pronouns with numerals or quantifiers
\end{styleExampleTitle}

\ea
\label{Example_6.16}
\gll {\bluebold{tong}} {\bluebold{dua}} {mandi,} {pas} {Nofita} {de} {datang}\\ %
 \textsc{1pl}  two  bathe  precisely  Nofita  \textsc{3sg}  come\\

\glt
‘\bluebold{the two of us} were bathing, at that moment Nofita came’ \textstyleExampleSource{[081025-006-Cv.0326]}
\z

\ea
\label{Example_6.17}
\gll {saya} {liat} {\bluebold{kamu}} {\bluebold{smua}} {tapi} {kamu} {\ldots}\\ %
 \textsc{1sg}  see  \textsc{2pl}  all  but  \textsc{2pl}  \\

\glt
‘I see \bluebold{all of you} but you {\ldots}’ \textstyleExampleSource{[080921-006-CvNP.0006]}
\z

\ea
\label{Example_6.18}
\gll {sa} {maki} {\bluebold{dorang}} {\bluebold{brapa}} {\bluebold{itu}}\\ %
 \textsc{1sg}  abuse.verbally  \textsc{3pl}  several  \textsc{d.dist}\\

\glt
‘I verbally abused \bluebold{several of them there}’ \textstyleExampleSource{[080923-008-Cv.0012]}
\z

\ea
\label{Example_6.19}
\gll {sa} {maki} {\bluebold{dorang}} {\bluebold{banyak}} {\bluebold{itu}}\\ %
 \textsc{1sg}  abuse.verbally  \textsc{3pl}  many  \textsc{d.dist}\\

\glt
‘I verbally abused \bluebold{many of them there}’ \textstyleExampleSource{[Elicited BR111021.024]}
\z

\ea
\label{Example_6.20}
\gll {*} {\bluebold{dua}} {\bluebold{tong}} {mandi,} {pas} {Nofita} {de} {datang}\\ %
 { }  two  \textsc{1pl}  bathe  precisely  Nofita  \textsc{3sg}  come\\

\glt
Intended reading: ‘\bluebold{the two of us} were bathing, at that moment Nofita came’ \textstyleExampleSource{[Elicited ME151112.001]}
\z

\ea
\label{Example_6.21}
\gll {*} {saya} {liat} {\bluebold{smua}} {\bluebold{kamu}} {tapi} {kamu} {\ldots}\\ %
 { }   \textsc{1sg}  see  all  \textsc{2pl}  but  \textsc{2pl}  \\

\glt
Intended reading: ‘I see \bluebold{all of you} but you {\ldots}’ \textstyleExampleSource{[Elicited ME151112.002]}
\z


In the corpus, the numerals and quantifiers typically form constituents with the quantified pronouns. That is, floating numerals or quantifiers are unattested, with one exception though. Quantifier \textitbf{smua} ‘all’ can also float to a clause-final position, as shown in (\ref{Example_6.22}). (This observation that among the numerals and quantifiers only \isi{quantifier} \textitbf{smua} ‘all’ floats also applies to the \isi{modification} of nouns, as discussed in §\ref{Para_8.3}.)



\begin{styleExampleTitle}
Floating \isi{quantifier} \textitbf{smua} ‘all’
\end{styleExampleTitle}

\ea
\label{Example_6.22}
\gll {langsung} {mandi,} {\bluebold{kitong}} {mulay} {mandi} {\bluebold{smua}}\\ %
 immediately  bathe  \textsc{1pl}  start  bathe  all\\

\glt
‘[We arrived here, arrived by motorboat,] immediately (we) bathed, \bluebold{we} started \bluebold{all} bathing’ \textstyleExampleSource{[080917-008-NP.0131]}
\z



Further, pronouns can be modified with prepositional phrases as illustrated with \textitbf{dong} ‘\textsc{3pl}’ in (\ref{Example_6.23}), or with relative clauses as shown with short \textitbf{sa} ‘\textsc{1sg}’ in (\ref{Example_6.24}).



\begin{styleExampleTitle}
Modification of pronouns with prepositional phrases or relative clauses
\end{styleExampleTitle}

\ea
\label{Example_6.23}
\gll {tapi} {\bluebold{dong}} {\bluebold{di}} {\bluebold{sana}} {\bluebold{tu}} {tida} {taw} {pencuri}\\ %
 but  \textsc{3pl}  at  \textsc{l.dist}  \textsc{d.dist}  \textsc{neg}  know  thief/steal\\

\glt
‘but \bluebold{them over there (}\blueboldSmallCaps{emph}\bluebold{)} never steal’ (Lit. ‘don’t know to steal’) \textstyleExampleSource{[081011-022-Cv.0293]}
\z

\ea
\label{Example_6.24}
\gll {waktu} {de} {kawing} {mas-kawing} {itu} {\bluebold{sa}} {\bluebold{yang}} {\bluebold{ambil}}\\ %
 when  \textsc{3sg}  marry.unofficially  bride.price  that  \textsc{1sg}  \textsc{rel}  get\\
\glt
‘when she marries, that bride-price, (it’s) \bluebold{me who’ll get} (it)’ \textstyleExampleSource{[081006-025-CvEx.0024]}
\z


\subsection{Personal pronouns in adnominal possessive constructions}
\label{Para_6.1.3}
Pronouns also take the possessor slot in adnominal possessive constructions. Overall, the short forms are preferred over the long forms, as shown in \tabref{Table_6.7} (the percentages for the most frequent forms are underlined).



The corpus contains a total of 1,692 adnominal possessive constructions. In 160 constructions, \textitbf{ko} ‘\textsc{2sg}’ takes the possessor slot; again, it is excluded from further analysis given that it has only one form. This leaves 1,532 adnominal possessive constructions. In 1,097 constructions the possessor slot is filled with a short \isi{pronoun} (72\%) as compared to only 435 constructions (28\%) in which a long \isi{pronoun} takes the possessor slot. The exception is first person plural \textitbf{kitong}/\textitbf{tong} ‘\textsc{1pl}’: speakers employ long \textitbf{kitong} ‘\textsc{1pl}’ almost as often as short \textitbf{tong} ‘\textsc{1pl}’.


\begin{table}
\caption{Pronominally used pronouns in adnominal possessive constructions}\label{Table_6.7}
\begin{tabular}{llrrllrrlr}
\lsptoprule
& \multicolumn{3}{l}{ Long \isi{pronoun} forms} & & \multicolumn{3}{l}{ Short \isi{pronoun} forms} &   & Total\\
&  & \# & \% &  &  & \# & \% & &  \#\\
\midrule

\textsc{1sg} & \textitbf{saya} &  83 &  16\% &  & \textitbf{sa} &  422 &  \textstyleChUnderl{84\%} &   & 505\\
\textsc{3sg} & \textitbf{dia} &  106 &  17\% &  & \textitbf{de} &  508 &  \textstyleChUnderl{83\%} & &   614\\
\textsc{1pl} & \textitbf{kitorang} &  9 &  \textstyleChUnderl{90\%} & &  \textitbf{torang} &  1 &  10\% &   & 10\\
\textsc{1pl} & \textitbf{kitong} &  40 &  49\% & &  \textitbf{tong} &  42 &  51\% & &   82\\
\textsc{1pl} & \textitbf{kita} &  17 &  \textstyleChUnderl{93\%} & &  \textitbf{ta} &  1 &  7\% & &   29\\
\textsc{2pl} & \textitbf{kamu} &  12 &  27\% &  & \textitbf{kam} &  32 &  \textstyleChUnderl{73\%} & &   44\\
\textsc{3pl} & \textitbf{dorang} &  8 &  8\% &  & \textitbf{dong} &  91 &  \textstyleChUnderl{92\%} &  &  99\\
\midrule
& Total &  435 &  28\% &  & &   1,097 &  \textstyleChUnderl{72\%} & &   1,532\\
\midrule
{\textsc{2sg}} & \textitbf{ko} &  &   &  & &  &  &  &  160\\
\midrule
& Total &  &  &  & &  &   &  &  1,692\\
\lspbottomrule
\end{tabular}
\end{table}

In (\ref{Example_6.25}), one possessive construction is presented in context with long \textitbf{dia} ‘\textsc{3sg}’ taking the possessor slot in (\ref{Example_6.25}). (For a detailed discussion of adnominal possessive constructions see \chapref{Para_9}.)


\ea
\label{Example_6.25}
\gll {nanti} {\bluebold{dia}} {\bluebold{pu}} {\bluebold{maytua}} {tanya,} {ko} {dapat} {ikang} {di} {mana}\\ %
 very.soon  \textsc{3sg}  \textsc{poss}  wife  ask  \textsc{2sg}  get  fish  at  where\\
\glt
‘later \bluebold{his wife} will ask, ``where did you get the fish?''' \textstyleExampleSource{[080919-004-NP.0062]}
\z


\subsection{Personal pronouns in inclusory {conjunction} constructions}
\label{Para_6.1.4}
Papuan Malay also employs plural pronouns in inclusory \isi{conjunction} constructions, such that ``\textsc{pro-pl} \textitbf{(dua) dengang} \textsc{np}'' or ``\textsc{pro-pl} (two) with \textsc{np}''. The conjunct that designates the entire set is encoded by a plural \isi{pronoun}. This conjunct is inclusory in that it “identifies a set of participants that includes the one or those referred to by the lexical \isi{noun} phrase”, as \citet[1]{Lichtenberk.2000} points out.  Hence it is an “inclusory \isi{pronoun}” (\citeyear*[2]{Lichtenberk.2000}) or, as \citet[33]{Haspelmath.2007c} calls it, an “inclusory conjunct”. In Papuan Malay, both conjuncts are linked by means of overt coordination with the \isi{comitative} marker \textitbf{dengang} ‘with’, with its short form \textitbf{deng}. The inclusory conjunct precedes the included conjunct, as shown in (\ref{Example_6.26}) to (\ref{Example_6.29}).



Typically, the inclusory conjunct is encoded by a dual construction formed with a plural \isi{pronoun} and the adnominally used \isi{numeral} \textitbf{dua} ‘two’, such that ``\textsc{pro-pl} \textitbf{dua}''. In (\ref{Example_6.26}), for instance, the speaker talks about herself and her husband. That is, the entire set consists of two referents with the inclusory conjunct \textitbf{tong dua} ‘we two’ including the conjunct \textitbf{bapa} ‘father’ in its reference. Only rarely is the inclusory conjunct encoded by a bare plural \isi{pronoun}, as in (\ref{Example_6.27}). In this example, the entire set consists of the speaker, his wife, and their children, with the included conjunct \textitbf{ana{\Tilde}ana} ‘children’ being subsumed under the inclusory conjunct \textitbf{tong} ‘\textsc{1pl}’.



\begin{styleExampleTitle}
Plural and dual inclusory \isi{conjunction} constructions with the first person plural \isi{pronoun}
\end{styleExampleTitle}

\ea
\label{Example_6.26}
\gll {\ldots} {\bluebold{tong}} {\bluebold{dua}} {\bluebold{deng}} {\bluebold{bapa}} {\bluebold{tu}} {sayang} {dia}\\ %
 { }   \textsc{1pl}  two  with  father  \textsc{d.dist}  love  \textsc{3sg}\\

\glt
‘[but this child] \bluebold{I and (my) husband (}\blueboldSmallCaps{emph}\bluebold{)} love her’ \textstyleExampleSource{[081115-001a-Cv.0251]}
\z

\ea
\label{Example_6.27}
\gll {malam} {hari} {atur} {\bluebold{tong}} {\bluebold{deng}} {\bluebold{ana{\Tilde}ana}} {makang}\\ %
 night  day  arrange  \textsc{1pl}  with  \textsc{rdp}{\Tilde}child  eat\\

\glt
‘in the evening (my wife) arranges (the food), \bluebold{we and the children} eat’ \textstyleExampleSource{[080919-004-NP.0007]}
\z


All three plural pronouns can take the inclusory conjunct slot, such as first person plural \textitbf{tong} ‘\textsc{1pl}’ in (\ref{Example_6.26}) and (\ref{Example_6.27}), second plural \textitbf{kam} ‘\textsc{2pl}’ in (\ref{Example_6.28}) and third person plural \textitbf{dong} ‘\textsc{3pl}’ in (\ref{Example_6.29}). Most often the included conjunct is encoded by a \isi{proper noun} as in (\ref{Example_6.28}), or less frequently by a \isi{noun} phrase as in (\ref{Example_6.29}), or also in (\ref{Example_6.26}).



\begin{styleExampleTitle}
Inclusory \isi{conjunction} constructions formed with the second and third person plural pronouns
\end{styleExampleTitle}

\ea
\label{Example_6.28}
\gll {\bluebold{kam}} {\bluebold{dua}} {\bluebold{deng}} {\bluebold{Isabela}} {pergi} {cek} {kapal} {di} {plabuang}\\ %
 \textsc{2pl}  two  with  Isabela  go  check  ship  at  harbor\\

\glt
‘\bluebold{you(}\bluebold{\textsc{sg}}\bluebold{) and Isabela} go check the ship at the harbor’ \textstyleExampleSource{[080922-001a-CvPh.0035]}
\z

\ea
\label{Example_6.29}
\gll {{\bluebold{dong}}} {\bluebold{dua}} {\bluebold{dengang}} {\bluebold{Natanael}} {\bluebold{pu}} {\bluebold{maytua}} {langsung}  pake  {spit}\\ %
 {\textsc{3pl}}  two  with  Natanael  \textsc{poss}  wife  immediately  use  {speedboat}\\
\glt
‘\bluebold{he/she and Natanael’s wife} immediately took the speedboat’ \textstyleExampleSource{[081014-008-CvNP.0006]}
\z



In addition, the corpus contains two inclusory \isi{conjunction} constructions, presented in (\ref{Example_6.30}) and (\ref{Example_6.31}), in which the inclusory conjuncts are used for joining two \isi{noun} phrases. Such inclusory \isi{conjunction} constructions have also been described for other languages, especially in Polynesia, as \citet{Haspelmath.2007c} points out. More specifically, \citet[35]{Haspelmath.2007c} notes that in such a construction the “first conjunct precedes the inclusory \isi{pronoun}, which is then followed by the other included conjunct(s) in the usual way”.



\begin{styleExampleTitle}
Inclusory \isi{conjunction} constructions conjoining two \isi{noun} phrases
\end{styleExampleTitle}

\ea
\label{Example_6.30}
\gll {\bluebold{Dodo}} {\bluebold{kam}} {\bluebold{dua}} {\bluebold{deng}} {\bluebold{Waim}} {ceritrakang} {dulu}\\ %
 Dodo  \textsc{2pl}  two  with  Waim  tell  first\\

\glt
‘\bluebold{you (}\blueboldSmallCaps{sg}\bluebold{) Dodo and Waim} talk first’ \textstyleExampleSource{[081011-001-Cv.0001]}
\z

\ea
\label{Example_6.31}
\gll {\bluebold{Tinus}} {{\bluebold{dorang}}} {\bluebold{dua}} {{\bluebold{dengang}}} {{\bluebold{Martina}}} {{\bluebold{ini},}} {dong} {dua} {lari}\\ %
 Tinus  {\textsc{3pl}}  two  {with}  {Martina}  {\textsc{d.prox}}  \textsc{3pl}  two  run\\
\gll  {trus,}  {dorang}  dua  {lari}  {sampe}  di  {kali}\\
 {be.continuous}  {\textsc{3pl}}  two  {run}  {reach}  at  {river}\\

\glt
‘\bluebold{Tinus and Martina here}, the two of them drove continuously, the two of them drove all the way to the river’ \textstyleExampleSource{[081015-005-NP.0011]}
\z



The inclusory \isi{conjunction} constructions presented in (\ref{Example_6.26}) to (\ref{Example_6.29}) contrast with what \citet[33]{Haspelmath.2007c} calls “\isi{comitative} \isi{conjunction} constructions”, which denote additive relations. In Papuan Malay, such \isi{comitative} constructions are formed with \isi{comitative} \textitbf{dengang} ‘with’. The inclusory \isi{conjunction} constructions in (\ref{Example_6.26}) to (\ref{Example_6.29}) also contrast with ``\textsc{n} \textsc{pro-pl}'' \isi{noun} phrases with an \isi{associative} inclusory reading. Both contrasts are illustrated with the examples in (\ref{Example_6.32}) and (\ref{Example_6.33}).



As for the distinction of \isi{comitative} and inclusory \isi{conjunction} constructions, \citet[33]{Haspelmath.2007c} makes the following cross-linguistic observations. In a \isi{comitative} \isi{conjunction} construction, the \isi{conjunction} of “two set-denoting \textsc{np}s [{\ldots}] ‘\{A, B\} and \{C, D\}’ yields the set \{A, B, C, D\}” (\citeyear*[33]{Haspelmath.2007c}). In inclusory \isi{conjunction} constructions, by contrast, “some members of the second conjunct set are already included in the first conjunct set”; hence the result of the coordination is not the “union, but the \textitbf{unification}\textit{ }of the sets [such that] ‘\{A, B, C\} and \{B\}’ yields the set \{A, B, C\}” (\citeyear*[33]{Haspelmath.2007c}). This distinction also applies to Papuan Malay. While the constructions in (\ref{Example_6.26}) to (\ref{Example_6.29}) receive an inclusory reading, the \isi{comitative} ‘N1 \textitbf{dengang} ‘with’ \textsc{N2}’ \isi{conjunction} construction in (\ref{Example_6.32}) receives an additive reading. (Comitative \isi{conjunction} constructions with \textitbf{dengang} ‘with’ are discussed in see §\ref{Para_14.2.1.1}.)



\begin{styleExampleTitle}
Comitative ``\textsc{N1} \textitbf{dengang} ‘with’ \textsc{N2}'' \isi{conjunction} construction
\end{styleExampleTitle}

\ea
\label{Example_6.32}
\gll {{baru}} {siapa} {\bluebold{Sarles}} {\bluebold{dengang}} {\bluebold{dong}} {\bluebold{dua}} {turung}\\ %
 {and.then}  who  Sarles  with  \textsc{3pl}  two  descend\\
\gll bli  {ni}\\
 buy  {\textsc{d.prox}}\\

\glt
‘and then, who-is-it, \bluebold{Sarles and the two of them} came down and bought this’ \textstyleExampleSource{[081022-003-Cv.0012]}
\z



Papuan Malay inclusory \isi{conjunction} constructions, that is, ``\textsc{pro-pl} \textitbf{(dua) dengang} \textsc{np}’ constructions, are also distinct from ``\textsc{n} \textsc{pro-pl}'' \isi{noun} phrases with an \isi{associative} inclusory plural reading. The pragmatic differences between the constituent order found in (\ref{Example_6.26}) to (\ref{Example_6.29}) as compared to that found in (\ref{Example_6.33}) are similar to the differences found in \ili{Toqabaqit}, another \ili{Austronesian} language, as observed by \citet[27]{Lichtenberk.2000}: the contrast concerns “the relative degrees of discourse salience of the two sets of participants, the overtly and the covertly encoded ones”. This contrast also applies to Papuan Malay. In (\ref{Example_6.28}) to (\ref{Example_6.29}), the covertly encoded participants subsumed under the adnominal dual constructions are more salient and therefore mentioned first. The overtly encoded participants, by contrast, are less salient and therefore mentioned second. In the \textsc{n} \textsc{pro-pl} \isi{noun} phrase in (\ref{Example_6.33}), by contrast, the overtly encoded participant \textitbf{bapa} ‘father’ is more salient and therefore mentioned first. The covertly encoded participants subsumed under the adnominal dual construction \textitbf{dorang dua} ‘they two’ are less salient and of subordinate status. (For details on \textsc{n} \textsc{pro-pl} \isi{noun} phrases with an \isi{associative} inclusory plural reading, see §\ref{Para_6.2.2.2}.)



\begin{styleExampleTitle}
\textsc{n} \textsc{pro-pl} \isi{noun} phrase with an \isi{associative} reading
\end{styleExampleTitle}

\ea
\label{Example_6.33}
\gll {\bluebold{bapa}} {\bluebold{dorang}} {\bluebold{dua}} {pulang} {hari} {minggu} {cepat}\\ %
 father  \textsc{3pl}  two  go.home  day  Sunday  be.fast\\
\glt
‘\bluebold{father and he} returned home quickly on Sunday’ \textstyleExampleSource{[080925-003-Cv.0163]}
\z


\subsection{Personal pronouns in summary conjunctions}
\label{Para_6.1.5}
The plural pronouns also occur in “\isi{summary conjunction}” constructions, a term adopted from Haspelmath's (\citeyear*[36]{Haspelmath.2007c}) cross-linguistic study on coordination: following a set of conjoined \isi{noun} phrases, a final constituent “sums up the set of conjuncts and thereby indicates that they belong together and that the list is complete”. According to \citet[36]{Haspelmath.2007c}, however, this final constituent is a “\isi{numeral} or \isi{quantifier}”.



In Papuan Malay, by contrast, the final constituent that sums up the set of conjuncts is a plural \isi{pronoun}. This is illustrated in the examples in (\ref{Example_6.34}) to (\ref{Example_6.36}). The set can consist of just two conjuncts as in (\ref{Example_6.34}), or of three or more as in (\ref{Example_6.35}). Typically the conjuncts are conjoined without an overt coordinator, as in (\ref{Example_6.34}) and (\ref{Example_6.35}). When the set of conjuncts is limited to two, as in (\ref{Example_6.36}), the conjuncts may also be linked with an overt coordinator, usually \isi{comitative} \textitbf{dengang} ‘with’. (For details on the combining of \isi{noun} phrases, see §\ref{Para_14.2}.)



\begin{styleExampleTitle}
Resumptive plural pronouns in \isi{summary conjunction} constructions
\end{styleExampleTitle}

\ea
\label{Example_6.34}
\gll {\bluebold{mama}} {\bluebold{bapa}} {\bluebold{tong}} {mo} {sembayang}\\ %
 mother  father  \textsc{1pl}  want  worship\\

\glt
‘\bluebold{we mother and father} want to worship’ \textstyleExampleSource{[080917-003b-CvEx.0020]}
\z

\ea
\label{Example_6.35}
\gll {\ldots} {\bluebold{Hurki}} {e} {\bluebold{Herman}} {\bluebold{Nusa},} {em,} {\bluebold{Oktofina}} {\bluebold{kamu}} {duduk} {situ}\\ %
 { }  Hurki  uh  Herman  Nusa  uh  Oktofina  \textsc{2pl}  sit  \textsc{l.med}\\

\glt
‘[in the evening (I said),] ``\bluebold{you (}\blueboldSmallCaps{pl}\bluebold{) Hurki}, uh \bluebold{Herman, Nusa}, uh \bluebold{Oktofina} sit there''' \textstyleExampleSource{[081115-001a-Cv.0085]}
\z

\ea
\label{Example_6.36}
\gll {\bluebold{mama}} {\bluebold{deng}} {\bluebold{bapa}} {\bluebold{dong}} {su} {meninggal}\\ %
 mother  with  father  \textsc{3pl}  already  die\\

\glt
‘\bluebold{they mother and father} have already died’ \textstyleExampleSource{[080919-006-CvNP.0012]}
\z


When the number of conjuncts is limited to two, Papuan Malay speakers often employ a dual construction in which the adnominal \isi{pronoun} is modified with the \isi{numeral} \textitbf{dua} ‘two’ as in (\ref{Example_6.37}) and (\ref{Example_6.38}). In such a “dual \isi{conjunction}” construction, a term also adopted from \citet[36]{Haspelmath.2007c}, the conjuncts are most often conjoined with an overt coordinator, as in (\ref{Example_6.37}), although coordination without an overt coordinator is also possible, as in (\ref{Example_6.38}).



\begin{styleExampleTitle}
Resumptive plural pronouns in dual \isi{conjunction} constructions
\end{styleExampleTitle}

\ea
\label{Example_6.37}
\gll {\bluebold{sa}} {\bluebold{deng}} {\bluebold{Eferdina}} {\bluebold{kitong}} {\bluebold{dua}} {pi} {berdoa} {tugu} {itu}\\ %
 \textsc{1sg}  with  Eferdina  \textsc{1pl}  two  go  pray  monument  \textsc{d.dist}\\

\glt
‘\bluebold{I and Eferdina, the two of us} go (and) pray over that statue’ \textstyleExampleSource{[080917-008-NP.0003]}
\z

\ea
\label{Example_6.38}
\gll {Rahab} {de} {bilang,} {\bluebold{bapa}} {\bluebold{mama}} {\bluebold{kam}} {\bluebold{dua}} {liat} {dulu}\\ %
 Rahab  \textsc{3sg}  say  father  mother  \textsc{2pl}  two  see  first\\
\glt
‘Rahab said, ``\bluebold{father and mother, the two of you} have a look!''' \textstyleExampleSource{[081006-035-CvEx.0044]}
\z


\subsection{Personal pronouns in appositional constructions}
\label{Para_6.1.6}
Pronouns very commonly occur in \textsc{pro} \textsc{np} constructions in which a pronominally used \isi{pronoun} precedes a \isi{noun} or \isi{noun} phrase. These constructions are analyzed as appositional constructions, with appositions being defined as “two or more \isi{noun} phrases having the same referent and standing in the same syntactical relation to the rest of the sentence” {\citep[5193]{Asher.1994}}. Such \textsc{pro} \textsc{np} constructions are distinct from the \textsc{np} \textsc{pro} constructions discussed in §\ref{Para_6.2}, in which an adnominally used \isi{pronoun} follows its head nominal. To validate this distinction, appositional \textsc{pro} \textsc{np} constructions are described in some detail in this section.



Appositions may be restrictive or nonrestrictive. In restrictive \isi{apposition}, the second appositive limits or clarifies the first unit. In nonrestrictive \isi{apposition}, by contrast, the second appositive is added as an optional additional piece of information \citep[182–188]{Morley.2000}. The same applies to Papuan Malay \textsc{pro} \textsc{np} appositions; that is, depending on their semantic function within the clause, the appositions may be restrictive or nonrestrictive. The referent is typically human with consultants agreeing that \textsc{pro} \textsc{np} expressions with nonhuman referents are unacceptable. The corpus contains only one exception in which the referent is an inanimate entity, presented in (\ref{Example_6.1}), repeated as (\ref{Example_6.41}). The construction in (\ref{Example_6.41}), however, involves “a \textstyleChItalic{personification}\textit{ }of the nonhuman object that is addressed”, as \citet[314]{Abrams.2009} describe it (for details on such figures of speech see §\ref{Para_6.2.1.1.3}).



Appositional \textsc{pro} \textsc{np} constructions are formed with all persons and number; those with singular pronouns are presented in (\ref{Example_6.39}) to (\ref{Example_6.43}) and those with plural pronouns in (\ref{Example_6.44}) to (\ref{Example_6.47}). Dual constructions are also possible, as shown in (\ref{Example_6.48}). Appositions can be bare nouns as in (\ref{Example_6.39}), \isi{noun} phrases with modifiers as in (\ref{Example_6.40}), or coordinate \isi{noun} phrases as in (\ref{Example_6.48}). In terms of intonation, the data in the corpus does not indicate a clear pattern: the \isi{apposition} can be set off from the preceding \isi{pronoun} by a \isi{comma intonation} (“{\textbar}”), as in (\ref{Example_6.39}), or can follow it with no intonation break as in (\ref{Example_6.40}).



The appositional constructions with singular pronouns (``\textsc{pro-sg} \textsc{np}'') in (\ref{Example_6.39}), (\ref{Example_6.40}) and (\ref{Example_6.43}) are nonrestrictive with the appositions \textitbf{mama} ‘mother’ in (\ref{Example_6.39}), \textitbf{prempuang cantik} ‘beautiful woman’ in (\ref{Example_6.40}), and \textitbf{ana} ‘child’ in (\ref{Example_6.43}) providing additional optional information not needed for the identification of their pronominal referents. The constructions in (\ref{Example_6.41}) and (\ref{Example_6.42}), by contrast, are restrictive with the appositions \textitbf{sungay ko} ‘you river’ and \textitbf{Agus ni} ‘this Agus’ giving information needed for the identification of the referents \textitbf{ko} ‘\textsc{2sg}’ and \textitbf{dia} ‘\textsc{3sg}’, respectively.



\begin{styleExampleTitle}
Appositions with singular pronouns: \textsc{pro-sg} \textsc{np}
\end{styleExampleTitle}

\ea
\label{Example_6.39}
\gll {\ldots} {yo,} {akirnya} {\bluebold{sa}} {} \textup{\textbar} {} {\bluebold{mama}} {berdoa} {berdoa}\\ %
 { }  yes  finally  \textsc{1sg}  {} {} {}  mother  pray  pray\\

\glt
‘[so in fifth grade she broke-off school,] yes, finally \bluebold{I}, \bluebold{(a/her) mother}, prayed (and) prayed’ \textstyleExampleSource{[081011-023-Cv.0178]}
\z

\ea
\label{Example_6.40}
\gll {kalo} {{ko}} {tida} {skola} {\bluebold{ko}} {\bluebold{prempuang}} {\bluebold{cantik}}\\ %
 if  {\textsc{2sg}}  \textsc{neg}  go.to.school  \textsc{2sg}  woman  be.beautiful\\
 \gll {nanti}  {\ldots}\\
 {very.soon}  {}\\

\glt
‘if you don’t go to school, later \bluebold{you}, \bluebold{a beautiful woman}, {\ldots}’ \textstyleExampleSource{[081110-008-CvNP.0043]}
\z

\ea
\label{Example_6.41}
\gll {\ldots} {tida} {perna} {dia} {liat,} {\bluebold{ko}} {\bluebold{sungay}} {\bluebold{ko}} {bisa} {terbuka}\\ %
  { } \textsc{neg}  once  \textsc{3sg}  see  \textsc{2sg}  river  \textsc{2sg}  be.able  be.opened\\
\gll {begini}\\
 {like.this}\\
\glt
[Seeing the ocean for the first time:] ‘[never before has he seen, what, a river that is so very big like this ocean,] never before has he seen \bluebold{you}, \bluebold{you river} can be wide like this?’ \textstyleExampleSource{[080922-010a-CvNF.0212-0213]}
\z

\ea
\label{Example_6.42}
\gll {dia} {tanya} {\bluebold{dia}} {} \textup{\textbar} {} {\bluebold{Agus}} {\bluebold{ni},} {ko} {ada} {kapur} {ka}?\\ %
 \textsc{3sg}  ask  \textsc{3sg} {} {} {} Agus  \textsc{d.prox}  \textsc{2sg}  exist  lime  or\\

\glt
‘he asked \bluebold{him}, \bluebold{Agus here}, ``do you have lime (powder)?''' \textstyleExampleSource{[080922-010a-CvNF.0034]}
\z

\ea
\label{Example_6.43}
\gll {\ldots} {tapi} {\bluebold{de}} {\bluebold{ana}} {juga} {cepat} {ikut} {terpengaru}\\ %
 { }  but  \textsc{3sg}  child  also  be.fast  follow  be.influenced\\
\glt
‘\ldots but \bluebold{he/she}, \bluebold{a kid}, also quickly follows (others) to be influenced’ \textstyleExampleSource{[080917-010-CvEx.0001]}
\z


Most often appositional constructions are formed with plural pronouns, such that ``\textsc{pro-pl} \textsc{np}''. Semantically, \textsc{pro-pl} \textsc{np} are distinct from \textsc{pro-sg} \textsc{np} constructions in that they not only indicate the definiteness of the apposited \isi{noun} phrases, but also their plurality, as shown in (\ref{Example_6.44}) to (\ref{Example_6.47}). For instance, \textitbf{pemuda} ‘youth’ in (\ref{Example_6.44}) or \textitbf{IPA satu} ‘Natural Science I (student)’ in (\ref{Example_6.45}) receive their plural reading from the preceding plural pronouns. If deemed necessary, speakers can specify the number of the apposited \isi{noun} phrases with an adnominal \isi{numeral} or \isi{quantifier}, as in \textitbf{tiga orang itu} ‘those three people’ in (\ref{Example_6.46}), or in \textitbf{brapa prempuang} ‘several women’ in (\ref{Example_6.47}).



\begin{styleExampleTitle}
Appositions with plural pronouns: \textsc{pro-pl} \textsc{np}
\end{styleExampleTitle}

\ea
\label{Example_6.44}
\gll {\bluebold{tong}} {\bluebold{pemuda}} {\bluebold{ini}} {mati} {smua}\\ %
 \textsc{1pl}  youth  \textsc{d.prox}  die  all\\

\glt
‘\bluebold{we}, \bluebold{the young people here}, have all lost enthusiasm’ \textstyleExampleSource{[081006-017-Cv.0014]}
\z

\ea
\label{Example_6.45}
\gll {tadi} {\bluebold{kam}} {\bluebold{IPA}} {\bluebold{satu}} {tra} {maing}\\ %
 earlier  \textsc{2pl}  natural.sciences  one  \textsc{neg}  play\\

\glt
‘earlier, \bluebold{you}, \bluebold{the Natural Science I (students)}, didn’t play’ \textstyleExampleSource{[081109-001-Cv.0162]}
\z

\ea
\label{Example_6.46}
\gll {\bluebold{dong}} {\bluebold{tiga}} {\bluebold{orang}} {\bluebold{itu}} {datang} {duduk}\\ %
 \textsc{3pl}  three  person  \textsc{d.dist}  come  sit\\

\glt
‘\bluebold{they}, \bluebold{those three people}, came (and) sat (down)’ \textstyleExampleSource{[081006-023-CvEx.0074]}
\z

\ea
\label{Example_6.47}
\gll {\ldots} {sa} {maki} {\bluebold{dorang}} {\bluebold{brapa}} {\bluebold{prempuang}} {di} {situ}\\ %
 { }  \textsc{1sg}  abuse.verbally  \textsc{3pl}  several  woman  at  \textsc{l.med}\\

\glt
‘[last month,] I verbally abused \bluebold{them}, \bluebold{several women}, there’ \textstyleExampleSource{[080923-008-Cv.0001]}
\z


When the number of referents encoded by the apposited \isi{noun} phrase is limited to two, Papuan Malay speakers also use dual constructions in which the \isi{pronoun} is modified with the \isi{numeral} \textitbf{dua} ‘two’, such that ``\textsc{pro-pl} \textitbf{dua} \textsc{np}'', as in (\ref{Example_6.48}). In the corpus, however, such constructions are rare and the dual constructions are always formed with the third person plural \isi{pronoun}.\footnote{The \textsc{pro} \textsc{np} constructions presented in this section were analyzed as appositions. One question for further research is whether these constructions could instead be analyzed as \isi{noun} phrases with pre-head pronouns. It is expected that such preposed pronouns would have an individuating function given that other pre-head determiners, namely numerals and quantifiers, also have an individuating function (see §\ref{Para_8.3}). One problem with such an analysis, however, would be \textsc{pro} \textsc{np} constructions with singular pronouns, as in (\ref{Example_6.39}) to (\ref{Example_6.43}), given that singular pronouns would hardly have an individuating function. (For a discussion of the determiner function of post-head pronouns see §\ref{Para_6.2}.)}



\begin{styleExampleTitle}
Appositions with dual constructions: \textsc{pro-pl} \textitbf{dua} ‘two’ \textsc{np}
\end{styleExampleTitle}

\ea
\label{Example_6.48}
\gll {\bluebold{dorang}} {\bluebold{dua}} {\bluebold{ade}} {\bluebold{kaka}} {\bluebold{itu}} {Agus} {dengang} {Fredi} {tra} {baik}\\ %
 \textsc{3pl}  two  ySb  oSb  \textsc{d.dist}  Agus  with  Fredi  \textsc{neg}  be.good\\
\glt
‘\bluebold{the two of them}, \bluebold{those siblings}, Agus and Fredi, are not good’ \textstyleExampleSource{[081014-003-Cv.0012]}
\z


\section{Adnominal uses}
\label{Para_6.2}
Papuan Malay pronouns are very often employed as determiners in post-head position, such that ``\textsc{np} \textsc{pro}''. Cross-linguistically, this function of personal pronouns is rather common, as \citet[141]{Lyons.1999} points out: they “combine with nouns to produce expressions whose reference is thereby determined in terms of the identity of the referent”; hence, they are “personal determiners”. In this function as “\isi{definite} expressions”, adopting \citegen[26]{Helmbrecht.2004} terminology, they indicate that the addressees are assumed to be able to identify the referent of an expression (see also \citealt[11]{Bhat.2007}; \citealt[26–32]{Lyons.1999}; \citealt[454–455]{Lyons.1977}).



It is argued here that, given the lack of inflectional person-number marking on nouns and verbs, and further given the lack of \isi{definite} articles, the adnominally used Papuan Malay pronouns also have this determiner function. That is, they allow the unambiguous identification of the referents as speakers or addressees, or as individuals or entities being talked about. Hence, Papuan Malay post-head pronouns are neither resumptive pronouns nor proclitic agreement markers on verbs.



This is illustrated with the example in (\ref{Example_6.49}). In the \textsc{np} \textsc{2sg} \isi{noun} phrase \textitbf{Wili ko} ‘you Wili’, the second person \isi{pronoun} marks the person spoken to as the intended addressee. In the \textsc{np} \textsc{3sg} \isi{noun} phrase \textitbf{tanta dia itu} ‘that aunt’ (literally ‘that she aunt’), the third person \isi{pronoun} signals that the interlocutors are assumed to know the referent. The brackets indicate the constituent structure within the \isi{noun} phrase. Details are discussed in §\ref{Para_6.2.1} and §\ref{Para_6.2.2}.



\begin{styleExampleTitle}
\textsc{np} \textsc{2sg} and \textsc{np} \textsc{3sg} \isi{noun} phrases
\end{styleExampleTitle}

\ea
\label{Example_6.49}
\gll {[\bluebold{Wili}} {\bluebold{ko}]} {jangang} {gara{\Tilde}gara} {[\bluebold{tanta}} {\bluebold{dia}} {\bluebold{itu}]}!\\ %
 Wili  \textsc{2sg}  \textsc{neg.imp}  \textsc{rdp}{\Tilde}irritate  aunt  \textsc{3sg}  \textsc{d.dist}\\

\glt
[Addressing a young boy:] ‘\bluebold{you Wili} don’t irritate \bluebold{that aunt}!’ \textstyleExampleSource{[081023-001-Cv.0038]}
\z



Adnominal pronouns are available for all person-number values, with the exception of the first person singular. This unexpected restriction may have to do with the function of the adnominally used pronouns which is to disambiguate the participants in a speech act, as discussed in detail throughout this section. It seems that Papuan Malay presumes addressees to have difficulties in identifying first person plural, second person and third person participants. To disambiguate the referents, the respective nouns can be modified with the appropriate pronouns. With first person singular referents, however, no such difficulties are expected. Hence, such referents do not need to be disambiguated, as demonstrated with the example in (\ref{Example_6.50}).\footnote{See also Bickel and Witzlack-Makarevich’s (\citeyear*[15]{Bickel.2008}) cross-linguistic study on “Referential scales and case alignment”, which shows that “first person singular is indeed often treated differently from other persons”.}



The utterances in (\ref{Example_6.50}) are part of a conversation between a mother and her son. As the family wants to go on a trip, the son wants to obtain a leave of absence from school. He is afraid, though, that his mother will not remind him in time to ask for this leave. In trying to soothe him, his mother tells him that she will remind him in time and that she will not depart without him. In doing so, the speaker alternatively refers to herself with the \isi{noun} \textitbf{mama} ‘mother’ and with first person singular \textitbf{sa} ‘\textsc{1sg}’. In this context, \textitbf{mama} ‘mother’ unambiguously refers to the speaker. Hence, there is no need to further disambiguate the referent by adding the first person singular \isi{pronoun}.



\begin{styleExampleTitle}
Speech acts with first person singular referents
\end{styleExampleTitle}

\ea
\label{Example_6.50}
\gll {hari} {{jumat}} {ko} {{mo}} {{jalang,}} {{baru}} {{\bluebold{mama}}} {kas} {{taw}} {\ldots}\\ %
 day  {Friday}  \textsc{2sg}  {want}  {walk}  {and.then}  {mother}  give  {know}  \\
\gll \bluebold{sa}  {tida}  {bisa}  {kas}  {tinggal}  {ko}  {\ldots}  hari  {jumat}  {ko}\\
 \textsc{1sg}  {\textsc{neg}}  {be.able}  {give}  {stay}  {\textsc{2sg}}  {}  day  {Friday}  {\textsc{2sg}}\\
\gll {mo}  {jalang,}  {baru}  {\bluebold{mama}}  {kasi}  {ingat}\\
 {want}  {walk}  {and.then}  {mother}  {give}  {remember}\\
\glt
‘on Friday (when) you want to go (and ask for the leave), \bluebold{I (‘mama’)} will remind (you) {\ldots} \bluebold{I} cannot leave you (behind) {\ldots} on Friday (when) you want to go, \bluebold{I (‘mama’)} will remind you’ \textstyleExampleSource{[080917-003b-CvEx.0011/0015/0020]}
\z



\tabref{Table_6.8} and \tabref{Table_6.8a} give an overview of the adnominal uses of pronouns as determiners.\footnotetext{The free translations in \tabref{Table_6.8} and \tabref{Table_6.8a} are taken from the glossed texts. Therefore, the tenses may vary; likewise, the translations for \textitbf{dia}/\textitbf{de} ‘\textsc{3sg}’ vary.}


\begin{table}

\caption[Adnominal pronouns as determiners - Long Pronouns Form]{Adnominal pronouns as determiners - Long Pronouns Form\footnote{Documentation: Long \isi{pronoun} forms – 080923-009-Cv.0051, 081023-001-Cv.0038, 080924-001-Pr.0008, 081110-005-Pr.0107, 080923-012-CNP.0011, 080919-003-NP.0002; short \isi{pronoun} forms – 081011-023-Cv.0167, 081115-001a-Cv.0001, 081006-009-Cv.0013, 081014-015-Cv.0006, 081006-024-CvEx.0043.}}\label{Table_6.8}
\begin{tabularx}{\textwidth}{p{7 cm}p{5 cm}}
\lsptoprule
 \multicolumn{1}{c}{Example} &  \multicolumn{1}{c}{Free translation}\\
 \midrule
\textitbf{de bilang, a }\textitbfUndl{om ko ini} tra liat {\ldots} & ‘he said, ``ah \textstyleChUnderl{you}\\
\textsc{3sg}\textitbf{\textmd{\textup{ say ah uncle }}}\textsc{2sg} \textsc{d.prox} \textsc{neg} see &  \textstyleChUnderl{uncle here} didn’t see   {\ldots}'''\\
\\[-1em]
\textitbf{Wili ko jangang gara-gara }\textitbfUndl{tanta dia itu}! & ‘you Wili don’t irritate\\
Wili \textsc{2sg imp-neg} irritate aunt\textitbf{\textmd{\textup{ }}}\textsc{3sg}\textitbf{\textmd{\textup{ }}}\textsc{d.dist} &  \textstyleChUnderl{that aunt}!’\\
\\[-1em]
\textitbf{jadi}\textitbfUndl{ nene kitorang ini}\textitbf{ masak} & ‘so \textstyleChUnderl{we} \textstyleChUnderl{grandmothers here}\\
\textitbf{\textmd{\textup{so grandmother }}}\textsc{1pl}\textitbf{\textmd{\textup{ }}}\textsc{d.prox}\textitbf{\textmd{\textup{ cook}}} &   cook’\\
\\[-1em]
\textitbf{jadi }\textitbfUndl{laki{\Tilde}laki kitong}\textitbf{ harus bayar {\ldots}} & ‘so \textstyleChUnderl{we men} have to pay {\ldots}’\\
\textitbf{\textmd{\textup{so }}}\textsc{rdp}\textitbf{\textmd{\textup{{\Tilde}husband }}}\textsc{1pl}\textitbf{\textmd{\textup{ have.to pay}}} & \\
\\[-1em]
\textitbfUndl{bangsat kamu tu}\textitbf{ tinggal lari} & ‘\textstyleChUnderl{you rascals there} keep\\
\textitbf{\textmd{\textup{rascal }}}\textsc{2pl}\textitbf{\textmd{\textup{ }}}\textsc{d.dist}\textitbf{\textmd{\textup{ stay run}}} &  running’ \\
\\[-1em]
{{\ldots} biking malam untuk }\textitbfUndl{anjing dorang} & ‘[ the sagu porridge that   my\\
\hspace{3mm} \textitbf{\textmd{\textup{make night for dog }}}\textsc{3pl} &  wife] had made at night for \textstyleChUnderl{the dogs}’ \\
\lspbottomrule
\end{tabularx}

\end{table}
\begin{table}
\caption{Adnominal pronouns as determiners - Short Pronouns Form}\label{Table_6.8a}
\begin{tabularx}{\textwidth}{p{6.5 cm}p{5.5 cm}}
\lsptoprule
 \multicolumn{1}{c}{Example} &  \multicolumn{1}{c}{Free translation}\\
 \midrule
\textitbf{{\ldots} sampe }\textitbfUndl{bapa de}\textitbf{ pukul sa deng pisow} & ‘{\ldots} until (my) \textstyleChUnderl{husband} hit me with\\
\hspace{3mm} \textitbf{\textmd{\textup{until father }}}\textsc{3sg}\textitbf{\textmd{\textup{ hit }}}\textsc{1sg}\textitbf{\textmd{\textup{ with knife}}} &  a knife\\
\\[-1em]
\textitbf{itu yang }\textitbfUndl{Lodia torang}\textitbf{ bilang {\ldots}} & ‘that’s why \textstyleChUnderl{Lodia and her}\\
\textsc{d.dist}\textitbf{\textmd{\textup{ }}}\textsc{rel}\textitbf{\textmd{\textup{ }}}Lodia\textitbf{\textmd{\textup{ }}}\textsc{1pl}\textitbf{\textmd{\textup{ say}}} &  \textstyleChUnderl{companions including me} said {\ldots}’\\
\\[-1em]
\textitbf{{\ldots} }\textitbfUndl{Pawlus tong}\textitbf{ bicara sama dia itu} & ‘\textstyleChUnderl{Pawlus and his companions}\\
\hspace{3mm} Pawlus\textsc{ 1pl}\textitbf{\textmd{\textup{ speak to }}}\textsc{3sg}\textitbf{\textmd{\textup{ }}}\textsc{d.dist} & \textstyleChUnderl{including me} spoke to him (\textsc{emph})’\\
\\[-1em]
\textitbf{kamu }\textitbfUndl{ana prempuang kam}\textitbf{ latiang} & ‘you, \textstyleChUnderl{you girls} practice’\\
\textsc{2pl}\textitbf{\textmd{\textup{ child woman }}}\textsc{2pl}\textitbf{\textmd{\textup{ practice}}} & \\
\\[-1em]
\textitbf{tong biasa tanya sama }\textitbfUndl{kaka dong} & ‘we usually ask \textstyleChUnderl{the older siblings}’\\
\textsc{1pl} be.usual ask to oSb \textsc{3pl} & \\
\lspbottomrule
\end{tabularx}
\end{table}

Some of the examples in Table \ref{Table_6.8} and Table \ref{Table_6.8a} do not readily translate into English, given that “personal determiners” in English are subject to constraints concerning their person-number values \citep[27]{Lyons.1999}. In English, only ``we'' and ``you (\textsc{pl})'' occur freely as determiners, while ‘you (\textsc{sg})’ occurs in exclamations only; the remaining pronouns do not have any determiner uses.\footnote{English examples are ``we teachers'', ``you students'', or ``you idiot'' \citep[451–442]{Lyons.1999}.} Other languages, however, are less constrained. In \ili{German}, for example, the first and second persons, both singular and plural, occur as determiners, while the third person does not (\citealt[142]{Lyons.1999}; see also \citealt[189]{Helmbrecht.2004}) for the determiner uses of personal pronouns). Along similar lines, in the Oslo dialect of \ili{Norwegian}, the female third person singular \isi{pronoun} functions as a determiner \citep{Johannessen.2006}. In addition, pronouns can occur as determiners with proper names in some \ili{Germanic} languages, such as \ili{German}, \ili{Icelandic}, and \ili{Norwegian}. In \ili{German} it is the first or second person singular pronouns \citep[264-269]{Roehr.2005}, in \ili{Icelandic} it is the third person singular and the first and second person plural pronouns \citep[218-224]{Sigursson.2006}, and in \ili{Northern Norwegian} it is the third person singular \isi{pronoun} \citep[581]{Matushansky.2008}. Still other languages are “completely unconstrained in this respect” \citep[142]{Lyons.1999}, as for instance \ili{Warlpiri} (\citealt{Hale.1973} in \citealt[142]{Lyons.1999}).


\citet[134]{Lyons.1999} suggests “that personal pronouns are the pronominal counterpart of \isi{definite} articles”. This is the case for \ili{Warlpiri} which has “no \isi{definite} article” but “a full paradigm of personal determiners” (\citeyear*[142, 144]{Lyons.1999}). And it is also the case for Papuan Malay which has no \isi{definite} article either but an almost complete paradigm of personal determiners, the exception being the first singular person. Other \ili{Austronesian} languages, by contrast, which do have a \isi{definite} article also employ this article as a determiner with proper names. Examples, provided in \citet{Campbell.2000}, are \ili{Balinese} \citep{Kersten.1948}, \ili{Chamorro} \citep{Topping.1960}, and \ili{Fijian} \citep{Milner.1959, Schutz.1971}, and, presented in \citet{Campbell.2000b}, \ili{Malagasy} \citep{Arakin.1963}, \ili{Maori} \citep{Krupa.1967}, \ili{Minangkabau} \citep{Moussay.1981}, \ili{Tagalog} (\citealt{Bowen.1965}; \citealt{Ramos.1971}), \ili{Tahitian} \citep{Arakin.1981}, and \ili{Tongan} \citep{Churchward.1953}.



As for \isi{noun} phrases with adnominal pronouns in other \ili{regional Malay varieties}, only limited information is available. Brief descriptions or examples are offered for \ili{Ambon Malay} \citep{vanMinde.1997}, \ili{Balai Berkuak Malay} \citep{Tadmor.2002}, \ili{Dobo Malay}  (R. Nivens p.c. 2013), \ili{Kupang Malay} \citep{Grimes.2008}, \ili{Manado Malay} \citep{Stoel.2005}, and \ili{Sri Lanka Malay} \citep{Slomanson.2013}. In each case, however, the descriptions are limited to the \isi{associative plural} interpretation of ``\textsc{np} \textsc{pro-pl}'' expressions (see §\ref{Para_6.2.2.3}). A determiner function of the pronouns is not mentioned in any of these descriptions.



In addition, some descriptions of \ili{regional Malay varieties} mention \textsc{np} \textsc{pro} constructions, most of which are analyzed as topic-comment constructions.


\begin{itemize}
\item 
\ili{Ambon Malay}: \citet[284]{vanMinde.1997} mentions constructions in which “a preposed \textsc{np} is copied by a coreferential \isi{pronoun} in the remainder of the clause”. In each case, the \isi{pronoun} is the short third person singular \textitbf{de} ‘\textsc{3sg}’. In addition, \citet[285]{vanMinde.1997} presents examples in which a \isi{pronoun} follows a \isi{noun} phrase with an adnominal \isi{demonstrative} at its right periphery. 

\item 
\ili{Banda Malay}: \citet[165]{Paauw.2009} gives examples of \textsc{np} \textsc{pro} constructions which he also analyzes as topic-comment constructions. The \isi{pronoun} is third person singular \textitbf{dia} ‘\textsc{3sg}’ and the preceding \isi{noun} phrase is set off with an adnominal \isi{demonstrative}.

\item 
\ili{Northern Moluccan Malay}: \citet[5]{Voorhoeve.1983} analyzes similar constructions as topic-comment constructions “in which the topic is cross-refer\-enced by a \isi{pronoun} subject in the comment”. Again, the \isi{pronoun} is third person singular \textitbf{dia} ‘\textsc{3sg}’ and the preceding \isi{noun} phrase is set off with an adnominal \isi{demonstrative}.

\item 
Papuan Malay: \citet[166–168]{Paauw.2009} presents \textsc{np} \textsc{pro} constructions in which the short third person forms \textitbf{de} ‘\textsc{3sg}’ and \textitbf{dong} ‘\textsc{3pl}’ occur between a subject and a \isi{verb}. \citet{Paauw.2009} analyzes these pronouns as “proclitics” that function as subject agreement markers on verbs.

\end{itemize}

In the following sections, the adnominal uses of the pronouns are examined in detail. That is, these sections discuss the function of the pronouns to signal definiteness and person-number values, whereby they allow the unambiguous identification of the referents as speakers, addressees, or third-person participants.



The adnominal uses of the singular pronouns are discussed in §\ref{Para_6.2.1} and those of the plural pronouns in §\ref{Para_6.2.2}. For the singular pronouns a major issue is the question whether \textsc{np} \textsc{pro} expressions are indeed \isi{noun} phrases with adnominal pronouns or whether these expressions should be analyzed as topic-comment constructions, as in other \ili{regional Malay varieties}. For the plural pronouns, two interpretations of \textsc{np} \textsc{pro} constructions are discussed, additive, and \isi{associative} inclusory plurality. In giving examples for \textsc{np} \textsc{pro} expressions, brackets are used to signal the constituent structure within the \isi{noun} phrase, where deemed necessary.


\subsection{Adnominal singular personal pronouns}
\label{Para_6.2.1}
In their determiner uses, the singular personal pronouns indicate the definiteness as well as the person and the number values, namely singularity, of their referents. ``\textsc{np} \textsc{pro-sg}'' expressions with \textitbf{ko} ‘\textsc{2sg}’ are presented in §\ref{Para_6.2.1.1}, and those with \textitbf{dia}/\textitbf{de} ‘\textsc{3sg}’ in §\ref{Para_6.2.1.2}. In all examples given in §\ref{Para_6.2.1.1} and §\ref{Para_6.2.1.2}, the \textsc{np} \textsc{pro-sg} expressions constitute intonation units, unless mentioned otherwise; that is, the pronouns are not set off from their head nominals by a \isi{comma intonation}. In addition, however, the corpus also contains \textsc{np} \textsc{pro-sg} expressions in which the nouns are set off from the following pronouns by intonation; these \isi{noun} phrases are briefly discussed in §\ref{Para_6.2.1.3}. Finally, §\ref{Para_6.2.1.4} presents the reasons for analyzing \textsc{np} \textsc{pro-sg} expressions as \isi{noun} phrases with adnominal pronouns rather than as topic-comment constructions.
\subsubsection[\textsc{np} \textsc{2sg} {noun} phrases]{\textsc{np} \textsc{2sg} {noun} phrases}
\label{Para_6.2.1.1}
\textsc{np} \textsc{2sg} \isi{noun} phrases have three different functions: (1) in \isi{direct speech} they mark the person spoken to as the intended addressee, (2) in direct quotations they signal that the referent is the addressee of the \isi{reported speech}, and (3) as \isi{rhetorical figures of speech} they give an unexpected emotional impulse to a speaker’s discourse. These functions are explored one by one, followed by a summary of the syntactic and lexical properties of \textsc{np} \textsc{2sg} \isi{noun} phrases.\\
\setcounter{secnumdepth}{4}

\subsubsubsection{``\textsc{np 2sg}''  \textit{{noun} phrases in direct speech}\label{Para_6.2.1.1.1}}

In \isi{direct speech}, speakers employ \textsc{np} \textsc{2sg} \isi{noun} phrases when they want to send an unambiguous signal that the person spoken to is indeed the intended addressee. In such \isi{noun} phrases, the second person \textitbf{ko} ‘\textsc{2sg}’ marks the referent encoded in the head nominal as the addressee of the utterance. The head nominal can be a common \isi{noun} or a \isi{proper noun}, as shown in (\ref{Example_6.51}) to (\ref{Example_6.54}).



\begin{styleExampleTitle}
\textsc{np} \textsc{2sg} \isi{noun} phrases in \isi{direct speech}
\end{styleExampleTitle}

\ea
\label{Example_6.51}
\gll {[\bluebold{mama-ade}} {\bluebold{ko}]} {masak} {daging} {sa} {biking} {papeda} {e?}\\ %
 aunt  \textsc{2sg}  cook  meat  \textsc{1sg}  make  sagu.porridge  eh\\

\glt
‘\bluebold{you aunt} cook the meat, I make the sagu porridge, eh?’ \textstyleExampleSource{[080921-001-CvNP.0073]}
\z

\ea
\label{Example_6.52}
\gll {[\bluebold{mace}} {\bluebold{ko}]} {rasa} {lucu} {jadi}\\ %
 woman  \textsc{2sg}  feel  be.funny  so\\

\glt
[Reaction to a narrative:] ‘because \bluebold{you Madam} would have felt funny’ \textstyleExampleSource{[081010-001-Cv.0206]}
\z

\ea
\label{Example_6.53}
\gll {[\bluebold{Wili}} {\bluebold{ko}]} {jangang} {gara{\Tilde}gara} {tanta} {dia} {itu}!\\ %
 Wili  \textsc{2sg}  \textsc{neg.imp}  \textsc{rdp}{\Tilde}irritate  aunt  \textsc{3sg}  \textsc{d.dist}\\

\glt
[Addressing a young boy:] ‘\bluebold{you Wili} don’t irritate that aunt!’ \textstyleExampleSource{[081023-001-Cv.0038]}
\z

\ea
\label{Example_6.54}
\gll {[\bluebold{Susana}} {\bluebold{ko}]} {pigi} {kaka} {cebo}\\ %
 Susana  \textsc{2sg}  go  oSb  wash.after.defecating\\

\glt
[Addressing her three-year old daughter:] ‘\bluebold{you Susana}, go, (your) older sister will wash (you)!’ \textstyleExampleSource{[081014-006-CvPr.0048]}
\z



When the head nominal is a common \isi{noun}, second person \textitbf{ko} ‘\textsc{2sg}’ indicates which particular individual is being referred to. Thereby the \isi{pronoun} allows the unambiguous identification of the addressee as the intended referent. Often speakers choose this strategy when they address an individual in a group of several interlocutors as in (\ref{Example_6.51}) and (\ref{Example_6.52}), or when they give an order to someone, as in (\ref{Example_6.53}) and (\ref{Example_6.54}).



When \textitbf{ko} ‘\textsc{2sg}’ co-occurs with a \isi{proper noun}, as in (\ref{Example_6.53}) or (\ref{Example_6.54}), one might argue that such \isi{noun} phrases are redundant with the \isi{pronoun} as adnominal determiner being superfluous, since proper nouns are inherently \isi{definite}. In Papuan Malay, however, \textsc{np} \textsc{2sg} expressions constitute direct speech-act strategies which allow speakers to single out participants and to mark them unambiguously as the intended referents of the proper nouns. Being addressed with such a \textsc{np} \textsc{2sg} \isi{noun} phrase leaves the addressees little room for interpretation. Thereby, \textsc{np} \textsc{2sg} nouns phrases are much more direct than the indirect strategies presented in (\ref{Example_6.55}) and (\ref{Example_6.56}),



Most often, speakers do not address their interlocutor with a direct \textsc{np} \textsc{2sg} expression. Instead, they tend to use more indirect, face-preserving strategies by addressing their interlocutor with a \isi{kinship term} or their proper name. This applies especially when issuing a request or an order, as shown in (\ref{Example_6.55}) and (\ref{Example_6.56}). In (\ref{Example_6.55}), a daughter asks her father for money by addressing him with the \isi{kinship term} \textitbf{bapa} ‘father’. In (\ref{Example_6.56}), a father requests his daughter to talk to him by addressing her with her proper name \textitbf{Nofela}.



\begin{styleExampleTitle}
Indirect forms of address with bare proper nouns or kinship terms
\end{styleExampleTitle}

\ea
\label{Example_6.55}
\gll {\bluebold{bapa}} {ingat} {tong} {itu} {uang!}\\ %
 father  remember  \textsc{1pl}  \textsc{d.dist}  money\\

\glt
‘\bluebold{you (‘father’)} remember our, what’s-its-name, money!’ \textstyleExampleSource{[080922-001a-CvPh.0857]}
\z

\ea
\label{Example_6.56}
\gll {\bluebold{Nofela}} {bicara} {suda!}\\ %
 Nofela  speak  already\\
\glt
‘\bluebold{you (‘Nofela’)} speak (to me)!’ \textstyleExampleSource{[080922-001a-CvPh.0805]}
\z


\subsubsubsection{``\textsc{np 2sg}'' \textit{{noun} phrases in reported speech}\label{Para_6.2.1.1.2}}

Speakers also employ \textsc{np} \textsc{2sg} \isi{noun} phrases when they report \isi{direct speech}. This reporting is usually done through quoting. Cross-linguistically, direct quotations serve “to dramatize and highlight important elements in a narrative”, while in\isi{direct speech} “seems less vivid and colorful”, as \citet[552]{Bublitz.2006} point out. The same applies to Papuan Malay, as speakers typically use quotes when reporting \isi{direct speech}, as demonstrated in (\ref{Example_6.57}) and (\ref{Example_6.58}).



When relating what had been said to a particular individual, speakers usually begin the quote with an \textsc{np} \textsc{2sg} \isi{noun} phrase, as (\ref{Example_6.57}) and (\ref{Example_6.58}). This has two functions. First, it indicates the referent as the addressee of the \isi{reported speech}. Second, \textsc{np} \textsc{2sg} \isi{noun} phrases mark the referent as familiar or given. Thereby they signal to the hearers that they should be able to identify the referent. Subsequently, speakers continue the direct quote by referring to, or “addressing”, the referent with bare \textitbf{ko} ‘\textsc{2sg}’, as in (\ref{Example_6.58}). Note that the first occurrence of \textitbf{Iskia} in (\ref{Example_6.58}) is not part of the quote but the direct object of \textitbf{bilang} ‘say’.


\ea
\label{Example_6.57}
\gll {de} {bilang,} {\bluebold{Natalia}} {\bluebold{ko}} {bisa} {liat} {orang} {di} {luar?}\\ %
 \textsc{3sg}  say  Natalia  \textsc{2sg}  be.able  see  person  at  outside\\

\glt
[About hospitality:] ‘[(my father said to me,) ``if you close the door, can you see the people outside?'',] he said, ``can \bluebold{you Natalia} see the people outside?''' \textstyleExampleSource{[081110-008-CvNP.0104]}
\z

\ea
\label{Example_6.58}
\gll {tong} {{dua}} {{bilang}} {Iskia,} {{\bluebold{Iskia}}} {\bluebold{ko}} {{temani,}} {\bluebold{ko}} {temani}\\ %
 \textsc{1pl}  {two}  {say}  Iskia  {Iskia}  \textsc{2sg}  {accompany}  \textsc{2sg}  accompany\\
 \gll {karna}  {su}  {larut}  {malam}  {sedikit}\\
 {because}  {already}  {be.protracted}  {night}  {few}\\
\glt
‘the two of us said to Iskia, ``\bluebold{you Iskia} come with (us), \bluebold{you} come with (us) because it’s already a bit late in the evening''' \textstyleExampleSource{[081025-006-Cv.0323/0325]}
\z



\subsubsubsection{``\textsc{np 2sg}'' \textit{{noun} phrases as {rhetorical figures of speech} (“apostrophes”)}\label{Para_6.2.1.1.3}}

\textsc{np} \textsc{2sg} \isi{noun} phrases also serve as \isi{rhetorical figures of speech}. Speakers suddenly interrupt the flow of their discourse and employ a \isi{noun} phrase modified with second person \textitbf{ko} ‘\textsc{2sg}’, whereby they unexpectedly address a different audience of absent persons or nonhuman entities.



This “turning away from an audience and addressing a second audience” as a rhetorical figure of speech has been termed “\isi{apostrophe}” \citep[75]{Bussmann.1996}. Generally speaking, speakers employ apostrophes to give “a sudden emotional impetus” \citep[313]{Abrams.2009} to their discourse and thereby to create an emotional reaction in their audience. Following \citet{Kacandes.1994}, this emotional reaction to \isi{apostrophe} can be explained “by its power of calling another into being”; that is, “[t]he audience witnesses an invigoration of a being who previously was not ‘present’”. Moreover, the “[l]inguistic properties of the second-person \isi{pronoun} invite the hypothesis that one also reacts strongly to \isi{apostrophe} because one can so easily become the ‘you’ and thus feel oneself called into the relationship it creates” (\citeyear*{Kacandes.1994}).



This explanation also applies to \textsc{np} \textsc{2sg} \isi{noun} phrase apostrophes in Papuan Malay as illustrated in (\ref{Example_6.59}) to (\ref{Example_6.61}). Structurally, these utterances resemble direct quotations. Contrasting with the \isi{direct speech} situations in (\ref{Example_6.51}) to (\ref{Example_6.54}), however, the addressed referents were not present when the utterances occurred. And in contrast to the \isi{reported speech} situation in (\ref{Example_6.57}) and (\ref{Example_6.58}), the speakers in (\ref{Example_6.59}) to (\ref{Example_6.61}) do not relate direct quotes. Instead, they “turn away” from their audience to “address a second audience” of human or nonhuman referents.

\largerpage

The example in (\ref{Example_6.59}) is part of a story about a fight between \textitbf{Martin} and \textitbf{Fitri}, with the speaker relating how \textitbf{Martin} attacked \textitbf{Fitri}. Notably, neither \textitbf{Martin} nor \textitbf{Fitri} were present when the speaker recounted the incident. When mentioned the first time, \textitbf{Martin} is marked as a third-person actor, as is typical of narratives with non-speech-act participants. Later in the discourse, however, \textitbf{Martin} is marked as the addressee. More specifically, the speaker first refers to \textitbf{Martin} with the \textsc{np} \textsc{3sg} \isi{noun} phrase \textitbf{Martin dia lewat} ‘Martin went past’ (literally ‘he Martin’) (see also §\ref{Para_6.2.1.2}). Next, the speaker refers to \textitbf{Martin} with the third person singular \isi{pronoun} in \textitbf{de lompat} ‘he jumped’. Now \textitbf{Fitri} returns the attack and kicks \textitbf{Martin} badly. At this point, the speaker interrupts the flow of her narrative about the two non-speech-act participants and employs the \textsc{np} \textsc{2sg} \isi{noun} phrase \textitbf{Martin ko} ‘you Martin’ to relate that \textitbf{Martin} fell to the ground. In turning away from her audience and addressing absent \textitbf{Martin}, the speaker gives “emotional impetus” to the fact that \textitbf{Martin} went down after having been kicked by a woman, thereby creating an emotional reaction in her audience.



\begin{styleExampleTitle}
\textsc{np} \textsc{2sg} \isi{noun} phrases in apostrophes: Human referents
\end{styleExampleTitle}

\ea
\label{Example_6.59}
\gll {{\bluebold{Martin}}} {{\bluebold{dia}}} {{lewat}} {{tete,}} {{de}} {{lompat}} {{mo}} {pukul} {Fitri}\\ %
 {Martin}  {\textsc{3sg}}  {pass.by}  {grandfather}  {\textsc{3sg}}  {jump}  {want}  hit  Fitri\\
 \gll {\ldots} {Fitri}  {kas}  {naik}  {kaki}  di  {sini,}  {\bluebold{Martin}}  \bluebold{ko}  {jatu,}\\
  { }  {Fitri}  {give}  {ascend}  {foot}  at  {\textsc{l.prox}}  {Martin}  \textsc{2sg}  {fall}\\
 \gll {dia}  {lari}  {ke}  {mari,}  {dia}  {mo}  {pukul}  {Fitri}\\
 {\textsc{3sg}}  {run}  {to}  {hither}  {\textsc{3sg}}  {want}  {hit}  {Fitri}\\

\glt
[About a fight between Fitri and Martin:] ‘\bluebold{Martin} went past grandfather, he jumped (and) wanted to hit Fitri [and Fitri caught his foot and] Fitri kicked (Martin) here, \bluebold{you Martin} fell, (then) he ran (over) here, he wanted to hit Fitri’ \textstyleExampleSource{[081015-001-Cv.0018-0019]}
\z


\textsc{np} \textsc{2sg} apostrophes are also formed with nonhuman referents. They “imply a \textstyleChItalic{personification}\textit{ }of the nonhuman object that is addressed” {(Abrams and Harpham 2009: 314)}. In (\ref{Example_6.60}), for instance, the speaker recounts a stormy boat trip. Suddenly, she turns away from her audience to address the main protagonist \textitbf{anging} ‘wind’ with the \textsc{np} \textsc{2sg} \isi{noun} phrase \textitbf{anging ko} ‘you wind’. In the example in (\ref{Example_6.1}), repeated as (\ref{Example_6.61}), the speaker relates how one of his ancestors came down to the coast. Seeing the ocean for the first time, he mistakes it for a wide river. At this point the speaker turns away from his audience to address this \textitbf{sungay} ‘river’ with the \textsc{np} \textsc{2sg} \isi{noun} phrase \textitbf{sungay ko} ‘you river’. Note that the \isi{apostrophe} is part of an appositional \textsc{pro} \textsc{np} construction with preposed \textitbf{ko} ‘\textsc{2sg}’, such that \textitbf{ko sungay ko} ‘you, you river’ (see §\ref{Para_6.1.6}).



\begin{styleExampleTitle}
\textsc{np} \textsc{2sg} \isi{noun} phrases in apostrophes: Nonhuman referents
\end{styleExampleTitle}

\ea
\label{Example_6.60}
\gll {\ldots} {\bluebold{anging}} {\bluebold{ko}} {datang} {suda,} {hujang} {besar} {datang} {suda}\\ %
  { } wind  \textsc{2sg}  come  already  rain  be.big  come  already\\

\glt
[About a storm during a boat trip:] ‘\bluebold{you wind} already came up, a big rain already came up’ \textstyleExampleSource{[080917-008-NP.0137]}
\z

\ea
\label{Example_6.61}
\gll {\ldots} {tida} {perna} {dia} {liat,} {[\bluebold{ko}]} {[\bluebold{sungay}} {\bluebold{ko}]} {bisa} {terbuka}\\ %
{  }   \textsc{neg}  once  \textsc{3sg}  see  \textsc{2sg}  river  \textsc{2sg}  be.able  be.opened\\
\gll {begini}\\
 {like.this}\\
\glt
[Seeing the ocean for the first time:] ‘[never before has he seen, what, a river that is so very big like this ocean,] never before has he seen \bluebold{you, you river} can be wide like this?’ \textstyleExampleSource{[080922-010a-CvNF.0212-0213]}
\z


\subsubsubsection{``\textsc{np 2sg}'' \textit{{noun} phrases and their head nominals}\label{Para_6.2.1.1.4}}

This section summarizes the syntactic and lexical properties of \textsc{np} \textsc{2sg} \isi{noun} phrases.



In the corpus, \textsc{np} \textsc{2sg} \isi{noun} phrases typically take the subject slot in clause-initial position, as in (\ref{Example_6.57}) to (\ref{Example_6.61}). There are a few exceptions, however: in (\ref{Example_6.62}) \textitbf{babi ko} ‘you pig’ occurs as an exclamation in clause-final position; in (\ref{Example_6.64}) \textitbf{kaka ko} ‘you older sibling’ denotes the possessor in an \isi{adnominal possessive construction} which, in turn, takes the clausal object slot; and in (\ref{Example_6.65}) \textitbf{pace ko} ‘you man’ expresses the possessor in an \isi{adnominal possessive construction} which, in turn, takes the complement slot in a \isi{prepositional phrase}. The referent can be encoded with a common \isi{noun} as in (\ref{Example_6.62}), a \isi{proper noun} as in (\ref{Example_6.67}), or a \isi{noun} phrase with an adnominal modifier as in (\ref{Example_6.63}). The referent is typically human; it can, however, also be inanimate such as \textitbf{anging} ‘wind’ in (\ref{Example_6.60}).



The utterances in (\ref{Example_6.57}) to (\ref{Example_6.61}) and (\ref{Example_6.64}) to (\ref{Example_6.68}) also show that \textitbf{ko} ‘\textsc{2sg}’ is freely used as a determiner; that is, its uses are not limited to exclamations, such as \textitbf{babi (puti) ko} ‘you (white) pig’ in (\ref{Example_6.62}) and (\ref{Example_6.63}).


\ea
\label{Example_6.62}
\gll {\ldots} {dasar} {bodo} {\bluebold{babi}} {\bluebold{ko}}\\ %
{ }   base  be.stupid  pig  \textsc{2sg}\\
\glt
‘[you (\textsc{sg}) here, do you (\textsc{sg}) have ears (or) not,] (you are of course) stupid, \bluebold{you pig}’ \textstyleExampleSource{[081014-016-Cv.0047]}
\z

\ea
\label{Example_6.63}
\gll {\bluebold{babi}} {\bluebold{puti}} {\bluebold{ko}} {dari} {atas} {turung}\\ %
 pig  be.white  \textsc{2sg}  from  top  descend\\

\glt
[About an acquaintance:] ‘\bluebold{you white pig} came down from up (there)’ \textstyleExampleSource{[081025-006-Cv.0260]}
\z

\ea
\label{Example_6.64}
\gll {sa} {taw} {\bluebold{kaka}} {\bluebold{ko}} {pu} {ruma}\\ %
 \textsc{1sg}  know  oSb  \textsc{2sg}  \textsc{poss}  house\\

\glt
‘I know \bluebold{you older brother}’s house’ \textstyleExampleSource{[080922-010a-CvNF.0238]}
\z

\ea
\label{Example_6.65}
\gll {nanti} {kitong} {lewat} {di} {\bluebold{pace}} {\bluebold{ko}} {pu} {kampung} {itu}\\ %
 very.soon  \textsc{1pl}  pass.by  at  man  \textsc{2sg}  \textsc{poss}  village  \textsc{d.dist}\\

\glt
‘later we’ll pass by \bluebold{you man}’s village there’ \textstyleExampleSource{[081012-001-Cv.0017]}
\z

\ea
\label{Example_6.66}
\gll {de} {blang,} {a,} {\bluebold{om}} {\bluebold{ko}} {\bluebold{ini}} {tra} {liat} {\ldots}\\ %
 \textsc{3sg}  say  ah!  uncle  \textsc{2sg}  \textsc{d.prox}  \textsc{neg}  see  \\

\glt
‘he said, ``ah, \bluebold{you uncle here} didn’t see {\ldots}''' \textstyleExampleSource{[080923-009-Cv.0051]}
\z

\ea
\label{Example_6.67}\label{bkm:Ref353011958}
\gll {\bluebold{Barce}} {\bluebold{ko}} {\bluebold{ini}} {ko} {takut}\\ %
 Barce  \textsc{2sg}  \textsc{d.prox}  \textsc{2sg}  feel.afraid(.of)\\

\glt
‘\bluebold{you Barce here}, you feel afraid’ \textstyleExampleSource{[081109-001-Cv.0131]}
\z

\ea
\label{Example_6.68}\label{bkm:Ref353011959}
\gll {\bluebold{Eferdina}} {\bluebold{ko}} {\bluebold{itu}} {ko} {taw} {kata} {pis} {ka} {tida}\\ %
 Eferdina  \textsc{2sg}  \textsc{d.dist}  \textsc{2sg}  know  word  please[E]  or  \textsc{neg}\\
\glt
‘\bluebold{you Eferdina there}, do you know the word ``please'' or not?’ \textstyleExampleSource{[081115-001a-Cv.0145]}
\z


\subsubsection[\textsc{np} \textsc{3sg} {noun} phrases]{\textsc{np} \textsc{3sg} {noun} phrases}
\label{Para_6.2.1.2}
In \textsc{np} \textsc{3sg} \isi{noun} phrases, the determiner \isi{pronoun} indicates and accentuates that the speakers assume their interlocutors to know the referents, encoded by the head nominals. That is, marking referents as familiar or given, \textitbf{dia}/\textitbf{de} ‘\textsc{3sg}’ signals to the hearers that they should be in a position to identify them. The determiner uses of \textitbf{dia}/\textitbf{de} ‘\textsc{3sg}’ can be situational or anaphoric. Both uses are discussed one by one, followed by a summary of the syntactic and lexical properties of \textsc{np} \textsc{3sg} \isi{noun} phrases.

\subsubsubsection{Situational uses of \textitbf{dia}/\textitbf{de} ‘\textsc{3sg}’ in ``\textsc{np 3sg}'' {noun} phrases\label{Para_6.2.1.2.1}}
In the situational uses of adnominal pronouns, “the physical situation in which the speaker and hearer are located contributes to the familiarity of the referent of the \isi{definite} \isi{noun} phrase” \citep[4]{Lyons.1999}. This cross-linguistic observation also applies to the situational uses of adnominally used \textitbf{dia}/\textitbf{de} ‘\textsc{3sg}’, as illustrated in (\ref{Example_6.69}) to (\ref{Example_6.71}).



In (\ref{Example_6.69}), the situation is an obvious one: the hearer \textitbf{Wili} has been irritating his \textitbf{tanta} ‘aunt’ and is told to stop doing this. In (\ref{Example_6.70}), the speaker illustrates local bride-price customs with an example. The determiner \textitbf{de} ‘\textsc{3sg}’ marks the familiarity of the referent \textitbf{bapa} ‘father’. This in turn leads the interlocutor to interpret \textitbf{bapa} ‘father’ as the speaker’s husband. In (\ref{Example_6.71}), the interlocutors discuss motorbike problems. Suddenly, the speaker quotes what \textitbf{Dodo de} ‘Dodo’ (literally ‘he Dodo’) had said. \textitbf{Dodo} had not been mentioned earlier and was not present at this conversation. Determiner \textitbf{de} ‘\textsc{3sg}’, however, signals to the hearers that they are familiar with the referent which, in turn, leads them to interpret the referent as the speaker’s older brother \textitbf{Dodo}.


\ea
\label{Example_6.69}
\gll {Wili} {ko} {jangang} {gara{\Tilde}gara} {[\bluebold{tanta}} {\bluebold{dia}} {\bluebold{itu}]}!\\ %
 Wili  \textsc{2sg}  \textsc{neg.imp}  \textsc{rdp}{\Tilde}irritate  aunt  \textsc{3sg}  \textsc{d.dist}\\

\glt
‘you Wili don’t irritate \bluebold{that aunt}!’ \textstyleExampleSource{[081023-001-Cv.0038]}
\z

\ea
\label{Example_6.70}
\gll {macang} {kalo} {[\bluebold{bapa}} {\bluebold{de}]} {kasi} {nona} {ini,} {a,} {nanti} {\ldots}\\ %
 variety  if  father  \textsc{3sg}  give  girl  \textsc{d.prox}  ah!  very.soon  \\

\glt
[About bride-price children:] ‘for example, if (my) \bluebold{husband} gives this (our) girl (to our relatives), ah, later {\ldots}’ \textstyleExampleSource{[081006-024-CvEx.0079]}
\z

\ea
\label{Example_6.71}
\gll {[\bluebold{Dodo}} {\bluebold{de}]} {bilang,} {adu} {coba} {ko} {kas} {taw} {sa}\\ %
 Dodo  \textsc{3sg}  say  oh.no!  if.only  \textsc{2sg}  give  know  \textsc{1sg}\\
\glt
‘\bluebold{Dodo} said, ``oh no, if only you had let me know''' \textstyleExampleSource{[081014-003-Cv.0029]}
\z


\subsubsubsection{Anaphoric uses of \textitbf{dia}/\textitbf{de} ‘\textsc{3sg}’ in ``\textsc{np 3sg}'' {noun} phrases\label{Para_6.2.1.2.2}}

In the anaphoric uses of adnominal pronouns, the referents of the \isi{definite} \isi{noun} phrase are “familiar not from the physical situation but from the linguistic context” {\citep[4]{Lyons.1999}}, as they were mentioned earlier in the discourse. The same observation applies to the anaphoric uses of adnominally used \textitbf{dia}/\textitbf{de} ‘\textsc{3sg}’.



In Papuan Malay, when introducing new protagonists, speakers typically introduce these individuals or entities with bare common or proper nouns. At their next mention, these non-speech-act participants are encoded with \textsc{np} \textsc{3sg} \isi{noun} phrases, with the third person \isi{pronoun} marking the referents as \isi{definite}. This, in turn, signals to the hearers that they are assumed to be familiar with the referents. This strategy is illustrated with the two narrative extracts in (\ref{Example_6.72}) and (\ref{Example_6.73}).\footnote{Introducing new characters with a bare \isi{noun} and subsequently marking them as familiar with the adnominally used third person \isi{pronoun} as a potential discourse strategy in Papuan Malay was brought to the author’s attention by A. van Engelenhoven (p.c. 2013).}



The utterances in (\ref{Example_6.72}) are part of a narrative about some bad news that the speaker received from his grandmother. The speaker introduces his grandmother as a new protagonist with the bare \isi{kinship term} \textitbf{nene} ‘grandmother’. This introduction involves two mentions of \textitbf{nene} ‘grandmother’; the \isi{repetition} gives the speaker time to reflect who it was that had been accompanying his grandmother when they met. Following this introduction, the speaker employs the \textsc{np} \textsc{3sg} \isi{noun} phrase \textitbf{nene de} ‘grandmother’ (literally ‘she grandmother’), which marks the new character as given and familiar.



\begin{styleExampleTitle}
Anaphoric uses of \textitbf{dia}/\textitbf{de} ‘\textsc{3sg}’: Example \#1
\end{styleExampleTitle}

\ea
\label{Example_6.72}
\gll {\ldots} {{pas}} {ketemu} {{deng}} {sa} {{pu}} {{\bluebold{nene},}} {\bluebold{nene},}\\ %
 { }  {precisely}  meet  {with}  \textsc{1sg}  {\textsc{poss}}  {grandmother}  grandmother\\
\gll {trus}  kaka  {laki{\Tilde}laki,}  {mama-tua}  {pu}  {ana}\\
 {next}  oSb  {\textsc{rdp}{\Tilde}husband}  {aunt}  {\textsc{poss}}  {child}\\
\glt
‘[I passed by (and) reached the village market there, I was sitting, standing there,] right then (I) met my \bluebold{grandmother}, \bluebold{grandmother} and then (my) older brother, aunt’s child’\\

 \gll baru  \bluebold{nene}  \bluebold{de}  mulay  tanya  saya,  \bluebold{de}  blang  {\ldots}\\
 and.then  grandmother  \textsc{3sg}  start  ask  \textsc{1sg}  \textsc{3sg}  say  \\
\glt
‘and then \bluebold{grandmother} started asking me, \bluebold{she} said, {\ldots}’ \textstyleExampleSource{[080918-001-CvNP.0056-0057]}
\z


The utterance in (\ref{Example_6.73}) occurred during a narrative about a bad-mannered intruder and a young woman named \textitbf{Rahab} who observed this person’s behavior. Employing a bare \isi{proper noun}, the speaker introduces \textitbf{Rahab} as a new character on the scene. At her next mention, this new protagonist is encoded by the \textsc{np} \textsc{3sg} \isi{noun} phrase \textitbf{Rahab de} ‘Rahab’ (literally ‘she Rahab’), which marks this non-speech-act participant as given and familiar. In the following, the speaker refers to \textitbf{Rahab} with the bare third person \isi{pronoun} \textitbf{de} ‘\textsc{3sg}’.



\begin{styleExampleTitle}
Anaphoric uses of \textitbf{dia}/\textitbf{de} ‘\textsc{3sg}’: Example \#2
\end{styleExampleTitle}

\ea
\label{Example_6.73}
\gll {baru} {{de}} {{luda{\Tilde}luda}} {keee,} {{\ldots}} {{\bluebold{Rahab}}} {{yang}} {liat,} {{\bluebold{Rahab}}} {\bluebold{de}}\\ %
 and.then  {\textsc{3sg}}  {\textsc{rdp}{\Tilde}spit}  spoot!  {}  {Rahab}  {\textsc{rel}}  see  {Rahab}  \textsc{3sg}\\
\gll {jemur{\Tilde}jemur}  {pakeang}  {begini}  {baru}  {\bluebold{de}}  {perhatikang,}  {\ldots}\\
 {\textsc{rdp}{\Tilde}be.dry}  {clothes}  {like.this}  {and.then}  {\textsc{3sg}}  {observe}  {}\\
\glt
[About a bad-mannered intruder:] ‘and then he was spitting ``spoot!'' {\ldots} (it was) \bluebold{Rahab} who saw (it), \bluebold{Rahab} was drying clothes at that moment, then \bluebold{she} noticed {\ldots}’ \textstyleExampleSource{[081006-035-CvEx.0042]}
\z


\subsubsubsection{``\textsc{np 3sg}'' {noun} phrases and their head nominals\label{Para_6.2.1.2.3}}

This section summarizes the syntactic and lexical properties of \textsc{np} \textsc{3sg} \isi{noun} phrases.



\textsc{np} \textsc{3sg} \isi{noun} phrases in the corpus typically take the subject slot in clause-initial position, as in (\ref{Example_6.70}) to (\ref{Example_6.73}). Other slots, however, are also possible, such as the direct object slot in (\ref{Example_6.69}), or the possessor slot in (\ref{Example_6.74}). The referent can be expressed with common nouns as in (\ref{Example_6.72}), proper nouns as in (\ref{Example_6.73}), or \isi{noun} phrases with adnominal modifiers, as in (\ref{Example_6.74}). Further, determiner \textitbf{dia}/\textitbf{de} ‘\textsc{3sg}’ also occurs in complex \isi{noun} phrases, as in \textitbf{bapa dari Jepang dia} ‘the man from Japan’ (literally ‘he man from Japan’) in (\ref{Example_6.75}), or \textitbf{kaka pendeta di Mambramo de tu} ‘that older pastor sibling in the Mambramo area’ (literally ‘that he older pastor in the Mambramo area’) in (\ref{Example_6.76}). The referents in \textsc{np} \textsc{3sg} \isi{noun} phrases are usually human, but they can also be animate nonhuman such as \textitbf{kaswari} ‘cassowary’ in (\ref{Example_6.77}), or inanimate such as \textitbf{bua mangga} ‘mango fruit’ in (\ref{Example_6.78}).


\ea
\label{Example_6.74}
\gll {\ldots} {di} {dano} {situ} {di} {[[\bluebold{kaka}} {\bluebold{laki{\Tilde}laki}} {\bluebold{de}]} {pu} {[tempat} {situ]]}\\ %
{ }   at  lake  \textsc{l.med}  at  oSb  \textsc{rdp}{\Tilde}husband  \textsc{3sg}  \textsc{poss}  place  \textsc{l.med}\\
\glt
‘[we wanted to pray a whole night while picnicking, at what’s-its-name,] at the lake there, at \bluebold{the older brother}’s place there’ \textstyleExampleSource{[080922-002-Cv.0090]}
\z

\ea
\label{Example_6.75}
\gll {\ldots} {{karna}} {ini} {\bluebold{bapa}} {\bluebold{dari}} {\bluebold{Jepang}} {\bluebold{dia}} {suda} {kutuk}  {kota}  {ini}\\ %
 { }  {because}  \textsc{d.prox}  father  from  Japan  \textsc{3sg}  already  curse  {city}  {\textsc{d.prox}}\\
\glt
‘{\ldots} because, what’s-his-name, \bluebold{the gentleman from Japan} already cursed this city’ \textstyleExampleSource{[080917-008-NP.0021]}
\z

\ea
\label{Example_6.76}
\gll {\bluebold{kaka}} {\bluebold{pendeta}} {\bluebold{di}} {\bluebold{Mambramo}} {\bluebold{de}} {\bluebold{tu}} {jual} {RW}\\ %
 oSb  pastor  at  Mambramo  \textsc{3sg}  \textsc{d.dist}  sell  cooked.dog.meat\\

\glt
‘\bluebold{that older sibling pastor in (the) Mambramo (area)} sells cooked dog meat’ \textstyleExampleSource{[081011-022-Cv.0105]}
\z

\ea
\label{Example_6.77}
\gll {\ldots} {ato} {\bluebold{kaswari}} {\bluebold{dia}} {ada} {berdiri} {pas} {perhatikang} {begini} {\ldots}\\ %
 { }   or  cassowary  \textsc{3sg}  exist  stand  be.exact  watch  like.this  \\

\glt
‘[if you see a cassowary’s footprint] or \bluebold{the cassowary} is standing right there watching (you) like this, {\ldots}’ \textstyleExampleSource{[080923-014-CvEx.0022]}
\z

\ea
\label{Example_6.78}
\gll {\ldots} {{bawa}} {{anaang}} {{pinang,}} {{anaang}} {{sagu,}} {{bibit}} {klapa,} {{bibit}}\\ %
  { } {bring}  {offspring}  {betel.nut}  {offspring}  {sago}  {seedling}  coconut  {seedling}\\
 \gll {pisang,}  {\ldots}  {mungking}  {\bluebold{bua}}  {\bluebold{mangga}}  {\bluebold{de}}  punya  {bibit}  {\ldots}\\
 {banana}  {}  {maybe}  {fruit}  {mango}  {\textsc{3sg}}  \textsc{poss}  {seedling}  \\
\glt
[About wedding customs:] ‘[(when) we bring (our son,] (we) bring betel nut seedlings, sago seedlings, coconut seedlings, banana seedlings, {\ldots} maybe \bluebold{seedlings of the mango fruit}, {\ldots}’ \textstyleExampleSource{[081110-005-CvPr.0056-0057]}
\z


\subsubsection[\textsc{np} \textsc{pro-sg} expressions with {comma intonation}]{\textsc{np} \textsc{pro-sg} expressions with comma intonation}
\label{Para_6.2.1.3}
The corpus also contains \textsc{np} \textsc{pro-sg} expressions in which the nouns are set off from the following pronouns by a \isi{comma intonation} (“{\textbar}”), as in (\ref{Example_6.79}) to (\ref{Example_6.82}).



In \textsc{np} \textsc{pro-sg} expressions with second person \textitbf{ko} ‘\textsc{2sg}’, the marked-off nouns function as vocatives, that is, as forms of direct address, such that ``\textsc{voc} \textsc{pro}''. Cross-linguistically, \textsc{voc} \textsc{pro} expressions serve to specify “a person out of a group of persons while using a second person singular \isi{pronoun}” with the vocative \isi{noun} being “separated from the rest of the sentence by intonation” \citep[46]{Bhat.2007}. This strategy of singling out and addressing particular individuals through a \textsc{voc} \textsc{pro} expression is shown in (\ref{Example_6.79}) and (\ref{Example_6.80}), respectively: \textitbf{mama} ‘mother’ and \textitbf{Ise} are vocatives which are set off from second person \textitbf{ko} ‘2\textsc{sg}’ with a distinct \isi{comma intonation} (“{\textbar}”). Hence, these expressions cannot be interpreted as \textsc{np} \textsc{2sg} \isi{noun} phrases.



\begin{styleExampleTitle}
Topic-comment constructions with \isi{comma intonation}: \textsc{np} {\textbar} \textsc{2sg}
\end{styleExampleTitle}

\ea
\label{Example_6.79}
\gll {trus} {Martina} {de} {tanya} {saya,} {\bluebold{mama}} {} \textup{\textbar} {} {\bluebold{ko}} {rasa} {bagemana?}\\ %
 next  Martina  \textsc{3sg}  ask  \textsc{1sg}  mother {} {} {}  \textsc{2sg}  feel  how\\

\glt
‘and then Martina asked me, ``\bluebold{mother}, how do \bluebold{you} feel?'' \textstyleExampleSource{[081015-005-NP.0018]}
\z

\ea
\label{Example_6.80}
\gll {{jadi}} {Ise} {{ni}} {tong} {su} {bilang} {dia,} {\bluebold{Ise}} {} \textup{\textbar} {} {\bluebold{ko}} {tinggal}\\ %
 {so}  Ise  {\textsc{d.prox}}  \textsc{1pl}  already  say  \textsc{3sg}  Ise  {} {} {} \textsc{2sg}  stay\\
\gll di  {sini}  {suda!}\\
 at  {\textsc{l.prox}}  {already}\\

\glt
‘so Ise here, we already told her, ``\bluebold{Ise}, \bluebold{you }stay here!''' \textstyleExampleSource{[080917-008-NP.0026]}
\z


\textsc{np} \textsc{3sg} expressions with a \isi{comma intonation} are analyzed as topic-comment constructions. Cross-linguistically, in topic-comment constructions, the “topic is generally expected to continue” and therefore “third person pronouns [{\ldots}] are used in order to represent the continued occurrence of a topic” \citep[209]{Bhat.2007}. This observation also applies to Papuan Malay \textsc{np} {\textbar} \textsc{3sg} expressions, that is, the preposed \isi{noun} phrase signals the topic, while coreferential \textitbf{dia}/\textitbf{de} ‘\textsc{3sg}’ has comment function. This strategy of forming topic-comment constructions is shown in (\ref{Example_6.81}) and (\ref{Example_6.82}): \textitbf{orang Senggi} and \textitbf{Klara} designate the topics while \textitbf{dia} ‘\textsc{3sg}’ and \textitbf{de} ‘\textsc{3sg}’ function as comments, respectively.



\begin{styleExampleTitle}
Topic-comment constructions with \isi{comma intonation}: \textsc{np} {\textbar} \textsc{3sg}
\end{styleExampleTitle}

\ea
\label{Example_6.81}
\gll {{baru}} {dia} {{datang,}} {{orang}} {{Jayapura}} {sana,} {kawang} {itu,}\\ %
 {and.then}  \textsc{3sg}  {come}  {person}  {Jayapura}  \textsc{l.dist}  friend  \textsc{d.dist}\\
\gll  \bluebold{orang}  {\bluebold{Senggi}} {} \textup{\textbar} {}  \bluebold{dia}  {datang}  {de}  {duduk}\\
 person  {Senggi}  {} {} {} \textsc{3sg}  {come}  {\textsc{3sg}}  {sit}\\

\glt
[Talking about a friend:] ‘and then she came, (the) person (from) Jayapura over there, that friend, \bluebold{(the) person (from) Senggi}, \bluebold{she} came (and) she sat (down)’ \textstyleExampleSource{[080917-008-NP.0107]}
\z

\ea
\label{Example_6.82}
\gll {Klara} {} \textup{\textbar} {} {\bluebold{de}} {lompat} {satu} {kali} {tu}\\ %
 Klara  {} {} {} \textsc{3sg}  jump  one  time  \textsc{d.dist}\\

\glt
‘\bluebold{Klara}, \bluebold{she} jumped once (\textsc{emph})’ \textstyleExampleSource{[081025-006-Cv.0216]}
\z


Topic-comment constructions with no \isi{comma intonation} are also possible, however. In this type of constructions, the topic is expressed in a \isi{noun} phrase with a \isi{pronoun} determiner and \isi{demonstrative} modifier, such that ``\textsc{n pro-sg dem}''. This is illustrated in (\ref{Example_6.67}), repeated as (\ref{Example_6.83}), and in (\ref{Example_6.68}) in §\ref{Para_6.2.1.1} (p. \pageref{Example_6.68}). Very often, however, the preposed topical \isi{noun} phrase does not contain a \isi{pronoun} determiner, such that ``\textsc{n dem}''. This is demonstrated with the topic-comment constructions \textitbf{ade ini de} ‘this younger sibling, he/she’ in (\ref{Example_6.84}), and \textitbf{Ise ni de} ‘Ise here, she’ in (\ref{Example_6.85}). In [\textsc{n (pro-sg) dem]} \textsc{pro-sg} topic-comment constructions, the \isi{demonstrative} sets aside the topic, and therefore no \isi{comma intonation} is needed.



\begin{styleExampleTitle}
Topic-comment constructions with \isi{demonstrative}: \textsc{np dem} \textsc{pro-sg}
\end{styleExampleTitle}

\ea
\label{Example_6.83}
\gll {[\bluebold{Barce}} {\bluebold{ko}} {\bluebold{ini}]} {[\bluebold{ko}]} {takut}\\ %
 Barce  \textsc{2sg}  \textsc{d.prox}  \textsc{2sg}  feel.afraid(.of)\\

\glt
‘\bluebold{you Barce here}, \bluebold{you} feel afraid’ \textstyleExampleSource{[081109-001-Cv.0131]}
\z

\ea
\label{Example_6.84}
\gll {baru} {[\bluebold{ade}} {\bluebold{ini}]} {[\bluebold{de}]} {sakit}\\ %
 and.then  ySb  \textsc{d.prox}  \textsc{3sg}  be.sick\\

\glt
‘and then \bluebold{this younger sibling, he/she} is sick’ \textstyleExampleSource{[080917-002-Cv.0020]}
\z

\ea
\label{Example_6.85}
\gll {\ldots} {[\bluebold{Ise}} {\bluebold{ni}]} {[\bluebold{de}]} {su} {mulay} {takut} {ini}\\ %
 { }  Ise  \textsc{d.prox}  \textsc{3sg}  already  start  feel.afraid(.of)  \textsc{d.prox}\\

\glt
‘[this tree began shaking, shaking like this, and] \bluebold{Ise here,} \bluebold{she} already started feeling afraid’ \textstyleExampleSource{[080917-008-NP.0028]}
\z



At this stage in the research on Papuan Malay, it is not possible to tell if there are rules governing the choice between \textsc{np} {\textbar} \textsc{pro-sg} and \textsc{np dem} \textsc{pro-sg} topic-comment constructions. To answer this question more research is needed.


\subsubsection[Analysis of \textsc{np} \textsc{pro-sg} expressions as {noun} phrases and not as topic{}-comment constructions]{Analysis of \textsc{np} \textsc{pro-sg} expressions as {noun} phrases and not as topic-comment constructions}
\label{Para_6.2.1.4}
There are four reasons for analyzing the \textsc{np} \textsc{2sg} expressions in (\ref{Example_6.51}) to (\ref{Example_6.68}) and the \textsc{np} \textsc{3sg} constructions in (\ref{Example_6.69}) to (\ref{Example_6.78}) as \isi{noun} phrases with a pronominal determiner and not as topic-comment constructions.



First, \textsc{np} \textsc{pro-sg} expressions can occur in positions other than the clause-initial subject slot, as shown with the \textsc{np} \textsc{2sg} \isi{noun} phrases in (\ref{Example_6.62}), (\ref{Example_6.64}), and (\ref{Example_6.65}), and the \textsc{np} \textsc{3sg} \isi{noun} phrases in (\ref{Example_6.69}) and (\ref{Example_6.74}). In these positions, however, the respective common nouns cannot be interpreted as topics in topic-comment constructions. This is due to the fact that topicalized constituents do not remain in-situ but are fronted to the clause-initial position (see also Example (\ref{Example_1.4}) in §\ref{Para_1.6.1.4}, p. \pageref{Example_1.4}).



Second, an \textsc{np} \textsc{pro-sg} expression can be modified with a \isi{demonstrative}, as in the \textsc{np} \textsc{2sg} \isi{noun} phrases in (\ref{Example_6.66}) to (\ref{Example_6.68}), or the \textsc{np} \textsc{3sg} \isi{noun} phrases in \ref{Example_6.69}) and (\ref{Example_6.76}). In these ``\textsc{np} \textsc{pro-sg dem}'' expressions, the demonstratives have scope over the pronouns. The fact that the pronouns occur in \isi{noun} phrases with adnominal \isi{demonstrative}, in turn, supports the conclusion that in \textsc{np} \textsc{pro-sg} expressions the pronouns function as determiners. Moreover, in two of the examples, the \textsc{np} \textsc{pro-sg} \textsc{dem} expressions have topic function in topic-comment constructions, namely the \textsc{np} \textsc{2sg} \textsc{dem} \isi{noun} phrases in (\ref{Example_6.67}) and (\ref{Example_6.68}). In both cases, the preposed \isi{noun} phrases are copied by coreferential \textitbf{ko} ‘\textsc{2sg}’ which has comment function.\footnote{There is no \isi{comma intonation} between the topical \isi{noun} phrases and the pronominal comments in (\ref{Example_6.67}) and (\ref{Example_6.69}).} Neither bare \textitbf{Barce} in (\ref{Example_6.67}), nor bare \textitbf{Eferdina} in (\ref{Example_6.68}) can be topics in topic-comment constructions. Instead it is the entire \isi{noun} phrase, including determiner \textitbf{ko} ‘2\textsc{sg}’, which has topic function. This, in turn, also supports the conclusion that in \textsc{np} \textsc{pro-sg} expressions, the \isi{pronoun} functions as a pronominal determiner.



Third, by indicating person, singularity, and definiteness of their referents, determiner pronouns have pertinent discourse functions. In \isi{direct speech}, \textsc{np} \textsc{pro-sg} expressions with second person \textitbf{ko} ‘2\textsc{sg}’ mark the referent of the head nominal unambiguously as the intended addressee. In \isi{reported speech}, \textsc{np} \textsc{2sg} \isi{noun} phrases indicate that the referent is the addressee of the direct quotation. In addition, they signal to the hearers that they are in a position to identify the referent. Finally, as apostrophes in rhetoric figures of \isi{direct speech}, they serve as “exclamatory addressees”. \textsc{np} \textsc{pro-sg} expressions with third person \textitbf{dia}/\textitbf{de} ‘3\textsc{sg}’ signal and accentuate that the speakers expect their hearers to be familiar with the referents encoded by their head nominals. In other words, it is communicated to the interlocutors that they should be able to identify the referents.



Fourth, the corpus includes a number of utterances, in which speakers repeat an \textsc{np} \textsc{pro-sg} expression as a form of hesitation or delay; in each case the \isi{pronoun} is third person \textitbf{dia}/\textitbf{de} ‘3\textsc{sg}’. Two of these repetitions are presented in (\ref{Example_6.86}) and (\ref{Example_6.87}). It is noted that the speakers do not repeat the respective bare nouns \textitbf{pace} ‘man’ and \textitbf{Markus}, but the \textsc{np} \textsc{3sg} expressions \textitbf{pace de} ‘the man’ (literally ‘he man’) and \textitbf{Markus de} ‘Markus’ (literally ‘he Markus’). This suggests that they perceive these expressions to be cohesive entities which, in turn, supports their analysis as single \isi{noun} phrases.


\ea
\label{Example_6.86}
\gll {[\bluebold{pace}} {\bluebold{de}],} {[\bluebold{pace}} {\bluebold{de}]} {mandi} {rapi,} {de} {mandi} {rapi}\\ %
 man  \textsc{3sg}  man  \textsc{3sg}  bathe  be.neat  \textsc{3sg}  bathe  be.neat\\

\glt
‘\bluebold{the man}, \bluebold{the man} bathed neatly, he bathed neatly’ \textstyleExampleSource{[081109-007-JR.0002]}
\z

\ea
\label{Example_6.87}
\gll {akirnya} {[\bluebold{Markus}} {\bluebold{de}],} {[\bluebold{Markus}} {\bluebold{dia}]} {turung} {begini}\\ %
 finally  Markus  \textsc{3sg}  Markus  \textsc{3sg}  descend  like.this\\
\glt
‘finally \bluebold{Markus}, \bluebold{Markus} came down (to the coast) like this’ \textstyleExampleSource{[080922-010a-CvNF.0204]}
\z


\subsection{Adnominal plural personal pronouns}
\label{Para_6.2.2}
Plural pronouns also function as determiners in \isi{noun} phrases, such that as illustrated in (\ref{Example_6.88}) and (\ref{Example_6.89}). They signal the definiteness and person-number values of their referents, and thereby allow their unambiguous identification.


\ea
\label{Example_6.88}
\gll {[\bluebold{pemuda}} {\bluebold{dong}]} {snang} {skali}\\ %
 youth  \textsc{3pl}  feel.happy(.about)  very\\

\glt
‘\bluebold{the young people} feel very happy’ (Lit. ‘\bluebold{youth they}’) \textstyleExampleSource{[080925-003-Cv.0220]}
\z

\ea
\label{Example_6.89}
\gll {[\bluebold{Ise}} {\bluebold{dong}]} {su} {datang}\\ %
 Ise  \textsc{3pl}  already  come\\

\glt
‘\bluebold{Ise and her companions including herself} already came’ (Lit. ‘\bluebold{Ise they}’) \textstyleExampleSource{[080925-003-Cv.0169]}
\z


The examples in (\ref{Example_6.88}) and (\ref{Example_6.89}) also show that \textsc{n} \textsc{pro-pl} \isi{noun} phrases have two readings.



First with an \isi{indefinite} referent, such as \textitbf{pemuda} ‘youth’ in (\ref{Example_6.88}), \textsc{n} \textsc{pro-pl} \isi{noun} phrases have an additive plural reading. Second with a \isi{definite} referent such as \textitbf{Ise} in (\ref{Example_6.89}), \textsc{n} \textsc{pro-pl} \isi{noun} phrases receive an \isi{associative} inclusory plural reading. This makes Papuan Malay belong to the large group of languages in Asia where the “\isi{associative plural} marker [{\ldots}] is also used to express additive plurals” \citep[5–6]{Daniel.2013}.



The \isi{additive plural interpretation} of \textsc{n} \textsc{pro-pl} \isi{noun} phrases is discussed in §\ref{Para_6.2.2.1} and the \isi{associative} inclusory plural reading in §\ref{Para_6.2.2.2}. These descriptions are followed in §\ref{Para_6.2.2.3} by a brief overview of the \isi{associative plural} in other \ili{regional Malay varieties}.


\subsubsection[Additive plural interpretation]{Additive plural interpretation}
\label{Para_6.2.2.1}
In \textsc{n} \textsc{pro-pl} \isi{noun} phrases with \isi{indefinite} referents, adnominal plural pronouns have two functions. They signal the definiteness of their referents and an additive plural reading of the respective \isi{noun} phrases with the basic meaning of ``the Xs''.



Cross-linguistically, the additive interpretation implies referential homogeneity of the group. That is, “every referent of the plural form is also a referent of the stem” \citep[1]{Daniel.2013}. This additive reading of Papuan Malay \textsc{n} \textsc{pro-pl} is illustrated in (\ref{Example_6.90}) to (\ref{Example_6.92}). In (\ref{Example_6.90}), \textitbf{kitorang} ‘\textsc{1pl}’ denotes the plurality of its bare head nominal \textitbf{nene} ‘grandmother’, while in (\ref{Example_6.91}) \textitbf{kamu} ‘\textsc{2pl}’ signals the plurality of \textitbf{bangsat} ‘rascal’, and in (\ref{Example_6.92}) \textitbf{dong} ‘\textsc{3pl}’ indicates the plurality of \textitbf{anjing} ‘dog’. These examples also show that the referent is always animate. It can be human as in (\ref{Example_6.90}) and (\ref{Example_6.91}), or nonhuman as in (\ref{Example_6.92}); inanimate referents are unattested.



\begin{styleExampleTitle}
Additive plural interpretation with bare head nominal
\end{styleExampleTitle}

\ea
\label{Example_6.90}
\gll {jadi} {\bluebold{nene}} {\bluebold{kitorang}} {\bluebold{ini}} {masak}\\ %
 so  grandmother  \textsc{1pl}  \textsc{d.prox}  cook\\

\glt
‘so \bluebold{we grandmothers here} cook’ \textstyleExampleSource{[080924-001-Pr.0008]}
\z

\ea
\label{Example_6.91}
\gll {\bluebold{bangsat}} {\bluebold{kamu}} {\bluebold{tu}} {tinggal} {lari} {ke} {sana} {ke} {mari}\\ %
 rascal  \textsc{2pl}  \textsc{d.dist}  stay  run  to  \textsc{l.dist}  to  hither\\

\glt
‘\bluebold{you rascals there} keep running back and forth’ \textstyleExampleSource{[080923-012-CvNP.0011]}
\z

\ea
\label{Example_6.92}
\gll {\ldots} {di} {mana} {\bluebold{anjing}} {\bluebold{dong}} {gong-gong}\\ %
 { }   at  where  dog  \textsc{3pl}  bark(.at)\\

\glt
‘[I just ran closing in on the pig] where \bluebold{the dogs} were barking’ \textstyleExampleSource{[080919-003-NP.0007]}
\z


In (\ref{Example_6.90}) to (\ref{Example_6.92}) the number of referents is left unspecified. When this number is limited to two, speakers very often use a dual construction, such that ``bare \textsc{n} \textsc{pro-pl} \textitbf{dua}''. In such a construction, the two referents are not explicitly mentioned but subsumed under the postposed adnominal \isi{numeral} \textitbf{dua} ‘two’, as in (\ref{Example_6.93}) and (\ref{Example_6.94}).



\begin{styleExampleTitle}
Additive dual interpretation
\end{styleExampleTitle}

\ea
\label{Example_6.93}
\gll {\bluebold{laki{\Tilde}laki}} {\bluebold{kam}} {\bluebold{dua}} {sapu}\\ %
 \textsc{rdp}{\Tilde}husband  \textsc{2pl}  two  sweep\\

\glt
‘\bluebold{you two boys} sweep’ \textstyleExampleSource{[081115-001b-Cv.0010]}
\z

\ea
\label{Example_6.94}
\gll {\bluebold{pace}} {\bluebold{dorang}} {\bluebold{dua}} {\bluebold{ini}} {ke} {atas}\\ %
 man  \textsc{3pl}  two  \textsc{d.prox}  to  top\\
\glt
‘\bluebold{the two men here} (went) up (there)’ \textstyleExampleSource{[081006-034-CvEx.0010]}
\z


\subsubsection[Associative {inclusory plural interpretation}]{Associative inclusory plural interpretation}
\label{Para_6.2.2.2}
\textsc{n} \textsc{pro-pl} \isi{noun} phrases with a \isi{definite} referent and an adnominal plural \isi{pronoun} receive an \isi{associative} inclusory plural reading. 



In her cross-linguistic semantic analysis of \isi{associative} plurals, \citet[470–471]{Moravcsik.2003} defines “\isi{associative} plurals as “constructions whose meaning is ‘X and X’s associate(s)’, where all members are individuals, X is the focal referent, and the associate(s) form a group centering around X”. In Papuan Malay, the focal referent is always encoded with a \isi{noun} or \isi{noun} phrase heading the phrasal construction, while the associates are encoded with a post-head plural \isi{pronoun}. In (\ref{Example_6.95}) and (\ref{Example_6.96}), for instance, \textitbf{Lodia} and \textitbf{Pawlus} are the focal referents while the pronouns \textitbf{torang} ‘\textsc{1pl}’ and \textitbf{dorang} ‘\textsc{3pl}’ denote the associates, respectively.



The reading of Papuan Malay \textsc{n} \textsc{pro-pl} \isi{noun} phrases is not only \isi{associative}, however. Adopting \citegen[479]{Moravcsik.2003} analysis, the reading of such \isi{noun} phrases is also “inclusory”, in that “all members of the plural set are summarily referred to by a \isi{pronoun}” (see also \citealt[25]{Haspelmath.2004}; \citealt{Gil.2009}). That is, the reference of the plural \isi{pronoun} in a Papuan Malay \textsc{n} \textsc{pro-pl} \isi{noun} phrase includes the reference of the focal referent, such that ``\textsc{pro} including X''. In (\ref{Example_6.95}), for instance, the \isi{pronoun} \textitbf{kam} ‘\textsc{2pl}’ includes not only the companions and the speaker, but all members of the plural set, “including \textitbf{Oktofina}”. That is, the \textsc{n} \textsc{pro-pl} \isi{noun} phrase \textitbf{Oktofina kam} does not signal an additive relation in the sense of ``Oktofina  plus you companions''. Likewise in (\ref{Example_6.96}), the reference of \textitbf{dorang} ‘\textsc{3pl}’ includes not only the associates of the focal referent \textitbf{Pawlus}, but all members of the plural set, ``including Pawlus''.


\begin{styleExampleTitle}
Associative \isi{inclusory plural interpretation}
\end{styleExampleTitle}

\ea
\label{Example_6.95}
\gll {\bluebold{tanta}} {\bluebold{Oktofina}} {\bluebold{kam}} {pulang} {jam} {brapa?}\\ %
 aunt  Oktofina  \textsc{2pl}  go.home  hour  several\\

\glt
‘what time did \bluebold{you aunt Oktofina and your companions including you (Oktofina)} come home?’ \textstyleExampleSource{[081006-010-Cv.0001]}
\z

\ea
\label{Example_6.96}
\gll {tanta} {ada} {mara} {\bluebold{Pawlus}} {\bluebold{dorang}}\\ %
 aunt  exist  be.angry  Pawlus  \textsc{3pl}\\

\glt
‘aunt is being angry with \bluebold{Pawlus and his companions including Pawlus}’ \textstyleExampleSource{[081006-009-Cv.0002]}
\z



In the following, the semantic properties of \isi{associative} inclusory expressions are examined. Also discussed are the lexical classes used in these expressions and the types of relationships expressed within the associated groups.



Cross-linguistically, \isi{associative} inclusory expressions imply three distinct semantic properties, namely “referential heterogeneity”, “reference to groups”, and “asymmetry” \citep{Daniel.2013, Moravcsik.2003}. The notion of “referential heterogeneity” implies that “the \isi{associative plural} designates a heterogeneous set” \citep[1]{Daniel.2013}. The semantic property of “reference to groups” refers to a high degree of internal cohesion within the plural construction; that is, the focal referent and the associates form “a spatially or conceptually coherent group” \citep[471]{Moravcsik.2003}. The notion of “asymmetry” implies that the groups are “ranked”, in that the \isi{associative plural} names its pragmatically most salient or highest ranking member, the focal referent \citep[471]{Moravcsik.2003}.



Referential heterogeneity of Papuan Malay \isi{associative} inclusory expressions is illustrated with the examples in (\ref{Example_6.97}) to (\ref{Example_6.99}). In (\ref{Example_6.97}), \textitbf{bapa Iskia dong} ‘father Iskia and them’ does not denote several people called \textitbf{Iskia}; neither does \textitbf{bapa desa dorang} ‘father mayor and them’ refer to more than one mayor. The same applies to the examples in (\ref{Example_6.98}) and (\ref{Example_6.99}) (in this context \textitbf{dokter} ‘doctor’ has a \isi{definite} reading as the local hospital has only one doctor). In each case, the plural \isi{pronoun} encodes a heterogeneous set of associates “centering around X”, the focal referent. Moreover, the pronouns include the focal referents in their reference.


\begin{styleExampleTitle}
Associative \isi{inclusory plural interpretation} with the third person plural \isi{pronoun}
\end{styleExampleTitle}

\ea
\label{Example_6.97}
\gll {{\bluebold{bapa}}} {{\bluebold{Iskia}}} {\bluebold{dong}} {bunu} {babi,} {\bluebold{bapa}} {\bluebold{desa}} {\bluebold{dorang}}\\ %
 {father}  {Iskia}  \textsc{3pl}  kill  pig  father  village  \textsc{3pl}\\
\gll dong  {bunu}  {babi}\\
 \textsc{3pl}  {kill}  {pig}\\
\glt
‘\bluebold{father Iskia and his companions including Iskia} killed a pig, \bluebold{father mayor and his companions including the mayor}, they killed a pig’ \textstyleExampleSource{[080917-008-NP.0120]}
\z

\ea
\label{Example_6.98}
\gll {Ise} {ko} {tinggal} {di} {sini} {suda} {deng} {\bluebold{mama-tua}} {\bluebold{dorang}!}\\ %
 Ise  \textsc{2sg}  stay  at  \textsc{l.prox}  just  with  aunt  \textsc{3pl}\\

\glt
‘you Ise just stay here with \bluebold{aunt and her companions including aunt}!’ \textstyleExampleSource{[080917-008-NP.0026]}
\z

\ea
\label{Example_6.99}
\gll {\bluebold{dokter}} {\bluebold{dorang}} {bilang} {begini} {\ldots}\\ %
 doctor  \textsc{3pl}  say  like.this  \\

\glt
‘\bluebold{the doctor and his companions including the doctor} said like this, {\ldots}’ \textstyleExampleSource{[081015-005-NP.0047]}
\z


The semantic property “reference to groups” is shown in (\ref{Example_6.97}) and (\ref{Example_6.98}). In the two examples, the \textsc{n} \textsc{pro-pl} \isi{noun} phrases denote coherent groups of inherently associated individuals, namely \textitbf{bapa Iskia dong} ‘father Iskia and them’, \textitbf{bapa desa dorang} ‘father mayor and them’, and \textitbf{mama-tua dorang} ‘aunt and them’, respectively. These examples also illustrate the notion of “ranking” in \isi{associative} inclusory expressions. The pragmatically highest ranking members are the focal referents \textitbf{bapa Iskia} ‘father Iskia’ and \textitbf{bapa desa} ‘father mayor’ in (\ref{Example_6.97}), and \textitbf{mama-tua} ‘aunt’ in (\ref{Example_6.98}). The remaining members of the plural sets, by contrast, are not fully enumerated but subsumed under the plural \isi{pronoun} \textitbf{dong/dorang} ‘3\textsc{pl}’.



Typically, the associates are encoded with the third person plural \isi{pronoun}. Less frequently, the associates are encoded with the first person plural \isi{pronoun}, as in (\ref{Example_6.95}), repeated as (\ref{Example_6.101}), or with the second person plural \isi{pronoun} as in (\ref{Example_6.101}) and (\ref{Example_6.102}). In \isi{associative} inclusory expressions formed with the second person plural \isi{pronoun}, the focal referent is typically the addressee as in (\ref{Example_6.101}). Alternatively, although much less often, one of the associates can be the addressee as in (\ref{Example_6.102}) (the focal referent \textitbf{Lodia} was not present during this conversation).

\exewidth{(123)}
\begin{styleExampleTitle}
Associative \isi{inclusory plural interpretation} with the first and second person plural pronouns
\end{styleExampleTitle}

\ea
\label{Example_6.100}
\gll {itu} {yang} {\bluebold{Lodia}} {\bluebold{torang}} {bilang} {begini} {\ldots}\\ %
 \textsc{d.dist}  \textsc{rel}  Lodia  \textsc{1pl}  say  like.this  \\

\glt
‘that’s why \bluebold{Lodia and her companions including me} said like this, {\ldots}’ \textstyleExampleSource{[081115-001a-Cv.0001]}
\z

\ea
\label{Example_6.101}
\gll {\bluebold{tanta}} {\bluebold{Oktofina}} {\bluebold{kam}} {pulang} {jam} {brapa?}\\ %
 aunt  Oktofina  \textsc{2pl}  go.home  hour  several\\

\glt
‘what time did \bluebold{you aunt Oktofina and your companions including you (Oktofina)} come home?’ \textstyleExampleSource{[081006-010-Cv.0001]}
\z

\ea
\label{Example_6.102}
\gll {\bluebold{Lodia}} {\bluebold{kam}} {pake} {trek} {ke} {sana} {baru} {sa} {\ldots}\\ %
 Lodia  \textsc{2pl}  use  truck  to  \textsc{l.dist}  and.then  \textsc{1sg}  \\

\glt
‘\bluebold{Lodia and her companions including you (addressee)} took the truck to (go) over there, and then I {\ldots}’ \textstyleExampleSource{[081022-001-Cv.0001]}
\z



In (\ref{Example_6.95}) to (\ref{Example_6.102}), the number of referents is not specified. When only two participants are involved, however, that is the focal referent plus one associate, Papuan Malay speakers very often use a dual construction, such that ``bare \textsc{n} \textsc{pro-pl} \textitbf{dua}'', as in (\ref{Example_6.103}). Like dual constructions with an additive reading (§\ref{Para_6.2.2.1}), the associate is not explicitly mentioned but subsumed under the post-head \isi{numeral} \textitbf{dua} ‘two’.



\begin{styleExampleTitle}
Associative inclusory dual interpretation
\end{styleExampleTitle}

\ea
\label{Example_6.103}
\gll {\bluebold{om}} {\bluebold{kitong}} {\bluebold{dua}} {kluar} {mo} {pergi} {cari} {pinang}\\ %
 uncle  \textsc{1pl}  two  go.out  want  go  search  betel.nut\\

\glt
‘\bluebold{uncle and I} went out and wanted to look for betel nuts’ \textstyleExampleSource{[081006-009-Cv.0014]}
\z



As for the lexical classes employed in \isi{associative plural} expressions, \citet[3]{Daniel.2013} observe “a clear preference for \isi{associative} plurals formed from proper names over kin terms over non-kin human common nouns over nonhuman nouns”. This also applies to Papuan Malay, in that the focal referents in \isi{associative} inclusory expressions are formed from human nouns while nonhuman animate focal referents are unattested. Among human nouns in the corpus, however, kin terms as in (\ref{Example_6.98}) are more common than proper names as in (\ref{Example_6.95}). This has to do with the fact that culturally people prefer not to use proper names, if they have another option, especially if the person is older and/or present. In addition, although not very often, \isi{associative plural} expressions are formed from non-kin terms such as the title \isi{noun} expression \textitbf{bapa desa} ‘father mayor’ in (\ref{Example_6.97}), or the common \isi{noun} \textitbf{dokter} ‘doctor’ in (\ref{Example_6.99}). (See also {Moravcsik 2003: 471–473}.)



Concerning the relationship between the focal referent X and the associates, {Daniel and \citet[3]{Moravcsik.2013}} note that “the group may be: (i) X’s family, (ii) X’s friends, or familiar associates, or (iii) an occasional group that X is a member of” with “kin forming the most commonly understood associates”. Papuan Malay also conforms to this cross-linguistic finding in that the associates are most often X’s family as in (\ref{Example_6.98}). Less commonly, X’s associates are friends or companions in a shared activity as in (\ref{Example_6.99}). Associative plurals denoting occasional groups or, according to {\citet[473]{Moravcsik.2003}}, “incidental association”, have not been identified in the corpus.


\subsubsection[Associative plural in other {regional Malay varieties}]{Associative plural in other regional Malay varieties}
\label{Para_6.2.2.3}
The \isi{associative plural} interpretation for \isi{noun} phrases with adnominal plural \isi{pronoun} is also quite common for other \ili{regional Malay varieties}, such as Ambon, Bali Berkuak, Dobo, Kupang, Manado, or \ili{Sri Lanka Malay}. In \ili{Ternate Malay}, however, pronouns do not have adnominal functions {\citep[141]{Litamahuputty.2012}}. The \isi{associative plural} reading of \isi{noun} phrases with adnominal plural pronouns found in \ili{regional Malay varieties} is illustrated in the examples in (\ref{Example_6.104}) to (\ref{Example_6.109}).



In Ambon, Dobo, Kupang, and \ili{Sri Lanka Malay}, the adnominal \isi{pronoun} is postposed as in Papuan Malay, as demonstrated in (\ref{Example_6.104}) to (\ref{Example_6.107}). In Balai Berkuak or \ili{Manado Malay}, by contrast, the \isi{pronoun} is in pre-head position, as shown in (\ref{Example_6.108}) and (\ref{Example_6.109}).



In all examples, the \isi{pronoun} is the third person plural \isi{pronoun}. In most varieties only the short \isi{pronoun} form is used as for instance in Ambon or \ili{Dobo Malay}, as shown in (\ref{Example_6.104}) and (\ref{Example_6.105}). Only in \ili{Manado Malay} are the short and long forms used, as shown in (\ref{Example_6.109}). Contrasting with Papuan Malay, these \ili{regional Malay varieties} do not use the first and second person plural pronouns to express \isi{associative} plurality.

\begin{styleExampleTitle}
\ili{Ambon Malay} \citep[169]{vanMinde.1997}
\end{styleExampleTitle}


\ea

\label{Example_6.104}
\gll {\bluebold{mama}} {\bluebold{dong}}\\ %
 mother  \textsc{3pl}    \\
 \glt ‘mother and the others’\\
\z

 
\begin{styleExampleTitle}
\ili{Dobo Malay} (R. Nivens p.c. 2013)
\end{styleExampleTitle}


\ea

\label{Example_6.105}
\gll {\bluebold{pa}} {\bluebold{Kace}} {\bluebold{dong}}\\ %
 man  Kace  \textsc{3pl}  \\
 \glt ‘Mr. Kace and his associates’\\
\z

 
\begin{styleExampleTitle}
\ili{Kupang Malay} \citep{Grimes.2008}
\end{styleExampleTitle}


\ea

\label{Example_6.106}
\gll
\bluebold{Yan}  \bluebold{dong}\\
 Yan  \textsc{3pl}    \\
 \glt ‘Yan and his family / mates’\\
\z

 
\begin{styleExampleTitle}
\ili{Sri Lanka Malay} \citep{Slomanson.2013}
\end{styleExampleTitle}


\ea

\label{Example_6.107}
\gll {\bluebold{Miflal}} {\bluebold{derang}}\\ %
 Miflal  \textsc{3pl}\\
 \glt ‘Miflal and his friends’\\
\z


\begin{styleExampleTitle}
\ili{Balai Berkuak Malay} \citep[7]{Tadmor.2002}
\end{styleExampleTitle}


\ea

\label{Example_6.108}
\gll {\bluebold{sidaq}} {\bluebold{Katalq}}\\ %
 \textsc{3pl}  Katalq    \\
 \glt ‘Katalq and her gang’\\
\z


\begin{styleExampleTitle}
\ili{Manado Malay} \citep[30]{Stoel.2005}
\end{styleExampleTitle}


\ea

\label{Example_6.109}
\gll {\bluebold{dong/dorang}} {\bluebold{Yoram}}\\ %
 \textsc{3pl}  Yoram    \\
 \glt ‘Yoram and his family’\\
\z



In short, among the \ili{eastern Malay varieties} Papuan Malay is unique given that \isi{associative plural} expressions are formed with all three plural persons, including the long and the short \isi{pronoun} forms. This different behavior of Papuan Malay \textsc{n} \textsc{pro-pl} \isi{noun} phrases supports the conclusion put forward in §\ref{Para_1.8} that the history of Papuan Malay is different from that of the other \ili{eastern Malay varieties}.\footnote{It is important to note, though, that the observed differences could also result from gaps in the descriptions of the other \ili{eastern Malay varieties}.}


\section{Summary}
\label{Para_6.3}
The Papuan Malay \isi{pronoun} system distinguishes singular and plural numbers and three persons. In addition to signaling the person-number values of their referents they also signal their definiteness. Each \isi{pronoun} has at least one long and one short form, with the exception of second person singular \textitbf{ko} ‘\textsc{2sg}’. The use of the long and short forms does not mark grammatical distinctions but represents speaker preferences. The pronouns have pronominal and adnominal uses.



In their pronominal uses, the pronouns substitute for \isi{noun} phrases and designate speech roles. The long and short \isi{pronoun} forms occur in all syntactic slots within the clause. For the direct and oblique object slots, however, speakers use the long forms much more often. These preferences interrelate with the preferred use of the ``heavy'' long \isi{pronoun} forms in clause-final position. This, in turn, reflects the cross-linguistic tendency for the clause-final position to be taken by ``heavy'' constituents. In adnominal possessive constructions, the pronouns only take the possessor slot; most often it is the short pronouns that take this slot. Pronouns also occur in inclusory \isi{conjunction}, \isi{summary conjunction}, and appositional constructions.



In their adnominal uses, the pronouns occur in post-head position and function as determiners. That is, signaling definiteness and person-number values, the pronouns allow the unambiguous identification of their referents. As determiners, the \isi{pronoun} forms of all person-number values are employed, with the exception of the first person singular. \textsc{np} \textsc{pro} \isi{noun} phrases with plural pronouns have two possible interpretations. With \isi{indefinite} referents, they have an additive plural reading, while with \isi{definite} referents they have an \isi{associative} inclusory reading.

