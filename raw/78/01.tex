\chapter[Introduction]{Introduction}\label{Para_1}

Papuan Malay is spoken in West Papua,\footnote{Formerly, West Papua was known as  ``Irian Jaya'' or  ``West Irian''.} which covers the western part of the island of New Guinea. The language is a nonstandard variety of Malay, belonging to the \ili{Malayic} branch within the \ili{Austronesian} language family.\footnote{The \ili{Malayic} branch also includes other \ili{eastern Malay varieties} as well as Standard Malay and Indonesian {(\citealt{Blust.2013}: xxiv–xl)}. (See §\ref{Para_1.2} for more details on the genetic affiliations of Papuan Malay.)}

Within the larger \ili{Malay continuum}, Papuan Malay forms a distinct, structurally coherent unit.

In West Papua, Papuan Malay is the language of wider communication and the first or second language for an ever-increasing number of people of the area (ca. 1,100,000 or 1,200,000 speakers). While Papuan Malay is not officially recognized, and therefore not used in formal government or educational settings or for religious preaching, it is used in all other domains, including unofficial use in formal settings, and, to some extent, in the public media.

This grammar describes Papuan Malay as spoken in the \ili{Sarmi} area, which is located about 300 km west of Jayapura. Both towns are situated on the northeast coast of West Papua (see \figref{Figure_0.1} on p. \pageref{Figure_0.1} and \figref{Figure_0.2} on p. \pageref{Figure_0.2}). After a general introduction to the language, presented in this chapter, the grammar discusses the following topics, building up from smaller grammatical constituents to larger ones: \isi{phonology}, word formation, word classes, \isi{noun} phrases, adnominal possessive relations, prepositional phrases, verbal and nonverbal clauses, non-declarative clauses, and conjunctions and \isi{constituent combining}.

This chapter provides an introduction to Papuan Malay. The first section gives a brief introduction to the larger geographical setting of Papuan Malay (§\ref{Para_1.1}). The genetic affiliations and the \isi{dialect situation} of the language are discussed in §\ref{Para_1.2} and §\ref{Para_1.3}, respectively. The \isi{linguistic setting} of Papuan Malay is examined in §\ref{Para_1.4}, followed in §\ref{Para_1.5} by a description of its \isi{sociolinguistic profile} and in §\ref{Para_1.6} of its \isi{typological profile}. Pertinent \isi{demographic information} is given in §\ref{Para_1.7}, and an overview of the history of Papuan Malay is presented in §\ref{Para_1.8}. Previous research on the language is summarized in §\ref{Para_1.9}, followed in §\ref{Para_1.10} by a brief overview of available materials in Papuan Malay. Finally, in §\ref{Para_1.11}, methodological aspects of the present study are described.


\section{Geographical setting}\label{Para_1.1}
\largerpage
Papuan Malay is mostly spoken in the coastal areas of West Papua. As there is a profusion of terms related to this geographical area, some terms need to be defined before providing more information on the geographical setting of Papuan Malay.

 ``West Papua'', the term adopted in this book, denotes the western part of the island of New Guinea. More precisely, the term describes the entire area west of the Papua New Guinea border up to the western coast of the Bird’s Head, as shown in \figref{Figure_0.1} (p. \pageref{Figure_0.1}; see also §\ref{Para_1.11.2} regarding the larger setting of the \isi{research location}).\footnote{The term ‘West Papua’ is also used in the literature, as for instance in \citet{King.2004}, \cite{Kingsbury.2002}, and \citet{Tebay.2005}. More recently, \citet{Gil.2014} has proposed the Malay term \textit{Tanah Papua} ‘Land of Papua’ for the western part of the island of New Guinea.}

In addition to the name  ``West Papua'', two related terms are used in subsequent sections, namely  ``Papua province'' and  ``Papua Barat province''. Both refer to administrative entities within West Papua. As illustrated in \figref{Figure_0.2} (p. \pageref{Figure_0.2}), Papua province covers the area west of the Papua New Guinea border up to the Bird’s Neck; the provincial capital is Jayapura. Papua Barat province, with its capital Manokwari, covers the Bird’s Head.

West Papua occupies the western part of New Guinea which belongs to the eastern Malay Archipelago. With its 317,062 square km, it covers about 40\% of New Guinea’s landmass. Its length from the border with Papua New Guinea in the east to the western tip of the Bird’s Head is about 1,200 km. Its north-south extension along the border with Papua New Guinea is about 700 km. The central part of West Papua is dominated by the Maoke Mountains. They are an extension of the mountain ranges of Papua New Guinea and, for the most part, covered with tropical rainforest. The northern and southern lowlands are covered with lowland rainforests and freshwater swamp forests which are drained by major river systems, such as the Mamberamo in the north and the Digul in the south. (See \citealt{EncyclopaediaBritannica.2001b}; \citeyear*{EncyclopaediaBritannica.2001}.)

{Major areas with }substantial concentrations of Papuan Malay speakers are the coastal urban areas of Jayapura and \ili{Sarmi} on the north coast, Merauke and Timika on the south coast, Fakfak and Sorong in the western part and Manokwari in the northeastern part of the Bird’s Head, and Serui on Yapen Island in Cenderawasih Bay. Other areas with substantial speaker numbers most likely include Nabire in the Bird’s Neck, \ili{Biak} Island in Cenderawasih Bay, and possibly Wamena in the highlands in central West Papua. (See {\citealt{Scott.2008}: 10}; see also \figref{Figure_0.2} on p. \pageref{Figure_0.2}.).

\section{Genetic affiliations}\label{Para_1.2}
As a Malay language, Papuan Malay belongs to the \ili{Malayic} sub-branch within the \ili{Mala\-yo-Polynesian} branch of the \ili{Austronesian} language family. A review of the literature suggests, however, that the exact classification of Papuan Malay is difficult for three reasons. First, as discussed in §\ref{Para_1.2.1}, the internal classification of the \ili{Malayo-Polynesian} subgroup is problematic. Moreover, there is a debate in the literature over the classification of the \ili{Malayic} languages within Western-\ili{Malayo-Polynesian}. Secondly, as discussed in §\ref{Para_1.2.2}, there is disagreement among scholars regarding the status of the \ili{eastern Malay varieties}, including Papuan Malay, as to whether they are non-\ili{creole} descendants of Low Malay or \ili{Malay-based creoles}. Thirdly, there is an ongoing debate over the legitimacy of Papuan Malay as a distinct language, as discussed in §\ref{Para_1.2.3}.

\subsection[Papuan Malay, a {Malayic} language]{Papuan Malay, a {Malayic} language within Malayo-Polynesian}\label{Para_1.2.1}
As a \ili{Malayic} language, Papuan Malay belongs to the \ili{Malayo-Polynesian} branch. Its classification within this branch is problematic, however.


In the literature the Malay languages are frequently classified as  ``Western \ili{Malayo-Polynesian}'' or  ``West-\ili{Malayo-Polynesian} (see for instance \citealt[227]{Adelaar.2001}; \citealt[677]{Nothofer.2009}; \citealt[791]{Tadmor.2009b}).


The existence of the Western \ili{Malayo-Polynesian} subgroup, however, is not well established. \citet[68]{Blust.1999}, for instance, points out that  ``Western \ili{Malayo-Polynesian} does not meet the minimal criteria for an established subgroup''. Hence, {Blust} concludes that Western \ili{Malayo-Polynesian} instead constitutes a  ``residue'' of languages which do not belong to the Central- and \ili{Eastern-Malayo-Polynesian} sub-branch \citep[68]{Blust.1999}. Along similar lines, \citet[14]{Adelaar.2005c} notes that Western \ili{Malayo-Polynesian}  ``does not have a clear linguistic foundation [\ldots] and the genetic affiliations of its putative members remain to be investigated''. \citet{Donohue.2008} also discuss the problematic status of the Western \ili{Malayo-Polynesian} subgroup. Based on phonological, morphological, and semantic innovations, the authors conclude that there is no basis for the Western \ili{Malayo-Polynesian} and Central/Eastern-\ili{Malayo-Polynesian} subgroups. In 2013, the status of the Western-\ili{Malayo-Polynesian} (WMP) subgroup remains problematic, with \citet[31]{Blust.2013} maintaining that it  ``is possible that WMP is not a valid subgroup, but rather consists of those MP [\ili{Malayo-Polynesian}] languages that do not belong to CEMP [Central-Eastern \ili{Malayo-Polynesian}]'' (see also \citealt[741–742]{Blust.2013}.



Moreover, there is disagreement among scholars with respect to the classification of the \ili{Malayic} languages within Western \ili{Malayo-Polynesian}. Based on phonological and morphological innovations, \citet[31ff]{Blust.1994} groups them within Malayo-\ili{Chamic} which is one of five subgroups within Western-\ili{Malayo-Polynesian}. The two branches of this grouping refer to the \ili{Malayic} languages of insular Southeast Asia, and the \ili{Chamic} languages of mainland Southeast Asia (see also \citealt[32]{Blust.2013}). \citet{Adelaar.2005d}, by contrast, suggests that \ili{Malayic} is part of a larger collection of languages, namely \ili{Malayo-Sumbawan}. This group has three branches. One includes the sub-branches Malay\-ic, \ili{Chamic}, and \ili{Balinese}-\ili{Sasak}-Sumbawa, while the other two include \ili{Sundanese} and \ili{Madurese}. \citet{Blust.2010}, however, rejects this larger \ili{Malayo-Sumbawan} grouping. Based on lexical innovations, he argues that \ili{Malayic} and \ili{Chamic} form  ``an exclusive genetic unit'' and should not be grouped together with \ili{Balinese}, \ili{Sasak}, and Sumbawanese (\citealt[80–81]{Blust.2010}; see also \citealt[736]{Blust.2013}). Hence, \citep[xxxii]{Blust.2013} classifies Papuan Malay as a \ili{Malayic} language within Malayo-\ili{Chamic}. This classification for the Malay languages within Malayo-\ili{Chamic} is also adopted by the \textstyleChItalic{Ethnologue} {\citep{Lewis.2016}}.

\newpage  %due to footnote i first lines of following section
\subsection{Papuan Malay, a non-{creole} descendant of low Malay}\label{Para_1.2.2}
Papuan Malay is a non-\ili{creole} descendant of \ili{low Malay}.\footnote{The term  ``\ili{low Malay}'' refers to  ``the colloquial form of Malay'', a trade language  ``existing in a diglossic situation [\ldots] with  ``\ili{High Malay}'' [\ldots] (which is usually defined as the classical literary language based upon the court language of Riau-Johor [\ldots])'' (\citealt[18–19]{Paauw.2003}; see also \citealt[18–25]{Paauw.2009}).}

There is an ongoing discussion in the literature, however, regarding the status of the \ili{eastern Malay varieties}, including Papuan Malay; that is, whether they are indeed non-\ili{creole} descendants of \ili{low Malay} or rather \ili{Malay-based creoles}.

Three factors contribute to this discussion: (1) the ``simple structure'' of Papuan Malay and the other \ili{eastern Malay varieties}, with their lack of inflectional \isi{morphology} and limited derivational processes (see §\ref{Para_1.6.1.2}), (2) the influence from non-\ili{Austronesian} languages which these languages, including Papuan Malay, show (see §\ref{Para_1.6.2}), and (3) the history of Malay as a trade language (see §\ref{Para_1.8}). These pertinent characteristics of the \ili{eastern Malay varieties} receive different interpretations.

Scholars such as {\citet[675]{Adelaar.1996}, } \citet{Donohue.2007b, Donohue.2011}, and \citet{McWhorter.2001} conclude that these languages best be characterized as Malay-based pidgins or creoles.

By contrast, other scholars, such as {\cite{Collins.1980}}, {\citet{Gil.2001}}, {\citet{Bisang.2009}}, and {\citet{Paauw.2013}}, and also earlier contributions by \citet{Donohue.2003} and \citet{Donohue.1998}, challenge the alleged \ili{creole} origins of the \ili{eastern Malay varieties}, given that structural simplicity is also found in inherited Malay varieties and also given that linguistic borrowing is not limited to pidgins or creoles.

This latter view is also the one adopted in the present description of Papuan Malay. The fact that Papuan Malay has a comparatively simple surface structure and some features typically found in Papuan but not in \ili{Austronesian} languages is not sufficient evidence to classify Papuan Malay as a \ili{creole}.

Throughout the remainder of this section, the different positions regarding the \ili{creole} versus non-\ili{creole} status of the \ili{eastern Malay varieties} are presented in more detail. The view that the \ili{eastern Malay varieties} are creolized languages is discussed first.

\citet[675]{Adelaar.1996} propose a list of eight structural features which illustrate the reduced \isi{morphology} of the \ili{eastern Malay varieties} and some of the linguistic features they borrowed from local languages. According to the authors, these features, which distinguish the \ili{eastern Malay varieties} from the \ili{western Malay varieties} and literary Malay, point to the pidgin origins of the \ili{eastern Malay varieties}, including those of West Papua. Hence, \citet{Adelaar.1996} propose the term \textstyleChItalic{Pidgin Malay Derived} dialects for these varieties. In a later study, {\citet[202]{Adelaar.2005d}} refers to the same varieties as \textstyleChItalic{Pidgin-Derived} Malay varieties. Another researcher who supports the view that the (eastern) Malay varieties are creolized languages is \citeauthor{McWhorter.2001} (\citeyear*{McWhorter.2001}; \citeyear*{ McWhorter.2005}; \citeyear*[197–251]{McWhorter.2007}). Considering the structural simplicity of Malay and its history as a trade language, he comes to the conclusion that Malay is an  ``anomalously decomplexified'' language which shows  ``the hallmark of a grammar whose transmission has been interrupted to a considerable degree (\citeyear*[197, 216]{McWhorter.2007}). The \textstyleChItalic{Ethnologue} \citep{Lewis.2016} also adopts the view that the \ili{eastern Malay varieties} are creolized languages and classifies them as \textstyleChItalic{Malay-based creoles}; these varieties include Ambon, Banda, Kupang, Larantuka, Manado, North Moluccan, and Papuan Malay. (See also \citealt{Roosman.1982}; \citealt{Burung.2007}.)

This view that the \ili{regional Malay varieties} are creolized languages is further found in descriptions of individual \ili{eastern Malay varieties} such as \ili{Ambon Malay}, \ili{Kupang Malay}, and \ili{Manado Malay}.

For \ili{Ambon Malay}, {\citet[115]{Grimes.1991}} argues that the language is a \ili{creole} or nativized pidgin. This conclusion is based on linguistic, sociolinguistic, and historical data, which the author interprets in light of \citegen[35]{Thomason.1988} framework of  ``contact-induced language change''. Following this framework, nativized pidgins are the long-term  ``result of mutual linguistic accommodation'' and  ``simplification'' in multilingual contact situations (\citeyear*[174, 205, 227]{Thomason.1988}). Along similar lines, {\citet[337]{Jacob.2011}} consider \ili{Kupang Malay} to be a Malay-based \ili{creole} that displays a substantial amount of influence from local substrate languages (see also \citealt{Jacob.2006}). \ili{Manado Malay} is also taken to be a \ili{creole} that developed from a local variety of \ili{Bazaar Malay} which is a \ili{Malay-lexified pidgin} (\citealt[411]{Prentice.1994}; \citealt[8]{Stoel.2005}).



Van Minde (\citeyear*{vanMinde.1997}), in his description of \ili{Ambon Malay}, and \citet{Litamahuputty.1994}, in her grammar of \ili{Ternate Malay}, by contrast, make no clear statements as to whether they consider the respective \ili{eastern Malay varieties} to be creolized languages or not.



In fact, the alleged \ili{creole} status and pidgin origins of the regional (eastern) Malay varieties have been contested by a number of scholars. \citet{Collins.1980}, \citet{Wolff.1988}, \citet{Gil.2001}, {\citet{Bisang.2009}}, and {\citet{Paauw.2013}}, for instance, argue that structural simplicity per se is not evidence for the pidgin origins of a language. Nor is the borrowing of linguistic features. {\citet{Blust.2013}} seems to have a similar viewpoint, although he does not overtly state this. Less clear is Donohue's (\citeyear*{Donohue.2003}; \citeyear*{Donohue.2007}; \citeyear*{Donohue.2007b}; \citeyear*{Donohue.2011}) and \citegen{Donohue.1998} position concerning the \ili{creole}/non-\ili{creole} status of the \ili{eastern Malay varieties}.



{\citet[35]{Bisang.2009}} challenges the view that low degrees of complexity should be taken as an indication to the pidgin/creole origins of a given language. In doing so, he specifically addresses the viewpoints put forward by \cite{McWhorter.2001, McWhorter.2005}. Paying particular attention to the languages of East and Southeast Mainland Asia, {\citet{Bisang.2009} }makes a distinction between overt and hidden complexity. The author shows that languages with a long-standing history may also have  ``simple surface structures [\ldots] which allow a number of different inferences and thus stand for hidden complexity'' (\citeyear*[35]{Bisang.2009}). That is, such languages do not oblige their speakers to employ particular structures if those are understood from the linguistic or extralinguistic context.



As far as particular \ili{regional Malay varieties} are concerned, \citet{Collins.1980}, for example, comes to the conclusion that \ili{Ambon Malay} is not a \ili{creole}. Examining sociocultural and linguistic evidence, the author compares \ili{Ambon Malay} to Standard Malay and to the nonstandard Malay variety \ili{Trengganu}. \ili{Ambon Malay} is spoken in a language-contact zone and held to be a \ili{creole}. \ili{Trengganu} Malay, by contrast, is spoken on the Malay Peninsula and considered an inherited Malay variety. This Malay variety, however, is also characterized by structural simplifications typically held to be characteristics of \ili{creole} languages. In consequence, \ili{Trengganu} Malay could well be classified as a \ili{creole} Malay just like \ili{Ambon Malay} \citep[42--53, 57--58]{Collins.1980}. As a result of his study, {Collins} questions the basis on which Malay varieties such as \ili{Ambon Malay} are classified as \ili{creole} languages, while other varieties such as \ili{Trengganu} are not. Arguing that the overly simplified categorization offered by \ili{creole} theory does not do justice to the \ili{Austronesian} languages, he comes to the following conclusion (\citeyear*[58--59]{Collins.1980}):

\begin{quote}
The term \ili{creole} has no predictive strength. It is a convenient label for linguistic phenomena of a certain time and place but it does not encompass the linguistic processes which are taking place in eastern Indonesia.
\end{quote}


In the context of his study on \ili{Banjarese Malay}, a variety spoken in southwestern Borneo, \cite{Wolff.1988} expresses a similar viewpoint. The author examines the question of whether \ili{Banjarese Malay} represents a direct \isi{continuation} of \ili{old Malay} or is the result of rapid language change, such as creolization. {Wolff} concludes that there is  ``absolutely no proof that any of the living dialects of Indonesian/Malay are indeed creoles'' (\citeyear*[86]{Wolff.1988}).



Another critique concerning the use of the term \textstyleChItalic{creoles} with respect to \ili{regional Malay varieties} is put forward by {\citet{Steinhauer.1991}} in his study on \ili{Larantuka Malay}. Given that too little is known about the origins and historical developments of the \ili{eastern Malay varieties}, the author argues that the label \textstyleChItalic{creole} is not very useful. Moreover, it becomes  ``meaningless'' if it is too  ``broadly defined'' in terms of the type of borrowing it takes for a language to be labeled a \ili{creole} (\citeyear*[178]{Steinhauer.1991}).



\citet{Gil.2001} also refutes the classification of the \ili{regional Malay varieties} as creolized languages and {\citegen{Adelaar.1996}} notion of \textstyleChItalic{Pidgin Malay Derived} dialects. More specifically, he argues that \cite{Adelaar.1996} do not give sufficient evidence that the original trade language was indeed a pidgin. Based on his research on \ili{Riau Indonesian}, \citet{Gil.2001} maintains that structural simplicity in itself is not sufficient evidence to conclude that a language is a \ili{creole}.



\citet{Paauw.2005,Paauw.2007,Paauw.2009,Paauw.2013} also takes issue with the classification of the \ili{eastern Malay varieties} as creolized languages. In his 2005 paper, {Paauw }points out that the features found in \textstyleChItalic{Pidgin Malay Derived} varieties \citep{Adelaar.1996} are also found in most of the inherited Malay varieties. Therefore, these features are better considered  ``markers of ‘low’ Malay, rather than contact Malay'' {\citep[17]{Paauw.2005}}. In another paper addressing the influence of local languages on the \ili{regional Malay varieties}, {\citet{Paauw.2007}} discusses some of the features which have been taken as evidence that these Malay varieties are creolized languages. He comes to the conclusion that borrowing in itself does not prove creolization. Otherwise,  ``it would be hard to find any language which couldn’t be considered a \ili{creole}'' (\citeyear*[3]{Paauw.2007}). In discussing the alleged pidgin origins and creolization of the \ili{eastern Malay varieties}, {\citet[26]{Paauw.2009}} maintains that there is not enough linguistic evidence for the claim that these are creoles. Likewise, {\citet[11]{Paauw.2013}} points out that there is no linguistic evidence for the pidgin origins of the \ili{eastern Malay varieties}, even though they developed under sociocultural and historical conditions which are typical for creolization. Instead, these varieties show many similarities with the inherited Malay varieties with respect to their lexicon, isolating \isi{morphology}, and syntax.



It seems that \citet{Blust.2013} also questions the classification of the \ili{eastern Malay varieties} as creoles. First, he lists the \ili{eastern Malay varieties} as Malayo-\ili{Chamic} languages rather than as creoles (\citeyear*[xxvii]{Blust.2013}). Second, in discussing pidg\-inization and creolization among \ili{Austronesian} languages, \citet[65–-66]{Blust.2013} refers in detail to \citegen{Collins.1980} study on \ili{Ambon Malay}. \citeauthor{Blust.2013} does not overtly state that he agrees with \citeauthor{Collins.1980}. He does, however, quote \citegen[58--5]{Collins.1980} above-mentioned conclusion that the label  ``\ili{creole} has no predictive strength'', without critiquing it. This, in turn, suggests that {Blust} has a similar viewpoint on this issue.



{Donohue’s} position about the \ili{creole}/non-\ili{creole} status of \ili{regional Malay varieties}, including Papuan Malay, is less clear. \citet[68]{Donohue.1998} argue that the different Malay varieties cannot be explained in terms of a single parameter such as  ``pure'' versus ``mixed or creolize''. With regard to Papuan Malay, \citet[1]{Donohue.2003} remarks that the fact that Papuan Malay displays six of the eight features found in {\citegen{Adelaar.1996}} \textstyleChItalic{Pidgin Malay Derived} varieties does not prove the pidgin origins of this Malay variety. Due to areal influence these features may also have developed independently in non-pidgin or non-\ili{creole} Malay varieties. In a later study on voice in Malay, \citet{Donohue.2007} takes a slightly different position in evaluating the contact which the Malay languages of eastern Indonesia had with non-\ili{Austronesian} languages. He concludes this contact caused  ``some level of language \isi{assimilation}'' and  ``language adaptation'', but he does not assert that this contact had to result in creolization (\citeyear*[1496]{Donohue.2007}). In another 2007 publication on voice \isi{variation} in Malay, \citet[72]{Donohue.2007b} notes that those Malay varieties spoken in areas far away from their traditional homeland show characteristics not found in the inherited Malay varieties. Moreover, in some areas these  ``transplanted'' Malay varieties have undergone  ``extensive creolization''. Finally, in his 2011 study on the \ili{Melanesian} influence on Papuan Malay \isi{verb} and clause structure, {Donohue} refers to Papuan Malay as one of the  ``ill-defined ‘eastern’ creoles'' spoken between New Guinea and Kupang. As such, it does not represent  ``an \ili{Austronesian} speech tradition'', with the exception of its lexicon (\citeyear*[433]{Donohue.2011}).



In concluding this discussion about the \ili{creole} versus non-\ili{creole} status of Papuan Malay, the author agrees with those scholars who challenge the view that the \ili{eastern Malay varieties} are creolized languages. Moreover, the author agrees with \citet[35, 43]{Bisang.2009}, who argues that complexity is not limited to the \isi{morphology} or syntax of a language. Instead, complexity may instead be found in the pragmatic inferential system as applied to utterances in their discourse setting. Such  ``hidden complexity'' is certainly a pertinent trait of Papuan Malay, as shown throughout this book. Two examples of hidden complexity are presented in (\ref{Example_1.1}) and (\ref{Example_1.2}). Due to the lack of morphosyntactic marking in Papuan Malay, a given construction can receive different readings, as illustrated in (\ref{Example_1.1}). Depending on the context, the \textitbf{kalo {\ldots} suda} ‘when/if {\ldots} already’ construction can receive a temporal or a counterfactual reading.\footnote{One anonymous reviewer suggests an alternative analysis for the example in (\ref{Example_1.1}). Rather than being ambiguous and exemplifying a case of  ``hidden complexity'', the \textitbf{kalo {\ldots} suda} `when/if {\ldots} already' construction expresses an unspecified reason-consequence relation, with the context supplying the information on whether the reason has place.} Example (\ref{Example_1.2}) illustrates the pervasive use of \isi{elision} in Papuan Malay. Verbs allow but do not require core arguments. Therefore, core arguments are readily elided when they are understood from the context ( ``Ø'' represents the omitted arguments).



\begin{styleExampleTitle}
{Examples of hidden complexity}
\end{styleExampleTitle}
\ea
\label{Example_1.1}
\gll \bluebold{kalo} de \bluebold{suda} kasi ana prempuang, suda tida ada\\ %
 if \textsc{3sg} already give child woman already \textsc{neg} exist\\
\gll prang suku lagi\\
war ethnic.group again\\
\glt \squarebrackets{About giving children to one’s enemy:}\\
Temporal reading: `\bluebold{once} she has given (her) daughter (to the other group), there will be no more ethnic war'\\
Counterfactual reading: ‘\bluebold{if} she \bluebold{had} given (her) daughter (to the other group), there would have been no more ethnic war’ \textstyleExampleSource{[081006-027-CvEx.0012]}
\z

\ea
\label{Example_1.2}
\gll {\ldots} karna de tida bisa bicara bahasa, maka \bluebold{Ø} pake\\ %
 { } because \textsc{3sg} \textsc{neg} be.able speak language therefore {} use\\
\gll bahasa orang bisu, {\ldots} baru \bluebold{Ø}  \bluebold{Ø} foto, foto,\\
language person be.mute {} and.then { } { } photograph photograph\\
\gll a, \bluebold{Ø} snang, prempuang bawa babi,  \bluebold{Ø} kasi \bluebold{Ø} \bluebold{Ø}\\
ah! { } {feel.happy (.about)} woman bring pig { } give \\
\glt
\squarebrackets{First outside contact between a Papuan group living in the jungle and a group of pastors:} `\squarebrackets{but they can’t speak Indonesian,} because she can’t speak Indonesian, therefore (\bluebold{she}) uses sign language {\ldots} (\bluebold{the pastor is taking}) pictures, pictures, ah, (\bluebold{the women} are) happy, the women bring a pig, (\bluebold{they}) give (\bluebold{it to the pastors})’ \textstyleExampleSource{[081006-023-CvEx.0073]}
\z



\subsection{Papuan Malay, a distinct language within the Malay continuum}\label{Para_1.2.3}
Papuan Malay is part of the larger Malay language continuum. The Malay varieties are situated geographically in a contiguous arrangement from the Malay Peninsula (Malay\-sia and Singapore) in the west across Malaysia, the Sultanate of Brunei Darussalam, and the Indonesian archipelago all the way to West Papua in the east (see \figref{Figure_0.1} on p. \pageref{Figure_0.1}).


This arrangement suggests a chaining pattern for the \ili{Malay continuum} in which the individual Malay speech groups have contact relationships with the other Malay groups surrounding them which results in the linguistic \isi{similarity} of adjoining groups. In consequence, adjacent varieties are likely to have higher degrees of inherent intelligibility than varieties that are situated at some distance to each other. That is, intelligibility decreases as the distance between the varieties along the chain increases, due to the increasing dissimilarities between the respective language systems (see \citealt{Karam.2000}: 126).



The chaining pattern of the Malay cluster raises the question whether Papuan Malay is a distinct language or a dialect of a larger Malay language, such as \ili{Standard Indonesian} which is expected to serve as a transvarietal standard {for other \ili{regional Malay varieties}.} To answer this question, three factors need to be taken into account: structural \isi{similarity}, inherent intelligibility, and shared ethnolinguistic identity with other Malay varieties. These are also the three criteria applied by the ISO 639-3 standard  ``for defining a language in relation to varieties which may be considered dialects'' (\citealt{Lewis.2016b}; see also \citealt{Hymes.1974}: 123).

 
First, structural \isi{similarity} with other Malay varieties: As a Malay variety, Papuan Malay shares many structural and lexical features with other Malay varieties. At the same time, however, Papuan Malay also exhibits a considerable amount of unique phonological, morphological, syntactic, lexical, and discourse features. These structural characteristics distinguish the language from other \ili{eastern Malay varieties}, such as Ambon, Manado, or North Moluccan Malay, as well as from the standard varieties of Malay, such as \ili{Standard Indonesian}. (See \citealt[3]{Anderbeck.2007}; \citealt[1]{Donohue.2003}; \citeyear*[73]{Donohue.2007b}; \citealt[20]{Paauw.2009}; \citealt[110--111]{Scott.2008}.)



Second, inherent intelligibility with other Malay varieties: For Papuan Malay speakers with no prior contact, the mentioned structural uniqueness has direct implications for their comprehension of other Malay varieties, in that they have difficulties understanding these varieties. That is, there is only limited or no inherent intelligibility between Papuan Malay and other Malay varieties. This applies especially to \ili{Standard Indonesian} and the \ili{western Malay varieties} in general. (See \citealt[3]{Anderbeck.2007}; \citealt[72--73]{Donohue.2007b}; \citealt[20]{Paauw.2009}; \citealt[27--28]{Suharno.1979}; \citealt[213--214]{Yembise.2011}.)



Third, shared ethnolinguistic identity with other Malay varieties: ethnolinguistically, Papuans typically identify with their respective indigenous vernacular languages, regardless as to whether or not they are still active speakers of that language. Beyond this local identity, they have a well-established, distinct identity as Papuans, especially vis-à-vis Indonesians from the western parts of Indonesia. This has largely to do with the ongoing Indonesian occupation (since 1963) and the negative attitudes that the Indonesian government and Indonesian institutions express toward Papuans and  ``Papuaness'' and also toward Papuan Malay (for more details on language attitudes see §\ref{Para_1.5.2}). Vice versa, Papuan attitudes towards Indonesia and  ``Indonesianess'' are also rather negative. (See for instance \citealt{Chauvel.2002}; \citealt{King.2004}.) Papuans summarize their distinct identity as follows: \textitbf{suku beda, bahasa beda, agama beda, adat beda} ‘(our) ethnicity is different, (our) language(s) is/are different, (our) religion is different, (and our) customs are different’. This statement was made to the author on numerous occasions during her stays in West Papua. This distinct ethnolinguistic identity vis-à-vis Papuan Malay is also evidenced by the names which Papuans use to refer to their language, names such as \textitbf{logat Papua} ‘Papuan speech variety’ or \textitbf{bahasa tanah} ‘home language’. Indonesian, by contrast, is always \textitbf{bahasa Indonesia} ‘Indonesian language’. These names for Papuan Malay not only indicate a strong, indigenous identification with their language. They also imply that Papuans are able to distinguish between their language and Indonesian \citep[19]{Scott.2008}. (See also the discussion on language awareness in §\ref{Para_1.5.2} ‘Language attitudes’.)



Given its structural uniqueness, limited or nonexistent inherent intelligibility, and the lack of shared ethnolinguistic identity with other Malay varieties, it is concluded here that Papuan Malay is a distinct language within the larger \ili{Malay continuum}. The ISO 639-3 code for Papuan Malay is [pmy] {\citep{Lewis.2016b}}.


\section{Dialect situation}\label{Para_1.3}
\largerpage
Papuan Malay is a structurally coherent unit with slight dialectal variations across the various regions where the language is spoken.



The identification of regional varieties of Papuan Malay is complicated, however, due to its linguistic and socio\isi{linguistic setting}, as \citet{Paauw.2009} points out. In West Papua both Papuan and \ili{Austronesian} languages are spoken.  ``Each of these languages has its own grammatical and phonological system which can influence the Malay spoken by individuals and communities'' (\citeyear*[75]{Paauw.2009}). Besides,  ``a large number of speakers of Papuan Malay are second-language speakers, and this too influences the linguistic systems of individuals and communities'' (\citeyear*[76]{Paauw.2009}).



To explore how many distinct varieties of Papuan Malay exist, a linguistic and sociolinguistic survey of the language was conducted in 2007 across West Papua \citep{Scott.2008}. The survey was carried out in and around seven coastal urban areas, namely Fakfak, Jayapura, Manokwari, Merauke, Timika, Serui, and Sorong (for details see §\ref{Para_1.9.2} and §\ref{Para_1.9.3}; see also \figref{Figure_0.2} on p. \pageref{Figure_0.2}). In these locations different Papuan and \ili{Austronesian} languages are spoken and second-language Papuan Malay speakers come from different linguistic backgrounds (see also §\ref{Para_1.4}).



The survey results suggest that regional differences of Papuan Malay are minor and limited to  ``differences in accent, pronunciation, and perhaps some differences in vocabulary'' \citep[18]{Scott.2008}.



With respect to the \isi{phonology}, \citet[24–44]{Scott.2008} mention the following regional features: (1) word-final voiceless plosives seem to be present in the eastern but not in the western parts West Papua; (2) the word-final lateral seems to fluctuate freely with the flap in the eastern part of West Papua, while the word-final lateral seems to be missing in the western part; (3) nasal \isi{assimilation} seems to occur in the western but not in the eastern parts of West Papua; (4) vowel harmony of [ə] to a vowel in another segment possibly occurs in the western but not in the eastern parts; and (5) the glottal fricative may be missing in the urban areas of Merauke. Overall, however, these differences are minor. At most, they possibly support an Eastern and Western Papuan Malay divide with Timika  ``sometimes following the Western regions of Fakfak and Sorong and sometimes following the Eastern regions of Jayapura and Merauke'' \citep[43]{Scott.2008}. This  ``possible East-West divide'', however, requires further research (\citeyear*[44]{Scott.2008}). Some of {\citegen{Scott.2008}} findings are modified by the current study (see \chapref{Para_2}): As for the word-final lateral, the corpus data does not show any fluctuation with the flap;\footnote{The word-final rhotic trill, however, may be devoiced if it occurs before a pause or in utterance-final position (§\ref{Para_2.3.1.3}).} nasal \isi{assimilation} does occur (§\ref{Para_2.2.1}). As far as the lexicon is concerned, regional differences also appear to be minor \citep[46, 96, 99]{Scott.2008}. Regional differences with respect to the grammar were not observed.



Overall, the data indicates that Papuan Malay as spoken across West Papua forms a structurally coherent unit despite its larger linguistic and socio\isi{linguistic setting}.



Moreover, while speakers are able  ``to identify others from different regions'' according to their usage of Papuan Malay, these regional variations do not impede comprehension:  ``Papuan Malay spoken in different regions of Papua is readily intelligible by Papuans from different regions of the province''; even children would understand Papuan Malay speakers from different regions  ``upon first exposure'' \citep[18]{Scott.2008}.



Taken together, these findings suggest that regional varieties of Papuan Malay are dialects of the same language rather than distinct, albeit closely related, languages. (See also \citealt[3]{Anderbeck.2007}, and the ISO 639-3 criteria for language identification in \citealt{Lewis.2016b}.\footnote{The ISO 639-3 standard applies three basic criteria for defining a language in relation to varieties which may be considered dialects. The first criterion considers intelligibility between speech varieties:  ``Two related varieties are normally considered varieties of the same language if speakers of each variety have inherent understanding of the other variety at a functional level (that is, can understand based on knowledge of their own variety without needing to learn the other variety)'' \citep{Lewis.2016}.})


The proposition that Papuan Malay is a structurally coherent unit modifies \citegen[1]{Donohue.2003} conclusion that  ``[it] is in a very real sense misleading to write about ‘Papuan Malay’ [\ldots] as if there was one unified variety of Malay spoken in the west of New Guinea''. \citeauthor{Donohue.2003} suggests that there are at least four distinct Papuan Malay varieties, without, however, addressing the question  ``whether these different varieties of Malay constitute an entity that can be called Papuan Malay in any linguistic sense'' (\citeyear*[1]{Donohue.2003}). Instead, \citeauthor{Donohue.2003} leaves this question  ``for a later date'' (\citeyear*[2]{Donohue.2003}). The most salient Papuan Malay varieties are listed below (\citeyear*[1--2]{Donohue.2003}; see also \figref{Figure_0.2} on p. \pageref{Figure_0.2}):

%\setcounter{itemize}{0}
\begin{enumerate}
\item 
\ili{North Papua Malay}, spoken along West Papua’s north coast between \ili{Sarmi} and the Papua New Guinea border; it shows a clear influence from \ili{Manado Malay} / North Moluccan Malay.

\item 
\ili{Serui Malay}, spoken in Cenderawasih Bay (except for the Numfor and \ili{Biak} islands); it is rather similar to \ili{North Papua Malay}.

\item 
\ili{Bird’s Head Malay}, spoken on the west of the Bird’s Head (in and around Sorong, Fakfak, Koiwai), is closely related to \ili{Ambon Malay}; the varieties spoken on the east of the Birds’ Head (in and around Manokwari and other towns) are similar to \ili{Serui Malay}.

\item 
\ili{South Coast Malay}, spoken in and around Merauke.

\end{enumerate}

\citet[2]{Donohue.2003} maintains, as mentioned,{ that the northern Papuan Malay varieties show  ``}a clear influence'' from \ili{Manado Malay} and/or North Moluccan Malay. As one example of this influence, he presents the lexical item \textitbf{kelemarin} ‘yesterday’, which is found in North-Moluccan Malay \citep[3]{Voorhoeve.1983}, but not in \ili{South Coast Malay}. The present corpus, by contrast, does not include any \textitbf{kelemarin} tokens. Instead, all attested 153 Papuan Malay tokens for ‘yesterday’ are realized with an alveolar rhotic. Neither do \citet{Scott.2008} make reference to the alternative realization of the word-internal rhotic as a lateral.



In summary, the findings of a linguistic and sociolinguistic language survey of different coastal regions of West Papua suggest that Papuan Malay forms a structurally coherent unit. Regional variations do occur, but they are minor and the observed differences support at most dialectal divisions, such as a possible East-West divide.

\section{Linguistic setting}\label{Para_1.4}
West Papua is the home of 274 languages, according to \citet{Lewis.2016}. Of these, 216 are non-\ili{Austronesian}, or Papuan, languages (79\%).\footnote{For a discussion of the term ‘\ili{Papuan languages}’ see Footnote \ref{Footnote_1.24} in §\ref{Para_1.6.2} (p. \pageref{Footnote_1.24}).}
 The remaining 58 languages are \ili{Austronesian} (21\%).\footnote{The \textstyleChItalic{Ethnologue} \citep{Lewis.2016} lists Papuan Malay as a Malay-based \ili{creole}, while here it is counted among the \ili{Austronesian} languages (see also §\ref{Para_1.2.2}). A listing of West Papua’s languages is available at \url{http://www.ethnologue.com/country/id/languages} and \url{http://www.ethnologue.com/map/ID_pe_} (accessed 8 January 2016).}


In the \ili{Sarmi} regency, where most of the research for this description of Papuan Malay was conducted, both Papuan and \ili{Austronesian} languages are found, as shown in \figref{Figure_0.3} (p. \pageref{Figure_0.3}). Between \ili{Bonggo} in the east and the Mamberamo River in the west, 23 \ili{Papuan languages} are spoken. Most of these languages belong to the \ili{Tor}-\ili{Kwerba} language family (21 languages). One of them is \ili{Isirawa}, the language of the author’s host family. The other twenty \ili{Papuan languages} are \ili{Airoran}, \ili{Bagusa}, \ili{Beneraf}, \ili{Berik}, \ili{Betaf}, \ili{Dabe}, \ili{Dineor}, \ili{Itik}, \ili{Jofotek-Bromnya}, \ili{Kauwera}, \ili{Keijar}, \ili{Kwerba}, \ili{Kwerba} Mamberamo, \ili{Kwesten}, \ili{Kwinsu}, \ili{Mander}, Maw\-es, \ili{Samarokena}, \ili{Trimuris}, and \ili{Wares}. The remaining two languages are \ili{Yoke} which is a \ili{Lower Mamberamo} language, and the isolate \ili{Massep}. In addition, eleven \ili{Austronesian} languages are spoken in the \ili{Sarmi} regency. All eleven languages belong to the \ili{Sarmi} branch of the \ili{Sarmi}-Jayapura Bay subgroup, namely \ili{Anus}, \ili{Bonggo}, \ili{Fedan}, \ili{Kaptiau}, \ili{Liki}, \ili{Masimasi}, \ili{Mo}, \ili{Sobei}, \ili{Sunum}, \ili{Tarpia}, and \ili{Yarsun}. While all of these languages are listed in the \textstyleChItalic{Ethnologue} \citep{Lewis.2016}, three of them are not included in \figref{Figure_0.3} (p. \pageref{Figure_0.3}), namely \ili{Jofotek-Bromnya} and \ili{Kaptiau}, both of which are spoken in the area around \ili{Bonggo}, and \ili{Kwinsu} which is spoken in the area east of \ili{Sarmi}.

\largerpage[2]
\begin{table}[h]
\caption{Status values and their numeric equivalents}\label{Table_1.0}
\begin{tabular}{ll}
\lsptoprule
Status & Numeric Value\\
\midrule
Developing & 5\\
Vigorous & 6a\\
Threatened & 6b\\
Shifting & 7\\
Moribund & 8a\\
Nearly extinct & 8b\\
\lspbottomrule
\end{tabular}
\end{table}
\clearpage

Of the 23 \ili{Papuan languages}, one is  ``developing'' (\ili{Kwerba}) and five are  ``vigorous'' (see \tabref{Table_1.0} and \tabref{Table_1.1}). The remaining languages are  ``threatened'' (7 languages),  ``shifting'' to Papuan Malay (7 languages),  ``moribund'' (1 language), or  ``nearly extinct'' (2 languages). One of the threatened languages is \ili{Isirawa}, the language of the author’s host family.\footnote{\label{Footnote_1.10}The \textstyleChItalic{Ethnologue} {\citep{Lewis.2016b}} gives the following definitions for the status of these languages: 5 (Developing) – The language is in vigorous use, with literature in a standardized form being used by some though this is not yet widespread or sustainable; 6a (Vigorous) – The language is used for face-to-face communication by all generations and the situation is sustainable; 6b (Threatened) – The language is used for face-to-face communication within all generations, but it is losing users; 7 (Shifting) – The child-bearing generation can use the language among themselves, but it is not being transmitted to children; 8a (Moribund) – The only remaining active users of the language are members of the grandparent generation and older; 8b (Nearly Extinct) – The only remaining users of the language are members of the grandparent generation or older who have little opportunity to use the language. For details see \url{http://www.ethnologue.com/about/language-status} (accessed 8 January 2016). See also \tabref{Table_1.0}.}


Most of the 23 \ili{Papuan languages} are spoken by populations of 500 or less (16 languages), and another three have between 600 and 1,000 speakers. Only three have larger populations of between 1,800 and 2,500 speakers. One of them is the  ``developing'' language \ili{Kwerba}.




\renewcommand{\ili}[1]{#1\il{#1}} %to avoid line breaks in table 1.2
%\begin{savenotes}
\begin{table}[b]
\caption[Papuan languages in the Sarmi regency]{Papuan languages in the Sarmi regency: Status and populations.\\  Numeric status values are shown \tabref{Table_1.0}
\footnote{See also Footnote \ref{Footnote_1.10} on p. \pageref{Footnote_1.10} for more information.}
}\label{Table_1.1}

%\todo[inline]{maybe create a separate table to explain the numeric status values. This will make producing them in each instance obsolete}

\begin{tabular}{lllr}
\lsptoprule
 Name & ISO 639-3 code & Status & \multicolumn{1}{r}{Population}\\
\midrule

Aironan & [air] & (Vigorous) &  1,000\\
\ili{Bagusa} & [bqb] & (Vigorous) &  600\\
\ili{Beneraf} & [bnv] & (Shifting) &  200\\
\ili{Berik} & [bkl] &  (Shifting) &  200\\
\ili{Betaf} & [bfe] &  (Threatened) &  600\\
\ili{Dabe} & [dbe] & (Shifting) &  440\\
\ili{Dineor} & [mrx] & (Moribund) &  55\\
\ili{Isirawa} & [srl] & (Threatened) &  1,800\\
\ili{Itik} & [itx] & (Threatened) &  80\\
\ili{Jofotek-Bromnya} & [jbr] & (Threatened) &  200\\
\ili{Kauwera} & [xau] & (Vigorous) &  400\\
\ili{Keijar} & [kdy] & (Shifting) &  370\\
\ili{Kwerba} & [kwe] & (Developing) &  2,500\\
\ili{Kwerba} Mamberamo & [xwr] & (Vigorous) &  300\\
\ili{Kwesten} & [kwt] & (Shifting) &  2,000\\
\ili{Kwinsu} & [kuc] & (Shifting) &  500\\
\ili{Mander} & [mqr] & (Nearly extinct) &  20\\
\ili{Massep} & [mvs] &  (Nearly extinct) &  25\\
\ili{Mawes} & [mgk] & (Threatened) &  850\\
\ili{Samarokena} & [tmj] & (Threatened) &  400\\
\ili{Trimuris} & [tip] & (Vigorous) &  300\\
\ili{Wares} & [wai] & (Shifting) &  200\\
\ili{Yoke} & [yki] & (Threatened) &  200\\
\lspbottomrule
\end{tabular} 
\end{table}

%\end{savenotes}



Three of the 23 \ili{Papuan languages} have been researched to some extent, namely  ``shifting'' \ili{Berik},  ``threatened'' \ili{Isirawa}, and  ``developing'' \ili{Kwerba}. The resources on these languages include word lists, descriptions of selected grammatical topics, issues related to literacy in these languages, anthropological studies, and materials written in these languages. \ili{Isirawa} especially has a quite substantial corpus of resources, including the New Testament of the Bible. Moreover, the language has seen a five-year literacy program. In spite of these language development efforts, the language is losing its users. In four languages, a \isi{sociolinguistic study} was carried out in 1998 {\citep{Clouse.2002}}, namely in Aironan, \ili{Massep}, \ili{Samarokena}, and \ili{Yoke}. Limited lexical resources are also available in \ili{Samarokena} and \ili{Yoke}, as well as in another eight languages (\ili{Beneraf}, \ili{Dabe}, \ili{Dineor}, \ili{Itik}, \ili{Kauwera}, \ili{Kwesten}, \ili{Mander}, and \ili{Mawes}). For the remaining eight languages no resources are available except for their listing in the \textstyleChItalic{Ethnologue} {\citep{Lewis.2016b}} and \textstyleChItalic{Glottolog} {\citep{Nordhoff.2013}}: \ili{Bagusa}, \ili{Betaf}, \ili{Jofotek-Bromnya}, \ili{Keijar}, \ili{Kwerba} Mamberamo, \ili{Kwinsu}, \ili{Trimuris}, and \ili{Wares}. (For more details see Appendix \ref{Para_C}.)\footnote{The \textstyleChItalic{Ethnologue} \citep{Lewis.2016b} provides basic information about these languages including their linguistic classification, alternate names, dialects, their status in terms of their overall development, population totals, and location. The \textstyleChItalic{Ethnologue} is available at \url{http://www.ethnologue.com} (accessed 8 January 2016). \textstyleChItalic{Glottolog} {\citep{Nordhoff.2013}} is an online resource that provides a comprehensive catalogue of the world’s languages, language families and dialects. \textstyleChItalic{Glottolog} is available at \url{http://glottolog.org/} (accessed 8 January 2016).}

Of the eleven \ili{Austronesian} languages, one is threatened, four are  ``shifting'' to Papuan Malay, five are  ``moribund'', and one is  ``nearly extinct'' (see \tabref{Table_1.2}). Most of these languages have less than 650 speakers. The exception is \ili{Sobei} with a population of 1,850 speakers. \ili{Sobei} is also the only \ili{Austronesian} language that has been researched to some extent. The resources on \ili{Sobei} include word lists, descriptions of some of its grammatical features, anthropological studies, and one lexical resource. In another four languages limited lexical resources are available. For the remaining six languages no resources are available, except for their listing in the \textstyleChItalic{Ethnologue} {\citep{Lewis.2016b}} and \textstyleChItalic{Glottolog} \citep{Nordhoff.2013}: \ili{Fedan}, \ili{Kaptiau}, \ili{Liki}, \ili{Masimasi}, \ili{Sunum}, and \ili{Yarsun}. (For more details see Appendix \ref{Para_C}.)

\begin{table}[b]

\caption[Austronesian languages in the Sarmi regency]{Austronesian languages in the Sarmi regency: Status and populations.\\ Numeric status values are shown in \tabref{Table_1.0}}
\label{Table_1.2}
%\todo[inline]{maybe create a separate table to explain the numeric status values. This will make producing them in each instance obsolete}

\begin{tabular}{lllr}
\lsptoprule
 Name & ISO 639-3 code & Status &  \multicolumn{1}{r}{Population}\\
\midrule
\ili{Anus} & [auq] & (Shifting) &  320\\
\ili{Bonggo} & [bpg] & (Moribund) &  320\\
\ili{Fedan} & [pdn] & (Moribund) &  280\\
\ili{Kaptiau} & [kbi] & (Shifting) &  230\\
\ili{Liki} & [lio] & (Moribund) &  11\\
\ili{Masimasi} & [ism] & (Nearly extinct) &  10\\
\ili{Mo} & [wkd] & (Shifting) &  550\\
\ili{Sobei} & [sob] & (Shifting) &  1,850\\
\ili{Sunum} & [ynm] & (Threatened) &  560\\
\ili{Tarpia} & [tpf] & (Moribund) &  630\\
\ili{Yarsun} & [yrs] & (Moribund) &  200\\
\lspbottomrule
\end{tabular}
\end{table}



\renewcommand{\ili}[1]{\il{#1}#1} %to avoid line breaks in table 1.2

\section{Sociolinguistic profile}\label{Para_1.5}
This section discusses the \isi{sociolinguistic profile} of Papuan Malay. In summary, this profile presents itself as follows:


\begin{itemize}
\item 
Strong and increasing language vitality of Papuan Malay;

\item 
Substantial language contact between Papuan Malay and Indonesian;

\item 
Functional distribution of Papuan Malay as the \textsc{low} variety, and Indonesian as the \textsc{high} variety, in terms of  {\citegen{Ferguson.1972}} notion of diglossia;

\item 
Positive to somewhat am\isi{bivalent} language attitudes toward Papuan Malay; and

\item 
{Lack of language awareness of many Papuan Malay speakers about the status of Papuan Malay as a language distinct from Indonesian.}
\end{itemize}

Papuan Malay is spoken in a rich linguistic and sociolinguistic environment, which includes indigenous Papuan and \ili{Austronesian} languages, as well as Indonesian and other languages spoken by migrants who have come to live and work in West Papua (see §\ref{Para_1.4} and §\ref{Para_1.7.1}). As in other areas of New Guinea, many Papuans living in the coastal areas of West Papua speak two or more languages (\citealt{Foley.1986}: 15–47; see also \citealt{Muhlhausler.1996}). The linguistic repertoire of individual speakers may include one or more local Papuan and/or \ili{Austronesian} vernaculars, Papuan Malay, and – depending on the speaker’s education levels – Indonesian, and also English, all of which are being used as deemed necessary and appropriate.



Many of the indigenous Papuan and \ili{Austronesian} languages are threatened by extinction. By contrast, the vitality of Papuan Malay is strong and increasing. This applies especially to urban coastal communities where Papuan Malay serves as a language of wider communication between members of different ethnic groups \citep[10–18]{Scott.2008}. In the \ili{Sarmi} regency, for instance, many vernacular languages are shifting, or have shifted, to Papuan Malay (see §\ref{Para_1.4}).



There is also substantial language contact between Papuan Malay and Indonesian.



The coexistence and interaction of indigenous vernacular languages, Papuan Malay, and Indonesian with their varying and overlapping roles creates a triglossic situation. More investigation is needed, however, to determine whether the interplay between all three best be explained in terms of \citegen[44–50]{Fasold.1984} notion of ``double overlapping diglossia'' or whether their functional distribution represents an instance of ``linear polyglossia''. For the present discussion, however, the status of the indigenous vernaculars vis-à-vis Papuan Malay and Indonesian is not further taken into consideration. Instead, the remainder of this section focuses on the interplay of Papuan Malay and Indonesian.



Both languages are in a diglossic distribution. In this situation, according to \citegen{Ferguson.1972} notion of diglossia, Indonesian serves as \textsc{h}, the \textsc{high} variety, which is acquired through formal education, and Papuan Malay as \textsc{l}, the \textsc{low} variety, which is acquired in informal domains, including the home domain.



Papuan Malay speakers display the typical language behavior of \textsc{low} speakers in their \isi{language use} patterns as well as with respect to their language attitudes. Language use and the diglossic distribution of Papuan Malay and Indonesian are discussed in §\ref{Para_1.5.1}, and language attitudes, together with language awareness, in §\ref{Para_1.5.2}.


\subsection{Language use}\label{Para_1.5.1}
The diglossic, or functional, distribution of Indonesian as the \textsc{high} variety and Papuan Malay as the \textsc{low} variety implies that in certain situations Indonesian is more appropriate while in other situations Papuan Malay is more appropriate.



In terms of \citegen[86]{Fishman.1965}  ``domains of language choice'', three factors influence such language choices: the topics discussed, the relationships between the interlocutors, and the locations where the communication takes place. Another factor to be taken into account is speaker education levels, given that Indonesian is acquired through formal education. Below the four factors are discussed in more detail.\footnote{Not further taken into account here is the growing influence of the mass media, namely TV, even in more remote areas which exposes Papuans more and more to colloquial varieties of Indonesian, especially Jakartan Indonesian (see also \citealt{Sneddon.2006}).}

%\setcounter{itemize}{0}
\begin{enumerate}

\item Speaker education levels\label{Item_1.1}

In diglossic situations, the \textsc{low} variety is known by everyone while the \textsc{high} variety is acquired through formal education \citep{Ferguson.1972}. This also applies to the diglossic distribution of Papuan Malay and Indonesian. While Papuan Malay is known by almost everyone in West Papua’s coastal areas, knowledge of Indonesian depends on speakers’ education levels.

The results of the mentioned 2007 survey \citep[14–17]{Scott.2008} show that bilingualism/multilingualism is  ``a common feature of the Papuan linguistic landscape''. The report does not, however, give details about the degree to which Papuans are bilingual in Indonesian, but notes that bilingualism levels remain uncertain.

During her 3-month fieldwork in \ili{Sarmi} (see §\ref{Para_1.11.3}), the author did not investigate bilingualism in Indonesian. She did, however, note changes in speakers’ language behavior depending on their education levels. Papuan Malay speakers with higher education levels displayed a general and marked tendency to  ``dress up'' their Papuan Malay with Indonesian features. This tendency was even more pronounced when discussing high topics (see Factor 2 ``Topical regulation''), or when interacting with group outsiders (see Factor 3 ``Relationships between interlocutors''). The observed features include lexical choices. Such choices are made between lexical items that are distinct in both languages, for example Indonesian \textitbf{desa} ‘village’ or \textitbf{mereka} ‘\textsc{3pl}’ instead of Papuan Malay \textitbf{kampung} ‘village’ and \textitbf{dorang}/\textitbf{dong} ‘\textsc{3pl}’, respectively. Lexical choices are also made between lexical items that are rather similar in both languages, such as Indonesian \textitbf{adik} [\textstyleChCharisSIL{ˈa.dɪk̚}] ‘younger sibling’ or \textitbf{tidak} [\textstyleChCharisSIL{ˈti.dɐk̚}] ‘\textsc{neg}’, instead of Papuan Malay \textitbf{ade} [\textstyleChCharisSIL{ˈa.dɛ}] ‘younger sibling’ and \textitbf{tida} [\textstyleChCharisSIL{ˈti.da}] ‘\textsc{neg}’, respectively. Other features are syntactic ones, such as Indonesian causatives formed with suffix \textitbf{\-kan} ‘\textsc{caus}’, passives formed with prefix \textitbf{di\-} ‘\textsc{uv}’, or possessives formed with suffix \textitbf{-nya} ‘\textsc{3possr}’.\footnote{For detailed grammatical descriptions of Indonesian see for instance {\citet{Mintz.1994}} and {\citet{Sneddon.2010}}.}

Less-educated speakers, by contrast, did not display this general tendency of mixing and switching to Indonesian given their more limited exposure to the \textsc{high} variety Indonesian. They only showed this tendency to  ``dress-up'' their Papuan Malay with Indonesian features or lexical items when discussing \textsc{high} topics (see Factor 2 ``Topical regulation''), or when interacting with fellow-Papuans of higher social standing or with group outsiders (see Factor 3 ``Relationships between interlocutors'').


\item Topical regulation\label{Item_1.2}


As {\citet[71]{Fishman.1965}} points out,  ``certain topics are somehow handled better in one language than in another''. The results of the 2007 survey provide only limited information about this issue, however. The findings only state that Papuan Malay is the preferred language for humor and that politics are typically discussed in the indigenous vernaculars (\cite{Scott.2008}: 17). The author’s own observations during her 3-month fieldwork in late 2008 modify these findings (see §\ref{Para_1.11.3}). The observed Papuan Malay speakers displayed a notable tendency to change their language behavior when discussing \textsc{high} topics. That is, when talking about topics associated with the formal domains of government, politics, education, or religion they tended to ``dress up'' their Papuan Malay and make it more Indonesian-like.

\item Relationships between interlocutors\label{Item_1.3}

Language behavior is not only influenced by the topics of communication and speaker education levels, but also by role relations. That is, individual speakers display certain language behaviors depending on the role relations between them \citep[76]{Fishman.1965}.

As for Papuan Malay, the 2007 survey results \citep[13, 14]{Scott.2008} indicate that family members and friends typically communicate in Papuan Malay or in the vernacular, but not in Indonesian. The same applies to informal interactions between customers and vendors, or between patients and local health workers. Teachers may also address their students in Papuan Malay in informal interactions (in informal interactions in primary school, students may even address their teachers in Papuan Malay). The report does not discuss which language(s) Papuans use when they interact with fellow-Papuans of higher social standing or with outsiders.

During her 3-month fieldwork in \ili{Sarmi} (see §\ref{Para_1.11.3}), however, the author did note changes in speakers’ language behavior depending on the role relations between interlocutors in terms of their status and community membership.

In interactions with fellow-Papuans of equally low status, less-educated Papuans typically used the \textsc{low} variety Papuan Malay. (At times, they also switched to \ili{Isirawa}, the vernacular language for most of them.) By contrast, when interacting with fellow-Papuans of higher social standing, such as teachers, mayors and other government officials, and pastors, or when conversing with group outsiders, that is non-Papuans, the observed speakers showed a marked tendency to change their language behavior. That is, in such interactions, their speech showed influences from the \textsc{high} variety Indonesian, similar to the general language behavior of better-educated speakers, described under Factor 1 ``Speaker education levels''. As for the language behavior of better-educated speakers, their general tendency to  ``dress-up'' their Papuan Malay with Indonesian features was even more marked when they interacted with group outsiders, such as the author. This tendency to  ``dress-up'' one’s Papuan Malay with Indonesian features reflects role relations, in that the use of Papuan Malay indicates intimacy, informality, and equality, while the use of Indonesian features signals social inequality and distance, as well as formality (see also \citealt[70]{Fishman.1965}).\footnote{All observed Papuans of higher social standing were also better educated, whereas none of the observed less-educated Papuans was of high social standing.}

\item Locations\label{Item_1.4}

Language behaviors are also influenced by the locations where communication takes place, in that speakers consider certain languages to be more appropriate in certain settings \citep[71, 75]{Fishman.1965}. This also applies to Papuan Malay. In certain domains, Papuan Malay speakers consider Indonesian to be more appropriate than Papuan Malay due to the diglossic distribution of both languages \citep[11–18]{Scott.2008}. That is, Indonesian is the preferred language for formal interactions in the education and religious domains (such as formal instruction, leadership, or preaching) or other public domains such as government offices. Papuan Malay strongly dominates all other domains. In addition, it is also the preferred language for informal interactions in public domains such as schools, churches, and government offices.
\end{enumerate}

\subsection{Language attitudes}\label{Para_1.5.2}
\citegen[70]{Fishman.1965} considerations of intimacy and distance, informality and formality also apply to Papuan Malay.


The findings of the 2007 survey indicate that Papuans associate Papuan Malay with intimacy and informality, while they associate Indonesian with social distance and formality \citep{Scott.2008}. The names which the interviewees used to refer to Papuan Malay reflect these positive feelings toward their language: \textitbf{bahasa tanah} ‘home language’, \textitbf{bahasa santay} ‘language to relax’, \textitbf{bahasa sehari-hari} ‘everyday language’, or \textitbf{bahasa pasar} ‘market/trade language’. Especially the name \textitbf{bahasa tanah} ‘home language’ suggests  ``a strong, indigenous identification with this speech form'' (\citeyear*[18]{Scott.2008}). Most interviewees also stated that they are interested in the development of Papuan Malay. Moreover, the majority of interviewees stated that Papuan Malay and Indonesian are of equal value and that Indonesian speakers do not deserve more respect than Papuan Malay speakers. Given these findings, the researchers come to the conclusion that among the interviewed Papuans attitudes toward Papuan Malay are  ``remarkably positive'' (\citeyear*[18–22]{Scott.2008}).


The expressed attitude that Papuan Malay and Indonesian are of equal value is remarkable, given that in diglossic communities speakers usually consider the \textsc{high} variety to be superior. The \textsc{low} variety, by contrast, is usually held  ``to be inferior, even to the point that its existence is denied'' \citep[36]{Fasold.1984}.


The author’s own observations agree with the survey findings that Papuans find Papuan Malay suitable for intimate communication, while they feel at a distance with Indonesian. Many Papuan Malay speakers she met referred to their speech variety as \textitbf{logat Papua} ‘Papuan speech variety’, a name that like \textitbf{bahasa tanah} ‘home language’ indicates a strong, indigenous identification with their language.


At the same time, though, it is questioned here to what extent Papuans feel at ease with Papuan Malay and how positive their attitudes really are. While most of the 2007 interviewees said that Papuan Malay and Indonesian are of equal value, the same interviewees also stated that Indonesian was more appropriate in certain domains. Besides, the author’s own observations suggest that Papuans also consider Indonesian to be more appropriate for certain topics and with certain interlocutors. These language behaviors suggest that language attitudes toward Papuan Malay are somewhat am\isi{bivalent} as far as formal domains are concerned.



A  ``low level of correlation between attitudes and actual behavior'' is not unusual, though, as scholars such as \citet[140]{Agheyisi.1970} point out (see also \citealt[10]{Cooper.1974}; \citealt[16]{Baker.1992}). As for Papuan Malay, the observed mismatch can perhaps be accounted for in terms of \citegen{Kelman.1971} distinction of sentimental and instrumental attachments. Applying this distinction, one can say that Papuans are  ``sentimentally attached'' to Papuan Malay but  ``instrumentally attached'' to Indonesian. Papuan Malay is associated with sentimental attachments, in that it makes Papuans feel good about being Papuan. Indonesian, by contrast, is associated with instrumental attachments in that it allows them to achieve social status and their education and to get things done \citep[25]{Kelman.1971}.


\newpage
In this context, the attitudes which Indonesians and Indonesian institutions express toward Papuan Malay are also important. Overall, it seems that Indonesians who live in West Papua but do not speak Papuan Malay consider the language to be poor or bad Indonesian \citep[19]{Scott.2008}. In West Papua, this view is implicitly communicated by Indonesian government institutions, for instance by hanging banners across major roads which demand \textstyleChItalic{mari kita berbicara \ili{bahasa Indonesia} yang baik dan benar} ‘let us speak good and correct Indonesian’. Such negative language attitudes are widespread and at times rather demeaning. Moreover, they are not only directed towards the language but also towards its speakers. \citet[94]{King.2002}, for instance, reports that Indonesians in Papua consider Papuans to be stupid and backwards:  ``‘Papua bodoh’ – stupid Papuans; backward Papuans''. Moreover, negative attitudes towards Papuan Malay, and the \ili{eastern Malay varieties} in general, are also found among Indonesian academics. \citet[106]{Masinambow.2002}, for example, report that scholars in Indonesia continue to regard the \ili{eastern Malay varieties} as second-class, mixed languages which are opposed by the pure \ili{High Malay} language.\footnote{\citet{Masinambow.2002} uses  ``\ili{High Malay}'' as a cover term which also includes \ili{Standard Indonesian}.}

(For a discussion of Indonesian language planning see \citealt[114–143]{Sneddon.2003}; for a discussion of the role of Papuan Malay in the context of Indonesian language politics see \citealt[13–17]{Besier.2012}.)

Hence, the need for Papuans to distinguish between sentimental and instrumental attitudes is confounded by the negative attitudes which Indonesian institutions and individuals have toward Papuan Malay.



Notably, Papuan Malay is not recognized by the Papuan independence movement OPM (\textstyleChItalic{Organisasi Papua Merdeka} – ‘Free Papua Movement’) either.


The First Papuan People’s Congress, held on 16--19 October 1961, issued a manifesto which declared that \textstyleChItalic{Papua Barat} ‘West Papua’ would be the name of their nation, \textstyleChItalic{Papua} the name of the people, \textstyleChItalic{Hai Tanahku Papua} ‘My land Papua’ the national anthem, the \textstyleChItalic{Bintang Kejora} ‘Morning Star’ the national flag, the \textstyleChItalic{burung Mambruk} ‘Mambruk bird’ the national symbol, and \textstyleChItalic{Satu Rakyat dan Satu Jiwa} ‘One People One Soul’ the national motto. Moreover, the Congress decided that the national language should not be Malay, as it was the colonizer’s language \citep[40--43]{Alua.2006}. The Second Papuan People’s Congress, held from 29 May until 4 June 2000 at Cenderawasih University in Jayapura, reconfirmed the national anthem, flag, and symbol, and again rejected Papuan Malay as the national language. Instead the Congress decided that English should be the official language. In addition, Papuan Malay and \ili{Tok Pisin} should serve as  ``common'' languages \citep[50]{King.2004}.\footnote{The report in \cite{King.2004} is based on an \textstyleChItalic{Agence France Presse} summary, dated 6 January 2000, which is titled  ``The constitution of the ‘State of Papua’ as envisaged in Jayapura''.}

Likewise, the Third Papuan People’s Congress, held from 17--19 October 2011 in Abepura, rejected Papuan Malay as the national language \citep[19]{Besier.2012}.


\largerpage
This desire of Papuan nationals  ``of a clean linguistic break'' is an utopian dream, however, as {\citet[407]{Rutherford.2005}} points out. Moreover, it presents a dilemma since only few people in West Papua speak these other languages, whereas Papuan Malay is the de facto language of wider communication. (\citealt[See also][17--22]{Besier.2012}.)



The fact that Papuan Malay has not been officially recognized in spite of its large numbers of speakers reflects the lack of esteem held by the main stakeholders vis-à-vis this language, by the Indonesian or OPM stakeholders. (\citealt[See also][-32]{Besier.2012}.)



Another factor to be considered in the context of language attitudes is the issue of language awareness.



The findings of the 2007 sociolinguistic survey indicate a potential lack of language awareness. Papuan interviewees stated that  ``lesser educated [{\ldots} Papuan Malay] speakers would likely be unaware of the differences'' between their language and Indonesian; that is,  ``they would consider the speech form they use to be coincident with standard Indonesian'' \citep[11]{Scott.2008}. Along similar lines, {\citet[76]{Paauw.2009}} reports that many Papuan Malay speakers are not aware of the fact that their speech variety is distinct from Indonesian. (See also \citealt[5--7]{Burung.2008}.)



The author made similar observations during her 2008 fieldwork in \ili{Sarmi}. Many Papuan Malay speakers she met thought that they were speaking Indonesian with a local Papuan flavor when conversing with other Papuans.



This lack of language awareness is not surprising, however, given the negative language attitudes that Papuans experience from the Indonesian government and Indonesian institutions which sanction Indonesian as the only acceptable variety of Malay. Through this  ``ideological erasure'' of Papuan Malay from official quarters, the language has become  ``invisible'', using  \citegen[974]{Gal.1995} terminology. This erasure has led to the perception among many Papuans that Papuan Malay does not exist as a distinct language. \citep[See also][30]{Errington.2001}.



In summarizing this discussion on language attitudes, it is concluded that overall Papuans’ attitudes toward Papuan Malay are positive to somewhat am\isi{bivalent}, rather than wholly positive.


\section{Typological profile of Papuan Malay}
\label{Para_1.6}
This section presents an overview of the \isi{typological profile} of Papuan Malay as described in this book. General typological features of the language are discussed in §\ref{Para_1.6.1}, followed in §\ref{Para_1.6.2} by a comparison of some of its features with those found in \ili{Austronesian} and in \ili{Papuan languages}. In §\ref{Para_1.6.3}, some features of Papuan Malay are compared to those found in other \ili{eastern Malay varieties}.


\subsection{General typological profile}\label{Para_1.6.1}
\largerpage
In presenting the pertinent typological features of Papuan Malay, an overview of its \isi{phonology} is given in §\ref{Para_1.6.1.1}, its \isi{morphology} in §\ref{Para_1.6.1.2}, its word classes in §\ref{Para_1.6.1.3}, and its \isi{basic word order} in §\ref{Para_1.6.1.4}.

\subsubsection[Phonology]{Phonology}\label{Para_1.6.1.1}
Papuan Malay has 18 consonant and five vowel phonemes. The \isi{consonant system} consists of the following phonemes: /\textstyleChCharisSIL{p}, \textstyleChCharisSIL{b}, \textstyleChCharisSIL{t}, \textstyleChCharisSIL{d}, \textstyleChCharisSIL{g}, \textstyleChCharisSIL{k}, \textstyleChCharisSIL{tʃ}, \textstyleChCharisSIL{dʒ}, \textstyleChCharisSIL{s}, \textstyleChCharisSIL{h}, \textstyleChCharisSIL{m}, \textstyleChCharisSIL{n}, \textstyleChCharisSIL{ɲ}, \textstyleChCharisSIL{ŋ}, \textstyleChCharisSIL{r}, \textstyleChCharisSIL{l}, \textstyleChCharisSIL{j}, \textstyleChCharisSIL{w}/. All consonants occur as onsets,\footnote{Velar /\textstyleChCharisSILviiivpt{ŋ}/ however, only occurs in the root-internal and not in the word-initial onset position.} while the range of consonants occurring in the coda position is much smaller. The five vowels are /\textstyleChCharisSIL{i}, \textstyleChCharisSIL{ɛ}, \textstyleChCharisSIL{u}, \textstyleChCharisSIL{ɔ}, \textstyleChCharisSIL{a}/. All five occur in stressed and unstressed, open and closed syllables. A restricted sample of like segments can occur in sequences. Papuan Malay shows a clear preference for disyllabic roots and for CV and CVC syllables; the maximal syllable is CCVC. Stress typically falls on the penultimate syllable. Adding to its 18 native \isi{consonant system}, Papuan Malay has adopted one loan segment, the voiceless labio-dental fricative /\textstyleChCharisSIL{f}/. (\chapref{Para_2})


\subsubsection[Morphology]{Morphology}\label{Para_1.6.1.2}
Papuan Malay is a language near the isolating end of the analytic-synthetic continuum. That is, the language has very little productive \isi{morphology} and words are typically single root morphemes. Inflectional \isi{morphology} is lacking, as nouns and verbs are not marked for any grammatical category such as gender, number, or case. Word formation is limited to the two derivational processes of \isi{reduplication} and \isi{affixation}.



Reduplication is a very productive process. Three types of \isi{lexeme formation} are attested, namely full \isi{reduplication}, which is the most common one, partial and imitative \isi{reduplication}. Usually, content words undergo \isi{reduplication}; \isi{reduplication} of function words is rare. The overall meaning of \isi{reduplication} is  ``a \textsc{higher}/\textsc{lower} \textsc{degree} \textsc{of} \ldots'', employing \citegen[1151]{Kiyomi.2009} terminology. (\chapref{Para_4})



Affixation has very limited productivity. Papuan Malay has two affixes which are somewhat productive. Verbal prefix \textscItal{ter-} ‘\textsc{acl}’ derives \isi{monovalent} verbs from mono- or \isi{bivalent} bases. The derived verbs denote accidental or unintentional actions or events. Nominal suffix \textitbf{-ang} ‘\textsc{pat}’ typically derives nominals from verbal bases. The derived nouns denote the patient or result of the event or state specified by the \isi{verbal base}. In addition, Papuan Malay has one nominal prefix, \textscItal{pe(n)-} ‘\textsc{ag}’, which is, at best, marginally productive. The derived nouns denote the agent or instrument of the event or state specified by the \isi{verbal base}.\footnote{The small caps designate the abstract representation of affixes that have more than one form of realization; prefixes \textscItal{ter-} and \textscItal{pe(n)-} have two allomorphs each, namely \textitbf{ter-} and \textitbf{ta-} (§\ref{Para_3.1.2.1}), and \textitbf{pe(}\textscItal{n}\textitbf{)-} and \textitbf{pa(}\textscItal{n}\textitbf{)-} (small-caps \textscItal{n} represents the different realizations of the nasal) (§\ref{Para_3.1.4.1}), respectively.} (§\ref{Para_3.1}, in \chapref{Para_3})

Compounding is a third \isi{word-formation} process. Its degree of productivity remains uncertain, though, as the demarcation between compounds and phrasal expressions is unclear. (§\ref{Para_3.2}, in \chapref{Para_3})



Papuan Malay has no morphologically marked passive voice. Instead, speakers prefer to encode actions and events in active constructions. An initial survey of the corpus shows that speakers can use an analytical construction to signal that the undergoer is adversely affected. This construction is formed with \isi{bivalent} \textitbf{dapat} ‘get’ or \textitbf{kena} ‘hit’, as in \textitbf{dapat pukul} ‘get hit’ or \textitbf{kena hujang} ‘hit (by) rain’.\footnote{In this book, Papuan Malay strategies to express passive voice are not further discussed; instead, this topic is left for future research.}\\


\subsubsection[Word classes]{Word classes}\label{Para_1.6.1.3}
The open word classes in Papuan Malay are nouns, verbs, and adverbs. The major closed word classes are personal pronouns, interrogatives, demonstratives, locatives, numerals, quantifiers, prepositions, and conjunctions. The distinguishing criteria for these classes are their syntactic properties, given the lack of inflectional \isi{morphology} and the limited productivity of derivational patterns. A number of categories display membership overlap, most of which involves verbs. This includes overlap between verbs and nouns as is typical of Malay languages and other \ili{Austronesian} languages of the larger region.



One major distinction between nouns and verbs is that nouns cannot be negated with \textitbf{tida}/\textitbf{tra} ‘\textsc{neg}’ (§\ref{Para_5.2} and §\ref{Para_5.3}, in \chapref{Para_5}). In his discussion of pertinent typological characteristics of  ``western \ili{Austronesian}'' languages,\footnote{\citet[111]{Himmelmann.2005} employs the term  ``western \ili{Austronesian}'' as a  ``rather loose geographical expression''; it is  ``strictly equivalent to \textit{non-\ili{Oceanic} \ili{Austronesian} languages}''.} \citet[128]{Himmelmann.2005} points out that  ``in languages where negators provide a diagnostic context for distinguishing nouns and verbs, putative adjectives always behave like verbs''. This also applies to Papuan Malay, in that the semantic types usually associated with adjectives are encoded by \isi{monovalent} stative verbs. Verbs are divided into \isi{monovalent} stative, \isi{monovalent} dynamic, \isi{bivalent}, and \isi{trivalent} verbs. A number of adverbs are derived from \isi{monovalent} stative verbs (§\ref{Para_5.14}, in \chapref{Para_5}). Personal pronouns, demonstratives, and locatives are distinct from nouns in that all four of them can modify nouns, while nouns do not modify the former (\chapref{Para_5}).


\subsubsection[Basic word order]{Basic word order}\label{Para_1.6.1.4}
Papuan Malay has a basic SVO word order, as is typical of western \ili{Austronesian} languages (\citealt[141–144]{Himmelmann.2005}; see also \citealt[355–359]{Donohue.2007c}). This VO order is shown in (\ref{Example_1.3}). Very commonly, however, arguments are omitted if the identity of their referent was established earlier. This is the case with the omitted subject \textitbf{tong} ‘\textsc{1pl}’ in the second clause and the direct object \textitbf{bua} ‘fruit’ in the third clause. An initial survey of the corpus also shows that topicalized constituents are always fronted to the clause initial position, such as the direct object \textitbf{bapa desa pu motor itu} ‘that motorbike of the mayor’ in (\ref{Example_1.4}).\footnote{{\citet[433]{Donohue.2011}} suggests that the frequent topicalization of non-subject arguments  ``is an adaptive strategy that allows the OV order of the substrate languages in New Guinea [\ldots] to surface in what is nominally a VO language, Papuan Malay''.\\
In this book the issue of topicalization is not further discussed; instead, this topic is left for future research.}


\begin{styleExampleTitle}
{Word order: Basic SVO order, \isi{elision} of core arguments, and fronting of topicalized arguments}
\end{styleExampleTitle}
\ea
\label{Example_1.3}
\gll tong \bluebold{liat} bua, Ø \bluebold{liat} bua dang tong \bluebold{mulay} \bluebold{tendang{\Tilde{}}tendang} Ø\\ %
 \textsc{1pl}  see fruit { } see fruit and \textsc{1pl} start  \textsc{rdp}{\Tilde}kick \\

\glt 
‘we \bluebold{saw} a fruit, (we) \bluebold{saw} a fruit and we \bluebold{started kicking} (it)’ \textstyleExampleSource{[081006-014-Cv.0001]}
\z

\ea
\label{Example_1.4}
\gll \bluebold{bapa} \bluebold{desa} \bluebold{pu} \bluebold{motor} \bluebold{itu} Hurki de ada\\ %
 father village \textsc{poss} motorbike \textsc{d.dist} Hurki \textsc{3sg} exist\\
\gll taru \bluebold{Ø} di Niwerawar\\
 put { } at Niwerawar\\
\glt  ‘(as for) \bluebold{that motorbike of the mayor}, Hurki is storing (\bluebold{it}) at Niwerawar’ \textstyleExampleSource{[081014-003-Cv.0024]}
\z

A Papuan Malay \isi{verb} takes maximally three arguments, that is, the subject and two objects, namely a recipient-like R argument and a theme-like T argument. In double object constructions with \isi{trivalent} verbs, the typical word order is ‘\textsc{subject} – \textsc{verb} – R – T’. However, \isi{trivalent} verbs do not require, but allow three syntactic arguments. Most often, speakers use alternative strategies to reduce the number of arguments. (§\ref{Para_11.1.3}, in \chapref{Para_11})



As is typical cross-linguistically, the SVO word order correlates with a number of other word order characteristics, as discussed in \citet{Dryer.2007c}.



Papuan Malay word order agrees with the predicted word order with respect to the order of \isi{verb} and adposition, \isi{verb} and adpositional phrase, main \isi{verb} and auxiliary \isi{verb}, mark and standard, parameter and standard, clause and \isi{complementizer}, and head nominal and relative clause. In two aspects, the word order differs from the predicted order. In adnominal possessive constructions, the possessor precedes rather than follows the possessum, and in \isi{interrogative} clauses, the question marker is clause-final rather than clause-initial. Six word order correlations do not apply to Papuan Malay. The word order of \isi{verb} and manner ad\isi{verb}, of copula and predicate, and of article or plural word and \isi{noun} are nonapplicable, as Papuan Malay does not have manner adverbs, a copula, an article, and a plural word. Nor does the order of main and subordinate clause and the position of adverbial subordinators apply, as in combining clauses Papuan Malay does not make a morphosyntactic distinction between main and subordinate clause (see \tabref{Table_1.3}).


\begin{table}
\caption{Predicted word order for VO languages {\citep[130]{Dryer.2007}} versus Papuan Malay word order\label{Table_1.3}}


\begin{tabular}{lll}
\lsptoprule
 \multicolumn{1}{c}{Predicted word order} & Papuan Malay word order &  Examples\\
\midrule
prepositions & as predicted & (\ref{Example_1.5}), (\ref{Example_1.6})\\
\tablevspace
\isi{verb} – \isi{prepositional phrase} & as predicted & (\ref{Example_1.5}), (\ref{Example_1.6})\\
\tablevspace
auxiliary \isi{verb} – main \isi{verb} & as predicted & (\ref{Example_1.5}),\\
\tablevspace
mark – standard\tablefootnote{{\citet[130]{Dryer.2007}} uses the term  ``marker'' rather than  ``mark''. The terminology for comparative constructions employed in this book, however, follows \citegen{Dixon.2008} \tablevspace
terminology; hence,  ``mark'' rather than  ``marker'' (see §\ref{Para_11.5}).}
 & as predicted & (\ref{Example_1.7}), (\ref{Example_1.8})\\
\tablevspace
parameter – standard & as predicted & \REF{Example_1.7}, \REF{Example_1.8}\\
\tablevspace
initial \isi{complementizer} & as predicted & (\ref{Example_1.9})\\
\tablevspace
\isi{noun} – relative clause & as predicted & (\ref{Example_1.10})\\
\tablevspace
\isi{noun} – genitive & \textsc{possessor} \textsc{lig} \textsc{possessum} & (\ref{Example_1.11})\\
\tablevspace
initial question particle & clause final question & (\ref{Example_1.12})\\
\tablevspace
\isi{verb} – manner ad\isi{verb} & nonapplicable & \\
\tablevspace
copula – predicate & nonapplicable & \\
\tablevspace
article – \isi{noun} & nonapplicable & \\
\tablevspace
plural word – \isi{noun} & nonapplicable & \\
\tablevspace
main clause – subordinate clause & nonapplicable & \\
\tablevspace
initial adverbial subordinator & nonapplicable & \\
\lspbottomrule
\end{tabular}
\end{table}


Papuan Malay has prepositions, with the \isi{prepositional phrase} following the \isi{verb}, as illustrated in (\ref{Example_1.5}) and (\ref{Example_1.6}); auxiliary verbs precede the main \isi{verb} as shown in (\ref{Example_1.5}) (§\ref{Para_13.3}, in \chapref{Para_13}\footnote{Auxiliary verbs are briefly mentioned in §\ref{Para_13.3}, in \chapref{Para_13}; a detailed description of these verbs is left for future research.})  \citep[see also][373–379]{Donohue.2007}. The example in (\ref{Example_1.6}) shows that aspect-marking adverbs also precede the \isi{verb} (§\ref{Para_5.4.1}, in \chapref{Para_5}); cross-linguistically, however, the order of aspect marker and \isi{verb} does not correlate with the order of \isi{verb} and object \citep[130]{Dryer.2007}.



\begin{styleExampleTitle}
{Word order: Auxiliary \isi{verb} – main \isi{verb} – prepositional phrase}
\end{styleExampleTitle}
\ea
\label{Example_1.5}
\gll {ko} {\bluebold{harus}} {pulang} {\bluebold{ke}} {\bluebold{tempat}}\\ %
  \textsc{2sg} have.to  go.home to place\\

\glt 
‘you \bluebold{have to} go home \bluebold{to (your own) place}’ \textstyleExampleSource{[080922-010a-CvNF.0143]}
\z

\ea
\label{Example_1.6}
\gll {de} {\bluebold{suda}} {naik} {\bluebold{di}} {\bluebold{kapal}}\\ %
 \textsc{3sg} already ascend at ship\\


\glt 
‘he \bluebold{already} went \bluebold{on board}’ \textstyleExampleSource{[080923-015-CvEx.0025]}
\z


In Papuan Malay comparison clauses, the parameter precedes the mark, both of which precede the standard, as in (\ref{Example_1.7}) and (\ref{Example_1.8}). The position of the index differs depending on the type of comparison clause. In \isi{degree-marking} clauses the parameter follows the index, as in the superlative clause in (\ref{Example_1.7}). In \isi{identity-marking} clauses, by contrast, the parameter precedes the index as in the \isi{similarity} clause in (\ref{Example_1.8}), or it is omitted. The word-order of index and parameter, however, does not correlate with that of \isi{verb} and object \citep[130]{Dryer.2007}. (§\ref{Para_11.5}, in \chapref{Para_11})


\begin{styleExampleTitle}
{Word order: \textsc{parameter} – \textsc{mark} – \textsc{standard}}
\end{styleExampleTitle}
\ea
\label{Example_1.7}
\glll {\textupsc{comparee}} {\textupsc{index}} {\textupsc{parameter}} {\textupsc{mark}} {\textupsc{standard}}\\ %
 {dia}  {\bluebold{lebi}}  {\bluebold{tinggi}}  {dari}  {saya}\\
 {\textsc{3sg}}  {more}  {be.high}  {from}  {\textsc{1sg}}\\
\glt 
‘he/she is \bluebold{taller} than me’ (Lit. ‘be \bluebold{more tall} from me’) \textstyleExampleSource{[Elicited BR111011.002]}
\z

 
\ea
\label{Example_1.8}
\glll {\textupsc{comparee}} {\textupsc{parameter}} {\textupsc{index}} {\textupsc{mark}} {\textupsc{standard}}\\ %
 de  \bluebold{sombong}  \bluebold{sama}  deng  ko\\
 \textsc{3sg}  be.arrogant  be.same  with  \textsc{2sg}\\
\glt 
‘she’ll be \bluebold{as arrogant as} you (are)’ (Lit. ‘be \bluebold{arrogant same} with you’) \textstyleExampleSource{[081006-005-Cv.0002]}
\z


The \isi{complementizer} \textitbf{bahwa} ‘that’ occurs in clause-initial position, with the complement clause following the \isi{verb}, as in (\ref{Example_1.9}). (§\ref{Para_14.3.1}, in \chapref{Para_14})

\begin{styleExampleTitle}
Word order: Initial complementize
\end{styleExampleTitle}

\ea
\label{Example_1.9}
\gll {sa} {tida} {taw} {\bluebold{bahwa}} {jam} {tiga} {itu} {de} {su} {meninggal}\\ %
\textsc{1sg} \textsc{neg} know that hour three \textsc{d.dist} \textsc{3sg} already die\\

\glt 
‘I didn’t know \bluebold{that} by three o’clock (in the afternoon) she had already died’ \textstyleExampleSource{[080917-001-CvNP.0005]}
\z


Within the \isi{noun} phrase, the relative clause follows its head nominal, as shown in (\ref{Example_1.10}) (§\ref{Para_8.2.8}, in \chapref{Para_8}). Other modifiers, such as demonstratives, or \isi{monovalent} stative verbs, also occur to the right of the head nominal. This order of head nominal and modifier is typical for western \ili{Austronesian} languages (\citealt[142]{Himmelmann.2005}; see also \citealt[359–373]{Donohue.2007c}). Cross-linguistically, however, the order of head nominal and \isi{demonstrative}, \isi{numeral}, or stative \isi{verb} does not correlate with the order of \isi{verb} and object \citep[130]{Dryer.2007}. Numerals and quantifiers precede or follow the head nominal, depending on the semantics of the phrasal structure (§\ref{Para_8.3}, in \chapref{Para_8}).



\begin{styleExampleTitle}
{Word order: Head nominal – relative clause}\label{Word_order_head_nominal}
\end{styleExampleTitle}
\ea
\label{Example_1.10}
\gll {\ldots} {karna} {liat} {ada} {makangang} {dalam} {\bluebold{kantong}} {\bluebold{yang}} {saya} {bawa}\\ %
 { }  because see exist food inside bag \textsc{rel} \textsc{1sg} bring\\
\glt 
‘[she was already glad] because (she) saw there was food in \bluebold{the bag that} I brought’ \textstyleExampleSource{[080919-004-NP.0032]}
\z


Likewise in \isi{noun} phrases with adnominally used nouns, the modifier \isi{noun} follows the head nominal, as in \textitbf{tulang bahu} ‘shoulder bone’ (§\ref{Para_8.2.2}, in \chapref{Para_8}). By contrast, \isi{adnominal possession} in Papuan Malay is typically expressed with a construction in which the \textsc{possessor} precedes the \textsc{possessum}; both are linked with the possessive marker \textitbf{pu(nya)} ‘\textsc{poss}’, as illustrated in (\ref{Example_1.11}) (\chapref{Para_9}). This word order does not correlate with the general VO order, but it is typical for the \ili{eastern Malay varieties} in general and other \ili{Austronesian} languages of the larger region, as discussed in more detail in §\ref{Para_1.6.2}.

\begin{styleExampleTitle}
Word order: \textsc{possessor} – \textsc{possessum}
\end{styleExampleTitle}
\ea
\label{Example_1.11}
\gll {\ldots} {sa} {pegang} {\bluebold{sa}} {pu} {parang} {\bluebold{sa}} {punya} {jubi} {\ldots}\\ %
 { } \textsc{1sg} hold \textsc{1sg} \textsc{poss} short.machete \textsc{1sg} \textsc{poss} bow.and.arrow  \\

\glt 
‘[so, in the morning I got up, I fed the dogs,] I took \bluebold{my} short machete, \bluebold{my} bow and arrows \ldots’ \textstyleExampleSource{[080919-003-NP.0003]}
\z


In alternative \isi{interrogative} clauses, the question marker occurs in clause-final position. Such questions are formed with the alternative-marking \isi{conjunction} \textitbf{ka} ‘or’ which is also used to mark \isi{interrogative} clauses, as demonstrated in (\ref{Example_1.12}) (§\ref{Para_13.2.3}, in \chapref{Para_13}; see also §\ref{Para_14.2.2.2}, \chapref{Para_14}). Again, this word order does not correlate with the general VO order.



\begin{styleExampleTitle}
{Word order: Clause-final question marker \textitbf{ka} ‘or’}
\end{styleExampleTitle}
\ea
\label{Example_1.12}
\gll {ko} {sendiri} {\bluebold{ka}?}\\ %
\textsc{2sg}  be.alone or\\

\glt 
‘are you alone \bluebold{or (not)}?’ \textstyleExampleSource{[080921-010-Cv.0003]}
\z


As mentioned, in a number of aspects the predicted word order does not apply to Papuan Malay. Papuan Malay has no manner adverbs. Instead \isi{monovalent} stative verbs express manner; they take a postpredicate position (§\ref{Para_5.4.8}, in \chapref{Para_5}). The language has no copula either. Hence, in nonverbal predicate clauses, the nonverbal predicate is juxtaposed to the subject (\chapref{Para_12}). Neither does Papuan Malay have an article or plural word. Instead, free personal pronouns signal the person, number, and definiteness of their referents (\chapref{Para_6}). In combining clauses, Papuan Malay makes no morphosyntactic distinction between main and subordinate clauses; dependency relations are purely semantic (§\ref{Para_14.2}, in \chapref{Para_14}).



In negative clauses, the negators occur in prepredicate position: \textitbf{tida}/\textitbf{tra} ‘\textsc{neg}’ negates verbal, existential, and nonverbal prepositional clauses, while \textitbf{bukang} ‘\textsc{neg}’ negates nonverbal clauses, other than prepositional ones; besides, \textitbf{bukang} ‘\textsc{neg}’ also marks contrastive \isi{negation} (§\ref{Para_13.1}, in \chapref{Para_13}). This negator-predicate order is typical for western \ili{Austronesian} languages \citep[141]{Himmelmann.2005}. Cross-linguistically, however, it does not correlate with the order of \isi{verb} and object \citep[130]{Dryer.2007c}.


\subsection{Papuan Malay as a language of the Papuan contact zone}\label{Para_1.6.2}
In this section, some of the typological features of Papuan Malay are compared to pertinent features found in \ili{Austronesian} languages in general, as well as to features typical for \ili{Austronesian} languages spoken in the larger region, and to some features of \ili{Papuan languages}.\footnote{\label{Footnote_1.24}The term  ``Papuan'' is a collective label used for  ``the non-\ili{Austronesian} languages spoken in New Guinea and archipelagos to the West and East''; that is, the term  ``does not refer to a superordinate category to which all the languages belong'' \citep[107]{Klamer.2008}.}



The reason for this investigation is the observation that Papuan Malay is lacking some of the features typical for \ili{Austronesian} languages, while it has a number of features which are found in \ili{Papuan languages}. This investigation is not based on a comparative study, which would explore whether and to what extent Papuan Malay, as spoken in \ili{Sarmi} on West Papua’s northeast coast, has adopted features found in the languages of the larger region, such as \ili{Isirawa}, a \ili{Tor}-\ili{Kwerba} language and the language of the author’s hosts, or the \ili{Tor}-\ili{Kwerba} languages \ili{Kwesten} and \ili{Samarokena}, or the \ili{Austronesian} languages \ili{Mo} and \ili{Sobei}. Such a study is left for future research. (See also \tabref{Table_1.1} in §\ref{Para_1.4}.)



Instead this investigation is based on studies on areal diffusion. For a long time, scholars have noted that in the area east of Sulawesi, Sumba, and Flores, all the way to the Bird’s Head of New Guinea, a number of linguistic features have diffused from Papuan into \ili{Austronesian} languages and vice versa.



{\citet{Klamer.2008}} and  \cite{Klamer.2010} propose the term  ``East Nusantara'' for this area. More specifically, \citet[1]{Klamer.2010} define\footnote{As \citet[1]{Klamer.2010} point out, though, there is an ongoing discussion about  ``the exact geographic delimitations of the East Nusantara region'' and  ``whether (parts of) New Guinea are also considered to be part of it'' (see also Footnote 3 in \citealt[1]{Klamer.2010}).}

\begin{quote}
East Nusantara as a geographical area that extends from Sumbawa to the west, across the islands of East Nusa Tenggara, Maluku [\ldots] including Halm\-ahera, and to the Bird’s Head of New Guinea in the east [\ldots]. In the northwest, the area is bounded by Sulawesi.
\end{quote}



According to the above definition, only parts of West Papua belong to East Nusantara, namely the Bird’s Head but not West Papua’s north coast. Yet, it seems useful to examine the \isi{typological profile} of Papuan Malay in light of the observed diffusion of linguistic features, discussed in \citet{Klamer.2008} and  \citet{Klamer.2010}.



This comparison shows that Papuan Malay is lacking some of the features which are typical for \ili{Austronesian} languages. At the same time, it has a number of features which are untypical for \ili{Austronesian} languages, but which are found in \ili{Austronesian} languages of East Nusantara. Moreover, Papuan Malay has some features not typically found in \ili{Austronesian} languages of East Nusantara but found in \ili{Papuan languages}. These features are summarized in \tabref{Table_1.4} to \tabref{Table_1.6}; the listed features are taken from \citet{Klamer.2008} and \citet{Klamer.2010}, unless mentioned otherwise.


\largerpage
\tabref{Table_1.4} presents seven features found in \ili{Austronesian} languages in general, six of which are listed in {\citet[113]{Klamer.2008}}.\footnote{The noun-genitive order is not explicitly mentioned in {\citet{Klamer.2008}}.} Papuan Malay shares five of these features. It does not, however, share the typical noun-genitive order which is used to express \isi{adnominal possession}. Papuan Malay \isi{noun} phrases with posthead nominal modifiers are used to denote important features for subclassification of the head nominal rather than for \isi{adnominal possession} (§\ref{Para_8.2.2}, in \chapref{Para_8}). Also, Papuan Malay does not distinguish between inclusive and exclusive first person plural in its pronominal paradigm.


\begin{table}
\caption{Pertinent features of Austronesian languages in general vis-à-vis Papuan Malay features\label{Table_1.4}}
\begin{tabular}{lll}
\lsptoprule
 \multicolumn{1}{c}{\ili{Austronesian} languages} & \multicolumn{2}{c}{ Papuan Malay}\\
\midrule

Phonemic l/r distinction & yes & (Chap. \ref{Para_2})\\
 \tablevspace
Preference for CVCV roots & yes & (Chap. \ref{Para_2})\\
 \tablevspace
Reduplication & yes & (Chap. \ref{Para_4})\\
 \tablevspace
Head-initial & yes & (Chap. \ref{Para_8})\\
 \tablevspace
Negator precedes the predicate & yes & (Chap. \ref{Para_13})\\
 \tablevspace
Noun-genitive order & no & (Chap. \ref{Para_8} \& \ref{Para_9})\\
 \tablevspace
Inclusive-exclusive distinction in personal pronouns & no & (Chap. \ref{Para_5} \& \ref{Para_6})\\
\lspbottomrule
\end{tabular}
\end{table}

% \largerpage[-2] %long distance

\newpage 
\tabref{Table_1.5} lists 17 linguistic features  ``found in many of the \ili{Austronesian} languages of East Nusantara'' {\citep[10]{Klamer.2010};\footnote{This list of features in \citet{Klamer.2010} builds on \citet{Klamer.2002}, {\citet{Himmelmann.2005}}, {\citet{Donohue.2007c}}, and {\citet{Klamer.2008}}.} some of these features are also listed in   \tabref{Table_1.4}. Papuan Malay shares eight of them, such as the preference for CVCV roots or the lack of a productive voice system on verbs. Another eight features, however, are unattested in the corpus, such as metathesis or clause-final negators.
%\setcounter{TempFootnote}{\value{footnote}}
%\begin{savenotes}

\begin{table} 
\caption{Pertinent features of Austronesian languages of East Nusantara vis-à-vis Papuan Malay features\label{Table_1.5}}
\begin{minipage}{\textwidth}
%\todo[inline]{add chaprefs in this table}
\begin{tabularx}{\textwidth}{Xll}
\lsptoprule
\multicolumn{1}{c}{\ili{Austronesian} languages of East Nusantara} & \multicolumn{2}{c}{ Papuan Malay}\\
\midrule
\multicolumn{3}{l}{Phonology}\\
\midrule
 Preference for CVCV roots & yes & (Chap. \ref{Para_2})\\
 Prenasalized consonants & no & (Chap. \ref{Para_2})\\
 Metathesis & no & (Chap. \ref{Para_2})\\
\midrule
\multicolumn{3}{l}{Morphology}\\\midrule
 No productive voice system on verbs & yes & (Chap. \ref{Para_3} \& \ref{Para_5})\\
 Left-headed compounds\footnote{In Papuan Malay the demarcation between compounds and phrasal expressions is unclear, however. Hence, it remains uncertain to what degree \isi{compounding} is a productive
  process. (For more details see §\ref{Para_3.2}.)}
  & yes & (Chap. \ref{Para_3})\\
 Agent/subject indexed on \isi{verb} as prefix/proclitic & no & (Chap. \ref{Para_3} \& \ref{Para_5})\\
Inclusive-exclusive distinction in personal pronouns & no & (Chap. \ref{Para_5} \& \ref{Para_6})\\
 Morphological distinction between alienable and inalienable nouns & no & (Chap. \ref{Para_3} \& \ref{Para_5})\\
\midrule
\multicolumn{3}{l}{Syntax}\\
\midrule
 Verb-object order & yes & (Chap. \ref{Para_11})\\
 Prepositions & yes & (Chap. \ref{Para_10})\\
 Genitive-\isi{noun} order ( ``preposed possessor'') & yes & (Chap. \ref{Para_8} \& \ref{Para_9})\\
 Noun-Numeral order & yes & (Chap. \ref{Para_8})\\
 Absence of a passive construction & yes & (Chap. \ref{Para_11})\\
 Clause-final negators & no & (Chap. \ref{Para_13})\\
 Clause-initial indigenous complementizers\footnote{The Papuan Malay \isi{complementizer} is \textitbf{bahwa} ‘that’. According to \citet{Jones.2007} it originates from \ili{Sanskrit}.} 
 & no & (Chap. \ref{Para_14})\\
 Formally marked adverbial/complement clauses & no & (Chap. \ref{Para_14})\\
\midrule
\multicolumn{3}{l}{Other}\\\midrule
 Parallelisms without stylistic optionality & \multicolumn{2}{l}{not yet researched}\\
\lspbottomrule

\end{tabularx}

\end{minipage}
\end{table}
%\end{savenotes}




Two of the nonshared morphological and two of the shared syntactic features require additional commenting, that is, indexing on the \isi{verb}, the distinction between alienable and inalienable nouns, the noun-\isi{numeral} order, and the absence of a passive construction.

 
Papuan Malay does not have indexing on the \isi{verb}. Instead, Papuan Malay uses free personal pronouns. (\chapref{Para_6})



Overall, Papuan Malay does not distinguish between alienable and inalienable possessed items, with one exception: adnominal possessive constructions with omitted possessive marker signal inalienable possession of body parts or kinship relations. Just as commonly, however, inalienable possession of these entities is encoded in the same way as possession of alienable items, that is, in a \textsc{possessor} \textsc{ligature} \textsc{possessum} construction. Examples are \textitbf{sa maytua} ‘my wife’, \textitbf{dia pu maytua} ‘his wife’, or \textitbf{sa pu motor} ‘my motorbike’ (literally ‘\textsc{1sg} wife’, ‘\textsc{3sg} \textsc{poss} wife’, ‘\textsc{1sg} \textsc{poss} motorbike’). (\chapref{Para_9})



In Papuan Malay \isi{noun} phrases, numerals and quantifiers follow the head nominal. As mentioned in §\ref{Para_1.6.1}, however, they can also precede the head nominal, depending on the semantics of the phrasal structure. (§\ref{Para_8.3}, in \chapref{Para_8})


\newpage 
Like other East Nusantara \ili{Austronesian} languages, Papuan Malay does not have a dedicated passive construction. Instead, speakers encode actions and events in active constructions (see also §\ref{Para_1.6.1.2}).\footnote{As mentioned in §\ref{Para_1.6.1.2}, passive constructions are not further discussed in this book; instead, this topic is left for future research.}

 
 
East Nusantara \ili{Austronesian} languages also often make use of parallelisms without stylistic optionality.\footnote{\citet[370, 371]{Klamer.2002} defines ‘Parallelisms without stylistic optionality’ as follows:  ``Many languages in Eastern Indonesia employ the verbal art form of parallelism [{\ldots} It] is a structurally defined verbal art form that functions as a stylistic device in the ritual language [\ldots] In parallelism, semantically synonymic words or phrases are combined in (minimally two) parallel utterances. [\ldots] Though parallelism is a property of oral literature, it is not purely stylistic: the pairings are obligatory; there is generally no stylistic optionality involved in the choice of a proper pair.''} Whether, and to what extent, Papuan Malay employs this feature has not been researched for the present study; instead this topic is left for future research.

 
Papuan Malay also has a number of features which are not usually found in the East Nusantara \ili{Austronesian} languages. Instead, these features are typical characteristics of \ili{Papuan languages}.



\begin{table} 
\caption{Pertinent features of Papuan languages vis-à-vis Papuan Malay features\label{Table_1.6}}
\begin{tabularx}{\textwidth}{Xll}
\lsptoprule
\multicolumn{1}{c}{Papuan languages} & \multicolumn{2}{c}{Papuan Malay}\\
\midrule
\multicolumn{3}{l}{Phonology}\\\midrule
No phonemic l/r distinction & no & (Chap. \ref{Para_2})\\
\midrule
 \multicolumn{3}{l}{Morphology}\\\midrule
No inclusive-exclusive distinction in personal pronouns & yes & (Chap. \ref{Para_5}\& \ref{Para_6})\\
 Marking of gender & no & (Chap. \ref{Para_3} \& \ref{Para_5})\\
 Subject marked as suffix on \isi{verb} & no & (Chap. \ref{Para_3} \& \ref{Para_5})\\
 Morphological distinction between alienable and inalienable nouns & no & (Chap. \ref{Para_3} \& \ref{Para_5})\\
\midrule
 \multicolumn{3}{l}{Syntax}\\\midrule
Subject-\isi{verb} order & yes & (Chap. \ref{Para_11})\\
 Genitive-\isi{noun} order ( ``preposed possessor'') & yes & (Chap. \ref{Para_8} \& \ref{Para_9})\\
 Serial \isi{verb} constructions\footnote{Serial \isi{verb} constructions are briefly mentioned in §\ref{Para_11.2}, in \chapref{Para_11}; a detailed description of this topic is left for future research.}
 & yes & (Chap. \ref{Para_11})\\
 Clause-chaining & yes & (example (\ref{Example_1.13}))\\
 Tail-head linkage & yes & (example (\ref{Example_1.13}))\\
 Clause-final conjunctions & few & (Chap. \ref{Para_14})\\
 Object-\isi{verb} order & no & (Chap. \ref{Para_11})\\
 Postpositions & no & (Chap. \ref{Para_10})\\
 Clause-final negator & no & (Chap. \ref{Para_13})\\
 Switch reference & no & (Chap. \ref{Para_14})\\
\lspbottomrule
\end{tabularx} 
\end{table}

\tabref{Table_1.6} presents 15 linguistic features typically found in \ili{Papuan languages} \citep[10]{Klamer.2010}.\footnote{This list of features in \citet{Klamer.2010} builds on Foley (\citeyear*{Foley.1986}; \citeyear*{Foley.2000}), \citet{Pawley.2005b}, and \cite{Aikhenvald.2007b}.\\
Tail-head linkage is not mentioned in \citet{Klamer.2008}. It is, however, a typical Papuan feature (see \citealt[200–201]{Foley.1986}; \citeyear*[390]{Foley.2000}).}
 Papuan Malay shares six of them, such as the subject-\isi{verb} order, or the genitive-\isi{noun} order. There is also limited overlap between Papuan Malay and \ili{Papuan languages} with respect to the position of conjunctions. All Papuan Malay conjunctions are clause-initial, but two of them can also take a clause-final position (\chapref{Para_14}). Eight of the 15 features are not found in Papuan Malay, such as gender marking or postpositions.

Among the syntactic features, there are three that need to be commented on, namely clause-chaining, switch reference, and tail-head linkage.



Clause chaining is not discussed in the present study. An initial survey of the corpus indicates, however, that it is very common in Papuan Malay. One example is given in (\ref{Example_1.13}).



\begin{styleExampleTitle}
{Clause-chaining in Papuan Malay}\label{Clause_chaining}
\end{styleExampleTitle}
\ea
\label{Example_1.13}
\gll {langsung} {\bluebold{sa}} {\bluebold{pegang}} {\bluebold{sa}} {\bluebold{putar}} {\bluebold{sa}} {\bluebold{cari}}\\ %
  immediately \textsc{1sg} hold \textsc{1sg} turn.around \textsc{1sg}  search\\

\glt 
‘immediately \bluebold{I held} (the plate), \bluebold{I turned around}, \bluebold{I looked around}’ \textstyleExampleSource{[081011-005-Cv.0034]}
\z


Following \citet[11]{Klamer.2010}, clause-chaining in \ili{Papuan languages} is often characterized by  ``some concomitant switch reference system''. This, however, does not seem to apply to Papuan Malay. That is, so far dedicated switch-reference devices have not been identified, a finding which contrasts with \citegen{Donohue.2011} observations. \citet[431–432]{Donohue.2011} suggests that the sequential-marking \isi{conjunction} \textitbf{trus} ‘next’  ``is a commonly used connective when there is a same-subject coreference \isi{condition} between clauses'', while the sequential-marking \isi{conjunction} \textitbf{baru} ‘and then’ tends  ``to indicate switch reference''. An initial investigation of the attested \textitbf{trus} ‘next, and then’ and \textitbf{baru} ‘and then’ tokens in the corpus shows, however, that both \isi{conjunction} more often link clauses with a switch in reference, than those with same-subject coreference (§\ref{Para_14.2.3.1} and §\ref{Para_14.2.3.2}, in \chapref{Para_14}). Neither do any of the other conjunctions function as dedicated switch-reference devices.

 
Tail-head linkage is not treated in the present study. This feature denotes a  ``structure in which the final clause of the previous sentence initiates the next sentence, often in a reduced form'' (\citealt[390]{Foley.2000}; see also \citealt{deVries.2005}). An initial survey of the corpus shows, however, that tail-head linkage is very common in Papuan Malay. In the example in (\ref{Example_1.14}), for instance, the speaker repeats part of the first clause at the beginning of the second clause: \textitbf{kasi senter} ‘give a flashlight’.


\begin{styleExampleTitle}
Tail-head linkage in Papuan Malay
\end{styleExampleTitle}
\ea
\label{Example_1.14}
\gll {skarang}  {dong} {\bluebold{kasi}} {dia} {\bluebold{senter},} {\bluebold{kasi}} {\bluebold{senter}} {dong}\\ %
{now} {\textsc{3pl}} give \textsc{3sg} flashlight give flashlight \textsc{3pl}\\
\gll mo {kasi} {pisow}\\
want {give} {knife}\\
\glt
‘now they \bluebold{give} him \bluebold{a flashlight}, (having) \bluebold{given} (him) \bluebold{a flashlight}, they want to give (him) a knife’ \textstyleExampleSource{[081108-003-JR.0002]}
\z


\subsection{Papuan Malay as an eastern Malay variety}\label{Para_1.6.3}
This section compares some of the features found in Papuan Malay to those found in other \ili{eastern Malay varieties}, namely in \ili{Ambon Malay} (AM) (\citealt{vanMinde.1997}), \ili{Banda Malay} (BM) {\citep{Paauw.2009}}, \ili{Kupang Malay} (KM) {\citep{Steinhauer.1983}}, \ili{Larantuka Malay} (LM) \citep{Paauw.2009}, \ili{Manado Malay} (MM) {\citep{Stoel.2005}}, North Moluccan or \ili{Ternate Malay} (NMM/TM) (\citealt{Taylor.1983}; \citealt{Voorhoeve.1983}; \citealt{Litamahuputty.2012}).\footnote{In their contributions, {\citet{Taylor.1983}} and {\citet{Voorhoeve.1983}} label the Malay variety spoken in the northern Moluccas as North Moluccan Malay, while {\citet{Litamahuputty.2012}} uses the term \ili{Ternate Malay} for the same variety in her in-depth grammar. Given that the three studies differ in depth, all three of them are included here, with \citegen{Taylor.1983} and \citegen{Voorhoeve.1983} summarily listed under North Moluccan Malay.}



These comparisons are far from systematic and exhaustive. Instead, they pertain to a limited number of topics as they came up during the analysis and description of the \isi{phonology}, \isi{morphology}, and syntax of Papuan Malay. (A detailed typological study of the \ili{eastern Malay varieties} is \citealt{Paauw.2009}.) The comparisons discussed here touch upon the following phenomena:


\begin{itemize}
\item 
Affixation (§\ref{Para_3.1}, in \chapref{Para_3})

\item 
Reduplication (\chapref{Para_4})

\item 
Adnominal uses of the personal pronouns (§\ref{Para_6.2}, in \chapref{Para_6})

\item 
Existence of diphthongs (§\ref{Para_2.1.2}, in \chapref{Para_2})

\item 
Noncanonical functions of the possessive ligature in adnominal possessive constructions (§\ref{Para_9.3}, in \chapref{Para_9})

\item 
Argument \isi{elision} in verbal clauses (§\ref{Para_11.1}, in \chapref{Para_11})

\item 
Morphosyntactic status of the \isi{reciprocity marker} \textitbf{baku} ‘\textsc{recp}’ (§\ref{Para_11.3}, in \chapref{Para_11})

\item 
Contrastive uses of negator \textitbf{bukang} ‘\textsc{neg}’ (§\ref{Para_13.1.2}, in \chapref{Para_13})
)

\end{itemize}


The remainder of this section gives an overview how Papuan Malay compares to the other \ili{eastern Malay varieties} with respect to these phenomena. (In \tabref{Table_1.7} to \tabref{Table_1.10} empty cells signal that a given feature is not mentioned in the available literature. One reason could be that the respective feature is nonexistent. It is, however, just as likely that such empty cells result from gaps in the available literature.)



Affixation is one area in which Papuan Malay has a number of features which are distinct from those found in other \ili{eastern Malay varieties}. \tabref{Table_1.7} presents three prefixes and one suffix and shows that the Papuan Malay affixes are different both in terms of their form and their degree of productivity. In most of the \ili{eastern Malay varieties}, the three prefixes are realized as \textit{ta-}, \textit{pa(}\textscItal{n}\textitbf{)}-, and \textit{ba-}. By contrast, the Papuan Malay affixes \textscItal{ter-} ‘\textsc{acl}’, \textscItal{pe(n)-} ‘\textsc{ag}’, and \textscItal{ber-} ‘\textsc{vblz}’ are most commonly realized as \textit{ter}-, \textit{pe(}\textscItal{n}\textitbf{)}\textitbf{-}, and \textit{ber}-, respectively; hence, they have more resemblance with the corresponding \ili{Standard Indonesian} affixes.



Papuan Malay prefix \textscItal{ter-} has only limited productivity, while prefix \textscItal{ber-} is unproductive. In the other \ili{eastern Malay varieties}, by contrast, the corresponding prefixes \textit{ta-} and \textit{ba-} are very productive. Papuan Malay prefix \textscItal{pe(n)-} is, at best, marginally productive. In \ili{Manado Malay} \textit{paŋ-} is productive (in addition an unproductive form \textit{pa}\textsc{-} exists). Likewise, in North Moluccan / \ili{Ternate Malay} prefixation with \textit{pang-} is productive \citep[30]{Litamahuputty.2012}.\footnote{\citet[4]{Voorhoeve.1983}, by contrast, suggests that \textit{pa}-  ``is no longer morphologically distinct''.} In \ili{Ambon Malay} the prefix occurs but it is unproductive. The Papuan Malay prefix -\textitbf{ang} has only limited productivity. In \ili{Ambon Malay}, the suffix also occurs but according to \citet[106]{vanMinde.1997} it is difficult to determine whether and to what degree it is productive.


\begin{table}
\caption{Affixation: Form and productivity}\label{Table_1.7}

\begin{tabular}{lllllllll}
\lsptoprule
& PM & AM & BM & KM & LM & MM & {NMM /} & {TM}\\
\midrule
\multicolumn{9}{l}{Prefix \textscItal{ter-}}\\
\midrule
Form & \textscItal{ter-} & \textitbf{ta-} & \textitbf{ta-} & \textitbf{ta-} & \textitbf{tə(r)-} & \textitbf{ta-} & \textitbf{ta-} &  \textitbf{ta-}\\
\textsc{prod} & lim. & yes & yes & yes & yes & yes & yes &  yes\\
\midrule
\multicolumn{9}{l}{Prefix \textscItal{pe(n)-}}\\
\midrule
Form & \textscItal{pe(n)-} & \textitbf{pa(}\textscItal{n}\textitbf{)}- &  &  &  & \textitbf{paŋ-} & \textitbf{pa}\textitbf{-} &  \textitbf{pang-}\\
\textsc{prod} & marg. & no &  &  &  & yes & no &  yes\\
\midrule
\multicolumn{9}{l}{Prefix \textscItal{ber-}}\\
\midrule
Form & \textscItal{ber-} & \textitbf{ba-} & \textitbf{ba-} & \textitbf{ba-} & \textitbf{bə(r)-} & \textitbf{ba-} & \textitbf{ba-} &  \textitbf{ba-}\\
\textsc{prod} & no & yes & yes & yes & yes & yes & yes &  yes\\
\midrule
\multicolumn{9}{l}{Prefix \textitbf{-ang}}\\
\midrule
\textsc{prod} & lim. & ? &  &  &  &  &  & \\
\lspbottomrule
\end{tabular}
\end{table}

Reduplication is another phenomenon in which Papuan Malay displays a number of features which differ from those described for other \ili{eastern Malay varieties} (\chapref{Para_4}). As shown in \tabref{Table_1.8}, Papuan Malay and the other \ili{eastern Malay varieties} employ full \isi{reduplication}. Partial and imitative \isi{reduplication}, however, is only reported for Papuan Malay, \ili{Ambon Malay}, and \ili{Larantuka Malay}. Besides, Papuan Malay shares especially many features with \ili{Ambon Malay} regarding the morpheme types which can undergo full \isi{reduplication} (§\ref{Para_4.3.1}, in \chapref{Para_4}).

 
In general, \isi{reduplication} conveys a wide range of different meaning aspects. These meaning aspects differ with respect to the range of word classes they attract for \isi{reduplication}. Among the \ili{eastern Malay varieties}, the attested meaning aspects in Papuan Malay attract the largest range of different word classes, followed by a medium range of attracted word classes in \ili{Ambon Malay}. In the other \ili{eastern Malay varieties}, by contrast, this range of attracted word classes seems to be much smaller. (§\ref{Para_4.3.2}, in \chapref{Para_4})



In Papuan Malay, the reduplicated items can also undergo  ``\isi{interpretational shift}'' or  ``type coercion''. This feature is also attested in Ambon, Larantuka, Manado, and \ili{Ternate Malay}. Again, Papuan Malay and \ili{Ambon Malay} share pertinent features, in that in both varieties nouns and verbs can undergo \isi{interpretational shift}, while in \ili{Manado Malay} only nouns and in Larantuka and \ili{Ternate Malay} only verbs are affected. (§\ref{Para_4.3.3}, in \chapref{Para_4})



These findings suggest that \isi{reduplication} in Papuan Malay has more in common with \ili{Ambon Malay} than with the other \ili{eastern Malay varieties}.


\begin{table}
\caption{Reduplication}\label{Table_1.8}

\begin{tabular}{llllllll}
\lsptoprule
 & PM & AM & BM & KM & LM &  {NMM /} &  {TM}\\
\midrule

\multicolumn{8}{l}{Type of reduplication}\\
\midrule
Full & yes & yes & yes & yes & yes & yes &  yes\\
Partial & yes & yes &  &  & yes &  & \\
Imitative & yes & yes &  &  & yes &  & \\
\midrule
\multicolumn{8}{l}{Meaning aspects and range of attracted word classes}\\
\midrule
Range & large & med. & small & small & small & small &  small\\
\midrule
\multicolumn{8}{l}{Interpretational shift of reduplicated lexemes}\\
\midrule
Shift & yes & yes &  &  & yes &  &  yes\\
\lspbottomrule
\end{tabular}
\end{table}

\largerpage
Papuan Malay is also distinct from other \ili{eastern Malay varieties} with respect to the adnominal uses of its personal pronouns (\tabref{Table_1.9}; see also §\ref{Para_6.2}, in \chapref{Para_6}). In Papuan Malay, the second and third singular person pronouns have adnominal uses. They signal definiteness and person-number values, whereby they allow the unambiguous identification of their referents. In other \ili{eastern Malay varieties}, by contrast, ‘\textsc{n} \textsc{pro-sg}’ expressions are analyzed as topic-comment constructions. Besides, the first, second, and third person plural pronouns in Papuan Malay also have adnominal uses; they express \isi{associative} plurality. In the other \ili{eastern Malay varieties}, by contrast, \isi{associative plural} expressions are only formed with the third person plural \isi{pronoun}.

\newpage 
\begin{table}
\caption{Personal pronouns: Adnominal uses of singular and plural pronouns}\label{Table_1.9}

\begin{tabular}{lllllllll}
\lsptoprule
 & PM & AM & BM & KM & LM & MM & {NMM /} &   {TM}\\
\midrule
\textsc{2/3sg} & yes & no & no &  &  &  & no & \\
\textsc{1/2pl} & yes & no &  & no &  & no &  &  no\\
\textsc{3pl}\tablefootnote{Adnominal uses of the third person plural \isi{pronoun} are also reported for \ili{Balai Berkuak Malay} {\citep[7]{Tadmor.2002}}, \ili{Dobo Malay}  (R. Nivens p.c. 2013), and \ili{Sri Lanka Malay} {\citep{Slomanson.2013}}; in \ili{Balai Berkuak Malay} and \ili{Manado Malay} the personal \isi{pronoun} occurs in prehead position.}
 & yes & yes &  & yes &  & yes &  &  no\\
\lspbottomrule
\end{tabular}
\end{table}

In addition, Papuan Malay is compared to the other \ili{eastern Malay varieties} in terms of one phonological and four syntactic features, summarized in \tabref{Table_1.10}.



Papuan Malay has no diphthongs; instead the vowel combinations /\textstyleChCharisSIL{ai}/ and /\textstyleChCharisSIL{au}/ are analyzed as V.V or VC sequences (§\ref{Para_2.1.2}, in \chapref{Para_2}). The same analysis applies to Larantuka and \ili{Manado Malay}. For Ambon and North Moluccan / \ili{Ternate Malay}, by contrast, the same vowel sequences are analyzed as diphthongs. Most likely, though, the different analyses result from differences between the analysts rather than from distinctions between the respective Malay varieties.



In adnominal possessive constructions, the ligature \textitbf{pu(nya)} ‘\textsc{poss}’ not only marks possessive relations, but also has a number of noncanonical functions, such as that of an emphatic marker. Such noncanonical functions of the ligature are also reported for two other \ili{eastern Malay varieties}, namely Ambon and \ili{Ternate Malay}.



In Papuan Malay verbal clauses, core arguments are very often elided (see §\ref{Para_1.6.1.4} and §\ref{Para_11.1}, in \chapref{Para_11}). The same observation applies to Ambon and \ili{Manado Malay}.



In Papuan Malay verbal clauses, the \isi{reciprocity marker} \textitbf{baku} ‘\textsc{recp}’ is analyzed as a separate word (§\ref{Para_11.3}, in \chapref{Para_11}). For Ambon, Banda, Kupang, Manado, and North Moluccan / \ili{Ternate Malay}, by contrast, the same marker is analyzed as a prefix. Most likely, this different analysis is again due to differences between the analysts rather than due to linguistic differences between the respective Malay varieties.



In Papuan Malay negative clauses, the negator \textitbf{bukang} ‘\textsc{neg}’ not only negates nouns and nominal predicate clauses, but also signals contrast (§\ref{Para_13.1.2}, in \chapref{Para_13}). The same observation applies to Ambon, Manado, and \ili{Ternate Malay}.


\begin{table}
\caption{Some phonological and syntactic features in Papuan Malay and other eastern Malay varieties}\label{Table_1.10}


\begin{tabularx}{\textwidth}{lXXXXXXXX}
\lsptoprule

\multicolumn{9}{l}{Phonology: Diphthongs}\\
\midrule
& PM & AM & BM & KM & LM & MM & \multicolumn{2}{l}{ NMM   /   TM}\\
\textsc{diph} & no & yes &  &  & no & no & yes & yes\\
\midrule
\multicolumn{9}{l}{Adnominal possessive constructions: Noncanonical uses of the ligature (\textsc{lig})}\\
\midrule
& PM & AM & BM & KM & LM & MM & \multicolumn{2}{l}{ NMM   /   TM}\\
\textsc{lig} use & yes & yes &  &  &  &  &  &  yes\\
\midrule
\multicolumn{9}{l}{Verbal clauses: Argument elision}\\
\midrule
& PM & AM & BM & KM & LM & MM & \multicolumn{2}{l}{ NMM   /   TM}\\
Elision & yes & yes &  &  &  & yes &  & \\
\midrule
\multicolumn{9}{l}{Verbal clauses: Morphosyntactic status of \isi{reciprocity marker} \textitbf{baku} ‘\textsc{recp}’}\\
\midrule
& PM & AM & BM & KM & LM & MM & \multicolumn{2}{l}{ NMM   /   TM}\\
\textsc{recp} & word & prefix & prefix & prefix &  & prefix & prefix &  prefix\\
\midrule
\multicolumn{9}{l}{Negative clauses: Contrastive function of \textitbf{bukang} ‘\textsc{neg}’}\\
\midrule
& PM & AM & BM & KM & LM & MM & \multicolumn{2}{l}{ NMM   /   TM}\\
\textsc{cst} & yes & yes &  &  &  & yes &  &  yes\\
\lspbottomrule
\end{tabularx}
\end{table}

The overview presented in this section shows several differences and commonalities between Papuan Malay and the other \ili{eastern Malay varieties}.



The differences pertain to \isi{affixation} (form and degree of productivity of the affixes), and the adnominal uses of the personal pronouns. The discussed commonalities involve \isi{reduplication}, the noncanonical uses of the possessive ligature, \isi{elision} of core arguments in verbal clauses, and the contrastive uses of negator \textitbf{bukang} ‘\textsc{neg}’. The observed commonalities suggest that Papuan Malay has more in common with \ili{Ambon Malay} than with the other \ili{eastern Malay varieties}. It is important to note, however, that these differences and commonalities could also result from gaps in the descriptions of the other \ili{eastern Malay varieties}. The noted differences concerning the morphosyntactic status of the \isi{reciprocity marker} and the phonological status of VV sequences most likely result from differences between the analysts rather than from linguistic differences between the compared Malay varieties.



Overall, the noted distinctions and similarities support the conclusion put forward in §\ref{Para_1.8} that the history of Papuan Malay is different from that of the other \ili{eastern Malay varieties}, and that \ili{Ambon Malay} was influential in its genesis. (See §\ref{Para_1.8} ‘History of Papuan Malay’ for more details.)


\section{Demographic information}\label{Para_1.7}
This section presents \isi{demographic information} about the Papuan Malay speakers. Numbers of speakers are discussed in §\ref{Para_1.7.1}, occupation details in §\ref{Para_1.7.2}, education and literacy rates in §\ref{Para_1.7.3}, and religious affiliations in §\ref{Para_1.7.4}.


\subsection{Speaker numbers}\label{Para_1.7.1}
The conservative assessment presented in this section estimates the number of Papuan Malay speakers in West Papua to be about 1,100,000 or 1,200,000.



Previous work provides different estimates for the number of people who speak Papuan Malay. With respect to first language speakers, {\citet[1]{Clouse.2000}} estimates their number at 500,000. As for its uses as a language of wider communication, \citet{Burung.2007}, for instance, give an estimate of one million speakers, while {\citet[71]{Paauw.2009}} approximates their number at 2.2 million speakers. None of the authors provides information, however, on how they arrived at these numbers.



The attempt here to approximate the number of Papuan Malay speakers is based on the 2010 census, conducted by the Non-Departmental Government Institution Badan Pusat Statistik (BPS-Statistics Indonesia). More specifically, the speaker estimate is based on the statistics published by the BPS-Statistics branches for Papua province and Papua Barat province.\footnote{Statistics from BPS-Statistics Indonesia are available at \url{http://www.bps.go.id/} (accessed 8 January 2016). Statistics for Papua province are available at \url{http://papua.bps.go.id} (accessed 8 January 2016), and statistics for Papua Barat province are available at \url{http://irjabar.bps.go.id/} (accessed 8 January 2016).}



According to the BPS-Statistics for Papua province and Papua Barat province, the total population of West Papua is 3,593,803; this includes 2,833,381 inhabitants of Papua province and 760,422 inhabitants of Papua Barat province\footnote{\label{Footnote_1.38}Population totals for Papua province are also available at \url{http://papua.bps.go.id/yii/9400/index.php/post/552/Jumlah Penduduk Papua} (accessed 21 Oct 2013), and for Papua Barat province at \url{http://irjabar.bps.go.id/publikasi/2011/Statistik Daerah Provinsi Papua Barat 2011/baca_publikasi.php} (accessed 21 Oct 2013).} (\citealt[11--14]{BidangNeracaWilayahdanAnalisisStatistik.2011}; \citealt[92]{BidangNeracaWilayahdanAnalisisStatistik.2012}). The census data does not discuss the number of Papuan Malay speakers. The (online) data does, however, give information about ethnicity (Papuan versus non-Papuan\footnote{A  ``Papuan'' is defined as someone who has at least one Papuan parent, is married to a Papuan, has been adopted into a Papuan family, or has been living in Papua for 35 years \citep[11]{BidangNeracaWilayahdanAnalisisStatistik.2011b}.}) by regency (for detailed population totals see Appendix \ref{Para_E}).



The present attempt at approximating the number of Papuan Malay speakers is based on the following assumptions: (1) Papuans who live in the coastal regencies of West Papua are most likely to speak Papuan Malay, (2) Papuans living in the interior regencies are less likely to speak Papuan Malay, and (3) non-Papuans living in West Papua are less likely to speak Papuan Malay. It is acknowledged, of course, that there might be older Papuans living in remote coastal areas who do not speak Papuan Malay, that there might be Papuans living in the interior who speak Papuan Malay, and that there might be non-Papuans who speak Papuan Malay.



For Papua province, the census data by regency and ethnicity gives a total of 2,810,008 inhabitants, including 2,150,376 Papuans (76.53\%) and 659,632 non-Papuans (23.47\%), who live in its 29 regencies.\footnote{\label{Footnote_1.40}The statistics for Papua province do not give population details by regency and ethnicity per se. They do, however, include this information in providing population details by religious affiliation under the category \textstyleChItalic{Sosial Budaya} ‘Social (affairs) and Culture’; see \url{http://papua.bps.go.id/yii/9400/index.php/site/page?view=sp2010} (accessed 21 Oct 2013). By adding up the population details according to religious affiliation it is possible to arrive at overall totals by regency and ethnicity.} (This total of 2,810,008 more or less matches the total given for the entire province which lists the entire population of Papua province with 2,833,381). Of the 29 regencies, 14 are essentially coastal; the remaining 15 are located in the interior.\footnote{Coastal regencies: Asmat, \ili{Biak} Numfor, Jayapura, Kota Jayapura, Keerom, Yapen, Mamberamo Raya, Mappi, Merauke, Mimika, Nabire, \ili{Sarmi}, Supiori, Waropen.\\
Interior regencies: Boven Digoel, Deiyai, Dogiyai, Intan Jaya, Jayawijaya, Lanny Jaya, Mamberamo Tengah, Nduga, Paniai, Pegunungan Bintang, Puncak, Puncak Jaya, Tolikara, Yahukimo, Yalimo.} The total population for the 14 coastal regencies is 1,364,505, which includes 756,335 Papuans and 608,170 non-Papuans. Based on the above assumptions that Papuans living in coastal areas can speak Papuan Malay, and that non-Papuans are less likely to speak it, the number of Papuan Malay speakers living in Papua province is estimated at 760,000 speakers.



For Papua Barat province, the census data by regency and ethnicity gives a total of 760,422 inhabitants, including 405,074 Papuans (53.27\%) and 355,348 non-Papuans (46.73\%) living in its 11 regencies.\footnote{Papua Barat regencies: Fakfak, Kaimana, Kota Sorong, Manokwari, Maybrat, Raja Ampat, Sorong, Sorong Selatan, Tambrauw, Teluk Bintuni, and Teluk Wondama.} Ten of its regencies are essentially coastal; the exception is Maybrat, which is located in the interior. The total population for the ten regencies is 727,341, including 373,302 Papuans and 354,039 non-Papuans. Based on the above assumptions, the number of Papuan Malay speakers living in Papua Barat province is estimated with 380,000 speakers. \citep[11--14]{BidangNeracaWilayahdanAnalisisStatistik.2011b}



These findings give a total of between 1,100,000 to 1,200,000 potential speakers of Papuan Malay. This estimate is conservative, as people living in the interior are excluded. Moreover, non-Papuans are excluded from this total. However, the results of a sociolinguistic survey carried out in 2007 by the Papuan branch of SIL Indonesia in several costal regencies indicate  ``substantive use'' of Papuan Malay by  ``non-Papuan residents of the region'' \citep[11]{Scott.2008}.



The population estimate presented here does not make any statements about the potential number of first language Papuan Malay speakers. The results of the 2007 survey indicate, however, that large numbers of children learn Papuan Malay at home: all of the 14 interviewed focus groups stated that Papuan Malay is spoken in their region; moreover, 70\% of the focus groups indicated that Papuan Malay is  ``the first language children learn in the home as well as the language most commonly used in their region'' \citep[11]{Scott.2008}.

\subsection{Occupation details}\label{Para_1.7.2}
\largerpage
Most of West Papua’s population works in the agricultural sector: 70\% in Papua province, and 54\% in Papua Barat province. As subsistence farmers, they typically grow bananas, sago, taro, and yams in the lowlands, and sweet potatoes in the highlands; pig husbandry, fishing, and forestry are also widespread. The second most important domain is the public service sector. In Papua province, 10\% of the population works in this sector, and 17\% in Papua Barat province. Furthermore, 9\% in Papua province and 12\% in Papua Barat province work in the commerce sector. Other minor sectors are transport, construction, industry, and communications. (\citealt[21]{BidangNeracaWilayahdanAnalisisStatistik.2011b}; \citeyear*[12]{BidangNeracaWilayahdanAnalisisStatistik.2012}; \citealt{EncyclopaediaBritannica.2001b}; \citeyear*{EncyclopaediaBritannica.2001};  see also \citealt[83]{BidangNeracaWilayahdanAnalisisStatistik.2012b}).



The census data does not provide information about occupation by ethnicity. However, the author made the following observations for the areas of \ili{Sarmi} and Jayapura (see \figref{Figure_0.2} on p. \pageref{Figure_0.2} and \figref{Figure_0.3} on p. \pageref{Figure_0.3}). Papuans typically work in the agricultural sector; those living in coastal areas are also involved in small-scale fishing. Those with a secondary education degree usually (try to find) work in the public sector. The income generating commerce and transportation sectors, by contrast, are in the hands of non-Papuans. This assessment is also shared by {\citet[124]{Chauvel.2002}} who maintains that  ``Indonesian settlers dominate the economy of [West] Papua''. The author does not provide details about the origins of these settlers. Given Indonesia’s ``transmigration'' program, however, it can be assumed that most, or at least substantial numbers, of these settlers originate from the overcrowded islands of Java, Madura, Bali, and/or Lombok. Moreover, substantial numbers of active and retired military personnel have settled in West Papua.\footnote{\phantomsection\label{Footnote_1.43} ``Transmigration'' is a program by the Indonesian government to resettle millions of inhabitants. Coming from the overcrowded islands of Java, Madura, Bali, and Lombok, they settle in the less populated areas of the archipelago, such as West Papua. The first transmigration project was launched in 1905 {\citep[553]{Fearnside.1997}}. During the second World War, the project was put on hold,  ``until the current transmigration program was launched in 1950'' \citep[554]{Fearnside.1997}. Between 1905 and 1989,  ``a cumulative total of approximately [\ldots] five million people [\ldots] had been shipped to the outer islands as part of the official program, plus anywhere from two to three times this many had moved independent of the program'' (\citealt[554]{Fearnside.1997}; see also \citealt{EmbassyoftheRepublicofIndonesiainLondon.2009}).} (See \citealt{Fearnside.1997}; \citealt{EmbassyoftheRepublicofIndonesiainLondon.2009}.)


\subsection{Education and literacy rates}\label{Para_1.7.3}
The 2010 census data provides information about school enrollment and literacy rates in \ili{Standard Indonesian}. In West Papua, most children attend school. For older teenagers and young adults, however, the rates of those who are still enrolled in a formal education program are much lower. Literacy rates for the adult population aged 45 years or older are lower than the rates for the younger population. Overall, education and literacy rates are (much) lower for Papua province than for Papua Barat province. Details are given in \tabref{Table_1.11} to \tabref{Table_1.13}.



Most children under the age of 15 go to school, as shown in \tabref{Table_1.11}. However, the data also indicates that this rate is much lower for Papua province than for Papua Barat province. The number of teenagers aged between 16 and 18 who are still enrolled in school is much lower for both provinces, again with Papua province having the lower rate. As for young adults who are still enrolled in a formal education program, the rate is even lower, at less than 15\%. The data in \tabref{Table_1.11} gives no information about the school types involved. That is, these figures also include children and teenagers who are enrolled in a school type that is not typical for their age group.\footnote{The school participation rates by age groups in \tabref{Table_1.11} are available at \url{http://www.bps.go.id/eng/tab\_sub/view.php?kat=1\& tabel=1\& daftar=1\& id\_subyek=28\& notab=3} (accessed 21 Oct 2013).} (For enrollment figures by school types see  \tabref{Table_1.12}.)


\begin{table}
\caption{Formal education participation rates by age groups}\label{Table_1.11}
%\todo[inline]{add ``values in percent''?}


\begin{tabular}{l*{4}{r}}
\lsptoprule

%   & \multicolumn{4}{c}{Values in percent}\\
    & \multicolumn{1}{c}{7--12 } & \multicolumn{1}{c}{13--15} & \multicolumn{1}{c}{16--18} &  \multicolumn{1}{c}{19--24}\\
 \midrule
% Province\\
% \midrule
Papua &  76.22\% &  74.35\% &  48.28\% &  13.18\%\\
Papua Barat &  94.43\% &  90.25\% &  60.12\% &  14.66\%\\
\lspbottomrule
\end{tabular}
\end{table}

The 2010 census data also shows that most children get a primary school education (76.22\% in Papua province, and 92.29\% in Papua Barat province). Enrollment figures for junior high school are considerably lower with only about half of the children and teenagers being enrolled. Figures for senior high school enrollment are even lower, at less than 50\%. The data in \tabref{Table_1.12} also shows that overall Papua Barat province has higher enrollment rates than Papua province, especially for primary schools.\footnote{The enrollment rates by school types in \tabref{Table_1.12} are available at \url{http://www.bps.go.id/eng/tab_sub/view.php?kat=1 & tabel=1 & daftar=1 & id_subyek=28 & notab=4} (accessed 21 Oct 2013).}


\begin{table}
\caption{School enrollment rates by school type}\label{Table_1.12}


\begin{tabular}{lrrr}
\lsptoprule

%  Province 
 & \multicolumn{1}{c}{Primary} & \multicolumn{1}{c}{Junior high} &  \multicolumn{1}{c}{Senior high}\\
 \midrule
Papua &  76.22\% &  49.62\% &  36.06\%\\
Papua Barat &  92.29\% &  50.10\% &  44.75\%\\
\lspbottomrule
\end{tabular}
\end{table}

Literacy rates in 2010 differ considerably between the populations of both provinces. In Papua province only about three quarters of the population is literate, while this rate is above 90\% for Papua Barat province, as shown in \tabref{Table_1.13}. In Papua province, the literacy rates are especially low in the Mamberamo area, in the highlands, and along the south coast \citep[27–30]{BidangNeracaWilayahdanAnalisisStatistik.2011}.\footnote{The literacy rates in \tabref{Table_1.13} are available at \url{http://www.bps.go.id/eng/tab_sub/view.php?kat=1 & tabel=1 & daftar=1 & id_subyek=28 & notab=2} (accessed 21 Oct 2013).}


\begin{table}
\caption{Illiteracy rates by age groups}\label{Table_1.13}


\begin{tabular}{lrrr}
\lsptoprule

%  Province 
 & \multicolumn{1}{c}{{\textless}15} & \multicolumn{1}{c}{15--44} &  \multicolumn{1}{c}{45+}\\
 \midrule
Papua &  31.73\% &  30.73\% &  36.14\%\\
Papua Barat &  4.88\% &  3.34\% &  9.91\%\\
\lspbottomrule
\end{tabular}
\end{table}

The census data provides no information about education and literacy rates according to rural versus urban regions. The author assumes, however, that education and literacy rates are lower in rural than in urban areas. The census data also does not include information about education and literacy rates by ethnicity. As mentioned in §\ref{Para_1.7.2}, the author has the impression that Papuans typically work in the agriculture sector while non-Papuans are more often found in the income generating commerce and transportation sectors. This, in turn, gives non-Papuans better access to formal education, as they are in a better position to pay tuition fees.


\subsection{Religious affiliations}\label{Para_1.7.4}
West Papua is predominantly Christian. For most Papuans their Christian faith is a significant part of their Papuan identity. It distinguishes them from the Muslim Indonesians who have come from Java, Madura, and Lombok and settled in West Papua, as a result of Indonesia’s transmigration program (see Footnote \ref{Footnote_1.43} in §\ref{Para_1.7.2}, p. \pageref{Footnote_1.43}).
% \todo{recheck hyperlink to footnote}


Papua province has 2,810,008 inhabitants, including 2,150,376 Papuans and 659,632 non-Papuans. Almost all Papuans are Christians (2,139,208 = 99.48\%), while only 10,759 are Muslims (0.05\%); the remaining 0.02\% has other religious affiliations. Of the 659,632 non-Papuans, two thirds are Muslims (439,337 = 66.60\%), while one third are Christians (216,582 = 32.83\%); the remaining 0.57\% has other religious affiliations.\footnote{Detailed data by regency is available under the category \textstyleChItalic{Sosial Budaya} ‘Social (affairs) and Culture’ at \url{http://papua.bps.go.id/yii/9400/index.php/site/page?view=sp2010} (accessed 21 Oct 2013).}



Papua Barat province has 760,422 inhabitants, including 405,074 Papuans and 355,348 non Papuans. For Papua Barat province, no census data is published by ethnicity and religion. Based on the data given in \citet[11–14]{BidangNeracaWilayahdanAnalisisStatistik.2011}, however, the following picture emerges: most Papuans are Christians (352,171 = 86.94\%), while 52,903 are Muslims (13.06\%), most of whom live in the Fakfak regency. Of the 355,348 non-Papuans, about two thirds are Muslims (239,099 = 67.29\%) and one third are Christians (110,166 = 31.00\%); the remaining 1.71\% have other religious affiliations.


\section{History of Papuan Malay}\label{Para_1.8}
Papuan Malay is a rather young language. It only developed over approximately the last 130 years, unlike other Malay languages in the larger region. As discussed in this section, though, the precise origins of Papuan Malay remain unclear. That is, it is not known exactly which Malay varieties had which amount of influence in which regions of West Papua in the formation of Papuan Malay.



Malay has a long history as a trade language across the Malay Peninsula and the Indonesian archipelago. The language spread to the Moluccas through extensive trading networks. It was already firmly established there before the arrival of the first Europeans in the sixteenth century. (See \citealt{Adelaar.1996}; \citealt{Collins.1998}; \citealt[42--79]{Paauw.2009}.)  From the Moluccas, Malay spread to West Papua where it developed into today’s Papuan Malay.



The southwestern part of West Papua was under the influence of the island of Seram in the central Moluccas, with trade relationships firmly established from about the fourteenth century, long before the first Europeans arrived. A special \ili{lingua franca}, called Onin, was used in the context of these trade relations. Onin was  ``a mixture of Malay and local languages spoken along the coasts of the Bomberai Peninsula'' {\citep[1]{Goodman.2002}}. Unfortunately, {\citet{Goodman.2002}} does not discuss the relationship between Onin and Malay in more detail. It is noted, though, that today Malay is spoken in Fakfak, the main urban center on the Bomberai Peninsula, as well as in the areas around Sorong and Kaimana. According to \citet[2]{Donohue.2003}, the Malay spoken in these areas  ``is essentially a variety of \ili{Ambon Malay}'' \citep[see also][]{Walker.1982}.


The Bird’s Head and Geelvink Bay, now Cenderawasih Bay, were under the authority of the Sultanate of Tidore. The first mention of Tidore’s authority over this part of West Papua dates back to 15 January 1710 and can be found in the \textstyleChItalic{Memorie van Overgave} ‘Memorandum of Transfer’ by the outgoing Governor of Ternate Jacob Claaszoon. In summarizing this memorandum,\footnote{While \citet[192--195]{Haga.1884} gives no further bibliographical details for this memorandum, the following details are found in {\citet[262]{Andaya.1993}}: VOC 1794. Memorie van overgave, Jacob Claaszoon, 14 July 1710, fols 55--56.} \citet[192--195]{Haga.1884} lists the locations on New Guinea’s coast which belonged to Tidore’s territory. Included in this list is the west coast of Geelvink Bay, with {Haga} pointing out that Tidore also claimed authority over Geelvink Bay’s south coast. In the second half of the nineteenth century, however, Tidore’s authority over Geelvink Bay declined after the \ili{Dutch} banned Tidore’s raiding expeditions to New Guinea on 22 February 1861 \citep[28--29]{Bosch.1995}. Roughly 35 years later, in 1895, the outgoing Resident of Ternate, J. van Oldenborgh noted that, due to this ban, Tidore’s authority on New Guinea had been reduced to zero as the sultans no longer had the means to enforce their authority in this area \citep[81]{vanOldenborgh.1995}. In 1905, the last sultan of Tidore, Johar Mulki (1894--1905), relinquished all rights to western New Guinea to the \ili{Dutch} (\citealt{vanderEng.2004}; see also \citealt[138]{Overweel.1995}).



Due to \ili{Tidorese} influence in the eighteenth and nineteenth centuries, the Bird’s Head and Geelvink Bay were firmly connected with the wider Moluccan trade network (\citealt[72]{Seiler.1982}; \citealt[2--3]{Timmer.2002}; \citealt[314–315]{vanVelzen.1995}; see also \citealt{Huizinga.1998} on the relations between Tidore and New Guinea’s north coast in the nineteenth century). However, scholars disagree on how firmly Malay was established in this area, especially in Geelvink Bay, during these early trading relations.


\largerpage
\citet[53]{Rowley.1972}, for instance, suggests that the Malay presence along West Papua’s western coast may date back to the fourteenth century. Malay influence began with \ili{Javanese} trading settlements and then continued with trading settlements which were under the control of Seram and Tidore. At that time, the \ili{Dutch} did not yet show any direct interest in this region. It was the British who, in 1793, established the first European post at Dorey, now Manokwari, which they maintained for two years. During this period Dorey was already under the influence of Tidore and its inhabitants had to pay an annual tribute to the Tidore sultan. Van Velzen (\citeyear*[314–315]{vanVelzen.1995}) also claims that Malay was a regional language of wider communication long before the arrival of the first Europeans is. He refers to \citegen{Haga.1885} account of one of the first European visits to the Yapen Waropen area, which took place in 1705. On Yapen Island the crew was able to communicate in Malay with some of the local inhabitants. Given that these inhabitants were ethnically \ili{Biak}, \citet{vanVelzen.1995} concludes that it may have been the \ili{Biak} who first introduced Malay to Geelvink Bay.\footnote{Along similar lines \citet[3]{Samaun.1979} states that Malay, namely Ambon or \ili{Ternate Malay},  ``was long ago introduced'' in West Papua. The author does not, however, provide a more precise date, instead maintaining that Malay has been used in West Papua  ``for more than a century'' (\citeyear*[3]{Samaun.1979}).}

 
This claim of the long-standing presence of Malay in the Geelvink Bay is not, however, supported by the reports of explorers who visited the Geelvink area in the nineteenth century. These early visits occurred after the \ili{Dutch} had first shown interest in this region. This was only in 1820, after the British had established their post at Dorey in 1793; this first \ili{Dutch} interest  ``was due in part to the fear that other attempts would be made'' \citep[53]{Rowley.1972}.



For instance, when the French explorer and rear admiral \citet[606]{DumontdUrville.1833} stayed in Dorey (Manokwari) in September 1827, he noted that the Papuans, who formed the majority of inhabitants in Dorey, hardly knew any Malay; only the upper-class of Dorey spoke Malay more or less fluently. A similar statement about the Papuans’ abilities to speak Malay comes from \citet{vanHasselt.1936}. He reports how the first missionaries to West Papua, the Germans Ottow and Geissler, together with his father van Hasselt and the \ili{Dutch} researcher Croockewit attempted to learn and study the local language after they had arrived in Geelvink Bay in 1858. The author notes that it was very difficult for them to learn the local language, as the Papuans knew little or no Malay \citep[116]{vanHasselt.1936}. Along similar lines, the British naturalist \citet[380]{Wallace.1890} relates that, when he came to Dorey in 1858, the local Papuans could not speak any Malay.



Based on these reports, it can be concluded that in the early eighteen hundreds Malay was not yet well established in Geelvink Bay, including the area in and around today’s Manokwari. Hence, the author agrees with \citet[73]{Seiler.1982}, who comes to the conclusion that, in light of accounts such as the one by \citet{DumontdUrville.1833},


\begin{quote}
[t]here is no reason to assume that Malay was better known at other places along New Guinea’s north coast; Manokwari was one of the most visited places in the area and if anything, Malay should have been known to a larger extent there than anywhere else.
\end{quote}


The history of Malay along West Papua’s north and northeast coast is also disputed among scholars.



\citet[56--57]{Rowley.1972} states that  ``Malay adventurers'' went eastwards to the Sepik area  ``in expeditions for birds of paradise''. Even long before the nineteenth century, Malay traders made sporadic visits to the northeastern coasts of New Guinea and the Bismarck Archipelago. Hence, Rowley concludes that Malay influence along West Papua’s north and northeast coast began long before the \ili{Dutch} started taking an interest this area.



The Danish anthropologist \citet{Parkinson.1900} came to a similar conclusion after having visited the north coast of today’s Papua New Guinea. Based on his acquaintanceship with some Malay-speaking inhabitants, Malay artifacts, and some inherited Malay words, the explorer concludes that Malay seafarers from the East India islands have undertaken trips along the coast of New Guinea  ``for a long time'' (\citeyear*[20--21]{Parkinson.1900}).



This conclusion is not supported, however, by the observations of other European explorers who visited West Papua’s northeast coast in the nineteenth century after the \ili{Dutch} had annexed the western part of New Guinea in 1828.\footnote{In 1828, the \ili{Dutch} annexed today’s West Papua as far as 141 degrees of east longitude (today’s border with Papua New Guinea) \citep[509]{Burke.1831}.}



Twenty years after this annexation, in 1848, the \ili{Dutch} laid formal claim on West Papua’s north coast, including Humboldt Bay in the east, now Yos Sudarso Bay with the provincial capital Jayapura \citep[56]{Rowley.1972}. In 1850, the \ili{Dutch} sent a first expedition fleet eastwards to mark their claim; this expedition included Sultanese boats and a number of pirate boats. The fleet did not, however, reach Humboldt Bay, although the Cyclops Mountains were in sight. Two years later, though, the \ili{Dutch} were able to establish a garrison in Humboldt Bay; the troops were from Ternate. However, it seems that this garrison did not include any Europeans, because, according to \citet[74]{Seiler.1982}, it was only in the course of the  ``Etna expedition'' in 1858 that the \ili{Dutch} first reached Humboldt Bay. The report of this expedition states that the Papuans living in Humboldt Bay did not know any Malay and had had no contact with the outside world \citep[182--183]{CommissievoorNieuwGuinea.1862}.



Twenty years later it was still not possible to communicate in Malay with the Papuans of Humboldt Bay. \citet[127--129]{RobidevanderAa.1879}, for instance, reported that when the Government commissioner van der Crab visited Humboldt Bay in 1871, his interpreter could not communicate with the local population because of their very poor Malay. The commissioner also noted that outside trading in this area was very limited due to tense relations between the Papuan population and outside traders and due to the wild sea.



Around this time, however, outside trading between the Moluccas and West Papua’s northeast coast, including Humboldt Bay and the areas to its east, started to take off. As a result of this increase in outside contacts, knowledge of Malay, especially of the North Moluccan varieties, also started to spread rapidly in this region. Seiler (\citeyear*{Seiler.1982}; \citeyear*{Seiler.1985}) gives an overview of these developments, citing government officials, merchants, and missionaries who visited West Pap\-ua’s northeast coast in the late nineteenth century.



One of them was the Protestant missionary \citet{Bink.1894}. In 1893, about twenty years after van der Crab’s 1871 visit to this area, {Bink} travelled to Humboldt Bay. In his report he noted the presence of Malay traders from Ternate who were shooting birds of paradise in the area (\citeyear*[325]{Bink.1894}). Another observer is the \ili{German} geologist \citet{Wichmann.1917}. In 1903, he travelled to Humboldt Bay and Jautefa Bay, where today’s Abepura is located. Wichmann reported the presence of Malay traders who were living on Metu Debi Island in Jautefa Bay (\citeyear*[150]{Wichmann.1917}). A third observer is \citet{vanHasselt.1926}. When he visited Jamna Island (located off the northeast coast between \ili{Sarmi} and Jayapura) in 1911, he noted that several Papuans could already speak Malay, because they had been in regular contact with traders (\citeyear*[134]{vanHasselt.1926}).



Based on the reports of these observers, \citet[147]{Seiler.1982} comes to the following conclusion:

\begin{quote}
It would appear that Malays started regular trading visits to areas east of Geelvink Bay sometime after the middle of the 19th century, at the same time as the \ili{Dutch} began to explore their long-forgotten colony. This was just prior to the beginning of the \ili{German} activities in the area. Twenty years or so of contact between the local people and Malays could easily account for the knowledge of Malay on the part of the coastal people.
\end{quote}


In the early twentieth century, the use of Malay throughout West Papua increased when the \ili{Dutch} decided to increase their influence in this area and to enforce the use of Malay in the domains of education, administration, and proselytization. A major resource for these efforts was the Malay-language school system already established in the Moluccas. It provided the \ili{Dutch} with the personnel necessary for bringing the population and the resources of West Papua under their control {\citep[64]{Collins.1998}}. Therefore West Papua saw a constant influx of \ili{Ambon Malay} speaking teachers, clerks, police, and preachers during this period \citep[254–255]{Donohue.2007d}. This link between West Papua and Ambon was especially close, as until 1947 West Papua was part of the Moluccan administration, which had its capital in Ambon. So \ili{Ambon Malay} played an important role in the genesis of Papuan Malay, as well as North Moluccan Malay.



After World War II, the \ili{Dutch} government recruited additional personnel for West Papua from other areas, such as North Sulawesi, Flores, Timor, and the Kei Islands. In addition, fishermen and traders from Sulawesi and, to some extent, from East Nusa Tenggara came to West Papua. (\citealt[96]{Roosman.1982}; \citealt[682]{Adelaar.1996}; \citealt[254--255]{Donohue.2007d}.) At the same time, increasing numbers of Papuans received a primary school education. Furthermore, the \ili{Dutch} established schools to train Papuans for public services. As a result, more and more Papuans became government officials, teachers, and police officers. During this period, Standard Malay was the official language in public domains, including trade and the religious domain. (\citealt[120]{Chauvel.2002}; \citealt[255]{Donohue.2007d}; see also \citealt[234]{Adelaar.2001}.) Outside the coastal urban centers, however, Malay played only a very limited role. This is evidenced by that fact that along West Papua’s north coast Papuan Malay is still  ``restricted to a coastal fringe, and does not extend inland to any great extent except where agricultural projects were in force'' \citep[255]{Donohue.2007d}.



After Indonesia annexed West Papua in 1963, \ili{Standard Indonesian} became the official language of West Papua. It is used in all public domains, including primary school education, the mass media, and the religious domain.



West Papua’s Malay, by contrast, is not recognized as a language in its own right vis-à-vis Indonesian (for details on the \isi{sociolinguistic profile} of Papuan Malay, see §\ref{Para_1.5}). Only recently has Papuan Malay received attention from linguistics as an independent language (for details see §\ref{Para_1.9}). Materials in Papuan Malay are equally recent (for details see §\ref{Para_1.10}).



In speaking about Papuan Malay and its history and genesis one aspect needs to be highlighted, however. As \citet[73]{Paauw.2009} points out, there is linguistic evidence that both North Moluccan Malay (on the north and east coasts of the Bird’s Head and in parts of Cendrawasih Bay, including the islands of \ili{Biak} and Numfoor) and \ili{Ambon Malay} (in the western and southern Bird’s Head, the Bomberai peninsula, and in other parts of Cendrawasih Bay, including the island of Yapen) have been influential.



It is still unknown, though, exactly how much influence each variety had in the various regions of West Papua. Overall, however, regional differences in the usage of Papuan Malay across the language area seem to be minor, as discussed in §\ref{Para_1.3}.

\newpage 
The developments described in this section show that the history of Papuan Malay is quite distinct from that of other \ili{eastern Malay varieties}. Other \ili{eastern Malay varieties} were already well established before the first Europeans arrived in these areas in the sixteenth century. This applies to Ambon and North Moluccan Malay, both of which contributed to Papuan Malay. It also applies to \ili{Manado Malay}, which apparently developed out of North Moluccan Malay. Likewise, it applies to \ili{Kupang Malay}. (\citealt[42--79]{Paauw.2009}; see also \citealt{Adelaar.1996}; \citealt{Collins.1998}.) Papuan Malay, by contrast, only developed over the last 130 years or so.


\section{Previous research on Papuan Malay}\label{Para_1.9}
Until the second half of the twentieth century, the Malay varieties spoken in New Guinea had received almost no attention. Linguists only started taking more notice of the language in the second half of the twentieth century. An overview of these early studies is given in §\ref{Para_1.9.1}. More recent studies, starting from the early years of the twenty first century, are discussed in §\ref{Para_1.9.2}. In addition, Papuan Malay has received attention in the context of sociolinguistic and sociohistorical studies (§\ref{Para_1.9.3}).


\subsection{Early linguistic studies on the Malay varieties of West Papua}\label{Para_1.9.1}
\citet{Zoller.1891} mentions Malay in his description of the \textstyleChItalic{Papua Sprachen} ‘languages of Papua’ (\citeyear*[351–426]{Zoller.1891}), as well as in his 300-item \isi{word list} of 48 languages of Papua (\citeyear*[443–529]{Zoller.1891}); the 48 languages include 29 languages of \ili{German} New Guinea, and 17 languages of British New Guinea, as well as Malay and Numfor of Netherlands New Guinea (for comparative reasons, the \isi{word list} also includes \ili{Maori} and \ili{Samoan}, besides the 48 languages of Papua).



Likewise,{ \citet{Teutscher.1954}} mentions Malay in his article on the languages spoken in New Guinea. As a \ili{lingua franca} it is used in formal and informal domains. Moreover, for Papuans this Malay has become a \textstyleChItalic{tweede moedertaal} ‘second mother tongue’ (\citeyear*[123]{Teutscher.1954}).



Also available is a \textstyleChItalic{Beknopte leergang Maleis voor Nieuw-Guinea} ‘A concise language course in the Malay variety spoken in New Guinea’ \citep{BureauCursussenenVertalingen.1950}.



The Malay of New Guinea is also mentioned by \citeauthor{Anceaux.1960} in their Malay-\ili{Dutch}-\ili{Dani} \isi{word list} (\citeyear*{Anceaux.1960}) as well as in their penciled New Guinea Malay-\ili{Dutch} \isi{word list} {(no date)}.



In addition, {\citet[49]{Teeuw.1961}} states that after 1950 a variety of publications were produced specifically for western New Guinea; they were written in Malay with a  ``distinctly local colour''. At the same time, however, the author notes that there were no publications which discussed the Malay of Netherlands New Guinea or the language policies regarding this Malay variety.



Around the same time, {\citet{Moeliono.1963}} mentions Indonesian in his study of the languages spoken in West Papua. The author refers to the language as a \textstyleChItalic{logat bahasa Indonesia} ‘speech variety of the Indonesian language’ without, however, discussing its features. The author does state, though, that this  ``dialect'' is spoken in the coastal and urban areas of West Papua and used by the \ili{Dutch} colonial government for letters and announcements. Moreover, it is used as a \ili{lingua franca}, both in formal and informal domains.



Early linguistic studies on the Malay varieties spoken in West Papua date back to the second half of the twentieth century.



\citet{Samaun.1979} highlights some morphological, syntactical, and lexical features where the \textstyleChItalic{dialek Indonesia Irian} ‘\ili{Irian Indonesian} dialect’ of Jayapura differs from \ili{Standard Indonesian}. While explaining these differences as mere simplifications, the author also notes that due to some of these modifications, this \textstyleChItalic{dialek} of Indonesian sounds non-Indonesian.



Along similar lines, \citet{Suharno.1979, Suharno.1981} describes some aspects of Papuan Malay \isi{phonology}, \isi{morphology}, lexicon, and grammar in comparison to \ili{Standard Indonesian}. While referring to Papuan Malay as an Indonesian dialect, the author suggests that this variety of Indonesian is autonomous and deserves more research. The author also maintains that this dialect is a suitable language for development programs. In formal situations, however, the language variety is unacceptable.



Unlike \citet{Samaun.1979} and \citet{Suharno.1979, Suharno.1981}, \citet{Roosman.1982} does not refer to Papuan Malay as a dialect of Indonesian. Instead, he considers Papuan Malay as a form of \ili{Ambon Malay} which has  ``\ili{pidgin Malay} as its basic stratum'' (\citeyear*[1]{Roosman.1982}). In his paper, the author presents phonetic inventories of \ili{Ambon Malay} (Irian Malay), Pidgin Malay, and Indonesian and comments on some of the differences he found.



Another scholar who mentions various features of the Malay spoken in West Papua is \citet{Walker.1982}. In the context of his study on \isi{language use} at Namatota, a village located on West Papua’s southwest coast, the author discusses some of the similarities which Malay shares with Indonesian and some of the distinctions between both languages.



\citet{Ajamiseba.1984} mentions the Malay variety spoken in West Papua in the context of his study on the linguistic diversity found in this part of New Guinea. Referring to this speech variety as ``\ili{Irian Indonesian}'', the author compares some of its features to those of other languages spoken in West Papua. This comparison, however, seems to be based on \ili{Standard Indonesian} rather than on Papuan Malay.



In  \citeyear*{vanVelzen.1995}, \citeauthor{vanVelzen.1995} published his  \citetitle{vanVelzen.1995} (\citeyear*{vanVelzen.1995}). Similar to previous studies, the author highlights some aspects of \ili{Serui Malay} in comparison to \ili{Standard Indonesian}. Based on phonological, morphological, and lexical features, \citet[315]{vanVelzen.1995} concludes that \ili{Serui Malay} and the other Malay varieties of West Papua’s north coast  ``are probably more closely related to \ili{Tidorese} or Ternatan Malay'' than to \ili{Ambon Malay}, as suggested by \citet{Roosman.1982}.\footnote{With respect to this quote, R. Nivens (p.c. 2013) suggests that \citet[315]{vanVelzen.1995} made this comment  ``because the sultan of Tidore once claimed sovereignty over parts of Papua'', but it is doubtful  ``that he had any actual linguistic data to back up this claim''.}


\subsection{Recent linguistic descriptions of Papuan Malay}
\label{Para_1.9.2}
More recently, Papuan Malay has received attention from linguistics as a language in its own right vis-à-vis the other \ili{eastern Malay varieties} as well as vis-à-vis Indonesian. Three studies give an overview of the most pertinent features of Papuan Malay: \citet{Donohue.2003}, \citet{Paauw.2009}, and \citet{Scott.2008}.



\citet{Donohue.2003} discusses various linguistic features of Papuan Malay as spoken in the area around Geelvink Bay. The described features include, among others, \isi{phonology}, \isi{noun} phrases, verbal morphosyntax, and clause linkages.



In the context of his typological study of seven \ili{eastern Malay varieties}, {\citet{Paauw.2009}} compares Papuan Malay with Ambon, Banda, Kupang, Larantuka, Manado, and North Moluccan Malay.\footnote{The basis for the description of Papuan Malay is textual data collected in Manokwari \citep[35]{Paauw.2009}, as well as data available in previous studies: \citet{Suharno.1981};  \citet{vanVelzen.1995}; \citet{Donohue.2003}; \cite{Burung.2007}; {\citet{Kim.2007}} (this study is an earlier version of \citealt{Scott.2008}); {\citet{Sawaki.2007}}.} The described features include \isi{phonology}, lexical categories, word order, clause structure, \isi{noun} phrases, prepositional phrases, and \isi{verb} phrases.



\citegen{Scott.2008} study is part of a larger sociolinguistics language survey of the Papuan Malay varieties of West Papua (see §\ref{Para_1.9.3}). The authors describe different aspects of the lexicon, \isi{phonology}, \isi{morphology}, syntax, and discourse of Papuan Malay as spoken in (and around) the urban areas of Fakfak, Jayapura, Manokwari, Merauke, Timika, Serui, and Sorong (see also \figref{Figure_0.2} on p. \pageref{Figure_0.2}).



In addition, there are a number of studies which explore specific aspects of Papuan Malay.



One of the investigated features is the personal \isi{pronoun} system. \cite{Donohue.2007d} examine the innovative forms and functions of the \isi{pronoun} system in Papuan Malay as spoken along West Papua’s north coast. In their study on the development of \ili{Austronesian} first-person pronouns, \cite{Donohue.1998} explore the loss of the inclusive-exclusive distinction in non-singular personal pronouns in Papuan Malay as spoken in Serui and Merauke, as well as in other nonstandard Malay varieties. \citet{Saragih.2012} investigates the use of person reference in everyday language on the social networking service Facebook.

Besides the personal \isi{pronoun} system, the voice system – that is to say, the lack thereof – has also received attention. {\citet{Donohue.2007}} investigates the \isi{variation} in the voice systems of six different Indonesian/Malay varieties, including Papuan Malay as spoken in the areas around Jayapura and Serui (see also \citealt{Donohue.2005b}; \citeyear*{Donohue.2007b}).\footnote{\citet{Donohue.2007} refers to Papuan Malay as spoken in the area of Serui as ``\ili{Serui Malay}''.}

In a more recent study on the \ili{Melanesian} influence on Papuan Malay, \citet{Donohue.2011} investigates pronominal agreement, aspect marking, serial \isi{verb} constructions, and various aspects of clause linkage in Papuan Malay.



In addition to these more in-depth studies on Papuan Malay, initial research has been conducted on a variety of different topics. \citet{Burung.2004} examines comparative constructions in Papuan Malay. \citet{Burung.2005} discusses three types of textual continuity, namely topic, action, and thematic continuity. \citet{Burung.2007} describe different types of \isi{causative} constructions. \citet{Burung.2008} presents a brief \isi{typological profile} of Papuan Malay. \citet{Burung.2008b} investigates how Papuan Malay expresses the semantic prime FEEL, applying the Natural Semantic Metalanguage (NSM) framework. {\citet{Lumi.2007}} investigates similarities and differences of the plural personal pronouns in Ambon, Manado, and Papuan Malay. {\citet{Sawaki.2004}} discusses serial \isi{verb} constructions and word order in different clause types, and gives an overview of the pronominal system. {\citet{Sawaki.2007}} investigates how Papuan Malay expresses passive voice. {\citet{Warami.2005}} examines the uses of a number of different lexical items, including selected interjections and conjunctions.



Other materials on Papuan Malay mentioned in the literature but not consulted by the author are the following (listed in alphabetical order): \citegen{Donohue.1997} study on contact and change in Papuan Malay as spoken in Merauke,\footnote{{\citet{Donohue.1997} refers to Papuan Malay as spoken in the Merauke area as ``\ili{Merauke Malay}''.}\\} \citegen{Hartanti.2008} analysis of SMS texts in Papuan Malay, \citegen{Mundhenk.2002} description of final particles in Papuan Malay, \citegen{Podungge.2000} description of slang in Papuan Malay, \citegen{Sawaki.2005} paper on nominal agreement in Papuan Malay, \citegen{Sawaki.2005b} paper \textstyleChItalic{Melayu Papua: Tong Pu Bahasa}, and \citeauthor{Silzer.1978}'s (\citeyear{Silzer.1978}; \citeyear{Silzer.1979}) \textstyleChItalic{Notes on Irianese Indonesian}.

\subsection{Sociolinguistic and sociohistorical studies}
\label{Para_1.9.3}
To date, sociolinguistic studies on Papuan Malay are scarce.



The earliest one is \citegen{Walker.1982} study on \isi{language use} at Namatota, mentioned in §\ref{Para_1.9.2}. Examining the different functions Malay and other languages have in this multilingual community, the author highlights the pervasive role of Malay in the community.



A more recent study is the sociolinguistic survey mentioned in §\ref{Para_1.3}, §\ref{Para_1.5}, and §\ref{Para_1.9.2}, which the Papuan branch of \iai{SIL International} carried out in (and around) the coastal urban areas of Fakfak, Jayapura, Manokwari, Merauke, Timika, Serui, and Sorong \citep{Scott.2008}. In the context of this study, sociolinguistic and linguistic data was collected to explore how many distinct varieties of Papuan Malay exist and which one(s) of those varieties might be best suited for language development and standardization efforts. (See also \figref{Figure_0.2} on p. \pageref{Figure_0.2}.)



Another study on Papuan Malay, mentioned in §\ref{Para_1.5}, is \citegen{Besier.2012} thesis. The author explores the role of Papuan Malay in society in terms of the language policies of the Indonesian government, as well as its role in the independence movement, in formal education, and in the church and mission organizations.



\citet{Burung.2008} discusses the issue of Papuan Malay language awareness and vitality. Unlike \citet[10–17]{Scott.2008} (see §\ref{Para_1.5}), \citet{Burung.2008} suggests that Papuan Malay is increasingly losing domains of use to \ili{Standard Indonesian} due to the increasing influence of Indonesian throughout West Papua and the lack of language awareness among Papuans. (\citealt[See also][]{Burung.2009}.)



In addition to these sociolinguistic studies, there are also three sociohistorical studies, which need to be mentioned: \citet{Adelaar.1996}, \citet{Gil.1997}, and \citeauthor{Paauw.2005} (\citeyear*{Paauw.2005}; \citeyear*{Paauw.2007}). These studies propose classifications of Malay in general and of the \ili{eastern Malay varieties} in particular, including Papuan Malay, from a sociohistorical perspective.



Focusing on the period of European colonialism, \cite{Adelaar.1996} identify three distinct sociolects of Malay: (1)  ``literary Malay'', (2)  ``\ili{lingua franca} Malay'', and (3)  ``inherited Malay''. Within this framework, Papuan Malay is classified as a ( ``Pidgin Malay Derived'') \ili{lingua franca} or trade language (\citeyear*[675]{Adelaar.1996}), as already discussed in §\ref{Para_1.2.2}.



Another,  ``tentative typology of Malay/Indonesian dialects'' is proposed by \cite{Gil.1997}. As their primary parameter, the authors propose the  ``lectal cline'', and thus distinguish between acrolectal (that is, Standard Malay/Indonesian) and basilectal (that is, nonstandard) Malay varieties (\citeyear*[1]{Gil.1997}). The basilectal varieties are further divided into varieties with and without native speakers. For the former, a classification according two parameters is proposed: (1) ethnically homogeneous versus ethnically heterogeneous and (2) ethnically Malay versus ethnically non-Malay. According to this typology, Papuan Malay is classified as an  ``ethnically heterogeneous / non-Malay'' variety (\citeyear*[1]{Gil.1997}).



A different approach is taken by \citet{Paauw.2005, Paauw.2007}. Taking into account the diglossic nature of Malay, Paauw distinguishes between  ``national languages'',  ``inherited varieties'', and  ``contact varieties''. Among the latter, \citet[2]{Paauw.2007} further differentiates four subtypes, one of them being the eastern Malay  ``nativized'' varieties. Within this framework, Papuan Malay is classified as a  ``nativized'' eastern Malay  ``contact variety'' (\citeyear*[2]{Paauw.2007}; \citealt[see also][14]{Paauw.2005}).


\section{Available materials in Papuan Malay}\label{Para_1.10}
At this point, materials in Papuan Malay are still scarce. Most of them seem to come in the form of jokes, or \textitbf{mop} ‘humor’. These jokes are published in newspapers or posted on dedicated websites, such as \textstyleChItalic{MopPapua}. Some of them are also published in book form, such as Warami’s (\citeyear*{Warami.2003}; \citeyear*{Warami.2004}) jokes collections. Humor in Papuan Malay also comes in the form of comedy, such as the sketch series \textstyleChItalic{Epen ka, cupen toh} ‘Is it important? It’s important enough, indeed!’ from Merauke, which is accessible via YouTube.\footnote{
\textstyleChItalic{MopPapua} is available at \url{https://instagram.com/moppapua/} (accessed 8 January 2016).\\
\textstyleChItalic{Epen ka, cupen toh} is available at \url{http://www.youtube.com/watch?v=IWiQK0qKIj8} (accessed 8 May 2015).
}



In 2006, the movie \textstyleChItalic{Denias} came out, a film in Papuan Malay about a boy from the highlands who wants to go to school.\footnote{%
\textstyleChItalic{Denias} is available at \url{http://www.youtube.com/watch?v=kc683zv6H_E} (accessed 8 January 2016).}



Other materials in Papuan Malay are only available on the internet, such as:



\begin{enumerate}
\item
\textstyleChItalic{Kamus Bahasa Papua} ‘Dictionary of the Papuan Language’

\begin{itemize}
\item 
A Papuan Malay – Indonesian dictionary with currently 164 items (last updated on 24 March 2011)
\item 
Online URL: \url{http://kamusiana.com/index.php/index/20.xhtml} (access\-ed 8 January 2016)

\end{itemize}

\item
\textstyleChItalic{Kitong pu bahasa} ‘Our Language’

\begin{itemize}
\item 
A Christian website in Papuan Malay, Indonesian, and English which includes information about the Papuan Malay language and its history, the books of Jonah and Ruth from the Old Testament and the Easter story from the New Testament of the Bible in PDF format, and Christian texts and songs in audio format.


\item{Online URL: \url{http://kitongpubahasa.com/en/_5699} (accessed 8 January 2016)}

\end{itemize}
\end{enumerate}

Also, mention needs to be made of a language development program launched by Yayasan Betania Indonesia, a Papuan nongovernmental organization located in Abepura, West Papua. The program’s goal is to develop written and audio resources with a focus on Bible translation, seeking to promote and develop the use of the language in the religious domain (L. Harms p.c. 2015).



An online resource providing materials on issues relevant to West Papua is ‘West Papua Web’.\footnote{`West Papua Web’ is available at \url{http://www.papuaweb.org/} (last updated in January 2012) (accessed 8 January 2016).} This resource is hosted by The University of Papua, Cenderawasih University, and the Australian National University. To date, however, the website does not provide materials in Papuan Malay.


\section{Present study}\label{Para_1.11}
This study primarily deals with the Papuan Malay language as it is spoken in the \ili{Sarmi} area, which is located about 300 km west of Jayapura. Both towns are located on West Papua’s northeast coast. The description of the language is based on 16 hours of recordings of spontaneous conversations between Papuan Malay speakers.



The following sections provide pertinent background information for the study. After discussing some theoretical considerations in §\ref{Para_1.11.1}, the general setting of the \isi{research location} \ili{Sarmi} is presented in §\ref{Para_1.11.2}. The \isi{methodological approach} and the field work are described in §\ref{Para_1.11.3}. Details on the recorded corpus and the sample of speakers contributing to this corpus are presented in §\ref{Para_1.11.4}. The procedures for the \isi{data transcription} and analysis are discussed in §\ref{Para_1.11.5}. Finally, §\ref{Para_1.11.6} describes the procedures involved in eliciting the \isi{word list}.


\subsection{Theoretical considerations}\label{Para_1.11.1}
Papuan Malay is spoken in a rich linguistic and sociolinguistic environment in the coastal areas of West Papua (see §\ref{Para_1.4} and §\ref{Para_1.5}). Many Papuans speak two or more languages which they use as deemed appropriate and necessary. That is, depending on the setting of the \isi{communicative event}, speakers may use one or the other code or switch between them.



The conversations, recorded in \ili{Sarmi} in late 2008, reveal some of this linguistic richness. They include conversations in which the interlocutors freely switch between different codes, such as Papuan Malay, \ili{Isirawa}, and Indonesian. These recordings illustrate how intertwined and close to the speakers’ minds the languages that are part of their linguistic repertoire are.



With a few exceptions, however, this description of Papuan Malay does not take into account language contact issues and therefore does not reflect the rich linguistic environment which Papuan Malay is part of. Instead, the description creates an abstraction of Papuan Malay as if it were a linguistic entity spoken in isolation, rather than spoken in the context of a larger, complex linguistic and sociolinguistic reality.



That is, in terms of  \citegen{deSaussure.1959} distinction between \textstyleChItalic{langue} and \textstyleChItalic{parole}, this description of Papuan Malay focuses on the language system as  ``a collection of necessary conventions'' (\citeyear*[9]{deSaussure.1959}). The rationale for this abstraction is twofold. First, it is needed in order to identify, analyze, illustrate, and discuss pertinent linguistic features which are characteristics of Papuan Malay and which distinguish this speech variety from others, such as other \ili{eastern Malay varieties}. Second, the abstraction is necessary in order to appreciate the complexity of Papuan Malay as \textstyleChItalic{parole}; as discussed below, however, the investigation of this complexity is beyond the scope of the present research.



It is pointed out, however, that this abstraction of Papuan Malay as \textstyleChItalic{langue} is based on natural speech or \textstyleChItalic{parole}, which represents  ``the executive side of speaking'' \citep[13, 14]{deSaussure.1959}. Moreover, Papuan Malay as \textstyleChItalic{langue} is accessible and recognized by its speakers, although not without some difficulty. Furthermore, in being extracted from a  ``heterogeneous mass of speech facts'', employing \citegen[14]{deSaussure.1959} terminology, the examples and texts presented in this book reflect at least part of the larger linguistic reality of the recorded speakers.



Given this focus on \textstyleChItalic{langue}, the present isolated analysis of Papuan Malay remains incomplete. After having extracted Papuan Malay from its complex (socio)linguistic reality, the next step in presenting an adequate linguistic description of the language needs to focus on Papuan Malay as \textstyleChItalic{parole}, with its  ``heterogeneous mass of speech facts'' \citep[14]{deSaussure.1959}. More specifically, this next step needs to consider the larger linguistic environment and the interactions between the different codes which are at the disposal of the coastal Papuan communities. This step, however, is beyond the scope of this book and is left for future research.


\subsection{Setting of the research location}\label{Para_1.11.2}
The research for the present description of Papuan Malay was conducted in \ili{Sarmi}, the capital of the \ili{Sarmi} regency (see \figref{Figure_0.3} on p. \pageref{Figure_0.3}). In the planning stages of this research, it was suggested to the author that \ili{Sarmi} would be a good site for collecting Papuan Malay language data, due to its location, which was still remote in late 2008 when the first period of this research was conducted (see also §\ref{Para_1.11.3}). It was anticipated that Papuan Malay as spoken in \ili{Sarmi} would show less Indonesian influence than in other coastal urban areas such as Jayapura, Manokwari, or Sorong.



The coastal stretch of West Papua’s north coast, where \ili{Sarmi} is located, is dominated by sandy beaches. The flat hinterland is covered with thick forest and gardens grown by local subsidiary farmers. The town of \ili{Sarmi} is situated on a peninsula, about 300 km west of Jayapura on West Papua’s northeast coast; in 2010, the town had a population of 4,001 inhabitants; the regency’s population was 32,971.\footnote{Detailed 2010 census data is available at \url{http://bps.go.id/eng/download_file/Population_of_Indonesia_by_Village_2010.pdf} (accessed 21 Oct 2013) (see also §\ref{Para_1.7.1}).}


During the first period of this research, in late 2008, it was still difficult to get to \ili{Sarmi}, as there were no bridges yet across the Biri and \ili{Tor} rivers, located between \ili{Bonggo} and \ili{Sarmi}. Both rivers had to be crossed with small ferries with the result that public transport between Jayapura and \ili{Sarmi} was limited, time-consuming, and expensive. A cheaper alternative was travel by ship, since the \ili{Sarmi} harbor allows larger ships to anchor. This was also time-consuming, as the traffic between both cities was limited to about one to two ships per week. There is also a small airport but in 2008 there were no regular flight connections and tickets were too expensive for the local population. Today, there are bridges across the Biri and \ili{Tor} rivers and public transport between \ili{Sarmi} and Jayapura is both regular and less time-consuming and expensive than in 2008.



In late 2008, the most western part of the \ili{Sarmi} regency was not yet accessible by road; the sand/gravel road ended in Martewar, 20 km west of \ili{Sarmi} town. The villages between Martewar and Webro, that is, Wari, Aruswar, Niwerawar, and Arbais, were accessible by motorbike via the beach during low tide; the villages further west, that is, Waim, Karfasia, Masep, and Subu, were only accessible by boat. Today, the coastal road extends to Webro. The villages further west are still not accessible via road. Travel to the inland villages (Apawer Hulu, Burgena, Kamenawari, Kapeso, Nisro, Siantoa, and Samorkena) is also difficult as there are no proper roads to these remote areas. Some villages located along rivers are accessible by boat. Other villages are at times accessible via dirt road, constructed by logging enterprises. After heavy rains, however, these roads are impassable for most cars and trucks.



Most of the \ili{Sarmi} regency’s Papuan population work as subsistence farmers. Employment in the public sector is highly valued, and those who have adequate education levels try to find work as civil servants in the local government offices, in the health sector, or in the educational domain. However, secondary school education is not widely available. While the larger villages west of \ili{Sarmi} have primary and junior high schools, there are no senior high schools in these villages. Hence, teenagers from families who have the financial means to pay tuition fees have to come to \ili{Sarmi}. Here, they usually live with their extended families. This also applies to the author’s host family, most of whom are from Webro (see §\ref{Para_1.11.3}).


\largerpage
Public health services are basic in the regency. There is a small hospital in \ili{Sarmi}, but its medical services are rather limited. For surgery and the treatment of serious illnesses, the local population has to travel to Jayapura. Financial and postal services are available in \ili{Sarmi} but not elsewhere in the regency. Communication via cell-phone is also possible in \ili{Sarmi} and the surrounding villages, but it is limited in the more rural areas. Many villages are still not connected to telecommunication networks, as there are not enough cell sites to cover the entire regency.

\subsection{Methodological approach and fieldwork}\label{Para_1.11.3}
The description of Papuan Malay is based on 16 hours of recordings of spontaneous conversations between Papuan Malay speakers. The corpus includes only a few texts obtained via \isi{focused elicitation}. The rationale for this \isi{methodological approach} is discussed below.



The fieldwork was conducted in West Papua in four periods between September 2008 and December 2011. The first period took place in \ili{Sarmi} from the beginning of September until mid-December 2008. During this time the texts which form the basis for the present study were recorded. The remaining three fieldwork periods took place in \ili{Sentani}, located about 40 km west of Jayapura, from early October until mid-December 2009, from mid-October until mid-December 2010, and from early September until the end of November 2011. During these periods, the recordings were transcribed, about one third of the texts was translated into English, additional examples were elicited, and \isi{grammaticality judgment} tests were conducted (see §\ref{Para_1.11.5}). During the fourth fieldwork in late 2011, the \isi{word list} was recorded (see §\ref{Para_1.11.6}), and a 150-minute extract of the corpus was transcribed more thoroughly.



During the first fieldwork I lived with a pastor, Kornelius\textsuperscript{†} Merne, his wife Sarlota\textsuperscript{†}, and three of their five children. Also living in the house were one of Sarlota’s sisters and eight teenagers (three males and five females). The teenagers were part of the extended family and came from the Mernes’ home village Webro, located about 30 km west of \ili{Sarmi}, or nearby villages, which, like Webro, belong to the Pante-Barat district. At that time, the eight teenagers were junior or senior high school students. Furthermore, there was a constant coming and going of guests from villages of the \ili{Sarmi} regency: relatives, pastoral workers, and/or local officials passing through or staying for several days up to several weeks. Hence, the household included between 14 and about 30 persons. The Mernes, their household members and many guests belonged to the \ili{Isirawa} language group (\ili{Tor}-\ili{Kwerba} language family), to which Webro and the neighboring villages belong. Some guests originated from other language areas, such as the \ili{Papuan languages} \ili{Samarokena}, \ili{Sentani}, and \ili{Tor}, or the \ili{Austronesian} languages \ili{Biak} and \ili{Ambon Malay}.



At the beginning of my stay with his family, pastor Merne had given me permission to do recordings in his house. Besides recording spontaneous conversations, I had planned to elicit different text genre such as narratives, procedurals, and expositories. This, however, soon proved to be impossible for two reasons, namely the diglossic distribution of Papuan Malay and Indonesian, and the lack of language awareness, discussed in §\ref{Para_1.5}. As a result of these two factors, it proved de facto impossible for the household members and guests to talk with me in Papuan Malay. They always switched to Indonesian. This made both \isi{focused elicitation} and language learning difficult. Therefore, after a few unsuccessful attempts to elicit texts, I decided to refrain from further elicitation and to record spontaneous conversations instead. From then on, I always carried a small recording device with internal microphone which I turned on when two or more people were conversing. After a few days the household members were used to my constant recording. I never had the impression that they were trying to avoid being recorded (there were only two situations in which speakers distanced themselves from me in order not to be recorded). Most of the sixteen hours of text were recorded in this manner, as discussed in more detail in §\ref{Para_1.11.4.1}. There are a few exceptions, though, which are also discussed in §\ref{Para_1.11.4.1}.



Given that my hosts and their guests typically switched to Indonesian when talking with me, most of my language learning was by listening to Papuans talking to each other in Papuan Malay, by applying what I observed during these conservations and in the recorded data, and by discussing these observations with those speakers who were interested in talking about language related issues. The procedures involved in transcribing and analyzing the recorded texts are described in §\ref{Para_1.11.5}.



During the fourth period of fieldwork, from the beginning of September until the end of November 2011, I recorded a 2,458-item \isi{word list} \citep{Kluge.2014}. The items were extracted from the transcribed corpus and recorded in isolation to investigate the Papuan Malay \isi{phonology} at the word level. The consultants from whom the list was recorded were two Papuan Malay speakers, Ben Rumaropen and Lodowik Aweta. The procedures involved in recording this list are described in §\ref{Para_1.11.6}. 


\subsection{Papuan Malay corpus and speaker sample}\label{Para_1.11.4}
During the first fieldwork period in late 2008, 220 texts totaling almost 16 hours were recorded. Almost all of them were recorded in \ili{Sarmi} (217/220 texts); the remaining three were recorded in Webro. The texts were recorded from a sample of about 60 different Papuan Malay speakers. The corpus is described in §\ref{Para_1.11.4.1}, and the sample of recorded speakers in §\ref{Para_1.11.4.2}.


\subsubsection[Recorded texts]{Recorded texts}
\label{Para_1.11.4.1}
The basis for the current study is a 16-hour corpus. In all, 220 texts were recorded (see Appendix \ref{Para_C}). The texts were recorded in the form of WAV files with a Marantz PMD620 using the recorder’s internal microphone. Each WAV file was labeled with a record number which includes the date of its recording, a running number for all texts recorded during one day, and a code for the type of text recorded. This is illustrated with the record number 080919-007-CvNP: 080919 stands for  ``2008, September 19''; 007 stands for  ``\isi{recorded text} \#7 of that day''; and CvNP stands for  ``Personal Narrative (NP) which occurred during a Conversation (Cv)''. The same record numbers are used in Toolbox for the transcribed texts (see §\ref{Para_1.11.5.1}) and the examples given in this book (see ‘Conventions for examples’, p. \pageref{Para_0}).

Most texts are spontaneous conversations which occurred between two or more Papuan speakers (157/220 texts – 71.4\%), as shown in \tabref{Table_1.1.14}. Details concerning the contents of these conversations are given in \tabref{Table_1.15}. The remaining 63 texts (28.6\%) fall into two groups: conversations with the author (see \tabref{Table_1.16}) and elicited texts (see \tabref{Table_1.17}). (See also Appendix \ref{Para_C} for a detailed listing of the 220 recorded texts.)
%}

\begin{table}[p]
\caption{Overview of 16-hour corpus}\label{Table_1.1.14}


\begin{tabular}{lrrrr}
\lsptoprule
 \multicolumn{1}{c}{Text types} & \multicolumn{2}{c}{Texts} & \multicolumn{2}{c}{Hours}\\\cmidrule(lr){2-3}\cmidrule(lr){4-5}
 & \multicolumn{1}{c}{Count} & \multicolumn{1}{c}{\%} & \multicolumn{1}{c}{Count} & \multicolumn{1}{c}{\%}\\
\midrule

Spontaneous conversations &  157 &  71.4 &  10:08:02 &  63.4\\
Conversations with the author &  40 &  18.2 &  04:27:15 &  27.9\\
Elicited texts &  23 &  10.4 &  01:23:17 &  8.7\\
\midrule
Total &  220 &  100 &  15:58:34 &  100\\

\lspbottomrule
\end{tabular}
\end{table}



\begin{table}[p]
\caption[Spontaneous conversations]{Spontaneous conversations\footnote{As percentages are rounded to one decimal place, they do not always add up to 100\%.}}\label{Table_1.15}

\begin{tabular}{lrrrr}
\lsptoprule
\multicolumn{1}{c}{Contents} & \multicolumn{2}{c}{Texts} & \multicolumn{2}{c}{Hours}\\\cmidrule(lr){2-3}\cmidrule(lr){4-5}
& \multicolumn{1}{c}{Count} & \multicolumn{1}{c}{\%} & \multicolumn{1}{c}{Count} & \multicolumn{1}{c}{\%} \\
\midrule
Casual conversations &  105 &  66.9 & 05:59:55 &  59.2\\
Expositories &  14 &  8.9 &  00:59:48 &  9.8\\
Hortatories &  5 &  3.2 &  00:03:48 &  0.6\\
Narratives (folk stories) &  2 &  1.3 &  00:39:45 &  6.5\\
Narratives (personal experiences) &  25 &  15.9 &  01:05:17 &  10.7\\
Phone conversations &  5 &  3.2 &  01:13:19 &  12.1\\
Procedurals &  1 &  0.6 &  00:06:10 &  1.0\\
\midrule
Total &  157 &  100 &  10:08:02 &  100\\
\lspbottomrule
\end{tabular}

\end{table}


\begin{table}[p]
\caption{Conversations with the author}\label{Table_1.16}

\begin{tabular}{lrrrrr}
\lsptoprule
 \multicolumn{1}{c}{Contents} & \multicolumn{2}{c}{Texts} & \multicolumn{2}{c}{Hours}\\\cmidrule(lr){2-3}\cmidrule(lr){4-5}
 & \multicolumn{1}{c}{Count} & \multicolumn{1}{c}{\%} & \multicolumn{1}{c}{Count} & \multicolumn{1}{c}{\%} \\
\midrule
Casual conversations &  13 &  32.5 &  01:17:05 &  28.8\\
Expositories &  17 &  42.5 &  02:10:15 &  48.7\\
Narratives (personal experiences) &  8 &  20.0 &  00:50:36 &  18.9\\
Procedurals &  2 &  5.0 &  00:09:19 &  3.5\\
\midrule
Total &  40 &  100 &  04:27:15 &  100\\
\lspbottomrule
\end{tabular}
\end{table}



\begin{table}
\caption{Elicited texts}\label{Table_1.17}


\begin{tabular}{lrrcr}
\lsptoprule
\multicolumn{1}{c}{Contents} & \multicolumn{2}{c}{Texts} & \multicolumn{2}{c}{Hours}\\\cmidrule(lr){2-3}\cmidrule(lr){4-5}
& \multicolumn{1}{c}{Count} & \multicolumn{1}{c}{\%} & \multicolumn{1}{c}{Count} & \multicolumn{1}{c}{\%} \\
\midrule
Jokes &  14 &  60.9 &  00:13:12 &  15.8\\
Narratives (personal experiences) &  7 &  30.4 &  01:06:47 &  80.2\\
Procedurals &  2 &  8.7 &  00:03:18 &  4.0\\
\midrule
Total &  23 &  100 &  01:23:17 &  100\\
\lspbottomrule
\end{tabular}
\end{table}

Most of the texts in the corpus are spontaneous conversations between two or more Papuans. While being present during these conversations, I usually did not participate in the talks unless being addressed by one of the interlocutors. The recorded conversations cover a wide range of text genre and topics. The majority of conversations are casual and about everyday topics related to family life, relations with others, work, education, politics, and religion. Five conversations were conducted over the phone. A substantial number of the recorded conversations are narratives about personal experiences such as journeys or childhood experiences. Included are also 14 expositories, five hortatories, two folk stories, and one brief procedural.\footnote{In expository discourse the speaker describes or explains a topic. In hortatory discourse the speaker attempts to persuade the addressee to fulfill the commands given in the discourse. In procedural discourse the speaker describes how to do something. \citep{Loos.2003}} In all, the corpus contains 157 such conversations (157/220 – 71.4\%), accounting for about ten hours of the 16-hour corpus (63.4\%).



The corpus also includes 40 texts which I recorded when visiting two relatives of the Merne family. Unlike the other family members and guests of the Merne household, two of Sarlota Merne’s relatives, a young female pastor and her husband who also lived in \ili{Sarmi}, had no difficulties talking to me in Papuan Malay. I visited them regularly to chat, elicit personal narratives, and discuss local customs and beliefs. In all, the corpus contains 40 such texts (40/220 – 18.2\%) (see \tabref{Table_1.16}). These texts account for about four and a half hours of the 16-hour corpus (27.9\%).


The corpus also contains 23 elicited texts (23/220 – 10\%) (see \tabref{Table_1.17}). These texts account for about one and a half hours of the 16-hour corpus (8.7\%). During the first two weeks of my first fieldwork, I elicited a few texts, as mentioned in §\ref{Para_1.11.3}. Two were short procedurals which I recorded on a one-to-one basis. Besides, I elicited three personal narratives with the help of Sarlota Merne, who was one of the few who were aware of the language variety I wanted to study and record. She was present during these elicitations and explained that I wanted to record texts in \textitbf{logat Papua} ‘Papuan speech variety’. She also monitored the speech of the narrators; that is, when they switched to Indonesian, she made them aware of the switch and asked them to continue in \textitbf{logat Papua}. Toward the end of my stay in \ili{Sarmi}, when I was already well-integrated into the family and somewhat proficient in Papuan Malay, I recorded one narrative in a group situation from one of Sarlota Merne’s sisters and another three personal narratives on a one-to-one basis from one of the teenagers living with the Mernes. Also toward the end of this first fieldwork, I recorded 14 jokes which two of the teenagers also living in the house told each other. A sample of texts is presented in Appendix \ref{Para_B}.


\subsubsection[Sample of recorded Papuan Malay speakers]{Sample of recorded Papuan Malay speakers}\label{Para_1.11.4.2}
The corpus was recorded from about 60 different speakers. This sample includes 44 speakers personally known to the author. \tabref{Table_1.18} to \tabref{Table_1.20} provide more information with respect to their language backgrounds, gender, age groups, and occupations.



The sample also includes a fair number of speakers who visited the Merne household briefly and who took part in the recorded conversations. In transcribing their contributions to the ongoing conversations, their gender and approximate age were noted; additional information on their language backgrounds or occupations is unknown, however.



\tabref{Table_1.18} presents details with respect to the vernacular languages spoken by the 44 recorded Papuan Malay speakers. Most of them are speakers of \ili{Isirawa}, a \ili{Tor}-\ili{Kwerba} language (38/44 – 86). The vernacular languages of the remaining six speakers are the \ili{Austronesian} languages \ili{Biak} and \ili{Ambon Malay}, and the \ili{Papuan languages} \ili{Samarokena}, \ili{Sentani}, and \ili{Tor}.


\begin{table}
\caption{The recorded Papuan Malay speakers by vernacular languages}\label{Table_1.18}

\begin{tabular}{lr}
\lsptoprule
 Vernacular language &  Total\\

\midrule
\ili{Isirawa} &  38\\
\ili{Ambon Malay} &  1\\
\ili{Biak} &  1\\
\ili{Samarokena} &  2\\
\ili{Sentani} &  1\\
\ili{Tor} &  1\\
\midrule
Total &  44\\
\lspbottomrule
\end{tabular}
\end{table}
\tabref{Table_1.19} gives an overview of the recorded 44 speakers in terms of their gender and age groups. The sample includes 20 males (45\%) and 24 females (55\%). Age wise, the sample is divided into three groups: 19 adults in their thirties or older (19/44 – 43\%), 20 young adults in their teens or twenties (20/44 – 45\%), and five children of between about five to 13 years of age.


\begin{table}
\caption{The recorded Papuan Malay speakers by gender and age groups}\label{Table_1.19}
\begin{tabular}{lrrr}
\lsptoprule
 \multicolumn{1}{c}{Age groups} & Males & Females &  Total\\
\midrule

Adult (thirties and older) &  10 &  9 &  19\\
Young adult (teens and twenties) &  6 &  14 &  20\\
Child (5--13 years) &  4 &  1 &  5\\
\midrule
Total &  20 &  24 &  44\\
\lspbottomrule
\end{tabular}
\end{table}

\tabref{Table_1.20} provides an overview of the speakers and their occupations. The largest subgroups are pupils (13/44 – 30\%), farmers (10/44 – 23\%), and government or business employees (5/44 – 11\%). Eight of the 13 students were the teenagers living in the Merne household. The two BA students were the Merne’s oldest children who were studying in Jayapura and only once in a while came home to \ili{Sarmi}. In addition to the ten full-time farmers, three of the government employees worked as part-time farmers. Of the total of five children, three were not yet in school; the remaining two were in primary school.

\begin{table}
\caption{The recorded Papuan Malay speakers by occupation}\label{Table_1.20}


\begin{tabular}{lr@{ }lrr@{ }l}
\lsptoprule
 \multicolumn{1}{c}{Occupation} & \multicolumn{2}{c}{Males}   & \multicolumn{1}{c}{Females} &  \multicolumn{2}{c}{Total} \\


\midrule
Farmer &  2 & (+3) &  8 &  10 & (+3)\\
Pupil (high school) &  1 & &  4 &  5& \\
Pupil (middle school) &  1&  &  5 &  6& \\
Pupil (primary school) &  2&  &  0 &  2& \\
Employee (government/business) &  5 & &   0 &  5& \\
Pastor &  2 &  &  1 &  3& \\
Child &  2 & &  1 &  3& \\
Housewife &  0 & &  2 &  2& \\
(ex-)Mayor &  2 & &   0 &  2& \\
Student (BA studies) &  1 & &  1 &  2& \\
BA graduate &  0 & & 1 &  1& \\
Church verger &  1 & &  0 &  1& \\
Nurse &  1 & &  0 &  1& \\
Teacher &  0 & &  1 &  1& \\
\midrule
Total &  24 & &  20 &  44& \\
\lspbottomrule
\end{tabular}
\end{table}
\subsection{Data transcription, analysis, and examples}\label{Para_1.11.5}
This section discusses the transcription and analysis of the recorded Papuan Malay texts. In §\ref{Para_1.11.5.1}, the procedures for transcribing and translating the recorded data are discussed. In §\ref{Para_1.11.5.2}, the procedures related to the data analysis are described, including grammaticality judgments and \isi{focused elicitation}.


\subsubsection[Data transcription and translation into English]{Data transcription and translation into English}\label{Para_1.11.5.1}
Two Papuan Malay consultants transcribed the recorded texts during the second fieldwork in late 2009 and the third fieldwork in late 2010. The two consultants were Ben Rumaropen, who was one of my main consultants throughout the entire research project, and Emma Onim.



B. Rumaropen grew up in Abepura, located about 20 km west of Jayapura; his parents are from \ili{Biak}. In 2004, B. Rumaropen graduated with a BA in English from Cenderawasih University in Jayapura. From 2002 until 2008, he worked with the SIL Papua survey team. During this time he was one of the researchers involved in the mentioned 2007 sociolinguistics survey of Papuan Malay \citep{Scott.2008}. E. Onim grew up in Jayapura; her parents are from Wamena. In 2010, E. Onim graduated with a BA in finance from Cenderawasih University in Jayapura. Since then, she has been the finance manager of a local NGO.

 
The two consultants transcribed the texts in Microsoft Word, listening to the recordings with Speech Analyzer, a computer program for acoustic analysis of speech sound, developed by \iai{SIL International}.\footnote{Speech Analyzer is available at \url{http://www-01.sil.org/computing/sa/} (accessed 8 January 2016).} B. Rumaropen transcribed 121 texts, and E. Onim 99 texts; each text was transcribed in a separate Word file. Using Indonesian orthography, both consultants transcribed the data as literally as possible, including hesitation markers, false starts, truncation, speech mistakes, and nonverbal vocalizations, such as laughter or coughing. Once a recording had been transcribed, I checked the transcription by listening to the recording. Transcribed passages which did not match with the recordings were double-checked with the consultants. After having checked the transcribed texts is this manner, I imported the Word files into Toolbox, a data management and analysis tool developed by \iai{SIL International}.\footnote{Toolbox is available at \url{http://www-01.sil.org/computing/toolbox/} (accessed 8 January 2016).} In Toolbox, I interlinearized the 220 texts into English and Indonesian and compiled a basic dictionary. Each text was imported into a separate Toolbox record, receiving the same record number as its respective WAV file (for details see §\ref{Para_1.11.4.1}).



During the second fieldwork in late 2009, B. Rumaropen and I translated 83 of the 220 texts into English, which accounts for a good five hours of the 16-hour corpus. The translated texts also contain explanations and additional comments which B. Rumaropen provided during the translation process. Appendix \ref{Para_B} presents 12 of these texts.



During the fourth fieldwork period in late 2011, B. Rumaropen transcribed a 150-min\-ute extract of the corpus more thoroughly, that is, close to phonetically. In addition to the wordlist (§\ref{Para_1.11.6}), this extract also aided in the analysis of the Papuan Malay \isi{phonology}.



The entire text material, including the recordings and the Toolbox files are archived with \iai{SIL International}. Due to privacy considerations, however, they are not publically available. The examples in this book are taken from the entire corpus; that is, examples taken from the 137 texts which have not yet been translated were translated as needed. In the examples, proper names are substituted with aliases to guard anonymity.


\subsubsection[Data analysis, grammaticality judgments, and {focused elicitation}]{Data analysis, grammaticality judgments, and focused elicitation}
\label{Para_1.11.5.2}
In early 2010, after B. Rumaropen had transcribed a substantial number of texts and we had translated the mentioned 83 texts, I started with the analysis of the Papuan Malay corpus. This analysis was greatly facilitated by the Toolbox concordance tool, in which all occurrences of a word, phrase, or construction can be retrieved. The retrieved data was imported into Word for further sorting and analysis. Another helpful feature was the Toolbox export command, which allows different fields to be chosen for export into Word, such as the text, morpheme, or speech part fields.

 
During the analysis, I compiled a list of questions about analytical issues and comprehension problems encountered in the corpus. During the third and fourth fieldwork periods in late 2010 and late 2011, I worked through these questions with Papuan Malay consultants. Most of this work was done with B. Rumaropen. I also consulted informally with other Papuan Malay speakers on various occasions.



During both fieldwork periods in 2010 and 2011, I also worked with B. Rumaropen on grammaticality judgments. That is, based on the analysis of the corpus data, I constructed sentences which I submitted to B. Rumaropen to comment upon. When I found gaps in the data, I discussed them with B. Rumaropen to establish whether a given expression or construction exists in Papuan Malay, and I asked him to provide some example sentences. Beyond these fieldwork periods, B. Rumaropen and I stayed in contact via email and Skype and continued working on grammaticality judgments and the elicitation of example sentences, as needed.



The elicited examples and the constructed sentences for grammaticality judgments were entered into a separate Toolbox database file. Where used in this grammar, these examples are explicitly labeled as  ``elicited''. All other examples are taken from the Papuan Malay corpus. Throughout this book all generic statements, both positive and negative, are based on the occurrences in the corpus, unless stated otherwise.


\subsection{Word list}\label{Para_1.11.6}
During the fourth fieldwork period in late 2011, I recorded a 2,458-item \isi{word list} with two Papuan Malay consultants, namely B. Rumaropen and Lodowik Aweta. Originally from Webro, L. Aweta was one of the young people living in the Mernes’ household during my first fieldwork in 2008. In 2011, L. Aweta was a student at Cenderawasih University.



The \isi{word list} was extracted from the compiled Toolbox dictionary. During the elicitation, B. Rumaropen provided the stimulus, while L. Aweta repeated the stimulus within one of two different frame sentences.



The frame sentences, which are given in (\ref{Example_1.15}) and (\ref{Example_1.16}), were used alternatively and served two purposes. First, I anticipated that by repeating the target word within a larger sentence, L. Aweta would potentially be less influenced by B. Rumaropen’s pronunciation. This precaution was taken in case that the pronunciations of the two consultants differed, with one being from \ili{Sentani} and the other one from \ili{Sarmi}. Second, eliciting the target word as part of a larger sentence allowed me to analyze how some of the word-final segments were pronounced when they occurred in sentence final position and when they were followed by another word. This proved especially helpful in analyzing the realizations of the plosives and the rhotic when occurring in the word-final coda position (see §\ref{Para_2.1.1.1}, §\ref{Para_2.3.1.2}, and §\ref{Para_2.3.1.3} in \chapref{Para_2}).



\begin{styleExampleTitle}
{Frame sentences for \isi{word list} elicitation}
\end{styleExampleTitle}

\ea
\label{Example_1.15}
\gll {sa} {blum} {taw} {ko} {pu} {kata} {itu,} {kata} {\_\_\_}\\ %
 \textsc{1sg}  not.yet know \textsc{2sg} \textsc{poss} word \textsc{d.dist} word  \_\_\_\\

\glt 
‘I don’t yet know that word of yours, the word \_\_\_’
\z

\ea
\label{Example_1.16}
\gll {ko} {pu} {kata} {\_\_\_} {itu,} {sa} {blum} {taw}\\ %
 \textsc{2sg} \textsc{poss} word  \_\_\_  \textsc{d.dist} \textsc{1sg} not.yet know\\

\glt 
‘that word \_\_\_ of yours, I don’t yet know (it)’
\z

B. Rumaropen recorded each elicited word in a separate WAV file, using Speech Analyzer. Subsequently, I transcribed the recorded target words as separate records in Toolbox. Each record includes the orthographic representation of the target word, its phonetic transcription, English gloss, and the word class it belongs to. The \isi{word list} is found in Appendix A. The sound files and the Toolbox database file are found in \citet{Kluge.2014}.



After having entered the target words in Toolbox, I analyzed the lexical data with Phonology Assistant. This analysis tool, developed by \iai{SIL International}, creates consonant and \isi{vowel inventory} charts and assists in the phonological analysis.\footnote{Phonology Assistant is available at \url{http://phonologyassistant.sil.org} (accessed 8 January 2016).}




The description of the Papuan Malay \isi{phonology} in \chapref{Para_2} is based on a \isi{word list} of 1,117 lexical roots, extracted from the 2,458-item list. In addition, 380 items, \isi{historically derived} by (unproductive) \isi{affixation} of Malay roots, are investigated. The corpus also includes a large number of loanwords, originating from different donor languages, such as \ili{Arabic}, \ili{Chinese}, \ili{Dutch}, English, \ili{Persian}, \ili{Portuguese}, or \ili{Sanskrit}. Hence, a sizeable percentage of the attested lexical items are loanwords. So far, 719 items of the 2,458-item \isi{word list} (29\%) have been identified as loanwords, using the following sources: \citet{Jones.2007} and \citet{Tadmor.2009} (on borrowing in Malay in general see also \citealt[151–156]{Blust.2013}). Upon further investigation, some of the 1,117 lexical roots listed as inherited Papuan Malay words may also turn out to be loanwords. In addition, the corpus includes a number of lexical items which are typically used in \ili{Standard Indonesian} but not in Papuan Malay; examples are Indonesian \textitbf{desa} ‘village’ and \textitbf{mereka} ‘\textsc{3pl}’ (the corresponding Papuan Malay words are \textitbf{kampung} ‘village’ and \textitbf{dorang}/\textitbf{dong} ‘\textsc{3pl}’, respectively). Given that these words are inherited Malay lexical items, they are not treated as loanwords in this book. However, neither are these items included in the \isi{word list} in Appendix \ref{Para_A}.
