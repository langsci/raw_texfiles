\documentclass[output=paper]{langsci/langscibook} 
\title{Igbo-English intrasentential codeswitching and the Matrix Language Frame model} 
\author{% 
 Kelechukwu Ihemere \affiliation{University of Westminster, London UK}
}
\chapterDOI{10.17169/langsci.b121.498} %will be filled in at production


\abstract{
This paper uses data from Igbo-English intrasentential codeswitching involving mixed nominal expressions to test the Matrix Language Frame (MLF) model. The MLF model is one of the most highly influential frameworks used in the study of grammatical aspects of codeswitching. Three principles associated with it, the Matrix Language Principle, the Asymmetry Principle and the Uniform Structure Principle, were tested on data collected from informal conversations by educated adult Igbo-English bilinguals resident in Port Harcourt. The results of the analyses suggest general support for the three principles and for identifying Igbo-English as a “classic” case of codeswitching. 
}


\shorttitlerunninghead{Igbo-English intrasentential codeswitching}
\maketitle
\begin{document}


% \title{[Warning: Draw object ignored][Warning: Draw object ignored][Warning: Draw object ignored][Warning: Draw object ignored][Warning: Draw object ignored][Warning: Draw object ignored][Warning: Draw object ignored][Warning: Draw object ignored][Warning: Draw object ignored][Warning: Draw object ignored][Warning: Draw object ignored][Warning: Draw object ignored][Warning: Draw object ignored][Warning: Draw object ignored][Warning: Draw object ignored][Warning: Draw object ignored][Warning: Draw object ignored][Warning: Draw object ignored][Warning: Draw object ignored][Warning: Draw object ignored][Warning: Draw object ignored][Warning: Draw object ignored][Warning: Draw object ignored][Warning: Draw object ignored][Warning: Draw object ignored]\ili{Igbo}-English Intrasentential Codeswitching and the Matrix Language Frame model}



 

% \textbf{Key words:} \ili{Igbo}, Codeswitching, Matrix Language Frame model

\section{Introduction}\label{sec:ihemere:1}

It goes without saying that codeswitching (CS) is conceivably one of the most studied and discussed outcomes of language contact. Its study has been approached from two main dimensions: (i) the sociolinguistic and (ii) the linguistic. Researchers concerned with the sociolinguistic aspects of CS tend to seek to identify patterns of occurrence of CS and the impact of social-psychological factors on this bilingual behavior. Those working from a purely linguistic dimension focus on the structural aspects of CS; their aim is to uncover the syntactic and morphological characteristics of codeswitched utterances. Here, the focus is on the purely linguistic study of CS. Several constraints have been proposed by a number of researchers interested in the linguistic study of CS.  

For instance, \citet{PoplackMeechan1998} and their associates are interested with formulating constraints on points in a sentence where CS can take place on the grounds of surface-level linear differences between the languages concerned. These researchers view restrictions on CS along the lines of dissimilarities in word order, either across clauses (intersentential) or on phrases within clauses (intrasentential). In particular, Poplack’s Equivalence Constraint (EC) is based on the premise that switching is not permitted when the syntax of two languages does not match at a potential switch point. Researchers such as \citet{Bhatt2001} have put forward many counter-examples to such models.\footnote{See also the examples from \ili{Igbo}-English CS in \sectref{sec:ihemere:4}-\sectref{sec:ihemere:6} which appear to falsify the EC.} 

Another group of researchers looks for explanations at a more abstract level than linear structure. These researchers \citep{BelaziEtAl1994,DiSciulloEtAl1986,MacSwan2009} do this by structuring their explanations along the lines of what are considered generative theories of syntax. They assert that the grammatical organization of CS can be accounted for in terms of the principles of current syntactic theories, even though these theories were initially formulated to explain monolingual data. They do not recognize any theoretical (or useful) value in recognizing the asymmetry between a Matrix Language (ML) and an Embedded Language (EL). These approaches differ from the Matrix Language Frame (MLF) model \citep{MyersScotton1993,MyersScotton2002,MyersScotton2013}.

The MLF model makes the case for a distinction between the ML and the EL. The ML plays a dominant role in shaping the overall morphosyntactic properties of codeswitched utterances. In other words, the model posits two hierarchies in reference to mixed constituents: both languages do not participate equally; only one language is the source of the abstract morphosyntactic frame. This language (and the frame) is called the ML and the other language is called the EL. This idea is formalized as the Morpheme Order Principle (MOP) and the System Morpheme Principle (SMP) of the MLF model. These are testable hypotheses referring to the existence of asymmetry between the languages implicated in CS. On the basis of these principles, only one language (the source of the frame) supplies both morpheme order and frame-building system morphemes to the frame (see \sectref{sec:ihemere:3} for more details on the MLF model). 

Accordingly, this study assesses the veracity of this notion of asymmetry in the roles played by the languages participating in CS using some \ili{Igbo}-English data involving mixed nominal expressions. The rationale for focusing on mixed nominal expressions for the analyses reported in this paper stems from the fact that the languages differ in the relative order of head (H) and complement (C) within the nominal argument phrase $-$ NP (or what is now termed determiner phrase – DP, after \citealt{Abney1987}). In so doing, this study contributes directly to research on the linguistic analysis of CS by showing: (i) through exemplifications from \ili{Igbo}-English bilingual discourse what happens to the grammatical structures when two languages are in contact in the same clause; (ii) that CS does not crash when the surface structures of two languages do not map onto each other; (iii) that CS is possible between a functional head and its complement; and (iv) that \ili{Igbo}-English is a classic case of CS. Classic CS includes elements from two (or more) language varieties in the same clause, but only one of these varieties is the source of morphosyntactic frame for the clause \citep[8]{MyersScotton2002}.

\largerpage
The rest of the paper is structured as follows. \sectref{sec:ihemere:2} introduces the type of CS that is the focus of our analysis. \sectref{sec:ihemere:3} focuses on the MLF model and its associated principles, which are tested in this paper. \sectref{sec:ihemere:4} introduces the speakers, sampling and transcription procedures, and the bilingual data. \sectref{sec:ihemere:5}-7 present the analysis and discussion of the mixed nominal expressions in \ili{Igbo}-English CS. \sectref{sec:ihemere:8} is the conclusion to the paper.  

\section{Types of codeswitching}\label{sec:ihemere:2}

It is often the case that researchers make a distinction between intersentential and intrasentential switching (see \citealt{Clyne2003}). Intersentential switching exemplifies the employment of different languages at sentence or clause boundaries, as in \REF{ex:ihemere:1}, whereas intrasentential switching involves the coding of different elements within a single clause in different languages, as in (\ref{ex:ihemere:2}-\ref{ex:ihemere:5}). All examples are from \ili{Igbo}-English CS; the switched element is in bold font.

\ea\label{ex:ihemere:1}
{\ili{Igbo} and English intersentential switching\rmfnm{}}\\
\gll le    nü  ihe  o   me-re    		[PAUSE] \textbf{Did} \textbf{you} \textbf{see} \textbf{that}?\\
     look  at  thing  he do-\textsc{ind}        ~         did  you see that\\
\glt ‘Look at what he did. Did you see that?’
\z
\footnotetext{ A list of abbreviations used in the codeswitching glosses is placed at the end of the paper.}


In \REF{ex:ihemere:1}, the example includes full sentences in both \ili{Igbo} and English. Each sentence is a single clause:  \textit{Did you see that?} is a sentence in English and it is a single clause; \textit{Le nü ihe o mere} ‘look at what he did’ is a sentence in \ili{Igbo} and it is a single clause. Within each clause there is no switching of languages, but there is switching between the clauses. This type of CS is not particularly interesting for researchers concerned with the purely linguistic analysis of CS because the two languages are not in contact in the same clause, unlike in the following examples:


\ea\label{ex:ihemere:2}
{\ili{Igbo}-English \textsc{CS}}\\
\gll ha    ga-anö     \textbf{week}  abüö  na  Abuja.\\
     they  \textsc{aux}-stay  week  \textsc{num/d}  \textsc{prep} Abuja\\
\glt ‘They will stay for two weeks in Abuja.’
\z

\ea\label{ex:ihemere:3}
{\ili{Igbo}-English \textsc{CS}}\\
\gll   eze  anyï  bïa-ra        gbaghee  \textbf{hospital}  ahü.  \\
       chief   our  come-\textsc{ind}  open        hospital  \textsc{dem/d}\\
\glt   ‘Our chief came and opened that hospital.’
\z

\ea\label{ex:ihemere:4}
{\ili{Igbo}-English \textsc{CS}}\\
\gll   ï-ga-abia  ï-hü  \textbf{councilor}  ha  na  \textbf{opening} \textbf{ceremony} a?\\
       \textsc{cl-aux}-come  \textsc{inf}-see  councilor  \textsc{prn/d}  \textsc{prep}  opening ceremony \textsc{dem}\\
\glt   ‘Will you come to meet their councillor at the opening ceremony?’
\z

\ea\label{ex:ihemere:5}
{\ili{Igbo}-English \textsc{CS}}\\
\gll ha    küda-ra     \textbf{booth} dum   na    \textbf{polling} \textbf{station} ahü.\\
     they break-\textsc{ind}  booth  \textsc{q}  \textsc{prep}  polling station   \textsc{dem/d}\\
\glt ‘They broke all (the) booths at that polling station.’   
\z

In this study, we shall focus on mixed nominal expressions within a single clause, as illustrated in (2-5), and shall take as our unit of analysis the bilingual clause. This will be defined as a clause containing one or more morphemes from more than one language.

\section{The Matrix Language Frame model}\label{sec:ihemere:3}

The MLF model was first articulated by Myers-Scotton in her book \textit{Duelling languages} in 1993. The model posits that the key to understanding feature mismatches in CS is to recognize one of the asymmetries in language that is especially evident in CS: structural conflicts are resolved in favor of only one of the participating languages. According to \citet[72]{JakeEtAl2002}, the model captures this generalization in its theoretical assumptions about the nature of linguistic competence and also about operations involved in language production. This view is conceptualized under what they term a Uniform Structure Principle (USP) and its corresponding two hierarchies that indicate how the model relates to linguistic competence. 

The USP: A given constituent type in any language has a uniform abstract structure and the requirements of well-formedness for this type must be observed whenever the constituent appears. In bilingual speech, the structures of the ML are always preferred, but some embedded structures are allowed if ML clause structure is observed \citep[8--9]{MyersScotton2002}.

When this principle is applied to bilingual speech, it gives rise to the first hierarchy, which states that in bilingual speech, the languages involved do not participate equally: one language uniformly sets the morphosyntactic frame and this frame is referred to as the ML. The second of the two hierarchies of the USP is the distinction in the MLF model between the roles of content morphemes (similar to lexical elements) and system morphemes (similar to functional elements). 

Content morphemes (e.g. nouns and verbs) are those that either assign or receive thematic roles; they refer to relations within the sentence such as whether a noun is the Agent or the Patient of the verb. Under this model, content morphemes along with one type of system morpheme called an early system morpheme, are specifically characterized as conceptually activated. \citet{MyersScotton2002} explains that \textit{conceptually activated} means that speaker pre-linguistic intentions activate (or select) content morphemes and any early system morphemes that may accompany them on the surface. This activation occurs at the first level of what is termed \textit{the mental lexicon}.

The mental lexicon is said to consist of elements called \textit{lemmas}\footnote{Lemmas are defined as the morphological and syntactic properties which a word is said to inherently possess, which determine its co-occurrence and selectional restrictions, after \citegen{Levelt1989} Speech Production model.} that are tagged for specific languages; the speaker’s intentions call up language-specific lemmas, which contain the information necessary to produce surface-level forms. Furthermore, lemmas in the mental lexicon that underlie content morphemes are directly activated through the speaker’s intention. In turn, these lemmas activate the lemmas underlying early system morphemes. Early system morphemes flesh out the meaning of the lemmas of the content morphemes that call them. These system morphemes are called “early” because of their early activation in the language production process. Examples of early system morphemes \citep[268]{MyersScotton2006} include plural markings, determiners (e.g. the definite article \textit{the} and the indefinite articles \textit{a, an} in English), and those prepositions (also called satellites) that change the meanings of phrasal verbs in certain contexts (e.g. \textit{out} as in \textit{Alice} \textbf{\textit{looks out for}}\textit{ her little brother} or \textit{through} in \textit{the actor} \textbf{\textit{ran through}}\textit{ his lines before the performance}).

The other two types of system morphemes (bridge late system morphemes and outsider late system morphemes) are called “late” because the model claims that they are not activated until a later production level, at a second abstract level that is called \textit{the formulator}. According to the model, the formulator is viewed as an abstract mechanism that receives directions from lemmas in the mental lexicon (those underlying content morphemes); the directions from the lemmas underlying content morphemes tell the formulator how to assemble larger constituents, such as combinations of noun phrases (NPs)/determiner phrases (DPs) and inflection (I)/verb phrases (VPs), resulting in a full clause. 

Regarding bridge late system morphemes, they occur between phrases that make up a larger constituent. An excellent example of a bridge is the associative or possessive element that occurs between a possessor noun (N) and the element that is possessed in a number of languages. For instance, \textit{of} is a bridge, as in \textit{the house of}\textbf{\textit{}} \textit{Gina}. Also, the model considers the possessive -\textit{s} in English to be a bridge morpheme, as in \textit{Gina’s house}. A bridge morpheme depends on the well-formedness conditions of a specific constituent in order for it to appear; such a constituent is not well-formed without the bridge morpheme. 

Outsider late system morphemes like bridge late system morphemes also satisfy well-formedness conditions. However, they are said to differ from bridges in that the presence and form of an outsider depends on information that is outside the element with which it occurs and therefore outside its immediate constituent. That information is said to come from an element in another constituent or from the discourse as a whole. \citet{MyersScotton2002,MyersScotton2006,MyersScotton2013} gives a clear example of an outsider late system morpheme as the element that shows subject-verb agreement on the verb in English. She explains that the form of the agreement marker depends on the subject. Thus, English speakers would say \textit{the dog like-s chewing bones,} but \textit{dogs like-Ø chewing bones.} The suffix \textit{-s} only occurs when there is a third person singular content element in the present tense to call that suffix; otherwise, in English, there is no suffix (Ø = ‘zero’ marker).

Crucially, the model claims that the frame-building system morphemes in mixed constituents come from only one language, the source of the ML. This theoretical notion is formalized as three testable hypotheses, claimed to be universally applicable in cases involving classic CS:

\begin{itemize}
 \item \textbf{The Matrix Language Principle (MLP)} states that it is always possible in cases of classic CS to identify the ML in a bilingual clause. The ML will be the language supplying the morphosyntactic frame of the clause \citep[8]{MyersScotton2002}. 

 \item \textbf{The Asymmetry Principle (AP)}  states that bilingual speech is characterized by asymmetry in terms of the participation of the languages involved in CS \citep[9]{MyersScotton2002}. 

 \item \textbf{The Uniform Structure Principle (USP)} states that in bilingual speech, the structures of the ML are always preferred \citep[8-9]{MyersScotton2002}.

\end{itemize}
 
We shall offer more specific details about how the principles apply to the \ili{Igbo}-English data in subsequent sections of this paper.

\section{Methodology}\label{sec:ihemere:4}
\subsection{Sampling procedure}\label{sec:ihemere:4.1}

The sampling was done through personal contacts among speakers who knew each other well and shared the same friendship network in Port Harcourt, Nigeria. Port Harcourt is the capital of Rivers State, and the metropolitan area has a population of over a million, composed of people from different parts of Nigeria, including a large \ili{Igbo} population. Hence, through pre-existing contacts in the city it was possible to recruit 50 educated adult \ili{Igbo}-English bilinguals of both sexes for the study. Thirty-eight out of the fifty speakers were university graduates, while twelve were undergraduates at the time of the fieldwork in the summer of 2011. All fifty speakers speak Nigeria Standard English (NSE) as their second language.

\subsection{The speakers}\label{sec:ihemere:4.2}

\largerpage[-1]
The speakers involved ranged in age from 20 to 55, including secondary school teachers (N = 15), engineers (N = 10), physicians (N = 4), nurses (N = 5), business owners (N = 4), and undergraduate students (N = 12). \tabref{tab:Ihemere:1} summarizes the occupation and sex of the speakers. 
 
\begin{table}
\caption{Distribution of speakers according to occupation and sex}
\label{tab:Ihemere:1}

\begin{tabularx}{.66\textwidth}{Xrrr}
\lsptoprule
Occupation & Male & Female & Total\\
\midrule
Teachers & 6 & 9 & 15\\
Engineers & 9 & 1 & 10\\
Physicians & 4 & - & 4\\
Nurses/midwives & - & 5 & 5\\
Business owners & 4 & - & 4\\
Undergraduates & 6 & 6 & 12\\
\midrule
Total & 29 & 21 & 50\\
\lspbottomrule
\end{tabularx}

\end{table} 

All the speakers were born and grew up in Nigeria and have \ili{Igbo} as their first language. They are bilingual, being native speakers of \ili{Igbo} and proficient in English. The second language was learned at school age. Having studied in Nigeria, where English is the official language and the language of all advanced education, the speakers have all been educated almost entirely in English. To ensure their anonymity, the speakers’ real names are not used in the examples; where names appear in the examples, these are pseudonyms.

\subsection{Data collection and transcription procedure}

A number of studies on the grammatical aspects of CS still rely on data sourced from grammaticality judgements of native speakers (see \citealt{MacSwan2009}). The problem with such studies is that the judgements of two native speakers may vary with respect to a particular utterance; and as \citet{Daller1991} noted much earlier, such judgements tend to reflect attitudes towards language mixture rather than grammaticality or otherwise. This is mostly the case in communities where CS is stigmatized – most speakers have been known to judge CS as ungrammatical or to deny the very utterances they have been captured using, documented in recordings (\citealt[99]{MacSwan1999}). Nevertheless, most studies on the grammar of CS continue to be based mainly on corpora of authentic everyday speech (as in \citealt{MyersScottonJake2014}). The present study is based primarily on a corpus of naturally occurring speech involving the fifty men and women described in \sectref{sec:ihemere:4.1} and \sectref{sec:ihemere:4.2}. Specifically, the corpus consists of naturally occurring group conversations among members of a friendship network. From twenty minutes to one hour of tape-recorded informal conversations involving each of the speakers were recorded by this researcher in the summer of 2011. 

The transcriptions use the normal orthography of \ili{Igbo} and English, respectively. However, after \citet{Echeruo1998}, instead of using subscript dots (.) for the three \ili{Igbo} closed vowels \textit{i, o}, and \textit{u}, we will use umlauted symbols (ï; ö; ü), which are readily available on standard word processing software. Also, since there is no instance in which “ch” is in complementary distribution with “c” in \ili{Igbo}, we will use “c” in all \ili{Igbo} words with a sound similar to the voiceless palato-alveolar affricate [\textstyleipa{ʧ}]. 

\subsection{The bilingual data}

The resulting corpus contains substantial examples of different types of CS. However, this paper is concerned with the following CS structures attested in the \ili{Igbo}-English data:

\begin{itemize}[noitemsep]
 \item Singly occurring embedded language (EL) nouns (Ns)/noun phrases (NPs) in mixed determiner phrases (DPs) (N = 1057/1599), as in (\ref{ex:ihemere:2}--\ref{ex:ihemere:5}) above. 

 \item Multi-word nominal sequences framed by a Matrix Language (ML) element (N = 192/1599), as in (\ref{ex:ihemere:4}--\ref{ex:ihemere:5}) above.

 \item Singly occurring English EL Ns/NPs + \ili{Igbo} ML Ns/NPs in genitive/associative constructions (N = 165/1599), as in \REF{ex:ihemere:6}.
 
\newpage 
\ea\label{ex:ihemere:6}
{\ili{Igbo}-English \textsc{CS}}\\
\gll a-si    na  a-ga-eme   \textbf{wedding}  Ngozi  ma   ö  gbakee.\\
     \textsc{cl}-said  \textsc{c}  \textsc{cl-aux}-do  wedding  Ngozi  \textsc{c}  she  recovers\\
\glt ‘They said that they will hold Ngozi’s wedding when she recovers.’
\z

\item Singly occurring English EL Ns/NPs + \ili{Igbo} ML adjectives (N = 73/1599), as in \REF{ex:ihemere:7}.
 

\ea\label{ex:ihemere:7}
{\ili{Igbo}-English \textsc{CS}}\\
\gll ö  na-cö    ï-zü    \textbf{portmanteau}  öhüü. \\
     she  \textsc{aux}-want  \textsc{inf}-buy  portmanteau  new\\
\glt ‘She wants to buy (a/the) new portmanteau.’
\z
\item English EL single Ns that occur as bare forms in otherwise \ili{Igbo} utterances (N = 112/1599), as in \REF{ex:ihemere:8}.


\ea\label{ex:ihemere:8}
{\ili{Igbo}-English \textsc{CS}}\\
\gll ha    fe-re    \textbf{exam}  na   Abuja.\\
     they  pass-\textsc{ind}  exam  \textsc{prep}  Abuja \\
\glt ‘They passed (the) exam in Abuja.’
\z

\end{itemize}

In the sections that follow, we test the application of the three principles of the MLF model outlined in \sectref{sec:ihemere:3}. This will involve exemplification and illustration of the principles, followed by the results of a quantitative analysis relating to each principle. 

\section{Testing the MLP}\label{sec:ihemere:5}

If the principle holds, it will be possible to identify a ML in all bilingual clauses in the \ili{Igbo}-English data by specific criteria (rather than just assume that \ili{Igbo} is necessarily the ML). Two specific criteria will be employed to identify the ML of each bilingual clause: (i) the morpheme order criterion; and (ii) the system morpheme criterion. These two criteria follow from two additional principles, the AP and the USP, which are tested in \sectref{sec:ihemere:6} and \sectref{sec:ihemere:7}.

\subsection{The morpheme order criterion}\label{sec:ihemere:5.1}

The morpheme order criterion follows the Morpheme Order Principle (MOP), which predicts that in ML+EL constituents consisting of singly occurring EL lexemes and any number of ML morphemes the surface morpheme order will be that of the ML \citep[59]{MyersScotton2002}. 

To operationalize this criterion, we will interpret it to mean that it will be applicable wherever the two languages involved in CS have distinct surface orders. This is true of \ili{Igbo} and English, since they differ in the relative order of head (H) and complement (C) within the nominal argument phrase (NP/DP). The usual order in \ili{Igbo} is C followed by H rather than the H–C order of English. To illustrate this difference in the configuration of the NP/DP in both languages, consider the monolingual \ili{Igbo} sentence in \REF{ex:ihemere:9}:

\ea\label{ex:ihemere:9}
{Igbo}\\
\gll ha    bi  na   ülö  öhüü  ahü.\\
     they  live  \textsc{prep}  house  new  that\\
\glt ‘They live in that new house.’
\z

In \REF{ex:ihemere:9}, we observe what is typical within the \ili{Igbo} DP: both the adjective (A) \textit{öhüü} ‘new’ and determiner (D) \textit{ahü} ‘that’ are post-posed to the nominal element (N) \textit{ülö} ‘house’; the reverse order is usually the case in English. 

Also, in \ili{Igbo} a N can follow another N to form a genitival relationship, as in \REF{ex:ihemere:10}: 

\ea\label{ex:ihemere:10}
{Igbo}\\
\gll ö    na-agba   igwe    Kanye.\\
     \textsc{3sg}    \textsc{aux}-ride  bicycle    Kanye\\
\glt ‘He rides Kanye’s bicycle.’
\z

The situation in \REF{ex:ihemere:10} is different from that of a language like English, where usually only the N in the genitive case is inflected. \ili{Igbo} Ns are neither declined for case nor inflected for number like those of English. Therefore, in constructions like \REF{ex:ihemere:10}, it is the genitival N which comes second in the \ili{Igbo} NP (see \citealt{Emenanjo1978}; \citealt{Uwalaka1997}). We can illustrate how the morpheme-order criterion would apply to utterance \REF{ex:ihemere:3}, repeated here as \REF{ex:ihemere:11}:

\ea\label{ex:ihemere:11}
{\ili{Igbo}-English CS}\\
\gll eze    anyï  bïa-ra        gbaghee  \textbf{hospital}  ahü.  \\
     chief   our  come-\textsc{ind}  open        hospital    \textsc{dem/d}\\
\glt ‘Our chief came and opened that hospital.’
\z

In this example the EL noun complement \textit{hospital} precedes its \ili{Igbo} D head \textit{ahü} ‘that’, reflecting \ili{Igbo} complement-head (C-H) order and we would thus identify the mixed DP \textit{hospital}\textbf{\textit{}} \textit{ahü} as following \ili{Igbo} order. A similar conclusion would be reached in the case of \REF{ex:ihemere:7}.

In \REF{ex:ihemere:7} we observe that the \ili{Igbo} true adjective \textit{öhüü} ‘new’ is post-posed to the EL noun \textit{portmanteau}. This configuration is in sharp contrast with the situation in English, where the order is reversed. Therefore, we would identify the mixed constituent as following \ili{Igbo} order. 

A few of the examples involve two Ns in genitival relationship, as in \REF{ex:ihemere:6} with \textit{wedding Ngozi} ‘Ngozi’s wedding’. 

Firstly, we note that English also allows an analytic type of genitive (e.g. ‘the wedding of Ngozi’) alongside the synthetic type. However our two languages differ in the following ways: (i) in \ili{Igbo}, the N+N genitive construction does not make use of a bridge morpheme (like \textit{of}) to link the two Ns/NPs; and (ii) \ili{Igbo} N+N genitive constructions do not include the use of overt determiners; if determiners are used at all, they are always post-posed to the nominal elements. Secondly, looking at the bilingual genitive construction in \REF{ex:ihemere:6}, we observe that unlike what obtains in English, where usually only the N in the genitive case is inflected, in \ili{Igbo}, the preceding N is said to be in a pre-genitival position \citep{Uwalaka1997}, while the second N is the possessor. On the strength of this evidence, we will conclude that the word order in \REF{ex:ihemere:6} reflects that of \ili{Igbo}. 

At first glance, the pre-posed \ili{Igbo} N \textit{nnukwu} ‘big/bigness’ in \REF{ex:ihemere:12a} appears to pose a problem for the morpheme order criterion:

\ea\label{ex:ihemere:12}
\ea\label{ex:ihemere:12a} \ili{Igbo}-English CS\\
\gll obodo  anyï  nö    na  nnukwu  \textbf{trouble}.\\
     country  our    \textsc{be}  \textsc{prep} big     trouble\\
\glt ‘Our country is in big trouble’

\ex\label{ex:ihemere:12b} \ili{Igbo} \\
\gll [\textsubscript{NP}[\textsubscript{N}~nnukwu]  [\textsubscript{N}~nsogbu]] {\textasciitilde} [\textsubscript{NP}[\textsubscript{N}~nsogbu] [\textsubscript{N}~nnukwu]]  \\
	    {\hphantom{[\textsubscript{NP}[\textsubscript{N}~}}big/bigness  {\hphantom{[\textsubscript{N}~}}trouble   {}                {\hphantom{[\textsubscript{NP}[\textsubscript{N}~}}trouble       {\hphantom{[\textsubscript{N}~}}big/bigness    \\
\glt ‘big trouble’
\z
\z

It is important, however, to point out that the \ili{Igbo} word \textit{nnukwu} is described by \citet[47-8]{Emenanjo1978} and \citet[237]{MadukaDurunze1990} as a ‘qualifactive’ noun. These \ili{Igbo} grammarians argue that the \ili{Igbo} true adjectives occur only post-nominally, as in \REF{ex:ihemere:7}. Notably, while the \ili{Igbo} qualifactive nouns functioning as adjectives can occur pre-/post-nominally as in \REF{ex:ihemere:12b}, in English, adjectives typically occur pre-nominally within DP. Therefore, we can submit that when \ili{Igbo} Ns are used as adjectives, as in \REF{ex:ihemere:12a}, they behave like the adjectives found in English which typically occur pre-nominally because they are in what may be termed associative constructions. Since the surface word order of the mixed NP in \REF{ex:ihemere:12a} is compatible with that of both languages, we have coded all instances (N = 37/1599) in the data corpus represented by this example as ‘either’ according to the morpheme order criterion. 

Another seemingly problematic case for identifying morpheme order in the \ili{Igbo}-English data involves English NP compounds framed by a post-posed \ili{Igbo} functional element; example \REF{ex:ihemere:5} is repeated here as \REF{ex:ihemere:13}:

\ea\label{ex:ihemere:13}
{\ili{Igbo}-English CS}\\
\gll ha    küda-ra  \textbf{booth} dum   na    \textbf{polling} \textbf{station} ahü.\\
     they break-\textsc{ind}  booth  \textsc{q}  \textsc{prep}  polling station   \textsc{dem/d}\\
\glt ‘They broke all (the) booths at that polling station.’   
\z

The EL NP \textit{polling station}\textbf{\textit{}} shows a structural dependency relation that makes it well-formed in English. For instance, \textit{station} heads the nominal sequence pre-modified by the N \textit{polling} that denotes the type of station. \citet{MyersScotton2002} argues that such examples do not pose a problem for the MLF model since the other elements surrounding the EL materials follow the MOP. That is, we agree with \citet[139]{MyersScotton2002} that such phrases do not pose a problem for the MLF model because the EL multi-word nominal sequence is part of a full DP headed by the post-posed \ili{Igbo} demonstrative determiner \textit{ahü} ‘that’ in \REF{ex:ihemere:13}. Thus, with the postposed \ili{Igbo} functional element, the full DP now has a C-H surface word order in-line with \ili{Igbo} grammar. 

Lastly, we consider the case of English Ns which occur in \ili{Igbo} utterances with zero (Ø) determiners, as in ‘(the) exam’ of example \REF{ex:ihemere:8}. 

In \REF{ex:ihemere:8}, the NP \textit{exam} seems to express some kind of specific reference but without using any determiner. In other words, the NP appears in a context that requires the use of an overt determiner obligatorily in English, but not in \ili{Igbo}. This claim is supported by the presence of a pre-posed determiner in the monolingual English translation of \REF{ex:ihemere:8}. According to \citet[106]{MyersScottonJake2001}, EL bare forms are content morphemes that occur in a mixed constituent frame prepared by the ML, but that is missing some or all of the required ML system morphemes. Therefore, a compromise strategy is activated and used with the result that the EL content morpheme is not placed in a slot projected by its ML counterpart; rather, it is realized as a bare form or as a part of an EL island. 

We, however, disagree with this explanation for why EL bare forms occur with respect to \ili{Igbo}-English CS. Firstly, the EL N \textit{exam} in \REF{ex:ihemere:8} is the direct equivalent of its \ili{Igbo} counterpart which occurs as a bare form in similar contexts. For instance, the English N \textit{exam} is congruent with its \ili{Igbo} counterpart \textit{ule} in \textit{Ha fe-re ule na Abuja} ‘They passed (the) exam in Abuja’. Secondly, the EL N is not inserted with any noticeable compromise strategy either as suggested by \citet[106]{MyersScottonJake2001}. Instead, the N occurs in exactly the same position as its \ili{Igbo} counterpart. In fact, \ili{Igbo} already permits ‘null determiners’ in its grammar (see \citealt[64--65]{Obiamalu2013}). Given this state of affairs, an alternative explanation can be proffered for the occurrence of EL bare forms in \ili{Igbo}-English CS. 

As a first step, we must account for the variation observed in the bilingual determinate DPs in \REF{ex:ihemere:2}--\REF{ex:ihemere:5} which follow \ili{Igbo} C-H order. By adopting the DP-analysis of mixed nominal expressions in \ili{Igbo}-English CS, which assumes that the NP is headed by a functional element, the structures where the N/NP precedes the D seem problematic for a theory that assumes that the functional head is higher in the structure and has scope over the NP which it c-commands. According to \citet{Kayne1994}, heads must always precede their associated complement position, even though the surface word order in some languages may be H-C (e.g. English) and in some others C-H, as in \ili{Igbo}. Under this view, in languages like \ili{Igbo}, the C is said to undergo left adjunction to the specifier (Spec) position. The claim is that the universal ordering between a head and its dependents is Spec-Head-Complement, as represented in \REF{ex:ihemere:14}:

\ea\label{ex:ihemere:14}
\begin{forest} baseline
     [XP
	[Spec,name=Spec]    [Xˈ[X\textsuperscript{0}][YP]]]
\end{forest}
\z

If so, then the bilingual determinate DPs discussed earlier could be analyzed as having the structure in \REF{ex:ihemere:15}. The structure in \REF{ex:ihemere:15} says that the bilingual determinate DPs are headed by a functional head that takes an NP as C. The NP complement moves to the Spec position in surface syntax giving rise to the C-H order. For instance, the mixed DP \textit{hospital}\textbf{\textit{}} \textit{ahü} in \REF{ex:ihemere:11} will have the structure in \REF{ex:ihemere:16}.

\ea\label{ex:ihemere:15}
\begin{forest} baseline
	 [FP
		 [Spec]    [Fˈ[F][NP,name=NP]]
     ]
\draw[->,dashed] (NP.south) -- ++(0,-.5\baselineskip) -| (Spec.south);     
\end{forest}
\z 

\ea\label{ex:ihemere:16} 
\begin{forest} baseline
	[DP
	[Spec[\textbf{hospital},tier=word]][Dˈ                                                 
		[D[ahü,tier=word]][NP[t\textsubscript{i},tier=word]]]]                                                   
\end{forest}
\z

In \REF{ex:ihemere:16}, the N ‘hospital’ is shown to move into its surface position where it appears before the demonstrative D \textit{ahü} ‘that’; thus, creating two possibilities: first, the N head could move to the head of the functional category in a head to head movement; or second, the NP could move to the Spec position of the functional projection (FP). Given that there is no agreement morphology between the N and the associated functional category in \ili{Igbo}, we assume the latter for the bilingual determinate DPs, as illustrated in \REF{ex:ihemere:15} and \REF{ex:ihemere:16}. Consequently, to maximize structural symmetry between determinate and indeterminate nominals, we shall assume that the latter are DPs headed by a following null determiner, in line with \ili{Igbo} grammar. If our supposition about the determinate DPs is correct, then the bare EL form in \REF{ex:ihemere:8} will have the structure in \REF{ex:ihemere:17}. 

\ea\label{ex:ihemere:17} 

\begin{forest} baseline
	[DP
		[Spec[\textbf{exam},tier=word]]
		[Dˈ[D[Ø,tier=word]][NP[ t\textsubscript{i}, tier=word]]]
	]
\end{forest}
\z

Bare nominals can be interpreted as definite, indefinite or generic, which are features associated with the functional category D \citep{Radford2004}. Therefore, we conclude that in languages like \ili{Igbo} where there are bare nominals, there is a related null D head which carries the D-features. Moreover, the same analysis can be applied to account for the variation observed in all the mixed nominal expressions presented in this study. 

Next, we consider our second criterion for identifying the ML of a bilingual clause.

\subsection{The system morpheme criterion}

The system morpheme criterion follows from the System Morpheme Principle (SMP), which predicts that in ML+EL constituents, all system morphemes which have grammatical relations external to their head constituent will come from the ML \citep[59]{MyersScotton2002}. It is immediately apparent from the way the system morpheme criterion is stated that all system morphemes are not the same in terms of whether they have grammatical relations external to their heads. Recall that in \sectref{sec:ihemere:3} we pointed out that the MLF model makes a distinction between content versus system morphemes. This distinction is on the grounds that content morphemes, such as the lone English origin noun \textit{hospital} in \REF{ex:ihemere:11} and the \ili{Igbo} verb \textit{bïa} ‘come’ in the same example, can both assign or receive thematic roles, whereas system morphemes do not assign or receive thematic roles. Furthermore, system morphemes subdivide into \textsc{early} versus \textsc{late} system morphemes, according to whether or not they are conceptually activated or directly linked to the speaker’s intentions (see already \sectref{sec:ihemere:3}). 

The \textsc{early} versus \textsc{late} distinction is predicated on assumptions about how early or late in the language production process the relevant morphemes are accessed. Hence, Myers-Scotton suggests that the lemmas underlying early system morphemes, like the post-posed \ili{Igbo} demonstrative D \textit{ahü} ‘that’ in \REF{ex:ihemere:11}, are activated when the lemmas supporting content morphemes (such as the EL N \textit{hospital}) point to them. These indirectly elected lemmas further realize the conceptual content of the semantic/pragmatic feature bundles. For example, in \REF{ex:ihemere:11} \textit{ahü} adds definiteness/specificity to its complement \textit{hospital}. In other words, the same semantic/pragmatic feature bundle activates both \textit{ahü} and \textit{hospital.} 

Late system morphemes are then divided into bridge and outsider late system morphemes, the latter being co-indexed with forms outside the head of their maximal projection \citep[75]{MyersScotton2002}, while the former are not. That is, outsider late system morphemes are the system morphemes mentioned in \sectref{sec:ihemere:3} which have grammatical relations external to their head. Examples are subject-verb agreement, clitics and case affixes. For our purposes we shall re-define outsider late system morphemes in terms of such grammatical categories as auxiliary verb, tense, aspect, mood, and sentence negation, which are associated with the verb in both languages; rather than in terms of relations outside a morpheme’s maximal projection. This is because the concept of maximal projection tends to be theory-specific (cf. \citealt{Fukui2001}), and also in the case of an analytic language like \ili{Igbo}, there is no agreement morphology between verb and subject. Additionally, these grammatical categories are perhaps the most frequent kind of “outsider late morphemes” that one can find in both languages, since they occur in most clauses which have finite verbs. Thus, the language source of the earlier mentioned grammatical categories in bilingual clauses containing the mixed nominal expressions should enable us to identify the ML, and this criterion should lead to the same result for each clause as the morpheme-order criterion discussed above. To take an example, the ML of (5/13) was identified as \ili{Igbo} in \sectref{sec:ihemere:5.1} according to the morpheme order criterion.

We can see that (5/13) is also \ili{Igbo} according to the system morpheme criterion, since the verbal inflectional morpheme \textit{-ra} (the affirmative indicative past tense suffix) on ‘break’ also comes from \ili{Igbo}. The same conclusion can be reached in the additional examples from \ili{Igbo}-English CS below: 

\ea\label{ex:ihemere:18}
\gll ndï    INEC  wepüta-ra  \textbf{election} \textbf{results}   dum  na  TV.\\
     people.of  INEC  bring.out-\textsc{ind}   election results  all  \textsc{prep}  TV\\
\glt ‘The INEC released all the election results on TV.’    
\z

\ea\label{ex:ihemere:19}
\gll \textbf{returning} \textbf{officer}  ahü       a-bü-ghï        onye    iberibe.\\
     returning officer     \textsc{dem}  \textsc{v-be-neg}  person  stupid\\
\glt ‘That returning officer is not a stupid person.’
\z

\ea\label{ex:ihemere:20}
\gll ma  \textbf{ceremony}   ahü  fu-ru      n̩nukwu   ego.\\
     but ceremony  \textsc{d}  cost-\textsc{ind}  big    money\\
\glt ‘but that ceremony cost a lot of money’
\z

\ea\label{ex:ihemere:21}
\gll \textbf{election}  afö  a  a-dï-ghï  mfe  m’ölï.\\
     election    year  \textsc{d}  \textsc{v-be-neg}  easy  at.all\\
\glt ‘This year’s election is not easy at all.’
\z

As can be observed in \REF{ex:ihemere:18}-\REF{ex:ihemere:21}, all the verbal inflectional morphology (defined as outsider late system morphemes in this study) come from the language determined as the matrix language in \sectref{sec:ihemere:5.1}, namely \ili{Igbo}. In addition, the nominal compounds in (5/13) \textit{polling}\textbf{\textit{}} \textit{station}, \REF{ex:ihemere:18} \textit{election results}, and \REF{ex:ihemere:19} \textit{returning officer} are termed internal EL islands. According to \citet[265]{MyersScotton2006}, internal EL islands are part of a larger phrase framed by the ML. All the EL elements qualify as islands because they are phrases within bilingual clauses (they are NPs), and their words show structural dependency relations that make them well-formed in the EL (English). However, in all the examples, the EL morphemes are part of full DPs dominated by \ili{Igbo} determiners: the demonstrative \textit{ahü} ‘that’ (in 5/13, 19 and 20), and the quantifier \textit{dum} ‘all’ in \REF{ex:ihemere:18}. Crucially, all the mixed constituents support the morpheme order criterion because with the post-posed \ili{Igbo} determiners the full DPs now have a complement-head order. 

Also, we saw in \sectref{sec:ihemere:5.1} that it was sometimes not possible to apply the morpheme order criterion where the word order of a mixed nominal expression was compatible with both \ili{Igbo} and English (see examples \REF{ex:ihemere:12}, and \REF{ex:ihemere:22}).

\ea\label{ex:ihemere:22}
\ea \ili{Igbo}-English CS\\
\gll üfödü  \textbf{students}  a-na-ghï   a-bïa    na   oge.\\
     \textsc{q}     students  \textsc{v-be-neg}  \textsc{v}-come/arrive  \textsc{prep}  time   \\
\glt ‘Some students do not arrive on time.’
\ex
{Igbo} \\
\gll [\textsubscript{DP}[\textsubscript{Q}~üfödü]  [\textsubscript{N}~ümü\_akwükwö]] {\textasciitilde} [\textsubscript{DP}[\textsubscript{N}~ümü\_akwükwö]  [\textsubscript{Q}~üfödü]]\\
    {\hphantom{[\textsubscript{DP}[\textsubscript{Q}~}}some %
    {\hphantom{[\textsubscript{N}~}}students %
       ~ %
    {\hphantom{[\textsubscript{DP}[\textsubscript{N}~}}students %
    {\hphantom{[\textsubscript{Q}~}}some\\
\glt ‘some students’
\z
\z

In examples like \REF{ex:ihemere:12} and \REF{ex:ihemere:22} where there is no word order conflict between \ili{Igbo} and English as far as the switched element is concerned (see already discussion in \sectref{sec:ihemere:5.1} above), we shall consider the source language of the verbal inflectional morphology to identify the ML according to the system morpheme criterion only. It is equally important to underline that \textit{üfödü} ‘some’ and the numeral \textit{otu} ‘one’ are the only quantifiers in \ili{Igbo} that may pre-modify their nouns. \textit{Üfödü} is unique in that, unlike \textit{otu}, it can pre- or post-modify its noun. This is not the case in English, where \textit{some} always occurs in pre-position to the element it modifies. In fact, \citet[239]{MadukaDurunze1990} observes that the \ili{Igbo} words \textit{nnukwu} and \textit{üfödü} are qualifactive nouns which, when they precede their nouns, become emphatic in their descriptive meaning or ambiguously suggest an inherent as opposed to a descriptive meaning. Crucially, since the surface word order of the mixed constituents in \REF{ex:ihemere:12} and \REF{ex:ihemere:22} are compatible with that of both languages, we have coded all instances in the data corpus represented by these examples as ‘either’ according to the morpheme order criterion. However, examining the source language of the verbal inflectional morphology, we observe that \ili{Igbo} and not English is in charge of the outsider late system morphemes in the examples (e.g. the bridge late system morpheme \textit{nö} \textsc{be} in \REF{ex:ihemere:12} and the negative suffix \textit{-ghi} in \REF{ex:ihemere:22}). 

As it turns out, there is no case in the sample data under consideration where English supplies the morpho-syntax and \ili{Igbo} the EL elements. Therefore, if the MLP is correct, then all system morphemes in the same bilingual clause will come from the same source language; also, this source language will be the same as that identified by the first criterion, that of morpheme order (\sectref{sec:ihemere:5.1}).

\subsection{Quantitative analysis of the MLP}

In this section we present the results of a quantitative analysis conducted in order to test the MLP. The two criteria for identifying the ML were applied to each bilingual clause in order to test the MLP, according to which it is always possible in classic CS to identify the ML in a bilingual clause. The combined results are given in \tabref{tab:Ihemere:2}. 
 
\begin{table}
\caption{Identification of ML according to morpheme order and system morpheme criteria; {English nouns/NPs (N = 1599)}}

\begin{tabularx}{\textwidth}{Xp{2cm}p{2cm}rr}
\lsptoprule
Nominal expressions                                             & Source of morpheme order & Source of outsider late system morpheme & Number & \% \\
\midrule
English Ns/NPs + \\
Post-posed \ili{Igbo} Ds                   & + \ili{Igbo}   &   + \ili{Igbo} & 1249 & (78.1\%) \\
English Ns/NPs + \ili{Igbo} Ns \\
in genitival relationship             &     √     &    √       & 128 & (8\%)     \\
English Ns/NPs + \\
Post-posed \ili{Igbo} As                            &  √      &     √      &  73  &(4.6\%)   \\

English Ns/NPs with \\
zero determiner                            &  √     &      √      &  112  &(7\%)   \\

Pre-posed \ili{Igbo} Ns + \\
English Ns in associative \\
constructions   &   Either    &       √      & 37 & (2.3\%)   \\
\lspbottomrule
\end{tabularx}
\label{tab:Ihemere:2}
\end{table} 

\tabref{tab:Ihemere:2} reveals that the source language of the system morphemes utilized to mark the grammatical categories of tense, aspect and mood is \ili{Igbo} (100\%). 78.1\% (N = 1249/1599) of all English Ns/NPs occur with post-posed \ili{Igbo} determiners. This finding seems to violate the Functional Head Constraint (FHC: \citealt{BelaziEtAl1994}) and similar CS frameworks (\sectref{sec:ihemere:1}), which predict that the language feature of the complement f-selected by a functional head, like all other relevant features, \textit{must} match the corresponding feature of that functional head. As is clear from the analysis in \sectref{sec:ihemere:5.1}, switching is not blocked between a functional head (D) and its complement (N/NP) in \ili{Igbo}-English CS. 

According to \citet[59]{MyersScotton2002}, only if the terms of morpheme order and one type of system morpheme (an outsider late system morpheme) are satisfied by one and the same language, can the ML be identified as that language. On this basis, we determine that the ML of 97.7\% of the bilingual clauses is \ili{Igbo} unequivocally; whereas, the ML of only 2.3\% of the bilingual clauses is \ili{Igbo} according to just the system morpheme criterion. The finding of one language as the overwhelming source of ML in \ili{Igbo}-English CS parallels the findings in \ili{Hungarian}-English CS (\ili{Hungarian} is the ML: \citealt{Bolonyai2005}), \ili{Ewe}-Kabiye CS (\ili{Ewe} is the ML: \citealt{Essizewa2007}), and \ili{Welsh}-English CS (\ili{Welsh} is the ML: \citealt{DaviesDeuchar2010}). That is, the evidence from \ili{Igbo}-English CS confirms that the two languages do not contribute equally in the creation of mixed utterances. In the abstract interaction between the two grammars, the matrix language (\ili{Igbo}) is more activated than the embedded language (English), resulting in \ili{Igbo} contributing the morpho-syntactic structure into which English elements are inserted.  

\section{Testing the AP}\label{sec:ihemere:6}

As we highlighted in \sectref{sec:ihemere:3}, the Asymmetry Principle (AP) states that bilingual speech is characterized by asymmetry in terms of the roles played by the languages involved in CS \citep[9]{MyersScotton2002}. This asymmetry is evident in the foregoing analysis reported in \tabref{tab:Ihemere:2}. 

\begin{itemize}[noitemsep]
 \item Asymmetry in source of verb inflections (outsider late system morphemes): As the analysis reported in \tabref{tab:Ihemere:2} clearly indicates, all verb inflections (100\%) of the bilingual clauses come from only one of the participating languages, namely \ili{Igbo}.

 \item Asymmetry in the resolution of conflict in word order: Again, analysis of the sample reveals that wherever there is a conflict in word order between the two languages, as in 97.7\% (N = 1562/1599) of the examples, the order of \ili{Igbo} (the ML) prevails over that of English (the EL). 

 \item Asymmetry in the supply of content morphemes: The asymmetry in the roles played by both languages in the \ili{Igbo}-English data extends even to the supply of content morphemes (words). A morpheme count indicates that over 90\% of all the content words in the sample under consideration come from \ili{Igbo}.
 
\end{itemize}
 
 

These findings illustrate the overwhelming influence of just one language, \ili{Igbo}, as the source of the morphosyntactic frame in the CS examples. 

\section{Testing the USP}\label{sec:ihemere:7}

Recall that the Uniform Structure Principle (USP) states that in bilingual speech, the structures of the ML are always preferred (\sectref{sec:ihemere:3}). We have already seen that outsider late system morphemes can only come from the ML of a clause. However, the USP goes further to predict that other system morphemes, such as early system morphemes, which can come both from the ML and the EL, will be drawn preferentially from the ML of a bilingual clause \citep[8-9]{MyersScotton2002}. As it turns out, \ili{Igbo} (ML) contributes the overwhelming majority (98.7\%) of all the early system morphemes (e.g. demonstratives, pronominal modifiers, quantifiers, numerals, pronouns, and so on); whereas English (EL) contributes only 1.3\% of the early system morphemes in the form of the plural marker \textit{-s} on the EL N \textit{tractor+s} in \REF{ex:ihemere:23}: 

\ea\label{ex:ihemere:23}
{\ili{Igbo}-English CS}\\
\gll \textbf{tractors}  ndï    ahü  emebi-ca-la.\\
     tractors  \textsc{prn/d}    \textsc{dem/d}  damage-\textsc{encl-perf} \\
\glt ‘Those tractors have been damaged completely’
\z

It is important to underline that there is no instance in the \ili{Igbo}-English data where a lexical noun is in \ili{Igbo} and the determiner is from English. Thus, the role of the EL is limited to supplying nearly only content morphemes. Broadly, the analyses clearly show that the structures of \ili{Igbo} (the ML) are preferred in the mixed constituents discussed in this paper.

\section{Conclusion}\label{sec:ihemere:8}

In this study, we have evaluated the MLF model of CS with some \ili{Igbo}-English data and have concluded that the data can indeed be considered a classic case of CS in that a ML can be clearly identified in bilingual clauses. We have established this by a qualitative and quantitative analysis, uncovering overwhelming supportive evidence for the MLP, the AP and the USP. The findings also confirm that CS is not blocked when the surface structures of two languages do not map onto each other; additionally, CS is possible DP-internally between a functional head and its complement. 

The overall implication for the MLF model is that its predictive power lies in its recognition that there will be an asymmetry between the ML and the EL in their roles in setting the morphosyntactic frame of the bilingual clause. The regularity with which \ili{Igbo} supplies both the frame building elements (system morphemes) and sets morpheme order wherever there is a conflict in word order in \ili{Igbo}-English CS bears this out. 

Nevertheless, further research on a much larger corpus might reveal more problematic examples than we have identified in the present study. Also, an aspect of \ili{Igbo}-English intrasentential CS that this study has not touched  concerns uncovering what motivates the speakers to codeswitch in the first place. A number of reasons have been adduced in the literature as motivating factors for codeswitching. These range from social-pragmatic to grammatical considerations. Future research on \ili{Igbo}-English CS will seek to ascertain the particular motivations why \ili{Igbo}-English bilinguals engage in codeswitching.

\section*{\textbf{Abbreviations}}

\begin{tabularx}{.45\textwidth}{lX}
\textsc{aux} & auxiliary\\

\textsc{be} & copular verb\\

\textsc{cl} & clitic\\

\textsc{c} & complementiser\\

\textsc{d} & determiner\\

\textsc{dem} & demonstrative\\

\textsc{fut} & future\\
 
\textsc{ind}  &  affirmative indicative \\
\end{tabularx}
\begin{tabularx}{.45\textwidth}{lX}
\textsc{inf} & infinitive\\

\textsc{neg} & negation\\

\textsc{perf} & perfective\\

\textsc{pl} & plural\\

\textsc{prep} & preposition\\

\textsc{prn} & pronoun\\

\textsc{q} & quantifier\\
\\
\end{tabularx}

\printbibliography[heading=subbibliography,notkeyword=this]

\end{document}