\documentclass[output=paper]{langscibook} 
\author{Anna Volkova\affiliation{National Research University} \lastand Svetlana Toldova\affiliation{Higher School of Economics, Russian Federation}}

\title{Reflexivity in Kazym Khanty} 

\abstract{This paper discusses reflexivity strategies in the Kazym dialect of Khanty, an endangered Uralic language spoken in northwestern Siberia. Khanty is a language without dedicated reflexive pronouns \citep{Nikolaeva1995, Nikolaeva1999Ostyak}; to express reflexivity Kazym Khanty speakers use personal pronouns, a doubled pronoun construction or add a particle to a personal pronoun. For a closed class of verbs in Kazym Khanty detransitivising suffixes can be employed to convey the reflexive meaning. The absence of dedicated reflexive pronouns is a typological rarity, cross-linguistically they are considered the ‘norm’ \citep{Moyse-Faurie2008, HeineMiyashita2008}. The paper presents a hypothesis of how Kazym Khanty avoids exceeding anaphoric ambiguity}

\IfFileExists{../localcommands.tex}{
 \usepackage{langsci-optional}
\usepackage{langsci-gb4e}
\usepackage{langsci-lgr}

\usepackage{listings}
\lstset{basicstyle=\ttfamily,tabsize=2,breaklines=true}

%added by author
% \usepackage{tipa}
\usepackage{multirow}
\graphicspath{{figures/}}
\usepackage{langsci-branding}

  
\newcommand{\sent}{\enumsentence}
\newcommand{\sents}{\eenumsentence}
\let\citeasnoun\citet

\renewcommand{\lsCoverTitleFont}[1]{\sffamily\addfontfeatures{Scale=MatchUppercase}\fontsize{44pt}{16mm}\selectfont #1}
  
  %% hyphenation points for line breaks
%% Normally, automatic hyphenation in LaTeX is very good
%% If a word is mis-hyphenated, add it to this file
%%
%% add information to TeX file before \begin{document} with:
%% %% hyphenation points for line breaks
%% Normally, automatic hyphenation in LaTeX is very good
%% If a word is mis-hyphenated, add it to this file
%%
%% add information to TeX file before \begin{document} with:
%% %% hyphenation points for line breaks
%% Normally, automatic hyphenation in LaTeX is very good
%% If a word is mis-hyphenated, add it to this file
%%
%% add information to TeX file before \begin{document} with:
%% \include{localhyphenation}
\hyphenation{
affri-ca-te
affri-ca-tes
an-no-tated
com-ple-ments
com-po-si-tio-na-li-ty
non-com-po-si-tio-na-li-ty
Gon-zá-lez
out-side
Ri-chárd
se-man-tics
STREU-SLE
Tie-de-mann
}
\hyphenation{
affri-ca-te
affri-ca-tes
an-no-tated
com-ple-ments
com-po-si-tio-na-li-ty
non-com-po-si-tio-na-li-ty
Gon-zá-lez
out-side
Ri-chárd
se-man-tics
STREU-SLE
Tie-de-mann
}
\hyphenation{
affri-ca-te
affri-ca-tes
an-no-tated
com-ple-ments
com-po-si-tio-na-li-ty
non-com-po-si-tio-na-li-ty
Gon-zá-lez
out-side
Ri-chárd
se-man-tics
STREU-SLE
Tie-de-mann
}
  \bibliography{localbibliography}
 %\togglepaper[1]%%chapternumber
}{}

\begin{document}
\maketitle 

\section{Introduction}\label{sec:Volkova:1}

The present paper discusses reflexivity strategies in the Kazym dialect of Khanty, an endangered Uralic language spoken in northwestern Siberia. Khanty is known in the literature to be a language without dedicated reflexive pronouns \citep{Nikolaeva1995, Nikolaeva1999Ostyak}. That is true also for the Kazym dialect of Khanty: personal pronouns function as reflexive pronouns -- cf. \REF{ex:Volkova:1}:

\ea 
\label{ex:Volkova:1}
 \gll Evi-j-en λʉw-ti šiwaλ-əs-λe.\\
 girl\textsc{{}-obl-}%
\textsc{poss.2sg} (s)he\textsc{{}-acc} see\textsc{{}-pst-3sg>sg}\\
 \glt ‘The girl saw him / herself.’\footnote{In Khanty, the 2\textsuperscript{nd} person singular possessive suffix is often used in a non-possessive function to indicate discourse salience. In such uses, the link to the possessive meaning is preserved: when translating from Khanty to Russian, native speakers often convey the meaning with a 2\textsuperscript{nd} person singular pronoun. The non-possessive uses of possessive affixes in Khanty are often reminiscent of definite articles, but the correspondence is not full. Thus, their distribution and referential properties need further investigation.}
\z


A terminological note is warranted before we proceed. We use the term \textit{reflexivity} for the phenomenon where two roles in a situation are performed by the same participant. The ways a natural language encodes reflexivity are referred to as \textit{reflexivity} \textit{strategies} (e.g. reflexive pronouns,
verbal reflexive affixes). We use the term \textit{binding} for an anaphoric dependency within a sentence, especially if the antecedent is non-referential (we, however, use this term loosely and refrain from any theoretical claims as to the nature of this dependency); we reserve the term \textit{coreference} for the cross-sentential anaphoric dependencies. \textit{Local} \textit{binding} refers to an anaphoric dependency between coarguments of a verb. The term \textit{covaluation} is used as a cover term for both \textit{binding} and \textit{coreference}. We also use the term %
%reflexive possessive marker? reflexive possessive pronouns? (in general, adjectives such as “reflexive” and “possessive” should be used with a noun for clarity)
%Martin
%May 13, 2020 12:45 PM
\textit{reflexive} \textit{possessive} \textit{construction}, if the possessor of an argument is covalued with another argument in the clause. 

Kazym Khanty differs from the northern varieties of Khanty discussed in the literature: for the 3\textsuperscript{rd} person pronoun to be locally bound, the verb is not required to bear any special kind of agreement (we will address this in detail in \sectref{sec:Volkova:3}; the reverse pattern with obligatory object agreement on the verb is described for the Tegi variety in \citealt{VolkovaReuland2014} and for the Obdorsk variety in \citealt{Nikolaeva1999Ostyak}). Apart from employing personal pronouns to encode reflexivity, Kazym Khanty speakers also make use of %
%better more precise: doubled pronoun construction (or similar)
%Martin
%May 13, 2020 12:47 PM
a doubled pronoun construction or add a particle to a personal pronoun (\sectref{sec:Volkova:4}). A closed class of verbs in Kazym Khanty allows detransitivising suffixes to express reflexivity (\sectref{sec:Volkova:5}). \sectref{sec:Volkova:6} deals with reflexive possessive constructions which combine a personal pronoun and a possessive affix on the possessed noun. Different means of intensification are discussed in \sectref{sec:Volkova:7}. 

The absence of dedicated reflexive pronouns is typologically unusual, cross-linguistically they are considered the ‘norm’ \citep{Moyse-Faurie2008, HeineMiyashita2008}. We discuss how Kazym Khanty avoids exceeding anaphoric ambiguity in \sectref{sec:Volkova:8}. \sectref{sec:Volkova:9} concludes. 

The Kazym data and generalizations provided in this paper come primarily from the elicitation sessions conducted during the HSE University team field trips to Kazym (2018-2019). These examples are given below with no reference to the source. However, in illustrating language facts of Kazym Khanty we also (where possible) resort to providing examples from texts. They come from either the Western Khanty corpus collected and glossed by Egor Kashkin (WKhC) or the text corpus collected by our team during the fieldwork (KKhC).

\section{{Khanty:} {a} {profile}}\label{sec:Volkova:2}
\subsection{{Sociolinguistics}}\label{sec:Volkova:2.1}

Khanty (Ostyak) is a member of the Ob’-Ugric subgroup of the Ugric group (includes also Mansi (Vogul) and Hungarian) of the Uralic language family. It is spoken by some 9,500 people (2010 census). The ethnic population totals 28,700 people spread out over several thousand square kilometers in northwestern Siberia, Russia \citep{LewisEtAl2013} from the upper reaches of Pechora, in the northern Urals, to the Yugan, Vasyugan, and Vakh rivers in the Tomsk region. The majority of Khanty people live in the Khanty-Mansi and Yamalo-Nenets Autonomous Regions, smaller groups reside in the Tomsk region. Due to limited contact between groups of speakers, the Khanty have developed a
dialectal continuum, the opposite ends of which diverge greatly in both grammar and lexicon and are mutually incomprehensible \citep{Nikolaeva1999Ostyak}. The most commonly accepted classification of dialects goes back to \citep{Steinitz1937}. They can be subdivided into three groups: i) Eastern dialects (dialects of Vakh-Vasyugan, Surgut and Salym); ii) Southern dialects (dialects of Irtysh and Demyanka); iii) Northern dialects (dialects of Middle-Ob’, Kazym, Shuryshkary and Obdorsk). At present, the southern dialects have almost died out, the eastern dialects are highly endangered. The northern dialects are used primarily by the older generation (50+). 

The variety reported in this paper is spoken in the village of Kazym in the Beloyarsky District in the northern part of the Khanty-Mansi Autonomous Region, just to the east of the Ob’ river. Another idiom we mention is the Berezovo Khanty variety spoken in the Tegi village which is situated in the Ob’ basin.

\subsection{{Nominal} {system}}\label{sec:Volkova:2.2}

The nominal system has three cases: Nominative, Dative, and Locative. The language distinguishes three numbers: singular, dual and plural. Personal pronouns also distinguish three cases, but unlike nouns, they have dedicated affixes for Accusative and lack Locative. The pronominal system has three persons: 1\textsuperscript{st}, 2\textsuperscript{nd} and 3\textsuperscript{rd}. 

Like many other Finno-Ugric languages, Khanty employs a full set of nominal suffixes encoding number and person of a possessor on a possessed noun. A possessor expressed by a full nominal or a free personal pronoun may or may not be present in addition – see \REF{ex:Volkova:2a} and \REF{ex:Volkova:2b} respectively. In Kazym Khanty, possessive affixes are obligatory only in the case of an %
overt free personal pronoun possessor \REF{ex:Volkova:2a} and are optional otherwise\footnote{According to \citep[52]{Nikolaeva1999Ostyak}, in Khanty lexical possessors do not trigger possessive marking on the head. In contrast, in our Kazym data (including data of WKhC) we register some cases of headmarking with lexical possessors. Thus, the distribution of possessive markers in Kazym is different from that in Ob’ dialects, but es(s)lishing the exact rules of it is outside the scope of present paper.}. In \REF{ex:Volkova:2c}, in a noun phrase “Leshtan’s elder son” a possessive marker is present on the head noun “son” (2textcyrillic{с), while in a noun phrase “sister’s dress” in \REF{ex:Volkova:2d} it is absent on the head noun “dress”. λ}
\ea 
\label{ex:Volkova:2}
	\ea 
	\label{ex:Volkova:2a}
  	 \gll Ma puχ-\textbf{ɛm} / *puχ wʉn woš-ən wɵχ.\\
  	 I son-\textsc{poss.1sg} / son big town\textsc{{}-loc} live \\
  	 \glt ‘My son lives in a big town.’

	\ex
	\label{ex:Volkova:2b}
  	 \gll Akɛ-\textbf{m} tiw χăt-əmt-əs.\\
  	 uncle\textsc{{}-poss.1sg} here move\textsc{{}-punct-pst}\\
  	 \glt ‘My uncle came in.’ [WKhC, “Russian doll”]
	
	\ex
	\label{ex:Volkova:2c}
  	 \gll λeśtan-leŋke-λ wǫn poχ-\textbf{əλ} joχt-əs.\\
  	 Leshtan\textsc{{}-dim-poss.3sg} big son\MakeUppercase{{}-}\textsc{poss.3sg} come\textsc{{}-pst}\\
  	 \glt ‘The elder son of Leshtan came in’ [WKhC, “Bogatyr”]
  	 
  	 \ex
	\label{ex:Volkova:2d}
  	 \gll Ma upɛ-m jɛrnas λʉmt-s-əm.\\
  	 I sister\textsc{{}-poss.1sg} dress put.on\textsc{{}-pst-1sg}\\
  	 \glt ‘‎I put on my sister’s dress.’
	\z
\z


Possessive affixes can also be attached to postpositions \REF{ex:Volkova:3}:


\ea 
\label{ex:Volkova:3}
 \gll Ma λiw piλ{}-\textbf{aλ}{}-a kulaś-ti pit{}-λ{}-əm.\\
 I they with\textsc{{}-poss.3pl-dat} fight\textsc{{}-ipfv.part} become\textsc{{}-npst-1sg}\\
 \glt ‘I’ll fight with them!’ [WKhC, “The river land man and Ob’ river land man”]
\z


Possessive affixes in Khanty also have a number of non-possessive functions: they can mark semantic/discourse features of a noun phrase%
%“nominal” would be more neutral (the NOUN PHRASE hypothesis is contentious)
%Martin
%May 14, 2020 11:49 AM
 such as definiteness, topicality, familiarity as in \REF{ex:Volkova:4a} (see \citealt{Nikolaeva1999Ostyak, Simonenko2017,Mikhailov2018} for a detailed discussion). The 2\textsuperscript{nd} person singular possessive suffix is also used in a discourse function \REF{ex:Volkova:4b}, marking what can be roughly described as discourse salience. This is particularly frequent with person names.
%
%I would find it better if it were glossed as DEF in such cases. The 2SG gloss is confusing.
%Martin
%May 14, 2020 12:10 PM

\ea 
\label{ex:Volkova:4}
	\ea
	\label{ex:Volkova:4a}
  	 \gll I ike-λ lup-λ.\\
  	 one man\textsc{{}-poss.3sg} say\textsc{{}-npst[}3sg]\\
  	 \glt ‘One man (the river land man) says:’ [WKhC, “The river land man and Ob’ river land man”]

	\ex
	\label{ex:Volkova:4b}
  	 \gll Worŋa imi-\textbf{j-en} pa lop-t-aλ.\\
  	 raven woman\textsc{{}-obl-poss.2sg} \textsc{add} tell\textsc{{}-evid.prs-3sg}\\
  	 \glt ‘(It appears that) The (female) raven says:’ [WKhC, “The raven and the gull”]
	\z
\z

\subsection{{Verbal} {system}}\label{sec:Volkova:2.3}

Kazym Khanty distinguishes two tenses\footnote{There is also a separate paradigm for evidential forms. These forms are formed with participles in a predicative position inflected with possessive affixes for subject agreement.}: past and non-past. A verb has three %
%“argument marking patterns” would be better here (algnment is the contrast between accusative and ergative)
%Martin
%May 14, 2020 11:49 AM
argument marking patterns: subject agreement, subject-object agreement and passive. In the case of subject agreement, the verb obligatorily agrees with the subject in number (\textsc{sg}, \textsc{du}, \textsc{pl}) and person (1\textsuperscript{st}, 2\textsuperscript{nd}, 3\textsuperscript{rd}). Both intransitive \REF{ex:Volkova:5} and transitive verbs \REF{ex:Volkova:6a} attach subject agreement suffixes. 

\ea 
\label{ex:Volkova:5}
 \gll Tăm još eweλt ńɛwrem-ət aškola-j-a jăŋχ-λ-\textbf{ət}.\\
 this road from child\textsc{{}-pl} school\textsc{{}-obl-dat} go\textsc{{}-npst-}\textbf{\textsc{3pl}}\\
 \glt ‘Children go to school along this road.’
\z

 
Like other Ob’-Ugric languages, Khanty employs differential object marking. In the absence of the Accusative case marker (except for pronouns), it comes in the form of object agreement. Transitive verbs in Khanty can optionally agree in number (singular vs. non-singular) with the direct object – this is expressed by subject-object agreement paradigm \REF{ex:Volkova:6b}. According to some reference grammars (e.g. \citealt{Honti1984}), agreement with the object is licensed by the definiteness of the direct object. In Kazym, the system is more complex with aspect playing a role (see below in \sectref{sec:Volkova:2.4}).

\ea 
\label{ex:Volkova:6}
	\ea
	\label{ex:Volkova:6a}
  	 \gll Vas’a-j-en ar arij-s.\\
  	 Vasja\textsc{{}-obl-poss.2sg} song sing\textsc{{}-pst[3sg]}\\
  	 \glt ‘Vasja sang a song.’

	\ex
	\label{ex:Volkova:6b}
  	 \gll Vas’a-j-en ar-əλ arij-s-\textbf{əλλe}.\\
  	 Vas’a\textsc{{}-obl-poss.2sg} song\textsc{{}-poss.3sg} sing\textsc{{}-pst-}\textbf{\textsc{3sg>sg}}\\
  	 \glt ‘Vasja sang the/his song.’
	\z
\z

The third argument marking pattern is passive \REF{ex:Volkova:7}. The passive affix follows the tense markers on the verb, and then subject agreement affixes are attached. The logical subject is demoted to an oblique locative position. Apart from direct objects, in Kazym Khanty, Recipients and low Applicatives \REF{ex:Volkova:7} can be promoted into the subject position \citep{Nikolaeva1999Ostyak,ColleyPrivoznov2019}:
\ea 
\label{ex:Volkova:7}
 \gll (Ma) aŋk-ɛm-ən jɛrnas-ən jont-s-\textbf{aj}{}-m.\\
 I mother\textsc{{}-poss.1sg-loc} dress{}-\textsc{loc} sew\textsc{{}-pst-}\textbf{\textsc{pass}}\textsc{{}-1sg} \\
 \glt ‘My mother sewed a dress for me.’ (lit. ‘I was sewn by my mother with a dress.’)
\z


%
%dress{}-LOC
%Martin
%May 14, 2020 11:50 AM


Like Hungarian, Khanty has a rich system of detachable preverbs which are grammaticalized adverbs. Some of them have the source semantics of space relations (cf. \textit{nuχ} ‘up’, \textit{iλ} ‘down’). A number of them have developed aspectual meanings (e.g. telicity, \citealt{Kozlov2019}): 

\ea 
\label{ex:Volkova:8}
	\ea
	\label{ex:Volkova:8a}
 	 \gll Waśka-j-e kinška λʉŋt-əs.\\
  	 Vasja\textsc{{}-obl-poss.2sg} book read\textsc{{}-pst[3sg]}\\
  	 \glt ‘Vasja read the book / read the book for a while / was reading the book.’

	\ex
	\label{ex:Volkova:8b}
  	 \gll Waśka-j-en kinška nuχ λʉŋt-əs-λe.\\
  	 Vasja\textsc{{}-obl-poss.2sg} book up read-\textsc{pst-3sg>sg}\\
  	 \glt ‘Vasja read the book (to the end) / \#read the book for a while / \#was reading the book.’
	\z
\z

In \REF{ex:Volkova:8a}, the verb \textit{λʉŋtati} ‘read’ in past tense can have an atelic, a telic and a progressive meaning. In contrast, in \REF{ex:Volkova:8b} when accompanied with the preverb \textit{nuχ} this verb can have only a telic reading, the atelic reading is unavailable. 

Khanty also makes use of a number of polyfunctional verbal affixes to encode %
%better: valency{}-changing
%Martin
%May 14, 2020 11:51 AM
valency-changing operations (causative, reflexive, middle, impersonal and antipassive). This point is illustrated in \REF{ex:Volkova:9}: 

\ea 
\label{ex:Volkova:9}
	\ea
	\label{ex:Volkova:9a}
  	 \gll Aŋk-ɛm jɛrnas upe-m{}-a jɵnt-λ.\\
  	 mother\textsc{{}-poss.1sg} dress sister\textsc{{}-poss.1sg{}-dat} sew\textsc{{}-npst[}3\textsc{sg}]\\
  	 \glt ‘My mother is sewing a dress for my sister.’

	\ex
	\label{ex:Volkova:9b}
  	 \gll Aŋk-ɛm jɵnt-\textbf{əs}{}-λ.\\
  	 mother\textsc{{}-poss.1sg} sew\textsc{{}-detr-npst[3sg]}\\
  	 \glt ‘My mother is sewing.’
	\z
\z
 

Adding detransitivising suffix -\textit{əs-} to a transitive verb \textit{jɵntati} ‘sew’ \REF{ex:Volkova:9a} makes it intransitive \REF{ex:Volkova:9b}. The use of such verbal affixes is lexically restricted and not productive.

\subsection{{Clause} {structure}}\label{sec:Volkova:2.4}

Khanty is a SOV language, but the word order is relatively free \citep{Nikolaeva1999Ostyak}. Khanty employs accusative alignment. The choice between the three argument marking patterns discussed in the previous subsection depends on the information structure of the clause. Object agreement is used if the object is a “secondary topic” (this feature often correlates with definiteness of a noun phrase, see \citealt{Nikolaeva1999Agreement}). In Kazym Khanty, some speakers disfavour subject agreement on the verb if the direct object is a pronoun (disregarding whether it is bound or not) or a definite noun phrase. However, one more factor comes into play: the aspectual and actional properties of the verb \citep{Kozlov2019}. The interaction of the aspectual interpretation of the clause and the object marking on the verb is rather complicated. Roughly, a definite object and subject agreement on the verb are possible only if the clause has an imperfective reading \REF{ex:Volkova:10a}; on the other hand, with certain telic verbs the definite direct object%
%it is best to avoid abbreviations in the text; I think “direct object” can be used here
%Martin
%May 14, 2020 11:51 AM
 requires subject-object agreement under perfective interpretation \REF{ex:Volkova:10b}. Consequently, both subject and subject-object agreement patterns on the verb are compatible with a definite/pronominal direct object:
\ea 
\label{ex:Volkova:10}
	\ea
	\label{ex:Volkova:10a}
  	 \gll Petra χʉw măn-ti woχ-əs.\\
  	 Peter long.time I\textsc{{}-acc} call{}-\textsc{pst[3sg]}\\
  	 \glt ‘Peter was calling for me for a long time.’

	\ex
	\label{ex:Volkova:10b}
  	 \gll Petra măn-ti woχ-s-əλλe.\\
  	 Peter I\textsc{{}-acc} call\textsc{{}-pst-3sg>sg}\\
  	 \glt ‘Peter called me up.”
	\z
\z

Passive is a basic topic maintaining device (\citealt[30]{Nikolaeva1999Ostyak}; \citealt[35]{Koshkaryova2002}). Topic is encoded as a subject \citep{Nikolaeva1995, Nikolaeva1999Ostyak}. Thus, passive is used to promote a non-subject argument (e.g. Theme, Recipient) into the subject position under topicalization (for a more detailed discussion of passive properties see \citealt{Kiss2019, ColleyPrivoznov2019}), while focused subjects of transitive verbs are usually illicit:\footnote{Under certain conditions some speakers allow focused subjects \REF{ex:Volkova:fn1}, but such examples are rare. As for intransitive verbs, the focused wh-word \textit{χuj} ‘who’ can be used with a verb in active voice \REF{ex:Volkova:fn2a}, however, for some verb classes passive is also an option \REF{ex:Volkova:fn2b} with a low applicative being promoted into the subject position.
\ea 
	\label{ex:Volkova:fn1}
	\gll  {$X$uj} {mɛt} {χuw-a} {juwət-λ-əλe.}\\
	who most long-\textsc{adv} throw-\textsc{npst}-3sg>sg\\
	\glt ‘Who will through [this stick] the farthest’ [WKhC, “The Tale of the Priest and of His Workman Balda”]
\z

\ea
	\label{ex:Volkova:fn2}
	\ea 
	\label{ex:Volkova:fn2a}
	\gll {Jetən} {oλəŋ-a} {śi} {ji-s} {χuj} {śi} {χuwat} {muw-a} {măn-əλ.}\\
	evening begin-\textsc{dat} \textsc{foc} become-\textsc{pst} who this length land-\textsc{dat} go-\textsc{npst[3sg]}\\
	\glt ‘– It’s evening, who’ll go all the way out here?’ [WKhC, “Pashit-Wort”]
	
	\ex 
	\label{ex:Volkova:fn2b}
	\gll {Maša-j-en} {χot} {χuj-at-ən} {λuŋ-s-a.}\\
	  Masha-\textsc{obl-poss.2sg} house who-\textsc{indef-loc} enter-\textsc{pst-pass[3sg}]\\
	\glt ‘Masha’s house was entered by someone.’ (Nikita Muravyev, p.c.)
\z
\z

}



\ea 
\label{ex:Volkova:11}
	\ea
	\label{ex:Volkova:11a}
  	 \gll Tăm ar-əλ χuj-ən ari-s-a?\\
  	 this song\textsc{-poss.3sg} who\textsc{-loc} sing\textsc{-pst-pass[3sg]}\\
  	 \glt ‘Who sang this song?’ (lit. ‘By whom this song was sung’)

	\ex
	\label{ex:Volkova:11b}
  	 \gll *Xuj tăm ar-əλ ari-s(-əλλe)?\\
  	 who this song\textsc{{}-poss.3sg} sing\textsc{{}-pst-3sg>sg}\\
  	 \glt Int.: Who sang this song?
	\z
\z

Example \REF{ex:Volkova:11a} is a translation into Khanty of the sentence “Who sang this song?”: “this song” is promoted into the subject position, while the focused wh-word \textit{χuj} ‘who’ is marked by locative; if the focused wh-word occupies the subject position, the sentence is illicit \REF{ex:Volkova:11b}.

Kazym Khanty also uses subject pro-drop. In \REF{ex:Volkova:12}, the subject is expressed only on the verb, there is no overt 2\textsuperscript{nd} person pronoun in the sentence. In \REF{ex:Volkova:13}, a series of clauses has the same subject ‘grandfather’ which is never expressed as a full nominal. %\textcyrillic{{\U\CYRA}}

\ea 
\label{ex:Volkova:12}
 \gll Ńaλm-en χoti wɛr-s-ən?\\
 tongue\textsc{{}-poss.2sg} what do\textsc{{}-pst-}2sg\\
 \glt ‘– What have you done with your tongue?’ [WKhC, “A woman preparing sinews”]
\z

 \ea 
\label{ex:Volkova:13}
 \gll textcyrillic{{UCYRA}n} pa wɵ-λ{}-əm jăm{}-a moś mońś-əs. Ar moś wɵ-s.  Moś-λ-aλ χʉw-ət.\\
 \textsc{neg} \textsc{add} know\textsc{{}-npst-1sg} good\textsc{{}-dat} tale tell\textsc{{}-pst[3sg]} many tale know\textsc{{}-pst[3sg]} tale-\textsc{pl-poss.3sg} long\textsc{{}-pl}\\
 \glt ‘[He] knew a lot of tales. [His] tales are long.’ [KKhC]
\z
 


Object pro-drop is also possible: 


 \ea 
\label{ex:Volkova:14}
 \gll Śempər kew potali juwət-s-a λʉw katəλ-s-əλλe.\\
 Schemper stone lump throw\textsc{{}-pst-pass} (s)he catch\textsc{{}-pst-}\textsc{3sg>s}\textsc{g}\\
 \glt ‘[They] threw the Schemper stone, he caught [it].’ [WKhC, “The Schemper stone”]
\z



%
%3SG>SG?
%Martin
%May 14, 2020 11:53 AM
\REF{ex:Volkova:14}, the argument of the verb \textit{katəλsəλ}\textit{λ}\textit{$e$} ‘caught’ occupying the direct object position (in the second clause) is not expressed overtly. It refers to the Schemper stone mentioned in the first clause.

It should be noted, however, that object drop does not license a reflexive interpretation:
\ea 
\label{ex:Volkova:15}
 \gll Upi pa jaj išək-λ-əλλən. \\
 older.brother and older.sister praise\textsc{{}-npst-3du>nsg}\\
 \glt \{LC: The younger sister and brother performed very well.\} ‘The older brother and sister praise [them / *themselves].’
\z



\subsection{\textbf{Personal} \textbf{pronouns}}\label{sec:Volkova:2.5}

In Khanty, personal pronouns have three case forms: Nominative, Accusative, and Dative. The pronominal system distinguishes three persons – 1\textsuperscript{st}, 2\textsuperscript{nd} and 3\textsuperscript{rd} – across three numbers: singular, dual and plural. The paradigms of Kazym Khanty personal pronouns are presented in the table below.

\begin{table}
\caption{Personal pronouns}\label{tab:Volkova:1}

\fittable{
\small
\begin{tabularx}{\textwidth}{p{0.4cm}p{0.9cm}p{0.8cm}p{1cm}p{1cm}p{0.9cm}p{0.8cm}p{1.2cm}p{0.8cm}p{0.8cm}} 
\lsptoprule
& \textsc{1sg} & \textsc{2sg} & \textsc{1du} & \textsc{2du} & \textsc{1pl} & \textsc{2pl} & \textsc{3sg} & \textsc{3du} & \textsc{3pl}\\
\hline
\textsc{nom} & ma & năŋ & min & niŋ & mʉŋ & nin & λʉw & λin & λiw (λij)\\
\textsc{acc} & măn-ti & năŋ-ti & min-t & nin-t & mʉŋ-t & nin-t & λʉw-ti / λʉweλ\footnote{In Kazym Khanty, the accusative and dative forms of pronouns differ from those in the Ob’ region. However, there are speakers in Kazym who use the Ob’ variants (\textit{λʉweλ} [(s)he.\textsc{acc]} and \textit{λʉweλa} [(s)he.\textsc{dat]}).} & λin-t & λiw-t\\
\textsc{dat} & mănɛm & năŋen & minam(a) & ninen(a) & mʉŋew & ninen & λʉweλ(a) & λinan(a) & λiweλ\\
\lspbottomrule
\end{tabularx}
}
\end{table}




The 3\textsuperscript{rd} person pronouns in Kazym Khanty are only used with animate antecedents. If an antecedent is inanimate, speakers of Khanty resort to object drop, repeating the full NP or using a demonstrative. In \REF{ex:Volkova:16}, using the 3\textsuperscript{rd} person pronoun \textit{λʉwti} to refer to the bowl is illicit; instead, the object is either dropped or the full NP \textit{an-λ} ‘her bowl’ is pronounced. \REF{ex:Volkova:17} exemplifies the use of a demonstrative \textit{śi} ‘that one’.

\ea 
\label{ex:Volkova:16}
 \gll Maša-en λöt-əs χuram an. Ik-əλ-a (an-λ / *λʉw-ti) išək-s-əλλe.\\
 Masha\textsc{{}-poss.2sg} buy\textsc{{}-pst[3sg]} beautiful bowl husband\textsc{{}-poss.3sg-dat} bowl-poss.3sg / (s)he\textsc{-acc} praise-\textsc{pst-3sg>sg} \\
 \glt ‘Masha bought a beautiful bowl. [She] praised [it] to her husband.’
\z

 
\ea 
\label{ex:Volkova:17}
 \gll Van’a-en śi-ti išək-λ{}-əλλe. \\
 Vanja\textsc{{}-poss.2sg} that.one\textsc{{}-acc} praise\textsc{{}-npst-3sg>sg} \\
 \glt ‘Vanja praises it / him /*himself.’ 
\z



There are no dedicated possessive pronouns in Khanty, instead the Nominative form of a personal pronoun is used in possessive constructions:
\ea 
\label{ex:Volkova:18}
 \gll Tăm năŋ λajm-en?\\
 this you axe\textsc{{}-poss.2sg}\\
 \glt ‘- Is it your axe?’ (WhKC, “The golden axe”]
\z


 

\section{{Locally} {bound} {pronouns}}\label{sec:Volkova:3}
\subsection{{Direct} {object}}\label{sec:Volkova:3.1}

In Kazym Khanty, the majority of speakers use personal pronouns (non-reflexive forms) to encode binding:

\ea 
\label{ex:Volkova:19}
 \gll Maša-j-en\textsubscript{i} λʉw-ti\textsubscript{i/j} λapət-λe.\\
 Masha\textsc{{}-}\textsc{obl-poss.2sg} (s)he-\textsc{acc} feed-\textsc{mpst[3sg]}\\
 \glt ‘Masha feeds herself / him.’
\z

%I would find it better if this were glossed as DEF. The 2SG gloss is confusing.
%Martin
%May 14, 2020 12:08 PM

In \REF{ex:Volkova:19}, a 3\textsuperscript{rd} person pronoun can be interpreted both as covalued with the subject of the clause or as coreferential to someone in the previous context.

The constraints on bound vs. disjoint reading of pronouns in such cases vary across the speakers\footnote{At this point in our discussion we are focusing on the so called extroverted (or other-oriented) verbs. The differences in encoding reflexivity between extroverted and introverted (self-oriented) verbs will be addressed in \sectref{sec:Volkova:5}}. For some speakers, the presence of object agreement on the verb licenses the bound reading of the pronoun \REF{ex:Volkova:20a}, while %
%This is an odd way of putting it – both cases are covalued, but \REF{ex:Volkova:20b} is unacceptable.
%Martin
%May 14, 2020 12:11 PM
the subject agreement on the verb forces the disjoint reading \REF{ex:Volkova:20b}:

\ea 
\label{ex:Volkova:20}
	\ea
	\label{ex:Volkova:20a}
  	 \gll λin λin{}-ti išǝk{}-λ-əλλen.\\
 	 they[\textsc{du}] they[\textsc{du]-acc} praise\textsc{{}-npst-3du>nsg}\\
  	 \glt ‘They praised themselves’

	\ex
	\label{ex:Volkova:20b}
  	 \gll λin λin{}-ti išǝk{}-λ-əŋən.\\
  	 they[\textsc{du}] they[\textsc{du]-acc} praise\textsc{{}-npst-3du}\\
  	 \glt *‘They praise themselves.’ / ‘They praise them.’
	\z
\z
 
This pattern is identical to the one described for Tegi Khanty in \citet{VolkovaReuland2014}. For other speakers, verbal agreement seemingly plays no role, and a personal pronoun can get a bound or a disjoint reading either way. Consider \REF{ex:Volkova:21a} and \REF{ex:Volkova:21b}: in \REF{ex:Volkova:21a}, the verb carries object agreement while in \REF{ex:Volkova:21b} it agrees only with the subject; in both cases, the 3\textsuperscript{rd} person pronoun \textit{λʉw} can be interpreted as bound or as referring to someone mentioned in the previous discourse. 
\ea 
\label{ex:Volkova:21}
	\ea
	\label{ex:Volkova:21a}
  	 \gll Kašəŋ $\chi ɵ$-j-at\textsubscript{i} λʉwti\textsubscript{i/j} išək-s-əλλe. \\
  	 every man\textsc{{}-obl-indef} he.\textsc{acc} praise\textsc{{}-pst-3sg>sg}\\
  	 \glt ‘Every man praised himself / him.’

	\ex
	\label{ex:Volkova:21b}
  	 \gll Kašəŋ $\chi ɵ$\textsubscript{i} λʉw-ti\textsubscript{i/j} išək-əλ.\\
  	 every man he\textsc{{}-acc} praise\textsc{{}-npst.3sg}\\
  	 \glt ‘Every man praises himself/him.’
	\z
\z

 
%
%Here (and elsewhere) you use initial capitalization and a final period. But other examples have no initial capitalization and no final period. It would be best to be consistent, and to use one of these styles throughout the paper. (My preference is for capitalization\& period, but it’s not a strong preference.)
%Martin
%May 14, 2020 1:04 PM


Judgments on examples like \REF{ex:Volkova:21} in Kazym Khanty often vary from speaker to speaker and from example to example elicited from the same speaker. 

\subsection{{Indirect} {Object}}\label{sec:Volkova:3.2}

Personal pronouns also encode reflexivity in the position of indirect (dative) object. Example \REF{ex:Volkova:22} illustrates the point, \textit{λʉw} is encoding Experiencer in Dative:

\ea 
\label{ex:Volkova:22}
 \gll Paša-j-en λʉweλa kăλ.\\
 Pasha\textsc{{}-obl-poss.2sg (}s)he \textsc{dat} be.visible.\textsc{npst[3sg]}\\
 \glt ‘Pasha is visible to himself / him.’ ( {\textasciitilde} Pasha is able to see himself / him.)
\z
%
%The gloss in \REF{ex:Volkova:23} is different. And it’s odd to have a possessive suffix on a pronoun.
%Martin
%May 14, 2020 1:06 PM

For Recipient \REF{ex:Volkova:23}, Benefactive \REF{ex:Volkova:24} and other semantic roles that are encoded in Khanty by Dative, the strategy is the same: a locally bound personal pronoun. Depending on the context, in all these examples \textit{λʉweλa} can also have a disjoint interpretation. 

\ea 
\label{ex:Volkova:23}
 \gll Nɛm $\chi ɵ$-j-at λʉweλa šiməλ-šək ăn pun-λ.\\
 no.kind man-\textsc{obl-indef} (s)he\textsc{.dat} few\textsc{{}-att} \textsc{neg} put\textsc{{}-npst[3sg]}\\
 \glt ‘Nobody puts less to himself (than to others).’
\z

 
\ea 
\label{ex:Volkova:24}
 \gll Waśka{}-j-en λʉw-eλa χot os-əs.\\
 Vasja\textsc{{}-obl-poss.2sg} (s)he\textsc{{}-dat} house build\textsc{{}-pst[3sg]}\\
 \glt ‘Vasja built the house for himself / him.’
\z

The 3\textsuperscript{rd} person pronoun in the indirect object position cannot be anteceded by a direct object \REF{ex:Volkova:25a}, however, if it occupies a direct object position, an indirect object can serve as its antecedent \REF{ex:Volkova:25b}.
\ea 
\label{ex:Volkova:25}
	\ea
	\label{ex:Volkova:25a}
  	 \gll .\textsuperscript{*}Ma χur-ən Pet{}'a λʉw-eλa wantλta-s-ɛm.\\
  	  I image-\textsc{loc} Petja (s)he\textsc{{}-dat} show\textsc{{}-pst-1sg>sg}\\
  	 \glt Int.: I showed Petja to himself on the photo.

	\ex
	\label{ex:Volkova:25b}
  	 \gll Ma χur-ən Pet{}'a{}-j{}-en{}-a λʉw-ti wantλta-s-ɛm.\\
  	 I image-\textsc{loc} Petja-\textsc{obl-poss.2sg-dat} (s)he\textsc{{}-dat} show\textsc{{}-pst-1sg>sg}\\
  	 \glt ‘I showed to Petja himself on the photo.’
	\z
\z


\subsection{{Binding} {conditions} {for} {\textit{λʉw}}}\label{sec:Volkova:3.3}

As mentioned above, personal pronouns can be bound by non-referential expressions such as quantifiers. In example \REF{ex:Volkova:26}, the 3\textsuperscript{rd} person pronoun \textit{λʉw} occupies the position of a direct object, and in \REF{ex:Volkova:27} it %
%(Note that repetition often sounds better in English academic prose than gapping.)
%Martin
%May 14, 2020 1:07 PM
occupies the position of an indirect dative object.

\ea 
\label{ex:Volkova:26}
 \gll Nɛm χɵ-j-at λʉw-t ăn šɵka-λ.\\
 no man-\textsc{obl-indef} (s)he\textsc{{}-acc neg} offend\textsc{{}-npst[3sg]}\\
 \glt ‘Nobody will offend himself.’
\z

 
 \ea 
\label{ex:Volkova:27}
 \gll Kašəŋ ewi-ja jont-λ  λʉw-eλa tʉtśaŋ  χir.\\
 every girl sew\textsc{{}-npst[3sg]} (s)he\textsc{{}-dat} for.needlework pouch\\
 \glt ‘Every girl sews herself a pouch for needlework.’
\z




In general, when a subject of a clause is a quantified expression, speakers prefer the bound interpretation of \textit{λʉw}, but provided an appropriate context they allow the disjoint interpretation as well \REF{ex:Volkova:28}. 

\ea 
\label{ex:Volkova:28}
 \gll Pet’a-j-en nuχ pit-əs.  Kašəŋ kort-əŋ $\chi ɵ$jat-əw λʉw-t išk-əλ.\\
 Peter-\textsc{obl-poss.2sg} up become-\textsc{pst[3sg}] every village\textsc{{}-attr} man\textsc{{}-poss.1pl} (s)he\textsc{{}-acc} praise-\textsc{pst[3sg]}\\
 \glt ‘Peter won (the game). Every man from the village praise him.’
\z


If the antecedent is referential, there is no clear preference in favour of a bound or a disjoint reading, both are available. In \REF{ex:Volkova:29}, the verb in the first conjoined clause bears subject-object agreement while in the second clause it agrees only with the subject; in both clauses, the pronoun \textit{λʉw} can get either a bound or a disjoint reading.

\ea 
\label{ex:Volkova:29}
 \gll Maša-j-en\textsubscript{i} šuwaλ-əs-λe λʉw-ti\textsubscript{i/j} χur χośi i Daša-j-en\textsubscript{k} λʉw-ti\textsubscript{i/k/j} pa šuwaλ-əs.\\
 Masha\textsc{{}-obl-poss.2sg} see\textsc{{}-npst-3sg>sg} (s)he\textsc{{}-acc} image to and Dasha\textsc{{}-obl-poss.2sg} (s)he\textsc{{}-acc} \textsc{add} see\textsc{{}-pst[}3sg] 
\\
 \glt ‘Masha saw her(self) on the photo and Dasha saw her(self) too.’
\z

 %
%I think it would be better to use “Masha”, “Pasha”, etc. in the translations, just like “Vasja”. Translating given names into English is odd.
%Martin
%May 14, 2020 1:09 PM


The 3\textsuperscript{rd} person pronoun \textit{λʉw} can also get a sloppy reading – cf. \REF{ex:Volkova:30a}. For the strict reading the speakers prefer repeating the full noun phrase as in \REF{ex:Volkova:30b}:
\ea 
\label{ex:Volkova:30}
	\ea
	\label{ex:Volkova:30a}
  	 \gll Maša-j-en  šʉwaλ-əs-λe λʉw-ti χur χośi i Daša-j-en pa.\\
  	 Masha\textsc{{}-obl-poss.2sg} see\textsc{{}-npst-3sg>sg} (s)he\textsc{{}-acc} image to and Dasha-\textsc{obl-poss.2sg} \textsc{add}\\
  	 \glt ‘Masha saw herself in the photo and Dasha did so too (Dasha saw herself).’

	\ex
	\label{ex:Volkova:30b}
  	 \gll Maša-j-en  šʉwaλ-əs-λe λʉw-ti χur χosi i Daša-j-en iśi Maša-j-əλ šuwaλ-əs.\\
  	 Masha\textsc{{}-obl-poss.2sg} see\textsc{{}-pst-3sg>sg} (s)he\textsc{{}-acc} image to and Dasha\textsc{{}-obl-poss.2sg} too Masha\textsc{{}-obl-poss.3sg} see\textsc{{}-pst[3sg]}\\
  	 \glt ‘Masha saw herself in the photo and Dasha saw Masha too.’
	\z
\z


\subsection{{Postpositional} {phrases}}\label{sec:Volkova:3.4}

Some postpositions in Khanty can attach case and possessive suffixes (e.g. \textit{ewəλt-ɛm-a} [from\textsc{{}-poss.1sg-dat}] ‘from me’), similarly to possessed nouns (see \sectref{sec:Volkova:2.2} and \sectref{sec:Volkova:6}). The complement noun phrase overtly expressed as a free personal pronoun triggers the agreement on the postposition.
\ea 
\label{ex:Volkova:31}
 \gll ${\emptyset}$\textsubscript{i} Xɵλ-mit χătəλ šuwaλ-əs jɵš χoś-a \textbf{ λʉw\textsubscript{i}} jeλpe-\textbf{λ}-ən wɵn taś pa mir.\\
 three\textsc{{}-ord} day see\textsc{{}-npst[3sg]} road near\textsc{{}-dat} (s)he in.front.of\textsc{{}-poss.3sg-loc} big herd \textsc{add} people\\
 \glt ‘On the third day he sees a big herd and people in front of him near the road.’ [WKhC, “The three wise words”]
\z


%nonpast?
%Martin
%May 14, 2020 1:08 PM




\ea 
\label{ex:Volkova:32}
 \gll Paša-j-en\textsubscript{i} (i) \textbf{λʉw}\textsubscript{i/j} oλŋ-\textbf{əλ}{}-ən putərt-əs.\\
 Pasha\textsc{{}-obl-poss.2sg pt} (s)he about\textsc{{}-poss.3sg-loc} tell\textsc{{}-pst[3sg]}\\
 \glt 'Pasha told about him/himself.’
\z

 

\ea 
\label{ex:Volkova:33}
 \gll Maša-j-en ńawrɛm-λ-aλ\textsubscript{i} \textbf{λiw\textsubscript{i}} oλŋ-\textbf{eλ}{}-ən putərt-əs.\\
 Masha\textsc{{}-obl-poss.2sg} child-\textsc{pl-poss.3pl} they about\textsc{{}-poss.3pl-loc} tell\textsc{{}-pst[3sg]}\\
 \glt ‘Masha told the children about them.’
\z

In \REF{ex:Volkova:31} and \REF{ex:Volkova:32}, personal pronoun \textit{λʉw} is covalued with the subject of the clause. \REF{ex:Volkova:32} illustrates the fact that both bound and disjoint readings are available for \textit{λʉw} in a postpositional phrase, as in object position. In \REF{ex:Volkova:33}, \textit{λʉw} is covalued with a noun phrase in the direct object position.

Kazym Khanty also employs uninflected postpositions. They can also take pronouns as their complements, and the pronouns can be covalued with the subject as shown in \REF{ex:Volkova:34}:

\ea 
\label{ex:Volkova:34}
 \gll Mit$\chi ɵ$\textsubscript{i} λʉw\textsubscript{i} rot-a nɵməs-ij-əλ.\\
 servant (s)he along\textsc{{}-dat} think\textsc{{}-ipfv-npst[3sg]}\\
 \glt ‘The servant thinks to himself:’ [WKhC, “The Quick-witted servant of the king”]
\z


Personal pronouns with the postposition \textit{kut-ən} ‘between’ form a reciprocal postpositional phrase:


 \ea 
\label{ex:Volkova:35}
 \gll λin\textsubscript{i} kut-ən\textsubscript{i}{}-ən jăm-a wʉ-s-ŋən.\\
 they\textsc{.du} interval\textsc{{}-2/3du-loc} good\textsc{{}-adv} live\textsc{{}-pst-3du}\\
 \glt ‘They had a good rapport with each other.’ [WKhC, “The Quick-witted servant of the king”]
\z


There is also a dedicated %
%Shoudn’t this be called a reflexive pronoun?
%Martin
%May 14, 2020 1:15 PM
lexeme \textit{panən} meaning ‘with oneself’. This lexeme has the properties of a dedicated presuppositional comitative in terms of \citet{Perkova2018}, meaning the involvement of one of the coparticipants is presupposed.


 \ea 
\label{ex:Volkova:36}
 \gll Joχ i măn-s-ət ime-λ pănən tɵ-s-ət.\\
 back go\textsc{{}-pst-3pl} wife\textsc{{}-3sg} with.self carry\textsc{{}-pst}\\
 \glt ‘Back they went (and) took his wife with them.’ [WKhC, “The younger daughter of the sun”]
\z




\ea 
\label{ex:Volkova:37}
 \gll Pănən χăλ-i χir-əλ-ən tɵp χ ǫλ əm aj ńań tăj-əs.\\
 with.self food.for.travel\textsc{{}-attr} sack\textsc{{}-poss.3sg-loc} only three small bread take\textsc{{}-pst[3sg]}\\
 \glt ‘‎He took only three little loaves of bread in his sack with him‎.’ [WKhC, “The boy from the other side”]
\z



\ea 
\label{ex:Volkova:38}
 \gll Mitχ ɵ χon pănən λ-ti ɵms-əs.\\
 servant king %
%why is the gloss different here?
%Martin
%May 14, 2020 1:15 PM
with.self eat\textsc{{}-ipfv.par} sit\textsc{{}-pst[3sg]}
\\
 \glt ‘The servant and the king with him sat down to eat.’ [WKhC “The Quick-witted servant of the king”]
\z

 

 \ea 
\label{ex:Volkova:39}
 \gll Amp-ew muŋ piλ-\textbf{aw}{}-a pănən ji-s.\\
 dog\textsc{{}-poss.1pl} we with\textsc{{}-poss.1pl-dat} with.self go\textsc{{}-pst[}3sg]\\
 \glt ‘Our dog went with us (together with us).’ [WKhC, “On the river bank”]
\z


Summing up, in all relevant contexts Kazym Khanty employs locally bound personal pronouns to express reflexivity. The agreement pattern on the verb does not play a crucial role in the availability of a bound reading the way it does in northern dialects of Khanty.


\section{{Pronoun} doubling}\label{sec:Volkova:4}
\subsection{{Doubling} {\textit{λʉw}}}\label{sec:Volkova:4.1}

Some speakers prefer or even require a doubling strategy for coargument binding. Examples in \REF{ex:Volkova:40} and \REF{ex:Volkova:41} elicited from different speakers illustrate the cross-speaker variation. In \REF{ex:Volkova:40}, \textit{ λʉw} \textit{λʉwti} forms a single unit which ensures a bound interpretation, cf. the impossibility to drop \textit{λʉw} in \REF{ex:Volkova:40b}.

\ea 
\label{ex:Volkova:40}
	\ea
	\label{ex:Volkova:40a}
  	 \gll Maša-j-en\textsubscript{i}  [λʉw λʉwti]\textsubscript{i/*j} λapət-λe. \textup{(Speaker X)}\\
  	 Masha\textsc{{}-obl-poss.2sg} (s)he (s)he\textsc{{}-dat} feed\textsc{{}-npst-3sg>sg}\\
  	 \glt ‘Masha maintains herself by her own efforts (lit. Masha feeds herself).’

	\ex
	\label{ex:Volkova:40b}
  	 \gll $X$uj *(λʉw) λʉw-ti muλχatλ išk-əs-əλλe.\\
  	 who (s)he (s)he\textsc{{}-acc} yesterday praise\textsc{{}-pst-3sg>sg}\\
  	 \glt ‘Somebody praised himself yesterday.’
	\z
\z
 
Other speakers disprefer this strategy \REF{ex:Volkova:41a} or reinterpret \textit{λʉw} \textit{λʉwti} as a combination of an intensifier and a pronominal (on the use of \textit{λʉw} as a self-%
%better: self{}-intensifier
%Martin
%May 14, 2020 1:16 PM
intensifier see \sectref{sec:Volkova:7}). In \REF{ex:Volkova:41}, both interpretations (bound and disjoint) are available for a simple pronoun. 
\ea 
\label{ex:Volkova:41}
	\ea
	\label{ex:Volkova:41a}
  	 \gll Maša-j-en\textsubscript{i}  (* λʉw) λʉw-ti\textsubscript{i/j} λapət-λe. \textup{(Speaker Y)}\\
  	Masha\textsc{{}-obl-poss.2sg} (s)he (s)he\textsc{{}-dat} feed\textsc{{}-npst-3sg>sg} \\
  	 \glt ‘Masha feeds herself / him.’

	\ex
	\label{ex:Volkova:41b}
  	 \gll Vas’a-j-en  λʉw λʉw-ti ăn wɵ-λ-λe.\\
  	 Vasja\textsc{{}-obl-poss.2sg} (s)he (s)he\textsc{{}-acc neg} know\textsc{{}-npst-3sg>sg}\\
  	 \glt ‘Vasja himself doesn’t know himself.’
	
	\ex
	\label{ex:Volkova:41c}
  	 \gll Maša-j-en  λʉw λʉweλa jontə kɛrnas.\\
  	 Masha\textsc{{}-obl-poss.2sg} (s)he (s)he\textsc{.dat} sew\textsc{{}-npst[3sg]} dress\\
  	 \glt ‘Masha (herself) sew herself a dress.’
	\z
\z



The order of the elements is also not fixed. Some speakers use the nominative form followed by the case form \REF{ex:Volkova:40}, one speaker also used the reversed order \REF{ex:Volkova:42}. In \REF{ex:Volkova:42a}, the verb bears subject-object agreement, in \REF{ex:Volkova:42b}, it agrees only with the subject, thus both options can be combined with the doubled pronoun.
\ea 
\label{ex:Volkova:42}
	\ea
	\label{ex:Volkova:42a}
  	 \gll Učitel’-ət\textsubscript{i} λiw-ti λiw\textsubscript{i/*j} išək-s-ə λ aλ. \textup{(Speaker Z)}\\
  	 teacher-\textsc{pl} they-\textsc{acc} they praise-\textsc{pst-3pl>nsg}\\
  	 \glt ‘The teachers praised themselves / *them.’

	\ex
	\label{ex:Volkova:42b}
  	 \gll Učitel’-ət\textsubscript{i} λiw-ti λiw\textsubscript{i/*j} išək-s-ət. \\
  	 teacher-\textsc{pl} they-\textsc{acc} they praise-\textsc{pst-3pl}\\
  	 \glt ‘The teachers praised themselves / *them.’
	\z
\z


\subsection{{Combining} {\textit{λʉw} }{and} {\textit{i}}}\label{sec:Volkova:4.2}

Some Kazym Khanty speakers also use a combination of a discourse particle \textit{i} and a 3\textsuperscript{rd} person pronoun to encode reflexivity. This option unambiguously yields a bound interpretation. For some, it does not depend on the type of agreement on the verb (can be combined with both the subject and the subject-object agreement) – \REF{ex:Volkova:43}, others consider subject agreement on the verb in combination with \textit{iλʉwti} illicit \REF{ex:Volkova:44}. 

\ea 
\label{ex:Volkova:43}
  \gll Van’a-en i λʉw-ətti išək-λ(-əλλe). \\
 Vanja\textsc{{}-poss.2sg} \textsc{pt} (s)he\textsc{{}-acc} praise-\textsc{npst(-3sg>sg)} \\
  \glt ‘Vanja praises himself / *him.’
\z

\ea 
\label{ex:Volkova:44}
  \gll Evi-en i λʉw-ti iśn’i lis-ən šiwaλ-əs*(-λe). \\
 girl\textsc{{}-poss.2sg} \textsc{pt} (s)he\textsc{{}-acc} window glass\textsc{{}-loc} see\textsc{{}-pst-3sg>sg}\\
  \glt ‘The girl saw herself in the window glass.’
\z



Summing up, personal pronouns in Kazym Khanty can have both a bound and a disjoint interpretation. If a speaker wants to avoid ambiguity, she can resort to an alternative strategy such as doubling of a 3\textsuperscript{rd} person pronoun or adding a discourse particle \textit{i} to a 3\textsuperscript{rd} person pronoun. Both of these strategies are neither fully grammaticalized, nor accepted by all the speakers.

\section{{Verbal} {reflexivization}}\label{sec:Volkova:5}

In Kazym Khanty, two detransitivising suffixes – \textit{{}-əs- (}also \textit{-as-,} \textit{-aś-)} and -\textit{ijλ-} – can function as verbal reflexivizers in combination with a closed class of verbs (grooming, bodily posture etc.). The use of the detransitivising suffix \textit{{}-əs-} as a verbal reflexive is exemplified in \REF{ex:Volkova:45}.

\ea 
\label{ex:Volkova:45}
	\ea
	\label{ex:Volkova:45a}
  	 \gll Maša-j-en λurt-as{}-əs.\\
  	 Masha\textsc{{}-obl-poss.2sg} cut.hair\textsc{{}-detr-pst[3sg}]\\
  	 \glt ‘Masha got her hair cut.’

	\ex
	\label{ex:Volkova:45b}
  	 \gll Maša-j-en puχ-əλ λurt-s-əλλe.\\
  	 Masha\textsc{{}-obl-poss.2sg} son\textsc{{}-poss.3sg} cut.hair\textsc{{}-pst-3sg>sg} \\
  	 \glt ‘Masha cut her son’s hair.’ 
	\z
\z


The suffix \textit{{}-əs-} can also mark reciprocity \REF{ex:Volkova:46}:

\ea 
\label{ex:Volkova:46}
	\ea
	\label{ex:Volkova:46a}
  	 \gll λin λin kʉtən-ən taŋ-as{}-λ{}-əŋən \\
  	 they[\textsc{du]} they[\textsc{du]} between\textsc{{}-poss.3du} persuade\textsc{{}-detr-npst-3du}\\
  	 \glt ‘They persuaded each other.’

	\ex
	\label{ex:Volkova:46b}
  	 \gll Pet’a{}-j{}-en Vas’a-j-λ taŋ-s{}-əλe χot omas-ti.\\
  	 Peter-\textsc{obl-poss.2sg} Vasja\textsc{{}-obl-poss.3sg} persuade\textsc{{}-pst-3sg>sg} house build\textsc{{}-nfin.npst}\\
  	 \glt ‘Peter persuaded Vasja to build a house.’
	\z
\z
 


It also covers most of the meanings in the reflexive-middle domain on Kemmer’s semantic map \citep{Kemmer1993}, including middle and antipassive, cf. \REF{ex:Volkova:47b} for deobjective and \REF{ex:Volkova:47c} for potential passive (possibilitive).

\ea 
\label{ex:Volkova:47}
	\ea
	\label{ex:Volkova:47a}
  	 \gll Aŋk-ɛm jɵn-λ  jɛrnas.\\
  	 mother\textsc{{}-poss.1sg} sew\textsc{{}-npst.3sg} dress\\
  	 \glt ‘My mother is sewing a dress.’

	\ex
	\label{ex:Volkova:47b}
  	 \gll Aŋk-ɛm jɵnt-əs-λ.\\
  	 mother-poss.1sg sew-detr-npst[3sg]\\
  	 \glt ‘My mother sews (clothes).’
  	 
  	 \ex
	\label{ex:Volkova:47c}
  	 \gll Tam šaškan jăma jɵnt-əs-λ.\\
  	 this textile good sew\textsc{{}-detr-npst[3sg]}\\
  	 \glt ‘This textile is easy (good) to sew.’
	\z
\z

Example \REF{ex:Volkova:48} and \REF{ex:Volkova:49} illustrate the use of suffix -\textit{ijλ-} as a verbal reflexive: 
\ea 
\label{ex:Volkova:48}
	\ea
	\label{ex:Volkova:48a}
  	 \gll Ewi-je-n  λuχit-ijλ-əs.\\
  	 girl\textsc{{}-dim-poss.2sg} wash\textsc{{}-detr-pst[3sg]}\\
  	 \glt ‘The girl washed.’

	\ex
	\label{ex:Volkova:48b}
  	 \gll Maša-j-en još-ŋəλ λuχit-s-əλλe.\\
  	 Masha\textsc{{}-obl-poss.2sg} hand\textsc{{}-poss.3du} wash\textsc{{}-pst-}3sg>
\\
  	 \glt ‘Masha washed her hands.’
	\z
\z

\ea 
\label{ex:Volkova:49}
	\ea
	\label{ex:Volkova:49a}
  	 \gll Jivan-en ar vuχ rɵpət-əs pa išək-ijλ.\\
  Ivan\textsc{{}-poss.2sg} a.lot money earn\textsc{{}-pst[}3sg] and praise\textsc{{}-detr.npst[3sg]}\\
  	 \glt ‘Ivan earned a big sum of money and praises himself / boasts.’


	\ex
	\label{ex:Volkova:49b}
  	 \gll Jivan-en jaj-əλ išək-əλ.\\
  	 Ivan\textsc{{}-poss.2sg} brother\textsc{{}-poss.3sg} praise\textsc{{}-npst[3sg]}\\
  	 \glt ‘Ivan praises his brother.’
	\z
\z


The suffix -\textit{ijλ-} can also be used to mark reciprocity \REF{ex:Volkova:50}. 

\ea 
\label{ex:Volkova:50}
	\ea
	\label{ex:Volkova:50a}
  	 \gll Pet’a-en Maša-en pilä mosəλt-ijəλ-s-əŋən.\\
  	 Petja\textsc{{}-poss.2sg} Masha\textsc{{}-poss.2sg} with kiss\textsc{{}-detr-pst-3du}\\
  	 \glt ‘Petja and Masha kissed.’ (lit. Petja kissed with Masha)

	\ex
	\label{ex:Volkova:50b}
  	 \gll Im-əλ eweλt mosəλt-əs.\\
  	 wife-\textsc{poss.3sg} from kiss-\textsc{pst[3sg}]\\
  	 \glt ‘(He) kissed his wife.’
	\z
\z


However, its primary function is to mark frequentative \citep{Kaksin2007}, as can be seen from the contrast between \REF{ex:Volkova:51a} and \REF{ex:Volkova:51b}:

\ea 
\label{ex:Volkova:51}
	\ea
	\label{ex:Volkova:51a}
  	 \gll Want-i sorəm muw-n oλ śi wojəmt-λ-a.\\
  	 look\textsc{{}-imp.so} dry ground\textsc{{}-loc} lay\textsc{.npst[3sg]} \textsc{foc} fall.asleep\textsc{{}-npst-pass[3sg]}\\
  	 \glt ‘Look, (he) lies on dry ground, and he is about to fall asleep’ [WKhC, “The river land man and the Ob’ land man”]

	\ex
	\label{ex:Volkova:51b}
  	 \gll At-λ λiλ-əŋ tɛλ-n oməs-s-əλλe χuta wojəmt-\textbf{ijəλ}{}-s-a moj χuta ăntǫ.\\
  	 night\textsc{{}-poss.3sg} soul\textsc{{}-attr} full\textsc{{}-loc} sit\textsc{{}-pst-3sg.so} where fall.asleep\textsc{{}-ipfv-pst-pass[}3sg] or where \textsc{neg}
\\
  	 \glt ‘…And so he spent the night, sometimes falling asleep, sometimes not.’ [WKhC, “The river land man and the Ob’ land man”]
	\z
\z

The division of labour between \textit{{}-əs-} and -\textit{ijλ-} is lexically motivated. The existence of a certain suffixed form depends on a particular verb stem (cf.\textit{λurt-} ‘to cut hair’ 〜 \textit{λurt-əs-} [cut.hair-\textsc{detr}] ‘to cut self’s hair’ vs. *\textit{λuχit-əs-} [wash-\textsc{detr}]).

With detransitivised verbs, \textit{λʉw} can occasionally be used as a self-%
%self{}-intensifier
%Martin
%May 14, 2020 1:18 PM
intensifier modifying the subject in a dedicated construction with the postposition \textit{satta-/saχt}, cf. \REF{ex:Volkova:52} (see \sectref{sec:Volkova:7.1} for details). 

\ea 
\label{ex:Volkova:52}
 \gll Maša-j-en  λʉw saχt-əλ-a  λuχit-\textbf{ɨjλ}{}-s.\\
 Masha\textsc{{}-obl-poss.2sg} (s)he with\textsc{{}-poss.3sg-dat} wash\textsc{{}-detr-pst[3sg]}\\
 \glt ‘Masha herself washed herself.’
\z


The use of a bound personal pronoun or a doubled pronoun is also possible with grooming verbs \REF{ex:Volkova:53}--\REF{ex:Volkova:54}, but speakers consider such examples artificial or triggering the meaning that by default the participant is incapable of performing this action on her own.

\ea 
\label{ex:Volkova:53}
 \gll Maša-j-en  (λʉw) λʉw-t λuχt-s-əλλe.\\
 Masha\textsc{{}-obl-poss.2sg} (s)he (s)he\textsc{{}-acc} wash\textsc{{}-pst-3sg>sg}\\
 \glt ‘Masha (herself) washed herself.’
\z
 


\ea 
\label{ex:Volkova:54}
 \gll Ajk-en  λʉw-ti λomλa-s.\\
 boy\textsc{{}-poss.2sg} (s)he\textsc{{}-acc} dress\textsc{{}-pst[}3sg]\\
 \glt ‘The boy (himself) dressed himself (the boy is usually dressed by somebody else, but now he has managed to do this himself).’
\z


Therefore, to encode reflexivity with introverted verbs, speakers primarily use detransitivising suffixes or possessive constructions (see \sectref{sec:Volkova:6.2}).

\section{{Reflexive} {possessive} {constructions}}\label{sec:Volkova:6}
\subsection{{Adpossessive} {domain}}\label{sec:Volkova:6.1}

To encode an anaphoric dependency between the subject of a clause and the possessor of a non-subject argument, Kazym Khanty employs a possessive affix sometimes accompanied by a free personal pronoun in the position of the possessor in a corresponding noun phrase:

\ea 
\label{ex:Volkova:55}
	\ea
	\label{ex:Volkova:55a}
  	 \gll [Kašəŋ $\chi ɵ$jăt]\textsubscript{i} arij-s (λʉw\textsubscript{i/j}) ar-əλ.\\
  	 every man sing\textsc{{}-pst[3sg]} (s)he song\textsc{{}-poss.3sg}\\
  	 \glt ‘Every man sang his (own) / his song.’

	\ex
	\label{ex:Volkova:55b}
  	 \gll [Kašəŋ $\chi ɵ$jăt]\textsubscript{i} nɵm-əλ-λe  (λʉw\textsubscript{i/j}) kɵrt-əλ.\\
  	 Every man remember\textsc{{}-npst-3sg>sg} (s)he village\textsc{{}-poss.3sg}\\
  	 \glt ‘Every man remembers his\textsubscript{} (own) / his village.’
	\z
\z

A bound reading for the possessor of a direct object is available independently of the presence of object agreement on the verb: the verb agrees only with the subject in \REF{ex:Volkova:55a} and with the subject and object in \REF{ex:Volkova:55b}. This comes in contrast with data reported for the Obdorsk dialect in \citep{Nikolaeva1999Ostyak}. In the Obdorsk dialect, a possessive affix is bound if the verb carries object agreement and can be interpreted as bound or disjoint in the case of subject agreement on the verb. In Kazym Khanty, both readings are available for both cases. The combination of a personal pronoun in the possessor position and a possessive affix is also used in 1\textsuperscript{st} and 2\textsuperscript{nd} person: 

\ea 
\label{ex:Volkova:56}
 \gll Ma ma muw-ɛm-ən jăχ{}-λ{}-əm.\\
 I I land\textsc{{}-poss.1sg-loc} go\textsc{{}-npst-}1\textsc{sg}\\
 \glt ‘I am walking through my land.’ [WKhC, “The Quick-witted servant of the king”]
\z
 

Some speakers who adhere to the non-doubling strategy of encoding reflexivity consider the overt pronoun redundant \REF{ex:Volkova:57} and use it only to add emphasis.
%
%not restrictive, hence no commas
%Martin
%May 14, 2020 1:20 PM


 \ea 
\label{ex:Volkova:57}
 \gll Vas’a-j-en (?λʉw) ar-əλ ari-s-əλλe.\\
 Vasja\textsc{{}-obl-poss.2sg} s(he) song\textsc{{}-poss.3sg} sing\textsc{{}-pst-3sg>sg}\\
 \glt ‘Vasja sang his own song.’
\z

Some speakers strongly prefer a bound interpretation if the possessor position is occupied by an overt pronoun. In \REF{ex:Volkova:58}, the first sentence provides a context which identifies Peter as the author of the song. Despite that, in \REF{ex:Volkova:58a} and \REF{ex:Volkova:58b} presented to speakers with this context, this interpretation (Peter is the author of the song) is not readily available. Sentence \REF{ex:Volkova:58a} has a local antecedent in the Locative while the possessive noun phrase is the subject of the passive construction. Sentence \REF{ex:Volkova:58b} exemplifies active alignment with subject agreement on the verb:
\ea 
\label{ex:Volkova:58}
 \gll Pet’a-j-en isa arij{}-s  λʉw ar-əλ. \\
 Peter\textsc{{}-obl-poss.2sg} always sing\textsc{{}-pst[3sg}] (s)he song\textsc{{}-poss.3sg}\\
 \glt ‘Peter always sang his (own) song.’ 
  \ea
	\label{ex:Volkova:58a}
  	 \gll Muλχatλ kašəŋ χɵ-j-ăt-ən aris-a λʉw ar-əλ.\\
  	 Yesterday every man-\textsc{obl-indef-loc} sing\textsc{{}-pass[}3\textsc{sg}] (s)he song\textsc{{}-poss.3sg}\\
  	 \glt 1) ‘Yesterday, every man sang his (own) song.’2) ‘\%Peter sang his (own) song. Yesterday every man sang his (Peter’s) song.’

	\ex
	\label{ex:Volkova:58b}
  	 \gll Kašəŋ χɵ-j-ăt arij-s  λʉw ar-əλ.\\
  	 every man-\textsc{obl-indef} sing\textsc{{}-pst[3sg]} (s)he song\textsc{{}-poss.3sg}\\
  	 \glt 1) ‘Yesterday, every man sang his (own) song.’
2) ‘\%Peter sang his (own) song. Yesterday every man sang his (Peter’s) song.’
	\z

\z

As was mentioned in \sectref{sec:Volkova:2.2}, some discourse prominent noun phrases (the noun phrases under the scope of the pragmatic presupposition or noun phrases with secondary topic status, according to \citealt{Nikolaeva1999Ostyak}) are marked with possessive affixes. In Kazym Khanty, direct objects with possessive affixes trigger object agreement on the verb (excluding imperfective clauses and noun phrases within the focus domain). There is a tendency among speakers to interpret such direct objects as belonging to subjects (associated with subject’s personal domain) even if the relationship between the subject and the direct object is not possessive in the proper sense of the word:

\ea 
\label{ex:Volkova:59}
 \gll Pet’a tut juχ-λ-aλ  χuλ sewər-s-əλλe.\\
 Peter fire tree\textsc{{}-pl-poss.3sg} all cut\textsc{{}-pst-3sg>nsg}\\
 \glt ‘Peter cut all his firewood.’
\z


In \REF{ex:Volkova:59}, the relationship between subject (Peter) and the direct object (firewood) is established on the basis of the involvement in the same situation and on the basis of the presence in the same scene (presupposed under the same conditions).

In Kazym Khanty, object agreement on the verb does not force subject orientation for the possessive affixes, as can be seen in \REF{ex:Volkova:60a} and \REF{ex:Volkova:60b}. In example \REF{ex:Volkova:60a}, the possessive suffix -\textit{əλ-} on the direct object “her son” is covalued with the noun phrase %
%"from this woman"; I suggest that you add this for the sake of clarity
%Volkova, A. (Anna)
%August 14, 2020 3:43 PM
within a PP “from this woman”; in \REF{ex:Volkova:60b}, the possessive suffix on the direct object is covalued with the zero subject (‘the woman’ mentioned in the previous clause). In both cases, the verb carries object agreement. 

\ea 
\label{ex:Volkova:60}
	\ea
	\label{ex:Volkova:60a}
  	 \gll λʉw śi imi ewəλt poχ-əλ woχ-ti pit-s-əλλe.\\
  	 (s)he this woman from son\textsc{{}-poss.3sg} beg\textsc{{}-ipfv.part} become\textsc{{}-pst-3sg.so}\\
  	 \glt ‘He started begging this woman for her son.’ [WKhC, “Bogatyr”]

	\ex
	\label{ex:Volkova:60b}
  	 \gll Śăλta mԑt jɵχət poχ-əλ tini-j-s-əλλe  śi śoras χɵ-j-a.\\
  	 then most later son\textsc{{}-poss.3sg} sell\textsc{{}-obl-pst-3sg>sg} this goods man\textsc{{}-obl-dat}\\
  	 \glt ‘(The woman)...then sold her son to this merchant.’ [WKhC, “Bogatyr”]
	\z
\z


Example \REF{ex:Volkova:61} showcases that the antecedent of the possessor expressed with a possessive affix can be the direct object, which is possible both with subject-object agreement \REF{ex:Volkova:61a} and with subject-only agreement on the verb \REF{ex:Volkova:61b}:
\ea 
\label{ex:Volkova:61}
	\ea
	\label{ex:Volkova:61a}
  	 \gll Maša-j-en ak-et\textsubscript{i} χot-eλ\textsubscript{i}{}-a kit{}-s{}-əλλe.\\
  	 Masha\textsc{{}-obl-poss.2sg} boy\textsc{{}-pl} house\textsc{{}-poss.3pl-dat} send\textsc{{}-pst-3sg>nsg}\\
  	 \glt ‘Masha sent the boys to their house.’

	\ex
	\label{ex:Volkova:61b}
  	 \gll Maša{}-en ajk{}-et\textsubscript{i} χot{}-eλ\textsubscript{i}{}-n   šiwaλ{}-əs.\\
  	 Masha\textsc{{}-poss.2sg} boy\textsc{{}-pl} house\textsc{{}-poss.3pl-loc} see\textsc{{}-pst[}3\textsc{sg}]\\
  	 \glt ‘Masha saw boys in their house’
	\z
\z

In Kazym Khanty, at least for some speakers the unmarked direct object (indefinite direct object) does license the covalued interpretation of a possessive marker on another noun phrase \REF{ex:Volkova:61b}. In this respect, Kazym Khanty also differs from the Obdorsk dialect of Khanty described by \citet{Nikolaeva1999Ostyak}.

\subsection{{Possessive} {constructions} {in} {encoding} {argument} {binding}}\label{sec:Volkova:6.2}

Possessive constructions are widely used with introverted verbs, in particular, they are preferred with grooming verbs:


\ea 
\label{ex:Volkova:62}
	\ea
	\label{ex:Volkova:62a}
  	 \gll Vas’a-j-en tʉš-λ-aλ  λur-s-əλλe  / λurt-əs.\\
  	 Vasja-\textsc{obl-poss.2sg} whiskers\textsc{{}-pl-poss.3nsg} cut.hair\textsc{{}-pst-3sg>nsg} / cut.hair\textsc{{}-pst[}3\textsc{sg}]\\
  	 \glt ‘Vasja shaved his whiskers.’

	\ex
	\label{ex:Volkova:62b}
  	 \gll Vas’a-j-ən tʉš-λ-aλ  λurt-s-aj-t.\\
  	 Vasja\textsc{{}-obl-loc} whiskers.\textsc{{}-pl-poss.3nsg} cut.hair\textsc{{}-pst-pass-3pl}\\
  	 \glt ‘Vasja shaved his whiskers.’ lit. ‘His whiskers were shaved by Vasja.’
	\z
\z
 



\ea 
\label{ex:Volkova:63}
 \gll Maša-j-en  ɵpət-λ-aλ  nʉχ kunš-s-əλλe.\\
 Masha\textsc{{}-obl-poss.2sg} hair\textsc{{}-pl-poss.3nsg} up comb\textsc{{}-pst-3sg>nsg}\\
 \glt ‘Masha combed her hair (herself).’
\z

 
Possessive constructions can also be used with extroverted verbs to encode argument binding. In \REF{ex:Volkova:64}, instead of using the 3\textsuperscript{rd} person pronoun \textit{λuw} in the direct object position (as in ‘saw him(self)’), speakers prefer a possessive construction ‘(his) shadow image’ (=reflection):

\ea 
\label{ex:Volkova:64}
 \gll Was’a-j-en jiŋk lot-a šɵš-əs. Śăta šiwaλ-əs-λe (λuw) is xur-əλ.\\
 Vasya\textsc{{}-obl-poss.2sg} water pit-\textsc{dat} walk-\textsc{pst[3sg]} there see\textsc{{}-pst-3sg>nsg} (s)he shadow image\textsc{{}-poss.3sg}\\
 \glt ‘Vasya came up to a puddle. He saw there his (own) reflection.’
\z

To sum up, in Kazym Khanty there are no dedicated reflexive possessive pronouns or dedicated reflexive possessive affixes. The reflexivity in this context is encoded by means of possessive affixes. Besides, the possessor can be overtly expressed with a free personal pronoun in the possessor position in the noun phrase. Not only subjects but also direct objects can antecede possessive affixes irrespective of the agreement patterns on the verb. Possessive constructions are also often used both with introverted (especially, grooming verbs) and extroverted verbs in place of other ways of encoding reflexivity.

\section{{Self-Intensification}}\label{sec:Volkova:7}
\subsection{{The} {postpositional} {phrase} {with} {\textit{satta-/saχt}}}\label{sec:Volkova:7.1}

Kazym Khanty employs a dedicated grammaticalized postpositional construction as an intensifier with the meaning ‘on one’s own, by oneself’. It consists of a personal pronoun and a postposition \textit{satta-/saχt-} with a corresponding possessive affix:

\ea 
\label{ex:Volkova:65}
 \gll Ma ma satt-ɛm-a  śit wɛr-λ-ɛm.\\
 I I with.self\textsc{{}-poss.1sg-dat} this do\textsc{{}-npst-1sg>sg}\\
 \glt ‘I do it myself.’
\z
 
This intensifier is controlled by the subject. The subject triggers the possessive agreement – cf. \REF{ex:Volkova:65} for the 1\textsuperscript{st} sg and \REF{ex:Volkova:66} for the 3\textsuperscript{rd} sg:


\ea 
\label{ex:Volkova:66}
	\ea
	\label{ex:Volkova:66a}
	   \gll λʉw saχt-əλ-a moləpś-əλ λɵmt-s-əλλe.\\
  	 (s)he with.self\textsc{{}-poss.3sg-dat} deer.skin.coat\textsc{{}-poss.3sg} put.om\textsc{{}-pst-3sg>sg}\\
  	 \glt ‘(He) himself put on his malitsa (deer skin coat) (without anybody’s help)’
  	 

	\ex
	\label{ex:Volkova:66b}
	\gll λʉw saχətt-əλ-a  λɵmt-λ-əs.\\
  	 (s)he with.self\textsc{{}-poss.3sg-dat} put.on\textsc{{}-prs}\\
  	 \glt ‘(He) dresses up by himself.’
  	
	\z
\z
 

According to \citet{Kaksin2007}, the postposition \textit{satta} ‘with’ occurs only with personal pronouns. The final affix \textit{{}-a} is a dative or an adverbial affix. The construction can be literally translated as ‘me with myself’ \citep[93]{Kaksin2007}. This construction is never used in the sense ‘alone, separately’ or in a contrastive context.

\subsection{ λt{ʉ} {as} {an} {intensifier}}\label{sec:Volkova:7.2}

Some native speakers use the anaphoric pronoun \textit{λʉw} as an intensifier meaning ‘alone, separately’:

\ea 
\label{ex:Volkova:67}
 \gll Maša-j-en  λʉw juχt-əs petr-əλ ănt λawəλ{}-s{}-əλλe.\\
 Masha\textsc{{}-obl-poss.2sg} (s)he come\textsc{{}-pst[}3sg] Peter\textsc{{}-poss.3sg} \textsc{neg} wait\textsc{{}-pst-3sg>sg}\\
 \glt ‘Masha came herself, she did not wait for Peter.’
\z
 
\ea 
\label{ex:Volkova:68}
 \gll Maša-j-en  λʉw wɛr-s ar.\\
 ‘Masha\textsc{{}-obl-poss.2sg} (s)he do\textsc{{}-pst[}3sg] song.’\\
 \glt ‘Masha made the song by herself.’
\z
 
\subsection{{Other} {expressions} {for} {intensification}}\label{sec:Volkova:7.3}

In Kazym Khanty, there are several other expressions (adjectives and adverbs) conveying intensification or reflexive possession meanings. An adjective \textit{jukan} ‘own, personal’ forces the coreferential reading of the possessor of a noun phrase and the subject of the clause: 


 
 

\ea 
\label{ex:Volkova:69}
 \gll λʉw năŋ ńań ănt λɛ-λ  λʉw (λʉw) jukan ńań-əλ  wɛr-λ.\\
 (s)he you bread \textsc{neg} eat\textsc{{}-npst[3sg]} (s)he (s)he own bread\textsc{{}-poss.3sg} do\textsc{{}-npst[}3sg]\\
 \glt ‘She won’t eat your bread, she will cook her own bread.’
\z




There is also a derivative \textit{jukana} with the meaning ‘on one’s own, separately, for personal usage’: \textit{jukana} \textit{wɵλti} ‘to live by himself’ (Solovar~2014: 102), cf. \REF{ex:Volkova:70}:


\ea 
\label{ex:Volkova:70}
 \gll Kǫrt-əŋ joχ λiw jukan-eλ-a tǫp iχuśjaŋ wʉλi tăj-λ-ət.\\
 village\textsc{{}-attr} people they own\textsc{{}-poss.3pl-dat} only eleven deer have\textsc{{}-npst-3pl}\\
 \glt ‘The camp people own only eleven deers privately.’ [WKhC, “In the camp”]
\z


Another lexeme with a similar meaning is an adjective \textit{ateλt} ‘alone’ and a corresponding adverb \textit{ateλta}:


 \ea 
\label{ex:Volkova:71}
 \gll Ma ateλta wɛr-λ-əm.\\
 I separately live-\textsc{npst-1sg}\\
 \glt ‘I live on my own.’
\z



Intensification across languages is often expressed by the same form as reflexivity. In Kazym Khanty, in the absence of dedicated reflexive pronouns, this function can be performed by personal pronouns (for the 3\textsuperscript{rd} person), by a grammaticalized postpositional construction with the postposition \textit{satta-/saχt-} or with the help of dedicated adjectives like \textit{jukan} ‘own, personal’ or \textit{ateλt} ‘alone’ and adverbs derived from them.


\section{{Strategies} {for} {overcoming} {the} {ambiguity}}\label{sec:Volkova:8}

The Kazym Khanty data is typologically unusual: There are no dedicated reflexive pronouns; personal pronouns, including the 3\textsuperscript{rd} person pronoun \textit{λʉw} ‘(s)he’, are used in reflexive contexts. Thus, the 3\textsuperscript{rd} person pronoun can have both a reflexive and a disjoint reading. The question naturally arises, what are the ways of overcoming this ambiguity? When answering this question, the following factors should be taken into consideration. Firstly, the choice of discourse anaphora devices depends on the distribution of discourse topics and, hence, on the particular information structure of a clause: pronominal noun phrases tend to encode discourse prominent referents (discourse topics, cf. accessibility hierarchy of Gundel~1996), they refer to given information in a clause, and predominantly they are topics or secondary topics (Lambrecht~1994, Nikolaeva~1999b). Secondly, there is a direct mapping between information structure and an %
%argument marking
%Martin
%May 14, 2020 1:24 PM
argument marking pattern (passive, object agreement) in Khanty. Thirdly, Khanty is a pro-drop language with possibility of direct object and possessor pro-drop. 

Khanty exploits two primary strategies to avoid the conflict between reflexive vs. disjoint reading of the 3\textsuperscript{rd} person pronoun in a non-subject position. As has been shown by \citet{Nikolaeva1999Ostyak, Nikolaeva1999Agreement, ColleyPrivoznov2019, Kiss2019}, information structure is the crucial factor that licenses a particular argument marking pattern in the clause in Khanty. Topics occupy the subject position in Khanty. If a pronominal argument is coreferential with a noun phrase from the previous discourse, it is likely to be a topic (it is given, presupposed). The following possibilities are available for it: (i) this argument is topical while the other argument in the clause is not topical (new), (ii) both core arguments of a predicate are topical.

The case when one argument is topical and the other is new is illustrated in example \REF{ex:Volkova:72}. The subject of the first clause is the agent, \textit{Paša}. In the second clause, a new participant is introduced as an agent of the verb ‘to praise’, \textit{Paša} loses its agent role but preserves its topical status – the passive construction is required: 

\ea 
\label{ex:Volkova:72}
 \gll Pašă-j-en\textsubscript{i}  χot λaŋəλ λeśit-s-əλλe. ${\varnothing}$\textsubscript{i} Aŋk-əλ-ən išək-s-a.\\
 Pasha\textsc{{}-obl-poss.2sg} house roof repair\textsc{{}-pst-3sg>sg} ${\varnothing}$ mother\textsc{{}-poss.3sg-loc} praise\textsc{{}-pst-pass[3sg]}\\
 \glt ‘Pasha repaired the roof. [He] was praised by his mother.’
\z
 


In the second clause in \REF{ex:Volkova:69}, the agent of the verb ‘to praise’ is \textit{aŋkəλ} ‘his mother’, it is new, it cannot occupy the subject position. Hence, it is demoted to the oblique position marked with locative. The verb bears the passive marker. The topical noun phrase coreferential to \textit{Paša} occupies the subject position and has no overt expression in the clause. The accusative argument marking as in \REF{ex:Volkova:73} is not ungrammatical \textit{per} \textit{se}, but it is not a natural continuation for the first sentence in \REF{ex:Volkova:72} as it violates discourse coherence.

\ea 
\label{ex:Volkova:73}
 \gll Aŋk-əλ  λʉw-ti išək-s-əλλe.\\
 Mother\textsc{{}-poss.3sg} (s)he\textsc{{}-acc} praise\textsc{{}-pst-3sg>sg}\\
 \glt ‘His mother praised him.’
\z
 

A similar case is presented in \REF{ex:Volkova:74}.

\ea 
\label{ex:Volkova:74}
 \gll Aš-ɛm muλχattəλ sort katλ-əs, śi sort(-əλ) ma jaj-ɛm-ən. \\
 father\textsc{{}-poss.1sg} yesterday pike catch\textsc{{}-pst[3sg]} \textsc{foc} pike(-\textsc{poss.3sg}) I brother\textsc{{}-poss.1sg-loc} up let.go\textsc{{}-pst-pass[3sg]}\\
 \glt ‘My father caught the fish, my brother set it free.’
\z


In \REF{ex:Volkova:74}, the noun phrase \textit{sort} ‘pike’ is mentioned in the first clause and is the topic of the second one where it is the patient of the verb \textit{ɛsaλti} ‘let go’. It is promoted to the subject position, the full noun phrase is repeated, and the verb in the second clause is in passive. Summing up, in Kazym Khanty, the topicalization of an argument is usually %
%usually accompanied?
%Martin
%May 14, 2020 1:26 PM
accompanied by passivization: the topicalized argument is promoted into the subject position where it is either repeated as a full noun or dropped.

If both arguments in the clause are topical, the subject is a topic introduced in the previous discourse and the direct object is a secondary topic (“an entity such that the utterance is construed to be about the relationship between it and the primary topic”, \citealt{Nikolaeva1999Ostyak, Nikolaeva1999Agreement}, cf. also “tail” in \citealt{Vallduvi1992}). This is the context where object-drop is used: 

\ea 
\label{ex:Volkova:75}
 \gll Want-λ-əλλe χot χări kut-λ-əp-ən nawərnɛ-lɛ\textsubscript{i} ari-man oməs-əλ. Pupi poχ-ije ${\varnothing}$ i wu-s-λe još păte-λ ${\varnothing}$i χătśə-s-λe nawərnɛ-lɛ wośləχ-a ji-s.\\
 look\textsc{{}-npst-3sg>sg} house open.place distance\textsc{{}-poss.3sg-attr-loc} frog\textsc{{}-dim} sing\textsc{{}-conv} sit\textsc{{}-npst} bear boy\textsc{{}-dim} take\textsc{{}-pst-3sg>sg} hand bottom\textsc{{}-3sg} hit\textsc{{}-pst-3sg>sg} frog\textsc{{}-dim} mud\textsc{{}-dat} become\textsc{{}-pst[3sg]}\\
 \glt ‘[He] looks, a frog is sitting on the floor and singing. The bear took [her], hit [her] with his hand, the frog turned into mud’ (WhKC, “Little chipmunk”]
\z


Example \REF{ex:Volkova:75} is a fragment of a tale. The bear is a discourse topic in this part of the text. The bear goes to the house where he sees a frog. The frog is introduced in the first sentence and is also a discourse topic in this piece of text. In the consequent clauses the direct object referring to the frog has no overt lexical expression but is cross-referenced on the verb with the help of the subject-object agreement marker. 

In other words, Kazym Khanty has an array of strategies (passivization, subject and object drop) that allow it to avoid 3\textsuperscript{rd} person pronouns in the direct object position in the contexts where a familiar Standard Average European would have used a coreferential personal pronoun. This observation is also supported by the quantitative data. In the WhKhC corpus which has 2883 sentences in total there are only 17 clauses where \textit{λʉw} occupies the direct object position. Five of them are cases where the subject and the direct object differ in their grammatical features (in person or number). The majority of the other cases stem from a retelling of a Russian tale and can be attributed to the influence of Russian. 

Speakers of Kazym Khanty also employ a number of strategies to avoid locally bound 3\textsuperscript{rd} person pronouns in the direct object position. These include replacing them with reflexive possessive constructions (\sectref{sec:Volkova:6.2}) or using a detransitivised form of a verb instead of a transitive one. However, a 3\textsuperscript{rd} person pronoun in the direct object position is a regular variant in isolated elicited sentences even though the native speakers are not consistent in their judgments on bound vs. disjoint readings. We hypothesize that the overt free pronoun in Kazym Khanty is, in a sense, reserved for reflexive contexts – see \REF{ex:Volkova:76} where the bound 3\textsuperscript{rd} person pronoun is contrastively focused. 

\ea 
\label{ex:Volkova:76}
 	\gll Was’a{}-j{}-en  Pet’a{}-j{}-λ{}-a  χur wan{}-əλt{}-əs. Nɵməs{}-əs \textup{ś}ăta Pet’a-j-en pa (i) λʉw-t śi χur-əλ-ən uš-a wɛr-s-əλλe.\\
 		Vasja\textsc{{}-obl-poss.2sg} Peter\textsc{{}-obl-poss.3sg-dat} image look			\textsc{{}-caus-pst[3sg]} think{}-\textsc{pst[3sg]} there Peter\textsc{{}-obl-poss.2sg add pt} (s)he\textsc{{}-acc foc} image\textsc{{}-poss.3sg-loc} brain\textsc{{}-dat} do\textsc{{}-pst-3sg>sg}\\
  	\glt ‘Vasya was showing a photo to Petya. (He) thought that Petya was there, (but instead) found himself on the photo’
\z


In naturally occurring texts, coreference (discourse-level anaphora) is usually expressed by other means, therefore there is no real competition between a bound and a disjoint reading for a 3\textsuperscript{rd} person pronoun. But it may arise in isolated sentences presented to speakers. 

To sum up, there are no grammatical constraints on the 3\textsuperscript{rd} person pronoun in the direct object position in Kazym Khanty, but in naturally occurring texts its use is rare.

\section{{Conclusions}}\label{sec:Volkova:9}

Kazym Khanty uses locally bound personal pronouns to express reflexivity. Their behavior, unlike in other dialects of Khanty, is not grammatically constrained. In other words, in most of the cases we considered, a pronoun can have both a bound and a disjoint reading, and one cannot predict the interpretation solely based on grammatical factors.

This is typologically unusual. Other languages reported in the literature to allow locally bound 3rd person pronouns are Frisian \citep{Everaert1986}, Old English (\citealt{Gelderen2000}), and Haitian Creole \citealt{Zribi-HertzGlaude2007}). In general, the use of dedicated strategies is considered the norm \citep{Moyse-Faurie2008,HeineMiyashita2008}. Binding in Khanty thus violates the Principle B of the Binding Theory \citep{Chomsky1981}. It is problematic for both the syntactic Reflexivity theory \citep{ReinhartReuland1993, Reuland2011} and the semantics-based theory of \citet{Schlenker2005}, as well as for the theories that argue for the Disjointness presumption \citep{FarmerHarnish1987,KoenigSiemund2000} or for a blocking and obviation account \citep{Kiparsky2012}.

In our paper we discussed factors influencing the encoding of reflexivity in Kazym Khanty and offered an account in terms of distribution of labour. Unlike many European languages, Kazym Khanty avoids ambiguity when a 3\textsuperscript{rd} person pronoun is used. Coreference (discourse-level anaphora) is expressed by different strategies which depend on topic domains and patterns of their encoding. The two crucial factors are: (a) the choice of verbal argument marking %
%argument marking
%Martin
%May 14, 2020 1:28 PM
regulated by the information structure and (b) the patterns for subject and object pro-drop. The use of 3\textsuperscript{rd} person pronouns in a direct object position is rare and is reserved for a bound reading even if it can also get a disjoint reading. 


\section*{Abbreviations}

\begin{tabularx}{.45\textwidth}{lQ}
\textsc{add} & additive\\ 
\textsc{ADV}	& adverbial\\ 
\textsc{att} & attenuative\\ 
\textsc{attr} & attributive\\ 
\textsc{caus} & causative\\ 
\textsc{conv} &	converb\\ 
\textsc{detr} &	detransitivizing affix\\ 
\textsc{dim} &	diminutive\\ 
\textsc{du} &	dual\\ 
\textsc{evid} &	evidential\\ 
\textsc{imp} &	imperative\\ 
\textsc{indef} &	indefinite\\ 
\textsc{ipfv} &	imperfective\\ 
KKhC  &	Kazym Khanty Corpus\\ 
\textsc{nfin} & 	non-finite\\ 
\textsc{npst} & 	nonpast\\ 
\textsc{nsg} & 	non-singular\\ 
\textsc{obl} & 	oblique\\ 
\textsc{ord} &  	ordinal\\ 
\textsc{pass} & 	passive\\ 
\textsc{prs} & 	present\\ 
\textsc{pst} & 	past\\ 
\textsc{pt} &  	particl\\ 
\textsc{punct} & 	punctual\\ 
\textsc{so} & 	subject-object agreement\\ 
WKhC &	Western Khanty Corpus\\ 
\end{tabularx}


\sloppy\printbibliography[heading=subbibliography,notkeyword=this]



\end{document} 
