\documentclass[output=paper]{langscibook}

\author{Simon Overall\affiliation{University of Otago}}
\title{Reflexive and reciprocal in Aguaruna} 
\abstract{This paper describes the grammatical means for expressing reflexive and reciprocal situations in Aguaruna (Chicham). The two functions are marked with dedicated verbal derivational suffixes which reduce the valency of the verb. There are some clear examples of lexicalized reflexive and reciprocal markers, with attendant semantic narrowing, but in general the semantic effects of these markers are predictable and combinatorial. Reflexive and reciprocal suffixes can co-occur with valency increasing derivational suffixes (causative and applicative) and are mutually exclusive with inflectional object agreement markers.
[New paragraph:] Aguaruna is spoken between the Andes and the Amazon Basin, and its use of valency reducing derivations to mark reflexive and reciprocal situations is consistent with areal tendencies. However, the presence of distinct markers for reflexive and reciprocal makes Aguaruna more like the Andean Quechuan languages, as Amazonian languages tend to have a single multipurpose valency reducing derivation.}

\IfFileExists{../localcommands.tex}{
  \addbibresource{localbibliography.bib}
  \usepackage{langsci-optional}
\usepackage{langsci-gb4e}
\usepackage{langsci-lgr}

\usepackage{listings}
\lstset{basicstyle=\ttfamily,tabsize=2,breaklines=true}

%added by author
% \usepackage{tipa}
\usepackage{multirow}
\graphicspath{{figures/}}
\usepackage{langsci-branding}

  
\newcommand{\sent}{\enumsentence}
\newcommand{\sents}{\eenumsentence}
\let\citeasnoun\citet

\renewcommand{\lsCoverTitleFont}[1]{\sffamily\addfontfeatures{Scale=MatchUppercase}\fontsize{44pt}{16mm}\selectfont #1}
   
  %% hyphenation points for line breaks
%% Normally, automatic hyphenation in LaTeX is very good
%% If a word is mis-hyphenated, add it to this file
%%
%% add information to TeX file before \begin{document} with:
%% %% hyphenation points for line breaks
%% Normally, automatic hyphenation in LaTeX is very good
%% If a word is mis-hyphenated, add it to this file
%%
%% add information to TeX file before \begin{document} with:
%% %% hyphenation points for line breaks
%% Normally, automatic hyphenation in LaTeX is very good
%% If a word is mis-hyphenated, add it to this file
%%
%% add information to TeX file before \begin{document} with:
%% \include{localhyphenation}
\hyphenation{
affri-ca-te
affri-ca-tes
an-no-tated
com-ple-ments
com-po-si-tio-na-li-ty
non-com-po-si-tio-na-li-ty
Gon-zá-lez
out-side
Ri-chárd
se-man-tics
STREU-SLE
Tie-de-mann
}
\hyphenation{
affri-ca-te
affri-ca-tes
an-no-tated
com-ple-ments
com-po-si-tio-na-li-ty
non-com-po-si-tio-na-li-ty
Gon-zá-lez
out-side
Ri-chárd
se-man-tics
STREU-SLE
Tie-de-mann
}
\hyphenation{
affri-ca-te
affri-ca-tes
an-no-tated
com-ple-ments
com-po-si-tio-na-li-ty
non-com-po-si-tio-na-li-ty
Gon-zá-lez
out-side
Ri-chárd
se-man-tics
STREU-SLE
Tie-de-mann
} 
  \togglepaper[27]%%chapternumber
}{}

\begin{document}
\maketitle


\section{Introduction} %1. /
\label{sec:overall:1}

This paper describes the grammatical means for expressing %
reflexive and reciprocal situations in Aguaruna, a Chicham language spoken in north Peru.\footnote{I use the name Aguaruna when writing in English, as this is the most frequently encountered term. The language is officially named \textit{awajún} in Peru, and native speakers I have worked with typically refer to it as \textit{iinia chicham}. The ISO 639-3 code is agr, and glottocode agua1253.} While the paper is largely descriptive in nature, it also aims to situate the description in the typological literature as much as possible.\footnote{I thank all the Aguaruna speakers who have helped me learn and understand their language, especially Lady Akintui Tsajuput, Yanua Atamain Uwarai and Eduardo Cungumas Kujancham for answering my questions about reflexive constructions. The paper also benefitted from the editors’ very helpful comments. Of course I take full responsibility for any remaining errors of fact or interpretation.}

The Chicham family (formerly known as Jivaroan) consists of five closely related varieties, defined politically as distinct languages. In addition to Aguaruna, the other languages are Shuar, Wampis, Shiwiar, and Achuar. Aguaruna is the most distinct, at least from a phonological perspective, but speakers of all varieties are generally able to converse, although this may involve some initial difficulty in accommodating to differences (see \citep{OverallKohlbergerinprep}, for more detailed description of the Chicham family). All five languages are spoken in the south of Ecuador and north of Peru, in an area mostly within the Santiago, Pastaza and Marañón River basins (see Map 1). This area is linguistically diverse, and in addition to Chicham languages there are populations speaking languages from other families (Quechuan, Kawapanan) as well as some isolates (Kandozi-Chapra). Linguistic diversity in the Marañón River basin appears to have been even higher at the time of the Spanish invasion (\citealt{AdelaarMuysken2004}: 172–173). Although this paper relates solely to Aguaruna, the facts are substantially the same for the other languages of the
family.

The description presented here is based on field data collected in various visits since 2004. Examples are cited in the same orthography used by \citet{Overall2017}, in which the following graphs differ from IPA values: <ch> = /t͡ʃ/, <sh> = /ʃ/, <y> = /j/. Where examples are not taken from a published source, they are cited with the filename of the relevant recording; these recordings are currently being prepared for archiving. Examples are from recorded narrative except where otherwise specified. Original orthograpy is indicated by angle brackets.

The structure of the paper is as follows: \sectref{sec:overall:2} gives an overview of Aguaruna grammar. \sectref{sec:overall:3} describes the formal marking of
reflexive and reciprocal constructions, and their interaction with other categories, and \sectref{sec:overall:4} goes into more detail regarding the semantic range of reflexive and reciprocal. \sectref{sec:overall:5} discusses how Aguaruna fits into areal patterns, and \sectref{sec:overall:6} offers a brief conclusion.

\section{Typological profile and grammar overview} %2. /
\label{sec:overall:2}

Aguaruna is nominative-accusative, and typically shows SV/APV constituent order. The morphology is almost entirely suffixing, basically agglutinating, and shows both head and dependent marking: at the clause level, subjects and speech act participant (SAP) objects are indexed with verbal suffixes, and NP arguments are marked for case; and within the possessive NP, possessed nouns are morphologically marked as possessed, along with person and number of the possessor, and possessors are also marked (lexical nouns take a genitive form, pronominal possessors take accusative case). Example \REF{ex:overall:1} illustrates a possessive NP with lexical possessor. Note that there is no grammatical way to disambiguate third person possessors (e.g. \textit{she}\textsubscript{i} \textit{cut her}\textsubscript{i/j} \textit{hand}) – see \sectref{sec:overall:4.2} for examples.

\ea%1
    \label{ex:overall:1}
 \glll washí yakah\'ĩ\\
    {}[washi  yaka-hĩ]\textsubscript{NP}\\
  monkey.\textsc{gen}  arm-\textsc{pssd.3}\\
  \glt  ‘the monkey’s arm’ [agr040723\_29]
\z

While the phonology is relatively straightforward, productive processes of vowel elision can obscure the agglutinating nature of the morphology. Vowel nasality is contrastive and spreads to adjacent vowels and glides, and the nasal consonants /m, n/ may be denasalized when followed by oral vowels (see \citealt{Overall2017}: 67–71 for details).

\subsection{Finite and non-finite verbs} %2.1 /
\label{sec:overall:2.1}

Verbs are obligatorily inflected, and verbal morphology shows a clear distinction of finiteness. Finite verbs are marked for the following verbal grammatical categories: aspect, tense, person/number and mood/modality. Verbal morphology is entirely suffixing apart from an unproductive causative prefix, and can be usefully viewed in terms of morphological slots, as in the schematic overview in \figref{fig:overall:1}.


\begin{figure}
\begin{tabularx}{\textwidth}{XXXXXXXX}
\lsptoprule
& A & B & C & D & E & F & G\\
\hline
\textsc{root} & Valency & Object & Aspect & Negation & Tense & Subject & Mood\\
\lspbottomrule
\end{tabularx}
\caption{Morphological slots in the verbal word}
\label{fig:overall:1}
\end{figure}

\tabref{tab:overall:1} shows the slot F suffixes that mark subjects in finite verbs. For second and third person, there is some allomorphy triggered by tense. The distinction between singular and plural number is only consistently maintained in first person; third person does not distinguish number and the second person “singular” form may also be found with plural reference, where number is irrelevant or apparent from the context. Note that plural subject can also optionally be specified along with aspect marking in slot C, independently of the person marking.


\begin{table}
\caption{Finite verbal subject markers}
\label{tab:overall:1}
\begin{tabularx}{0.8\textwidth}{p{1.3cm}p{3.5cm}p{1.7cm}p{1.7cm}}
\lsptoprule
\textsc{person} & \textsc{tense} & \textsc{marker} & \\
&  & \textsc{sg} & \textsc{pl}\\
\midrule
1 & all tenses & \textit{\nobreakdash-ha} & \textit{\nobreakdash-hi}\\
2 & past tenses & \-\textit{\nobreakdash-umɨ} & \textit{\nobreakdash-uhumɨ}\\
& non-past tenses & \textit{\nobreakdash-mɨ} & \textit{\nobreakdash-humɨ}\\
3 & present and definite future tenses & \multicolumn{2}{c}{\textit{\nobreakdash-wa}}\\
& other tenses & portmanteau + person & tense markers\\ %\multicolumn{2}{c}{portmanteau tense + person markers}\\
\lspbottomrule
\end{tabularx}
\end{table}

The categories of slots B to G are obligatorily specified, but not always overtly morphologically marked: in some slots, absence of a marker contrasts meaningfully with presence of a marker.

Aguaruna makes heavy use of non-finite clause types in clause-chaining constructions, especially in narrative texts. These clauses are morphosyntactically dependent in that they can only appear in a construction with an associated finite predicate: the verbs of dependent clauses are marked for most of the same categories as finite verbs, but lack tense and mood marking; they are also marked for switch-reference (same-subject versus different-subject, and some more specific relations). Nominalizations are also widely used, forming relative and complement clauses and also functioning in lieu of finite verbs in some contexts such as traditional narratives (\citealt{Overall2017}: 537–540; and see detailed discussion in \citealt{Overall2018}).

Reflexive and reciprocal markers are valency changing derivations and appear in slot A; they can appear in all verb forms, including subordinate verbs and nominalizations. 

\subsection{Grammatical relations and object marking in the verb} %2.2 /
\label{sec:overall:2.2}

Aguaruna shows nominative-accusative alignment. This is manifested in case marking of NPs and verbal agreement, as well as grammatical processes such as nominalization and switch-reference, which distinguish subject (S or A) from non-subject (objects and obliques). For example, the non-subject nominalizer \textit{{}-taĩ} forms a nominal that may refer to the notional object (\textit{yu-taĩ} eat-\textsc{nmlz} ‘food’), instrument (\textit{aɰa-taĩ} write-\textsc{nmlz} ‘pen’) or location (\textit{kanu-taĩ} sleep-\textsc{nmlz} ‘dormitory’) \citep[267]{Overall2017}.
The objects of underived ditransitive clauses, as well as those added by valency increasing derivation, are also apparently identical to those of monotransitive clauses in their case marking, agreement, and syntactic behaviours such as nominalization and switch-reference \citep[269]{Overall2017}. Compare the \textsc{1sg} object of an underived simple transitive clause in \REF{ex:overall:2}, recipient of the underived ditransitive clause in \REF{ex:overall:3}, and object of applicative derivation in \REF{ex:overall:4}, all of which are identically marked with accusative case and with verbal object agreement.

\ea%2
    \label{ex:overall:2}
    \glll mína huhuktá\\
 mi=na  hu-hu-ki-ta\\
  \textsc{1sg}=\textsc{acc}  carry-\textsc{1sg.obj-pfv-imp}\\
  \glt  ‘carry me!’ \citep[281]{Overall2017}
\z

\ea%3
    \label{ex:overall:3}
    \glll mína suhustá\\
  mi=na  su-hu-sa-ta\\
  \textsc{1sg=acc}  give\textsc{{}-1sg.obj-pfv-imp}\\
  \glt  ‘give it to me’ \citep[243]{Overall2017}
\z

\ea%4
    \label{ex:overall:4}
    \glll mína túhutmɨ\\
  mi=na  tu-hu-tu-mı̵\\
  \textsc{1sg}=\textsc{acc}  say-\textsc{appl-1sg.obj-recpst.3.decl}\\
  \glt  ‘(she) told me’ \citep[304]{Overall2017}
\z

Verbs fall into two classes, manifested in the forms of the applicative suffix in slot A and the object marking suffixes in slot B, which show initial /h/ or /t/ depending on the class of the verb. The applicative suffix has the form \textit{{}-hu} or \textit{{}-tu}, and the first person singular object suffix has the same form – but where applicative and first singular object co-occur, they alternate \textit{h/t} forms (as in example 4 above). The second person object suffix has the basic forms \textit{\nobreakdash-hama} or \textit{\nobreakdash-tama}, with a variant \textit{\nobreakdash-pa} that seems to be phonologically conditioned \citep[244]{Overall2017}. First plural object is generally marked identically to second person, except that the form -\textit{kahatu} can be used where second person is specifically excluded, and is also used to mark generic human objects. Only SAP objects are indexed with verbal suffixes – third person objects are always zero-marked. There is no difference in verbal indexing of notional direct, indirect or derived objects, but only one object can be indexed on the verb. \citet[275]{Overall2017} shows that speakers avoid grammatical configurations that trigger competition for this marking slot, that is, clauses that include two SAP objects. Object marking is obligatory, and may co-occur with overt object NPs, as in examples (2–4) above. Examples \REF{ex:overall:5} and \REF{ex:overall:6} illustrate simple SAP object marking, and \REF{ex:overall:7} shows a SAP object added by the applicative derivation.

\ea%5
    \label{ex:overall:5}
    \glll ũỹũ\'ntusta\\
  uyun-tu-sa-ta\\
  accompany-\textsc{1sg.obj-pfv-imp}\\
  \glt  ‘go with me!’ [agr040721\_07]
\z

\ea%6
    \label{ex:overall:6}
    \glll áu waipákmɨ\\
  au  wai-pa-ka-mɨ\\
  \textsc{dem.dist}  see-\textsc{2.obj-pfv-recpst.3.decl}\\
  \glt  ‘s/he saw you’ \citep[314]{Overall2017}
\z


\ea%7
    \label{ex:overall:7}
    \glll pasún miníthamkũĩsh\\
  pasun  mini-tu-hama-ku-ĩ=sha\\
  evil.spirit  arrive-\textsc{appl-2.obj.ipfv-sim-ds=conces}\\
  \glt  ‘even though an evil spirit arrives to your detriment’ [agr041005\_21]
\z

The combination of first person A and second person P does not involve object marking in slot B (Object), instead being marked in slot F (Subject) with the suffix \textit{{}-hamɨ} if both arguments are singular (as in example 8) or \nobreakdash-\textit{himɨ} if either or both of the arguments is plural. Although these forms are clearly based on first person markers \textit{\nobreakdash-ha} (\textsc{sg)} \textsc{/} \textsc{\nobreakdash-}\textit{hi} (\textsc{pl}) + second person \textit{\nobreakdash-mɨ}, their non-combinatorial semantics with respect to number leads \citet[244–245]{Overall2017} to treat them as portmanteau morphs.

\ea%8
    \label{ex:overall:8}
    \glll kamɨ yabái wíshakam dɨkáhuahamɨ\\
  kamɨ  yabai  wi=shakama  dɨka-hu-a-hamɨ\\
  indeed  now  1\textsc{sg}=\textsc{add}  know-\textsc{appl-pfv-1sg.sbj/2sg.obj.decl}\\
  \glt  ‘now I know that about you too’ [agr041005\_21]
\z

Two productive valency-increasing operations are marked with suffixes in slot A (valency), these are applicative \textit{{}-hu}/\textit{{}-tu} and causative \textit{{}-mitika} . Both operations increase the valency of the verb by one, adding an object to the clause. Applicative derivation straightforwardly adds an object argument, semantically typically a beneficiary (as is the added 1\textsc{sg} object in example 9) or maleficiary (as in example 7 above). In the case of causative, there is a rearrangement of roles from the underived clause, as the added ‘causer’ argument is the subject and the notional subject of the causativized verb becomes an object (‘causee’) (example 10). 

\ea%9
    \label{ex:overall:9}
    \glll minásh batáɨ ukuithúkta\\
  mi=na=sha  bataɨ  ukui-tu-hu-ka-ta\\
  1\textsc{sg}=\textsc{acc=add}  chambira  detach-\textsc{appl-1sg.obj-pfv-imp}\\
  \glt  ‘get some chambira (fruit sp.) for me too!’ \citep[302]{Overall2017}
\z

\ea%10
    \label{ex:overall:10}
    \glll ámɨ mína dushímtihamɨ\\
  amɨ  mi=na  dushi-mitika-ha-mɨ\\
  2\textsc{sg}  1\textsc{sg}=\textsc{acc}  laugh\textsc{{}-caus-1sg.obj.ipfv-2sg.decl}\\
  \glt  ‘you are making me laugh’ \citep[300]{Overall2017}
\z

A set of verbs form causatives not with the slot
A (valency) suffix but with a prefixed vowel whose quality is not completely predictable: \textit{ɨ-tsɨkɨ-} (\textsc{caus-}jump-) ‘startle’; \textit{i-ta-} (\textsc{caus\nobreakdash-}arrive\nobreakdash-) ‘bring’.

A few verb roots show unproductive phonological alternants with differing transitivity values. In general, the intransitive variant is the more marked member of such pairs, for example \textit{shiki-} ‘urinate on (transitive)’, \textit{shiki-pa-} ‘urinate (intransitive)’, with unproductive detransitivizer \textit{{}-pa}.

Reflexive and reciprocal markers are the only productive valency reducing operators, and their formal properties form the topic of the following section.

\section{Reflexive and reciprocal marking in the verb}
\label{sec:overall:3}

Reflexive and reciprocal constructions encode situations in which there is coreference between two semantic participants. Reflexive applies to verb roots that typically appear in transitive clauses, and signals coreferentiality of the notional A and P arguments. Reciprocal marking similarly signals coreference of A and  P arguments, but acting on each other rather than on themselves. The reciprocal construction therefore implies two or more participants, at least semantically.

In Aguaruna, both reflexive and reciprocal derivations are marked with verbal suffixes in slot A (\figref{fig:overall:1}): reflexive
\textit{{}-m(a)} or \textit{{}-mam(a)}; and reciprocal \textit{{}-n(a)i}, with denasalized form \textit{{}-d(a)i}.\footnote{The bracketed vowels are elided in phonologically predictable environments. The selection of \textit{{}-ma} or \textit{{}-mama} appears to be lexically conditioned.}


At first glance, these markers appear to function as members of the object-marking paradigm. Like object markers, reflexive and reciprocal are obligatory whenever there is an appropriate configuration of subject and object. In the examples in \REF{ex:overall:11}, the SAP object markers in (a) and (b) appear to form a paradigm with the reflexive marker in (c). Similarly, compare the verb marked with the reciprocal suffix in \REF{ex:overall:12} with the same verb marked for second person object in \REF{ex:overall:6} above – both the reciprocal and the object suffix appear directly following the root, and preceding the aspect marker.

\ea%11
    \label{ex:overall:11}
\ea
    \glll áu tsupíŋkamɨ\\
    au  tsupi-hu-ka-mɨ\\
    \textsc{dem.dist}  cut\textsc{{}-1sg.obj-pfv-recpst.3.decl}\\
    \glt ‘s/he cut me’ \citep[247]{Overall2017}

\ex
\glll tsupíŋmakmɨ\\
    tsupi-hama-ka-mɨ\\
    cut\textsc{{}-2.obj-pfv-recpst.3.decl}\\
\glt    ‘he has cut you’ \citep[307]{Overall2017}

\ex
\glll {tsupímakmɨ}\\
    tsupi-ma-ka-mɨ\\
    cut\textsc{{}-refl-pfv-recpst.3.decl}\\
\glt    ‘he has cut himself’ \citep[307]{Overall2017}
\z
\z

\ea%12
    \label{ex:overall:12}
    \glll ãhúm wainiámi\\
  ãhum  wai-nai-a-mi\\
  later  see-\textsc{recp-pfv-hort}\\
\glt ‘let’s meet (i.e. see each other) later’ \citep[424]{Overall2017}
\z

But \citet[306]{Overall2017} points out that
reflexive and reciprocal markers are not compatible with overt object NPs. This indicates that they are in fact valency reducing, and can be considered to constitute
\textsc{reflexive} \textsc{voice} and \textsc{reciprocal} \textsc{voice}, respectively (in the sense of \citealt{Kulikov2011}; and see Haspelmath, this volume, §5.2--5.3). \todo{crossreference to be added}In contrast, the object markers are compatible with overt NPs (13, 14) and are therefore more like agreement. Outside of elicitation contexts, overt pronouns are more likely to appear in emphatic contexts such as \REF{ex:overall:15}, where the pronominal object NP is separated from the verb by the multi-word subject NP.

\ea%13
    \label{ex:overall:13}
    \glll mína ɨsátnɨ\\
  mi=na  ɨsa-tu-ini-ɨ\\
  1\textsc{sg=acc}  bite\textsc{{}-1sg.obj-pfv-3.decl}\\
  \glt  ‘it bit me’ \citep[293]{Overall2017}
\z

\ea%14
    \label{ex:overall:14}
    \glll mína suhustá\\
  mi=na  su-hu-sa-ta\\
  1\textsc{sg=acc}  give\textsc{{}-1sg.obj-pfv-imp}\\
  \glt  ‘give it to me’ \citep[243]{Overall2017}
  \z

\ea%15
    \label{ex:overall:15}
    \glll amina apahui tukɨ puhuwa nuu yaimpakti\\
  ami=na [apahui tukɨ puhu-wa nu] yaĩ-pa-ka-ti\\  2\textsc{sg=acc}\textit{\textsubscript{i}}  God  always  live\textsc{{}-3} \textsc{ana} help-\textsc{2.obj}\textit{\textsubscript{i}}\textsc{{}-pfv-juss}\\
  \glt  ‘may God, who is eternal, help you’ (personal correspondence)
\z

There is no reflexive or reciprocal pronoun, and indeed the valency-reducing nature of these constructions means that there would be no function for such a pronoun, since it would be expected to occupy the object role.

As noted above, overt pronouns are used in emphatic contexts. Example \REF{ex:overall:16} illustrates such a context with a reflexive marked verb: a man (subject of the final nominalized verb \textit{wainkau} ‘saw’) discovers that his younger brother is turning into a monster and eating himself. The verb ‘eat’ is marked with the reflexive suffix, and the unexpected nature of this situation is signaled by representing the subject with an overt pronoun marked with the enclitic \textit{=ki} (glossed ‘restrictive’ following \citealt{Overall2017}, and indicating exhaustive focus). Note that a bilingual speaker translated this into Spanish with the emphatic reflexive \textit{sí mismo}.

\ea%16
    \label{ex:overall:16}
    \glll níŋki yúmamak puhúttaman wainkáu\\
  {}[nĩ=ki  yu-mama-a-kũ  puhu-tatamana]  waina-ka-u\\
  3\textsc{sg}=\textsc{restr}  eat-\textsc{refl-ipfv-sim.3.ss}  live-\textsc{sbj>obj}  see-\textsc{pfv-nmlz}\\
  \glt  ‘he\textit{\textsubscript{i}} saw that he\textit{\textsubscript{j}} was eating himself’ (\textit{vio que estaba comiendo en sí mismo}) (agr040720\_22)\footnote{The final verb is nominalized and functioning as a finite verb, a frequent construction in traditional narratives (cf. \sectref{sec:overall:2.1}). The auxiliary verb ‘live’ in the bracketed clause is marked for switch-reference indicating that its subject is coreferent with the object of the final verb (see \citealt{Overall2017}, \sectref{sec:overall:13.6}).}
\z

While their interaction with the object marking paradigm and their obligatoriness make reflexive and reciprocal markers appear more like traditional inflection, they also show properties that align them with traditional derivation. In particular, some stems are lexicalized and show non-combinatorial semantics. Lexicalized reflexives include \textit{su-ma-} (give-\textsc{refl-}) ‘buy’ (not ‘give to oneself’; but cf. reciprocal ‘give to each other’ in example 30 below); and \textit{wai-ma}{}- (see-\textsc{refl-}) ‘see a vision under the influence of hallucinogens’. In order to express the meaning ‘see oneself’, a different verb root \textit{nii-} ‘look at’ is used: \textit{nii-ma}{}- (look.at-\textsc{refl-}) ‘look at oneself’
.\footnote{Yanua Atamain, personal communication and Eduardo Cungumas, personal communication.}

Lexicalized reciprocal forms include \textit{ɨŋkɨ-ni}{}- ‘hold hands’ < \textit{ɨŋkɨ}{}- ‘put away, keep safe, load gun’; and \textit{maa-ni-} (kill-\textsc{recp-}) ‘fight’.\footnote{The verb ‘kill’ shows some variation, surfacing as /ma/, /maa/ or /mã/ (cf. example 18) depending on its morphological context.} In order to express the sense ‘kill each other’, one can use a different verb, such as \textit{amu-} ‘finish off’ – this verb can refer to finishing up a serving of food or drink, or to exterminating a group of people. Its reciprocal marked form appears in the place name \textit{wɨɰa amuníkbau} \REF{ex:overall:17}, the site of a historic battle with many casualties.

\ea%17
    \label{ex:overall:17}
    \glll wɨɰa amuníkbau\\
  wɨɰa  amu-nai-ka-mau\\
  ancestor  finish.off-\textsc{recp-pfv-nmlz}\\
  \glt  ‘place of the ancestors killing each other’ [agr041005\_18]
\z

In sum, although I have labelled reflexive and reciprocal as derivational markers (cf. Haspelmath, this volume \sectref{sec:overall:5.2}), I note that “the traditional division into derivational and inflectional morphology is not a very useful one for Aguaruna verbs” (\citealt{Overall2017}: 233; cf. \citealt{Plungian2001}).

\subsection{Applicative and reflexive verbal markers}
\label{sec:overall:3.1}
Reflexive and reciprocal markers can co-occur with the applicative suffix, which they may precede or follow, depending on the semantics. The lexicalized reflexive and reciprocal verb stems, with non-combinatorial semantics, are treated like underived roots in having the applicative derivation added to them. The verb root \textit{ɨkɨ-} ‘move something into another position’, ‘put’ has a lexicalized reflexive form \textit{ɨkɨ-ma-} (put-\textsc{refl-)} with the specific meaning ‘sit down’. This stem may then take the applicative suffix \textit{ɨkɨ-ma-tu-} (put-\textsc{refl-appl-)} giving the meaning ‘sit on something’ (\citealt{Overall2017}: 308–309). On the other hand, reflexive and reciprocal markers can occupy the morphological slot immediately following the applicative suffix, marking the notional object of the applicative and giving a self-benefactive construction, as in \REF{ex:overall:18} where the applicativized stem \textit{mã-tu-} (kill-\textsc{appl-)} \textsc{‘}kill for someone’ is reflexivized to give the
sense ‘kill for oneself’; similarly in \REF{ex:overall:19}.

\ea%18
    \label{ex:overall:18}
    \glll wɨkaɨɰák wɨuwai kuntínun mantumaátatus\\
  wɨkaɨɰa-k\~u  wɨ-u=ai  [kuntinu=na mã-tu-ma-a-tatus]\\
  walk\textsc{.ipfv-sim.3.ss}  go\textsc{.pfv-nmlz=cop.3.decl}  animal\textsc{=acc} kill\textsc{{}-appl-refl-pfv-intent.3.ss}\\
  \glt  ‘he went walking to kill animals for himself’ (i.e. ‘he went hunting’) \citep[492]{Overall2017}\footnote{Note that the main verb in this example (‘he went’) is nominalized and formally marked as the complement of the copula enclitic (see detailed discussion of this construction in \citealt{Overall2018}).}
\z

\ea%19
    \label{ex:overall:19}
    \glll yúpichu huhumtáyamɨ\\
  yupichu  hu-hu-ma-tayamɨ\\
  easy  take-\textsc{appl-refl-norm}\\
  \glt  ‘we easily take it away (for ourselves)’ \citep[617]{Overall2017}
\z

Similar examples can be found for reciprocal marking. The non-combinatorial stem \textit{maa-ni-} (kill-\textsc{recp-}) ‘fight’ (not ‘kill each other’), can be applicativized to give \textit{maa-ni-tu-} (kill-\textsc{recp-appl-}) ‘fight for something’. On the other hand, the verb root \textit{kanu-} ‘sleep’ can be applicativized to give a stem meaning ‘reach the same spirit power as someone by having the same dream’, and this stem in turn can take a reciprocal marker following the applicative suffix: \textit{kanu-tu-dai-} (sleep-\textsc{appl-recp-}) ‘reach the same spirit power as each other’.

\subsection{Reciprocal and plurality} %3.2 /
\label{sec:overall:3.2}

Although a reciprocal situation must involve multiple participants semantically, these are not necessarily encoded as plural subjects. \citet{Overall2017} gives the following elicited example \REF{ex:overall:20} of the derived verb stem \textit{maa-ni-} (kill-\textsc{recp-}) ‘fight’. Although there is semantically more than one participant, the verb is marked for first person singular subject, and no other participant is mentioned.

\ea%20
    \label{ex:overall:20}
    \glll kashín wíi maániktathai\\
  kashini  wi  maa-nai-ka-tata-ha-i\\
  tomorrow  1\textsc{sg}  kill-\textsc{recp-pfv-fut-1sg-decl}\\
  \glt  ‘tomorrow I’m going to fight’ \citep[311]{Overall2017}
\z

There is no direct NP coordination in Aguaruna, instead the comitative enclitic \textit{=haĩ} may be used to express plural participants. NPs marked with this enclitic may be treated as conjoined or simply oblique; that is, [NP\textsubscript{SUBJECT} NP=\textit{haĩ}] may trigger singular or plural subject marking. Example \REF{ex:overall:20} can be read as having an implied second participant treated as an oblique NP and therefore not reflected in the verb inflection.

The narrative passage in \REF{ex:overall:21} illustrates this use of comitative \textit{=haĩ}, combined with the indeterminacy of number marking. The subordinate verbs are marked simply for third person subject, unspecified for number. The woman was the subject of the previous clause and is the implied subject here; the husband must be interpreted as a semantic participant but it remains ambiguous as to whether he is treated as a syntactic subject.\todo{check brackets}

\ea%21
    \label{ex:overall:21}
    \glll aíshihãĩ maá maániak\~uã nuwanṹĩ chicháman ɨpɨŋkã huwáku túwahamɨ\\
 [aishĩ=haĩ  maa  maa-nai-a-kawã]  nuwanu=ĩ   [chichama=na  ɨpɨŋkɨ-kã]  huwa-ka-u  tuwahamɨ\\
  husband\textsc{.pssd.3=com}  \textsc{redup}  kill\textsc{{}-recp-ipfv-repet.3.ss}  \textsc{ana=loc} problem\textsc{=acc} resolve\textsc{{}-pfv.3.ss} stay\textsc{{}-pfv-nmlz}  \textsc{narr}\\
  \glt  ‘(the woman) fighting with her husband, they then resolved their problems, so the story goes’ \citep[311]{Overall2017}
\z

\section{Semantics of reflexive constructions} %4. /
\label{sec:overall:4}

The previous section has described the details of formal marking of
reflexive and reciprocal constructions. As shown above, the reflexive and reciprocal suffixes interact with a paradigm of object markers on the verb, clearly distinguishing situations in which the notional subject and object are coreferent from those in which they are not. At the level of the clause, these constructions reduce valency, making the appearance of an object NP impossible. This section goes into more detail regarding the semantic effects of the reflexive and reciprocal constructions in Aguaruna.

\subsection{Extroverted and introverted verb types} %4.1 /
\label{sec:overall:4.1}

Extroverted verbs describe actions that prototypically apply to a second participant, while introverted verbs are those that describe prototypically self-directed actions
\citep[803]{Haiman1983}. There is no evidence that the Aguaruna reflexive or reciprocal constructions behave differently in their morphology or syntax with different semantic classes of verbs, but a few examples of verbs with inherently reflexive semantics but no overt reflexive marking are all of the introverted semantic type, as predicted by \citet{Haiman1983}.

The extroverted verb type was illustrated with the verb \textit{tsupi-} ‘cut’ in examples (11 a–c) in \sectref{sec:overall:3} above. Similarly, \textit{ɨtɨ}{}- ‘beat with nettle’ (< \textit{ɨtɨ} ‘wasp’?), forms the reflexive as \textit{ɨtɨ-ma}{}- (beat.with.nettle-\textsc{refl{}-) ‘}beat oneself with nettle’. The extroverted verb \textit{ma-} ‘kill’ is illustrated in example \REF{ex:overall:22}.

\ea%22
    \label{ex:overall:22}
    \glll ã\'\~{w}ĩ dakáka maámi\\
  au=ĩ  daka-ka  ma-a-mi\\
  \textsc{dem.dist}=\textsc{loc}  wait-\textsc{pfv.1pl.ss}  kill-\textsc{pfv-hort}\\
  \glt  ‘let’s ambush him there and kill him!’ [agr041005\_19]
\z

Adding reflexive gives the sense ‘kill oneself’ (\citealt{UwaraiEtAl1998}: 76 translate the stem \textit{maa-ma-} (kill-\textsc{refl}{}-) into Spanish as \textit{suicidarse} ‘commit suicide’). Example \REF{ex:overall:23}, from a translation of the New Testament, relates how a jailer had drawn his sword to kill himself after thinking that the people he was guarding had escaped.\footnote{The relevant passage is Acts 16:28, translated in the \textit{New International Version} as: “But Paul shouted, ‘Don't harm yourself! We are all here!’”.}

\ea%23
    \label{ex:overall:23}
  \glll <Nunitai Pablo senchi untsuká: –Maamawaipa, jutiik ashí betek batsatji, –tiuwai.> \\
  nuni-taĩ  Pablo  sɨnchi  untsu-kã  “maa-ma-aw-aipa hutii=ka  ashi  bɨtɨka  batsata-hi”  ti-u=ai\\
  do.that-\textsc{3.ds}  Paul  strongly  call-\textsc{pfv.3.ss}  kill-\textsc{refl-pfv-proh} 1pl=\textsc{top}  all  full  be.\textsc{pl.ipfv-1pl}  say.\textsc{pfv-nmlz=cop.3.decl}\\
  \glt  ‘when he did that, Paul called out loudly, “don’t kill yourself! we are all here!” he said.’ (\citealt{LaLigaBiblica2008}: 245)
  \z

The introverted verb type can be illustrated with the verb \textit{ayamhu-} ‘defend’. Example \REF{ex:overall:24} shows a simple transitive use of this verb; in \REF{ex:overall:25} it is marked with first person singular object; and in \REF{ex:overall:26} it is reflexivized to give ‘defend oneself’.

\ea%24
    \label{ex:overall:24}
    \glll makíshkish ayamhúkchahui\\
  makichiki=sha  ayamhu-ka-cha-aha-u=i\\
  one=\textsc{add}  defend-\textsc{pfv-neg-pl-nmlz=cop.3.decl}\\
  \glt  ‘not even one defended him’ \citep[195]{Overall2017}
\z

\ea%25
    \label{ex:overall:25}
    \glll “ikámỹã\~{w}ã tukúhui, ayamhútkata!” tus untsúmu\\
  {}[ikama\_yawaã  tuku-hu-a-wa-i  ayamhu-tu-ka-ta tus]  untsuma-u\\
  jaguar  attack-\textsc{1sg.obj-ipfv-3-decl}  defend-\textsc{1sg.obj-pfv-imp} say\textsc{.sbd.3.ss}  call\textsc{.ipfv-nmlz}\\
  \glt  ‘“A jaguar is attacking me! Help me!” he was calling’ \citep[561]{Overall2017}
  \z

\ea%26
    \label{ex:overall:26}
    \glll yuwáta táma nuní áyamhumak {…}\\
  {}[yu-a-ta-ha  ta-ma]  nuni ayamhu-ma-kã { }\\
  eat\textsc{{}-pfv-ifut-1sg} say\textsc{.ipfv-nsbj>sbj}  thus  defend-\textsc{refl-pfv.3.ss} { }\\
  \glt  ‘when (the puma) tried to eat him, he defended himself like that…’ lit. when (the puma) said “I will eat him!”… \citep[565]{Overall2017}\footnote{Note that the verb ‘say’ in the bracketed clause is marked for switch-reference indicating that a non-subject participant (the object, in this example) is the subject of the controlling clause (see \citealt{Overall2017}: \S 13.6).}
\z

Verbs of grooming fall into the introverted semantic class, and are typically reflexivized, with the unmarked root being transitive. For example, \textit{tɨmashi-} ‘comb someone’s hair’, \textit{tɨmash-ma-} (comb.hair-\textsc{refl-}) ‘comb one’s own hair’, as shown in \REF{ex:overall:27}.

\ea%27
    \label{ex:overall:27}
    \glll wíi tɨmáshmahai\\
  wi  tɨmashi-ma-ha-i\\
 1\textsc{sg}  comb-\textsc{refl.ipfv-1sg-decl}\\
  \glt  ‘I am combing my hair’ (cf. \citealt{Overall2017}: 306)
\z

The verb \textit{ikiŋ-ma-} ‘wash one’s hands’ is also reflexive, the stem \textit{ikihu}{}- meaning ‘wash someone’s hands’.\footnote{This stem may include the causative prefix \textit{V-}, and is perhaps related to semantically similar verbs \textit{kita}{}- ‘drip’, \textit{kitama}{}- ‘be thirsty’, \textit{kiha}{}- ‘absorb liquid nasally’. It may also include the applicative suffix \textit{{}-hu}.} These verbs treat the person being groomed as the object, not the specific affected body part (i.e. ‘hair’ and ‘hands’ in these examples are encoded as part of the verbal semantics and not treated as participants).

Although most introverted actions are expressed with reflexivized verbs, there are also some underived verbs of this type, as predicted by \citet[803ff]{Haiman1983}. For example, the verb \textit{niha-} ‘wash (clothes etc.)’ is not reflexivized to describe people washing themselves, instead there is an underived intransitive verb \textit{maa}{}- ‘bathe’. This verb can in turn be causativized to give \textit{i-ma-} (\textsc{caus-}bathe-) ‘bathe someone (such as a child)’.

Verbs describing inherently reciprocal actions tend to be basically transitive and take reciprocal marking: \textit{iŋku-ni-} (meet-\textsc{recp}{}-) ‘meet each other’, \textit{kumpam-dai-} ‘greet each other’,\footnote{The /kumpa/ element is from Spanish \textit{compadre} ‘close friend’.} in addition to \textit{maa-ni-} (kill-\textsc{recp-}) ‘fight’ already mentioned above. 

\subsection{Exact and partial coreferences} %4.2 /
\label{sec:overall:4.2}

I have not encountered any clear examples of the contrast between exact and inclusive coreference, of the type that would distinguish \textit{he defended himself} from \textit{he defended [himself and others]}. The comitative marker described in
\sectref{sec:overall:3.2} above would presumably allow such non-exact coreference to be encoded with the standard reflexive construction.

With respect to actions directed at body parts, the examples of grooming verbs given above (\sectref{sec:overall:4.1}) illustrate a strategy of lexicalizing the action as a transitive verb with the possessor of the body part (not the body part itself) as object. These introverted verbs can be reflexivized with the standard reflexive construction (as in example 27 above). With extroverted verbs directed at body parts, however, the body part itself is the grammatical object, heading its own NP. Compare example \REF{ex:overall:28}, in which the subject of the verb \textit{hu-} ‘take’ is possessor of the object, the possessed noun \textit{katĩ} ‘his penis’, and example \REF{ex:overall:29}, in which the subject of the same verb \textit{hu-} ‘take’ is different from the possessor of the object NP headed by the possessed noun \textit{bakui-chi-hĩ} (thigh.\textsc{pssd}{}-\textsc{dim-pssd.3}) ‘his little thigh’. As noted in \sectref{sec:overall:2} above, there is no way to disambiguate third-person possessors (‘his’ vs ‘his own’) other than by adding a lexical possessor NP: the same suffix \textit{{}-hĩ} (\textsc{{}-pssd.3sg}) is used in the situation of coreference in \REF{ex:overall:28}, and in disjoint reference in \REF{ex:overall:29}. As can be seen in \REF{ex:overall:28}, the reflexive construction is not used when the object is a body-part of the
subject.\todo{ex 28 incomplete - ikák?}

\ea%28
    \label{ex:overall:28}
    \glll katín uwɨhín húkĩ ikák\\
  katĩ=na  uwɨ-hĩ=nĩ  hu-kĩ\\
  penis.\textsc{pssd.3}  hand-\textsc{pssd.3=loc}  take-\textsc{pfv.3.ss}\\
  \glt ‘[the devil] having taken his (own) penis in his hand…’ \citep[467]{Overall2017}
\z

\ea%29
    \label{ex:overall:29}
    \glll núna yachiuchíhin bakuichíhin hukíuwai\\
  nu=na  yachi-uchi-hĩ=na  bakui-chi-hĩ=na   hu-ki-u=ai\\
  \textsc{ana=acc}  brother.\textsc{pssd-dim-pssd.3=acc}  thigh.\textsc{pssd-dim-pssd.3=acc} take-\textsc{pfv-nmlz=cop.3.decl}\\
  \glt  ‘he took his little brother’s little thigh’ [agr041005\_14]
\z

\subsection{Coreference of the subject with various semantic roles} %4.3 /
\label{sec:overall:4.3}

Examples thus far have illustrated verbs whose objects are semantically patients or themes, and these are the targets of reflexive marking.
When combined with applicative derivation, reflexive targets a beneficiary or maleficiary as a grammatical object, as described above
(\sectref{sec:overall:3.1}, \ref{ex:overall:18}--\ref{ex:overall:19}).

The underived ditransitive verb \textit{su-} ‘give’ has a gift and a recipient object, the latter of which is
more likely to be human and therefore potentially coreferent with the subject. There is a semantic change when this verb combines with reflexive, giving the stem
\textit{su-ma-} (give-\textsc{refl}\nobreakdash-) ‘buy’, not ‘give to oneself’. With reciprocal, however, the meaning is compositional \textit{su-nai-} (give-\textsc{recp}{}-) ‘give to each other’, as in \REF{ex:overall:30}.

\ea%30
    \label{ex:overall:30}
    \glll nuwanúi sudáisauwai\\
  nuwanu=ĩ  su-nai-sa-u=ai\\
  \textsc{ana=loc}  give-\textsc{recp-pfv-nmlz=cop.3.decl}\\
  \glt  ‘then they gave each other (their songs)’ [agr041005\_17]
\z

Note that the reflexivized stem \textit{su-ma-} (give-\textsc{refl-)} ‘buy’ has a self-benefactive reading (i.e. ‘buy for oneself’). To express the notion of buying for someone else, the applicative suffix can be added, as in \REF{ex:overall:31}.

\ea%31
    \label{ex:overall:31}
    \glll wíi haánchin sumáŋkathamɨ\\
  wi  haanchi=na  su-ma-hu-ka-ta-hamɨ\\
  1\textsc{sg}  clothes=\textsc{acc}  give-\textsc{refl-appl-pfv-ifut-1sg.sbj/2sg.obj.decl}\\
  \glt  ‘I will buy you clothes’ \citep[309]{Overall2017}
  \z

The verb \textit{tu}{}- ‘say’ takes a speech report complement and may also take an object referring to the addressee, or to a person being spoken about. The latter type of object is the target of reflexive in \REF{ex:overall:32}.

\ea%32
    \label{ex:overall:32}
    \glll atákɨk tumámipa\\
  atakɨ=ka  tu-mami-ipa\\
  again=\textsc{top}  say-\textsc{refl.pfv-proh}\\
  \glt  ‘don’t say that about yourself again’ (agr041005\_22)
\z

It seems clear, then, that any grammatical object is a potential target of reflexivization, regardless of the semantic role it encodes.

\subsection{Long-distance coreference} %4.4 /
\label{sec:overall:4.4}

Where coreference involves an argument in a subordinate clause whose antecedent is in a matrix clause, there may be the possibility of ambiguity of the type seen in English \REF{ex:overall:33}, and reflexive marking may be used to disambiguate in the case of coreference.

\ea%33
    \label{ex:overall:33}
    She\textsubscript{i} thought that she\textsubscript{i/j} had enough money
\z

In Aguaruna, reflexive is not used in such constructions, and in fact there is no chance of ambiguity as subordinate clauses are not finite, and are marked for switch-reference. The nearest construction to a finite subordinate clause is the speech report construction, which is used not only to report direct speech but also for complements of thought, intention and purpose. Because speech reports are always direct speech, there is no chance of the ambiguity seen in \REF{ex:overall:33}, as the equivalent would look like \REF{ex:overall:34} or \REF{ex:overall:35}.

\ea%34
    \label{ex:overall:34}
    She\textsubscript{i} thought saying “she\textsubscript{j} has enough money”
    \z

\ea%35
    \label{ex:overall:35}
    She\textsubscript{i} thought saying “I\textsubscript{i} have enough money”
    \z

The following text examples illustrate coreference and disjoint reference in subjects of subordinate clauses formed with speech reports. In \REF{ex:overall:36} the subject of the matrix clause is the same as that of the apprehensive clause, and since this is a direct speech report it is expressed as first person singular. In \REF{ex:overall:37} the subject of the verb in the speech report is different from that of the matrix clause, consequently it is expressed as third person.

\ea%36
    \label{ex:overall:36}
    \glll áimak ɨmamkɨmas “intáhaiŋ” tus]\\
  aima-a-k\~u  ɨmamkɨma-sã [inta-ha-i-ha  tus]\\ say.\textsc{sbd.3.ss}
  fill-\textsc{ipfv-sim.3.ss}  take.care-\textsc{sbd.3.ss}  break-\textsc{pfv-appr-1sg}\\
 \glt  ‘Filling them carefully, lest he should break them’ lit. saying “may I not break them” \citep[363]{Overall2017}
 \z

\ea%37
    \label{ex:overall:37}
    \glll iwíyahi “tɨpɨstí” tusá\\
  iwi-ya-hi  [tɨpɨ-sa-ti  tu-sa]\\
  raise.hand-\textsc{rempst-1pl.decl}  lie.down-\textsc{pfv-juss}  say-\textsc{sbd.1pl.ss}\\
  \glt  ‘we raised our hands saying “may it stop!”’ \citep[350]{Overall2017}
\z

\section{Areal tendencies} %5. /
\label{sec:overall:5}

Reflexive and reciprocal are valency-reducing derivations in Aguaruna, and this is in keeping with a common pattern in Amazonian languages, but Aguaruna lacks the vagueness that characterizes the detransitivizers of other languages, for example \citet[44]{Derbyshire1999} describes a verbal detransitivizing derivation in most Carib languages “which is added to a transitive stem and carries the meanings of ‘reflexive’ or ‘reciprocal’, or simply ‘intransitive’ which is often best translated as a passive in languages like English”. Similarly: “A number of [Tupí] languages have a general intransitivizing prefix, which covers reflexive, reciprocal and passive” \citep[120]{Rodrigues1999}. Summarizing this trend, \citet[596]{Payne2001} suggests a general detransitivizing affix as an areal feature of Amazonian languages. Aguaruna is only partially in keeping with this trend, as its reflexive and reciprocal markers are detransitivizing verbal derivations, but their semantically specific nature means that they do not follow the tendency towards a single semantically vague detransitivizing derivation. In this, Aguaruna is more akin to the Quechuan languages spoken to the west, which have a range of semantically specific valency changing derivations including reflexive and reciprocal, as well as valency increasing causative and applicative (\citealt{AdelaarMuysken2004}: 229). \citet[31-32]{Overall2017} has observed that Aguaruna grammar shows features of both Amazonian and Andean types, as is to be expected given its location in the foothills of the Andes at the western edge of the Amazon basin. 

\section{Concluding remarks} %6. /
\label{sec:overall:6}

This paper has described the processes of reflexive and reciprocal marking in Aguaruna grammar. The most notable point is that the markers of these functions straddle the divide between traditional notions of derivation and inflection. They reduce the valency of the verb, but they are obligatory and form a paradigm with inflectional categories of participant agreement. The function of reducing valency, rather than marking reflexivity within a syntactically transitive clause, is
 consistent with patterns found in neighbouring Quechuan languages (mentioned in \sectref{sec:overall:5}) and in the wider Amazonian area \citep[187]{Birchall2014}.

There are some clear examples of lexicalized reflexive and reciprocal markers, with attendant semantic narrowing, but these are the exception. For the most part, the semantic effects of these markers are predictable and combinatorial, and this is more like Quechuan languages, in contrast to the Amazonian tendency towards a single, semantically indeterminate, valency reducing derivation.

The description presented above is largely based on textual examples. Future research focusing on elicitation will no doubt help to tease out more details of the subtleties of reflexive and reciprocal marking in
Aguaruna.

\section*{Abbreviations}

Glossing and abbreviations follow the Leipzig standards, with the following abbreviations:\medskip

\noindent
\begin{tabularx}{.45\textwidth}[t]{lQ}
\textsc{1,} \textsc{2,} \textsc{3}  &  first, second, third person\\
\textsc{acc}  &  accusative\\
\textsc{add}  &  additive\\
\textsc{ana}  &  anaphoric pronoun\\
\textsc{appl}  &  applicative\\
\textsc{appr}  &  apprehensive\\
\textsc{caus}  &  causative\\
\textsc{cntr.ex}  &  counter expectation\\
\textsc{com}  &  comitative\\
\textsc{conces}  &  concessive\\
\textsc{cop}  &  copula\\
\textsc{decl}  &  declarative\\
\textsc{dem}  &  demonstrative\\
\textsc{dim}  &  diminutive\\
\textsc{dist}  &  distal\\
\textsc{ds}  &  different subject\\
\textsc{ep}  &  epenthetic segment\\
\textsc{fut}  &  future\\
\textsc{gen}  &  genitive\\
\textsc{hort}  &  hortative\\
\textsc{ideo}  &  ideophone\\
\textsc{ifut}  &  immediate future\\
\textsc{imp}  &  imperative\\
\textsc{intent}  &  intentional\\
\textsc{ipfv}  &  imperfective\\
\textsc{juss}  &  jussive\\
\end{tabularx}
\begin{tabularx}{.45\textwidth}[t]{lQ}
\textsc{loc}  &  locative\\
\textsc{narr}  &  narrative modality\\
\textsc{neg}  &  negative\\
\textsc{nmlz}  &  nominalizer\\
\textsc{norm}  &  normative\\
\textsc{nsbj}  &  non-subject\\
\textsc{obj}  &  object\\
\textsc{pfv}  &  perfective\\
\textsc{pl}  &  plural\\
\textsc{proh}  &  prohibitive\\
\textsc{pssd}  &  possessed form of noun\\
\textsc{recp}  &  reciprocal\\
\textsc{recpst}  &  recent past\\
\textsc{redup}  &  reduplication\\
\textsc{refl}  &  reflexive\\
\textsc{rempst}  &  remote past\\
\textsc{repet}  &  repetitive\\
\textsc{restr}  &  restrictive\\
\textsc{sap}  &  speech act participant\\
\textsc{sbd}  &  subordinate\\
\textsc{sbj}  &  subject\\
\textsc{sg}  &  singular\\
\textsc{sim}  &  simultaneous\\
\textsc{sr}  &  switch-reference\\
\textsc{ss}  &  same subject\\
\textsc{top} & topic
\end{tabularx}

\sloppy\printbibliography[heading=subbibliography,notkeyword=this]
\end{document}