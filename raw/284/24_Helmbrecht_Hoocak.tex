\documentclass[output=paper]{langscibook}

\author{Johannes Helmbrecht \affiliation{University of Regensburg, Germany}}
\title{Reflexive constructions in Hoocąk} 
\abstract{}

\IfFileExists{../localcommands.tex}{
 \addbibresource{localbibliography.bib}
 \usepackage{langsci-optional}
\usepackage{langsci-gb4e}
\usepackage{langsci-lgr}

\usepackage{listings}
\lstset{basicstyle=\ttfamily,tabsize=2,breaklines=true}

%added by author
% \usepackage{tipa}
\usepackage{multirow}
\graphicspath{{figures/}}
\usepackage{langsci-branding}

 
\newcommand{\sent}{\enumsentence}
\newcommand{\sents}{\eenumsentence}
\let\citeasnoun\citet

\renewcommand{\lsCoverTitleFont}[1]{\sffamily\addfontfeatures{Scale=MatchUppercase}\fontsize{44pt}{16mm}\selectfont #1}
   
 %% hyphenation points for line breaks
%% Normally, automatic hyphenation in LaTeX is very good
%% If a word is mis-hyphenated, add it to this file
%%
%% add information to TeX file before \begin{document} with:
%% %% hyphenation points for line breaks
%% Normally, automatic hyphenation in LaTeX is very good
%% If a word is mis-hyphenated, add it to this file
%%
%% add information to TeX file before \begin{document} with:
%% %% hyphenation points for line breaks
%% Normally, automatic hyphenation in LaTeX is very good
%% If a word is mis-hyphenated, add it to this file
%%
%% add information to TeX file before \begin{document} with:
%% \include{localhyphenation}
\hyphenation{
affri-ca-te
affri-ca-tes
an-no-tated
com-ple-ments
com-po-si-tio-na-li-ty
non-com-po-si-tio-na-li-ty
Gon-zá-lez
out-side
Ri-chárd
se-man-tics
STREU-SLE
Tie-de-mann
}
\hyphenation{
affri-ca-te
affri-ca-tes
an-no-tated
com-ple-ments
com-po-si-tio-na-li-ty
non-com-po-si-tio-na-li-ty
Gon-zá-lez
out-side
Ri-chárd
se-man-tics
STREU-SLE
Tie-de-mann
}
\hyphenation{
affri-ca-te
affri-ca-tes
an-no-tated
com-ple-ments
com-po-si-tio-na-li-ty
non-com-po-si-tio-na-li-ty
Gon-zá-lez
out-side
Ri-chárd
se-man-tics
STREU-SLE
Tie-de-mann
} 
 \togglepaper[1]%%chapternumber
}{}


\begin{document}
\maketitle


\section{Some basics of Hoocąk morphosyntax}\label{sec:Helmbrecht:1}

Hoocąk is an indigenous language of North America that belongs to the Mississippi Valley group of the Siouan language family. Hoocąk (also called Ho-Chunk) is a highly endangered language still spoken by approximately a hundred elderly speakers in Wisconsin. 

Hoocąk is grammatically quite distinct from better known European languages. The verb is morphologically highly complex with a remarkable wealth of morphological positions before and after the verbal root (see further below).

From a syntactic point of view, the most remarkable property of Hoocąk is the way how core arguments of the clause are coded grammatically. Arguments of the verbal predicate are filled morphologically by means of pronominal affixes. A pronominally inflected verb in principle represents a complete clause, and lexical NPs are not grammatically necessary, either with a nominal or with a pronominal head; see \REF{ex:Helmbrecht:1}.


\ea \label{ex:Helmbrecht:1} 
  \textit{wasgerá hakaráixuxšąną }\\  
 	\gll  {wasge=ra}   {Ø-ha-kara-gíxux=šąną}\\	
 	{dish=\textsc{def}} {\textsc{obj.3sg}-1\textsc{e.a}-\textsc{poss.refl}-break=\textsc{decl}}\\
  \glt `I broke my dish.' (\citealt[14]{Eagle1988})
\z 

The verbal predicate at the end of the clause in \REF{ex:Helmbrecht:1}. contains two pronominal prefixes, the third person singular object \textit{Ø-} followed by the first person singular actor (A) prefix \textit{ha-}. The object prefix refers to the referent of the NP \textit{wasge=ra} `the dish', the actor prefix to the speaker. The declarative enclitic \textit{=šąną} is not obligatory and marks the entire clause as a statement. The possessive reflexive marker \textit{kara-} (\textsc{poss.refl}) indicates that the referent of the A argument owns the referent of the U argument.\footnote{The terms “A argument” and “U argument” are taken from Role and Reference Grammar (cf. \citealt{Vanvalin1997}, \citealt{Vanvalin2013}), where they are defined as macro-roles, i.e. as generalized semantic roles that subsume various different and more specific agent-like and patient-like semantic roles. I use these terms here to refer to the two different paradigms of person affixes for intransitive verbs in Hoocąk and the arguments of a transitive verb that are filled by person affixes of these paradigms. The A paradigm is required for intransitive verbs that designate different kinds of actions, the U paradigm is required for verbs that designate states, properties and uncontrolled processes. In addition, I use these terms here to refer to the first argument of a transitive verb, the A argument, and the second argument of a transitive verb, the U argument, because these arguments are filled with person affixes of the respective A and U paradigms. Note that – because of valence increasing morphological processes – there may be more than on U argument in a verb, which is a particularity of Hoocąk. In these cases, I distinguish the two U arguments of a verb terminologically as e.g. “patient U argument”, or “recipient U argument”, or “benefactive U argument”.} The NP `the dish' may be dropped without affecting the grammaticality of the clause. Note also that the syntactic function of the two arguments in (\ref{ex:Helmbrecht:1}). are exclusively marked by the pronominal affixes on the verb. There is no case marking of the noun and word order would not help in this case either. 

There are up to seven prefix slots of the verb that may be filled with different kinds of grammatical prefixes; see \tabref{tab:Helmbrecht:1} for an abstract overview.


\begin{table}[H]
	\begin{tabular}{p{3cm} p{9cm}}
		\lsptoprule
		Morphological slots   & \\
		\hline 
	 -7	& pronominal prefixes I \\
	-6	& outer applicatives (instrument and locative) \\
	-5	& outer instrumentals \\
	-4	& pronominal prefixes II (Undergoer and Actor)  \\
	-3	&benefactive applicative, reflexive marker (kii-/ki- \textsc{refl}), reciprocal marker (kii-/kiki- \textsc{recp}), possessive reflexive marker (kara-/ kV/ k- poss.refl) \\
	-2	& pronominal prefixes III   \\
	-1	& inner instrumentals \\
	0	& verbal root \\
	1-n	& suffixes  \\
	\lspbottomrule 
	\end{tabular}
	\caption{Template presentation of prefixes of the Hoocąk verb (cf. \citealt{Helmbrecht2008})}\label{tab:Helmbrecht:1}
\end{table}


There are pronominal prefixes that index the core arguments S\textsubscript{A}, S\textsubscript{U}, A, and U of the clause (labelled pronominal prefixes I-III with slots -7/-4/-2 in \tabref{tab:Helmbrecht:1}). There are four different applicative prefixes that augment the valency of the verb stem (the outer applicatives and the benefactive applicative in \tabref{tab:Helmbrecht:1}). There are eight so-called instrumental prefixes that enrich and specialize the semantics of the verb (similarly labelled outer and inner instrumentals in \tabref{tab:Helmbrecht:1}). And there are a reflexive and a reciprocal marker, which mark the identity of the actor (A) and undergoer (U) of a transitive verb (both in the morphological slot -3 in \tabref{tab:Helmbrecht:1}). 

The reflexive marker \textit{kii-} signals that the referent of the A argument is identical with the referent of the U argument. In this case the U argument is never marked separately by a pronominal prefix. The same reflexive marker may also have a reciprocal meaning, if the A argument is plural. This holds for first and second persons as well as for third persons. In addition, it has to be stressed that A and U third person arguments always have a disjoint reference, if there is no reflexive marker. Compare the examples in \REF{ex:Helmbrecht:2}.

\ea \label{ex:Helmbrecht:2}
 \ea \textit{ hajáną }  \\
	\gll {ha<Ø-Ø>já=ną}\\
 {\textsc{<obj.3sg-sbj.3sg>}see=\textsc{decl}}\\
 \glt `He\textsubscript{1} sees him\textsubscript{2}' [DL XI: 15]
 \ex \textit{ hakijáną }\\
 	\gll	{ha<Ø>\textbf{ki-}já=ną}\\ 
 	{<\textsc{sbj.3sg}>\textbf{\textsc{\textsc{refl}}}{}-see=\textsc{decl}}\\
	\glt `He\textsubscript{1} sees himself\textsubscript{1}'
 \z 
\z 

There is no way to interpret the two arguments in (\ref{ex:Helmbrecht:2}a) as coreferential. If coreference between the two arguments is intended, the reflexivizer \textit{kii-} \textsc{refl} has to be used; see (\ref{ex:Helmbrecht:2}b). The reflexive marker \textit{kii-} may also be interpreted with a reciprocal meaning in case that the A argument is a non-singular referent. In this function, \textit{kii-} competes with the reduplicated form \textit{kiki-} that always marks reciprocal meaning (see examples \ref{ex:Helmbrecht:6} and \ref{ex:Helmbrecht:7} below).

In addition, there is a possessive reflexive marker indicating a possessive relation between the A and the U argument (illustrated already in example \ref{ex:Helmbrecht:1} above). This form will be discussed in \sectref{sec:Helmbrecht:5.1} below.

Some further comments on the pronominal prefixes are necessary. Although there are three morphological slots of pronominal prefixes, there are in fact only two different paradigms of pronominal affixes, one indicating the person category of the S\textsubscript{A} argument, i.e. the intransitive subject of a verb with active semantics, and the second one indicating the person category of the S\textsubscript{U} argument, i.e. the intransitive subject of a verb with inactive semantics. This marking pattern is lexically fixed for each intransitive verb. Compare the paradigm of personal affixes for intransitive inactive verbs such as \textit{š'aak} `be old' in \tabref{tab:Helmbrecht:2} and for intransitive active verbs such as \textit{šgáač} `play' in 

Intransitive active verbs designate controlled movements like `come', `go', `arrive' `swim' etc. and actions such as `dance', `get dressed', `travel', etc. \tabref{tab:Helmbrecht:3}. Inactive intransitive verbs designate properties like `be red', `be big', `be strong' etc. and uncontrolled processes such as `float', `boil', `slip', etc. 

\begin{table}
	\begin{tabular}{lll}
\lsptoprule
1\textsc{sg} & \textit{hį-š'ák}  & `I am old'\\
2\textsc{sg} & \textit{nį-š'ák}  & `you are old'\\
3\textsc{sg} & \textit{Ø-š'áak} & `he is old'\\
1\textsc{i.du} & \textit{waągá-š'ák} & `you and I are old'\\
1\textsc{i.pl} & \textit{waągá-š'ák-wi} & `we (incl.) are old'\\
1\textsc{e.pl} & \textit{hį-š'ák-wi}  & `we (excl.) are old'\\
2\textsc{pl} & \textit{nį-š'ák-wi} & `you (all) are old)'\\
3\textsc{pl} & \textit{š'áak-ire} & `they are old'\\
\lspbottomrule
\end{tabular}
\caption{Paradigm of the intransitive inactive verb \textit{š'áak} `to be old'}\label{tab:Helmbrecht:2}
\end{table}

Intransitive active verbs designate controlled movements like `come', `go', `arrive' `swim' etc. and actions such as `dance', `get dressed', `travel', etc. 


\begin{table}
\begin{tabular}{lll}
\lsptoprule
1\textsc{sg} & \textit{ha-šgáč} & `I play'\\
2\textsc{sg} & \textit{ra-šgáč} & `you play'\\
3\textsc{sg} & \textit{Ø-šgáač} & `he plays'\\
1\textsc{i.du} & \textit{hį-šgáč} & `you and I play'\\
1\textsc{i.pl} & \textit{hį-šgáč-wi} & `we (incl.) play'\\
1\textsc{e.pl} & \textit{ha-šgáč-wi} & `we (excl.) play'\\
2\textsc{pl} & \textit{ra-šgáč-wi} & `you (all) play'\\
3\textsc{pl} & \textit{šgáač-ire} & `they play'\\
\lspbottomrule
\end{tabular}
\caption{Paradigm of an intransitive active verb}\label{tab:Helmbrecht:3}
\end{table}


The A (transitive subject) and the U (transitive object) arguments of transitive verbs are filled by a combination of pronominal affixes from both paradigms. Hoocąk is thus a head-marking language on the clause level and belongs to the so-called split-S marking type. It has to be stressed that this marking pattern – also called active/ inactive alignment type – holds only for first and second persons (speech act participants); see \figref{fig:helmbrecht:1} (cf. \citealt{Hartmann2013}: 1268).

\begin{figure}
 \centering
 \includegraphics[width = 0.5\textwidth]{figures/helmbrecht1.png}
 \caption{Active/ inactive alignment (for first and second persons)}\label{fig:helmbrecht:1}
\end{figure}

\begin{figure}
 \centering
 \includegraphics[width = 0.5\textwidth]{figures/helmbrecht2.png}
 \caption{Accusative alignment (for third persons)}\label{fig:helmbrecht:2}
\end{figure}


Third persons show accusative alignment, i.e. S\textsubscript{A}, S\textsubscript{U} and A are coded in all cases identically, either by the 3\textsc{sg} zero form \textit{Ø}{}-, or the 3\textsc{pl} form –\textit{ire} (see \ref{fig:Helmbrecht:2}). The transitive object U is marked either by a zero form for 3\textsc{sg}, or by a special pronominal affix \textit{wa-} (\textsc{obj}.3\textsc{pl}) for third person plural objects. Note that this special obj.3\textsc{pl} form \textit{wa-} is used only, if the U argument is definite.

The right side of the verb root is likewise morphologically complex, but in a very different way. There are a few suffixes and a large number of enclitics that appear in a fixed order after the verb root. These bound forms designate tense, aspect, and mood categories and are generally less grammaticalized than the prefixes. One manifestation of this is that the prefixes are highly synthetic and undergo plenty of morphophonemic processes, while the suffixes and enclitics are rather agglutinating and stable with regard to their phonological form.

While verbs are easy to identify based on their morphology, nouns are problematic in this respect. There is no noun-specific morphology such as case marking, number marking or gender. Nouns can be identified by their semantics and by their structural and distributional properties, especially as heads of nominal expressions.

The order of the major clausal constituents is quite regular, exhibiting SOV order in traditional terms. The predicate is strictly clause final. Other constituents such as argument NPs, adverbials and subordinate clauses precede the predicate, but may show different orders depending on pragmatic factors (see \citealt{Helmbrecht2021}). 

\section{Basic uses of the reflexivizer}\label{sec:Helmbrecht:2}

There are no reflexive pronouns in Hoocąk (like English \textit{himself}), and reflexive scenarios are never expressed by personal pronouns such that there are coreferential A and U personal pronouns both inflected by the corresponding cases. Hoocąk has only two personal pronouns (\textit{nee} `first and second person', and \textit{{}'ee} `third person'), which are not case marked, and which are used exclusively for emphatic reasons in specific focus constructions. Instead, Hoocąk has one verbal reflexive marker \textit{kii-,} which indicates coreference of the U with the A argument. A second and closely related meaning of the reflexivizer \textit{kii-} is the reciprocal meaning. In addition, this reflexivizer may be reduplicated in order to express explicitly a reciprocal meaning \textit{kiki-}\textsc{recp}. The relationship between these two forms - \textit{kii}\textsc{-refl} and \textit{kiki}\textsc{-recp-} and the two meanings - reflexive and reciprocal - will be illustrated in the subsequent sections.  

Reflexivization is usually seen as a detransitivizing operation, not necessarily in terms of a structural syntactic transitivity, but from the point of view of transitivity as a prototype notion (cf. \citealt{HopperThompson1980}). The action is less transitive, because the undergoer, who is the endpoint of the action, is the same as the actor, i.e. there is only one true independent participant. Reflexive verbs are therefore often grammatically treated as detransitivized verbs in one way or another in the languages of the world. In Hoocąk, reflexivization is clearly a detransitivizing process. Formally, this is manifest by the blocking of the U pronominal affixes. Only the A argument is marked by a person prefix. A typical example of a reflexive construction is given in the following utterance from the DOBES corpus of Hoocąk texts.\footnote{Data for this study come from a large digital corpus of Hoocąk text, which were collected within the DOBES funding initiative of the Volkswagen Foundation (\url{http://dobes.mpi.nl//}). The glossed texts and the audio and video files of the Hoocąk documentation project are stored in the digital archive of the Max Planck Institute for Psycholinguistics called "The Language Archive" (\url{http://dobes.mpi.nl/projects/hocank/}). The DOBES project "Documentation of the Hoocąk Language" was led by Johannes Helmbrecht and Christian Lehmann at the University of Erfurt, Germany. The data taken from the Hoocąk corpus are supplemented by data elicited by the author during various fieldtrips to Wisconsin between 1997 and 2007. Abbreviations such as ECO027 specify the text from which the example is taken (here the "Ecology speech") as well as the number of the utterance (here number "027)".} 

\ea \textit{ Hąąke hųųkišgacnikjawi.}
 \label{ex:Ref30155749}\\
	\gll {hąąke} {hį-ho<kii>šgac=nį=kje-wi}\\
	{\textsc{neg.in}}  {\textsc{1\textsc{incl}.A-<refl>}abuse=\textsc{neg.fin=FUT-\textsc{pl}}}\\
	\glt `Let's not abuse ourselves.' [ECO027]
\z 

The transitive verb \textit{hošgac} `to abuse someone' in example (\ref{ex:Helmbrecht:6}c). has an A and a U argument in its argument structure. The reflexivizer \textit{kii-} indicates that the referent of the U argument is identical to the referent of the A argument. The latter is marked by a pronominal affix \textit{hį-} (first person inclusive actor; 1incl.A) and the plural marker for first and second persons \textit{{}-wi} (\textsc{pl}) at the end. The future marker \textit{=kje} has to be interpreted as a hortative in this context.

In Hoocąk, the reflexive prefix \textit{kii-} can, in principle, be used with every transitive verb, if its semantics allows such a derivation, i.e. it must be possible that the action of the verb can be exerted on oneself. In most cases, the agent A argument is coreferential with the patient U argument of the transitive verb; but other coreference relations are possible (see \sectref{sec:Helmbrecht:5} below). 

 The patient U argument is the first target of the reflexivizer, and this does not change, even if there are other U arguments around. These other U arguments could be introduced into the argument structure of the verb by means of an applicative marker. There are four different applicative markers in Hoocąk, a superessive, an inessive, an instrumental and a benefactive applicative marker, which have in common that they open a new undergoer argument slot in the verb. These semantically different undergoer arguments can be wholly or partly coreferential with the A argument (see \sectref{sec:Helmbrecht:5} below). This is, however, not possible with an additional instrumental U argument, certainly for semantic reasons. In this case, the coreference remains between agent (A) and patient (U); see example \REF{ex:Helmbrecht:4} below. It is hard to imagine a situation in which the instrument is coreferential/ identical with the actor of this action.

\ea   \textit{ Mąąhįpahi himą́kicgisšąną } \label{ex:Helmbrecht:4}.\\
 \gll {mąąhį-paahi}  {hi-mąą-Ø-ha-ki-cgis=šąną}\\
  {knife-be.sharp}  {\textsc{appl.inst}-cut-\textsc{3sg.u}-\textsc{1e.a}-\textsc{\textsc{refl}}-cut=\textsc{decl}}\\
  \glt `I cut myself with a sharp knife.'
 (\citealt{Hartmann2013}: ex. 216)\footnote{The Hoocąk data collected by Iren \citet{Hartmann2013} can all be found on the website of the Valency Pattern Leipzig project at the Max Planck Institute for Evolutionary Anthropology ({http://www.valpal.info/languages/hoocak}).}
\z   
  

Similar semantic restrictions apply for verbs that designate an action that cannot be exerted on the actor her/himself. For instance, the reflexivizer \textit{kii-} cannot be used with the transitive verb \textit{ru'ą} `to carry something' `to lift something', with a reflexive meaning, because it is pragmatically not possible to lift oneself, or to carry oneself, at least when the literal sense is meant. Despite this pragmatic constraint, \textit{ru'ą} `to carry someone' can be marked with \textit{kii-} yielding a reciprocal meaning `each other'. This reciprocal use is only possible with a plural A argument. For instance, \textit{kii-ru'ą-ire} would mean `they carry each other'. 

The same holds for the transitive verb \textit{hoki'\'{ų́}} `to imitate something/someone'. This verb cannot receive a reflexive meaning, because it is literally not possible to imitate oneself. However, this verb may receive a reciprocal meaning by adding \textit{kii-} such that it becomes \textit{ho-ki-kí'ų́-ire} `they imitate each other'. Note that in this case, the single \textit{kii-} has likewise a reciprocal meaning. The middle syllable /\textit{ki}{}-/ in \textit{hoki'\'{ų́}} is part of the stem. (It may well be that it is the historically lexicalized reflexivizer \textit{kii-}.)

As is often the case in the Hoocąk lexicon, otherwise productive derivational means are frequently found in words where they are fossilized as part of the stem. This holds for the reflexive marker \textit{kii-} too. In these cases, the addition of \textit{kii-} results in verbs with a morpheme sequence \textit{kiki-}, which can be interpreted as reflexive or reciprocal. For instance, in the transitive verb \textit{hiki'ó} `to touch something', the \textit{kii-} form in the middle is part of the stem and cannot be interpreted as a reflexive marker. If the \textit{kii-} is added as in \textit{hikikí'o,} a polysemous verb emerges. The first meaning is – as one expects – `to touch oneself', the second meaning is the reciprocal meaning `to touch each other', and the third meaning is `to touch something repeatedly' (PM XI:17). Reduplication in Hoocąk can be utilized to indicate iterativity, thus the addition of \textit{kii-} may simply be interpreted as a mere reduplication of the middle syllable of the stem. 

Another example of this sort is \textit{hokit'é} `to talk to someone'. The \textit{kii-} part in the middle is lexicalized and has no reflexive meaning. \textit{Yaakit'é} would be `I speak to someone' and not *`I speak to myself'. However, to make it reflexive, one can insert \textit{kii-} \textsc{refl} and gets \textit{yaa-ki-kit'e} `I speak to myself' (PM XI.10). Another verb that cannot receive a reflexive meaning with \textit{kii-} \textsc{refl} is the verb provided in (\ref{ex:Helmbrecht:5}a--b). 

\ea  \label{ex:Helmbrecht:5}  
	\ea \textit{  hat'\'{ą}p  }  {}`to jump on something'
	\ex \textit{ha-ki-t'\'{ą}p  } {}`to jump on each other'
 \z 
\z

The reflexive marker \textit{kii-} in (\ref{ex:Helmbrecht:5}b). cannot be interpreted as `to jump on oneself' for pragmatic reasons. Therefore it is interpreted as reciprocal, which again demonstrates the close semantic relationship between both meanings. The semantic extension from reflexive to reciprocal is conceptually easy, the polysemous encoding of reflexive and reciprocal meanings is therefore widespread among the languages of the world (see \citealt{Maslova2013}).

The reciprocal usage of the \textit{kii-} reflexive marker sometimes competes with the reciprocal marker \textit{kiki-}\textsc{recp}, which is a reduplication of the reflexive marker \textit{kii-}. The reciprocal marker \textit{kiki-} \textsc{recp} is used only, if the meaning of the reflexivizer \textit{kii-} is ambiguous, and only the reciprocal meaning is intended, or if the speaker wants to particularly stress the reciprocal meaning; cf. the examples in (\ref{ex:Helmbrecht:6}) and (\ref{ex:Helmbrecht:7}).

\ea   \label{ex:Helmbrecht:6}
 \ea \textit{ hajá } {}`to see sth.'
 \ex \textit{  hakijá } {}`to see oneself'
 \ex  \textit{ 	 haakícaaną } \\
	\gll {ha<ha-\textbf{kí}>ca-ną}\\
 {see<\textsc{1e.a}-\textbf{\textsc{\textsc{refl}}}>see{\textbackslash}\textsc{1e.a=decl}}\\
 \glt {}`I see myself (e.g. in the mirror)' 
 \ex  \textit{  'eejá haakíkicawiiną } \\  
	\gll   {'eejá}   {ha-<ha-\textbf{kíki}>ca-wi=ną}\\
  {there}  {<\textsc{1e.a}-\textbf{\textsc{recp}}>see{\textbackslash}\textsc{IE.A-\textsc{pl}=decl}}\\
	\glt `We see each other there (in the mirror)' [DL XI:15]
 \z
\z


\ea \textit{hegų nąącge hąąke pįį hąąke hiįkijawinį hinųbahąňą} \label{ex:Helmbrecht:7}\\ 	
 \gll   {hegų} {nąącge}  {hąąke}  {pįį}   {hąąke}   {<hį>ha<\textbf{kii}>ja-wi=nį} {hi-nųųp-ahą=ra}\\
  {that.way} {heart} {\textsc{neg.in}} {be.good} {\textsc{neg.in}} {<\textsc{1I.A}><\textbf{\textsc{\textsc{refl}}}>see-\textsc{\textsc{pl}}=\textsc{neg.fin}} {\textsc{ORD}-two-times=\textsc{def}}\\
 \glt `We never see each other with good hearts anymore.' [DAP107]
\z 

The transitive verb \textit{hajá} `to see something' may infix the reflexivizer \textit{kii-} yielding a reflexive meaning `to see oneself'; see the inflected example in (\ref{ex:Helmbrecht:6}c). for the first person. The reflexivizer may also receive a reciprocal meaning with this verb, if A is pluralized; see example \REF{ex:Helmbrecht:7}. If there is some doubt, and if the reciprocal meaning is intended, the reflexivizer may be reduplicated to underline that only the reciprocal interpretation is intended; see (\ref{ex:Helmbrecht:7}d).

Transitive verbs are inflected for person by a combination of forms from the A paradigm and the U paradigm. The general morphological pattern is that the U form precedes the A form, but some exceptions apply. First, the first inclusive dual and plural A form \textit{hį-} (\textsc{1incl}.A) always precedes all other prefixes of the verb. Secondly, there is a portmanteau prefix \textit{nįį-} for the first person acting on a second person (1\& 2) that does not allow a further segmentation. The general and schematic paradigm of pronominal affixes for a transitive verb form of the first and most regular conjugation is given in \tabref{tab:Helmbrecht:4} (cf. \citealt{Helmbrecht2021})

\begin{table}
\begin{tabular}{l|llllll}
\lsptoprule
 &   &   &  U  &  &  & \\
 \hline
 & -  & 1\textsc{sg}   & 2\textsc{sg}   & 3\textsc{sg}   & &   \\
 & 1\textsc{sg}   & -  & nįį-V & Ø-ha-V   & &   \\
 & 2\textsc{sg}   & hį-ra-V  & -  & Ø-ra-V   & &   \\
 & 3\textsc{sg}   & hį-Ø-V   & nį-Ø-V   & Ø-Ø-V & &   \\
A & 1\textsc{incl}.\textsc{du} & -  & -  & hį-Ø-V   & &   \\
 & 1\textsc{incl}.\textsc{pl} & -  & -  & hį-Ø-V-wi   & &   \\
 & 1\textsc{excl}.\textsc{pl} & -  & nįį-V-wi & Ø-ha-V-wi   & &   \\
 & 2\textsc{pl}   & hį-ra-V-wi  & -  & Ø-ra-V-wi   & &   \\
 & 3\textsc{pl}   & hį-V-ire & nį-V-ire & Ø-V-ire  & &   \\
  \lspbottomrule
\end{tabular} 

\fittable{
 \begin{tabular}{l|llllll}
 \lsptoprule
 &   &   &  U & & & \\
 \hline
  & & 1\textsc{incl}.\textsc{du} & 1\textsc{incl}.\textsc{pl} & 1\textsc{excl}.\textsc{pl} & 2\textsc{pl}   & 3\textsc{pl}  \\
 & 1\textsc{sg}   & -  & -  & -  & nįį-V-wi & wa-ha-V \\
 & 2\textsc{sg}   & -  & -  & hį-ra-V-wi  & -  & wa-ra-V \\
 & 3\textsc{sg}   & wąągá-Ø-V   & wąągá-Ø-V-wi   & hį-Ø-V-wi   & nį-Ø-V-wi   & wa-Ø-V  \\
 & 1\textsc{incl}.\textsc{du} & -  & -  & -  & -  & hį-wa-V \\
 A & 1\textsc{incl}.\textsc{pl} & -  & -  & -  & -  & hį-wa-V-wi \\
 & 1\textsc{excl}.\textsc{pl} & -  & -  & -  & nįį-V-wi & wa-ha-V-wi \\
 & 2\textsc{pl}   & -  & -  & hį-ra-V-wi  & -  & wa-ra-V-wi \\
 & 3\textsc{pl}   & wąągá-V-ire & wąągá-V-ire-wi & hį-V-ire-wi & nį-V-ire-wi & wa-V-ire  \\
 \lspbottomrule
\end{tabular}
}
\caption{Transitive paradigm of person markers (first conjugation)}\label{tab:Helmbrecht:4}
\end{table}


\tabref{tab:Helmbrecht:4} covers all combinations of person/ number values of the A and U arguments that are in principle possible. Most of the pronominal affixes precede the verb root (V), but the plural marker \textit{{}-wi} (\textsc{pl}) for first and second persons, and the subject third person plural marker \textit{{}-ire} (sbj.3\textsc{pl}) follow the verb root. The white cells with a hyphen in \tabref{tab:Helmbrecht:4} indicate that this combination of person/ number values cannot be expressed by the corresponding person affixes in Hoocąk. These white cells have in common that the referent of the A argument is completely coreferential, or partially coreferent, with the referent of the U argument. Some of these "white" reflexive scenarios can be expressed with the pronominal affix of the A argument and the reflexive marker \textit{kii-}. Others cannot be expressed at all with pronominal affixes in Hoocąk. I will illustrate some of these restrictions briefly.

 The transitive verb \textit{mąącgís} `to cut something (with a cutting instrument like a scissor)' consists of the bound verb root \textit{{}-cgis} `cut something' and the instrumental prefix \textit{mąą-} that adds a manner/ instrument meaning to the lexical meaning of the verb root (such as `with a knife/ with a pair of scissors’, or the like). Note that this instrumental prefix does not provide a new argument slot to the verb root. Reflexive events such as `I cut myself' or `you cut yourself' etc. (see \ref{ex:Helmbrecht:8}a and \ref{ex:Helmbrecht:9}a.) cannot be expressed by a combination of the respective A and U pronominal affixes. Instead, the A prefix \textit{ha-} (1e.a) has to be used plus the reflexive marker \textit{kii-}, which indicates the coreference of A and U; see (\ref{ex:Helmbrecht:8}b). and (\ref{ex:Helmbrecht:9}b). The coreferential U argument is not marked at all.


\ea \label{ex:Helmbrecht:8} 
   \ea [*]{
   \gll {mąa-hį-ha-cgis}\\
 {by.cutting-\textsc{1e.u-1e.a}-cut}\\
	\glt `I cut myself (with a cutting instrument like a scissor)'}
   \ex []{ \textit{mą́ ąkicgis}\\
	\gll  {mąą-ha-ki-cgis}\\
	  {by.cutting-\textsc{1e.a-\textsc{refl}}-cut}\\
	\glt `I cut myself (with a cutting instrument like a scissor)'[PM XI:19]\footnote{Underlying /h/ in \textit{ha-} (1e.a) and \textit{hį-} (1e.u) always drops word internally.}}
	\z 
\z   
 

\ea \label{ex:Helmbrecht:9}  
	\ea [*]{
	\gll {mąą-nį-ra-cgis}\\
	 {by.cutting-\textsc{2U-2A}-cut}\\
	\glt `You cut yourself (with a cutting instrument like a scissor)'}
	\ex []{ \textit{mąąną́ kicgis}\\
	\gll {mąą-ra-ki-cgis}\\
	{by.cutting-\textsc{2A-\textsc{refl}}-cut}\\
	\glt `You cut yourself (with a cutting instrument like a scissor)'}
	\z
\z 

The examples in (\ref{ex:Helmbrecht:8}) and (\ref{ex:Helmbrecht:9}) represent reflexive events, in which the referent of A (first person singular/ second person singular) is fully identical to the referent of U. If A is a third person singular (zero marked \textit{Ø-} 3\textsc{sg}), the reflexive marker indicates that A and U are coreferential. If there is no reflexive marker, we have a normal transitive construction with two zero-marked third-person arguments with different referents (see the examples \ref{ex:Helmbrecht:2}a--b above). 

Things are more complicated, if plural referents are involved. Hoocąk has three different first person plural markers, first person dual inclusive (you and me; (1\textsc{incl.du})), first person plural inclusive (we all including you; (1\textsc{incl.pl})), and first person plural exclusive (we all, but not you; (1\textsc{excl.pl})). These A forms can be combined with the reflexive marker \textit{kii-} with the result that the respective first person non-singular group is an A and U argument at the same time. The inclusive/ exclusive distinction is maintained. However, there is a systematic polysemy in the way that either each member of the group acts on him-/herself, or that the members of this group act on each other; a reciprocal meaning. 

What is not possible to express pronominally in Hoocąk is that a first person non-singular group acts on the first person singular, with or without the reflexivizer. English does not allow this scenario either (*\textit{we} \textit{see} \textit{myself} / \textit{me} \textit{in} \textit{the} \textit{mirror}); see \citet{HampeLehmann2013}. The inverse situation with a first person singular acting on a first person non-singular is, however, possible in English: \textit{I} \textit{see} \textit{us} \textit{in} \textit{the} \textit{mirror}. No matter whether \textit{us} is interpreted as an inclusive plural or an exclusive plural, it is a kind of partial reflexive situation, which is not marked as reflexive. Hoocąk cannot express this situation with its pronominal affixes and the reflexivizer. Therefore, it is marked white plus hyphen in \tabref{tab:Helmbrecht:4}. We have a similar situation with the second person singular as A argument. Hoocąk does not allow a 2\textsc{sg.a} acting on a first person inclusive non-singular (1\textsc{incl}.\textsc{du}/\textsc{pl.a}). The English equivalent sounds odd, too: \textsuperscript{?}{}`\textit{you} (\textsc{sg}) \textit{see} \textit{us} \textit{in} \textit{the} \textit{mirror} (including yourself)'. If the first plural pronoun \textit{us} is interpreted as exclusive, it is no longer odd. Then it is no longer a reflexive construction in English and in Hoocąk.

The reflexive \textit{kii-} derivation is generally not possible with intransitive active and inactive verbs. There are no such reflexive formations as `something breaks by itself' (*\textit{kiišížre)} or `something is cooked for oneself' (*\textit{kiitúc)}. 

\section{Emphatic meaning of the reflexivizer}\label{sec:Helmbrecht:3}

The reflexivizer may be used to express emphasis, which is comparable to some uses of English reflexive pronouns. Compare the examples in (\ref{ex:Helmbrecht:10}) and (\ref{ex:Helmbrecht:11}) from different texts.

\ea  \label{ex:Helmbrecht:10}
  \glll hegų waicekjį wa'ųąježe waįsisikįk wa'ųąježe yaakiregają \\
 {hegų} {wa<hį>cek=xjį} {wa<ha>'ų-ha-jee=že}  {wa<hį>sisik=nįk} {wa<ha>'ų-ha-jee=že} {hi<ha-\textbf{kii}>re=gają}\\
 {that.way} {<\textsc{1e.u}>be.young=\textsc{ints}} {<\textsc{1e.a}>do/be-\textsc{1e.a}-\textsc{pos.vert}=\textsc{quot}}   {<\textsc{1e.u}>be.agile=\textsc{dim}} {<\textsc{1e.a}>do/be-\textsc{1e.a}-\textsc{pos.vert}=\textsc{quot}}   {<\textsc{1e.a}-\textbf{\textsc{\textsc{refl}}}>think=\textsc{seq}}\\
	\glt `Well, I thought, I was young and fast on foot.' [MOV026]
\z 
 
The speaker in MOV026 expresses his surprise that the old man in this story ran much faster than he. The transitive verb \textit{hiré} `think something' has two arguments, the thinker as A and the content of the thinking as U argument. The reflexivizer in this example cannot indicate coreference of A and U, but rather emphasizes that the narrated reality contradicts all expectations. A more idiomatic translation in English could perhaps be `I \textbf{really} thought \textbf{for} \textbf{myself} ....' using the English reflexive pronoun as a self-intensifier within a prepositional phrase as a kind of adverbial to the main verb `thought'. A similar usage of the reflexivizer in Hoocąk is shown in (\ref{ex:Helmbrecht:11}).

\ea  \label{ex:Helmbrecht:11} 
 \glll  yaa nįįšge 'eejaxjį saacąxjį hoto\v{g}ocra hegų (hą)ke wažą nąąkixgųnį  \\
   {yaa}  {nįį=šge}   {'eejaxjį}  {saacą=xjį}   {hoto\v{g}oc=ra}  {hegų} {hąke} {wažą}   {nąą<\textbf{kii}>xgų=nį}\\
  {\textsc{affirm}} {me=also}  {about.there} {five=\textsc{ints}} {look.at{\textbackslash}\textsc{1e.a=def}} {that.way}  {\textsc{neg.in}}  {something} {<\textbf{\textsc{\textsc{refl}}}>understand=\textsc{neg.fin}}\\
 \glt `Yeah, me too, even when I looked at the story about five times, I couldn't understand a thing.' [RRT073]
\z 

The transitive verb \textit{nąąxgų́} `to hear something', `to understand something' has the first person speaker as A argument and the content of what has not been understood (`thing') as U argument. The reflexivizer does not indicate coreference of A and U, but emphasizes the fact that despite all the efforts the speaker did not succeed in understanding.

\section{Reflexive scenarios with body parts as target}\label{sec:Helmbrecht:4}

As far as I can judge from the data I have at hand, there is no systematic constructional difference between reflexive scenarios expressed by introverted or extroverted verbs. Introverted verbs demand the same reflexive construction used for extroverted verb. 

However, one can find some constructional variation in reflexive scenarios that seem to be linked to different degrees of involvement of the patient argument as it is the case with parts of the body of the A argument. This constructional variation can be observed also with some introverted verbs. Reflexive scenarios with introverted verbs, i.e. verbs that designate body care (grooming) actions such as `to wash oneself' or `to shave oneself' (see \citealt{haspelmath2019comparing}) occur sometimes with additional morphological material in Hoocąk. In addition to the reflexivizer, the verb `to wash oneself' may occur with the possessive reflexive marker. The possessive reflexive marker is a verbal marker that indicates that the A argument of a transitive verb possesses the U argument; compare the examples in (\ref{ex:Helmbrecht:12}a--c).

\ea  \label{ex:Helmbrecht:12}
	\ea  \textit{ ruž\'{ą}}  {}`to wash something'
	\ex \textit{  ku-ruž\'{ą}  }  {}`to wash one's own'
	\ex  
	
	\glll wažątírera waakúružąąną\\
	  {wažątíre=ra} {wa-ha-\textbf{kú-}ružą=ną}\\
   {car=\textsc{def}}   {\textsc{obj.3\textsc{pl}-1e.a}-\textbf{\textsc{poss.refl}}-wash=\textsc{decl}}\\
	\glt `I wash my cars.' [ PM (XI:8)]
	 \z 
	\z  

The verb \textit{ruž\'{ą}} `to wash something' requires the \textsc{washer} as A argument and what is \textsc{washed} as U argument. The reflexive possessive marker, which has three allomorphs (\textit{kara-/kV-/k-} \textsc{poss.refl}) indicates that the referent of A possesses the referent of U; cf. (\ref{ex:Helmbrecht:12}b and c). The clause (\ref{ex:Helmbrecht:12}c) without this marker would simply mean `I wash the cars'. The possessive reflexive does not increase the valency of the verb, but indicates an additional relation between A and U and is a good indicator for transitivity. Only transitive verbs may take it.

If \textit{ruž\'{ą}} `to wash something' is used to express the reflexive scenario `to wash oneself', the possessive reflexive marker may appear in addition to the reflexivizer. See the elicited examples in (\ref{ex:Helmbrecht:13}) and (\ref{ex:Helmbrecht:14}).

\ea  \label{ex:Helmbrecht:13} 
 \glll {hakikúružąąną }\\
       {ha-\textbf{ki-}\textbf{kú-}ružą=ną}\\
	  {\textsc{1e.a}-\textbf{\textsc{refl}}-\textbf{\textsc{poss.refl}}-wash=\textsc{decl}}\\
 \glt `I wash myself.' [PM XI:8]
\z 

\ea \label{ex:Helmbrecht:14} 
    \glll {hakikikuružąwi}\\
  {ha-\textbf{kiki}-\textbf{ku}-ružą-wi}\\
  {\textsc{1e.a}-\textbf{\textsc{recp}}-\textbf{\textsc{poss.refl}}-wash-\textsc{pl}}\\
  \glt `We wash each other.' [PM XI:8]
\z 

Both constructions, the reflexive construction and the reciprocal construction, may take the possessive reflexive verbal marker. This constructional alternative, i.e. the combination of reflexivizer plus a possessive reflexive, can be found also with other semantic types of verbs. Compare the following two clauses from Hartmann's database. 

\ea \label{ex:Helmbrecht:15} 
\glll  Wa'į šjuuc yaákikuruką.\\
  {wa'į} {šjuuc}   {<hi>   ha-<ha-\textbf{ki-}\textbf{ku}>  ruką}\\
  {blanket} {be.warm} {<\textsc{appl.inst><1e.a-\textbf{\textsc{refl}-poss.refl}}>cover}\\
  \glt `I covered myself with a warm blanket.' (\citealt{Hartmann2013}:example No.8)
\z 

\ea  \label{ex:Helmbrecht:16} 
   \glll Wa'į šjuuc {yaa'ųanąga haákituką} \\
  {wa'į} {šjuuc} {hi<ha>'ų= anąga ha<ha- \textbf{ki}>tuką}\\
  {blanket} {be.warm} {<\textsc{1e.a}>use=and  <\textsc{1e.a}-\textbf{\textsc{\textsc{refl}}}>cover{\textbackslash}\textsc{1e.a}}\\
 \glt `I covered myself with a warm blanket.' (lit. 'I use a warm blanket, and I covered myself ')  (\citealt{Hartmann2013}:example No.730)
\z 

In the first clause (\ref{ex:Helmbrecht:15}), the transitive verb `to cover something' takes both verbal markers, the reflexivizer and the possessive reflexive marker, and in the second (\ref{ex:Helmbrecht:16}) only the reflexivizer. Both clauses are semantically almost equivalent; the difference may perhaps be found in the completeness of the covering, which is partial in the first and complete in the second clause. The combination of reflexivizer plus possessive reflexive may thus correlate with a partial reflexive scenario. 

We also find partial reflexive scenarios involving body parts that are expressed only with the possessive reflexive marker and no reflexivizer. For instance, in order to express `to shave (oneself)' in Hoocąk, one has to use the transitive verb \textit{gik'o} `to scrape off something'. In order to get the `shave' meaning one has to modify the verb; see the following examples. 

\ea  \label{ex:Helmbrecht:17}
 \glll 'iihįra gigik'o\\
  {'ii-hį=ra} {Ø-Ø-\textbf{gi}{}-gik'o}\\
  {mouth-hair=\textsc{def}} {\textsc{3sg.u-3sg.a-appl.ben}-scrape.off}\\
 \glt `He\textsubscript{1} shaves him\textsubscript{2}' (lit. `He\textsubscript{1} shaves the beard for him\textsubscript{2}') [PM XI:9]
\z 


\ea \label{ex:Helmbrecht:18} 
  \glll iihįrą karaik'o \\
  'ii-hį=rą  Ø-Ø-\textbf{kara}-gik'o\\ 
  mouth-hair=def 3\textsc{sg.u-3sg.a-poss}.\textsc{refl}-scrape.off\\
  \glt `He\textsubscript{1} shaved himself\textsubscript{1}' (lit. `He\textsubscript{1} shaves his\textsubscript{1}, the mouth hair')
\z 

In (\ref{ex:Helmbrecht:17}), the verb \textit{gik'o} `to scrape off something' is derived by means of a benefactive applicative \textit{gi-}\textsc{appl.ben} that opens a benefactive U argument. Without it, the translation would simply be `he\textsubscript{1} scrapes it (beard) off'. Note that the clause in (\ref{ex:Helmbrecht:17}). is not a reflexive construction and the two third person arguments have different referents. The \textit{{}'iihį} `beard' remains the patient U argument of the verb \textit{gik'o} `to scrape off something'.

On the other hand, example (\ref{ex:Helmbrecht:18}) is a reflexive construction, but without the reflexivizer \textit{kii-}. Instead, the possessive reflexive marker is used. The reflexive scenario here is partial, because the U argument of \textit{gik'o} `scrape off something', the `mouth hair' is a body-part of the A argument. The construction literally says that `A shaves his mouth hair' rendered in English as `A shaves himself'. There are further examples that suggest that the degree of affectedness of the patient U in a partial reflexive scenario triggers the choice of different constructions. Compare, for example, the two clauses in (\ref{ex:Helmbrecht:19}) from our text corpus.


\ea \label{ex:Helmbrecht:19}
  WL:  \textit{kirucecere 'anąga nąąjurašge wakurucgisires'a}\\
  BO: \textit{ hąhą }\\
  \ea 
  \gll {WL:\textbf{kii}{}-ru<ce>ce-ire}    {'anąga}   {nąąju=ra=šge}  {wa-\textbf{ku}-rucgis-ire-s'a}\\
  {\textsc{\textsc{refl}}<\textsc{rdp}>pull.off.a.piece-\textsc{sbj.3\textsc{pl}}}  {and}  {hair=\textsc{def}=also}   {\textsc{obj.3\textsc{pl}}-\textsc{poss.\textsc{refl}}-cut.with.scissors-\textsc{sbj.3{pl}-iter}}\\
  \glt `they cut themselves and they also cut their hair'
  \ex 
  \gll {BO:}  {hąhą}\\
  {}  {yes}\\   
  	\glt `yes'  [RRT068]
  	\z 
\z 

In the first clause, the transitive verb \textit{rucé} `pull off a piece (of soft substance)', which is reduplicated \textit{rucecé} `pull off many pieces', takes the reflexivizer \textit{kii-}. This construction is translated by our speakers as `they cut themselves' implying that the action affects the referents of the A argument completely. The second clause describes a second reflexive scenario with the same referents as A, but in this case the action affects the actors only partially; \textit{rucgis} is like \textit{rucece} a transitive verb of cutting; see also the examples (\ref{ex:Helmbrecht:4}), (\ref{ex:Helmbrecht:8}) and (\ref{ex:Helmbrecht:9}) with the related verb \textit{mąącgis} `to cut something (with a cutting instrument)'.

Even less affected is the patient U argument in the reflexive scenarios in the following examples. The transitive verb \textit{horak} `to tell something' is used to express the reflexive scenario `to talk about oneself'. In both cases, reflexivity is marked solely by the possessive reflexive marker.

\ea \textit{Hįxųųnųįgregi...  hižą hokarakre... hąhąo heesge haakje.} \\
	\gll   {hį-xųųnų=nįk=regi}  {hižą}   {ho<\textbf{ka}>rak=re}   {hąhą'o} {heesge} {haa-kje}\\ 
  {\textsc{1e.u}-be.small=\textsc{dim}=\textsc{sim/loc}} {one}   {<\textbf{\textsc{poss.refl}}>tell=\textsc{imp}} 	{yes}   {that's.why}  {make/\textsc{caus}{\textbackslash}\textsc{1e.a}-\textsc{fut}}\\
 \glt `When I was little ... tell something about \textbf{yourself}! ... yes, guess I'll do that.' [HOR008]
\z 


\ea \textit{'Éegi hokarakšg\'{ų́}nį žéesge hiraireną}.\\
 \gll {'éegi}   {ho<\textbf{ka}>rak=šgų́ nį}
   {žeesgé}  {hiré-ire=ną}\\
  {then}  {<\textbf{\textsc{poss.\textsc{refl}}}>tell(\textsc{sbj.3sg})=\textsc{dub}} {thus}   {think.through-\textsc{sbj.3\textsc{pl}=decl}}\\
  \glt `Then she told about \textbf{herself}, that's what they thought.' [OH1.2\_024]
\z 

There are also examples in the Hoocąk corpus with \textit{horak} `to tell something' in which the combination of reflexive marker plus possessive reflexive appear (ED11019); similarly for verbs of thinking (FOX027a).

The examples discussed so far suggest that the partiality of the reflexive scenarios correlates with the type of reflexive construction. However, this is only a very loose tendency. We find also clear examples in the corpus where the reflexive scenario is partial, but it is still the canonical reflexive construction that is used. Compare the utterance HOR086 in (\ref{ex:Helmbrecht:22}). 


\ea \label{ex:Helmbrecht:22}  
   \textit{  Hegų 'eeja hamįąnąkšąną jaagų hegų hįgixgu ną'įkje wagi'ųňą jaagu paara hegų   nąąsura hąąke   nįį howacip rokigųnįge. }\\
  \gll {hegų} {'eeja}  {hamį<ha>nąk=šąną}   {jaagu} {hegų}  {hį-gixgu}   {nąą'į-kje}  {wa<gi>'ų=ra} {jaagu}  {paa=ra}  {hegų} {nąąsu=ra}  {hąąke}   {nįį}   {ho-wacip} {roo<\textbf{kii}>gų-nį=ge}\\
  {that.way} {there} {<\textsc{1e.a}>sit.on(\textsc{obj.3sg})=only} {what}  {that.way}  {\textsc{1e.u}-buck.off}  {try(\textsc{sbj.3sg})-\textsc{FUT}}  {<\textsc{appl.ben}>do/be(\textsc{sbj.3sg})=\textsc{def}} {what}   {nose=\textsc{def}} {that.way} {head=\textsc{def}}   {\textsc{neg.in}} {water} {\textsc{appl.iness}-dump} {<\textbf{\textsc{\textsc{refl}}}>want(\textsc{sbj.3sg})-\textsc{neg.fin=causal}}\\
 \glt `Then I sat on her, and she was going to try to buck me off, but she didn't want to put   her nose or head in the water.' (lit. ...., but she did not want to put \textbf{herself} in the water   \textbf{with} \textbf{regard} \textbf{to} \textbf{the} \textbf{nose} \textbf{and} \textbf{the} \textbf{head}{}')[HOR086]
\z 

In HOR086, the actor is the `horse', which is introduced in previous clauses of this text. The reflexive construction is used to express the situation that the horse did not want to put parts of its body (`nose and head') under the water. This clause has to be read literally: `she didn't want to put herself under the water, the nose (and) the head'. The reflexive scenario is thus partial, but the canonical reflexivizer is used. From the perspective of the English translation, one would expect the possessive reflexive to mark the possession of the body parts (`nose'/ `head'). 

\section{Coreference of the subject with various semantic roles}\label{sec:Helmbrecht:5}

\subsection{Possessor}\label{sec:Helmbrecht:5.1}
As has already been illustrated with a few examples (see \ref{ex:Helmbrecht:1}, \ref{ex:Helmbrecht:12}, \ref{ex:Helmbrecht:18}, and \ref{ex:Helmbrecht:19} above), there is a possessive reflexive marker \textit{kara-/} \textit{kV}/ \textit{k-} \textsc{poss}.\textsc{refl} that indicates a possessive or other close relationship between the A argument and the U argument of the transitive verb. A canonical example from the text corpus would be BF1006 in (\ref{ex:Helmbrecht:23}).

\ea \label{ex:Helmbrecht:23}  
 BO:  \textit{ hegų mįįkeeja mįįnąkanąkšąną wiižukra kurusgenąkšąną }\\
 	\gll  {hegų}  {mįįk='eeja}  {mįįnąk-a=nąk=šąną}  {wiižuk=ra} {Ø-Ø-\textbf{ku}{}-rusge=nąk=šąną}\\
  {that.way}  {lie.down=there} {sit(\textsc{sbj.3sg})-0=\textsc{pos.ntl=decl}}	{gun=\textsc{def}}  {\textsc{obj.3sg-sbj.3sg}-\textbf{\textsc{poss.\textsc{refl}}}{}-clean.up=\textsc{pos.ntl=decl}}\\
  \glt `He was sitting on his cot, he was cleaning \textbf{his} rifle.' [BF1006]
\z 

The transitive verb \textit{rusgé} `to clean something' has a 3\textsc{sg} A argument and a 3SG U argument; both are marked zero. The A argument `he' is the topic of this stretch of discourse, the U argument indexes the NP `the rifle', which is part of the clause. The \textsc{poss.refl} marker indicates the possessive relation between A and U. If the possessor of the U argument is not coreferent with the A argument, another construction has to be used. In Hoocąk, usually the benefactive applicative \textit{gi-} \textsc{appl.ben} is used. This form translates in English as `for someone', but, in addition, the benefactive applicative systematically has a possessive meaning. Compare the following example.



\ea \label{ex:Helmbrecht:24} \textit{ 'iihįra gigik'o }\\
	\gll 'ii-hį=ra Ø-Ø-\textbf{gi}-gik'o\\
  mouth-hair=\textsc{def}   \textsc{3sg.u-3sg.a-appl.ben}-scrape.off\\
   \glt `He\textsubscript{1} shaves him\textsubscript{2}' (lit. `He\textsubscript{1} shaves the beard for him\textsubscript{2}{}', or `He\textsubscript{1} shaves his\textsubscript{2} beard') [PM XI:9]
\z 

The transitive verb \textit{gik'o} `to scrape something off' receives a second U argument, which is semantically a benefactive or a possessor. The actor shaves the beard for someone else, which always implies that this someone else is or may be the possessor of the beard. The beneficiary of the shaving is never coreferent with the actor (A argument), and it is this benefactive/ possessor U argument that is - in most cases - pronominally marked on the verb. The interpretation of the beneficiary as possessor is always available with intransitve inactive verbs, as well as with transitive verbs, and does not depend on the patient U argument, i.e. does not presuppose that the patient U argument is a body part (cf. \citealt{helmbrecht2003possession,Helmbrecht2021}). This is demonstrated in (\ref{ex:Helmbrecht:25}).

\ea  \label{ex:Helmbrecht:25}
  \ea \textit{hi'é} 'to find something'
  \ex \textit{  hi-gi-'é} 'to find something for someone'
  \ex \textit{ wažątírera hįįgí'eeną  }  \\
	\gll  {wažątíre=ra} {hi-< Ø-hį-Ø{}-> gí-'e=ną}\\
	{car-\textsc{def}}{<3sg.u\textsubscript{pat}-\textsc{1e.u}\textsubscript{ben}-3\textsc{sg.a}-3>\textsc{appl.ben}-find=decl}\\
 \glt `He found the car \textbf{for} \textbf{me}'/ `He found \textbf{my} car' \citep[29]{helmbrecht2003possession}
 \z 
\z 

Here, the patient U argument (`the car') is a third person and thus zero marked. If the patient U argument were plural, the \textsc{obj}.3\textsc{pl} marker \textit{wa-} would have been used. The beneficiary is licensed by the \textit{gi-} applicative marker, and is likewise marked by a pronominal affix of undergoer paradigm. This beneficiary may always be interpreted as the possessor of the patient U argument (`the car'), no matter if the patient U argument is a body part or not.

\subsection{Locational participants}\label{sec:Helmbrecht:5.2}

The two personal pronouns in Hoocąk mentioned above are used only in certain focus constructions. They never fill argument positions of verbs or prepositions. In addition, Hoocąk has no real adpositions, thus, construction like `She\textsubscript{1} saw a snake \textbf{besides} \textbf{her}\textsubscript{1}{}' in English, where the pronominal complement of the preposition `besides' is coreferent with the subject of the clause do not exist. However, locative U arguments exist, in particular, if they are added to the argument frame of the verb by means of a locative applicative. One of these applicative markers is \textit{ha-} \textsc{appl.supess}, which can be translated as `on something' `over something' and the like. Together with the reflexivizer it is possible to mark coreference between the locative U argument and the A argument, as seen in (\ref{ex:Helmbrecht:26})\todo{25 in the original}.

\ea   \label{ex:Helmbrecht:26}  \textit{Kutei, nįį haakipaxų́!} \\
  	\gll {kutei} {nįį} {ha-ha-\textbf{kii}-paaxų}\\
  {\textsc{intj}(male)} {water} {\textsc{appl.supess}-\textsc{1e.a}-\textbf{\textsc{refl}}-pour{\textbackslash}1\textsc{e.a}}\\
	 \glt `Oh, I poured water over \textbf{myself}.' \citep[example No. 31]{Hartmann2013}
\z 

There is a second locative applicative \textit{ho-} \textsc{appl.iness} that is usually translated as `in something', `into something'. With the reflexivizer and the locative inessive applicative, partial coreference is marked with the A argument; compare (\ref{ex:Helmbrecht:27}): 

\ea   \label{ex:Helmbrecht:27} \textit{Wanąą, nąącawara nįį waakipaxų }\\
 \gll  {wanąą} {nąącawa=ra}  {nįį}   {ho-ha-kii-paaxų}́ \\
  {\textsc{intj}(female)}   {ear=\textsc{def}} {water} {\textsc{appl.iness-1e.a}-\textbf{\textsc{refl}}-pour{\textbackslash}\textsc{1e.a}}\\
  \glt `I poured water into my ear.' \citep[example No. 32]{Hartmann2013}
\z 

Both examples have a non-reflexive meaning, if the reflexivizer is dropped.

\subsection{Benefactive/ recipient}\label{sec:Helmbrecht:5.3}
In addition to the above-mentioned locational applicatives, Hoocąk has a benefactive applicative \textit{gi-} appl.BEN that introduces a beneficiary or recipient U argument into the argument frame of the verb. One could expect that this applicative marker may co-occur with the reflexivizer in the same manner as the locative applicatives co-occur with the reflexivizer, thus indicating coreference of the A argument with the benefactive/ recipient U argument. Interestingly, this is not possible. I did not find a single instance of this combination in the entire Hoocąk corpus (of more than 100 texts) and such a combination does not occur on Hartmann's database of examples \citep{Hartmann2013} either. However, there are many instances in the corpus where the reflexivizer \textit{kii-} alone has a beneficiary reading. See the following two examples.



\ea \label{ex:Helmbrecht:28} \textit{ "Žee mąąšųų rakišuruxurukikjane," hįįge. }\\
 \gll  {žee}  {mąąšųų}  {ra-\textbf{kii-}šu-ruxuruk-i-kjane}   {hi<hį>ge}\\
  {that}  {feather}  {\textsc{2.a}-\textbf{\textsc{refl}}-\textsc{2.a}-accomplish-0-\textsc{fut}} {<\textsc{1e
  e.u}>say.to(\textsc{sbj.3sg})}\\
  \glt `You'll earn \textbf{yourself} a feather," he said to me.' [BOF061]
\z 


\ea  
\label{ex:Helmbrecht:29}
\textit{Heesge ha'ų woorák te'e hegų hakurukézixjį yaakíre.}\\
 \gll   {heesge} {ha-'ųų} {woorak} {te'e} {hegų}   {ha-kurukezi=xjį} {hi<ha-\textbf{kii-}>re}\\
  {that's.why} {\textsc{1e.a}-do/make} {story} {this} {that.way}  {\textsc{1e.a}-hold.highly=\textsc{ints}} {<\textsc{1e.a}-\textbf{\textsc{refl}}>think}\\
  \glt `That's why I thought I would bring out this story.' (Lit. `That's why I did it, I bring this story, I thought it \textbf{for} \textbf{myself'}) [WIL134]
\z 

In both cases, the \textit{kii-} \textsc{refl} marker produces a kind of autobenefactive meaning. The U argument in both utterances is not identical with the A argument, but a kind of third participant is introduced that is the beneficiary of the action. In (\ref{ex:Helmbrecht:27}), it is the addressee of the direct speech of the grandfather of the speaker; in WIL134, it is the speaker himself, but he is not the patient U of the verb \textit{hiré} `to think something'. Note that the double marking of the second person A in (\ref{ex:Helmbrecht:27}) has nothing to do with reflexivity and is just a peculiarity of the morphology of the Hoocąk verb.

\section{Coreference between non-subject arguments}\label{sec:Helmbrecht:6}

Hoocąk does not have adpositional phrases (as clausal adjuncts) containing free personal pronouns or reflexive pronouns. Thus, constructions like \textit{He} \textit{spoke} \textit{with} \textit{John\textsubscript{1}} \textit{about} \textit{himself\textsubscript{1}}, or \textit{She} \textit{told} \textit{us\textsubscript{1}} \textit{about} \textit{ourselves\textsubscript{1}} do not exist. The only way to express these states of affairs in Hoocąk is to split these clauses in two such that the verbal predicate is repeated: `He spoke with/to John, and he spoke about him'. 

\section{Contrast between object and nominal adpossessor}\label{sec:Helmbrecht:7}

Hoocąk has no possessive pronouns. Instead, Hoocąk has two kinds of external possessor marking, one with the possessive reflexive marker, and one with the benefactive applicative. The possessive reflexive is used when the referent of A is, at the same time, the possessor of U (see \ref{ex:Helmbrecht:23} above). The benefactive applicative is used, if the possessor of U is someone else (see \ref{ex:Helmbrecht:24} and \ref{ex:Helmbrecht:25} above). Another construction that allows expressing that the possessor of U is different than A is the NP-internal possessive construction with \textit{hani}. Contrast the two following examples.


\ea \label{ex:Helmbrecht:30}
  \ea \textit{nįįkją́k waakáragigųsšąną}\\ 
   \gll {nįįkj\'ąk}  {wa-ha-\textbf{kára}-gigųs=šąną}\\
 {child} {\textsc{obj.3\textsc{pl}-1\textsc{ea}-\textbf{\textsc{poss}.\textsc{refl}}}-teach-\textsc{decl}}\\
   \glt `I\textsubscript{1} taught my\textsubscript{1} children.'
  \ex  \textit{nįįkją́ k waanàń ą waagígųsšąną}\\
  \gll {nįįkj\'ąk} {wa-ha-Ø{}-nį\'{} =ra}   {wa-ha-gígųs=šąną}\\ 
 {child}  {\textsc{obj.3pl}-own-\textsc{sbj.3sg}-own=\textsc{def}}  {\textsc{obj.3pl-1e.a}-teach=\textsc{decl}}\\
  \glt `I\textsubscript{1} teach his\textsubscript{2} children.' [DL XXIII:3]
  \z 
\z 

The first one (\ref{ex:Helmbrecht:30}a) employs the possessive reflexive indicating that A is the possessor of U. The second one (\ref{ex:Helmbrecht:30}b) is used because the possessor of U is not A, but someone else. The construction is a kind of NP-internal possessive construction with a fully inflected transitive verb \textit{hanį} `to own sth.' that is nominalized with the definite article. The NP can be translated literally as 'child(ren)\textsubscript{1} (that) he owns them\textsubscript{1}'. 

\section{Contrast between exact and inclusive coreference}\label{sec:Helmbrecht:8}

Exact coreference between a third person A and a coreferent third person U is expressed with the reflexivizer \textit{kii-} as in (\ref{ex:Helmbrecht:2}a) above. On the other hand, there is no easy and direct way to express inclusive coreference of the type `\textit{she\textsubscript{1}} \textit{sees} \textit{herself} \textit{and} \textit{the} \textit{others\textsubscript{1+x}}{}'. The reason is, again, that Hoocąk has no free personal pronouns or free reflexive pronouns that can enter into a syntactic coordination with `\textit{pro} \textit{and} \textit{the} \textit{others}{}'. 

\section{Long distance coreference}\label{sec:Helmbrecht:9}

Reflexive marking (reflexivizer, reflexive possessive) is restricted to the clausal domain. There is no special construction in Hoocąk indicating coreference of, for instance, subject arguments across clause boundaries, as can be found in complement clauses of the type `\textit{She\textsubscript{1}} \textit{thought} \textit{that} \textit{she\textsubscript{1}} \textit{had} \textit{enough} \textit{money}{}'. However, Hoocąk allows the suspension of person indexing of the S/A arguments in the complement clause, if this argument is coreferent with one of the arguments S/A/U) of the matrix clause; see for instance CHT064a below.\todo{maybe better example number?} 

\ea  \textit{woorák te'é, hiperés nąąn\'{į}gi'įgé, 'eesgé wáa'\'{ų́}ną}  .\\
 \gll  [{woorák} {te'é}  {\textbf{hiperés}}  {\textbf{nąą<nįį-gi>'į=ge}}]   [{'eesgé} {wa<ha>'ųų=ną}]\\
  [{story}  {this}  {\textbf{know}} {\textbf{<1\&}\textbf{\textsc{2-appl.ben}>want=\textsc{causal}}}]   [{thus}  {<\textsc{1e.a}>be/do=\textsc{decl}}]\\
  \glt `Because I wanted you to know this story, I did this.' [CHT064b]
\z 

The embedded transitive verb \textit{hiperes} `know' is not marked for the A argument which should be a second person A argument that is at the same time the U argument of the matrix verb \textit{nąą'į} `to attempt sth., to want sth.'. The verb \textit{hiperés} `to know something' of the complement clause is still a finite verb; it still inflects for the U argument (i.e. a zero affix here). This is thus a construction that signals coreference, but since it has no special form, it is not a reflexive construction.

\section*{Acknowledgements}

I am grateful to the corrections and suggestions of the editors and reviewers to improve the paper. I am responsible for all shortcomings and mistakes.   

\section*{Abbreviations}
\begin{tabularx}{\textwidth}{lQ}
1,2,3   & first, second, third person,  \\
\textsc{A}  & actor; agent   \\
\textsc{appl.iness }  & inessive applicative prefix;  \\
\textsc{appl.inst} & instrumental applicative;  \\
\textsc{appl.supess } & superessive applicative prefix;  \\
\textsc{assump}   & assumptive;    \\
\textsc{causal} & causal;  \\
\textsc{coll}  & collective marker;   \\
\textsc{cont}  & continuative;  \\
\textsc{ctv} & complement taking verb  \\
\textsc{decl }  & declarative;   \\
\textsc{def }   & definite;   \\
\textsc{dem.dist}   & demonstrative;    \\
\textsc{dem.dist}  & demonstrative, distal;  \\
\textsc{dem.dist}  & demonstrative, proximal;   \\
\textsc{dim} & diminutive;    \\
\textsc{dist }  & distal;  \\
\textsc{du } & dual;    \\
\textsc{dub} & dubitative;    \\
\textsc{emph }  & emphatic;   \\
\textsc{excl }  & exclusive;  \\
\textsc{foc} & focus;   \\
\textsc{freq}   & frequentative;    \\
\textsc{fut}   & future;  \\
\textsc{hab} & habitual;   \\
\textsc{hyp}   & hypothetical   \\
\textsc{imp.post} & delayed imperative;  \\
\textsc{incl}  & inclusive;  \\
\textsc{infer}  & inferential;   \\
\textsc{ints}   & intensifier;   \\
\textsc{iter}   & iterative;  \\
\textsc{neg.fin}   & final negator  \\
\textsc{obj} & object;  \\
\textsc{opt} & optative;   \\
\textsc{pl} & plural;  \\
\textsc{pos.hor}  & `be (lying/ horizontal position)';  \\
\textsc{pos.ntl}  & `be (sitting/ neutral position)';   \\
\end{tabularx}

\begin{tabularx}{\textwidth}{lQ}
\textsc{pos.vert} & `be (standing/ vertical position)'; \\
\textsc{poss.refl}   & possessive reflexive;   \\
\textsc{pot}   & potential;  \\
\textsc{prop}  & proper name marker;  \\
\textsc{prox}  & proximal;   \\
\textsc{recp}   & reciprocal  \\
\textsc{refl}   & reflexive marker  \\
\textsc{S\textsubscript{a}}  & intransitive (actor) subject; \\
\textsc{sbj}   & subject;    \\
\textsc{seq}   & sequential;    \\
\textsc{sg } & singular;   \\
\textsc{sim} & simultaneous;  \\
\textsc{sim/loc}  & simultaneity/ locative; \\
\textsc{S\textsubscript{u}}  & intransitive (undergoer) subject;   \\
\textsc{top} & topic;   \\
\textsc{u}  & undergoer; patient;   \\
\end{tabularx}

{\sloppy\printbibliography[heading=subbibliography,notkeyword=this]}
\end{document}
