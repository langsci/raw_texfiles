\documentclass[output=paper]{langscibook}

\author{Karin Michelson\affiliation{University at Buffalo}}

\title{Reflexive prefixes in Oneida}

\abstract{Oneida expresses coreference (or coindexing) by means of two verbal prefixes: the reflexive and the semi-reflexive. Coindexing is strictly a matter of morphology; there are no reflexive nominals, and the verbal prefixes are not grammatical voice morphemes. Both prefixes have other functions as well; for example the semi-reflexive derives anticausative verbs and verbs of nontranslational motion, and the reflexive can express reciprocity.}

\IfFileExists{../localcommands.tex}{
 \addbibresource{localbibliography.bib}
 \usepackage{langsci-optional}
\usepackage{langsci-gb4e}
\usepackage{langsci-lgr}

\usepackage{listings}
\lstset{basicstyle=\ttfamily,tabsize=2,breaklines=true}

%added by author
% \usepackage{tipa}
\usepackage{multirow}
\graphicspath{{figures/}}
\usepackage{langsci-branding}

 
\newcommand{\sent}{\enumsentence}
\newcommand{\sents}{\eenumsentence}
\let\citeasnoun\citet

\renewcommand{\lsCoverTitleFont}[1]{\sffamily\addfontfeatures{Scale=MatchUppercase}\fontsize{44pt}{16mm}\selectfont #1}
   
 %% hyphenation points for line breaks
%% Normally, automatic hyphenation in LaTeX is very good
%% If a word is mis-hyphenated, add it to this file
%%
%% add information to TeX file before \begin{document} with:
%% %% hyphenation points for line breaks
%% Normally, automatic hyphenation in LaTeX is very good
%% If a word is mis-hyphenated, add it to this file
%%
%% add information to TeX file before \begin{document} with:
%% %% hyphenation points for line breaks
%% Normally, automatic hyphenation in LaTeX is very good
%% If a word is mis-hyphenated, add it to this file
%%
%% add information to TeX file before \begin{document} with:
%% \include{localhyphenation}
\hyphenation{
affri-ca-te
affri-ca-tes
an-no-tated
com-ple-ments
com-po-si-tio-na-li-ty
non-com-po-si-tio-na-li-ty
Gon-zá-lez
out-side
Ri-chárd
se-man-tics
STREU-SLE
Tie-de-mann
}
\hyphenation{
affri-ca-te
affri-ca-tes
an-no-tated
com-ple-ments
com-po-si-tio-na-li-ty
non-com-po-si-tio-na-li-ty
Gon-zá-lez
out-side
Ri-chárd
se-man-tics
STREU-SLE
Tie-de-mann
}
\hyphenation{
affri-ca-te
affri-ca-tes
an-no-tated
com-ple-ments
com-po-si-tio-na-li-ty
non-com-po-si-tio-na-li-ty
Gon-zá-lez
out-side
Ri-chárd
se-man-tics
STREU-SLE
Tie-de-mann
} 
 \togglepaper[1]%%chapternumber
}{}

\begin{document}
\maketitle 

\section{Introduction}\label{sec:oneida:1}

Oneida (Northern Iroquoian), a polysynthetic language of North America, expresses coreference (coindexing) within a clause morphologically by means of two prefixes to the verb stem: 
the \textbf{reflexive} prefix {-atat-/-atate-} and the formally related \textbf{semi-reflexive} pre\-fix {-at-/-ate-/-atʌ-/-an-/-al-/-a-}.\footnote{The term {coindexing} is used here rather than {coreference}, following the cogent critique of the term {coreference} in the context of (reflexive) pronouns in \citet{BachPartee1980} and subsequent work.} There are no reflexive nominals in Oneida. Verb forms with the reflexive and semi-reflexive prefixes are given in \REF{ex:oneida:1} and \REF{ex:oneida:2}. The pronominal inflections in \REF{ex:oneida:1a} and \REF{ex:oneida:2a} mark a relation between two distinct animate arguments: first person exclusive plural acting on third person masculine singular in \REF{ex:oneida:1a}, and first person singular acting on third person masculine singular in \REF{ex:oneida:2a}. The pronominal inflections in \REF{ex:oneida:1b} and \REF{ex:oneida:2b} mark a single animate argument. The verb form in \REF{ex:oneida:1b}, with the reflexive prefix, is inflected with the pronominal prefix that marks a first person exclusive plural argument, and the verb form in \REF{ex:oneida:2b}, with the semi-reflexive prefix, is inflected with the pronominal prefix that marks a first person singular argument.\footnote{\textrm{[FEFF?]}In the Oneida orthography the vowel {u} is a high-mid back mildly rounded nasalized vowel and {ʌ} is a low-mid central nasalized vowel. A raised period indicates vowel length. Underlining indicates devoicing, a common phenomenon at the end of an utterance. Excerpts from \citet{MichelsonKennedyDoxtator2016} give speaker and title of the recording. Note that in the excerpts not every particle is glossed; a sequence of particles may be translated into English with a single word.}\todo{put the "titles of examples at the top level rather than under a)}


\ea%1
    The reflexive prefix\\
   \label{ex:oneida:1}
    \ea  
     \label{ex:oneida:1a}
    \glll {waʔshakwaste·lísteʔ}\\
     waʔ-shakwa-stelist-eʔ\\
         \textsc{fact}-\textsc{1ex.pl>3m.sg}-laugh.at-\textsc{pnc}\\
   \glt ‘we laughed at him’
  
  \ex 
   \label{ex:oneida:1b}
  \glll {waʔakwatateste·lísteʔ}\\
    waʔ-yakw-\textbf{atate}{}-stelist-eʔ\\
        \textsc{fact}-\textsc{1ex.pl.a-}\textbf{\textsc{refl}}-laugh.at-\textsc{pnc}\\
    \glt ‘we laughed at ourselves’
    \z
\z

\ea%2 
The semi-reflexive prefix
    \label{ex:oneida:2}
    \ea 
    \label{ex:oneida:2a}
    \glll {wahitsiʔnyuhklo·kéweʔ}\\
     wa-hi-tsiʔnyuhkl-okew-eʔ\\
         \textsc{fact-1sg>3m.sg}-snot-wipe-\textsc{pnc}\\
    \glt ‘I wiped his nose’

    \ex  
    \label{ex:oneida:2b}
    \glll {waʔkattsʔnyuhklo·kéweʔ}\\
     waʔ-k-\textbf{at}{}-tsiʔnyuhkl-okew-eʔ\\
       \textsc{fact-1sg.a-}\textbf{\textsc{semirefl}}-snot-wipe-\textsc{pnc}\\
    \glt ‘I wiped my nose’    
   \z
\z

Reflexive meaning in Oneida is expressed exclusively in the verbal morphology. However the reflexive and semi-reflexive are not grammatical voice morphemes; although they do express meanings that in other languages are associated with reflexive or middle voice, there is no evidence for an inflectional voice category in Oneida. In addition it should be noted that the functions of the reflexive and semi-reflexive prefixes are not restricted to coindexing. This is especially so for the semi-reflexive, where the outcome of affixa\-tion often has an unpredictable semantic component. Oneida does have independent personal pronouns, used primarily for emphasis and contrast, but there is no reflexive pronoun.

The next section of the paper provides a very brief overview of Oneida morphology that is relevant for understanding the interaction of the reflexive and semi-reflexive with verbal pronominal marking. \sectref{sec:oneida:3} describes the functions of the reflexive, and \sectref{sec:oneida:4} is about the semi-reflexive. \sectref{sec:oneida:4} also compares the distribution of the various forms of the two prefixes. \sectref{sec:oneida:5} describes how participant roles interact with coindexing, including alternative structures to coindexing for certain roles. The last section ends the paper with some final observations.

\section{Background}\label{sec:oneida:2}

Oneida is a Northern Iroquoian language spoken by fewer than forty speakers who learned Oneida as their first language. Historically the Oneida nation was located in upstate New York, just east of Syracuse, but in the 1800s groups moved to southwestern Ontario and to northeastern Wisconsin near Green Bay. While today the Oneida, or Onʌyote’a·ká· (People of the Standing Stone), reside in all three locations, Oneida is spoken by first-language speakers only at the Oneida Nation of the Thames territory in Ontario. (Map downloaded 25-06-20 from \url{http://www.angelfire.com/on3/oneida/page2d.html}.)

%{\textbackslash}  
%%please move the includegraphics inside the {figure} environment
%%\includegraphics[width=\textwidth]{figures/ReflexivePrefixesinOneidarevJune2020-img001.gif}

Oneida is a polysynthetic language and like other Northern Iroquoian languages is remark\-able for its complex verbal morphology, including around sixty or so bound pronominal prefixes, an intricate distribution of prepronominal prefixes that include meanings having to do with negation, locations, and quantity (to mention a few), and robust noun incorporation. Despite the proximity to the dominant English-speaking towns, Oneida has relatively few borrowings, instead using mostly conventionalized inflected verb forms as labels for new concepts. There are over 150 uninflected particles with a wide range of meanings; they can, for example, express locations, negation, quantitative and modal concepts, and link clauses in various ways. Most of the examples in this paper may be found in \citep{MichelsonDoxtator2002} dictionary or in the texts published in \citet{MichelsonKennedyDoxtator2016}.

Traditionally, Northern Iroquoian is described as having three morphological parts of speech – verbs, nouns and uninflected particles, with kinship terms more recently recog\-nized as a fourth (see \citealt{KoenigMichelson2010}). Verbs, nouns and kinship terms are obligatorily inflected with pronominal prefixes. The semantic categories distinguished by the prefixes are person (first, second, third, plus inclusive versus exclusive), number (singular, dual, plural), and gender (masculine, feminine-zoic, feminine). The feminine-zoic singular refers to some female persons, animals, and some inanimates in motion (\citealt{Abbott1984}, \citealt{Michelson2015}). The femi\-nine occurs only in the singular; all nonsingular female persons are referred to with feminine-zoic prefixes.\footnote{The label ‘neuter’ sometimes is used in place of feminine-zoic for some of the languages related to Oneida, such as Cayuga and Seneca which no longer distinguish reference to single female ‘zoic’ persons from inanimates.} An indefinite (or unspecified) category is syncretic throughout the system with the feminine singular, and ‘feminine-indefinite’ is the traditional label for the feminine singular plus indefinite categories.

There are three paradigmatic classes of pronominals, and their distribution is relevant for understanding the morphology of reflexive verbs as compared with corresponding non-reflexive verbs. First, there is a class of portmanteau-like prefixes that occur with verbs that have two semantic arguments that are both animate (which includes most kinship terms). For example, the verb form in \REF{ex:oneida:3} is inflected with the prefix that references a first person singular proto-agent and a third person masculine singular proto-patient (the terms proto-agent and proto-patient are adopted from \citealt{Dowty1991} for semantic roles not confined to canonical agent and patient). The verb forms in \REF{ex:oneida:1a} and \REF{ex:oneida:2a} in the introduction also have prefixes that reference two animate arguments. The other two classes of pronominals, Agent and Patient, occur with verbs that have only one animate semantic argument. Verbs with Agent pronominals are exemplified in \REF{ex:oneida:4} and \REF{ex:oneida:5}. The verb {-ye-} ‘wake up’ in \REF{ex:oneida:4} has one animate semantic argument, third person masculine singular, and it is inflected with the Agent prefix {ha-}. The verb {{}-ket-} ‘scrape, grate’ in \REF{ex:oneida:5} has two semantic arguments but only one animate argument, third person masculine singular; the Agent prefix {ha-} references this animate argument, and the inanimate argument is not referenced. When a verb has no animate arguments, the verb is inflected with the feminine-zoic singular prefix, since every verb must have a pronominal prefix. Often, the selection of Agent versus Patient paradigm may be evident from the meaning of the verb, but in many cases the semantic motivation has become obscured and the selection of Agent/Patient prefixes is considered by all Iroquoian scholars to be lexically determined by the verb. (See \citealt{KoenigMichelson2015} for a detailed discussion about the realization of arguments in Oneida and the distribution of pronominal prefixes, including arguments for the feminine-zoic singular prefix as the default prefix.)

\ea%3 Verb with two animate arguments
    \label{ex:oneida:3}
    \glll wahihle·wáhteʔ \\
     wa-\textbf{hi}-hlewaht-eʔ\\
        \textsc{fact}-\textbf{\textsc{1sg>3m.sg}}-punish-\textsc{pnc}\\
    \glt ‘I punished him’
    \z


\ea%4
Monadic verb with one animate argument: Agent prefix\\
    \label{ex:oneida:4}
    \glll waha·yé·\\
     wa-\textbf{ha}-ye-ʔ\\
       \textsc{fact-}{\textsc{3m.sg.a}}-wake.up-\textsc{pnc} \\
   \glt ‘he woke up’
   \z

 

\ea%5
    \label{ex:oneida:5}
    Dyadic verb with one animate argument: Agent prefix \\
    \glll waha·kéteʔ \\
    wa-\textbf{ha}-ket-eʔ\\
      \textsc{fact-}\textbf{\textsc{3m.sg.a}}-scrape,grate-\textsc{pnc} \\
    \glt  ‘he scraped or grated it’
  \z

Reflexive and semi-reflexive prefixes occur between the pronominal prefix and the verb root. The verbs {{}-nuhlyaʔk-} ‘hurt’ in \REF{ex:oneida:6} and {-ahseht-} ‘hide’ in \REF{ex:oneida:8} have two distinct animate arguments and bear prefixes referenc\-ing both arguments – the same arguments as the verb form in \REF{ex:oneida:3}. The form in \REF{ex:oneida:7}, with the reflexive {-atat-}, is inflected with the first person singular Agent prefix referencing the single distinct animate argument. Likewise, the verb forms in \REF{ex:oneida:9} and \REF{ex:oneida:10}, the latter with the semi-reflexive, have only one animate argument and both are inflected with an Agent prefix.\footnote{Verbs with the reflexive prefix always occur with the Agent paradigm of pronominal prefixes. Verbs with the semi-reflexive can select the Patient paradigm. Some verbs, such as {{}-ahseht-} ‘hide’ in \REF{ex:oneida:8}-(10), require the incorporated root {{}-yaʔt-} ‘body’ when the affected argument is animate, as is the case in \REF{ex:oneida:8}; see \citealt{MichelsonDoxtator2002}.}

\ea%6
   \label{ex:oneida:6}
   Verb with two distinct animate arguments\\
    \gll wa-\textbf{hi}-nuhlyaʔk-eʔ\\
      \textsc{fact}-\textbf{\textsc{1sg>3m.sg}}-hurt-\textsc{pnc} \\
   \glt ‘I hurt him’
  \z
   

\ea%7
    \label{ex:oneida:7}
    Reflexive verb with one distinct animate argument: Agent prefix\\
   \glll {wahatatnú·lyahkeʔ}\\
   wa-\textbf{k}-\textbf{atat}-nuhlyaʔk-eʔ\\
    \textsc{fact-}\textbf{\textsc{1sg.a}}\textsc{-}\textbf{\textsc{refl}}-hurt-\textsc{pnc} \\
   \glt ‘I hurt myself’
  \z

    
\ea%8
   \label{ex:oneida:8}
  Verb with two animate arguments\\
 \glll {wahiyaʔtáhsehteʔ}\\
  wa-\textbf{hi}-yaʔt-ahseht-eʔ\\
 \textsc{fact}-\textbf{\textsc{1sg>3sg}}-body-hide-\textsc{pnc} \\
  \glt ‘I hid him’
 \z

\ea%9
    \label{ex:oneida:9}
   Verb with one animate argument: Agent prefix\\
  \glll {waʔkáhsehteʔ}\\
   waʔ-\textbf{k}-ahseht-eʔ\\
  \textsc{fact}-\textbf{\textsc{1sg.a}}-hide-\textsc{pnc}  \\
   \glt  ‘I hid it’
 \z


\ea%10
    \label{ex:oneida:10}Semi-reflexive verb with one distinct animate argument: Agent prefix\\
   \glll {waʔkatáhsehteʔ}\\
   waʔ-\textbf{k}-\textbf{at}-ahseht-eʔ\\ 
    \textsc{fact-}\textbf{\textsc{1sg.a}}\textsc{-}\textbf{\textsc{semirefl}}-hide-\textsc{pnc}\\
  \glt ‘I hid’
  \z

Oneida does have free-standing pronouns, but they are used only for emphasis and contrast. First and second person pronouns are uninflected particles, i.e. have a constant form: {í·} for first person, and {isé·} for second person. Third person forms are based on a stem {{}-ulhaʔ}, inflected with the appropriate pronominal prefixes (from the Patient paradigm). This stem is often glossed ‘self’ in work on Iroquoian, but it is an intensifier and its function does not include coindexing. The excerpts in \REF{ex:oneida:11} and \REF{ex:oneida:12} are examples of how it is used.
\todo{What are the references in the examples? Not in reference list, no year}
    

%\ea%11
%    \label{ex:oneida:11}
%   \glll {Kwáh}  {akwekú} {lonulhá·} {lotiyʌthu.} \\
%    { } { } \textbf{lon-ulhaʔ} { }\\    
%    just  all \textbf{\textsc{3m.pl.p}}\textbf{-self} \textsc{3m.pl.p}-plant-\textsc{stv}\\
%    \glt ‘They grew everything themselves.’ Verland Cornelius, {A} {Lifetime} {of} {Memories}
%  \z
  
\ea%11  with both original language
    \label{ex:oneida:11}
    Intensifier {{}-ulhaʔ}\\
    \glll {Kwáh}  {akwekú} {lonulhá·} {lotiyʌthu.} \\
   Kwáh akwekú \textbf{lon-ulhaʔ}     loti-yʌtho-u\\    
    just  all \textbf{\textsc{3m.pl.p}}\textbf{-self} \textsc{3m.pl.p}-plant-\textsc{stv}\\
    \glt ‘They grew everything themselves.’ Verland Cornelius, {A} {Lifetime} {of} {Memories}
  \z

%\ea%12
%    \label{ex:oneida:12}
%   \glll  {nʌ} {akaulhá·} {sʌ·} {oskanʌha}  {waʔenhotu·kó·,}\\
%     { } \textbf{aka-ulhaʔ} { } { } { } waʔ-ye-nhotukw-ʔ  \\  
%    just  \textbf{\textsc{3fi.p}}\textbf{-self}  also  quietly %\textsc{fact-3fi.a}-open.a.door-\textsc{pnc}\\
%    \glt ‘then she herself also quietly opened the door,’ Norma %Kennedy, {The} {Girl} {With} {the} {Bandaged} {Fingers}
 % \z
\ea12
    \label{ex:oneida:12}
   \glll  {nʌ} {akaulhá·}    {sʌ·} {oskanʌha}  {waʔenhotu·kó·,}\\
     nʌ \textbf{aka-ulhaʔ} sʌ· oskanʌha waʔ-ye-nhotukw-ʔ  \\  
    just  \textbf{\textsc{3fi.p}}\textbf{-self}  also  quietly \textsc{fact-3fi.a}-open.a.door-\textsc{pnc}\\
    \glt ‘then she herself also quietly opened the door,’ Norma Kennedy, {The} {Girl} {With} {the} {Bandaged} {Fingers}
  \z

The next two sections give more detail about the distribution and functions of the reflexive and semi-reflexive prefixes.

\section{Reflexive prefix}\label{sec:oneida:3}

The reflexive prefix {-atat-/-atate-} functions to identify an instigator of an event as identical with the affected participant, i.e. coindexes a proto-agent and proto-patient participant. Some verbs that are attested with the reflexive are listed in \REF{ex:oneida:13}. The distribution of {{}-atat-} and {{}-atate-}is phonological: {{}-atate-} occurs when the prefixation of {{}-atat-} to the verb stem would result in a non-occur\-ring consonant cluster.

\ea%13
\label{ex:oneida:13}Verbs with the reflexive prefix\footnote{Some of these are inter\-nally complex; the composition of complex stems is given in \citet{MichelsonDoxtator2002} as part of the entry for the stem. Also, stems in Oneida may require a particular prepronominal prefix; for reasons of space, throughout this paper, stems are listed without these prefixes but again this information can be retrieved by consulting \citet{MichelsonDoxtator2002}.}\\          
{{}-aweʔest-} ‘prick, pierce, sting’ \&  {{}-atataweʔest-} ‘prick oneself’\\
{-hlen-} ‘cut into, incise’ \&  {-atathlen-} ‘cut oneself’\\
{{}-hloli-} ‘talk about someone’ \&  {{}-atathloli-} ‘talk about oneself’\\
{-itʌl-} ‘pity someone’ \&  {-atatitʌl-} ‘feel sorry for oneself’\\
{{}-kaly-} ‘bite someone’ \&  {{}-atatkaly-} ‘bite oneself’\\
{-kuʔtslihal-} ‘weigh something’ \&  {-atatkuʔtslihal-} ‘weigh oneself’\\
{-lyo-/-liyo-} ‘kill’ \&  {-atatliyo-} ‘kill oneself’\\
{-nutu-} ‘put something into someone’s mouth’ \&  {-atatnutu-} ‘feed oneself’\\
{-shnye-} ‘look after someone, nurture’ \&  {-atateshnye-} ‘look after oneself’\\
{-stelist-} ‘laugh at someone’ \&  {-atatestelist-} ‘laugh at oneself’\\
{-wyʌnataʔ-} ‘get something ready’ \& {-atatewyʌnataʔ-} ‘get oneself ready’\\
{-yaʔtakenha-} ‘help someone out’ \&  {-atatyaʔtakenha-} ‘help oneself’\\
{-ʔnikuhloli-} ‘entertain someone’ \&  {-atateʔnikuhloli-} ‘entertain or amuse oneself’\\
{-ʔnutanhak-} ‘blame someone’ \&  {-atateʔnutanhak-} ‘blame oneself’\\
{-ʔskuthu-} ‘burn someone’ \&  {-atateʔskuthu-} ‘burn oneself’

\z

An additional use of the reflexive prefix is with kinship terms. The reflexive can occur with a few kinship terms to indicate a dyadic relation; an ex\-ample is {{}-atatyʌha} ‘parent and child’ in the excerpt in \REF{ex:oneida:14a}. The effect of the reflexive with kinship terms is to express a reciprocal relation. Otherwise reciprocals normally require the dualic prepronominal prefix, as discussed later on in this section. Without the reflexive, the kinship term refers to one of the members only, as in \REF{ex:oneida:14b}.

%\ea%14
%    \label{ex:oneida:14}
%   The reflexive with kinship terms
 %   \ea 
%    \label{ex:oneida:14a}
%    \glll {yotinuhsóta} {kaʔikʌ} {onatatyʌha,} {tahnú·} {nʌ} {yaʔkáheweʔ} {a·kyatekhu·ní·,}\\
 %    yoti-nuhsota  {kaʔikʌ} on-\textbf{atat}-yʌha {tahnú·} {nʌ}     {yaʔkáheweʔ}   aa-ky-atekhuni-ʔ\\
 %  \textsc{3fz.pl.p-}have.a.home.together  this    \textsc{3fz.pl.p-}\textbf{\textsc{refl}}-parent:child and    then  it came time  \textsc{opt-3fz.du.a-}eat\textsc{-pnc}\\
%  \glt  ‘(Once upon a time) this mother and daughter had a home together, and when it       came time for the two of them to eat,’ Norma Kennedy, {The} {Bird}
 
%\ex  
% \label{ex:oneida:14b}
% \glll {Né·} {kwí·} {né·n}   {liyʌha} ...        {wahaya·kʌneʔ,} {yahaʔslo·tʌ·,}\\
%   { } { } { } { } wa-ha-yakʌʔ-neʔ y-a-h-aʔsl-ot-ʌʔ\\
%  \textsc{assertion}    \textsc{1sg>3m.sg}-parent:child  \textsc{fact-3m.sg.a-}exit\textsc{-pnc} \textsc{transloc-fact-3m.sg.a-}axe-stand\textsc{-pnc} \\
 % \glt ‘So my son, (if it seems like the weather is going to get real bad ...) he goes out, he plants an axe in the ground,’ Mercy Doxtator, {How} {to} {Divert} {a} {Storm}
 % \z
% \z

\ea14
    \label{ex:oneida:14}
   The reflexive with kinship terms
    \ea 
    \label{ex:oneida:14a}
    \glll {yotinuhsóta} {kaʔikʌ} {onatatyʌha,} {tahnú·} {nʌ} {yaʔkáheweʔ} {a·kyatekhu·ní·,}\\
     yoti-nuhsota  {kaʔikʌ} on-\textbf{atat}-yʌha {tahnú·} {nʌ}    {yaʔkáheweʔ}   aa-ky-atekhuni-ʔ\\
   \textsc{3fz.pl.p-}have.a.home.together  this    \textsc{3fz.pl.p-}\textbf{\textsc{refl}}-parent:child and    then  it came time  \textsc{opt-3fz.du.a-}eat\textsc{-pnc}\\
  \glt  ‘(Once upon a time) this mother and daughter had a home together, and when it       came time for the two of them to eat,’ Norma Kennedy, {The} {Bird}
    
 \ex  
 \label{ex:oneida:14b}
 \glll {Né·} {kwí·} {né·n}   {liyʌha} ...        {wahaya·kʌneʔ,} {yahaʔslo·tʌ·,}\\
   Né· kwí·né·n li-yʌha wa-ha-yakʌʔ-neʔ y-a-h-aʔsl-ot-ʌʔ\\
  \textsc{assertion}    \textsc{1sg>3m.sg}-parent:child  \textsc{fact-3m.sg.a-}exit\textsc{-pnc} \textsc{transloc-fact-3m.sg.a-}axe-stand\textsc{-pnc} \\
  \glt ‘So my son, (if it seems like the weather is going to get real bad ...) he goes out, he plants an axe in the ground,’ Mercy Doxtator, {How} {to} {Divert} {a} {Storm}
  \z
 \z

The reflexive can encode some additional meaning. For example, with certain one-place predicates that describe a physical attribute or kind of personality, the reflexive adds a component of meaning that may be rendered into English as ‘think oneself so’ or ‘act so’, as in \REF{ex:oneida:15a}. With some verbs the reflex\-ive adds a component that suggests effort, as with the verb ‘apply oneself’ in \REF{ex:oneida:15b}. Other verbs that cannot be derived compo\-sitionally from the meaning of the non-reflexive verb are ‘hire oneself out’ in \REF{ex:oneida:15c} and ‘turn oneself into (another being) in \REF{ex:oneida:15d}.
%\ea%15
 %    Reflexive verbs with some additional meaning \\
 %   \label{ex:oneida:15}
 %   \ea \label{ex:oneida:15a}
 %   \glll  {Shayá·tat} {kaʔikʌ} {kʌʔ} {nithoyʌha,} {yah} {kwí·} {teʔwé·ni}      {nihatatnikʌhteleʔ.}   \\ 
 %    { } { } { } { } yah   kwí· teʔwé·ni ni-h-\textbf{atat-}nikʌhtle-ʔ\\
 %   {he is one person}  this {young guy}  not  really    {it’s incredible} \textsc{part-3m.sg.a-}\textbf{\textsc{refl}}\textsc{-}be.handsome\textsc{-stv}  \\
 % \glt ‘So there’s this one young man, it’s incredible the way he considers himself so good-looking.’ Georgina Nicholas, {The} {Flirt}
    
%\ex 
%     \label{ex:oneida:15b}
%  \glll {tsiʔ} {a·hutataskénhaʔ,} {a·hotiyo·tʌ·,} {ta·huthwatsiláshnyeʔ,}\\
%   tsiʔ aa-hu-\textbf{atat}{}-askenha-ʔ aa-hoti-yotʌ-ʔ t-aa-hu-at-hwatsil-a-shnye-ʔ\\
%  \textsc{comp}    \textsc{opt-3m.pl.a-}\textbf{\textsc{refl}}\textsc{-}fight.over\textsc{-pnc}  \textsc{opt-3m.pl.p-}work\textsc{-pnc} \textsc{dlc-opt-3m.pl.a-semirefl-}family-\textsc{join}{}-look.after-\textsc{stv}  \\
%  \glt ‘(they told them) that they should apply themselves, they should work, they should       look after their families,’ Pearl Cornelius, {Family} {and} {Friends}
    
%\ex   \label{ex:oneida:15c}
%   \glll {{nʌ} {kiʔ} {ok} {aleʔ} {wí·}}  {wahutaténhaneʔ,} {{kátshaʔ} {ok} {nú·}}   {tahuwatínhaneʔ,}\\
%   { } wa-hu-\textbf{atate}{}-nhaʔ-neʔ { } t-a-huwati-nhaʔ-neʔ\\
%    {then again} \textsc{fact-3m.pl.a-}\textbf{\textsc{refl}}\textsc-}hire\textsc{-pnc} somewhere    %\textsc{cisloc-fact-3>3m.pl-}hire-\textsc{pnc}\\
%  \glt ‘and then again they would hire themselves out, someone would hire them somewhere,’ Mercy Doxtator, {All} {About} {Tobacco}
    
%  \ex  \label{ex:oneida:15d}
%  \glll {Aulhá·}  {né·} {thikʌ}  {kóskos}  {yotatunihátyehseʔ}\\
%   { } { } { } { }  yo-\textbf{atat-}uni-hatye-hseʔ\\
%   herself \textsc{assertion} that  pig \textsc{3fz.sg.p-}\textbf{\textsc{refl}}\textsc{-}make\textsc{-prog-hab}  \\
%  \glt ‘And it was her that would turn herself into a pig’ Verland Cornelius, {A} {Witch} {Story}
%  \z 
 %\z

\ea%15
     Reflexive verbs with some additional meaning \\
    \label{ex:oneida:15}
    \ea \label{ex:oneida:15a}
    \glll  {Shayá·tat} {kaʔikʌ} {kʌʔ} {nithoyʌha,} {yah} {kwí·} {teʔwé·ni}      {nihatatnikʌhteleʔ.}   \\ 
     Shayá·tat kaʔikʌ kʌʔ nithoyʌha, yah   kwí· teʔwé·ni ni-h-\textbf{atat-}nikʌhtle-ʔ\\
    {he is one person}  this {young guy}  not  really    {it’s incredible} \textsc{part-3m.sg.a-}\textbf{\textsc{refl}}\textsc{-}be.handsome\textsc{-stv}  \\
  \glt ‘So there’s this one young man, it’s incredible the way he considers himself so good-looking.’ Georgina Nicholas, {The} {Flirt}
    
\ex 
     \label{ex:oneida:15b}
  \glll {tsiʔ} {a·hutataskénhaʔ,} {a·hotiyo·tʌ·,} {ta·huthwatsiláshnyeʔ,}\\
   tsiʔ aa-hu-\textbf{atat}{}-askenha-ʔ aa-hoti-yotʌ-ʔ t-aa-hu-at-hwatsil-a-shnye-ʔ\\
  \textsc{comp}    \textsc{opt-3m.pl.a-}\textbf{\textsc{refl}}\textsc{-}fight.over\textsc{-pnc}  \textsc{opt-3m.pl.p-}work\textsc{-pnc} \textsc{dlc-opt-3m.pl.a-semirefl-}family-\textsc{join}{}-look.after-\textsc{stv}  \\
  \glt ‘(they told them) that they should apply themselves, they should work, they should       look after their families,’ Pearl Cornelius, {Family} {and} {Friends}
    
\ex   \label{ex:oneida:15c}
   \glll {{nʌ} {kiʔ} {ok} {aleʔ} {wí·}}  {wahutaténhaneʔ,} {{kátshaʔ} {ok} {nú·}}   {tahuwatínhaneʔ,}\\
   {nʌ kiʔ ok aleʔ wí·} wa-hu-\textbf{atate}{}-nhaʔ-neʔ {{kátshaʔ} {ok} nú·} t-a-huwati-nhaʔ-neʔ\\
    {then again} \textsc{fact-3m.pl.a-}\textbf{\textsc{refl}}\textsc{-}hire\textsc{-pnc} somewhere    \textsc{cisloc-fact-3>3m.pl-}hire-\textsc{pnc}\\
  \glt ‘and then again they would hire themselves out, someone would hire them somewhere,’ Mercy Doxtator, {All} {About} {Tobacco}
    
  \ex  \label{ex:oneida:15d}
  \glll {Aulhá·}  {né·} {thikʌ}  {kóskos}  {yotatunihátyehseʔ}\\
   Aulhá· né· thikʌ kóskos yo-\textbf{atat-}uni-hatye-hseʔ\\
   herself \textsc{assertion} that  pig \textsc{3fz.sg.p-}\textbf{\textsc{refl}}\textsc{-}make\textsc{-prog-hab}  \\
  \glt ‘And it was her that would turn herself into a pig’ Verland Cornelius, {A} {Witch} {Story}
  \z 
 \z

Finally, reciprocal meaning is expressed with the reflexive plus a prepronominal prefix with diverse functions, the dualic (duplicative) prefix {te-}. (The basic meaning of the dualic/duplicative is usually described as involving ‘twoness’, but its functions are quite diverse; see, for example, \citealt{Lounsbury1953}.) Just like reflexive verbs, verbs that have the reciprocal structure occur with the Agent paradigm of pronominal prefixes. This is shown in the excerpt in \REF{ex:oneida:16}, which includes two instances of the verb {-nʌskw-} ‘steal (from)’. The last verb form in \REF{ex:oneida:16}, without the reflexive, bears the prefix {hak-}, referencing two animate arguments, third person masculine singular and first person singular. The first verb form in \REF{ex:oneida:16} is a reciprocal with both reflexive and dualic prefixes; it is inflected with the first person exclusive dual Agent prefix {yaky-}.
%\ea%16
%    \label{ex:oneida:16}
%    Reciprocal verb with the reflexive and dualic prefixes\\
%    \glll {teyakyatatnʌskwas,}... {ókhaleʔ}  {tho} {tehahyakwilotátiʔ} {wahakkʌhanʌskoʔ.} \\
 %   \textbf{te}-\textbf{yaky}-\textbf{atat-}nʌskw-as { }  { } { } wa-\textbf{hak}{}-kʌh-a-nʌsko-ʔ\\
%  \textbf{\textsc{dlc}}\textsc{-}\textbf{\textsc{1excl.pl.a}}\textsc{-}\textbf{\textsc{refl}}\textsc{-}steal\textsc{-hab}  and    there  he is coming on his tiptoes \textsc{fact-}\textbf{\textsc{3m.sg>1sg}}\textsc{-}blanket-\textsc{join}{}-steal\textsc{-hab}\\
%   \glt ‘we would steal [the blanket] from each other, ..he'd come tiptoeing and steal the blanket from me.’ Pearl Cornelius, {Family} {and} {Friends}
%  \z

\ea%16
    \label{ex:oneida:16}
    Reciprocal verb with the reflexive and dualic prefixes\\
    \glll {teyakyatatnʌskwas,}... {ókhaleʔ}  {tho} {tehahyakwilotátiʔ} {wahakkʌhanʌskoʔ.} \\
    \textbf{te}-\textbf{yaky}-\textbf{atat-}nʌskw-as {ókhaleʔ}  {tho} {tehahyakwilotátiʔ} wa-\textbf{hak}{}-kʌh-a-nʌsko-ʔ\\
  \textbf{\textsc{dlc}}\textsc{-}\textbf{\textsc{1excl.pl.a}}\textsc{-}\textbf{\textsc{refl}}\textsc{-}steal\textsc{-hab}  and    there  he is coming on his tiptoes \textsc{fact-}\textbf{\textsc{3m.sg>1sg}}\textsc{-}blanket-\textsc{join}{}-steal\textsc{-hab}\\
   \glt ‘we would steal [the blanket] from each other, ..he'd come tiptoeing and steal the blanket from me.’ Pearl Cornelius, {Family} {and} {Friends}
  \z


Many verbs can express both reflexive and reciprocal meaning (for example -{atatyaʔtakenha-} ‘help oneself’ and {te-} ... {-atatyaʔtakenha-} ‘help each other’) but some verbs can express only reciprocal meaning (for example {-atatnʌskw-} ‘steal from one another’, {{}-atatlʌʔnha-} ‘get to know one another, become acquainted’, and {-atatkahnle-} ‘look at one another’).\footnote{There is a reflexive verb ‘see one\-self’, {{}-atatkʌ-}, but it is based on a different verb, {{}-kʌ-} ‘see’.}

\section{Semi-reflexive prefix}\label{sec:oneida:4}

The semi-reflexive prefix {-at-/-ate-/-atʌ-/-an-/-al-/-a-} occurs widely in Oneida (the different forms are discussed at the end of this section). The semi-reflexive has a number of functions including use with verbs of grooming, deriving anticausative verbs, and deriving verbs that involve change of position and manner of self-propulsion. These are meanings that are expressed  in some languages by the middle voice. But the semi-reflexive can also change the semantic role of one of the arguments of the verb, and often the result of affixing the semi-reflexive is at least partially unpredictable. These functions are discussed in turn below.

The semi-reflexive is found with most verbs of grooming and body care, including those whose meaning involves the whole body and those that target just a part of it. Many of these verbs have an incorporated noun that denotes the affected body part. The verb form in \REF{ex:oneida:17a} involves adornment of the whole body while the one in \REF{ex:oneida:17b} is directed just at teeth. Additional grooming verbs are listed in \REF{ex:oneida:17c}.
      

\ea%17  
\label{ex:oneida:17}
 Semi-reflexive with grooming verbs\\
    \ea \label{ex:oneida:17a}
    \glll {yakotyaʔtahsluní} \\
     yako-\textbf{at}{}-yaʔt-a-hsluni\\
        \textsc{3fi.p}{}-\textbf{\textsc{semirefl}}{}-body-\textsc{join}{}-dress,prepare[\textsc{stv}]\\
  \glt ‘she is all dressed up’ 
    
    \ex 
     \label{ex:oneida:17b} 
     \glll {yutnawilóhaleheʔ}\\
    yu-\textbf{at}{}-nawil-ohale-heʔ\\
    \textsc{3fi.a}{}-\textbf{\textsc{semirefl}}{}-tooth-wash-\textsc{hab}\\
  \glt ‘she is brushing her teeth’
    
 \ex 
 \label{ex:oneida:17c}
 {-atewyʌʔt-} ‘fix, put away, take care of’, {-atatewyʌʔt-} ‘make oneself presentable’ {-hsluni-} ‘dress someone’, {-atsluni-} ‘get dressed’, {{}-kustuʔlhyaʔk-} ‘cut a beard, shave someone’, {{}-atkustuʔlhyaʔk-} ‘shave oneself’, {{}-nathalho-} ‘comb someone’s hair’, {-atnathalho-} ‘comb one’s (own) hair’, {{}-wisklalho-} ‘smear with white’, 
 {{}-atwisklalho-} ‘put face powder on’
    \z
\z
         

The semi-reflexive derives anticausatives; some derived anticausative verbs are listed in \REF{ex:oneida:18}. The verbs in \REF{ex:oneida:19} represent a sizeable cohort of derived stems with both the semi-reflexive prefix and an overt causative suffix {-t-/-ht-/-ʔt-/-st-}. However, with these stems, a canonical causative meaning cannot always be discerned, and furthermore the result of affixing the semi-reflexive can be unpredictable. In other words, while the verbs in \REF{ex:oneida:17} are relatively transparent anticausatives, the verbs in \REF{ex:oneida:18} are less so. 

\ea%18
    Semi-reflexive derives anticausative verbs\label{ex:oneida:18}\\
    \resizebox{\linewidth}{!}{\begin{tabular}[t]{@{} ll @{}}
    {{}-hyoʔkt-} ‘dull something, make dull’ \&  {{}-athyoʔkt-} ‘become dull’\\
    {{}-kaʔtshyu-} ‘undo’ \&  {{}-atkaʔtshyu-} ‘come undone’\\
    {{}-khahsyu-} ‘separate, divide, share’ \&  {{}-atekhahsyu-} ‘come apart, separate’\\
    {{}-hwanhak-} ‘tie up’ \&  {{}-athwanhak-} ‘get tied up’    \\
    {{}-lanyu-} ‘rub something’ \&  {{}-atlanyu-} ‘rub against’\\
    {{}-laʔnekalu-} ‘burst something’ \&  {{}-atlaʔnekalu-} ‘burst’\\
    {{}-laʔnʌtahsyu-} ‘peel something’ \&  {{}-atlaʔnʌtahsyu-} ‘peel off’\\
    {{}-tenihʌ-} ‘shake something’ \&  {{}-attenihʌ-} ‘flap’
    \end{tabular}}
\z


\ea%19
    Anticausative verbs with a causative suffix and semi-reflexive prefix\label{ex:oneida:19}\\
    \resizebox{\linewidth}{!}{\begin{tabular}[t]{ @{} ll  @{} }
    {{}-ahkatste-}   ‘be tough, endure’     \&    {{}-atahkatstat-} ‘toughen up, make oneself tough’\\
    {{}-anowʌ-}      ‘be a liar’            \&   {{}-atanowʌht-} ‘doubt, not believe’\\
    {{}-ksaʔtaksʌ-}  ‘be a bad child’       \&        {{}-ateksaʔtaksaʔt-} ‘misbehave’\\
    {{}-lakal(ehl)-} ‘for a noise to sound’ \& {{}-atlakalehlast-} ‘make noise’\\
    {{}-lhale-}      ‘be ready, expecting’  \&  {{}-atelhalat-} ‘get (oneself) ready’\\
    {{}-shnole-}     ‘be fast’              \& {{}-ateshnolat-} ‘go fast, do quickly’\\
    {{}-shw-}        ‘smell, get a whiff of’\& {{}-ateshwaht-} ‘smell something’\\
    {{}-ʔniskw-}     ‘be late’              \& {{}-atʌʔniskwaht-} ‘do late, slowly, behind schedule’
    \end{tabular}}
\z

 The semi-reflexive verbs in \REF{ex:oneida:20} and \REF{ex:oneida:21} describe a change in posture or orientation, or have to do with motion in a particular manner. The verbs in \REF{ex:oneida:21} are derived from stative verbs.

\ea%20
    \label{ex:oneida:20}
   Semi-reflexive derives verbs with a change in orientation or manner of motion\\
   {{}-awʌhlat-} ‘put something over something’, {{}-atawʌhlat-} ‘spill over, go over’

    {-awʌlye-} ‘stir something’, {-atawʌlye-} ‘wander, stroll’

    {{}-kalhateny-} ‘turn something over’, {-atkalhateny-} ‘turn around’, 

    {{}-kalhatho-} ‘turn or knock over, plow’, {{}-atkalhatho-} ‘turn or roll over’

    {{}-ketskw-} ‘right something’, {-atketskw-} ‘right oneself, sit up’ 

    {-kwiʔt-} ‘move something’, {-atkwiʔt-} ‘move over’

    {{}-ukoht-} ‘penetrate, force through’, {{}-atukoht-} ‘pass by, continue on’

    {{}-ʔsle-/-iʔsle-} ‘drag something’, {{}-ateʔsle-} ‘crawl’
    \z

          

    

\ea%21
    \label{ex:oneida:21}
    Semi-reflexive verbs derived rom stative verbs\\
   {-haʔkwawelu-} ‘have one’s head back with throat exposed’, {-athaʔkwawelu-} ‘put         one’s head back’

    {{}-nʌtshotalho-} ‘have one’s arm hooked through something, {{}-atnʌtshotalho-} ‘hook       one’s arm (through someone else’s)’

    {{}-utshot-} ‘be kneeling’, {{}-atutshot-} ‘kneel down’

    {{}-ʔnoyot-} ‘be stooped’, {{}-ateʔnoyot-} ‘stoop over (something)’
    \z
   With a significant number of verbs, the semi-reflexive changes the participant role of one of the arguments of the verb in an unpredictable way, or it just derives a verb with a different and unpredictable meaning. Examples are listed in \REF{ex:oneida:22}.

\ea%22
    \label{ex:oneida:22}
    The semi-reflexive derives a verb with unpredictable meaning\\
     {{}-ahlist-} ‘forbid someone’, {{}-atahlist-} ‘put a stop to’

    {{}-hloli-} ‘tell someone something’, {{}-athloli-} ‘talk about someone or something’

    {{}-hninu-} ‘buy’, {{}-atʌhninu-} ‘sell’

    {{}-itʌht-} ‘be poor’, {{}-anitʌht-} ‘plead’

    {{}-khuni-} ‘prepare food, cook’, {{}-atekhuni-} ‘eat a meal’

    {{}-kweny-} ‘beat at, best someone’, {{}-atkweny-} ‘win’

    {{}-liyo-/-lyo-} ‘beat, kill’, {{}-atliyo-} ‘fight’

    {{}-niha-} ‘lend’, {{}-atʌhniha-} ‘borrow’

    {{}-oʔkt-} ‘come to the end of, finish, end’, {{}-atoʔkt-} ‘run out of’

    {{}-nyeht-} ‘send something with someone’, {{}-atʌnyeht-} ‘send someone something’

    {{}-olishʌ-} ‘be out of breath, pant’, {{}-atolishʌ-} ‘rest’

    {{}-tsyʌʔt-} ‘cure someone’, {{}-atetsyʌʔt-} ‘treat someone’

    {{}-nhaʔ-} ‘hire someone, get someone to do something’, {{}-atʌnhaʔ-} ‘hire labour’

    {{}-ʔtshaʔ-} ‘get beaten, stumped’, {{}-atʌʔtshaʔ-} ‘earn’
    \z

          

   

The semi-reflexive has been described by \citet[74]{Lounsbury1953} for Oneida, \citet[237--243]{Woodbury2018} for Onondaga, and \citet[55--58]{Chafe2015} for Seneca. There is an additional function mentioned in these sources that is relevant here, which is to indicate ownership. An example with an English translation that suggests that an entity, in this case ‘shoes’, belongs to the proto-agent is \REF{ex:oneida:23}. The last verb form in \REF{ex:oneida:15b} above, ‘look after one’s family’, also suggests a kind of ownership.

\ea%23
    \label{ex:oneida:23}
    Semi-reflexive and ownership\\
  \glll  {waʔtkaláhtaneʔ} \\
    waʔ-t-k-\textbf{al}-ahtaʔ-neʔ\\
  \textsc{fact-dlc-1sg.a-}\textbf{\textsc{semirefl}}\textsc{-}put.on.shoes\textsc{-pnc} \\
 \glt ‘I put on my shoes’
\z
    
However, possession is not entailed. Often, pragmatically it makes sense to think of the object as belonging to the instigator, but (outside of body parts of course) the entity can belong to someone else, or to no one. The semi-reflexive verb just indicates some sort of physical or perceived proximity. In fact, for many verbs, it would be odd to think of the entity as being owned. The verb form in \REF{ex:oneida:24} was used in the context of the narrator’s grandmother making baskets, which she sold or traded for goods. The same verb ({{}-uni-} ‘make’) occurs in \REF{ex:oneida:25}, with the affected entity expressed externally rather than by an incorporated noun. Here the narrator is talking about her grandmother making her own butter and cheese. In these contexts, it makes little sense to talk of belongings; rather the sense is making baskets herself for her own purpose; or butter and cheese herself, for her and the family’s use.

\ea%24
    \label{ex:oneida:24}
    \glll {{Né·} {s} {kwí·}} {yakolʌʔnhá·u}    {a·yutaʔahslu·ní·}\\
   {{Né·} {s} {kwí·}} {yakolʌʔnhá·u}  aa-yu-\textbf{at}-aʔahsl-uni-ʔ\\
      \textsc{assertion} {she knows how}   \textsc{opt-3fi.a-}\textbf{\textsc{semirefl}}\textsc{-}basket-make\textsc{-pnc} \\
 \glt ‘she really knew how to make baskets.’ Georgina Nicholas, {An} {Oneida} {Childhood}
\z
  

\ea%25
    \label{ex:oneida:25}
  \glll  {{né·} {s} {kwí·} {né·}}  {owistóhsliʔ} {waʔutu·ní·} {kháleʔ} {cheese,} {cottage} {cheese} \\
   {né·} {s} {kwí·} {né·} {owistóhsliʔ} waʔ-yu-\textbf{at}-uni-ʔ {kháleʔ} {cheese,} {cottage} {cheese} \\
    \textsc{assertion} butter \textsc{fact-3fi.a-}\textbf{\textsc{semirefl}}-make\textsc{-pnc} and cheese, cottage cheese\\
    \glt   ‘she made butter, and cheese, cottage cheese.’ Verland Cornelius, {A} {Lifetime} {of} {Memories}
\z

\todo{In all these examples - the titles- are these books? Corpora?}
This section ends with a brief description of the distribution of the different forms of the semi-reflexive, {-at-/-ate-/-atʌ-/-an-/-al-/-a-}, and the overlap with the forms of the re\-flexive, {{}-atat-/-atate-}. As already mentioned, the reflexive is {-atate-} when adding {{}-atat-} to the stem would result in a sequence of consonants that is not permitted in Oneida. Similarly, the semi-reflexive form {-ate-} occurs if otherwise a non-occurring cluster would result. If there were no other semi-reflexive realizations, the reflexive would simply constitute a sequence of two semi-reflexives. However, the semi-reflexive does have additional forms, and the distribution of the forms is only partly phonological: {-atʌ-} (mainly before stems that begin in {n} or {hn}), {-al-} (before lexically-specified roots that begin in the vowel {a}), {-an-} (before lexically-specified roots that begin in the vowel {i}), and {-a-} (before a few lexically-specified roots beginning in {n} or {ʔn}). The same stem can occur predictably with the {-atat-} or {{}-atate-} reflexive but select a semi-reflexive form that is not {{}-at-} or {{}-ate-}. For example, the verb {-hninu-} ‘buy (from)’ occurs with the semi-reflexive {-atʌ-} in {{}-atʌhninu-} ‘sell’, listed in \REF{ex:oneida:22} above, but with the reflexive {{}-atat-} (see \REF{ex:oneida:26} below). Another example is {{}-nhaʔ-} ‘hire someone’, {{}-atʌnhaʔ-} ‘hire labour’ with the semi-reflexive {{}-atʌ-} in \REF{ex:oneida:22}, and {{}-atatenhaʔ-} ‘hire oneself out’ with the reflexive {{}-atate-} in \REF{ex:oneida:15c}.

\section{Semantic roles}\label{sec:oneida:5}

This section is a discussion of pairs of participant roles other than canonical proto-agent and proto-patient that can be coindexed in Oneida, as well as some participant roles that require or allow a reflexive structure in some languages but do not involve the (semi-) reflexive prefixes in Oneida.

A relatively productive suffix in Oneida is the benefactive-applicative, and stems with this suffix can be prefixed with the reflexive to derive stems with arguments that are coindexed, as in the excerpt in \REF{ex:oneida:26}. Other benefactive verbs are {{}-atatlihunyʌni-} ‘teach one\-self’ (literally, make the matter for oneself) and {{}-atatyoʔtʌhse-} ‘work for oneself’. (There are several forms of the benefactive suffix, some phonologically unrelated; for example {{}-ʌni-} and {{}-hs(e)-}.)

\ea%26
    \label{ex:oneida:26}
     Reflexive with the benefactive\\
  \glll  {né·} {tsiʔ}  {í·} {akhwístaʔ}  {wá·katsteʔ}  {waʔkatathninúnyuhseʔ}  {tsyoʔk} {nahté·shuʔ,}\\
  {né·} {tsiʔ} {í·} ak-hwist-aʔ  waʔ-k-atst-eʔ waʔ-k-\textbf{atat}-hninu-nyu-hs-eʔ {tsyoʔk} {nahté·shuʔ,}\\
  because  \textsc{first} \textsc{person}  \textsc{1sg.poss}-money\textsc{-nsf}  \textsc{fact-1sg.a}-use\textsc{-pnc} \textsc{fact-1sg.a-}\textbf{\textsc{refl}}-buy\textsc{-distr-ben-pnc}      all kinds of things\\
 \glt ‘because I used my money to buy all these things for myself,’ Norma Kennedy, {My}   {First} {Job} {in} {Tobacco}
  \z

Interestingly while verbs whose meaning includes a benefactive argument are quite productive with the reflexive, verbs whose meaning includes a recipient seem to be unat\-tested. For example, in Oneida, one cannot give or send something to oneself; but one can give things to one another, as with the reciprocal of the verb {{}-awi-/-u-} ‘give’ in \REF{ex:oneida:27a}. For ‘talk to oneself’ a speaker provided the circumlocution in \REF{ex:oneida:27b}. Here, a form of the em\-phatic pronoun {{}-ulhaʔ-} (see additional examples in \REF{ex:oneida:11}--\REF{ex:oneida:12}) meaning ‘I am all alone’ is followed by a verb that asserts I am talking; indeed this is perhaps a more careful interpre\-tation of what it means to say ‘talk to oneself’, namely, ‘there I am all alone, and {still} (nevertheless) I am talking’.
\ea%27
    \label{ex:oneida:27}
    \ea Reciprocal verb (but no corresponding reflexive)\label{ex:oneida:27a}\\
    \glll {Thoʔnʌ} {ʌhsí·luʔ,} {‘tsyoní·tuʔ}     {tetyatatnawi·lúʔ.}\\
    {Thoʔnʌ} {ʌhsí·luʔ,} {‘tsyoní·tuʔ} \textbf{te-}ty-\textbf{atat}{}-nawil-u-ʔ\\
    {and then} {you will say} beaver      \textbf{\textsc{dlc}}\textsc{{}-1incl.du.a-}\textbf{\textsc{refl}}\textsc{{}-}tooth-give\textsc{{}-pnc}\\
   \glt ‘And then you will say, ``beaver let’s trade teeth{''}'. (Mercy Doxtator, {Beaver, Let’s Trade Teeth})
   \ex \label{ex:oneida:27b}
   \glll {Akulhaʔtsí·waʔ} {tho} {wakéthaleʔ.}\\
         ak-ulhaʔ-tsí·waʔ {tho} wake-thal-eʔ\\
          \textsc{1sg.poss-}self-intensely    there \textsc{1sg.p-}talk,converse\textsc{{}-stv}\\
    \glt ‘I am all alone (and) still I am talking.’ (Spoken by Olive Elm.)    
    \z
\z

%\ea%27
 %   \label{ex:oneida:27}
 %   \ea Reciprocal verb (but no corresponding reflexive)\label{ex:oneida:27a}\\
 %   \glll {Thoʔnʌ} {ʌhsí·luʔ,} {‘tsyoní·tuʔ}     {tetyatatnawi·lúʔ.}\\
 %   { } { } { } \textbf{te-}ty-\textbf{atat}{}-nawil-u-ʔ\\
 %   {and then} {you will say} beaver      %\textbf{\textsc{dlc}}\textsc{{}-1incl.du.a-}\textbf{\textsc{refl}}\textsc{{}-}tooth-give\textsc{{}-pnc}\\
 %  \glt ‘And then you will say, ``beaver let’s trade teeth{''}'. (Mercy Doxtator, {Beaver, Let’s Trade Teeth})
 %  \ex \label{ex:oneida:27b}
 %  \glll {Akulhaʔtsí·waʔ} {tho} {wakéthaleʔ.}\\
 %        ak-ulhaʔ-tsí·waʔ { } wake-thal-eʔ\\
  %        \textsc{1sg.poss-}self-intensely    there \textsc{1sg.p-}talk,converse\textsc{{}-stv}\\
 %   \glt ‘I am all alone (and) still I am talking.’ (Spoken by Olive Elm.)    
 %   \z
%\z
 
There are no special reflexive forms used for possession. Alienably-possessed entities in Oneida can be inflected with possessive prefixes (related to the Patient series of verbal pronominal prefixes) that identify the person, number and gender of the possessor; inalienably-possessed entities take Agent prefixes. (\citealt{KoenigMichelson2019}, 2020 are detailed discussions of possession in Oneida.) The excerpts in \REF{ex:oneida:28a} and \REF{ex:oneida:28b} both have the alienably-possessed form {laohwístaʔ} ‘his money’ with the third person masculine singular possessive prefix {lao-}. These excerpts come from a report about a man who regularly left his money with the owner of the local store. When the man died, his son asked the store owner for the old man’s money, but the store owner denied having the money. In \REF{ex:oneida:28a} the possessor is coindexed with the masculine singular argument of the verb {-atye-} ‘throw’, but in \REF{ex:oneida:28b} the possessor is disjoint from the masculine singular argument of {-hawe-} ‘hold, have’. (Out of context, without mention of an overt possessor, the usual interpretation is that the possessor is coreferential with the coargument of the verb.)

%%   \label{ex:oneida:28}
  
%   \ea \label{ex:oneida:28a}
%   Possession\\
%   \glll {Tho} {s} {yakʌʔ} {nú·} {yehótyehseʔ} {laohwístaʔ,} {la·té·} {latewyʌ·tuheʔ.}\\
%   { } { } { } { } ye-ho-atye-hseʔ \textbf{lao}-hwist-aʔ { } { }\\
   % that’s   habitually   reportedly   where %\textsc{transloc-3m.sg.p-}throw\textsc{-hab}   \textbf{\textsc{3m.sg.poss}}\textsc{-}money\textsc{-nsf} he said  he is saving it\\
%%    \glt ‘That’s where he\textsubscript{i} left his\textsubscript{i} money, he\textsubscript{i} said he\textsubscript{i} was saving it.’ Olive Elm, {The} {Dreamer}
    
  %  \ex 
 %   \label{ex:oneida:28b}
 %   \glll {tsiʔ} {lonúhteʔ} {kʌʔ} {láhaweʔ} {laohwístaʔ} {kʌ·,}\\
%    tsi lo-anuhte-ʔ kʌʔ la-haw-eʔ \textbf{lao}-hwist-aʔ kʌ·,\\
%    \textsc{comp} \textsc{3m.sg.p-}know\textsc{{}-stv} {right there} \textsc{3m.sg.a-}hold,have\textsc{-stv} \textbf{\textsc{3m.sg.poss}}\textsc{-}money\textsc{-nsf}  y’know \\
 %  \glt ‘because he\textsubscript{i} knew he\textsubscript{j} was holding his\textsubscript{i} money right there,’ Olive Elm, {The} {Dreamer}
%\z
%\z

\ea%28
    \label{ex:oneida:28}
  
   \ea \label{ex:oneida:28a}
   Possession\\
   \glll {Tho} {s} {yakʌʔ} {nú·} {yehótyehseʔ} {laohwístaʔ,} {la·té·} {latewyʌ·tuheʔ.}\\
   {Tho} {s} {yakʌʔ} {nú·} ye-ho-atye-hseʔ \textbf{lao}-hwist-aʔ {la·té·} {latewyʌ·tuheʔ.}\\
    that’s   habitually   reportedly   where \textsc{transloc-3m.sg.p-}throw\textsc{-hab}   \textbf{\textsc{3m.sg.poss}}\textsc{-}money\textsc{-nsf} he said  he is saving it\\
    \glt ‘That’s where he\textsubscript{i} left his\textsubscript{i} money, he\textsubscript{i} said he\textsubscript{i} was saving it.’ Olive Elm, {The} {Dreamer}
    
    \ex 
    \label{ex:oneida:28b}
    \glll {tsiʔ} {lonúhteʔ} {kʌʔ} {láhaweʔ} {laohwístaʔ} {kʌ·,}\\
    tsi lo-anuhte-ʔ kʌʔ la-haw-eʔ \textbf{lao}-hwist-aʔ kʌ·,\\
    \textsc{comp} \textsc{3m.sg.p-}know\textsc{{}-stv} {right there} \textsc{3m.sg.a-}hold,have\textsc{-stv} \textbf{\textsc{3m.sg.poss}}\textsc{-}money\textsc{-nsf}  y’know \\
   \glt ‘because he\textsubscript{i} knew he\textsubscript{j} was holding his\textsubscript{i} money right there,’ Olive Elm, {The} {Dreamer}
\z
\z

English-like constructions involving coreference with oblique arguments or coreference with a non-subject (patient) do not occur in Oneida. Equivalents of these English-like constructions are expressed differently in Oneida. The excerpt in \REF{ex:oneida:29} includes a typical locative structure. There are no adpositions in Oneida and the equivalent phrases require a particle specifying a location ({ohnaʔkʌ·shuʔ}  ‘along behind’) and the orientation or movement of the located entity (in this case, someone – an unknown and frightening being – is coming along). The excerpt in \REF{ex:oneida:30} is given here as an example of a typical multi-clausal structure that is used where in English there is coindexing of a non-subject (e.g. ‘they would talk to them about themselves’). Instead of a prepositional phrase (‘about themselves’) Oneida requires a clause; in this case ‘what their life (or lifestyle) should be like’.

%\ea%29
%    \label{ex:oneida:29}
%    Locative clause\\
% \glll   {Né·n} {lothu·té·} {thikʌ} {tsiʔ} {úhkaʔ} {ok} \textbf{{ohnaʔkʌ·shuʔ}} {ta·yʌ·,} \\
% { } lo-athute-ʔ { } { } { } { } { } t-a-yʌ-e-ʔ\\
% \textsc{assertion} \textsc{3m.sg.p-}hear\textsc{-stv} that  \textsc{comp} someone \textbf{along} \textbf{behind}  \textsc{cisloc-fact-3fi.a-}come,go\textsc{{}-pnc}\\
 %   \glt ‘And so he heard someone coming along behind (him),’ Norma Kennedy,   {My}   {Father’s} {Encounter}

  %\z
   
%\ea%30
 %   \label{ex:oneida:30}
 %  Coindexing across clauses\\
%   \glll  {washakotihlo·lí·} {tsiʔ} {na·hotilihoʔtʌhakeʔ}
   %{wahotínyakeʔ,} \\
%    wa-shakoti-hloli-ʔ { } n-aa-hoti-lihw-oʔtʌ-hak-eʔ { } %wa-hoti-nyak-eʔ\\
%\textsc{fact-3>3m.pl-}tell\textsc{-pnc}  \textsc{comp}    %\textsc{part-opt-3m.pl.p-}matter-kind.of\textsc{{}-cont-pnc} when %\textsc{fact-3m.pl.p-}marry\textsc{{}-pnc}\\
%\glt  ‘they would tell them what their life should be like when they got married,’ Hazel   Cornelius, {Starting} {Life} {Together}
 %  \z
  
 \ea%29
    \label{ex:oneida:29}
    Locative clause\\
 \glll   {Né·n} {lothu·té·} {thikʌ} {tsiʔ} {úhkaʔ} {ok} \textbf{{ohnaʔkʌ·shuʔ}} {ta·yʌ·,} \\
 {Né·n} lo-athute-ʔ {thikʌ} {tsiʔ} {úhkaʔ} {ok} \textbf{{ohnaʔkʌ·shuʔ}} t-a-yʌ-e-ʔ\\
 \textsc{assertion} \textsc{3m.sg.p-}hear\textsc{-stv} that  \textsc{comp} someone \textbf{along} \textbf{behind}  \textsc{cisloc-fact-3fi.a-}come,go\textsc{{}-pnc}\\
    \glt ‘And so he heard someone coming along behind (him),’ Norma Kennedy,   {My}   {Father’s} {Encounter}
\z
   
\ea30
    \label{ex:oneida:30}
   Coindexing across clauses\\
   \glll  {washakotihlo·lí·} {tsiʔ} {na·hotilihoʔtʌhakeʔ} {wahotínyakeʔ,} \\
    wa-shakoti-hloli-ʔ { } n-aa-hoti-lihw-oʔtʌ-hak-eʔ {nʌ} wa-hoti-nyak-eʔ\\
\textsc{fact-3>3m.pl-}tell\textsc{-pnc}  \textsc{comp}    \textsc{part-opt-3m.pl.p-}matter-kind.of\textsc{{}-cont-pnc} when \textsc{fact-3m.pl.p-}marry\textsc{{}-pnc}\\
\glt  ‘they would tell them what their life should be like when they got married,’ Hazel   Cornelius, {Starting} {Life} {Together}
   \z  

\section{Conclusion}\label{sec:oneida:6}

Two verbal prefixes in Oneida, the reflexive and the semi-reflexive, function to coindex arguments of the verb. The basic function of the reflexive is to coindex a proto-agent and proto-patient; the dualic prepronominal prefix adds reciprocal meaning. The semi-reflexive is used for verbs of grooming and body care; it also derives anticausatives and meanings expressed by the middle voice in other languages. Both the reflexive and semi-reflexive derive verbs with meanings that cannot be determined simply from combining a coindexing function of the prefixes with the meaning of the verb to which the prefixes are added, and this is especially true of the semi-reflexive. This unpredictability is not surprising for a morphological for\-mation.

\section*{Acknowledgements}

As always, I would like to acknowledge the masterful speakers who contributed to \citet{MichelsonKennedyDoxtator2016}. I am very grateful to Olive Elm for working with me on the intriguing puzzles of her language, and to Jean-Pierre Koenig, Samantha Cornelius, and Hanni Woodbury for valuable comments on earlier versions of this paper. Also, I thank the internal reviewer of my paper for this volume, Maria Khachaturyan, for her careful reading and many useful comments.



    
\section*{Abbreviations}

\begin{tabularx}{.45\textwidth}[t]{lQ}
\textsc{a}        &   agent\\
\textsc{ben}      &   benefactive\\
\textsc{caus}     &    causative\\
\textsc{cisloc}   &    cislocative,\\
\textsc{comp}     &  complementizer\\
\textsc{cont}     &    continuative\\
\textsc{distr}    &   distributive\\
\textsc{dlc}      &   dualic\\
\textsc{du}       &  dual\\            
\textsc{excl}     &    exclusive\\
\textsc{fact}     &    factual mode\\
\textsc{fi}       &    feminine-indefinite\\
\textsc{fut}      &   future mode\\
\textsc{fz}       &    feminine-zoic\\
\textsc{hab}      &   habitual aspect\\
\textsc{incl}     &    inclusive\\
\textsc{join}     &    joiner vowel\\
\textsc{m}        &   masculine\\
\textsc{nsf}      &   noun suffix\\
\textsc{opt}      &   optative\\
\textsc{p}        &   Patient\\
\textsc{part}     &    partitive\\
\textsc{pl}       &    plural\\
\textsc{pnc}      &   punctual aspect\\
\textsc{poss}     &    possessive\\
\textsc{prog}     &    progressive\\
\textsc{semirefl} &  semi-reflexive\\
\textsc{sg}       & singular\\
\textsc{stv}      & stative aspect\\
\textsc{transloc} & translocative\\
\end{tabularx}


{\sloppy\printbibliography[heading=subbibliography,notkeyword=this]}
\end{document}
