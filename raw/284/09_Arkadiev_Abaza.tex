\documentclass[output=paper]{langscibook}

\author{Peter Arkadiev\affiliation{Institute of Slavic Studies of the Russian Academy of Sciences, Russian State University for the Humanities} \lastand Sonia Durneva\affiliation{Higher School of Economics, Moscow}}
\title{Reflexive constructions in Abaza} 
\abstract{In this article we describe reflexivization constructions in Abaza (Northwest Caucasian), a polysynthetic language characterized by consistent head marking and morphological ergativity. Abaza features two dedicated reflexivization markers: (i) the prefix \textit{č{ə}}{}- used to reflexivize the absolutive argument, and (ii) the lexical reflexive based on the noun \textit{qa} ‘head’, which is able to reflexivize arguments of different types. Besides that, coreferentiality of arguments can be expressed by the “doubling” of ordinary person-number prefixes, which is primarily used when an indirect object of a transitive verb is coreferential to its ergative subject. The absolutive reflexive prefix also has such uses as anticausative and autocausative. A possible path of diachronic development of the Abaza system of reflexivization markers is also briefly discussed.}


\IfFileExists{../localcommands.tex}{
  \usepackage{langsci-optional}
\usepackage{langsci-gb4e}
\usepackage{langsci-lgr}

\usepackage{listings}
\lstset{basicstyle=\ttfamily,tabsize=2,breaklines=true}

%added by author
% \usepackage{tipa}
\usepackage{multirow}
\graphicspath{{figures/}}
\usepackage{langsci-branding}

  
\newcommand{\sent}{\enumsentence}
\newcommand{\sents}{\eenumsentence}
\let\citeasnoun\citet

\renewcommand{\lsCoverTitleFont}[1]{\sffamily\addfontfeatures{Scale=MatchUppercase}\fontsize{44pt}{16mm}\selectfont #1}
   
  %% hyphenation points for line breaks
%% Normally, automatic hyphenation in LaTeX is very good
%% If a word is mis-hyphenated, add it to this file
%%
%% add information to TeX file before \begin{document} with:
%% %% hyphenation points for line breaks
%% Normally, automatic hyphenation in LaTeX is very good
%% If a word is mis-hyphenated, add it to this file
%%
%% add information to TeX file before \begin{document} with:
%% %% hyphenation points for line breaks
%% Normally, automatic hyphenation in LaTeX is very good
%% If a word is mis-hyphenated, add it to this file
%%
%% add information to TeX file before \begin{document} with:
%% \include{localhyphenation}
\hyphenation{
affri-ca-te
affri-ca-tes
an-no-tated
com-ple-ments
com-po-si-tio-na-li-ty
non-com-po-si-tio-na-li-ty
Gon-zá-lez
out-side
Ri-chárd
se-man-tics
STREU-SLE
Tie-de-mann
}
\hyphenation{
affri-ca-te
affri-ca-tes
an-no-tated
com-ple-ments
com-po-si-tio-na-li-ty
non-com-po-si-tio-na-li-ty
Gon-zá-lez
out-side
Ri-chárd
se-man-tics
STREU-SLE
Tie-de-mann
}
\hyphenation{
affri-ca-te
affri-ca-tes
an-no-tated
com-ple-ments
com-po-si-tio-na-li-ty
non-com-po-si-tio-na-li-ty
Gon-zá-lez
out-side
Ri-chárd
se-man-tics
STREU-SLE
Tie-de-mann
} 
  \togglepaper[1]%%chapternumber
}{}

\begin{document}
\maketitle 




\section{Introduction}\label{sec:Arkadiev:1}

\subsection{Classification and location of Abaza; sources of data}\label{sec:Arkadiev:1.1}



Abaza (\textit{abaza-b{ə}zsa}, ISO 639-3 abq) belongs to the Northwest Caucasian language family, and together with the closely related Abkhaz, it forms the Abkhaz-Abaza branch of this family. The language is spoken by about 50 thousand people, mainly in the Abazinsky district of the Karachay-Cherkess Republic in the Russian North Caucasus and in Turkey, see the map in \figref{fig:Arkadiev:1}. 



\begin{figure}
\includegraphics[width=\textwidth]{figures/Abaza.png}
\caption{Map of Abaza}
\label{fig:Arkadiev:1}
\end{figure}


In Russia, Abaza enjoys the status of one of the official languages of the Karachay-Cherkess Republic and has a written standard used in press, teaching and books. Despite that, the language is mostly used in colloquial situations and rural environments and is undergoing a constant pressure from Russian. Most if not all speakers of Abaza in Russia are bilingual in Russian, and many are also fluent in Kabardian, the distantly related language of the same family with which Abaza has been in intense contact. The major dialect of Abaza is Tapanta, often considered to be the only “Abaza proper” variant (see the genealogical tree of the Abkhaz-Abaza dialects in \citealt{Chirikba2003}: 14).



The data in our paper mainly comes from the fieldwork conducted in the village Inzhich-Chukun (\textit{j{ə}n{ʒ}’{ə}g’-č{̣}’k{̣}{ʷ}{ə}n}) of the Abazinsky district of the Karachay-Cherkess Republic. The data was collected in 2017–2019 during fieldtrips organized by the National Research University “Higher School of Economics” and the Russian State University for the Humanities (Moscow). Most examples are elicited, but data from a small collection of oral narratives recorded and annotated by the members of our research team as well as from published texts are also used. Published descriptions of Abaza include the grammars by \citet{Genko1955} and \citet{Tabulova1976} (in Russian), a short sketch by \citet{LomtatidzeHewitt1989} and a generative account of certain aspects of morphosyntax by \citet{OHerin2002} (in English).



\subsection{Salient grammatical features}\label{sec:Arkadiev:1.2}
 \subsubsection{Clause structure and polysynthesis}\label{sec:Arkadiev:1.2.1}

Like all languages of the Northwest Caucasian family (see \citealt{ArkadievLander2020}), Abaza is polysynthetic and predominantly morphologically ergative. Its morphosyntax is consistently head-marking on both clausal and phrasal levels, all arguments being indexed by prefixal pronominal markers on verbs, see example \REF{ex:Arkadiev:1}\footnote{Abaza examples are given in the Caucasological transcription rather than in IPA (see \citealt{ArkadievLander2020}: 372–376). The most important divergences from IPA are as follows: ejective consonants are marked by a dot below or above the symbol; palatalization is marked by an apostrophe; \textit{c} = [{ʦ}], \textit{č} = [{ʧ}], \textit{š} = [{ʃ}], \textit{{ʒ}} = [{ʣ}], \textit{{ʒ}} = [{ʤ}], \textit{ž} = [{ʒ}], \textit{s} = [{ɕ}], \textit{z} = [{ʑ}], \textit{c} = [{ʨ}].}, possessed nouns and postpositions, see example \REF{ex:Arkadiev:2}. Overt nominals cross-referenced by pronominal prefixes are optional and do not show any case marking, see examples \REF{ex:Arkadiev:2} and \REF{ex:Arkadiev:3}.



\ea 
\label{ex:Arkadiev:1}
    \gll j-g’-{ʕ}a-s{ə}-r{ə}-m-t-χ-t \\
    \textsc{3sg.n.abs-neg.emp-csl-1sg.io-3pl.erg-neg}{}-give(\textsc{aor)-re-dcl}\\
    \glt ‘They did not give it back to me.’ [textual example]
\z
 



\ea 
\label{ex:Arkadiev:2}
    \gll {h-babuška}  l-pn{ə}  h-{ʕ}a-n-χa-n\\
    \textsc{1pl.io}{}-granny[\textsc{r}] \textsc{3sg.f.io}{}-at \textsc{1pl.abs-csl}{}-\textsc{loc}{}-remain-\textsc{pst}\\
    \glt ‘We remained at our granny’s.’ [textual example]
\z

 
\ea 
\label{ex:Arkadiev:3}
    \gll \textit{ph{ʷ}{ǝ}s-{ḳ}}\textsubscript{i}  \textit{l-sab{ǝ}j}\textsubscript{j}  \textit{d}\textsubscript{j}\textit{{}-{ʕ}a-l}\textsubscript{i}\textit{{}-q-ašt{ǝ}l{ǝ}-n}\\
    woman-\textsc{indf}  \textsc{3sg.f.io}{}-child  \textsc{3sg.h.abs-csl-3sg.f.io-loc}{}-forget-\textsc{pst.dcl}\\
    \glt ‘A woman forgot about her child.’ [textual example]
\z
 
Basic word order tends to be head-final, but this is not strictly so for clauses. In addition to person-number-gender prefixes, verbs are inflected for tense, aspect, mood and finiteness and besides that can include affixes expressing negation, causative, various applicatives, as well as spatial, aspectual, modal and evaluative meanings. Verbal forms heading main and subordinate clauses are in most cases formally distinct, with overt affixes expressing both the independent status of predication and various types of subordination (relativization, nominalization, different types of converbs). The general verbal template is given in \tabref{tab:Arkadiev:1}.


\begin{table}
    \centering
    \begin{tabularx}{0.6\textwidth}{p{1.3cm}p{0.5cm}p{3.8cm}}
    \lsptoprule
      \multirow{4}{*}{endings}  & +7 & subordinators, force   \\
        & +6 & tense, mood\\
        & +5 & negation\\
        & +4 & aspect\\
        \hline
       \multirow{6}{*}{stem}  & +3 & plural \\
        & +2 & event operators\\
        & +1 & directional suffixes\\
        & 0 & root\\
        & -1 & sociative\\
        & -2 & causative\\
        \hline
        & -3 & negation\\
        \hline
       \multirow{7}{*}{preverbs} & -4 & ergative  \\
        & -5 & indirect object\\
        & -6 & locative preverbs \\
        & -7 & directional preverbs \\
        & -8 & applicatives \\
        & -9 & potential \\
        & -10 & repetitive \\
        \hline
        & -11 & subordinators, negation \\
        & -12 & absolutive  \\
      \lspbottomrule  
    \end{tabularx}
    \caption{Verbal template}
    \label{tab:Arkadiev:1}
\end{table}

%\begin{table}
%\begin{tabularx}{\textwidth}{XXXXXXXXXXXXXXXXXXXX} 
%&  & \multicolumn{7}{X}{ preverbs} &  & \multicolumn{6}{X}{ stem} & \multicolumn{4}{X}{ endings}\\
%\lsptoprule
% –12 & –11 & –10 & –9 & –8 & –7 & –6 & –5 & –4 & –3 & –2 & –1 & 0 & & +1 & +2 & +3 & +4 & +5 & +6 & +7\\
 % &  &  &  &  &  &  &  &  &  &  &  &  &  &  & &  &  &  & \\
%\lspbottomrule
%\caption{Verbal template}\label{tab:Arkadiev:1}
%\end{tabularx}
%\end{table}

Abaza shows “omnipredicativity” \citep{Launey2004}, whereby almost any content word, including nouns and adjectives as well as their combinations, can function as a predicate without a copula and be inserted into the regular verbal morphology, cf. example \REF{ex:Arkadiev:4}.


\ea 
\label{ex:Arkadiev:4}
    \gll \textit{sara}  \textit{s-an}  \textbf{\textit{d-ad{ə}g’a-b} }\\
    \textsc{1sg}  \textsc{1sg.io}{}-mother  \textsc{3sg.h.abs}{}-Circassian-\textsc{npst.dcl}\\
    \glt ‘My mother is Circassian.’ [textual example]
\z
 

\subsubsection{Noun phrases}\label{sec:Arkadiev:1.2.2}
Noun phrases in Abaza minimally contain a noun, which can be inflected for number, definiteness, indefiniteness, possession and oblique cases and take modifiers such as demonstrative, possessor, simple or complex numeral, adjectives, other nouns and relative clause. With such modifiers as adjectives, non-referential nouns and simple numerals, the head noun forms the so-called nominal complex — a tightly integrated word-like entity with rigid internal order, which is inflected and modified as a whole, see example \REF{ex:Arkadiev:5}. Other modifiers do not form part of the nominal complex; most notably, the adnominal possessor forms a full noun phrase and is obligatorily cross-referenced by a possessive (= indirect object) prefix, as in example \REF{ex:Arkadiev:6}.


\ea 
\label{ex:Arkadiev:5}
    \gll \textit{a-}[\textit{b{ə}zsa–d{ə}r{ə}-{ʕ}{ʷ}{}-ca–d{ə}w}]\textit{{}-k{ʷ}a}\\
    \textsc{def}{}-language–know-\textsc{nag-plh}–big-\textsc{pl}\\
    \glt ‘the great linguists’ [textual example]
\z
 
\ea 
\label{ex:Arkadiev:6}
    \gll \textbf{\textit{s-an}}  \textbf{\textit{l}}\textit{{}-aš’a}\\
    \textsc{1sg.io}{}-mother  \textsc{3sg.f.io}{}-brother\\
    \glt ‘my mother’s brother’ [textual example]
\z





As said above, noun phrases cross-referenced by person-number-gender prefixes, including verbal core arguments, do not bear any case marking and are optional. Abaza distinguishes singular and plural number and human and non-human gender, with human being further subdivided into masculine and feminine. Gender is reference-based and manifests itself almost exclusively in pronominal markers on verbs and other argument-taking expressions.



\subsubsection{Independent and bound pronouns}\label{sec:Arkadiev:1.2.3}



Abaza has both independent and bound person forms, the two classes being clearly formally related. Independent pronouns are optional and, like other nominals, lack core case marking, while bound person forms distinguish the absolutive and the oblique (=ergative/indirect object) series and are generally obligatory. The two types of person forms are shown in \tabref{tab:Arkadiev:2}.



\begin{table}
\caption{Independent and bound person forms}
\label{tab:Arkadiev:2}
\begin{tabularx}{0.7\textwidth}{XXXX} 
\lsptoprule
& Absolutive & Oblique & Independent\\
\hline
1\textsc{sg} & {s({ə}){}-} & {s({ə}){}-/z-} & {sara}\\
2\textsc{sgm} & {w({ə})-} & {w({ə})-} & {wara}\\
2\textsc{sgf} & {b({ə})-} & {b({ə})-/p{}-} & {bara}\\
3\textsc{sgm} & {d({ə})-} & {j({ə})-} & {jara}\\
3\textsc{sgf} &  & {l({ə})-} & {lara}\\
3\textsc{sgn} & {j({ə}){}-} & {a{}-/na{}-} & {jara}\\
1\textsc{pl} & {h({ə})-} & {h({ə})-/{ʕ}-} & {hara}\\
2\textsc{pl} & {s({ə})-} & {s({ə})-/z-} & {sara}\\
3\textsc{pl} & {j({ə}){}-} & {r({ə})-/d({ə})-} & {dara}\\
\lspbottomrule
\end{tabularx}
\end{table}

Independent third person pronouns shown in \tabref{tab:Arkadiev:2} occur only rarely and are mainly used for emphasis; normally, demonstratives are used in this function. These are shown in \tabref{tab:Arkadiev:3}.



\begin{table}
\begin{tabularx}{0.5\textwidth}{XXX} 
\lsptoprule
& Sg & Pl\\
\hline
proximal & {ar{ə}j} & {arat}\\
medial & {an{ə}j} & {anat}\\
distal & {aw{ə}j} & {awat}\\
\lspbottomrule
\end{tabularx}
\caption{Demonstratives}
\label{tab:Arkadiev:3}
\end{table}

The prefixes of the absolutive series occur in the slot –12 and encode the S argument of intransitive verbs (example \REF{ex:Arkadiev:7a}) and the P argument of transitive verbs (example \REF{ex:Arkadiev:7b}), while the prefixes of the oblique series encode the A argument of transitive verbs in slot –4 (example \REF{ex:Arkadiev:7b}), indirect and applied objects in slots –8, –6 and –5 (example \REF{ex:Arkadiev:7c}), as well as objects of postpositions and adnominal possessors (examples \REF{ex:Arkadiev:2} and \REF{ex:Arkadiev:6} above).



\ea 
\label{ex:Arkadiev:7}
    \ea 
    \label{ex:Arkadiev:7a}
    \gll \textbf{{h}}{{}-bzaza-d}\\
     \textbf{\textsc{1pl.abs}}\textsc{{}-}live(\textsc{aor)-dcl}\\
    \glt ‘We lived.’ [textual example]
    
    \ex 
    \label{ex:Arkadiev:7b}
    \gll  {awaʔa} \textbf{{h{ə}}}{{}-c{̣}a-}\textbf{{d{ə}}}{{}-r-c{̣}a-χ-n{ə}s}\\
    there  \textbf{\textsc{1pl.abs}}\textsc{{}-loc-}\textbf{\textsc{3pl.erg}}\textsc{{}-caus}{}-put-\textsc{re-purp}\\
    \glt ‘so that they bury us there’ [textual example]
    
    \ex 
    \label{ex:Arkadiev:7c}
    \gll  {j-{ʕ}a-}\textbf{{h{ə}}}{{}-r-t{ə}-n}\\
   \textsc{3sg.n.abs-csl-}\textbf{\textsc{1pl.io}}\textsc{{}-3pl.erg}{}-give-\textsc{pst}\\
    \glt ‘They gave it to us.’ [textual example]
    \z
\z
 



Verbal pronominal prefixes are obligatorily overt with one general exception: 3\textsuperscript{rd} person singular non-human and 3\textsuperscript{rd} person plural prefixes of the absolutive series, both looking as \textit{j({ə})}{}-, are usually dropped if the predicate is immediately preceded by the corresponding full noun phrase. Contrast example \REF{ex:Arkadiev:8a}, where the absolutive object follows the verb furnished with an absolutive prefix, with \REF{ex:Arkadiev:8b}, where the prefix {j}{}- is absent in the presence of the immediately preceding absolutive NP.


\ea 
\label{ex:Arkadiev:8}
    \ea 
    \label{ex:Arkadiev:8a}
    \gll {mhamat-g’ar{ə}j} \textbf{{j}}\textsubscript{i}{{}-{ʕ}a-j{ə}-r-t-t{̣}} \textbf{{adg’{ə}l}}\textsubscript{i}\\
     Muhamat-Girey  \textsc{3sg.n.abs-csl-3sg.m.io-3pl.erg-}give(\textsc{aor)-dcl}  land\\
    \glt ‘They gave land to Muhamat-Girey.’ [textual example]
    
    \ex 
    \label{ex:Arkadiev:8b}
    \gll  \textbf{z-za-{ʒ}{ə}{}-k} ʕa-h-χʷʕa-n\\
    cow-one-\textsc{cln-adnum}  \textsc{csl-1pl.erg}-buy-\textsc{pst.dcl}\\
    \glt ‘We had bought one cow.’ [textual example]
    \z
    
\z

\subsubsection{Verb classes, valency and applicatives}\label{sec:Arkadiev:1.2.4}



Abaza verbs can be monovalent, bivalent or polyvalent, and non-monovalent verbs can be transitive, intransitive and inverse %
%
%This term is clumsy but is better than “dative{}-absolutive”.
%.
%4 августа 2020 г. 19:37
(or “oblique-absolutive”). The valency classes are defined by patterns of verbal cross-reference, as shown in \tabref{tab:Arkadiev:4}.



\begin{table}
\begin{tabularx}{\textwidth}{XXXXX} & A-like argument & P-like argument & other arguments & example\\
\lsptoprule
transitive & Erg & Abs & (IO, Appl) & {d{ə}r} ‘know’, 

{t(a)} ‘give’\\
intransitive & Abs & (IO, Appl) & (Appl) & {bzaza} ‘live’, 

{pš{ə}} ‘look at’, 

{cqra{ʕ}a} ‘help’\\

inverse & IO, Appl & Abs & (Appl) & {ma} ‘have’

{q-ašt{ə}l} ‘forget’\\
\lspbottomrule
\end{tabularx}
\caption{Valency classes of verbs}
\label{tab:Arkadiev:4}

\end{table}

Examples (\ref{ex:Arkadiev:9a}--c) illustrate the three verb classes.

\ea 
\label{ex:Arkadiev:9}
    \ea 
    \label{ex:Arkadiev:9a}
    \gll \textbf{{s{ə}}}{{}-}\textbf{{l}}{{}-ba-t{̣}}\\
     \textbf{\textsc{1sg.abs}}\textsc{{}-}\textbf{\textsc{3sg.f.erg}}{}-see(\textsc{aor)-dcl}\\
    \glt  ‘She (Erg) saw me (Abs).’ (transitive)
    
    \ex 
    \label{ex:Arkadiev:9b}
    \gll  \textbf{{s{ə}}}{{}-}\textbf{{l}}{{}-pš{ə}-t{̣}}\\
    \textbf{\textsc{1sg.abs}}\textsc{{}-}\textbf{\textsc{3sg.f.io}}{}-look(\textsc{aor)-dcl}\\
    \glt ‘I (Abs) looked at her (IO).’ (intransitive)
    
    \ex 
    \label{ex:Arkadiev:9c}
    \gll  \textbf{{s{ə}}}{{}-}\textbf{{l}}{{}-q-ašt{ə}l-t{̣}}\\
   \textbf{\textsc{1sg.abs}}\textsc{{}-}\textbf{\textsc{3sg.f.io}}\textsc{{}-loc}{}-forget(\textsc{aor)-dcl}\\
    \glt ‘She (IO) forgot me (Abs) [lit. I got forgotten on her].’ (inverse)
    \z
\z
    



Abaza possesses a rich system of applicative prefixes occurring in slots –8 and –6, which freely combine with verbs of all valency types and introduce indirect objects expressed by personal prefixes immediately preceding the corresponding applicative prefix (see e.g. \citealt{OHerin2001}). Despite being prone to lexicalization, most applicatives are highly productive. Below we provide examples of the benefactive \REF{ex:Arkadiev:10a}, malefactive \REF{ex:Arkadiev:10b}, comitative \REF{ex:Arkadiev:10c}, instrumental \REF{ex:Arkadiev:10d}, and judicative \REF{ex:Arkadiev:10e} applicatives; the latter mostly combines with non-verbal stems and introduces the role of a person evaluating the situation.


\ea 
\label{ex:Arkadiev:10}
    \ea 
    \label{ex:Arkadiev:10a}
    \gll  d-\textbf{s{ə}}-\textbf{{z}}{{}-{ʕ}a-r-g-χ-t{̣}}\\
     \textbf{\textsc{1sg.abs}}\textsc{{}-}\textbf{\textsc{3sg.f.erg}}{}-see(\textsc{aor)-dcl}\\
     \glt ‘They brought him to me.’ [textual example]
    
    \ex 
    \label{ex:Arkadiev:10b}
    \gll  {j-g’{ə}-}\textbf{{j}}{{}-}\textbf{{c{ə}}}{{}-c{̣}a-h-k{̣}-wa-m}\\
    \textsc{3sg.n.abs-neg.emp-}\textbf{\textsc{3sg.m.io}}\textsc{{}-}\textbf{\textsc{mal}}\textsc{{}-loc}:under-\textsc{1pl.erg}{}-hold-\textsc{ipf-neg}\\
    \glt  ‘We do not conceal it from him.’ \citep[184]{Tabulova1976}
    
    \ex 
    \label{ex:Arkadiev:10c}
    \gll  {buχgalter–qada-ta}     {d-}\textbf{{s{ə}}}{{}-}\textbf{{c{ə}}}{{}-n-χ-{ə}j-t{̣}}\\
    accountant[\textsc{r}]–chief-\textsc{adv}  \textsc{3sg.h.abs-}\textbf{\textsc{1sg.io}}\textsc{{}-}\textbf{\textsc{com}}\textsc{{}-loc}{}-work-\textsc{prs-dcl}\\
    \glt ‘She works with me as a chief accountant.’ [textual example] (inverse)
    
    \ex 
    \label{ex:Arkadiev:10d}
    \gll  a-c{ə}r{ʁ}{ʷ}{ə}  a-zerno\textbf  {a}{-}\textbf{la}{}-{ʕ}-ca-r-g-{ə}j-t\\
    \textsc{def}-spade \textsc{def}-corn[\textsc{r}] \textbf{\textsc{3sg.n.io}}\textsc{{}-}\textbf{\textsc{ins}}\textsc{{}-csl-loc}:under-\textsc{3pl.erg}{}-carry-\textsc{prs-dcl}\\
    \glt ‘They gather corn with a spade.’ [textual example]
    
    \ex 
    \label{ex:Arkadiev:10e}
    \gll  {d-}\textbf{{r{ə}-ma}}{{}-λapa-p-ta}  {aχč’a} {g’-j{ə}-r-t-wa-m}\\
   \textsc{3sg.h.abs-3pl.io-jud}{}-expensive\textsc{{}-npst.dcl-adv} money  \textsc{neg.emp-3sg.m.io-3pl.erg}{}-give-\textsc{ipf-neg}\\
    \glt ‘They consider him expensive [lit. he appears expensive to them] and don’t pay him.’ [textual example]
    \z
\z


Besides that, many of the numerous locative prefixes (“preverbs”) occurring in the slot –7 (see e.g. \citealt{Klychev1995}) are also applicatives and introduce indirect objects, consider example \REF{ex:Arkadiev:11} with a preverb meaning ‘behind’.


\ea 
\label{ex:Arkadiev:11}
    \gll šami{ɮ}    č{ə}{ʕ}{ʷ}-ta  d-na-\textbf{{s{ə}}}{{}-}\textbf{š’ta}-l{ə}-n\\
  Shamil  horseman-\textsc{adv}  \textsc{3sg.h.abs-trl-}\textbf{\textsc{1sg.io}}\textsc{{}-}\textbf{\textsc{loc}}\textbf{:behind}{}-go.in-\textsc{pst.dcl}\\
    \glt ‘Shamil followed me on horseback.’ [textual example]
\z
 



\section{Reflexive constructions}\label{sec:Arkadiev:2}

There are two dedicated reflexive constructions in Abaza, one verbal (morphological) and one nominal (lexical). The verbal reflexive construction involves the prefix {č{ə}}{}- occurring in the absolutive slot –12 and limited to the reflexivization of the absolutive argument, as illustrated in example \REF{ex:Arkadiev:12}; it will be discussed in \sectref{sec:Arkadiev:2.1}. The %
%
%reflexive nominal?
%Martin
%4 августа 2020 г. 19:37
%
%
%we decided to stick to our original formulation for the sake of parallelism with “verbal reflexive”
%.
%9 августа 2020 г. 12:27
nominal reflexive construction employs the body-part noun \textit{qa} ‘head’ with a possessor prefix coreferential with the A-like argument of the verb, cf. example \REF{ex:Arkadiev:13}. The %
%
%reflexive nominal? (note that „the nominal reflexive” must be short for a full form, but what would it be?) “the nominal reflexive form”?)
%Martin
%4 августа 2020 г. 19:37
nominal reflexive can be used to reflexivize different syntactic positions, including the absolutive, where it competes with the verbal reflexive prefix. It will be discussed in \sectref{sec:Arkadiev:2.2}. Apart from this, certain types of coreference between arguments can be expressed by the use of the appropriate pronominal prefixes in two distinct slots, as seen in example \REF{ex:Arkadiev:14}; even though this strategy is not restricted to reflexivization, it deserves attention and will be discussed in section \sectref{sec:Arkadiev:2.3}.


\ea 
\label{ex:Arkadiev:12}
    \gll \textbf{{č}}{{}-}\textbf{{h{ə}}}{{}-r-pχ-{ə}w-n}\\
     \textbf{\textsc{rfl.abs}}\textsc{{}-}\textbf{\textsc{1pl.erg}}\textsc{{}-caus}{}-warm-\textsc{ipf-pst}\\
    \glt ‘We were warming ourselves up.’ [textual example]
\z

 
\ea 
\label{ex:Arkadiev:13}
    \gll \textbf{{p-qa}}  \textbf{{b}}{{}-a-ps{ə}}\\
    \textbf{\textsc{2sg.f.io}}\textbf{{}-head}  \textbf{\textsc{2sg.f.abs}}\textsc{{}-3sg.n.io}{}-look(\textsc{imp)}\\
    \glt‘Look at yourself!’ (said to a woman)
\z

\ea 
\label{ex:Arkadiev:14}
    \gll zak{ə}-zak haq{ʷ}{ə} \textbf{{s{ə}}}{{}-c-t{ə}-}\textbf{{z}}{{}-g-{ə}w-š-t{̣}}\\
    one-one stone \textbf{\textsc{2pl.io}}\textsc{{}-com-loc-}\textbf{\textsc{2pl.erg}}{}-carry-\textsc{ipf-fut-dcl}\\
    \glt ‘Each of you will take along (lit. with you) a stone.’ [textual example]
\z



\subsection{Reflexive constructions with the absolutive reflexive prefix}\label{sec:Arkadiev:2.1}



The absolutive reflexive prefix \textit{č{ə}}{}- normally occurs in slot –12 and is used in situations when the absolutive argument is coreferential with some other argument higher in agentivity which is encoded in the usual way. The most common situation of this kind is attested with transitive verbs, where the absolutive reflexive indicates coreference of the ergative agent and the absolutive patient. For transitive verbs, the use of the absolutive reflexive \textit{č{ə}}{}- seems to be fully productive; in particular, extroverted and introverted verbs behave similarly in this respect. Example \REF{ex:Arkadiev:15} shows an extroverted verb ‘injure’ and \REF{ex:Arkadiev:16} shows an introverted verb ‘wash’ \REF{ex:Arkadiev:16}.


\ea 
\label{ex:Arkadiev:15}
    \ea 
    \label{ex:Arkadiev:15a}
    \gll {s{ə}-j-$\chi {ʷ}{ə}$-t{̣}}\\
     \textsc{1sg.abs-3sg.m.erg}{}-injure\textsc{(aor)-dcl}\\
    \glt  ‘He injured me.’
    
    \ex 
    \label{ex:Arkadiev:15b}
    \gll  \textbf{{č{ə}}}{{}-}\textbf{{j}}{{}-$\chi {ʷ}{ə}$-t{̣}}\\
    \textbf{\textsc{rfl.abs}}\textsc{{}-}\textbf{\textsc{3sg.m.erg}}\textsc{{}-}injure(\textsc{aor)-dcl}\\
    \glt ‘He injured himself.’
    \z
\z

\ea 
\label{ex:Arkadiev:16}
    \ea 
    \label{ex:Arkadiev:16a}
    \gll {j{ə}-l-{ʒ}{ʒ}{}-{ə}j-t{̣}}\\
       \textsc{3sg.n.abs-3sg.f.erg}{}-wash-\textsc{prs-dcl}\\
    \glt  ‘She is washing it.’
    
    \ex 
    \label{ex:Arkadiev:16b}
    \gll  \textbf{{c{ə}}}{{}-}\textbf{{l}}{{}-{ʒ}{ʒ}{}-{ə}j-t{̣}}\\
    \textbf{\textsc{rfl.abs}}\textsc{{}-}\textbf{\textsc{3sg.f.erg}}{}-wash-\textsc{prs-dcl}\\
    \glt  ‘She is washing (herself).’
    \z
    
\z
 
 
Importantly, the absolutive reflexive prefix does not render the verb intransitive and hence cannot be regarded as a valency-reducing device. This is evidenced not only by the presence of the ergative prefix in examples \REF{ex:Arkadiev:15b} and \REF{ex:Arkadiev:16b}, but also by the formation of the imperative. Imperative forms of Abaza transitive verbs obligatorily lack the ergative prefix corresponding to the 2\textsuperscript{nd} person singular actor, and this occurs in ordinary transitive \REF{ex:Arkadiev:17a} and reflexive \REF{ex:Arkadiev:17b} constructions alike.


\ea 
\label{ex:Arkadiev:17}
    \ea 
    \label{ex:Arkadiev:17a}
    \gll {a-sab{ə}j  d-{ʒ}{ʒ}a}\\
       \textsc{def}{}-child  \textsc{3sg.h.abs}{}-wash(\textsc{imp})\\
    \glt  ‘Wash the child!’
    
    \ex 
    \label{ex:Arkadiev:17b}
    \gll  \textbf{{č{ə}}}{{}-{ʒ}{ʒ}a}\\
    \textsc{rfl.abs}{}-wash(\textsc{imp})\\
    \glt  ‘Wash yourself!’
    \z
\z
 

The use of the reflexive prefix under coreference of the absolutive with a higher ranking argument is obligatory, as indicated by example \REF{ex:Arkadiev:18a}, where the doubling of the first person prefix results in ungrammaticality, as opposed to example \REF{ex:Arkadiev:18b} with the reflexive prefix, and by example \REF{ex:Arkadiev:18c} showing that the use of the ordinary third person human absolutive prefix is only compatible with a disjoint interpretation.

\ea
\label{ex:Arkadiev:18}
    \ea[*]{  
    \label{ex:Arkadiev:18a}
    \gll \textbf{{s{ə}}}{{}-}\textbf{{z}}{{}-d{ə}r-{ə}j-t{̣}}\\
      \textbf{\textsc{1sg.abs}}\textsc{{}-}\textbf{\textsc{1sg.erg}}{}-know-\textsc{prs-dcl}\\
    \glt   intended: ‘I know myself.’
    }
    
    \ex[]{ 
    \label{ex:Arkadiev:18b}
    \gll \textbf{{č{ə}}}{{}-}\textbf{{z}}{{}-d{ə}r-{ə}j-t{̣}}\\
    \textbf{\textsc{rfl.abs}}\textsc{{}-}\textbf{\textsc{1sg.erg}}{}-know-\textsc{prs-dcl}\\
    \glt   ‘I know myself.’
    }
    
    \ex[]{ 
    \label{ex:Arkadiev:18c}
    \gll  \textbf{{d{ə}}}{{}-}\textbf{{l}}{{}-{ʒ}{ʒ}a-t{̣}}\\
        \textsc{3sg.h.abs-3sg.f.erg}{}-wash\textsc{(aor)-dcl}\\
    \glt   ‘She washed her/him/*herself.’
    }
    \z
\z
    
The absolutive reflexive prefix is also used when the antecedent is an indirect object rather than the ergative. This happens, first, in inverse constructions derived from transitive verbs by means of the potential prefix \textit{z{ə}}{}-, as in example \REF{ex:Arkadiev:19}, and the involuntative prefix \textit{mqa}{}-, as in example \REF{ex:Arkadiev:20}. Both these prefixes induce the shift of the A-like argument from the ergative to the indirect object (cf. \citet[185]{OHerin2002}), see the difference between the transitive construction in (\ref{ex:Arkadiev:19a}, b) and the inverse construction in (\ref{ex:Arkadiev:19c}, d).

\ea 
\label{ex:Arkadiev:19}
    \ea 
    \label{ex:Arkadiev:19a}
    \gll \textit{s{ə}-j-k{̣}{ʷ}aba-t{̣}}\\
    \textsc{1sg.abs-3sg.m.erg}{}-bathe(\textsc{aor)-dcl} \\
    \glt   ‘He bathed me.’
    
    \ex 
    \label{ex:Arkadiev:19b}
    \gll \textbf{{č{ə}}}{{}-j-k{̣}{ʷ}aba-t{̣}}\\
    \textbf{\textsc{rfl.abs}}\textsc{{}-3sg.m.erg}{}-bathe(\textsc{aor)-dcl}\\
    \glt  ‘He bathed [himself].’
    
    \ex 
    \label{ex:Arkadiev:19c}
    \gll  {s{ə}-}\textbf{{j}}{{}-}\textbf{{z{ə}}}{{}-k{̣}{ʷ}aba-t{̣}}\\
       \textsc{1sg.abs-}\textbf{\textsc{3sg.m.io}}\textsc{{}-}\textbf{\textsc{pot}}{}-bathe(\textsc{aor)-dcl}\\
    \glt    ‘He managed to bathe me [lit. I bathed to him].’
    
    \ex 
    \label{ex:Arkadiev:19d}
    \gll   \textbf{{c{ə}-j}}{{}-}\textbf{{z{ə}}}{{}-k{̣}{ʷ}aba-t{̣}}\\
        \textbf{\textsc{rfl.abs-3sg.m.io-pot}}{}-bathe(\textsc{aor)-dcl}\\
    \glt  ‘He managed to bathe [lit. to him bathed himself].’
    \z
\z

 \ea 
\label{ex:Arkadiev:20}
    \ea 
    \label{ex:Arkadiev:20a}
    \gll {s{ə}-}\textbf{{j}}{{}-}\textbf{{mqa}}{{}-χ{ʷ}{ə}-t{̣}}\\
    \textsc{1sg.abs-}\textbf{\textsc{3sg.m.io}}\textsc{{}-}\textbf{\textsc{invol}}{}-injure(\textsc{aor)-dcl}\\
    \glt  ‘He accidentally injured me [lit. I got injured on him].’
    
    \ex 
    \label{ex:Arkadiev:20b}
    \gll \textbf{{c}}{{}-}\textbf{{j{ə}}}{{}-}\textbf{{mqa}}{{}-$\chi {ʷ}$-t{̣}}\\
    \textbf{\textsc{rfl.abs-3sg.m.io-invol}}{}-injure(\textsc{aor)-dcl}\\
    \glt   ‘He accidentally injured himself [lit. on him got injured himself].’
    \z
\z



Second, the absolutive reflexive can be coreferential with an indirect object encoding the causee (original ergative subject) in morphological causatives based on transitive verbs. In such cases two interpretations are possible, with the antecedent being either the original agent (the causee IO), as in example (\ref{ex:Arkadiev:21c}.i) (21c.i) or the new agent (the ergative causer), as in example (\ref{ex:Arkadiev:21c}.ii) and \REF{ex:Arkadiev:22}.

\ea 
\label{ex:Arkadiev:21}
    \ea 
    \label{ex:Arkadiev:21a}
    \gll {j{ə}-z-{ʒ}{ʒ}a-t{̣}}\\
    \textsc{3sg.n.abs-1sg.erg}{}-wash(\textsc{aor)-dcl}\\
    \glt ‘I washed it.’
    
    \ex 
    \label{ex:Arkadiev:21b}
    \gll {j-}\textbf{{s{ə}}}{{}-j-}\textbf{{r{ə}}}{{}-{ʒ}{ʒ}a-t{̣}}\\
   \textsc{3sg.n.abs-}\textbf{\textsc{1sg.io}}\textsc{{}-3sg.m.erg-}\textbf{\textsc{caus}}\textsc{{}-}wash(\textsc{aor)-dcl}\\
    \glt  ‘He made me wash it.’
    
    \ex 
    \label{ex:Arkadiev:21c}
    \gll \textbf{{c}}{{}-}\textbf{{s{ə}-j}}{{}-r{ə}-{ʒ}{ʒ}a-t{̣}}\\
     \textbf{\textsc{rfl.abs-1sg.io-3sg.m.erg}}\textsc{{}-caus}{}-wash(\textsc{aor)-dcl}\\
    \glt   i.  ‘He made me\textsubscript{i} wash (myself\textsubscript{i}).’\\ ii. ‘He\textsubscript{i} made me wash him\textsubscript{i}.’
    \z
\z


 \ea 
    \label{ex:Arkadiev:22}
    \gll zawa{ɮ} a-{ʒ}{ə} \textbf{{č}}{{}-a-}\textbf{{j}}{{}-}\textbf{{r{ə}}}{{}-q{ʷ}ara-χ-t{̣}}\\
    Zawal  \textsc{def}-water  \textsc{rfl.abs-3sg.n.io-3sg.m.erg-caus}{}-strangle-\textsc{re(aor)-dcl}\\
    \glt ‘Zawal drowned himself (lit. he\textsubscript{i} let the water strangle him\textsubscript{i}).’ [textual example]
\z


Third, the absolutive reflexive can occur in non-derived inverse verbs where its antecedent is an experiencer rather than an agent, as in example \REF{ex:Arkadiev:23}.


 \ea 
\label{ex:Arkadiev:23}
    \ea 
    \label{ex:Arkadiev:23a}
    \gll {d-s-c{ə}-ma{ʁ}-p}\\
    \textsc{3sg.h.abs-1sg.io-mal}{}-be.unpleasant-\textsc{npst.dcl}\\
    \glt  ‘I hate him.’
    
    \ex 
    \label{ex:Arkadiev:23b}
    \gll \textbf{{c}}{{}-}\textbf{{s}}{{}-c{ə}-ma{ʁ}-χ-p}\\
    \textbf{    }\textbf{\textsc{rfl.abs}}\textsc{{}-}\textbf{\textsc{1sg.io}}\textsc{{}-mal-}be.unpleasant-\textsc{re-npst.dcl}\\
    \glt    ‘I hate myself.’\footnote{Reflexive constructions of all types can optionally include the refactive suffix {{}-χ} (on its uses in Abaza see \citealt{Panova2019}) serving to reinforce the reflexive meaning. On such uses of refactive markers see \citet{Stoynova2010}.}
\z
\z

Finally, the absolutive reflexive can be used in inverse denominal predicates derived by the judicative applicative \textit{ma}{}-, see example \REF{ex:Arkadiev:24}.

\ea 
\label{ex:Arkadiev:24}
    \ea 
    \label{ex:Arkadiev:24a}
    \gll {d-s{ə}{}-}\textbf{{ma}}{{}-ps {ʒ}a-t{̣}}\\
     \textsc{3sg.h.abs-1sg.io-}\textbf{\textsc{jud}}{}-beautiful(\textsc{aor)-dcl}\\
    \glt  ‘I considered him/her beautiful.’
    
    \ex 
    \label{ex:Arkadiev:24b}
    \gll \textbf{{c} }{{}-}\textbf{{s{ə}}}{{}-ma-ps {ʒ}a-t{̣}}\\
     \textbf{\textsc{rfl.abs-1sg.io}}\textsc{{}-jud}{}-beautiful(\textsc{aor)-dcl}\\
    \glt   ‘I considered myself beautiful.’
    \z
\z    


The absolutive reflexive cannot be used in polyvalent intransitive verbs that encode their A-like argument in the absolutive slot (cf. \REF{ex:Arkadiev:9b} above).

\ea
    \label{ex:Arkadiev:25}
    \gll {č{ə}-l-pš-{ə}j-t{̣}}\\
   \textsc{rfl.abs-3sg.f.io}{}-look-\textsc{prs-dcl}\\
    \glt intended: ‘She looked at herself.’
\z


\subsection{Reflexive constructions with the reflexive pronoun}
\label{sec:Arkadiev:2.2}

The reflexive pronoun (or rather the reflexive noun) in Abaza is based on the noun root \textit{qa} ‘head’ obligatorily furnished with a possessive (indirect object) prefix with the person, number and gender features matching those of the antecedent. The reflexive pronoun itself is cross-referenced by a 3\textsuperscript{rd} person non-human marker in the appropriate slot. Example \REF{ex:Arkadiev:26b} shows the reflexive in the absolutive position, and example \REF{ex:Arkadiev:27b} shows the indirect object reflexive. The corresponding (a) examples feature ordinary nouns in the same syntactic positions. In \REF{ex:Arkadiev:26b} the reflexive pronoun immediately precedes the verb, hence the corresponding absolutive prefix is absent.

\ea 
\label{ex:Arkadiev:26}
    \ea 
    \label{ex:Arkadiev:26a}
    \gll {sara}  {s-an}  {d{ə}-z-ba-t{̣}}\\
      \textsc{1sg}  \textsc{1sg.io}{}-mother  \textsc{3sg.h.abs-1sg.erg}{}-see(\textsc{aor)-dcl}\\
    \glt ‘I saw my mother.’
    
    \ex 
    \label{ex:Arkadiev:26b}
    \gll  {sara} {a-{ʕ}{ʷ}{ə}ga-la} \textbf{{s-qa}}  {z-ba-χ-t{̣}}\\
     \textsc{1sg}  \textsc{def}{}-mirror-\textsc{ins}  \textsc{1sg.io}{}-head  \textsc{1sg.erg}{}-see-\textsc{re(aor)-dcl}\\
    \glt    ‘I saw myself in the mirror.’
\z
\z

    
\ea 
\label{ex:Arkadiev:27}
    \ea 
    \label{ex:Arkadiev:27a}
    \gll {j-an}  {d{ə}-l-c-qra{ʕ}-{ə}j-t{̣}}\\
       \textsc{3sg.m.io}{}-mother  \textsc{3sg.h.abs-3sg.f.io-com}{}-help-\textsc{prs-dcl}\\
    \glt ‘He helps his mother.’
    
    \ex 
    \label{ex:Arkadiev:27b}
    \gll  \textbf{j-qa} {d-}\textbf{{a}}{{}-c-qra{ʕ}a-χ-{ə}j-t{̣}}\\
     \textbf{\textsc{3sg.m.io}}\textbf{-head} \textsc{3sg.h.abs-}\textbf{\textsc{3sg.n.io}}\textsc{{}-com}{}-help-\textsc{re-prs-dcl}\\
    \glt ‘He helps himself.’
\z
\z

With a plural antecedent, the reflexive pronoun can optionally take the plural suffix \textit{k{ʷ}a}, in which case it is cross-referenced by a plural prefix, see examples (\ref{ex:Arkadiev:28a},b).

  
\ea 
\label{ex:Arkadiev:28}
    \ea 
    \label{ex:Arkadiev:28a}
    \gll {hara}  \textbf{h-qa} {j-}\textbf{a}-\textbf{zə}-\textbf{h}-χʷʕa-t\\
       \textsc{1pl} \textbf{\textsc{1pl.io}}\textbf{-head} \textsc{3sg.n.abs-}\textbf{\textsc{3sg.n.io}}\textsc{{}-}\textbf{\textsc{ben}}-\textbf{\textsc{1pl.erg}}-buy(\textsc{aor)-dcl}\\
    \glt 
    
    \ex 
    \label{ex:Arkadiev:28b}
    \gll {hara} \textbf{{h-qa-kʷa}} {jə-}\textbf{r}{{}-}\textbf{{zə}}{{}-}\textbf{{h}}{{}-χ {ʷ}{ʕ}a-t{̣}}\\
     \textsc{1pl}  \textbf{\textsc{1pl.io}}\textbf{{}-head-}\textbf{\textsc{pl}}  \textsc{3sg.n.abs-}\textbf{\textsc{3pl.io}}\textsc{{}-}\textbf{\textsc{ben}}\textsc{{}-}\textbf{\textsc{1pl.erg}}{}-buy(\textsc{aor)-dcl}\\
    \glt a=b ‘We bought it for ourselves.’
\z
\z


The reflexive pronoun is the only reflexivization strategy available for intransitive verbs like ‘look at’ or ‘help’ in examples \REF{ex:Arkadiev:25} and \REF{ex:Arkadiev:27} above, but is used more widely. With transitive verbs, it competes with the verbal reflexive prefix, which seems to be the default option and is especially preferable in those cases when the use of the nominal reflexive may induce a body-part rather than a reflexive interpretation, as seen in examples \REF{ex:Arkadiev:29} and \REF{ex:Arkadiev:30}–-\REF{ex:Arkadiev:31}. 

 
\ea 
\label{ex:Arkadiev:29}
    \ea 
    \label{ex:Arkadiev:29a}
    \gll {d-s{ə}-r-q{ʷ}anc’-{ə}j-t{̣}}\\
       \textsc{3sg.h.abs-1sg.erg-caus-}guilty-\textsc{prs-dcl}\\
    \glt ‘I accuse him/her.’
    
    \ex 
    \label{ex:Arkadiev:29b}
    \gll \textbf{{s-qa}}  \textbf{{s{ə}}}{{}-r-q{ʷ}anč’-{ə}j-{ṭ}}\\
      \textbf{\textsc{1sg.f.io}}\textbf{{}-head}  \textbf{\textsc{1sg.erg}}\textsc{{}-caus}{}-guilty-\textsc{prs-dcl}\\
    \glt ‘I accuse myself.’ / ??‘I accuse my own head’
    
    \ex 
    \label{ex:Arkadiev:29c}
    \gll \textbf{{c}}{{}-}\textbf{{s{ə}}}{{}-r-q{ʷ}anc’-{ə}j-t{̣}}\\
     \textbf{    }\textbf{\textsc{rfl.abs}}\textsc{{}-}\textbf{\textsc{1sg.erg}}\textsc{{}-caus-}guilty-\textsc{prs-dcl}\\
    \glt ‘I accuse myself.’
\z
\z

\ea 
\label{ex:Arkadiev:30}
    \ea 
    \label{ex:Arkadiev:30a}
    \gll \textbf{{c{ə}}}{{}-}\textbf{{l}}{{}-{ʒ}{ʒ}{}-{ə}j-t{̣}}\\
       \textbf{rfl.abs}{}-\textbf{3sg.g.erg}{}-wash-\textsc{prs-dcl}\\
    \glt ‘She is washing (herself).’
    
    \ex 
    \label{ex:Arkadiev:30b}
    \gll \textbf{{l-qa}  }{l-{ʒ}{ʒ}{}-{ə}j-t{̣}}\\
     \textbf{\textsc{3sg.f.io}}\textbf{{}-head}  \textsc{3sg.f.erg}{}-wash-\textsc{prs-dcl}\\
    \glt ‘She is washing her head.’ / *‘She is washing.’
\z
\z

\ea 
\label{ex:Arkadiev:31}
    \ea 
    \label{ex:Arkadiev:31a}
    \gll \textbf{{č}}{{}-a-c{ə}-s-χč’a-t{̣}}\\
      \textbf{\textsc{rfl.abs}}\textsc{{}-3sg.n.io-mal-1sg.erg}{}-protect(\textsc{aor})-\textsc{dcl} \\
    \glt ‘I protected myself from it.’
    
    \ex 
    \label{ex:Arkadiev:31b}
    \gll \textbf{{s-qa}}  {a-c{ə}-s-χč’a-t{̣}}\\
    \textbf{\textsc{1sg.io-}}\textbf{head}  \textsc{3sg.n.io-mal-1sg.erg}{}-protect(\textsc{aor})-\textsc{dcl} \\
    \glt ‘I protected myself / my head from it.’
\z
\z

The nominal reflexive can also be used instead of the verbal reflexive in inverse verbs, cf. example \REF{ex:Arkadiev:32}.

\ea 
    \label{ex:Arkadiev:32}
    \gll \textbf{{s-qa}}  {j-}\textbf{{s{ə}}}{{}-c{ə}-ma{ʁ}-χ-p}\\
   \textbf{\textsc{1sg.io}}\textbf{{}-head}  \textsc{3sg.n.abs-}\textbf{\textsc{1sg.io}}\textsc{{}-mal}{}-be.unpleasant-\textsc{re-npst.dcl}\\
    \glt ‘I hate myself.’
\z


The reflexive pronoun also occurs in the position of indirect or applied object with transitive verbs, where its antecedent is the ergative agent, see examples \REF{ex:Arkadiev:33} and \REF{ex:Arkadiev:34}; as we show in the next section (\sectref{sec:Arkadiev:2.3}), this pattern of coreference can be expressed by mere doubling of pronominal prefixes.

\ea 
\label{ex:Arkadiev:33}
    \ea 
    \label{ex:Arkadiev:33a}
    \gll {sara}  {bara}  {j-b-a-s-h{ʷ}-t{̣}}\\
      \textsc{1sg}  \textsc{2sg.f}  \textsc{3sg.n.abs-2sg.f.io-dat-1sg.erg}{}-say(\textsc{aor)-dcl}\\
    \glt ‘I said it to you (woman).’
    
    \ex 
    \label{ex:Arkadiev:33b}
    \gll aw{ə}j \textbf{{l-qa}}  j-a-l-h{ʷ}-χ-t\\
    \textsc{dist} \textsc{3sg.f.io}{}-head  \textsc{3sg.n.abs-3sg.n.io-3sg.f.erg}{}-say-\textsc{re-dcl}\\
    \glt ‘She said it to herself.’
\z
\z

\ea 
\label{ex:Arkadiev:34}
    \ea 
    \label{ex:Arkadiev:34a}
    \gll {d-b-c{ə}-s-χč’a-t{̣}}\\
      \textsc{3sg.h.abs-2sg.f.io-mal-1sg.erg}{}-protect\textsc{(aor)-dcl}\\
    \glt ‘I protected him from you (woman).’
    
    \ex 
    \label{ex:Arkadiev:34b}
    \gll \textbf{s-qa}  d-a-\textbf{c{ə}-s}-χc’a-t\\
   \textbf{\textsc{1sg.io}}\textbf{{}-head} \textsc{3sg.h.abs-3sg.n.io-}\textbf{\textsc{mal}}\textsc{{}-}\textbf{\textsc{1sg.erg-}}protect(\textsc{aor)-dcl}\\
    \glt ‘I protected him/her from myself.’
\z
\z

Finally, the nominal reflexive can also express coreference with a non-subject argument, e.g. with the absolutive P as in example \REF{ex:Arkadiev:35}, where the nominal reflexive is an applied object.

\ea 
    \label{ex:Arkadiev:35}
    \gll {aslan}  \textbf{{j-qa}} \textbf{{d}}{{}-a-c{ə}-s-χč’a-χ-t{̣}}\\
   Aslan  \textsc{3sg.m.io}{}-head  \textsc{3sg.h.abs-3sg.n.io-mal-1sg.erg}{}-protect-\textsc{re(aor)-dcl}\\
    \glt ‘I protected Aslan from himself.’
\z


The nominal reflexive can co-occur with the verbal reflexive when both the absolutive and the indirect object are coreferential with the ergative participant, as in example \REF{ex:Arkadiev:36}.


\ea 
    \label{ex:Arkadiev:36}
    \gll \textbf{{s-qa}}  \textbf{{č}}{{}-a-c{ə}-}\textbf{{s}}{{}-χč’a-t{̣}}\\
   \textbf{\textsc{1sg.io}}\textbf{{}-head}  \textbf{\textsc{rfl.abs}}\textsc{{}-3sg.n.io-mal-}\textbf{\textsc{1sg.erg}}{}-protect(\textsc{aor)-dcl}\\
    \glt ‘I protected myself from myself.’
\z


The nominal reflexive cannot be used as an intensifier, this function being expressed by (simple or reduplicated) third person pronouns, see \citet{Panova2020}. This is shown in example \REF{ex:Arkadiev:37a}, where the reduplicated third person masculine pronoun {jara} functions as a {self}{}-intensifier, while the use of the reflexive noun in the same position renders the sentence infelicitous \REF{ex:Arkadiev:37b}.


\ea 
\label{ex:Arkadiev:37}
    \ea[ ]{ 
    \label{ex:Arkadiev:37a}
    \gll za{ʒ}g’{ə}j a{}-č’k{ʷ}{ə}n d-g’-p-j{ə}{}-m{}-q{ə}{}-t{̣}, \textbf{{jara{\textasciitilde}jara}}  {j-qa}  {p{ə}-j-q{ə}-χ-t{̣}}  \\
      nobody  \textsc{def}{}-boy  \textsc{3sg.h.abs-neg-loc-3sg.m.erg}{}-cut(\textsc{aor)dcl} \textbf{\textsc{3sg.m{\textasciitilde}intf}}  \textsc{3sg.m.io}{}-head  \textsc{loc-3sg.m.erg}{}-cut-\textsc{re(aor)-dcl}\\
    \glt  ‘Nobody injured the boy, he injured himself.’
    }
    
    \ex[\#]{ 
    \label{ex:Arkadiev:37b}
    \gll ... \textbf{{j-qa}}  aw{ə}j  d-p-na-q{ə}-χ-t\\
    { } \textbf{\textsc{3sg.m.io}}\textbf{{}-head}  \textsc{dist}  \textsc{3sg.h.abs-loc-3sg.n.erg}{}-cut-\textsc{re(aor)-dcl}\\
    \glt ‘his head cut him.’
      }
    \z
\z

The third person pronoun is also used to disambiguate the reflexive and disjoint readings in adpossessive constructions, see example \REF{ex:Arkadiev:38a}; the nominal reflexive is ungrammatical in this position \REF{ex:Arkadiev:38b}.
\ea 
\label{ex:Arkadiev:38}
    \ea[]{
    \label{ex:Arkadiev:38a}
    \gll das{ə}wzlakg’{ə}j \textbf{{jara}}  j-t{ʕ}aca d{ə}-r-z{ə}-nχ-{ə}j-\\
      whoever.it.is  \textsc{3sg.m} \textsc{3sg.m.io}{}-family \textsc{3sg.h.abs-3pl.io-ben}-work-\textsc{prs-dcl}\\
    \glt ‘Everyone works for his own family.’
    }
    
    \ex[*]{
    \label{ex:Arkadiev:38b}
    \gll das{ə}wzlakg’{ə}j \textbf{j-qa}  a-t{ʕ}aca  d{ə}-r-z{ə}-nχ-{ə}j-t\\
    whoever.it.is  \textsc{3sg.m.io-}head  \textsc{3sg.n.io}-family  \textsc{3sg.h.abs-3pl.io-ben}-work-\textsc{prs-dcl}\\
    \glt intended: ‘=a’
    }
    \z
\z


 
The nominal reflexive cannot occur in the position of the subject, i.e. as the ergative argument of transitive verbs, example \REF{ex:Arkadiev:39}, or the absolutive argument of intransitive verbs \REF{ex:Arkadiev:40}.
\ea 
\label{ex:Arkadiev:39}
    \ea[ ]{ 
    \label{ex:Arkadiev:39a}
    \gll a-phʷ{ə}spa\textsubscript{i}  a-ʕʷ{ə}ga  a-pn{ə} \textbf{{l-qa}}\textsubscript{i}  l-ba-χ-{ə}j-d\\
       \textsc{def}{}-girl \textsc{def}{}-mirror \textsc{3sg.n.io}-at  \textsc{3sg.f.io}-head  \textsc{3sg.erg}{}-see-\textsc{re-prs-dcl}\\
    \glt ‘The girl sees herself in the mirror.’ (\citealt{Testelets2017}: ex. 10a)
    }
    
    \ex[\#]{
    \label{ex:Arkadiev:39b}
    \gll \textbf{{l-qa}}\textsubscript{i}  {a-ph{ʷ}{ə}spa}\textsubscript{j/*i}  a-{ʕ}{ʷ}{ə}ga  a-pn{ə}  d-a-ba-χ-{ə}j-d\\
    \textsc{3sg.f.io}{}-head  \textsc{def}{}-girl  \textsc{def}{}-mirror  \textsc{3sg.n.io}{}-at  \textsc{3sg.h.abs-3sg.n.erg}{}-see-\textsc{re-prs-dcl}
\\
    \glt  ‘\#Her head again sees the girl in the mirror.’ (*=a) (\citealt{Testelets2017}: ex. 10b)
    }
    \z
\z

\ea 
\label{ex:Arkadiev:40}
    \ea[]{ 
    \label{ex:Arkadiev:40a}
    \gll \textbf{l-qa} d-a-c-s-{ə}j-t\\
      \textsc{3sg.f.io}-head \textsc{3sg.h.abs-3sg.n.io-mal}-fear-\textsc{prs-dcl}\\
    \glt  ‘She fears herself.’
    }
    
    \ex[*]{ 
    \label{ex:Arkadiev:40b}
    \gll  \textbf{{l-qa}}  {j{ə}-l-c-s-{ə}j-t{̣}}\\
    \textsc{3sg.f.io}-head \textsc{3sg.n.abs-3sg.f.io-mal}{}-fear-\textsc{prs-dcl}\\
    \glt (only \#‘Her head is afraid of her.’)
    }
    \z
\z


Normally the antecedent of the nominal reflexive must belong to the same clause, but some of our consultants allowed examples like \REF{ex:Arkadiev:41} with the matrix subject anteceding a reflexive in a non-finite clause.

\ea 
    \label{ex:Arkadiev:41}
    \gll {aslan}\textsubscript{i}  [{r{ə}wslan}\textsubscript{j}  \textbf{{j}}\textsubscript{i/j}\textbf{{}-}\textbf{{qa}}  {d-a-z-{ʒ}{ə}r{ʕ}{ʷ}{ə}-rn{ə}s}] {j-a-j-h{ʷ}-t{̣}}\\
   Aslan  Ruslan   \textsc{3sg.m.io}{}-head  \textsc{3sg.h.abs-3sg.n.io-ben}{}-listen\textsc{{}-purp} \textsc{3sg.m.io-dat-3sg.m.erg}{}-say(\textsc{aor)-dcl}\\
    \glt ‘Aslan told Ruslan to listen to himself (=Ruslan / \%=Aslan).’
\z



\subsection{Domains not covered by the dedicated reflexive constructions}\label{sec:Arkadiev:2.3}

In addition to the dedicated verbal and nominal reflexives, coreference in Abaza can be %
%
%I cannot see why there is any problem with labelling this way of expressing coreference with this word (for the sake of simplicity when mentioning it more than once)
%Sonia Durneva
%4 августа 2020 г. 19:37
expressed by the use of the same personal prefixes in two distinct slots, which we call “doubling” . In particular, this is the only strategy available for the reflexivization of the adnominal possessor or postpositional object, cf. examples \REF{ex:Arkadiev:42} and \REF{ex:Arkadiev:43}.
\ea 
    \label{ex:Arkadiev:42}
    \gll \textbf{w{ə}}-nb{ʒ}’a{ʕ}{ʷ}-ca-k{ʷ}a z-{ʕ}a-\textbf{{w{ə}}}{{}-m-d-ja}\\
   \textbf{\textsc{2sg.m.io}}{}-friend-\textsc{plh-pl}  \textsc{rel.rsn-csl-}\textbf{\textsc{2sg.m.erg}}\textsc{-neg-}lead-\textsc{qn}\\
    \glt ‘Why didn’t you (man) bring your friends here?’ [textual example] (=Ruslan / \%=Aslan).’
\z

\ea 
    \label{ex:Arkadiev:43}
    \gll \textbf{{j}}{}-pn{ə}  w-a-n-\textbf{{j{ə}}}{{}-r-pχ’a-wa}\\
   \textbf{\textsc{3sg.m.io}}at  \textsc{2sg.m.abs-3sg.n.io-loc-}\textbf{\textsc{3sg.m.erg}}\textsc{-caus}-spend.night-\textsc{ipf}\\
    \glt ‘he lets you (man) spend night at his (place)’ [textual example]
\z


Besides these rather expected cases, doubling of personal prefixes systematically occurs in transitive verbs as well to indicate coreference between the ergative agent and an indirect object. This happens in morphological causatives from transitive verbs (cf. \citealt[188]{Tabulova1976}), see example \REF{ex:Arkadiev:44}. 
\ea 
\label{ex:Arkadiev:44}
    \ea 
    \label{ex:Arkadiev:44a}
    \gll l{ə}-b{ə}zsa-g’{ə}j  h-l{ə}-r-d{ə}r-t\\
      \textsc{3sg.f.io}{}-language-\textsc{add}  \textsc{1pl.io-3sg.f.erg-caus}-know(\textsc{aor)-dcl}\\
    \glt ‘She taught (lit. caused to know) us her language.’ [textual example] 
    
    \ex 
    \label{ex:Arkadiev:44b}
    \gll  {j-{ʕ}a-}\textbf{{s}}{{}-}\textbf{{s{ə}}}{{}-r-d{ə}r-{ə}j-t{̣}}\\
     \textsc{3sg.n.abs-csl-}\textbf{\textsc{1sg.io}}\textsc{{}-}\textbf{\textsc{1sg.erg}}\textsc{{}-caus}{}-know-\textsc{prs-dcl}\\
    \glt ‘I learn it (lit.» I cause myself to know it”).’ \citep[188]{Tabulova1976}
\z
\z

As expected, the coreferent interpretation is obligatory only with the first and second person prefixes, while verb forms with identical third person prefixes may have both coreferential and disjoint interpretations depending on the context, see example \REF{ex:Arkadiev:45}.

\ea 
\label{ex:Arkadiev:45}
    \ea 
    \label{ex:Arkadiev:45a}
    \gll {j-{ʕ}a-j-l{ə}-r-ba-t{̣}}
\\
      \textsc{3sg.n.abs-csl-3sg.m.io-3sg.f.erg-caus}{}-see(\textsc{aor)-dcl}\\
    \glt ‘She showed it to him.’
    
    \ex 
    \label{ex:Arkadiev:45b}
    \gll  {j-{ʕ}a-}\textbf{{l}}{{}-}\textbf{{l{ə}}}{{}-r-ba-t{̣}}\\
     \textsc{3sg.n.abs-csl-}\textbf{\textsc{3sg.f.io}}\textsc{{}-}\textbf{\textsc{3sg.f.erg}}\textsc{{}-caus}{}-see(\textsc{aor)-dcl}\\
    \glt ‘She\textsubscript{i} showed it to her\textsubscript{j}/herself\textsubscript{i}.’
\z
\z

Expression of coreference by doubling of personal prefixes is widespread with applied objects of transitive verbs. It is attested with the comitative, see example \REF{ex:Arkadiev:14} above, benefactive \REF{ex:Arkadiev:46}, malefactive \REF{ex:Arkadiev:47}\footnote{Note that our consultants allow a broader application of this strategy than reported by O’Herin (2001: 490–491), who claims it to be disallowed with benefactive and malefactive.}, as well as with some locative preverbs \REF{ex:Arkadiev:48}.


\ea 
\label{ex:Arkadiev:46}
    \ea 
    \label{ex:Arkadiev:46a}
    \gll {j{ə}-}\textbf{{l}}{{}-}\textbf{{z{ə}}}{{}-w-$\chi {ʷ}{ʕ}$-{ə}j-t{̣}}\\
      \textsc{3sg.n.abs-}\textbf{\textsc{3sg.f.io}}\textsc{{}-}\textbf{\textsc{ben}}\textsc{{}-2sg.m.erg}{}-buy-\textsc{prs-dcl}\\
    \glt ‘You (man) buy it for her.’
    
    \ex 
    \label{ex:Arkadiev:46b}
    \gll  {j{ə}-}\textbf{{w}}{{}-}\textbf{{z{ə}}}{{}-}\textbf{{w}}{{}-$\chi {ʷ}{ʕ}$-{ə}j-t{̣}}\\
    \textsc{3sg.n.abs-}\textbf{\textsc{2sg.m.io}}\textsc{{}-}\textbf{\textsc{ben}}\textsc{{}-}\textbf{\textsc{2sg.m.erg}}{}-buy-\textsc{prs-dcl}\\
    \glt  ‘You (man) buy it for yourself.’
\z
\z

\ea 
\label{ex:Arkadiev:47}
    \ea 
    \label{ex:Arkadiev:47a}
    \gll {d-}\textbf{{a}}{{}-}\textbf{{c}}{{}-a-s-χc’a-t{̣}}\\
      \textsc{3sg.h.abs-}\textbf{\textsc{3sg.n.io}}\textsc{{}-}\textbf{\textsc{mal}}\textsc{-1sg.erg}-protect(\textsc{aor)-dcl}\\
    \glt ‘I protected him/her from it.’
    
    \ex 
    \label{ex:Arkadiev:47b}
    \gll  {d-}\textbf{{s{ə}}}{{}-}\textbf{{c{ə}}}{{}-}\textbf{{s}}{{}-χc’a-t{̣}}\\
    \textsc{3sg.h.abs-}\textbf{\textsc{1sg.io}}\textsc{{}-}\textbf{\textsc{mal}}\textsc{{}-}\textbf{\textsc{1sg.erg}}{}-protect(\textsc{aor)-dcl}\\
    \glt  ‘I protected him/her from myself.’
\z  
\z
 
 \ea 
\label{ex:Arkadiev:48}
    \ea 
    \label{ex:Arkadiev:48a}
    \gll {j-{ʕ}{}-}\textbf{{a}}{{}-}\textbf{{c{̣}a}}{{}-w-c{̣}-{ə}j-t{̣}}\\
       \textsc{3sg.n.abs-csl-}\textbf{\textsc{3sg.n.io}}\textsc{{}-}\textbf{\textsc{loc}}\textbf{:under}{}-\textsc{2sg.m.erg}{}-put-\textsc{prs-dcl}\\
    \glt  ‘You (man) put this under that.’
    
    \ex 
    \label{ex:Arkadiev:48b}
    \gll  {j-{ʕ}a-}\textbf{{w{ə}}}{{}-}\textbf{{c{̣}a}}{{}-}\textbf{{w}}{{}-c{̣}-{ə}j-t{̣}}\\
    \textsc{3sg.n.abs-csl-}\textbf{\textsc{2sg.m.io}}\textsc{{}-}\textbf{\textsc{loc}}\textbf{:under}{}-\textbf{\textsc{2sg.m.erg}}{}-put-\textsc{prs-dcl}\\
    \glt   ‘You (man) put it under yourself.’
\z  
\z


When the semantics allow it, it is possible to combine the doubling strategy with one of the dedicated reflexivization devices, cf. example \REF{ex:Arkadiev:49a} with the verbal reflexive and example \REF{ex:Arkadiev:49b} with the nominal reflexive; cf. also example \REF{ex:Arkadiev:36} above.

 \ea 
\label{ex:Arkadiev:49}
    \ea 
    \label{ex:Arkadiev:49a}
    \gll \textbf{{c}}{{}-}\textbf{{s}}{{}-c{ə}-}\textbf{{s}}{{}-χc’a-t{̣}}\\
       \textbf{\textsc{rfl.abs}}\textsc{{}-}\textbf{\textsc{1sg.io}}\textsc{{}-mal-}\textbf{\textsc{1sg.erg}}\textsc{{}-}protect(\textsc{aor)-dcl}\\
    \glt 
    
    \ex 
    \label{ex:Arkadiev:49b}
    \gll \textbf{{s-qa}}  {j{ə}-}\textbf{{s}}{{}-c{ə}-}\textbf{{s}}{{}-χč’a-t{̣}}\\
    \textbf{\textsc{1sg.io}}\textbf{{}-head}  \textsc{3sg.n.abs-}\textbf{\textsc{1sg.io}}\textsc{{}-mal-}\textbf{\textsc{1sg.erg}}{}-protect(\textsc{aor)-dcl}\\
    \glt   a=b ‘I protected myself from myself.’
\z  
\z

%I fear that this notation will puzzle some readers.
%Martin
%4 августа 2020 г. 19:37
%
%
%I don’t see why. This notation is absolutely normal, moreover, a reader unfamiliar with it can easily infer what is meant by it.
%.
%4 августа 2020 г. 19:37
%
%
%agree
%Sonia Durneva
%4 августа 2020 г. 19:37


A special case of doubling of personal prefixes occurs in constructions involving relative verbal forms, i.e. relative clauses, content questions (see \citealt{Arkadiev2020}) and argument focus constructions. Here a coreferential (or more precisely: covarying, i.e. semantically bound) interpretation is only available if all occurrences of the relevant personal prefix are replaced by the relative prefix \textit{z{ə}}{}- in the same slot (see a discussion in \citealt{OHerin2002}: 264–265). This happens both in verbs with indirect objects, example \REF{ex:Arkadiev:50a}, and in adpossessive constructions, example \REF{ex:Arkadiev:51a}. If the regular personal prefix is used instead of the relative prefix in the lower position, only the disjoint interpretation is possible, cf. examples \REF{ex:Arkadiev:50b} and \REF{ex:Arkadiev:51b}.
 \ea 
\label{ex:Arkadiev:50}
    \ea 
    \label{ex:Arkadiev:50a}
    \gll {aw{ə}j} \textbf{{z-z{ə}}}{{}-r-d{ə}r-wa-z-da?}\\
       \textsc{dist}  \textbf{\textsc{rel.io-rel.erg}}\textsc{{}-caus-}know-\textsc{ipf-pst.nfin-qh}\\
    \glt ‘Who learned (lit. caused oneself to know) it?’
    
    \ex 
    \label{ex:Arkadiev:50b}
    \gll {aw{ə}j} \textbf{{j-z{ə}}}{{}-r-d{ə}r-wa-z-da?}\\
    \textsc{dist}  \textbf{\textsc{3sg.m.io-rel.erg}}\textsc{{}-caus}{}-know-\textsc{ipf-pst.nfin-qh}\\
    \glt ‘Who taught him that?’ / *‘Who learned it?’
\z
\z

\ea
\label{ex:Arkadiev:51}
    \ea[]{ 
    \label{ex:Arkadiev:51a}
    \gll \textbf{{z}}-χa{ɮ}at-k{ʷ}a-la  čə-\textbf{z{ə}}{{}-m-bž’a-χ-wa  d-laga-p} \\
      \textsc{rel.io-}mistake-\textsc{pl-ins} \textsc{rfl.abs-rel.erg-neg-}educate-\textsc{re-ipf} \textsc{3sg.h.abs}{}-fool-\textsc{npst.dcl}\\
    \glt ‘The one\textsubscript{i} who does not learn by his/her\textsubscript{i} own errors is a fool.’
    }
    
    \ex[\#]{ 
    \label{ex:Arkadiev:51b}
    \gll \textbf{{j}}-χa{ɮ}at{̣}-k{ʷ}a-la č{ə}-\textbf{z{ə}}-m-bž’a-χ-wa  d-laga-p\\
    \textsc{3sg.m.io-}mistake-\textsc{pl-ins} \textsc{rfl.abs-rel.erg-neg-}educate-\textsc{re-ipf} \textsc{3sg.h.abs}{}-fool-\textsc{npst.dcl}\\
    \glt  ‘The one\textsubscript{i} who does not learn by his\textsubscript{j} (someone else’s) errors is a fool.’
    }
\z
\z

The distribution of the three types of expression of coreference in Abaza, including two dedicated reflexivization strategies and the doubling of personal prefixes, is shown in \tabref{tab:Arkadiev:5}.



\begin{table}

\begin{tabularx}{\textwidth}{p{3.5cm}XXXXX}
\lsptoprule
strategy & Erg>Abs & IO>Abs & Erg>IO & Abs>IO & X>Poss\\
\hline
verbal reflexive \textit{č{ə}}- & + & + & – & – & –\\
nominal reflexive \textit{qa} & + & + & + & + & –\\
doubling of personal prefixes & – & – & + & – & +\\
\lspbottomrule
\end{tabularx}
\caption{Distribution of reflexivization strategies in Abaza}
\label{tab:Arkadiev:5}
\end{table}


\section{Related functions of the absolutive reflexive prefix}\label{sec:Arkadiev:3}

The verbal reflexive has autocausative and anticausative uses with both controlling animate and non-controlling inanimate subjects. Verbs allowing such a use of reflexive include verbs denoting caused motion, example \REF{ex:Arkadiev:52}, caused change of posture, example \REF{ex:Arkadiev:53}, and certain verbs of caused change of state, example \REF{ex:Arkadiev:54}.
\label{ex:Arkadiev:52}
    \ea 
    \label{ex:Arkadiev:52a}
    \ea
    \gll {sara}  \textbf{{č}}{{}-a-ca-s{ə}-r-pa-t{̣}}\\
      \textsc{1sg}  \textbf{\textsc{rfl.abs}}\textsc{{}-3sg.n.io-loc:}back\textsc{{}-1sg.erg-caus-}turn(\textsc{aor})\textsc{{}-dcl}\\
    \glt  ‘I turned (lit. myself) back.’
    
    \ex 
    \label{ex:Arkadiev:52b}
    \gll {a-fljuger}  \textbf{{č}}{{}-a-ca-na-r-pa-t{̣}}\\
    \textsc{def}{}-vane[\textsc{r}]  \textbf{\textsc{rfl.abs}}\textsc{{}-3sg.n.io-loc:}back\textsc{{}-3sg.n.erg-caus-}turn(\textsc{aor})\textsc{{}-dcl}\\
    \glt  ‘The weather-vane turned (lit. itself).’\z
\z

\ea
\label{ex:Arkadiev:53}
    \ea 
    \label{ex:Arkadiev:53a}
    \gll {nana}  \textbf{{č{ə}}}{{}-na-l{ə}-r-q{ʷ}-t{̣}}\\
      granny  \textbf{\textsc{rfl.abs}}\textsc{{}-trl-3sg.f.erg-caus}{}-bend(\textsc{aor)-dcl}\\
    \glt  ‘Granny bent (to get something from the floor).’ 
    
    \ex 
    \label{ex:Arkadiev:53b}
    \gll {a-c{̣}la} \textbf{{č}}{{}-na-na-r-q{ʷ}-t{̣}}\\
    \textsc{def}{}-tree  \textbf{\textsc{rfl.abs}}\textsc{{}-trl-3sg.n.erg-caus}{}-bend(\textsc{aor)-dcl}\\
    \glt  ‘The tree bent.’ 
\z
\z

\ea
\label{ex:Arkadiev:54}
    \ea 
    \label{ex:Arkadiev:54a}
    \gll {aw{ə}j} \textbf{{č}}{{}-a-k{ʷ}-j{ə}-r-{ʁ}{ʷ}{ʁ}{ʷ}a-t{̣}}\\
     \textsc{dist}  \textbf{\textsc{rfl.abs}}\textsc{{}-3sg.n.io-loc:}on\textsc{{}-3sg.m.erg-caus}{}-straight(\textsc{aor})-\textsc{dcl}\\
    \glt ‘He stretched (lying on a bench).’
    
    \ex 
    \label{ex:Arkadiev:54b}
    \gll {a-napa-k{ʷ}a} \textbf{{č}}{{}-d{ə}-r-{ʁ}{ʷ}{ʁ}{ʷ}a-χ-t{̣}}\\
    \textsc{def}{}-page-\textsc{pl}  \textbf{\textsc{rfl.abs}}\textsc{{}-3pl.erg-caus}{}-straight-\textsc{re(aor)-dcl}\\
    \glt ‘The pages became smooth again (after the book was put under a press).’
\z
\z    

From the data we have, it may appear that most of the verbs that allow such use of the reflexive are morphological causatives, but simplex verbs allow it as well, see examples \REF{ex:Arkadiev:55} and \REF{ex:Arkadiev:56}.
\ea 
    \label{ex:Arkadiev:55}
    \gll {a-q{ə}s-k{ʷ}a} \textbf{{č}}{{}-{ʕ}a-r-t{̣}{ə}-t{̣}}\\
   \textsc{def}{}-window\textsc{{}-pl} \textbf{\textsc{rfl.abs}}\textsc{{}-csl-3pl.erg}{}-open\textsc{(aor)-dcl}\\
    \glt ‘The windows opened.’ \citep[362]{Tugov1967}
\z

\ea 
    \label{ex:Arkadiev:56}
    \gll \textbf{č}-a-d-h-klə-n  z{ə}m{ʕ}{ʷ}a-g’{ə}j\\
   \textbf{\textsc{rfl.abs}}\textsc{{}-3sg.n.io-loc-1pl.erg}{}-gather-\textsc{pst}  all-\textsc{add}\\
    \glt ‘we all gathered there’ [textual example]
\z
 
A less trivial use of the reflexive prefix is attested only in combination with the morphological causative and involves the meaning of simulation or pretence, cf. examples \REF{ex:Arkadiev:57} and \REF{ex:Arkadiev:58}.
\ea 
    \label{ex:Arkadiev:57}
    \gll \textbf{{č{ə}}}{{}-j-}\textbf{{r{ə}}}{{}-g{ʷ}zaza-wa-n}
\\
   \textbf{\textsc{rfl.abs}}\textsc{{}-3sg.m.erg-}\textbf{\textsc{caus}}{}-hurry-\textsc{ipf-pst}\\
    \glt ‘He pretended to be in a hurry.’
\z

\ea 
    \label{ex:Arkadiev:58}
    \gll \textbf{{č}}{{}-j{ə}-}\textbf{{r}}{}-laga-t\\
   \textbf{\textsc{rfl.abs}}\textsc{-3sg.m.erg-}\textbf{\textsc{caus}}{}-fool(\textsc{aor)-dcl}\\
    \glt ‘He pretended to be a fool.’
\z
 

 diversity of reflexivization strategies attested in Abaza and their distribution can be explained as a result of successive cycles of grammaticalization (i.e. \textit{layering}, \citealt{Hopper1991}). The etymology of the absolutive reflexive \textit{č{ə}}{}- is unclear, but comparative data from Abkhaz \citep[77--78]{Hewitt1979} indicates that it goes back to a noun with a possessive prefix incorporated into the absolutive slot of the verb, as shown in example \REF{ex:Arkadiev:59}. 
\ea 
    \label{ex:Arkadiev:59}
    \gll \textbf{{l}}{{}-}\textbf{{č{ə}}}{{}-l-k{̣}{ʷ}aba-jt{̣}}\\
   \textbf{\textsc{3sg.f.io-rfl}}\textsc{{}-3sg.f.erg}{}-bathe(\textsc{aor)-dcl}\\
    \glt ‘She bathed.’ Abkhaz (\citealt{Hewitt1979}: 78, transcription and glossing adapted)
\z




This diachronic process has reached a more advanced stage in Abaza than in Abkhaz and must have started with the absolutive arguments of highly transitive verbs, which is commonly recognized as the most natural reflexive context, see \citet[3]{Faltz1977}; \citet[42-52]{Kemmer1993}; \citet{Haspelmath2008}; \citet[16--17]{Haspelmath2019constraints}, then extending to derived and lexical inverse predicates by analogy.



The nominal reflexive {qa} ‘head’ with a possessive prefix is nothing but a newer instance of the same development. The grammaticalization path from ‘head’ to reflexive is cross-linguistically recurrent (see e.g. \citealt{Schladt2000}; \citealt{HeineKuteva2002}: 168–169; \citealt{EvseevaSalaberri2018}\footnote{It should be noted that the data on Abaza and Abkhaz adduced in these works are erratic and probably all stem from errors in the table given by \citet[108]{Schladt2000} without reference to sources.}) and is common in the languages of the Caucasus, being attested across the Northwest Caucasian family as well as in the Kartvelian languages. The strategy with doubling of pronominal prefixes is probably a vestige of an earlier state with no dedicated reflexive marking, ousted to the periphery of the system when the specialized means of expression emerged.



\section*{Acknowledgments}

This article was prepared as part of the project № 18-05-0014 realized through ‘The National Research University – Higher School of Economics’ Academic Fund Program in 2019–2020 and financed through the Russian Academic Excellence Project 5-100. We thank all our Abaza consultants, especially Dina Usha, for their patience and generosity, as well as Martin Haspelmath, Ekaterina Lyutikova, Maria Polinsky, Nicoletta Puddu and an anonymous reviewer for their useful feedback on the preliminary versions of this paper. All faults and shortcomings remain ours.



\section*{Abbreviations}

\begin{tabularx}{.45\textwidth}{lQ}
\textsc{1} & 1\textsuperscript{st} person\\
\textsc{2} & 2\textsuperscript{nd} person\\
\textsc{3} & 3\textsuperscript{rd} person\\
\textsc{abs} & absolutive\\
\textsc{add} & additive\\
\textsc{adnum} & adnumerative\\
\textsc{adv} & adverbial\\
\textsc{aor} & aorist\\
\textsc{ben} & benefactive\\
\textsc{caus} & causative\\
\textsc{cln} & non-human classifier\\
\textsc{com} & comitative\\
\textsc{csl} & cislocative\\
\textsc{dat} & dative applicative\\
\textsc{dcl} & declarative\\
\textsc{def} & definite\\
\textsc{dist} & distal demonstrative\\
\textsc{emp} & emphatic\\
\textsc{erg} & ergative\\
\textsc{f} & feminine\\
\textsc{fut} & future\\
\textsc{h} & human\\
\textsc{imp} & imperative\\
\textsc{indf} & indefinite\\
\textsc{ins} & instrumental\\
\textsc{intf} & intensification\\
\textsc{invol} & involuntative\\
\end{tabularx}
\begin{tabularx}{.45\textwidth}{lQ}
\textsc{io} & indirect object\\
\textsc{ipf} & imperfective\\
\textsc{jud} & judicative\\
\textsc{loc} & locative applicative\\
\textsc{m} & masculine\\
\textsc{mal} & malefactive\\
\textsc{n} & non-human\\
\textsc{nag} & agent nominal\\
\textsc{neg} & negation\\
\textsc{nfin} & non-finite\\
\textsc{npro} & nominal proform\\
\textsc{npst} & nonpast\\
\textsc{pl} & plural\\
\textsc{plh} & human plural\\
\textsc{pot} & potential\\
\textsc{prs} & present\\
\textsc{pst} & past\\
\textsc{purp} & purposive\\
\textsc{qh} & human interrogative\\
\textsc{qn} & non-human interrogative\\
\textsc{r} & Russian loan\\
\textsc{re} & refactive\\
\textsc{rel} & relativizer\\
\textsc{rfl} & reflexive\\
\textsc{rsn} & reason subordinator\\
\textsc{sg} &  singular\\
\textsc{trl} & translocative.
\end{tabularx}

%\section*{Acknowledgements}
%\citet{Nordhoff2018} is useful for compiling bibliographies.

{\sloppy\printbibliography[heading=subbibliography,notkeyword=this]}

\end{document}
