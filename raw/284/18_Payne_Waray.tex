\documentclass[output=paper]{langscibook}

\author{Thomas E. Payne\affiliation{University of Oregon and SIL International} \lastand Voltaire Q. Oyzon\affiliation{Leyte Normal University}}

\title{Reflexive constructions in Waray} 

\abstract{Waray is an Austronesian language spoken in the Eastern Visayas region of the Philippines. In this paper we argue that reflexive constructions of all types employ a morphologically complex reflexive nominal \textit{kalugaríngon}. This nominal, based on the root \textit{lugaring} 'to self-rely'/'do on ones own', obligatorily expresses the undergoer when actor and undergoer in the same clause are coreferential. It also may refer to locative, and genitive elements within a clause, and elements of dependent clauses (long-distance coreference), when these are coreferential with a qualifying antecedent. Depending on the context, the use of the reflexive nominal as an oblique nominal, genitive nominal, or in long distance coreference may not be required, but rather has a self-intensifying function. Finally, several examples from a large corpus of natural texts are presented and discussed.} 
\IfFileExists{../localcommands.tex}{
 \addbibresource{localbibliography.bib}
 \usepackage{langsci-optional}
\usepackage{langsci-gb4e}
\usepackage{langsci-lgr}

\usepackage{listings}
\lstset{basicstyle=\ttfamily,tabsize=2,breaklines=true}

%added by author
% \usepackage{tipa}
\usepackage{multirow}
\graphicspath{{figures/}}
\usepackage{langsci-branding}

 
\newcommand{\sent}{\enumsentence}
\newcommand{\sents}{\eenumsentence}
\let\citeasnoun\citet

\renewcommand{\lsCoverTitleFont}[1]{\sffamily\addfontfeatures{Scale=MatchUppercase}\fontsize{44pt}{16mm}\selectfont #1}
   
 %% hyphenation points for line breaks
%% Normally, automatic hyphenation in LaTeX is very good
%% If a word is mis-hyphenated, add it to this file
%%
%% add information to TeX file before \begin{document} with:
%% %% hyphenation points for line breaks
%% Normally, automatic hyphenation in LaTeX is very good
%% If a word is mis-hyphenated, add it to this file
%%
%% add information to TeX file before \begin{document} with:
%% %% hyphenation points for line breaks
%% Normally, automatic hyphenation in LaTeX is very good
%% If a word is mis-hyphenated, add it to this file
%%
%% add information to TeX file before \begin{document} with:
%% \include{localhyphenation}
\hyphenation{
affri-ca-te
affri-ca-tes
an-no-tated
com-ple-ments
com-po-si-tio-na-li-ty
non-com-po-si-tio-na-li-ty
Gon-zá-lez
out-side
Ri-chárd
se-man-tics
STREU-SLE
Tie-de-mann
}
\hyphenation{
affri-ca-te
affri-ca-tes
an-no-tated
com-ple-ments
com-po-si-tio-na-li-ty
non-com-po-si-tio-na-li-ty
Gon-zá-lez
out-side
Ri-chárd
se-man-tics
STREU-SLE
Tie-de-mann
}
\hyphenation{
affri-ca-te
affri-ca-tes
an-no-tated
com-ple-ments
com-po-si-tio-na-li-ty
non-com-po-si-tio-na-li-ty
Gon-zá-lez
out-side
Ri-chárd
se-man-tics
STREU-SLE
Tie-de-mann
} 
 \togglepaper[1]%%chapternumber
}{}


%\usepackage{comment}
\begin{document}

\maketitle


\section{Introduction}
\label{sec:Payne:1}
Waray (also called Waray-Waray, Winaray, or Leyte-Samarnon) is the mother tongue and language of wider communication for most inhabitants of the provinces of Samar, Eastern Samar, Leyte and parts of Biliran in the Eastern Visayas region of the Philippines.\footnote{We would like to thank many Waray students, writers and intellectual leaders from various regions who participated in discussions of the data appearing in this paper, including especially Amado Arjay Babon, Firie Jill Ramos, and many others who willingly participated on social media. We would also like to thank Ricardo Nolasco, Doris Payne, Martin Haspelmath, Katarzyna Janic and an anonymous reviewer for comments on earlier drafts. Of course, any and all errors are our own. We also acknowledge and appreciate the support of Leyte Normal University, SIL Philippines, SIL International and the US Fulbright Foundation. The ISO 639-3 code for the language represented in this study is 'war'.}  With over three million speakers, it is the sixth most widely spoken language in the country. Unless otherwise specified, examples appearing in this paper are from Northern Leyte. \figref{fig:Payne:1} illustrates the areas where Waray is primarily spoken, with the variety represented in this paper highlighted.

%insert figure {fig:Payne:1}
Waray is a member of the Greater Central Philippine (GCP) sub-family of the Malayo-Polynesian family, Austronesian phylum (\citet{Blust 1991}). Although we have not systematically investigated reflexive constructions in all GCP languages, deep personal experience with several GCP languages leads us to believe the generalizations presented here are applicable throughout the subfamily.
Nevertheless, specific data and analyses in this paper are applicable only to Waray, and not necessarily to all GCP, much less to all "Philippine type" languages.

The present study is based on native-speaker competence, a large corpus of spoken and written data (3NS Corpora project – hereafter referred to as "the corpus"), published material in Waray, and extensive input from teachers, students, and intellectual leaders throughout the Waray speaking region. 

In this paper we show that reflexivity in Waray is consistently expressed by the nominal reflexivizer \textit{kalugaríngon} 'self'. Agent-patient coreference can sometimes be expressed by simple intransitive constructions, but such examples may have non-reflexive interpretations depending on the context (see examples such as (\ref{ex:Payne:18})b below). 

We observe that such a phonologically "large" and morphologically complex nominal reflexivizer runs counter to the observation that reflexive constructions are usually expressed by phonologically reduced, or in other ways grammaticalized forms (pronouns, clitics or affixes) in the world's languages. We speculate that this counter expectation may be explained by the fact that in Waray traditional culture, doing something "to oneself", "by oneself", "with oneself" or "for oneself" is considered culturally odd, marginal or aberrant. Indeed, ideas expressed by \textit{kalugaríngon} constructions often have socially undesirable connotations that are unexpected given the free English translations.

The outline of the paper is the following. We begin with a brief description of the morphosyntactic typology of Waray (\ref{sec:Payne:2}), and the pronoun system (\ref{sec:Payne:3}). Out-of-context examples of various types of reflexive constructions are presented in (\ref{sec:Payne:4}) through (\ref{sec:Payne:8}). In (\ref{sec:Payne:9}) we speculate on the possible motivations for the nominal character of the Waray reflexive word. In (\ref{sec:Payne:10}) we present and discuss several examples of \textit{kalugaríngon} from a large corpus of natural texts. Our general conclusions are presented in (\ref{sec:Payne:11}). A list of formatting conventions and abbreviations follows, before the references.



\section{Morphosyntactic typology}\label{sec:Payne:2}
Waray exhibits the typical morphological typology of Greater Central Philippine languages, with a high degree of morphological synthesis in verbal predicates, and a much lower degree of synthesis in referential expressions. Referential expressions (nouns and noun phrases) can be derived from multimorphemic verbs, but such morphological complexity is due to the verbal character of such nominalized forms rather than any specifically nominal categories. The following are some preliminary examples illustrating the morphological typology of Waray:

\ea 
\label{ex:Payne:1}
\glll Ini hi Nánay nagpípinamulod.\\
      {} {} {}  na-g-\textsc{red}1-<in>pag-N-pulod\\
       \textsc{demo1} \textsc{abs.p} Mom \textsc{intr.r-del-imperf-<it>inf-dist}{}-cut.wood \\
\glt ‘Mother went about cutting wood.'\citep[72]{Alunan2016}:72
\z

\ea\label{ex:Payne:2}
\glll Nagkapot-kapot  kami  han  am'  mga  kamot.\\
na-g-\textsc{red}2-kapot\\
\textsc{intr.r-del-attn}{}-hold  1\textsc{excl.abs}  \textsc{obl} 1\textsc{excl.gen} \textsc{pl}  hand \\
\glt ‘We playfully held hands.'
\z

Note that in example (\ref{ex:Payne:1}) the verb form consists of a root and six morphological elements, including prefixes, partial reduplication, an infix, and a highly fusional nasal element \textit{N}{}- indicating distributive action. In contrast, the referential expression \textit{ini} \textit{hi} \textit{Nanay} consists of three distinct morphological elements, two free words \textit{ini} 'this' and \textit{Nanay} 'Mom', and a proclitic case marking determiner \textit{hi}. The verb form in (\ref{ex:Payne:1}) illustrates what we have found to be the maximum number of morphemes in a naturally occurring predicative word in the corpus, though more complex, yet grammatical, constructions can be concocted out of context. We find the verbal inflectional system to consist of twelve paradigmatic affixes (represented by \textit{na}{}- 'intransitive realis' in examples (\ref{ex:Payne:1}) and (\ref{ex:Payne:2})). We have also documented thirteen productive and non-paradigmatic stem-forming (or loosely "derivational") verbal elements, including all the other elements in examples (\ref{ex:Payne:1}) and (\ref{ex:Payne:2}) (\citet{OyzonPayneinprep}).

Example (\ref{ex:Payne:2}) illustrates another relatively synthetic verb containing a root and three morphological elements, including full root reduplication expressing what we call "attenuation." The effect of full root reduplication (\textsc{red}2) is that the event is less genuine, less serious or more random than the root alone would imply. The effect in the context from which this example is extracted is reasonably captured by the adverb "playfully" in the English translation. Example (\ref{ex:Payne:2}) also illustrates that even such a central category as nominal plurality (really collectivity) in a referential expression is expressed analytically in Waray, via the particle spelled \textit{mga} (pronounced [máŋa]). It is safe to say that there are no morphologically expressed inflectional categories affecting nouns. All morphological complexity in referential expressions is stem-forming, and most of that is identical to verb morphology, nominalization being a central feature of Waray discourse.

Grammatical transitivity is an important dimension in Waray morphosyntax. Most inflected verbs are explicitly marked as being grammatically intransitive or transitive, as will be clear from the glosses of the inflected verbs appearing in this paper. A grammatically intransitive clause is one that contains an absolutive argument expressing the most affected participant, but no separate controller or starting point. A grammatically transitive clause is one which contains a controller or starting point that is separate from the absolutive argument. The separate controller or starting point is either expressed in the ergative case, or is strongly implied. This grammatical distinction is independent of the semantic (inherent or ontological) transitivity of the verb root. Semantically transitive roots (those that evoke scenes that imply the participation of an undergoer and a separate actor) may be expressed in grammatically transitive or intransitive constructions, depending on discourse-pragmatic considerations. This is the basis of the famous Philippine voice (or ``focus") systems. We will have no more to say about this issue in this paper, but refer interested readers to the extensive literature on Philippine voice systems, most recently  \citet{PayneOyzon2018} and references cited therein.

The syntactic typology of Waray is broadly predicate-initial, prepositional and head marking. Clausal arguments or obliques may occur before the main predicate (an inflected verb or uninflected nominal predicate). There are three cases, absolutive, ergative/genitive and oblique. These are indicated via pronominal form (see \tabref{tab:Payne:1}), or prenominal determiners. In addition to case, the determiners distinguish personal names from all other nouns, and identifiability (comparable, though not identical to definite vs. indefinite categories). Oblique roles are divided between locative (determiner \textit{ha}) and general (determiners \textit{kan} 'personal name,' \textit{han} 'perpetual' and \textit{hin} 'generic'). Justification for these terms, and extensive additional details of Waray morphosyntax are forthcoming in\citet{OyzonPayneinprep} 



\section{The personal pronoun system}\label{sec:Payne:3} 

Personal pronouns in Waray vary for case (absolutive, ergative/genitive, and oblique), person, and number. An inclusive vs. exclusive first-person plural distinction is also made. \tabref{tab:Payne:1} displays the system of personal pronouns:

\begin{table}
\fittable{
	\small
  \begin{tabularx}{\textwidth}{p{0.7cm}p{1cm}p{0.9cm}p{0.9cm} p{0.02cm} p{1cm} p{1.6cm}p{1.5cm}p{1.3cm}}
  \lsptoprule
         & & \multicolumn{2}{c}{Absolutive} & & \multicolumn{3}{c}{Ergative/Genitive}	& \multirow{2}{*}{Oblique}\\
         \cmidrule{3-4}  \cmidrule{6-8}
         Person &  & Enclitic & Full form  & & Enclitic & Post-posed full form & Pre-posed full form\\
         \hline
         \multirow{3}{*}{1} & \textsc{sg} & -- &	ako	& & =ko	& nákon & ákon & ákon/akô\\
       & \textsc{incl} & 	--	& kita &	 & =ta	& náton	& áton	& áton/atô\\
    	& \textsc{excl} & -- &	kami &	& --	 & námon & ámon &	ámon\\
    	\multirow{2}{*}{Comp} & 1\textsc{sg}>2\textsc{sg} & \multicolumn{7}{c}{=ta ikaw}\\
    	& 1\textsc{sg}>2\textsc{pl} & \multicolumn{7}{c}{=ta kamo}\\
    	\multirow{2}{*}{2} & \textsc{sg} &	=ka	& ikaw & & =mo	 & nímo & ímo & ímo\\
    	& \textsc{pl} &	--	& kamo	& & -- & níyo & íyo & íyo\\
    	\multirow{2}{*}{3} & \textsc{sg} & -- &	hiya/siya &	& -- & níya & íya	& íya\\
    	& \textsc{pl} & -- & hira/sira & & -- & níra	& íra & íra\\
    \lspbottomrule
    \end{tabularx}
    }
    \caption{Personal pronouns of Waray}
    \label{tab:Payne:1}
\end{table}




Note that when first-person singular acts on a second person, the enclitic form of the 1\textsuperscript{st} person inclusive plural pronoun =\textit{ta} occurs, rather than the expected =\textit{ko}. This may be seen as a kind of actor-undergoer coreferentiality in that the speaker identifies with the undergoer when the undergoer is second person -{}- as though the speaker is saying 'We (including you) act on you', for example:

\ea\label{ex:Payne:3}
\glll Isusumat ta ikaw kan Nanay.\\
i-\textsc{red}1-sumat\\
\textsc{appl}2-\textsc{incompl}-tell 1\textsc{inc.erg} 2\textsc{sg.abs} \textsc{obl.p} Mom\\
\glt ‘I will tell on you to Mom.' (Lit:  'We (including you) will tell on you to Mom.')
\z

\ea\label{ex:Payne:4}
\glll
Higugmaon ta kamo.\\
higugma-on\\
love-\textsc{tr.ir} 1\textsc{inc.erg} 2\textsc{exc.abs}\\
\glt `I love you all.' (Lit. `We (including you) love you all.') 
\z

This quasi coreferentiality is a common feature of Philippine pronoun systems. In some languages, the components of these sequences have merged to become distinct forms, though in Waray the two parts of each composite form are still pronounced as individual units. 

Note also that there are two or three forms for each category in the genitive pronoun column – a preposed form, a postposed form, and for some categories an enclitic form. Example (\ref{ex:Payne:5}) illustrates the three alternative possessive constructions: 

\ea\label{ex:Payne:5}
\ea
 Enclitic genitive pronoun: \textit{an balay \textbf{ko}} `my house'\\
\ex  Preposed genitive pronoun:\textit{ an \textbf{ákon} balay} `my house'\\
\ex  Postposed genitive pronoun: \textit{an balay \textbf{nákon}} `my house'\\
\z
\z

There are subtle semantic and/or pragmatic distinctions among these three possibilities. These nuances are relevant for the use of the reflexive nominal \textit{kalugaríngon} as discussed in the following sections. 

\section{Basic Reflexive Constructions}\label{sec:Payne:4}

Waray employs the noun \textit{kalugaríngon} 'self' in many situations involving coreference between an actor and some other clause constituent. We consider \textit{kalugaríngon} to be a noun, rather than a pronoun for the following reasons. First, it does not vary morphologically for case, person or number the way pronouns do. Rather, its case is indicated via case-marking determiners, just as with nouns. Second, its person and number are indicated via adnominal genitive pronouns. Third, referring expressions headed by \textit{kalugaríngon} may be modified like nouns in ways that pronouns may not. These properties will be illustrated in the following examples:

Examples (\ref{ex:Payne:6})--(\ref{ex:Payne:9}) illustrate basic actor-undergoer coreferentiality expressed obligatorily with a reflexive construction:

\ea\label{ex:Payne:6}
\glll
Nakità ko an akon kalugaríngon ha salamin.\\
na-kità\\
\textsc{r.spon}-see 1\textsc{sg.erg}  \textsc{abs} 1\textsc{sg.gen} self \textsc{loc} mirror\\
\glt ‘I saw myself in a mirror.'  
\z

\ea\label{ex:Payne:7}
\glll
Nasísina an akon sangkay ha íya kalugaríngon. \\
na-\textsc{red}1-sina\\
\textsc{r.spon-imperf}-hate \textsc{abs} 1\textsc{sg.gen} friend \textsc{loc} 3\textsc{sg.gen} self\\
\glt ‘My friend hates (is angry with) himself.'
\z

\ea\label{ex:Payne:8}
\glll
Gindayaw niya an íya kalugaríngon.\\
<in>g-dayaw  \\
\textsc{<tr.r>del}-praise 3\textsc{sg.erg} \textsc{abs} 3\textsc{sg.gen} self\\
\glt ‘She praised herself.'
\z


\ea\label{ex:Payne:9}
\glll Ginpatay han tawo an íya kalugaríngon.\\
<in>g-patay\\
\textsc{<tr.r>del}-kill \textsc{erg} man \textsc{abs} 3\textsc{sg.gen} self\\
\glt ‘The man killed himself.'
\z


Note that a prenominal genitive pronoun occurs before \textit{kalugaríngon} in all of these examples. This is the dominant pattern for actor-undergoer coreferential reflexive constructions in Waray, and the first to come to mind when inventing examples out of context. Post-nominal and enclitic genitive pronouns are also grammatically possible, but far less common. Out of 323 examples of \textit{kalugaríngon} in the corpus, all but one have an adnominal genitive possessor (ex. (\ref{ex:Payne:43}) below is the exception). Of the 322 examples of possessed \textit{kalugaríngon}, there are five examples of enclitic genitives (see, e.g., examples (\ref{ex:Payne:41}) and (\ref{ex:Payne:42}) below), and no examples of post-posed genitive possessors (either pronominal or full NPs) of the reflexive nominal. In all the examples in this paper, \textit{kalugaríngon} may be replaced by any semantically compatible noun with no other changes in the sentence, e.g., 'I saw my brother in a mirror' (ex. \ref{ex:Payne:6}), 'she praised her teacher' (ex. \ref{ex:Payne:8}), etc. However, for possessed nominals other than \textit{kalugaríngon}, enclitic and post-posed genitive possessors are proportionally more common than they are for \textit{kalugaríngon}. Thus it appears there is an emerging special pattern of genitive possession for \textit{kalugaríngon} that distinguishes it from other nouns. This may be an initial step toward grammaticalization of \textit{kalugaríngon} as a dedicated reflexive pronoun. 

Another nominal property of \textit{kalugaríngon} is that it may be modified in the same way as other nouns. First, it takes the nominal collective marker \textit{mga} to mark plurality, just as common nouns do: \textit{áton} \textit{mga} \textit{kalugaríngon} `ourselves' (see example \ref{ex:Payne:45} below). Second, certain attribute words may occur as attributive modifiers in NPs headed by \textit{kalugaríngon}:

\ea \label{ex:Payne:10}
\ea 
\gll an ákon minimingaw nga kalugaríngon\\
 \textsc{abs} 1\textsc{sg.gen} lonely \textsc{lk} self\\
 \glt `my lonely self'
\ex 
\gll an ákon nasísina nga kalugaríngon\\
\textsc{abs} 1\textsc{sg.gen} angry \textsc{lk} self\\
\glt `my angry self'
\z
\z

None of the 323 examples of \textit{kalugaríngon} found in the corpus for this study have adnominal attributive modifiers, so this phenomenon is clearly uncommon. However, the fact that it is even possible to modify this word distinguishes it from the class of pronouns.

The reflexive nominal \textit{kalugaríngon} is a nominalized form based on the root \textit{lugaring}, meaning roughly 'self-rely, or 'on one's own.' Here are some examples of this root used outside of its common reflexive context:

\ea \label{ex:Payne:11}
\glll Naglúlugaring na ako.\\
na-g-\textsc{red}1-lugaring\\
\textsc{intr.r}-\textsc{del-imperf}-on.own now 1\textsc{sg.abs}\\
\glt ‘I'm living on my own.'
\z

\ea
\label{ex:Payne:12}
\glll{Paglugaring!}\\
pag-lugaring\\
\textsc{inf}-on.own\\
\glt ‘Do it yourself!' 
\z

Example (\ref{ex:Payne:12}) is a basic intransitive imperative construction employing the infinitive marker \textit{pag}\nobreakdash-. This utterance is a mild rebuke to someone, perhaps a child, asking the speaker to do something for them. 

The reflexive nominal is obligatory in an absolutive role (examples (\ref{ex:Payne:6}), (\ref{ex:Payne:8}) and (\ref{ex:Payne:9}) above) when coreferential with the actor of the clause. It is also obligatory when an oblique is coreferential with the actor, as in (\ref{ex:Payne:7}), and the following. In examples (\ref{ex:Payne:13}) and (\ref{ex:Payne:14}), if a simple 3\textsc{sg} pronoun replaces the NP headed by \textit{kalugaríngon}, coreference with the actor is impossible:

%filled out example
\ea
\label{ex:Payne:13}
\glll Ginpadara niya an surat ha íya kalugaríngon.\\
<in>g-pa-dara {} {} {} {} {} {}\\
<\textsc{tr.r}>\textsc{del-caus}-carry 3\textsc{sg.erg} \textsc{abs} letter \textsc{loc} 3\textsc{sg.gen} self\\
\glt'S/he sent the letter to her/himself.' (or 'S/he had someone carry the letter to her/himself').
\z

%filled out example
\ea
\label{ex:Payne:14}
\glll
Nahuwad niya an kape ha íya kalugaríngon.\\
na-huwad { } { } { } { } { } { }\\
\textsc{r.spon}-spill 3\textsc{sg.erg} \textsc{abs} coffee \textsc{loc} 3\textsc{sg.gen} self\\
\glt ‘S/he spilled the coffee on her/himself.'
\z

Note that the verb \textit{huwad} in spontaneous mood is translated as 'spill' in English (example \ref{ex:Payne:14}). The same root in deliberate mood, \textit{ginhuwad}, would be more insightfully translated as 'pour'.

The reflexive nominal does not naturally occur in an ergative role (\ref{ex:Payne:15}a) or in an absolutive role in an intransitive construction (\ref{ex:Payne:15}b): 

%16
\ea
\label{ex:Payne:15}
\ea[*]{
\glll
 Ginpatay han íya kalugaríngon an tawo.\\
 <in>g-patay { } { } { } { } { }\\
 <\textsc{tr.r>del}-kill \textsc{erg} 3\textsc{sg.gen} self \textsc{abs} man.\\
\glt ('*Himself killed the man.')
}

\ex[*]{
\glll
{Nagpatay} {an} {íya} {kalugaríngon} {hin} {tawo.} \\
 na-g-patay { } { } { } { } { } \\
 \textsc{intr.r-del}{}-kill \textsc{abs} 3\textsc{sg.gen} self \textsc{obl.indef} man\\
\glt ('*Himself killed a man.')
}
\z
\z


These constructions, if interpretable at all, are extremely awkward and confusing. In other words, the \textit{actor}, whether ergative or absolutive, cannot reflect a distinct nominal in the clause or elsewhere. This is one property that \citet{Schachter1977} called a "role-related subject property" of Tagalog.

However, an oblique nominal can reflect an actor argument whether the actor is ergative (examples (\ref{ex:Payne:13}) and (\ref{ex:Payne:14}) above) or absolutive in a detransitive (or ``antipassive", \citealt{OyzonPayneinprep}.) construction, as in (\ref{ex:Payne:7}) above, and the following: 

%17
\ea \label{ex:Payne:16}
\glll
{Nagpatay} {an} {tawo} {ha} {íya} {kalugaríngon.}\\
na-g-patay  { } { } { } { } { }\\
\textsc{intr.r-del}{}-kill \textsc{abs} man \textsc{loc} 3\textsc{sg.gen} self\\
\glt ‘Humanity has killed itself,' or 'The man killed himself.'
\z 

Example \ref{ex:Payne:16} is a detransitive version of example (\ref{ex:Payne:9}), but the interpretation may be quite different. In (\ref{ex:Payne:16}) \textit{an} \textit{tawo} can be understood in the generic sense as "humanity." This is consistent with a general tendency for this particular word \textit{tawo} to have a generic sense in certain contexts. This fact is tangential to the notion of reflexivity. It is not the case that all absolutive actors in detransitive reflexive constructions are understood as generic (see, e.g., example (\ref{ex:Payne:7}) above).


\section{Contrast between introverted and extroverted verbs}\label{sec:Payne:5}
Transitive verbs that allow a human object can be divided semantically into introverted and extroverted classes \citep[803]{Haiman1980}. Prototypical extroverted actions express socially antagonistic events such as 'kill', 'kick', 'attack', 'hate' and 'criticize', whereas introverted actions include body care (or grooming) actions such as 'shave', 'comb' and 'bathe'. In Waray, extroverted actions are expressed with inherently transitive verbs, i.e., their underived forms may be used in a transitive frame. Introverted actions, on the other hand, tend to be expressed by inherently intransitive verbs. In an intransitive frame, such verbs tend to be understood as reflexive, even without use of the reflexive nominal. In order to occur in a transitive frame, such verbs require the addition of a valence increasing morphological element.

The examples in (\ref{ex:Payne:17}) and (\ref{ex:Payne:23}) (further below) illustrate extroverted verbs expressed in transitive and detransitive reflexive constructions, in what we are calling "deliberate" (prefix \textit{g}{}-) and "controlled" (infix \nobreakdash-\textit{um-}) moods. Deliberate mood presents a situation as something that the actor goes out of their way to perform. The situation is not something the actor normally does, but is a special, conscious act. Controlled mood depicts situations as being under the control of the actor, but with emphasis on the \textit{effect} of the situation on the absolutive argument (whether the absolutive happens to be the actor or not). Often, events in controlled mood are presented as situations the controller always, naturally, effortlessly or inevitably does. In the following examples, the transitive versions are understood as more harsh, more effective or more intense than the corresponding detransitive versions. Similarly, the deliberate mood detransitives are understood as more intense than the corresponding controlled mood forms:

%numbering in word misplaced
%orig: 18
\ea
\label{ex:Payne:17}
\ea  {Transitive, deliberate mood}\\
\glll Ginkagat han áyam an íya kalugaríngon.\\
 <in>g-kagat { } { } { } { } { }\\
\textsc{<tr.r}>\textsc{del-}bite \textsc{erg} dog \textsc{abs} 3\textsc{sg.gen} self\\
\glt ‘The dog bit itself.'
 
\ex 
{Detransitive, deliberate mood}\\
\glll
 {Nagkagat} {an} {áyam} {\-\-ha} {íya} {kalugaríngon.} \\
 na-g-kagat { } { } { } { } { } \\
 \textsc{intr.r-del}{}-bite \textsc{abs} dog \textsc{loc} 3\textsc{sg.gen} self\\
 \glt ‘The dog nipped at itself.'
 
\ex {Transitive, controlled mood}\\
\glll
 {Kinagat} {han} {áyam} {an} {íya} {kalugaríngon.}\\
 <in>-kagat { } { } { } { } { }\\
 \textsc{tr.r}{}-bite \textsc{erg} dog \textsc{abs} 3\textsc{sg.gen} self\\
 \glt ‘The dog bit/bites itself (as usual).'

\ex 
{Detransitive, controlled mood}\\
\glll
 {Kumágat} {an} {áyam} {\-\-ha} {íya} {kalugaríngon.}\\
 <um>kagat { } { } { } { } { } \\
 <\textsc{intr.r.ctrl}>bite \textsc{abs} dog \textsc{loc} 3\textsc{sg.gen} self\\
 \glt ‘The dog (casually) nips/nipped at itself.'
 \z
 \z

Many introverted verb roots are inherently intransitive, as evidenced by the fact that they may occur in transitive frames only with the addition of causative or applicative morphology (see \citealt{OyzonPayneinprep} for a discussion of verb classes). For example, the root \textit{karigò} 'bathe' may occur in a simple intransitive frame, as in the following:

%one example missing, not changing everything
\ea
\label{ex:Payne:18} 
\ea 
Intransitive, controlled\\
\glll
Kumarigò an babáyi (*?ha íya kalugaríngon) \\
<um>karigò { } { } { } { } { }\\
 <\textsc{intr.r.ctrl}>bathe \textsc{abs} woman \textsc{loc} 3\textsc{sg.gen} self\\
\glt ‘The woman bathed (herself).' (Expected, normal activity.)

\ex
Intransitive,deliberate\\
\glll Nagkarigò an babáyi (ha íya kalugaríngon) \\
na-g-karigò { } { } { } { } { }\\
\textsc{intr.r-del}-bathe \textsc{abs} woman \textsc{loc} 3\textsc{sg.gen} self\\
\glt ‘The woman bathed (herself).' (Unexpected in some way.)

\z
\z


Example (\ref{ex:Payne:18}a) illustrates an intransitive construction in controlled mood, implying that the event is unsurprising, effortless, normal, and fully expected. In this case, the addition of the reflexive nominal in an oblique role may be grammatical but sounds extremely odd (indicated by the double notation "*?"). Example (\ref{ex:Payne:18}b)depicts a similar scene, but in deliberate mood. This implies that the event is unusual, unexpected, effortful, or surprising in some way. In this case, without the reflexive nominal, coreferentiality is still the implication ('she bathed herself'), but the clause is open to other interpretations, e.g., 'she bathed (someone else, recoverable from the context).' Still, the reflexive nominal in an oblique role forces a reflexive interpretation and the event is assumed to be unexpected for some other contextual reason. For example, the sentence becomes more interpretable with the addition of some context, such as \textit{hin} \textit{petrolyo} 'with gasoline.' Bathing oneself with gasoline would be a highly unusual activity, and hence would require the use of deliberate modality, and the explicit reflexive nominal.

As mentioned above, inherently intransitive introverted verbs may be expressed in a transitive frame with the addition of transitivizing morphology, such as the applicative suffix \textit{an}. In this case, the actor is expressed in the ergative case and the patient in the absolutive. For the clause to express actor-undergoer coreference, the reflexive nominal is required:
%20
\ea \label{ex:Payne:19}
Transitive, applicative\\
\glll
 Ginkarigoan han babáyi an íya kalugaríngon. \\
 <in>g-karigò-an { } { } { } { } { }\\
 \textsc{<tr.r}>\textsc{del-}bathe-\textsc{appl}1 \textsc{erg} woman \textsc{abs} 3\textsc{sg.gen} self\\
\glt'The woman bathed herself.' 
\z


The detransitive version of this construction is not grammatical, since the applicative -\textit{an} always derives a grammatically transitive stem. Rather, the intransitive forms without the applicative (examples in \ref{ex:Payne:18}) serve the function of a detransitive applicative.

Other verbs that follow this pattern are \textit{ahit} 'shave hair' and \textit{sudlay} 'comb hair'. Here are some examples with \textit{sudlay}:

%
%21
\ea \label{ex:Payne:20}
\ea
\glll Nagsudlay hiya (han íya bungot).\\
 na-g-sudlay { } { } { } { }\\
 \textsc{intr.r-del}-comb 3\textsc{sg.abs} {obl} 3\textsc{sg.gen} beard\\
\glt ‘He\textsubscript{i} combed (his\textsubscript{i/j} beard).'

\ex
\glll Ginsudlayan han barbero an íya bungot.\\
 <in>g-sudlay-an { } { } { } { } { }\\
 \textsc{<tr.r}>\textsc{del-}comb-\textsc{appl} \textsc{erg} barber \textsc{abs} 3\textsc{sg.gen} beard\\
 \glt ‘The barber\textsubscript{i} combed his\textsubscript{i/j} beard.'
 \z
 \z
 
The root \textit{sudlay} does not naturally occur in the controlled mode ??\textit{sumudlay}. In example (\ref{ex:Payne:20}a), in the absence of a clarifying oblique, the actor's head hair is the usual interpretation of the undergoer. However, this assumption can be cancelled with the mention of another kind of hair, e.g., \textit{bungot} 'beard', expressed as an oblique. Also, in (\ref{ex:Payne:20}b) the first impression is that the actor and the possessor of the beard are not coreferential -{}- because that is a typical thing for barbers to do. Though, again, this is not necessary -{}- the barber may be combing his own beard. 

In all cases in which a possessor may or may not be coreferential with the actor of the clause, a coreferential meaning may be enforced by the use of \textit{kalugaríngon} in a genitive role. This is fully grammatical, but unusual in discourse, since in fact the coreference relations are normally clear enough in actual conversation. As discussed further below, the reflexive nominal in a genitive role usually functions as a kind of self-intensifier (see, e.g., \citealt{Haspelmath2020} this volume)\todo{this volume}, stressing the fact that the actor accomplishes the action on her or his own possession, and that this is unexpected for some reason:

%"filled out" example
\ea
\label{ex:Payne:21}
\ea
\glll
{Nagsudlay} {hiya} {han} {íya} {kalugaríngon} {bungot.}\\
 na-g-sudlay { } { } { } { } { }\\
 \textsc{intr.r-del}{}-comb 3\textsc{sg.abs} {obl} 3\textsc{sg.gen} self beard\\
 \glt'He\textsubscript{i} combed his\textsubscript{i} own beard.' (cf. \ref{ex:Payne:20}a)

\ex
\glll
 {Ginsudlayan} {han} {barbero} {an} {íya} {kalugaríngon} {bungot.}\\
 <in>g-sudlay-an { } { } { } { } { } { }\\
 \textsc{<tr.r}>\textsc{del-}comb-\textsc{appl} \textsc{erg} barber \textsc{abs} 3\textsc{sg.gen} self beard\\
 \glt'The barber\textsubscript{i} combed his\textsubscript{i} own beard.' (cf. \ref{ex:Payne:20}b)
 \z
 \z

In (\ref{ex:Payne:21}a) and (\ref{ex:Payne:21}b), \textit{íya} \textit{kalugaríngon} 'his self' is treated as a nominal possessor of 'beard', and \textit{íya} must be coreferential with the actor of the clause. Compare (\ref{ex:Payne:21}a) to the following. In this case, \textit{íya} \textit{amay} 'her/his father' is the nominal possessor of \textit{bungot}, and coreference between \textit{íya} and the actor is the expected, but not necessary interpretation:

%23
\ea
\label{ex:Payne:22}
\glll
Nagsudlay hiya han bungot han íya amay.\\
na-g-sudlay { } { } { } { } { } { } \\
\textsc{intr.r-del}-comb 3\textsc{sg.abs} \textsc{obl} beard \textsc{obl} 3\textsc{sg.gen} father\\
\glt ‘S/he\textsubscript{i} combed her/his\textsubscript{i/(j)} father's beard.'
\z

Interestingly, the roots \textit{suson} 'criticize' and \textit{sina} 'hate/be angry with' fall into the grammatical class of introverted actions, though semantically they may be considered "socially antagonistic." The basic, underived forms of these verbs are intransitive, and the transitive forms must be derived:

\ea
Intransitive,controlled\\
\label{ex:Payne:23}
\ea
\glll
Sumuson an politiko (ha íya kalugaríngon). \\
<um>suson { } { } { } { } { } \\
 <\textsc{intr.r.contr>}criticize \textsc{abs} politician \textsc{loc} 3\textsc{sg.gen} self\\
 \glt ‘The politician criticized himself.' (Gently, self-reflecting)
 
\ex
Intransitive, deliberate\\
\glll
Nagsusón an politiko (ha íya kalugaríngon).\\
na-g-suson { } { } { } { } { } \\
 \textsc{intr.r}{}-\textsc{del}{}-criticize \textsc{abs} politician \textsc{loc} 3\textsc{sg.gen} self\\
\glt ‘The politician criticized himself.' (Deliberate, public.)

\ex
Transitive, deliberate applicative\\
\glll
Ginsusnan han politiko an íya kalugaríngon.\\
 <in>g-suson-an { } { } { } { } { } \\
 \textsc{<tr.r}>\textsc{del-}criticize-\textsc{appl} \textsc{erg} politician \textsc{abs} 3\textsc{sg.gen} self\\
 \glt ‘The politician criticized himself.' (Mercilessly, harshly.)
\z
\z

\ea
\label{ex:Payne:24}
 Intransitive, spontaneous\\
\ea
\glll
Nasísina hiya ha íya kalugaríngon.\\
 na-\textsc{red}1-sina { } { } { } { } .\\
 \textsc{r.spon-imperf}-hate 3\textsc{sg.abs} \textsc{loc} \textsc{3sg.gen} self\\
 \glt'He hates (is angry with) himself.'
 

 
 \ex
  Transitive, deliberate\\
  \glll
 Ginsinahan níya an íya kalugaríngon.\\
 <in>g-sina-an { } { } { } { } \\
 \textsc{<tr.r}>\textsc{del-}hate-\textsc{appl}1 3\textsc{sg.erg} \textsc{abs} 3\textsc{sg.gen} self\\
\glt ‘He hated (or got angry with) himself.'
\z
\z


We speculate that these roots follow the pattern of introverted verbs because there is no physical effect on the criticized/hated person. The relevant semantic distinction in Waray seems to be between events that cause a physical change vs. those that do not, rather than strictly extroverted vs. introverted actions. 

Here is an example of a verb that falls into the extroverted category, even though it does not describe a socially antagonistic act. It is more similar, semantically, to a grooming verb. In this case, however, the affected body part must be mentioned, probably because, unlike 'comb', there is no particular part of the body for which scratching is a normal, everyday activity:
%filled out
\ea
\label{ex:Payne:25}
\glll
{Ginkalot} {niya} {an} {íya} {(kalugaríngon)} {likod.} \\
<in>g-kalot { } { } { } { } { } \\
\textsc{<tr.r}>\textsc{del}{}-scratch 3\textsc{sg.erg} \textsc{abs} 3\textsc{sg.gen} self back\\
\glt'S/he scratched her/his (own) back.' 
\z


Without \textit{kalugaríngon}, example (\ref{ex:Payne:25}) is ambiguous as to whether the possessor of the back is coreferential with the actor. With \textit{kalugaríngon}, the reflexive interpretation is enforced. Although the effect of scratching may or may not be visible, it does involve physical rather than solely psychological effects. We speculate that it is for this reason that \textit{kalot} 'scratch' falls into the class of "extroverted" (or physical effect) verbs. 

\section{Coreference between non-actor arguments} 
\label{sec:Payne:6}
The reflexive nominal may be used to enforce coreference between non-actor arguments. For example:
%fill out examples
\ea
 \label{ex:Payne:26}
 \glll
 Ginsumatan kami niya bahin han ámon kalugaríngon.\\
 <in>g-sumat-an { } { } { } { } { } { } \\
\textsc{<tr.r}>\textsc{del-}tell-\textsc{appl} 1\textsc{excl.abs} 3\textsc{sg.erg} about \textsc{obl} 1\textsc{excl.gen} self\\
 \glt ‘He told us about ourselves.'
\z

When the target and its reflection are both non-actors and first or second person, as in (\ref{ex:Payne:26}) and (\ref{ex:Payne:28}), the reflexive nominal is possible, but not necessary. Examples (\ref{ex:Payne:26}) and (\ref{ex:Payne:27}) are nearly synonymous. (\ref{ex:Payne:26}) simply emphasizes the importance of the coreference relation (similar to the self-intensifying function described above for \textit{kalugaríngon} in a genitive role):%\todo{incomplete examples}

%28
\ea
 \label{ex:Payne:27}
 \glll
Ginsumatan kami níya bahin han ámon.\\
 <in>g-sumat-an { } { } { } { } { } \\
\textsc{ <tr.r}>\textsc{del-}tell-\textsc{appl} 1\textsc{excl.abs} 3\textsc{sg.erg} about \textsc{obl} 1\textsc{excl.obl}\\
 \glt'He told us about us.'
\z

%29
\ea
\label{ex:Payne:28}
\glll
 Ginpakità ta ikaw han ímo ladawan.\\
 <in>g-pa-kità { } { } { } { } { } \\
 \textsc{<tr.r>del-caus}-see 1\textsc{inc.erg} \textsc{2sg.abs} \textsc{obl} 2\textsc{sg.gen} picture\\
 \glt ‘I showed you a picture of you.' (or 'your picture')\footnote{Recall that \textit{ta} \textit{ikaw} and \textit{ta} \textit{kamo} are 'composite' forms used whenever a first person participant acts on a second person participant. While =\textit{ta} is an inclusive plural (first + second person) form, it always stands for first person singular when the absolutive is second person.}
 \z

However, when the actor and the non-actor nominal are third person and the same number, there is no non-paraphrastic way to disambiguate. The examples in (\ref{ex:Payne:29}) are ambiguous with or without the presence of the reflexive nominal:

%30
\ea
\label{ex:Payne:29}
\ea
Transitive:\\
\glll
 Ginpakità ni Juan hi Maria hin íya kalugaríngon ladawan.\\
 <in>g-pa-kità\\
 \textsc{<tr.r>del-caus}-see \textsc{erg.p} John \textsc{abs.p} Mary \textsc{obl} 3\textsc{sg.gen} self picture\\
 \glt ‘John showed Mary a picture of him/herself.'
 
\ex
Detransitive:\\
\glll
Nagpakità hi Juan kan Maria hin íya kalugaríngon ladawan.\\
na-g-pa-kità \\
\textsc{intr.r-del-caus}-see \textsc{abs.p} John \textsc{obl.p} Mary \textsc{obl} 3\textsc{sg.gen} self picture\\
\glt ‘John showed Mary a picture of him/herself.'
\z
\z

Without \textit{kalugaríngon}, (\ref{ex:Payne:29}a)--(\ref{ex:Payne:29}b) would be triply ambiguous. The picture could be of John, of Mary, or of some other 3\textsuperscript{rd} person singular referent. It should be emphasized that this type of construction, though completely grammatical, is rare in conversation. In face-to-face discourse, coreference relations are usually clear from the context. This optional use of \textit{kalugaríngon} may be seen as a kind of self-intensifying function, emphasizing the coreference relationship, or contrasting coreference with a presumption of disjoint reference. 

\section{Contrast between exact and inclusive coreference} 
\label{sec:Payne:7}

There is no essential contrast between reflexive constructions involving exact vs. inclusive coreference. The expression \textit{ngan} \textit{iba} 'and others' can simply be added to the reflected referential expression to indicate others are included with the referent of the reflexive nominal.

\ea
\label{ex:Payne:30}
\glll
 Dinádayaw niya an íya kalugaríngon ngan an iba.\\
 <in>\textsc{red}1-dayaw \\
 <\textsc{tr.r.ctrl}>\textsc{imperf}-praise 3\textsc{sg.erg} \textsc{abs} 3\textsc{sg.gen} self and \textsc{abs} other\\
 \glt'He praises himself and others.'
\z

This strategy seems to be available for any construction involving \textit{kalugaríngon}.

\section{Long-distance coreference} 
\label{sec:Payne:8}

In long distance co-reference, the reflexive nominal may be used to enforce coreference relations: 

\ea\label{ex:Payne:31}
\gll
Húnahúna ni Pedro may adâ an íya kalugaríngon igo nga kwarta.\\
 think \textsc{erg.p} Pedro \textsc{exist} {}  \textsc{abs} 3\textsc{sg.gen} self enough \textsc{lk} money\\
 \glt ‘Pedro thinks that he himself has enough money.'
\z

The construction in (\ref{ex:Payne:31}), though grammatical, is unusual in actual conversation. Normally a simple 3\textsc{sg.abs} pronoun would imply, though not strictly code, coreference in situations like this:

\ea
\label{ex:Payne:32}
\gll
 Húnahúna ni Pedro may adâ hiya igo nga kwarta.\\
 think \textsc{erg.p} Pedro \textsc{exist} {}  3\textsc{sg.abs} enough \textsc{lk} money\\
 \glt ‘Pedro\textsubscript{i} thinks that he\textsubscript{i(j)} has enough money.'
\z

Again, this (rather uncommon) usage of \textit{kalugaríngon} may be seen as a kind of self-intensifying usage. However, unlike self-intensifiers in European languages (e.g., Latin \textit{ipse}, German \textit{selbst}, or Spanish \textit{mismo/misma}), \textit{kalugaríngon} is syntactically constrained -{}- it may not appear as an appositive (\ref{ex:Payne:33}a, b), or in an actor role (see ex. (\ref{ex:Payne:15}) above):

\ea
\label{ex:Payne:33}
 Spanish: \textit{Viene la reina misma.} \\
 German: \textit{Die Königin selbst kommt.}\\
 Latin: \textit{Regina ipsa venturus est.\todo{this should be ventura est}}\\
 English: \textit{The queen herself is coming.}\\
 
\ea[*]{
Waray:\\
\gll
  Makanhi (íya) \textbf{kalugaríngon} an reyna.\\
 coming 3\textsc{sg.gen} self \textsc{abs} queen\\
 }
 
 
\ex[*]{
\gll Nagkúkuha (íya) \textbf{kalugaríngon} han reyna an tinapay.\\
getting 3\textsc{sg.gen} self \textsc{erg} queen \textsc{abs} bread\\
\glt (for 'The queen herself is getting the bread.')
}
\z
\z


Such self-intensifying functions are available in Waray using the Spanish borrowing \textit{mismo}, though this usage is not particularly common: 

%35 in word
\ea
\label{ex:Payne:34}
\textit {Makanhi mismo an reyna.}\\
\textit {Makanhi an reyna mismo.}\\
\glt ‘The queen herself is coming.'
\z

Of the 256 examples of \textit{mismo} in the corpus, there are none that clearly exhibit this usage. Furthermore, \textit{mismo} never functions as a reflexivizer:

%36 in word
\ea[*]{
\label{ex:Payne:35}
Ginpatay han tawo an íya mismo. (cf. \ref{ex:Payne:9})\\
(for: 'The man killed himself.')
}
\z


\section{Speculations regarding the awkwardness of \textit{kalugaríngon} constructions in Waray}
\label{sec:Payne:9}


As mentioned in the introduction, we find it surprising that the reflexive form, {kalugaríngon}, is such a phonologically large and morphologically complex nominal. Most languages, it seems, have well structuralized and phonologically reduced patterns for expressing reflexive ideas. One may especially expect languages with highly synthetic verb morphology, such as Waray, to have some verb or verb-phrase element that expresses at least some varieties of reflexivity. Indeed, the verb morphology of Waray offers many ways of adjusting the argument relations and event type expressed by a clause, including causative, applicative (two types), reciprocal, precative, associative (one action done together with others), distributive associative (multiple actions done randomly with others), distributive (action done randomly), counter expectation, imperfective, iterative, attenuative, and others. One finds it surprising, in this context, that reflexivity should be a category that is not also well grammaticalized. Instead, we find a rather cumbersome and often awkward full nominal expression. 

Our speculation on this topic is grounded in the observation that Philippine cultures, Waray in particular, are very communal societies. Acting together with others is a high cultural value. Consequently, it is often unusual, and rather aberrant that someone should act exclusively on one's own, or upon oneself. This fact is expressed in the grammar in the multiplicity of associative, mutual action and reciprocal categories in the verb, and in the inclusive/exclusive distinction in the pronoun system. The colloquial expression \textit{paglugaring!} 'Do it on your own' or 'don't involve me/us with what you are doing' is an indicative example. This expression (based on the root \textit{lugaring}), is used as a rebuke to somewhat ostracize somebody from a group. This is because in Waray, traditionally things are done collectively. 

For another example, in traditional contexts drinking \textit{tubâ} 'coconut wine' is a social activity. Waray even has the following lexicalized expression employing the associative infix \textit{\nobreakdash-Vr-}:

\ea
\label{ex:Payne:36}
\glll irignom \\
<Vr>g-inom\\
\textsc{<assoc}>\textsc{del}-drink\\
\glt ‘drinking session' 
\z

Traditionally there is one \textit{tagayan}, a cup that is passed from person to person in a drinking session. Warays never drink alcohol alone. So, to do things alone, especially social activities, is odd, and a serious breach of social norms. We consider these observations to be speculation, since one must be careful not to jump too quickly from cultural observations to linguistic analyses. In this case, however, we find the speculation particularly intriguing, and perhaps worthy of serious future research.

\section{The use of \textit{kalugaríngon} in discourse}
\label{sec:Payne:10}

In a corpus of 1,753,050 words (3NS Corpora Project), we find 323 examples of \textit{kalugaríngon}, or .08\% of the total number of words. It is the 268\textsuperscript{th} most common word in the corpus. For comparison, there are 117,231 examples of standard reflexive pronouns in the British National Corpus (\citealt{Davies2004}), advertised to contain "100 million words". Thus approximately .11\% of the advertised total number of words in the English corpus are reflexive pronouns. Furthermore, we did not include possessors with \textit{own} in our search of the BNC, even though \textit{kalugaríngon} is used this way in Waray. From this we can conclude that reflexive constructions with \textit{kalugaríngon} are proportionally less common than similar large reflexives in English. Whether this difference is significant or not we will leave to the statisticians.

The following are a few naturally occurring examples of \textit{kalugaríngon} from the corpus, with some observations concerning its usages. We include these examples to balance the examples earlier in the paper, most of which are devised by speakers specifically in response to a questionnaire. The out-of-context examples are fully grammatical, but apart from a discourse context, it is often unclear why a speaker would choose to use \textit{kalugaríngon} or not.

\paragraph*{{\textit{Kalugaríngon}} {as} {an} {absolutive} {nominal:}} Examples \REF{ex:Payne:37}--\REF{ex:Payne:39} are examples of reflexive constructions in which the reflexive nominal is obligatory. In these examples, the reflexive nominal is in the absolutive case, and its antecedent, the second-position enclitic pronoun =\textit{ko}, is in the ergative:

%38
\ea
\label{ex:Payne:37}
\glll
Di' ko man puydi ig-stress \textbf{tak'} \textbf{kalugaríngon}  ha iyo.\\
 dili { } { } { } i-g-stress iton-ákon { } { } { } \\
 \textsc{neg} 1\textsc{sg.erg} so can \textsc{appl2-del}-stress \textsc{demo1.abs-}1\textsc{sg.gen} self \textsc{loc} 2\textsc{sg.obl}\\
\glt ‘I cannot stress myself for you.'
\z

%39
\ea
\label{ex:Payne:38}
\glll
 Ginpakamatayan ko na hin tawâ \textbf{an} \textbf{ak'} \textbf{kalugaríngon}.\\
<in>g-pag-ka-matay-an { } { } { } { } { } ákon { }\\
<\textsc{tr.r}>\textsc{del-inf-cntrex}{}-kill-\textsc{appl}1 1\textsc{sg.erg} \textsc{compl} \textsc{obl} laugh \textsc{abs} 1\textsc{sg.gen} self\textbf\\
\glt ‘I killed myself with laughter.'
\z

%40 in word
\ea
\label{ex:Payne:39}
\glll
Nag-newyear resolution man gud ak' nga pupugson ko \textbf{tak'} \textbf{kalugaríngon}  magsurat hin bisan ano kada búlan.\\
na-g-new.year { } { } { } { } { } \textsc{red}1-pugas-on { } iton-ákon {} ma-g-surat {} {} {} {} {}\\
\textsc{intr.r}{}-new.year resolution so indeed 1\textsc{sg.abs} \textsc{lk}  {imperf-}force-\textsc{tr.r} 1\textsc{sg.erg} \textsc{demo1.abs-}1\textsc{sg.gen} self \textsc{intr.ir-del}{}-write \textsc{obl.indef} about what every month\\
\glt ‘I made a new year resolution that I will force myself to write about something every month.'
\z

In examples \REF{ex:Payne:37} and \REF{ex:Payne:39}, the form \textit{tak'} is a blend and contraction of the demonstrative \textit{iton} plus the pronoun \textit{ákon}. 

\paragraph*{\textit{Kalugaríngon} {as} {a} {genitive} {modifier:}} All examples of \textit{kalugaríngon} functioning as a genitive modifier that occur in the corpus express intensification of the coreference relation (or self-intensification). In example (\ref{ex:Payne:40}), \textit{áton} \textit{kalugaríngon} 'our self' is a genitive modifier within the noun phrase \textit{an} \textit{áton} \textit{kalugaríngon} \textit{nga} \textit{dila}, 'our own tongue,' literally 'our self's tongue'. The absolutive case determiner, \textit{an}, specifies the head, \textit{dila}, and not \textit{kalugaríngon}. 

\ea
%41 in word
\label{ex:Payne:40}
\glll
Yana nga may MTBMLE na kita gin-aaghat an mga manunurat an mga Waraynon nga gamiton an \textbf{áton} \textbf{kalugaríngon} nga dila ha kada adlaw nga pakigkaharampang ngan pakig-istorya ngan pati na ha panutdoan hin aton mga eskwelahan\\
{ } { } { } { } { } { } <in>g-\textsc{red}1-aghat { } { } ma-g-\textsc{red}1-N-surat { } { } waray-non { } gamit-on { } { } { } { } { } { } { } { } { } pag-ki-g-<Vr>kahampang { } pag-ki-g-istorya { } { } { } { } pag-N-tutdo-an { } { } { } eskwela-an\\
now \textsc{lk} \textsc{exist} MTBMLE \textsc{compl} 1\textsc{incl.abs} <\textsc{tr.r}>\textsc{del-imperf}-encourage\textsc{} \textsc{abs} \textsc{pl} \textsc{inr.r-del-imperf-plc-}write \textsc{abs} \textsc{pl} Waray-\textsc{person.nom} \textsc{lk} use-\textsc{nom} \textsc{abs} 1\textsc{plinc.gen} self \textsc{lk} tongue in every day \textsc{lk} \textsc{inf-prec}-\textsc{del-plc}-socialize and \textsc{inf-prec-del}-speak and
even \textsc{compl} \textsc{loc} \textsc{inf}-\textsc{plc}-teach-\textsc{nom} \textsc{loc} 1\textsc{plinc} \textsc{pl} education-\textsc{loc.nom}\\
\glt ‘Now that we have MTBMLE,\footnote{Mother-Tongue Based Multi-Lingual Education.} the writers, the Waray are encouraged use our own tongue in our everyday socializing, conversation and even in teaching in our school.
\z

This usage of \textit{kalugaríngon} is notionally redundant, since \textit{an} \textit{áton} \textit{dila} 'our tongue' would have been perfectly clear. However, its usage here emphasizes the fact that the language referred to is \textit{our} \textit{own\-}, i.e., something that belongs to us. In a technical sense, this example also involves "long distance" reflexivization, since the antecedent for \textit{ákon} \textit{kalugaríngon} is in the previous clause, \textit{yana} \textit{nga} \textit{may} \textit{mtbmle} \textit{na} \textbf{\textit{kita}} . . . . ‘Now that \textbf{we} have MTBMLE . . ." However, this use of \textit{kalugaríngon} is more intensive than reflexive/coreferential. The speaker is stressing that writers are using Waray, as opposed to the other languages that Waray writers usually employ. 

Example (\ref{ex:Payne:41}) also illustrates \textit{kalugaríngon} functioning as a self-intensifying genitive modifier within an NP. Again, this usage is technically redundant -{}- \textit{an} \textit{akon} \textit{kahímo} \textit{nga} \textit{dugúan} ‘my bloody face' would have been perfectly clear.

%42 in word
\ea
\label{ex:Payne:41}
\glll
Nasiplatan ko an \textbf{kalugaríngon} \textbf{ko} nga kahímo dugúan, buklad an mata, laylay an dila, luho an agtang.\\
na-siplat-an { } { } { } { } { } { } dugô-an { } { } { } { } { } { } { } { } { }\\
\textsc{r.spon}{}-stare-\textsc{appl} 1\textsc{sg.erg} \textsc{abs} self 1\textsc{sg.erg} \textsc{lk} face blood-\textsc{nom.loc} wide.open \textsc{abs} eye hang.flacidly \textsc{abs} tongue hole \textsc{abs} forehead\\
\glt ‘I stared at my own bloody face, eyes wide open, tongue hanging flaccidly, forehead pierced.'
\z

Example (\ref{ex:Payne:42}) also illustrates \textit{kalugaríngon} as a self-intensifying genitive modifier.

\ea
\label{ex:Payne:42}
\glll
An mababatián mo la mao an hururingay san mga lanyog nga humay o kun di man an mga huni san iba-iba nga mananap ngan tamsi o kun di man an \textbf{kalugaríngon} \textbf{mo} nga pagginhawa.\\
{ } ma-\textsc{red}1-bati-án { } { } ámo { } <Vr>huring-ay { } { } { } { } { } { } { } { } { } { } { } { } { } { } { } { } { } { } { } { } { } { } { } { } { } { } pag-ginhawa\\
\textsc{abs} \textsc{nom.ir-imperf}-hear-\textsc{nom} 2\textsc{sg.gen} just like \textsc{abs} \textsc{dist.plc}-whisper-\textsc{nom}
\textsc{gen} \textsc{pl} ripe \textsc{lk} rice or if not so \textsc{abs} \textsc{pl} call \textsc{gen} different \textsc{lk} animal and bird or if not so \textsc{abs} self 2\textsc{sg.gen} \textsc{lk} \textsc{inf}-breathe\\
\glt ‘What you will hear is like the whispering of the ripe rice, if not the call of different animals and birds, if not your own breathing.'\footnote{This example is from the Northern Samar variety of Waray. This is evident by the use of \textit{san} as the genitive case particle, in place of \textit{han} as used in Leyte. Also, this lexical item, \textit{mao}, is characteristic of Calbayog City and Northern Samar. The form in Leyte is \textit{ámo} or, \textit{asya}.}
\z

Once again, the use of \textit{kalugaríngon} is technically redundant, since \textit{an} \textit{pagginhawa} \textit{mo} 'your breathing' would have been perfectly clear. 

\paragraph*{{Kalugaríngon} {in} {an} {oblique} {role:}} Example (\ref{ex:Payne:43}) is one of the few examples in the corpus in which \textit{kalugaríngon} appears with no possessor. Normally one would expect either the prenominal \textit{ákon} (as in example \ref{ex:Payne:44}), or the post-nominal enclitic =\textit{ko} '1\textsc{sg.gen}{}' in this construction. However, it is a general characteristic of Waray discourse that first person forms may be omitted when the speaker's intention is clear. Therefore, one might say there is a ``zero" possessor of \textit{kalugaríngon} in this example. In this case, the reflexive nominal is required in order to express coreference between the actor and the oblique nominal.

%44 in word
\ea
\label{ex:Payne:43}
\glll
Ako nahipausa \textbf{ha} \textbf{kalugaríngon}.\\
{ } na-hipausa\\
1\textsc{sg.abs} \textsc{r.spon}{-astonish} \textsc{loc} self\\
\glt ‘I was astonished at myself.'
\z

%45 in word
\ea
\label{ex:Payne:44}
\glll
Nakatalwas gad ako hit' nga akon tigdaay nga pag-emcee pero adi la gihap an kaawod \textbf{ha} \textbf{ákon} \textbf{kalugaríngon} nga bisan ako nga ungod nga waraynon banyaga nga dila an nahigaraan.\\
na-ka-talwas { } { } { } { } { } { } { } { } { } { } { } { } { } ka-awod { } { } { } { } { } { } { } { } { } waray-non { } { } { } { } na-higara-an\\
\textsc{r.spon-abl}{}-overcome really 1\textsc{sg.abs} \textsc{demo1} \textsc{lk} 1s\textsc{g.gen} sudden \textsc{lk} \textsc{inf}{}-MC however \textsc{demo2} just also \textsc{abs} \textsc{vrblzr}{}-shy \textsc{loc} 3\textsc{sg.obl} self \textsc{lk} although 1\textsc{sg.abs} \textsc{lk} true \textsc{lk} Waray-\textsc{nom.person} stranger \textsc{lk} tongue \textsc{abs} \textsc{r.spon-}accustom-\textsc{appl}1\\
\glt ‘I was able to pull off my sudden emceeing, though the embarrassment with myself still lit necessary, since ngers, that even though I am a true Waray, I am used to a foreign tongue.'
\z


Once again, the use of \textit{kalugaríngon} in example (\ref{ex:Payne:45}) is technically nothe actor and the coreferential NP are 1\textsuperscript{st} person inclusive. However, in this case it intensifies the seriousness, or challenging connotations of the rhetorical question that follows.

\ea
\label{ex:Payne:45}
\glll
Igpakiana ta ini \textbf{ha} \textbf{áton} \textbf{mga} \textbf{kalugaríngon}: ginpoprotektahan ta ba an aton kalibúngan?\\
i-g-pakiana { } { } { } { } { } { } <in>g-\textsc{red}1-protekta-an { } { } { } { } ka-libong-an\\
\textsc{appl}2-\textsc{del}-ask 1\textsc{plinc.erg} \textsc{demo}1.\textsc{abs} \textsc{loc} 1\textsc{plinc.gen} \textsc{pl} self 
\textsc{<tr.r>del-imperf}{}-protect-\textsc{appl}1 1\textsc{plinc.erg} \textsc{qp} \textsc{abs} 1\textsc{plinc.gen} \textsc{nom}{}-surroundings-\textsc{loc.nom}\\
\glt ‘Let us ask this of ourselves: Are we protecting our environment?'
\z

\paragraph*{Long distance coreference}
In example \REF{ex:Payne:46} \textit{kalugaríngon} occurs in a nominalized (or "headless relative") clause, inside an adverbial clause following the subordinating conjunction \textit{kay} 'because'. Its antecedent occurs in the main clause, \textit{grabe} \textit{nga} \textit{mga} \textit{tawo}. However, the ergative argument of the nominalized clause is "zero" (indicated by parenthetical "they" in the English translation) under coreference with the absolutive of the main clause. In this case, \textit{kalugaríngon} is necessary to express coreferentiality. Without \textit{kalugaríngon}, the sentence would imply that extreme people consider them (some other group) to be gods. 

\ea
\label{ex:Payne:46}
\glll
Grabe nga mga tawo makaharadlok kay (an) mga \textbf{kalugaríngon} an ginkikilala na nga diyos.\\
{ } { } { } { } ma-ka<Vr>hadlok { } { } { } { } { } g<in>\textsc{red}1-kilala { } { } { }\\
extreme \textsc{lk} \textsc{pl} person \textsc{stv-vblzr-assoc}-afraid because \textsc{abs} \textsc{pl} self \textsc{abs} \textsc{del<tr.r>imperf}{}-recognize \textsc{compl} \textsc{lk} God.\\
\glt ‘Extreme people are frightening because (they) are ones who consider themselves as God.'\footnote{The original of this sentence omits the absolutive marker. However, this kind of omission of the absolutive noun-marker is common in spoken discourse, and speakers agree that it should be added at this point.%Original "omits the aboslutive marker here." Changed it because footnote cannot go in gloss
} 
\z

\section{Conclusion}
\label{sec:Payne:11}

In summary, we find that reflexive constructions in Waray are expressed mostly via an analytic strategy involving the reflexive nominal \textit{kalugaríngon}, 'self'. For certain "grooming" activities, actor-undergoer coreference may be expressed in a simple intransitive construction, but this is not common. We call \textit{kalugaríngon} a nominal (or noun) rather than a pronoun because it has almost all properties of ordinary full nouns: it follows a prenominal case marker/determiner, and its person and number are expressed via free possessive pronouns. Also, it may be marked for plurality and modified in the same way nouns can. Pronouns, on the other hand, vary morphologically for case and person/number, and may not take "adpronominal" modifiers. The only respect in which \textit{kalugaríngon} departs from prototypical nounhood is that it may not easily express the semantic role of actor, e.g., the Waray equivalents of "*Herself saw Mary", and "*Himself sat down" are as ungrammatical as these English strings. This fact is the basis for one of the "role related subject properties" discussed by \citet{Schachter1977} for Tagalog.

\textit{Kalugaríngon} may occur in any case, except ergative, as mentioned above. It may also reflect antecedents in any case, including obliques and genitives. The function of \textit{kalugaríngon} in a genitive role almost always intensifies, rather than simply codes a coreference relation. Antecedents in main clauses may also be reflected by \textit{kalugaríngon} in subsequent complement, relative or adverbial clauses, but again, such usages are usually intensive, rather than simply reflexive.

Despite this high degree of flexibility, we find the use of \textit{kalugaríngon} to be proportionally less common in our corpus than are English reflexive pronouns in the British National Corpus. We speculate that this pattern may be due to one or both of two factors: 1. \textit{Kalugaríngon} is a rather cumbersome, often awkward locution. It has not developed a "streamlined" grammaticalized form as one often finds for reflexive constructions in the world's languages. 2. Since Waray traditional culture is very communal and cooperative, self-action is somewhat socially stigmatized. It is often a mark of ostracism and/or disdain for someone to do something "by oneself", "to oneself," or "for oneself." Future research may reveal additional insights in this direction.

\section*{Formatting conventions and abbreviations}

Data in this paper are presented in an interlinear format. The top line is the official Waray orthography, as described in Nolasco, 2012, Nolasco et al 2017\todo{missing reference} \citet{NolascoEtAl2012}, revision currently under consideration by the Department of Education). A second line provides morphological analyses when helpful for the point illustrated by the example. A third line gives the morpheme-by-morpheme glosses. Finally, the last line gives a free English translation.

In the current official orthography, syllable prominence (either word stress, vowel length, or both) is not indicated when it is predictable. When it is unpredictable given the context, an acute accent indicates syllable prominence. Briefly, if the final syllable is prominent, no accent is needed. If there is a "heavy" syllable (CVC, or CV:) anywhere in the word other than the last syllable, the prominence predictably moves to the left, and so is not indicated. All other prominent syllables in indigenous Waray words are indicated with an acute accent. In Spanish and English loan words, stress is not indicated at all. Syllable prominence alone may distinguish lexical items. In addition, many grammatical categories are expressed or accompanied by changes in syllable prominence patterns. The glottal stop is indicated in one of four ways. 1) Sequences of vowel graphemes always involve an intervening glottal stop, e.g., \textit{tiil} [tiˈʔil], 'foot'. 2) Following a consonant, the glottal stop is indicated with a hyphen, e.g. \textit{mag\-áanak} [magˈʔaʔanak] 'will give birth' 3) At the end of a word in a prominent syllable, it is indicated with a circumflex over the final vowel, e.g., \textit{kitâ} [kiˈtaʔ] 'to see'. 4) At the end of a word in a non-prominent syllable, it is indicated with a grave accent over the final vowel, e.g., \textit{sikò} [ˈsikoʔ] 'elbow'. In such cases the penultimate syllable is predictably prominent. Unfortunately, most published material in Waray does not employ diacritics at all.

In this paper, morphological analyses are expressed in the following ways. Prefixes are followed by a hyphen, e.g., \textit{g}-, \textit{pa}-; suffixes are preceded by a hyphen, e.g., -\textit{an}, -\textit{i}: Infixes are surrounded by hyphens when cited in the text, e.g., \nobreakdash-\textit{in}{}-, -\textit{um}{}-, but are surrounded by angled brackets when occurring in morphological analyses of cited data, e.g., <\textit{in}>, <\textit{um}>. 

Abbreviations employed in Waray examples are the following. Note that default features are omitted simply to save space. For example, the determiner \textit{an} is glossed simply as \textsc{abs} 'absolutive', though technically it should be \textsc{abs.def.nonp} 'absolutive, definite/identifiable, non-personal name.' It contrasts with \textit{it,} glossed \textsc{abs.indef} 'absolutive, indefinite/non-identifiable, non-personal name' and \textit{hi} glossed \textsc{abs.p} 'absolutive, personal name'.

\begin{tabularx}{.45\textwidth}{lQ}
\textsc{1excl} & first person plural exclusive\\

1\textsc{sg} & first person singular\\

2\textsc{sg} & second person, singular\\

\textsc{3sg} & third person, singular\\

3\textsc{pl} &  third person, plural\\

\textsc{abs} & absolutive case\\

\textsc{appl1} & applicative 1 (locative or recipient applicative, -\textit{an})\\

\textsc{appl2} & applicative 2 (benefactive or transferred item applicative, \textit{i-})\\

\textsc{caus} & causative\\

\textsc{ctrl} & controlled mood\\

\textsc{compl} & completive particle\\

\textsc{def} & definite/identifiable\\

\textsc{del} & deliberate mood\\

\textsc{demo1} & demonstrative pronoun/adjective, near speaker and hearer.\\

\textsc{demo2} & demonstrative pronoun/adjective, near hearer, away from speaker.\\
\textsc{dist} & distributive (e.g., \textsc{dist.plc}  'distributive pluraction')\\
\end{tabularx}
\begin{tabularx}{.45\textwidth}{lQ}
\textsc{erg} & ergative case\\

\textsc{exist} & existential phrase (\textit{may} \textit{adâ})\\

\textsc{gen} & genitive case\\

\textsc{imperf} & imperfective\\

\textsc{indef} & indefinite/non-identifiable\\

\textsc{inf} & infinitive\\

\textsc{intr} & intransitive\\

\textsc{ir} & irrealis mood\\

\textsc{lk} & linker\\

\textsc{loc} & locative\\

\textsc{neg} & negative\\

\textsc{nom} & nominalizer\\

\textsc{obl} & oblique\\

\textsc{p} & personal name\\

\textsc{pl} & plural\\

\textsc{plc} & pluraction\\

\textsc{qp} & question particle\\

\textsc{r} & realis mood\\

\textsc{red1} & partial (\# CV-) reduplication\\

\textsc{red1~} & full root reduplication\\

\textsc{sg} & singular\\

\textsc{spon} & spontaneous mood\\

\textsc{stv} & stative\\
\textsc{tr} & transitive\\
\textsc{vblzr} & verbalizer\\

\end{tabularx}



{\sloppy\printbibliography[heading=subbibliography,notkeyword=this]}
\end{document}
