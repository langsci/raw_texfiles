\documentclass[output=paper]{langscibook}
\author{Verónica Orqueda\affiliation{Pontificia Universidad Católica de Chile}\orcid{}\lastand Roland Pooth\affiliation{Köln-Nippes}\orcid{}}
\title{Reflexive constructions in Early Vedic}
\abstract{This chapter addresses the diverse reflexive constructions and related functions found in Early Vedic, the earliest attested Indo-Aryan language of the Indo-European family. In particular, we analyze constructions with the middle voice, the nominal strategy \textup{tan\'{ū}-}, and the reflexive adjective \textup{svá-}. Furthermore, we suggest different diachronic pathways that may explain the historical development of the system synchronically developed here.}
% \keywords{Vedic; reflexivity; middle voice; Indo-European}
\IfFileExists{../localcommands.tex}{
  \addbibresource{localbibliography.bib}
  \usepackage{langsci-optional}
\usepackage{langsci-gb4e}
\usepackage{langsci-lgr}

\usepackage{listings}
\lstset{basicstyle=\ttfamily,tabsize=2,breaklines=true}

%added by author
% \usepackage{tipa}
\usepackage{multirow}
\graphicspath{{figures/}}
\usepackage{langsci-branding}

  
\newcommand{\sent}{\enumsentence}
\newcommand{\sents}{\eenumsentence}
\let\citeasnoun\citet

\renewcommand{\lsCoverTitleFont}[1]{\sffamily\addfontfeatures{Scale=MatchUppercase}\fontsize{44pt}{16mm}\selectfont #1}
  
  %% hyphenation points for line breaks
%% Normally, automatic hyphenation in LaTeX is very good
%% If a word is mis-hyphenated, add it to this file
%%
%% add information to TeX file before \begin{document} with:
%% %% hyphenation points for line breaks
%% Normally, automatic hyphenation in LaTeX is very good
%% If a word is mis-hyphenated, add it to this file
%%
%% add information to TeX file before \begin{document} with:
%% %% hyphenation points for line breaks
%% Normally, automatic hyphenation in LaTeX is very good
%% If a word is mis-hyphenated, add it to this file
%%
%% add information to TeX file before \begin{document} with:
%% \include{localhyphenation}
\hyphenation{
affri-ca-te
affri-ca-tes
an-no-tated
com-ple-ments
com-po-si-tio-na-li-ty
non-com-po-si-tio-na-li-ty
Gon-zá-lez
out-side
Ri-chárd
se-man-tics
STREU-SLE
Tie-de-mann
}
\hyphenation{
affri-ca-te
affri-ca-tes
an-no-tated
com-ple-ments
com-po-si-tio-na-li-ty
non-com-po-si-tio-na-li-ty
Gon-zá-lez
out-side
Ri-chárd
se-man-tics
STREU-SLE
Tie-de-mann
}
\hyphenation{
affri-ca-te
affri-ca-tes
an-no-tated
com-ple-ments
com-po-si-tio-na-li-ty
non-com-po-si-tio-na-li-ty
Gon-zá-lez
out-side
Ri-chárd
se-man-tics
STREU-SLE
Tie-de-mann
}
  \togglepaper[13]%%chapternumber
}{}

\begin{document}
\maketitle


\section{Introduction} %1 /
\label{sec:Orqueda:1}

\subsection{Vedic and Early Vedic}
\label{sec:Orqueda:1.1}


{Vedic (or Vedic Sanskrit) is the earliest attested Indo-Aryan language of the Indo-Iranian (or Indo-Iranic) branch of the Indo-European family. It was spoken from the mid-2nd millennium BCE through to the beginning of the 1st millennium BCE, within the area of today’s Afghanistan, northern Pakistan and northern India \citep[160]{Witzel2006}, see \figref{fig:Orqueda:1}.}\footnote{{{File: Early Vedic Culture (1700-1100 BCE).png. (2020, June 1).}{\textit{Wikimedia Commons, the free media repository.}}{Retrieved June 14, 2020 from https://commons.wikimedia.org/w/index.php?title=File:Early\_Vedic\_Culture\_(1700-1100\_BCE).png\& oldid=423076564.}}}

\begin{figure}
\includegraphics[width=\textwidth]{figures/OrquedaPoothnew-img001.png}
\caption{Geographical distribution of speakers of Early Vedic}
\label{fig:Orqueda:1}
\end{figure}

{Vedic is attested in the oldest religious texts of Hinduism and Sanskrit literature, the \textit{Saṃhitās} ‘collections’: \textit{R̥gveda-Saṃhitā} (RV), \textit{Sāmaveda-Saṃhitā}, \textit{Black} (\textit{kr̥ṣṇa}) and \textit{White} (\textit{śukla}) \textit{Yajurveda-Saṃhitā} (YV), and \textit{Atharvaveda-Saṃhitā} (AV). The texts were composed for the ritual recitation of sacred poetic formulas (\textit{mantrās}) with fixed metrical structures alongside parts in prose; they were memorized and verbally transmitted with astonishingly high fidelity by oral tradition across generations up to the present day, preserved in several recensions or ‘schools’ (\textit{śākhās} ‘branches’, e.g. AV of the \textit{Paippalāda-Śākhā}). Their written fixation and canonization was subsequent to the ongoing process of their creation and continual re-arrangements within the oral transmission.}

{Given a timescale of roughly 1000 years, it is difficult to speak of a homogeneous language. Therefore, diverse labels are used to differentiate historical varieties: Early Vedic, Old Vedic, Late Vedic.}\footnote{{{“Old Vedic” is the language of the Mantra period, subsequent to Early Vedic, and datable to ca. 1150 BC with the beginning of the iron age \citep[280]{Witzel1997}. It is followed by “Late Vedic”, from ca. 1200-800 to ca. 500 BCE.}}}{ Early Vedic (henceforth, EV) is the language of the core of the \textit{R̥gveda-Saṃhitā}, especially the language of the “family books” (Maṇḍalas 2-7) and RV 1.51-191, 8.1-66 \citep{Oldenberg1912}, and presumably several parts of RV 9, which is a later compilation of hymns. Although it is not possible to date these phases with complete accuracy, the earliest sections (RV 5) may have been composed by people who also spoke the language in everyday life around 1400 BCE (\citealt{Witzel1989}: 124-7, \citealt{Witzel1997}). The term “Late Early Vedic” refers to the language of RV 1.1-50, 8.67-103, and RV 10.}

{This is a corpus-based investigation and the focus of this paper is on the language of the RV, which most likely corresponds to the period in which Early Vedic was spoken. The structure of this paper is as follows: in the rest of this first section, we offer an overview of the language under investigation and some relevant remarks about its grammar. In \sectref{sec:Orqueda:2}, we analyze the diverse strategies for reflexive marking: verbal (\sectref{sec:Orqueda:2.1}), head noun (\sectref{sec:Orqueda:2.2}) and adjunct auto-possessive (\sectref{sec:Orqueda:2.3}). In each of these sections, we further offer an overview of the different values associated with those strategies. We express our conclusions in \sectref{sec:Orqueda:3}.}



\subsection{General remarks on Early Vedic grammar}
\label{sec:Orqueda:1.2}



Vedic has fusional morphotaxis with cumulative exponence of grammatical categories. The dominant marking strategy is suffixation; partial reduplication is frequent with verbs (e.g. perf. act. {\textit{ca-cákṣ-\textsuperscript{a}} ‘[he/she/it] has seen [him/her/it]’, from ${\surd}$\textit{cákṣ-}).}\footnote{{{In \sectref{sec:Orqueda:1}, we follow the conventions of Vedic philology by giving the}{\textsc{3sg}}{form of verbs as citation form, and by hyphenating the stem (e.g.}{\textit{ca-cákṣ-}}).{The}{\textsc{3sg}}{ending suffix is given as a superscript when not illustrative. The symbol ${\surd}$ is used to cite the root. The traditional category “present” is rather an imperfective aspect plus present tense. “Present stems” (this is, imperfective stems) are traditionally numbered from}{I}{\textsuperscript{st}}{through X}{\textsuperscript{th}}{. For the sake of space, examples are translated but left unglossed in this section. In general, we follow the Leipzig glossing rules (see the Abbreviations section at the end for gloss abbreviations). Morphs are not segmented unless absolutely necessary to follow the argumentation in the paper.}}}{ The fusional marking strategy includes portmanteau suffixes (“endings”) for person, number, TAM, voice (see below), or case, number, gender, e.g. acc. sg. f. \textit{{}-am} of \textit{v\'{ā}c-am} (Patient, Theme or Goal) from \textit{v\'{ā}k} (\textit{v\'{ā}c-}) ‘speech’. There is a high degree of stem variation including root and stem suppletion (e.g. \textit{purú-} adj. m. n. ‘much, many’ vs. \textit{pūrv\'{ī}-} adj. f., and root and/or stem ablaut with multiple morph variants (e.g. ${\surd}$\textit{han-/ghn-/ghan-/ghāṃ-/ja-} ‘to slay, kill’). Several diachronically innovative roots lack ablaut (e.g. 3sg X\textsuperscript{th} pres. ind. act. \textit{cakṣáyati} from \textit{cakṣ-} ‘to see’). Verbs and pronouns may show root suppletion, the former depending on TAM: e.g. perfective \textit{á-vadh-īt,} (${\surd}$\textit{vadh-} ‘to slay, kill’) vs. imperfective \textit{hán-ti} (${\surd}$\textit{han-} ‘to slay, kill’); the latter depending on case, e.g. anaphoric pronouns \textit{sá-s} (\textsc{nom.m}) vs. \textit{tá-m} (\textsc{acc.m}).}

{Verbs inflect via endings that encode simultaneously {person} (1, 2, 3), {number} (\textsc{sg}, \textsc{du}, \textsc{pl}), {voice} (active vs. middle), and TAM distinctions. Present {tense} is only coded by endings, e.g. 3sg pres. ind. active \textit{{}-ti} of \textit{hán-ti} ‘[he/she/it] is slaying [him/her/it/them]’ or middle \textit{{}-te} of \textit{jíghna-te} with the same meaning. Past tense is coded by the prefix \textit{á-} combined with endings (e.g. \textit{á-vadh-īt}, \textit{á-han} ‘[he/she/it] slew, killed [him/her/it/them]’. Future tense is coded by a tense stem \textit{haniṣyá-\textsuperscript{t(i)}} ‘will slay, kill’, which is rare in Early Vedic, future tense being more often coded by the subjunctive-future stem. Coding of {mood} is by endings (indicative \textit{hán-ti, jíghna-te}, imperative 2sg \textit{ja-hí,} 3sg \textit{hán-tu}) or by use of modal stems, e.g. “subjunctive” \textit{hána-\textsuperscript{t(i)}} (exhibiting subjunctive-future polyfunctionality), optative \textit{hany\'{ā}-}\textit{\textsuperscript{t}}, desiderative-conative \textit{jígh\'{ā}ṃsa-}\textit{\textsuperscript{ti}}. There is an archaic non-tensed category called “injunctive”, e.g. \textit{hán} ‘[he/she/it] slew, slays, will slay [it/him/her/them]’, underspecified for tense and non-irrealis modal distinctions. Verbs inflect for {aspect} via varying stems, following a “root and pattern” stem formation principle (\citealt{Pooth2014}: 113ff.): imperfective (traditionally called “present stem”) \textit{hán-\textsuperscript{ti}}, intensive I \textit{jaṅghán-\textsuperscript{ti}}, intensive II \textit{ghánighn-ant-} (participle), perfective (traditionally called “aorist stem”) \textit{á-vadh-\textsuperscript{īt}}, anterior (traditionally called “perfect stem”) \textit{jagh\'{ā}n-}\textit{\textsuperscript{a}}.}

{Nouns and adjectives (e.g. \textit{kr̥ṣṇa-} m. ‘blackbuck, antelope \textit{cervicapra}’, \textit{kr̥ṣṇá-} adj. ‘black’) inflect for three genders (feminine, masculine, and neuter), three numbers (singular, dual, plural), and eight cases (nominative, accusative, instrumental, dative, ablative, genitive, locative, vocative). Nouns have lexical gender. Adjectives generally inflect like nouns but for all three genders.}

{Vedic {alignment} is of the {nominative-accusative} {type}. The nominative typically encodes A = S, while the accusative encodes P (patient), T (theme), G (goal), and even R (recipient); alternations of accusative G and R with dative and locative are not infrequent. The instrumental may express the oblique agent of passive constructions. Vedic lacks the valency relation of necessary complementation \citep[281-301]{Pooth2014}; all arguments can be pragmatically non-overt and covert. Vedic word order is basically discourse-configurational. Noun phrases can be discontinuous. }


\section{Reflexivizers in Early Vedic}
\label{sec:Orqueda:2}

Early Vedic lacks a prototypical reflexive pronoun, but has diverse strategies for coreference of arguments within the minimal clause.\footnote{{{To our knowledge, a thorough study on Vedic long-distance reflexives is still lacking. As in other ancient Indo-European languages, such a pronoun is absent. It seems possible that the demonstrative pronoun}{\textit{sá-}}{may be used in some cases. Further study on this topic is still needed.}}}{ Following the cross-linguistic classification of \citet{Faltz1977}, these are basically the middle voice and a head noun strategy: \textit{tan\'{ū}-} ‘body’. As well, there is a compound strategy with \textit{svá-} (+ noun), used mostly for partial coreference. Early Vedic also has an elaborate system of personal pronouns (1\textsuperscript{st} and 2\textsuperscript{nd} person, sg., du., pl.) and demonstrative pronouns (3\textsuperscript{rd} person sg., du., pl.), which when used in the genitive case (e.g. \textit{máma} 1sg gen., \textit{táva} 2sg gen.), encode both coreferential and disjoint possession.}\footnote{{{There are also possessive pronominal adjectives (e.g.}{\textit{mámaka-}}{‘my’), but these rare in Early Vedic \citep[305]{Macdonell1910}.}}}



\subsection{Verbal reflexivizers}
\label{sec:Orqueda:2.1}

\subsubsection{General remarks on the Early Vedic middle voice and its polysemy}
\label{sec:Orqueda:2.1.1}


In EV, middle inflection is polyfunctional:\footnote{{{The high degree of polysemy and lability in EV middle forms strengthens the hypothesis that the Vedic middle more generally goes back to a Proto-Indo-European “off-valency-processing” detransitivizing category \citep{Pooth2014}.}}}{ following the terminology of \citetv{chapters/02_Haspelmath_Reflexive_constructions} its functions include {auto-pathic} (i.e. direct reflexive), as the first 3\textsc{pl} form in \REF{ex:Orqueda:1},}\footnote{{{All translations are our own, unless explicitly stated.} }}{ {auto-benefactive}, as in \REF{ex:Orqueda:2}, {auto-receptive}/{auto-directed}, as in \REF{ex:Orqueda:3}, or {auto-possessive} (reflexive possessive), as in \REF{ex:Orqueda:4}.}\footnote{{{We prefer the labels “recipro-pathic” and “auto-possessive”, as these terms show with more accuracy that these are different functions and that they belong to a complex net of connected functions (autopathic, auto-benefactive, recipro-possessive, etc.).}}}{ The subject (mainly nom.) is either beneficiary, recipient/goal, or possessor:}

\ea%1
 \label{ex:Orqueda:1}  
 \gll \textbf{{añjáte}} vy  añjate  sám  añjate\\
    anoint.3\textsc{pl.prs.mid}  \textsc{rec}  anoint.3\textsc{pl}.\textsc{prs.mid}  together  anoint.\textsc{3pl.prs.mid}\\
\glt ‘They anoint themselves, they anoint each other, together they anoint each other’ [RV 9.86.43a]
\z

\ea%2
    \label{ex:Orqueda:2}
\gll yáje  tám\\
  worship.\textsc{1sg.prs.ind.mid}  \textsc{dem.acc}\\
\glt ‘I worship him for my benefit’ [RV 2.9.3c]
\z

\ea%3
    \label{ex:Orqueda:3}
\gll \'{ā}  devó    dade ...  vásūni\\
  (t)hither  god.\textsc{nom.sg}  give/take/receive.\textsc{3sg.pf.ind.mid} { } good.\textsc{acc.pl}\\
\glt ‘The god has taken the goods to/for himself’ [RV 7.6.7a]
\z

\ea%4
    \label{ex:Orqueda:4}
\gll úc  chukrám  átka-m  ajate\\
  out  bright.\textsc{acc.sg}  garment.\textsc{acc.sg}  drive.\textsc{3sg.prs.ind.mid}\\
\glt ‘He pulls out his (own) bright garment’ [RV 1.95.7c]
\z

{With plural subjects, middle inflection can show corresponding reciprocal meanings: {recipro-pathic}, as illustrated by the second 3pl form in example \REF{ex:Orqueda:1} (often with the particle \textit{ví} an additional marker; \citealt{Kulikov2007Reciprocal}), {recipro-benefactive} (‘'for each other’s ‘benefit’), {recipro-receptive/amphi-directed} (‘to each other’), {recipro-possessive} (‘each other’s ACC’). With plural subjects, middle inflection also encodes {joint} {action} ('together with each other'), as in \REF{ex:Orqueda:5}, often additionally encoded by the particle \textit{sám} ‘together’:}

\ea%5
    \label{ex:Orqueda:5}
\gll sám  áyanta  \'{ā}  díśaḥ\\
  together  go.\textsc{3pl.prs.ind/subj.mid}  (t)hither  direction.\textsc{acc.pl}\\
\glt ‘They (will) go together in all directions’ [RV 1.119.2b]
\z

Moreover, middle inflection can encode an {indefinite} {Agent}, as in \REF{ex:Orqueda:6}, and can even have a passive function with an optional oblique Agent (normally in the instrumental case), as in \REF{ex:Orqueda:7}

\ea%6
    \label{ex:Orqueda:6}
\gll yáthā  vidé\\
  like  know.\textsc{3sg.prs/pf.ind.mid}\\
\glt ‘As (is) known’ [RV 1.127.4a]
\z

\ea%7
    \label{ex:Orqueda:7}
\gll tvay\'{ā}  yát  stavante …  vīr\'{ā}s\\
  2\textsc{sg.ins}  when  praise.\textsc{3pl.prs.ind/subj.mid} { } man.\textsc{nom.pl}\\
\glt ‘When - by you (oblique agent) - the men are praised’ [RV 6.26.7c]
\z

{Middle inflection is often lexicalized with experiencer-stimulus verbs, verbs of sentience and cognition (e.g. \textit{mányate} ‘to think something, think of someone’), emotive speech, motion, change in body posture, states (e.g. \textit{\'{ā}ste} ‘to sit, sit down’). This conforms to a well-known middle marking pattern \citep{Kemmer1993}.}\footnote{{{Middle inflection is also lexicalized with verbs indicating a lower degree of control, e.g.}{\textit{pard-} }{‘to fart’ (*}{\textit{pá}}{\textit{rdate}}{is not attested in the earliest texts but can be reconstructed based on Classical Sanskrit}{\textit{pardate}}{; see \citealt{Pooth2014}).}}}{ Lexicalized middle inflection allows \textit{man-} ‘to think’ to be used in a predicative reflexive construction, as in \REF{ex:Orqueda:8}:}

\ea%8
    \label{ex:Orqueda:8}
\gll mánye  rev\'{ā}n  iva\\
   think.\textsc{1sg.prs.mid}  wealthy.\textsc{nom.sg}  as\\
\glt ‘I think of myself as a wealthy man’ [RV 8.48.6cd]
\z

{In a few cases, middle inflection indicates that the acc. is a non-affected goal, whereas corresponding active forms indicate that the acc. is an affected patient, e.g. middle \textit{jíhīte} ‘to go away to someone (acc.), to give way to someone (acc.)’ vs. active \textit{jáhāti} ‘'to leave someone (acc.) behind’ (\citealt{Pooth2014}: 154ff.). The distinction of active \textit{yé tvāṃ … pádyanti} ‘who are stepping forward to you’ (RVKhil 4.2.7a) vs. \textit{pádyate, ápādi} ‘to fall down’ (\textit{pad-}) seems to reflect an agentive active vs. non-agentive middle opposition.}

{When judged from its entire functional scope, the EV middle voice category is “off-valency-detransitivizing” \citep{Pooth2014}. This implies that it is \textit{not necessarily} a valency-changing category, and that \textit{per se} middle inflection does not \textit{categorically} decrease the number of participants involved in the event, but can do so, and does, if such an interaction between verb stem and middle inflection is lexicalized.}

{As illustrated in \REF{ex:Orqueda:9}, middles (e.g. 3pl \textit{áranta/aranta}) can show labile syntactic and semantic behavior. They are used intransitively (‘came together’) or convey \textit{indirect causative} meaning (where \textit{indirect causative} means causing a change of state in P without direct physical contact or manipulation).}

\ea%9
    \label{ex:Orqueda:9}
\ea  
    \label{ex:Orqueda:9a}
    \gll sáṃ …  vām  uśánā  áranta  dev\'{ā}ḥ\\
    together { } 2\textsc{du.acc}  uśánā.\textsc{ins}  meet.3\textsc{pl.aor.mid}  god.\textsc{nom.pl}\\
\glt    ‘The gods made you two come together with Uśanā’ [RV 5.31.8d]

\ex
     \label{ex:Orqueda:9b}
\gll sáṃ …  aranta  párva\\
    together { } meet.\textsc{3pl.aor.mid} limb.\textsc{nom.pl}\\
\glt ‘The limbs came together’ [RV 4.19.9d]
    \z
    \z

{In \REF{ex:Orqueda:9a}, the gods (\textit{dev\'{ā}s}) cause a change of state in the 2sg, whereas the meaning of \REF{ex:Orqueda:9b} does not include causation (‘the limbs’ undergo a change of state). Active forms can also exhibit transitive/intransitive lability or similar kinds of polysemy, as in \REF{ex:Orqueda:10}.}

\ea%10
    \label{ex:Orqueda:10}
\gll táva  bhāgásya  tr̥pṇuhi\\
  you.\textsc{gen.sg}  portion.\textsc{gen.sg}  sate.oneself/become.saturated.\textsc{2sg.imp.a}\\
\glt ‘Sate yourself / be / become sated from your portion!’ [RV 2.36.4cd]
\z

{The verb \textit{tr̥p-/tarp-} is stative-processual ‘to be/become sated’ but also allows an agentive reflexive meaning ‘to sate oneself, make oneself be saturated’.}\footnote{{{The stem formation pattern with thematic pres.}{\textit{tr̥ṃpá-}}{, thematic aor.}{ \textit{átr̥pa-,} }{perf. mid.}{\textit{tātr̥pur}}{, participle}{ \textit{tātr̥pāná-}}{points to a preceding deponent verb (“proto-middle tantum”; \citealt{Pooth2014}), as also indicated by the “middle-ish” polysemous semantics.}{The active}{\textit{{}-nu-}}{present forms seem to be innovative.}}}{}

{Thus, not all TAM stems and active vs. middle forms are equally specified for valency in EV. Transitive/intransitive lability vs. non-lability is licensed by a \textit{lexicalized interaction} between the lexical meaning and the meaning of the respective TAM stem formation viz-à-viz active vs. middle inflection \citep{Pooth2014}. Consequently, the valency-decreasing function of middle inflection operates \textit{as lexicalized interaction} with TAM stems specified for valency, e.g. “present passive” stems like \textit{pūyá-\textsuperscript{te}} ‘is purified’ vs. active IX\textsuperscript{th} “present” \textit{pun\'{ā}ti} ‘purifies someone (acc.)’, etc. (\citealt{Kulikov2012}; \citealt{Pooth2014}).}\footnote{{{A diachronic tendency to introduce the valency-changing function by narrowing active or middle forms of formerly labile verbs to either transitive or intransitive function is evident from the relation of active forms of archaic stems of motion verbs (e.g.}{\textsuperscript{1}}{\textit{r̥}}{‘to rise, raise’) to corresponding active forms of evidently innovative stems \citep{Pooth2012}. The restriction of transitive valency to active forms of innovative present stems is also evident from active forms}{\textit{pínva-ti} }{vs. middle forms}{\textit{pínva-te}}{of the verb}{\textit{pinv-}}{‘to swell’. Whereas active forms of the I}{\textsuperscript{st}}{present stem}{\textit{pínva-}}{are restricted to transitive function (‘to swell someone’), corresponding middle forms are more dominantly intransitive (‘to swell’), although there are a few relics with indirect causative meaning. The narrowing of several middle forms to valency-decreasing function and the restriction of TAM stems to either transitive or intransitive valency is an ongoing innovative functional change within the EV language \citep{Pooth2014}.}}}{ Various works have described typical lability introduced by special TAM formations, e.g. that of perfect active forms \citep{Kuemmel2000}, athematic middle -\textit{āna-} participles, etc. \citep{Kulikov2014}.}


\subsubsection{Verbal reflexive constructions in the auto-pathic domain}
\label{sec:Orqueda:2.1.2}

{Auto-pathic reflexives set the coreference between subject and object. Such cases can be expressed by the middle voice in all kinds of clauses, and both with extroverted, as was seen in \REF{ex:Orqueda:1} above and also in \REF{ex:Orqueda:11a}, and extroverted events, as in \REF{ex:Orqueda:11b}, according to \citegen{Haiman1983} terminology:}

\ea\label{ex:Orqueda:11}
 \ea
  \label{ex:Orqueda:11a}
 \gll pr̥ché  tád  éno  varuṇa\\
      ask.\textsc{1sg.ind.mid}  \textsc{dem.acc.n}  sin.\textsc{acc.n}  Varuṇa.\textsc{voc}\\
\glt ‘I ask myself about that sin, o Varuṇa’ [RV 7.86.3a]

\ex
 \label{ex:Orqueda:11b}
\gll uṣámāṇaḥ  \'{ū}rṇām\\
      clothe.\textsc{ptc.mid.nom.sg}  wool.\textsc{acc.sg}\\
\glt ‘Clothing himself in wool’ or ‘Being clothed / dressed in wool’ [RV 4.22.2c]
\z
\z

{In auto-pathic reflexive constructions, the middle voice is an almost obligatory marking that can co-occur with the nominal strategy, as shown below in \sectref{sec:Orqueda:2.2} There is a tendency to use middle inflection as a reflexivizing strategy without additional marking when middle forms have corresponding transitive active forms, as in \REF{ex:Orqueda:12}, while otherwise the additional nominal marking strategy can be used.}\footnote{{{The high number of reflexive examples with an athematic middle participle (especially with the -}{\textit{āna-} }{suffix) combined with}{\textit{tan\'{ū}-}}{is consistent with the idea that these participles are ambiguous between different interpretations, as already pointed by Kulikov in various papers (e.g. \citealt{Kulikov2006}).}}}

 \ea\label{ex:Orqueda:12}
 \gll táva   śriyé   marútaḥ   marjayanta\\
   you.\textsc{gen.sg}   splendour.\textsc{dat.sg}   marut.\textsc{nom.pl}   rub.\textsc{3pl.caus.mid}\\
\glt ‘For your splendour, the Maruts rubbed themselves’ [RV 5.3.3a]
\z

{As for introverted events, the EV verb stem \textit{vás}{}-\textit{te} is restricted to middle inflection, while the causative stem \textit{vāsáya-\textsuperscript{ti}} can be active and transitive ‘to cloth someone (A acting on P)’ As illustrated in \REF{ex:Orqueda:11b}, the meaning of the middle participle \textit{uṣámāṇa-} can have a two-place structure with a P subject (nom.) and a theme (acc.), but it can also have a stative interpretation (‘is dressed/clothed’). Thus, \textit{váste} shows stative-dynamic polysemy ‘to be clothed in (acc.), to clothe oneself in (acc.). The reason why the auto-pathic reflexive reading in \REF{ex:Orqueda:11b} does not co-occur with a nominal strategy may be that \textit{váste} is already a special “introverted verb stem” in EV.}



\subsection{Head noun reflexivizers}
\label{sec:Orqueda:2.2}

\subsubsection{General remarks on \textit{tan\'{ū}-} }
\label{sec:Orqueda:2.2.1}


{The feminine noun \textit{tan\'{ū}-} ‘body, person, self’ can be used in direct (in the accusative case) and indirect (in an oblique case) reflexive constructions, with an animate (and highly agentive) antecedent, as in \REF{ex:Orqueda:13}:}

\ea%12, wrong in the original
    \label{ex:Orqueda:13}
\gll ágne  yájasva  tanvàṃ  táva  sv\'{ā}m\\
   agni.\textsc{voc}  worship.\textsc{2sg.imp.mid}  \textsc{self.acc.sg}  your.\textsc{sg}  own.\textsc{acc.sg}\\
\glt ‘Agni, worship yourself’ [RV 6.11.2d]
\z

However, \textit{tan\'{ū}-}is not a pure reflexivizer without lexical meaning, because it is not wholly grammaticalized as a reflexive marker (\citealt{Pinault2001}, \citealt{Orqueda2019}).\footnote{ The use of}{\textit{tan\'{ū}-} }{as a reflexivizer in Early Vedic illustrates a well-known cross-linguistic development of reflexives from body-nouns and body-part-nouns, as shown by \citet{Schladt2000}, among others.}{ While many cases are ambiguous between a lexical and a reflexive interpretation, others display only a lexical interpretation, as the comparison between \REF{ex:Orqueda:14a} and \REF{ex:Orqueda:14b} shows:}

\ea%13
    \label{ex:Orqueda:14}
\ea
 \label{ex:Orqueda:14a}
\gll s\'{ū}raḥ  upāké  tanvàṃ  dádhānaḥ\\
      sun.\textsc{gen.sg}  in.front.\textsc{loc.sg}  body/self\textsc{.acc.sg}  put.\textsc{prs.ptc.mid.nom.sg}\\
\glt ‘Placing your body/yourself in front of the sun’ [RV 4.16.14a]

\ex
 \label{ex:Orqueda:14b}
\gll áśmā  bhavatu  naḥ  tan\'{ū}ḥ\\
    rock.\textsc{nom.sg}  be/become.\textsc{3sg.imp.act}  we.\textsc{gen.pl}  body.\textsc{nom.sg}\\
\glt ‘Let our body be/become a rock’ [RV 6.75.12b]
\z
\z

{In ambiguous cases like \REF{ex:Orqueda:14a}, only the context may help to disambiguate the polysemy (\citealt{Pinault2001}, \citealt{Kulikov2007Reflexive}). Both as a reflexivizer and as a lexical item, \textit{tan\'{ū}-} is far more frequent in the singular, although there are also some plurals and a few duals. Besides, as expected, the accusative case is most frequent, although there are also cases of coreference in oblique cases, as in \REF{ex:Orqueda:18} below.}


\subsubsection{Head noun reflexive constructions with \textit{tan\'{ū}-}}
\label{sec:Orqueda:2.2.2}


As shown in \sectref{sec:Orqueda:2.1}, the middle voice is the primary reflexivizer in EV, so \textit{tan\'{ū}-} is mostly used as an additional mark of reflexivity to emphasize the reflexive interpretation, and this explains why practically all reflexive constructions with \textit{tan\'{ū}-} are also marked with the middle voice. However, there are no examples of \textit{tan\'{ū}-} with middle-marked and typically introverted events (e.g. \textit{vas-} ‘to be clothed, cloth’). Besides, not all extroverted reflexives allow the addition of \textit{tan\'{ū}-}.

{The reflexive strategy with \textit{tan\'{ū}-} can operate for all three persons and all three genders. The singular accusative with a singular referent is the most frequent structure, although it is also possible to find both a plural reflexivizer with a plural referent, as in \REF{ex:Orqueda:15} below, and a singular reflexivizer with a plural referent.}

\ea\label{ex:Orqueda:15}
\gll yátra    ś\'{ū}rāsaḥ    tanvàḥ  vitanvaté\\
    where    brave.\textsc{nom.pl}    body/self.\textsc{acc.pl}  stretch.\textsc{mid.prs.3pl}\\
\glt ‘Where the brave ones/heroes stretch their bodies/themselves’ [RV 6.46.12a]
\z

{The rarer cases of non-agreement are always ambiguous between a reflexive and a lexical interpretation, but they are worth noting as they explain the incomplete grammaticalization of this item. If \textit{tan\'{ū}-} had undergone complete grammaticalization as a reflexivizer, we could perhaps expect the loss of its declension and/or agreement, which is not the case.}

In the {autopathic} {domain,} there is a tendency to use middle inflection as a reflexivizer without additional marking when middle forms have transitive active uses within the same stem. Otherwise the additional nominal marking strategy is often used as a disambiguating device.\footnote{{{The high number of reflexive examples with an athematic middle participle (especially with the -}{\textit{āna-} }{suffix) combined with}{\textit{tan\'{ū}-} }{is consistent with the idea that these participles are ambiguous between different interpretations, as already pointed by Kulikov in diverse papers (e.g. \citealt{Kulikov2006}).}}}{ For instance, the present stem of \textit{yaj-}‘to worship’ can be used both as intransitive (without acc.) and indirect causative, as in \REF{ex:Orqueda:16a}; and it occurs with \textit{tan\'{ū}-} to reinforce the reflexive interpretation, as in \REF{ex:Orqueda:16b}. In turn, \REF{ex:Orqueda:17} shows that a typically two-place verb form (a X\textsuperscript{th} causative stem) does not occur with an additional marker:}

\ea%15
    \label{ex:Orqueda:16}
    \ea
     \label{ex:Orqueda:16a}
\gll yájasva   hotar   iṣitáḥ   yájīyān\\
      worship.\textsc{2sg.imp.mid}   priest\textsc{.voc.sg}  sent.out.\textsc{voc.sg}   worshipper.\textsc{voc.sg}\\
\glt ‘Make (our offering) worshipped when prompted, O priest and worshipper!’ [RV 6.11.1a]

\ex
 \label{ex:Orqueda:16b}
\gll ágne   yájasva  tanvàṃ  táva   sv\'{ā}m\\
      Agni.\textsc{voc}  worship.\textsc{2sg.imp.mid}  self.\textsc{acc.sg}  you.\textsc{gen.sg}   own.\textsc{voc.sg}\\
\glt ‘Agni, worship yourself / your own body’ [RV 6.11.2d]
\z
\z

\ea%16
    \label{ex:Orqueda:17}
\gll táva   śriyé   marútaḥ   marjayanta\\
      you.\textsc{gen.sg}   splendour.\textsc{dat.sg}   marut.\textsc{nom.pl}   rub.3\textsc{pl.caus.mid}\\
\glt ‘For your splendour, the Maruts rubbed themselves’ [RV 5.3.3a]
\z

{\textit{Tan\'{ū}-} combined with \textit{svá-} can function as a complex compound reflexive, with no difference in meaning to constructions with \textit{tan\'{ū}-} and without \textit{svá-}. Interestingly, a possessive pronoun or a genitive personal pronoun can also occur within this complex construction, as in \REF{ex:Orqueda:16b} above, but not if \textit{svá}{}- is missing.}

{In EV, reflexive \textit{tan\'{ū}-} plus active-marked verbs are infrequent and restricted to causative stems and the 3pl perfect active form \textit{māmr̥juḥ}, as in \REF{ex:Orqueda:17}, which suggests an ongoing diachronic change towards the collapse of the active/middle voice distinction and a decline of middle marking of reflexivity:}\footnote{{{In fact, middle and active voice slowly converge in the history of Sanskrit, and this is in line with a growing use of the masculine noun}{\textit{ātmán-}}{‘self’ as a nominal reflexive marker, regardless of the active/middle verbal endings from the AV (Post Early Vedic) onwards:}{\textit{yáṃ}}{ \textit{vay}}{\textit{áṃ dviṣmáḥ sá} }{{\textit{ātm\'{ā}naṃ} }}{\textit{dveṣṭu} }{(A.) ‘The one who we hate, let that one hate himself’[AV 16.7.5b];}{{\textit{ātm\'{ā}naṃ}}}{ \textit{pitáraṃ putráṃ paútraṃ … / yé priy\'{ā}s t\'{ā}n úpa hvaye} }{(MID) ‘To myself, the father, the son, the grandson, those that are dear, I invoke’ [AV 9.5.30ab].}}}

\ea%17
    \label{ex:Orqueda:18}
\gll váśaṃ  dev\'{ā}sas  tanv\`{ī}  ní  māmr̥juḥ\\
   power.\textsc{acc.sg}    god.\textsc{nom.pl}  self.\textsc{loc.sg}  down/into  rub.\textsc{3pl.pf.a}\\
\glt ‘The gods rubbed their power upon (literally down to/into) themselves’ [RV 10.66.9d]
\z

{The antecedent of \textit{tan\'{ū}-} is most usually the subject (in the nominative case). The few examples of non-subject antecedents (marked with a non-nominative case) are ambiguous, as in \REF{ex:Orqueda:18}}\footnote{{{In this example,}{\textit{svayám}}{is an Actor-oriented intensifier. Although it is not a reflexivizer, it is usually found in reflexive constructions. This can be explained by the fact that Actor-oriented intensifiers are frequently found with highly agentive subjects and these are a requirement for auto-pathic reflexives in Early Vedic.}}}{ below, where a meaning ‘body’ is possible, too. Here, the antecedent of the indirect reflexive \textit{tanvè} is found in the accusative \textit{árīḷham vatsám}.}

\ea%18
    \label{ex:Orqueda:19}
\gll árīḷham  vatsám  caráthāya  māt\'{ā}\\
  unlicked.\textsc{acc.sg}  calf.\textsc{acc.sg}  wander.\textsc{inf.dat}  mother.\textsc{nom.sg}\\

\gll svayám  gātúm  tanvè  ichámānam\\
    by.himself  way.\textsc{acc.sg}   body.\textsc{dat.sg}  seek.\textsc{ptc.mid.acc.sg}\\
\glt `The mother (leaving) the calf unlicked for wandering, [him] who is now seeking by himself a way for himself / his body’ [RV 4.18.10cd]
\z


We may include these cases in this survey, as the reflexive interpretation is possible.

The head noun reflexive strategy also expresses {indirect} {reflexivity}. In these cases, the subject (in the nominative) and an oblique case (e.g., dative, locative, instrumental) are coreferential, as in \REF{ex:Orqueda:20}.

\ea \label{ex:Orqueda:20}
\ea
 \label{ex:Orqueda:20a}
\gll utá   sváyā  tanv\`{ā}  sám  vade   tát\\
      and  own.\textsc{ins.sg}  body.\textsc{ins.sg}  with  say.\textsc{1sg.prs.mid}   this.\textsc{acc.sg}\\
\glt ‘And I discuss this with myself’ [RV 7.86.2a]

\ex
 \label{ex:Orqueda:20b}
\gll janáyan  mitráṃ  tanvè  sv\'{ā}yai\\
      generate.\textsc{ptc.prs.a.nom.sg}  friend.\textsc{acc.sg}    body.\textsc{dat.sg}  own.\textsc{dat.sg}\\
\glt     ‘Generating a friend for yourself’ [RV 10.8.4d] with antecedent 2g nom. (\textit{tvám})
\z
\z

{Indirect reflexive constructions with \textit{tan\'{ū}-} (often with extra emphatic elements, such as \textit{svá-}) are polysemous as for semantic roles; this is not due to the reflexive nature of \textit{tan\'{ū}-} but rather due to the functional scope of the dative.}

{As prototypical indirect reflexives imply coreference with an argument of a three-slot verb in the clause \citep[77-78]{Kemmer1993}, but many EV verbs are underspecified for valency (even \textit{dā-} ‘to give, take, receive, get, grab’), there are problems with describing these constructions as prototypical indirect reflexives in a syntactic sense.}


\subsubsection{The polysemy of \textit{tan\'{ū}-} }
\label{sec:Orqueda:2.2.3}


\textit{Tan\'{ū}-} can also occur as a {reciprocal} {marker} and as an {intensifier}, which corresponds to a frequent kind of polysemy cross-linguistically. Reflexives may be formally identical to both intensifiers and reciprocals (Geniušien\.e 1987, \citealt{Kemmer1993}, \citealt{KoenigSiemund2000}, \citealt{KoenigGast2006}).

{As a recipro-pathic, the use of \textit{tan\'{ū}-}, as in \REF{ex:Orqueda:21}, is an optional additional marker: it is not frequent in the corpus and in all cases it occurs in interaction with other reciprocal markers (the dual number, the middle voice and, often, the reciprocal adverb \textit{mitháḥ} ‘mutually’):}

%20
 \ea
\label{ex:Orqueda:21}
\gll indrāgn\'{ī}...  mitháḥ  hinvān\'{ā}  tanv\`{ā}\\
   indra.agni.\textsc{nom.du}  mutually  impel.\textsc{mid.ptc.nom.du}  body.\textsc{nom/acc.du}\\
\glt ‘Indra and Agni, impelling each other mutually’ [RV 10.65.2ab]
\z

As an intensifier, \textit{tan\'{ū}-} occurs in the nominative (as an adnominal intensifier), or in the instrumental (as an adverbial intensifier), as in \REF{ex:Orqueda:22a} and \REF{ex:Orqueda:22b}, respectively, and it is not restricted to constructions with middle-marked verbs:

\ea %21 in word
\label{ex:Orqueda:22}
\ea
\label{ex:Orqueda:22a}
\gll sv\'{ā}  tan\'{ū}ḥ  bala-déyāya\\
     own.\textsc{nom.sg}  body.\textsc{nom.sg}  power-give.\textsc{ger}\\

\gll    mā  \'{ā}   ihi\\
   1\textsc{sg.acc}  towards  go.\textsc{2sg.imp.act}\\
\glt     ‘Come to me to give me power in your own person’ (‘Come to give me strength yourself’) [RV 10.83.5d]

\ex
\label{ex:Orqueda:22b}
\gll mandasvā   ándhasaḥ\\
     rejoice.\textsc{2sg.imp.mid}    juice.\textsc{gen.sg}\\

\gll r\'{ā}dhase  tanv\`{ā}  mahé\\
     generosity.\textsc{dat.sg}  body.\textsc{ins.sg}  great.\textsc{dat.sg}\\
\glt  ‘Rejoice from the (Soma) juice for the great generosity in person’
[RV 3.41.6ab, RV 6.45.27b]
\z
\z

As \REF{ex:Orqueda:22a} shows, \textit{tan\'{ū}-} can be combined with emphatic elements such as \textit{svá}{}- also when it is used as an intensifier (see \citealt{Kulikov2007Reflexive} and \citealt{Orqueda2019}), thus structurally running in parallel with its use as reflexivizer.



\subsection{Adjunct auto-possessive constructions}
\label{sec:Orqueda:2.3}



As mentioned, Early Vedic has diverse strategies for the expression of the auto-possessive function: the middle voice (see \sectref{sec:Orqueda:2.1.3}), the less frequent use of demonstrative or personal pronouns in the genitive case, as illustrated in \REF{ex:Orqueda:10} above ({\textit{táva} }\textit{bhāgásya tr̥pṇuhi} ‘Sate yourself /become saturated from \textit{your} portion!’, RV 2.36.4cd), and the noun phrase integrated by the adjective \textit{svá-} plus a noun for the possessee, as outlined in \sectref{sec:Orqueda:2.3.1}


\subsubsection{Constructions with \textit{svá-}}
\label{sec:Orqueda:2.3.1}


The adjective \textit{svá-}, etymologically connected to Indo-European cognates that can express (reflexive) possession, such as Latin \textit{suus} and Latvian \textit{savs}, is also highly polysemous, both within the clause and in word-formation. Within the area of functions related to reflexivity, it can be used in {auto-possessive} function within the clause. In \REF{ex:Orqueda:23}, for example, it indicates partial coreference with the subject. It can also be used as an {intensifier}, marking contrastive focus, as in \REF{ex:Orqueda:24}. Furthermore, \textit{svá-} can be used as a {disjoint} {possessive} {marker}, as in \REF{ex:Orqueda:25}, and as the primary strategy for intensifying/reflexive {nominal} {compounds} (see \sectref{sec:Orqueda:2.3.2}). In none of these cases is it restricted to the combination with middle endings.%\todo{are these verses? is this the reason for the line separation?}

\ea%22
    \label{ex:Orqueda:23}
\gll vádhīm   vr̥trám …\\
  kill.\textsc{1sg.a}  Vr̥tra.\textsc{acc.sg}\\

\gll   \textbf{svéna} bh\'{ā}mena  taviṣáḥ  babhūv\'{ā}n\\
  own.\textsc{ins.sg}  rage.\textsc{ins.sg}  strong.\textsc{nom.sg}  become.\textsc{ptc.a.nom.sg}\\
\glt ‘I have killed Vr̥tra, having become strong through my own rage’ [RV 1.165.8ab]
\z

\ea%23
    \label{ex:Orqueda:24}
\ea
 \label{ex:Orqueda:24a}
\gll pibatu  vr̥trakhādáḥ  sutám  sómam\\
    drink.\textsc{3sg.imp.a}  vr̥tra.gnawer.\textsc{nom.sg}  pressed.\textsc{acc.sg}  soma.\textsc{acc.sg}\\

\gll    dāśúṣaḥ \textbf{své} sadhásthe\\
    worshipper.\textsc{gen.sg}  own.\textsc{loc.sg}  place.\textsc{loc.sg}\\
\glt    ‘Let the Vr̥tra-gnawer drink the pressed soma in the worshipper’s own/very seat’  [RV 3.51.9cd]

\ex
 \label{ex:Orqueda:24b}
\gll \textbf{sváḥ} svāya  dhāyase\\
      self.\textsc{nom.sg}  own.\textsc{dat.sg}  nourishing.\textsc{dat.sg}\\

      \gll kr̥ṇutām  r̥tvíg  r̥tvíjam\\
      make.\textsc{mid.imp.3sg}  priest.\textsc{nom.sg}     priest.\textsc{acc.sg}\\
\glt ‘Let the priest himself (and not someone else) make the priest for his own nourishing’ [RV 2.5.7a]
\z
\z

\ea%24
    \label{ex:Orqueda:25}
\gll  {...} te  ápa       s\'{ā}  nú        vájrāt\\
  { } you.\textsc{gen.sg}    away  she.\textsc{nom.sg}   just  thunderbolt.\textsc{abl.sg}\\

\gll dvit\'{ā}  anamat  bhiyásā \textbf{svásya} manyóḥ\\
  just.so  bent.\textsc{3sg.impf.a}  fear.\textsc{ins.sg}  own.\textsc{gen.sg}  fury.\textsc{gen.sg}\\
\glt ‘Now, she bent away just so from your thunderbolt out of fear of your fury’ [RV 6.17.9ab]
\z

{As examples \REF{ex:Orqueda:23} through \REF{ex:Orqueda:24} show, the use of \textit{svá}{}- is not restricted to specific syntactic slots. As for the person feature of its antecedent, 3rd person singular antecedents are in the majority, although the 2nd or 1st person are also frequent, as in \REF{ex:Orqueda:22} and \REF{ex:Orqueda:24}, respectively. Regarding the case of the antecedent, it is usually in the nominative subject position (see \citealt{Vine1997}), but there are examples with an oblique case antecedent in non-subject position, as in \REF{ex:Orqueda:24b}. The cases of genitive antecedents seem to be restricted to a few nouns, to 2nd personal pronouns and demonstratives, while there are no 1\textsuperscript{st} person genitive antecedents.}\footnote{{{\citet{Hock1991} claims that cases as in \REF{ex:Orqueda:24b} confirm that genitives controlling reflexives have more agentive-like features. But see also \citet[212-213]{Vine1997}, who considers that in these cases the genitive indicates the introduction of a new “rhematic” element that binds the auto-possessive marker.}}} Example \REF{ex:Orqueda:26}, in turn, shows that antecedents of \textit{svá-} can be subjects of passive constructions \citep{GrestenbergerNoDate}. This confirms that the antecedents for \textit{svá-} need not be highly agentive.

\ea
\label{ex:Orqueda:26}
\gll mārjālyàḥ  mr̥jyate  své  dámūnāḥ\\
   fit.for.grooming.\textsc{nom.sg}  groom.\textsc{3sg.pass}  own.\textsc{loc.sg}  house.master.\textsc{nom.sg}\\

\gll kavi-praśastáḥ  átithiḥ  śiváḥ  naḥ\\
   poet-praised.\textsc{nom.sg}  guest.\textsc{nom.sg}  kind.\textsc{nom.sg}  our.\textsc{gen.pl}\\
\glt ‘Fit to be groomed, he is groomed in his own [house] as master of the house, praised by poets, our kind guests’ [RV 5.1.8ab]
\z


\subsubsection{Nominal compounds with \textit{svá-}}
\label{sec:Orqueda:2.3.2}


As the first member of a nominal compound,\footnote{{{Interestingly,}{\textit{tan\'{ū}}}{{}- and}{\textit{svayám}}{are also first members of nominal compounds in EV. However, the former is only used with its lexical meaning (e.g.,}{\textit{tanū-tyájaḥ ‘}}{leaving their (own) bodies’), while the latter, with only two occurences in the RV, has an intensifying/anticausative meaning (e.g.}{\textit{svayaṃ-j\'{ā}ḥ} }{‘self-produced’ (RV 7.49.2b), in reference to waters that arise by themselves (springs), in opposition to waters that are found by digging (well water).}}}{ \textit{svá}{}- may be added to a deverbal noun or adjective, giving rise to a reflexive (e.g., the first compound in \ref{ex:Orqueda:27a}), auto-possessive, as in \REF{ex:Orqueda:27b}, intensifying (e.g., the second compound in \ref{ex:Orqueda:27a}), or anticausative interpretation, as in \ref{ex:Orqueda:27c}:}

\ea
\label{ex:Orqueda:27}
\ea
 \label{ex:Orqueda:27a}
\gll svá-kṣatrāya  sváya-śase\\
      self-ruling.\textsc{dat.sg}  self-glorious.\textsc{dat.sg}\\
\glt ‘For the self-ruling and the self-glorious’ [RV 5.48.1cb]

\ex
 \label{ex:Orqueda:27b}
\gll  sva-dháyā  mādáyethe\\
      self-power.\textsc{ins.sg}  rejoice.\textsc{2du.caus.mid}\\
\glt ‘You two rejoice with your own power’ [RV 1.108.12b]

\ex
 \label{ex:Orqueda:27c}
\gll yé  sva-j\'{ā}ḥ  vavr\'{ā}saḥ\\
    who.\textsc{nom.pl}  self-generate.\textsc{nom.pl}  hole.\textsc{nom.pl}\\
\glt    ‘Who are self-generated, like (earth-)holes’ [RV 1.168.2a]
\z
\z

{Notably, unambiguous reflexive examples are rare and most usually can also be interpreted as intensifiers. This confirms the formal overlap between reflexives and intensifiers, which is cross-linguistically frequent in word formation (compounding or derivation; \citealt{Koenig2011}).}\footnote{{The complex polysemic nature of}{\textit{svá-}}{may be explained through its diachrony from PIE. Contrary to the common opinion that it develops from an original reflexive root in Proto Indo-European, we believe that a possessive marker was eventually formed on the base of an original deictic marker (a proximative demonstrative stem) that was high in the features of topicality and animacy. This would explain, particularly, the uses with a genitive antecedent and the disjoint possessive. A brief list of facts that support this interpretation is: first, that in practically all cases}{\textit{svá-}}{has an animate referent (which usually is not a requisite for disjoint possessives); secondly, that}{\textit{svá-} }{frequently occurs in prominent slots in the stanza, mostly the initial position of the clause, in Early Vedic but not in later varieties (contrarily, reflexive markers and possessives need not to be linked to prominent clause slots; thirdly, due to the higher number of cases of intensification in nominal compounds versus the number of clearly reflexive compounds (see especially \citealt{Orqueda2017} for an extensive overview of this claim).}}


\section{Final remarks}
\label{sec:Orqueda:3}


We can draw the following conclusions regarding the reflexive constructions in Early Vedic. First, we showed that polysemy is widespread for the different strategies linked to reflexivity. Secondly, we showed that, while the middle voice is used for both autopathic reflexives and auto-possessives, the use of differential markings for auto-pathic and auto-possessive constructions arises already in Early Vedic. Thirdly, non-nominative subjects controlling autopathic reflexives are not an ordinary case, although they are attested, as long as they are agent-like NPs. This suggests that antecedents of reflexivizers are mainly selected according to semantic features rather than syntactic functions.

{Lastly, we proposed some diachronic explanations for the strategies under study. In particular, we have shown the emergent use of nominal marking for autopathic reflexives, which is in line with the eventual loss of voice distinction in later stages of the language. Reflexives have progressively come to require that the antecedent is an NP high in the features of volition and control, thus distinguishing reflexives from other related functions (such as anticausatives or statives). From our perspective, this development is consistent with changes from a more semantically determined proto-language towards a more configurational syntax. Further research on these topics in later descendants would undoubtedly contribute to a better understanding of these diachronic developments.}


\section*{Acknowledgements}

The research for this paper was financially supported by the CONICYT-FONDECYT, project no. 11170045.


\section*{Abbreviations}

\begin{tabularx}{.45\textwidth}{>{\scshape}lQ}
1 &  1st person\\
2 &  2nd person\\
3 &  3rd person\\
\textsc{a} &  active voice\\
\textsc{acc} &  accusative\\
\textsc{aor} &  aorist\\
\textsc{caus} &  causative\\
\textsc{dat} &  dative\\
\textsc{dem} &  demonstrative\\
\textsc{du} &  dual\\
\textsc{gen} &  genitive\\
\textsc{ger} &  gerund\\
\textsc{imp} &  imperative\\
\textsc{impf} &  imperfect\\
\textsc{ind} &  indicative\\
\end{tabularx}
\begin{tabularx}{.45\textwidth}{>{\scshape}lQ}
\textsc{inf} &  infinitive\\
\textsc{inj} &  injunctive\\
\textsc{ins} &  instrumental\\
\textsc{loc} &  locative\\
\textsc{mid} &  middle voice\\
\textsc{n} &  neuter\\
\textsc{nom} &   nominative\\
\textsc{pf} &  perfect\\
\textsc{pl} &  plural\\
\textsc{prs} &  present\\
\textsc{ptc} &  participle\\
\textsc{rec} &  reciprocal\\
\textsc{sg} &  singular\\
\textsc{subj} &  subjunctive\\
\textsc{voc} &  vocative\\
\end{tabularx}


\sloppy\printbibliography[heading=subbibliography,notkeyword=this]
\end{document}
