\documentclass[output=paper]{langscibook}
\author{Joseph Brooks\affiliation{University of California, Santa Barbara}\orcid{}}
\title{Reflexive and Middle Constructions in Chini}
\abstract{In this paper I rely primarily on examples from discourse in Chini, a language of northeastern Papua New Guinea, in order to describe how reflexivity and autopathic semantic relations are expressed. First I describe the reflexive possessive construction. I suggest that the coreferential association is between the possessor and the most topicworthy participant(s), which often but not always corresponds to the clause-internal subject. I then describe the middle construction and argue that its primary function is to identify the main participant in a clause as a semantic patient. The potential for autopathic readings of clauses headed by middle verb forms depends on the degree of the participant's control over the activity and furthermore involves interplays between lexical semantics and contextual interpretation. Finally I discuss certain specialized middle constructions where the reflexive or reciprocal interpretation is made absolute.}
\IfFileExists{../localcommands.tex}{
 \addbibresource{localbibliography.bib}
 \usepackage{langsci-optional}
\usepackage{langsci-gb4e}
\usepackage{langsci-lgr}

\usepackage{listings}
\lstset{basicstyle=\ttfamily,tabsize=2,breaklines=true}

%added by author
% \usepackage{tipa}
\usepackage{multirow}
\graphicspath{{figures/}}
\usepackage{langsci-branding}

 
\newcommand{\sent}{\enumsentence}
\newcommand{\sents}{\eenumsentence}
\let\citeasnoun\citet

\renewcommand{\lsCoverTitleFont}[1]{\sffamily\addfontfeatures{Scale=MatchUppercase}\fontsize{44pt}{16mm}\selectfont #1}
   
 %% hyphenation points for line breaks
%% Normally, automatic hyphenation in LaTeX is very good
%% If a word is mis-hyphenated, add it to this file
%%
%% add information to TeX file before \begin{document} with:
%% %% hyphenation points for line breaks
%% Normally, automatic hyphenation in LaTeX is very good
%% If a word is mis-hyphenated, add it to this file
%%
%% add information to TeX file before \begin{document} with:
%% %% hyphenation points for line breaks
%% Normally, automatic hyphenation in LaTeX is very good
%% If a word is mis-hyphenated, add it to this file
%%
%% add information to TeX file before \begin{document} with:
%% \include{localhyphenation}
\hyphenation{
affri-ca-te
affri-ca-tes
an-no-tated
com-ple-ments
com-po-si-tio-na-li-ty
non-com-po-si-tio-na-li-ty
Gon-zá-lez
out-side
Ri-chárd
se-man-tics
STREU-SLE
Tie-de-mann
}
\hyphenation{
affri-ca-te
affri-ca-tes
an-no-tated
com-ple-ments
com-po-si-tio-na-li-ty
non-com-po-si-tio-na-li-ty
Gon-zá-lez
out-side
Ri-chárd
se-man-tics
STREU-SLE
Tie-de-mann
}
\hyphenation{
affri-ca-te
affri-ca-tes
an-no-tated
com-ple-ments
com-po-si-tio-na-li-ty
non-com-po-si-tio-na-li-ty
Gon-zá-lez
out-side
Ri-chárd
se-man-tics
STREU-SLE
Tie-de-mann
} 
 \togglepaper[1]%%chapternumber
}{}

\begin{document}
\maketitle 
%\shorttitlerunninghead{}%%use this for an abridged title in the page headers


\section{Introduction}\label{sec:brooks:1}

Here I describe the possessive reflexive and the middle construction in Chini, a language of northeastern Papua New Guinea (PNG). I provide background about Chini in \sectref{sec:brooks:1.1} and my methods in \sectref{sec:brooks:1.2}. In \sectref{sec:brooks:2} I provide an overview of relevant areas of the grammar, especially participant roles and clause structure. I describe the workings of the reflexive possessive pronoun \textit{ŋɨ}= in \sectref{sec:brooks:3} and the middle marker \textit{nji}- in \sectref{sec:brooks:4} I conclude in \sectref{sec:brooks:5}.

\subsection{{The} {Chini} {language}}\label{sec:brooks:1.1}

Chini is the traditional language of the Awakŋi people of Andamang village and the Yavɨnaŋri of Akrukay. Both villages are associated with a distinct dialect, each with a social as well as geographic dimension. The villages themselves correspond to multiple hamlets on the lower Sogeram River in Madang Province, Papua New Guinea (PNG). Local speech practices are characterized by code-switching between Chini and Tok Pisin, the national lingua franca of PNG and areal language of shift. Currently, young adults are mostly bilingual listeners but do not actively use Chini. Most adults in their 40s and older (about 50 people) are active users, and some are multilingual in neighboring languages. Dialect differences and any Tok Pisin material are maintained in examples.

\begin{figure}
\caption{Chini in Areal Perspective}
{Thank you to Monika Feinen for creating this map.}
\includegraphics[width=\textwidth]{figures/BrooksChini-img001.jpg}
\label{fig:brooks:1}
*Red denotes Trans New Guinea languages, Green: Ramu languages, White: uninhabited
\end{figure}

Chini belongs to the Tamolan subgroup in the Ramu family (\citealt{Brooks2018Realis, Foley2005, Zgraggen1971}), a grouping of at least 20 languages along the lower and middle Ramu River and in adjacent areas. Few descriptive materials are available on these languages.

\subsection{{Methodological} {background}}\label{sec:brooks:1.2}

The ongoing fieldwork on which this paper is based has been conducted across multiple trips totaling 12 months between 2012 and 2019. My fieldwork practice has ethnographic, linguistic, and documentary components. The corpus is housed at the Endangered Languages Archive (ELAR) (see \citealt{Brooks2018Documentation} for the web address). The annotated part of the corpus consists of some 15 hours of connected speech in Chini, including narrative but mostly conversation. This is supplemented by my field notes which include many key examples from unrecorded interactions.

The angle I take in this paper is to describe the possessive reflexive and the middle constructions in a way that reflects Chini grammar and usage, as limited by the extent of my understanding. I rely mostly on examples from connected speech. These are identified to their location in my fieldnotes or to recordings in the Endangered Languages Archive. Examples labeled 'Offered' were proposed by native speakers as appropriate utterances for me to parrot. 'Elicited' examples are from targeted elicitation, either from translation of something someone said in context in Tok Pisin, or from transcribing naturalistic speech. While all recordings have the consent of participants to be public, any examples I feel present a concern are not accompanied by identifiers. Common everyday expressions are not cited. Translations aim to reflect the original Chini as much as possible without being too infelicitous in English. Translations that depart significantly from the Chini are labeled as free translations. Likewise, descriptive labels and glosses are not intended rigidly or as representations of universal concepts, but as tools to represent language-specific associations between form and meaning \citep{Reesink2008}.

\section{{Grammatical} {background}}\label{sec:brooks:2}

Here I provide an overview of the areas of the grammar relevant to the possessive reflexive and (especially) the middle construction, namely participant categories and how their semantic and pragmatic roles relate to clause structure and valency behavior. 

\subsection{{The} {noun} {phrase}}\label{sec:brooks:2.1}

Noun phrase structure is [noun][adjective][numeral] with mostly dependent-head order in genitive constructions. The position of deictic determiners is based on semantic scope. Nominal categories include a plural/non-plural relative number distinction (where ‘non-plural’ is semantically akin to a paucal), diminutive, augmentative, and authentic (i.e., an original version of something). Noun phrases are not flagged for core cases. Postpositional enclitics provide different kinds of semantic and/or pragmatic information about the role of the noun phrase in the clause. It is not unusual for multiple enclitics to co-occur. This allows for fairly complex ideas to be expressed in a single noun phrase, including (as it relates to reflexivity) autopathic concepts. In particular, concepts involving self-reflection tend to rely on roundabout (and often, translation-resistant) expressions, without overt reflexive material. Agusta said \REF{ex:brooks:1} after complaining her eyesight had become too poor to see her knitting properly:
\ea\label{ex:brooks:1}
\glll ku   pavimɨŋaŋgamika!\\
ku  pa=avi=mɨŋɨ=aŋgɨ=ami=k-a=a!\\
\textsc{1sg.nom}  before=\textsc{new=trans=lh.npl=sim=prox-def=excl}\\
\glt (Free translation) 'Woe am I now, in contrast to the bright-eyed me from before!'
{}[Agusta Njveni, afi021218m\_7:09]
\footnote{Certain graphemic conventions diverge from a phonemically-based orthography. Between vowels or glides, <g> represents the velar approximant /ɰ/. <ŋ> represents /ŋ/, but <ŋg> represents the prenasalized stop /\textsuperscript{ŋ}ɡ/. <g> is also used for [ɡ], an allophone of /\textsuperscript{ŋ}ɡ/ that occurs before /ŋ/. <h> represents the breathy voice quality of certain stops when it is phonemically contrastive (and co-occurs with ingressive airflow, which is not represented). Other instances of murmur are not represented. <c> occurs in <cm> to represent the voiceless palatal stop in the prestopped nasal /\textsuperscript{c}m/ and in <ch>, for the affricate /tʃ/. Other conventions include <v> for /β/, <ñ> for /ɲ/ (but note <nj> for /\textsuperscript{ɲ}ɟ/).}
\footnote{Example citations indicate the source of the original utterance. In addition to the speaker’s name, an identifier like ‘afi021218m\_7:09’ indicates the ISO code (afi), the date of the recording, the number of participants (s for ‘singular’, m ‘multiple’), and the time stamp.}
\z

\subsection{{Participant} {categories} {for} {pronouns} {and} {nouns}}\label{sec:brooks:2.2}

Whereas many Papuan languages are known for the reduced functional role of nominals in discourse (\citealt{DeVries2005}), in Chini the functional load of nouns and pronouns in referential tracking (among other uses) is high. The language has an abundance of core argument categories for object-like participants. These tend to be given lexical expression, especially Instruments. As a result, nominal-heavy clauses are not so uncommon in Chini discourse as they might be in other Papuan languages. Another reason for this relates to the fact that clause chaining in Chini is not based on reference. Instead, the chain linkage devices code dependency relations that demarcate topical information off from the comment, among other related discourse-pragmatic functions. This can be glimpsed in \REF{ex:brooks:2}, where the prosody and the chain linker =\textit{va} demarcate the topical background information from the following comment, which is headed by the final clause. The pragmatic unity between the two clauses in the comment is signaled by the linker =\textit{kɨ}. In each clause, reference is clarified by pronouns.

\ea\label{ex:brooks:2}
\ea
\glll ku   ŋgaŋgukŋimapava\\
ku   ŋgɨ=aŋgu.kŋi-m-apa=\textbf{va}\\
\textsc{1sg.nom}  \textsc{3sg.dat}=ask\textsc{-ipfv-r=}\textbf{\textsc{pre.r}}\\
\glt `I had been asking her (Dorin) but'

\ex
\glll anɨ  ŋɨrkŋɨ    nɨŋaviandikɨ\\
anɨ  ŋɨ=ɨrk-ŋɨ   nɨ=ŋɨ=avia.ndi=\textbf{kɨ}\\
\textsc{3sg} \textsc{poss.refl}=talk-\textsc{npl} \textsc{ins=1sg.acc}=withhold.\textsc{r=}\textbf{\textsc{cnt.r}}\\
\glt `she withheld her plans (lit. her talk) from me and'

\ex
\glll ku   yanɨ  pupmu   kuavɨyi.\\
ku   yanɨ  pupmu  ku-avɨ-yi\\
\textsc{1sg.nom}  just  alone   cross-\textsc{tloc.pc-r}\\
\glt `I went all alone to the other side of the river (to collect greens).'

[Dorothy Paul, afi051116m\_15:14]
\z
\z



\subsubsection{{Pronouns}}\label{sec:brooks:2.2.1}

The Chini pronouns can be seen in \tabref{tab:brooks:1} below. An initial (a) in 3\textsc{sg} forms indicates a dialect difference; the initial vowel is maintained only in the Akrukay dialect. Nominative and also the dual forms are unbound, all others are bound proclitics.

 In Chini, verbs that can be used transitively (that is, occurs with reference to object-like participants) are associated with one (or sometimes, more than one) pronominal object case. (Recall that nominals are not marked for case, however.) The three pronominal cases are Accusative, Dative, and Benefactive.\footnote{A handful of verbs take the Benefactive, for example: \textit{ndɨ}- ‘like, think of’, \textit{anu}- ‘worry about’, \textit{ayi}- ‘wait for’, \textit{kɨ}- ‘propel, kick, throw’. Others take the Dative: \textit{ñu}- ‘chase off, after’, \textit{aŋgu}- ‘request information’, the sense ‘hog someone’s time, be possessive over (someone)’ of \textit{amru}- ‘seize’, \textit{ndu}- ‘perceive (\textsc{pc})’. The majority take an Accusative: \textit{kɨ}- ‘tell’, \textit{amba}- ‘take care of (someone)’, \textit{amá}- ‘transport, take (someone somewhere)’, \textit{ŋgɨn}- ‘perceive (\textsc{pl})’.}

\begin{table}
\caption{Pronouns in Chini}
\label{tab:brooks:1}
\fittable{
\small
\begin{tabularx}{\textwidth}{p{1.2cm}p{0.8cm}p{0.5cm}p{0.6cm}p{0.7cm}p{0.6cm}p{1.7cm}p{0.6cm}p{0.6cm}p{0.6cm}}
\lsptoprule
& \textsc{distal} & \textsc{1sg} & \textsc{2sg} & \textsc{3sg} & 1\textsc{pl} & {1/2\textsc{/3pl}}\newline collective & {\textsc{1du}} & {\textsc{2du}} & \textsc{3du}\\
\midrule
\textsc{nom} &  x\multirow{4}{*}{mɨ} &  \shadecell ku &   \multirow{2}{*}{ nu} &  {anɨ} &  \multirow{2}{*}{añi} &  \multirow{2}{*}{ñi} & \multirow{4}{*}{aŋgɨ} & \multirow{2}{*}{{ŋgu}} &  \multirow{4}{*}{maŋuñi*}\\
\textsc{acc} & &  \multirow{2}{*}{ŋɨ} & &  {(a)nɨ} & & & & & \\
%\hhline%%replace by cmidrule{-~-----~~~}
\textsc{dat} & & &  \multirow{2}{*}{{ŋgu}} & {  { {(a)ŋgɨ}}} &  \multirow{2}{*}{anji} &  \multirow{2}{*}{ {nji}} & & & \\
%\hhline%%replace by cmidrule{-~-----~~~}
\textsc{poss} & &  { {ku}} & & & & & & & \\
\textsc{foc.poss} &  { {-}} &  { {-}} &  { {ɨnku}} &  { {ankɨ}} &  { {ainkɨ}} &  { {iŋkɨ}} & \multicolumn{3}{c}{ -}\\
\textsc{ben} & {{mbɨ}} &  {{mbɨ}} &  {{ndvu}} &  {{(a)ndvɨ}} &  {{anjvɨ}} &  { {njvɨ}} & \multicolumn{3}{c}{(co-occur with \textsc{ben}  {vɨ}=)}\\
\lspbottomrule
\end{tabularx}
}
*Lit. `those two', sometimes:  \textit{kaŋuñi} `these two'
\end{table}

The pronouns exhibit several divisions. The 1\textsc{sg} \textit{ŋɨ}= conflates Accusative and Dative case. 2\textsc{sg}, 3\textsc{sg}, and 1/2/3\textsc{pl}\footnote{The collective pronouns represent any 2 or more persons as a unit. The \textsc{du} and 1\textsc{pl} distinctive pronouns represent multiple persons in terms of some property of distinctiveness. Often the difference is subtle.} conflate Nominative and Accusative while distinguishing the Dative. As I discuss in \sectref{sec:brooks:2.3}, constituent order in object-initial main clauses justifies grouping Accusatives and Datives as ‘Patients’ in the sense of ‘the most semantically patient-like argument in a multivalent clause’. As I discuss in \sectref{sec:brooks:3}, the reflexive possessive pronoun \textit{ŋɨ}= refers to topical possessors.

\subsubsection{{Allatives,} {benefactives,} {and} {instruments} {as} {core} {participants}}\label{sec:brooks:2.2.2}

Any lexical noun (and some nominalized verb forms) having a certain semantic role of goal, beneficiary, or instrument is considered a core participant in Chini clause structure. That status is cross-referenced by a proclitic that attaches to the verb complex: Allative \textit{mɨ}=, Benefactive \textit{vɨ}=, and Instrumental \textit{nɨ}=. These language-specific categories exhibit some semantic variability, for instance nouns having the semantic role of goal or path count as Allatives:\footnote{Also apparent in \REF{ex:brooks:3} is the possiblity for a noun phrase marked by an adessive or vialis postposition to count as an Allative, a grey area in the core versus oblique distinction.}

\ea\label{ex:brooks:3}
\glll ku  Amɨŋarɨ  mayikɨ\\
ku  Amɨŋarɨ  \textbf{mɨ=}ayi=kɨ\\
\textsc{1sg.nom}  [Ramu\_river]\textsc{\textsubscript{all} } \textbf{\textsc{all}}\textbf{=}go/come\_upriver.\textsc{irr=cnt.r} \\

\glll achikɨ   tɨŋɨ    mayuku  yu.\\
 achi-kɨ  tɨ=ŋɨ    \textbf{mɨ=}ayuku  yu\\
 [upriver\textsc{-prox}  path=\textsc{adess]}\textsc{\textsubscript{all}} \textbf{\textsc{all}}\textbf{=}quickly go/come.\textsc{irr}\\
 \glt `I'll go upriver on the Ramu (River), going quickly on the upriver route.'
 [Dorothy Paul, afi260814m\_29:03]
\z


Instruments include concrete and abstract instruments, gifts, entities manipulated by human hands, certain roles and capacities, and adverbial manner.

\ea\label{ex:brooks:4}
\glll ka  ku  mmhɨ  nɨmɨnkɨ.\\
k-a  ku  mmhɨ  \textbf{nɨ=}mɨ=nkɨ\\
\textsc{prox-def}  \textsc{1sg.nom}  [bamboo]\textsc{\textsubscript{ins} } \textbf{\textsc{ins}}\textsc{=dist}=light.\textsc{r}\\
 \glt `This (the matchwood) I lit using the bamboo.' [Anton Mana, afi271016m\_12:17]
\z


The Benefactive indicates beneficiaries, maleficiaries, purposes, and reasons. As seen in \REF{ex:brooks:5}, this participant category is the only one shared by pronouns and nominals (here, a nominalized verb):%\todo{incomplete example}

\ea\label{ex:brooks:5}
\glll andvambrimbri varatmapaye {(...anɨ papmɨ mɨndavarka.)}\\
\textbf{andvɨ=}ambri{\textasciitilde}mbri   \textbf{vɨ=}ara-tm-apa-y-e { }  \\
\textbf{\textsc{3sg.ben}}=hurry{\textasciitilde}\textsc{nmlz}  \textbf{\textsc{ben}}=move\_along-\textsc{ipfv-r-z-ctrst} { }\\
\glt `I was on my way in order to hurry for him but (...he had forgotten all about it.)' [Emma Airɨmarɨ, afi051116m\_2\_15:59]
\z


Benefactive pronouns in fact conflate Benefactive and Allative functions. Pronominal recipients of directional transfers (from a source to a goal, along a path) take the Benefactive form. \REF{ex:brooks:6} concerns a soccer game that had taken place:

\ea\label{ex:brooks:6}
\glll ndvɨkavɨ!\\
\textbf{ndvɨ=}kɨ-avɨ\\
\textbf{\textsc{3sg.ben}}=propel-\textsc{tloc.opt.pc}\\
\glt `Kick (the ball) to him.' [Elicited, 2016 Fieldnotes]
\z

 My basic point here is that Chini biases its users to attend to specific types of participants, including ones not always thought of as candidates for core arguments (see \citealt{Mithun2005}). In the next section, I discuss some similarities and distinctions between Patients and Instruments when they pattern as topics in clause-initial position.

\subsection{{Pragmatically-determined} {constituent} {order} {in} {main} {clauses}}\label{sec:brooks:2.3}

Main clauses are verb-final, and the order of nominal constituents is pragmatically-based. For transitive clauses with a semantic agent (‘A’) and patient (‘P’), APV is the most frequent order. Like in \REF{ex:brooks:7}, it is used when A is the default topic-worthy argument. In this exchange between a folkloric husband and wife, the P argument has no special pragmatic status; the participation is normative and unremarkable in relation to the activity:

\ea\label{ex:brooks:7}
\glll ŋgɨmanɨ       ŋgaŋgukŋi   "nu  ŋgu  aryindani?"\\
ŋgɨ=manɨ      ŋgɨ=aŋgu-kŋi    nu  ŋgu  ar-yi-nda=n-i\\
\textsc{3sg.poss}=husband  \textsc{3sg.dat}=ask-\textsc{irr}  2\textsc{sg}  fish  catch-\textsc{irr-neg=z-q.irr}\\
 \glt `Her husband asked her: "Did you not catch any fish?" '
[Frank Mana, afi260612s\_1:19]
\z


 The construction that serves to activate the topicworthiness of a lexical Patient relies on clause-initial placement and a pronominal clitic cross-referenced on the verb complex.\footnote{The distal deictic \textit{mɨ}= is used mostly for non-humans. Human Patients are cross-referenced by their relevant (human) pronoun. Accusative \textit{ŋɨ}= is used for the 1\textsc{sg}. For the \textsc{2sg}, 3\textsc{sg}, and all \textsc{pl} persons, the Accusative or Dative is used, depending on the verb.} In \REF{ex:brooks:8}, Emma activates ‘sago’ as a topic, suggesting (in jest) to her addressee that he has been remiss in his work:

\ea\label{ex:brooks:8}
\glll anjɨgɨ  nu   mɨñu?\\
 {anjɨgɨ} nu   \textbf{mɨ}=ñu\\
 {sago} \textsc{2sg} \textbf{\textbf{dist}}{=carve.}\textsc{irr}\\
\glt `Are you ever going to (harvest) that sago?' [Emma Airɨmarɨ, afi250814m\_3:14]
\z


Instrument and Benefactive (but not Allative) participants may appear as topics (in initial position) and are cross-referenced on the verb complex just like topicworthy Patients:

\ea\label{ex:brooks:9}
\glll ...ayi  pirkɨ  añi manɨmɨñi.\\
ayi  pir-kɨ  añi ma=\textbf{nɨ=}mɨ=ñi\\
{}[something  bad-\textsc{npl}]\textsc{\textsubscript{ins}} 1\textsc{pl}  \textsc{foc=}\textbf{\textsc{ins}}=\textsc{dist}=get.\textsc{r.pc}\\
\glt `(The money, we didn’t get it in a good way...) it was by something bad (by selling cannabis) that we got it.' 
\z

A topicalized object may pattern as both Patient and Instrument. In \REF{ex:brooks:10} \textit{vrinkɨ} 'reeds' occurs in clause-initial position as a topicworthy participant. It is cross-referenced on the verb as an Instrument (by the first \textit{nɨ}= in the clause, whereas the second \textit{nɨ}= refers to the fire as a second Instrument), due to the alteration of its state by human hands. As the affected participant, it is also a Patient, as indicated by \textit{mɨ}=:%\todo{incomplete example}

\ea\label{ex:brooks:10}
\glll {(gwu nɨmɨkavɨ!)} vrinkɨ  \textbf{nɨ}gwu   nɨ\textbf{mɨ}kavɨmɨ...\\
{ } vrinkɨ  \textbf{nɨ=}gwu  nɨ=\textbf{mɨ=}kɨ-avɨ=mɨ\\
    { } reed.\textsc{pl} \textbf{\textsc{ins}}=fire  \textsc{ins=}\textbf{\textsc{dist}}=throw-\textsc{tloc.opt.pc=pre.irr}\\
\glt `(Set fire to it!) Set fire to the reeds (...and then the dogs will kill the pig as it emerges).' [Alfons Garɨmbɨni, afi160714m\_8:43]
\z


My point here is that the Chini patterns evince a more complex array of possibilities for participant roles than the term ‘object’ implies (\citealt{MithunChafe1999}). At the same time, object-initial clauses do evince a participant category of Patient.

\section{{The} {reflexive} {possessive} {construction} {(\textit{ŋɨ}})}\label{sec:brooks:3}

Here I describe the uses of the reflexive possessive pronoun \textit{ŋɨ}=, the only \textit{bona fide} reflexivizer in Chini. In \sectref{sec:brooks:3.1} I show how many examples reflect the common analysis of reflexive relations in terms of clause-internal coreference (between possessor and syntactic subject). Then, in \sectref{sec:brooks:3.2} I discuss how other examples point to topics and (to a lesser extent) agents (rather than subjects) as coreferential with reflexive possessors. This can be seen in instances of partial coreference but also clause-external coreference, where the discourse topicality of the antecedent possessor supercedes the topicality of the subject in the clause where \textit{ŋɨ}= appears.

\subsection{{Clause-internal} {coreference} {between} {subject} {and} {possessor}}\label{sec:brooks:3.1}

In \REF{ex:brooks:11}, the 2\textsc{sg} possessor is straightforwardly coreferential with the subject:%\todo{incomplete example}

\ea\label{ex:brooks:11}
\glll “nu \textbf{ŋɨ}manɨ kɨramɨ   {(...anɨ avɨgɨtɨ mayi.)”}\\
nu \textbf{ŋɨ=}manɨ  kɨ-ra=mɨ { } \\
\textsc{2sg}  \textbf{\textsc{poss.refl}}\textbf{=}husband   tell-\textsc{opt=pre.irr} { } \\
 \glt ` "You tell \textbf{your} husband (...he must come down and spear the crocodile)." '
 [Anton Mana, afi260514s\_2:28]
\z


Note that this construction is also used for reciprocal possession (English: ‘each other’s’):

\ea\label{ex:brooks:12}
\glll añi mɨyi  vɨndɨ   mɨ,  añi  \textbf{ŋ}ɨrkŋɨ akikina?\\
    añi mɨ-yi  vɨ-ndɨ  mɨ   añi \textbf{ŋɨ=ɨrk-ŋɨ}  \textbf{aki{\textasciitilde}ki=n-a}\\
 \textsc{1pl} \textsc{dist-}what \textsc{ben}-think  \textsc{dist} \textsc{1pl}  \textbf{\textsc{poss.refl}}=talk-\textsc{npl}  spear\textsc{{\textasciitilde}ipfv=z-q.r}\\
 \glt `Why do we not heed/deflect (lit. spear) \textbf{each} \textbf{other's} talk?'
 [Dorothy Paul, afi260814m\_34:55]
\z


In general, when the possessor referent is not the subject (or established topic), a non-reflexive possessive pronoun is used (\tabref{tab:brooks:1}). Here Emma uses the non-reflexive collective possessive \textit{nji}= as she complains about a very relatable problem:

\ea\label{ex:brooks:13}
\glll ainkɨtwavɨŋgayi aŋri \textbf{nji}rkŋɨ   ŋgɨnɨmichinda.\\
ainkɨ=twavɨŋgayi aŋ-ri \textbf{nji=}ɨrk-ŋɨ   ŋgɨnɨ-m-i-chi-nda\\
\textsc{1pl.foc.poss}=child.\textsc{pl} man-\textsc{pl}  \textbf{\textsc{pl.poss}}=talk-\textsc{npl} perceive\textsc{-ipfv-irr-z-neg}\\
 \glt `The young men of ours don't listen (lit. perceive/heed \textbf{any} \textbf{of} \textbf{our} talk).'
[Emma Airɨmarɨ, afi260814m\_34:59]
\z


A possessor in a phrasal afterthought takes the non-reflexive form. The prosodic break (here, a pause indicated by the comma) between the clause and the phrasal afterthought is enough for the latter to be treated as clause-external:

\ea\label{ex:brooks:14}
\glll mumuŋu  ŋakɨ ɨvki, \textbf{ŋg}ambɨgɨ. \\
 mumuŋu  ŋa-kɨ ɨvk-i \textbf{ŋgɨ=}ambɨgɨ\\
auntie   riverwards-\textsc{prox}  be.sitting.\textsc{pc-irr}  \textbf{\textsc{3sg.poss}}=house\\
\glt `Auntie (Agusta) is sitting over there riverwards, (in) \textbf{her} house.'
 [Anton Mana, afi111016m\_43:41]
\z


Reflexive possessors need not be human, so long as the animal \REF{ex:brooks:15} or inanimate entity \REF{ex:brooks:16} is an agentive topic:

\ea\label{ex:brooks:15}
\glll chavɨ   \textbf{ŋɨ}miatmɨ   nɨŋaurua.\\
chavɨ \textbf{ŋɨ=}miatmɨ   nɨ=ŋɨ=auru-a\\
poison.frog  \textbf{\textsc{poss.refl=}}poison  \textsc{ins=1sg.acc}=wash-\textsc{r}\\
\glt `The poison frog shot ('washed') me (in the eye) with its poison.'
[Anton Mana, 2018 Fieldnotes, offered example]
\z


\ea\label{ex:brooks:16}
\glll mɨ\textbf{ŋ}atugu mɨchagɨyi.\\
mɨ=\textbf{ŋɨ=}atugu mɨ=chagɨ-yi\\
\textsc{dist=}\textbf{\textsc{poss.refl}}\textbf{=}limit  \textsc{all}=arrive-\textsc{r.pc}\\
\glt `It has reached its limit.' [Anton Mana, 2014 Fieldnotes, offered example]
\z


Note that non-reflexive animal and inanimate possessors rely on the distal deictic \textit{mɨ}=:

\ea\label{ex:brooks:17}
\glll \textbf{mɨ}yẽntmɨ ara.\\
mɨ=yim-tmɨ ar-a\\
\textbf{\textsc{dist}}=chew.betel.nut\textsc{-nmlz}  good\textsc{-r}\\
\glt `\textbf{Its} (the meat of the betel nut in question) chewing is good (for getting a buzz).'
[Alfons Garɨmbɨni, 2014 Fieldnotes, offered example]
\z


\subsection{{Partial} {coreference} {between} {topic} {(or} {agent)} {and} {possessor}}\label{sec:brooks:3.2}

Coreference does not always involve full identity of the possessor with the subject, however. In instances of partial coreference, the possessor almost always refers to the more topicworthy member within a plural subject. Ros addressed \REF{ex:brooks:18} to Anton and me as we emerged from the bush in her part of the village. The possessor and topic is me (not me and Anton, since the recently deceased woman Ikivim is my classificatory grandmother but Anton's aunt). The reference of the possessor and its topicworthiness is then reinforced in the 3\textsc{sg} benefactive pronoun \textit{ndvɨ}=.

\ea\label{ex:brooks:18}
\glll na  ñi   ŋɨñinmɨ   aŋgɨnɨ ndvɨmbruindani?\\
 na  ñi \textbf{ŋɨ}=ñinmɨ aŋgɨnɨ ndvɨ=mbru-i-nda=n-i\\
and  \textsc{pl}   \textsc{poss.refl}=maternal.anc banana \textsc{3sg.ben}=cut-\textsc{irr-neg=z-q.irr}\\
\glt `And did you lot not cut any (savory) bananas of \textbf{his} (i.e. me) (deceased) grandmother’s (i.e. her garden) for him?' [Ros Njveni, afi111016m\_44:50]
\z


Similarly, in \REF{ex:brooks:19} the partial coreference is based on the topical participant within a plural subject. That participant is a (folkloric) village man, as introduced in the first clause and as understood as the protagonist of the folktale. He is a subset of the plural subject (i.e., the villagers who carried the pig along with him to his homestead):

% \todo{incomplete example}
\ea\label{ex:brooks:19}
\gll ramɨ anɨ mayindaka,\\ \\
 \glt `He (the village man) shot the pig and then,'

\glll ñi manjurakɨ   chakɨ \textbf{ŋɨ} ŋgɨgɨ   mɨga...\\
ñi mɨ=anjur-a=kɨ  ch-a=kɨ  \textbf{ŋɨ=}ŋgɨgɨ   mɨ=g-a\\
\textsc{pl}  \textsc{dist}=carry-\textsc{r=cnt.r}  ascend-\textsc{r=cnt.r}  \textbf{\textsc{poss.refl}}\textbf{=}homestead  \textsc{dist}=set.down-\textsc{r}\\
\glt `they (the villagers, including the man) carried it, went up, and laid it down in \textbf{his} homestead... ' [Paul Guku, afi100514s\_12:07]
\z


In one specialized construction, the interpretation of the reference of the possessor hinges on semantic agency rather than pragmatic topicworthiness. This construction expresses accompaniment or “attendant action” (\citealt{ZaliznjakShmelev2007}:214). Its function is based on asymmetries in agency within a plural subject, where one member merely attends the action and is not an agent. Of the two members of the subject in \REF{ex:brooks:20}, the wife is expressed as the agent, since she is headed to her matrilineally inherited bush ground with her husband, who merely accompanies her.\footnote{There is an underlying cultural component that drives the use of this construction. It is often used to describe movements into the bush. In Chini society, the bush is subdivided into chunks, each associated with a particular moiety and associated subclan. (Spouses belong to opposing moieties.) The chunks are inherited through a system of mostly matrilineal land tenure according to moiety and clan membership. So, the agent in these situations is that person whose clan owns the land. In Chini they are referred to as \textit{mbɨpapayaŋgɨ} ‘the one who goes first to it’. Just as that person (the candidate for the topical agent in this construction) ‘goes first’, their spouse (or other associate) is seen as accompanying them.}

% \todo{incomlete example}
\ea\label{ex:brooks:20}
Aŋgwamɨ pata ŋgɨmanɨ, maŋuñi, bmu nɨgɨ, maŋuñi anjɨgɨ vuwuyi.


\glt `Aŋgwamɨ and her husband, those two, a day later they went to (harvest) sago.'

\glll \textit{maŋuñi} \textbf{ŋɨ}manɨnmɨ avkɨkɨ anjɨgɨ  ŋumapa.\\
maŋuñi \textbf{ŋɨ=}manɨ=nmɨ av-kɨ=kɨ anjɨgɨ  ŋu-m-apa\\
\textsc{3du} \textbf{\textsc{poss.refl}}=husband=\textsc{accom} descend-\textsc{r.pc=cnt.r}  sago  carve-\textsc{ipfv-r}\\
\glt `The two of them, (she) with \textbf{her} husband went down (to the bush) and harvested sago.' [Anton Mana, afi051116s\_0:51]
\z


\subsection{{Clause-external} {coreference} {between} {topic} {and} {possessor}}\label{sec:brooks:3.3}

The above examples of full and partial coreference uphold the general view of reflexive relations as a clause-internal matter. However, examples from Chini discourse reveal that reflexivity can involve clause-external coreference. Such uses arise when the discourse topicality of an antecedent supercedes that of the subject, for instance in long stretches of discourse like clause chains where multiple subjects are introduced. The chain in \REF{ex:brooks:21} is about an oxbow marsh that several Andamang villagers share with a neighboring village called Watabu. The subject in the third line below is elided, but it is clear from the context that it would be the Awakŋi boys (\textit{agŋiŋri}) fencing off the marsh. It is also clear that the discourse topic (and possessor) is not the boys themselves, but rather the Awakŋi owners of their half of the marsh (Anton and his family), the 'we' from the first clause:\footnote{The boys, while potentially a subset of the 1\textsc{pl} argument in the first clause, are not so easily identified as such. The marsh belongs to a specific clan. The event has also not yet occurred, and the boys represent multiple clans. So, these two referents turn out to represent separate topics. \citet{Comrie1998} points out how breaks in topic continuity often motivate the use of more marked prononimal forms to reactivate the discontinuous topic. However Chini does not distinguish pronouns in this way.}

\ea\label{ex:brooks:21}
\glll \textbf{añi}  ŋɨyãrkŋɨ ndumɨ,  \\
\textbf{añi} ŋɨ=yãrkŋɨ ndu=mɨ\\
\textbf{\textsc{1pl} } \textsc{poss.refl}=’side.of.things’  perceive.\textsc{pc.mod=pre.irr}\\
\glt `\textbf{We} need to attend to our side of things so,'

\glll agŋiŋri rindata vienɨ agarɨndata, \\
agŋi-ŋri ri=nda-ta vienɨ ag-arɨ=nda-ta\\
post.initiate.boy-\textsc{pl}  head.downriver.\textsc{mod=seq-irr}  sago.palm.frond  cut-\textsc{mod=seq-irr}\\
\glt ‘once the (older) boys have gone downriver and cut dried sago palm fronds and,'

\glll \textbf{\textit{ŋaŋgɨ} } \textit{tɨrɨmɨ...} \\
 \textbf{ŋɨ=aŋgɨ}  tɨ-rɨ=mɨ\\
\textbf{\textsc{poss.refl==lh.npl}} cut.\textsc{pc-mod}=\textsc{pre.irr}\\
 \glt `fenced off (lit. cut) \textbf{ours} (side of the marsh)...' [Anton Mana, afi260814m\_1:57]
\z


For the possessive reflexive construction, coreference most often involves full identity between the possessor and the topical subject. Partial coreference and the possibility for clause-external coreference with a topical antecedent reveal that possessive reflexivity may be more complex than clause-internal relations between syntactic categories. Where clause-external coreference is concerned, some explanation may be found in the potential for newly introduced subject participants to be ephemeral in discourse versus topics which are established as given and definite, and thus more highly recoverable from context \citep{Lambrecht1994}. In other words, highly topical participants enjoy high candidacy for coreference as reflexive possessors, and may in that capacity override subjects (cf. \citealt{Reesink1983}).

\section{{The} {middle} {construction} {(\textit{nji})}}\label{sec:brooks:4}

Here I describe the workings of the Chini middle, formed by the verbal prefix \textit{nji-} (or the proclitic \textit{nji}= in a few specialized constructions discussed in ). There are no reflexive pronouns beyond the possessive \textit{ŋɨ}, and so the middle construction is the primary grammatical expression for autopathic and mutual relations. As I discuss here, the function of the middle is to represent the action of the verb events as affecting (rather than being fully controlled by) the main participant. That is, the main participant in a middle-marked clause is essentially a semantic patient.\footnote{Middle situation types in Chini correspond mostly to Kemmer’s (1993, 1994) findings, with some exceptions. In Chini, middles are mostly not used for changes in body posture, emotive speech actions, cognition, or grooming. Like other languages, Chini middles are characterized somewhat by lexical idiosyncrasy. The generic verbs for ‘grow’ include a middle for human and animal growth, but an unmarked intransitive for plant growth.} My main focus here will be on illustrating how this function interrelates with autopathic and mutual semantic readings. I argue that those readings are strongest when the main participant has significant control over the action, and much weaker the less control they perceive to have.\footnote{By ‘mutual’ events I refer to \citegen{Nedjalkov2007} work on reciprocals, where participants act “to/of/against/from/with each other” \REF{ex:brooks:6}. I generally follow \citet{Haspelmath2020} in reserving ‘reflexive’ and ‘reciprocal’ for grammatical markers. I also use them to refer to those middles where reflexive or reciprocal meanings are always involved. For middle verbs where such meanings are more tenuous or a matter of interpretation, I use the terms ‘autopathic’ and ‘mutual’.}

 The current documentation records 70 middle verb forms in Chini, which corresponds to approximately 20\% of the verbal lexicon (where middles are considered separate lexemes, either as deponents or as derivations of non-middle counterparts). Historically, the Chini middle appears to predate the diversification of the Tamolan subgroup. This is hinted at by cognate middle forms and their unmarked transitive counterparts for ‘bathe’ and ‘wash’ in Chini’s nearest relatives \citep{Zgraggen1974}. The historical relation to the plural collective dative pronoun of the same form, \textit{nji}= (see \tabref{tab:brooks:1}), is unclear, but the two are almost certainly related. In what follows, I give a brief overview of the transitivity patterns of middles \sectref{sec:brooks:4.1} In \sectref{sec:brooks:4.2} I discuss the semantics of middles in terms of how the presence, absence, or mitigated control yields differences with respect to autopathic (and/or mutual) interpretations.

\subsection{{Transitivity} {patterns} {and} {argument} {structural} {behavior} {of} {middles}}\label{sec:brooks:4.1}

Middles exhibit a range of possibilities with respect to their unmarked counterparts:

\begin{table}
\caption{Transitivity patterns for unmarked counterparts}
\label{tab:brooks:2}
\begin{tabularx}{\textwidth}{QQQ}
\lsptoprule
 {Transitivity} {pattern} {of} {counterpart} & {Unmarked} {counterpart} & {Middle} {form}\\
 \midrule
No known counterpart (Deponent forms) & - & \textit{njimim}- 'urinate, shoot projectile poison'\\
& & \textit{njagi-}\glt `paddle (a canoe)'\\
%\tablevspace 
 Intransitive & \textit{ch}- 'exist, live, be left/remain' & \textit{njich}- 'exist unto itself/oneself, let something/someone be, nevermind’\\
& \textit{pu}- ‘get upset’ & \textit{njipu}- 'thrash about, get all riled up'\\
%\tablevspace 
 Ambitransitive & \textit{mbɨn}- ‘last (time); well up (water); increase in pressure; pressure someone; stop by pressing (e.g. a recorder)' & \textit{njimbɨn}- 'dry up (e.g. a swamp)'\\
& \textit{pu}- 'float; set afloat, adrift (\textsc{tloc})' & \textit{njipu}- 'drift off (downriver) (\textsc{tloc})'\\
%\tablevspace 
 Transitive & \textit{yirɨv}- 'turn (something) over' & \textit{njiyirɨv}- 'avert one's gaze'\\
& \textit{yu}- ‘pick/lift up’ & \textit{njiyu}- ‘jump up, onto’\\
%\hhline%%replace by cmidrule{~--}
\lspbottomrule
\end{tabularx}
\end{table}

Note that the evidence does not quite support an analysis of \textit{nji}- as a syntactic valency-decreasing device.\footnote{Transitivity in Chini is best described as semantically based. The coding frames and argument structural combinations of any given verb depend to a great extent on lexical semantics. For some verbs, the patterns generally cohere with the semantic maps fine-tuned by \citet{ComrieEtAl2015}. However, area- and language-particular conceptualizations of verbal meanings also play a major role (cf. \citealt{Pawley2000}). For example, the verb \textit{ám}- 'cook' never indicates an accomplishment, only an (intransitive) activity. The affected participant of \textit{mu}- 'become dusk' is obligatorily (transitively) expressed (\textit{bmu ŋɨmu} 'dusk dusked me'). For some ambivalent verbs, intransitive and transitive uses hardly differ: \textit{nju}- 'bear offspring (\textsc{intr}); give birth to (\textsc{tr})'. For others, intransitive versus transitive meanings are more distinct: \textit{nji}- 'reside, be settled, settle (one's body) into a spot (\textsc{intr})' but: 'set something down in upright position; plant sweet potato, taro, sugar cane, greens (\textsc{tr})'.} While most middles may have transitive counterparts, this reflects the much greater proportion of transitive-patterning to intransitive-patterning verbs in the lexicon. The presence of intransitive counterparts and the occasional unpredictability of the argument structural alternations that occur between transitive-middle pairs suggest that \textit{nji-} does not function to decrease valency (even if decreased valency is often characteristic of clauses headed by middles). The middle form \textit{njiyɨyiyi}- means ‘scratch (oneself)’ but its transitive counterpart \textit{yɨyiyi}- means ‘itch’ as in “my skin itches me” (and not: “(someone else) scratches me”). As in \REF{ex:brooks:24}, some middles can even take patient-like objects. The patterns can be understood as syntactic effects of underlying semantic principles.

\subsubsection{{Argument} {structural} {behavior} {of} {middles}}\label{sec:brooks:4.1.1}

In \sectref{sec:brooks:2.2.1} I described how verbs that take an object-like participant are associated with Accusative, Dative, or Benefactive participant categories. It is precisely these argument types that rarely co-occur with middles. This can be seen in the middle forms of the paucactional \REF{ex:brooks:22} and pluractional \REF{ex:brooks:23} roots for ‘perceive, know’. The former (\textit{ndu}-) specifies a Dative, the latter (\textit{ŋgɨn}-) an Accusative. The erstwhile Benefactive is exemplified in \REF{ex:brooks:24}. Reflexive (or reciprocal) relations can be based on coreference between the subject and any of these three object-like participant types:

\ea\label{ex:brooks:22}
\glll agŋiŋri   agamkɨ  \textbf{nji}nduindaka...  \textup{(Erstwhile Dative)}\\
agŋi-ŋri   agamkɨ  \textbf{nji-}ndu.i=nda-ka...\\
post.initiate.boy\textsc{-pl}  all   \textbf{\textsc{mid}}\textbf{-}perceive.\textsc{pc.r=seq-r}\\
\glt `All the boys looked at each other and then...' [Anton Mana, afi021218m\_27:16]
\z


\ea\label{ex:brooks:23}
\glll agŋiŋri   agamkɨ  \textbf{nji}ŋgɨninda.  \textup{(Erstwhile Accusative)}\\
agŋi-ŋri   agamkɨ  \textbf{nji-}ŋgɨn-i-nda\\
post.initiate.boy\textsc{-pl}  all   \textbf{\textsc{mid-}}perceive\textsc{.pl-irr-neg}\\
\glt `None of the boys looked at each other.' [Elicited example, 2018 Fieldnotes]
\z

\ea\label{ex:brooks:24}
\glll anɨ  ñimɨŋɨ   nɨ\textbf{nji}kavɨ.   \textup{(Erstwhile Benefactive)}\\
anɨ  ñimɨŋɨ   nɨ=\textbf{nji-}kɨ-avɨ\\
3\textsc{sg}  black   \textsc{ins=}\textbf{\textsc{mid}}\textbf{-}propel-\textsc{tloc.r.pc}\\
\glt `He painted himself black.' [2014 Fieldnotes]
\z

As \REF{ex:brooks:24} also illustrates, middle clauses need not have monovalent argument structure. The most common multivalent pattern is the inclusion of an Instrument. Although object-like participants are generally absent in middle clauses, it is nevertheless possible for some middles to co-occur with a patient-like argument. Consider the use of \textit{njag-} ‘surpass, put clothes on upper body’ in \REF{ex:brooks:25}:

\ea\label{ex:brooks:25}
\glll achamɨ  \textbf{nj}ara!\\
 achamɨ  \textbf{nj}ara\\
clothing.item  \textbf{\textsc{mid}}-put.clothes.on.upper.body.\textsc{opt-opt}\\
 \glt `Put a shirt on!' [Offered example, Anton Mana, 2014 Fieldnotes]
\z

\subsection{{Uses} {of} {the} {Chini} {middle}}\label{sec:brooks:4.2}

Uses of the Chini middle have in common the expression of a general type of action where, whatever degree of control the main participant has, they become affected or altered by it in the course of their participation. In \sectref{sec:brooks:4.2.1} I discuss how, while the majority of uses and lexical meanings include reflexivity (or reciprocity), that inclusion hinges upon the degree of control of the agent. In \sectref{sec:brooks:4.2.2}, I discuss extensions of middle marking.

\subsubsection{{Three} {semantic} {subtypes} {of} {Chini} {middles}}\label{sec:brooks:4.2.1}

In \sectref{sec:brooks:4.2.1.1} I discuss reflexive and reciprocal middles, where the main participant is equally agent and patient. In \sectref{sec:brooks:4.2.1.2} I discuss unaccusative middles, where the main participant is purely a patient. In \sectref{sec:brooks:4.2.1.3} I discuss the partially autopathic middles for verbal actions where the control of the agent is mitigated or otherwise ambiguous.

\subsubsubsection{{Reflexive} {and} {reciprocal} {middles}}\label{sec:brooks:4.2.1.1}

A common understanding of middles is a situation where “the participant both performs and undergoes the event” \citep[1563]{Lichtenberk2007}. This is the most general and frequently encountered situation type for Chini middles, both in discourse and as represented in the lexicon. Drawing on \citegen{Kemmer1994} notion of the relative elaboration of events in terms of participants, three possibilities in Chini are shown in \tabref{tab:brooks:3}. While some events are interpretable as autopathic (agents acting upon themselves), others are mutual (agents acting upon each other), while some may be interpreted either way as dependent on context.

\begin{table}
\caption{Autopathic and mutual interpretations of reflexive middles}
\label{tab:brooks:3}
\begin{tabularx}{\textwidth}{QQQ}

\lsptoprule

 {Transitive} {counterparts} {(unmarked)} & { {Middle-derived} {forms}}
 {(\textit{nji}}{-)} & { {Elaboration} {of} {events}}\\
\midrule
 (no known counterpart) & \textit{njag-} ‘surpass, put shirt on (oneself)’ & { Strong autopathic interpretation}\\
& \textit{njiŋgɨ-} ‘put trousers on (oneself)’ & \\
%\hhline%%replace by cmidrule{--~}
 \textit{aku}- ‘pull (something) out’ & \textit{njaku}- ‘push (oneself, itself) out, sprout’ & \\
%\hhline%%replace by cmidrule{--~}
 \textit{ña}- ‘hide (something)’ & \textit{njiña}- ‘hide (oneself)’ & \\
 \textit{auru}- ‘wash (something)’ & \textit{njauru}- ‘bathe (oneself, each other)’ & {{ Strong autopathic or mutual interpretation}

 (based on participant number, context)}\\
 \textit{apri}- ‘teach (someone)’ & \textit{njapri}- ‘learn (teach oneself, each other)’ & \\
%\hhline%%replace by cmidrule{--~}
 \textit{yiru-} ‘declare, call out, name’ & \textit{njiyiru-} ‘designate (oneself, each other)’ (also: ‘claim’) & \\
%\hhline%%replace by cmidrule{--~}
 \textit{aigŋ}- 'write, draw' & \textit{njaigŋ}- 'decorate (oneself, each other) in traditional paint, garb (for dance songs)' & \\
 (no known counterpart) & \textit{njigwri-} ‘argue’ & { Strong mutual interpretation}\\
& { \textit{njiŋgɨ-} ‘race (each other);}

 talk over (each other)’ & \\
%\hhline%%replace by cmidrule{--~}
 \textit{akɨ-} ‘marry (one’s partner)’ & \textit{njakɨ-} ‘marry (each other)’ & \\
%\hhline%%replace by cmidrule{--~}
 \textit{achim-} ‘amass, collect, gather’ & \textit{njachim-} ‘meet (up), gather (each other)’ & \\
%\hhline%%replace by cmidrule{--~}
{ \textit{agɨ}- 'press against,}

 push (someone)' & \textit{njagɨ}- 'be stuck, crammed together' & \\
%\hhline%%replace by cmidrule{--~}
 \textit{ayi}- ‘help (someone) out’ & \textit{njayi}- ‘help (each other) out’ & \\
%\hhline%%replace by cmidrule{--~}
\lspbottomrule
\end{tabularx}
\end{table}

 While the autopathic or mutual reading of many middle verb forms is uncontroversial (e.g., \textit{njiña}- ‘hide oneself’), some arise via a Chini-specific interpretation of events. The middle form \textit{njaku}- is used to express (among other things) the sprouting of a plant. Upon comparison with its transitive counterpart \textit{aku}- ‘pull (something) out’, the Chini expression of a plant sprouting (\textit{njaku}-) involves the (conceptually autopathic) action of the plant “pushing itself out”.

 Unlike reflexive constructions in many European languages for instance, in Chini, middles rarely involve part-whole relations, but there are a handful of middles that do. In addition to the differentiation between clothing one’s upper (\textit{njag}-) versus lower body (\textit{njiŋgɨ}-) (both deponent forms), transitive \textit{yirɨv}- ‘turn (something) over’ pairs with the middle \textit{njiyirɨv}- which means ‘avert (one’s) gaze (i.e., in shame)’. The transitive verb \textit{tɨ}- ‘plant a garden, tubers’ has a middle counterpart \textit{njitɨ}- with part-whole semantics related to self-decoration:

\ea\label{ex:brooks:26}
\glll ...ayemŋgra nɨnjitɨga.\\
 ayemŋgr-a nɨ=nji-tɨ-ga\\
bird.of.paradise-\textsc{npl}   \textsc{ins=mid-}plant-\textsc{r}\\
\glt `...planted bird of paradise (feathers) (in their own hair)'
[Ayirɨvɨ Mana, afi140514s\_4:47]
\z


In Chini, some situations commonly expressed by reflexivizers or middles cross-linguistically are expressed by other means, for instance by unmarked intransitives (e.g., \textit{ambia}- ‘boil’). Some situations are hardly expressed at all. What might be normal autopathic construals of events for an English speaker can prove absurd in the Chini sociocultural world (e.g. ‘giving a gift to oneself’). Certain private autopathic actions like ‘speaking to oneself’ are in Chini expressed in terms of ‘doing X \textit{alone’}. It is only once multiple participants are involved, that a middle form can be used to express the event (and then, to express mutual relations):

\ea\label{ex:brooks:27}
\glll apwatɨ   mɨkɨnɨŋirati... ma   añi  ikɨ  \textbf{njichi}.\\
 apwatɨ  mɨ=kɨ-nɨŋi-ra-ti  m-a   añi  ikɨ \textbf{nji-ch-i}\\
out.in.the.open  \textsc{all}=propel\textsc{-tloc-irr-neg  dist-def} \textsc{1pl} only \textbf{\textsc{mid}}\textbf{-talk}\textbf{\textsc{-irr}}\\
\glt `Don’t throw it out in the open... that, we shall only \textbf{discuss} \textbf{amongst} \textbf{ourselves}.' [Ayirɨvɨ Mana, afi040814m\_29:58]
\z


While the use of socially antagonistic verbs (‘hate/kill/criticize/demean oneself’) to express certain autopathic actions is standard in many languages, Chini linguistic practices (including in Tok Pisin) do not make use of such intentionally self-destructive concepts, at least not in overtly autopathic terms. A few middle forms do involve mutual actions with socially antagonistic verbs: \textit{njaki-} ‘fight’ (based on its transitive counterpart \textit{aki}- ‘attack, shoot with spear/arrow’), and the deponent form \textit{njigwri}- ‘argue’.

\subsubsubsection{{Unaccusative} {middles}}\label{sec:brooks:4.2.1.2}

Unaccusative middles involve a main participant that exerts no control over the situation that affects them. If an agent is involved, they are clause-external. Their defining characteristic is how straightforwardly their meanings are copied from their unmarked transitive counterparts (see \tabref{tab:brooks:4} below). \citegen{Haspelmath2016} distinction between ‘automatic’ and ‘costly’ unaccusative meanings ia useful here. The unaccusative middles in Chini refer (mostly) to automatic situations (i.e., which need not involve external energy input) while their transitive counterparts refer to costly situations (and require external energy input). At least three situation types are distinguished.

\begin{table}
\caption{Unaccusative middles}
\label{tab:brooks:4}
\begin{tabularx}{\textwidth}{XXX}

\lsptoprule
 {Unmarked} {counterparts*} & {Middle-derived} {forms} {(\textit{nji}}{-)} & { {Situation} {type}}\\
\midrule
 \textit{vua}\~{} - “” & \textit{njivuã}- ‘break, burst, crack (via multiple fissures or holes)’ & {{ Unaccusative}

 destruction}\\
 \textit{aivɨ-} (\textsc{pc}), \textit{ayima}- (\textsc{pl}) “” & { \textit{njaivɨ}- (\textsc{pc}), \textit{njayima}- \textsc{(pl})}

 ‘break and collapse (tall narrow things)’ & \\
%\hhline%%replace by cmidrule{--~}
 \textit{irk}- (\textsc{pc}), \textit{mbu-} (\textsc{pl})“” & { \textit{njirk}- (\textsc{pc}), \textit{njimbu}- (\textsc{pl})}

 ‘break, cut (into separate parts)’ & \\
%\hhline%%replace by cmidrule{--~}
 \textit{ŋu}- “(Eng. fell)” & { \textit{njiŋu}-}

 ‘fall (mature, non-palm trees only)’ & \\
 (no known counterpart) & \textit{njiyɨvr}- 'grow, change in size' & {{ Unaccusative}

{ appearance}

}\\
{ \textit{vr}- ‘be unable or unwilling}

 to perceive or use’ & \textit{njivr-} ‘become unrecognizable’ & \\
%\hhline%%replace by cmidrule{--~}
 \textit{agɨ}- 'split into separate parts' & \textit{njagɨ}- 'split, fork (a road or river)' & \\
%\hhline%%replace by cmidrule{--~}
 \textit{yu}- 'pick/lift up' & \textit{njiyu}- '(re)surface (the riverbed)' & \\
 \textit{kɨ}- 'remove from enclosed space' & \textit{njikɨ-} 'come loose, fall from enclosed space' & {{ Unaccusative}

 movement}\\
{ \textit{pu}- ‘float in place (\textsc{tloc,} \textsc{intr});}

 set adrift (\textsc{tloc,} \textsc{tr})’ & \textit{njipu}- ‘be adrift (\textsc{tloc})’ & \\
%\hhline%%replace by cmidrule{--~}
\lspbottomrule
\end{tabularx}
* “” indicates identical meaning for transitive counterpart except in terms of agency
\end{table}

There is one verb whose event type is outside those identified in \tabref{tab:brooks:4}. Uses of the unmarked (ambitransitive) verb \textit{mba}- 'deceive, mislead, do/behave improperly' imply control of the main participant over the deception (including telling an actual lie):

\ea\label{ex:brooks:28}
\glll na  nu   mɨnɨgɨ  ndvɨrkɨkɨ   mbãmhichi?\\
 na  nu   mɨ=nɨgɨ ndvɨ=ɨr-kɨ=kɨ   mba-mh-i=ch-i\\
 and  \textsc{2sg}  \textsc{dist}=another  \textsc{3sg.ben}=cut.\textsc{pc-r}=\textsc{cnt.r}   mislead-\textsc{ipfv-irr=z-q.irr}\\
 \glt `And as if you had cut some (savory bananas) for him, now here you are   being misleading (i.e. acting as if he had behaved properly according to  expectation).' [Ros Njveni, afi111016m\_44:52]
\z


In contrast, uses of the middle form \textit{njimba}- 'deceive, be wrong, do/behave improperly' imply the absence of control (i.e., intentionality) in the act of deception (or the improper behavior). In \REF{ex:brooks:29}, Emma informs Dorothy that she found the strainer she had at first forgot she had brought over for them to cook with:

\ea\label{ex:brooks:29}
\glll ku mamɨgɨ avkɨ, mɨkani mɨkani, ku njimba.\\
 (ku mamɨgɨ avkɨ mɨkani mɨkani)  ku nji-mb-a\\
 I brought it down here it is here it is \textsc{1sg.nom}  \textsc{mid}-deceive-\textsc{r}\\
 \glt `I brought it down, here it is here it is, I was wrong.'
 [Emma Airɨmarɨ, 051116m, 22:44]
\z


Chini thus makes use of the middle to make important semantic distinctions, for instance willful versus accidental behavior.

 Unaccusative middles generally preclude autopathic or mutual readings (unlike reflexive and reciprocal middles \sectref{sec:brooks:4.2.1.1} and autopathic causal middles \sectref{sec:brooks:4.2.1.3}). For example, when the sediment base of the riverbed surfaces on a canoe journey, no use of the middle form \textit{njiyu-} ‘surface’ can be conceived of in terms of the sediment resurfacing or lifting itself. It is always the external agent of the receded water level that is to blame.\footnote{Just like other verbs, middles can be polysemous. The unmarked transitive \textit{yu-} ‘pick, lift up’ is not polysemous. Its middle form is: \textit{njiyu}- ‘resurface (the riverbed); jump up, onto’.} However, for a few verbs there are occasional exceptions where an autopathic \REF{ex:brooks:30} or mutual \REF{ex:brooks:31} interpretation is possible. These arise when external control is obliquely present in the context of the utterance:

\ea\label{ex:brooks:30}
\glll anɨ  njichi.\\
anɨ  nji-ch-i\\
\textsc{3sg}  \textsc{mid-}exist-\textsc{irr}\\
\glt `(He’s sleeping,) leave him be (“let him exist unto himself”).'
[2018 fieldnotes, elicited example]
\z


\ea\label{ex:brooks:31}
\glll mɨnjagwuwa.\\
 mɨ=nji-agwu-ga\\
\textsc{dist=mid-}put/pile.inside-\textsc{r.pl}\\
\glt `They (the dried tobacco leaves) are overly piled up (i.e., on each other).'
[Dorothy Paul, afi151116m\_35:54]
\z


\subsubsubsection{{Mitigated} {control} {and} {partially} {autopathic} {middles}}\label{sec:brooks:4.2.1.3}

This middle subtype refers to verbal meanings where the control of the agent is mitigated by some external force or is somehow otherwise ambiguous. For these situations, the question of the main participant’s control over the activity may be less straightforward than clear presence \sectref{sec:brooks:4.2.1.1} or absence \sectref{sec:brooks:4.2.1.2} As I discuss below in \sectref{sec:brooks:4.2.1.3.2}, there is a tendency for partially autopathic readings, though this is not always the case. The verbs in \tabref{tab:brooks:5} give an initial impression.

\begin{table}
\caption{\label{tab:brooks:5}Middles involving mitigated or ambiguous control}
\footnote{A number of middle verbs of motion and of bodily function listed in \tabref{tab:brooks:5} may first appear to represent instances of lexical idiosyncrasy, something understood to be characteristic of middles \citep{Kemmer1994}. Part of my argument in this section, however, is that the marking of some verbs as middles may not be idiosyncratic as it seems, but is instead due to semantic properties like mitigated control.}
\fittable{
\small
\begin{tabularx}{\textwidth}{Xp{5cm}X}
\lsptoprule
 {Unmarked} {counterparts*} &
  {Middle-derived forms (\textit{nji}-)} &
  {{Situation type}}\\
\midrule
{\textit{ambɨñ}- 'laugh at (someone) (i.e. with amusement)'} & \textit{njambɨñ}- 'laugh' & {{Externally-oriented bodily function or emotion}}\\
{{\textit{pu}- 'be upset} (at someone, about something)'} & \textit{njipu}- 'get (oneself) riled up (i.e. over something), thrash about in anger' & \\
%\hhline%%replace by cmidrule{--~}
{(no known counterparts)} & \textit{njumia}- ‘vomit’ & \\
%\hhline%%replace by cmidrule{--~}
& \textit{njimim}- ‘urinate, shoot projectile poison (frogs)’ & \\
%\hhline%%replace by cmidrule{~-~}
& \textit{njavi}- ‘defecate’ & \\
%\hhline%%replace by cmidrule{~-~}
& \textit{njimbovɨ}- ‘burp’ & \\
%\hhline%%replace by cmidrule{~--}
& \textit{njagi}- 'paddle (a canoe)' & {{ Action or state leading to further action or state}}\\
%\hhline%%replace by cmidrule{~--}
& \textit{njigwunɨŋi}- ‘dance about (with each other) (\textsc{tloc})’ & \\
%\hhline%%replace by cmidrule{~-~}
& \textit{njari}- 'be off, get up to leave' & \\
%\hhline%%replace by cmidrule{~-~}
& \textit{njinku}- 'do repetitive back-and-forth or up-and-down motion (e.g. swing, see-saw, do pull-ups)' & \\
%\hhline%%replace by cmidrule{~-~}
& \textit{njirɨv}- 'jump down, off' & \\
%\hhline%%replace by cmidrule{~-~}
& \textit{njaŋgu-} `(cause, allow oneself to) waste time, dilly-dally' & \\
%\hhline%%replace by cmidrule{--~}
{\textit{yu}- ‘pick/lift up’} & \textit{njiyu}- ‘jump up, onto’ & \\
% \hhline%%replace by cmidrule{--~}
{\textit{aŋvu}- 'reduce (something)'} & \textit{njaŋ(v)u}- 'bend (oneself) down' & \\
%\hhline%%replace by cmidrule{--~}
{\textit{ñi}- 'get, retrieve (someone or something)'} & \textit{njiñi}- 'for something to make contact with itself via movement, esp. back-and-forth' & \\
%\hhline%%replace by cmidrule{--~}
{\textit{yim-} `chew betel nut (the action of chewing it)'} & \textit{njiyim-} `chew betelnut (and experience its narcotic effect)' & \\
%\hhline%%replace by cmidrule{--~}
\lspbottomrule
\end{tabularx}
}
\end{table}

\subsubsubsubsection{{Mitigated} {control}}\label{sec:brooks:4.2.1.3.1}

Mitigated control over the action is especially true of activities where the participant exerts agentivity as an initiator of the action, but then loses control in some way to become affected by the outcome. Chewing betel nut includes not only the agentive process of combining the ingredients and physically chewing them, but also a chemical reaction resulting in a slightly narcotic effect and heightened sociability. So, the participant is construable as a patient in the chemical and social process, and this is reflected in the grammar of the Chini middle. The transitive verb \textit{yim}- ‘chew betel nut’ and its corresponding middle form \textit{njiyim}- ‘chew betel nut’ subtly distinguish the two possibilities for this event in terms of control. To indicate only the action of the chewing without reference to the chemical or social effect, the transitive form is used:

\ea\label{ex:brooks:32}
\glll nu  miagɨ   yiminɨkaya\\
 nu  mia-gɨ   yim-i-n-ɨ=ka=ya\\
2\textsc{sg}  betel.nut-\textsc{npl}  chew.betel.nut-\textsc{irr-nmlz-npl=prox.def=top}\\
 \glt `You being in the midst of chewing betel nut like that,'

 \glll miagɨ kɨyi iŋkɨri ŋɨnɨŋɨ magwupmɨtɨ makamɨndɨ.\\
 \\
 \\
\glt `I wish you wouldn’t keep filling the inside of your (mouth) with betel nut pulp and talking.' [Emma Airɨmarɨ, afi260814m\_2:48]
\z
% \todo{incomlete example}


In contexts like \REF{ex:brooks:32}, the complete control of the agent over the act of chewing the betel nut (versus spitting it out) is subtly expressed by the transitive form. When the middle form is used, it is instead the semantic patienthood of the main participant that becomes subtly present. One day, a couple of people saw I was chewing betel nut from across the way. In their question as they smiled and shouted over to me, they used the middle form \textit{njiyim}-, thereby referring to the full process of chewing betel nut including its positive psychosocial effects:

\ea\label{ex:brooks:33}
\glll nu  njiyimkɨyi?!\\
 nu nji-yim-kɨ=y-i\\
 \textsc{2sg}  \textsc{mid}-chew.betel.nut-\textsc{r=z-q.irr}\\
 \glt `Are you chewing betel nut (i.e., and feeling pleasant/chatty)?!' [2016 Fieldnotes]
 \z

 Differing degrees of control might help explain some cross-linguistic differences in terms of which situation types get marked as middles. Kemmer (1993, 1994) describes the cross-linguistic tendency for middles to be used in situations of translational and non-translational motion, including posture. But in Chini, only those motions and postures where the control of the main participant is mitigated count as middles. Going/coming (\textit{aŋɨ}-), heading upriver (\textit{agɨ}-), downriver (\textit{ri}-), sitting down (\textit{pɨ}-) and many others typically involve an action over which the main participant has full control, and where the main participant is not necessarily drawn into further subsequent activity. In contrast, bending down (\textit{njaŋ(v)u}-) requires that one eventually bend back up; jumping up (\textit{njiyu}-) or down (\textit{njirɨv}-) leads to some further trajectory, as does getting up to leave (\textit{njari}-) (which leads, inevitably, to that person leaving). For some verbs, especially of motion and posture, the participant's control may be seen as only minimally compromised (e.g. swinging or paddling). For others, it may be more strongly compromised. Bodily functions arguably fall into this category. Only those bodily functions where some degree of control is (at least initially!) exerted (see \tabref{tab:brooks:5}) occur as middles. Bodily functions seen as involving no exertion of control occur as unmarked intransitives (\textit{ayi}- ‘sneeze’ and \textit{chã}- ‘cough’).

\subsubsubsubsection{{Partially} {autopathic} {readings}}\label{sec:brooks:4.2.1.3.2}

Here I discuss how motions, postures, bodily functions and other situation types involving a loss of control are readily interpretable in terms of partial autopathy.

 Lexical semantics can prove quite important to understanding why certain verbal events expressed by middles have autopathic readings. For middles of motion and posture, the potential for autopathic readings could be related to resultative semantics \citep{Nedjalkov1988}. Where resultatives express a “state produced by the corresponding action” \citep[498]{Kozinskij1988}, middles like \textit{njinku}- ‘swing back and forth’, \textit{njiyim-} ‘chew betel nut’, and others, express a secondary action or change of state produced by the initial action of the verb. So, one’s choice to participate in an event leading to a loss of control allows for a reading of partial (or mitigated) autopathy. This principle is also evident in the semantic differences between some middle forms with their transitive counterparts (e.g., \textit{yu-} ‘pick, lift up’ versus \textit{njiyu-} ‘jump up, onto’ in the sense of ‘pick, lift oneself up, onto’ and \textit{aŋvu}- ‘reduce (something)’ versus \textit{njaŋ(v)u}- ‘bend down, over’ in the sense of ‘reduce oneself’).

 For some middles, however, the felicitousness of an autopathic reading may be more questionable as a matter of context or even individual interpretation. Consider the (deponent) middle verb form \textit{njagi}- ‘paddle (i.e., oneself, each other along)’. Participation involves dipping and pushing the oar, at which point the resulting force of the push propels the canoe and its occupant(s) across the water. Another example is \textit{njambɨñ}- ‘laugh’. It derives from its transitive counterpart \textit{ambɨñ}- ‘laugh at (someone)’. On the one hand, laughter can involve a loss of control. Yet one can spur oneself and (especially) others to laughter, leading to the possibility for autopathic or mutual readings for the middle form ('make oneself/each other laugh'). As in other languages, the control of main participants in emotional-psychological states and also excretive bodily functions can be seen as ambiguous, though context often resolves any apparent ambiguity in the lexical semantics.

 I have described how the Chini middle functions to express the main participant in a clause as a semantic patient. Along the way, I have argued that the intertwining of autopathic (and mutual) meaning arises as a secondary semantic effect. The more control the main participant is understood to exert, the more felicitous the autopathic reading is likely to be. The link is not grammatically rigid, but rather depends on the interplay between lexical semantics, context, and interpretation. While the division of three subtypes I have proposed here is in one sense a mere artefact of my description, it arguably reflects differences in control across middle situation types.

\subsubsection{{Extensions} {of} {middle} {marking}}\label{sec:brooks:4.2.2}

In a few constructions, the middle marker attaches not as a verbal prefix but as a proclitic to the verb complex. In that capacity it functions as a reflexivizer or reciprocal marker. While I have argued that the middle marker is not in fact a reflexivizer but that autopathic and mutual interpretations of middles arise as a secondary feature of the main participant’s limited control over events, in these constructions, the autopathic and/or mutual meaning appears to be what motivates the presence of the middle marker.

In \sectref{sec:brooks:4.2.1} I mentioned bodily functions as a common middle situation type in Chini and alluded to the related squeamish theoretical question of how construable those events are in terms of autopathy and control. In contexts where one participant is negatively affected by the bodily functions of another, the entirety of the action is not construable as autopathic (even if the bodily function itself is):

\ea\label{ex:brooks:34}
\glll mɨnɨmhinjavia.\\
 mɨ=nɨ=mhi=nji-avi-a\\
\textsc{dist=3sg.acc=foc.all=} \textsc{mid}-defecate-\textsc{r}\\
\glt `It (the puppy) pooped on her.' [Elicited example, 2018 Fieldnotes]
\z

Bodily functions become undeniably autopathic in those unfortunate situations when the main participant is both agent and patient. This is expressed in Chini by a construction where the middle marker is introduced by the focused allative. This ‘double middle’ construction is restricted to those pronominal person-number combinations that distinguish a dative case. (1\textsc{sg} and all dual participants require the expected accusative or invariant pronominal forms instead of the middle marker.)

\ea\label{ex:brooks:35}
\glll anɨ  vrɨmɨ   {nji}mhi{nji}mimkɨ.  \textup{Reflexive ‘Double’ Middle}\\
anɨ  vrɨmɨ   \textbf{nji=}mhi=\textbf{nji-}mim-kɨ\\
\textsc{3sg}  mistakenly  \textbf{\textsc{mid}}\textsc{=foc.all=} \textbf{\textsc{mid}}\textsc{-}urinate\textsc{-r}\\
\glt `S/he mistakenly urinated on him/herself.' [Elicited example, 2018 Fieldnotes]
\z

Finally, the middle marker occurs as part of the reciprocal comitative construction \REF{ex:brooks:36} and the reciprocal sociative construction \REF{ex:brooks:37}. (As \citet{ZaliznjakShmelev2007} describe for Latin, the sociative in Chini expresses “participation on equal grounds” (213).)

\ea\label{ex:brooks:36}
\glll aŋgɨ  \textbf{njiŋgɨ}   yu.  \textup{Reciprocal Comitative}\\
aŋgɨ  \textbf{nji=ŋgɨ}  yu\\
\textsc{1du}  \textbf{\textsc{mid=com}}  go.\textsc{irr}  \\
\glt `We two will go with each other.'
\z

\ea\label{ex:brooks:37}
\glll aŋgɨ  \textbf{njavɨgɨ}   yu. \textup{Reciprocal Sociative} \\
aŋgɨ  \textbf{nji=avɨgɨ}   yu\\
\textsc{1du}  \textbf{\textsc{mid}}\textbf{=upper.arm}  go\textsc{.irr}  \\
 \glt `We two will go together (i.e., side by side, in friendship, etc.).'
 \z

\section{{Conclusion}}\label{sec:brooks:5}

In this paper I have described those constructions in Chini where autopathic (and/or mutual) relations between participants figure prominently in linguistic expression. One is the reflexive possessive construction, where the form \textit{ŋɨ}= is based on coreference between the possessor and the topic (whether subject or otherwise) or semantic agent.

The other is the middle construction. Middles can be distiguished in terms of the differing degrees of agency of the main participant, whether agency is more or less present \sectref{sec:brooks:4.2.1.1}), absent \sectref{sec:brooks:4.2.1.2}), or mitigated \sectref{sec:brooks:4.2.1.3}). The Chini middle is not used to indicate autopathic relations between participants per se, but rather indicates the semantic patienthood of the main participant across different types of situations. Autopathic (and mutual) readings are possible to the extent that the main participant exerts full or partial control over the action or as permitted by lexical semantics and/or the context of the utterance. Yet autopathic meaning is deeply bound up with the Chini middle. That this is true is seen in the extensions of middle marking to other constructions, namely the double middle for accidental bodily functions, and the reciprocal comitative and sociative constructions (\sectref{sec:brooks:4.2.2}).

\section*{Abbreviations}
Standard Leipzig glossing abbreviations not listed
\noindent

\begin{tabularx}{.45\textwidth}{lQ}
\textsc{accom} &   Accompaniment (postposition)\\
\textsc{adess} &   Adessive (postposition)\\
\textsc{cnt(.r/irr)} &  Continuity of Information (clause chain linkers, realis vs. irrealis)\\
\textsc{ctrst} &   Contrastive (clause combining construction)\\
\textsc{irr} &   Irrealis mood\\
\textsc{lh} &   Light head (for noun phrases)\\
\textsc{mid} & Middle (‘voice’)\\
\textsc{mod} &   Modal verb base (abstract modal interpretation in \textsc{pre}, \textsc{seq} medial clauses)\\
\textsc{npl} &   Non-plural nominal number\\
\textsc{opt} &   Optative mood\\
\textsc{pc} &   Paucactional verbal number\\
\textsc{pl} &   Plural nominal number or pluractional verbal number\\
\textsc{pre(.r/irr)} &  Presuppositional Information (clause chain linkers, realis vs. irrealis)\\
\end{tabularx}
\begin{tabularx}{.45\textwidth}{lQ}
\textsc{q(.r/irr)} &  Question suffix (realis/content question vs. irrealis/yes-no question)\\
\textsc{r} &   Realis mood\\
\textsc{seq(.r/irr)} &  Temporal succession (clause chain linkers, realis vs. irrealis)\\
\textsc{sim} &   Simulative\\
\textsc{tloc} &   Translocative directionality\\
\textsc{trans} &   Translational directionality\\
\textsc{z} &   Category-conditioned suffix form that marks a wide range of clause types\\
\end{tabularx}

\section*{Acknowledgments}

I am thankful in particular to Bernard Comrie, Marianne Mithun, and Lise Dobrin. Their input has so greatly helped shape my thinking over the past several years and relates to much of what I describe and argue for in this paper. I am so grateful to everyone in Andamang for their hospitality and wisdom, especially to my \textit{apakɨ} Anton Mana who has spent so much time teaching me and working with me.


\sloppy\printbibliography[heading=subbibliography,notkeyword=this]
\end{document}
