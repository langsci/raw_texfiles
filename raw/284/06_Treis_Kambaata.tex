\documentclass[output=paper]{langscibook}
\title{Reflexive constructions in Kambaata}
\abstract{Kambaata (Cushitic, Ethiopia) has a nominal and a verbal reflexivizer. The nominal reflexivizer \textit{gag-á} `self', a case-inflecting noun of masculine gender, is used to mark coreference between the subject and a direct, indirect or oblique object. Whereas the antecedent of the reflexive noun is most commonly the subject of the same clause, this chapter argues that \textit{gag-á} `self' also qualifies as a long-distance reflexive. As such, it can mark coreference between an NP in an infinite or finite subordinate clause and the subject of the matrix clause. Apart from being used in reflexive constructions, \textit{gag-á} `self' is a self-intensifier. The middle morpheme -\textit{aqq} / -\textit{'} on verbs is multifunctional. Most productively, it expresses autobenefactivity. It can also mark coreference between the subject and the direct object in the same clause clause. However, in typical reflexive situations (e.g. `see oneself'), it is  rarely the only reflexivizer but cooccurs with the reflexive noun \textit{gag-á}.}
\author{Yvonne Treis\affiliation{LLACAN (CNRS, INALCO)}\orcid{}}  
%move the following commands to the "local..." files of the master project when integrating this chapter

\IfFileExists{../localcommands.tex}{
 \addbibresource{localbibliography.bib}
 \usepackage{langsci-optional}
\usepackage{langsci-gb4e}
\usepackage{langsci-lgr}

\usepackage{listings}
\lstset{basicstyle=\ttfamily,tabsize=2,breaklines=true}

%added by author
% \usepackage{tipa}
\usepackage{multirow}
\graphicspath{{figures/}}
\usepackage{langsci-branding}

 
\newcommand{\sent}{\enumsentence}
\newcommand{\sents}{\eenumsentence}
\let\citeasnoun\citet

\renewcommand{\lsCoverTitleFont}[1]{\sffamily\addfontfeatures{Scale=MatchUppercase}\fontsize{44pt}{16mm}\selectfont #1}
   
 %% hyphenation points for line breaks
%% Normally, automatic hyphenation in LaTeX is very good
%% If a word is mis-hyphenated, add it to this file
%%
%% add information to TeX file before \begin{document} with:
%% %% hyphenation points for line breaks
%% Normally, automatic hyphenation in LaTeX is very good
%% If a word is mis-hyphenated, add it to this file
%%
%% add information to TeX file before \begin{document} with:
%% %% hyphenation points for line breaks
%% Normally, automatic hyphenation in LaTeX is very good
%% If a word is mis-hyphenated, add it to this file
%%
%% add information to TeX file before \begin{document} with:
%% \include{localhyphenation}
\hyphenation{
affri-ca-te
affri-ca-tes
an-no-tated
com-ple-ments
com-po-si-tio-na-li-ty
non-com-po-si-tio-na-li-ty
Gon-zá-lez
out-side
Ri-chárd
se-man-tics
STREU-SLE
Tie-de-mann
}
\hyphenation{
affri-ca-te
affri-ca-tes
an-no-tated
com-ple-ments
com-po-si-tio-na-li-ty
non-com-po-si-tio-na-li-ty
Gon-zá-lez
out-side
Ri-chárd
se-man-tics
STREU-SLE
Tie-de-mann
}
\hyphenation{
affri-ca-te
affri-ca-tes
an-no-tated
com-ple-ments
com-po-si-tio-na-li-ty
non-com-po-si-tio-na-li-ty
Gon-zá-lez
out-side
Ri-chárd
se-man-tics
STREU-SLE
Tie-de-mann
} 
 \togglepaper[1]%%chapternumber
}{}


%custom footer for preprints

%%DEAR BOOK EDITORS
%Still to be done:
%%Is it possible to format Table 1 in landscape? If not, it needs to be split up in two parts.
%%Page breaks still cut through examples. I tried the \protectedex-command, but it doesn't work. 
%%The command \REF does not seem to work. I have used (\ref) instead.
%%First line of examples not yet italicized (it seems that the file does not yet contain series information)
%%I quoted §2 of your book below as Haspelmath (this volume), but didn't include it in the bibliography, as I wasn't sure whether you would rather insert a section hyperlink than a reference link
%Other notes:
%%*Ethiopian* authors are quoted by their first name, please don't change this in the bibliography, the way it is now is correct (sorry for this note, but editors tend to mess up my bibliographies ...)
%%Please copy information from localhyphenation.tex into the book



\begin{document}
\maketitle

\section{Introduction}\label{sec:treis:1}
Kambaata is a Highland East Cushitic (HEC) language spoken by more than 600,000 people \citep[74]{CSA2007} in the Kambaata-Xambaaro Zone in the Southern Region of Ethiopia. Immediate neighbors are speakers of other HEC languages (Hadiyya and Alaaba) and Ometo languages of the Omotic family (Wolaitta and Dawro). The most widespread second language of Kambaata speakers is the Ethiopian lingua franca Amharic. The description of reflexive constructions presented here is based on data from diverse sources obtained during field research between 2002 and 2019: a corpus of recorded narratives and conversations, my field notes of elicited sentences and mock-dialogues as well as a corpus of written texts, including locally published collections of oral literature, schoolbooks, a dictionary, religious texts and the translation of the \textit{Little Prince}. Gaps in the data were filled and open questions were discussed in interviews on the phone or through text and voice messages with a native speaker in 2020.

\subsection{Typological profile}\label{sec:treis:1.1}

The constituent order of Kambaata is consistently head-final, hence all modifiers precede the noun in the NP, and all dependent clauses precede independent main clauses. The last constituent in a sentence is usually a fully finite main verb or a copula. Kambaata is agglutinating-fusional and, except for one partial pre-reduplication process,\footnote{See (\ref{ex:treis:27}) for an example of a pre-reduplicated noun.} strictly suffixing. Inflectional morphology is realized by segmental suffixes together with stress suprafixes. The following open word classes are defined on morphosyntactic grounds: nouns, adjectives, verbs, ideophones and interjections. 

Kambaata has a nominative-accusative case-marking system. The nominative is the subject case, see ‘girls’ in (\ref{ex:treis:1}). The accusative marks direct objects – see ‘good place’ in (\ref{ex:treis:1}) – and certain adverbial constituents, it also serves as the citation form of nouns and adjectives. Nouns distinguish nine case forms: nominative, accusative, genitive, dative, ablative, instrumental, locative, oblique and predicative. Nouns are marked for gender (masculine vs. feminine), the assignment of grammatical gender is mostly arbitrary. Attributive adjectives, such as ‘good’ in (\ref{ex:treis:1}), agree with their head noun in case and gender.\footnote{Transcriptions in this chapter use the official Kambaata orthography, which is based on the Roman script (\citealt[73-80]{Treis2008}; \citealt{Alemu2016}). One important adaptation is here made to the official orthography: phonemic stress is marked by an acute accent on the vowel. The following graphemes are not in accordance with IPA conventions: <ph> /p’/, <x> /t’/, <q> /k’/, <j> /dʒ/, <c> /tʃ’/, <ch> /tʃ/, <sh> /ʃ/, <’l> /l’/, <’r> /r’/, <y> /j/ and <’> /Ɂ/. Geminate consonants and long vowels are marked by doubling, e.g. <shsh> /ʃː/ and <ee> /eː/. In clusters of a glottal stop and a sonorant, the sonorant is, by convention, written double, e.g. <’nn> for /Ɂn/ and <’rr> /Ɂr/. Nasalization is marked by a macron, e.g. <ā> /ã/.}

\ea\label{ex:treis:1} 
\gll (...) masal-aakk-atí-i danaam-íta ma’nn-íta aat-táa-s\\
     {} girl-\textsc{plv2-f.nom-add} good-\textsc{f.acc} place-\textsc{f.acc} give-3\textsc{f.ipfv-3m.obj}\\
\glt (Speaking about the winner of a wrestling competition) ‘(…) and also the girls honor him (lit. give him a good place).’ [Conversation about circumcision traditions, EK2016-02-23\_001]
\z

Fully finite main verbs are distinguished from several types of dependent verbs, which are reduced in finiteness, i.e. relative verbs, converbs, purposive verbs and (infinite) verbal nouns. Verbs inflect for aspect, mood, polarity and syntactic dependence. All verb forms apart from verbal nouns obligatorily index their subject; see the portmanteau morpheme \textit{-táa} in (\ref{ex:treis:1}), which encodes imperfective aspect and indexes a third person feminine subject. Object suffixes on verbs, such as the third person masculine object suffix \textit{-s} in (\ref{ex:treis:1}) and the first person singular object suffix \nobreakdash-\textit{’e} in (\ref{ex:treis:2}), are pronominal in nature and usually substitute for object nominals. A finite verb form alone can constitute a complete utterance (\ref{ex:treis:2}).

\ea\label{ex:treis:2} 
\gll qéel-teente-’e\\
     defeat-2\textsc{sg.prf-1sg.obj}\\
\glt (Complete turn in a dialogue:) ‘You have defeated me.’ [Narrative, EK2016-02-12\_003]
\z

\subsection{A preview of reflexive constructions}\label{sec:treis:1.2}

Kambaata uses a reflexive noun \textit{gag-á} ‘self’ plus a possessive suffix (\ref{ex:treis:3}) or a reflexive voice marker \textit{-aqq~/ -’}, labelled “middle (derivation)” (\textsc{mid}) (\ref{ex:treis:4}), to express coreference between the subject and an accusative object.

\ea\label{ex:treis:3} 
\gll \textbf{Gag-á-s} ba’-íshsh-o\\
     self-\textsc{m.acc-3m.poss} be.destroyed-\textsc{caus1-3m.pfv}\\
\glt (Speaking about the actual cause of someone’s death whom the addressee thought to have died from an illness) ‘He killed himself (lit. his self).’ [Elicited, DW2020-01-24]\z

\ea\label{ex:treis:4} 
\gll Kabár gagmooxx-íin \textbf{xuud-aqq-aammí}=da áda móok-i-’i sabáb-b darsh-ítee’u\\
     today.\textsc{m.obl} mirror-\textsc{m.icp} see-\textsc{mid-1sg.ipfv.rel=cond} then cheek-\textsc{f.nom-1sg.poss} ensue-\textsc{3f.pfv.cv} become.swollen-\textsc{3f.prf}\\
\glt (Speaking about the consequences of a brawl) ‘Then when I saw myself in the mirror today, my cheek was badly swollen.’ [Elicited, DW2020-01-24]\z

In (\ref{ex:treis:5}), both reflexivizers cooccur in the same clause. The verb \textit{saaxx}- ‘praise oneself’ is the middle derivation of \textit{saad}- ‘praise (someone)’.

\ea\label{ex:treis:5} 
\gll Isú mánn-u galaxx-u’nnáachchi-s birs-í-n-in ís \textbf{gag-á-s} \textbf{saaxx-án} biir-óochch biir-úta zahh-áyyoo’u\\
     \textsc{3m.acc} people-\textsc{m.nom} thank-\textsc{3m.neg4-3m.obj} do.before-\textsc{3m.pfv.cv-emp-emp} \textsc{3m.nom} self-\textsc{m.acc-3m.poss} praise.\textsc{mid-3m.ipfv.cv} office-\textsc{f.abl} office-\textsc{f.acc} walk-\textsc{3m.prog}\\
\glt ‘Before people (could) thank him (for the job), he walked from office to office praising himself.’ [Elicited, DW2020-01-24]\z

In the following sections, I will first introduce the personal pronoun system of Kambaata (\sectref{sec:treis:2}) and then discuss the form and functions of the noun \textit{gag-á} ‘self’ (\sectref{sec:treis:3}). Apart from being used as a reflexivizer in various syntactic functions (except in the subject function), it is used as a self-intensifier. In \sectref{sec:treis:4}, I present the multifunctional middle derivation, whose most productive function is to signal coreference between the subject and a beneficiary (a dative adjunct). It also marks coreference between the subject and a direct (accusative) object, but here it usually cooccurs with the reflexive noun. Thirdly, the middle derivation has an intersubjective use and expresses the emotional involvement of the subject in a state-of-affairs. Together with the passive morpheme, the middle derivation marks reciprocity. In the conclusion (\sectref{sec:treis:5}), I lay out the contexts in which the reflexive noun is preferred over the middle morpheme and when double expression is preferred over the use of only one reflexive marker.

\section{Personal pronouns}\label{sec:treis:2}

Kambaata has free (\sectref{sec:treis:2.1}) and bound personal pronouns (\sectref{sec:treis:2.2}), but no pronoun-like reflexive nominals (i.e. pronominoids). Personal pronouns are used to refer to humans, less often to other animates, and usually not to inanimate referents like things or events, for which demonstratives are preferred.

\subsection{Free personal pronouns}\label{sec:treis:2.1}

Free personal pronouns (\tabref{tab:treis:1}) distinguish person, number, gender (in the third person), honorificity (in the second and third person) and case. The case paradigm of personal pronouns is partly suppletive; see, for instance, the different stems that are used for the nominative and accusative forms of \textsc{1sg,} \textsc{2sg}, \textsc{2hon,} \textsc{1pl} and \textsc{2pl}. In principle, personal pronouns distinguish as many case forms as nouns. However, there is systematic syncretism of the instrumental-comitative-perlative \textsc{(icp)} and locative \textsc{(loc)} forms for all persons except \textsc{3m}. Furthermore, the oblique and the predicative case forms are only minimally distinct in the first person plural. The singular predicative forms combine with the copula (\textsc{cop3}) \nobreakdash-V\textit{t}. In the plural, the copula (\textsc{f.cop2)} ´-\textit{taa} is required (see \citealt[397-426]{Treis2008} for information on the distribution of Kambaata copulas).

\begin{sidewaystable}
%remove this command and format table in landscape
\caption{Free personal pronouns}
\label{tab:treis:1}
\begin{tabularx}{\textwidth}{lXp{1.7cm}Xp{2cm}p{2cm}XXXX} 
\lsptoprule
& {\textsc{nom}} & {\textsc{acc}} & {\textsc{gen}} & {\textsc{dat}} & {\textsc{abl}} & {\textsc{icp}} & {\textsc{loc}} & {\textsc{obl}} & {\textsc{pred}}\\
\midrule
\textsc{1sg} & {án} & {ées} & {íi} & {esáa(ha)} & {esáachch} & {esáan} & {esáan} & {áne} & {áne}\\
\textsc{2sg} & {át} & {kées} & {kíi} & {kesáa(ha)} & {kesáachch} & {kesáan} & {kesáan} & {áte} & {áte}\\
\textsc{2hon} & {á’nnu} & {ki’nnéta} & {ki’nné} & {ki’nnée(ha)} & {ki’nnéechch} & {ki’nnéen} & {ki’nnéen} & {á’nno} & {á’nno}\\
\textsc{3m} & {ís} & {isú} & {isí} & {isíi(ha)} & {isíichch} & {isíin} & {isóon} & {íso} & {íso}\\
\textsc{3f} & {íse} & {iséta} & {isé} & {isée(ha)} & {iséechch} & {iséen} & {iséen} & {íse} & {íse}\\
\textsc{3hon} & {íssa} & {issáta} & {issá} & {issáa(ha)} & {issáachch} & {issáan} & {issáan} & {íssa} & {íssa}\\
\textsc{1pl} & {na’óot} & {nées} & {níi} & {nesáa(ha)} & {nesáachch} & {nesáan} & {nesáan} & {na’ó} & {na’óo}\\
\textsc{2pl} & {a’nno’óot} & {ki’nne’éeta} & {ki’nne’ée} & {ki’nne’ée(ha)} & {ki’nne’éechch} & {ki’nne’éen} & {ki’nne’éen} & {a’nno’óo} & {a’nno’óo}\\
\textsc{3pl} & {isso’óot} & {isso’óota} & {isso’óo} & {isso’óo(ha)} & {isso’óochch} & {isso’óon} & {isso’óon} & {isso’óo} & {isso’óo}\\
\lspbottomrule
\end{tabularx}
\end{sidewaystable}

\subsection{Bound personal pronouns}\label{sec:treis:2.2}

Bound object pronouns on verbs and bound possessive pronouns on nouns and adjectives (\tabref{tab:treis:2}) are only minimally distinct: for \textsc{1sg} possessors and \textsc{2sg} objects, speakers can choose between two freely distributed allomorphs, whereas only one of the allomorphs is admitted for the respective \textsc{1sg} object and the \textsc{2sg} possessor form. A comparison with free pronouns (\sectref{sec:treis:2.1}) shows that bound pronouns neutralize the distinction between honorific and plural referents in the second and third person. The right column of \tabref{tab:treis:2} exemplifies the use of possessive suffixes on the accusative form of the reflexive noun \textit{gag-á} ‘self’.

\begin{table}
\caption{Bound personal pronouns and the reflexive noun}
\label{tab:treis:2}
\begin{tabularx}{0.9\textwidth}{lp{4.5cm}p{4.5cm}} 
\lsptoprule 
& {Pronominal Suffixes} & {Reflexive noun (\textsc{acc}) with possessive suffix}\\
\midrule 
\textsc{1sg.obj} & {-’e} & \\
\textsc{1sg.poss}& {-’e {\textasciitilde} -’} & {gag-á-’e} {{\textasciitilde} gag-á-’}\\
\textsc{2sg.obj} & {-(k)ke {\textasciitilde} -he} & \\
\textsc{2sg.poss}& {-(k)k} & {gag-á-kk}\\
\textsc{3m} & {-s} & {gag-á-s}\\
\textsc{3f} & {-se} & {gag-á-se}\\
\textsc{1pl} & {-(n)ne} & {gag-á-nne}\\
\textsc{2pl~(=~2hon)} & {-(k)ki’nne {\textasciitilde} -’nne} & {gag-á-kki’nne} {{\textasciitilde} gag-á-’nne}\\
\textsc{3pl~(=~3hon)} & {-(s)sa} & {gag-á-ssa}\\
\lspbottomrule
\end{tabularx}
\end{table}

Possessive pronouns never cooccur with full nominal possessors. Object pronouns typically substitute for direct or indirect object nominals; recall (\ref{ex:treis:1}). However, in case of high referential prominence, an object can be doubly expressed by a full object nominal – a noun or pronoun phrase – and a bound object pronoun on the verb, as seen in (\ref{ex:treis:6}) and later in (\ref{ex:treis:14}).

\ea\label{ex:treis:6} 
\gll Harr-ée buud-á \textbf{kesáa} m-á buchch-íichch eeb-ó<\textbf{kke}>ta-at?\\
     donkeys-\textsc{f.gen} horn-\textsc{f.dat} \textsc{2sg.dat} what-\textsc{m.acc} soil-\textsc{m.abl} bring-\textsc{1sg.purp.ss<2sg.obj>-cop3}\\
\glt ‘From where on earth am I going to bring you a donkey horn?’ [Narrative, EK2016-02-12\_003]\z

\section{Reflexive noun}\label{sec:treis:3}
\subsection{Form and meaning}\label{sec:treis:3.1}

Kambaata uses the reflexive noun \textit{gag-á} ‘self’, usually combined with a possessive suffix (\tabref{tab:treis:2}),\footnote{There are two instances in the Gospel of John in which the possessor of \textit{gag-á} ‘self’ is expressed by a free genitive pronoun, e.g. \textit{íi} (\textsc{1sg.gen)} \textit{gag-íi} (self-\textsc{m.dat)} ‘for myself’. For the use of unmodified reflexive nouns see \sectref{sec:treis:3.2.5}.} to express coreference between the subject and another participant in the clause. \textit{Gag-á} ‘self’ is clearly noun-like in nature. It inflects for case (\tabref{tab:treis:3})\footnote{In \tabref{tab:treis:3}, the notation \textit{-i\_´} of the genitive morpheme indicates that the case is realized by a segment \textit{-i} and a stress suprafix on the rightmost syllable of the word.} like any regular common noun of the masculine declension \textsc{m1} \citep[103]{Treis2008}. In the text of this chapter, the reflexive noun is always cited in its accusative form \textit{gag-á}.

\begin{table}
\caption{The case paradigm of \textit{gag-á} ‘self’}
\label{tab:treis:3}
\begin{tabular}{llll}
\lsptoprule
\textsc{acc} & \textit{gag-á} & \textsc{abl} & \textit{gag-íichch}\\
\textsc{nom} & \textit{gág-u} & \textsc{icp} & \textit{gag-íin}\\
\textsc{gen} & \textit{gag-i\_´} & \textsc{loc} & \textit{gag-áan}\\
\textsc{dat} & \textit{gag-íi(ha)} & \textsc{obl=pred} & \textit{gág-a}\\
\lspbottomrule
\end{tabular}
\end{table}

\textit{Gag-á} ‘self’ is a transnumeral noun and thus allows for singular and plural reference. It is not attested with plurative (\textsc{plv)} marking, but a singulative (\textsc{sgv)} example is presented in (\ref{ex:treis:13}). The reflexive noun is marked for distributivity through partial pre-reduplication (‘each … oneself’), as seen in (\ref{ex:treis:27}). Other morphemes that can attach to the reflexive noun are the emphasis marker \textit{-n} (\ref{ex:treis:13}), the additive marker -V (\ref{ex:treis:20}), and – when ‘self’ is the head or modifier of the non-verbal predicate (\ref{ex:treis:32}) – the copula. The stem of the reflexive noun can be the input of the status noun derivation with \textit{-oom-áta} \citep[171]{Treis2008}: \textit{gag-oom-áta} ‘identity (lit. selfhood, selfness)’ \citep[349]{Alemu2016}, as shown in (\ref{ex:treis:7}).\footnote{All examples taken from publications in the Kambaata language are stress-marked, segmented, glossed and translated to English by the present author.}

\ea\label{ex:treis:7} 
\gll \textbf{Gag-oom-á-nne} caakk-is-soonti-nné=g-a<n>ka bír-i-kk caakk-ítu\\
     self-\textsc{stat-f.acc-1pl.poss} become.light-\textsc{caus1-2sg.pfv-1pl.obj.rel=sim-m.acc<emp>} future-\textsc{f.nom-2sg.poss} become.light-3\textsc{f.bdv}\\
\glt ‘As you brought our identity to light, may your future be bright!’ \citep[4]{AdaneNodate}\z

The noun \textit{gag-á} ‘self’ can be used metaphorically to express a ‘person like oneself’, or a ‘close relative’, as illustrated in (\ref{ex:treis:8}).

\ea\label{ex:treis:8} Proverb \\
\gll \textbf{Gág-u} buud-á woqqarr-ó=da allagg-íchch-u ill-íta qas-áno\\
     self-\textsc{m.nom} horn-\textsc{m.acc} strike-\textsc{3m.pfv.rel=cond} strangers-\textsc{sgv-m.nom} eye-\textsc{f.acc} poke-\textsc{3m.ipfv}\\
\glt ‘If a next of kin (lit. a self) strikes the horn (of your bull), a stranger (can) poke (you in your) eye.’ \citep[52]{AlamuAlamayo2017}
\z

While ‘head’ is the most common source for reflexive nominals in the languages of the world \citep{Schladt1999} – see also the reflexivizer \textit{ras} ‘head’ in Amharic \citep[57-58]{Leslau1995}, the primary contact language of Kambaata, and the reflexivizer \textit{umo} ‘head’ in the closely related HEC language Sidaama \citep[184-187]{Kawachi2007} –, there is no indication that Kambaata \textit{gag-á} goes back to a noun ‘head’. A reflexive noun cognate to that of Kambaata is used in the HEC languages Alaaba, K’abeena and Hadiyya (\citealt[188-199]{Schneider-Blum2007}; \citealt[257-259]{Crass2005}; \citealt[90-91]{Tadesse2015}).

\subsection{Reflexive constructions}\label{sec:treis:3.2}
\subsubsection{Autopathic domain}\label{sec:treis:3.2.1}

Coreference between the subject and its direct object in a monotransitive clause is expressed by an accusative-marked reflexive noun. The possessive suffixes on \textit{gag-á} ‘self’ are coreferential with the subject of the clause, e.g. \textsc{3m} in (\ref{ex:treis:3}), \textsc{1sg} in (\ref{ex:treis:9}) and (\ref{ex:treis:13}), \textsc{2pl} in (\ref{ex:treis:10}) and \textsc{3pl} in (\ref{ex:treis:11}). The examples (\ref{ex:treis:9})-(\ref{ex:treis:11}) illustrate that the subject is not necessarily expressed by an independent nominative NP, it suffices to have it indexed on the verb. As the seven subject indexes and the seven possessive suffixes are not fully congruent, a mismatch is observed in (\ref{ex:treis:11}). The ordered persons are indexed as \textsc{3f} (= \textsc{3pl)} on the verb \textit{torr-} ‘throw’ but as \textsc{3pl} (= \textsc{3hon)} on ‘self’.\footnote{Free personal pronouns distinguish nine forms (\tabref{tab:treis:1}), possessive/object pronouns (\tabref{tab:treis:2}) and subject indexes only seven. In the possessive/object paradigm, we see the following syncretism: \textsc{1sg,} \textsc{2sg,} \textsc{3m}, \textbf{\textsc{3f}}, \textsc{1pl}, \textsc{2pl} (= \textsc{2hon)}, \textbf{\textsc{3pl (= 3hon)}}. Another type of syncretism is found in the subject index paradigm: \textsc{1sg}, \textsc{2sg}, \textsc{3m}, \textbf{\textsc{3f (= 3pl)}}, \textbf{\textsc{3hon}}, \textsc{1pl}, \textsc{2pl (= 2hon)}.}

\ea\label{ex:treis:9} 
\gll (…) \textbf{gag-á-’} isso’oo-sí qax-á<n>ka ass-í {kot-íshsh=ké’ (…)}\\
     {} self-\textsc{m.acc-1sg.poss} \textsc{3pl.gen-def} level-\textsc{m.acc<emp>} do-\textsc{1sg.pfv.cv} become.small-\textsc{caus1.1sg.pfv.cv=seq}\\
\glt ‘(…) I lower myself to their level (…).’ \citep[11]{Saint-Exupéry2018}\z

\ea\label{ex:treis:10} 
\gll \textbf{Gag-á-’nne} xa’mm-iyyé: \'{ā}ā-ndo \'{ā}ā-bay?\\
     self-\textsc{m.acc-2pl.poss} ask-\textsc{2pl.imp} yes-\textsc{q} yes-\textsc{neg1.q}\\
\glt ‘Ask yourselves: Yes or no (lit. not yes)?’ \citep[93]{Saint-Exupéry2018}\z

\ea\label{ex:treis:11} 
\gll Át gashsh-itaantí mann-á \textbf{gag-á-ssa} már-t baar-í aazéen torr-ítunta {azzaz-zoonti-ssá=da (…)}\\
     \textsc{2sg.nom} pass.the.night.\textsc{caus1-2sg.ipfv.rel} people-\textsc{m.acc} self-\textsc{m.acc-3pl.poss} go-\textsc{3f.pfv.cv} sea-\textsc{m.gen} inside-\textsc{m.loc} throw-\textsc{3f.purp.ds} order-\textsc{2sg.pfv-3pl.obj.rel=cond}\\
\glt ‘If you ordered the people you govern (lit. make pass the night) to go and throw themselves into the sea (…)’ \citep[38]{Saint-Exupéry2018}\z

In (\ref{ex:treis:12}), the subject that serves as the antecedent of the reflexive noun is expressed by a possessive pronoun (\textit{-ssa} \textsc{3pl.obj}) on the infinite verbal noun.\footnote{Unlike other verb forms, verbal nouns cannot index their subjects. The subject is either expressed by a nominative NP, a genitive NP or a possessive pronoun.}

\ea\label{ex:treis:12} 
\gll (…) \textbf{gag-á-ssa-n} íkko ées haww-íichch fa’-is-ú-ssa dag-áam-ba’a\\
     {} self-\textsc{m.acc-3pl.poss-emp} or \textsc{1sg.acc} trouble-\textsc{m.abl} be.saved-\textsc{caus1-m.acc-3pl.poss} know-\textsc{1sg.ipfv-neg1}\\
\glt ‘(…) I don’t know whether they can save themselves and me (lit. I don’t know their saving themselves or me) from trouble.’ \citep[3.118]{KEB1989}\z

The transnumeral reflexive noun can indicate coreference with singular and plural subjects, see\textit{gag-á} in (\ref{ex:treis:9}) and (\ref{ex:treis:10})-(\ref{ex:treis:12}), respectively. However, we still find a small number of overtly singulative-marked forms in the corpus (\ref{ex:treis:13}). The pragmatic reason for this marking is still unknown.\footnote{Note that one of the functions of singulative marking on transnumeral nouns is to express affection \citep[118f]{Treis2014}.} In contrast, overt plurative marking (hypothetically *\textit{gag-g-áta} \textsc{plv1} or *\textit{gag-aakk-áta} \textsc{plv2} ‘selves’) is unattested and was rejected by the native speaker I consulted. 

\ea\label{ex:treis:13}  
\gll \textbf{Gag-ichch-ú-’e-n} ikk-oommí=da esáa woyy-áno-’e\\
     self-\textsc{sgv-f.acc-1sg.poss-emp} become-\textsc{1sg.pfv.rel=cond} \textsc{1sg.dat} become.better-\textsc{3m.ipfv-1sg.obj}\\
\glt (Protagonist of a story who has adopted body parts of other animals:) ‘It would be better if I became myself (again).’ [Narrative, TD2016-02-11\_001]\z

A non-reflexive free accusative pronoun or a non-reflexive object suffix on the verb is necessarily interpreted as being referentially disjoint with the subject. See, for instance, the clause marked in bold in (\ref{ex:treis:14}): the free accusative pronoun \textit{isú} ‘him’ and the object suffix \nobreakdash-\textit{sí} (here infixed into the purposive verb) are always interpreted as being referentially disjoint from the subject of ‘help’ (reflected in the subject index \textsc{3m}). The same is true of the object suffix \textit{-s} on the main verb ‘ask’; neither in this nor in any other context can it be coreferential with the subject ‘little prince’.

\ea\label{ex:treis:14} 
\gll “(…)” y-í=ké’ xa’mm-ée-s qakkíchch-u láah-u, \textbf{isú} \textbf{kaa’ll-o<sí>ta} hashsh-o-sí=biiha\\
     {} say-\textsc{3m.pfv.cv=seq} ask-\textsc{3m.pfv-3m.obj} little-\textsc{m.nom} prince-\textsc{m.acc} \textsc{3m.acc} help-\textsc{3m.purp.ss<3m.obj>} want-\textsc{3m.pfv-3m.obj.rel=reas2}\\
\glt ‘“(…)” said the little prince to him (*himself), because he wanted to help him (*himself).’ \citep[44]{Saint-Exupéry2018}\z

\subsubsection{Oblique domain}\label{sec:treis:3.2.2}

Kambaata also makes use of the reflexive noun \textit{gag-á} ‘self’ to signal coreference between the subject of a clause and its indirect or oblique objects. In (\ref{ex:treis:15}), the dative-marked beneficiary is coreferential with the subject ‘doves’. In (\ref{ex:treis:16}), the ablative-marked source is coreferential with the \textsc{2sg} subject. In (\ref{ex:treis:17}), the locative-marked indirect object is coreferential with the \textsc{3f} \textsc{(=} \textsc{3pl)} subject of its clause. 

\ea\label{ex:treis:15}
\gll Wól-i-s handar-ití-i (…) \textbf{gag-íiha-n-sa} it-táa=r-a bajig-óon hacc-itáyyoo’u \\
     other-\textsc{f.nom-def} doves-\textsc{f.nom-add} {} self-\textsc{m.dat-l-3pl.poss} eat-\textsc{3f.ipfv.rel=nmz4-m.acc} happiness-\textsc{f.icp} look.for.\textsc{mid-3f.prog}\\
\glt ‘And the other doves (...) were happily looking for food (lit. what they eat) for themselves.’ \citep[8.19f]{KEB1989}\z

\ea\label{ex:treis:16} 
\gll Át káan y-itaantí-i {\textbf{gag-íichchi-kke}-eti-ndo (…)}?\\
     \textsc{2sg.nom} \textsc{p\_dem1.m.acc} say-\textsc{2sg.ipfv.rel-nmz1.m.nom} self-\textsc{m.abl-2sg.poss-cop3-disj}\\
\glt (John 13:32) ‘Is this your own idea (lit. is it from yourself that you say this) or (…)?’ \citep[83]{KHTPH2005}\z

\ea\label{ex:treis:17} 
\gll \textbf{Gag-áan-ta-ssa} dikka’-áa-na wol-ú mann-á gad-dán “Ná’oot xumm-áan-n-u-a” y-itáa mann-íi (…) kúll-o-ssa\\
     self-\textsc{m.loc-l-3pl.poss} rely-\textsc{3f.ipfv.rel-crd} other-\textsc{m.acc} people-\textsc{m.acc} despise-\textsc{3f.ipfv.cv} {\db}\textsc{1pl.nom} peace-\textsc{ag-plv3-m.pred-m.cop2} say-\textsc{3f.ipfv.rel} people-\textsc{m.dat} {} tell-\textsc{3m.pfv-3pl.obj}\\
\glt ‘(He) said (…) to the people who trusted in themselves, who despised others and who said, “We are righteous”.’ \citep[16]{BSENodate}\z

The reflexive noun is also attested in morphologically complex oblique object NPs, e.g. those that are headed by a case-marked relational noun, such as \textit{al-éen} ‘on top’ (\ref{ex:treis:18}), or a case-marked nominalizer (\ref{ex:treis:19}). Relational nouns and nominalizers govern genitive-marked modifiers.

\ea\label{ex:treis:18} 
\gll \textbf{Gag-i-kkí} \textbf{al-éen} gar-é murat-úta aass-itaantí manch-ú {ik-koontí=da (…)}\\
     self-\textsc{m.gen-2sg.poss} top-\textsc{m.loc} justice-\textsc{f.gen} judgement-\textsc{f.acc} give-\textsc{2sg.ipfv.rel} person.\textsc{sgv-m.acc} become-\textsc{2sg.pfv.rel=cond}\\
\glt ‘If you are a person who (can) pass a fair judgment on yourself (lit. on top of your self) (…).’ \citep[40]{Saint-Exupéry2018}\z

\ea\label{ex:treis:19} 
\gll Ku mánch-u \textbf{gag-i-sí=tann-ée} xall-íi {sáww=y-u’nnáan (…)}\\
     \textsc{a\_dem1.m.nom} person.\textsc{sgv-m.nom} self-\textsc{m.gen-3m.poss=nmz3-f.dat} only-\textsc{m.dat} think=say-\textsc{3m.neg4}\\
\glt ‘This man does not only think about himself (lit. for the one of his self) (…).’ \citep[52]{Saint-Exupéry2018}\z

Kambaata does not have any adpositions but uses case markers or case-marked relational nouns to mark circumstantial adjuncts, e.g. locative adjuncts. Circumstantial adjuncts usually do not contain a reflexive noun in case of coreference with the subject. The phrase \textit{shiin-áan-ta-se} ‘beside her, at her side’ in (\ref{ex:treis:20}) is ambiguous and can be interpreted as ‘at her own side’ or ‘at her (= another feminine referent’s) side’. (See also \sectref{sec:treis:3.2.4} on the reflexive possessor.)

\ea\label{ex:treis:20} 
\gll Worr-iichch-ú mexx-é-nin \textbf{shiin-áan-ta-se} {xúud-d (…)}\\
     snakes-\textsc{sgv-m.acc} single-\textsc{mult-emp-emp} side-\textsc{m.loc-l-3f.poss} see-\textsc{3f.pfv.cv}\\
\glt ‘She suddenly saw a snake beside her(self) (lit. at her side) (…).’ [Elicited, DW2020-01-24]\z

\subsubsection{Long-distance domain}\label{sec:treis:3.2.3}

%Create link to Haspelmath this volume
In Kambaata, the antecedent of the reflexive noun does not have to be an argument of the same minimal clause. Even though my database does not provide a large number examples, there is sufficient proof that \textit{gag-á} ‘self’ qualifies as a long-distance reflexive, i.e. a reflexive noun that “can occur in a subordinate clause and take its antecedent in the matrix clause” (Haspelmath this volume: §9). In some diagnostic examples, the reflexive noun is found in an infinite verbal noun clause (\textsc{vnc)} and its antecedent in the matrix clause. In (\ref{ex:treis:21}), the antecedent of ‘self’ is the subject of the matrix clause – see the \textsc{1sg} index on the main verb. In (\ref{ex:treis:22}), the antecedent is the indirect object, expressed as a \textsc{2sg} object pronoun, of the main verb. 

\ea\label{ex:treis:21} 
\gll {\ob}\textbf{Gag-a-’í-i}, min-i-nné-e, hegeeg-u-’í-i muccur-ú ass-íi{\cb}\textup{\textsubscript{VNC}} abb-á yakitt-á ass-áamm\\
     {\db}self-\textsc{m.acc-1sg.poss-add} house-\textsc{m.acc-1pl.poss-add} area-\textsc{m.acc-1sg.poss-add} clean-\textsc{m.acc} make-\textsc{m.dat} big-\textsc{m.acc} effort-\textsc{m.acc} make-\textsc{1sg.ipfv}\\
\glt ‘I will make a great effort to clean myself, our house and my environs.’ \citep[4.120]{KEB1989}\z

\ea\label{ex:treis:22} 
\gll {\ob}(…) \textbf{gag-á-kk} mann-íi hor-íi<n>ka lall-íis-u{\cb}\textup{\textsubscript{VNC}} hasis-áno-he\\
     {\db}{} self-\textsc{m.acc-2sg.poss} people-\textsc{m.dat} all-\textsc{m.dat<emp>} appear-\textsc{caus1-m.nom} be.necessary-\textsc{3m.ipfv-2sg.obj}\\
\glt (John 7:4) ‘(…) you need to show yourself to everybody (lit. to show yourself to everybody is necessary for you).’ \citep[32]{KHTPH2005}\z

%Create link for Haspelmath this volume
Examples (\ref{ex:treis:21})--(\ref{ex:treis:22}) do not seem surprising from the perspective of European languages where reflexive pronouns can be employed in the non-finite long-distance domain (cf. Haspelmath (this volume: §9)). However, Kambaata goes a step further. As (\ref{ex:treis:23}) illustrates, an antecedent can just as well be coreferential with a reflexive noun in a finite subordinate clause. The ablative-marked standard of comparison \textit{gag-íichchi-s} ‘from/than himself’ – found in a relative clause inside another relative clause that modifies the subject of the main clause – is coreferential with the \textsc{3m} subject of the hierarchically superior matrix clause, i.e. the subject indexed on \textit{he’-anó} ‘(who) lives’.

\ea\label{ex:treis:23} 
\gll {\ob}Mát-o dooll-áan {\ob}{\ob}haraarím-a-s mát-o \textbf{gag-íichchi-s} kank-á<n>ka abb-itúmb-o{\cb}\textup{\textsubscript{RC}} plaaneet-í al-éen he’-anó{\cb}\textup{\textsubscript{RC}}-na {\ob}jaal-á has-áyyoo{\cb}\textup{\textsubscript{RC}} qakkíchch-u láah-u yóo’ ikke]\textup{\textsubscript{Main C}}\\
     {\db}one-\textsc{m.obl} time-\textsc{m.loc} {\db}{\db}width-\textsc{f.nom-3m.poss} one-\textsc{m.obl} self-\textsc{m.abl-3m.poss} that.much-\textsc{m.acc<emp>} become.big-\textsc{3f.neg5-m.obl} planet-\textsc{m.gen} top-\textsc{m.loc} live-\textsc{3m.ipfv.rel-crd} {\db}friend-\textsc{m.acc} look.for.\textsc{mid-3m.prog} little-\textsc{m.nom} prince-\textsc{m.nom} \textsc{cop1.3} \textsc{pst}\\
\glt ‘Once upon a time there was a little prince who lived on a planet the width of which was not much bigger than (the little prince) himself and who was looking for a friend.’ \citep[20]{Saint-Exupéry2018}\z

\subsubsection{Adpossessive domain}\label{sec:treis:3.2.4}

The adnominal possessor of a non-subject participant can be coreferential or non-coreferential with the subject. Kambaata does not make a distinction between subject-coreferential and subject-disjoint free possessor (genitive) pronouns or possessive suffixes. In (\ref{ex:treis:24}), the suffix \textit{-s} \textsc{3m.poss} on an instrumental-comitative-perlative participant is coreferential with the subject ‘Father God’, whereas the subject ‘he’ (as indexed on the verb) and the possessor are disjoint in (\ref{ex:treis:25}).

\ea\label{ex:treis:24} 
\gll Ánn-u Magán-u \textbf{beet-íin-ta-s} {ább-unta (…)}\\
     father-\textsc{m.nom} God-\textsc{m.nom} son-\textsc{m.icp-l-3m.poss} be.glorified-\textsc{3m.purp.ds}\\
\glt (John 14:13) (Literal translation of the Kambaata version:) ‘So that Father God is glorified through his (own) son (…).’ \citep[68]{KHTPH2005}\z

\ea\label{ex:treis:25} 
\gll \textup{A:} Manch-í\textup{\textsubscript{i}} min-í márr-o\textup{\textsubscript{j}}-ndo? -- \\
     {} person.\textsc{sgv-m.gen} house-\textsc{m.acc} go-\textsc{3m.pfv-q} {} \\
\gll \textup{B:} Márr-ee\textup{\textsubscript{j}} íkke, mánch-u-s\textup{\textsubscript{i}} yóo-ba’a, \textbf{beet-íin-ta-s}\textup{\textsubscript{i}} daqq-ámm-ee’u\textup{\textsubscript{j}} \\
     {} go-\textsc{3m.prf} \textsc{pst} person.\textsc{sgv-m.nom-def} \textsc{cop1.3-neg1} son-\textsc{m.icp-l-3m.poss} meet.\textsc{mid-pass-3m.prf}\\
\glt A: ‘Did he\textsubscript{j} go to the man’s\textsubscript{i} house?’ – B: ‘He\textsubscript{j} went there, (but) the man\textsubscript{i} was not there, he\textsubscript{j} met his\textsubscript{i} (= the man’s) son.’ [Elicited, DW2020-02-22]\z

Explicit coreference between the subject and the possessor of a non-subject participant in the same clause is expressed with a genitive-marked reflexive noun plus a possessive suffix, see ‘the mother’ and ‘her (own) part’ in (\ref{ex:treis:26}), ‘these’ and ‘their (own) language’ in (\ref{ex:treis:27}) and ‘they’ and the distributive phrase ‘(each) their (own) people’ in (\ref{ex:treis:28}).

\ea\label{ex:treis:26} 
\gll (…) am-atí-i \textbf{gag-i-sé} wud-íin qixxan-táa’u\\
     {} mother-\textsc{f.nom-add} self-\textsc{m.gen-3f.poss} side-\textsc{m.icp} get.ready-\textsc{3f.ipfv}\\
\glt ‘(…) and the mother gets ready for her (own) part.’ [Conversation about circumcision traditions, EK2016-02-23\_001]\z

\ea\label{ex:treis:27} 
\gll “Kúru \textbf{gag-i-ssá} afóo haasaaww-ú iitt-ít bá’-ee-haa=rr-a” y-isiicc-iyyé!\\
     {\db}\textsc{p\_dem1.pl.m.nom} self-\textsc{m.gen-3pl.poss} mouth-\textsc{m.acc} speak\textsc{-m.nom} love-\textsc{3f.pfv.cv} do.very.much-\textsc{3f.prf.rel-m.cop2=nmz4-m.pred} say-\textsc{caus2.mid-2pl.imp}\\
\glt ‘Make them say to themselves: “These are (people) who love to speak their (own) language (lit. mouth) very much.”’ [Symposium speech, DW2016-09-24]\z

\ea\label{ex:treis:28}
\gll \textbf{Gág-gag-i-ssá} mann-á<n>ka aag-is-sáa-haa\\
     \textsc{red-}self-\textsc{m.gen-3pl.poss} people-\textsc{m.acc<emp>} enter-\textsc{caus1-3f.ipfv.rel-m.cop2}\\
\glt ‘They intermarry in their own kin-group (lit. they marry each their own people).’ [Elicited, DW2004-11-03]\z

However, the genitive-marked reflexive noun is not strictly subject-oriented. It may also signal coreference between a possessor and a non-subject participant in the same clause. In my database, one finds, among others, examples in which the antecedent is the dative NP in a predicative possessive construction with \textit{yoo-} ‘exist’ (\textsc{cop1)}, see ‘for the ones who hunt’ in (\ref{ex:treis:29}), or an accusative object, see ‘the flower’ in (\ref{ex:treis:30}).

\ea\label{ex:treis:29} 
\gll (…) ées hugaax-xaa=r-iihá-a\textup{\textsubscript{i}} \textbf{gag-i-ssá}\textup{\textsubscript{i}} seer-u yóo-haa\\
     {} \textsc{1sg.acc} hunt-\textsc{3f.ipfv.rel=nmz4-m.dat-add} self-\textsc{m.gen-3pl.poss} rule-\textsc{m.nom} \textsc{cop1.3.rel-m.cop2}\\
\glt ‘(…) and the ones who hunt me have their own rules (lit. for the ones\textsubscript{i} who hunt me, there are their\textsubscript{i} own rules).’ \citep[70]{Saint-Exupéry2018}\z

\ea\label{ex:treis:30}
\gll (…) qakkíchch-u láah-u fiit-ichch-úta\textup{\textsubscript{i}} ankar-í ankar-í \textbf{gag-i-sé}\textup{\textsubscript{i}} burcuq-óonin {iffíshsh (…)} \\
     {} little-\textsc{m.nom} prince-\textsc{m.nom} flower-\textsc{sgv-f.acc} night-\textsc{m.acc} night-\textsc{m.acc} self-\textsc{m.gen-3f.poss} glass-\textsc{m.loc-emp} close.3\textsc{m.pfv.cv}\\
\glt ‘(…) the little prince shuts the flower\textsubscript{i} under her\textsubscript{i} glass (globe) every night and (…)’ \citep[91]{Saint-Exupéry2018}\z

%Create link for Haspelmath this volume
There are even several attested instances in which the reflexive noun is coreferential with an antecedent in an embedded clause: In (\ref{ex:treis:31}), \textit{gag-i-sí} ‘his own’ is coreferential with the direct object \textit{man-ch-ú} ‘man’ (\textsc{acc)} in the relative clause (\textsc{rc}). In (\ref{ex:treis:32}), \textit{gag-i-ssá} ‘their own’ is coreferential with the dative possessor in the conditional clause. In the adpossessive domain, Kambaata thus violates the cross-linguistic tendency of antecedent-reflexive asymmetry, which states that “[t]he antecedent must be higher on the rank scale of syntactic positions than the reflexive pronoun” (Haspelmath this volume: §7).\footnote{A consulted native speaker confirmed that \textit{-sí} could in principle also be coreferential with \textit{díinu} ‘enemy’ \textsc{(nom}) but that world knowledge would make a listener favor the first interpretation.}

\ea\label{ex:treis:31} Periphrasis of proverb in common speech \\
\gll {\ob}Manch-ú\textup{\textsubscript{i}} abbíshsh gen-anó{\cb}\textup{\textsubscript{RC}} díin-u\textup{\textsubscript{j}} \textbf{gag-i-sí}\textup{\textsubscript{i}} ilam-íichch ful-áno\\
     {\db}person.\textsc{sgv}-\textsc{m.acc} exceed.\textsc{caus1.3m.pfv.cv} harm-\textsc{3m.ipfv.rel} enemy-\textsc{m.nom} self-\textsc{m.gen-3m.poss} relatives-\textsc{m.abl} come.out-\textsc{3m.ipfv}\\
\glt ‘A person’s worst enemy is found among his relatives (lit. An enemy\textsubscript{j} who harms a person\textsubscript{i} very much comes out from his\textsubscript{i} own relatives).’ \citep[115]{AlamuAlamaayo2017}\z

\ea\label{ex:treis:32} Periphrasis of proverb in common speech\\
\gll {\ob}Ám-at il-áa\textup{\textsubscript{i}} ánn-u gizz-íi\textup{\textsubscript{j}} yoo-ba’í=dda{\cb} \textbf{gag-i-ssá}\textup{\textsubscript{i+j}} hé’-u<n>ku bárch-i-ta\\
    {\db}mother-\textsc{f.nom} children-\textsc{f.dat} owner-\textsc{m.nom} cattle-\textsc{m.dat} \textsc{cop1.3-neg1.rel=cond} self-\textsc{m.gen-3pl.poss} live-\textsc{m.nom<emp>} misery-\textsc{f.pred-f.cop2}\\
\glt ‘If children\textsubscript{i} have no mother (and) cattle\textsubscript{j} no owner (lit. if there is not a mother for children (and) an owner for cattle) their\textsubscript{i+j} life is a misery.’ \citep[10]{AlamuAlamaayo2017}\z

The use of the reflexive noun in the adpossessive domain is optional and serves the purpose of emphasis. This can be illustrated with examples from natural language use, such as (\ref{ex:treis:33}), in which possession is expressed by juxtaposing a regular genitive pronoun and a genitive reflexive noun. 

\ea\label{ex:treis:33} 
\gll Kúun ammoonsíi kíi-haa-ba’a, \textbf{íi-haa}, \textbf{gag-í-’e-a<n>ka} béet-u\\
     \textsc{p\_dem1.m.nom} however \textsc{2sg.gen-m.cop2-neg1} \textsc{1sg.gen-m.cop2} self-\textsc{m.gen-1sg.poss-m.cop<emp>} son-\textsc{m.pred}\\
\glt ‘But this is not yours, (it) is mine, (it) is my own son.’ [Narrative, TH2003-05-28\_001]\z

The optionality of the reflexive noun is also reflected in two variants of the same proverb in (\ref{ex:treis:34})-(\ref{ex:treis:35}): the first uses the genitive pronoun \textit{isé} 3\textsc{f.gen} ‘her’ (\ref{ex:treis:34}), while the second uses the reflexive noun \textit{gag-i-sé} ‘her own’ (\ref{ex:treis:35}).

\ea\label{ex:treis:34} Proverb variant 1 \\
\gll Ball-ó wonan-á mogga’-óo beet-í=biit \textbf{isé} beet-í ar-é bar-í wonan-á hoog-gáa’i\\
     mother-in-law-\textsc{f.gen} enset.ring-\textsc{m.acc} steal-\textsc{3f.pfv.rel} son-\textsc{m.gen=nmz2.f.nom} \textsc{3f.gen} son-\textsc{m.gen} wife-\textsc{f.gen} day-\textsc{m.acc} enset.ring-\textsc{m.acc} loose-\textsc{3f.ipfv}\\
\glt ‘The son’s (wife) who stole (her) mother-in-law’s enset ring loses (her) enset ring on the day of \textbf{her} son’s wife(’s arrival).’\footnote{The enset (\textit{Ensete ventricosum}) is a multi-purpose plant cultivated in the highlands of southern Ethiopia. The fermented corm, the fermented pulp and the starch are used for human consumption. Fresh or dried leaves, midribs and leaf sheaths as well as the fibers extracted from the plant serve to produce household utensils and packaging material.} \citep[28]{Geetaahun2002}\z

\ea\label{ex:treis:35} Proverb variant 2 \\
\gll (…) \textbf{gag-i-sé} beet-í ar-é {bar-í (…)}\\
     {} self-\textsc{m.gen-3f.poss} son-\textsc{m.gen} wife-\textsc{f.gen} day-\textsc{m.acc}\\
\glt ‘(…) on the day of \textbf{her} \textbf{own} son’s wife(’s arrival).’ \citep[24]{AlamuAlamaayo2017}\z

\subsubsection{Bare reflexive noun}\label{sec:treis:3.2.5}

The possessive suffix on the reflexive noun can be dispensed with in contexts where the antecedent and the reflexive are impersonal or generic, as is often the case in proverbs (\ref{ex:treis:37}), in conversations about traditions (\ref{ex:treis:38}) or in general truths (\ref{ex:treis:39}). The suffix is also missing in the idiom \textit{gag-á} \textit{daqq-} ‘become an adult, come of age (lit. find oneself)’.

\ea\label{ex:treis:36}  Proverb\\
\gll Gaazhzh-ó hór-u<n>ku \textbf{gag-í}i fun[n]úq\\
     wage.war-\textsc{3m.pfv.rel} all-\textsc{m.nom<emp>} self-\textsc{m.dat} shove.away.\textsc{ideo}\\
\glt ‘All who wage war struggle for themselves (i.e. not for the collective good).’ \citep[51]{AlamuAlamaayo2017}\z

\ea\label{ex:treis:37} Conversation about mourning traditions\\
\gll (…) \textbf{gag-í} ilan-ch-ú, onxan-é ilan-ch-ú moog-eennó-o iill-án qax-ée waas-á qammas-áno-ba’a\\
     {} self-\textsc{m.gen} relatives-\textsc{sgv-m.acc} nearness-\textsc{f.gen} relatives-\textsc{sgv-m.acc} bury-\textsc{3hon.ipfv.rel-nmz1.m.acc} reach-\textsc{3m.ipfv.cv} extent-\textsc{m.dat} enset.food-\textsc{m.acc} take.a.bite-\textsc{3m.ipfv-neg1}\\
\glt ‘(…) (one) did not (even) take a bite of food until (people) buried one’s relative, (one’s) near relative.’ [EK2016-02-23\_003]\z

\ea\label{ex:treis:38} 
\gll \textbf{Gag-á} haww-íichch fa’-is-íi dánd-u ammóo qoorím-a-ta\\
     self-\textsc{m.acc} trouble-\textsc{m.abl} be.saved-\textsc{caus1-m.dat} be.able-\textsc{m.nom} however wisdom-\textsc{f.pred-f.cop2}\\
\glt (The horse advises the hare: It is good to have friends.) ‘But being able to save oneself from trouble is wise(r).’ \citep[3.118f]{KEB1989}\z

\subsection{Self-intensifying constructions}\label{sec:treis:3.3}

As in many languages of the world (see, among others, \citealt{Koenigetal2013, KoenigSiemund2000, GastSiemund2006}), the reflexive noun \textit{gag-á} is also used as a self-intensifier. The description in this section is preliminary, as the diverse non-reflexive functions of \textit{gag-á} are not yet well understood and still require further investigation. However, my corpus clearly shows that \textit{gag-á} has self-intensifying functions when used adnominally (in apposition to a preceding noun phrase) or on its own as an argument or adverbial adjunct. In the typological literature (\citealt{KoenigSiemund2000, Gast2002, GastSiemund2006}), the adnominal use of self-intensifiers is associated with an alternative-evoking function (roughly paraphrasable as ‘no one other than N’, ‘as opposed to others related to N’), whereas two functions linked to the adverbial use are labeled “adverbial-exclusive” or “actor-oriented” (‘on one’s own, alone, without help’) and “adverbial-inclusive” or “additive” (‘also, too’). However, in Kambaata, no correlation between syntactic position and meaning can be observed.\footnote{The following examples may give the (wrong) impression that the appositional use correlates with the alternative-evoking function and the non-appositional use with the “exclusive” and “inclusive” functions. This is, however, not the case, as other examples in my data show. Also note that – although all self-intensifiers in (\ref{ex:treis:39})-(\ref{ex:treis:41}) are (parts of) subjects – alternative-evoking and “inclusive” self-intensifiers are also attested as direct objects, indirect objects, and predicates.}

In (\ref{ex:treis:39}), \textit{gag-á} is used in apposition to a subject noun with which it shares case and gender values. The central referent, \textit{Kambáat-u} ‘Kambaata people’, is opposed to the contextually given foreign, non-native speaker of the Kambaata language.

\ea\label{ex:treis:39} 
\gll (…) Kambáat-u \textbf{gág-u<n>ku-s} haasaaww-anó=hanní=g-a ass-ámm hiir-ámm-ee’i-i íh-u hasis-áno-a\\
     {} Kambaata-\textsc{m.nom} self-\textsc{m.nom<emp>-3m.poss} speak-\textsc{3m.ipfv.rel=nmz2.m.gen=sim-m.acc} do-\textsc{pass.3m.pfv.cv} translate-\textsc{pass-3m.prf.rel-nmz1.m.nom} become-\textsc{m.nom} be.necessary-\textsc{3m.ipfv.rel-m.cop2}\\
\glt (Context: We didn’t want that the dialogues in the book sounded as if they were spoken by a foreigner.) ‘(The book) had to be translated in a way (that it sounded) as if Kambaata people themselves would speak.’ [Book launch speech, DW2018-03-12]\z

In (\ref{ex:treis:40}), the self-intensifying \textit{gag-á} expresses that the (male) addressee does not delegate or seek assistance but carries out the action himself.\footnote{See also (\ref{ex:treis:42}).} The example illustrates the so-called “adverbial-exclusive” function. The typological label is hardly suitable for Kambaata, as the self-intensifier is not used adverbially in (\ref{ex:treis:40}) but is the subject of the main clause.\footnote{Note, however, that ‘on one’s own’ could, alternatively, be expressed by the \textsc{icp-}marked form of ‘self’, i.e. \textit{gag-íin-}\textsc{poss} ‘by, with, through oneself’, in adverbial function.}

\ea\label{ex:treis:40} 
\gll (…) át harde’-oom-áan yoontí j-áata qabatt-óon \textbf{gág-u-kki-n} qo’rr-ít has-soontí=b-a mar-táant íkke\\
     {} \textsc{2sg.nom} youngsters-\textsc{stat-f.loc} \textsc{cop1.2sg.rel} time-\textsc{f.acc} belt-\textsc{f.icp} self-\textsc{m.nom-2sg.poss-emp} gird.\textsc{mid}-\textsc{2sg.pfv.cv} want-\textsc{2sg.pfv.rel=plc-m.acc} go-2\textsc{sg.ipfv} \textsc{past}\\
\glt (John 21:18) ‘When you were in your youth you dressed yourself and went where you wanted.’ (Following context: But when you are old you will stretch out your hands, and someone else will dress you and lead you where you do not want to go.) \citep[95]{KHTPH2005}\z

The third self-intensifying function, the so-called “adverbial-inclusive” function, is exemplified in (\ref{ex:treis:41}). Again, the self-intensifier is not used adverbially in Kambaata but on its own as the subject.

\ea\label{ex:treis:41} 
\gll (…) hamiil-agúd-aa bonx-ichch-í al-éen qakkíchch-ut gaaroríin-ch-ut afuu’ll-ítee’; \textbf{gág-u<n>ku-se-n} hamiil-agud-áta agud-dáyyoo’u \\
     {} cabbage-seem-\textsc{m.obl} leaf-\textsc{sgv-m.gen} top-\textsc{m.loc} tiny.\textsc{sgv-f.nom} chameleon-\textsc{sgv-f.nom} sit-\textsc{3f.prf} self-\textsc{m.nom<emp>-3f.poss-emp} cabbage-seem-\textsc{f.acc} seem-\textsc{3f.prog}\\
\glt (The chameleon, which we, which I see here now,) the tiny chameleon sits on a cabbage-colored leaf; (and) she, too (lit. herself), seems cabbage-colored.’ [Narrative, TD2016-02-11\_001]\z

One and the same clause can contain two forms of \textit{gag-á}, one in reflexive and the other in self-intensifying use, as seen in (\ref{ex:treis:42}). The genitive form \textit{gag-i-kkí} (lit.) ‘your self’s’ indicates coreference between the \textsc{2sg} subject and the possessor, the nominative form \textit{gág-u-kk} stressed that the addressee has to enforce their rights on their own.

\ea\label{ex:treis:42} Periphrasis of a proverb\\
\gll \textbf{Gag-i-kkí} gar-íta \textbf{gág-u-kk} aphph-íi aphphám-i\\
     self-\textsc{m.gen-2sg.poss} right-\textsc{f.acc} self-\textsc{m.nom-2sg.poss} grab.\textsc{mid-m.dat} struggle\textsc{-2sg.imp}\\
\glt ‘Enforce (lit. struggle to grab) your own rights yourself!’ (i.e. Nobody grants them to you.) \citep[138]{AlamuAlamaayo2017}\z

Self-intensifying functions constitute only a subset of the non-reflexive uses of \textit{gag-á}. The corpus also shows it in contexts such as (\ref{ex:treis:43}), in which \textit{gag-á} does not lend itself to a self-intensifying interpretation. With respect to (\ref{ex:treis:43}), a native speaker I consulted considered it interchangeable with a free personal pronoun (\sectref{sec:treis:2.1}), which here would be \textit{isso’ootí-i} \textsc{3pl.nom-add.}\footnote{Note also that in a synonym matching exercise in a schoolbook, \textit{gág-u-nne} (self-\textsc{m.nom-1pl.poss)} ‘ourselves’ has to be paired with the personal pronoun \textit{ná’oot} (1\textsc{pl.nom)} ‘we’ \citep[4.122]{KEB1989}.}

\ea\label{ex:treis:43}
\gll (…) \textbf{gag-u-ssá-a} ammóo ma’nn-íta {af-fúmb-u-a=rr-a (…)}\\
     {} self-\textsc{m.nom-3pl.poss-add} and place-\textsc{f.acc} take-\textsc{3f.neg5-m.pred-m.cop2=nmz4-m.pred}\\
\glt (Context: They had only one ring of petals,) and they (lit. themselves) took up no room (…).’ \citep[30]{Saint-Exupéry2018}\z

\section{Middle derivation}\label{sec:treis:4}

Kambaata verb roots end in a single consonant or a consonant cluster.\footnote{Only a single verb root ends in a vowel: \textit{re-} ‘die’. If the root is followed by a vowel-initial morpheme, \textit{h} is inserted to avoid a vowel sequence.} The root can be followed by one or several word-class maintaining or word-class changing derivational morphemes, which in turn are followed by inflectional morphemes. The most productive derivational categories on verbs are causative, passive, middle and reciprocal. Kambaata has a short (or simple) causative -\textit{(i)s} (\textsc{caus1)} and a long (or double) causative -\textit{(i)siis} (\textsc{caus2).} Their distribution is partly determined by the valency of the base, but is also partly lexicalized (and thus not predictable). The passive is marked by \textit{-am}, e.g. \textit{shol-} ‘cook’ > \textit{shol-am-} ‘be cooked’, \textit{biix-} ‘break (tr.)’ > \textit{biix-am-} ‘be broken, break (intr.)’. Kambaata only has one labile verb: \textit{gid-} ‘be(come) non-tactile cold; make (someone) feel non-tactile cold’.

The middle is realized by two predominately phonologically conditioned allomorphs: -\textit{aqq} /ak’ː/ and –\textit{’} /ʔ/. The first allomorph is used on verb stems ending in a consonant cluster, e.g. \textit{iyy-} ‘carry’ > \textsc{mid:} \textit{iyy-aqq-} ‘carry for one’s benefit, endure’, \textit{quss-} ‘rub’ > \textit{quss-aqq-} ‘rub oneself’, or on stems ending in an ejective consonant, e.g. \textit{x} /t’/ in \textit{maax-} ‘hide’ > \textit{maax-aqq-} ‘hide for/in oneself’. The second allomorph is suffixed to stems that end in a sonorant, that in turn triggers metathesis to satisfy the phonotactic constraints of Kambaata, see e.g. \textit{mur-} ‘cut’ > \textit{mu’rr-} /muʔr-/ ‘cut oneself’, \textit{fan-} ‘open’ > \textit{fa’nn-} /faʔn-/ ‘open for one’s benefit’. Stems ending in a single obstruent can either be marked as middle with \textit{-aqq}, e.g. \textit{xuud-} ‘see’ > \textit{xuud-aqq-} ‘see, consider oneself’, or with the second allomorph. In the latter case, the sequence of an obstruent plus a glottal stop is realized as a geminate ejective consonant, e.g. /g+ʔ/ = /k’ː/ in \textit{dag-} ‘know, find’ > \textit{daqq-} ‘know, find for one’s benefit’ and /f+ʔ/ = /p’ː/ in \textit{huf-} ‘comb’ > \textit{huphph-} ‘comb oneself’. The choice of the first or second allomorph after single obstruents seems partly lexically determined, partly a case of free variation.

The middle does not reduce the valency of the verb. It has three discernibly different functions, the expression of autobenefactivity (\sectref{sec:treis:4.1}), reflexivity (\sectref{sec:treis:4.2}) and emotional involvement of the speaker (\sectref{sec:treis:4.3}). The middle is also part of the reciprocal derivation (\sectref{sec:treis:4.4}).

\subsection{Autobenefactive}\label{sec:treis:4.1}

As in all East Cushitic languages (cf. \citealt{Mous2004}), the most productive interpretation of the middle marker in Kambaata is to express that the subject of the clause is the beneficiary of the event expressed by the verb. There are apparently no semantic restrictions on the verbs that can be used with an autobenefactive middle marker. In (\ref{ex:treis:44}) the autobenefactive middle morpheme is on the verb \textit{laa’ll-} \textsc{‘}search and call (for a missing animal)’, in (\ref{ex:treis:45}) on the verb \textit{xa’mm-} ‘ask’, and in (\ref{ex:treis:46}) on the verbs \textit{ass-} ‘do’ (irregular middle form: \textit{eecc-}) and \textit{min-} ‘build’.

\ea \label{ex:treis:44} 
\gll Laa’ll-\textbf{aqq}-ayyoo’í-i xuud-eemma=dá-a m-á y-éen maassa’-éenno-la?\\
     search.call\textsc{-mid-3m.prog.rel-nmz1.m.acc} see-\textsc{3hon.pfv.rel=cond-add} what-\textsc{m.acc} say-\textsc{3hon.pfv.cv} bless-\textsc{3hon.ipfv-mit}\\
\glt ‘And if one comes across (lit. sees) someone who is searching and calling (for a missing animal) for his/her own benefit, what does one say to bless (him/her)?’ [Conversation on blessings, AN2016-02-19\_002]\z

\ea\label{ex:treis:45} 
\gll Mát-u qabaaxxáam-u adab-óohu qabaaxxáam-oa<n>ka manch-í min-í márr-ee’u, beet-úta xa’mm-\textbf{aqq}-óta\\
     one-\textsc{m.nom} rich-\textsc{m.nom} boy-\textsc{m.nom} rich-\textsc{m.obl<emp>} person.\textsc{sgv-m.gen} house-\textsc{m.gen} go-\textsc{3m.prf} daughter-\textsc{f.acc} ask-\textsc{mid-}3\textsc{m.purp.ss}\\
\glt ‘A rich young man (lit. boy) went to a rich man’s house in order to ask for the daughter for his own benefit.’ [Narrative, EK2016-02-12\_003]\z

\ea\label{ex:treis:46} 
\gll Gizz-á hoolam-á ir-á xáaz-z qú’mm=\textbf{eecc}-ít min-í \textbf{mi’nn}-itóo’u\\
     money-\textsc{m.acc} much-\textsc{m.acc} time-\textsc{m.acc} gather-\textsc{3f.pfv.cv} gather=do.\textsc{mid}\textsc{-3f.pfv.cv} house-\textsc{m.acc} build.\textsc{mid}-\textsc{3f.pfv}\\
\glt ‘After having saved money for many years, they could build a house for their own benefit.’ [Elicited, DW2020-01-24]\z

The autobenefactive function of the middle derivation could, in principle, also be analyzed as a subtype of the reflexivizing function, namely as one indicating coreference of the subject and a dative beneficiary adjunct.
     
\subsection{Reflexive}\label{sec:treis:4.2}

In (\ref{ex:treis:4}), the middle derivation was shown to be able to mark on its own that the subject and the direct (accusative) object are coreferential; another example is given in (\ref{ex:treis:47}). Overall, however, examples of this type seem to be rare. There are no clear cases in which the middle derivation alone marks coreference of the subject and a participant other than the direct object (if we exclude the beneficiary adjunct of \sectref{sec:treis:4.1}). And even in prototypical reflexive situations, as in (\ref{ex:treis:4}) and (\ref{ex:treis:47}), the middle morpheme is often not the only reflexivizer but rather an additional reflexivizing device besides the reflexive noun, as elaborated on at the end of this section).

\ea \label{ex:treis:47}
\gll Sull-\textbf{aqq}-ée’u\\
     choke.with.rope-\textsc{mid-3m.pfv}\\
\glt (Speaking about the actual cause of someone’s death whom the addressee thought to have died from an illness) ‘He hung himself.’ [Elicited, DW2020-01-24]\z

In contrast, we commonly find the middle morpheme on verbs of grooming and bodily care in Kambaata. Grooming and bodily care is typically self-directed, so the coreference of the carer and the cared is expected, and in many languages of the world, this coreference relations remains unmarked or marked by shorter morphemes if compared to the marking of prototypical reflexive situations (cf. \citealt{Kemmer1994}). In Kambaata, with verbs of grooming and bodily care, reflexivity cannot be doubly expressed by a middle morpheme and a reflexive noun. If the noun \textit{gag-á} ‘self’ is used with such verbs, it does not have a reflexive but a self-intensifying meaning; recall the self-intensifier with the verb \textit{qo’rr-} ‘gird’ in (\ref{ex:treis:40}).

Sometimes the root from which a middle verb was derived is not, or is no longer, attested in the language, and the middle verb forms a pair with a causative verb (\tabref{tab:treis:4}). Here the speaker is bound to overtly express whether the action is carried out by the subject on him- or herself, or on someone else. 

\begin{table}
\caption{Grooming verbs (middle vs. causative stem)}
\label{tab:treis:4}
\begin{tabularx}{\textwidth}{lllX}
\lsptoprule
Root &  & Derivative & Translation\\
\hline
\textit{*aal-} & \textsc{mid}  & \textit{aa’ll-} (\ref{ex:treis:48}) & ‘wash (oneself)’\\
& \textsc{caus1} & \textit{aansh-} & \textsc{‘}wash (something/someone)’\\
\textit{*odd-} & \textsc{mid}  & \textit{odd-aqq-} (\ref{ex:treis:48}) & ‘wear, put on (one’s clothes)’\\
& \textsc{caus1} & \textit{odd-iis-} & ‘have (someone) wear, put on (clothes)’\\
\textit{*gunguul-} & \textsc{mid}  & \textit{gunguu’ll-}  & ‘cover one’s head’\\
& \textsc{caus1} & \textit{gunguushsh-} & ‘cover someone’s head’\\
\textit{*qor-} & \textsc{mid}  & \textit{qo’rr-} (\ref{ex:treis:40}) & ‘gird, put on (belt, skirt, trousers)’\\
& \textsc{caus2} & \textit{qor-siis-} & ‘have (someone) gird, put on 
(belt, skirt, trousers)’\\
\lspbottomrule
\end{tabularx}
\end{table}

\ea\label{ex:treis:48} 
\gll Bór-a gassim-á xóqq=y-ít miin-í-se \textbf{aa’ll}-ít \textbf{odd-aqq}-ít \textbf{huphph}-ít xaaloot-á mar-tóo’u\\
     \textsc{pn-f.nom} morning-\textsc{m.acc} get.up=say-\textsc{3f.pfv.cv} face-\textsc{f.acc-3f.poss} wash.\textsc{mid-}3\textsc{f.pfv.cv} put.on-\textsc{mid-3f.pfv.cv} comb.\textsc{mid-3f.pfv.cv} church-\textsc{m.acc} go-3\textsc{f.pfv}\\
\glt ‘Bora got up in the morning, washed her face, got dressed, combed her hair and went to church.’ [Elicited, DW2020-01-24]\z

The middle verbs in \tabref{tab:treis:5} are based on a verb root that usually\footnote{In the corpus we also find some rare examples in which the unextended verb root is used even if the target of bodily care is the subject itself.} expresses that an action of bodily care is carried out on a person that is non-coreferential with the subject. In contrast, the middle-derived form can only be interpreted as expressing coreference between the subject and the patient of bodily care. The clothes that are put on and the body parts that are the targets of bodily care can be overtly expressed as accusative objects, irrespective of whether the middle verb is of the type given in \tabref{tab:treis:4} or in \tabref{tab:treis:5}; see, e.g., \textit{miin-í-se} ‘her face’ in (\ref{ex:treis:48}).

\begin{table}
\caption{Grooming verbs (root vs. middle stem)}
\label{tab:treis:5}
\begin{tabularx}{\textwidth}{lllX}
\lsptoprule
Root & Translation & Middle & Translation\\
\hline
\textit{buur-} & ‘butter, anoint (s.o.)’ & \textit{buu’rr-} & ‘butter, anoint (oneself)’\\
\textit{dad-} & ‘braid, plait (s.o.’s hair)’ & \textit{daxx-} & ‘braid, plait (one’s own hair)'\\
\textit{huf-} & ‘comb (s.o.’s hair)’ & \textit{huphph-} & ‘comb (one's own hair)’ (\ref{ex:treis:48})\\
\textit{meed-} & ‘shave (s.o.)’ & \textit{meexx-} & ‘shave (oneself)’\\
\textit{miiq-} & ‘brush (s.o.’s) teeth’ & \textit{miiq-aqq-} & ‘brush (one’s own) teeth’\\
\textit{xaax-} & ‘wrap, tie around, have & \textit{xaax-aqq-} & \textsc{‘}wrap, tie around (oneself),\\
& (s.o.) wear (e.g. a scarf)' & & wear (e.g. a scarf)’\\
\lspbottomrule
\end{tabularx}
\end{table}

In cases of non-default coreference of subject and direct object (in the prototypical reflexive situation), it is common to find two reflexivizers, the reflexive noun and the middle derivation, in the same clause, as we saw in (\ref{ex:treis:5}) and is further illustrated in (\ref{ex:treis:49})-(\ref{ex:treis:50}). The reflexive noun seems to be the primary reflexivizer and the middle derivation an addition. The native speaker I consulted was reluctant to omit the reflexive nouns in (\ref{ex:treis:50}) and preferred the combination of the nominal and verbal reflexivizer. (An autobenefactive interpretation of the middle derivation in (\ref{ex:treis:50}) can be excluded.)

\ea\label{ex:treis:49} 
\gll \textbf{Gag-á-’} \textbf{egexx}-íi dand-áam-ba’a\\
     self\textsc{-m.acc-1sg.poss} hold.up.\textsc{mid-m.dat} be.able-\textsc{1sg.ipfv-neg1}\\
\glt ‘I cannot contain myself.’ \citep[37]{Saint-Exupéry2018}\z

\ea\label{ex:treis:50} 
\gll Jáal-a-’ \textbf{gag-á-se} abbís-s qac-úta \\
     friend-\textsc{f.nom-1sg.poss} self-\textsc{m.acc-3f.poss} exceed.\textsc{caus1-3f.pfv.cv} thin-\textsc{f.acc}\\
\gll lókk-a-se ammóo cúlu=at-tumb-úúta ass-ít xuud-\textbf{aqq}-ít\\
     leg-\textsc{f.nom-3f.poss} and beautiful=do-\textsc{3f.neg5-nmz1.f.acc} do-\textsc{3f.pfv.cv} see-\textsc{mid-3f.pfv.cv}\\
\gll \textbf{gag-á-se} shigíg=\textbf{eecc}-ít bá’-ee-taa\\
     self-\textsc{m.acc-3f.poss} repel=do.\textsc{mid-3f.pfv.cv} do.very.much-\textsc{3f.prf.rel-f.cop2}\\
\glt ‘My friend considers herself too thin and her legs ugly, she hates herself deeply.’ [Elicited, DW2020-01-24]\z

\subsection{Emotional involvement}\label{sec:treis:4.3}

The middle derivation has also acquired an intersubjective meaning and expresses the emotional involvement of the speaker – and not the subject – in a state-of-affairs. The three functions of the middle derivation – reflexive, autobenefactive and emotive – are contrasted in (\ref{ex:treis:51})-(\ref{ex:treis:53}), which all contain the verb \textit{aass-} ‘give’. In (\ref{ex:treis:51}), the subject and the indirect object, the recipient of ‘give’, are coreferential. In (\ref{ex:treis:52}), the subject is the beneficiary of a gift (or rather a bribe), but not the recipient. In (\ref{ex:treis:53}), the speaker is emotionally touched by the event that he observes.

\ea\label{ex:treis:51} Reflexive\\
\gll Gag-íiha-n-se abb-áta ma’nn-íta aass-\textbf{aqq}-itóo’u\\
     self-\textsc{m.dat-l-3f.poss} big-\textsc{f.acc} place-\textsc{f.acc} give-\textsc{mid-3f.pfv}\\
\glt ‘She attributed (lit. gave) an important place to herself.’ [Elicited, DW2020-01-24]\z

\ea\label{ex:treis:52} Autobenefactive\\
\gll Dáann-u isíi fírd-unta-s gizz-á aass-\textbf{aqq}{}-ée’u\\
     judge-\textsc{m.nom} \textsc{3m.dat} judge-\textsc{3m.purp.ds-3m.obj} money-\textsc{m.acc} give-\textsc{mid-3m.pfv}\\
\glt ‘So that the judge would decide for him\textsubscript{i}, he\textsubscript{i} gave (the judge) money for his\textsubscript{i} own benefit.’ [Elicited, DW2020-01-24]\z

\ea\label{ex:treis:53} Emotive\\
\gll Ább-u mánn-u aass-áni-yan xúujj ciil-uhú-u m-á-ndo aass-\textbf{aqq}-ée’u \\
     big-\textsc{m.nom} people-\textsc{m.nom} give-\textsc{3m.ipfv.cv-ds} see.\textsc{3m.pfv.cv} infant-\textsc{m.nom-add} what-\textsc{m.acc-q} give-\textsc{mid-3m.pfv}\\
\glt (How amazing! How moving!) ‘The little child saw adults give (something, e.g. to the guests), then he also gave something (to them).’ [Elicited, DW2020-01-24]\z

\subsection{Reciprocity}\label{sec:treis:4.4}

A sequence of a middle and a passive morpheme regularly gives rise to a reciprocal, e.g. \textit{gomb-} ‘push’ > \textit{gomb-aqq-am-} ‘push each other’, \textit{dag-} ‘find’ > (*\textit{dag-ʔ-am-} >) \textit{daqq-am-} ‘meet (lit. find each other)’ (\ref{ex:treis:25}), \textit{mazees-} ‘injure’ > (*\textit{mazees-ʔ-am-} >) \textit{mazeecc-am-} ‘injure each other’, \textit{y-} ‘say’ > \textit{y-aqq-am-} ‘say to each other’ (\ref{ex:treis:54}).

\ea\label{ex:treis:54} 
\gll \MakeUppercase{ā}ā, āā, kúun y-\textbf{aqq-am}-móommi-a bár-i\\
     yes yes \textsc{p\_dem1.m.nom} say-\textsc{mid-pass-1pl.pfv.rel-m.cop2} day-\textsc{m.pred}\\
\glt ‘Yes, yes, it is the day we agreed on (lit. we said to each other).’ \citep[83]{Saint-Exupéry2018}\z

\section{Conclusion}\label{sec:treis:5}

Kambaata has a nominal and a verbal reflexivizer, both of which are multifunctional and also used in non-reflexive functions. 

The reflexive noun \textit{gag-á} ‘self’, which regularly combines with a possessive suffix, is primarily used to signal that the direct, indirect or oblique object is coreferential with the subject of the same clause. If the reflexive noun were replaced by a free personal pronoun or a bound object pronoun on the verb, the subject and these object pronouns would necessarily be considered referentially disjoint. 
While the reflexive noun most commonly expresses a coreference relation between arguments of a minimal clause (\sectref{sec:treis:3.2.1}, \sectref{sec:treis:3.2.2}), I have also presented evidence that the antecedent of \textit{gag-á} `self' can be found outside this restricted syntactic domain. Examples in which the reflexive noun in an infinite or finite subordinate clause is coreferential with the subject of the matrix clause justify the analysis of \textit{gag-á} `self' as a long-distance reflexive (\sectref{sec:treis:3.2.3}).

Whereas a non-reflexive (in)direct or oblique object pronoun rules out a coreference relation with the subject NP, an adnominal possessor of a non-subject noun phrase can be interpreted in two ways: as coreferential or non-coreferential with the subject. In the adpossessive domain, the reflexive noun serves to signal coreference explicitly and thus has a disambiguating function. As shown in \sectref{sec:treis:3.2.4}, the antecedent of the adnominal reflexive noun is not necessarily the subject of the clause but may also be another participant, even in a subordinate clause. 

Apart from having a reflexive function, the noun \textit{gag-á} ‘self’ is also used as a self-intensifier (\sectref{sec:treis:3.3}).

The middle derivation -\textit{aqq} / –\textit{’} can serve as a reflexivizer in prototypical reflexive situations, i.e. situations in which coreference between arguments is unexpected. It can only signal coreference between the subject and a direct (accusative) object - but even in this context it is rarely the only reflexivizing means in its clause. Instead it often cooccurs with a reflexive noun (\sectref{sec:treis:4.2}). In less typical reflexive situations in which subject-object coreference (self-affectedness of the subject) is the default, as in the case of grooming and bodily care, the middle morpheme is used as the sole marker of coreference. If the noun \textit{gag-á} ‘self’ occurs in the clause of grooming and bodily care verbs, it necessarily has a self-intensifying function. As in related East Cushitic languages, the most productive synchronic function of the middle derivation is the expression of autobenefactivity (\sectref{sec:treis:4.1}). In Kambaata; it has furthermore adopted an intersubjective interpretation (\sectref{sec:treis:4.3}).

\section*{Acknowledgements}

Work on this chapter was in part supported by a grant of the \textit{Institut} \textit{des} \textit{Sciences} \textit{Humaines} \textit{et} \textit{Sociales} of the \textit{Centre} \textit{National} \textit{de} \textit{la} \textit{Recherche} \textit{Scientifique} (CNRS) under the programme \textit{Soutien} \textit{à} \textit{la} \textit{mobilité} \textit{international} \textit{2019.} I am indebted to my Kambaata consultants Alemu Banta, Teshome Dagne, Tessema Handiso, Eremiyas Keenore, Aman Nuriye and all the other Kambaata speakers I have been working with since 2002. I am especially grateful to Deginet Wotango Doyiso for his native speaker expertise and the fruitful discussions of the analyses proposed here. I thank the editors, an anonymous reviewer and Abbie Hantgan for their valuable comments on an earlier version of this paper.

%One could consider deleting the abbreviations that are already included in the Leipzig list.
\section*{Abbreviations}
\begin{tabularx}{.45\textwidth}{lQ}
\textsc{a\_dem} & adjectival demonstrative\\
\textsc{abl} & ablative\\
\textsc{acc} & accusative\\
\textsc{add} & additive\\
\textsc{ag} & agentive\\
\textsc{bdv} & benedictive\\
\textsc{caus} & causative\\
\textsc{cond} & conditional\\
\textsc{cop} & copula\\
\textsc{crd} & coordinative\\
\textsc{cv} & converb\\
\textsc{dat} & dative\\
\textsc{def} & definite\\
\textsc{disj} & disjunctor\\
\textsc{ds} & different subject\\
\textsc{emp} & emphasis\\
\textsc{f} & feminine\\
\textsc{gen} & genitive\\
\textsc{HEC} & Highland East Cushitic\\
\textsc{hon} & honorific, impersonal\\
\textsc{icp} & instrumental-comitative-perlative\\
\textsc{ideo} & ideophone\\
\textsc{imp} & imperative\\
\textsc{ipfv} & imperfective\\
\textsc{l} & linker\\
\textsc{loc} & locative\\
\textsc{m} & masculine\\
\textsc{mid} & middle\\
\end{tabularx}
\begin{tabularx}{.45\textwidth}{lQ}
\textsc{mit} & mitigator\\
\textsc{mult} & multiplicative\\
\textsc{neg} & negation\\
\textsc{nmz} & nominalizer\\
\textsc{nom} & nominative\\
\textsc{obj} & object\\
\textsc{obl} & oblique\\
\textsc{p\_dem} & pronominal demonstrative\\
\textsc{pl} & plural\\
\textsc{pfv} & perfective\\
\textsc{plc} & place nominalizer\\
\textsc{plv} & plurative\\
\textsc{pn} & proper noun\\
\textsc{poss} & possessive\\
\textsc{pred} & predicative\\
\textsc{prf} & perfect\\
\textsc{prog} & progressive\\
\textsc{pst} & past\\
\textsc{purp} & purposive\\
\textsc{q} & question\\
\textsc{reas} & reason clause marker\\
\textsc{red} & reduplication\\
\textsc{rel} & relative\\
\textsc{seq} & sequential\\
\textsc{sg} & singular\\
\textsc{sgv} & singulative\\
\textsc{sim} & similative, manner nominalizer\\
\textsc{ss} & same subject\\
\textsc{stat} & status noun derivation\\
\end{tabularx}

{\sloppy\printbibliography[heading=subbibliography,notkeyword=this]}

\end{document}