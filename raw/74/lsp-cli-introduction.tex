% \chapter{Introduction}
\label{sec:intro}

% \citet{Good54,Wilk89,Bend65,Dela12,Kohl58,Mane69a,Mane69b,Klei97,Taux21,Taux24,Ratt32a,Ratt32b,Nico52,Dupe84,Reim75,Tomp02,Klei99,Daku05,Daan94,Popp93,Doug66,Sali08,Reco03,Cans61,Trap06,Comr08b,Rowl65,Crou66,Gray69,Toup95,Crou03,Mane79,Dau97,Casa03a,Stew67,Casa03,Casa08,Kedr97,Madd84,Madd09,Awed02,Daku02a,Park02,Hyma85,Hyma07,Bonv88,Dixo10b,Amek05a,Amek06,mek08,Levi06,Brin05,Stas08,Dik97,Gund93,Waa71,Tcha07,Rowl66,Mora06,Mane82,Awed07,Assi07,Comr89,Bodo94,Mane75,Mane99,Awed9,Awed03,Delp79,Arit87,Ours89,Klei00,Tcha72,Alla01,Rytz66,Hasp08,Hell07,Hein97,Hein84,Amek07c,Gree78b,Chan09,Comr08,Card7,Corb06,Dixo86,Corb91,Grin00,Daku70,Vend57,Olaw99,Bodo97,Saan03,Daku07b,Daku08,Bour92,Nico07,Fraw92,Blen03,Jose08,Daku09,Muff02,Schw10,Wray05,Lade93,Aust09}


\chapter{General remarks on the language}
\label{sec:background}

Chakali  ({\sls tʃàkálɪ́ɪ́})\footnote{ISO 639-3: cli  \citep{lewi16}; Glottocode: chak1271 \citep{glot16}} is a language  spoken in seven communities in the Wa East District, Upper West Region of Ghana.   It is currently classified  into the \ili{Grusi} Southwestern (or Western) subgroup of the \ili{Gur} family, alongside \ili{Dɛg}, \ili{Vagla}, \ili{Tampulma}, \ili{Kyitu}/\ili{Siti}, \ili{Phuie}, \ili{Winyé}, and varieties of \ili{Sisaala} \citep{lewi14, glot16}. These minority languages are spoken in northwest Ghana, southwest Burkina Faso,  and northeast Ivory Coast.  The languages  \ili{Tampulma}, \ili{Vagla}, \ili{Dɛg}, and  \ili{Pasaale} -- a variety of  \ili{Sisaala} --  are the closest to Chakali in terms of mutual intelligibility.


The area where the language is spoken  is bordered to the east  by areas inhabited by \ili{Waali} ({\sls wáálɪ́ɪ́}) and \ili{Bulengi} ({\sls búléŋĩ́ĩ́}) speakers. These two languages are virtually undocumented languages, which,   diachronically, can provisionally be classified as Western  \ili{Oti-Volta} based on folk linguistic factors. \ili{Waali}, the language spoken in Wa and some surrounding villages (see Figure \ref{fig:map-b-and-w}), can be considered to be the lingua franca of the Upper West Region of Ghana  \citep{brin15d}. \ili{Bulengi}, on the other hand, is the language of \isi{Bulenga} (and some surrounding villages like Gilan, Chagu, and Dupari), a fast-growing town in terms of population and development. To the north, Chakali is bordered by \ili{Pasaale}-speaking villages, and Kpalewagu, whose inhabitants maintain a \ili{Mande} language known as \ili{Kantosi}. \ili{Tampulma} speakers are mainly found in some villages of the Northern Region, but a few villages to the west are within the Upper West Region's border (i.e. Holumuni and Belezing).  To the south and southwest lie \ili{Vagla}-speaking villages and the uninhabited Mole National Park.


\begin{sidewaysfigure}
 \centering

\includegraphics[height=4.8in]{Graphic/project-5.jpg}
% climap.pdf: 0x0 pixel, 0dpi, 0.00x0.00 cm, bb=

\caption[]{Chakali-speaking villages: \isi{Gurumbele},
\isi{Ducie}, \isi{Motigu}, \isi{Sogla} (variant spelling Sawla), \isi{Tuosa}, \isi{Tiisa}, and  \isi{Katua} (Reproduced from \citet{brin16})\label{fig:map-b-and-w}}
\end{sidewaysfigure}

\newpage 
\section{Previous work}
\label{sec:chrono}


The late English anthropologist Jack Goody presented the first linguistic data on the Chakali language, namely 38 words gathered  on August 29th, 1952,  in \isi{Katua} \citep[33]{Good54}. He is responsible for the identification of the existence of the language and the people who speak it.\footnote{There may be British and/or French colonial documents somewhere which mention {\it Chakali}. For instance,  it is known that French Captain  Louis Gustave Binger and his troop attacked some of Babatu's men in  \isi{Ducie}.  Binger's reports were impossible to get hold of. \citet[133]{Wilk89} writes ``Zabarima occupation of \isi{Ducie} occurred probably early in May 1897''.} The passage reads: 

\begin{quote} I do not know of any previous record of the existence of the 
group speaking this dialect. Although now living entirely within the 
administrative district of Wa, there is in their midst the village of Kandia 
inhabited only by \ili{Guang}-speaking Gonjas. The chiefship of Kandia was an 
important office in the \ili{Gonja} political system. Either at the time of the 
arrival of the British military forces or a little before, during the course of 
a war between the State of Wa, allied with Bole, and the Yabumwura, the senior 
chief of \ili{Gonja}, it fell within the orbit of Wa. The western section of the 
group comprising the villages of Chago, Bisikan, and Bulinga speaks Wala, i.e. the 
dialect of Dagari spoken within the State of Wa, and was certainly under the 
influence of the Chiefs of Wa before the European conquest. The Chief of 
Bulinga, the central village of this section, claims to have been a Kamboŋa (a 
semi-dependent war-chief) in relation to Wa. The eastern group of the Chakalle 
speak Chakalle and seem to have been under the suzerainty of the \ili{Gonja} Chief at 
Kandia. This group consists of the villages of \isi{Katua}, \isi{Tuosa}, \isi{Sogla}, \isi{Motigu}, 
\isi{Chasia}, \isi{Ducie} and \isi{Gurumbele}. \citep[3]{Good54} 
\end{quote}

Approximately ten years later, Chakali data is used to confirm the 
\ili{Grusi} 
cluster in \citet{Bend65}.\footnote{\ili{Grusi}  as a language cluster 
has been defined and confirmed in several publications \citep{Dela12, Kohl58, 
Bend65, Mane69a, Mane69b, Klei97}, but  the term  {\it Grusi} and its 
spelling \is{variant}variants (i.e. {\it Gurunsi}, {\it Grunshie}, {\it Gourounsi}, etc.)  
have always existed in the French and English colonial vocabulary without great 
unanimity on its designation \citep{Taux21, Taux24, Ratt32a, Ratt32b, Nico52, 
Dupe84}. }  The material, a list of 97 words, is said to have been produced by  
Mr. E. R. Rowland.  His notes have not been located and remain unpublished.  
\citet{Mane69a, Mane69b} reconstructs  a {\it  gurunsi commun}  based on an 
average of 80 words from twenty-six \ili{Grusi} languages.  He uses  
only 36 Chakali words, all of them extracted from \citet{Bend65}. 

\newpage 
In 
1974 and 1994, sociolinguistic surveys were carried out in the Chakali area by 
the Ghana Institute of Linguistics, Literacy and Bible Translation (GILLBT), 
formerly Ghana Institute of Linguistics (GIL), which is the Ghanaian branch of 
the Summer Institute of Linguistics (SIL) \citep{Reim75, Tomp02}. For these two 
surveys, the main goal  was to investigate the need of Chakali language 
development and to assess  \ili{Waali} comprehension. No language data is offered in 
  \citet{Tomp02}, and \citet{Reim75}  could not be found at the 
GILLBT headquarters in Tamale when I visited in 2008, nor obtained from one of its authors, the late 
Regina Blass.  In 1999, Ulrich Kleinewillinghöfer spent a few hours in Wa 
with Godfrey Bayon Tangu \citep{Klei99}. In this short period,  he  gathered 
approximately 150 words  and from them  inferred some generalizations on Chakali 
nominals. In 2001, a Brazilian  known as Pastor Ronaldo worked with two  
language consultants in order to start a vernacular literacy project. The 
initiative came from the Evangelical Church of Ghana.  Two illustrated booklets 
were written,  aiming at adult literacy.  The first 
booklet introduces the designed alphabet  and the second  
consists of  syllables and short sentences thematically organized. In 2005, 
 Mary Esther Kropp Dakubu spent two  days with an informant from Jayiri, 
gathering general information on Chakali  \citep{Daku05}. Her intention was to 
investigate the situation on site for a possible documentation project. Due to 
the condition of the road, she was not able to reach the villages where Chakali is spoken by the majority of the inhabitants.  Her unpublished report  presents  data which was believed to be representative of  Chakali, but which transpired to be an idiosyncratic mix of \ili{Waali} and Chakali, and some \ili{Bulengi},  the language spoken in \isi{Bulenga} and surrounding villages. Finally, there are other studies that deserve to be mentioned:  Henry Seidu Daannaa,   a native Chakali from \isi{Tuosa},  presents a retrospective study of the practice of indirect rule  which affected the social and political organization of  Chakali during the colonial administration \citep{Daan94};    Cesare Poppi conducted anthropological research which focused on issues related to knowledge, secrecy,  and initiation \citep{Popp93}, and  theoretical issues concerning the analysis of the representational status of masks, particularly the {\it \isi{Sigmaa}} masks which are cornerstones in the  Chakali belief system; finally, the work of \citet{Doug66},  \citet{Wilk89}, and \citet{Sali08}  are good  overviews on the role of the Chakali land and people in the  political and cultural history of Wa.

This was the complete list of work written on Chakali when I started the research in 2007. It shows that the language has been known to exist since 1954, yet very little work had been done, and much that was written  remains unpublished. Since then, some work has been published or distributed locally \citep{Kang07b, Kang07a,  brin08b, brin08, brin10, brin11, Brin12, brin15c, brin16}.\footnote{All of the information used in Sections \ref{sec:lects} and \ref{sec:vital}  are taken from \citet{brin15c}, a work on the vitality of the Chakali language and culture.}

\section{Chakali lects}
\label{sec:lects}

With Chakali,    three concepts can be identified. The term may be used  to name a land,  an ethnic group,   or a language.  However it would be wrong to assume that a member of the Chakali ethnic group  or someone living in Chakali land necessarily speaks the language.  This is what \citeauthor{Good54} describes when he writes: ``[t]he Chakalle who inhabit the eastern part of the Wa district are split into those speaking a language of the Mossi group and those speaking a \ili{Grusi} language. `Speaking a language' refers to the tongue which dominates in the child's play group; the eastern Chakalle who use a \ili{Grusi} language in this context are in fact mostly bilingual. The common name for the group derives from a recognition of uniformity in other social activities.''   \citet[2]{Good54}.  It is crucial to keep in mind that the notions of land, ethnicity, and language are intricately interwoven.   For instance, according to \citet{Daan94},  {\it Chakali}  consists of  thirteen communities and their inhabitants:  \isi{Bulenga}, \isi{Tiisa}, \isi{Sogla} ({variant} spelling Sawla), \isi{Tuosa}, Chagu, \isi{Motigu}, \isi{Ducie}, \isi{Katua}, Bisikan, Kandia, Dupari, Gilan, and \isi{Gurumbele}.  By contrast, the sociolinguistic censuses which I carried out indicate that {\it Chakali} is the language of the inhabitants and forefathers of  \isi{Tiisa}, \isi{Sogla}, \isi{Tuosa},  \isi{Motigu}, \isi{Ducie}, \isi{Katua}, and  \isi{Gurumbele} exclusively. 

The collective demonym for the people of the latter seven villages literally translates to {\sls m̩̀ ŋmá kàà} {\it (lit.)}   `I say that', whereas that of  the people of \isi{Bulenga} and surrounding villages translate to   {\sls ŋmɪ́nɪ́ŋ dʒɔ̀ŋ}  `What is it?'.   In this folk-sociolinguistic categorisation, the Waala are the {\sls ǹ̩ jɛ́ jàà} `I say that'.\footnote{\citet[525]{Ratt32b} writes that the Awuna, a \ili{Kasem} dialect also  known as Aculo \citep[147]{nade89}, has earned its appellation based on a habit of ``prefacing an observation with the words''  {\it a wun a} `I say'.  It is indeed the case that a Chakali can open a sentence with  {\it m̩̀ ŋmá kàà, ...}  `I say that, (...)'. To hear the  Ghanaian English opening expression {\it à sé ɛ̃̀ɛ̃̀}  `I say eh, (...)'   in Wa, with the last word being a complementiser introducing a new clause, is not unusual.}  

Another popular distinction is that of `black' and `white' Chakali: respectively,  {\sls tʃàkàlbúmmò} `Black Chakali'  is a notion which connotes with secretive individuals and possessors of powerful medicine. To the best of my knowledge, this is equivalent to what {\sls m̩̀ ŋmá kàà} represents. The notion of  {\sls tʃàkàlpʊ̀mmá}  `White Chakali' corresponds, according to my ‘Black Chakali’ consultants, to talkative people who cannot hold back. They comprise the inhabitants of \isi{Bulenga}, Dupari, Bisikan, Chagu, and Gilan, that
is, those villages included in what \citet[2--3]{Daan94} identifies as Chakali people, minus the
villages where the language is said to be indigenous.  Obviously, if one asks the same question
in \isi{Bulenga} and surrounding villages one may get a different interpretation of the distinction
between ‘black’ and ‘white’.\footnote{\citet[14--15]{Good54} reports a `Black Waala' and `White Waala' division, the former being the dominated group, that is commoners and pagan, while the latter being the dominant group, that is members of the chiefly lineage and Muslim. Tony Naden (p.c.) confirmed to me the existence of  `Black Dagomba', with no correlative `White', and suspected it to refer to the descendants of the original inhabitants in contrast to the aristocracy, therefore roughly Black = ‘commoner’ vs. White = ‘aristocracy’. In the case considered here, the interviews with `Black Chakali' individuals tell us about the resources people have available for telling their world and creating an identity. Assuming that the connotation of the division black/white is ruled/ruler, dominated/dominant, or commoner/chief, then it appears that despite being labeled as ‘black’, one can exploit this sense of the concept in order to associate one’s group with more positive cultural implications. This social categorisation is in need of further study.}


\begin{table} 
\centering
\caption{Collective Demonyms and associated villages}
\label{tab:demonym}
\begin{tabular}{llll}
\toprule

Demonym 1  & {\sls m̩̀ ŋmá kàà}  & {\sls ŋmɪ́nɪ́ŋ dʒɔ̀ŋ} & {\sls ǹ̩ jɛ́ jàà} \\\midrule
   Demonym 2   &      {\sls tʃàkàlbúmmò}    &      {\sls tʃàkàlpʊ̀mmá}   &   --      \\\midrule
   \citet[2-3]{Good54}    &      Eastern Chakali    &      Western Chakali   &   Waala     \\\midrule
   Village  &    \isi{Ducie}      &      \isi{Bulenga}    &   \isi{Wa}      \\
	      &   \isi{Gurumbele}      &     Dupari    & Busa        \\
	      &    \isi{Motigu}     & Bisikan         & Gurupie        \\
	    &      \isi{Sogla}   & Chagu         & Loggu        \\
	    &     \isi{Tiisa}    & Gilan         & Jayiri        \\
	    &       \isi{Tuosa}  &         & \isi{Chasia}        \\
      &       \isi{Katua}  &         &        \\ 
      \bottomrule
\end{tabular}
\end{table}

Table \ref{tab:demonym} organizes the information for convenience. It also constitutes a hypothesis to be tested since the denominations do not necessarily map one-to-one, the Western Chakali  and   Waala would need to be extended,  and discussions I had about these self-identifications were often confusing. For instance, some men interviewed in \isi{Tuosa} in 2014 told me that \isi{Tiisa}, \isi{Tuosa}, and \isi{Katua} are not {\sls m̩̀ ŋmá kàà}, but are {\sls tʃàkàlbúmmò}.

All the Chakali lects are mutually intelligible. Still, each village is recognised to have a set of unique features. Examples of lectal variation are provided in \citet{brin15c} and the dictionary  includes some lectal usages, but one recurrent illustration of folk-dialectology is  how each village would express `to eat yam':   \isi{Motigu}, \isi{Gurumbele}, \isi{Tuosa}, \isi{Tiisa},  and \isi{Katua}  `chew' yam {\it (tie)}, whereas \isi{Ducie}  `eat' yam {\it (di)}.  And while `yam' is pronounced {\it kpããŋ} in \isi{Motigu}, \isi{Gurumbele},  and \isi{Ducie}, it is pronounced {\it pɪɪ} in \isi{Tuosa}, \isi{Tiisa},  and \isi{Katua}. Thus, if someone says {\it tie kpããŋ}, he/she is easily identified as someone from either \isi{Gurumbele} or \isi{Motigu}.  The expression {\it di kpããŋ} is typically uttered by someone from  \isi{Ducie},  and {\it tie pɪɪ} by someone from \isi{Tuosa},  \isi{Tiisa}, and \isi{Katua}.

\section{Language vitality}
\label{sec:vital}

The number of Chakali speakers is close to 3500 individuals. It is spoken by all community members in \isi{Gurumbele} and \isi{Ducie}, and by the majority in \isi{Motigu} and \isi{Katua}. It is spoken to a lesser extent in \isi{Sogla}, \isi{Tuosa},  and \isi{Tiisa}.  In the other villages which are considered as  parts of Chakali land, people speak a language similar to \ili{Waali}, the language of  Wa, or \ili{Bulengi}, the language of \isi{Bulenga}. \ili{Waali} is known by the majority of Chakali speakers,  but is used differently from community to community.   Chakali is believed to be on the road to extinction: some believe that \ili{Waali} and \ili{Bulengi} are the languages which will be spoken throughout the whole of the Chakali villages in the coming decades.



\begin{sidewaysfigure}
\footnotesize
  \centering
 \begin{tabular}{p{5cm}p{3.5cm}p{3.5cm}p{3.5cm}}


 %\begin{Atabular}{ll}
\toprule
    Factors & \multicolumn{3}{c}{Measures}\\[1ex]\midrule
 &E1&E2&E3\\[1ex]\midrule

1. Intergenerational language transmission &  {severely endangered} 
(2) & {unsafe} (4)& {safe} (5)\\

2.  Absolute number of speakers  &
\multicolumn{3}{l}{[{ 3484}]}\\ 

3. Proportion of speakers within the total population &
\multicolumn{3}{l}{[{severely
endangered}  (2) ]}\\

4.  Trends in existing language domains & {highly limited domains}
(2)& {dwindling domains} (3)&{multilingual parity} (4)\\


5.  Response to new domains and media & 
\multicolumn{3}{l}{[{inactive-minimal}(0-1)]}\\

6.  Materials for language education and literacy & 
\multicolumn{3}{l}{[{no {orthography} available} (0)]}\\

7. Governmental and institutional language attitudes
and policies, including official status and use &
\multicolumn{3}{l}{[{active
assimilation} (2) ]}\\


8. Community members attitudes toward their own language &-&-&{all
members value their language and wish to see it promoted} (5) 
\\

9.  Amount and quality of documentation &
\multicolumn{3}{l}{[{undocumented-inadequate} (0-1) ]}\\


\bottomrule
  \end{tabular}
  
\caption[Estimated degree of endangerment for the E1, E2, and E3 
villages]{Estimated degree of endangerment for the E1 \{\isi{Tuosa}, \isi{Tiisa}, \isi{Sogla}\}, 
E2 \{\isi{Katua}, \isi{Motigu}\} and E3 \{\isi{Gurumbele}, \isi{Ducie}\}. A value within square brackets 
 applies to E1, E2, and E3 villages as a whole. The number in parentheses is a 
relative grade used in the  language vitality assessment  
\citep[see][7]{Reco03}} \label{tab:estimate-endangerment}

\end{sidewaysfigure}



\citet{brin15c} determines the vitality of  Chakali by i) examining sociological and historical factors that may be seen as linked to the language’s vitality and responsible for language change, and ii) using the answers to the questionnaire developed in \citet{Reco03}. It suggests a division of the Chakali villages into three groups, which are presented in Figure \ref{tab:estimate-endangerment}. \isi{Sogla}, \isi{Tiisa},  and \isi{Tuosa} correspond to the villages where the intergenerational transmission is ineffective and where \ili{Waali} is used in formal and informal domains. They are the endangered-1  villages (E1).  \isi{Motigu} and \isi{Katua} correspond to E2 villages. In both villages,  \ili{Waali} is encroaching on  Chakali in formal and informal domains. The situation is not alarming since Chakali is spoken by the majority and  the intergenerational transmission is effective, but, as outlined in the survey \citep[Section 2.2.2 in ][]{brin15c},  given the average population size of the villages and the recent conversion to Islam of their youth, among other factors, it is worth considering that a language shift to \ili{Waali} may take place within a short period of time.  A. B. Sakara and H. S. Daanaa, both born in \isi{Tuosa} and prominent Chakali figures, told me that Chakali was spoken by everyone in their village when they were children, i.e. in the 1950s and 1960s. There  are no signs indicating that the same language replacement which took place in \isi{Tuosa}  cannot take place in \isi{Motigu} and \isi{Katua}. Finally, the E3 villages, \isi{Gurumbele} and \isi{Ducie}, show the most effective intergenerational transmission of the Chakali language. Both villages also establish local alliances (i.e. marriage, common shrines, one assemblyman for both villages, etc.). \ili{Waali} is spoken and understood, yet it is usually spoken in specific  domains, essentially  in official visits from the district or regional capital conducted by governmental bodies,  and  to  \ili{Waali}-speaking visitors, traders,  or migrant farmers.


\section{Data collection method}
\label{sec:method}

Nearly every year since 2007 I  made a field trip to the Wa East District of Ghana, usually in the dry season, i.e.  a  period between February and  May.  Most of my stays were spent in a Chakali-speaking village. The linguistic data was gathered mainly in \isi{Ducie}, and sociolinguistic surveys were conducted in \isi{Katua}, \isi{Motigu}, \isi{Sogla}, \isi{Ducie}, and \isi{Gurumbele}.  I had several overnight stays in \isi{Motigu}, \isi{Gurumbele},  and Wa, and a few day trips to \isi{Katua}, \isi{Tiisa}, \isi{Tuosa}, and \isi{Sogla}. 

%a person correctly identified the ethnic group of my tutor by the way I was 
%speaking

Different elicitation techniques were used to gather linguistic and encyclopedic data, most of them influenced by language documentation methods \citep[see][]{Lukp09}. The most authentic and natural data comes from impressionistic and manual auditory transcription of audio recordings involving events such as transactions at the market, meetings with elders,  and interviews with commoners. In these cases wordlists were created out of the transcriptions. The least natural data are pieces of translation work  or exchanges of information with consultants of the type `how do you say X' or  `what is X' where X stands for an intended entity or proposition, using English or Chakali as the medium of communication. Translations from English to Chakali and from Chakali to English  were performed through a  collaboration with my main consultants, namely: Daniel Kanganu Karija (male, 58 Y.O., \isi{Ducie}), Fuseini Mba Zien (male, 54 Y.O., \isi{Ducie}), Awie Bakuri Ahmed (male, 31 Y.O., \isi{Gurumbele}), and Afia Kala Tangu (female, 34 Y.O., \isi{Ducie}). Small-scale quantitative studies required at times as many as 30 different speakers, all of them from \isi{Ducie}. In such studies, the method of elicitation consisted of having a significant number of native speakers interpreting, identifying and expressing perceived stimuli, which provided me with a level of authenticity unattainable in (bilingual) elicitation of wordlists.  The degree of consensus within the responses was interpreted as signalling core, secondary, or `accidental' meaning. The same method was also useful in practical lexicography sessions when the discovery procedure involved  taxonomies unknown to me. The domains of animals and plants required the identification of species and their associated pronunciation. A problem arises when the visual access to some species  is practically impossible, e.g.   wild animals or seasonal plants. While working on  the lexical database, many species were identified using illustrations. One known disadvantage with this approach to lexicon and grammar discovery is that standard stimuli  face the problem  of cross-cultural applicability.  In the context of  northern Ghana, unfamiliar items or scenes depicted cause disagreement in the overall description, if not confusion.  Another obstacle is that pictures and illustrations may lack elementary features, such as texture, odour, size, etc., which are crucial for the identification of a species.  For instance, arriving at a consensus when identifying  species of snake has proved difficult since  only  illustrations and pictures found in \citet{Cans61, Trap06} were used. However, in the research context,  I believe the most satisfactory data collection strategies were used. Needless to say, every piece of Chakali data in this book comes from my own transcription of speech.

 
\chapter{User's guide}
\label{sec:cont-descr}

The book is divided into four parts: a general introduction, a Chakali-English dictionary, an English-Chakali reversal index, 
and a part containing grammar outlines. At a macrostructure level, the dictionary is followed by the reversal index. They both contain information extracted  from a  lexical database which I started  collecting  in 2007 using the software {\it Field Linguist’s Toolbox}. The data was imported in {\it FieldWorks Language Explorer}  (FLEx) in  2012.  The entries appearing in the dictionary are made out of  only a selection of  entries and lexicographic fields/values available in the lexical database.

The passage from unwritten language to written language has the inevitable consequence of favouring a dialect. A literate native speaker of Chakali could easily identify from the entries that \isi{Ducie} was the community where the majority of the data was collected.  Corresponding expressions from other varieties of Chakali are present, when they exist,  but more work is definitely needed.   Addressing the issue of convention and standardisation will require a  group of devoted contributors from distinct communities. There is no reason to treat the decisions taken in this book, especially regarding the \isi{orthography},  as the standard. Despite the fact that the \isi{Ducie} lect is not a ``standard'', it is important to keep in mind that  a set of forms was produced by the lexicographical practice, the location of data collection, and the idiolects of the consultants.

\section{Chakali-English dictionary}
\label{sec:cli-eng-entry}

The Chakali-English dictionary consists of over 3500 Chakali \is{headword}headword entries (a.k.a. \is{lemma}lemmas). The transcription employs an alphabetic system motivated by the phonological   description presented in Part \ref{part:gram-part}.  It uses a Latin alphabet supplemented with symbols from the International Phonetic Alphabet (IPA), so the spelling-sound correspondence is direct.  A full list of orthography symbols used in the dictionary and some guidance to their pronunciations are displayed in Table \ref{tab:orth-symb}.


\begin{table}
  \caption{Dictionary orthography and other symbols\label{tab:orth-symb}}
\small
 \begin{tabular}{llll}
p & voiceless bilabial plosive & w&labio-velar approximant\\
b & voice bilabial plosive &j & palatal approximant \\
t & voiceless alveolar plosive &r & alveolar trill/flap\\
d &  voiced alveolar plosive&o& close-mid back rounded\\
k & voiceless velar plosive  &ɔ& open-mid back rounded\\
g & voiced velar plosive &e& close-mid front unrounded\\
ʔ & glottal stop &ɛ& open mid front unrounded\\
kp & voiceless labio-velar plosive &u& close back rounded\\
gb & voiced labio-velar plosive&ʊ& near close near back rounded\\
f & voiceless labio-dental fricative &i& close front unrounded\\
v & voiced labio-dental fricative  &ɪ& near close near front unrounded\\
s & voiceless alveolar fricative &a& open front unrounded\\
z & voiced alveolar fricative &ə& mid central\\
ɣ & voiced velar fricative&$[$ $]$ & phonetic representation\\
h & voiceless glottal fricative & ː & emphasis over or long segment\\
tʃ & voiceless postalveolar affricate & V̆ & extra short vowel\\
dʒ & voiced postalveolar affricate&C̩& syllabic consonant\\
m & bilabial nasal &  Ṽ & nasalized   vowel\\
n & alveolar nasal& V̀ & low tone \\
ɲ & palatal nasal &  V̄ & mid tone \\
ŋ & velar nasal& V́ & high tone \\
ŋm & velar-labial nasal&V̏  & extra-low tone \\
l & alveolar lateral approximant  & &\\
\end{tabular}

 \end{table}
 
%capitalization convention 

\newpage 
For users accustomed to the literacy work of GILLBT\footnote{Reference is made to the literacy work on \ili{Vagla},  \ili{Tampulma}, and \ili{Pasaale}  of Marjorie Crouch, Patricia Herbert, Noah Ampen, Kofi Mensah, Mike Toupin, Vicky Toupin, Ian Gray,  and Claire Gray.} the correspondences in Table \ref{tab:corr-our} identify the differences between the transcriptions: the one adopted in this book appears to the right side of the arrows. 

\begin{table} 
\caption[]{Correspondences of orthographies\label{tab:corr-our}}
 \begin{center}
\framebox[3in]{
\begin{tabular}{lllllll}
% ng &$\leftarrow$& ŋ  & \hspace*{5ex} y&$\leftarrow$& j\\
ny &$\leftarrow$& ɲ  &\hspace*{5ex} ng &$\leftarrow$& ŋ\\
ch  &$\leftarrow$& tʃ  &\hspace*{5ex}  i&$\leftarrow$&ɪ, i\\
j &$\leftarrow$&dʒ  &\hspace*{5ex}  u&$\leftarrow$& ʊ, u \\
y &$\leftarrow$&j  &\hspace*{5ex} Vh&$\leftarrow$&  Ṽ \\ 
\end{tabular}  
}
\end{center}
\end{table}



The \is{headword} headwords  are structured alphabetically  although an arbitrary decision was taken to place the letter ``{dʒ}'' after ``{d}'', ``{gb}'' after ``{g}'', ``{kp}'' after ``{k}'',  ``{ɲ}'',  ``{ŋm}'', and ``{ŋ}'',  successively after ``{n}'',  and ``{tʃ}'' after ``{t}''.  All headwords are equal and appear at the left side of the column. Four representative entries of the Chakali-English dictionary are presented in Table \ref{tab:illu-dict}.\footnote{The circled numbers are there for reference purposes only.}

\begin{table}
\caption[]{Illustrations of dictionary entries\label{tab:illu-dict}}
 \begin{center}
\framebox[4in]{
\begin{tabular}{p{10cm}}

 \Circled{1}{\bf  fi} \Circled{2}[fí]  \Circled{3}{\itshape num.}
\Circled{5}ten \\[0.5ex]

\Circled{1}{\bf   bʊzaal}  \Circled{2} [bʊ́záàl]  \Circled{3}{\itshape n.} \Circled{8}cf: bɪɪzimii.   \Circled{5} Stone partridge, type of bird  \Circled{9}{\it (Ptilopachus petrosus)}   \Circledd{11} pl: bʊzaalɛɛ. \\[0.5ex]

\Circled{1}{\bf suoŋbii} \Circled{2}[sùómbíí] \Circledd{10}{\itshape lit.} shea.nut-seed
 \Circled{3}{\itshape n.} kidney \Circledd{11} pl. {\sls suoŋbie}.\\[0.5ex]

\Circled{1}{\bf   kpa}  \Circled{2}[kpà] 
\Circled{3}{\itshape v.} \Circled{8}cf: {paa}; {jʊʊ}$_{1}$.  \Circled{4}\textbf{1.} \Circled{5} take 
\Circled{6}{\sls kpá à pár tɪ̄ɛ̄ŋ.} \Circled{7}Give me the hoe.  
\Circled{4}\textbf{2} \Circled{5} to marry a woman \Circled{6}  {\sls ʊ̀ kpáʊ́ rà.}  
\Circled{7}He married her.



\end{tabular}
}
\end{center}
\end{table}

The convention is for an entry to start with a  headword (\Circled{1}), which is immediately followed by its phonetic representation (\Circled{2}). This representation adds tones and other information on the pronunciation. Words which do not bear tones in the phonetic representation field are considered as either toneless or  unresolved. The grammatical category (\Circled{3}) provides  the word class of the headword. A \isi{headword} may be accompanied by a literal translation ({\it lit}) \Circledd{10} to isolate the English meaning of each stem.  In the literal translation field, a hyphen (-) separates stems and a full stop (.) joins spacing between English words. A \isi{plural} form  is provided for the majority of the nouns \Circledd{11}. Cross references  (\Circled{8}) appear after  the phonetic form and the part-of-speech. Variations to which different spellings or forms have to be assigned are placed after the phonetic form. It offers some lectal and generational variations in the following way: {\it var.} introduces a standard's \isi{variant} and  {\it var. of} sends the reader back to the \isi{headword} treated as standard. 


The meaning  is represented in the following way: if the headword has only one sense,  the part of speech immediately precedes the English definition (\Circled{5}).  If the headword has more than one sense,  a boldface number (\Circled{4}) enumerates the different senses. When Chakali is translated into English using many expressions, these are separated by a comma. If a word typically collocates with a semantic property or properties, this is explicitly stated using examples in the English translation.  For instance, the definition of the verb {\sls zɪna} is given  as   `to drive, ride, or sit on e.g. bicycle, motorcycle, horse'.  An example of usage (\Circled{6}) precedes its English free translation (\Circled{7}). Only verbal and functional words  are backed up by example sentences.  If literal and/or not easily translatable, the free translation contains further clarifications.


\subsection{Capitalization}
\label{sec:capitalization}

Despite the existence of case \is{variant}variants in the \isi{orthography}, a decision was made in this dictionary to present the Chakali data in unicase, i.e. without \is{capitalization}capitalization rules.  In the current state, there are many practical questions that need answers and an orthography development would need to consider issues beyond linguistic ones.

\subsection{Prosody}
\label{sec:INT-prosody}

The example sentences are all marked with diacritics which attempt to capture the intonation as I perceived it during the transcription work. The convention for marking tone is:  high (    ́   ), low (    ̀  ), mid   (  ̄  ),  and super-low ( ̏  ).   An overview of tone and intonation is provided in Section \ref{sec:tone-intonation}. At this stage, the transcription and description of tone will require an analysis of considerable sophistication, something which deserves a separate study. There are several issues linked to doing the transcription by ear and lacking a more elaborated convention. For instance, due to the general F0 downtrends over the course of an utterance,  the prosody on single words is easier to represent with this simple convention as opposed  to longer expressions. Further, as they are  not always perceived and/or transcribed, there is inconsistency in the tonal marking of consonants in syllable final positions.

\subsection{Scientific name}
\label{sec:INT-sci-name}

To  add the referential stability needed for future comparison between traditional  and scientific  taxonomies, scientific names appear in italics  (\Circled{9}).   References to scientific names of  plants and trees were taken from \citet{hawt06},  scientific names of  snakes from \citet{Cans61}  and \citet{Trap06}, and scientific names of  birds from \citet{borr02}.

\subsection{Grammatical category}
\label{sec:INT-other-lex-field}

The grammatical categories (a.k.a  word classes or parts of speech)  used in the dictionary are elaborated in Part \ref{part:gram-part}.  They are distinguished using distributional and inflectional criteria. 
% Table \ref{tab:dict-abb} on page  \pageref{tab:dict-abb} offers a  list of the sections in the grammar outline where each grammatical  category can be found, together with their dictionary abbreviation.


\subsection{Loans and their etymology}
\label{sec:INT-loan-ety}


Loan words are given a source, and when necessary, the source's pronunciation and 
gloss are provided. If a gloss does not appear, it is assumed that the meanings in Chakali and in the source language are practically the same. Some origins are well-established, others are intuitive. The 
word  {\it ultimately} (abbreviated as {\it ultm.}) may be placed prior to the source language
to mean that the loan word might not have been borrowed directly from the 
speakers of the language with which the word is associated. For example, it is 
most likely that all English words entered Chakali  through contact with 
speakers of other Ghanaian languages.  Section \ref{sec:GRM-borr-noun} offers an 
overview of languages from which Chakali may have borrowed.  References to 
etymologies are mainly taken from \citet{newm07hausa}, \citet{daku07},   \citet{bald08},
\citet{daku09ga},  \cite{vagl80},  \citet{Dume11}, and \citet{vydr15}. Besides language names as sources,  expressions that are known to be found in other languages without necessarily being identifiable to one particular source are given various source values. Such items cross ethnic and/or geographical boundaries although they may not be known in other parts of the country.  For instance, {\it Ghanaianism} (Ghsm) refers to an expression known to be found in most Ghanaian languages,  and {\it Gur} refers to an expression that has been reconstructed for most \ili{Gur} languages. 


\section{English-Chakali reversal index}
\label{sec:eng-cli-entry}

The English-Chakali reversal index is a list of  alphabetically organized 
English \is{headword}headwords (\Circled{1}). As shown in Table 
\ref{tab:illu-index},  the headword may be 
associated with  more than one  Chakali 
gloss entry (\Circled{5}).  



\begin{table}
\caption[]{Illustration of an English-Chakali reversal 
index entry\label{tab:illu-index}}
\begin{center} 
\framebox[3in]{

\begin{tabular}{lll}
 \Circled{1}grasshopper (type of) & \Circled{3}{\itshape n.} & 
\Circled{5}{hɔ̃ʊ̃} \\
 & \Circled{3}{\itshape n.} & \Circled{5}{tʃɛlɪntʃɪɛ} \\
 & \Circled{3}{\itshape n.} & \Circled{5}{kɔkɔlɪkɔ} \\
\end{tabular} 
}
\end{center} 
\end{table}


English  \is{headword}headwords are reduced to minimal terms in order to have the index easily searchable. Several English expressions can be associated with one Chakali word: for instance, all Chakali tree names get {\it tree (type of)} but only some have known English expressions associated to  them, e.g. {\it Shea tree}. Each Chakali word is preceded by its word class (\Circled{3}).  Since users are expected to  look for English keywords, not all dictionary entries are found in the reversal index.




\section{Grammatical outlines}
\label{sec:intro-outline}

Part \ref{part:gram-part} is divided into two sections.
The first section presents a brief outline of the phonology. It 
is principally based on phonetic representations available in the lexical database.   
The phoneme inventory,  syllable structures, and minimal pairs are identified. In 
addition, phonotactics and suprasegmentals are briefly discussed. The software 
{\it Dekereke} was used to investigate phonotactic generalizations and search 
for specific features and environments.\footnote{Thanks to its creator Rod 
Casali for his continual help.}  Based on the transcriptions of various 
narrative types and controlled elicitation (Section \ref{sec:method}), the second section, entitled `Gramm outline'   offers an overview of  the essentials of word and sentence 
formations in the language, as well as topics of linguistic usages of cultural 
relevance. The glossing tags in the abbreviations list (page \pageref{sec-ABB}) are for the most part 
equivalent to the conventions designed in  \citet{Comr08b} and \citet{hasp14}.  
As a rule,   a three-line morpheme-by-morpheme  glossing for textual data is 
provided, but  four lines may exceptionally appear.  The first line is a 
representation of the object language, the second line consists of   tags 
representing  rough approximations  of the morpheme in the object   language 
(e.g. function, meaning,  and part-of-speech), whereas the third line is a free  
translation capturing the general meaning  conveyed in the object language's 
line. An additional line can appear when details are not evident in the gloss, 
or when another level of analysis is intended.  Small capital letters in 
the  free translation may be used to represent a focused constituent.  The  non-overt 
expression of a feature is enclosed within round brackets. An interlinearized example  may  be accompanied by a 
reference  to a particular corpus text or  a situation in which the utterance 
was collected. Most  examples are taken from elicitation data. Corpus sentences are mainly selected in three texts: the Python story (PY), the Clever boy story (CB), and the Law breaker story (LB). The three stories consist of oral third person traditional folk tales. 
 The first was performed by Kotia Nwabipe  and the other two by Daniel Kanganu Karija.  They were recorded and transcribed in \isi{Ducie} in 2007. The latter two are contained in the first appendix in \citet[471-500]{brin11}.The corpus texts are not provided in this edition. 



\section{Abbreviations}
\label{LEX:abbrev}


 Two alphabetically ordered lists of abbreviations are provided: a list to be used with Part \ref{part:part1} and Part  \ref{part:index} is given on page \pageref{sec-AB} and a list to be used with Part \ref{part:gram-part} is given on page \pageref{sec-ABB}. The former  list gives alongside the abbreviations and their meaning the section or sections of the 
grammar that cover the related topic. 