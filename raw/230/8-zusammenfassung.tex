\chapter{Zusammenfassung und Ausblick}\label{kapitel:zusammenfassung}

Die vorliegende Arbeit hat die \isi{Konstruktionalisierung} von [\object{dër}\,+\,N] aus drei Blickwinkeln betrachtet. Erstens wurde die semantische Ausbleichung des ursprünglichen Demonstrativartikels \is{Demonstrativartikel} beleuchtet. Zweitens wurde gezeigt, mit welchen Substantivklassen \is{Substantiv} der N-Slot besetzt wird. Drittens wurden Strukturmerkmale \is{Nominalsyntax} der gesamten Nominalphrase \is{Nominalphrase (NP)} offengelegt. 
Diese Herangehensweise trägt der Tatsache Rechnung, dass die Entwicklung des Definitartikels \is{Definitartikel} nicht nur auf Morphemebene \is{Morphem} abläuft (aus einem demonstrativen adnominalen Element wird ein auf \isi{Definitheit} reduziertes Artikelwort), sondern die gesamte Nominalphrase \is{Nominalphrase (NP)} im Althochdeutschen betrifft. Dieser gesamtheitliche Blick steht im Einklang mit der \isi{Konstruktionsgrammatik}, welche Sprache als dynamisches Netzwerk von miteinander assoziierten Konstruktionen \is{Konstruktikon} betrachtet. Sie nimmt eine kognitive Perspektive auf Grammatik (und Sprache im Allgemeinen) ein, weil davon ausgegangen wird, dass kognitive Grundprinzipien (vor allem Ka"-te"-gori"-sier"-ungs- und Ab"-strak"-tions"-prozesse) das Sprachsystem formen. In der vorliegenden Arbeit\linebreak konnte gezeigt werden, wie der kognitiv-linguistische Faktor \isi{Belebtheit}, d.h. die außersprachliche Einordnung von Entitäten in \textsc{menschlich, belebt, unbelebt} die Setzung von \object{dër} beeinflusst. Auch \is{Entrenchment} Entrenchmentprozesse, welche die Fähigkeit zu abstrahieren und analogische \is{Analogie} Bezüge herzustellen voraussetzen, wurden erstmals mit der Entwicklung des Definitartikels \is{Definitartikel} in Verbindung gebracht.

Im Gegensatz zu bisherigen Studien zum \isi{Definitartikel} beruhen die Analysen auf einer breiten Datenbasis, die mithilfe korpuslinguistischer Methoden \is{Korpuslinguistik} untersucht wurde. Als Textgrundlage \is{Korpus} dienten die fünf größten ahd. Textdenkmäler, die über das \object{Referenzkorpus Altdeutsch} zugänglich sind: Isidor (um 790), Monseer Matthäus (um 810), Tatian (um 840), Otfrids Evangelienbuch (um 870) und Notkers Boethius (um 1025). 

Die Funktionsanalyse von \object{dër} hat gezeigt, dass die Entwicklung des Definitartikels \is{Definitartikel} schon im frühen Althochdeutschen weit fortgeschritten ist. Bereits im Isidor finden sich in 100 zufällig ausgewählten NPs mehr \object{dër}-Belege in \is{Semantische Definita} semantisch-definiten, d.h. situationsunabhängigen Kontexten, als in \is{Pragmatische Definita} pragmatisch-definiten, d.h. situationsabhängigen Kontexten. Gemessen an den später datierten Texten steigt diese \is{Expansion} Kontextexpansion: Während im Tatian knapp ein Drittel der se\-man\-tisch-definiten \is{Semantische Definita} Fälle mit einer \object{dër}-Phrase ausgedrückt werden (Unika \is{Unikum} und generische Belege \is{generisch} ausgenommen), sind es bei Otfrid schon die Hälfte. Auch die Ergebnisse zu den inhärent definiten Superlativkonstruktionen \is{Superlativ} sowie den Unika \is{Unikum} spiegeln diese funktionale Verschiebung. Bei Notker, dem jüngsten Text, dominiert in diesen Kontexten die \object{dër}-Setzung. Es konnte gezeigt werden, dass der anaphorische \is{anaphorisch} und der anamnestische \is{anamnestisch} Gebrauch als mögliche Brückenkontexte \is{Brückenkontext} und damit Startpunkte für die Entwicklung in Frage kommen. Der generische \is{generisch} Gebrauch ist früher möglich, als es die in der Forschung bisher vorgeschlagen Grammatikalisierungsskalen \is{Grammatikalisierungspfad} postulieren, weshalb der Hauptentwicklungspfad, der eine \isi{Expansion} von pragmatisch-definiten \is{Pragmatische Definita} Kontexten zu semantisch-definiten Gebrauchskontexten \is{Semantische Definita} vorsieht, um einen generischen \is{generisch} \herkur{Seitenpfad} erweitert wurde. Es bleibt zukünftigen Studien überlassen, die Faktoren offenzulegen, die für den variablen Gebrauch von \object{dër} bei generischen \is{generisch} Ausdrücken verantwortlich sind. Möglicherweise spielt die Art des generischen \is{generisch} Verweises eine Rolle (\object{kind-referring NP} mit und ohne \object{characterizing sentences}, vgl. Abschnitt \ref{sec:nicht-referentiell}). 

Ob ein \isi{Substantiv} mit \object{dër} kombiniert wird, ist von seinem Belebtheitsgrad \is{Belebtheit} abhängig. Grob zusammengefasst werden in den ahd. Texten insbesondere Menschen, aber auch Konkreta \is{Konkretum} eher determiniert als Abstrakta \is{Abstraktum} und \isi{Massennomen}. Die Unterschiede zwischen den Texten zeigen, dass \object{dër} mit zunehmender Obligatorisierung und semantischer Ausbleichung entlang der \isi{Belebtheitshierarchie} auf neue Substantivklassen expandiert. Während der Belebtheitsfaktor \is{Belebtheit} im Isidor nicht sichtbar wird, zeigen die Auswertungen zum Monseer Matthäus, dass menschliche und konkrete \is{Konkretum} Referenten einen größeren Anteil innerhalb der \object{dër}-Belege einnehmen als bei den Belegen ohne \object{dër}. Im Tatian stechen zwei Gruppen signifikant heraus: Zum einen Menschen, welche überzufällig häufig determiniert werden, und zum anderen \is{Abstraktum} Abstrakta, die überzufällig häufig undeterminiert bleiben. Betrachtet man nur die Hapax Legomena, so nehmen menschliche Referenten bei Otfrid eine ähnliche Sonderrolle ein, da sie stärker als alle anderen Substantivtypen zur \object{dër}-Setzung neigen. Diese Präferenz hängt damit zusammen, dass Menschen kognitiv auffällig und maximal handlungsfähig \is{Agentivität} sind. Sie sind besonders wichtig für den Diskurs, weswegen Sprecherinnen und Sprecher sie sprachlich hervorheben \herkur{wollen}. Mit seiner ursprünglich demonstrativen Funktion ist \object{dër} hierfür prädestiniert.  
Bei Notker hat die Belebtheitsanalyse \is{Belebtheit} gezeigt, dass die  \object{dër}-Präferenz für Menschen, Tiere und Konkreta \is{Konkretum} gleichermaßen hoch ist. Anders als in den älteren Texten ist die Abneigung der \is{Abstraktum} Abstrakta gegenüber der Determinierung jedoch nicht mehr so stark. Der Faktor \isi{Relevanz} ist auf unterschiedliche Weise sichtbar geworden. So sind es im Monseer Matthäus und im Tatian vor allem gesellschaftlich ranghohe und männliche Referenten, die regelmäßig determiniert werden, so dass hier \isi{Relevanz} mit \isi{Belebtheit} positiv korreliert. Bei Otfrid werden mithilfe von \object{dër} thematisch wichtige Referenten hervorgehoben, darunter auch viele \is{Konkretum} Konkreta. Im Isidor und auch bei Notker scheinen viele Abstrakta \is{Abstraktum} auch thematisch relevant zu sein, so dass dies das relativ hohe Vorkommen von \object{dër} erklären könnte. Hier müssten zukünftig noch weitere textuelle Tiefbohrungen erfolgen.

Die Anfertigung von \isi{Annotationsrichtlinien} sowie die doppelten \is{Annotation} Annotationen, welche über \herkur{Inter Annotator Agreements} evaluiert wurden, haben für ein hohes Maß an Transparenz und Objektivität \is{Operationalisierung} bei der \is{Belebtheit}  Belebtheitsannotation \is{Annotation} gesorgt. In zukünftigen Untersuchungen könnten mit einer ähnlichen methodischen Herangehensweise auch die semantischen Rollen \is{Semantische Rolle} untersucht werden. Die Ergebnisse der Stichprobenanalysen zum Isidor, Tatian und Otfrid deuten zwar darauf hin, dass \isi{Agentivität} die \object{dër}-Setzung begünstigt, allerdings sind agentive Referenten meist auch belebt. Um die Wechselwirkung zwischen \isi{Belebtheit} und semantischer Rolle \is{Semantische Rolle} offenzulegen, müsste die semantische Rolle noch feiner ausdifferenziert und dann systematisch auf die ahd. Texte übertragen werden. 

Die Ergebnisse zur Struktur \is{Nominalsyntax} der \is{Nominalphrase (NP)} Nominalphrase haben sichtbar gemacht, dass schon ab dem frühen Althochdeutschen in der Nominalphrase ein pränominaler \is{Wortstellung} Determiniererslot \is{Determiniererschema} angelegt ist. In allen Textdenkmälern wird der Großteil aller definiten Phrasen von determinierenden Elementen (neben \object{dër} vor allem \is{Possessivum} Possessivartikel, aber auch das \isi{Demonstrativum} \object{dëser} oder \is{Genitivattribut} Genitivattribute) begleitet. Die flektierbaren \is{Flexion} Einleiter sind darüber hinaus auch noch außerordentlich stellungsfest. Es ist wahrscheinlich, dass Sprecherinnen und Sprecher aus diesem empirischen Input ein \isi{Determiniererschema} ableiten, in dem \object{dër} aufgrund seiner hohen Gebrauchsfrequenz den Determiniererslot standardmäßig besetzt, was die Herausbildung \is{Definitartikel} der Definitartikelkonstruktion \is{Konstruktion} [\object{dër}\,+\,N] fördert. Zudem begünstigt das \isi{Schema}  [\object{dër}\,+\,Adjektiv\textsubscript{schwach}\,+\,N] als Resultat semantisch bedingter Kollokationen die Obligatorisierung von \object{dër}. Auch spezifische, hochfrequente Konstruktionen \is{Konstruktion} wie \object{dër heilant} sind für den Wandel förderlich. Sie treten primär in semantischen \is{Semantische Definita} Definitheitskontexten \is{Definitheitskontext} auf und können damit als erste Instanzen des Schemas \is{Schema} [Definitartikel + Nomen] analysiert werden. In dieser Funktion dienen sie als Vorbild für die analogische \is{Analogie} Ausbreitung der \isi{Konstruktion}. 

\hspace*{-1.82977pt}Alle Korpusdaten, Annotationsrichtlinien und R-Skripte wurden in 
\textcite{HZKYL4_2020} veröffentlicht, so dass das Vorgehen sowie die Ergebnisse der  vorliegenden Untersuchung transparent dokumentiert sind und zukünftige Studien an die Materialien anknüpfen können.
