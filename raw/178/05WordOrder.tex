% !TEX root = ../main.tex
\chapter{Word Order}\label{WordOrder}

%%% ニガの語順を調べる
%%% Strongly active topicsは文頭に現れても良い

%%% 代名詞にアクセントがないことで、日本語の語頭アクセント規則(LH or HL)を破っている。よって後ろに現れるほうが良い。

%%% 前置された名詞はゼロになりやすいが、そうでない名詞はずっと繰り返し現れる。

%%% 文頭に現れる要素のすべてにトピック・マーカーがついているわけではない。トピック・マーカーはactivation statusに敏感だが、語順はidentityの方に敏感。

%%% Vallduv\UTF{00ED} (1994) のLink & Tailとの並行性を指摘。

%%% Gundel (1988)
%(52) Given before new principle
%     State what is given before what is new in relation to it.
%
%(53) First things first principle
%     Provide the most important information first.

%%% Rizziらのカートグラフィーに対する反論
%%% [TOPIC [FOCUS ...] ]
%%% 初期状態でこのような構造をしているという根拠はなく、トピックは文頭と文末、それぞれに現れる認知的動機付けが存在する

%%% フォーカスは動詞の直前 (Kuno, 1980)
%%% Tanaka (2005) 後置文の分析:選好応答は後置文

%%% 必ずしもDefiniteが文頭に来るわけではないというBackgroundのdefinitenessの節と矛盾

%%% 「へーテスト」にかける

%%----------------------------------------------------
%%----------------------------------------------------
\section{Introduction}\label{WO:Intro}

This chapter discusses how the \isi{information structure} of a clause affects
\isi{word order}.

Figure \ref{DEPositionAllF} shows the overall distribution of elements
in terms of their positions in a clause.
%i.e.,
%elements' positions from the beginning of the clause in question
%to the last element of the current clause boundary.
Elements are counted by phrases (so called \ci{bunsetsu}).
The y-axis indicates the frequency of the elements and
the x-axis indicates the position of the elements:
\code{1} means that the element in question appeared in the first position of the clause,
\code{2} means that it appeared in the second position,
and so on.
I used the values of \code{nth} originally included in CSJ.
The reason why the frequencies of \code{1} and \code{2} are lower than those of \code{3} is that
the linguistic categories that appear in the first or second position
are typically fillers, connectives, and adjectives and are excluded from the analysis.
The fact that the elements later than fifth in the clause appear very frequently
might be counterintuitive based on the ordinary idea of a clause, since a clause consists of a single predicate and at most three arguments and a few more adjuncts.
In spoken language, however,
there are many fillers, intensifiers such as \ci{hontooni} `really', and paraphrases,
which make the clause longer.
Since \code{nth} simply counts the position of a phrase in terms of linear position, and not structurally,
embedded clauses such as relative clauses are also included in the count.
\chd{I assume that it is worth including these intervening expressions
to analyze where a phrase can be interrupted by them and where it cannot.
In fact, the following results show that most non-\isi{anaphoric} elements appear immediately before the predicate, not interrupted by fillers, intensifiers, and so on (see \S \ref{WOPrePredEles}).}
Moreover,
CSJ has a unique definition of clause, which is not always the same as the intuitive definition;
\chd{rather, a clause in CSJ is closer to a single series of clause chains.}
For example,
some subordinate markers such as \ci{-to} `if' and \ci{-te} `and' do not work as clause boundaries.
These characteristics cause more elements to appear in later positions. See \citeA{maruyamaetal06} for a detailed definition of clause unit.

Figure \ref{DEPositionISF} and \ref{DEPositionPerF} show
element positions and their frequencies based on \isi{information status} and persistence, respectively.
The \isi{information status} ``\isi{anaphoric}'' in this study just means that the element in question has a co-referential \isi{antecedent} and ``non-\isi{anaphoric}'' means that it does not. ``Persistent'' means that the referent in question is also mentioned in the following \isi{discourse}, and ``non-persistent'' means that it is not. %
%%%ISSUE "means" out of alignment.
 See \S \ref{FW:Cor:AnaRel} for the details of the annotation procedure.
\chd{As was discussed in \S \ref{TopPar},
a linear mixed effects model was employed to predict \isi{information status} (\isi{anaphoric} vs.~non-\isi{anaphoric}).
As fixed effects, \isi{word order} (\code{nth} in CSJ, see \S \ref{WO:Intro} for the definition of this annotation), particles (\ci{toiuno-wa, wa, mo, ga, o, ni}), and intonation (phrasal vs.~\isi{clausal} IU, see \S \ref{Int:PIUCIU} for the definitions) were included, and
as a random effect, the speaker (\code{TalkID}) was included.
The model with the effects of \isi{word order}, particles, and intonation is significantly different from the models without each of them, which indicates that
\isi{word order}, particles, and intonation respectively contribute to the prediction of \isi{information status}.
The model with all three effects is significantly different from the model without the effect of \isi{word order} (likelihood ratio test, $p < 0.01$);
it is significantly different from the model without the effect of particles ($p < 0.001$) and the model without the effect of intonation ($p<0.05$)}

\chd{As was also discussed in \S \ref{TopPar},
a linear mixed effects model was also applied to predict persistence (persistent vs.~non-persistent).
Word order, particles, and intonation were included as fixed effects, and
the speaker (\code{TalkID}) was included as a random effect.
The model with the effects of \isi{word order} and particles is again significantly different from the models without either of them (likelihood ratio test, $p < 0.01$ for the model without \isi{word order}, $p < 0.001$ for the model without particles).
However, the model with the effect of intonation is not significantly different from the model without it ($p=0.423$)}.
The results are to be discussed in more detail in \S \ref{WOSentInitEles}.

\begin{figure}
%\begin{minipage}{0.5\textwidth}
%	\begin{center}
	\includegraphics[width=0.7\textwidth]{figure/DEPositionAll.pdf}
	\caption{Order of all elements}
	\label{DEPositionAllF}
%	\end{center}
%\end{minipage}
\end{figure}
\begin{figure}
%\begin{minipage}{0.5\textwidth}
%	\begin{center}
	\includegraphics[width=0.7\textwidth]{figure/DEPositionIS.pdf}
	\caption{Word order vs.\ infoStatus}
	\label{DEPositionISF}
%	\end{center}
%\end{minipage}
\end{figure}
\begin{figure}
%\begin{minipage}{0.5\textwidth}
%	\begin{center}
	\includegraphics[width=0.7\textwidth]{figure/DEPositionPer.pdf}
	\caption{Word order vs.\ persistence}
	\label{DEPositionPerF}
%	\end{center}
%\end{minipage}
\end{figure}

%%%ISSUE Figures and texts are too far away because only one figure appears on one page. Also, some figures are too big.

Figure \ref{DiffAllF} shows the overall distribution of elements in terms of their distance from the predicate;
%\code{0} indicates that the element in question is the predicate itself,
\code{1} indicates that the element appears right before the predicate,
\code{2} indicates that there is one element between the preceding element and the following predicate, and so on.
If the element appears right after the predicate,
the distance is counted as \code{-1}.
Since the number of post-predicate elements is too small to make any generalization, they are excluded from the figures.
Post-predicate elements will be discussed in comparison with dialogues in \S \ref{WOPostPreEles}.

Figures \ref{DiffInfoStatusF} and \ref{DiffPerF} show the distance between the element and the predicate
depending on \isi{information status} and persistence.
\chd{A linear mixed effects model of \isi{information status} (with the distance from the predicate and particles as fixed effects and the speaker as a random effect) indicates that
whereas the model with particles is significantly different from the model without them (likelihood ratio test, $p<0.001$),
the difference between the models with and without the distance from the predicate is only marginally significant ($p=0.060$).
This entails that the effect of particles significantly contributes to the model,
but the effect of distance is inconclusive (see \S \ref{WOPrePredEles} for discussion).
On the other hand, a linear mixed effects model of persistence (fixed and random effects are the same as above) shows that the effects of both particles and distance are significant to the model ($p<0.01$ for both the model without particle and that without the distance).}
The results are also to be discussed in further detail in \S \ref{WOPrePredEles}.


\begin{figure}
%\begin{minipage}{0.5\textwidth}
%	\begin{center}
	\includegraphics[width=0.7\textwidth]{figure/DiffAll.pdf}
	\caption{Distance from predicate}
	\label{DiffAllF}
%	\end{center}
%\end{minipage}
\end{figure}
\begin{figure}
%\begin{minipage}{0.5\textwidth}
%	\begin{center}
	\includegraphics[width=0.7\textwidth]{figure/DiffInfoStatus.pdf}
	\caption{Distance from predicate vs.\ Information status}
	\label{DiffInfoStatusF}
%	\end{center}
%\end{minipage}
\end{figure}
\begin{figure}
%\begin{minipage}{0.5\textwidth}
%	\begin{center}
	\includegraphics[width=0.7\textwidth]{figure/DiffPersistence.pdf}
	\caption{Distance from predicate vs.\ persistence}
	\label{DiffPerF}
%	\end{center}
%\end{minipage}
\end{figure}

%%----------------------------------------------------
%%----------------------------------------------------
\section{Clause-initial elements}\label{WOSentInitEles}

This section discusses clause-initial elements. In \ref{GivenAppearClause-Initially}, it will be argued that shared elements (i.e., unused, declining, \isi{inferable}, or evoked elements) tend to appear clause-initially, and in \S \ref{PersistentAppearClause-Initially}, that persistent elements also do.  
%It will be argued that shared elements (i.e., unused, declining, \isi{inferable}, or evoked elements) tend to appear clause-initially in \S \ref{GivenAppearClause-Initially}, and that persistent elements also tend to appear clause-initially in \S \ref{PersistentAppearClause-Initially}.
From these observations,
it will be generalized that topics tend to appear clause-initially,
as predicted from the previous literature.
Finally in \S \ref{TopicAppearClause-Initially},
I discuss the reasons why topics appear clause-initially.


%%----------------------------------------------------
\subsection{Shared elements tend to appear clause-initially}\label{GivenAppearClause-Initially}

%\begin{figure}
%\begin{minipage}{0.5\textwidth}
%	\begin{center}
%	\includegraphics[width=0.95\textwidth]{figure/WO.pdf}
%	\caption{Word order within a clause}
%	\label{WOF}
%	\end{center}
%\end{minipage}
%\begin{minipage}{0.5\textwidth}
%	\begin{center}
%	\includegraphics[width=0.95\textwidth]{figure/WO2.pdf}
%	\caption{Word order within a clause (\# of argument $>$ 1)}
%	\label{WO2F}
%	\end{center}
%\end{minipage}

%\begin{minipage}{0.5\textwidth}
%	\begin{center}
%	\includegraphics[width=0.95\textwidth]{figure/WOISGiven.pdf}
%	\caption{Word order vs.\ ASP (given, \# of argument $>$ 1)}
%	\label{WOISGivenF}
%	\end{center}
%\end{minipage}
%\begin{minipage}{0.5\textwidth}
%	\begin{center}
%	\includegraphics[width=0.95\textwidth]{figure/WOISNew.pdf}
%	\caption{Word order vs.\ ASP (new, \# of argument $>$ 1)}
%	\label{WOISNewF}
%	\end{center}
%\end{minipage}
%\end{figure}


Figure \ref{DEPositionISF} shows the frequency of elements and their positions
based on \isi{information status}.
Anaphoric elements appear most frequently in the third position.
On the other hand, non-\isi{anaphoric} elements appear most frequently in the fourth position,
but those in the fifth and sixth positions also appear frequently.
%This generalization still holds with another criterion.
%Figure \ref{WOF} shows the position of arguments within a clause (the arguments sharing the same predicate
%%considering only arguments of predicates
%potentially coded by \ci{ga} `\ab{nom}', \ci{o} `\ab{acc}', \ci{ni} `\ab{loc}'%
%	\footnote{
%	In Japanese, the same form \ci{ni} represent both locative and dative.
%	Here I simply included all \ci{ni}-coded elements as arguments.
%	}%
%).
%whereas Figure \ref{WOF} shows the overall orders of arguments.
%In Figure \ref{WOISF},
%clauses where more than one argument appears are considered.
%Figure \ref{WOISGivenF} shows the position of given arguments within a clause, considering only clauses where more than or equal to two arguments appear.
%Figure \ref{WOISNewF} shows the position of new arguments within a clause.
%As shown more clearly in these figures,
%given elements most frequently appear in the initial position,
%while new elements most frequently appear in the second position
%although those in the initial position are still frequent.
These distribution of elements in different information statuses appear to replicate the classic observation that
topics tend to appear earlier in a clause,
i.e., the from-old-to-new principle \cite{mathesius28,firbas64,danes70,kuno78,gundel88}. This principle is explicitly formulated in \Next.
%
\ex. \label{oldnewprinciple}\tl{From-old-to-new principle}:
 In languages in which \isi{word order} is relatively free,
 the unmarked \isi{word order} of constituents is old,
 predictable information first and new, unpredictable information last.
 \hfill{(\citeA[][54]{kuno78}, \citeA[][p.\ 326]{kuno04})}

This principle is motivated by the accumulative nature of utterance processing;
old (or given) elements work as anchors that relate
the previous \isi{utterance} and the following \isi{utterance}.
%This does not indicate that all topics appear sentence initially.
%First,
%some kind of \isi{topic} appear after the predicate rather than at the beginning as will be argued in \S \ref{WORdis},
%where
%the conditions of topics that appear sentence initially and post-predicatively will be discussed.
%Second,
%given foci can also appear at the beginning.
%As shown in \Next, for example,
This principle appears to be supported by examples such as the following.
%\ci{keeki-o} `cake-\ab{acc}' in line (ii),
%which refers to cake and ice cream in line (i),
%can be repeated after \ci{hee}
%and be treated as news.
%\ci{Keeki-o} `keeki-\ab{acc}' can also appear naturally before the predicate as shown in (ii)$^{\prime}$.
%%
%\ex. \a. \ag. koko-ni keeki-to aisu-ga oi-te at-ta-n-da-kedo \\
%		here-\ab{loc} cake-and ice.cream-\ab{nom} put-and exist-\ab{past}-\ab{nmlz}-\ab{cop}-though \\
%		`There were a piece of cake and ice cream,'
%	\bg. \EM{keeki-o} hanako-ga tabe-tyat-ta mitai \\
%		cake-\ab{acc} Hanako-\ab{nom} eat-\ab{pfv}-\ab{past} apparently \\
%		`but Hanako ate the cake.'
%	\bg.[(ii)$^{\prime}$] hanako-ga \EM{keeki-o} tabe-tyat-ta mitai \\
%		Hanako-\ab{nom} cake-\ab{acc} eat-\ab{pfv}-\ab{past} apparently \\
%		`but Hanako ate the cake.'
% \z.
% \b. hee, \{\EM{keeki-o} / Hanako-ga\}
%
In \Next,
\ci{sore} `it' in line c, referring back to \ci{kasi-pan} `sweetbread' in line b, precedes the A element \ci{oziityan} `grandfather'.
%
\largerpage
\ex. \label{PronIni1}
% \ag. sarani uti-no sohu-tteiuno-ga okasi-ga sukina mono-de \\
%   moreover \ab{1}\ab{pl}-\ab{gen} 
% \a. In addition, my grandfather likes sweets,
 \ag. uti-no sohu-tteiuno-ga okasi-ga sukina mono-de \\
 		out-\ab{gen} grandfather-\ci{toiuno}-\ci{ga} sweet-\ci{ga} favorite thing-\ab{cop} \\
		`Our grandfather likes sweets.'
 \bg. yoku pan-ya-san-de \EMi{kasi-pan}-o kat-te kuru-n-desu-ga \\
   often bread-store-\ab{hon}-\ab{loc} sweet-bread-\ci{o} buy-and come-\ab{nmlz}-\ab{cop}.\ab{plt}-though \\
   `(He) often buys sweet bread and comes home,'
 \bg. e n \EM{sore-o} i maa yoowa \EMi{oziityan-wa} issyookenmee \fbox{taberu}-n-desu-keredomo \\
   \ab{fl} \ab{frg} it-\ci{o} \ab{frg} \ab{fl} in.a.word grandfather-\ci{wa} trying.best eat-\ab{nmlz}-\ab{cop}.\ab{plt}-though \\
   `that, he tries his best to eat it, but'
 \b. he cannot eat all and
 \b. gives the leftovers to the dog...
  \hfill{(\code{S02M0198: 244.48-262.82})}
%S02M0198|00244484L|244.484271|252.337563|L|更に(0.433)うちの(0.569)祖父っていうのが(0.6)お菓子が好きなもの(0.204)で(0.39)よくパン屋さんで菓子パンを買ってくるんですが|/並列節ガ/|
%S02M0198|00253226L|253.226374|254.606404|L|買い過ぎてしまいまして|/テ節/|
%S02M0198|00255210L|255.209674|272.289576|L|(F え)(F ん)(0.648)それを(D い)(0.147)(F まー)要はお爺ちゃんは一生懸命食べるんですけれども(0.412)余って(0.268)それを犬に(0.194)あげ(0.315)てしまうので(1.403)(F その)残飯で太り(0.447)菓子パンで太り(0.696)味は覚えてグルメになるという最悪の(0.322)育ち方をしてしまいまして|/テ節/|
%

Note that \ci{sore} `it' in line c is not coded by \ci{wa} but by \ci{o}.
This shows that clause-initial shared elements are not necessarily coded by \isi{topic} markers,
although it is predicted that elements coded by \isi{topic} markers would be more likely to appear clause-initially than those coded by case markers (see the discussion in \S \ref{WO:ClauseInit:Ident:Topic}).

\largerpage
Similarly in \Next,
\ci{sore} `it' in line c refers back to \ci{buraunkan} `cathode ray tube'
and appears at the beginning of the clause,
preceding other elements.
%
\ex.\label{PronIni2}
 \ag. oo-gata-no-ne \\
   large-type-\ab{gen}-\ab{fp} \\
   `(It's) a larger type (of cathode ray tube).'
 \bg. yoku maa a hooru-toka-ni aru-yoona oo-gata-no ee {\EMi{buraunkan}}-nan-da-kedomo \\
   often \ab{fl} \ab{fl} hall-etc.-\ab{dat} exist-like large-tyle-\ab{gen} \ab{fl} cathode.ray.tube-\ab{nmlz}-\ab{cop}-though \\
   `(It's) a large type of cathode ray tube typically equipped in a large hall, and'
 \bg. \EM{sore-o}-ne koo \EMi{kotti-kara} \EMi{kotti-ni} \fbox{moti-ageru}-toiu-yoona \\
  that-\ci{o}-\ab{fp} this.way here-from here-from bring-rise-\ab{quot}-like \\
  `this (cathode ray tube), (people) brought it from here to there.'
  \b. some people were doing something like that.
   \hfill{(\code{S05M1236: 471.26-490.38})}

%S05M1236|00462313L|462.313166|477.177522|L|(F ま)ブラウン管て言ってもですね(0.24)(F あのー)小型のブラウン管ではなくて(0.732)(F えー)(0.728)プロジェクション(F えー)チューブっつって(0.377)大きい(0.107)大型のね(0.135)よく(F まー)(F あ)(0.387)ホールとかに(0.134)あるような大型の(1.01)(F えー)ブラウン管なんだけども|/並列節ケドモ/|
%S05M1236|00477852L|477.851579|490.382975|L|(D す)それをね(0.401)こう(0.348)こっちからこっちに持ち上げると(0.5)いうような(0.155)朝から(0.457)朝から(D い)(0.421)夕方までこう(0.383)こっちからこっちに移すと(0.621)いうような仕事をしてる(F うー)同期の人もいました|[文末]|

However,
this is not the whole story;
there are many counter-examples where non-\isi{anaphoric} elements precede \isi{anaphoric} ones. %
%%%ISSUE "where" out of alignment.
 Table \ref{GNT} shows the number of cases
where \isi{anaphoric} precedes non-\isi{anaphoric} and non-\isi{anaphoric} precedes \isi{anaphoric} within the same clause.
There are 102 cases where \isi{anaphoric} precedes non-\isi{anaphoric},
while there are 63 cases where non-\isi{anaphoric} precedes \isi{anaphoric}.
The cases where \isi{anaphoric} precedes non-\isi{anaphoric} only slightly outnumber the cases where non-\isi{anaphoric} precedes \isi{anaphoric}.
63 cases (39.4\%) is too large a number to believe that they are mere exceptions to the principle in \ref{oldnewprinciple}.

\begin{table}
\centering
	\caption{Order of anaphoric \& non-anaphoric elements}
\begin{tabular}{rr}
	\lsptoprule
	Anaphoric $\to$ Non-\isi{anaphoric} & Non-\isi{anaphoric} $\to$ Anaphoric \\
	\hline
	102 & 63 \\
	\lspbottomrule
\end{tabular}
	\label{GNT}
\end{table}

I do not claim that the principle in \ref{oldnewprinciple} is not correct,
but I do claim that the principle does not apply to all cases.
Anaphoric elements precede non-\isi{anaphoric} elements
if the \isi{anaphoric} elements are assumed to refer to the ``same'' entity which has already been mentioned.
In other words,
shared elements precede non-\isi{anaphoric} elements.
%I argue that identifiability is also one of the features of topichood.
%What I mean by ``topical'' is that an element has features that correlate with \isi{topic} according to (\ref{ISFeatures}) in Chapter \ref{Framework}.
%In this case,
%being definite is a correlating feature with \isi{topic} and definite elements are called ``topical''.
For example, in \Next,
\ci{mizu} `water' is repeatedly mentioned in the \isi{utterance},
but it is never produced clause-initially.
I argue that this is because
\ci{mizu} `water' in \Next[b] and later is not assumed to refer to the ``same'' entity already mentioned in the previous \isi{discourse}.
\largerpage
%
\ex.\label{WO:ClauseInit:Given:mizu}
 \ag. desukara daitai iti-niti-ni ni-rittoru-no \EM{mizu-o} \ul{tot}-te kudasai-to iw-are-te \\
 so approximately one-day-for two-liter-\ab{gen} water-\ci{o} drink-and please-\ab{quot} tell-\ab{pass}-and \\
 `So we were told to drink two liters of water per day,'
 \bg. syokuzi-no toki-wa kanarazu magukappu-de ni-hai-bun-no \EM{mizu-o} \ul{nomi}-masu-si \\
 	meal-\ab{gen} time-\ci{wa} surely mug-with two-cup-amount-\ab{gen} water-\ci{o} drink-\ab{plt}-and \\
	`whenever we have a meal, we drink two cups of water,'
 \bg. totyuu totyuu-de-mo kanarazu \EM{mizu-o} ho anoo \ul{nomi}-taku-naku-temo \\
 		on.the.way on.the.way-\ab{loc}-also surely water-\ci{o} \ab{frg} \ab{fl} drink-want-\ab{neg}-even.if \\
		`also on the way, even if we didn't want to drink water,'
 \bg. nom-as-areru-to iu kanzi-de \\
 	drink-\ab{caus}-\ab{pass}-\ab{quot} say feeling-\ab{cop} \\
	`we were forced to drink (water).'
 \b. they think that drinking water is very important.
  \hfill{(\code{S01F0151: 339.78-366.29})}
%S01F0151|00339776L|339.775512|341.443878|L|でこのティータイムなんですけれども|/並列節ケレドモ/|
%S01F0151|00341699L|341.698978|349.557717|L|この(0.43)標高の高いところでは(0.141)高山病という非常に危険な(0.329)可能性があるので(0.243)(F えー)水が非常に重要になります|[文末]|
%S01F0151|00349955L|349.954875|366.290044|L|ですから大体一日に二リットルの水を取ってくださいと言われて(0.316)食事の時は必ずマグカップで二杯分の水を(0.145)飲みますし(0.285)途中途中でも必ず水を(0.113)(D ほ)(0.114)(F あのー)(0.432)飲みたくなくても飲まされるという感じで(0.403)水分補給(0.297)を(0.632)重視しておりました|[文末]|


In the same way,
\ci{tenkan} `epilepsy' appears many times in \Next,
but it never appears clause-initially.
%
\ex.\label{WO:ClauseInit:Given:tenkan}
 \ag. ato ik-kai \EM{tenkan} \EMi{okosi}-tara sinu-tte it-te-ta-n-desu-kedo \\
      moreover one-time.\ab{cl} epilepsy cause-\ab{cond} die-\ab{quot} say-\ab{past}-\ab{nmlz}-\ab{cop}.\ab{plt}-though \\
      `(The doctor) said that, if (my dog) gets an epilepsy seizure once more, (the dog) would die, but...'
 \bg. mata so sookoo si-teru uti-ni \EM{tenkan} \EMi{okosi}-masi-te \\
      again \ab{frg} meanwhile do-\ab{prog} while-\ab{dat} epilepsy cause-\ab{plt}-and \\
      `meanwhile, (the dog) has an epilepsy seizure, and...'
 \b. The dog recovered this time, but had an epilepsy seizure several times and finally died. (130.8 sec omitted.)
 \bg. sono boku-ga dekakeru toki-ni moo noki-sita-de \EM{tenkan} \EMi{okosi}-te \\
      \ab{fl} \ab{1}\ab{sg}-\ci{ga} go.out when-\ab{dat} already eave-under-\ab{loc} epilepsy cause-and \\
      `When I left (home), (the dog) had already had an epilepsy seizure, and...'
 \bg. tabun sin-dei-ta-n-da-roo-to \\
      probably die-\ab{prog}-\ab{past}-\ab{nmlz}-\ab{cop}-\ab{infr}-\ab{quot}\\
      `probably died...'
 \bg. ta noki-sita-de \EM{tenkan} \EMi{okosi}-ta-ga tame-ni \\
      \ab{frg} eave-under-\ab{loc} epilepsy cause-\ab{past}-\ab{gen} reason-\ab{dat} \\
      `just because (the dog) has an epilepsy seizure under the eaves...'
 \b. the dog could not get out of there and died, we [the family members] were talking like that.
 \src{S02M0198: 558.7-712.8}
%  \ex [飼い犬が]あと一回\EM{てんかん}起こしたら死ぬって言ってたんですけど
%  \ex またそそうこうしてるうちに\EM{てんかん}起こしまして
%  \ex ... (130.8秒省略。このときは復活したがその後数回てんかんを繰り返し、死んでしまう。)
%  \ex \label{ExTenkan3} その僕が出掛ける時にもう軒下で\EM{てんかん}起こして
%  \ex 多分死んでいたんだろうと
%  \ex \label{ExTenkan4} (た)軒下で\EM{てんかん}起こしたが為に
%  \ex その要は出られなくて
%  \ex 引っ掛かっちゃって
%  \ex 出られなくてそのまま死んじゃったんじゃないのっていう話をしたんですけど\\
%\src{S02M0198: 558.7-712.8}


Whether the speaker refers to the shared entity mentioned previously
depends on the speaker's subjective judgement rather than on objective reasoning.
In \Next, for example,
the \isi{anaphoric element} \ci{kuruma} `car' in line c does not appear clause-initially for the same reason as in \LLast and \Last.
However, \ci{kuruma} `car' in line b and d are clearly the same entity.
%
\ex.\label{WO:ClauseInit:Given:kuruma}
 \ag. kirauea-kazan-mo mappu-o kai-masi-te \\
 		Kilauea-volcano-also map-\ci{o} buy-\ab{plt}-and \\
		`Also for Kilauea, (we) bought a map and'
 \bg. de zibun-tati-de ma rentakaa \EM{kuruma-o} \ul{tobasi}-te e iki-masi-ta \\
 		then self-\ab{pl}-by \ab{fl} rent-a-car car-\ci{o} drive-and \ab{fl} go-\ab{plt}-\ab{past} \\
		`(we) drove there by rent-a-car by ourselves.'
 \b.[] (83.52 sec talking about the mountain.)
 \bg. de anoo jibun-no koko koko-de tyotto tome-te miyoo-to omot-ta toko-ni koo \EM{kuruma-o} \ul{tome}-te \\
 		and \ab{fl} self-\ab{gen} \ab{frg} here-\ab{loc} a.bit stop-and try-\ab{quot} think-\ab{past} place-\ab{dat} this.way car-\ci{o} stop-and \\
	 	`At the place (we) wanted to stop, (we) stopped the car,'
 \b. you can take pictures and so on.
 \hfill{(\code{S00F0014: 843.23-940.34})}
%
%S00F0014|00843233L|843.23315|850.162512|L|(F あのー)キラウエア火山もマップを買いましてで自分達で(0.363)(F ま)レンタカー(0.102)車を(0.376)飛ばして(0.367)(F え)行きました|[文末]|
% ...
%S00F0014|00933685L|933.684509|940.33856|L|で(0.299)(F あのー)自分のここここでちょっと止めてみようと思ったとこにこう車を止めて(0.398)(F まー)その写真を撮ったり|<タリ節>|大きい切れ目−係り先なし

I argue that, in this case, the speaker does not care about the identity of the car.
Rather, she focuses on talking about her trip to Kirauea;
the car she was in is not important for this speech.
As will be discussed in \S \ref{PersistentAppearClause-Initially}, the
importance as well as the identity of the entity contributes to \isi{word order} in spoken Japanese.
Important (i.e., persistent) elements appear clause-initially.

Interestingly,
these elements which are repeatedly mentioned but never appear clause-initially are not referred to by zero or overt pronouns.
It is especially difficult to zero-pronominalize \ci{tenkan} `epilepsy' in \LLast[b-f] and \ci{kuruma} `car' in \Last[d].%
 \footnote{
 It is difficult to apply this test in \ref{WO:ClauseInit:Given:mizu} because \ci{mizu} `water' accompanies numeral modifiers such as 
 `of two liters' and `two cups of'.
 }
Zero pronouns are considered to be the most accessible topics \cite[17]{givon83}.
To zero-pronominalize,
the speaker needs to provide signals to let the \isi{hearer} know which is the \isi{topic}, as will be discussed in \ref{TopicAppearClause-Initially}.

%%
%\ex.
% \ag. koko-ni kuri-gohan-to yaki-zakana-ga oi-te at-ta-kara minna-de tabe-yoo-to omot-ta-n-da-kedo \\
% 		here-\ab{loc} chestnut-rice-and baked-fish-\ab{nom} put-and exist-because all-by eat-will-\ab{quot} think-\ab{past}-\ab{nmlz}-\ab{cop}-though \\
%		`Here there had been chestnut rice and baked fish and I wanted to eat them with everybody, but'
% \bg. \EM{gohan-\{o/wa/{\O}\}} hanako-ga tabe-tyat-ta-n-da-tte \\
% 		rice-\{\ab{acc}/\ci{wa}/{\O}\} Hanako-\ab{nom} eat-\ab{pfv}-\ab{past}-\ab{nmlz}-\ab{cop}-\ab{rep} \\
%		`Hanako ate the rice (and the rice is gone).'
% \bg.[b$^{\prime}$.] hanako-ga \EM{gohan-\{?o/??wa/{\O}\}} tabe-tyat-ta-n-da-tte \\
% 		Hanako-\ab{nom} rice-\{\ab{acc}/\ci{wa}/{\O}\} eat-\ab{pfv}-\ab{past}-\ab{nmlz}-\ab{cop}-\ab{rep} \\
%		`Hanako had eaten meal (so we cannot eat together).'
%%%% イントネーションも重要。b'で「ごはん」に強勢があると、indefiniteとしか解釈されない。
%
%Although the judgement is subtle,
%in \Last[b], where \ci{gohan} `rice' precedes the agent `Hanako',
%\ci{goahn} `rice' inclines to be interpreted as \ci{kuri-gohan} `chestnut rice' which has appeared in the previous line a.
%On the other hand,
%in \Last[b$^{\prime}$], where \ci{gohan} `rice' is preceded by the agent `Hanako',
%\ci{gohan} `rice' inclines to be interpreted as meal in general.%
%	\footnote{
%	In Japanese, \ci{gohan} `rice' is ambiguous between rice and meal in general.
%	}
%Moreover,
%\ci{o}-, \ci{wa}- and zero-codings are possible in \Last[b],
%whereas only zero-coding is natural in \Last[b$^{\prime}$].
%In \Last, \ci{o}-coding is possible because it is separated from the predicate.
%In spoken Japanese,
%P elements tend to be overtly coded if they are separated from the predicate \cite{fry01}.
%\ci{Wa}-coding is possible because \ci{gohan} `rice' appeared in the previous \isi{discourse}.
%I believe the \isi{zero particle} in this case is {\O$_{t}$},
%which is felicitous regardless of the \isi{activation status} of the referent of \ci{gohan} `rice'.
%The \isi{information structure} of \Last[a] can be represented as \Next.
%%
%\ex.
%% \ag. [gohan-o hanako-ga tabe-tyat-ta]$_{F}$ \\
%%			rice-\ab{acc} Hanako-\ab{nom} eat-\ab{pfv}-\ab{past} \\
%	\ag.  [gohan-\{o/wa/{\O}\}]$_{T}$ [hanako-ga tabe-tyat-ta]$_{F}$ \\
%			rice-\{\ab{acc}/\ci{wa}/{\O}\} Hanako-\ab{nom} eat-\ab{pfv}-\ab{past} \\
%
%I argue that \ci{o}-coded \ci{gohan} `rice' in the initial position is still \isi{topic} to some extent,
%although it is not coded by the \isi{topic} marker \ci{wa};
%P in the initial position, preceding focus A,
%only refers to elements which has been activated in the \isi{hearer}'s mind.
%\LLast[b] cannot receive the interpretation \LLast[b$^{\prime}$].
%If this position is for focus, there should be no such constraint;
%focus is by default non-\isi{anaphoric} and has not been activated in the \isi{hearer}'s mind.
%Moreover, most given Ps are still coded by \ci{o} `\ab{acc}' as will be discussed below.
%
%In \LLast[b$^{\prime}$], on the other hand,
%\ci{gohan} `rice' cannot be coded by \ci{o} or \ci{wa}.
%Assuming that the \isi{information structure} of \Last[b] is \Next,
%\ci{gohan} is felicitously coded by \ci{o} `\ab{acc}' because
%\isi{non-contrastive focus} P elements cannot felicitously coded by \ci{o}.
%\exg. [hanako-ga gohan-{\O} tabe-tyat-ta]$_{F}$ \\
%		Hanako-\ab{nom} rice-{\O} eat-\ab{pfv}-\ab{past} \\
%
%I claim \ci{wa}-coding is impossible because
%of the incompatible characteristics of \ci{wa}-coding and the element's position;
%\ci{wa} codes activated element at issue and the element between the focus agent and the focus predicate is also interpreted as focus.
%Since focus elements are tend to be \isi{indefinite} and non-\isi{anaphoric},
%\ci{gohan} `rice' in \LLast[b$^{\prime}$] inclines to be interpreted as \isi{indefinite} and new.
%Still, I claim that \ci{gohan-o} `rice-\ab{acc}' in \LLast[b] is focus but given
%because it appears in the initial position preceding other focus elements.

%This claim seems difficult to test in the corpus because
%there is no definite markers in Japanese.
From the discussion above,
there are at least two predictions that can be tested in the corpus.
Firstly, since evoked and \isi{inferable} elements are coded by \isi{topic} markers, as was shown in Chapter \ref{Particles},
it is predicted that
elements coded by \isi{topic} markers tend to appear earlier in a clause (\S \ref{WO:ClauseInit:Ident:Topic}).
This is because elements assumed by the speaker to be evoked or \isi{inferable}
are also assumed to be shared.
Secondly,
since pronouns essentially code shared elements which have been mentioned,
pronouns are also predicted to appear earlier in a clause (\S \ref{WO:ClauseInit:Ident:Pron}).
Both predictions are confirmed in the following investigations.
Thirdly, I will show that clause-initial elements are not sensitive to \isi{activation cost};
unused elements can also appear clause-initially (\S \ref{WO:ClauseInit:Ident:ActStatus}).
Evoked, \isi{inferable}, declining, and unused elements are shared
(see Table \ref{ActStatusCorpus}).
Therefore, the claim that shared elements appear clause-initially is supported.

%%----------------------------------------------------
\subsubsection{Topic-coded elements appear clause-initially}\label{WO:ClauseInit:Ident:Topic}

\begin{figure}
%\begin{minipage}{0.5\textwidth}
	\begin{center}
	\includegraphics[width=0.7\textwidth]{figure/WOTopPar.pdf}
	\caption{Order of arguments coded by topic markers}
	\label{WOTopParF}
	\end{center}
%\end{minipage}
\end{figure}
\begin{figure}
%\begin{minipage}{0.5\textwidth}
	\begin{center}
	\includegraphics[width=0.7\textwidth]{figure/WOCasePar.pdf}
	\caption{Order of arguments coded by case markers}
	\label{WOCaseParF}
	\end{center}
%\end{minipage}
\end{figure}

%%% \ref{Par:ArgStr}で一部議論重複
Let us test the prediction that elements coded by \isi{topic} markers tend to appear earlier in a clause.
Figure \ref{WOTopParF} shows the distribution of topic-coded elements
and their positions.
Compare this figure with Figure \ref{WOCaseParF},
which shows the distribution of case-coded elements and their positions.
It is clear that elements coded by \isi{topic} markers are more skewed towards earlier positions within a clause than those coded by case markers.

\Next is an example of a \ci{wa}-coded element appearing clause-initially.
The \ci{wa}-coded element \ci{hone} `bone' in line a,
which has been discussed in the previous \isi{discourse},
is separated from the predicate by an intervening locative (a tomb for animals in the temple).
The intervening part is long and the predicate finally appears in line d.
%
\ex.
	\ag. ee suriipii-no itibu-no oo \EM{hone-wa} \\
		\ab{fl} Sleepy-\ab{gen} part-\ab{gen} \ab{fl} bone-\ci{wa} \\
		`Part of the bones of Sleepy (dog's name),'
		%`Part of bone of Sleepy (dog's name),'
	\bg. sono morimati-no watasi-no senzo-no o hait-teru otera-no \\
		that Morimachi-\ab{gen} \ab{1}\ab{sg}-\ab{gen} ancestor-\ci{gen} \ab{fl} enter-\ab{prog} temple-\ab{gen} \\
		`the temple in Morimachi where my ancestors were,'
	\bg. yahari ano doobutu-no kuyootoo-ga ari-masu \\
		again that animal-\ab{gen} tomb-\ci{ga} exist-\ab{plt} \\
		`there are tombs for animals,'
	\bg. sotira-no hoo-ni \ul{osame}-masite-ne \\
		that-\ab{gen} direction-\ab{dat} place-\ab{plt}-and \\
		`(we) placed (his bones) there.'
	\hfill{(\code{S02M1698: 620.12-634.26})}
%S02M1698|00620119L|620.118778|634.26275|L|(F えー)スリーピーの一部の(0.379)(F おー)骨は(0.438)その<FV>森町の(0.669)私の先祖の(0.598)(F お)入ってるお寺の(0.465)やはり(F あの)動物の供養塔があります(0.358)そちらの方に納め(0.347)ましてね|/テ節/|

In \Next,
\ci{sono ko} `that puppy',
whose referent has appeared in line a,
is also an example of a \ci{wa}-coded element appearing clause-initially.
The element is also separated from the predicate
by an intervening argument, `distemper'.
%
\ex.
	\ag. mosi \EMi{koinu-o} kat-tesimat-tara \\
		if puppy-\ci{o} keep-\ab{pfv}-\ab{cond} \\
		`If you decided to keep a new puppy,'
	\bg. \EM{sono} \EM{ko-wa} mata zisutenpaa-ni \ul{kakat}-te sin-zyau-kara \\
			that puppy-\ci{wa} again distemper-\ab{dat} catch-and die-\ab{pfv}-because \\
			`the puppy would die of distemper again, so'
	\b. keep a new puppy after this winter, this is what we were told by the vet.
	\hfill{(\code{S02M0198: 108.68-126.70})}
%
%S02M0198|00108681L|108.680591|126.700531|L|冬を越さないと(0.407)ジステンパーの細菌が(0.47)庭にいるままで死んでくれないから(0.431)もし小犬を飼っ(0.843)てしまったら(0.382)その子はまたジステンパーに掛かって死んじゃうから(0.584)冬を越してから(0.403)新しい犬を飼ってくれ(0.356)と僕らは言われていたもので|<並列節デ>|大きい切れ目−係り先なし

%\Next is an example of given elements that are neither coded by \isi{topic} markers nor put in clause-initially.
%The element \ci{mizu} `water' appears three times in \Next,
%all of which are given.
%\ex.\label{WO:Ex:mizu}
% \ag. desukara daitai iti-niti-ni ni-rittoru-no \EM{mizu-o} \ul{tot}-te kudasai-to iw-are-te \\
% so approximately one-day-for two-liter-\ab{gen} water-\ab{acc} drink-and please-\ab{quot} tell-\ab{pass}-and \\
% `So we were told to drink two liters of water per day,'
% \bg. syokuzi-no toki-wa kanarazu magukappu-de ni-hai-bun-no \EM{mizu-o} \ul{nomi}-masu-si \\
% 	meal-\ab{gen} time-\ci{wa} surely mug-with two-cup-amount-\ab{gen} water-\ab{acc} drink-\ab{plt}-and \\
%	`whenever we have meal, we drink two cups of water,'
% \bg. totyuu totyuu-de-mo kanarazu \EM{mizu-o} ho anoo \ul{nomi}-taku-naku-temo \\
% 		on.the.way on.the.way-\ab{loc}-also surely water-\ab{acc} \ab{frg} \ab{fl} drink-want-\ab{neg}-even.if \\
%		`also on the way, even if we didn't want to drink water,'
% \bg. nom-as-areru-to iu kanzi-de \\
% 	drink-\ab{caus}-\ab{pass}-\ab{quot} say feeling-\ab{cop} \\
%	`we were forced to drink (water).'
% \b. they think that drinking water is very important.
%  \hfill{(\code{S01F0151: 339.78-366.29})}
%%S01F0151|00339776L|339.775512|341.443878|L|でこのティータイムなんですけれども|/並列節ケレドモ/|
%%S01F0151|00341699L|341.698978|349.557717|L|この(0.43)標高の高いところでは(0.141)高山病という非常に危険な(0.329)可能性があるので(0.243)(F えー)水が非常に重要になります|[文末]|
%%S01F0151|00349955L|349.954875|366.290044|L|ですから大体一日に二リットルの水を取ってくださいと言われて(0.316)食事の時は必ずマグカップで二杯分の水を(0.145)飲みますし(0.285)途中途中でも必ず水を(0.113)(D ほ)(0.114)(F あのー)(0.432)飲みたくなくても飲まされるという感じで(0.403)水分補給(0.297)を(0.632)重視しておりました|[文末]|

%I argue that `water' in \Last is interpreted as \isi{indefinite} and the givenness of `water' does not matter in this narrative.
%This is because \ci{mizu} `water' appears non-initial position.
%Of course there are definite referents appearing non-initial position.
%However, the givenness of those referents are still not important.
%Consider the following example.
%%
%\ex.\label{WO:Ex:kuruma}
% \ag. kirauea-kazan-mo mappu-o kai-masi-te \\
% 		Kilauea-volcano-also map-\ab{acc} buy-\ab{plt}-and \\
%		`Also for Kilauea, (we) bought a map and'
% \bg. de zibun-tati-de ma rentakaa \EM{kuruma-o} \ul{tobasi}-te e iki-masi-ta \\
% 		then self-\ab{pl}-by \ab{fl} rent-a-car car-\ab{acc} drive-and \ab{fl} go-\ab{plt}-\ab{past} \\
%		`(we) drove there by rent-a-car by ourselves.'
% \b.[] (83.52 sec talking about the mountain.)
% \bg. de anoo jibun-no koko koko-de tyotto tome-te miyoo-to omot-ta toko-ni \\
% 		and \ab{fl} self-\ab{gen} \ab{frg} here-\ab{loc} a.bit stop-and try-\ab{quot} think-\ab{past} place-\ab{loc} \\
%	 	`At the place (we) wanted to stop,'
% \bg. koo \EM{kuruma-o} \ul{tome}-te \\
% 		this.way car-\ab{acc} stop-and \\
%		`(we) stopped the car,'
% \b. you can take pictures and so on.
% \hfill{(\code{S00F0014: 843.23-940.34})}
%%
%%S00F0014|00843233L|843.23315|850.162512|L|(F あのー)キラウエア火山もマップを買いましてで自分達で(0.363)(F ま)レンタカー(0.102)車を(0.376)飛ばして(0.367)(F え)行きました|[文末]|
%% ...
%%S00F0014|00933685L|933.684509|940.33856|L|で(0.299)(F あのー)自分のここここでちょっと止めてみようと思ったとこにこう車を止めて(0.398)(F まー)その写真を撮ったり|<タリ節>|大きい切れ目−係り先なし
%
%In \Last,
%\ci{kuruma} `car' is mentioned in lines b and d.
%Since it is reasonable to think that the speakers drove the same car while they were travelling,
%\ci{kuruma} `car' in line d is inferred as definite.
%However, the \isi{definiteness} or givenness of the car here is irrelevant to the \isi{discourse};
%the speaker is talking about their travel to Kilauea not about the car.
%Therefore,
%\ci{kuruma} is coded by the \isi{case marker} \ci{o} and appears non-initial position, i.e., immediately before the predicate as will be discussed in \S \ref{WOPrePredEles}.

\ci{Wa} appearing in initial position is already conventionalized, and
it is possible to test this with acceptability judgements.
It is not acceptable for a \ci{wa}-coded P to appear between the focus agent and the predicate except in contrastive readings of \ci{wa}.
%	\footnote{
%	Contrastive \ci{wa} appearing in this position will be
%	discussed more in detail in \S \ref{WODiscussion}.
%	}
As the contrast between \Next[a-c] shows,
the zero-coded P \ci{hon} `book' in \Next[a] right before the predicate is acceptable,
while the \ci{wa}-coded \ci{hon} `book' in the same position in \Next[b] is not acceptable.
To express the idea of \Next[b],
the \ci{wa}-coded P should precede the A, \ci{taroo} `Taro'.
%
\ex. \ag. \ul{taroo-ga} \EM{hon} yon-deru-yo \\
		Taro-\ci{ga} book read-\ab{prog}-\ab{fp} \\
		`Taro is reading a book.'
	\bg. ??\ul{taroo-ga} \EM{hon-wa} yon-deru-yo \\
		Taro-\ci{ga} book-\ci{wa} read-\ab{prog}-\ab{fp} \\
		`Taro is reading the book.'
	\bg. \EM{hon-wa} \ul{taroo-ga} yon-deru-yo \\
		book-\ci{wa} Taro-\ci{ga} read-\ab{prog}-\ab{fp} \\
		`Taro is reading the book.'
		\hfill{(Constructed)}

\largerpage
There is only one example (out of 9 \ci{wa}-coded Ps) in the corpus
where \ci{wa}-coded P is preceded by \ci{ga}-coded A.
This \ci{wa}-coded P is contrastive, a case which will be discussed in \S \ref{WODiscussion}.

I propose the hypothesis that elements which belong to the same unit of \isi{information structure} appear adjacent within a clause.
I call this the information-structure \isi{continuity principle} in \isi{word order}.
%
\ex. \label{IScontinuityP}\tl{Information-structure continuity principle}:
 A unit of \isi{information structure} is continuous in a clause;
 i.e., elements which belong to the same unit are adjacent to each other.

This principle explains why \LLast[b] is not acceptable,
while \LLast[a,c] are.
The \isi{information structure} of each of the examples in \LLast is represented in \Next.
In \Next[b],
the \isi{topic} P element \ci{hon-wa} `book-\ci{wa}' intervenes between two focus elements, \ci{taroo-ga} `Taro-\ci{ga}' and \ci{yon-deru} `read-\ab{prog}', which is not acceptable.
In \Next[c], on the other hand,
the \isi{topic} P does not split up the domain of focus,
and the whole sentence is acceptable.
In \Next[a],
all the elements including \ci{hon} `book' belong to focus
and hence \ci{hon} in this position is acceptable.
%
\ex. \ag. [\ul{taroo-ga} \EM{hon} yon-deru]$_{F}$-yo \\
		Taro-\ci{ga} book read-\ab{prog}-\ab{fp} \\
		`Taro is reading a book.'
	\bg. ??[\ul{taroo-ga}]$_{F}$ [\EM{hon-wa}]$_{T}$ [yon-deru]$_{F}$-yo \\
		Taro-\ci{ga} book-\ci{wa} read-\ab{prog}-\ab{fp} \\
		`Taro is reading the book.'
	\bg. [\EM{hon-wa}]$_{T}$ [\ul{taroo-ga} yon-deru]$_{F}$-yo \\
		book-\ci{wa} Taro-\ci{ga} read-\ab{prog}-\ab{fp} \\
		`Taro is reading the book.'

Interestingly,
it is possible for a \ci{wa}-coded A to be preceded by an \ci{o}-coded P, as shown in \Next[a] (compare this with \Next[b]).
%
\ex.
\ag. \ul{hon-o} \EM{taroo-wa}  yon-deru-yo \\
		book-\ci{o} Taro-\ci{wa} read-\ab{prog}-\ab{fp} \\
		`Taro is reading the book.'
\bg. \ul{hon-o} \EM{taroo-ga} yon-deru-yo \\
		book-\ci{o} Taro-\ci{ga} read-\ab{prog}-\ab{fp} \\
		`Taro is reading the book.'

\largerpage
As was argued above,
the preposed P, \ci{hon-o} `book-\ci{o}' in \Last,
is topical, which is represented as in \Next.
%
\ex.
\ag. [\ul{hon-o} \EM{taroo-wa}]$_{T}$  [yon-deru]$_{F}$-yo \\
		book-\ci{o} Taro-\ci{wa} read-\ab{prog}-\ab{fp} \\
		`Taro is reading the book.'
\bg. [\ul{hon-o}]$_{T}$ [\EM{taroo-ga} yon-deru]$_{F}$-yo \\
		book-\ci{o} Taro-\ci{ga} read-\ab{prog}-\ab{fp} \\
		`Taro is reading the book.'

As shown in \Last[a],
the two \isi{topic} elements \ci{hon-o} `book-\ci{o}' and \ci{taroo-wa} `Taro-\ci{wa}' are adjacent to each other and hence this sentence is acceptable.
Also in \Last[b],
the only \isi{topic} element \ci{hon-o} `book-\ci{o}' does not split up the focus elements \ci{taroo-ga yon-deru}, which is predicted to be acceptable.
\ci{Hon-o} `book-\ci{o}' could be a focus instead of a \isi{topic} in \LLast[b], since given elements can be foci.
But it is reasonable to think of a situation where given focus elements are preposed
so that there is a smooth transition from the previous sentence.
The information-structure \isi{continuity principle} in \ref{IScontinuityP} still holds in either case.

Note that \ref{IScontinuityP} does not refer to \isi{word order};
rather, it is about adjacency.
I argue that this principle is also at work in intonation (see Chapter \ref{Intonation}).

What is the difference between clause-initial elements coded by \isi{topic} markers and those coded by case markers?
As was discussed in \S \ref{Par:ArgStr:TopHierarchy},
there is a hierarchy of \isi{topic} coding \ref{ASPGivenSchema},
which is repeated here as \Next.
%
\ex.
 A, S $>$ P

The hierarchy indicates that an
evoked or \isi{inferable} A or S is more likely to be coded by a \isi{topic} marker than a P in the same activation status.
Word order is not affected by this hierarchy.
Figures \ref{WOSGivenF} and \ref{WOPGivenF} show \isi{word order} of
\isi{anaphoric} S and P, respectively.
Compare these with Figures \ref{WOSNewF} and \ref{WOPNewF},
which show the \isi{word order} of non-\isi{anaphoric} S and P.
The word order of A is omitted because the number is too small.
As can be seen from the contrasts between Figures \ref{WOSGivenF} and \ref{WOSNewF} and between Figures \ref{WOPGivenF} and \ref{WOPNewF},
\isi{anaphoric} elements are more likely to appear earlier in the clause than non-\isi{anaphoric} elements.
Although the contrast is less clear between \isi{anaphoric} vs.~non-\isi{anaphoric} P,
what is especially notable is that there are three times as many \isi{anaphoric} Ps as non-\isi{anaphoric} Ps in the third position.
(There are 27 \isi{anaphoric} Ps in the third position,
while there are only 10 non-\isi{anaphoric} Ps.)
I speculate that the contrast is less clear in \isi{anaphoric} vs.~non-\isi{anaphoric} P than S because there are cases like \ref{WO:ClauseInit:Given:mizu} and \ref{WO:ClauseInit:Given:tenkan},
where the element is annotated as \isi{anaphoric} but is considered to not be shared.
In this case, P appears pre-predicatively rather than clause-initially.
Therefore, I argue that,
while elements coded by \isi{topic} markers are likely to appear earlier in the clause,
\isi{word order} is independent of \isi{topic} marking.
Topic markers are sensitive to the given-new taxonomy, as was discussed in Chapter \ref{Particles};
clause-initial position is sensitive to sharedness.
Topic markers and \isi{word order} are sensitive to different aspects of topichood.

\begin{figure}
%\begin{minipage}{0.5\textwidth}
	\begin{center}
	\includegraphics[width=0.6\textwidth]{figure/WOSGiven.pdf}
	\caption{Word order of anaphoric S}
	\label{WOSGivenF}
	\end{center}
%\end{minipage}
\end{figure}
\begin{figure}
%\begin{minipage}{0.5\textwidth}
	\begin{center}
	\includegraphics[width=0.6\textwidth]{figure/WOPGiven.pdf}
	\caption{Word order of anaphoric P}
	\label{WOPGivenF}
	\end{center}
%\end{minipage}
\end{figure}
\begin{figure}
%\begin{minipage}{0.5\textwidth}
	\begin{center}
	\includegraphics[width=0.6\textwidth]{figure/WOSNew.pdf}
	\caption{Word order of non-anaphoric S}
	\label{WOSNewF}
	\end{center}
%\end{minipage}
\end{figure}
\begin{figure}
%\begin{minipage}{0.5\textwidth}
	\begin{center}
	\includegraphics[width=0.6\textwidth]{figure/WOPNew.pdf}
	\caption{Word order of non-anaphoric P}
	\label{WOPNewF}
	\end{center}
%\end{minipage}
\end{figure}

%%----------------------------------------------------
\subsubsection{Pronouns appear clause-initially}\label{WO:ClauseInit:Ident:Pron}

Next, let us examine the position of pronouns.
Figure \ref{WOExpTypeF} shows the position of pronouns.
Figure \ref{DEPositionAllF}, repeated as Figure \ref{DEPositionAllF2} for comparison,
represents the distribution of all elements.
Although the number of pronouns is small,
it is clear, comparing with the overall distribution of elements in Figure \ref{DEPositionAllF2}, that
the order of pronouns is skewed towards initial positions within a clause.
Hence, it is reasonable to conclude that
pronouns are likely to appear earlier in a clause.
Examples of pronouns appearing earlier in a clause are shown in \ref{PronIni1} and \ref{PronIni2} above.
This result is compatible with \citeA{yamashita02} and \citeA{kondoyamashita07}.


\begin{figure}
%\begin{minipage}{0.5\textwidth}
	\begin{center}
	\includegraphics[width=0.6\textwidth]{figure/DEPositionAll.pdf}
	\caption{Order of all elements}
	\label{DEPositionAllF2}
	\end{center}
%\end{minipage}
\end{figure}
\begin{figure}
%\begin{minipage}{0.5\textwidth}
	\begin{center}
	\includegraphics[width=0.6\textwidth]{figure/WOExpType.pdf}
	\caption{Order of pronouns}
	\label{WOExpTypeF}
	\end{center}
\end{figure}

%\begin{table}
%\begin{minipage}{0.5\textwidth}
%\centering
%	\caption{Markers for ASP (given)}
%	\label{ASPParGivenT}
%	\begin{tabular}{lrrr}
%	\lsptoprule
%			 & A & S & P \\
%	\midrule
%	\ci{o} 	& 0 & 0 & 166 \\
%	\ci{ga} 	& 29 & 191 & 0 \\
%	\midrule
%	\isi{case marker} & 29 & 191 & 166 \\
%	  & (50.9\%) & (59.0\%) & (91.2\%) \\
%	\midrule
%	\midrule
%	\ci{wa} 	& 27 & 104 & 14 \\
%	\ci{toiunowa} 	& 1 & 29 & 2 \\
%	\midrule
%	\isi{topic} marker & 28 & 133 & 16 \\
%	  & (49.1\%) & (41.0\%) & (8.8\%) \\
%	\midrule
%	\midrule
%	sum & 57 & 324 & 182 \\
%	  & (100\%) & (100\%) & (100\%) \\
%	\lspbottomrule
%	\end{tabular}
%\end{minipage}
%\begin{minipage}{0.5\textwidth}
%\centering
%	\caption{Markers for ASP (new)}
%	\label{ASPParNewT}
%	\begin{tabular}{rrr}
%	\lsptoprule
%	A & S & P \\
%	\midrule
%	0 & 0 & 211 \\
%	15 & 351 & 0 \\
%	\midrule
%	15 & 351 & 211 \\
%	(78.9\%) & (76.5\%) & (92.5\%) \\
%	\midrule
%	\midrule
%	3 & 90 & 14 \\
%	1 & 18 & 3 \\
%	\midrule
%	4 & 108 & 17 \\
%	(21.1\%) & (23.5\%) & (7.5\%) \\
%	\midrule
%	\midrule
%	19 & 459 & 228 \\
%	(100\%) & (100\%) & (100\%) \\
%	\lspbottomrule
%	\end{tabular}
%\end{minipage}
%\end{table}

%%----------------------------------------------------
\subsubsection{Unused elements appear clause-initially}\label{WO:ClauseInit:Ident:ActStatus}

Not only evoked, \isi{inferable}, and declining elements,
but also unused elements appear clause-initially.
Elements coded by the \isi{copula} followed by \ci{ga} or \ci{kedo} are unused elements, as was discussed in Chapter \ref{Particles}.%
 \footnote{
 \chd{See \S \ref{BackSubSubKedo} for en explanation why
 an element coded by the \isi{copula} followed by \ci{ga} or \ci{kedo} is not considered to be a clause.}
 }
It is very unnatural for them to be preceded by other arguments.
For example,
as shown in the contrast between \Next[a] and \Next[b],
\ci{rei-no ken} `that issue' cannot be felicitously preceded by another argument, in this case \ci{kotira-de} `this side'.
%
\ex.
 \ag. \EM{rei-no} \EM{ken-desu-ga} kotira-de nantoka nari-sou-desu \\
      that-\ab{gen} issue-\ab{cop}.\ab{plt}-though this.side-\ab{loc} whatever become-will-\ab{cop}.\ab{plt} \\
      `Regarding that issue, (I) guess (we) figured the way out.'
      \hfill{\cite[modified from][283]{niwa06}}
 \bg.[a$^{\prime}$.] ??kotira-de \EM{rei-no} \EM{ken-desu-ga} nantoka nari-sou-desu \\
      this.side-\ab{loc} that-\ab{gen} issue-\ab{cop}.\ab{plt}-though whatever become-will-\ab{cop}.\ab{plt} \\

\largerpage
In a similar manner,
\ci{yamada-no koto} `the issue of Yamada' cannot naturally be preceded by an \isi{adverbial}, \ci{ano mama} `that way',
as shown in the contrast between \Next[a] and \Next[b].
%
\ex.
 \ag. \EM{yamada-no} \EM{koto-da-kedo} ano mama hot-toi-te ii-no-kana \\
      Yamada-\ab{gen} issue-\ab{cop} that way leave-let-and good-\ab{nmlz}-\ab{q} \\
      `Regarding Yamada, is it OK to just leave him?'
      \hfill{\cite[283]{niwa06}}
 \bg.[a$^{\prime}$.] ??ano mama \EM{yamada-no} \EM{koto-da-kedo} hot-toi-te ii-no-kana \\
      that way Yamada-\ab{gen} issue-\ab{cop} leave-let-and good-\ab{nmlz}-\ab{q} \\


Unused elements also include \isi{indefinite} elements, even though it is counter-intuitive to consider \isi{indefinite} NPs as being ``shared''.
For example, as was mentioned in \S \ref{Fr:Definition:TFFeathers:Definite},
an \isi{indefinite} element can appear clause-initially
if the speaker assumes the \isi{hearer} to remember that the speaker (or somebody else) has talked about a category the element refers to.
For example, as shown in \Next[Y],
repeated from \ref{Fr:Definition:TFFeathers:Definite:Ex:Mango1} in \S \ref{Fr:Definition:TFFeathers:Definite},
having mentioned the category ``mango'' makes it possible for \ci{mangoo} `mango' to appear clause-initially, even though \ci{mangoo} `mango' is clearly \isi{indefinite}
since the \isi{hearer} has no way to tell which mango the speaker ate. I regard this as unused and hence shared.
%
\ex. Context:
	Y told H that he had never seen or eaten mangoes.
	H told Y that they are delicious.
	Several days later, Y finally ate a mango.
	\ag.[Y:] \EM{mangoo} konoaida miyako-zima-de tabe-ta-yo \\
			mango the.other.day Miyako-island-\ab{loc} eat-\ab{past}-\ab{fp} \\
			`(I) ate (a) mango (we talked about) in Miyako island the other day.'
	\bg.[Y$^{\prime}$:] konoaida miyako-zima-de \EM{mangoo} tabe-ta-yo \\
			the.other.day Miyako-island-\ab{loc} mango eat-\ab{past}-\ab{fp} \\
			`(I) ate (a) mango in Miyako island the other day.'

In this case, however,
\ci{mangoo} `mango' in the pre-predicate position is also felicitous,
as in \Last[Y$^{\prime}$],
which indicates that this is a borderline case;
\ci{mangoo} can be a \isi{topic} in the sense that
it is unused and the speaker has talked about it before,
while it can be a focus in the sense that
it is new to the \isi{discourse} and \isi{indefinite}.

On the other hand, in \Next[Y],
%repeated from (\ref{Fr:Definition:TFFeathers:Definite:Ex:Mango2}),
where the speaker does not assume the \isi{hearer} to remember that
the speaker has talked about mangoes,
clause-initial \ci{mangoo} `mango' is infelicitous,
whereas pre-predicate \ci{mangoo} is perfectly acceptable.
%
\ex. Context:
	Y and H have not met for a few months.
	\a.[H:] What did you do these days?
	\bg.[Y:] ??\EM{mangoo} konoaida miyako-zima-de tabe-ta-yo \\
			mango the.other.day Miyako-island-\ab{loc} eat-\ab{past}-\ab{fp} \\
		\hfill(=\LLast[Y])
	\bg.[Y$^{\prime}$:] konoaida miyako-zima-de \EM{mangoo} tabe-ta-yo \\
			the.other.day Miyako-island-\ab{loc} mango eat-\ab{past}-\ab{fp} \\
			`(I) ate (a) mango in Miyako island the other day.'
		\hfill(=\LLast[Y$^{\prime}$])

\newpage
Therefore, it is reasonable to conclude that
shared elements include those which refer to categories the speaker (or somebody else) has talked about, and that they can appear clause-initially.


%%----------------------------------------------------
\subsection{Persistent elements tend to appear clause-initially}\label{PersistentAppearClause-Initially}

Persistent elements are skewed to earlier positions more than non-persistent elements,
as shown in Figure \ref{DEPositionPerF}.
%This is especially clear in the contrast between Figure \ref{WOASPPerF} and \ref{WOASPNPerF},
%which show the \isi{word order} of A, S, and P for persistent and non-persistent elements
%in a clause which contains more than one argument.
%Most Exs and As are persistent and precede other arguments as in Figure \ref{WOASPPerF} compared to Figure \ref{WOASPNPerF}.

%\begin{figure}
%\begin{minipage}{0.5\textwidth}
%	\begin{center}
%	\includegraphics[width=0.95\textwidth]{figure/WOASPPer.pdf}
%	\caption{Word order vs.\ grammatical function (persistent)}
%	\label{WOASPPerF}
%	\end{center}
%\end{minipage}
%\begin{minipage}{0.5\textwidth}
%	\begin{center}
%	\includegraphics[width=0.95\textwidth]{figure/WOASPNPer.pdf}
%	\caption{Word order vs.\ grammatical function (non-persistent)}
%	\label{WOASPNPerF}
%	\end{center}
%\end{minipage}
%\end{figure}
The following are examples of persistent elements appearing clause-initially.
In \Next,
\ci{hihu-byoo} `skin-disease' in line a, coded by the \isi{topic} marker \ci{toiuno-wa},
appears clause-initially.
The predicate appears in line c,
separated from the subject by a proposition in line b and by another clausal argument (\ci{hito-ni} `person-by') in line c.
Also, in line d, \ci{kore-wa} `this-\ci{wa}', referring to `skin-disease', appears clause-initially.
%
\ex.
 \ag. \EM{hihu-byoo-toiuno-wa} \\
 		skin-disease-\ci{toiuno-wa} \\
		`The skin disease,'
 \bg. damat-tei-temo \\
 		keep.silent-\ab{prog}-even.if \\
		`even if you don't tell people about it,'
 \bg. hito-ni \ul{mir-are-te-simau} mono-dat-ta-node \\
 		person-by see-\ab{pass}-and-\ab{pfv} thing-\ab{cop}-\ab{past}-because \\
		`people can see it, so'
 \bg. \EM{kore-wa} ano omot-ta izyooni seesintekini \ul{kutuu-desi}-ta \\
 		this-\ci{wa} \ab{fl} think-\ab{past} more mentally painful-\ab{cop}-\ab{past} \\
		`this was more mentally painful than I had expected.'
		\hfill{(\code{S02F0100: 222.75-231.09})}
%S02F0100|00222751L|222.750605|231.088045|L|皮膚病というのは(0.428)黙っていても(0.285)人に見られてしまうものだったので(0.429)これは(F あの)思った以上に精神的に苦痛でした|[文末]|

Similarly, in \Next,
\ci{sore-wa} `that-\ci{wa}' in line b and g,
and \ci{sore-dake-wa} `that-only-\ci{wa}' in line i,
all of which refer to `chelow kebab' in line a,
appear clause-initially.
%
\ex.
 \a. There is a dish called \EM{chelow kebab}.
 \bg. de \EM{sore-wa} eeto gohan-ni eeto bataa-o maze-te \\
 	and that-\ci{wa} \ab{fl} rice-to \ab{fl} butter-\ci{o} mix-and \\
	`That, you mix rice with butter...'
 \b. on top of that you put spice,
 \b. on top of that you put mutton,
 \b. you mix it and eat it.
 \b. There were many dishes of this kind.
 \bg. \EM{sore-wa} kekkoo sonnani hituzi-no oniku-no kusasa-mo naku-te \\
 	that-\ci{wa} to.some.extent not.really sheep-\ab{gen} meat-\ab{gen} smell-also not.exist-and \\
	`It did not have smell of mutton...'
 \b. I thought it was delicious.
 \bg. \EM{sore-dake-wa} anoo iran-ryoori-no naka-de \ul{taberu} koto-ga ano deki-ta ryoori-desu \\
 		that-only-\ci{wa} \ab{fl} Iran-dish-\ab{gen} inside-\ab{loc} eat thing-\ci{ga} \ab{fl} can-\ab{past} dish-\ab{cop} \\
		`This is the only dish I could eat among \ili{Iranian} dishes.'
 \hfill{(\code{S03F0072: 446.03-447.66})}
%S03F0072|00446026L|446.026013|447.663208|L|チェロカバブというのがありまして|/テ節/|
%S03F0072|00448150L|448.150018|463.667799|L|でそれは(0.114)(F えーと)御飯に(0.376)(F えーと)バターを混ぜてその上に香辛料を振って(0.174)その上に羊のお肉が乗っていて(0.289)それをこう(0.11)混ぜてぐちゃぐちゃに混ぜて食べるという(0.441)(F えーと)お料理が(0.707)(F あのー)(0.681)多かったんですけれども|/並列節ケレドモ/|
%S03F0072|00464361L|464.360734|476.870208|L|それは結構そんなに羊の(0.42)お肉の臭さもなくて(1.094)(F あのー)(0.156)おいしいな(0.178)って(0.418)思ってそれだけは(0.475)(F あのー)イラン料理の中で(0.273)食べることが(0.28)(F あの)できた料理です|[文末]|

\chd{As was mentioned in \ref{WO:Intro},
both \isi{word order} and particles significantly contribute to predict persistence,
contrary to the result of \citeA{imamura17},
who concludes that ``scrambling [PSV order] is pertinent to anaphorically prominent but cataphorically non-prominent objects and that topicalization is especially germane to `continuing \isi{topic}' as the referent of the object'' (p.~78).
There are a few potential reasons why the results of the present work are different from those of \citeA{imamura17}.
One potential reason is the difference of modalities:
\citeA{imamura17} employed a corpus of written Japanese (\ci{the Balanced Corpus of Contemporary Written Japanese}, BCCWJ), while the present study employs spoken Japanese.
Related to the first point,
clause-chaining -- which, as I will point out, is one of the reasons why clause-initial elements tend to be persistent (see the next section) --
only appears in spoken Japanese, but not in written Japanese.
In any case, this is a mere speculation and further studies are needed to analyze
why the results of these two studies differ.}


%%----------------------------------------------------
\subsection{Motivations for topics to appear clause-initially}\label{TopicAppearClause-Initially}

As was pointed out by many linguists,
topics tend to appear clause-initially
because they function as an anchor to the previous \isi{discourse}.
The principle in \ref{oldnewprinciple} is motivated by this processing convenience \cite[e.g.,][]{keenan77}.
Clause-initial locatives and other adjectives can also be explained by this motivation.
This anchoring function works best when the \isi{activation cost} of the referent is relatively high \cite{givon83};
i.e.,
when the referent of the element in question is \isi{inferable} or declining.
When the \isi{activation cost} is low, i.e., when the \isi{topic} is continuous from the previous \isi{discourse},
the element in question that refers to the \isi{topic} is expected to be zero \cite{givon83,gundeletal93,ariel90};
there is no need for anchoring because the \isi{topic} is already evoked and the \isi{hearer} expects it to also be mentioned in the current sentence.
This explanation predicts that the distance between the element in question and the \isi{antecedent} is larger when the element in question is expressed in the form of an NP instead of zero.
Figure \ref{DistExpTypeF} appears to support this prediction,
\chd{although a statistical analysis indicates that the expression type does not significantly contribute to predict distance.
This paragraph discusses NPs with long distance.
See the discussion below for NPs with shorter distance.}
The whisker plot in Figure \ref{DistExpTypeF} shows the distance between the element in question (NP vs.\ (explicit) \isi{pronoun} vs.\ zero \isi{pronoun}) and its \isi{antecedent}.
It measures the time between the production of the  \isi{first mora} of the element in question and the production of the \isi{first mora} of the \isi{antecedent}.
The figure shows that, in many cases, the distance between the NP and the \isi{antecedent} is larger than that of zero and the \isi{antecedent}.
Zero pronouns are assumed to be produced at the time
when the \isi{first mora} of the predicate is uttered.

This pattern is exemplified in \Next,
\chd{where zero pronouns are indicated by \ci{\O}.}
In line b, \ci{san-nin-me} `the last person' precedes adjuncts (`last fall') and is coded by a variation of \ci{toiuno-wa} (\ci{ttuuno-wa}).
\chd{Zero pronouns \ci{\O} are inserted right before the predicate for the purpose of presentation,
but this does not affect the analysis.}
Since this person is one of the three people mentioned in line a,
s/he is \isi{inferable}
through a part-whole relation.
The \isi{topic} moves on to another person in line f, who is also one of the three people mentioned in line a.
In line j, the speaker again refers to the person mentioned in line b.
Also, this time the element \ci{moo hitori-wa} `the other person' appears near the beginning of the clause, preceding other arguments.
The referent continues to be mentioned until line q.
%where the elements that refer to this person is either pronouns, as in line h and k, or zero, as in line i, l, and o.
Finally, the speaker starts talking about himself in line r,
in which case the element \ci{boku-wa} `\ab{1}\ab{sg}-\ci{wa}' appears near the beginning of the clause.
%
\ex.
 \a. All of us three quit this job, interestingly, or strangely.
 \bg. de anoo \EM{san-nin-me-ttuuno-wa} tui se ee kyonen-no o aki-ni yame-ta-n-desu-kedomo \\
 	and \ab{fl} three-\ab{cl}-\ab{ord}-\ci{toiuno}-\ci{wa} just \ab{frg} \ab{fl} last.year-\ab{gen} \ab{fl} fall-in quit-\ab{past}-\ab{nmlz}-\ab{cop}.\ab{plt}-though \\
	`The last person quit this fall.'
 \bg. \EM{soitu-wa} maa itiban saisyo-ni yame-tai yame-tai ttut-ta ningen-nan-desu-kedomo \\
 		\ab{3}\ab{sg}-\ci{wa} \ab{fl} most first-in quit-want quit-want \ab{quot}.say-\ab{past} person-\ab{nmlz}-\ab{cop}.\ab{plt}-though \\
		`He was the first person who said he wanted to quit.'
 \b. This kind of thing happens often.
 \b. All of us three quit eventually.
 \bg. ndee \ul{hitori-wa}-desu-ne \\
 		then one.person-\ci{wa}-\ab{cop}.\ab{plt}-\ab{fp} \\
		`Concerning another person,'
 \b. I guess this is closely related to the fact that we worked in Mobara.
 \bg. de hitotu \ul{sono} \ul{hito-wa} ee ma yappari tonikaku hatarai-te okane-ga koo te-ni \EM{\O} hairu-tte iu koto-ni itiban-no kati-o miidasi-ta wake-desu-ne sono ziki-ni \\
 		then one.thing that person-\ci{wa} \ab{fl} \ab{fl} as.expected any.way work-and money-\ci{ga} this.way hand-to {\O} get.in-\ab{quot} say thing-to most-\ab{gen} value-\ci{o} find-\ab{past} reason-\ab{cop}.\ab{plt}-\ab{fp} that time-at \\
		`At that time this person found it most valuable to work hard and gain money.'
 \b. (Explanation about his view on working. 9.3 sec.)
 \bg. de moo \EM{hitori-wa} maa \EM{kare-mo} hi hizyooni mobara-o aisi-teru-n-desu-ga \\
 	then more one.person-\ci{wa} \ab{fl} \ab{3}\ab{sg}.\ab{m}-also \ab{frg} very Mobara-\ci{o} love-\ab{prog}-\ab{nmlz}-\ab{cop}.\ab{plt}-though \\
	`The other one, who also loves Mobara (a place name),'
 \bg. kondo-no sigoto-tte atarasiku \EM{\O} tui-ta sigoto-tteiuno-wa \\
 		next-\ab{gen} job-\ab{quot} newly {\O} acquire-\ab{past} job-\ci{toiuno}-\ci{wa} \\
		`(his) next job, the new job (he) acquired is...'
 \bg. maa inaka-no hoo-no sigoto-nan-desu-ne \\
 	\ab{fl} rural-\ab{gen} area-\ab{gen} job-\ab{nmlz}-\ab{cop}.\ab{plt}-\ab{fp} \\
	%`in rural area.'
	in a rural area.'
 \bg. de \EM{kare} iwaku-desu-ne \\
 	then \ab{3}\ab{sg}.\ab{m} say-\ab{plt}-\ab{fp} \\
	`According to what he says,'
 \bg. sono yama-ga nai tokoro-ni-wa \EM{\O} sum-e-nai-to \\
 	\ab{fl} mountain-\ci{ga} not.exist place-at-\ci{wa} {\O} live-can-\ab{neg}-\ab{quot} \\
 	`He says that he cannot live in places without mountains.'
 \b. Though Mobara does not have mountains, the sky in Mobara is clear.
 \b. We call it Mobara sky. Mobara has such an idyllic scene.
 \bg. sore-ga maa doositemo nai-to \EM{\O} sum-e-nai-tte iu koto-o sono ziki-ni \EM{\O} sato-ta-n-zya-nai-ka-to \\
 	that-\ci{ga} \ab{fl} by.all.means not.exist-\ab{cond} {\O} live-can-\ab{neg}-\ab{quot} say thing-\ci{o} that time-in {\O} learn-\ab{past}-\ab{nmlz}-\ab{cop}-\ab{neg}-\ab{q}-\ab{quot} \\
	`(He) learned at that time that (he) can't live without such scene (I guess).'
 \b. de \EMi{boku-wa}-to ii-masu-to \\
 	then \ab{1}\ab{sg}-\ab{quot} say-\ab{plt}-\ab{cond} \\
	`Talking about myself...'
 \b. ...
   \hfill{(\code{S05M1236: 639.40-738.22})}
%S05M1236|00639399L|639.399427|647.222898|L|て(0.466)(F まー)これがまた(?)(0.115)(F あのー)(0.273)不思議なことにと言うか(0.211)<咳>面白いことにその三人共ですねその会社を(0.187)(F えー)(0.378)辞めました|[文末]|
%S05M1236|00647283L|647.283|653.466794|L|<笑>で(0.587)(F あのー)三人目っつうのはつい(D せ)(F えー)去年の(0.222)(D (? お))秋に辞めたんですけども|/並列節ケドモ/|
%S05M1236|00653467L|653.466794|656.353546|L|そいつは(F まー)一番最初に辞めたい辞めたいっつった人間なんですけども|/並列節ケドモ/|
%S05M1236|00656478L|656.47798|657.577155|L|えてしてそういうもんですが|/並列節ガ/|
%S05M1236|00658101L|658.101232|660.387564|L|(F えー)(0.742)三人共辞めました|[文末]|
%S05M1236|00660711L|660.710985|667.679344|L|んで一人はですね(0.671)(F えー)(0.238)(F ま)これやっぱり茂原で働いたっていうことに大きく関係してそうなんですねその辞めた理由っていうのが||倒置−つなぎ切り
%S05M1236|00667930L|667.930126|678.94475|L|で一つ(D い)その人は(0.608)(F えー)(0.293)(F ま)やっぱり(0.376)とにかく(0.172)働いてお金がこう手に入るっていうことに(0.215)一番の価値観を(0.47)見出だした訳ですねその(0.157)その(D 時)(F えー)時期に||倒置−つなぎ切り
%S05M1236|00679812L|679.811821|689.132829|L|それで今でもですね(0.563)(F まー)(0.166)一番お金が(0.132)入るところっていうんで(0.267)色々職を転々として(0.54)(F まー)(0.294)相当金持ちに(0.284)なってるようですけども|/並列節ケドモ/|
%S05M1236|00689447L|689.446897|694.030141|L|(F まー)一つの生き方かなと(0.427)いう風に(0.372)(F えー)(0.685)僕なんかも見てますけども|/並列節ケドモ/|
%S05M1236|00694880L|694.880285|707.26424|L|でもう一人は(0.959)(F まー)(0.656)彼も(D (? ひ))非常に茂原を愛してるんですが(0.285)(F えー)(0.466)(F ま)今度の仕事っていうのは(0.473)(F あのー)(0.294)今度の仕事って新しく就いた仕事っていうのは(0.39)(F まー)田舎の方(D2 で)(0.275)(F えー)の仕事なんですね|[文末]|
%S05M1236|00708165L|708.164927|711.5187|L|で彼曰くですね(0.192)(F その)山がないところには住めないと|[と文末]|
%S05M1236|00711927L|711.927|728.390145|L|<笑>で茂原ってのは(F まー)山がある訳じゃないんだけど(0.393)(F あのー)(0.308)非常に澄み切った(0.517)(F えー)空でそれのことを(0.24)(F あのー)僕ら茂原晴れと言ってるんですけども(0.602)(F えー)(F まー)(D 非)非常に(F あのー)(0.697)(F まー)(0.215)そういう(0.561)のどかな風景がある訳ですね|[文末]|
%S05M1236|00729305L|729.305466|734.964084|L|で(0.259)<咳>(0.75)それが(F まー)どうしてもないと住めないっていうことをその時期に悟ったんじゃないかと|[と文末]|
%S05M1236|00736425L|736.425057|738.218691|L|で(0.664)僕はと言いますと|/条件節ト/|
%S05M

In this type of example,
clause-initial elements, especially those coded by \isi{topic} markers, function as an anchor to the previous \isi{discourse}.

\begin{figure}
%\begin{minipage}{0.5\textwidth}
	\begin{center}
	\includegraphics[width=0.5\textwidth]{figure/DistExpType.pdf}
	\caption{Anaphoric distance vs.\ expression type}
	\label{DistExpTypeF}
	\end{center}
%\end{minipage}
\end{figure}


However,
Figure \ref{DistExpTypeF} also indicates that
(explicit) pronouns (\ci{kore} `\ab{dem}.\ab{prox} (this)', \ci{sore} `\ab{dem}.\ab{med} (this/that)', \ci{are} `\ab{dem}.\ab{dist} (that)', \ci{kare} `\ab{3}\ab{sg}.\ab{m} (he)', \ci{kanozyo} `\ab{3}\ab{sg}.\ab{f} (she)')%
	\footnote{
	\ci{Kare} `\ab{3}\ab{sg}.\ab{m} (he)' and \ci{kanozyo} `\ab{3}\ab{sg}.\ab{f} (she)' are very rare in spoken Japanese.
	Instead, \ci{kono hito} `this person' or similar expressions are used more frequently.
	However, this study does not count them as pronouns.
	}
and zero pronouns do not differ from each other.
%Or pronouns even seem to have shorter distance than zeros.
Moreover, there are NPs which refer to the immediate \isi{antecedent}.
\chd{Whereas more than half of the NPs have a longer distance than explicit and zero pronouns,
the figure also shows that many NPs have distances as short as those of explicit and zero pronouns.
In fact, a fixed effects analysis for distance (with expression type as a fixed effect and speaker as a random effect) indicates that expression type is not a significant factor to predict distance.}
For example, in example \Last,
the referent of \ci{hitori} `one person' in line f is mentioned in line h as \ci{sono hito} `that person' again,
although the distance is not very large.%
	\footnote{
	The impression of line g is that of an inserted clause rather than a \isi{topic} shift.
	}
In a similar manner,
the referent of \ci{san-nin-me} in line b is mentioned in the immediately following clause (line c) as \ci{soitu} `\ab{3}\ab{sg}'.
These examples are not mere exceptions.
In fact, 74.1\% of referents mentioned for the second time are still expressed in the form of an NP;
only 21.4\% are expressed as zero and 4.6\% as a \isi{pronoun},
as shown in Table \ref{AnaCountExpTypeT} and Figure \ref{AnaCountExpTypeF}.
Figure \ref{AnaCountExpTypeF} and Table \ref{AnaCountExpTypeT} show
the expression type of the element in question based on how many times the referent is mentioned.
``2" indicates that the element in question is mentioned for the second time,
``3" indicates that it is mentioned for the third time, and so on.
The ratio of zero increases as the referent keeps being mentioned.
The fact that the referent introduced is mentioned repeatedly is also reported in \citeA{clancy80}, who investigates Pear Stories;
this pattern is not unique to the corpus of the current study.
\Next is another example
of two NPs which refer to the same referent and which adjacent.
In this example,
the very long word \ci{yuugosurabia-syakaisyugi-kyoowakoku} `Socialist Federal Republic of Yugoslavia' is repeated twice.
%
\ex.\label{WO:TopicAppearClause-Initially:Ex:Yuugo}
 \ag. ee kon ma kono tiiki ee yu ma \EM{kyuu-yuugosurabia-syakaisyugi-kyoowakoku}-toiu tokoro-nan-desu-keredomo \\
 	\ab{fl} \ab{frg} \ab{fl} this area \ab{fl} \ab{frg} \ab{fl} former-Yugoslavia-socialist-republic-\ab{quot} place-\ab{nmlz}-\ab{cop}.\ab{plt}-though \\
	`This area is called Socialist Federal Republic of Yugoslavia,'
 \bg. kono \EM{yuugosurabia-syakaisyugi-kyoowakoku}-tteiuno-wa motomotoga ee minzoku-tairitu-no hagesii tiiki-de-gozai-masi-te \\
 	this Yugoslavia-socialist-republic-\ci{toiuno}-\ci{wa} originally \ab{fl} ethnic-conflict-\ab{gen} severe area-\ab{cop}-\ab{plt}-\ab{plt}-and \\
	`this Socialist Federal Republic of Yugoslavia is an area with severe ethnic conflicts...'
	\src{S00M0199: 81.95-94.42}
%S00M0199|00081950L|81.949813|94.424971|L|(F えー)(0.143)(D こん)(F ま)この地域(F えー)(D ユ)(0.253)(F ま)旧ユーゴスラビア社会主義共和国というところなんですけれども(0.426)このユーゴスラビア社会主義共和国っていうのは元々が(0.26)(F えー)民族対立の激しい地域でございまして|/テ節/|
%S00M0199|00094875L|94.874776|96.351778|L|(F えー)(0.217)(F ま)一つの||言いさし−言い直しあり
%S00M0199|00096450L|96.449668|124.995828|L|(F えー)(F え)戦後(F えー)第二次大戦終了後(0.115)(F おー)から言われてた言葉として(0.418)(F えー)(D ひ)一つの(0.152)(F えー)(D す)(F ま)政党ですね共産党が支配する(0.418)(F えー)二つの(0.152)(F えー)(F ま)二つの(0.108)(F え)言葉(F え)(F う)(0.196)文字を持ち(0.438)三つの宗教があり(0.219)(F えー)四つの(0.132)(F えー)言語を話す(0.471)で五つの(F えー)民族によって構成される(0.434)国と言われる(F まー)(0.447)(F えー)歴史的に見ても類い稀な(F あ)(F えー)モザイク国家と(0.282)いうことだったんですが|/並列節ガ/|

Why does the speaker repeat the same referent next to its previous mention,
although s/he can fairly assume that the it has already been evoked with the first mention?
In fact, the second `Socialist Federal Republic of Yugoslavia' in line b cannot be omitted contrary to what is claimed about the nominal forms \cite{givon83,gundeletal93,ariel90}.
Why?

\begin{table}
	\centering
	\tblcaption{Nth mention vs.\ expression type}
	\begin{tabular}{lrrrrr}
	\lsptoprule
          &  2  & 3   &  4 & 5  & 6+ \\
    \midrule
  NP      & 260 & 135 & 83 & 54 & 255 \\
          & \rt{(74.1\%)} & \rt{(64.9\%)} & \rt{(58.0\%)} & \rt{(52.4\%)} & \rt{(40.5\%)} \\
  Pronoun & 16  & 14  & 9  & 13 & 20 \\
          & \rt{(4.6\%)} & \rt{(6.7\%)} & \rt{(6.3\%)} & \rt{(12.6\%)} & \rt{(3.2\%)} \\
  Zero    & 75  & 59  & 51 & 36 & 355 \\
          & \rt{(21.4\%)} & \rt{(28.4\%)} & \rt{(35.7\%)} & \rt{(35.0\%)} & \rt{(56.3\%)} \\
    \midrule
  Sum     & 351 & 208 & 143 & 103 & 630 \\
%          & \rt{(100\%)} & \rt{(100\%)} & \rt{(100\%)} & \rt{(100\%)} & \rt{(100\%)} \\
    \lspbottomrule
	\end{tabular}
	\label{AnaCountExpTypeT}
%\end{table}
%         2   3   4   5  6+
%  NP   260 135  83  54 255
%  Pron  16  14   9  13  20
%  Zero  75  59  51  36 335
\end{table}

\begin{figure}
	\centering
	\includegraphics[width=0.5\textwidth]{figure/AnaCountExpType.pdf}
	\caption{Nth mention vs.~expression type}
	\label{AnaCountExpTypeF}
\end{figure}

\largerpage[-1]
Since the most frequent \isi{pronoun} in Japanese is the zero \isi{pronoun}, as indicated in Figure \ref{AnaCountExpTypeF} and Table \ref{AnaCountExpTypeT},
the speaker needs to make sure that the \isi{hearer} understands which referent zero pronouns refer to.
Therefore, the speaker needs to establish the referent as a \isi{topic}
before s/he uses zero.%
 \footnote{\chd{
 As pointed out by one of the reviewers (Morimoto),
 it is possible to replace `this Socialist Federal Republic of Yugoslavia' in line b of \ref{WO:TopicAppearClause-Initially:Ex:Yuugo} with a pronoun-like form such as \ci{kono kuni} `this country'.
 My argument here still holds because the pronoun-like form `this country' is
 much more informative than the zero \isi{pronoun}.
 The following argument by \citeA{lambrecht94} also suggests that
 focus can be the \isi{antecedent} of overt pronouns, but not zero pronouns.
 See examples \ref{WO:TopicAppearClause-Initially:Ex:John} and \ref{WO:TopicAppearClause-Initially:Ex:Rosa}.
 }}
This might be related to the observation in \citeA[136]{lambrecht94} that
focus elements cannot be the \isi{antecedent} of zero,
while \isi{topic} elements can.
Compare \Next and \NNext (the acceptability judgements are based on Lambrecht. Information structure is added by the present author).
In \Next, \ci{John} is interpreted as \isi{topic} (by default) in \Next[b],
in which case zero is acceptable.
%
\ex.\label{WO:TopicAppearClause-Initially:Ex:John}
 \a. John married Rosa, but he didn't really love her.
 \b. [John]$_{T}$ [married Rosa]$_{F}$, but {\O} didn't really love her.

On the other hand,
in \Next,
\ci{John} is the focus because it is the answer to the question,
in which case zero is not acceptable, as in \Next[b].
Only an explicit \isi{pronoun} is acceptable, as shown in \Next[a].
%
\ex.\label{WO:TopicAppearClause-Initially:Ex:Rosa}
 \a.[Q:] Who married Rosa?
 \b.[A:]
   \a.[a.] John married Rosa, but he didn't really love her.
   \b.[b.] *?[John]$_{F}$ [married Rosa]$_{T}$, but {\O} didn't really love her.


\begin{figure}
%\begin{minipage}{0.5\textwidth}
	\begin{center}
	\includegraphics[width=0.6\textwidth]{figure/ExpTypePrevWO.pdf}
	\caption{Antecedent's word order of NPs}
	\label{ExpTypePrevWOF}
	\end{center}
%\end{minipage}
\end{figure}
\begin{figure}
%\begin{minipage}{0.5\textwidth}
	\begin{center}
	\includegraphics[width=0.6\textwidth]{figure/ExpTypePrevWOZero.pdf}
	\caption{Antecedent's word order of zero pronoun}
	\label{ExpTypePrevWOZeroF}
	\end{center}
%\end{minipage}
\end{figure}

Why do these pronouns or NPs that refer to the immediate \isi{antecedent} appear (almost) clause-initially?
I argue that, in addition to the from-old-to-new principle \ref{oldnewprinciple},
the persistent-element-first principle works in \isi{spontaneous speech}.
%
\ex. \label{PerFirstPrinciple}\tl{Persistent-element-first principle}:
 In languages in which \isi{word order} is relatively free,
 the unmarked \isi{word order} of constituents is persistent element first and non-persistent element last.

\chd{One of the factors which motivate this principle is  clause-chaining.
In spoken Japanese, a chain of clauses is frequently observed, as schematized in \Next,
where the speaker announces the \isi{topic} at the beginning and continues to talk about it in a chain of multiple clauses.%
 \footnote{
 This is also pointed out by Michinori Shimoji (p.c.) with reference \ili{Ryukyuan} Languages,
 which belong to the same language family as Japanese.
 }
}
%
\ex.
 \a. \fbox{Topic}
 \b. \fbox{Clause1}
 \b. \fbox{Clause2}
 \b. \fbox{Clause3}
 \b. ...

A specific example of clause-chaining is shown in \Next,
where the \isi{topic} `Everest Trail' in line a is preannounced,
and the following clauses (b--f) are about this \isi{topic}.
%
\ex.
\ag. \EM{kono} \EM{eberesuto-kaidoo-toiuno-wa} \\
	this Everest-trail-\ab{quot}-\ci{wa} \\
	`{This Everest Trail} is'
\bg. tibetto-to nepaaru-no kooeki-ro-ni-mo nat-te ori-masi-te \\
	Tibet-\ab{com} Nepal-\ab{gen} trade-road-\ab{dat}-also become-and \ab{prog}-\ab{plt}-and \\
	`also used for trading  between Tibet and Nepal.'
\cg. ma zissai-wa nihon-de iu-to \tp{\dvline} \\
	\ab{fl} actual-\ci{wa} Japan-\ab{loc} say-\ab{cond} \\
	`Say, in Japan for example,'
\dg. \EM{\O} takao-san-mitaina yama-miti-nan-desu-keredomo \\
	{\O} Takao-mountain-like mountain-road-\ab{nmlz}-\ab{cop}.\ab{plt}-though \\
	`it's like a road in Mt.\ Takao or something.'
\eg. genti-no hito$\sim$bito-nitotte-wa ee \EM{\O} tuusyoo-ro-to iu-yoona \\
	local-\ab{gen} person$\sim$\ab{pl}-for-\ci{wa} \ab{fl} {\O} trade-road-\ab{quot} say-like \\
\bg. insyoo-no \EM{\O} miti-desi-ta \\
	 impression-\ab{gen} {\O} road-\ab{cop}.\ab{plt}-\ab{past} \\
 `{it} was like a trading road for local people.'
 \b.[] \hfill{(\code{S01F0151: 105.73-120.14})}
%このエベレスト街道というのは
%チベットとネパールの交易路にもなっておりまして
%ま実際は日本で言うと
%高尾山みたいな山道なんですけれども
%現地の人々にとってはえー通商路というような印象の道でした (S01F0151: 105.73-120.14)

\chd{This pattern is useful because the referent talked about in the chain of clauses in question is referred to at the beginning of the chain and the speaker can use the zero \isi{pronoun} in the following clauses.}

%This principle is motivated by processing;
%it is easier for the \isi{hearer} to process if the \isi{topic} to be talked about is stated first.
%It is easy also for the speaker to say the \isi{topic} first
%because s/he has already activated the \isi{topic} to be talked about,
%but not necessarily how to express the following content.
%	\footnote{
%	This might be the same as the from-old-to-new principle (\ref{oldnewprinciple}).
%	}
Figure \ref{ExpTypePrevWOF} and \ref{ExpTypePrevWOZeroF} show the \isi{word order} of the antecedents of NPs and zero pronouns, respectively.
Although the contrast is subtle,
the antecedents of zero pronouns are more skewed towards earlier positions than NPs.
%The figure is a bar plot of expression types of elements based on the word orders of their antecedents.
%The x-axis indicates the \isi{word order} of the antecedents and y-axis indicates the raw frequencies of elements.

Consider example \Next.
The speaker mentions the \isi{topic} `the participants of the trekking' first in line a,
and expands on this in the following \isi{discourse}.
After \Next[f],
the speaker extends the \isi{topic} and describes each participant.
%
\ex.\label{trekking}
 \ag. e \EM{torekking-sankasya}-nituki-masite-wa \\
 	\ab{fl} trekking-participant-about-\ab{plt}-\ci{wa} \\
	`Concerning the participants of this trekking,'
 \bg. moo hontooni ni-zyuu-go-sai-no ooeru-san-kara \\
 	\ab{fl} really two-ten-five-years.old-\ab{gen} working.woman-\ab{hon}-from \\
	`from the 25-year-old working lady,'
 \bg. nana-zyuu-ni-sai-no ozii-san-made \\
 	seven-ten-two-years.old-\ab{gen} old.guy-\ab{hon}-till \\
	`to the 72-year-old elderly man,'
 \bg. hizyooni takusan-no hito$\sim$bito-ga \\
 	very many-\ab{gen} person$\sim$\ab{pl}-\ci{ga} \\
	`many people...'
 \b. no, not many people,
 \bg. ta-syu-ni wataru nenree-soo-no hito-ga i-te omosirokat-ta-desu \\
 		many-kind-\ab{dat} cover age-tier-\ab{gen} person-\ci{ga} exist-and interesting-\ab{past}-\ab{plt} \\
		`there were many kinds of people from a wide age range and it was interesting.'
		\src{S01F0151: 597.67-610.87}
%S01F0151|00597665L|597.665333|610.870275|L|(F え)トレッキング参加者につきましては(0.192)もう本当に二十五歳の(A オーエル;OL)さんから七十二歳のお爺さんまで(0.278)非常にたくさんの人々が(0.539)(F えー)(0.131)<FV>たくさんのじゃない(0.11)多種に渡る(0.237)年齢層の人がいて面白かったです|[文末]|
%S01F0151|00611451L|611.450993|614.022662|L|で(0.429)みんな人によって趣味が違いまして|/テ節/|
%S01F0151|00614377L|614.37677|626.616677|L|登山から(0.325)(F え)登山が趣味の方もいれば写真を撮るのが趣味の方もいて(0.352)後は(D こせ)高山植物に非常に詳しい方もいれば(0.328)バードウォッチングに詳しくて物凄く立派な双眼鏡を持ってくる方もいたりして|<テ節>|直後がまとめ表現
%S01F0151|00626928L|626.928382|634.442324|L|(F ま)私達はっきり言って何も知らないで参加したので全部人に(0.301)あれは何ですかとか言って聞きながら(0.292)教えていただいて(0.317)登っていました|[文末]|

In this kind of example,
clause-initial elements do not refer to zero pronouns as constituents in the following clauses,
but are only pragmatically associated with the constituents in the following clauses (see also \S \ref{Par:Subj:Ex}).


\begin{table}
 \tblcaption{Antecedent's particle vs.~current expression type}
 \label{ExpTypePrevParT}
\begin{tabular}{lrrrr}
 \lsptoprule
          & \ci{toiuno-wa} & \ci{wa} & \ci{ga} & \ci{o} \\
 \midrule
 NP       & 11             & 38      & 80      & 89 \\
          & \rt{(36.7\%)}  & \rt{(46.3\%)} & \rt{(63.0\%)} & \rt{(74.8\%)} \\
 Pronoun  & 4              & 3       & 5       & 3 \\
          & \rt{(13.3\%)}  & \rt{(3.7\%)} & \rt{(3.9\%)} & \rt{(2.5\%)} \\
 Zero     & 15             & 41      & 42      & 27 \\
          & \rt{(50.0\%)}  & \rt{(50.0\%)} & \rt{(33.1\%)} & \rt{(22.7\%)} \\
 \midrule
 Sum      & 30             & 82      & 127     & 119 \\ 
%          & \rt{(100\%)}   & \rt{(100\%)} & \rt{(100\%)} & \rt{(100\%)} \\
 \lspbottomrule
%           NP Pron Zero
%  toiuno-wa 11    4   15
%  wa        38    3   41
%  ga        80    5   42
%  o         89    3   27
\end{tabular}
\end{table}

\begin{figure}
	\begin{center}
	\includegraphics[width=0.5\textwidth]{figure/ExpTypePrevPar.pdf}
	\caption{Antecedent's particle vs.~current expression type}
	\label{ExpTypePrevParF}
	\end{center}
\end{figure}

Not all clause-initial antecedents of zero pronouns are coded by \isi{topic} markers.
Figure \ref{ExpTypePrevParF} is a bar plot of expression types of elements based on the particles of their antecedents.
According to the figure, the antecedents of zero pronouns are more likely to be coded by \ci{wa} or \ci{toiuno-wa} than
those of overt NPs,
although many antecedents of zero pronouns are coded by \ci{ga} or \ci{o}.

In example \Next,
%\ci{syoo-doobutu} `an small animal' in line b is new 
%%
%\ex.
% \a. When we were having dinner,
% \bg. n mado-no tokoro-to iu-ka beranda-ni nanika \EM{syoo-doobutu-ga} koo tyokotyoko-to ki-ta-n-desu-ne \\
% 	\ab{fl} window-\ab{gen} place-\ab{quot} say-\ab{q} balcony-to something small-animal-\ab{nom} this.way \ab{ono}-\ab{quot} come-\ab{past}-\ab{nmlz}-\ab{cop}.\ab{plt}-\ab{fp} \\
%	`a small animal came near the window, or the balcony.'
% \bg. de saisyo koo ano sotira-no soto-no hoo-kara \EM{\O} nozoi-ta mon-desu-kara \\
% 	then at.first this.way \ab{fl} that-\ab{gen} outside-\ab{gen} direction-from {\O} look-\ab{past} thing-\ab{nmlz}.\ab{cop}-because \\
%	`At first, (it) appeared from the outside, that way, so'
% \bg. watasi-wa saisyo \EM{\O} risu-kana-to omot-ta-n-desu \\
% 	\ab{1}\ab{sg}-\ci{wa} at.first {\O} squirrel-\ab{q}-\ab{quot} think-\ab{past}-\ab{nmlz}-\ab{cop}.\ab{plt} \\
%	`at first I thought (it) was a squirrel.'
% \bg. de tu sat-to koo are-to omot-te it-tara sat-to \EM{\O} nige-tyai-masi-te \\
% 	then \ab{frg} \ab{ono}-\ab{quot} this.way \ab{fl}-\ab{quot} think-and go-then \ab{ono}-\ab{quot} {\O} escape-\ab{pfv}-\ab{plt}-and \\
%	`(I) was wondering and approached (the balcony), then (it) ran away.'
%	\src{S00F0014: 612.71-621.50}
%S00F0014|00612711L|612.710582|621.496514|L|それで(0.344)(F あのー)<FV>(0.176)夜に(0.554)(F あの)食事をこうしてましたら(0.748)(F ん)窓のところと言うかベランダに何か小動物がこうちょこちょこと来たんですね|[文末]|
%S00F0014|00621815L|621.814615|627.660077|L|で最初こう(0.282)(F あの)(0.12)そちらの外の方から覗いたもんですから(0.312)私は最初(0.265)リスかなと思ったんです|[文末]|
%S00F0014|00628000L|627.999519|631.707987|L|で(0.329)(D (? つ))さっとこう(F あれ)(0.235)と思って行ったらさっと逃げ(0.169)ちゃいまして|/テ節/|
clause-initial \ci{waru-gaki} `brats', coded by \ci{ga} in line a, is the \isi{antecedent} of the zero pronoun in line b.
\largerpage
%
\ex.
 \ag. a dokka-no kinzyo-no \EM{waru-gaki-ga} sute-inu-o mi-te \\
		\ab{fl} somewhere-\ab{gen} neighborhood-\ab{gen} bad-brat-\ci{ga} abandon-dog-\ci{o} look-and \\
		`Brats around here found this abandoned dog, and'
 \bg. akai penki-o hana-no ue-ni \EM{\O} nut-ta-n-daroo-to \\
 	red paint-\ci{o} nose-\ab{gen} above-\ab{dat} {\O} paint-\ab{past}-\ab{nmlz}-\ab{infr}-\ab{quot} \\
	`(they) must have painted the dog's nose red.'
 \b. (we) were talking like this.
 \src{S02M0198: 176.26-184.61}
%S02M0198|00176259L|176.258754|184.61063|L|(F あ)どっかの(0.161)近所の悪がきが(0.461)捨て犬を見て(0.599)赤いペンキを(0.263)鼻の上に塗ったんだろうと(0.193)話したんですけども|/並列節ケドモ/|

\largerpage
This might sound \ci{a priori} to some readers because Japanese is traditionally argued to be an SOV language:
of course \ci{ga}-coded elements are subjects and precede other arguments.
However, what I claim is that
the persistent-element-first principle in \ref{PerFirstPrinciple}, in addition to the from-old-to-new principle in \ref{oldnewprinciple}, is one of the reasons why so-called subjects (A and S) precede other arguments.
%As \citeA{lambrecht94} points out, topics tend to be outside of a clause, i.e., they do not belong to the arguments of a clause as in \Next.
%In \Next[a],
%the expression \ci{the African elephant} is not an argument of the predicate \ci{fan};
%\ci{fan} has A as \ci{he} and P as \ci{himself}.
%Similarly, in \Next[b],
%\ci{the typical family today} is also non-argument;
%the argument of the \isi{verb} \ci{work} is \ci{the husband and the wife}.
%%
%\ex.
% \a. (Six year old girl, explaining why the African elephant has bigger ears than the Asian elephant)
% 	\EM{The African elephant}, it's so hot there, so he can fan himself.
% \b. (From a TV interview about the availability of child care)
% 	That isn't the typical family anymore.
%	\EM{The typical family today}, the husband and the wife both work.
%	\hfill{\cite[][p.~193 (emphasis added)]{lambrecht94}}
%
%In the same way,
%`the participants of the trekking' in (\ref{trekking}) is not an argument of the clause (\ref{trekking}f).
%The argument of the \isi{verb} \ci{i(ru)} `exist' is \ci{hito} `person'.

Another motivation has been proposed for clause-initial \isi{topic}s repeated immediately after the first mention.
%Another motivation has been pointed out for topic elements immediately repeated clause-initially.
\citeA{dennakagawa13} discuss cases where clause-initial topics are used as fillers.
Since topics have already been evoked in the speaker's mind, the cost of producing topics is lower than that of producing new elements.
While the speaker utters the \isi{topic},
s/he plans the following \isi{utterance}.
\citeA{dennakagawa13} investigated conversations and found that the \isi{topic} elements repeated immediately after the previous speaker's \isi{utterance}
complementarily distribute with fillers.
They also found that the length of the final mora of the \isi{topic} phrase (typically \ci{wa}) correlates with the length of the following \isi{utterance}
\cite[see also][]{watanabeden10}.
%This might be another motivation for topics being outside of a clause.
In the following example \Next,
not only is `Serbian people' repeated twice in line a and b, almost the whole sentence is repeated;
the sentences in line a and b convey almost the same proposition.
This is another piece of evidence that supports Den \& Nakagawa's claim;
while repeating almost the same proposition,
the speaker can plan what to say next about this \isi{topic}.
%
\largerpage
\ex.
 \ag. sono \EM{serubia-zin-no} \EM{kata-tati}-ga soko-ni-wa ma hazimete ee serubia-teekoku-toiu kokka-o tukuru-no-ga maa zyuu-ni-seeki-no ma owari-gurai-nan-desu-ga \\
 	that Serbia-people-\ab{gen} person.\ab{plt}-\ab{pl}-\ci{ga} there-\ab{dat}-\ci{wa} \ab{fl} first.time \ab{fl} Serbia-empire-\ab{quot} nation-\ci{o} make-\ab{nmlz}-\ci{ga} \ab{fl} ten-two-century-\ab{gen} \ab{fl} end-around-\ab{nmlz}-\ab{cop}.\ab{plt}-though \\
	`Those Serbian people built a nation called the Serbian Empire towards the end of the eleventh century.'
 \bg. ee kono ziki maa \EM{serubia-no} \EM{kata-tati}-ga maa koko-ni tu kokka-o tukut-te ee serubia-teekoku-toiu koto-de \\
 	\ab{fl} this time \ab{fl} Serbia-\ab{gen} person.\ab{plt}-\ci{ga} \ab{fl} here-\ab{dat} \ab{frg} nation-\ci{o} make-and \ab{fl} Serbia-empire-\ab{quot} thing-\ab{cop}.and \\
	`Around this time Serbian people built a nation, this is the Serbian Empire and'
 \bg. ee ryuusee-o \EM{\O} kiwame \\
 		\ab{fl} flourish-\ci{o} {\O} be.extreme \\
	`(it) flourished.'
 \b. At that time Catholics were coming from the north, and from the south, the Greek Orthodox were coming,
 \b. though they are both Christian,
 \bg. ee ni-keetoo-no syuukyoo-no naka-de seekatu-o \EM{\O} si-te-iku naka-de \\
 		\ab{fl} two-stream-\ab{gen} religion-\ab{gen} inside-\ab{loc} life-\ci{o} {\O} do-and-go inside-\ab{loc} \\
		`While (they) were living surrounded by two streams of religion,'
 \bg. ee serubia-teekoku-tosite ma dotira-o erabu-ka-tteiu na ko ee koto-no naka-de \\
 	\ab{fl} Serbia-empire-as \ab{fl} which-\ci{o} choose-\ab{q}-\ab{quot} \ab{frg} \ab{frg} \ab{fl} thing-\ab{gen} inside-\ab{cop}.and \\
	`(they) faced the question of which one to choose.'
 \bg. ee ma minami-gawa-no girisya-seekyoo-o \EM{\O} toru wake-nan-desu-ga \\
 	\ab{fl} \ab{fl} south-side-\ab{gen} Greek-Orthodox-\ci{o} {\O} choose reason-\ab{nmlz}-\ab{cop}.\ab{plt}-though \\
	`(They) eventually chose the Greek Orthodox.'
	\src{S00M0199: 212.34-221.02}
%S00M0199|00212336L|212.336343|221.023973|L|そのセルビア人の方達がそこには(F ま)初めて(0.163)(F えー)セルビア帝国という(0.373)(F えー)国家を作るのが(F まー)十二世紀の(0.436)(F ま)終わりぐらいなんですが|/並列節ガ/|
%S00M0199|00221455L|221.455248|229.24769|L|(F えー)この時期(F まー)(0.976)セルビアの方達が(F まー)ここに(D つ)国家を作って(0.151)(F えー)セルビア帝国ということで(0.266)(F えー)隆盛を極め|/連用節/|
%S00M0199|00229652L|229.651772|254.15569|L|(F えー)そしてその後(F まー)(F あのー)(F まー)その当時宗教としては(0.337)(F えー)北からは(F えー)(0.13)(D く)(F ま)キリスト教のカトリック(0.128)(F えー)南からは(0.372)(F えー)(F い)ギリシャ正教ということで(F えー)二つの(F えー)(F ま)同じキリスト教ですが(F えー)二系統の宗教の中で(0.327)生活をしていく中で(0.391)(F えー)セルビア(0.177)帝国として(F ま)どちらを選ぶかっていうな(0.212)(D こ)(F えー)ことの(D ん)中で(F えー)(0.14)(F ま)(0.561)南側のギリシャ正教を取る訳なんですが|/並列節ガ/|


%%----------------------------------------------------
\subsection{Summary of clause-initial elements}

This section investigated characteristics of clause-initial elements.
It turned out that
shared and persistent elements tend to appear clause-initially.
Not only did this study confirm the classic observation that
topics tend to appear clause-initially,
this section and the next  analyze what kind of topics appear clause-initially.
I also discussed motivations for clause-initial topics.


%%----------------------------------------------------
%%----------------------------------------------------
\section{Post-predicate elements}\label{WOPostPreEles}

While Japanese is reported to be a \isi{verb-final language} \cite{hinds86,shibatani90},
some elements appear after the \isi{verb} in spoken Japanese \cite{kuno78,onosuzuki92,fujii95,takami95a,takami95b,ono06,nakagawaetal08_paper}.
The following are examples of post-predicate elements.
Since post-predicate elements are very rare in monologues,
the examples are from the dialogue part of CSJ.
\ci{Kono hito} `this person' in \Next and \ci{terii itoo} `Terry Ito (a person's name)' in \NNext are produced after the predicates \ci{yat} `do' and \ci{kake} `wear', respectively.
%
\ex.
\ag.[R:] nani \EMi{yat}-teru-no \EM{kono} \EM{hito} \\
 		what do-\ab{prog}-\ab{nmlz} this person \\
		`What is (he) doing, this person?'
		\src{D02F0028: 193.30-194.45}

\ex.\label{D02F0015_TerryIto}
 \ag.[L:] sangurasu-toka \EMi{kake}-te-masu-yo-ne \EM{terii} \EM{itoo-tte} \\
		sunglasses-\ab{hdg} wear-\ab{prog}-\ab{plt}-\ab{fp}-\ab{fp} Terry Ito-\ab{quot} \\
		`(He) is wearing sunglasses, isn't he, Terry Ito?'
		\src{D02F0015: 359.17-362.42}

This section investigates the \isi{information structure} of post-predicate constructions of this kind.
Although post-predicate expressions could be adverbs, connectives, and other adjuncts,
this study examines only noun phrases.
%I call all the elements which appear before the predicate
%``elements before the predicate'' or ``preposed elements''
%as opposed to ``post-predicate'' or ``postposed'' elements.
%I keep the term ``pre-posed'' elements
%to indicate elements which appear immediately before the predicate
%(to be discussed in \S \ref{WOPrePredEles}).

%%----------------------------------------------------
\subsection{Strongly evoked elements appear after the predicate}\label{WORdis}

\citeA[][136]{takami95a} argues that
postposed elements are elements other than the focus.
For example,
the answer to a question or \ci{wh}-phrase cannot be postposed naturally.
\Next is an example of a \isi{postposed element}, `a 10-carat diamond ring', as the answer to a `what' question .
While the sentence itself is natural,
the \isi{postposed element} cannot felicitously answer the question.
%
\ex.
 \a.[Q:] What did Taro buy for Hanako?
 \bg.[A:] \#taroo-wa hanako-ni kat-te yat-ta-yo \EM{zyuk-karatto-no} \EM{daiya-no} \EM{yubiwa-o} \\
 		Taro-\ci{wa} Hanako-for buy-and give-\ab{past}-\ab{fp} 10-carat-\ab{gen} diamond-\ab{gen} ring-\ci{o} \\
		`Taro bought (it) for Hanako, a 10-carat diamond ring.'

Similarly,
\ci{wh}-phrases such as \ci{dore} `which' cannot be postposed, as shown in \Next.
\exg. *itiban oisii-desu-ka \EM{dore-ga}? \\
		most delicious-\ab{cop}.\ab{plt}-\ab{q} which-\ci{ga} \\
		`The most delicious one, which?'


%As has been pointed out in \citeA{onosuzuki92,takami95b,ono06,nakagawaetal08_paper},
%topics can be postposed.
\citeA{nakagawaetal08_paper} found that
there are two types of \isi{post-predicate construction}:
the single-contour type and the double-contour type.
In the single-contour type, the post-predicate elements are uttered without a pause and do not have the F$_{0}$ peak. In the double-contour construction, on the other hand, post-predicate elements are uttered with a pause and have the F$_{0}$ peak.
The \isi{pitch} contours of each \isi{utterance} are shown in Figure \ref{kome1F} for the single-contour type (\Next[A] and \NNext[A]) and \ref{kome2F} for the double-contour type (\Next[A$^{\prime}$] and \NNext[A$^{\prime}$]),
both of which were produced by the author.
The post-predicate part is \ci{kome-wa} `rice-\ci{wa}',
whose accent nucleus is on \ci{me} and whose overall accent is supposed to be LHL (L indicates low and H indicates high in \isi{pitch}).
In Figure \ref{kome1F}, where the \isi{postposed element} is uttered with the same continuous contour as the \isi{main clause},
one can neither observe the F$_{0}$ peak in \ci{me} nor a pause between the predicate and the \isi{postposed element}.
In Figure \ref{kome2F}, on the other hand,
where the \isi{postposed element} is uttered in a separate contour from the \isi{main clause},
one can observe the F$_{0}$ peak in \ci{me} and a pause between the predicate and the \isi{postposed element}.

\citeA{nakagawaetal08_paper} investigated the difference between these two construction types in terms of \isi{information structure} and found that
the post-predicate elements of the single-contour type are evoked
by being mentioned immediately before or through physical context.
On the other hand,
the post-predicate elements of the double-contour type are not necessarily evoked.
For example, compare examples \Next and \NNext,
where the bold-faced letters indicate that
they are high in \isi{pitch}.%
	\footnote{
	Here I assume that the \isi{pitch accent} of \ci{oisii} `good' is LHHH
	and that that of \ci{kome-wa} `rice-\ci{wa}' is LHL.
	}
The referent `rice' in \Next is evoked
because it is mentioned in \Next[Q] immediately before the answer to Q is uttered.
In this case,
\Next[A$^{\prime}$],
where the \isi{post-predicate element} \ci{kome-wa} `rice-\ci{wa}' has its own F$_{0}$ peak and is preceded by a pause,
is not acceptable,
while \Next[A],
where the \isi{post-predicate element} without its own F$_{0}$ peak is uttered immediately after the predicate without a pause,
is acceptable.
%
\ex. \tl{The referent `rice' evoked}
 \a.[Q:] I don't like rice.
	\bg.[A:] o\pk{isii}-yo \ul{kome-wa} \\
			good-\ab{fp} rice-\ci{wa} \\
	\bg.[A$^{\prime}$:] ?o\pk{isii}-yo, \ul{ko\pk{me}-wa} \\
			good-\ab{fp} rice-\ci{wa} \\
			`RICE is good (but other things are not).'
			\hfill{\cite[][7]{nakagawaetal08_paper}}

On the other hand,
in \Next, where `rice' is not evoked before the speaker utters \Next[A] or \Next[A$^{\prime}$],
only the double-contour type \Next[A$^{\prime}$] is acceptable
and the single-contour type \Next[A] is not natural.
\ex. \tl{The referent `rice' not evoked}
 \a.[Q:] Is that sushi bar good?
	\bg.[A:] ??o\pk{isii}-yo \ul{kome-wa} \\
			good-\ab{fp} rice-\ci{wa} \\
	\bg.[A$^{\prime}$:] o\pk{isii}-yo, \ul{ko\pk{me}-wa} \\
			good-\ab{fp} rice-\ci{wa} \\
			`RICE is good (but other things are not).'
			\hfill{(ibid.)}

A remaining issue is
to investigate the difference between elements before and after the predicate in terms of \isi{information structure}.

\begin{figure}
%\begin{minipage}{0.5\textwidth}
	\begin{center}
	\includegraphics[width=0.6\textwidth]{sounds/kome1.pdf}
	\caption{Post-predicate construction: single-contour type}
	\label{kome1F}
	\end{center}
%\end{minipage}
\end{figure}
\begin{figure}
%\begin{minipage}{0.5\textwidth}
	\begin{center}
	\includegraphics[width=0.6\textwidth]{sounds/kome2.pdf}
	\caption{Post-predicate construction: double-contour type}
	\label{kome2F}
	\end{center}
%\end{minipage}
\end{figure}

\citeA{nakagawaetal08_paper} measured the referential distance (RD) between post-predicate elements and their antecedents,
i.e., they measured the number of inter-pausal units between the element in question and its \isi{antecedent}.
They modified the definition of RD from the original one \cite{givon83} and decided to use inter-pausal unit as a measure of RD,
since clause boundaries are sometimes difficult to identify in spoken Japanese.
Their results are shown in Table \ref{RDPostT}.
The table shows that the average RD of the post-predicate elements of the single-contour type is 6.9,
whereas that of the double-contour type is 39.7.
What about elements before the predicate?

I conducted the same investigation for elements before the predicate, but this time I used the monologues employed throughout this study
because the dialogues Nakagawa and her colleagues used in their study lack the information about the RD of elements before the predicate.%
	\footnote{
	\citeA{nakagawaetal08_paper} counted the RD of non-\isi{anaphoric} elements as 100 (the maximum value of RD),
	but the present study did not include non-\isi{anaphoric} elements,
	since I thought that this is ad hoc.
	This modification makes the RD of elements before the predicate (conducted in this study) smaller.
	This has only a small effect and the overall conclusion does not change because
	according to our result,
	the RD of pre-predicate elements are larger than that of post-predicate elements;
	if this study employed the same criteria as Nakagawa et al.,
	the RD of elements before the predicate would be expected to be even larger.
	}
Further studies are needed to make sure that elements before the predicate in monologues and dialogues have the same characteristics.
Table \ref{RDPreT} shows the average RDs of elements before the predicate based on their \isi{word order}.
Here, I simplified \isi{word order} to only count arguments (excluding fillers, fragments, adverbs, adjectives, etc.).
\code{1} indicates that the element in question is the first argument in a clause,
\ci{2} indicates that it is the second argument, and so on.
The RD of the first argument is 20.9 on average,
that of the second argument is 23.0, and
the third argument is 41.1.
%the fourth is 70.5.
The table shows that the RD of elements before the predicate are larger than that of postposed elements of the single-contour type, regardless of their word order.
The RD of double-contour postposed elements is similar to that of preposed elements in the third position.
I do not have an explanation for the RD of double-contour postposed elements.
I believe that postposed elements of the double-contour type are heterogeneous;
some might be an afterthought,
some might have \isi{interactional} functions \cite{ono07},
while others might be something else (\citeA{tanaka05,kakuden12}, see also the discussion in \S \ref{WO:PostP:Motiv:Double}).
What I want to emphasize here is that the RD of the single-contour postposed elements is smaller than that of elements before the predicate.
The postposed elements of the single-contour type are evoked when they are uttered;
their \isi{activation cost} is low.
Taking into consideration the fact that
many of the post-predicative elements are pronouns or nouns preceded by demonstratives \cite{nakagawaetal08_paper},
I propose that post-predicative elements are often strongly evoked.
On the other hand, the \isi{activation cost} of preposed elements is higher than that of postposed elements.%
 \footnote{
 The average RD of zero pronouns is 5.0,
 which shows that post-predicate elements of the single-contour type is
 close to zero pronouns.
 }

\begin{table}
%\begin{minipage}{0.5\textwidth}
 \centering
 \caption{RD of post-predicate elements}
 \begin{tabular}{lrr}
 \lsptoprule
   & Single-contour & Double-contour \\
 \midrule
  RD & 6.9 & 39.7 \\
 \lspbottomrule
 \end{tabular}
 \label{RDPostT}
% \end{minipage}
\end{table}
\begin{table}
% \begin{minipage}{0.5\textwidth}
  \centering
 \caption{RD of elements before predicate}
 \begin{tabular}{lrrrr}
 \lsptoprule
  &  1  & 2 & 3 \\
 \midrule
 RD & 20.9 & 23.0 & 41.1 \\
 \lspbottomrule
 \end{tabular}
 \label{RDPreT}
% \end{minipage}
\end{table}

The following are examples of post-predicate constructions from dialogues.
\Next and \NNext are examples of the single-contour type.
The postposed elements of this construction are typically pronouns or elements modified by the  demonstratives \ci{kono} `\ab{dem}.\ab{prox} (this)', \ci{sono} `\ab{dem}.\ab{med} (this/that)', or \ci{ano} `\ab{dem}.\ab{dist} (that)'.
In \Next,
the \isi{postposed element} is the \isi{pronoun} \ci{kore} `\ab{dem}.\ab{prox} (this)'.
The participants are working on a task about ranking famous people based on how much they earn.
The \isi{utterance} is produced in the middle of this task and
the \isi{demonstrative} \ci{kore} refers to the ranking so far.
Therefore, the referent of \ci{kore} is expected to be evoked in the participants' mind.
As shown in Figure \ref{D02F0025_sugoi_tatakaiF},
where the upper box indicates the intensity of the \isi{utterance}
and the lower box indicates the F$_{0}$,
the \isi{postposed element} \ci{kore} does not have an F$_{0}$ peak.
%
\ex. \label{D02F0025_sugoi_tatakai}
	\ag.[L:] sugoi tatakai-da-yo-ne \EM{kore} \\
	awful battle-\ab{cop}-\ab{fp}-\ab{fp} this \\
	`(It) is an awful battle, this?'
		\src{D02F0025: 463.93-465.81}

In \Next,
where the participants are involved in the same task as \Last,
\ci{kono hito} `this person' is the famous person under discussion right now -- hence the referent is evoked in the participants' mind.
Figure \ref{D02M0028_konohitoF} shows the intensity and the F$_{0}$ of the \isi{utterance} in \Next.
Although the F$_{0}$ of the \isi{postposed element} is not shown because the speaker's spoke quietly,
the intensity tells us that
the \isi{postposed part} is uttered without a pause.
Also, the fact that the intensity is low indicates that the \isi{postposed element} is only weakly uttered because the referent is sufficiently evoked.
%
\ex.\label{D02M0028_konohito}
	\ag.[R:] nani yat-teru-no \EM{kono} \EM{hito} \\
			what do-\ab{prog}-\ab{nmlz} this person \\
			`What is (he) doing, this person?'
		\src{D02M0028: 193.30-194.45}

Common nouns can also be postposed elements of the single-contour type, as shown in \Next.
In \Next, where the participants are involved in the same task,
the \isi{postposed element} \ci{syasin} `photo' is uttered without a pause or F$_{0}$ peak, as shown in Figure \ref{D02F0015_syasinF}.
Since R, the other participant, is physically holding the photos and this is part of the rules of the task,
it is reasonable to assume that the participants have already evoked the photos.
%
\ex.\label{D02F0015_syasin}
	\ag.[L:] siro-kuro-desu-ka \EM{syasin} \\
		white-black-\ab{cop}.\ab{plt}-\ab{q} photo \\
		`Are (they) black-and-white, the photos?'
		\src{D02F0015: 313.95-315.26}

\begin{figure}
%\begin{minipage}{0.5\textwidth}
	\begin{center}
	\includegraphics[width=0.6\textwidth]{sounds/D02F0025_sugoi_tatakai.pdf}
	\caption{Intensity and F$_{0}$ of the single-contour type \ref{D02F0025_sugoi_tatakai}}
	\label{D02F0025_sugoi_tatakaiF}
	\end{center}
%\end{minipage}
\end{figure}
\begin{figure}
%\begin{minipage}{0.5\textwidth}
	\begin{center}
	\includegraphics[width=0.6\textwidth]{sounds/D02M0028_konohito.pdf}
	\caption{Intensity and F$_{0}$ of the single-contour type \ref{D02M0028_konohito}}
	\label{D02M0028_konohitoF}
	\end{center}
%\end{minipage}
\end{figure}
\begin{figure}
%\begin{minipage}{0.5\textwidth}
	\begin{center}
	\includegraphics[width=0.6\textwidth]{sounds/D02F0015_syasin.pdf}
	\caption{Intensity and F$_{0}$ of the single-contour type \ref{D02F0015_syasin}}
	\label{D02F0015_syasinF}
	\end{center}
%\end{minipage}
\end{figure}


On the other hand, postposed elements in the double-contour type have not been sufficiently evoked or they are contrastive at the time of \isi{utterance}.
In \Next, where the participants are again involved in the task of ranking famous people based on their income,
\ci{kotti-wa} `on my side' is uttered in a separate contour from the \isi{main clause}, and there is a pause between the \isi{main clause} and the \isi{postposed element}, as shown in Figure \ref{D02F0015_kottiwaF}.
`On my side' is necessary information in the sense that
the other participant, L, was talking about how many people were listed on her own side.
Therefore, participant R might have thought that `there are ten people' is not enough and added `on my side' later.
The F$_{0}$ peak of the postposed element \ci{kotti-wa} `on my side' is still lower than \ci{zyuu} `ten' in the \isi{main clause},
and the intensity is also lower.
This is because the \isi{postposed element} is not the focus, as \citeA{takami95a,takami95b} has pointed out.
Foci are typically new in the given-new taxonomy and need both an F$_{0}$ peak and intensity in order for the \isi{hearer} to understand clearly what is said.
%
\ex.\label{D02F0015_kottiwa}
 \a.[L:] There are eleven people (listed on my side).
 \bg.[R:] zyuu-nin-desu \EM{kotti-wa} \\
 		ten-people-\ab{cop}.\ab{plt} this.side-\ci{wa} \\
		`There are ten people on my side.'
	\src{D02F0015: 3.27-9.03}

%Similarly in \Next,
%where the participants are involved in the same task,
%the participants were talking about how the compensation system works in the show business and .
%%
%\ex.\label{D02F0015_geenookai}
% \ag.[L:] kore-wa dan-zyo-no sa-wa nai-n-desu-ka-ne geenookai-tte \\
% 		this-\ci{wa} male-female-\ab{gen} difference-\ci{wa} not.exist-\ab{nmlz}-\ab{cop}.\ab{plt}-\ab{q}-\ab{fp} show.business-\ab{top} \\
%		`Isn't there difference between men and women (income), the show business?'
%	\src{D02F0015: 890.51-893.82}

\newpage
In \Next, L is interviewing R about her study on differences among Japanese dialects.
R utters `eastern area' in a separate contour from the predicate because this is the only area where she found no differences between smaller areas (prefectures) when comparing different dialects.
%because R compares different dialects and, only in the eastern area, did she find no differences among smaller areas (prefectures).
Therefore `the eastern area' is contrasted with other areas.
In this case, the F$_{0}$ peak and the intensity of the \isi{postposed element} are as high as those of the \isi{main clause},
as shown in Figure \ref{D04F0050_kantooF}.
%
\ex.\label{D04F0050_kantoo}
 \ag.[R:] kooiu sa-ga aru-ne-tte iu-koto-wa ie-nai zyootai-desi-ta-ne \EM{kantoo-no} \EM{hoo-wa} \\
 	such.and.such difference-\ci{ga} exist-\ab{fp}-\ab{q} say-thing-\ci{wa} say-\ab{neg} situation-\ab{cop}.\ab{plt}-\ab{past}-\ab{fp} east-\ab{gen} direction-\ci{wa} \\
	`One cannot say that there is such and such difference, (in the) eastern area.'
	\src{D04F0050: 338.54-349.27}

\begin{figure}
%\begin{minipage}{0.5\textwidth}
	\begin{center}
	\includegraphics[width=0.6\textwidth]{sounds/D02F0015_kottiwa.pdf}
	\caption{Intensity and F$_{0}$ of the double-contour type \ref{D02F0015_kottiwa}}
	\label{D02F0015_kottiwaF}
	\end{center}
%\end{minipage}
\end{figure}
\begin{figure}
%\begin{minipage}{0.5\textwidth}
	\begin{center}
	\includegraphics[width=0.6\textwidth]{sounds/D04F0050_kantoo.pdf}
	\caption{Intensity and F$_{0}$ of the double-contour type \ref{D04F0050_kantoo}}
	\label{D04F0050_kantooF}
	\end{center}
%\end{minipage}
\end{figure}



%%----------------------------------------------------
\subsection{Motivations for topics to appear post-predicatively}\label{WO:PostP:Motivations}

It has been pointed out that
topics or given elements tend to appear clause initially \cite{mathesius28,firbas64,danes70}.
%Why do they also appear post-predicatively?
What are the motivations for them to appear post-predicatively?
In this section I mainly discuss the post-predicate elements of the single-contour type in comparison with the elements before the predicate.
Elements of the double-contour type are heterogeneous, as discussed above, and need further investigation.


%%----------------------------------------------------
\subsubsection{Low activation cost and general characteristics of intonation units}\label{WO:PostP:Motivations:IU}

Before getting directly into the question of
why some topics appear post-pred\-i\-cat\-ively, %
%%%ISSUE "post-predicatively" does not allign properly.
let us begin with the question of why some topics do not appear clause-initially.
As discussed in \S \ref{GivenAppearClause-Initially} and this section,
the \isi{activation cost} of preposed topics is higher than
the \isi{activation cost} of postposed topics and zero pronouns.
The \isi{low activation cost} of post-predicate elements suggests that
they are not anchors to the previous \isi{discourse};
since they are already sufficiently evoked,
they do not have to relate to the previous context and the current \isi{utterance}.
Therefore, they have a motivation for not appearing clause-initially.
Why do they appear post-predicatively?

I argue that the element whose \isi{activation cost} is low tends to appear post-predicatively
because in Japanese and many other languages
an \isi{intonation unit} starts from a high F$_{0}$ and gradually declines toward the end 
\cite{libermanpierrehumbert84,cruttenden86,duboisetal93,chafe94,prieto96,truckenbrodt04,denetal10}.
Since the elements with \isi{low activation cost} do not require a high F$_{0}$,
their preferred position is toward the end of the \isi{intonation unit}.
This kind of phenomenon has already been reported in \ili{Siouan}, Caddoan, and \ili{Iroquoian} languages of North America \cite{mithun95}.
In these languages,
this newsworthy-first (i.e., given-last) \isi{word order} is fully grammaticalized, and Mithun proposes the hypothesis that the given-last \isi{word order} comes from right-detachment constructions, i.e., the postposed constructions discussed in this section.
She argues that this \isi{word order} is motivated by the general tendency of intonation units to form a high F$_{0}$, which gradually declines.
This tendency of intonation units is physiologically motivated,
as \citeA{cruttenden86} discusses:
%
\begin{quote}
The explanation for declination has often been related to the decline in transglottal pressure as the speaker uses up the breath in his lungs.
A more recent explanation suggests that an upward change of \isi{pitch} involves a physical adjustment which is more difficult than a downward change of \isi{pitch},
the evidence being that a rise takes longer to achieve than a fall of a similar interval in fundamental frequency.
\cite[][168]{cruttenden86}
\end{quote}
%

Moreover, \citeA[][89]{comrie89} argues that unstressed constituents such as \isi{clitic} pronouns are cross-linguistically ``subject to special positioning rules only loosely, if at all, relating to their \isi{grammatical relation}'';
therefore, he argues that ``sentences with pronouns can be discounted in favour of those with full noun phrases''.
%I believe that this tendency is also the case not only with pronouns but also given full NPs in Japanese
%because, in Japanese, pronouns are not very frequent and full NPs are frequently used to refer to activated referent and they are frequently unstressed \cite[\S 6.3]{venditti00}.
Arguing against the hypothesis \cite{givon79}
that one can reconstruct the ancient \isi{word order} of a language based on \isi{pronominal} affixes and clitics,
Comrie suggests that the order of these elements in a clause is more likely to be influenced by stress rhythm properties \cite[][218]{comrie89}.

I argue that the order of Japanese unstressed pronouns and NPs is also affected by phonetic constraints, as Comrie suggests.
As will be discussed in Chapter \ref{Intonation}, some unstressed pronouns and NPs referring to highly evoked entities
have a decrease in \isi{pitch} peak and are produced only in \isi{low pitch}.
%As will be discussed in Chapter \ref{Intonation}, some unstressed pronouns and NPs referring to highly evoked entities lose the \isi{pitch} peak and are produced only in \isi{low pitch}.
However, an accent rule in Japanese forbids lexical items starting with two \isi{low pitch} morae in a row.
Therefore, the best position for unstressed items is the sentence-final or \isi{post-predicate position},
in which unstressed items are allowed.
For a phonetic analysis of unstressed items,
see Chapter \ref{Intonation}.


%%----------------------------------------------------
\subsubsection{Why the post-predicate construction mainly appears in dialogue and the source of its ``emotive'' usage}

The declination of F$_{0}$ does not fully explain post-predicate constructions in Jap\-a\-nese. %
%%%ISSUE "Japanese" out of alignment.
The discussion above does not explain why the Japanese \isi{post-predicate construction} mainly appears in dialogues, but not in monologues.
Moreover, Japanese post-predicate constructions are reported to have ``\isi{emotive}'' characteristics \cite{ono07}.
As examples for \isi{emotive} characteristics of post-predicate constructions, consider the following constructed example.
Let us assume that a boy gave a present to his girlfriend.
The girl happily received the gift and opened it.
After seeing the gift, say a banana case,%
	\footnote{
	Bananas of all sizes can fit into this banana case.
	}
she uttered \Next or \NNext.
Since the most frequent \isi{word order} in Japanese is predicate-final,
the canonical order is the one in \Next  , whereas
\NNext can be regarded as a \isi{post-predicate construction}.
%
\exg.\label{korenani}kore nani \\
	this what \\
	`What's this?'
	\hfill{(Canonical \isi{word order})}

\exg.\label{nanikore}nani kore \\
	what this \\
	`What's this (weird thing)?'
	\hfill{(Post-predicate construction)}

These two utterances consist of the same constituents \ci{kore} `this' and \ci{nani} `what'.
As was pointed out in \citeA{onosuzuki92} and \citeA{ono07},
however, the implicatures of these two are different.
In \LLast,
she simply does not know what she received,
probably because she has never seen it before.
By contrast, in \Last,
she knows what she received (it's a banana case) but she did not like it, as we expected.
\chd{In other contexts, \Last can be used to express the speaker's surprise,
excitement, etc.
However, \Last can never be a neutral question.}
Where does this \isi{implicature} come from?

Since these two utterances consist of exactly the same elements,
it is obvious that the \isi{implicature} in \Last cannot be derived from the meaning of each constituent. In this study, I propose that two factors are involved in the questions why post-predicate constructions mainly appear in dialogues and what the source of this ``\isi{emotive}'' usage is:
\isi{word order} and intonation.

Firstly, I discuss why the \isi{post-predicate construction} appears mainly in dialogues.
My point is that,
since the intonation-unit-final position is a position for expressions with \isi{interactional} functions,
the \isi{post-predicate element} (of the single-contour type) plays some \isi{interactional} role.
As has traditionally been argued \cite[e.g.,][]{watanabe71},
the \isi{post-predicate position} is for interaction in Japanese.
\citeA{iwasaki93} extended this argument and claimed that
in fact the intona\-tion-unit-final position is the position for interaction;
the \isi{post-predicate position} is only one example of this intonation-unit-final position.
Consider the following example.
Each line corresponds to a single \isi{intonation unit}.
The lines a, b, and c end with the \isi{interactional} markers \ci{ne} and \ci{sa},
which is indicated by \EM{IT}.
As the examples in \Next show,
these \isi{interactional} markers appear IU-finally.%
	\footnote{
	IT stands for ``\isi{interactional} component'',
	one of the four component types in an \isi{intonation unit}.
	Other types are:
	LD (lead component (e.g., fillers)),
	ID (ideational component), and
%	SU (subjective component), and
	CO (cohesive component).
	The order of an \isi{intonation unit} is proposed to be
	LD ID CO IT in Japanese \cite[][44]{iwasaki93}.
	}
%
\ex.
 \a.
 \glll sooiu sito-ga siki si-te-\EM{ne} \\
 	such person-\ci{ga} lead do-and-\ab{fp} \\
	ID ID ID ID-CO-\EM{IT}	 \\
	\glt `Such people led, and'
 \b.
 \glll sinin-o asoko-e minna-\EM{ne} \\
 		corpses-\ci{o} there-\ab{dir} all-\ab{fp} \\
		ID ID ID-\EM{IT} \\
%	\glt `'
 \b.
 \glll ano dote-no ue-e-\EM{sa} \\
 		that bank-\ab{gen} top-\ab{dir}-\ab{fp} \\
		ID ID ID-ID-\EM{IT} \\
%	 \glt `'
 \glll atsume-te \\
 		gather-and \\
		ID-CO \\
	\glt `gathered dead bodies on top of that bank...'
	\hfill{\cite[][47, gloss and transcription modified by the current author]{iwasaki93}}

As \citeA{morita05} suggests,
a general function of \isi{interactional} particles such as \ci{ne} and \ci{sa} is ``to foreground a certain stretch of talk as an `interactionally relevant unit' to be operated on
-- whether that unit is itself a whole \isi{utterance} or merely one particular component of that \isi{utterance}'' (p.~92).
Since post-predicate elements follow these \isi{interactional} particles within the same \isi{intonation unit} -- as in \ref{D02F0015_TerryIto} and \ref{D02F0025_sugoi_tatakai},
where the post-predicate elements follow \ci{ne} -- they are also expected to have some \isi{interactional} function.
\citeA{kakuden12} report that
77.6\% of post-predicate constructions have \isi{interactional} particles of this kind after the predicate,
whereas only 47.0 \% of non-post-predicate constructions have \isi{interactional} particles.
This also suggests that post-predicate constructions are related to
some \isi{interactional} characteristics.
Further investigation is necessary to uncover what kind of \isi{interactional} functions they have, possibly employing conversational analysis.

Secondly, I argue that  the source of the ``\isi{emotive}'' \isi{implicature} of \ref{nanikore} in contrast with \ref{korenani} comes from an intonational constraint on the \isi{post-predicate element}.
In Japanese, \ci{wh}-questions can optionally be uttered with \isi{rising intonation}.
%%% 要文献
However, the \isi{post-predicate element} is always falling and the \isi{rising intonation} is not natural.
Figure \ref{nanikoreF} shows the \isi{pitch contour} of the \isi{utterance} \ci{nani kore} `what's this (weird thing)?' \ref{nanikore},
while Figure \ref{korenaniF} shows the \isi{pitch contour} of the neutral order \ci{kore nani} `what's this?' \ref{korenani}.
As indicated in the figures, the neutral \isi{word order} \ref{korenani} in Figure \ref{korenaniF} is uttered with \isi{rising intonation},
and I believe that this is the most frequent intonation,
whereas the \isi{post-predicate construction} \ref{nanikore} in Figure \ref{nanikoreF} has a \isi{falling intonation},
in which case it is impossible to utter \ci{kore} with \isi{rising intonation}.
%Questions with \isi{falling intonation} convey negative emotion of the speaker.
It is this constraint on the intonation of post-predicate elements that yields the \isi{emotive} \isi{implicature} of the \isi{utterance} in \ref{nanikore}.
In fact, the neutral \isi{word order} \ci{kore nani} can be uttered with \isi{falling intonation}, as shown in Figure \ref{korenani_fallF}.
In this case, as predicted from the discussion, the \isi{falling intonation} conveys the emotion of the speaker.
It is possible for \ci{nani} `what' in \ref{nanikore} to be uttered with \isi{rising intonation} as indicated in Figure \ref{nanikore_riseF},
in which case the \isi{emotive} nuance of \ref{nanikore} disappears.


\begin{figure}
%\begin{minipage}{0.5\textwidth}
	\begin{center}
	\includegraphics[width=0.6\textwidth]{sounds/korenani.pdf}
	\caption{Pitch contour of \ci{kore nani} \ref{korenani} with rising intonation}
	\label{korenaniF}
	\end{center}
%\end{minipage}
\end{figure}
\begin{figure}
%\begin{minipage}{0.5\textwidth}
	\begin{center}
	\includegraphics[width=0.6\textwidth]{sounds/nanikore.pdf}
	\caption{Pitch contour of \ci{nani kore} \ref{nanikore}}
	\label{nanikoreF}
	\end{center}
%\end{minipage}
\end{figure}
\begin{figure}
%\begin{minipage}{0.5\textwidth}
	\begin{center}
	\includegraphics[width=0.6\textwidth]{sounds/korenani_fall.pdf}
	\caption{Pitch contour of \ci{kore nani} \ref{korenani} with falling intonation}
	\label{korenani_fallF}
	\end{center}
%\end{minipage}
\end{figure}
\begin{figure}
%\begin{minipage}{0.5\textwidth}
	\begin{center}
	\includegraphics[width=0.6\textwidth]{sounds/nanikore_rise.pdf}
	\caption{Pitch contour of \ci{nani kore} \ref{nanikore} with rising intonation of \ci{nani}}
	\label{nanikore_riseF}
	\end{center}
%\end{minipage}
\end{figure}

%%----------------------------------------------------
\subsubsection{Post-predicate elements of the double-contour type}\label{WO:PostP:Motiv:Double}

Finally, in this section,
I briefly review some intriguing studies on post-predicate constructions which I assume belong to the double-contour type.
The first study is \citeA{kakuden12}.
They investigated whether the \isi{hearer} responds (including back-channel responses) to the speaker near and after the predicate, and showed that
the speaker adds post-predicate elements
when the \isi{hearer} does not respond to the predicate.
Their further analysis suggests that the speaker produces post-predicate elements to get a response from the \isi{hearer} and to achieve mutual belief.
Let us see example \Next,
which comes from the dialogue part of CSJ they employed.
The duration of silences is shown in seconds inside parentheses, since it is important for the discussion.
In \Next[-L2], where the speaker postposes the element \ci{kono kenkyuu} `this study',
there are pauses between the \isi{verb} phrase and the postposed \isi{demonstrative} \ci{kono} `this', as well as between the \isi{demonstrative} and the postposed NP \ci{kenkyuu} `study',
which is enough time for L to realize that
R does not respond to L.
Note that R, the listener of the \isi{postposed construction},
does not respond until second 604.33,
0.32 seconds after L finished the post-predicate part.
Also note that
these pauses differentiate post-predicate constructions of the double-contour type from those of the single-contour type.
%
\ex.
 \ag.[L1:] ima nan-nin-gurai-de (0.588) a (0.29) ohi \\
           now what-\ab{cl}.person-\ab{hdg}-with {} \ab{fl} {} \ab{frg} \\
           `Right now, how many people... oh,'
 \bg.[L2:] kihontekini-wa hitori-de (0.161) \EMi{yat}-te rassyaru-desu-mon-ne
           (0.12) \EM{kono} (0.585) \EM{kenkyuu} \\
           basically-\ci{wa} alone-with {} do-and \ab{prog}.\ab{hon}-\ab{cop}-\ab{nmlz}-\ab{fp} {} this {} study \\
           `basically, (you) do (it) by yourself, this study?'
 \b.[] \src{D04M0010: 597.20-604.01}
 \bg.[R3:] ettoo (0.434) a (0.137) boku-no syozoku-si-teru kenkyuu-situ-de(0.44)-wa hanasi-kotoba-no ninsiki-o yat-teru-no-wa (0.143) m soo-desu-ne \\
           \ab{fl} {} \ab{fl} {} \ab{1}\ab{sg}-\ab{gen} belong-do-\ab{prog} study-room-\ab{loc}-\ci{wa} speech-language-\ab{gen} recognization-\ci{o} do-\ab{prog}-\ab{nmlz}-\ci{wa} {} \ab{frg} so-\ab{cop}.\ab{plt}-\ab{fp} \\
           `Lets see... in the lab I belong to the one who studies speech recognition it's, yes...,'
 \bg.[R4:] boku hitori-desu-ne \\
      \ab{1}\ab{sg} alone-\ab{cop}.\ab{plt}-\ab{fp} \\
      `it's just me.'
           \src{D04M0010: 604.33-612.09}
  \b.[] \hfill{\cite[287]{kakuden12}}
%D04M0010|00597201L|597.201362|604.007857|L|今何人ぐらいで(0.588)(F あ)(0.29)(D おひ(? てぃ))基本的には一人で(0.161)やってらっしゃる(0.207)ですもんね(0.12)この(0.585)研究||倒置−つなぎ切り
%D04M0010|00604328R|604.327743|612.085642|R|(F えっとー)(0.434)(F あ)(D (? む))(0.137)僕の所属してる研究室で(0.44)は話し言葉の認識をやってるのは(0.143)(D む)そうですね僕一人ですね|[文末候補]|


\citeA{tanaka05} investigates postposed and preposed constructions
in terms of \isi{interactional} structures:
preferred vs.~dispreferred structures.
See the discussion in \S \ref{Back:CharJ:WO:PostP} for detail.
%Preferred structures include assessment followed by agreement and request followed by acceptance,
%while dispreferred structures include assessment followed by disagreement and request followed by refusal.
%According to \citeA[332ff.]{levinson83},
%preferred second parts including agreement with an assessment and acceptance of a request are typically produced in a 
%simple and direct form without delay;
%on the other hand,
%dispreferred second parts including disagreement with an assessment and
%refusal of a request are typically produced in a 
%complex and indirect form with delay.
%
%%
%\ex.
% \ag.[C:] ima-no katati-to mattaku onnazi.= \\
%          now-\ab{gen} shape-with exactly same \\
%          `(It's) exactly the same shape as the ones in vogue now.'
% \bg.[K:] =onnazi-yo$\downarrow$ =[\EM{eri-mo}. \\
%          same-\ab{fp} collar-also \\
%          `(It's) the same, the collar too.'
% \bg.[E:] {\hspace{2.45cm}} [a! honto::. \\
%          {} oh really \\
%          {\hspace{2.45cm}}`Oh re::ally.'
%          \hfill{\cite[406]{tanaka05}}
%
%\ex. (A response to an inquiry about an advertisement)
% \ag. ${\uparrow}$e:to{\textasciitilde}${\downarrow}$ g{\textasciitilde} (0.3) ano:: .hhhh \\
%      uh:m {} {} uh::m \\
%      `Uh:m- g- uh::m .hhhh'
% \bg. sono $<$\ul{nakami}$>$-made \\
%      its content-as.for \\
%      `when it comes down to its contents,'
% \bg. tyotto-ne \\
%      a.bit-\ab{fp} \\
%      `sort of'
% \bg. kookoku-no$\tilde{}$ gn $>$-ga-tte-no-wa \\
%      advert-\ab{gen} \ab{frg} -\ab{nom}-\ab{quot}-\ab{nmlz}-\ab{top} \\
%      `when it comes to the (gn) of the advert,'
% \bg. tyotto \\
%      a.bit \\
%      `sort.of'r
% \bg. kotira-de-wa \\
%      here-\ab{loc}-\ab{top} \\
%      `on our side,'
% \bg. wakara-nai-n-desu-keredomo$<$, .hhhh \\
%      know-\ab{neg}-\ab{nmlz}-\ab{cop}.\ab{plt}-though \\
%      `(we) have no knowledge of, .hhhh'
%       \hfill{\cite[412-413]{tanaka05}}

%%----------------------------------------------------
\subsection{Summary of post-predicate elements}

In this section
I investigated post-predicate elements.
It turned out that
the activation cost of postposed elements is much lower than
that of preposed elements,
elements that appear before the predicate.
This suggests that topics also appear post-predicatively.
I also discussed why
topics appear post-predicatively as well as clause-initially
in terms of the shape of intonation and its constraints on Japanese grammar.

The characteristic found in this study is one of many features of post-predicate elements.
In future research,
it is necessary to explore how these features are related to each other.


%%----------------------------------------------------
%%----------------------------------------------------
\section{Pre-predicate elements}\label{WOPrePredEles}

This section discusses pre-predicate elements,
elements which appear immediately before the predicate.
In \S \ref{WO:PreP:New},
I show results that indicate that
new, i.e.\ focus, elements tend to appear right before the predicate.
In \S \ref{WO:PreP:Motivation},
I discuss reasons for why focus elements appear near the predicate.

\begin{figure}
%\begin{minipage}{0.5\textwidth}
	\begin{center}
	\includegraphics[width=0.7\textwidth]{figure/DEPositionIS.pdf}
	\caption{Word order vs.\ information status}
	\label{DEPositionISF2}
	\end{center}
%\end{minipage}
\end{figure}
\begin{figure}
%\begin{minipage}{0.5\textwidth}
	\begin{center}
	\includegraphics[width=0.7\textwidth]{figure/DiffInfoStatus.pdf}
	\caption{Distance from predicate vs.\ InfoStatus}
	\label{DiffInfoStatusF2}
	\end{center}
%\end{minipage}
\end{figure}


%%----------------------------------------------------
\subsection{New elements appear right before the predicate}\label{WO:PreP:New}

%\begin{figure}
%\begin{minipage}{0.5\textwidth}
%	\begin{center}
%	\includegraphics[width=0.95\textwidth]{figure/DiffCaseGiven.pdf}
%	\caption{}
%	\label{Diff1}
%	\end{center}
%\end{minipage}
%\begin{minipage}{0.5\textwidth}
%	\begin{center}
%	\includegraphics[width=0.95\textwidth]{figure/DiffCaseNew.pdf}
%	\caption{}
%	\label{Diff2}
%	\end{center}
%\end{minipage}
%\end{figure}
%
%\chd{
%According to the statistical analysis reported in \S \ref{WO:Intro},
%the effect of the distance between the element in question and the predicate to predict \isi{information status} is only marginally significant.
%}

As shown in Figure \ref{DEPositionISF} and \ref{DiffInfoStatusF},
repeated here as Figure \ref{DEPositionISF2} and \ref{DiffInfoStatusF2} for convenience,
new elements or focus elements tend to appear immediately before the predicate.
Figure \ref{DEPositionISF2} shows the element position based on \isi{information status} including all expressions such as fillers, adjectives, etc.;
Figure \ref{DiffInfoStatusF2} shows the distance between each of the elements and the predicate based on their \isi{information status}.
%Figure \ref{WOISGivenF2} and \ref{WOISNewF2} are histograms of position of arguments excluding other expressions and clauses with one argument.
%They also show the distribution of A, S, and P in each position.
As indicated in Figure \ref{DEPositionISF2},
the distribution of \isi{anaphoric} elements is skewed towards clause-initial position,
whereas that of non-\isi{anaphoric} elements is not.
Drawing from Figure \ref{DiffInfoStatusF2}, we can also see that many new elements appear immediately before the predicate.
%Considering clauses which contain equal to or more than two arguments,
%given elements appear at the initial position most frequently
%as shown in Figure \ref{WOISGivenF2},
%whereas new elements appear at the second position most frequently as shown \ref{WOISNewF2}.
%Especially given A elements are more frequent than new A elements in the initial position.
%This is compatible with the observation that A tend to be given in \ili{Sakapultec Maya} and many other languages \cite{dubois87,duboisetal03}.
%New S or P elements appear more frequently than given S or P elements in the second position.
\chd{As discussed in \ref{WO:Intro}, the mixed effects model of \isi{information status} (the distance between the predicate and the element in question) shows that
the contribution of distance is only marginally significant.
However, a further analysis in this section shows that distance is also a significant factor for predicting \isi{information status}.
As is clear from Table \ref{ParInfoStatusCTT} and \ref{Par:InfoStatusPar:LSMEANST},
datives tend to code new elements (especially, as opposed to \ci{wa}).
Datives can appear anywhere, from pre-predicate to clause-initial positions,
which is shown in Figure \ref{WO:Prep:New:DiffASPF}.
%In fact, the model excluding datives indicates that
%both the distance and particles are significant factors to predict \isi{information status} (likelihood ratio test, $p<0.05$ without the distance, $p<0.001$ without particles).
Therefore, I tentatively conclude that the distance between the predicate and the element in question (excluding \ci{ni}-coded elements) is an important factor for \isi{information status}, and that new elements appear before the predicate.
This supports a classic observation from other languages that focus appears close to the predicate (\citeA{bresnan94,morimoto99} on Bantu languages, \citeA{jacennikdryer92} on \ili{Polish}, \citeA{erguvanli84} on \ili{Turkish}, see \citeA{morimoto00} for a summary of studies on both VO and OV languages).
Further studies are necessary to obtain conclusive evidence.}

\begin{figure}
	\begin{center}
	\includegraphics[width=0.8\textwidth]{figure/DiffASP.pdf}
	\caption{Distance from predicate vs.~grammatical function}
	\label{WO:Prep:New:DiffASPF}
	\end{center}
\end{figure}


The following are examples of non-\isi{anaphoric} elements appearing close to the predicate.
\Next and \NNext are examples of non-\isi{anaphoric} P occurring immediately before the predicate.
In \Next,
\ci{kyoomi} `interest' appears immediately before the predicate \ci{moti} `have',
and, in \NNext,
\ci{aidenthithii} `identity' in line a, \ci{inoti} `life' in line b, and \ci{ti} `blood' in line c appear right before the predicates \ci{kake} `risk' and \ci{nagasi} `bleed', respectively. Non-\isi{anaphoric} Ps are typically abstract concepts like \ci{kyoomi} `interest' in \Next, \ci{aidenthithii} `identity' in \NNext[a], and \ci{inoti} `life' in \NNext[b], or
\isi{indefinite} like \ci{ti} `blood' in \NNext[c].
%
\exg. de ee sono ri-too-no hoo-ni sono \EM{kyoomi-o} \ul{moti} hazime-masi-te \\
		then \ab{fl} \ab{fl} remote-island-\ab{gen} direction-\ab{dat} \ab{fl} interest-\ci{o} have start-\ab{plt}-and \\
	`(We) are starting to be interested in remote islands (in Hawaii).'
	\src{S00F0014: 149.92-153.33}
%S00F0014|00149920L|149.919959|153.334445|L|で(0.325)(F えー)(F その)離島の方に(F その)興味を持ち始めまして|/テ節/|

\ex.\label{S00M0199_inoti}
 \ag. tasuu-no serubia-zin-ga minzoku-no ee \EM{aidenthithii-o} \ul{kake}-te \\
 	many-\ab{gen} Serbia-people-\ci{ga} ethnic-\ab{gen} \ab{fl} identity-\ci{o} risk-and \\
	`Serbian people bet their identity, and'
 \bg. \EM{inoti-o} \ul{kake}-te \\
 		life-\ci{o} risk-and \\
		`risked their lives, and'
 \bg. \EM{ti-o} \ul{nagasi}-ta-to iu \\
 		blood-\ci{o} bleed-\ab{past}-\ab{q} say \\
		`bled (in battles),'
 \bg. rekisi-ga ee sono-go tenkai s-are-masu \\
 	history-\ci{ga} \ab{fl} that-later progress do-\ab{pass}-\ab{plt} \\
	`history went on this way.'
	\src{S00M0199: 343.53-351.77}
%S00M0199|00337016L|337.015835|351.772351|L|その時(0.488)(F えー)(F ま)その聖地(0.157)を(0.142)守るという意味で(F あのー)このコソボ平原において(0.422)(F えー)多数のセルビア人が民族の(0.204)(F えー)アイデンティティーを賭けて命を賭けて血を流したという(0.379)歴史が(0.17)(F えー)その後(0.167)展開されます|[文末]|

Non-\isi{anaphoric} S elements also appear immediately before the predicate.
They tend to be abstract or \isi{indefinite} like non-\isi{anaphoric} Ps.
In \Next,
\ci{kanzi} `impression', an abstract concept, is the only argument of the predicate \ci{tigau} `differ' and is therefore S.
This element appears immediately before the predicate.
%
\ex.\label{S00F0014_kanzi}
 \ag. sono kontorasuto-toiuno-wa nanka totemo koo ekizotikku-to-iu-ka \\
 	that contrast-\ci{toiuno}-\ci{wa} somehow very such exotic-\ab{quot}-say-\ab{q} \\
	`The contrast [between black and blue] is very exotic, I would say,'
 \bg. husigina \EM{kanzi}-ga \ul{si}-masi-te \\
 		mysterious impression-\ci{ga} do-\ab{plt}-and \\
		`the impression was mysterious.'
		\src{S00F0014: 1042.88-1047.03}
%S00F0014|01042882L|1042.88247|1047.031717|L|そのコントラストというのは何かとてもこうエキゾチックと言うか不思議な感じがしまして|/テ節/|
%
%\ex.
% \ag. iya risu-nisite-wa tyotto sippo-no \EM{kanzi}-ga tigau-na-to \\
% 		no squirrel-for-\ci{wa} a.bit tail-\ab{gen} impression-\ab{nom} different-\ab{fp}-\ab{quot} \\
%		`No, the impression of (the animal's) tail is different from that of squirrels,'
% \bg. omot-te-ta-n-desu-ne \\
% 	think-\ab{prog}-\ab{nmlz}-\ab{cop}.\ab{plt}-\ab{fp} \\
%	`(I) was thinking like that.'
%	\src{S00F0014: 635.38-639.92}
%S00F0014|00635381L|635.381427|639.920666|L|いやリスにしてはちょっと尻尾の感じが違うなと(0.352)思ってたんですね|[文末]|

In \Next,
\ci{hito} `person' is \isi{indefinite} and appears before the predicate.
\exg. naka-ni-wa byooin-okuri-ni naru \EM{hito}-mo \ul{i}-masi-ta-kedomo \\
		inside-\ab{dat}-\ci{wa} hospital-send-to become person-also exist-\ab{plt}-\ab{past}-though \\
		`Some people were sent to the hospital (lit. People who were sent to the hospital also exist).'
		\src{S05M1236: 578.30-581.49}
%S05M1236|00578305L|578.304793|581.489255|L|中には病院送りになる(0.445)<笑>人もいましたけども|/並列節ケドモ/|


%%----------------------------------------------------
\subsection{Motivations for a focus to appear close to the predicate}\label{WO:PreP:Motivation}

%Why do non-\isi{anaphoric} elements most frequently appear close to the predicate?
I argue that the information-structure \isi{continuity principle} \ref{IScontinuityP} is also at work here, which is repeated below as \Next for the purpose of convenience.
%
\ex. \label{IScontinuityP2}\tl{Information-structure continuity principle}:
 A unit of \isi{information structure} is continuous in a clause;
 i.e., elements which belong to the same unit are adjacent to each other.

I assume that
the predicate is most frequently in the domain of focus \cite{lambrecht94},
optionally with one focus element.
Since the predicate and the new element are in the same domain of focus,
they also appear together most frequently.

In fact, only few studies pay attention to the \isi{information status} (and namely \isi{information structure}) of predicates.%
	\footnote{
	\citeA{hopperthompson80} is an important exception.
	}
Unfortunately this study is not an exception.
Typically, definite markers such as \ci{the} in \ili{English} and \ci{der} in \ili{German} attach to nouns, not to verbs.
Also \isi{topic} markers such as \ci{wa} in Japanese typically attach to nouns.
Therefore, nouns have attracted more attention than verbs.
Typically verbs are followed by tense or aspect markers, subordinate-clause markers, realis vs.\ irrealis markers, and so on.
I believe that these verbal markers are also related to \isi{information structure},
but this is beyond the scope of this study.

However,
it is obvious that \isi{argument-focus structure},
where the predicate is not in the domain of focus,
is the least frequent type among all three types of focus constructions (predicate-focus, sentence-focus, and argument-focus structures).
Given that the corpus employed in this study consists of monologues, it is to be expected that there are even fewer examples of argument-focus structures because these structures typically appear as the answer to a who/what question, as shown in \Next, where the capital letters indicate prominence.
%
\ex.
 \a.[Q:] Who went to school?
 \b.[A:] [The CHILDREN]$_{F}$ [went to school]$_{T}$.
 \hfill{\cite[][121]{lambrecht94}}

Since there are no (explicit) questions in monologues,
we find fewer argument-focus structures.

Another context in which sentences with \isi{argument-focus structure} appear is the ``A not B'' context.
In \isi{monologue}s, ``A not B'' contexts typically appear in self-repair,
which is also rare in our relatively smooth monologues.
Therefore, it is not unreasonable to assume that the predicate is in the domain of focus most of the time,
and I argue that the information-structure \isi{continuity principle} \ref{IScontinuityP2} explains why new elements (i.e., focus elements) tend to appear immediately before the predicate.

One piece of evidence that supports the information-structure \isi{continuity principle} is the fact that
it is difficult for presupposed elements to appear immediately before the predicate,
interrupting the focus domain.
Compare \Next[A] and \Next[A$^{\prime}$],
which are answers to the question in \Next[Q].%
	\footnote{
	Note that they are not a perfect minimal pair because of the \isi{accusative marker} of \ci{o}.
	The presence or absence of \ci{o} is determined by \isi{word order}, and
	\isi{information structure} is a kind of side effect in this case.
	See the discussion in \S \ref{CasePar} for more detail.
	}
In \Next[A],
the presupposed elements \ci{taroo-ni} `to Taro' and \ci{hanako-ni} `to Hanako' are interrupting the domain of focus `gave a travel ticket' and `gave a cake'.
Therefore this sentence is not acceptable.
Conversely, in \Next[A$^{\prime}$],
the presupposed elements do not intervene the domain of focus and therefore the answer is acceptable.
%
\ex.
 \a.[Q:] What did you do for Taro and Hanako for their birthdays?
 \bg.[A:] ?[ryokoo-ken-o]$_{F}$ [taroo-ni]$_{T}$ [age-te]$_{F}$ [keeki-o]$_{F}$ [hanako-ni]$_{T}$ [tukut-te age-ta]$_{F}$-yo \\
 			travel-ticket-\ci{o} Taro-\ab{dat} give-and cake-\ci{o} Hanako-\ab{dat} make-and give-\ab{past}-\ab{fp} \\
			`(I) gave travel tickets to Taro and gave cake to Hanako.'
 \bg.[A$^{\prime}$:] [taroo-ni]$_{T}$ [ryokoo-ken age-te]$_{F}$ [hanako-ni]$_{T}$ [keeki tukut-te age-ta]$_{F}$-yo \\
 			Taro-\ab{dat} travel-ticket give-and Hanako-\ab{dat} cake make-and give-\ab{past}-\ab{fp} \\
			`(I) gave Taro travel tickets and gave Hanako cake.'

A more natural context for \Last[A] is one where Q asks
what A did for the travel ticket and the cake.
\citeA{kuno78} proposes that
the pre-predicate position is for new elements,
but he limits this principle to cases
where the predicate is given.
%
\ex.
 In cases where the predicate is given,
 the position immediately before the predicate is the position for new.
 \hfill{\cite[][60, translated by the current author]{kuno78}}

I argue that this observation also applies to cases where the predicate is new.

Moreover,
as will be discussed in Chapter \ref{Intonation},
the domain of focus is uttered in a single \isi{intonation unit},
whereas the \isi{topic} is uttered separately from the domain of focus.
Figure \ref{S00M0199_aidenthithiF} to \ref{S00F0014_kanziF} show
the \isi{pitch} contours of examples \ref{S00M0199_inoti} and \ref{S00F0014_kanzi} we discussed in the last section.
As we can see,
there is no pause between the predicate and the previous element, and
the \isi{pitch range} is larger in the elements than in the predicates.
In Figure \ref{S00M0199_tiF},
it is difficult to see the \isi{pitch range} because \ci{ti} `blood' does not have accent nucleus.
From the first lowering of \ci{na} in \ci{nagasi-ta} `bled' being cancelled,%
	\footnote{
	The \isi{pitch accent} of \ci{nagasi-ta} is LHLL.
	}
one can see that
\ci{ti-o} `blood-\ci{o}' and \ci{nagasi-ta} `bleed' form a single \isi{intonation unit}.


\begin{figure}
%\begin{minipage}{0.5\textwidth}
	\begin{center}
	\includegraphics[width=0.6\textwidth]{sounds/S00M0199_aidenthithi.pdf}
	\caption{Pitch contour of a in \ref{S00M0199_inoti}}
	\label{S00M0199_aidenthithiF}
	\end{center}
%\end{minipage}
\end{figure}
\begin{figure}
%\begin{minipage}{0.5\textwidth}
	\begin{center}
	\includegraphics[width=0.6\textwidth]{sounds/S00M0199_inoti.pdf}
	\caption{Pitch contour of b in \ref{S00M0199_inoti}}
	\label{S00M0199_inotiF}
	\end{center}
%\end{minipage}
\end{figure}
\begin{figure}
%\begin{minipage}{0.5\textwidth}
	\begin{center}
	\includegraphics[width=0.6\textwidth]{sounds/S00M0199_ti.pdf}
	\caption{Pitch contour of c in \ref{S00M0199_inoti}}
	\label{S00M0199_tiF}
	\end{center}
%\end{minipage}
\end{figure}
\begin{figure}
%\begin{minipage}{0.5\textwidth}
	\begin{center}
	\includegraphics[width=0.6\textwidth]{sounds/S00F0014_kanzi.pdf}
	\caption{Pitch contour of b in \ref{S00F0014_kanzi}}
	\label{S00F0014_kanziF}
	\end{center}
%\end{minipage}
\end{figure}

%%----------------------------------------------------
\subsection{Summary of pre-predicate elements}

The results of this section showed that
new elements, namely focus elements, tend to appear right before the predicate.
A similar claim has been made by \citeA{kuno78} and \citeA{endo14}
through constructed examples.
This study supported their claim by examining naturally occurring utterances.
I also discussed explanations why the focus appears right before the predicate.

%%----------------------------------------------------
%%----------------------------------------------------
\section{Discussion}\label{WODiscussion}

This section first discusses possible confounding effects on \isi{word order} in Japanese,
in particular in association with basic \isi{word order} (\S \ref{WO:Dis:Confounding}).
Second,
I discuss Giv{\'{o}}n's \isi{topicality} hierarchy (\S \ref{WO:Dis:Givon}).
I provide some counter-examples to this hierarchy and
propose modifications to it.
Finally, 
I discuss the implications of this study's findings as regards \isi{word order} typology (\S \ref{WO:Dis:WOTypology}).



%%----------------------------------------------------
\subsection{Possible confounding effects}\label{WO:Dis:Confounding}

\begin{figure}
%\begin{minipage}{0.5\textwidth}
	\begin{center}
	\includegraphics[width=0.6\textwidth]{figure/DEPositionASPA.pdf}
	\caption{Word order of A}
	\label{DEPositionASPAF}
	\end{center}
%\end{minipage}
\end{figure}
\begin{figure}
%\begin{minipage}{0.5\textwidth}
	\begin{center}
	\includegraphics[width=0.6\textwidth]{figure/DEPositionASPS.pdf}
	\caption{Word order of S}
	\label{DEPositionASPSF}
	\end{center}
%\end{minipage}
\end{figure}
\begin{figure}
%\begin{minipage}{0.5\textwidth}
	\begin{center}
	\includegraphics[width=0.6\textwidth]{figure/DEPositionASPP.pdf}
	\caption{Word order of P}
	\label{DEPositionASPPF}
	\end{center}
%\end{minipage}
\end{figure}
\begin{figure}
%\begin{minipage}{0.5\textwidth}
	\begin{center}
	\includegraphics[width=0.6\textwidth]{figure/DEPositionASPDAT.pdf}
	\caption{Word order of dative}
	\label{DEPositionASPDATF}
	\end{center}
%\end{minipage}
\end{figure}


It is necessary to take other features into account
to see the exact effect of topichood and focushood on \isi{word order}. %
%%%ISSUE odd word break in "topichood".
 Especially, the effect of ``basic \isi{word order}'' should not be ignored.
%I did not employ any statistical analysis because
%there are still not enough data in my corpus.
Here I provide some evidence to support my argument that
\isi{information structure} contributes to \isi{word order} in spoken Japanese.
Figures \ref{DEPositionASPAF} to \ref{DEPositionASPDATF} show the
\isi{word order} and \isi{information status} of each type of grammatical function (A, S, P, and dative).
These figures indicate that
\isi{anaphoric} elements of all grammatical function types are still more likely to appear earlier in a clause than new elements.
A and S are more likely to appear earlier in a clause than P because of the basic \isi{word order}.
However, my argument still holds for the same grammatical function types.
In cases with new elements,
one can see the effect of basic \isi{word order};
the peak of S is 4,
which means the 4th position is the most popular for new S (Figure \ref{DEPositionASPSF}),
whereas the peak of P is 6, which means the 6th position is the most popular for new P (Figure \ref{DEPositionASPPF}).
The distribution of A is not clear because there are few examples.
But the trend still seems to hold for A.

%%----------------------------------------------------
\subsection{Giv\'on's topicality hierarchy and word order}\label{WO:Dis:Givon}

\citeA{givon83} proposes a hierarchy of \isi{topicality}, shown in \Next (terminology modified by the author).
``RD'' refers to referential distance,
which is one of the approximations to measure \isi{topicality}.
Low RD means high \isi{topicality},
while high RD means low \isi{topicality}.
%
\ex.\label{TopicHierarchy}
 \a.[$\uparrow$] High RD
 \b. Referential \isi{indefinite} NPs
 \b. Cleft/focus constructions
 \b. Y-moved NPs (`contrastive topicalization')
 \b. Preposed definite NPs
 \b. Neutral-ordered definite NPs
 \b. Postposed definite NPs
 \b. Stressed/independent pronouns
 \b. Unstressed/bound pronouns or grammatical agreement
 \b. Zero anaphora
 \b.[$\downarrow$] Low RD
 \hfill{\cite[][7]{givon83}}

Here I point out two counter-examples to this hierarchy.
First,
as has already been shown in Table \ref{RDPostT} and \ref{RDPreT},
which are repeated as Table \ref{RDPostT2} and \ref{RDPreT2} for convenience,
the average RD of elements in the clause-initial position (20.9) is lower than that in the second (23.0) or third position (41.1).
To see this more in detail,
I divided the results of Table \ref{RDPreT2} on the basis of grammatical function.
This is shown in Table \ref{RDASP}.
Regardless of whether the element is A, S, or P,
the overall tendency is that
the elements closer to the predicate have a higher average RD.%
 \footnote{
 For now I do not have an explanation for S in the second position.
 It is necessary to test whether the difference between Ss in the first and the second positions is statistically significant or not.
 }
%%% 2Sに関しては説明が必要
The \isi{topicality} hierarchy in \ref{TopicHierarchy} predicts that
clause-initial elements (d in \ref{TopicHierarchy}) have a lower RD than elements in the neutral-ordered position (e in \ref{TopicHierarchy}).%
	\footnote{
	I assume that all elements that have antecedents (and therefore also RDs) are definite.
	}
Especially P is against the \isi{topicality} hierarchy in \ref{TopicHierarchy},
according to which P in the second or third positions should have a lower RD than P in the first position, since the neutral position of P in Japanese is the second or third position.
However, this is not the case.
At least in Japanese,
the data show that
elements closer to the predicate have higher RDs because the pre-predicate position is for focus and hence for new elements.

\begin{table}[bht]
%\begin{minipage}{0.5\textwidth}
\centering
\caption{RD of post-predicate elements}
\begin{tabular}{lrr}
\lsptoprule
  & Single-contour & Double-contour \\
\midrule
RD & 6.9 & 39.7 \\
\lspbottomrule
\end{tabular}
\label{RDPostT2}
%\end{minipage}
\end{table}
\begin{table}
%\begin{minipage}{0.5\textwidth}
\centering
\caption{RD of pre-predicate elements (based on argument order)}
\begin{tabular}{lrrrr}
\lsptoprule
  &  1  & 2 & 3 \\
\midrule
RD & 20.9 & 23.0 & 41.1 \\
\lspbottomrule
\end{tabular}
\label{RDPreT2}
%\end{minipage}
\end{table}

\begin{table}
%\begin{minipage}{0.5\textwidth}
\centering
\caption{RD of pre-predicate elements (based on grammatical function)}
\label{RDASP}
\begin{tabular}{lrrr}
\lsptoprule
		& 1	& 2	& 3 \\
\midrule
%	Ex	& 15.2	& 15.0 & --	\\
	A	& 10.3	& 47.3	& -- \\
	S	& 22.5	& 21.7	& 73.5 \\
	P	& 22.4	& 36.6	& 49.1 \\
\lspbottomrule
\end{tabular}
%\end{minipage}
\end{table}

Second,
the average RD of zero pronouns is as high as that of postposed NPs
according to Table \ref{RDPostExpTypeT} and \ref{RDPreExpTypeT}.
This is against the \isi{topicality} hierarchy in \ref{TopicHierarchy}, which states that
preposed definite NPs (d in \ref{TopicHierarchy}) and neutral-ordered definite NPs (e in \ref{TopicHierarchy}) have
higher RDs than
postposed definite NPs.
As discussed above,
elements are postposed for \isi{interactional} purposes and/or intonational reasons.


\begin{table}
%\begin{minipage}{0.5\textwidth}
\centering
\caption{RD of postposed elements of the single-contour type (based on expression type)}
\begin{tabular}{lrr}
\lsptoprule
  & Pronoun & NP \\
\midrule
RD & 15.1 & 5.0 \\
\lspbottomrule
\end{tabular}
\label{RDPostExpTypeT}
%\end{minipage}
\end{table}
\begin{table}
%\begin{minipage}{0.5\textwidth}
\centering
\caption{RD of pre-predicate elements (based on expression type)}
\begin{tabular}{lrrr}
\lsptoprule
     & Zero & Pronoun & NP \\
\midrule
 RD  & 5.0  & 5.8     & 27.8 \\
\lspbottomrule
\end{tabular}
\label{RDPreExpTypeT}
%\end{minipage}
\end{table}

The final point is an additional suggestion for \ref{TopicHierarchy} rather than a counter-ex\-am\-ple.
The RD of postposed elements of the double-contour type is much higher than Giv\'on predicts.
As will be argued in Chapter \ref{Intonation},
a unit of \isi{information structure} corresponds to a unit of intonation.
Since postposed elements of the single-contour type by definition belong to the same \isi{intonation unit} as the main predicate,
the predicate and the \isi{postposed element} form a single unit (construction) and postposed elements are relatively homogeneous and easy to characterize.
However,
postposed elements of the double-contour type are heterogeneous, as discussed above, and they are difficult to characterize
because the element itself corresponds to a single unit.
There are different reasons why such elements are uttered. 
%The motivations for such elements to be uttered are heterogeneous.
The function of these postposed elements is determined by the sequence of conversation.


%%----------------------------------------------------
\subsection{Information structure and word order typology}\label{WO:Dis:WOTypology}

Since focus elements are most frequently patients according to the correlating features in \ref{ISFeatures},
which is repeated here as \Next,
the information-structure \isi{continuity principle} in \ref{IScontinuityP} predicts that, cross-linguistically,
P (the patient-like argument in a \isi{transitive clause}) and V (the predicate) tend to appear together most frequently and,
if the \isi{word order} is fixed in the language in question,
P and V tend to appear together.
%
\ex.
\begin{tabular}[t]{lll}
	 & \isi{topic} & focus \\
	a. & presupposed & asserted \\
	b. & evoked & brand-new \\
	c. & definite & \isi{indefinite} \\
	d. & specific & non-specific \\
	e. & \isi{animate} & \isi{inanimate} \\
	f. & agent & patient \\
	g. & \isi{inferable} & non-\isi{inferable} \\
%	h. & entity & proposition \\
\end{tabular}
%

In fact, this has already been claimed and tested in \citeA[Chapter 4]{tomlin86}.
Tomlin proposes this claim in terms of the Verb-Object Bonding.
%
\ex. \tl{Verb-Object Bonding (VOB):}
	the object of a \isi{transitive verb} is more tightly bounded to the \isi{verb}
	than is its subject.
	\hfill{\cite[][74]{tomlin86}}
%``[i]n general the object of a \isi{transitive clause} is syntactically and semantically more tightly `bound' to the \isi{verb} than is the subject of a \isi{transitive clause}'' (p.~73).

He also states that
``[e]xactly why there should be such a bond between a \isi{transitive verb} and its object is not entirely clear'' (ibid.).
I propose the information-structure \isi{continuity principle}
as the motivation for such bond.
He enumerates many cross-linguistic pieces of evidence that support VOB.
I introduce a few of them to keep the discussion simple.

First,
in many languages,
there exists some clause-level phonological behavior
(reductions or sandhis) which occur between object and \isi{verb},
but not between subject and \isi{verb} (op.~cit., p.~97).
In \ili{French}, for example,
liaison does not occur between the subject and a \isi{transitive verb},
but it does between the object and the \isi{verb} \cite[see also][]{selkirk72}.
There is no liaison between the subject \ci{les gens} and the \isi{verb} \ci{ach\`{e}tent} in \Next,
whereas there can be liaison between the \isi{verb} \ci{donnerons} and the object \ci{une pomme} in \NNext.%
%%%ISSUE IPA fonts (inside the command \tp{}) are not shown properly.
%
\ex.
 \a. \glll
	les gens ach\`{e}tent beaucoup de \c{c}a \\
	\tp{le} \tp{Z\~a} \tp{aSEt} \tp{boku} \tp{d@} \tp{sa} \\
	the people buy.\ab{3}\ab{pl} much of that \\
	`Those people buy a lot of that.' \hfill{(no liaison)}
 \b. *\tp{le} \tp{Z\~a} \tp{zaSEt} \tp{boku} \tp{d@} \tp{sa} 
 		 \hfill{(*liaison)}

\ex.
 \a. \glll
	nous donnerons une pomme \`a notre m\`ere \\
  	\tp{nu} \tp{dOn@r\~o} \tp{zyn} \tp{pOm} \tp{a} \tp{notr} \tp{mEr} \\
	we give.\ab{3}\ab{pl} a apple to our mother \\
	`We will give an apple to our mother.'
	\hfill{(liaison)}
\begin{flushright}
\cite[][pp.~98-99, transcription modified based on standard \ili{French}]{tomlin86}
\end{flushright}

Another case is \ili{Yoruba} (Niger-Congo) \isi{vowel} deletion \cite[from][]{bamgbose64}.
In verb-noun sequences in this language,
when the object begins with a \isi{vowel},
the last \isi{vowel} of the \isi{verb} is sometimes deleted.
This happens between \isi{verb} and object, but not between subject and \isi{verb}.
%
\ex.
 \ag. \tp{gb\'e} + \tp{od\'o} $\to$ \tp{gb'\'od\'o} \\
		brought + motor \\
 \bg. \tp{jE} \tp{iy\'On} $\to$ \tp{j'iy\'On} \\
 		eat pounded.yam \\
 \bg. \tp{Se} \tp{\`ow\`o} $\to$ \tp{S'\`ow\`o} \\
 		do trade \\
		\hfill{\cite[][pp.~29--30]{bamgbose64}}

These phonological phenomena in \ili{French} and \ili{Yoruba} suggest that
the object and predicate are bound more tightly than the subject and predicate.
In a similar manner,
in Japanese
the focus element and the predicate form a single \isi{intonation unit},
but the \isi{topic} element and the predicate do not,
as we will see in Chapter \ref{Intonation}.

The second piece of evidence that supports VOB is \isi{noun incorporation}.
In \ili{Mokilese} (Oceanic), for example,
there is a set of verbs into which an \isi{indefinite} object may be incorporated \cite[from][]{harrison76}.
\Next[a] is a \isi{transitive clause} with a definite object,
which is not incorporated into the \isi{verb},
whereas \Next[b] is a clause with an \isi{indefinite} object,
which is incorporated into the \isi{verb}.
Note that the incorporated object \ci{rimeh} `bottle' in \Next[b] is between the \isi{verb} and the aspect suffix \ci{la}.
%
\ex.
 \ag. ngoah audoh-\EMi{la} \EM{rimeh}-i \\
		\ab{1}\ab{sg} fill-\ab{pfv} bottle-this \\
		`I filled this bottle.'
 \bg. ngoah audohd \EM{rimeh}-\EMi{la} \\
		\ab{1}\ab{sg} fill bottle-\ab{pfv} \\
		`I filled bottles.'
		\hfill{\cite[162]{harrison76}}

Similarly,
compare \Next[a] and \Next[b].
\Next[a] is a case where the object \ci{suhkoah} `tree' is definite and is not incorporated,
while \Next[b] is a case where the object is \isi{indefinite} and is incorporated into the \isi{verb}.
\ex.
 \ag. ngoah poadok-\EMi{di} \EM{suhkoah}-i \\
 	\ab{1}\ab{sg} plant-\ab{pfv} tree-this \\
	`I planted this tree'
 \bg. ngoah poad \EM{suhkoah}-\EMi{di} \\
 	\ab{1}\ab{sg} plant tree-\ab{pfv} \\
	`I planted trees.'
	\hfill{(ibid.)}

As \citeA{mithun84} observes,
in some languages patient Ss can also be incorporated into verbs,
but languages that allow patient S-incorporation also allow P-incorporation \cite[see also][]{baker88}: %
%%%ISSUE "P-incorporation" does not align properly.
there is a universal hierarchy as in \Next.
The last two (agent S and A) are in brackets because
they are not attested.
%
\ex. \label{NIhierarchy}P $>$ patient S ($>$ agent S $>$ A)

In Southern Tiwa (Tanoan), for example,
the patient Ss `dipper' and `snow' are incorporated in \Next,
while agent Ss such as `dog' cannot be incorporated as in \NNext.
	\ex. \ag. \tp{l-\EM{k'uru}-k'euwe-m} \\
			{\sc B}-\EM{dipper}-old-\ab{pres} \\
			`The dipper is old.'
		\bg. \tp{we-\EM{fan}-lur-mi} \\
			{\sc C}.\ab{neg}-\EM{snow}-fall-\ab{pres}.\ab{neg} \\
			`Snow isn't falling. (It is not snowing.)'  \hfill{(patient S)}
		\b.[] \hfill{\cite{allenetal84,baker88}}
	
	\ex. \ag. \tp{\EM{khwien}-ide} \tp{{\O}-teurawe-we} \\
			\EM{dog}-{\sc suf} {\sc A}-run-\ab{pres} \\
			`The dog is running.'
		\bg. *\tp{{\O}-\EM{khwien}-teurawe-we} \\
			{\sc A}-\EM{dog}-run-\ab{pres} \\
			`The dog is running.'
			\hfill{(agent S)}
		\b.[] \hfill{\cite{allenetal84,baker88}}

In Japanese,
\citeA{kageyama93} reports that
patient S and P (in his terminology, internal arguments) are widely incorporated into verbs and form noun-\isi{verb} compounds.
He also reports the existence of agent S and A (external arguments) incorporated into verbs,
but claims that they are exceptional.
%%% 例を追加
The hierarchy of \isi{noun incorporation} \ref{NIhierarchy} is similar to the hierarchy of zero-marking in Japanese.
This is because
they are both hierarchies based on focus structure (see also \S \ref{Disc:HardConst}).

Finally, VOB and the information-structure \isi{continuity principle} with correlating features of \isi{information structure} \ref{ISFeatures} predict that
cross-linguistically,
P and V appear together most frequently.
Table \ref{wals-81} shows the order of subject (S in the table, A in our terminology), object (O in the table, P in our terminology), and \isi{verb} \cite{wals-81}.
``[O]ne order is considered dominant if text counts reveal it to be more than twice as common as the next most frequent order; if no order has this property, then the language is treated as lacking a dominant order for that set of elements '' \cite{wals-s6}.
The table shows that
SOV and SVO are the most popular dominant \isi{word order}s among all other possibilities as predicted,
while the next popular order is VSO,
which is against our prediction.
However, note that in deciding which \isi{word order} is dominant in a language,
Dryer included only
``a \isi{transitive clause}, more specifically declarative clauses in which both the subject and object involve a noun (and not just a \isi{pronoun})'' \cite{wals-81}.
Therefore, this dominant \isi{word order} might not be that of \isi{predicate-focus structure}.
 Since both of the full noun phrases can be new,
the clause have a sentence-focus structure.
\citeA{dryer97} \cite[as well as][]{wals-81} points out that
\isi{transitive} clauses with full lexical nouns do not occur frequently;
it is more common that
one of the two arguments is \isi{pronominal},
which is more likely to have a \isi{predicate-focus structure}.
For now,
a cross-linguistic examination of word orders controlling \isi{information structure} is very difficult
and I leave this problem for future studies.

\begin{table}
%\begin{minipage}{0.5\textwidth}
\centering
\caption{Order of subject, object, and verb \cite{wals-81}}
\label{wals-81}
\begin{tabular}{lr}
	\lsptoprule
	Word Order & \# of Lgs \\
	\midrule
	SOV &	565 \\
	SVO &	488 \\
	VSO &	95 \\
	VOS &	25 \\
	OVS &	11 \\
	OSV &	4 \\
	No dominant order &	189 \\
	\lspbottomrule
\end{tabular}
\end{table}



% 「太郎が本は読んでるよ」ContrastiveならOKの理由


%%%----------------------------------------------------
%\subsection{Kuno's information order theory}

%%% 「[太郎は]T [何回]F [ヨーロッパに行った]Tの?」




%%----------------------------------------------------
%%----------------------------------------------------
\section{Summary}

%%----------------------------------------------------
\subsection{Summary of this chapter}

This chapter analyzed associations between \isi{word order} and \isi{information structure} in spoken Japanese.
I made it clear that
shared topics appear clause-initially,
while strongly evoked topics appear post-predicatively.
Also, new, i.e., focus, elements appear immediately before the predicate.
Based on these findings,
I proposed the information-structure \isi{continuity principle},
in addition to the from-old-to-new principle and the persistent-element-first principle.


%%----------------------------------------------------
\subsection{Remaining issues}

As I briefly discussed in \S \ref{WO:Dis:Confounding},
\isi{information structure} is not the only feature contributing to \isi{word order} in spoken Japanese.
It is necessary to employ statistical analyses including other features to investigate the effect of information structure..











