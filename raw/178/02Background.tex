% !TEX root = ../main.tex
\chapter{Background}\label{Background}

%%% Stalnaker Assertion re-visited.
%%% assertionとは、可能世界の排除
%%% domain (\isi{topic}) に関して何かをpredicate (focus) する (Chafe)
%%% contrastive topicはdomainの選択
%%% assertionよりも前に行われる

%%% E.Kissにも触れる
%%% 砂川、対比のハのformal analysis

%%% 誰が大金持ちですか? -- ビル・ゲイツは大金持ちです(Kuroda, 2005: p.7)
%%% -> among othersの「は」
%%% 「1例を挙げるなら」
%%% Presupposition = [X is rich.]
%%% Xの例をあげよ

%%% 誰がパーティーに来たの? -- 太郎「は」来たよ
%%% ちゃんと質問に答えていない

%%% focusとexhaustive listingの関係を論じる
%%% 完全なexhaustive listingの答えが要求される
%%% 不完全な答えは会話の公準違反


%%----------------------------------------------------
%%----------------------------------------------------
\section{Introduction}


%\begin{itemize}
%	\item Focus is an element by which the information in assertion is different from the \isi{presupposition} in a given context.
%	\item An element which is unpredictable or unrecoverable.
%		\cite[][121]{lambrecht94}
%\end{itemize}
%
%
%\ex. {Predicate focus}
%	\a.[Q:] What did the children do the next?
%	\b.[A:] {The children went to SCHOOL.}
%
%\ex. {Argument focus}
%	\a.[Q:] Who went to school?
%	\b.[A:] {The CHILDREN went to school.}
%
%\ex. {Sentence focus}
%	\a.[Q:] What happened?
%	\b.[A:] {The CHILDREN went to SCHOOL.}
%
%\ex. {Zero focus} \\
%	John was very busy that morning. After {the children went to SCHOOL},
%	he had to clean the house and go shopping for the party.

This chapter provides an overview of various definitions of (or notions frequently associated with) topics (\S \ref{BackSecTopic}) and foci (\S \ref{BackSecFocus}).
In each section,
I first introduce the definition of topic and focus used in this study.
Then I review the literature.
%The characterizations of topics and foci proposed on the literature are based on either meaning or form.
%I argue that almost all the characterization based on meaning consist of features of topics, rather than they are mutually exclusive.
%I do not employ characterization based on form because this paper is attempting to demonstrate the association of forms (particles, \isi{word order}, and intonation) with \isi{information structure}.
%If \isi{topic} and focus are identified based on some linguistic form,
%the goal of this paper would be ruined because the conclusion is circular.
Topic is roughly equivalent to ``psychological subject'' \cite{gabelentz69}, ``theme'' \cite[e.g.,][]{danes70,halliday04}, ``ground'', ``background'', and ``link'' \cite{vallduvi94},
although there are many (sometimes crucial) differences among these notions.
In the same manner,
focus is roughly equivalent to ``psychological predicate'', ``rheme'', ``foreground'', and ``comment''.
\citeA{gundel74} and \citeA{kruijff-korbayovasteedman03} provide a useful summary of the history of these notions.

In reviewing the literature,
I emphasize two aspects:
the importance of the definition of topic and focus proposed in the study and, at the same time,
the heterogeneous characteristics of these notions.
The present study argues that \isi{topics} and foci in different languages form prototype categories composed of various features that are present to different degrees. This position is similar to \citeA{firbas75} and \citeA{givon76},
who viewed \isi{topic} as a gradient notion,
although the features they propose are not exactly the same.
Also, I assume a single flat layer of \isi{information structure} with multiple features, rather than the multiple layers assumed by many researchers (such as the topic-comment vs.~focus-background layers).
%rather than the multiple layers,
%Also, I only assume a single layer of \isi{information structure}
%rather than the multiple layers, such as the topic-comment vs.~focus-background layers. While many researchers hypothesize multiple layers of \isi{information structure},
%I instead assume a flat layer of \isi{information structure} with multiple features.


Finally, in \S \ref{BackSecCharJap} 
I review the literature on Japanese particles, \isi{word order}, and intonation.


%%----------------------------------------------------
%%----------------------------------------------------
\section{Topic}\label{BackSecTopic}

In this section, I give a brief overview of the definitions of \isi{topic}.
The notion of \isi{topic} is controversial and has a complicated history.
I classify these complicated notions into several representative categories in the following subsections.
Before the overview, I first introduce the definition of \isi{topic} assumed in this study to make the discussion clearer.

%%----------------------------------------------------
\subsection{The definition of topic in this study}\label{BackSubsecDefTopic}

%%% Gundel (1988) \isi{topic} familiarity conditionを引用
%%% Gundel (1985) も見る

%%% predictability, recoverability

%%% Communicative Dynamism
%%% Firbas (1966: 270) "By the degree of CD carried by a sentence element we understand the extent to which the sentence element contributes to the development of the communication, to which it 'pushes the communication forward', as it were."

%%% Mathesius (1982): theme: represents 'what is known or at least obvious in a given situation and from which the speaker proceeds in his \isi{discourse}' -> shared の支持に利用

Since I assume that \isi{information structure} is a cognitive notion, I define \isi{topic} from a cognitive standpoint.
The definition is stated in \Next.
%
\ex. \label{BackDefTopic} The topic is a \isi{discourse} element that the speaker assumes or presupposes to be shared (known or taken for granted) and uncontroversial in a given sentence both by the speaker and the \isi{hearer}.

This definition follows and elaborates the idea of topics (\ci{daimoku-tai} `\isi{topic} form') in \citeA{matsushita28},
who states that ``the theme of judgement [\isi{topic}] should not be changed before the judgement'' (p.~774, translated by NN).
Also, he states that the \isi{topic} is ``determinate'' (p.~775).

In terms of the given-new taxonomy proposed by \citeA{prince81}, shown in \Next,
topics defined in \Last include unused, declining (to be discussed below), \isi{inferable}, and evoked elements \cite[\S 4.4.2]{lambrecht94}.%
 \footnote{
 Inferable elements are further divided into containing and non-containing, and
 evoked elements are divided into textually and situationally evoked.
 I omit these distinctions since they are irrelevant to the discussion.
 }
%
By the statement that topics are ``shared'',
I mean that topics are either unused, declining, \isi{inferable}, or evoked.
\vspace{0.5cm}
%
\ex.\label{Back:Top:DefTop:GNTaxonomy}

\resizebox{.95\textwidth}{!}{ \small \Tree [.{Assumed Familiarity} [.New [.Brand-new Unanchored Anchored ] Unused ] [.\EM{Declining} ] [.Inferable ] [.Evoked ] ]}

\hfill{\cite[modified from][237]{prince81}}
\vspace{0.5cm}

A new element refers to an entity first introduced by the speaker into the \isi{discourse};
in other words, ``[the speaker] tells the \isi{hearer} to `put it on the counter''' \cite[235]{prince81}.
A brand-new element refers to a new entity that ``the \isi{hearer} may have had to create'' (ibid.).
There are two types of brand-new elements: anchored and unanchored.
``A \isi{discourse} entity is Anchored if the NP representing it is linked, by means of another NP, or `Anchor', properly contained in it, to some other \isi{discourse} entity'' (op.cit.: 236).
According to Prince, ``\ci{a bus} [...] is Unanchored, or simply Brand-New, whereas \ci{a guy I work with} [...], containing the NP \ci{I}, is Brand-new Anchored, as the \isi{discourse} entity the \isi{hearer} creates for this particular guy will be immediately linked to his/her \isi{discourse} entity for the speaker'' (ibid.).
An unused element refers to an entity ``the \isi{hearer} may be assumed to have a corresponding entity in his/her own model and simply has to place it in (or copy it into) the discourse-model'' (ibid.) such as \ci{Noam Chomsky}.
An NP refers to an evoked entity ``if [the] NP is uttered whose entity is already in the discourse-model, or `on the counter''' (ibid.).
``A \isi{discourse} entity is Inferable if the speaker assumes the \isi{hearer} can infer it, via logical -- or, more commonly, plausible -- reasoning, from \isi{discourse} entities already Evoked or from other Inferables'' (ibid.).

%In addition, I put declining elements in the taxonomy.
%CB, version 1 + reference (does not make much sense): 
In addition, I include what I call ``declining elements" \citep{Prince1981} in the taxonomy. A declining element refers to an entity which has been mentioned a while ago but is assumed to be declining in the \isi{hearer}'s mind because it has not been referred to for a while.
Declining elements are assumed to be in a \isi{semi-active state} in terms of \citeA{chafe87,chafe94}.
The referents of declining elements are in a \isi{semi-active state} especially through ``deactivation from an earlier active state'' \cite[29]{chafe87}.
Chafe's concept of ``semi-active'' also includes \isi{inferable} entities. I introduce a new term in order to distinguish declining from \isi{inferable} entities. 
%Since I want to distinguish \isi{inferable} from declining,
%I introduce a new term.

Note that the condition that the speaker assumes the element to be shared is a necessary but not a sufficient condition of \isi{topic};
\isi{topic}s are assumed by the speaker to be shared with the \isi{hearer},
but it is not necessarily the case that
all shared elements are topics.
The \isi{topic} element must also be assumed to be uncontroversial,
and I argue that this is a necessary and sufficient condition for \isi{topic},
(see \S \ref{FrameworkTopic} for details).

Also note that the definition of \isi{topic} in \ref{BackDefTopic} includes the heterogeneous elements in \Last.
%For example, topics in (\ref{BackDefTopic}) include so-called given elements, 
Therefore, definition \ref{BackDefTopic} does not necessarily contradict the definitions proposed in the previous literature.
Rather, it includes many of the previous definitions and restates them in terms of a cognitive viewpoint.

In the following sections, I provide a brief overview of different notions of topic proposed in the previous literature, and compare them with the notion I propose in the present study.

%%----------------------------------------------------
\subsection{Aboutness}

One of the most representative definitions of \isi{topic} is that
a \isi{topic} is what the sentence is about.
This definition is employed by various linguists such as \citeA{matsushita28,kuno72,gundel74,reinhart81,dik78,lambrecht94}; and \citeA{erteschik-shir07}.
%%%ISSUE Something wrong with bib file? The closing parenthesis has no link.
Topics as things under discussion \cite[e.g.,][]{heycock08} are also classified here.
Here I will discuss \citeA{reinhart81} because it is one of the most detailed and influential works.
% Topic as under discussion (Heycock, 08: p.65ff.)

\citeA{reinhart81},
inspired by \citeA{strawson64},
posits that topics should be characterized in terms of \EMt{aboutness}.
More precisely,
``an expression will be understood as representing the \isi{topic}
if the assertion is understood as intending to expand our knowledge of this \isi{topic}'' \cite[59]{reinhart81}.%
  \footnote{
  Although Reinhart's definition of \isi{topic} is basically from Strawson,
  the discussion in this work is based on \citeA{reinhart81}.
  This is because she notes that her ``presentation of [the criteria of topics] may not be fully loyal to [Strawson's] original intentions'' since ``[Strawson's] criteria are introduced in a rather parsimonious manner'' (59).
  }
Moreover, the truth value of a sentence is assessed with respect to the \isi{topic} ({ibid.}).
She proposes some tests to identify a \isi{topic} in a sentence.
The first one is an \ci{as for}/\ci{regarding} test;
an expression X is a \isi{topic} if it is felicitously paraphrased as \{\ci{as for}/\ci{regarding}\} X (p.~63, see also \citet{kuno72,kuno76,gundel74}).
Therefore, \ci{Matilda} in \Next[a] and \ci{your second proposal} \Next[b] are topics.
%
\ex. \a. \EM{As for Matilda}, she can't stand Felix.
     \b. \EM{Regarding your second proposal}, the board has found it unfeasible.
     \b.[] \hfill{\cite[59]{reinhart81}}

As she cautions, however,
not all topics can be identified in this way
because \ci{as for} and \ci{regarding} are typically used to change the current \isi{topic} \cite{keenanschieffelin76,durantiochs79}.
For example, \ci{as for this book} in \Next is awkward even though it is clearly a \isi{topic}.
This is because the book has already been the \isi{topic} of the previous sentence.
%
\ex. \EMi{Kracauer's book} is probably the most famous ever written on the subject of the cinema.
 ??\EM{As for this book}, many more people are familiar with its catchy title then[sic] are acquainted with its [turgid] text.
 \hfill{\cite[64]{reinhart81}}

Therefore, she proposes a ``more reliable test'' (ibid.),
which embeds the sentence in question in \ci{about} sentences.
This is exemplified in \Next,
where the book is correctly identified as a \isi{topic}.
%
\ex. He said \EM{\{about/of\} the book} that many more people are familiar with its catchy title than are acquainted with its turgid text.
  \hfill{(op.~cit., 65)}


To formalize this intuition, Reinhart introduces the notion of possible pragmatic assertions.
It is assumed that ``each declarative sentence is associated with a set of possible pragmatic assertions (PPA), which means that that sentence can be used to introduce the content of any of these assertions into the context set'' (p.~80).
The context set of a given \isi{discourse} at a given point is a set of propositions that both the speaker and the \isi{hearer} have accepted to be true at that point \cite{stalnaker78}.
The set of PPA's of a given sentence S is defined in \Next,
where $\phi$ indicates the proposition expressed by S.
%
\ex. \label{BackExPPA} PPA$_{(S)}$ = $\phi$ together with [$<\alpha,\phi>$: $\alpha$ is the interpretation of an NP expression in S]
   \hfill{\cite[80-81]{reinhart81}}

Assuming \Last, the \isi{topic} expression of a sentence S in a context C
is defined as in \Next.
%
\ex. \label{BackExAboutness} Topic is ``the expression corresponding to $\alpha_{i}$ in the pair $<\alpha_{i},\phi>$ of PPA$_{(S)}$ which is selected in C''.
    \hfill{(op.~cit., 81)}

This is achieved in the following steps:
(i) ``if possible, the proposition $\phi$ expressed in S will be assessed by the \isi{hearer} in C with respect to the subset of propositions already listed in the context set under $\alpha_{i}$'', and
(ii) ``if $\phi$ is not rejected it will be added to the context set under the entry $\alpha_{i}$'' ({ibid.}).

Since this definition of \isi{topic} in terms of \isi{aboutness} is attractive and seems to coincide with our intuition,
many linguists adopt it \cite[e.g.,][]{lambrecht94,erteschik-shir07}.
However, I do not employ this definition
even though my criteria for topics in \ref{BackDefTopic} and Reinhart's \ref{BackExAboutness} are apparently very similar, and even though the elements covered by these two definitions overlap most of the time.
Given that I am interested in finding \isi{topic} expressions in corpora,
\isi{aboutness} is not clear enough for my purpose.
For example, \citeA{vallduvi94} presents the following hypothetical mini-conversation between a newly-appointed White House butler (H$_{1}$) and the Foreign Office Secretary after returning from a trip to Europe (S$_{0}$).
%
\ex. \a.[H$_{1}$:] I am arranging things for the president's dinner. Anything I should know?
     \b.[S$_{0}$:] Yes. [The president]$_{TOP}$ [hates the Delft china set]$_{FOC}$.\\
     \begin{flushright}
     {\cite[9, 12]{vallduvi94}}
     \end{flushright}

In this example, Vallduv\'{\i} identifies \ci{hates the Delft china set} as focus; however, it passes the \ci{about} test as shown in \Next.
%
\ex. The Foreign Office Secretary said \EM{about the Delft china set} that the president hates it.
%%% 要確認

Since I am assuming that topics are in complementary distribution with focus elements,
the element in question is not a focus if it is a \isi{topic}, and vice versa.

On the other hand, the \ci{no}- and \ci{aha}-tests proposed in \S \ref{FrameworkTopic} correctly identify \ci{the president} as a \isi{topic} and \ci{the Delft china set} as a focus.
As shown in \Next[H$_{2}$] and \NNext[H$_{2}$],
the \isi{topic} \ci{the president} cannot be argued against or repeated as news,
whereas the focus \ci{the Delft china set} can.
%
\ex. \a.[H$_{1}$:] I'm arranging things for the president's dinner. Anything I should know?
     \b.[S$_{0}$:] Yes. [The president]$_{TOP}$ [hates the Delft china set]$_{FOC}$.
     \b.[H$_{2}$:] ?No, \EM{the first lady} hates the Delft china set.
     \b.[H$_{2}^{\prime}$:] No, the president hates \EM{Rockingham Pottery}.

\ex. \a.[H$_{1}$:] I'm arranging things for the president's dinner. Anything I should know?
     \b.[S$_{0}$:] Yes. [The president]$_{TOP}$ [hates the Delft china set]$_{FOC}$.
     \b.[H$_{2}$:] ?Aha, \ci{the president}.
     \b.[H$_{2}^{\prime}$:] Aha, \EM{the Delft china set}.
%%% 要確認


Therefore, I conclude that
the definition in \ref{BackDefTopic} identifies \isi{topic}s better than the \isi{aboutness} test,
even though \isi{aboutness} captures some aspects of our intuition about topics.


%%----------------------------------------------------
\subsection{Evokedness}\label{BackEvoked}

Evoked information is commonly called ``given'' or ``old'' information.
However, as pointed out in \citeA{prince81},
the terms ``given'' and ``old'' are too ambiguous.
Following Prince,
I use the term ``evoked information'' for a referent that has been mentioned in the previous \isi{discourse} or has been physically present in the speaker's and \isi{hearer}'s attention
and hence ``in the consciousness of the addressee [(or the \isi{hearer})] at the time of \isi{utterance}'' \cite[30]{chafe76}.
The term ``the focus (center) of attention'', ``\isi{anaphoric}'', ``predictable'' \cite{kuno72}, and ``active'' \cite{portner07} are understood in the same way.

Most researchers agree that evoked information is not the \isi{topic} itself \cite[\EMt{inter alia}]{reinhart81,gundel88,lambrecht94}.
As it is well known, evoked elements can be a focus instead of a \isi{topic}, as shown in \Next[B].
%


\newpage
\ex.\label{BackExHimself} \a.[A:] Who did Felix praise?
     \b.[B:] [Felix praised]$_{TOP}$ [himself.]$_{FOC}$
     \b.[] \hfill{\cite[72, style modified by NN]{reinhart81}}

In \Last[B], it is obvious that \ci{himself} is evoked information
since the referent is mentioned in the previous context as well as the sentence itself.
At the same time, it is a \isi{focus} because
it is the answer to the \ci{wh}-question (see also the discussion on \isi{focus} in \S \ref{BackSecFocus} below).
Given that foci cannot be topics,
\ci{himself} in \Last[B] is not a \isi{topic}.

Moreover, as has been pointed out by many scholars \cite[see][\EMt{inter alia}]{li76,givon83,halliday04},
topics are frequently evoked, but this is not always the case.

%%----------------------------------------------------
%\subsection{Communicative dynamism}

%%%----------------------------------------------------
%\subsection{Topic is the focus of attention}
%
%Topics are the center or focus of attention of the \isi{hearer}
%\cite{schachter73,garcia75}.
%
%As argued in \citeA{reinhart81},
%focus can be characterized exactly in the same way.
%For example, 
%\citeA{erteschik-shirlappin79} employs the term focus
%as something to draw the \isi{hearer}'s attention.
%Therefore, it is confusing to use this as the definition of \isi{topic}.
%

%%----------------------------------------------------
\subsection{Subject}

As pointed out in \citeA{li76},
topics are frequently, but not always, subjects.
For example, the whole \isi{utterance} in \Next[a-d] can be the answer to the question ``what happened?'',
indicating that the subjects in these utterances are part of the focus,
and therefore cannot be a \isi{topic}.
%
\ex.
  \a.[] What happened?
  \b. [A man shot a lion.]$_{FOC}$
  \b. [It is snowing.]$_{FOC}$
  \b. [Someone came in.]$_{FOC}$
  \b. [The Mets beat the A's.]$_{FOC}$
  \b.[] \hfill{\cite[49, modified by NN]{gundel74}}


Topics are not always subjects, either.
Objects and other elements can also be topics.
In \Next,
the object of each sentence is a topic.
The \isi{information structure} is annotated by the author; note, however, that a context would be necessary to clarify the \isi{information structure} in this example. 
%It is necessary to specify the context to determine the detailed \isi{information structure}.
%
\ex.
 \a. [Beans]$_{TOP}$ he won't eat.
 \b. [As for that dress]$_{TOP}$, I promise I won't wear [it.]$_{TOP}$
 \b. (What about) [beans]$_{TOP}$, does he like [them?]$_{TOP}$
 \b.[] \hfill{\cite[27, modified by NN]{gundel74}}

However, it is also important to note that topics are frequently subjects \cite{li76}.


%%----------------------------------------------------
\subsection{Sentence-initial elements}

\citeA{chomsky65} and \citeA{halliday67} characterize \isi{topic}s as the sentence-initial element
(more recently, see \citeA{hajicovaetal00}).
To define the \isi{topic} in terms of linguistic form pre-empts the goal of this study, namely, to figure out the association between information structures (\isi{topic} and \isi{focus}) and linguistic forms (particles, \isi{word order}, and intonation).

Moreover, there are cases where sentence-initial elements are not topics.
For example, the sentences in \LLast in the last section are topicless, meaning that the sentence-initial elements cannot be topics. Conversely, topics do not always appear sentence-initially:
%
\ex. (What about the proposal?) -- [Archie rejected]$_{FOC}$ [\{it/the proposal\}.]$_{TOP}$

We will examine topics which appear after the predicate
in Chapter \ref{WordOrder}.
As will be discussed, topics frequently appear sentence-finally in casual spoken Japanese and in many other languages, and in this position have their own characteristics.
%We will see
%topics which appear after the predicate
%in Chapter \ref{WordOrder}.
%As will be discussed in Chapter \ref{WordOrder},
%topics frequently appear sentence-finally in casual spoken Japanese and many other languages;
%and post-predicate topics have their own characteristics.

%%----------------------------------------------------
%\subsection{Unstressed elements}\label{BackSecUnstressed}
%
%%%% 要確認
%Topics are assumed to be unstressed elements in \citeA{chomsky96}.
%However, \citeA{buring99} finds many examples of stressed (``accented'') topics in \ili{English} and \ili{German} (see also \citeA{steedman00}).
%As will be discussed in Chapter \ref{Intonation},
%not having a stress is a property of evoked elements rather than topics in general.
%As has been discussed in \S \ref{BackEvoked} above,
%not all evoked elements are topics and not all topics are evoked,
%although this is frequently the case.

%----------------------------------------------------
%\subsection{\textit{Wa}-coded element}
%
%In the tradition of Japanese linguistics,
%topics are simply assumed to be \ci{wa}-coded elements or ``what the sentence is about'';
%the exact characteristics of topics have rarely been discussed \cite[e.g.,][]{mikami60,noda96,kijutubumpokenkyukai09}.
%However, as stated in the sections above,
%since this study aims to find the association between \isi{information structure} and linguistic forms,
%it is problematic to define \ci{wa}-coded elements as topics \EMt{a priori}.
%Moreover, there are many \isi{topic} markers including \ci{toiuno-wa}, \ab{cop}-\ci{kedo}, and the zero particles,
%but the difference among these are not very clear.
%
%In \citeA[254]{kijutubumpokenkyukai09}, for example,
%\ab{cop}-\ci{kedo} (literally `\ab{cop}-although') is reported to ``be used to introduce a new element that has not been a \isi{topic} of the current \isi{discourse},''
%which appears to be a reasonable description.
%They also point out that one cannot use \ci{wa} instead of \ab{cop}-\ci{kedo}.
%However, there is no explanation of why \ci{wa} is not natural:
%i.e., what characteristics of \ci{wa} prohibits \ci{wa} to be used in this context.
%A unified framework that accounts for all \isi{topic} markers is necessary.
%

%%----------------------------------------------------
%\subsection{Notes on ``contrastive topic''}


%%----------------------------------------------------
%%----------------------------------------------------
\section{Focus}\label{BackSecFocus}

In this section, I review different definitions of \isi{focus}, as well as notions closely associated with it.
Like \isi{topic}, \isi{focus} is also a controversial notion and the literature disagrees on its definition as well as its properties.
In the following subsections, I again classify different definitions of \isi{focus} into representative groups, but discuss my own definition of the term first for clarity. 
%Here I again categorize different notions of \isi{focus} into several representative groups in the following subsections.
%But first, I introduce my definition of \isi{focus} in order for the discussion to be clear.
%Then, I give an overview of each definition of \isi{focus} in the literature.

%%----------------------------------------------------
\subsection{The definition of focus in this study}\label{BackSubsecDefFocus}

Since I try to capture phenomena of \isi{information structure} in a single layer,
I believe that \isi{topic} and focus should be mutually exclusive rather than overlapping with each other,
as has been mentioned above.
Therefore, I define the notion of focus as in \Next
(see also the discussion in \S \ref{FrameworkFocus}).
%
\ex. The focus is a \isi{discourse} element that the speaker assumes to be news to the \isi{hearer} and possibly controversial.
S/he wants the \isi{hearer} to learn the relation of the \isi{presupposition} to the focus by his/her \isi{utterance}.
In other words, focus is an element that is asserted.
%The speaker typically does not assume the element to be shared (known or taken for granted) with the \isi{hearer}.
\label{BackFocDef}

Like \ref{BackDefTopic},
this definition also follows and elaborates the idea of focus (\ci{heisetsu-tai} `plain form') in \citeA{matsushita28}.
He states that ``whereas the theme of judgement [\isi{topic}] should not be changed before the judgement, materials to be used for the judgement [focus] are indeterminate, variate, and free since the speaker uses these materials at his/her own choice'' (p.~774, translated by NN).

I believe the statement that the speaker ``wants the \isi{hearer} to learn the relation of the \isi{presupposition} to the \isi{focus}'' in \ref{BackFocDef} is essentially the same as the definition of comment in \citeA{gundel88},
which states as follows.
%
\ex. A \isi{predication}, P, is the comment of a sentence, S, iff in using S the speaker intends P to be assessed relative to the \isi{topic} of S.
     \hfill{\cite[210]{gundel88}}

\citeA{lambrecht94} \cite[based on][]{halliday67} also employs the same definition of focus as stated in \Next.
%
\ex. [T]he focus of a sentence, or more precisely,
  the focus of the proposition expressed by a sentence
  in a given \isi{utterance} context,
  is seen as the element of information whereby the \isi{presupposition} 
  and the assertion \EMt{differ} from each other.
  The focus is that portion of a proposition which
  cannot be taken for granted at the time of speech.
  It is the \EMt{unpredictable} or pragmatically
  \EMt{non-recoverable} element in an \isi{utterance}.
  \hfill{\cite[207, underlined by the original author]{lambrecht94}}

Unpredictability or non-recoverability \cite[see also][]{kuno72} is also very similar to the definition in \ref{BackFocDef}.

I use the term \EMt{assertion} in the sense of \citeA{stalnaker04}.
He argues that, among possible worlds, a single world is chosen by the assertion.
I consider this to be equivalent to ``being news to the \isi{hearer}.''
The reason why I do not simply say ``focus is the element being asserted'' is that
to single out a world from many possible worlds might be confused with \isi{contrastiveness}.
As will be discussed in \S \ref{Back:Foc:Contr},
focushood and \isi{contrastiveness} are similar but different notions.

As has been pointed out in many studies \cite[e.g.,][]{matsushita28,chomsky65,gundel74},
the answer corresponding to a \ci{wh}-question is a typical focus.
The following examples are from \citeA[121]{lambrecht94}.
The interpretation of \isi{information structure} is by the author
and might slightly differ from Lambrecht's original intention.
%
\ex. \label{BackLambPredFoc}{Predicate focus}
	\a.[Q:] What did the children do next?
	\b.[A:] {[The children]$_{TOP}$ [went to school.]$_{FOC}$}

\ex. \label{BackLambArgFoc}{Argument focus}
	\a.[Q:] Who went to school?
	\b.[A:] {[The children]$_{FOC}$ [went to school.]$_{TOP}$}

\ex. \label{BackLambAllFoc}{Sentence focus}
	\a.[Q:] What happened?
	\b.[A:] {[The children went to school.]$_{FOC}$}


Focus is news \cite[or newsworthy in][]{mithun95} for the \isi{hearer} and can be repeated as what s/he learned from the current \isi{utterance}.
For example, in \Next,
the \isi{topic} \ci{John} in \Next[A] cannot be repeated as news by B,
whereas (part of) the focus \ci{teacher} can be repeated by B$^{\prime}$.
%
\ex. \a.[A:] [\{As for/Regarding\} John]$_{TOP}$, [he]$_{TOP}$ [is a teacher]$_{FOC}$.
     \b.[B:] ??Aha, \EM{John}.
     \b.[B$^{\prime}$:] Aha, \EM{a teacher}.
%%% 要確認

\ci{No} tests based on \citeA{erteschik-shir07} are also available.
See discussion in \S \ref{FrameworkFocus}.
The identfication of focus using
\ci{wh}-question-answer pairs, such as (\ref{BackLambPredFoc}--\ref{BackLambAllFoc}), or the \ci{aha} test \Last
rests on the assumption that
foci are news or newsworthy,
while \ci{no} tests like \ref{BackExJohn2} in \S \ref{FrameworkFocus} are based on the assumption that
foci can be controversial.

In the following sections,
I review various notions associated with foci
and how they relate to the discussion of foci in the present work.

%%----------------------------------------------------
\subsection{Newness}

Newness is known to correlate with focushood \cite[\EMt{inter alia}]{li76,givon83,halliday04}.
Although different researchers use the term \ci{new} to refer to different concepts,
I use this term to indicate strictly ``new'' in terms of \citeA{prince81} or ``what the speaker assumes he is introducing into the addressee's consciousness by what he says'' \cite[30]{chafe76}.
Other \isi{newness}, what is called ``relational new'' in \citeA{gundel88},
is excluded from the current discussion.
According to \citeA[177]{gundelfretheim06}, relational \isi{newness} is described as follows.
%
\ex. Y [focus] is new in relation to X [\isi{topic}] in the sense that
     it is new information that is asserted, questioned, etc.~about X.
     Relational [...] \isi{newness} thus reflects how the informational content of a particular event or state of affairs expressed by a sentence is represented and how its truth value is to be assessed.

The notion of ``relational new'' corresponds to focus in this study and the notion of comment in \citeA{gundel88}.
%%% 不要?


The literature agrees that
not all foci are new.
As discussed in \S \ref{BackEvoked},
focus can be an evoked element.
\ref{BackExHimself}, repeated here as \Next,
is an example of this case;
\ci{himself} in \Next[B] is evoked because the referent ``Felix'' has already been mentioned in the preceding \isi{utterance} \Next[A],
and, at the same time, it serves as focus because it corresponds to the answer part of the \ci{wh}-question in \Next[A].
%
\ex. \a.[A:] Who did Felix praise?
     \b.[B:] [Felix praised]$_{TOP}$ [himself.]$_{FOC}$
     \b.[] \hfill{\cite[72, style modified by NN]{reinhart81}}


On the other hand,
all new elements can be foci.
It is well known that, in \ili{English}, (specific or non-generic) \isi{indefinite} noun phrases cannot be topics.
For example, \citeA{gundel74}, discussing the following examples,
concludes that \isi{indefinite} noun phrases cannot be topics.
As shown in \Next[a] and \NNext[a],
\isi{indefinite} noun phrases cannot be put in the frame \ci{concerning} and \ci{about};
nor can they appear in the frame \ci{what about}.
%
\ex. 
  \a. *Concerning a \ili{French} king, he married his mother.
  \b. *What about a \ili{French} king? -- He married his mother.
  \b.[] \hfill{\cite[54]{gundel74}}

\ex.
  \a. *About a lion, Bill shot him.
  \b. *What about a lion? -- Bill shot him.
     \hfill{(\EMt{ibid.})}


I argue that new elements that have been known to the \isi{hearer} before the \isi{utterance}, i.e., ``unused'' in terms of \citeA{prince81}, can be either \isi{topic}s or foci.
They are new in the sense that the speaker is introducing them into the \isi{hearer}'s consciousness by what s/he says;
but they are given in the sense that they are assumed by the speaker to be shared with the \isi{hearer}.
In Chapter \ref{WordOrder},
I argue that, in fact, unused elements have characteristics of both topics and foci.
%I am not sure whether new elements that have been known to the \isi{hearer} before the \isi{utterance}, i.e., ``unused'' in terms of \citeA{prince81}, is always foci or not.
%For example, is it natural to produce \Next out of blue
%if the speaker and the \isi{hearer} has not seen Jay for a year?
%What if the \isi{hearer} seems not to care about Jay so much
%although the \isi{hearer} met Jay yesterday?
%What if the speaker believes that the \isi{hearer} always cares about Jay so much even though the \isi{hearer} has not seen him for a long time?
%%
%\ex. \a. Concerning Jay, he finally got a PhD!
%     \b. About Jay, he got married!
%%%% 要確認
%
%I leave the question open whether ``unused'' elements can be topics out of blue.
%Some languages might allow the speaker to express them as topics,
%whereas other languages might not.
%In Chapter \ref{Particles},
%I argue that in Japanese ``unused'' elements can be topics
%when the speaker believes that the \isi{hearer} has the referent in mind.

%\citeA{hornby71}
%"The part of the sentence which constitutes what the speaker is talking about is being called the \isi{topic} of the sentence in the preset work.
%The rest of the sentence, the comment, provides new information about the \isi{topic}."


%%----------------------------------------------------
\subsection{Contrastiveness}\label{Back:Foc:Contr}

Many studies, particularly in generative linguistics,
associate focushood with \isi{contrastiveness} (frequently accompanied by a \isi{pitch peak}).
Here I base my discussion on \citeA{rooth85,rooth92},
who was inspired by \citeA{vonstechow91},
since his theory is one of the most influential studies on focus as contrastiveness.

\chd{In his theory, alternative semantics,
where focus is related to the intuitive notion of contrast},
Rooth argues that the function of focus is to evoke alternatives;
in other words,
the focus element is contrasted with the alternatives.
For example, consider \Next in two cases, one 
in which \ci{Mary} is focused and one in which \ci{Sue} is focused.
%
\ex. Mary likes Sue.

The former case evokes the set of propositions of the form `x likes Sue',
as formalized in \Next[a],
whereas the latter case evokes the set of propositions of the form `Mary likes y', as formalized in \Next[b].
%
\ex.
  \a. $\llbracket$[$_{S}$ [Mary]$_{F}$ likes Sue]$\rrbracket^{f}$ = \{\textbf{like}(x,\textbf{s}) $\mid$ x $\in$ $E$\}, where $E$ is the domain of individuals.
  \b. $\llbracket$[$_{S}$ Mary likes [Sue]$_{F}$]$\rrbracket^{f}$ = \{\textbf{like}(\textbf{m},y) $\mid$ y $\in$ $E$\}
  \b.[] \hfill{\cite[76]{rooth92}}

Among the members of these sets,
Mary is chosen as the one who likes Sue in \Last[a],
and Sue is chosen as the one who Mary likes in \Last[b].

The characterization and formalization of focus in alternative semantics is
clear and seems to work well.
However, characterizing foci as contrastive is problematic
for our assumptions:
we have assumed that \isi{topic} and focus are mutually exclusive, and yet there can be \isi{contrastive topic}s and contrastive foci,
as has been pointed out in \citeA{vallduvivilkuna98}.
Especially problematic for us is the existence of contrastive topics.
If \isi{contrastiveness} is equal to focushood,
one has to admit that a \isi{contrastive topic} is both \isi{topic} and focus.
%Although many linguists are perfectly comfortable with this,
Following \citeA{vallduvivilkuna98},
I argue that this is very confusing for a theory of \isi{information structure} and
it is more plausible to assume that \isi{contrastiveness} is a feature independent of both topichood and focushood.
For example,
as will be discussed in Chapter \ref{Particles},
the particle \ci{wa} in Japanese is sensitive to some properties of topichood,
whereas the particle \ci{ga} is sensitive to some properties of focushood.
In addition to this,
these two particles are also sensitive to \isi{contrastiveness}:
they are obligatory when contrast is involved but are optional in other cases.
Still, contrastive \ci{wa} and \ci{ga} are sensitive to topichood and focushood, respectively.
Therefore, this study
%, which studies Japanese \isi{information structure},
assumes that
\isi{contrastiveness} is independent of \isi{topic} and focus.
However, it is highly likely that other languages work differently.
Further study is needed to investigate whether \isi{contrastiveness} is independent of \isi{topic} and focus in all languages.

%%%----------------------------------------------------
%\subsection{Importance}
%
%highlighted
%
%%----------------------------------------------------
\subsection{Pitch peak}

\chd{Some studies assume that focus involves a \isi{pitch peak}.
For example, \cite[100]{chomsky96} states that
``phrases that contain the intonation center [\isi{pitch peak} in the present work] may be interpreted as focus of \isi{utterance}''.}
As \citeA[230]{gundel88} reports,
the association between \isi{pitch peak} and focus
is found in typologically, genetically, and geographically diverse languages
and concludes that this association seems to be universal.
According to her,
a focus is given a \isi{pitch peak} at least in \ili{English}, Guarani, \ili{Russian}, and \ili{Turkish}
with the only exception of \ili{Hixkaryana} \cite[see also the references in her work and][]{buring07}.%
\footnote{
See \citeA{downing12} for more exceptions.
}
%%% BolingerとかSteedmanも?

As has been pointed out in previous studies on other languages \cite[e.g., ][\S 6.2]{jackendoff72},
%and discussed in \S \ref{BackSecUnstressed},
however,
I do not employ the definition of focus as a \isi{pitch peak}
because the goal of this study is to investigate the association between \isi{information structure} and linguistic forms including intonation;
the definition of focus as a \isi{pitch peak} spoils the goal of our study.
Moreover,
I will argue in Chapter \ref{Intonation} that 
elements other than focus are given a \isi{pitch peak}.
For example, a \isi{topic} that is reintroduced in the \isi{discourse} is produced prominently \cite[see also][]{gundel99}.
It is also well known that
\isi{contrastiveness} correlates with \isi{pitch peak}.
Therefore, regarding focus as an element with \isi{pitch peak} causes great confusion.

%%----------------------------------------------------
%\subsection{Structural analysis of focus}

%Rizzi

%%% 反論は\cite{buring07}
%%% フォーカスのvacuous movementを証拠なしに仮定しなければならない
%%% backgroundも証拠なしにどかさないといけない

%%% Culicover "Topicalization, inversion, and complementizers in \ili{English} Topicalization, inversion, and complementizers in \ili{English}"


%%----------------------------------------------------
%%----------------------------------------------------
\section{Characteristics of Japanese}\label{BackSecCharJap}

In this section, I provide a rough overview of the typological characteristics of Japanese.
Most of the literature on Japanese is based on written language;
therefore, most of this section is also based on written Japanese -- except for the parts that have to do with sound, such as intonation.
I discuss differences between written and spoken Japanese where necessary.


%%----------------------------------------------------
\subsection{General characteristics}\label{BackSubSecGeneralChar}

Japanese is an SOV language, with typical OV characteristics
in terms of \citeA{dryer07}:
it has postpositions (which are called particles in this study),
genitives precede nouns,
\isi{adverbial} subordinators appear after the verb,
main verbs precede auxiliary verbs,
question particles and complementizers appear after the verb,
subordinate clauses precede main clauses, and
relative clauses precede nouns
\cite{shibatani90,masuokatakubo92}.
Moreover,
nouns are preceded by adjectives and demonstratives,
and verbs are followed by many kinds of suffixes indicating tense, modality, negation, passive voice, causativity, and so on.
\Next shows some examples of Japanese sentences.
``A'' stands for the \isi{agent-like argument} of \isi{transitive} clauses;
``S'' stands for the only argument of \isi{intransitive} clauses; and
``P'' stands for the patient-like argument of \isi{transitive} clauses.
%
\ex.
     \ag. taroo-ga hanako-ni hon-o yat-ta \\
        Taro-\ab{nom} Hanako-\ab{dat} book-\ab{acc} give-\ab{past} \\
        `Taro gave a book to Hanako.' \hfill{(A + DAT + P + V)}
     \bg. sono san-nin-no ookina otoko \\
          that three-\ab{cl}.person-\ab{gen} big man \\
          `those three big men' \hfill{(Adj + N)}
     \bg. taroo-no hon \\
          Taro-\ab{gen} book \\
          `Taro's book' \hfill{(GEN + N)}
     \bg. [taroo-ga kat-ta] hon \\
           Taro-\ab{nom} buy-\ab{past} book \\
           `the book Taro bought' \hfill{(Rel + N)}
     \bg. ik-e-nai \\
          go-\ab{cap}-\ab{neg} \\
          `cannot go' \hfill{(V + SFX1 + SFX2)} \\
     \b.[] \hfill{\cite[257--258, glosses modified by NN]{shibatani90}}

The features of Japanese most relevant for this study are the order of the subject, object, and the \isi{verb} and the order of nouns and particles.
Also, as will be discussed in \ref{BackSubSecWO},
arguments such as subjects and objects can be `scrambled',
i.e., word orders other than the basic \isi{word order} are found in both spoken and written Japanese.

In written Japanese, the particles \ci{ga} and \ci{o}, which follow nouns, are considered to be a \isi{nominative} particle and an \isi{accusative} particle respectively,
and Shibatani glossed them as such.
As will be discussed below, however,
zero particles are extensively used in spoken Japanese and
the characterization of \ci{ga} as the \isi{nominative} marker and \ci{o} as the \isi{accusative marker} does not necessarily reflect the exact properties of these particles.
Since the literature is mainly based on written Japanese,
I keep the glosses of \ab{nom} for \ci{ga} and \ab{acc} for \ci{o} in this chapter.
In the same way, I will use \ab{top} for \ci{wa} since most of the literature agrees that \ci{wa} is a \isi{topic} marker (no matter what it means),
although, again, the \isi{zero particle} is extensively used in the spoken language.
However, the reader should keep in mind that the glosses in this chapter are tentative.
I will not use \ab{nom} \ab{acc}, and \ab{top} in the following chapters;
instead, I will just gloss \ci{ga}, \ci{o}, and \ci{wa} for each particle.

Japanese extensively employs so-called zero pronouns.
In \Next, for example,
pronouns such as `I', `him', and `it' are not explicitly uttered.
%
\ex.
 \ag. zyon-ga ki-ta-node, ai-ni it-ta \\
      John-\ab{nom} come-\ab{past}-since meet-\ab{dat} go-\ab{past} \\
      `Since John came, (I) went to see (him),'
 \bg. zyon-ga dekire-ba suru-desyoo \\
      John-\ab{nom} can-if do-will \\
      `If John can (do it), (he) will do (it).'
      \hfill{\cite[17]{kuno73}}

These omitted pronouns are sensitive to the \isi{information status} of the referents \cite[see][Chapter 1]{kuno78}.


The language has five vowels and 15 consonants (although the number may vary depending on the analysis).
The syllable structure is relatively simple:
a syllable basically consists of a consonant and a \isi{vowel},
where long vowels, geminates, and final nasal codas are possible.
Also, /y/ ([\tp{j}]) can appear between a consonant and a \isi{vowel}
as in \ci{kyoo} ([\tp{kjo:}]) `today' as opposed to \ci{koo} ([\tp{ko:}]) `this way'.
\isi{Pitch accent} plays an important role in Japanese.
The systems of \isi{pitch accent} vary among dialects; here I review the accent system of Standard Japanese (spoken around Tokyo),
which is the variety investigated in the present study.
First, in Standard Japanese,
the \isi{pitch} is either high or low, and
the pitches of the first and the second syllables are different.
If the first syllable is high, the second syllable is low,
and vice versa.
Second, the accent nucleus (indicated by {\tcorner}) specifies where the \isi{pitch} falls.
For example,
[\tp{ha{\tcorner}Ci}] `chopsticks' indicates that [\tp{ha}] is high and [\tp{Ci}] is low.
On the other hand, [\tp{haCi{\tcorner}}] `bridge' indicates that
[\tp{ha}] is low and [\tp{Ci}] is high.
Words without nucleus accent are also possible as in the case of [\tp{haCi}] `edge',
which is pronounced in the same way as `bridge'.
The distinction between [\tp{haCi{\tcorner}}] `bridge' and [\tp{haCi}] `edge' can be made, for example, by examining the accentless particles following them.
For example, when \ci{ga} `\ab{nom}' follows [\tp{haCi{\tcorner}}] `bridge', the \isi{pitch} of \ci{ga} is low
because the accent nucleus specifies where the \isi{pitch} falls.
On the other hand, when \ci{ga} follows [\tp{haCi}] `edge', \ci{ga} is produced in a high \isi{pitch}.
Thereby [\tp{haCi{\tcorner}}] `bridge' and [\tp{haCi}] `edge' can be distinguished from each other.
In addition to phonemes and \isi{pitch} accents, issues on intonation will be discussed in more detail in \S \ref{BackSubIntonation}, since they are one of the main topics of this study.


%%----------------------------------------------------
\subsection{Particles}\label{BackSubSecParticles}

As mentioned above,
nouns in Japanese are followed by various particles or postpositions.
In general, they are believed to be clitics and indicate the status of a noun in a clause.%
 \footnote{
 Although the equal sign (=) is usually used for \isi{clitic} boundaries,
 I use the hyphen (-) and do not distinguish clitics from affixes for the sake of simplicity.
 }
In this section, I review the literature on the particles that will be investigated in this study, namely \ci{ga}, \ci{o}, and \ci{wa}. Note again that the literature is mainly on written Japanese.
In \S \ref{BackSubSubZero},
I present a review of the literature on zero particles,
which are widely used in spoken Japanese in place of \ci{ga}, \ci{o}, and \ci{wa}.

%%----------------------------------------------------
\subsubsection{Case particles vs.~adverbial particles}

In the present study,
I discuss two kinds of particles that attach to nouns:
case particles and \isi{adverbial} particles.
Case particles such as \ci{ga} and \ci{o}
code the grammatical relations of the nouns.
For example, in \Next,
\ci{ga}, which follows the noun \ci{taroo}, codes \isi{nominative case},
whereas \ci{o}, which follows the noun \ci{hon} `book', codes \isi{accusative} case.
%
\exg.\label{ExShibatani90257}taroo-\EM{ga} hanako-ni hon-\EM{o} yat-ta \\
        Taro-\ab{nom} Hanako-\ab{dat} book-\ab{acc} give-\ab{past} \\
        `Taro gave a book to Hanako.'
  \hfill{\cite[257]{shibatani90}}


Adverbial particles, on the other hand,
sometimes follow and sometimes replace case particles
and add additional meaning to the sentence.
The \isi{adverbial particle} discussed in this study is \ci{wa}.%
 \footnote{
 There are other \isi{adverbial} particles such as \ci{mo} `also' and \ci{dake} `only',
 which also follow or replace case particles.
 As the glosses `also' and `only' suggest,
 they are translated as adverbs in \ili{English},
 which is why they are called ``\isi{adverbial}'' particles.
 }
\ci{Wa} can replace \ci{ga} and \ci{o} and turn the noun into a \isi{topic}.
It sometimes replaces and sometimes follows \ci{ni} `\ab{dat}'.
For example,
each noun in \Last can be \ci{wa}-marked in the following ways.
%
\ex. \ag. taroo-\EM{wa} hanako-ni hon-o yat-ta \\
        Taro-\ab{nom}-\ab{top} Hanako-\ab{dat} book-\ab{acc} give-\ab{past} \\
        `Regarding Taro, he gave a book to Hanako.'
 \bg. hon-\EM{wa} taroo-{ga} hanako-ni yat-ta \\
      book-\ab{top} Taro-\ab{top} Hanako-\ab{dat} give-\ab{past} \\
        `Regarding the book, Taro gave it to Hanako.'
 \bg. hanako-(ni)-\EM{wa} taroo-{ga} hon-o yat-ta \\
      Hanako-(\ab{dat})-\ab{top} Taro-\ab{top} book-\ab{acc} give-\ab{past} \\
        `Regarding Hanako, Taro gave a book to her.'

There are complex interactions between \ci{wa}-marking and \isi{word order} \cite[e.g.,][]{kuroda79},
which will be discussed in Chapter \ref{WordOrder}.


%%----------------------------------------------------
\subsubsection{\textit{Ga}}

Almost all studies agree that
\ci{ga} in contemporary Japanese is a \isi{case marker} that codes \isi{nominative case} \cite[e.g.,][]{yamada36,kuno73,tanaka77,shibatani90}.
\ci{Ga} is also said to code the ``subject'' \cite[e.g.,][164]{kuroda79}.
%which I will not discuss in detail in this study.
In addition,
it can code genitive case and the object (in terms of this study, P).
I do not introduce these usages since they are irrelevant to the present work.
See, for example, \citeA{ono75,nishida77,yasuda77,kuno73,shibatani01}.

Recent studies are more interested in the mapping between
surface form (such as \ci{ga} and \ci{o})
and the semantic (or deep) structure of predicates.
See \citeA{kondo03} for a survey of such studies.


%%%----------------------------------------------------
%\paragraph{Genitive \ci{ga}}
%
%\ci{ga} is sometimes used as a genitive marker.
%This is a residual of classic Japanese;
%in classic Japanese, \ci{ga} used to be a genitive marker,
%which gradually developed into a \isi{nominative} marker
%\cite[e.g.,][]{ono75,nishida77,yasuda77}.
%In classic Japanese,
%\ci{ga} can be used productively as a genitive marker
%as shown in \Next.
%%
%\ex.
% \ag. wa-\EM{ga} kuni \\
%      \ab{1}\ab{sg}-\ab{gen} nation \\
%      `my nation'
% \bg. ani-\EM{ga} kataki \\
%      older.brother-\ab{gen} enemy \\
%      `(my) older brother's enemy'
%      \hfill{\cite[7]{ono75}}
%
%In contemporary Japanese, however, 
%this use of \ci{ga} is unproductive and
%it is possible for \ci{ga} to attach to a very limited number of words such as \ci{wa} `\ab{1}\ab{sg}' in \Last[a].%
% \footnote{
% \ci{Wa} `\ab{1}\ab{sg}' is also fossilized and
% \ci{wa-ga} is almost a fixed expression that cannot be analyzed.
% Contemporary speakers do not use \ci{wa} to refer to themselves.
% }
%
%%%----------------------------------------------------
%\paragraph{Object marking}
%
%\citeA[373ff.]{tokieda41} proposed that \ci{ga} sometimes codes ``objects''.
%%Objective marking, I believe, can be either exhaustive listing or neutral description.
%In \Next, for example,
%\ci{ga} appears to code ``the objects'' of the predicates \ci{hanaseru} `can speak', \ci{hosii} `want', and \ci{suki-da} `be fond of', respectively.
%``The subject'' appears to be \ci{watakusi} `I'.
%%
%\ex.
% \ag. watakusi-wa eigo-\EM{ga} hanas-eru \\
%      \ab{1}\ab{sg}-\ab{top} \ili{English}-\ab{obj} speak-can \\
%      `I can speak \ili{English}.'
% \bg. watakusi-wa okane-\EM{ga} hosii \\
%      \ab{1}\ab{sg}-\ab{top} money-\ab{obj} want \\
%      `I want money.'
% \bg. watakusi-wa mearii-\EM{ga} suki-da \\
%      \ab{1}\ab{sg}-\ab{top} Mary-\ab{obj} fond.of-\ab{cop} \\
%      `I like Mary.' \hfill{\cite[79]{kuno73}}
%
%Others such as \citeA[44]{martin62} argue that
%\ci{ga}-coded nouns in cases like \Last are also ``subjects''
%because the predicate such as \ci{hosii} `want' and \ci{suki-da} `be fond.of' are adjectives rather than verbs.
%More accurately, these predicates are translated as `desirable' for \ci{hosii} and `appealing' for \ci{like}.
%The predicate \ci{hanaseru} in \Last[a] can be translated as
%`possible to speak'.
%\citeA{kuno73} argues that this analysis is peculiar because
%one has to admit that there are two subjects
%in \Next[a-b],
%where each sentence has two \ci{ga}-coded nouns.
%%
%\ex.
% \ag. dare-\EM{ga} eiga-\EM{ga} suki-desu-ka \\
%      who-\ab{nom} movie-\ab{nom} fond.of-\ab{cop}.\ab{plt}-\ab{q} \\
%      `Who likes movies?'
% \bg. watakusi-\EM{ga} eiga-\EM{ga} suki-desu \\
%      \ab{1}\ab{sg}-\ab{nom} movie-\ab{nom} fond.of-\ab{cop}.\ab{plt} \\
%      `I like movies.'
%      \hfill{\cite[80]{kuno73}}
%
%I do not step into the issues of
%what a subject of a sentence is,
%whether a sentence should have a single subject or not,
%whether all sentences should have a subject, and
%how to identify a subject in a sentence.
%For detailed discussion, see \citeA[280ff.]{shibatani90}.
%The important generalization for now is that
%the predicates of \ci{ga}-coded ``objects'' represent states, not actions \cite[81]{kuno73}.
%Also, predicates which represent psychological events or states
%have \ci{ga}-coded ``objects'' \cite[373ff.]{tokieda41}.
%As summarized in \citeA{onishi01},
%non-canonical coding of core arguments are found
%cross-linguistically with
%predicates of low transitivity and those which represent psychological events or states.
%The \ci{Ga}-coded ``object'' is one of these non-canonical coding
%and is independent of \isi{information structure}.
%See \citeA{shibatani01} for more detail on this type of \ci{ga}-coding.

%%----------------------------------------------------
\paragraph{Exhaustive listing vs.~neutral description}

\citeA{kuno73} distinguishes two types of \ci{ga}:
exhaustive listing and neutral description.
In terms of the present study,
exhaustive listing corresponds to \isi{argument focus} (or \isi{narrow focus}),
while neutral description corresponds to part of \isi{predicate focus} and sentence focus (or \isi{broad focus}),
although whether the latter \ci{ga} codes focus or not is controversial
as will be discussed below.
Examples \Next[a-b] are instances of exhaustive listing and
neutral description, respectively.
%
\ex.
 \a. \tl{Exhaustive listing}
 \bg.[] zyon-\EM{ga} gakusei-desu \\
      John-\ab{nom} student-\ab{cop}.\ab{plt} \\
      `(Of all the people under discussion) John (and only John) is a student.' \\
      `It is John who is a student.'
 \b. \tl{Neutral description}
 \bg.[] ame-\EM{ga} hutte i-masu \\
      rain-\ab{nom} fall \ab{prog}-\ab{plt} \\
      `It is raining.'
      \hfill{\cite[38]{kuno73}}

Kuno, following \citeA{kuroda79},
proposes that
\ci{ga} of neutral description can only code
the subject (As and Ss in this study) of action verbs,
existential verbs, and
adjectives/nominal adjectives
that represent changing states,
whereas \ci{ga} of exhaustive listing can attach to any kinds of nouns.
This is not the \isi{topic} of the present work,
which does not examine
the associations between \isi{information structure} and predicate type,
although this is a very important \isi{topic}.
See \citeA[Chapter 4]{masuoka00},
which extensively discusses this issue.


%%----------------------------------------------------
\paragraph{\textit{Ga} as focus marker}
%%% Kuroda (2005) では明確に否定されているらしい (Oxford Handbook)
Lastly but most importantly in the present work,
\ci{ga} is sometimes described as a \isi{focus marker}.
\ci{Ga} of exhaustive listing in \citeA{kuno73} corresponds to
\ci{ga} as a \isi{focus marker} \cite{heycock08}.
%But it seems to me that
%\ci{ga} of neutral description also express some kind of focus.
\ci{Ga} coding new (unpredictable) information \cite[Chapter 25]{kuno73j} is also related to \ci{ga} coding focus.

\citeA{noda95} classifies \ci{ga} of exhaustive-listing as focus markers, or \ci{toritate} particles,
while he argues that \ci{ga} of neutral description is a \isi{case marker}.%
 \footnote{
 \citeA{tokieda50} classifies some uses of \ci{ga} into ``particles which represent limitation'' (p.~188ff.),
 which are also close to focus markers.
 }
\ci{Toritate} can be literally translated as `taking up'
and is intended to mean `to make something remarkable'.
\ci{Toritate} particles are defined as
particles that make part of a sentence or a phrase remarkable and emphasize that part \cite[178]{miyata48}.
\ci{Toritate} particles include \ci{mo} `also', \ci{sae} `even',
\ci{dake} `only', etc.,
which are in general classified into focus markers in other languages.
Therefore, I conclude that \ci{toritate} particles\chd{, including \ci{ga} with exhaustive-listing readings,} correspond to
focus particles.%
 \footnote{
 However, many researchers also classify the so-called \isi{topic} marker 
 \ci{wa} into \ci{toritate} particles;
 some of them only include contrastive \ci{wa}  \cite{okutsu74,okutsu86,numata86},
 others include both contrastive and non-contrastive \ci{wa}
 \cite{miyata48,suzuki72,teramura81,noda95}.
 Although I do not believe that \ci{wa}, including contrastive \ci{wa}, is a \isi{focus marker},
 the notions of focushood and \isi{contrastiveness} are frequently confused,
 but should be discussed independently.
 Therefore, I regard \ci{toritate} particles as the equivalent of focus markers
 in other languages.}

\citeA{onoetal00} go further and claim that
\ci{ga} in natural conversation does not code As and Ss;
rather, they claim that
``\ci{ga} is well characterized as marking that its NP is to be construed as a participant in the state-of-affairs named by the predicate in pragmatically highly marked situations'' (p.~65).
In other words,
``\ci{ga} is found in pragmatically highly marked situations where
there is something unpredictable about the relationship between
the \ci{ga}-marked NP and the predicate such that
an explicit signalling of that relationship becomes interactionally or cognitively relevant'' (ibid.).
Although it is not perfectly clear what they mean by
``pragmatically marked situations'',
part of what they mean is that
\ci{ga} functions as a \isi{focus marker}, since they use \ci{ga} coding new or unpredictable information
as a piece of evidence that supports their claim.
In \Next[b], for example,
\ci{ga} codes the answer to the question
`what club (are you going to) join?' in \Next[a].
%
\ex.
 \ag. nani-ni hai-n-da \\
      what-\ab{dat} enter-\ab{nmlz}-\ab{cop} \\
      `What (club are you going) to join?'
 \bg. handobooru-\EM{ga} ii-kana-toka omotte [...] \\
      handball-\ab{nom} good--\ab{q}-\ab{hdg} think \\
      `(It's) handball (I want to join), (I) think.' \\
 \b.[]     \hfill{\cite[70]{onoetal00}}


%%----------------------------------------------------
\paragraph{Remaining issues}

It is indeed the case that
\ci{ga} sometimes follows nouns that are in a case that is not the \isi{nominative}, as shown in \Next.
(See Chapter \ref{Particles} for detailed discussion.)
In \Next[a], \ci{ga} follows the postposition \ci{kara} `from (\ab{abl})', meaning that the noun cannot be \isi{nominative}.
In a similar manner,
\ci{ga} follows \ci{to} `with (\ab{com})' in \Next[b] and
\ci{made} `til (\ab{lim})' in \Next[c].%
 \footnote{\ref{ExFocGa} is not acceptable for some people.}
%%% フォーカスの「が」は生産的ではない
%%% 「までが」「からが」しか言えない
%
\ex.
 \ag. kore-\EM{kara}-\EM{ga} hontoo-no zigoku-da \\
      this-\ab{abl}-\ci{ga} true-\ab{gen} hell-\ab{cop} \\
      `From this the true hell starts.'
      \hfill{(Vegeta in \ci{Dragon Ball}%
      \footnote{
      Toriyama, Akira (1990) \ci{Dragon Ball} 23, p.~149. Tokyo: Shueisha.
      }
      )}
 \bg.\label{ExFocGa}kotira-wa nihonsyu-\EM{to}-\EM{ga} au-desyoo \\
      this-\ab{top} sake-\ab{com}-\ci{ga} match-will \\
      `This one goes well with sake.'
      \hfill{(A review from \ci{Tabelog}%
       \footnote{http://tabelog.com/ehime/A3801/A380101/38006535/dtlrvwlst/2992604/, last accessed on 03/23/2015}
      )}
  \bg. ie-ni kaeru-\EM{made}-\EM{ga} ensoku-desu \\
       home-\ab{dat} return-\ab{lim}-\ab{nom} excursion-\ab{cop}.\ab{plt} \\
       `Until (you) arrive at home is the excursion. (Before you arrive at home, you are on the way of excursion.)'
       \hfill{(Common warning by school teachers)}%
       \footnote{
       I found 32,700 websites using this expression with Google exact search (searched on 06/17/2015).
       }
% ここからが本当の地獄だ (ドラゴンボールのベジータのセリフ)
% こちらは日本酒とが合うでしょう。(http://tabelog.com/ehime/A3801/A380101/38006535/dtlrvwlst/2992604/, last accessed on 03/23/2015)
% 家に帰るまでが遠足ですよ (先生がよく言うセリフ)

As will be discussed in detail in Chapter \ref{Particles},
this type of \ci{ga} codes focus rather than \isi{nominative case}.
However, it is too extreme to claim that no kind of \ci{ga} codes the \isi{nominative}.
For example, it is never possible to replace
\ci{o} in \ref{ExShibatani90257} with \ci{ga}
no matter how much \ci{hon} `book' is focalized.
It is clear that \ci{ga} sometimes codes \isi{nominative case}, sometimes codes focus, and sometimes codes both.
Also, as will be outlined below,
zero particles are extensively used in spoken Japanese.
Therefore, the question is under what conditions \ci{ga} codes focus,
under what conditions it codes \isi{nominative},
and when is \ci{ga} used instead of the zero particles.
Also, what motivates \ci{ga} to code focus?
This is not the place to discuss whether \ci{ga} codes focus or \isi{nominative case}.
%It is not appropriate to discuss whether \ci{ga} codes focus or \isi{nominative case}.
I discuss these issues in Chapter \ref{Particles}.



%%----------------------------------------------------
\subsubsection{\textit{O}}

There are fewer studies on the particle \ci{o} and,
as far as I am aware, almost all studies agree that \ci{o} is an \isi{accusative marker}
and that it codes the patient-like argument in \isi{transitive clause}s \cite[e.g.,][]{yamada36,shibatani90}.
There are two non-canonical usages of the particle \ci{o}: coding time and place of transferring \cite{yamada36}.
%In this section, I briefly review two non-canonical usages of the particle \ci{o}:


%%%----------------------------------------------------
%\paragraph{Place of transferring}
%In addition to coding P,
%\ci{O} can code the place of transferring.
%This is exemplified in \Next,
%where the places of transferring are coded by \ci{o} and treated as ``object''
%instead of being coded by other postpositions for the place of action
%such as \ci{ni}, \ci{de}, etc.
%%
%\ex.
% \ag. mon-\EM{o} deru \\
%      gate-\ab{acc} go.out \\
%      `To get out of the gate (go through the gate)'
% \bg. sora-\EM{o} tobu \\
%      sky-\ab{acc} fly \\
%      `To fly in the sky'
% \bg. kuni-\EM{o} saru \\
%      country-\ab{acc} leave \\
%      `To leave the (home) country'
%      \hfill{\cite[414]{yamada36}}
%
%%Whereas this use of \ci{o} sometimes sounds a little bit too formal,
%Examples like \Last can be used normally also in spoken Japanese.
%As noted earlier, however, the zero particles are predominant in everyday conversation.
%
%
%%%----------------------------------------------------
%\paragraph{Time}
%
%Time expressions, with predicates such as `spend' and `pass', are also coded by \ci{o},
%as exemplified in \Next.
%In \Next[a],
%the time expression \ci{toki}, with the predicate \ci{sugosu} `spend',
%is coded by \ci{o}.
%In \Next[b],
%the expression \ci{tosi} `year', with the predicate \ci{heru} `pass',
%is coded by \ci{o}.
%%
%\ex.
% \ag. ie-de saigo-no toki-\EM{o} sugosu tame-ni \\
%      home-\ab{loc} last-\ab{gen} time-\ab{acc} spend purpose-for \\
%      `To spend the last minue (of your life) at home'
%      \hfill{(A handout on home-visit nursing%
%       \footnote{
%       http://www.nihonkaigaku.org/library/university/i100911-t1.pdf,
%       last accessed on 03/24/2015
%       }
%      )}
% \bg. tosi-\EM{o} heru goto-ni huuai-ga masi [...] \\
%      year-\ab{acc} pass every-at texture-\ab{nom} increase \\
%      `As it passes years (as years pass by), the texture changes...'
%      \hfill{(A description of furniture%
%       \footnote{
%       http://www.ikea.com/jp/ja/catalog/categories/series/28865/,
%       last accessed on 03/24/2015
%       }
%      )}
%
%

%%----------------------------------------------------
\paragraph{Remaining issues}

Both of these non-canonical usages of \ci{o} concern the mapping
between surface forms and semantic structures,
as discussed in the paragraph on \ci{ga} and ``object'' marking.
Therefore, I consider these issues to be independent of \isi{information structure}.

As with \ci{ga},
zero particles are extensively used instead of \ci{o} in spoken Japanese.
It is therefore necessary to investigate the distribution of zero particles and \ci{o}.
I propose conditions for the use of zero particles and \ci{o} in Chapter \ref{Particles}.
I will give an overview of the literature on the zero particles in \S \ref{BackSubSubZero}.




%%----------------------------------------------------
\subsubsection{\textit{Wa}}\label{Back:GeneralChar:Wa}

%%% Hinds, John and Iwasaki, Shoichi and Maynard, Senko. Eds. 1987. Perspectives on Topicalization: The Case of Japanese Wa. John Benjamins.
%%% Iwasaki, Shoichi. 2002. Japanese. John Benjamins.
%%% Heycock, Caroline. 2008. Japanese -Wa, -Ga, and Information Structure. In Shigeru Miyagawa & Mamoru Sait (Eds.) The Oxford Handbook of Japanese Linguistics. pp.54--83.


The \isi{adverbial particle} \ci{wa} has been widely discussed in the literature
because the conditions on where it appears are very complex and subtle.

In the early literature of modern Japanese linguistics,
\ci{wa} was confused with a \isi{nominative} marker
because most of the time the particle codes so-called \isi{nominative case} in place of \ci{ga}.
According to \citeA[2]{aoki92},
who studied more than 10,000 examples of \ci{wa}
in novels and essays,
76.7\% of \ci{wa} codes \isi{nominative case}, and
84.7\% of \ci{wa}-marked nouns code \isi{nominative case}.
Moreover,
\ci{wa} appears to ``replace'' \ci{ga}.
For example,
the sentences in \Next[a] with \ci{wa} and \Next[b] with \ci{ga}
are truth-conditionally equivalent, and
replacing one particle with the other does not affect the truth value
of the sentence.
%
\ex.
 \ag. zyon-\EM{wa} gakusei-desu \\
      John-\ab{top} student-\ab{cop}.\ab{plt} \\
      `John is a student.'
 \bg. zyon-\EM{ga} gakusei-desu \\
      John-\ab{nom} student-\ab{cop}.\ab{plt} \\
      `John is a student.'
      \hfill{\cite[38]{kuno73}}

In the same way,
\Next[a] and \Next[b] are truth-conditionally equivalent.
%
\ex.
 \ag. ame-\EM{wa} hutte i-masu-ga... \\
      rain-\ab{top} fall \ab{prog}-\ab{plt}-though \\
      `It is raining, but...'
 \bg. ame-\EM{ga} hutte i-masu \\
      rain-\ab{nom} fall \ab{prog}-\ab{plt} \\
      `It is raining.'
      \hfill{(ibid.)}

Therefore, \ci{wa} was considered to code \isi{nominative case} like \ci{ga}.

\citeA[472ff.]{yamada36} pointed out that
\ci{wa} should be classified as an \isi{adverbial particle} (\ci{kakari joshi})%
 \footnote{
 Yamada distinguishes \ci{kakari joshi} from \ci{fuku joshi}.
 Although the \ili{English} term \ci{\isi{adverbial} particle} sounds closer to
 \ci{fuku joshi},
 I use the term \ci{\isi{adverbial} particle} to include both
 \ci{kakari joshi} and \ci{fuku joshi}
 because this distinction does not matter for now. 
 }
and should not be confused with case particles such as \ci{ga}.
However, since \ci{wa} codes \isi{nominative case} most of the time,
\ci{wa} has been analyzed as opposed to \ci{ga}.
Since the nature of \ci{wa} has been widely discussed,
I can only give a simplified overview of representative analyses, each of which captures a certain aspect of the particle.
\citeA{onoe77} is a useful survey of the history of studies on \ci{wa},
and \citeA{noda96} is a good summary of contemporary studies.
Here I focus on \ci{wa}-marked nouns and put aside the other uses of the particle. For other types of \ci{wa},
see, for example, \citeA[Chapter 7]{teramura91}.
%\ci{Wa} attaching nouns consist of 90.5\% in novels and essays according to \citeA{aoki92}.

The most popular analysis of \ci{wa}
is that it is a \isi{topic} marker,
which was proposed by \citeA{matsushita28}.%
 \footnote{
 According to \citeA{onoe77},
 this was first proposed in \ci{Ayuish\^{o}} by Fujitani Nariakira (1778).
 }
However, the definition \isi{topic} itself is controversial in the literature
as we have seen in \S \ref{BackSecTopic}.
So, the question of what a \isi{topic} marker is still remains.
In what follows, I outline various proposals in this regard found in the literature.

%%----------------------------------------------------
\paragraph{Givenness}

The first characterization of \ci{wa} is that it codes given information
\cite[233]{chafe70}.
\citeA{kuno73} also makes a similar claim:
\ci{wa} codes \isi{anaphoric} information,
i.e.,
information that has been ``entered into the registry of the
present \isi{discourse}'' (45).
According to \citeA{kuno73}, for example,
\Next[a] is unacceptable because
\ci{ame} `rain' has not been entered into the present registry,
whereas \Next[b] is acceptable because
\ci{wa}-coded \ci{ame} `rain' has been registered.
Note that the first-mentioned \ci{ame} was coded by \ci{ga} in \Next[b].
%
\ex.
 \ag. *ame-\EM{wa} hutte i-masu \\
       rain-\ab{top} fall \ab{prog}-\ab{plt} \\
       `Speaking of rain, it is falling.'
 \bg. asa hayaku ame-\EMi{ga} huri dasi-ta... yoru-ni natte-mo ame-\EM{wa} hutte i-ta \\
       morning early rain-\ab{nom} fall start-\ab{past} night-\ab{dat} become-also rain-\ab{top} fall \ab{prog}-\ab{past}\\
       `It started raining early in the morning...
       Speaking of the rain, it was still falling even at night.'
       \hfill{\cite[45]{kuno73}}


The analysis that \ci{wa} codes given information
explains the fact that
\ci{wa} cannot attach to nouns such as \ci{wh}-phrases like \Next[a],
quantified noun phrases like \Next[b],
and \isi{indefinite} pronouns like \Next[c].
They represent new information and have not been entered into the registry of temporary \isi{discourse}.
%
\ex.
 \ag. *dare-\EM{wa} ki-masi-ta-ka \\
       who-\ab{top} come-\ab{plt}-\ab{past}-\ab{q} \\
       `Who came?'
       \hfill{\cite[37]{kuno73}}
 \bg. *oozei-no hito-\EM{wa} paathii-ni ki-masi-ta \\
       many-\ab{gen} person-\ab{top} party-\ab{dat} come-\ab{plt}-\ab{past} \\
       `Speaking of many people, they came to the party.'
       \hfill{(op.cit.: 45)}
 \bg. *dareka-\EM{wa} byooki-desu \\
       somebody-\ab{top} sick-\ab{cop}.\ab{plt}\\
       `Speaking of somebody, he is sick.'
       \hfill{(ibid.)}


Although I believe that Kuno's observation explains
a condition of \ci{wa}-coding well,
his claim needs to be supported by more natural data
because his grammatical judgements are not unanimously shared by all native speakers of Japanese.
Moreover,
as will be discussed in Chapter \ref{Particles},
78 (41.1\%) out of 190 cases of \ci{wa} code new (non-\isi{anaphoric}) information, i.e., nouns without antecedents in the previous contexts. 
Most of them are neither generic nor contrastive and need explanation.
I will discuss the conditions of the use of \ci{wa}
in Chapter \ref{Particles}.


%%----------------------------------------------------
\paragraph{Generic \textit{wa}}

\citeA{kuroda72} and \citeA{kuno73} argue that
generic nouns can be always marked by \ci{wa}.%
 \footnote{
 \citeA{kuroda72} pays more attention to generic events rather than
 just nouns.
 }
According to \citeA{kuno72},
this is because they are ``in the permanent registry of \isi{discourse},
and do not have to be reentered into the temporary registry for each \isi{discourse}'' (p.~41).
For example, the sentences in \Next are acceptable in an out-of-the-blue context.
%
\ex.
 \ag. kuzira-\EM{wa} honyuu-doobutu-desu \\
      whale-\ab{top} mammal-animal-\ab{cop}.\ab{plt} \\
      `Speaking of whales, they are mammals. (A whale is a mammal.)'
      \hfill{\cite[44]{kuno73}}
 \bg. hito-\EM{wa} sinu (mono-desu) \\
      person-\ab{top} die (thing-\ab{cop}.\ab{plt})\\
      `Human beings die. (All humans are mortal.)'
      \hfill{(Constructed)}


In Chapter \ref{Particles}, however,
I will show that not all generic nouns can be felicitously coded by this particle in an out-of-the-blue context.
Instead, I propose that
the generic condition of \ci{wa}-coding is integrated into its
the givenness condition.


%%----------------------------------------------------
\paragraph{Contrastive \textit{wa}}

\citeA{kuno73} distinguishes between the \ci{wa} coding given information (in his sense, \isi{anaphoric} information)
and the one coding contrastive information.
He argues that contrastive \ci{wa} can code new (in his term, ``non-\isi{anaphoric}'') information
as shown in the contrast between \Next[a] and \Next[b].
According to Kuno,
\ci{oozei-no hito} `many people' in \Next[a] is new and non-con\-tras\-tive;
therefore, the sentence is not acceptable.
On the other hand, \ci{oozei-no hito} `many people' in \Next[b] is new and it contrasted with \ci{omosiroi hito} `interesting person';
in this case, the sentence is acceptable.
Contrastive \ci{wa} is typically accompanied by high \isi{pitch}.
Note that the examples and acceptability judgements are by Kuno, and that
in particular \Next[b] is not acceptable to some people (including the author).
%
\ex.
 \ag. *oozei-no hito-\EM{wa} paathii-ni ki-masi-ta \\
       many-\ab{gen} person-\ci{EM} party-\ab{dat} come-\ab{plt}-\ab{past} \\
       `Speaking of many people, they came to the party.'
       \hfill{(Non-contrastive)}
 \bg. oozei-no hito-\EM{wa} paathii-ni ki-masi-ta-ga omosiroi hito-\EM{wa} hitori-mo i-mase-n-desi-ta \\
       many-\ab{gen} person-\ab{top} party-\ab{dat} come-\ab{plt}-\ab{past}-though interesting people-\ab{top} single-also exist-\ab{plt}-\ab{neg}-\ab{plt}-\ab{past} \\
       `Many people came to the party indeed, but there was none who was interesting.'
       \hfill{(Contrastive)}
 \b.[] \hfill{\cite[47]{kuno73}}

The contrast between \Next[a] and \Next[b] is explained in the same way.
\ex.
 \ag. *ame-\EM{wa} hutte i-masu \\
       rain-\ab{top} fall \ab{prog}-\ab{plt} \\
       `Speaking of rain, it is falling.'
       \hfill{(Non-contrastive)}
 \bg. ame-\EM{wa} hutte i-masu-ga taisita koto-wa ari-mase-n \\
       rain-\ab{top} fall \ab{prog}-\ab{plt}-though serious matter-\ab{top} exist-\ab{plt}-\ab{neg} \\
       `It is raining, but it is not much.'
       \hfill{(Contrastive)}
 \b.[] \hfill{\cite[46]{kuno73}}
%
%\ex.
% \ag. watakusi-\EM{wa} tabako-\EM{wa} sui-masu-ga sake-\EM{wa} nomi-mase-n \\
%      \ab{1}\ab{sg}-\ci{wa} cigarette-\ci{wa} smoke-\ab{plt}-though alcohol-\ci{wa} drink-\ab{plt}-\ab{neg} \\
%      `Speaking of myself, I do smoke, but I don't drink.'
%  \bg. watakusi-\EM{wa} syuumatu-ni-\EM{wa} hon-\EM{wa} yomi-masu-ga benkyoo-\EM{wa} si-mase-n \\
%       \EM{1}\ab{sg}-\ci{wa} weekend-on-\ci{wa} book-\ci{wa} read-\ab{plt}-though study-\ci{wa} do-\ab{plt}-\ab{neg} \\
%       `Speaking of myself, I read books on the weekend, but I don't do any studying.'
%       \hfill{\cite[48--49]{kuno73}}

While some studies like \citeA{kuno73} assume that
contrastive non-contrastive \ci{wa} are independent and mutually exclusive,
others like \citeA{teramura91} speculate that
they are governed by the same condition(s).
\citeA{teramura91}
claims that the basic property of the particle is to indicate contrast with other elements, and non-contrastive \ci{wa} appears when the contrasted elements are not noticed.

%Note that other particles can be contrastive such as \ci{ga} and \ci{o} and the difference among the contrastive \ci{wa}, \ci{ga}, and \ci{o} is left unexplained in \citeA{kuno73}.
%%
%\ex.
% \ag. 僕がウナギで、花子がそばで、太郎が天丼
% \bg. 


\citeA{hara08}
shows that contrastive \ci{wa} always induces scalar implicatures as in \Next[a] and
proposes a formal analysis of the particle.
Furthermore,
\citeA{hara06} argues that these implicatures
are conventional rather than conversational implicatures.
%
\ex.
 \ag. nanninka-\EM{wa} ki-ta \\
      some.people-\ab{top} come-\ab{past} \\
      `Some people came.'\\
      (Implicature: it is possible that it is not the case that
      everyone came.)
  \bg. \#minna-\EM{wa} ki-ta \\
        everyone-\ab{top} come-\ab{past} \\
        `Everyone came.'\\
        (No \isi{implicature} possible.)
  \b.[] \hfill{\cite[36]{hara06}}
%\ex.
% \a. Who came to the party?
% \bg. zyon-\EM{wa} ki-ta \\
%      John-\ab{top} come-\ab{past} \\
%      `As for John, he came.'\\
%      (Implicature: it is possible that it is not the case that John and Mary came. $\thickapprox$ I don't know about others.)
% \bg. zyon-\EM{ga} ki-ta \\
%      John-\ci{ga} come-\ab{past} \\
%      `John came.'
%      (Complete answer)


The present study does not aim at investigating detailed characteristics of contrastive \ci{wa};
rather, I am more interested in capturing various aspects of \ci{wa} as a whole,
including its contrastive uses,
and in giving a unified explanation for all of them.
Therefore, issues like the syntactic position of contrastive \ci{wa},
the interaction between contrast and negation or quantifiers,
and their formal analyses
are outside of the scope of this study.
In Chapter \ref{Particles},
I will argue that contrastive and non-contrastive \ci{wa} can be explained consistently with a single principle along the lines of \citeA{teramura91}.

%%% Contrastive waをフォーカスとしている例

%%%----------------------------------------------------
%\paragraph{\ci{Kakari musubi}}
%
%Traditionally,
%the particle \ci{wa} has been considered to be a \ci{kakari joshi}
%(since \citeA{yamada36} following \ci{Kotoba-no Tama-no O} by Motoori Norinaga (1785))
%and the function of \ci{kakari} is the nature of \ci{wa}.
%Before the review of \ci{wa} as a \ci{kakari joshi},
%I outline \ci{kakari-musubi} phenomena in classic Japanese,
%from which Yamada got the idea.
%
%In classic Japanese,
%\ci{Kakari-musubi} (translated as ``hanging-tying'' in \citeA[247]{frellesvig10}) is
%``a construction in which some constituent is marked by one of the
%\ci{kakari particles} [\ci{kakari joshi}]
%(a) \ci{ka, ya, so/zo, namo/namu} or (b) \ci{koso}
%and the sentence predicate it relates to is in the
%(a) adnominal [in the present study, attributive (\ab{att})] or (b) exclamatory [concessive]%
% \footnote{
% \citeA{ono64} argues that the meaning of this predicate form (\ci{izenkei}) is
% presupposed concession.
% Therefore, I gloss this as concessive (\ab{conc}).
% }
%form, rather than in the conclusive [finite] form 
%generally used to conclude sentences'' (ibid.).
%Examples in \Next
%are from \ci{Man'y\^{o}sh\^{u}},
%which is a collection of poems around the 8th century.
%\Next[a] exemplifies \ci{zo} and the attributive form \ci{keru} (the finite form is \ci{keri}),
%and \Next[b] is an example of \ci{koso} and the concessive form \ci{kere}.
%%
%\ex.
% \ag. sak-u hana-no iro-ha kahara-zu momosiki-no oomiyahito-\EM{zo} tatikahari-\EM{ker-u} \\
%      bloom-\ab{att} flower-\ab{gen} color-\ab{top} change-\ab{neg} palace-\ab{gen} courtier-\ci{zo} change-\ab{past}-\ab{att} \\
%      `While the color of blooming flowers do not change,
%      just courtiers in the palace have changed.'
%      \hfill{(Man'y\^{o}sh\^{u} 1061)}%
%      \footnote{
%      \citeA[175]{kojimaetal95}
%      }
% \bg. yuu-sara-ba kimi-ni aha-mu-to omohe-\EM{koso} hi-no kuru-raku-mo uresikari-\EM{ker-e} \\
%      dusk-become-because \ab{2}\ab{sg}-\ab{dat} see-will-\ab{quot} think-\ci{koso} sun-\ab{gen} fall-\ab{nmlz}-also happy-\ab{past}-\ab{conc} \\
%      `Just because I will see you after dusk,
%      I was happy to see the sunset.'
%      \hfill{(Man'y\^{o}sh\^{u} 2922)}%
%      \footnote{
%      \citeA[310]{kojimaetal95_2}
%       }
%% 咲く花の色は変らずももしきの大宮人ぞ立ちかはりける (万葉1061)
%% 夕さらば君に会はむと思へこそ日の暮るらくもうれしかりけれ (万葉2922)
%
%
%In modern Japanese, however,
%clear \ci{kakari-musubi} of this kind is lost;
%instead, \citeA{yamada36} argues that
%\ci{wa} as \ci{kakari joshi} relates to \ci{chinjutsu} (``statement'') of the predicate,
%which corresponds to an illocutionary act rather than a \isi{verb} morpheme.
%\ci{Chinjutsu} is
%``an expression of an mental association between subjective and objective concepts, by stating whether these two concepts agree or not'' \cite[679]{yamada36}.
%Since the idea of \ci{chinjutsu} itself is very controversial in the literature \cite[see e.g.,][]{tokieda37a,tokieda37b},
%I simplify the issue and try to capture the essence of what Yamada means.
%Compare the following incomplete sentences \Next[a] with \ci{ga} and
%\Next[b] with \ci{wa}.
%Intuitively, \ci{tori} `bird' with \ci{ga} in \Next[a] relates to
%the predicate \ci{tobu},
%whereas \ci{tori} with \ci{wa} in \Next[b] does not.
%%
%\ex.
% \ag. tori-\EM{ga} tobu toki \\
%      bird-\ab{nom} fly when \\
%      `when a bird flies...'
% \bg. tori-\EM{wa} tobu toki \\
%      bird-\ab{top} fly when \\
%      `when a bird flies...'
%      \hfill{\cite[489]{yamada36}}
%
%\ci{Wa}-coded \ci{tori} requires a predicate of the \isi{main clause}.
%Therefore, when \ci{wa}-coded \ci{tori} is not an argument of the \isi{main clause},
%the sentence results in anomaly,
%whereas \ci{ga}-coded \ci{tori} is acceptable under the same condition.
%For example,
%if the \isi{main clause} meaning `(you should) look up!' is added to \Last,
%where the agent corresponds to (implicit) `you' and
%the patient corresponds to `above',
%\Next[a] with \ci{ga}-coding is acceptable,
%while \Next[b] with \ci{wa}-coding is not.
%%
%\ex.
% \ag. tori-\EM{ga} tobu toki ue-o mi-nasai \\
%      bird-\ab{nom} fly when above-\ab{acc} look-\ab{imp} \\
%      `When a bird flies, look up.'
% \bg. ??tori-\EM{wa} tobu toki ue-o mi-nasai \\
%      bird-\ab{top} fly when above-\ab{acc} look-\ab{imp} \\
%      `When a bird flies, look up.'
%
%
%Whereas Yamada's claim is criticized in the literature \cite[e.g.,][]{saji74},
%his idea also triggered many interesting studies.
%However,
%since the definition of \ci{chinjutsu} is not clear and
%is difficult to investigate in quantitative studies like the present work,
%I leave this issue open for further studies.

%%----------------------------------------------------
\paragraph{Characterization of \textit{wa} based on judgement types}

%Some studies investigate the association between \ci{wa} (and \ci{ga}) and judgement types.
%For example, \citeA{mio48}, inspired by \citeA{sakuma40},
%distinguishes four types of judgements
%in terms of ``field'' of speech.
%The ``field'' is ``a set of conditions which
%influence sentences in some way at a given moment'' (p.~38).
%The first type is the sentence corresponding to a field,
%which introduces a new field.
%They are separated from the previous context and 
%something different comes up in the sentence.
%For example,
%\Next[a] ``introduces new field (or scene), which branches off from the previous context'' (p.~47).
%\Next[b] is also explained in the same way.
%As can be seen from these examples,
%\ci{ga}, rather than \ci{wa}, is used in sentences of this type.
%This is equal to sentence-focus structure in terms of the present work.
%%
%\ex. \tl{Sentences corresponding to a field}
% \ag. a, ame-\EM{ga} hut-teru \\
%      oh rain-\ab{nom} fall-\ab{prog} \\
%      `Oh, it's raining!'
% \bg. mukasi mukasi aru kaigan-ni mesu-no kuzyaku-to osu-no kuzyaku-\EM{ga} sunde i-masi-ta \\
%      long.ago long.ago certain coast-\ab{dat} female-\ab{gen} peacock-and male-\ab{gen} peacock-\ab{nom} live \ab{prog}-\ab{plt}-\ab{past} \\
%      `Once upon a time, there lived male and female peacocks in the coast area.'
%      \hfill{\cite[46--47]{mio48}}
%
%The second type is the sentence which contains the field.
%\Next[b] is a sentence of this type,
%which is the answer to a question \Next[a].
%Since the answer assumes a question,
%this kind of sentence is not equal to a field;
%instead, it contains the field of ``question''.
%\Next[b] is equal to \isi{predicate-focus structure} in terms of the present study.
%%
%\ex. \tl{A sentence which contains the field}
% \a. (Continuing from \LLast[b]) What did the peacocks do?
% \bg. kuzyaku-tati-\EM{wa} sumi yoi sima-o mituke-masi-ta \\
%      peacock-\ab{pl}-\ci{wa} live good island-\ab{acc} find-\ab{plt}-\ab{past} \\
%      `The peacocks found an island suitable to live.'
%      \hfill{(op.cit.:p.~51)}
%
%The examples in \Next[b-c] are also sentences which contain the field.
%\Next[b] is equal to \isi{predicate-focus structure}, and
%\Next[c] is equal to \isi{argument-focus structure}.
%%
%\ex. \a. Which is mine?
% \bg. anata-no-\EM{wa} kore-da \\
%      \ab{2}\ab{sg}-\ab{nmlz}-\ab{top} this-\ab{cop} \\
%      `Yours is this.'
%      \hfill{(Constructed)}
% \bg. kore-\EM{ga} anata-no-da \\
%       this-\ab{nom} \ab{2}\ab{sg}-\ab{nmlz}-\ab{cop} \\
%       `This is yours.'
%       \hfill{(op.cit.:53)}
%
%The third type is the sentence which orients the field.
%This type of sentence is not enough to express the field itself;
%rather, it expresses part of the field.
%For example,
%sentences in \Next express incompletely in the sense that
%they only partially describe the field
%(the situations where it is raining in \Next[a] and
%plum flowers are blooming in \Next[b]).
%They ``orient richer fields by poorer expressions''
%\cite[55]{mio48}.
%I assume that sentences of this type are equal to sentence-focus structure.
%%
%\ex. \tl{Sentences which direct the field}
% \ag. a, ame-da\\
%      oh rain-\ab{cop} \\
%      `Oh, (it's) rain(ing).'
% \bg. a, ume-da\\
%      oh plum-\ab{cop} \\
%      `Oh, plum flowers (are blooming).'
%      \hfill{(op.cit.:55)}
%
%The fourth and last type is the sentence which complements the field.
%This is similar to the second type (the sentence which contains the field) but the question included in the second type is missing.
%\Next[b], for example,
%where the \isi{pronoun} corresponding to `they' is not expressed,
%is a sentence which complements the field,
%while the sentence with `they' is of the second type (the sentence which contains the field).
%%
%\ex. \tl{Sentences which complement the field}
% \a. (What are they?)
% \bg. ume-da \\
%      plum-\ab{cop} \\
%      `(They are) plum flowers.'
%      \hfill{(op.cit.:56)}
%
%Similarly, in \Next,
%`I' and `it' are not expressed,
%which results in a sentence which complement the field.
%%
%\exg. yomi-tai \\
%      read-want \\
%      `(I) want to read (it).'
%      \hfill{(op.cit.:57)}
%
%I assume that the fourth type is equal to \isi{predicate-focus structure}.

\citeA{kuroda72},
inspired by Branz Brentano and Anton Marty,
proposed the distinction between \ci{wa} vs.~\ci{ga}
based on categorical vs.~\isi{thetic} judgements.
According to Kuroda,
``the categorical judgement is assumed to consist of two separate acts,
one, the act of recognition of that which will be made the subject,
and the other, the act of affirming or denying what is expressed by the predicate about the subject'' (p.~154).
On the other hand,
the \isi{thetic} judgement ``represents simply the recognition or rejection of material of a judgement'' (ibid.).
Kuroda argues that
sentences with \ci{wa}, like \Next[a], correspond to the categorical judgement, and
those with \ci{ga}, like \Next[b], correspond to the \isi{thetic} judgement.
%
\ex.
 \ag. inu-\EM{wa} neko-o oikakete iru \\
      dog-\ab{top} cat-\ab{acc} chase \ab{prog} \\
      `The dog is chasing a/the cat.'
      \hfill{(Categorical judgement)}
 \bg. inu-\EM{ga} neko-o oikakete iru \\
      dog-\ab{nom} cat-\ab{acc} chase \ab{prog} \\
      `A/The dog is chasing a/the cat.'
      \hfill{(Thetic judgement)}
 \b.[] \hfill{\cite[161]{kuroda72}}

The categorical judgement roughly corresponds to the \isi{predicate-focus structure}, and
the \isi{thetic} judgement corresponds to the sentence-focus structure.

%Although the approach to \ci{wa} (and \ci{ga}) from judgement types
%potentially have more issues on \isi{information structure},
I assume that some part of judgement types can be reduced to particles.
Therefore, the theory of judgement types and particles are compatible and complement each other.
In the present study,
I only focus on the particles
and leave the rest for future studies.

%%----------------------------------------------------
\paragraph{Cohesion}

\citeA{clancydowning87},
analyzing spoken narratives, suggest that
``\ci{wa}-marking is not necessary to establish thematic status, nor does \ci{wa}-marking, when it appears, necessarily indicate that the participant in question is thematic, to the extent that
thematicity can be equated with the measures that [they] have considered,
i.e., the frequency of appearance, persistence, or ability to elicit zero switch reference'' (p.~24),
contrary to other studies such as \citeA{maynard80}.
They conclude that
``the primary function of \ci{wa} is to serve as a local cohesive device,
linking textual elements of varying degrees of contrastivity'' (p.~46)
because ``the majority of \ci{wa} uses in [their] data,
whether thematic or locally contrastive or both,
occurred on switch subjects,
i.e., references to participants who by definition had been non-subjects when last mentioned'' (ibid.).

I investigated whether this generalization applies to
my data, CSJ (\ci{the Corpus of Spontaneous Japanese}), which also includes spoken narratives as will be explained in the next chapter.
First, I extracted all \ci{wa}-coded NPs and pronouns and their \isi{antecedent} NPs and pronouns.
Then, I categorized the antecedents into
so-called subjects (\ci{ga}-coded NPs),
objects (\ci{o}-coded NPs), and datives (\ci{ni}-coded NPs) and
counted their numbers.
As a result,
it turned out that
13 subjects, 11 objects, and 10 datives were the antecedents of \ci{wa}-coded NPs or pronouns.
Although the numbers are very small and it is inappropriate to generalize based on them,
it is clear that
Clancy and Downing's claim does not hold in my data.

Moreover, \citeA{watanabe89} argues, analyzing corpora, that
\ci{wa} codes important and definite nouns, contrary to \citeA{clancydowning87}.
Therefore, it is necessary to re-examine their claim.

%%% Maynard (1980): ``-wa functions in the Japanese paragraph primarily to identify the NP which the writer has chosen as theme'' ``[in narratives,] to signal which one is the more prominent, constant figure on our thematic stage.''

%%----------------------------------------------------
\paragraph{Isolation}

%%% 松下大三郎『改撰標準日本文法』

It has been pointed out that
\ci{wa} isolates the nouns marked by it from the rest of the sentence.
\citeA{onoe77} reports that this issue was observed in the 19th century
in studies like \ci{Colloquial Japanese} by Brown (1863) and
\ci{Japansche Spraakleer} by Hoffmann (1868).
\citeA[103]{onoe81}, supporting this view,
argues that a sentence with \ci{ga} as in \Next[a] expresses a unified situation,
whereas a sentence with \ci{wa} as in \Next[b] isolates or separates
the noun from the predicate,
in this case
\ci{sora} `sky' from \ci{aoi} `blue',
and then associates these two.
%
\ex.
 \ag. sora-\EM{ga} aoi \\
      sky-\ab{nom} blue \\
      `The sky is blue.'
 \bg. sora-\EM{wa} aoi \\
      sky-\ab{top} blue \\
      `The sky is blue.'

He further argues that
\ci{wa} ``drastically confirms the \isi{thetic} judgement `the sky is blue'{''} (ibid.).

While I believe that this characterization partly captures the nature of \ci{wa},
it needs to be expressed within a theory and supported by more data.%
 \footnote{
 Onoe seems to think that the existence of the contrastive \ci{wa}
 supports the particle's ``isolation'' function.
 However, the connection between isolation and \isi{contrastiveness} is not clear to me.
 }
For example,
\ci{ga} in \Last[a] also separates \ci{sora} from \ci{aoi} because
there is a phrase boundary.
Where does the intuition of \ci{wa}'s ``isolation'' come from?
In Chapter \ref{Intonation},
I argue that there is an intonation boundary between a \isi{topic} and a focus;
therefore, topics including \ci{wa}-coded elements are intonationally separated from foci.


%%----------------------------------------------------
\paragraph{Remaining issues}

As I have mentioned above,
the aim of this study is to give a consistent explanation of \ci{wa}-coding,
rather than a detailed model of some aspect of the particle.
The characteristics summarized above reflect some of the aspects \ci{wa}.
Later on, I will propose conditions \ci{wa} as a whole.
As I also stated above,
the properties of predicates and sentence types are outside the scope of this study.
However, I believe that characterizing the particle \ci{wa} will help us to understand other unexplained features in the future.

%%----------------------------------------------------
\subsubsection{\textit{Toiuno-wa}}\label{Back:GeneralChar:Toiunowa}

In this section, I discuss the marker \ci{toiuno-wa}, which will also be investigated in the present study. The marker consists of at least four morphemes, as shown in \Next.
\exg. to iu-no-wa \\
	\ab{quot} call-one-\ci{wa} \\

The first morpheme \ci{to} is a quotation marker,
and \ci{iu} corresponds to `call' (or, more closely, `hei{\ss}en' in \ili{German}).
\Next is an example of how \ci{to} and \ci{iu}, which are realized as \ci{to ii}, are used.
%
\exg. hasi-wa tyuugoku-go-de nan-\EM{to} \EM{ii}-masu-ka \\
		chopstick-\ab{top} China-language-in what-\ab{quot} call-\ab{plt}-\ab{q} \\
		`How do you call ``chopsticks'' in \ili{Chinese}?'
		\hfill{\cite[][p.\ 81]{masuokatakubo92}}

The morpheme \ci{no} is a nominalizer which corresponds to `one' (as in \ci{this one}) in \ili{English}.
It can be used when restrictively modified nouns are repeated or are clear from the context (p.\ 160).
%
\exg. kono seetaa-wa tiisai-node ookii-\EM{no}-to kaete kudasai \\
	this sweater-\ab{top} small-because big-one-with exchange please \\
	`Since this sweater is too small, please exchange this with a bigger one.'
	\hfill{(op.\ cit.: p.\ 160)}

\citeA{masuokatakubo92} point out that
the combination of noun + \ci{to iu} + \ci{mono} (`thing') is used
when the speaker is talking about the category in general,
rather than a specific referent of the noun.
For example, \ci{kyoosi} `teacher' in \Next[a] simply refers to specific teachers,
whereas \ci{kyoosi} followed by \ci{-to iu mono} in \Next[b] refers to teachers in general.
%
\ex.
 \ag. sotugyoo-paatii-ni-wa \EM{kyoosi}-ga 20-mei seito-ga 140-mei syusseki si-ta \\
	graduation-party-\ab{dat}-\ab{top} teacher-\ab{nom} 20-\ab{cl} student-\ab{nom} 140-\ab{cl} attend do-\ab{past} \\
	`20 teachers and 140 students participated in the graduation party.'
	\hfill{(Specific teachers)}
 \bg. \EM{kyoosi-to} \EM{iu} \EM{mono}-wa tuneni aizyoo-o mot-te seeto-o mitibika-nakere-ba nara-nai \\
 teacher-\ab{quot} call thing-\ab{top} always love-\ab{acc} have-and student-\ab{acc} lead-\ab{neg}-\ab{cond} become-\ab{neg} \\
 `Teachers always must lead their students with love.'
 \hfill{(Teachers in general)}
 \begin{flushright}
 	(op.\ cit.: p.\ 34)
 \end{flushright}

%They discuss \ci{mono} instead of \ci{no}.
This also applies to \ci{no}, which also refers to some category in general rather than a specific entity.
In fact, \ci{mono} in \Last[b] can be replaced with \ci{no} without changing the meaning.
The morpheme \ci{wa} is the particle discussed in the previous section.

Unless I am discussing the compositional meanings of \ci{to iu no-wa},
I will put no spaces in \ci{toiuno}
because it is sometimes reduced to \ci{(t)teno}, \ci{t(y)uuno}, or even [\tp{tW:n@}].
I separate \ci{wa} to keep the relationships between \ci{toiuno-wa} and \ci{wa} transparent,
although \ci{wa} sometimes merges with \ci{toiuno}
and the sequence is realized as [\tp{tW:n@:}], [\tp{t:Ena:}], [\tp{tsW:na:}], etc.

While other combinations such as \ci{toiuno-ga} and \ci{toiuno-o} are possible,
I focus on \ci{toiuno-wa} because other combinations are rare in the corpus.
Since there are only a few studies on \ci{toiuno-wa} itself,
I also include studies on \ci{toiu} (without \ci{no-wa}) in the following overview.
% and more colloquial form \ci{tte},
%which is similar but not exactly the same as \ci{toiuno-wa}.

%%----------------------------------------------------
\paragraph{Basic usage}

According to \citeA{takubo89},
the combination of
\ci{toiu} and basic category nouns (such as \ci{hito} `person' and \ci{mono} `thing') is sometimes used to introduce proper names that the \isi{hearer} is assumed not to know.
%
\exg. kinoo tanaka siroo-\EM{toiu} \EM{hito}-ni ai-masi-ta \\
	yesterday Tanaka Shiro-called person-\ab{dat} meet-\ab{plt}-\ab{past} \\
	`Yesterday I met a person called Shiro Tanaka.'
	\hfill{\cite[][p.\ 218]{takubo89}}


Similarly, \citeA{kijutubumpokenkyukai09} describes \ci{toiuno-wa} as
``presenting an expression as a \isi{topic} and explaining the meaning or
attributing a noun to a specific referent'' (p.~230).
\Next[a] exemplifies the former, and
\Next[b] exemplifies the latter.
%
\ex.
 \ag. dokukinhoo-\EM{toiuno-wa} dokusen-kinsi-hoo-no ryaku-dearu \\
      \ci{dokukinhoo}-\ci{toiuno-wa} monopoly-ban-law-\ab{gen} abbreviation-\ab{cop} \\
      `The expression \ci{dokukinhoo} stands for \ci{dokusen-kinsi-hoo} (competition law).'
 \bg. satoo-san-\EM{toiuno-wa} eigyoo-bu-no satoo-san-desu-ka zinzi-bu-no satoo-san-desu-ka\\
      Sato-\ab{hon}-\ci{toiuno-wa} sales-section-\ab{gen} Sato-\ab{hon}-\ab{cop}-\ab{q} personnel-section-\ab{gen} Sato-\ab{hon}-\ab{cop}-\ab{q}\\
      `Which do you mean by ``Mr.Sato'', the person in the sales section or the person in the personnel section?'
      \hfill{\cite[230]{kijutubumpokenkyukai09}}


Sentences with \ci{toiuno-wa} also express
general properties of the topic or a judgement on what it should be.
\Next[a] is an example of the former, and
\Next[b] is an example of the latter.
%
\ex.
 \ag. suzuki-\EM{tteiuno-wa} aaiu yatu-da-yo \\
      Suzuki-\ci{toiuno-wa} that.kind guy-\ab{cop}-\ab{fp} \\
      `Suzuki is that kind of guy.'
 \bg. kagaku-\EM{toiuno-wa} honrai heewa-no tame-ni yakudateru-beki mono-da \\
      science-\ci{toiuno-wa} essentially peace-\ab{gen} sake-for use-should thing-\ab{cop} \\
      `We should use science for the sake of peace.'
      \hfill{(op.cit.: 231)}


%%----------------------------------------------------
\paragraph{Characterization of \textit{toiuno-wa} based on {predication} types}

\citeA{masuoka12}, who was inspired by \citeA{sakuma41},
analyzes the association between \isi{predication} types and
the marker \ci{toiuno-wa} and concludes that
\ci{toiuno-wa} is a \isi{topic} marker only for property \isi{predication} (or individual-level \isi{predication}),
as opposed to event \isi{predication} (or stage-level \isi{predication}).
Property \isi{predication} states the property of a referent \cite{masuoka87,masuoka08}, which is unbounded by space or time.
Masuoka states that property \isi{predication} corresponds to the
individual-level \isi{predication} proposed in \citeA{carlson77}.%
 \footnotemark
 \footnotetext
 {
 However, property \isi{predication} and individual-level \isi{predication} are
 not exactly the same because
 according to \citeA{masuoka08p},
 the following examples are classified into property \isi{predication},
 which is typically considered to be stage-level rather than individual-level \isi{predication}.
 \ex.
  \a. That person is busy.
  \b. My friend \{has been to / went to\} France many times.
  \b.[] \hfill{\cite[5--6, translated by NN]{masuoka08p}}
 
 Masuoka states that they are atypical property \isi{predication}.
 Anyway, I do not get involved in the issue of predicate types in the present study.
 }
\Next exemplifies property \isi{predication}, which is true regardless of time and space and hence also unbound by time and space.
%
\ex.
 \a. Japan is an island country.
 \b. That person is kind.
 \b.[] \hfill{\cite[4, translated by NN]{masuoka08p}}

On the other hand,
event \isi{predication} describes an event bound by time and space as in \Next.
%
\ex. A child smiled.  \hfill{(op.cit.: 5)}

This corresponds to stage-level \isi{predication} in \citeA{carlson77}.

To see that \ci{toiuno-wa} is a marker of property \isi{predication} only,
compare the following examples.
In \Next[a], which expresses event \isi{predication} bound by space and time,
\ci{toiuno-wa} cannot be used felicitously,
while in \Next[b], which expresses property \isi{predication}
unbound by space and time,
\ci{toiuno-wa} can be inserted.
%
\ex.\label{ExSatiko}
\ag. *satiko-\EM{toiuno-wa} uso-o tui-ta \\
     Sachiko-\ci{toiuno-wa} lie-\ab{acc} commit-\ab{past} \\
     `Regarding Sachiko, she lied.'
     \hfill{\cite[96]{masuoka12}}
\bg. satiko-\EM{toiuno-wa} uso-tuki-da \\
     Sachiko-\ci{toiuno-wa} lie-commiter-\ab{cop} \\
     `Regarding Sachiko, she is a liar.'
     \hfill{(Constructed)}


%\citeA{masuoka12}, inspired by \citeA{sakuma41}, points out that



%%----------------------------------------------------
\paragraph{Remaining issues}

Masuoka's characterization of \ci{toiuno-wa} well captures
some aspects of this marker.
In the present work, I will discuss \ci{toiuno-wa} from different perspectives
and will not go into detail in what respects \isi{predication} types.
I also aim at describing the relationships among other \isi{topic} markers
such as \ci{wa} and \ci{kedo}/\ci{ga},
which will be discussed below.



%%----------------------------------------------------
\subsubsection{\textit{Kedo} and \textit{ga}}\label{BackSubSubKedo}

Sometimes conjunctions can be used as \isi{topic} markers.
The present study discusses \ci{kedo} and \ci{ga} preceded by a \isi{copula},
both of which correspond to `although' or `whereas' in \ili{English}.
\ci{Kedo} and \ci{ga} are differ mainly in terms of register;
\ci{kedo} can be used in both casual and formal styles,
whereas \ci{ga} is mainly used in the formal style.
\ci{Ga} in \Next[a] and \ci{kedo} in \Next[b],
which are preceded by copulas,
function as \isi{topic} markers in the sense that
they newly introduce topics at the beginning of a \isi{discourse} or a paragraph, or are used to state different aspects of the current \isi{topic}
\cite{koide84,takahashi99}.
Intuitively,
`that issue' in \Next[a] and `Yamada' in \Next[b]
are considered to be newly introduced.
%
\ex.
 \ag. rei-no ken-desu-\EM{ga} nantoka nari-sou-desu \\
      that-\ab{gen} issue-\ab{cop}.\ab{plt}-though whatever become-will-\ab{cop}.\ab{plt} \\
      `Regarding that issue, (I) guess (I) figured the way out.'
 \bg. yamada-no koto-da-\EM{kedo} ano mama hot-toi-te ii-no-kana \\
      Yamada-\ab{gen} issue-\ab{cop}-though that way leave-let-and good-\ab{nmlz}-\ab{q} \\
      `Regarding Yamada, is it OK to just leave him?'
      \hfill{\cite[283]{niwa06}}


Note that the so-called \isi{nominative} \ci{ga} is different from
the conjunctive \ci{ga} in various ways.
For example,
conjunctive \ci{ga} does not directly follow nouns; rather, nouns must be followed by the \isi{copula} (\ci{desu}), as shown in \Next[a].
On the other hand, the so-called \isi{case marker} \ci{ga} can directly follow nouns,
as shown in \Next[b].
%
\ex.
 \ag. taroo-wa sensei-\EM{desu-ga} hanako-wa kangosi-desu \\
      Taro-\ab{top} teacher-\ab{cop}.\ab{plt}-though Hanako-\ab{top} nurse-\ab{cop} \\
      `Taro is a teacher, while Hanako is a nurse.'
      \hfill{(Conjunctive \ci{ga})}
 \bg. sensei-\EM{ga} ki-masi-ta-yo \\
      teacher-\ab{nom} come-\ab{plt}-\ab{past}-\ab{fp} \\
      `The teacher has come.'
      \hfill{(Nominative \ci{ga})}


Note also that \ci{ga} and \ci{kedo} as \isi{topic} markers are different from
conjunctive \ci{ga} and \ci{kedo}.
Conjunctive \ci{ga} and \ci{kedo} by definition follow clauses
instead of phrases;
on the other hand,
the corresponding \isi{topic} markers cannot follow clauses.
Since \ci{kedo}- or \ci{ga}-coded NPs like \ci{rei-no ken} `that issue' in \LLast[a] and \ci{yamada-no koto} `yamada's issue' in \LLast[b]
appear to be the predicates of copular sentences,
the subjects of these copular sentences should also be present. %there should be the subjects of copular sentences.
However, no subjects can be added in sentences like \LLast.

%%----------------------------------------------------
\paragraph{Remaining issues}

The characterization of \ci{kedo} and \ci{ga} as \isi{topic} markers
which introduce topics captures the distributions of these particles.
In Chapter \ref{Particles},
I aim at capturing these markers as well as other \isi{topic} particles from a unified point of view.


%%----------------------------------------------------
\subsubsection{Zero particles}\label{BackSubSubZero}

%While overt particles almost always follow nouns in written Japanese,
While nouns in written Japanese are almost always followed by overt particles,
zero particles ({\O}) are ubiquitous in spoken Japanese.
All kinds of core arguments (A, S, and P) can be basically coded by them, as exemplified in \Next.
%
\ex. \a.[] \tl{\ci{Ga} vs.~{\O}}
	\bg. {taroo-\{\EM{{\O}/ga}\}} {kaet-teru-no-\{\EM{{\O}/o}\}} sitte iru? \\
		Taro-\{{\O}/\ab{nom}\} return-\ab{prog}-\{{\O}/\ab{acc}\} know be \\
		`Do (you) know that Taro is back?' \hfill{(A \& P)}
	\b.[] \tl{\ci{O} vs.~{\O}}
	\bg. ima kono {hon-\{\EM{{\O}/o}\}} yon-deru-nen \\
		now this book-\{{\O}/\ab{acc}\} read-\ab{prog}-\ab{par} \\
		`Now (I'm) reading this book.' \hfill{(P)}
	\b.[] \tl{\ci{Wa} vs.~{\O}}
	\bg. {kimi-\{\EM{{\O}/wa}\}} dare-ga suki? \\
		\ab{2}\ab{sg}-\{{\O}/\ab{top}\} who-\ab{nom} like \\
		`Who do you like?' \hfill{(S)}
		\begin{flushright}
		{\cite[pp.\ 367-368, glosses modified]{shibatani90}}
		\end{flushright}

Although I employ the symbol {\O} and
use expressions like ``zero-coding'' and ``zero particles'',
I do not necessarily claim that zero particles exist. Rather, I see them as equivalent to ``bare NPs'' or ``NPs not followed by any particle'', and consider the difference a matter of notation.
For the sake of clarity, however,
I use the symbol {\O} and refer to bare nouns as ``zero-coding''.
Also, I do not get involved in the discussion of whether
zero particles are in fact zero or are simply omitted.
%Kuno (1972, 282)
%Ga for subject marking in the matrix sentence cannot be deleted in informal speech.
%All instances of subject with no overt particles in the matrix sentence
%are the result of Wa deletion.
I assume that each production of a \isi{zero particle} in everyday usage is governed by unique and complex conditions.
When somebody says ``the particle X can be replaced with {\O} in this context,''
I consider it to mean ``the conditions of producing X and {\O} in this context are not predictable in the current model''.

In this section, I review conditions of zero-coding that have been proposed in the literature.
Note that other parts of \S \ref{BackSubSecParticles}
focus on written Japanese,
while this part focuses on spoken Japanese.
\citeA{shimojo06} and \citeA{fry01} are useful surveys of the previous literature and
I rely on them to review the literature here.

%%----------------------------------------------------
\paragraph{Socio-linguistic factors}

\citeA{tsutsui84} points out that zero particles are acceptable
in less formal situations.
%For example,
%in \ci{the Corpus of Spontaneous Japanese} \cite[CSJ:][]{maekawa03,maekawaetal04},
%where participants did not know each other before the recording,
%speakers rarely use the zero particles
%because they use polite forms such as \ci{desu} `\ab{cop}.\ab{plt}' and \ci{masu} `\ab{plt}'.
%For example, in a segment \Next from CSJ,
%the speaker uses the overt particles \ci{ga} and \ci{o}
%instead of the zero particles.
%Note that the speaker also employs polite form \ci{desu} in line c.
%%
%\ex.
% \ag. e kono ni-hiki-\EM{ga} ookina karada-\EM{o} yusuri-nagara \\
%      \ab{fl} this 2-\ab{cl}.animal-\ci{ga} big body-\ab{acc} shake-while \\
%      `These two (dogs) shake their body,'
% \bg. kawaii kao site \\
%      cute face do \\
%      `with cute faces,'
% \bg. hurahura aruki mawat-teru-n-\EMi{desu} \\
%      zigzag walk wander-\ab{prog}-\ab{nmlz}-\ab{cop}.\ab{plt} \\
%      `walk around in a zig-zag manner'.
%      \src{S00F0031: 187.05-192.60}
%% S00F0031|00187051L|187.051137|192.598282|L|(F え)この二匹が大きな体を揺すりながらかわいい顔してふらふら歩き回ってるんです|[文末]|
%
%On the other hand,
%casual conversations like \ci{the Chiba three-party conversation corpus} \cite{Den_2007_SAC},
%speakers frequently employ the zero particles
%because the speakers had known each other well before the recording and
%talk very casually.
%As in \Next[a],
%the speaker, talking informally,
%uses the zero particles instead of overt particles.
%Note that the expressions like \ci{suggee} `madly' and \ci{mon} `\ab{nmlz}' are casual forms.
%%
%\ex.
% \ag. watasi-\EM{\O} ima-no mahuraa-\EM{\O} nakusi-tara \\
%      \ab{1}\ab{sg}-{\O} now-\ab{gen} scarf-{\O} lose-\ab{cond} \\
%      `If I lose my current scarf,'
% \bg. \EMi{suggee} kanasii-\EMi{mon} \\
%      madly sad-\ab{nmlz} \\
%      `(I) am madly sad.'
%      \src{chiba0832: 486.86-489.73}
%% 486.8587 489.7325 B: わたし今のマフラーなくしたらすっげえ悲しいもん
%
%
Also, it has been reported that
zero particles are used differently in different dialects \cite[e.g.,][]{sasaki06,nakagawa13m}.
%\citeA{fry01} investigated the dialect difference between Tokyo (Standard) and Kansai (Western) dialects and report that
%the difference was not statistically significant.
%According to \citeA{nakagawa13m}, however,
%who studied conditions of zero vs.~overt particles in these two dialects,
%the conditions that allows the zero particles are different in Tokyo and \ili{Kansai Japanese} (see also \citeA{satonakagawa12}).
%
I discuss the zero particles in casual forms spoken around Tokyo
to control for stylistic and dialectal differences.

%%----------------------------------------------------
\paragraph{Word and sentence length}

\citeA[98ff.]{tsutsui84} also proposes that
zero particles following monosyllabic nouns are less natural than
those following multisyllabic nouns.
%%
%\ex.
% \ag. zyon-wa \EM{me}-\{\EM{ga/?{\O}}\} ii-naa \\
%      John-\ab{top} eye-\{\ci{ga/{\O}}\} good-\ab{fp} \\
%      `John has good eyes.'
% \bg. zyon-wa \EM{atama}-\{\EM{ga/{\O}}\} ii-naa \\
%      John-\ab{top} head-\{\ci{ga/{\O}}\} good-\ab{fp} \\
%      `John is smart. (Lit. John has a good head.)'
%      \hfill{\cite[99]{tsutsui84}}
%
\citeA[123]{fry01} reports that
40\% of multisyllabic words are zero-coded,
while 27\% of monosyllabic words are zero-coded.%
 \footnote{
 However, his results are more complex;
 the difference between the zero-coding ratios of multisyllabic words
 and monosyllabic words are significant for As and Ss, but not for Ps.
 }
Moreover,
\citeA[44]{jorden74} has claimed that
zero-coding is frequent especially in short sentences.
\citeA[122ff.]{fry01},
compared short utterances with less than 10 words with
long utterances with equal to or more than 10 words, and
found that zero particles appear more often in short utterances.
%Fry, inspired by \citeA{alfonso66},
%suggests that ``longer sentences exhibit more syntactic complexity,
%and hence introduce more potential ambiguities'' (p.~122).
%To avoid ambiguities,
%overt particles preferred over the zero particles.
%
%Henceforth, I focus on overt vs.~zero particles following multisyllabic NPs in short sentences to control for this factor.
Henceforth, I focus on overt vs.~zero particles following multisyllabic NPs in short sentences to avoid this factor.

%%----------------------------------------------------
\paragraph{Contrast and narrow focus}

Contrasted elements are always followed by \ci{wa} \cite[53ff.]{tsutsui84}.
In \Next[a], for example,
\ci{boku} `I' and \ci{biru} `Bill' are contrasted
and cannot be followed by zero particles.
%
\ex.
 \ag. boku-\{\EM{wa/*{\O}}\} oyoi-da-kedo biru-\{\EM{wa/*{\O}}\} oyoga-nakat-ta-yo \\
      \ab{1}\ab{sg}-\{\ab{top}/{\O}\} swim-\ab{past}-though Bill-\{\ci{{wa}/{\O}}\} swim-\ab{neg}-\ab{past}-\ab{fp} \\
      `I swam, but Bill didn't swim.'
 \bg. boku-\{\EM{wa/{\O}}\} biiru-\{\EM{wa/*{\O}}\} nomu-kedo sake-\{\ci{wa/*{\O}}\} noma-nai \\
      \ab{1}\ab{sg}-\{\ci{wa/{\O}}\} beer-\{\ci{wa/{\O}}\} drink-though sake-\{\ci{wa/{\O}}\} drink-\ab{neg} \\
      `I drink beer but not sake.'
      \hfill{\cite[Modified from][]{tsutsui84}}%
      \footnote{
      Many of Tsutsui's examples employ formal and polite forms
      rather than casual forms.
      Therefore, I modified all of his examples cited
      in the present study into casual forms
      to exclude the effect of formality.
      }

%%% 田中くん来たけど、山田くんまだ来ないね <- 言える

As \citeA[93ff.]{tsutsui84} also pointed out,
zero particles cannot be felicitously used
in narrow-focus contexts
(the \isi{argument focus} structure or ``exclusivity'' in Tsutsui's term).
In these contexts, overt particles are obligatory
\cite[see also][]{fujiiono00}.
As shown in \Next[B], where \ci{suteeki} `steak' is focused, for example,
the overt particle \ci{o} is natural,
while the \isi{zero particle} {\O} is not.
%
\ex.
 \a.[A:] Did you eat spaghetti in the restaurant?
 \bg.[B:] boku-wa suteeki-\{\EM{o/*{\O}}\} tabe-ta-n-da-yo \\
          \ab{1}\ab{sg}-\ab{top} steak-\{\ci{o/{\O}}\} eat-\ab{past}-\ab{nmlz}-\ab{fp} \\
          `I ate steak (not spaghetti).'
          \hfill{\cite[93, context added]{tsutsui84}}

In a similar manner,
\ci{hon} `book' in \Next[B] can be naturally followed by \ci{ga},
but not by {\O},
because \ci{hon} is narrow-focused.
%
\ex.
 \a.[A:] Which book is interesting?
 \bg.[B:] kono hon-\{\EM{ga/*{\O}}\} omosiroi-yo \\
          this book-\{\ci{ga/*{\O}}\} interesting-\ab{fp} \\
          `This book is interesting.'
          \hfill{(op.cit.: 94, context added)}

Based on these facts,
\citeA{shimojo06}, following \citeA{leed02},
proposes that the function of zero particles is to
``withhold[...] reference to other referents 
which are potentially related to the proposition denoted by the sentence'' (p.~131).


On the other hand,
\citeA{matsuda96} and \citeA{fry01} report that
\ci{wh}-word Ps (such as \ci{nani} `what' and \ci{dare} `who') are
more likely to be zero-coded than non-\ci{wh}-word Ps.
Fry found that 71\% of \ci{wh}-Ps are zero-coded,
whereas 51\% of non-\ci{wh}-Ps are zero-coded.
As exemplified in \Next,
zero-coded \ci{wh}-Ps are not rare.%
 \footnote{
 However, I did not find any examples of \ci{dare} as P in
 \ci{the Chiba three-party conversation corpus}.
 }
%
\ex.
 \ag. de satosi ima \EM{nani-{\O}} si-ten-no \\
      then Satoshi now what-{\O} do-\ab{prog}-\ab{q} \\
      `So, what are you doing now, Satoshi?'
      \src{chiba1232: 349.08-349.98}
 \bg. \EM{nani-{\O}} turu-no \\
      what-{\O} fish-\ab{q} \\
      `What do you fish?'
      \src{chiba0732: 491.59-492.07}
% 349.0820 349.9822 A: で(R_聡史)今何してんの
% chiba0732.txt:491.5876 492.0735 B: 何釣るの

The fact that \ci{wh}-words are more likely to be zero-coded than non-\ci{wh}-words contradicts Tsutsui's observation because,
in general, \ci{wh}-questions are considered to be in \isi{narrow focus}.
Similarly, \citeA[Chapter 10]{niwa06} reports that
objects corresponding to the answer to a \ci{wh}-question are acceptable,
which are also considered to be in \isi{narrow focus} and are therefore another counter-example to Tsutsui's claim.
As shown \Next[A],
the object \ci{kootya} `tea', which is the answer to a \ci{wh}-question, can be coded by either \ci{o} or {\O}.
%
\ex.
 \a.[Q:] Which do you wanna drink, coffee or tea?
 \bg.[A:] zyaa \EM{kootya-\EM{\{o/{\O}\}}} nomu-wa \\
          then tea-\ci{o/{\O}} drink-\ab{fp} \\
          `Then, (I) drink tea.'
          \hfill{\cite[291]{niwa06}}

To complicate matters,
\ci{wh}-subjects can be zero-coded,
but subjects corresponding to the answer to a \ci{wh}-question cannot \cite{niwa06}.
As exemplified in \Next,
the \ci{wh}-subject \ci{dare} `who' can be either zero-coded or \ci{ga}-coded,
but the subject corresponding to the answer cannot be felicitously zero-coded.
%
\ex.
 \ag. ima \EM{dare-\EM{\{ga/{\O}\}}} ki-teta-no? \\
       now who-\ci{ga/{\O}} come-\ab{pfv}-\ab{q} \\
       `Who came a moment ago?'
 \bg. \EM{taroo-\{ga/?{\O}\}} ki-teta-n-da \\
       Taro-\ci{\{ga/{\O}\}} come-\ab{pfv}-\ab{nmlz}-\ab{cop} \\
       `Taro came.'
          \hfill{\cite[291]{niwa06}}

\citeA{fry03} reports that
the ratio of zero particles coding \ci{wh}-words for As and Ss (25\%) is lower than the ratio of zero-coding for non-\ci{wh}-As and Ss (32\%),
although the difference is not significant in a $\chi^{2}$-test.


%%%(13) コーヒー?紅茶?どっち飲む? -- じゃあ、紅茶{φ/を}飲むわ
%%%(14) (眼の前に本を差し出して)どっち読む? -- こっち{φ/を}読むね
%%%(15) 今、誰{φ/が}来てたの? -- 太郎{?φ/が}来てたんだ
%%%(16) 誰{?φ/が}犯人? -- 太郎{?φ/が}犯人

%%----------------------------------------------------
\paragraph{Word order}

\citeA[108ff.]{tsutsui84} argues that
zero particles can be used naturally
``if the NP [...] is preceded by the subject of the sentence and immediately followed by the predicate'' (p.~108).
As instantiated in \Next,
Tsutsui claims that the zero-coded NP \ci{eigo} `\ili{English}' in \Next[a] is natural
because it is preceded by the subject \ci{boku} `I' and immediately followed by the predicate \ci{umai} `good',
while the zero-coding in \Next[b] is unnatural because
it is not immediately followed by the predicate.
%
\ex.
 \ag. \EMi{boku-\{{wa/{\O}}\}} hanako-yori \EM{eigo-\{{ga/{\O}}\}} umai-yo \\
      \ab{1}\ab{sg}-\{\ci{wa/{\O}}\} Hanako-than \ili{English}-\{\ci{ga/{\O}}\} good-\ab{fp} \\
      `I'm better at \ili{English} than Hanako.'
 \bg. \EMi{boku-\{{wa/{\O}}\}} \EM{eigo-\{{ga/??{\O}}\}} hanako-yori umai-yo \\
      \ab{1}\ab{sg}-\{\ci{wa/{\O}}\} \ili{English}-\{\ci{ga/{\O}}\} Hanako-than good-\ab{fp} \\
      `I'm better at \ili{English} than Hanako.'
      \hfill{\cite[110]{tsutsui84}}

This is supported by \citeA{matsuda96} and \citeA{fry01}.
\citeA[124]{fry01}, for example, found that
58\% of verb-adjacent Ps are zero-coded,
whereas 41\% of non-verb-adjacent Ps are zero-coded.

\citeA[291ff.]{niwa06} points out that verb-adjacent NPs
can be zero-coded more naturally when the NPs are non-\isi{topic}s (foci).%
 \footnote{
 There may be elements in a sentence that are
 neither \isi{topic}s nor foci.
 The present study, however, assumes that all core arguments are
 either \isi{topic}s or foci;
 therefore, if an element is not a \isi{topic},
 it is assumed that it is a focus.
 }
On the other hand,
Niwa also found that clause-initial NPs can be naturally zero-coded
when the NPs are topics.
Compare \Next and \NNext.
\ci{Sugoi kawaii ko} `very cute girl' in \Next is in focus
because the NP is \isi{indefinite} and is treated as news.
In this case,
the verb-adjacent NP can be felicitously zero-coded as in \Next[a],
whereas the non-verb-adjacent NP cannot naturally be zero-coded \Next[b].
%
\ex.
 \ag. oi keiri-ka-ni \EM{sugoi} \EM{kawaii} \EM{ko}-\{\EM{ga/{\O}}\} hait-ta-zo \\
      hey accounting-section-\ab{dat} very cute girl-\{\ci{ga/{\O}}\} enter-\ab{past}-\ab{fp}\\
      `Hey, a very cute girl joined the accounting section.'
 \bg. oi \EM{sugoi} \EM{kawaii} \EM{ko}-\{\EM{ga/?{\O}}\} keiri-ka-ni hait-ta-zo \\
      hey very cute girl-\{\ci{ga/{\O}}\} accounting-section-\ab{dat} enter-\ab{past}-\ab{fp}\\
      `Hey, a very cute girl joined the accounting section.'
      \hfill{\cite[293]{niwa06}}

On the contrary,
\ci{ano ko} `that girl' in \Next is topical
because the NP is definite and the participants have previously discussed her.
In this case,
both the verb-adjacent and the non-verb-adjacent NPs can felicitously be zero-coded.
%
\ex. (People have discussed a female newcomer \ci{ano ko} `that girl'.)
 \ag. oi keiri-ka-ni \EM{ano} \EM{ko}-\{\EM{ga/{\O}}\} hait-ta-zo \\
      hey accounting-section-\ab{dat} that girl-\{\ci{ga/{\O}}\} enter-\ab{past}-\ab{fp}\\
      `Hey, that girl joined the accounting section.'
 \bg. oi \EM{ano} \EM{ko}-\{\EM{ga/{\O}}\} keiri-ka-ni hait-ta-zo \\
      hey that girl-\{\ci{ga/{\O}}\} accounting-section-\ab{dat} enter-\ab{past}-\ab{fp}\\
      `Hey, that girl joined the accounting section.'
      \hfill{(ibid.)}
%(40) a. おい、経理課にすごいかわいい子{φ/が}入ったぞ
%     b. おい、すごいかわいい子{?φ/が}経理課に入ったぞ
%(42) (新人の「あの子」のことが前から話題になっている)
%     a. おい、経理課にあの子{φ/が}入ったぞ
%     b. おい、あの子{φ/が}経理課に入ったぞ

%Likewise,
%\citeA{tsutsui84} speculates that zero-coding in \Next[a] is natural
%because it is preceded by the subject `Mary',
%while that in \Next[b] is unnatural
%because it is not preceded by the subject.
%%
%\ex.
% \ag. mearii-\EMi{ga} asi-\{\EM{ga/{\O}}\} nagai-no sitte-ta \\
%      Mary-\ci{ga} leg-\{\ci{ga/{\O}}\} long-\ab{nmlz} know-\ab{past} \\
%      `Did you know that Mary has long legs? (Lit. Did you know that Mary, (her) legs are long?)'
%  \bg. mearii-\EMi{no} asi-\{\EM{ga/??{\O}}\} nagai-no sitte-ta \\
%      Mary-\ab{gen} leg-\{\ci{ga/{\O}}\} long-\ab{nmlz} know-\ab{past} \\
%      `Did you know that Mary's legs are long?'
%      \hfill{(op.cit.: p.~113)}
%\ex.
% \ag. atasi-\{{wa/{\O}}\} sinu-hodo onaka-\{\EM{ga/{\O}}\} sui-teru-no \\
%      \ab{1}\ab{sg}-\{\ci{wa/{\O}}\} die-like stomach-\{\ci{ga/{\O}}\} get.empty-\ab{prog}-\ab{fp} \\
%      `I'm starving to death. (Lit. My stomach has gotten empty like I would die.)'
% \bg. atasi-\{{wa/{\O}}\} onaka-\{\EM{ga/??{\O}}\} sinu-hodo sui-teru-no \\
%      \ab{1}\ab{sg}-\{\ci{wa/{\O}}\} stomach-\{\ci{ga/{\O}}\} die-like get.empty-\ab{prog}-\ab{fp} \\
%      `I'm starving to death. (Lit. My stomach has gotten emply like I would die.)'
%      \hfill{(op.cit.: pp.~110-111)}

%However, this is not supported by \citeA{fry01}.
%According to Fry's result,
%31\% of the verb-adjacent subjects are zero-coded,
%whereas 33\% of the non-verb-adjacent subjects are zero coded.
%This difference is not significant.
%In fact, Tsutsui's grammatical judgement in \Last is
%intuitively not agreeable to me.



%%----------------------------------------------------
\paragraph{Types of predicates}

\citeA{tateishi89} argues that zero particles are natural only inside V$^{\prime}$.
The subjects of a stage-level predicate or of an \isi{unaccusative} predicate can be naturally zero-coded
because they are realized inside V$^{\prime}$.
On the other hand, the subjects of an individual-level predicate or an \isi{unergative} predicate
are realized outside V$^{\prime}$ \cite[see also][56--57]{kageyama93}.
As shown by the contrast between \Next and \NNext,
the subjects of \isi{unaccusative} predicates \Next can naturally be either zero- or \ci{ga}-coded,
while those of \isi{unergative} predicates \NNext can only be coded by \ci{ga};
zero-coding results in anomaly.
%
\ex. Unaccusative predicate
 \ag. tanaka-san-\{\EM{ga/{\O}}\} \EMi{nakunat}-ta-no sira-nakat-ta \\
      Tanaka-\ab{hon}-\{\ci{ga/{\O}}\} pass.away-\ab{past}-\ab{nmlz} know-\ab{neg}-\ab{past} \\
      `(I) didn't know that Mr.~Tanaka passed away.'
 \bg. terebi-no nyuusu-de tankaa-\{\EM{ga/{\O}}\} \EMi{tinbotu} suru tokoro mi-ta-yo \\
      TV-\ab{gen} news-at tanker-\{\ci{ga/{\O}}\} sink do place see-\ab{past}-\ab{fp} \\
      `(I) saw a tanker sinking in the TV news.'
      \hfill{\cite[56]{kageyama93}}

\ex. Unergative predicate
 \ag. kodomo-tati-\{\EM{ga/?*{\O}}\} \EMi{sawagu}-no mi-ta koto nai \\
      child-\ab{pl}-\{\ci{ga/{\O}}\} mess.around-\ab{nmlz} see-\ab{past} thing not.exist \\
      `(I've) never seen the children messing around.'
 \bg. kanzya-\{\EM{ga/?*{\O}}\} \EMi{abare}-ta-no sit-te-masu-ka \\
      patient-\{\ci{ga/{\O}}\} go.violent-\ab{past}-\ab{nmlz} know-\ab{prog}-\ab{plt}-\ab{q} \\
      `Did (you) know that the patient went violent?'
      \hfill{(ibid.)}


\citeA{yatabe99} points out that there are counter-examples to Tateishi's generalization,
citing an example from \citeA{niwa89}.
The predicate \ci{happyoo suru} `give a presentation' is an ergative predicate and it is possible to zero-code the agent of this action,
as shown in \Next.
%
\exg. kondo gengo-gakkai-de yamada-san-\{\EM{ga/{\O}}\} happyoo suru-n-da-tte \\
      next.time linguistic-conference-\ab{loc} Yamada-\ab{hon}-\{\ci{ga/{\O}}\} presentation do-\ab{nmlz}-\ab{cop}-\ab{quot} \\
      `I heard that Mr.~Yamada is going to give a presentation at the next linguistic conference.'
      \hfill{\cite[49]{niwa89}}

Note, however, that this example is topical zero-coding,
rather than focal zero-coding, and
these two might be different from each other.

Yatabe also argues against Tateishi's claim that
zero particles cannot naturally follow
the subject of an individual-level predicate.
Although I do not get involved in this discussion
because it is outside the scope of the present study,
I suggest that this is also attributable to
the distinction between \isi{topic} vs.~focus zero particles.
%
%\ex.
% \ag. komaru-yo-naa mado-\{\EM{ga/{\O}}\} chiisakat-tara \\
%      troublesome-\ab{fp}-\ab{fp} window-\{\ci{ga/{\O}}\} small-\ab{cond} \\
%      `We'll have a problem, won't we -- if the wondow is small.'
% \bg. komaru-yo-naa sono gakusei-\{\EM{ga/{\O}}\} nihon-jin-dat-tara \\
%      troublesome-\ab{fp}-\ab{fp} that student-\{\ci{ga/{\O}}\} Japan-person-\ab{cop}-\ab{cond} \\
%      `We'll have a problem, won't we -- if that student is Japanese.'
%      \hfill{\cite[87]{yatabe99}}


%%----------------------------------------------------
\paragraph{Types of nouns}

The hierarchy of features proposed in \citeA{silverstein76,silverstein81}
also plays a crucial role in zero-coding in spoken Japanese.
\citeA{minashima01} reports that
\isi{indefinite} or \isi{inanimate} objects are more likely to be zero-coded
than definite or \isi{animate} objects.
The results in \citeA[128ff.]{fry01} support Minashima's generalization.%
 \footnote{
 In Fry's data, zero-codings of \isi{animate} and \isi{inanimate} objects are not
 significantly different.
 He speculates that this might be because of the small number of
 \isi{animate} objects in his corpus.
 }
\citeA{kurumadajaeger13,kurumadajaeger15}, by conducting experiments on speaker's choice between overt vs.~zero particles,
also report that speakers are more likely to attach the overt particle (\ci{o}) to \isi{animate} objects.
On the other hand,
\citeA[128ff.]{fry01} reports that
``strongly definite'' subjects (proper nouns and personal pronouns)
are more likely to be zero-coded than other kinds of subjects.
Also, \isi{animate} subjects are more likely to be zero-coded than
\isi{inanimate} subjects.
Fry points out that this tendency follows the typological generalization
proposed in \citeA{comrie79,comrie83}.

\citeA{niwa06} suggests that
the predictability of nouns influences the coding of particles.
Compare \Next[a] and \Next[b], for example.
The only difference between these two examples is what
might fall from the sky;
in \Next[a], rain might fall,
while, in \Next[b], hail might fall,
which is more surprising.
In \Next[a], both the overt particle \ci{ga} and the \isi{zero particle}
are acceptable.
By contrast, in \Next[b] only the overt particle is acceptable.
%
\ex. (The sky looks threatening.)
 \ag. \EM{ame}-\{\EM{ga/{\O}}\} huru-kamosirenai-n-da-tte \\
      rain-\{\ci{ga/{\O}}\} fall-\ab{pot}-\ab{nmlz}-\ab{cop}-\ab{quot} \\
      `I heard that it might rain.'
 \bg. \EM{hyoo}-\{\EM{ga/?{\O}}\} huru-kamosirenai-n-da-tte \\
      hail-\{\ci{ga/{\O}}\} fall-\ab{pot}-\ab{nmlz}-\ab{cop}-\ab{quot} \\
      `I heard that it might hail.'
      \hfill{\cite[290]{niwa06}}

\citeA{kurumadajaeger13}\footnote{See also \cite[p.~863][]{kurumadajaeger15}.} argue that
\begin{modquote}
Japanese speakers prefer to produce an object NP without case marking
when the grammatical function of a noun is made more predictable
given the semantics of the noun (e.g., \isi{animacy}) and
the other linguistic elements in the sentence
(e.g., plausibility of [grammatical-function]-assignment given the subject, object, and \isi{verb})
\end{modquote}
For example,
doctors are more likely to do something to patients,
rather than vice versa.
Therefore, case in \Next[a] is more predictable than in \Next[b],
meaning that \ci{isya} in \Next[b] is more likely to be overtly coded than
\ci{kanzya} in \Next[a].
%
\ex.
 \ag. \EMi{isya}-ga \EM{kanzya}-\{\EM{o/{\O}}\} byoositu-de teate si-ta \\
      doctor-\ab{nom} patient-\{\ci{o/{\O}}\} hospital.room-in treat do-\ab{past} \\
      `The/a doctor treated the/a patient in a hospital room.'
 \bg. \EMi{kanzya}-ga \EM{isya}-\{\EM{o/{\O}}\} byoositu-de mat-ta \\
      patient-\ab{nom} doctor-\{\ci{o/{\O}}\} hospital.room-in wait-\ab{past} \\
      `The/a patient waited for the/a doctor in a hospital room.'
  \b.[]    \hfill{(Translated from \citeA[860]{kurumadajaeger13})}

They argue that their study
``constitutes strong support for the view that
language production is optimized to maximize the efficiency of information transmission'',
referring to \citeA{levyjaeger07} and \citeA{jaeger10}.

%``Shared information'' condition in \citeA[121ff.]{tsutsui84}.

%\ci{Ga} cannot be dropped if it marks the subject of an individual predicate.

% 意外性の関与 (車田さんの研究も引用)
%(10) a. 雲行きが怪しいな -- 雨{φ/が}降るかもしれないんだって
%     b. 雲行きが怪しいな -- 雹{?φ/が}降るかもしれないんだって



%%----------------------------------------------------
\paragraph{Other pragmatic factors}
\hspace*{-3mm}
\citeA{makinotsutsui86}
and
\citeA{backhouse93}
point out that
NPs in interrogatives tend to be zero-coded.
This is supported by \citeA{fry01},
who studied a large corpus.
For example,
in \Next from the corpus of \citeA{fry01},
\ci{pen}, whose existence is in question, is zero-coded.
%
\exg. nanka kami-to pen-\EM{\O} aru? \\
      um paper-and pen-\ci{\O} exist \\
      `Um, do you have pen and paper?'
      \hfill{\cite[120]{fry01}}

Sentences of this type have attracted particular attention
because the \isi{zero particle} in this sentence is not optional;
\ci{wa} and \ci{ga} (and, of course, \ci{o}) cannot be used in this context.
According to \citeA{onoe87},
these obligatory zero particles typically appear in sentences like
the following:
%
\ex.
 \ag. kore-\EM{\O} oisii-yo \\
      this-\ci{\O} good-\ab{fp} \\
      `This is delicious.'
 \bg. huzi-san-\EM{\O} mi-eru? \\
      Fuji-mountain--\ci{\O} see-\ab{cap} \\
      `Can you see Mt.~Fuji? (Is Mt.~Fuji visible to you?)'
 \bg. rosia-go-\EM{\O} yom-eru? \\
      Russia-language-\ci{\O} read-\ab{cap} \\
      `Can you read \ili{Russian}? (Is \ili{Russian} readable to you?)'
      \hfill{\cite[48]{onoe87}}


Also, \citeA[118ff.]{tsutsui84} observes that
zero particles code information the \isi{hearer} expects to hear.
As shown in the contrast between \Next and \NNext,
the \isi{zero particle} (as well as \ci{ga} in this case) can naturally code \ci{basu} `bus' in \Next
if the speaker and the \isi{hearer} are waiting for a bus and hence
the \isi{hearer} expects to hear the word \ci{basu} `bus';
on the other hand,
zero-coded \ci{basu} in \NNext is unnatural
because the \isi{hearer} does not expect to hear \ci{basu}.
%
\ex.
 \a.[] Situation: the speaker and the \isi{hearer} are waiting for a bus,
       and the speaker sees the bus coming.
 \bg.[] a basu-\{\EM{ga/{\O}}\} ki-ta \\
      oh bus-\{\ci{ga/{\O}}\} come-\ab{past} \\
      `Oh here comes a bus.'
      \hfill{\cite[120]{tsutsui84}}

\ex.
 \a.[] Situation: the speaker sees a bus coming at a place
       where there is no bus service.
 \bg.[] a basu-\{\EM{ga/*{\O}}\} ki-ta \\
      oh bus-\{\ci{ga/{\O}}\} come-\ab{past} \\
      `Oh here comes a bus.'
      \hfill{(ibid.)}


Some researchers argue that
\isi{discourse} structure affects the selection of \ci{wa} vs.~{\O}.
Analyzing casual interviews, \citeA{suzuki95}
claims that
``relatively speaking, zero-marked phrases tend to represent
minor [\isi{discourse}] boundaries in contrast to major boundaries represented
by \ci{wa}-phrases'' (p.~615).
On the other hand,
\citeA{kurosaki03},
investigating scenarios of TV dramas,
argues that zero particles are employed to introduce new topics
\cite[see also][]{niwa06},
which implies that they appear at major \isi{discourse} boundaries.
For now,
I suppose that it is extremely difficult to identify \isi{discourse} boundaries in a reliable way,
let alone the difference between major and minor boundaries.
Therefore, we need to wait for breakthroughs in this area.

%%----------------------------------------------------
\paragraph{Remaining issues}

As we can see from the outline of studies on zero particles,
factors that affect zero- vs.~overt-codings are complex,
and some results are contradictory.
A theory that explains zero-coding is necessary.
I propose a unified theory that predicts zero-coding in terms of
\isi{information structure}
based on \citeA{nakagawa13m}.
Along the lines of \citeA{comrie79,comrie83},
I propose a frequency account of zero vs.~overt coding of particles.
I believe that this account is congruent with
the theory proposed in \citeA{levyjaeger07,kurumadajaeger13} and \citeA{kurumadajaeger15}.



%%%----------------------------------------------------
%\subsubsection{Other particles}\label{BackSubSubOthers}
%
%There are other particles used in spoken Japanese,
%which are not discussed in the present study.
%In the following,
%I present a brief review of some important particles.
%
%%%----------------------------------------------------
%\paragraph{Case particles}
%
%Of course, Japanese has many other particles which express grammatical functions.
%All of the particles are post-nominal.
%Since I included \ci{ni} in the corpus investigation to compare \ci{ni} `\ab{dat}' with \ci{ga} and \ci{o},
%I briefly discuss \ci{ni}.
%Even though I investigate \ci{ni} for comparison,
%I will not discuss \ci{ni}
%because, as will be outlined below, \ci{ni} expresses heterogeneous meanings and it is difficult to identify each of the meanings of \ci{ni} in the corpus.
%The meaning of \ci{ni} is mainly locative,
%but it also covers so-called dative.
%I decided to gloss \ci{ni} as dative (\ab{dat})
%since there is another locative marker \ci{de}.
%\Next shows examples of locative-like \ci{ni},
%which attaches the location of a thing in \Next[a]
%or to the goal of an action in \Next[b].
%%
%\ex.
% \ag. bokuzyoo-\EM{ni} usi-ga iru \\
%      pasture-\ab{dat} cow-\ab{nom} exist \\
%      `There is a cow in a pasture.'
%% \bg. tuki-ga tyuu-kuu-\EM{ni} kakaru \\
%%      moon-\ab{nom} middle-sky-\ci{ni} hang \\
%%      `The moon hangs in the middle of the sky.'
% \bg. kisya-\EM{ni} noru \\
%      train-\ab{dat} ride \\
%      `Get on a train'
%      \hfill{\cite[369]{tanaka77}}
%
%\Next shows examples of dative-like \ci{ni},
%which attaches to the causee (`younger sister') in causative construction \Next[a]
%or to the agent (`somebody') in passive construction \Next[b].
%%
%\ex.
% \ag. imooto-\EM{ni} kanbyoo-o sa-seru \\
%      younger.sister-\ab{dat} nursing-\ab{acc} do-\ab{caus} \\
%      `make (one's) younger sister take care of (oneself)'
% \bg. hito-\EM{ni} damasa-reru \\
%      person-\ab{dat} trap-\ab{pass} \\
%      `be trapped by somebody'
%      \hfill{(ibid.)}
%
%
%I assume that verb-specific semantic roles such as `giver' and `runner'
%are mapped onto thematic relations such as `agent',
%which are further mapped onto grammatical functions such as `subject'
%\cite{vanvalin01}.
%The mappings are language-specific.
%I leave open the issue of what roles non-core arguments play in \isi{information structure}.
%
%%%----------------------------------------------------
%\paragraph{Topic particles}
%
%\citeA{morishige65,fujitayamazaki06} and \citeA{kijutubumpokenkyukai09}
%are good collections of other \isi{topic} particles
%including those which are not discussed in the present study.
%For example, the conditional marker \ci{nara} can be used as \isi{topic} marker.
%%
%\ex.
% \a. I want to go to the zoo today.
% \bg. doobutu-en-\EM{nara} kyoo-wa yasumi-da-yo \\
%      animal-garden-\ab{cond} today-\ci{wa} closed-\ab{cop}-\ab{fp} \\
%      `(If you are talking about) the zoo, (it) is closed today.'
%      \hfill{\cite[244]{kijutubumpokenkyukai09}}
%
%Another conditional marker, \ci{ba},
%combined with \ci{to} `\ab{quot}' and \ci{iu} `call',
%can also function like a \isi{topic} marker.
%%
%\ex.
% \a. Travel to Okinawa is now becoming popular.
% \bg. Okinawa-\EM{to-ie-ba} aoi umi-ya mabusii yookoo-ga omoi ukabu \\
%      Okinawa-\ab{quot}-{call}-\ab{cond} blue ocean-and bright sunlight-\ab{nom} think float \\
%      `Speaking of Okinawa, it reminds (me) of blue ocean and bright sunlight.'
%      \hfill{(op.cit.: 247)}
%
%Another conditional marker, \ci{tara},
%combined with \ci{to} `\ab{quot}' and the \isi{verb} \ci{kuru} `come',
%also function as a \isi{topic} marker.
%%
%\ex.
% \ag. tikagoro-no wakamono-\EM{to}-\EM{ki}-\EM{tara} hubensa-ni taeru-to iu koto-o sira-nai \\
%      recent-\ab{gen} kid-\ab{quot}-come-\ab{cond} inconvenience-\ab{dat} endure-\ab{quot} call thing-\ab{acc} know-\ab{neg} \\
%      `When it comes to kids these days, they don't know how to stand up inconvenience.'
%  \bg. ano otoko-no iikagensa-\EM{to}-\EM{ki}-\EM{tara} mattaku haradatasii \\
%       that man-\ab{gen} looseness-\ab{quot}-come-\ab{cond} indeed annoying \\
%       `When it comes to that guy's looseness, it is indeed annoying.'
%       \hfill{(op.cit.: 242)}
%
%See also \citeA{masuoka12} for discussion on \ci{to-ki-tara}.
%
%There are many more \isi{topic} markers and their variations.
%It is intriguing to investigate the distribution of all these markers.
%Also, studying interactions between topics and conditionals will provide
%fruitful insights both topics and conditionals.
%However, since there are few examples of these particles in my corpus,
%it is difficult to explore them in this study.
%
%%%----------------------------------------------------
%\paragraph{Focus particles}
%
%Focus particles such as \ci{dake} `only', \ci{made} `as far as', \ci{sae} `even', and \ci{sika} `anything but' are discussed as \ci{toritate} particles in Japanese linguistics \cite[e.g.,][]{numata86,teramura91} and as focus particles in formal linguistics \cite[e.g.,][]{aoyagi06}.
%The present work cannot discuss these focus particles because
%they are rare in my corpus (and spoken data in general).
%It is difficult to investigate rare items in corpora;
%As far as I notice,
%these focus particles are studied based on constructed examples and
%acceptability or grammaticality judgements of authors.
%Production or sentence-judging experiments are necessary for
%empirical investigation.
%

%%----------------------------------------------------
\subsection{Word order}\label{BackSubSecWO}
\largerpage[2]
While the basic \isi{word order} in Japanese is APV (or SOV in more popular terminology),
other variants are also possible.
Example \Next[a] shows the basic \isi{word order}, and
examples \Next[b--f] show other possibilities.
According to \citeA[260]{shibatani90},
not all possibilities are equally natural in out-of-the-blue contexts,
as shown by `?' before the sentence.
%
\ex.
 \ag. taroo-ga hanako-ni sono hon-o yat-ta \\
      Taro-\ab{nom} Hanako-\ab{dat} that book-\ab{acc} give-\ab{past} \\
      `Taro gave a book to Hanako.' \hfill{(A + DAT + P + V)}
 \bg. hanako-ni taroo-ga sono hon-o yat-ta \\
      Hanako-\ab{dat} Taro-\ab{nom} that book-\ab{acc} give-\ab{past} \\
      \hfill{(DAT + A + P + V)}
 \bg. sono hon-o taroo-ga hanako-ni yat-ta \\
      that book-\ab{acc} Taro-\ab{nom} Hanako-\ab{dat} give-\ab{past} \\
      \hfill{(P + A + DAT + V)}
 \bg. taroo-ga sono hon-o hanako-ni yat-ta \\
      Taro-\ab{nom} that book-\ab{acc} Hanako-\ab{dat} give-\ab{past} \\
      \hfill{(A + P + DAT + V)}
 \bg. ?hanako-ni sono hon-o taroo-ga yat-ta \\
      Hanako-\ab{dat} that book-\ab{acc} Taro-\ab{nom} give-\ab{past} \\
      \hfill{(DAT + P + A + V)}
 \bg. ?sono hon-o hanako-ni taroo-ga yat-ta \\
      that book-\ab{acc} Hanako-\ab{dat}  Taro-\ab{nom} give-\ab{past} \\
      \hfill{(P + DAT + A + V)}
 \b.[] \hfill{\cite[260]{shibatani90}}

 \clearpage
In spoken Japanese,
NPs (and adverbs) sometimes appear post-predicatively as exemplified in \Next[b].
%
\ex.
 \ag. \EM{taroo-ga} ki-ta \\
      Taro-\ab{nom} come-\ab{past} \\
      `Taro came.' \hfill{(S + V)}
 \bg. ki-ta-yo \EM{taroo-ga} \\
      come-\ab{past}-\ab{fp} Taro-\ab{nom} \\
      Lit.~`Came, Taro.' \hfill{(V + S)}
 \b.[] \hfill{\cite[258--259]{shibatani90}}



Different theories are interested in different aspects of \isi{word order} phenomena in Japanese.
As far as I can see, generative linguists and psycholinguists are mainly interested in `scrambling':
\isi{word order} variations of the subject, the object, the dative, and possibly other arguments,
all of which appear before the predicate.
More recently, generative linguists have also been interested in the `left periphery',
which is tightly connected with \isi{information structure}.
Some construction grammarians study dative-alternation-like phenomena in Japanese.%
 \footnote{
 I do not discuss the dative alternation in this study.
 See \citeA{nakamotoetal06},
 who found that the choice between DAT+P+V and P+DAT+V is determined
 by the meaning of a sentence as a whole.
 More specifically, they showed that P+DAT+V is preferred for caused motion.
 On the other hand, their results also show that
 ``there is an overall tendency for Japanese speakers to prefer [DAT+P+V] order to [P+DAT+V]'' (p.~1).
 They argue that ``the strength of the preference is not constant among different supralexical meanings '' (ibid.).
 }
Functional linguists and, more recently, \isi{interactional} linguists have been interested in post-predicate constructions,
partially because they are mainly working on spoken language,
and post-predicate constructions in Japanese only appear in spoken language.
On the other hand,
traditional Japanese linguists have not discussed the \isi{word order} phenomena
that I am interested in \cite[except for][]{noda83}.
Instead of \isi{word order} variations, they concentrate on \isi{affix} ordering and dependency relations \cite[see e.g.,][]{saeki98}.

I outline previous studies on basic \isi{word order} and other word order variation in the following sections.
Note that different approaches are skewed to different sections for the reasons stated above.



%%----------------------------------------------------
\subsubsection{Basic word order}

As far as I can tell,
all Japanese linguists agree that
the basic \isi{word order} in Japanese is SOV (APV in terms of this study).
For example,
\citeA{shibatani90} states that
``Japanese is an `ideal' SOV (Subject-Object-Verb) language
in the sense that the \isi{word order} of `dependent-head' is consistently maintained with regard to all types of constituent'' (p.~257).
%
\ex.
 \ag. taroo-ga ki-ta \\
      Taro-\ab{nom} come-\ab{past} \\
      `Taro came.' \hfill{(S + V)}
 \bg. taroo-wa ki-ta-ka \\
      Taro-\ab{top} come-\ab{past}-\ab{q} \\
      `Did Taro come?'  \hfill{(S (Topic) + V)}
 \bg. taroo-ga hon-o kat-ta \\
      Taro-\ab{nom} book-\ab{acc} buy-\ab{past} \\
      `Taro bought a book.' \hfill{(A + P + V)}
 \bg. taroo-ga hanako-ni hon-o yat-ta \\
      Taro-\ab{nom} Hanako-\ab{dat} book-\ab{acc} give-\ab{past} \\
      `Taro gave a book to Hanako.' \hfill{(A + DAT + P + V)}
 \bg. taroo-ga nani-o kat-ta-ka sira-nai \\
      Taro-\ab{nom} what-\ab{acc} buy-\ab{past}-\ab{q} know-\ab{neg} \\
      `(I) don't know what Taro bought.'  \hfill{(Clause + V)}
 \b.[]     \hfill{\cite[257--258]{shibatani90}}


\citeA{chujo83} conducted a sentence-comprehension experiment and
reports that it takes longer to judge the grammaticality of the PAV order than that of the APV order.%
 \footnote{
 There is one exceptional case:
 if P is human and is not followed by the particle \ci{o},
 the time difference between APV and PAV disappears.
 }
It has also been confirmed that
the PAV order is more difficult to process than the basic APV order 
in other experiments such as
phrase-by-phrase reading tasks \cite{miyamototakahashi02},
eye-movement experiments \cite{mazukaetal02}, and
ERP experiments \cite{uenokluender03}.
%fMRI \cite{kimetal04}.

In my data from \ci{the Corpus of Spontaneous Japanese},
which will be explained in the next chapter,
39 examples appear in APV order,
whereas 9 examples appear in PAV order.
Therefore, APV is the basic (most frequent) \isi{word order} in the corpus.%
 \footnote{
 Other non-verb-final orders such as VAP or AVP are extremely rare.
 }
Note, however, that
these numbers are very small compared to
examples where a single full NP appears in a clause;
644 examples appear in the SV order,
336 examples appear in the PV order (without A), and
526 examples appear in the DAT + V order.%
 \footnote{
 However, the AV pattern appears only in 8 examples.
 }
That clauses with two or more full NPs within the same clause are infrequent
has already been reported for Japanese \cite{matsumoto03} and for other languages \cite{dubois87,dryer97},
and the observation is also supported in my data.


%%% Nemoto, フィアラに言及

%%----------------------------------------------------
\subsubsection{Clause-initial elements}

Although clause-initial NPs can also be called
``preposed'' or ``scrambled'' NPs,
I call them clause-initial
because terms like ``preposing'' and ``scrambling'' assume
movement of the NPs.
Some even call all clause-initial NPs ``topicalized'' NPs, a term that I do not employ either because
it already attributes a special function to the NPs in question.
On the other hand, the term ``clause-initial'' does not assume movement or any other special function of clause-initial NPs.

%%----------------------------------------------------
\paragraph{Topic}

Functional linguists and recent generative grammarians who are working on cartography agree that topic-like NPs appear clause initially.
As has traditionally been pointed out,
topics, which correlate with given information,
tend to appear clause-initially
\cite{mathesius28,firbas64,danes70,kuno78}.
These topics function as ``anchors''
that associate previous and upcoming utterances.
Generative grammarians \cite[e.g.,][]{endo14} assume the universal hierarchy in \Next proposed by \citeA{rizzi04}
and argue that Japanese also follows this hierarchy.
In generative grammar,
it is assumed that a language (structure) is uniform
unless there is strong counter-evidence for it
\cite[the Uniformity Principle:][2]{chomsky01}.
%
 \ex. Force Top* Int Top* Focus Mod* Top* Fin IP
    \hfill{\cite[242]{rizzi04}}

``Force'' stands for clause types such as declarative, interrogative, and imperative;
``Top'' for \isi{topic},
``Int'' for higher \ci{wh}-elements \cite{rizzi01},
``Mod'' for modifiers such as adverbs, and
``Fin'' for finiteness.

\chd{
\citeA{ferreirayoshita03} conducted a \isi{production experiment} and found that
Japanese speakers produced given arguments before new arguments,
especially ``when the previous mention of the given argument involved the same lexical content'' (p.~688).
\citeA{imamura17} employed \ci{the Balanced Corpus of Contemporary Written Japanese} (BCCWJ) and concluded that
``the direct objects in OSV [non-canonical ``scrambled'' \isi{word order}] and \ci{wa}-marked entities are generally given information.
Yet, \isi{word order} changes from SOV [canonical \isi{word order}] to OSV do not influence the cataphoric prominence of a referent'' (p.~78).
}

%%----------------------------------------------------
\paragraph{Weight}
\hspace*{-1.5mm}Another important factor that affects \isi{word order} is the weight of the NP.
\citeA{yamashitachang01} pointed out that
in Japanese
heavy NPs tend to precede light NPs,
whereas in SVO languages like \ili{English}
light NPs precede heavy NPs \cite[e.g.,][]{arnoldetal00}.
They also report that topics and subjects tend to precede other NPs,
and that the weight and topichood of an NP compete to decide the order of the NPs \cite[see also][]{kondoyamashita07}.


%%----------------------------------------------------
\paragraph{Remaining issues}

The previous literature agrees that topics,
correlating with given information, appear clause-initially.
This is also motivated from a cognitive perspective.
The results of Chapter \ref{WordOrder}, however, show that
not all given elements appear clause-initially.
Moreover, there are post-predicate elements which correspond to topics in Japanese.
It is therefore also necessary to explain why some topics appear after the predicate.
In Chapter \ref{WordOrder},
I will show that sharedness,
rather than givenness in general,
affects \isi{word order} in Japanese, and that
\isi{activation status} determines whether
NPs appear clause-initially or post-predicatively.
Also, whether the referent in question is mentioned in the following \isi{discourse} or not affects \isi{word order} in addition to the effect of particles,
contrary to the finding of \citeA{imamura17}.

%%----------------------------------------------------
\subsubsection{Post-predicate elements}\label{Back:CharJ:WO:PostP}

I call NPs that appear after the predicate ``post-predicate'' or ``postposed'' NPs.
As has been stated earlier,
they appear mainly in the spoken language.
While adverbs and noun-modifying phrases are also postposed frequently in conversation,
the present study only discusses postposed NPs,
which are exemplified in \Next.
%
\ex.
 \ag. yurusite kun-nai-yo \EM{syatyoo-ga} \\
      allow give-\ab{neg}-\ab{fp} president-\ab{nom} \\
      `(He) would not allow (us to do such a thing), the president.'
      \hfill{\cite[431]{onosuzuki92}}
 \bg. omosiroi-kamo \EM{haikei-ga} \\
      interesting-may.be background-\ab{nom} \\
      `It's interesting, the background.'
      \hfill{\cite[9]{nakagawaetal08_paper}}


%%----------------------------------------------------
\paragraph{Afterthoughts}

Some researchers consider postposed elements to be ``afterthoughts'' \cite[259]{shibatani90}:
a clarification for an omitted element.
\citeA{kuno78,hinds82}; and \citeA{onosuzuki92} also make a similar point.
However,
it has been pointed out that
some postposed elements are produced in a coherent \isi{intonation contour} without pause (\citeA[436]{onosuzuki92}; \citeA[\S 2]{ono07}),
which suggests the possibility that
the speaker does not have time to plan to produce the \isi{postposed part};
rather, the \isi{postposed part} has been planned as such.




%%----------------------------------------------------
\paragraph{Non-focus}

\citeA{takami95b},
modifying \citeA{kuno78},
proposes that NPs that are postposed are not foci.
When focus NPs are postposed,
the sentences are not acceptable,
as shown in \Next,
where the \ci{wh}-word \ci{nani} `what' in \Next[a] and
\ci{mizu} `water' in \Next[b] are considered to be foci.
%
\ex.
% \ag. *kimi-wa tabe-ta-n-desu-ka \EM{nani-o} \\
%      \ab{2}\ab{sg}-\ab{top} eat-\ab{past}-\ab{nmlz}-\ab{plt}-\ab{q} what-\ci{o} \\
%      `Did you eat, what?'
%      \hfill{\cite[227]{takami95a}}
 \ag. *ato iti-nen-de teinen-nan-desu-ka \EM{dare-ga} \\
      remaining one-year-within retiring-\ab{cop}-\ab{plt}-\ab{q} who-\ab{nom} \\
      `Is (he) going to retire within a year, who?'
      \hfill{\cite[160]{takami95b}}      
 \bg. ??boku-wa nomi-tai \EM{mizu-ga} \\
      \ab{1}\ab{sg}-\ab{top} drink-want water-\ab{nom} \\
      `I want to drink, water.'
      \hfill{(op.cit.: p.~161)}

\citeA{takami95a} argues that
the NPs in the following examples can be postposed because
they are not the most important information,
although they are part of the focus.
%
\ex.
 \ag. akegata yatto umare-masi-ta \EM{otoko-no} \EM{ko-ga} \\
      dawn finally born-\ab{plt}-\ab{past} male-\ab{gen} child-\ab{nom} \\
      `At dawn, (he) was finally born, a male baby.'
 \bg. taroo-wa hanako-ni katte yat-ta-yo \EM{zyuk-karatto-no} \EM{daiya-no} \EM{yubiwa-o} \\
      Taro-\ab{top} Hanako-\ab{dat} buy give-\ab{past}-\ab{fp} 10-carat-\ab{gen} diamond-\ab{gen} ring-\ab{acc} \\
      `Taro gave Hanako, a 10-carat diamond ring.'
      \hfill{\cite[236]{takami95a}}

I suppose that Takami's ``important information'' is equal to focus.
In \Last, part of the focus is postposed,
but it is not ``the most focalized part'';
so the sentences in \Last are acceptable.
Therefore,
Takami's generalization that foci (or the most focalized part) cannot be postposed still holds.

\citeA{fujii91} argues that pragmatically important parts (such as focus and contrast) are uttered first,
which results in postposed constructions.
I consider this argument to be similar to Takami's argument and
include Fujii in this section of postposed elements as non-focus.


%%----------------------------------------------------
\paragraph{Emphasis}
\largerpage
\citeA{hinds82} argues that some postposed elements add emphasis to the \isi{utterance}.
\citeA[437]{onosuzuki92} also highlight postposed elements that
``strengthen the speaker's stance toward the proposition.''
% \footnote{
% They made this statement on \isi{adverbial} phrases.
% However, I believe that their ``\isi{emotive} type'' has similar characteristics
% }

Although it is not clear how to identify ``emphasis'',
their argument is important at least in two ways.
First,
when the postposed elements are produced in a coherent contour with the predicate,
they are similar to final particles such as \ci{ne} and \ci{yo}.
For example, in \Next,
the \isi{postposed element} \ci{watasi} `I' follows the final particle \ci{yo}.
%
\exg. sukii itte ki-masi-ta-\EM{yo} \EM{watasi} \\
      ski go come-\ab{plt}-\ab{past}-\ab{fp} \ab{1}\ab{sg} \\
      `(I) went skiing, me.'
      \hfill{\cite[438]{onosuzuki92}}

Given that final particles can appear in a row (e.g., \ci{oisii \EM{yo ne}} `good, isn't it?'),
it is no wonder that postposed elements behave as final particles, adding some kind of speaker attitude toward the proposition.

Second,
as \citeA{onosuzuki92} pointed out,
the implicatures of some \isi{postposed constructions} are dramatically different from the corresponding pre-pred\-i\-cate constructions. %
%%%ISSUE OUT OF ALIGNMENT.
For example, compare \Next[a] and \Next[b], which are composed of exactly the same elements and only differ in their word order.
In \Next[a], \ci{sore} `that' is postposed;
in \Next[b], \ci{sore} is in the basic position.
Therefore, they are expected to convey exactly the same meaning.
However, \Next[a] is not a simple question;
rather it is closer to a rhetorical question implying that the speaker doesn't like \ci{sore}.
On the other hand, \Next[b] is a simple neutral question.
%
\ex.
 \ag. nani \EM{sore} \\
      what that \\
      `What!?'
      \hfill{(op.cit.: p.~440)}
 \bg. \EM{sore} nani \\
       that what \\
       `What's that?'


Based on the evidence discussed above,
\citeA{ono07} claims that the \isi{postposed construction} has already been grammaticalized and is part of Japanese grammar.


%%----------------------------------------------------
\paragraph{Activation cost}

\citeA{nakagawaetal08_paper} divided postposed NPs into two types
based on intonation,
following \citeA{onosuzuki92}:
\isi{postposed elements} uttered in the same \isi{intonation contour} as the predicate (single-contour type) and
the ones uttered separately from the predicate (double-contour type).
They measured the Referential Distance (RD) between the \isi{postposed element} in question and and its immediate \isi{antecedent} by inter-pausal unit.
The RD approximates the \isi{activation cost} of the referent.
A smaller RD indicates that the referent has been mentioned relatively recently and hence the \isi{activation cost} is low;
a larger RD indicates that it has been mentioned less recently
and hence the \isi{activation cost} is high.

Nakagawa et al.~found that
the RD of the single-contour type is much smaller than that of the double-contour type.
They argue that the \isi{activation cost} of the single-contour type is small
and the referent is discussed currently as a \isi{topic}.
On the other hand, they report that the double-contour type is affected by multiple factors.

\newpage
%%----------------------------------------------------
\paragraph{Preferred {interactional} structure}

\citeA{tanaka05} argues that
\isi{interactional} factors affect \isi{word order} in Japanese conversation.
In sequences of conversation,
there are preferred and dispreferred organizations \cite{schegloffetal77,heritage84,pomerantz84}.
Preferred organizations are, for example,
an assessment followed by agreement and a request followed by acceptance.
On the other hand,
dispreferred organizations include
an assessment followed by disagreement and a request followed by refusal.
Preferred second parts -- such as agreement following an assessment or acceptance following a request -- are simple, direct, and are uttered without delay.
On the other hand,
dispreferred second parts -- such as disagreement following an assessment and refusal following a request -- are complex, indirect, and are uttered with delay.
\citeA[332ff.]{levinson83} compares preferred vs.~dispreferred organizations to unmarkedness vs.~markedness in morphology.

Based on this,
\citeA{tanaka05} found that preferred second parts begin with the predicate, followed by NPs and other adverbs and \isi{adverbial} clauses,
while dispreferred second parts end with the predicate,
preceded by NPs and other elements.
Tanaka argues that this contrast is observed because
it is the predicate that expresses the conclusion, i.e.\ the agreement, disagreement, acceptance, or refusal.

Let us take a closer look at an example of an assessment-agreement sequence.
In \Next,
Chikako (C), Keiko (K), and Emiko (E) are talking about current fashion trends which have been revived from their youth.
First, Chikako comments that current fashion is exactly the same as the fashion trends of their youth.
Then Keiko immediately agrees with Chikako
by uttering the predicate followed by an NP.
Note that the sign ``='' indicates that there is no pause between utterances.
%
\ex.
 \ag.[C:] ima-no katati-to \ul{ma}ttaku onnazi.= \\
                now-\ab{gen} form-\ab{com} exactly same \\
                `(It's) exactly the same shape as the ones in vogue now.'
 \bg.[K:] =\fbox{onnazi-yo}$\downarrow$ =[\EM{eri-mo} \\
          \hspace{0.2cm}same-\ab{fp} \hspace{0.3cm}collar-also \\
          `(It's) the same, the collar too.'
 \bg.[E:] {\hspace{2.5cm}} [a! honto::. \\
          {} oh really \\
          {\hspace{2.5cm}}`Oh re::ally.'
          \hfill{\cite[406]{tanaka05}}

On the other hand, in the next example -- a dispreferred second part --
the speaker delays the predicate expressing refusal
by putting several NPs and adverbs before the predicate.
In the context preceding the second part in \Next,%
 \footnote{
 I modified the transcription symbol ``- (hyphen)'' to ``{\textasciitilde} (tilde)'' because hyphens are used to express morphological boundaries in this study.
 The tilde (originally, a hyphen) indicates a sudden stop of an \isi{utterance} (typically a word) on the way to utter it.
 I will not explain other transcription symbols here because
 they are irrelevant to the current discussion.
 For more detail on transcription symbols,
 see \citeA{jefferson04} and \citeA{hepburnbolden13}.
 }
the speaker was asked about the content of an advertisement in a magazine.
%
\ex.
 \ag. sono \EM{$<$\ul{nakami}$>$-made} tyotto-ne \\
      its \hspace{0.2cm}content-even a.bit-\ab{fp} \\
      `When it comes down to its contents, sort of...'
 \bg. \EM{kookoku-no}{\textasciitilde} gn \EM{$>$ga-tte-no-wa} tyotto \\
      advert-\ab{gen} \ab{df} \ab{nom}-\ab{quot}-\ab{nmlz}-\ab{top} a.bit \\
      `when it comes to (the content) of the advert, sort of...'
 \bg. \EM{kotira-de-wa} \\
      here-\ab{loc}-\ab{top} \\
      `on our side...'
 \bg. \fbox{wakara-nai}-n-desu-keredomo$<$, .hhhh \\
      know-\ab{neg}-\ab{nmlz}-\ab{plt}-though \\
      `(we) have no knowledge of.'
      \hfill{(op.cit.: 413)}

The speaker could have simply said ``we have no knowledge of (it)''
because all other NPs are clear from the context.
However, the speaker chose to utter the NPs (and adverbs) instead of omitting them presumably to delay the conclusion.

%%----------------------------------------------------
\paragraph{Remaining issues}

Postposed constructions have been well studied in various theories.
However, few studies examine the difference between postposed NPs and
other NPs such as clause-initial and pre-predicate NPs.
\citeA{tanaka05} does not explain why speakers sometimes produce
post-predicate elements and sometimes not.
In Chapter \ref{WordOrder},
I will investigate these three kinds of NP in terms of \isi{information structure},
especially \isi{activation cost}.
Also, I will discuss the possible \emph{raison d'{\^{e}}tre} of post-predicate elements.

%%% 他に読まないとダメそうな文献:田窪「統語構造と文脈情報」

%%----------------------------------------------------
\subsubsection{Pre-predicate elements}

I call NPs that appear immediately before the predicate pre-predicate elements.
The previous discussion of basic \isi{word order} in Japanese implied that
Ps most frequently appear pre-predicatively and that this is the basic order.
Following almost all theories, I assume that that Ps appear pre-predicatively in the basic \isi{word order} and
I provide a review of other characteristics of NPs that appear pre-predicatively.

%Alternative:
% I assume with most theories that Ps appear pre-predicatively in their basic word order, and provide a review of other characteristics of pre-predicative NPs


%%----------------------------------------------------
\paragraph{Unaccusativity}

\hspace*{-1.5mm}Since \citeA{perlmutter78},
it is widely assumed that there are two types of \isi{intransitive} verbs:
\isi{unergative} verbs, which involve an agent, and
\isi{unaccusative} verbs, which involve only a patient (theme).
Especially among generative linguists,
it is also assumed that the argument of an \isi{unergative} \isi{verb} syntactically appears in the same position as the subject (A) of \isi{transitive} clauses,
while the argument of an \isi{unaccusative} \isi{verb} appears in the same position as the object (P) of \isi{transitive} clauses.
\citeA{kageyama93}, who applied this idea to Japanese,
provides rich examples to support this analysis of the surface structures of Japanese sentences.
As can be seen in examples \Next to \ref{ExKageyamaUnacc}, \ci{otoko-no ko} `boy' -- which is the argument of an \isi{unergative} \isi{verb} in \NNext \--- appears in the same position as \ci{kodomo} `child' in \Next \--- which is the subject (A) of a \isi{transitive verb}. On the other hand, \ci{ki-no eda} `tree branch' in \ref{ExKageyamaUnacc}, which is the argument of an \isi{unaccusative} \isi{verb}, appears in the same position as \ci{ki-no eda} in \Next, the object (P) of a \isi{transitive verb} .
%
\ex. \tl{Transitive verb}
 \ag. \EMi{kodomo}-ga \EM{ki-no} \EM{eda}-o ot-ta \\
      child-\ab{nom} tree-\ab{gen} branch-\ab{acc} break-\ab{past} \\
      `A child broke a tree branch.'
 \b. \Tree [.VP [.NP$_{1}$ \EMi{kodomo}-ga ] [.V$^{\prime}$ \qroof{\EM{ki-no} \EM{eda}-o}.NP$_{2}$ [.V ot-ta ] ] ]
  \hfill{\cite[46]{kageyama93}}

\ex.\label{ExKageyamaUner} \tl{Intransitive (Unergative) verb}
 \ag. \EMi{otoko-no} \EM{ko}-ga abare-ta \\
      male-\ab{gen} child-\ab{nom} go.violent-\ab{past} \\
      `A boy went violent.'
 \b. \Tree [.VP \qroof{\EMi{otoko-no ko}-ga}.NP$_{1}$ [.V$^{\prime}$ [.V abare-ta ] ] ]
 \hfill{(ibid.)}

\ex.\label{ExKageyamaUnacc} \tl{Intransitive (Unaccusative) verb}
 \ag. \EM{ki-no} \EM{eda}-ga ore-ta \\
      tree-\ab{gen} branch-\ab{nom} break-\ab{past} \\
      `A tree branch broke.'
 \b. \Tree [.VP [.NP$_{1}$ ] [.V$^{\prime}$ \qroof{\EM{ki-no eda}}.NP$_{2}$ ] [.V ore-ta ] ]
 \hfill{(ibid.)}

The important point for our purposes is that
the arguments of \isi{unaccusative} verbs and the objects (P) of \isi{transitive} verbs
are structurally closer to the \isi{verb};
i.e., they appear pre-predicatively in Japanese, which is basically a \isi{verb-final language}.


%%----------------------------------------------------
\paragraph{Focus}

\citeA{kuno78} and \citeA{takami95a} point out that
pre-predicate elements are foci (``most important information'').
\citeA[\S 4.2.]{endo14} also states that foci appear pre-predicatively.
Compare the following examples.
In \Next[A],
where `Boston' appears pre-predicatively and is preceded by `Hanako',
responding only to Boston is felicitous \Next[A],
while responding only to Hanako is not \Next[A$^{\prime}$].
%
\ex.
 \a.[Q:] ziroo-wa \EM{hanako-to} \EM{bosuton-ni} it-ta? \\
         Jiro-\ab{top} Hanako-with Boston-\ab{dat} go-\ab{past} \\
         `Did Jiro go to Boston with Hanako?'
 \bg.[A:] un \EM{bosuton-ni} it-ta-yo \\
           yeah Boston-\ab{dat} go-\ab{past}-\ab{fp} \\
           `Yeah, I went to Boston.'
 \bg.[A$^{\prime}$:] *un \EM{hanako-to} it-ta-yo \\
           yeah Hanako-with go-\ab{past}-\ab{fp} \\
           `Yeah, I went with Hanako.'
           \hfill{\cite[52]{kuno78}}

In \Next,
on the other hand,
where `Hanako' is preceded by `Boston',
responding only to Hanako is a natural answer, as illustrated in \Next[A$^{\prime}$],
while responding only to Boston is not, as shown in \Next[A].
%
\ex.
 \a.[Q:] ziroo-wa \EM{bosuton-ni} \EM{hanako-to} it-ta? \\
         Jiro-\ab{top} Boston-\ab{dat} Hanako-with go-\ab{past} \\
         `Did Jiro go to Boston with Hanako?'
 \bg.[A:] *un \EM{bosuton-ni} it-ta-yo \\
           yeah Boston-\ab{dat} go-\ab{past}-\ab{fp} \\
           `Yeah, I went to Boston.'
 \bg.[A$^{\prime}$:] un \EM{hanako-to} it-ta-yo \\
           yeah Hanako-with go-\ab{past}-\ab{fp} \\
           `Yeah, I went with Hanako.'
           \hfill{\cite[54]{kuno78}}

This implies that focus appears pre-predicatively.
The results reported in Chapter \ref{WordOrder} basically support this observation.

%%----------------------------------------------------
\paragraph{Remaining issues}

The observations discussed in the literature above imply that
Ps, the arguments of \isi{unaccusative} verbs, and foci appear pre-predic\-a\-tively.
The results in Chapter \ref{WordOrder} show that both patienthood and \isi{newness} contribute to \isi{word order} in Japanese.
The next question is what kind of theory allows both patients and new elements to appear pre-predicatively.
Throughout this study,
I aim at showing the plausibility of a theory that captures multiple variables at the same time, i.e., the theory of competing motivations \cite{dubois85}.



%%----------------------------------------------------
\subsection{Intonation}\label{BackSubIntonation}

I employ the term intonation and prosody roughly in the same way. Here I outline studies on the associations between intonation and functions such as \isi{information structure}.
For detailed phonetic descriptions and analyses of Japanese intonation,
see \citeA{beckmanpierrehumbert84,pierrehumbertbeckman88,sugito94ed,venditti00,igarashietal06,igarashi15}. %
%%%ISSUE Igarashi15 OUT OF ALIGNMENT, SOMETHING WRONG IN PARENTHESIS
Also, I only discuss units smaller than the clause;
I do not discuss \isi{discourse} structure
although there are many interesting interactions between intonation and \isi{discourse} structure in Japanese \cite[e.g.,][]{nakajimaallen93,vendittiswerts96,muraiyamashita99,koisoetal03,okuboetal03,koisoishimoto12}.
I focus on 
%prominence (\S \ref{BackSubSubSecProminence}), nonprominence (\S \ref{BackSubSubSecNonProminence}), and down-stepping (\S \ref{BackSubSubSecDownStep})
%since they have been considered to be related to \isi{information structure}.
studies on intonation units and \isi{information structure}.
%%% イントネーションといえばフォーカスの研究だったが、本研究はユニットとしてのイントネーションとISの関連を調べる。
%%% prominenceとの関連を述べる

%%%----------------------------------------------------
%\subsection{Pitch accent in Japanese}\label{PitchAccentJapanese}
%
%Before intorucing how to define intonation units,
%I will briefly discuss accent patterns of Japanese as background knowledge.
%Standard spoken Japanese, which is investigated in this study, is a \isi{pitch accent} language,
%a language which lexically distinguish high vs.\ \isi{low pitch} with a single accent nucleus within a word.
%In standard Japanese,
%the accent nucleus specifies where the \isi{pitch} falls.
%Once the \isi{pitch} falls, it never rises within a word.
%Standard Japanese also has a principle that the pitches of the first and the second morae must be different;
%when the \isi{first mora} is high, the \isi{second mora} must be low,
%the the \isi{first mora} is low, the \isi{second mora} must be high.%
%	\footnote{
%	So-called ``special morae'' such as moraic nasal, geminates, and long vowels are exceptions.
%	}
%Words in standard Japanese are divided into those with accent nucleus and those without it.
%The vast majority of words with accent nucleus have the accent nucleus on the third-last (antepenultimate) mora,
%while minor words has accent nucleus somewhere else \cite{kubozono06}.
%To summarize, three-mora words in standard Japanese have accent patterns as shown in \Next.
%The bold face letter indicates accent nucleus,
%H indicates high \isi{pitch}, and
%L indicates \isi{low pitch}.%
%	\footnote{One might wonder how to distinguish (\ref{AccentPatternJ}a) and (\ref{AccentPatternJ}d).
%	When these words are followed by particles without accent specification such as \ci{ga} `\ab{nom}',
%	three-mora words without accent nucleus followed by \ci{ga} are pronounced as LHH-H,
%	while those with accent nucleus followed by \ci{ga} are pronounced as LH\nuc{H}-L,
%	which is schematized in \Next.
%	\ex.
%	\a.[(\ref{AccentPatternJ}a)] LHH-\ci{ga} $\to$ LHH-H 
%	\b.[(\ref{AccentPatternJ}d)] LH\nuc{H}-\ci{ga} $\to$ LH\nuc{H}-L
%	
%	 }
%%
%\ex.\label{AccentPatternJ}
%\begin{tabular}{lll}
% & Accent type & Example \\
% a. & LHH & nezumi `mouse' \\
% b. & \nuc{H}LL & \nuc{i}noti `life' \\
% c. & L\nuc{H}L  & ko\nuc{ko}ro `heart' \\
% d. & LH\nuc{H} & oto\nuc{ko} `man' \\
%\end{tabular}\\
%\begin{flushright}
% {\cite[][14]{kubozono06}}
%\end{flushright}
%
%Intonation is considered to interact with these accent rules in standard spoken Japanese.
%

%%----------------------------------------------------
\subsubsection{Definition of intonation unit}

Before reviewing the previous literature,
I briefly discuss how an \isi{intonation unit} is defined.
The definition of \isi{intonation unit} makes use of a labeling system for Japanese prosodic information called X-JToBI,
which has already been annotated in \ci{the Corpus of Spontaneous Japanese}.
I discuss X-JToBI in the following paragraph, and introduce intonation units afterwards. 

%%----------------------------------------------------
\paragraph{X-JToBI and intonational phrases}

X-JToBI \cite{maekawaetal02,igarashietal06} is based on J-ToBI, proposed in \citeA{venditti97,venditti00} --
which is itself modified from ToBI (Tones and Break Indices), a labeling system for \ili{English} prosody \cite{silvermanetal92,pitrellietal94,beckmanelam97}.%
%%%ISSUE igarashietal06 OUT OF ALIGNMENT

Here I mainly discuss the break indices (BI) tier of X-JToBI
since this is the most relevant feature for intonation units.
The BI labelings are determined by human annotators and represent the strength of the prosodic boundaries \cite{maekawaetal02,igarashietal06}.
BI labelings basically consist of \code{1}, \code{2}, and \code{3}.%
	\footnote{
	In addition, there are diacritics: \code{m, -, p}.
	There are also labels for disfluency;
	word fragments, fillers, and so on.
	See \citeA{igarashietal06} for a detailed description.
	}
\code{1} corresponds to a word boundary,
\code{2} corresponds to an \isi{accentual-phrase boundary}, and
\code{3} corresponds to an \isi{intonational-phrase boundary}.
An \isi{intonational phrase} consists of more than or equal to one \isi{accentual phrase}.
An \isi{accentual phrase} consists of a \isi{pitch contour} with a single F$_{0}$ peak.
Intonational-phrase boundaries are the place where a \isi{pitch reset} occurs;
if the \isi{pitch range} of the current \isi{accentual phrase} is smaller than the next \isi{accentual phrase},
an \isi{intonational-phrase boundary} is identified in the current \isi{accentual-phrase boundary}.

Below is an example of an \isi{intonational-phrase boundary} (label \code{3}),
the boundary type most relevant to our study.
Figure \ref{BIexF} shows the \isi{pitch contour} of the \isi{utterance} in \Next.
%
\exg. aoi yane-no ie-ga mieru \\
	blue roof-\ab{gen} house-\ab{nom} visible \\
	`A house with the blue roof is visible.'

The vertical lines in the figure across the \isi{pitch contour} indicate
the peak and the bottom of F$_{0}$.
A contour with a single \isi{pitch peak} corresponds to a single \isi{accentual phrase}.
Comparing the first (\ci{aoi} `blue') and the second (\ci{yane-no} `roof-\ab{gen}') accentual phrases,
the \isi{pitch range} of the second is smaller than the first one;
i.e., \isi{downstepping} occurs in the second \isi{accentual phrase}.
Downstepping, a.k.a.~catathesis, is ``a phonological process by which the [\isi{pitch}] range is compressed after a lexical accent'' (\citeA[17]{venditti00}, see \citeA{poser84,beckmanpierrehumbert84,pierrehumbertbeckman88,kubozono93}).
%This is schematically represented in Figure \ref{FigDownStep},
%where ``IP'' indicates \isi{intonational phrase} and ``AP'' indicates \isi{accentual phrase} \cite[18]{venditti00}.
In Figure \ref{BIexF}, the first \isi{accentual-phrase boundary} is not an intonational-phase boundary.
On the other hand,
comparing the second (\ci{yane-no} `roof-\ab{gen}') and the third (\ci{ie-ga} `house-\ab{nom}') accentual phrases,
the second \isi{pitch range} is smaller than the third one.
Therefore,
the second \isi{accentual-phrase boundary} is an \isi{intonational-phrase boundary}.

\begin{figure}
%\begin{minipage}{0.5\textwidth}
 \centering
 \includegraphics[width=0.7\textwidth]{figure_BIex.pdf}
 \caption{An example of annotation of BI \cite[][412]{igarashietal06}}
 \label{BIexF}
%\end{minipage}
%\end{figure}
%\begin{figure}
%\begin{minipage}{0.5\textwidth}
%\vspace{2cm}
% \centering
% \includegraphics[width=0.5\textwidth]{figure_Venditti00_18.pdf}
% \caption{A schematic representation of down-stepping and intonational-phrase reseting in Japanese \cite[18]{venditti00}}
% \label{FigDownStep}
%\end{minipage}
\end{figure}


%An \isi{intonational phrase} roughly corresponds to
%an \isi{intermediate phrase} and \isi{utterance} in \citeA{pierrehumbertbeckman88},
%and a major phrase and \isi{intonational phrase} in \citeA{selkirktateishi91}.
%The correspondence is not one-to-one because
%some level in a theory lacks in another theory.
%For example, \citeA{selkirktateishi91} assume three levels above the major phrase (or \isi{accentual phrase}) as shown in \Next,
%whereas \citeA{pierrehumbertbeckman88} assume no levels between the major phrase (their \isi{intermediate phrase}) and the \isi{utterance}.
%In X-JToBI, 
%Pierrehumbert-Beckman's intermediate level and \isi{utterance} are further integrated into a single level of intonation phrase,
%which means that there is a single level above the \isi{accentual phrase}.
%The correspondence is summarized as a table in \Next,
%where ``ST91'' indicates \citeA{selkirktateishi91} and
%``PB88'' indicates \citeA{pierrehumbertbeckman88}.%
% \footnote{
% In \citeA{selkirk09}, however, the distinction between
% major vs.~minor phrases is abandoned and only a single level, a phonological phrase, is hypothesized.
% The level of \isi{utterance} is also dismissed.
% }
%%
%\ex. \EM{Levels of prosody in different theories} \\
%\begin{tabular}{llll}
%\toprule
%   & ST91 & PB88 & X-JToBI \\
%\midrule
% a. & \EM{Utterance} & Utterance & \rdelim\}{3}{1.5in}[Intonational phrase]  \\
% b. & \EM{Intonational phrase} & \rdelim\}{2}{1.5in}[Intermediate phrase] & \\
% c. & \EM{Major phrase} & & \\
% d. & {Minor phrase} & & \\
%    & {(Accentual phrase)} & & \\
% e. & Prosodic word & & \\
% f. & Foot & & \\
% g. & Syllable & & \\
% h. & Mora & & \\
%\bottomrule
%\end{tabular}
%
%What is more complicated is that
%the definition of each level varies depending on theories.
%However, major phrase, \isi{intermediate phrase}, and \isi{intonational phrase} basically based on the domain of \isi{downstepping}.
%The present study is based on X-JToBI because
%this is the only label extensively annotated in a large spoken corpus, namely CSJ.
%However, I leave open the question of whether this label is the best or not.
%I will briefly discuss this issue in Chapter \ref{Intonation}.
%Note that, in reviewing the literature in the following section,
%different theories assume different levels of prosody and
%the definition of each level also varies.

%%----------------------------------------------------
\paragraph{Intonation unit}

Based on X-JToBI,
\citeA{denetal10} and \citeA{denetal11} propose the definition of \isi{intonation unit}
which I will employ in this study.
They call it short utterance-unit as opposed to long utterance-unit,
but I use the term ``\isi{intonation unit} (IU)'' throughout
since I do not discuss long utterance-units.
An intonation-unit boundary is identified
where there is an \isi{intonational phrase} (the boundary labelled as \code{3} in CSJ) discussed above,
a clause boundary,%
	\footnote{
	To be more precise, this is a long utterance-unit boundary.
	See \citeA{denetal11} for the definition of this unit.
	}
or
a pause equal to or more than 0.1 seconds.
As discussed in \citeA{enomotoetal04},
it is difficult for human annotators to agree when deciding on intonation-unit boundaries based on the system proposed in \citeA{duboisetal92} and \citeA{iwasaki08}.
Den and his colleagues made it possible to identify intonation units in \isi{spontaneous speech} consistently across annotators.

In the following section, however,
I review studies on various kinds of intonation units including those defined \citeA{duboisetal92,maekawaetal02,iwasaki08,denetal11}.
Also, whereas prominence marking, down-step\-ping, and boundary \isi{pitch} movements are more popular topics than intonation units,
I review those studies in relation to the current study.
See \citeA{vendittietal08} for an overview of such studies.


%%----------------------------------------------------
%\subsubsection{Prominence (pitch peak)}\label{BackSubSubSecProminence}
%
%Focus
%New \& non-adjacent Given elements \cite{venditti00}
%
%%----------------------------------------------------
%\subsubsection{Nonprominence (pitch valley)}\label{BackSubSubSecNonProminence}
%
%Adjacent given elements \cite{venditti00}
%
%%----------------------------------------------------
%\subsubsection{Down-stepping}\label{BackSubSubSecDownStep}





%%----------------------------------------------------
%\subsubsection{Pause}\label{BackSubSubSecPause}

%%% 会話分析による、日本語の話題導入の方法

%%----------------------------------------------------
\subsubsection{Intonation units and related phenomena}

In this section,
I present a review of the literature on the association between prosodic units and related characteristics of language.
Note again that
the review includes various kinds of prosodic units
based on slightly different definitions,
although they agree in many cases.

%%----------------------------------------------------
\paragraph{Prominence and downstepping}

Prominence and \isi{downstepping} are crucial features in determining intonation units.
It is well known that
a focus receives prominence (\isi{pitch peak}).
\citeA[99--101]{pierrehumbertbeckman88} report that
``sequences with focus on the noun almost always had an \isi{intermediate phrase} [i.e., \isi{intonational phrase}] boundary between the \isi{adjective} and the noun[...] an \isi{intermediate phrase} boundary blocks catathesis [i.e., \isi{downstepping}]''. The conclusion was reached through production experiments where subjects were asked to produce a sequence of an \isi{adjective} and a noun with different focus positions.
The target sentences and contexts used by Pierrehumbert and Beckman are like the ones in \Next.
The capital letters indicate that those words are in focus, and
the bold-faced letters indicate that they are the target of analysis.
%
%\ex.
% \a.[Q:] [In America,] are there sweet beans like there are in Japan?
% \bg.[A:] mame-wa ari-masu-ga \EM{AMAI} \EM{mame}-wa ari-mase-n \\
%      bean-\ab{top} exist-\ab{plt}-though sweet bean-\ab{top} exist-\ab{plt}-\ab{neg} \\
%      `There are beans, but there aren't SWEET beans.'
%  \b.[]    \hfill{\cite[59]{pierrehumbertbeckman88}}
%
\ex.
 \a.[Q:] [In America,] are there sweet beans or carrots like there are in Japan?
 \bg.[A:] amai {NINZIN}-wa ari-masu-ga \EM{amai} \EM{MAME}-wa ari-mase-n \\
          sweet carrot-\ab{top} exist-\ab{plt}-though sweet bean-\ab{top} exist-\ab{plt}-\ab{neg} \\
      `There are sweet CARROTS, but there aren't sweet BEANS.'
 \b.[]   \hfill{\cite[59]{pierrehumbertbeckman88}}

Pierrehumbert and Beckman showed that
there is an \isi{intonational phrase} (i.e., \isi{intermediate phrase}) boundary
between the \isi{adjective} (\ci{amai} `sweet' in \Last[A]) and the noun (\ci{mame} `bean' in \Last[b])
when the noun is a focus, as in \Last.
Although the results are complicated,
they conclude that their generalization applies to both accented and unaccented words.%
 \footnote{
 \citeA{kubozono07} compared two definitions of \isi{downstepping} (syntagmatic and paradigmatic) and investigated whether a \isi{pitch reset} occurs before the focus.
 He found conflicting results: from a syntagmatic perspective, the focus receives higher \isi{pitch} than the preceding phrase, which indicates that \isi{downstepping} is blocked.
 From a paradigmatic perspective, on the other hand, he had to conclude that \isi{downstepping} is not blocked before the focus.
 The present study employs the definition of syntagmatic \isi{downstepping}
 and assumes that the conclusions in \citeA{pierrehumbertbeckman88} and \citeA{kubozono07} do not contradict each other.
 See \citeA{kubozono07} for detailed discussion on this issue.
 }

%%% フォーカスがピッチリセットを引き起こすか(kubozono, PB, Poser)
%%% contrastive focusと普通のフォーカスの音調の違い (郡)


%%----------------------------------------------------
\paragraph{Focus projection}

There has been a cross-linguistic question of how
human beings distinguish \isi{broad focus} and \isi{narrow focus}:
%in the literature, there is the question of how human beings distinguish between \isi{broad focus} and \isi{narrow focus}
the issue of focus projection.
This has been investigated for \ili{English}, \ili{German} and \ili{Dutch}
\cite{selkirk84,gussenhoven83}.
\citeA{ito02}, who investigated this question in Japanese,
compared the response time and acceptability of each of the intonation types in \Next[A1-A3]
followed by a \isi{broad focus} question like \Next[Q].
The capital letters indicate the phrases whose \isi{pitch range} is expanded.
\ex.
 \ag.[Q:] yokoyama-kun-wa boonasu morat-tara doo suru-no \\
          Yokoyama-\ab{hon}-\ab{top} bonus get-\ab{cond} how do-\ab{q} \\
          `What will Mr.Yokoyama do when he gets a bonus?'
 \bg.[A1:] kare-wa \EM{DAIBINGU-o} \EM{HAZIMERU-n-da-yo} \\
           \ab{3}\ab{sg}-\ab{top} diving-\ab{acc} begin-\ab{nmlz}-\ab{cop}-\ab{fp} \\
           `He starts (scuba) diving.'
 \bg.[A2:] kare-wa \EM{DAIBINGU-o} \EM{hazimeru-n-da-yo} \\
           \ab{3}\ab{sg}-\ab{top} diving-\ab{acc} begin-\ab{nmlz}-\ab{cop}-\ab{fp} \\
           `He starts (scuba) diving.'
 \bg.[A3:] kare-wa \EM{daibingu-o} \EM{HAZIMERU-n-da-yo} \\
           \ab{3}\ab{sg}-\ab{top} diving-\ab{acc} begin-\ab{nmlz}-\ab{cop}-\ab{fp} \\
           `He starts (scuba) diving.'
% \ag. kare-wa \EM{ie-o} \EM{kariru-n-da-yo} \\
%      \ab{3}\ab{sg}-\ab{top} house-\ab{acc} rent-\ab{nmlz}-\ab{cop}-\ab{fp} \\
%      `He rents a house'
% \bg. kare-wa \EM{ie-o} {kariru-n-da-yo} \\
%      \ab{3}\ab{sg}-\ab{top} house-\ab{acc} rent-\ab{nmlz}-\ab{cop}-\ab{fp} \\
% \bg. kare-wa {ie-o} \EM{kariru-n-da-yo} \\
%      \ab{3}\ab{sg}-\ab{top} house-\ab{acc} rent-\ab{nmlz}-\ab{cop}-\ab{fp} \\
      \hfill{\cite[412]{ito02}}

Ito found that
``though dual prominence [like \Last[A1]] is preferred for answers to \isi{broad focus} questions,
utterances with a single intonational prominence on the object [like \Last[A2]] may be comprehended equally quickly as those with dual prominence'' (op.cit.: 413) -- 
where A1 is significantly more acceptable than A2.
Also, she reports that the response time and acceptability of the A3-type do not significantly differ from those of A1 and A2.
She concluded that
``it is possible that the relation between argument structure and intonational focus marking is not universal'' (ibid.).

%The results reported in \citeA{kori11}, on the other hand,
%suggest that the intonation of broad and \isi{narrow focus} is different.
%
\largerpage
\citeA{kori11} investigated the intonation of broad and \isi{narrow focus} and
reports that, by default,
only the first word receives \isi{pitch peak}, whereas the following word is
suppressed -- 
although some speakers put prominence on the second word too.
\Next[a] is the target sentence that he asked participants to read aloud
and \Next[b-c] are the contexts.
In \Next[b-c], both \ci{aoi} `blue' and \ci{mahuraa} `scarf' are focused, because both of them contrast with `red' and `gloves' or `sweater', respectively.
In \Next[d], \ci{aoi} `blue' is narrowly focused
because it is the only element that contrasts with `red', while `scarf' is not contrasted.
%
\ex.
 \ag. \EM{aoi} \EM{mahuraa}-dat-ta-n-desu \\
      blue scarf-\ab{cop}-\ab{past}-\ab{nmlz}-\ab{cop}.\ab{plt} \\
      `(It) was a blue scarf.'
 \b. I ordered \EMi{red gloves}, but I received \EM{a blue scarf}. \hfill{(Broad focus)}
 \b. I ordered \EMi{a red sweater}, but I received \EM{a blue scarf}.\hfill{(Broad focus)}
 \b. I ordered \EMi{a red scarf}, but I received \EM{a blue scarf}. \hfill{(Narrow focus)}

Kori concludes that
the default intonation for \isi{broad focus} is to suppress the second word (\ci{mahuraa} `scarf' in this case)
because most of the participants produced the sentences as such,
although some participants chose the sentence with prominence both on \ci{aoi} `blue' and \ci{mahuraa} `scarf'
when they were asked to choose a good sentence.

%%%----------------------------------------------------
%\paragraph{Syntactic structure}
%
%\citeA{selkirk09} propose the hypothesis that
%``the clause structure of a sentence in Japanese
%corresponds to a domain for certain of the phonological and phonetic phenomena that contribute to defining the intonational patterns of Japanese sentence.
%This domain has been referred to as the $\iota$-domain, or \isi{intonational phrase}'' \cite[66]{selkirk09}.
%Selkirk proposes two pieces of evidence that support her hypothesis,
%one of which is from \citeA{kawaharashinya08},
%who investigated the intonation of gapping and coordination in Japanese.
%Here I only introduce the first piece of evidence Selkirk provides
%because the second one seems to me to require further investigations.
%Kawahara and Shinya assume that
%prosodic and syntactic structures have some correspondence
%and explored the associations between them.
%Assuming that there are four levels above the phonological word,
%they hypothesized the correspondence as schematized in \Next.
%%
%\exg.
% {Syntactic boundaries:} $_{Sentence}$[... $_{Clause}$[... $_{VP}$[... {\hspace{0.2cm}} $_{NP}$[... ] ...] ...] ...] \\
% {\it Prosodic boundaries:} Utt IP MaP {\hspace{0.2cm}} MiP \\
% \hfill{\cite[65]{kawaharashinya08}}
%
%In a clause, corresponding to their \isi{intonational phrase},
%they found initial rise, \isi{pitch reset}, final lowering, final pause, and final creakiness.
%In a VP, assumed to correspond to their major phrase,
%they found initial rise and \isi{pitch reset},
%both of which are smaller than those of intonational phrases,
%but found no final lowering, final pause, and final creakiness.
%
%%The second piece of evidence for the hypothesis of syntax-phonology correspondence is from \citeA{ishihara02}.
%%\exg. 
%
%%%% 統語構造の曖昧性を韻律で解決する (心理言語学の人たち)
%%%% Kawahara & Shinya (2008): 節頭の\isi{pitch} riseのほうが節中の\isi{pitch} riseよりも大きい → intonational phraseの存在。intonational phraseは統語構造と一致している。(Selkirk, 2009)
%%%% Deguchi & Kitagawa (2002); Ishihara (2002): 
%
%Functions of prosody to disambiguate syntactic structures are also well studied in the literature.
%For example,
%\citeA{uyenoetal80} studied how prosody affects the interpretation of syntactically ambiguous sentences like \Next,
%where \ci{ototoi} `the day before yesterday' is ambiguous over
%whether it is included in the relative clause (left-branching interpretation) or in the \isi{main clause} (center-embedded interpretation).
%They manipulated the \isi{pitch} peaks of relative clauses
%and had participants listen to the recording and judge whether the sentences are interpreted as left-branching or center-embedded.
%%
%\ex.
% \ag. [\EM{ototoi} koron-da otona]-ga warat-ta \\
%      day.before.yesterday fall-\ab{past} adult-\ab{nom} laugh-\ab{past} \\
%      `The adult who fell the day before yesterday laughed.'
%      \hfill{(Left-branching)}
% \bg. \EM{ototoi} [koron-da otona]-ga warat-ta \\
%      day.before.yesterday fall-\ab{past} adult-\ab{nom} laugh-\ab{past} \\
%      `The adult who fell laughed the day before yesterday.'
%      \hfill{(Center-embedded)}
% \b.[] \hfill{\cite[225]{uyenoetal80}}
%
%They found that
%``[w]hen the \isi{pitch} assigned to the relative clause is the same or higher than that of the preceding portion of the sentence,
%it tends to be interpreted as center-embedded, otherwise as left-branching'' (op.cit.: 234).

%%----------------------------------------------------
\paragraph{Functional and cognitive motivations for intonation units}
\largerpage
\citeA{iwasaki93},
applying the style of IU identification proposed in \citeA{duboisetal92} and \citeA{chafe94}  to Japanese,
argues that a Japanese \isi{intonation unit} corresponds to a phrase rather than a clause,
in contrast to the English IU, which corresponds to a clause according to \citeA{chafe87,chafe94}.
According to Iwasaki's survey,
42.2\% of IUs in Japanese are \isi{clausal}, whereas 57.8\% are phrasal.
%\isi{clausal} IUs in Japanese are 42.2\%,
%whereas phrasal IUs are 57.8\%.
Their \isi{intonation unit} is a ``stretch of speech uttered under a single coherent \isi{intonation contour}'' \cite[17]{duboisetal92}.
\citeA[39]{iwasaki93} states that
the beginning of an IU ``is often, though not always, marked by a pause, hesitation noises, and/or resetting of the baseline \isi{pitch} level'',
whereas the ending of an IU ``is often, again though not always, marked by a lengthening of the last syllable.''
\citeA{iwasaki93} provides \Next to exemplify how intonation units in Japanese correspond to phrases.
Each line in \Next corresponds to a single \isi{intonation unit}
and \Next[a-e] as a whole consist of a single proposition,
``I heard that broadcast at home with my family.''
%
\ex.
 \ag. atasi-wa-ne:* \\
      \ab{1}\ab{sg}-\ab{top}-\ab{fp} \\
      `I, you know...'
 \bg. uti-de kii-ta-no-ne? \\
      home-\ab{loc} hear-\ab{past}-\ab{nmlz}-\ab{fp} \\
      `heard at home, you know...'
 \bg. sono are-wa-ne? \\
      that that-\ab{top}-\ab{fp} \\
      `that thing, you know...'
 \bg. hoosoo-wa-ne? \\
      broadcast-\ab{top}-\ab{fp} \\
      `that broadcast, you know,'
 \bg. kazoku-de. \\
      family-with \\
      `with my family.'
 \hfill{\cite[40]{iwasaki93}}

The \isi{pitch} and intensity of \Next are shown in Figure \ref{IUExF} from
\citeA[109]{iwasaki08}, in which the same example and figure are explained.
%where he explains the same example with the figure.
The IU \Next[a] ends with final \isi{vowel} lengthening,
whereas boundary \isi{pitch} movements are observed in the ending of IUs \Next[b-d],
which are indicated by ``?''.
\Next[e] ends with a final lowering, indicated by ``.''.

%Iwasaki divided ``functional components'' into four types.
Iwasaki distinguishes between four types of "functional components":
\ex.
 \tl{Four functional components}
 \a. \EM{Lead (LD)} such as fillers, which have no substantial meaning.
 \b. \EM{Ideation (ID)}, which conveys the content of speech.
 \b. \EM{Cohesion (CO)} such as conjunctives and \ci{wa}, which relate the previous and the current IUs.
 \b. \EM{Interaction (IT)} such as \ci{ne} `\ab{fp}' and \ci{yo} `\ab{fp}', which are associated with communication.

Based on this,
he shows similarities among different IUs.
For example, 
\Next[a] is an IU which only contains an NP followed by particles, whereas
\Next[b] is an IU which only contains a VP, also followed by particles.
%According to the functional component classification in \Last,
The structure of these two IUs is essentially the same in terms of functional components,
although they are different in terms of grammatical structure.
%
\ex.
 \a.
  \glll [mami-ni-dake] [-wa] [-ne] \\
        Mami-\ab{dat}-only -\ab{top} -\ab{fp} \\
        \EM{ID} \EM{CO} \EM{IT} \\
 \b.
  \glll [ik-ase-ta-rasii] [-no] [-yo] \\
         go-\ab{caus}-\ab{rep} -\ab{nmlz} -\ab{fp} \\
         \EM{ID} \EM{CO} \EM{IT} \\
  \glt  `(I heard that she) let only Mami go.'
% \a.
% \glll [sooyuu sito-ga siki si] [-te] [-ne*] \\
%       such person-\ab{nom} lead do -and -\ab{fp} \\
%       \EM{ID} {} {} {} \EM{CO} \EM{IT} \\
% \glt `Such people led and...'
% \b.
% \glll [sinin-o asoko-e minnna] [-ne*] \\
%        corpse-\ab{acc} there-\ab{gl} all -\ab{fp} \\
%        \EM{ID} {} {} \EM{IT} \\
% \glt   `all the corpses to there...'
% \b.
% \glll [ano] [dote-no ue-e] [-sa*] \\
%       \ab{fl} bank-\ab{gen} top-\ab{gl} -\ab{fp} \\
%       \EM{LD} \EM{ID} {} \EM{IT} \\
% \glt  `to the top of the bank...'
% \b.
% \glll [atume] [-te.] \\
%      gather and \\
%      \EM{ID} \EM{CO} \\
% \glt `gathered and...'
% \hfill{\cite[47]{iwasaki93}}

Iwasaki analyzed his data based on his classification and
found that more than 80\% of the IUs consist of
two or less functional components.
He states that
``this might be due to the limitation of work that the speaker can handle within one IU. [...] Japanese speakers [...] are faced with a constraint which permits them to exercise up to two functions per \isi{intonation unit}'' (p.~49).


\begin{figure}
 \centering
 \includegraphics[width=0.85\textwidth]{sounds/iwasaki_IU.png}
 \caption{Example of an intonation unit \cite[109]{iwasaki08}}
 \label{IUExF}
\end{figure}


By contrast, \citeA[68]{matsumoto00} reports that
``one clause comprises an average of 1.2 IUs''
and argues that
``the clause is the syntactic exponent of Japanese substantive IU''.
She proposes the ``one new NP per IU'' constraint in Japanese,
comparing it to the one new idea at a time constraint in \citeA{chafe87,chafe94}.
However, \citeA[\S 5.6]{matsumoto03} also reports that
one new or given NP per IU is preferred in Japanese conversation.
Therefore,
new as well as given NPs appear in an \isi{intonation unit}
without other NPs.
%although she proposes ``one new NP per IU'' constraint
%also in this paper.


%\citeA{matsumoto03} extensively investigates the association between
%intonation units and syntactic structure, \isi{information structure}, and functional structure.
%Here I review her study focusing on the association between
%intonation units and \isi{information structure}.

%%%\citeA{sakutafujii}

\citeA{nakagawaetal10_piu} focused on the difference between
phrasal IUs and \isi{clausal} IUs and
analyzed them in terms of \isi{information structure}.
They measured referential distance and persistence \cite{givon83} and concluded that
one of the functions of phrasal IUs is to introduce or re-introduce
important topics in \isi{discourse}.
They compare this function of phrasal IUs to left-dislocations observed in many languages.

%%----------------------------------------------------
%\paragraph{Other factors}
%
%length of a unit % \citeA{ghini93}
%speech rate % \citeA{hayeslahiri91}



%%----------------------------------------------------
\paragraph{Remaining issues}

Most studies on phonetics and phonology concentrate on
foci rather than topics.
Among different focus types, most of the studies (except for those on focus projection) concentrate on \isi{narrow focus} rather than \isi{broad focus}.
Moreover, almost all of them are experimental studies rather than
corpus studies.
By contrast, I focus here on
the differences between broad foci and topics in \isi{spontaneous speech},
although I also carry out a \isi{production experiment}.

Previous functional studies such as \citeA{iwasaki93,matsumoto00,matsumoto03}; and \citeA{nakagawaetal10_piu}
have methodological issues since they rely on an impressionistic definition of intonation units.
This study, on the contrary, is based on a
strict definition of intonation unit and
aims at revealing associations between intonation and \isi{information structure}.

The results in Chapter \ref{Intonation} show that
an \isi{intonation unit} corresponds to a unit of \isi{information structure} -- e.g., \isi{topic} or focus --
which frequently but not always overlaps with a unit of the syntactic structure.

%%----------------------------------------------------
\subsubsection{Pause}

\citeA{sugito94} showed in a perceptual experiment that
pauses appear before \isi{pitch reset}. %by means of a perceptual experiment.
She recorded trained announcers reading the news and had subjects listen to the recording.
She found that, when pauses were eliminated,
subjects perceived the voice as though two people were overlapping with each other when the pauses were substituted by \isi{pitch} resets.
According to her,
it is in fact impossible to reset \isi{pitch} without pauses and
vocal cords are tensed 0.1 seconds before speech production.
Based on this, I assume that pauses correlate with \isi{pitch reset}.

%%----------------------------------------------------
%%----------------------------------------------------
\section{Summary}

In this chapter,
I outlined the previous literature on topics and foci as well as the characteristics of Japanese relevant to this study,
and enumerated the remaining questions to be investigated.

In Chapters from \ref{Particles} to \ref{Intonation},
I investigate the associations between \isi{information structure} and particles, \isi{word order}, and intonation in spoken Japanese.
Before this, in the next chapter I introduce the framework adopted in this study.




