\documentclass[output=paper,
modfonts,nonflat
]{langsci/langscibook} 
\author{Bruna Franchetto\affiliation{Museu Nacional, Federal University of Rio de Janeiro, Brazil}\lastand 
Kristine Stenzel\affiliation{Federal University of Rio de Janeiro, Brazil}
}
\title{Amazonian narrative verbal arts and typological gems}   
\abstract{\noabstract}

\ourchaptersubtitle{~}
\ourchaptersubtitletrans{~}
\ChapterDOI{10.5281/zenodo.1008775}

\maketitle

\begin{document}  
\section{Origins}
This volume owes its development to a confluence of circumstances, not least of which is the veritable explosion of scholarship on Amazonian languages that has taken place over the last several decades. Though the description and analysis of the 300 or so still-existing languages spoken in Amazonia\footnote{Following \citealt{EppsSal2013} “Amazonia" is understood here as comprising both the Amazon and Orinoco basins, covering parts of Brazil, Bolivia, Peru, Ecuador, Colombia, Venezuela, Suriname, and the Guianas. For more on the distribution and state of endangerment of Amazonian languages, see \citealt{Moore2008}.} is still far from comprehensive, repositories of linguistic and anthropological academic references, such as the \textit{Etnolinguistica} web site, clearly reflect exponential growth in the field since the 1990s.\footnote{\url{http://www.etnolinguistica.org/}. Of the 358 dissertations or theses on Amazonian languages on file as of May 2017, just 6 were written before 1980, the number jumping to 19 during the next decade and then to 41 during the 1990s (representing some 18 percent of the total on record). Between 2000 and 2010, contributions increased more than fourfold, to 170 (47 percent of the archive), and another 123 have been added in the last six years. We should note that researchers make their own academic works available on this site, so the numbers cited do not represent a fully comprehensive view of all scholarship.}This same period of expanding academic focus on Amazonian languages also saw the rise of new language documentation efforts and the establishment of archives of cultural and linguistic materials in which languages of the region are well represented.\footnote{The DoBeS archive (Volkswagen Foundation, Germany) has materials from 14 Amazonian languages; ELAR (Endangered Languages Archive, University of London/SOAS) over 40; AILLA (Archive of the Indigenous Languages of the Americas, University of Texas Austin) an additional 60. More than 80 languages are included in the documentation archive maintained at the Emilio Goeldi Museum (MPEG, in Pará, Brazil) and another 18 in  Indigenous Languages Documentation Project (PRODOCLIN) archive at the Museum of Indigenous Peoples (Museu do Índio/FUNAI, Rio de Janeiro, Brazil).}
The interdisciplinary and highly collaborative nature of most new documentation projects in Amazonia\footnote{The “participatory" or “collaborative" paradigm is widely adopted in current documentation projects in Amazonia, which prioritize training of indigenous researchers and high levels of community involvement (see \citealt{Franchetto2014}; \citealt{Stenzel2014}).}
has in turn strengthened dialog between anthropologists and field linguists who recognize the narrative genre as a prime source of both cultural understanding and verbal artistry, especially when offered by knowledgeable and eloquent orators such as those whose voices are represented here. Thus, text analysis — a longstanding element of language documentation in classic Boasian terms — is itself making a welcome comeback.
 
\largerpage
Our idea to gather a set of narratives from recent documentation projects into an organized volume is a product of this renaissance.\footnote{As is the \textit{Texts in the Indigenous Languages of the Americas} series, a recently re-established yearly supplement to the \textit{International Journal of American Linguistics.}} However, as word of our initiative began to circulate, the response from interested colleagues quickly threatened to swell the project to near-Amazonian proportions, and we found ourselves forced to make difficult choices. Fully recognizing that our final selection is but a sample of the rich materials available, we can only hope to see more collections of this type organized in the future.
    
The narratives themselves led us to organize the volume into three broad themes that are highly significant for Amazonian ethnology and its recent developments. The first theme — \textit{Life, death, and the world beyond} — refers to crucial cosmological dimensions and forces us to rethink notions such as death, the dead, life, embodiment, the soul, the spirit, and post-mortem destiny, which are often not well translated or are cannibalized by Western/non-indigenous concepts. The second theme — \textit{Beginnings} — includes fragments of Amerindian philosophy, in which reflection on the origin of beings does not pass through \textit{ex-nihilo} creation, there being no “genesis" in the Judeo-Christian vein. The third theme — \textit{Ancestors and tricksters} — introduces us to a few members of the Amerindian repertoire of comic and crafty characters, and leads us to memories of historical events and into realms of relations, whether among relatives or between enemies, that lie at the heart of societal living, with all its fluid frontiers and rituals.

\newpage
\largerpage
\section{A contribution to Amazonian ethnology}

Each chapter of this book presents a single narrative, an ever-present and much appreciated genre among almost all Amazonian peoples. Each embodies a unique rendition offered by a specific narrator, in circumstances and settings that vary widely: some were offered in a village, town, or intimate home setting in response to a specific request, one was recorded during a community language workshop (Kotiria), others in the course of everyday activities or within the context of a ritual.\footnote{Links to the audio or video renditions are provided in each chapter.} As we contemplate these diverse settings, we are reminded that the act of narration is never monologic: there is always an audience, there are always interlocutors and “what-sayers". Narration is itself both a communicative and formative act. It not only transmits collective or individual memories, weaving the continuity of a people, clan, sib, or family, but also establishes the limits of social and antisocial behavior (and their consequences), revealing transformations, original and potential, creative or destructive.

At the same time, we can extract from these narratives mythical structures comparable to others in and beyond the Americas, following the paths of Levi Strauss’s \textit{esprit humain}. Through narratives, thought is molded, instruction and knowledge are transmitted and refined.  The Ka’apor and Kuikuro narratives, for instance, exemplify diffused bits and pieces of pan-Amazonian mythology, crossing frontiers between genres, peoples, and regions. Scatological and obscene, the Ka’apor narrative finds parallels in the oral traditions of many Amazonian groups. The Kuikuro narrative is not only an element of the Upper-Xinguan network, in which peoples of distinct origins and languages share rituals, myths, discourses and each other, but is also a unique female rendition of a narrative heard before only in masculine voices. Feminine voices resound in the Trumai, Hup, Kwaza, and Kotiria narratives as well.
    
A classic theme in Amazonian mythology, the origins of crucial cultural items – such as songs, rituals, and cultivated plants — are often viewed as gifts or as bounty seized in encounters involving confrontation or alliance between enemies or occupants of “other" worlds. In the Sakurabiat narrative, for example, the origin of corn involves knowledge captured by great shamans from neighboring groups. 
    
The Kalapalo and Trumai live in the same Upper Xingu regional multilingual cultural system, occupying distinct niches due to different degrees of adaptation and incorporation into the system. A comparison of the Kalapalo and Trumai narratives is particularly interesting because both describe funerary rituals and practices, recounting the origins of the Trumai chanted lamentations and some of the Kalapalo songs performed during the Xinguan mortuary ritual. A Kalapalo man married to a Snake-Woman acquires the songs from his father-in-law; the Trumai people receive their chanted lamentations from the Smooth-billed Ani, a bird. Likewise, the origins of places, such as the Kotiria sacred cemeteries, and elements of the natural environment, such as the Deer’s Tomb Constellation of the Hup narrative, lie in similar transformational fluidity and transposition of boundaries between this and other worlds.
    
Metamorphosis is a pervasive and relevant theme in Amerindian shamanic thought and contemporary Lowland South American ethnology.  It evidences communication and change of perspectives between humans and non-humans, between the living and the dead, between blood relatives and affines, us and “others", a challenge to the irreducible and naturalized distinctions in Western thought. Translation, understood in its most ample sense, is a necessary but not mechanical mediation, since translation itself moves, modifies, and creates. In “The death-path teachings”, two Marubo spirit-shamans, able to cross the world of spirits and dead people, connect exoteric knowledge with instructive speech. Likewise, a Kuikuro woman travels, still alive, to the upside-down world of the dead and there converses with them and hears their “twisted" words. 
    
Narrative events occur in what is for us a remote “past" or mythological \textit{illo tempore}, or better yet, as one Kuikuro chief puts it, a time “when we were all hyper-beings” speaking the same “language" or making ourselves understood through languages. It was or still is a time, a dimension out of time, or between times, peopled by ancestors and “monstrous" beings, such as the clumsy people-eater \textit{Khátpy} of the Kĩsêdjê narrative. Indeed, the terms “myth" or “mythological narratives", and “history" or “historical narratives" are frequently used to define or at least suggest what might be considered narrative sub-genres. However, as the Kotiria narrative shows, this is a more-than-fluid frontier where the supposedly self-evident opposition between regimes of memory crumbles.
    
This fluidity is nowhere clearer than in comparative analysis of evidentials and/or epistemic markers used in narratives, markers that take more into account than the mere qualification of source of information. Such elements may be manipulated by the narrator, sensitive to the occasion and audience, to mark voices of authority. Evidentials or epistemic markers — crucial and often obligatory — first of all define the epistemological status of narrative speech, as we see in the use of the Ka’apor reportative, but above all, reveal ambiguities and porous boundaries. Is the Kuikuro narrative a “myth" about the inverted life of the dead or a “memory" of a live woman’s journey to another world and return to narrate what she saw to fellow  members of the living world? The narrator tempers her own assertions with markers typical of “historical" facts transmitted through collective memory and with the non-certainty of events not directly and visually witnessed, marking that is impossible in “mythical" narratives, which speak of origins, indistinctions among species, and transformations. The Suruí narrative vividly evokes episodes from a not-too-distant past — though still prior to times known by adults today — replete with battles between neighboring peoples, yet in this narrative we observe the “deletion of non-witnessed evidentiality" characteristic of “myths".

\section{Narrative verbal artistry}

To narrate is not just to verbally express an account in prosaic form. As we have noted, the act of narration is a performance, whether public or private, offered to interlocutors and audiences and open for evaluation, criticism, and praise. The narrator is often a “master" in the art of oration, a specialist of “good and beautiful speech", recognized as such and fully aware of his or her role in the chain of transmission of abilities and content. The master’s artistic skills include manipulation of distinct protagonists’ perspectives, balancing of repetitions with nuanced variation, control of the necessary detours from the advancing storyline, full command of all the varied means of capturing and holding the listeners’ attention. 
Such mastery is evident in the Marubo narrative genre \textit{yoã vana}, distinct from the sung narrative genre \textit{saiti vana}, but both highly poetic performances. Cesarino’s division of lines in the written text attempts to reproduce, if  only partially, the dramatic effect produced by the rhythm of the oral performance and by thoughts-utterances whose understanding requires careful exegesis.
    
Similarly, the “masters" of the Kuikuro and Kalapalo narratives share like abilities and the narratives themselves reveal similar structures: formulaic openings and closings, scenes, blocks, parallelisms; movement verbs and logophoric connectives mark sequences and the development of events and actions. In the Hup and Kotiria narratives, skilled use of tail-head linking strategies guarantee sequential cohesion. Even more impressive is the Kwaza narrator's domination of anticipatory switch-reference marking as she constructs the narrative, in van der Voort’s words, as “one long sentence, each chained clause being either in a subordinate mood or in a cosubordinate mood.” 
    
    The rarity, or near absence, of indirect reported speech in Amazonian narratives draws our attention to the preponderance of direct reported speech, observed throughout the volume. Our narrators are masters in performance of such speech, leading us to wonder about other possibilities of embedding and recursive structures. In fact, we are dealing not only with cited dialogues, but also the expression of inner thoughts, which take the form of images, perceptions, emotions, plans. 
For instance, almost half of the Kuikuro and Kalapalo narratives is animated by dialogues between the characters, with a predominance of verbal forms inflected by performative modes (imperative, hortative, imminent future), as well as epistemic markers that modulate the attitudes and communicative intentions of the interacting characters. Cesarino mentions “the extensive use of reported speech, which allows the (Marubo) narrator to shift between voices.” Last but not least, we highlight the “embedded quotations of successive narrators of the events” in the Surui narrative, as Yvinec observes. 

    These are but a few of the many and varied narrative discourse structures resources present in this volume, calling our attention to the richness and diversity of narrative verbal artistry in Amazonia. 

\section{A host of typological gems}
\largerpage[2]
This volume not only introduces us to a rich panorama of narrative styles and cultural themes, it also demonstrates the astounding genetic and structural diversity of Amazonian languages. Although not all recent research on Amazonian languages has been fully explored and incorporated into typological databases,\footnote{ Such as such as the \textit{World Atlas of Linguistic Structures} (WALS) http://wals.info/ and its more recently organized counterpart, (SAILS)  \textit{South American Indigenous Language Structures} \url{http://sails.clld.org/}. }
the picture that is emerging is one of much greater structural diversity within the Amazonian basin than was previously supposed. Indeed, the impetus to define a set of recognizably distinct “Lowland Amazonian" linguistic features \citep{Payne1990,Dixon1999,Aikhenvald2012} wanes in light of empirical evidence underscoring vast regional diversity \citep{van2000,Campbell2012,EppsSal2013}. Additionally, analyses such as \citegen{Birchall2014} work on argument coding patterns in South American languages suggest that broader Western/Eastern South American perspectives may actually be more significant to understanding patterns of structural similarity and difference than earlier assumptions of an Andean/Lowland Amazonian dichotomy (see also \citealt{O’Connor2014}).\footnote{Other chapters in the same volume focus on specific typological features, including OV order, nominalization as a subordination strategy, post-verbal negation, and use of desiderative morphemes, that appear to characterize South American languages as a whole. }
  
This debate is far from concluded, and as research continues to pour in, it is certain to bring new insights into deep genetic relationships, pre-historical movements and patterns of contact, as well as contemporary areal phenomena, all of which serving to refine our typological profiles. For the moment, suffice it to say that even the small selection of languages in our volume clearly shows that there is no easy answer to the question: “What does an \textit{Amazonian language} look like?”  
 
 
\begin{figure}[t]
\includegraphics[width=\textwidth]{figures/amazon.pdf}
\caption{Peoples and languages represented in this volume}
\end{figure}

The twelve languages in this volume come from a variety of geographic locations within Amazonia, and include three linguistic isolates and members of the Carib, East Tukano, Nadahup, Jê, Tupi, and Pano families — only a fraction of the more than four dozen distinct genealogical units that compose the Amazonian linguistic landscape \citep[1]{EppsSal2013}. Three regions characterized by longstanding and systemic cultural and linguistic interaction are also represented by different subsets of these languages. Kotiria and Hup are spoken in the Upper Rio Negro region of northwestern Amazonia in the Brazil-Colombia borderlands (see \citealt{Aikhenvald2002}, \citealt[73--84]{Aikhenvald2012}, \citealt{EppsStenz2013}), and the Guaporé-Mamoré region of Southern Rondônia and northeastern Bolivia is represented by Kwaza, Aikanã, and Sakurabiat \citep{Crevels2008}. Indeed, the chapters by Epps and van der Voort in this volume discuss features that support characterization of these two regions as “linguistic areas" in which contact and multilingual practices have led to structural similarities among genetically unrelated languages. The third multilingual system, represented by Kuikuro, Kalapalo, and Trumai, is the Upper Xingu in central Brazil \citep{Franchetto2011}. The chapters by Franchetto, Guerreiro, and Guirardello-Damian, point out that, in contrast to the Upper Negro and Guaporé-Mamoré regions, in the Upper Xingu context, multilingualism emerges and is evidenced primarily as a component of Xinguan ritual arts. 
    
Kuikuro and Kalapalo are actually variants of a single language, baptized by Franchetto as the “Upper Xingu Carib Language”. Though viewed as \textit{dialects} for the linguist, they are \textit{languages} for their speakers for two substantive reasons. First, because within the Upper Xingu multilingual regional system, they are diacritics of local political identities. Secondly, because attributing the status of “language" to both establishes their equal value, counterbalancing the tendency for indigenous languages labeled as “dialects" to be viewed as having an inferior or marginal existence. We have strategically opted to present the Kuikuro and Kalapalo narratives in sequence so that the reader can appreciate the obvious similarities between the syntax of the two languages as well as the  differences — sometimes quite subtle — in morphology and lexicon.  Unfortunately, the written medium masks a crucial dimension of dialectal difference occurring on the prosodic level, where Kuikuro and Kalapalo clearly exemplify the notion of words “dancing to the beats of different drummers". Equally strategic is the sequencing of the Kwaza and  Aikanã narratives, versions of the same story offered by speakers of two language isolates in the same multilingual region. 
    
A broad overview of the twelve languages reveals a handful of common structural features, including agglutinative and preferentially suffixing morphology, as well as predominantly head-final constituent order (the exception being the relatively free word order of Kwaza). However, a closer look shows interesting variations in clausal constituent ordering, including object-initial order, which first came to light in languages of the Carib family\footnote{Several Carib languages are analyzed as having OVS as the dominant order, and OVS is also found in some East Tukano, Tupi, Arawak languages (see \citealt[155]{Derbyshire1999}; \citealt[273--275]{Campbell2012}).} and which can be seen in numerous lines of the Kuikuro and Kalapalo narratives, such as \REF{ex:intro:1}: 

\ea\label{ex:intro:1} tüti ilü leha iheke \\[.3em]
\gll  tüti i-lü leha \textbf{i-heke} \\
\textsc{refl.}mother fight-\textsc{pnct} \textsc{compl} \textsc{\textbf{3-erg}} \\
\glt ‘He fought with his own mother’ [\textsc{kuikuro}, line 243]
\z 

\newpage 
As a frequently occurring alternate order, OVS is found in many other Amazonian languages, including Kotiria, where known, non-focused subjects are sentence-final, as we see in \REF{ex:intro:2}.
\ea\label{ex:intro:2} “hiphiti a’ri phinitare naita yʉ’ʉ" nia.  \\[.3em]
\gll híphiti	a’rí	{\textasciitilde}phídi-ta-re	{\textasciitilde}dá-i-ta 	\textbf{yʉ’ʉ́}	{\textasciitilde}dí-a \\
     everything	\textsc{dem.prox}	right.here-\textsc{emph-obj}	get-\textsc{m-intent}	\textbf{1\textsc{sg}}	say-\textsc{assert.pfv}\\
\glt ‘“All of these things here I'm taking away,” (\textit{Dianumia Yairo}) said.’ [\textsc{kotiria}, line 242]
\z 
Another striking feature observed throughout the volume is the rampant use of derivational processes to create new lexical concepts, counterbalance parsimonious lexical class distinctions, and define contexts of complementation and subordination \citep{vanGijn2011,Bruno2011}. Some interesting examples of verbalizations are the derived forms for ‘teaching’ in Kalapalo \REF{ex:intro:3}, ‘body painting (with genipapo)’ in Kuikuro \REF{ex:intro:4}, and ‘marrying’ in Kotiria \REF{ex:intro:5}.
    
\ea\label{ex:intro:3}  akihata iheke\\[.3em]
\gll \textbf{aki-ha}-ta i-heke\\
     \textbf{word-\textsc{vblz}}-\textsc{dur} 3-\textsc{erg}\\
\glt ‘He was teaching.’ [\textsc{kalapalo}, line 78] 
\z 

\ea\label{ex:intro:4} engü isangatelü leha\\[.3em]
\gll engü	is-\textbf{anga-te}-lü		leha\\
then 	 3-\textbf{jenipa-\textsc{vblz}}-\textsc{pnct} 	\textsc{compl} \\
\glt ‘Then she was painted with genipapo’ [\textsc{kuikuiro}, line 10]\\
\z
 
\ea\label{ex:intro:5} phʉaro numia, phʉaro numia ti phapʉre namotia tire himarebʉ, tiaro numiapʉ bʉhkʉthurupʉre. \\[.3em]
\gll phʉá-ro	{\textasciitilde}dúbí-á	phʉá-ro	{\textasciitilde}dúbí-á	ti=phá-pʉ-re	\textbf{{\textasciitilde}dabó-tí}-á  tí-re	hí-{\textasciitilde}bare-bʉ	tiá-ro	{\textasciitilde}dúbí-á-pʉ́	bʉkʉ́-thúrú-pʉ́-ré\\
     two\textsc{-sg}	woman\textsc{-pl}	two\textsc{-sg}	woman-\textsc{pl}	\textsc{anph}=time-\textsc{loc-obj}  	\textbf{wife-\textsc{vbz}}-\textsc{assert.pfv} \textsc{anph-obj}  	\textsc{cop-rem.ipfv-epis}	three\textsc{-sg}	woman-\textsc{pl-loc}	ancestor-times-\textsc{loc-obj}  \\
\glt ‘In those olden times, the custom was to marry two wives, two or even three.’ [\textsc{kotiria}, line 23] 
\z

A far vaster set of morphemes are employed in nominalizations, a small sample being the Sakurabiat ‘hammock’ in \REF{ex:intro:6}, the Kwaza ‘olden times’ in \REF{ex:intro:7}, and in \REF{ex:intro:8}, the Kĩsêdjê autodenomination. 

\ea\label{ex:intro:6} Pɨbot nẽãrã setoabõ\\[.3em]
\gll pɨbot neara se-\textbf{top-ap}=õ\\
     arrive again \textsc{3cor}-\textbf{lying.down-\textsc{nmlz}}=\textsc{dat}\\
\glt ‘He arrived again at his own hammock.’ [\textsc{sakurabiat}, line 15]
%\glt ‘E chegou na sua rede (na casa dele) novamente.' 	
\z

\ea\label{ex:intro:7} a'ayawɨ cwata unɨ̃tetawata txarwa hakahɨ̃ awɨ\\[.3em]
\gll a\textasciitilde a-ya-wɨ cwa-ta unɨ̃teta-wa-ta txarwa
\textbf{haka-hɨ̃ } \textbf{a-wɨ}\\
exist\textasciitilde exist-\textsc{iobj}-time \textsc{isbj-cso} converse-\textsc{isbj-cso} first \textbf{old-\textsc{nmlz}} \textbf{exist-time}\\
\glt ‘Speaking today about our olden times,’ [\textsc{kwaza}, line 55]
\z

\ea\label{ex:intro:8} Kĩsêdjê \\[.3em]
\gll kĩ      sêt-∅        jê    \\
     village burn-\Nmlz{} \Pl{} \\
\glt ‘The ones who burn villages’ [\textsc{kĩsêdjê}, line 2] \\
\z  
% \todo{in 8, there is a language name instead of the orthographic line}
In Kuikuro and Kalapalo, there are locative, agent, non-agent, and instrument nominalizers, the latter used with the root \emph{hü} (Kuikuro) / \emph{hüti} (Kalapalo) ‘to feel shy/respect/shame’, in the derivation of terms for one’s parents-in-law \REF{ex:intro:9}. 
\ea\label{ex:intro:9}  ihütisoho kilü\\[.3em]
\gll i-\textbf{hüti-soho} ki-lü\\
     3-\textbf{shame-\textsc{ins}} say-\textsc{pnct}\\
\glt ‘His father-in-law said.’ [\textsc{kalapalo}, line 130]
\z 

Aikanã has a nominalizer for actions \REF{ex:intro:10},  Kotiria one for reference to events/lo\-ca\-tions \REF{ex:intro:11}, and Sakurabiat one exclusively used for syntactic objects, seen in \REF{ex:intro:12}. 

\ea\label{ex:intro:10}   üre'apa'ine xarükanapɨire'ẽ kukaẽ \\[.3em]
\gll \textbf{üre-apa'i}-ne xa-rüka-napa-ire-'ẽ kuka-ẽ\\
    \textbf{hide-\textsc{act.nmlz-loc}} \textsc{1pl-dir:}around-\textsc{clf:}forest-almost\textsc{-imp} tell-\textsc{decl}\\
\glt  `{``}We will sneak around them,'' said Fox.'  [\textsc{aikanã}, line 25]\\
\z

\ea\label{ex:intro:11} do'poto	to hiro	hia. \\[.3em]
\gll      \textbf{do'pó-to}	to=hí-ro	hí-a \\
          \textbf{origin/roots-\textsc{nmlz.loc/evnt}}	3\textsc{sg.poss=cop-sg} 	\textsc{cop-assert.pfv} \\
\glt ‘It's his (\textit{Ñahori}’s) origin site.’ [\textsc{kotiria}, line 36]
\z
 
\ea\label{ex:intro:12} Kʷai mariko kɨpkɨba 'a mariko sete\\[.3em]
\gll kʷai mat \textbf{i-ko} kɨpkɨba 'a mat i-ko sete\\
     stone ? \textbf{\textsc{obj.nmlz}-ingest} tree fruit ? \textsc{obj.nmlz}-ingest \textsc{3sg}\\
\glt ‘He only eats stone and fruit (as if he were not human).’ (Lit. ‘Stone is what he eats, and fruit is what he eats.’) [\textsc{sakurabiat}, line 55]
\z

Valence-increasing operators include the productively used transitivizing auxiliary of Marubo, shown in \REF{ex:intro:13}.
\ea\label{ex:intro:13} vanavanakwãi avai kayakãisho \\[.3em]
\gll vana-vana-kawã-i             \textbf{a}-vai                 kaya-kãi-sho            \\
     speak-speak-go-\textsc{prog} \textbf{\textsc{aux.trns}-con} leave-\textsc{inc-sssa} \\
\glt `Calling and calling she left' [\textsc{marubo}, line 16]
\z

Marubo also has morphological causatives, as do Suruí, \textit{-ma} in ‘torching the house’ in \REF{ex:intro:14}, Ka’apor, \textit{-mu} in ‘opening one’s anus to fart’ in \REF{ex:intro:15}, and Kuikuro, \textit{-nhe} in ‘moving the woman up’ in \REF{ex:intro:16}.

\ea\label{ex:intro:14} ““Eebo oyena \~{G}oxorsabapa yã” iyã” de.\\[.3em]
\gll ee-bo o-ya-ee-na \~{G}oxor-sab-\textbf{ma}-apa a i-ya \(\varnothing\)-de\\
\textsc{endo-advers} \textsc{1sg-nwit-endo-foc} Zoró-house-\textbf{\textsc{caus}}-burn \textsc{sfm.nwit} \textsc{3sg-nwit} \textsc{3sg-wit}\\
\glt ‘““Thus I burnt down the Zoró's house.””’{\footnotemark}  [\textsc{suruí}, line 42]\\
\z
\footnotetext{Multiple sets of quotation marks in the Suruí narrative indicate layers of embedding in quoted speech, as Yvinec discusses  in \fnref{fn:surui:multquot} of chapter 12.}

\ea\label{ex:intro:15} xape ai jumupirar te’e xoty je \\[.3em]
\gll i-ʃapɛ ai ju-\textbf{mu}-piɾaɾ tɛʔɛ i-ʃɔtɪ jɛ \\
3-anus bad \textsc{refl}-\textbf{\textsc{caus}}-open free 3-towards \textsc{hsy} \\
\glt ‘Her disgusting asshole opened towards the boy.’ [\textsc{ka’apor}, line 21]\\
\z 

\ea\label{ex:intro:16} itükanhenügü letüha iheke itükanhenügü itükanhenügü \\[.3em]
\gll itüka\textbf{-nhe}-nügü	üle=tü=ha	i-heke itüka-\textbf{nhe}-nügü itüka-\textbf{nhe}-nügü \\
3.move.up-\textsc{\textbf{tr}-pnct} 	\textsc{log=uncr}=\textsc{ha}	3-\textsc{erg} 3.move.up-\textsc{\textbf{tr}-pnct} 3.move.up\textsc{\textbf{-tr}-pnct} \\
\glt ‘Then, she moved her up, she moved her up, she moved her up’ [\textsc{kuikuro}, line 98]\\
\z

Hup, on the other hand, has causative constructions formed with serialized roots, such as \textit{k'ët-} ‘stand', used repeatedly in \REF{ex:intro:17} to indicate indirect or “sociative" causation.

\ea\label{ex:intro:17}  Yúp mah, yɨno yö́’ mah yúp, yúp hõ̀p tɨh k’ët wédéh, hõ̀p tɨh k’ët wèd, mòh tɨh k’ët wèd, nííy mah.\\[.3em] [.6em]
\gll yúp=mah, yɨ-no-yö́ʔ=mah yúp, yúp hõ̀p tɨh \textbf{k’ët}-wéd-éh hõ̀p tɨh \textbf{k’ët}-wèd, mòh tɨh \textbf{k’ët}-wèd, ní-íy=mah.\\
     \textsc{dem.itg=rep} \textsc{dem.itg-}say\textsc{-seq=rep} \textsc{dem.itg} \textsc{dem.itg} fish \textsc{3sg} \textbf{stand}-eat\textsc{-decl} fish \textsc{3sg} \textbf{stand}-eat tinamou \textsc{3sg} \textbf{stand}-eat be\textsc{-dynm=rep}\\
\glt ‘Having said that, it’s said, he gave her fish to eat; he went on giving her fish to eat, to give her tinamous to eat, it’s said.'  [\textsc{hup}, line 27, see also \fnref{fn:hup:caus} in chapter 7.]
\z

Valence-decreasing derivational processes include morphological intransi\-tiv\-iz\-ers in Sakurabiat \REF{ex:intro:18}, and Kuikuro \REF{ex:intro:19}, while \REF{ex:intro:20} gives an example of the productive noun incorporation found in Trumai. 

\ea\label{ex:intro:18} Kɨrɨt sĩit jãj etsɨgɨka\\[.3em]
\gll kɨrɨt sĩit jãj \textbf{e}-sɨgɨ-ka\\
     child \textsc{dim} tooth \textbf{\textsc{intrvz}}-drop-\textsc{vblz}\\
\glt ‘(That's why) kids' teeth drop out.’ [\textsc{sakurabiat}, line 44]
%\glt ‘(Por isso que agora) dente de criança cai tudo.' 
\z

\ea\label{ex:intro:19} luale utimükeĩtai \\[.3em]
\gll luale \textbf{ut}-imükeiN-tai \\
sorry \textbf{\textsc{1.dtr}}-turn.face-\textsc{fut.im} \\
\glt ‘“Sorry! I will turn my face back”’ [\textsc{kuikuro}, line 224]\\
%‘“Desculpe! Eu vou virar meu rosto para trás”’  
\z

\ea\label{ex:intro:20} ina hen esak ji hen mal husa husa ke ine jik, det'a hen jaw at̪u tsula nawan de.\\[.3em]
\gll ina           hen     esak  ji   hen\\     
\textsc{disc.con}  then    hammock  \textsc{prag.in}  then\\     

\hfill

\gll \textbf{mal husa husa} ke ine ji=k\\
\textbf{edge   tie       tie}       \textsc{disloc.abs} \textsc{3anaph.masc}  \textsc{prag.in}=\textsc{erg}\\

\hfill

\gll det'a   hen    jaw  at̪u  tsula  nawan  de\\
well  then  human.being  dead  be.lying   similar  already\\
\glt ‘Then he tied the hammock, it became very similar to a dead person lying.' [\textsc{trumai}, line 10, see also note 13 in chapter 5].
\z

    The languages in our collection also vary significantly in the extent to which they employ bound morphology. \REF{ex:intro:18} and \REF{ex:intro:20} above clearly show the more analytical profiles of Sakurabiat and Trumai, contrasting with the distinctly synthetic morphology of languages such as Kwaza \REF{ex:intro:21} and Suruí \REF{ex:intro:22}. 

\ea\label{ex:intro:21} tsɨwɨdɨte xareredɨnãiko adɨ'ata\\[.3em]
\gll tsɨwɨdɨte	xarere-dɨnãi-ko	a-dɨ-a-ta\\
girl			crazy-manner-\textsc{ins}	exist-\textsc{caus-1pl.incl-cso}\\
\glt ‘We let girls act crazy like that, in our present life.’ [\textsc{kwaza}, line 59]
\z

\ea\label{ex:intro:22} Omamõperedene.\\[.3em]
\gll o-ma-amõ-pere-de-na-e\\
\textsc{1sg-poss}-grandfather-\textsc{iter-wit-foc-sfm.wit}\\
\glt ‘My grandfather did that again and again.’ [\textsc{suruí}, line 47]\\
%\glt ‘Meu avô sempre fazia isso.’ 
\z

Highly complex verbal morphology is especially striking in Aikanã \REF{ex:intro:23} and Kotiria, particularly in the latter's productive use of verb serialization to code aspectual, modal, and adverbial spatial/manner distinctions \REF{ex:intro:24}. Similar constructions with serialized roots are seen in Hup verbal words \REF{ex:intro:25}, one of the structural features likely diffused through centuries of language contact \citep{Epps2007}.

\ea\label{ex:intro:23}   yãw'ẽ wikere xü'iaxanapetaka'ĩwãte kukaẽ \\[.3em]
\gll yãw'ẽ wikere \textbf{xü'i-a-xa-nape-ta-ka-'ĩwã-te} kuka-ẽ\\
    let's.go.\textsc{imp} peanut \textbf{dig-uproot-\textsc{1pl-dir:}forest-\textsc{rem.fut}-\textsc{clf}:pieces-\textsc{admon-pst}} tell-\textsc{decl}\\
\glt    `{``}Let's go digging up peanuts as planned,'' he told her.' [\textsc{aikanã}, line 14]\\
%`{``}Vamos arrancar amendoim como combinamos,'' ele falou.' 
\z

 
\ea\label{ex:intro:24} pha'muri mahsa õre pha'muyohataa.  \\[.3em]
\gll {\textasciitilde}pha'bú-rí	{\textasciitilde}basá 	{\textasciitilde}ó-ré	\textbf{{\textasciitilde}pha'bú-yóhá-tá}-a \\
     originate-\textsc{nmlz}	people  	\textsc{deic.prox-obj}  	\textbf{originate-go.upriver-come}-\textsc{assert.pfv} \\
\glt ‘The origin beings appeared coming upriver here.’ [\textsc{kotiria}, line 15]
%\glt ‘Os seres de transformação apareceram subindo pelo rio aqui.’ 
\z

  
\ea\label{ex:intro:25}  Yɨno yö́’ mah yúp, tɨ́hàn hɨd dö’ híayáh.\\[.3em]
\gll yɨ-no-yö́ʔ=mah yúp, tɨ́h-àn hɨd \textbf{döʔ-hí}-ay-áh\\
     \textsc{dem.itg-}say\textsc{-seq=rep} \textsc{dem.itg} \textsc{3sg-obj} \textsc{3pl} \textbf{take-descend}\textsc{-inch-decl}\\
\glt ‘Saying thus, it’s said, they took (the baby deer) down.' [\textsc{hup}, line 99]
%\glt ‘Falando assim, dizem, eles baixarem (o nene).' 
\z

Indeed, the narratives of this volume attest the rich means use to code movement and spatial relations in Amazonian languages (see also \citealt{Bozzi2013}). A few languages employ locative postpositions or case markers with spatially specific semantics: inessive and allative markers in Kuikuro, Kalapalo and Aikanã, ablatives in Suruí and Sakurabiat, and the “provenence locative" marker in Marubo \REF{ex:intro:26}. 

\ea\label{ex:intro:26} Vei Maya vei mai nãkõsh wenímarivi, shavo wetsa. \\[.3em]
\gll Vei   Maya vei   mai nãkõ- \textbf{sh}                  wení-ma-rivi          shavo wetsa \\
     death Maya death land nectar- \textbf{\textsc{loc.prov}} rise-\textsc{neg-emp} woman other \\
\glt `Vei Maya did not come from the Death-Land nectar; she is another woman.' [\textsc{marubo}, line 4]
\z

Many more integrate detailed spatial or movement information in verbal morphology, through root serialization showing associated motion or direction (as seen in the Kotiria and Hup examples \REF{ex:intro:24}--\REF{ex:intro:25} above), or with bound directional/locational morphemes indicating notions such as ‘outside’, ‘hither’, ‘close’, etc., in Aikanã and Kwaza \REF{ex:intro:27}. 
\ea\label{ex:intro:27} watxile karɛ͂xu katsutyata xareyawata axehɨ̃ko tsadwɛnɛ\\[.3em]
\gll watxile	karɛ͂xu				katsu-tya	ta		xareya-wa-ta		axe-hɨ̃-tya	tsadwɛ-\textbf{nɛ}\\
finally		dry.heartwood	cross-\textsc{cso}	\textsc{cso}	search-\textsc{isbj-cso}		find-\textsc{nmlz-cso}	onto.path-\textbf{\textsc{dir}:hither}\\
\glt ‘Later, crossing the dry log, they then searched and then got back onto the path.’ [\textsc{kwaza}, line 32]
\z

Directional auxiliaries, such as ‘go uphill’ in Trumai \REF{ex:intro:28}, are also commonly found. Similar directional verbs occur in Kotiria, e.g. ‘go upriver’ (in lines 225 and 253 of chapter 6),  and Hup ‘go upstream’ (in lines 4 and 12 of chapter 7, among others).
\ea\label{ex:intro:28} kaʔʃɨ  t̪'axer lahmin.\\[.3em]
\gll kaʔʃɨ    t̪'axer   \textbf{lahmi}=n\\
walk    poorly  \textbf{go.uphill}=\textsc{3abs}\\
\glt ‘She left.' [\textsc{trumai} line 3, see also \fnref{fn:5:uphill} in chapter 5]
\z

Sakurabiat uses verbal auxiliaries for associated movement and to indicate the body position of subjects \REF{ex:intro:29}, and has positional demonstratives that code the body position of other referents \REF{ex:intro:30}.
\ea\label{ex:intro:29} Pɨ ke itoa enĩĩtse\\[.3em]
\gll pɨ ke i-\textbf{to}-a eni=ese\\
     lying \textsc{dem} \textsc{3sg-\textbf{aux.lie}-thv} hammock=\textsc{loc}\\
\glt ‘He (\textit{Arɨkʷajõ}) was there just lying in the hammock.’ [\textsc{sakurabiat}, line 3]
%\glt ‘E ele (\textit{Arɨkʷajõ}) estava lá deitado na rede.'
\z 

\ea\label{ex:intro:30} Tamõ'ẽm porẽtsopega petsetagiat:\\[.3em]
\gll \textbf{ta}=bõ='ẽp porẽsopeg-a pe=se-tak-iat\\
     \textbf{\textsc{dem.stand}}=\textsc{dat}=\textsc{emph} ask-\textsc{thv} \textsc{obl}=\textsc{3cor}-daughter-\textsc{col}\\
\glt ‘He just got there and asked to his daughters:’ [\textsc{sakurabiat}, line 16]
%\glt ‘Entrou, foi direto perguntar pra filharada dele.' 
\z

Kuikuro, on the other hand, makes an interesting centripetal/centrifugal distinction in its imperative suffixes, the latter seen in \REF{ex:intro:31}. 
\ea\label{ex:intro:31} ouünko tuhipe kunhigake ika kigeke \\[.3em]
\gll o-uüN-ko	tuhi-pe		     ku-ng-ingi\textbf{-gake}	ika	kigeke	\\
2-father-\textsc{pl} 	garden-\textsc{ntm} 	     1.2-\textsc{obj}-see-\textbf{\textsc{imp.ctf}} 		wood 	let’s.go\\
\glt ‘“Let’s go see your father’s old garden, let’s go to cut wood!”’ [\textsc{kuikuro}, line 15]\\
%‘“Vamos lá ver a roça que era do pai de vocês, vamos catar lenha!”’ 
\z 

 Nearly half of the languages in the collection have switch-reference systems, with notable variation in terms of the contexts in which markers occur and the additional grammatical categories they may express. In Kotiria, overt switch-reference marking occurs only in contexts of clause subordination. In contrast, clause coordination is the relevant context in Kĩsêdjê, which has some ‘different subject’ forms that further indicate anticipatory subject agreement, and if third person, tense distinctions as well. Switch-reference markers in Kwaza and Aikanã can signal a new foregrounded topic or important turn of events in discourse. As is the case with most languages in the Pano family, Marubo has a complex system in which switch-reference markers code distinctions of same and different subjects as well as simultaneous or sequential actions (see \citealt{vanGijn2016}).\footnote{Other chapters in the same volume offer case studies of switch-reference systems in diverse Amazonian languages.} 
     
An even larger set of languages have grammaticalized markers with evidential and/or epistemic semantics. The complex systems of obligatory evidential marking in Hup and Kotiria have four or five categories that contrast hearsay/reported information with different subtypes of direct sensory (visual, non-visual) and indirect (inferred, presumed) evidential sources. Other languages, such as Suruí, have a basic witnessed/non-witnessed distinction, but can employ evidential markers pragmatically in discourse to prioritize focus on particular events over identification of source of evidence. Suruí evidential markers can also occur recursively with an utterance containing embedded quoted speech, as we see in \REF{ex:intro:32}, as can the Sakurabiat evidential \textit{eba} (e.g. line 20 of chapter 8). 
\ea\label{ex:intro:32} ““Nem, olobaka \~{G}oxoriyã” iyã” de.\\[.3em]
\gll nem o-sob-aka \~{G}oxor-\textbf{ya} i-\textbf{ya} \(\varnothing\)-\textbf{de}\\
\textsc{intj} \textsc{1sg}-father-kill Zoró-\textbf{\textsc{nwit}} \textsc{3sg-\textbf{nwit}} \textsc{3sg-\textbf{wit}}\\
\glt ‘““Well, a Zoró killed my father.””’ [\textsc{suruí}, line 3]\\
%\glt ‘““Um Zoró matou meu pai.””’ 
\z 
Trumai and Ka’apor typically make use of hearsay evidentials to indicate narratives as having a non-firsthand source of information, while Kuikuro and Kalapalo have a large number of optional evidential and epistemic markers that occur primarily in the quoted speech of narrative protagonists, indicating their attitudes and intentions in interaction. 

Ergativity is a well-known feature of Amazonian languages (see \citealt{Gildea2010}) and occurs in some form in nearly half of the languages in this volume. Fully ergative systems are seen in Kuikuro and Kalapalo, in which the morpheme (-)\textit{heke} always marks the ergative argument, as in \REF{ex:intro:1}, \REF{ex:intro:3} and \REF{ex:intro:16} above, and absolutive arguments are formally unmarked. Trumai makes use of ergative clitics and absolutive bound pronouns, while ergative marking in Marubo — in keeping with patterns found throughout Panoan languages — involves suprasegmental nasality, easily observed in pronominal forms such as \textit{e} ‘1\textsc{sg.abs}’ vs. \textit{ẽ} ‘1\textsc{sg.erg}’ and \textit{mato} ‘2\textsc{pl.abs}’ vs. \textit{mã} ‘2\textsc{pl.erg}’. Kĩsêdjê has a split system, with nominative-accusative alignment in main clauses and ergative-ab\-so\-lu\-tive alignment in embedded clauses, as we see in \REF{ex:intro:33}. 

\ea\label{ex:intro:33}  Kôt hry jatuj khãm khutha. \\[.3em]
\gll {\ob}  \textbf{kôt }          hry   j-atu-j         {\cb}  khãm \textbf{khu}-ta                    \\
     {} \textbf{\Third.\Erg}{} trail \E-stop-\Nmlz{} {} in   \textbf{\Third}-put.standing.\Sg{} \\
\glt `He put it down [where he had stopped making the trail].' [\textsc{kĩsêdjê }, line 51]\\
\z 
Finally, Galucio describes the mixed system of Sakurabiat as “nominative-ab\-so\-lu\-tive". Verbal prefixes index the subject of an intransitive verb or object of a transitive verbs (the absolutive argument), while transitive subjects (A) are obligatorily expressed as free pronouns. The same free pronominal forms can also be used as subjects in intransitive sentences, revealing nominative (S/A) alignment. 

    While on the topic of pronouns, we should note that eight of the twelve languages in this collection have an inclusive/exclusive distinction in their pronominal paradigms. In \REF{ex:intro:34}, we see that Trumai additionally marks a \textit{dual} inclusive/exclusive value. 
\ea\label{ex:intro:34} “huk'anik, huta.kaʃ ka a huʔtsa kawa."\\[.3em]
\gll huk'anik    huta.kaʃ   \textbf{ka}  \textbf{a}       huʔtsa  kawa\\
\textsc{expr}  later \textbf{\textsc{1incl}} \textbf{\textsc{du}} see go\\
\glt ‘“Wait, later we are going to see her (i.e., take care of her)."' [\textsc{trumai}, line 23]
\z

Turning our attention very briefly to the “sounds" one hears in Amazonian voices, an overview of the phonological systems of the languages in our volume reveals the frequency of a high central vowel [ ɨ ], which occurs in ten of the twelve languages as a phoneme or commonly used allophone. As for consonants, Kuikuro has a unique uvular flap and for an Amazonian language, Trumai has an whoppingly large 23-consonant inventory that includes a lateral fricative, as well as ejectives and plosives that make a distinction between alveolar and dental points of articulation. Nasality (a suprasegmental feature in Hup and Kotiria), nasal-harmony or spreading processes (in Kotiria and Sakurabiat), tone (in Kotiria, Hup, and Suruí), and glottalic sounds — full glottal stops, glottalized and aspirated consonants, and laryngealized vowels — are other prominent phonological features. In Kĩsêdjê, infixed aspiration of voiceless plosives has a syntactic function, marking third person agreement, as seen in \REF{ex:intro:35}. 
\ea\label{ex:intro:35}  Akwyn nen thẽn khatho. \\[.3em]
\gll akwyn ne=n             thẽ=n             k\textbf{<h>}atho               \\
     back  be.so=\AAnd.\Ss{} go.\Sg=\AAnd.\Ss{} \textbf{<\Third>}come.out.\Sg{} \\
\glt `He came back and came out (of the forest).' [\textsc{kĩsêdjê}, line 43]\\
\z 
Our tour of the fascinating structural features of Amazonian languages could go on an on, adding the noun classifiers of Kwaza and Kotiria, the “nominal tense" suffix of Kuikuro, the five-way past-tense distinction of Marubo, and the suppletive verbal forms of Kĩsêdjê — among others — to this initial collection of typological gems. However, we will stop here in the hope your curiosity has now been sufficiently sparked and you are ready to explore for yourself the delights, details, and discoveries our contributors have provided in the chapters that follow. 

{\sloppy
\printbibliography[heading=subbibliography,notkeyword=this]
} 
\end{document}
