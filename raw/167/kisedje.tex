\documentclass[output=paper,
modfonts,nonflat
]{langsci/langscibook} 
\author{Rafael Nonato\affiliation{UMass Amherst, USA}%
\and Kujusi Suyá%
\and Jamthô Suyá%
\lastand Kawiri Suyá%
}%
\title{Kĩsêdjê}
\lehead{R.\ Nonato, Kujusi Suyá, Jamthô Suyá \& Kawiri Suyá}
\ourchaptersubtitle{Khátpy re wapãmjê thõ thurun tho thẽm nda}
\ourchaptersubtitletrans{`The one (story) in which the Khátpy monster loads a forefather of ours onto his back and carries him away'}
% \abstract{noabstract}
\ChapterDOI{10.5281/zenodo.1008789}

\maketitle

\begin{document}

\section{Introduction} %(((

In this narrative, an ancestor of the Kĩsêdjê goes hunting and kills a monkey
up in a tree. He climbs up the tree to fetch it and when he comes down, the
monster \textit{Khátpy} is already waiting for him. \textit{Khátpy} hits him unconscious and puts
him in a basket to carry home and feed to his hungry children. Stopping mid-way to
open a trail in the forest, \textit{Khátpy} leaves the basket on the ground unattended.
When he finishes cutting part of the trail open, he comes back, fetches the
basket and carries it up to the end of the trail. He repeats this process a few
times until the Kĩsêdjê ancestor wakes up and realizes what is happening. The
ancestor quietly waits for the monster to leave the basket unattended again and
counts the time it takes for the monster to come back. \textit{Khátpy} comes back,
fetches the basket and carries it up to the end of the trail. He then leaves
the basked unattended once more to continue opening the trail. The ancestor
then comes out of the basket but, before running away, he fills it with rocks
so the monster won't realize he has escaped. Though \textit{Khátpy} feels that the
basket has become heavier, he still carries it home without checking its
contents. When he arrives home, his wives tell him there are only rocks and a
few monkeys in the basket, but none of the big prey \textit{Khátpy} claims he's killed.
Enraged, \textit{Khátpy} fetches his club and goes back into the forest after the
Kĩsêdjê ancestor. That is when the story ends, with the narrator excusing
himself for only knowing what happened up to that point. That is how his father
told him this story, how his people tell it. The title of the story, the
expression the Kĩsêdjê use to refer to it, is basically a summary of how it
begins, turned into a headless relative clause:

\ea Khátpy re wapãmjê thõ thurun tho thẽm nda \\[.3em]
\gll {\ob}  Khátpy=re wa-pãm-jê=thõ              thu-ru=n                       \\
     {} K.=\Erg{} \First\Incl-father-\Pl=one load.on.back-\Nmlz=\AAnd.\Ss{} \\
% \vtop{%
\gll t$\langle$h$\rangle$o        ∅-thẽ-m           {\cb}=nda   \\ 
     $\langle$\Third$\rangle$with \Third-go-\Nmlz{} {}=\Def \\ 
\glt `The one (story) in which the \textit{Khátpy} monster loads a forefather of ours onto his back and carries him away'%
% }%
\label{exe:title}
\z

%)))
\subsection{The circumstances of the narration} %(((

This story was narrated by Kuiussi Suyá, the chief of the Kĩsêdjê. He is
recognized in the community as a great storyteller and knower of their
traditions. He told it from his hammock, in his house at the \textit{Ngôjhwêrê} village,
on December 5\textsuperscript{th} 2009. It was recorded by Rafael Nonato as
part of PRODOCLIN--Kĩsêdjê,%
\footnote{\url{http://prodoclin.museudoindio.gov.br/index.php/etnias/kisedje}}
a documentation project for the Kĩsêdjê language sponsored by the Museu do
Índio.\footnote{The “Museum of Indigenous Peoples", located in Rio de Janeiro
is an organ of FUNAI, the Brazilian Bureau of Indigenous Affairs.} This narrative
was transcribed and translated by Jamthô Suyá, and was interlinearized by
Rafael Nonato with assistance from Jamthô Suyá and Kawiri Suyá. It had also
been previously adapted into a short film.\footnote{At the time of writing, the
short could be accessed at \url{https://www.youtube.com/watch?v=wmtwNxYCUvo}.}

%)))
\subsection{The Kĩsêdjê people} %(((

The Kĩsêdjê are roughly 450 people, most of whom live in the Wawi Indigenous
Land, in the State of Mato Grosso, Brazil. The largest Kĩsêdjê village, named
\textit{Ngôjhwêrê}, `the origin of the water', is located near the southern borders of
this land at $11^\circ51'53''$~S; $52^\circ54'02''$~W.\footnotemark{} The Wawi
Indigenous Land is situated in the southern fringes of the Amazon forest,
encompassing most of the basin of the Suyá river, a western tributary of the
Xingu river, itself a southeast tributary of the Amazon. 

\begin{figure}[t]
\includegraphics[height=.4\textheight]{figures/wawi.pdf}
  \caption{The Wawi Indigenous Land.}
\end{figure}

The Wawi Indigenous
Land is contiguous to and located to the east of the Xingu Indigenous Park,
where the Kĩsêdjê used to live until the recent official recognition of their
own land. They arrived in the region of the Xingu basin in the latter part of
the 19\textsuperscript{th} century and have since forged an intricate history
of alliances, wars, and exchanges of technology with the peoples that inhabited
the region prior to their arrival.%
\footnotetext{\url{https://goo.gl/maps/i7kyoGZAb6L2}}

The Kĩsêdjê used to be known by the exonym ``Suyá'', after the river whose
basin they inhabit. This name was given to the river by another people,
possibly the Trumai, and the Kĩsêdjê dislike the term, and ask the researchers
that work with them to avoid using it. Their autodenomination, ``Kĩsêdjê''
(\ref{exe:kisedje}), makes reference to the traditional technique they employ
to create village sites, namely, burning a patch of forest into a circular
clearing, on the rims of which they then proceed to build their houses.

  
\ea Kĩsêdjê \\[.3em]
\gll kĩ      sêt-∅        jê    \\
     village burn-\Nmlz{} \Pl{} \\
\glt ‘The ones who burn villages’ \\
\label{exe:kisedje}
\z

%)))

\newpage 
\subsection{The Kĩsêdjê language} %(((

The Kĩsêdjê speak a Northern Jê language (Jê family, Macro-Jê stock). Similar
to other Northern Jê languages, Kĩsêdjê is strictly head-final, with the
exception of a single head, to be mentioned below.

The main verb is always clause-final. In the neutral order, verbs are
immediately preceded by their direct argument, whether it be a noun phrase or a
verb phrase. Any postpositional phrase that is also argumental must come
immediately before the direct argument, and these are preceded by any adjunct
postpositional phrases and/or adverbs.\footnote{(\ref{exe:adjppdirobjver})
exemplifies the order \textit{adjunct PP + direct object + verb}.
(\ref{exe:adjppargppver}) exemplifies the order \textit{adjunct PP + argument
PP + verb}. The text doesn't contain any sentences that exemplify the order
\textit{adjunct PP + argument PP + direct object + verb}.} The subject comes
before all the constituents mentioned above.%
\footnote{(\ref{exe:subadjppdirargver}) exemplifies the order
\textit{subject + adjunct PP + direct argument}.} 

The only exception to the pervasive head-final character of the language are
the TAM particles. They are obligatory in main clauses and ungrammatical in
embedded clauses. Main clauses must bear a single TAM particle, either in
initial\footnote{(\ref{exe:wajpripos}) and (\ref{exe:kotpripos}) exemplify 
TAM particles in initial position.} or in second position. When they appear in
second position, the TAM particles must be preceded by a dislocated
constituent marked for topic or focus interpretation.%
\footnote{(\ref{exe:nasecposathaj}) and (\ref{exe:nasecpos}) exemplify 
TAM particles in second position.} 

\ea  Word-order in the clausal domain  \\
\glt (Foc/Top) [ Mood/Tense [ S (Adjuncts) (PP Args) [ (DO) V ] ] ] 
\label{sch:wordorderclause}
\z

\noindent
Note that when they are in initial position, the \textit{factual future} and the
\textit{factual non-future} TAM particles can be deleted. The contexts
for their deletion are in almost perfect complementary distribution, though, and
for that reason their meaning can usually be recovered. The factual future particle
can be deleted when it precedes a nominative participant pronoun, whereas the
factual non-future particle cannot be deleted precisely when it precedes a
nominative participant pronoun. Both can be deleted when the subject is phrasal,
though. In narrative style, deletion of the factual non-future particle is very
pervasive, as we can notice in the narrative to be presented below.

Consistent with Kĩsêdjê's head-finality, postpositions follow their arguments,
possessed nouns follow their possessors and nominal determiners follow the nouns
they modify.\footnote{(\ref{exe:nomdet}) exemplifies the order \textit{noun +
determiner}.} There are no nominal categories that express amount (numerals) or
quality (adjectives). The amount and quality of a noun are expressed
verbally,\footnote{(\ref{exe:ithawyti}) exemplifies the order \textit{noun +
determiner + amount-denoting verb}.} often through relative clauses, which in
Kĩsêdjê are internally headed.\footnote{(\ref{exe:relcla1}) and
(\ref{exe:relcla2}) exemplify internally headed relative clauses. The head of
the former is \textit{hry} ‘trail’ and the head of the latter is
\textit{khukwâj} ‘monkey’.}

\ea  Word-order in the sub-clausal domain \\ 
\glt {[ [ (Possessor) Noun ] (Det) ] (P)} 
\z 

\noindent Kĩsêdjê is a strictly dependent-marking language, with a single
phenomenon reminiscent of agreement: when a direct argument is dislocated
(either to the first position, for topic/focus purposes, or to a position
preceding the adjuncts it normally follows, for less clear discourse reasons),
a resumptive pronoun obligatory marks its base position.%
\footnote{(\ref{exe:ithawyti}) and (\ref{exe:nasecpos}) exemplify this
phenomenon.} A nominative-accusative frame is found in main clauses and an
ergative-absolutive frame in embedded clauses.\footnote{(\ref{exe:nommain})
exemplifies an intransitive main verb with nominative subject and
(\ref{exe:nommaintran}) exemplifies a transitive main verb with nominative
subject and accusative object. (\ref{exe:koreemb}) exemplifies a transitive
embedded verb with ergative subject and (\ref{exe:ergabsemb}) exemplifies
intransitive embedded verbs with absolutive subjects, as well as the use of
ergative pronouns to double the subject of intransitive embedded verbs.} Most
verbs show two distinct forms: a morphologically simpler one used in main
clauses and a derived (nominalized) one used in embedded clauses.%
\footnote{(\ref{exe:ergabsemb}) exemplifies many different nominalizing
suffixes.} Case on noun phrases is marked by phrasal enclitics, with distinct
ergative and nominative forms. Noun phrases in the absolutive and accusative
cases are unmarked. As for the pronouns, their ergative forms are free accented
words, their nominative forms are phonological clitics and their accusative and
absolutive forms are prefixes. Only 3\textsuperscript{rd} person pronouns have
distinct accusative and absolutive forms, and only in certain restricted
environments. \textcite{san97} and \textcite{non14} give more detailed
descriptions of the language.

%)))
\newpage 
\section{Khátpy re wapãmjê thõ thurun tho thẽm nda} %(((
\translatedtitle{`When the \textit{Khátpy} monster loaded our forefather onto his back and carried him away'}\\

\translatedtitle{\hspace*{-3.5mm}`Quando o monstro \textit{Khátpy} botou nosso ancestral nas costas e levou embora'}\footnote{Recordings of this story are available from \url{https://zenodo.org/record/997437}}

\ea  Ne nhy ne. \\[.3em]
\gll ne=nhy                                       ne    \\
     be.so\footnotemark{}=\AAnd.\Ds\footnotemark{} be.so \\
\glt `Then it was like this.' \\
     `Aí foi assim.' \\
\addtocounter{footnote}{-1}
\footnotetext{The verb \textit{ne} can be translated as `to do so' or as `to be so'. Each occurrence will be glossed in the most appropriate way.}
\addtocounter{footnote}{+1}
\footnotetext{The form of the Kĩsêdjê coordinating conjunction marks a number of distinctions. The most salient is the distinction, labeled “switch-reference marking" by \textcite{jac67}, between the coordination of clauses with the same subject (\Ss) and that of clauses with different subjects (\Ds). Moreover, in certain syntactic contexts, the \textsc{ds} conjunction has distinct forms indicating agreement with the subject of the next clause. If that subject is of the third person, the \textsc{ds} conjunction also marks tense.}
\z

\ea  Ajipãmjê thõ ra, ajipãmjê ra, \\[.3em]
\gll aj-i-pãm-jê=thõ=ra                              aj-i-pãm-jê=ra                 \\
     \Pl\footnotemark-\First-father-\Pl{}=one=\Nom{} \Pl-\First-father-\Pl{}=\Nom{} \\
\glt `A forefather of ours, our forefathers,' \\
     `Um dos nossos antepassados, nossos antepassados,' \\
\footnotetext{Though Kĩsêdjê makes a distinction between inclusive and exclusive first person plural, there is no specialized first person exclusive morphology. There is specialized morphology used to mark first person plural \textit{inclusive} (as in (\ref{exe:switch-to-inclusive})), while the simple pluralization of first person results in the exclusive interpretation. The latter is the form employed in this sentence, since the narrator is telling this story to a non-Kĩsêdjê person.}
\z

\ea  khajkhwa khrat mã, khajkhwa khrat mã\ldots{} \\[.3em]
\gll khajkhwa khrat=mã     khajkhwa khrat=mã     \\
     sky      beginning=to sky      beginning=to \\
\glt `towards the east, towards the east \ldots{}' \\
     `na direção do leste, na direção do leste \ldots{}' \\
\z

\newpage 
\ea  tê, khajkhwajndo mã khátpy ra khatxi kumẽn nhy ajipãmjê ra sarẽn ndo pa. \\[.3em]
\gll tê   khajkhwa=jndo=mã kátpy=ra       khatxi      kumẽn=nhy             aj-i-pãm-jê=ra                 s-arẽ-n=ndo                    pa         \\
     oops sky=end=to       monster=\Nom{} be.numerous be.intense=\AAnd.\Ds{} \Pl-\First-father-\Pl{}=\Nom{} \Third-talk.about-\Nmlz{}=with stay.\Pl{} \\
\glt `I mean, towards the west there were many \textit{khátpy} monsters and our forefathers always told us so.' \\
     `Quero dizer, na direção do oeste tinha muitos monstros \textit{khátpy} e nossos antepassados sempre contavam isso.' \\
\z

\ea  Nenhy ajipãmjê thõ ra pá khôt thẽ. \\[.3em]
\gll ne=nhy           aj-i-pãm-jê=thõ=ra                 pá     khôt  thẽ      \\
     be.so=\AAnd.\Ds{} \Pl-\First-father-\Pl{}=one=\Nom{} forest along go.\Sg{} \\
\glt `And so one of our forefathers went hunting in the forest.' \\
     `E então um dos nossos antepassados foi caçar na floresta.' \\
\label{exe:nomdet}
\z

\ea  Pá khôt thẽ\ldots{} pá khôt thẽn jowi, \\[.3em]
\gll pá=khôt      thẽ-\ldots{}\footnotemark{} pá=khôt      thẽ=n             jowi        \\
     forest=along go.\Sg{}-\Ints{}            forest=along go.\Sg=\AAnd.\Ss{} as.they.say \\
\glt `He was in the forest for a while, he was in the forest and then, as they say,' \\
     `Ele ficou na floresta bastante tempo, ele estava na floresta e dizem que,' \\
\footnotetext{Suspension points are used thorough the text to indicate phonological lengthening. In English or Portuguese, lengthening doesn't always serve the same function as it does in Kĩsêdjê. In particular, in Kĩsêdjê, verb lengthening can indicate an intensification or prolongation of the action depicted by the verb. Whenever that is the case, I add an appropriate adverbial to the free translation.}
\z

\ea  khyj wê khukwâj sak nhy jêt. \\[.3em]
\gll khyj=wê    khukwâj sak=nhy           jêt        \\
     above=from monkey  pierce=\AAnd.\Ds{} hang.\Sg{} \\
\glt `he shot a monkey with an arrow and the monkey got stuck up there.' \\
     `ele flexou um macaco e o macaco ficou preso lá em cima.' \\
\label{exe:adjppdirobjver}
\z

\largerpage
\ea  Khukwâj sak nhy jêt nhy \\[.3em]
\gll khukwâj sak=nhy           jêt=nhy               \\
     monkey  pierce=\AAnd.\Ds{} hang.\Sg{}=\AAnd.\Ds{} \\
\glt `He shot the monkey with an arrow, it got stuck and' \\
     `Ele flexou o macaco, ele ficou preso e' \\
\z

\newpage 
\ea  swârâ apin, swârâ apin kukwâj me nhy thẽn ``ty''\ldots{} ne nhy \\[.3em]
\gll ∅-swârâ        api=n               ∅-swârâ        api=n               kukwâj me=nhy                                thẽ=n             ty            ne=nhy           \\
     \Third-towards climb.up=\AAnd.\Ss{} \Third-towards climb.up=\AAnd.\Ss{} monkey throw.\Sg{}\footnotemark{}=\AAnd.\Ds{} go.\Sg=\AAnd.\Ss{} \Ontp:falling be.so=\AAnd.\Ds{} \\
\glt `he climbed up after the monkey, pushed it, and then the monkey went and ``\textit{ty}''\ldots{} and then' \\
     `ele subiu atrás do macaco, empurrou e aí o macaco foi e ``\textit{ty}''\ldots{} e aí' \\
\footnotetext{Verbal number in Kĩsêdjê is indicated by suppletive forms. Not all verbs distinguish between a singular and a plural form, but many of the high-frequency verbs do. The plural form indicates either that the absolutive argument (intransitive subject or transitive object) is numerous (usually 3 or more) or that the event depicted by the verb is somehow extended.}
\z

\ea  arâ kátpy ra jáwi arâ sahwan tho ta. \\[.3em]
\gll arâ     kátpy=ra       jáwi        arâ     s-ahwa=n               t<h>o                     ta          \\
     already monster=\Nom{} as.they.say already \Third-wait=\AAnd.\Ss{} <\Third>\footnotemark{}at stand.\Sg{} \\
\glt `as they say, the \textit{khátpy} monster was already waiting for him, was standing underneath him.' \\
     `como dizem, o monstro \textit{khátpy} já estava esperando por ele, estava esperando embaixo dele.' \\
\footnotetext{Many heads that begin with an unaspirated `t' or `k' mark agreement with third person through aspiration of their initial consonant.}
\z

\ea  Sahwan tho ta nhy \\[.3em]
\gll s-ahwa=n               t<h>o                     ta=nhy                 \\
     \Third-wait=\AAnd.\Ss{} <\Third>at stand.\Sg{}=\AAnd.\Ds{} \\
\glt `He was waiting for him, was underneath him and then' \\
     `Ele estava esperando ele, estava debaixo dele e aí' \\
\z

\ea  khukwâj me ne nen, akwyn rwâk mã thẽ, tên\ldots{} khátpy ra arâ hwaj wê ta. \\[.3em]
\gll khukwâj me=ne                  ne=n             akwyn rwâ-k=mã              thẽ      tên          khátpy=ra      arâ     ∅-hwaj=wê        ta          \\
     monkey  throw.\Sg{}=\AAnd.\Ss{} do.so=\AAnd.\Ss{} back  climb.down-\Nmlz{}=to go.\Sg{} unexpectedly monster=\Nom{} already \Third-feet=from stand.\Sg{} \\
\glt `he (the forefather) had thrown the monkey and so was climbing back down, but to his surprise the \textit{khátpy} monster was already standing under him.' \\
     `ele (o antepassado) tinha jogado o macaco e estava descendo de volta, mas para a sua surpresa o monstro \textit{khátpy} já estava esperando embaixo dele.' \\
\label{exe:nommain}
\z

% juntei duas frases da transcrição e resegmentei
\ea  ``Tuu\ldots{} wâtân tã khupẽ kasák, khupẽ kasák ta ikhôt thẽ?'' \\[.3em]
\gll tuu  wâtâ=n        tã     khupẽ     kasák   khupẽ     kasák=ta       i-khôt       thẽ      \\
     damn what=\N\Fut{} though foreigner be.evil foreigner be.evil=\Nom{} \First-along go.\Sg{} \\
\glt `{}``Damn! Why did the evil foreigner follow me, though?''{}' \\
     `{}``Droga! Mas então por que o estrangeiro malvado me seguiu?''{}' \\
\z

\ea  Pá txi, pá txi, wet khôt athẽm wa itôra nhimbry jawê ithẽm na thã! \\[.3em]
\gll pá     txi    pá     txi    [\footnotemark{} wet   khôt  a-thẽ-m=wa                               i-tôra             nhimbry j-awê    i-thẽ-m               ]=na        thã            \\
     forest be.big forest be.big {}               other along \Second-go.\Sg-\Nmlz{}=\AAnd.\Ds.\First{} \First-differently game    \E-after \First-go.\Sg-\Nmlz{} {}=\N\Fut{} \Intj:surprise \\
\glt `{}``The forest is big, the forest is big, you should have gone to a part of it and I should have gone to a different part after my game!''{}' \\
     `{}``A floresta é grande, a floresta é grande, você devia ter ido para uma parte dela e eu devia ter ido para uma outra parte atrás da minha caça!''{}' \\
\footnotetext{When I consider it relevant for understanding the structure of the sentence, I indicate the boundaries of an embedded clause with square brackets.}
\z

\ea  Wapãmjê thõ ra swârâ rwâk mã thẽ. \\[.3em]
\gll wa-pãm-jê=thõ=ra                                   ∅-swârâ        rwâ-k=mã              thẽ      \\
     \First\Incl\footnotemark{}-father-\Pl{}=one=\Nom{} \Third-towards climb.down-\Nmlz{}=to go.\Sg{} \\
\glt `Our  forefather came climbing down.' \\
     `Nosso antepassado veio descendo.' \\
\footnotetext{The narrator seems to have slipped into first inclusive forms, in spite of the fact that he's narrating this story to a non-Kĩsêdjê person. Since he usually tells this narrative to other Kĩsêdjê people, these are probably the forms he's used to employing, and doing otherwise may require conscious effort.}
\label{exe:switch-to-inclusive}
\z

\ea  Swârâ rwâk mã thẽn\ldots{} thât rwâk mã thẽ nhy khuthã khura. \\[.3em]
\gll ∅-swârâ        rwâ-k=mã              thẽ=n             thât   rwâ-k=mã              thẽ=nhy             k<h>uthã       k<h>ura     \\
     \Third-towards climb.down-\Nmlz{}=to go.\Sg=\AAnd.\Ss{} though climb.down-\Nmlz{}=to go.\Sg{}=\AAnd.\Ds{} <\Third>before <\Third>hit \\
\glt `Came climbing down towards him\ldots{} He was going to finish coming down, but he (the \textit{khátpy} monster) hit him before he did so.' \\
     `Veio descendo para perto dele\ldots{} Ele ia terminar de descer, mas ele (o monstro \textit{khátpy}) bateu nele antes dele acabar.' \\
\z

\ea  Khuthã khura nhy thẽn ``ty'' nen thãm ne no. \\[.3em]
\gll k<h>uthã       k<h>ura=nhy            thẽ=n             ty            ne=n             thãm=ne                    no             \\
     <\Third>before <\Third>hit=\AAnd.\Ds{} go.\Sg=\AAnd.\Ss{} \Ontp:falling do.so=\AAnd.\Ss{} fall.down.\Sg{}=\AAnd.\Ss{} lie.down.\Sg{} \\
\glt `He (the \textit{khátpy} monster) hit him before he did so, he (the forefather) went and ``\textit{ty}'', and so he fell and lay there.' \\
     `Ele (o monstro \textit{khátpy}) bateu nele antes dele acabar (de descer), ele (o antepassado) foi e ``\textit{ty}'', e assim foi que ele caiu e ficou deitado.' \\
\z

\ea  Thẽn thãm nhy: \\[.3em]
\gll thẽ=n             thãm=nhy                   \\
     go.\Sg=\AAnd.\Ss{} fall.down.\Sg{}=\AAnd.\Ds{} \\
\glt `He went and fell, and he (the \textit{khátpy} monster):' \\
     `Ele foi e caiu, e ele (o monstro \textit{khátpy}):' \\
\z

% juntei duas frases da transcrição e resegmentei
\ea  ``Haha\ldots{} hahaa hwararo ikhrajê re samdep khãm sãm ndo sambak nhy ire ngrytxi pĩrĩ wyrák thã.'' \\[.3em]
\gll haha               hahaa              [  hwararo   i-khra-jê=re              s-amdep-∅                khãm s-ã-m=ndo                           s-amba-k=nhy                   ire           ngry-txi  pĩ-rĩ                         ]  wyrák\ thã    \\
     \Intj:satisfaction \Intj:satisfaction {} yesterday \First-child-\Pl{}=\Erg{} \Third-be.hungry-\Nmlz{} in   \Third-be.standing.\Sg-\Nmlz{}=with \Third-mind-\Nmlz{}=\AAnd.\Ds{} \First.\Erg{} beast-big kill.\Sg\footnotemark-\Nmlz{} {} happen.indeed \\
\glt `{}``That's good, that's good. Yesterday my children were hungry all day long and today I have killed a big beast indeed.''{}' \\
     `{}``Isso é bom, isso é bom. Ontem minhas crianças ficaram com fome o dia inteiro e hoje eu matei um animal grande mesmo.''{}' \\
\footnotetext{The verb \textit{pĩ} actually means `to wound unconscious`. For simplicity, I gloss it as `kill'.}
\z

\newpage 
\ea  Nen khatyp khôsátxi nhihwêt ne thore khãm sarõn ne khutá. \\[.3em]
\gll ne=n             ∅-khatyp   khôsátxi nh-ihwêt=ne        thore ∅-khãm    s-arõn=ne              khu-tá                  \\
     do.so=\AAnd.\Ss{} \Third-for basket   \E-make=\AAnd.\Ss{} then  \Third-in \Third-fold=\AAnd.\Ss{} \Third-put.inside.\Sg{} \\
\glt `He did (said) so, made a basket for him (the ancestor), and then he (the monster) folded him and put him inside.' \\
     `Ele fez (disse) assim, fez um cesto para ele (o ancestral), e então ele (o monstro) dobrou e colocou ele dentro.' \\
\z

\ea  Thore sĩpy khukwâj ngrên hwan nen arêkmã khuthun tho mo. \\[.3em]
\gll thore s-ĩpy         khukwâj ngrê-n                 hwa=n             ne=n             arêkmã khu-thu=n                      t<h>o        mo       \\
     then  \Third-on.top monkey  put.inside.\Pl-\Nmlz{} finish=\AAnd.\Ss{} be.so=\AAnd.\Ss{} soon   \Third-load.on.back=\AAnd.\Ss{} <\Third>with go.\Pl{} \\
\glt `Then he finished putting the monkeys inside on top of him, and this way he loaded it onto his back and carried it far away.' \\
     `Então ele acobu de botar os macacos em cima dele, e dessa forma ele botou nas costas e foi carregando lá para longe.' \\
\z

\ea  Khwã hry ro thẽn\ldots{} sĩthep ne, khwã hry ro thẽn sĩthep ne, \\[.3em]
\gll kh-wã     hry   ro   thẽ=n-\ldots{}           s-ĩthep=ne             kh-wã     hry   ro   thẽ=n             s-ĩthep=ne             \\
     \Third-to trail with go.\Sg=\AAnd.\Ss-\Ints{} \Third-stop=\AAnd.\Ss{} \Third-to trail with go.\Sg=\AAnd.\Ss{} \Third-stop=\AAnd.\Ss{} \\
\glt `He would be opening the trail for a while and then stop, would be opening the trail and then stop,' \\
     `Ele ia abrindo a trilha por um tempo e então parava, ia abrindo a trilha e então parava,' \\
\label{exe:adjppargppver}
\z

\ea  khwã hwĩ khrakhrak to thẽn khwã hry ro thẽn hwĩ khrakhrak to thẽn sĩthep ne, \\[.3em]
\gll kh-wã     hwĩ    khrakhrak to   thẽ=n             kh-wã     hry   ro   thẽ=n             hwĩ    khrakhrak to   thẽ=n             s-ĩthep=ne             \\
     \Third-to branch break     with go.\Sg=\AAnd.\Ss{} \Third-to trail with go.\Sg=\AAnd.\Ss{} branch break     with go.\Sg=\AAnd.\Ss{} \Third-stop=\AAnd.\Ss{} \\
\glt `he was breaking branches, opening a trail, breaking branches and then he stopped,' \\
     `ele estava quebrando galhos, abrindo uma trilha, quebrando galhos então ele parou,' \\
\z

\ea  akwyn thẽn khuthun, khuthun tho mo. \\[.3em]
\gll akwyn thẽ=n             khu-thu=n                      khu-thu=n                      t<h>o        mo       \\
     back  go.\Sg=\AAnd.\Ss{} \Third-load.on.back=\AAnd.\Ss{} \Third-load.on.back=\AAnd.\Ss{} <\Third>with go.\Pl{} \\
\glt `came back and loaded it onto his back, loaded it onto his back and carried it far away.' \\
     `voltou e botou nas costas, botou nas costas e carregou para longe.' \\
\z

\ea  Tho mon\ldots{} kôre hwĩ khrakhrak ne hry nhithep khãm khutan, \\[.3em]
\gll t<h>o        mo=n-\ldots{}            [  kôre          hwĩ    khrakhrak-∅=ne           hry   nh-ithep ]  khãm khu-ta=n                           \\
     <\Third>with go.\Pl=\AAnd.\Ss-\Ints{} {} \Third.\Erg{} branch break-\Nmlz{}=\AAnd.\Ss{} trail \E-stop  {} in   \Third-put.standing.\Sg=\AAnd.\Ss{} \\
\glt `He carried it for a long time and placed it where he had stopped breaking branches, at the end of the trail, and' \\
     `Ele carregou muito tempo e colocou no lugar onde ele parou de quebrar galho, no fim da trilha, e' \\
\label{exe:koreemb}
\z

\ea  amu nen khwã hwĩ khrakhrak to thẽn, \\[.3em]
\gll amu     ne=n             kh-wã     hwĩ    khrakhrak-∅   to   thẽ=n             \\
     farther do.so=\AAnd.\Ss{} \Third-to branch break-\Nmlz{} with go.\Sg=\AAnd.\Ss{} \\
\glt `continued breaking branches,' \\
     `continuou quebrando galhos,' \\
\z

\ea  hwĩ khrakhrak to thẽn sĩthep ne \\[.3em]
\gll hwĩ    khrakhrak to   thẽ=n             s-ĩthep=ne             \\
     branch break     with go.\Sg=\AAnd.\Ss{} \Third-stop=\AAnd.\Ss{} \\
\glt `he went breaking branches and then he stopped and' \\
     `ele foi quebrando galhos e então parou e' \\
\z

\ea  akwyn nen thẽn khuthu nen amu tho thẽ. \\[.3em]
\gll akwyn ne=n             thẽ=n             khu-thu             ne=n             amu     t<h>o        thẽ      \\
     back  do.so=\AAnd.\Ss{} go.\Sg=\AAnd.\Ss{} \Third-load.on.back be.so=\AAnd.\Ss{} farther <\Third>with go.\Sg{} \\
\glt `went back, loaded it onto his back and so carried it farther.' \\
     `voltou, botou nas costas e assim carregou mais além.' \\
\z

\ea  Tho thẽn\ldots{} kôre hwĩ khrakhrak to thẽm nhithep khãm khuta. \\[.3em]
\gll t<h>o        thẽ=n-\ldots{}             [  kôre          hwĩ    khrakhrak-∅   to   thẽ-m          nh-ithep ]  khãm khu-ta                    \\
     <\Third>with go.\Sg=\AAnd.\Ss{}-\Ints{} {} \Third.\Erg{} branch break-\Nmlz{} with go.\Sg-\Nmlz{} \E-stop  {} in   \Third-put.standing.\Sg{} \\
\glt `He carried it for a while and then placed it where he had stopped breaking branches.' \\
     `Carregou por um tempo e aí colocou onde ele tinha parado de quebrar galhos.' \\
\z

\ea  Amu nen tho thẽ. \\[.3em]
\gll amu     ne=n             t<h>o        thẽ      \\
     farther do.so=\AAnd.\Ss{} <\Third>with go.\Sg{} \\
\glt `Doing so he continued carrying it farther.' \\
     `Fazendo assim ele continuou carregando.' \\
\z

\ea  Amu nen khuthun tho mon\ldots{} kôre hry ro thẽm ne hwĩ khrakhrak nhithep khãm khuta. \\[.3em]
\gll amu     ne=n             khu-thu=n                      t<h>o        mo=n-\ldots{}              [  kôre          hry   ro   thẽ-m=ne                  hwĩ    khrakhrak-∅   nh-ithep-∅      ]  khãm khu-ta                    \\
     farther do.so=\AAnd.\Ss{} \Third-load.on.back=\AAnd.\Ss{} <\Third>with go.\Pl=\AAnd.\Ss{}-\Ints{} {} \Third.\Erg{} trail with go.\Sg-\Nmlz{}=\AAnd.\Ss{} branch break-\Nmlz{} \E-stop-\Nmlz{} {} in   \Third-put.standing.\Sg{} \\
\glt `He did so farther, loaded it onto his back, carried it a long time and placed it where he had stopped opening the trail and breaking branches.' \\
     `Ele continuou assim e botou nas costas, carregou muito tempo e colocou onde ele tinha parado de fazer trilha e quebrar galhos.' \\
\z

\largerpage
\ea  Khuta nhy hõnen ndo khá tha. \\[.3em]
\gll khu-ta=nhy                           hõne=n              ndo khá  tha     \\
     \Third-put.standing.\Sg{}=\AAnd.\Ds{} be.ready=\AAnd.\Ss{} eye skin rupture \\
\glt `He placed it on the ground and then he (the forefather) was ready and woke up.' \\
     `Ele colocou no chão e então ele (o antepassado) estava pronto e acordado.' \\
\z

\ea  Amtysamdep ta khambrô khôt khunta nhy nen ndo khátha. \\[.3em]
\gll amty s-amdep-∅=ta                    ∅-khambrô    khôt  khu-nta=nhy            ne=n             ndo khá  tha     \\
     wasp \Third-be.hungry-\Nmlz{}=\Nom{} \Third-blood after \Third-bite=\AAnd.\Ds{} be.so=\AAnd.\Ss{} eye skin rupture \\
\glt `A wasp of the \textit{amtysamdep} species bit him to suck his his blood and he woke up.' \\
     `Uma vespa da espécio \textit{amtysamdep} mordeu ele para sugar o sangue e assim ele acordou.' \\
\label{exe:subadjppdirargver}
\z

\ea  Tên\ldots{} khátpy ra khuthun tho mo nhy ndo khá than jáwi arak anhi khãm mbaj to no. \\[.3em]
\gll tên          khátpy=ra      khu-thu=n                      t<h>o        mo=nhy              ndo khá  tha=n              jáwi        arak    anhi khãm ∅-mba-j             to   no             \\
     unexpectedly monster=\Nom{} \Third-load.on.back=\AAnd.\Ss{} <\Third>with go.\Pl{}=\AAnd.\Ds{} eye skin rupture=\AAnd.\Ss{} as.they.say already self in   \Third-hear-\Nmlz{} with lie.down.\Sg{} \\
\glt `Unexpectedly, the \textit{khátpy} monster had loaded it (the basket) on his back and was carrying it, and he (the ancestor) woke up and was already lying down listening.' \\
     `De repente, o monstro \textit{khátpy} tinha botado (o cesto) nas costas e estava carregando, e ele (o ancestral) acordou e já estava deitado escutando.' \\
\label{exe:nommaintran}
\z

\ea  ``Tu âââ\ldots{} khupẽ kasák, khupẽ kasák na ithát ne itho mo,'' \\[.3em]
\gll tu\ âââ khupẽ     kasák   khupẽ     kasák=na         i-thát=ne               i-tho       mo       \\
     gosh    foreigner be.evil foreigner be.evil=\N\Fut{} \First-wound=\AAnd.\Ss{} \First-with go.\Sg{} \\
\glt `{}``Gosh\ldots{} the evil monster, it was the evil monster that wounded me and was carrying me,''{}' \\
     `{}``Nossa\ldots{} o monstro malvado, foi o monstro malvado que me feriu e estava me carregando,''{}' \\
\z

\ea  nenhy kôt hwĩ khrakhrak khôt hry ro thẽm nda ro thẽn hry jatuj khãm khutha. \\[.3em]
\gll ne=nhy           [  kôt           hwĩ    khrakhrak-∅   khôt  hry   ro   thẽ-m          ]=nda                  ro thẽ=n             hry   j-atu-j         khãm khu-ta                    \\
     be.so=\AAnd.\Ds{} {} \Third.\Erg{} branch break-\Nmlz{} along trail with go.\Sg-\Nmlz{} {}=\Def\footnotemark{} at go.\Sg=\AAnd.\Ss{} trail \E-stop-\Nmlz{} in   \Third-put.standing.\Sg{} \\
\glt `he thought so and then he (the \textit{khátpy} monster) finished walking the trail that he had built along the broken branches and placed it (the basket) at the end of the trail.' \\
     `ele pensou assim e então ele (o monstro \textit{khátpy}) percorreu a trilha que ele tinha construindo ao longo dos galhos quebrandos e colocou (o cesto) no fim da trilha.' \\
\footnotetext{Relative clauses are internally headed. The head of this one is \textit{hry}.}
\label{exe:relcla1}
\z

\ea  Nenhy amu nen khwã hwĩ khrakhrak to thẽ. \\[.3em]
\gll ne=nhy           amu     ne=n             kh-wã     hwĩ    khrakhrak-∅   to   thẽ      \\
     be.so=\AAnd.\Ds{} farther do.so=\AAnd.\Ss{} \Third-to branch break-\Nmlz{} with go.\Sg{} \\
\glt `And then he (the \textit{khátpy} monster) continued breaking branches.' \\
     `E então ele (o monstro \textit{khátpy}) continuou quebrando galhos.' \\
\z

\ea  Hwĩ khrakhrak to thẽ nhy athũm\ldots{} nhy nen akwyn nen thẽ. \\[.3em]
\gll hwĩ    khrakhrak-∅   to   thẽ=nhy             athũm-\ldots{}=nhy            ne=n             akwyn ne=n             thẽ      \\
     branch break-\Nmlz{} with go.\Sg{}=\AAnd.\Ds{} take.time-\Ints{}=\AAnd.\Ds{} be.so=\AAnd.\Ss{} back  be.so=\AAnd.\Ss{} go.\Sg{} \\
\glt `He was breaking branches, it took some time and then he went back.' \\
     `Ele estava quebrando galhos, demorou um certo tempo e então ele voltou.' \\
\z

\ea  Akwyn nen thẽn khatho. \\[.3em]
\gll akwyn ne=n             thẽ=n             k<h>atho               \\
     back  be.so=\AAnd.\Ss{} go.\Sg=\AAnd.\Ss{} <\Third>come.out.\Sg{} \\
\glt `He came back and came out (of the forest).' \\
     `Ele voltou e saiu (da floresta).' \\
\z

\ea  Khathon akhum khuthun nen tho mo. \\[.3em]
\gll k<h>atho=n                      akhum khu-thu=n                      ne=n             t<h>o        mo       \\
     <\Third>come.out.\Sg=\AAnd.\Ss{} again \Third-load.on.back=\AAnd.\Ss{} be.so=\AAnd.\Ss{} <\Third>with go.\Pl{} \\
\glt `He came out, loaded it again on his back and so continued taking it away.' \\
     `Ele saiu, botou nas costas de novo e assim continuou carregando para longe.' \\
\z

% juntei duas frases da transcrição e resegmentei
\ea  Tho mon kôt hry nhithep khãm khuthan amu thẽ nhy thore tho sujakhre ro no. \\[.3em]
\gll t<h>o        mo=n              [  kôt           hry   nh-ithep-∅      ]  khãm khu-ta=n                           amu     thẽ=nhy             thore t<h>o        s-ujakhre-∅          ro   no             \\
     <\Third>with go.\Pl=\AAnd.\Ss{} {} \Third.\Erg{} trail \E-stop-\Nmlz{} {} in   \Third-put.standing.\Sg=\AAnd.\Ss{} farther go.\Sg{}=\AAnd.\Ds{} then  <\Third>with \Third-count-\Nmlz{} with lie.down.\Sg{} \\
\glt `He took it, left it at the place where he stopped the trail and went farther and then he (the forefather) stayed lying down counting the time.' \\
     `Carregou, deixou no lugar onde ele tinha parado a trilha e continuou adiante e então ele (o antepassado) ficou deitado contando o tempo.' \\
\z

\ea  Tho sujakhre ro no nhy athũm nhy nen akwyn khatho. \\[.3em]
\gll t<h>o        s-ujakhre-∅          ro   no=nhy                    athũm=nhy            ne=n             akwyn k<h>atho               \\
     <\Third>with \Third-count-\Nmlz{} with lie.down.\Sg{}=\AAnd.\Ds{} take.time=\AAnd.\Ds{} be.so=\AAnd.\Ss{} back  <\Third>come.out.\Sg{} \\
\glt `He stayed lying counting the time, it took some time and then he (the \textit{khátpy} monster) came back out.' \\
     `Ele ficou deitado contando o tempo, demorou um tempo e então ele (o monstro \textit{khátpy}) saiu de volta.' \\
\z

\ea  Akwyn swârâ khathon akhum khuthun. \\[.3em]
\gll akwyn ∅-swârâ        k<h>atho=n                      akhum khu-thu=n                                   \\
     back  \Third-towards <\Third>come.out.\Sg=\AAnd.\Ss{} again \Third-load.on.back=\AAnd.\Ss\footnotemark{} \\
\glt `He came back out towards him and loaded it again on his back.' \\
     `Ele saiu de volta pra perto dele e botou de novo nas costas.' \\
\footnotetext{Here, though the coordinating conjunction marks subject maintenance, there actually is a subject switch.}
\z

\newpage
\ea  ``Hyhy! Athaj nhyry ri wa thẽ.'' \\[.3em]
\gll hyhy ∅      athaj nhy-ry              ri=wa                    thẽ      \\
     aha  \Fut{} there do.so.\Nmlz-\Nmlz{} during=\AAnd.\Ds.\First{} go.\Sg{} \\
\glt `{}``Aha! While he is doing that over there I will go away.''{}' \\
     `{}``Aha! Enquanto ele estiver fazendo isso lá eu vou embora.''{}' \\
\z

\ea  Arêkmã, arêkmã wi anhi khãm, anhi mã sumbaj to no jowi. \\[.3em]
\gll arêkmã arêkmã wi     anhi khãm anhi=mã s-umba-j             to   no             jowi        \\
     soon   soon   indeed self in   self=to \Third-think-\Nmlz{} with lie.down.\Sg{} as.they.say \\
\glt `Soon, as they say, he (the forefather) stayed lying down thinking to himself.' \\
     `Logo, segundo dizem, ele (o antepassado) ficou deitado pensando sozinho.' \\
\z

\ea  Tho thẽn\ldots{} akhum khuthun tho mon khuta. \\[.3em]
\gll t<h>o        thẽ=n             akhum khu-thu=n                      t<h>o        mo=n              khu-ta                    \\
     <\Third>with go.\Sg=\AAnd.\Ss{} again \Third-load.on.back=\AAnd.\Ss{} <\Third>with go.\Pl=\AAnd.\Ss{} \Third-put.standing.\Sg{} \\
\glt `He (the \textit{khátpy} monster) kept taking it\ldots{} loaded him on his back again, took it and put it down.' \\
     `Ele (o monstro \textit{khátpy}) foi levando\ldots{} botou nas costas de novo, levou e depois colocou no chão.' \\
\z

\ea  Kôt hry jatuj khãm khutha. \\[.3em]
\gll [  kôt           hry   j-atu-j         ]  khãm khu-ta                    \\
     {} \Third.\Erg{} trail \E-stop-\Nmlz{} {} in   \Third-put.standing.\Sg{} \\
\glt `He put it down where he had stopped making the trail.' \\
     `Colocou no chão onde ele tinha parado de fazer a trilha.' \\
\z

\largerpage
\ea  Akhum thẽ, akhum thẽ nhy akhum tho sujakhre mbet to no nhy\ldots{} \\[.3em]
\gll akhum thẽ      akhum thẽ=nhy             akhum t<h>o        s-ujakhre-∅          mbet    to   no=nhy                    \\
     again go.\Sg{} again go.\Sg{}=\AAnd.\Ds{} again <\Third>with \Third-count-\Nmlz{} be.good with lie.down.\Sg{}=\AAnd.\Ds{} \\
\glt `He went away again and then he (the forefather) lay down counting time with attention and\ldots{}' \\
     `Ele foi pra longe de novo e então ele (o antepassado) ficou deitado contando o tempo com atenção e\ldots{}' \\
\z

\ea  thẽm nda tho athũm nhy, ``Hy! Kê nhyry ri wa ikatho ra'' nen, \\[.3em]
\gll thẽ-m=nda             t<h>o        athũm=nhy            hy kê     nhy-ry              ri=wa                    i-katho               ra     ne=n             \\
     go.\Sg-\Nmlz{}=\Nom{} <\Third>with take.time=\AAnd.\Ds{} Ha \Fut{} do.so.\Nmlz-\Nmlz{} during=\AAnd.\Ds.\First{} \First-come.out.\Sg{} indeed be.so=\AAnd.\Ss{} \\
\glt `his trip took some time, then he (the forefather) said, ``Ha! While he does so I will come out indeed,''{}' \\
     `a viagem dele demorou um certo tempo, aí ele (o antepassado) disse, ``Ha! Enquanto ele estiver fazendo isso eu vou sair mesmo,''{}' \\
\z

\ea  athũm kharõ nhy nen akwyn khatho. \\[.3em]
\gll athũm-∅           kharõ=nhy         ne=n             akwyn k<h>atho               \\
     take.time-\Nmlz{} appear=\AAnd.\Ds{} be.so=\AAnd.\Ss{} back  <\Third>come.out.\Sg{} \\
\glt `a little more time passed and he (the \textit{khátpy} monster) came back out.' \\
     `passou um pouco mais de tempo e ele (o monstro \textit{khátpy}) saiu de volta.' \\
\z

\ea  ``Hyhy! Kê nhyry ri wa thẽ.'' \\[.3em]
\gll hyhy kê     nhy-ry              ri=wa                    thẽ      \\
     aha  \Fut{} be.so.-\Nmlz{} during=\AAnd.\Ds.\First{} go.\Sg{} \\
\glt `{}``Aha! While he's doing so I will go.''{}' \\
     `{}``Aha! Enquanto ele estiver fazendo isso eu vou embora.''{}' \\
\z

\ea  Khathon, akhum khuthun tho mo. \\[.3em]
\gll k<h>atho=n                      akhum khu-thu=n                      t<h>o        mo       \\
     <\Third>come.out.\Sg=\AAnd.\Ss{} again \Third-load.on.back=\AAnd.\Ss{} <\Third>with go.\Pl{} \\
\glt `He (the \textit{khátpy} monster) came out, loaded it again on his back and took it away.' \\
     `Ele (o monstro \textit{khátpy}) saiu de volta, botou do novo nas costas e levou embora.' \\
\z

\largerpage[2]
\ea  Akhum khuthun tho mon\ldots{} kôt hry jatuj khãm khutha. \\[.3em]
\gll akhum khu-thu=n                      t<h>o        mo=n-\ldots{}              [  kôt           hry   j-atu-j         ]  khãm khu-ta                    \\
     again \Third-load.on.back=\AAnd.\Ss{} <\Third>with go.\Pl=\AAnd.\Ss{}-\Ints{} {} \Third.\Erg{} trail \E-stop-\Nmlz{} {} in   \Third-put.standing.\Sg{} \\
\glt `He loaded it again on his back, took it all the way and put it down where he had stopped the trail.' 
\newpage 
     `Ele botou de novo nas costas, levou o caminho todo e colocou no chão onde ele tinha parado a trilha.'  
\z

\ea  Hry jatuj khãm khuthan, \\[.3em]
\gll hry   j-atu-j         khãm khu-ta=n                           \\
     trail \E-stop-\Nmlz{} in   \Third-put.standing.\Sg=\AAnd.\Ss{} \\
\glt `He put it down where he had stopped the trail, and' \\
     `Ele botou no chão onde ele tinha parado a trilha, e' \\
\z

\ea  thât hwĩ khrakhrak to, khwã hry ro thẽ\ldots{} nhy ary\ldots{}re nhy. \\[.3em]
\gll thât   hwĩ    khrakhrak-∅   to   kh-wã     hry   ro   thẽ-\ldots{}=nhy             ary-\ldots{}-re          nhy              \\
     though branch break-\Nmlz{} with \Third-to trail with go.\Sg{}-\Ints{}=\AAnd.\Ds{} for.long-\Ints{}-little be.sitting.\Sg{} \\
\glt `was uselessly breaking branches, building a long trail, and then he (the forefather) sat for a little long.' \\
     `estava inutilmente quebrando galhos, construindo uma longa trilha, e então ele (o antepassado) ficou um tempinho sentado.' \\
\z

\ea  Nen khatho ne\ldots{} anhi ndêt khukwâj to thêk ne khathon, \\[.3em]
\gll ne=n             k<h>atho=ne                       anhi ndêt      khukwâj to   thêk=ne             k<h>atho=n                      \\
     do.so=\AAnd.\Ss{} <\Third>come.out.\Sg{}=\AAnd.\Ss{} self away.from monkey  with push.out=\AAnd.\Ss{} <\Third>come.out.\Sg=\AAnd.\Ss{} \\
\glt `He did so and then came out\ldots{} Pushed the monkeys out away from him, came out, and' \\
     `Ele fez isso e então saiu\ldots{} Empurrou os macacos para fora, para longe de si, saiu, e' \\
\z

\ea  thore khẽn tha ro thẽn khuthán, \\[.3em]
\gll thore khẽn tha  ro   thẽ=n             khu-tá=n                         \\
     then  rock this with go.\Sg=\AAnd.\Ss{} \Third-put.inside.\Sg=\AAnd.\Ss{} \\
\glt `then he brought a rock and put it inside,' \\
     `então ele trouxe uma pedra e botou dentro,' \\
\z

\largerpage[2]
\ea  khẽn tha ro thẽn khuthán, \\[.3em]
\gll khẽn tha  ro   thẽ=n             khu-tá=n                         \\
     rock this with go.\Sg=\AAnd.\Ss{} \Third-put.inside.\Sg=\AAnd.\Ss{} \\
\glt `he brought another rock and put it inside,' \\
     `trouxe outra pedra e botou dentro,' 
\z
\newpage 

\ea  khẽn tha ro thẽn khuthán, sĩpy khukwâj ngrên hwan, \\[.3em]
\gll khẽn tha  ro   thẽ=n             khu-tá=n                         s-ĩpy         khukwâj ngrê-n                 hwa=n             \\
     rock this with go.\Sg=\AAnd.\Ss{} \Third-put.inside.\Sg=\AAnd.\Ss{} \Third-on.top monkey  put.inside.\Pl-\Nmlz{} finish=\AAnd.\Ss{} \\
\glt `he brought another rock and put it inside, and on top of them finished putting all the monkeys inside,' \\
     `trouxe outra pedra e colocou dentro, e no topo delas terminou de colocar todos os macacos em cima,' \\
\z

\ea  hõnen arêkmã khukwâj itha wytin khupyn nen thẽ. \\[.3em]
\gll hõne=n              arêkmã khukwâj itha wyti=n        khu-py=n                    ne=n             thẽ      \\
     be.ready=\AAnd.\Ss{} soon   monkey  this be.one=\AAnd{} \Third-fetch.\Sg=\AAnd.\Ss{} do.so=\AAnd.\Ss{} go.\Sg{} \\
\glt `he was done and then took this one monkey and so went away.' \\
     `ele acabou e logo pegou um macaco e foi embora.' \\
\label{exe:ithawyti}
\z

\ea  Thẽn\ldots{} athũm nhy akwyn khatho. \\[.3em]
\gll thẽ=n-\ldots{}             athũm=nhy            akwyn k<h>atho               \\
     go.\Sg=\AAnd.\Ss{}-\Ints{} take.time=\AAnd.\Ds{} back  <\Third>come.out.\Sg{} \\
\glt `He went far away, it took time and he (the \textit{khátpy} monster) came back out.' \\
     `Ele foi embora para longe, demorou um tempo e ele (o monstro \textit{khátpy}) saiu de volta.' \\
\z

\ea  Akwyn khathon khuthun, khẽn mã wikamẽn pa. \\[.3em]
\gll akwyn k<h>atho=n                      khu-thu=n                      khẽn mã ∅-wikamẽ=n                  pa         \\
     back  <\Third>come.out.\Sg=\AAnd.\Ss{} \Third-load.on.back=\AAnd.\Ss{} rock to \Third-be.pulled=\AAnd.\Ss{} stay.\Pl{} \\
\glt `He came back out, loaded it on his back and kept being pulled down towards the rocks.' \\
     `Ele saiu de volta, botou nas costas e ficou sendo puxado para baixo em direção às pedras.' \\
\z

\newpage
\ea  Khẽn mã wikamẽn pan, ``Haaa! Ngrytxi thyk mbet ne khajkhit na!'' \\[.3em]
\gll khẽn mã ∅-wikamẽ=n                  pa=n                Haaa ngry-txi  thy-k       mbet-∅=ne                  ∅-khajkhit-∅=na                  \\
     rock to \Third-be.pulled=\AAnd.\Ss{} stay.\Pl=\AAnd.\Ss{} Argh beast-big die-\Nmlz{} be.good-\Nmlz{}=\AAnd.\Ss{} \Third-be.light-\Nmlz{}=\N\Fut{} \\
\glt `He kept being pulled down towards the rocks and said ``Argh!'' A big beast that is well dead is supposed to be light!''{}' \\
     `Ele ficou sendo puxado para baixo em direção às pedras e disse ``Argh!'' Um animal grande que está bem morto era para ser leve!''{}' \\
\z

\ea  ``Ngrytxi thyk mbet ne khajkit na!'' \\[.3em]
\gll ngry-txi  thy-k       mbet-∅=ne                  ∅-khajkhit-∅=na                  \\
     beast-big die-\Nmlz{} be.good-\Nmlz{}=\AAnd.\Ss{} \Third-be.light-\Nmlz{}=\N\Fut{} \\
\glt `{}``A big beast that is well dead is supposed to be light!''{}' \\
     `{}``Um animal grande que está bem morto era para ser leve!''{}' \\
\z

\ea  Nenhy tho huuuh\ldots{} nen sarĩn tho thẽ. \\[.3em]
\gll ne=nhy           t<h>o        huuuh ne=n             s-arĩ=n                t<h>o        thẽ      \\
     be.so=\AAnd.\Ds{} <\Third>with umpf  be.so=\AAnd.\Ss{} \Third-jump=\AAnd.\Ss{} <\Third>with go.\Sg{} \\
\glt `And then he grunted and jumped and took it away.' \\
     `E aí ele grunhiu e pulou e levou embora.' \\
\z

\ea  Khuthun tho mo, tho mo\ldots{} hry jatuj khãm khuta. \\[.3em]
\gll khu-thu=n                      t<h>o        mo       t<h>o        mo-\ldots{}       hry   j-atu-j         khãm khu-ta                    \\
     \Third-load.on.back=\AAnd.\Ss{} <\Third>with go.\Pl{} <\Third>with go.\Pl{}-\Ints{} trail \E-stop-\Nmlz{} in   \Third-put.standing.\Sg{} \\
\glt `He loaded it on his back and took it all the way, took it all the way and put it down at the end of the trail.' \\
     `Ele botou nas costas e levou até o fim, levou até o fim e botou no chão no final da trilha.' \\
\z

\ea  Amu khwã hry ro thẽ\ldots{} \\[.3em]
\gll amu     kh-wã     hry   ro   thẽ      \\
     farther \Third-to trail with go.\Sg{} \\
\glt `He continued opening the trail\ldots{}' \\
     `Ele continuou abrindo a trilha\ldots{}' \\
\z

\ea  Akhum hry jatuj khãm khuta. \\[.3em]
\gll akhum hry   j-atu-j         khãm khu-ta                    \\
     again trail \E-stop-\Nmlz{} in   \Third-put.standing.\Sg{} \\
\glt `He put it down again where he had stopped the trail.' \\
     `Ele deixou de novo no lugar onde ele tinha parado a trilha.' \\
\z

\ea  Hry jatuj khãm khutan, amu nen hry ro thẽ. \\[.3em]
\gll hry   j-atu-j         khãm khu-ta=n                           amu     ne=n             hry   ro   thẽ      \\
     trail \E-stop-\Nmlz{} in   \Third-put.standing.\Sg=\AAnd.\Ss{} farther be.so=\AAnd.\Ss{} trail with go.\Sg{} \\
\glt `He put it down where he had stopped the trail and opened the trail farther.' \\
     `Ele botou no chão onde ele tinha parado a trilha e então continuou abrindo a trilha.' \\
\z

\ea  Hry ro thẽn\ldots{} sĩthep ne akwyn nen khatho. \\[.3em]
\gll hry   ro   thẽ=n-\ldots{}             s-ĩthep-∅=ne                   akwyn ne=n             k<h>atho               \\
     trail with go.\Sg=\AAnd.\Ss{}-\Ints{} \Third-stop-\Nmlz{}=\AAnd.\Ss{} back  be.so=\AAnd.\Ss{} <\Third>come.out.\Sg{} \\
\glt `He opened a long trail, stopped and so came back out.' \\
     `Ele abriu uma trilha comprida, parou e então voltou.' \\
\z

\ea  Akwyn thẽn khathon kê khwã wikamẽn pan \\[.3em]
\gll akwyn thẽ=n             k<h>atho=n                      kê   kh-wã     ∅-wikamẽ=n                  pa=n                \\
     back  go.\Sg=\AAnd.\Ss{} <\Third>come.out.\Sg=\AAnd.\Ss{} also \Third-to \Third-be.pulled=\AAnd.\Ss{} stay.\Pl=\AAnd.\Ss{} \\
\glt `He went back, came out and kept again being pulled down towards them (the rocks),' \\
     `Ele voltou, saiu e ficou de novo sendo puxado para baixo em direção a elas (as pedras),' \\
\z

\ea  khẽn mã wikamẽn pan, ``Haaa! Ngrytxi thyk mbet ne khajkhit na!'' \\[.3em]
\gll khẽn mã ∅-wikamẽ=n                  pa=n                haaa ngry-txi  thy-k       mbet-∅=ne                  ∅-khajkhit-∅=na                  \\
     rock to \Third-be.pulled=\AAnd.\Ss{} stay.\Pl=\AAnd.\Ss{} argh beast-big die-\Nmlz{} be.good-\Nmlz{}=\AAnd.\Ss{} \Third-be.light-\Nmlz{}=\N\Fut{} \\
\glt `he kept being pulled down towards the rocks and said ``Argh! A big beast that is well dead is supposed to be light!''{}' \\
     `ficou sendo puxado para baixo em direção às pedras e disse ``Argh! Um animal grande que está bem morto era para ser leve!''{}' \\
\z

\ea  ``Ngrytxi thyk mbet ne khajkhit na!'' \\[.3em]
\gll ngry-txi  thy-k       mbet-∅=ne                  ∅-khajkhit-∅=na                  \\
     beast-big die-\Nmlz{} be.good-\Nmlz{}=\AAnd.\Ss{} \Third-be.light-\Nmlz{}=\N\Fut{} \\
\glt `{}``A big beast that is well dead is supposed to be light!''{}' \\
     `{}``Um animal grande que está bem morto era para ser leve!''{}' \\
\z

\ea  Khwã wikamẽn pa jahôt to huuuh nen sarĩn amu tho thẽ. \\[.3em]
\gll kh-wã     ∅-wikamẽ=n                  pa-∅             j-ahô-t                 to   huuuh ne=n             s-arĩ=n                amu     t<h>o        thẽ      \\
     \Third-to \Third-be.pulled=\AAnd.\Ss{} stay.\Pl-\Nmlz{} \E-come.out.\Pl-\Nmlz{} with umpf  be.so=\AAnd.\Ss{} \Third-jump=\AAnd.\Ss{} farther <\Third>with go.\Sg{} \\
\glt `He grunted his way out of being pulled down towards the rocks, jumped up and took it farther.' \\
     `Ele ficou grunhindo sendo puxado para baixo em direção às pedras, pulou e levou mais adiante.' \\
\z

\ea  Tho thẽn\ldots{} tho pâj, sũrũkhwã mã tho pâj. \\[.3em]
\gll t<h>o        thẽ=n-\ldots{}             t<h>o        pâj    s-ũrũkhwã    mã t<h>o        pâj    \\
     <\Third>with go.\Sg=\AAnd.\Ss{}-\Ints{} <\Third>with arrive \Third-house to <\Third>with arrive \\
\glt `He carried it all the way and arrived home with it.' \\
     `Ele carregou (a cesta) o resto do caminho e chegou em casa com ela.' \\
\z

\ea  Sũrũkhwã me, sũrũkhwã khãm khrajê me hrõjê swârâ tho pâj. \\[.3em]
\gll s-ũrũkhwã=me,\footnotemark{} s-ũrũkhwã    khãm ∅-khra-jê=me              ∅-hrõ-jê          swârâ   t<h>o        pâj    \\
     \Third-house=\AAnd{}          \Third-house at   \Third-child-\Pl{}=\AAnd{} \Third-wife-\Pl{} towards <\Third>with arrive \\
\glt `At home, he brought it towards his children and wives.' \\
     `Em sua casa, ele levou para junto dos seus filhos e esposas.' \\
\footnotetext{This was a mistake by the narrator.}
\z

% juntei duas frases da transcrição e resegmentei
\ea  Khrajê ra khathon ``Haaa! Waj turê ra ngrytxi pĩn tho mo\ldots{} Waj turê ra ngrytxi pĩn tho mo.'' \\[.3em]
\gll ∅-khra-jê=ra              k<h>atho=n                      haaa waj  turê=ra    ngry-txi  pĩ=n                t<h>o        mo-\ldots{}       waj  turê=ra    ngry-txi  pĩ=n                t<h>o        mo       \\
     \Third-child-\Pl{}=\Nom{} <\Third>come.out.\Sg=\AAnd.\Ss{} ah   must dad=\Nom{} beast-big kill.\Sg=\AAnd.\Ss{} <\Third>with go.\Pl{}-\Ints{} must dad=\Nom{} beast-big kill.\Sg=\AAnd.\Ss{} <\Third>with go.\Pl'' \\
\glt `His childen came out and (said) ``Dad must have killed a big beast and brought it all the way here.''{}' \\
     `Seus filhos saíram e (disseram) ``Papai deve ter matado um animal grande e trouxe até aqui.''{}' \\
\label{exe:wajpripos}
\z

\ea  Tho thẽn\ldots{} khikhre kape mã khẽn me nhy thuk nen ta nhy, \\[.3em]
\gll t<h>o        thẽ=n-\ldots{}             khikhre kape  mã khẽn me=nhy                 thuk ne=n             ta=nhy                 \\
     <\Third>with go.\Sg=\AAnd.\Ss{}-\Ints{} house   front to rock throw.\Sg{}=\AAnd.\Ds{} thud do.so=\AAnd.\Ss{} stand.\Sg{}=\AAnd.\Ds{} \\
\glt `He carried it over, threw the rock in front of the house, it (the basket) made a `\textit{thuk}' and stood there, and' \\
     `Ele carregou até lá, jogou a pedra na frente da casa, ele (o cesto) fez `\textit{thuk}' e ficou lá, e' \\
\z

\ea  nenhy,  ``ndotê, ndotê, ndotê, ndotê'' ne\ldots{} \\[.3em]
\gll ne=nhy           ndotê    ndotê    ndotê    ndotê    ne    \\
     be.so=\AAnd.\Ds{} be.quick be.quick be.quick be.quick do.so \\
\glt `so he said ``Come here, come here, come here, come here\ldots''{}' \\
     `então ele disse ``Venham aqui, venham aqui, venham aqui, venham aqui\ldots''{}' \\
\z

\ea  ``Wakhrajê nho thát, ngrytxi kukhráthá thõ kê khukhrẽ.'' \\[.3em]
\gll wa-khra-jê              nh-o    thát ngry-txi  kukhráthá\ thõ=kê                        khu-khrẽ         \\
     \First\Incl-child-\Pl{} \E-food for  beast-big do.something.with=\AAnd.\Ds.\Third.\Fut{} \Third-eat.\Pl{} \\
\glt `{}``For our childen's food, do something with the big beast and they will eat it.''{}' \\
     `{}``Para a comida dos nossos filhos, faça alguma coisa com o animal grande e eles vão comer.''{}' \\
\z

\ea  ``Hwararo wakhrajê re hrãm khãm sãm.'' \\[.3em]
\gll hwararo   wa-khra-jê=re                  ∅-hrãm-∅              khãm s-ã-m                          \\
     yesterday \First\Incl-child-\Pl{}=\Erg{} \Third-desire-\Nmlz{} in   \Third-be.standing.\Sg-\Nmlz{} \\
\glt `{}``Yesterday our children were hungry''{}' \\
     `{}``Ontem nossos filhos estavam com fome''{}' \\
\z

\ea  ``Khwã ngrytxi pĩn tho mo\ldots{}'' \\[.3em]
\gll kh-wã     ngry-txi  pĩ=n                t<h>o        mo-\ldots{}       \\
     \Third-to beast-big kill.\Sg=\AAnd.\Ss{} <\Third>with go.\Pl{}-\Ints{} \\
\glt ``{}`(I) killed a big beast for them and brought it all the way''{}' \\
     ``{}`(Eu) matei um animal grande para eles e trouxe até aqui''{}' \\
\z

\ea  ``Ka ndotên khwã khukhrátá thõ kê rêt thõ khrẽ.'' \\[.3em]
\gll ka             ndotê=n             kh-wã     k<h>ukhrátá\ thõ=kê                              ∅-rêt=thõ        khrẽ      \\
     \Second.\Nom{} be.quick=\AAnd.\Ss{} \Third-to <\Third>do.something.with=\AAnd.\Ds.\Third.\Fut{} \Third-offal=one eat.\Pl{} \\
\glt `{}``Come quickly and do something with it and then they will eat some offall.''{}' \\
     `{}``Venha logo fazer alguma coisa com ele e aí eles vão comer os miúdos.''{}' \\
\z

\ea  Hrõ ra ``Hy!'' nen khatho. \\[.3em]
\gll ∅-hrõ=ra           hy  ne=n             k<h>atho               \\
     \Third-wife=\Nom{} yes do.so=\AAnd.\Ss{} <\Third>come.out.\Sg{} \\
\glt `His wife said ``Yes!'' and came out.' \\
     `Sua esposa disse ``Sim!'' e saiu.' \\
\z

\ea  Khathon, khukwâj rẽn ndo thẽ. \\[.3em]
\gll k<h>atho=n                      khukwâj rẽ-n              ndo  thẽ      \\
     <\Third>come.out.\Sg=\AAnd.\Ss{} monkey  throw.\Pl-\Nmlz{} with go.\Sg{} \\
\glt `She came out and was throwing monkeys (out of the basket).' \\
     `Ela saiu e ficou jogando macacos (para fora do cesto).' \\
\z

\ea  Wapãmjê thõ wê sĩmbry wê kôt khukwâj hwaj ta, tho hôt to thẽn, tho hôt to thẽn, ``the!'' \\[.3em]
\gll wa-pãm-jê=thõ=wê                  s-ĩmbry     wê     kôt           khukwâj hwa-j=ta                t<h>o        hô-t                 to   thẽ=n             t<h>o        hô-t                 to   thẽ=n             the   \\
     \First\Incl-father-\Pl{}=one=from \Third-game \Cop{} \Third.\Erg{} monkey  kill.\Pl-\Nmlz{}=\Det{} <\Third>with pull.out.\Pl-\Nmlz{} with go.\Sg=\AAnd.\Ss{} <\Third>with pull.out.\Pl-\Nmlz{} with go.\Sg=\AAnd.\Ss{} hey! \\
\glt `The game from our forefather, the monkeys that he had killed, she kept pulling out and (said) ``Hey!'' \\
     `A caça do nosso antepassado, os macacos que ele tinha matado, ela ficou puxando para fora e (disse) ``Oras!'' \\
\label{exe:relcla2}
\z

\ea  Khôsátxi khre khahwãrã ro ta. \\[.3em]
\gll khôsátxi khre       khahwã-rã     ro   ta          \\
     basket   inner.side probe-\Nmlz{} with stand.\Sg{} \\
\glt `She kept probing the inner side of the basket.' \\
     `Ela ficou inspecionando o lado de dentro do cesto.' \\
\z

\ea  ``Nhintã ngrytxi jarẽn nda?'' \\[.3em]
\gll nhintã ngry-txi  j-arẽ-n=nda                  \\
     where  beast-big \E-talk.about-\Nmlz{}=\Det{} \\
\glt `{}``Where is the big beast he talked about?''{}' \\
     `{}``Onde está o animal grande de que ele falou?''{}' \\
\z

\ea  ``Athaj na wa ngrytxi pĩn tho mo.'' \\[.3em]
\gll athaj=na       wa            ngry-txi  pĩ=n                t<h>o        mo       \\
     there=\N\Fut{} \First.\Nom{} beast-big kill.\Sg=\AAnd.\Ss{} <\Third>with go.\Pl{} \\
\glt `{}``I killed a big beast and brought all the way there.''{}' \\
     `{}``Eu matei um animal grande e trouxe até aí.''{}' \\
\label{exe:nasecposathaj}
\z

\ea  ``Ngry txire! Kôt ka sõmun khêrê?'' \\[.3em]
\gll ngry  txi-re     kôt        ka             s-õmu-n            khêrê  \\
     beast big-\Adj{} can.\Fut{} \Second.\Nom{} \Third-see-\Nmlz{} not.be \\
\glt `{}``The beast is big! Can't you see?''{}' \\
     `{}``O animal é grande! Você não consegue ver?''{}' \\
\label{exe:kotpripos}
\z

\newpage
\ea  ``Nhintã tôra ngrytxi jarẽn nda?'' \\[.3em]
\gll nhintã tôra   ngry-txi  j-arẽ-n=nda                  \\
     where  though beast-big \E-talk.about-\Nmlz{}=\Det{} \\
\glt `{}``Where, though, is the big beast you talk about?''{}' \\
     `{}``Onde então é que está esse animal grande de que você fala?''{}' \\
\z

\ea  ``Ngrytxi khêrê weri!'' \\[.3em]
\gll ngry-txi  khêrê  weri   \\
     beast-big not.be indeed \\
\glt `{}``There is indeed no big beast!''{}' \\
     `{}``Não tem mesmo animal grande nenhum!''{}' \\
\z

\ea  ``Ngry txire! kôt ka sõmun khêrê?'' \\[.3em]
\gll ngry  txi-re     kôt        ka             s-õmu-n            khêrê  \\
     beast big-\Adj{} can.\Fut{} \Second.\Nom{} \Third-see-\Nmlz{} not.be \\
\glt `{}``The beast is big! Can't you see it?''{}' \\
     `{}``O animal é grande! Você não consegue ver?''{}' \\
\z

\ea  ``Rik tã, ndotên wakhrajê nho thát khukhrátá thõ kê khwê rêt thõ khrẽ!'' \\[.3em]
\gll rik     tã     ndotê=n             wa-khra-jê              nh-o    thát k<h>ukhrátá\ thõ=kê                              kh-wê       ∅-rêt=thõ        khrẽ    \\
     quickly indeed be.quick=\AAnd.\Ss{} \First\Incl-child-\Pl{} \E-food for  <\Third>do.something.with=\AAnd.\Ds.\Third.\Fut{} \Third-from \Third-offal=one eat.\Pl \\
\glt `{}``Go quickly and do something with it for our children's food and they'll eat some offall!''{}' \\
     `{}``Se apresse e faça alguma coisa para a comida das nossas crianças e aí eles vão comer alguns miúdos!''{}' \\
\z

% juntei duas frases da transcrição e resegmentei
\ea  ``Hwararo, wakhrajê re hrãm khãm sãm nhy ire ithẽm ne ire ngrytxi pĩrĩ ne tho imorõ wyráká.'' \\[.3em]
\gll hwararo   [  wa-khra-jê=re                  ∅-hrãm-∅              khãm s-ã-m=nhy                                 ire           i-thẽ-m=ne                       ire           ngry-txi  pĩ-rĩ=ne                    t<h>o        i-mo-rõ               ]  wyráká \\
     yesterday {} \First\Incl-child-\Pl{}=\Erg{} \Third-desire-\Nmlz{} in   \Third-be.standing.\Sg-\Nmlz{}=\AAnd.\Ds{} \First.\Erg{} \First-go.\Sg-\Nmlz{}=\AAnd.\Ss{} \First.\Erg{} beast-big kill.\Sg-\Nmlz{}=\AAnd.\Ss{} <\Third>with \First-go.\Pl-\Nmlz{} {} happen \\
\glt `{}``It happened that yesterday our children were hungry, I went, killed this big beast and brought it all the way here.''{}' \\
     `{}``Aconteceu que ontem nossos filhos estavam com fome e eu fui, matei um animal grande e trouxe até aqui.''{}' \\
\label{exe:ergabsemb}
\z

\ea  ``Rik athaj aj khwã khukhrátá thõ kê aj khukhrẽ.'' \\[.3em]
\gll rik     athaj aj    kh-wã     k<h>ukhrátá\ thõ=kê                              aj    khu-khrẽ         \\
     quickly there \Pl{} \Third-to <\Third>do.something.with=\AAnd.\Ds.\Third.\Fut{} \Pl{} \Third-eat.\Pl{} \\
\glt `{}``Quick! Go there do something with it for them and so they will eat it all''{}' \\
     `{}``Rápido! Vão lá fazer alguma coisa com ele e então eles vão comer tudo''{}' \\
\z

\ea  ``Ngrytxi khêrê\ldots{} Amne thẽn sõmu.'' \\[.3em]
\gll ngry-txi  khêrê  amne    thẽ=n             s-õmu      \\
     beast-big not.be to.here go.\Sg=\AAnd.\Ss{} \Third-see \\
\glt `{}``There is no big beast\ldots{} Come here and see it.''{}' \\
     `{}``Não tem animal grande nenhum\ldots{} Venha aqui e veja.''{}' \\
\z

\ea  Sãm thãmthã, tê, norõ thãmthã khatho. \\[.3em]
\gll s-ã-m                          thãmthã        tê   ∅-no-rõ                 thãmthã        k<h>atho               \\
     \Third-be.standing.\Sg-\Nmlz{} last.long.time oops \Third-lie.down-\Nmlz{} last.long.time <\Third>come.out.\Sg{} \\
\glt `He was standing for a long time, oops, I mean, he was lying down for a long time and then came out.' \\
     `Ele ficou em pé muito tempo, quero dizer, ele ficou deitado muito tempo e então saiu.' \\
\z

\ea  Swârâ thẽn, ``the!'' Khôsátxi khre kahrãn tên\ldots{} khẽn wit nhy. \\[.3em]
\gll ∅-swârâ        thẽ=n             the khôsátxi khre       kahrã=n          tên          khẽn wit  nhy              \\
     \Third-towards go.\Sg=\AAnd.\Ss{} hey basket   inner.side probe=\AAnd.\Ss{} unexpectedly rock only be.sitting.\Sg{} \\
\glt `He went towards her and (said) ``Hey!'' He had probed the inner side of the basket and there were unexpectedly only rocks.' \\
     `Ele foi para junto dela e (disse) ``Oras!'' Ele tinha vasculhado o interior do cesto e inesperadamente só tinha pedras.' \\
\z

\newpage 
\ea  ``Mãn khẽn wit khrĩ!'' \\[.3em]
\gll mãn  khẽn wit  khrĩ             \\
     here rock only be.sitting.\Pl{} \\
\glt `{}``There are only rocks sitting here!''{}' \\
     `{}``Só tem pedras aqui!''{}' \\
\z

\ea  Sõmun ndo tan, ``Ty a\ldots{} tharãm na ngrytxi thẽ!'' nen, \\[.3em]
\gll s-õmu-n            ndo  ta=n                 ty\ a      tharãm=na         ngry-txi  thẽ      ne=n             \\
     \Third-see-\Nmlz{} with stand.\Sg=\AAnd.\Ss{} goddamn.it long.ago=\N\Fut{} beast-big go.\Sg{} do.so=\AAnd.\Ss{} \\
\glt `He stood there looking and said ``Goddamn it! It was long ago that the big beast went away!''{}' \\
     `Ele ficou ali olhando e disse ``Desgraça! Faz tempo que o animal grande foi embora!''{}' \\
\z

\ea  ``Nhum nda iwê awythárá mã?'' nen khô pyn, akhum atán thẽ. \\[.3em]
\gll nhum=nda   i-wê        a-wythá-rá              mã             ne=n             ∅-khô       py=n                 akhum atá=n                thẽ      \\
     who=\Nom{} \First-from \Second-protect-\Nmlz{} be.forthcoming do.so=\AAnd.\Ss{} \Third-club fetch.\Sg=\AAnd.\Ss{} again enter.\Sg=\AAnd.\Ss{} go.\Sg{} \\
\glt `{}``Who is going to protect you from me?'' he said, fetched his club, entered (the forest) again and went (after the forefather).' \\
     `{}``Quem que vai proteger você de mim?'' ele disse, pegou sua bordina, entrou (na floresta) de novo e foi (atrás do antepassado).' \\
\z

\ea  Akhum atán thẽn\ldots{} kôre pĩrĩ tá mã khathon \\[.3em]
\gll akhum atá=n                thẽ=n-\ldots{}             [  kôre          ∅-pĩ-rĩ                 ]  tá    mã k<h>atho=n                      \\
     again enter.\Sg=\AAnd.\Ss{} go.\Sg=\AAnd.\Ss{}-\Ints{} {} \Third.\Erg{} \Third-kill.\Sg-\Nmlz{} {} place to <\Third>come.out.\Sg=\AAnd.\Ss{} \\
\glt `Entered again, went back all the way and came out to the place where he wounded him (the forefather) unconscious,' \\
     `Entrou de novo, voltou o caminho todo e saiu no lugar onde ele tinha machucado ele (o ancestral) mortalmente,' \\
\z

\newpage 
\ea  ``Ry\ldots{}a\ldots{}'' khô ro anhi kathwân sãm thãmthã. \\[.3em]
\gll ry\ldots{}a\ldots{} ∅-khô       ro   anhi kathwâ=n              s-ã-m                          thãmthã        \\
     argh\ldots{}        \Third-club with self straighten=\AAnd.\Ss{} \Third-be.standing.\Sg-\Nmlz{} last.long.time \\
\glt `{}``Argh\ldots{}'' He straightened himself with the club (against the ground) and stood there for a long time.' \\
     `{}``Argh\ldots{}'' Ele se endireitou com a borduna (contra o chão) e ficou lá de pé por muito tempo.' \\
\z

\ea  Jowi anhi tho? nenhy amu sarẽn nda mbaj khêrê, Khupyry. \\[.3em]
\gll jowi        anhi tho    ne=nhy           amu     s-arẽ-n=nda                      mba-j        khêrê  Khupyry       \\
     as.they.say self do.how be.so=\AAnd.\Ds{} farther \Third-talk.about-\Nmlz{}=\Det{} know-\Nmlz{} not.be howler.monkey \\
\glt `How is it, as they say? It was so and I don't know what else to say about this, Howler Monkey.'\footnotemark \\
     `Como que é, que eles dizem? Aconteceu assim e eu não sei dizer mais nada sobre isso, Guariba.' \\
\footnotetext{The narrator is talking to the linguist, whom the Kĩsêdjê nicknamed \textit{Khypyry}.}
\z

\ea  Itha wit mẽ ra, tho nen sarẽn wit na wa khumba. \\[.3em]
\gll itha wit  mẽ=ra         t<h>o        ne=n             s-arẽ-n                   wit=na        wa            khu-mba     \\
     this only people=\Nom{} <\Third>with do.so=\AAnd.\Ss{} \Third-talk.about-\Nmlz{} only=\N\Fut{} \First.\Nom{} \Third-know \\
\glt `I know only this, that people used to tell this way.' \\
     `Eu só sei isso, que o pessoal costumava contar desse jeito.' \\
\label{exe:nasecpos}
\z

\ea  Ipãm nda nen tho sujarẽn nda ne wa khumba. \\[.3em]
\gll i-pãm=nda            ne=n             t<h>o        s-ujarẽ-n=nda              ne=wa                   khu-mba     \\
     \First-father=\Nom{} do.so=\AAnd.\Ss{} <\Third>with \Third-talk-\Nmlz{}=\Det{} do.so=\AAnd.\Ds.\First{} \Third-know \\
\glt `My father narrated this way and I used to hear it thus.' \\
     `Meu pai contava assim e eu costumava escutar (a história) desse jeito.' \\
\z

\newpage 
\ea  Nenhy khátpy ra nen mẽ thõj pĩn khuthun tho mo nhy, akhum khwê thẽ. \\[.3em]
\gll ne=nhy           khátpy=ra      ne=n             mẽ=thõj    pĩ=n                khu-thu=n                      t<h>o        mo=nhy              akhum kh-wê       thẽ      \\
     be.so=\AAnd.\Ds{} monster=\Nom{} do.so=\AAnd.\Ss{} people=one kill.\Sg=\AAnd.\Ss{} \Third-load.on.back=\AAnd.\Ss{} <\Third>with go.\Pl{}=\AAnd.\Ds{} again \Third-from go.\Sg{} \\
\glt `It was so, and so the \textit{khátpy} monster wounded one of our people unconscious, loaded him on his back and took him far away, and he (the forefather) ran away from him.' \\
     `Foi desse jeito, e assim que o monstro \textit{khátpy} feriu mortalmente um de nós, botou ele nas costas, carregou ele muito longe, e ele (o ancestral) fugiu dele.' \\
\z

\ea  Nenhy ne. \\[.3em]
\gll ne=nhy           ne    \\
     be.so=\AAnd.\Ds{} be.so \\
\glt `Then it was like this.' \\
     `Aí foi assim.' \\
\z

%)))
\section*{Non-standard abbreviations}

\begin{tabularx}{.45\textwidth}{lX}
\&             & clause coordination \\
\textsc{ds}    & different subject \\
\textsc{e}     & epenthetic morpheme \\
\textsc{intj}  & interjection \\
\end{tabularx}
\begin{tabularx}{.45\textwidth}{lX}
\textsc{ints}  & intensification \\
\textsc{ontp}  & onomatopoeia \\
\textsc{ss}    & same subject \\
\\
\end{tabularx}

%)))
{\sloppy
\printbibliography[heading=subbibliography,notkeyword=this]
}
\end{document}
