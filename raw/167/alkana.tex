\documentclass[output=paper,
modfonts,nonflat
]{langsci/langscibook} 
\author{Joshua Birchall\affiliation{Museu Paraense Emílio Goeldi}%
\and Hein van der Voort\affiliation{Museu Paraense Emílio Goeldi}%
\and Luiz Aikanã%
\lastand Cândida Aikanã%
}%
\title{Aikanã}
\lehead{J.\ Birchall, H.\ van der Voort, Luiz Aikanã \& Cândida Aikanã}
\ourchaptersubtitle{Eruerazũ kyã’apa’i}
\ourchaptersubtitletrans{‘The story of Fox’}  
% \abstract{noabstract}
\ChapterDOI{10.5281/zenodo.885273}

\maketitle

\begin{document}

\section{Introduction} 

The story of Fox is a myth told by different peoples of the southwestern Amazon, including the Aikanã. The Aikanã people speak an isolate language, which, with about 225 speakers out of an ethnic group of around 560, is to be considered seriously endangered. The speakers of Aikanã live in two different indigenous reserves and in several towns and villages in southeastern Rondônia, Brazil, surrounded by the deforested lands of big cattle ranchers and soy farmers. The Aikanã represent the majority ethnic group in the Tubarão-Latundê reserve, which is shared with two minority populations: the Kwaza (\textsc{isolate}) and the Latundê (\textsc{northern nambikwara}). 
Several mixed Aikanã and Kwaza families live in another nearby reserve called Kwazá do Rio São Pedro. 
Although the Aikanã language is still passed on to members of the youngest generations in both indigenous reserves, knowledge of the oral and musical traditions is disappearing rapidly. In addition to the two reserves in southeastern Rondônia, there are two other reserves in the southwest and the north of Rondônia where Aikanã populations live together with other ethnic groups. The Aikanã language is neither used nor remembered in those reserves, which are located far from traditional Aikanã lands. 

\begin{figure}[t]
% \includegraphics[width=\textwidth]{figures/aikana-kwaza-map}


\includegraphics[width=\textwidth]{figures/aikana-kwaza-map-mod}
  \caption{The indigenous reserves where the Aikanã live, shown in yellow.}
\end{figure}

The Aikanã language is morphologically highly complex. Most of this complexity concerns the verb, whereas fewer morphemes are used exclusively on nouns. However, due to the availability of highly productive nominalization strategies, nouns can also be morphologically complex. Aikanã has a great number of classifier and directional-like suffixes, several valency-changing suffixes, and suffixes marking tense, modality and aspect. Frequently occurring sequences of bound morphemes may become fixed with a derived meaning that is related to that of the constituent parts.\footnote{When verb roots and suffixes enter into such a bond, person markers sometimes may intervene and therefore occur as infixes in the morphemic analysis (e.g.: lines 49, 53, 82, 108).} There is a wealth of main clause and adverbial clause mood suffixes, and extensive clause chains can be built using a switch reference marking system similar to that of Kwaza. The person marking system involves several inflectional paradigms for subject, object, beneficiary, and reflexive functions. There are different paradigms for subject marking — some of them suffixing and one prefixing — depending on verbal classes that are not yet fully understood. Third person subjects are often unmarked. Aikanã displays a basic distinction between future and unmarked non-future tense, but there are additional past and remote future tenses. Future tense and desiderative modality canonically involve double person marking: a  person marker at the end of the verb stem, just before the mood inflection, and a person marker adjacent to the verb root, which is obligatorily a first person singular or plural, expressing an embedded perspective similar to that of quotation \citep{VanderVoort2013,VanderVoort2016}. Although the language isolates of Rondônia, namely Aikanã, Kanoé and Kwaza, display several similar lexical and grammatical traits, there is no compelling evidence that they should be considered genealogically related \citep{hvw:vanderVoort:Kwaza}.

The following story was told by Luiz Aikanã during his visit in June 2013 to the Museu Paraense Emílio Goeldi in Belém.
The story was recorded in audio and video formats as part of a documentation project funded through the DoBeS programme. After recording, the story was transcribed, analyzed and translated into Portuguese with Luiz's help. Luiz was born in 1952 and learned this and various other stories from his grandmother, \textit{Kwã’ĩ}. He has been living and working on the Tubarão-Latundê reserve since it was officially settled in 1973. Because of his knowledge, experience and interest, he is one of the principal sources of information on Aikanã language and culture. Additional consultation on the analysis involved Cândida Aikanã, who is a native speaker of the Aikanã language with full command of Portuguese and is also a member of the DoBeS project team.
 
The story of Fox takes place in mythological times, when animals transformed into humans at will. As in the Kwaza story of Grandfather Fox, Fox in the Aikanã story is very smart and knows how to trick people. Also similar to the Kwaza story, Fox leads a young woman astray (and in this case  her younger sister as well)  after having found out about her plans for the next day. And again, the lesson of the story is that one should avoid speaking about one’s plans for the future because that will attract adversity. The Aikanã story is quite different from the Kwaza one in several respects, but it similarly conveys this warning on the danger of talking about the future, which can be considered a taboo that still forms part of the present way of life of the Aikanã people despite the enormous changes that they and the other indigenous peoples of Rondônia have undergone during the 20th century.

The story is presented with a rather broad phonetic transcription on the first line, and is then segmented phonologically and morphologically on the second line. The third line contains the glosses and the fourth and fifth lines contain free translations in English and Portuguese. 
It is worth pointing out that description of the Aikanã language is still ongoing and the analysis presented here will be further refined as this work continues to progress. Aikanã has had a native writing tradition since the late 1980s when an orthography was developed by missionaries. This orthography is used with varying success at the schools on the reserve, in Bible translation, in a recent dictionary by \citet{Silva2013} and in the present text. The <s> usually corresponds to IPA [ts], the <x> corresponds to [tʃ], the <y> corresponds to [j], the <z> often corresponds to [ð], and the <'> corresponds to [ʔ]. Vowels following a nasal consonant are usually nasalized, but this is not marked in the orthography used here. The central vowel [ɨ] and its nasal counterpart [ɨ̃] are allophones of the phonemes /a/ and /ã/, respectively. They occur only before an [i], but since they are part of the existing orthography they are preserved here. 

% Interesting because it contains a lot of verbs related to now-out-of-use shamanic practices, blow, pray, cure, transform, etc...  

% Silva et al. is from 2013. Please change this!


\section{Eruerazũ kyã’apa’i}

\translatedtitle{‘The story of Fox’}\\
\translatedtitle{‘A História do Raposa’}\footnote{Recordings of this story are available from \url{https://zenodo.org/record/885240}}
\ea   hisa xüxü xüxüwe Kwã'ĩ kyã'arisukudiweye kyãkarekaẽ \\[.3em]
\gll hisa xüxü xüxüwe Kwã'ĩ kyã-are-isu-ku-diwe-ye kyã-ka-re-ka-ẽ\\      1\textsc{sg}  \textsc{1sg.poss} grandmother Kwã'ĩ speak-poor-\textsc{rem.pst-1sg.ben-pst.nmlz-obj} speak-\textsc{1sg-fut-1sg-decl} \\

\glt `I am going to tell a story my grandmother \textit{Kwã'ĩ} told me.' \\
`Eu vou contar o que minha avó \textit{Kwã'ĩ} contava pra mim.'\\
\z

\newpage
\ea   hena detyamɨi namɨi hiku'ete kuka'i'ete wareyü̃pü \\[.3em]
\gll he-na detya-mɨi namɨi hiku-ete kuka-i-ete ware-yü̃-pü\\
then-\textsc{ds} woman-\textsc{dim} cousin other-\textsc{all} tell-\textsc{nmlz-all} go-\textsc{dir:}close-\textsc{ss}\\
\glt    `Once a young woman went to talk with her cousin.' \\
`Daí uma moça foi falar com a prima dela.'
\z

\ea   derinena hikiri'ikana wikere axawapata'ẽ kukaẽ \\[.3em]
\gll deri-ne-na hikiri-'ika-na wikere a-xa-wa-pa-ta-'ẽ kuka-ẽ \\ 
light-\textsc{pfv-ds} dark-\textsc{intens-ds} peanut uproot-\textsc{1pl}-\textsc{dir:}upwards-\textsc{tr}-\textsc{fut}-\textsc{imp} tell-\textsc{decl} \\
\glt   `{``}Let's go dig up peanuts early tomorrow morning early,'' the cousin said.' \\
`{``}Vamos lá arrancar amendoim amanhã cedo,'' falou a prima.'
\z

\ea  hena kadupɨi kaxata'ereĩ \\[.3em]
\gll he-na kadupɨi ka-xa-ta-'ere-'ẽ\\
then-\textsc{ds} alright do-\textsc{1pl-rem.fut-hort-imp}\\
\glt   `{``}OK, let's do it,'' (the girl replied).' \\
`{``}Está bem, vamos fazer,'' (a moça respondeu).'\\
\z

%EX5

\ea   hena hepü ka'yareyada eruera anapayü̃zadeare \\[.3em]
\gll  he-na he-pü ka-'ya-re-yada eruera anapa-yü̃za-de-are\\
then-\textsc{ds} say-\textsc{ss} \textsc{1sg}-come-\textsc{fut-reas} fox listen-\textsc{dir:}next-\textsc{dir:}outside-\textsc{infr}\\
\glt    `{``}Then I'll come back,'' (she replied), but Fox was listening through the wall.' \\
`{``}Então vou voltar,'' (ela respondeu), mas o Raposa estava escutando elas através da parede.' \\
\z          

%-re-yada is a kind of exhortative of reason?, like -re?e, and therefore doesn´t need a second person marker? Or should it be translated as "therefore she was coming"?

\ea   anapayü̃zadepü kãwãyada  hikiri'ikana \\[.3em]
\gll anapa-yü̃za-de-pü kãwã-yada  hikiri-ika-na\\
listen-\textsc{dir:}next--\textsc{dir:}outside-\textsc{ss} be.like-\textsc{reas} dark-\textsc{intens-ds}\\
\glt   `He was listening from behind the wall, it was very dark.' \\
`Ele estava escutando atrás da parede bem de manhã cedo.' \\
\z
%JB: -de 'outside' instead of dii? 

\newpage
\ea   derinena hikiri'ikana mẽyãpü tawĩmeata'ẽ kukaẽ \\[.3em]
\gll deri-ne-na hikiri-'ika-na mẽ-yã-pü tawĩ-me-a-ta-'ẽ kuka-ẽ\\
light-\textsc{pfv-ds} dark-\textsc{intens-ds} 2\textsc{sg}-come-\textsc{is} await-\textsc{2sg-1sg-fut-imp} tell-\textsc{decl}\\
\glt    `{``}So you come get me early morning tomorrow!'' the girl said.' \\
 `{``}Então você me chama amanhã cedo!'' a moça falou.' \\
\z
%the verb tãwĩ- 'await' is also used in the sense of to 'call, call forth' and may originate from the interrogative combination tãwã-ĩ 'what, when, why'. (tãwã- used as a verb means 'what do, why do'.) 

\ea   	hepü hukadupɨi hepü xünehepü \\[.3em]
\gll 	he-pü hukadupɨi he-pü xüne-he-pü\\
then-\textsc{ss} alright then-\textsc{ss} return-\textsc{3sg-ss}\\
\glt 	`{``}OK,'' the cousin said and left.' \\
`{``}OK,'' a prima falou, e foi embora.'

\z

\ea   hena zune iriane \\[.3em]
\gll  he-na zune iriane \\
then-\textsc{ds} night middle\\
\glt  `Then in the middle of the night ...' \\
`Daí no meio da noite...' \\
\z

%EX10
\ea   derikanerena mɨitü ɨitüderine \\[.3em]
\gll deri-ka-ne-re-na mɨitü ɨitü-deri-ne\\
day-\textsc{1sg-pfv-fut-ds} only be.different-\textsc{nmlz}-\textsc{emph}\\
\glt `It was going to be dawn soon.' \\
`Estava querendo amanhecer ainda.' \\
\z
%JB: miitu 'weak' 'watered down' 
%JB: split zamiya
%JB: 
% \ea   derikanere zamiyamɨitü ixü derine \\[.3em] ORIGINAL
%\gll deri-ka-ne-re zamiya-mɨitü-ixü deri-ne\\
%    day-\textsc{1sg-pfv-fut} now-only-X day-\textsc{pfv}\\


\ea  yã'i eruera apa'ixüte \\[.3em]
\gll yã-i eruera apa-ixüte\\
come-\textsc{nmlz} fox say-\textsc{rep}\\
\glt   `Fox came, they say.' \\
`O Raposa veio, disseram.'  \\
\z

\ea  eruera yãpü \\[.3em]
\gll eruera yã-pü\\
fox come-\textsc{ss}\\
\glt  `Fox came.' \\
`O Raposa veio.'
\z

\newpage
\ea   namɨi namɨi kukana hãw heẽ \\[.3em]
\gll namɨi namɨi kuka-na hãw he-ẽ \\
cousin cousin tell-\textsc{ds} huh say-\textsc{decl}   \\
\glt  `{``}Cousin, cousin!'' he called. ``Yes?'' she responded.' \\
`{``}Prima, prima!'' ele chamou. ``Sim?'' ela respondeu.'
\z

\ea   yãw'ẽ wikere xü'iaxanapetaka'ĩwãte kukaẽ \\[.3em]
\gll yãw'ẽ wikere xü'i-a-xa-nape-ta-ka-'ĩwã-te kuka-ẽ\\
let's.go.\textsc{imp} peanut dig-uproot-\textsc{1pl-dir:}forest-\textsc{rem.fut}-\textsc{clf}:pieces-\textsc{admon-pst} tell-\textsc{decl}\\
\glt    `{``}Let's go digging up peanuts as planned,'' he told her.' \\
`{``}Vamos arrancar amendoim como combinamos,'' ele falou para ela.' \\
\z
%-iwã-te habitual? other modal/aspectual category?
%JB: why separate the iwate? let's go digging up peanuts
%\ea   yãw'ẽ wikere xü'ixanapetaka'ĩwãte kukaẽ \\[.3em] (HV original)
%\gll yãw'ẽ wikere xüi a-xa-nape-ta-ka-'ẽ ĩwãte kuka-ẽ\\
%    let's.go.\textsc{imp} peanut dig uproot-\textsc{1pl-dir:}forest-%\textsc{fut-X-imp} X tell-\textsc{decl}\\


\ea   erünuna hikiri'ika'iwã hikiri'ikaẽ \\[.3em]
\gll erünuna hikiri-ika-iwã hikiri-ika-ẽ \\
\textsc{expl} dark-\textsc{int-admon} dark-\textsc{int-decl}\\
\glt    `{``}But damn, it's still dark outside, really dark,'' (she replied).' \\
`{``}Mas poxa, ainda está escuro lá fora, bem escuro,'' (ela respondeu).' \\
\z

\ea   tawãxapü kaxatɨi kukaẽ \\[.3em]                                
\gll tawã-xa-pü ka-xa-ta-i kuka-ẽ \\
what-\textsc{1pl-ss} do-\textsc{1pl-fut-int} tell-\textsc{decl}\\
\glt    `{``}Why do we have to go now?{''}' \\
`{``}Porque temos que fazer agora?{''}' \\
\z

\ea  hinaẽ derinedupa kawaẽ  \\[.3em]
\gll hina-ẽ deri-ne-dupa kawa-ẽ \\
no-\textsc{decl} light-\textsc{pfv-conc} be-\textsc{decl}\\
\glt    `{``}No, dawn is almost here.{''}' \\
`{``}Não, está clareando já.{''}' \\
\z

\largerpage
\ea   kawãte izata'ẽ kapü kadiẽ iza'idepeta'ẽ kapü kadiẽ hẽ \\[.3em]
\gll kawãte iza-ta-'ẽ ka-pü kadi-ẽ iza-idepe-ta-'ẽ ka-pü kadi-ẽ he-ẽ \\
because far-\textsc{rem.fut-decl} do-\textsc{ss} affirm-\textsc{decl} far-\textsc{dir:}garden-\textsc{rem.fut-decl} do-\textsc{ss} affirm-\textsc{decl} say-\textsc{decl}\\
\glt    `{``}It's because the garden is far away, really far away,'' he said.' \\
`{``}É porque a roça fica longe, bem longe mesmo,'' ele falou.'  \\
\z

 
\ea   hena mamaderi hürüwanipü \\[.3em]
\gll 	he-na mama-deri hürü-wa-ne-pü\\
then-\textsc{ds} mother-\textsc{3.poss} rise-\textsc{dir:}up-\textsc{pfv-ss}\\
\glt `Then her mother awoke.' \\
`Daí a mãe dela acordou.' \\
\z

\ea   hena hikiri'ika'iwã tãwãmeapü kameazati \\[.3em]
\gll  he-na hikiri-ika-iwã tãwã-mea-pü ka-meaza-ti \\
then-\textsc{ds} dark-\textsc{intens-admon} what-\textsc{2pl-ss} do-\textsc{2pl-fut.int}\\
\glt    `{``}But it's still dark out, why are you going?{''} she said.' \\
 `{``}Mas está escuro ainda. Por que vocês vão fazer agora?'' ela falou.'
\z

\ea   izaderineipita'ẽ eyedupa kukadupa \\[.3em]
\gll iza-deri-ne-i-pita-'ẽ eye-dupa kuka-dupa\\
far-light-\textsc{pfv-nmlz-proc-imp} \textsc{3pl.obj-conc} tell-\textsc{conc}\\
\glt    `{``}Let the sun come up first,'' she told them, but...' \\
 `{``}Deixa clarear mais,'' ela falou pra elas, mas...'\\
\z

%pita procrastinative?

\ea   hinaẽ kapü derinedupa kawaẽ eyepü \\[.3em]
\gll  hina-ẽ ka-pü deri-ne-dupa kawa-ẽ eye-pü\\
no-\textsc{decl} do-\textsc{ss} light-\textsc{pfv-conc} be-\textsc{decl} \textsc{3pl.obj-ss}\\
\glt    `{``}No, it's already dawn,'' Fox said to them.'' \\
 `{``}Não, está clareando já,'' Raposa falou para elas.''
\z

\ea   hedupana purikɨi'eneke bubu'he'ĩwã tãwãxeapü warexatɨi kukaẽ \\[.3em]
\gll  he-dupana purikɨi-'ene-ke bubu-'he-'ĩwã tãwã-xea-pü ware-xa-ti kuka-ẽ \\
say-\textsc{temp} flute-\textsc{col-also} dance-\textsc{3-admon} what-\textsc{1pl-ss} go-\textsc{1pl-fut.int} tell-\textsc{decl}\\
\glt   `But when he said that, the girl said, ``The musicians are also dancing, how shall we get past?{''}'\footnote{At this point, men are still playing flutes and dancing, which lasts all night. In accordance with traditional custom, women are not allowed to witness the event and see or even hear the flutes, which are sacred.} \\
 `Mas na hora que ele falou isso, a menina respondeu, ``Os músicos estão dançando ainda, como é que vamos passar?{''}' \\
\z

\newpage 
 
\ea   hinaẽ kapü xarükanapɨire'ẽ \\[.3em]
\gll hina-ẽ ka-pü xa-rüka-napai-re-'ẽ \\
no-\textsc{decl} do-\textsc{ss} \textsc{1pl-dir:}around-\textsc{clf:}forest-\textsc{fut-imp}\\
\glt   `{``}No, we'll go around them through the brush (behind the house).{''}' \\
 `{``}Não, vamos desviar pelo mato (atrás da casa).{''}' \\
\z

\ea   üre'apa'ine xarükanapɨire'ẽ kukaẽ \\[.3em]
\gll üre-apa'i-ne xa-rüka-napa-ire-'ẽ kuka-ẽ\\
hide-\textsc{act.nmlz-loc} \textsc{1pl-dir:}around-\textsc{clf:}forest-almost\textsc{-imp} tell-\textsc{decl}\\
\glt  `{``}We will sneak around them,'' said Fox.' \\
`{``}Vamos desviar eles escondidos,'' falou o Raposa.'\\
\z

\ea  tãwãmeapü waremea'ĩ kukaẽ \\[.3em]
\gll tãwã-mea-pü ware-mea-'ĩ kuka-ẽ\\
what-\textsc{2pl-ss} go-\textsc{2pl-int} say-\textsc{decl}\\
\glt `{``}How is it that you came?'' the girl said.' \\
`{``}Como é que você veio?'' ela falou.'\\
\z

\ea   baba hapükemukahana kãyãpü kayaẽ \\[.3em]
\gll baba hapü-ke-muka-a-na ka-yã-pü ka-yã-ẽ\\
father hold-\textsc{3-clf:}eye-\textsc{1sg.obj-ds} \textsc{1sg}-come-\textsc{ss} \textsc{1sg}-come-\textsc{decl}\\
\glt    `{``}Father covered my face, and then I came,'' (Fox said).'\\
`{``}Meu pai segurou meu rosto, daí eu vim,'' (Raposa falou).'\\
\z

\ea   hukadupɨi kukaẽ \\[.3em]
\gll hukadupɨi kuka-ẽ \\
alright tell-\textsc{decl}\\
\glt  `{``}Alright,'' she said.' \\
`{``}Tudo bem,{''} ela falou.' \\
\z

\ea  kadupɨi kaxare'ereĩ hepü hena \\[.3em]
\gll kadupɨi ka-xa-re-'ẽ he-pü he-na\\
alright do-\textsc{1pl-fut-imp} say-\textsc{ss} then-\textsc{ds}  \\
\glt    `{``}Alright, then let's go,'' she said, and then ...'\\
`{``}Tudo bem, então vamos,'' ela falou, e daí ...'\\
\z

%the -ĩ may be just [hĩ], which is apparently a meaningless filler that Luiz uses a lot (perhaps a tag question hina'ĩ "isn´t it?").
%we alter between hena 'say' and hena 'then'


\newpage 
\ea   mamaderi kãwãyada tanimɨiye keapa'ẽĩ  deri zãme mama'ĩ diarekaẽ \\[.3em]
\gll mama-deri kãwã-yada tanimɨi-ye kea-pa-'ẽ deri zãme mama'ĩ dia-re-ka-ẽ\\
mother-\textsc{3.poss} be.like-\textsc{reas} sister-\textsc{obj} get-\textsc{tr-imp} light today chicha \textsc{1sg-fut-1sg-decl} \\
\glt    `Her mother said, ``In that case, bring your little sister. I am going to make \textit{chicha} today.{''}'\footnote{In Rondônia, this is usually a lightly alcoholic drink based on boiled fermented maize, manioc or yams, which is prepared by women. Elsewhere it is also known as \textit{kashiri} or \textit{caxiri}.},\footnote{The form \textit{deri} is a false start: the narrator begins to say \textit{derinena} `at daybreak' (lit. `light-\textsc{pfv-ds}') but then corrects himself, saying \textit{zãme} `today'.}\\
 `A mãe dela falou, ``Então leva sua irmãzinha, vou fazer chicha hoje.{''}'
\z

%kea- apparently also means 'to get/pick objects that are attached' (here the child, in 10:46,line 111 the larvae)
%the -ĩ as in the previous line may be just the filler [hĩ].
%deri is a slip of the tongue
%Unclear whether this is sister or brother. Some were translated irmão, others irmã
%JB: mark the hinai deri as a repair in some way? (yes, indeed)

\ea   yoayoakuka'isuwate'ẽ \\[.3em]
\gll  yoa\textasciitilde yoa-kuka-isuwate-'ẽ \\
cry\textasciitilde cry-\textsc{clf:}body-\textsc{hab-decl}\\
\glt  `{``}She is always crying.{''}' \\
`{``}Ela fica chorando.{''}'\\
\z

\ea   deü'edika'ẽ tanimɨ̃iye kukaẽ hukadupɨi \\[.3em]
\gll  deü-edika-'ẽ tanimɨi-ye kuka-ẽ hukadupɨi\\
put.heavy-\textsc{clf:}back-\textsc{imp} sister-\textsc{obj} tell-\textsc{decl} alright\\
\glt   `{``}Carry your little sister on your back,'' (Fox) told her. ``Alright,'' (she responded).'\\
`{``}Leva sua irmã nas costas,'' (Raposa) falou. ``Tudo bem,'' (ela respondeu).'\\

\z

\ea   tanimɨideriye de'edikapü yãw'ẽ \\[.3em]
\gll  tanimɨi-deri-ye deü-edika-pü yãw'ẽ \\
sister-\textsc{3.poss-obj} put.heavy-\textsc{clf:}back-\textsc{ss} let's.go.\textsc{imp}\\
\glt  `She put her little sister on her back and ``Let's go!{''} (she said).' \\
`Ela colocou a irmãzinha nas costas e ``Vamos embora!{''} (ela falou).' \\
\z

\ea  hikadepapü.  hepü hikiri'ine \\[.3em]
\gll hika-de-pa-pü.  he-pü hikiri-'i-ne\\
leave-\textsc{dir:}outside-\textsc{tr-ss} then-\textsc{ss} dark-\textsc{nmlz-loc} \\
\glt   `She went outside. It was dark.'\\
`Ela saiu pra fora. Já escureceu.'\\
 \z       

% why -pa- = trans? Not CL?


\ea   hikirihedupapü hidepɨidukaripü \\[.3em]
\gll hikiri-he-dupa-pü hidepai-dukari-pü \\
dark-\textsc{3-conc-ss} garden-\textsc{3pl-ss}\\
\glt `Although it was dark, they went to the garden.' \\
`Mesmo no escuro, eles foram pra roça.'
\z

% -dupa conces or dupa inside? Probably the -ss suffix is a slip of the tongue; it doesn´t make sense and it was not repeated in the transcription.

\ea   bate kameyẽ'ete'i erünuna iza'iza'ĩwã  kukapü iza hepü \\[.3em]
\gll bate ka-me-yẽ-'ete-'i erünuna iza\textasciitilde iza-ĩwã  kuka-pü iza-he-pü \\
where be-\textsc{2sg-dub-all-int} wow! far\textasciitilde far-\textsc{admon} tell-\textsc{ss} far-\textsc{3sg-ss}\\
\glt    `{``}But where is it?'' asked the girl. ``Wow, it's still far away. It's far,'' he replied.' \\
`{``}Mas onde que é?'' ela perguntou. ``Nossa, está longe ainda. Fica longe.'' ele respondeu.'
%me-eye-i?
%?Where? you said to them.???
\z

\ea   iza kukapü hete'i hũka kukaẽ   \\[.3em]
\gll  iza kuka-pü he-te'i hũka kuka-ẽ \\
far tell-\textsc{ss} say-\textsc{emp} in.vain tell-\textsc{decl}\\
\glt   `{``}Didn't I tell you it was far?'' he said, fooling them.'\\ 
`{``}Já não falei que fica longe?'' falou, enganando elas.' \\
%HV hypothesis: he-te'i hũka- (CHK the construction; a problem is the lacking person marking here)
\z

\ea   zamiya wãderi'ete wareyü̃papü tikiri dürü'i'ete \\[.3em]
\gll zamiya wã-deri-'ete ware-yü̃-pa-pü tikiri dürü-'i-'ete\\
now live-\textsc{nmlz-all} go-\textsc{clf:}close-\textsc{tr-ss} mound sit-\textsc{nmlz-all} \\
\glt   `Then they arrived at a termite mound where he lives.' \\
 `Daí chegaram numa casa de cupins de barro onde ele mora.'\\
% -pa- = TR or clf:big.thing? (in line 38, 40, 41, 43)
% the nominaliser -deri also adds focus (that one which, the one where, etc.)
\z
 
\ea  hena henudu ẽ perüka'i inekapü wãheyada \\[.3em]
\gll he-na henu-du-pe-rüka-'i-ne ka-pü wã-he-yada\\
then-\textsc{ds} hole-\textsc{dir:}in-\textsc{clf:}round-\textsc{dir:}around-\textsc{nmlz-loc} do-\textsc{ss} live-\textsc{3-reas}\\
\glt `It's of course full of holes where he lives.'\\
`Está cheio de buracos onde ele mora.'\\
\z

\ea  hena hapükemukapü waredurikapaẽ \\[.3em]
\gll he-na hapü-ke-muka-pü ware-durika-pa-ẽ\\
then-\textsc{ds} hold-\textsc{3-clf:}eye-\textsc{ss} go-\textsc{dir:}inside-\textsc{tr-decl} \\
\glt   `Then he covered their eyes and entered.'\\
`Daí tampou os olhos delas e entrou.'\\
\z


\ea  waredurikapapü \\[.3em]
\gll ware-durika-pa-pü\\
go-\textsc{dir:}inside-\textsc{tr-ss}\\
\glt `They went inside.'\\
`Entraram pra dentro.'\\
\z
 
\ea  hena kɨinezũ keza ɨitühene'ena  \\[.3em]
\gll he-na kɨine-zũ keza ɨitü-he-ne'e-na\\
then-\textsc{ds} \textsc{3sg-poss} house be.different-\textsc{3-ite-ds}\\
\glt  `His house was very different.'\\
`Daí a casa dele estava diferente.'\\
%not clear to MA & RA why LU used -ne'e- 'again'
\z

%the negative expression hina is probably the filler [hĩ].

\ea   kawã kayaparehãyãpü kawãtena ãrüakukapederiame hena \\[.3em]
\gll kawã ka-ya-pa-re-hãyã-pü kawãte-na ãryüa-kuka-pe-deri-ame he-na\\
be.like \textsc{1sg}-come-\textsc{tr-fut-1pl.obj-ss} because-\textsc{ds} know-\textsc{clf:}body-\textsc{clf}:round-\textsc{nmlz-sup} then-\textsc{ds}\\
\glt    `{``}He really is leading us astray," the older and somewhat more knowledgeable girl thought.'\\
`{``}Ele falou isso só para judiar de nós,'' aquela que é mais sabida pensou.' 

%The AIK expression 'take away' and POR 'levar ao engano' are similar. Perhaps lg. contact or is there sth more universal (e.g. ENG mislead, take for a ride)?
%-(h)ame = kind of superlative 'pure, only, just' (2:115-6)
% -pe- = CL 'round'? (-kukape= 7:115)
\z

\ea   wãwã'ĩ'ikaderihame kapü hina'ĩ  hina kukaẽ \\[.3em]
\gll  wãwã'ĩ-'ika-deri-hame ka-pü hina-ĩ  hina kuka-ẽ \\
child-\textsc{intens-nmlz-sup} do-\textsc{ss} no-\textsc{nmlz} no say-\textsc{decl}\\
\glt    `The little child didn't worry at all, didn't say anything.' \\
`A criança mais nova nem se liga, não fala nada.' \\
\z

\ea   tãwãxeapü kaxa'i erükazapa'ĩ kawã dukumɨi kawãtena \\[.3em]
\gll tãwã-xea-pü ka-xa-'i erükazapa'ĩ kawã dukumɨi kawãte-na\\
what-\textsc{1pl-ss} do-\textsc{1pl-int} wow! be.like ruin because-\textsc{ds}\\
  \glt    `{``}How is it that we stopped here? Wow, it must be some spirit messing with us.{''}' \\
`{``}Como é que nós paremos aqui? Poxa, é sombração que está mexendo com a gente.{''}'
\z          

\newpage 
\ea   namɨi kayareapɨite'i \\[.3em]
\gll namɨi ka-ya-re-a-pa-i-te'i\\
cousin \textsc{1sg}-come-\textsc{fut-1sg.obj-tr-nmlz-emph}\\
\glt  `{``}I thought it was my cousin!{''}' \\
`{``}Pensei que era minha prima!{''}' \\
%CHK the whole analysis has to be checked! Probably -te'i = 'EMP cleft' as in line (37)
%kaya- = 'take away'? kaza- pasa- = abduct?
\z  

\ea   he'ẽ kapü ã'apakukaẽ \\[.3em]
\gll  he-'ẽ ka-pü ã-'a-pa-kuka-ẽ \\
say-\textsc{decl} do-\textsc{ss} think-\textsc{3sg.refl-tr-clf:}body-\textsc{decl}\\
\glt `She went on thinking and became sad.'\\
`Ela foi pensando e ficou triste.'
%discontinuous verb: ã--pakuka- 'be sad'
\z  

\ea   nake tãwã ãanaẽ kapü hina'ĩ'ẽ \\[.3em]
\gll nake tãwã ã-a-na-ẽ ka-pü hina-'ĩ-'ẽ \\
\textsc{cond} what think-\textsc{1sg.obj-neg-decl} do-\textsc{ss} no-\textsc{nmlz-decl}\\
\glt    `{``}What can I do? There is nothing.{''}'\\
`{``}Como que posso fazer? Não podemos fazer nada.{''}'\\
%maybe the construction is same as in line (51). Or is it different, with ã-'a- 'think-1sg.obj', or with -na = int / neg? Or is there an 3sg.refl or impersonal -'a- there? 
%nake seems to be a slip of the tongue
\z  

\ea   hena'ẽ zamiya hiku is'ideri dukanuẽ hepü yoa\\[.3em]
\gll hena-'ẽ zamiya hiku ise-'i-deri d-u<ka>nu-ẽ he-pü yoa\\
quiet-\textsc{imp} now other small-\textsc{nmlz-nmlz} \textsc{1sg}-hungry<\textsc{1sg}>-\textsc{decl} say-\textsc{ss} cry\\
\glt   `The little one was crying from hunger.' \\
`A outra pequena estava chorando de fome.'
%'hungry' is a 'strong' and discontinuous verb root/stem, so an additional person marker is infixed.
%-deri here adds focus 'the one who is the small one'
\z  

\ea   he'ẽ kapü  kɨine hiku ti'iweke ũnenudupa\\[.3em]
\gll  he-'ẽ ka-pü  kɨine hiku ti'iwe-ke ũ<ne>nu-dupa\\
say-\textsc{decl} do-\textsc{ss} \textsc{3sg} other grow-\textsc{com} hungry<\textsc{pfv}>-\textsc{conc}\\
\glt   `The grown one also had gotten hungry but (she held on).' \\
`A outra grande também estava com fome mas aguentou.'\\

%(grande, crescido usado como raiz nominal!)
% it sounds like tu d-ũ<ne>nu-dupa "XX 1sg-hungry<pfv>-conc". The d- may be part of a variant of u- hungry. Maybe "tu" is a slip of the tongue, or sth else.
\z  

\ea   tãwã'ãnaẽ he'ẽ yoaẽ  \\[.3em]
\gll tãwã-'ã-na-ẽ he-'ẽ yoa-ẽ \\
what-\textsc{impers-neg-decl} say-\textsc{decl} cry-\textsc{decl}\\                            
\glt  `{``}What can one do?'' she said. ``She is crying.{''}'\\
 `{``}O que pode fazer?'' ela falou. ``Ela está chorando.{''}'
% Maybe same as in line (48)? Other expressions: tãwãdianãe / tãwã'anãe = "como é que posso fazer?"
%  he'ẽ can also be an ideophone for crying.
\z  


\ea eruerazũ mamaderi \\[.3em] 
\gll eruera-zũ mama-deri \\
fox-\textsc{poss} mother-\textsc{3.poss}\\           
\glt  `Now Fox's mother (comments):' \\
`Agora a mãe do Raposa (comenta):'\\
\z  

\ea dukanuẽ ũnenuxaẽ he'ẽ yoayoaredukari'ĩwã \\[.3em]
\gll d-u<ka>nu-ẽ ũ<ne>nu-xa-ẽ he'ẽ yoa\textasciitilde yoa-are-dukari-'ĩwã \\
\textsc{1sg}-hungry<\textsc{1sg}>-\textsc{decl} hungry<\textsc{pfv}>-\textsc{1pl-decl} say-\textsc{decl} cry\textasciitilde cry-poor-\textsc{3pl-admon} \\
\glt   `{``}The poor dears are crying `I am hungry, we are hungry.{'}{''}' \\
`{``}Os coitados estão chorando `Estou com fome, estamos com fome.{'}{''}' \\

\z  

\ea   tara kawhepire'ẽ eyepü kawa'ĩ \\[.3em]
\gll tara kaw-he-pi-re-'ẽ eye-(na)-pü kawa-'ĩ\\
what eat-\textsc{3-proc-fut-imp} \textsc{3pl.obj-(neg)-ss} be-\textsc{int}\\
\glt   `{``}Why didn´t he find something for them to eat first?'' (the mother thought).' \\
`{``}Porque não procurou uma coisa pra eles comerem primeiro?'' (a mãe pensou).'\\ 
%MA/RA: it should be -eye-na-pü (15:18)
\z  

\ea   kawãnunu hãemepe'eyepü kawãte'i kukaẽ \\[.3em]
\gll kawã-nunu hãe-me-pe-'eye-pü kawãte-'i kuka-ẽ\\
be.like-\textsc{mir} grab-\textsc{2sg-clf:}round-\textsc{3pl.obj-ss} because-\textsc{int} say-\textsc{decl} \\
\glt  `{``}Why the hell did you catch them?'' she said to him angrily.' \\
`{``}Então porque você pegou elas?'' ela falou com raiva.'\\

%-pe- = CL? yes
%hãe or hãy instead of hã'i? yes
\z  

\ea   kawayada xoakarüperekaẽ urikɨi hepü hikade'ẽ \\[.3em]
\gll kawa-yada xoa-ka-rüpe-re-ka-ẽ urikɨi he-pü hika-de-'ẽ\\
be-\textsc{reas} see-\textsc{1sg-dir:}ground-\textsc{fut-1sg-decl} food say-\textsc{ss} leave-\textsc{dir:}outside-\textsc{decl}\\
\glt  `So he said ``I will look for food,'' and left.' \\
`Então ele falou ``Vou procurar comida,'' e saiu pra fora.' 
%CHK -rüpe- = lexicalised?
\z  


\newpage 
\ea hikadepü kapü wã'apaderi'ete yü̃'eyeẽ kapü \\[.3em]
\gll hika-de-pü ka-pü wã-'apaderi-'ete yü̃-'eye-ẽ ka-pü\\
leave-\textsc{dir:}outside-\textsc{ss} do-\textsc{ss} live-\textsc{act.nmlz-all} \textsc{dir:}close-\textsc{3pl.obj-decl} do-\textsc{ss}\\
\glt  `He went outside and left for where the girls' parents lived.' \\
`Ele saiu para onde o povo morava.' \\
%-'apa-i creates an action noun or result noun; it may consist of an impersonal -'a-, a transitiviser or classifier -pa- and a nominaliser (in this case the focus nominaliser -deri is used)
%yü̃ is almost pronounced as nu
\z  

\ea   tara pu'apaderiye kuraruye kapü\\[.3em]
\gll tara pu-'apaderi-ye kuraru-ye ka-pü\\
what raise-\textsc{act.nmlz-obj} chicken-\textsc{obj} do-\textsc{ss} \\
\glt   `There he got things that one raises, chickens.'  \\
`Alí ele pegou coisas que a gente cria, galinha,'
\z  

\ea   düdü pu'apa'iye kikireye hãehãekepesa'eye  \\[.3em]
\gll düdü pu-'apa'i-ye kikire-ye hãe\textasciitilde hãe<ke>pe-sa-'eye\\
parrot raise-\textsc{act.nmlz-obj} parakeet-\textsc{obj} grab\textasciitilde grab<\textsc{3sg>-mal-3pl.obj}\\
\glt   `Pet parrots, parakeets, he grabbed them from the residents.' \\
`Papagaio, periquito, ele pegou dos moradores.'\\
\z  

\ea   nasunapaẽ wãderi'ete duxüpanepü hiba'eyenakedupa kapü  \\[.3em]
\gll na-suna-pa-ẽ wã-deri-'ete du-xü-pa-ne-pü hiba-'eye-nake-dupa ka-pü\\
bring-\textsc{dir:}hither-\textsc{tr-decl} live-\textsc{nmlz-all} \textsc{dir:}in-\textsc{dir:}return-\textsc{tr-pfv-ss} give-\textsc{3pl.obj-cond-conc} do-\textsc{ss}  \\
\glt `He brought those home, and entering into their residence, gave (the food) to the children, however ...' \\
`Ele trouxe de volta pra casa e entrou dentro da residência e deu (a comida) para elas, mas...' \\
%; na-pa- 'bring' is discontinuous. 
%DIR du 'in' is used as (elliptic?) verb stem
%xüne is discontinuous?
\z  


\ea  haradukarinake  \\[.3em]
\gll hara-dukari-nake\\
not.want-\textsc{3pl-cond}\\
\glt `They really didn't want anything.' \\
 `Elas não queriam mesmo.' \\ 
 %(nake 'cond' has an emphatic connotation) 
\z 

\newpage 
\ea   	haradukarinake tãwãanaẽ kapü  \\[.3em]
\gll 	hara-dukari-nake tãwã-a-na-ẽ ka-pü\\
not.want-\textsc{3pl-cond} what-\textsc{impers-neg-decl} do-\textsc{ss}\\
\glt   `{``}They didn't want anything, now what can one do?{''}'\\
`{``}Não queriam, mas fazer o quê?{''}' \\                                   
\z   

\ea   yoahedukarina kapü kãwãẽ  zamiya mamaderi  \\[.3em]
\gll 	yoa-he-dukari-na ka-pü kãwã-ẽ  zamiya mama-deri\\
cry-\textsc{3-3pl-ds} do-\textsc{ss} be.like-\textsc{decl} now mother-\textsc{3.poss}\\
\glt  `They kept on crying as ever. And then his mother went:'  \\
`Ficaram chorando.  Agora a mãe do Raposa falou:'\\
%yoa seems to be part phonotactically of the previous utterance
\z  

\ea   tara hü'a'iye xoawe'epü tãwãmepü urikɨiiye kamezɨi  \\[.3em]
\gll 	 tara hü'a-'i-ye xoa-we'eye-pü tãwã-me-pü urikɨi-ye ka-meza-i\\
what good-\textsc{nmlz-obj} see-\textsc{3pl.ben-ss}  what-\textsc{2sg-ss} food-\textsc{obj}	do-\textsc{2sg.caus-int}\\
\glt `{``}Find them something good, you only bring bad stuff.{''}'  \\
`{``}Procure um coisa boa para elas, você só traz coisa ruim.{''}'\\
%the interrogative probably has an exclamative function expressing indignation here
\z  

%Gloss for -keza/-meza markers? Caus? Yes, I think so. (it is usually used on trans verbs that change the state of something. salgar, avermelhar, etc.)

\ea   tara dukumɨi'iye kaw'i kawã'ĩwãwã  \\[.3em]
\gll 	tara dukumɨi-'i-ye kaw-'i kawã-'ĩwã\textasciitilde wã\\
what ruin-\textsc{nmlz-obj} eat-\textsc{nmlz} be.like-\textsc{admon}\textasciitilde \textsc{red}\\
\glt  `{``}They don't eat worthless things.{''}'\footnote{Domestic(ated) animals are not eaten, even if the same animals would represent game in the wild context.} \\
`{``}Elas não comem coisa que não presta.{''}'\\
% CHK or -i-wãwã '-nmlz-seem'
\z  
 
\ea  kawã'ĩ dukumɨi'iye kawxare'ẽyaremina  \\[.3em]
\gll kawã-ĩ dukumɨi-'i-ye kaw-xa-re-'ẽ-are-mina\\
 be.like-\textsc{nmlz} ruin-\textsc{nmlz-obj} eat-\textsc{1pl-fut-imp-infr-emp.neg}\\
\glt `{``}They don't even think of eating what's worthless.{''}' \\
`{``}Nem pensam em comer aquilo que não presta.{''}' \\
\z  
 
\largerpage[2]
\ea     hü'anɨi apaduri'iwa he'ẽ kukana\\[.3em]
\gll 	hü'a-na-i apa-dukari-'iwa he-'ẽ kuka-na\\
good-\textsc{neg-nmlz} find-\textsc{3pl-admon} say-\textsc{decl} tell-\textsc{ds}\\
\glt    `{``}They are suffering,'' she said to him.' \\
`{``}Estão sofrendo,'' ela falou para ele.'
%(lit. find badness)
% CHK analysis of [duri'iwa he'ẽ]. 
%durika = duraka bottom?
\z  

 
\ea     iza hepü  \\[.3em]
\gll 	iza he-pü\\
far say-\textsc{ss}\\
 \glt     `{``}Go far away,'' she said.'  \\
 `{``}Vai longe,'' ela falou.' \\
 %vai longe, falou
\z  

\ea   tara hü'a'iye takepewe'eye'ẽ kukaẽ  \\[.3em]
\gll 	 tara hü'a-'i-ye ta-ke-pe-we'eye-'ẽ kuka-ẽ\\
what good-\textsc{nmlz-obj} shoot-\textsc{3-clf:}round-\textsc{3pl.ben-imp} tell-\textsc{decl}\\
\glt   `{``}Kill something good for them!'' she told him.' \\
 `{``}Mata coisa boa pra elas!'' ela falou para ele.' \\
%mata coisa boa pra eles
%-ke-pe- also kind of object?
\z  

\ea   kadupɨi kayapitaẽ \\[.3em]
\gll 	 kadupɨi ka-ya-pita-ẽ \\
alright \textsc{1sg}-go-\textsc{proc-decl}\\
\glt   `{``}OK, then I'll go,'' and he left.'\\
`{``}Está bem, já vou então,'' e ele foi.'\\
%está bem, eu vou então (this is not there: falou pra eles)
\z  


\ea  hikadepü iza izapa'apü wã'apa'i'ete hidüka'eye'i apa'ixüte  \\[.3em]
\gll 	hika-de-pü iza iza-pa-'a-pü wã-apa'i-ete h-idüka-eye-'i apa-'ixüte\\
leave-\textsc{dir:}outside-\textsc{ss} far far-\textsc{tr-impers-ss} live-\textsc{act.nmlz-all} \textsc{3sg-dir:}thither-\textsc{3pl.obj-nmlz} tell-\textsc{rep}\\
\glt    `He left the house and went to an inhabited place very far away, they say.' \\
`Ele saiu de casa e foi para uma moradia que fica bem longe, eles dizem.' \\
%iza-pa-a- 'one does it far'.
%wã-apa'i 'habitation' (but also 'neighbour') (lit. action noun: liv-ing)
%ele saiu de casa e foi para uma moradia que fica bem longe
%X-i apa-i-xüte 'X, they say' is a formula the etymology of which is unresolved. Often -xüte is omitted but understood.
\z  

\ea   hidüka'eyepü tarawã pu'apaderiye kapü hãekepepü\\[.3em]
\gll 	h-idüka-'eye-pü tara-wã pu-'apaderi-ye ka-pü hãe<ke>pe-pü\\
\textsc{3sg-dir:}thither-\textsc{3pl.obj-ss} what-?{\rmfnm} raise-\textsc{act.nmlz-obj} do-\textsc{ss} grab<\textsc{3sg>-ss}\\
\glt   `He went there far away and got what people were raising.' \\
`Ele foi lá longe e pegou o que o pessoal estava criando.' \\
%CHK -?ee- = -?eye- MA/RA yes he went there to other people
%-wã = slip of tongue? Or is it 'like'? MA/RA: no it´s a mistake
%the spelling hãe is a solution for [hãi] (that is not [hã'i] or [hɨ̃i])
\footnotetext{The uninterpretable form \textit{-wã} is apparently a slip of the tongue.}
\z  

\newpage 
\ea  kapü kẽrikukapeyada kẽriẽ kukayada  \\[.3em]
\gll ka-pü kẽri-kuka-pe-yada kẽri-ẽ kuka-yada\\
do-\textsc{ss} linger-\textsc{clf:}body-\textsc{?-reas} linger-\textsc{imp} tell-\textsc{reas}\\
\glt  `Obviously this took a while. It should take some time.'\\
`Obviamente demorou um pouco. É para demorar um pouco mesmo.'
%demorou um pouco
%CHK what kind of expression is this
%dia-re-na '1sg-fut-ds' was an unintended utterance so I deleted it for the printed version
\z  

\ea   ĩwã'arena kaxupane'eyãre'ẽ mamaderi hepü  \\[.3em]
\gll 	ĩwã-are-na ka-xü-pa-ne-'eyã-re-'ẽ mama-deri he-pü\\
like.that-poor-\textsc{ds} \textsc{1sg-dir:}return-\textsc{tr-pfv-2pl.obj-fut-decl} mother-\textsc{3.poss} say-\textsc{ss}\\
\glt    `{``}How unfortunate, let me bring you back!'' his mother said to the girls.' \\
 `{``}Coitadas! Eu vou levar vocês de volta!'' a mãe dele falou para as meninas.'
%como demorou, a mãe do raposa levou eles embora
%CHK (i)wã(te) everywhere. ĩwãena = 'quando é assim' See also Fátima p.192)
%notice pronunciation [xu] for /xü/, probably because of assimilation in rapid speech.
\z  
% pane 'accompany'? perhaps! I think there is no verb stem, it may be elliptic

\ea   eruerazũ mamaderi eyepü hina'ĩ hü'anɨi'apazaẽ  \\[.3em]
\gll 	 eruera-zũ mama-deri eye-pü hina-'ĩ hü'a-na-i h-apa-za-ẽ \\
fox-\textsc{poss} mother-\textsc{3.poss} \textsc{3pl.obj-ss} no-\textsc{nmlz} good-\textsc{neg-nmlz} 2-find-\textsc{pl-decl}\\
\glt   `Fox's mother said to them, ``You are suffering.{''}'\\
`A mãe do Raposa falou, ``Vocês estão sofrendo.{''}'\\
\z  

\ea   tara kawxamɨirumia'ẽ hü'ana'i hapazaẽ \\[.3em]
\gll 	tara kaw-xa-mɨiriu-mia-'ẽ hü'a-na-i h-apa-za-ẽ\\
what eat-\textsc{1pl-desi-2pl-decl} good-\textsc{neg-nmlz} 2-find-\textsc{pl-decl}\\
\glt   `{``}You want to eat and you are suffering.{''}' \\
`'{}``Vocês querem comer e estão sofrendo.{''}'\\
%hene'ĩ was a slip of the tongue
%notice how also the desiderative may employ a quotative construction, just like the future!
\z 


\ea     kaxupane'eyãre'ẽ eyepü  \\[.3em]
\gll 	 ka-xü-pa-ne-'eyã-re-'ẽ eye-pü\\
\textsc{1sg-dir:}return-\textsc{tr-pfv-2pl.obj-fut-decl} \textsc{3pl.obj-ss}\\
\glt    `{``}Let me bring you back!'' she said to them.' \\
`{``}Eu vou levar vocês de volta'', falou pra elas.'\\
\z  

\newpage 
\ea   hapükika'eyepü hikadepa'eyepü  \\[.3em]
\gll 	 hapü-ke-ika-'eye-pü hika-de-pa-'eye-pü \\
hold-\textsc{3-clf:}finger-\textsc{3pl.obj-ss} leave-\textsc{dir:}outside-\textsc{tr-3pl.obj-ss}\\
\glt   `She took them by the hand and left.' \\
`Ela segurou a mão delas e levou para fora.'\\
\z  

\ea    katemɨi nuxupane'enunu  \\[.3em]
\gll 	kate-mɨi nu-xu-pa-ne'e-nunu\\
there-\textsc{dim} come-\textsc{dir:}return-\textsc{tr-ite-mir}\\
\glt    `Now they were arriving close to home again.' \\
`Estavam chegando perto de casa.' \\
\z  

\ea   anapahidepenunu mamaderi babaderi  yoahedukariẽ \\[.3em]
\gll 	anapa-hidepe-nunu mama-deri baba-deri  yoa-he-dukari-ẽ\\
hear-\textsc{dir:}garden-\textsc{mir} mother-\textsc{3.poss} father-\textsc{3.poss} cry-\textsc{3-3pl-decl}\\
\glt    `They heard the children's mother and father crying in the garden.' \\
`Ouviram a mãe e o pai delas chorando na roça.' \\
\z  
%hide 'strong/forcibly'?

\ea   puidepena  \\[.3em]
\gll 	pu-idepe-na\\
go.\textsc{pl-dir:}garden-\textsc{ds}\\
\glt   `{``}They are walking over there ...{''}' \\
`{``}Estão andando por aí ...{''}'
\z  

\ea    ite hüridawaperekaẽ ite darüpa'eyã'ẽ eyepü \\[.3em]
\gll 	ite hüri<da>wa-pe-re-ka-ẽ ite darüpa-'eyã-'ẽ eye-pü\\
here return\textsc{<1sg.refl>-?-fut-1sg-decl} here stay.\textsc{pl-2pl.obj-decl} \textsc{3pl.obj-ss}\\
\glt    `{``}From here I will return and you stay put,'' she told them.' \\
`{``}Daqui eu vou voltar e vocês ficam,'' ela falou pra elas.'\\
\z  

\ea   hiba mama baba'i'ene yoayoahedukariẽ \\[.3em]
\gll 	hiba mama baba-'i-'ene yoa\textasciitilde yoa-he-dukari-ẽ\\
this mother father-\textsc{nmlz-col} cry\textasciitilde cry-\textsc{3-3pl-decl}\\
\glt    `Well, the mother and father were crying.'\\
`Daí a mãe e o pai estavam chorando.'
\z  

\ea   ite katemɨiyana \\[.3em]
\gll 	ite katemɨi yã-na \\
here close come-\textsc{ds} \\
\glt   `They were coming close.' \\
`A mãe e o pai estavam chegando perto delas.'\\
\z 

\ea   mama  baba memekuka'ana \\[.3em]
\gll 	mama  baba bee-me-kuka-a-na\\
mother father arrive-\textsc{2sg-clf:}body-\textsc{1sg.obj-ds}\\
\glt   `{``}(When you call them) ``Mother, father, come to me.{''}{''} (Fox's mother explained).' \\
`{``}(Quando chama eles) ``Mãe, pai, vem para cá.{''}{''} (a mãe do Raposa explicou).'\\
\z 

\ea   wareyã'ẽyãpü pane'ẽyãta'ẽ \\[.3em]
\gll 	 ware-yã-ẽyã-pü pane-ẽyã-ta-'ẽ\\
go-come-\textsc{2pl.obj-ss} bring\textsc{-2pl.obj-rem.fut-decl}\\
\glt  `{``}And when they come to you, they will take you home.{''}' \\
`{``}Quando chegarem, vão levar vocês pra casa.{''}'\\
\z  

\ea   ite hüridawaperekaẽ hepü \\[.3em]
\gll 	ite hüri<da>wa-pe-re-ka-ẽ he-pü \\
here return\textsc{<1sg.refl>-?-fut-1sg-decl} say-\textsc{ss}\\
\glt   `{``}Here I will return back,'' she said.' \\
`{``}Daqui eu vou voltar pra trás,'' ela falou.'\\
\z  

%üri (original) written hüri- above (also hüri in Fatima)

\ea   hepü daedaedɨikasa'eyena \\[.3em]
\gll 	he-pü dae\textasciitilde dae-dɨika-sa-'eye-na\\
say-\textsc{ss} walk\textasciitilde walk-\textsc{dir:}remain-\textsc{mal-3pl.obj-ds}\\                           
\glt  `She walked back behind them.' \\
`Ela voltou por trás delas.'\\
\z  

%two hepü's?

\ea  darüpaena darüpaẽ \\[.3em]
\gll 	darüpa-e-na darüpa-ẽ \\
stay.\textsc{pl}-well-\textsc{ds} stay.\textsc{pl}-\textsc{decl}\\
\glt    `They stayed there for a while.'\\
`Ficaram um tempo lá.' \\
\z  
%darüpa- 'stay, stop, sit' is only used in the plural. In the singular one would use dürü- 'sit' (which is not sg only).
%CHK -e- 'well', which is a hypothesis based on some other example somewhere else.  

\ea   zamiya babaderi mamaderi'i'ene kapü yoahe'ẽ \\[.3em]
\gll 	zamiya baba-deri mama-deri-'i-'ene ka-pü yoa-he-'ẽ\\
now father-\textsc{3.poss} mother-\textsc{3.poss-nmlz-col} do-\textsc{ss} cry-\textsc{3-decl}\\
\glt  `Then their father and mother were still crying.' \\
`Daí o pai e a mãe delas ainda estavam chorando.'\\
\z  
%H: I´ve seen this before: N N'i'ene is a sort of coordinating construction ('the mamas and the papas'?)

\ea   beeyü̃'eyena hepü xãyãrehãyãdukariẽ tãwĩhedukarina \\[.3em]
\gll 	bee-yü̃-'eye-'ẽ-na he-pü xã-yã-re-hãyã-dukari-ẽ tãwĩ-he-dukari-na\\
arrive-\textsc{dir}:near-\textsc{3pl.obj}-well-\textsc{ds} say-\textsc{ss} \textsc{1pl}-come-\textsc{1pl.obj-3pl-decl} await-\textsc{3-3pl-ds}\\
\glt    `As they were getting nearby, ``They are coming close to us,'' the children said and waited for them.' \\
`Estavam indo perto deles, daí, ``Estão chegando perto de nós,'' as crianças falaram e esperaram eles.' \\
%CHK is this extra -e- 'well then' real here? Perhaps it´s not there. Why did we put it there? (In line (89) it is real.)
\z           

\ea   hena tawĩhedupana baba baba mama mama \\[.3em]
\gll 	he-na tawĩ-he-dupana baba baba mama mama\\
then-\textsc{ds} await-3-\textsc{temp} father father mother mother\\
\glt    `Then they called out, ``Father! Father! Mother! Mother!{''}' \\
`Dai elas chamaram, ``Papai! Papai! Mamãe! Mamãe!{''}'
\z

\largerpage
\ea   erüarekũyẽi hepü xoahenunu \\[.3em]
\gll 	erüare-kũyã-i he-pü xoa-he-nunu\\
feel.sorry-\textsc{1pl.ben-nmlz} say-\textsc{ss} see-\textsc{3-mir} \\
\glt    `The parents said, ``Our poor dears!" as they saw them.' \\
`Os pais falaram, ``Nossas coitadas!'' quando viram elas.' \\
%eles falar coitado deles e viu as crianças
\z
%-kuya? eruari (fatima sp)?


\ea    darüpa'aredukarina \\[.3em]
\gll 	darüpa-are-dukari-na\\
stay.\textsc{pl}-poor-\textsc{3pl-ds}\\
\glt    `The poor kids are sitting there.' \\
`As coitadas estão lá.'
\z

 
\ea     erüarekumɨizɨi eyepü yoa'eyepü yoahepü \\[.3em]
\gll 	erüare-kuma-i-za-i eye-pü yoa-'eye-pü yoa-he-pü\\
feel.sorry-poor-\textsc{nmlz-assoc-nmlz} \textsc{3pl.obj-ss} cry-\textsc{3pl.obj-ss} cry-\textsc{3-ss}\\
\glt   `{``}You poor little things," they said to them, crying.'  \\
`{``}Coitado de vocês,'' falou para elas chorando.'\\
%(unless -mɨ̃ izei is sth. else, f.ex. an expression containing -mã 'respect' or -kumã- 'pitiful') 
\z

\ea   hepü hikuye hürükewanunu hameri \\[.3em]
\gll 	he-pü hiku-ye hürü-ke-wa-nunu hameri\\
then-\textsc{ss} other-\textsc{obj} rise-\textsc{3-dir:}up-\textsc{mir} already\\
\glt    `They lifted up one of the girls and were ready to go, but now ...' \\
`Dai levantou uma delas para ir embora, mas...'\\
\z

\ea   eruera urumekarepü hameri wãeditehe hikutehe kyã'i'apa'i \\[.3em]
\gll  eruera urume-ka-re-pü hameri wãedi te-he hiku te-he kyã-'i-'apa'i\\
fox transform-\textsc{1sg-fut-ss} already tail have-\textsc{3sg} other have-\textsc{3sg} speak-\textsc{nmlz-act.nmlz}\\
\glt   `One was changing into a fox and was already sprouting a tail, and the other one as well, that's what the story says.' \\
`Uma estava se transformando em raposa e já estava nascendo rabo, e a outra também. É assim que a história conta.'\\
\z

\ea   kawãdupa yãw'ẽ he'eyepü hapükika'eyepü \\[.3em]
\gll 	kawã-dupa yãw'ẽ he-'eye-pü hapü-ke-ika-'eye-pü\\
be.like-\textsc{conc} let's.go.\textsc{imp} say-\textsc{3pl.obj-ss} hold-\textsc{3-clf:}hand-\textsc{3pl.obj-ss}\\
\glt    `{``}Even so, let's leave,'' the parents said to them, and they held hands.'  \\
 `{``}Mesmo assim vamos embora,'' falou para elas, e segurou as mãos delas.'  \\
%
%CHK [kika]: ke is probably not the empty root ka- as in ka-ika 'hand' but may be -ke- '3(obj)'?
\z
%combined hapu with following word, as earlier
% -ika and -kika both hand?

\ea   pane'eyepü \\[.3em]
\gll 	pane-'eye-pü\\
bring-\textsc{3pl-ss}\\
\glt   `They brought them along.' \\
`Levaram elas.'\\
\z

%EX100 
\ea   keza'ete wareduxüpane'eyena kapü zamiya hena \\[.3em]
\gll 	keza-'ete ware-du-xü-pane-'eye-na ka-pü zamiya he-na\\		house-\textsc{all} go-\textsc{dir:}in-\textsc{dir:}return-bring-\textsc{3pl.obj-ds} do-\textsc{ss} now then-\textsc{ds} \\
\glt   `They entered the house with them, but ...' \\
`Entraram na casa com elas, mas...'
\z 
%entrou na casa com eles; 
%changed dü to du, xu to xü; 
%hina'ĩ is glossed as "mas/but" in FLEx (indeed, it contrasts what was said  the previously, but its literal meaning is something like 'nothing'. See also the next note.).
 
\largerpage[2]
\ea     hameri eruera urumekareheyada kapü hina'ĩ \\[.3em]
\gll 	hameri eruera urume-ka-re-he-yada ka-pü hina-ĩ\\
already fox transform-\textsc{1sg-fut-reas} do-\textsc{ss} no-\textsc{nmlz}\\
\glt   `Since they were already becoming foxes it was no good.' \\
`Agora que já se transformaram em raposas, não foi bem.'
%ka-pü hina-ĩ translated by Luiz as 'já era, já foi'. It seems an idiomatic expression and it is sometimes pronounced as [kapöni] or [kapini].
%agora já se transformaram em raposas (that was still going on (therefore the FUT))
%JB: -yada purposive?. No, it´s an adverbial clause of reason.
\z

\ea   hadite'ete büxuheku'ẽ kukana \\[.3em]
\gll 	hadite-'ete büxu-he-ku-'ẽ kuka-na \\
shaman-\textsc{all} cure-3-\textsc{1sg.ben-imp} tell-\textsc{ds}\\
\glt    `Father told the shaman, ``Cure them for me!{''}' \\
`O pai pediu ao pajé, ``Cura elas para mim!{''}'
%He asked a shaman to cure them for him and he went there.
%pediu o pajé curar eles, e dai o pajé veio           
%lit. perhaps "at the shaman´s he said "cure him for me""
\z

\ea  hadite wareyãpü büxühepü  \\[.3em]
\gll 	hadite ware-yã-pü büxü-he-pü\\
shaman  go-come-\textsc{ss} cure-\textsc{3sg-ss}\\
\glt `The shaman came and he cured them.' \\
`O pajé veio e curou.' \\
%curou
%JB: changed clitic to affix. compound verbs are rarer than we first thought
\z

\ea   arerekekukahepü    \\[.3em]
\gll 	arere-ke-kuka-he-pü \\
blow-\textsc{3}-\textsc{clf}:body-3\textsc{sg}-\textsc{ss}\\                       
\glt `He blew on and cleansed the body.'\footnote{The process of sucking and blowing away maladies is a central part of Aikanã shamanic healing and is a common practice among many lowland South American groups.}    \\
`Ele assoprou e limpou o corpo.' \\
%assoprou e limpou o corpo
\z


%EX105

\ea   keapü dupakapü \\[.3em]
\gll 	kea-pü dupa ka-pü\\
get-\textsc{ss} really do-\textsc{ss}\\
\glt   `He did it just like this.' \\
`Ele fez assim mesmo.'\\
%On the basis of Cândida´s translation and the story context, -dupa- could be identified as the emphatic infix 'really'/'mesmo', but I find it unexpected as a free particle. Alternatively, it could be seen as the concessive suffix -dupa 'although'/'assim mesmo', but then a verb root is lacking and the context does not support this interpretation.
\z

\ea  wɨiwɨimezakukane'eta'ẽ kukaẽ \\[.3em]
\gll 	wɨiwɨi-meza-ku-ka-ne-'eta-'ẽ kuka-ẽ\\
repeat-\textsc{2sg.caus-1sg.ben-tr-pfv-rem.fut-imp} tell-\textsc{decl}\\
\glt    `{``}You must do that again for me,'' (the father) said to him.'\\
`{``}Repete mais uma vez para mim,'' (o pai) falou pra ele.'
%repete mais uma vez pra mim
%CHK the analysis: -ne- 'PFV' here? And what about -eta-??
\z

\ea   zamiya mama'ĩ mama'ĩkea'ẽ detyaderi'ete kukapü \\[.3em]
\gll  zamiya mama'ĩ mama'ĩ-kea-'ẽ detya-deri-'ete kuka-pü\\
now chicha chicha-\textsc{3-imp} woman-\textsc{3.poss-all} tell-\textsc{ss}\\
\glt    `{``}Now make chicha!'' he told his wife.'\\
 `{``}Daí faz chicha então!'' ele falou para sua esposa.'\\
%daí faz chicha então, falou para a mulher dele
%-kya- functions as an object marker here or as a transitiviser. Or is it kea- 'get, make, extract'?
\z

\ea   hisa zamumuye tara ari'iye takawaparekaẽ hadite kawhepü \\[.3em]
\gll 	hisa zamumu-ye tara ãri'i-ye ta<ka>wa-pa-re-ka-ẽ hadite kaw-he-pü\\
\textsc{1sg} patawa.larva-\textsc{obj} what mamuí.larva-\textsc{obj} break<\textsc{1sg}>-\textsc{tr-fut-1sg-decl} shaman eat-3-\textsc{ss}\\
\glt    `{``}I am going to get some patawá and mamuí larvae for the shaman to eat.{''}'\footnote{The indigenous peoples of Rondônia cultivate the protein-rich larvae of specific beetle species by cutting down patawa (\textit{Oenocarpus bataua}) or buriti (\textit{Mauritia flexuosa}) palm trees and wild papaya (\textit{Jaracatia spinosa}, in Portuguese \textit{mamuí}) trees, leaving them to be eaten from the inside by these larvae. After about half a year the trunks can be cracked open and the delicious larvae can be harvested.} \\
`{``}Eu vou tirar coró de patauá e de mamuí para o pajé comer.{''}'\\
%CHK discontinuous verb root ta-wa- 'to crack open wood (in order to get to the coró)' maybe related to tau- 'break (egg, wood)'? (it is different from kira- 'split wood (for firewood)'
\z

\ea   ü'ükekukakũyare'ẽ wãwã'iye detyaderi'ete kukapü \\[.3em]
\gll 	ü'ü-ke-kuka-kũya-re-'ẽ wãwã'i-ye detya-deri-'ete kuka-pü\\
 save-3-\textsc{clf:}body-\textsc{1pl.ben-fut-imp} child-\textsc{obj}  woman-\textsc{3.poss-all} tell-\textsc{ss}\\
\glt    `{``}He will fix the bodies of the children for us,'' he said to his wife.' \\
`{``}Ele vai concertar o corpo das crianças para nós,'' falou para a mulher dele.'
%ü'ü- 'to heal, repair' is also 'to keep, harvest', so perhaps a gloss like 'save' covers both of these connotations.
\z
%JB: -kuya-are?
%EX110

\ea   babaderi daedaena'ĩ  daedaenapü \\[.3em]
\gll 	 baba-deri dae\textasciitilde dae-na-'ĩ  dae\textasciitilde dae-na-pü \\
father-\textsc{3.poss} walk\textasciitilde walk-go-\textsc{nmlz}  walk\textasciitilde walk-go-\textsc{ss}\\
\glt  `Her father went walking (in the forest).'  \\
`O pai delas foi andando (no mato).'

%there is a verb (that takes person prefixes) -na-, which means 'go away' ('vir embora')
\z
%-na reduced -nape 'in the woods'
 
\ea   zamumuye takewapü   \\[.3em]
\gll 	zamumu-ye ta<ke>wa-pü  \\
patawá.larva-\textsc{obj} break<3>-\textsc{ss}\\ 
\glt  `He removed patawá larvae (from the wood).' \\
`Ele tirou coró de patauá.'\\
\z
 
\newpage %solves several footnote problems at once
\ea  ari'i-ye keapü \\[.3em]
\gll 	 ãri'i-ye kea-pü \\
mamuí.larva-\textsc{obj} get-\textsc{ss} \\
\glt  `He got mamuí larvae.' \\
`Pegou coró de mamuí também.'   \\
\z

\ea    nusunapapü \\[.3em]
\gll 	nu-suna-pa-pü\\
come-\textsc{dir:}return-\textsc{tr-ss}\\
\glt   `He brought them back home.'\\
`Ele trouxe de volta pra casa.'\\ 
\z

\ea   amakea'ẽ detyaderi'ete kukapü amamakezaẽ \\[.3em]
\gll 	ama-kea-'ẽ detya-deri-'ete kuka-pü ama\textasciitilde ma-keza-ẽ\\
cook-\textsc{3-imp} woman-\textsc{3.poss-all} tell-\textsc{ss} cook\textasciitilde \textsc{red}-\textsc{3sg.caus-decl}\\
\glt    `{``}Cook it!'' he told his wife, and she cooked it.'  \\
`{``}Cozinha aí!'' ele falou para a mulher dele, e ela cozinhou.'\\ 
%cozinha aí, ele falou para a mulher dele, e ela cozinhou
\z

%EX115

\ea   hikiririkapedupana zamiya \\[.3em]
\gll 	hikiri-rika-pe-dupana zamiya \\
dark-\textsc{clf}:floor-\textsc{clf:}round-\textsc{temp} now\\
\glt   `As it was getting dark inside ...' \\
`Enquanto estava escurecendo lá dentro...' \\
%it was getting dark inside the traditional round straw house
\z

\ea   hadite tãwĩkukapü irüpü \\[.3em]
\gll 	hadite tãwĩ-kuka-pü irü-pü\\
shaman await-\textsc{clf}:body-\textsc{ss} trance-\textsc{ss}\\
\glt   `He called the shaman to enter into a trance.'\footnote{In this state the shaman is sitting down on his/her bench while pulling down the invisible lines that form the net on which his/her spirit can travel, the \textit{haditaezũ daruma} `shaman's sling'. In order to heal, he/she performs acts such as sucking, blowing smoke, gestures of collecting, extracting, expelling, etc. The last Aikanã shaman passed away in 1985, but elderly people remember the tradition and are often able to interpret the work of shamans from other ethnic groups.}\\
`Ele chamou o pajé para rezar.'\\
%irü pray? He is probably in trance. Cf. the reflexive verb irü-pe- 'tired'.
%the switch-reference rules are violated here, unless the the shaman is subject of both verbs
\z
 
\newpage 
\ea   wãwã'ĩ ukikekukaku'ẽ awexü urumeẽ \\[.3em]
\gll 	 wãwã'ĩ uki-ke-kuka-ku-'ẽ awexü urume-ẽ \\
child clean-3-\textsc{clf:}body-\textsc{1sg.ben}-\textsc{imp} demon transform-\textsc{decl} \\
\glt   `{``}Cleanse the body of my daughter, who has transformed into a demon!{''}'\footnote{The \textit{awexü} is a dangerous and powerful spirit of the forest that can transform itself into any being and is able to make people lose their mind. Especially when someone is alone in the forest or on a remote cultivated plot, the \textit{awexü} may trick someone and lead him/her astray or directly attack and kill a person. Unexpected death and psychotic illness are often explained as the work of the \textit{awexü}. Experiences with the \textit{awexü} are always traumatic and accounts of them are harrowing.}  \\
`{``}Limpa o corpo da criança que se transformou em bicho do mato!{''}'\\

%Fatima has uki- 'limpar'
%treat urume as a unsegmented verb, transform
\z
   
\ea   kukaku'ẽ kukana  \\[.3em]   
\gll 	 kuka-ku-'ẽ kuka-na\\
tell-\textsc{1sg.ben-imp} tell-\textsc{ds}\\
\glt   `{``}Talk to him for me!'' he said.' \\
`{``}Fala para ele para mim!'' ele falou.' \\
%fala pra ele pra mim, falou
\z

\ea    hadite wareduapü \\[.3em]
\gll 	hadite ware-dua-pü \\
shaman go-\textsc{dir}:inside-\textsc{ss}\\
\glt    `The shaman went inside.' \\
`O pajé entrou para dentro.'\\
\z

%EX120

\ea     wareriakapü uruhepü \\[.3em]
\gll 	ware-riaka-pü uru-he-pü\\
go-\textsc{dir}:middle-\textsc{ss} sing-3-\textsc{ss}\\
\glt    `He went to the middle of the house and sang.'\\
`Ele chegou no meio da casa e cantou.'\\
%JB: changed ria-ka to -rika; HV: I changed it back, because it is said, and I elicited a whole paradigm with it on the left page, so I think it is real.
\z

\ea   hadite kɨineke wareyü̃panake büxü'ẽ hepü \\[.3em]
\gll 	hadite kɨine-ke ware-yü̃-pa-nake büxü-'ẽ he-pü\\
shaman \textsc{3sg-com} go-\textsc{dir}:near-\textsc{tr-cond} cure-\textsc{imp} say-\textsc{ss} \\
\glt   `As soon as the shaman brought her with him, (the father) said ``Cure her!{''}' \\
`Quando o pajé levou ela junto com ele, (o pai) falou ``Cura ela!{''}'
%JB: are these -ke all comitatives? HV: only the first -ke is comitative.
%JB: changed hinake to kiineke after listening to recording. Could it be hizake? would correspond to 2sg.obj on verb, could also be buxuhee...
%HV: I think your change makes the best sense. I transcribed hinake (hizãkje), but if that is a 2nd person, it does not makes sense in the context (and I don´t think the verb has a 2.obj inflection). 
\z

\ea     arerekekukaxüne'ẽ \\[.3em]
\gll 	arere-ke-kuka-xüne-'ẽ \\
blow-\textsc{3}-\textsc{clf}:body-\textsc{dir}:return-\textsc{imp} \\
\glt    `{``}Blow and cleanse the body again!" (the father) said.' \\
`{``}Assopra e limpa o corpo de novo!" (o pai) falou.' \\
%assoprou e limpou o corpo dela de novo
%JB: separate -ke as 3rd person?
\z
%arere blow?

  
\ea     hibaye awexüye hukedurakaxüne'ẽ \\[.3em]
\gll 	hiba-ye awexü-ye hu-ke-duraka-xüne-'ẽ\\
this-\textsc{obj} demon-\textsc{obj} remove-\textsc{3}-\textsc{dir}:inside-\textsc{dir}:return-\textsc{imp}   \\
\glt    `{``}Remove this demon from inside of her!{''}' \\
`{``}Tire esse bicho do mato que está dentro dela!{''}'\\
\z

 
\ea     pawpawkezakaxüne'ẽ kyãkukapü \\[.3em]
\gll 	paw\textasciitilde paw-keza-ka-xüne-'ẽ kyã-kuka-pü\\
run\textasciitilde run-\textsc{3sg.caus}-\textsc{tr}-\textsc{dir}:return-\textsc{imp} speak-tell-\textsc{ss}\\
\glt    `{``}Make it (the demon) run away!" he said to him.' \\
 `{``}Espanta o espírito para fora!" falou para ele.'\\
%espantou o espirito pra fora, falou pra ele  amanheceu o dia
%JB: not sure about -ka transitive
\z

%EX125

\ea   derinena zamiya deripanena  \\[.3em]
\gll 	deri-ne-na zamiya deri-pa-ne-na \\
light-\textsc{pfv-ds} now  light-\textsc{tr}-\textsc{pfv-ds} \\
\glt  `Dawn came and then day came.' \\
`Clareou e amanheceu o dia.'\\
%amanheceu
\z

\ea   zamiya kapü kukapü daexünepü \\[.3em]
\gll 	zamiya ka-pü kuka-pü dae-xüne-pü\\
now do-\textsc{ss} tell-\textsc{ss} walk-\textsc{dir}:return-\textsc{ss}\\
\glt  `{``}It's done," he said, and (the demon) went back to where he came from.' \\
`{``}Está pronto,'' ele falou e (o bicho do mato) voltou para de onde veio.'   \\
\z

\ea   hena wɨiwɨiyeye wãkanayeyepü urikɨieye hürakenupapü \\[.3em]
\gll 	he-na wɨiwɨi-yeye wãkana-yeye-pü urikɨi-ye hüra-ke-nupa-pü\\
then-\textsc{ds} repeat-\textsc{ite} late.morning-\textsc{ite-ss} food-\textsc{obj} place-\textsc{3}-\textsc{dir}:outside-\textsc{ss}\\
\glt `Then it had to be repeated during the day, and food was placed in the yard (for the shaman and the possessed child) .' \\
`Daí ele fez outra vez de dia, e ele deixou comida no terreiro (para o pajé e a criança).'
%JB: hüra 'poison' ? did she leave poisoned food in the yard?
%HV: hüra- means 'to place food', and often, but not always, with a connotation of to poison. I dont think poison is the case here. hüra-wa- 'put in mouth', hüra-nupa- 'place in yard'. Apparently in particular food, in order for someone to take it.
\z
%her instead of him? HV: yes

\ea  kariye kawhepü \\[.3em]
 \gll 	kari-ye kaw-he-pü\\ 
that-\textsc{obj} eat-3-\textsc{ss}\\
\glt    `She ate some of that.'\\
`Ela comeu aquilo.' \\
\z
%difference kari and hiba? Yes: kari = Dem.pron. 'that, that one', hiba = particle or verb root 'here, take this'.

\ea büxühekunehe'ẽ kukapü büxühenehepü xükeakapapü\\[.3em]
\gll büxü-he-ku-ne'e-'ẽ kuka-pü büxü-he-ne'e-pü xü-kea-ka-pa-pü \\
cure-3-\textsc{1sg.ben-ite-imp} tell-\textsc{ss} cure-3-\textsc{ite-ss} finish-\textsc{3sg}-\textsc{clf:}piece-\textsc{tr-ss}\\
\glt `{``}Perform the cure again for me,'' he said, and the shaman did so again, and finished (removing the demon's spirit from the head of the child).' \\
`{``}Cura de novo para mim,'' ele falou, e o pajé curou de novo, e terminou (tirando o espírito do bicho da cabeça da criança).'
%o pajé curou de novo, falou e curou novamente, tirando a alma
\z
%kapa as a single transitivizer (I doubt it)
%xü-kea- also in the formigas text. Related to xüi- 'dig'?

\ea   taraye kawahenake xükeakapapü \\[.3em]
\gll 	tara-ye kaw-a-he-nake xü-kea-ka-pa-pü\\
what-\textsc{obj} eat-\textsc{impers}-3-\textsc{cond} finish-\textsc{3sg}-\textsc{clf:}piece-\textsc{tr-ss} \\
\glt    `Through eating (the food) he could remove the demon's spirit (from the bodies of the girls).' \\
`Comendo as coisas ele tirou a alma do bicho (dos corpos das meninas).'\\
%comendo as coisas ele tirou a alma do bicho
%JB: doubts on impers
\z

%EX130
\ea  kapünepü zamiya kapü zamiya hü'akaxünerewaẽ wãwã'ĩ \\[.3em]
\gll 	kapü-ne-pü zamiya ka-pü zamiya hü'a-ka-xüne-re-wa-ẽ wãwã'ĩ\\
finish-\textsc{pfv-ss} now do-\textsc{ss} now good-\textsc{1sg}-\textsc{dir:}return-\textsc{fut}-\textsc{2sg.ben}-\textsc{decl} child\\                        
\glt  `He finished and said, ``Now that it's done, your children will get well again.{''}' \\  
`Ele terminou e falou, ``Agora que foi feito, suas filhas vão sarar.{''}'
% JB: "return to being clean"? 1sg? HV: No, this is a future construction with a 2sg beneficiary (the father of the girls).
\z

\ea   zare kaxünerewaẽ \\[.3em] 
\gll 	zare ka-xüne-re-wa-ẽ\\
person \textsc{1sg}-\textsc{dir:}return-\textsc{fut}-\textsc{2sg.ben}-\textsc{decl}\\                        
\glt  `{``}They will be people again for you.{''}' \\
`{``}Vão se tornar em gente de novo para você.{''}' \\
%-wa up? (No: 2sg.ben)
\z

\ea  kuka'i apa'ixüte kukana \\[.3em]
\gll 	kuka-'i apa-'ixüte kuka-na\\
tell-\textsc{nmlz} say-\textsc{rep} tell-\textsc{ds}\\
\glt  `This is how he spoke to the father.' \\
`Assim que ele falou para o pai.' \\ 
\z

\ea    hukadupɨi hepü \\[.3em]
\gll 	hukadupɨi he-pü\\
alright say-\textsc{ss}\\
\glt   `{``}Alright,'' he said.'\\
`{``}Está certo,'' ele falou.'\\
%está certo, falou
\z

\ea   wãwã'ĩke hina'ĩ hũka hũka eryüanahe'ẽ  hukakeaderi \\[.3em]
\gll 	 wãwã'ĩ-ke hina-'ĩ hũka hũka eryüana-he-'ẽ hũka-kea-deri\\
child-\textsc{com} no-\textsc{nmlz} in.vain in.vain sick-3-\textsc{decl} in.vain-\textsc{3-nmlz}\\
\glt    `The children used to be ill and were really going crazy.'\\
`As crianças também não tinham sussego, viviam bagunçando.'\\
%HV: not sure about hina'ĩ (glossed here consistently with the rest). Perhaps Luiz was correcting himself, wanting to say wãwã'ĩ-'ene-ke, but saying wãwã'ĩ-ke, ene.
%JB: eryüana "sick" or eryüa-na "stay-DS" "live-ds" 
%HV: it should mean 'sick' here, lit. 'not existing (well)', i.e. 'not healthy'.
%BETTER TRANSLATION hukakeaderi? Yes, with -kya- '3/tr/make' and the focus nominaliser -deri
%HV: can´t we just gloss 'vain'?
%JB: No, vain alone means usually vainglorious/pretentious 
\z

%EX135
\ea     zamiya zɨ̃izɨ̃i eryüaxünena zamiya \\[.3em]
\gll 	zamiya zɨ̃izɨ̃i eryüa-xüne-na zamiya\\
now correct live-\textsc{dir:}return-\textsc{ds} now \\
\glt    `But now they were behaving well again.' \\
`Mas agora ficaram direitinhas de novo.' \\
%zɨ̃izɨ̃i 'direitinho' is a fixed particle expression, base on the verb root zɨ̃i- 'be well', 
\z

\ea   wãwã'ĩ hikaderike hepü hü'axüneẽ \\[.3em]
\gll 	wãwã'ĩ-ika-deri-ke he-pü hü'a-xüne-ẽ\\
child-\textsc{intens-nmlz-com} then-\textsc{ss} good-\textsc{dir:}return-\textsc{decl}\\
\glt    `The youngest also got better.' \\
`A criança mais nova melhorou também.'
% JB: "the child that left"?  removed 'e from hika'ederike 
%HV: see line (44) (and (49))
%HV: I changed kapü to hepü
\z

\ea    hiba tiwenederi hü'axüneẽ \\[.3em]
\gll 	hiba ti'iwe-ne-deri hü'a-xüne-ẽ \\
this grow-\textsc{pfv-nmlz} clean-\textsc{dir:}return-\textsc{decl}\\
\glt    `The older one got better.'\\
`Essa mais velha melhorou.'\\
%wene seems to have something to do with being old
% Fatima give ti'ive-ne 'get old, age' on p.160
%HV: Fatima is correct, see line (50)
\z

\newpage 
\ea    kyã'i apatena \\[.3em]
\gll 	kyã-'i apa-te-na\\
speak-\textsc{nmlz} say-\textsc{pst}-\textsc{ds}\\
\glt   `This is how they told it.' \\
`Assim que contaram.'\\
%JB: kea'i 3-nmlz? 
\z

\ea   dupana zarikapasapü zarikahedupana \\[.3em]
\gll 	dupana zarika-pa-sa-pü zarika-he-dupana\\
while delay-\textsc{tr-mal-ss} delay-3-\textsc{temp}\\
\glt    `But a while after (Fox had abducted them)...'\footnote{Here the narrator goes back to an earlier phase in the story, adding the part concerning the real cousin after the girls had been abducted.}\\
`Mas um pouco depois (que o Raposa sequestrou elas)...'
%mas depois demorou com ele
%Fatima gives both 'demorar' and 'sarar' for zarika. in this context, either one could be correct. HV: demorar
\z

%EX140
\ea   hiku kapü kaxare'ẽ namɨideri'ika kukaderiye wareyü̃pü \\[.3em]
\gll 	hiku ka-pü ka-xa-re-'ẽ namɨi-deri-ika kuka-deri-ye ware-yü̃-pü\\
other do-\textsc{ss} do-\textsc{1pl-fut-imp} cousin-\textsc{3.poss-intens} tell-\textsc{nmlz}-\textsc{obj} go-\textsc{dir}:close-\textsc{ss}\\
\glt   `The real cousin, the one who said to do it (to get peanuts) arrived (at the girls' house).'  \\
`Aquela prima delas que tinha combinado com ela (arrancar amendoim) primeiro chegou (na casa das meninas).'\\
\z

\ea   kaxare'ẽ ka'ĩwãte yãw'ẽ deripahãyã'ẽ namɨi kukaẽ \\[.3em]
\gll 	ka-xa-re-'ẽ ka-'ĩwã-te yãw'ẽ deri-pa-hãyã-ẽ namɨi kuka-ẽ\\
do-\textsc{1pl-fut-imp} do\textsc{-admon-pst}  let's.go.\textsc{imp} day-\textsc{tr}-\textsc{1pl.obj-decl} cousin tell-\textsc{decl}\\
\glt    `{``}We had agreed to do it, let's go! It's becoming day for us,'' the cousin said.' \\
`{``}Vamos lá fazer o que concordamos! O dia está amanhecendo em nós,'' a prima falou.' \\
%vamos lá, que está amanhecendo o dia, a prima falou
\z

\ea   mamaderi warehikadepü keriẽ hĩzã kamezakukateare apɨire'i \\[.3em]
\gll 	mama-deri ware-hika-de-pü keriẽ hĩzã ka-meza-kuka-te-are apa-ire-'i \\
mother-\textsc{3.poss} go-leave-\textsc{dir}:outside-\textsc{ss}  whoa! \textsc{2sg} do-\textsc{2sg.caus}-\textsc{clf}:body-\textsc{pst}-\textsc{infr} say-almost-\textsc{int} \\
\glt   `Her mother went outside and said: ``Whoa! Aren't you the one that was going to call them?{''}' \\
`A mãe saiu pra fora e falou, ``Nossa! Não foi você que chamou elas?{''}'\\
%a mãe saiu pra fora e falou, você não era aquele que chamou elas?
%isn't EVID a class of morphemes to relate information source? M
%changed kere to kerie. HV: I changed it back
%changed mea to meza
%change tiare to teare
\z

\ea   hameri'ẽ  tara kawãte'i wareyãpü namɨi namɨi kukapü \\[.3em]
\gll 	hameri-h-ẽ  tara kawãte-'i ware-yã-pü namɨi namɨi kuka-pü\\
already-\textsc{3-decl} what because-\textsc{int} go-come-\textsc{ss} cousin cousin say-\textsc{ss} \\
\glt    `{``}Who was the one that came already and said, `Cousin! cousin!' then?{''}'\\
`{``}Quem será que veio e chamou `Prima! Prima!' naquela hora?{''}'
%
%JB: unsure of analysis of kawã-te-'i. HV: kawãte is also a word meaning 'because' (glossed consistently here). With an interrogative and tara it means 'que(m) era, então?'.
\z

 
\ea   yãw'ẽ kukapü warehikadepapü hameri pa'ĩwãte \\[.3em]
\gll 	yãw'ẽ kuka-pü ware-hikade-pa-pü hameri pa-'ĩwã-te\\
let's.go.\textsc{imp} say-\textsc{ss} go-\textsc{dir}:outside-\textsc{tr-ss} already  unsuccessful-\textsc{admon-pst}\\
\glt   `{``}Let's go!'' she had said to her, but they had already left.'\\
 `{``}Vamos embora!'' falou para ela, mas elas já tinham saido.'\\
%chamou eles mas já tinham saido embora
%JB: Fatima pa- "ser panema' HV: excellent solution!
\z

%EX145
\ea   hĩzã kamezɨiare ka'ĩwãte kukaẽ \\[.3em]
\gll 	hĩzã ka-meza-i-are ka-'ĩwã-te kuka-ẽ\\
\textsc{2sg} do-\textsc{2sg.caus-nmlz-infr} do-\textsc{admon-pst} say-\textsc{decl}\\
\glt   `{``}I thought it was you,'' she said to the cousin.' \\
 `{``}Pensei que era você,'' ela falou para a prima.' \\
%pensei que era você, falou pra ela
%changed kameziyare to kamezɨiare
\z

\ea   hinaẽ hisa kayana'ĩwãte kapü derinena kayata'ẽ ka'ĩwã \\[.3em]
\gll 	hina-ẽ hisa ka-ya-na-'ĩwã-te ka-pü deri-ne-na ka-ya-ta-'ẽ ka-'ĩwã\\
no-\textsc{decl}  \textsc{1sg} \textsc{1sg}-come-\textsc{neg-admon-pst} do-\textsc{ss} light-\textsc{pfv-ds} \textsc{1sg}-come-\textsc{rem.fut-decl} do-\textsc{admon} \\
\glt  `{``}No, it wasn't me. I let it dawn first.{''}'  \\
`{``}Não foi eu não, deixei clarear o dia primeiro.{''}'\\
%somewhat unsure of last few words; is it imperative (I had it dawn, lit. 'dawn!')?
\z

\ea   ka'ĩwãte kawã zarena'ĩ kawã kazapasahãyãtena \\[.3em]
\gll 	ka-'ĩwã-te kawã zare-na-'ĩ kawã ka-za-pa-sa-hãyã-te-na\\
do-\textsc{admon-pst} be.like person-\textsc{neg-nmlz} be.like do-\textsc{caus-tr-mal-1pl.obj-pst}-\textsc{ds}\\
\glt    `{``}So it must not have been a person that took them from us.{''}' \\
`{``}Então não era uma pessoa que levou elas de nós.{''}' \\
%unsure of glossing fore zarena'i. HV: I changed int to nmlz.
%same pasa as in line (140) 'carregou'. First I decided to translate it as 'abduct', but -pa- is really a transitiviser and -sa- is a malefactive. The story is less explicit lexically, and most of the action is expressed in grammatical elements, resulting in literal expressions like 'he did it to them on us' etc.. However, the analysis of kaza is not certain. -za- could be sth. else.
\z

\ea     babaderi mamaderi he'i he'i apɨixüte \\[.3em]
\gll 	baba-deri mama-deri he-'i he-'i apa-ixüte \\
father-\textsc{3.poss} mother-\textsc{3.poss} say-\textsc{nmlz} say-\textsc{nmlz} say-\textsc{rep} \\
\glt    `Their father and mother said this. This is what was said.'\\
`O pai e a mãe delas falou isso. Assim que falaram.'\\
\z


\ea     hepü hikirinena kawã kawã hikirinena \\[.3em]
\gll 	he-pü hikiri-ne-na kawã kawã hikiri-ne-na \\
then-\textsc{ss} dark-\textsc{pfv-ds} be.like be.like dark-\textsc{pfv-ds}\\
\glt    `Then it got dark.'\\
`Daí escureceu, escureceu mesmo.'\\
\z

%EX150
\ea     bari wareyãepü kaxata'ẽ hikiri'ikana \\[.3em]
\gll 	bari ware-yã-e-pü ka-xa-ta-'ẽ hikiri-'ika-na\\
who go-come-\textsc{2sg.obj-ss} do-\textsc{1pl-rem.fut-imp} dark-\textsc{intens-ds}\\
\glt   `{``}When someone comes for you, saying, ‘Let's do it early in the morning!'{''} ...' \\
`{``}Quando alguém vier para você falando ‘Vamos lá amanhã cedo!'{''} ...'
%`{``}Whoever comes for you, let's go there early.{''}' 
%quem veio pra você, vamos lá amanhã cedo
%better translation?: whoever comes for you, it should be early
%JB: segmented bari
\z

\ea   hapa'aparete'ẽ apa'i apɨixüte apa'ẽ kyãapɨisuwãẽ he'ẽ xüxüe kyã'isuwãẽ xüxüe xüxü xüxüe Kwã'ĩ \\[.3em]
\gll 	h-apa\textasciitilde apa-rete'ẽ apa-'i apa-ixüte apa-'ẽ kyã-apa-isuwã-ẽ he-'ẽ xüxüe kyã-'isuwã-ẽ xüxüe xüxü xüxüe Kwã'ĩ \\
\textsc{2sg}-say\textasciitilde say-\textsc{neg.imp} say-\textsc{nmlz} say-\textsc{rep}	say-\textsc{decl} speak-say-\textsc{rem.pst-decl} say-\textsc{decl} grandmother speak-\textsc{rem.pst-decl} grandmother \textsc{1sg.poss} grandmother Kwã'i\\
\glt   `{``}You can't talk like that with people,'' my grandmother used to say.\footnote{Here, reference is made to the moral of the story, also mentioned in the introduction, that one should not talk about one's plans.} Grandmother Kwã'ĩ.' \\
`{``}Você não pode falar assim com os outros,'' assim que falava minha avó. Vovó Kwã'ĩ.
%você não pode falar assim com outro, assim que falava minha avó
%Fatima give Kiã- as "falar, conversar, aconselhar"
\z

\ea   kariyame ãryüaka'ĩwã  \\[.3em]
\gll 	kari-ame ãryüa-ka-'ĩwã  \\
this-\textsc{sup} know-\textsc{1sg-admon} \\
\glt  `I know just this.' \\
`Só isso que eu sei.' \\
%assim que minha vovó Kwã'ĩ contou pra mim, só isso que eu sei
% JB: 2sg? curious.... HV: there is no hizã there in the recording, nor in my transcription, so I deleted it. I also took out one of the kawães.
\z
 
\ea   kawãẽ \\[.3em]
\gll 	kawã-ẽ \\
be.like-\textsc{decl}\\
\glt   `That's it.'  \\
`É assim.'
\z
 


%%{\footnotemark}
%\footnotetext{Footnotes use a slightly strange notation when occurring in examples as the footnotemark and the footnotetext are separated
%}


%\section{Sebastian: how to apply multiple footnotes}
%\ea 
%watxile karɛ͂xu katsutyata xareyawata axehɨ̃ko tsadwɛnɛ\\[.3em]
% \gll watxile karɛ͂xu  katsu-tya    ta    xareya-wa-ta axe-hɨ̃-tya{\footnotemark}    tsadwɛ-nɛ\\
% finally      dry.heartwood    cross-\textsc{cso}    \textsc{cso} search-\textsc{isbj-cso}   find-\textsc{nmlz-cso}
%onto.path-\textsc{dir}:hither\\
% \glt 'Later, crossing the dry log, they{\footnotemark} then searched
%and got back onto the path.'
% \z
 %\addtocounter{footnote}{-1}
 %\footnotetext{In the transcription, Mario replaced the instrumental
 %marker -\textit{ko} by the cosubordinative marker -\textit{tya}.}
%\stepcounter{footnote}
 %\footnotetext{The girl's mother and the rest of her family.} 

 \section*{Acknowledgments}
Generous funding by the VolkswagenStiftung of DoBeS (Dokumentation Bedrohter Sprachen) project nr. 85.611 is hereby gratefully acknowledged. 
In addition, Luiz Aikanã's visit to the Museu Goeldi was kindly funded by the Brazilian National Science Foundation CNPq (Conselho Nacional de Pesquisa Científica) within Vilacy Galucio's project \textit{Documentação de línguas indígenas e a sua integração no acervo digital de línguas indígenas do Museu Goeldi}. Additional comments and corrections were kindly provided by Raimunda and Mario Aikanã. 
 
\section*{Non-standard abbreviations}

\begin{tabularx}{.45\textwidth}{lX}
\textsc{act} & action \\
\textsc{admon} & admonitory \\
\textsc{ag} & agent \\
\textsc{col} & collective \\
\textsc{conc} & concessive \\
\textsc{desi} & desiderative \\
\textsc{dim} & diminutive \\
\textsc{dir } & directional \\
\textsc{ds } & different subject\\
\textsc{dub} & dubitative\\
\textsc{hab} & habitual aspect\\
\textsc{hort} & hortative \\
\textsc{impers} & impersonal \\
\textsc{infr} & inferential mood \\
\end{tabularx}
\begin{tabularx}{.45\textwidth}{lX}
\textsc{int} & interrogative \\
\textsc{intens } & intensifier \\
\textsc{ite} & iterative \\
\textsc{mal} & malefactive \\
\textsc{mir} & mirative \\
\textsc{proc} & procrastinative \\
\textsc{reas} & reason adverbial \\
\textsc{red } & reduplication \\
\textsc{rem} & remote \\
\textsc{rep} & reported past \\
\textsc{ss} & same subject\\
\textsc{sup} & superlative \\
\textsc{temp} & temporal adverbial \\
\\
\end{tabularx}

{\sloppy
\printbibliography[heading=subbibliography,notkeyword=this]
}
\end{document}
