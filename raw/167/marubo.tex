\documentclass[output=paper,
modfonts,nonflat
]{langsci/langscibook} 
\author{Pedro de Niemeyer Cesarino\affiliation{University of São Paulo, Brazil}%
\and Armando Mariano Marubo Cherõpapa Txano%
\lastand Robson Dionísio Doles Marubo%
}%
\title{Marubo}
\lehead{P.\ N.\ Cesarino, Armando Marubo \& Robson Marubo}
\ourchaptersubtitle{Vei vai yoã}
\ourchaptersubtitletrans{`The death-path teachings'}  
% \abstract{noabstract}
\ChapterDOI{10.5281/zenodo.1008779}

\maketitle

\begin{document}

\section{Introduction}

The Marubo are Panoan-speakers from the Javari River Indigenous Reservation (Terra Indígena Vale do Javari, state of Amazonas, Brazil), who live along the headwaters of the Ituí and Curuçá Rivers, as well as in the cities of Cruzeiro do Sul (in the state of Acre) and Atalaia do Norte (Amazonas). Their population is currently estimated at 1,700. The Marubo were reasonably unaffected by the rubber trade that devastated vast portions of the Amazon region during the nineteenth and early twentieth centuries, including the Juruá river basin, where other Panoan-speakers still live. Far from the urban centers, their lands protected a society that was created at the turn of the twentieth century by an important chief and shaman, João Tuxáua, and his relatives \citep{Ruedas2001,Welper2009}. João Tuxáua was responsible for gathering a number of dispersed Panoan-speakers and creating a new society out of earlier cultural and linguistic traditions, adopting the language of one group -- the Chai Nawavo -- that now comprise the contemporary Marubo. In fact, earlier groups (whose names are always followed by \textit{nawavo} or `people', as in \textit{Chai Nawavo} or `Bird People' and \textit{Vari Nawavo} or `Sun People') became segments of Marubo social and kinship system. The Marubo continue to live in longhouses, which have been abandoned by other Panoan-speakers, such as the Kaxinawa, Katukina, Yaminawa, Sharanawa, and Shipibo-Conibo, but are also maintained by the Matis, Mayoruna, and Korubo: three other Panoan-speaking peoples from the Javari river basin.

\begin{figure}[H]
\includegraphics[width=\textwidth]{figures/marubo.pdf}
  \caption{The Javari basin and location of the Marubo.}
\end{figure}

The Marubo preserve a very active ritual life characterized by the work of prayer-shamans (\textit{kẽchĩtxo}) and spirit-shamans (\textit{romeya}). Complex initiation and ritual knowledge transmission processes are ongoing, and involve the performance and instruction of verbal genera, such as curing songs (\textit{shõki}), spirit songs (\textit{iniki}), chiefly speeches (\textit{tsãiki}), instructive speeches (\textit{ese vana}), and mythical narratives (see \citealt{Montagner1985,Montagner1996,Cesarino2011,Cesarino2013}, among others). The latter can be performed in two ways: narrated (\textit{yoã vana}) with the special use of parallelism, rhythm, metaphors, and gestures, or sung (\textit{saiti vana}), by use of constant melodic phrases (one for each story) and fixed meters. The vast \textit{yoã vana}, a collection of mythical narratives, is the cornerstone of Marubo ritual knowledge; its episodes can be transferred to other verbal arts for ritual efficacy or counsel (see \citealt{Cesarino2011} for a detailed study). Prayer-shamans are responsible for verbal knowledge transmission and understanding, while spirit-shamans (who are also prayer-shamans) circulate through the realms of spirits and dead people that compose Marubo cosmology.

The research presented here was conducted with two spirit-shamans, Robson Dionísio Doles Marubo and Armando Mariano Marubo, as well as with other important prayer-shamans (Antonio Brasil Marubo, Lauro Brasil Marubo, and Paulino Joaquim Marubo). The now deceased prayer-shaman, Armando, authored the narrative that follows, which connects a traditional narrative about the formation of the Death Path (\textit{Vei Vai yoã}) with an instructive speech (\textit{ese vana}) about eschatological conceptions. The narrative was performed and recorded at Alegria village (Upper Ituí river) in 2007, after three years of collaboration between author and researcher. The original audio digital recording was transcribed, reviewed, and translated with the help of Robson Dionísio, a shaman, bilingual researcher and schoolteacher. The complete literary translation of this narrative was published elsewhere in Portuguese \citep{Cesarino2012} and can be compared with a sung version of the Death Path narrative, also previously translated and published \citep[303ff]{Cesarino2011}. The present version revises and adds details to the original narrative, including the unpublished interlinear segmentation.

The first part of Armando's narrative synthesizes the formation of the Death Path by \textit{Vei Maya} and the tree spirits; the second part connects this narrative with the moral teachings involved in the journey along this dangerous path. This is the kind of teaching that Marubo youngers should attend to, so as to prepare themselves for the afterlife. The lines of narrative were divided according to rhythm and parallelism, in order to reproduce the dramatic effect, a prominent characteristic of the original oral performance, in the written version. One of the central features conveying this effect (which is also didactic) is the extensive use of reported speech, which allows the narrator to shift between voices, be it the voice of an ancestor (as in line 25) or of a generic dead person (as in line 71). It is also important to note that the first Portuguese translation, from which this detailed and segmented version is derived, was actually conceived to be literary rather than completely literal.

Linguistic data on the Marubo language was first obtained through Costa's (\citeyear{Costa1992}, \citeyear{Costa1998}, \citeyear{Costa2000}) preliminary phonological and morphological research, which I later revised and expanded for my ethnographic research and verbal arts translations. My research was also based on other linguistic studies of Panoan languages (\citealt{Valenzuela2003} for Shipibo-Conibo; \citealt{Fleck2003} for Matsés; \citealt{Camargo1995,Camargo1996a,Camargo1996b,Camargo1998,Camargo2003,Camargo2005} for Kaxinawá, among others), as well as on a revision of the orthographic conventions used for the Marubo language by the New Tribes Mission linguists since 1950. Despite the traits that it shares with other Panoan languages -- such as agglutinative morphology, easily identifiable morpheme boundaries, the presence of ergative-absolutive case marking, and a complex switch-reference system that distinguishes same/different subjects and sequential/simultaneous actions -- the classification of the Marubo language within the Panoan linguistic group is still being debated \citep[55]
{Valenzuela2003}. The Marubo phonemic system, with orthographic conventions indicated by <>, is composed of fourteen consonants (p <p>, m <m>, v <v>, w <w>, t <t>, n <˜>, s <s>, ts <ts>, ɾ <r>, ʃ <sh>, tʃ <tx>, ʂ < ch>, y < y >, k <k>) and four vowels (i <i>, ɨ (e), a <a>, u <o>).

      \section{The teachings of the Death-Path}
\translatedtitle{Ensinamentos do Caminho-Morte}

\noindent
I. A história de Vei Maya%\footnote{Recordings of this story are available from \url{some url}}%no media files available for this story


\ea Txõtxõ Koro shavo, winin aká shavo, \\[.3em]
\gll Txõtxo Koro shavo winin    aká               shavo \\
     bird   grey women erection \textsc{aux.trns} women \\
\glt `Bird-women, seductive women,' \\
`Mulheres-pássaro, as mulheres sedutoras,' \\
\z

\ea Atisho vei ooki vei oo atisho. \\[.3em]
\gll a-ti-sho                vei   oo-iki            vei   oo  a-ti-sho                \\
     3\textsc{dem-nmlz-sssa} death cry-\textsc{vblz} death cry 3\textsc{dem-nmlz-sssa} \\
\glt `women of the death-cry, of the death-cry.' \\
`aquelas que soltaram o grito-morte, aquelas do grito-morte.' \\
\z

\ea Aivo askásevi Vei Maya askásevi, \\[.3em]
\gll a-ivo              aská-sevi        Vei   Maya aská-sevi        \\
     3\textsc{dem-genr} \textsc{sml-con} death Maya \textsc{sml-con} \\
\glt `And her also, Vei Maya also,' \\
`e também ela, Vei Maya também,' \\
\z

\ea Vei Maya vei mai nãkõsh wenímarivi, shavo wetsa. \\[.3em]
\gll Vei   Maya vei   mai  nãkõ-sh                  wení-ma-rivi          shavo wetsa \\
     death Maya death land nectar-\textsc{loc.prov} rise-\textsc{neg-emp} woman other \\
\glt `Vei Maya did not come from the Death-Land; she is another woman.'{\footnotemark} \\
`Vei Maya não surgiu do néctar da terra-morte, é outra mulher.' \\
\footnotetext{The narrator is saying that \textit{Vei Maya} was not born in the Death-Land that she later created. `Nectar' (\textit{nãko}) is a shamanic term for a special transformational substance.}
\z

\newpage 
\ea Aská akĩ, aská akĩ, isĩ akĩ, \\[.3em]
\gll aská         a-kĩ             aská         a-kĩ                   isĩ    a-kĩ                   \\
     \textsc{sml} do-\textsc{sssa} \textsc{sml} \textsc{aux.trns-sssa} strong \textsc{aux.trns-sssa} \\
\glt `Doing this, doing this, doing this strongly,' \\
`Fazendo assim, fazendo assim, fazendo forte,' \\
\z

\ea aská akĩ, isĩ akĩ rishkikinã. \\[.3em]
\gll aská         a-kĩ                   isĩ    a-kĩ                   rishki-ki-nã          \\
     \textsc{sml} \textsc{aux.trns-sssa} strong \textsc{aux.trns-sssa} beat-\textsc{ass-foc} \\
\glt `doing this strongly, her husband beat her.' \\
`fazendo assim, fazendo forte, [o marido] ia mesmo espancando.' \\
\z

\ea Awẽ amaĩnõ wetsarotsẽ a venemesh merasho rishkiti tenãi. \\[.3em]
\gll awẽ  a-maĩnõ               wetsa-ro-tsẽ           a              vene-mesh mera-sho           rishki-ti         tenã-i            \\
     what \textsc{aux.trns-con} other-\textsc{top-con} \textsc{3.dem} man-?     find-\textsc{sssa} beat-\textsc{ins} \textsc{die-pst1} \\
\glt `And doing so, the husband killed his other wife.' \\
`E assim fazendo, o homem matou a sua outra mulher.' \\
\z

\ea Askámãi wetsarotsẽ, wetsa westí tsaopakea aivorotsẽ, \\[.3em]
\gll aská-mãinõ       wetsa-ro-tsẽ,          wetsa westí tsao-pake-a             a-ivo-ro-tsẽ                \\
     \textsc{sml-con} other-\textsc{top-con} other alone seat-\textsc{distr-rlz} 3.\textsc{dem-genr-top-con} \\
\glt `But the other, the one who sat alone,' \\
`Mas a outra, aquela que ficou sozinha sentada,' \\
\z

\ea aro awẽ vene rishkia. \\[.3em]
\gll a-ro               awẽ           vene rishki-a          \\
     3.\textsc{dem-top} \textsc{poss} man  beat-\textsc{rlz} \\
\glt `the husband beat.' \\
`o marido nela bateu.' \\
\z

  \largerpage
\ea Awẽ chinã naíai tsaõ, \\[.3em]
\gll awẽ           chinã   naí-ai            tsaõ                \\
     \textsc{poss} thought sad-\textsc{incp} seated.\textsc{loc} \\
\glt `There she sat with a sad thought,' \\
`Ficou sentada com o pensamento entristecido,' \\
\z

\ea vei ari kenai, vei ari kenai. \\[.3em]
\gll vei   a-ri                kena-i             vei   a-ri                kena-i             \\
     death 3.\textsc{dem-refl} call-\textsc{prog} death 3.\textsc{dem-refl} call-\textsc{prog} \\
\glt `alone, calling for death, alone, calling for death.' \\
`pela morte sozinha chamava, pela morte sozinha chamava.' \\
\z

\ea Vei Mayanã. \\[.3em]
\gll Vei   Maya-nã           \\
     death Maya-\textsc{foc} \\
\glt \textbf{`}This is Vei Maya.' \\
`É Maya-Morte.' \\
\z

\ea Aivo vei ari kenaiti. \\[.3em]
\gll a-ivo               vei   a-ri                kena-i-ti               \\
     3.\textsc{dem-genr} death 3.\textsc{dem-refl} call-\textsc{prog-pst5} \\
\glt `The one that, in the old times, called for death.' \\
`A que há tempos pela morte chamava.' \\
\z

\ea Aská akĩserotsẽ ari iniki vanai. \\[.3em]
\gll aská         a-kĩ-se-ro-tsẽ                     a-ri                iniki vana-i               \\
     \textsc{sml} \textsc{aux.trns-sssa-ext-top-con} 3\textsc{.dem-refl} song  speech-\textsc{prog} \\
\glt `This way she called, this way she sang for herself.' \\
`Assim mesmo chamando, ela sozinha cantofalava.'  \\
\z

\ea Ronorasĩ kenaiti. \\[.3em]
\gll rono-rasĩ          kena-i-ti               \\
     snake-\textsc{col} call-\textsc{prog-pst5} \\
\glt `She called for the snakes long ago.' \\
`Chamava pelas cobras,' \\
\z

  
\ea Vanavanakwãi avai kayakãisho \\[.3em]
\gll vana-vana-kawã-i             a-vai                 kaya-kãi-sho            \\
     speak-speak-go-\textsc{prog} \textsc{aux.trns-con} leave-\textsc{inc-sssa} \\
\glt `Calling and calling she left' \\
`falando e falando foi saindo,'\\
\z

\newpage 
\ea kayã nachima. \\[.3em]
\gll kayã               nachi-ma            \\
     river.\textsc{loc} bathe-\textsc{caus} \\
\glt `to bathe in the river.' \\
`foi banhar no rio.'\\
\z

\ea A nachia tsaosmãis, a rono anõ rakákawãs nachai. \\[.3em]
\gll a              nachi-a            tsao-se-mãinõs       a              rono  anõ          raká-kawãs nacha-i            \\
     3.\textsc{dem} bathe-\textsc{rlz} seat-\textsc{ext-con} 3.\textsc{dem} snake \textsc{fin} lie-go     bite-\textsc{pst1} \\
\glt `While she sat to bathe, a passing snake bit her.' \\
`Enquanto sentava-se para banhar, uma cobra que ali ficava a mordeu.' \\
\z

\ea Tenãseiti. \\[.3em]
\gll tenã-se-iti           \\
     die-\textsc{ext-pst5} \\
\glt `And she died a long time ago.'\footnotemark \\
`Morreu mesmo há muito tempo.' \\
\z

\footnotetext{There are at least three verbal forms for `death' in the present text: \textit{vopia}, `to die in this world'; \textit{veia}, a `second possible death and/or transformation in the afterlife'; \textit{tenãia}, `to be physically injured to the point of death'.}

\ea Aská akaivo voshõ, \\[.3em]
\gll aská         aka-i-vo                  vo-shõ                 \\
     \textsc{sml} \textsc{aux.trns-prog-pl} arrive.\textsc{pl-dssa} \\
\glt `And then they arrived.' \\
`E assim então eles chegaram,' \\
\z

\ea Shono Yove Nawavo pakeivo paraiki voshõ. \\[.3em]
\gll Shono        Yove   Nawa-vo            pake-i-vo             para-iki                vo-shõ.                \\
     samaúma.tree spirit people-\textsc{pl} fall-\textsc{prog-pl} come.down-\textsc{vblz} arrive.\textsc{pl-con} \\
\glt `Samaúma Spirit-People were coming down, arriving.' \\
`O Povo-Espírito da Samaúma foi descendo, chegando.' \\
\z

\newpage
\ea Anosho chinãi, \\[.3em]
\gll ano-sho                 chinã-i             \\
     there-\textsc{loc.prov} think-\textsc{prog} \\
\glt `And there she thought,' \\
`E ali elá pensou,' \\
\z

\ea ato chinãmakĩ, \\[.3em]
\gll ato              chinã-ma-kĩ              \\
     3\textsc{pl.dem} think-\textsc{caus-sssa} \\
\glt `about them she was thinking,' \\
`Sobre eles ficou pensando,' \\
\z

\ea ato chinãmakĩ. \\[.3em]
\gll ato              chinã-ma-kĩ              \\
     3\textsc{pl.dem} think-\textsc{caus-sssa} \\
\glt `About them she was thinking.'\footnotemark \\
`Sobre eles pensou.' \\
\z

\footnotetext{The Samaúma Spirit-People came down from the \textit{Tama Shavá}, a dwelling in the tree canopies, a better world to which all the deceased were destined in ancient times, regardless of their moral qualities. \textit{Vei Maya} is outraged with this common destiny and thus provokes an eschatological transformation. Samaúma (\textit{ceiba petandra}) is one of the tallest Amazonian trees; its spirit-people are some of the most important in Marubo shamanism. The next two trees mentioned in the narrative could not be identified in Portuguese, but the Marubo used to call the \textit{chai} tree with the regional term “envireira".}

\ea ``Ramaro nokẽ chinã naíai, nõ neskái, \\[.3em]
\gll rama-ro         nokẽ             chinã   naí-ai            nõ               neská-ai          \\
     now-\textsc{top} \textsc{1pl.gen} thought sad-\textsc{incp} \textsc{1pl.gen} \textsc{sml-incp} \\
\glt ‘“Now our thought saddened, so we became,' \\
`“Agora que ficamos com o pensamento entristecido,' \\
\z

\ea noke neská akavo, noke. \\[.3em]
\gll noke             neská        aka-vo               noke             \\
     \textsc{1pl.abs} \textsc{sml} \textsc{aux.trns-pl} \textsc{1pl.abs} \\
\glt `now we will do it this way.' \\
`agora vamos fazer assim.'\\
\z

\ea Txipo shavá otapa roai askátanivai ari shavámisvo. \\[.3em]
\gll txipo shavá otapa roa-i                 aská-ta-ni-vai         a-ri                shavá-misi-ivo            \\
     after time  come  sorcery-\textsc{prog} \textsc{sml-ass-?-con} 3.\textsc{dem-refl} live-\textsc{possib-genr} \\
\glt `The future we will change, so that they might suffer.' \\
`A época que virá vamos transformar para que os outros sofram.'\\
\z

\ea Vei Vai arina shovimakĩ! \\[.3em]
\gll Vei   Vai  a-ri-na                 shovi-ma-ki                  \\
     death path \textsc{aux.trns-imp-?} make.built-\textsc{caus-ass} \\
\glt `Come and make the Death-Path!' \\
`Vamos, façam logo o Caminho-Morte!' \\
\z

\ea Vei Vai arina shovimakĩ!'' ikiti. \\[.3em]
\gll Vei   Vai  a-ri-na            shovi-ma-ki                  iki-ti            \\
     death path \textsc{aux-imp-?} make.built-\textsc{caus-ass} say-\textsc{pst5} \\
\glt `Come and make the Death-Path!" she commanded long ago.' \\
`Façam logo o Caminho-Morte!”, disse ela há muito tempo.'\\
\z

\ea Askáka akátõsh tanamakinãnãi. \\[.3em]
\gll aská-aka              aká-tõsho             tana-ma-iki-nãnã-i                                 \\
     \textsc{sml-aux.trns} \textsc{aux.trns-cns} understand.decide-\textsc{caus-ass-recp-pst1} \\
\glt `And they arranged everything amongst themselves.' \\
`Assim eles entre si tudo combinaram.' \\
\z

\ea Chai Yove Nawavo, \\[.3em]
\gll Chai           Yove   Nawa-vo            \\
     envireira.tree spirit people-\textsc{pl} \\
\glt `Spirit People of the Envireira Tree,' \\
`Povo-Espírito da Envireira,' \\
\z

\ea Shono Yove Nawavo, \\[.3em]
\gll Shono        Yove   Nawa-vo            \\
     samaúma.tree spirit people-\textsc{pl} \\
\glt `Spirit People of the Samaúma Tree,' \\
`Povo-Espírito da Samaúma,' \\
\z

\ea Tama Yove Nawavo, \\[.3em]
\gll Tama Yove   Nawa-vo            \\
     tree spirit people-\textsc{pl} \\
\glt `Spirit People of the Tama Tree,' \\
`Povo-Espírito das Árvores,'\\
\z

\ea ati tanamakinãnãvaikis, \\[.3em]
\gll a-ti                tana-ma-iki-nãnã-vaikis                      \\
     3.\textsc{dem-nmlz} understand.decide-\textsc{caus-aux-recp-con} \\
\glt `they decided everything amongst themselves,' \\
`são estes os que entre si tudo combinaram,'\\
\z

\ea awẽ vana anõkis akavo \\[.3em]
\gll awẽ           vana   anõ-ki-se            aka-vo               \\
     \textsc{poss} speech \textsc{fin-ass-ext} \textsc{aux.trns-pl} \\
\glt `and obeyed her' \\
`a ordem obedeceram e fizeram,'\\
\z

\ea Vei Vai shovimakĩ. \\[.3em]
\gll Vei   Vai  shovi-ma-kĩ            \\
     death path make-\textsc{caus-ass} \\
\glt `building the Death-Path.' \\
`construíram o Caminho-Morte.'\\
\z

\ea Atiãro yora veiya roase, \\[.3em]
\gll atiã-ro          yora   vei-ya           roa-se            \\
     \textsc{temp-top} people die-\textsc{prf} easy-\textsc{ext} \\
\glt `At that time people died easily,' \\
`Naquela época as pessoas morriam tranquilas,' \\
\z

\ea Vopitani tachikrãse, \\[.3em]
\gll vopi-ta-ni        tachi-krã-se              \\
     die-\textsc{ass-?} arrive-\textsc{dir.c-ext} \\
\glt `died and arrived there {[}in the world in the tree canopies{]},' \\
`faleciam e já chegavam [na Morada Arbórea],' \\
\z

\newpage 
\ea vopitani tachikrãseika. \\[.3em]
\gll vopi-ta-ni           tachi-krã-se-i-ka                \\
     death-\textsc{ass-?} arrive-\textsc{dir.c-ext-pst1-?} \\
\glt `died and really arrived there.'\footnotemark \\
`faleciam e já chegavam mesmo .' \\
\z

\footnotetext{In the world of the tree canopies.}

\ea Akámẽkirotsẽ ãtõ atovo, \\[.3em]
\gll aká-mẽkĩ-ro-tsẽ               ãtõ                  ato-vo              \\
     \textsc{aux.trns-con-top-con} 3\textsc{pl.dem.erg} 3\textsc{pl.abs-pl} \\
\glt `So it was, but then they made it,' \\
`Assim era, mas ela ordenou e fizeram,'\\
\z

\ea Vei Vai aská akĩ shovimai akavo. \\[.3em]
\gll Vei   Vai  aská         a-kĩ                   shovi-ma-i               aka-vo               \\
     death path \textsc{sml} \textsc{aux.trns-sssa} built-\textsc{caus-prog} \textsc{aux.trns-pl} \\
\glt `Death-Path they made.' \\
`construíram o Caminho-Morte.'\\
\z

\ea Shovo Yove Nawavo aská vei chinãya shokoma, \\[.3em]
\gll Shovo        Yove   Nawa-vo            aská         vei   chinã-ya                  shoko-ma          \\
     samaúma.tree spirit people-\textsc{pl} \textsc{sml} death thought-\textsc{atr.perm} live-\textsc{neg} \\
\glt `Samaúma Spirit People do not live with death-thought,' \\
`Povo-Espírito da Samaúma não vive assim com pensamento-morte,'\\
\z

\ea Tama Yove Nawavo vei chinãya shokoma, \\[.3em]
\gll Tama Yove   Nawa-vo            vei   chinã-ya                  shoko-ma          \\
     tree spirit people-\textsc{pl} death thought-\textsc{atr.perm} live-\textsc{neg} \\
\glt `Tama Spirit People do not live with death-thought,' \\
`Povo-Espírito das Árvores não vive com pensamento-morte,'\\
\z

\ea Chai Yove Nawavo vei chinãya shokoma. \\[.3em]
\gll Chai           Yove   Nawa-vo            vei   chinã-ya                  shoko-ma          \\
     envireira.tree spirit people-\textsc{pl} death thought-\textsc{atr.perm} live-\textsc{neg} \\
\glt `Chai Spirit People do not live with death-thought.' \\
`Povo-Espírito da Envireira não vive com pensamento-morte.' \\
\z

\ea Akámẽkĩtsẽ ãtõ ato vanaka, \\[.3em]
\gll aká-mẽki-tsẽ              ãtõ                  ato              vana   aka               \\
     \textsc{aux.trns-con-con} \textsc{3pl.dem.erg} \textsc{3pl.abs} speech \textsc{aux.trns} \\
\glt `So they are, but she commanded,' \\
`Assim mesmo são, mas ela os ordenou,' \\
\z

\ea chinãmakinãnãvaikis akavo, \\[.3em]
\gll chinã-ma-ki-nãnã-vaikis          aka-vo               \\
     think-\textsc{caus-ass-recp-con} \textsc{aux.trns-pl} \\
\glt `and they decided amongst themselves,' \\
`eles pensaram entre si e então fizeram,' \\
\z

\ea a vai shovimakinã. \\[.3em]
\gll a              vai  shovi-ma-ki-nã              \\
     3.\textsc{dem} path built-\textsc{caus-ass-foc} \\
\glt `and built that path.' \\
`construíram aquele caminho.'\\
\z

\ea Atõ aská ati, \\[.3em]
\gll atõ              aská         a-ti              \\
     \textsc{3pl.dem} \textsc{sml} \textsc{aux-pst5} \\
\glt `Long ago they made it,' \\
`Assim há tempos fizeram,' \\
\z

\ea atõ aská atisho. \\[.3em]
\gll atõ              aská         a-ti-sho                    \\
     3\textsc{pl.dem} \textsc{sml} \textsc{aux.trns-pst5-sssa} \\
\glt `Long ago they did it.' \\
`assim há tempos eles fizeram.' \\
\z

\ea Akĩ vai roa aina, vai roakama, \\[.3em]
\gll a-kĩ                vai  roa     a-ina                     vai  roaka-ma          \\
     3.\textsc{dem-sssa} path arrange \textsc{aux.trns-con.fin} path good-\textsc{neg} \\
\glt `The path they arranged, a bad path,' \\
`Ajeitaram o caminho, caminho ruim,' \\
\z

\newpage 
\ea anõsh txipo kaniaivo askái shavánõ, \\[.3em]
\gll anõ-sh           txipo kania-ivo          askái        shavá-nõ          \\
     for-\textsc{dat} after grow-\textsc{genr} \textsc{sml} live-\textsc{fin} \\
\glt `so that the youngsters might experience it,' \\
`para que os depois nascidos padeçam,'\\
\z

\ea txipo kaniaivo anõ yostánõ. \\[.3em]
\gll txipo kania-ivo          anõ yostá-nõ            \\
     after grow-\textsc{genr} for suffer-\textsc{fin} \\
\glt `So that they suffer.' \\
`Para que os depois nascidos sofram.' \\
\z

\noindent
II. A travessia%\footnote{Recordings of this story are available from \biberror{\url{some url }}}%no media files available for this story


\ea Wetsaro vei ikitai, \\[.3em]
\gll wetsa-ro           vei   iki-ta-i              \\
     other-\textsc{top} death \textsc{cop-ass-pst1} \\
\glt `This one is dead,' \\
`Um já está morrido,' \\
\z

\ea wetsaro vei ikitai, \\[.3em]
\gll wetsa-ro           vei   iki-ta-i              \\
     other-\textsc{top} death \textsc{cop-ass-pst1} \\
\glt `this one is dead,' \\
`outro já está morrido,'\footnotemark{}\\
\footnotetext{The Portuguese “morrido” translates the difference between two possible deaths conceived by Marubo eschatology: the death of the carcass-body (\textit{vopia}) and the death of the double (\textit{veia}). The first is translated as `morto' and the second one as `morrido', thus mirroring a popular Brazilian  expression that also distinguishes two kinds of death: “morte matada e morte morrida”.} 
\z

\ea wetsaro vei matsá pakei, \\[.3em]
\gll wetsa-ro           vei   matsá pake-i             \\
     other-\textsc{top} death mud   fall-\textsc{pst1} \\
\glt `the other one has fallen in the death-mud,' \\
`Outro caiu no lamaçal-morte,'\\
\z

\newpage 
\ea wetsaro vimi noiaivo, \\[.3em]
\gll wetsa-ro           vimi  noia-ivo           \\
     other-\textsc{top} fruit like-\textsc{genr} \\
\glt `the other, fond of fruit,' \\
`outro, o fã de frutas,' \\
\z

\ea awẽ vimi amaĩnõ anosho atxitai. \\[.3em]
\gll awẽ           vimi  a-maĩnõ               ano-sho                 atxi-ta-i               \\
     \textsc{poss} fruit \textsc{aux.trns-con} there-\textsc{loc.prov} stuck-\textsc{ass-pst1} \\
\glt `became stuck in the fruit.' \\
`come a fruta e ali mesmo fica preso.' \\
\z

\ea Akáakarasĩ aská atõ veikãse aya. \\[.3em]
\gll aká-aká-rasĩ                   aská         atõ              vei-kãia-se          aya \\
     \textsc{aux.trns-aux.trns-col} \textsc{sml} 3\textsc{pl.dem} \textsc{die-inc-ext} be  \\
\glt `Doing this and that they keep dying there.' \\
`Assim fazendo eles ali ficam morridos.' \\
\z

\ea Askámãi yora ese vanaya, \\[.3em]
\gll aská-mãinõ       yora   ese    vana-ya                  \\
     \textsc{sml-con} person wisdom speech-\textsc{atr.perm} \\
\glt `But the person with wise speech,' \\
`Mas a pessoa de fala sabida,' \\
\z

\ea yora vanaya, \\[.3em]
\gll yora   vana-ya                  \\
     person speech-\textsc{atr.perm} \\
\glt `the talkative person,' \\
`a pessoa faladora,' \\
\z

\ea vana shatesmaivo yora, \\[.3em]
\gll vana   shate-se-ma-ivo           \\
     speech cut-\textsc{ext-neg-genr} \\
\glt `the person of constant speech,' \\
`a pessoa de fala firme,' \\
\z

\newpage 
\ea yora akáro aská: \\[.3em]
\gll yora   aká-ro                aská         \\
     person \textsc{dem.genr-top} \textsc{sml} \\
\glt `this person is like this:' \\
`esta é assim.' \\
\z

\ea aro na mai shavápashõ nishõ, \\[.3em]
\gll a-ro na        mai  shavá-pa-shõ                   ni-shõ             \\
     3.\textsc{dem-top} \textsc{dem.prox} land dwelling-\textsc{loc-loc.prov} live-\textsc{dssa} \\
\glt `having lived in this land,' \\
`Esta, tendo vivido nesta terra,' \\
\z

\ea wa shavo kai wetsa, \\[.3em]
\gll wa                shavo ka-i             wetsa   \\
     \textsc{dem.dist} woman go-\textsc{prog} another \\
\glt `with that woman,' \\
`com aquela mulher,' \\
\z

\ea wa shavo kai wetsa, \\[.3em]
\gll wa                shavo ka-i             wetsa   \\
     \textsc{dem.dist} woman go-\textsc{prog} another \\
\glt `with that other woman,' \\
`com aquela mulher,' \\
\z

\ea wa shavo kai wetsa, akama. \\[.3em]
\gll wa                shavo ka-i             wetsa   aka-ma           \\
     \textsc{dem.dist} woman go-\textsc{prog} another \textsc{aux-neg} \\
\glt `with that other woman, he didn't go.' \\
`com aquela outra mulher não saía.' \\
\z

\ea Mato mã aĩ viá keská, \\[.3em]
\gll mato            mã       aĩ    viá  keská        \\
     2\textsc{pl.abs} \textsc{2pl.erg} woman take \textsc{sml} \\
\glt `Just like when you choose a woman,'\footnotemark \\
`Como vocês que escolhem as suas mulheres,' \\
\z

\footnotetext{This is a reference to me (Cesarino) and monogamous white people.}

\ea a westí verõsho oĩa akavo, \\[.3em]
\gll a              westí    verõ-sho         oĩa aka-vo               \\
     3.\textsc{dem} only.one eye-\textsc{con} see \textsc{aux.trns-pl} \\
\glt `person who looks with only one eye'\footnotemark{} \\
`pessoas que olham com apenas um olho,' \\
\footnotetext{``Person who looks with only one eye" is a metaphor for those who search for only one women, as white people. The Marubo polygamy was once restricted to shamans and chiefs but nowadays is practised with more relaxed criteria, which produces this kind of criticism by elder shamans as Armando.} 
\z

\ea yoratsẽ Vei Maya vei akatĩpa, \\[.3em]
\gll yora-tsẽ            Vei   Maya vei   aka-tĩpa                 \\
     person-\textsc{con} death Maya death \textsc{aux.trns-imposs} \\
\glt `this kind of person, Vei Maya cannot hold,' \\
`esse tipo de pessoa Vei Maya não consegue pegar' \\
\z

\ea askárasĩ vei akatĩpa. \\[.3em]
\gll aská-rasĩ        vei   aka-tĩpa                 \\
     \textsc{sml-col} death \textsc{aux.trns-imposs} \\
\glt `this kind of person cannot die.' \\
`pessoas assim não podem ficar morridas.' \\
\z

\ea Wa mai shavápashõ, \\[.3em]
\gll wa                mai  shavá-pa-shõ                   \\
     \textsc{dem.dist} land dwelling-\textsc{loc-loc.prov} \\
\glt ‘“In that land,' \\
`{}``Na morada daquela terra,' \\
\z

\ea wa mai shavapashõ, \\[.3em]
\gll wa                mai  shavá-pa-shõ                   \\
     \textsc{dem.dist} land dwelling\textsc{-loc-loc.prov} \\
\glt `in that land,' \\
`na morada daquela terra,' 
\z

\largerpage
\ea wa shavo kai wetsa, \\[.3em]
\gll wa                shavo ka-i              wetsa   \\
     \textsc{dem.dist} woman go-\textsc{prog} another \\
\glt `with that and that woman,' \\
`com aquela mulher,' \\
\z

\newpage 
\ea wa shavo kai wetsa, \\[.3em]
\gll wa                shavo ka-i             wetsa   \\
     \textsc{dem.dist} woman go-\textsc{prog} another \\
\glt `with that and that woman,' \\
`com aquela mulher,' \\
\z

\ea ẽ yora onã shavorasĩ, \\[.3em]
\gll ẽ                yora   onã  shavo-rasĩ         \\
     1\textsc{sg.gen} people know women-\textsc{col} \\
\glt `with my relatives' wives,' \\
`com mulheres conhecidas,'\\
\z

\ea akĩ ichná kawãi ẽ niámarvi. \\[.3em]
\gll a-kĩ                   ichná kawã-i         ẽ                niá-ma-rivi           \\
     \textsc{aux.trns-sssa} bad   go-\textsc{prog} \textsc{1sg.erg} live-\textsc{neg-emp} \\
\glt `I did not live by flirting.' \\
`eu não fiquei mesmo fazendo besteira.' \\
\z

\ea ẽ oĩtivoivo, \\[.3em]
\gll ẽ                oĩ-ti-vo-ivo                     \\
     1\textsc{sg.gen} see.choose-\textsc{nmlz-pl-genr} \\
\glt `Only with my chosen one,' \\
`Apenas com a minha escolhida,'\\
\z

\ea shavo ninivarãsh, \\[.3em]
\gll shavo ni-ni-varã-sh                       \\
     woman live-live-\textsc{dir.c-dspa} \\
\glt `the woman that I brought to live with me,' \\
`A mulher que eu trouxe para viver comigo,' \\
\z

\ea aivo shavo oĩ inishõ neskái. \\[.3em]
\gll a-ivo               shavo oĩ  i-ni-shõ                neská-i           \\
     3.\textsc{dem-genr} woman see \textsc{aux-assoc-dssa} \textsc{sml-pst1} \\
\glt `only with this one I've lived.' \\
`por ter vivido apenas com ela é que fiquei assim.' \\
\z

\newpage 
\ea Vei kayapai ẽ neskámaĩnõ. \\[.3em]
\gll vei   kaya-pai                     ẽ                neská-maĩnõ      \\
     death true.principal-\textsc{comp} \textsc{1sg.gen} \textsc{sml-con} \\
\glt `An honest dead I now am.' \\
`Por isso agora sou morto íntegro.'\\
\z

\ea  Matõ neskánamãsh, ea vei akatĩpa ea. \\[.3em]
\gll matõ             neská-namã-sh           ea               vei   aka-tĩpa                 ea               \\
     2\textsc{pl.gen} \textsc{sml-loc-dspa}   1\textsc{sg.abs} death \textsc{aux.trns-imposs} \textsc{1sg.abs} \\
\glt `In this place of yours, you cannot kill me.''{}' \\
`Por isso vocês aqui não podem, não podem me matar.''{}'\\
\z

\ea Ikitõ awẽ ese vanase ainai, \\[.3em]
\gll iki-tõ           awẽ           ese    vana-se             a-ina-i                                \\
     say-\textsc{cns} \textsc{poss} wisdom speech-\textsc{ext} \textsc{aux.intr-mov}.up-\textsc{prog} \\
\glt `His wise words he says ascending,'\footnotemark \\
`Assim ele vai então dizendo sua fala sabida,'\\
\z

\footnotetext{This refers to the speech of a deceased person, who is crossing the path.}

\ea awẽ ese vanase vevo ashõ kai, \\[.3em]
\gll awẽ           ese    vana-se             vevo   a-shõ                  ka-i           \\
     \textsc{poss} wisdom speech-\textsc{ext} before \textsc{aux.trns-dssa} go-\textsc{pr} \\
\glt `with wise words he goes,' \\
`tendo dito sua fala sabida ele avança,' \\
\z

\ea katsese vana ikitai tapi, \\[.3em]
\gll katsese    vana   iki-ta-i              tapi \\
     everything speech say-\textsc{ass-prog} go   \\
\glt `speaking with everything he continues,'\footnotemark \\
`Falando com tudo ele segue,'\\
\z

\footnotetext{He refers to all of the path's dangers, which the dead should know in their numerous forms (\textit{shovia}). The person should acquire this knowledge during his/her life in order to face the challenges of the afterlife.}

\newpage 
\ea awá shao tapã vana ikitase, \\[.3em]
\gll awá   shao tapã   vana   iki-ta-se            \\
     tapir bone bridge speech say-\textsc{ass-ext} \\
\glt `speaking with the tapir bone bridge,' \\
`com a ponte de osso de anta ele fala,'\\
\z

\ea awá shao tapã masotanáiri \\[.3em]
\gll awá   shao tapã   maso-taná-iri              \\
     tapir bone bridge upon-arranged-\textsc{dir} \\
\glt `with the sharp shell heap,' \\
\glt `coma as cortantes conchas,' \\
\z

\ea pao shokoarasĩ vana ikitase \\[.3em]
\gll pao   shokoa-rasĩ       vana   iki-ta-se            \\
     shell heap-\textsc{col} speech say-\textsc{ass-ext} \\
\glt `above the tapir bridge he speaks,'\footnotemark \\
\glt `sobre a ponte de ossos de anta ele fala,' \\
\z

\footnotetext{A heap of shells that cut and kill the dead.}

\ea vei yochĩrasĩ vanaainase \\[.3em]
\gll vei   yochĩ-rasĩ          vana-a-ina-se                           \\
     death spirit-\textsc{col} speech-\textsc{rel-mov}.up-\textsc{ext} \\
\glt `speaking with the dead spirits he goes,' \\
`com todos os espectros-morte ele fala,' \\
\z

\ea vimirasĩ vanaainase. \\[.3em]
\gll vimi-rasĩ          vana-a-ina-se                           \\
     fruit-\textsc{col} speech-\textsc{rel-mov}.up-\textsc{ext} \\
\glt `speaking with the fruits he goes.'\footnotemark \\
`com os frutos todos ele fala.'\\
\z

\footnotetext{Death-fruits (\textit{vei vimi}) that he might eat instead of continuing his ascent.}

\newpage 
\ea  Wa mai shavápashõ, vimi ichnárasĩ yaniakĩ niáma, \\[.3em]
\gll wa                mai  shavá-pa-shõ                   vimi  ichná-rasĩ       õsipa  yania-kĩ           niá-ma            \\
     \textsc{dem.dist} land dwelling-\textsc{loc-loc.prov} fruit bad-\textsc{col} varied feed-\textsc{sssa} live-\textsc{neg} \\
\glt ‘“In that land, I didn't live by eating bad and varied fruit,' \\
`“Naquela terra, não vivi me alimentando de ruins e fartos frutos.'\\
\z

\ea eri píti koĩ meramashõrivi ea anõ yanini. \\[.3em]
\gll e-ri              píti koĩ       mera-ma-shõ-rivi          ea               anõ          yani-ini        \\
     1\textsc{sg-refl} food real.true find-\textsc{cs-sspa-emp} \textsc{1sg.abs} \textsc{fin} feed-\textsc{?} \\
\glt `I've worked to have my own real food.' \\
`Eu mesmo procurava comida de verdade para me alimentar.' \\
\z

\ea Aki ea anõ mato ea mã veikatĩpa. \\[.3em]
\gll a-ki             ea               anõ          mato            ea               mã              vei-aka-tĩpa                   \\
     \textsc{aux-ass} \textsc{1sg.abs} \textsc{fin} \textsc{2pl.abs} \textsc{1sg.abs} \textsc{2pl.erg} death-\textsc{aux.trns-imposs} \\
\glt `That's how I've lived, so you cannot kill me.''' \\
`É assim que sou, vocês não podem me matar!”'\\
\z

\ea A kaisa vanania. \\[.3em]
\gll a              kai-sa vana-ina               \\
     3.\textsc{dem} go-?   speech-\textsc{mov}.up \\
\glt `There he goes ascending and speaking.' \\
`Assim ele sobe falando.' \\
\z

\ea Vei shõparasĩ askásevi, \\[.3em]
\gll Vei   shõpa-rasĩ         aská-sevi \\
     death papaya-\textsc{col} \textsc{sml-con}  \\
\glt `With papaya-death also,' \\
`Com os mamãos-morte também,'\\
\z

\ea askárasĩ awe kẽvo anõ inã askásevi, \\[.3em]
\gll aská-rasĩ        awe   kẽ-vo             anõ  inã   aská-sevi        \\
     \textsc{sml-col} thing desire-\textsc{pl} \textsc{fin} offer \textsc{sml-con} \\
\glt `with all the alluring things also,' \\
`com todas as coisas sedutoras também,'\\
\z

\ea askásevi askásevi vana akitase kãi, \\[.3em]
\gll aská-sevi       aská-sevi vana-a-ki-ta-se                  kãi \\
     \textsc{sml-con} \textsc{sml-con}  speech-\textsc{rel-sssa-ass-ext} go  \\
\glt `with all the things he speaks and speaks,' \\
`e também e também, com tudo ele vai mesmo falando.' \\
\z

\ea vanaarasĩ nokorivi, \\[.3em]
\gll vana-a-rasĩ             noko-rivi           \\
     speech-\textsc{rlz-col} arrive-\textsc{emp} \\
\glt `speaking with everything he arrives,' \\
`Falando com tudo ele chega mesmo,' \\
\z

\ea ese vanase vevo oshõ kãi nokorivi. \\[.3em]
\gll ese    vana-se             vevo   o-shõ              kã-i             noko-rivi           \\
     wisdom speech-\textsc{ext} before come-\textsc{dssa} go-\textsc{prog} arrive-\textsc{emp} \\
\glt `having walked with wise words he arrives.' \footnotemark \\
`tendo antes falado sabiamente ele chega mesmo.'\\
\z

\footnotetext{Arrives at the end of the Death-Path, where he/she will find the ancient people.}

\ea Askámaĩnõ wetsaro, awẽ ese vana keyonamãsho, \\[.3em]
\gll Aská-maĩnõ       wetsa-ro          a-ivo    awẽ   ese    vana   keyo-namã-sho         \\
     \textsc{sml-con} other-\textsc{top} \textsc{dem-genr} \textsc{poss} wisdom speech over-\textsc{loc-con} \\
\glt `But the other one, in that place where his speech failed,' \\
`Mas aquele outro, naquele lugar mesmo em que sua fala sabida acabou,' \\
\z

\ea awẽ keyovãianamãsho atxitase. \\[.3em]
\gll awẽ           keyo-vãia-namã-sho        atxi-ta-se            \\
     \textsc{poss} over-\textsc{inc-loc-con} hold-\textsc{ass-ext} \\
\glt `in that place where it failed he gets stuck.' \\
`ali mesmo onde a fala acabou ele fica preso.' \\
\z

\ea Nokẽ shenirasĩ, ramama itivorasĩ, \\[.3em]
\gll Nokẽ             sheni-rasĩ           rama-ma          i-ti-vo-rasĩ              \\
     1\textsc{pl.gen} forbear-\textsc{col} now-\textsc{neg} live-\textsc{pst5-pl-col} \\
\glt `Our forbearers, those born long ago,' \\
`Os nossos antigos, os antepassados de outros tempos,' \\
\z

\ea askásevi veikenaivorasĩ. \\[.3em]
\gll askásevi     vei-ke-na-ivo-rasĩ              \\
     \textsc{sml} death-\textsc{compl-?-genr-col} \\
\glt `they also used to die.' \footnotemark \\
 `ficavam também morridos.'\\
\z

\footnotetext{Meaning that they also used to die or become transformed along the path, because of their lack of knowledge and/or good moral behavior.}

\ea Rave nokoma, rave nokoa, Vei Naí Shavaya nokoma, \\[.3em]
\gll rave noko-ma             rave noko-ma             vei   naí shavaya  noko-ma             \\
     part arrive-\textsc{neg} part arrive-\textsc{neg} death sky dwelling arrive-\textsc{neg} \\
\glt `Some didn't arrive, some didn't arrive, in the Death-Sky Dwelling they couldn't arrive,' \\
`Uns não chegavam, uns não chegavam, na Morada do Céu-Morte não chegavam.'\\
\z

\ea ravero nokoai, ravero nokoma, ravero nokoai. \\[.3em]
\gll rave-ro           noko-ai              rave-ro           noko-ma             rave-ro           noko-ai              \\
     part-\textsc{top} arrive-\textsc{incp} part-\textsc{top} arrive-\textsc{neg} part-\textsc{top} arrive-\textsc{incp} \\
\glt `some arrived, some couldn't arrive, some arrived.' \\
`Uns chegavam, uns não chegavam, outros chegavam.' \\
\z

\ea Akarivi \\[.3em]
\gll aka-rivi              \\
     \textsc{aux.trns-emp} \\
\glt `That's how it happened.' \footnotemark \\
`Assim mesmo é.' \\
\z

\footnotetext{The narrative continues with the exposition of other dangers of the path, giving the sequence of the history of \textit{Vei Maya}, as well as with its moral speculations (see \citealt{Cesarino2012} for the complete version).}

\section*{Non-standard abbreviations}

\begin{tabularx}{.5\textwidth}{lQ}
\textsc{ass } & assertive \\
\textsc{assoc } & associative \\
\textsc{atr.perm } & attributive, permanent \\
\textsc{atr.trns } & attributive, transitional \\
\textsc{cns } & consecutive \\
\textsc{comp } & comparative \\
\textsc{con } & connective \\
\textsc{con.fin } & connective of finality \\
\textsc{dir } & direction \\
\textsc{dir.c } & direction, centripetal \\
\textsc{distr } & distributive \\
\textsc{dspa } & different subject, previous action \\
\textsc{dssa } & different subject, simultaneous action \\
\textsc{emp } & emphatic \\
\textsc{ext } & existential predication \\
\textsc{fin } & finality \\
\textsc{fut } & future \\
\textsc{genr } & generic \\
\textsc{imposs } & impossible \\
\end{tabularx}
\begin{tabularx}{.45\textwidth}{lQ}
\textsc{inc } & inchoative \\
\textsc{incp } & incompletive \\
\textsc{loc.prov } & provenance \\
\textsc{mov.up } & movement up \\
\textsc{possib } & possibility \\
\textsc{pst1} & past (immediate)\\
\textsc{pst2 } & past (months)\\
\textsc{pst3} & past (years, decades)\\
\textsc{pst4} & past (decades, centuries)\\
\textsc{pst5} & past (remote, narrative)\\
\textsc{rlz } & realized action \\
\textsc{sml } & similitive \\
\textsc{sspa } & same subject, previous action \\
\textsc{sssa } & same subject, simultaneous action \\
\textsc{temp } & temporal subordination \\
\textsc{vblz } & verbalizer \\
\end{tabularx}

 

{\sloppy
\printbibliography[heading=subbibliography,notkeyword=this]
}
\end{document}
