\documentclass[output=paper,
modfonts,nonflat
]{langsci/langscibook} 
\author{Edileusa Kwaza\and 
 Mario Aikanã\lastand Hein van der Voort\affiliation{Museu Paraense Emílio Goeldi}
}
\title{Hakai Darija}  
\abstract{noabstract}
% \ChapterDOI{}

\maketitle

\begin{document}

\section{Introduction} 
 
The story of Grandfather Fox represents a myth as told by the Kwaza people of the southwestern Amazon. Kwaza is an isolate language that is highly endangered with about 25 speakers. The speakers of Kwaza live in two different indigenous reserves and in a nearby village, in the southeastern corner of the Brazilian state of Rondônia, amidst a sea of deforested lands owned by big cattle ranchers and soy farmers. In one of the reserves, the Kwaza and Latundê (\textsc{northern nambikwara}) form minority populations among the Aikanã (\textsc{isolate}). In another reserve live several mixed Kwaza and Aikanã families. In spite of its very fragmented speakers’ community, the language is still the first language of the youngest generation in two families. 

  The story was told by Edileusa Kwaza, who had learnt it from her late monolingual Kwaza father, Antonhão. As she told the story, Edileusa was accompanied by her husband Zezinho Kwaza. Edileusa and Zezinho live most of the time with their children and grandchildren in the little village outside the first reserve, with very little contact with other Kwaza speakers. It should be considered admirable that, in spite of their Portuguese-only and often anti-indigenous environment, Kwaza is maintained as the home language of this family. The story was recorded on audio and video in August 2014, as part of a documentation project funded through the DoBeS programme.\footnote{ Funding by the VolkswagenStiftung of DoBeS (Dokumentation Bedrohter Sprachen) project nr. 85.611 is hereby gratefully acknowledged.} The story has been transcribed and analysed with the help of Mario Aikanã, bilingual native speaker of Kwaza and Aikanã and, like most Indians of southeastern Rondônia, full command of Portuguese.

  The story of Grandfather Fox takes place in mythological times. In those days animals transformed into humans at will. Grandfather Fox is a very smart animal who has many tricks up his sleeve. In this story he seduces a young woman after finding out about her plans for the next day. One of the lessons of this and several other traditional stories is that one should avoid speaking about one’s plans for the future, because that will attract adversity. The Aikanã also tell a mythological Fox story and, although it is very different from the Kwaza one, its edifying note on the danger of referring to future plans is similar. In spite of the enormous cultural and ecological changes that the Kwaza and Aikanã peoples have undergone during the 20th century, this taboo is also still very much part of their present way of life.

  The story is transcribed phonetically on the first line, and segmented phonologically and  morphologically on the second line, glossed on the third line and t5he fourth line contains the free translation. Zezinho’s response is given in \{brackets\}. 

\section{Story of Grandfather Fox}

\ea  hakai darija\\[.6em]
\gll hakai      darija\\
     grandparent   fox\\
\glt ‘Grandfather Fox’
\z 

\ea  etaytohoi tsywydyte hurujale arakate tja tsywydyte tja hurujalɛta ta ata.\\[.6em]
\gll etay-tohoi      tsywydyte  huruja-lɛ  arakate    tja    tsywydyte  tja    huruja-lɛ-ta    ta    a-ta\\
     woman-\textsc{cl}:child  girl      like-\textsc{reci} young.man  \textsc{cso}  girl      \textsc{cso}  like-\textsc{reci-cso}    \textsc{cso}  exist-\textsc{cso}\\
\glt ‘An adolescent girl and a young man liked one another, they liked each other, that’s how it was.’
\z

\ea  hadeja hajanỹtsyratiwy tomaʔiʔĩta tja anãi tjarahỹta tjarahỹta tsylehỹ\\[.6em]
\gll hadeja  haja-nỹ-tsy-rati-wy    toma=ĩʔĩta-tja    a-nãi-tjara-hỹ-ta{\footnotemark}    tjara-hỹ-ta    tsy-le-hỹ\\
     night  day-\textsc{refl-ger-foc}{}-time  bathe=always-\textsc{cso} exist\textsc{{}-nom-proc-nom-cso}  \textsc{proc-nom-cso}  \textsc{ger-frust-nom}\\
\glt ‘Some time after midnight she would usually take a bath, that’s how she happily lived, but...’
\z
\footnotetext{The combination -\textit{tjarahỹta} gives the preceding verbal or zero-verbalised stem the connotation ‘luckily’ or ‘happily’.}

\ea  darija tswa aretja orita tsywydytewã hỹdɛ maʔỹtɛ tomaja tjata\\[.6em]
\gll darija  tswa  are-tja    orita    tsywydyte-wã  hỹdɛ    maʔỹtɛ    tomaja  tjata\\
     fox  man  turn-\textsc{cso}  go.there  girl-\textsc{ao} let’s.go  cousin    bathe-\textsc{exp}  say\\
\glt ‘Fox turned into a man and went up to the girl, saying “Let’s go! Cousin, let’s take a bath!”’ 
\z

{\sloppy
\printbibliography[heading=subbibliography,notkeyword=this]
}
\end{document}