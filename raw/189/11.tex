\documentclass[output=paper,modfonts,newtxmath,hidelinks]{langscibook} 
\ChapterDOI{10.5281/zenodo.2545527}

\title{Head directionality in Old Slavic}

\author{Krzysztof Migdalski\affiliation{University of Wrocław}}


\abstract{This paper investigates the issue of head directionality in Old Slavic. This issue has played an important role in diachronic studies on Germanic, in which a switch in head directionality was assumed to have triggered word order changes in the history of these languages. Within Slavic, Old Bulgarian and Old Church Slavonic have been claimed to partly feature head-final grammars by \citet{pancheva2005,pancheva2008} and \citet{dimitrovavulchanova-vulchanov2008}, in contrast to contemporary Slavic languages, which are head-initial. This paper shows that there is little evidence for head-finality in Old Slavic.

\sloppy\keywords{directionality parameter, clitics, participle movement, Old Chuch Slavonic, Old Bulgarian}
}

\begin{document}
\maketitle


\section{Head directionality}\label{11:s1}

The hypothesis of \isi{head directionality} has its roots in \citeposst{greenberg1963} empirical generalizations concerning the position of the verb with respect to the \isi{direct object} in the verb phrase and the correlation between object placement and the ordering of other elements. \citeauthor{greenberg1963} observed that the order within VP has typological implications: VO languages have prepositions, whereas OV languages have postpositions. Within the framework of Principles and Parameters, this correlation is straightforwardly captured through the postulate of the head parameter, which implies that languages show variation concerning the order of the head with respect to its complement (see \citealt{Vennemann1972} and \citealt{Dryer1992,Dryer2007} for discussion). On the assumption that in spite of crosslinguistic variation the head--complement order within a single language is invariant, in head-initial languages the complement always follows the head, hence the object follows the verb and the preposition precedes its \isi{nominal} complements. Correspondingly, in head-final languages the object precedes the \isi{verbal} head, the way a \isi{nominal} complement precedes its postposition.

It has been observed, however, that not all languages display a consistent setting of the head parameter (see \citealt{Hawkins1980,Hawkins1982}). For instance, a well-known case of inconsistency is that of \ili{German}. Although \ili{German} is predominantly head-initial, the verb is final in non-finite verb phrases, while \isi{adjective} phrases may be both head-final and head-initial. In diachronic studies, it has been postulated that the setting of the head parameter may switch in language history. For instance, \citet{pintzuk1991}
shows that although Old \ili{English} (OE) featured mainly OV (head-final) structures, there were also minor instances of VO orders, as evidenced by exceptional structures involving particles, see \REF{11:ex1a}, and personal pronouns following the verb, see \REF{11:ex1b}.
% example 1
\ea \label{11:ex1}
	\ea[]{ \label{11:ex1a}
	\gll þa wolde seo Sexburh æfter syxtyne gearum don hire swustor ban of ðære byrgene \textbf{up}\\
         then wanted the Sexburh after sixteen years take her sister’s bones from the burial.place up\\
	\glt `After sixteen years Sexburh wanted to take up her sister’s bones from their burial-place'
    }
	\ex[]{ \label{11:ex1b}
    \gll We wyllað secgan \textbf{eow} sum bigspell\\
         We want tell you a parable\\ 
    \glt `We want to tell you a parable' \hfill (OE, \citealt[141]{fischer-etal2004})
    }
	\z 
\z
On \citeauthor{pintzuk1991}’s analysis, the post-\isi{verbal} placement of particles and objects is indicative of the head-initial setting of VP, which in Old \ili{English} constitutes a minority pattern. This pattern is assumed to be in competition with the more common head-final VP order instantiated by OV structures.

The hypothesis of grammar competition was postulated by \citet{kroch1989} in order to capture a period of diachronic variation between two structures that are not compatible with each other within a single grammar. Such two structures are assumed to represent two contradictory parameter settings (such as head-final versus head-initial constructions), or, within the Minimalist framework, the presence of lexical items with contradictory features (see also \citealt[278]{pintzuk2002}). The postulate of grammar competition has resulted in many fruitful analyses of diachronically unstable structures. For example, \citet{haeberli-pintzuk2006} investigate the position of the main verb and the auxiliary with respect to adjuncts and complements in verb clusters in Old \ili{English} and attribute the observed \isi{word order} variation to a switch in \isi{head directionality} of functional projections in Old \ili{English}.

Within \ili{Slavic}, a switch in \isi{head directionality} is assumed to trigger a change in the \isi{cliticization} in \citeposst{pancheva2005} analysis. This paper argues for a different view, and it is organized as follows. \sectref{11:s2} examines the arguments for head finality provided by \citet{pancheva2005} on the basis of a diachronic modification of \isi{cliticization} patterns in \ili{Bulgarian}. \sectref{11:s3} overviews \citeposst{pancheva2008} argumentation related to participle--auxiliary orders and the position of \isi{negation} in Old Church \ili{Slavonic}.\footnote{This paper presents a further development of the analysis proposed in \citet{migdalski2016}.}

\section{\citeposst{pancheva2005} analysis of head directionality in Old Slavic}\label{11:s2}

\sloppy Most analyses of Old Church \ili{Slavonic} syntax \citep{willis2000,jung2015,jung-migdalski2015,migdalski2016} assume that it was head-initial on a par with Modern \ili{Slavic} languages. The exceptions are accounts due to \citet{dimitrovavulchanova-vulchanov2008}, who postulate that it was X\textsuperscript{0}{}-final in the VP-domain and X\textsuperscript{0}{}-initial in the CP-domain, as well as \citet{pancheva2005,pancheva2008}, who argues that it was T\textsuperscript{0}{}-final on the basis of the position of \isi{pronominal} clitics, \isi{negation}, and participles with respect to the auxiliary. However, a challenge that these analyses face is the fact that a switch in \isi{head directionality} should have triggered a major modification of the syntactic structure of these languages. Such a modification did not occur; moreover, in contrast to \ili{Germanic} languages, all contemporary \ili{Slavic} languages are strictly head-initial. In view of this, the subsequent section will show that there is little evidence for head-finality in Old \ili{Slavic}. In \sectref{11:s2.1} I provide an overview of \citeauthor{pancheva2005}'s analysis of diachronic \ili{Bulgarian} data. In section \sectref{11:s2.2} I present a criticism of her account.

\subsection{\citeposst{pancheva2005} study the diachrony of cliticization patterns in Bulgarian}\label{11:s2.1} 

\citet{pancheva2005} provides a detailed analysis of the diachrony of \isi{cliticization} patterns in the history of \ili{Bulgarian}. She establishes that in the earliest stages (9\textsuperscript{th}{}--13\textsuperscript{th} c.), Old \ili{Bulgarian} displays largely the same distribution of clitics as Old Church \ili{Slavonic}. Namely, the clitics occur after the verb, as shown in \REF{11:ex2}. As the verb does not need to be located clause-initially, they are clearly not \isi{second position} clitics. Although contemporary \ili{Bulgarian} also features verb-adjacent \isi{cliticization}, it normally disallows post-\isi{verbal} \isi{clitic placement}.



% \gll 	svęt\textcyrillic{ь} bô mõž\textcyrillic{ъ} stvoril\textcyrillic{ъ} ja jest\textcyrillic{ь} \\


% example 2
\ea \label{11:ex2}
\gll svętь bô mõžъ stvorilъ \textbf{ja} jestь \\
     holy because man create.\textsc{part.m.sg} them.\textsc{acc} is.\textsc{aux}\\
\glt `Because a holy man has created them' \hfill (9\textsuperscript{th} c. Bg, \citealt[139]{pancheva2005})
\z
\citeauthor{pancheva2005} assumes, following \citet{kayne1991}, \citet{Chomsky1995}, and corresponding analyses of verb-adjacent \isi{cliticization} that underlyingly \isi{pronominal} clitics are generated as VP arguments. They move from XP-positions in VP and left-adjoin to T\textsuperscript{0} as heads. Crucially, the fact that the \isi{accusative} \isi{pronominal} \isi{clitic} precedes the auxiliary verb in \REF{11:ex2} leads her to assume that Old \ili{Bulgarian} is a T\textsuperscript{0}{}-final language, but all the other heads are initial.

% example 3
% \ea%3
%     \label{11:ex:key:3}
%     \gll\\
%         \\
%     \glt
%     \z
\ea \label{11:ex3}
\glt [\textsubscript{TP} [\textsubscript{vP} [\textsubscript{V’} t\textsubscript{i} V\textsuperscript{0} ]] [\textsubscript{T} CL\textsubscript{i} T\textsuperscript{0}]]  \hfill \citep[139]{pancheva2005}
\z
Another assumption made by \citet[146]{pancheva2005} is that although in Old \ili{Bulgarian} lexical verbs leave vP, they do not reach T\textsuperscript{0} but only Asp\textsuperscript{0} located below T\textsuperscript{0}. This means that her evidence for the final T\textsuperscript{0} comes from the position of the auxiliary `be' (such as \textit{estь} in \REF{11:ex2}) located in T\textsuperscript{0} with respect to \isi{pronominal} clitics (such as \textit{ja} in \REF{11:ex2}).

The post-\isi{verbal} \isi{cliticization} was the dominant pattern in \ili{Bulgarian} until the 13\textsuperscript{th} century. Subsequently, Wackernagel (\isi{second position}) \isi{cliticization} prevailed and remained the default type until the 17\textsuperscript{th} century. \citeauthor{pancheva2005} attributes this change to a switch in the head parameter of T\textsuperscript{0}, which became head-initial. She claims that as a result of the switch \isi{pronominal} clitics begin to appear in front of T\textsuperscript{0} and their position with respect to the verb becomes reversed, as shown in the derivation in (4a). Since other elements may now occur between the verb and the \isi{clitic}, the verb is no longer analyzed as the \isi{clitic} host by the speakers. The clitics remain phonologically enclitic and receive prosodic support from their new hosts located in SpecTP, see \REF{11:ex4b} and \REF{11:ex4c}, or SpecCP.

% example 4
\ea \label{11:ex4}
	\ea[]{
    	\glt [\textsubscript{TP} [\textsubscript{TP} (CL) [\textsubscript{TP} XP =CL T [\textsubscript{AspP} [\textsubscript{Asp} V Asp]]]]] \hfill \citep[151]{pancheva2005}
	} \label{11:ex4a}
	\ex[]{
    	\gll tova \textbf{se} pomoli Juda bogu\\
        	 that.\textsc{top} \textsc{refl} ask.\textsc{part.sg.m} Judas God \\
        \glt `Judas asked God that.\textsc{top}'\hfill (18\textsuperscript{th} c. Bg, \citealt[154]{pancheva2005})
	} \label{11:ex4b}
	\ex[]{
    	\gll a tïa \textbf{gy} zlě mõčaše \\
         	 and she them.\textsc{acc} badly tortured\\
        \glt `She tortured them badly' \hfill (17\textsuperscript{th} c. Bg, \citealt[123]{pancheva2005})
	}\label{11:ex4c}
	\z
\z
\citeauthor{pancheva2005} notes a syntactic restriction on the lexical elements preceding \isi{second position} clitics during this period. She observes that in contrast to contemporary \ili{Slavic} languages with Wackernagel clitics, the clitics in the \ili{Bulgarian} corpus data from that period occur strictly after the first word, which in some cases results in Left Branch Extraction. There are no instances of clitics following the first branching phrase. The same observation is made by \citet[Chapter 3]{radanovickocic1988} for the earliest stages of the development of Wackernagel \isi{cliticization} in Old \ili{Serbian}. Second position \isi{cliticization} with clitics preceded by unambiguous phrasal elements became available in \ili{Serbian} only at a later stage. I take this correlation to mean that the Old \ili{Bulgarian} data analyzed by \citet{pancheva2005} exemplify the initial stage of the emergence of \isi{second position cliticization}, which was not completed. Incidentally, this syntactic restriction on \isi{second position cliticization} cannot be captured by \citeauthor{pancheva2005}’s derivation presented in \REF{11:ex4a}, given that she assumes that the pre-\isi{clitic} element is located in an XP-projection: SpecTP or SpecCP.\footnote{\label{11:fn1}In some \ili{Slavic} languages, such as Serbo-\ili{Croatian}, the \isi{second position} \isi{clitic} \textit{li}, which functions as a focus or interrogation marker, may also be preceded exclusively by single words, as illustrated in \REF{11:exfni}, following \citeauthor{Boskovic2001}'s (\citeyear[27]{Boskovic2001}) observation.

\ea \label{11:exfni}
\gll Skupe (\hspace{-2pt} \textbf{li}) knjige (*\hspace{-2pt} \textbf{li}) Ana čita?\\
expensive {} \textsc{q} books {} \textsc{q} Ana reads\\
\glt `Does Ana read expensive books?' \hfill(S-C, \citealt[27]{Boskovic2001})
\z

\noindent \citet[31ff.]{Boskovic2001} attributes the restriction to the syntactic deficiency of \textit{li} in Serbo-\ili{Croatian}, which is not able to support a \isi{specifier}, and the focus feature of \textit{li} may only be checked through head movement. In fact, this is a special property of “operator clitics” expressing the illocutionary force of a \isi{clause}, which in many \ili{Slavic} languages display special requirements concerning the categorial and syntactic status of their preceding element, in contrast to \isi{pronominal} and auxiliary \isi{second position} clitics. See \citet[Chapter 3]{migdalski2016} for discussion.}

The third stage of the diachronic change investigated by \citeauthor{pancheva2005} takes place from the 17\textsuperscript{th} c. onwards, when \isi{second position} clitics in \ili{Bulgarian} are reanalyzed as preverbal clitics. This pattern prevails in the 19\textsuperscript{th} century and continues to be the default \isi{cliticization} type in contemporary \ili{Bulgarian}. \citeauthor{pancheva2005} points out that this change was contemporaneous with the loss of obligatory \isi{topicalization} to SpecTP. The \isi{topicalization} affected a number of unrelated categories, including the demonstrative \textit{tova} in \REF{11:ex4b} and the subject \textit{tïa} in \REF{11:ex4c}. \citeauthor{pancheva2005} argues that the decline of \isi{topicalization} had repercussions for the syntax of clitics: as SpecTP became filled less frequently, the clitics were no longer analyzed as hosted in \isi{second position} by a constituent located in SpecCP or SpecTP. Instead, the clitics started to appear more frequently adjacent to the verb. In syntactic terms this meant, in \citeauthor{pancheva2005}’s view, that they were reinterpreted as items merged in X\textsuperscript{0} positions, adjoined to functional heads in the extended projections of the verb, see \REF{11:ex5a}, rather than as XP elements that move from argument positions within VP and head-adjoin to T\textsuperscript{0}.\footnote{\label{11:fn2}An anonymous reviewer points out that Pancheva’s account on the reanalysis of clitics fits into the economic factor assumed in grammaticalization, “Merge as a head, not a phrase.” However, \citet{jung-migdalski2015} show that this factor is challenged by the degrammaticalizaiton of \isi{pronominal} clitics into weak pronouns, which occurred in Old \ili{Russian} and Old \ili{Polish}.} With the loss of \isi{second position} interpretation, the clitics could be located lower in the structure, next to the verb, as shown in illustrated in \REF{11:ex5b} for the reflexive \isi{clitic} \textit{sa}, which is left-adjacent to the verb \textit{javi}. 

% example 5
\ea \label{11:ex5}
	\ea[]{
    	\glt [\textsubscript{TP} ...T\textsuperscript{0}...[\textsubscript{XP} [\textsubscript{X} CL X\textsuperscript{0} ] ... [\textsubscript{vP} V\textsuperscript{0} ]]] \hfill \citep[137]{pancheva2005}
	} \label{11:ex5a}
	\ex[]{
    	\gll i archangel Michailь pak \textbf{sa} javi Agari\\
        	 and archangel Michael again \textsc{refl} appeared Agara\\
        \glt `And Archangel Michael appeared to Agara again'\\
        {}\hfill(18\textsuperscript{th} c. Bg, \citealt[120]{pancheva2005})
	} \label{11:ex5b}
	\z
\z


\subsection{Empirical problems with \citeposst{pancheva2005} analysis}\label{11:s2.2}

\citeauthor{pancheva2005}’s analysis addresses a remarkably large set of data, covering different \isi{cliticization} patterns in the history of \ili{Bulgarian}. Although her empirical observations are impressive, the analysis suffers from a number of serious shortcomings. 

First, the postulated link between \isi{head directionality} and a \isi{cliticization} pattern does not receive any support from synchronic considerations. As is well-known, contemporary \ili{Slavic} languages display two distinct patterns of \isi{cliticization} (see, e.g., \citealt{Franks-King2000}). On the one hand, \ili{Czech}, Serbo-\ili{Croatian}, \ili{Slovak}, and \ili{Slovenian} feature \isi{second position} clitics, which obligatorily occur after the clause-initial element virtually irrespective of its category. This type of \isi{clitic} distribution is illustrated in \REF{11:ex6} for a sequence of auxiliary and \isi{pronominal} clitics in Serbo-\ili{Croatian}. The clitics can be preceded by a number of different categories, including the subject, see \REF{11:ex6a}, a wh-element, see \REF{11:ex6b}, and an \isi{adverb}, see \REF{11:ex6c}. 

% example 6
\ea \label{11:ex6}
	\ea[]{
		\gll Mi \textbf{smo} \textbf{mu} \textbf{je} predstavili juče.\\
      		 we are.\textsc{aux} him.\textsc{dat} her.\textsc{acc} introduce.\textsc{part.pl} yesterday\\
		\glt `We introduced her to him yesterday.'
	} \label{11:ex6a}
    \ex[]{
		\gll Zašto \textbf{smo} \textbf{mu} \textbf{je} predstavili juče?\\
       		 why are.\textsc{aux} him.\textsc{dat} her\textsc{acc} introduce.\textsc{part.pl} yesterday\\
		\glt `Why did we introduce her to him yesterday?'
	}\label{11:ex6b}
    \ex[]{
		\gll Juče \textbf{smo} \textbf{mu} \textbf{je} predstavili.\\
      		 yesterday are.\textsc{aux} him.\textsc{dat} her.\textsc{acc} introduce.\textsc{part.pl}\\
		\glt `Yesterday we introduced her to him.' \hfill (S-C, \citealt[8--9]{Boskovic2001})
	}\label{11:ex6c}
	\z
\z
On the other hand, two \ili{Slavic} languages, \ili{Bulgarian} and \ili{Macedonian}, have verb-adjacent clitics, which may not be separated from the verb by any intervening material, see \REF{11:ex7a}. As shown in \REF{11:ex7b}, these clitics do not need to target \isi{second position}. 

% example 7
\ea \label{11:ex7}
	\ea[]{
 		\gll Vera \textbf{mi} \textbf{go} (*\hspace{-2pt} včera) dade.\\
      		 Vera me.\textsc{dat} it.\textsc{acc} {} yesterday gave\\
 		\glt `Vera gave it to me yesterday'
 	}\label{11:ex7a}
 	\ex[]{
     		Včera Vera \textbf{mi} \textbf{go} dade. \hfill (Bg, \citealt[ex. (111d,c)]{franks2010})\\ 
 	}\label{11:ex7b}
 	\z
\z
The \ili{Slavic} languages that display these two \isi{cliticization} patterns differ in a number of ways. For instance, only the languages with verb-adjacent clitics have definite articles (see \citealt{boskovic2016}) and tense morphology (see \citealt{migdalski2015,migdalski2016}). Crucially, they are all head-initial irrespective of their \isi{cliticization} system.

Diachronically, the verb-adjacent pattern of clitics predates \isi{second position cliticization}. It has been observed by \citet{radanovickocic1988} and \citet{pancheva2005} that in Old Church \ili{Slavonic} \isi{pronominal} clitics were predominantly verb-adjacent, as shown for the \isi{dative} \isi{clitic} \textit{mi} in \REF{11:ex8a} and for the \isi{accusative} \isi{clitic} \textit{tę} in \REF{11:ex8b}.

% example 8
\ea \label{11:ex8}
	\ea[]{
		\gll Oca moego vь tĕxъ dostoitъ \textbf{mi} byti\\
      	 	 father.\textsc{gen} my.\textsc{gen} in these be.appropriate.\textsc{inf} me.\textsc{dat} be.\textsc{inf}\\
 		\glt `I had to be in my Father's house?'\\{}\hfill (OCS, \textit{Luke} 2:49, \citealt{pancheva-etal2007})
	}\label{11:ex8a}
	\ex[]{
		\gll Ašte desnaĕ tvoĕ rõka sъblažněetъ \textbf{tę}\\
      		 if right your hand sin.\textsc{pres.1sg} you.\textsc{acc}\\
 		\glt `If your right hand causes you to sin'\\{}\hfill (OCS, \textit{Matthew} 5:30, \citealt[154]{radanovickocic1988})
	}\label{11:ex8b}
	\z
\z
Although \isi{pronominal} clitics could occur in \isi{second position} in Old Church \ili{Slavonic}, especially when the clause-initial element was a verb (and hence they were verb-adjacent), \citet{radanovickocic1988} points out that only three clitics appeared in \isi{second position} without exception: the question/focus particle \textit{li}, the \isi{complementizer} \isi{clitic} \textit{bo} ‘because,’ and the focus particle \textit{že,} see \REF{11:ex9a}--\REF{11:ex9c}. 

% example 9
\ea \label{11:ex9}
	\ea[]{
 		\gll Približi \textbf{bo} sę crstvie nbskoe.\\
      		 approach.\textsc{aor.3sg} because \textsc{refl} kingdom heaven\\
 		\glt `For the kingdom of heaven is at hand.'\\{}\hfill (OCS, \textit{Matthew} 3:2, \citealt[152]{radanovickocic1988})
	}\label{11:ex9a}
	\ex[]{
		\gll Mati \textbf{že} jego živĕaše blizъ vratъ.\\
     		 mother \textsc{foc} his live.\textsc{imp.3sg} near gates\\
		\glt `And his mother lived near the gates.'\\{}\hfill (OCS, \citealt[152]{radanovickocic1988})
	}\label{11:ex9b}
	\ex[]{
 		\gll Ašte \textbf{li} oko tvoĕ lõkavo bõdetъ\\
      		 if \textsc{q} eye your evil be.\textsc{pres.sg.n} \\
 		\glt `If your eye should be evil'\\{}\hfill (OCS, \textit{Matthew} 6:23, \citealt[151]{radanovickocic1988})
	}\label{11:ex9c}
	\z
\z
I observe in \citet{migdalski2016} that the \isi{second position} clitics exemplified in \REF{11:ex9a}--\REF{11:ex9c} form a natural class of sentential (operator) clitics. The semantic property that unifies them is that they all encode the illocutionary force of a \isi{clause}. The counterparts of these clitics in contemporary \ili{Slavic} languages also target \isi{second position}, regardless of whether their \isi{pronominal} and auxiliary clitics also occupy Wackernagel position or whether they are verb-adjacent. Thus, as shown in \REF{11:ex10}, although \ili{Bulgarian} has verb-adjecent clitics, the \isi{clitic} \textit{li} is in \isi{second position}, separated from the \isi{accusative} \isi{clitic} \textit{ja} and the auxiliary \isi{clitic} \textit{je}.

% example 10
\ea \label{11:ex10}
	\gll Včera \textbf{li} Penka \textbf{ja} \textbf{e}  dala knigata na Petko?\\
     	 yesterday Q Penka her.\textsc{refl} is\textsubscript{}.\textsc{aux} give.\textsc{part.f.sg} book.the to Petko \\
	\glt `Was it yesterday that Penka gave the book to Petko?'\\{}\hfill (Bg, \citealt[833]{tomic1996})
\z
The fact that \citet{pancheva2005} disregards the categorial status of clitics located in respective positions in her estimates of the different types of \isi{clitic placement} is a major drawback of her analysis. In fact, this problem has been also pointed out by \citet{dimitrovavulchanova-vulchanov2008}, who, referring to \citeposst{pancheva2005} analysis, note that in \textit{Codex Suprasliensis} (a late Old Church \ili{Slavonic} relic) the distribution of clitics is quite consistent and regular, and it does not seem to be a matter of statistical frequency or choice. \citeauthor{dimitrovavulchanova-vulchanov2008} observe that in \textit{Codex Suprasliensis} clitics are found in \isi{second position} if SpecCP is filled, otherwise they are post-\isi{verbal}. Although \citeauthor{dimitrovavulchanova-vulchanov2008} do not provide any data in support of their observation, it is likely that that SpecCP is filled in the presence of operator clitics of the type exemplified in \REF{11:ex9}, which are uniformly hosted in \isi{second position}. 

In \citet{migdalski2016} I further observe that \citeauthor{pancheva2005}’s analysis is challenged by synchronic and diachronic \isi{cliticization} data from \ili{Slavic}. On the synchronic side, a problematic empirical fact is that the \isi{clitic} forms of the auxiliary verb ‘to be’ in South \ili{Slavic} languages occupy a different position with respect to \isi{pronominal} clitics depending on their person feature content. Namely, as indicated for Serbo-\ili{Croatian} in \REF{11:ex11}, the 3\textsuperscript{rd} person auxiliary \isi{clitic} (such as \textit{je} in \REF{11:ex11a}) is located to the right of the \isi{pronominal} clitics, while all the other auxiliary variants (such as the 1\textsuperscript{st} person form \textit{sam} in \REF{11:ex11b}) are hosted to the left of the \isi{pronominal} clitics. 

% example 11
\ea \label{11:ex11}
	\ea[]{
 		\gll On \textbf{mu} \textbf{ih}  \textbf{je}  dao.\\
      		 he him.\textsc{dat} them.\textsc{acc} is.\textsc{aux} give.\textsc{part.sg.m}\\
		\glt `He gave them to him.'
	}\label{11:ex11a}
	\ex[]{
		\gll Ja \textbf{sam} \textbf{mu} \textbf{ih} dao.\\
      		 I am.\textsc{aux} him.\textsc{dat} them.\textsc{acc} give.\textsc{part.sg.m}\\
 		\glt `I gave them to him indeed.' \hfill (S-C, \citealt[839]{tomic1996})
	}\label{11:ex11b}
	\z
\z
If \citeauthor{pancheva2005}’s account of \isi{cliticization} were to be adopted to account for the auxiliary \isi{clitic placement}, it would imply that in contemporary South \ili{Slavic} languages T\textsuperscript{0} is head-final in the structures with the 3\textsuperscript{rd} person singular auxiliary, and that T\textsuperscript{0} is head-initial with all the other auxiliary forms. This is not a welcome result given that the auxiliaries assume a different position in the structure purely depending on their person/number feature specification. The nature of this morphological contrast suggests that it does not involve alleged competition between two grammars that differ with respect to T\textsuperscript{0}{}-initial and T\textsuperscript{0}{}-final placement but rather that the contrast is entirely synchronic.

\largerpage
On the diachronic side, \citeauthor{pancheva2005}’s proposal of the switch in the \isi{head directionality} of T\textsuperscript{0}, which relies on the position of \isi{pronominal} clitics with respect to the auxiliary, is seriously challenged by the timing of the diachronic modification of the auxiliary placement in the history of \ili{Bulgarian}. I report in \citet[283--284]{migdalski2016}, following \citeposst{slawski1946} observations, that in Old \ili{Bulgarian} all auxiliary forms followed \isi{pronominal} clitics, as in the pattern in \REF{11:ex2} above, which is used by \citeauthor{pancheva2005} as evidence for the T\textsuperscript{0}{}-final order. Two additional Old \ili{Bulgarian} examples in which a non-third person auxiliary follows the \isi{pronominal} clitics are given in \REF{11:ex12}. At first sight they may seem to lend support to \citeauthor{pancheva2005}'s analysis, since in contrast to contemporary \ili{Slavic} languages, all auxiliary forms are located to the right of the \isi{pronominal} clitics.

% example 12
\ea \label{11:ex12}
	\ea[]{
		\gll pustila \textbf{mę} \textbf{sta} oba carĕ\\
      		 let.go{}.\textsc{part.f.dual} me{}.\textsc{acc} are{}.\textsc{aux.2dual} two tsars\\
 		\glt `Two tsars have sent me' \hfill (14\textsuperscript{th} c. Bg)
	}\label{11:ex12a}
	\ex[]{
 		\gll tvoè zlàto što \textbf{mu} \textbf{si} pròvodilь\\
      		 your gold that him{}.\textsc{dat} are{}.\textsc{aux.2sg} send.\textsc{part.sg.m}\\
 		\glt `Your gold that you have sent to him' \hfill (17\textsuperscript{th} c. Bg, \citealt[76]{slawski1946})
	}\label{11:ex12b}
	\z
\z
However, in the 17\textsuperscript{th}--18\textsuperscript{th} century the auxiliary placement in \ili{Bulgarian} underwent a modification: the first and second auxiliary forms shifted across the \isi{pronominal} clitics, adopting the current distribution \citep[76--77]{slawski1946}, as exemplified in \REF{11:ex13}. The timing of the modification is a problem for \citet{pancheva2005}, as it took place when according to her analysis \ili{Bulgarian} had featured T\textsuperscript{0}{}-initial grammar for several centuries, with no \isi{second position} clitics left.

% example 13
\ea \label{11:ex13}
	\ea[]{
		\gll deto \textbf{si} \textbf{së} javilь na mòata žena\\
      		 where are.\textsc{aux.2sg} \textsc{refl} appear.\textsc{part.sg.m} to my.the wife\\
 		\glt `Where you have appeared to my wife' \hfill (17\textsuperscript{th} c. Bg, \citealt[77]{slawski1946})
	}\label{11:ex13a}
	\ex[]{
 		\gll nó \textbf{sa} \textbf{gi} zváli gotïi\\
      		 and are.\textsc{aux.3pl} them.\textsc{acc.pl} call.\textsc{part.pl} Goths\\
 		\glt `And they called them Goths' \hfill (18\textsuperscript{th} c. Bg, \citealt[77]{slawski1946})
 }\label{11:ex13b}
 	\z
 \z
I observe that the timing of the switch of the auxiliary forms indicates that \isi{second position cliticization} is not related to the alleged loss of T\textsuperscript{0}{}-finality or the position of \isi{pronominal} clitics with respect to the auxiliary. The lack of the correlation between these properties is also independently confirmed by \citeposst{jung2015} study of the auxiliary placement in Old \ili{Russian} data. \citeauthor{jung2015} points out that even though Old \ili{Russian} had \isi{second position} clitics until the 14\textsuperscript{th} century, the first and second person forms of the auxiliary rigidly followed the \isi{pronominal} clitics throughout this period. Furthermore, in \citet{migdalski2015,migdalski2016} I develop an analysis of a diachronic switch from verb-adjacent to Wackernagel clitics in Serbo-\ili{Croatian}, \ili{Slovenian}, and \ili{Polish}, showing that it was contemporaneous with the loss of tense morphology, analyzed as the loss of TP. It remains to be determined whether a related analysis can be applied to the Old \ili{Bulgarian} facts noted by \citet{pancheva2005}.

\section{\citeposst{pancheva2008} arguments for the final T\textsuperscript{0} related to participle-auxiliary orders and the distribution of negation} \label{11:s3}

This section examines the arguments for the T\textsuperscript{0}{}-finality of Old Church \ili{Slavonic} that \citet{pancheva2008} provides in her later work. They are related to the syntax of compound tenses formed with the \textit{l}{}-\isi{participle} and the auxiliary `be' and the interaction between \isi{negation} and verb placement.

\subsection{Participle--auxiliary orders in Old Church Slavonic} \label{11:s3.1}

Most South and West \ili{Slavic} languages feature a compound tense construction formed with the auxiliary ‘be’ and the \textit{l}{}-\isi{participle}; see \REF{11:ex14a} for \ili{Bulgarian}. The \textit{l}{}-\isi{participle} may be fronted across the auxiliary, as in \REF{11:ex14b}. 

% example 14
\ea \label{11:ex14}
	\ea[]{
		\gll Az sŭm čel knigata.\\
     		 I am.\textsc{aux}\textsubscript{} read.\textsc{part.sg.m} book.the\\
             \glt `I have read the book.'
     }\label{11:ex14a}
	\ex[]{
		\gll Čel sŭm knigata.\\
     		 read.\textsc{part.sg.m} am.\textsc{aux} book.the\\
		\glt `I have read the book.' \hfill (Bg)
	}\label{11:ex14b}
	\z
\z
This operation has received considerable attention in the literature since \citeposst{lema-rivero1989} analysis of the fronting in terms of Long Head Movement, which on their account proceeds via head raising of the \textit{l}{}-\isi{participle} from V\textsuperscript{0} to C\textsuperscript{0} across the auxiliary located in I\textsuperscript{0}, as shown in \REF{11:ex15}.

% example 15
\ea \label{11:ex15}
\glt [\textsubscript{CP} [\textsubscript{C} Part\textit{\textsubscript{i}}] [\textsubscript{IP} Aux [\textsubscript{VP} [\textsubscript{V} t\textit{\textsubscript{i}}] DP]]]
\z
The operation has also been analyzed as head adjunction of the \isi{participle} to C\textsuperscript{0} \citep{wilder-cavar1994}, to Aux\textsuperscript{0} \citep{boskovic1997}, or to a focus projection Delta\textsuperscript{0}  \citep{lambova2003}. I proposed in my previous work \citep{broekhuis-migdalski2003,migdalski2006} that the movement involves predicate inversion, which proceeds via XP remnant movement of the \textit{l}{}-\isi{participle} to SpecTP. This proposal accounts for a number of properties of the movement that had been unexplained in the previous analysis, such as the dependency of the phrasal movement on the presence of the auxiliary `be' and the subject gap requirement, a property that will be important for the analysis presented in the remainder of this article.

\citet{pancheva2008} addresses similar cases of clause-initial \isi{participle} placement in Old Church \ili{Slavonic}, as illustrated in \REF{11:ex16b}.

% example 16
\ea \label{11:ex16}
	\ea[]{
		\gll iže běaxŏ prišъli otъ vьsěkoję vьsi\\
     		 who.\textsc{foc} be.\textsc{\isi{past}.3pl} come.\textsc{part.pl} from every village\\
		\glt `who had come from every village' \hfill (OCS, \textit{Luke} 5.17)
	}\label{11:ex16a}
	\ex[]{
		\gll učenici bo ego ošъli běaxõ vъ gradъ\\
     		 disciples for his go.\textsc{part.pl} be.\textsc{\isi{past}.3pl} in town\\
		\glt `because his disciples had gone to the town'\\{}\hfill (OCS, \textit{John} 4.8, \citealt{pancheva2008})
	}\label{11:ex16b}
	\z
\z
In principle, the Old Church \ili{Slavonic} structure in \REF{11:ex16b} most likely illustrates a counterpart of \isi{participle} fronting attested in Modern \ili{Slavic}, as has been argued for by \citet[325--327]{willis2000}. \citet{pancheva2008} postulates, however, that on the assumption that Old Church \ili{Slavonic} was T\textsuperscript{0}{}-final, the ordering presented in \REF{11:ex16b} could be taken to be the basic one, whereas the auxiliary–\isi{participle} pattern in \REF{11:ex16a} could be derived via rightward \isi{participle movement}. In order to determine which order is the derived one, she calculates the ratio of both patterns. 

Importantly, \citet{pancheva2008} notes that the participle--auxiliary order may be more frequent than the auxiliary--\isi{participle} when the auxiliary is a \isi{clitic} that needs prosodic support to its left. In order to limit the impact of the prosodic requirements on \isi{word order}, she chooses to restrict her analysis to the structures involving the \isi{past tense} auxiliary, which has a strong, non-\isi{clitic} form. Furthermore, she assumes that the pattern that is a result of an optional operation will be statistically less common than the one that instantiates the basic order. 

The results of her quantitative study show that both orders occur in a balanced proportion in Old Church \ili{Slavonic}, though the participle--auxiliary pattern is less common than the auxiliary--\isi{participle} pattern: 41\% versus 59\%. By contrast, in Modern \ili{Bulgarian} the auxiliary--\isi{participle} order is considerably more frequent and constitutes 97\% of the data investigated by \citeauthor{pancheva2008}, versus 3\% of the participle--auxiliary orders. Pancheva states that on the assumption that Modern \ili{Bulgarian} is T\textsuperscript{0}{}-initial and that participle--auxiliary sequences are a result of \isi{participle movement} to the left, the contrast in the ratio of the two constructions across the centuries indicates that Old Church \ili{Slavonic} was a T\textsuperscript{0}{}-final language.

The diachronic contrast in the ratio of participle--auxiliary orders is certainly interesting and requires an explanation, though it should be noted that even in Old Church \ili{Slavonic} the participle--auxiliary pattern is less frequent. \citet{pancheva2008} makes use of additional argumentation to support her analysis. Namely, she acknowledges the fact that the different ratios of the \isi{participle}/auxiliary patterns across centuries may have been due to different discourse factors that are reflected through these two orders rather than due to the switch in the T$^0$-head parameter setting. Thus, it may well be the case that a particular discourse context started or ceased to be expressed through \isi{participle movement} at a certain point in the history of \ili{Bulgarian}. Yet, she ultimately rejects this possibility, referring to an observation of different ratios between active and passive participles preceding the auxiliary. She shows that in \textit{Codex Marianus}, an Old Church \ili{Slavonic} relic, active participles are placed in front of the auxiliary in 16\% of cases, while passive participles precede the auxiliary in as many as 67\% of cases. In Modern \ili{Bulgarian} the rate is not that high. \citeauthor{pancheva2008} argues that this contrast may point to a situation in which two grammars (T\textsuperscript{0}{}-final and T\textsuperscript{0}{}-initial) are in competition, and that the switch in the setting of the T\textsuperscript{0}{}-head parameter was initiated among active participles, which as a result gave rise to a higher rate of the active participle--auxiliary orders.

I would like to propose an alternative explanation of the observed diachronic frequency contrast in the participle--auxiliary orders. As has been examined in detail by \citet{lambova2003}, \isi{participle} fronting in Modern \ili{Bulgarian} triggers different discourse conditions depending on whether it occurs across the \isi{present} perfect auxiliary \isi{clitic} (see \REF{11:ex17a} below as well as \REF{11:ex14b} above) or the strong \isi{past} perfect auxiliary, as in \REF{11:ex17b}. Given that the auxiliary in \REF{11:ex17a} is prosodically deficient and needs to be supported to its left, the fronting of the \isi{participle} (or of some other element) to the position in front of the \isi{clitic} is obligatory. In contrast, movement of the \isi{participle} across the non-\isi{clitic} auxiliary, as in \REF{11:ex17b}, is optional. As was mentioned above, \citeauthor{pancheva2008} restricts her diachronic analysis to the orders involving \isi{participle} fronting across the \isi{past tense} auxiliary, which correspond to the one in \REF{11:ex17b}, and in this way she avoids a potential influence of the \isi{clitic} prosodic requirement on \isi{word order} possibilities.

%SOLL NUR AUS a) und b) BESTEHEN MIT a' UND b' !!!!!!!!!!!!!!!!!!!!!!!!!!!
%SOLL NUR AUS a) und b) BESTEHEN MIT a' UND b' !!!!!!!!!!!!!!!!!!!!!!!!!!!
%SOLL NUR AUS a) und b) BESTEHEN MIT a' UND b' !!!!!!!!!!!!!!!!!!!!!!!!!!!
% example 17
\ea \label{11:ex17}
	\ea[]{
		\gll Gledali \textbf{sa} filma.\\
     		 watch.\textsc{part.pl} are.\textsc{aux.3pl} movie.the\\
		\glt `They have watched the movie.'
	}\label{11:ex17a}
	\exi{a'.}[*]{\gll
		 	 \textbf{Sa} gledali filma.\\
             are.\textsc{aux.3pl} watch.\textsc{part.pl} movie.the\\
		\glt Intended: `They have watched the movie.'
	}\label{11:ex17a2}
	\ex[]{
		\gll Gledali bjaxa filma.\\
     		 watch.\textsc{part.pl} were.\textsc{aux.3pl} movie.the\\
		\glt `They had \textsc{watched} the movie.'
	}\label{11:ex17b}
	\exi{b'.}[]{
		\gll	 Bjaxa gledali filma.\\
        were.\textsc{aux.3pl} watch.\textsc{part.pl} movie.the\\
		\glt `They had watched the movie' \hfill (Bg, \citealt[111--112]{lambova2003})
	}\label{11:ex17b2}
	\z
\z
\citet{lambova2003} points out that whereas the \isi{participle movement} across the auxiliary \isi{clitic} illustrated in \REF{11:ex17a} is perceived as neutral, the fronting across the \isi{past tense} auxiliary exemplified in \REF{11:ex17b} necessarily produces detectable semantic effects and is perceived as “marked.” This fact is reflected in the translation of \REF{11:ex17b}, with the main verb capitalized to show a focused interpretation. \citet[113]{lambova2003} argues that \isi{participle} fronting across the \isi{past tense} auxiliary is felicitous when “the speaker is presenting the activity under discussion as an alternative.” Thus, the sentence in \REF{11:ex17b} can be produced in a situation in which “the discourse contains either explicit or implied reference to the movie being in possession, i.e. rented or owned.” \citep[113]{lambova2003}. In such a scenario, a potential paraphrase of this example is `They have only seen the movie.' The main verb is pronounced with a high tone, as is typical of contrastively focused constituents in \ili{Bulgarian}. These properties lead \citeauthor{lambova2003} to suggest that when the \isi{participle} raises across the \isi{past tense} auxiliary, it lands in a higher projection than it does during the fronting across the auxiliary \isi{clitic}. She terms this projection Delta Phrase and assumes it is a discourse-related projection located above CP, where focus is licensed.

In Modern \ili{Bulgarian} \isi{participle} fronting across the \isi{past tense} auxiliary results in a special discourse effect, so it is not surprising that it is not often found in the corpus examined by \citeauthor{pancheva2008}. What needs to be determined is whether a related discourse effect was produced by the corresponding \isi{participle} reordering in Old Church \ili{Slavonic}. It is likely that it did not. In fact, in \sectref{11:s2.1} above I refer to a discourse-related syntactic change reported in \citet[153--154]{pancheva2005}, which occurred in \ili{Bulgarian} between the 17\textsuperscript{th} and the 19\textsuperscript{th} centuries, and which involved the decline of obligatory \isi{topicalization} targeting SpecTP. This change was accompanied by a reinterpretation of Wackernagel \isi{pronominal} clitics as preverbal elements. Examples of the obligatory \isi{topicalization} are given in \REF{11:ex4} above and \REF{11:ex18}--\REF{11:ex20} below, and they include clauses with a topicalized object, see \REF{11:ex4b}, an \isi{adverbial} \isi{participle}, see \REF{11:ex18}, a finite verb, see \REF{11:ex19}, and an \isi{adverb}, see \REF{11:ex20}. Pancheva notes that in Modern \ili{Bulgarian} the corresponding structures are not felicitous.\footnote{\label{11:fn3}\citet{dimitrovavulchanova-vulchanov2008} observe a high frequency of structures of this type in Old Church \ili{Slavonic}, which leads them to assume that VP is head-final in this language. However, they do not exclude the possibility of VP being head-initial, with the \isi{topicalization} derived via movement.}

\largerpage
% example 18
\ea \label{11:ex18}
	\gll i otvěštavь starecъ reče emu: {\dots} i vъ drugõõ ned(ě)lę prïide starecъ kъ bratu\\
     and answering old.monk told him {} and in other Sunday came old.monk to young.monk\\
	\glt `And in response, the old monk told him: {\dots} And the next Sunday, the old monk came to the young one' \hfill (14\textsuperscript{th} c. Bg)
\z

% example 19
\ea \label{11:ex19}
	\gll se priõtъ b(og)ъ pokaanïe tvoe\\
     thus accepts God repentance your \\
	\glt `Thus God accepts your repentance' \hfill (14\textsuperscript{th} c. Bg)
\z

% example 20
\ea \label{11:ex20}
	\gll pakъ utide angelъ i vtorïju patъ\\
     again went angel and second time\\
	\glt `The angel went there again for the second time' \hfill (18\textsuperscript{th} c. Bg)
\z
Even though the \isi{topicalization} data provided by \citet[153--154]{pancheva2005} does not include examples with clause-initial \textit{l}{}-participles, it is quite likely that they were also subject to the rule of obligatory \isi{topicalization}. \citet{broekhuis-migdalski2003} and \citet{migdalski2006} argue on the basis of Modern \ili{Bulgarian} that fronted \textit{l}{}-participles target SpecTP. If the same analysis can be applied to Old Church \ili{Slavonic} (see \citealt{willis2000}) and Old \ili{Bulgarian}, the historically high ratio of \isi{participle movement} receives a straightforward explanation: it is a product of the obligatory \isi{topicalization} to SpecTP.

Another factor that may have given rise to the higher frequency of participle-initial orders in Old Church \ili{Slavonic} is the fact that the complex tense structures formed with the \textit{l}{}-\isi{participle} and the auxiliary `be' were considerably less common in Old Church \ili{Slavonic} than they are in the contemporary South \ili{Slavic} languages. Thus, \citeposstpg{dostal1954}{599ff.} estimates indicate that the \textit{l}{}-perfect tense was used sporadically in Old Church \ili{Slavonic}, and usually in subordinate clauses. \citeauthor{dostal1954}’s corpus study lists 10 thousand usages of the aorist, 2300 of the imperfect tense, and approximately only 600 instances of the perfect tenses (that is, approximately 5\% of all the tense forms). The scarcity of the usage of the \textit{l}{}-perfect compound tense in Old \ili{Slavic} has been attributed to a number of factors (see \citealt{migdalski2006}: 26--27 for discussion). For instance, \citeauthor{bartula1981} (\citeyear[100]{bartula1981}; see also \citealt{damborsky1967}) notes that there are few examples of \isi{present} perfect structures in the earliest Old Church \ili{Slavonic} relics. They become more frequent in later manuscripts, such as \textit{Codex Suprasliensis} and \textit{Savvina kniga} (both from the 11\textsuperscript{th} century). Most likely, the structures formed with the \textit{l}{}-\isi{participle} may have felt too novel and innovative for formal biblical texts. The fact that these structures were far less common in Old \ili{Slavic} than in present-day \ili{Slavic} languages may have repercussions for the different ratios in the participle--auxiliary patterns investigated by \citet{pancheva2008}.

\subsection{The position of negation in Old Church Slavonic} \label{11:s3.2}

The final observation used by \citet{pancheva2008} to support of her T\textsuperscript{0}{}-final analysis of Old Church \ili{Slavonic} is related to the interaction between \isi{negation} and verb placement. It has been observed in the literature (see e.g. \citealt{rivero1991}) that in Modern \ili{Slavic} \isi{negation} may attract and incorporate into verbs, as a result of which the two elements form a single prosodic word. The process of incorporation is evidenced by the placement of \isi{second position} clitics in languages such as Serbo-\ili{Croatian}, which follow the sequence of \isi{negation} and the finite verb, as in \REF{11:ex21}.

% example 21
\ea \label{11:ex21}
\gll Ne \{*\hspace{-2pt} \textbf{ga}\} vidim \{\hspace{-2pt} \textbf{ga}\}\\
     \textsc{neg} {} him.\textsc{acc} see.\textsc{pres.1sg} {} him.\textsc{acc}\\
\glt `I don’t see him' \hfill (S-C, \citealt{rivero1991}: 338)
\z
As will be discussed in more detail below, contemporary \ili{Slavic} languages differ with respect to whether \isi{negation} attracts the (finite) auxiliary verb or the \textit{l}{}-\isi{participle}. \citet{pancheva2008} shows that in Old Church \ili{Slavonic} \isi{negation} may attract finite verbs, see \REF{11:ex22a}, including the auxiliary, see \REF{11:ex22b}, and, in contrast to Modern \ili{Bulgarian}, in some cases also the \textit{l}{}-\isi{participle}, see \REF{11:ex22c}.

% example 22
\ea \label{11:ex22}
	\ea[]{
		\gll ne ostavitъ \textbf{li} devęti desętъ i devęti vъ pustyni\\
     		 \textsc{neg} leaves \textsc{q} nine ten and nine in wilderness\\
		\glt `Does he not leave the ninety-nine in the wilderness?'\\{}\hfill (OCS, \textit{Luke} 15.4)
	}\label{11:ex22a}
	\ex[]{
		\gll sego avraamъ něstъ sъtvorilъ\\
     		 this Abraham \textsc{neg.}is.\textsc{aux} do.\textsc{part.sg.m}\\
	\glt `Abraham did not do this' \hfill (OCS, \textit{John} 8.40)
	}\label{11:ex22b}
	\ex[]{
		\gll ne moglъ \textbf{bi} tvoriti ničesože\\
    		 \textsc{neg} can.\textsc{part.sg.m} be.\textsc{cond.3sg} do.\textsc{inf} nothing\\
		\glt `He couldn’t do anything' \hfill (OCS, \textit{John} 9.33, \citealt{pancheva2008})
	}\label{11:ex22c}
	\z
\z
\citeauthor{pancheva2008} assumes that in Old Church \ili{Slavonic} NegP is located above TP. In view of this assumption, the fact that \isi{negation} may attract the \textit{l}{}-\isi{participle} and as a result produce the negation--participle--auxiliary pattern is taken by \citeauthor{pancheva2008} to indicate a potential T\textsuperscript{0}{}-final structure. According to her analysis, a T\textsuperscript{0}{}-final structure can also be postulated for negation--auxiliary--\isi{participle} orders on the assumption that \isi{negation} attracts the auxiliary across the \isi{participle}. Importantly, \citeauthor{pancheva2008} claims that since Old Church \ili{Slavonic} shows variation in the \isi{verbal} structures involving \isi{negation}, allowing both negation--\isi{participle} and negation--auxiliary orders, it is likely that Old Church \ili{Slavonic} features two grammars (T\textsuperscript{0}{}-final and T\textsuperscript{0}{}-initial), which are in competition.

I observe that \citeposst{pancheva2008} hypothesis of the two competing grammars, posited on the basis of the distribution of \isi{negation}, is challenged by diachronic and empirical facts. 

Diachronically, the position of \isi{negation} with respect to the verb exhibits categorial and semantic contrasts, which suggests that it is not related to grammar competition. Thus, \citeauthor{vecerka1989} (\citeyear[34]{vecerka1989}; quoted in \citealt[328]{willis2000}) observes that the negation--auxiliary order is four times as frequent as the negation--\isi{participle} order. Correspondingly, \citet[329]{willis2000} shows that the auxiliary--negation--\isi{participle} pattern is not found in matrix clauses. This type of variation is unexpected if grammar competition is involved.\footnote{\label{11:fn4}An anonymous reviewer points out though that embedded contexts may pattern differently in processes of \isi{language change}. They may be more conservative than non-embedded contexts in the case of diffusion of a change.}

Furthermore, in subordinate clauses the position of the conditional auxiliary \textit{bi} is related to the \isi{semantics} expressed by the \isi{complementizer}, which in turn may have a repercussion for the position of \isi{negation} with respect to the auxiliary and the \textit{l}{}-\isi{participle}. As observed by \citet[330]{willis2000}, in Old Church \ili{Slavonic} complementizers may attract the conditional auxiliary. The attraction is obligatory in the case of \isi{complementizer} \textit{a,} which introduces conditional clauses, see \REF{11:ex23}, but not with the \isi{complementizer} \textit{da,} which introduces indicative clauses, see \REF{11:ex24}.

% example 23
\ea \label{11:ex23}
	\ea[]{
		\gll A \textbf{by} bylъ sьde\\
     		 if \textsc{cond.3sg} be.\textsc{part.sg.m} here\\
		\glt `If he had been here'
	}\label{11:ex23a}
	\ex[]{
			 \gll A \textbf{by} sьde bylъ\\
             if \textsc{cond.3sg} here be.\textsc{part.sg.m}\\
		\glt `If he had been here'
	}\label{11:ex23b}
		\ex[]{
			\gll A \textbf{by} bylъ prorokъ\\
     			 if \textsc{cond.3sg} be.\textsc{part.sg.m} prophet\\
		\glt `If he had been the prophet'  \hfill (OCS, \citealt{vaillant1977}: 219)
	}\label{11:ex23c}
	\z
\z

% example 24
\ea \label{11:ex24}
	\ea[]{
		\gll Drъžaaxõ \textbf{i} da ne \textbf{bi} otъšelъ otъ nixъ\\
     		 held.\textsc{3pl} him that \textsc{neg} \textsc{cond.3sg} leave.\textsc{part.sg.m}\textsubscript{} from them\\
             \glt `And they held him, so that he would not leave them'\\{}\hfill (OCS, \textit{Codex Marianus}, \citealt[330]{willis2000})
     }\label{11:ex24a}
     \ex[]{
		\gll Drъžaaxõ \textbf{i} da \textbf{bi}  ne otъšlъ otъ nixъ\\
     		 held.\textsc{3.pl} him that \textsc{cond.3sg} \textsc{neg} leave.\textsc{part.sg.m} from them\\
		\glt `And they held him, so that he would not leave them'\\{}\hfill (OCS, \textit{Codex Zographensis}, \citealt[330]{willis2000})
	}\label{11:ex24b}
	\z
\z

\noindent It can be assumed then that in subordinate clauses headed by the \isi{complementizer} \textit{a}, there will be no instances of the negation--auxiliary pattern, and that only the negation--\isi{participle} order will be observed. Such a contextual, semantic-dependent restriction would be surprising if the variation were due to grammar competition. Rather, it seems that at least in the environments presented in \REF{11:ex23} and \REF{11:ex24}, the position of \isi{negation} with respect to the verb is dictated by a syntactic mechanism, which in specific contexts becomes obligatory.\footnote{\label{11:fn5}An anonymous reviewer provides an additional empirical fact that challenges Pancheva’s assumption of a link between the position of \isi{negation}, \isi{cliticization}, and \isi{head directionality}. Namely, Old North \ili{Russian} displayed both the negation--\isi{participle} order (though \isi{negation} could directly precede the copular `be') and \isi{second position} \isi{clitic} system until the 14\textsuperscript{th} century. On Pancheva’s analysis the co-occurrence of these two properties would indicate that Old North \ili{Russian} was simultaneously T\textsuperscript{0}{}-initial and T\textsuperscript{0}{}-final.}  

Synchronically, Pancheva’s assumption of the potential relation between the position of \isi{negation} and the directionality of T\textsuperscript{0} is challenged by properties of complex tense structures in contemporary \ili{Polish} and \ili{Czech}. \ili{Polish}, which is clearly a T\textsuperscript{0}{}-initial language, permits \isi{negation} to either precede the auxiliary or the \isi{participle}. The type of possible order depends on the type of the auxiliary involved. For example, \isi{negation} attracts the \isi{future} auxiliary (which morphologically is the \isi{perfective} form of the verb `be'), as shown in \REF{11:ex25}, but it adjoins to the \textit{l}{}-\isi{participle} rather than the perfect auxiliary in structures characterizing \isi{past} events, as indicated in \REF{11:ex26}.

% example 25
\ea \label{11:ex25}
	\ea[]{
		\gll Nie będziesz parkował tutaj samochodu.\\
     		 \textsc{neg} be.\textsc{perf.1sg} park.\textsc{part.sg.m} here car\\
		\glt `You won’t park your car here.'
	}\label{11:ex25a}
	\ex[*]{
			 Nie parkował będziesz tutaj samochodu. \hfill (Pl)\\
	}\label{11:ex25b}
	\z
\z

% example 26
\ea \label{11:ex26}
	\ea[]{
		\gll Nie parkowali-śmy tutaj samochodu.\\
     		 \textsc{neg} park.\textsc{part.pl.m-aux.1pl} here car\\
		\glt `We didn’t park the car here.'
	}\label{11:ex26a}
	\ex[*]{
			 Nie-śmy parkowali tutaj samochodu. \hfill (Pl)\\
	}\label{11:ex26b}
	\z
\z
A corresponding variation is observed in \ili{Czech}, which is also a T\textsuperscript{0}{}-initial language. Thus, \isi{negation} is adjoined to the \textit{l-}\isi{participle}, and it may not be adjoined to the auxiliary `be'. However, \isi{negation} adjoins to the verb `be' when it is used as a copula. The distributional contrast is presented in \REF{11:ex27} and \REF{11:ex28}.

% example 27
\ea \label{11:ex27}
	\ea[]{
		\gll Přišel jsi.\\
     		 come.\textsc{part.sg.m} are.\textsc{aux.2sg}\\
		\glt `You have come.'
	}\label{11:ex27a}
	\ex[]{
		\gll Nepřišel jsi.\\
    		 \textsc{neg.}come.\textsc{part.sg.m} are.\textsc{aux.2sg}\\
		\glt `You haven’t come.'
	}\label{11:ex27b}
	\ex[*]{
		\gll Nejsi přišel.\\
    		 \textsc{neg.}are.\textsc{aux.2sg} come.\textsc{part.sg.m}\\
             \glt Intended: `You haven't come.'{}\hfill (Cz, \citealt{toman1980})
    }\label{11:ex27c}
    \z
\z

%example 28
\ea \label{11:ex28}
	\ea[]{
		\gll Jsi hlupák / zdráv / na řadě.\\
     		 are.\textsc{2sg} idiot {} healthy {} on row\\
		\glt `You are an idiot / healthy / It’s your turn.'
	}\label{11:ex28a}
	\ex[]{
		\gll Nejsi hlupák / zdráv / na řadě.\\
     		 \textsc{neg.}are.\textsc{2sg} idiot {} healthy {} on row\\
		\glt `You’re not an idiot / healthy / It’s not your turn.'
	}\label{11:ex28b}
	\ex[*]{
		\gll Jsi nehlupák / nezdráv / ne na řadě.\\
        are.\textsc{2sg} \textsc{neg.}idiot {} \textsc{neg.}healthy {} \textsc{neg} on row\\
        \glt Intended: `You're not an idiot / healthy / It's not your turn.'\\{}\hfill (Cz, \citealt{toman1980})\\
	}\label{11:ex28c}
	\z
\z
Since in \ili{Czech} auxiliaries and copula verbs are morphologically identical (except for the fact that the auxiliary form is null and the copula form is overt in the 3\textsuperscript{rd} person singular and \isi{plural}), the position of \isi{negation} is clearly related to the categorial distinction between these two variants of the verb `be'. Thus, in both \ili{Czech} and \ili{Polish} the position of \isi{negation} and the verb is evidently contextually dependent.\footnote{\label{11:fn6}According to an anonymous reviewer, another factor that favors a categorial distinction between the copula and the auxiliary is the different timing of their loss in East \ili{Slavic} languages such as \ili{Russian}.} It is not a result of statistical frequency and it is not contingent on the \isi{head directionality} of TP. 

\section{Conclusion} \label{11:s4}

To conclude, this paper examined arguments provided in the literature, mainly by \citet{pancheva2005,pancheva2008}, in favor of head finality in \ili{Slavic} on the basis of diachronic changes in the placement of clitics in the history of \ili{Bulgarian} as well as the syntax of participles and the position of \isi{negation} in Old Church \ili{Slavonic}. It has showed that there is little evidence in support of head finality in Old \ili{Slavic}, and that this claim is also challenged by empirical facts concerning the distribution of the auxiliary `be' in the history of \ili{Bulgarian}. Furthermore, the diagnostics used in favor of the head final analysis have been demonstrated to give wrong predictions when applied to the same patterns found in Modern \ili{Slavic}.




\section*{Abbreviations}

\begin{tabularx}{.5\textwidth}{@{}lQ@{}}
\textsc{aor}&aorist\\
\textsc{aux}&auxiliary\\
Bg&{Bulgarian}\\
\textsc{cl}&{clitic}\\
\textsc{cond}&conditional\\
Cz&{Czech}\\
\textsc{dat}&{dative}\\ 
\textsc{dual}&dual number\\
\textsc{f}&{feminine}\\
\textsc{foc}&focus particle\\
\textsc{gen}&{genitive}\\
\textsc{imp}&imperfect tense\\
\textsc{inf}&{infinitive}\\
\textsc{m}&{masculine}\\
\end{tabularx}%
\begin{tabularx}{.5\textwidth}{@{}lQ@{}}
\textsc{n}&{neuter}\\
\textsc{neg}&{negation}\\
OE&Old {English}\\
OCS&Old Church {Slavonic}\\
\textsc{part}&{participle}\\
\textsc{past}&{\isi{past} tense}\\
Pl&{Polish}\\
\textsc{pl}&{plural}\\
\textsc{pres}&\isi{present}\\
\textsc{q}&question particle\\
\textsc{refl}&reflexive\\
S-C&Serbo-{Croatian}\\
\textsc{sg}&singular\\
\textsc{top}&topic\\
\end{tabularx}

\section*{Acknowledgements}

I wish to thank Željko Bošković, Hakyung Jung, Tanja Milićev, the FDSL-12 and FASL-25 audiences, and two anonymous reviewers for very helpful comments and discussion. All errors are mine.

\sloppy
\printbibliography[heading=subbibliography,notkeyword=this]

\end{document}
