\chapter{Why communication is central to meaning} \label{ch:why communication is central}

%\section{What is Meaning?} \label{sec:what is meaning}

\is{ontology|(}There are many kinds of things in the world. There are individuals such as planets and stars, ants and humans, tables and chairs, and books and civilizations. Each of these individuals has an infinite number of properties such as having a size and shape or being more or less complex. And these individuals also stand in relations such as a planet circling a star, an ant being smaller than a human, a chair being beside a table, or a book belonging to modern times. Complexes of these, individuals having properties or standing in relations, form \emph{infons}, and collections of infons form situations, parts of the world we occupy as agents and move about in. A situation may contain the earth circling around the sun, a person on earth sitting at a desk writing a book, and a scurrying ant. Such an inventory, consisting of individuals, properties, relations, infons, situations, and some other kinds of entities, makes up \emph{reality}.


All agents, whether ants or humans, carve up reality and create an \emph{informational space} or \emph{ontology} to navigate the world. That is, while the entities in such a space are part of reality, they need to be individuated as distinct entities for them to become part of an agent's informational space. An ant does not individuate a mountain just as a person may not discriminate between two shades of blue.

In carving up reality, one of the key things agents do is discover or create connections or constraints between two or more items of information such as, say, smoke and fire or an utterance and its content. When such a connection occurs with a certain regularity, we call it a \emph{meaning}. Indeed, we say that smoke means fire and, likewise, that an utterance of ``There is a fire'' in suitable circumstances means there is a fire in some situation. Reality is full of such constraints or meanings. Science, broadly construed, is just the attempt to comprehend them and, in this sense, it can be said to be \emph{the search for meaning}.\is{ontology|)}


\section{Semantics} \label{sec:semantics}
This vast domain of constraints in reality can be roughly divided into natural and artificial, that is, those that are independent of humans and those that depend on them. Smoke meaning fire is a natural constraint and an utterance of ``There is a fire'' meaning there is a fire in some situation is an artificial constraint.


Some artificial constraints like the one above involve \emph{symbol systems}\is{symbol systems|(} that enable a person to represent one object by another and convey this to another person. A traffic light tells one to stop or to go. Representation does not just exist in the world; it requires agency. It appears that a symbolic consciousness emerged a few hundred thousand years ago with the \isi{Neanderthals}, or possibly much earlier with \emph{Homo erectus},\is{Homo erectus} although this has yet to be confirmed.

Semantics is or ought to be the study of symbol systems. In some traditions, this discipline is called semiotics\is{semiotics} or semiology.\is{semiology} My use of ``semantics'' is considerably broader than simply the representing of utterance content as in formal \is{formal semantics}semantics.\footnote{I say more about this in \sectref{sec:semantics and pragmatics}.} One of the most complex symbol systems is (verbal) language and so semantics has often come to be more narrowly identified with the study of linguistic meaning. A study of four traditions -- Sanskrit, Greek, Hebrew, and Arabic -- reveals that semantics emerged as an independent discipline roughly three thousand years ago from the exegesis of mostly religious texts and concerned the relationship between language, reality, and knowledge.\footnote{See \citet[286]{bhsv:es}. See also \citet{db:cwp} for other traditions.} Its original context was \emph{communicative}, initially between people and divine powers when the world itself was largely read as signs from above. Today, semantics belongs primarily to philosophy, linguistics, psychology,\is{psychology} and artificial intelligence, and derivatively to other fields, each studying it from different points of view and with different ends. The reason for its wide scope is that symbolic meaning is central to life itself.\is{symbol systems|)}

Arguably, its main problem is to understand how language acquires meaning. How is it that an utterance, whether spoken or written, carries one or more meanings from a speaker to an addressee? Because language has so many varied devices to convey meaning, this turns out to be a very large question made up of very many subquestions. To follow some parts of this problem, consider a simple puzzle concerning names.

 
\section{A classic example}\label{sec:classic example}

Consider the following sentence:

\begin{exe}
\ex Hesperus is Phosphorus. ($\psi$)\label{ex:1}
\end{exe}

Many readers will recognize $\psi$ as a variant of a sentence considered by Gottlob \citet{frege:sr},\ia{Frege, Gottlob@Frege, Gottlob|(} perhaps the founding figure of modern semantics. A first attempt at apprehending its content might be to say that a name refers directly to the object it names. For $\psi$ this would imply that both the words \Expression{Hesperus} and \Expression{Phosphorus} refer to the planet Venus. Because \Expression{is} can be taken to stand for equality, we can infer that $\psi$ asserts the same thing as ``Hesperus is Hesperus.'' But this is a problem because the latter utterance is a trivial identity whereas $\psi$ involves an empirical discovery. This is \isi{Frege's puzzle} of informative identities: how can we explain the new information conveyed by $\psi$?

Frege's solution was to posit an intermediate layer of meaning called the \emph{sense} of each name and the object Venus they both stand for would be the \emph{reference}. His idea was that every name expresses a sense which in some cases leads to a reference. Though the referents of the two names in $\psi$ are identical, their senses are different, the first being the evening star and the second being the morning star, and it is these different senses that capture the cognitive significance of $\psi$ unlike the trivial identity ``Hesperus is Hesperus'' where even the senses are the same. This solution runs into insuperable problems with more complex sentences although his two-tier account of meaning was on the right track. For now, I want to point out a pivotal thing he left implicit.

He did not sufficiently articulate the distinction between a sentence and an utterance. A sentence -- like a word -- is an abstract object whereas an utterance of a sentence is an action performed in some situation.\is{utterance} Everyone knows the difference between the two but Frege mentioned it mainly when he discussed \is{force}force.\footnote{By ``force'' Frege meant assertion or other such acts performed as part of uttering a sentence.} But it is vital even for his theory of sense and reference because a sentence by itself does not have a content: it is only when it is \emph{used} appropriately in an utterance situation that it acquires a content. It appears to us that $\psi$ has a content only because we implicitly evaluate it as an utterance in some unspecified but familiar situation. But a little thought would reveal that in a different situation a speaker may be referring to the figure in Greek mythology who was a personification of the evening star and not to the evening star itself. In other words, \Expression{Hesperus} is \emph{ambiguous}\is{ambiguity} and the addressee needs to know which meaning is intended. Differently put, Frege abstracted from its use and assumed the meaning of interest to him. A more complete theory would have to start by showing first how the utterance is disambiguated.

Frege and the logicist tradition\is{logicism} he inaugurated correctly identified reference\is{reference} as a key property of language. In so doing, the ideal language philosophers succeeded in stating one of the two main subproblems of semantics: how does language relate to reality, how does it connect with the world? The other main subproblem, bequeathed from ancient times, concerns the possible knowledge it brings about. But by sticking to sentences and ignoring utterances they left out the key means to their solution: the use of language -- or \emph{communication}. Just as Frege did, many who followed him chose to abstract from communication and focus on particular linguistic devices and their problems, as we have just seen with the puzzle of informative identities. They did this generally by employing truth as a way to get at meaning because Frege had pointed to a truth value as the reference of a sentence. It was the later \citet{wittgenstein:pi}\ia{Wittgenstein, Ludwig@Wittgenstein, Ludwig|(} followed by \citet{austin:pp, austin:htdtww} and \citet{grice:sitwow} and other ordinary language philosophers who realized the importance of seeing language as a situated activity but did not quite succeed in unraveling this elusive concept.\footnote{See the fascinating account by \citet{sen:swg} of how the economist Piero Sraffa\ia{Sraffa, Piero@Sraffa, Piero} (and indirectly the political theorist Antonio Gramsci)\ia{Gramsci, Antonio} may have influenced Wittgenstein in making his celebrated move from his early logicist ideas to his later use-based ideas. As Sen puts it, ``Wittgenstein told a friend (Rush Rhees, another Cambridge philosopher) that the most important thing that Sraffa taught him was an ``anthropological way'' of seeing philosophical problems. In his insightful analysis of the influence of Sraffa and Freud, Brian \citet[36--39]{mcguinness:wt} discusses the impact on Wittgenstein of ``the ethnological or anthropological\is{anthropology} way of looking at things that came to him from the economist Sraffa.'' While the \emph{Tractatus} tries to see language in isolation from the social circumstances in which it is used, the \emph{Philosophical Investigations} emphasizes the conventions and rules that give the utterances particular meaning. The connection of this perspective with what came to be known as `ordinary language philosophy' is easy to see.''}

It would seem there are two ways to approach the connection between language and world. One is to break down language into various constructions such as noun and verb phrases, select some specific devices, and create a framework to see how they work via their truth conditions. Some adherents still stick to sentences as the bearers of truth values but many have shifted to utterances because the situated nature of language makes the former increasingly difficult as the field advances and more details come into view.\largerpage

The other is to recognize that unless one first fathoms simple cases of communication in their full complexity across a number of devices, one cannot create a \emph{general} framework that will be able to address the many problems they raise in a uniform way. And doing so allows us to largely avoid the tricky notion of truth because meaning would be derived from its use in communication.

The first approach is bottom-up and the second top-down. There is a grave danger that the first will lead to a proliferation of incompatible theories presupposing different views of meaning at a more foundational level. My view is that once communication and content are understood thoroughly with simple constructions in a general way, it will be much easier to tackle more complex constructions with an elaboration of the same tools so that the many subquestions of semantics all cohere into an integrated solution to its main problem.

From a related but different vantage point, \citet[Chapter~3]{dummett:oap} offers the following insights.

\begin{quote}
In fact, it is by grasping what would render [a sentence] true that we apprehend what it means. There can therefore be no illuminating account of the concept of truth which presupposes meaning as already given: we cannot be in the position of grasping meaning but as yet unaware of the condition for the truth of propositions. Truth and meaning can only be explained \emph{together}, as part of a single theory. (page 15)
\[\vdots\]
Or, at least, they have to be explained together so long as Frege's insight continues to be respected, namely that the concept of truth plays a central role in the explanation of sense. On this Fregean view, the concept of truth occupies the mid-point on the line of connection between sense and use. On the one side, the truth-condition of the sentence determines the thought it expresses, in accordance with the theory developed by Frege and adapted by Davidson;\ia{Davidson, Donald} on the other, it governs the use to be made of the sentence in converse with other speakers, in accordance with the principles left tacit by both of them. \emph{That leaves open the possibility of describing the use directly, and regarding it as determining meaning, relegating the concept of truth to a minor, non-functional role.} [my italics] This was the course adopted by Wittgenstein in his later work. The concept of truth, no longer required to play a part in a theory explaining what it is for sentences to mean what they do, now really can be characterized on the assumption that their meanings are already given. (page 19)
\[\vdots\]
Rather than characterizing meaning in terms of truth-conditions, and then explaining how the use of a sentence depends upon its meaning as so characterized, this approach requires us to give a direct description of its use: this will then \emph{constitute} its meaning.

The disadvantage of this approach lies in its unsystematic nature. This, for Wittgenstein, was a merit: he stressed the diversity of linguistic acts and of the contributions made to sentences by words of different kinds. Systematization is not, however, motivated solely by a passion for order: like the axiomatic presentation of a mathematical theory, it serves to isolate initial assumptions. A description of the use of a particular expression or type of sentence is likely to presuppose an understanding of a considerable part of the rest of the language: only a systematic theory can reveal how far linguistic meaning can be explained without a prior supply of semantic notions. The ideal would be to explain without taking any such notions as given: for it would otherwise be hard to account for our coming by such notions, or to state in any non-circular manner what it is to possess them. It is unclear from Frege's work whether this ideal is attainable. The indefinability of truth does not, of itself, imply the inexplicability of truth, although Frege himself offered no satisfactory account of elucidations that fall short of being definitions. Perhaps the concept of truth can be adequately explained if a substantive analysis of the concepts of assertion and of judgement is feasible: but Frege leaves us in the dark about this. Wittgenstein, equally, leaves us in the dark about whether his programme can be executed: it is another disadvantage for the repudiation of system that it leaves us with no way of judging, in advance of the attainment of complete success, whether a strategy is likely to be successful. (pages 20--21)
\end{quote}


Dummett provides a further reason to pursue meaning via communication: to sidestep the interdependence of truth and meaning. My approach squarely pursues the possibility of describing use directly and deriving meaning from use. However, contra Wittgenstein's ``repudiation of system'' I will attempt a complete and systematic account of communication and meaning. 

%It turns out that there \emph{is} a functional but minor need for truth so things are unfortunately not as simple as they might have been. But this interdependence is not viciously circular as implied by Dummett as I explain in \sectref{sec:meaning and truth}.  

There is a third advantage to reducing meaning to communication. It allows us to see meaning as a natural phenomenon as opposed to something otherworldly. With the truth-conditional strategy, this possibility remains murky at best. \citet{grice:sitwow} may have been the first person to outline a concrete program for such a reduction even though its details are different from my approach. I discuss his framework in \chapref{ch:Grice}.

Lastly, ever since the later Wittgenstein mentioned it, various writers have been trying to pin down the normativity\is{normativity} of language and meaning. I will try to show that even this intangible property emerges more clearly once communication is grasped in a precise way.\ia{Frege, Gottlob@Frege, Gottlob|)}\ia{Wittgenstein, Ludwig@Wittgenstein, Ludwig|)}

%I believe Chomsky's doubts about the possibility of a theory of performance\footnote{``As soon as questions of will or decision or reason or choice of action arise, human science is at a loss.'' Noam Chomsky\ia{Chomsky, Noam} (1978) Incidentally, performance, an old notion of Chomsky's, is usually viewed in a much narrower psycholinguistic way and solely from the point of view of the speaker. I am using the term more broadly to depict the entire process of communication not only as it involves a speaker and an addressee in the small but also as it involves all of society in the large.} have their origins in the work of the later Wittgenstein, and if the strategy I adopt proves to be successful, then they will also be laid to rest.  


\section{A snapshot of semantics} \label{sec:snapshot}

%In principle, linguistic meaning can serve either the function of thought or communication. There does not seem to be any third possibility. But since language is sparked in babies through communication, it is arguable that the function of communication is primary and thought secondary.\footnote{Chomsky and his followers seem to have championed the first of these as the main purpose of communication.} At least it is unlikely that thought that requires language precedes communication. This conclusion is further borne out when one actually studies what linguistic meaning might be, how it emerges, and how it is used. Why, for example, does ``desk'' conventionally mean \emph{something made for writing}? It is hard to imagine such a meaning not arising in a whole community of interacting speakers and addressees. Once such a socially sourced word and meaning enter a person's lexicon, they can also be used in their thinking but the order in which these two things occur is clear.

%Given the number of benefits of approaching meaning via communication, it is surprising that it has been so little studied. This may partly have to do with its difficulty. It 

Communication has at least six dimensions: its contextual or situated nature, its involving actions or utterances and interpretations, its encompassing epistemic, practical, and social interactions between speaker and addressee, and its being computationally tractable. For the moment, they can be understood intuitively.

\tabref{tab:SummaryofTheories} displays how some prominent approaches to communication fare against these dimensions.

\begin{table}
\resizebox{\textwidth}{!}{\begin{tabular}{lccccc}
\lsptoprule
	& Logicism\is{logicism} & Wittgenstein\ia{Wittgenstein, Ludwig@Wittgenstein, Ludwig} & Austin\ia{Austin, J. L.@Austin, J. L.} & Grice\ia{Grice, Paul@Grice, Paul} & Lewis\ia{Lewis, David@Lewis, David}							\\\midrule
context					& marginal 	& implicit	& implicit	& implicit	& implicit		\\	
action					& partial		& yes	& yes	& yes	& yes			\\	
%force					& no		& no		& yes	& no		& no		& no		& yes			\\	
epistemic		
interaction				& no		& no		& implicit		& yes	& yes	\\	
practical 				
interaction				& no		& implicit		& no		& no 	& partial			\\	
%choice					& no		& no		& no		& no		& partial			\\	
social					
interaction				& no		& no 	& no		& no		& no	 	\\ 					
computable				& no		& no		& no		& partial		& partial			\\ \lspbottomrule
%structure				& no		& no		& no		& no		& no	       \\ \hline\hline
\end{tabular}}
\caption{Summary of theories of communication\label{tab:SummaryofTheories}}
\end{table}

\noindent The table identifies how different frameworks deal with the \emph{how} of communication, not the what. It deals with its process rather than its content. Two distinct ``yesses'' in the same row do not imply that both accounts do equal justice to the dimension. As will become clear, my framework -- called Equilibrium Semantics -- scores a ``yes'' in all rows except the one pertaining to social interaction, where it scores a ``partial'' rating.


\section{Equilibrium Semantics} \label{sec:equilibrium semantics}

%that the study of language is better seen as a social science, even a branch of economic science (understood as the science of choice), rather than as a branch of cognitive psychology, although the latter discipline is crucially important not just for our innate capacities but also for how we choose and act

\noindent Building on my books \emph{The Use of Language} (\citeyear{parikh:ul}) and \emph{Language and Equilibrium} (\citeyear{parikh:le}), I present a new framework in this book that can serve as a core part of a science of communication usable primarily by philosophy, linguistics, and artificial intelligence, and also perhaps by cognitive and social science, especially economics, and related disciplines. I will try to show that the models I develop are not only philosophically sound and mathematically solid but also computationally tractable and empirically adequate, and I will connect them to the subjects above in multiple ways. The nature of communication is such that I will also have to draw upon insights from these diverse areas, especially philosophy, linguistics, psychology,\is{psychology} and economics. This poses a challenge for both the writer and reader, as I have to strive to make all of it accessible and you may have to take an interest in occasionally unfamiliar ideas, concerns, and methods. But I believe the journey will be well worth the effort as I offer a substantial advance on the state of the art with results that are both more general and more precise.

In my view, the mathematics best suited for modeling communication comprises certain innovations in game theory and situation theory that I have described earlier and that I add to here. \is{game theory}\is{situation theory}

Informal game-theoretic thinking is quite old, going back to classical times in multiple cultures. Possibly its modern roots can be traced to Machiavelli\ia{Machiavelli, Niccolo} and Hobbes.\ia{Hobbes, Thomas} Its first formal results were obtained by \citet{zermelo:chess} in connection with the game of chess. As \citet[1]{myerson:gt} says, \emph{interactive decision theory} might have been a more descriptively accurate name for the theory as it allows one to consider practical and epistemic interactions between people in a computationally tractable way.

\citet{lewis:c} was the first to introduce game theory into semantics but he used rather simple games from \citet{schelling:sc} and viewed them as dispensable scaffolding. At this early moment, game theory was not seen as offering a framework that could partly replace logic. The Austin-inspired idea of use conditions as an alternative to truth conditions was often entertained but such conditions were also seen as a part of logic.

As I have shown in my dissertation \emph{Language and Strategic Inference} (\citeyear{parikh:diss}) and since, game theory allows one to discard such logically stated use conditions and offer precise mathematical derivations of content instead. This work, together with some aspects of Lewis's way of modeling games, has led to a small but growing field in recent times with many papers and also books such as \citet{benz:gtp}, \citet{pietarinen:gtlm}, \citet{clark:mg}, and \citet{benz:lge}. \citet[Chapter~8]{slb:ms} describe some of my ideas in the context of distributed systems. The key difference between some of the game-theoretic linguistic approaches that have emerged and my framework is that the former are largely orthodox Gricean\ia{Grice, Paul@Grice, Paul} and the latter is not, as will become especially clear in this book. This has meant that the former have focused mainly on deriving Gricean implicatures whereas I have tried to derive literal meaning as well.  

The theory of situations was originally developed by \citet{bp:sa} and \citet{b:sl} and drew upon ideas from \citet{shannon:mtc} and \citet{austin:htt}.\ia{Austin, J. L.@Austin, J. L.} It is essentially a qualitative account of information and I have already described some of its key ideas informally, especially the idea of a situation which forms the context for an utterance. Later we will see how it also allows us to capture the content of an utterance in a very general way.\pagebreak

%\emph{Communication and Content} is entirely self-contained so no acquaintance with my past books is required and, in addition, the game theory is introduced from scratch through very simple examples. Naturally, the topics covered in each volume are somewhat different though each successive publication represents a deepening of the overall perspective.

I will confine myself to language although the primary means by which we convey meanings{\interfootnotelinepenalty=10000\footnote{I will use ``meaning'' and ``content''\is{content} interchangeably in many contexts although I will also use ``content'' more widely to include the grammatical and phonetic structure of the sentence uttered. There will be times when I will explicitly distinguish conventional meanings\is{meaning!conventional} from referential meanings,\is{meaning!referential} the latter being the same as semantic contents.}} to one another include words, images, and gestures. Indeed, gestures preceded words and images in our evolution. My account can be extended  straightforwardly to images and gestures as well as to other symbol systems and so will also be of interest to the corresponding fields.

One of my aims in this book is to show how little needs to be assumed to construct a science of communication, mainly facts about ontology, the partial rationality of agents, and a language and its grammar and phonetic system. I want to do this in an informally foundational way so I will introduce various concepts when they are required, in a more or less linear fashion. I will therefore start with information as the most basic ``building block'' instead of plunging directly into communication. 

A science of communication rests on a science of information for two reasons. The first is that the world is information as such and we are embedded in it. Communication therefore occurs within an informational space and cannot be understood without it. The second is that communication involves conveying information in one form or another even when we ask questions, issue commands, or express feelings. It relays \emph{content} and has an \emph{effect} and both are informational, although the former affects us cognitively and the latter may involve the whole person. So I begin with a brief tour of this pervasive landscape.
