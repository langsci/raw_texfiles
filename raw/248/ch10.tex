\chapter{Universality, Frege's principles, indeterminacy, and truth}
\section{The universality of games of partial information}\label{sec:universality}

The games of partial information shown so far were formed with respect to particular examples but it is possible to construct such games more generally by the maps $g_u$ and $g'_u$ on utterances to obtain two monoidal systems $({\cal G}, \otimes)$ and $({\cal G}', \otimes')$ with zeroes as mentioned in \sectref{sec:4.3}.\footnote{The identity of $({\cal G}, \otimes)$ is the empty game $g_e$, which is just the trivial game with a single initial node (with prior probability 1) and a branch labeled with $e$, the empty string, issuing from it, a further branch labeled $\mathbf{1}$, the identity of $({\cal I}, \odot)$, issuing from the node that ends the first branch, and any terminal payoff, preferably $0$. The reader should check that for any game $g \in {\cal G}$, we get $g \otimes g_e = g_e \otimes g = g$ and that $g_u(e) = g_e$ and $f_u(g_e) = \mathbf{1}$. There is also a similar zero game $g_0$ with a single branch labeled $0$, the zero of $({\cal L}, \circ)$, and a further branch labeled $\mathbf{0}$, the zero of $({\cal I}, \odot)$, such that $g \otimes g_0 = g_0 \otimes g = g_0$ for all $g \in {\cal G}$ and that $g_u(0) = g_0$ and $f_u(g_0) = \mathbf{0}$. Pictorially:

%%%\item Identity of $\cal L$: $e$    %\item Zero of $\cal L$: $0$     %\item Identity of $\cal I$: $\mathbf{1}$   %\item Zero of $\cal I$: $\mathbf{0}$

\ea Identity of $({\cal G}, \otimes)$: $g_e$\\
\begin{picture}(120,40)
% Upper Tree
%\put(171,50){\input{unit1}}
\put(0,0){\input{figures/unit1}}
%\put(224,60)
\put(0,0)
{\begin{picture}(120,54)
%{\begin{picture}(120,15)
\put(108,27){\circle*{3}}
\put(54,27){\vector(1,0){54}}
\end{picture}}
% End of Upper Tree
%--------------------------------
% Upper Labels
\put(0,0)
{\begin{picture}(120,54)
\put(0,18){\makebox(0,0){$\rho_e = 1$}}
\put(0,36){\makebox(0,0){$s_e$}}
% % % \put(54,36){\makebox(0,0){$t_e$}} % Author wanted to remove this label, e-mail on Sep 28, 2019
\put(27,33){\makebox(0,0){$e$}}
\put(81,33){\makebox(0,0){$\mathbf{1}$}}
\put(126,27){\makebox(0,0){$0,0$}}
\end{picture}}
% End of Upper Labels
%--------------------------------------
% End of Figure 5
%-----------------------------------
\end{picture}
\z
\ea Zero of $({\cal G}, \otimes)$: $g_0$\\
%Figure 5
\begin{picture}(120,40)
% Upper Tree
%\put(171,50){\input{unit1}}
\put(0,0){\input{figures/unit1}}
%\put(224,60)
\put(0,0)
{\begin{picture}(120,54)
%{\begin{picture}(120,15)
\put(108,27){\circle*{3}}
\put(54,27){\vector(1,0){54}}
\end{picture}}
% End of Upper Tree
%--------------------------------
% Upper Labels
\put(0,0)
{\begin{picture}(120,54)
%{\begin{picture}(120,15)
\put(0,18){\makebox(0,0){$\rho_0 = 1$}}
\put(0,36){\makebox(0,0){$s_0$}}
% % % \put(54,36){\makebox(0,0){$t_0$}} % Author wanted to remove this label, e-mail on Sep 28, 2019
\put(27,33){\makebox(0,0){$0$}}
\put(81,33){\makebox(0,0){$\mathbf{0}$}}
\put(126,27){\makebox(0,0){$0,0$}}
\end{picture}}
% End of Upper Labels
%--------------------------------------
% End of Figure 5
%-----------------------------------
\end{picture}
\z

The identity $g'_e$ and zero $g'_0$ of $({\cal G}', \otimes')$ can be defined analogously by replacing $\mathbf{1}$ with $t_e$, the identity of $({\cal T}, \star)$, and by replacing $\mathbf{0}$ with $t_0$, the zero of $({\cal T}, \star)$, in the above games.} The solution to locutionary global games can then be compactly described by the commutative diagram in Figure~\ref{fig:cd2}.

\begin{figure}[h]
\begin{picture}(350,160)(10,-140)
\xymatrix@=1.5in{ & ({\cal L},\circ)  \ar[d]|{(g_u,g'_u)} \ar[dl]_{{\cal C}_u} \ar[dr]^{{\cal C}'_u} \\
({\cal I},\odot) & ({\cal G},\otimes) \times ({\cal G}',\otimes') \ar[l]_{f_u} \ar[r]^{f'_u}  & ({\cal T},\star)  }
\end{picture}
\caption{The commutative diagram: ${\cal C}_u = f_u \circ g_u$ and ${\cal C}'_u = f'_u \circ g'_u$}\label{fig:cd2}
\end{figure}

${\cal C}_u$ and ${\cal C}'_u$ are the semantic and syntactic content maps and they can be decomposed into two sets of maps $(g_u,g'_u)$ and $(f_u, f'_u)$ where the latter represent functions from the relevant games to their solutions. In general, families of interdependent games corresponding to a whole utterance will need to be considered in order to specify a solution as their interdependence through the prior conditional probabilities requires their simultaneous solution. This diagram makes the uniform treatment of all contents by Equilibrium Linguistics even clearer. The following fundamental results establishing the universality\is{universality} of games of partial information for the semantics and syntax of natural language are stated without proof as they are straightforward.\footnote{They require formal definitions of $g_u$ and $g'_u$ as shown in the Appendix.} They are so basic that from the standpoint of Equilibrium Linguistics they could well be called ``The Main Theorems of (Linguistic) Communication.''

\begin{theorem}

$g_u: \cal L \functionarrow \cal G$ and $g'_u: \cal L \functionarrow {\cal G}'$ are isomorphisms. \label{thm:isomorphism}

\end{theorem}

\begin{theorem}

Given a semantic interpretation function ${\cal C}_u: \cal L \functionarrow \cal I$ and given the semantic game function $g_u: \cal L \functionarrow \cal G$, there is a unique function $f_u: \cal G \functionarrow \cal I$ such that ${\cal C}_u = f_u \circ g_u$.

\end{theorem}

\begin{theorem}

Given a syntactic interpretation function ${\cal C}'_u: \cal L \functionarrow \cal T$ and given the syntactic game function $g'_u: \cal L \functionarrow {\cal G}'$, there is a unique function $f'_u: {\cal G}' \functionarrow \cal T$ such that ${\cal C}'_u = f'_u \circ g'_u$.

\end{theorem}

These theorems imply that there is always a unique solution to the locutionary global game. It may happen that the solution involves a mixed strategy and so in fact involves multiple contents, each with some probability as in a pun, but this mixed strategy is then uniquely given. These results involving the universality of $g_u$ and $g'_u$ and therefore of semantic and syntactic games of partial information have the following fundamental consequence: if there is \emph{any} other theory (e.g.\ mainstream rule and convention based semantical theory, syntactic theory, optimality theory, relevance theory, etc.) that succeeds in providing an account of the locutionary content functions ${\cal C}_u$ and ${\cal C}'_u$, the theorems say that it \emph{has} to be essentially equivalent to the maps $f_u$ and $f'_u$. Thus, they state that games of partial information are essential to communication or are ``universal.''\footnote{For more about this, see \citet[Section~4.11]{parikh:le}.} If we wish to represent mixed semantic-syntactic products then the two triangles in Figure~\ref{fig:cd2} can be folded into one by defining just a single monoid of semantic-syntactic games and a single map to it and a single solution map from it. 

Finally, if phonetics is included in the framework, there would be a further set of phonetic games and more or less the same kind of treatment would be available where the words uttered and the equilibrium parse and meaning can be simultaneously derived from the speech wave produced by a speaker.


\section{Frege's compositionality and context principles} \label{sec:compositionality/context}

It should be evident that Theorem~\ref{thm:simple equation} in \sectref{sec:solving locutionary global games} is related to Frege's century-old principle of compositionality and context principle because it gives a way to determine the referential meaning\is{meaning!referential} of an utterance from first principles just as these principles attempt to do. The compositional principle states that the meaning of a composite expression is a function of the \emph{independently given} meanings of its component parts and the context principle states that the meaning of a word or subsentential expression is what it contributes to the meaning of the whole sentence. Many, notably \citet[4--5]{dummett:f}, have remarked on the tension between these two principles because one requires the meaning of a sentence to depend on the meanings of its constituent words and the other requires the meanings of the constituent words to depend on the meaning of the sentence.\footnote{There are different ways of interpreting especially the context principle as pointed out by \citet{ms:cp}. Indeed, they also show that there were quite remarkable debates in classical Indian philosophy of a similar sort from roughly the 5th century CE to the 17th. Their key proponents espoused ``sentence holism'' (similar to the context principle) and what is called ``designation before connection'' (similar to compositionality) and ``connected designation'' (intermediate between the two principles). These discussions went on for centuries. It is a shame that Western philosophers do not generally include such non-Western ideas in their writings.}

The first thing to note is that the Fundamental Equation of Equilibrium Linguistics is about utterances rather than sentences as contrasted with Frege's two principles. That is, it takes account fully of the context in which sentences are uttered as sentences by themselves cannot convey contents.\footnote{See \citet{strawson:or} and \citet[20--21 and Section~2.7]{parikh:le} and \sectref{sec:classic example} of this book.} I believe there is currently no other competing account that shows how to compute the (locutionary) meaning of an utterance from scratch. Following \citet{grice:sitwow},\ia{Grice, Paul@Grice, Paul} the tendency is simply to say that it is given by convention with the actual processes of lexical and structural disambiguation and pronoun reference fixing left completely mysterious.

The second thing the Fundamental Equation makes clear is that the optimal (referential) meanings of words, subsentential expressions, and sentences are \emph{interdependent}: they codetermine one another via the interdependence of prior probabilities and the concomitant interdependence of the corresponding locutionary games of partial information as expressed through a fixed point process. This shows that Frege's principle of compositionality does not hold for referential meanings because the meanings of the component parts in turn depend on the meanings of the other parts and also on the whole. It is possible to re-express the fundamental equation as follows:
\begin{eqnarray*}
{\cal C}_u(\hbox{S}) & = & h_1({\cal C}_u(\hbox{NP}), {\cal C}_u(\hbox{VP})) \\
{\cal C}_u(\hbox{NP}) & = & h_2({\cal C}_u(\hbox{S}), {\cal C}_u(\hbox{VP})) \\
{\cal C}_u(\hbox{VP}) & = & h_3({\cal C}_u(\hbox{S}), {\cal C}_u(\hbox{NP}))
\end{eqnarray*}
\noindent where $h_1$, $h_2$, $h_3$ are suitable functions derived from Equation~\ref{eq:simple} in \sectref{sec:maintheorem}. The function $h_1$ in particular is obtained by taking the product of the relevant components of $z^{\star}$ in Equation~\ref{eq:simple} to get ${\cal C}_u(S)$. A similar result can be derived from the more general Equation~\ref{eq:simple with payoffs} in \sectref{sec:maintheorem} as well. The noun phrase NP and verb phrase VP can be further broken down into their constituents in these equations. It is the presence of the second and third equations above that disqualifies the Fregean principle of compositionality since the principle requires the meanings of the component parts to be determined independently and directly. This set of equations can perhaps be more perspicuously expressed as follows:
\begin{eqnarray*}
{\cal C}_u(\hbox{S}) & = & h_1({\cal C}_u(\hbox{NP}), {\cal C}_u(\hbox{VP})) \\
{\cal C}_u(\hbox{NP}) & = & h_4({\cal C}_u(\hbox{VP})) \\
{\cal C}_u(\hbox{VP}) & = & h_5({\cal C}_u(\hbox{NP}))
\end{eqnarray*}
\noindent and as follows:
\begin{eqnarray*}
{\cal C}_u(\hbox{S}) & = & h_1({\cal C}_u(\hbox{NP}), {\cal C}_u(\hbox{VP})) \\
{\cal C}_u(\hbox{NP}) & = & h_6({\cal C}_u(\hbox{S})) \\
{\cal C}_u(\hbox{VP}) & = & h_7({\cal C}_u(\hbox{S}))
\end{eqnarray*}
\noindent the first set corresponding to the first equality in Equation~\ref{eq:simple} and the second set to the second equality.

Here, too, it is the second and third equations in each set that disqualify Fregean compositionality. Only in the special case where there are no lexical ambiguities\is{ambiguity} will the Fregean principle work. In general, we will have to resort to Theorem~\ref{thm:simple equation} to determine the meanings of all expressions, whether simple or composite. This is a complex process of simultaneous interaction between the potential meanings of each component part and whole which results in the actual (optimal) meanings of the parts and the whole. Thus, the \emph{principle of compositionality} has to be replaced by the more general \emph{fixed point principle} as indicated by Equation~\ref{eq:simple} or its more general variant Equation~\ref{eq:simple with payoffs}.

For greater clarity, I urge the reader to see how the examples we have just discussed, \Expression{Bill ran} and \Expression{Fed raises interest}, fit into these equations. For example:
\begin{eqnarray*}
{\cal C}_u(\Expression{Bill ran}) & = & h_1({\cal C}_u(\Expression{Bill}), {\cal C}_u(\Expression{ran})) \\
{\cal C}_u(\Expression{Bill}) & = & h_4({\cal C}_u(\Expression{ran})) \\
{\cal C}_u(\Expression{ran}) & = & h_5({\cal C}_u(\Expression{Bill}))
\end{eqnarray*}
\noindent and:
\begin{eqnarray*}
{\cal C}_u(\Expression{Bill ran}) & = & h_1({\cal C}_u(\Expression{Bill}), {\cal C}_u(\Expression{ran})) \\
{\cal C}_u(\Expression{Bill}) & = & h_6({\cal C}_u(\Expression{Bill ran})) \\
{\cal C}_u(\Expression{ran}) & = & h_7({\cal C}_u(\Expression{Bill ran}))
\end{eqnarray*}

Also, since there are no conventional meanings at the level of phrases or sentences, Frege's principle automatically fails to hold at the level of conventional meaning.\is{meaning!conventional}

Thus, what I am calling the fixed point principle represents a generalization of Frege's principle of compositionality. But these very equations, the third set in particular, also show how the former principle generalizes the context principle because the optimal referential meanings of subsentential expressions depend on the optimal referential meaning of the whole sentence \emph{and} vice versa. Indeed, at this more general level, both principles are reconciled without any tension at all between them. This is a remarkable result.\is{meaning!pipeline view of}

But this is not all. The \emph{third} observation is that implicit in Frege's compositional principle (and, arguably, in the context principle) is the priority of syntax over semantics and the idea that semantics \emph{mirrors} syntax. The function of the meanings of the component parts of a sentence depends on first optimally parsing the sentence. It is only on the basis of this optimal parse that the component meanings can be properly \emph{identified} and \emph{composed}. The fixed point principle, on the other hand, asserts that semantics and syntax, and phonetics for that matter, are completely interdependent and more loosely \emph{reflect} one another. \emph{Nothing} has priority; everything is simultaneous and interdependent! In particular, each bit of semantics influences each bit of syntax and vice versa. Each component $z^{\ast}_k$ of the vector $z^{\ast}$ is affected by all the other components $z^{\ast}_{-k}$ (not to mention by all the suboptimal content values) irrespective of whether they are semantic or syntactic. All the (possible) contents are intermixed. When an utterance is being processed in real time, the fixed point principle will be modified to take account of the temporal appearance of words as I discuss in \sectref{sec:psycholinguistics}. But, in either case, the process of inferring the locutionary \emph{content} of an utterance -- its semantic, syntactic, and phonetic values -- involves a thoroughgoing interdependence. This, then, is a further generalization of both the compositionality principle and the context principle of Frege. 

Not only does Equation~\ref{eq:simple} provide a way to reduce the computation of locutionary content to a mere comparison of probabilities (or expected values in a more general setting as indicated by Equation~\ref{eq:simple with payoffs}), that is, to mere arithmetic, this fixed point principle of Equilibrium Linguistics encompasses all the relevant dimensions -- generalizing from sentence to utterance, generalizing the nature of the interdependence between word meaning and sentence meaning, and generalizing the relation between syntax and semantics -- seamlessly. Frege's principle of compositionality and the context principle are completely transcended. 

In fact, if these results are seen in the context of the discussion about Wittgenstein,\ia{Wittgenstein, Ludwig@Wittgenstein, Ludwig} Austin,\ia{Austin, J. L.@Austin, J. L.} and Grice\ia{Grice, Paul@Grice, Paul} in \sectref{sec:macro-semantics} and \chapref{ch:Grice}, they show how certain aspects of both Fregean ideas about reference and Wittgensteinian, Austinian, and Gricean ideas about use and communication are not only unified but also superseded. As I said in \citet{parikh:le} and \chapref{ch:romantic tradition}, the fundamental concepts of semantics are \emph{reference}, \emph{use}, \emph{indeterminacy}, and \emph{equilibrium}. The foregoing shows in a precise way how three of these four notions are unified in Equilibrium Linguistics. The earlier ideas are also superseded because they are rendered more precisely and perhaps more generally. 

Interestingly, Equation~\ref{eq:simple} and its more general variant Equation~\ref{eq:simple with payoffs} also apply to other symbol systems, including images and gestures. I discuss in \citet[Section~7.5]{parikh:le} how the meanings of pictures, for example, can be inferred from appropriate part-whole or mereological relationships in a visual ``utterance.'' The syntactic side of the equation has to be interpreted carefully as most symbol systems lack the elaborate structure that language possesses.

\section{Indeterminacy} \label{sec:indeterminacy}
Indeterminacy, the fourth notion mentioned above, is dealt with in some detail in \citet[Chapter~5]{parikh:le} so I will be brief here. It is pervasive in both locutionary and especially illocutionary communication for a variety of reasons. One of them is the occasional absence of full common knowledge with the result that each component of the relevant Communication Game can separate into distinct objective and subjective games. Apart from the speaker and addressee inferring slightly different contents, there can also be outright miscommunication. Another factor is the occasional presence of mixed strategies in the global game owing to equal prior probabilities in one or more games of partial information as can happen with puns where the multiple meanings are intended. An interesting aspect of indeterminacy, mentioned at the end of \sectref{sec:interpretation game}, is the uncertainty surrounding each agent's inferred meaning in the other agent's mind owing to the invisibility of interpretations and, often, also of addressee responses. A key source, perhaps completely overlooked by all writers on meaning, is the indeterminacy of utterance situations. When $u$ is differently carved out by the interlocutors, as happens frequently in literature and art and less often in ordinary communication, there is no unique meaning that can be ascribed. Because the speaker's intentions can be partially \emph{implicit} in examples like ``His weight is 150 lbs.,'' the primacy given to speaker meaning is purely a prejudice issuing from the misplaced Gricean \ia{Grice, Paul@Grice, Paul} focus on speaker meaning. Addressee meanings are equally important and, in general, there will be as many meanings as there are agents in a conversation. As a result, there is also no need to proclaim \emph{the death of the author} as \citet{barthes:da} did a half-century ago or \citet{wb:if} did much earlier, although less dramatically. However, Barthes was more correct than the New Criticism of Wimsatt and Beardsley\ia{Wimsatt, William K.@Wimsatt, William K.} and others in allowing for an open text with open-ended interpretations, a consequence Equilibrium Linguistics effortlessly enables simply by altering $u$ and related aspects of the discourse in a systematic and methodical manner rather than through vague declarations of the kind many writers influenced by Continental philosophy revel in.

A major source of indeterminacy that I did not consider in my previous book is \isi{vagueness}. I give a thorough account of this phenomenon in the next chapter.

I will also say a little more in \partref{part:IV} about indeterminacy as it relates to illocutionary meaning because that is where the bulk of the indeterminacy lies, especially when $u$ is itself indeterminate, but it should be apparent how this fourth concept also ties in smoothly with reference, use, and equilibrium, and flows ineluctably out of the framework. It is something that has largely been neglected in the literature, partly owing to a lack of precise ways to handle it and partly owing to sheer prejudice against the very idea, as it threatens the rigid picture of meaning the ideal language philosophers bequeathed to the field. Uncertainty, generally, can be quite unsettling. But when the tools of probability and game theory are brought to bear on the phenomenon, it is possible to handle it with precision and rigor.


\section{Meaning and truth} \label{sec:meaning and truth}
So far, we have managed to derive content from use without any mention of truth. However, there is a minor need for truth in some cases in computing content. Since interpretation involves disambiguation, the truth of a possible proposition, if independently available, can help in this task. If the situation described by an utterance is $c$ and if $c$ is, say, perceptually accessible, then a possible content $\sigma$ can be evaluated against $c$ to see if it makes the proposition $c \vDash \sigma$ true. If it does, then this fact can bolster the choice of $\sigma$ against some other possible content $\sigma'$. Naturally, there will be many other factors that also push toward one or other choice of content so truth cannot play an overriding role; it is just one among many pushes and pulls on alternative contents.

Specifically, the way in which truth enters the computation of content is via the prior probabilities of local partial information games that are members of the global game. As we have seen, the priors are affected by multiple objective and subjective elements, and truth, when it is independently available, is one of them. This kind of role does not result in the kind of vicious circularity between meaning and truth referred to in \citet{dummett:oap} and in \sectref{sec:classic example}.

The determination of truth can occur through the fixing of the described situation but the selection of resource situations and even the utterance situation itself can also play a role. Thus, this role of truth in deriving meaning can be reversed and help in fixing or narrowing these three types of situation as well: described, resource, and utterance. When this happens, meaning and truth become more interdependent. For example, a particular choice of described situation might make a particular choice of content more likely via the resulting proposition's being true, and vice versa, the choice of that content might make that corresponding described situation more likely. Such mutual reinforcement and ``circularity'' are not vicious because the \emph{possible} meanings are not defined in terms of truth as they are based wholly on the Phonetic, Syntactic, and Semantic Constraints which, in turn, depend only on the grammar and on conventional meaning and on the utterance situation. Thus, only the \emph{choice} of optimal meaning may be based partly on truth.

It has never been clear exactly what the place of \citegen{lewis:slg} principle of accommodation \is{accommodation, principle of} is within a theory of meaning. Now that the role of truth in the derivation of meaning from use has been spelled out, it should be possible to extend it to include Lewis's idea but I do not pursue this here.
