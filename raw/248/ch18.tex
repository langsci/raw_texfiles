\chapter{Overview of illocutionary meaning}
\section{Review of the argument for indirect meaning} \label{sec:review}

I have now described in full detail how the notions of relevance and distance that constitute the illocutionary Semantic Constraint can, together with the illocutionary Flow Constraint, yield the indirect meanings of an utterance. I have also shown how these meanings are generally indeterminate and involve multiple infons with different probabilities, thus justifying the picture of meaning as smeared out on the infon lattice, one region for each interlocutor in the exchange.

Combining this model with the one in \partref{part:III} for locutionary meaning gives us a way to compute both direct and indirect meanings. In particular, I have discussed how lexical and structural disambiguation and reference-fixing or saturation on the side of locutionary meaning and free enrichment, implicature, and modulation on the side of illocutionary meaning are handled. This also provides insight into how many tropes work so figurative speech also falls within their scope. The one aspect of meaning I have not explicitly considered is direct and indirect illocutionary force, that is, the stating, commanding and requesting, asking, promising, and other forms of doing things with words, to use Austin's\ia{Austin, J. L.@Austin, J. L.} felicitous phrase, that accompany every utterance. Force does not pose any special obstacles and can be resolved by the same partial information games as worked with other direct and indirect meanings.

A small adjustment that needs to be made in the model for illocutionary meaning is that I have assumed that the baseline from which distances are measured is either the locutionary meaning or one of its proper parts. But when all types of indirect meanings are evaluated together, the baseline may change to, for example, a modulated meaning or an enriched meaning or both. I will show one example of this in the next section. In any case, such changes to the model are straightforward to make.

One unfinished matter mentioned in \chapref{ch:free enrichment} is whether informational proximity implies sufficient relevance. Because the account of distance is difficult to formalize it may be hard to give a proof of this. But one might argue as follows. First, it could be noted that some information $\tau$ is relevant just in case it fulfills the shared conversational goal because that is what leads to an increased \emph{value} for the agent. If $\tau$ is already adequately near the locutionary meaning (or its proper part), then it (or information derivable from it or information it is derived from) must abductively or otherwise fulfill the shared conversational goal within the open ball. Thus, proximity implies fulfillment of goals. And fulfillment of goals implies increased value of information, which is relevance. So relevance is superfluous to the derivation of indirect meaning. I am not sure if this informal argument can stand up to greater scrutiny but if it does it would greatly simplify the account as then only distances within the open ball need to be calculated to find illocutionary meanings. That would also emasculate the problems raised by intensionality as well as other difficulties for the notion of relevance. And it would sever ties completely with the Gricean maxims\ia{Grice, Paul@Grice, Paul} while retaining the Cooperative Principle and also the Gricean reasoning involved in the analysis of particular examples.

In the Gricean scheme it is the maxims that provide a ``standard'' against which literal meaning is measured. If it fails, an implicature is triggered to restore adequacy. In my account it is the goals of the agents as they emerge in the Setting Game that provide the standard against which locutionary meaning is measured. When such goals are not met, illocutionary meanings are triggered to restore adequacy. So the maxims can be seen in a somewhat new light, as approximations to goals or as high-level goals that can be \emph{overridden} by the more specific goals, as I argued in \sectref{sec:theory of conversation}.\footnote{Earlier, in \citet[Section~7.7]{parikh:ul}, I had seen them as approximations to rationality which seems a bit unspecific relative to goals.} When viewed this way, it becomes clear why maxim-based explanations of implicature are often unsatisfactory and vague. It is because maxims can only partially account for the breadth and variety of goals that interlocutors can have and it may often seem that we are force-fitting them into one or another maxim. \citet{horn:tp} offers a very rich set of examples to show how maxims such as being orderly are, in fact, realized in many different ways in different subcommunities and sets of situations, which helps to particularize otherwise ``universal'' maxims, thereby bridging the gap somewhat between abstract and concrete goals.

Another issue is to what degree the requirements of common knowledge can be said to be met in computing illocutionary meaning. This is a delicate matter and in theoretical work it is best to just assume it. But its lack is often responsible for all kinds of miscommunication and partial communication as captured by the overlapping regions of the infon space representing each agent's content. So when our interest lies in such aspects of communication it may become necessary to more carefully specify the nature of the shared knowledge as I have discussed in \emph{The Use of Language} (\citeyear[Chapters~5 and 6]{parikh:ul}) in some detail. There, I also  tackled other flows of information, such as \emph{suggesting}, where full common knowledge does not obtain.

I remind the reader how thoroughly pragmatic and practical my model for both locutionary and illocutionary meaning is as opposed to the more epistemic inclinations of most other researchers as mentioned in \sectref{sec:communication as rational activity}. The goals and preferences of the agents play a crucial role in all the derivations. This is completely missing in Relevance Theory and is relatively absent in Grice as well despite his talk about purposes and despite some of his actual analyses. As most others have followed Grice, they also share the same limitations.


\section{A complete example} \label{sec:a complete example}

I now deal relatively briefly with a more or less complete example bringing especially the Content Selection Game back into focus. This will help connect what we have accomplished in Parts~\ref{part:III} and \ref{part:IV}.

Suppose a seven-year-old child is playing quietly by himself in a room. His activity may partly be described by his ongoing actions and partly by choices he may be considering. His mother is working on her book on communication, \emph{Deconstructing Derrida},\ia{Derrida, Jacques} in the same room. Her activity may likewise be described by her thoughts and possible actions like writing. Now, the child cuts himself on his finger very slightly and a drop of blood appears. He begins to cry loudly. The mother sees it is a very minor cut and, as she is absorbed in her thoughts, she wants him to stop crying while she interrupts her work to attend to the cut. This forms the context or the background situation $u$ for the communication that transpires between them. The Setting Game $SG_u$ incorporates the partly convergent and partly divergent interests of the two agents. From all the examples I have looked at so far it should be apparent that such situations can be extremely varied. What is remarkable is that situation theory and situated game and decision theory allow us to  represent this wide range of possibilities quite compactly.

Informally, what happens next is that she utters the sentence, ``The cut is not serious.'' And the child hears her and stops crying. How is this exchange to be analyzed and understood?

Based on $SG_u$ the mother $\cal A$ wants to stop the child $\cal B$'s crying. She could bring this about in a number of ways. One way is to directly convey to the child that he should stop crying. This includes both the locutionary content to stop crying and the corresponding illocutionary force of a ``command'' or firm request. Another way is to convey a more circuitous content that the cut is not serious and \emph{so} the child should stop crying. There are many other possibilities depending partly even on the creativity of the mother.\footnote{For example, she could try to distract the child.} If telepathy were possible, the mother would convey these direct or indirect contents \emph{im}mediately, that is, without the mediation of language. But before she can do any conveying, either via telepathy or via language, she has to select what to convey, either the direct or indirect content or some other content altogether.

Such contents do not come ready-made into $\cal A$'s mind. They have to be individuated based on her \emph{goal} to get $\cal B$ to stop crying because the cut is not serious. As I said in \sectref{sec:value of information}, the contents possess what I call \emph{act-equivalence} since they have the same purpose. This is a kind of intensionality\is{intensionality} of choice and action as equivalent contents may induce different responses.

On the one hand, the human brain is extraordinarily fast, but on the other, we are not usually aware of such content selection when we speak, let alone all the other decisions that are involved. What actually happens in the body is not known, but we can try to identify the broad structure of constraints -- the \emph{philosophical} psychology -- that must be respected in a \emph{rough} way. For example, I am presenting these constraints as a problem of \emph{choice}. It is not necessary that the choices be made in the detailed way I am about to describe, but the constraints predict that choice of certain kinds must be present.\footnote{See \citet{glimcher:dub} for some preliminary evidence that supports this view.} 

The Content Selection Game that results is shown in Figure~\ref{fig:content game}. 
\vfill
\begin{figure}[h] 
\input{figures/pixmotherchildcontent.tex} 
\caption{The mother and child Content Selection Game}
\label{fig:content game}
\end{figure}
\pagebreak

The utterance situation $u$ contains a great deal of information, some of which is shared and some of which is not. For example, the fact that the child is crying loudly is shared, in fact, it is common knowledge between them in the nonwellfounded way described earlier. But the child isn't sure whether the mother has the mental space to fuss over him or merely attend to him and return forthwith to her work. This uncertainty on the child's part may be represented by a set of two possible situations, $s$ itself where she will not fuss and $s'$ where she will fuss. Both these situations are a part of $u$. As far as he is concerned, he could be either in $s$ or in $s'$, the two situations being identical except for this single uncertainty about fussing. The mother knows that she cannot spare the time to fuss and so knows that she is in $s$. But she is also aware that the child does not know if she will fuss or not, that is, if the situation that obtains is $s'$ or $s$. So she is forced to consider $s'$ even though she knows it is not a factual situation. Because enough of the foregoing is shared between them, these two possibilities become common knowledge between them. Since there is no further information about the situations between them, it is common knowledge between them that the probabilities $\rho$ and $\rho'$ of these initial situations $s$ and $s'$ being factual are the same.\footnote{As I also pointed out in \sectref{sec:back to communication}, there is a certain subtlety involved here because the mother actually knows that $s$ is factual and so that $\rho = 1$. One could say either that the game is played prior to her knowing this or, what I prefer, that they have common knowledge of the child's belief of equiprobability. That is, the mother adopts the child's ignorance for the sake of the situation.}

I will restrict myself to just the two contents explicitly considered above, the indirect content labeled $\tau$ and the direct content labeled $\tau'$. These are the two contents the mother could convey in both situations and they are represented by four branches emanating from them. At this stage, there are different possibilities. We can assume the child is aware of these possible contents, though the argument for that in this setting is somewhat thin. Or we can assume that only the mother has these possibilities in mind and chooses between them and so only she represents the full strategic situation between them. Different settings will dictate different local assumptions: both can be accommodated within the choice situation I am developing. 

In response to conveying $\tau$ or $\tau'$, the child can either stop crying -- the action represented by $a$ -- or continue crying -- the action represented by $a'$. And once both mother and child have acted, certain consequences ensue for each of them that are most conveniently represented by utilities. The particular numbers have been chosen to reflect their relative preferences for these consequences in the setting. These preferences are dictated by the complexities of the background situation which involve the following salient features:

\begin{enumerate}

\item The shared goal of getting the child to stop crying by attending to him
\item The partly convergent, partly divergent personal subgoals of the participants because the mother does not want to fuss and the child wants the mother to fuss
\item The effort of conveying and interpreting a content
\item The psychological fact that if the child has a reason to stop crying, he will be more inclined to do so. Such a reason presumably functions like an Aristotelian enthymeme.\ia{Aristotle}
\item The desire of the mother to reassure her child and to reinforce her parental bond, that is, their relationship

\end{enumerate}

The first three of these have been amply examined so far but the last two have been looked at less thoroughly. The latter two facts involve a moral dimension as mentioned in \citet{taylor:halpp} and in several places in Parts~\textrm{I}, \textrm{II}, and \textrm{III}. Such moral facts, whether of a major kind or a minor kind, are often present in interactions involving communication and this normative dimension of language forms its second fundamental aspect, the first being its acquisition of meaning in the first place. But the two are intertwined as is evident from this example and from \chapref{ch:vagueness}.

For example, if the mother chooses to convey $\tau$ in $s$ and if the child chooses $a$, then the mother gets a payoff of $7$ and the child gets $5$. These payoffs include both costs and benefits. If the mother chooses the direct content $\tau'$ in $s$, and if the child once again chooses to stop crying, then they get $7$ and $-3$, respectively. Here, the mother's net payoff stays roughly the same owing to two opposing pressures: the lower effort of conveying $\tau'$ reduces the cost but the ignoring of the third and fourth factors raises the cost, with the result that the payoff remains roughly the same. On the other hand, the child's reward plummets as he is not offered any basis for his action and is not reassured by his mother's self-centered response. Remember that the situation $s$ involves the mother's not wanting to fuss over the child and it is in this context that the child makes his evaluations. The other payoffs can be analyzed similarly keeping in mind that $s'$ involves the alternative situation where the mother is inclined to fuss. The key thing to note is that the payoffs reflect both cooperation and conflict and also costs and benefits.

By now, having seen the examples in \sectref{sec:expanded content selection game} and \chapref{ch:vagueness}, it should be clear how quite complex psychological, social, anthropological, and moral features of the utterance situation become part of the Communication Game.\is{psychology}\is{sociology}\is{anthropology} The idea is that the \emph{structural} essences of these attributes are \emph{abstracted} from the concrete situation and inserted into the abstract game-theoretic model, that is all. This is what would allow the so-called \emph{thick} aspects of the utterance situation and even the larger environment, the focus of so much inquiry in the social sciences, to be accommodated by the framework of Equilibrium Linguistics.

It turns out that the solution to the choice situation in Figure~\ref{fig:content game} is that the mother should choose to convey $\tau$ in $s$, which is the situation that matters since it is factual. But now the mother, having solved the content selection problem, has to figure out how to convey the indirect content that the cut is not serious and so the child should stop crying. For this, she turns to language as one modality among others. So far, the child knows nothing of what is brewing except perhaps that she has noticed his crying.

As we know, this figuring out of how to convey a content via language is also a choice situation called the Generation Game. In a sense, I got ahead of myself because I expressed $\tau$ in English. Needless to say, $\tau$ has to have some more abstract representation in the brain. And the mother's task is to convert this abstract representation into an English sentence. Part of this task is what to make explicit and what to leave implicit, that is, what words to utter and what to convey as an illocutionary content (e.g.\ implicature). For example, the mother could say any of the following:

\begin{itemize}

\item The cut is not serious so please stop crying

\item The cut is not serious

\item The cut is small

\item It's a small cut

\hspace{0.4in} \vdots 

\hspace{0.4in} \vdots

\end{itemize}

We are not usually conscious of how such choices get made, but the brain nevertheless does have to make them. The second, third, and fourth options, for example, say something about the cut but leave the child to figure out the implicature that he should stop crying \emph{as a result}. As mentioned in \sectref{sec:solving generation games}, there are objective (e.g.\ length, complexity) and subjective (e.g.\ style, maintaining or altering relationships, avoiding errors) aspects of cost that determine the optimal sentence and what is made explicit and left implicit. In addition, it is not just the sentence that has to be selected but also how it should be produced, with what intonation and so on. Once the best sentence is uttered the Interpretation Game emerges.

Overall, the explanation for the mother's utterance of ``The cut is not serious'' and the child's stopping crying runs as follows. First, there is the setting $u$ within which everything takes place and an associated Setting Game. Next, the mother plays the induced Content Selection Game and selects the indirect content $\tau$ over $\tau'$ as that is the equilibrium of that game. This is so because their preferences were based on the fact the child would be more likely to respond as desired if given a reason. Without it, he is unlikely to be persuaded not to mention the reassurance he receives. After this selection, the mother chooses a sentence $\varphi$ (``The cut is not serious,'' possibly) by playing the Generation Game. Then this sentence $\varphi$ is uttered and the child plays the Interpretation Game. He infers the full meaning of $\varphi$ (i.e.\ the literal content + implicature) which is $\tau$ and then plays the Content Selection Game where he chooses the appropriate response to $\tau$, namely, $a$, and so stops crying. In inferring $\tau$ the methods and results of both Parts~\textrm{III} and \textrm{IV} have to be used. 

Thus, the mother (i.e.\ the speaker) starts with the Content Selection Game and then plays the Generation Game. The child (i.e.\ the addressee) starts with the Interpretation Game and ends with the Content Selection Game which could also be called the Response Game from his point of view. This double set of choices that both agents have to make -- selecting a content and then the utterance for the speaker and choosing an interpretation and then an action for the addressee -- are quite reasonable once one realizes that one choice in each set of choices would have been required even if telepathy were possible. Both players then return to the Setting Game.

As we have been seeing throughout, this is more or less the full structure of communication and it constitutes what I have called the Communication Game. 

%One thing I have implicitly relied on in the foregoing description is that all these games are solved rationally. But, as I have mentioned several times, people are not always rational and so it becomes necessary to specify how, in fact, these games should be solved. In my view, the global game involving both the locutionary and illocutionary global games has the kind of structure that does not lead people to deviate from utility theory. Thus, the only place where limited rationality may enter is the Content Selection Game. And this is where alternative theories of bounded rationality may play a role.
%
%To repeat, there is also a subtler way in which I have already built limited rationality into the structure of communication. By separating it into different and independent levels rather than trying to build a single large encompassing game, I have implicitly assumed that people are finite agents and cannot process everything at once. Such a model also makes things clearer conceptually by isolating the different tasks involved. At the level of psychology and neuroscience, however, it is quite possible that there is one seamless structure through which everything happens. Even so, the hypothesis is that some of the choices identified will need to be made.


A more complex example that builds on this analysis can also be considered. Assume $u$ is the same as before but now there is a slight edge in the mother's response to the child's crying, maybe because she is concentrating on a particularly challenging point in her writing. She might then utter a slightly sharper sentence like ``You are not going to die.'' This would result from a somewhat different but related Content Selection Game where the content $\tau''$ is that the child will not die \emph{from the cut} and so he should stop crying because the cut is not serious. This time there is not only a locutionary content and an implicature but also a completion \emph{from the cut} and the saturation of the pronoun \Expression{You}. Indeed, as mentioned in  \chapref{ch:implicature}, the implicature is discovered starting from the baseline of the enriched meaning, not the locutionary meaning. The payoffs in the Content Selection Game would also be a little different as they would account for the chance the mother gets to express her annoyance at the interruption. The reader should now be able to supply the details for such an utterance.

Small variations of this kind in the exact choice of content and sentence will always be possible in a situation and this makes the process of content selection and generation a fundamentally creative process to some degree even when the goals from the Setting Game and $u$ are fully specified. This is what makes it difficult to fully spell out the structure of the Communication Game. But I have shown that it is possible to go quite far in this process nevertheless.

If I had to find one line to summarize the ground traversed, it would have to be that the influence of logicism\is{logicism} and truth-conditional approaches to meaning more generally have kept the field from appreciating a fuller idea of agency and therefore of communication and language and perception and metaphysics. In a sense, the field has more to do with rhetoric than logic, and game theory and situation theory provide the right means to approach it. Utility is perhaps more important than truth in semantics and communication.\is{truth}
