\chapter{Implicature} \label{ch:implicature}

This is the classical type of illocutionary or indirect meaning. It involves a range of examples that are often differently classified. As I have already analyzed a couple of examples of implicature informally in the previous two chapters, I will not develop a more full-blown account here. The considerations are very similar to those in \chapref{ch:free enrichment} in the discussion of free enrichment.

Once the need for one or more implicatures is felt because the locutionary content does not fully realize the goals of the conversation that are more or less public in the Setting Game, the addressee undertakes a local search using the Distance and Relevance sub-Constraints and then submits the candidates thus found to the Flow Constraint. This is in fact the procedure more or less for all indirect meanings and especially for free enrichment and implicature.

Just to amplify my discussion, I will briefly analyze two more examples. The first is a slightly modified example from \citet{grice:lc}.\ia{Grice, Paul@Grice, Paul} One agent asks another where Pierre lives, saying he wants to send him something, and the reply he gets is:

\begin{exe}
\ex Somewhere in the south of France. ($\eta$)
\end{exe}

\noindent First, $\eta$ needs to be enriched to \emph{Pierre lives somewhere in the south of France} because without this the first agent's goal would not be met. The enrichment is not only derivable at a close distance and high relevance, it also sails through the Flow Constraint. But even with this enrichment, $\cal B$'s goal is not fulfilled. There is no implicature available given the information in $u$ that allows an actual address to be inferred. So the only possible conclusions that can be drawn are that $\cal A$ does not know or is unwilling to cooperate (i.e.\ the Cooperative Principle does not hold). This is within derivational reach, that is, they are within the open ball as they can be inferred from an ambient fact like \emph{if $\cal A$ had known and if he were cooperating, he would have given the details} and they are also relevant in the sense that they provide a negative answer to the query. And the two possibilities can be shown to go through the Flow Constraint with appropriate probabilities.

This is not all. $\cal A$ has not simply said he will not reveal Pierre's address, he has provided partial information regarding his whereabouts. This indicates he is not being completely uncooperative which might be taken as rude given some prior relationship between the two agents. As I said in \chapref{ch:distance}, cooperation is often partial. This further fact -- that the Cooperative Principle is nevertheless partially observed -- can also be inferred as an implicature in the same way. Such contents based on maintaining relationships and being polite play a role in the corresponding Content Selection Game as well.

The next example is from \citet[22]{huglysayward:implicature} who argue convincingly that Grice's own way of handling it was circular. In a conversation about whether Eisenhower\ia{Eisenhower, Dwight} was a great US president, one participant offers evidence of Eisenhower's generalship during the war and his great popularity, upon which $\cal A$ says:

\begin{exe}
\ex And he had a wonderful grin too. ($\eta'$)
\end{exe}

The locutionary content of this contribution does not achieve the conversational goal. So it triggers the search for an implicature. It can then be inferred in a few short steps that it would be common knowledge that it \emph{obviously} does not meet the goal because having a wonderful grin has nothing to do with being a president, let alone a great one. Because of this obviousness, it draws a parallel with the other statements about Eisenhower's generalship and popularity as also being irrelevant, although less obviously so. This parallel is drawn because otherwise there would have been no good reason to make such an obviously irrelevant statement. In other words, its real meaning is to draw attention to the earlier statements as being similar with regard to its salient irrelevance. Thus, a locutionary content that had no value in the locutionary information it provided with respect to the conversational goal leads to a conclusion within the derivational ball that is also relevant to the goal. Further, it also goes through the Flow Constraint as the ambient facts are all common knowledge and so is the corresponding illocutionary game.

One can see from such examples that the kind of reasoning the agents have to undertake to derive implicatures can be quite sophisticated. It employs the full breadth of inferential modes and strategies available to us and so cannot be easily formalized. It includes even analogical reasoning as happened in the example above where the blatant irrelevance of a statement implies a relation of similarity to the less clear-cut irrelevance of the preceding contributions. Grice\ia{Grice, Paul@Grice, Paul} was right to emphasize such complex inferences and the Relevance theorists'\is{Relevance Theory} approach becomes manifestly inadequate because they have no place for goal- and purpose-driven derivations, only for the completely blind process of deriving a large \emph{number} of implications. 

How do these analyses square with the maxim of Communication I introduced in \sectref{sec:theory of conversation}? Pretty well, except that the more precise procedure I have offered in this chapter dispenses with the need for such an informally stated rule as implicatures are obtained from basic principles of rational agents trying to fulfill their goals, that is all.

Before I turn to modulation, I should mention that it is not only the addressee who infers implicatures and enrichments. Modulo various indeterminacies, the speaker, too, has to go through the calculations in his Generation Game by imagining how the addressee would carry out the required calculations. Only then can he choose his sentence optimally so that it achieves its ends in an optimal way by making certain things explicit and leaving others implicit.

Lastly, scalar implicatures of the kind discussed by \citet{horn:sploe, horn:qr}, \citet{levinson:pragmatics, levinson:pm}, and many others can also be handled similarly. The key thing to note here is that not all the implicatures that a scale may license will actually be generated because the relevant goals may not be present. It will all depend on the situation. In this sense, there are possibly no generalized implicatures of the kind Grice\ia{Grice, Paul@Grice, Paul} had originally envisaged. Incidentally, \citet{ross:gip} was probably the first to employ game-theoretic methods to analyze \textit{complex} scalar implicatures. He showed that games of partial information (and Pareto-Nash equilibria) scale effortlessly toward this end.

