\chapter{Translation} \label{ch:translation}

My purpose here is to highlight just one contribution my framework can make to the task of translation. Translation requires not only a grasp and command of the source and target languages but also a deep understanding of the text to be translated. \citet[Chapter~25]{jm:slp2} describe a computational approach to the problem.

Different languages individuate the world in different ways. And different languages affect addressees of discourses in different ways. These simple facts have a profound consequence: perfect translations do not exist. This compels us to seek approximate translations and to view translation as a process of approximation. Moreover, languages differ in several other ways as well, in the information they make explicit (e.g.\ definiteness and number information), in their syntax, and so on. In this section, I describe in broad outline how Equilibrium Linguistics can bring precision to the idea of translation as approximation.

The basic idea is very simple. Let the source language be ${\cal L}$ and the target language be ${\cal L}'$. Consider an utterance of $\varphi \in {\cal L}$ in $u$. Then $\varphi' \in {\cal L}'$ relative to $u'$ will be a translation of $\varphi$ in $u$ if their meanings are sufficiently close to each other. Suppose $\sigma$ is the (locutionary and illocutionary) meaning of $\varphi$ in $u$ and $\sigma'$ is the (locutionary and illocutionary) meaning of $\varphi'$ in $u'$. That is, ${\cal C}_u(\varphi) = \sigma$ and ${\cal C}_{u'}(\varphi') = \sigma'$. (Alternatively, it is possible to let $\sigma$ and $\sigma'$ be the intended meanings of the respective utterances that include weaker flows that are not common knowledge.) Then $\sigma$ and $\sigma'$ must be sufficiently near each other. How should nearness be measured? There are two possible ways, each with somewhat different properties. One is to use the idea of distance $d_u$ from \chapref{ch:distance} and the other is to use the idea of distance $\delta_u$ from \sectref{sec:information}. The former is more practical as it takes account of the interlocutors' goals in $u$\footnote{This needs some interpretation as $\sigma$ presumably already meets the goals of the agents as it is both the locutionary and illocutionary content of $\varphi$ in $u$, so what does it mean to traverse the infon space from $\sigma$ to $\sigma'$? In this context, it would mean that $\sigma'$ also fulfills the goals. That is, goal fulfillment is preserved in going from $\sigma$ to $\sigma'$.} and the latter is more logical. I personally think $d_u$ is likely to be more useful than $\delta_u$. Presumably the goals in $u'$ are aligned with the goals in $u$ so it does not matter which situation is used.

Assume we have a translation threshold $\epsilon_{t,u}$. Then we can give the following definition.

\begin{definition}

$\varphi'$ in $u'$ is a translation of $\varphi$ in $u$ if and only if $d_u(\sigma, \sigma') < \epsilon_{t,u}$ (alternatively, $\delta_u(\sigma, \sigma') < \epsilon_{t,u}$).

\label{def:translation}
\end{definition}

There will generally be multiple translations available and the task would then be to identify the best one according to some criterion. One possibility is to minimize the distance but this may not always yield the ideal outcome.

This still leaves open how an actual translation would be derived as the definition merely characterizes the set of translations. I have shown in detail how to go from $\varphi$ in $u$ to $\sigma$. Then reasoning of one sort or another may be used to go from $\sigma$ to $\sigma'$. But it still remains to go from $\sigma'$ to $\varphi'$ in $u'$. This is the inverse direction of generation rather than interpretation and as the generation problem remains partially solved, so does the problem of translation. Indeed, the task is to find a nearby $\sigma'$ such that it is the content of some $\varphi' \in {\cal L}'$ in $u'$. The last two steps are thus interdependent and involve a back-and-forth interaction between them.

This definition of translation is a little narrow as it omits any consideration of the intended effects of the source and target texts. As just discussed, such effects involve classification or detailed description and the former approach can be used to broaden the definition to include effects that are also \emph{near} each other.\\\\
I believe \partref{part:IV} gives a more or less complete solution to the problem of illocutionary communication that is, as advertised in \chapref{ch:why communication is central}, philosophically sound, mathematically solid, reasonably computationally tractable, and empirically adequate. The Setting Game, Content Selection Game, Generation Game, and Interpretation Game that constitute an (illocutionary) Communication Game are developed in detail. So are the two illocutionary Constraints: Semantic and Flow. The universality of locutionary partial information games demonstrated in \partref{part:III} can be readily extended to illocutionary partial information games. I consider meanings that lie beyond direct and indirect meaning and classify the entire realm of meaning, both intended and unintended. I complete this conceptual geography of meaning by approaching the effects of communication as either a classificatory or descriptive undertaking. I take a brief look at translation. Again, as promised in \chapref{ch:why communication is central}, I connect arguments in philosophy, linguistics, psychology, and computer science, unifying them through the framework of Equilibrium Linguistics and relating them to the hermeneutic sciences. 

In the next Part, I look at Language Games and at how conventional meanings originate and change.



