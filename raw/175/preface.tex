\addchap{Vorwort}
\begin{refsection}

Diese Arbeit ist eine überarbeitete und vor allem gekürzte Version meiner Habilitationsschrift, die im November 2016 von der Fakultät für Linguistik und Literaturwissenschaft der Universität Bielefeld als schriftliche Habilitationsleistung angenommen wurde.\\
\noindent
\\ Von den ersten Überlegungen bis zu der Form, in der sie nun vorliegt, sind nicht nur 6,5 Jahre vergangen, sondern es haben mich auch diverse Personen auf ganz unterschiedliche Weise unterstützt.\\
\noindent
\\ Danken möchte ich deshalb:\\
\noindent
\\ Meinem Bielefelder Chef Ralf Vogel für die Freiräume, die er mir gelassen hat, und dafür, dass ihm kaum eine Idee zu unkonventionell war, er aber auch bestimmte Annahmen/Vorgehensweisen sehr kritisch infragegestellt hat. Ebenfalls sehr förderlich für meine Arbeit waren die Konferenzbesuche, die er und die Fakultät für Linguistik und Literaturwissenschaft mir ermöglicht haben.\\
\noindent
\\ Meinen damaligen Mitstreitern am Lehrstuhl Anna Kutscher, Julia Jonischkait und Panagiotis Kavassakalis für die Kolloquiumssitzungen und den gemeinsamen Unialltag.\\
\noindent
\\ Sandra Pappert, die mir in der Uni als Statistikexpertin, aber auch außerhalb des Unibetriebs eine große Hilfe war.\\
\noindent
\\ Horst Lohnstein für die Unterstützung über Uni- und Stadtgrenzen hinweg.\\
\noindent
\\ Sebastian Bank und vor allem Antonios Tsiknakis, die spät in mein Projekt eingestiegen sind, aber in der letzten Phase als meine \LaTeX-Experten unverzichtbar waren, sowie Jan Schattenfroh für die Erstellung der \BibTeX-Einträge.
\pagebreak
\noindent
\\Philippa Cook, Anke Holler und Cathrine Fabricius-Hansen für die Aufnahme der Arbeit in die Reihe \textit{Topics at the Grammar-Discourse Interface}, Annika Hübl für die Endkorrektur sowie Sebastian Nordhoff und Felix Kopecky für die Endformatierung.\\
\noindent
\\ Meinen Bielefelder Nachbarn, meinen Kölner Freunden und Bekannten und last but not least meiner Familie und Markus für alles, was nichts mit Uni zu tun hat.\\\\
\noindent
Sonja Müller
\hfill\hbox{Köln, im Februar 2018}





\printbibliography[heading=subbibliography]
\end{refsection}

