\documentclass[output=collectionpaper]{langsci/langscibook}
\ChapterDOI{10.5281/zenodo.3462756}

\title{Canonical, complex, complicated?}

\author{Jenny Audring\affiliation{Leiden University}}

\abstract{Investigating the complexity of grammatical gender begins with the question: What are the dimensions of variation? This question is addressed by Canonical Typology, which provides us with a cross-linguistic road map of gender systems (\citealt{Corbett2016}). Compass and measuring rod are the principles of canonicity, which organise the theoretical space around a canonical centre and then situate real gender systems in this space. In this chapter I compare and contrast the principles of canonicity with those of complexity, and discuss both of them in relation to difficulty. While canonicity, complexity, and difficulty are related notions, it will be argued that they are not identical: individual phenomena can be complex but canonical, or complex but not difficult. The aim of the chapter is to tease apart issues of methodology, description, and theory in order to arrive at a clearer understanding of the complexity of gender.
\medskip

\keywords{gender, complexity, canonicity, difficulty, learnability, economy, transparency, independence, redundancy}
}%

\begin{document}
\maketitle

\section{The typology of gender}

\subsection{Introduction}

Typologies are descriptive spaces shaped by the dimensions of cross-linguistic variation. Once laid out, such spaces can be profiled according to various theoretical aims. In the domain of grammatical gender, the best example of this method is the Canonical Typology approach (e.g.\ \citealt{Corbett2006}; \citealt{Brown2013}; \citealt{Bond2019}; \citealt{Corbett2016} for gender). By organising the typological variation in gender systems according to the principles of canonicity, we arrive at a better understanding of the feature, from its most canonical manifestations at the centre to the non-canonical systems at the periphery.\footnote{For a collection of interesting outlier systems, see \citet{Fedden2018}.}

The aim of this paper is to further explore the typological space of grammatical gender by comparing and contrasting canonicity with two other evaluative measures: complexity and difficulty.%
\footnote{The terms ``canonicity'', ``complexity'' and ``difficulty'' are used as technical terms throughout the paper. \sectref{sec:Audr:2} briefly outlines the relevant theory.} %
The three notions appear to intersect: one might expect canonical gender systems to be the least complex, and the least complex systems to be the least difficult to acquire or use. However, there are theoretical reasons to assume that canonicity can imply greater complexity, and empirical reasons to believe that lower complexity does not necessarily mean lower difficulty.

The chapter is organised as follows. I first lay out the theoretical perspective taken in this chapter. This section also serves as an overview of the terminology used. Then I introduce the notion of ``profiling'', which means organising a typological space according to certain principles. \sectref{sec:Audr:2} discusses the principles involved in profiling the typology of gender according to canonicity on the one hand and complexity on the other. In \sectref{sec:Audr:3}, I apply the principles to the typological space and compare the results. \sectref{sec:Audr:4} widens the discussion to cross-linguistic evidence on difficulty in first language acquisition. \sectref{sec:Audr:Concl} concludes the paper.

With regard to the three notions compared \textendash{} canonicity, com\-plex\-i\-ty, and dif\-fi\-cul\-ty \textendash{} the text has an asymmetric structure: canonicity is taken as the baseline for an assessment of complexity, but difficulty is introduced independently and then linked to the other two notions.

\subsection{Theoretical perspective and terminology}
\label{sec:Audr:1.2}
The theoretical perspective taken in this chapter is in line with \citet{Corbett1991,Corbett2013,Corbett2013a,Corbett2013b}. Grammatical gender systems are understood as systems of agreement classes. This means that we follow Hockett's famous dictum that ``[g]enders are classes of nouns reflected in the behaviour of associated words'' \citep[231]{Hockett1958} and take agreement as a definitional property of gender. Nouns serve as agreement \textit{controllers} that determine the form and feature structure of agreeing \textit{target} words. An example is (\ref{ex:Audr:pasta1}) from \ili{Italian}, where the definite article and the predicative adjective agree in gender with the feminine noun \textit{pasta}.

\protectedex{%
\ea
\label{ex:Audr:pasta1}
\ili{Italian} (Anna Thornton, p.c.)\\
\gll la pasta è squisit-a          \\
     \textsc{def.sg.f} pasta\textsc{(f).sg} be.\textsc{prs.3sg} delicious-\textsc{sg.f}\\
\glt `The pasta is delicious.'
\z
}%

The syntactic configurations in which we find the agreement controller and its targets are called \textit{domains}. The most local domain for gender agreement is the noun phrase (although, of course, finer subdivisions can be made with regard to hierarchical or linear distance within the noun phrase). Many languages, including \ili{Italian}, show gender agreement in more than one domain. Larger domains are the clause (with predicative agreement targets such as verbs) and the sentence (with relative pronouns as clause-external but sentence-internal agreement targets), but anaphoric agreement can reach beyond the sentence and even span more than one turn in conversation.

The number of different agreement patterns corresponds to the number of gender \textit{values} distinguished in a language (this is less straightforward when languages have inconsistent or mismatching agreement patterns). Gender values often have names, e.g. \textit{feminine} or \textit{uter}, especially in smaller systems with fewer values and when values line up with particular semantic properties. The values in larger systems are commonly labeled by numbers. Some linguistic traditions, e.g.\ the Bantuist literature, speak of noun classes rather than genders and distinguish numbered singular and plural classes (see example (\ref{ex:Audr:3}) below).

Nouns usually have a consistent gender value as an inherent lexical property. \textit{Assignment rules} that regulate which noun goes with which gender are easy to identify in a number of languages, but less so in others. Such rules can refer to semantic, phonological, or morphological properties of nouns. Consider, for example, the following rules proposed for \ili{German} \citep[Chapter~3]{Koepcke1982}.\footnote{These rules are not categorical but reflect statistical tendencies; counterexamples can be found for every proposed rule.}

\begin{itemize}
\item Semantic assignment rule

\begin{itemize}
\item Nouns denoting lexical categories are neuter (e.g. \textit{das Substantiv} `the noun', \textit{das Verb} `the verb', \textit{das Pronomen} `the pronoun')
\end{itemize}

\item Phonological assignment rule

\begin{itemize}
\item Monosyllabic nouns ending in /ʃ/ are masculine (e.g. \textit{der Mensch} `the human', \textit{der Busch} `the bush/shrub', \textit{der Marsch} `the march')
\end{itemize}

\item Morphological assignment rules

\begin{itemize}
\item Nouns that take the plural suffix \textit{{}-(e)n} are feminine (e.g. \textit{die Tür} `the door', \textit{die Stirn} `the forehead', \textit{die Flut} `the flood')
\end{itemize}
\end{itemize}

Phonological and morphological rules are often subsumed under ``formal rules'' \citep{Corbett2013b}. In addition, as defended in \citet{Audring2017}, it may be useful to distinguish between general rules that account for a large part of the noun vocabulary, and `parochial' rules with a narrower scope.\footnote{For an insightful discussion of parochial or ``crazy'' rules and associated theoretical issues see \citet{Enger2009}.} This distinction cross-cuts the semantic/formal split. The \ili{German} examples above represent parochial rules; they constitute a small part of a large and complex rule system.

Taken together, the number and nature of the assignment rules, the properties of the controllers, the range of values, and the behaviour of the targets in each domain can be used to broadly characterise the gender system of a language and compare it to others.


\subsection{Profiling}

In typologies of grammatical (sub)systems, all instances of cross-linguistic variation can be treated equally by simply cataloguing the available options. \tabref{tab:Audr:Props}, for example, lists a selection of options for gender systems.

\begin{table}[htb]
\begin{tabular}{ll}
\lsptoprule
Controllers: & Noun, pronoun, ...\\
Targets: & Adjectives, verbs, pronouns, articles, ...\\
Domains: & Noun phrase, clause, ...\\
Values: & 2 gender values/ 10 gender values, ...\\
Assignment rules: & Semantic, phonological, ...\\
\lspbottomrule
\end{tabular}
\caption{Possible properties of gender systems (selection)}
\label{tab:Audr:Props}
\end{table}

However, it might be useful to profile the typology. For example, typologists might sort the various options according to commonness or rarity. Alternatively, we might want a typology of gender to say that a gender system with nothing but pronominal targets is a non-canonical gender system \textendash{} hence the persistent disagreement in the linguistic literature on whether or not \ili{English} has grammatical gender.%
\footnote{See \citetvo{chapters/12} for a different view on pronominal gender.} %
Such differences can be captured by defining a ``canonical'' or ideal gender system and then situating real systems according to their relative distance from this baseline. This is the method of Canonical Typology (\citealt{Corbett2006,Corbett2012}; \citealt{Brown2013}; \citealt{Corbett2016}); we will discuss it in more detail in \sectref{sec:Audr:2} and \sectref{sec:Audr:3}.

Profiling \textendash{} be it in terms of commonality, canonicity, or any other evaluative measure \textendash{} organises the typological space according to certain principles and thereby enriches the description, allowing for a deeper understanding of the grammatical (sub)system in question. In the present paper, I will compare two profiles for grammatical gender, the canonicity profile and the complexity profile, and relate both to the issue of difficulty. First, however, we need to establish principles that allow us to ask which properties count as canonical or complex, and why.

\section{Principles}
\label{sec:Audr:2}

\subsection{Introduction: Principles}

The method I have referred to as ``profiling'' creates organised typological spaces. Organisation requires principles. In this section, I will review the principles of canonicity as proposed in the literature, and then suggest a number of possible principles for complexity and difficulty (again, guided by the relevant literature).

Since the issues are themselves highly complex, the representation will be uncomfortably sketchy in places. Especially for canonicity, the reader is referred to the original sources for a more extensive motivation of the approach, for discussion, and for further examples.

\largerpage[-1]
\subsection{Principles of canonicity}
\label{sec:Audr:2.2}

The main purpose of the canonical approach to typology is to define a linguistic equivalent of the zero on the Kelvin thermometer: an absolute calibration point in the space of possibilities \citep{Fedden2015}. Unlike the scale of a thermometer, however, a canonical typology is multi-dimensional. \citet{Corbett2016} define the calibration point for grammatical gender and the variational space around it with the help of a number of principles. Since gender is a morphosyntactic feature involving agreement, most of the principles for canonical gender systems follow from those for canonical morphosyntactic features \citep{Corbett2012} and canonical agreement \citep{Corbett2006}, respectively. \citet{Corbett2016} present the clusters of principles separately; in the following they will be represented jointly. In order to allow for easier cross-reference to the source, the original numbering is retained. This necessitates a minor adjustment: Principle I for canonical morphosyntactic features appears as Principle Ia, Principle I for canonical agreement as Principle Ib. Moreover, I have added names to the principles for easier reference throughout the text.\footnote{All principles in this chapter are capitalised.}

According to Corbett and colleagues, the relevant principles for canonicity are the following (after \citealt{Corbett2016}):

\begin{itemize}
\item[] \textit{Principle Ia: Clarity}
\begin{itemize}
\item[] The feature gender and its values are clearly distinguished by formal means.
\end{itemize}

\item[]\textit{Principle Ib: Redundancy}
\begin{itemize}
\item[] Canonical gender agreement is redundant rather than informative.
\end{itemize}

\item[]\textit{Principle II: Simple Syntax}

\begin{itemize}
\item[] In a canonical gender system, the use of the feature and its values is determined by simple syntactic rules. Canonical gender agreement is syntactically simple.
\end{itemize}

\item[]\textit{Principle III: Exponence}


\begin{itemize}
\item[] In a canonical gender system, the feature and its values are expressed by canonical inflectional morphology.
\end{itemize}

\item[]\textit{Principle IV: Orthogonality}

\begin{itemize}
\item[] Canonical gender and canonical parts of speech are fully orthogonal.
\end{itemize}

\item[]\textit{Principle V: Matching Values}

\begin{itemize}
\item[] In a canonical system of grammatical gender the contextual values match the inherent values.
\end{itemize}

\item[]\textit{Canonical Gender Principle (CGP)}

\begin{itemize}
\item[] In a canonical gender system, each noun has a single gender value.
\end{itemize}
\end{itemize}

The principles are operationalised by means of criteria that specify for individual properties or behaviour whether they are more or less canonical. Greatly simplifying the complex and sophisticated account in \citet{Corbett2016}, the principles and criteria for canonical gender say that gender

\begin{itemize}
\item
should be expressed by means of affixes

\item
should involve dedicated and unique markers that express gender and nothing else

\item
should be marked consistently, regularly, and obligatorily

\item
is not impinged upon by syntax, lexical restrictions, or other grammatical features.
\end{itemize}

Controller and target should

\begin{itemize}
\item
have gender and express it overtly

\item
have matching values (thus rendering the gender information on the target redundant).
\end{itemize}

Furthermore, there should not be any syntactic complications such as inconsistent controllers or special agreement rules for different parts of speech. In principle, all relevant parts of speech should have access to all gender values. The exception is nouns, which \textendash{} canonically \textendash{} should only have a single, fixed gender value.

Anticipating a more detailed discussion in \sectref{sec:Audr:3}, let us look again at \ili{Italian} to see how the principles play out.%
\footnote{See \citet[3]{Fedden2017} for a similar assessment.} %
Example (\ref{ex:Audr:pasta1}) is repeated as (\ref{ex:Audr:pasta2}a); example (\ref{ex:Audr:pasta2}b) is added for contrast.

\ea
\label{ex:Audr:pasta2}
\ili{Italian}\\
\begin{xlist}
\ex
\gll la pasta è squisit-a          \\
     \textsc{def.sg.f} pasta\textsc{(f).sg} be.\textsc{prs.3sg} delicious-\textsc{sg.f}\\
\glt `The pasta is delicious.'
\ex
\gll il cibo è squisit-o          \\
     \textsc{def.sg.m} food\textsc{(m).sg} be.\textsc{prs.3sg} delicious-\textsc{sg.m}\\
\glt `The food is delicious.'
\end{xlist}
\z

\ili{Italian} marks gender mostly by suffixes, which are consistent, regular, and obligatory. However, some cumulative exponence occurs: the definite articles fuse stem and gender marker, and all gender markers double as number markers. Both controllers and targets distinguish two values (masculine and feminine); these match across domains. The great majority of nouns have a constant gender value, and many nouns show their gender overtly. Gender agreement is redundant in most cases. Hence, the \ili{Italian} gender system comes fairly close to being canonical.

Generalising, we can state that a canonical gender system is defined by formal clarity, syntactic and morphological simplicity, orthogonality to all other compatible linguistic properties, and consistency in the behaviour of all items involved. Viewed in this way, it is easy to see that canonicity involves similar considerations to complexity. Indeed, Principle II (Simple Syntax) makes explicit reference to simplicity. Turning to complexity next, we ask what principles can be brought to bear in order to identify a particular property or behaviour as more or less complex.

\subsection{Principles of complexity}
\label{sec:Audr:2.3}

The literature on linguistic complexity is vast, and many sources propose principles of complexity. The following section draws on \citet{Audring2017}, a detailed study of the complexity of gender systems; the principles are inspired by earlier work, chiefly \citet{Kusters2003}, \citet{Miestamo2008}, and \citet{DiGarbo2014,DiGarbo2016}. Here, as in most sources (with the exception of \citealt{Kusters2003}), discussion will be restricted to absolute or descriptive complexity (\citealt{Miestamo2008}; \citealt{Sinnemaeki2011,Sinnemaeki2014}) in order to keep relative complexity, i.e.\ difficulty, a separate issue (for which see ‎\sectref{sec:Audr:4}).

The most common principle applied in judging complexity is that less equals less complex. This kind of assessment can be used for properties that can be counted or measured. For example, a language with two gender values is less complex than a language with four. Other countable properties are, for example, the number of distinct forms in a paradigm or the number of allomorphs for a given grammatical formative. Following \citet{Kusters2003}, this might be called the Principle of Economy (but see \citealt{Miestamo2008}; \citealtvo{chapters/11} who call it ``Principle of Fewer Distinctions'') and be defined as follows:

\begin{quote}
Principle of Economy: The more distinctions or forms a grammatical feature involves, the more complex the feature.
\end{quote}

The Principle of Economy needs to be supplemented by other principles, since not all phenomena lend themselves to quantification. For example, it might be argued that dedicated, unique markers are less complex than polyfunctional markers. This is not a matter of quantity, but a matter of mapping function to form. Polyfunctionality comes in various guises; the most common are markers that are syncretic across gender values or that simultaneously express another grammatical feature. The examples in ‎(\ref{ex:Audr:3}) from \ili{Chichewa} (\ili{Niger-Congo} (\ili{Bantoid}), \citealt{Bentley2001}) illustrate both situations.

\ea
\label{ex:Audr:3}
Agreement in \ili{Chichewa}\\
\begin{xlist}
\ex
\begin{minipage}[t]{0.4\textwidth}
\gll mwa-muna a-kuyimba\\
     1-man 1-sing.\textsc{prs} \\
\glt `The man is singing.'\\
\end{minipage}%
\begin{minipage}[t]{0.4\textwidth}
\gll a-muna a-kuyimba\\
     2-man 2-sing.\textsc{prs}\\
\glt `The men are singing.'
\end{minipage}
\ex
\begin{minipage}[t]{0.4\textwidth}
\gll chi-patso chi-kugwa  \\
     7-fruit 7-fall.\textsc{prs}\\
\glt `A piece of fruit is falling.'\\
\end{minipage}%
\begin{minipage}[t]{0.4\textwidth}
\gll zi-patso zi-kugwa\\
     8-fruit 8-fall.\textsc{prs}\\
\glt  `Pieces of fruit are falling.'
\end{minipage}
\end{xlist}
\z

The nominal and verbal prefixes in ‎(\ref{ex:Audr:3}) express noun class as well as number: 1 and 7 are singular classes, 2 and 8 are plural classes. ‎(\ref{ex:Audr:3}b) shows the expected situation: the markers for class 7 and 8 are distinct. In ‎(\ref{ex:Audr:3}a) the verbal prefix is syncretic for singular and plural and hence polyfunctional (the same marker also returns as the marker of the plural class 14; \citealt[6]{Mchombo2004}).

In order to capture the intuition that polyfunctional markers are more complex than dedicated markers, we assume a principle that is well-represented in the complexity literature, the Principle of Transparency (again, I follow the terminology of \citealt{Kusters2003}; \citealt{Miestamo2008} and \citealtvo{chapters/11} call it ``Principle of One-Meaning-One-Form''). This principle states that:

\begin{quote}
Principle of Transparency: Minimal complexity is characterised by a 1:1 mapping of meaning and form.
\end{quote}

The examples in ‎(\ref{ex:Audr:3}) violate this principle by showing forms with more than one function (cumulative expression of noun class and number in (\ref{ex:Audr:3}a) and ‎(\ref{ex:Audr:3}b), syncretic markers for class 1 and 2 in ‎(\ref{ex:Audr:3}a)). It should be noted that otherwise the \ili{Chichewa} examples are remarkably transparent: they involve clearly separable prefixes which are even alliterative between controller and target in class 7, 8 and 2.%
\footnote{\citet[15]{Corbett2006} includes alliterative form as a criterion for canonical agreement.}

Certain cases of polyfunctionality produce complex situations for which it seems justified to posit a separate complexity principle. Following \citet{DiGarbo2014,DiGarbo2016}, I call it the Principle of Independence. This principle states that:

\begin{quote}
Principle of Independence: In the least complex situation, a grammatical feature is independent of other grammatical features or other linguistic properties.\footnote{See also \citet[170, 174]{Corbett2012} for related criteria for canonical features.}
\end{quote}

Independence is compromised when gender marking is neutralised for a part of the paradigm. Well-known examples are gender neutralisation in the plural and in the local persons. \tabref{tab:Audr:2} illustrates the latter case. \ili{Ngala} (\citealt{Siewierska2013}, data from \citealt{Laycock1965}) distinguishes gender in all three persons of the singular personal pronouns, while in \ili{Arabic} \citep[298--299]{Ryding2005} only the second and the third person mark gender. \ili{Italian} shows gender in the third person only.

\begin{table}[htb]
\begin{tabularx}{.8\textwidth}{l*{6}{Z{.9cm}}}
\lsptoprule
\bfseries Language & \multicolumn{2}{c}{\bfseries Ngala}  & \multicolumn{2}{c}{\bfseries \ili{Arabic} }  & \multicolumn{2}{c}{\bfseries \ili{Italian} }\\
 & \multicolumn{2}{c}{\bfseries (\ili{Sepik})}  & \multicolumn{2}{c}{\bfseries (\ili{Afro-Asiatic})}  & \multicolumn{2}{c}{\bfseries (IE)}\\
{Gender} & \textsc{m} & \textsc{f} & \textsc{m} & \textsc{f} & \textsc{m} & \textsc{f} \\
\midrule
1st person & \itshape wn & \itshape ñǝn & \multicolumn{2}{c}{ \itshape anaa} & \multicolumn{2}{c}{ \itshape io}\\
2nd person & \itshape mǝn & \itshape  yn &  \itshape anta &  \itshape anti & \multicolumn{2}{c}{ \itshape tu}\\
3rd person &\itshape kǝr &  \itshape yn & \itshape huwa &  \itshape hiya &  \itshape lui &  \itshape lei\\
\lspbottomrule
\end{tabularx}
\caption{Gender marking in personal pronouns (singular)}
\label{tab:Audr:2}
\end{table}


In \ili{Arabic} and \ili{Italian} we see that gender depends on another property, in this case another grammatical feature. According to the Principle of Independence, this represents increased complexity because it necessitates longer descriptions of the system. The idea is the same as limited orthogonality in canonicity (Principle IV (Orthogonality) for canonical morphosyntactic features, \sectref{sec:Audr:2.2} above): not all logically possible pairings of cross-cutting properties occur. Limitations to Independence can involve properties such as part of speech, other features such as person, number, definiteness, or case, lexical restrictions such as lack of productivity of morphological markers, or interventions from the side of the speaker for semantic or pragmatic purposes.

In contrast to canonicity, where the principles and criteria should converge on the same outcome, the three principles of complexity \textendash{} Economy, Transparency and Independence \textendash{} are autonomous and can lead to different evaluations. Consider again the \ili{Arabic} and \ili{Italian} paradigms in \tabref{tab:Audr:2}. From the perspective of Economy the paradigms are simpler than the paradigm of \ili{Ngala}: they contain fewer forms. However, they violate Transparency by requiring a non-1:1 mapping of features and forms, as \textit{anaa}, \textit{io} and \textit{tu} have to map onto both gender values.%
\footnote{Note that we are still dealing with grammatical gender here and not just with the sex of the speaker or the addressee. In \ili{Hebrew}, which has a system similar to \ili{Arabic}, addressing an inanimate entity (say, an egg rolling off the table or a misbehaving computer) would require the use of a second-person pronoun in the appropriate grammatical gender value (feminine for the egg, masculine for the computer) (Lior Laks, personal communication).} %
The \ili{Arabic} and \ili{Italian} data also show higher complexity from the perspective of Independence, since gender is not fully orthogonal with person.

The upshot is that we cannot speak of the complexity of gender as a unitary phenomenon. Rather, we can employ the three principles (and potentially others) to evaluate observable properties or behaviour. A profiled typology or ``complexity space'' of gender does not have a single calibration point of minimal complexity. Violations of any of the principles constitute a more complex situation.

Note that we are only considering languages that have a gender system. Hence, we disregard the fact that having gender in the first place complexifies a language. Nor will we ask about a gender system's usefulness or functionality. Such issues are addressed elsewhere \textendash{} see for example \citetv{chapters/04} and \citetvo{chapters/13}.

\section{Canonicity vs.\ complexity}
\label{sec:Audr:3}

\subsection{Profiling}

Profiling the typological space by means of the principles introduced above, we can draw up a comparison for canonicity and complexity. This will be done separately for five parameters: the controller (\sectref{sec:Audr:3.2}), the targets (\sectref{sec:Audr:3.3}), the values (\sectref{sec:Audr:3.4}), the domains (\sectref{sec:Audr:3.5}), and the assignment rules (\sectref{sec:Audr:3.6}). In each section, we will ask what properties are more canonical and what properties are less canonical, building on \citet{Corbett2006,Corbett2012} and \citet{Corbett2016}.%
\footnote{\cite[514--517]{Corbett2016} discuss the properties of values under the heading of ``Features''.} %
Then we will evaluate the options according to the principles of complexity. For reasons of space, only a selection of properties will be discussed; see \citet{Audring2017} for a fuller account. Please refer back to \sectref{sec:Audr:2.2} and \sectref{sec:Audr:2.3} for the principles.

\subsection{Controller}
\label{sec:Audr:3.2}

As we saw in \sectref{sec:Audr:2.2}, the principles of canonicity lead to certain expectations with regard to properties and behaviour. For canonical controllers in gender systems, these are the following.

\subsubsection{Controller: canonicity}

A canonical controller is present and expresses gender overtly. This is due to Clarity as well as to Redundancy, since an explicit controller renders the agreement redundant. According to Simple Syntax as well as to the Canonical Gender Principle, the controller should be consistent in the agreements it takes and have a single, lexically specified gender value.

Systems that deviate from these expectations are less canonical. The question to explore here is whether they are also more complex. Let us consider the properties one by one.

\subsubsection{Controller: complexity}

\fussy
While an overtly present controller may be expected throughout, absent controllers are cross-linguistically common in pro-drop languages. Consider the \ili{Spanish} example in (\ref{ex:Audr:4}), where the adjective agrees with an implicit third-person controller.
\sloppy 

\ea
\label{ex:Audr:4}
\ili{Spanish}\\
\gll está rot-a\\
     be.\textsc{prs}.\textsc{3sg} break-\textsc{f.sg}\\
\glt `It/she is broken.'
\z

In terms of complexity, an absent controller increases Economy because the syntagmatic structure is simpler. By contrast, it constitutes a case of higher complexity from the point of view of Transparency, since there is no form that goes with the controller function. Moreover, a controller that is absent in some cases but present in others is at odds with Independence, since its distribution is influenced by other factors, e.g.\ pragmatics.

Aside from their presence or absence, controllers differ in whether or not they mark gender overtly. The opposite of overt gender is covert gender; languages with covert gender express the feature only by agreement. An example for a language with overt gender is \ili{Turkana} (\ili{Nilotic}, examples \ref{ex:Audr:5}a); a covert system is found in \ili{Dutch} (examples \ref{ex:Audr:5}b). Other languages may show intermediate degrees of overtness.

\ea
\label{ex:Audr:5}
Overt vs.\ covert gender\\
\begin{xlist}
\ex
Overt gender (\ili{Turkana}, \citealt{Dimmendaal1983}: 224)\\
\begin{minipage}[t]{0.4\textwidth}
\gll ɛ{}-sikin-a  \\
     \textsc{m.sg}{}-breast-\textsc{sg}\\
\glt `breast'\\
\end{minipage}%
\begin{minipage}[t]{0.4\textwidth}
\gll a-ŋasep\\
     \textsc{f.sg-}placenta\\
\glt `placenta'\\
\end{minipage}
\goodbreak
\ex
Covert gender (\ili{Dutch})\\*
\begin{minipage}[t]{0.4\textwidth}
\gll vloek    \\
     curse(\textsc{c).sg}\\
\glt `curse'
\end{minipage}%
\begin{minipage}[t]{0.4\textwidth}
\gll boek\\
     book(\textsc{n).sg}\\
\glt `book'
\end{minipage}
\end{xlist}
\z

The nouns in (\ref{ex:Audr:5}a) show overt gender in the form of class prefixes. The nouns in (\ref{ex:Audr:5}b) do not provide any formal indication of gender. Covert gender is more complex from the point of view of Transparency, since covert gender involves function without form. On the other hand, overt marking involves additional morphological material and an additional locus of marking, so it is more complex from the perspective of Economy. Independence is affected when overt marking is subject to conditions. An example can be found in the \ili{Khoisan} language \ili{Sandawe}, where gender marking on the noun is restricted to a number of nouns referring to female persons, which constitutes a lexical condition motivated by semantics \citep[57]{Steeman2011}.

The next property to be considered is the behaviour of the controller with regard to its targets. According to both Transparency and Independence, nouns should be consistent controllers that trigger the same agreement on any target under any circumstance. This captures the insight that hybrid nouns such as \ili{Dutch} \textit{meisje} `girl', which takes neuter agreement on attributive targets and (mostly) feminine agreement on others, are a complexifying phenomenon in a gender system.

According to the Canonical Gender Principle (henceforth CGP), nouns should have only a single gender value each. Thus, a language like \ili{Savosavo} (Papuan, \citealt{Wegener2012}), which allows for manipulation of the gender value for pragmatic purposes, constitutes a non-canonical situation (example \ref{ex:Audr:6}).

\ea
\label{ex:Audr:6}
\ili{Savosavo} \citep[64]{Wegener2012}\\
\gll Ai lo tuvi=na ko tuvi k-aughi ngai-sa patu.\\
     this \textsc{det.sg.m} house=\textsc{nom} \textsc{det.sg.f} house \textsc{3sg.f.obj}{}-exceed big-\textsc{vblz} \textsc{bg.ipfv}\\
\glt `This house (\textsc{m}) is bigger than that house (\textsc{f}).', lit.\ `This house (\textsc{m}) is big exceeding that house (\textsc{f}).'\footnote{\textsc{vblz}=verbalizing morpheme, \textsc{bg}=background}
\z

In the example, the noun \textit{tuvi} `house' is used first with masculine agreements matching its lexical gender, but later with feminine agreements; this has the effect of emphasising, diminutive-like, the smallness of the house.

Languages like \ili{Savosavo}, which systematically recategorise nouns for evaluative statements about size or merit (\citealt[123]{Corbett2014}; \citealt[179]{DiGarbo2014}), are not only less canonical, but also more complex. They violate Transparency by a 1:2 mapping of nouns and genders as well as compromising Independence, as the recategorisation involves semantic or pragmatic factors.

\tabref{tab:Audr:3} collates the controller properties and their evaluation in terms of canonicity and complexity. A tick indicates alignment between maximal canonicity and minimal complexity. A cross indicates canonicity but increased complexity. A dash means that a principle is not relevant. In \tabref{tab:Audr:3} we see that maximal canonicity lines up fairly well with minimal complexity. An exception is Economy disagreeing with Clarity and Redundancy: more formal evidence makes for a clearer and hence more canonical gender system, but at the cost of parsimony.{}

\begin{table}
\small
\begin{tabularx}{\textwidth}{Xccc}
\lsptoprule
\bfseries The controller\ldots & \bfseries Economy & \bfseries Transparency & \bfseries Independence\\
\midrule
\ldots is present (Clarity, Redundancy) & \xmark & \cmark & \cmark\\
\padding
\ldots has overt expression of gender (Clarity, Redundancy) & \xmark & \cmark & $-$/\cmark\\
\padding
\ldots is consistent in the agreements it takes (Simple Syntax, CGP) & $-$ & \cmark & \cmark\\
\lspbottomrule
\end{tabularx}
\caption{Canonicity and complexity of the controller}
\label{tab:Audr:3}
\end{table}

\subsection{Targets}
\label{sec:Audr:3.3}

The list of target properties figuring in the canonicity profiling is extensive. In the following I will restrict the discussion to a number of central properties.

\subsubsection{Targets: canonicity}

Canonically, the gender value of the target is redundant and depends on the gender value of the noun. This is a consequence of the Principle of Redundancy, but it also touches on Orthogonality, as each target should have access to all gender values in the language. Virtually all principles demand that the target has gender values that match those of the controller; the Principle of Matching Values makes this explicit. According to Exponence, gender should be expressed by bound morphology. Moreover, the markers should be uniquely distinguishable across other logically compatible features and their values (Clarity).

\subsubsection{Targets: complexity}
\label{sec:Audr:3.3.2}

The informativity or redundancy of the gender value on the target can be illustrated with the help of example (\ref{ex:Audr:7}).

\ea
\label{ex:Audr:7}
\ili{French} (Françoise Kably, p.c.)\\
\begin{xlist}
\ex
\gll elle/il est idiot-e/idiot\\
     \textsc{3sg.f/3sg.m} be\textsc{.prs.3sg} stupid\textsc{{}-sg.f/}stupid\textsc{.sg.m}\\
\glt `She/he is stupid.'
\ex
\gll tu es idiot-e/idiot\\
     \textsc{2sg} be\textsc{.prs.2sg} stupid\textsc{{}-sg.f/}stupid\textsc{.sg.m}\\
\glt `You are stupid.'
\end{xlist}
\z

In (\ref{ex:Audr:7}a) the gender agreement on the adjective is redundant given the gender of the pronominal controller. In (\ref{ex:Audr:7}b), by contrast, the second person pronoun does not distinguish gender, so the gender value on the adjective is informative. How does the difference play out in complexity? Obviously, redundancy is a violation of Economy: it is uneconomical to express the same information twice. From the point of view of Transparency, two views are possible. In one sense, redundancy always violates Transparency since the same feature is marked more than once. In this view, the agreement targets formally realise the gender of the noun. However, it might be argued that the agreement targets themselves have gender as a contextual feature (in the sense of \citealt{Booij1996}), and whatever item has a feature should mark it. This would bring (\ref{ex:Audr:7}a) in line with Transparency after all. Paradigmatically, the evaluation depends on whether one assumes that the \ili{French} 2\textsuperscript{nd} person pronoun is syncretic for the two gender values or does not have gender at all. The first scenario constitutes a disruption of Transparency \textendash{} a single form with two functions \textendash{} but the second does not, as the absence of a distinct form would correlate with the absence of a feature. Finally, Independence attributes greater complexity to (\ref{ex:Audr:7}b) than to (\ref{ex:Audr:7}a) since the gender values \textsc{f} and \textsc{m} on the adjective in (\ref{ex:Audr:7}b) have to be inferred from elsewhere, e.g.\ from the sex of the addressee.

That targets should depend on the controller and match its values syntagmatically follows from the asymmetry of agreement. Note that this is not counted as a violation of Independence, since it is definitional for the controller-target relation. However, any additional dependency or influencing factor constitutes higher complexity in terms of Independence. Two such scenarios deserve discussion. The first is a target having `its own opinion' about value choice and taking on a different gender value than the controller's. A case in point is semantic agreement, for which \ili{Dutch} provides examples.

\ea
\label{ex:Audr:8}
Semantic agreement (\ili{Dutch})\\
\gll dat meisje dat uh die daar achter het stuur zat\\
     \textsc{dem.sg.n} girl(\textsc{n})\textsc{sg} \textsc{rel.sg.n} eh \textsc{rel.sg.c} there behind \textsc{def.sg.n} wheel(\textsc{n}) sit.\textsc{pst.3sg}\\
\glt `that girl who sat behind the wheel'\\
(Corpus Gesproken Nederlands © Nederlandse Taalunie 2014)
\z

In (\ref{ex:Audr:8}) the agreements that go with the neuter noun \textit{meisje} `girl' have two different values: the demonstrative determiner is neuter, while the speaker first chooses a neuter relative pronoun, then hesitates and picks a common gender form.

Semantic agreement is pervasive in \ili{Dutch} relative pronouns, personal pronouns, and possessive pronouns (\citealt{Audring2006,Audring2009}); the relative likelihood is in line with the Agreement Hierarchy \citep{Corbett1979}. This behaviour makes the system more complex because it involves semantics in a place where only syntax should matter; this is the Principle of Independence. Note that Economy is not affected, since there are no additional markers involved (at least not syntagmatically; for the paradigmatic situation see next paragraph). Neither does semantic agreement \textendash{} strictly speaking \textendash{} affect Transparency, as both form and feature value change.

The second deviation from matching values arises when certain targets are paradigmatically unable to match the controller. This happens when the target distinguishes other values than the controller. Again, \ili{Dutch} can serve as an example for this deviation from the canonical situation.

Most agreement targets in \ili{Dutch} distinguish two genders, referred to as common (\textsc{c}) and neuter (\textsc{n}) (\tabref{tab:Audr:4}). Two targets diverge from this pattern. The personal pronouns and the possessive pronouns show an additional distinction between masculine and feminine that is not available to the other targets nor, arguably, for the nouns.%
\footnote{Here we see an example where the agreement class approach mentioned in \sectref{sec:Audr:1.2} runs into analytical difficulties, as gender affiliation is a function of target behaviour, but the targets do not behave uniformly.} %
Note that gender agreement is restricted to the singular, so only singular forms are given.

\begin{table}
\begin{tabularx}{\textwidth}{lXXXXXl}
\lsptoprule
{\bfseries Target/}

\bfseries Gender & \bfseries \textsc{Def} & \bfseries \textsc{Dem} & \bfseries \textsc{Adj} & \bfseries \textsc{Rel} & \bfseries \textsc{Pro} & \bfseries \textsc{Poss}\\
\midrule
\textsc{c/m} & \multirow{2}{*}{\textit{de}} & \multirow{2}{*}{\textit{deze/die}} & \multirow{2}{*}{\textit{{}-e}} & \multirow{2}{*}{\textit{die}} & \textit{hij} & \textit{zijn}\\
\textsc{c/f} &  &  &  &  & \textit{zij} & \textit{haar}\\
\textsc{n} & \textit{het} & \textit{dit/dat} & \textit{ø}\footnotemark{} & \textit{dat} & \textit{het} & \textit{zijn}\\
\lspbottomrule
\end{tabularx}
\caption{Gender agreement in Dutch}
\label{tab:Audr:4}
\end{table}
%
\footnotetext{
\label{fn:Audr:14}
The common gender adjective has the suffix -\textit{e} and the neuter adjective is a bare stem. This formal distinction is restricted to indefinite contexts.}

The additional masculine/feminine split on the pronouns is a violation of the Principle of Independence, since it depends on the target type what gender values are available. Also, the choice of the pronouns requires external motivation. Again, and rather counterintuitively, Transparency is not affected, as each form viewed in isolation corresponds to a single value (an exception is the syncretism of the masculine and the neuter in the possessives which is not our concern here). From the point of view of Economy, the paradigmatic mismatch involves supernumerary distinctions, hence higher complexity.\footnote{One may argue that the reduced paradigm of the attributive targets results in lower complexity from the point of Economy. However, there is little reason to assume that \ili{Dutch} nouns still distinguish three genders \textendash{} speakers are no longer able to systematically distinguish masculine from feminine nouns \textendash{} and the pronouns (including, surprisingly, the neuter) mostly reflect semantic rather than syntactic properties (\citealt{Audring2006,Audring2009}). Therefore, it makes sense to say that the pronouns show more gender distinctions than the nouns, a case of increased complexity.} Note, however, that other languages might show the reverse pattern \textendash{} individual targets with fewer distinctions \textendash{} resulting in lower complexity from the point of view of Economy.

The principles of canonicity not only reflect expectations about the gender value of the target, but also about its morphology. In a canonical system, ``gender is realised through agreement by canonical inflectional morphology, which is affixal'' \citep[509]{Corbett2016}. Interestingly, the difference does not affect complexity as we have defined it here. Neither in terms of Economy nor in terms of Transparency or Independence do we see a compelling reason to say that a bound marker is less or more complex than a free marker (this has been pointed out by \citealt{Leufkens2014}). Hence, such differences do not affect our complexity evaluation.

\largerpage
More relevant for complexity is the final property considered here: the unique distinguishability of gender on the target. Here dedicated markers for gender contrast with portmanteau markers that also express other features (we have seen an example in (\ref{ex:Audr:3})). The Principle of Transparency decrees that a unique marker constitutes the least complex situation. This is in contrast with Economy, since dedicated markers make for more distinct forms. Transparency, in turn, agrees with canonicity in its preference for unique markers. Moreover, computing the form of a polyfunctional marker involves other features, which violates Independence.

Concluding this brief survey of target properties, we see that complexity agrees with canonicity for many properties (\tabref{tab:Audr:5}; again, a tick indicates alignment between maximal canonicity and minimal complexity, cross indicates canonicity but increased complexity, dash means that a principle is not relevant). Other properties leave complexity untouched. Disagreement is found in two cases: redundancy and non-syncretic markers are more complex in terms of Economy. The alignment between matching values and Economy depends on the individual language situation. Note again that the `inbuilt' dependency of the target on the controller is not counted as a violation of Independence.

\begin{table}[t]
\small
\begin{tabularx}{\textwidth}{Xccc}
\lsptoprule
\bfseries The target\ldots & \bfseries Economy & \bfseries Transparency & \bfseries Independence\\
\midrule
\ldots has a gender value that is redundant rather than informative (Redundancy) & \xmark & \cmark & \cmark\\
\padding
\ldots depends for its gender value on the gender value of the noun (Redundancy) & $-$ & $-$ & \cmark\\
\padding
\ldots has gender values that match those of the controller and of other targets (Redundancy, Simple Syntax, Matching Values, CGP) & $-$/\cmark/\xmark & $-$ & \cmark\\
\padding
\ldots has bound expression of agreement (Exponence) & $-$ & $-$ & $-$\\
\padding
\ldots has a gender value that can be uniquely distinguished across other logically compatible features and their values (Clarity) & \xmark & \cmark & \cmark\\
\lspbottomrule
\end{tabularx}
\caption{Canonicity and complexity of the target}
\label{tab:Audr:5}
\end{table}

\largerpage
\subsection{Values}
\label{sec:Audr:3.4}

The values of a feature are inextricably linked to the items that carry them: the controller and the targets. Therefore, most value-related properties have already been touched on in \sectref{sec:Audr:3.2} and \sectref{sec:Audr:3.3}, and this section can be brief.

\subsubsection{Values: canonicity}

Canonically, values have at least the two following properties. First, for any given controller and its targets, gender values do not vary. This is in line with Redundancy, Simple Syntax, Matching Values, and the Canonical Gender Principle, which say that target values should mirror controller values, and that controllers have gender as a lexical property. Invariance includes independence of other features and their values, as decreed by Clarity and Orthogonality. Second, gender values should form a closed class. This is due to Orthogonality: in a fully orthogonal system of lexical items and grammatical features, only the lexical items constitute an open class \citep[502--503]{Corbett2016}. Again, we ask if the canonical situation is also the least complex.

\subsubsection{Values: complexity}
\label{sec:Audr:3.4.2}

Gender values show variation when they are open to choice or change under the influence of other factors. We saw variable controller gender values in \sectref{sec:Audr:3.2}, example (\ref{ex:Audr:6}), and variable target gender values in \sectref{sec:Audr:3.3}, example (\ref{ex:Audr:8}). A more complex situation is found in \ili{Romanian}, where gender values appear to vary between singular and plural, as the neuter gender agreements resemble the masculine in the singular and the feminine in the plural (see \citealt[150--152]{Corbett1991} for an account in which the situation is interpreted not as a case of variation, but as a system with non-unique markers for the neuter gender).

In all cases we see a violation of the Principle of Independence. Independence supports invariant gender values, as a minimally complex gender system is self-contained and does not require reference to other morphosyntactic features such as number, or to non-syntactic factors such as semantics or pragmatics. Therefore, any variation or choice makes the system more complex.

The second property can be interpreted as concerning the number of gender values in a language. The higher this number (i.e.\ the closer to an open set), the greater the range of potential combinations of nouns and gender values, which makes it harder to establish orthogonality (\citealt[502--503]{Corbett2016}).%
\footnote{In the earlier literature, the number of values was used as a criterion for distinguishing gender from classifier systems, with the expectation that gender values should form a ``smallish'' set (\citealt{Dixon1982nounclasses}; \citealt[6]{Aikhenvald2000}).} %
In terms of complexity, fewer gender values also mean lower complexity, though for different reasons: Economy says that the simplest system has the fewest values.

Summarising, we see that the properties of the values affect complexity to a limited degree: the first affects Independence, the second Economy; the other principles are not affected (\tabref{tab:Audr:6}). For both properties, however, maximal canonicity coincides with minimal complexity.

\begin{table}
\small
\begin{tabularx}{\textwidth}{Xccc}
\lsptoprule
\bfseries The values\ldots & \bfseries Economy & \bfseries Transparency & \bfseries Independence\\
\midrule
\ldots do not vary for any given controller and its targets (Clarity, Redundancy, Simple Syntax, Orthogonality, Matching Values, Canonical Gender Principle) & $-$ & $-$ & \cmark\\
\padding
\ldots form a closed class (Orthogonality) & \cmark & $-$ & $-$\\
\lspbottomrule
\end{tabularx}
\caption{Canonicity and complexity of the values}
\label{tab:Audr:6}
\end{table}

\subsection{Domains}
\label{sec:Audr:3.5}

Moving on to domains \textendash{} the syntactic configurations in which agreement occurs \textendash{} we can identify three criteria that contribute to higher canonicity and that can be evaluated for complexity.

\subsubsection{Domains: canonicity}

For domains we can state that the most canonical domain of agreement is the local domain (i.e.\ within the phrase containing the controller; \citealt[21]{Corbett2006}). This is due to Simple Syntax. Indeed, the greater the syntactic distance between controller and target, the more linguistic theories are inclined to exclude the relation from agreement (e.g.\ by speaking of ``cross-reference'' instead; for discussion see \citealt{Barlow1991} and \citealt[134--152]{Barlow1992}, \citealt{Corbett1991}, \citeyear{Corbett2001} and \citeyear{Corbett2006}, and \citealt{Siewierska1999}). Moreover, Clarity increases when there are multiple domains, as more domains provide better analytical evidence for the existence of an agreement system. Multiple domains are also favoured by Orthogonality, as orthogonality between words and features increases with more agreement targets and hence more domains.

Corbett \& Fedden give a third criterion for canonical gender: ``In a canonical gender system the gender of a noun is constant across all domains in which a given language shows agreement'' \citep[517]{Corbett2016}. As this ties in with the lexically specified, single gender value of the controller, the matching gender values of controller and target, and the invariance of all targets for any given controller, all of which were covered in the previous sections, we will not discuss this criterion further.

\subsubsection{Domains: complexity}

When we compare canonicity and complexity (\tabref{tab:Audr:7}), the question arises whether gender agreement within the noun phrase should also count as less complex. Interestingly, within the realm of descriptive complexity that does not consider potential issues of (processing) difficulty, none of the three complexity principles favours one option over the other. Local agreement is neither more economical, nor more transparent or less dependent than agreement elsewhere.

The second domain-related property concerns the number of domains. In a canonical world, agreement involves not one domain but several. However, neither Transparency nor Independence penalises single domains, and with respect to Economy, each additional domain makes the system larger and therefore more complex. Here we see a clear case where canonicity and complexity disagree.

\begin{table}
\small
\begin{tabularx}{\textwidth}{Xccc}
\lsptoprule
\bfseries The domain\ldots & \bfseries Economy & \bfseries Transparency & \bfseries Independence\\
\midrule
\ldots is local (i.e.\ within the phrase containing the controller) (Simple Syntax) & $-$ & $-$ & $-$\\
\padding
\ldots is one of multiple domains (Clarity, Orthogonality) & \xmark & $-$ & $-$\\
\lspbottomrule
\end{tabularx}
\caption{Canonicity and complexity of domains}
\label{tab:Audr:7}
\end{table}

\subsection{Assignment}
\label{sec:Audr:3.6}

Gender assignment rules regulate which gender value is associated with any given noun. Canonicity has little to say about this issue.

\subsubsection{Assignment: canonicity}

Corbett \& Fedden list a single assignment-related criterion for canonical gender, which feeds the Canonical Gender Principle: ``In a canonical gender assignment system, the gender of a noun can be read unambiguously off its lexical entry'' (\citeyear[520]{Corbett2016}). The authors conclude that assignment based on semantics is the most canonical situation (see \citealt[65]{Audring2017}, footnote 22, for an argument against this position). Gender assignment based on formal properties is considered less canonical.

\subsubsection{Assignment: complexity}
\label{sec:Audr:3.6.2}

Complexity also favours semantic assignment rules, but for different reasons. The argument goes by several steps. In \sectref{sec:Audr:1.2} we introduced a distinction between general rules and parochial rules. While this distinction is primarily about scope, it also relates to the number of rules that are needed to account for the gender of every noun in the language: general rules cover a large portion of the noun vocabulary, so the system can operate with only a few such rules, whereas parochial rules take care of a smaller subset of the nouns, requiring more rules overall.

Another factor that is relevant for complexity is the variety of rule types. Does a language employ only semantic rules or also formal rules, and if the latter, are these phonological, morphological, or both?

Complexity is minimal if rules are large in scope (necessitating only a small number of different rules) and of a single type. This is due to Economy: fewer rules and fewer rule types are quantitatively simpler. If we link this to the typological finding that semantic rules can occur without formal rules but not vice versa (\citealt[64]{Corbett1991}, though see \citealt{Killian2015} and \citealtv{chapters/06} on the \ili{Koman} language \ili{Uduk}, which arguably uses only formal rules), we end up with the situation that complexity favours semantic rules. This is the same outcome as for canonicity, but for different reasons. \tabref{tab:Audr:8} summarises the overlap.

\begin{table}
\small
\begin{tabularx}{\textwidth}{Xccc}
\lsptoprule
\bfseries Assignment rules & \bfseries Economy & \bfseries Transparency & \bfseries Independence\\
\midrule
The gender of a noun can be read unambiguously off its lexical entry (CGP); assignment rules are entirely based on semantics & \cmark & $-$ & $-$\\
\lspbottomrule
\end{tabularx}
\caption{Canonicity and complexity of domains}
\label{tab:Audr:8}
\end{table}

\subsection{Summary: canonicity vs.\ complexity}
\label{sec:Audr:3.7}

\begin{table}
\renewcommand{\arraystretch}{1.3}
\footnotesize
\begin{tabularx}{\textwidth}{lXccc}
\lsptoprule
 & \bfseries Property & \bfseries Economy & \bfseries Transparency & \bfseries Independence\\
\midrule
\multirow{3}{*}{\raggedleft \rotatebox[origin=r]{90}{\textbf{Controller\ldots}}} &\footnotesize \ldots is present (Clarity, Redundancy) & \xmark & \cmark & \cmark\\
&\footnotesize \ldots has overt expression of gender (Clarity, Redundancy) & \xmark & \cmark & $-$/\cmark\\
 &\footnotesize \ldots is consistent in the agreements it takes (Simple Syntax, CGP) & $-$ & \cmark & \cmark\\
\midrule
\multirow{3}{*}{\raggedleft \rotatebox[origin=r]{90}{\textbf{Target\ldots}}} &\footnotesize \ldots has a gender value that is redundant rather than informative (Redundancy) & \xmark & \cmark & \cmark\\
&\footnotesize \ldots depends for its gender value on the gender value of the noun (Redundancy) & $-$ & $-$ & \cmark\\
 &\footnotesize \ldots has gender values that match those of the controller and of other targets (Redundancy, Simple Syntax, Matching Values, CGP) & $-$/\cmark/\xmark & $-$ & \cmark\\
 &\footnotesize \ldots has bound expression of agreement (Exponence) & $-$ & $-$ & $-$\\
 &\footnotesize \ldots has a gender value that can be uniquely distinguished across other logically compatible features and their values (Clarity) & \xmark & \cmark & \cmark\\
 &\footnotesize \ldots does not vary for any given controller and its targets (Clarity, Redundancy, Simple Syntax, Orthogonality, Matching Values, Canonical Gender Principle) & $-$ & $-$ & \cmark\\
\midrule
\raggedleft \rotatebox[origin=r]{90}{\textbf{Values\ldots}} &\footnotesize \ldots do not vary for any given controller and its targets (Clarity, Redundancy, Simple Syntax, Orthogonality, Matching Values, Canonical Gender Principle) & $-$ & $-$ & \cmark\\
&\footnotesize \ldots form a closed class (Orthogonality) & \cmark & $-$ & $-$\\
\midrule
\multirow{2}{*}{\raggedleft \rotatebox[origin=r]{90}{\textbf{Domain\ldots}}} &\footnotesize \ldots is local (i.e.\ within the phrase containing the controller) (Simple Syntax) & $-$ & $-$ & $-$\\
&\footnotesize \ldots is one of multiple domains (Clarity, Orthogonality) & \xmark & $-$ & $-$\\
\midrule
\raggedleft \rotatebox[origin=r]{90}{\textbf{Assignment}\hspace{-.2cm}} &\footnotesize The gender of a noun can be read unambiguously off its lexical entry (CGP);
assignment rules are entirely based on semantics & \cmark & $-$ & $-$\\
\lspbottomrule
\end{tabularx}
\caption{Canonicity vs.\ complexity, summary}
\label{tab:Audr:9}
\end{table}

The comparison of properties of gender systems in terms of canonicity vs.\ complexity is summarised in \tabref{tab:Audr:9}. A number of observations can be made. First, there are various properties that are relevant to canonicity but not to complexity, or only to a single complexity principle; these are indicated by dashes. If dashes are discarded (i.e.\ if only ticks and crosses are considered), an interesting pattern emerges. Transparency and Independence always line up with canonicity (again, ticks indicate maximal canonicity and minimal complexity). Economy, by contrast, disagrees with canonicity in the majority of the cases. There are only three properties for which the most canonical option is also maximally simple: mismatching values involving reduced values, fewer gender values, and a purely semantic assignment system. For the latter two, however, we saw that canonicity and Economy arrived at the same preference by different arguments (see \sectref{sec:Audr:3.4.2} and \sectref{sec:Audr:3.6.2}). Hence the alignment is even weaker than \tabref{tab:Audr:9} suggests.

What we see is that canonical gender systems can be complex, which means that there are areas where complexity is expected of \textendash{} perhaps even inherent to \textendash{} grammatical gender. The principles most at odds are Clarity and Redundancy on the side of canonicity and Economy on the side of complexity.

Having completed the comparison of canonicity and complexity, we move on to the third issue under consideration: difficulty. \sectref{sec:Audr:4.1} introduces difficulty and motivates the evidence selected for this paper. \sectref{sec:Audr:4.2} identifies and discusses factors that influence difficulty in first language acquisition. \sectref{sec:Audr:4.3} ties together the results and links them to the previous issues, canonicity and complexity.

\section{Difficulty}
\label{sec:Audr:4}

\subsection{Introduction: difficulty}
\label{sec:Audr:4.1}

In contrast to descriptive complexity, which is an absolute evaluative measure, difficulty is inherently relative: a particular structure is difficult for somebody in the context of some particular task. The experiencer can be a speaker, a hearer, or a learner, and the task can be, for instance, language processing or acquisition. The following section discusses difficulty in the context of first language acquisition. Adult second language acquisition is excluded because it increases the empirical space by many additional variables, chiefly the first language (Does it have a gender system? Are the systems of L1 and L2 similar?), the learner (age, motivation) and the learning context (amount of exposure, explicit instruction or not). This makes it much harder to isolate the specific factors that accelerate or delay acquisition of gender (though see \citealt{Kusters2003} for an account of relative complexity, i.e.\ difficulty, based on second language acquisition).

There is a wealth of literature available on first language acquisition of gender in a variety of languages. Unfortunately, the languages addressed are mostly \ili{Indo-European}, with the notable exception of \citet{Gagliardi2014} on \ili{Tsez}, and a number of studies on \ili{Bantu} languages (\ili{Niger-Congo}); see \citet{Demuth2003} for an overview.

Comparison is impeded by the diversity of the studies. Differences range from who is tested (single children, groups of children), when they are tested (the ideal period lies between 2 and 8 years, but most studies cover smaller time spans), how the data is collected (in diary studies, in the lab, naturally or experimentally) to what is tested (mostly production, sometimes comprehension) and on what items (often existing nouns, sometimes nonce nouns). Methodological choices have important theoretical consequences. Comprehension can reveal abilities that are not yet apparent in production (see e.g.\ \citealt{Heugten2010}), and performance on different types of item might reflect different types of learning. For example, correct use of gender with existing nouns can reflect item-based learning, while the ability to classify nonce words may indicate the successful discovery of assignment rules.

Also, there are differences in what is considered the point of successful acquisition. Correctness levels may vary between nouns and between genders, but also between agreement targets, whereby early success with targets close to the noun may reflect knowledge associated with individual lexemes or even combinations acquired as holophrases, amalgams, or chunks \citep[59--60]{MacWhinney1978}. Many studies adopt Brown's (\citeyear{Brown1973}) method of using 90\% correctness as threshold: an error rate of less than 10\% means that gender has been successfully acquired.

Such difficulties notwithstanding, the various studies present some indications of the properties of a language that aid or hinder the acquisition of its gender system. These will be discussed next.

\subsection{Evidence from first language acquisition}
\label{sec:Audr:4.2}

We assume that ease of acquisition is reflected in speed of acquisition: simple systems are acquired faster and/or earlier.%
\footnote{It might be desirable to distinguish fast from early acquisition, since delays can be due to maturational constraints or because one property relies on the mastery of another (and once the first property is mastered, the second is acquired fast; thanks to Bernhard Wälchli for pointing this out).  However, the evidence provided by the literature \textendash{} especially with regard to first language acquisition \textendash{} is usually on absolute time (early/late) rather than relative time (fast/slow), so the distinction has to be disregarded here.} %
Gender systems appear to be in place around the age of three in most languages reported in the literature. For the purposes of this section, the most relevant studies are those that compare acquisition in two or more languages and report faster or slower success for individual languages (e.g. \citealt{Mills1986}; \citealt{Eichler2013}) or that point out significant delays (e.g. \citealt{Mulford1985}; \citealt{Blom2008}).

A review of the relevant literature yields a consensus on four general factors that influence the acquisition of gender. These can be subsumed under the terms

\begin{itemize}
\item Frequency
\item Perspicuity
\item Consistency
\item Monofunctionality
\end{itemize}

Note that these factors are the result of observations rather than theoretical stipulations such as the principles used in canonicity and complexity profiling (\sectref{sec:Audr:2} and \sectref{sec:Audr:3}). Let us consider each in turn.


\subsubsection{Frequency}

Frequency reflects the number of times a child is exposed to a particular item or structure. Unsurprisingly, a positive effect of higher frequency is reported in a variety of studies. Particularly for the initial stages, acquisition is described as proceeding in a piecemeal, item-based manner. Correct use of gender morphology may initially be tied to specific lexical items or individual agreement markers which are mastered early because they often (co-)occur in the input (e.g.\ \citealt{Mariscal2009,Szagun2007}; \citealt[115]{Mills1986}). Conversely, patterns may be delayed because they are represented with insufficient frequency. \citet{Rodina2014}, for example, reports that \ili{Russian} children have difficulties with female person names ending in \textit{-ik} or \textit{-ok} and with nouns such as \textit{doktor} `doctor' when referring to a woman. These nouns contradict morphophonological rules (their form suggests masculine gender) in favour of semantics: adult speakers strongly prefer feminine agreement in accordance with natural gender. While children master the formal rules early, the semantically motivated exceptions are discovered late because such nouns are infrequent in the input.

Frequency can affect entire gender values. A well-known case is the neuter gender in \ili{Dutch}, which is acquired with an astonishing delay: children still show around 25\% errors at age 7 (\citealt{Blom2008}, see also \citealt{Keij2012} and references there). This is due to the much lower frequency of neuter nouns in the language, plus a condition on the neuter form of adjectives that restricts its presence in the input (see footnote~\ref{fn:Audr:14}).

Generalising to gender systems as a whole, we see that frequent marking in general paves the way to early acquisition. \citet{Szagun2007} remark that nouns co-occur with articles in most contexts in \ili{German}, which ensures early success in acquisition since articles are important gender cues. \citet{Eichler2013} suggest the same correlation for \ili{French}. Noun class markers in \ili{Bantu} appear on a broad range of agreement targets in a variety of domains and are therefore highly frequent. Acquisition studies report that they are in place by age 2;6--3 \citep{Demuth2003}, despite the large number of classes and their low degree of semantic motivatedness. By contrast, mastery of the apparently much simpler \ili{English} gender system is comparatively slow; gender errors with person names are found beyond age 4 and errors with non-persons beyond age 6 (\citealt[91, 103]{Mills1986}). The main reason is that there are few cues in the input, since agreement is restricted to pronouns.

Taken together, the evidence suggests that the difficulty of acquiring a gender system is influenced by the frequency with which the child hears the nouns in company of agreeing words. The more agreement targets there are in the language, and the higher their frequency in use, the earlier the system is detected and mastered.

\subsubsection{Perspicuity}
\label{sec:Audr:4.2.2}

If the morphological markers are the central cues to acquisition, such cues are expected to work best when they are perspicuous and clear. Formal perspicuity can be a function of phonological weight (including stress) and relative distinctness, but also of the degree to which a gender value is expressed by a typical form. \citet{Arias-Trejo2013}, for example, report that \ili{Spanish} children are able to use gender agreement as a predictor of form-meaning correspondences in novel nouns from an early age onwards; the authors attribute this to the clear presence of the suffixes \textit{-a} (feminine) and \textit{-o} (masculine) in the input.%
\footnote{Such explanations are interpretations, and the same facts are sometimes presented as evidence for opposing views. Thus, \citet{Mariscal2009} analyses the difference between \ili{Spanish} \textit{-a} and \textit{-o} as ``subtle'' and lists it among the properties that hinder rather than help acquisition (148, 149).} %
Similarly, the feminine definite article in \ili{Italian} is acquired before the masculine because it has fewer allomorphs \citep[514]{Pizzuto1992}. For the complex morphological paradigms of \ili{Bantu}, early and error-free acquisition is reported and explained by the perspicuity of the noun class prefixes \citep[213]{Demuth2003}.

Conversely, perspicuity is impeded by syncretism, especially when reaching across orthogonal features. The \ili{German} definite article \textit{der}, for example, is syncretic for nominative masculine and genitive feminine. \citet{Eichler2013} mention this factor as an explanation for the slower acquisition of \ili{German} gender as opposed to \ili{French} gender, the two systems being otherwise similar in complexity. A similar point is raised for \ili{Icelandic} (\citealt{Mulford1985}; \citealt{Levy1988}) where noun-final \textit{-a} and \textit{-i} can be cues for feminine respectively masculine gender, but both endings occur in various places within the complex inflectional class system, which makes it harder for the child to discover the correlation. Here, clarity overlaps with functionality, a point discussed in \sectref{sec:Audr:4.2.4} below.

There is interesting, though cursory, evidence that affixes might be more easily detectable than non-affixal phonological gender cues, being more perspicuous as a unit. Studies report that, in particular, diminutive affixes facilitate gender acquisition (e.g.\ \citealt{Kempe2003} for \ili{Russian} and \citealt{Cornips2008} for \ili{Dutch}).

Overall, there is a consensus in the literature that children use formal cues earlier or to better effect than semantic cues. This has been reported for \ili{Tsez} \citep{Gagliardi2014}, \ili{French} \citep{Karmiloff-Smith1979}, \ili{Spanish} \citep{PerezPereira1991}, \ili{German} \citep{MacWhinney1978,Mills1986}, and \ili{Russian} \citep{Rodina2014,Rodina2012}. The only dissenting study is \citet{Mulford1985}, who finds that \ili{Icelandic} children master semantic cues earlier (though see \citealt{PerezPereira1991} for methodological criticism). However, \ili{Icelandic} may be a language in which neither the semantic nor the formal cues are particularly clear, as \citet{Levy1988} hypothesises.

Perspicuity is not necessarily tied to form. Semantic cues to gender can also vary in semantic perspicuity, i.e.\ salience. Importantly, what is evident or salient for the adult speaker may not be so for the gender-acquiring child. Studies show that even natural gender, which seems an obvious and straightforward semantic parameter, is not apparent in the use of gender morphology by young children (\citealt{Szagun2007} for \ili{German}; \citealt{Rodina2014} for \ili{Russian}; \citealt{Mills1986} for \ili{English}). A similar argument is brought forward by \citet{Plaster2010} to refute the complex semantics suggested by \citet{Dixon1972} and \citet{Lakoff1987} for the gender system of \ili{Dyirbal} \textendash{} the proposed system would be unlearnable, since the semantic parameters would not yet be available to the child.

\subsubsection{Consistency}

The clearest cues to gender are also the most consistent: an ideal cue has a unique form that consistently represents a particular gender value. This holds for morphological markers as well as entire nouns. Consistency is broken by variation. For example, the female names ending in \textit{-ik} or \textit{-ok} discussed by \citet{Rodina2014} contain an inconsistent cue: the suffixes normally indicate masculine gender. However, such nouns are mastered earlier than the \textit{doktor}-type nouns included in the same study. It might be argued that the former represent a lower degree of inconsistency, as each individual suffixed noun is either masculine or feminine, whereas the latter show variation for every individual noun.

The basic insight for the acquisition of assignment rules is that categorial rules are the easiest to acquire \citep[114]{Mills1986}. Stochastic rules involving inconsistent cues are harder to figure out and appear to be learned later. The relevant parameter is sometimes called \textit{reliability} or \textit{validity} \citep{MacWhinney1978}, a prominent term in the Competition Model by \citet{MacWhinney1989}. Highly valid cues have high predictive power by being consistently associated with a certain gender value.

Summing up the three factors discussed so far, gender cues work best when they are ``sufficiently frequent, adequately valid and easily perceivable'' (\citealt[68]{Wegener1995} for \ili{German}, translation mine). Similar statements are made for \ili{Spanish} (\citealt{Mariscal2009}; \citealt{PerezPereira1991}) and \ili{Italian} \citep[545]{Pizzuto1992}. For the purposes of the present study a fourth factor, monofunctionality, is worth singling out, though it is not entirely independent of the previous three.

\subsubsection{Monofunctionality}
\label{sec:Audr:4.2.4}

Gender markers are dedicated or monofunctional when they express gender and nothing else. However, many languages have gender markers that are polyfunctional and encode two or more properties. Shared functions are usually other features such as number or case, inflectional class, or definiteness. Any kind of polyfunctionality affects both clarity and consistency.

The clearest evidence that gender acquisition is delayed by the parallel acquisition of case is adduced for \ili{German}. \citet{Eichler2013} observe that \ili{German} gender is acquired later than \ili{French}, \ili{Italian}, or \ili{Spanish} gender and attribute this to the influence of case. \citet{Bewer2004} reports an early peak in gender correctness followed by a relapse when case starts to emerge. Conversely, \citet{PerezPereira1991} notes that \ili{Spanish} gender agreement markers are more transparent because they do not vary with case.

In her famous study on \ili{Icelandic}, \citet{Mulford1985} finds that gender is acquired late, with a particular delay in the discovery of formal cues. An explanation is sought in the polyfunctionality of the markers in the highly complex \ili{Icelandic} inflectional class system, which obscures the correlations between the nominal suffixes and gender.

The impact of polyfunctionality on acquisition is strongest in cases where the child can be suspected of erroneously associating gender markers with other functional properties. \citet{Bittner2002} suggests that \ili{German} children might initially regard the masculine definite article \textit{der} as a marker of subjecthood or agentivity. \ili{Dutch} children appear to start out assuming that the \ili{Dutch} article \textit{de} is a definiteness marker, delaying the discovery of gender (\citealt{Keij2012}; \citealt{Cornips2008}).

Generally speaking, the earlier acquisition of formal cues reported in \sectref{sec:Audr:4.2.2} interestingly suggests that form-form correlations might be easier to acquire than form-function correlations, especially when various functions employ the same morphological markers.

Closing this section of literature review, two sporadic observations might be worth noting. Firstly, a variety of studies indicate early mastery of agreement in local domains, with more persistent errors in the use of distant targets such as pronouns. This suggests a correlation between difficulty and domains. Secondly, and partly contradicting the previous point, \citet[545]{Pizzuto1992} report tendentially better results for bound morphology over free markers in \ili{Italian}, with verbal inflection being acquired before pronouns and articles. However, there is little evidence for or against this pattern in the other literature consulted. Both points, however, are in line with what might be expected from the perspective of canonicity. This brings us to the final section, which ties together the three domains of evaluation.

\subsection{Summary: canonicity, complexity, difficulty}
\label{sec:Audr:4.3}

Returning to the question we set out with, we can now ask how the factors relevant to difficulty line up with those pertaining to canonicity and complexity. \tabref{tab:Audr:10} summarises the alignment of difficulty on the one hand with canonicity and the three types of complexity on the other. As in the previous tables, ticks indicate alignment (minimal difficulty, maximal canonicity, minimal complexity). Divergences (minimal difficulty, lower canonicity, higher complexity) are indicated by crosses. Dashes mean no alignment since a factor for difficulty is irrelevant to canonicity and/or complexity.

\begin{table}
\begin{tabularx}{\textwidth}{Xcccc}
\lsptoprule
\bfseries Difficulty & \bfseries Canonicity & \bfseries Economy & \bfseries Transparency & \bfseries Independence\\
\midrule
Frequency & $-$/ \cmark & $-$/\xmark & $-$/\xmark & $-$\\
Perspicuity & \cmark & \cmark/\xmark & \cmark & \cmark\\
Consistency & \cmark & $-$ & \cmark & \cmark\\
Monofunctionality & \cmark & \xmark & \cmark & \cmark\\
\lspbottomrule
\end{tabularx}
\caption{Difficulty vs.\ canonicity and complexity, summary}
\label{tab:Audr:10}
\end{table}

Starting with frequency, we saw that difficulty introduces parameters into the discussion that are of limited relevance to canonicity or complexity: the usage frequency of nouns and agreeing elements matters only to difficulty. Syntagmatic frequency as dependent on the number of targets, by contrast, is relevant to all three evaluative measures, but in contradictory ways: canonicity leads us to expect several targets in various domains (Principle of Redundancy, Principle of Orthogonality), which violates Economy and potentially Transparency and therefore results in a more complex system.%
\footnote{As noted in \sectref{sec:Audr:3.3.2}, the decision for Transparency depends on the theoretical perspective. Are agreement markers seen as redundantly realising the feature of the noun? Then agreement is always a violation of Transparency. Or do the agreement targets in fact express their own contextual feature (although the value is dependent on the noun)? In this case agreement is not necessarily non-transparent.} %
For difficulty, more targets mean greater perspicuity, hence facilitation of acquisition.

Perspicuity, in turn, lines up with Transparency, Economy, and Independence in that a perspicuous, i.e.\ alliterative, form without allomorphic variants makes for the best gender cue in acquisition, as well as the most transparent and the most economical agreement marker needing the least additional specifications. Such markers are also the most canonical. Similarly, perspicuity is greater in the absence of syncretism, as is Transparency. Economy, on the other hand, might be said to favour syncretism. It might also favour markers that are unstressed or phonologically light, in disagreement with perspicuity.

Not shown in \tabref{tab:Audr:10} is difficulty diverging from both canonicity and complexity in the preference for formal cues over semantic cues in the early stages of gender acquisition. This is surprising, as semantic motivations for gender are more canonical and potentially less complex.

The third factor relevant for difficulty, consistency, is clearly in line with canonicity: canonical agreement controllers, targets, and values are expected to show predictable, consistent behaviour. This is also the least complex situation according to Transparency and Independence. The Canonical Gender Principle, according to which each noun should have a single gender value, also describes the situation of least difficulty, as variation slows down acquisition.

Moving on to the fourth difficulty factor, monofunctional markers are the easiest to learn as well as the most transparent and the most independent. They are also the most canonical, as monofunctionality ensures the unique distinguishability of gender across other features. Again, this contradicts Economy, which might be said to favour cumulative markers or reduced paradigms.

A less expected outcome from the point of view of functionality is, again, that form-form relations might initially be easier to detect in the input than form-function relations, with functions being figured out at a later stage.

Finally, however, attention should be drawn to a pattern that might be expected but is not found: there is no evidence for slower acquisition of systems with higher numbers of gender values. Studies on \ili{Bantu} noun class acquisition (summarised in \citealt{Demuth2003}) report that agreement within the NP (demonstratives and possessives) is in place around age 2;4--2;6, followed by class prefixes on the noun (2;6--2;8 in \ili{Siswati} and \ili{Sesotho}, even earlier in \ili{Zulu}), then verb agreement. The entire noun class system is mastered by age 3. This matches the age of successful gender acquisition mentioned for \ili{Italian} and \ili{Spanish} (see the summary in \citealt[556]{Eichler2013}), despite the fact that these languages have two gender values while the cited \ili{Bantu} languages have around seven.%
\footnote{The number is an approximation, as the Bantuist tradition counts singular and plural classes separately and includes locative classes, which leaves some room for analytical variation.} %
By contrast, the acquisition of \ili{English} and \ili{Dutch}, which have far fewer gender values, shows much slower progress. This indicates that the number of classes, which seems such a central and obvious criterion for complexity (i.e.\ Economy), is in itself not at all relevant for difficulty. Here, canonicity, which ascribes no special status to the number of values, lines up better with difficulty than does complexity.

Summing up, we arrive at an interesting result. Of the three principles for complexity, Independence makes the most accurate predictions for difficulty: cross-cutting features, inter-feature syncretism, and one feature depending on another hinder acquisition, as does any compromise on consistency.

Violations of Transparency, in turn, make the system harder to acquire when there are fewer forms than functions. This holds both for the syntagmatic and the paradigmatic dimension, i.e.\ for syncretism as well as for cumulative exponence. However, syntagmatic transparency violations that involve overrepresented, i.e.\ redundantly repeated markers appear to be beneficial: redundancy increases the perspicuity of gender and thereby aids acquisition.

As in the comparison of canonicity and complexity (\sectref{sec:Audr:3.7}), Economy is the odd one out. Economy does not line up with canonicity, and violations of Economy often help rather than hinder learning. The burden of acquiring additional morphology and a greater range of agreement domains is eclipsed by the benefits in perspicuity and frequency. Even for the number of gender values no negative effect is found.

As a consequence, canonicity ends up a better predictor of difficulty than complexity. Economy, which is not a priority in canonicity, is also not a priority in difficulty. In fact, low economy with regard to syntagmatic exponence turns out to be an advantage.

\section{Conclusions}
\label{sec:Audr:Concl}

In this chapter I have compared and contrasted three evaluative measures: canonicity, complexity, and difficulty. By profiling the typological space of grammatical gender in terms of canonicity and complexity, individual linguistic properties are identified as being more or less canonical, and/or more or less complex. The general result is one of agreement: maximal canonicity lines up well with low complexity and minimal difficulty. The notable exception is the Principle of Economy, according to which maximal canonicity often means higher complexity.

The comparison is then extended to difficulty in first language acquisition. The result is similar: difficulty, canonicity, and complexity largely agree, with the exception of Economy. Violations of Economy can go hand in hand with maximal canonicity and early acquisition. This means that structures may be complex but canonical and easy to learn. This is due to the central role of Clarity respectively perspicuity: systems that offer rich cues and stand out in the grammar provide the best evidence for the linguist and for the language-acquiring child.

The study demonstrates that assessing the complexity, canonicity, and difficulty of gender systems requires typological understanding as well as explicit principles for evaluation in order to arrive at a motivated and consistent judgment.

\section*{Acknowledgements}

I am grateful to Grev Corbett, Ray Jackendoff, Anna Thornton, and to the editors of the volume for invaluable comments and advice. The support of the \ili{Dutch} national research organisation NWO (Veni grant \#275-70-036) is gratefully acknowledged.

\section*{Special abbreviations}

\noindent The following abbreviations are not found in the Leipzig Glossing Rules:
\medskip

\noindent
\begin{tabular}{ll ll ll}
\textsc{bg} & background & \textsc{vblz} & verbalizing morpheme & \textsc{c} & common gender \\
\end{tabular}


\printbibliography[heading=subbibliography,notkeyword=this]

\end{document}
