\documentclass[output=collectionpaper]{langsci/langscibook}
\ChapterDOI{10.5281/zenodo.3462754}

\title{Introduction}
\author{Francesca Di Garbo\affiliation{Stockholm University}\and Bruno Olsson\affiliation{Australian National University}\lastand Bernhard Wälchli\affiliation{Stockholm University}}
\abstract{\noabstract}

\rohead{}

\begin{document}
\maketitle

\noindent%
This chapter introduces the two volumes \textit{Grammatical gender and linguistic complexity I: General issues and specific studies} and \textit{Grammatical gender and linguistic complexity II: World-wide comparative studies}.

Grammatical gender is notorious for its complexity. \citet[1]{Corbett1991} characterizes gender as ``the most puzzling of the grammatical categories''. One reason is that the traditional definitional properties of gender \textendash{} noun classes and agreement \textendash{} are very intricate phenomena that can affect all major areas of language structure. Gender is an interface phenomenon par excellence and tends to form elaborate systems, which is why the question of how systems emerge in language development and change is highly relevant for understanding and modeling the evolution of gender systems. In addition, some of the recent literature on linguistic complexity claims that gender is `historical junk’ without any obvious function (\citealt[156]{Trudgill2011}) and is likely to be lost in situations of increased non-native language acquisition (\citealt{McWhorter2001,McWhorter2007}; \citealt{Trudgill1999}). Not only are its synchronic functions a matter of debate, but gender also tends to be diachronically opaque due to its high genealogical stability and entrenchment (\citealt[142]{Nichols1992}; \citealt{Nichols2003}), making gender a core example of a mature phenomenon (\citealt{Dahl2004}). However, despite the well-established connection between gender and linguistic complexity, and recent attempts to develop complexity metrics for gender systems (\citealt{Audring2014,Audring2017}; \citealt{DiGarbo2016}) and metrics for addressing the relationship between gender and classifiers (\citealt{Passer2016b}), there is so far no collection of articles particularly devoted to the relationship between grammatical gender and linguistic complexity.

The two companion volumes introduced here are an attempt to fill this gap. They address the topics of gender and linguistic complexity from a range of different perspectives and within a broadly functional--typological approach to the understanding of the dynamics of language. Specific questions addressed are the following:

\begin{itemize}
\item \textbf{Measurability of gender complexity:}\\
What are the dimensions of gender complexity, and what kind of metrics do we need to study the complexity of gender cross-linguistically? Are there complexity trade-offs between gender and other kinds of nominal classification systems? Does gender complexity diminish or increase under the pressure of external factors related to the social ecology of speech communities?
\item \textbf{Gender complexity and stability:}\\
How does gender complexity evolve and change over time? To what extent do the gender systems of closely related languages differ in terms of their complexity and in which cases do these differences challenge the idea of gender as a stable feature? How complex are incipient gender systems?
\item \textbf{Typologically rare gender systems and complexity:}\\
How do instances of typologically rare gender systems relate to complexity? What tools of analysis are needed to disentangle and describe these complexities?
\end{itemize}

\noindent Discussion around these topics was initiated during a two-day workshop on ``Grammatical gender and linguistic complexity'' that took place at the Department of Linguistics at Stockholm University, Sweden, November 20--21, 2015. Most chapters included in the two volumes are based on papers first presented and discussed during this workshop. However, some additional authors came on board after the workshop and all contributions went through considerable modifications on their way to being included in the collection of articles. The result consists of 14 chapters (including this introduction) in two volumes, which address the questions listed above, while investigating the many facets of grammatical gender through the prism of linguistic complexity.

The chapters discuss what counts as complex or simple in gender systems, and whether the distribution of gender systems across the world’s languages relates to the language ecology and social history of speech communities. The contributions demonstrate how the complexity of gender systems can be studied synchronically, both in individual languages and across large cross-linguistic samples, as well as diachronically, by exploring how gender systems change over time.

\subsection*{Organization of the two volumes}
The first volume, \textit{Grammatical gender and linguistic complexity I: General issues and specific studies} (henceforth referred to as Volume I), consists of three chapters on the theoretical foundations of gender complexity, and six chapters on languages and language families of Africa, New Guinea and South Asia. The second volume, \textit{Grammatical gender and linguistic complexity II: World-wide comparative studies} (henceforth referred to as Volume II), consists of three chapters providing diachronic and typological case studies, and a final chapter discussing old and new theoretical and empirical challenges in the study of the dynamics of gender complexity. The rest of this section is a roadmap providing summaries of the following thirteen chapters.

\subsubsection*{Volume~I: General issues and specific studies}
Part~\ptgeneral{}, General issues, in Volume~I, starts with \textbf{Jenny Audring}'s contribution. Building on previous work in Canonical Typology, Audring proposes that a maximally canonical gender system is one in which formal clarity and featural orthogonality reign, unperturbed by morphological cumulation and cross-category interactions. Canonical gender is also populated by well-behaved targets exhibiting unambiguous agreement, in accordance with the (transparently assigned) gender of their controllers. Alongside this hypothetical clustering of canonical properties, Audring, building on earlier literature, establishes three main dimensions according to which the complexity of a gender system can be gauged: economy (a system with fewer distinctions is less complex than one with many distinctions), transparency (a one-to-one mapping between meaning and form is less complex than a one-to-many mapping) and independence (a system in which all features are independent of each other is less complex than one where they interact). Starting from the postulate that the maximally canonical gender system should also be minimally complex, Audring examines how the canonicity parameters fare against the complexity measures, and finds that the criteria from canonicity and complexity largely converge, with economy being the glaring exception: a canonical gender system is an uneconomical one. The discussion then turns to the notion of difficulty, here understood as the speed with which children acquire the gender system of their first language. With the premise that a gender system of maximal canonicity and minimal complexity should also be the least difficult to acquire, Audring compares the criteria for canonicity and complexity with factors that are known to facilitate the acquisition of a gender system. The result of this comparison is general convergence between the three dimensions, again except for economy. An otherwise canonical and simple gender system will be easier to acquire if it also features ample redundancy.

Exploring the relationship between language structures and sociohistorical and environmental factors is one of the most debated issues in recent quantitative typological research. In his contribution, \textbf{Östen Dahl} asks whether there is a negative correlation between the complexity of grammatical gender and community size in line with the general claim that languages with large populations feature simpler morphology than smaller languages. Gender systems presuppose non-trivial patterns of grammaticalization and complex types of encoding in inflectional morphology. In addition, contact-induced erosion and loss of grammatical gender are well documented in the literature. Yet, Dahl shows that it is very hard to find any clear-cut statistically significant correlation between gender features as documented in the \textit{World atlas of language structures} (\textit{WALS}) and language size. Similarly, gender features do not clearly correlate with any of the inflectional categories represented in \textit{WALS}, with the exception of systems of semantic and formal gender assignment, which tend to be found in languages with highly grammaticalized nominal number marking. Dahl argues that in order to better understand the impact that language-external factors may have on the complexity of gender systems, areal and genealogical skewing in the distribution of types of gender systems and the demographic profile of the languages need to be taken into account. Furthermore, he suggests that more elaborate classifications of gender systems than those currently available in typological databases are needed in order to identify those aspects of gender marking that are most likely to adapt to the pressure of language-external factors, as well as a shift in perspective from synchronic to diachronic typologies.

\textbf{Johanna Nichols} uses canonicity as a starting point for her discussion of the relative complexity of gender agreement. As in Audring’s contribution, exponence of gender is non-canonical inasmuch as it departs from the structuralist ideal of biunique form--function correspondence. Nichols proposes the reasonable hypothesis that gender systems are in fact not complex in themselves. Rather, their complexity is a side-effect of gender arising primarily in languages that have already cultivated considerable complexity elsewhere in their grammars. But empirical testing of this hypothesis suggests that it must be rejected, because Nichols shows \textendash{} surprisingly perhaps \textendash{} that languages with grammatical gender do not display a higher degree of overall morphological complexity than languages without gender. The question is then which diachronic processes cause gender systems to accumulate complexity over time, even when the rest of the morphological system manages to avoid increased complexification. Nichols identifies one clue to this puzzle by comparing gender to participant indexation, and, more specifically, to cases in which such systems display hierarchical patterning (as when a verb form indexes the participant that ranks highest on a hierarchy such as 1, 2 > 3). In Nichols’ view, this is an example of a ``self-correcting mechanism'' that can act as a cap on complexification within indexation systems. Gender systems, on the other hand, do not have recourse to such mechanisms, because markers of gender agreement lack the referential function that participant indexes, such as pronouns, have.

Part~\ptafrica{} of Volume I focuses on languages of Africa. Gender systems in \ili{Niger-Congo} languages are among the most studied instances of grammatical gender cross-linguistically. Yet to a large extent this body of research is based on a tradition of analysis which is strongly \ili{Bantu}-centered and not easily applicable to other language families within and outside Africa. The chapter by \textbf{Tom Güldemann} and \textbf{Ines Fiedler} seeks to overcome this limitation by proposing a novel toolkit for the analysis of \ili{Niger-Congo} gender systems. The kit rests upon four notions: agreement class, nominal form class, gender and deriflection, and aims to be universally applicable to the description of any language-specific gender system as well as for the purpose of cross-linguistic comparison. While the notions of nominal form class and agreement class have to do with the concrete morphosyntactic contexts in which nominal and non-nominal gender marking occur, gender and deriflection are more concerned with the abstract, lexical dimension of grammatical gender. By using these analytical tools, Güldemann and Fiedler dismiss the notion of noun class which has been largely used in \ili{Niger-Congo} studies and which rests on the problematic assumption that there is a systematic one-to-one mapping between nominal form classes and agreement classes. The authors demonstrate the descriptive adequacy of the proposed approach by focusing on data from three genealogically and/or geographically coherent \ili{Niger-Congo} groups in West Africa: \ili{Akan}, \ili{Guang} and \ili{Ghana-Togo-Mountain}. They show how the new method reveals some important generalizations about \ili{Niger-Congo} gender systems. For instance, agreement class inventories are always simpler (or at least not more complex) than nominal form class inventories, both in terms of number of distinctions and types of structures. Diachronically, this means that the systems of nominal form classes can be more conservative than those of agreement classes.

The contribution by \textbf{Don Killian} discusses the gender system of \ili{Uduk}, a \ili{Koman} language of the Ethiopian-(South) Sudanese borderland, with special emphasis on some unusual properties of the agreement and assignment principles operating in the language. Gender agreement in \ili{Uduk} is primarily realized in a set of clitics that attach to the verb, and which mark the case role and gender of a core argument that immediately follows the verb. The fact that these postverbal clitics only appear when immediately followed by the corresponding argument points to the fundamental role of adjacency in this gender system, a point also illustrated by conjunctions and complementizers, which agree in gender with the following nominal. According to Killian, gender assignment is largely arbitrary, even for the highest segments of the animacy hierarchy, where one could expect to find assignment based on salient features of the referent (such as sex). Furthermore, the irrelevance of the referent for gender assignment extends to pronouns and demonstratives, which invariably trigger agreement according to Class I. Apart from a few formal rules (targeting derived nouns), there seem to be no clear-cut semantic patterns that could bring order to this unwieldy assignment system. Killian proposes that the \ili{Uduk} gender is non-canonical but relatively simple \textendash{} features that would easily make this gender system slip under the typologist’s radar.

In the first of three contributions focusing on languages of New Guinea (Part~\ptnewguinea{} of Volume~I), \textbf{Matthew Dryer} presents an overview of gender in \ili{Walman}, a \ili{Torricelli} language. Gender agreement in \ili{Walman} is shown in third person agreement on verbs, where the sets of subject and object affixes distinguish feminine and masculine agreement. Agreement is also found, albeit less systematically, on a subset of nominal modifiers, including some adjectives and demonstratives. Gender assignment is sex-based for humans and large animals, arbitrary for lower animals, whereas almost all inanimates are feminine, with spill-over into the masculine for some natural phenomena (which, like animates, are capable of autonomous force). Dryer presents two analytical puzzles for the description of \ili{Walman} gender. The first concerns the large group of pluralia tantum nouns, which trigger invariant plural agreement instead of the standard masculine or feminine (singular) agreement. This group of nouns is about twice as large as that of masculine nouns, so if the number of members is taken as decisive for the status of a category, then the pluralia tantum category in \ili{Walman} is clearly on a par with the two uncontroversial genders. The second puzzle concerns diminutive agreement. The \ili{Walman} diminutive is not marked on the noun itself (unlike some more familiar derivational diminutives), rather it is realized by dedicated diminutive affixes that replace the usual feminine and masculine gender agreement markers. This makes the diminutive look like an additional gender value, but Dryer points to the lack of inherently diminutive nouns and the fact that the diminutive sometimes co-occurs with masculine/feminine agreement as good reasons for questioning its status as a gender value. Like other contributions to this book, Dryer’s discussion is a good illustration of how interactions between gender and other categories of grammar conspire to make gender systems (as well as the task of analyzing them) more complex.

\textbf{Bruno Olsson} shows that the complexity of gender can be addressed from a diachronic point of view by advanced methods of internal reconstruction in the case of a family in which all languages except one are so far poorly documented. The language investigated is \ili{Coastal Marind}, an \ili{Anim} language of the Trans-Fly area of South New Guinea. \ili{Coastal Marind} gender is covert except in a few nouns displaying stem-internal vowel alternation (\textit{anem} ‘man [I \textsc{sg}]’, \textit{anum} ‘woman [II \textsc{sg}]’, \textit{anim} ‘people [I/II \textsc{pl}]’). Olsson endorses earlier comparative research arguing that vowel alternation within \ili{Anim} words derives from umlaut triggered by postposed articles inflecting for gender (as they still exist in the perhaps distantly related and areally not too remote \ili{Ok} languages). By means of statistical analysis, he identifies traces of umlaut for two classes even in non-alternating nouns. The lack of any statistical effect in a third class is explained by class shift of nouns for animals. In \ili{Coastal Marind}, gender and number are intricately intertwined in an unexpected way. The joint plural of the two animate classes behaves almost identically to gender IV, one of the two inanimate classes (which do not distinguish number). Olsson speculates that gender IV might have originated from pluralia tantum, but since there is no longer a semantic link (no inanimate plural), it is not possible to view gender IV as plural synchronically, despite systematic syncretism with the animate plural throughout a large number of different formal exponents, including stem suppletion. The case of \ili{Coastal Marind} thus demonstrates that a gender system can become more complex through very specific kinds of interaction with phonology on the one hand and with number on the other.

In the traditional literature on gender, not all continents are equally well represented. New Guinea is a major area that has been notoriously underrepresented so far. \textbf{Erik Svärd} investigates gender in New Guinea in an areally restricted variety sample of twenty languages and compares it to gender in Africa and beyond. Unlike Africa, where gender is amply represented in the large language families, the two large families in New Guinea, \ili{Austronesian} and \ili{Trans-New Guinea}, mostly lack gender, unlike many small language families and isolates in which gender is attested. As a consequence, gender in New Guinea is diverse and more akin to the global profile of gender in comparison with Africa. Despite the diversity of gender in New Guinea, Svärd is able to identify characteristic properties of gender in New Guinea. Most languages with gender have a masculine--feminine opposition (where either member can be unmarked), and several gender targets, typically including verbs. Unlike Africa and the Old World in general, formal assignment and overt marking of gender on nouns is rare in New Guinea and, in the few languages having formal assignment, it is usually limited to a subset of the gender classes. However, gender assignment in New Guinea is not typically simple, since many languages have what Svärd calls ``opaque assignment'', which does not mean lack of assignment patterns, but rather that exceptions abound. The relevance of size and shape, the existence of multiple noun class systems, and lack of gender in pronouns are further properties characteristic of many languages of New Guinea with gender. Svärd’s comparison of New Guinea and Africa concludes the part on languages in Africa and New Guinea.

In Part~\ptsouthasia{} of Volume~I, \textbf{Henrik Liljegren} investigates the properties of gender systems and their complexity in 25 of 28 \ili{Hindu Kush Indo-Aryan} languages. The languages under study are those for which there is enough data in published sources and/or the author's field data, and are examined against the background of other languages spoken in the area, namely other \ili{Indo-Aryan}, \ili{Nuristani}, \ili{Iranian}, \ili{Tibeto-Burman}, \ili{Turkic} and \ili{Burushaski}. The result is a cross-linguistic survey, which is an intra-genealogical, areal and micro-typological study in one. Despite the close genealogical relationship between the \ili{Hindu Kush Indo-Aryan} languages, their gender systems are remarkably diverse, ranging from languages with the inherited masculine--feminine distinction pervasively marked on many agreement targets in the southwest (for instance, in \ili{Kashmiri}) to the \ili{Chitral} languages \ili{Kalasha} and \ili{Khowar} in the northwest, which instead have an innovated copula-based animacy distinction. These two languages also reflect the earliest northward migration of Indo-Aryans in the region. In some languages in the southeast, the sex-based and animacy-based oppositions are combined in concurrent gender systems, as is the case in the \ili{Pashai} languages and \ili{Shumashti}, which yield the highest complexity scores among \ili{Hindu Kush Indo-Aryan} languages. Liljegren shows that the distribution of various kinds of gender systems has both genealogical and areal implications, with different \ili{Iranian} contact languages in the southeast and southwest yielding a variety of contact effects. Liljegren traces in detail how the entrenchment of gender in this language grouping gradually declines from the southeast to the northwest. Generally in \ili{Hindu Kush Indo-Aryan}, gender is stable only to the extent that related languages with inherited gender are neighbors. But there are also language-internal factors. The functional load of gender is higher in languages with ergative rather than accusative verbal alignment.

\subsubsection*{Volume~II: World-wide comparative studies}
After having introduced all chapters of Volume~I, we now turn to Volume~II.
To date, the study of gender complexity has largely focused on synchrony. \textbf{Francesca Di Garbo} and \textbf{Matti Miestamo} demonstrate that diachrony is indispensable for a deeper understanding of the relationship between gender and complexity. They investigate four types of diachronic changes affecting gender systems \textendash{} reduction, loss, expansion and emergence \textendash{} in fifteen sets of closely related languages (36 languages in total) from various families and continents. In exploring how the detected types of changes relate to complexity, they find that reduction of gender agreement does not necessarily entail reduction of complexity. Rather complexity can increase both in reducing and emerging gender systems. Across the languages of the sample, there are strong regularities in how different kinds of changes are mapped onto the Agreement Hierarchy. The two opposite poles of the hierarchy, attributive modifiers and personal pronouns, can often be identified as the places of origin for both the decline and rise of gender. Di Garbo and Miestamo argue that two opposite forces, syntactic cohesion and semantic agreement, are at work at the two different poles of the implicational hierarchy. In a similar vein, the two different processes involved in reduction \textendash{} morphophonological erosion and redistribution of agreement \textendash{} display different directions of change along the Agreement Hierarchy. Di Garbo and Miestamo consider various cases of language-internal rise of gender and contact-induced gender emergence, and detect striking similarities. The cases under consideration suggest that gender in the process of emergence is non-pervasive and constrained. While gender can disseminate by means of borrowing of lexical items, emergent gender systems in borrowing languages differ in structure from gender systems in donor languages.

Traditional definitions of grammatical gender rely on the notions of noun class, agreement and system. \textbf{Bernhard Wälchli} demonstrates that dispensing with these notions and pursuing a radically functional approach to the study of grammatical gender is possible and worthwhile. The chapter is a typological investigation of feminine anaphoric gender grams (as in \ili{English} \textit{she/her}) in a world-wide convenience sample of 816 languages, based on a corpus of parallel texts (the New Testament). The functional equivalence between the forms extracted from the corpus is ensured by the fact that they cover a single search space across all languages considered. Through this methodology, which is applied to the domain of grammatical gender for the first time, the study finds instances of simple patterns of gender marking in a large number of languages for which no such constructions had been documented before. Three types of simple gender are extracted from the corpus and analyzed in the paper: non-compositional complex noun phrases, reduced nominal anaphors and general nouns. These instances of simple gender are interpreted as incipient types of gender systems from a grammaticalization perspective. Conversely, cumulation with case in the encoding of grammatical relations is taken as a characteristic feature of complex and mature (i.e.\ highly grammaticalized) feminine anaphoric gender grams. After discussing the differences between simple and mature gender, the chapter concludes by proposing a functional network for the grammatical gender domain in which the gram approach is reconciled with more traditional approaches based on the notions of noun classes, agreement and system.

\largerpage
While languages can have both gender and classifier systems, the co-occur\-rence of the two is rare. This suggests that these two different types of nominal classification systems may actually be in complementary distribution with one another. \textbf{Kaius Sinnemäki} validates this claim statistically by investigating the distribution of gender and numeral classifier systems in a stratified sample of 360 languages. Complexity is operationalized as the overt coding of a given pattern in a given language and thus, in this case, as the presence of gender and/or numeral classifiers. The study’s main hypothesis is that there is an inverse relationship between presence of gender and presence of numeral classifiers. The hypothesis is tested using generalized mixed effect models, which also control for the impact of genealogical and areal relationships between languages on the distribution of the variables of interest. The results reveal a statistically significant inverse relationship between presence of gender and presence of numeral classifier systems and that in addition the two types of nominal classification systems have a roughly complementary areal distribution. Languages spoken within the Circum-Pacific region are more likely to have numeral classifiers than languages spoken outside this area, whereas the opposite distribution applies to gender. This inverse relationship also exists independently of language family and area and thus confirms the study’s main hypothesis. According to Sinnemäki, these results, which should be interpreted as a probabilistic rather than an absolute universal, suggest that there is a functionally motivated complexity trade-off between gender and numeral classifiers, whereby languages tend to avoid developing and maintaining more than one system at a time within the functional domain of nominal classification.

The concluding chapter, by \textbf{Bernhard Wälchli} and \textbf{Francesca Di Garbo}, pre\-sents a wide-ranging enquiry into the diachrony and complexity of gender systems, with an emphasis on gender systems as dynamic entities evolving over time. The authors re-examine a variety of phenomena that will be familiar to students of gender, such as gender and the animacy hierarchy, assignment rules, gender agreement, and cumulative expression with other inflectional categories. But casting the net wider, the chapter also examines various issues that have received less attention in the literature, and which arguably are crucial for understanding the origin, development and synchronic characteristics of gender systems. These include the introduction of inanimate nouns into sex-based gender classes, opaque assignment and the development from semantic to phonological assignment, nouns \textendash{} and clauses \textendash{} as targets of gender agreement, and relationships between controller and target that go beyond co-reference and syntactic dependency. Among the 12 sections of the chapter (all of which can be read independently), we also find an exploratory survey of accumulation of nominal marking in the NP (including markers that fall outside the realm of noun classification, such as \textit{one} in the NP \textit{the red one}), and a proposal for a definition of agreement that is intended to capture the fundamental asymmetry between controller and target (as the sites where gender originates and is realized respectively). These and other sections of the chapter question the solidity of some commonly made distinctions, such as that between agreement features and conditions on agreement, or the binary splits between e.g.\ semantic and formal assignment systems, or the assumption that the category of gender can always be distinguished from that of number. These emerge in a new guise once the dynamic perspective favored by the authors is adopted.

\subsection*{Acknowledgments}

The two volumes are the result of a collaborative endeavor in which not only the editors and authors of the chapters were involved. We would like to thank in particular Yvonne Agbetsoamedo, Jenny Audring, Lea Brown, Greville Corbett, Östen Dahl, Michael Daniel, Deborah Edwards-Fumey, Sebastian Fedden, Jeff Good, Pernilla Hallonsten Halling, Martin Haspelmath, Robert Hepburn-Gray, Dan Ke, Marcin Kilarski, Matti Miestamo, Manuel Otero, Robert Östling, Frank Seifart, Ruth Singer, Krzysztof Stroński, Anna Maria Thornton and 9 anonymous reviewers, whose comments have contributed to considerable improvement of all chapters. We would also like to thank the editorial board of the series Studies in Diversity Linguistics for having supported the book project from its very beginning. In particular, the words of encouragement of the series’ editor, Martin Haspelmath, have been very important from the outset to the final stages of the volumes’ production. We would also like to thank the Language Science Press team, and in particular Sebastian Nordhoff and Felix Kopecky, for their eagerness in supporting us in all matters of producing open access books. As mentioned earlier, the book project started with the workshop ``Grammatical gender and linguistic complexity'' which was held in Stockholm, November 20th--21st, 2015. We thank the Department of Linguistics at Stockholm University for hosting the workshop, as well as the workshop participants and the audience for contributing to a very inspiring and stimulating discussion venue. The workshop was ultimately made possible thanks to the vice-chancellor of Stockholm University, Astrid Söderbergh Widding, who, back in 2015, initiated a funding program for collaborative linguistics research between Stockholm University and the University of Helsinki. This funding scheme has since then sparked numerous collaborative projects between the two universities, under the coordination of its scientific committee, directed by Camilla Bardel. The Stockholm-Helsinki cooperation program has through the years led to a large number of joint publications of which the present two volumes are just one example.

{\sloppy
\printbibliography[heading=subbibliography,notkeyword=this]
}
\end{document}
