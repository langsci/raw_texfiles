\documentclass[english,output=paper,colorlinks,citecolor=brown]{../langscibook} 
\author{David Moore\affiliation{University of Western Australia}\orcid{}} 


\title{Crosscurrents in linguistic research: Humanism and positivism in Central Australia 1890--1910}
\shorttitlerunninghead{Crosscurrents in linguistic research}

\abstract{Trained in nineteenth century humanist traditions of philology, German \is{Lutherian missionaries}Lutheran missionaries conducted linguistic fieldwork in the Dieri (Diyari) language near Lake Eyre in South Australia and in the Aranda (Arrarnta, Arrernte) and Loritja (Luritja) languages at Hermannsburg in the Northern Territory. As the discipline became increasingly positivist in the late nineteenth century, anthropologists and linguists with this very different orientation also took an interest in the languages of Central Australia. In this paper I contrast humanist and positivist researchers of Central Australian languages arguing that common \is{metascientific orientations}metascientific orientations are more significant factors than nationality for understanding their research.}

\IfFileExists{../localcommands.tex}{
  \usepackage{langsci-optional}
\usepackage{langsci-gb4e}
\usepackage{langsci-lgr}

\usepackage{listings}
\lstset{basicstyle=\ttfamily,tabsize=2,breaklines=true}

%added by author
% \usepackage{tipa}
\usepackage{multirow}
\graphicspath{{figures/}}
\usepackage{langsci-branding}

  
\newcommand{\sent}{\enumsentence}
\newcommand{\sents}{\eenumsentence}
\let\citeasnoun\citet

\renewcommand{\lsCoverTitleFont}[1]{\sffamily\addfontfeatures{Scale=MatchUppercase}\fontsize{44pt}{16mm}\selectfont #1}
  
  %% hyphenation points for line breaks
%% Normally, automatic hyphenation in LaTeX is very good
%% If a word is mis-hyphenated, add it to this file
%%
%% add information to TeX file before \begin{document} with:
%% %% hyphenation points for line breaks
%% Normally, automatic hyphenation in LaTeX is very good
%% If a word is mis-hyphenated, add it to this file
%%
%% add information to TeX file before \begin{document} with:
%% %% hyphenation points for line breaks
%% Normally, automatic hyphenation in LaTeX is very good
%% If a word is mis-hyphenated, add it to this file
%%
%% add information to TeX file before \begin{document} with:
%% \include{localhyphenation}
\hyphenation{
affri-ca-te
affri-ca-tes
an-no-tated
com-ple-ments
com-po-si-tio-na-li-ty
non-com-po-si-tio-na-li-ty
Gon-zá-lez
out-side
Ri-chárd
se-man-tics
STREU-SLE
Tie-de-mann
}
\hyphenation{
affri-ca-te
affri-ca-tes
an-no-tated
com-ple-ments
com-po-si-tio-na-li-ty
non-com-po-si-tio-na-li-ty
Gon-zá-lez
out-side
Ri-chárd
se-man-tics
STREU-SLE
Tie-de-mann
}
\hyphenation{
affri-ca-te
affri-ca-tes
an-no-tated
com-ple-ments
com-po-si-tio-na-li-ty
non-com-po-si-tio-na-li-ty
Gon-zá-lez
out-side
Ri-chárd
se-man-tics
STREU-SLE
Tie-de-mann
}
  \bibliography{../localbibliography}
  \togglepaper[9]%%chapternumber
}{}



\begin{document}
\maketitle

\section{Introduction} 

 The period 1890--1910 saw the publication of the first comprehensive grammars and dictionaries of Central \il{Australian Aboriginal languages} Australian Aboriginal languages. \is{Lutheran missionaries}Lutheran missionaries from the Hermannsburg Missionary Institute in present-day Lower Saxony in Germany conducted linguistic fieldwork in the Dieri (Diyari) language near Lake Eyre in South Australia from 1866. After the Hermannsburg Mission in the Northern Territory was established in 1877 Hermann Kempe (1844--1928) researched the Aranda (Arrarnta, Arrernte) language. The Hermannsburg missionaries left Central Australia in 1891. Carl Strehlow (1871--1922) arrived in 1894 after training at the Neuendettelsau Mission Institute in Franconia, Germany in which humanist philology played a greater role than it had at Hermannsburg.
 
 The missionaries were not alone. During 1907 and 1908 five descriptions of Central Australian languages were published. In tracing the history of German anthropology \citet[51]{Kenny2013} claims that “Strehlow had little contact with his British-Australian contemporaries”. Discussing the antagonism between the English biologist and anthropologist Walter Baldwin Spencer and Strehlow, Kenny does not discuss like-minded English-speaking researchers such as R.H. Mathews (\citealt{Mathews1907}; \citealt{Thomas2007}) and N.W. Thomas who collaborated with Strehlow. The “German \textit{fin} \textit{de} \textit{siècle} anthropological tradition that was language based” \citep[99]{Kenny2013} was not monolithic and there is a need to take account of the discontinuities in German Ethnology which are so evident on reading the German sources. Citing \citet{Monteath2013} and acknowledging that Antihumanists were also “well represented among the Germans”, \citet[228]{Kenny2013} also fails to acknowledge the antagonism of German Antihumanists towards Strehlow.
 
 Previously I argued that Strehlow was engaged in philology and cultural translation rather than a form of ‘ethnography’ which was a transitional stage to modern anthropology (\citealt[336]{MooreRíos-Castaño2018}). In dual roles of missionary and researcher, he translated for distinct purposes and audiences. His investigation of indigenous religion was limited by the attitudes of the church authorities which viewed it as ‘heathen’ (\citealt[338]{MooreRíos-Castaño2018}).  For a brief time (1906--1910) he was engaged in disinterested scholarship. He kept his research separate from his evangelical work \citep[232]{Brock2017}. Further, “he stopped his investigations on completion of his manuscript and the death of his patron” \citep[236]{Brock2017}, returning to Bible translation from 1913 to 1919 (\citealt[336]{MooreRíos-Castaño2018}).
 
 In this paper I am seeking to explain the differences between Strehlow and some of his contemporaries in terms of their attitudes to linguistic research. I characterise two contrasting kinds of research as ``humanist'' and ``positivist'' according to the influences of nationality, education and training \citep[94]{Errington2008}, arguing that \is{metascientific orientations}metascientific orientations and \is{language ideoligies}language ideologies (\citealt{MooreForthcoming}) more than nationality determined their approaches to linguistic research. An understanding of these factors is necessary for interpreting their linguistic descriptions and also for understanding the collaborations between some researchers in the field and the antagonisms between others.  
 
\section{Humanist and positivist paradigms in the early twentieth century}

The labels \is{humanism}`humanism' and \is{positivism}`positivism' refer to philosophies or epistemologies or means to finding “a method by which humans could be classified and known” \citep[186]{Zimmerman2001}, reflecting a division that existed in Germany since the \textit{Aufklärung} (Enlightenment), into \is{Naturwissenschaften}\textit{Naturwissenschaften} (natural sciences) and \is{Geisteswissenschaften}\textit{Geisteswissenschaften} (human sciences).\footnote{The \textit{Geisteswissenschaften} included both the humanities and social sciences, for which there was no clear division in the nineteenth century \citep[282]{Adams1998}} \is{Humanism}Humanism arose in the Renaissance as the study of the classical world \citep[172]{Giustiniani1985}.\footnote{Another term which contrasts with ``positivist'' is ``idealist'' \citep{Vossler1904} which applies to developments in German philosophy later than the Humanist origins in the sixteenth century.} Although this term refers to diverse branches of scholarship \citep[258]{Adams1998}, I focus upon the adoption of Humanism by the sixteenth century Lutheran Reformers, the rise of the \is{German philosophy of language}German philosophy of language and its \is{language ideologies}language ideologies and the extension of philology to indigenous languages (\citealt[328]{MooreRíos-Castaño2018}; \citealt{MooreForthcoming}). Humanist scholars privileged the study of language for understanding other societies \citep[53]{Zimmerman2001}, developing the methods of textual criticism, hermeneutics and translation to understand texts. 

Early anthropologists such as James George Frazer (1854--1941) were trained in Classics as the study of the ancient civilizations of Greece and Rome. A division of labour developed as the two disciplines diverged (\citealt{Marett1908}; \citealt{Kluckhohn1961}). The subject material of classical studies was ‘civilized peoples’, that is, those with a written literature while the subject material of anthropology was the ‘natural peoples’, those without written literatures.\footnote{Anthropology is the only discipline which is allied to the humanities, social sciences and natural sciences.} The division between \is{humanism}humanism and \is{positivism}positivism is a cline rather than rigid, reflecting the situation in German academic life in the first decade of the twentieth century. Within Germany the move from the humanist philological sciences to positivist sciences was underway about 1850 \citep[26]{Smith1991}, reflecting wider changes in society in the ‘age of positivism’ \citep[120]{Massin1996}. There are strong similarities between general linguists, missionary linguists and \is{moderate positivists}moderate positivists. That they corresponded about the study of Australian languages is evidence of this affinity, reflecting the persistance of \is{language ideologies}language ideologies from the German philosophy of language.
 
\section{Central Australian missionary linguistics as Humanist research}

\subsection{Kempe and Strehlow}

Missionaries compiled grammars and wordlists for the purposes of biblical translation from German into Aboriginal languages (\citealt{MooreRíos-Castaño2018}). Initially, they sought words in Indigenous languages as translations of key theological terms in order to translate the Catechism and later, the Bible into Aboriginal languages. Kempe published a grammar and wordlist of ``the language of the Macdonnell Ranges'' \citep{Kempe1891}. His treatment of the language was tentative: ``the following pages, therefore, do not profess to contain \textit{a} complete vocabulary, nor one which would satisfy the learned philologist'' \citep[1]{Kempe1891}. The Neuendettelsau curriculum was based upon philology and Lutheran theology with the purpose of enabling the mission candidates to translate and interpret biblical texts. Strehlow's training replicated the ``classical orientation'' \citep[83]{Kenny2013} in which proficiency in reading ancient languages enabled scholars to understand the biblical and classical Greek and Roman worlds. Language training included instruction in Latin, Greek, Hebrew, English and German syntax and word formation, and prepared missionaries for translating languages (\citealt[8]{Völker2001}; \citealt[26]{Nobbs2005}). Strehlow revised Kempe’s earlier grammar, dictionary and hymnbook. Aranda became a language of interest to European scholarship and the need to obtain reliable data from the field prompted Freiherr Moritz von Leonhardi (1856--1910) to request information from Strehlow who extended his research to the collection of texts and translations of \textit{Die} \textit{Aranda-} \textit{und} \textit{Loritja-Stämme} \textit{in} \textit{Zentral-Australien} (\citealt{Strehlow1907}). 

For German \is{Lutheran missionaries}Lutheran missionaries the first step in understanding was to acquire Aboriginal languages. They learned the languages rapidly through social interaction. The Neuendettelsau-trained Lutheran missionary in North Queensland Wilhelm Poland said that ``[t]here was in fact only one way of learning the language, and that was to mix with the older generation in their camps'' \citep[103]{Poland1988}. The Aranda elder and Evangelist Moses Tjalkabota was ``surprised at the rapid rate of progress which Carl Strehlow made with the language'' through reading Kempe’s grammar and hymnbook \citep[65--66]{Latz2014}. The Aboriginal elders Loatjira, Pmala, Tjalkabota and Talku worked with Carl Strehlow on the compilation of \textit{Die} \textit{Aranda} \citep[29]{Kenny2013}. Talku (c.1867–1941) told him the Loritja myths for the 1908 and 1911 volumes of \textit{Die} \textit{Aranda.} Strehlow’s collections of Loritja texts, grammars and dictionary were the first comprehensive record of a Western Desert language.

Monolingual speakers of Aboriginal languages were often not able to understand the researcher’s questions. In the preface to his grammar, \citet[1]{Kempe1891} describes the problem for linguistic research as follows: 

\begin{quote}
    The result of an attempt to analyse a language of which the people speaking it have only a colloquial knowledge, and who are consequently incapable of answering or even understanding grammatical questions, must be in many respects imperfect. The difficulty is increased by the wandering habits of the people. 
\end{quote}

Kempe was aware that Europeans would be told what they wanted to hear because of the gratuitous concurrence which occurs when an informant agrees with the researcher from a desire to please the questioner \citep[198]{Liberman1985}. \citet[1]{Kempe1891} emphasised the need to check language statements thoroughly: “Concerning the vocabulary, it may be mentioned that it has been carefully compiled and revised several times with different natives, so that the words may be relied upon as correct.” 

Leonhardi discussed the difficulty of eliciting information:

\begin{quote}
    One should never develop his own view and then put the question, “Is this how it goes?” The question must be rather expressed, “What have your elders taught about the matter?” Then some blacks are smart enough, to find the answer. In this way one can go back and check, whether it is correct. \citep[286]{Leonhardi1907}
\end{quote}

In his time at Bethesda from 1892 to 1894, Carl Strehlow, with J.G. Reuther, evaluated Dieri (Diyari) terms which would be useful for the Dieri New Testament translation and gained experience in translation, building on the work of Hermannsburg trained missionaries who had preceded Johann Flierl \citep{Kneebone2001}. Strehlow became aware that Aboriginal languages are very different from European languages. For example, \textit{neji} in Diyari cannot be directly translated as ‘brother’ (J \citealt[83]{Strehlow2011} 83), as Aboriginal languages have separate words for ‘younger brother’ and ‘older brother’. He researched kinship with the ethnographic researcher Francis Gillen \citep{MulvaneyEtAl1997}. Leonhardi’s questions reflected such contemporary interests of European scholars as totemism, initiation rites and kinship, views of conception and ceremonial objects or \textit{Tjurunga} \citep{Schmidt1908}. 

It was the systematizing and generalizing by Spencer and Gillen that led Strehlow to record the particular and local to find out what Aranda and Loritja said in their own words. Strehlow recorded texts in order to understand Aboriginal culture. In criticising Spencer and Gillen, Leonhardi wrote to Strehlow:

\begin{quote}
    The big mistake of the books by these two researchers, it seems to me, is the fact that they systematise too much, that they try too hard to show universal views in a large area, while there should be no more than individual legends, local views and customs etc. and not a closed well-ordered system of custom. Only by providing individual stories and customs is it possible to bring out, through comparison, general aspects. (VL 1904-1-2, 28/8/04).
\end{quote}

This ‘emic’ approach was later identified by Kenneth Lee Pike as “studying behaviour from inside the system” (\citealt[37]{Pike1967}; \citealt[542]{Bolinger1975}), rather than taking an external perspective. Leonhardi’s \textit{Linguistische} \textit{Feststellungen} (linguistic findings) are the interlinear texts and the free translations of the myths and the songs of \textit{Die} \textit{Aranda}. Copious footnotes included translations and explanations of words which appear in the texts. The texts which were recorded only in German translation were apparently regarded as of less evidentiary value. Strehlow was also working on a grammar and comparative Aranda-German-Loritja dictionary which would help the reader to understand the \textit{Urtext.} 

The importance of the \textit{Urtext} can be understood from the comment of Leonhardi’s editorial successor \citet[285]{Hagen1991}: “It is of some importance to know that the most important matter, the focal point so to speak, viz. the intellectual culture of the Aranda and Loritja, are in the main secured.” The uncompleted sections of \textit{Die} \textit{Aranda} deal mainly with material culture. Most critical for humanist research was to record what ‘the Other’ said in their own words.

\subsection{Collaboration with philologists in linguistic research} 

Some German scholars were interested in language classification and typology, particularly the ‘general linguists’ of the Humboldtian school. However, contemporary comparative philology in Germany was narrowly focused upon the Indo-European languages:

\begin{quote}
    von der aufblühende historische-vergleichenden Sprachforschung wurde die typologische Sprachwissenschaft im Sinne Humboldts ziemlich in der Hintergrund gedrängt. 
    
    the flourishing historical-comparative language research pushed typological linguistics of the Humboldtian school somewhat into the background \citep[216]{Deeters1937}.  
\end{quote}

Although \citet[1]{Kempe1891} claimed that his grammar and wordlist were submitted ``in the hope that they would prove interesting to the philologist'', he doesn’t appear to have consulted with philologists. Contact between Strehlow and a general linguist was facilitated by Leonhardi, who could see the benefit of making Strehlow’s research available to European scholars. Franz Nikolaus Finck, a Professor of linguistics at the University of Berlin, provided comments on Carl Strehlow’s texts which Leonhardi sent him. In a letter to Strehlow Leonhardi writes: 

\begin{quote}
    I would like you to know that Prof. Finck in Berlin, to whom I had sent the ‘Aranda Legends’ has in the last few days expressed high praise for the Aranda texts in a letter to me. Since Prof. Finck is a first-rate authority on Austr. Oceanic languages and I had sent him your essay as well, as you know, I am very pleased about this recognition.  (VS 1908-1-1).\footnote{The correspondence from Moritz von Leonhardi to Carl Strehlow is held in the archives of the Strehlow Research Centre, Alice Springs.}
\end{quote}

\section{Positivism} 

Positivists based their research on the natural sciences. Their favourable valuation of the natural sciences followed contemporary trends and, by the late nineteenth century, led to a ``sense that scientific discourse was more correct than others'' \citep[154]{Crick1976}. For some positivists language was typically one category of human behavior among many behaviors that could be described. The view that visual observation provided the only reliable evidence about the ‘Other’ meant that ‘fieldwork’ became an essential practice within anthropology and began to replace ‘armchair’ scholarship around the turn of the century. Prominent in the development of fieldwork was Adolf Bastian (1826--1905), co-founder the \textit{Gesellschaft} \textit{für} \textit{Anthropologie,} \textit{Ethnologie} \textit{und} \textit{Urgeschichte} (Society for Anthropology, Ethnology and Prehistory) with Rudolf Virchow (1821--1902) and the first director of the Museum für Völkerkunde in Berlin in 1873 \citep{Köpping1983}. \textit{Völkerkunde} ``classified and generalized the results of a strictly descriptive ethnography'' (\citealt[19--25]{BuchheitKöpping2001}, cited in \citealt[87]{Gingrich2005}).\footnote{\textit{Völkerkunde} can be translated as `cultural anthropology'. \textit{Anthropologie} is translated as `physical anthropology' \citep[82]{Massin1996} and is therefore a false friend with current English `anthropology'.} Bastian typified the positivist view, attacking interpretation, history and literature as unreliable ways of understanding the ‘Other’ \citep[61]{Zimmerman2001}. However, he retained a respect for the philological tradition \citep[89]{Gingrich2005}. After his death in 1905, he was succeeded by his younger associates, whom \citet[91]{Gingrich2005} characterises as ``moderate positivists''. Positivists were not necessarily evolutionists and Bastian and others were opposed to evolutionism. In the following sections, the ‘moderate positivists’ \citep[99]{Gingrich2005} are contrasted with ``radical positivists'' or ``Antihumanists'' (\citealt{Zimmerman2001, Monteath2013}). 

\subsection{Moderate positivists}

The moderate positivists were overshadowed by the diffusionists in the first decade of the twentieth century in Germany but “remained as systematic fieldworkers and museum documentarists” \citep[92]{Gingrich2005}, closer to the international mainstream of anthropology and particularly close to the German-influenced linguistic anthropology that was emerging in the USA. Among the moderate positivists were Konrad Theodor Preuss (1869--1938) and Karl von den Steinen (1855--1929). \citet{Preuss1908, Preuss1909} reviewed Carl Strehlow’s \textit{Die Aranda} positively. 

\subsection{Antihumanists}

Antihumanists followed social evolutionary anthropology, which was the dominant paradigm in British anthropology by the turn of the twentieth century. Walter Baldwin Spencer (1860--1929) and Francis J. Gillen (1855--1912) may be characterized as Antihumanist. As Spencer admitted, ``my anthropological reading was practically confined to Sir Edward Tylor's `Culture' and Sir James Frazer's `Totemism'~'' \citep[184]{Spencer1928}. Spencer and Gillen followed Frazer’s lead and their monographs clearly show the influence of Frazer’s \textit{Golden} \textit{Bough} and the list of priorities for the collection of significant ethnographic ‘facts’ outlined in his short questionnaire \citep[45]{Urry1993}. Frazer separated particular facts from their cultural contexts and arranged them within a continuous discourse of evolutionary development. When Virchow’s ‘inductive positivism’ \citep[138]{Massin1996} was rejected, social evolutionism became more influential in Germany. I examine German Antihumanist researchers in the following sections. 

\subsubsection{Basedow}

Herbert Basedow (1881--1933) was a medical practitioner whose family had migrated from Berlin to South Australia in the 1850s. He trained in Breslau under Hermann Klaatsch (1863--1916), an anatomist and physical anthropologist who founded an institute of physical anthropology in Breslau in 1907 \citep[84]{Massin1996} and invited Basedow to study there in the same year (\citealt[ix]{Basedow1925}). Klaatsch was one of the first German physical anthropologists to adopt social evolutionary theory. In 1904 he travelled to Australia, visiting Melville Island, Tasmania and northwestern Australia \citep[423]{Oetteking1916}. He claimed that the Aborigines were “a relic of the oldest types of mankind” \citep[42]{McGregor1997}, based upon the anatomical comparison of Aboriginal people with the Neanderthals and other earlier humans. Australian languages were also primitive: ``The Australian dialects seem in many respects to be fragments of the primitive speech of man'' \citep[38]{Klaatsch1923}. Adopting the evolutionary view of his mentor, \citet[208]{Basedow1908} compared sounds made by speakers of Aboriginal languages with those made by apes: 

\begin{quote}
    Es ist von Interesse, dass Garner in seinem bekannten Werk über die Affensprache gefunden zu haben angibt, dass die von ihm beobachteten Affen denselben Laut „ng" besitzen und zwar im Zusammenhang mit dem Ausdruck der Zufriedenheit „ngkw-a". 

    It is of interest that Garner in his well known work on ape language, found that the apes observed by him use the same sound ‘ng’ in the context of an expression of satisfaction ‘ngk-wa’. \citep[208]{Basedow1908}.
\end{quote}

Basedow was a member of the South Australian Government North-West Prospecting Expedition, led by L.A. Wells. He expected his officers to learn Aboriginal languages and to “to treat the natives in a friendly and considerate, yet firm and masterly way” \citep[49]{Zogbaum2010}. Basedow collected “a vocabulary of about 1500 words of the Aluridja (Western Desert) and Aranda languages” (\citealt[v]{Harmstorf2004}). He admitted that he did not consult other sources and that his “article on language is not intended to be at all comprehensive” (\citealt[xii]{Basedow1925}). The short-term nature of Basedow’s trips were useful for compiling wordlists but not for learning to speak languages fluently. His wordlist \citep{Basedow1908} is rich in names for physical objects but not mental and religious aspects of culture. 

Although his evolutionary views are in strong contrast with Strehlow’s, Basedow appears to have been sympathetic to the \is{Lutheran missionaries}Lutheran missionaries and appreciative of their linguistic research (\citealt[vi]{Harmstorf2004}). He was a Lutheran and had strong connections to the Barossa Valley and South Australian Lutherans who supported the Hermannsburg Mission. He visited the mission station in 1919 and later wrote that with Strehlow’s death ``Science has lost and indefatigable and conscientious worker'' (\citealt[ix]{Basedow1925}). 

\subsubsection{Eylmann}

Erhard Eylmann (1860--1926) included two chapters about language in his study of Aboriginal people in Australia \citep{Eylmann1908}. \citet[34]{Monteath2013} characterises him as an ‘Antihumanist’ as he focused upon material culture rather than pursuing a humanist interest in language. He admitted the limitations of his understanding of Aboriginal languages:

\begin{quote}
    Über den Bau der Sprachen vermag ich keine nennenswerte Auskunft zu geben. Ich habe mich in Südaustralien nirgends solange aufgehalten, daß ich nach Erledigung der notwendigsten Arbeiten noch Sprachstudien treiben konnte. 

    Concerning the structure of the language, I cannot provide any great amount of information. I have not stayed anywhere in South Australia for a length of time that would have permitted me to pursue language studies after I had completed the work of the highest priority. \citep[81]{Eylmann1908}
\end{quote}

\citet[81]{Eylmann1908} admitted that he had difficulty eliciting a word which was equivalent to English ‘and’ from a speaker of the Awarai language. He became tired in a ``surprisingly short time''.

\section{Positivism in linguistic research}

August \citet{Schleicher1983}  first suggested that linguistics was a natural science, casting linguistics in terms of biological metaphors and created a ‘disciplinary matrix’ for a linguistics founded upon the natural sciences (see also \citealt{McElvenny2018}). Linguistics increasingly came under the influence of positivism in the late nineteenth century.

\subsection{Planert}

Wilhelm Planert (b. 1882) claimed to be ‘scientific’. In his inaugural dissertation at the University of Leipzig, \citet{Planert1907Suaheli} claimed: “In this treatise, for the first time, an attempt is made to correspond to the intentions of modern linguistics”. Planert was a student of Carl Meinhof (1857--1944), professor at the School of Oriental Studies in Berlin from 1905. Meinhof was involved in developing the Language Institutes (Seminars) as ‘Hypermetropolitan spaces’, laboratories where phonetic and linguistic information could be easily extracted from informants \citep[138]{Pugach2012}. Languages were recorded with phonographs, played back and ‘observed’. It was that “the new discipline of phonetics recast linguistics as a natural science, distancing it from humanistic philology by refocusing attention on bodies and the sounds they produced instead of written texts” \citep[93]{Pugach2012}. 

Planert’s usual method of operation at the Oriental Institutes was to interview language speakers who were visitors to Germany. He was limited to working in the metropole and the laboratory away from the context of language use. \citet{Planert1908} acknowledged in a response to Carl Strehlow’s criticisms of his Aranda Grammar that errors were made because of a lack of reliable informants and he was disparaging about the training of Missionary Nicolai Wettengel who was his informant for the Aranda Grammar \citep{Planert1907Aranda}.  Wettengel had worked at Hermannsburg in the Northern Territory of Australia from 1901--1906 (\citealt[1154]{Strehlow2011}) and gained some familiarity with the Aranda language. Planert worked with an informant who was not a native speaker of Aranda and who had a less than adequate grasp of the language. 

Languages were manipulated to serve colonial goals \citep[88]{Errington2008}. The nation required a “school-mediated, academy-supervised idiom codified for the requirements of reasonably precise bureaucratic and technological communication” \citep[57]{Gellner1983}. Planert’s PhD dissertation was published as \textit{Syntactic} \textit{relationships} \textit{in} \textit{Swahili} \citep{Planert1907Suaheli}. Swahili was a language of administration which the German colonists elaborated into a language of ‘civilization and progress’ to rule East Africa. Planert also described Bushma and Hottentot (\citeyear{Planert1905}), Nama (\citeyear{Planert1905}) and Jaunde \citep{Nekes1911}.\footnote{A language used in the German colony of Kamerun (Cameroon) and now written as ‘Ewondo’.} While the Germans were conducting a genocidal war against the Nama and Herero peoples in South West Africa \citep{Hull2005}, Planert was writing his grammar of the Nama language. He collaborated with the colonial authorities in language engineering and control, in contrast with the \is{moderate positivists}moderate positivists who had very little to do with colonialism or were even opposed to it. 

\section{Conclusions} 

Researchers from a wide variety of metascientific orientations attempted to understand the ‘Other’ through their languages. 

Kenny’s assertion that the ``humanism of German anthropology with its pluralistic outlook and anti-evolutionistic position lasted nearly to the eve of World War I'' fails to explain Klaatsch’s evolutionary anthropology. The views of Basedow, Eylmann and Planert reveal the degree to which Antihumanism had, in fact, become established and dominant in German Ethnology. There was more in common between researchers of different nationalities who shared a similar orientation than those of the same nationality who had different philosophical orientations.  It would be most accurate to say that Strehlow had little sympathy for those contemporaries who saw research in a very different way, anthropologists whose primary metascientific orientation was Antihumanist, including German Antihumanists. 

The critical difference between the missionaries and the Antihumanists in the Central Australian field was that the missionaries could understand the ‘Other’ through the strong focus upon language of their humanist training. Antihumanist interpretations were often hampered by literalism and misunderstandings. Although claiming to be ‘scientific’ and objective, they were often biased through their support for pre-existing theories and their affiliations to colonial forces. Significantly, missionary research filled in gaps in the knowledge of Central Australian languages at a time when neither anthropology nor comparative philology took an interest in the languages of Australia. Further research on these rich sources is needed to understand missionary research, \is{language ideologies}language ideologies and experiences of fieldwork. 


% % \section*{Acknowledgements}


{\sloppy\printbibliography[heading=subbibliography,notkeyword=this]}
\end{document}
