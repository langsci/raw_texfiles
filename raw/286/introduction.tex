\documentclass[output=paper]{langsci/langscibook} 
\title{Introduction} 
\author{Emilie Aussant\affiliation{Histoire des Théories Linguistiques (CNRS/Université de Paris)}\and Jean-Michal Fortis\affiliation{Histoire des Théories Linguistiques (CNRS/Université de Paris)}}  

\abstract{\noabstract}

\begin{document}
\maketitle

The present volume contains a selection of twelve papers delivered during the 14th international conference on the history of the language sciences (\textsc{ICHoLS XIV}), which took place at Université Sorbonne Nouvelle in Paris, from August 28 to September 1, 2017.

It is divided thematically into two parts: (I) \emph{Metalinguistic concepts and representations}, and (II) \emph{Fields, authors and disciplinary commitments}. Within each part, papers are arranged chronologically. There is of course a certain artificiality in this two-part division, given the variety of the topics and perspectives illustrated in the papers. Let us say that the first part is more concerned with descriptive concepts and the second with case studies involving specific fields and authors.

The first part begins with a paper by Mazhuga, who sets out to explore the origin of the technical use of the terms \emph{accidere\slash accidens} in Latin grammar. Is this origin Greek? Is it to be sought in the preceding ``technicization'' of what would be the Greek counterparts of \emph{accidens}, namely \textit{συμβεβηκός} and \textit{παρεπόμενον} in their grammatical usage? Mazhuga thinks not, and shows that the Latin and Greek terminologies have followed their own idiosyncratic courses of development. This leads him to discuss the Greek terms in some detail, before coming to the ultimate goal of the paper, which is to clarify the Latin terms. In this respect, it is of particular interest to mention the hypothesis he puts forth: \emph{accidentia}, he claims, was used in rhetoric to designate the qualities pertaining to a particular case (causa). It is from there that it gained its technical use in grammar.

Fortis' contribution (delivered as a plenary lecture, hence its greater length) does not bear on a linguistic category per se, but rather on a philosophically loaded perspective which has been defended throughout the history of Western linguistics. This perspective corresponds to a family of descriptions which can be conveniently brought together under the label of ``localism'', a name that first became established only during the 19th century, although the basic idea behind it is much older and can be traced back to Aristotle. Localism traditionally argues for the primacy of spatial relations in the semantics of particular categories, typically prepositions and cases. Remarkably, ideas reviving this tradition surged again in the second half of the 20th century, for reasons explored in this paper. After a historical retrospective and a presentation of the recent context, two main neolocalist lines of investigation are examined: localist descriptions in lexical-grammatical semantics, and localist theories of thematic roles.

The next paper cross-cuts and complements the preceding one. In her study, Chalozin-Dovrat discusses the status of space in linguistic discourse, scrutinizing three theories she samples in view of the important role they confer on spatializing descriptions of linguistic time. The theories in question are due to Beauzée, Guillaume and the strand she identifies as the ``conceptual school of cognitive linguistics'' (comprising Lakoff, Langacker, Talmy and Traugott). She suggests that spatializing descriptions, aside from their potential in modelling linguistic facts, also have a function of legitimation: by resorting to a fundamental category of Western science, space, they contribute to providing linguistics with scientific credentials. This strategy is what is designated as ``scientification'' in the paper.

It is a fortunate circumstance that the next chapter by Mazziotta provides yet another way to envisage the relation of linguistics to spatialization, in the guise, this time, of diagrammatic representations of grammatical structure. These appear, as the author claims, at a juncture when linguistics increasingly turns from a word- to a syntax-centered perspective. The focus is on early diagrams, especially those proposed by Clark, Reed and Kellogg around the mid-19th century. As can be expected, diagrams reflect pre-conceived grammatical analyses. But the graphical expedients employed to capture grammatical structure have their own logic, which may interfere with that of the analysis. These expedients are therefore not theoretically neutral, as the author shows, for instance, in the treatment of subordination.

If the subject of spatial relations is of timely importance, the same can be said of the question addressed in the following chapter: the origin of the term ``polysemy'', a designation which sums up a host of issues actively debated in recent times as a result of the renewed prominence of lexical semantics in linguistic scholarship. In his study, Courbon shows that, contrary to common belief, the term was not coined by Bréal but by Joseph Halévy (an orientalist), who applied it to cuneiform signs. The wide acceptance of the term and the fact that Bréal was credited with its invention show two things: that the context was favorable to the newly created \emph{Sémantique} (in part through lexicography), and that Bréal was in a better position to prescribe its use. Here, social-scientific factors come into play: we may say that Bréal’s advantage lay in his institutional position and the important role he played in the rise of modern general linguistics in France.

The relevance of polysemy for contemporary semantics and even syntax (if we think of construction grammar) is more than matched by the topic of the next paper. In his survey of the state of the art, Christy touches on the status of prefabs, idioms, phraseology etc. (``fixed expressions'') in today's linguistics, with an eye toward Saussure and Bally. Again, we see this issue's rise to prominence in the past decades, partly, we may say, to fill up a gap left by generative linguistics, and partly because the usage-based approach inherited from diachronic linguistics joined forces with cognitive and functional linguistics. The contemporary landscape is the focus of Christy's paper, which maps recent research and gives us an overview of the criterial features of fixed expressions, of the roles and functions assigned to them by different authors, and of the new tools available to data collection and analysis. As transpires from Christy's discussion, the importance of fixed expressions culminates in usage-based approaches. In these approaches, formulaic structures form the basis for creatively assembled sequences of any degree of regularity above non-productive, non-compositional idioms (which in this respect are a limiting case).

In the second part of this volume, we have brought together case studies which bear on a specific field or author. In all cases, these fields and authors are outlined against the backdrop of overarching views or broader concerns related to sociolinguistic, philosophical, ideological, or pedagogical issues, or to the circumscription of the disciplinary boundaries of linguistics.

In the opening chapter by Li, the overarching view in question is of a socio-historical nature. In the perspective advocated in this paper, data from a standard language are no longer prioritized and must give way to productions obtained from situations of communicative immediacy. An approach of this kind, shows Li, is especially fitting for the subject she considers, that is, Chinese Pidgin English. Since the socio-historical angle lays much importance on concrete speech situations, Li takes due account of the speech act participants, of the variety of their social roles, and highlights the ensuing variation of their linguistic productions. In addition, she offers a descriptive analysis of the Chinese Pidgin English preposition \emph{long}, and argues that syntactic-semantic variations in its usage are bound up with varying degrees of acculturation on the part of speakers. Of note here is an interplay with the Cantonese substrate: meanings of \emph{long} more specific to the Cantonese counterpart tend to inherit their syntactic behavior from Cantonese too. 

The following chapter is a contribution to the history of school grammar and the evolution of language pedagogy. Lhomond, who is the grammarian chosen by Piron in her case study, plays a very special part in this history. For Lhomond's \emph{Elémens de grammaire latine}, in addition to having achieved a huge editorial success, have been held up as the first truly ``modern'' Latin textbook, in the sense that it was to a large extent adapted to its audience of French-speaking learners. Since Latin grammaticography had always been essential for the description of vernacular languages, and since, in return, Lhomond's French grammar was intended as a kind of propedeutic introduction to Latin, it seemed all the more apposite to have a closer look at this reciprocal relationship. What Piron demonstrates is that, while still of course dependent on the Latin model, Lhomond decidedly takes a ``delatinizing'' turn. As an example, although Latin cases are treated as functionally equivalent to French \emph{de} and \emph{à} when government is at issue, cases are no longer employed for presenting the paradigm of French nouns. To be noted too, are such innovations as the separate treatment of adjectives as a part of speech distinct from substantives, tabular expositions of paradigms, and a French propaedeutic taking its departure from the word, that is, dealing with morphology and spelling, agreement and government as features of the word.

School grammar is of course endowed with a social role, that of codifying, standardizing and transmitting. The stakes of linguistic description are very high too when it is in addition subordinated to an enterprise of acculturation like religious conversion. This does not mean that missionary linguistics is necessarily more prejudiced or ill-intentioned than supposedly cool science. In his chapter on German missionaries and scholars in Australia, Moore shows precisely this: missionaries trained in philology were more intimately in touch with the local cultures and languages, and appeared to be more empathetic, than anthropologists and linguists whose secular religions were positivism and evolutionism. Thus, Strehlow, a missionary, professed an ``idiographic'' orientation, defiant of the sweeping claims of evolutionary and physical anthropology. On the other hand, the matter-of-fact approach of anthropologists and linguists could be conducive to a distant vision which, by its lack of empathy, ran the risk of being more susceptible to prejudices, value judgment and collaboration with colonial power. This was indeed the case for the strand of research which Moore characterizes as ``antihumanist''.

The chapter by Bergounioux is concerned not so much with the latter kind of rival epistemic orientations as with the constitution of a field, the study of regional dialects in France. However, broader epistemic issues are not absent, for the field was largely shaped by new methods that differentiated themselves both from philology and historical linguistics, and thus favored the emergence of linguistic subdisciplines. These methods were promoted by two important figures of French linguistics, namely, Gilliéron, for linguistic geography, and Rousselot, for experimental phonetics. Both scholars felt the need for a journal that would enable them to deploy their investigation in the field of dialectology, a project which was concretized in the \emph{Revue des Patois Gallo-Romans}. Each brought to the task his own technique and interests, and his own network of local researchers. However, because Rousselot's approach was in the direction of extremely fine-grained phonetic analysis — and as such rather unwieldy — and because descriptions were narrowly focused on very local varieties, the net result tended to be an ``atomization'' (Bergounioux' word) of the dialectal reality. Moreover, dialectology was at variance with the linguistic unification then actively implemented by the French republic, and with the dim view that was taken of so-called \emph{patois} in a large part of French society. In the end, facing a relative indifference, the journal and other publications with a similar purview were unable to perpetuate themselves.

In the case of linguistic geography and experimental phonetics, new methods help form subdisciplines and bring into focus specific aspects of linguistic and social reality. In the chapter by Frigeni, we see that a discipline already in existence and with a well-established methodology may be partly reshaped by a new perspective. In historical semantics, Meillet, it is recalled, laid a new emphasis on social and cultural matters. But was this perspective idiosyncratic enough to be regarded as a specific way of practicing historical semantics? Frigeni's answer is positive, and she argues that Meillet's perspective lived on in Benveniste's work, even if he made no explicit reference to Meillet in this connection. Evidence for a legacy, or at least a common orientation, is furnished by her case study, in which she examines Benveniste's semantic analysis of Indo-Iranian \emph{Mit(h)ra} in the light of Meillet's initial attempt. Both scholars, it is shown, conduct their etymological inquiry by expanding it into an inter-cultural comparison (between the Indian and the Iranian contexts) in which social reality takes on an important role. This concern for a wider cultural context leads them both to hypothesize for \emph{Mit(h)ra} a meaning with a social import, related to the notion of social pact or contract for Meillet, and to a function protective of the community for Benveniste.

The following chapter on Trần Đức Thảo by D'Alonzo illustrates yet another way of counteracting linguistic abstraction by bringing it in closer contact with the social reality of semiotic systems. In the eyes of the phenomenologist and materialist philosopher Trần Đức Thảo, the social and cognitive determinants of signs furnish the material conditions of human semiotic life. These conditions, he claims, undermine the view of sign-systems as made up of arbitrary elements with a differential value, in other words, the view commonly attributed to Saussure. Trần Đức Thảo's insistence on the motivation of signs is associated with the consideration of their antepredicative genesis (in particular in the form of indicative gestures). This would not be highly original if Husserlian phenomenology, to which this antepredicative layer refers back, were not encompassed in the same critique as Saussurian structuralism, and both characterized as idealistic. In other words, Trần Đức Thảo's critique reflects an original perspective which enrolls Marxism and genetic psychology into a reorientation of phenomenology, and in turn marshals this theoretical complex against Saussure.

{\sloppy\printbibliography[heading=subbibliography,notkeyword=this]}
\end{document}
