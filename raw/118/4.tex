\chapter[Noun morphology]{Noun morphology}\label{chap:4}
\hypertarget{RefHeading1211341525720847}{}
A Moloko noun functions as the head of a noun phrase. A noun phrase can serve as an argument within a clause.  The most prototypical nouns are those denoting something temporally stable, compact, physically concrete and made out of durable material, with a number of defining sub-features (\citealt[50--51]{Givon2001}), but the class extends also to include a range of more abstract concepts. The morphosyntactic criteria for identifying a noun in Moloko include: 

\begin{itemize}
\item They can be pluralised, taking the plural \textit{=ahaj} (\ref{ex:4:1}--\ref{ex:4:2}, see \sectref{sec:4.2.2}).
\end{itemize}

\ea \label{ex:4:1}
məze  ahay\footnote{The first line in each example is the orthographic form. The second is the phonetic form (slow speech) with morpheme breaks.}\\
\gll  mɪʒɛ=ahaj\\
      person=Pl\\
\glt  ‘people’
\z

\ea \label{ex:4:2}
ayah  ahay\\
\gll  ajax=ahaj\\
      squirrel=Pl\\
\glt  ‘squirrels’
\z

\begin{itemize}
\item They can take a possessive pronoun (\ref{ex:4:3}--\ref{ex:4:4}, see \sectref{sec:3.1.2}).
\end{itemize}

\ea \label{ex:4:3}
hor  əwla\\
\gll  hʷɔr=uwla\\
      woman={\oneS}.{\POSS}\\
\glt  ‘my wife’
\z

\ea \label{ex:4:4}
slərele  ango\\
\gll  ɬɪrɛlɛ=aŋgʷɔ\\
      work={\twoS}.{\POSS}\\
\glt  ‘your work’
\z

\begin{itemize}
\item They can be counted (\ref{ex:4:5}--\ref{ex:4:6}, see \sectref{sec:3.3.1}).
\end{itemize}

\ea \label{ex:4:5}
gəvah  bəlen\\
\gll  gəvax  bɪlɛŋ\\
      field  one\\
\glt  ‘one field’
\z

\ea \label{ex:4:6}
sla  ahay  kəro\\
\gll  ɬa=ahaj  kʊrɔ\\
      cow=Pl  ten\\
\glt  ‘ten cows’
\z

\begin{itemize}
\item They can be modified by a demonstrative (\ref{ex:4:7}--\ref{ex:4:8}, see \sectref{sec:3.2.1}-- \sectref{sec:3.2.2}).
\end{itemize}

\ea \label{ex:4:7}
war  nehe\\
\gll  war    nɛhɛ\\
      child  {\DEM}\\
\glt  ‘this child’
\z

\ea \label{ex:4:8}
ma    ndana\\
\gll  ma    ndana\\
      word  {\DEM}\\
\glt  ‘that word’ (just spoken)
\z

\begin{itemize}
\item They can take the derivational morpheme \textit{ga}  resulting in a derived adjective (\ref{ex:4:9}--\ref{ex:4:10}, \sectref{sec:5.3}).
\end{itemize}

\ea \label{ex:4:9}
gədan  ga\\
\gll  gədaŋ  ga\\
      strength  {\ADJ}\\
\glt  ‘strong’
\z

\ea \label{ex:4:10}
ɓərav    ga\\
\gll  ɓərav    ga\\
     heart  {\ADJ}\\
\glt  ‘perseverant’
\z

\begin{itemize}
\item They can be modified by a derived adjective (\ref{ex:4:11}--\ref{ex:4:12}, see \sectref{sec:4.3}).
\end{itemize}

\ea \label{ex:4:11}
memele  malan  ga\\
\gll  mɛmɛlɛ  malaŋ    ga\\
      tree    greatness  {\ADJ}\\
\glt  ‘a large tree’
\z

\ea \label{ex:4:12}
yam  pəyecece  ga\\
\gll  jam    pijɛtʃɛtʃɛ    ga\\
      water  coldness    {\ADJ}\\
\glt  ‘cold water’
\z

Moloko nouns (or noun phrases) carry no overt case markers themselves; the function of the various noun phrases in a clause is indicated by the word order in the clause, pronominal marking in verbs (see \sectref{sec:7.3}), and adpositions (\sectref{sec:5.6}). 

\section{Phonological structure of the noun stem}\label{sec:4.1}
\hypertarget{RefHeading1211361525720847}{}
\citet{Bow1997c} studied syllable patterns in nouns. \tabref{tab:4.24} (from \citealt{Bow1997c}) shows examples of one- to three-syllable noun words of each possible syllable pattern, with and without labialisation and palatalisation prosodies. Syllable pattern is independent of prosody. Bow found many nouns that are CVC but very few that are CV. However, many CVCV nouns actually contain a reduplicated syllable, (\ref{ex:4:13}--\ref{ex:4:15}).

\ea \label{ex:4:13}
dede\\
  dɛdɛ\\
\glt  ‘grandmother’
\z

\ea \label{ex:4:14}
sese\\
  ʃɛʃɛ\\
\glt  ‘meat’
\z

\ea \label{ex:4:15}
baba\\
\glt  ‘father’ 
\z

\begin{sidewaystable}
\caption{Syllable patterns in nouns with different prosodies\label{tab:4.24}}
\begin{tabular}{lllllll} \lsptoprule & {Neutral  } & {Gloss} & {Labialised} & {Gloss} & {Palatalised} & {Gloss}\\\midrule
{CV} & \textit{sla} & ‘cow’ &  &  &  & \\
{CVC} & \textit{fat} & ‘day/sun’ & \textit{hoɗ} & ‘stomach’ & \textit{jen} & ‘chance’\\
{V.CV} & \textit{ava} & ‘arrow’ & \textit{oko} & ‘fire’ & \textit{el\'{e}} & ‘eye’\\
{V.CVC} & \textit{ahar} & ‘hand/arm’ & \textit{otos} & ‘hedgehog’ & \textit{enen} & ‘snake’\\
{CV.CV} & \textit{gala} & ‘yard’ & \textit{sono} & ‘joke’ & \textit{jere} & ‘truth’\\
{CV.CVC} & \textit{mavaɗ} & ‘sickle’ & \textit{tohor} & ‘cheek’ & \textit{pembez} & ‘blood’\\
{V.CV.CV} & \textit{adama} & ‘adultery’ & \textit{oɓolo} & ‘yam’ & \textit{eteme} & ‘onion’\\
{V.CV.CVC} & \textit{adangay} & ‘stick’ & \textit{omboɗoc} & ‘sugar cane’ & \textit{emelek} & ‘bracelet’\\
{CV.CV.CV} & \textit{manjara} & ‘termite’ & \textit{mozongo} & ‘chameleon’ & \textit{zetene} & ‘salt’\\
{CV.CV.CVC} & \textit{maslalam} & ‘sword’ & \textit{dolokoy} & ‘syphilis’ & \textit{debezem} & ‘jawbone’\\
\lspbottomrule
\end{tabular}
\end{sidewaystable}

There are many Moloko nouns whose first syllable is V. This syllable may be historically an old /a-/ prefix. Nouns with these /a-/ prefixes can only be discovered by comparing Moloko vocabulary with that of other related languages where the nouns do not carry the prefix. \tabref{tab:4.25} illustrates three nouns in Moloko and in Mbuko\il{Mbuko}.\footnote{\citet{Mbuagbaw1995}, \citet{Gravina2001}. Judging from the number of nouns in the Moloko database that begin with m, there may be some kind of an old /\textit{m-}/ prefix as well.} 

\begin{table}
\begin{tabular}{lll}
\lsptoprule
{Moloko} & {Mbuko} & {Gloss}\\
\midrule\relax
[\textit{anzakar}]  & [\textit{nzakar}]  & ‘chicken’\\\relax
[\textit{azʊŋgʷɔ}] & [\textit{zʊŋgʷɔ}] & ‘donkey’\\\relax
[\textit{ɛtɛmɛ }] & [\textit{tɛmɛ}] & ‘onion’\\
\lspbottomrule
\end{tabular}
\caption{/a-/ prefix in Moloko compared with Mbuko\label{tab:4.25}}
\end{table}

\citet{Bow1997c} discovered that tonal melodies on nouns are different than for verbs (see \sectref{sec:6.7} for verb tone melodies). \tabref{tab:4.26} (from \citealt{Bow1997c}) shows how the underlying tone melodies are realised on the surface in one, two, and three syllable nouns.  The left column gives examples with no depressor consonants (see \sectref{sec:2.4.1}), and the right column contains nouns with depressor consonants which effect different tone melodies. For one syllable nouns, only two tonal melodies are possible (H or L). For two syllable nouns, H, L, HL, or LH are possible. For three syllable nouns, H, L, HL, LH, HLH, and LHL are possible. Note that a surface mid tone can result from two sources. It can be an underlying high tone that has been lowered by a preceding low tone\footnote{Therefore there are no surface LH combinations since an underlying LH will be realised as LM.} or it can be an underlying low tone in a word with no depressor consonants.\footnote{There are also very few examples of ML combinations in the surface form. The only example was [\textit{k\`{ɪ}m\={ɛ}dʒ\`{ɛ}}], an underlying LHL that had depressor consonants.} 

\begin{sidewaystable}
\begin{tabular}{p{2cm}lp{2cm}lp{2cm}p{2cm}l} 
\lsptoprule
& \multicolumn{3}{l}{{No depressor consonants}} & \multicolumn{3}{c}{{Depressor consonants present}}\\\cmidrule(lr){2-4}\cmidrule(lr){5-7}
{Underlying tonal melody} & {Surface tone} & {Phonetic transcription} & {Gloss} & {Surface tone} & {Phonetic transcription} & {Gloss}\\\midrule
H & H & [tsáf] & ‘shortcut’ & H & [záj] & ‘peace’\\
& HH & [tʃ\'{ɛ}tʃ\'{ɛ}] & ‘louse’ & HH & [b\'{ɔ}ɮ\'{ɔ}m] & ‘cheek’\\
& HHH & [m\'{ʊ}l\'{ɔ}kʷ\'{ɔ}] & ‘Moloko’ & HHH & [d\'{ə}ndárá] & ‘lamp’\\\midrule
L & M & [ɗ\={a}f] & ‘loaf’ & L & [gàr] & ‘difficulty’\\
& MM & [k\={ə}r\={a}] & ‘dog’ & LL & [dàndàj] & ‘intestines’\\
& MMM & [m\={ɪ}t\={ɛ}n\={ɛ}ŋ] & ‘bottom’ & LLL & [àdàŋgàj] & ‘stick’\\\midrule
HL & HM & [m\'{ɛ}k\={ɛ}tʃ] & ‘knife’ & HL & [dʒ\'{ɛ}r\`{ɛ}] & ‘truth’\\
& HMM & [át\={ʊ}kʷ\={ɔ}] & ‘okra’ & HLL & [m\'{ɔ}gʷ\`{ɔ}d\`{ɔ}kʷ] & ‘hawk’\\
& HHM & [m\'{ɔ}s\'{ɔ}kʷ\={ɔ}j] & ‘vegetable sauce’ & HHL & [áz\'{ʊ}ŋgʷ\`{ɔ}] & ‘donkey’\\\midrule
LH & MH & [ɬ\={ə}máj] & ‘ear/name’ & LM & [b\`{ɔ}gʷ\={ɔ}m] & ‘hoe’\\
& MMH & [k\={ɪ}t\={ɛ}f\'{ɛ}r] & ‘scoop’ & LLM & [g\`{ə}g\`{ə}m\={a}j] & ‘cotton’\\
& MHH & [\={a}m\'{ɛ}l\'{ɛ}k] & ‘bracelet’ & LMH & [g\`{ɛ}mb\={ɪ}r\'{ɛ}] & ‘dowry’\\\midrule
HLH & HMH & [ák\={ʊ}f\'{ɔ}m] & ‘mouse’ & HLM & [d\'{ɛ}d\`{ɪ}l\={ɛ}ŋ] & ‘black’\\\midrule
LHL & MHM & [s\={ə}sáj\={a}k] & ‘wart’ & LML & [k\`{ɪ}m\={ɛ}dʒ\`{ɛ}] & ‘clothes’\\
&  &  &  & MHL & [m\={ə}ŋgáhàk] & ‘crow’\\
\lspbottomrule
\end{tabular}
\caption{Tonal melodies on nouns\label{tab:4.26}}
\end{sidewaystable}

\section{Morphological structure of the noun word}\label{sec:4.2}
\hypertarget{RefHeading1211381525720847}{}
Moloko noun words are morphologically simple compared with verbs.  A noun can be comprised of just a noun stem,\footnote{We refer to the simplest form as a stem because it can be more complex than a root in that it can have an /a-/ prefix.} a compound noun, or a nominalised verb. 

\largerpage A noun stem can consist of a simple noun root \REF{ex:4:16} or two reduplicated segments \REF{ex:4:17}. These reduplicated elements actually form two separate phonological words (note the word-final alteration \textit{ŋ} in both segments) but are lexically one item.\footnote{Because there are word-final consonant changes for only /n/ and /h/, it is not known whether all similar reduplications necessarily form two separate phonological words.} 


\ea \label{ex:4:16}
hay\\
  hàj\\
\glt  ‘house’
\z

\ea \label{ex:4:17}
ndən nden\\
  ndəŋ ndɛŋ\\
\glt  ‘traditional sword’
\z

Nouns can be derived from verbs\is{Derivational processes!Verb to noun} by a potentially complex process where a prefix, a suffix, and palatalisation are added. The prefix is \textit{mə-} or \textit{me-}, depending on whether the verb has the \textit{/a-/} prefix or not. The suffix is \textit{-əye} or \textit{-e}, depending on whether the verb root has one or more consonants. The suffix carries palatalisation which palatalises the whole word. The resulting form is an abstract noun which cannot take the plural \textit{=ahay} but which otherwise has all the characteristics of a noun. This highly productive process is discussed further in \sectref{sec:7.6} but two nominalisations are shown here. In \REF{ex:4:18} and \REF{ex:4:19}, the underlying form, the {\twoS} imperative, and the nominalised form are given. A one-syllable verb with no prefix takes the prefix \textit{mə-} and the suffix \textit{{}-əye} \REF{ex:4:18}. A two consonant root with /a-/ prefix takes the prefix \textit{me-} and the suffix \textit{{}-e} \REF{ex:4:19}.

\ea \label{ex:4:18}
\textup{/ v \textsuperscript{e} / \hspace{30pt} ve  \hspace{82pt}    məvəye}\\
\hspace{55pt} [v-ɛ]  \hspace{74pt}    [mɪ-v-ijɛ]\\
\hspace{55pt}      pass[{\twoS}.{\IMP}]-{\CL} \hspace{27pt} {\NOM}{}-pass-{\CL}\
\glt \hspace{55pt} ‘Pass!’ (spend time) \hspace{10pt} ‘year’ (lit. passing of time)
\z

\ea \label{ex:4:19}
\textup{/a-m l-aj/ \hspace{10pt}   məlay  \hspace{70pt}    meməle}\\
\hspace{55pt}   [məl-aj]  \hspace{60pt}    [mɛ-mɪl-ɛ]\\
\hspace{55pt}      rejoice[{\twoS}.{\IMP}]-{\CL} \hspace{20pt} {\NOM}{}-rejoice-{\CL}\\
\glt \hspace{55pt} ‘Rejoice!’ \hspace{55pt}   ‘joy’
\z

Another nominalisation process can be postulated when noun stems and verb roots are compared. This second nominalisation process is irregular and non-productive. \tabref{tab:4.27} illustrates a few examples and compares verb roots with their counterpart regular and irregular nominalisations. In each case, the consonants in the nouns in both nominalised forms are the same as those for the underlying verb root.\is{Derivational processes!Verb to noun} These data show that in the irregular set of nominalisations, there is no set process of nominalisation --- in some cases an /\textit{a-}/ prefix is added (see lines 1 and 2); in other cases the prosody is changed to form the irregular nominalised form (from palatalised to neutral in line 4, from neutral to palatalised in lines 3, 5, and 6). 

When the irregular nominalisations are compared with the regular nominal\-ised form in \tabref{tab:4.27}, it can be seen that the two types of nouns relate to the sense of the verbs in different ways. The regular nominalisation refers to the event or the process itself (stealing, carrying, sending, etc.), whereas the irregular nominalisation denotes some kind of a referent involved in the event (thief, work, hand, etc.). 

\begin{table}
% \resizebox{\textwidth}{!}{
\begin{tabular}{lllll}
\lsptoprule
& & & \multicolumn{2}{c}{Nominalisation}\\\cmidrule{4-5}
{Line} & \multicolumn{1}{p{2.8cm}}{{Underlying form\newline of verb root}} & {{\twoS} imperative} & {Regular} & {Irregular}\\\midrule
1 & \textit{/k r/} & \textit{kar-ay}  & \textit{mə-ker-e}   & \textit{akar} \\
  &		   &  \textup{‘Steal!’} & \textup{‘stealing’} & ‘thief`\\
2 & \textit{/h r/} & \textit{har}  & \textit{mə-hər-e} & \textit{ahar} \\
  &		   & ‘Carry by hand!’ & ‘carrying’  & ‘hand’\\
3 & \textit{/h r ɓ\textsuperscript{o}/} & \textit{hərɓ-oy}  & \textit{mə-hərɓ-e}  & \textit{hereɓ} \\
  &		   & ‘Heat up!’ & ‘heating’ & ‘heat’\\
4 & \textit{/t w/} & \textit{təw-e}  & \textit{mə-təw-e}  & \textit{təway} \\
  &		   & ‘Cry!’ & ‘crying’ & ‘cry’\\
5 & \textit{/ɬ r/} & \textit{slar}  & \textit{mə-slər-e}  & \textit{slərele} \\
  &		   & ‘Send!’ & ‘sending’ & ‘work’\footnote{Probably a compound of \textit{slar} ‘send/commission’ \textit{+ ele} ‘thing’ (\sectref{sec:4.3}).}\\
6 & \textit{/dz n/} & \textit{jən-ay}  & \textit{məjene}  & \textit{jen} \\
  &		   & ‘Help!’ & ‘helping’ & ‘luck’\\
\lspbottomrule
\end{tabular}
% }
\caption{Derived nouns}\label{tab:4.27}
\end{table}

Two processes denominalise nouns; one forms adjectives (\sectref{sec:4.3}) and the other, adverbs (see \sectref{sec:3.5.2}). It is not possible to derive a verb from a noun root or stem in Moloko.

\subsection{Subclasses of nouns}\is{Noun class!Sub-classes of nouns|(}
\hypertarget{RefHeading1211401525720847}{}\largerpage
There are no distinct morphological noun classes in Moloko.  Those nouns with an /a-/ prefix\is{Noun class!A-prefix} could perhaps be considered a separate class (see \sectref{sec:4.1}), but this phenomenon is more of an interesting historical linguistic phenomenon rather than a marker of synchronically different Moloko noun classes. There appears to be no phonological, grammatical or semantic reason for the prefix or other consequences of the presence versus absence of /a-/. 

The plural construction is discussed in \sectref{sec:4.2.2}. Moloko has four subclasses of nouns that are distinguished by whether and how they become pluralised. These are concrete nouns (\sectref{sec:4.2.3}), mass nouns (\sectref{sec:4.2.4}), abstract nouns (\sectref{sec:4.2.5}), and irregular nouns (\sectref{sec:4.2.6}). 

\subsection{Plural construction}\label{sec:4.2.2}
\hypertarget{RefHeading1211421525720847}{}
Noun plurals\is{Plurality!Noun plurals|(} are formed by the addition of the clitic \textit{ahay} which follows the noun or the possessive pronoun. The plural clitic carries some features of a separate phonological word and some of a phonologically bound morpheme. The neutral prosody of [=\textit{ahaj}] does not neutralise the prosody of the word to which it cliticises (\ref{ex:4:20}, \ref{ex:4:21}), which would indicate a separate phonological word (see \sectref{sec:2.6.1}). 


\ea \label{ex:4:20}
 \textup{/atama\textsuperscript{e}} \textup{=ahj/  \hspace{20pt}  $\rightarrow$ \hspace{10pt}  [ɛtɛmɛhaj]}\\
  onion \hspace{8pt}    =Pl     \hspace{60pt}      ‘onions’\\
\z

\ea \label{ex:4:21}
\textup{/akfam\textsuperscript{o}} \textup{=ahj/} \hspace{15pt}   $\rightarrow$ \hspace{10pt} \textup{[}\textup{ɔkʷfɔmahaj]}\\
\glt  mouse \hspace{6pt}  =Pl  \hspace{60pt}      ‘mice’
\z

Two types of word-final changes indicate that the plural is phonologically bound to the noun. First, word-final changes for /h/ that demonstrate a word break do not occur between a noun and the plural \REF{ex:4:2}. 

Second, the stem-final deletion of /n/ before the /=ahj/ (shown in \tabref{tab:28}. adapted from \citealt{Bow1997c}) indicates that the plural is phonologically bound to the noun (\sectref{sec:2.6.1.5}). 

\begin{table}
%\resizebox{\textwidth}{!}
{\begin{tabular}{lllll}
\lsptoprule
& {Underlying form} & {Surface form} & &  {Gloss}\\\midrule
{Neutral } & /g s n/ & [gəsaŋ][=ahaj] \hspace{5pt}  $\rightarrow$  & [gəsahaj] & ‘bulls’\\
& & ‘bull’  \hspace{8pt}  Pl \\
{Labialised} & /t la l n\textsuperscript{o}/ & [tʊlɔlɔŋ][=ahaj] \hspace{3pt}  $\rightarrow$  & [tʊlɔlɔhaj] & ‘hearts’\\
& & ‘heart’ \hspace{8pt} Pl  \\
{Palatalised} & /da d n\textsuperscript{e}/ & [dɛdɛŋ][=ahaj] \hspace{5pt}  $\rightarrow$ & [dɛdɛhaj] & ‘truths’\\
& & ‘truth’ \hspace{5pt}  Pl \\
\lspbottomrule
\end{tabular}}
\caption{Word-final changes of /n/ between noun and plural clitic}\label{tab:28}
\end{table}

\newpage 
We consider the plural marker to be a type of clitic and not an affix\footnote{\citet{Bow1997c} considered the plural marker to be an affix. } because it does show some evidence of phonological attachment and because it binds to words of different grammatical classes in order to maintain its position at the right edge of the noun phrase permanent attribution construction (see \sectref{sec:5.4.2}). The plural  [=\textit{ahaj}] will cliticise to a noun \REF{ex:4:22}, possessive pronoun (\ref{ex:4:23}, \ref{ex:4:24}), or pronoun.  The plural modifies the entire construction in a permanent attribution construction (\sectref{sec:5.1} example \ref{ex:4:10}). 

\ea \label{ex:4:22}
\textup{/ɓ r ɮ n \hspace{7pt} =ahj/     \hspace{36pt}   $\rightarrow$ \hspace{10pt}  [ɓərɮahaj]}\\
\glt  mountain     =Pl   \hspace{71pt}       ‘mountains’
\z

\ea \label{ex:4:23}
\textup{/g l n  \hspace{10pt}      =ahn    \hspace{12pt}      =ahj/  \hspace{3pt}  $\rightarrow$  \hspace{10pt} [gəlahahaj]}\\
\glt  kitchen  {}   =\oldstylenums{3}\textsc{s}.{\POSS}     =Pl  \hspace{38pt}    ‘his/her kitchens’
\z

\ea \label{ex:4:24}
\textup{/plas\textsuperscript{e}} \hspace{1pt} \textup{=ahn   \hspace{12pt}    =ahj/ \hspace{10pt}  $\rightarrow$ \hspace{10pt} [pəlɛʃahahaj]}\\
\glt  horse {\hspace{2pt}}  =\oldstylenums{3}\textsc{s}.{\POSS}    =Pl \hspace{45pt}   ‘his horses’
\z

Note that in adjectivised noun phrases, other constituents must also be pluralised (Section 5.3 examples \ref{ex:5:47}--\ref{ex:5:49})

\subsection{Concrete nouns}\label{sec:4.2.3}\is{Clitics!Plural}
\hypertarget{RefHeading1211441525720847}{}
Concrete nouns (see \tabref{tab:29}) occur in both singular and plural constructions. The plural of these nouns is formed by the addition of the plural clitic \textit{=ahay}  within the noun phrase\is{Plurality!Pluralisation within the noun phrase}, following the head noun (further discussed in \sectref{sec:5.1}). Concrete nouns can also take numerals. 

\begin{table}
\begin{tabular}{lll}
\lsptoprule
{Singular} & {Plural}\footnote{Resyllabification occurs with the addition of plural marker. It is the same resyllabification that occurs at the phrase level (\sectref{sec:2.5.2}).} & {Plural with numeral}\\\midrule
\textit{anjakar}  & \textit{anjakar}\textit{=ahay}  & \textit{anjakar}\textit{=ahay zlom} \\
‘chicken’ & ‘chickens’ & ‘five chickens’\\\midrule
\textit{sləmay}  & \textit{sləmay=ahay}  & \textit{sləmay=ahay cew} \\
‘ear’/‘name’ & ‘ears’/‘names’ & ‘two ears’/‘two names’\\\midrule
\textit{jogo}  & \textit{jogo=ahay}  & \textit{jogo=ahay makar} \\
‘hat’ & ‘hats’ & ‘three hats’\\\midrule
\textit{albaya}  & \textit{albaya}\textit{=ahay}  & \textit{albaya}\textit{=ahay} \textit{kəro} \\
‘young man’ & ‘young men’ & ‘ten young men’\\\midrule
\textit{dede}  & \textit{dede}\textit{=ahay}  & \textit{dede}\textit{=ahay} \textit{məko} \\
‘grandmother’ & ‘grandmothers’ & ‘six grandmothers’\\
\lspbottomrule
\end{tabular}
\caption{Concrete noun plural}\label{tab:29}
\end{table}

\subsection{Mass nouns}\label{sec:4.2.4}
\hypertarget{RefHeading1211461525720847}{}
Mass nouns (shown in \tabref{tab:30}.) are non-countable --- the singular form refers to a collective or a mass, e.g. \textit{yam} ‘water.’  These nouns, when pluralised, refer to different kinds or varieties of that noun referent. These nouns cannot take numerals but they can be quantified (see \sectref{sec:3.3.4}). 

\begin{table}
\begin{tabular}{ll}
\lsptoprule
{Singular} & {Plural}\\\midrule
\textit{yam}  & \textit{yam=ahay} \\
‘water’ &  ‘waters’ (in different locations)\\
\textit{sese}  & \textit{sese=ahay} \\
‘meat’ & ‘meats’ (from different animals)\\
\textit{agwəjer}  & \textit{agwəjer=ahay} \\
‘grass’ & ‘grasses’ (of different species)\\
\lspbottomrule
\end{tabular}
\caption{Mass noun plural}\label{tab:30}
\end{table}

\subsection{Abstract nouns}\label{sec:4.2.5}
\hypertarget{RefHeading1211481525720847}{}
Abstract nouns are ideas or concepts and as such they are not ``singular'' or ``plural.'' In Moloko they do not take \textit{=ahay}\textit{,} e.g., \textit{fama} ‘intelligence, cleverness,’ \textit{slərele} ‘work.’ Although they cannot be pluralised, they can be quantified (see \sectref{sec:3.3.4}). 

\subsection{Irregular nouns}\label{sec:4.2.6}
\hypertarget{RefHeading1211501525720847}{}
Three nouns, all of which refer to basic categories of human beings, have irregular plural forms in that the noun changes in some way when it is pluralised. The singular and plural forms for these nouns are shown in \tabref{tab:31}. For \textit{hor} ‘woman’ and \textit{zar} ‘man,’ the plural forms resemble the singular but involve insertion of the consonant \textit{w} (\textit{hawər} and \textit{zawər}, respectively). For \textit{war} ‘child’ the plural form is completely suppletive (\textit{babəza}). For each of these three items, there is an alternate plural form which is formed by reduplicating the entire plural root. This alternate form is interchangeable with the corresponding irregular plural form.

\begin{table}
\begin{tabular}{lll}
\lsptoprule
{Singular} & {Plural} & {Alternate plural form}\\
\midrule
\textit{hor} & \textit{hawər=ahay} & \textit{hawər hawər} \\
 ‘woman’ & ‘women’  & ‘women’\\\midrule
\textit{zar}  & \textit{zawər=ahay} & \textit{zawər zawər} \\
‘man’ &  ‘men’  & ‘men’\\\midrule
\textit{war} & \textit{babəza=ahay} & \textit{babəza babəza} \\
 ‘child’ & ‘children’  & ‘children’\\
\lspbottomrule
\end{tabular}
\caption{Irregular noun plurals}\label{tab:31}
\end{table}
\is{Noun class!Sub-classes of nouns|)}\is{Plurality!Noun plurals|)}
\section{Compounding}\label{sec:4.3}
\hypertarget{RefHeading1211521525720847}{}
In a language like Moloko where   words meld together in normal speech, real compounds are difficult to identify, since two separate nouns can occur together juxtaposed within a noun phrase without a connecting particle (see \sectref{sec:5.4.2}). In general, if what might seem to be a compound phonologically can be analysed as separate words in a productive syntactic construction, we interpret them as such. We have found some genuine compound noun stems in Moloko, and proper names are often lexicalised compounds that in terms of their internal structure are structurally like phrases or clauses (\sectref{sec:4.4}). 

The grammatical and phonological criteria used to identify a compound are fourfold:
\largerpage
\begin{itemize}
\item The compound patterns as a single word in whatever class it belongs to, instead of as a phrase (that is, in terms of its outer distribution),
\item    The compound is seen as a unit in the minds of speakers,
\item The compound has a meaning that is more specific than the semantic sum of its parts,
\item The compound exhibits no word-final phonological changes that would necessitate more than one phonological word (see \sectref{sec:2.6}); for example, there are no word-final changes ([ŋ] and [x]) and prosodies spread over the entire compound.
\end{itemize}

\tabref{tab:32} shows several compounds made from \textit{ele} ‘thing,’ placed both before and after another root. The compounds in the table illustrate that compounds can be made from a noun plus another noun root (lines 1--3), or a noun plus a verb root (line 4). Note that when \textit{ele} ‘thing’ is the leftmost root in a compound (lines 1--2), \textit{ele}  loses its own palatalisation prosody, an indication that the roots comprise a phonological compound. When it is the rightmost root in the compound, its palatalisation prosody spreads leftwards, affecting the whole word.

\begin{table}
\begin{tabular}{lll@{ }l}
\lsptoprule

{Line} & {Compound noun} & \multicolumn{2}{l}{Elements}\\\midrule
1 & \textit{alahar} & \textit{ele} & \textit{ahar} \\
& ‘weapon, bracelet’ & thing & hand\\
2 & \textit{oloko} & \textit{ele} & \textit{oko}\\
& ‘wood’ & thing & fire\\
3 & \textit{memele} & \textit{mama} & \textit{ele} \\
& ‘tree’ & mother & thing\\
4 & \textit{slərele} & \textit{slar} & \textit{ele}\\
& ‘work’ & send & thing\\
\lspbottomrule
\end{tabular}
\caption{Compounds made with \textit{ele} ‘thing’}\label{tab:32}
\end{table}

\tabref{tab:33} shows two compounds made with \textit{ma} ‘mouth’ or ‘language.’

\begin{table}
\begin{tabular}{ll@{ }l}
\lsptoprule
{Compound} & \multicolumn{2}{l}{Elements}\\\midrule
mahay & \textit{ma} & \textit{hay}\\
‘door’ & mouth & house\\\midrule
\textit{maslar } & \textit{ma} & \textit{aslar}\\ 
‘front teeth’  & mouth & tooth\\
\lspbottomrule
\end{tabular}
\caption{Compounds made with ma}\label{tab:33}
\end{table}
 
A more complex example is \textit{ayva} ‘inside-house.’ It could be analysed as /\textit{a hay ava}/ ‘at house in’; however it distributes not as a locative adpositional phrase, but rather as a noun, in that it can be possessed \REF{ex:4:25} and it can be subject of the verb /s/\textit{ }‘want’ \REF{ex:4:26}.

\ea \label{ex:4:25}
Atərava  ayva  ahan.\\
\gll  a-tər=ava   ajva     =ahaŋ\\
      \oldstylenums{3}\textsc{s}-enter=in    {inside house}  =\oldstylenums{3}\textsc{s}.{\POSS}\\
\glt  ‘He goes into his house.'
\z

\ea \label{ex:4:26}
Asan  ayva  bay.\\
\gll  a-s=aŋ    ajva    baj\\
      \oldstylenums{3}\textsc{s}-please=\oldstylenums{3}\textsc{s}.{\IO}  {inside house}  {\NEG}\\
\glt  ‘He doesn’t want [to go] inside the house.' (lit. the inside of the house does not please him)
\z

\section{Proper Names}\label{sec:4.4}
\hypertarget{RefHeading1211541525720847}{}
Moloko proper nouns\is{Deixis!Proper Names} (names of people, tribes, and places) can be morphologically simple but often are compounds. In the case of names for people, the names often indicate something that happened around the time of the baby’s birth.  Names can also be compounds that encode proverbs. Thus, proper names can be simple nouns, compounds, prepositional phrases, verbs, or complete clauses.  \tabref{tab:34}. illustrates some proper names that are compounds, and shows the components of the name where necessary. Lines 1--5 show simple proper names and lines 6--11 show proper names that are compounds. 

\begin{sidewaystable}
\begin{tabular}{lllll}
\lsptoprule
{Line} & {Name} & {Type of name} & {Components of name}  & {Meaning}\\
& & & {(where applicable)} & \\
\midrule
1 & \textit{Jere} & person &  & ‘truth’\\
2 & \textit{Gajəlah} & person &  & ‘broken piece of pottery’\\
3 & \textit{Ftak} & person/village &  & (no meaning outside its name)\\
4 & \textit{Mokwəyo} & village &  & (no meaning outside its name)\\
5 & \textit{Maslay} & tribe &  & (no meaning outside its name)\\
6 & \textit{Məloko} & language & \textit{ma aloko} & ‘our language’ (Moloko)\\
& & & language=\oldstylenums{1}\textsc{Pin}.{\POSS} \\
7 & \textit{Anjakəyma} & person & \textit{a-njak-ay ma} & ‘here comes trouble’\\
& & & \oldstylenums{3}\textsc{s}-find{}-{\CL}     word \\
8 & \textit{Kosəyməze} & person & \textit{kos-ay məze} & ‘he unites the people’\\
& & & unite[{\twoS}.{\IMP}]-{\CL} people \\
9 & \textit{Kavəyaka} & person & \textit{kə avəya aka} & ‘in suffering’\\
& & & on suffering on \\
10 & \textit{Angaɗay} & person & \textit{a-ngaɗ-ay} & ‘he is joyful’\\
& & & \oldstylenums{3}\textsc{s}-rejoice-{\CL} \\
11 & \textit{Mərəyabay} & person & \textit{məray   abay} & ‘no shame’\\
& & & shame  {\EXT}+{\NEG} \\
\lspbottomrule
\end{tabular}
\caption{Proper names}\label{tab:34}
\end{sidewaystable}

Twins are usually given special names according to their birth order, \textit{Masay} ‘first twin,’ \textit{Aləwa} ‘second twin.’  A single child after a twin birth is named \textit{Aban}.  
