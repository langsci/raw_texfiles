\documentclass[output=paper,colorlinks,citecolor=brown,nonflat]{langsci/langscibook}
\ChapterDOI{10.5281/zenodo.3776541}
\author{Alexandra Cornilescu\affiliation{University of Bucharest}}
\abstract{The paper discusses Romanian data that had gone unnoticed so far and investigates the differences of grammaticality triggered by differentially marked direct objects in ditransitive constructions, in binding configurations. Specifically, while a bare direct object (DO) may bind a possessor contained in the indirect object (IO), whether or not the IO is clitic doubled, a differentially marked DO may bind into an undoubled IO, but cannot bind into an IO if the latter is clitic doubled. Grammaticality is restored if the DO is clitic doubled in its turn. The focus of the paper is to offer a derivational account of ditransitive constructions, which accounts for these differences. The claim is that the grammaticality contrasts mentioned above result from the different feature structures of bare DOs compared with differentially marked ones, as well as from the fact that differentially marked DOs and IO have common features. Differentially marked DOs interfere with IOs since both are sensitive to the animacy hierarchy, and include a syntactic [Person] feature in their featural make-up. The derivational valuation of this feature by both objects may create locality problems.
%{Keywords:} {dative,} {DOM,} {ditransitive} {construction,} {functional} {prepositions,} {binding}
}



\title{Ditransitive constructions with differentially marked direct objects in Romanian}
\shorttitlerunninghead{Ditransitive constructions with \textsc{dom}-ed direct objects in Romanian}

\IfFileExists{../localcommands.tex}{
  \usepackage{langsci-optional}
\usepackage{langsci-gb4e}
\usepackage{langsci-lgr}

\usepackage{listings}
\lstset{basicstyle=\ttfamily,tabsize=2,breaklines=true}

%added by author
% \usepackage{tipa}
\usepackage{multirow}
\graphicspath{{figures/}}
\usepackage{langsci-branding}

  
\newcommand{\sent}{\enumsentence}
\newcommand{\sents}{\eenumsentence}
\let\citeasnoun\citet

\renewcommand{\lsCoverTitleFont}[1]{\sffamily\addfontfeatures{Scale=MatchUppercase}\fontsize{44pt}{16mm}\selectfont #1}
  
  %% hyphenation points for line breaks
%% Normally, automatic hyphenation in LaTeX is very good
%% If a word is mis-hyphenated, add it to this file
%%
%% add information to TeX file before \begin{document} with:
%% %% hyphenation points for line breaks
%% Normally, automatic hyphenation in LaTeX is very good
%% If a word is mis-hyphenated, add it to this file
%%
%% add information to TeX file before \begin{document} with:
%% %% hyphenation points for line breaks
%% Normally, automatic hyphenation in LaTeX is very good
%% If a word is mis-hyphenated, add it to this file
%%
%% add information to TeX file before \begin{document} with:
%% \include{localhyphenation}
\hyphenation{
affri-ca-te
affri-ca-tes
an-no-tated
com-ple-ments
com-po-si-tio-na-li-ty
non-com-po-si-tio-na-li-ty
Gon-zá-lez
out-side
Ri-chárd
se-man-tics
STREU-SLE
Tie-de-mann
}
\hyphenation{
affri-ca-te
affri-ca-tes
an-no-tated
com-ple-ments
com-po-si-tio-na-li-ty
non-com-po-si-tio-na-li-ty
Gon-zá-lez
out-side
Ri-chárd
se-man-tics
STREU-SLE
Tie-de-mann
}
\hyphenation{
affri-ca-te
affri-ca-tes
an-no-tated
com-ple-ments
com-po-si-tio-na-li-ty
non-com-po-si-tio-na-li-ty
Gon-zá-lez
out-side
Ri-chárd
se-man-tics
STREU-SLE
Tie-de-mann
}
  \bibliography{../localbibliography}
  \togglepaper[1]%%chapternumber
}{}

\begin{document}
\maketitle


\section{Problem and aim} %1./

In this paper, I turn to data not discussed for Romanian so far and consider the differences of grammaticality triggered by differentially marked direct objects (i.e. DOs with Differential Object Marking, from now one, DOM-ed DOs) in ditransitive constructions, in \textit{binding} configurations.

Specifically,\footnote{Judgments on possessor binding in Romanian ditransitive constructions and some of the examples come from an experiment described in detail in \citet{CornilescuDinuTigău2017Dative}. Unless otherwise specified, examples and acceptability judgments belong to the author.} bare DOs easily bind a possessor contained in a dative IO, whether the latter is clitic doubled (from now on, CD-ed) or not, as in \REF{ex:cornilescu:1} - \REF{ex:cornilescu:2}. The picture changes when the DO is DOM-ed. It is still possible for a DOM-ed DO to bind into an undoubled IO \REF{ex:cornilescu:3}, but if the IO is doubled, the sentence is ungrammatical \REF{ex:cornilescu:4}. While co-occurrence of the DOM-ed DO with a dative clitic results in ungrammaticality, if the DOM-ed \textsc{DO} is doubled, sentences are grammatical, again irrespective of the presence/absence of the dative clitic, as in examples \REF{ex:cornilescu:5} and \REF{ex:cornilescu:6}.

\ea \label{ex:cornilescu:1} DP\textsubscript{theme} > DP\textsubscript{goal}\\ %1
        \gll {Banca}  {a} {retrocedat}  {multe} {case\textsubscript{i}}  {proprietarilor}   {lor\textsubscript{i}}  {de} {drept}.\\
bank.\textsc{def} has returned  many houses  owners\textsc{def}.\textsc{dat} their of right\\
        \glt ‘The bank returned the houses to their rightful owners.’ \citep[162]{CornilescuDinuTigău2017DOC}
\z


\ea %2
    \label{ex:cornilescu:2}
    DP\textsubscript{theme} > \textbf{cl}-DP \textsubscript{goal}\\
    \gll {Banca} \textbf{{le}}{\textsubscript{j}}{=a} {retrocedat} {multe} {case\textsubscript{i}} {proprietarilor\textsubscript{j}} {lor\textsubscript{i}} {de} {drept}.\\
        bank.\textsc{def} \textsc{3pl.dat}=has returned many houses owners.\textsc{def}.\textsc{dat} their of right\\
    \glt ‘The bank returned many houses to their rightful owners.’ \citep[162]{CornilescuDinuTigău2017DOC}
    \z





\ea%3
    \label{ex:cornilescu:3}
    DOM-ed DP\textsubscript{theme} > DP \textsubscript{goal}\\
    \gll Comisia {a} {repartizat} {pe} {mai} {mulți} {medici\textsubscript{i}} {rezidenți} {unor} {foști} {profesori} {de-ai} {lor\textsubscript{i}}.\\
        board.\textsc{def} has assigned \textsc{dom} more many medical residents some.\textsc{dat} former professors of theirs\\
    \glt ‘The board assigned several medical residents to some former professors of theirs.’
    \z

\newpage

\ea%4
    \label{ex:cornilescu:4}
    *DOM-ed DP\textsubscript{theme} > \textbf{cl}-DP\textsubscript{goal}\\
    \gll {*Comisia} \textbf{{le}}{=a} {repartizat} {pe} {mai} {mulți} {medici\textsubscript{i}} {rezidenți} {unor} {foști} {profesori} {de-ai} {lor\textsubscript{i}}.\\
        board.\textsc{def}  \textsc{3pl.dat}=has assigned  \textsc{dom} more many medical residents some.\textsc{dat} former professors of theirs\\
    \glt ‘The board assigned several medical residents to some former professors of theirs.’
    \z



\ea%5
    \label{ex:cornilescu:5}
    \textbf{cl}-DOM-ed DP\textsubscript{theme} > DP\textsubscript{goal}\\
    \gll {Comisia} \textbf{{i}}{=a} {repartizat} {pe} {mai} mulți {medici\textsubscript{i}} {rezidenți} {unor} {foști} {profesori} {de-ai} {lor\textsubscript{i}}.\\
		board.\textsc{def} \textsc{3pl.acc=}has assigned \textsc{dom} more many medical residents some.\textsc{dat} former professors of theirs\\
    \glt ‘The board assigned several medical residents to some former professors of theirs.’
    \z




\ea%6
    \label{ex:cornilescu:6}
    \textbf{cl}- DOM-ed DP \textsubscript{theme}>\textbf{cl}{}-DP \textsubscript{goal} \\
    \gll {Comisia} {i=l=a=repartizat} {pe} {fiecare} {medic} {rezident} {unei} {foste} {profesoare} {a} {lui}.\\
       board.\textsc{def} \textsc{3sg.f.dat}=\textsc{3sg.m.acc}=assigned \textsc{dom} each medical resident some.\textsc{dat} former professor.\textsc{f}.\textsc{dat} \textsc{gen} his\\
    \glt ‘The board assigned each resident doctor to a former professor of his.’
    \z




Critical is the difference between \REF{ex:cornilescu:2} and \REF{ex:cornilescu:4}, and also between \REF{ex:cornilescu:4} and \REF{ex:cornilescu:5}-\REF{ex:cornilescu:6} where the DO is doubled.

The aim of the chapter is to offer a derivational account of ditransitive constructions, which accommodates these differences. I claim that the grammaticality contrasts above result from the different feature structure of bare DOs compared with DOM-ed ones, and from the fact that DOM-ed DOs and IOs need to check the same [Person] feature against the same functional head.

\section{On Romanian dative DPs} %2./

\subsection{Inflectional datives and the animacy hierarchy} %2.1

In Romanian nouns have \textit{inflectional dative morphology} and, additionally, exhibit \textit{prepositional marking}, employing the locative preposition \textit{la} ‘at’/’to’. An essential property of inflectional datives (=Inf-\textsc{dat}) is that they are highly sensitive to the animacy hierarchy and have a higher cut-off point than \textit{la}\textit{{}-}datives, as seen in \REF{ex:cornilescu:8}.

\ea%7
    \label{ex:cornilescu:7}
    human > animate > inanimate
    \z



\ea%8
    \label{ex:cornilescu:8}
    \ea \label{ex:cornilescu:8a}
    \gll {Am}   {turnat}  {vin}   {la} musafiri/musafirilor.	\\
       	 have.\textsc{1sg}  poured   wine   at guests/guests.\textsc{def}.\textsc{dat}	 \\
    \glt ‘I poured wine to the guests.’
    \ex \label{ex:cornilescu:8b}
    	\gll {Am} {dat}    {apă}  {la} cai/{?}cailor. 	\\
    	have.\textsc{1sg} given   water   at horses/{?}horses.\textsc{def}.\textsc{dat} \\
    	\glt ‘I poured water to the horses.’
    \ex  \label{ex:cornilescu:8c}
    \gll {Am} {turnat}  {apă}  {la} flori/*?florilor.	\\
     		have.\textsc{1sg} poured   water   at flowers/*?flowers.\textsc{def}.\textsc{dat} \\
    \glt	‘I poured water to the flowers.’
   	\z
    \z

One theoretical difficulty that immediately arises is that of incorporating scalar concepts like the animacy hierarchy or the definiteness hierarchy into the discrete binary system of a minimalist grammar. \citet{Richards2008} argues that the animacy hierarchy and the definiteness hierarchy are semantic and pragmatic in nature and should be viewed as \textit{syntax-semantics interface phenomena}. Crucially, he proposes that nouns which are sensitive to these hierarchies should be lexically specified for a binary grammatical [Person] feature (\citealt{Rodríguez-Mondoñedo2007} for Spanish). It is this [Person] feature which triggers the interpretation of a given NP along the two hierarchies, checking its position on the two scales. Nouns which accept the Inf-\textsc{dat} enter the derivation lexically marked as [+Person]. Since this is a syntactic feature, it must be checked during the derivation.

\subsection{ On the internal structure of \textit{la}-datives}

The preposition \textit{la} ‘at’/’to’ is not only a \textit{functional dative marker}, but it is also the core \textit{lexical preposition} of the location and movement frames. The lexical preposition \textit{la} assigns accusative case to its object. This accusative cannot be replaced by a dative, and, as correctly pointed out by both reviewers, accusative \textit{la-}phrases do not co-occur with dative clitics. All movement and location verbs may combine with lexical accusative \textit{la}{}-phrases, rejecting, however, dative \textit{la}{}-phrases. An example is the verb \textit{merge} ‘go’, which is compatible only with lexical \textit{la}, but not with functional dative \textit{la}. Substitution of the \textit{la}{}-phrase with a dative DP is impossible \REF{ex:cornilescu:9a}, and a dative clitic is equally ungrammatical \REF{ex:cornilescu:9b}.

\ea%9
    \label{ex:cornilescu:9}
    \ea \label{ex:cornilescu:9a}
    \gll {Ion} {a} {mers} {la} Maria/**Mariei.	\\
    		Ion has gone at Maria.\textsc{acc}/Maria.\textsc{dat}\\
    \glt ‘Ion went to Maria.’
    \ex[*]{\label{ex:cornilescu:9b}
    \gll {Ion} {îi}={merge} (Mariei).\\
    	Ion \textsc{3sg.f.dat}=goes Maria.\textsc{dat}\\
    \glt ‘Ion is going to Maria.’}
    \z
    \z

One specification is required at this point. Even for unaccusative verbs like \textit{plăcea} ‘like’, which always select a dative Experiencer, either inflectional or prepositional, co-occurrence of a dative \textit{la-}phrase with a clitic is possible only in the third person. In the first and in the second person, the clitic may co-occur only with an inflectional dative strong pronoun, never with a prepositional dative, as apparent in \REF{ex:cornilescu:10b} below:

\ea%10
    \label{ex:cornilescu:10}
    \ea
    \gll {Ciocolata}             {le=place}                    {copiilor}                            /{la} {copii}. 	\\
    	chocolate.\textsc{def}  \textsc{3pl.dat}=like.\textsc{3sg} children.\textsc{def}.\textsc{dat} /at children	\\
    \glt‘Children like chocolate.’
    \ex \label{ex:cornilescu:10b}
    \gll {Ciocolata} {îmi=place} {şi} {mie}/*şi {la} mine.\\
    	chocolate.\textsc{def}  \textsc{1sg.dat}=like.\textsc{3sg} also I.\textsc{dat}/also at me\\
    \glt ‘I also like chocolate.’
    \z
    \z

Verbs in the movement frame do not behave uniformly regarding the realization of their Goal argument. While some never select a dative (e.g. \textit{merge} ‘go’), others (e.g. \textit{ajunge} ‘arrive’ or \textit{veni} ‘come’) may select a dative on condition that the Goal DP is [+Person]; the dative Goal is realized as a clitic, doubled by a strong pronoun or by a dative \textit{la-}phrase, provided that the clitic is third person, as already shown in \REF{ex:cornilescu:10}. Thus, in \REF{ex:cornilescu:11a} the \textit{la}-phrase is lexical; in \REF{ex:cornilescu:11b}, the Goal is a dative phrase realized as a clitic. The first person dative clitic can only be doubled by a dative strong pronoun, while the \textit{la}{}-phrase is out \REF{ex:cornilescu:11c}. The relevant example is however \REF{ex:cornilescu:11d}, an example attested in Google, where the Goal is a dative, and the dative clitic is doubled by a dative \textit{la}{}- phrase. As the comparison of \REF{ex:cornilescu:11a} and \REF{ex:cornilescu:11d} shows, the \textit{la-}phrase is interpreted as a dative only when it co-occurs with a dative clitic.

\ea%11
      \label{ex:cornilescu:11}
         \ea \label{ex:cornilescu:11a}
         \gll {Pachetul} {a} {ajuns} \textbf{{la}} \textbf{mine}/\textbf{la} \textbf{{Londra}} {ieri}.	\\
         	 parcel.\textsc{def} has arrived at \textsc{1sg.acc}/at London yesterday\\
         \glt ‘The parcel got to me/to London yesterday.’
         \ex \label{ex:cornilescu:11b}
         \gll {Pachetul} \textbf{{mi}}{=a} {ajuns} {şi} {mie} {ieri.}	\\
         	parcel.\textsc{def} \textbf{\textsc{1sg.dat}}=has arrived also \textsc{1sg.dat} yesterday\\
         %\glt ‘The parcel got to me too yesterday.’
         \ex \label{ex:cornilescu:11c}
         \gll {Pachetul} \textbf{{mi}}{=a} {ajuns} (*\textbf{la} \textbf{mine}) {şi} {mie} {ieri}.	\\
         parcel.\textsc{def} \textbf{\textsc{1sg.dat}}=has arrived (at \textsc{1sg.acc}) also \textsc{1sg.dat} yesterday\\
         \glt ‘The parcel got to me too yesterday.’
         \ex \label{ex:cornilescu:11d}
         \gll Acum \textbf{{le}}{=au} {venit} \textbf{{la}} \textbf{{mulți}} {deciziile} {de} {recalculare} a pensiilor.\\
         	now \textsc{3pl.dat}=have come at. many.\textsc{acc} decisions of  recalculation \textsc{gen.art} pensions.\textsc{def.gen}\\
         \glt ‘Now many have got their decisions for recalculation of their pensions.’
         \z
         \z

In the rest of this section I examine the internal structure of the \textit{la}{}-phrase when it is a dative, i.e. when it is clitic-doubled. As a dative-marker \textit{la} is puzzling, since it is described as a “dative preposition”, but, as seen above in \REF{ex:cornilescu:9a}, it clearly assigns accusative case to its complement (\figref{fig:cornilescu:1}). On the other hand, \textit{la}{}-phrases may take dative clitics \REF{ex:cornilescu:11c}, and, as well-known, clitics and their associates always agree in Case. This suggests that, as a dative marker, \textit{la} simply assigns Case to a DP \textit{subcomponent} of the dative phrase, while the whole \textit{la}{}-phrase has \textit{an uninterpretable valued dative feature} (\figref{fig:cornilescu:2}), which agrees with the clitic’s Case feature. The marker \textit{la} has become an \textit{internal constituent} which extends the dative phrase (\figref{fig:cornilescu:2}), i.e. a K(ase) marker like the marker of DOM \citep{López2012}. An additional role of this morpheme is that of a category shifter, which reanalyzes the PP into a KP, therefore, an extended DP.

The categorial change from P to K may be viewed as an instance of \textit{downward re-analysis} (\citealt{RobertsRoussou2003}), likely to have occurred out of the need to improve the correspondence between syntactic features and their PF representation.


\begin{figure}[h]%original ex. 12
	\begin{forest}
	for tree={s sep=20mm, inner sep=0}
		[PP
			[P\\{[}Case:{\longrule}{]}\\{[}Loc/Goal{]}\\la]
			[DP\\
			 {[}\textit{u} Case:Acc{]}\\
			 {(}{[}\textit{i}Person{]}{)}
			]
		]
	\end{forest}
	\caption{\label{fig:cornilescu:1} Lexical  \textit{la} assigns accusative Case}
\end{figure}

\begin{figure}
	\begin{forest}
		[KP
			[K\\{[}\textit{u}\sout{Case: \textsc{acc}}{]}\\{[}\textit{u}Case: \textsc{dat}{]}]
			[DP\\{[}\textit{u}\sout{Case: \textsc{acc}}{]}]
		]
		{\draw (.east) node[right]{[Case: \textsc{dat}]}; }
	\end{forest}
	\caption{\label{fig:cornilescu:2} K assigns accusative to DP, while the KP has an uninterpretable dative feature.}
\end{figure}

In time, there gradually emerged two different changes in the function of the Locative PP in \figref{fig:cornilescu:1}. One has been the extension of \textit{la} from verbs that have Goals or Possessor-Goals in their a-structure (verbs of giving and throwing) to verbs that select Beneficiaries (e.g. verbs of creation, like \textit{face} ‘make, do’, \textit{coace} ‘bake’, etc.), and even verbs that select Maleficiary or Source, i.e. the opposite of Goal, (e.g. \textit{fura} ‘steal’). Thus the preposition \textit{la} widens its thematic sphere, but it is partly desemanticized, since the thematic content of the \textit{la}-phrase almost completely follows from the descriptive content of the selecting verb. Secondly, while any kind of DP may assume the Location/Goal ${\theta}${}-role, these extended interpretations (e.g. Beneficiary, Maleficiary) are compatible only with nouns high in the animacy hierarchy. As explained, such nouns grammaticalize their inherent human feature as a syntactic [Person] feature \citep{Richards2008}.

\ea %original 14, now 12
      \label{ex:cornilescu:12}
      Possessor-Goal\\
      \gll {Mama} {(le)=a} {dat} {prăjituri} copiilor /\textbf{la} {copii.} \\
           mother.\textsc{def} \textsc{3pl.dat}=has given cakes children.\textsc{def}.\textsc{dat} /at children\\
      \glt ‘Mother gave cakes to the children.’
      \z



\ea%original 15, now 13
      \label{ex:cornilescu:13}
      Beneficiary \\
      \gll {Mama} {(le)=a} {copt} prăjituri copiilor /\textbf{la} {copii}.\\
             mother.\textsc{def} \textsc{3pl.dat}=has baked cakes children.\textsc{def}.\textsc{dat} /at children\\
      \glt ‘Mother baked cakes {for} the children.’
      \z




\ea%original 16, now 14
      \label{ex:cornilescu:14}
      Maleficiary/Source \\
      \gll  {Nişte} {vagabonzi} {le=au} {furat} copiilor /\textbf{la} {copii} {jucăriile.} \\
             some tramps \textsc{3pl.dat-}have stolen children.\textsc{def}.\textsc{dat} /at children toys.\textsc{def}\\
      \glt ‘Some tramps stole the toys from the children.’
      \z

At this point, there was an imperfect match between features and their exponents, since \textit{la} had partly lost its thematic content, and an obligatory syntactic [+Person] feature in the nominal matrix had no PF realization (\figref{fig:cornilescu:1}). This tension led to the re-analysis of \textit{la} as a PF exponent of the [Person] feature of the noun. As such \textit{la} becomes a higher K part of the nominal expression, where K is a spell-out of [\textit{i}Person]. Syntactically, K is a Probe that values an uninterpretable [\textit{u}Person:{\longrule}] feature of the DP through agreement (\figref{fig:cornilescu:3}).

\begin{figure} %originally ex. 17
	\begin{forest}
		[KP
			[K\\
			{[}\textit{i}Person{]}\\
			{[}\textit{u}Case:{\longrule}{]}\\
			{[}\textit{u}Case: \textsc{dat}{]}
				[la]
			]
			[DP
				[D\\{[}+D{]}\\{[}${\pm}$Def{]}\\{[}\textit{u}${\varphi}${]}\\{[}\textit{u}\sout{Case:Acc}{]}
				]
				[NP\\{[}+N{]}\\{[}\textit{i}${\varphi}${]}\\{[}\textit{u}Person{]}\\{[}+Animate{]}
				]
			]
		]
		{\draw (.east) node[right]{[\textit{i}Person, \textit{i}${\varphi}$,${\pm}$Def, \textit{u}Case: \textsc{dat}]}; }
	\end{forest}
	\caption{\label{fig:cornilescu:3} K is a spell-out of the [Person] feature.}
\end{figure}

Compared to \figref{fig:cornilescu:1}, the representation in \figref{fig:cornilescu:3} is “simpler”, since the grammatical feature [\textit{i}Person], synchretically realized by N in \figref{fig:cornilescu:1} is realized as a separate lexical item in \figref{fig:cornilescu:3}.

Like Inf-\textsc{dat}, \textit{la}{}-\textsc{dat} is sensitive to the animacy hierarchy, but the selectional properties of \textit{la} are not identical to those of the dative inflection. For instance, both types of datives are compatible with names of corporate bodies \REF{ex:cornilescu:15}, but only Inf-\textsc{dat} is also felicitous with \textit{abstract} nouns \REF{ex:cornilescu:16}.

\ea%orig 18, now 15
      \label{ex:cornilescu:15}
      \gll {(Le)=a} {împărţit} {banii} {la} {nişte} {asociaţii} {caritabile}/{unor} {asociaţii} {caritabile}.\\
      	 {}(\textsc{3pl.dat})=has handed\_out money.\textsc{def} at some associations charitable/some.\textsc{dat} associations charitable\\
      \glt  ‘He handed out the money to some charities.’
      \z

\ea%orig 19, now 16
      \label{ex:cornilescu:16}
      \gll {A} {supus} {proiectul} *{la} {atenţia} bordului/{atenției} {bordului}. \\
            has submitted project.\textsc{def} *at attention board.\textsc{def}.\textsc{gen}/attention.\textsc{def}.\textsc{dat} board.\textsc{def}.\textsc{gen}\\
      \glt  ‘He submitted the project to the board’s attention.’
   \z

 {Conclusions} {so} {far}:
 \begin{enumerate}
	\item Nouns may come from the lexicon with an unvalued [\textit{u}Person] feature.
	\item Dative \textit{la} is a K component which spells-out an [\textit{i}Person] feature, historically resulting through downward re-analysis of the homonymous [Location] preposition. K selects DPs which are [\textit{u}Person] and values their [\textit{u}Person] feature.
	\item A KP nominal expression is complex, since it contains a smaller DP. The K-head case-licenses the smaller DP. K also contains an \textit{uninterpretable valued} dative feature checked during the derivation.
	\end{enumerate}

\subsection{Why a clitic is sometimes required}

In theory, like any functional head, the clitic should be a response to some internal need of the \textit{la}-phrase. It is plausible that dative \textit{la}, an [\textit{i}Person] spell-out, further eroded semantically, becoming an uninterpretable [\textit{u}Person], at least sometimes.\footnote{An important empirical generalization \citep{Cornilescu2017} regarding Romanian dative clitics is that they are obligatory for non-core datives, but optional for core datives. In the analysis proposed here, this means that the [Person] feature on dative KPs is uninterpretable by default and can be interpretable only for \textit{core datives}, which have the property of being s-selected by the verb.} The KP continues to have all the features in \REF{ex:cornilescu:17}, except that [Person] is uninterpretable \REF{ex:cornilescu:17}.

\ea%orig 20, now 17
      \label{ex:cornilescu:17}
      KP [\textit{u}Person, +D, ${\pm}$Def, \textit{i}${\varphi}$, \textit{u}Case: \textsc{dat}]
      \z

Given this feature structure a pronominal clitic is required to derivationally supply an [\textit{i}Person] feature. Clitics are known to be sensitive to features like [+D, +Def, …] and cannot combine with nominal projections smaller than DP. They may, however, combine with projections larger than DPs, i.e. KPs, where these features are specified, since they percolate from the D-head.

Concluding, \textit{la}+DP constituents are KPs, where K is a dative head. With verbs of movement and location, including ditransitive ones, \textit{la} + DP are also still analyzable as PPs expressing Goal/Location.

\subsection{{The} {internal} {structure} {of} {the} {inflectional} {dative} {phrase}}%2.4
\label{sec:cornilescu:2.4}

The analysis of [\textit{la\textsubscript{K}}] above suggests a parallel treatment for the dative morphology, K\textsubscript{dative}, which I will also consider a Person exponent. Nouns inflected for the dative are endowed with [\textit{u}Person\_\_], given their sensitivity to the animacy hierarchy. This feature is valued KP-internally, when K\textsubscript{dative} has an interpretable Person feature, i.e. K is [\textit{i}Person, Case-Dative{\longrule}]. Alternatively, if K’s semantic feature is bleached, then K\textsubscript{dative} is [\textit{u}Person] and simply realizes Case. In such situations, CD is obligatory and [\textit{u}Person] is checked KP-externally, using a clitic derivation.

Importantly, like \textit{la-}\textsc{dat}, Inf-\textsc{dat} are ambiguous between a KP and a PP categorization. The PP analysis is, for example, required for adjectives that select Inf-\textsc{dat} complements (e.g. \textit{util} ‘useful’, \textit{folositor} ‘useful’, \textit{necesar} ‘necessary’). Since adjectives are not case-assigners, the Dative is licensed by a null preposition which finally incorporates into the adjective.

Inside \textit{vP}, when the Inf-\textsc{dat} is clitic doubled or, at least, may have been clitic doubled, the Inf-\textsc{dat} is analyzable as a KP. However, when doubling is impossible, the Inf-\textsc{dat} must be projected as a PP, since otherwise it cannot check either Case or Person. One example is that of sentences containing two Inf- \textsc{dat} phrases, of which the higher must be CD-ed and the lower cannot be CD-ed (since they compete for the same \textit{v}P internal position at some point).

\ea%orig 21, now 18
      \label{ex:cornilescu:18}
      \gll {Ion} \textbf{{şi}}{=a} {vândut} {casa} {unor} rude /la {nişte} {rude}.\\
           Ion \textsc{3sg.m.refl}.\textsc{dat}=has sold house.\textsc{def} some.\textsc{dat} relatives /{at} some relatives \\
      \glt ‘Ion sold his house to some relatives.’
   \z


 {Some} {results}:
\begin{enumerate}
	\item Datives inside \textit{v}P –whether \textit{la}{}- \textsc{dat} or Inf- \textsc{dat} - are uniformly either KPs or PPs.
	\item \textit{La-} and K\textsubscript{dative} are exponents of Person which encode sensitivity to the animacy hierarchy.
	\item When K is [\textit{i}Person], the Person feature of datives is checked KP-internally, while the Case feature is checked derivationally. The clitic is unnecessary and thus impossible.
	\item When K is [\textit{u}Person], the ultimate exponent of Person is the clitic, whose presence is mandatory.
\end{enumerate}

 A consequence:
 \begin{itemize}
	\item Given the feature structure of datives [\textit{u/i} Person, \textit{u} Case: Dative], the applicative verb that licenses them should be endowed with the following features: V\textsubscript{appl}[ \textit{u}Person, \textit{u}Case:{\longrule}].
\end{itemize}

\section{Briefly on the syntax of Romanian DOM} %3./

\subsection{{Background}}

The obligatory marker of Romanian DOM is the spatial preposition \textit{pe} ‘on’/‘to\-wards’/‘against’, similar to Spanish \textit{a}. Unlike \textit{a,} however, \textit{pe} assigns accusative case to its object. Therefore, Romanian is not among the many languages where DOM-ed DOs and IOs share the same dative/oblique case, sameness of case representing an explicit connection between the two (\citealt{ManziniFranco2016}).

One of the reviewers stresses that DOM \textit{pe} derives from the \textit{directional} uses of the Old Romanian preposition \textit{p(r)e}, which was often used with directional/Goal verbs (e.g. \textit{striga} ‘call’, \textit{asculta} ‘listen to’, \textit{întreba} ‘ask’), as well as with verbs which entailed the presence of an opponent (e.g. \textit{lupta} ‘fight’), as in the following example:

\ea%19
      \label{ex:cornilescu:19}
      Old Romanian \citep[395]{HillMardale2017}\\
      \gll Au  ascultat	\textbf{pre} \textbf{mine}.\\
            have listened \textsc{dom} me\\
      \glt ‘They have listened to me.’
      \z



Significant research on the history of DOM has demonstrated that in Old Romanian  the presence of the functional preposition \textit{p(r)e} was a means of upgrading the object, signaling a \textit{contrastive topic} interpretation (\citealt{Hill2013, HillMardale2017}). Furthermore, in Old Romanian , \textit{p(r)e} was not restricted to animate nouns, as shown in \REF{ex:cornilescu:20} below:

\ea%orig 23 , now 20
      \label{ex:cornilescu:20}
      Old Romanian  (\citealt[396]{HillMardale2017}) \\
      \gll  Şi deaderă lui Iacov \textbf{pre} \textbf{bozii} cei striini.\\
             and gave \textsc{dat} Jakob \textsc{dom} weeds.\textsc{def} the foreign\\
      \glt ‘And they gave to Jakob the foreign weeds.’
      \z

In Modern Romanian, the noun classes compatible with DOM have been reduced to animate, predominantly [+human] nouns. This restriction is in line with the change in the discourse function of DOM, “from a marker of Contrastive Topic […] to a \textit{backgrounding device} for the [+human] noun in the discourse \citep[147]{Hill2013}”. Thus in Modern Romanian, the most frequent discourse role of DOM-ed objects is that of \textit{familarity topic}, a role which is strengthened by the frequent association of DOM with clitic doubling (\citealt{HillMardale2017}).
% Hi Anna. I am in the chat window

Reinterpreting these important results in the framework of our analysis, it follows that although they do not share Case, DOM-ed DOs and IOs share other properties in Romanian, too. Thus, DOM is sensitive to the animacy hierarchy, which means that both DOM-ed DOs and IOs grammaticalize [Person].

Similarly, the DOM marker \textit{pe} ‘on’/‘to(wards)’ can easily be analyzed as a K head (\citealt{López2012, HillMardale2017}), a spell-out of Person, behaving in all respects like dative \textit{la}, except that \textit{pe}{}-phrases check an accusative feature. Tentatively, the feature structure of \textit{pe}-KPs is as follows: [\textit{u/i}Person, \textit{u}Case:Acc]. When \textit{pe} selects the [\textit{u}Person] option, a clitic extends the KP, forming a chain that ultimately values the [\textit{u}Person] feature.

In harmony with its familiar topic discourse role, DOM is also sensitive to the definiteness hierarchy \REF{ex:cornilescu:21}, which arranges nominal expression by order of their referential stability. Thus, DOM is obligatory for personal pronouns and proper names, which are always referentially stable, it is felicitous but optional with definite and indefinite DPs, and it is impossible with determinerless nouns.

\ea%orig 24, 21now
      \label{ex:cornilescu:21}
       personal pronouns > proper names > definite phrases > indefinite specific > indefinite non-specific  > bare plurals > bare singular
      \z

In its turn, CD is \textit{possible} and \textit{optional} for all accusative KPs, while being \textit{obligatory} only for personal pronouns. Finally CD is not possible for bare DOs, i.e. it operates on KPs, not DPs, presumably because only KPs are marked for [Person].

\subsection{{The} {syntax} {of} {DOM}}

As for the syntax of DOM, I have provisionally adapted to Romanian the analysis in \citet{López2012}. \citeauthor{López2012} maintains the classical view that accusative case originates in \textit{v}. In DOM languages there are two strategies of checking the accusative. Some objects remain \textit{in situ} and satisfy their Case requirement by incorporation into the lexical verb V, which finally incorporates into \textit{v}. DOM-ed objects scramble to the specifier of an αP located between the little \textit{v} and the lexical VP, a position where they are directly probed by little \textit{v}, as in \figref{fig:cornilescu:4}.


\begin{figure}%orig ex. 25
	\begin{forest}
		[\textit{v}P
			[Subject
			]
			[\textit{v}'
				[\textit{v}]
				[αP
					[α]
					[VP
						[V]
						[DO]
					]
				]
			]
		]
	\end{forest}
	\caption{\label{fig:cornilescu:4} Structure of vP proposed by \citep{López2012}}
\end{figure}

The background assumption is that the grammar operates with nominals of different sizes \REF{ex:cornilescu:22}, which may have different syntactic and semantic properties.

\ea%orig 26, now 22
      \label{ex:cornilescu:22}
        KP   >   DP   > NumP   > NP
      \z


In Romanian the cut-off point between objects that scramble and objects that remain \textit{in situ} is the NumP: i.e. NumP and NPs remain \textit{in situ}, DPs may scramble, KPs \textit{must scramble}. On the semantic side, \textit{in situ} objects are interpreted as predicates, objects that scramble are interpreted as arguments.

\section{Dative clitics and CD-Theory} %4./

\subsection{{On} {clitics}}

As already shown, with CD, both dative and accusative clitics select KPs [\textit{u}Person], showing sensitivity to the animacy hierarchy. Accusative clitics also observe the definiteness hierarchy. For instance they exclude bare quantifiers; in contrast, dative doubling is possible for any nominal provided that it has an overt determiner \citep{Cornilescu2017}.

For the current analysis what matters most is that CD-ed DOs and IOs exit the \textit{v}P, passing through a \textit{v}P-periphery position which allows them to bind and outscope the subject in Spec, \textit{v}P (\citealt{Dobrovie-Sorin1994, CornilescuDinuTigău2017DOC,Tigau2011}). Binding of the subject is impossible for undoubled objects. Thus in \REF{ex:cornilescu:27}, the CD-ed dative \textit{fiecărui profesor ‘}every.\textsc{dat} professor\textit{’} binds and outscopes the preverbal subject \textit{câte doi studenţi} ‘some two students’. Similarly, in \REF{ex:cornilescu:28}, the post-verbal doubled DO may bind a possessive in the preverbal subject, but this is not possible for the undoubled DO.

\ea%orig 27, now 23
      \label{ex:cornilescu:27}
      \gll {Câte} {doi} {studenţi} {i=au}  {ajutat} {fiecărui} {profesor}.\\
            some two students \textsc{3sg.m.dat}=have helped each.\textsc{dat} professor\\
      \glt ‘Each professor was helped by two students.’
      \z


\ea%orig 28, now 24
      \label{ex:cornilescu:28}
      \ea
      \gll {Muzica} {lor\textsubscript{i}} {îi} {=plictiseşte}  {pe} mulţi\textsubscript{i}/\textsubscript{j}.\\
      		music.\textsc{def} their \textsc{3pl.acc} bores \textsc{dom} many\\
      \glt ‘Their own music bores many people.’
      \ex
      \gll {Muzica} {lor\textsubscript{*i/j}}  {plictiseşte} {pe} {mulţi\textsubscript{i}}.	\\
      	 music.\textsc{def} their\textsubscript{*i/j} bores on many\textsubscript{i}\\
      \glt  ‘Their music bores many people.’
      \z
      \z



The identity of the \textit{v}P periphery projection through which clitics pass on the way to T is still under debate. Some researchers (e.g. \citealt{Ciucivara2009}) propose that this is a projection where clitics check Case, while others argue that it is a PersonP at the \textit{v}P periphery \citep{Belletti2004Probus, Stegovec2015}, in whose specifier any [\textit{u}Person] nominal can value this feature (\figref{fig:cornilescu:5}). In line with the analysis above, I have adopted the second proposal. Since Person is an agreement feature, rather than an operator one, Spec, PersonP is an \textit{argumental position}. In conclusion, before going to the Person field above T \citep{Ciucivara2009}, the clitic phrase reaches a \textit{Person P}, at the \textit{v}P periphery, above the subject constituent \figref{fig:cornilescu:5}.

\begin{figure}[h] %orig ex. 29
	\begin{forest}
		[PersonP
			[KP]
			{ \draw (.south) node[below]{[\textit{u}Pers]}; }
			[Person
				[Person]
				{ \draw (.south) node[below]{[\textit{i}Pers]}; }
				[\textit{v}P
					[Subject...]
				]
			]
		]
	\end{forest}
	\caption{\label{fig:cornilescu:5} Configuration of Person checking at the vP periphery}
\end{figure}

\subsection{{A} {suitable} {clitic} {theory:} {\citealt{Preminger2019}} }%4.2

Of the many available theories of CD, I have selected \citet{Preminger2019}, which is theoretically more motivated and also simpler. For instance, it does not require a big DP. Rather the starting point is a standard DP/KP. In Preminger’s interpretation, CD is an instance of \textit{long D-head movement}, as in \figref{fig:cornilescu:6}. The D moves from its DP position and adjoins to little \textit{v}, skipping the V head (which is why this is an instance of long head movement).


\begin{figure}%orig ex. 30
	\begin{forest}
	%for tree={s sep=20mm, inner sep=0}
	[vP
		[D\textsuperscript{0}-v\textsuperscript{0}]
		[VP
			[V
			]
			[DP
				[D\textsuperscript{0}]
				[NP]
			]
		]
	]
	\end{forest}
	\caption{\label{fig:cornilescu:6} Configuration of cliticization proposed by \citep{Preminger2019} }
\end{figure}

What is specific to the CD chain is that both copies of D are pronounced, the higher one is the clitic, the lower one is (part of) the associate DP. Pronunciation of two copies of a chain is allowed only if a phasal boundary is crossed (the DP boundary in \figref{fig:cornilescu:6}). The two copies are often phonologically distinct.

\section{On the syntax of ditransitives} %5./

\subsection{{Previous} {results}}%5.1

My analysis of ditransitives assumes the syntax of DOM above. For reasons presented in detail elsewhere \citep{CornilescuDinuTigău2017DOC}, I have adopted a \textit{classical derivational analysis} of the dative alternation \citep{HaradaLarson2009, OrmazabalRomero2017}. Previous research on Romanian ditransitives (\citealt{DiaconescuRivero2007, CornilescuDinuTigău2017DOC}) has brought to light several properties relevant for ditransitive binding configurations.

\begin{enumerate}[a.]
	\item Binding evidence points to the fact that in Romanian ditransitives the internal arguments show a Theme-over-Goal structure. Thus, as sentences \REF{ex:cornilescu:1} and \REF{ex:cornilescu:2} above indicate, the bare DO can bind, not only into an undoubled dative, but also into a doubled one. A Theme-over-Goal base configuration has also been argued for other Romance languages (see, for instance, \citetv{chapters/cepeda} on Portuguese).

	\item In ditransitive constructions, the DO and IO show \textit{symmetric binding potential}, so that there is often an ambiguity between direct and inverse binding for the same pattern. The preferred reading is the one where the surface order corresponds to the direction of binding. For lack of space I will ignore these ambiguities in the analysis below.

	\item There is no difference between the CD-ed and the clitic-less constructions, as far as c-command configurational properties are concerned \citep{CornilescuDinuTigău2017DOC}, i.e. the DO and the IO have symmetric binding abilities irrespective of the presence of the clitic.
\end{enumerate}


I claim that Romanian possesses a genuine alternation between a Prepositional Dative construction, similar to the \textit{to}{}-construction in English, and a pattern similar to the Double Object Construction, where the dative is analyzed as a KP. In the Prepositional Dative construction, the P is null and has the usual role of case-licensing its KP complement. If the null P incorporates, the dative is licensed by an applicative head with the features V\textsubscript{appl} [\textit{u}Person, \textit{u}Case:{\longrule}\_], for reasons explained in \sectref{sec:cornilescu:2.4} above.

The focus of the analysis that follows is to understand why some otherwise available binding configurations become degraded when the DO is DOM-ed.

In order to bring out the contribution of DOM in ditransitive constructions, we compare derivations where the DO is a DP, not a KP, in which case it is not marked for [Person], with derivations in which the DO is DOM-ed, and has [Person] marking. The IO is also successively a PP, a KP, a cl+KP.

\subsection{{The} {DO} {is} {a} {DP} {(i.e.} {it} {is} {not} {DOM-ed)}}%5.2

{In} {the} {basic} {ditransitive} {configuration} the dative is a PP. This configuration, which corresponds to example \REF{ex:cornilescu:1} above {unambiguously} expresses a Theme > Goal interpretation (well-attested). The null P checks Case, and K is [{i}Person], irrespective of whether the IO is an Inf-\textsc{dat} or a {la}{}-\textsc{dat}.

\begin{figure}%orig ex. 31
	\begin{forest}
		[VP
			[DP\textsubscript{theme}\\
				 {[}Case: \textsc{acc}{]}
			]
			[V'
				[V
				]
				[PP
					[P\\
					 {[}$\varnothing${]}
					]
					[KP\\
					 {[}\textit{u}Case: \textsc{dat}{, i}Pers{]}
					]
				]
			]
		]
	\end{forest}
	\caption{\label{fig:cornilescu:7} Case checking when the IO is a PP}
\end{figure}

When null P incorporates, as in \figref{fig:cornilescu:8}, V\textsubscript{appl} [\textit{u}Pers, \textit{u}Case:{\longrule}] is projected. In \figref{fig:cornilescu:8}, both nominals in the domain of V\textsubscript{appl} could value the Case feature of V\textsubscript{appl}, but only the Goal can value its [\textit{u}Pers{\longrule}] feature, since the Theme is a DP not marked for [Person]. Suppose a derivation is intended where the IO binds and precedes the DO, as in example \REF{ex:cornilescu:25} below. In this case, the DO need not move, while the IO should raise past it to Spec, Appl. This derivation is straightforward. V\textsubscript{appl} is allowed to case-license the Theme first, since V\textsubscript{appl} encounters the DO first, when it probes its domain. Next, adopting the locality theory in \citet{Dogget2004} in order to maintain the standard direction of Agree, the Goal moves to an outer Spec,VP, above the Theme. In this position it can be probed by V\textsubscript{appl}, which thus values its own [\textit{u}Pers] feature. At the following step, the Goal KP moves further up to Spec, V\textsubscript{appl}P where it checks Case by Agree with little \textit{v}.


\begin{figure}%orig ex. 32
	\begin{forest}
		[\textit{v}P
			[\textit{v}
			]
			[V\textsubscript{Appl}P
				[{V\textsubscript{Appl}\\
					{[}\textit{u}Pers:\_\_{, u}Case:\_\_{]}}
				]
				[VP
					[DP\textsubscript{Theme}\\{[}Case: \textsc{acc}{]}
					]
					[V'
						[{V\\P+V}
						]
						[{KP\textsubscript{Goal}\\{[}\textit{u}Case: \textsc{dat}, \textbf{iPers}{]}}
						]
					]
				]
			]
		]
	\end{forest}
	\caption{\label{fig:cornilescu:8} Applicative configuration where the IO is a KP which values the Person feature of Appl}
\end{figure}

\ea%25
      \label{ex:cornilescu:25}
      IO before DO; IO > DO \\
      \gll Recepționerul arătă \textbf{fiecărui} \textbf{turist\textsubscript{i}} camera \textbf{lui\textsubscript{i}}.\\
           receptionist.\textsc{def} showed each.\textsc{dat} tourist room.\textsc{def} his\\
      \glt ‘The receptionist showed each tourist his room.’ \citep{CornilescuDinuTigău2017DOC}
      \z

Cliticization is unnecessary, since the Goal is s-selected, and its Person feature is interpretable. Symmetric binding is predicted to be available, since in the initial structure, Theme c-commands Goal, and in the derived structure(s), Goal c-commands Theme. Next we consider \REF{ex:cornilescu:26}, where a CD-ed IO binds and precedes a bare DO.

\ea%orig 34, now 26
 \label{ex:cornilescu:26}
 \gll {Statul} \textbf{le}=a {estituit} \textbf{foştilor} \textbf{proprietari} {casele} {naționalizate}.\\
 state.\textsc{def} \textsc{3pl.dat}=has returned former.\textsc{def}.\textsc{dat} owners houses.\textsc{def} nationalized\\
 \glt ‘The state returned the nationalized houses to their former owners.’
 \z

The presence of the clitic shows that the dative KP is [\textit{u}Pers], as in \figref{fig:cornilescu:9}. For the sake of simplicity I will again consider a derivation where the DO does not scramble and V\textsubscript{appl} checks its Case feature through Agree. At this point, both of the Goal’s features are unchecked, and V\textsubscript{appl} still has an unvalued [\textit{u}Person] feature.

\begin{figure}%orig ex. 35
	\begin{forest}
	%for tree={s sep=20mm, inner sep=0}
		[ApplP
			[Appl\\
				{[}\textit{u}Pers:\_\_ {]}\\
				{[}\textit{u}Case:{\longrule}{]}
			]
			[VP
				[DP\textsubscript{Theme} \\
				 {[}Case:\textsc{acc}{]}
				]
				[V'
					[V
					]
					[KP\textsubscript{Goal}\\{[}\textbf{\textit{u}Pers:{\longrule}}{,} Case:\textsc{dat}{]}
					]
				]
			]
		]
	\end{forest}
	\caption{\label{fig:cornilescu:9} Applicative configuration where the IO KP cannot value the Person feature of Appl}
\end{figure}


The Goal moves to a position (an outer specifier of VP) where it is accessible to V\textsubscript{appl} and there is Agree between V\textsubscript{appl} and the dative, which now shares a [\textit{u}Person] feature, but neither feature is deleted, since both occurrences of the features are unvalued and uninterpretable. The two features are related by agreement and count as instances of the same feature (\citealt{PesetskyTorrego2007}). As in the preceding derivation, the Goal raises to Spec, Appl and checks Case with little \textit{v}, but its [\textit{u}Person] feature is still unvalued. This is what forces movement to the PersonP, at the  \textit{v}P-periphery, as in \figref{fig:cornilescu:10}. When all the features of the Goal have been valued, the goal undergoes cliticization.


 \begin{figure}%orig ex. 36
 \small
	\begin{forest}
	for tree={s sep=3mm, inner sep=0}
		[PersP
			[Pers\\
				{[}\textit{i}Pers{]}
			]
			[\textit{v}P
				[DP\textsubscript{Goal}\\
					{[}\textit{u}Pers{]}
				]
				[\textit{v}P
					[Subject
					]
					[\textit{v}’
						[\textit{v}
						]
						[V\textsubscript{Appl}P
							[<DP\textsubscript{Goal}>
							]
							[V\textsubscript{Appl}'
								[V\textsubscript{Appl}\\
								 {[}\textit{u}Pers{]}
								]
								[VP
									[<DP\textsubscript{Goal}>\\
									 {[}\textit{u}Pers{]}
									]
									[VP
                                                                                    [DP\textsubscript{Theme}]
                                                                                    [V
                                                                                            [V]
                                                                                            [<DP\textsubscript{Goal}>]
                                                                                    ]
									]
								]
							]
						]
					]
				]
			]
		]
 	\end{forest}
	\caption{\label{fig:cornilescu:10} DP Goal raises to the vP periphery to check its uninterpretable Person feature}
\end{figure}

CD was obligatory because the Goal’s Person feature could not be checked inside \textit{v}P.

\subsection{{When} {DOM-ed} {themes} {and} {dative} {goals} {combine}}%5.3

Sentences with DOM and datives create locality problems, since both objects are KPs marked for the same [\textit{i/u} Person] feature and both may value the [\textit{u}Person] feature of V\textsubscript{appl}. The empirical facts are summed up in \REF{ex:cornilescu:30}:

\ea%orig 37, now 30
   \label{ex:cornilescu:30}
  	\ea A \textit{pe}-marked DO binds an undoubled IO without problems (sentence \REF{ex:cornilescu:3} above) \label{ex:cornilescu:30a}
  	\ex A \textit{pe}-marked DO cannot bind a CD-ed IO (sentence \REF{ex:cornilescu:4} above). \label{ex:cornilescu:30b}
  	\ex A CD-ed \textit{pe-}marked Object can bind an IO, irrespective of CD (sentences \REF{ex:cornilescu:5}-\REF{ex:cornilescu:6} above). \label{ex:cornilescu:30c}
   \z
   \z


These facts follow from the analysis. A natural explanation for why a \textit{pe}{}-marked object can bind an IO (= \REF{ex:cornilescu:30a}) is that, in this case the IO stays low and may be (re)analyzed as a PP, thus not competing with the DO.

\begin{figure}%orig ex. 38
	\begin{forest}
		[\textit{v}P
			[\textit{v}\\
				 {[}\textit{u}Case:\_\_{]}
			]
			[${\alpha}$P
				[KP\textsubscript{DO}\\
					 {[}\textit{i}Pers{]}\\
					 {[}\textit{u}Case:\textsc{acc}{]}
				]
				[${\alpha}$P
					[${\alpha}$'
					]
					[VP
						[<KP\textsubscript{DO}>
						]
						[V'
							[V]
							[PP
								[P]
								[DP\textsubscript{IO}]
							]
						]
					]
				]
			]
		]
	\end{forest}
	\caption{\label{fig:cornilescu:11} Configuration of accusative Case checking for DOM-ed DO}
\end{figure}


The \textit{pe}{}-marked DO in \figref{fig:cornilescu:11} scrambles, and it is only for this reason that a landing site is projected between little \textit{v} and VP, as in \citeauthor{López2012}’s analysis. The DO is [\textit{i}Person] and does not need to move beyond its case checking position (Spec, ${\alpha}$P). Let me turn to situations \REF{ex:cornilescu:30b}-\REF{ex:cornilescu:30c} now. When the IO is CD-ed and there is DOM, the result is ungrammatical, as in sentence \REF{ex:cornilescu:4} above. A CD-ed \textit{pe}{}-object restores grammaticality, as in \REF{ex:cornilescu:5} above. Since CD-ed DOM-ed objects are unproblematic, it could be suggested that sentence \REF{ex:cornilescu:4} is ungrammatical because, at the current stage in the evolution of Romanian, \textit{pe}{}-DOs are well-formed only if they are also CD-ed. The following Google example shows however, that CD is not obligatory for \textit{pe}{}-DOs, except for personal pronouns.
\newpage

\ea%39, now 31
   \label{ex:cornilescu:31}
   \gll  {Zavaidoc} {a} {tocmit} \textbf{pe} \textbf{un} \textbf{asasin} {care} {a} {injunghiat=o} {mortal} {pe} {Zaraza}.\\
       Zavaidoc has hired \textsc{dom} an assassin who has stabbed=\textsc{3sg.f.acc} mortally \textsc{dom} Zaraza\\
   \glt ‘Zavaidoc hired an assassin who stabbed Zaraza to death.’ (presentation of the Zaraza restaurant on Google)
   \z

Therefore, the marginality of \REF{ex:cornilescu:4} cannot be attributed to the absence of the clitic, but to some kind of “interference” between the \textit{pe}{}-DOs and CD-ed IOs. I suggest that the problem concerns the locality of Agree, interfering with the feature structure of the two objects.

Consider the following intermediate stage (\figref{fig:cornilescu:12}) in the derivation of sentences like \REF{ex:cornilescu:4}. If the IO is CD-ed, then its Person feature is uninterpretable and the dative KP must check both Person and Case against appropriate functional heads. On the other hand the DOM-ed DO is [\textit{i}Pers] (since it does not need a clitic) and must only value its Case.

When V\textsubscript{Appl} probes its c-command domain, the DOM-ed object is the first that it encounters, so V\textsubscript{Appl} agrees with the closer goal and values its own Person and Case features and it further attracts the KP-DO to its Specifier, since, by assumption, DOM-ed DOs scramble \citep{López2012}. The IO is trapped in its merge position, and cannot check Case and Person anymore, so that the derivation crashes.

  \begin{figure}%orig ex. 40
	\begin{forest}
		[\textit{v}P
			[\textit{v}
			]
			[V\textsubscript{Appl}P
				[V\textsubscript{Appl}\\
					\textbf{{[}{\textit{u}}{Pers}{]}}\\
					{[}\textit{u}Case:\textsc{acc}{]}
				]
				[VP
					[KP\textsubscript{DO}\\
						\textbf{{[}{\textit{i}}{Pers}{]}}\\
                                                {[}\textit{u}Case:\textsc{acc}{]}
					]
					[V'
						[V]
						[KP\textsubscript{do}\\
							\textbf{{[}{\textit{u}}{Pers}{]}}\\
							{[}\textit{u}Case:\textsc{dat}{]}
						]
					]
				]
			]
		]
	\end{forest}
	\caption{\label{fig:cornilescu:12} The DOM-ed DO checks both features of the applicative head.}
\end{figure}



The problem disappears if the DO is CD-ed, as in sentences \REF{ex:cornilescu:5} and \REF{ex:cornilescu:6} above. For simplicity’s sake I will examine sentences where the CD-ed \textit{pe}-DO binds an undoubled IO. In this case, the \textit{pe}-DP is endowed with an uninterpretable Person feature, which will be checked in the \textit{v}P periphery PersonP, just as with datives.

The accusative clitic’s role is syntactic: intuitively “it moves the Theme out of the Goal’s way” \citep{Anagnostopoulou2006}, as in \figref{fig:cornilescu:13}. The DO moves to Spec, V\textsubscript{Appl}, a position where it can be probed by little \textit{v} which checks its accusative Case. Next it targets the PersP at the \textit{v}P periphery, where it Agrees with the [\textit{i}Pers] head and values [\textit{u}Pers]. When all the DO’s features have been checked, cliticization is mandatory. The features of V\textsubscript{Appl} have not been valued yet and the IO is free to move to the outer Spec, VP, where the IO is probed by V\textsubscript{appl} checking its case. The IO, whose person feature is interpretable, values the Person feature of V\textsubscript{appl} and needs to raise no further. Resort to the Accusative clitic is a repair strategy: while the *DOM-ED DP \textsubscript{theme}>{cl}{}- DP \textsubscript{goal} pattern is severely degraded\textsubscript{,} the pattern {cl}{}- DOM-ED DP \textsubscript{theme}>{cl}{}-DP \textsubscript{goal ,} which differs from the preceding only through the presence of the accusative clitic, is fully grammatical.

\begin{figure}%orig ex. 41
	\begin{forest}
		[PersP
			[Pers
			]
			[\textit{v}P
				[KP\textsubscript{Theme}
				]
				[\textit{v}P
					[DP\textsubscript{Agent}
					]
					[\textit{v}'
						[\textit{v}
						]
						[V\textsubscript{Appl}P
								[KP\textsubscript{Theme}
								]
								[V'\textsubscript{Appl}
									[V\textsubscript{Appl}
									]
									[VP
										[KP\textsubscript{Goal}
										]
									]
								]
						]
						]
					]
				]
			]
		]
	\end{forest}
	\caption{\label{fig:cornilescu:13} A clitic doubled DOM-ed DO moves to the vP periphery to check Person.}
\end{figure}


\section{Some theoretical implications of the analysis}\label{sec:cornilescu:6} %6./

Summing up the data we started with in \REF{ex:cornilescu:1} – \REF{ex:cornilescu:6} above and considering the categorial status of the arguments, as well as their (non)-clitic status, we obtain the patterns in \REF{ex:cornilescu:35}.

\ea%35
   \label{ex:cornilescu:35}
   \ea\parbox{1.5cm}{KP-DO} *KP-IO/PP-IO \label{ex:cornilescu:35a}
   \ex\parbox{1.5cm}{Cl-KP}     KP-IO \label{ex:cornilescu:35b}
   \ex\parbox{1.5cm}{Cl-KP}  Cl-KP-IO \label{ex:cornilescu:35c}
   \ex\parbox{1.5cm}{*K-DO} Cl-DP IO \label{ex:cornilescu:35d}
   \ex\parbox{1.5cm}{DP-DO} Cl-KP-IO \label{ex:cornilescu:35e}
   \z
   \z

The critical property of the patterns is the need to check the [\textit{u}Pers] against the Appl head. Sentences of type \REF{ex:cornilescu:35e}, where the DO is a bare DP, which does not need to check Person, are fine irrespective of whether the IO is doubled or undoubled. In contrast, patterns \REF{ex:cornilescu:35a}-\REF{ex:cornilescu:35d} contain two nominals (KPs) that check Person, the DOM-ed direct object and the IO. These types of sentences rely on the configuration in \REF{ex:cornilescu:36}, where the same Appl head should Agree with two arguments, a configuration familiar from the analysis of PCC effects (see \citetv{chapters/sheehan} and the references therein).

\ea%43, now 36
   \label{ex:cornilescu:36}
   Appl[\textit{u}Person] \hspace{0.8cm} DOM DO [\textit{i}/\textit{u}Person] \hspace{0.8cm} IO [\textit{i/u}Person]
   \z


\newpage
What differentiates between \REF{ex:cornilescu:35e} and \REF{ex:cornilescu:35a}-\REF{ex:cornilescu:35d} is that in \REF{ex:cornilescu:35a}- \REF{ex:cornilescu:35d}, but not in \REF{ex:cornilescu:35e}, not only the IO, but also the DO agrees with Appl. Recall that according to \citet{Preminger2019}, PCC effects are likely to occur whenever the relevant DO agrees with \textit{v} or Appl. Indeed the distribution of the asterisks in \REF{ex:cornilescu:35a}- \REF{ex:cornilescu:35d} may be restated as a form of PCC, as also suggested for Spanish ditranstives with DOM by \citet{OrmazabalRomero2013Borealis}.

\ea%44, now 37
   \label{ex:cornilescu:37}
   PCC-like effects in Romanian ditransitives

In a combination of DOM-ed DO and IO, the IO can be doubled (or a clitic) only if the DO is also doubled (or a clitic).
   \z

The admissible patterns in \REF{ex:cornilescu:35a}-\REF{ex:cornilescu:35d} fall in line with this generalization. Pattern \REF{ex:cornilescu:35a}, where neither argument is provided with a clitic would be ungrammatical if the dative had been a KP[\textit{u}Person]. This ungrammaticality is not detected, since the dative is a second, locative argument and can be analyzed as a PP which checks the Case and Person feature of the DP, PP internally, as shown in the discussion of \figref{fig:cornilescu:11} above. Projection as a PP in \figref{fig:cornilescu:11} functions as a repair strategy. In the ungrammatical \REF{ex:cornilescu:35d}, the undoubled DO blocks the lower clitic-doubled dative, preventing it from checking Person (and Case) and producing a PCC-like effect. Patterns \REF{ex:cornilescu:35b}-\REF{ex:cornilescu:35c} are fine since the DO and IO arguments check Person against different heads (Person P, ApplP, respectively), avoiding the problem of multiple arguments agreeing with the same head.

Finally, the data analyzed in this paper provide further evidence for Sheehan’s (this volume) insight that PCC-like phenomena do not depend on (non)clitic status of the arguments, but on the emergence of a configuration of type \REF{ex:cornilescu:36}. In the ungrammatical pattern \REF{ex:cornilescu:4}/\REF{ex:cornilescu:35d}, the DO, in the intervener role, is not a clitic, only the IO is.

\section{Conclusions} %7./

\begin{itemize}
\item DOM-ed DOs interfere with IOs since both are sensitive to animacy hierarchy, codified as Person.
\item
The interaction of DOM-ed DO and IOs in Romanian is a classical locality problem based on the fact that the same applicative head matches two nominals in its c-command domain, regarding Person. The head agrees with the closer object, i.e. the DO. In such configurations, the IO must be a PP, i.e. it cannot be doubled.
\item
 When the DO object is CD-ed, the IO may be a KP and may access V\textsubscript{appl} and it may even be CD-ed.
\end{itemize}


\section*{Abbreviations}
The abbreviations used in the glosses of this chapter follow the Leipzig Glossing Rules.


\section*{Acknowledgements}
I would like to express my gratitude for the wonderful help I got from the reviewers and the editors in finalizing the paper. Remaining errors are all mine.



\sloppy\printbibliography[heading=subbibliography,notkeyword=this]\end{document}
