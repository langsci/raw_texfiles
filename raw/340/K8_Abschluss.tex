% Chapter Template

\chapter{Fazit und Ausblick} % Main chapter title

\label{K8} % Change X to a consecutive number; for referencing this chapter elsewhere, use \ref{ChapterX}

{

%\subsubsection{Generelle Zusammenfassung}
Die vorliegende Arbeit hat gezeigt, dass der Einsatz von maschineller Übersetzung (in Echtzeit) im Rahmen der CvK einen Einfluss auf die Wahrnehmung der Kommunikation hat. Hierzu wurde eine nach bestem Wissen und Gewissen bislang so zuvor noch nicht durchgeführte, explorative Fallstudie entwickelt, die das Chatverhalten von Versuchspersonen in einem Setting mit maschineller Übersetzung mit dem Verhalten in einem monolingualen Chat vergleicht.
%\subsubsection{Situierung und Einpassung}
Die Ergebnisse der vorgenommenen Analysen in dieser Arbeit lassen sich mit denen anderer, in jüngster Vergangenheit durchgeführter naturalistisch orientierter Studien vergleichen und reihen sich damit in den aktuellen Forschungskontext mit kommunikations- und MÜ-bezogener Ausrichtung ein.
%\subsubsection{Theoretischer Mehrwert}
Einen theoretischen Mehrwert liefert die Arbeit dadurch, dass sie aus kommunikationswissenschaftlicher Sicht die Sphäre der Gruppenchats verlässt und eine Konstellation von lediglich zwei am Chat beteiligten Personen aufgreift und diese wiederum mit dem bestehenden Forschungsstand der Translationstechnologien verknüpft. Während die Mensch-Maschine"=Interaktion sowie die Informationsbeschaffung mittels Chatbots mittlerweile einen eigenen, erschlossenen Forschungsbereich darstellen, bestehen mehrere blinde Flecken auf dem Gebiet der CvK in Verbindung mit Echtzeit"=MÜ. Die vorliegende Arbeit weist auch durch die kritische Betrachtung der eigenen Methoden und Ergebnisse auf diese Punkte hin und trägt ein wenig zu diesem Brückenschluss bei.
%\subsubsection{Methodologischer Mehrwert}
Für die methodologische Wegbereitung wurden Elemente der naturalistisch orientierten Forschungskonzeption auf den Gebieten des Eye-Trackings sowie der CvK auf das kleinste erdenkliche Setting mit lediglich zwei beteiligten Personen übertragen. Die besondere Herausforderung lag in der Anpassung bestehender Methoden auf den Einsatz von maschineller Übersetzung innerhalb einer Chatanwendung wie -- in diesem konkreten Fallbeispiel -- dem Skype Translator. Dabei haben sich die einzelnen für die Analyse verwendeten Indikatoren als unterschiedlich aussagekräftig erwiesen.
%\subsubsection{Praktischer Mehrwert}
Der praktische Wert der Arbeit besteht in der Untersuchung des ungeleiteten, am Alltag orientierten Textchats zwischen Personen, die der Muttersprache der anderen beteiligten Person nicht mächtig sind. Aus dieser Betrachtung entstammen Erkenntnisse sowohl in Bezug auf das Chatverhalten als auch auf die Anforderungen an eine möglichst selbsterklärende und intuitiv konzipierte Kommunikationssoftware, damit eine solche Kommunikationssituation überhaupt erfolgreich verläuft. Besonders hervorzuheben ist die Erkenntnis, dass der Echtzeit-MÜ einerseits substanziell hohe Aufmerksamkeit zukommt, andererseits im Bewusstsein der Chatteilnehmer{\textperiodcentered}innen weniger präsent zu sein scheint als angenommen.
%\subsubsection{Ausblick}
Nicht zuletzt bietet die Arbeit einen praktischen Ausblick auf die Anwendbarkeit der Ergebnisse. Kommunikationsanwendungen unterliegen einem stetigen Wandel, was auch über den Verlauf dieses gesamten Projektes spürbar wurde. Eine kontinuierliche Untersuchung ist gerade deshalb unerlässlich.

% Ende von onehalfspacing
}
