\chapter{Existential and locational predication}\label{existpred}

%\setcounter{exx}{0}


\section{Introduction}

In this chapter, two types of predications are discussed: existential\is{existential predication|(} and locational predications\is{locational predication}.
The two types are exemplified by the English\il{English} sentences in (\ref{EngExist}) and (\ref{EngLoc}) respectively.

\eal
\ex\label{EngExist}
There is a tree (in the garden).
\ex\label{EngLoc}
The tree is in the garden.
\zl

\noindent
While in the existential construction (\ref{EngExist}) a statement about the existence of an entity is made, existence is presupposed in the locational\is{locational predication} construction (\ref{EngLoc}) and said entity is categorized with respect to its location in space.
In many languages the formal properties of the constructions, such as definiteness/indefiniteness\is{argument!definiteness of} of the arguments, correlate with these pragmatic implications of the two structures.\is{existential predication|)}

From\is{locational predication!semantics of|(} a descriptive as well as a formal semantic point of view, existential and locational sentences have been treated as similar to one another, if not identical in their underlying semantic structure.\is{locational predication!semantics of|)}
Sometimes, other contexts such as predicative possession\is{possession!predicative} and nominal predication\is{nominal predication} are also put into the same category \citep[111--113]{Payne:1997}.
Nominal predication in marked"=S languages has already been discussed in Chapter~\ref{nompred}. 
I have chosen to treat that topic separately since in some languages in my sample nominal predication has a number of special properties that are not shared with existential or locational predications. 
In contrast, the context of predicative possession\is{possession!predicative|(} did not reveal any special properties in my study.
The languages of my sample employ two strategies for expressing this context: either there is a transitive verb `have'\is{verb class!`have'} or predicative possession\is{possession!predicative} uses the same construction as existentials (while adding the possessor either as an adpositional phrase or an attributive possessor). 
These are also the two main types that \citet{Stassen:2009} distinguishes in his typology of predicative possession. 
He further introduces three subtypes of the locational possessive construction -- the `locational possessive', `with-possessive' and `topic-possessive' -- the details of which are not relevant here.
Another approach to the classification of types of predicative possession are the eight types of possessive `event schemata' distinguished by \citet[47]{Heine:1997}. 
Five of these eight schemata use a formula including a predicate `exist' or `be located', while a sixth uses a predicate `be with' which can be considered a locational concept. 
This approach also indicates a strong relation between the encoding of location\is{locational predication!semantics of} and existence, on the one hand, and possession, on the other.
For those languages of my sample that have an existential/locational/possessive construction, the data from possessive contexts are included in this chapter.
Otherwise, this context is not treated in this study.\is{possession!predicative|)}

For\is{existential predication!positive versus negative|(} a small number of languages in my sample, a different case-form is used for the subject of negative and positive existential predications.
From a cross-linguistic perspective, this behavior is not unheard of, though also not very common (Matti Miestamo\aimention{Miestamo, Matti}, p.c.).
For instance, in Russian\il{Russian} and Finnish\il{Finnish}, subject case-marking is different for positive and negative clauses in a number of contexts. 
While positive copula clauses mark their subjects with Nominative case, in the negative counterparts, Finnish\il{Finnish} employs Partitive case while Russian\il{Russian} uses the Genitive \citep[167]{Dixon:2010-2}.\is{existential predication!positive versus negative|)}   

The overwhelming majority of languages in my sample use the same construction to express locational\is{locational predication!semantics of} and existential predication.
This is, however, not a peculiar fact about languages of the marked"=S type, but has been noted for the majority of the world's languages.
Historical as well as philosophical explanations have been given in order to account for this relation. %(Section~\ref{locexistrel}). 
In addition, when the two predications are not encoded by the same construction, the structural differences appear to be triggered by the same types of factors across languages. 
A brief overview of the literature treating these topics is given in Section~\ref{lingproploc}.
Afterwards, I will present the different patterns found in existential and locational constructions for the languages of my sample and formulate the research questions for the present study (Section~\ref{dataexist}). 
In the subsequent sections, I will present data from Nilo-Saharan (Section~\ref{ExistNilo}), Afro"=Asiatic (Section~\ref{ExistAfro}), North-American languages (Section~\ref{ExistNA}), and languages from the Pacific area (Sectin~\ref{ExistOc}).
Finally, I summarize the languages in my sample in Section~\ref{ExistSum}.
 
\section{Linguistic properties}\label{lingproploc}

%\subsection{Relations between location and existence}\label{locexistrel}

\citet{Lyons:1967,Lyons:1968}\is{locational predication!semantics of|(}\is{existential predication!semantics of|(} argues that existential constructions are historically derived from locational constructions in most of the languages of the world, unless the two kinds of constructions are completely identical to each other.
Indeed, the locational nature is still very obvious in the existentials of many languages since they require some locational phrase -- be it as vague as `here' or `there' -- to be present in this construction (cf. the English\il{English} existential construction `there is a X').
As a motivation for this historical connection, \citet[499]{Lyons:1968} argues for an ontological relation between existence and location since existence implies existence at a specific (though possibly unspecified) location. And, conversely, absence of a entity from all locations implies non-existence.\is{existential predication!semantics of|)} 

While Lyons' discussion is concerned with the semantic and ontological relation of the two types of constructions, other scholars have concentrated on the syntactic relation between the two. 
Among these scholars is \citet{Freeze:1992}, who argues that the underlying syntactic structure of existentials\is{existential predication!syntax of} and locationals\is{locational predication!syntax of} is identical. 
Any differences in the surface realization of the two structures in a given language are triggered by other factors such as definiteness\is{argument!definiteness of} of the S argument.\is{locational predication!semantics of|)}

One structural correlate of these factors is an alternation of word order\is{word order} in the two types of constructions. 
These word order effects are the main focus of the study by \citet{Clark:1978} on existential, locational and possessive constructions.
She\is{argument!definiteness of}(, argues that the ordering correlates with the properties of the subject in terms of definiteness\is{argument!definiteness of} and specificity.
This is shown, for example, by the English\il{English} data (cf. example (\ref{EngExist}) and (\ref{EngLoc}) above), in which the indefinite subject of existentials\is{existential predication!syntax of} is not in the canonical subject position but instead a dummy location is inserted in this position.\footnote{The example such as (i.a) and (i.b) are possible in English\il{English}, but very unusual. Example (i.a) gets better when the locational phrase is added.
\eal
\ex A tree is (in the garden).
\ex A tree exists.
\zllast
}
The (usually) definite subject of locationals\is{locational predication!syntax of} on the other hand preferably occurs in the canonical subject position (i.e. sentence initially).
Clark's findings suggest that this is not only the case in English\il{English}, but that the correlation between word order and predication type is a cross-linguistic tendency, since the overwhelming majority of her sample of 30 languages (with some bias toward European languages) showed this tendency. 
The correlation between word order and existential vs. locational sentences was particularly high for languages without a morphosyntactic means to distinguish definites and indefinites.

Though Clark's findings are intriguing, her collapsing of the categories existential and locational with the notion of indefinite versus definite subject may be somewhat problematic. To distinguish between existentials and locationals, the criterion whether the subject of a clause in definite or indefinite is a good approximation, but counterexamples do occur. 
The following made-up tabloid headline would probably be interpreted as a statement about existence rather than location by most speakers of English\il{English}, yet the subject is marked with the definite article (\ref{Yeti}).

\begin{exe}\ex\label{Yeti}The Yeti exists.
\end{exe} 

So the question has to be answered whether Clark's correlation really is between word order\is{word order} and existential versus locational predications, or rather between word order and definiteness, which in most cases coincides with the distinction between existentials and locationals\is{argument!definiteness of|)}.

In studies of existential and locational predication, not much is said about the case-marking of subjects in these constructions. 
Or, to put it in other terms, the question is whether the S-like argument in existentials and locationals behaves like other S elements. 
Given the topic of this study, this is my main interest with regard to these contexts.
\citet[123]{Payne:1997} notes that there ``[u]sually is no or reduced evidence of grammatical relations in existential constructions.''
If this is true, one would not expect S of existential predications\is{existential predication!syntax of} to be encoded like more typical intransitive subjects in marked"=S languages.  


\section{Research question}\label{dataexist}

In the subsequent sections, I will present data on locational and existential predications in the languages of the marked"=S type. 
The special focus is on the case-forms employed for the S-like arguments in these clauses. 
More specifically, I selected three contexts: locational predications, as well as positive and negative existential predications. 
In each of these contexts, the marking of the respective subject is investigated. Thus, data for the following three roles were collected for each language of the sample:

\begin{itemize}
\item subject of positive existential predication
\item subject of negative existential predication
\item subject of locational predication
\end{itemize}

The\is{existential predication!positive versus negative|(} distinction between negative and positive predications is only made for existentials here.
If a language uses the same construction for existential and locational contexts, any differences between negative and positive existentials will also be found with negative locationals. 
However, there are languages in which the difference between positive and negative contexts is only found with existentials to the exclusion of locationals, while no language makes such a distinction exclusively in the locational context.\is{existential predication!positive versus negative|)} %For several reasons data on one or the other of these contexts are missing for a number of languages in my sample.

Most marked"=S languages use the same constructions for existential and locational predications. 
Usually, these constructions encode their subjects like subjects of regular intransitive clauses. 
A distinction between the encoding of subjects of positive and negative contexts is only found in few of the languages. 
Not all languages appear to have dedicated constructions for locationals and/or existentials. 
The contexts (or subset of these contexts) are often expressed through the use of a generic intransitive verb\is{verb class!positional verb} expressing some kind of local orientation, such as `sit', `stand' or `lie'. 
In these cases, the locational and existential predications can be regarded as instances of regular intransitive clauses. 
Thus subjects are expected to be in the S-case.\footnote{Recall that the label \textsc{S-case} is a shorthand for: the nominative case if a language has nominative"=accusative alignment and the absolutive case if a language has ergative"=absolutive alignment.}  

First, I will give an example of this majority pattern. 
The S element in (\ref{MojEx}) and (\ref{MojLoc}) is marked with the Nominative case-suffix \emph{-\v c} in Mojave\il{Mojave}, just as any intransitive S argument is. 
Note that Mojave\il{Mojave} does not have a single existential or locational verb\is{verb class!positional verb}, using instead a number of stative verbs\is{verb class!positional verb} in both existential and locational predications.
%\footnote{The following abbreviations are used:  CL = classifier; DIST\_PST = distant past; DSCN = discontinuous; INCEP = inceptive;  PF = perfect; \prosp{} = prospective; }

\begin{exe}\ex\langinfo{Mojave}{Yuman; Arizona}{\citealp[33, 212]{Munro:1976}}
 \begin{xlist}
\ex\label{MojEx}\gll\textipa{hukTar\textbf{-\v c}} \textipa{\textglotstop avi:-T-l\super{y}} \textipa{idi:-k}\\
coyote-\nom{} mountain-\dem{}-\loc{} lie-\tns{}\\
\glt `There are coyotes in those hills.'

\ex\label{MojLoc}\gll\textipa{pi:pa} \textipa{n\super{y}amaTa:m} \textipa{k\super{w}@loyaw} \textipa{k\super{w}-tapoy-h-n\super{y}\textbf{-\v c}} \textipa{\textglotstop ava:-l\super{y}} \textipa{iva-m}\\
person tomorrow chicken \relativ{}-kill-\irr{}-\dem{}-\nom{} house-\loc{} sit-\tns{}\\
\glt `The man who's going to kill the chicken tomorrow is in the house.'
\end{xlist}
\end{exe} 

Nias\il{Nias} also uses the same type of construction to encode locational and existential meanings. 
However\is{existential predication!positive versus negative|(}, different constructions are used for positive and negative contexts.
While the construction used for positive contexts (\ref{NiasExistExamp}a) employs the S element in the Mutated form of a noun (i.e. the same as for regular intransitive S), in negative contexts the S-like element is in the Unmutated form (\ref{NiasExistExamp}b).

\begin{exe} \ex\label{NiasExistExamp}\langinfo{Nias}{Sundic; Sumatra, Indonesia}{\citealp[344, 358]{Brown:2001}}
\begin{xlist} \ex \gll ga so \textbf{g\"ocoa}\\
here exist cockroach.\mut{}\\
\glt `There's a cockroach here.'

\ex \gll l\"ona \textbf{ba{\ss}i} ba mbanu ha'a\\
\Neg{}.exist pig \loc{} village.\mut{} \prox{}\\
\glt `There are no pigs in this village.'
\end{xlist} 
\end{exe}\is{existential predication!positive versus negative|)}

Finally, there is one language in my sample in which different constructions are used for existentials and locationals (at least by some speakers).
While in Tennet\il{Tennet} existentials the subject can be zero-coded (\ref{TenExiExa}), with locationals the Nominative case is always used (\ref{TenLocExa}).

\begin{exe} \ex\langinfo{Tennet}{Surmic; Sudan}{\citealp[236, 236]{Randal:1998}}
 \begin{xlist}
\ex\label{TenExiExa}\gll\textipa{\'any\'ak} \textbf{\textipa{m\'am}} \textipa{c\'{\=*e}\'{\=*e}z-a}\\
have water house-\obl{}\\
\glt `There is water in the house.' 

\ex\label{TenLocExa}\gll\textipa{\'{\=*a}ve} \textbf{\textipa{l\=*o\'{\=*u}d\'{\=*o}}} \textipa{ke\'et-\'a} \textipa{v\'{\=*u}rt-\^{\=*a}}\\
be\_located Loudo.\nom{} tree-\obl{} under-\obl{}\\
\glt `Loudo is under the tree.'
\end{xlist}
\end{exe}

The following sections provide a detailed study of the contexts of positive and negative existential predication and locational predication in marked"=S languages. 
The data are divided by genealogical and areal grouping into the Nilo-Saharan (\ref{ExistNilo}) and Afro"=Asiatic languages (\ref{ExistAfro}), and the languages of North America (\ref{ExistNA}) and the Pacific area (\ref{ExistOc}). 
In many cases, it has been difficult to obtain information on the contexts studied here for individual languages. 
This is probably due to the fact that clauses of the existential and locational type are often encoded like regular intransitive clauses and thus are not explicitly discussed in many grammars. 
Hence, in the following sections there are no data on one or the other context for a number of languages.    

\section{Nilo-Saharan}\label{ExistNilo}

For most marked"=S languages of the Nilo-Saharan stock, the S arguments of existential and locational predications are encoded alike, since the same constructions are used in both contexts. However, some languages show interesting patterns, especially in having alternative constructions in the different subdomains.

In Murle\il{Murle}, the prototypical situation is attested, in which parallel constructions are used for existential (\ref{MurEx}) and locational predication (\ref{MurLoc}). 
And indeed this construction is also parallel to other intransitive verbs\is{verb class!positional verb} (\ref{MurS}). 
Nandi\il{Nandi} (\ref{NanExist}) and Datooga\il{Datooga} (\ref{DatExist}) behave similarly.

%\pagebreak
\begin{exe}\ex\label{MurExist}\langinfo{Murle}{Surmic; Sudan}{\citealp[49, 50]{Arensen:1982}}
\begin{xlist}
\ex\label{MurEx} \gll \textipa{abil} \textipa{guumun\textbf{-i}} \textipa{kEEt} \textipa{taddina}\\
stands owl-\nom{} tree up\\
\glt `There is an owl up in the tree.'
\ex\label{MurLoc}\gll\textipa{EEl} \textipa{tor-Et\textbf{-a}} \textipa{ceeza}\\
stand gun-\pl{}-\nom{} in\_house\\
\glt `The guns are in the house.'
%Murle\il{Murle} \citet[52]{Arensen:1982}
\ex\label{MurS}\gll\textipa{akO} \textipa{agul\textbf{-i}} \textipa{ci} \textipa{appi} \textipa{liila}\\
goes crocodile-\nom{} \relativ{} big into\_river\\
\glt `The big crocodile goes into the river.'
\end{xlist}
\end{exe} 

\begin{exe}\ex\label{NanExist}\langinfo{Nandi}{Nilotic; Kenya}{\citealp[123]{Creider:1989}} %Compare [47]{Creider:1989} for the formation of the noun `lion'
\begin{xlist}
\ex\gll\textipa{m\`I:t-\'ey} \textipa{\textbf{nget\'un}-ta}\\
\cop{}-\ipfv{} lion.\nom{}-\them{}\\
\glt `There is a lion.' %\end{exe} \citet[123]{Creider:1989}

\ex\gll\textipa{m\`I:t-\'ey} \textipa{\textbf{k\'Ipro:no}} \textipa{kit\^a:li}\\
\cop{}-\ipfv{} Kiprono.\nom{} Kitale\\
\glt `Kiprono is in Kitale.'
%\end{exe} \citet[123]{Creider:1989} 
\end{xlist}
\end{exe}

%\pagebreak
\begin{exe}\ex\label{DatExist}\langinfo{Datooga}{Nilotic; Tanzania}{\citealp[184, 171]{Kiessling:2007}}
\begin{xlist} 
\ex\gll m\`a-nd\'a \textbf{d\'uu}-s\`u j\'aa g\'a-w\'a gw\'a-r\'oo\textltailn \'\i\\
3.\Neg{}-be\_there cattle.\nom{}-\prox{}.\pl{} \nom{}.\fut{}.\relativ{} 3-go 3-meet\\
\glt `There are none of these cattle that he may go to meet.'
%\end{exe}
% Nominative \citep[171, ex. 41]{Kiessling:2007}

\ex\gll gw\'and\`a \textbf{g\'ad\'eemg\'a} j\`eed\'a d\^uhw\textsubdot{a}\\
3-be\_there women.\nom{} among cattle.\acc{}\\
\glt `The women were among the cattle.'
\end{xlist}
\end{exe}

As noted before, the same is true for the majority of languages in my sample. 
However, there are some languages which have an alternative construction for one of these two types of predication that differs from the encoding of the other type. 
Also, in some languages at least some types of existential and/or locational predications do not encode their subject in the same way as prototypical intransitive clauses encode their subjects (S). 
In the following, I will focus on these languages.   

The first Nilo-Saharan language which exhibits some variation with respect to the encoding of the S argument in existential and locational predications is Turkana\il{Turkana}.
At least two different constructions are used in Turkana\il{Turkana} for encoding existential and locational contexts.
First, existentials can be encoded like nominal predicates. 
As seen in the previous chapter (\ref{NomPredNilo}), this construction usually does not have a verb, unless it is negated or in non-present tense. 
In those verbless clauses, the S argument is in the Accusative\is{case!individual forms!accusative} case (\ref{TurExZero}a). 
If a verb is present -- whether to encode negation or past tense, or because construction with a lexicalized verb is used, as in the next example -- Nominative\is{case!individual forms!nominative} case is used for the S argument (\ref{TurExZero}b).\footnote{The example in (\ref{TurExZero}b) is an idiomatic expression, in which the verb `drink' is deprived of its lexical meaning. 
The high potential of verbs of eating and drinking to undergo metaphorical extensions is discussed in \citet{Newman:2009}.}

\begin{exe}\ex\label{TurExZero}\langinfo{Turkana}{Nilotic; Kenya}{\citealp[74, 75]{Dimmendaal:1982}}
\begin{xlist}
\ex\gll\textipa{NI-dE\`{}} \textipa{omwOn\`{}}\\
\NC{}-children.\acc{} four\\
\glt `There are four children.'

\ex\gll\textipa{\`E-m\`aa-s\`e} \textipa{\textbf{NI-d\`E}} \textipa{omwOn\`{}}\\
3-drink-\pl{} children.\nom{} four\\
\glt `There are four children.'
\end{xlist}
\end{exe}

%\begin{exe}\ex\gll\textipa{\`e-y\`e-i\`{}} \textipa{a-k\`ayi} \textipa{a-c\`E} \textipa{n\`ege\`{}}, \textipa{na-e-y\`a} \textipa{Ni-mu\textltailn-in}\\
%3-be-A house other here where-3-be snakes\\
%`there is another house here, where there are snakes.'
%\end{exe} [331]

%\subsubsection*{Subject of locative clauses:}

The second construction I will discuss here is interpreted as either existential, locational or possessive.
Other than the nominal predication construction, in which an overt copula\is{copula!absence versus presence} only occurs when it is needed to host negation\is{existential predication!positive versus negative} or tense marking, the copula is usually used in all cases. 
As is to be expected in constructions which have an overt verb, the Nominative\is{case!individual forms!nominative} case is used for the S argument (\ref{TurExLoc}).
In the possessive interpretation of this construction, the possessee is always interpreted as being indefinite (\ref{TurLocZero}a). 
If one wants to formulate a possessive sentence with a definite possessee, the non-verbal construction used in nominal predications has to be employed (\ref{TurLocZero}b)  according to \citet[82]{Dimmendaal:1982}.
%
%\begin{exe}\ex\gll\textipa{\`e-y\`aka-s\`I} \textipa{Ni-k\`a\`al-a}\\
%3-be-\pl{} \NC{}-camel-\pl{}\\
%`the camels are there/ there are camels.'\end{exe}
%The same construction can have locational, existential or possessive meaning:


\begin{exe}\ex\label{TurExLoc}\langinfo{Turkana}{}{\citealp[82]{Dimmendaal:1982}}
\begin{xlist}
\ex\gll\textipa{\`e-y\`aka-s\`I} \textipa{Na-\textbf{\`at\`uk}}\\
3-be-\pl{} \NC{}-cows.\nom{}\\
\glt `There are cows (or the cows are there).'

\ex\gll\textipa{\`e-y\`e-i\`{}} \textipa{a-\textbf{p\`EsE}} \textipa{\textbf{a-p\`ey}}\\
3-be-\asp{} \NC{}-girl.\nom{} \NC{}-one.\nom{}\\
\glt `There is one girl (or one girl is there).'
%\end{exe} [82]
\end{xlist}
\end{exe} %[82]

\pagebreak
\begin{exe}
 \ex\label{TurLocZero}\langinfo{Turkana}{}{\citealp[82]{Dimmendaal:1982}}
\begin{xlist}
\ex\gll\textipa{\`e-y\`aka-s\`I} \textipa{a-yON\`{}} \textipa{Na-\textbf{\`at\`uk}}\\
3-be-\pl{} \NC{}-me \NC{}-cows.\nom{}\\
\glt `I have cows.'
\ex\gll\textipa{\textbf{Na-atuk\`{}}} \textipa{Nugu\`{}} \textipa{Na-kaN\`{}}\\
cows.\acc{} these.\acc{} \NC{}-mine\\
\glt `These cows are mine'
%\end{exe} [82]
\end{xlist}
\end{exe}

Maa\il{Maa} is another language that shows some variation on the constructions used for existential and locational contexts. 
According to \citet{Payne:2007}, there are two types of existentials in Maa\il{Maa}, those constructed with the verb\is{verb class!positional verb} \emph{tii} `be at' and those constructed with the verb \emph{ata} `have'\is{verb class!`have'}. %\footnote{All Maa\il{Maa} examples are from \citet{Payne:2007}.}
The first construction, i.e. the one with \emph{tii}, encodes both existential and locational contexts. 
In this construction, the S argument is always marked with the  Nominative\is{case!individual forms!nominative} case (\ref{MaaExLoc}a, b).
Existentials constructed with the verb \emph{ata}\is{verb class!`have'} on the other hand have zero-coded S arguments and do not have a locational meaning (\ref{MaaZeroEx}).


\begin{exe}
 \ex\label{MaaExLoc}\langinfo{Maa}{Nilotic; Kenya}{\citealp[ex.\,19a, ex.\,17, ex.\,20a]{Payne:2007}}
\begin{xlist}
\ex\gll \textipa{\neu{}-\'e-ti\'I} \textipa{ap\'a}, \textbf{\textipa{Ol-mUrran\'I}} \textbf{\textipa{\'obo}}\\
        \con{}-3-be.at long\_ago \mas{}.\sg{}-warrior.\nom{} one.\nom{}\\
\glt `Long ago, there was a warrior.'
\ex\gll\textipa{e-t\'I\'I} \textbf{\textipa{Enk-\'ay\'I\'on\'I}} \textipa{ol-kEj\'U}\\
       3-be\_at \fem{}.\sg{}-boy.\nom{} \mas{}.\sg{}-leg.\acc{}\\
\glt `The boy is at the river.' (lit. `The boy is at the big leg.')
\ex\label{MaaZeroEx} \gll \textipa{n-\'e-yiol\'o-u} \textipa{\'a\`aj\`o}  \textipa{k-\'E-\'ata-I}  \textbf{\textipa{Enk-\'a\'I}} \textipa{n\'a-r\^a} \textipa{pap\^a}\\
\con{}-3-know-\incep{} that.\pl{} \dscn{}-3-exist-\pass{} \fem{}.\sg{}-God.\acc{} \relativ{}.\fem{}-be father.\acc{}\\
\glt `They knew that there is God who is the father.'
\end{xlist}
\end{exe}
%\citep[ex.17]{Payne:2007}


In the above example of the \emph{ata}-existential\is{verb class!`have'}, the verb is in the passive. 
Since passive verbs always take their subjects in the zero-coded Accusative\is{case!individual forms!accusative} form in Maa\il{Maa}, this is not surprising.\footnote{The following examples, from \citet[ex.16, ex.15]{Payne:2007}, demonstrate the Maa\il{Maa} Passive and the corresponding active clause: 
\begin{exe}\ex
\begin{xlist}\ex\gll\textipa{E-tE-En-\'ak-\`I} \textbf{\textipa{Ol-ap\'urr\`on\`I}}\\
3-\prf{}-tie-\prf{}-\pass{} \mas{}.\sg{}-thief.\acc{}\\
\glt `The thief was arrested.'
\ex\gll\textipa{E-Ib\'UN-\'a} \textipa{I-s'IkarIn\'I}	\textbf{\textipa{Ol-ap\'urr\`on\`I}}\\
3-catch-\prf{} \pl{}-police.\nom{} \mas{}.\sg{}-thief.\acc{}\\
\glt `The policemen have arrested the thief.'
\end{xlist}
\end{exe}}
However, there are some non-passive \emph{ata}-existentials which nevertheless take zero-coded subjects. 
Examples of the type demonstrated in (\ref{MaaExistNoPas}) make up a quarter of the instances of \emph{ata}-existentials in Payne's corpus. 

\begin{exe}\ex\label{MaaExistNoPas}\langinfo{Maa}{}{\citealp{Payne:2007}}
\begin{xlist}
\ex\gll \dots \textipa{am\^U} \textipa{m-E-\'at\`a} \textbf{\textipa{Ol-tUN\'an\`I}} \textipa{\'o-\'Itieu}\\
because \Neg{}-3-exist \mas{}.\sg{}-person.\acc{} \mas{}.\sg{}.\relativ{}.\acc{}-dare\\
\glt `\dots because there is no one who can face him.'

\ex\raggedright\gll\textipa{\mas{}-E-\'Et\`a} \textbf{\textipa{Ol-m\'Urr\'an\`I}} \textipa{l\'E-m-\'e-ny\'Orr} \textipa{te=n-e-i-pus-\'I\'ek-\`I} \textipa{Enk-\'a\'In\'a}\\ 
\Neg{}-3-exist \mas{}.\sg{}-warrior.\acc{} \relativ{}.\mas{}-\Neg{}-3-like \obl{}=\con{}-3-\Verb{}-blue-\instr{}-\pass{} \fem{}.\sg{}-arm.\acc{}\\
\glt `There isn't a warrior who doesn't want to (have his) hand be made blue.'
\end{xlist}
\end{exe}

In the previous section, data from Tennet\il{Tennet} have already been introduced.
\citet[236]{Randal:1998} notes that in Tennet\il{Tennet} not all speakers use parallel constructions for existential and locational predications.
Some speakers use the standard locational construction for existentials as well. 
In this construction, the S argument is in the Nominative\is{case!individual forms!nominative} case (\ref{TenLocat}, b).
Other speakers use a different construction for existential contexts, which has a zero-coded S argument (\ref{TenExist}). 
For\is{existential predication!positive versus negative|(} negative existential and locational predications the subject is always zero-coded (\ref{TenNeg}).
The basic variation between the two groups of speakers is thus whether the positive existential context is covered by the same construction as the negative existential context or as the positive locational context. 

\begin{exe}\ex\langinfo{Tennet}{Surmic; Sudan}{\citealp[223]{Randal:1998}}
\begin{xlist}
\ex\label{TenLocat}\gll\textipa{\'{\=*a}ve} \textbf{\textipa{l\=*o\'{\=*u}d\'{\=*o}}} \textipa{ke\'et-\'a} \textipa{v\'{\=*u}rt-\^{\=*a}}\\
be\_located Loudo.\nom{} tree-\obl{} under-\obl{}\\
\glt `Loudo is under the tree.'

\ex\label{TenLocation}\gll\textipa{\'avte} \textipa{b\=*ur\'{\=*u}\textbf{-n\^a}} \textipa{lebel-\'a}\\
stay.\pl{} eggs-\nom{} platform-\obl{}\\
\glt `(The) eggs are on the platform.' 
\end{xlist}
\end{exe}

\pagebreak

\begin{exe}\ex\langinfo{Tennet}{}{\citealp[236]{Randal:1998}}
\begin{xlist}
\ex\label{TenExist}\gll\textipa{\'any\'ak} \textbf{\textipa{m\'am}} \textipa{c\'{\=*e}\'{\=*e}z-a}\\
have water house-\obl{}\\
\glt `There is water in the house.' 

\ex\label{TenNeg}\gll\textipa{\=*ill\'{\=*o}\'{\=*I}} \textbf{\textipa{m\'am}} \textipa{c\'{\=*e}\'{\=*e}z-a}\\
absent water house-\obl{}\\
\glt `There's no water in the house.' 
\end{xlist}
\end{exe}\is{existential predication!positive versus negative|)}

The Nilo-Saharan data are summarized in Table~\ref{NiloOverviewExistLoc}. 
The data from Maa\il{Maa} and Tennet\il{Tennet} are split up between two lines for each of the two languages. 
For Maa\il{Maa}, the first line represents the construction with \emph{tii} `be at', while the second line represents the construction with \emph{ata} `have'. 
In Tennet\il{Tennet}, the two lines represent the inter-speaker variation regarding which construction to use for positive existentials. 
The table shows that all languages use nominative\is{case!individual forms!nominative} case for locational subjects. 
Most languages also make use of the nominative\is{case!individual forms!nominative} for existential subjects, but in this context more variation is found. 
A distinction in encoding between negative and positive existenials\is{existential predication!positive versus negative} in only found in Turkana\il{Turkana} and with some Tennet\il{Tennet} speakers. 
While in Turkana\il{Turkana} negative existentials receive Nominative\is{case!individual forms!nominative} case-marking, in Tennet\il{Tennet} this context is zero-coded. 
\begin{table}[ht]
\centering
\begin{tabular}{lccc}
\hline \hline
\bfseries language&\bfseries S exist. (+)&\bfseries S exist.(-)&\bfseries S loc. pred.\\
\hline
Datooga\il{Datooga}&\textbf{\nom{}}&\textbf{\nom{}}&\textbf{\nom{}}\\
%\hdashline
Maa\il{Maa} (\textit{be at})&\textbf{\nom{}}&{-}&\textbf{\nom{}}\\
{Maa\il{Maa} (\textit{have})}&\acc{}&\acc{}&n.a.\\
%\hdashline
Murle\il{Murle}&\textbf{\nom{}}&{-}&\textbf{\nom{}}\\
%\hdashline
Nandi\il{Nandi}&\textbf{\nom{}}&{-}&\textbf{\nom{}}\\
%\hdashline
Tennet\il{Tennet} (\textit{variety 1})&\acc{}&\acc{}&\textbf{\nom{}}\\
{Tennet\il{Tennet} (\textit{variety 2})}&\textbf{\nom{}}&\acc{}&\textbf{\nom{}}\\
%\hdashline
Turkana\il{Turkana}&\acc{}/\textbf{\nom{}}&\textbf{\nom{}}&\textbf{\nom{}}\\
\hline \hline
\end{tabular}
\caption{Overview of the marking of existential and locational predication in the Nilo-Saharan languages}\label{NiloOverviewExistLoc}%\\
\end{table}

\section{Afro"=Asiatic}\label{ExistAfro}

For the Afro"=Asiatic marked"=S languages, very little information on existential and locational predications is given in the relevant grammars. 
Most of the data given in the following were gathered by extensively studying all examples provided throughout the grammars and trying to identify the ones with locational or existential meanings.
The data that could be gathered on the relevant contexts did not reveal any remarkable patterns. 
Whether a grammar provided data on existentials, locationals or both types, the subject element always was marked with the S-case. 
A minor exception to this pattern was attested in Harar\il{Oromo (Harar)} Oromo and will be discussed below.
Also, no variation between negative and positive\is{existential predication!positive versus negative} contexts could be identified in any Afro"=Asiatic language, but then again, hardly any negative examples were found at all.

The only Afro"=Asiatic language in which alternations in case-marking on the S argument of existential and locationals can be observed is the Harar\il{Oromo (Harar)} dialect of Oromo. 
The subject of locational phrases is normally in Nominative\is{case!individual forms!nominative} case, especially when definite\is{argument!definiteness of} (\ref{HarLoc}b). 
In some situations, the emphatic subject form is used (\ref{HarLoc}c) and thus no Nominative\is{case!individual forms!nominative} case-marking occurs on the subject. 
The construction with the emphatic subject-marker appears to be limited to the existential reading, but this might just be a tendency parallel to the correlation between indefiniteness and existential reading observed by \citet{Clark:1978} and not an absolute selectional restriction. 
For negative contexts, no examples were found.

\begin{exe}
\ex\label{HarLoc}\langinfo{Oromo (Harar)}{Eastern Cushitic; Ethiopia}{\citealp[101, 109]{Owens:1985}} 
\begin{xlist}
\ex\gll c'uf-t\'ii x\'ees\'a jir-an\\
all-\nom{} in exist-\pl{}\\
\glt `All are inside.'
\ex \gll kitaab\textbf{-n\'ii} miiz\'a rr\'a jira\\
book-\nom{} table on exist\\
\glt `The book is on the table.'
\ex \gll miiz\'a rr\'a \textipa{kitaab\'aa}-t\'uu jira\\
table on book-\emphat.\sbj{} exist\\
\glt `There is a book on the table.' 
\end{xlist}
\end{exe}

%\end{exe} [101]

%\begin{exe}\ex\gll nam-n\'ii dures-\'ii man\'a bite \'as jira\\
%man-\nom{} rich-\nom{} house bought here exist\\
%`the rich man who bought the house is here'
%\end{exe} [101] 

In K'abeena\il{K'abeena} (\ref{KabExist}), locationals as well as existentials  mark the S argument in Nominative\is{case!individual forms!nominative} case.
Also, there does not seem to be any alternation between positive and negative sentences -- unless the non-accessible negative existentials reveal an alternative pattern. 
However, since the same verb is used for existential and locational predications, even though \emph{yoo} is sometimes glossed as `to exist' and sometimes as `to be located' by \citet{Crass:2005}, negative existentials very likely employ the same pattern as negative locationals. 
 
%\enlargethispage{2\baselineskip}
\pagebreak
\begin{exe}\ex\label{KabExist}\langinfo{K'abeena}{Eastern Cushitic; Ethiopia}{\citealp[98, 115, 98]{Crass:2005}}
\begin{xlist}
\ex\gll wiim\textsuperscript{u} \textbf{'abogod\'aa'nut\textsuperscript{i}} yoo-s\textsuperscript{i}\\
many friends.\nom{}.\pl{} exist.3-3\sg{}.\mas{}.\obj{}\\
\glt `He has got many friends,' lit.:`Many friends are to him.'\\ 
original translation: `Er hat viele Freunde'\\
lit.: `Viele Freunde exis\-tie\-ren [bei] ihm.')

\ex\gll \textbf{m\'ancu-s\textsuperscript{e}} bokku yoo\\
man.\nom{}-\defsc{}.\mas{} house.\acc{} be\_located.3\\
\glt `The man is in the house.' \\
original translation: `Der Mann ist im Haus.'

\ex\gll wolk'itt'een\textsuperscript{i} tees\textsuperscript{u} \textbf{wuu} yoo-ba\\
Wolkite.\loc{} now water.\nom{} exist.3-\Neg{}\\
\glt `In Wolkite there is no water right now.'\\ 
original translation: `In Wolkite gibt es jetzt kein Wasser.'
\end{xlist}
\end{exe}

%\citep[115]{Crass:2005}

%\begin{exe}\ex\gll saa-s\textsuperscript{e} '\'azut\textsuperscript{i} yoo-s\textsuperscript{e}\\
%cow.\dat{}-\defsc{}.\fem{} milk.\nom{} exist.3-3\sg{}.\fem{}.\obj{}\\
%`The cow has got milk.' (literally: `Milk exists to the cow.')\\
%Original translation: `Die Kuh hat Milch.' (w\"ortlich: `Milch existiert der Kuh.')
%\end{exe}

%more examples on page 115 with special verb `to be located/exist'

%\begin{exe}\ex\gll m\'ancu-s\textsuperscript{e} bokku yoo\\
%wife.\nom{}-\defsc{}.\fem{} house.\acc{} be\_located.3\\
%`The woman/wife is inside the house.'
%Original translation: `Die Frau ist im Haus.'
%\end{exe} 

In Arbore\il{Arbore}, only examples of the existential predication could be identified. 
The subject of this construction in the Nominative\is{case!individual forms!nominative} case as demonstrated in (\ref{ArboreExistF}).\footnote{The Nominative form \emph{\textglotstop iNgir\'e} is distinct from the zero-coded form of the noun \emph{\textglotstop ingir}.}

\begin{exe} 
\ex\label{ArboreExistF}\langinfobreak{Arbore}{Eastern Cushitic; Ethiopia}{\citealp[132]{Hayward:1984}}
\gll \textglotstop iNgir\textbf{-\'e} \textglotstop a-y g\'irta\\
louse-\nom{} \pvs{}-3\sg{} exist.3\sg{}.\fem{}\\
\glt `There is a louse.'
\end{exe}

For Boraana\il{Oromo (Boraana)} Oromo (\ref{BorLoc}), Gamo\il{Gamo} (\ref{GamLoc}), and Wolaytta\il{Wolaytta} (\ref{WolLoc}), only locational examples could be extracted from the grammatical descriptions.
The Nominative case is always used to encode the S argument.

\begin{exe}\ex\label{BorLoc} \langinfobreak{Oromo (Boraana)}{Eastern Cushitic; Kenya}{\citealp[54]{Owens:1982}}
\gll n\`am\textbf{-i} j\`ann\textbf{-i} d\`ibi\textbf{-in} s\`un arm j\`ir \\
man-\nom{} brave-\nom{} other-\nom{} \dem{} here be\\
\glt `That other brave man is here.' 
\end{exe}
%In locative expressions nouns are in the Nominative form \citep[52]{Stroomer:1995}
%\begin{exe}\ex\begin{xlist}
%\ex\gll obboleetti-ni tiya worra jir-ti\\
%sister-\nom{} 1\sg{}.\poss{} home be-3\sg{}.\fem{}.\prs{}\\
%`My sister is home.'
%\ex \gll innii worra taa\\
%3\sg{}.\nom{} house stay.3\sg{}.\mas{}.\prs{}\\
%`He is at home.'\end{xlist}
%\begin{flushright}\citet[69,~52]{Stroomer:1995}\end{flushright}
%\end{exe}

%\pagebreak 
\begin{exe}\ex\label{GamLoc} \langinfobreak{Gamo}{Omotic; Ethiopia}{\citealp[384]{Hompo:1990}}
 \gll iza na\textglotstop i-t\textbf{-ii} go\v s\v san\v ca-z-aa-ko-n d-ettes\\
his child-\pl{}-\nom{} peasant-\defsc{}-\acc{}-\com{}-\loc{} be-\complx{}\\
\glt `His children are around the peasant.'
\end{exe}

\begin{exe}\ex\label{WolLoc} \langinfobreak{Wolaytta}{Omotic; Ethiopia}{\citealp[587, ex: 332]{Lamberti:1997}}
\gll eet\textbf{-i} banta horaatta keettaa-n\textsuperscript{i} de\textglotstop osoona\\
3\pl{}-\nom{} 3\pl{}.\poss{} new house-\pl{}-\loc{} be.3\pl{}\\
\glt `They are in their new houses.' 
\end{exe}

For Zayse\il{Zayse}, finally, it was not completely clear whether the only relevant sentence that could be found should be classified as an existential, as the English translation suggests, or rather as a locational. 
Regardless of this question, the construction demonstrated uses the Nominative\is{case!individual forms!nominative} case for the S argument (\ref{ZayLoc}). 

\begin{exe}\ex\label{ZayLoc}\langinfobreak{Zayse}{Omotic; Ethiopia}{\citealp[261]{Hayward:1990}}
\gll \textglotstop a\`s\textbf{-\'\i} kar\'a yesatte\\
man-\nom{} indoors exist-\cop{}\\
\glt `There is a man in the house.'
\end{exe}

The data are summarized in Table~\ref{AfroOverviewExistLoc}. 
There are a lot of missing data for the Afro"=Asiatic languages on the contexts of existential and locational predications. 
Therefore, any tendencies described here have to be viewed as a preliminary result. 
The languages of this family encode existential as well as locational subjects in the nominative\is{case!individual forms!nominative} case. 
No differences between the encoding of subjects in positive and negative existential predications could be found in the Afro"=Asiatic marked"=S languages.
\enlargethispage{\baselineskip}

\begin{table}[ht]
\centering
\begin{tabular}{lccc}
\hline \hline
\bfseries language&\bfseries S exist. (+)&\bfseries S exist.(-)&\bfseries S loc. pred.\\
\hline
Arbore\il{Arbore}&\textbf{\nom{}}&{-}&{-}\\
%\hdashline
Gamo\il{Gamo}&{-}&{-}&\textbf{\nom{}}\\
%\hdashline
K'abeena\il{K'abeena}&\textbf{\nom{}}&\textbf{\nom{}}&\textbf{\nom{}}\\
%\hdashline
Oromo (Boraana\il{Oromo (Boraana)})&{-}&{-}&\textbf{\nom{}}\\
%\hdashline
Oromo (Harar\il{Oromo (Harar)})&emphatic subjet&{-}&\textbf{\nom{}}\\
%\hdashline
Wolaytta\il{Wolaytta}&{-}&{-}&\textbf{\nom{}}\\
%\hdashline
Zayse\il{Zayse}&\textbf{\nom{}}&{-}&\textbf{\nom{}}\\
\hline \hline
\end{tabular}
\caption{Overview on the marking of existential and locational predication in the Afro"=Asiatic languages}\label{AfroOverviewExistLoc}%\\
\end{table}

\section{North America}\label{ExistNA}

The marked"=S languages of North America tend to have no dedicated constructions for encoding existential and locational predications. 
They usually employ stative verbs\is{verb class!positional verb} in these contexts. 
However, at least for the Yuman languages, the option not to use the S-case on subjects in existential contexts seems to exist, or even to be preferred or obligatory for some languages. This is generally the case if the S-case is an optional marker (cf. also Section~\ref{usage-based}). 
This tendency is in accordance with the claim by \citet[123]{Payne:1997} that existentials mark grammatical relations only to a limited degree.

Mojave\il{Mojave} has been shown in Section~\ref{dataexist} to use the same type of construction for locational (\ref{MojExLoc}a) and existential predications (\ref{MojExLoc}b).
In this construction, a number of stative verbs can occur, and the S argument is usually encoded with the Nominative\is{case!individual forms!nominative} case. 
Hence, these contexts can best be analyzed as being regular intransitive clauses.
Negative existentials also exhibit this intransitive pattern with Nominative marking on the S argument (\ref{MojExLoc}c).  


\begin{exe}\ex\label{MojExLoc}\langinfo{Mojave}{Yuman; Arizona}{\citealp[21, 212, 70]{Munro:1976}}
 \begin{xlist}
\ex \gll\textipa{pi:pa\textbf{-\v c}} \textipa{k\super{w}@\v ca:nava:-l\super{y}} \textipa{uwa:-k}\\
person-\nom{} Yuma-\loc{} be\_in-\tns{}\\
\glt `There is someone in Yuma \dots' 
%\end{exe}
%\begin{flushright}\citet[21]{Munro:1976}\end{flushright}
%\gll\textipa{hukTar-\v c} \textipa{v-iv\textglotstop aw-m}\\
%coyote-\nom{} here-stand-\tns{}\\
%`There's a coyote standing there.'
%\end{exe} \begin{flushright}\citet[49]{2Mojave\il{Mojave}}\end{flushright}

\ex\gll\textipa{pi:pa} \textipa{n\super{y}amaTa:m} \textipa{k\super{w}@loyaw} \textipa{k\super{w}-tapoy-h-n\super{y}{-\v c}} \textipa{\textglotstop ava:-l\super{y}} \textipa{iva-m}\\
person tomorrow chicken \relativ{}-kill-\irr{}-\dem{}-\nom{} house-\loc{} sit-\tns{}\\
\glt `The man who's going to kill the chicken tomorrow is in the house.'
%\end{exe} \begin{flushright}\citet[212]{Munro:1976} \end{flushright}

\ex\gll\textipa{hat\v coq} \textipa{havasu:{-\v c}} \textipa{kava:r-ta:han-e}\\
dog blue-\nom{} not-very-\augv{}\\
\glt `There are no blue dogs'
%\end{exe} \begin{flushright}\citet[70]{Munro:1976}\end{flushright}
\end{xlist}
\end{exe}

%compare with the following example:
%\begin{exe}\ex\gll\textipa{hat\v coq-\v c} \textipa{havasu:-p\v c}\\
%dog-\nom{} blue-\tns{}\\
%`The dog is blue.'
%\end{exe} 
%\begin{flushright}\citet[188]{Munro:1976}\end{flushright} 
%
%\begin{exe}\ex\gll\textipa{n\super{y}a-m-iyem-m} \textipa{maka-\v c} \textipa{ido-mpot\v c}\\
%when-2-go-\dsbj{} someone-\nom{} be-\Neg{}\\
%`Nobody came while you were gone'; `When you were gone there was nobody (here)'
%\end{exe} 
%\begin{flushright}\citet[52]{Munro:1976}\end{flushright}
%
%\begin{exe}\ex\gll\textipa{\textglotstop avi:-\v c} \textipa{va:r-t-k}\\
%money-\nom{} not-\emphat{}-\tns{}\\
%`It's not expensive'; `the money's not (there)'
%\end{exe}
%\begin{flushright}\citet[57]{Munro:1976} translated as `there's no money in it' in example (359) page 70 \end{flushright}
%
%\begin{exe}\ex\gll\textipa{hukTar-\v c} \textipa{\textglotstop avi:-T-l\super{y}} \textipa{idi:-k}\\
%coyote-\nom{} mountain-\dem{}-\loc{} lie-\tns{}\\
%`There are coyotes in those hills.' 
%\end{exe}
%\begin{flushright}\citet[33]{Munro:1976}\end{flushright}
%
%\begin{exe}\ex\gll\textipa{pi:pa-\v c} \textipa{k\super{w}@\v ca:nava:-l\super{y}} \textipa{uwa:-k}\\
%person-\nom{} Yuma-\loc{} be\_in-\tns{}\\
%`There is someone in Yuma \dots' 
%\end{exe}
%\begin{flushright}\citet[21]{Munro:1976}\end{flushright}
%
%Existential in subordinate clause?
%
%\begin{exe}\ex\gll\textipa{su:va:r} \textipa{\textglotstop asent-m} \textipa{\textglotstop-su:paw-m}\\
%song one-\dsbj{} 1-know-\tns{}\\
%`I only know one song.'; `There's only one song I know.'
%\end{exe} 
%\begin{flushright}\citet[40]{Munro:1976}\end{flushright}
%
%But there is Nominative case-marking in subordinate clauses as in (b):
%\begin{exe}\ex
%\begin{xlist}
%\ex\gll\textipa{\textglotstop avu:ya-n\super{y}} \textipa{m-supet-m} \textipa{\textglotstop-a\textglotstop a:v-k}\\
%door-\dem{} 2-close-\dsbj{} 1-hear-\tns{}\\
%`I heard you close the door'; `When you closed the door, I heard it.'
%\ex\gll\textipa{\textglotstop avu:ya-n\super{y}-\v c} \textipa{s@pet-m} \textipa{\textglotstop-a\textglotstop a:v-k}\\
%door-\dem{}-\nom{} close-\dsbj{} 1-hear-\tns{}\\
%`I heard the door close'; `When the door closed, I heard it.'
%\end{xlist}
%\end{exe}
%\begin{flushright}\citet[40]{Munro:1976}\end{flushright}
% 
%locative clauses: see (4) in \citet[100]{Munro.subj:1976}
%
%\begin{exe}\ex \gll \textipa{\textglotstop i:pa-\v c} \textipa{\textglotstop asent-k} \textipa{\textglotstop ava-l\super y} \textipa{u:va-m}\\
%man-\nom{} one-same house-\loc{} located-\tns{}\\
%`One man is in the house.'
%\end{exe}
%
%\begin{exe}\ex\gll\textipa{hukTar-\v c} \textipa{\textglotstop avi:-T-l\super{y}} \textipa{idi:-k}\\
%coyote-\nom{} mountain-\dem{}-\loc{} lie-\tns{}\\
%`There are coyotes in those hills.' 
%\end{exe}
%\begin{flushright}\citet[33]{Munro:1976}\end{flushright}
%
%
%\begin{exe}\ex\gll\textipa{\textglotstop ave:-\v c} \textipa{@-l\super{y}-u:nu-k}\\
%mouse-\nom{} EPTH-\loc{}-be\_in-\tns{}\\
%`There is a mouse there.'
%\end{exe} 
%\begin{flushright}\citet[171]{Munro:1976}\end{flushright}
%
%Munro treats the above example as an instance of a zero-pronoun marked with the Locative case and then incorporated into the verb (an epethetic vowel is inserted in place of the zero-pronoun).
%
%\begin{exe}\ex\gll\textipa{pa-\v c} \textipa{pay} \textipa{iva:-k-@}\\
%person-\nom{} all arrive-\tns{}-AUG\\
%`Everyone is here'; `Everyone came'
%\end{exe} 
%\begin{flushright}\citet[82]{Munro:1976}\end{flushright}
%
%Whole relative clause serves as subject of the locational predication. 
%
%\begin{exe}\ex\label{MojRelSub}\gll\textipa{pi:pa} \textipa{n\super{y}amaTa:m} \textipa{k\super{w}@loyaw} \textipa{k\super{w}-tapoy-h-n\super{y}-\v c} \textipa{\textglotstop ava:-l\super{y}} \textipa{iva-m}\\
%person tomorrow chicken \relativ{}-kill-\irr{}-\dem{}-\nom{} house-\loc{} sit-\tns{}\\
%`The man who's going to kill the chicken tomorrow is in the house.'
%\end{exe} 
%\begin{flushright}\citet[212]{Munro:1976} \end{flushright}
%
%The same is true in questions:
%
%\begin{exe}\ex\gll\textipa{maka-\v c} \textipa{\textglotstop ava:-l\super{y}} \textipa{iva}\\
%who-\nom{} house-\loc{} sit\\
%`Who's in the house?'
%\end{exe} 
%\begin{flushright}\citet[87]{Munro:1976}\end{flushright}
%
%\begin{exe}\ex\gll\textipa{pa-\v c} \textipa{k-l\super{y}@vi:-k} \textipa{\textglotstop ava:-l\super{y}} \textipa{u:nu}\\
%person-\nom{} \question{}-like-\ssbj{} house-\loc{} be\_in\\
%`How many people are in the house.'
%\end{exe} 
%\begin{flushright}\citet[91]{Munro:1976}\end{flushright}

%\pagebreak

Out of the dozens of existential examples that I found for Mojave\il{Mojave}, the S argument is always in the Nominative case, with one exception.
Example (\ref{MojZeroExist}) suggests that the Nominative case-marker can be missing on this argument. 
Since \citet{Munro:1976} notes the optionality\is{case-marking!optional} of the Nominative case-marker, this is no surprise. 

\enlargethispage{2\baselineskip}

\begin{exe}\ex\label{MojZeroExist}\langinfobreak{Mojave}{}{\citealp[70]{Munro:1976}}
\gll\textipa{n\super{y}a-v-k} \textipa{\textbf{\textglotstop aha:}} \textipa{kava:r-k}\\
this-\dem{}-\loc{} water not-\tns{}\\
\glt `There's no water here'
\end{exe} 

In Jamul\il{Jamul Tiipay} Tiipay S arguments of presentational clauses are always zero-coded (\ref{JamExist}a), whereas in locational contexts Nominative\is{case!individual forms!nominative} marking does occur (\ref{JamExist}b). 
In the closely-related language Diegue\~no\il{Diegue\~no (Mesa Grande)}\footnote{Until recently, Jamul\il{Jamul Tiipay} Tiipay was treated as a dialect of Diegue\~no.}, S arguments of existential clauses are apparently also zero-coded (\ref{DieExLoc}a). Whether they also allow for encoding of the S argument in the Nominative like locational clauses (\ref{DieExLoc}b) is not clear, since unfortunately none of the materials on the language give information on this question. 

\begin{exe}\ex\label{JamExist}\langinfo{Jamul Tiipay}{Yuman; Mexico}{\citealp[231]{Miller:2001}; \citealp[148]{Miller:1990}}
 \begin{xlist}
\ex\gll \textbf{toor} tewa-ch u-wiiw\\
bull be\_located-\ssbj{} 3-see\\
\glt `There was a bull there, and he saw (the boy).'
%\end{exe} \begin{flushright}\citet[334]{Miller:2001}\end{flushright}

\ex\gll nyaa\textbf{-ch} peyii ta'-wa-ch-pu puu-ch ny-u'yaaw\\
1-\nom{} here 1-be\_located-\NR{}-\dem{} 3$>$1-see {???}\\
\glt `She knows that I was there.' 
%\end{exe} \begin{flushright}\citet[148]{Miller:1990}\end{flushright}
\end{xlist}
\end{exe}

%\begin{exe}\ex\gll\textipa{puu-ch} \textipa{kaamp-lly} \textipa{nyewaayk} [+nyeway-(a/-chu'u)]\\
%those-\nom{} Campo-\loc{}	live [+be\_locd-(-\question{}/-\question{})]\\
%`They used to live in Campo, didn't they.' 
%\end{exe}
%\begin{flushright}\citet[129]{Miller:1990}\end{flushright}

\begin{exe}\ex\label{DieExLoc}\langinfo{Diegue\~no (Mesa Grande)}{Yuman; California}{\citealt[27]{Gorbet:1976}, \citealt[176]{Langdon:1970}}
\begin{xlist} 
%comp. ex.(66) \citep[27]{Gorbet:1976}
\ex\gll \textbf{'i:k\textsuperscript{w}ic} 'xin n\textsuperscript{y}wa:yp t+wa: i:tay+pu+i\\
man one live \prog{}+be\_sitting forest+\dem{}+\loc{}\\
\glt `There was a man who lived in the forest.'
%\end{exe}
%locative clauses: Subject coding \citep[176]{Langdon:1970} 
\ex\gll \textglotstop ik\textsuperscript{w}ic pu\textbf{=c} n\textsuperscript{y}uk pa\\
man that\_one=\nom{} already he\_got\_here\\
\glt `That man is already here.'
\end{xlist}
\end{exe}

For Yavapai\il{Yavapai}, all examples listed in the grammar are of an existential nature if one takes the English translation into account. 
 Whether a locational reading is also possible cannot conclusively told from the information in the grammar. 
All S arguments are in the Nominative\is{case!individual forms!nominative} case. 
And finally, in Havasupai\il{Havasupai}, only one example was found, which is existential according to the translation provided. 
In this example the S argument is in the Nominative\is{case!individual forms!nominative} case (\ref{HavEx}).

\begin{exe}\ex\langinfo{Yavapai}{Yuman; Arizona}{\citealp[28, 98]{Kendall:1976}}
\begin{xlist} 
\ex\gll\textipa{h\super{w}at\textbf{-c}} \textipa{viya-k} \textipa{wi-o-m}\\
blood-\nom{} here-\loc{} have-\appl{}-\faff{}\\
\glt `There is blood here.' %\citet[28]{Kendall:1976} \end{exe}

\ex\gll\textipa{cmyul} \textipa{\~n-wa:-c-\textbf{c}} \textipa{via} \textipa{\textglotstop-wa:-v-m} \textipa{pay-a} \textipa{yo:}\\
ant \poss{}-house-\pl{}-\nom{} here 1-house-\dem{}-around all-\tns{} exist\\
\glt `There are ant hills all around my house.'
%\citet[98]{Kendall:1976}\end{exe}

\ex\gll\textipa{cnapuk\textbf{-c}} \textipa{miyyul-l} \textipa{yu-m}\\
red\_ant-\nom{} sugar-\loc{} be-\faff{}\\
\glt `There is an ant in the sugar.'
%\citet[98]{Kendall:1976}
\end{xlist}
\end{exe}


\begin{exe}\ex\label{HavEx}\langinfobreak{Havasupai}{Yuman; Arizona}{\citealp[61]{Kozlowski:1972}}
\gll pa-c\textbf{-v} hlah-l \texttheta a-l yu-k-yu\\
man-\dem{}-\nom{} moon-\loc{} there-in be-\ind{}-\aux{}\\
\glt `There is a man in the moon.' \end{exe}

The Wappo\il{Wappo} data provide a mirror image of the Yavapai\il{Yavapai} situation. 
The translations suggest an existential reading of the following examples.
Possibly, examples such as (\ref{WapExist}c) can also be interpreted as locationals, depending on the previous discourse, but the grammar does not provide any information on this topic.
One reason for this might be the lack of textual data due to the fact that, as \citet{Thompsonetal:2006} note in the introduction to their grammar, their informant (and last speaker of the language) did not enjoy working on narratives. 
Be that as it may, all the examples have Nominative\is{case!individual forms!nominative} S arguments, and no difference is made between positive (\ref{WapExist}a) and negative\is{existential predication!positive versus negative} clauses (\ref{WapExist}b).

\begin{exe} \ex\label{WapExist}\langinfo{Wappo}{Wappo-Yukian; California}{\citealp[105]{Thompsonetal:2006}}
\begin{xlist}
 \ex\gll pol'\textbf{-i} \v oi:-khi\textglotstop\\
 dirt-\nom{} exist-\stat{}\\
\glt `There's a bucket of dirt.'
 %\end{exe}[105]
 %\ex\gll layh-te eniya{\textglotstop} le\textglotstop a-khi{\textglotstop} cew\\
 %white-\pl{} too many-\stat{} there\\
 %`There are too many whites there.'
% \end{exe} [104] 
 %negative:
  \ex\gll heta hut'\textbf{-i} la-khi\textglotstop\\
 here coyote-\nom{} missing-\stat{}\\
\glt `There aren't any coyotes here.'
 %\end{exe} [104] 
% with location:
 \ex\gll c'ic'a-t\textbf{-i} hol-wil'uh le\textglotstop a-hki\textglotstop\\
 bird-\pl{}-\nom{} tree-on many-\stat{}\\
\glt `There are lots of birds on the tree.'
 %\end{exe} [104]
 
 %\eX\gll ew-i e\v cum-uh yo\textglotstop-khi\textglotstop\\
 %fish-\nom{} river-\loc{} exist-\stat{}\\
 %`there are fish in the river'
% [105]
\end{xlist}
\end{exe}

In Maidu\il{Maidu} the S argument of existential (\ref{MaiExist}) and locational clauses (\ref{MaiLoc}) is marked with Nominative\is{case!individual forms!nominative} case.  

\begin{exe}\ex\label{MaiExist}\langinfobreak{Maidu}{Maiduan; California}{\citealt[10]{Maidu:1912}}
\gll ``\textipa{Unu\~n'} \textipa{ko'doi-di} \textipa{kan} \textipa{sede\textbf{-m'}} \textipa{uma'pem},'' \textipa{atsoi'a}\\
this word-\loc{} and blood-\nom{} shall\_exist say.\pst{}.3\sg{}\\
\glt ` ``There shall be blood in the world'', he said' 
%\citet[10]{Maidu:1912}
\end{exe}
 
%\begin{exe}\ex\gll\textipa{k\^ad'om-tsanan'te-di} \textipa{atsoi'am} \textipa{hin'wo-m} \textipa{mai'd\"u-m} \textipa{p\={i}-m} \textipa{atsoi'a}\\
%north-?-\loc{} ? first-\nom{} person-\nom{} many-\nom{} ?\\
%`In the north, it is said, there were many first people.'
%\end{exe}
%\citet[140f.]{Maidu\il{Maidu}:1912}

\begin{exe}\ex\label{MaiLoc}\langinfobreak{Maidu}{}{\citealp[58]{Shipley:1964}}
\raggedright\gll {\textglotstop}ad\'om my-k\textraiseglotstop\'i p\'andak-\textbf{am} kyk\'ym ma-{\textglotstop}\'am mym\'y-k kap\'ota-m k\textraiseglotstop an\'ajdi\\
then \dem{}-\gen{}   rifle-\nom{}  \pstrem{} be-PT\_PST.3 he-\gen{} coat-\nom{} under\\
\glt `His rifle was there under his coat.' 
\end{exe}

A summary of the data is provided in Table~\ref{NAOverviewExistLoc}. 
In Jamul\il{Jamul Tiipay} Tiipay and Diegue\~no\il{Diegue\~no (Mesa Grande)}, existential subjects are in the zero-coded accusative\is{case!individual forms!accusative}. 
Both languages mark locational subjects with the Nominative. 
All other marked"=S languages of this area appear to use the nominative case in locational as well as existential contexts. 
Although for many of the examples, the grammars give an existential translation, the same sentences can probably also be translated as locationals in a given contexts since they employ regular intransitive verbs\is{verb class!positional verb} such as `sit/stand/lie', and in many cases a locational phrase is added. 
For none of the languages of this area was any variation found in the subject-marking of positive and negative\is{existential predication!positive versus negative} existentials.


\begin{table}[ht]
\centering
\begin{tabular}{lccc}
\hline \hline
\bfseries language&\bfseries S exist. (+)&\bfseries S exist.(-)&\bfseries S loc. pred.\\
\hline
Diegue\~no\il{Diegue\~no (Mesa Grande)} (Mesa Grande) &\acc{}&{-}&\textbf{\nom{}}\\
%\hdashline
Havasupai\il{Havasupai}&\textbf{\nom{}}&{-}&\textbf{\nom{}}\\
%\hdashline
Jamul\il{Jamul Tiipay} Tiipay&\acc{}&{-}&\textbf{\nom{}}\\
%\hdashline
Mojave\il{Mojave}&\textbf{\nom{}}&\textbf{\nom{}}&\textbf{\nom{}}\\
%\hdashline
Yavapai\il{Yavapai}&\textbf{\nom{}}&{-}&\textbf{\nom{}}\\
%\hdashline
Maidu\il{Maidu}&\textbf{\nom{}}&{-}&\textbf{\nom{}}\\
%\hdashline
Wappo\il{Wappo}&\textbf{\nom{}}&\textbf{\nom{}}&\textbf{\nom{}}\\
\hline \hline
\end{tabular}
\caption{Overview of the marking of existential and locational predication in the languages of North America}\label{NAOverviewExistLoc}%\\
\end{table}

\section{Pacific}\label{ExistOc}

The languages of the Pacific region, though there are only three of them with informative data in my sample, exhibit the most interesting patterns with regard to existential and locational predications.
All languages have at least two different constructions to encode this domain of grammar. 
The semantic distinctions that individual constructions encode vary to quite an extent between the languages. 
Differences between negative and positive\is{existential predication!positive versus negative} contexts are wide-spread in this very limited selection of Austronesian and non-Austronesian languages of this region. 

The\is{existential predication!positive versus negative|(} distinction between positive and negative existentials in Nias\il{Nias} has already been demonstrated in Section~\ref{dataexist}.
Now I will discuss the data in more detail.
In Nias\il{Nias}, existential and locational predications use parallel constructions (possessive constructions use the same pattern as well).
For both types of predication, there is one construction that is used for positive sentences (existence, location) and another one for negative ones (non-existence, absence).
Positive existential/locational constructions are built with the verb \emph{ga} which takes the Mutated form of the noun it predicates over (\ref{NiaPosExist}).  
Negative existential/locational constructions contain the verb \emph{l\"ona}\footnote{\textit{Löna} is also the form of the standard verbal negator in Nias\il{Nias}.  
When used as verbal negator, the case-marking is the same as it would be with the non-negated verb \citep[471--475]{Brown:2001}.}, which takes a noun in Unmutated form (\ref{NiaNegExist}).

%\enlargethispage{\baselineskip}
\begin{exe} \ex\label{NiaPosExist}\langinfo{Nias}{Sundic; Sumatra, Indonesia}{\citealp[344, 570]{Brown:2001}}
\begin{xlist} \ex \gll ga so \textbf{g\"ocoa}\\
here exist cockroach.\mut{}\\
\glt `There's a cockroach here.'

\ex \gll so \textbf{nono}-nia do-\textbf{mbua}\\
exist child.\mut{}-3\sg{}.\poss{} two-\clf{}.\mut{}\\
\glt `She has two children.'
\end{xlist}
\end{exe}

%{Nias\il{Nias}} \citep[358,~575]{Brown:2001}
\begin{exe} \ex\label{NiaNegExist}\langinfo{Nias}{}{\citealp[358, 575]{Brown:2001}}
\begin{xlist} \ex \gll l\"ona \textbf{ba{\ss}i} ba mbanu ha'a\\
exist.\Neg{} pig \loc{} village.\mut{} \prox{}\\
\glt `There are no pigs in this village.'

\ex \gll l\"ona \textbf{ono}-nia.\\
exist.\Neg{} child-3\sg{}.\poss{}\\
\glt `She doesn't have any children.'
\end{xlist} 
\end{exe}\is{existential predication!positive versus negative}


Aji\"e\il{Aji\"e} is an Austronesian language from New Caledonia. 
It has two positive existential constructions, one positive locational construction, and one construction used for both negative existentials and locationals.
First there is the `unmarked' existential verb \emph{wii/wi} (\ref{AjiExist},\ref{AjiExLoc}).  
With this verb, the Nominative\is{case!individual forms!nominative} marker\is{case-marking!via adposition} is optionally used \citep[109]{Lichtenberk:1978}.

\pagebreak
\begin{exe}\ex\langinfo{Aji\"e}{Oceanic; New Caledonia}{\citealp[102]{Lichtenberk:1978}}
\begin{xlist}
\ex\label{AjiExist}\gll\textipa{na} \textipa{wii} \textbf{\textipa{rha}} \textbf{\textipa{m2\textglotstop u}} \textipa{ka} \textipa{kani} \textipa{@}\\
3\sg{} exist one yam \relativ{} grow good\\
\glt `There is a yam that grows good.' 
\ex\label{AjiExLoc}\gll\textipa{wi} \textipa{tO-wE} \textbf{\textipa{na}} \textbf{\textipa{p\~u-\~e}}?\\
exist be\_where \nom{} trunk-3\sg{}\\
\glt `Where is its trunk?' 
\end{xlist} 
\end{exe}

Apart from this construction, there is the human existential verb \emph{ta/t\textturnv}. 
The Nominative\is{case!individual forms!nominative} is always used to mark the subject with this verb (\ref{AjiExHum}). 
In addition, there is a locational verb \emph{t\textopeno} `be at a place'.
Most examples given of this verb do not have an overt subject nominal. 
Those that do have one mark it with the Nominative preposition \emph{na} (\ref{AijLoc}). 
For the negative\is{existential predication!positive versus negative} contexts, the same construction is used for existentials and locationals.
For both non-human and human S arguments in negative existential/locational predications, it is not possible to be marked with the Nominative preposition, thus they are always zero-coded.
Those sentences are constructed with the negative existential \emph{y\textepsilon ri} (\ref{AijLoc2}). 

%\pagebreak

\begin{exe}\ex\langinfo{Aji\"e}{}{\citealp[110]{Lichtenberk:1978}}
\begin{xlist}
\ex\label{AjiExHum}\gll\textipa{na} \textipa{ta} \textipa{m\~a} \textbf{\textipa{na}} \textbf{\textipa{bwe\textglotstop}} \textipa{Ge} \textipa{kaunuaE}\\
3\sg{} exist \pstrem{} \nom{} woman from K.\\
\glt `Long ago there was a woman from Kaunuae.'

\ex\label{AijLoc}\gll\textipa{gE} \textipa{yE} \textipa{ta} \textipa{tO-a} \textbf{\textipa{na}} \textbf{\textipa{gEi}}\\
2\sg{} \prosp{} be be\_there \nom{} you\\
\glt `You are going to stay over there.'
\end{xlist}
\end{exe}

\begin{exe}\ex\label{AijLoc2}\langinfo{Aji\"e}{}{\citealp[110]{Lichtenberk:1978}}
\begin{xlist}
\ex\gll\textipa{na} \textipa{yEri} \textbf{\textipa{kamo\textglotstop}} \textipa{rro-i}\\
3\sg{} \Neg{}.exist man in\_there\\
\glt `There was no-one there.' 

\ex\gll\textipa{na} \textipa{yEri} \textbf{\textipa{mwane}}\\
3\sg{} \Neg{}.exist money\\
\glt `There is no money.'
\end{xlist}
\end{exe}

Savosavo\il{Savosavo} -- the only non-Austronesian languages of the Pacific discussed in this chapter -- has a large number of locational constructions. 
Only one of these constructions can have an existential interpretation. 
In this type of locational construction (a subtype of what \citet{Wegener:2008} calls `predicate-subject locational') the predicate is marked by the focus emphatic \emph{=e}, and the following S argument may (\ref{SavExist}a) or may not (\ref{SavExist}b) be marked with the Nominative clitic.
%The second structure appears to allow for an existential reading.
Other than for positive existentials, which do not seem to have a dedicated construction, negative existentials\is{existential predication!positive versus negative} are formed with the verb \emph{baighoza} `not exist'. 
The S argument is marked with the Nominative in this construction (\ref{SavExist}c).

\begin{exe}\ex\label{SavExist}\langinfo{Savosavo}{Solomons East Papuan; Solomon Islands}{\citealp[209, 122]{Wegener:2008}}
\begin{xlist} 
\ex\gll apoi ata=e te lo keva\textbf{=na}\\
because here=\emphat{} \emphat{} \deter{}.\sg{}.\mas{} path=\nom{}\\
\glt `Because here (is ) the road.'
\ex\gll lo lo buringa=la=e edo kola=ga\\
3\sg{}.\mas{} 3\sg{}.\mas{}.\gen{} back=\loc{}.\mas{}=\emphat{} two tree=\pl{}\\
\glt `At his back (are) two trees./ There are two trees at his back.'
%\end{exe} \citet[209]{Wegener:2008}
\ex\raggedright\gll lo mama mau lo-va nanaghiza\textbf{=na} te baighoza-i\\
\deter{}.\sg{}.\mas{} mother father 3\sg{}.\mas{}-\gen{}.\mas{} teaching=\nom{} \emphat{} \Neg{}.exist-\fin{}\\
\glt `The teaching of the parents does not exist (any more)'
%\end{exe} \citet[122]{Wegener:2008}
\end{xlist}
\end{exe}

%\begin{exe}\ex\gll Zu koata=e, baigho=e keda=na\\
%but before=\emphat{} not.exist=\emphat{} fire=\nom{}\\
%`But before, there was no fire.'
%\end{exe}
%\citet[238]{Wegener:2008}

%There are three different types of locational sentences:

%\begin{itemize}
%\item Subject-predicate locationals 
%\item Predicate-subject locationals 
%\begin{itemize}
%\item with particle subject enclitic
%\item with emphatic predicate (subject enclitic impossible)
%\end{itemize}
%\end{itemize}

%Only the last type allows for an existential reading.

%\citet[207ff.]{Wegener:2008} distinguishes between three types of locational sentences. The first devision being between those where the order is subject predicate (those are discussed in the next section) and the ones which have a predicate subject order.
%With the predicate-subject locationals a further subdivision is found. There is a construction where the clause initial predicate is followed by the particle \emph{te} to which the subject pronoun enclitic is attached (possibly followed by a full subject NP with Nominative marking). 
%`Enclitic personal pronouns can only be used for activated, non-emphasized referents,
%therefore the first type of predicate-first locational clause, which involves enclitic pronouns,
%is usually used when the emphasis is on the location as the particularly relevant,
%new information. The subject referent tends to be already established, and the primary
%function of the clause is to specify the location of this referent.' \citep[210]{Wegener:2008} 
%This description indicates that this structure is also of purely locational nature (and hence will be discussed in the next section).
%\subsubsection*{Subject of locative clauses:}

Locational contexts can be expressed with a number of different constructions in Savosavo\il{Savosavo}. 
In addition to the `predicate-subject locational with an emphatic predicate' (\ref{SavLoc}a) already discussed above, there is also the `predicate-subject locational with a particle subject enclitic' (\ref{SavLoc}b) as well as the `subject-predicate locational' construction (\ref{SavLoc}c). 
The S argument of all these locational constructions is in the Nominative case. However, as noted above, zero-coding is possible for the `predicate-subject locational' with an emphatic predicate. %(\ref{SavLoc}a+).

%\pagebreak
\begin{exe}
\ex\label{SavLoc}\langinfo{Savosavo}{}{\citealp[92, 208, 209]{Wegener:2008}}
\begin{xlist}
%\end{exe}
%\citet[92]{Wegener:2008}
%\subsubsection*{Predicate-subject locationals (with particle subject enclitic):} 
%\end{exe} \citet[208]{Wegener:2008}
%\subsubsection*{Predicate-subject locationals (emphatic predicate, subject enclitic impossible):}
%In this construction the subject is not obligatorily case-marked \citet[209f.]{Wegener:2008}
%\begin{exe}\ex
%\begin{xlist}
\ex\gll Apoi ata=e te lo keva\textbf{=na}\\
because here=\emphat{} \emphat{} \deter.\sg.\mas{} path=\nom{}\\
\glt `Because here (is) the road.'
%\ex \gll Te kulo lo sukulu=gha ze kulo=e lo \textbf{naba} \textbf{uan}\\
%\conj{} seawards \deter{}.\pl{} school=\pl{} 3\pl{}[\gen{}] seawards=\emphat{} \deter{}.\sg{}.\mas{} number one\\
%`And, seawards, seawards from the school buildings was the number one.'
\ex\gll ny-omata te\textbf{=lo}\\
1\sg{}-at \partic{}=3.\sg{}.\mas{}.\nom{}\\
\glt `With me (is) it.' lit.: `At me (is) it.'
\ex\gll lo-va sokasoka\textbf{=na} lo-va kata papale=la\\
3\sg{}.\mas-\gen.\mas{} brush=\nom{} 3\sg{}.\mas-\gen.\mas{} bushwards side=\loc.\mas{}\\
\glt `His brush is at his bushwards side.'
\end{xlist}
\end{exe}

%With other verbs: 
%
%\begin{exe}\ex\gll lo-va tovi papale=la=lo te alu-i, lo sokasoka=na\\
%3\sg{}.\mas{}-\gen{}.\mas{} right side=\loc{}=3\sg{}.\mas{}.\nom{} \emphat{} stand-\fin{} \deter{}.\sg{}.\mas{} brush=\nom{}\\
%`It is standing at his (the man's) right side, the brush.'
%\end{exe}
%\citet[93]{Wegener:2008}

The data from the marked"=S languages of the Pacific area are summarized in Table~\ref{PacOverviewExistLoc}. 
Subjects of locational sentences are predominantly coded by the overt S-case (Nominative in Savosavo\il{Savosavo} and Aji\"e\il{Aji\"e}, Absolutive\is{case!individual forms!absolutive} in Nias\il{Nias}), but in one of the Savosavo\il{Savosavo} locational constructions they can also be zero-coded.\footnote{Remember that for the locationals only the positive context is included here, since no language has a variation between negative and positive locationals but not with existentials. 
Thus the Nias\il{Nias} negative locational construction is not represented in the table.}  
All languages of this region have -- at least optionally -- variation between negative and positive existential\is{existential predication!positive versus negative} contexts. 
While this variation is always found in Nias\il{Nias}, Aji\"e\il{Aji\"e} and Savosavo\il{Savosavo} have two coding options for postive existentials -- overt nominative case or zero-coding -- but only one in negative contexts. 
Aji\"e\il{Aji\"e} negative existentials are always zero-coded, Savosavo\il{Savosavo} on the other hand codes negative existentials with Nominative case. 

\begin{table}[ht]
\centering
\begin{tabular}{lccc}
\hline \hline
\bfseries language&\bfseries S exist. (+)&\bfseries S exist.(-)&\bfseries S loc. pred.\\
\hline
Aji\"e\il{Aji\"e}&\textbf{\nom{}}/\acc{}&\acc{}&\textbf{\nom{}}\\
%\hdashline
Nias\il{Nias}&\textbf{\abs{}}&\erg{}&\textbf{\abs{}}\\
%\hdashline
Savosavo\il{Savosavo}&\textbf{\nom{}}/\acc{}&\textbf{\nom{}}&\textbf{\nom{}}/(\acc{})\\
\hline \hline
\end{tabular}
\caption{Overview of the marking of existential and locational predication in the marked"=S languages of the Pacific}\label{PacOverviewExistLoc}%\\
\end{table}

\section{Summary}\label{ExistSum}


%\section{Stategies:}
%\begin{itemize}
 %\item existential predication uses same strategy as locational predication
 % \item existential predication uses different strategy then locational predication
%  \item positive and negative existential predication use same strategy
 %  \item positive and negative existential predication use different strategy
%\end{itemize}

%Crosstable those strategies

The encoding of subjects in positive and negative existential predications as well as locational predications is summarized in Table~\ref{OverviewExistLoc}. 
In locational predications, all languages allow for the marking of subjects with the overt S-case. 
In most languages, this is the only pattern available for this role. 
Languages that show variation in the encoding of existentials (either between positive or negative contexts, or simply have multiple coding options) may also exhibit the same variation in locationals (e.g. Nias\il{Nias} and Savosavo\il{Savosavo}). 
With existentials in a number of languages encoding the subject in the zero-case is at least one of the options. 
This is, for example, the case for North American Jamul\il{Jamul Tiipay} Tiipay and Diegue\~no\il{Diegue\~no (Mesa Grande)}, Nilo-Saharan Tennet\il{Tennet} and Turkana\il{Turkana} and all three Pacific languages. 
Variation\is{existential predication!positive versus negative|(} in subject-marking between positive and negative existentials can be found in a small number of languages (for example Nias\il{Nias} and Turkana\il{Turkana}). 
However, there is no clear directionality in the distribution of overt versus zero-coding between positive and negative contexts. 
Aji\"e\il{Aji\"e}, Nias\il{Nias}, and Tennet\il{Tennet} use the zero-coded form in the negative contexts, while the positive contexts have overtly coded case-forms (at least as an option). 
In Turkana\il{Turkana} and Savosavo\il{Savosavo}, overt marking is used in the negative context, while positive existentials can have zero-coded subjects.\is{existential predication!positive versus negative|)} 
Since zero-coding of subjects is more commonly found with existentials than with locationals, the data to some extent support the claim that existentials exhibit a limited degree of grammatical relation marking \citep[123]{Payne:1997}.

\begin{table}[ht]
\centering
\begin{tabular}{lccc}
\hline \hline
\bfseries language&\bfseries S exist. (+)&\bfseries S exist.(-)&\bfseries S loc. pred.\\
\hline 
%\endfirsthead
%\bfseries language&\bfseries emphatic S &\bfseries non-emphatic S&\bfseries Basic word order &\bfseries emphatic word order &\bfseries Predicate Nominal&\bfseries zero copula\\
%\endhead
%\hline \multicolum{3}{r}{\emph{Continued on next page}}
%\endfoot
%\hline
%\endlastfoot
Aji\"e\il{Aji\"e}&\textbf{\nom{}}/\acc{}&\acc{}&\textbf{\nom{}}\\
%\hdashline
Arbore\il{Arbore}&\textbf{\nom{}}&{-}&{-}\\
%\hdashline
Datooga\il{Datooga}&\textbf{\nom{}}&\textbf{\nom{}}&\textbf{\nom{}}\\
%\hdashline
Diegueno&\acc{}&{-}&\textbf{\nom{}}\\
%\hdashline
Gamo\il{Gamo}&{-}&{-}&\textbf{\nom{}}\\
%\hdashline
Havasupai\il{Havasupai}&\textbf{\nom{}}&{-}&\textbf{\nom{}}\\
%\hdashline
Jamul\il{Jamul Tiipay} Tiipay&\acc{}&{-}&\textbf{\nom{}}\\
%\hdashline
K'abeena\il{K'abeena}&\textbf{\nom{}}&\textbf{\nom{}}&\textbf{\nom{}}\\
%\hdashline
Maa\il{Maa}&\textbf{\nom{}}/\acc{}&\acc{}&\textbf{\nom{}}\\
%\hdashline
Maidu\il{Maidu}&\textbf{\nom{}}&{-}&\textbf{\nom{}}\\
%\hdashline
Mojave\il{Mojave}&\textbf{\nom{}}&\textbf{\nom{}}&\textbf{\nom{}}\\
%\hdashline
Murle\il{Murle}&\textbf{\nom{}}&{-}&\textbf{\nom{}}\\
%\hdashline
Nandi\il{Nandi}&\textbf{\nom{}}&{-}&\textbf{\nom{}}\\
%\hdashline
Nias\il{Nias}&\textbf{\abs{}}&\erg{}&\textbf{\abs{}}\\
%\hdashline
Oromo (Boraana\il{Oromo (Boraana)})&{-}&{-}&\textbf{\nom{}}\\
%\hdashline
Oromo (Harar\il{Oromo (Harar)})& emphatic subjet &{-}& \textbf{\nom{}}\\
%\hdashline
Savosavo\il{Savosavo}&\textbf{\nom{}}/\acc{}&\textbf{\nom{}}&\textbf{\nom{}}/\acc{}\\
%\hdashline
Tennet\il{Tennet}&\textbf{\nom{}}/\acc{}&\acc{}&\textbf{\nom{}}\\
%\hdashline
Turkana\il{Turkana}&\textbf{\nom{}}/\acc{}&\textbf{\nom{}}?&\textbf{\nom{}}\\
%\hdashline
Wappo\il{Wappo}&\textbf{\nom{}}&\textbf{\nom{}}&\textbf{\nom{}}\\
%\hdashline
Wolaytta\il{Wolaytta}&{-}&{-}&\textbf{\nom{}}\\
%\hdashline
Yavapai\il{Yavapai}&\textbf{\nom{}}&{-}&\textbf{\nom{}}\\
%\hdashline
Zayse\il{Zayse}&\textbf{\nom{}}&{-}&\textbf{\nom{}}\\
\hline \hline
\end{tabular}
\caption{Overview of the marking of existential and locational predication}\label{OverviewExistLoc}
\end{table}

                                     
