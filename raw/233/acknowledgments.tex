\addchap{\lsAcknowledgementTitle} 

Many people have contributed to the completion of this book, which started out as my dissertation project at the University of T\"ubingen, on a position in a project financed by the ERC Advanced Grant 324246 EVOLAEMP. During the four years that I was working on it, my advisor Gerhard Jäger provided me with a stable and rich research environment, with lots of opportunities to meet fascinating people, while always leaving me a maximum of freedom to explore my linguistic interests. He also provided the seed idea which this thesis grew out of, and when initial experiments failed due to low data quality, he gave me the chance to spend time and resources on collecting high-quality data. He also helped with many suggestions nudging me towards a more empirical approach in many design decisions, and was always very quick to help with technical issues. Finally, I thank him for his patience when many parts of the work described here took longer than expected. Fritz Hamm, my second advisor, is not only the person who got me interested in causal inference, but over the past twelve years, he has also been the person to waken my interest first in logic, then in mathematics, and has been a source of encouragement and inspiration for my more formal side ever since.

Igor Yanovich has given me much advice on which parts of my work to prioritize, and accompanied the process of working out the mathematical details with a critical eye, also providing vital moral support whenever a new algorithmic idea did not lead to the results I had hoped for. I also thank Armin Buch and Marisa Köllner for the enjoyable collaboration on some of the research leading up to this thesis, as in determining the concept list for NorthEuraLex, and allowing me to test parts of my implementation in different contexts. I am also very grateful to all the other EVOLAEMP members for helpful discussion and feedback on the many occasions where I presented preliminary results to the group. I particularly enjoyed teaching with Christian Bentz and Roland Mühlenbernd, and exchanging experiences and knowledge with Johannes Wahle and Taraka Rama on many occasions.

Among the many other researchers I have had the pleasure to communicate with during the past four years, there are some which provided particularly important bits of advice and information, which is why I would like to mention them here. Johann-Mattis List, Søren Wichmann, Gereon Kaiping, Harald Hammarström, Michael Dunn, and Robert Forkel provided valuable advice about lexical databases, issues of standardization, and best practices. Johanna Nichols and Balthasar Bickel inspired my interest in typology, and helped me to see why this is the linguistic discipline where I feel most at home. I also enjoyed learning some Nenets with Polina Berezovskaya, who gave me valuable insights into linguistic fieldwork.

Then, there are the many student assistants who assisted me in compiling the NorthEuraLex database, and with the many tasks involved in releasing it. Thora Daneyko has been extremely helpful in contributing many small programs and helpful knowledge to this large endeavor, and Isabella Boga continues to be a very enthusiastic and productive supporter of lifting the database to the next level. Pavel Sofroniev was of invaluable help whenever web programming was necessary (e.g. for the NorthEuraLex website), and Alessio Maiello helped many times by fixing problems in the project infrastructure in a very quick and competent manner. Thanks are also due to the former data collectors Alla Münch, Alina Ladygina, Natalie Clarius, Ilja Grigorjew, Mohamed Balabel, and Zalina Baysarova.

For their contribution to the last steps on the long road towards the publication of this book, I would like to thank John Nerbonne, Sebastian Nordhoff, three anonymous reviewers, and the volunteer proofreaders for their encouragement, their helpful feedback, and especially their patience.

When completing a dissertation, it is time to look back and think about the factors which most influenced one's development up to this point. Towering above all else, I see the luck of the time and place I was born into. Being born into Europe, this inspiring and complex continent which somehow manages to uphold universal healthcare and affordable tuition, and into Germany, this safe and stable country of many opportunities which still puts some value on the humanities, are two factors which allowed me to study extensively without worrying too much about my future. Without this mindset, I would not have risked the jump into the uncertain perspectives of academia. Also, being able to study and then to continue my career in a calm and international place imbued with academic tradition like Tübingen can be added to this list of lucky geographical coincidences.

Coming to the people who shaped me intellectually during my undergraduate studies, I would like to at least mention Frank Richter, Dale Gerdemann, Detmar Meurers, and Laura Kallmeyer, who introduced me to four very different kinds of computational linguistics, and Michael Kaufmann, in whose algorithmics group I have learned the most valuable pieces of knowledge and thinking which I took away from studying computer science.

Among other factors which contributed to my making it through the past years, I cannot overstate the luck of having a stable and loving family in the North which always provided unconditional support, and a close-knit circle of Tübingen friends which was available for socializing and talking about problems whenever the need arose, often providing me with much-needed perspective on the scale of my problems and worries. Also, I am glad to still be able to count some other people I studied with among my close friends, even if they moved away from Tübingen by now.

Finally, there is my wife Anna, the gratitude towards whom I find difficult to put into words. All the support, the patience, the willingness to be sad and happy together, to put up with late-night enthusiasm and despair despite a five-hour difference in sleep patterns -- it would be a very different life without all this. We have been through a lot, and I am looking forward to building on what we have finally achieved for many more years.
