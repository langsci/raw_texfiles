\documentclass[output=paper]{langsci/langscibook}
\ChapterDOI{10.5281/zenodo.4729793}

\author{Berthold Crysmann\affiliation{Université de Paris, Laboratoire de linguistique formelle, CNRS}}

\title{Deconstructing exuberant exponence in Batsbi} 

\abstract{In this article, I shall discuss ``exuberant exponence'' in Batsbi
  \citep{harris_a09}, an extreme case of extended exponence where
  identical gender-number markers can surface multiple times within
  the same word, subject to the presence of certain triggering stems
  or affixes. I shall also evaluate in some detail the challenge the
  Batsbi data pose for extant formal theories of inflection and
  show that these challenges cut across the divide between lexical
  and inferential theories. In the analysis, I shall highlight the
  dependent nature of the agreement exponents and propose a formal
  account that draws crucially on two central properties of
  Information-based Morphology, namely the recognition of many-to-many
  relations at the most fundamental level of description, and the
  possibility to extract (partial) generalisations over rules by means
  of cross-classifying inheritance hierarchies. As a result,
  cross-classification of agreement rule types with those for the
  triggering stems and affixes will capture the dependent nature
  directly, while at the same time ensuring the reuse of inflectional
  resources. Thus, the decomposition of Batsbi exuberant exponence
  improves considerably over a pure word-based approach and emphasises
  the need to afford both atomistic and holistic views within a theory
  of inflection.}




\begin{document}
\maketitle


\section{Introduction}

Ever since \citet{Matthews72}, extended (or multiple) exponence has
been one of the core phenomena highlighting the one-to-many nature of
inflectional morphology (see \citealt{Harris17} for a typological
survey). In this chapter, I shall discuss exuberant exponence in
Batsbi \citep{Harris09}, an extreme case of extended exponence, where
one and the same morphosyntactic property may end up being marked over
and over again within a word. Outside Batsbi, the phenomenon has been
reported for a variety of languages, including Archi, Khinalug,
Chamalal (see \citealp{Harris09} for an extended list).


Exuberant exponence in Batsbi is manifest in gender/number agreement
on verbs, giving rise to up to four realisations of agreement with the
same argument, the absolutive. What is more, the shape of the
exponents across multiple realisations stays the same.

\begin{exe}
  \ex  \label{ex:Batsbi}\gll \textbf{y}-ox-\textbf{y}-$\emptyset$-o-\textbf{y}-anǒ\\
  \textsc{cm}-rip-\textsc{cm-tr-pres-cm-evid1}\\
  \glt `Evidently she ripped it.’ \hfill\citep[277]{Harris09}
  
\end{exe}

What makes exuberant exponence particularly interesting from the
viewpoint of formal grammar is that the phenomenon can serve as a
stress-test for current theories of inflectional morphology. First,
exuberant exponence will be less troublesome for theories that fully
embrace extended exponence as a basic property of inflectional
morphology, rather than providing limited workarounds on the basis of
an essentially morphemic model. Second, the identity of exponents
observed in Batsbi calls for inflectional models that provide a notion
of resource reusability. Third, as I shall discuss below, the
presence of agreement markers is dependent on adjacent triggering
stems and suffixes, which suggests that agreement markers cannot be
derived on their own, but rather compose with the affixes that license
their occurrence into inflectional constructions. I shall argue more
specifically that the dependent nature of Batsbi exuberant exponence
calls for a model of morphology that addresses the many-to-many nature
of inflection at the most basic level, a property characteristic of
the framework adopted here.
 

The presentation of the empirical facts about Batsbi exuberant
exponence is based on the original paper by
\citet{Harris09}. Thus, this paper aims at making contributions in
two areas: first explore in more detail the implications of the
data for different incarnations of inferential-realisational and
inferential-lexical approaches, and second, provide a fully
formalised treatment of this challenging case of dependent multiple exponence
within the framework of Information-based Morphology (=IbM;
\citealp{Crysmann:Bonami:2016}).

The chapter is organised as follows: in Section \ref{sec:Data}, I
shall rehearse the basic empirical data concerning Batsbi exuberant
exponence, starting with the inventory of (productive) gender markers,
followed by a discussion of class marking on stems as well as  
affixal material. % The description of the basic empirical pattern
% concludes with a discussion of person/number/case marking, which may
% partially overlap with gender/number marking, possibly giving rise to
% yet another level of multiple exponence.
Section \ref{sec:Discussion} will serve to evaluate extant theories of
inflection with respect to their capability to address the phenomenon
at hand, taking as a starting point the typology developed in
\citet{Stump01}: While incremental theories prove to be inadequate, a
somewhat striking observation is that exuberant exponence does not
distinguish between lexical-realisational and
incremental-realisational models as a class, but is rather sensitive
to details of formal expressivity of the concrete theory.

Section \ref{sec:Analysis} will finally provide an analysis within the
framework of Information-based Morphology (henceforth: IbM), an
inferential-realisational model of morphology cast entirely in terms
of inheritance hierarchies of typed feature structures. I shall
provide a brief sketch of IbM and then show how cross-classification
in monotonic inheritance hierarchies is well-suited to capture reuse of form and
the dependent nature of exuberant exponence at the same time.

\section{Data}
\label{sec:Data}

\subsection{Properties of class marking in Batsbi}

Batsbi has a rather elaborate gender system, distinguishing eight
gender categories, each with singular and plural forms, out of which
at least five are productive, while the following three are not,
according to \citet{Corbett:91} and \citet{holisky:gagua:94}: genders IV
(2 nouns), VIII (4), and VII (15). Lexical counts are indicated in
parentheses.

Exponence of gender/class agreement is detailed in Table
\ref{tab:ClassExp}. As can be seen, /d/ is quite prevalent as an
exponent, which is why Harris occasionally uses it as a representative
for the entire set of class markers. 

\begin{table}[htb]
  \centering
  \begin{tabular}{rcc}
    \lsptoprule
    & \textsc{singular} & \textsc{plural}\\
    \midrule
    I & v & b\\ 
    II & y & d\\ 
    III & y & y\\ 
    (IV) & b & b\\
    V & d & d\\ 
    VI & b & d\\
    (VII) & b & y \\
    (VIII) & d & y\\
    \lspbottomrule
  \end{tabular}
  \caption{Gender agreement markers in Batsbi}
  \label{tab:ClassExp}
\end{table}

Gender/number agreement is controlled by the absolutive argument,
i.e. the S argument of intransitives, as witnessed in (\ref{ex:AgrS}),
and the O argument of transitives, as shown by (\ref{ex:AgrO}).   


\begin{exe}
  % \ex \label{ex:PreRoot}
  % \ex \gll  don-e-v taylz-i  \textbf{d}-ek’-\textbf{d}-iy-ẽ\\ 
  % horse(b/d)-OBL-ERG saddlebags(/d)-PL.ABS CM-fall-CM-TR-AOR\\
  % \glt ‘The horse threw off the saddlebags.
    
  \ex{ \gll  xen-go-ħ  potl-i \textbf{d}-ek’-ĩ\\
    tree-\textsc{all-loc} leaf(d/d)-\textsc{pl.abs} \textsc{cm}-fall-\textsc{aor}\\
    \glt ‘The leaves of the tree were
    falling.’\hfill\citep[274]{Harris09} } \label{ex:AgrS}
  
  \ex {\gll  pst’uyn-čo-v
    bader \textbf{d}-iy-ẽ\\
    married.woman(y/y)-\textsc{obl-erg} child(d/d).\textsc{abs}
    \textsc{cm}-do-\textsc{aor}\\
    \glt ‘The (married) woman bore a child.’
    \hfill\citep[274]{Harris09}} \label{ex:AgrO}

\end{exe}



\subsection{Class marking on stems}
\label{sec:Data:Roots}

As we have seen in example (\ref{ex:Batsbi}) above, Batsbi class
marking can surface multiple times within a word, and when it does,
we always find the same exponents. However, as pointed out by
\citet{Harris09}, presence of class markers in this language is
contingent on the right-adjacent marker: just as we may find words
with multiple class markers, as in (\ref{ex:Batsbi}), we may equally
find words showing a single marker, as in
(\ref{ex:AgrS}), (\ref{ex:AgrO}) or (\ref{ex:NoPreRoot}), or even no
overt class making at all, as e.g. in (\ref{ex:None}).

\begin{exe}
  
  \ex \label{ex:NoPreRoot}{\gll oqus mot: k’edl-e-guy tat:-\textbf{b}-iy-ẽ\\
    \textsc{3sg.erg} bed(b/d).\textsc{abs} wall-\textsc{obl}-towards
    push-\textsc{cm-tr-aor}\\
    \glt ‘S/he pushed the bed towards the wall.’ \hfill \citep[275]{Harris09}}
  \ex \label{ex:None}{\gll  qan simind
    lapsdan matx
    ot’-ǒ\\
    tomorrow corn(d/d).\textsc{abs} to.dry
    sun(b/d) spread-\textsc{fut}\\
    \glt ‘Tomorrow [they] will spread the corn in the sun to dry.’
    \hfill (idem)}
\end{exe}

Stems are one of the elements that may require or disallow left adjacent
class markers: according to \citet[fn.~23]{Harris09}, 468 (21.53\%)
out of 2178 verbs in the dictionary by \citet{kadagize84} feature a
pre-radical class marker. While none of the stems in
(\ref{ex:NoPreRoot}) or (\ref{ex:None}) appears to take a class marker
to its immediate left, the verbs \textit{ek'}
`fall' and  \textit{iy} `do' in fact do, as illustrated in (\ref{ex:AgrS}) and (\ref{ex:AgrO})  above.
    

% \begin{exe}
%   \ex \label{ex:PreRoot}\gll  pst’uyn-čo-v
%   bader \textbf{d}-iy-ẽ\\
%   married.woman(y/y)-\textsc{obl-erg} child(d/d).\textsc{abs}
%   \textsc{cm}-do-\textsc{aor}\\
%   \glt ‘The (married) woman bore a child.’ \hfill\citep[274]{Harris09}
% \end{exe}


\citet{holisky:gagua:94} note that some verbs distinguish the perfective
from the imperfective stem by means of an agreement marker,
contrasting, e.g. \textit{\textbf{d}-ek'-ar} `fall.\textsc{pfv}' with
\textit{ak'-ar} `fall.\textsc{ipfv}'. \citet{Harris09} provides a list
of minimal pairs, where lexical meaning is solely distinguished by
presence of a pre-radical marker, including e.g. \textit{ot:-ar}
`stand, stay' vs. \textit{\textbf{d}-ot:-ar} `pour into'.  Thus, it
appears that the presence vs.\ absence of a pre-radical agreement marker
is lexically determined, i.e. it is a property of individual stems, or
else of the entire lexeme. Choice of the shape of the marker, by contrast,
is clearly an inflectional property.

\subsection{Class marking on suffixes}
\label{sec:Data:Suffixes}

Similar to pre-stem class markers, post-stem gender/number markers
appear left-adjacent to certain triggering suffixes. These comprise
the transitivity markers \textit{-al} (\textsc{intr}) and \textit{-iy}
(\textsc{tr}), as well as the affirmative and negative evidentiality
markers \textit{-anǒ} (\textsc{evid1.aff}) and \textit{-a}
(\textsc{evid1.neg}). Pre-stem and pre-suffixal class markers are
controlled by the same argument, the absolutive, and their shape is
identical. However, their presence is conditioned
independently.  

\begin{table}[htb]
  \centering
  \begin{tabular}[t]{lll}
    \lsptoprule
    \textsc{stem} & \textsc{(in)trans} & \textsc{evid}\\
  \midrule
  stem & $\emptyset$ & $\emptyset$/-inǒ\\
  \textbf{d}-stem & $\emptyset$ & $\emptyset$/-inǒ\\
  stem & $\emptyset$ & \textbf{d-}anǒ\\
  \textbf{d}-stem & $\emptyset$ & \textbf{d-}anǒ\\
  stem & \textbf{d}-iy/al & $\emptyset$/-inǒ\\
      \textbf{d}-stem & \textbf{d}-iy/al & $\emptyset$/-inǒ\\
  stem & \textbf{d}-iy/al & \textbf{d-}anǒ\\
  \textbf{d}-stem & \textbf{d}-iy/al & \textbf{d-}anǒ\\
    \lspbottomrule
\end{tabular}

\caption{Patterns of dependent class marking in Batsbi \citep{Harris09}}
\label{tab:MarkingPatterns}
\end{table}
\subsubsection{Intransitive marker \textit{-al}}  

The basic function of the intransitive marker \textit{-al} is to  derive intransitives
from transitives, as illustrated in
(\ref{ex:Batsbi:Intrans}).\footnote{Note that the intransitive marker
  can also be found with some intransitive bases, e.g.
  \textit{ak’-\textbf{d}-al-ar} ‘light up, catch fire’ vs.
  \textit{ak’-ar} ‘burn, be alight’. % BC: Cite Harris 
}


\begin{exe}
  \ex \label{ex:Batsbi:Intrans}
  \begin{xlist}
    \ex {\gll p’erang-mak-aħǒ xalat
      \textbf{y}-opx-ǒ\\
      shirt-on-\textsc{loc}
      house.coat(y/y).\textsc{abs} \textsc{cm}-put.on-\textsc{prs}\\
      \glt ‘[She] puts on a house coat over her shirt.’ \hfill\citep[275]{Harris09}}
    
    \ex \gll sẽ yoħ taguš \textbf{y}-opx-\textbf{y}-al-in=ě\\
    me.\textsc{gen} girl(y/d).\textsc{abs} beautifully \textsc{cm}-put.on-\textsc{cm-intr-aor}=\&\\
    \glt ‘My daughter dressed beautifully and ...’ \hfill\citep[275]{Harris09}
    
  \end{xlist}
    
\end{exe}

When this marker is present, it is obligatorily accompanied by the
class marker to its left. Presence of the post-radical marker is
triggered independently of the stem, as shown by the contrast between
(\ref{ex:Batsbi:Intrans}a,b) and (\ref{ex:Batsbi:Intrans:weigh}).

\begin{exe}
  \ex \gll psare(ħ) oc’-\textbf{v}-al-in-es ...\\
  yesterday weigh.\textsc{pfv-cm-intr-aor-1sg.erg}\\
  \glt ‘I (masculine) weighed yesterday ...’
  \hfill\citep[275]{Harris09} \label{ex:Batsbi:Intrans:weigh}

\end{exe}

\subsubsection{Transitive marker \textit{-iy}}

While the intransitive marker \textit{-al} derives intransitives from
transitive bases, the transitive marker \textit{-iy} signals the
opposite, namely transitives derived from intransitive
bases.\footnote{This marker may occasionally serve to distinguish
  transitives. % BC: Cite Harris
} Again, this marker is immediately preceded by the class marker, as
illustrated in the examples in (\ref{ex:Batsbi:Trans}).

\begin{exe}
  \ex \label{ex:Batsbi:Trans}
  \begin{xlist}
    \ex \gll don-e-v taylz-i \textbf{d}-ek’-\textbf{d}-iy-ẽ \\
      horse(b/d)-\textsc{obl-erg} saddlebags(/d)\footnotemark-\textsc{pl.abs} \textsc{cm}-fall-\textsc{cm-tr-aor}\\
      \glt ‘The horse threw off the saddlebags.'\hfill\citep[274]{Harris09}
    
    \ex\gll kuyrc’l-e-x qečqečnayrẽ daq’r-i lal-\textbf{d}-iy-ẽ makaħǒ\\
      wedding-\textsc{obl-con} various food(d/d)-\textsc{pl.abs} go-\textsc{cm-tr-aor} above\\
      \glt ‘At the wedding [they] passed around various foods.’\hfill(idem)
  \end{xlist}
\end{exe}

\footnotetext{`Saddlebags' is a plurale tantum
  \citep[][274]{Harris09}. Lacking an attested singular form, its gender could be any of II, V, or VI.}


\subsubsection{Present evidential}

The third suffixal marker that takes the class marker, again to its
immediate left, is the present evidential marker
\textit{-anǒ}. According to \citet{Harris09}, this marker productively
combines with any lexeme. Compare the examples in
(\ref{ex:Batsbi:PrsEvid}): adding the present evidential to an example
like (\ref{ex:Batsbi:PrsEvid}a), with already two class markers (one
triggered by the stem and one triggered by the transitive
marker), adds a third instance of class marking, yielding a total of
three exponents, as shown in (\ref{ex:Batsbi:PrsEvid}b).

\begin{exe}
  \ex \label{ex:Batsbi:PrsEvid}
  \begin{xlist}
    \ex {\gll k’ab
      \textbf{y}-ox-\textbf{y}-iy-ẽ\\
      dress(y/y).\textsc{abs} \textsc{cm}-rip-\textsc{cm-tr-aor}\\
      \glt ‘[She] ripped the dress.’\hfill\citep[277]{Harris09}}
    
    \ex  {\gll \textbf{y}-ox-\textbf{y}-$\emptyset$-o-\textbf{y}-anǒ\\
      \textsc{cm}-rip-\textsc{cm-tr-prs-cm-evid1}\\
      \glt `Evidently she ripped it.’\hfill\citep[277]{Harris09}}
  \end{xlist}
\end{exe}

Again, class inflection of the evidential is independent of that of the stem, i.e. it 
is triggered by the present evidential, regardless of whether the stem is
already marked with the gender marker, as in
(\ref{ex:Batsbi:PrsEvid}b), or not, as in
(\ref{ex:Batsbi:PrsEvid:cut}).
    
\begin{exe}
  \ex \label{ex:Batsbi:PrsEvid:cut} \gll tet’-\textbf{d}-anǒ\\
  cut-\textsc{cm-evid1}\\
  \glt ‘Evidently s/he was cutting it.’ \hfill
  \citep[][181]{holisky:gagua:94}
\end{exe}

The present evidential \textit{-anǒ} (\textsc{evid1}) contrasts with,
e.g. the aorist evidential \textit{-inǒ}, which never
takes a gender/number marker. % BC: Example



% \subsection{Person/number/case marking}

% \citet{Harris09} cites an additional source of exuberance, namely
% person-number-case marking, which can be considered another redundant
% marking of number. 

% % \begin{itemize}
% %   \item  Paradigm of person/number/case markers (3rd person marking is
% %     zero)
    
% %     \begin{table}[htb]
% %       \centering
% %       \begin{tabular}[t]{lll}
% %         & \textsc{nom} & \textsc{erg}\\
% %         \hline
% %         \textsc{1sg} & -s(ǒ) & -as\\
% %         \textsc{2sg} & -ħ(ǒ) & -a(ħ)\\
% %         \textsc{1ex} & -tx(ǒ) & -atx\\
% %         \textsc{2pl} & -š(ǒ) & -eš,-iš
% %       \end{tabular}
      
% %       \caption{Paradigm of Batsbi person/number/case markers}
% %       \label{tab:PNC}
% %     \end{table}


    
% %     Table \ref{tab:PNC} presents the paradigm of person/number/case markers:
% % evidently, as the name suggests  these markers are differentiated for
% % case, i.e. absolutive and ergative. Note further that third person
% % marking is always zero. % What about 1inc?    

% Extended exponence of number between person/number/case markers and
% class markers can e.g. be observed with intransitives, as in
% (\ref{ex:Batsbi:PNC}):

% \begin{exe}
%   \ex \label{ex:Batsbi:PNC}
%   \begin{xlist}
%     \ex \gll mič-iv-ħ \textbf{b}-uyt’-ayšǐ k’nat-i\\
%     where-\textsc{dir-loc} \textsc{cm}-go-\textsc{2pl.erg}
%     boy(v/b)-\textsc{pl.abs}\\
%     \glt ‘Where are you going, boys?’
%     \ex \gll xširoš \textbf{y}-uyt’-\textbf{y}-aγ-\textbf{y}-ano kalik\\
%     often \textsc{cm}-go-\textsc{cm}-come-\textsc{evid1.2sg.erg} city\\
%     \glt `You (\textsc{f}) evidently often come and go to the city.'
%     \hfill\citep[280]{Harris09}
%   \end{xlist}
% \end{exe}
    
% However, this type of extended exponence appears to be of an entirely
% different nature than the one we find for class markers: not only are
% we confronted with different exponence and different combinations of
% morphosyntactic properties (gender/number vs. person/number/case), the
% controllers of these two types of agreement may also differ, as
% illustrated by the transitive examples in (\ref{ex:Batsbi:PNC:Trans}).  


% \begin{exe}
%   \ex
%   \label{ex:Batsbi:PNC:Trans}
% \gll duq kaniz \textbf{y}-ayq-n-\textit{atx}\\
%     many grape[s](\textbf{y/y}).\textsc{abs} \textsc{cm}-eat-\textsc{aor}-\textsc{1ex.erg}\\
%     \glt ‘We(\textsc{ex}) ate a lot of grapes.'
%     \hfill\citep[279]{Harris09}
% \end{exe}

% I shall conclude that  we are actually dealing with two independent
% agreement processes here which may but need not be controlled by the same
% argument. If this is the case, we may safely assume that these two
% agreement processes have independent representations on the
% morphosyntactic property set, so we are not confronted with a
% one-to-many relation in the strict sense. 


\subsection{Wordhood}

The implications of exuberant exponence for morphological theory
depend of course on the crucial question whether the relevant domain
is morphology, i.e. whether we are dealing with complex words, or
syntax. \citet{Harris09} provides extensive tests showing that we are
indeed confronted with massive extended exponence within a single
word, rather than agreement across several syntactically independent
words. This is even more important given that most markers involved
here used to be independent words diachronically, e.g. the evidential
marker derives from the verb `to be'. % , and the person/number/case
% markers derive from independent pronouns.  

Regarding the status of class markers, \citet{Harris09} provides five
tests in total.\footnote{\citet{Harris09} presents a
  total of seven tests, two of which are confined to the status of
  person/number markers. These markers may incidentally be controlled
  by the same argument, which leads Harris to regard them as yet
  another instance of (partial) exuberant agreement. However, given
  that the controllers need not be the same
  \citep[see][ex. 33]{Harris09}), I shall rather treat this as
  accidental, and thus ignore person/number marking for the
  purposes of this paper.} I shall give a brief description of the
tests, and summarise the results, which uniformly point towards the
affixal status of the class markers (see Table~\ref{tab:Wordhood} for
a summary of the results, and
% . As for the person/number/gender markers,
% evidence overwhelmingly points in the same direction, the only
% exception being the nasal reduction test (for details, see
\citealt[sec.~5]{Harris09} for details).

\begin{description}
% \item[Phonology] Two phonological reduction processes targeting word-final vowels
%   and nasals. These tests are only applicable to the
%   person/number/case markers. 
\item[Agreement controller:] Establishes whether
  auxiliary or evidential markers share an argument structure with the
  verb: true auxiliaries behave like intransitives (regardless of main
  verb), evidentials reflect the main verb's transitivity, suggesting
  bound status.
\item[Intervention:] Two related tests based on the possibility for
  intervention of negative marker and clitic conjunction:
  the possibility for intervention is independently established for
  auxiliaries, yet all markers under discussion uniformly prohibit
  intervention. 
\item[Coordination \& Gapping:] {\sloppy Two tests that assess
  whether or not markers can be suppressed in coordinate structures. While
  auxiliaries and main verbs can be elided in the second conjunct,
  transitive markers and evidentials cannot. }
\end{description}

\begin{table}
  \begin{tabular}{lcccccc}
    \lsptoprule
%    &&&& \multicolumn{2}{c|}{\textsc{per/num}}\\ 
    Test & \textsc{tr} & \textsc{intr} & \textsc{evid1} &
                                       %                    \textsc{erg}
                                       % & \textsc{abs} &
                                                        \textsc{aux}\\
    \midrule
%    V reduction & & & & aff & aff &\\
%    /n/ reduction & & & & aff & wd\\
    Agreement trigger & & & aff
%                       &&
                                       & wd \\
    Intervention (neg) & aff & aff & aff
%                       & &
                                                        & wd\\
    Intervention (clitic) & & & aff
                       % &&
                                       & wd\\
    Conjoining & aff & aff & aff
                       % & aff & aff
                                                        & wd\\
    Gapping & aff &&
         % &&
                       & wd\\
    \lspbottomrule
  \end{tabular}
  \caption{Tests for word vs. affix status \citep{Harris09}}
  \label{tab:Wordhood}
\end{table}



% I shall discuss three of these test in
% detail, focusing on the coordination test and the intervention test,
% which are applicable to the full range of markers in question. In
% particular, I shall skip the phonological test (word-final vowel),
% which are only ever applicable to the person/number/case
% markers. % Similarly, I shall equally skip the gapping test, since it is
% % in fact quite similar to the conjunction test.




% \subsubsection{Intervention}

% The first intervention test Harris presents, and the one which I will
% discuss in detail concerns the  negation marker \textit{co} `not'. 
% As show by \citet{Harris09}, this marker is obligatorily placed
% directly adjacent to the verb or the auxiliary, cf. (\ref{ex:co:Verb})
% and (\ref{ex:co:AUX}). 

% \begin{exe}
%   \ex \label{ex:co:Verb}
%   \begin{xlist}
%     \ex[]{ \gll oqus γaziš co ʕam-d-or\\
%     he.\textsc{erg} well \textsc{neg}  study-\textsc{cm}-\textsc{impf}\\
%     \glt ‘He did not study well.’}
%     \ex[]{co ʕam-d-or oqus γaziš}
%     \ex[*]{oqus co γaziš ʕam-d-or}
%     \ex[*]{oqus γaziš ʕam-d-or co \hfill\citep[p.~285]{Harris09}} 
%   \end{xlist}
%   \ex \label{ex:co:AUX}
%   \begin{xlist}
%     \ex[]{\gll so sk’ol-i v-ot’u-yn co v-a-sŏ\\
%     I-\textsc{absl} school-in \textsc{cm}-go-\textsc{ptcpl}
%     \textsc{neg} \textsc{cm}-be-\textsc{1sg.abs}\\
%     \glt `I (male) am not going to school.'}
%   \ex[*]{\gll so sk’ol-i co v-ot’u-yn v-a-sŏ\\
%     I-\textsc{absl} school-in \textsc{neg} \textsc{cm}-go-\textsc{ptcpl}
%      \textsc{cm}-be-\textsc{1sg.abs}\\
%      \glt }
%    \ex[]{\gll son co leʔ v-ax-a n\\
%      I.\textsc{dat} \textsc{neg} want \textsc{cm}-go-\textsc{inf}\\
%      \glt `I (male) don’t want to go.' \hfill \citep[p.~285]{Harris09}
%    }
%   \end{xlist}
% \end{exe}

% Despite the fact that the evidential marker historically derives from
% the verb `to be', it does not permit placement of negative \textit{co}
% to its immediate left (see (\ref{ex:co:evid})), unlike the auxiliary
% in (\ref{ex:co:AUX}a). Rather the negative marker will be found to the
% left of the main verb.

% \begin{exe}
%   \ex \label{ex:co:evid}
%   \begin{xlist}
%     \ex[]{\gll co tet’-d-anŏ \\
%       \textsc{neg} cut-\textsc{cm}-\textsc{evid1} \\
%       \glt `S/he was evidently not cutting it.'}
%     \ex[*]{tet’ co d-anŏ\hfill \citep[p.~287]{Harris09}}
%   \end{xlist}
% \end{exe}

% Similarly, the negative marker cannot intervene between the transitive
% or intransitive markers and the stem, as shown in (\ref{ex:co:tr}) and
% (\ref{ex:co:intr}).  

% \begin{exe}
%   \ex \label{ex:co:tr}
%   \begin{xlist}
%     \ex[] {\gll aqsb-a-x gagn-i co c’eg-y-o\\
%       Easter(d/)-\textsc{obl-con} egg(y/y)-\textsc{PL.ABS} \textsc{neg} red-\textsc{cm-prs}\\
%       \glt `At Easter they didn't paint (lit. make) eggs red.'\hfill
%       \citep[p.~286]{Harris09}}
%       \ex[*]{aqsb-a-x gagn-i c’eg co y-o} 
%   \end{xlist}
%   \ex \label{ex:co:intr}
%   \begin{xlist}
%     \ex[]{\gll yoħ co c’eg-y-al-i$^n$\\
%       girl(y/d).\textsc{abs} \textsc{neg} red-\textsc{cm-intr-aor}\\
%       \glt ‘The girl did not blush.’}
%     \ex[*]{yoħ c’eg co y-al-i$^n$ \hfill\citep[p.~286]{Harris09}}
%   \end{xlist}
% \end{exe}

% Harris stresses that ``both \textit{y-o} and \textit{y-al-i$^n$} exist
% as independent verb forms in other contexts.'' She therefore concludes
% that the transitive and intransitive markers do not enjoy the status
% of independent auxiliaries. 

% Finally, the negative marker cannot be found adjacent to the
% person/number/case markers either, as shown in
% (\ref{ex:co:pnc}). 

% \begin{exe}
%   \ex \label{ex:co:pnc}
%   \begin{xlist}
%     \ex[]{\gll so osi co v-a-ra-sŏ\\
%       I.\textsc{ABS} there \textsc{NEG} \textsc{CM}-be-\textsc{IMPF-1SG.ABS}\\
%       \glt `I was not there.'}
%     \ex[*]{ so osi  v-a-ra co sŏ \hfill\citep[p.~287]{Harris09}}
%   \end{xlist}
% \end{exe}

% \citet{Harris09} complements the test pertaining to the
% placement of the negative marker with a second one using the enclitic
% coordination marker `and'. While this marker can separate a preverb
% from the verb stem, it cannot occur in e.g. between the stem and the
% evidential markers.  See \citet[p.~287--288]{Harris09} for details.

% \subsubsection{Coordination}

% The second test I will discuss in detail concerns coordination. Like
% the intervention test with the negative marker, this diagnostic
% appears to be applicable to both transitivity markers and
% evidentials. 

% With VP coordination, auxiliaries can either be repeated on each
% conjunct, as shown in  (\ref{ex:CoordAUX}a), or else the auxiliary can
% elide, as illustrated in  (\ref{ex:CoordAUX}b).

% \begin{exe}
%   \ex \label{ex:CoordAUX}
%   \begin{xlist}
%     \ex[]{\gll andri qor dargo-b-oš v-a, bubk’-i leħ-d-oš v-a\\
%       Andre.\textsc{abs} apple(b/d).\textsc{abs} plant-\textsc{cm-absl} \textsc{cm}-be flower(d/d)-pl.abs pick-\textsc{cm-absl} \textsc{cm}-be\\
%       \glt ‘Andre is planting an apple and is picking flowers.’
%     }
%     \ex[]{\gll andri qor dargo-b-oš v-a, bubk’-i leħ-d-oš \\
%       Andre.\textsc{abs} apple(b/d).\textsc{abs} plant-\textsc{cm-absl} \textsc{cm}-be flower(d/d)-pl.abs pick-\textsc{cm-absl}\\
%       \glt ‘Andre is planting an apple and picking flowers.’\\\hfill\citep[p.~288]{Harris09}}
%   \end{xlist}
% \end{exe}

% Given that the markers in question, i.e. the evidential and the
% transitivity markers historically derive from auxiliaries, there is an
% expectation for these markers to permit elision, if they are still
% independent words. However, in contrast to the true auxiliary in
% (\ref{ex:CoordAUX}), elision is not possible for the evidential
% marker, as shown in (\ref{ex:CoordEvid}).  

% \begin{exe}
%   \ex  \label{ex:CoordEvid}
%   \begin{xlist}
%     \ex[]{\gll andri-s qor dargo-b-o-b-anŏ ye bubk’-i laħ-d-o-d-anŏ\\
%       Andre-\textsc{erg} apple(b/d).\textsc{abs}
%       plant-\textsc{cm-prs-cm-evid1} and flower(d/d)-\textsc{pl.abs}
%       pick-\textsc{cm-prs-cm-evid1}\\
%       \glt ‘Andre will evidently plant an apple and picks flowers.’}
%     \ex[*]{\gll andri-s qor dargo-b-o-b-anŏ ye bubk’-i laħ-d-o\\
%       Andre-\textsc{erg} apple(b/d).\textsc{abs}
%       plant-\textsc{cm-prs-cm-evid1} and flower(d/d)-\textsc{pl.abs}
%       pick-\textsc{cm-prs}\\
%       \glt \hfill\citep[p.~288]{Harris09}}
%     \end{xlist}
% \end{exe}

% Applying the test to intransitive markers (\ref{ex:CoordIntr}) or
% transitive markers (\ref{ex:CoordTr}) provides identical results: if
% either were still independent words, their inability to elide in
% coordination is surprising, whereas it is fully expected if these
% markers have the status of bound affixes. 

% \begin{exe}
%   \ex \label{ex:CoordTr}
%   \begin{xlist}
%     \ex[]{\gll as y-ot:-y-al-n-as zoreš c’eg-y-al-in-sŏ\\
%       I.\textsc{erg} \textsc{cm}-agitated-\textsc{cm-intr-aor-1sg.erg} much red-\textsc{cm-intr-aor-1sg.abs}\\
%       \glt `I became agitated and blushed very much.'}
%     \ex[*]{as y-ot:-y-al-n-as zoreš c’eg}
%   \end{xlist}
%   \ex \label{ex:CoordIntr}
%   \begin{xlist}
%     \ex[]{\gll as qor dargo-b-o-s ye bubk’-i laħ-d-o-s\\
%       I.\textsc{erg} apple(b/d).abs plant-\textsc{cm-prs-1sg.erg} and
%       flower(d/d)-\textsc{pl.abs} pick-\textsc{cm-prs-1sg.erg}\\
%       \glt `I (will) plant an apple and pick flowers.'
%     }
%     \ex[*]{as qor dargo-b-o-s ye bubk’-i laħ}
%   \end{xlist}
% \end{exe}


% \subsubsection{Agreement}

% The last piece of evidence towards bound affix status is provided by
% the respective agreement patterns observed by auxiliaries and the
% evidential marker.

% \citet{Harris09} shows on the basis of agreement controllers that
% evidentials (and transitive markers) do not pattern like auxiliaries,
% but rather reflect the argument structure of the lexical verb. As
% illustrated in (\ref{ex:AGRAUX}), auxiliaries pattern like
% intransitives, agreeing with the absolutive subject. 

% \begin{exe}
%   \ex \label{ex:AGRAUX} {\gll as qor dargo-b-o-s v-a\\
%     Andre(v/b).\textsc{abs} apple(b/d).\textsc{abs}
%     plant-\textsc{cm-absl} \textsc{cm}-be\\
%   \glt `Andre is planting an apple....' \hfill \citep[p.~284]{Harris09}}
% \end{exe}

% The evidential however does not appear to have an argument structure
% independently of that of the lexical verb: while agreement of the class
% marker is still controlled by the absolutive, it is the object `apple'
% of the transitive verb that controls agreement, as shown in (\ref{ex:AGRevid}). 
 

% \begin{exe}
%   \ex \label{ex:AGRevid}{\gll manana-s qor tet’-b-anŏ...\\
%     Manana(y/d)-\textsc{erg} apple(b/d).\textsc{abs} cut-\textsc{cm-evid1}\\
%   \glt `Evidently Manana is cutting an apple...'\citep[p.~284]{Harris09}}
%   \ex \label{ex:AGRtr}{\gll andri-s qor dargo-b-o-b-anŏ...\\
%     Andre(v/b)-\textsc{erg} apple(b/d).\textsc{abs} plant-\textsc{cm-prs-cm-evid1}\\
%   \glt `Evidently Andre is planting an apple...'\citep[p.~284]{Harris09}}
% \end{exe}

% A similar observation can be made for the transitive marker
% (phonologically elided here): in (\ref{ex:AGRtr}), both the class
% marker licensed by the transitive and the one introduced by the
% evidential do reflect the transitive valency of the main verb, as
% witnessed by the agreement with the absolutive object. If these markers
% are bound affixes, this behaviour is not at all surprising in that it
% just reflects the standard selection of the absolutive argument as the
% controller, i.e. the object of a transitive or the subject of an
% intransitive. If any of these markers were still an independent
% auxiliary, this would beg the question why their agreement patterns
% does not follow the intransitive model featured by the auxiliary in
% (\ref{ex:AGRAUX}).


To summarise, the evidence \citet{Harris09} provides robustly points
in the same direction, namely that transitivity markers and
evidential markers are bound affixes. Therefore, the issue of
exuberant exponence and the dependent nature of the class markers are
to be addressed in the domain of morphology rather than relegating
them to syntax.

\section{Discussion}
\label{sec:Discussion}

Exuberant exponence can probably be regarded as just another case of
extended (or multiple) exponence, so we would expect theories that
embrace the notion of many-to-many relations between function and form
to outperform those which picture  morphology
in terms of (classical) morphemes. This is indeed the
line of argumentation put forth by \citet{Harris09}. In her article,
she discusses the theoretical significance of extended exponence in
general and exuberant exponence in particular and confronts the Batsbi
facts with claims made by various theoretical frameworks. In
particular, she observes that incremental theories are uniformly hard
pressed to cover the empirical patterns, since these approaches assume
that morphological operations must always add information, as in the
lexical-incremental theory of \citet{Wunderlich95}, or must always
express information not yet expressed, as in the
inferential-incremental approach of \citet{Steele95}.




\subsection{Implications for lexical-realisational theories}

\citet{Harris09} already discusses in some depth the implications of
the Batsbi data for two instances of Distributed Morphology, a
lexical-realisational theory in terms of the typology of morphological
theories proposed by \citet{Stump01}.  She shows convincingly that
the theory of primary and secondary exponence advanced by
\citet{Noyer92} restricts extended exponence to maximally two
occurrences, which makes it impossible to capture the Batsbi data,
even though it may be adequate for Berber and Arabic, the languages
Noyer based his theory on.

In a paper on extended exponence in German, \citet{MuellerGereon07} suggests
to complement the theory of impoverishment (used in Distributed Morphology,
 \citealt{Halle93}, to account for syncretism) with a
theory of enrichment, in order to facilitate the treatment of extended
exponence. In the interest of limiting the formal complexity of a system that
recognises both deletion and insertion rules, he suggests that
enrichment may only redundantly add features already present. As shown by
\citet{Harris09}, enrichment rules indeed make it possible for a
lexical-realisational theory such as DM to cover the Batsbi
data.  The criticism she raises against the theory of enrichment is
more of a conceptual nature, essentially stating that lexical theories
are not well-equipped to capture the relevant generalisations directly,
but rather force the surface patterns into a
morphemic mould. 

While I concur with Harris's general assessment of the two DM
approaches, it is still worth noting that the problems faced by
\citet{Noyer92} and by \citet{MuellerGereon07}, are of an
entirely different nature: while  Müller's approach can indeed be
criticised for favouring a morphemic ideal and deriving exuberant
exponence by means of a ``workaround'', as argued convincingly in
\citet{Harris09}, it is equally clear that the theory of enrichment
meets at least the criterion of weak generative capacity, unlike
\citet{Noyer92}. One might even suggest that the division between a
morphemic core and enrichment could be motivated by considerations of
what is or could be considered typologically canonical or
unmarked. Noyer's theory contrasts sharply with that of Müller: 
his theory fails on grounds of weak capacity, i.e. it cannot even
describe the set of acceptable surface words. What is more, the reason for this
failure is located not at the level of the theory, where one might
just drop some universal claim in favour of a language-particular
constraint, but rather it is implemented at the level of the
underlying logic of feature discharge, meaning there is just no chance
of repair. To summarise, exuberant exponence falsifies Noyer's theory
of feature discharge, while Müller's theory appears to be flexible
enough to describe the facts.\largerpage[-1]

The observation that there is no clear alignment with general
properties of the approaches, but rather a strong dependence on the
details of implementation suggests that a typology of morphological
theories can only give a coarse indication of the analytical
properties of a theory and therefore still needs to be complemented by
careful investigation of the formal properties of the individual
approaches.
      

\subsection{Batsbi and inferential-realisational theories}

In contrast to both morpheme-based (=lexical) and incremental
theories, in\-feren\-tial-realisa\-tional theories generally embrace
extended exponence as a recurrent property in inflectional systems.
However, it seems that this very fact has led \citet{Harris09} to take
for granted that every approach within this family of theories will be
able to capture the empirical patterns. While there certainly is no
general obstacle, we shall see in this section that not all
word-and-paradigm theories are equally well-equipped to account for
the Batsbi data in an insightful and maximally general fashion. To
illustrate this point, I shall briefly discuss A-morphous Morphology
\citep[=AM;][]{Anderson92} and Paradigm Function Morphology
\citep[=PFM;][]{Stump01} and argue that it is important to submit to
further scrutiny the architectural decisions and the formal devices
offered by each theory.

A-morphous Morphology (AM) organises inflectional rules into a system
of ordered rule blocks that is used to derive affix order. While there
is preemption within rule blocks, by way of extrinsic rule ordering,
preemption does not generally apply across different blocks, thereby
making it possible in principle that a morphosyntactic property may
get expressed more than once.  However, AM does not provide any device
permitting reuse of resources across different rule blocks. Thus, while
extended exponence or even exuberant exponence per se is not a problem
at all for Anderson's model, the absence of, e.g. rules of referral
makes it difficult to capture the generalisation that exponents of
gender marking are indeed identical across different surface positions
in the word. Thus in addition to massive duplication of gender-marking
rules across different rule blocks, surface identity is pictured as
entirely accidental.\largerpage

Paradigm Function Morphology also builds on a system of extrinsically
ordered rule blocks and it equally limits rule competition to rules
within the same block. In contrast to AM, however, PFM does provide
rules of referral, either in terms of rules of referral to an ordered
rule block (cf. \citealt{Stump93}), or by means of ``conflation''
\citep{Stump17}. A solution along these lines clearly improves on
Anderson's theory, which addresses the question of weak but not strong
generative capacity. However, having both ordered and unordered rule
blocks, or rule blocks and conflation, provides for a rather baroque
structure that appears to work around what I consider a design flaw of
a rule block approach: being amorphous, PFM may look like the simpler
model as far as derived structure is concerned, but this comes at the
expense of an overly elaborate derivation structure.  Thus the absence
of morphological structure at the top-level is more than compensated
by having several layers of structure in the cascade of rules of
exponence and conflation rules, with intermediate representations at
every level. The morphous inferential-realisational analysis that I
shall present in Section~\ref{sec:Analysis}, by contrast, invokes no
structure at all beyond the assumption that exponents are segmentable,
an assumption which is by the way implicitly made by the PFM rule
system.


While at first sight, the move from ordinary
extended exponence to exuberant exponence appeared as a mere
quantitative difference, exuberance is actually a game-changer,
inducing a qualitative difference when confronted with concrete formal
theories: while incremental theories can indeed be discarded
en bloc, the ability to account for exuberant exponence does
not align with the distinction between lexical-realisational and
inferential-realisational theories. As we have seen there are
approaches of either type that can successfully analyse the data, as
well as approaches that fail to do so. That means that the ability to
capture exuberant exponence does not depend so much on the broad
affiliation within the typology of morphological theories but rather
on the specifics of the formal implementation.

\section{Harris's word-based approach}
\citet{Harris09} herself proposes a word-based analysis of Batsbi
class marking, inspired inter alios by \citet{Blevins06}, see
\citet{Blevins14} for a more recent reference.  Under a word-based
perspective, speakers are assumed to store paradigms of high frequency
words and establish analogical relationships between the cells of the
paradigm. Such analogical relations are abstracted from full or
partial paradigms, their application enabling speakers to form new
word forms from already memorised ones. For instance, given a stored
paradigm, word-to-word relations between paradigm cells can be
abstracted out, like the one in (\ref{ex:HarrisAnalog}):

\begin{exe}
  \ex \label{ex:HarrisAnalog} {[Gender $n$] $\sim$ [ CM$_n$-X] ↔ [Gender $m$] $\sim$ [ CM$_m$-X]}
\end{exe}\largerpage

According to her, such abstract relations, or the concrete
instantiations thereof, to gender/number features and their
corresponding surface exponents, make it possible to infer new forms
from known forms, e.g. \textit{yet:ŏ} `s/he pours milk' from
\textit{det:ŏ} `s/he pours tea' (recall that agreement is with
the absolutive, which is the object of a transitive in this case).

For Batsbi, \citet{Harris09} assumes that lexical items and affixes
each give rise to two basic schemata, one that features a class marker
(\textbf{d}-\textsc{lex}/\textbf{d}-\textsc{aff}), and one that does
not (\textsc{lex}/\textsc{aff}). Based on the lexical schemata, Harris
suggests that basic verbs like \textit{\textbf{d}-ek’-i$^n$} ‘they
fell' and \textit{ot’-ŏ} `they spread it' can be schematised as
[\textbf{d}-\textsc{lex}]$_V$ and [\textsc{lex}]$_V$,
respectively. 

She then moves on to ``first order'' extensions, including transitive
and intransitive markers and suggests two abstract schemata
[V-\textbf{d}-\textsc{aff} ]$_V$ and [V-\textsc{aff} ]$_V$ the first
of which is instantiated in the following sub-schemata (\tabref{ex:HarrisFirstOrder}).  

\begin{table}[h]
  \caption{Transitive/intransitive first order extensions \label{ex:HarrisFirstOrder}}
  \begin{tabular}{rlll}
    \lsptoprule
      & Sub-schema & Example & Translation\\
    \midrule
    a. & [[\textbf{d-}\textsc{lex} ]$_V$ -\textbf{d-}i-]$_V$ &
                                                      \textbf{d-}ek’-\textbf{d-}iy-en & ‘threw it off’\\
    b. & [[\textbf{d-}\textsc{lex} ]$_V$ -\textbf{d-}al]$_V$ &
                                                      \textbf{y}-opx-\textbf{y}-al-in=e & ‘dressed and’\\
    c. & [[\textsc{lex}]$_V$ -\textbf{d-}i-]$_V$ &
                                          tat:-\textbf{b}-iy-en & ‘pushed it’  \\
    d. & [[\textsc{lex}]$_V$ -\textbf{d-}al-]$_V$ &
                                           oc’-\textbf{v}-al-in-es & ‘I weighed’ \\
    \lspbottomrule
  \end{tabular}
\end{table}

In order to incorporate second order extensions such as the evidential
I and the aorist evidential, \citet{Harris09} proposes even more
complex sub-schemata, illustrated in \tabref{ex:HarrisSecondOrder}.

\begin{table}
  \caption{Second order schemata\label{ex:HarrisSecondOrder}}
  \begin{tabularx}{\textwidth}{rlQ}
    \lsptoprule
    &  Sub-schema
    & Explanation\\
    \midrule
    a. & [[\textbf{d}-\textsc{lex}]$_V$ -\textbf{d}-anŏ]$_V$  & evidential I of simple verb with preradical CM\\
    b. & [[[\textbf{d}- \textsc{lex} ]$_V$  -\textbf{d}-i]$_V$  -\textbf{d}-anŏ]$_V$  & evidential I of derived transitive with preradical CM\\
    c. & [[\textbf{d}- \textsc{lex} ]$_V$ -inŏ]$_V$  & aorist evidential of simple verb with preradical CM\\
    d. & [[[\textbf{d}- \textsc{lex} ]$_V$  -\textbf{d}-i]$_V$ -inŏ]$_V$ & aorist evidential of derived transitive with preradical CM\\
    \lspbottomrule
  \end{tabularx}
\end{table}

As indicated by \citet{Harris09}, the sub-schemata in
\tabref{ex:HarrisSecondOrder} are only a subset of the actual number of
schemata. Factoring in only the stem and transitive/intransitive
schemata, the number grows to 16. Once we factor in TAM markers
(e.g. present, imperfective or aorist), we end up with a considerably
greater number. The word-based approach therefore does not appear to
be a very economical way of capturing the dependency of a class marker
on the marker that licenses it. What is more, such a view will hardly
scale up to the description of morphologically even more complex
languages. Finally, a word-based view misses the utterly local nature
of licensing involved with class marking.

It is rather clear what the basic intuitions are that Harris intends
to capture with her (informal) analysis: to account for the dependent
nature of gender markers (via schemata) and their uniform pattern of
alternation (via analogy). It is far less clear though how the
different abstractions of intermediate structures that she offers are
to be interpreted in a word-based model. As a result, there are two
basic readings of her analysis that I shall assume as plausible for
the rest of this chapter: a purely word-based view, where intermediate
abstractions are just abbreviatory devices \citep[298]{Harris09},
or a constructional view where such abstractions are meant to have
some theoretical status. Depending on which of the two readings
is correct, the current paper will make a different contribution: if
the latter, it will provide a formal interpretation of
\citet{Harris09}, leading to a clearer understanding of what the different
variables (depicted in bold face or small caps) are and how they can
be interpreted in a generative grammar that makes use of typed feature
logic.  If, however, the former, word-based interpretation is more
faithful, it will show in addition how a schema-based approach of
Batsbi can be formalised in a rigorous fashion, without necessitating
a fully holistic, word-based view.

In the next section, I shall therefore present an alternative analysis
of Batsbi exuberant exponence, one which completely avoids unfolding the
entire morphotactics into primary and secondary sub-schemata, but
relies instead on a typed feature logic to give a formal
interpretation to the basic combination of class-markers and the
exponents that license their occurrence.  

  


\section{Analysis}
\label{sec:Analysis}

In this section, I shall present Information-based Morphology, an
inferential-realisational theory of inflection and show how the two
basic analytical devices, inheritance and cross-classification in
typed feature structures, are sufficient to provide an analysis of
Batsbi exuberant exponence that captures simultaneously the dependent
nature of class markers and the uniformity of their
exponence. Furthermore, this analysis will highlight how Harris'
original proposal, when understood in constructional rather than
word-based terms, can be given a straightforward formal interpretation
using the IbM framework.



\subsection{Information-based Morphology}
\label{sec:IbM}

In this section\footnote{This section has been largely reproduced from
  \citet{Crysmann:Bonami:2017:HPSG}. For an overview of alternative
  approaches to morphology within HPSG and constraint-based grammar,
  please see \citet{Bonami15b}.

  One difference between the current version of IbM and previous ones
  is that we have now settled on considering \textsc{mph} as a list rather
  than a set.}, I shall present the basic architecture of
Information-based Morphology
\citep[IbM, ][]{Crysmann:Bonami:2016,Crysmann:14:OUP}, an
inferential-realisa\-tional theory of inflection \citep[cf.][]{Stump01}
that is couched entirely within typed feature logic, as assumed in
HPSG \citep{Pollard87,Pollard94}. In IbM, realisation rules embody
partial generalisations over words, where each rule may pair $m$
morphosyntactic properties with $n$ morphs that serve to express them.
IbM is a morphous theory \citep{Crysmann:Bonami:2016}, i.e. exponents
are described as structured morphs, combining descriptions of shape
(=phonology) and position class. As a consequence, individual rules
can introduce multiple morphs, in different, even discontinuous
positions. By means of multiple inheritance hierarchies of rule types,
commonalities between rules are abstracted out: in essence, every
piece of information can be underspecified, including shape, position,
number of exponents, morphosyntactic properties, etc.

In contrast to other realisational theories, such as Paradigm Function
Morphology \citep{Stump01} or A-morphous Morphology
\citep{Anderson92}, IbM does away with procedural concepts such as
ordered rule blocks. Moreover, rules in IbM are non-recursive,
reflecting the fact that inflectional paradigms in general constitute
finite domains.  Owing to the absence of rule blocks, IbM embraces a
strong notion of Pāṇini's principle or the elsewhere condition
\citep{kiparsky_p85} which is couched purely in terms of informational
content (=subsumption) and therefore applies in a global fashion
\citep{Crysmann:14:OUP}, thereby including discontinuous bleeding
\citep{Noyer92}.

\subsubsection{Inflectional rules as partial abstraction over words}
From the viewpoint of inflectional morphology, words can be regarded
as associations between a phonological shape (\textsc{ph}) and a
morphosyntactic property set (\textsc{ms}), the latter including, of
course, information pertaining to lexeme identity. This correspondence
can be described in a maximally holistic fashion, as shown in Figure
\ref{fig:Word}. Throughout this section, I shall use German
(circumfixal) passive/past participle (\emph{ppp}) formation, as
witnessed by \textit{ge-setz-t} `put', for illustration.

\begin{figure}
  \avm{
    [ ph & \upshape gesetzt\\
      ms & \{[lid & setzen],[tma & ppp]\}
    ]
    }  
  \caption{Holistic word-level association between form (\textsc{ph}) and
    function (\textsc{ms})}
  \label{fig:Word}
\end{figure}

Since words in inflectional languages typically consist of multiple
segment\-able parts, realisational models provide means to index
position within a word: while in AM and PFM ordered rule blocks
perform this function, IbM uses a list of morphs (\textsc{mph}) in order
to explicitly represent exponence. Having morphosyntactic
properties and exponents represented as sets and lists, standard issues in
inflectional morphology are straightforwardly captured at the level of rules: cumulative
exponence corresponds to the expression of $m$ properties by 1 morph,
whereas extended (or multiple) exponence corresponds to 1 property
being expressed by $n$ morphs. Overlapping exponence finally
represents the general case of $m$ properties being realised by $n$
exponents. Figure \ref{fig:WordMph} illustrates the word-level $m:n$
correspondence of lexemic and inflectional properties to the multiple
morphs that realise it. By means of simple underspecification,
i.e. partial descriptions, one can easily abstract out realisation of
the past participle property, arriving at a direct representation of circumfixal
realisation.

\begin{figure}
    \begin{minipage}[t]{.55\textwidth}\centering\small
      Word:\medskip\\
      \avm{
        [ ph & \rm gesetzt\\mph & <[ph &  \rm ge\\ pc & $-1$], [ph &  \rm setz\\ pc & $0$], [ph & \rm t\\ pc & $1$]>\\
          ms & \{[lid & setzen],[tma & ppp]\}
        ]}
      \end{minipage}\begin{minipage}[t]{.45\textwidth}\centering\small
      Abstraction of circumfixation ($1:n$):\medskip\\
       \avm{
         [ mph & <[ph &  \rm ge\\ pc & $-1$],[ph & \rm t\\ pc & $1$],\ldots>\\
           ms &  \{[tma & ppp],\ldots\} ]
         }
      \end{minipage}
  \caption{Structured association of form (\textsc{mph}) and function (\textsc{ms}) }
  \label{fig:WordMph}
\end{figure}

Yet, a direct word-based description does not easily capture situations
where the same association between form and content is used more
than once in the same word, as is arguably the case for Swahili
\citep{Stump93,Crysmann:Bonami:2016,Crysmann:Bonami:2017:HPSG} or,
even more importantly for Batsbi \citep{Harris09}. By way of
introducing a level of \textsc{r(ealisation) r(ules)}, reuse of
resources becomes possible. Rather than expressing the relation
between form and function directly at the word level, IbM assumes
that a word's description includes a specification of which rules
license the realisation between form and content, as shown in Figure
\ref{fig:WordRR}.


\begin{figure}
    \avm[pic,picname=CRYS1]{
      [ mph & <\node{xx}{[ph & \rm ge\\ pc & $-1$]} \node{aa}{[ph &  \rm setz\\ pc & $0$]},\node{ab}{[ph &  \rm t\\ pc & $1$]}> \bigskip\\
        rr & \{	[	mph & \{\node{ba}{[ph &  \rm setz\\ pc & $0$]}\}\\ mud & \{\node{ca}{[lid & setzen]}\} ], [	mph & \{\node{yy}{[ph &  \rm ge\\ pc & $-1$]},\node{bb}{[ph &  \rm t\\ pc & $1$]}\}\\ mud & \{\node{cb}{[tma & ppp]}\} ] \}\bigskip\\
        ms & \{\node{da}{[lid & setzen]},\node{db}{[tma & ppp]}\}
      ]}
    \begin{tikzpicture}[remember picture, overlay]
    \foreach \x/\y in {xx/yy, aa/ba, ab/bb, ca/da, cb/db}
      \path[{Stealth[]}-{Stealth[]},dotted, lsGuidelinesGray, out=-90, in=90] (CRYS1-\x) edge (CRYS1-\y);
    \end{tikzpicture}
% %     \psset{arrows=<->,linecolor=gray,linestyle=dotted,angleA=-90,angleB=90}
% %     \nccurve{xx}{yy} \nccurve{aa}{ba} \nccurve{ab}{bb}
% %     \nccurve{ca}{da} \nccurve{cb}{db}
      \caption{Association of form and function mediated by rule\label{fig:WordRR}}
\end{figure}

Realisation rules (members of set \textsc{rr}) pair a set of
morphological properties to be expressed, the morphology under
discussion (\textsc{mud}) with a list of morphs that realise them
(\textsc{mph}). In order to facilitate generalisations about shape and
position in an independent fashion, IbM recognises each of them as
first order properties of morphs, where \textsc{ph} represents
a description of the phonological shape,\footnote{For ease of
  presentation, I shall use strings to represent phonological
  contributions. More generally, \textsc{ph(on)} value should be
  considered descriptions of phonological events, as suggested e.g.\ by
\citet{Bird:Klein:94}.} whereas \textsc{pc} corresponds
to position class information.  A general principle of morphological
well-formedness (Figure \ref{fig:MCC}) ensures that the properties
expressed by rules add up to the word's property set and that the
rules' \textsc{mph} list add up to that of the word, i.e. no
contribution of a rule may ever be lost.\footnote{The principle of
  general well-formedness in Figure~\ref{fig:MCC} bears some
  resemblance to LFG's principles of completeness and coherence
  \citep{bresnan_j82}, as well as to the notion of ``Total
  Accountability'' proposed by \citet{Hockett47}. Since $m:n$ relations
  are recognised at the most basic level, i.e. morphological rules,
  mappings between the contributions of the rules and the properties
  of the word can and should be $1:1$. } In essence, a word's sequence
of morphs, and hence, its phonology will be obtained by shuffling
($\bigcirc$) the rules' \textsc{mph} lists in ascending order of
position class (\textsc{pc}) indices (see \citealt{bonami-crysmann:2013} for details). 
Similarly, a word's morphosyntactic
property set (\textsc{ms}) will correspond to the non-trivial set
union ($\uplus$) of the rules' \textsc{mud} values.\footnote{While
  standard set union ($\cup$) allows for the situation that elements
  contributed by two sets may be collapsed, non-trivial set union
  ($\uplus$) insists that the sets to be unioned must be disjoint.}
Finally, the entire morphosyntactic property set of the word (\textsc{ms}) is
exposed on each realisation rule by way of structure sharing
(\negmedspace\avm{\0}).

\begin{figure}
    \avm{
      \textit{word} $\rightarrow$
      [mph & \tag{$e_1$} $\bigcirc\dots\bigcirc$ \tag{$e_n$}\smallskip\\
        rr & \{ [mph & \tag{$e_1$}\\mud & \tag{$m_1$}\\ms & \0] ,\ldots\ ,
        [mph & \tag{$e_n$}\\mud & \tag{$m_n$}\\ ms & \0] \}\\
        ms & \tag{$m_1$} $\uplus\dots\uplus$ \tag{$m_n$}
      ]
    }
  \caption{Morphological well-formedness\label{fig:MCC}}
\end{figure}

This latter aspect, i.e. the relationship between \textsc{mud} and
\textsc{ms} in rule descriptions, surely deserves some more
clarification in the context of this chapter. IbM makes a deliberate
distinction between expression of a property and conditioning on a
property: while \textsc{mud} represents expression of properties,
constraints on the \textsc{ms} set serve to capture allomorphic
conditioning, in the sense of \citet{Carstairs87}. There are two
important consequences of this distinction \citep{Crysmann:14:OUP}:
first, it becomes possible to make the application of inflectional
rules a direct function of the information to be expressed, without
having to postulate a system of (ordered) rule blocks. Second, it
paves the way for a global notion of Pāṇinian competition, being able
to distinguish between situations of discontinuous bleeding
\citep{Noyer92} and multiple or overlapping exponence. Thus, a rule
with [\textsc{mud} \{~$\alpha,\beta$~\}] would preempt a rule with
[\textsc{mud}~\{~$\beta$~\}], since every morphosyntactic property is
licensed (expressed) by exactly one rule. The rules
[\textsc{mud}~\{~$\alpha$~\}, \textsc{ms}~\{~$\beta$~\}] and
[\textsc{mud}~\{~$\beta$~\}], by contrast, would give rise to
overlapping exponence (provided exponents do not compete for
position). Here, expression of $\alpha$ is merely conditioned on a
property that is independently expressed: $\beta$. See
\citet{Crysmann:14:OUP} for extensive discussion of preemption and
overlapping exponence in Swahili.

Realisation rules conceived like this essentially constitute partial
abstractions over words, stating that some collection of morphs
jointly expresses a collection of morphosyntactic properties. In the
example in Figure \ref{fig:WordRR}, we find that realisation rules
thus conceived implement the $m:n$ nature of inflectional morphology
at the most basic level: while the representation of
classical morphemes as $1:1$ correspondences is permitted, it is but one
option. The circumfixal rule for past participial inflection directly
captures the $1:n$ nature of extended exponence.


 
\subsubsection{Levels of abstraction}

The fact that IbM, in contrast to PFM or AM, recognises $m:n$
relations between form and function at the most basic level of
organisation, i.e. realisation rules, means that morphological
generalisations can be expressed in a single place, name\-ly simply as
abstractions over rules. Rules in IbM are represented as typed feature
structures organised in an inheritance hierarchy, such that properties
common to leaf types can be abstracted out into more general
supertypes. This vertical abstraction is illustrated in Figure
\ref{fig:Vertical}. Using again German past participles as an example,
the commonalities that regular circumfixal \textit{ge-...-t} (as in
\textit{gesetzt} `put') shares with subregular \textit{ge-...-en} (as
in \textit{geschrieben} `written') can be generalised as the
properties of a rule supertype from which the more specific leaves
inherit. Note that essentially all information except choice of
suffixal shape is associated with the supertype. This includes the
shared morphotactics of the suffix.

\begin{figure}
    \begin{forest}
      [\avm{
          [mud & \{ [tma & ppp] \}\\
            mph & < [ph & \rm ge\\ pc & $-1$ ],
            [pc & $1$ ] > ]
        }
       [\avm{[mph & < ..., [ph & \rm t ] > ]}]
       [\avm{[mph & < ..., [ph & \rm en] > ]}] 
      ]
    \end{forest}
  \caption{Vertical abstraction by inheritance\label{fig:Vertical}}
\end{figure}

In addition to vertical abstraction by means of standard monotonic
inheritance hierarchies, IbM draws on online type construction
\citep{Koenig94}: using dynamic cross-classification, leaf types from
one dimension can be distributed over the leaf types of another
dimension. This type of horizontal abstraction permits modelling of
systematic alternations, as illustrated once more with German past
participle formation:

\begin{exe}
  \ex \label{ex:ppp}
  \begin{xlist}

    \ex \textbf{ge}-setz-\textbf{t} `set/put'
    \ex über-setz-\textbf{t} `translated'
    \ex \textbf{ge}-schrieb-\textbf{en} `written'
    \ex über-schrieb-\textbf{en} `overwritten'
  \end{xlist}
\end{exe}

In the more complete set of past participle formations shown in
(\ref{ex:ppp}), we find alternation not only between choice of suffix
shape (\textit{-t} vs. \textit{-en}), but also between presence
vs. absence of the prefixal part (\textit{ge-}).

\begin{figure}

  \oneline{
    \begin{forest}
      [\avm
         {[mud & \{[tma & ppp]\}\\
          mph & < ..., [pc & $1$] >]}
        [\DIMBOX{PREF}, for children={anchor=north}
          [\avm{[mph & < [ph & \rm ge\\pc & $-1$], [ ~ ]>]}]
          [\avm{[mph & < [ ~ ] > ]}]
        ]
        [\DIMBOX{SUFF}, for children={anchor=north}
          [\avm{[mph & < ..., [ph & \rm t ]>]}]
          [\avm{[mph & < ..., [ph & \rm en]>]}]
        ]
      ]
    \end{forest}
    }
  \caption{Horizontal abstraction by dynamic cross-classification\label{fig:Horizontal}}
\end{figure}

Figure \ref{fig:Horizontal} shows how online type construction enables
us to generalise these patterns in a straightforward way: while the
common supertype still captures properties true of all four different
realisations, namely the property to be expressed and the fact that it
involves at least a suffix, concrete prefixal and suffixal realisation
patterns are segregated into dimensions of their own (indicated by
\DIMBOX{PREF} and \DIMBOX{SUFF}). Systematic cross-classification
(under unification) of types in \DIMBOX{PREF} with those in
\DIMBOX{SUFF} yields the set of well-formed rule instances,
e.g. distributing the left-hand rule type in \DIMBOX{PREF}  over the types
in \DIMBOX{SUFF} yields the rules for \textit{ge-setz-t} and
\textit{ge-schrieb-en}, whereas distributing the right hand rule type in
\DIMBOX{PREF} gives us the rules for \textit{über-setz-t} and
\textit{über-schrieb-en}, which are characterised by the absence of
the participial prefix.

\subsection{An information-based account of Batsbi exuberant exponence }
\label{sec:IbM:Batsbi}

Having introduced the basic workings of IbM, we are now in a position
to approach the analysis of exuberant exponence in Batsbi. For the
purposes of the following discussion, recall the two most central
observations made in Section \ref{sec:Data}: first that the shape of class markers
remains identical across all occurrences, and second, that the presence
vs. absence of a class marker depends on their right-adjacent
marker. Thus we saw both stems that trigger presence of an immediately
preceding class marker, and stems that do not. Similarly, some classes
of affixal exponents are obligatorily accompanied by a left-adjacent
marker, whereas others do not license presence of such a marker. As a
consequence, a word may surface with multiple identical class
markers, a single pre-stem class marker or a single suffixal class
marker, or even with no overt class marker at all.

The analysis I shall put forth in this section is that stems and
affixes that trigger presence of overt agreement are actually 
allomorphically conditioned on gender marking properties, but that
expression of gender marking can be zero, in the limiting case. 

By way of illustration, let us start with a sample analysis of a word
featuring exuberant exponence. As given in Figure~\ref{fig:Batsbi},
the word's \textsc{mph} list features two occurrences of the gender V/VI plural
marker \textit{d}, each adjacent to a trigger, the stem \textit{ek'}
and the transitive marker \textit{-iy}. 

\begin{figure}
  
  \avm[pic, picname=CRYS2]{
    [ ph(on) & \upshape\node{p1}{\textbf{d}}\node{p2}{{ek'}}\node{p3}{\textbf{d}}\node{p4}{{iy}}\node{p5}{ẽ}\bigskip\bigskip\\
      mph & <
      \node{m1}{\tag{a}}[ph & \upshape\node{p11}{\textbf{d}}\\ pc & $-1$],
      \node{m2}{\tag{b}}[ph & \upshape\node{p21}{{ek'}}\\ pc &    $0$],
      \node{m3}{\tag{c}}[ph & \upshape\node{p31}{\textbf{d}}\\ pc & $1$],
      \node{m4}{\tag{d}}[ph & \upshape\node{p41}{{iy}}\\ pc & $2$],
      \node{m5}{\tag{e}}[ph & \upshape\node{p51}{ẽ}\\ pc & $3$]
      >\bigskip\\
      rr & \{
      [mph & <\node{m11}{\tag{a}}, \node{m21}{\tag{b}}>\\ms & \{\tag{w}, ...\}\\ mud & \{\node{s1}{\tag{t}}\}],
      [mph & <\node{m31}{\tag{c}},\node{m41}{\tag{d}}>\\ms & \{\tag{w}, ...\}\\ mud & \{\node{s2}{\tag{u}}\}],
      [mph & <\node{m51}{\tag{e}}>\\ mud & \{\node{s3}{\tag{v}}\}], [mph & < >\\ mud & \{\node{s4}{\tag{w}}\}]
      \}\\ \\
      ms & \{ \node{s11}{\tag{t}}[\type*{lid}
        stem1 & < [ph \normalfont ak']>\\
        stem2 & < ..., \tag{b}[ph \normalfont ek']>
      ], \node{s21}{\tag{u}}{\myit trans}, \node{s31}{\tag{v}}{\myit aor},
      \node{s41}{{\tag{w}}}[\type*{abs-agr} num & pl\\gend & V/VI]
      \}
    ]}
  \begin{tikzpicture}[remember picture, overlay]
    \foreach \x/\y in {m2/m21, m3/m31, m4/m41, m5/m51, s1/s11, s2/s21, s3/s31, s4/s41}
      \path[{Stealth[]}-{Stealth[]},dotted, black!50, out=-90, in=90] (CRYS2-\x) edge (CRYS2-\y);
    \foreach \x/\y in {p1/p11, p2/p21, p3/p31, p4/p41, p5/p51}
      \draw[{Stealth[]}-{Stealth[]},dotted, black!50, out=-90, in=90, looseness=.25] (CRYS2-\x.base) edge (CRYS2-\y);
  \end{tikzpicture}
%   \psset{arrows=<->,linestyle=dotted,linecolor=gray,angleA=-90,angleB=90,nodesep=1pt,arm=0pt}
%   \ncdiag{p1}{p11} \ncdiag{p2}{p21} \ncdiag{p3}{p31}
%   \ncdiag{p4}{p41} \ncdiag{p5}{p51} \nccurve{m1}{m11}
%   \nccurve{m2}{m21} \nccurve{m3}{m31} \nccurve{m4}{m41}
%   \nccurve{m5}{m51} \nccurve{s1}{s11} \nccurve{s2}{s21}
%   \nccurve{s3}{s31} \nccurve{s4}{s41}
  % \nccurve[angle=270]{s11}{s12} \nccurve{s21}{s22}
  % \nccurve[angleB=0]{s31}{s32} \nccurve{s41}{s42}
  \caption{Sample analysis of Batsbi exuberant exponence\label{fig:Batsbi}}
\end{figure}

As indicated by coindexation, each instance of the agreement marker is
introduced by the same realisation rule as its trigger, e.g. a single
rule introduces both the stem \textit{ek'} (\negmedspace\avm{\tag{b}}) and its
dependent class marker (\negmedspace\avm{\tag{a}}). The same holds for the
transitivity marker \textit{-iy} (\negmedspace\avm{\tag{d}}) and its accompanying
class marker (\negmedspace\avm{\tag{c}}). Each of these complex rules expresses some
property other than class agreement, as indicated by their
\textsc{mud} value, e.g. lexemic identity (\negmedspace\avm{\tag{t}}), or
transitivity (\negmedspace\avm{\tag{u}}), but both are conditioned on the
morphosyntactic property of gender/number agreement (\negmedspace\avm{\tag{w}}),
specified as a constraint on the entire \textsc{ms} set. Since
gender/number agreement has no expression independent of a trigger,
and since in many words there is no overt exponent of class marking
agreement, owing to the fact that only around 25\% of stems and a
select few suffixes license these dependent markers, I shall assume
that class marking is expressed by default zero realisation, i.e. a
rule that realises any property that has no more specific realisation
rule by the empty set of morphs.\footnote{This rule is similar in
  spirit to the identity function default of \citet{Stump01}. Note
  that in IbM, just like in PFM, this kind of default reasoning is
  part of the logic, based entirely on the notion of
  information. Furthermore, it only applies between rule instances,
  i.e. leaves of the hierarchy, leaving multiple inheritance in the
  type hierarchy entirely monotonic. This contrasts sharply with
  Network Morphology \citep[=NM][]{Brown11}, where defaults are used
  at the description level and at any node in the hierarchy,
  necessitating strong assumptions about orthogonality of properties
  in order to keep resolution of defaults sound. In the remainder of
  this chapter, I shall make no further reference to NM, for the
  simple reason that, as far as I am aware, the two areas under
  discussion here, i.e. multiple exponence and morphotactics, have not
  been the focus of research in that framework, making
  it difficult to assess its predictions. } When class agreement does
indeed surface, its dependent nature is best understood in terms of
inflectional allomorphy.

\subsection{Rule types for gender/number marking}\largerpage

Having sketched the overall line of analysis,  
I shall now present a  description of the actual rule system starting
with the type hierarchy that associates gender/number agreement features with
any particular shape of class marker.  

At the top of the hierarchy in Figure \ref{fig:CM}, we find properties
common to all class markers. First and foremost, the morphotactic
description on \textsc{mph} captures the fact that all class marking
is dependent, consisting of two adjacent morphs. This basic property
is expressed by means of requiring the list of morphs to be
contributed by any class-marking rule to be bimorphic, i.e. a list of
length 2. The phonology (\textsc{ph}) and position class (\textsc{pc})
of the morphs thus contributed are further constrained to have a
consonantal morph immediately followed by a vowel-initial one, as
dictated by the strictly consecutive position class indices. Second,
the general rule type and its subtypes are restricted to have an
\textit{abs-agr} feature structure on the morphosyntactic property
set.


\begin{sidewaysfigure}
  \small
  \begin{forest}
    [\avm{[ms & \{ [\type{abs-agr}], ... \}\\
         mph & < [ph & \upshape C\\
            pc & \tag{i} ], [ph &  V ...\\pc & \tag{i} + $1$ ] > ]}, for children={anchor=north}
         [\avm{[ ms & \{[\type*{abs-agr}
                  gend & I\\
                  num & sg], ... \}\\
                mph & < [ph & \upshape v  ], ... > ]}]
         [\avm{ [mph & < [ph &  \upshape y ], ...>] }
             [\avm{ [ms \{[\type*{abs-agr} gend & III], ...\} ]}]
             [\avm{ [ms \{[\type*{abs-agr} gend & II\\ num & sg], ...\}]}]
         ]
         [\avm{[mph & < [ph & \upshape d  ], ... > ]}]
         [\avm{[mph & < [ph & \upshape b  ], ... > ]}
             [\avm{[ms \{[\type*{abs-agr} gend & I\\num & pl], ...\}]}]
             [\avm{[ms \{[\type*{abs-agr} gend & VI\\num & sg],...\}]}]
         ]
    ]
  \end{forest}
  \caption{Subhierarchy of class marker rules\label{fig:CM}}
\end{sidewaysfigure}

Subtypes in the hierarchy in Figure~\ref{fig:CM} now further constrain
the shape of the class marker. At the first level down in the
hierarchy, the phonological shape of the initial consonantal marker is
fixed. While \textit{v-} is restricted to the singular of gender I and
\textit{d-} is treated as the default class marker, the two remaining
markers \textit{b-} and \textit{j-} are both subject to unmotivated
syncretism. This can be captured in a straightforward way by fixing
their morphosyntactic constraints extensionally on the subtypes they
dominate. This is possible since rule instances in IbM are only ever
based on leaf types, following \citet{Koenig99}. 

As given in Figure~\ref{fig:CM} (page~\pageref{fig:CM}), 
the rule type for default CM marking
is fully underspecified. The version of  Pāṇini's principle that IbM
assumes will actually preempt application of any more general rule in
the presence of a more specific one. 

\ea Pāṇinian competition (PAN) \hfill \citep{Crysmann:14:OUP} 
  \begin{xlistn}
    \ex For any leaf type $t_1$[\textsc{mud} $\mu_1$,\textsc{ms}
    $\sigma$], $t_2$[\textsc{mud} $\mu_2$,\textsc{ms}
    $\sigma \wedge \tau$] is a morphological competitor, iff
    $\mu_1 \subseteq \mu_2$.
    
    \ex For any leaf type $t_1$ with competitor $t_2$, expand
    $t_1$'s \textsc{ms} $\sigma$ with the negation of $t_2$'s
    \textsc{ms} $\sigma \wedge \tau$:
    $\sigma \wedge \neg (\sigma \wedge \tau) \equiv \sigma \wedge
    \neg \tau$.
  \end{xlistn}
\z

According to Pāṇinian competition, which is a closure operation on the
type hierarchy, the \textsc{ms} set of the more general description
for the default marker \textit{d-} will end up being specialised to
the description in Figure~\ref{fig:Panini:BC}, which is essentially
complementation with respect to the descriptions of its
competitors. 

\begin{figure}
 \avm{
   [mph & < [ph & \upshape d \\
     pc & \tag{$i$}], [ph & \upshape V...\\pc &  \tag{$i$} + 1] >\\
              ms & \{\myit abs-agr,...\}
            $~\wedge$ (
            $\neg$ \{[\type*{abs-agr} gend & I], ... \} $~\wedge$\smallskip\\
            \punk{$\neg$ \{[\type*{abs-agr} gend & II\\ num & sg], ... \} $~\wedge$}{} \smallskip\\
            $\neg$ \{[\type*{abs-agr} gend & III], ...\} $~\wedge$\smallskip\\
            $\neg$ \{[\type*{abs-agr} gend & VI\\ num & sg], ...\}
            )
    ]
    }
  \caption{Pāṇinian competition applied to default CM marker \textit{d-}\label{fig:Panini:BC}}
\end{figure}

    
\subsection{Deconstructing class marking (suffixes)}

Having introduced the partial constraints on the shape and position of
the class markers, we are now in a position to bring them together with
the suffixal markers on which they depend. The essential analytic
device we shall rely on is online type construction \citep{Koenig94},
which enables us to state constraints on class markers and their
licensors in dimensions of their own, yet distribute rule types in one
dimension over the types in the other. Thus, each individual ingredient can be
described in the most general way, while at the same time we can ensure
their systematic combination.

\begin{sidewaysfigure}\footnotesize
  \begin{forest}
      [\avm{
        [\type*{realisation-rule}
          mud & \tag{$m$}\\
          ms  & \tag{$m$} $\cup$ set\\
          mph & list]}
          [\DIMBOX{ALLOMORPHY}, for children = {anchor=north}
              [\avm{
                [ms  & \{ [\type{abs-agr}], ... \}\\
                 mph & < [ph & C \\
                 pc & \tag{$i$} ], [ph &  V ... \\
                 pc & \tag{$i$}$ + 1$ ] > 
                 ]
               }
                   [...] 
                   [\avm{[mph < [ph  \normalfont d  ], ...> ]}]
               ]
               [[\avm{[mph & < [ ~ ] >]}]]
          ]
          [\DIMBOX{EXPONENCE}
              [\avm{[mud & \{[\type{intr}]\}\\
                mph & < ..., [ph & \normalfont al\\
                pc & 2]>]}]
              [\avm{[mud & \{[\type{evid1}]\}\\
              mph & < ..., [ph & \normalfont anǒ \\
                pc & 5]>]}]
              [\avm{
            [mud & \{[\type{aor}]\}\\
              mph & < [ph & \normalfont en\\
                  pc & 3]>] ... }]
% %               [\avm{
% %           [mud & \{[\type*{lid} stem1 & \tag{$s$}],\\
% %             [\type{perf}]\}\\
% %             mph & \tag{$s$}]}]
          ]
      ]\end{forest}
  \caption{Hierarchy for suffix and class marking rule types\label{fig:CrossSuff}}
\end{sidewaysfigure}

The hierarchy of rule types in Figure~\ref{fig:CrossSuff} is organised
into two dimensions, labelled \DIMBOX{ALLOMORPHY} and
\DIMBOX{EXPONENCE}. In the former, one finds the type hierarchy of
class  marking from Figure~\ref{fig:CM}, with class-marking leaf types
abbreviated by the representative rule type for the
\textit{d-}marker. In the \DIMBOX{EXPONENCE} dimension, we find
realisation rule types for markers that show class-marking allomorphy,
such as the present evidential or the intransitive, and some that do
not. All realisation rules in this dimension specify a morphosyntactic
property to be expressed via their non-empty \textsc{mud} set, and all
of them pair this property with a constraint on the exponent that
serves to express this property, consisting of a phonological
description and a position class index. The crucial difference between
exponents that are accompanied by a class marker and those that are not
is the constraint on the cardinality of the \textsc{mph} set: while
the latter specify a closed list (of length~1), those that do
require a class marker are characterised by an open list.

Building on online type construction \citep{Koenig94}, IbM obtains the
set of rule instances by systematic intersection, under unification,
of every leaf type from every dimension with every leaf type from
every other dimension. The rule instances thus inferred from the type
hierarchy are then subject to Pāṇinian competition.

Rule types that do not take a class marker specify a monomorphic
\textsc{mph} set and therefore fail to unify with any of the class
marking constraints, which are constrained to have a bimorphic
\textsc{mph} set, as specified on their supertype. Thus, rule types
such as the one for the aorist can only combine with the rightmost
leaf type in the \DIMBOX{ALLOMORPHY} dimension, which merely
constrains the cardinality of the \textsc{mph} set to 1. Rule types
that do take class markers, by contrast, do unify with the class
marking constraints, yielding all combinations of class markers with
the triggering marker.\footnote{To be exact, triggering markers will
  also combine with the underspecified monomorphic rule type. However,
  these rules will always be preempted by the more specific rules
  showing allomorphic gender/number variation.} When unifying
class-marking and triggering types, unification of the phonological
descriptions will ensure that morphs introduced in the two dimensions
will receive the correct position class indices, thereby enforcing
left adjacency of the class-marker to the triggering marker.

\begin{figure}
    \avm{[mud & \{ [  ] \}\\
          mph &    <  > ]}
  \caption{Default zero realisation\label{fig:IFD}}
\end{figure}

Finally, since expression of agreement properties does not necessarily
have to be overt, I shall propose that agreement in Batsbi is
expressed by a default rule of zero realisation, as shown in
Figure~\ref{fig:IFD}. So any single morphosyntactic property that
does not have any more specific expression can be realised without
introducing any morphs. This will capture the vast number of cases
where indeed no overt marking of agreement is found: as stated above,
only a quarter of stems in the Batsbi lexicon license class agreement
markers and only a select few affixes. If we assume that class
agreement in Batsbi does not necessarily have an overt expression, we
can treat those cases where we do find agreement as allomorphic
variations of certain stems and affixes, as sketched in the analysis
in \figref{fig:Batsbi}. Thus, by taking the majority case of zero
exponence as our point of departure and treat dependent overt
exponence as inflectional allomorphy, we avoid making arbitrary or even
conflicting decisions about which overt exponents are realisations of
agreement and which ones are just allomorphs.

\subsection{Deconstructing class marking (stems)}

As we have seen in Section~\ref{sec:Data:Roots}, agreement marking of
stems is ultimately decided in the lexicon: some stems take a class
marker, some do not, and for some lexical entries we even find
alternation where one stem in a lexeme's stem space comes with a class
marker, but the other does not. To make sense of this lexically
conditioned alternation, I shall build on the notion of stem spaces as
proposed by \citet{Bonami06}.  In IbM, stem spaces are provided by the
lexeme and stem introduction rules, a subtype of realisation rules,
serve to select an appropriate stem from the stem space and insert it into
\textsc{mph} (see \citet{Bonami17b} for details on the interface
between lexemes and the inflectional system).

As a first step to integrate inflecting and non-inflecting stems, I
shall sketch a sample lexical entry for the alternating verb
\textit{ak'/ek'} and subsequently show how the general stem selection
rules of the language will thread this lexemic information into the
inflectional system, where it will take part in the allomorphic
alternation we described above.  

\begin{figure}
  \avm{
    [synsem & [loc [cat & [head & verb\\
          val & [subj & < NP[{\myit abs}]\textsubscript{\1} >\\
          comps & <  > ] ]\\
          cont & [rels & <[pred & fall\\
          arg1 & \1
            ]
            >
          ]
        ]
      ]
      \\
      morph & [ms & \{
        [\type*{lid}
          stem1 & < [ph & \normalfont ak'  ]>\\
          stem2 & < [  ], [ph & \normalfont ek' ] >] , ...  \}
      ]
    ]}  
  \caption{Sample lexical entry of a Batsi verb\label{fig:Lexeme}}
\end{figure}

At the lexical level, all it takes is to differentiate in the stem
space between inflecting and non-inflecting stems. A most
straightforward way of doing this is to replicate in the specification
of stems a distinction we have already drawn for affixal markers,
namely between monomorphic and bimorphic. Thus, an alternating stem
such as \textit{ak'/ek'} will have a singleton list as the value of
\textsc{stem1}, but a two-elementary list as the value of
\textsc{stem2}, as shown in Figure~\ref{fig:Lexeme}.  

Stem introduction rules are given in the rule type hierarchy in
Figure~\ref{fig:Stem}: just like the realisation rules for the aorist,
evidential, transitive etc. in Figure~\ref{fig:CrossSuff} above, the stem
introduction rules are part of the \DIMBOX{EXPONENCE} dimension, so
they are available for cross-classification with the class marking rule
types. The two stem selection rules given here identify their
\textsc{mph} value with that of a stem value in \textsc{mud},
\textsc{stem2} in the perfective, and \textsc{stem1} otherwise. Note
that neither stem selection rule limits the arity of the stem
values or of their \textsc{mph} list. Thus, they  both unify freely
with any of the types in the \DIMBOX{ALLOMORPHY} dimension, including all
of the class-marking rule types, as well as the
non-marking monomorphic type. Thus, cross-classification by online type
construction will derive both bimorphic class-marking  and monomorphic
non-marking stem selection rules.    
  
\begin{sidewaysfigure}
  \begin{forest}
    [\avm{[\type*{realisation-rule}
          mud & \tag{$m$}\\
          ms & \tag{$m$} $\cup$ set\\
          mph & list]}
        [\DIMBOX{ALLOMORPHY}
            [\avm{[ms & \{ [\type*{abs-agr}], ... \}\\
                  mph & < [ph &  C \\
                           pc & \tag{$i$} ], [ph & V ...\\
                                              pc & \tag{$i$} + $1$ ] > ]}
                   [...]
                   [\avm{[mph < [ph & \upshape d  ], ...> ]}]
            ]
            [\avm{[mph < [ ] >]}]]
         [\DIMBOX{EXPONENCE}
             [ \avm{[mud & \{ [\type{lid} stem2 & \tag{$s$}],
                    [\type{pfv}] \}\\
                  mph & \tag{$s$} < ...,[pc & $0$] >]}
              ]
              [\avm{[mud & \{[\type*{lid} stem1 & \tag{$s$}]\}\\
                  mph & \tag{$s$} < ...,[pc & $0$] > ] ...  }
              ]
          ]
      ]
  \end{forest}
        % \lf{
        % \begin{avm}
        %   [mud & \{[\type*{intr}]\}\\
        %     mph & < ..., [ph & < \normalfont al >\\
        %       pc & 2]>]
        % \end{avm}
        % }
        %   \lf{
        %   \begin{avm}
        %     [mud & \{[\type*{evid1}]\}\\
        %       mph & < ..., [ph & < \normalfont ano >\\
        %         pc & 5]>]
        %   \end{avm}
        % }
        %   \lf{
        %   \begin{avm}
        %     [mud & \{[\type*{aor}]\}\\
        %       mph & < [ph & < \normalfont en >\\
        %         pc & 3]>]
        %   \end{avm}
        % }
  \caption{Stem selection and class-marking rule types\label{fig:Stem}}

\end{sidewaysfigure}

However, once any of the stem selection rules is applied to a concrete
lexeme, bimorphic class marking rules will only be applicable to stem
values of arity two, whereas monomorphic non-marking stem
selection rules will exclusively apply to stem values of arity
one.

To conclude, the present analysis of exuberant exponence in Batsbi
exploits the fact that IbM recognises many-to-many relations between
morphosyntactic properties at the most basic level of representation,
namely realisation rules. Using  online type
construction in an inheritance  hierarchy of rule types, the two most
central generalisations regarding exuberant exponence in this language
can be given a unified and straightforward account, by separating
constraints on the shape of class markers from licensing their
presence: Thus, while triggering affix rules and stems ultimately
decide on whether they must (or may not) combine with a class marker,
the constraints on class-marking are stated separately, distributing
over rules of exponence.  

\subsection{Reflections on the dependent nature of exuberant exponence}

The kind of exuberant exponence expounded in Batsbi witnesses two
important properties: first, agreement marking is dependent on an
adjacent triggering marker, a stem or some affix, and the number of
class markers found then depends on the number of triggering stems or
suffixes present in the word, yielding a variable degree of exuberant
exponence. The formal analysis does justice to these two observations
by treating the dependent class marker as morphologically conditioned
allomorphy of the triggering stem or suffix. This raises the obvious
question whether exuberant exponence must in general be of the
dependent type.\footnote{Thanks to Jean-Pierre Koenig for drawing
  my attention to this.} Fully redundant multiple exponence involving
more than two markers is rare, so I shall extrapolate from what we
know about multiple exponence in general.

To answer this question, let us consider pre-prefixation in Nyanja
\citep{Stump01,Crysmann:14:OUP}: in
this language a subclass of adjectives takes two agreement markers,
one from the set of adjectival markers, the other from the set of
verbal agreement markers.\largerpage

{\multicolsep=.25\baselineskip
\begin{multicols}{2}
\begin{exe}
  \ex 
  \begin{xlist}
    \ex \label{ex:NyanjaConc} 
      \gll ci-lombo ci-kula.\\
      \textsc{cl7}-weed \textsc{conc7}-grow\\
      \glt `A weed grows.'
    \ex \label{ex:NyanjaQual} 
      \gll ci-manga ca-bwino\\
      \textsc{cl7}-maize \textsc{qual7}-good\\
      \glt `good maize'
  \end{xlist}
  \end{exe}
  \end{multicols}}
\begin{exe}
  \ex
    \gll ci-pewa ca-ci-kulu\\
    \textsc{cl7}-hat(7/8) \textsc{qual7}-\textsc{conc7}-large\\
    \glt `a large hat'
\end{exe}

Multiple exponence in Nyanja is solely determined by inflection class
membership, and the two agreement markers surface adjacent to each
other, without any additional triggering exponent. In IbM, this
situation has been analysed by means of composing simple verbal and
adjectival markers into a class-specific morphotactically complex
marker \citep{Crysmann:14:OUP}. However, what we find here is
composition of similar yet non-identical markers, each of which is
attested independently. 

The crucial difference between Nyanja and Batsbi is that the number of
exponents is fixed in the former for any given inflection class, but
it is variable and dependent on the presence of concrete stems and
suffixes in the latter. Whenever multiple exponence is
morphotactically dependent, the formal approach sketched here, which
composes each instance of multiple marking with a triggering exponent,
is to be preferred. It so happens that this approach is also much
more apt at handling variable degrees of exuberance, a property
that is actually expected, if exponence is dependent on a triggering
marker. Composition among the instances of multiple exponence, by
contrast, is the way to go, if multiple exponence is morphotactically
independent and fixed with respect to the degree  of exuberance.     

\section{Conclusion}

In this paper I have discussed exuberant exponence in Batsbi
\citep{Harris09}. I have shown that the design property of IbM to
recognise $m:n$ relations between form and function at the level of
realisation rules lends itself naturally to accounting for the dependent
nature of these markers. Thus, under the perspective offered here,
exuberant class marking in Batsbi is just a case of allomorphy on the
markers/stems they depend on, conditioned by number and gender
properties. The uniformity of shape of these markers has been captured
by a system of cross-classifying type hierarchies along the dimensions
of allomorphy and exponence, building on the formal notion of online
type construction \citep{Koenig94} standardly embraced by IbM. As a
result, I have offered a theory of Batsbi exuberant exponence that is
as holistic as necessary to capture dependence, and at the same time as
atomistic as possible, thereby facilitating reuse. In other words, the
current approach captures the constructional properties of the system
within a formal generative model. 

% \begin{itemize}
% \item Information-based account captures dependent nature of Batsbi
%   exuberant gender/number marking by means of syntagmatic
%   constraints on morphs
  
% \item Gender/number markers essentially treated as allomorphy on 
%   the triggering exponents 
  
% \item Re-use of inflectional rule schemata ensured by constructing
%   rules from underspecified rule descriptions in a multi-dimensional
%   type hierarchy
  
%   device identical to that used for parallel position classes and
%   conditioned reordering  (Crysmann \& Bonami 2012, 2013, under
%   review; Bonami \& Crysmann 2012, 2013)
  
  
% \item Inherent, positional and conflated exponence all reducible to
%   combination of partial rule descriptions

Finally, this paper provided some meta-theoretical result, showing
that there is only limited a priori superiority of
inferential-realisational approaches over lexical-realisational ones:
just as much as the conceptual foundations, it is the formal
expressivity of the specific framework that determines its adequacy in
light of exuberant exponence.




% \begin{frame}{Beyond Batsbi}
  
%   \begin{itemize}
%   \item Archi \citep{kibrik94:_archi}  displays exuberant
%     exponence with possessives \citep{Corbett:91} 
    
%     4 exponents of agreement with the possessum in total
    
%   \item In contrast to Batsbi, exponence is not always fully alliterative 

%     \begin{exe}
%       \ex {\gll \textbf{d}-as̄á-\textbf{r}-ej-\textbf{r}-u-t̄u-\textbf{r}\\
%         \textsc{ii}-of.myself-\textsc{ii-suffix-ii-suffix-suffix-ii}\\
%         \glt ‘my own’ [female]}
%       \ex {\gll \textbf{w}-as̄á-\textbf{w}-ej-\textbf{w}-u-t̄u-∅\\
%         \textsc{I}-of.myself-\textsc{i-suffix-i-suffix-suffix-i}\\
%         \glt ‘my own’ [male]}
%     \end{exe}

%     \item word-final /w/ regularly elides

%     \item 

%%% Did not find any initial /r/ so far. Dropping Archi for the moment.  
      
%   \end{itemize}


% 





% \begin{table}
% \caption{Frequencies of word classes}
% \label{tab:1:frequencies}
%  \begin{tabular}{lllll} 
%   \lsptoprule
%             & nouns & verbs & adjectives & adverbs\\ 
%   \midrule
%   absolute  &   12 &    34  &    23     & 13\\
%   relative  &   3.1 &   8.9 &    5.7    & 3.2\\
%   \lspbottomrule
%  \end{tabular}
% \end{table}

\section*{Abbreviations}

The glosses in this chapter follow the original description by
\citet{Harris09}, slight\-ly adapted to adhere more fully to the Leipzig
conventions. Here is a list of additional abbreviations being used:
\textsc{cm} (class marker), \textsc{pres} (present), \textsc{aor}
(aorist), \textsc{evid1} (evidential 1), \textsc{con} (contact
case). Furthermore, inherent noun class is indicated by means of the
exponents of the singular and plural class markers.\largerpage

\section*{Acknowledgements}
Preliminary versions of this work have been presented at the 16th
international morphology meeting (Budapest, May 2014), the 3rd
European workshop on Head-driven Phrase Structure Grammar (Paris,
November 2014), and the workshop on building blocks (Leipzig,
November 2014). I would like to thank the audiences at these venues
for their questions and comments, in particular, Doug Ar\-nold,
Matthew Baerman, Olivier Bonami, Bernard Fradin, Alain Kihm,
Jean-Pierre Koenig, Frank Richter, Barbara Stiebels, Greg Stump and
Geraldine Wal\-ther.

This work was partially supported by a public grant overseen by the
French National Research Agency (ANR) as part of the program
``Investissements d’Avenir'' (reference: ANR-10-LABX-0083). It
contributes to the IdEx Université de Paris (ANR-18-IDEX-0001).

{\sloppy\printbibliography[heading=subbibliography,notkeyword=this]}
\end{document}
