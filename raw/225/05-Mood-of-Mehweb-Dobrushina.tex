\documentclass[output=paper]{langsci/langscibook} 
\ChapterDOI{10.5281/zenodo.3402062}

% Chapter 5

\title{Moods of Mehweb}

\author{Nina Dobrushina\affiliation{National Research University Higher School of
    Economics, Linguistic Convergence Laboratory,
ndobrushina@hse.ru}}


\abstract{The paper is a description of moods in Mehweb, a lect of the Dargwa branch of East Caucasian (Nakh-Daghestanian) languages, Republic of Daghestan. The data were collected in the course of several field trips to the village of Mehweb. The forms of non-indicative moods and common constructions where these forms occur are described. Mehweb has inflectional forms for the imperative, prohibitive, optative, irrealis and apprehensive. Hortative and jussive are expressed periphrastically.

\emph{Keywords:} Nakh-Daghestanian, East Caucasian, modality, mood,
imperative, hortative, jussive, optative, irrealis, conditional,
apprehensive, volitional.}


\begin{document}

\maketitle

\exewidth{(200)}

\let\exfont\rm
\let\eachwordone\rm

% 1.
\section{Introduction}

This paper is a description of non-indicative moods in Mehweb. Mehweb
moods are briefly discussed in \citet{magometov1982,khajdakov1985} and in a
sketch of Mehweb morphology by Nina \citeauthor{sumbatova:mehweb-grammar} (manuscript). The data for
this paper were collected in the course of field trips to Mehweb in
2013, 2014 and 2015.

I describe morphological forms of non-indicative moods as well as
periphrastic constructions used for the expression of some categories
which are rendered by non-indicative moods in many languages of the
world.

There are five forms which can be considered as inflectional forms of
mood in Mehweb: second person imperative, prohibitive, optative,
irrealis, and apprehensive. I also briefly describe the converbs which
are used in the subordinate part of conditional clauses, because these
forms are functionally close to the non-indicative moods, and in many
languages, non-indicative forms are used in these clauses. The
hypothetical conditional converb is derived from the same irrealis stem
in \emph{-a} as optative, irrealis, and apprehensive, thus manifesting
similarity with non-indicative moods.

I also consider two periphrastic constructions: one is used for the
hortative (=first person plural imperative, or inclusive imperative),
and the second for the jussive (third person imperative).

The paper is structured in accordance with the semantics of
non-indicative forms and constructions. It starts with volitional
categories. In \sectref{second-person-imperative}, the formation of second person imperative is
considered, and typical constructions with second person imperative are
described. \sectref{prohibitive} describes the prohibitive – the negative
imperative which is expressed in Mehweb, as in most East Caucasian
languages, by a dedicated morphological marker. Several interjections
with imperative meaning are considered in \sectref{imperative-interjections}. \sectref{hortative} and \sectref{jussive}
describe the form and semantics of periphrastic constructions which are
used for hortative and jussive. In \sectref{optative}, the semantics of the optative
is discussed, as well as some typical constructions involving the
optative. After volitionals, the forms with the irrealis meaning are
considered in \sectref{irreal-forms}; as in most East Caucasian languages, they occur
almost exclusively in conditional clauses. Last, I consider the
apprehensive form, used to introduce a situation the speaker is afraid
of (\sectref{apprehensive}). In \sectref{discussion} (Discussion), I compare the system of
Mehweb non-indicative moods with that of five other Dargwa languages and
dialects.

% 2.
\section{Second person imperative}\label{second-person-imperative}

\is{imperative|(}
Second person imperative expresses commands and requests addressed to
the hearer. In this section, I analyze the formation of second person
imperatives in their relation to transitivity and controllability of the
verbs, the agreement of imperatives with the addressee, and the forms of
address in the imperative constructions.

% 2.1.
\subsection{Formation of imperatives}\label{formation-of-imperatives}

The second person imperative of imperfective verbs is always marked by
the suffix \emph{-e} (\ref{ex:5:1}, \ref{ex:5:2}), unlike the imperative of perfective
verbs. The second person imperative of perfective verbs is marked either
by \emph{-e} or \emph{-a} depending on the \isi{transitivity} of the verb.
Intransitive verbs take the suffix \emph{-e}, transitive verbs take the
suffix \emph{-a} (see \tabref{tab:5:1}):

\ea \label{ex:5:1} % (1)
\gll \emph{niʔ}  \emph{\textbf{urt'-e}!}\\
milk  \textbf{pour:\textsc{ipfv}-\textsc{imp}}\\
\glt `Pour the milk!'
\pagebreak

\ex \label{ex:5:2} % (2)
\gll \emph{ħu} \emph{\textbf{w-aqnal}} \emph{\textbf{duc'}} \emph{\textbf{ulq-e}!}\\
 you.sg(\textsc{nom})  \textbf{\textsc{m}-often}  \textbf{run}  \textsc{m}.\textsc{lv}:\textsc{ipfv}-\textsc{imp}\\
\glt `Run more often!'

\ex % (3)
\gll \emph{niʔ}  \emph{\textbf{art'-a}!}\\
 milk  \textbf{pour:\textsc{pfv}-\textsc{imp}.\textsc{tr}}\\
\glt `Pour the milk!'

\ex % (4)
\gll \emph{\textbf{qa-d-iʔ~-e}}  \emph{heše-r.}\\
\textbf{down-\textsc{f1}-sit:\textsc{pfv}-\textsc{imp}}  here-\textsc{f}(\textsc{ess})\\
\glt `Sit down here.'
\z


\begin{table}
% Table 1.
\caption{Formation of second person imperatives}\label{tab:5:1}
\begin{tabular}{@{}lcc@{}}
\toprule
& transitive & intransitive\tabularnewline  \midrule
Perfective & \emph{-a} & \emph{-e} \tabularnewline
Imperfective & \emph{-e} & \emph{-e} \tabularnewline
\bottomrule
\end{tabular}

\end{table}

As \emph{-e} as an imperative marker is an unmarked choice, it is
glossed simply as \textsc{imp}.

Labile\is{labile verbs} perfective verbs can form two imperatives, one that follows the
transitive pattern, the other that follows the intransitive one. Cf.\
\emph{abxes} `open, \textsc{pfv}', (\emph{b})\emph{aˤldes} `hide, \textsc{pfv}',
(\emph{b})\emph{erqʷes} `become worn, \textsc{pfv}':

\ea % (5)
 \emph{rasul,} \emph{qali} \emph{\textbf{abx-a}!}\\
Rasul house \textbf{open:\textsc{pfv}-\textsc{imp}.\textsc{tr}}\\
\glt `Rasul, open the house!'

\ex % (6)
\gll \emph{qali,} \emph{\textbf{abx-e}!}\\
house \textbf{open:\textsc{pfv}-\textsc{imp}}\\
\glt `House, open up!'

\ex % (7)
\gll \emph{ʡali,} \emph{\textbf{b-aˤld-a}} \emph{ʁarʁa!}\\
Ali \textbf{\textsc{n}-hide:\textsc{pfv}-\textsc{imp}.\textsc{tr}} stone\\
\glt `Ali, hide the stone!'

\ex % (8)
\gll \emph{ʡali,} \emph{\textbf{w-aˤld-e}} \emph{ʁarʁa-la} \emph{ʡa}‹\emph{w}›\emph{ad!}\\
Ali \textbf{\textsc{m}-hide:\textsc{pfv}-\textsc{imp}} stone-\textsc{gen} ‹M›behind\\
\glt `Ali, hide behind the stone!'

\ex % (9)
\gll \emph{ʡali,} \emph{\textbf{b-erqʷ-a}} \emph{ħawa!}\\
Ali, \textbf{\textsc{n}-tear:\textsc{pfv}-\textsc{imp}.\textsc{tr}} dress\\
\glt `Ali, tear the dress!'

\ex % (10)
\gll \emph{ħawa,} \emph{\textbf{b-erqʷ-e}!}\\
dress \textbf{\textsc{n}-tear:\textsc{pfv}-\textsc{imp}}\\
\glt `Dress, get torn!'
\z

Some verbs have irregular and/or suppletive imperative forms. For
example the verb \emph{es} `say' has the imperative \emph{bet'a}; other
cases are considered in \citet{daniel2019} [this volume].

Imperatives from verbs that denote events and situations over which
the speaker exerts no control are acknowledged as grammatical by some
speakers only. In most cases speakers are able to come up with a special
context. For example, one can say \emph{Bemže!} `Get hot!' as if one was addressing a stove.

Imperatives of some perfective verbs which denote uncontrollable events
are presented in \tabref{tab:5:2}.

\begin{table}
  % Table 2.
  \caption{Imperative of intransitive uncontrollable verbs}\label{tab:5:2}

  \begin{tabular}{@{}ll@{}}
\toprule
Verb & intransitive imperative\tabularnewline \midrule
\emph{-ac'es} (\textsc{pfv}) `melt' & \emph{b-ac'e} (addressing snow)\tabularnewline
\emph{-arχes} (\textsc{pfv}) `touch' (unintentionally) & \emph{w-arχe} \tabularnewline
\emph{-ebk'es} (\textsc{pfv}) `die' & \emph{w-ebk'e} \tabularnewline
\emph{-emžes} (\textsc{pfv}) `become hot' & \emph{b-emže} (addressing a stove) \tabularnewline
\emph{-erħes} (\textsc{pfv}) `become rotten' & \emph{b-erħe} \tabularnewline
\emph{-ertes} (\textsc{pfv}) `curdle' & \emph{d-erte} (addressing milk) \tabularnewline
\emph{-erʔʷes} (\textsc{pfv}) `become dry' & \emph{b-erʔʷe} \tabularnewline
\emph{-ikes} (\textsc{pfv}) `happen' & \emph{b-ike} \tabularnewline
\emph{-uʔes} (\textsc{pfv}) `become spoilt' & \emph{b-uʔe} \tabularnewline
\emph{-emχes} (\textsc{pfv}) `swell' & \emph{b-emχe} \tabularnewline
\emph{kalʔes} (\textsc{pfv}) `be left, remain' & \emph{kalʔe} \tabularnewline
\emph{-arʡaˤs} (\textsc{pfv}) `become cold, freeze' & \emph{d-aˤrʡe} \tabularnewline
\bottomrule
\end{tabular}
\end{table}

% \largerpage

Most two-place experiencer verbs\is{experiential verbs} have two imperatives, with suffix \emph{-a} and
with suffix \emph{-e}. There is no clear difference in meaning between these two forms.

\ea % (11)
\gll \emph{ħa-ze} \emph{arʁ-e!}\\
 you.sg.\textsc{obl}-\textsc{inter}(\textsc{lat})  understand:\textsc{pfv}-\textsc{imp}\\
\glt `[You] understand!'

\ex % (12)
\gll  \emph{ħa-ze}  \emph{arʁ-a!}\\
 you.sg.\textsc{obl}-\textsc{inter}(\textsc{lat})  understand:\textsc{pfv}-\textsc{imp}.\textsc{tr}\\
\glt `[You] understand!'
\z

Imperatives from experiencer verbs are shown in \tabref{tab:5:3}. Not all
speakers acknowledge both imperative forms of these verbs; the less
accepted forms are marked by a question mark.

\begin{table}[h]
  % Table 3.
  \caption{Imperative from experiencer verbs}\label{tab:5:3}
  
\begin{tabular}{@{}lll@{}}
\toprule
% {experiencer verb (the meaning)}
  & \multicolumn{1}{m{4.5em}<{\raggedright}}{transitive imperative} &
\multicolumn{1}{m{4.5em}<{\raggedright}@{}}{intransitive imperative}\tabularnewline\midrule
\emph{-ahas} (\textsc{pfv}) `know' & \emph{b-ah-a} & \emph{b-ah-e} \tabularnewline
\emph{-arges} (\textsc{pfv}) `find' & \emph{b-arg-a} & \emph{b-arg-e} \tabularnewline
\emph{\(-\)iges} (\textsc{ipfv}) `love, want' & \textsuperscript{??}\emph{dig-a} & \emph{dig-e} \tabularnewline
\emph{arʁes} (\textsc{pfv}) `understand, hear' & \emph{arʁ-a} & \emph{arʁ-e} \tabularnewline
\emph{gʷes} (\textsc{pfv}) `see' & \textsuperscript{?}\emph{gʷ-a} & \emph{*gʷ-e} \tabularnewline
\emph{qumartes} `forget' & \emph{qumart-a} & \textsuperscript{?}\emph{qumart-e}\tabularnewline
\emph{uruχ k'es} (\textsc{ipfv}) `be afraid' & \emph{*uruχ k'-a} & \emph{uruχ k'-e} \tabularnewline
\bottomrule
\end{tabular}

\end{table}

Notably, verbs that show semantic restrictions on the formation of
imperatives easily produce imperatives within the \isi{jussive} construction.
The jussive is built as a combination of an imperative of the main verb with
the imperative of the verb \emph{es} `say' (see \sectref{jussive}):

\ea \label{ex:5:13} % (13)
\gll  \emph{\textbf{gʷ-e}}  \emph{bet'-a!}\\
\textbf{see:\textsc{pfv}-\textsc{imp}}  say:\textsc{pfv}-\textsc{imp}.\textsc{tr}\\
\glt `Let him see!' (he should make attempts to see)
\z

Some intransitive verbs that allow just one form of second person
imperative have the jussive construction with two imperative forms, the
one in \emph{-e} and the one in \emph{-a}. Speakers' first choice is
usually the form in \emph{-e}. They do not see any semantic difference
between the jussive based on the imperative in \emph{-e} and the jussive based on the 
imperative in \emph{-a}. Cf. examples (\ref{ex:5:13}) and (\ref{ex:5:14}):

\ea \label{ex:5:14} % (14)
\gll  \emph{\textbf{gʷ-a}}  \emph{bet'-a!}\\
 \textbf{see:\textsc{pfv}-\textsc{imp}.\textsc{tr}}  say:\textsc{pfv}-\textsc{imp}.\textsc{tr}\\
\glt  `Let him see!' (he should make attempts to see)
\z

Examples of the \isi{jussive} constructions with intransitive and experiencer
verbs are shown in \tabref{tab:5:4}.

\begin{table}
  % Table 4.
  \caption{Examples of jussive construction with uncontrollable verbs}\label{tab:5:4}
\begin{tabular}{@{}lll@{}}
\toprule
 & \multicolumn{1}{m{9em}<{\raggedright}}{jussive construction with imperative in \emph{-e}}
  & \multicolumn{1}{m{9em}<{\raggedright}@{}}{jussive construction with imperative in \emph{-a}} \tabularnewline \midrule
\emph{gʷes} (\textsc{pfv}) `see' & \emph{gʷe bet'a} & \emph{gʷa bet'a}\tabularnewline
\emph{-ac'es} (\textsc{pfv}) `melt' & \emph{b-ac'e bet'a} & \emph{b-ac'a bet'a}\tabularnewline
\emph{-emχes} (\textsc{pfv}) `become swollen' & \emph{b-emχe bet'a} & \emph{b-emχa bet'a} \tabularnewline 
\emph{-ertes} (\textsc{pfv}) `curdle' & \emph{d-erte bet'a} & \emph{d-erta bet'a} \tabularnewline
\emph{-emžes} (\textsc{pfv}) `become hot' & \emph{b-emže bet'a} & \emph{b-emža bet'a} \tabularnewline
\bottomrule
\end{tabular}
\end{table}

% 2.2.
\subsection{Number and gender of the addressee}

All verbs in the imperative obligatorily add a dedicated imperative
plural suffix \emph{-na} to convey the plurality\is{imperative, plural}
of the addressee.

Intransitive\is{transitivity} verbs which have a prefixal agreement slot agree in gender
and number with the nominative argument. Since this nominative
argument and the addressee coincide in intransitive verbs, the plural
imperative suffix \emph{-na} agrees with the same argument as the prefix
(\ref{ex:5:17}).

\ea % (15)
\gll  \emph{w-ak'-e!}\\
 \textsc{m}-come:\textsc{pfv}-\textsc{imp}\\
\glt `Come to me (addressing a men)!'

\ex % (16)
\gll  \emph{d-ak'-e!}\\
 \textsc{f1}-come:\textsc{pfv}-\textsc{imp}\\
\glt `Come to me (addressing a girl)!'

\ex \label{ex:5:17} % (17)
\gll  \emph{b-ak'-e-na!}\\
 \textsc{hpl}-come:\textsc{pfv}-\textsc{imp}-\textsc{imp}.\textsc{pl}\\
\glt `Come to me (addressing several people)!'
\z

Transitive\is{transitivity} verbs with a prefixal agreement slot also agree with their
nominative argument. Here, however, the addressee is the agent in the
ergative case. The prefixal agreement and the plural imperative suffix
are triggered by different arguments (\ref{ex:5:19}).

\ea % (18)
\gll  \emph{\textbf{b-aˤbʡ-a}}  \emph{urš-be!}\\
 \textbf{\textsc{hpl}-kill:\textsc{pfv}-\textsc{imp}.\textsc{tr}}  boy-\textsc{pl}\\
\glt `Kill these boys (addressing one person)!'

\ex \label{ex:5:19} % (19)
\gll  \emph{\textbf{w-aˤbʡ-a-na}}  \emph{rasul!}\\
 \textbf{\textsc{m}-kill:\textsc{pfv}-\textsc{imp}.\textsc{tr}-\textsc{imp}.\textsc{pl}}  Rasul\\
\glt `Kill Rasul (addressing several people)!'
\z

The suffix \emph{-na} as a plurality of addressee marker is also used on
prohibitive forms (see \sectref{prohibitive}).

In some Dargwa dialects (e.g.\ in {Tanti}\il{Tanti Dargwa} – \citealt[146]{sumbatova-lander2014})
the imperative form is not used if the P of the transitive construction
is a first person argument. The optative is used instead. This is not true for
Mehweb – there is no restriction on the usage of the imperative with
the first person:

\ea % (20)
\gll \emph{nu} \emph{dub} \emph{aˤʡ-aq-a!}\\
I eat \textsc{lv}-\textsc{caus}-\textsc{imp}.\textsc{tr}\\
\glt `Feed me!'
\z


% 2.3.
\subsection{Subject and forms of address}\label{subject-and-forms-of-address}

The agent of the imperative is not usually expressed, but it can be
indicated by an overt second person pronoun if it is stressed:

\ea % (21)
\gll  \emph{ħu}  \emph{učitel}  \emph{\textbf{uʔ-e}!}\\
 you.sg(\textsc{nom})  teacher  \textbf{\textsc{m}.be:\textsc{pfv}-\textsc{imp}}\\
\glt `[You] become a teacher!'

\ex % (22)
\gll  \emph{ħu-ni}  \emph{deč'}  \emph{\textbf{b-aq'-a}!}\\
 you.sg-\textsc{erg}  song  \textbf{\textsc{n}-do:\textsc{pfv}-\textsc{imp}.\textsc{tr}}\\
\glt `[You] sing the song!'
\z

% An imperative utterance can include nominal address.
Imperative utterances may contain forms of address expressed by a noun phrase in the nominative. The form of \isi{address}
is in the nominative even when referring to the agent of transitive
verbs:

\ea % (23)
\gll  \emph{muħammad,}  \emph{deč'}  \emph{\textbf{b-aq'-a}.}\\
 Muhammad(\textsc{nom})  song  \textbf{\textsc{n}-do:\textsc{pfv}-\textsc{imp}.\textsc{tr}}\\
\glt `Muhammad, sing the song.'

\ex % (24)
\gll  \emph{muħammad,}  \emph{učitel}  \emph{\textbf{uʔ-e}!}\\
 Muhammad(\textsc{nom})  teacher  \textbf{\textsc{m}.be:\textsc{pfv}-\textsc{imp}}\\
\glt `Muhammad, become a teacher!'
\z

Second person pronouns and demonstratives (used as third person
pronouns) cannot be used as forms of address:

\ea % (25)
\gll  \emph{*ħu}  \emph{deč'}  \emph{\textbf{b-aq'-a}}\\
 you.sg(\textsc{nom})  song  \textbf{\textsc{n}-do:\textsc{pfv}-\textsc{imp}.\textsc{tr}}\\

\ex % (26)
\gll  \emph{*it}  \emph{deč'}  \emph{\textbf{b-aq'-a}}\\
 this(\textsc{nom})  song  \textbf{\textsc{n}-do:\textsc{pfv}-\textsc{imp}.\textsc{tr}}\\
\z
 
The second person imperative construction can however include a third
person {NP} which is not a form of address. It is marked by the ergative
with transitive verbs and by the nominative with intransitive verbs.
Although the construction formally includes a third person {NP}, it is
addressed to the hearer whose name is Muhammad:

\ea % (27)
\gll  \emph{muħammad-ini}  \emph{deč'}  \emph{\textbf{b-aq'-a}.}\\
 Muhammad-\textsc{erg}  song  \textbf{\textsc{n}-do:\textsc{pfv}-\textsc{imp}.\textsc{tr}}\\
\glt `[Muhammad] sing the song.'

\ex % (28)
\gll  \emph{it-ini}  \emph{deč'} \emph{\textbf{b-aq'-a}.}\\
 this-\textsc{erg}  song  \textbf{\textsc{n}-do:\textsc{pfv}-\textsc{imp}.\textsc{tr}}\\
\glt `[He] sing the song.'

\ex % (29)
\gll  \emph{it}  \emph{\textbf{w-ak'-e}.}\\
 that(\textsc{nom})  \textbf{\textsc{m}-come:\textsc{pfv}-\textsc{imp}}\\
\glt `[He] come.'
\z

Speakers often build this construction with the additive particle
\emph{-ra}:

\ea % (30)
\gll  \emph{muħammad-ini꞊ra}  \emph{deč'}  \emph{\textbf{b-aq'-a}.}\\
 Muhammad-\textsc{erg}꞊\textsc{add}  song  \textbf{\textsc{n}-do:\textsc{pfv}-\textsc{imp}.\textsc{tr}}\\
\glt `[Muhammad] sing the song.'

\ex % (31)
\gll  \emph{it꞊ra}  \emph{\textbf{w-ak'-e}!} \\
 that(\textsc{nom})꞊\textsc{add}  \textbf{\textsc{m}-come:\textsc{pfv}-\textsc{imp}} \\
\glt `[He] come!'
\z

The construction with a third person {NP} and the imperative is primarily
used when the speaker addresses several people. The following sentences
can be uttered by the teacher who is addressing the whole class and
chooses the pupils to perform certain actions:

\ea % (32)
\gll  \emph{pat'imat꞊ra}  \emph{\textbf{d-ak'-e,}}  \emph{asijat꞊ra}  \emph{\textbf{d-ak'-e}.}\\
 Patimat(\textsc{nom})꞊\textsc{add}  \textbf{\textsc{f1}-come:\textsc{pfv}-\textsc{imp}}  Asijat(\textsc{nom})꞊\textsc{add}  \textbf{\textsc{f1}-come:\textsc{pfv}-\textsc{imp}}\\
\glt `Patimat come, and Asijat come.'

\ex % (33)
\gll  \emph{pat'imat-li}  \emph{deč'}  \emph{\textbf{b-aq'-a,}}  \emph{asijat-li}  \emph{deč'}  \emph{\textbf{bel'č'-a}.}\\
 Patimat-\textsc{erg}  song  \textbf{\textsc{n}-do:\textsc{pfv}-\textsc{imp}.\textsc{tr}}  Asijat-\textsc{erg}  song \textbf{read:\textsc{pfv}-\textsc{imp}.\textsc{tr}}\\
\glt `Patimat sing the song, and Asijat read the rhyme.'
\z

The following example with the word \emph{ca} as third person
imperative subject comes from the corpus:

\ea % (34)
\gll  \emph{mallarasbadij-ni}  \emph{ib}  \emph{iš-di-li-ze~:}  \emph{\textbf{ca}}  \emph{udi-di} \textbf{\emph{w-iz-e-na},}  \emph{\textbf{ca}}  \emph{aqu-di} \emph{\textbf{w-iz-e-na,}}  \emph{urga-w} \emph{nu} \emph{w-iz-iša,} \emph{nu-ni} \emph{ħuša} \emph{k'ʷi-jal-la} \emph{χʷasar} \emph{b-aq'-iša} \emph{ca-ca} \emph{ʁuruši-ze.}\\
Molla~Nasreddin.\textsc{obl}-\textsc{erg}  say:\textsc{pfv}.\textsc{aor}  that-\textsc{pl}-\textsc{obl}-\textsc{inter}(\textsc{lat}) \textbf{one}  below-\textsc{trans} \textbf{\textsc{m}-stand:\textsc{ipfv}-\textsc{imp}-\textsc{imp}.\textsc{pl}}  \textbf{one}  up-\textsc{trans}  \textbf{\textsc{m}-stand:\textsc{ipfv}-\textsc{imp}-\textsc{imp}.\textsc{pl}}  between-\textsc{m} I(\textsc{nom})  \textsc{m}-stand:\textsc{ipfv}-\textsc{fut}.\textsc{ego}  I-\textsc{erg}  you.pl  two-\textsc{ord}-\textsc{add} rescue  \textsc{hpl}-do:\textsc{pfv}-\textsc{fut}.\textsc{ego}  one-one  rouble-\textsc{inter}(\textsc{lat})\\
\glt 
`Molla Nasreddin told them: one of you stand higher, the other stand
lower, I will stand between you two, I will rescue the two of you for
one rouble each.'
\z


A similar phenomenon – the possibility to use 2\textsuperscript{nd}
person imperative with 3\textsuperscript{rd} person subject with
reference to the addressee - is found in other East Caucasian languages
(cf.\ \citealt[323]{dobrushina2001}).

% 2.4.
\subsection{Imperative with particles}

\is{imperative particle|(}

The imperative can be used with particles \emph{-w} and/or \emph{-ca}.
Although the particle \emph{-w} resembles the masculine gender marker, it
does not depend on the gender of the addressee:

\ea % (35)
\gll  \emph{deč'}  \emph{\textbf{b-aq'-a-w}!}\\
 song  \textbf{\textsc{n}-do:\textsc{pfv}-\textsc{imp}.\textsc{tr}-\textsc{ptcl}}\\
\glt `Sing a song! (addressing women or men)'
\z

The particle \emph{-w} is identical to the question particle
\emph{-w}/\emph{-u}. The particle \emph{ca} is formally identical to the word
\emph{ca} `one' and probably originates from it.

\ea % (36)
\gll  \emph{ʜaˤramir-ti-la}  \emph{ʁuša-ne}  \emph{\textbf{elʔ-a-ca}.}\\
 Haramirt-\textsc{pl}-\textsc{gen}  house-\textsc{pl} \textbf{count:\textsc{pfv}-\textsc{imp}.\textsc{tr}-\textsc{ptcl}}\\
\glt `List the families of the Haramirt (clan).' (Text 19. Clans, 1.6)
\z

Neither of the particles can be used if the imperative utterance
expresses permission:

\ea % (37)
\gll \emph{abaj,} \emph{b-uh-es꞊u} \emph{nu-ni} \emph{g-es} \emph{rasuj-s} \emph{k'amp'it'\rlap?}\\
mother \textsc{n}-become:\textsc{pfv}-\textsc{inf}꞊\textsc{q} I-\textsc{erg} give:\textsc{pfv}-\textsc{inf} Rasul-\textsc{dat} sweet\\ 
\glt `– Mother, can I give a sweet to Rasul?'

\vspace{\jot}

\gll \emph{b-uh-es} \emph{b-eg-a} / \textsuperscript{??}\emph{b-eg-a-w}  / \textsuperscript{??}\emph{b-eg-a-ca.}\\
\textsc{n}-become:\textsc{pfv}-\textsc{inf} \textsc{n}-give:\textsc{pfv}-\textsc{imp}.\textsc{tr} / \textsc{n}-give:\textsc{pfv}-\textsc{imp}.\textsc{tr}-\textsc{ptcl} / \textsc{n}-give:\textsc{pfv}-\textsc{imp}.\textsc{tr}-\textsc{ptcl}\\
\glt `– You can, give it to him.'
\z

The particle \emph{-w} expresses a more categorical demand than that
expressed by the particle \emph{-ca}. Therefore, it is not used in situations when the speaker has a status lower than the addressee, or
when the speaker has no right to demand. In the following example, the
child asks her mother to give her the sweet; with the particle \emph{-w}
she is rather too direct, as if her mother must give it to her; with the
particle \emph{-ca} the utterance sounds as a mild request.

\ea % (38)
\gll \emph{Abaj} \emph{\textbf{ag-a}} / \emph{\textbf{ag-a-ca}}  / \textsuperscript{?}\emph{\textbf{ag-a-w}} \emph{nab} \emph{k'amp'it'.}\\
Mother \textbf{give:\textsc{pfv}-\textsc{imp}.\textsc{tr}} / \textbf{give:\textsc{pfv}-\textsc{imp}.\textsc{tr}-\textsc{ptcl}} / \textbf{give:\textsc{pfv}-\textsc{imp}.\textsc{tr}-\textsc{ptcl}} I.\textsc{dat} sweet\\
\glt `Mother, give me a sweet.'
\z

In example (\ref{ex:5:39}), the imperative with the particle \emph{-w} would have
been completely inappropriate, since the pupil addresses his request to
the teacher. The imperative with the particle \emph{-ca} is better,
although it is not the typical way to address the teacher.

\ea \label{ex:5:39} % (39)
\gll \textsuperscript{?}\emph{Maisarat} \emph{Magomedovna} \emph{\textbf{ag-a-ca}} \emph{di-ze} \emph{kung.}\\
 Maisarat Magomedovna \textbf{give:\textsc{pfv}-\textsc{imp}.\textsc{tr}-\textsc{ptcl}} I.\textsc{obl}-\textsc{inter}(\textsc{lat})  book\\
\glt `Maisarat Magomedovna, give me the book please.'
\z

The particles \emph{-w-} and \emph{-ca} can occur together:

\ea % (40)
\gll \emph{pat'imat} \emph{ħu} \emph{\textbf{d-ak'-e-w-ca}!}\\
 Patimat you.sg(\textsc{nom}) \textbf{\textsc{f1}-come:\textsc{pfv}-\textsc{imp}-\textsc{ptcl}-\textsc{ptcl}}\\
\glt `Patimat, [you] come!'
\z

According to the corpus, the particle \emph{-ca} is used very
frequently; the particle \emph{-w} was not found in the corpus.

\removelastskip
\is{imperative particle|)}

% 2.5.
\subsection{Coordinated constructions with imperatives}

\is{imperative in clause chaining|(}

If several imperatives are combined, the chain of verb forms can either
consist of imperatives or combine imperative(s) with converb(s):

\ea % (41)
\gll \emph{\textbf{b-uc-a}} \emph{maza} \emph{\textbf{aʔ-a}} \emph{b-uħna.}\\
 \textbf{\textsc{n}-catch:\textsc{pfv}-\textsc{imp}.\textsc{tr}} sheep \textbf{drive:\textsc{pfv}-\textsc{imp}.\textsc{tr}}  \textsc{n}-inside(\textsc{lat})\\
\glt `Catch the sheep, let it inside.'

\ex % (42)
\gll \emph{pat'imat} \emph{kaltuška꞊ra} \emph{d-urʔun}  \emph{\textbf{d-aq'-i-le}} \emph{ħarši}  \emph{\textbf{d-aq'-a}!}\\
 Patimat potato꞊\textsc{add} \textsc{npl}-clean \textbf{\textsc{npl}-do:\textsc{pfv}-\textsc{aor}-\textsc{cvb}} soup  \textbf{\textsc{npl}-do:\textsc{pfv}-\textsc{imp}.\textsc{tr}}\\
\glt `Patimat, peel the potato and make the soup!'

\ex % (43)
\gll \emph{k'amp'it'-une} \emph{\textbf{as-i-le}}  \emph{tukaj-ħe-la} \emph{ħu-ni꞊jal} \emph{\textbf{mu-d-uk-adi}.}\\
 sweet-\textsc{pl} \textbf{take:\textsc{pfv}-\textsc{aor}-\textsc{cvb}} shop-\textsc{in}-\textsc{el}  you.sg-\textsc{erg}꞊\textsc{emph} \textbf{\textsc{negvol}-\textsc{npl}-eat:\textsc{ipfv}-\textsc{proh}}\\
\glt `Buy some sweets, (but) don't eat them.'
\z

Further examples and some discussion of the contrast between the chains
with imperatives and the chains with converbs can be found in \citet{kustova2019}
(this volume).

\removelastskip
\is{imperative in clause chaining|)}
\is{imperative|)}

% 3.
\section{Prohibitive}\label{prohibitive}

\is{prohibitive|(}

The prohibitive is a negative imperative which is expressed by a
dedicated affix. It is formed with the prefix \emph{mV-} with an
unspecified vowel which assimilates to the next vowel (see discussion in
\citealt{moroz2019} [this volume] and \citealt{daniel2019} [this volume]), and the suffix
\emph{-adi}, sometimes truncated to \emph{-ad}. In \sectref{discussion}, I~give
some information on the origin of this marker. The gender agreement
marker \emph{b-} (N~or \textsc{hpl}) assimilates to the \textsc{negvol} marker
\emph{m}V- (see \citealt{moroz2019} [this volume]). Sometimes, prohibitive formation
involves reduplication, as in (\ref{ex:5:46}) – see the discussion in \citet{daniel2019} [this
volume].

\ea  % (44)
\gll \emph{deč'} \emph{\textbf{mi-m-iq'-ad\(i\)}!}\\
 song \textbf{\textsc{negvol}-\textsc{n}-do:\textsc{ipfv}-\textsc{proh}}\\
\glt `Don't sing!'
\z

The prohibitive can be derived only from imperfective stems. Therefore,
each verb has two imperatives but only one prohibitive. There is no
formal distinction between transitive\is{transitivity} and intransitive prohibitives.

\ea % (45)
\gll \emph{\textbf{mu-lug-adi~}} \emph{d-uk'-a-k'a-ra,}  \emph{maja} \emph{g-i-le} \emph{le-l-le} \emph{hub-li-s.}\\
 \textbf{\textsc{negvol}-give:\textsc{ipfv}-\textsc{proh}} \textsc{f1}-say:\textsc{ipfv}-\textsc{irr}-\textsc{cond}-\textsc{add}  Maja give:\textsc{pfv}-\textsc{aor}-\textsc{cvb} \textsc{aux}-\textsc{f}-\textsc{cvb} husband-\textsc{obl}-\textsc{dat} \\
\glt `Although she said: `Don't give', they still married Maja off'. (Text 14. Laces, 1.3)

\ex \label{ex:5:46} % (46)
\gll \emph{gurda} \emph{b-ik'-uwe} \emph{le-b}  \emph{sinka-li-ze} \emph{\textbf{b-is-mi-m-is-adi}} \emph{ħu.}\\
fox \textsc{n}-say:\textsc{ipfv}-\textsc{cvb.ipfv} \textsc{aux}-\textsc{n} bear-\textsc{obl}-\textsc{inter}(\textsc{lat}) \textbf{\textsc{n}-cry-\textsc{negvol}-\textsc{n}-cry-\textsc{proh}} you.sg(\textsc{nom})\\
\glt `The fox said [to the bear]: ``Don't cry''.' (Text M. A bear, a wolf
and a fox, 1.11)
\z

The prohibitive has the same marker of plurality\is{imperative, plural}
\emph{-na} as the imperative. The prohibitive suffix cannot be truncated before the plural
marker.

\ea % (47)
\gll \emph{deč'} \emph{\textbf{mi-m-iq'-adi-na}!}\\
 song \textbf{\textsc{negvol}-\textsc{n}-do:\textsc{ipfv}-\textsc{proh}-\textsc{imp}.\textsc{pl}}\\
\glt `Don't sing!' (addressing several speakers)

\ex % (48)
\gll \emph{*deč'} \emph{mi-m-iq'-ad-na!}\\
 \emph{song} \textbf{\textsc{negvol}}-\textsc{n}-do:\textsc{ipfv}-\textsc{proh}-\textsc{imp}.\textsc{pl}\\
\glt Intended: `Don't sing!' (addressing several speakers)
\z

The prohibitive can be used with forms of \isi{address} in the same way as the imperative (\sectref{subject-and-forms-of-address}):

\ea % (49)
\gll \emph{pat'imat,} \emph{deč'} \emph{\textbf{mi-m-iq'-adi}.}\\
 Patimat song \textbf{\textsc{negvol}-\textsc{m}-do:\textsc{ipfv}-\textsc{proh}}\\
\glt `Patimat, don't sing the song.'

\z

Constructions with third person subject are also available for the
prohibitive:

\ea % (50)
\gll \emph{pat'imat-li} \emph{deč'}  \emph{\textbf{mi-m-iq'-adi}.}\\
 Patimat-\textsc{erg} song \textbf{\textsc{negvol}-\textsc{m}-do:\textsc{ipfv}-\textsc{proh}}\\
\glt `[Patimat] don't sing the song.'
\z

The prohibitive can take the particle \emph{-ca}:

\ea % (51)
\gll \emph{\textbf{mi-m-iq'-adi-ca}} \emph{hel}  \emph{deč'!}\\
 \textbf{\textsc{negvol}-\textsc{m}-do:\textsc{ipfv}-\textsc{proh}-\textsc{ptcl}} this song\\
\glt `Don't sing this song!'
\z

\removelastskip
\is{prohibitive|)}

% 4.
\section{Imperative interjections}\label{imperative-interjections}

There are several words which function as imperatives although they are
not related to any verb. They are used to urge the addressee to perform
an action, and some of them can attach the imperative plural marker
\emph{-na}.

The interjection\is{imperative interjection} \emph{ma} `take, hold' is known in various languages of
Daghestan (e.g.\ Archi, Agul). In Mehweb, it may attach the plural
marker \emph{-na}:

\ea % (52)
\gll \emph{ma!}\\
\textsc{intj}\\
\glt `Take!'

\ex % (53)
\gll \emph{ma-na!}\\
\textsc{intj}-\textsc{imp}.\textsc{pl}\\
\glt `Take (addressed to several people)!'
\z

The interjection \emph{ma} can be combined with other imperative forms:

\ea % (54)
\gll \emph{ma} \emph{as-a!}\\
\textsc{intj} take:\textsc{pfv}-\textsc{imp}.\textsc{tr}\\
\glt `Take!'

\ex % (55)
\gll \emph{ma-na} \emph{as-a-na!}\\
\textsc{intj}-\textsc{imp}.\textsc{pl} take:\textsc{pfv}-\textsc{imp}.\textsc{tr}-\textsc{imp}.\textsc{pl}\\
\glt `Take (addressed to several people)!'
\z

The imperative interjection \emph{hara} is used to attract the visual
attention of the addressee. It also can attach the plural marker
\emph{-na}:

\ea % (56)
\gll \emph{hara!}\\
\textsc{intj}\\
\glt `Look!'

\ex % (57)
\gll \emph{hara-na!}\\
\textsc{intj}-\textsc{imp}.\textsc{pl}\\
\glt `Look! (addressing several people)'
\z

Two imperative interjections are used to urge the addresses to be quiet
and keep silence. For example, the teacher can use them in order to make
children silent: \emph{q'ah!} `Shhh!' and \emph{c'it'!} `Shhh!'. These
interjections cannot combine with the plural marker \emph{-na}.

% 5.
\section{Hortative (first person inclusive imperative)}\label{hortative}

\is{hortative|(}

The term \emph{hortative} is used here for the constructions which
express the inducement to perform an action together with the speaker,
cf.\ English \emph{Let's go}. There is no dedicated hortative morphology
in Mehweb, but the periphrastic construction is widely used to express
invitation to common action.

The hortative construction consists of the infinitive of the main verb and
the form \textsc{cl}\emph{-aš-e}, where \textsc{cl} is a gender marker.

\ea % (58)
\gll\emph{\textbf{w-aš-e}} \emph{χal} \emph{\textbf{w-aq'-as~}}  \emph{ħa-la} \emph{urtaq'.}\\
\textbf{\textsc{m}-go:\textsc{ipfv}-\textsc{imp}} seek \textbf{\textsc{m}-do:\textsc{pfv}-\textsc{inf}} you.sg.\textsc{obl}-\textsc{gen}  friend\\
\glt `Let's look together for your friend' (Aspectual test 1, 1.121)
\z

The form \textsc{cl}\emph{-aš-e} is an imperative of the verb \textsc{cl}\emph{-aš-es} `go/come (ipfv)'.
Alone, this form can be used as a second person
imperative and as a hortative. There are no other words in Mehweb which
combine these two meanings in one form; there are also no other
hortatives which are expressed lexically, in one word.

\ea % (59)
\gll \emph{pat'imat,} \emph{\textbf{d-aš-e}}  \emph{di-šu!}\\
 Patimat, \textbf{\textsc{f1}-go:\textsc{ipfv}-\textsc{imp}} I.\textsc{obl}-\textsc{ad}(\textsc{lat})\\
\glt `Patimat, come to me!'

\ex % (60)
\gll \emph{\textbf{d-aš-e}} \emph{tukaj-ħe!}\\
 \textbf{\textsc{f1}-go:\textsc{ipfv}-\textsc{imp}} shop-\textsc{in}(\textsc{lat})\\
\glt `Let's go to the shop!' (addressing a women)

\ex % (61)
\gll \emph{ʡali,} \emph{\textbf{w-aš-e}}  \emph{di-šu!}\\
 Ali, \textbf{\textsc{m}-go:\textsc{ipfv}-\textsc{imp}} I.\textsc{obl}-\textsc{ad}(\textsc{lat})\\
\glt `Ali, come to me!'

\ex % (62)
\gll \emph{\textbf{w-aš-e}} \emph{tukaj-ħe!}\\
 \textbf{\textsc{m}-go:\textsc{ipfv}-\textsc{imp}} shop-\textsc{in}(\textsc{lat})\\
\glt `Let's go to the shop!' (addressing a man)
\z

This pattern of hortative construction – with an infinitive and a
particle originating from an imperative or hortative form of a motion
verb – is attested in some other East Caucasian languages (Khwarshi
\citep{khalilova2009}, Lak and Rutul (personal fieldnotes)).

The imperative \textsc{cl}\emph{-aš-e} followed by the plural\is{imperative, plural} marker \emph{-na}
is used as a second person plural imperative or as an inducement to
several addressees to perform an action together. There is an irregular
change of \emph{-e} to \emph{-i} when the plural suffix is added:
\emph{w-aše – b-ašina}:

\ea % (63)
\gll \emph{\textbf{b-aš-ina}} \emph{tukaj-ħe!}\\
\textbf{\textsc{hpl}-go:\textsc{ipfv}-\textsc{imp}.\textsc{pl}} shop.\textsc{obl}-\textsc{in}(\textsc{lat})\\
\glt `Go to the shop!' / `Let's go to the shop!' (addressing several people)
\z

In the hortative construction, the form \textsc{cl}\emph{-aš-e} agrees with the
addressee, while the infinitive of the main verb agrees with the
nominative. In the constructions with intransitive\is{transitivity} imperatives, the
addressee and the nominative participant coincide (\ref{ex:5:64}, \ref{ex:5:65}). In the
constructions with transitive imperatives, the addressee coincides with
the ergative participant; therefore, the main verb and the auxiliary
form \textsc{cl}\emph{-aš-e} agree with different arguments (\ref{ex:5:66}–\ref{ex:5:69}).

\ea \label{ex:5:64} % (64)
\gll \emph{w-aš-e} \emph{uz-es!}\\
\textsc{m}-come:\textsc{ipfv}-\textsc{imp} \textsc{m}.work:\textsc{ipfv}-\textsc{inf}\\
\glt `Let us work! (addressing a boy)'

\ex \label{ex:5:65} % (65)
\gll \emph{d-aš-e} \emph{d-uz-es!}\\
\textsc{f1}-come:\textsc{ipfv}-\textsc{imp} \textsc{f1}-work:\textsc{ipfv}-\textsc{inf}\\
\glt `Let us work! (addressing a girl)'

\ex \label{ex:5:66} % (66)
\gll \emph{d-aš-e} \emph{deč'} \emph{b-aq'-as!}\\
 \textsc{f1}-come:\textsc{ipfv}-\textsc{imp} song \textsc{n}-do:\textsc{pfv}-\textsc{inf}\\
\glt `Let's sing a song! (addressing a girl)'

\ex % (67)
\gll \emph{w-aš-e} \emph{deč'} \emph{b-aq'-as!}\\
 \textsc{m}-come:\textsc{ipfv}-\textsc{imp} song \textsc{n}-do:\textsc{pfv}-\textsc{inf}\\
\glt `Let's sing a song! (addressing a boy)'

\ex % (68)
\gll \emph{d-aš-e} \emph{urši} \emph{w-it'-es!}\\
 \textsc{f1}-go:\textsc{ipfv}-\textsc{imp} boy \textsc{m}-draw:\textsc{pfv}-\textsc{inf}\\
\glt `Let's draw a boy! (addressing a girl)'

\ex \label{ex:5:69} % (69)
\gll \emph{w-aš-e} \emph{dursi} \emph{d-it'-es.}\\
 \textsc{m}-go:\textsc{ipfv}-\textsc{imp} girl \textsc{f1}-draw:\textsc{pfv}-\textsc{inf}\\
\glt `Let's draw a girl (addressing a boy)'
\z

The plural suffix\is{imperative, plural} \emph{-na} is added to the verb \textsc{cl}\emph{-aše} when the
hortative construction is addressed to several people and the action is
thus meant to be performed by more than two participants, including the
speaker:

\ea % (70)
\gll \emph{\textbf{b-aš-ina}} \emph{deč'}  \emph{\textbf{b-aq'-as}.}\\
 \textbf{\textsc{hpl}-come:\textsc{ipfv}-\textsc{imp}.\textsc{pl}} song  \textbf{\textsc{n}-do:\textsc{pfv}-\textsc{inf}}\\
\glt `Let's sing a song (addressing several people)!'
\z

The hortative construction can contain the first person plural pronoun
as a subject:

\ea % (71)
\gll \emph{\textbf{d-aš-e}} \emph{nuša} \emph{tukaj-ħe}  \emph{\textbf{b-uˤq'-as}.}\\
 \textbf{\textsc{f1}-go:\textsc{ipfv}-\textsc{imp}} we shop.\textsc{obl}-\textsc{in}(\textsc{lat})  \textbf{\textsc{hpl}-go:\textsc{pfv}-\textsc{inf}}\\
\glt `Let's go to the shop (addressing a girl)'

\ex % (72)
\gll \emph{\textbf{b-aš-e}} \emph{sinka} \emph{b-erkʷ-es~}  \emph{nuša-jni!}\\
 \textsc{n}-go:\textsc{ipfv}-\textsc{imp} bear \textsc{n}-eat:\textsc{pfv}-\textsc{inf} we-\textsc{erg}\\
\glt `Let's eat the bear!' (fox addressing wolf) (Text M. A bear, a wolf and
a fox)
\z

In the hortative construction, negation\is{polarity} is marked on the main verb, since
the illocution is not under the scope of negation:

\ea % (73)
\gll \emph{\textbf{d-aš-e}} \emph{deč'}  \emph{\textbf{ħa-b-aq'-as}.}\\
 \textbf{\textsc{f1}-come:\textsc{ipfv}-\textsc{imp}} song \textbf{\textsc{neg}-\textsc{n}-do:\textsc{pfv}-\textsc{inf}}\\
\glt `Let's not sing a song (addressing a girl)'

\ex % (74)
\gll \emph{\textbf{d-aš-e}} \emph{urši} \emph{\textbf{ħa-jt'-es}.}\\
 \textbf{\textsc{f1}-go:\textsc{ipfv}-\textsc{imp}} boy \textbf{\textsc{neg}-\textsc{m}.draw:\textsc{pfv}-\textsc{inf}}\\
\glt `Let's not draw a boy (addressing a girl)'
\z

Constructions with the negated verb of motion are not interpreted as
hortatives:

\ea % (75)
\gll \emph{\textbf{mi-d-ik'-adi}} \emph{deč'} \emph{\textbf{b-aq'-as}.}\\
 \textbf{\textsc{negvol}-\textsc{f1}-come:\textsc{ipfv}-\textsc{proh}} song \textbf{\textsc{n}-do:\textsc{pfv}-\textsc{inf}}\\
 \glt `Don't come to sing a song.'
\z

If a hortative occurs in the coordinative construction\is{imperative in clause chaining}, one of the
predicates can be expressed by a perfective converb (\ref{ex:5:76}), or both
predicates are expressed by infinitives (\ref{ex:5:77}); in the latter case, one
hortative auxiliary can belong to both infinitives:

\ea \label{ex:5:76} % (76)
\gll \emph{\textbf{b-aš-ina}} \emph{qali꞊ra } \emph{\textbf{b-aq'-i-le,}} \emph{q'ʷaˤl} \emph{\textbf{as-es}.}\\
\textbf{\textsc{hpl}-go:\textsc{ipfv}-\textsc{imp}.\textsc{pl}} house꞊\textsc{add} \textbf{\textsc{n}-do:\textsc{pfv}-\textsc{aor}-\textsc{cvb}} cow \textbf{take-\textsc{inf}}\\

\ex \label{ex:5:77} % (77)
\gll \emph{\textbf{b-aš-ina}} \emph{qali꞊ra } \emph{\textbf{b-aq'-as,}} \emph{q'ʷaˤl꞊ra} \emph{\textbf{as-es}.}\\
 \textbf{\textsc{hpl}-go:\textsc{ipfv}-\textsc{imp}.\textsc{pl}} house꞊\textsc{add} \textbf{\textsc{n}-do:\textsc{pfv}-\textsc{inf}} cow꞊\textsc{add} \textbf{take-\textsc{inf}}\\
\glt `Let's build the house and buy the cow.'
\z

The motion verb almost always takes the first place in hortative
constructions (\ref{ex:5:78}), but its final position is not completely
ungrammatical (\ref{ex:5:79}).

\ea \label{ex:5:78} % (78)
\gll \emph{\textbf{b-aš-ina}} \emph{qali} \emph{\textbf{b-aq'-as}.}\\
\textbf{\textsc{hpl}-go:\textsc{ipfv}-\textsc{imp}.\textsc{pl}} house \textbf{\textsc{n}-do:\textsc{pfv}-\textsc{cvb}}\\
\glt `Let's build the house.'

\ex \label{ex:5:79} % (79)
\gll \textsuperscript{?}\emph{qali} \emph{\textbf{b-aq'-as}} \emph{\textbf{b-aš-ina}.} \\
house \textbf{\textsc{n}-do:\textsc{pfv}-\textsc{cvb}} \textbf{\textsc{hpl}-go:\textsc{ipfv}-\textsc{imp}.\textsc{pl}}\\
\glt `Let's build the house.'
\z

The particle of mild request \emph{-ca} can be used with the hortative:

\ea % (80)
\gll \emph{\textbf{w-aš-e-ca~}} \emph{heč'} \emph{xunul} \emph{ʡaˤχ} \emph{r-aq'-as.}\\
\textbf{\textsc{m}-go:\textsc{ipfv}-\textsc{imp}-\textsc{ptcl}} that.higher woman good \textsc{f}-do:\textsc{pfv}-\textsc{inf}\\
\glt `Let's help that women.' (Text 06. Mahmud Omar who was friends with
devils, 1.11)
\z


\removelastskip
\is{hortative|)}

% 6.
\section{Jussive (third person imperative)}\label{jussive}

\is{jussive|(}

Jussive is a form or construction which is used to express an inducement
to a third person, most often transferred via the addressee. Some East
Caucasian languages have a dedicated form for this meaning; often, the
meaning of jussive is covered by the optative \citep{dobrushina2012}. In Mehweb,
the meanings of the jussive and optative are expressed separately, by a
periphrastic construction and by an inflectional form respectively. In
\sectref{jussive-construction}, the structure of the jussive construction is described.
\sectref{semantics-of-the-jussive} discusses the semantics of the jussive construction.
The optative is considered in \sectref{optative}.

% 6.1.
\subsection{Jussive construction}\label{jussive-construction}

The Mehweb jussive consists of the imperative of the verb `say'
\emph{bet'a} (irregular form; see \citealt{daniel2019} [this volume]) and the
imperative of the main verb. The jussive is conceived as a transfer of a
command or request to the non-locutor via the addressee (Tell him
``Work!'' → Let him work!):

\ea % (81)
\gll \emph{musa} \emph{\textbf{uz-e}} \emph{\textbf{bet'-a}.}\\
 Musa \textbf{\textsc{m}.work:\textsc{ipfv}-\textsc{imp}} \textbf{say:\textsc{pfv}-\textsc{imp}.\textsc{tr}}\\
\glt `Let Musa work.'

\ex \label{ex:5:82} % (82)
\gll \emph{sa‹w›i-jal} \emph{\textbf{uˤq'-e}} \emph{\textbf{bet'-a}} \emph{heʔʷan-i} \emph{ʁiz-be-ču.}\\
 ‹1›self-\textsc{emph} \textbf{\textsc{m}.go:\textsc{pfv}-\textsc{imp}} \textbf{say:\textsc{pfv}-\textsc{imp}.\textsc{tr}} similar-\textsc{atr} hair-\textsc{pl}-\textsc{comit}\\
\glt `With this kind of hair, let him drive on his own.' (Aspectual test 1,
1.141)
\z

Jussive semantics does not require the verb to designate a controllable
action (see \sectref{semantics-of-the-jussive}). Therefore, verbs which denote
uncontrollable actions can occur in the jussive construction in the form
which is morphologically imperative, while normally the second person
imperative of these verbs is not used (see also \sectref{formation-of-imperatives}):

\ea % (83)
\gll \emph{d-aq-a,} \emph{niʔ} \emph{\textbf{d-ert-e}} / \emph{\textbf{d-ert-a}}  \emph{bet'-a.}\\
 \textsc{npl}-let:\textsc{pfv}-\textsc{imp}.\textsc{tr} milk \textbf{\textsc{npl}-spoil:\textsc{pfv}-\textsc{imp}} / \textbf{\textsc{npl}-spoil:\textsc{pfv}-\textsc{imp}.\textsc{tr}} say:\textsc{pfv}-\textsc{imp}.\textsc{tr}\\
\glt `Leave it, let the milk spoil.'
\z

The imperative of the verb `say' does not have an agreement slot. It can
only agree with the addressee in number, as all imperatives:

\ea % (84)
\gll \emph{urš-be-jni} \emph{deč'} \emph{\textbf{b-aq'-a}} \emph{\textbf{bet'-a}.}\\
 boy-\textsc{pl}-\textsc{erg} song \textbf{\textsc{n}-do:\textsc{pfv}-\textsc{imp}.\textsc{tr}} \textbf{say:\textsc{pfv}-\textsc{imp}.\textsc{tr}}\\
\glt `Let the boys sing a song (addressing one person).'

\ex % (85)
\gll \emph{urš-be-jni} \emph{deč'} \emph{\textbf{b-aq'-a}} \emph{\textbf{bet'-a-na}.}\\
 boy-\textsc{pl}-\textsc{erg} song \textbf{\textsc{n}-do:\textsc{pfv}-\textsc{imp}} \textbf{say:\textsc{pfv}-\textsc{imp}.\textsc{tr}-\textsc{imp}.\textsc{pl}}\\
\glt `Let the boys sing a song (addressing several people).'
\z

The jussive construction shows some evidence of grammaticalization. The
agent of the jussive construction usually bears A or S marking (ergative
with transitive verbs and nominative with intransitive verbs):

\ea \label{ex:5:86} % (86)
\gll \emph{muħammad-ini} \emph{deč'} \emph{\textbf{b-aq'-a}} \emph{\textbf{bet'-a}.}\\
 Muhammad-\textsc{erg} song \textbf{\textsc{n}-do:\textsc{pfv}-\textsc{imp}.\textsc{tr}} \textbf{say:\textsc{pfv}-\textsc{imp}.\textsc{tr}}\\
\glt `Let Muhammad sing a song.'

\ex \label{ex:5:87} % (87)
\gll \emph{musa} \emph{uz-e } \emph{\textbf{bet'a}.}\\
 Musa \textsc{m}.work:\textsc{ipfv}-\textsc{imp} \textbf{say:\textsc{pfv}-\textsc{imp}.\textsc{tr}}\\
\glt `Let Musa work.'
\z

Although, as was shown above (\sectref{subject-and-forms-of-address}), second person imperative in
Mehweb can be used with 3\textsuperscript{rd} person subject and {A/S}
marking, such constructions are clearly peripheral. They do not occur in
the texts; they are used in a special pragmatic type of context 
(addressing several people in distributional meaning); and they cannot apply
to non-animate subject. Examples (\ref{ex:5:82}–\ref{ex:5:86}) hence cannot be interpreted
as cases of reported speech.

The addressee of the verb `say' is normally marked by the inter-lative. The
availability of S or A marking in the jussive construction shows that
the jussive has developed into a periphrastic form distinct from the
complement construction of the verb `say'. Cf.\ example (\ref{ex:5:87}) with a
complement clause-like structure with addressee marking in (\ref{ex:5:88}):

\ea \label{ex:5:88} % (88)
\gll \emph{musa-ze } \emph{uz-e }  \emph{\textbf{bet'a}.}\\
 Musa-\textsc{inter}(\textsc{lat}) \textsc{m}.work:\textsc{ipfv}-\textsc{imp} \textbf{say:\textsc{pfv}-\textsc{imp}.\textsc{tr}}\\
\glt `Tell Musa to work.'
\z

In jussive constructions, the verb `say-\textsc{imp}' follows the imperative of
the main verb. The following sentence is ungrammatical:

\ea % (89)
\gll \emph{*musa} \emph{\textbf{bet'-a}} \emph{\textbf{uz-e}.}\\
 Musa \textbf{say:\textsc{pfv}-\textsc{imp}.\textsc{tr}} \textbf{\textsc{m}.work:\textsc{ipfv}-\textsc{imp}}\\
 \z
 
As with the hortative, negation\is{polarity} is marked on the lexical verb of the jussive
construction:

\ea % (90)
\gll \emph{muħammad-ini} \emph{deč'} \emph{\textbf{mi-m-iq'-adi}} \emph{\textbf{bet'-a}.}\\
 Muħammad-\textsc{erg} song \textbf{\textsc{negvol}-\textsc{n}-do:\textsc{ipfv}-\textsc{proh}} \textbf{say:\textsc{pfv}-\textsc{imp}.\textsc{tr}}\\
\glt `Let Muhammad not sing a song.'
\z

% 6.2.
\subsection{Semantics of the jussive}\label{semantics-of-the-jussive}

The jussive is used in exhortations to actions by third person agents:

\ea % (91)
\gll \emph{išbari} \emph{muħammad-ini} \emph{t'ult'} \emph{\textbf{b-aq'-a}} \emph{\textbf{bet'-a}.}\\
today Muhammad-\textsc{erg} bread \textbf{\textsc{n}-do:\textsc{pfv}-\textsc{imp}.\textsc{tr}} \textbf{say:\textsc{pfv}-\textsc{imp}.\textsc{tr}}\\
\glt `Let Muhammad bake bread today.'
\z

The jussive can also express permission:

\ea % (92)
\gll \emph{b-uh-es꞊u} \emph{muħammad-ini} \emph{k'amp'it'} \emph{as-es?}\\
 \textsc{n}-become:\textsc{pfv}-\textsc{inf}꞊\textsc{q} Muhammad-\textsc{erg} sweet take:\textsc{pfv}-\textsc{inf}\\
\glt `– May Muhammad take a sweet?'

\gll \emph{b-uh-es,} \emph{\textbf{as-a}} \emph{\textbf{bet'-a}.}\\
\textsc{n}-become:\textsc{pfv}-\textsc{inf} \textbf{take:\textsc{pfv}-\textsc{imp}.\textsc{tr}} \textbf{say:\textsc{pfv}-\textsc{imp}.\textsc{tr}}\\
\glt `– (He) may, let him take one.'
\z

Jussives can have inanimate subjects. The jussive construction with an
inanimate subject expresses the speaker's indifference towards the
situation (indifference is semantically close to permission). The
implication is that the addressee should not interfere with the
realization of the situation; for instance, s/he should not take the
boiling soup from the stove:

\ea % (93)
\gll \emph{\textbf{rurž-e}} \emph{\textbf{bet'-a}} \emph{ħarši.}\\
 \textbf{boil:\textsc{ipfv}-\textsc{imp}} \textbf{say:\textsc{pfv}-\textsc{imp}.\textsc{tr}} soup\\
\glt `Let the soup boil.'

\ex % (94)
\gll \emph{\textbf{d-uh-e}} \emph{\textbf{bet'-a}} \emph{dig-uj-s.}\\
 \textbf{\textsc{f1}-become:\textsc{pfv}-\textsc{imp}} \textbf{say:\textsc{pfv}-\textsc{imp}.\textsc{tr}} love-\textsc{ptcp}.\textsc{obl}-\textsc{dat}\\
\glt `Let her get married with anyone (lit.\ become to whoever she wants).'
\z

Constructions with inanimate subjects show again that the jussive
construction is highly grammaticalized, because the imperative
\emph{bet'a} has lost its original meaning `say!'.

The jussive is available only in the third person. First and second
person pronouns cannot occur in jussive constructions:

\ea % (95)
\gll \emph{it-ini} \emph{\textbf{as-a}} \emph{\textbf{bet'-a}} \emph{k'amp'it'.}\\
 that-\textsc{erg} \textbf{take:\textsc{pfv}-\textsc{imp}.\textsc{tr}} \textbf{say:\textsc{pfv}-\textsc{imp}.\textsc{tr}} sweet\\
\glt `Let him take your sweet.'

\ex % (96)
\gll \emph{*nu-ni} \emph{\textbf{as-a}} \emph{\textbf{bet'-a}} \emph{k'amp'it'.}\\
 I-\textsc{erg} \textbf{take:\textsc{pfv}-\textsc{imp}.\textsc{tr}} \textbf{say:\textsc{pfv}-\textsc{imp}.\textsc{tr}} sweet\\
\glt Intended: `Let me take a sweet.'

\ex % (97)
\gll \emph{*ħu-ni} \emph{\textbf{as-a}} \emph{\textbf{bet'-a}} \emph{k'amp'it'.}\\
 you.sg-\textsc{erg} \textbf{take:\textsc{pfv}-\textsc{imp}.\textsc{tr}} \textbf{say:\textsc{pfv}-\textsc{imp}.\textsc{tr}} sweet\\
\glt Intended: `Let you take a sweet.'
\z

The semantics of indifference is the source for the constructions where
the jussive has a concessive\is{concession} meaning:

\ea % (98)
\gll \emph{\textbf{uz-e}} \emph{\textbf{bet'-a,}} \emph{saʁʷa-l꞊la} \emph{miski-je} \emph{uʔ-es-i} \emph{it.}\\
 \textbf{\textsc{m}.work:\textsc{ipfv}-\textsc{imp}} \textbf{say:\textsc{pfv}-\textsc{imp}.\textsc{tr}} how-\textsc{atr}꞊\textsc{add} poor-\textsc{advz} 1.be:\textsc{ipfv}-\textsc{inf}-\textsc{atr} that\\
\glt `Let him work, he will still be poor (=Even if he works, he will still be poor)'

\ex % (99)
\gll \emph{\textbf{d-uʔ-e}} \emph{\textbf{bet'-a}} \emph{хʷaldili}  \emph{amma} \emph{quli-b} \emph{ʜaˤnči} \emph{ħa-b-iq'-an.}\\
\textbf{\textsc{f1}-be:\textsc{ipfv}-\textsc{imp}} \textbf{say:\textsc{pfv}-\textsc{imp}.\textsc{tr}} beautiful but home.\textsc{in}-\textsc{n}(\textsc{ess}) work \textsc{neg}-\textsc{n}-do:\textsc{ipfv}-\textsc{hab}\\
\glt `Let her be beautiful, but she does not do her work at home (Though
she is beautiful, she does not work at home).'
\z

Unlike the \isi{optative}, the jussive is not used to express wishes\is{expression of wish}.
Accordingly, example (\ref{ex:5:100}) is acknowledged to be grammatical, but
semantically inappropriate; one of the speakers suggested that this
sentence can be uttered by an atheist who thinks that God can be forced
to perform an action. The correct choice would be to use the optative (\ref{ex:5:101}).

\ea \label{ex:5:100} % (100)
\gll \textsuperscript{?}\emph{aradeš} \emph{\textbf{ag-a}}  \emph{\textbf{bet'-a}.}\\
 health \textbf{give:\textsc{pfv}-\textsc{imp}.\textsc{tr}} \textbf{tell:\textsc{ipfv}-\textsc{imp}.\textsc{tr}}\\
\glt `\textsuperscript{?}Let [Allah] make [you] healthy.'

\ex \label{ex:5:101} % (101)
\gll \emph{aradeš} \emph{\textbf{g-a-b}!}\\
health \textbf{give:\textsc{pfv}-\textsc{irr}-\textsc{opt}}\\
\glt `May [Allah] make [you] healthy!'
\z

When the jussive is used do denote uncontrollable situations, it is
interpreted as expression of indifference or allowance but not as wish.
The following utterance can be pronounced when the speaker does not care
about the rain, e.g.\ because he has already done his work in the field:

\ea % (102)
\gll \emph{\textbf{d-aq'-a}} \emph{\textbf{bet'-a}}  \emph{zab.}\\
 \textbf{\textsc{npl}-do:\textsc{pfv}-\textsc{imp}.\textsc{tr}} \textbf{say:\textsc{pfv}-\textsc{imp}.\textsc{tr}} rain\\
\glt `Let it rain (I don't care).'
\z

If the speaker wants the rain to fall, she would rather use the form of
optative:

\ea % (103)
\gll \emph{\textbf{d-aq'-a-b}} \emph{zab!}\\
\textbf{\textsc{npl}-do:\textsc{pfv}-\textsc{irr}-\textsc{opt}} rain\\
\glt `May it rain!'
\z


\removelastskip
\is{jussive|)}

% 7.
\section{Optative}\label{optative}

\is{optative|(}
\is{expression of wish|(}

The optative is used to convey good and bad wishes. In Mehweb, as in
many other East Caucasian languages, the optative is expressed by a
dedicated inflectional form (for a discussion of optatives in languages
of the Caucasus see \citealt{dobrushina2011}). The formation of the optative is
described in \sectref{morphology-of-the-optative}, its semantics in \sectref{optative-constructions}, and typical
constructions involving the optative form~– in \sectref{semantics-of-the-optative}.

% 7.1.
\subsection{Morphology of the optative}\label{morphology-of-the-optative}

The optative is marked by the suffix \emph{-b} added to the irrealis
stem in \emph{-a-}:

\ea % (104)
\gll \emph{aradeš} \emph{\textbf{g-a-b}!}\\
 health \textbf{give:\textsc{pfv}-\textsc{irr}-\textsc{opt}}\\
\glt `May [Allah] make [you] healthy!'
\z

The optative can be derived from both the perfective and imperfective
stems: \emph{g-a-b} (give:\textsc{pfv}-\textsc{irr}-\textsc{opt}) – \emph{lug-a-b}
(give:\textsc{ipfv}-\textsc{irr}-\textsc{opt}); \emph{d-ic-a-b} (\textsc{npl}-sell:\textsc{pfv}-\textsc{irr}-\textsc{opt}) –
\emph{d-ilc-a-b} (\textsc{npl}-sell:\textsc{ipfv}-\textsc{irr}-\textsc{opt}).

The negative optative is derived from the imperfective stem with the
prefix \emph{mV-} (the same negative volitional marker which is used in the
prohibitive). The negative optative may also be formed with the regular
negative prefix \emph{ħa-}. The negative optative with the prefix
\emph{mV-} usually comes as a first choice of the speaker when s/he
translates wishes with negation, but the forms with the prefix
\emph{ħa-} are also often considered grammatical. Forms in \emph{ħa-}
are more easily accepted from perfective verbs, thus filling the gap of
the perfective negative optative. Sometimes, however, an imperfective
negative optative with the prefix \emph{ħa-} is also accepted by the
speakers (see \tabref{tab:5:5}). Negative optative is not a frequent form, it
does not occur in the corpus. I was unable to compare the actual
frequency of these two negative forms.

\begin{table}[h]
 % Table 5.
 \caption{Forms of the positive and negative optative}\label{tab:5:5}
\begin{tabular}{@{}llllm{.18\textwidth}<{\raggedright}@{}}
\toprule
&  \multicolumn{2}{c}{{positive}} & \multicolumn{2}{c@{}}{{negative}}\tabularnewline \cmidrule(lr){2-3} \cmidrule(l){4-5}
& {perfective} & {imperfective} & {perfective} & {imperfective}\\ \midrule
  `give' & \emph{g-a-b} & \emph{lug-a-b } & \emph{ħa-g-a-b} & \emph{mu-lug-a-b }

                                                              \textsuperscript{??}\emph{ħa-lu-ga-b}\\
  `sell' & \emph{d-ic-a-b } & \emph{d-ilc-a-b} & \emph{ħa-dic-a-b} & \emph{mi-d-ilc-a-b }

                                                                     *\emph{ħa-d-ilc-a-b}\\
  `find' & \emph{b-arg-a-b} & \emph{b-urg-a-b} & \emph{ħa-b-arg-a-b} & \emph{mu-m-urg-a-b}

                                                                       *\emph{ħa-b-urg-a-b}\\
  `eat' & \emph{b-erkʷ-a-b} & \emph{b-uk-a-b} & \emph{ħa-b-erkʷ-a-b} & \emph{mu-m-uk-a-b}

                                                                       \emph{ħa-b-uk-a-b}\\
  `drink' & \emph{b-erž-a-b} & \emph{b-už-a-b} & \emph{ħa-b-erž-a-b} & \emph{mu-m-už-a-b}

                                                                       \emph{ħa-b-už-a-b}\\
  `happen' & \emph{b-ik-a-b} & \emph{b-irk-a-b} & \emph{ħa-b-ik-a-b} & \emph{mi-m-irk-a-b}

\emph{ħa-b-irk-a-b}\\
\bottomrule
\end{tabular}
\end{table}

Some optatives have a reduced form without any suffixes:
\emph{w-ebk'-a-b} `may [he] die!' – \emph{w-ebk'} `may [he] die!'

\ea % (105)
\gll \emph{kapul-le} \emph{\textbf{w-ebk'-a-b}!}\\
 pagan-\textsc{advz} \textbf{\textsc{m}-die:\textsc{pfv}-\textsc{irr}-\textsc{opt}}\\
\glt `May he die impious!'

\ex % (106)
\gll \emph{kapul-le} \emph{\textbf{w-ebk'}!}\\
 pagan-\textsc{advz} \textbf{\textsc{m}-die:\textsc{pfv}(\textsc{opt})}\\
\glt `May he die impious!'

\ex % (107)
\gll \emph{ħa-la} \emph{abaj} \emph{\textbf{r-ebk'}!}\\
 you.sg.\textsc{obl}-\textsc{gen} mother \textbf{\textsc{f}-die:\textsc{pfv}(\textsc{opt})}\\
\glt `May your mother die!' (\ldots{}can be uttered by a mother of a child,
and addressed to the child if something bad is going to happen to
her/him – i.e.\ may I die in your stead!)
\z

Apart from the verb `die', the reduced form was attested for the
verbs \textsc{cl}-\emph{erʔʷes} `become dry',
\emph{če}-\textsc{cl}-\emph{uqes} `grow', and \textsc{cl}-\emph{alqaqas} `grow (causative)'. However, not all speakers accept all these examples
(unlike \emph{w-ebk'} which is frequent).

\ea \label{ex:5:108} % (108)
\gll \emph{maˤq'ʷ} \emph{\textbf{b-erʔʷ-a-b}.}\\
 root \textbf{\textsc{n}-become.dry:\textsc{pfv}-\textsc{irr}-\textsc{opt}}\\
\glt `May the roots dry out.' (a bad wish, suggesting that the 
clan of the person against whom the bad wish is directed should disappear)

\ex % (109)
\gll \emph{maˤq'ʷ} \emph{\textbf{b-erʔʷ}.}\\
 root \textbf{\textsc{n}-become.dry:\textsc{pfv}(\textsc{opt})}\\
\glt `May the roots dry out.' (same as (\ref{ex:5:108}))

\ex % (110)
\gll \emph{maˤq'ʷ} \emph{ha-b-le}  \emph{\textbf{če-b-uq-a-b}.}\\
 root front-\textsc{n}-\textsc{advz} \textbf{\textsc{pv}-\textsc{n}-grow:\textsc{pfv}-\textsc{irr}-\textsc{opt}}\\
\glt `May it all grow roots up.'

\ex \label{ex:5:111} % (111)
\gll \emph{maˤq'ʷ} \emph{ha-b-le}  \emph{\textbf{če-b-uq}.}\\
 root front-\textsc{n}-\textsc{advz} \textbf{\textsc{pv}-\textsc{n}-grow:\textsc{pfv}(\textsc{opt})}\\
\glt `May it all grow roots up.'

\ex \label{ex:5:112} % (112)
\gll \emph{qu} \emph{\textbf{b-alq-aq-ab}!}\\
 field \textbf{\textsc{n}-grow:\textsc{ipfv}-\textsc{caus}-\textsc{opt}}\\
\glt `May the field grow!'

\ex \label{ex:5:113} % (113)
\gll \emph{qu} \emph{\textbf{b-alq-aq}!}\\
 field \textbf{\textsc{n}-grow:\textsc{ipfv}-\textsc{caus}(\textsc{opt})}\\
\glt `May the field grow!'
\z


Truncated\is{truncation} forms of the optative are also attested in Akusha
\citep[34]{vandenberg2001}, Ashty (\citealt{belyaev:ocherk}, manuscript), Shiri
(\citealt{belyaev:shiri}, manuscript), Tanti
\citep{sumbatova-lander2014}, and Sanzhi (\citeauthor{forker:sanzhi},
in preparation) lects of Dargwa.

Some optative forms have a causative suffix which is not motivated
semantically. Cf.\ examples (\ref{ex:5:111}), (\ref{ex:5:112}), (\ref{ex:5:114}), (\ref{ex:5:115}), (\ref{ex:5:116}) and (\ref{ex:5:117}).
When the speakers discuss the difference between the optative with and
without the causative suffix, they usually say that the sentences with
causative suffix \emph{-aq-} imply an appeal to God:

\ea \label{ex:5:114} % (114)
\gll \emph{qu} \emph{\textbf{b-alq-a-b}!}\\
 field \textbf{\textsc{n}-grow:\textsc{ipfv}-\textsc{irr}-\textsc{opt}}\\
\glt `May the field grow!'

\ex \label{ex:5:115} % (115)
\gll \emph{qu} \emph{\textbf{b-alq-aq-a-b}!}\\
 field \textbf{\textsc{n}-grow:\textsc{ipfv}-\textsc{caus}-\textsc{irr}-\textsc{opt}}\\
\glt `May the field grow [with the help of Allah]!'

\ex \label{ex:5:116} % (116)
\gll \emph{hum-be} \emph{ʡaˤχ}  \emph{\textbf{d-uh-a-b}!}\\
 road-\textsc{pl} good \textbf{\textsc{npl}-become:\textsc{pfv}-\textsc{irr}-\textsc{opt}}\\
\glt `May you have a good trip!'

\ex \label{ex:5:117} % (117)
\gll \emph{hum-be} \emph{ʡaˤχ}  \emph{\textbf{d-uh-aq-a-b}!}\\
 way-\textsc{pl} good \textbf{\textsc{npl}-become:\textsc{pfv}-\textsc{caus}-\textsc{irr}-\textsc{opt}}\\
\glt `May Allah give you a good trip!'
\z

This semantic difference between the ordinary and the causative optative
is due to the fact that the causative derivation adds a new participant
to the situation. The sentences with the causative suffix may include
the ergative of Allah (\ref{ex:5:118}, \ref{ex:5:119}). If the participant is not overtly
expressed in the sentence, this new participant in the causativized
optative construction is by default understood as Allah. In another
Daghestanian language, Archi (Lezgic), the ergative of Allah can be
included even in intransitive optative constructions meaning `with the
help of Allah', where the ergative may be interpreted as the ergative of
the cause, one of the known functions of the ergative case \citep{dobrushina2011}.
In Mehweb, most speakers reject intransitive\is{transitivity} optative sentences
with Allah in the ergative (\ref{ex:5:120}, \ref{ex:5:121}).

\ea \label{ex:5:118} % (118)
\gll \emph{allah-li-ni} \emph{hum-be} \emph{ʡaˤχ} \emph{\textbf{d-uh-aq-ab}!}\\
 Allah-\textsc{obl}-\textsc{erg} way-\textsc{pl} good \textbf{\textsc{npl}-become:\textsc{pfv}-\textsc{caus}-\textsc{opt}}\\
\glt `May Allah give you a good trip!'

\ex \label{ex:5:119} % (119)
\gll \emph{allah-li-ni} \emph{qu} \emph{\textbf{b-alq-aq-ab}!}\\
 Allah-\textsc{obl}-\textsc{erg} field \textbf{\textsc{n}-grow:\textsc{ipfv}-\textsc{caus}-\textsc{opt}}\\
\glt `May the field grow with the help of Allah!'

\ex \label{ex:5:120} % (120)
\gll \emph{*allah-li-ni} \emph{hum-be} \emph{ʡaˤχ} \emph{\textbf{d-uh-a-b}!}\\
 Allah-\textsc{obl}-\textsc{erg} way-\textsc{pl} good \textbf{\textsc{npl}-become:\textsc{pfv}-\textsc{irr}-\textsc{opt}}\\
\glt Intended: `May Allah give you a good trip!'

\ex \label{ex:5:121} % (121)
\gll \emph{*allah-li-ni} \emph{qu} \emph{\textbf{b-alq-ab}!}\\
 Allah-\textsc{obl}-\textsc{erg} field \textbf{\textsc{n}-grow:\textsc{ipfv}-\textsc{opt}}\\
\glt Intended: `May the field grow with the help of Allah!'
\z

If there is another overt ergative participant in the sentence, the
clause is interpreted as an ordinary causative construction; cf.\ (\ref{ex:5:124}):

\ea % (122)
\gll \emph{rasul} \emph{\textbf{w-ebk'-a-b}!}\\
 Rasul \textbf{\textsc{m}-die:\textsc{pfv}-\textsc{irr}-\textsc{opt}}\\
\glt `May Rasul die!'

\ex % (123)
\gll \emph{rasul} \emph{\textbf{w-ebk'-aq-a-b}!}\\
 Rasul \textbf{\textsc{m}-die:\textsc{pfv}-\textsc{caus}-\textsc{irr}-\textsc{opt}}\\
\glt `May Allah make Rasul die!'

\ex \label{ex:5:124} % (124)
\gll \emph{pat'imat-ini} \emph{rasul}  \emph{\textbf{w-ebk'-aq-ab}!}\\
 Patimat-\textsc{erg} Rasul \textbf{\textsc{m}-die:\textsc{pfv}-\textsc{caus}-\textsc{irr}-\textsc{opt}}\\
\glt `May Patimat make Rasul die!'
\z


% 7.2.
\subsection{Optative constructions}\label{optative-constructions}

The optative form is available for all persons, but with the first
person the construction is pragmatically less felicitous.

Third person optative construction

\ea % (125)
\gll \emph{dursi} \emph{d-arš-i-le} \emph{\textbf{kalʔ-a-b}} \emph{ħa-la}.\\
 girl \textsc{f1}-be.beautiful:\textsc{pfv}-\textsc{aor}-\textsc{cvb} \textbf{stay:\textsc{pfv}-\textsc{irr}-\textsc{opt}} you.sg.\textsc{obl}-\textsc{gen}\\
\glt `May your daughter be beautiful.'

\ex % (126)
\gll \emph{urši} \emph{q'uwat } \emph{le-b-le} \emph{\textbf{kalʔ-a-b}} \emph{ħa-la.}\\
 boy strength be-\textsc{n}-\textsc{cvb} \textbf{stay:\textsc{pfv}-\textsc{irr}-\textsc{opt}} you.sg.\textsc{obl}-\textsc{gen}\\
\glt `May your son be strong.' (lit.\ May your boy stay having strength)
\z

Second person optative construction

\ea % (127)
\gll \emph{d-arš-ib-i} \emph{\textbf{kalʔ-a-b}} \emph{ħu.}\\
 \textsc{f1}-be.beautiful:\textsc{pfv}-\textsc{aor}-\textsc{ptcp} \textbf{stay-\textsc{irr}-\textsc{opt}} you.sg\\
\glt `May you be beautiful.'

\ex % (128)
\gll \emph{q'uwat} \emph{le-w-i} \emph{\textbf{kalʔ-a-b}} \emph{ħu.}\\
 strong be-\textsc{m}-\textsc{ptcp} \textbf{stay:\textsc{pfv}-\textsc{irr}-\textsc{opt}} you.sg\\
\glt `May you be strong.'
\z

First person optative construction

\ea % (129)
\gll \emph{nu} \emph{\textbf{r-ebk'}} / \emph{\textbf{r-ebk'-ab}!}\\
 I \textbf{\textsc{f}-die:\textsc{pfv}(\textsc{opt})} / \textbf{\textsc{f}-die:\textsc{pfv}-\textsc{opt}}\\
\glt `May I die [but not you – addressing the child]!'
\z

In optative constructions, frozen formulaic expressions are typical, and central
participants are often left implicit. Cf.\ examples (\ref{ex:5:108}), (\ref{ex:5:114}), (\ref{ex:5:116})
where the person affected by the wish is overtly expressed.
However, mentioning this person is not ungrammatical, as in the
following examples:

\ea % (130)
\gll \emph{\textbf{muħammadi-s}} \emph{hum-be} \emph{ʡaˤχ} \emph{\textbf{d-uh-aq-a-b}!}\\
 \textbf{Muhammad-\textsc{dat}} way-\textsc{pl} good \textbf{\textsc{npl}-become:\textsc{pfv}-\textsc{caus}-\textsc{irr}-\textsc{opt}}\\
\glt `May Muhammad have a good trip!'

\ex % (131)
\gll \emph{\textbf{muħammad-ini}} \emph{ʁačne} \emph{ʡaˤχ-le} \emph{\textbf{d-ic-a-b}.}\\
 \textbf{Muhammad-\textsc{erg}} calf.\textsc{pl} good-\textsc{advz} \textbf{\textsc{npl}-sell:\textsc{pfv}-\textsc{irr}-\textsc{opt}}\\
\glt `May Muhammad sell calves with a profit.'
\z

Another possible participant of the optative situation is Allah. Most
often it occurs in optative sentences as a form of \isi{address}:

\ea % (132)
\gll \emph{\textbf{ja-allah}} \emph{ħušab} \emph{taliħ} \emph{\textbf{g-a-b}!}\\
 \textbf{\textsc{ptcl}-Allah(\textsc{nom})} you.pl.\textsc{dat} luck \textbf{give:\textsc{pfv}-\textsc{irr}-\textsc{opt}}\\
\glt `May [Allah] give [you] luck!'
\z

In transitive constructions, Allah can also be expressed as an Agent,
assuming ergative marking:

\ea % (133)
\gll \emph{allah-li}\,\footnotemark{} \emph{ara-deš} \emph{\textbf{g-a-b}!}\\
Allah-\textsc{obl}(\textsc{erg}) healthy-\textsc{nmlz} \textbf{give:\textsc{pfv}-\textsc{irr}-\textsc{opt}}\\
\glt `May [Allah] give [you] health!'

\footnotetext{The ergative forms \emph{Allahlini} \textasciitilde{} \emph{Allahli} are morphological variants.}

\ex % (134)
\gll \emph{\textbf{m-irq-ab}} \emph{ħu} \emph{allah-li.}\\
\textbf{\textsc{negvol}-\textsc{m}.let.go:\textsc{ipfv}-\textsc{opt}} you.sg(\textsc{nom}) Allah-\textsc{obl}(\textsc{erg})\\
\glt `May Allah stay with you.' (= may Allah not let something bad happen to
you) (Aspectual test 1, 1.156)
\z

The Ergative form of the word \emph{Allah} cannot co-occur with another
agent in the ergative case:

\ea % (135)
\gll \emph{*allah-li} \emph{ħu-ni} \emph{b-iz-il} \emph{t'ult'} \emph{\textbf{b-aq'-a-b}.}\\
Allah-\textsc{erg} you.sg-\textsc{erg} \textsc{n}-tasty-\textsc{atr} bread \textbf{\textsc{n}-do:\textsc{pfv}-\textsc{irr}-\textsc{opt}}\\
\glt Intended: `May you make good bread with the help of Allah.'
\z

% 7.3.
\subsection{Semantics of the Optative}\label{semantics-of-the-optative}

Optative forms are dedicated to the expression of good or bad wishes.

\ea % (136)
\gll \emph{ʔaq'} \emph{\textbf{lug-a-b,}} \emph{balhni} \emph{\textbf{g-a-b}.}\\
 intellect \textbf{give:\textsc{ipfv}-\textsc{irr}-\textsc{opt}} knowledge \textbf{give:\textsc{pfv}-\textsc{irr}-\textsc{opt}}\\
\glt `May [Allah] give [you] intellect, may [Allah] give [you] knowledge.'
\z

Unlike the jussive, the optative does not denote an action which is
meant to be fulfilled by the addressee or by a third person. If the
optative is derived from a verb which typically denotes controllable
actions, the sentence is interpreted as a wish that Allah fulfills the
action. The following example can be interpreted as a wish which can be
made real by Allah, but not as an indirect command to the third person to
give money:

\ea % (137)
\gll \emph{d-aq-il} \emph{arc} \emph{\textbf{g-a-b}.}\\
 \textsc{npl}-much-\textsc{atr} money \textbf{give:\textsc{pfv}-\textsc{irr}-\textsc{opt}}\\
\glt `May you be given [by Allah] a lot of money.'
\z

The optative cannot refer to the \isi{past}, cf.\ examples (\ref{ex:5:138}) and (\ref{ex:5:139}):

\ea \label{ex:5:138} % (138)
\gll \emph{\textbf{w-ebk'-a-b}} \emph{nu!}\\
\textbf{\textsc{m}-die-\textsc{irr}-\textsc{opt}} I\\
\glt `May I die! `

\ex \label{ex:5:139} % (139)
\gll \emph{*dag} \emph{\textbf{w-ebk'-a-b}} \emph{nu!}\\
 yesterday \textbf{\textsc{m}-die-\textsc{irr}-\textsc{opt}} I\\
\glt Intended: `I wish I had died yesterday!'
\z

Optative forms are widely used in everyday life. Below are some
traditional optative formulae:

\ea % (140)
\gll \emph{q'uwat} \emph{\textbf{g-a-b}!}\\
 strength \textbf{give:\textsc{pfv}-\textsc{irr}-\textsc{opt}}\\
\glt `May [Allah] give [you] strength!'

\ex % (141)
\gll \emph{k'ʷabaq'ala} \emph{\textbf{g-a-b}.}\\
 god.help\footnotemark{} \textbf{give:\textsc{pfv}-\textsc{irr}-\textsc{opt}}\\
\glt `May you have enough strength [to do your work].'

\footnotetext{This word occurs only in this formula and so far seems to be unanalyzable.}

\ex % (142)
\gll \emph{\textbf{w-ebk'-a-b}} \emph{ħu!}\\
 \textbf{\textsc{m}-die:\textsc{pfv}-\textsc{irr}-\textsc{opt}} you.sg\\
\glt `May you die!'

\ex % (143)
\gll \emph{ja-allah} \emph{\textbf{d-alq-aq-a-b}!} \\
 \textsc{ptcl}-Allah \textbf{\textsc{npl}-grow:\textsc{ipfv}-\textsc{caus}-\textsc{irr}-\textsc{opt}} \\
\glt `May [it] grow! (wish formula addressed to the person who is
planting something)'
\z

\largerpage


% 7.4.
\subsection{Expression of wish by means of forms in  \emph{-q'alle}}\label{expression-of-wish}


The wish of the speaker can also be expressed by forms ending in
\emph{-q'alle}. The derivation of these forms is described in \sectref{irreal-forms}.
Forms in \emph{-q'alle} show some properties of {converbs}\is{converb} (see \sectref{irreal-forms}
and \citealt{sheyanova2019} [this volume]); the wish-constructions with forms in
\emph{-q'alle} must be considered as cases of \isi{insubordination} (in terms
of \citealt{evans2007}).

The counterfactual conditional converb in \emph{-q'alle} can be used in
a main clause in order to express the speaker's wish (similar to the
forms of the conditional protasis in many European languages, as well as
other languages of the East Caucasian family, cf.\ \citealt{belyaev2012}).
Independent converbs in \emph{-q'alle} differ semantically from the
optative. While the optative form expresses blessings and curses,
constructions with conditional converbs denote dreams and desires of
speaker about some uncontrollable events. In \citet{dobrushina2011}, these two
types of optative were referred to as performative optative and
desiderative optative. East Caucasian languages often have a dedicated
inflectional form for the former, but the latter is usually expressed by
conditional forms, as in Mehweb.

\ea % (144)
\gll \emph{ca} \emph{di-la} \emph{qali} \emph{\textbf{b-uʔ-ib-q'alle}!}\\
 \textsc{ptcl} I.\textsc{obl}-\textsc{gen} house \textbf{\textsc{n}-become:\textsc{pfv}-\textsc{aor}-\textsc{ctrf}}\\
\glt `If only I had a house!'

\ex % (145)
\gll \emph{di-la} \emph{adami} \emph{žaˤwal} \emph{\textbf{ʡaˤš-w-irq-ul-q'alle}!}\\
 I.\textsc{obl}-\textsc{gen} husband early \textbf{\textsc{pv}-\textsc{m}-come.back:\textsc{ipfv}-\textsc{atr}-\textsc{ctrf}}\\
\glt `If only my husband came back soon!'
\z

The speaker's wish\is{expression of wish} can also be expressed by a combination of the
infinitive with the counterfactual marker \emph{-q'alle}:

\ea % (146)
\gll \emph{nu-ni} \emph{čaj} \emph{\textbf{d-erž-es-q'alle}!}\\
 I-\textsc{erg} tea \textbf{\textsc{npl}-drink:\textsc{pfv}-\textsc{inf}-\textsc{ctrf}}\\
\glt `I wish I had some tea!'
\z

Unlike other {converbs}\is{converb} in \emph{-q'alle}, the converb derived from the 
infinitive is not used in reference to the past:

\largerpage

\ea % (147)
\gll \emph{dag} \emph{\textbf{w-ebk'-ib-q'alle}} \emph{nu!}\\
 yesterday \textbf{\textsc{m}-die:\textsc{pfv}-\textsc{aor}-\textsc{ctrf}} I\\
\glt `If only I had died yesterday!'

\ex % (148)
\gll \emph{*nu-ni} \emph{dag} \emph{čaj} \emph{\textbf{d-erž-es-q'alle}!}\\
 I-\textsc{erg} yesterday tea \textbf{\textsc{npl}-drink:\textsc{pfv}-\textsc{inf}-\textsc{ctrf}}\\
\glt Intended: `I wish I had some tea yesterday!'
\z

The hypothetical conditional converb in \emph{-k'a} (see \sectref{irreal-forms})
cannot be used in independent constructions.

\ea % (149)
\gll \emph{*nu-ni} \emph{čaj} \emph{\textbf{d-erž-a-k'a}!}\\
 I-\textsc{erg} tea \textbf{\textsc{npl}-drink-\textsc{irr}-\textsc{cond}}\\
\glt Intended: `I wish I had some tea yesterday!'

\ex % (150)
\gll \emph{nu-ni} \emph{čaj} \emph{\textbf{d-erž-a-k'a,}} \emph{ʡaˤχ-le} \emph{b-uʔ-a-re.}\\
 I-\textsc{erg} tea \textbf{\textsc{npl}-drink-\textsc{irr}-\textsc{cond}} good-\textsc{advz} \textsc{n}-become-\textsc{irr}-\textsc{pst}\\
\glt `If I had some tea, it would be good.'
\z


\removelastskip
\is{optative|)}
\is{expression of wish|)}

% 8.
\section{Irreal forms}\label{irreal-forms}

\is{irrealis|(}

Cross-linguistically, forms with irreal meaning are most often found in
conditional constructions and in complement clauses \citep{mauri-sanso2016}.
In Mehweb, as in many other languages of Daghestan, complement clauses
do not employ irreal forms. Mehweb conditional constructions have
non-finite forms in the subordinate clause (conditional converbs), and a
finite form in the main clause (irrealis). In this section, the derivation
of conditional converbs (\sectref{conditional-converbs}) and irrealis (\sectref{irrealis}) will be discussed. In
\sectref{counterfactual-conditional-clauses}, \sectref{hypothetical-conditional-constructions}, and \sectref{real-conditional-constructions}, conditional constructions of different types
will be considered.

% 8.1.
\subsection{Conditional converbs}\label{conditional-converbs}

\is{conditional|(}

There are two markers of conditional clauses in Mehweb. They are
distributed according to the degree of (ir)reality: the suffix
\emph{-k'a} is used in conditional clauses which may come true
(hypothetical\is{hypothetical conditional} marker), the suffix \emph{-q'alle} designates situations
which did not and cannot take place (\isi{counterfactual} marker).

The suffix \emph{-k'a} presumably originates from the particle \emph{k'a}.
The particle \emph{k'a} is used for \isi{topicalization} of words of different
classes. In example (\ref{ex:5:151}), it attaches to the noun \emph{sinkala}, in
example (\ref{ex:5:152}) – to the perfective stem of the verb. In the latter
example, the particle is used together with reduplication\is{stem copy}, typical for
predicate topicalization \citep{maisak2010}: \emph{dargk'a dargira}.


\ea \label{ex:5:151} % (151)
\gll \emph{sinka-la} \emph{\textbf{k'a}} \emph{abzul-le} \emph{ʁʷaˤn-ne} \emph{d-elʔ-un-na} \emph{wahaj-le꞊l} \emph{ʁʷaˤn-ne~} \emph{luʔ-es} \emph{w-aʔ-i-ra.}\\
 bear-\textsc{gen} \textbf{\textsc{ptcl}} all-\textsc{advz} lie-\textsc{pl} \textsc{npl}-tell:\textsc{pfv}-\textsc{aor}-\textsc{ego} very-\textsc{advz}꞊\textsc{emph} lie-\textsc{pl} tell:\textsc{ipfv}-\textsc{inf} \textsc{m}-begin:\textsc{pfv}-\textsc{aor}-\textsc{ego}\\
\glt `As for the bear, I did actually tell fibs.' (Aspectual test 1, 1.89)

\ex \label{ex:5:152} % (152)
\gll \emph{\textbf{d-arg-k'a}} \emph{il-di} \emph{qulle} \emph{di-ze} \emph{d-arg-i-ra} \emph{huni꞊ra} \emph{b-arg-i-ra.}\\
 \textbf{\textsc{npl}-find:\textsc{pfv}-\textsc{ptcl}} this-\textsc{pl} house.\textsc{pl} I.\textsc{obl}-\textsc{inter}(\textsc{lat}) \textsc{npl}-find:\textsc{pfv}-\textsc{aor}-\textsc{ego} road꞊\textsc{add} \textsc{n}-find:\textsc{pfv}-\textsc{aor}-\textsc{ego}\\ 
\glt `As for getting there, I did reach those houses and found the street.' (Aspectual test 1, 1.164)
\z

Elicitation gave examples with \isi{predicate topicalization} marked by the
particle \emph{k'a} alone, without reduplication:

\ea % (153)
\gll \emph{\textbf{luč'-ib-k'a}} \emph{il} \emph{ʡaˤχ-le.}\\
 \textbf{learn:\textsc{ipfv}-\textsc{ipft}-\textsc{ptcl}} this good-\textsc{advz}\\
\glt `As for studies, he did study well.'

\ex % (154)
\gll \emph{\textbf{luč'-an-k'a}} \emph{il} \emph{ʡaˤχ-le} \emph{amma} \emph{abaj-s} \emph{zahmat} \emph{d-urh-an} \emph{il} \emph{d-aχ-as.}\\
\textbf{learn:\textsc{ipfv}-\textsc{hab}-\textsc{ptcl}} this good-\textsc{advz} but mother.\textsc{obl}-\textsc{dat} difficult \textsc{f1}-be:\textsc{ipfv}-\textsc{hab} this \textsc{f1}-support-\textsc{inf} \\
\glt 
`As for studies, she makes good progress. But it is difficult for her
mother to support her.'
\z

That the suffix of conditional clause originates from the \isi{topicalization}
particle is in conformity with the close relation between topic and
condition as described in \citet{haiman1978}. It is likely that the suffix of
counterfactual condition \emph{-q'alle} also originates from the marker
of topicalization. In Mehweb, the only function of \mbox{\emph{-q'alle}} is to
mark counterfactual conditional converbs, but in some other Dargwa
languages there is a particle \emph{q'al} (\emph{q'alli}) with a wide
range of meanings including topicalization (\citealt{maisak2010};
\citealt{mutalov-sumbatova2003}; \citeauthor{forker:sanzhi} in preparation).
The following examples come from two Dargwa dialects:

Dargwa (Khuduts village) (\citealt{maisak2010}; example elicited by D.~Ganenkov)

\ea % (155)
\gll \emph{buč'꞊\textbf{q'al}} \emph{buč'unni} \emph{cab} \emph{cik'al} \emph{hankalgunnekːu.}\\
read:\textsc{ipfv}꞊\textbf{\textsc{ptcl}} read:\textsc{ipfv}.\textsc{cvb} \textsc{cop} nothing  remember:\textsc{ipfv}.\textsc{cvb}+\textsc{cop}.\textsc{neg}\\
\glt `As for reading, he reads (the book), but does not remember anything.'
\z

Dargwa (Icari village) (\citealt{maisak2010}; example suggested by R.~Mutalov)

\ea % (156)
\gll \emph{buč'-ni-la} \emph{\textbf{q'alli}} \emph{buč'atːa} \emph{cacajnaqːilla} \emph{behelra\ldots{}}\\
 read:\textsc{ipfv}-\textsc{nmlz}-\textsc{gen} \textbf{\textsc{ptcl}} read.\textsc{prs}.\textsc{1sg} sometimes  however\\
\glt `As for reading, I read (books), but...'
\z

Forms in \emph{-q'alle} and in \emph{-k'a} can be embedded. This is an
argument in favor of their converbial status.

\ea % (157)
\gll \emph{nu} \emph{[di-la} \emph{urši-li-ni} \emph{xunul} \emph{\textbf{k-a-k'a]}} \emph{iχ-di-li-šu-r} \emph{d-uʔ-es-i.}\\
 I I.\textsc{obl}-\textsc{gen} boy-\textsc{obl}-\textsc{erg} wife  \textbf{bring:\textsc{pfv}-\textsc{irr}-\textsc{cond}} that-\textsc{pl}-\textsc{obl}-\textsc{ad}-\textsc{f}(\textsc{ess}) 
\textsc{f1}-be:\textsc{pfv}-\textsc{inf}-\textsc{atr} \\
\glt `If my son gets married, I will live at their place.'

\ex % (158)
\gll \emph{nu꞊ra} [\emph{iχ} \emph{\textbf{w-ebk'-ib-q'alle}}] \emph{d-ubk'-a-re.}\\
 I꞊\textsc{add} this \textbf{\textsc{m}-die:\textsc{pfv}-\textsc{aor}-\textsc{ctrf}} \textsc{f1}-die:\textsc{ipfv}-\textsc{irr}-\textsc{pst}\\
\glt `If he died, I would have also died.'
\z

In \sectref{hypothetical-conditional-converb} and \sectref{counterfactual-conditional-converb}, the derivation of conditional converbs in
\emph{-k'a} and \emph{-q'alle} will be considered in more detail.

% 8.1.1.
\subsubsection{Hypothetical conditional converb}\label{hypothetical-conditional-converb}

In the Hypothetical conditional converb, the suffix \emph{-k'a} can be added
to the Irreal stem of imperfective and perfective verbs. Therefore,
every verb has two conditional converbs in \emph{-k'a}:
\textsc{cl}-\emph{elč'es} `read, \textsc{pfv}' – \textsc{cl}-\emph{elč'ak'a}; \emph{luč'es} `read, \textsc{ipfv}' – \emph{luč'ak'a}.

Conditional clauses with converbs in \emph{-k'a} denote that the
situation can come true in the future:

\ea % (159)
\gll \emph{hel} \emph{deħ} \emph{\textbf{b-aq'-a-k'a}} \emph{sinka-li} \emph{nuša} \emph{k'ʷi-jal-la~} \emph{b-erg-es.} \\
this smell \textbf{\textsc{n}-do:\textsc{pfv}-\textsc{irr}-\textsc{cond}} bear-\textsc{obl}(\textsc{erg}) we two-\textsc{card}-\textsc{add} \textsc{hpl}-eat:\textsc{pfv}-\textsc{inf} \\
\glt `If the bear smells this, he will eat us both.' (Text M. A bear, a wolf
and a fox, 1.6)
\z

Followed by the additive particle \emph{-ra}, hypothetical conditional
converbs are used in concessive clauses (\ref{ex:5:160}). This pattern of marking
concessive clauses – by a combination of conditional converb and
emphatic or additive particle, also well known in Latin and Romance
languages – is attested in the majority of East Caucasian languages
(cf.\ Tanti \citep[138]{sumbatova-lander2014}, Agul \citep{dobrushina-merdanova2012};
\citealt{forker2016} for generalizations).

\pagebreak

\ea \label{ex:5:160} % (160)
\gll \emph{iti-s} \emph{rasul} \emph{hune-če} \emph{\textbf{w-ik-a-k'a-ra,}}  \emph{it-ini} \emph{beʁi-če} \emph{waˤb-ʜaˤ-baˤt-ur.}\\
 this.\textsc{obl}-\textsc{dat} Rasul way-\textsc{super}(\textsc{lat}) \textbf{\textsc{m}-happen:\textsc{pfv}-\textsc{irr}-\textsc{cond}-\textsc{add}} this-\textsc{erg} wedding.\textsc{obl}-\textsc{super}(\textsc{lat}) call-\textsc{neg}-\textsc{lv}:\textsc{pfv}-\textsc{aor}\\
\glt `Although she met Rasul, she did not call him to the wedding.'

\ex % (161)
\gll \emph{mu-lug-adi~} \emph{\textbf{d-uk'-a-k'a-ra}}, \emph{maja} \emph{g-i-le} \emph{le-l-le} \emph{hub-li-s.}\\
\textbf{\textsc{negvol}-give:\textsc{ipfv}-\textsc{proh}} \textsc{f1}-say:\textsc{ipfv}-\textsc{irr}-\textsc{cond}-\textsc{add} Maja give:\textsc{pfv}-\textsc{aor}-\textsc{cvb} \textsc{aux}-\textsc{f}-\textsc{cvb} husband-\textsc{obl}-\textsc{dat} \\
\glt `Although she said: `Don't give', they still married Maja'. (Text 14.
Laces,~1.3)
\z

% 8.1.2.
\subsubsection{Counterfactual conditional converb}\label{counterfactual-conditional-converb}

% \is{counterfactual conditional|(}
\is{counterfactual|(}

The counterfactual marker \emph{-q'alle} can be added to the past and
infinitive forms, but not to the present. The speakers of Mehweb
sometimes consider \emph{-q'alle} as a separate word, but it cannot be
separated from the verb. In this description, we consider \emph{-q'alle}
as a suffix. \tabref{tab:5:6} summarizes the combinations of the verbal stems and
the suffix \emph{-q'alle}: possible combinations are marked as (+),
impossible combinations are marked as (–); the perfective present form does not exist in Mehweb.
Examples are presented in \tabref{tab:5:7}.

\begin{table}[h]
 % Table 6.
 \caption{Stems which can add the counterfactual suffix \emph{q'alle}}\label{tab:5:6}
 
\begin{tabular}{@{}lcccc@{}}
\toprule
& past & present & infinitive & participle\tabularnewline \midrule
imperfective & (+) & (–) & (+) & (+)\tabularnewline
perfective & (+) & & (+) & (+)\tabularnewline
\bottomrule
\end{tabular}
\end{table}


\begin{table}[b]
 % Table 7.
 \caption{Examples of the forms with the counterfactual suffix \emph{-q'alle}}\label{tab:5:7}

\begin{tabular}{@{}lllll@{}}
\toprule
& & past & infinitive & participle\tabularnewline \midrule
\raisebox{-6pt}[0pt][0pt]{`find'} & imperfective & \emph{b-urg-ib-q'alle} & \emph{b-urg-es-q'alle}
& \emph{b-urg-ul-q'alle}\tabularnewline
& perfective & \emph{b-arg-ib-q'alle} & \emph{b-arg-es-q'alle} &
                                                                 \emph{b-arg-ib-i-q'alle}\\ \midrule
 \raisebox{-6pt}[0pt][0pt]{`read'} & imperfective & \emph{luč'-ib-q'alle} & \emph{luč'es-q'alle} &
\emph{luč'-ul-q'alle}\\
& perfective & \emph{b-elč'-un-q'alle} & \emph{b-elč'-es-q'alle} &
\emph{b-elč'-un-i-q'alle}\\
\bottomrule
\end{tabular}
\end{table}

\pagebreak


Counterfactual converbs in \emph{q'alle} are used in subordinate clauses
of conditional constructions (example (\ref{ex:5:162}), more details in
\sectref{counterfactual-conditional-clauses}), and in independent clauses with the meaning of wish (example (\ref{ex:5:163}),
more details in \sectref{expression-of-wish}). This latter usage may be considered a case
of \isi{insubordination}, typical for the forms used in conditional clause.


\ea \label{ex:5:162} % (162)
\gll \emph{hete-r} \emph{hed-di} \emph{malʔun-t-ini} \emph{r-uc-es} \emph{q'-oˤwe} \emph{le-l-le} \emph{k'ʷan,} \emph{nu} \emph{ca-ʁida} \emph{ajʁaj} \emph{\textbf{r-uh-ub-q'alle}.}\\
 there-\textsc{f}(\textsc{ess}) that.far.away-\textsc{pl} devil-\textsc{pl}-\textsc{erg}  \textsc{f}-catch:\textsc{pfv}-\textsc{inf} go:\textsc{ipfv}-\textsc{cvb.ipfv} \textsc{aux}-\textsc{npl}-\textsc{cvb} \textsc{quot} I(\textsc{nom}) one-few tarry \textbf{\textsc{f}-become:\textsc{pfv}-\textsc{aor}-\textsc{ctrf}} \\
\glt `If I would tarry there for just a minute, these devils would get to me
for sure.' (Text 03, Story told by Aminat, 1.29)

\ex \label{ex:5:163} % (163)
\gll \emph{ca} \emph{di-la} \emph{urši-li-ni} \emph{xunul} \emph{\textbf{d-ik-ul-q'alle}!}\\
 \textsc{ptcl} I.\textsc{obl}-\textsc{gen} boy-\textsc{obl}-\textsc{erg} wife \textbf{\textsc{f1}-bring:\textsc{ipfv}-\textsc{ptcp}-\textsc{ctrf}}\\
\glt `If only my son got married!'
\z

\removelastskip
\is{counterfactual|)}
% \is{counterfactual conditional|)}
\is{conditional|)}


% 8.2.
\subsection{Irrealis}\label{irrealis}

The predicate of the main clause of conditional constructions is
expressed by the form with the suffixal cluster \emph{-a-re}:
\textsc{cl}\emph{-ubk'are} `would die'. The cluster consists of the suffix of the
Irreal stem \emph{-a-} and the suffix of the Past \emph{-re}
(\emph{-a-re} – \textsc{irr}-\textsc{pst}). The marker \emph{-are} is used only for the
expression of irrealis, so the form must be considered as a dedicated
irrealis. The past suffix \emph{-re} is not productive. Apart from
irrealis, the suffix \emph{-re} occurs regularly only in several
lexemes: in the past copula \emph{le-}\textsc{cl}\emph{-re}, negative copula
\emph{agʷire}, in the lexeme \emph{burgare} `likely,
probably' (originally irrealis), and the form \emph{digibre} `would
like':

\ea % (164)
\gll \emph{k'ala-li-ze-b} \emph{\textbf{le-b-re}} \emph{doˤʜi.}\\
 Kala-\textsc{obl}-\textsc{inter}-\textsc{n}(\textsc{ess}) \textbf{{be}-\textsc{n}-\textsc{pst}} snow\\
\glt `There was snow in Kala.' (Text 15, Lost Donkeys)

\ex % (165)
\gll \emph{nab} \emph{\textbf{d-ig-ib-re}} \emph{čaj.}\\
 I.\textsc{dat} \textbf{\textsc{npl}-want:\textsc{ipfv}-\textsc{ipft}-\textsc{pst}} tea\\
\glt `I would like some tea.'
\z

Some speakers acknowledge other forms in \emph{-re} derived from the
past stem of imperfective verbs as grammatical, such as \emph{luč'ibre}
(\emph{luč'es} `read, study, \textsc{ipfv}'), \emph{isibre} (\emph{ises}, `take,
buy, \textsc{ipfv}'), \emph{urcibre} (\emph{urces} `fly, \textsc{ipfv}'). These forms are
also interpreted as irrealis:

\ea % (166)
\gll \emph{\textsuperscript{?}tukaj-ħe-la} \emph{si-k'al} \emph{\textbf{is-ib-re}} \emph{nu-ni꞊ra,} \emph{arc} \emph{d-uʔ-ib-q'alle.}\\
 shop-\textsc{in}-\textsc{el} what-\textsc{indef} \textbf{take:\textsc{ipfv}-\textsc{ipft}-\textsc{pst}} I-\textsc{erg}꞊\textsc{add}  money  \textsc{npl}-be:\textsc{ipfv}-\textsc{aor}-\textsc{ctrf} \\
\glt `I would have bought something in the shop, if (I\textsc{)} had some money.'
\z

These forms however are never used spontaneously, do not occur in texts,
and many speakers do not recognize them at all. Even the speakers who
can come up with an example using one of these forms, tend to replace it
by the regular irrealis in \emph{-are}.

The irrealis form in \emph{-are} is used in the main clause of
conditional clauses (most often counterfactual) (\ref{ex:5:167}) as well as for the
expression of irreal situations in independent clauses beyond
conditional constructions (\ref{ex:5:168}):

\ea \label{ex:5:167} % (167)
\gll \emph{iχ} \emph{w-ebk'-ib-q'alle,} \emph{nu꞊ra} \emph{\textbf{d-ubk'-a-re}.}\\
 this \textsc{m}-die:\textsc{pfv}-\textsc{aor}-\textsc{ctrf} I꞊\textsc{add} \textbf{\textsc{f1}-die:\textsc{ipfv}-\textsc{irr}-\textsc{pst}}\\
\glt `If he had died, I would have also died.'

\ex \label{ex:5:168} % (168)
\gll \emph{rasuj-ni} \emph{qu} \emph{\textbf{išq-aˤ-re}} \emph{dag,} \emph{amma} \emph{ʜaˤ-q'-un.}\\
 Rasul-\textsc{erg} field \textbf{mow:\textsc{ipfv}-\textsc{irr}-\textsc{pst}}  yesterday but \textsc{neg}-\textsc{m}.go:\textsc{pfv}-\textsc{aor} \\
\glt `Rasul could have mowed the field yesterday, but he didn't go.'
\z

% 8.3.
\subsection{Counterfactual conditional clauses}\label{counterfactual-conditional-clauses}

\is{counterfactual|(}
\is{converb|(}

Counterfactual conditional clauses contain a converb in \emph{-q'alle}
in the protasis, and the irrealis in the apodosis. The constructions
with the converb in \emph{-q'alle} and irrealis in \emph{-are} denote
situations which did not take place in the past (\ref{ex:5:169}), and most likely
will not take place in the future (\ref{ex:5:170}).

\ea \label{ex:5:169} % (169)
\gll \emph{urši-li-ni} \emph{xunul} \emph{\textbf{k-ib-q'alle,}} \emph{nu} \emph{iχ-di-li-šu-r} \emph{\textbf{d-uʔ-a-re}.}\\
 boy-\textsc{obl}-\textsc{erg} wife \textbf{take:\textsc{pfv}-\textsc{aor}-\textsc{ctrf}} I that-\textsc{pl}-\textsc{obl}-\textsc{ad}-\textsc{hpl(ess)} \textbf{\textsc{f1}-become:\textsc{pfv}-\textsc{irr}-\textsc{pst}}\\
\glt `If my son had got married, I would have lived at their place.'

\ex \label{ex:5:170} % (170)
\gll \emph{c'able} \emph{\textbf{w-ebk'-ib-q'alle,}} \emph{nu꞊ra} \emph{\textbf{d-ubk'-a-re}.}\\
 tomorrow \textbf{\textsc{m}-die:\textsc{pfv}-\textsc{aor}-\textsc{ctrf}} I꞊\textsc{add} \textbf{\textsc{f1}-die:\textsc{ipfv}-\textsc{irr}-\textsc{pst}}\\
\glt `If you should die tomorrow, I would also die.'
\z

A conditional clause with a counterfactual converb derived from an
aorist refers to the past; if the converb is derived from an
imperfective participle, it refers to the present:

\ea % (171)
\gll \emph{iχ} \emph{dag} \emph{\textbf{ʡaˤš-w-aqˤ-ib-q'alle}} \emph{ʡaˤχ-le} \emph{b-uʔ-a-re.}\\
 this yesterday \textbf{\textsc{pv}-\textsc{m}-come.back:\textsc{pfv}-\textsc{aor}-\textsc{ctrf}} good-\textsc{advz} \textsc{n}-be:\textsc{pfv}-\textsc{irr}-\textsc{pst}\\
\glt `If he had come yesterday, it would have been good.'

\ex % (172)
\gll \emph{iχ} \emph{išbari} \emph{\textbf{ʡaˤš-w-irq-ul-q'alle}}  \emph{ʡaˤχ-le} \emph{b-uʔ-a-re.}\\
 this today \textbf{\textsc{pv}-\textsc{m}-come.back:\textsc{ipfv}-\textsc{ptcp}-\textsc{ctrf}} good-\textsc{advz} \textsc{n}-be:\textsc{pfv}-\textsc{irr}-\textsc{pst}\\
\glt `If he came today, it would be good.'
\z

Converbs in \emph{-q'alle} based on infinitives refer to the future, but
there is an additional meaning of wish\is{expression of wish}. They are also used in
independent clauses (\sectref{expression-of-wish}) to express wish. In conditional
protasis, they denote desirable situations (\ref{ex:5:173}). Therefore, the converb
``infinitive + \emph{-q'alle}'' is not appropriate if the conditional
construction denotes non-desirable situations (\ref{ex:5:175}):

\ea \label{ex:5:173} % (173)
\gll \emph{nu-ni} \emph{čaj} \emph{d-erž-es-q'alle} \emph{nu}  \emph{wana} \emph{urh-a-re.}\\
 I-\textsc{erg} tea \textsc{npl}-drink:\textsc{pfv}-\textsc{inf}-\textsc{ctrf} I warm 1.become:\textsc{ipfv}-\textsc{irr}-\textsc{pst}\\
\glt `If I had tea, I would get warm.'

\ex % (174)
\gll \emph{abaj} \emph{\textbf{d-ebk'-ib-q'alle,}} \emph{il}  \emph{eh-il} \emph{urh-a-re.}\\
mother \textbf{\textsc{f1}-die:\textsc{pfv}-\textsc{aor}-\textsc{ctrf}} this bad-\textsc{atr} 1.become:\textsc{ipfv}-\textsc{irr}-\textsc{pst}\\
\glt `If his mother had died, he would have become a bad person.'

\ex \label{ex:5:175} % (175)
\gll \emph{*abaj} \emph{\textbf{d-ebk'-es-q'alle,}} \emph{il} \emph{eh-il} \emph{urh-a-re.}\\
 mother \textbf{\textsc{f1}-die:\textsc{pfv}-\textsc{inf}-\textsc{ctrf}} this bad-\textsc{atr} 1.become:\textsc{ipfv}-\textsc{irr}-\textsc{pst}\\
\glt Intended: `If his mother had died, he would have become a bad person.'
\z

\removelastskip
\is{counterfactual|)}
\is{converb|)}

% 8.4.
\subsection{Hypothetical conditional constructions}\label{hypothetical-conditional-constructions}

\is{hypothetical conditional|(}

Hypothetical conditional constructions denote situations which can
either be true in the present, or can be realized in the future, or are
habitual. The protasis of a hypothetical construction is expressed by
the converb in \emph{-k'a}. The apodosis can have different forms
depending on the semantics of the clause.

\ea % (176)
\gll \emph{iχ-ini} \emph{b-arx-le} \emph{b-urh-a-k'a,} \emph{iχ} \emph{w-atur} \emph{aʔ-as-i.}\\
 that-\textsc{erg} \textsc{n}-be.right-\textsc{cvb} \textsc{n}-tell:\textsc{ipfv}-\textsc{irr}-\textsc{cond} that(\textsc{nom}) \textsc{m}-free drive:\textsc{pfv}-\textsc{inf}-\textsc{atr}\\
\glt `If he tells the truth, they will let him go.'
\z

Clauses with \isi{perfective} and \isi{imperfective} hypothetical conditional
converbs in \emph{-k'a} contrast as denoting single \emph{vs.} multiple
actions:

\ea % (177)
\gll \emph{het} \emph{kung} \emph{\textbf{b-elč'-a-k'a}} \emph{nu-ni} \emph{ħa-ze} \emph{b-urh-iša} \emph{hel-li-ja} \emph{χabar.}\\
 that book \textbf{\textsc{n}-read:\textsc{pfv}-\textsc{irr}-\textsc{cond}} I-\textsc{erg} you.sg.\textsc{obl}-\textsc{inter}(\textsc{lat}) \textsc{n}-tell:\textsc{ipfv}-\textsc{fut}.\textsc{ego} this-\textsc{obl}-\textsc{gen} story\\
\glt `If you read this book, I will tell you his story.'

\ex % (178)
\gll \emph{d-aq-il} \emph{kung-ane} \emph{\textbf{luč'-a-k'a}} \emph{d-aq-il} \emph{si-k'al} \emph{nuša-ze} \emph{d-alh-ul.}\\
 \textsc{npl}-much-\textsc{atr} book-\textsc{pl} \textbf{read:\textsc{ipfv}-\textsc{irr}-\textsc{cond}} \textsc{npl}-much-\textsc{atr} what-\textsc{indef} we-\textsc{inter}(\textsc{lat}) \textsc{npl}-know:\textsc{ipfv}-\textsc{ptcp} \\
\glt `If we read many books, we know many things.'
\z

\removelastskip
\is{hypothetical conditional|)}


% 8.5.
\subsection{Real conditional constructions}\label{real-conditional-constructions}

Real conditional clauses presuppose that the state of affairs in the
subordinate clause is true. Real conditionals are sometimes treated as
\isi{reason clauses}, since\pagebreak[3] they lack the main feature of conditionals – the
lack of knowledge about the state of affairs denoted in the subordinate
clause. In Mehweb, this type of conditionals has a special mode of
marking, using an analytic construction with the verb
\textsc{cl}\emph{-arges} `find, \textsc{pfv}'. This verb is found in many languages
of Daghestan in semi-grammaticalised constructions designating direct
(visual) evidence (cf.\ \citealt{maisak-daniel2018}).

Conditional clauses of real conditional constructions have an auxiliary
verb \textsc{cl}\emph{-arges} marked by the conditional suffix \emph{-k'a}, and the
lexical verb.

The main clause of real conditional constructions can have different
indicative forms depending on the semantics of the situation. In example
(\ref{ex:5:179}), the situation of the matrix clause belongs to the past, in
examples (\ref{ex:5:180}) and (\ref{ex:5:181}) it belongs to the future:

\ea \label{ex:5:179} % (179)
\gll \emph{ili-s} \emph{hune-če} \emph{w-ik-i-le} \emph{\textbf{w-arg-a-k'a}} \emph{rasul,} \emph{il-ini} \emph{beʁi-če} \emph{waˤb-aˤt-ur-i} \emph{il.}\\
 this-\textsc{dat} way-\textsc{super}(\textsc{lat}) \textsc{m}-happen:\textsc{pfv}-\textsc{aor}-\textsc{cvb} \textbf{\textsc{m}-find:\textsc{pfv}-\textsc{irr}-\rlap{\textsc{cond}}} Rasul this-\textsc{erg} wedding-\textsc{super}(\textsc{lat}) call-\textsc{lv}:\textsc{pfv}-\textsc{aor}-\textsc{ptcp} this\\\
\glt `If she met Rasul [according to what you know about it], she called
him to the wedding.'

\ex \label{ex:5:180} % (180)
\gll \emph{anwar} \emph{w-ak'-i-le} \emph{\textbf{w-arg-a-k'a,}} \emph{abaj-šu} \emph{uˤq'-es.}\\
 Anwar \textsc{m}-come:\textsc{pfv}-\textsc{cvb} \textbf{\textsc{m}-find:\textsc{pfv}-\textsc{irr}-\textsc{cond}} mother-\textsc{ad}(\textsc{lat}) \textsc{m}.go:\textsc{pfv}-\textsc{fut}\\
\glt `As [it turned out that] Anwar came, he will go to his mother.'

\ex \label{ex:5:181} % (181)
\gll \emph{rasuj-ze} \emph{arc} \emph{kʷe} \emph{d-ik-i-le} \emph{\textbf{d-arg-a-k'a,}} \emph{il-ini} \emph{abaj-s} \emph{sajʁat} \emph{as-es.}\\
 Rasul.\textsc{obl}-\textsc{inter}(\textsc{lat}) money in.hands(\textsc{lat}) \textbf{\textsc{npl}-happen:\textsc{pfv}-\textsc{aor}-\textsc{cvb}} \textbf{\textsc{npl}-find:\textsc{pfv}-\textsc{irr}-\textsc{cond}} this-\textsc{erg} mother-\textsc{dat} gift take:\textsc{pfv}-\textsc{inf}\\
\glt `As Rasul [as it turned out] has got the money, he will buy the gift
to his mother.'
\z

\removelastskip
\is{irrealis|)}

% 9.
\section{Apprehensive}\label{apprehensive}

\is{apprehensive|(}

Mehweb has a dedicated form to express apprehension. When used in
independent clauses, the apprehensive means that the speaker is afraid
that some undesirable situation may come true. The apprehensive is
formed with the suffix \emph{-la} attached to the irrealis stem:
\emph{-a-la}.

\ea % (182)
\gll \emph{d-arʔ-a} \emph{mura,} \emph{zab} \emph{\textbf{d-aq'-a-la}.}\\
 \textsc{npl}-gather:\textsc{pfv}-\textsc{imp}.\textsc{tr} hay rain \textbf{\textsc{npl}-do:\textsc{pfv}-\textsc{irr}-\textsc{appr}}\\
\glt `Collect the hay, it might rain.'
\z


The apprehensive has a negative counterpart:

\ea % (183)
\gll \emph{zab} \emph{\textbf{ħa-d-aq'-a-la}} \emph{hab,} \emph{d-aˤq-a} \emph{šin} \emph{agarod-le-ħe.}\\
 rain \textbf{\textsc{neg}-\textsc{npl}-do:\textsc{pfv}-\textsc{irr}-\textsc{appr}} ahead \textsc{npl}-hit:\textsc{pfv}-\textsc{imp}.\textsc{tr} water vegetable.garden-\textsc{obl}-\textsc{in}(\textsc{lat}) \\
\glt `Turn on the water in the garden, [because/in case] it might not
rain.' 
\z

Apprehensive forms are commonly used to express warnings about something
that may happen to the addressee:

\ea % (184)
\gll \emph{q'eju,} \emph{\textbf{w-igʷ-a-la}.}\\
 slow \textbf{\textsc{m}-burn:\textsc{pfv}-\textsc{irr}-\textsc{appr}}\\
\glt `Be careful, beware not to get burnt.'

\ex % (185)
\gll \emph{q'eju,} \emph{\textbf{ar-d-ik-a-la}.}\\
 slow \textbf{down-\textsc{f1}-fall:\textsc{pfv}-\textsc{irr}-\textsc{appr}}\\
\glt `Be careful, beware not to fall down.'
\z

Apprehensives are often accompanied by the particle \emph{ʡaj}:

\ea % (186)
\gll \emph{ħu} \emph{ʁanq'} \emph{\textbf{uh-a-la}} \emph{ʡaj.}\\
 you.sg drown \textbf{\textsc{m}.become:\textsc{pfv}-\textsc{irr}-\textsc{appr}} \textsc{ptcl}\\
\glt `Beware not to drown.'
\z

First and third person subjects are also available in the apprehensive
constructions:

\ea % (187)
\gll \emph{nu} \emph{ʁanq'}  \emph{\textbf{uh-a-la}.}\\
 I drown \textbf{\textsc{m}.become:\textsc{pfv}-\textsc{irr}-\textsc{appr}}\\
\glt `May I not drown.'

\ex % (188)
\gll \emph{hara} \emph{nu} \emph{\textbf{ar-d-uk-a-la}!}\\
 \textsc{ptcl} I(\textsc{nom}) \textbf{away-\textsc{f1}-lead:\textsc{pfv}-\textsc{irr}-\textsc{appr}}\\
\glt `Be careful, someone may abduct me!'

\ex % (189)
\gll \emph{žanawal-li-ni} \emph{maza} \emph{\textbf{ar-b-uk-a-la}.}\\
 wolf-\textsc{obl}-\textsc{erg} sheep \textbf{away-\textsc{n}-lead:\textsc{pfv}-\textsc{irr}-\textsc{appr}}\\
\glt `The wolf can steal the sheep.'
\z

The apprehensive has an inherent negative value. If it is used with
reference to situations which are usually viewed as positive, the
situation changes its value from positive to negative. Example (\ref{ex:5:190})
is grammatical only if the speaker wants to have a daughter more than a
son (which is unusual for Daghestan). Example (\ref{ex:5:191}) is only grammatical if the
speaker does not want to recover from his illness.

\ea \label{ex:5:190} % (190)
\gll \emph{urši} \emph{\textbf{w-aq'-a-la}} \emph{ħu-ni} \emph{d-aq'-a} \emph{dursi!}\\
 boy \textbf{\textsc{m}-do:\textsc{pfv}-\textsc{irr}-\textsc{appr}} you.sg-\textsc{erg} \textsc{f1}-do:\textsc{pfv}-\textsc{imp}.\textsc{tr} girl\\
\glt `[I am afraid that] you give birth to a boy, [better] give birth to a girl!'

\ex \label{ex:5:191} % (191)
\gll \emph{ara} \emph{\textbf{d-uh-a-la}!}\\
 healthy \textbf{\textsc{f1}-become:\textsc{pfv}-\textsc{irr}-\textsc{appr}}\\
\glt `[I am afraid that] I become healthy!'
\z

Apprehensive predicates are regularly used in the complement clauses of
verbs of fear followed by the complementizer \emph{ile} (which is
the perfective converb of the verb `say'):

\ea % (192)
\gll \emph{nu} \emph{uruχ} \emph{k'-uwe} \emph{le-w-ra} \emph{žanawal-li-ni}  \emph{maza} \emph{\textbf{ar-b-uk-a-la}} \emph{ile.}\\
 I be.afraid \textsc{lv}:\textsc{ipfv}-\textsc{cvb.ipfv} be-\textsc{m}-\textsc{ego} wolf-\textsc{obl}-\textsc{erg}  sheep \textbf{away-\textsc{n}-lead:\textsc{pfv}-\textsc{irr}-\textsc{appr}} \textsc{comp} \\
\glt `I am afraid that the wolf steals a sheep.'

\ex % (193)
\gll \emph{nu} \emph{uruχ} \emph{k'-as} \emph{ħu} \emph{iz-es} \emph{\textbf{d-aʔ-a-la}} \emph{ile.}\\
 I be.afraid \textsc{lv}:\textsc{ipfv}-\textsc{hab}.\textsc{ego} you.sg be.ill:\textsc{ipfv}-\textsc{inf} \textbf{\textsc{f1}-begin:\textsc{pfv}-\textsc{irr}-\textsc{appr}} \textsc{comp}\\
\glt `I am afraid that you might fall ill.'
\z

If the subject of the apprehensive complement clause is coreferent to
the subject of the main clause, the logophoric pronoun \emph{sa}‹\textsc{cl}›\emph{i} is
used (see \citealt{kozhukhar2019} [this volume]). This is a phenomenon common to other
cases of subordination with the complementizer \emph{ile}.

\ea % (194)
\gll \emph{baba} \emph{uruχ} \emph{k'-uwe} \emph{le-r} \emph{χʷe} \emph{q'ac'} \emph{\textbf{b-ik-a-la}} \emph{ile.}\\
 granny be.afraid \textsc{lv}:\textsc{ipfv}-\textsc{cvb.ipfv} \textsc{aux}-\textsc{f} dog  bite \textbf{\textsc{n}-\textsc{lv}:\textsc{pfv}-\textsc{irr}-\textsc{appr}} \textsc{comp} \\
\glt `My grandmother is afraid that the dog bites her.'

\ex % (195)
\gll \emph{baba} \emph{uruχ} \emph{k'-uwe} \emph{le-r,} \emph{sa‹r›i} \emph{\textbf{ar-d-ik-a-la}} \emph{ile.}\\
 granny be.afraid \textsc{lv}:\textsc{ipfv}-\textsc{cvb.ipfv} \textsc{aux}-\textsc{f} self‹\textsc{f}› \textbf{\textsc{pv}-\textsc{f1}-fall:\textsc{pfv}-\textsc{irr}-\textsc{appr}} \textsc{comp}\\
\glt `The grandmother is afraid of falling down.'
\z

Apprehensives cannot refer to a situation in the past. The next example
is ungrammatical (\ref{ex:5:196}), and has to be modified as in (\ref{ex:5:197}).

\pagebreak

\ea \label{ex:5:196} % (196)
\gll \emph{*nu} \emph{uruχ} \emph{k'-as} \emph{dag} \emph{anwal-li-če} \emph{χʷe} \emph{q'ac'} \emph{\textbf{*b-ik-a-la}.}\\
 I be.afraid \textsc{lv}:\textsc{ipfv}-\textsc{hab}.\textsc{ego} yesterday Anwar-\textsc{obl}-\textsc{super}(\textsc{lat}) dog bite \textbf{\textsc{n}-\textsc{lv}:\textsc{pfv}-\textsc{irr}-\textsc{appr}} \\
\glt 
Intended: `I am afraid that the dog bit Anwar yesterday.'

\ex \label{ex:5:197} % (197)
\gll \emph{nu} \emph{uruχ} \emph{k'-as} \emph{dag-ʔʷanal} \emph{anwal-li-če} \emph{χʷe} \emph{q'ac'} \emph{\textbf{b-ik-a-la}} \emph{ile.}\\
 I be.afraid \textsc{lv}:\textsc{ipfv}-\textsc{hab}.\textsc{ego} yesterday-like Anwar-\textsc{obl}-\textsc{super}(\textsc{lat}) dog bite \textbf{\textsc{n}-happen:\textsc{pfv}-\textsc{irr}-\textsc{appr}} \textsc{comp} \\
\glt `I am afraid that the dog might bite Anwar as it happened yesterday.'
\z

The clause with the apprehensive and complementizer can be inserted into
the main clause:

\ea % (198)
\gll \emph{musa-ni}  \emph{mura} \emph{d-arʔ-ib} [\emph{dunijal} \emph{\textbf{ur-a-la}} \emph{ile}].\\
 Musa-\textsc{erg} hay \textsc{npl}-gather:\textsc{pfv}-\textsc{aor} world \textbf{rain-\textsc{irr}-\textsc{appr}} \textsc{comp}\\
\glt `Musa collected the hay out of fear that rain starts.'

\ex % (199)
\gll \emph{musa-ni}  [\emph{dunijal} \emph{\textbf{ur-a-la}} \emph{ile}] \emph{mura} \emph{d-arʔ-ib.}\\
 Musa-\textsc{erg} world \textbf{rain-\textsc{irr}-\textsc{appr}} \textsc{comp} hay \textsc{pl}-gather:\textsc{pfv}-\textsc{aor}\\
\glt `Musa collected the hay out of fear that rain starts.'
\z

The apprehensive construction without the complementizer cannot be
inserted into the main clause:

\ea % (200)
\gll \emph{eli} \emph{šula-le} \emph{b-uc-a} [\emph{ʁadara} \emph{b-oˤrʡ-aq-a-la}].\\
 child tight-\textsc{advz} \textsc{n}-hold:\textsc{pfv}-\textsc{imp}.\textsc{tr} dish \textsc{n}-break:\textsc{pfv}-\textsc{caus}-\textsc{irr}-\textsc{appr}\\
\glt `Hold the child tight, it might break the dish.'

\ex % (201)
\gll \emph{*eli} [\emph{ʁadara} \emph{b-oˤrʡ-aq-a-la}] \emph{šula-le} \emph{b-uc-a.}\\
 child dish \textsc{n}-break-\textsc{caus}-\textsc{irr}-\textsc{appr} tight-\textsc{advz} \textsc{n}-hold:\textsc{pfv}-\textsc{imp}.\textsc{tr}\\
\glt Intended: `Hold the child tight so that it does not break the dish.'

\ex % (202)
\gll \emph{sumka} \emph{b-uχ-a} \emph{mataħ} \emph{\textbf{ar-d-uʔ-a-la}.}\\
 bag \textsc{n}-bring:\textsc{pfv}-\textsc{imp}.\textsc{tr} money \textbf{\textsc{pv}-\textsc{npl}-lose:\textsc{pfv}-\textsc{irr}-\textsc{appr}}\\
\glt `Take the bag not to lose the money.'

\ex % (203)
\gll \emph{*sumka} [\emph{mataħ} \emph{\textbf{ar-d-uʔ-a-la}}] \emph{b-ux-a.}\\
 bag money \textbf{\textsc{pv}-\textsc{npl}-lose:\textsc{pfv}-\textsc{irr}-\textsc{appr}} \textsc{n}-bring:\textsc{pfv}-\textsc{imp}.\textsc{tr}\\
\glt Intended: `Take the bag not to lose the money.'
\z

Apprehensive is used to express negative purpose:

\ea % (204)
\gll \emph{w-aˤld-e} \emph{adaj-ni} \emph{ħu} \emph{dam} \emph{\textbf{w-aq'-a-la}.}\\
 \textsc{m}-hide:\textsc{pfv}-\textsc{imp} father-\textsc{erg} you.sg(\textsc{nom}) beat \textbf{\textsc{m}-do:\textsc{pfv}}\textbf{-\textsc{irr}-\textsc{appr}}\\
\glt `Hide, so that your father does not beat you.'

\ex % (205)
\gll \emph{c'a-li-če} \emph{ħule} \emph{w-iz-e,} \emph{\textbf{b-uš-a-la}.}\\
 fire-\textsc{obl}-\textsc{super}(\textsc{lat}) look \textsc{m}-\textsc{lv}:\textsc{pfv}-\textsc{imp} \textbf{\textsc{n}-die(of.fire):\textsc{pfv}-\textsc{irr}-\textsc{appr}}\\
\glt `Watch the fire so that it does not go out.'
\z

The purpose converb in \emph{-alis} is also used to express negative
purpose. Unlike apprehensive, negation in the purpose converb is overtly
marked by prefix \emph{ħa-}:

\ea % (206)
\gll \emph{w-aˤld-e} \emph{adaj-ni} \emph{ħu} \emph{dam} \emph{\textbf{ħa-q'-a-lis}.}\\
 \textsc{m}-hide:\textsc{pfv}-\textsc{imp} father-\textsc{erg} you.sg(\textsc{nom}) beat \textbf{\textsc{neg}-\textsc{m}.do:\textsc{pfv}-\textsc{irr}-\textsc{purp}}\\
\glt `Hide, so that your father does not beat you.'

\ex % (207)
\gll \emph{c'a-li-če} \emph{ħule} \emph{w-iz-e} \emph{\textbf{ħa-b-uš-a-lis}.}\\
 fire-\textsc{obl}-\textsc{super}(\textsc{lat}) look \textsc{m}-\textsc{lv}:\textsc{pfv}-\textsc{imp} \textbf{\textsc{neg}-\textsc{n}-die(of.fire):\textsc{pfv}-\textsc{irr}-\textsc{purp}}\\
\glt `Watch the fire so that it does not go out.'
\z

As some other verbal forms, apprehensives can be part of constructions
with topicalizing reduplication.

\ea % (208)
\gll \emph{it} \emph{\textbf{w-erχʷ}} \emph{\textbf{ħa-rχʷ-a-la}} \emph{nu} \emph{le-l-la} \emph{uruχ} \emph{k'-uwe.}\\
 this \textbf{\textsc{m}-enter:\textsc{pfv}} \textbf{\textsc{neg}-\textsc{m}.enter:\textsc{pfv}-\textsc{irr}-\textsc{appr}} I \textsc{aux}-\textsc{f}-\textsc{ego} be.afraid \textsc{lv}:\textsc{ipfv}-\textsc{cvb.ipfv} \\
\glt `I worry that he may not enter [the university].'
\z

\removelastskip
\is{apprehensive|)}

% 10.
\section{Discussion}\label{discussion}

In this section, I will compare the Mehweb system of non-indicative
forms with that of several other Dargwa lects (languages or dialects):
Akusha, Ashty, Shiri, Tanti, and Icari. Akusha is especially interesting
for this study, because it is suggested that Mehwebs came to the place
where they now live from the areas where the Akusha dialect is spoken
(\citealt{dobrushina2019a} [this volume]). If this hypothesis is true, we might expect
that Mehweb will show more similarity with Akusha than with other Dargwa
lects. Another object for the comparison is Avar~– the language which
is spoken in the vicinity and which could have influenced Mehweb.

The main prominent feature of Mehweb is the absence of personal endings
in all non-indicative forms. In this respect, Mehweb is presumably
unique among Dargwa languages and dialects. Akusha, Tanti, Shiri, Ashty,
Icari – all distinguish persons in the forms of optative and in
conditional forms. The loss of personal endings may be due to the
influence of Avar, since the latter has no personal paradigm.

Some traces of the former personal endings are still present in the
grammar of non-indicative mood forms. The Mehweb prohibitive ends in
\emph{-ad\(i\)}. In Akusha Dargwa, \emph{-ad} of prohibitive coincides
with the second person Future marker \citep[36]{vandenberg2001}. Shiri,
Ashty and Icari use the endings \emph{-t}/\emph{-t}: in the prohibitive, which are
the markers of the second person in some other forms of these lects
(\citealt{belyaev:shiri} manuscript; \citealt{mutalov-sumbatova2003}).
Mehweb, however, has the
marker \emph{-ad\(i\)} only in the prohibitive, hence synchronically it
does not denote person. Sumbatova suggested that the Mehweb prohibitive
marker originates from the second person ending \citep[590]{sumbatova-lander2014}.

In other respects, however, the Mehweb prohibitive is similar to that of
the other Dargwa lects: it is formed by a special negative prefix
\emph{ma-} (used only for the prohibitive and the negative optative) and
the suffix \emph{-ad\(i\)}. In Avar, the prohibitive is expressed
by a suffix.

There are several more features which distinguish Mehweb non-indicative
mood forms from what is typical for Dargwa lects.

The system of imperative marking is simpler in Mehweb than in other
Dargwa dialects. In Akusha, Tanti, Ashty, Shiri, and Icari, the choice
of the imperative marker is triggered by three factors: transitivity,
aspect and the formal class of the verb. In Mehweb, the formal class is
irrelevant for the choice of the imperative marker. The only relevant
factors are transitivity and aspect.

It is interesting that the marker of the imperative itself is formally
identical to that of Tanti but not to that of Akusha (which is supposed
to be closest to Mehweb). In Akusha, Ashty, Shiri and Icari, the marker
for perfective transitive imperatives is \emph{-a}, other types of
imperative have \emph{-i} or some other marker depending on the class
of verb (\citealt[48]{vandenberg2001}; \citealt{belyaev:ocherk} \& \citeyear{belyaev:shiri} manuscripts; \citealt{mutalov-sumbatova2003}). In Mehweb, the second class of imperatives takes \emph{-e}, like
the Tanti dialect \citep[142]{sumbatova-lander2014}. The marker \emph{-e}
in Mehweb could have been supported by the imperative of Avar, but the
distribution of Avar markers is opposite to that of Mehweb: \emph{-e}
for transitive imperatives, \emph{-a} for intransitive \citep[105]{charachidze1981}.

Mehweb differs from other Dargwa idioms in using the marker \emph{-na}
for the plural imperative and prohibitive. Akusha, Ashty, Shiri, Tanti,
and Icari also mark the plurality of the addressee by a special ending,
but in these dialects this marker is identical to the marker of the
second person plural in other forms. The Mehweb imperative/prohibitive
plural marker differs from other Dargwa lects even formally. In Mehweb,
the plural imperative/prohibitive is \emph{-na}; compare to \mbox{\emph{-ja}}/\mbox{\emph{-aja}} in Akusha \citep[48]{vandenberg2001}, \emph{-a}: in Ashty (\citealt{belyaev:ocherk}
manuscript), \mbox{\emph{-aja}} in Shiri (\citealt{belyaev:shiri} manuscript), \mbox{\emph{-a}}/\mbox{\emph{-ja}}
in Tanti \citep[142]{sumbatova-lander2014}, \emph{-aja} in Icari \citep{mutalov-sumbatova2003}.
Note that Avar has no special endings for the second person
plural imperative. For the moment, I have no suggestions as to the
origin of the marker \emph{-na}.

Unusual for Dargwa idioms are also Mehweb conditional markers. In
Akusha, Ashty, Shiri, Tanti, and Icari, conditional forms are marked by
the suffix \emph{-li} or \emph{-le}. Counterfactual conditionals in
all these lects are derived from hypothetical conditionals with the
marker of the past tense. Mehweb conditionals differ both in terms of
content and in terms of structure. Mehweb conditionals have other
markers than these Dargwa dialects (\emph{-k'a} for hypothetical
conditional converb and \emph{-q'alle} for counterfactual; see
\sectref{conditional-converbs} on the probable origin of these markers). The counterfactual form is
not formally related to the hypothetical. It seems therefore that the
proto-Dargwa conditonal forms were completely substituted in Mehweb by
new forms.


The optative of Mehweb has the same marker \emph{-b} as other Dargwa
lects. Another similarity is the presence of truncated optative forms in
Mehweb as well as in Akusha, Ashty, Shiri and Tanti (see references in \sectref{morphology-of-the-optative}).
The difference from other Dargwa lects is that the Mehweb optative
has one form for all persons, as I mentioned before.
Mehweb is also
% Another way where the Mehweb system is
simpler than the related idioms is that it does not
use the optative for commands which have first person object, as do
Tanti, Shiri, Ashty, and Icari (I have no information about this
construction in Akusha).
{\looseness1\par}

As most other Dargwa dialects, Mehweb lacks a dedicated form for the
hortative. The meaning of the hortative is regularly expressed by the
combination of the particle based on the imperative of `go' and the
infinitive. Unfortunately, there is no sufficient information on how the
hortative is expressed in Akusha, Ashty, Shiri, Tanti, and Icari.

As for the jussive, Mehweb uses a periphrastic construction to express
it. The combination of the imperative of the verb with the imperative of
the verb of speech (lit.~`verb-imp say') is found in several East
Caucasian languages (Akhvakh (\citeauthor{creissels:optative} manuscript), Lak and Archi
\citep{dobrushina2012}), but not among the Dargwa lects discussed above.

Apprehensives seem to be rare in East Caucasian (as well as in other
languages of the world). To my knowledge, apart from Mehweb, the
apprehensive is attested only in Archi \citep{kibrik1977}. These forms
however are rarely looked for by linguists, so the reason for the
infrequency of these forms can as well be their undocumentedness.

% 11.
\section{Conclusion}

As this study has shown, there are several features which show the
special position of Mehweb among other Dargwa lects. In several cases,
Mehweb differs from the other five lects used for comparison, while those
five show affinity between them. The study of non-indicative moods did
not show any special similarity between Mehweb and Akusha. The influence
of Avar, however, is also not attested in these forms. The only feature
of the Mehweb system of non-indicative moods which can result from intensive
contact with other languages is that, in several respects, it is simpler
than the system of other Dargwa lects.

\section*{List of abbreviations}

\begin{longtable}[l]{@{}ll@{}}
\textsc{1sg}	& first person singular \\
\textsc{ad}	& spatial domain near the landmark \\
\textsc{add}	& additive particle \\
\textsc{advz}	& adverbializer \\
\textsc{aor}	& aorist \\
\textsc{appr}	& apprehensive \\
\textsc{atr}	& attributivizer \\
\textsc{aux}	& auxiliary \\
\textsc{card}	& cardinal numeral \\
\textsc{caus}	& causative \\
\textsc{cl}	& gender (class) agreement slot \\
\textsc{comit}	& comitative \\
\textsc{comp}	& complementizer \\
\textsc{cond}	& conditional \\
\textsc{cop}	& copula \\
\textsc{ctrf}	& counterfactual \\
\textsc{cvb}	& converb \\
\textsc{dat}	& dative \\
\textsc{ego}	& egophoric \\
\textsc{el}	& motion from a spatial domain \\
\textsc{emph}	& emphasis (particle) \\
\textsc{erg}	& ergative \\
\textsc{ess}	& static location in a spatial domain \\
\textsc{f}	& feminine (gender agreement) \\
\textsc{f1}	& feminine (unmarried and young women gender prefix) \\
\textsc{fut}	& future \\
\textsc{gen}	& genitive \\
\textsc{hab}	& habitual (durative for verbs denoting states) \\
\textsc{hpl}	& human plural (gender agreement) \\
\textsc{imp}	& imperative \\
\textsc{in}	& spatial domain inside a (hollow) landmark \\
\textsc{indef}	& indefinite particle \\
\textsc{inf}	& infinitive \\
\textsc{inter}	& spatial domain between multiple landmarks \\
\textsc{intj}	& interjection \\
\textsc{ipft}	& imperfect \\
\textsc{ipfv}	& imperfective (derivational base) \\
\textsc{irr}	& irrealis (derivational base) \\
\textsc{lat}	& motion into a spatial domain \\
\textsc{lv}	& light verb \\
\textsc{m}	& masculine (gender agreement) \\
\textsc{n}	& neuter (gender agreement) \\
\textsc{neg}	& negation (verbal prefix) \\
\textsc{negvol}	& negation in volitional forms (negative imperative, negative optative) \\
\textsc{nmlz}	& nominalizer \\
\textsc{nom}	& nominative \\
\textsc{npl}	& non-human plural (gender agreement) \\
\textsc{obl}	& oblique (nominal stem suffix) \\
\textsc{opt}	& optative \\
\textsc{ord}	& ordinal numeral \\
\textsc{pfv}	& perfective (derivational base) \\
\textsc{pl}	& plural \\
\textsc{proh}	& prohibitive \\
\textsc{prs}	& present \\
\textsc{pst}	& past \\
\textsc{ptcl}	& particle \\
\textsc{ptcp}	& participle \\
\textsc{purp}	& purposive converb \\
\textsc{pv}	& preverb (verbal prefix) \\
\textsc{q}	& question (interrogative particle) \\
\textsc{quot}	& quotative (particle) \\
\textsc{super}	& spatial domain on the horizontal surface of the landmark \\
\textsc{tr}	& transitive \\
\textsc{trans}	& motion through a spatial domain \\
\end{longtable}

\printbibliography[heading=subbibliography,notkeyword=this]

\end{document}


%%% Local Variables:
%%% mode: latex
%%% TeX-master: "../main"
%%% End:
