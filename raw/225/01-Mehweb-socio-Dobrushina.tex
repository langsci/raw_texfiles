\documentclass[output=paper]{langsci/langscibook}
\ChapterDOI{10.5281/zenodo.3402054}

% Chapter 1
 
\title{The language and people of Mehweb}

\author{Nina Dobrushina\affiliation{National Research University Higher School of
  Economics, Linguistic Convergence Laboratory, ndobrushina@hse.ru}}


\abstract{This paper describes the sociolinguistic~situation of Mehweb, a lect of the Dargwa branch of East Caucasian, spoken in the Republic of Daghestan. In the course of several field trips to the village of Mehweb (officially, Megeb), sociolinguistic interviews were conducted in Mehweb and four neighbouring Avar- and Lak-speaking villages. The paper describes the demographic situation in Mehweb, the villagers' official status, their social and economic life in the past and at present. The multilingual repertoire of Mehwebs and their neighbours is described in both qualitative and quantitative terms. I conclude that, while there are no signs of language loss, the traditional patterns of multilingualism in Mehweb are highly endangered.}

\begin{document}
\maketitle


\section{% 1. 
Introduction}

Mehweb belongs to the Dargwa group of the East Caucasian
(Nakh-\hskip0pt Daghestanian) language family. It is sometimes considered a
dialect of Dargwa \citep{magometov1982}, but more often, it is treated as a
separate language \citep{khajdakov1985,koryakov-sumbatova2007}. Mehweb
is spoken in a single village called Mehweb\footnote{Russian
  \emph{Мегеб} – [megeb], the native term is [mehʷe], while
  [mehʷeb] is the Avar spelling which includes the final \emph{-b}
  of the locative form.} and geographically separated from all
other Dargwa languages. While Dargwa languages generally constitute a
continuous area, Mehweb is surrounded by speakers of Avar and Lak, which
are languages of other branches of the family.

The village of Mehweb is located in Gunibskij region, in the central
part of Daghestan at 1800 meters above sea level. The total number of
speakers is estimated to be between 800 and 900, 600 to 700 of whom
live in Mehweb itself. About 100 to 200 live in the so called
\emph{kutan}\footnote{Originally, kutans were territories for lowland
  herding in the winter. At the present time, people often prefer to
  stay in these lowland settlements for the whole year, thus
  establishing new villages.} \emph{kolkhoza imeni Gadzhieva} (located
350 km away from Mehweb and four kilometres away from the sea coast, near
the village Krainovka). Kutan was not examined from either a linguistic
or sociolinguistic point of view. All data in this paper come only from
Mehweb. There are also Mehweb families in Makhachkala, Kizlyar and
Bujnaksk, and a few elsewhere. All Mehweb-speaking families originate
from the village Mehweb.

Like most Daghestanians, Mehwebs are Muslim. Mehweb has no \isi{literacy} tradition. 
The Mehwebs write in \ili{Avar} or \ili{Russian}.
We have no evidence that Mehweb was ever written in the Arabic or
Cyrillic script in the observable past. At least, the residents of
Mehweb could not recall any manuscripts in Mehweb (unlike some other
minority languages of Daghestan – see \citealt{magomedkhanov2009} about the
Archi manuscript).

So far, there are no indications of \isi{language loss} in Mehweb. All
villagers speak Mehweb, and Mehweb is the first language acquired by
children.

The Mehwebs often suggest that their idiom is more conservative than
other Dargwa lects and contains some archaic features. This opinion is
also expressed in some descriptions of Mehweb \citep{magometov1982,khajdakov1985}.
Recent studies on Dargwa languages show that at least some
phenomena (such as various properties of agreement) are innovative in
Mehweb compared to other Dargwa lects \citep{sumbatova-lander2014}.

The command of \ili{Russian}, \ili{Avar}, and \ili{Lak} is spread in Mehweb (see \sectref{neighbours-and-language-contact}
for details). The proficiency in standard Dargwa is infrequent. In
\sectref{mehweb-officially}, the official status of the Mehweb language is discussed.
\sectref{the-past-of-mehweb} and \sectref{the-present-of-mehweb} briefly describe social and economic life of the
village in the past and at the present time. \sectref{neighbours-and-language-contact} is devoted to the
multilingual repertoires of Mehwebs and the neighbouring villages. A
brief conclusion summarizes the paper.

% 2. 
\section{Mehweb officially}\label{mehweb-officially}

Mehweb is located in a district where Avars are numerically dominant. As
a result, Mehwebs are in some respects considered to be Avars \citep[98]{tishkov-kisriev2007}.

Firstly, paradoxically, they are taught Avar at \isi{school} during lessons
called \emph{native language} (Russian \emph{родной язык}, lit.\ `native
language'), even though Avar belongs to another group of East Caucasian
and is genealogically distant from Mehweb. Mehweb children begin
learning two foreign languages in the first grade (at 6–7 years
old) – \ili{Avar} and \ili{Russian}, which, according to their parents, is not
easy for them. Another result of learning only Avar at school is
that Mehwebs are not acquainted with standard Dargwa, unlike most people
who speak other lects of Dargwa.

Secondly, most Mehwebs are registered as Avars\il{Avar} in their passports. That
continued until the 1990s, when the obligatory indication of \isi{ethnicity}
in passports was cancelled in Russia. The villagers explain that those
Mehwebs who got their passports at the village council were registered
as Avars, while those who got their passports in the cities were
registered as Dargis.

In the 2002 and 2010 censuses of the Russian Federation, Mehwebs were not
mentioned. Residents of Mehweb were registered as Dargis or as Avars. In
2002, 747 Dargis and 98 Avars were reported as residents of Mehweb. In
2010 the numbers were 572 Dargis and 124 Avars. The difference between
the data of the two censuses has no reasonable explanation. Mehweb is
very homogenous both ethnically and linguistically, as are most villages
of highland Daghestan. There are no outsiders in the village except for
several Avar women taken as wives. Most probably, the ethnic population
of Mehweb has not changed in at least the last hundred years, and the
census information does not reflect the true ethnic structure of Mehweb
in any way.

According to interviews with the villagers, Mehweb residents identify
themselves as Dargis. They are well aware of the closeness of their
language to Dargwa, and have regular contacts with the Dargwa people from
the village Mugi (see \sectref{the-past-of-mehweb}).

Data from the censuses\is{census} on native language are again controversial. The
Mehweb language is not mentioned. It follows from the 2002 census that
792 residents indicated Dargwa as their first language, while 53 indicated \ili{Avar}.
According to the 2007 census, this has changed: 566 indicated Dargwa as
their first language, and 113, Avar. The mention of Dargwa as a first language
is most likely because Mehweb is usually considered a variation of Dargwa, and
therefore the residents of Mehweb may have referred to their native
language as Dargwa. But there are no reasonable explanations for the
mention of Avar as a first language: there is no one in Mehweb who
speaks Avar as a first language, apart from the two or three women who
married in.

Mehwebs are not officially recognised as an ethnic group, nor is Mehweb
officially recognised as a language.

% 3. 
\section{The past of Mehweb}\label{the-past-of-mehweb}

There is a common belief that the village of Mehweb was founded by
re-settlers from the Dargwa-speaking village of Mugi \citep{uslar1892}. Mugi
is located in the Akushinskij district (in the central part of Daghestan, about 70 km from Mehweb; it takes two to three hours by car).
As far as I know, there is no tangible historical evidence for the
connections between Mehweb and Mugi, apart from oral testimony. Mehwebs
are convinced that Mugi is their ancestral homeland, and have several
versions of how they left it. One of the local stories reports that
there was an isolated part of Mugi which was in the way of Timur's
(Tamerlane's) army. When they realized they could not resist
Tamerlane, the residents fled and settled higher in the mountains.
According to this version, Mehweb was founded in the 14th century.
\citet[101]{khajdakov1985} dates the migration of Mehwebs to somewhere between the 8th and 
9th centuries, reporting the opinion of a respected Mehweb resident.
An early report by Komarov says that the Mehweb people are (descendants of)
refugees from another village, but \isi{Mugi} is not mentioned
\citep{komarov1868}\footnote{\textrussian{«Недалеко от Чоха есть большое селение Меге, по преданию,
  основанное даргинцами, в разное время искавшими спасения от
  кровомщения».}}.

According to lexicostatistical analysis, Mehweb belongs to the
Northern-\hskip0pt central group of Dargwa languages, and is closer to
Murego-Gubden lects than to the dialect of Mugi \citep{koryakov2013}.

Although it is not clear if this view on the origin of Mehwebs has
historical grounds, the residents of Mehweb and Mugi are quite positive.
They have established intensive contacts: they practice reciprocal group
visits, and invite each other to important festivities. Most of the
Mehwebs I had spoken to said they did not understand the dialect of Mugi
and preferred communicating with the Mugis in Russian.

The relations of the Mehwebs with Avars were much more intensive. The
main road to Mehweb was through a big Avar village, Chokh, and through
another, smaller Avar village, Obokh. In the 19th century, Mehweb was a
part of the so-called Andalal free association which mainly consisted of
Avar villages. After the revolution of 1917, Mehweb became a part of the
Charoda district. In 1928, it was transferred to the Gunib district.
Both districts are dominated by Avars. Between 1929 and 1934, it was
transferred to the Lak district, and then was re-transferred to Gunib.
Therefore, from the administrative point of view, the
Mehwebs were mostly connected with Avars.

\pagebreak

Avars were, and still are, the closest neighbours of the Mehwebs – it
takes about 40 minutes to walk to Obokh. Although the more distant Lak
neighbours were also important for Mehweb, because the Mehwebs would regularly
go to the Kumukh market where the communication was in Lak. The distance is about 
15 kilometres from Mehweb to \isi{Kumukh}, taking four to five
hours to get there by foot. Some women would go there every
Thursday. Visits to the \isi{market} in Kumukh gradually became less frequent
after the 1950–60s.

Mehweb was and still is one of the biggest villages in the
neighbourhood. 
%According to Komarov, in 1886 there were 727 residents \cite{komarov1868}.
This number has remained stable over the 20th century: 710 in 1926, 780
in 2007.

The main occupation of Mehwebs was breeding sheep and cattle. They also
grew corn and potatoes. The specialty of Mehweb was cultivating black peas which
usually yielded a good harvest. There were no fruit trees before the 1950s,
although at the present moment Mehwebs grow apples, pears and apricots.
Mehwebs were neither rich nor poor in comparison to other settlements of
highland Daghestan.

As Mehwebs had enough corn, they did not need to look for jobs outside
the village. According to the recollections of local people, seasonal
employment outside the village was not customary. Only a few Mehweb
people are reported to have practiced tinsmithing, like their Lak
neighbours. We were also told by the residents of the neighbouring
village of Shangoda that Mehwebs were good stone masons and builders,
and were invited to other villages. Another reason for inter-ethnic
contact was shepherding on remote pastures (transhumance), which
resulted in irregular contact with Avars and Kumyks. In general, the Mehweb
people did not migrate a lot.

Mehweb people rarely married out. As in all of Daghestan \citep{comrie2008,
wixman1980}, a marriage partner from Mehweb was preferable. Often the
spouse was chosen from the same patrilineal clan (\emph{tukhum}). In the
infrequent cases of mixed marriages the wife was taken from one of the
neighbouring Avar villages. The tradition of endogamic marriages\is{endogamy} started
to die away only in the beginning of the 21th century.

% 4. 
\section{The present of Mehweb}\label{the-present-of-mehweb}

Today, Mehweb has between 600 and 700 residents. The population has not
decreased as much as in many other neighbouring villages. For example, the \ili{Avar}
villages \isi{Obokh} and \isi{Shangoda} were twice as populated in the past. The \ili{Lak}
villages \isi{Mukar} and \isi{Uri} are on the verge of complete abandonment. However, several
families still live in the \ili{Lak} villages \isi{Palisma} and \isi{Kamakhal}, which were
recently among the biggest in the neighbourhood. The \ili{Avar} village of
\isi{Shitlib} (Shitli) has been abandoned. The only village in the
neighbourhood which did not lose a significant part of its population,
apart from Mehweb, is the Avar village \isi{Bukhty}. Mehweb is the biggest and
the most vital village in the vicinity, with a large \isi{school} and a sizeable population of 
children. Still, the locals report a slight
population decrease: the school had more pupils in the 1980s than now.

Apart from the regular \isi{school}, Mehweb has a special boarding school for
training boys in freestyle wrestling. There are usually about 10–15 boys
from other parts of Daghestan who live in Mehweb and study with local
children. These boys have different native languages (most often Avar)
and communicate with the locals in Russian.
%
There is a \isi{kindergarten} where local teachers communicate with the
children in Mehweb and in Russian.  The village boasts a large
social centre. It has a billiard room and, on occasions, hosts concerts
and dances. A small medical centre employs three nurses.

As elsewhere in Daghestan, the Mehwebs complain about local
unemployment. Those who are not employed at the school, kindergarten,
social centre or the medical centre, can make their living only by going
away for construction jobs, or by selling meat and cheese. There are
also several small shops run by local families.

People in Mehweb, as in all other villages in the neighbourhood, have
had TV since the 1980s. Regular access to broadcasts became possible
from the 1990s when a transmission tower was constructed in Sogratl. The broadcasts are mainly in Russian. Apparently this has influenced the
level of bilingualism in Russian.

The Mehwebs take pride in the fact that several of its residents
distinguished themselves during WW2. Two men were decorated as Hero of
the Soviet Union for their military service during the war. Mehweb has a
war memorial, and Victory Day (May 9) is also of special importance to the
village.

% 5. 
\section{Neighbours and language contact}\label{neighbours-and-language-contact}

The level of \isi{multilingualism} was studied in Mehweb and in four
neighbouring villages: Obokh and Shangoda (\ili{Avar}) and Uri and Mukar (\ili{Lak})
– see \figref{Figure1}. During fieldtrips in 2012–2015, a series of
sociolinguistic surveys was conducted to study the multilingual
repertoires of the residents\footnote{Sociolinguistic study of
  multilingualism in Mehweb and neighbouring villages is a part of a
  larger project documenting patterns of multilingualism in Daghestan
  (\url{https://multidagestan.com/}).}.

\begin{figure}
\includegraphics[width=\textwidth]{Figure3.png}

% Figure 1. 
\caption{Mehweb and neighbouring villages (map courtesy of Yuri Koryakov)}
\label{Figure1}
\end{figure}

% 5.1. 
\subsection{Data and methodology}\label{data-and-method}

In order to obtain quantitative data about the command of other
languages in each of these villages, the method of retrospective family
interviews (introduced in \citealt{dobrushina2013}) was applied\footnote{I am
  deeply grateful to Darya Barylnikova, Ilya Chechuro, Michael Daniel,
  Violetta Ivanova, Aleksandra Khadzhijskaya, Marina Korshak, Aleksandra
  Kozhukhar, Marina Kustova, Yevgenij Mozhaev, Olga Shapovalova, Marija
  Sheyanova, Semen Sheshenin, Aleksandra Sheshenina who ran the
  interviews on multilingualism in Mehweb together with me.}. The
dynamics of multilingualism is accessed through, and based on, short interviews with speakers of different generations, thus resembling
apparent time studies\footnote{Apparent time studies of language change
  use surveys of people of different ages, with an assumption that the
  speaker's speech reflects the speech patterns acquired in the
  childhood.} \citep{bailey2013}. The important difference from the apparent
time method is that data are obtained not only about the respondents
themselves, but also about their deceased relatives.

The method aims at capturing multilingual repertoires of the speakers of
the recoverable past in order to reconstruct traditional (i.e.\
pre-Soviet) patterns of language contact. It was typical for highland
Daghestani to have large families where parents lived together with
their youngest son and communicated with other children on a daily
basis, looked after their grandchildren and helped to run the household.
The younger generation was usually well acquainted with their
grandparents. By asking 60 to 80 year old villagers about language
repertoires of their grandparents, the data collected sometimes dates
back to the end of the 19th century, and even to the mid-19th century.
\tabref{Table1} provides an example of a questionnaire completed for one person.

\begin{table}[ht]
\begin{tabularx}{\textwidth}{@{}p{.38\textwidth}X@{}}
\toprule
\emph{Questions} & {Answers}\tabularnewline \midrule 
\emph{name} & Amin\tabularnewline
\emph{year of birth} & 1908\tabularnewline
\emph{year of death} & 1985\tabularnewline
\emph{is a relative of} & father of Mohammad, father-in-law of
Mariam\tabularnewline
\emph{information was given by} & Mohammad (son of Amin)\tabularnewline
\emph{education and occupation} & studied in madrasah, was a shepherd, a
foremen in kolkhoz\tabularnewline
\emph{command of Quranian Arabic} & could read the Arabic script, but
did not understand the text\tabularnewline
\emph{Lak} & yes\tabularnewline
\emph{Avar} & yes\tabularnewline
\emph{Russian} & no\tabularnewline
\emph{other languages} & Akusha dialect of Dargwa\tabularnewline
\bottomrule
\end{tabularx}

\caption{Example of a filled in sociolinguistic questionnaire}
\label{Table1}
\end{table}

The choice of respondents was more or less random. The aim of the study
is to reconstruct the multilingualism of the past; so the eldest
possible respondents were preferred, and younger generations were
included for the sake of comparison. The controlled parameters of the
sample were thus the respondents' age and gender.

The shortcomings of this method include, first of all, the subjective
character of judgments about language proficiency. No test of
proficiency of the respondent was undertaken (and obviously no such test was
possible for his\pagebreak[4] or her late relatives). Estimations of the level of
bilingualism were based on the respondents' judgments. The second
shortcoming is the fact that the respondent's memories of e.g.\ his
mother and father were limited to their older period of life.
Third and probably most importantly, judgments may reflect stereotyped
notions about past multilingualism widespread in the village, rather
than being based on personal memories of individual linguistic repertoires.
For a further discussion, see \citet{dobrushina2013}.

Multilingualism\is{multilingualism} is a social behaviour developed through interaction.
Hence sociolinguistic surveys were conducted not only in the village of
Mehweb but also in four neighbouring villages. The data from
retrospective family interviews in neighbouring villages helped us to
better understand how the communication between neighbouring villages
was performed. Were both languages used for communication or was one of
them preferred? For example, if we only found that most Mehwebs spoke
Avar and Lak, we still would not know whether Avar and Lak neighbours of
Mehwebs could speak Mehweb or not, and therefore could not estimate the role of the
Mehweb language in the area.

The closest neighbours of Mehweb are the \ili{Avar} villages \isi{Obokh} and
\isi{Shangoda} (\figref{Figure1}).

Obokh villagers speak a dialect of Avar. In their opinion, this variety
differs from the dialects of other villages in the area. At school, the
Obokhs learn standard Avar. There is an opinion among them that their
village is the oldest in the neighbourhood. They support this idea by
the size of the cemetery. Another fact which might prove that Mehweb is
younger than Obokh is that Obokh possesses more land than Mehweb,
although the village itself is smaller.

Shangoda, another Avar village, is further away from Mehweb than Obokh.
The track goes up and down, and it takes about 90 minutes to reach
Shangoda. Slightly closer than Shangoda was the Avar village Shitlib,
which is now abandoned. After Shangoda, there are what were earlier the Lak villages
Palisma and Kamakhal (about 30 minutes walk from Shangoda). They are
also abandoned now. In the 19th century and at the beginning of the
20th century, Shangoda belonged to the Kazikumukh district, dominated by
Laks. It was connected to Kumukh by a mountain path. Until the 1930s,
when Shangoda was transferred to the Gunib district, the inhabitants of
Shangoda had their administrative centre in the village of Palisma.
Therefore, relations with Laks were more important for Shangoda than
relations with Mehwebs or with Avar villages.

Lak villages are further away from Mehweb than Obokh or Shangoda, but
the contacts with them were essential for Mehwebs because of their
regular visits to the Lak market in Kumukh. In Lak villages, the Mehweb
people had friends with whom they could stay on their way to the Kumukh
market.

All five villages (Mehweb, Obokh, Shangoda, Mukar, Uri) are located at
more or less the same height above sea level (1500–1800 meters). In the
observable past, the economic life and the standards of living in all these
villages were similar.

In Mehweb, the sociolinguistic survey was the most extensive. Our
database contains 240 entries, including 90 people who are deceased. The
databases for other villages have less entries: 80 in Shangoda, 80 in
Uri, 103 in Obokh, 110 in Mukar (note that these villages are presently
much less populated than Mehweb).

People were divided into two groups: those who were born before and those
who were born after 1919. The reason for establishing 1919 as a cut-off
point was that in the 1930s, Soviet schools were opened in all villages.
The teaching was done in Russian. The generation born after 1919
therefore usually had a secular education, often had some level of
literacy, had less opportunities to learn Arabic script (because of the
atheistic politics of the USSR), and most often spoke some Russian. The
generation born before 1919 was closer to what we consider traditional
patterns of \isi{multilingualism}, as will be shown in the next section.

% 5.2. 
\subsection{Multilingualism among the residents born before 1919}\label{multilingualism-among-the-residents-born-before-1919}

According to our study, Mehwebs communicated with the Avars and Laks in Avar
and Lak respectively. This follows from the level of mutual bilingualism
of the Mehwebs and their neighbours. Almost 100\% of Mehwebs born before
1919 spoke Avar and Lak (see \tabref{Table2}). Their neighbours from Avar and
Lak villages had no command of Mehweb at all. Only 8\% of the people
from Obokh, the closest Avar village, were reported to speak Mehweb
(\tabref{Table2}).

Mehwebs acquired \ili{Avar} through their communication with the neighbouring
Avar villages, Obokh and Shangoda, and bigger villages which were more
distant but important economically and socially, including Sogratl,
Chokh, and Gunib. There were no \ili{Lak} villages located as close as Obokh
and Shangoda to Mehweb and the main source of knowledge of Lak was the
\isi{market} in \isi{Kumukh}. The role of this market in the area was important
enough for the Mehwebs to acquire Lak.

Occasionally, Mehwebs also mentioned their command of Kumyk. Kumyk was
acquired by those who brought sheep to the lowlands where Kumyks lived.
This practice was apparently not very common – only 2–3\% of the people
born before 1919 spoke Kumyk.

About 45–50\% of the Mehwebs born before 1919 could read the
\isi{Quran}\footnote{See also \citet{kozhukhar-barylnikova2013} about the dynamics
  of literacy in Mehweb.}. Note that the reported ability to read does
not imply ability to understand Arabic, but only to recite the text. The
knowledge of Arabic was usually limited to the knowledge of the phonetic
meaning of letters. If a person was reported to be able to read Arabic,
the researchers asked more specific questions about the ability to
translate (understand) Arabic text. According to our study, only 6\% of
Mehwebs could understand and translate the Quran.

About 20\% of Mehwebs of this generation spoke \ili{Russian}. The command of
Russian was much more common among men who travelled in order to earn
money.

As for the residents of Avar villages, the knowledge of Lak was reported
significantly more often in \isi{Shangoda} (93\%), than in \isi{Obokh} (22\%). This
is not surprising. Lak villages were very close to Shangoda (30 minutes
walk), and the residents of Shangoda and the Lak villages were socially
and economically connected. For both Shangoda and Obokh, the market in
Kumukh was very important, but Kumukh was much closer to Shangoda. There
was a striking difference between Obokh and Mehweb. The villages were
almost at the same walking distance from Lak villages, but the
difference in the level of Lak is huge: 95\% in Mehweb and 22\% in
Obokh. There is only one plausible explanation for this discrepancy.
Mehwebs as speakers of a minor language were disposed to speaking other
languages, while Avars, being the majority in the district, were in
general oriented to use their own language in all circumstances.

The residents of \ili{Lak} villages also had some command of \ili{Avar}, but the
level of their bilingualism was lower than in Avar villages (\tabref{Table2}).

\begin{table}[hb]
\begin{tabular}{@{}lccccc@{}}
\toprule
& {Mehweb} & {Avar} & {Lak} & {Russian}\tabularnewline \midrule
{Mehweb} & native & 97\% & 95\% & 21\%\tabularnewline
{Obokh} & 8\% & native & 22\% & 22\%\tabularnewline
{Shangoda} & 0\% & native & 93\% & 50\%\tabularnewline
{Uri} & 0\% & 78\% & native & 40\%\tabularnewline
{Mukar} & 0\% & 40\% & native & 50\%\tabularnewline
\bottomrule
\end{tabular}

\caption{The level of multilingualism in five villages: generations born
before 1919}
\label{Table2}
\end{table}

Mehwebs were the most multilingual\is{multilingualism} people of the villages in the area.
The language contact between Mehwebs and their neighbours was
asymmetrical\is{multilingualism, asymmetrical}. They spoke the languages of their neighbours, while the
neighbours did not speak Mehweb. Presumably, Mehweb was never used as a
second language (we cannot be positive about this because we have no information
about the more distant past). The reason for this asymmetry in the
linguistic relations between neighbours was obviously the fact that
Mehweb was spoken only in one village and had no importance at the
supralocal level.


% 5.3. 
\subsection{Multilingualism among the residents born after
1920}\label{multilingualism-among-the-residents-born-after-1920}

In the second half of the 20th century, knowledge of local languages
decreased, while knowledge of Russian increased significantly. People in
Mehweb and Obokh spoke virtually no \ili{Lak} (\tabref{Table3}). In Shangoda, the
command of Lak persisted longer, but it was almost lost in the
generation born after 1960. The command of \ili{Avar} in Lak villages Uri and
Mukar was also practically lost.

\begin{table}[ht]
\begin{tabular}{@{}lcccc@{}}
\toprule
& {Mehweb} & {Avar} & {Lak} & {Russian}\tabularnewline \midrule
{Mehweb} & native & 85\% & 17\% & 91\%\tabularnewline
{Obokh} & 4\% & native & 6\% & 83\%\tabularnewline
{Shangoda} & 0\% & native & 42\% & 86\%\tabularnewline
{Uri} & 0\% & 37\% & native & 96\%\tabularnewline
{Mukar} & 0\% & 17\% & native & 88\%\tabularnewline
\bottomrule
\end{tabular}

\caption{The level of multilingualism in the generation born after 1920}
\label{Table3}
\end{table}

There are several factors which triggered the drastic changes in local
\isi{multilingualism}. The first reason is the spread of Russian as a \isi{lingua franca}
across Daghestan. The command of Russian substituted local bilingualism.
Secondly, the relations within the neighbourhood started to lose their
economic significance, being substituted by connections with bigger
towns. At present, the Mehwebs prefer shopping in Makhachkala rather
than in Kumukh. Villagers also ceased cultivating fields, the borders
with the neighbours have lost their significance, and communication
became rarer.

There are rare cases of some Obokhs\is{Obokh} speaking Mehweb among those born in
the 1960s. This is because, until the 2000s, there was no senior school in
Obokh, and some children continued their education in Mehweb. Several
people reported the ability to understand Mehweb, acquired during their
school years.

In Mehweb, people born after the 1950s speak almost no Lak, but the
command of Avar is still very high. Avar was supported by schooling\is{school} and
communication with neighbours and with the Avar administration. Mehwebs
born after 1990, however, do not speak Avar. This might be a
manifestation of the same process of the loss of local \isi{multilingualism}
as in other villages, but it could also be a pattern of age-based
multilingualism, whereby a neighbouring language is acquired when people
start to work. In the latter case, this generation will speak Avar after
their professional socialization, at the age of 30–40. Only later
research will show what pattern the now young Mehwebs will follow.

\pagebreak

Some Mehwebs reported a command of the Akusha dialect of \ili{Dargwa}. In the
1950–1970s, Mehweb did not have enough shepherds, and the Dargis from the
Akusha district worked in the Mehweb kolkhoz as sheepherders. The
Mehwebs remember communicating with these shepherds and with their
wives, who came to see their husbands when they returned to Mehweb with
the sheep. As a result, some of the Mehwebs acquired the Akusha dialect,
which is otherwise not intelligible to them.

Another change concerned \isi{literacy}. The atheistic politics of the USSR
resulted in a dramatic loss of Arabic literacy. Only 5\% of Mehwebs born
after 1920 knew the Arabic script (as compared to the 48\% in the
generation born before 1919). A similar change happened in other
villages. At the same time, most villagers became literate in Cyrillic
and could read and write Russian and Avar.


% 6. 
\section{Summary}\label{summary}

Mehweb is a minority language, spoken in only one village. As mentioned
in the introduction, there are no signs of language shift in Mehweb. In
the village, everybody speaks Mehweb, and since the 19th century the
number of speakers has not decreased. There is, however, a strong
tendency towards the loss of traditional patterns of multilingualism.
Over the 20th century, knowledge of neighbouring languages in highland
villages was substituted by knowledge of Russian, because Russian spreaded
all over Daghestan and started to serve as the lingua franca (the level
of bilingualism is shown in \figref{Figure2}).
%
\begin{figure}[ht]
\includegraphics[width=.95\textwidth]{Figure2-new.png}

% Figure 2. 
\caption{Multilingualism in five villages: before 1919 and after 1920
  (map courtesy Yuri Koryakov)}
\label{Figure2}
\end{figure}
%
A good command of Russian was
supported by the arrival of television and by intensive migration to
towns. Today, almost every family has relatives who live elsewhere and come to
the village for vacation or on some special occasion (such as weddings
and funerals). Children who were born in cities usually only speak
Russian, and pass Russian to their peers who live in the village
\citep{daniel-dobrushina-knyazev2011}. Therefore, until recently the languages that could
influence the vocabulary and the grammar of Mehweb were Avar and Lak.
This role has now been assumed by Russian. In spite of the changes in
the multilingualism patterns, the Mehweb community still remains,
comparatively, more multilingual than other neighbouring communities.


\section*{Acknowledgements}

I am very grateful to Olesya Khanina and Francesca Di Garbo who read
the paper and gave valuable feedback.



\nocite{komarov1873}

\printbibliography[heading=subbibliography,notkeyword=this]


\end{document}

%%% Local Variables:
%%% mode: latex
%%% TeX-master: "../main"
%%% End:
