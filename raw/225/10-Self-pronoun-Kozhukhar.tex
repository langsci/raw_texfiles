\documentclass[output=paper]{langsci/langscibook} 
\ChapterDOI{10.5281/zenodo.3402072}

% Chapter 10

\title{The self-pronoun in Mehweb}

\author{Aleksandra Kozhukhar\affiliation{National Research University Higher School of Economics}}

\abstract{This study deals with the phenomenon of the pronominal multifunctionality in Mehweb. The pronominal stem glossed as `self' has four functions (reflexive, logophoric, intensifier, and resumptive) which are described in some detail.

\emph{Keywords}: logophoricity, reflexivization, long-distance
reflexives}


\begin{document}
\maketitle

% 1 
\section{Introduction}

In many typologically distinct languages, a formal relationship between
reflexive pronouns, logophoric pronouns and intensifiers is attested 
(see \citealt{könig-etal2013}). In Mehweb these functions are fulfilled by the pronominal
stem \emph{sa}‹\textsc{cl}›\emph{i}, `self', which can also be used as a
resumptive.

In this paper I will describe the formal and functional aspects of the
pronoun \emph{sa}‹\textsc{cl}›\emph{i}, starting with a description of the
structure of the relevant forms in \sectref{morphology}. In \sectref{logophoric-and-reflexive-contexts} I will
discuss their reflexive and logophoric usages, followed by a description
of free logophors in \sectref{discourse-usage}. \sectref{intensifier} is dedicated to the usage of
\emph{sa}‹\textsc{cl}›\emph{i} as an intensifier and in \sectref{resumptive} some examples
of the resumptive function will be discussed.

% 2.
\section{Morphology}\label{morphology}

The pronoun \emph{sa}‹\textsc{cl}›\emph{i} can appear in the form of what I refer to
as a ``bare pronoun'', consisting of a pronominal stem inflected for
number and case. A ``complex pronoun'' can be formed by adding the suffix
\emph{-al} to the bare pronoun. Both forms are described below.

% 2.1.
\subsection{Bare stem}

Mehweb employs the bare pronoun to refer to the antecedents in the
long-distance domain (see \sectref{distant-domain}) and possessive domain (see
\sectref{possessive-domain}). The pronoun \emph{sa}‹\textsc{cl}›\emph{i} agrees in number, person and
gender with the antecedent and can attach case suffixes (see \tabref{tab:10:1}).

\begin{table}[h]
  % Table 1.
  \caption{The paradigm of the bare pronoun}\label{tab:10:1}

  \advance\tabcolsep-1.5pt
\begin{tabular}{@{}llllllll@{}}
\toprule
{number} & {gender}\footnotemark & \textsc{nom} & \textsc{erg} & \textsc{dat} & \textsc{gen}
& \textsc{inter}-\textsc{lat} & \textsc{comit}\tabularnewline \midrule
& \textsc{m} & \emph{sa‹w›i} & & & & &\tabularnewline
\textsc{sg} & \textsc{f} & \emph{sa‹r›i} & \emph{sune-jni} & \emph{sune-s} & \emph{sune-la} & \emph{sune-ze} & \emph{sune-ču}\tabularnewline
& \textsc{n} & \emph{sa‹b›i} & & & & &\tabularnewline
\textsc{pl} & \textsc{hpl} & \emph{sa‹b›i} & \emph{ču-ni} & \emph{ču-s} & \emph{ču-la} & \emph{ču-ze} &
\emph{ču-ču}\tabularnewline
& \textsc{npl} & \emph{sa‹r›i} & & & & &\tabularnewline
\bottomrule
\end{tabular}
\end{table}

\footnotetext{In \tabref{tab:10:1} the genders
  are given as abbreviations as follows: \textsc{m} – masculine,
  \textsc{f} – feminine, \textsc{n} – neutral
  (i.e.\ all inanimate and animate non-human entities),
  \textsc{hpl} – human plural entities, \textsc{npl}~– non-human plural
  entities.}

The bare pronoun has three suppletive allomorphs. The first,
\emph{sa}‹\textsc{cl}›\emph{i}, is the nominative stem, which is the same in the
singular and in the plural and carries a gender marker infix, agreeing
with the antecedent of the pronoun. The second, \emph{sune-}, is the
oblique stem of the third person singular and can attach case suffixes.
The third, \emph{ču-}, is the oblique stem of the third person plural and
can attach case suffixes.

% 2.2
\subsection{Complex pronouns}

The stem \emph{sa}‹\textsc{cl}›\emph{i} may attach the particle \emph{-al}. The
particle functions as emphatic when attached to nominal stems and
demonstratives\footnote{Suffix \emph{-al} also marks cardinal numerals
  \citep[58]{magometov1982}.}:

\ea
\gll {it dursi-li-če꞊l ħule w-iz-ur.}\\
this girl-\textsc{obl}-\textsc{super}(\textsc{lat})꞊\textsc{emph} look \textsc{m}-\textsc{lv}:\textsc{pfv}-\textsc{aor}\\ 
\glt `He looked only at this girl.'

\ex
\gll {urši iti-če꞊l ħule w-iz-ur.}\\
boy this-\textsc{super}(\textsc{lat})꞊\textsc{emph} look \textsc{m}-\textsc{lv}:\textsc{pfv}-\textsc{aor}\\ 
\glt `The boy\textsubscript{i} looked only at
him\textsubscript{y}/her\textsubscript{y}.'
\z

A partial paradigm of the complex pronoun is given in \tabref{tab:10:2}. For the
sake of comparison, inflected forms of the first and second person
pronouns are also presented.

\begin{table}[h]
  % Table 2.
  \caption{The paradigm of the complex pronoun}\label{tab:10:2}

  \advance\tabcolsep-1pt
  \footnotesize
\begin{tabular}{@{}llm{2.5em}<{\raggedright}llllll@{}}
\toprule
% \multicolumn{1}{@{}m{2em}<{\raggedright}}{{num\-ber}}  & \multicolumn{1}{m{2em}<{\raggedright}}{{per\-son}} & \multicolumn{1}{m{2em}<{\raggedright}}{{gen\-der}}
 \makebox[16pt][l]{\raisebox{-8pt}[16pt][10pt]{\rotatebox{45}{number}}} &
 \makebox[12pt][l]{\raisebox{-7pt}[0pt][0pt]{\rotatebox{45}{person}}} &
 \makebox[12pt][l]{\raisebox{-7pt}[0pt][0pt]{\rotatebox{45}{gender}}} 
 & \textsc{nom} &
\textsc{erg} & \textsc{dat} & \textsc{gen} & \textsc{inter}-\textsc{lat} &
\textsc{comit}\tabularnewline \midrule
& {1} & – & \emph{nu-wal} & \emph{nu-ni-jal} & \emph{nab-al} & \emph{di-la-l} & \emph{di-ze-l} &
\emph{di-ču-wal} \tabularnewline
& {2} & – & \emph{ħu-wal} & \emph{ħu-ni-jal} & \emph{ħ} & \emph{ħu-la-l} & \emph{ħu-ze-l} &
\emph{ħu-ču-wal} \tabularnewline \cmidrule{2-9}
\raisebox{3pt}[0pt][0pt]{\textsc{sg}} & & \textsc{m} & \emph{sa‹w›i-jal} & & & & &\tabularnewline
& {3} & \textsc{f} & \emph{sa‹r›i-jal} & \emph{sune-jni-jal} & \emph{sune-s-al} &
\emph{sune-la-l} & \emph{sune-ze-l} & \emph{sune-ču-wal} \tabularnewline
& & \textsc{n} & \emph{sa‹b›i-jal} & & & & &\tabularnewline \midrule
& {1} & – & \emph{nuša-l} & \emph{nuša-jni-jal} & \emph{nušab-al} & \emph{nuša-la-l} &
\emph{nuša-ze-l} & \emph{nuša-ču-wal} \tabularnewline
\raisebox{-8pt}[0pt][0pt]{\textsc{pl}} & {2} & – & \emph{ħuša-l} & \emph{ħuša-jni-jal} & \emph{ħušad-al} &
\emph{ħuša-la-l} & \emph{ħuša-ze-l} & \emph{ħuša-ču-wal} \tabularnewline \cmidrule{2-9}
& {3} & \multicolumn{1}{m{2em}<{\raggedright}}{\textsc{hpl} \textsc{npl}} & \multicolumn{1}{m{4em}<{\raggedright}}{\emph{sa‹b›i-jal} \emph{sa‹r›i-jal}} & \emph{ču-ni-jal} & \emph{ču-s-al} & \emph{ču-la-l}
& \emph{ču-ze-l} & \emph{ču-ču-wal} \tabularnewline
\bottomrule
\end{tabular}
\end{table}


The suffix \emph{-al} is preceded by an epenthetic consonant or deletion
of the vowel in the suffix. If the last vowel of the stem is \emph{-u}-,
the epenthetic consonant is \emph{-w}- (e.g.\ \emph{nuwal}). If the last
vowel of the stem is \emph{-i}-, the epenthetic consonant is \emph{-j}-
(e.g.\ \emph{sawijal}). If \emph{-al} follows \emph{-e-} or \emph{-a}-
then the vowel in the suffix is dropped (e.g.\ \emph{ħušal} and
\emph{sunezel}). In the dative case, \emph{-al} is simply attached to
the case suffix. The distribution of these forms is discussed in the
following sections.

% 3.
\section{Logophoric and reflexive contexts}\label{logophoric-and-reflexive-contexts}

\is{logophoric pronoun|(}
\is{reflexive pronoun|(}

In this section, I will discuss the reflexive and logophoric functions of
the pronominal stem.

Reflexives are typically used to show the coreference of the non-subject
argument of the clause to another clause-mate argument \citep{könig-etal2013}.
\citet{testelets-toldova1998} argue that reflexives may be bound by a
higher syntactic priority position (i.e.\ subject) which occurs in the
same sentence. Logophoric pronouns are used to indicate
``coreferenciality or conjoint reference with the argument of a higher
predicate of communication or mental experience'' \citep{sells1987}.
{\looseness1\par}

% 3.1
\subsection{Local domain}

\is{local domain|(}

The reflexive is bound within the local domain if it occurs within the
same clause as its antecedent. Mehweb demonstrates no constraints on the
syntactic position a reflexive can take in the clause. It can occupy the
position of P as in (\ref{ex:10:3}) and (\ref{ex:10:6}), the indirect object position as in (\ref{ex:10:4}),
or it can fulfill the role of adjunct (\ref{ex:10:5}). The antecedent, however, has
to be the subject (cf. infelicitous (\ref{ex:10:7})). This means it requires
ergative marking with a transitive predicate, nominative for
intransitive, and dative, inter-lative or inter-elative for experiential
predicates (cf. examples (\ref{ex:10:3}), (\ref{ex:10:4}) and (\ref{ex:10:6})). Within the local domain, the
form of the pronoun is constrained: a bare pronoun with an antecedent in
the local domain is considered ungrammatical and can only be interpreted
as having logophoric meaning (compare (\ref{ex:10:3}) and (\ref{ex:10:8})).

\ea \label{ex:10:3} % {2}
\gll  rasuj-ni sa‹w›i-jal w-it-ib.\\
Rasul.\textsc{obl}-\textsc{erg} ‹\textsc{m}›self-\textsc{emph} \textsc{m}-beat:\textsc{pfv}-\textsc{aor}\\
\glt `Rasul\textsubscript{i} beat himself\textsubscript{i}.'

\ex \label{ex:10:4} % {3}
\gll  rasul sune-če-l ħule w-iz-ur.\\
Rasul self.\textsc{obl}-\textsc{super}(\textsc{lat})-\textsc{emph} look \textsc{m}-\textsc{lv}:\textsc{pfv}-\textsc{aor}\\ 
\glt `Rasul\textsubscript{i} looked at himself\textsubscript{i}.'

\ex \label{ex:10:5} % {4}
\gll  rasul sune-če-w-al duč'i-rq'-uwe le-w.\\
Rasul self.\textsc{obl}-\textsc{super}-\textsc{m}(\textsc{ess})-\textsc{emph} laugh-\textsc{lv}:\textsc{ipfv}-\textsc{cvb.ipfv} \textsc{aux}-\textsc{m}\\ 
\glt `Rasul\textsubscript{i} laughed at himself\textsubscript{i}.'

\ex \label{ex:10:6} % {5}
\gll  rasuj-ze sa‹w›i-jal daˤʜmic'aj-ħe-w gu-b.\\
Rasul.\textsc{obl}-\textsc{inter}(\textsc{lat}) ‹\textsc{m}›self-\textsc{emph} mirror-\textsc{in}-\textsc{m}(\textsc{ess}) see:\textsc{pfv}-\textsc{aor}\\ 
\glt `Rasul\textsubscript{i} saw himself\textsubscript{i} in the mirror.'

\ex \label{ex:10:7} % {6}
\ea % a.
\gll *sune-jni-jal rasul w-it-ib.\\
self.\textsc{obl}-\textsc{erg}-\textsc{emph} Rasul \textsc{m}-beat:\textsc{pfv}-\textsc{aor}\\ 
\glt `Rasul\textsubscript{i} beat himself\textsubscript{i}.'
(lit.\ `Himself\textsubscript{i} beat Rasul\textsubscript{i}.')

\ex % b.
\gll *sune-ze-l rasul gu-b.\\
self.\textsc{obl}-\textsc{inter}(\textsc{lat}) Rasul.\textsc{obl}-\textsc{erg} see:\textsc{pfv}-\textsc{aor}\\ 
\glt `Rasul saw himself.'
(lit.\ `Himself\textsubscript{i} saw Rasul\textsubscript{i}.')
\z

\ex \label{ex:10:8} % {7}
\gll  *rasuj-ni sa‹w›i w-it-ib.\\
Rasul.\textsc{obl}-\textsc{erg} ‹\textsc{m}›self ‹\textsc{m}›-beat:\textsc{pfv}-\textsc{aor}\\ 
\glt `Rasul\textsubscript{i} beat himself\textsubscript{i}.'
\z


Because Mehweb is a pro-drop language, the reflexive can get a
zero-antecedent, which is obligatorily in the subject position, as in
(\ref{ex:10:9}).

\ea \label{ex:10:9} % {8}
\ea %   a.
\gll it-ini sune-s-al jaˤbu as-ib.\\
that-\textsc{erg} self.\textsc{obl}-\textsc{dat}-\textsc{emph} horse take:\textsc{pfv}-\textsc{aor}\\

\ex % b.
\gll sune-s-al jaˤbu as-ib.\\
self.\textsc{obl}-\textsc{dat}-\textsc{emph} horse take:\textsc{pfv}-\textsc{aor}\\ 
\glt `(He\textsubscript{i}) bought himself\textsubscript{i} a horse.'
\z
\z

The reflexive pronoun can be bound by a quantified NP.

\ea % {9}
\gll  har-il urši-li-ni sune-s-al jaˤbu as-ib.\\
each-\textsc{atr} boy-\textsc{obl}-\textsc{erg} self.\textsc{obl}-\textsc{dat}-\textsc{emph} horse take:\textsc{pfv}-\textsc{aor}\\ 
\glt `Each boy\textsubscript{i} bought himself\textsubscript{i} a horse.'
\z

Subordinate clauses work the same way. In a subordinate clause, the bare
pronoun cannot be bound within the subordinate clause (\ref{ex:10:11}), while the
complex pronoun has to be bound within it (\ref{ex:10:12}).

\ea \label{ex:10:11} % {10}
\gll  rasuj-s dig-uwe le-w adaj-ze sa‹w›i daˤʜmic'aj-ħe-w gʷ-es.\\
Rasul.\textsc{obl}-\textsc{dat} want:\textsc{ipfv}-\textsc{cvb.ipfv} \textsc{aux}-\textsc{m} father-\textsc{inter}(\textsc{lat}) ‹\textsc{m}›self mirror-\textsc{in}-\textsc{m}(\textsc{ess}) see:\textsc{pfv}-\textsc{inf}\\ 
\glt `Rasul\textsubscript{i} wants his father\textsubscript{y} to see
him\textsubscript{i} in the mirror.'

\ex \label{ex:10:12} % {11}
\gll  rasuj-s dig-uwe le-w adaj-ze sa‹w›i-jal daˤʜmic'aj-ħe-w gʷ-es.\\
Rasul.\textsc{obl}-\textsc{dat} want:\textsc{ipfv}-\textsc{cvb.ipfv} \textsc{aux}-\textsc{m} father-\textsc{inter}(\textsc{lat}) ‹\textsc{m}›self-\textsc{emph} mirror-\textsc{in}-\textsc{m}(\textsc{ess}) see:\textsc{pfv}-\textsc{inf}\\ 
\glt `Rasul\textsubscript{i} wants his father\textsubscript{y} to see
himself\textsubscript{y} in the mirror.'
\z

In example (\ref{ex:10:12}) the antecedent of the reflexive is within the local
domain, whereas in (\ref{ex:10:11}) it is located in the distant domain (the latter
will be discussed further in \sectref{distant-domain}). The two domains differ as to
which pronoun is used: the local domain employs the complex pronoun,
whereas for an antecedent in the distant domain the bare pronoun is
used.
%
\is{local domain|)}

% 3.2
\subsection{Possessive domain}\label{possessive-domain}

\is{possessive domain|(}

The possessive domain contains contexts where a genitive reflexive
occurs in an NP within the same clause as its antecedent. In Mehweb,
this domain is  %\pagebreak[4]
distinguished from the local domain in that both bare
pronouns and complex pronouns can be employed\footnote{This fact may
  serve as evidence for the idea that the possessive domain is a
  transition point between the local domain and the distant domain.}, as
in (\ref{ex:10:13}).

\ea \label{ex:10:13} % {12}
\ea % a.
\gll sune-la quli-w ħa-jz-ur.\\
self.\textsc{obl}-\textsc{gen} house-\textsc{m}(\textsc{ess}) \textsc{neg}-live-\textsc{aor}\\ 
\glt `(He\textsubscript{i}) did not live in his\textsubscript{i} house.'

\ex % b.
\gll sune-la-l quli-w ħa-jz-ur.\\
self.\textsc{obl}-\textsc{gen}-\textsc{emph} house-\textsc{m}(\textsc{ess}) \textsc{neg}-live-\textsc{aor}\\ 
\glt `(He\textsubscript{i}) did not live in his\textsubscript{i} house.'
\z

\ex \label{ex:10:14} % {13}
\gll  sune-la xunul quli-r r-aq'-a.\\
self.\textsc{obl}-\textsc{gen} woman house-\textsc{f}(\textsc{ess}) \textsc{f}-leave:\textsc{pfv}-\textsc{imp}\\ 
\glt `Leave your wife at home.'
(corpus, Brother and Sister: 1.34 \citep{magometov1982})

\ex \label{ex:10:15} % {14}
\gll  hel-di zamaj-ze-b ib urši-li-ni sune-la-l gurda-li-ze.\\
this-\textsc{pl} time-\textsc{inter}-\textsc{n}(\textsc{ess}) say:\textsc{pfv}.\textsc{aor} boy-\textsc{obl}-\textsc{erg} self.\textsc{obl}-\textsc{gen}-\textsc{emph} fox-\textsc{obl}-\textsc{inter}(\textsc{lat})\\ 
\glt `Then the boy\textsubscript{i} said to his\textsubscript{i} fox.'
(corpus, Two Sons: 1.86 \citep{magometov1982})
\z


Consider also the following examples where the complex and the bare
pronoun are used in similar contexts by the same speaker:

\ea \label{ex:10:16} % {15}
\gll  sunela ħalmic'ir-t-iču‹w›ijal urši helle w-erχ-ur.\\
self.\textsc{obl}-\textsc{gen} animal-\textsc{pl}-\textsc{comit}‹\textsc{m}› boy here(\textsc{lat}) \textsc{m}-enter:\textsc{pfv}-\textsc{aor}\\ 
\glt `The boy\textsubscript{i} entered with his\textsubscript{i} animals.'
(corpus, Two Sons: 1.126 \citep{magometov1982})

\ex \label{ex:10:17} % {16}
\gll  habala-habal sune-la-l ħalmic'ir-t d-aχ-un.\\
start-start self.\textsc{obl}-\textsc{gen}-\textsc{emph} animal-\textsc{pl} \textsc{npl}-feed:\textsc{pfv}-\textsc{aor}\\ 
\glt `First he\textsubscript{i} fed all his\textsubscript{i} animals.'
(corpus, Two Sons: 1.198 \citep{magometov1982})
\z


Examples (\ref{ex:10:14}) to (\ref{ex:10:17}) prove that in natural texts the bare pronoun is
available in possessive contexts. Consultants provide contradictory
grammaticality judgements of constructed stimuli with the reflexive
genitive. The majority consider (\ref{ex:10:13}a) and (\ref{ex:10:13}b) to have the same meaning
and to be fully grammatical. Some consultants suggest that
\emph{sunelal} adds emphatic meaning (`his own'), whereas \emph{sunela}
simply indicates possession. Other consultants suggest that the bare
pronoun \emph{sunela} is not bound within the sentence (for further
discussion see \sectref{discourse-usage}), i.e.\ (\ref{ex:10:13}a) can be translated as `He is living
in his (someone else's) house'. Finally, some consultants consider
\emph{sunela} to be ungrammatical, apparently extending the constraints
on the occurrence of bare pronouns in the same clause as their
antecedents to possessive NPs.
%
\is{possessive domain|)}


% 3.3
\subsection{Distant domain}\label{distant-domain}

Distant domain contexts are sentences in which the pronoun and its
antecedent occur in different clauses. In Mehweb, the order of the
antecedent and the pronoun is relevant within the local domain. The
pronoun cannot precede its antecedent, otherwise it gets the free
logophoric reading (more on free logophors in \sectref{discourse-usage}). The distant
domain requires using the bare pronoun (see (\ref{ex:10:18})).

\ea \label{ex:10:18} % {17}
\gll  sune-s dig-uwe le-w adaj-ze rasul daˤʜmic'aj-ħe-w gʷ-es.\\
self.\textsc{obl}-\textsc{dat} want:\textsc{ipfv}-\textsc{cvb.ipfv} \textsc{aux}-\textsc{m} father-\textsc{inter}(\textsc{lat}) Rasul mirror-\textsc{in}-\textsc{m}(\textsc{ess}) see:\textsc{pfv}-\textsc{inf}\\ 
\glt `Rasul\textsubscript{i} wants his father\textsubscript{y} to see
him\textsubscript{i} in the mirror.'

lit.\ `Himself\textsubscript{i} wants his father\textsubscript{y} to see
Rasul\textsubscript{i} in the mirror.'
\z

The bare stem can take subject and non-subject positions (P, IO,
adjunct) in the subordinate or main clause and can be used in both
finite and non-finite subordinate clauses, as shown in the following
section.

% 3.3.1
\subsubsection{Finite subordinate clauses}

Mehweb employs finite subordinate clauses with predicates of speech and
thought. Finite subordinate clauses in Mehweb may or may not be followed
by the converb \emph{ile} `having said' and utilize either personal
pronouns or a bare pronoun.

\ea \label{ex:10:19} % {18}
\gll  adaj-ni ib sune-ze žanawar gu-b (ile).\\
father-\textsc{erg} say:\textsc{pfv}.\textsc{aor} self.\textsc{obl}-\textsc{inter}(\textsc{lat}) wolf see:\textsc{pfv}-\textsc{aor} say:\textsc{pfv}.\textsc{cvb}\\
\glt `Father\textsubscript{i} said he\textsubscript{i} saw a wolf.'

\ex \label{ex:10:20} % {19}
\gll  adaj-ni ib sune-ze žanawar gu-b-ra (ile).\\
father-\textsc{erg} say:\textsc{pfv}.\textsc{aor} self.\textsc{obl}-\textsc{inter}(\textsc{lat}) wolf see:\textsc{pfv}-\textsc{aor}-\textsc{ego} say:\textsc{pfv}.\textsc{cvb}\\ 
\glt `Father\textsubscript{i} said he\textsubscript{i} saw a wolf.'
\z

Considering Chechen and Ingush, \citet{nichols2000} refers to 
contexts such as (\ref{ex:10:20}) as semi-direct speech. In semi-direct speech ``quoted
matter is identical to the reported speech act except that coreferents
to the speaker are reflexivized and the clause is marked with a
quotative particle'' \citep{nichols2000}. According to Nichols, Chechen
uses reflexives to refer to the speaker, i.e.\ the subject of the main
clause, only if subordinate finite clauses are marked by the quotation clitic
\emph{eanna}, while direct speech contexts use personal pronouns
(1\textsc{sg} pronouns) and do not use the clitic.

In Mehweb, the quotative converb \emph{ile} is optional with both types
of reference. Compare the pronouns in (\ref{ex:10:19}) and (\ref{ex:10:20}) to those in (\ref{ex:10:21}) and
(\ref{ex:10:22}); in all of these cases, the use of \emph{ile} is optional.

\ea \label{ex:10:21} % {20}
\gll  adaj-ni ib di-ze žanawar gu-b (ile).\\
father-\textsc{erg} say:\textsc{pfv}.\textsc{aor} I.\textsc{obl}-\textsc{inter}(\textsc{lat}) wolf see:\textsc{pfv}-\textsc{aor} say:\textsc{pfv}.\textsc{cvb}\\ 
\glt `Father\textsubscript{i} said he\textsubscript{i} saw a wolf.'

\ex \label{ex:10:22} % {21}
\gll  adaj-ni ib di-ze žanawar gu-b-ra (ile).\\
father-\textsc{erg} say:\textsc{pfv}.\textsc{aor} I.\textsc{obl}-\textsc{inter}(\textsc{lat}) wolf see:\textsc{pfv}-\textsc{aor}-\textsc{ego} say:\textsc{pfv}.\textsc{cvb}\\ 
\glt `Father\textsubscript{i} said he\textsubscript{i} saw a wolf.'
\z

\largerpage[.5]

\begin{table}[b]
  % Table 3.
  \caption{Summary on the stimuli and antecedents}\label{tab:10:3}

\begin{tabular}{@{}ll@{}}
\toprule
{stimulus} & {antecedent of the pronoun}\tabularnewline\midrule
\emph{adajni ib dize žanawar gub} & actual speaker\tabularnewline
\emph{adajni ib dize žanawar gubra} & subject of the main
clause\tabularnewline
\emph{adajni ib suneze žanawar gub} & \isi{subject} of the main
clause\tabularnewline
\emph{adajni ib suneze žanawar gubra} & subject of the main
clause\tabularnewline
\bottomrule
\end{tabular}
\end{table}

\tabref{tab:10:3} provides a summary of options for a pronoun used in a
subordinate finite clause. It shows that \emph{dize} behaves as a
personal pronoun, since it can change its antecedent between the actual
and the \isi{reported speaker}. The pronoun \emph{suneze} on the other hand,
behaves as a logophoric pronoun and always refers to the subject of the
main clause; cf.\ the following examples:

\ea \label{ex:10:23} % {22}
\gll  adaj-ni ib rasuj-ze di-ze žanawar gu-b (ile).\\
father-\textsc{erg} say:\textsc{pfv}.\textsc{aor} Rasul.\textsc{obl}-\textsc{inter}(\textsc{lat}) I.\textsc{obl}-\textsc{inter}(\textsc{lat}) wolf see:\textsc{pfv}-\textsc{aor} say:\textsc{pfv}.\textsc{cvb}\\ 
\glt `Father\textsubscript{i} said to Rasul that he\textsubscript{i} saw a
wolf.'

\ex \label{ex:10:24} % {23}
\gll  adaj-ni ib rasuj-ze di-ze žanawar gu-b-ra (ile).\\
father-\textsc{erg} say:\textsc{pfv}.\textsc{aor} Rasul.\textsc{obl}-\textsc{inter}(\textsc{lat}) I.\textsc{obl}-\textsc{inter}(\textsc{lat}) wolf see:\textsc{pfv}-\textsc{aor}-\textsc{ego} say:\textsc{pfv}.\textsc{cvb}\\ 
\glt `Father\textsubscript{i} said to Rasul that he\textsubscript{i} saw a
wolf.'

\ex \label{ex:10:25} % {24}
\gll  adaj-ni ib rasuj-ze sune-ze žanawar gu-b-ra (ile).\\
father-\textsc{erg} say:\textsc{pfv}.\textsc{aor} Rasul.\textsc{erg}-\textsc{inter}(\textsc{lat}) self.\textsc{obl}-\textsc{inter}(\textsc{lat}) wolf see:\textsc{pfv}-\textsc{aor}-\textsc{ego} say:\textsc{pfv}.\textsc{cvb}\\ 
\glt `Father\textsubscript{i} said to Rasul that he\textsubscript{i} saw a
wolf.'

\z



Examples (\ref{ex:10:26}) and (\ref{ex:10:27}) additionally show subordinate clauses headed by
different matrix predicates.

\ea \label{ex:10:26} % {25}
\gll  it-ini pikri b-aq-ib sa‹w›i q'am uh-ub-le le-w (ile).\\
that-\textsc{erg} thought \textsc{n}-do:\textsc{pfv}-\textsc{aor} ‹\textsc{m}›self late become:\textsc{pfv}-\textsc{aor}-\textsc{cvb} \textsc{aux}-\textsc{m} say:\textsc{pfv}-\textsc{cvb}\\
\glt `He\textsubscript{i} had a thought that he\textsubscript{i} was late.'

\ex \label{ex:10:27} % {26}
\gll  iti-s b-ik-ib sa‹w›i q'am uh-ub-le le-w (ile).\\
that-\textsc{dat} \textsc{n}-think:\textsc{pfv}-\textsc{aor} ‹\textsc{m}›self late become:\textsc{pfv}-\textsc{aor}-\textsc{cvb} \textsc{aux}-\textsc{m} say:\textsc{pfv}-\textsc{cvb}\\ 
\glt `He\textsubscript{i} thought that he\textsubscript{i} was late.'
\z

% 3.3.2
\subsubsection{Non-finite subordinate clauses}

Non-finite subordinate clauses in Mehweb can employ converbs,
nominalizations or infinitives, depending on the predicate of the matrix
clause. Non-finite clauses can occur with a bare pronoun or with a zero
pronoun in the subject position. Grammaticality of first person personal
pronouns referring to the subject of the main clause in non-finite
subordinate clauses is a matter of variation among the consultants (cf.\
\ref{ex:10:28} and \ref{ex:10:31}). In non-finite subordinate clauses, the self-pronoun can
occupy subject and non-subject positions (cf.\ \ref{ex:10:32}).

Examples (\ref{ex:10:28}) and (\ref{ex:10:29}) demonstrate the use of the self-pronoun in
subject and non-subject position in a subordinate clause headed by an
infinitive.

\ea \label{ex:10:28} % {27}
\gll  it uruχ k'-uwe le-w sa‹w›i \(\textsuperscript{?}nu\) ʁaˤm-le w-ik-es (ile).\\
this be.afraid \textsc{lv}:\textsc{ipfv}-\textsc{cvb.ipfv} \textsc{aux}-\textsc{m} ‹\textsc{m}›self (\textsuperscript{?}I) wrong-\textsc{advz} \textsc{m}-become:\textsc{pfv}-\textsc{inf} say:\textsc{pfv}.\textsc{cvb}\\ 
\glt `He is afraid of making a mistake.'

\ex \label{ex:10:29} % {28}
\gll  rasuj-s dig-uwe le-b adaj sune-če-l ħule w-iz-es.\\
Rasul.\textsc{obl}-\textsc{dat} want:\textsc{ipfv}-\textsc{cvb.ipfv} \textsc{aux}-\textsc{n} father self.\textsc{obl}-\textsc{super}(\textsc{lat})-\textsc{emph} look \textsc{m}-\textsc{lv}:\textsc{pfv}-\textsc{inf}\\ 
\glt `Rasul\textsubscript{i} wants his father\textsubscript{y} to look at
himself\textsubscript{y}.'
\z

Subordinate clauses with an infinitive in Mehweb are employed as a
strategy for marking sentential arguments, and can also express an aim
(see (\ref{ex:10:30}–\ref{ex:10:32})). In (\ref{ex:10:31}), the personal pronoun \emph{nu} `I' is
grammatical.

\ea \label{ex:10:30} % {29}
\gll  ʡali-ni g-ib rasuj-ze arc il armi-li-ze uˤq'-es.\\
Ali-\textsc{erg} give:\textsc{pfv}-\textsc{aor} Rasul.\textsc{obl}-\textsc{inter}(\textsc{lat}) money that army-\textsc{obl}-\textsc{inter}(\textsc{lat}) \textsc{m}.go:\textsc{pfv}-\textsc{inf}\\ 
\glt `Ali bribed Rasul so that he (Rasul or another person) go to the army.'
(lit.\ `Ali gave money to Rasul in order that Rasul (or another person)
went to the army.')

\ex \label{ex:10:31} % {30}
\gll  ʡali-ni g-ib rasuj-ze arc nu armi-li-ze uˤq'-es.\\
Ali-\textsc{erg} give:\textsc{pfv}-\textsc{aor} Rasul.\textsc{obl}-\textsc{inter}(\textsc{lat}) money I army-\textsc{obl}-\textsc{inter}(\textsc{lat}) \textsc{m}.go:\textsc{pfv}-\textsc{inf}\\ 
\glt `Ali bribed Rasul to go the army.'
(lit.\ `Ali gave money to Rasul in order Ali went to the army.')\pagebreak[4]


\ex \label{ex:10:32} % {31}
\gll  ʡali-ni g-ib rasuj-ze arc sa‹w›i armi-li-ze uˤq'-es.\\
Ali-\textsc{erg} give:\textsc{pfv}-\textsc{aor} Rasul.\textsc{obl}-\textsc{inter}(\textsc{lat}) money ‹\textsc{m}›self army-\textsc{obl}-\textsc{inter}(\textsc{lat}) \textsc{m}.go:\textsc{pfv}-\textsc{inf}\\ 
\glt `Ali bribed Rasul to go the army.'

lit.\ `Ali gave money to Rasul in order Ali went to the army.'
\z

Examples (\ref{ex:10:33}) and (\ref{ex:10:34}) demonstrate the self-pronoun in a subordinate
clause headed by a specialized converb.

\ea \label{ex:10:33} % {32}
\gll  abaj-ni g-ib dursi ruzi-li-ze sune-s ʡaˤχ-le b-uʔ-alis.\\
mother-\textsc{erg} give:\textsc{pfv}-\textsc{aor} girl sister-\textsc{obl}-\textsc{inter}(\textsc{lat}) self.\textsc{obl}-\textsc{dat} good N-be:\textsc{pfv}-\textsc{purp}\\ 
\glt `Mother\textsubscript{i} gave her\textsubscript{i}
daughter\textsubscript{y} to her\textsubscript{i}
sister\textsubscript{z} in order she\textsubscript{i} felt good.'

\ex \label{ex:10:34} % {33}
\gll  baba uruχ k'-uwe le-r sa‹r›i ar-d-ik-ala (ile).\\
grandmother be.afraid \textsc{lv}:\textsc{ipfv}-\textsc{cvb.ipfv} \textsc{aux}-\textsc{f} ‹\textsc{f}›self \textsc{pv}-\textsc{f1}-fall:\textsc{ipfv}-\textsc{appr} say:\textsc{pfv}.\textsc{cvb}\\ 
\glt `Grandmother\textsubscript{i} is afraid of falling down.'
\z

Examples (\ref{ex:10:35}–% ), (\ref{ex:10:36}), and (
\ref{ex:10:37}) show the use of the bare pronoun in a
subordinate clause headed by an action nominal (masdar). In Mehweb there
are two suffixes available for the derivation of action nominals:
\emph{-ri} and \emph{-deš}. In most cases, these suffixes are
interchangeable.

\ea \label{ex:10:35} % {34}
\gll  ʡali-ze b-ah-ur rasuj-ze-la sune-s premia b-ak'-ri.\\
Ali-\textsc{inter}(\textsc{lat}) \textsc{n}-know:\textsc{pfv}-\textsc{aor} Rasul-\textsc{inter}-\textsc{el} self.\textsc{obl}-\textsc{dat} prize \textsc{n}-come:\textsc{pfv}-\textsc{nmlz}\\ 
\glt `Ali\textsubscript{i} found out from Rasul that he\textsubscript{i} got
money.'

\ex \label{ex:10:36} % {35}
\gll  iti-ze-la b-ah-ur-ra sune-jni maza b-erh-un-deš \textup/ b-erh-ri.\\
that-\textsc{inter}-\textsc{el} \textsc{n}-know:\textsc{pfv}-\textsc{aor}-\textsc{ego} self.\textsc{obl}-\textsc{erg} ram \textsc{n}-slaughter:\textsc{pfv}-\textsc{aor}-\textsc{nmlz} / \textsc{n}-slaughter:\textsc{pfv}-\textsc{nmlz}\\ 
\glt `(He\textsubscript{i}) found out from him\textsubscript{y} that
he\textsubscript{i} killed a ram.'

\pagebreak[4]

\ex \label{ex:10:37} % {36}
\gll  it-ini pikri b-aq-ib sa‹w›i q'am uh-ub-le le-w-deš (ile).\\ 
that-\textsc{erg} thought \textsc{n}-do:\textsc{pfv}-\textsc{aor} ‹\textsc{m}›self late become:\textsc{pfv}-\textsc{aor}-\textsc{cvb} \textsc{aux}-\textsc{m}-\textsc{nmlz} say:\textsc{pfv}.\textsc{cvb}\\ 
\glt `He\textsubscript{i} thought that he\textsubscript{i} was late.'
\z


The purpose of the examples above is to show that bare pronouns can be
used in non-finite subordinate clauses. This fact blurs the distinction
between the two functions the bare pronoun fulfills – that of the
long-distant reflexive and the logophoric pronoun.

% 3.3.3
\subsubsection{Subject orientedness of the self-pronoun}

\is{subject|(}

In a finite subordinate clause, the bare pronoun occurring in subject
position is subject oriented. This means it is co-referent to the
subject of the main clause, as in (\ref{ex:10:25}). Non-finite subordinate clauses
on the other hand, show variation in what is interpreted to be the
referent of the pronoun, depending on the presence of the suffix
\emph{-al}.

Most consultants interpret the self-pronoun with the suffix \emph{-al}
as subject oriented as well (see \sectref{distant-domain}). In the case of two
embedded predications, both the bare pronoun and the personal pronoun
\emph{nu} choose the subject of the embedded matrix clause; cf.\ (\ref{ex:10:38}–% ) and (\ref{ex:10:39}) to (
\ref{ex:10:40}).

\ea \label{ex:10:38} % {37}
\gll  ʡali-ni ib rasuj-ni ib sune-jni eža as-i-ra.\\
Ali-\textsc{erg} say:\textsc{pfv}.\textsc{aor} Rasul.\textsc{obl}-\textsc{erg} say:\textsc{pfv}.\textsc{aor} self.\textsc{obl}-\textsc{erg} goat take:\textsc{pfv}-\textsc{aor}-\textsc{ego}\\ 
\glt `Ali\textsubscript{y} said that Rasul\textsubscript{i} said that
he\textsubscript{i} bought a goat.'

\ex \label{ex:10:39} % {38}
\gll  ʡali-ni ib rasuj-ni ib nu-ni eža as-i-ra.\\
Ali-\textsc{erg} say:\textsc{pfv}.\textsc{aor} Rasul.\textsc{obl}-\textsc{erg}
say:\textsc{pfv}.\textsc{aor} I-\textsc{erg} goat take:\textsc{pfv}-\textsc{aor}-\textsc{ego}\\ 
\glt `Ali\textsubscript{y} said that Rasul\textsubscript{i} said that
he\textsubscript{i} bought a goat.'

\ex \label{ex:10:40} % {39}
\gll  ʡali-ni ib rasuj-ni ib sune-jni-jal eža as-i-ra.\\
Ali-\textsc{erg} say:\textsc{pfv}.\textsc{aor} Rasul.\textsc{obl}-\textsc{erg} say:\textsc{pfv}.\textsc{aor} self.\textsc{obl}-\textsc{erg}-\textsc{emph} goat take:\textsc{pfv}-\textsc{aor}-\textsc{ego}\\ 
\glt `Ali\textsubscript{y} said that Rasul\textsubscript{i} said that
he\textsubscript{i} bought a goat.'
\z

If a demonstrative is used instead of the self-pronoun or a personal
pronoun, it does not take an antecedent in the same sentence:

\ea \label{ex:10:41} % {40}
\gll  ʡali-ni ib rasuj-ni ib il-ini꞊jal eža as-i-ra.\\
Ali-\textsc{erg} say:\textsc{pfv}.\textsc{aor} Rasul.\textsc{obl}-\textsc{erg} say:\textsc{pfv}.\textsc{aor}  this-\textsc{erg}꞊\textsc{emph} goat take:\textsc{pfv}-\textsc{aor}-\textsc{ego}\\ 
\glt `Ali\textsubscript{i} said that Rasul\textsubscript{y} said that
he\textsubscript{z} bought a goat.'
\z

The subject of the external embedded clause can be the antecedent of the
logophoric pronoun if and only if the subject of the first embedded
clause does not agree in person and/or number with the logophoric
pronoun.

\ea % {41}
\gll  ʡali-ni ib nu-ni ib sune-jni eža asi-ra.\\
Ali-\textsc{erg} say:\textsc{pfv}.\textsc{aor} I-\textsc{erg} say:\textsc{pfv}.\textsc{aor} self.\textsc{obl}-\textsc{erg}-\textsc{emph} goat take:\textsc{pfv}-\textsc{ego}\\ 
\glt `Ali\textsubscript{i} said that I said that he\textsubscript{i} bought a
goat.'
\z

\removelastskip
\is{subject|)}

% 3.3.4
\subsubsection{Non-subject orientedness: a hypothesis}\label{non-subject-orientedness}

\is{subject|(}

A bare pronoun in subject position in a subordinate clause, whether it
is finite or non-finite, is always `subject oriented'. This means it is
coreferent to the subject of the closest embedded clause (unless there
is a mismatch in person or number properties).

In some speakers, the complex pronoun behaves in the same way. In other
speakers, however, the complex pronoun has to be coreferent to the
non-subject argument of the matrix clause (when present) (cf.\ \ref{ex:10:43}–% , 44,45,
\ref{ex:10:46}).

\ea \label{ex:10:43} % {42}
\gll  ʡali-ni ib rasuj-ze sa‹w›i-jal q'ar iˤšq-es uˤq'-es-i.\\
Ali-\textsc{erg} say:\textsc{pfv}.\textsc{aor} Rasul.\textsc{obl}-\textsc{inter}(\textsc{lat}) ‹\textsc{m}›self(-\textsc{emph}) grass mow:\textsc{pfv}-\textsc{inf} M.go:\textsc{pfv}-\textsc{inf}-\textsc{atr}\\
\glt `Ali\textsubscript{i} said to Rasul\textsubscript{y} that
he\textsubscript{y} should go mow the grass.'

\ex % {43}
\gll  ʡali-ze b-ah-ur rasuj-ze-la sune-s-al premia b-aq'-ri.\\
Ali-\textsc{inter}(\textsc{lat}) \textsc{n}-know:\textsc{pfv}-\textsc{aor} Rasul.\textsc{obl}-\textsc{inter}-\textsc{el}  self.\textsc{obl}-\textsc{dat}(-\textsc{emph}) money \textsc{n}-do:\textsc{pfv}-\textsc{nmlz}\\ 
\glt `Ali\textsubscript{i} found out from Rasul\textsubscript{y} that
he\textsubscript{y} got money.'

\ex % {44}
\gll  ʡali-ni g-ib rasuj-ze arc sa‹w›i-jal armi-li-ze uˤq'-es.\\
Ali-\textsc{erg} give:\textsc{pfv}-\textsc{aor} Rasul-\textsc{inter}(\textsc{lat}) money ‹\textsc{m}›self(-\textsc{emph}) army-\textsc{obl}-\textsc{inter}(\textsc{lat}) \textsc{m}.go:\textsc{pfv}-\textsc{inf}\\
\glt `Ali\textsubscript{i} gave Rasul\textsubscript{y} money for
him\textsubscript{y} to go to the army.'

\ex \label{ex:10:46} % {45}
\gll  abaj-ni g-ib dursi ruzi-li-ze sune-s-al ʡaˤχ-le b-uʔ-alis.\\
mother-\textsc{erg} give:\textsc{pfv}-\textsc{aor} daughter sister-\textsc{obl}-\textsc{inter}(\textsc{lat}) self.\textsc{obl}-\textsc{dat}-\textsc{emph} good-\textsc{advz} \textsc{n}-be:\textsc{pfv}-\textsc{purp}\\ 
\glt `Mother\textsubscript{i} gave her\textsubscript{i}
daughter\textsubscript{y} to her\textsubscript{i}
sister\textsubscript{z} in order for her\textsubscript{y} to feel good.'
\z

In the four examples above, the self-pronoun takes the non-subject
argument of the main clause as its antecedent. The referent of the
embedded subject shifts from the subject to the non-subject argument of
the embedding clause if the main clause contains more than one argument
that can serve as an antecedent for the self-pronoun and matches it in
person and number.

If all these conditions are satisfied, then the bare pronoun takes its
reference from the subject of the main clause, whereas the complex
pronoun takes its reference from another argument of the main clause.
These rules apply to all complementation strategies and all predicates
of the main clause that allow a second argument or adjunct as a
potential antecedent. If the main clause lacks other arguments, or if the
arguments of the main clause do not match the self-pronoun in person and
number, the subject-to-non-subject shift does not occur.

The complex pronoun cannot take an argument outside the clause as its
antecedent. The non-subject argument of the main clause thus may not be
an immediate antecedent of the complex pronoun inside the subordinate
clause. Examples (\ref{ex:10:43}) to (\ref{ex:10:46}) can be explained by introducing a zero
pronoun in the subject position of the subordinate clause. This zero
pronoun is non-subject-oriented (see Schema 1). On the other hand, the
reference of the bare pronoun combined with an intensifier
(\emph{sunejni sunejnijal),} is always subject-oriented (that is,
whenever the nearest subject matches the self-pronoun in person and/or
number) – see (\ref{ex:10:49}).

\begin{figure}[h]
\centerline{\footnotesize Schema 1: Non-subject-oriented zero pronoun}

\medskip

[S intransitive predicate IO][self non-finite predicate]

[S intransitive predicate IO][∅ self-\textsc{emph} non-finite
predicate]

[S intransitive predicate IO][self self-\textsc{emph} non-finite
predicate]
\end{figure}


\ea % {46}
\gll  ʡali-ze b-ah-ur rasuj-ze-la sune-s premia b-aq'-ri.\\
Ali-\textsc{inter}(\textsc{lat}) \textsc{n}-know:\textsc{pfv}-\textsc{aor} Rasul.\textsc{obl}-\textsc{inter}-\textsc{el} self.\textsc{obl}-\textsc{dat} money \textsc{n}-get:\textsc{pfv}-\textsc{nmlz}\\ 
\glt `Ali\textsubscript{i} found out from Rasul that he\textsubscript{i} got
money.'

\ex % {47}
\gll  ʡali-ze b-ah-ur rasuj-ze-la ∅ sune-s-al premia b-aq'-ri.\\
Ali-\textsc{inter}(\textsc{lat}) \textsc{n}-know:\textsc{pfv}-\textsc{aor} Rasul.\textsc{obl}-\textsc{inter}-\textsc{el} ∅ self.\textsc{obl}-\textsc{dat}-\textsc{emph} money \textsc{n}-get:\textsc{pfv}-\textsc{nmlz}\\ 
\glt `Ali\textsubscript{i} found out from Rasul\textsubscript{y} that
he\textsubscript{y} got money.'

\ex \label{ex:10:49} % {48}
\gll  ʡali-ze b-ah-ur rasuj-ze-la sune-s sune-s-al premia b-aq'-ri.\\
Ali-\textsc{inter}(\textsc{lat}) \textsc{n}-know:\textsc{pfv}-\textsc{aor} Rasul.\textsc{obl}-\textsc{inter}-\textsc{el}  self.\textsc{obl}-\textsc{dat} self.\textsc{obl}-\textsc{dat}-\textsc{emph} money \textsc{n}-get:\textsc{pfv}-\textsc{nmlz}\\ 
\glt `Ali\textsubscript{i} found out from Rasul\textsubscript{y} that
he\textsubscript{y} got money.'
\z

An alternative explanation is that the complex pronoun in the subject
position in the subordinate clause serves as the real subject of the
clause and, unable to be bound within the local domain, takes the
closest argument outside its clause as an antecedent. However, there is
no evidence that an intensifier can serve as a subject of the clause.
%
\is{logophoric pronoun|)}
\is{reflexive pronoun|)}
\is{subject|)}

% 4
\section{Discourse usage}\label{discourse-usage}

In discourse the bare pronoun\footnote{There is evidence that the bare
  pronoun in its free logophoric function can be intensified with the
  suffix \emph{-al} without changing the reference of the pronoun. The corpus,
  however, does not provide appropriate examples.} can be used to refer
to the narrator of a story. In the following contexts, the bare pronoun
is used in various syntactic positions and does not have an antecedent
within the sentence\footnote{It can also be hypothesized that the bare
  pronoun in its free logophoric function may refer to other
  participants of the narrative. The texts from the corpus do not
  provide any evidence in support of this, however, and the topic thus
  requires further investigation.}.

% \noindent (corpus, Poisoning: 1.20)

\ea % {49}
\gll  sa‹r›i duc' d-uq-un-na k'ʷan ʡaj illi-šu.\\
‹\textsc{f}›self run \textsc{f1}-lv:\textsc{pfv}-\textsc{aor}-\textsc{ego} \textsc{quot} perhaps that-\textsc{ad}(\textsc{lat})\\
\glt `I\textsubscript{i} (the narrator) ran to her\textsubscript{y}.' (corpus, Poisoning: 1.20)

% \noindent (corpus, Poisoning: 1.8)

\ex % {50}
\gll  sune-jni i-ra k'ʷan abaj-la heš dursi꞊ra d-aχ-uwe d-uʔ-a-k'a ħu d-u-es ʡaj.\\
self.\textsc{obl}-\textsc{erg} say:\textsc{pfv}-\textsc{ego} \textsc{quot} mother-\textsc{gen} this girl꞊and \textsc{f1}-look.after:\textsc{pfv}-\textsc{aor.cvb} \textsc{f1}-be-\textsc{irr}-\textsc{cond} you.sg \textsc{f1}-be:\textsc{pfv}-\textsc{inf} perhars\\ 
\glt `She\textsubscript{i} (the narrator) said that, my\textsubscript{y}
daughter, you\textsubscript{y} better take care of her daughter.' (corpus, Poisoning: 1.8)


% \noindent (corpus, Poisoning: 1.17)

\ex % {51}
\gll  sune-jni i-ra k'ʷan marijan ħad d-ig-a-k'a d-uh-e ʡaj ħad ʡaˤχ-le b-uʔ-a-re.\\
self.\textsc{obl}-\textsc{erg} say:\textsc{pfv}-\textsc{ego} \textsc{quot} marijan you.sg.\textsc{dat}  \textsc{f1}-want:\textsc{pfv}-\textsc{irr}-\textsc{cond} \textsc{f1}-become.pfv-\textsc{imp} perhaps you.sg.\textsc{dat} good-\textsc{advz} \textsc{n}-be:\textsc{pfv}-\textsc{irr}-\textsc{pst}\\  
\glt `She\textsubscript{i} (the narrator) said: Marijam\textsubscript{y}, if
you\textsubscript{y} want (to do this) marry him, maybe it would be good
for you\textsubscript{y}.' (corpus, Poisoning: 1.17)

% \noindent (corpus, Poisoning: 1.32)

\ex % {52}
\gll  sune-s k'ʷan ʡaj urče c'a aq'-ur.\\
self.\textsc{obl}-\textsc{dat} \textsc{quot} perhaps in.heart(\textsc{lat}) fire pour:\textsc{pfv}-\textsc{aor}\\ 
\glt `She (the narrator) felt bad.' (corpus, Poisoning: 1.32)

% \noindent (corpus, Speaking Lak: 1.14)

\ex % {53}
\gll  hanna raχkʷar r-uh-ub-le umma r-uk'-uwe gʷa k'ʷan ʡaj sune-če hel xunul.\\
now man \textsc{f}-become:\textsc{pfv}-\textsc{aor}-\textsc{cvb} kiss \textsc{f}-\textsc{lv}:\textsc{ipfv}-\textsc{cvb.ipfv} \textsc{ptcl} \textsc{quot} perhaps self.\textsc{obl}-\textsc{super}(\textsc{lat}) this.here woman\\ 
\glt `Then the woman started to kiss him (the narrator).' (corpus, Speaking Lak: 1.14)
\z

% 5
\section{Intensifier}\label{intensifier}

The complex pronoun in Mehweb can be used as an intensifier. The
intensifier is used in adposition to its head, which it emphasizes (cf.\
\ref{ex:10:55}). This pronoun is formally identical to the reflexive
pronoun\footnote{The functions of intensification and reflexivization
  are similarly combined in personal pronouns followed by the suffix
  \emph{-al}; also cf.\ \tabref{tab:10:2}.}. The bare pronoun alone cannot be used as an
intensifier (see \ref{ex:10:56}).

\ea \label{ex:10:55} % {54}
\gll  it-ini sune-jni-jal d-erk-un χinč'-e.\\
this-\textsc{erg} self.\textsc{obl}-\textsc{erg}-\textsc{emph} \textsc{n}-eat:\textsc{pfv}-\textsc{aor} khinkal-\textsc{pl}\\ 
\glt `He\textsubscript{i} himself\textsubscript{i} ate all khinkals.'

\ex \label{ex:10:56} % {55}
\gll  di-ze iti-ze-la b-ah-ur-ra ʡali-ni cula aħin-i it-ini sune-jni-jal꞊ra maza b-erh-ri.\\
I.\textsc{obl}-\textsc{inter}(\textsc{lat}) this-\textsc{inter}-\textsc{el} \textsc{n}-know:\textsc{pfv}-\textsc{aor}-\textsc{ego} Ali-\textsc{erg} only {be}:\textsc{neg}-\textsc{atr} this-\textsc{erg} self.\textsc{obl}-\textsc{erg}-\textsc{emph}꞊and ram \textsc{n}-slaughter:\textsc{pfv}-\textsc{nmlz}\\ 
\glt `I found out from him\textsubscript{i} that not only
Ali\textsubscript{y} but he\textsubscript{i} himself\textsubscript{i}
slaughtered the ram.'

\ex % {56}
\gll  *di-ze iti-ze-la b-ah-ur-ra ʡali-ni cula aħin-i it-ini sune-jni꞊ra maza b-erh-ri.\\
I.\textsc{obl}-\textsc{inter}(\textsc{lat}) this-\textsc{inter}-\textsc{el} \textsc{n}-know:\textsc{pfv}-\textsc{aor}-\textsc{ego} Ali-\textsc{erg} only {be}:\textsc{neg}-\textsc{atr} this-\textsc{erg} self.\textsc{obl}-\textsc{erg}꞊and ram \textsc{n}-slaughter:\textsc{pfv}-\textsc{nmlz}\\
\glt Intended: `I found out from him\textsubscript{i} that not only
Ali\textsubscript{y} but he\textsubscript{i} himself\textsubscript{i}
slaughtered the ram.'
\z

The complex pronoun may intensify an overt NP (cf.\ \ref{ex:10:58}), demonstratives
(cf.\ \ref{ex:10:59}), as well as pro-dropped pronouns in the subject position (cf.\
\ref{ex:10:60}). The intensifier agrees in number, case and gender with its head. It can
be used in all syntactic positions, including subject, P and other
positions.

\ea \label{ex:10:58} % {57}
\gll  rasuj-ni sune-s-al muħammadi-s eža as-ib.\\
Rasul.\textsc{obl}-\textsc{erg} self.\textsc{obl}-\textsc{dat}-\textsc{emph} muhammad-\textsc{dat} goat take:\textsc{pfv}-\textsc{aor}\\  
\glt `Rasul\textsubscript{i} bougth to Muhammad\textsubscript{y}
himself\textsubscript{y} a goat.'

\ex \label{ex:10:59} % {58}
\gll  it-ini sune-jni-jal d-erk-un χinč'-e.\\
this-\textsc{erg} self.\textsc{obl}-\textsc{erg}-\textsc{emph} \textsc{n}-eat:\textsc{pfv}-\textsc{aor} khinkal-\textsc{pl}\\ 
\glt `He\textsubscript{i} himself\textsubscript{i} ate all khinkals.'

\ex \label{ex:10:60} % {59}
\gll  sune-jni-jal d-erk-un χinč'-e.\\
self.\textsc{obl}-\textsc{erg}-\textsc{emph} \textsc{n}-eat:\textsc{pfv}-\textsc{aor} khinkal-\textsc{pl}\\
\glt `(He) himself ate the khinkals.'
\z

Some speakers are reluctant to accept intensification of NPs with low
animacy:

\ea % {60}
\gll  \textsuperscript{?}rasuj-ni muħammad-i-s sa‹b›i-jal eža as-ib.\\
Rasul.\textsc{obl}-\textsc{erg} muhammad-\textsc{obl}-\textsc{dat} ‹\textsc{n}›self-\textsc{emph} goat take:\textsc{pfv}-\textsc{aor}\\ 
\glt `Rasul bought to Muhammad this the very goat.'
\z

The intensifier may be preposed to its antecedent:

% \noindent (corpus, The story of Akula Ali: 1.7)

\ea % {61}
\gll  sa‹w›i-jal wazil-li b-arg-ib k'ʷan ʡilla꞊ra.\\
‹\textsc{m}›self-\textsc{emph} chief-\textsc{obl}(\textsc{erg}) \textsc{n}-find:\textsc{pfv}-\textsc{aor} \textsc{quot} reason꞊and\\
\glt `The chief\textsubscript{i} himself\textsubscript{i} found the reason.'
(corpus, The story of Akula Ali: 1.7 \citep{magometov1982})
\z

The intensifier can co-occur with complex pronouns used as reflexives,
as in (\ref{ex:10:63}) and (\ref{ex:10:64}). In such contexts, they seem to show a free relative
order. However, (\ref{ex:10:65}) shows that the compound consisting of two complex
pronouns cannot be split.

\ea \label{ex:10:63} % {62}
\gll  rasuj-ze sune-ze-l sa‹w›i-jal gu-b.\\
Rasul.\textsc{obl}-\textsc{inter}(\textsc{lat}) self.\textsc{obl}-\textsc{inter}(\textsc{lat})-\textsc{emph} ‹\textsc{m}›self-\textsc{emph} see:\textsc{pfv}-\textsc{aor} \\ 
\glt `Rasul\textsubscript{i} saw himself\textsubscript{i}.'

\ex \label{ex:10:64} % {63}
\gll  rasuj-ze sa‹w›i-jal sune-ze-l gu-b.\\
Rasul.\textsc{obl}-\textsc{inter}(\textsc{lat}) ‹\textsc{m}›self-\textsc{emph} self.\textsc{obl}-\textsc{inter}(\textsc{lat})-\textsc{emph} see:\textsc{pfv}-\textsc{aor}\\ 
\glt `Rasul\textsubscript{i} saw himself\textsubscript{i}.'

\ex \label{ex:10:65} % {64}
\gll  *rasuj-ze sune-ze-l gu-b sa‹w›i-jal.\\
Rasul.\textsc{obl}-\textsc{inter}(\textsc{lat}) self.\textsc{obl}-\textsc{inter}(\textsc{lat})-\textsc{emph}   see:\textsc{pfv}-\textsc{aor} ‹\textsc{m}›self-\textsc{emph}\\
\glt `Rasul\textsubscript{i} saw   himself\textsubscript{i}.'
\z

The intensifier can also be combined with a bare pronoun and can either
precede or follow it, with no semantic contrast (cf.\ \ref{ex:10:66} and \ref{ex:10:67}).

\ea \label{ex:10:66} % {65}
\gll  rasuj-s dig-uwe le-b sawi sune-če-l ħule w-iz-es.\\
Rasul.\textsc{obl}-\textsc{dat} want:\textsc{ipfv}-\textsc{cvb.ipfv} \textsc{aux}-\textsc{n} ‹\textsc{m}›self self.\textsc{obl}-\textsc{super}(\textsc{lat})-\textsc{emph} look \textsc{m}-\textsc{lv}:\textsc{pfv}-\textsc{inf}\\
\glt `Rasul\textsubscript{i} wants to look at himself\textsubscript{i}.'

\ex \label{ex:10:67} % {66}
\gll  rasuj-s dig-uwe le-b sune-če-l sa‹w›i ħule w-iz-es.\\
Rasul.\textsc{obl}-\textsc{dat} want:\textsc{ipfv}-\textsc{cvb.ipfv} \textsc{aux}-\textsc{n} self.\textsc{obl}-\textsc{super}(\textsc{lat})-\textsc{emph} ‹\textsc{m}›self look \textsc{m}-\textsc{lv}:\textsc{pfv}-\textsc{inf}\\
\glt `Rasul\textsubscript{i} wants to look at
himself\textsubscript{i}.'
\z

The intensifier can take the subject position in the subordinate clause
since subject pro-drop is also acceptable in subordinate clauses (cf.\
\ref{ex:10:43}–\ref{ex:10:46} above). The reference of the intensifier in subject
position is discussed in \sectref{non-subject-orientedness}.

% 6.

\section{Resumptive}\label{resumptive}

\is{resumptive pronoun|(}

The resumptive function of the self-pronoun is discussed in \citet{lander-kozhukhar2015}.
Resumptive pronouns are optionally used in the position
that is relativized (cf.\ \ref{ex:10:68}, \ref{ex:10:69}).

\ea \label{ex:10:68} % {67}
\gll  nu-ni ču-s kung gib-i ule b-aˤq'-un uškuj-ħe.\\
I-\textsc{erg} self.\textsc{pl}.\textsc{obl}-\textsc{dat} book give:\textsc{pfv}-\textsc{atr} child.\textsc{pl} \textsc{hpl}-go:\textsc{pfv}-\textsc{aor} school-\textsc{in}(\textsc{lat})\\ 
\glt `The children\textsubscript{i} to whom\textsubscript{i} I gave a book
went to school.'

\ex \label{ex:10:69} % {68}
\gll  šejtan ču-ze gu-b-i buk'unu-me uruχ b-aˤq-ib.\\
demon self.\textsc{pl}.\textsc{obl}-\textsc{inter}(\textsc{lat}) see:\textsc{pfv}-\textsc{aor}-\textsc{atr} shepherd-\textsc{pl} be.afraid \textsc{hpl}-\textsc{lv}:\textsc{pfv}-\textsc{aor}\\ 
\glt `The shepherds\textsubscript{i} who\textsubscript{i} saw a demon were
scared.'
\z

In resumptive contexts, the self-pronoun may also attach the suffix
\emph{-al}. As a result, the relativized argument is emphasized (cf.\ \ref{ex:10:70}
and \ref{ex:10:71}).

\ea \label{ex:10:70} % {69}
\gll  nu-ni sune-ze arc g-ib-i insaj-ni nab arc ħa-lug-an.\\
I-\textsc{erg} self.\textsc{obl}-\textsc{inter}(\textsc{lat}) money give:\textsc{pfv}-\textsc{aor}-\textsc{atr} man.\textsc{obl}-\textsc{erg} I.\textsc{dat} money \textsc{neg}-give:\textsc{ipfv}-\textsc{hab}\\ 
\glt `The man\textsubscript{i} to whom\textsubscript{i} I gave the money
doesn't give it back to me.'

\pagebreak

\ex \label{ex:10:71} % {70}
\gll  nu-ni sune-ze-l arc g-ib-i insaj-ni nab arc ħa-lug-an.\\
I-\textsc{erg} self.\textsc{obl}-\textsc{inter}(\textsc{lat}) money give:\textsc{pfv}-\textsc{aor}-\textsc{atr} man.\textsc{obl}-\textsc{erg} I.\textsc{dat} money \textsc{neg}-give:\textsc{ipfv}-\textsc{hab}\\ 
\glt `This very man\textsubscript{i} to whom\textsubscript{i} I gave money
doesn't give me them back.'
\z

Some consultants tend to use resumptives only with animate relative
heads (\ref{ex:10:72} and \ref{ex:10:73}).

\ea \label{ex:10:72} % {71}
\gll  \textsuperscript{?}sune-s ʡadidi ħark'ʷ b-aš-uwe le-b-i qali le-b rasuj-ja.\\
self.\textsc{obl}-\textsc{dat} behind river \textsc{n}-go:\textsc{ipfv}-\textsc{cvb.ipfv} \textsc{aux}-\textsc{n}-\textsc{atr} house {be}-\textsc{n} Rasul.\textsc{obl}-\textsc{gen}\\ 
\glt `The house\textsubscript{i} behind which\textsubscript{i} the river
flows belongs to Rasul.'

\ex \label{ex:10:73} % {72}
\gll  ʡadidi ħark'ʷ b-aš-uwe le-b-i qali le-b rasuj-ja.\\
behind river \textsc{n}-flow:\textsc{ipfv}-\textsc{cvb.ipfv} \textsc{aux}-\textsc{n}-\textsc{atr} house \textsc{aux}-\textsc{n} Rasul.\textsc{obl}-\textsc{gen}\\ 
\glt `The house\textsubscript{i} behind which\textsubscript{i} there is a
river belongs to Rasul.'
\z

For further discussion on resumptives see \citet{lander-kozhukhar2015}.
%
\is{resumptive pronoun|)}


% 7.
\section{Conclusion}

In this paper, I have considered the form and functions of the
pronominal stem \emph{sa}‹\textsc{cl}›\emph{i} in Mehweb. This stem has
at least the following functions: reflexive and long-distant reflexive,
logophoric (including free logophoric), intensifier and resumptive.
These functions, which are distinct from both syntactic and semantic
perspectives, show different constraints on their antecedents.

The complex pronoun functions as a locally bound reflexive and may
occupy any non-subject slots. The intensifier pronoun is homophonous to
the reflexive and receives the same case, number and gender values as its
head. The possible antecedents of an intensifier include locally bound
reflexives, long-distance reflexives and logophoric pronouns; it can
also be pro-dropped.

According to \citet{reuland2011} and \citet{sells1987}, logophoric pronouns are
pronouns used in finite subordinate clauses embedded under predicates of
speech and mental experience. For \citet{clements1975} and \citet{toldova1999}, the
main function of the logophoric pronoun is to define the point of view.
There are no typologically universal constraints on the syntactic
position the logophoric pronoun, but there is a strong tendency for the
antecedent to be in the subject position of the embedded clause. \citet{cole-etal2000}
however, discussing long-distance reflexives, argue
that these take either subject or non-subject position within non-finite
subordinate clauses. They also argue that long-distance reflexives
manifest subject orientation: their antecedents have to be subjects of
the main clause.

The pronoun \emph{sa}‹\textsc{cl}›\emph{i} covers both functions and
fits both the description of the logophoric pronoun and that of the
long-distance reflexive. Therefore, I suggest that in Mehweb, there is
neither a morphological nor a (sharp) syntactic distinction between
logophorics and long-distance reflexives.

\section*{List of abbreviations}

\begin{longtable}[l]{@{}ll@{}}
\textsc{ad}	& spatial domain near the landmark \\
\textsc{advz}	& adverbializer \\
\textsc{aor}	& aorist \\
\textsc{appr}	& apprehensive \\
\textsc{atr}	& attributivizer \\
\textsc{aux}	& auxiliary \\
\textsc{cl}	& gender (class) agreement slot \\
\textsc{comit}	& comitative \\
\textsc{cond}	& conditional \\
\textsc{cvb}	& converb \\
\textsc{dat}	& dative \\
\textsc{ego}	& egophoric \\
\textsc{el}	& motion from a spatial domain \\
\textsc{emph}	& emphasis (particle) \\
\textsc{erg}	& ergative \\
\textsc{ess}	& static location in a spatial domain \\
\textsc{f}	& feminine (gender agreement) \\
\textsc{f1}	& feminine (unmarried and young women gender prefix) \\
\textsc{gen}	& genitive \\
\textsc{hab}	& habitual (durative for verbs denoting states) \\
\textsc{hpl}	& human plural (gender agreement) \\
\textsc{imp}	& imperative \\
\textsc{in}	& spatial domain inside a (hollow) landmark \\
\textsc{inf}	& infinitive \\
\textsc{inter}	& spatial domain between multiple landmarks \\
\textsc{ipfv}	& imperfective (derivational base) \\
\textsc{irr}	& irrealis (derivational base) \\
\textsc{lat}	& motion into a spatial domain \\
\textsc{lv}	& light verb \\
\textsc{m}	& masculine (gender agreement) \\
\textsc{n}	& neuter (gender agreement) \\
\textsc{neg}	& negation (verbal prefix) \\
\textsc{nmlz}	& nominalizer \\
\textsc{nom}	& nominative \\
\textsc{npl}	& non-human plural (gender agreement) \\
\textsc{obl}	& oblique (nominal stem suffix) \\
\textsc{pfv}	& perfective (derivational base) \\
\textsc{pl}	& plural \\
\textsc{pst}	& past \\
\textsc{ptcl}	& particle \\
\textsc{purp}	& purposive converb \\
\textsc{pv}	& preverb (verbal prefix) \\
\textsc{quot}	& quotative (particle) \\
\textsc{super}	& spatial domain on the horizontal surface of the landmark \\
\end{longtable}


\nocite{asher-kumari1997,ogawa1998}
\printbibliography[heading=subbibliography,notkeyword=this]

\end{document}

%%% Local Variables:
%%% mode: latex
%%% TeX-master: "../main"
%%% End:
