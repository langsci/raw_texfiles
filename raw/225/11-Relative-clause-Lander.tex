\documentclass[output=paper]{langsci/langscibook} 
\ChapterDOI{10.5281/zenodo.3402074}

% Chapter 11

\title{Relative clause and resumptive pronouns in Mehweb}

\author{Yury Lander\affiliation{National Research University Higher School of Economics}\lastand Aleksandra Kozhukhar\affiliation{National Research University Higher School of Economics}}

\abstract{East Caucasian relative clause constructions (RCCs) are sometimes viewed as constructed mainly on the basis of semantic and pragmatic information. In this paper, we consider RCCs in Mehweb and argue that, despite the fact that the interpretation of some of them may rely exclusively on semantics, syntactic mechanisms may also come into play in their organization. We present evidence that Mehweb has grammaticalized the resumptive use of reflexive pronouns, which thus can be contrasted with other uses of reflexive pronouns due to the restrictions on animate antecedents observed only in RCCs.

\emph{Keywords}: relative clause, relativization, resumptive pronoun.}

\begin{document}
\maketitle

\section{Introduction}

Relativization is usually thought of as a mechanism which operates on an
argument or an adjunct of a subordinate clause (see, for example, \citealt{devries2002}).
For example, in \emph{the paper we are writing~\_\_}
the relativized argument is the direct object of the verb, while
\emph{the person that \_\_ wrote this sentence} presupposes that the
relativized argument is the verb's subject.\footnote{In both examples a
  gap is shown in the place of the relativized argument.} Note that many
scholars of relative clause constructions (RCCs) think of relativized
arguments and adjuncts as syntactic positions and not as semantic roles.
Indeed, studies of RCCs have revealed a number of restrictions on their
formation which clearly have\pagebreak[3] syntactic nature. These restrictions
include, for instance, the continuous distribution of relative
constructions along the Noun Phrase Accessibility Hierarchy (NPAH)\linebreak
\textsc{Subject~› Direct object › Indirect object › Oblique object ›
Possessor › Object of comparison} \citep{keenan-comrie1977}\footnote{This
  hierarchy was later extended and modified (for example, for ergative
  languages it was argued that the transitive undergoer has preference
  over the ergative argument); see \citet[211ff]{lehmann1984}, \citet{liao2000}, and
  specifically for Daghestanian languages, \citet{lyutikova1999,lyutikova2001}.} and
non-relativizability of nominals embedded in syntactic islands, like
indirect questions and parts of coordinating constructions \citep{ross1967}.

The universality of this conception was called into question by
% Comrie
\citet{comrie1996,comrie1998}, who proposed, following
% Matsumoto
\citet{matsumoto1988,matsumoto1997}, that some
languages may construct what is, in their descriptions, usually
considered an RCC on a semantic rather than on a syntactic basis. As was
shown in the above-mentioned works and the subsequent literature (see
especially the volume \citealt{matsumoto-etal2017}), such languages only need
to establish a plausible semantic link between the head of the noun
phrase and the subordinate clause which would be sufficient for the
characterization of this head. This link sometimes involves an argument
or an adjunct of the subordinate clause but it need not necessarily.
Hence a new term was coined for this phenomenon, namely
\emph{generalized noun modifying clause constructions}. Naturally, such
constructions do not display the syntactic restrictions proposed for
languages with ``canonical'' relative clauses.

As we will see below, the contrast between RCCs proper and generalized
noun-modifying clause constructions is not a clear-cut one. That is why
in this paper we will use the terms \emph{relative clause} and
\emph{relative clause construction} irrespectively of our stance as to
the mechanisms behind the attributive patterns we discuss.\footnote{The
  terms \emph{attributive clause} and \emph{noun-modifying clause
  construction} used in literature are misleading, since
  cross-linguistically relatives do not always function as syntactic
  attributes/modifiers of nouns (cf.\ internally-headed RCCs or the
  amazingly wide use of RCCs without ``head'' nouns in some languages).}
Nonetheless, we will distinguish between \emph{syntactically-oriented
RCCs} and \emph{semantically-oriented RCCs} depending on whether or not,
in a given case or set of cases, the syntactic information is relevant.
If a construction contains a grammaticalized means intended for
determining the relativized argument and displays clear syntactic
constraints, it is considered a syntactically-oriented RCC. Otherwise,
it may be considered semantically-oriented.

% \largerpage

This paper presents a preliminary description of Mehweb RCCs in the
perspective outlined above. At the clause level, Mehweb, as other Dargwa
languages, is double-marking: it has case marking and verb agreement.
Both kinds of marking display the ergative system, a remarkable
exception being person marking, the rules for which vary across Dargwa
varieties (\citealt{sumbatova2011}; for discussion of the Mehweb system of
personal agreement, see \citealt{ganenkov2019} [this volume]). As for word order, Mehweb
can be characterized as left-branching, although showing considerable
freedom in independent clauses.


This paper is based on our fieldwork in Mehweb in 2013, 2015 and 2016.
Most data were obtained through elicitation sessions. The structure of
the paper is as follows: in \sectref{east-caucasian-relative-clauses} we describe the context in which
we discuss Mehweb RCCs; in \sectref{relatives-in-mehweb} we provide background information
on relative clauses in this language; \sectref{syntactic-orientedness} is devoted to certain
aspects of Mehweb RCCs that point to their syntactic nature; and
\sectref{towards-an-explanation-of-the-Mehweb-pattern} discusses these data from a theoretical point of view. The last
section presents conclusions.

% 2.
\section{East Caucasian relative clauses}\label{east-caucasian-relative-clauses}

As is typical for a left-branching language, the basic RCC in East
Caucasian languages involves a relative clause preceding its head (if
any).\footnote{A survey of the data available for East Caucasian
  relatives can be found in \citet{barylnikova2015}.} In grammars, the form of
the verbal predicate of the subordinate clause is traditionally
described as a participle, although its real place in the verb paradigm
varies. The difficulties in the attribution of these forms are related
primarily to the fact that in many languages they coincide with some
finite forms.

At first glance, East Caucasian RCCs seem like good candidates to be
considered semantically-oriented. \citet[33]{kibrik1980} noticed that the
syntactic characteristics of the relativized argument are not crucial
for these constructions. Indeed, the role of the relativized argument
cannot be deduced from the form of the predicate of the relative clause,
neither can it be unambiguously recovered on the basis of any other
grammatical property of the construction. There are no dedicated
relative pronouns that mark the relativized argument, and the absence of
a corresponding NP cannot serve as a reliable clue, since East Caucasian
languages easily omit argument NPs even in independent clauses. Hence
\citet{comrie-polinsky1999}, who analyzed RCCs in \ili{Tsez}, argued that they
may be constructed on the basis of semantic frames, and \citet{comrie-etal2017}
continued this line of analysis for \ili{Hinuq} and \ili{Bezhta}, the
languages of the same Tsezic branch of East Caucasian as Tsez.
% Daniel \& Lander 
\citet{daniel-lander2008, daniel-lander2010}
  also proposed that RCCs in East Caucasian languages
are not based on syntactic information. In this section we will
illustrate the argumentation concerning these points with examples from
\ili{Tanti Dargwa}, a language belonging to the same branch of the family as
Mehweb (see \citealt{sumbatova-lander2014} for details).

In general, Tanti Dargwa does not show any restrictions on what
grammatical role is relativized. In this language, not only does the RCC
relativize all roles in NPAH, but it is also not sensitive to syntactic
islands\is{syntactic island}. The following examples (both elicited) demonstrate what should
presumably be described as relativization out of relative clauses and
coordination constructions:\footnote{For the reasons discussed in the
  paper, glossing occasionally follows rules that are different
  from other papers of the volume.}

\ea % {0}
\gll dam č-ib-se kːata b-ibšː-ib хːunul simi r-ač'-ib.\\
  I.\textsc{dat} bring:\textsc{pfv}-\textsc{aor}-\textsc{atr} cat  \textsc{n}-run.away:\textsc{pfv}-\textsc{aor} woman anger \textsc{f}-enter:\textsc{pfv}-\textsc{aor}\\
\glt `The woman such that the cat that she brought to me ran away got angry.'

\ex \label{ex:11:2} % {1}
\gll  aħmad-li꞊ra sun-ni꞊ra mura d-ertː-ib admi dila χːutːu꞊sa-j.\\
Ahmad-\textsc{erg}꞊\textsc{add} self-\textsc{erg}꞊\textsc{add} hay \textsc{npl}-mow:\textsc{pfv}-\textsc{aor} man I.\textsc{gen} father.in.law꞊\textsc{cop}-\textsc{m}\\
\glt `The man with whom Ahmad mowed the hay (lit., Ahmad and who mowed the
hay) is my father-in-law.'
\z

Therefore, it seems that \ili{Tanti Dargwa} lacks syntactic constraints on
relativization. Moreover, a relative clause can appear even if there is
no argument in the subordinate part that could be relativized. Cf.\
(\ref{ex:11:3}):\footnote{The presence of the \isi{attributive suffix} on the predicate of
  the relative clause in (\ref{ex:11:3}), which at first glance makes it different
  from the previous examples, is not related to any difference in the
  mechanisms of constructing the relation between the head and the
  relative clause. For a discussion of the distribution of the
  attributive suffix in Tanti Dargwa, see \citet{lander2014}.}

\ea \label{ex:11:3} % {2}
\gll  ʕuˤ dam muher-li-cːe-r r-iž-ib-se dila ʡamru alžana꞊ʁuna꞊sa-tːe.\\
you.sg I.\textsc{dat} dream-\textsc{obl}-\textsc{inter}-\textsc{f}(\textsc{ess}) \textsc{f1}-sit:\textsc{pfv}-\textsc{aor}-\textsc{atr} I.\textsc{gen} life heaven꞊like꞊\textsc{cop}-\textsc{npl}+\textsc{pst}\\
\glt `My life when I dreamt about you (lit., when you were sitting in my
dream) was heaven-like.'
\z

It is impossible to describe (\ref{ex:11:3}) as a result of any syntactic operation
which deals with an argument of the relative clause. Hence, this RCC is
likely to be semantically-oriented.

% \pagebreak

Still, it is doubtful that East Caucasian relatives never rely on
syntactic information. As \citet{daniel-lander2013} argued, the frequency
of relativization of a syntactic position may depend on whether a
language displays ergative features or not, even within this family. It
may be that syntax is still engaged, even though, sometimes, these
relatives only rely on semantics and pragmatics.

In addition, constraints on relativization have been reported for some
East Caucasian languages. For example, according to \citet[215]{tatevosov1996}, \ili{Godoberi} does not relativize possessors, objects of comparison and
objects of postpositions. \citet{mutalov-sumbatova2003} note that in
\ili{Itsari Dargwa} ``[r]elativization is impossible only for constituents
of coordinate clauses and at least doubtful for constituents of
adverbial clauses''. Lyutikova (\citeyear{lyutikova1999, lyutikova2001}) reports that \ili{Tsakhur} and
\ili{Bagwalal} prohibit relativization for the positions mentioned for Itsari
as well. Moreover, although the syntactic limits of relativization are
always quite loose, it is worth noting that informants do not always
accept relativization of all syntactically peripheral participants
without an appropriate context, even in languages whose RCCs are
commonly believed to be semantically-oriented.

Another problem for a purely semantic treatment is posed by the fact
that in many East Caucasian languages the relativized argument can be
expressed within a relative clause by a \isi{reflexive pronoun}, as in (\ref{ex:11:4}).
Such pronouns look like \isi{resumptive pronoun}s, which directly point to the
\emph{syntactic} position that is relativized.

\ea \label{ex:11:4} % {3}
\gll  du \textup(sun-ni-šːu\textup) qʼʷ-aˤn-se qali\\
I self-\textsc{obl}-\textsc{ad}(\textsc{lat}) go:\textsc{ipfv}-\textsc{prs}-\textsc{atr} house\\
\glt  `the house where I am going'
\z

Still, these pronouns differ from typical resumptives in various
significant ways.

First, to refer to relativized arugments, East Caucasian languages use
reflexive pronouns, while typical resumptives cited in the typological
literature are non-reflexive.\footnote{Note, however, that reflexives
  used as resumptives are found outside the East Caucasian family as
  well. For example, \citet{lee2004} provides a detailed discussion of the
  resumptive use of a reflexive pronoun in \ili{Korean}, \citet{csató-uchturpani2010}
  describe reflexive resumptives in \ili{Uyghur}, and \citet[219]{johanson-csató1998}
  report the resumptive function of reflexives in \ili{Turkish}.}
Yet the appearance of reflexive pronouns in RCCs may be related to the
fact that reflexive pronouns in this family have very wide distribution:
for example, they are used as \isi{logophoric pronoun}s or in independent
clauses both as \isi{intensifier}s and as pronominals \citep{testelets-toldova1998}.
This suggests that reflexive pronouns in East Caucasian languages
are much more neutral means of pronominal reference than their
counterparts in Standard Average European languages.

Second, East Caucasian languages sometimes allow resumptive reflexive
pronouns in the most privileged syntactic positions occupying the top of
NPAH, such as those of the intransitive subject (\ref{ex:11:5}), transitive actor
(\ref{ex:11:6}) and transitive undergoer (\ref{ex:11:7}). Cf.\ the following \ili{Tanti Dargwa}
examples:

\ea \label{ex:11:5} % {4}
\gll \(sa‹r›i\) dam-šːu r-ačʼ-ib rursːi\\
 self‹\textsc{f}› I.\textsc{obl}-\textsc{ad}(\textsc{lat})   \textsc{f}-come:\textsc{pfv}-\textsc{aor} girl\\
\glt  `the girl that came to me'

\ex \label{ex:11:6} % {5}
\gll \(sun-ni\) čutːu b-erkː-un umra\\
self-\textsc{erg} chudu \textsc{n}-eat:\textsc{pfv}-\textsc{aor} neighbour\\
\glt `the neighbour who ate chudu'

\ex \label{ex:11:7} % {6}
\gll  \(sa‹b›i\) umra-li b-erkː-un čutːu\\
self‹\textsc{n}› neighbour-\textsc{erg} \textsc{n}-eat:\textsc{pfv}-\textsc{aor} chudu\\
\glt `the chudu (a kind of pie) that the neigbor ate'
\z

Typical \isi{resumptive pronoun}s in relative clauses prefer the positions
that occur lower in syntactic hierarchies \citep[92]{keenan-comrie1977,maxwell1979}.
Hence, East Caucasian resumptives are different from
typical resumptives.\footnote{Again, there do exist languages which
  allow resumptives in the subject position, but these uses are usually
  considered exceptional. We do not have information on the degree of
  markedness of such uses as (\ref{ex:11:5}–\ref{ex:11:7}) in East Caucasian languages.}

\citet{daniel-lander2008} suggested that reflexives in relatives do not
serve to mark the relativized position, i.e.\ they are only anaphoric
devices, independent of relativization (cf.\ also \citealt[133]{comrie-etal2017}).
If so, their existence does not contradict the idea that East
Caucasian RCCs do not apply to syntactic information. The data from
Mehweb we proceed to present make the issue of the use of reflexives
more intriguing and return us to the idea that, after all, these can be
treated as resumptives.

% 3. 
\section{Relatives in Mehweb: first glance}\label{relatives-in-mehweb}

The basic RCC in Mehweb Dargwa involves a relative clause which precedes
the head of the noun phrase, if there is one. The predicate of the
relative clause is marked with an \isi{attributive suffix}, which has
allomorphs \emph{-il}, \emph{-i}, and \emph{-l}. The same
suffix is
found with some other attributes, such as adjectival attributes. Some
examples of RCCs are given in (\ref{ex:11:8}–\ref{ex:11:9}):

\ea \label{ex:11:8} % {7}
\gll   naˤʁ iz-u-l insan\\
 hand hurt:\textsc{ipfv}-\textsc{prs}-\textsc{atr} person\\
\glt   `a person whose hand hurts'

\ex \label{ex:11:9} % {8}
\gll  nu q'-oˤwe d-uʔ-ub-i huni\\
 I go:\textsc{ipfv}-\textsc{cvb.ipfv}  \textsc{f1}-be:\textsc{pfv}-\textsc{aor}-\textsc{atr} road\\
\glt  `the road I was going with'
\z

According to \citet[112–115]{magometov1982} and \citet[105–107]{khajdakov1985},
Mehweb distinguishes between three types of \isi{participle} with respect to
the stem they are formed with and the variant of the \isi{attributive suffix}
they adjoin; cf.\ \tabref{tab:11:1}.

\begin{table}[h]
  % Table 1.
  \caption{Participles in Mehweb Dargwa}\label{tab:11:1}

\begin{tabular}{@{}lll@{}}
\toprule
{participle} & {base} & {marker}\tabularnewline \midrule
Past & aorist & \emph{-i}\tabularnewline
Present & bare verbal stem + epenthetic vowel \emph{-i-} &
\emph{-u-l}\tabularnewline
Future & infinitive & \emph{-i}\tabularnewline
\bottomrule
\end{tabular}
\end{table}

While the past and future participles are morphologically transparent
and include just the corresponding base and the attributive suffix, the
present participle contains the former marker of the present tense
\emph{-u}, which is found in present converbs.\footnote{Michael Daniel
  (pers. com.) noted that it is most likely that imperfective converbs\is{converb, imperfective}
  are actually derived from imperfective participles.} While it is
glossed as \textsc{prs} in this paper,\footnote{Note that in using this
  gloss for \emph{-u}, our paper differs from other papers of this
  volume.} one should bear in mind that its distribution is limited to
few non-finite forms and it can be used as a marker of a relative tense
rather than as an absolute tense.\footnote{The finite present tense is
  expressed periphrastically by a combination of the present converb
  with a copula.}

We take the participles listed above as the canonical predicates of
relative clauses. However, it should be noted that the predicates of
relative clauses are not confined to these participles. For example, we
have RCCs where the attributive suffix is added to the \isi{copula}/existential verb, as in (\ref{ex:11:10}–\ref{ex:11:11}):

\pagebreak

\ea \label{ex:11:10} % {9}
\gll  kʷiha b-erh-u-we le-w-i adami-li-ze nu g-ub.\\
 ram \textsc{n}-slaughter:\textsc{pfv}-\textsc{prs}-\textsc{cvb} \textsc{aux}-\textsc{m}-\textsc{atr}   man-\textsc{obl}-\textsc{inter}(\textsc{lat}) I see:\textsc{pfv}-\textsc{aor}\\
\glt   `The man who had slaughtered a ram saw me.'\footnote{The example is
    additionally interesting because it relativizes one of the arguments
    of the so-called \isi{biabsolutive construction}. Cf.\ the original
    independent construction:\vspace{-\jot}

    \begin{exe}
      \exi{(i)}
      \gll adami kʷiha b-erh-u-we le-w\\
      man ram \textsc{n}-slaughter:\textsc{pfv}-\textsc{prs}-\textsc{cvb} \textsc{aux}-\textsc{m}\\
      \glt `The man slaughtered a ram.'
    \end{exe}

    \removelastskip
    \vspace{-\baselineskip}
  }

% \addtocounter{equation}{1}

\ex \label{ex:11:11} % {11}
\gll  qali le-b-i dursi d-ak'-ib.\\
  house {be}-\textsc{n}-\textsc{atr} girl \textsc{f1}-come:\textsc{pfv}-\textsc{aor}\\
\glt `The girl who has her own house came.'
\z

% \footnotetext{The example is
%     additionally interesting because it relativizes one of the arguments
%     of the so-called \isi{biabsolutive construction}. Cf.\ the original
%     independent construction:

%     \begin{exe}
%       \exi{(i)}
%       \gll adami kʷiha b-erh-u-we le-w\\
%       \footnotesize man ram \textsc{n}-slaughter:\textsc{pfv}-\textsc{prs}-\textsc{cvb} \textsc{aux}-\textsc{m}\\
%       \glt \footnotesize `The man slaughtered a ram.'
%     \end{exe}

%     \removelastskip
%     \vspace{-\baselineskip}
%   }


As shown by examples, the relativized argument need not be expressed
overtly within the relative clause. As in Tanti Dargwa, it is not
difficult to construct an example where the relation between the
relative clause and the head must be established by the context:

\ea % {11}
\gll  nu-ni b-erk-un-na itti b-urʁ-es b-aq-ib-i t'ult'.\\
  I-\textsc{erg} \textsc{n}-eat:\textsc{pfv}-\textsc{aor}-\textsc{ego} that   \textsc{hpl}-fight:\textsc{ipfv}-\textsc{inf}   \textsc{hpl}-let:\textsc{pfv}-\textsc{aor}-\textsc{atr} bread\\
\glt `I ate the bread which served as the reason for them to fight.'
\z


If the relativized argument can be reconstructed, it usually can be
expressed with a pronoun \emph{sa}‹\textsc{cl}›\emph{i} (here
\textsc{cl} is a gender marker), which has several suppletive forms
and whose partial paradigm is given in \tabref{tab:11:2}.
\begin{table}[h]
  % Table 2.
  \caption{Case-number-gender forms of the pronoun \emph{sa}‹\textsc{cl}›\emph{i}}
\label{tab:11:2}
\begin{tabular}{@{}llllll@{}}
\toprule
& {\textsc{nom}} & {\textsc{erg}} & {\textsc{gen}}
& {\textsc{dat}} & {\textsc{inter}-\textsc{lat}}\tabularnewline \midrule
{\textsc{3sg}} & {\textsc{m}} & \emph{sa‹w›i} & \emph{sune-jni} &
\emph{sune-la} & \emph{sune-s} \tabularnewline
& {\textsc{f}/\textsc{f1}} & \emph{sa‹r›i} & & &\tabularnewline
& {\textsc{n}} & \emph{sa‹b›i} & & &\tabularnewline
{\textsc{3pl}} & {\textsc{hpl}} & \emph{sa‹b›i} & \emph{ču-ni} & \emph{ču-la} & \emph{ču-s} \tabularnewline
& {\textsc{npl}} & \emph{sa‹r›i} & & &\tabularnewline
\bottomrule \hlx{v}
\end{tabular}
\end{table}
This pronoun also serves
as a \isi{reflexive pronoun} (both local and long-distance), as a \isi{logophoric
pronoun}, and as an \isi{intensifier} (see \citealt{kozhukhar2019} [this volume]).



Some examples of the use of \emph{sa}‹\textsc{cl}›\emph{i} as a
resumptive\is{resumptive pronoun} are given below. In (\ref{ex:11:13}) it appears in the indirect object
position, in (\ref{ex:11:14}) it serves as the possessor of the intransitive
subject, and in (\ref{ex:11:15}) it refers to the experiencer with the experiential
verb:

\ea \label{ex:11:13} % {12}
\gll  nu-ni ču-s kung gib-i ule b-aˤq'-un uškuj-ħe.\\
I-\textsc{erg} self.\textsc{pl}.\textsc{obl}-\textsc{dat} book give:\textsc{pfv}-\textsc{atr} child.\textsc{pl} \textsc{hpl}-go:\textsc{pfv}-\textsc{aor} school.\textsc{obl}-\textsc{in}(\textsc{lat})\\
\glt `The children to whom I gave a book went to school.'

\ex \label{ex:11:14} % {13}
\gll  sune-la kʷač' b-oˤrʡ-aq-ib-i gatu.\\
  self.\textsc{obl}-\textsc{gen} leg \textsc{n}-break:\textsc{pfv}-\textsc{caus}-\textsc{aor}-\textsc{atr}  cat\\
\glt  `the cat whose leg broke'

\ex \label{ex:11:15} % {14}
\gll  šejtan ču-ze g-ub-i buk'unu-me uruχ b-aˤq-ib.\\
demon self.\textsc{pl}.\textsc{obl}-\textsc{inter}(\textsc{lat}) see:\textsc{pfv}-\textsc{aor}-\textsc{atr} shepherd-\textsc{pl} be.afraid \textsc{hpl}-\textsc{lv}:\textsc{pfv}-\textsc{aor}\\
\glt `The shepherds who saw a demon were scared.'
\z

% 4. 
\section{Syntactic orientedness}\label{syntactic-orientedness}

Even though Mehweb data show considerable resemblance to \ili{Tanti Dargwa},
there are important differences between the two Dargwa varieties which
suggest that relativization in Mehweb may be syntactically-oriented.

% 4.1.
\subsection{Resumptives at the top of NPAH}

\is{resumptive pronoun|(}

Unlike in Tanti Dargwa, the Mehweb pronoun
\emph{sa}‹\textsc{cl}›\emph{i} is sometimes considered infelicitous at
the top of NPAH. Cf.\ the following example where the position
relativized into is the actor of a transitive clause:

\ea \label{ex:11:16} % {15}
\gll  \(*sune-jni\) kʷiha b-erh-un-i adami-li-ze nu g-ub.\\
  self.\textsc{obl}-\textsc{erg} ram \textsc{n}-slaughter:\textsc{pfv}-\textsc{aor}-\textsc{atr}  man-\textsc{obl}-\textsc{inter}(\textsc{lat}) I see:\textsc{pfv}-\textsc{aor}\\
\glt   `The man who slaughtered the ram saw me.'
\z

When used as \isi{intensifier}, \emph{sa}‹\textsc{cl}›\emph{i} is normally
accompanied by the emphatic clitic ꞊\emph{al} (with an allomorph
꞊\emph{jal} after vowels). Crucially, the same speaker who found the use
of the resumptive in (\ref{ex:11:16}) infelicitous allows the pronoun followed by
꞊\emph{al} in the same position:

\ea  % {16}
\gll  sune-jni꞊jal kʷiha b-erh-un-i adami-li-ze   nu g-ub.\\
  self.\textsc{obl}-\textsc{erg}꞊\textsc{emph} ram \textsc{n}-slaughter:\textsc{pfv}-\textsc{aor}-\textsc{atr} man-\textsc{obl}-\textsc{inter}(\textsc{lat})   I see:\textsc{pfv}-\textsc{aor}\\
\glt
  `The man who himself slaughtered the ram saw me.'
\z

This example demonstrates that the impossibility of using
\emph{sa}‹\textsc{cl}›\emph{i} in this position cannot be attributed to
any morphological rule that prohibits this pronoun in this position in
general: after all, it occurs there as an intensifier.

As noted by an anonymous reviewer, it could be that the emphatic clitic
changes the distribution of the pronoun. Yet there are also speakers who
have no problems with the use of the resumptive (lacking the emphatic
particle) in all core syntactic positions, including the positions of
the intransitive subject (\ref{ex:11:18}) and transitive actor (\ref{ex:11:19}):

\ea \label{ex:11:18} % {17}
\gll  sa‹b›i dupi-če-b b-urh-u-we b-uʔ-ub-i ule quli ʡaˤr-b-aˤq'-un.\\
self‹\textsc{hpl}› ball-\textsc{super}-\textsc{hpl}(\textsc{ess}) \textsc{hpl}-play:\textsc{ipfv}-\textsc{prs}-\textsc{cvb} \textsc{hpl}-be:\textsc{pfv}-\textsc{aor}-\textsc{atr} child.\textsc{pl} home.\textsc{in}(\textsc{lat}) away-\textsc{hpl}-go:\textsc{pfv}-\textsc{aor}\\
\glt `The children who played with the ball went home.'

\ex \label{ex:11:19} % {18}
\gll  ʜaˤnči ču-ni b-aq'-ib-i xuhe ʡaˤr-b-aˤq'-un quli.\\
work self.\textsc{obl}.\textsc{pl}-\textsc{erg} \textsc{n}-do:\textsc{pfv}-\textsc{aor}-\textsc{atr} woman.\textsc{pl} away-\textsc{hpl}-go:\textsc{pfv}-\textsc{aor} house.\textsc{in}(\textsc{lat})\\
\glt `The women who did all their work went home.'
\z

Our data concerning the possibility of the use of a resumptive at the
top of NPAH are not definitive. The fact that some speakers are more
restrictive in the use of \emph{sa}‹\textsc{cl}›\emph{i} in the
resumptive function suggests, however, that this function may be
governed by syntactic rather than semantic rules.

\is{resumptive pronoun|)}

% 4.2.
\subsection{Coordinate structure constraint}

\is{coordinate structure constraint|(}

Mehweb does not allow relativization out of a conjunct in the
coordination construction and hence follows one of the island
constraints, namely the coordinate structure constraint. (\ref{ex:11:20}a)
illustrates the coordination construction marked with the additive
clitic \emph{꞊ra}. (\ref{ex:11:20}b) demonstrates an unsuccessful attempt at
relativizing one of the conjuncts.

\ea \label{ex:11:20} % {19}
\ea %  a.
\gll musa-ni꞊ra di-la uzi-li-ni꞊ra heš kung   b-elč'-un.\\
  Musa-\textsc{erg}꞊\textsc{add} I.\textsc{obl}-\textsc{gen} brother-\textsc{obl}-\textsc{erg}꞊\textsc{add}   this book   \textsc{n}-read:\textsc{pfv}-\textsc{aor}\\
\glt
  `Musa and my brother read this book.'

  \ex % b.
  \gll *nu-ni꞊ra sune-jni꞊ra heš kung b-elč'-un-i  adami w-ak'-ib.\\
  I-\textsc{erg}꞊\textsc{add} self.\textsc{obl}-\textsc{erg}꞊\textsc{add} this book   \textsc{n}-read:\textsc{pfv}-\textsc{aor}-\textsc{atr}   man \textsc{m}-come:\textsc{pfv}-\textsc{aor}\\
  \glt
  (Expected: `The man who read this book together with me (lit., I and
  who read this book) came.')
  \z
  \z

This contrasts Mehweb with \ili{Tanti Dargwa}, where the coordinate structure
constraint does not apply (cf.\ (\ref{ex:11:2}) above), and again suggests that
syntactic rules might be at work here.

\is{coordinate structure constraint|)}

% 4.3.
\subsection{An argument for resumptive function}

\is{resumptive pronoun|(}

In general, reflexives in Dargwa languages and in Mehweb in particular
are insensitive to the \isi{animacy} or humanness of their antecedent. This is
shown in (\ref{ex:11:21}–\ref{ex:11:22}), where in the first example \emph{sunes} has an
animate (human) antecedent and in the second example \emph{sunela} has
an inanimate antecedent:

\ea \label{ex:11:21} % {20}
\gll  it-ini sune-s kung as-ib.\\
  this-\textsc{erg} self.\textsc{obl}-\textsc{dat} book take:\textsc{pfv}-\textsc{aor}\\
  \glt   `He bought a book for himself.'
  
\ex \label{ex:11:22} % {21}
\gll  nu-ni g-i-ra mažar sune-la weˤʡi-ze.\\
  I-\textsc{erg} give:\textsc{pfv}-\textsc{aor}-\textsc{ego} gun   self.\textsc{obl}-\textsc{gen} master-\textsc{inter}(\textsc{lat})\\
\glt   `I returned the gun to its owner.'
\z

However, some consultants claim that the appearance of
\emph{sa}‹\textsc{cl}›\emph{i} in the resumptive function is only
possible if the head of the relative clause is animate. Examples
(\ref{ex:11:23}–\ref{ex:11:24}) show the possibility of the use of the pronoun in RCCs with
human and non-human animate antecedents:

\ea \label{ex:11:23} % {22}
\gll nu-ni sune-s diʔ g-ib-i ħanq'aka-jni...\\
  I-\textsc{erg} self.\textsc{obl}-\textsc{dat} meat give:\textsc{pfv}-\textsc{aor}-\textsc{atr}   shepherd-\textsc{erg}\\
\glt 
  `the shepherd to whom I gave the meat'

\ex \label{ex:11:24} % {23}
\gll  sune-la kʷač' b-oˤrʡ-aq-ib-i gatu\\
  self.\textsc{obl}-\textsc{gen} leg \textsc{n}-break:\textsc{pfv}-\textsc{caus}-\textsc{aor}-\textsc{atr}  cat\\
\glt   `the cat whose leg broke' (= (\ref{ex:11:14}))
\z

On the contrary, (\ref{ex:11:25}) demonstrates that a resumptive reflexive with an
inanimate antecedent is infelicitous:

\ea \label{ex:11:25} % {24}
\gll   \(\textsuperscript{???}sune-la\) baˤʜ ark-ib-i qali\\
  self.\textsc{obl}-\textsc{gen} wall turn.into.ruin:\textsc{pfv}-\textsc{aor}-\textsc{atr}   house\\
\glt   `the house whose wall crashed down'
\z

Interestingly, this restriction is independent from the \isi{gender} system of
Mehweb which contrasts humans and non-humans rather than animates and
inanimates (see Footnote~\ref{fn14}).

The restriction of \emph{sa}‹\textsc{cl}›\emph{i} to animates\is{animacy} is crucial
exactly because it is not observed in non-resumptive uses. As such, it
separates the resumptive function from the other functions of the
pronoun and goes against Daniel \& Lander's (\citeyear{daniel-lander2008}) hypothesis that
reflexive pronouns in Daghestanian RCCs are not used as resumptives\is{resumptive pronoun} \emph{per se}.
If, according to some consultants' intuition, Mehweb has developed a
dedicated resumptive use of pronouns characterized by specific
restrictions, the RCCs involving such pronouns should be recognized as
syntactically oriented. Again, no constraint of this kind is observed in
\ili{Tanti Dargwa}, where the reflexive pronoun easily occurs in the place of
a relativized argument with an inanimate antecedent (\ref{ex:11:4}).

\is{resumptive pronoun|)}

% 4.4.
\subsection{Realizations of functions of
\emph{sa}‹\textsc{cl}›\emph{i}}

In theory, when referring to a relativized argument within a relative
clause, \emph{sa}‹\textsc{cl}›\emph{i} may fulfill not only the
resumptive function but also the intensifier function and the reflexive
proper function. These functions could in theory be distinguished on the
basis of (i) the restriction to animates in the resumptive function, and
(ii) the presence of the clitic \emph{꞊al} in the intensifier function.
In reality, however, the picture is more complex.

The \isi{intensifier} function of \emph{sa}‹\textsc{cl}›\emph{i} is indeed
observed, for example, in the following example:

\pagebreak

\ea \label{ex:11:26} % {25}
\gll  ʁarʁu-be ar-d-ik-ib sa‹r›i*(꞊jal) d-uʔ-ub-i   merʔ-ani-če-la\\
  stone-\textsc{pl} \textsc{pv}-\textsc{npl}-fall:\textsc{pfv}-\textsc{aor}   self‹\textsc{cl}›(*꞊\textsc{emph}) \textsc{npl}-be-\textsc{aor}-\textsc{atr}  place-\textsc{pl}-\textsc{super}-\textsc{el}\\
\glt 
  `The stones rolled from their own places.'
  (Lit., `The stones rolled from the place they themselves occupied.')
\z

In (\ref{ex:11:26}) \emph{sari} refers to the intransitive subject and requires the
emphatic clitic. Its inability to function as a resumptive\is{resumptive pronoun} (without the
clitic) may be explained either by its high position in NPAH or by its
inanimate\is{animacy} reference. Importantly, the inanimate reference does not block
its appearance in the intensifier function.

The realization of the reflexive\is{reflexive pronoun} function within a relative clause, on
the other hand, turns out to be impossible, as (\ref{ex:11:27}) shows:

\ea \label{ex:11:27} % {26}
\gll  nu-ni (*sune-la) weˤʡi-ze g-ib-i   mažar b-oˤrʡ-oˤb\\
  I-\textsc{erg} self.\textsc{obl}-\textsc{gen}  master-\textsc{inter}(\textsc{lat}) give:\textsc{pfv}-\textsc{aor}-\textsc{atr}  gun \textsc{n}-break:\textsc{pfv}-\textsc{aor}\\
\glt
  `The gun that I returned to its owner broke.'
\z

In this example, \emph{sunela} could be expected to mark the coreference
of the possessor with the undergoer argument (which is then
relativized), yet it does not. Since the reflexive is possible in the
same position in the independent clause (\ref{ex:11:22}), we suspect that the effect
observed in (\ref{ex:11:27}) is due to the fact that the pronoun is interpreted as a
resumptive, in which case it violates the animacy restriction.

Thus the resumptive function blocks the reflexive interpretation. This
rule is not likely to be based on any semantic principle independent of
the grammar, so we take it to be another piece of evidence for
grammaticalization of the resumptive function in this language.

% 5. 
\section{Towards an explanation of the Mehweb pattern}\label{towards-an-explanation-of-the-Mehweb-pattern}

To sum up, even though RCCs in Mehweb can be built on a semantic basis,
in many cases their functioning relies upon strict syntactic mechanisms
and constraints. At least when the relativized argument is animate, the
construction resembles RCCs described for better known languages in a
traditional fashion much more closely, since this argument can be
expressed with a resumptive pronoun proper. These data support the
conclusion made by \citet{daniel-lander2013} that the borderline between
RCCs involving syntactic mechanisms and RCCs which are based on the
semantic information is not strict.

We have no obvious explanation for the Mehweb pattern we observed above.
Nonetheless, below we present some speculations.

First, note that there are a number of languages where \isi{resumptive
pronoun}s are found in RCCs mostly or even only when the relativized
argument is animate; cf.\ \citet{bošković2009} on \ili{Serbo-Croatian} and \ili{Bulgarian}
(Slavic), \citet{csató-uchturpani2010} for \ili{Uyghur} (Turkic),
\citet[104–105]{erteschik-shir1992} for \ili{Hebrew} (Semitic), \citet{kawachi2007} for \ili{Sidaama}
(Cushitic). It may be that the Mehweb system results from
grammaticalization of a similar tendency. Still, there are languages
where at least in some contexts resumptives tend to be used for
inanimates rather than animates, such as \ili{Arabic} \citep{alzaghir2013}.
Sometimes this can be grammaticalized. \citet[474–475]{lyutikova1999}
reports that in another East Caucasian language, \ili{Tsakhur}, the
construction relativizing the object of a postposition only requires a
resumptive pronoun if the relativized argument is inanimate.

Second, we may suspect that the most typical uses of relatives are
associated with high accessibility of the relativized argument. This is
partly reflected in NPAH but can also manifest itself in other
parameters such as animacy, which is said to correlate with conceptual
accessibility (see some discussion in \citealt{vannice-dietrich2003}). Since
more typical uses are more likely to be grammaticalized (see \citealt{lander2015}
for discussion), it is expected that relativization based on syntactic
(i.e.\ grammatical) information is found for more accessible arguments.
Note, however, that the construction with resumptives retains
considerable semantic transparency \citep{keenan1975} and therefore is in a
sense less grammaticalized than constructions with the most accessible
arguments. In other words, the absence of resumptives at the top of NPAH
may be explained by the fact that this top is not primarily based on semantics,
but the absence of resumptives for less accessible arguments may be
explained by the fact that these constructions do not elaborate on
syntactic information.

Still, this approach has a notable shortcoming. The evidence that
relativization prefers animate arguments is somewhat scarce,\footnote{\label{fn14}For
  example, in Tsakhur, during elicitation the choice of what is
  relativized is sometimes influenced by animacy \citep[476–477]{lyutikova1999},
  and for \ili{Turkish} it is reported that headless RCCs by
  default have animate reference \citep{kerslake1998}. The latter, of course,
  may be just the property of headless relatives.} since most studies of
the interaction between animacy and relativization are devoted to the
way in which animacy affects the predictability of what is relativized.
Moreover, things may be turned the other way round. The most accessible
arguments are not normally described with a complex noun phrase with a
modifier, since their accessibility allows them to be more economically
expressed (such as by means of pronouns, proper names, simple noun
phrases, etc.), cf.\ \citet{ariel1990}. Since the inherent accessibility
features of the antecedent and the relativized argument are (normally)
identical, the very fact that the speaker has to use a highly complex
phrase based on a RCC would imply that the target of relativization need
not necessarily be accessible, at least as far as animacy is concerned.
In any case, more research is needed on the issue of the interaction
between \isi{animacy} and relativization.

% 6. 
\section{Conclusion}

In this paper, we provided a sketch of relativization in Mehweb against
the background of the remarkable freedom of relativization in (at least
some) other East Caucasian languages. In particular, we gave preliminary
evidence for the idea that this language has grammaticalized resumptives
and relies on syntactic information during relativization.

To be sure, these conclusions should not be taken for granted. In fact,
even for resumptives, which we specifically addressed above, it is not
clear whether all their uses should be considered alike; as argued by
\citet{erteschik-shir1992} among others,
different types of resumptives may even occur in one language. A deeper
investigation of the functioning of relatives in Mehweb and other East
Caucasian languages, including both corpus analysis and psycholinguistic
experiments, certainly may help to refine the conclusions presented
here.

\section*{Acknowledgements}

We are grateful to our consultants for their help and their patience
and to the editors and two anonymous reviewers for their comments on
an earlier draft of the paper. All errors are ours.

\section*{List of abbreviations}

\begin{longtable}[l]{@{}ll@{}}
\textsc{3pl}	& third person plural \\
\textsc{3sg}	& third person singular \\
\textsc{ad}	& spatial domain near the landmark \\
\textsc{add}	& additive particle \\
\textsc{aor}	& aorist \\
\textsc{atr}	& attributivizer \\
\textsc{aux}	& auxiliary \\
\textsc{caus}	& causative \\
\textsc{cl}	& gender (class) agreement slot \\
\textsc{cop}	& copula \\
\textsc{cvb}	& converb \\
\textsc{dat}	& dative \\
\textsc{ego}	& egophoric \\
\textsc{el}	& motion from a spatial domain \\
\textsc{emph}	& emphasis (particle) \\
\textsc{erg}	& ergative \\
\textsc{ess}	& static location in a spatial domain \\
\textsc{f}	& feminine (gender agreement) \\
\textsc{f1}	& feminine (unmarried and young women gender prefix) \\
\textsc{gen}	& genitive \\
\textsc{hpl}	& human plural (gender agreement) \\
\textsc{in}	& spatial domain inside a (hollow) landmark \\
\textsc{inf}	& infinitive \\
\textsc{inter}	& spatial domain between multiple landmarks \\
\textsc{ipfv}	& imperfective (derivational base) \\
\textsc{lat}	& motion into a spatial domain \\
\textsc{lv}	& light verb \\
\textsc{m}	& masculine (gender agreement) \\
\textsc{n}	& neuter (gender agreement) \\
\textsc{nom}	& nominative \\
\textsc{npl}	& non-human plural (gender agreement) \\
\textsc{obl}	& oblique (nominal stem suffix) \\
\textsc{pfv}	& perfective (derivational base) \\
\textsc{pl}	& plural \\
\textsc{prs}	& present \\
\textsc{pst}	& past \\
\textsc{pv}	& preverb (verbal prefix) \\
\textsc{super}	& spatial domain on the horizontal surface of the landmark \\
\textsc{1pl}	& first person plural \\
\end{longtable}



\printbibliography[heading=subbibliography,notkeyword=this]




\end{document}


%%% Local Variables:
%%% mode: latex
%%% TeX-master: "../main"
%%% End:
