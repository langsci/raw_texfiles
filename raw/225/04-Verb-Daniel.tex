\documentclass[output=paper]{langsci/langscibook} 
\ChapterDOI{10.5281/zenodo.3402060}

% Chapter 4

\title{Mehweb verb morphology}

\author{Michael Daniel\affiliation{National Research University Higher School of Economics}}

\abstract{The paper describes the morphology of the verb in Mehweb, a Dargwa lect of central Daghestan, Russia. The description is partly based on previous research (\citealt{magometov1982}, \citeauthor{sumbatova:unpublished} unpublished) and partly on the field data the author has been collecting from 2009 to the present. Mostly, formal morphology of synthetic verb forms and complex verbs are discussed.

\emph{Keywords}: East Caucasian, Dargwa, Mehweb, verb, inflection,
perfective, imperfective, transitivity, complex verbs.}

\begin{document}
\maketitle

\exewidth{(23)}

\let\exfont\rm
\let\eachwordone\rm

% 1.
\section{Introduction}\label{introduction-5}

In this chapter, I provide an overview of the verb morphology of Mehweb,
a lect of the Dargwa branch of East Caucasian languages, spoken in the
village of the same name in the Gunib district of the Republic of
Daghestan. The paper is mostly focused on formal and synthetic
morphology. Periphrastic forms are treated only peripherally, and the
semantics of the verbal categories is not discussed at all. As a result,
labels provided for different inflectional categories are conventional
and to a large extent based on previous research. While formation of
deverbal nominal forms – nominalizations and participles – is covered,
their further inflection as nominals is also left out. The previous
treatment of the Mehweb morphology, \citet{magometov1982}, provided the basis
for many analytical solutions.
{\looseness1\par}


Mehweb verbs agree in gender (noun class) with their nominative
argument, distinguishing three primary genders – masculine (M),
feminine (F) and neuter (N) in the singular, human plural (HPL) and
non-human plural (NPL) in the plural. There is an additional gender for
unmarried girls and women. Agreement marking is largely similar to
agreement in adjectives, spatial forms, numerals etc., which are not
treated in this chapter. Agreement morphology
% \pagebreak[3]
is discussed in \sectref{gender-agreement}.
Additionally, and unlike other parts of speech, some verbal forms show
special inflection with first or second person subjects, depending on
the illocutionary force (with first person in affirmative utterances and
with second person in interrogative ones). These are
discussed in \sectref{egophoric-forms}.


The whole inflectional paradigm of the verb is divided into two parallel
sets of forms, based on perfective and imperfective stems, whose
relation to each other is complex and follows several different formal
patterns with most verbs. The relation between the stems of a few verbs is irregular. Many
forms are formed from both stems. This is discussed in \sectref{aspectual-stems}.

In Mehweb, there are three distinct verbal inflectional classes,
distinguished by the suffix they take in the perfective past (aorist),
\emph{-ib} (\emph{-ub}), \emph{-ur} or \emph{-un}. The aorist stem is
used in the participle and the forms derived from it. Other forms,
including all forms in the imperfective, are however formed in the same
way for the verbs of all three classes. This is discussed in \sectref{conjugation-classes-and-the-issue-of-labialization},
which also provides a table showing all inflectional forms known so far.

Verbal negation is discussed in \sectref{polarity}. The structure of the verbal
paradigm as a whole is discussed in \sectref{synthetic-paradigm}. Some of the forms follow
specific rules, independent from the classification into three
inflectional classes. These include imperatives and infinitives and are
described in \sectref{imperative-and-infinitive}. Inflection of the auxiliary is discussed in~\sectref{auxiliary}.
Verbs with irregular morphology, including verbs of motion, are
discussed in \sectref{irregular-verbs}. \sectref{transitivity} presents data on transitivity,
including regular morphological causativization and lexically
constrained phenomena such as lability. \sectref{complex-verbs} explains the
morphological makeup of complex verbs, including verbs with vestigial
prefixes, light verbs and verbalizers and bound verbal roots.

% 2.
\section{Gender agreement}\label{gender-agreement}

\is{gender|(}
Mehweb nouns belong to one of the three primary genders – masculine,
feminine and neuter, glossed as M, F and N, respectively. Animate
non-human nouns belong to the neuter gender. In the plural, all human
nouns behave the same, so that only human plural (HPL) and non-human
plural (NPL) are distinguished. Additionally, nouns and pronouns
referring to girls or unmarried\is{feminine, unmarried women} women (glossed as F1) show a special
pattern of agreement – in the singular, they require the same marker as
non-human plurals. Many \isi{mass nouns} and some \isi{abstract nouns}, in the
singular, control NPL agreement.

The morphology of gender markers is shown in the following table and is
common to all targets of agreement – adjectives and verbs having a
prefix \mbox{agreement} slot, locative nominal forms – a suffix slot, etc.
Verbs may only have gender markers in the prefix position, and not all
verbs have this slot (though most do).

\begin{table}
  % Table 1.
  \caption{Gender agreement marking}

\begin{tabular}{@{}llll@{}}
\toprule
& \textsc{sg} & \textsc{pl} &\tabularnewline \midrule
\textsc{m} & w & & \tabularnewline
\textsc{f} & r & b & \textsc{hpl}\tabularnewline
\textsc{f1} & d-r & &\tabularnewline \midrule
\textsc{n} & b & d-r & \textsc{npl}\tabularnewline
\bottomrule
\end{tabular}
\end{table}

The marker of the masculine \emph{w-} is lost in forms where it is
preceded by a prefix, either grammatical (negation) or derivational. There is some evidence 
that this process is optional, at least with the prefix of negation.
Cf.:


\ea % (1)
\gll \emph{w-aχ-un} \quad {vs.\quad} \emph{ħa-χ-un}~(\textless{}~\emph{ħa-w-aχ-un})\\
\textsc{m}-foster:\textsc{pfv}-\textsc{aor} {} \textsc{neg}-\textsc{m}.foster:\textsc{pfv}-\textsc{aor}\\
\z

For more information on the morphology of negation see \sectref{polarity}.

\ea % (2)
\gll  \emph{w-ik-ib} \quad {vs.\quad} \emph{ar-ik-ib}~(\textless{}~\emph{ar-w-ik-ib})\\
\textsc{m}-fall:\textsc{pfv}-\textsc{aor} {} \textsc{pv}-\textsc{m}.fall:\textsc{pfv}-\textsc{aor}\\
\z

Note that, synchronically, most combinations of \isi{preverbs} with the root
are not compositional. Thus, the preverb \emph{ar-} etymologically means
`away', while the verb \emph{-ik-} synchronically means `happen'
(etymologically most probably `fall').

The masculine marker is also lost in stems with initial \emph{u}-, such
as:

\ea % (3)
\gll \emph{d-uq-un} \quad {vs.\quad} \emph{uq-un}~(\textless{}~\emph{w-uq-un})\\
\textsc{f1}-enter:\textsc{pfv}-\textsc{aor} {} \textsc{m}.enter:\textsc{pfv}-\textsc{aor}\\
\z

For more on preverbs, see \sectref{complex-verbs}.

\is{gender|)}

% 3.
\section{Egophoric forms}\label{egophoric-forms}

\is{egophoricity|(}

Some categories of the verb vary depending on whether they have a
subject in the first or second person or not. The forms signaling that
their subjects are speech act participants will be called egophoric
forms below. Unlike gender agreement, subject agreement shows an
accusative\is{accusativity} pattern and is controlled by S/A arguments. The peculiar
property of subject agreement in Mehweb as compared to other Dargwa
languages is that it is sensitive to the illocutionary type of the
utterance. The subject suffix appears with first person subjects in
declarative utterances but with second person subjects in interrogative
utterances. This distribution, known in typological studies as 
egophoric\is{egophoricity}, is sometimes dubbed disjunct vs.\ conjunct forms\is{disjunct vs.\ conjunct forms|see{ego\-phor\-icity}} 
and in East Caucasian languages is so far only attested in \ili{Akhvakh}
\citep{creissels2008,creissels2018} and Zakatala \ili{Avar} \citep{forker2018}. Below, this
inflection will be glossed as \textsc{ego}.

All TAME categories that have egophoric forms are shown in \tabref{tab:4:2}, in
both egophoric (\textsc{ego}) and unmarked (3) forms:

\begin{table}
\advance\tabcolsep2pt
  % Table 2.
  \caption{Egophoric forms and their unmarked counterparts}\label{tab:4:2}
\begin{tabular}{@{}lm{.1\textwidth}<{\raggedright}m{.15\textwidth}<{\raggedright}m{.16\textwidth}<{\raggedright}m{.15\textwidth}<{\raggedright}m{.16\textwidth}<{\raggedright}@{}}
\toprule
  & & \multicolumn{2}{c}{`come'} & \multicolumn{2}{c}{`put on'} \tabularnewline
 \cmidrule(lr){3-4} \cmidrule(l){5-6}
& & perfective & imperfective & perfective & imperfective\tabularnewline\midrule

\textsc{pst} & 
3

\textsc{ego} & 
\emph{꞊ak'-ib}

\emph{꞊ak'-i-ra} & 
\emph{꞊ik'-ib}

\emph{꞊ik'-i-ra} & 
\emph{ik'-ub}

\emph{ik'-ub-ra} & 
\emph{irk'ʷ-ib}

\emph{irk'ʷ-i-ra}\tabularnewline

\textsc{hab} & 
3

\textsc{ego} & 
– & 
\emph{꞊ik'an}

\emph{꞊ik'as} & 
– & 
\emph{irk'ʷ-an}

\emph{irk'ʷ-as}\tabularnewline
\textsc{fut} & 
3

\textsc{ego} & 
\emph{꞊ak'-as}

\emph{꞊ak'-iša} & 
\emph{꞊ik'-es}

\emph{꞊ik'-iša} & 
\emph{ik'ʷ-es}

\emph{ik'ʷ-iša} & 
\emph{irk'ʷ-es}

\emph{irk'ʷ-iša}\tabularnewline \midrule
  & & \multicolumn{2}{c}{`fly'} & \multicolumn{2}{c}{`read'} \tabularnewline
 \cmidrule(lr){3-4} \cmidrule(l){5-6}                                  
\textsc{pst} & 
3

\textsc{ego} & 
\emph{arc-ur}

\emph{arc-ur-ra} & 
\emph{urc-ib}

\emph{urc-i-ra} & 
\emph{꞊elč'-un}

\emph{꞊elč'-un-na} & 
\emph{luč'-ib}

\emph{luč'-i-ra}\tabularnewline

\textsc{hab} & 
3

\textsc{ego} & 
– & 
\emph{urc-an}

\emph{urc-as} & 
– & 
\emph{luč'-an}

\emph{luč'-as}\tabularnewline

\textsc{fut} & 
3

\textsc{ego} & 
\emph{arc-es}

\emph{arc-iša} & 
\emph{urc-es}

\emph{urc-iša} & 
\emph{꞊elč'-es}

\emph{꞊elč'-iša} & 
\emph{luč'-es}

\emph{luč'-iša}\tabularnewline
\bottomrule
\end{tabular}
\end{table}

In the past, the egophoric forms are marked with the suffix \emph{-ra},
assimilated to \emph{-na} after the nasal auslaut in the aorist. In the
imperfective past, the tense suffix \emph{-ib-} irregularly drops its
final \emph{-b}. In the future, non-egophoric forms are identical to the
infinitive, while the egophoric forms use a special suffix \emph{-iša}. In
the present habitual (which also serves as synthetic present for some stative verbs), there is an opposition of two special affixes,
\emph{-an} for non-egophoric and \emph{-as} for egophoric forms. Following
the idea that the basic distinction is between egophoric forms that are
marked and non-egophoric unmarked forms, I gloss \mbox{\emph{-an}} simply as
\textsc{hab} and \emph{-as} as \textsc{hab}.\textsc{ego} (similarly
with other forms). Egophoric forms are also present with the present form
of the auxiliary \emph{lewra} (\textsc{m}), \emph{lella}
(\textless{}~\emph{ler-ra}, \textsc{f} and \textsc{npl}),
% \pagebreak[4]
\emph{lebra}
(\textsc{n} and \textsc{hpl}) and the negative copula \emph{aħinna}
(\textless{}~\emph{aħin-ra}) – see \sectref{auxiliary} on inflection of
auxiliaries.
\is{egophoricity|)}

% 4.
\section{Aspectual stems}\label{aspectual-stems}

\is{aspectual stem|(}
\is{perfective|(}
\is{imperfective|(}

In Mehweb, the vast majority of the verbal categories are formed from
two different stems, perfective and imperfective. I will consider verbal
inflection as divided into perfective and imperfective paradigms. The
two paradigms are largely parallel. Most categories attested both in the
perfective and the imperfective paradigms use the same affixes. The
exceptions are listed in the following table:

\begin{table}[h]
  % Table 3.
  \caption{Asymmetries between perfective and imperfective paradigms}

\begin{tabular}{@{}lll@{}}
\toprule
& perfective & imperfective\tabularnewline \midrule
% \multicolumn{3}{p{.65\textwidth}}{\centering\emph{categories showing different marking\linebreak in the perfective vs. imperfective paradigms}} \tabularnewline \midrule
{past} & \emph{-ib\(-ub\)/-ur/-un} & \emph{-ib}\tabularnewline
participle & {past}+\emph{-i\(l\)} & \emph{-ul}\tabularnewline
converb & {past}+\emph{-le} & \emph{-uwe}
(\textless{}~\textsc{ptcp}+\emph{-le})\tabularnewline
imperative & \emph{-e/-a} & \emph{-e}\tabularnewline
infinitive & \emph{-es/-as} & \emph{-es}\tabularnewline\midrule
% \multicolumn{3}{p{.65\textwidth}}{\centering\emph{categories only compatible with one of the stems}} \tabularnewline \midrule 
present & – & \emph{-an/-as}\tabularnewline
prohibitive & – & \emph{m}(V)- \ldots{} \emph{-di}\tabularnewline
negative optative & – & \emph{m}(V)- \ldots{} \emph{-ab}\tabularnewline
\bottomrule
\end{tabular}
\end{table}

On the choice of one of the markers in the same category see the
relevant sections below. For the different markers of the aorist
(perfective past) see \sectref{conjugation-classes-and-the-issue-of-labialization}; for the choice of the vowels in the
imperative and the infinitive see \sectref{imperative-and-infinitive}; the second of the two
affixes in the present tense is the egophoric form (see \sectref{gender-agreement} above).
For the asymmetries in the system of special converbs see \citet{sheyanova2019}.
Other parallel categories in the two paradigms use the same
markers.

There are verbs that lack the perfective stem. When asked to produce
perfective forms for these verbs, the consultants suggest a combination
of the infinitive with perfective verbs, mostly \emph{꞊aɁes} `begin'.
These defective verbs denote states and some \isi{atelic} activities, such as
\emph{izes} `be ill', \emph{꞊iges} `want', \emph{꞊ukes} `itch',
\emph{ures} `rain', \emph{ruržes} `be shivering' (also `boil'),
\emph{rurqes} `flow', \emph{꞊uzes} `work', \emph{urʁes} `fight',
\emph{꞊ulqes} `dance'. Note that some of these verbs show a
morphological structure similar to one of the models of the imperfective
stem derivation – infixation of \emph{-r-} or \emph{-l-} – and may
historically go back to a regular two-stem verb. In fact, \emph{꞊ulqes}
`dance' is identical to the imperfective stem of \emph{꞊uqes}
\textasciitilde{} \emph{꞊ulqes} `go, run'. Another defective verb is the
bound root *\emph{k'es} (probably related to \emph{uk'es} (\textsc{ipfv}) `say')
that is used in some morphologically complex but unanalyzable verbs.

Some verbs have identical perfective and imperfective stems. These
include \emph{umces} `weigh, measure', \emph{irxes} `reap',
\emph{irc'es} `weed', \emph{꞊alces} `spin (thread)', \emph{꞊urhes}
`tell', \emph{꞊uhes} `scold', \emph{꞊uʔes} `be', \emph{꞊ises} `weep',
\emph{꞊aˤldes} `hide' (tr). Note again that some of these verbs have the
-V(l/r)C- structure typical of imperfective stems.

There are also several verbs whose imperfective stem is distinct from
the perfective stem in that it does not contain the gender\is{agreement slot} prefix slot:
(꞊)\emph{ižes} `lick', (꞊)\emph{iˤšqes} `mow, peel', (꞊)\emph{ites}
`beat', (꞊)\emph{igʷes} `burn'. More generally, there is an asymmetry
between perfective and imperfective stem in terms of the presence of the
gender \isi{agreement slot}: imperfective stems may lack it with those verbs
whose perfective stems have it, but not vice versa. Cf.\ the following
table:

\begin{table}[h]
  % Table 4.
  \caption{Asymmetries between perfective and imperfective paradigms}

  \begin{tabular}{@{}ll*2{p{2em}<{\centering}}@{}}
    \toprule
    & & \multicolumn{2}{c}{Imperfective}\tabularnewline \cmidrule{3-4}
    & & + & – \tabularnewline \midrule 
            \raisebox{-6pt}[0pt][0pt]{Perfective} & + & 66 & 29\tabularnewline
    & – & (2) & 21\tabularnewline
                \bottomrule
  \end{tabular}
\end{table}

The two verbs which exceptionally have gender\is{agreement slot} slots in the imperfective
stem but lack it in the perfective stem are \emph{kes} (\textsc{pfv})
\textasciitilde{} \emph{꞊ukes} (\textsc{ipfv}) `bring' and \emph{es} (\textsc{pfv})
\textasciitilde{} \emph{꞊uk'es} (\textsc{ipfv}) `say, tell', both of which are
morphologically irregular. The latter verb may be considered two
separate lexical items (`say' and `tell').


There are several highly \isi{irregular verbs}, all shown in \tabref{tab:4:5}. Note
that, again, with `see' and `give', the imperfective stems show one of
the regular patterns of imperfective stem formation (see below) and are
similar to their perfective stems, so that they represent a case of
weaker suppletion than fully irregular `say' and `go'.

\vspace{-.5\baselineskip}


\begin{table}[H]
  % Table 5.
  \caption{Aspectual stems of the irregular verbs}\label{tab:4:5}

\begin{tabular}{@{}lllll@{}}
\toprule
& `say' & `see' & `give' & `go'\tabularnewline \midrule
\textsc{pfv} & \emph{i-/e-/bet'-} & \emph{gʷ-} & \emph{\(꞊e\)g-} &
\emph{꞊aˤq'-/꞊uˤq'-/q'-꞊eʡ-}\tabularnewline
\textsc{ipfv} & \emph{uk'-} & \emph{irgʷ-} & \emph{lug-} &
\emph{꞊aš-}\tabularnewline
\bottomrule
\end{tabular}
\end{table}

% \pagebreak[4]

The attested patterns of the connection between the perfective and the
imperfective stems are summarized in \tabref{tab:4:6}. The choice of the pattern
is not fully independent of other formal properties of the verb, first
of all the perfective past formation and/or the presence of
\isi{labialization} (a labialized final consonant or \emph{u}); see the
explanations below the table.

\begin{table}

  % Table 6.
\caption{Patterns of aspectual stems formation}\label{tab:4:6}

\begin{tabular}{@{}m{.12\textwidth}<{\raggedright}m{.12\textwidth}<{\raggedright}m{.23\textwidth}<{\raggedright}lm{.15\textwidth}<{\raggedright}>{\raggedright}m{.18\textwidth}@{}}
\toprule
\footnotesize Model & 
\footnotesize Subtype & 
\footnotesize Example & 
\footnotesize No. & 
\footnotesize Constraints \& tendencies &  
\footnotesize Exceptions to constraints \tabularnewline \midrule
infixation

in \textsc{ipfv} & 
‹\emph{l}› & 
\emph{꞊ic'-} \textasciitilde{} \emph{꞊ilc'-} `fill' & 
18 & 
none & 
\tabularnewline \midrule

infixation

in \textsc{ipfv} & 
‹\emph{r}› & 
\emph{ih-\(ub\)} \textasciitilde{} \emph{irhʷ-} `throw' & 
5 & 
labialization & 
\emph{꞊ix-} \textasciitilde{} \emph{꞊irx-} `put'\tabularnewline \midrule
\emph{er-} in \textsc{pfv} & & \emph{꞊erž-} \textasciitilde{} \emph{꞊už-} `drink' &
17 & none &\tabularnewline \midrule

V\emph{l}C \textasciitilde{} \emph{lu}C & 
\emph{al}C \textasciitilde{} \emph{lu}C

\emph{el}C \textasciitilde{} \emph{lu}C & 
\emph{꞊elč'-\(un\)} \textasciitilde{} \emph{luč'-} `read' & 
9 & 
\textsc{aor} in \emph{-un} & 
\emph{꞊aˤlq'-} \textasciitilde{} \emph{luˤq'-} `rinse'\tabularnewline\midrule

ablaut & 
\emph{a-} \textasciitilde{} \emph{i-}

\emph{e-} \textasciitilde{} \emph{i-} & 
\emph{abx-} \textasciitilde{} \emph{ibx-} `open'

\emph{꞊eʔ} \textasciitilde{} \emph{꞊iʔ} `be~enough' & 
19 & 
(\textsc{aor} in \emph{-ib}) & 
\tabularnewline \midrule

ablaut & 
\emph{a-} \textasciitilde{} \emph{u-}

\emph{e-} \textasciitilde{} \emph{u-} & 
\emph{ar-\(un\)} \textasciitilde{} \emph{ur-} `sift'

\emph{꞊erg-} \textasciitilde{} \emph{꞊urg} `spin~(thread)' & 
22 & 
labialization

\textsc{aor} in \emph{-un} or \emph{-ur} & 
\emph{꞊arg-} \textasciitilde{} \emph{꞊urg-} `find'

\emph{꞊ebk'-} \textasciitilde{} \emph{꞊ubk'-} `die'\tabularnewline
\bottomrule
\end{tabular}
\end{table}

Infixation\is{infixation} of \emph{-l-} (18 verbs) is attested in all inflectional
classes, while \isi{infixation} of \emph{-r-} (seven verbs) is present in five
simple verbs, four of which are labialized (aorist in \emph{-ub}). The
model VlC \textasciitilde{} luC is typical specifically of the verbs
with aorist in \emph{-un}. Vowel alternation in V(C)C roots is usually
\emph{a-}/\emph{e-} \textasciitilde{} \emph{i-}, with \emph{i-} changing
to \emph{u-} in verbs with the aorist in \emph{-un}, \emph{-ur} or
\emph{-ub}.

\is{aspectual stem|)}
\is{perfective|)}
\is{imperfective|)}


% 5.
\section{Conjugation classes and the issue of labialization}\label{conjugation-classes-and-the-issue-of-labialization}

\is{conjugation classes|(}
\is{labialization|(}

I group Mehweb verbs into three inflectional classes according to the
marker of the perfective past they use – \emph{-ib}, \emph{-ur} or
\emph{-un}. Most verbs use the \emph{-ib} suffix, which I will consider
to be the default; the same suffix is used by verbs of all conjugations
with the imperfective stem as the imperfective past, so in fact it may
be considered to be simply a suffix (of the secondary derivational stem)
of the past, perfective or imperfective, the choice between the
perfective/imperfective interpretation being, in these forms, fully
determined by the aspectual characteristics of the stem. A small
additional fourth class is very similar to the `default' conjugation
except that all verbs in this class have labialization on the final
consonant of the stem and the aorist marker is realized as \emph{-ub};
it is shown as 1a in the following table. However, not all inflectional
properties of this 1a class may be explained as being a labialized
variety of the first class; see below. Here are some representative
forms:

\begin{table}
% Table 7.
\caption{Verbal inflectional classes}

\begin{tabular}{@{}llll@{}}
\toprule
& \textsc{pfv} \textsc{pst} & \textsc{ipfv} \textsc{pst} &\tabularnewline \midrule
1. & \emph{irx-ib} & \emph{irx-ib} & `reap'\tabularnewline
& \emph{꞊ic-ib} & \emph{꞊ilc-ib} & `sell'\tabularnewline
1a & \emph{꞊ig-ub} & \emph{꞊igʷ-ib} & `burn'\tabularnewline\midrule
2. & \emph{arc-ur} & \emph{urc-ib} & `fly'\tabularnewline
& \emph{꞊emž-ur} & \emph{꞊umž-ib} & `get warm'\tabularnewline \midrule
3. & \emph{꞊erg-un } & \emph{꞊ug-ib} & `eat'\tabularnewline 
& \emph{alʔ-un} & \emph{ulʔ-ib} & `cut'\tabularnewline
\bottomrule
\end{tabular}
\end{table}

In verbs with lexical \isi{pharyngealization}, the \emph{-u-} of the aorist
marker may be realized as \emph{-oˤ-} (on pharyngealization, see \citealt{moroz2019}).
Cf.:

% (4)
\ea
\emph{꞊oˤrʡ-oˤb} `break' (variant of \emph{-ub})

\ex % (5)
\emph{꞊iʡ-oˤn} `steal' (variant of \emph{-un}).
\z

Labialized stems also exist in the \emph{-un} and \emph{-ur} classes,
where the labialization is, however, lost before (absorbed by) the vowel
of the aorist suffix. It is also lost in the imperfective forms if the
stem vowel changes to \emph{-u-} – apparently, the root vowel absorbs
the labialization of the following consonant, including when there is
another consonant that comes between the root vowel and the labialized
consonant. Depending on the form and class, labialization of the stem is
thus realized as labialization of the last consonant of the stem (e.g.\
in the imperative), labialization of the stem vowel (in various
imperfective forms) or labialization of the suffix vowel (in \emph{-ib}
of the aorist).


Most verbs with \emph{-ub} in the aorist also have labialization in
other forms, so that one interpretation is that \emph{-ub} results from
the \emph{-ib} marker meeting the final labialization of the stem. The
two verbs that take \emph{-ub} but do not show labialization in other
forms – \emph{꞊oˤrʡ-} `break' and \emph{꞊uh-} `become' – both have
\emph{-u-} as the
% \pagebreak[3]
underlying vowel of the root (\emph{oˤ} is the result
of pharyngealization of \emph{u}). When comparing this to the fact that
the \emph{-u-} in the imperfective stem absorbs the labialization of the
final consonant, as shown in \tabref{tab:4:8}, it seems appropriate to
posit the deep form of the perfective stem of these two verbs as having
the labialized consonant whose labialization changes the aorist marker
\emph{-ib} to \emph{-ub} but is itself always absorbed *\emph{꞊oˤrʡʷ-},
*\emph{꞊uhʷ-}. Then, all verbs that take \emph{-ub} in the aorist have
final labialization. On the other hand, none of the \emph{-ib} verbs has
a labialized final consonant.

\begin{table}[h]
\vspace{-\jot}
% Table 8.
\caption{Labialized stems}\label{tab:4:8}

\begin{tabular}{@{}lllllll@{}}
\toprule
& \multicolumn{3}{c}{Perfective} & \multicolumn{3}{c@{}}{Imperfective} \tabularnewline 
\cmidrule(lr){2-4} \cmidrule(l){5-7}
 & \textsc{imp} & \textsc{inf} & \textsc{pst} & \textsc{imp} & \textsc{inf} & \textsc{pst}\tabularnewline \midrule
`dig' & \emph{꞊erʁʷa} & \emph{꞊erʁʷes} & \emph{꞊erʁub} & \emph{iʁʷe} &
\emph{iʁʷes} & \emph{iʁʷib}\tabularnewline
`slaughter' & \emph{꞊erhʷa} & \emph{꞊erhʷes} & \emph{꞊erhun} &
\emph{꞊urhe} & \emph{꞊urhes} & \emph{꞊urhib}\tabularnewline
`burn' & \emph{꞊alk'ʷa } & \emph{꞊alk'ʷes } & \emph{꞊alk'un} &
\emph{luk'e} & \emph{luk'es} & \emph{luk'ib}\tabularnewline
`go down' & \emph{꞊erχʷe} & \emph{꞊erχʷes} & \emph{꞊erχur} &
\emph{꞊urχe} & \emph{꞊urχes} & \emph{꞊urχib}\tabularnewline
\bottomrule
\end{tabular}
\end{table}

Given this evidence, it seems that the \emph{-ub} conjugation should
merely be considered a formal subtype of the \emph{-ib} conjugation.
However, the conjugation of the \emph{-ub} and \emph{-ib} verbs diverge
in two important points. First, both the aorist marker \emph{-ib} and
the homophonous imperfective past marker on all verbs lose the final
consonant when followed by \emph{-ra} in egophoric forms or the perfective
converb marker \emph{-le}. With \emph{-ub}, both forms keep the final
\emph{-b}. Second, the \emph{-ib} in the imperfective paradigm does not
change to \emph{-ub} after a labialized stem – something which we would
expect assuming that \emph{-ub} in the perfective paradigm results from
\ldots{}ʷ+\emph{-ib}.

\begin{table}[h]
% \vspace{-\jot}
  % Table 9.
\caption{Divergence between
the default \emph{-ib} and the \emph{-ub} conjugations}

\begin{tabular}{@{}llllll@{}}
\toprule
& & \textsc{imp} & \textsc{pst} & \textsc{pst}(\textsc{ego}) & \textsc{cvb} \tabularnewline \midrule
`come' & \textsc{pfv} & \emph{꞊ak'e} & \emph{꞊ak'ib} & \emph{꞊ak'ira} &
\emph{꞊ak'ile}\tabularnewline
& \textsc{ipfv} & \emph{꞊ik'e} & \emph{꞊ik'ib} & \emph{꞊ik'ira} &
\emph{꞊ak'uwe}\tabularnewline
`put on'& \textsc{pfv} & \emph{ik'ʷa} & \emph{ik'ub} & \emph{ik'ubra} &
\emph{ik'uble}\tabularnewline
& \textsc{ipfv} & \emph{irk'ʷa} & \emph{irk'ʷib} & \emph{irk'ʷira} &
\emph{irk'uwe}\tabularnewline
\bottomrule
\end{tabular}
\end{table}


% \pagebreak[4]

In other words, the suffix \emph{-ub} shows morphophonological behavior
which is significantly different from \emph{-ib.}

% \pagebreak[4]

Whatever the ultimate interpretation of the \emph{-ub} aorist should be,
it seems that this inflection type shows a position intermediate between
a separate conjugation class and a subtype of the default. The full list
of the attested labialized stems for all conjugations is as follows (in
the aorist form): \emph{꞊eˤʡub} `seed', \emph{꞊erkun} `eat', \emph{gub}
`see', \emph{ihub} `throw', \emph{꞊alk'un} `take fire', \emph{꞊igub}
`burn', \emph{ik'ub} `put on', \emph{꞊erhun} `slaughter', \emph{꞊usaˤʡun}
`fall asleep', \emph{꞊erʔub} `dry up', \emph{꞊aˤʜun} `get soaked',
\emph{꞊erq'ub} `become worn', \emph{꞊erʁub} `dig out', \emph{꞊alħun}
`wake up', \emph{꞊erχur} `come down'. As explained above, the verbs
\emph{꞊oˤrʡoˤb} `break' and \emph{꞊uhub} `become' are only labialized in
their underlying forms.

\is{conjugation classes|)}
\is{labialization|)}

% 6.
\section{Polarity}\label{polarity}

\is{polarity|(}
\is{negation|see{polarity}}
\is{polarity, negative volitional forms|(}

Verbal negation is expressed by one of the two prefixes, the standard
negation prefix \emph{ħa-} and the volitive negation prefix \emph{m}V-.
The latter is only used in volitional moods including the prohibitive
(negative imperative) and negative optative, and the former is used
elsewhere, both on finite and non-finite forms. Some speakers allow
using \emph{ħa-} in negative optative forms. The standard negation
\emph{ħa-} is, however, never used in prohibitive forms.

In periphrastic verbal forms, both the lexical and the auxiliary verb
may be negated. The standard negation \emph{ħa-} is placed immediately
before the verbal stem, thus following the preverb with preverbal verbs.
The full pre-root template of the verb is shown in the following
example:

\ea % (6)
\gll \emph{har-ħa-d-uq-un}.\\
\textsc{pv}-\textsc{neg}-\textsc{f1}-flee:\textsc{pfv}-\textsc{aor}\\
\glt `She did not run away.'
\z

Some of the negative forms of the verb \emph{꞊ak'-as} `come' are given
in % the following
\tabref{tab:4:10} as an example. As masculine forms
morphophonologically interact with the prefix (see below), feminine
(more specifically, F1 – girls gender) forms are given instead.

\begin{table}[t]
% Table 10.
\caption{Some negative forms of \emph{꞊ak'as} \textasciitilde{} \emph{꞊ik'es} `come'}\label{tab:4:10}

\begin{tabular}{@{}lll@{}}
\toprule
{stem} & {\textit{꞊ak'}} & {\textit{꞊ik'}}\tabularnewline \midrule
\textsc{pst} & \emph{ħadak'ib} & \emph{ħadik'ib}\tabularnewline
\textsc{inf} & \emph{ħadik'as} & \emph{ħadik'es}\tabularnewline
\textsc{hab} & – & \emph{ħadik'an}\tabularnewline
\textsc{opt} & – & \emph{midik'ab \(ħadik'ab\)}\tabularnewline
\textsc{proh} & – & \emph{midik'ad\(i\)}\tabularnewline
\textsc{cond} & \emph{ħadak'ak'a} & \emph{ħadik'ak'a}\tabularnewline
\textsc{ptcp} & \emph{ħadak'ibili} & \emph{ħadik'uli}\tabularnewline
\textsc{cvb} & \emph{ħadak'ile} & \emph{ħadik'uwe}\tabularnewline
\textsc{nmlz} & \emph{ħadak'ri} & \emph{ħadik'ri}\tabularnewline
\bottomrule
\end{tabular}
\vspace{-\jot}
\end{table}

The forms are morphophonologically straightforward except on vowel
initial bases, including those resulting from the elision of the
masculine prefix \emph{w-}, where the vowel \emph{-a} of the prefix interacts
with the initial vowel of the stem. The elision of the masculine prefix
\emph{w-} occurs after all prefixal elements, including the standard
negation prefix itself. After this, the following processes occur:

% \pagebreak

\ea % (7)
initial a- or e- of the base is dropped:
\begin{center}
ħa+aC\ldots{} → ħa-C\ldots{}

ħa+eC\ldots{} → ħa-C\ldots{}  
\end{center}



\ex % (8)
initial i → j:
\begin{center}
  ħa+iC\ldots{} → ħa-jC\ldots{}
\end{center}

\ex % (9)
\ldots{}and then dropped before a consonant cluster:
\begin{center}
ħa-jCC → ħa-CC\ldots{}
\end{center}

\ex % (10)
initial u → w:
\begin{center}
  ħa+uC\ldots{} → ħa-wC\ldots{}
\end{center}

\ex % (11)
\ldots{}and then dropped before a consonant cluster leaving
(probably optionally) \isi{labialization} on one of the consonants:

\begin{center}
  ħa-wCC → ħa-C\textsuperscript{(}ʷ\textsuperscript{)}C\textsuperscript{(}ʷ\textsuperscript{)}
\end{center}

% \vspace{-\baselineskip}

\z

This \isi{labialization} may only result from the initial \emph{u-} of the
root, not from the masculine prefix \emph{w}-, which is dropped after a
prefix, leaving no trace. Cf.\ the following forms with different types
of anlaut (masculine forms are given for the verbs with the initial
gender agreement slot):

\largerpage[1.25]

\begin{table}[H]
  % Table 11.
  \captionsetup{margin=0pt}
\caption{Standard negation on verbal stems with and without gender prefix slot}

\small
\begin{tabular}{@{}l*6{m{.11\textwidth}<{\raggedright}}@{}}
\toprule
\raisebox{-14pt}[0pt][0pt]{\parbox{.1\textwidth}{\raggedright with gender slot}} & \emph{꞊u}C- & \emph{꞊a}C- & \emph{꞊i}C- & \emph{꞊u}CC- & \emph{꞊a}CC- & \emph{꞊i}CC-\tabularnewline \cmidrule{2-7}
& `enter'

(\textsc{pfv}) & 
`nurture'

(\textsc{pfv}) & 
`come'

(\textsc{ipfv}) & 
`send'

(\textsc{ipfv}) & 
`nurture'

(\textsc{ipfv}) & 
`let go'

(\textsc{ipfv})\tabularnewline \midrule
\textsc{pst} \textsc{neg} (\textsc{m}) & \emph{ħa-wq-un} & \emph{ħa-χ-un} & \emph{ħa-jk'-ib} &
\emph{ħa-rxʷ-ib} & \emph{ħa-lχ-ib} & \emph{ħa-rq'-ib}\tabularnewline
\textsc{pst} (\textsc{m}) & \emph{uq-un} & \emph{w-aχ-un} & \emph{w-ik'-ib} &
\emph{urx-ib} & \emph{w-alχ-ib} & \emph{w-irq'-ib}\tabularnewline \midrule 
\raisebox{-14pt}[0pt][0pt]{\parbox{.1\textwidth}{\raggedright without gender slot}}
  & \emph{\#u}C & \emph{\#i}C & \emph{\#u}CC- & \emph{\#a}CC- &
\emph{\#i}CC- & \emph{\#e}CC-\tabularnewline \cmidrule{2-7}
& 
`sift'

(\textsc{ipfv}) & 
`take'

(\textsc{ipfv}) & 
`pour'

(\textsc{ipfv}) & 
`open'

(\textsc{pfv}) & 
`open'

(\textsc{ipfv}) & 
`count'

(\textsc{pfv})\tabularnewline \midrule
\textsc{pst} \textsc{neg} & \emph{ħa-wr-ib} & \emph{ħa-js-ib} & \emph{ħa-lq'ʷ-ib} &
\emph{ħa-bx-ib} & \emph{ħa-bx-ib} & \emph{ħa-lʔ-un}\tabularnewline
\textsc{pst} & \emph{ur-ib} & \emph{is-ib} & \emph{ulq'-ib} & \emph{abx-ib} &
\emph{ibx-ib} & \emph{ulʔ-ib}\tabularnewline
\bottomrule
\end{tabular}
\end{table}

The same processes apply to the optative forms when they use the
standard negation marker, cf.:

\begin{table}
% Table 12.
\caption{Negation on the optative forms}

\begin{tabular}{@{}llll@{}}
\toprule
&& Optative & Negative Optative\tabularnewline \midrule
\emph{꞊ik'es} & `come' (\textsc{ipfv}) & \emph{w-ik'-ab} (\textsc{m}) & \emph{ħa-jk'-ab} (\textsc{m})\tabularnewline
\emph{ures} & `rain' (\textsc{ipfv}) & \emph{ur-ab} & \emph{ħa-wr-ab}\tabularnewline 
\emph{ises} & `take' (\textsc{ipfv}) & \emph{is-ab} & \emph{ħa-js-ab}\tabularnewline
\emph{꞊irqes} & `let go' (\textsc{ipfv}) & \emph{w-irq-ab} (\textsc{m}) & \emph{ħa-rq-ab} (\textsc{m})\tabularnewline
\emph{꞊urxes} & `send' (\textsc{ipfv}) & \emph{urx-ab} (\textsc{m}) & \emph{ħa-rxʷ-ab}\tabularnewline
\bottomrule
\end{tabular}
\end{table}

Attested forms of negation in periphrastic forms use the negative auxiliary
\emph{agʷara}:

\ea % (12)
negation in periphrasis:

\ea % (a)
\xbox{.4\textwidth}{\gll \emph{luč'-uwe} \emph{le-w}.\\
read:\textsc{ipfv}-\textsc{cvb.ipfv} \textsc{aux}-\textsc{m}\\
\glt `He is reading.'}%
\exsameline % (b)
\xbox{.4\textwidth}{\gll \emph{luč'-uwe} \emph{agʷara}.\\
read:\textsc{ipfv}-\textsc{cvb.ipfv} \textsc{aux}:\textsc{neg}\\
\glt `He is not reading.'}
\z
\z


The morphophonology of the forms with the dedicated volitive negation
(\textsc{negvol}) marker is different. The prohibitive and the negative optative
forms both take the same consonantal prefix \emph{m-} (\emph{m}V- before
consonants) but two different suffixes. The masculine prefix \emph{w-}
is lost after the negative volitional \emph{m-}. When followed by
a consonant, either a gender prefix or the initial
% \pagebreak[3]
consonant of the stem,
the negative volitional copies the stem vowel. Finally, the neuter/human
plural prefix \emph{b-} is assimilated by the negative volitional and is
represented by~\emph{m-}.


\ea \label{ex:4:13} % (13)
morphophonology of the negative volitional prefix:

\ea % (a)
\gll \emph{m-uz-adi}~(\textless{}~\emph{m-w-uz-adi})\\
\textsc{negvol}-\textsc{m}.work:\textsc{ipfv}-\textsc{proh}\\
\glt `Do not work!' (to a man)

\ex % (b)
\gll \emph{mu-d-uz-adi}~(\textless{}~\emph{mV-d-uz-adi})\\
\textsc{negvol}-\textsc{f1}-work:\textsc{ipfv}-\textsc{proh}\\
\glt `Do not work!' (to a girl)

\ex % (c)
\gll \emph{buz} \emph{mu-m-uz-adi}~(\textless{}~\emph{mV-b-uz-adi})\\
(stem~copy) \textsc{negvol}-\textsc{n}-fry:\textsc{ipfv}-\textsc{proh}\\
\glt `Do not fry (it)!'
\z
\z

As (\ref{ex:4:13}c) also shows, the process of stem copy (see below) applies before assimilation in nasality takes place.


As to the suffix position, the negative optative and the prohibitive
have different suffixes. The negative optative takes the suffix
\emph{-ab}, same as the positive optative; the prohibitive takes a
dedicated suffix \emph{-adi}, whose final vowel is optionally dropped.
In both cases, the initial \emph{-a-} of the suffix is analyzed below as
a marker of a secondary derivational stem termed irrealis (see next
section). The following table shows the prohibitive of verbs with
different stem structures.

\begin{table}
% Table 13.
\caption{Volitional negation with different stem structures}

\small
\advance\tabcolsep-3.5pt
\begin{tabular}{@{}lllllllll@{}}
\toprule
& \multicolumn{2}{l}{Verb (\textsc{ipfv})} & \multicolumn{3}{c}{Negative Optative} & \multicolumn{3}{c@{}}{Prohibitive}\tabularnewline
\cmidrule(r){4-6} \cmidrule{7-9}
& & &  \multicolumn{1}{l}{\textsc{m}} & \multicolumn{1}{l}{\textsc{f1}/\textsc{npl}} & \multicolumn{1}{l}{\textsc{n}/\textsc{hpl}} & \multicolumn{1}{l}{\textsc{m}} &\multicolumn{1}{l}{\textsc{f1}/\textsc{npl}} & \multicolumn{1}{l@{}}{\textsc{n}/\textsc{hpl}} \tabularnewline\midrule 
\emph{꞊u}C... & \emph{꞊uzes} & `work' & \emph{uzab} & \emph{duzab} & \emph{buzab}
& \emph{muzadi} & \emph{muduzadi} & \emph{mumuzadi}\tabularnewline
\emph{꞊a}C... & \emph{꞊alχes} & `treat' & \emph{walχab} & \emph{dalχab} &
\emph{balχab} & \emph{malχadi} & \emph{madalχadi} &
\emph{mamalχadi}\tabularnewline
\emph{꞊e}C... & \emph{꞊elk'es} & `choose' & \emph{welk'ab} & \emph{delk'ab} &
\emph{belk'ab} & \emph{melk'adi} & \emph{medelk'adi} &
\emph{memelk'adi}\tabularnewline
\emph{꞊i}C... & \emph{꞊ilces} & `sell' & \emph{wilc'ab} & \emph{dilc'ab} &
\emph{bilc'ab} & \emph{milc'adi} & \emph{midilc'adi} &
\emph{mimilc'adi}\tabularnewline
\#VC & \emph{izes} & `be ill' & \emph{mizab} &
\emph{mizadi}\tabularnewline
CVC & \emph{luč'es} & `read' & \emph{muluč'ab} &
\emph{muluč'adi}\tabularnewline
\bottomrule
\end{tabular}
\end{table}


The prohibitive frequently appears with what looks like
reduplication\is{reduplication|see{stem copy}}; more specifically, a full copy of the stem\is{stem copy} together with
the gender marker is placed to the left of the negative volitional prefix.
The process is optional.

\ea % (14)
stem copy in the prohibitive:

\gll \emph{d-iz}~\textasciitilde{}~\emph{mi-d-iz-ad} (also~\emph{mi-d-iz-ad})\\
\textsc{f1}-wash:\textsc{ipfv}~\textasciitilde{}~\textsc{negvol}-\textsc{f1}-wash:\textsc{ipfv}-\textsc{proh} \phantom{(also~}\textsc{negvol}-\textsc{f1}-wash:\textsc{ipfv}-\textsc{proh} \\
\glt `Do not wash her!'
\z

Outside its use in the prohibitive, \isi{stem copy} is relatively common in
the context of standard negation and elsewhere with a certain added
expressive or pragmatic value (cf.\ \citealt{maisak2012} on similar processes in
other East Caucasian languages). Note that the \isi{stem copy} shows the
underlying form containing the masculine prefix, not the copy of the
actual realization of the stem in this specific context. This is seen in
standard negation involving stem copies; cf.\ (\ref{ex:4:15}) and (\ref{ex:4:16}):

\pagebreak

\ea \label{ex:4:15} % (15)
stem copy in standard negation:

\gll \emph{w-ak'~\textasciitilde{}~ħa-k'-ib-i} \emph{d-ak'}~\textasciitilde{}~\emph{ħa-d-ak'-ib-i}\\
\textsc{m}-come:\textsc{pfv}~\textasciitilde{}~\textsc{neg}-\textsc{m}.come:\textsc{pfv}-\textsc{aor}-\textsc{atr} \textsc{f1}-come:\textsc{pfv}~\textasciitilde{}~\textsc{neg}-\textsc{f1}-come:\textsc{pfv}-\textsc{aor}-\textsc{atr}\\
\glt `the one who did not come'

\ex \label{ex:4:16} % (16)
\glll \emph{w-ak'-ib-i} \quad \emph{ħa-k'-ib-i}\\ 
\textsc{m}-come:\textsc{pfv}-\textsc{aor}-\textsc{atr} \quad \textsc{neg}-\textsc{m}.come:\textsc{pfv}-\textsc{aor}-\textsc{atr}\\
`the~one~who~came' \quad `the~one~who~did~not~come'\\

\glll \emph{d-ak'-ib-i} \quad \emph{ħa-d-ak'-ib-i}\\
\textsc{f1}-come:\textsc{pfv}-\textsc{aor}-\textsc{atr} \quad \textsc{neg}-\textsc{f1}-come:\textsc{pfv}-\textsc{aor}-\textsc{atr}\\
`the~one~who~came' \quad `the~one~who~did~not~come'\\
\z


The process is not reduplication sensu stricto. I call it stem copying.
Structurally, the copy of the stem may be separated from the verb form
by other material, especially by the discourse particle, as in (\ref{ex:4:17}) and
(\ref{ex:4:18}), where it forms a separate wordform.

\ea \label{ex:4:17} % (17)
\isi{stem copy} in standard negation (Corpus)

\gll \emph{illi-če-la}  \emph{iz-uwe}  \emph{werħ}  \emph{d-aʔ-i-ra}  \emph{k'ʷan}  \emph{ʡaj} \emph{inc'-ul}  \emph{d-aʔ꞊ra}  \emph{ħa-d-aʔ-i-ra}  \emph{k'ʷan.}\\
this-\textsc{super}-\textsc{el}  be.ill:\textsc{ipfv}-\textsc{cvb.ipfv}  seven  \textsc{f1}-arrive:\textsc{pfv}-\textsc{aor}-\textsc{ego}  \textsc{quot}  \textsc{ptcl} more \textsc{f1}-arrive:\textsc{pfv}꞊\textsc{add} \textsc{neg}-\textsc{f1}-arrive:\textsc{pfv}-\textsc{aor}-\textsc{ego} \textsc{quot}\\
\glt `From this (day) she fell ill, seven days, she said, it took not more
than that, she said.'

\ex \label{ex:4:18} % (18)
\gll \emph{hanna}  \emph{hete}  \emph{b-aʔ-ib-i-jaʁle}  \emph{d-uc-ib}  \emph{nu}  \emph{buʁa}  \emph{muħamma-jni} \emph{q'uq'u-be-če,}  \emph{d-uc꞊ra}  \emph{d-uc-i-le}  \emph{χal}  \emph{b-aq'-ib.}\\
now  there(\textsc{lat})  \textsc{hpl}-arrive:\textsc{pfv}-\textsc{atr}-\textsc{cvb}.\textsc{ante} \textsc{f1}-take:\textsc{pfv}-\textsc{aor}  I  Buga Muhammad-\textsc{erg} knee-\textsc{pl}-\textsc{super}(\textsc{lat})  \textsc{f1}-take:\textsc{pfv}꞊\textsc{add}  \textsc{f1}-take:\textsc{pfv}-\textsc{aor}-\textsc{cvb}  seek \textsc{n}-do:\textsc{pfv}-\textsc{aor}\\
\glt `When we arrived there, Buga Muhammad took me on his lap; having taken
me, he examined (me).'

\z

In all contexts stem\is{stem copy} copying is optional. However, it is in the
prohibitive that these forms are very consistently produced as first
translations of the Russian stimuli with the relevant meaning. It seems
that expressive pragmatics of stem copying is being incipiently
grammaticalized in the expression of the prohibitive.

\is{polarity|)}
%\is{negation|)}
\is{polarity, negative volitional forms|)}

% 7.
\section{Synthetic paradigm}\label{synthetic-paradigm}

\leavevmode{\addfontfeature{LetterSpace=-1.5}
This section gives an overview of the synthetic paradigm of the Mehweb
verb. A summary table is provided at the end of the section. Polarity,
gender and ego\-phoric subject agreement and aspectual stem formation have been
discussed above.}


The derivation of forms is summarized in the following figure. For some more exceptional patterns, including 
derivation of special converbs from general converbs or from the infinitive stem, see \citet{sheyanova2019}. (An asterisk shows morphologically bound bases.)

\begin{figure}[h]
\includegraphics[scale=.8]{schema-daniel}

% Figure 14.
\setcounter{figure}{13}
\caption{Derivation of verbal forms}
\end{figure}
\refstepcounter{table}


The aspectual stem immediately derives the past (\isi{aorist} in the
\isi{perfective}, \isi{imperfective past} in the \isi{imperfective} paradigm; note that
the forms further derived from this secondary stem, e.g.\ converbs or
participles, do not necessarily have past reference), present habitual
(in the imperfective stem only), infinitive, the imperative, the
nominalization in \emph{-ri}.

Several other forms are based on a bound (hence the use of the asterisk) base produced by
adding \emph{-a-} to the aspectual stem; this base may be considered the
base of \isi{irrealis} (potential in terms of Nina Sumbatova, unpublished),
because it produces such forms as optative, conditional, apprehensive,
counterfactual and some other (see \citealt{dobrushina2019}). Support for this
analysis not confirmed diachronically by the data from other Dargwa lects,
comes from the \isi{counterfactual} form in \emph{-are}, one of the
irrealis series, segmentable into the irrealis marker \emph{-a-} and the
past marker \emph{-re}. The latter is attested elsewhere, including on
the auxiliary in the past forms (\emph{lewre} and \emph{agʷire}) and 
probably elsewhere (\emph{꞊igibre} from
\emph{꞊igib} `want' Ipft) – see \citet{dobrushina2019}. Note
the morphophonological difference between counterfactual \emph{-re} and
the egophoric \emph{-ra} – the latter causes the past marker \emph{-ib}
to drop the final \emph{-b}, while in the counterfactual
\emph{꞊igibre} it is preserved, just as in the egophoric forms of
the verbs in the \emph{-ub} subtype.

% \pagebreak


\begin{table}[t]
% Table 15.
\caption{Verbal inflection}\label{t5-15}

\begin{tabular}{@{}lllll@{}}
\toprule
& \multicolumn{2}{l}{\emph{꞊ak'as} `come'} & \multicolumn{2}{l@{}}{\emph{ik'ʷes} `put on'} \tabularnewline \midrule
stem & \emph{꞊ak'} & \emph{꞊ik'} &
\emph{ik'ʷ} & \emph{irk'ʷ}\tabularnewline
\textsc{hab} (3) & – & \emph{꞊ik'an} & – & \emph{irk'ʷan} \tabularnewline
\textsc{hab (ego)} & – & \emph{꞊ik'as} & – &  \emph{irk'ʷas}\tabularnewline
\textsc{imp} & \emph{꞊ak'e\(na\)} & \emph{꞊ik'e\(na\)} &
\emph{ik'ʷa\(na\)} & \emph{irk'ʷe\(na\)}\tabularnewline
\textsc{inf}/\textsc{fut} & \emph{꞊ak'as} & \emph{꞊ik'es} &
\emph{ik'ʷes} & \emph{irk'ʷes}\tabularnewline
\textsc{fut (ego)} & \emph{꞊ak'iša} & \emph{꞊ik'iša} &
\emph{ik'ʷiša} & \emph{irk'ʷiša}\tabularnewline
\textsc{nmlz} & \emph{꞊ak'ri} & \emph{꞊ik'ri} & \emph{ik'ʷri} &
\emph{irk'ʷri}\tabularnewline
\textsc{ptcp} & \emph{꞊ak'ibi\(l\)} & \emph{꞊ik'ul} &
\emph{ik'ubi\(l\)} & \emph{irk'ul}\tabularnewline
\textsc{pst} (3) & \emph{꞊ak'ib} &  \emph{꞊ik'ib} &  \emph{ik'ub} &   \emph{irk'ʷib} \tabularnewline
\textsc{pst (ego)} & \emph{꞊ak'ira} & \emph{꞊ik'ira} &\emph{ik'ubra} & \emph{irk'ʷira}\tabularnewline
\textsc{cvb} & \emph{꞊ak'ile} & \emph{꞊ik'uwe} & \emph{ik'uble}
& \emph{irk'uwe}\tabularnewline
\textsc{proh} & – & \emph{mi꞊ik'adi\(na\)} & &
\emph{mirk'ʷadi\(na\)}\tabularnewline
\textsc{opt} & \emph{꞊ak'ab} & \emph{꞊ik'ab} & \emph{ik'ʷab} &
\emph{irk'ʷab}\tabularnewline
\textsc{appr} & \emph{꞊ak'ala} & \emph{꞊ik'ala} &
\emph{ik'ʷala} & \emph{irk'ʷala}\tabularnewline
\textsc{cond} & \emph{꞊ak'ak'a} & \emph{꞊ik'ak'a} &
\emph{ik'ʷak'a} & \emph{irk'ʷak'a}\tabularnewline \midrule
& \multicolumn{2}{l}{\emph{arces} `fly'} & \multicolumn{2}{l@{}}{\emph{꞊elč'es} `read'}\tabularnewline \midrule 
stem & \emph{arc} & \emph{urc} & \emph{꞊elč'} & \emph{luč'}\tabularnewline
\textsc{hab} (3) & – &  \emph{urcan} & – &  \emph{luč'an} \tabularnewline 
\textsc{hab (ego)} & – & \emph{urcas} & – & \emph{luč'as}\tabularnewline
\textsc{imp} & \emph{arce\(na\)} & \emph{urce\(na\)} & \emph{꞊elč'a\(na\)} &
\emph{luč'e\(na\)}\tabularnewline
\textsc{inf}/\textsc{fut} & \emph{arces} & \emph{urces} & \emph{꞊elč'es} &
\emph{luč'es}\tabularnewline
\textsc{fut (ego)} & \emph{arciša} & \emph{urciša} & \emph{꞊elč'iša} &
\emph{luč'iša}\tabularnewline
\textsc{nmlz} & \emph{arcri} & \emph{urcri} & \emph{꞊elč'ri} &
\emph{luč'ri}\tabularnewline
\textsc{ptcp} & \emph{arcuri\(l\)} & \emph{urcul} & \emph{꞊elč'uni\(l\)} &
\emph{luč'ul}\tabularnewline
\textsc{pst} (3) & \emph{arcur} & \emph{urcib} & \emph{꞊elč'un} & \emph{luč'ib} \tabularnewline
\textsc{pst (ego)} & \emph{arcurra} & \emph{urcira} & \emph{꞊elč'unna} &  \emph{luč'ira}\tabularnewline
\textsc{cvb} & \emph{arculle} & \emph{urcuwe} & \emph{꞊elč'uwe} &
\emph{luč'uwe}\tabularnewline
\textsc{proh} & – & \emph{murc'adi\(na\)} & – &
\emph{muluč'adi\(na\)}\tabularnewline
\textsc{opt} & \emph{arcab} & \emph{urcab} & \emph{꞊elč'ab} &
\emph{luč'ab}\tabularnewline
\textsc{appr} & \emph{arcala} & \emph{urcala} & \emph{꞊elč'ala} &
\emph{luč'ala}\tabularnewline
\textsc{cond} & \emph{arcak'a} & \emph{urcak'a} & \emph{꞊elč'ak'a} &
\emph{luč'ak'a}\tabularnewline
\bottomrule
\end{tabular}

\vspace{-3\jot}
\end{table}

The {general converb}\is{converb, general} and the \isi{participle} are formed differently in the
perfective and the imperfective paradigms. In the perfective, the
attributive marker \emph{-i\(l\)} and the converb marker \emph{-le}
are added to the aorist. In the imperfective, the participle marker
\emph{-ul} and the converb marker \emph{-uwe} are added directly to the
imperfective stem. While the \emph{-l} of the imperfective participle
marker \emph{-ul} is always present, that of \emph{-i\(l\)} is
often dropped, and the distribution of the variants is not clear (though
it seems that at least in the predicative use of the participle in
\emph{-i\(l\)} the full variant is impossible).

It seems plausible to differentiate between \emph{-ul} as the participle
marker proper, used only with the imperfective stem of the verb, and the
attributive marker \mbox{\emph{-i\(l\)}}, attached to the aorist but also
used on infinitives (to form future participles, also used finitely),
auxiliaries (to form periphrastic participles) and adjectives. Note that the
imperfective converb ending \emph{-uwe} is more or less
straightforwardly analyzable into \emph{-ul-le}, where \emph{-le} is a
general converb marker (also in the perfective paradigm) and, more
generally, is used as a cross-categorial adverbializer, i.e.\ in forming
adverbs from adjective roots.

Special\is{converb, specialized} converbs may be based on the general converb form, as the causal
converb \emph{-na}, or on the participle, as anterior converb
\emph{-\(j\)aʁle}; see more on special converb formation in
\citet{sheyanova2019}.


Unlike the \isi{nominalization} in \emph{-ri}, which is formed directly from
the aspectual stem, nominalization in \emph{-deš} is formed from many
forms, including finite past, future, present (habitual), participles –
but not from volitional forms and not from the nominalization in
\emph{-ri}. Given that \emph{-deš} is also attached to adjectives and
nouns, the generalization seems to be that \emph{-deš} is not a
derivational morpheme but a cross-categorial predicate nominalizer. The
suffix does not combine with egophoric forms.

\tabref{t5-15} % below
summarizes synthetic verbal inflection. Forms are given
without gender agreement marking; for gender agreement see \sectref{introduction-5}.
The negative prefix may attach to all forms in the table (except the imperative); 
morphology of polarity marking is discussed in \sectref{polarity}. The
marker \emph{-na} is the marker of the plural\is{plural imperative|see{imperative, plural}}\is{imperative, plural} of the addressee in
volitional forms.

% 8.
\section{Imperative and infinitive}\label{imperative-and-infinitive}

\is{imperative|(}
\is{infinitive|(}

Both the imperative and the infinitive are formed from each of the two
stems. While in the imperfective paradigm the suffixes are invariably
\emph{-e} and \emph{-es}, respectively, the perfective imperative and
the perfective infinitive / perfective non-egophoric future both have two markers (\emph{-e} vs.\ \emph{-a}
in the imperative, \emph{-es} vs.\ \emph{-as} in the infinitive). The
choice of the allomorph in the two categories is independent.

% \largerpage

\begin{table}[h]
% Table 16.
\caption{Imperative and infinitive suffixes}

\begin{tabular}{@{}lll@{}}
\toprule
& markers & choice\tabularnewline \midrule 
Perfective imperative & \emph{-e/-a} & morphosyntactic\tabularnewline
Perfective infinitive/future & \emph{-es/-as} & phonological\tabularnewline
\bottomrule
\end{tabular}
\end{table}

The choice of the imperative vowel depends on the \isi{transitivity} of the
verb: transitive verbs take \emph{-a} and intransitive verbs take
\emph{-e}. Cf.\ \emph{꞊urs-a} `pound', \mbox{\emph{꞊iʡ-aˤ}} `steal', but
\emph{꞊alħʷ-e} `wake up', \emph{꞊uq-e} `go'. Note that the choice of the
marker is primarily based on transitivity rather than control, as e.g.\
motion verbs all take~\mbox{\emph{-e}}.


% \largerpage

P-\isi{labile verbs} (i.e.\ verbs that are used with and without agentive
argument) take \emph{-e} or \emph{-a} depending on the interpretation;
cf.\ \emph{w-aˤld-e} `hide (intr)' (to a man) vs.\ \emph{w-aˤld-a} `hide
it'. Other labile verbs also show similar behavior; cf.\ \emph{abx-a}
`open (it)' vs.\ \emph{abx-e} `open (intr)'; \emph{b-oˤrʡ-a} `break
(it)' vs.\ \emph{b-oˤrʡ-e} `break (intr)'. Although in these cases the
intransitive imperative might seem unlikely, it is readily interpreted
by my consultants as when talking to something that resists acting on it,
does not yield, or seems to take too long to achieve the result. There
is evidence that A-\isi{labile verbs} (i.e.\ verbs that may omit the patientive
argument ascribing nominative to the agentive argument) may also take
both markers; cf.\ \emph{꞊erq-a} `suck (e.g.\ milk)' vs.\ \emph{꞊erq-e}
`suck' (implicit, out-of-focus patient).

Experiential verbs\is{experiential verbs} do not behave in a unified way. Generally, they
prefer the intransitive suffix, but some also allow the transitive one,
without a clear meaning shift; cf.\ \emph{qumart-a} and \emph{qumart-e}
`forget', \emph{꞊ah-e} and \emph{꞊ah-a} `know'. One would expect an
interpretation with the imperative subject’s
% addressee's
increased control over the situation
but this is certainly not consistent through all the experiential verbs,
though some consultants do report this difference e.g.\ in the verb
\emph{꞊arg-e} vs.\ \mbox{\emph{꞊arg-a}} `find'. The verb \emph{gʷes} `see' does
not form a generally accepted imperative, but if it does, the form is
\emph{gʷ-a}.

There is no alternation in the imperfective imperative. A possible way
to account for this would be to consider all imperfective imperatives as
using the intransitive imperative suffix, which would amount to transitivity
decrease with obligatory promotion of the Agent. Imperfectives are
crosslinguistically more Agent-oriented forms. In an ergative language
like Mehweb, promoting the Agent would show up as decrease in
transitivity. The assumed promotion is, however, internal to verbal
morphology and does not change argument marking. P retains nominative
case, and A, if present, is marked with ergative.


The imperative of the verb `give' has two perfective stems, \emph{aga}
and \emph{꞊ega}, depending on the person of the recipient. The first
stem is used when the \isi{recipient} is the first person, otherwise the
second stem is used. Both stems are suppletive with respect to the
non-imperative stems, and the second stem additionally introduces an
agreement prefix slot. This pattern or the verb `give' is attested
elsewhere in Dargwa and in East Caucasian at large (see \citealt{comrie2003},
also \citealt{daniel-etal2010}). Another verb with an irregular imperative stem
is \emph{es} `say' (inf)~– \emph{bet'a} `say' (imperative). The verb
 \emph{꞊uˤq'es}\is{motion verbs} `go' has two imperatives, the regular \mbox{\emph{꞊uˤq'-e}} and
the irregular \emph{꞊eˤʡ-e}. The semantic distinction is not fully clear
but probably has to do with the final point, the first better translated
as `go there' and the second as `go away, leave'. Irregular imperatives
only exist in the perfective paradigm.

The forms \emph{꞊eg-a} `give (away)' and \emph{꞊eˤʡ-e} `go (away)'
contain the expected imperative suffixes (transitive and intransitive,
respectively). On the other hand, their stems are not present elsewhere
in the paradigm of these verbs, and neither can they be causativized. The
second form is also fully suppletive to all other stems of
\emph{꞊uˤq'es} `go'. They are thus close to the status of separate
lexical items – imperative interjections\is{imperative interjection}. This becomes clear when they
are compared to another suppletive imperative stem, \emph{bet'-a} (from
\emph{es} `say'), which has a clearly different morphological status.
First, it is the only imperative available for this verb. Second, the stem \emph{bet'-} is
optionally used in other forms (see \tabref{tab:4:18}), including causatives (see \tabref{tab:4:21}).

Imperatives show plural marking based on the number\is{imperative, plural} of the addressees
(thus showing, formally, an accusative\is{accusativity} pattern of agreement). Unlike in
the prefix slot~– and, for that matter, anywhere in Mehweb – this
marking is independent from the gender. The suffix is \emph{-na} and it
is regularly attached to the imperative marker as well as to the
irregular imperatives except in the verb \emph{꞊aš-e} `come here' vs.
\emph{꞊aš-ina} `come here' (plural addressee). Cf.:

\ea % (19)
intransitive imperative

\ea \xbox{.3\textwidth}{\gll \emph{uz-e}\\
  \textsc{m}.work:\textsc{ipfv}-\textsc{imp}\\
\glt `Work!' (to one person)}%  
\exsameline
\xbox{.4\textwidth}{\gll \emph{b-uz-e-na}\\
\textsc{hpl}-work:\textsc{ipfv}-\textsc{imp}-\textsc{imp}.\textsc{pl}\\
\glt `Work!' (to many)}
\z

% \pagebreak
\ex % (20)
transitive imperative

\ea \xbox{.37\textwidth}{\gll \emph{uc-a}\\
\textsc{m}.catch:\textsc{pfv}-\textsc{imp}.\textsc{tr}\\
\glt `Catch him!' (to one person)}%
\exsameline \xbox{.4\textwidth}{\gll \emph{b-uc-a}\\
\textsc{hpl}.catch:\textsc{pfv}-\textsc{imp}.\textsc{tr}\\
\glt  `Catch them!' (to one person)}

\ex \xbox{.37\textwidth}{\gll \emph{uc-a-na}\\
\textsc{m}.catch:\textsc{pfv}-\textsc{imp}.\textsc{tr}-\textsc{imp}.\textsc{pl}\\  
\glt `Catch him!' (to many)}%
\exsameline \xbox{.4\textwidth}{\gll \emph{b-uc-a-na}\\
\textsc{hpl}.catch:\textsc{pfv}-\textsc{imp}.\textsc{tr}-\textsc{imp}.\textsc{pl}\\
\glt   `Catch them!' (to many)}

\z
\z
On imperatives in Mehweb, see more in \citet{dobrushina2019}.

The choice of \emph{-es} vs.\ \emph{-as} in the perfective infinitive/non-egophoric future forms, on the other
hand, seems to have a purely formal motivation. The default marker is
clearly \emph{-es}, while \emph{-as} is only attested in about twenty
verbs which (a) have \emph{-a-} as a stem vowel that is (b) followed by a
stem final glottal, pharyngeal, uvular or velar consonant; cf.
\emph{꞊usaˤʡʷ-as} `fall asleep', \emph{꞊aʔ-as} `begin', \emph{꞊ah-as}
`know', \emph{꞊aˤʜʷaˤs} `get wet', \emph{aq'-as} `pour', \emph{꞊aχ-as}
`nurture', \emph{꞊ak-as} `smear'. Neither (a) nor (b) alone
seem to require \emph{-a-} as the vowel of the infinitive; cf.
\emph{꞊uˤq'-es} `go' (condition b but not a) or \emph{꞊ac'-es} `melt'
(condition a but not b).

There is a number of verbs where the consonant of the required place of
articulation is separated from the -a- of the stem by another consonant.
In these cases, the default seems to be \emph{-es}, including
\emph{ask'-es} `catch on', \emph{꞊alk'ʷ-es} `burn', \mbox{\emph{abx-es}}
`open', \emph{꞊arx-es} `send', \emph{꞊arχ-es} `touch', \emph{꞊aˤlq'-es}
`rinse', \emph{꞊alħʷ-es} `wake up', \mbox{\emph{꞊aˤld-es}} `hide'. However,
some verbs, including \emph{꞊aˤlq-aˤs} `give harvest', \mbox{\emph{꞊aˤbʡ-as}}
`kill', \mbox{\emph{꞊aˤrʡ-as}} `freeze', \emph{꞊aˤrʜ-as} `copulate' do choose
\emph{-a-} as the vowel of the infinitive.
{\looseness-1\par}

\is{imperative|)}
\is{infinitive|)}


% 9.
\section{Auxiliary}\label{auxiliary}

\is{auxiliary|(}

Mehweb verbal inflection heavily relies on periphrasis. Periphrastic
forms are used e.g.\ to form \isi{progressive}/\isi{durative} or \isi{resultative}/\isi{perfective} forms (combination of a converb with the auxiliary), \isi{future}
(combination of the infinitive with the auxiliary) and others. There are
some periphrastic forms based on auxiliary use of the verb \emph{꞊uɁes}
`be' (Pfv꞊Ipfv), but most forms in the \isi{periphrastic paradigm} use one of
the auxiliaries in \tabref{tab:4:17}. Complex forms (surcomposé) are also attested, using the
auxiliary, the second auxiliary in a converb form and yet
another converb of the lexical verb.

Periphrastic forms are also used to form jussive (combination of the
imperative of the lexical verb with the imperative of the verb `say';
see \citealt{dobrushina2019}) and perfective forms from defective verbs that only
have the imperfective stem.

The same verb is also used in locative and existential predication.

Inflection of the auxiliary is presented in the following table:


\begin{table}[H]
% Table 17.
\caption{Inflection of the auxiliary}\label{tab:4:17}

\begin{tabular}{@{}lllllll@{}}
\toprule
& 3 & \textsc{ego} & \textsc{pst} & \textsc{atr} & \textsc{ptcp} & \textsc{cvb}\tabularnewline \midrule
\textsc{m} & \emph{lew} & \emph{lewra} & \emph{lewre} & \emph{lewi} &
\emph{lewili} & \emph{lewle}\tabularnewline
\textsc{f}/\textsc{npl} & \emph{ler} & \emph{lella} & \emph{lelle} & \emph{leri} &
\emph{lerili} & \emph{lelle}\tabularnewline
3/\textsc{hpl} & \emph{leb} & \emph{lebra} & \emph{lebre} & \emph{lebi} &
\emph{lebili} & \emph{leble}\tabularnewline
\textsc{neg} \textsc{loc} & \emph{agʷara} & \emph{*} & \emph{agʷire} & \emph{agʷari} &
\emph{agʷarili} & \emph{agʷalle}\tabularnewline
\textsc{neg} \textsc{equ} & \emph{aħin} & \emph{aħinna} & \emph{aħinne} & \emph{aħini} &
\emph{aħinili} & \emph{aħije}\tabularnewline
\textsc{aux} & \emph{sabi} & \textsuperscript{?}\emph{sabi\(ra\)} &
\textsuperscript{?}\emph{sabire} & (=3) & (=3) & (=3) \tabularnewline
\bottomrule
\end{tabular}
\end{table}

The form \emph{sabi} is included in the list but has a very marginal
status in Mehweb. If used at all, it has the status of a particle rather
than of a true auxiliary/copula. It is clear that the \emph{-b-} of the
stem, etymologically a gender marker, has been fossilized.

Some forms, such as the converb of imminence, are not attested. Other
special converbs are well-formed: \emph{le꞊ijaʁle, sabijaʁle,
agʷirijaʁle} (but apparently not \emph{aħinijaʁle}), causal
\emph{le꞊lena}, \emph{agʷarlena}, concessive \emph{le꞊leʡur} and
\emph{agʷarleʡur}, additive \emph{le꞊lera} and \emph{agʷarlera} etc.
Nominalizations such as \emph{le꞊deš}, \emph{le꞊ideš}, \emph{sabideš,
aħindeš,} \emph{agʷiredeš}, \emph{agʷarideš} etc.\ are easily produced.
%
\is{auxiliary|)}


% 10.
\section{Irregular verbs}\label{irregular-verbs}

\is{irregular verbs|(}

There is a number of irregular verbs, including especially motion and
caused motion verbs. Several irregular verbs show irregularly short
roots, consisting only of one consonant. In the case of \emph{es} `say'
it may be argued that it has a zero stem in the perfective. With the
exception of the bound verb *\emph{k'es} (cf.\ \emph{uruχ k'es} `be
afraid of'; the verb itself is probably historically a reduced version
of the imperfective of \emph{꞊uk'es} `say, tell' Ipfv), all
these verbs are irregular in the perfective stem, while their
imperfective stem fits one of the regular patterns of stem formation
(cf.\ \emph{lug-} `give' and \emph{luk-} `saw', \emph{irgʷ-} `see' and
\emph{irk'ʷ-} `put on', \emph{uk'-} `say' and \emph{uk-} `eat').
{\looseness1\par}

\begin{table}[p]
% Table 18.
\caption{Inflection of the irregular verbs}\label{tab:4:18}

\begin{tabular}{@{}lllll@{}}
\toprule
stem & & \emph{*k'ib} (bound) & \emph{ib} `say' &
\emph{uk'} `say'\tabularnewline 
& & \textsc{ipfv} & \textsc{pfv} & \textsc{ipfv}\tabularnewline \midrule
\textsc{hab} (3) &  & \emph{k'an} & – & \emph{꞊uk'an}\tabularnewline
\textsc{hab (ego)} &  & \emph{k'as} & – & \emph{꞊uk'as}\tabularnewline
\textsc{imp} & & \emph{k'e\(na\)} & \emph{bet'a\(na\)} &
\emph{꞊uk'e\(na\)}\tabularnewline
\textsc{inf}/\textsc{fut} & & \emph{k'es} & \emph{es} &
\emph{꞊uk'es}\tabularnewline
\textsc{fut (ego)} & & \emph{k'iša} & \emph{iša} &
\emph{꞊uk'iša}\tabularnewline
\textsc{nmlz} & & \emph{k'ari} & \emph{ari} &
\emph{꞊uk'ri}\tabularnewline
\textsc{ptcp} & & \emph{k'ul} & \emph{ibi} &
\emph{꞊uk'ul}\tabularnewline
\textsc{pst} (3) &  & \emph{k'ib} & \emph{ib} & \emph{꞊uk'ib} \tabularnewline
\textsc{pst (ego)} &  & \emph{k'ira} &  \emph{ira} & \emph{꞊uk'ira}\tabularnewline
\textsc{cvb} & & \emph{k'uwe} & \emph{ile} &
\emph{꞊uk'uwe}\tabularnewline
\textsc{proh} & & – & – & \emph{mu꞊uk'adi}\tabularnewline
\textsc{opt} & & \emph{k'ab} & \emph{\(bet'\)ab} &
\emph{꞊uk'ab}\tabularnewline
\textsc{appr} & & \emph{k'ala} & \emph{\(bet'\)ala} &
\emph{꞊uk'ala}\tabularnewline
cond & & \emph{k'ak'a} & \emph{\(bet'\)ak'a} &
\emph{꞊uk'ak'a}\tabularnewline \midrule
stem & \emph{gub} `see' & \emph{irgʷ} & \emph{gib} `give' & \emph{lug}\tabularnewline
& \textsc{pfv} & \textsc{ipfv} & \textsc{pfv} & \textsc{ipfv} \tabularnewline \midrule
\textsc{hab} (3) & – & \emph{irgʷan} & – \emph{lugan} \tabularnewline
\textsc{hab (ego)} & – & \emph{irgʷas} & – & \emph{lugas}\tabularnewline
\textsc{imp} & – & \emph{irgʷe\(na\)} & \emph{aga\(na\)}
\emph{꞊ega\(na\)} & \emph{luge\(na\)}\tabularnewline
\textsc{inf}/\textsc{fut} & \emph{gʷes} & \emph{irgʷes} & \emph{ges} &
\emph{luges}\tabularnewline
\textsc{fut (ego)} & \emph{gʷiša} & \emph{irgʷiša} & \emph{giša} &
\emph{lugiša}\tabularnewline
\textsc{nmlz} & \emph{gʷari} & \emph{irgʷri} & \emph{gari} &
\emph{lugri}\tabularnewline
\textsc{ptcp} & \emph{gubi} & \emph{irgul} & \emph{gibi} &
\emph{lugul}\tabularnewline
\textsc{pst} (3) & \emph{gub} & \emph{irgʷib} & \emph{gib} & \emph{lugib} \tabularnewline
\textsc{pst (ego)} & \emph{gubra} &  \emph{irgʷira} &  \emph{gira} & \emph{lugira}\tabularnewline
\textsc{cvb} & \emph{guble} & \emph{irguwe} & \emph{gile} &
\emph{luguwe}\tabularnewline
\textsc{proh} & – & \emph{mirgʷadi\(na\)} & – &
\emph{mulugadi\(na\)}\tabularnewline
\textsc{opt} & \emph{gʷab} & \emph{irgʷab} & \emph{gab} &
\emph{lugab}\tabularnewline
\textsc{appr} & \emph{gʷala} & \emph{irgʷala} & \emph{gala} &
\emph{lugala}\tabularnewline
\textsc{cond} & \emph{gʷak'a} & \emph{irgʷak'a} & \emph{gak'a} &
\emph{lugak'a}\tabularnewline
\bottomrule
\end{tabular}
\end{table}

Note the marker of \isi{nominalization}, usually \emph{-ri}, is
\emph{-ari} on verbs that lack any vowel of the stem (\emph{ari} for
`say', \emph{gʷari} for `see', \emph{gari} for `give'), and the presence
of two different imperatives of `give' – `give to me' and `give to
someone else'. The inclusion of the stem \emph{-uk'-} as the
imperfective counterpart to the verb \emph{es} `say' is controversial.
The two stems differ in transitivity, the former being intransitive and
the latter transitive, so that they may be considered as separate
lexical items. However, \emph{꞊uk'es} is not an equivalent of `talk
(with/to)' but is an imperfective counterpart of \emph{es} `say'. In the
perfective, it lacks any segment at all except in the imperative and
irrealis series that share the stem \emph{bet'}, which is 
optional in irrealis forms.

Further, there are several highly irregular \isi{motion verbs}. The first one
is the basic verb of motion, \emph{꞊aˤq'-\(un\)} \textasciitilde{}
\emph{꞊aš-} `go', a non-ventive verb. In both perfective and
imperfective subparadigms, two different stems are present. In the
perfective, these are \emph{꞊aˤq'-} (the participle and forms based on
the participle stem, including aorist and general converb) and
\emph{꞊uˤq'} (imperative, infinitive, future, forms based on irrealis
\emph{a}-base and the action nominal). These are stems distributed
between different perfective forms.


In the imperfective, in addition to the stem \emph{꞊aš} that possesses
the full range of forms, there are several forms based on the stem
\emph{q'ˤ-}. Attested are the synthetic present forms, the conditional
form, the action nominal, the participle and the general converb;
probably, there are other, unelicited forms. Unlike other stems, these
forms lack the gender prefix altogether. The regular perfective
\emph{꞊aˤq'uwe} designates \isi{andative} situations (`go away from here') and
implies absence of the subject at the place of speech (`he is gone').
The converb \emph{q'uˤwe} is imperfective and designates an actual
\isi{ventive} situation (`he is coming'). The converb \emph{꞊ašuwe} is also
imperfective but conveys multiple or habitual situations. The perfective
situation `he has come' is conveyed by the perfective converb of the
regular verb \emph{꞊ak'es}.

A similar meaning (probably implying that the situation of coming is
visually attested) is conveyed by the present forms \emph{q'aˤn}
(non-egophoric) and \emph{q'aˤs} (egophoric); unlike other synthetic
presents that (at least tend to) have non-episodic (habitual)
interpretations, these forms seem to be progressives. The same
irregularities are observed in the andative verb
\emph{ʡaˤr꞊aˤq'-\(un\)} (\emph{ʡaˤr꞊uˤq'-}, \emph{ʡaˤr-q'ˤ-})
\textasciitilde{} \emph{ar꞊aš-}, which is a derivation of \emph{꞊aˤq'-}.

\begin{table}[h]
% Table 19.
\caption{Inflection of the motion verb \emph{꞊uˤq'es}}\label{tab:4:19}

\begin{tabular}{@{}lp{.15\textwidth}<{\raggedright}ll@{}}
\toprule
& \textsc{pfv} & ? & \textsc{ipfv} \tabularnewline \midrule
  \textsc{hab} 3 & – & \emph{q'aˤn} & \emph{꞊ašan} \tabularnewline
 \textsc{hab (ego)} & –  &  \emph{q'aˤs} & \emph{꞊ašas}\tabularnewline
\textsc{imp} & \emph{꞊uˤq'e\(na\)}, \emph{꞊eˤʡe\(na\)} &  & \emph{꞊aše\(na\)}\tabularnewline
\textsc{proh} & – & & \emph{ma꞊ašadi}\tabularnewline
\textsc{opt} & \emph{꞊uq'aˤb} & & \emph{꞊ašab}\tabularnewline
\textsc{appr} & \emph{꞊uq'aˤla} & & \emph{꞊ašala}\tabularnewline
\textsc{cond} & \emph{꞊uq'aˤk'a} & \emph{q'aˤk'a} & \emph{꞊ašak'a}\tabularnewline
\textsc{inf}/\textsc{fut} & \emph{꞊uˤq'es} & & \emph{꞊ašes}\tabularnewline
\textsc{fut (ego)} & \emph{꞊uˤq'iša} & & \emph{꞊ašiša}\tabularnewline
\textsc{nmlz} & \emph{꞊uˤq'ri} & \emph{q'aˤri} &
\emph{꞊ašr\textbf{i}}\tabularnewline
\textsc{ptcp} & \emph{꞊aˤq'uni} & \emph{q'oˤl} & \emph{꞊ašul}\tabularnewline
\textsc{pst} 3 & \emph{꞊aˤq'un} & & \emph{꞊ašib} \tabularnewline
\textsc{pst (ego)} & \emph{꞊aˤq'unna} &  & \emph{꞊ašira}\tabularnewline
\textsc{cvb} & \emph{꞊aˤq'uwe} & \emph{q'oˤwe} & \emph{꞊ašuwe}\tabularnewline
\bottomrule
\end{tabular}
\end{table}

\pagebreak

Further, there are two perfective imperatives\is{imperative}. The difference between
them is not very clear but is probably correlated with telicity (the
presence or absence of the final point of motion), as in `go away,
leave!' (\emph{꞊eˤʡe}) and `go there!' (\emph{꞊uˤq'e}). The imperfective
imperative is interpreted either as a multiple going event (habitual
interpretation, as `be visiting them!') or as a single \isi{ventive}
imperative event (as `come here!'). Single \isi{andative} imperative events
require the use of the perfective imperative.

As to the caused motion\is{caused motion verbs|see{motion verbs}}\is{motion verbs, caused motion} verbs, there are two series of forms, one based
on \emph{k-}, the other on \emph{χ-}. To the best of my knowledge, the
two series of forms are strictly parallel and designate
bringing/fetching events, the difference essentially being between fetching or
bringing animate entities (\emph{k-}) vs.\ bringing inanimate entities
(\emph{χ-}). I will further gloss them conventionally as `lead' vs.
`bring', though the contrast is not identical to the contrast between
\emph{lead} and \emph{bring} in English. In both series, the
monoconsonantal base expresses the meaning of ventive \mbox{(\emph{k-}}~and
\emph{χ-}) and is perfective, the \emph{꞊u}C- with a gender prefix slot
is perfective and elsewhere-oriented (\emph{꞊uk-}, \emph{꞊uχ-}), and the
\emph{꞊i}C base with a gender prefix slot is imperfective and orientation
neutral (\emph{꞊ik-}, \emph{꞊iχ-}). The strictly andative meaning
`lead/bring away from here' is expressed by a verb with a prefix
(\emph{ar꞊uk-} \textasciitilde{} \emph{ar꞊ik}-; \emph{ar꞊uχ-}
\textasciitilde{} \mbox{\emph{ar꞊iχ-}}).
{\looseness1\par}

In a sense, there are two pairs of stems, C~\textasciitilde{}~\emph{꞊i}C
and \emph{u}C~\textasciitilde{}~\emph{꞊i}C, with two perfective stems
sharing one imperfective counterpart. However, similarly to the `plain'
motion verbs (see above), the relation between the stems is probably
different from that in other perfective \textasciitilde{} imperfective
stems. The \emph{꞊i}C stem seems to convey the meaning of multiple
events while the C and \emph{꞊u}C stems designate single events. As a
result, the monoconsonantal verb behaves irregularly in that it has two
converbs, perfective \emph{kile} and several specifically imperfective
forms, including the imperfective converb \emph{kuwe}, general present
forms (with actual interpretation) \emph{kas} (non-egophoric) and
\emph{kan} (egophoric), and the participle (other parallel forms may be
present but unelicited). Unlike the non-causative motion verb described
above, the supplementary episodic imperfective forms \emph{kas, kan,
kuwe} (\emph{χas, χan, χuwe}) in the imperfective share the stem with
one of the perfective series. A different look at the paradigm would be
to consider each of the verbs of caused motion as including two
different verbs, the more or less regular Pfv\textsubscript{2}
\textasciitilde{} Ipfv\textsubscript{2} and the highly defective
Pfv\textsubscript{1} \textasciitilde{} (Ipfv\textsubscript{1}), probably
with the regular verb used as andative and the irregular as ventive –
but this needs further research into semantics and usage of motion
verbs.

\begin{table}[h]
% Table 20.
\caption{Inflection of the caused motion verbs \emph{kes} `bring (animate)' and \emph{χes} `bring (inanimate)'}\label{tab:4:20}

\small
\advance\tabcolsep-1.2pt
\begin{tabular}{@{}lllllllll@{}}
\toprule
& \emph{k-} & \emph{꞊uk-} & \emph{k-} &
\emph{꞊ik-} & \emph{χ-} & \emph{꞊uχ-} &
\emph{χ-} & \emph{꞊iχ-}\tabularnewline
& \textsc{pfv}\textsubscript{1} & \textsc{pfv}\textsubscript{2} & \textsc{ipfv}\textsubscript{1} &
\textsc{ipfv}\textsubscript{2} & \textsc{pfv}\textsubscript{1} & \textsc{pfv}\textsubscript{2} &
\textsc{ipfv}\textsubscript{1} & \textsc{ipfv}\textsubscript{2}\tabularnewline\midrule
\textsc{hab} (3) & – & – \emph{kas} & \emph{꞊ikas} & – & – & \emph{χas} & \emph{꞊iχas} \tabularnewline
\textsc{hab (ego)} &  – &  – &  \emph{kan} & \emph{꞊ikan} & – & – & \emph{χan} & \emph{꞊iχan}\tabularnewline
\textsc{imp} & \emph{ka\(na\)} & \emph{꞊uka\(na\)} & & \emph{꞊ike\(na\)} &
\emph{χa\(na\)} & \emph{꞊uχa\(na\)} & & \emph{꞊iχe\(na\)}\tabularnewline
\textsc{inf}/\textsc{fut} & \emph{kes} & \emph{꞊ukes} & & \emph{꞊ikes} & \emph{χes} &
\emph{꞊uχes} & & \emph{꞊iχes}\tabularnewline
\textsc{fut (ego)} & 
\emph{kiša} & 
\emph{꞊ukiša} & 
 & 
\emph{꞊ikiša} & 
\emph{χiša} & 
\emph{꞊uχiša} & 
 & 
\emph{꞊uχiša}\tabularnewline
\textsc{nmlz} & \emph{kari} & \emph{꞊ukri} & & \emph{꞊ikri} & \emph{χari} &
\emph{꞊uχri} & & \emph{꞊iχri}\tabularnewline
\textsc{ptcp} & \emph{kibi} & \emph{꞊ukibi} & \emph{kul} & \emph{꞊ikul} &
\emph{χibi} & \emph{꞊uχibi} & \emph{χul} & \emph{꞊iχul}\tabularnewline
\textsc{pst} (3) & \emph{kib} & \emph{꞊ukib} & & \emph{꞊ikib} & \emph{χib} & \emph{꞊uχib} & & \emph{꞊iχib} \tabularnewline
\textsc{pst (ego)} &  \emph{kira} & \emph{꞊ukira} &   & \emph{꞊ikira} & \emph{χira} &  \emph{꞊uχira} &  & \emph{꞊iχira}\tabularnewline
\textsc{cvb} & \emph{kile} & \emph{꞊ukile} & \emph{kuwe} & \emph{꞊ikuwe} &
\emph{χile} & \emph{꞊uχile} & \emph{χuwe} & \emph{꞊iχuwe}\tabularnewline
\textsc{proh} & – & – & & \emph{mi꞊ikadi} & – & – & &
\emph{mi꞊iχadi}\tabularnewline
\textsc{opt} & \emph{kab} & \emph{꞊ukab} & & \emph{꞊ikab} & \emph{χab} &
\emph{꞊uχab} & & \emph{꞊iχab}\tabularnewline
\bottomrule
\end{tabular}
\end{table}

Another irregularity of the caused motion verbs is morphosyntactic:
their imperfective stem is A-labile with an antipassive pattern; see the
following section.
{\looseness1\par}


\is{irregular verbs|)}

% 11.
\section{Transitivity}\label{transitivity}

\is{transitivity|(}

In this section, I consider several transitivity related issues, first
of all morphological causativization, but also change in argument structure or case assignment
% but also change in argument structure or marking
which is not marked by morphological means –
binominative constructions and related lexical phenomena, labile verbs
and antipassive verbs. I also briefly consider another type of verbal
derivation, typologically rare, probably even limited to (and within)
East Caucasian languages – the category of verificative.

The only regular process of valency change in Mehweb is causativization\is{causative}.
Periphrastic causativization is weakly grammaticalized in Mehweb; it is
based on the verbs \emph{aʔ\(ib\)} \textasciitilde{} \emph{iʔ-} `drive, cause to
go', \emph{꞊aq\(ib\)} \textasciitilde{} \emph{꞊irq-} `let go' and \emph{꞊aq'\(ib\)}
\textasciitilde{} \emph{꞊iq'-} `do', and is discussed in detail in
\citet{barylnikova2019}. This section limits the discussion to the
causativization in morphological and lexical domains. The discussion of
morphological causatives heavily relies upon the data collected by
Ekaterina Ageeva in 2012 (unpublished field report).

Mehweb verbs are very productively causativized\is{causative} through the suffixation
of \emph{-aq-}. The suffix is identical to the perfective stem of the
verb \emph{꞊aq\(ib\)} \textasciitilde{} \emph{꞊irq-} `let go'. Grammaticalization
of `let go' into a causative marker is not surprising, but the suffix
does not have the agreement slot present on the verb. Even though the
slot might have been lost in the process of grammaticalization, the
suggested path remains somewhat speculative. The suffix may combine both
with the perfective and imperfective stem, so that each form present in
the paradigm of the original, non-causative verb, also has its
causative counterpart. Note that all causative verbs follow the
\emph{-ib} inflectional class in the aorist, independently of the
inflectional class of the lexical verb: \emph{꞊aˤʜun} `get wet' –
\emph{꞊aˤʜaˤqib} `cause to get wet', \emph{꞊arcur} `get stuck' –
\emph{꞊arcaqib} `cause to get stuck'; just as \emph{꞊ac'ib} `melt'
\emph{꞊ac'aqib} `cause to melt'; labialized verbs preserve
labialization: \emph{꞊erq'ub} `tear apart' \textasciitilde{}
\emph{꞊erq'ʷaqib} `cause to tear apart'. In a periphrastic form, the
lexical verb but not the auxiliary is causativized:

\ea % (21)
\gll \emph{b-aš-aq-u-we} \emph{le-b-re}.\\
\textsc{hpl}-go:\textsc{ipfv}-\textsc{caus}-\textsc{cvb.ipfv} \textsc{aux}-\textsc{hpl}-\textsc{pst}\\
\glt `He made them go (repeatedly).'
\z

Causatives\is{causative} are formed from verbs with all types of argument structure,
including intransitive, experiential and transitive; cf.:

\ea % (22)
causative from intransitive~(Corpus)

\gll \emph{a-b-iz-aq-ib} \emph{abzul꞊la} \emph{χalq'-ane}.\\
\textsc{pv}-\textsc{hpl}-stand.up:\textsc{pfv}-\textsc{caus}-\textsc{aor} all꞊\textsc{add} people-\textsc{pl}\\
\glt `(She) woke up everybody.'

\ex % (23)
causative from experiential verb (Magometov's texts)

\nopagebreak

\gll \emph{hanna} \emph{uzi-li-ʔini} \emph{ruzi-li-ze} \emph{b-ah-aq-ib:}\\
now brother-\textsc{obl}-\textsc{erg} sister-\textsc{obl}-\textsc{inter}(\textsc{lat}) \textsc{n}-know:\textsc{pfv}-\textsc{caus}-\textsc{aor}:\\
\glt `Then the brother announced (made it known) to the sister: ...'

\ex % (24)
causative from transitive verbs (Corpus)

\gll \emph{d-aq'-ib} \emph{duboˤʡoˤr-t} \emph{niʔ-ane,} \emph{χajagun-t,} \emph{d-aq'-ib,} \emph{si-k'al} \emph{ħa-b-erkʷ-aq-i-le} \emph{w-aq-ħa-q-ib}.\\
\textsc{npl}-do:\textsc{pfv}-\textsc{aor} dish-\textsc{pl} milk-\textsc{pl}, fried.egg-\textsc{pl} \textsc{npl}-do:\textsc{pfv}-\textsc{aor} what-\textsc{indef} \textsc{neg}-\textsc{n}-eat:\textsc{pfv}-\textsc{caus}-\textsc{aor}-\textsc{cvb} \textsc{m}-let.go:\textsc{pfv}-\textsc{neg}-\textsc{m}.let.go:\textsc{pfv}-\textsc{aor}\\
\glt `(She) prepared meals, milk products, fried eggs (she) made, she did not
let me go before I ate something.'

\z

The causative\is{causative} from the ditransitive verb \emph{g\(ib\)} \textasciitilde{}
\emph{lug-} `give' is not attested in the corpus but is well-formed. It
is, however, morphologically irregular, as with several other verbs with
monoconsonantal stems. These verbs form causatives by adding the suffix
\emph{-aχ-}.

\begin{table}[h]
% Table 21.
\caption{Irregular perfective causatives}\label{tab:4:21}

\begin{tabular}{@{}ll@{\qquad}ll@{}}
\toprule
\emph{g\(ib\)}& `give' & \emph{g-aχaq-ib}& `caused to give'\tabularnewline 
\emph{g\(ub\)}& `see' & \emph{gʷ-aχaq-ib}& `caused to see'\tabularnewline
\emph{χ\(ib\)}& `bring' & \emph{χ-aχaq-ib}& `caused to bring'\tabularnewline
\emph{k\(ib\)}& `lead' & \emph{k-aχaq-ib}& `caused to lead'\tabularnewline
\emph{i-b}& `say' & \emph{aqaq-ib}& `caused to say'\tabularnewline
\bottomrule
\end{tabular}
\end{table}

The verb \emph{es} `say' forms the imperative from each of its two
perfective stems (see \tabref{tab:4:18}), \emph{a-} (\emph{aqaqib}) and
\emph{bet'-} (bet'aqib), both meaning `caused to say'. Caused motion
verbs with irregular paradigm structure (see \tabref{tab:4:20} above) apparently
form causatives from all three stems; cf.:

\ea % (25)
causatives of caused motion \emph{꞊uχes} `bring'

\ea % a.
\gll \emph{χ-aχaq-ib}\\
bring.?-\textsc{caus}-\textsc{aor}\\
\glt `caused to bring (it)'

\ex % b.
\gll \emph{ar-uχ-aq-iša}.\\
away-\textsc{m}.bring:\textsc{pfv}-\textsc{caus}-\textsc{fut}.\textsc{ego}\\
\glt `I will cause you to be brought away (by the river).'

\ex % c.
\gll \emph{ar-m-iχ-aq-adi}\\
away-\textsc{negvol}-\textsc{m}.bring:\textsc{ipfv}-\textsc{proh}\\
\glt `Let (the river) not bring me away!'
\z
\z

The non-caused motion verb \emph{꞊uˤq'es} does not form the \isi{causative}
from its short stem \emph{q'-} (see \tabref{tab:4:19}); the two available forms
are formed from the stems \emph{꞊uˤq'-} (perfective) and \emph{꞊aš-}
(imperfective):

\ea % (26)
causatives of motion verb \emph{꞊uˤq'es} `go'

\ea % a.
\gll *\emph{q'-aq-ib},~*\emph{q'-aχaq-ib}\\
go:\textsc{ipfv}-\textsc{caus}\\

\ex % b.
\gll \emph{b-uˤq'-aq-as}\\
\textsc{hpl}-go:\textsc{pfv}-\textsc{caus}-\textsc{inf}\\
\glt `cause (them) to go' (perfective causative infinitive)

\ex % c.
\gll \emph{b-aš-aq-uwe}\\
\textsc{hpl}-go:\textsc{ipfv}-\textsc{caus}-\textsc{cvb.ipfv}\\
\glt `making them come again and again' (imperfective causative converb)
\z
\z

Irregular causatives\is{causative} in the imperfective are not attested.

Morphologically possible and accepted by many speakers are double\is{causative, double}
causatives (noted in \citealt{ageeva2014}). In some cases, the forms convey the
compositional meaning of double causation (\ref{ex:4:27}), but sometimes
consultants interpret them as single causatives (\ref{ex:4:28}). Double causatives
are not attested in the corpus; elicited examples include:

\ea \label{ex:4:27} % (27)
compositional double causatives (from Ekaterina Ageeva's
data)

\ea % a.
\gll \emph{b-elʁ-aq-aq-ib}\\
\textsc{n}-eat.full:\textsc{pfv}-\textsc{caus}-\textsc{caus}-\textsc{aor}\\
\glt `made someone feed (an animal)'

\ex % b.
\gll \emph{b-erc'-aq-aq-ib}\\
\textsc{n}-fry:\textsc{pfv}-\textsc{caus}-\textsc{caus}-\textsc{aor}\\
\glt `made someone fry (it)'

\ex % c.
\gll\emph{d-aˤʜʷ-aˤq-aq-ib}\\
    \textsc{npl}-get.wet:\textsc{pfv}-\textsc{caus}-\textsc{caus}-\textsc{aor}\\
\glt `made someone get them (feet) wet'

\ex % d.
\gll \emph{b-alk'ʷ-aq-aq-ib}\\
  \textsc{n}-burn:\textsc{pfv}-\textsc{caus}-\textsc{caus}-\textsc{aor}\\
  \glt `made someone get (it) burning'

\ex % e.
\gll \emph{b-arχ-aq-aq-ib}\\
    \textsc{n}-touch:\textsc{pfv}-caus-caus\\
\glt `made someone touch (it)'

\ex % f.
\gll\emph{b-ac'-aq-aq-ib}\\
  \textsc{n}-melt-\textsc{caus}-\textsc{caus}-\textsc{aor}\\
  \glt `made someone melt (it)'
\z

\ex \label{ex:4:28} % (28)
non-compositional double causatives (from Ekaterina Ageeva's data)

\ea % a.
\gll \emph{d-alħ-aq-aq-ib}\\
\textsc{f1}-wake.up:\textsc{pfv}-caus-caus-aor\\
\glt `woke her up'


\ex % b.
\gll \emph{w-aˤrʡ-aq-aq-ib}\\
\textsc{m}-freeze:\textsc{pfv}-\textsc{caus}-\textsc{caus}-\textsc{aor}\\
\glt  `made him freeze'

\ex % c.
\gll \emph{w-aˤbʡ-aq-aq-ib}\\
\textsc{m}-kill:\textsc{pfv}-\textsc{caus}-\textsc{caus}-\textsc{aor}\\
\glt `made someone kill him'
\z
\z

The semantic contrast between double\is{causative, double} causatives in (\ref{ex:4:28}) and the respective
simple causatives ꞊\emph{alħaqas} `cause to wake up' etc.) is unclear, if it
exists at all. Except (\ref{ex:4:28}c), all verbs in (\ref{ex:4:27}) and (\ref{ex:4:28}) are
intransitive. (The verb \emph{꞊arχ-es} `touch' means literally
`something touched on something', with a natural interpretation of
getting one's hand in contact with something. The full meaning of the
form in (\ref{ex:4:27}e) is thus `caused someone\textsubscript{i} to cause
one\textsubscript{i}'s hand to contact something'.) These are all double
causative forms elicited by Ageeva. From a comparative East Caucasian
perspective, all these meanings tend or may be labile; and some are also
labile in Mehweb (e.g.\ (\ref{ex:4:27}d). This provides a tentative explanation of
why double causatives may be limited to these verbs. A simple causative
from a labile root is usually interpreted as a causative of its
intransitive rather than transitive meaning (schematically, `burn
(tr/intr)' → `burn (intr)'-\textsc{caus} (tr)). In such uses, the
causative suffix does not derive a new transitive meaning but emphasizes
the transitive semantics already present in the lexical meaning of the
labile verb as one of its possible interpretations. It may be considered
as a disambiguation mechanism for interpreting a labile root as
expressing specifically transitive meaning. As this causative suffix
does not have exactly the same function as regular causativization, it
allows for a second marker which serves as a regular causative
derivation.

The semantics of the simple \isi{causative} forms, on the other hand, is
always compositional, unless the whole causative derivation is
lexicalized. On the special use of the causative in optative
constructions see \citet{dobrushina2019}. Examples of lexicalized causatives
are, e.g.\ \emph{꞊aʔ-aq\(ib\)} `bring back' and also `hit' -
cf.\ \emph{꞊aʔ\(ib\)} `reach' (the latter probably from `reach with
hand', lit. `cause the hand to reach'), \emph{꞊ik-aq\(ib\)}
`put right' (of a joint etc.) – cf.\ \emph{꞊ik\(ib\)} `happen' (probably
from `fall', thus `make fall in place') etc.

Some verbs are equally available in transitive and intransitive
constructions without any morphological marking of the
(de)transitivization on the verb. There are two known types of \isi{labile verbs}, P-preserving labile verbs and A-preserving labile verbs. Note
that lability is strictly lexical and limited to small classes of verbs.
Additionally, there is a phenomenon formally similar to A-labiles that
includes one verb that may be called lexical antipassive.

\begin{table}[h]
% Table 22.
\caption{Lexical valency phenomena}

\begin{tabular}{@{}llll@{}}
\toprule
& P-labiles & A-labiles & antipassives\tabularnewline \midrule
transitive & A-\textsc{erg} verb P-\textsc{nom} & A-\textsc{erg} verb P-\textsc{nom} & A-\textsc{erg} verb P-\textsc{nom} \tabularnewline
intransitive & P-\textsc{nom} verb & A-\textsc{nom} verb & A-\textsc{nom} verb P-\textsc{erg} \tabularnewline
\bottomrule
\end{tabular}
\end{table}

In other words, in comparing intransitive uses of these verbs to the
transitive ones, P-labiles suppress their A-argument; A-labiles lose
their P-argument and re-assign nominative marking to the A-argument;
and, finally, antipassives re-assign nominative marking to the
A-argument without suppressing their \mbox{P-argument} but demoting it to an
oblique slot.

With P-preserving labiles, the problem is that, in an ergative language
with pro-drop, it is hard to distinguish between a transitive verb with
an omitted \mbox{A-argument} and intransitive use of a labile verb. Cf.\ their
schematic representation in English:

\ea % (29)
`(He) cut it.'

\ex % (30)
`(He) cooked it.' / `It cooked.'
\z

Although, in my experience, the speakers easily distinguish between the
availability of the intransitive reading with labile verbs and pro-drop
with strictly transitive verbs (e.g.\ by translating into Russian and
using mediopassive for the former and a non-referential third person
plural for the latter, or else adding \emph{it happened} \emph{all by
itself} vs.\ \emph{someone did it}), some kind of formal diagnostic may
also be used. This diagnostic is provided by the morphological
distinction between transitive and intransitive imperatives\is{imperative} in the
perfective paradigm. I thus classify a verb as labile if it is judged
grammatical with both imperative endings. The following labile verbs are
attested (note that the speakers' judgements do not seem to be fully consistent):

\ea % (31)
\emph{꞊ic'\(ib\)} \textasciitilde{} \emph{꞊ilc'-} `fill'

\ex % (32)
\emph{꞊erx\(un\)} \textasciitilde{} \emph{꞊urx-} `cook'

\ex % (33)
\emph{꞊erc'\(ib\)} \textasciitilde{} \emph{꞊uc'-} `fry' (in
intransitive use with human subjects, also `straighten up')

\ex % (34)
\emph{miʔ aʔ\(ur\)} \textasciitilde{} \emph{miʔ irʔʷ-}
`freeze' (?)

\ex % (35)
\emph{꞊oˤrʡ\(oˤb\)} \textasciitilde{} \emph{꞊oˤʡ-} `break'

\ex % (36)
\emph{꞊erq'\(ub\)} \textasciitilde{} \emph{꞊iq'ʷ-} `tear
apart, wear off' (?)

\ex % (37)
\emph{abx\(ib\)} \textasciitilde{} \emph{ibx-} `open'

\ex % (38)
\emph{ʡaj-k'\(ib\)} \textasciitilde{} \emph{ʡaj-k'-} `lock'

\ex % (39)
\emph{q'aˤbʡ\(ib\)} \textasciitilde{} \emph{q'iˤbʡ-} `close'

\ex % (40)
\emph{꞊aˤld\(un\)} \textasciitilde{} \emph{꞊aˤld-} `hide'

\ex % (41)
\emph{꞊arʔ\(ib\)} \textasciitilde{} \emph{꞊irʔ-} `gather'
\z

The labile verbs designate situations that may proceed unsupervised (such as
cooking events), may both be carried out on purpose or occur
spontaneously (such as breaking or opening/closing events) or may involve both
non-human/inanimate (thus non-intentional) or human undergoers (such as
`hide' or `gather'); on the semantics of lability in East Caucasian, see
\citet{haspelmath1993,daniel-maisak-merdanova2012}.

Another test that could have been applied to Mehweb labiles is marking of egophoricity.
% Because personal agreement works on the accusative rather than ergative basis (see \citealt{ganenkov2019}), after the A-argument is suppressed,
% the remaining P-argument controls personal agreement on the verb.
 Because personal agreement works on the accusative rather than ergative basis, after the A-argument is suppressed, the remaining P-argument controls personal agreement on the verb (see Section 3.1 in \citealt{ganenkov2019}).
%
However, I have only applied the imperative test. Note that both tests
are applied to labile verbs with some difficulty, or not equally well to
all of them. Most labile verbs, in their intransitive uses, typically
take inanimate subjects and thus are not compatible with first and
second person subjects and are not easily compatible with imperatives.
In the latter case, the speakers envisage a situation of urging a
process to proceed (see \citealt{dobrushina2019}) – and most of them very easily
accommodate to this interpretation.

No special study of semantics of the transitive/intransitive pattern
alternation with labile verbs has been carried out. The following two
examples from the text indicate that, in some cases, it may be connected
to the absence of the agent, the usually transitive situation proceeding
in a spontaneous way:

% \pagebreak

\ea \label{ex:4:42} % (42)
intransitive (Corpus)

\gll \emph{urx-ne} \emph{q'-aˤb-ib} \emph{k'ʷan,} \emph{unza} \emph{ʡaj-k'-i-le} \emph{b-ik-ib}.\\
key-\textsc{pl} \textsc{pv}-close:\textsc{pfv}-\textsc{aor} \textsc{quot} door lock-\textsc{lv}:\textsc{pfv}-\textsc{aor}-\textsc{cvb} \textsc{n}-happen:\textsc{pfv}-\textsc{aor}\\
\glt `The lock has locked itself, the door closed (=locked).'

\ex \label{ex:4:43} % (43)
transitive (Corpus)

\gll \emph{abaj} \emph{hil-l-ix-i-le} \emph{r-arg-i-ra,} \emph{unza꞊ra} \emph{ʡaj-k'-i-le,} \emph{hil-l-ix-i-le} \emph{r-arg-i-ra} \emph{abaj}.\\
mother \textsc{pv}-\textsc{f}-lie.down:\textsc{pfv}-\textsc{aor}-\textsc{cvb} \textsc{f}-find:\textsc{pfv}-\textsc{aor}-\textsc{ego} door꞊\textsc{add} lock-\textsc{lv}:\textsc{pfv}-\textsc{aor}-\textsc{cvb} \textsc{pv}-\textsc{f}-lie.down:\textsc{pfv}-\textsc{aor}-\textsc{cvb} \textsc{f}-find:\textsc{pfv}-\textsc{aor}-\textsc{ego} mother\\
\glt 
`I found (my) mother already gone to bed – I discovered that, having
locked the door, she lay down.' \pagebreak[4]
\z


Note that, in these examples, there is no direct morphosyntactic
evidence of transitive vs.\ intransitive use. It is only the context that
suggests these readings.
%
In (\ref{ex:4:42}), the agent is truly absent. In (\ref{ex:4:43}), it is omitted in the converb clause (`having locked the door') under co-reference to the subject of the main clause (`mother went to bed').
%
The first episode describes a situation of
spontaneous locking of the door, leaving the master of the apartment,
unexpectedly, outside the door and unable to go inside. The second
episode tells how the narrator, coming home quite late, discovered her
mother already asleep, and the door locked (apparently, by her mother,
prior to going to bed). Very often, however, the division of labour between
transitive and intransitive constructions with \isi{labile verbs} in East
Caucasian languages is more complex, so this needs further research.

In Mehweb, most \isi{experiential verbs} are intransitive, with the
experiencer marked by the inter-lative case. Some of these verbs take
either the transitive or intransitive imperative suffix (e.g.\
\emph{꞊arg\(ib\)} \textasciitilde{} \emph{꞊urg-} `find'; \emph{꞊ah\(ur\)}
\textasciitilde{} \emph{꞊alh-} `know'; \emph{qum-art\(ur\)}
\textasciitilde{} \emph{-urt-} `forget'). For two verbs, this correlates
with a change in argument marking – the experiencer changes from
inter-lative to ergative, and its agentivity increases (`know' = `learn
(so as to know)', `forget' = `try to forget'~– see \citealt{ganenkov2019}).

A-preserving labiles are less prominent in Mehweb and, generally, in
East Caucasian, and were not collected systematically, although, in
principle, the same imperative test could have been applied. It seems
that the following is an
% \pagebreak[4]
\mbox{example} of a verb that can be used both
intransitively and transitively while preserving its A-argument:
\emph{꞊erq\(ib\)} \textasciitilde{} \emph{꞊uq-} `suck (intr and tr – e.g.\ milk)'.

Finally, two caused motion\is{motion verbs, caused motion} verbs \emph{k\(ib\)} \textasciitilde{}
\emph{꞊uk\(ib\)} \textasciitilde{} \emph{꞊ik\(ib\)} `bring
(animate object)' and \emph{χ\(ib\)} \textasciitilde{} \emph{꞊uχ\(ib\)}
\textasciitilde{} \emph{꞊iχ\(ib\)} `bring (inanimate object)' exceptionally
follow the \isi{antipassive} pattern of valency change. The verb is primarily
transitive, but, exclusively (or at least preferably) in the
imperfective, it can also be used with the A-argument in the nominative
and the P-argument in the ergative.

\ea % (44)
transitive pattern (elicited)

\gll \emph{it-ini} \emph{mura} \emph{d-iχ-ib}.\\
this-\textsc{erg} hay \textsc{npl}-bring:\textsc{ipfv}-\textsc{ipft}\\
\glt `He was bringing hay.'

\ex % (45)
antipassive pattern (elicited)

\gll \emph{it} \emph{mura-li-ni} \emph{w-iχ-ib}.\\
this hay-\textsc{obl}-\textsc{erg} \textsc{m}-bring:\textsc{ipfv}-\textsc{ipft}\\
\glt `He was bringing hay.'\pagebreak[4]
\z

This pattern, to the best of my knowledge not documented in other Dargwa
varieties, was independently confirmed by several consultants.

Some morphologically simple verbs may be considered to be `lexical\is{causative, lexical} 
causatives' with respect to other simple verbs – i.e.\ forming pairs of
verbs whose mutual relation is more or less similar to that in causative
pairs but whose stems are not morphologically related. The list cannot
be exhaustive because it largely depends on what pairs one considers to
be in causative correlation, but, in a language with highly
productive causative derivation, lexical causatives are not expected to be many. One
example is \emph{꞊ebk'\(ib\)} \textasciitilde{} \emph{꞊ubk'-} `die' –
\emph{꞊aˤbʡ\(ib\)} \textasciitilde{} \emph{꞊iˤbʡ} `kill'; the other,
already much more questionable, is \emph{q'ˤ-} \textasciitilde{} \emph{꞊aˤq'\(un\)}
\textasciitilde{} \emph{꞊aš-} `go' – \emph{k\(ib\)} \textasciitilde{} \emph{꞊uk\(ib\)}
\textasciitilde{} \emph{꞊ik\(ib\)} `lead'.

The last phenomenon related to transitivity is the binominative\is{biabsolutive construction} (alias
biabsolutive) construction. In Mehweb, as in some other East Caucasian
languages, including the languages of the Dargwa branch, periphrastic
constructions license nominative marking of both A- and P-arguments.
Binominative constructions are only available in periphrastic forms
based on imperfective converbs (see \citealt{ganenkov2019}).

\ea % (46)
binominative construction (Corpus)

\gll \emph{q'us꞊ra}  \emph{w-iʔ-i-le}  \emph{dursi-la}  \emph{širbit-la} \emph{dubilhani} \emph{b-ilh-uwe}  \emph{le-w-re}  \emph{il}.\\
be.squatted꞊\textsc{add}  \textsc{m}-sit:\textsc{pfv}-\textsc{aor}-\textsc{cvb}  daughter-\textsc{gen}  shoe-\textsc{gen}  lace \textsc{n}-tie:\textsc{ipfv}-\textsc{cvb.ipfv}  \textsc{aux}-\textsc{m}-\textsc{pst} this\\
\glt `He (lit. this one) squatted and was tying (his) daughter’s shoelace.'
\z

The alternation between the expected ergative \textasciitilde{}
nominative and the binominative pattern in the periphrastic transitive
construction has been noticed and discussed by \citet[84ff.]{magometov1982}
The semantic effect that the binominative construction brings remains
unclear; in fact, Magometov suggests that, in Mehweb, it is the
binominative construction that is more natural in imperfective
periphrasis.
For further discussion of the syntax of binominative constructions in Mehweb, see
\citet{ganenkov2019,lander2019}.

Finally, I provide some examples of what has come to be called, in
recent research on East Caucasian, the \isi{verificative} construction. This
construction has not been controlled in elicitation; the only and few
examples that I have come from the corpus. The verificative construction
based on a verb P is a complex predicate whose meaning is, speaking
formally, `verify whether P is true' or `check what/who is x such that
P(x) is true', where x is the argument of P – see the examples below.
The verbal complex essentially includes two elements~– the lexical verb
followed by the interrogative particle followed by a more or less
grammaticalized form of the verb `see'; literally, `P-whether-see'. This
construction has been previously attested in two distantly related
Lezgic languages, \ili{Archi} \citep[291]{kibrik1977} and \ili{Agul} \citep{maisak-merdanova2004},
and later also reported in \ili{Chirag} by Dmitry Ganenkov. In \citet{daniel-maisak2014,maisak2016},
various properties of the verificative construction
are discussed, including that, while various forms may appear in
elicitation, the verificative is primarily used in purposive contexts
with the infinitive (`in order to check whether\ldots{}') or in the
imperative (`go and check whether\ldots{}'). These are exactly the forms
attested in the corpus; only the copula as the main verb is attested:

\ea % (47)
infinitive verificative, no question word (corpus)

\gll \emph{nomir꞊ra} \emph{χal} \emph{b-aq'-i-ra} \emph{k'ʷan} \emph{šula-le} \emph{le-b-u-g-es}.\\
number꞊\textsc{add} seek \textsc{n}-do:\textsc{pfv}-\textsc{aor}-\textsc{ego} \textsc{quot} tight-\textsc{advz} be-\textsc{n}-\textsc{q}-\textsc{verif}-\textsc{inf}\\
\glt  `I touched the number (plate), to see whether was fixed tightly.'

\ex % (48)
imperative verificative, question word (Magometov's texts)

\gll \emph{w-eˤʡe,} \emph{ħule} \emph{w-iz-e,} \emph{či-ja} \emph{le-b-u-gʷ-a}.\\
\textsc{m}-go:\textsc{pfv} look \textsc{m}-\textsc{lv}:\textsc{pfv}-\textsc{imp} who-\textsc{q} be-\textsc{n}-\textsc{q}-\textsc{verif}-\textsc{imp}\\
\glt `Go and look, see who is there.'
\z

In all East Caucasian languages where it has so far been attested, the
verificative results from univerbation of the interrogative form of the
main verb with the verb `see'. Our consultants tend to write these forms
together in transcription; otherwise,
the only formal indication of grammaticalization in Mehweb is the loss
of labialization in infinitive
verificatives (\emph{gʷ-es} → \emph{-g-es}). In other
languages the grammaticalization process is more advanced. To understand
the position of the Mehweb verificative with respect to the parameters
previously set up for Archi and Agul, further research is needed.
%
\is{transitivity|)}


% 12.
\section{Complex verbs}\label{complex-verbs}

\is{complex verbs|(}

In Mehweb, a verbal stem is a bound morpheme that typically consists of
one syllable, followed by one or more inflectional suffixes (an exception
being the truncated optative, where no suffix follows; see \citealt{dobrushina2019}).
Pre-root slots are optional. The presence of a gender prefix is
lexically determined – formally identical roots may be different in
having or not having a gender agreement\is{agreement slot} prefix (cf.\ \emph{umc-} `weight
(\textsc{ipfv})' and \emph{꞊umc-} `swell (\textsc{ipfv})'). After the agreement prefix,
the next slot to the left is that of the inflectional marker of {negation}\is{polarity}
(either standard or volitional). Then may follow a preverbal element.
Schematically, this template may be generalized as
\textsc{Preverb}-\textsc{Negation}-\textsc{Gender}-\textsc{Root}-\textsc{Inflection}.

I consider the position of the negation prefix to be a diagnostic of a
morphologically complex verb – if it is inserted inside what otherwise
seems a verbal stem that conveys single verbal meaning, then the
morphological element preceding the negation marker is a preverbal part
of the verb, however bound it is. For verbs possessing an agreement
slot, the position of this slot is another such diagnostic. Cf.\ the verb
\emph{qumartes} `forget' where neither \emph{qum-} or \emph{-art-} is
used without the other part, yet the negation is inserted between them.
In \emph{kajʔes} `sit down', the gender prefix comes after what
historically is a spatial preverb.

\ea % (49)
`forget' \emph{qumartur} – \emph{qum-art-ur} (\textsc{pfv}), cf.\ negative \emph{qum-ħa-rt-ur}

\ex % (50)
`sit down' \emph{kajʔib – ka-jʔ-ib}, the masculine \emph{w-} is
lost after vowel – cf.\ feminine \emph{ka-d-iʔ-ib} (see \sectref{gender-agreement})
\z

Unlike negation, positioning of a gender prefix at the beginning of a
verbal form does not prove its simplex status, because the preverbal
element may have its own gender agreement position. Then, the complex
status of a verbal stem is only unambiguously tested by the position of
the negation.

\ea % (51)
`pull' \emph{bit'ak'ib} (\textsc{n}), \emph{dit'ak'ib} (\textsc{f1}), cf.\ \emph{bit'-ħa-k'-ib}
\z

There is only one bisyllabic simplex root recorded so far – a root with
two syllables not split by negation:

\ea % (52)
`fall asleep' \emph{꞊usaʔ\(un\)} \textasciitilde{} \emph{꞊usulʔ-},
cf.\ negative \emph{ħa-wsaʔun}
\z

While many East Caucasian languages use some more or less bound
preverbal morphemes, some but not all of them also have a more or less
substantial set of true preverbs (derivational verbal prefixes).
Preverbs constitute a specific subclass of preverbal elements in that
they combine with several verbal stems~– first of all, motion\is{motion verbs} and
posture verbs, and have an isolatable meaning – often, spatial. While
many Dargwa languages possess a considerable inventory of preverbs, in
Mehweb they all ceased to be productive, so that many verbs with
preverbs ended up with non-compositional meanings. On the other hand,
there is a set of verbal stems that are more or less productively used
in complex verb formation. Finally, some complex verbs are combinations
of a preverbal element and a verbal stem that are only used together, as
\emph{qum-art-} above. I will consider them in turn.

Dargwa preverbs are identifiable in Mehweb first of all on etymological
grounds. The only typical preverb formations are the prefix \emph{ar-}
`away' (\emph{ʡaˤr-} in roots with pharyngealization, see \citealt{moroz2019}) in
various \isi{motion verbs}, in which a prefix with a clear directional meaning
combines with a motion verb. All other combinations show a strong degree
of idiomatization. The presence of highly idiomatic combinations seems
to contradict Magometov's (\citeyear[74]{magometov1982}) suggestion that, in Mehweb, the
system of prefixes has not been fully developed~– rather, it passed
away, leaving behind few vestiges. Below, all
preverb~\textasciitilde{} verb combinations attested so far are given as
perfective and imperfective, the perfective also showing the aorist
suffix in parentheses; the preverbs are provided with meaning labels
suggested by \citet[74–80]{magometov1982}, who based these suggestions on
comparison with other Dargwa languages.

\ea % (53)
Preverb \emph{ar-} `away'

\ea % (a)
\emph{ʡaˤr꞊aˤq'}-(\emph{un}) \textasciitilde{}
\emph{ar꞊aš-} `go away, leave' from \emph{꞊aˤq'-} `go'

\ex % (b)
\emph{ar꞊uk-\(ib\)} \textasciitilde{} \emph{ar꞊ik-}
`lead away'; cf.\ \emph{꞊uk-} \textasciitilde{} \emph{꞊ik-} `lead'

\ex % (c)
\emph{ar꞊uχ-\(ib\)} \textasciitilde{} \emph{ar꞊iχ-} `bring
away'; cf.\ \emph{꞊uχ-} \textasciitilde{} \emph{꞊iχ-} `bring'

\ex % (d)
\emph{ar꞊ik-\(ib\)} \textasciitilde{} \emph{ar꞊irk-} `fall
down, fall out'; cf.\ \emph{꞊ik-} \textasciitilde{} \emph{꞊irk-} `happen'
(etymologically probably `fall')

\ex % (e)
\emph{ar꞊ih\(ub\)} \textasciitilde{} \emph{꞊irhʷ-} `throw away, out from
somewhere'; cf.\ \emph{꞊ih\(ub\)} \textasciitilde{} \emph{꞊irhʷ-} `throw'

\ex % (f)
\emph{ar꞊as\(ib\)} \textasciitilde{} \emph{ar꞊is-} `take away'; cf.\ \emph{as\(ib\)} \textasciitilde{} \emph{is-} `take'

\ex % (g)
\emph{ar꞊uʔ-} \textasciitilde{} \emph{ar꞊ulʔ-} `lose'; cf.\
\emph{꞊uʔ-} \textasciitilde{} \emph{꞊ulʔ-} `spoil'
% \pagebreak[3]
\z


\ex % (54)
Preverb \emph{ka-} `down'

\ea % (a)
\emph{ka-lʔ\(un\)} \textasciitilde{} \emph{k-ulʔ-} `remain'; cf.\
\emph{alʔ-\(un\)} \textasciitilde{} \emph{ulʔ-} `cut'

\ex % (b)
\emph{ka꞊at\(ur\)} \textasciitilde{} \emph{ka꞊alt-} `leave'; cf.\ \emph{꞊atur}
\textasciitilde{} \emph{꞊alt-} `put on/under (?)' (the distribution of this
verbal stem in Mehweb is further discussed below)

\ex % (c)
\emph{ka꞊iʔ\(ib\)-} \textasciitilde{} \emph{ka꞊irʔ-} `sit down'; the stem is
not attested as a free verb
\z

\ex % (55)
Preverb \emph{har-} (not discussed by Magometov, highly
idiomatized)

\ea % (a)
\emph{har꞊ik\(ib\)} \textasciitilde{} \emph{har꞊irk-} `become first';
cf.\ \emph{꞊ik\(ib\)} \textasciitilde{} \emph{꞊irk} `happen'
(etymologically probably `fall')

\ex % (b)
\emph{har꞊uq\(un\)} \textasciitilde{}
\emph{har꞊ulq-} `run away, flee'; cf. \emph{꞊uq\(un\)}
\textasciitilde{} \emph{꞊ulq} `come, enter'
\z

\ex % (56)
Preverb \emph{če-} `surface' (highly idiomatized)

\ea % (a)
\emph{če꞊uq\(un\)} \textasciitilde{} \emph{če꞊ulq-}
`grow (of plants or hair)'; cf.\ \emph{꞊uq-} \textasciitilde{}
\emph{꞊ulq} `come, enter'

\ex % (b)
\emph{če-di꞊uq\(un\)} \textasciitilde{} \emph{če-di꞊ulq-}
`become arrogant'; cf.\ \emph{꞊uq-} \textasciitilde{} \emph{꞊ulq} `come,
enter'

\ex % (c)
\emph{če꞊꞊arc}-(ur) \textasciitilde{} \emph{če꞊꞊urc}-, the verb
which is described as `unmount a horse' by \citet[76]{magometov1982} but is
only attested in his texts once meaning `stay as a guest' (Magometov's
texts, Brother and sister); cf.\ \emph{꞊arc- \textasciitilde{} ꞊urc}
`stuck'
\z

\ex % (57)
Preverb \emph{q'a-} (not discussed by Magometov)

\ea % (a)
\emph{q'-aˤbʡ\(ib\)} \textasciitilde{} \emph{q'-ibʡˤ-} `close'; cf.\
\emph{ʡaˤbʡ\(ib\)} \textasciitilde{} \emph{ʡibʡˤ-} `shut someone up; cast
someone a spell of not being able to urinate or defecate (?)'

\ex % (b)
\emph{q'a꞊ik\(ib\)} \textasciitilde{} \emph{q'a꞊irk-} `become silent,
stop'; cf.\ \emph{꞊ik\(ib\)} \textasciitilde{} \emph{꞊irk-} `happen'
\z
\z

Some preverbs are only attested with one verbal root, and thus
synchronically indistinguishable from bound preverbal elements discussed
below:

\ea % (58)
\emph{hil꞊ixib} \textasciitilde{} \emph{hil꞊irxib} `lie down
(intr)'; cf.\ \emph{꞊ixib} \textasciitilde{} \emph{꞊irxib} `put'

\ex % (59)
\emph{a꞊izur \textasciitilde{} a꞊ilzib} `stand up'; cf.\ below on
the status of the verbal stem
\z

Like many East Caucasian languages, Mehweb has verbs that combine with
various elements in preverbal position to form non-compositional (or not
fully compositional) complex verbs. Сomplex verbs show different degrees
of univerbation, which may be viewed as a decrease in compositionality of
the complex and an increase in the boundedness of the preverbal element.
The latter includes the loss of categorical transparency of the
preverbal element, from autonomous noun, adverb or adjective for which
the verbal stem serves as a verbalizer, to a bound morpheme with no
clear autonomous semantics or categorical status. Assumedly,
intermediate cases are also possible, when the preverbal element is
recognized by the speakers as a separate word but is much more often
used in a verbal complex, but this issue has not been studied, so the
orthographic solutions are somewhat arbitrary. Whenever I have no
elicited evidence that the element is only used in this complex, I write
it separately below.

The most productive verbs include \emph{꞊uh\(ub\)} `become' and
\emph{꞊aq'\(ib\)} `do'. When combining with adjectives (the short form,
lacking the attributivizer \emph{-\(i\)l}), the two verbs form
inchoative \textasciitilde{} causative\is{causative}
% {causative (inchoative \textasciitilde{} causative alternation)}
pairs. Cf.\ \emph{ara ꞊uhes}
`recover' lit.\ `healthy become', \emph{ara ꞊aq'as} `heal' lit.\ `healthy
do' from \emph{ara\(l\)} `healthy'. Other verbs are only exceptionally
attested in inchoative constructions. I have one example: \emph{ʡaˤrʁa
꞊aʔib} `stretch'; cf.\ \emph{ʡaˤrʁa\(l\)} `long' and \emph{꞊aʔas} `begin'.

The verbs \emph{꞊uh\(ub\)} `become' and \emph{꞊aq'\(ib\)} `do' also form
less compositional derivations with nouns or elements of synchronically
unclear categorical status, e.g.\ \emph{deħ buh\(ub\)} `start stinking'
(\emph{deħ} `smell'), \emph{gʷer} \emph{baq'\(ib\)} `rock (a cradle)',
\emph{χal-baq'\(ib\)} `seek', \emph{dam-baq'\(ib\)} `beat up'.

The verb \emph{ib} `say' (\textsc{pfv}) is used in complex verbs designating
sound production or similar (\emph{šʷaˤt' ib} `whistle', \emph{tu ib}
`spit', \emph{aˤmču ib} `sneeze' etc.) The recorded complex verbs
designating motion\is{motion verbs} are based on the verb \emph{꞊uq\(un\)} \textasciitilde{}
\emph{꞊ulq} `come, enter' which has a limited distribution as a free verb but
is also used with prefixes (see above), or in combination with an adverbial
element \emph{dur\(a\)} `outside' in \emph{dura ꞊uq\(un\)} `exit'. The
complex verbs with \emph{꞊uq\(un\)}~\textasciitilde{} \emph{꞊ulq} `move, enter'
include \emph{t'aˤʜ ꞊uq\(un\)} `jump', \emph{čaˤχ ꞊uq\(un\)} `slip',
\emph{duc' ꞊uq\(un\)} `run', \emph{tir ꞊uq\(un\)} `wander' – it seems such
verbs tend to designate quick movement. The verb \emph{꞊aˤq\(ib\)
\textasciitilde{} ꞊irqˤ} `hit' is used in several complex verbs, from
highly compositional \emph{k'ʷama ꞊aˤq\(ib\)} `churn butter'
(\emph{k'ʷama} `butter') and \emph{urculi ꞊aˤq\(ib\)} `chop wood'
(\emph{urculi} `firewood') to non-transparent verbs with no common
semantic denominator, \emph{kal ꞊aˤq\(ib\)} `go stale' (\emph{kal}
`stale'), \emph{ʡaˤš꞊aˤq\(ib\)} `come back' and \emph{uruχ ꞊aˤq\(ib\)}
`become afraid'. The meaning `be afraid' in the imperfective may also be
rendered by \emph{uruχ k'-}, where \emph{k'-} is a bound verbal stem
only attested in the imperfective. It could be that the difference
between the two imperfective verbs, \emph{uruχ ꞊irqˤ\(ib\)} and \emph{uruχ
k'\(ib\)} is that between multiple episodic events (true imperfective of
\emph{uruχ ꞊aˤq\(ib\)}) vs.\ state, respectively – but the evidence for
this is not sufficient.

Other verbs include completely non-compositional combinations with roots
which do not serve as productive verbalizers, so that identification of
a light verb with a lexical verb is fully formal. These include:

\ea % (60)
\emph{xar b-aʔ\(ib\)} `ask' cf.\ \emph{꞊aʔ\(ib\)} `begin'

\ex % (61)
\emph{q'ac' b-ik\(ib\)} `bite' cf.\ \emph{꞊ik\(ib\)}
`happen' (\textless{}* `fall'?)
\z

While the common way of univerbation is the increase in boundedness of
the preverbal adverb or nominal with the stem of a free
verb, several complex verbs contain a stem whose identification is
problematic. Attested cases are:

\ea % (62)
\emph{miʔ aʔ\(ur\) \textasciitilde{} irʷ-} `freeze' (cf.\ \emph{miʔ}
`ice')

\ex % (63)
\emph{dub aˤʡib \textasciitilde{} ilʡˤ-} `eat' (cf.\ \emph{dub d-at\(ur\)} or \emph{b-uc\(ib\)} `be fasting')

\ex % (64)
\emph{qum-art-\(ur\) \textasciitilde{} qum-urt-}
`forget'

\ex \label{ex:4:65} % (65)
\emph{꞊uħ\(a\)-aq'-} (\textsc{ipfv} only?) `talk'

(note the absence of the agreement slot, thus not \emph{꞊aq'\(ib\)} `do')

\ex \label{ex:4:66} % (66)
\emph{꞊it'\(a\)-ak'\(ib\)} \textasciitilde{} \emph{꞊it'\(a\)-irk'-} `drag'

\ex \label{ex:4:67} % (67)
\emph{ʡaj-k'\(ib\)} \textasciitilde{} \emph{ʡaj-k'-} `lock'
\z

In (\ref{ex:4:65}) and (\ref{ex:4:66}), the (\emph{a}) appears before the negative prefix,
and is otherwise lost before the vowel of the stem. The verb in (\ref{ex:4:67}) has a negative form  \emph{ʡajk'-ħa-jk'-an} `does not (usually) lock', which suggests an underlying structure of the positive forms looking something like \emph{ʡajk'-k'\(ib\)}, with the two occurrences of \emph{k'} merging in one.

Two cases have an especially unclear morphological status in terms of
(un)boundedness of the verbal root. First, the verbal root
\emph{꞊at\(ur\)}~\textasciitilde{} \emph{꞊alt} seems to mean `put' (probably
from the original meaning `leave'), but it is a markedly rare choice in
this meaning (the common verb for `put' is \emph{꞊ix\(ib\)}). The
stem is much more common in several non-compositional structures,
including the prefixal verb \emph{ka꞊at\(ur\)} \textasciitilde{} \emph{ka꞊alt-}
`leave behind, lose' (also causative \emph{ka꞊at-aq-} `kidnap (cause to
be lost?)'), with designation of clothes meaning `take off', the noun
\emph{ši} `sting' (meaning `sting (verb)'), the apparently bound element
\emph{dub} (meaning `hold fast', cf.\ also \emph{dub buc\(ib\)} `hold fast'
and \emph{dub aˤʡ\(ib\)} `eat'), the word \emph{c'urʔa} in the sense
`become/leave orphan' and the spatial form \emph{hune꞊} `on the road'
meaning `see off' (`leave/put on the road'?). But it is also used in the
construction \emph{꞊atur ꞊aʔas} `let (someone pass/go)', where what
appears to be a finite form (an aorist \emph{꞊atur}) is used in apparent
subordination to the verb `begin'/'arrive'. Another probable use is the
complex verb \emph{waˤb-aˤt\(ur\)} \textasciitilde{} \emph{waˤb-aˤlt-} `call out'.
The verbal stem is similar, but, first, the putative
\emph{꞊at\(ur\)}~\textasciitilde{} \emph{꞊alt} is irregularly pharyngealized
(probably, pharyngealization has spread from the preverbal component,
but this is an irregular process, because pharyngealization in Mehweb
usually spreads leftwards – see \citealt{moroz2019}). And, second, in negative
forms, the \emph{b} splits in two (\emph{waˤb-ʜa-baˤt\(ur\)}). This may
mean that the former gender marker, now frozen because it was controlled
by the lexical noun which was the source of the bound preverbal element
\emph{waˤb-}, fused with the final \emph{-b} of this element when there
was no intervening negation prefix. But this process, again, is
irregular.

Second, the verbal root \emph{꞊iz\(ib\)} \textasciitilde{} \emph{꞊ilz-} is
attested with a preverb (see \emph{a꞊iz\(ib\)} `stand up' above), in
\emph{tir} \emph{꞊iz\(ib\)} \textasciitilde{} \emph{꞊ilz-} `turn
around' (cf.\ \emph{tir ꞊uq\(un\)} `wander, go in circles' above), and in
the expression \emph{ħule} \emph{꞊iz\(ib\)}, where \emph{ħule} is
an unclear form related to the noun `eye', while the complex verb agrees
with the subject – the one who looks). Otherwise, the verb
\emph{꞊iz\(ib\)}/\emph{꞊ilz-} does not seem to be used alone.

Finally, there are some idiomatic combinations of words of different
categories with verbs, showing more or less clear paths of semantic
derivation, e.g.\ \emph{liħi bixes} `listen' – lit.\ `ear put';
\emph{surat diltes} `draw', lit.\ `take out image'; \emph{himi abizes}
`become angry', lit.\ `the bill raises', \emph{aqu ihʷes} `cover',
lit.\ `throw up'; and less transparent synchronically \emph{žuχ wiʔ\(ib\)}
`urinate' and \emph{k'uč'e wiʔ\(ib\)} `defecate' – cf.\ the same root as a
bound root in \emph{ka꞊iʔ\(ib\)} \textasciitilde{} `sit down';
\emph{ask'es ꞊erχʷes} `fight' (lit.\ `catch/cling go') etc.
%
\is{complex verbs|)}


\section*{List of abbreviations}

\begin{longtable}[l]{@{}ll@{}}
\textsc{add}	& additive particle \\
\textsc{advz}	& adverbializer \\
\textsc{ante}	& anteriority converb \\
\textsc{aor}	& aorist \\
\textsc{appr}	& apprehensive \\
\textsc{atr}	& attributivizer \\
\textsc{aux}	& auxiliary \\
\textsc{caus}	& causative \\
\textsc{cond}	& conditional \\
\textsc{cvb}	& converb \\
\textsc{ego}	& egophoric \\
\textsc{el}	& motion from a spatial domain \\
\textsc{erg}	& ergative \\
\textsc{f}	& feminine (gender agreement) \\
\textsc{f1}	& feminine (unmarried and young women gender prefix) \\
\textsc{fut}	& future \\
\textsc{gen}	& genitive \\
\textsc{hab}	& habitual (durative for verbs denoting states) \\
\textsc{hpl}	& human plural (gender agreement) \\
\textsc{imp}	& imperative \\
\textsc{indef}	& indefinite particle \\
\textsc{inf}	& infinitive \\
\textsc{inter}	& spatial domain between multiple landmarks \\
\textsc{ipft}	& imperfect \\
\textsc{ipfv}	& imperfective (derivational base) \\
\textsc{lat}	& motion into a spatial domain \\
\textsc{loc}	& locative converb \\
\textsc{lv}	& light verb \\
\textsc{m}	& masculine (gender agreement) \\
\textsc{n}	& neuter (gender agreement) \\
\textsc{neg}	& negation (verbal prefix) \\
\textsc{negvol}	& negation in volitional forms (negative imperative, negative optative) \\
\textsc{nmlz}	& nominalizer \\
\textsc{nom}	& nominative \\
\textsc{npl}	& non-human plural (gender agreement) \\
\textsc{obl}	& oblique (nominal stem suffix) \\
\textsc{opt}	& optative \\
\textsc{pfv}	& perfective (derivational base) \\
\textsc{pl}	& plural \\
\textsc{proh}	& prohibitive \\
\textsc{pst}	& past \\
\textsc{ptcl}	& particle \\
\textsc{ptcp}	& participle \\
\textsc{pv}	& preverb (verbal prefix) \\
\textsc{q}	& question (interrogative particle) \\
\textsc{quot}	& quotative (particle) \\
\textsc{super}	& spatial domain on the horizontal surface of the landmark \\
\textsc{tr}	& transitive \\
\textsc{verif}	& verificative \\
\end{longtable}


\nocite{kustova2019,daniel2018:aspectual,daniel-maisak-merdanova2012}
\printbibliography[heading=subbibliography,notkeyword=this]

\end{document}

%%% Local Variables:
%%% mode: latex
%%% TeX-master: "../main"
%%% End:
