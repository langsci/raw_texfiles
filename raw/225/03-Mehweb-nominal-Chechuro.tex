\documentclass[output=paper]{langsci/langscibook} 
\ChapterDOI{10.5281/zenodo.3402058}

% Chapter 3

\title{Nominal morphology of Mehweb}

\author{Ilya Chechuro\affiliation{Linguistic Convergence Laboratory, National Research University Higher School of Economics}} % Yu. 

\abstract{This paper describes the nominal morphology of Mehweb. It deals with the following issues: the nominal paradigm, plural formation, the oblique stem, case formation and use, and irregular locatives. In this paper I analyze both the structure and the semantics of these forms.

\emph{Keywords}: nominal inflection, case, number, locative.}

\begin{document}
\maketitle


% 1.
\section{Introduction}

In this paper, I consider the following aspects of Mehweb grammar:
\begin{enumerate}[topsep=\medskipamount,itemsep=0pt,partopsep=0pt,parsep=0pt]
\item
  Nominal paradigm structure
\item
  Plural formation
\item
  Oblique stem formation
\item
  Grammatical cases\is{grammatical case}
\item
  Irregular locatives
\item
  Inflection of place names
\end{enumerate}

Since gender is not marked on nouns and is only reflected in verb
agreement, this aspect of the grammar is discussed in the chapter on
verbal morphology \citep{daniel2019}.

% 2.
\section{Structure of the nominal paradigm}

The the nominative singular form is identical to the nominal root.
Mehweb also has two intermediate derivational stems, the oblique stem
and the plural stem. The oblique stem is derived from the root by an
affix or, much more rarely, through a non-segmental operation, and
further derives all inflectional forms other than the nominative and the genitive case in
the singular, including the ergative case. The rules of oblique stem
formation are described in \sectref{oblique-stem}. The plural stem is derived from
the root and attaches plural suffixes. The rules of plural stem
formation are specific to each of the plural suffixes and are discussed
in the sections dealing with the corresponding suffixes. In the plural,
case suffixes follow the plural suffix. Figure \ref{fig:3:1} describes the general
mechanism of the formation of the plural and oblique stems, starting
from the root of a noun:

\begin{figure}[h]
 \vspace{-24pt}
\includegraphics[width=\textwidth]{Stem-Formation-2-crop}
  % Figure 1.
\caption{Plural and oblique stem formation}\label{fig:3:1}
\end{figure}


Or, in tabular form:

\begin{table}[b]
  % Table 1.
  \caption{Possible noun forms}\label{tab:3:1}


\begin{tabular}{@{}lll@{}}
\toprule
{Stem} & {Slot 1} & {Slot 2}\tabularnewline \midrule
Nominative stem & (\textsc{nom}) &\tabularnewline
Nominative stem & \textsc{gen} &\tabularnewline
Oblique stem & \textsc{dat/gen/erg/comit/repl/subst} &\tabularnewline
Oblique stem & localization suffix (see \sectref{nominal-inflection-system}) & orientation suffix (see \sectref{nominal-inflection-system})\tabularnewline
Plural stem+PL & (\textsc{nom}) &\tabularnewline
Plural stem+PL & \textsc{dat/gen/erg/comit/repl/subst} &\tabularnewline
Plural stem+PL & localization suffix (see \sectref{nominal-inflection-system}) & orientation suffix (see \sectref{nominal-inflection-system})\tabularnewline
\bottomrule
\end{tabular}
\end{table}

\pagebreak

As one can see from \tabref{tab:3:1}, the first slot is occupied by case or
localization markers, while the second slot is restricted to the
orientation markers and can be filled only if there is a localization
marker in the first slot.

Henceforth I distinguish between the \emph{stem} and the
\emph{root} of a word. The root of a word is the deepest level of
underlying representation of the unchangeable part of a noun, which
usually coincides with the nominative. The only exception is overt
gender marking, which is only characteristic of some nouns of adjectival origin,
such as \emph{uqna} `old man' (plural \emph{b-uqna-r-t} `old men'). Here the
gender markers, which are not part of the root, are present in both
singular and plural forms (masculine singular \emph{w-} is assimilated
with the [u] in the beginning of the word, \emph{b-} stands for the human plural). In this and similar cases I consider the gender
(also called \emph{class}) agreement slot a part of the root and mark it
as \textsc{cl}. This definition is slightly different from the canonical one
given in \citet[19]{haspelmath-sims2010} where the root is defined as the part of a
lexeme that remains after all affixes have been removed. I assume the agreement
slot (but not the marker itself) to be part of the nominal root.

The stem is a representation of a root, including intermediate
phonological and morphological representations. Thus, the root is an
abstraction that can correspond to a number of different stems, as in
the two forms in \tabref{tab:3:1}: \emph{ʁarʁa} `stone:\textsc{nom}.\textsc{sg}' and
\emph{ʁarʁ-u-be} `stone-\textsc{pl}.\textsc{stem}-\textsc{pl}'. In this example the root
is \emph{ʁarʁa}, while the stems are \emph{ʁarʁa} and \emph{ʁarʁ-u}.
Stems are never used without case suffixes (assuming a zero affix in the
nominative) and thus are also an abstraction.

The nominal paradigm of Mehweb consists of two parts (or sub-paradigms):
\emph{grammatical}, or \emph{functional}, cases and \emph{locative
forms}. The two types differ in their morphology: functional case
markers consist of one inflectional morpheme; locative forms include two
inflectional slots: \emph{localization} (LOC) and \emph{orientation}
(OR).

There are a number of nominal inflectional forms that can be
historically analyzed as former locatives but are synchronically
monomorphemic. These are \emph{the causal}, \emph{the substitutive},
\emph{the replicative} and probably \emph{the comitative}.

% 3.
\section{Plural}

\is{plural|(}

The description of plural formation in this chapter is based on
wordlists presented in \citet{magometov1982} and lexical data collected by
George Moroz during the field trips undertaken in 2013–2016 (\citealt{moroz2019} [this volume]).

% \largerpage

The category of number distinguishes three values: singular, plural and
associative. The singular is not marked. The plural is marked with the\pagebreak[4]
following suffixes: \emph{-t}, \emph{-be}, \emph{-me}, \emph{-ne},
\emph{-e}, \emph{-le}, \emph{-he}, \emph{-re}, \emph{-še}, \emph{-nube},
\emph{-tune}, \emph{-urbe}, \emph{-lume}. The associative plural suffix
is \emph{-qale}.

The suffixes \emph{-t}, \emph{-be}, \emph{-me}, \emph{-ne}, \emph{-e}
are frequent. The suffixes \emph{-le}, \emph{-he}, \emph{-re},
\emph{-še}, \emph{-nube}, \emph{-tune}, \emph{-urbe}, and \emph{-lume} are
limited to small classes of nominal stems.

% \pagebreak[4]

Strictly speaking, the choice of the plural suffix is lexical. In most
cases, it cannot be predicted either from  the formal properties of the
stem or from the semantics of the noun. The plural stem formation is not
always predictable, either.

On the other hand, each plural suffix has certain – and sometimes quite
strong~– constraints on the phonotactic structure of the stems to which
it can attach. There are different rules of plural stem formation for
different affixes, which, however, involve partially similar patterns.
For instance, the suffix \emph{-e} only attaches to one-syllable stems
(\sectref{the-plural-suffix--e}) and the suffix \emph{-re} usually changes the root vowel of
one-syllable nominative stems to [u] (\sectref{the-plural-suffix--re}). Another almost universal
process is final vowel syncope, which affects all stems except
monosyllabic words and borrowings. However, though the processes
discussed in this chapter often apply to most of the formally eligible
nouns, almost none of them is truly obligatory.

Below, I attempt to formalize (to some extent) the rules of plural
formation. Each of the subsections deals with a particular suffix. In
each subsection, I describe the restrictions observed, based on the
dictionary data and the data from \citet{magometov1982}. For the suffixes
\emph{-ne}, \emph{-e}, \emph{-le}, \emph{-he}, \emph{-re}, \emph{-še},
\emph{-nube}, \emph{-tune}, \emph{-urbe}, \emph{-lume}, and partly also for
the suffix \emph{-me}, I have been able to specify the classes of nouns that
take these suffixes. For the other suffixes, I have only been able to
specify the stem changes they cause.

I will use the following abbreviations: C for consonants, V for vowels,
R for sonorants.

% 3.1.
\subsection{The plural suffix \emph{-t}}
  \label{the-plural-suffix--t}

The plural suffix \emph{-t} is one of the most productive suffixes found with this function. In the presence of
this suffix, the stem undergoes the following changes:
\begin{enumerate}[topsep=\medskipamount,itemsep=0pt,partopsep=0pt,parsep=0pt,label={\arabic*})]
\item % 1)
  If a stem ends in a vowel, the vowel is dropped. The [a] of the
penultimate syllable changes to [u]\footnote{If this vowel is
  pharyngealized, it changes into [oˤ], the phonetic realization of
  /uˤ/.}. This rule does not apply to borrowed stems.

\item % 2)
  If a stem ends in a sonorant or [b], including after (1) is
applied, the plural suffix \emph{-t} can be attached directly to it.

\item % 3)
  If a stem is borrowed (or contains a borrowed morpheme), the plural
stem is formed by inserting the element \emph{-r-} (unless it ends in a
sonorant).

\item %  4)
  The word \emph{uqna} `old man' forms the plural stem by inserting \emph{-r-}
even though it is not borrowed.
\end{enumerate}

The rough generalization is that the suffix \emph{-t} attaches to stems
ending in sonorants.

% \pagebreak[4]

\tabref{tab:3:2} illustrates vowel drop and vowel change (Rule 1):

\begin{table}[h]
  % Table 2.
  \caption{Rule 1}\label{tab:3:2}
\begin{tabular}{@{}lll@{}}
\toprule
& \textsc{sg} & \textsc{pl}\tabularnewline \midrule
`a piece of firewood' & \emph{urculi} & \emph{urcul-t}\tabularnewline
`broom' & \emph{buškala} & \emph{buškul-t}\tabularnewline
`flue' & \emph{zamari} & \emph{zamur-t}\tabularnewline
`border' & \emph{durʡaˤri} & \emph{durʡoˤr-t}\tabularnewline
`mountain' & \emph{dubura} & \emph{dubur-t}\tabularnewline
`sunny hillside' & \emph{burhala} & \emph{burhul-t}\tabularnewline
`waterfall' & \emph{rurqaˤni} & \emph{rurqoˤn-t}\tabularnewline
\bottomrule
\end{tabular}
\end{table}

\tabref{tab:3:3} illustrates the second rule:

\begin{table}[h]
  % Table 3.
  \caption{Rule 2}\label{tab:3:3}
  
\begin{tabular}{@{}lll@{}}
\toprule
& \textsc{sg} & \textsc{pl}\tabularnewline \midrule
`blacksmith' & \emph{ustar} & \emph{ustar-t}\tabularnewline
`spoon' & \emph{k'uc'ul} & \emph{k'uc'ul-t}\tabularnewline
`bridle' & \emph{hurhur} & \emph{hurhur-t }\tabularnewline
`horse' & \emph{ʡaˤbul} & \emph{ʡaˤbul-t}\tabularnewline
`a piece of dry dung' & \emph{kupar} & \emph{kupar-t}\tabularnewline
`cauldron' & \emph{qazam} & \emph{qazam-t}\tabularnewline
`sack' & \emph{halban} & \emph{halban-t}\tabularnewline
`hand mill' & \emph{ulχab} & \emph{ulχab-t}\tabularnewline
`fairytale' & \emph{χabar} & \emph{χabar-t}\tabularnewline
`dream' & \emph{muʔer} & \emph{muʔer-t}\tabularnewline
\bottomrule
\end{tabular}
\end{table}


\tabref{tab:3:4} shows how the \emph{-t} suffix interacts with borrowed stems
ending in a vowel: the vowel drop does not apply.

\begin{table}
% \begin{longtable}{@{}llll@{}}
  % Table 4.

   \caption{Rules 3 and 4} \label{tab:3:4}
 \begin{tabular}{@{}llll@{}}
%  \caption{Rules 3 and 4 (continued)} % \tabularnewline \toprule
% & \textsc{sg} & \textsc{pl} & {Source}\tabularnewline \midrule  \endhead
%  \caption{Rules 3 and 4} \tabularnewline
 \toprule
 & \textsc{sg} & \textsc{pl} & {Source}\tabularnewline \midrule %  \endfirsthead
`reaper' & \emph{irxanči} & \emph{irxanči-r-t} & Turkic suffix
\emph{-či}\tabularnewline
`hunter' & \emph{awči} & \emph{awči-r-t} & Turkic \emph{avči}
`hunter'\tabularnewline %  \pagebreak[4]
`old man' & \emph{uqna} & \emph{b-uqna-r-t}\footnotemark &\tabularnewline
`time' & \emph{zamana} & \emph{zamana-r-t} & Arabic \emph{zamaːn}
`time'\tabularnewline
`sign' & \emph{išara} & \emph{išara-r-t} & Arabic \emph{ʔišaara}
`sign'\tabularnewline
`mine' & \emph{šaχta} & \emph{šaχta-r-t} & Russian \emph{šaχta} `mine'\tabularnewline
`car' & \emph{mašina} & \emph{mašina-r-t} & Russian \emph{mašina} `car'\tabularnewline
`oppression' & \emph{zulmu} & \emph{zulmu-r-t} & Arabic \emph{ðulm}
`injustice'\tabularnewline
`carriage' & \emph{ʡaˤraba} & \emph{ʡaˤraba-r-t} & Arabic \emph{ʕaraba}
`car'\tabularnewline
\bottomrule
  \end{tabular}

  %\end{longtable}
\end{table}


\pagebreak


Borrowed stems that end in a sonorant attach the \emph{-t} suffix
directly, as illustrated in \tabref{tab:3:5}:


\footnotetext{The word
  \emph{uqna} also contains a gender marker, which expresses the number
  and gender of this word. Thus, in the singular the marker is
  masculine singular \emph{w-} (dropped before the [u] of the
  stem), while in the plural the human plural marker \emph{b-} occurs.
  Several other nouns in Mehweb and other Dargwa dialects also include a
  gender marker.}

\begin{table}[H]
  % Table 5.
  \caption{Borrowed stems that attach the suffix \emph{-t} directly}\label{tab:3:5}
  
\begin{tabular}{@{}lll@{}}
\toprule
& \textsc{sg} & \textsc{pl}\tabularnewline \midrule
`sugar' & \emph{čakar} & \emph{čakar-t}\tabularnewline
`paper' & \emph{kaʁar} & \emph{kaʁar-t}\tabularnewline
`town' & \emph{šahar} & \emph{šahar-t}\tabularnewline
`soap' & \emph{sapun} & \emph{sapun-t}\tabularnewline
`person' & \emph{insan} & \emph{insan-t}\tabularnewline
`cure' & \emph{darman} & \emph{darman-t}\tabularnewline
`regent' & \emph{ħakim} & \emph{ħakim-t}\tabularnewline
`agronomist' & \emph{agranum} & \emph{agranum-t}\tabularnewline
`member' & \emph{čilen} & \emph{čilen-t}\tabularnewline
`table' & \emph{ustul} & \emph{ustul-t}\tabularnewline
`sack' & \emph{čantaj} & \emph{čantaj-t}\tabularnewline
\bottomrule
\end{tabular}
\end{table}

The plural suffix \emph{-t} also forms plurals of the words that denote
inhabitants of Mehweb and neighbouring villages. In \citet{magometov1982} this
use of the suffix \emph{-t} is described as a separate suffix
\emph{-n-t}. However, forms such as \emph{mehʷa-n} `a Mehweb
person' and \emph{surʁatla-n} `a~person from the village of Sogratl
suggest that \emph{-n} is a nominalizer and therefore not part of the
plural morpheme (see \tabref{tab:3:38} in \sectref{irregular-locatives}).

% 3.2.
\subsection{The plural suffix \emph{-ne}}
\label{the-plural-suffix--ne}

With the suffix \emph{-ne}, the stem undergoes the following changes:
\begin{enumerate}[topsep=\medskipamount,itemsep=0pt,partopsep=0pt,parsep=0pt,label={\arabic*})]
\item % 1)
  If a stem ends in a vowel, the vowel is dropped.

\item % 2)
  One-syllable words form the plural stem by attaching the morpheme
\emph{-a-}.

\item % 3)
  If the stem has two or more syllables and ends in a consonant,
including after Rule 1 has been applied, the plural stem is derived by
attaching the morpheme \emph{-u-}.
\end{enumerate}

\tabref{tab:3:6} illustrates the first rule:

\begin{table}[h]
   % Table 6.
  \caption{Rule 1}\label{tab:3:6}

  \begin{tabular}{@{}c@{\qquad}c@{}}
\toprule
    \begin{tabular}[t]{@{}lll@{}}
& \textsc{sg} & \textsc{pl}   \tabularnewline \midrule
`axe' & \emph{barda} & \emph{bard-ne}     \tabularnewline
`spring' & \emph{derga} & \emph{derg-ne}  \tabularnewline
`dew' & \emph{marka} & \emph{mark-ne}     \tabularnewline
`honey' & \emph{warʔa} & \emph{warʔ-ne}   \tabularnewline
`stain' & \emph{dabʁa} & \emph{dabʁ-ne}  \tabularnewline
`pile' & \emph{bek'a} & \emph{bek'-ne}    \tabularnewline
`mosquito' & \emph{k'ara} & \emph{k'ar-ne}\tabularnewline
\end{tabular} &
\begin{tabular}[t]{@{}lll@{}}
& \textsc{sg} & \textsc{pl}   \tabularnewline \midrule
 `place' & \emph{musa} & \emph{mus-ne}\tabularnewline
 `cover' & \emph{q'ap'a} & \emph{q'ap'-ne}\tabularnewline
 `mouse' & \emph{waca} & \emph{wac-ne}\tabularnewline
 `voice' & \emph{t'ama} & \emph{t'am-ne}\tabularnewline
 `bird' & \emph{čiqʷaˤ} & \emph{čiˤqʷ-ne}\tabularnewline
 `hedgehog' & \emph{satkʷa} & \emph{satkʷ-ne}\tabularnewline
&& \tabularnewline
\end{tabular} \tabularnewline
   \bottomrule
\end{tabular}
\end{table}

\tabref{tab:3:7} illustrates the mechanism of the plural formation of
one-syllable stems attaching the suffix \emph{-ne} (Rule 2):

\begin{table}
% \begin{longtable}{@{}lll@{}}
  % Table 7. 
  \caption{Rule 2}\label{tab:3:7} %  \tabularnewline

  \begin{tabular}{@{}c@{\qquad}c@{}}
    \toprule
  \begin{tabular}{@{}lll@{}}  
& \textsc{sg} & \textsc{pl}\tabularnewline \midrule %  \endfirsthead
%   \caption{Rule 2 (continued)} \tabularnewline \toprule
% & \textsc{sg} & \textsc{pl}\tabularnewline \midrule \endhead
`load' & \emph{deχ} & \emph{deχ-a-ne}\tabularnewline
`herd' & \emph{ħanq'} & \emph{ħanq'-a-ne}\tabularnewline
`manure' & \emph{dekʷ} & \emph{dekʷ-a-ne}\tabularnewline
`wedge' & \emph{č'ut'} & \emph{č'ut'-a-ne}\tabularnewline
`fist' & \emph{χunk'} & \emph{χunk'-a-ne}\tabularnewline
 `liver' & \emph{k'ac'} & \emph{k'ac'-a-ne}\tabularnewline % <>
 `place' & \emph{merʔ} & \emph{merʔ-a-ne}\tabularnewline
 \end{tabular} &
 \begin{tabular}{@{}lll@{}}  
& \textsc{sg} & \textsc{pl}\tabularnewline \midrule %  \endfirsthead
`pupil (of the eye)' & \emph{nur} & \emph{nur-a-ne}\tabularnewline % <>
`lightning' & \emph{parχ} & \emph{parχ-a-ne}\tabularnewline
`shelter (of branches)' & \emph{paž} & \emph{paž-a-ne}\tabularnewline
`yoke' & \emph{duk'} & \emph{duk'-a-ne}\tabularnewline
`strut' & \emph{t'al} & \emph{t'al-a-ne}\tabularnewline
`month' & \emph{baz} & \emph{baz-a-ne}\tabularnewline
`drop', `point' & \emph{t'ank'} & \emph{t'ank'-a-ne}\tabularnewline
 \end{tabular} \tabularnewline
\bottomrule
\end{tabular}                                  
% \end{longtable}
\end{table}

 
\tabref{tab:3:8} illustrates Rule 3:


\begin{table}[H]
  % Table 8.
\advance\tabcolsep-1bp
  \caption{Rule 3}\label{tab:3:8}
\begin{tabular}{@{}c@{\hskip1.5em}c@{}}
\toprule
  \begin{tabular}{@{}lll@{}}
& \textsc{sg} & \textsc{pl}\tabularnewline \midrule 
`scythe' & \emph{č'inik'} & \emph{č'inik'-u-ne}\tabularnewline
`shock/stook' & \emph{bizaq'} & \emph{bizaq'-u-ne}\tabularnewline
`chain' & \emph{raχas} & \emph{raχas-u-ne}\tabularnewline
`kidney' & \emph{urcec} & \emph{urcec-u-ne}\tabularnewline
`ploughshare' & \emph{uʔab} & \emph{uʔab-u-ne}\tabularnewline
`glue' & \emph{luʔmes} & \emph{luʔmes-u-ne}\tabularnewline
`trousers' & \emph{waχčag} & \emph{waχčag-u-ne}\tabularnewline
`fork' & \emph{χinč'ult'} & \emph{χinč'ult'-u-ne}\tabularnewline
`metal tray' & \emph{sarʁas} & \emph{sarʁas-u-ne}\tabularnewline
  \end{tabular} &
  \begin{tabular}{@{}lll@{}}
& \textsc{sg} & \textsc{pl}\tabularnewline \midrule 
`needle' & \emph{bureba} & \emph{bureb-u-ne}\tabularnewline
`corpse' & \emph{žanaza} & \emph{žanaz-u-ne}\tabularnewline
`pound' & \emph{qilawka} & \emph{qilawk-u-ne}\tabularnewline
`alms' & \emph{sadaq'a} & \emph{sadaq'-u-ne}\tabularnewline
`swallow' & \emph{určuti} & \emph{určut-u-ne}\tabularnewline
`nose' & \emph{šumšut'i} & \emph{šumšut'-u-ne}\tabularnewline
`whirligig' & \emph{c'alači} & \emph{c'alač-u-ne}\tabularnewline
`jug' & \emph{burbut'i} & \emph{burbut'-u-ne}\tabularnewline
`button' & \emph{mičawi} & \emph{mičaw-u-ne}\tabularnewline
  \end{tabular} \tabularnewline
\bottomrule
\end{tabular}
\end{table}

Rule 3 has one exception: the plural stem of the word \emph{ʁamas} `box'
is formed by syncope of the last vowel of the root:

\begin{table}[h]
  % Table 9.
  \caption{Exception (Rule 1)}
\begin{tabular}{@{}lll@{}}
\toprule
& \textsc{sg} & \textsc{pl}\tabularnewline \midrule
`box' & \emph{ʁamas} & \emph{ʁams-ne}\tabularnewline
\bottomrule
\end{tabular}
\end{table}

The nouns given in \tabref{tab:3:10} undergo haplology:

\begin{table}[h]
  % Table 10.
  \caption{Haplology}\label{tab:3:10}

\begin{tabular}{@{}lll@{}}
\toprule
& \textsc{sg} & \textsc{pl}\tabularnewline \midrule
`omelet' & \emph{χajqane} & \emph{χajq-u-ne}\tabularnewline
`moustache' & \emph{sersit'ane} & \emph{sersit'-u-ne}\tabularnewline
`lizard' & \emph{šuršut'ani} & \emph{šuršut'-u-ne}\tabularnewline
`fat tail' & \emph{urʁaˤdiq'aˤni} & \emph{urʁaˤdiq'-uˤ-ne}\tabularnewline
`bellows' & \emph{pušduk'ani} & \emph{pušduk'-u-ne}\tabularnewline
\bottomrule
\end{tabular}
\end{table}

The haplology here applies to the contiguous VR sequences: when after a
derivation there are two VR sequences with the same R next to each other,\pagebreak[4]
the first one is dropped, e.g.\
\emph{urʁadiq\textbf{a}ˤ\textbf{n}-\textbf{u}-\textbf{n}e} →
\emph{urʁadiq-\textbf{u}ˤ-\textbf{n}e}. These words can also be analyzed
as attaching the suffix \emph{-e} after dropping the final vowel.
However, since the suffix \emph{-e} prefers one-syllable stems, my
analysis seems more feasible\footnote{\citet[36]{magometov1982} treats these
  cases as cases of apophony rather than haplology. He analyzes the forms
  \emph{χajqune} and \emph{sersit'une} as follows: ``There are cases,
  albeit rare, when a word ending with \emph{-e} in the
  plural differs [from the singular] only by a vowel change in the stem.
  This vowel change, therefore, acquires a morphological meaning'' (translation from Russian by the author).}.

Several words form the plural stem by changing the vowel in the first
syllable (which is also the penultimate) into \emph{/u/}:

\begin{table}[h]
  % Table 11.
  \caption{Vowel change in the root}
\begin{tabular}{@{}lll@{}}
\toprule
& \textsc{sg} & \textsc{pl}\tabularnewline \midrule
`stomach' & \emph{ʁaga} & \emph{ʁug-ne}\tabularnewline
`frog' & \emph{ʡaˤt'a} & \emph{ʡoˤt'-ne}\tabularnewline
\bottomrule
\end{tabular}
\end{table}

% 3.3.
\subsection{The plural suffix \emph{-tune}}
\label{the-plural-suffix--tune}

The words \emph{qašqar} `bald man', \emph{wakil} `lawyer', \emph{arab}
`Arab' and \emph{sabab} `reason' attach the plural suffix \emph{-tune}.
Diachronically, these words employed the suffix \emph{-t(e)}, as in some
other Dargwa dialects, e.g.\ Kubachi. Presumably, this plural marking was
then reinforced by \emph{-ne}, which required the change of the final
vowel to \mbox{\emph{-u}}. Together, these suffixes formed the structure
\emph{-tune}, which is synchronically monomorphemic (\tabref{tab:3:12}):

\begin{table}[h]
  % Table 12.
  \caption{The plural suffix \emph{-tune}}\label{tab:3:12}
  
\begin{tabular}{@{}lllll@{}}
\toprule
& {Mehweb \textsc{sg}} & {Mehweb \textsc{pl}} & {Kubachi \textsc{sg}} & {Kubachi \textsc{pl}}\tabularnewline \midrule
`bald' & \emph{qašqar} & \emph{qašqar-tune} & \emph{qˤaˤšqˤaˤr} &
\emph{qˤaˤšqˤaˤr-te}\tabularnewline
`lawyer' & \emph{wakil} & \emph{wakil-tune} & \emph{wakil} &
\emph{wakil-te}\tabularnewline
`Arab' & \emph{arab} & \emph{arab-tune} & \emph{warab} &
\emph{warab-te}\tabularnewline
`reason' & \emph{sabab} & \emph{sabab-tune} & \emph{sabab} &
\emph{sabab-te}\tabularnewline
\bottomrule
\end{tabular}
\end{table}

% \pagebreak

% 3.4.
\subsection{The plural suffix \emph{-be}}
\label{the-plural-suffix--be}

With the suffix \emph{-be}, the stem undergoes the following changes:

\begin{enumerate}[topsep=\medskipamount,itemsep=0pt,partopsep=0pt,parsep=0pt,label={\arabic*})]
\item % 1)
  If a stem ends in a vowel, the vowel is dropped.

\item % 2)
  After dropping the final vowel, originally two-syllable words with
[a] in the first syllable often add \emph{-u-} to form their plural
stems.
\end{enumerate}

\tabref{tab:3:13} illustrates Rule 1:

\begin{table}[h]
  % Table 13.
  \caption{Rule 1}\label{tab:3:13}
  \begin{tabular}{@{}lll@{}}
\toprule
& \textsc{sg} & \textsc{pl}\tabularnewline\midrule
`bear' & \emph{sinka} & \emph{sink-be}\tabularnewline
`crust' & \emph{wank'a} & \emph{wank'-be}\tabularnewline
`tooth' & \emph{cula} & \emph{cul-be}\tabularnewline
`mill' & \emph{šinq'a} & \emph{šinq'-be}\tabularnewline
\bottomrule
\end{tabular}
\end{table}


\tabref{tab:3:14} illustrates Rule 2:

\begin{table}[H]
  % Table 14.
  \caption{Rule 2}\label{tab:3:14}
\begin{tabular}{@{}c@{\qquad}c@{}}
\toprule
  \begin{tabular}{@{}lll@{}}
& \textsc{sg} & \textsc{pl}\tabularnewline \midrule 
`leg' & \emph{daga} & \emph{dag-u-be}\tabularnewline
`heel' & \emph{qaˤč'a} & \emph{qaˤč'-u-be}\tabularnewline
`bone' & \emph{liga} & \emph{lig-u-be}\tabularnewline
`sledge' & \emph{čana} & \emph{čan-u-be}\tabularnewline
  \end{tabular}&
  \begin{tabular}{@{}lll@{}}
& \textsc{sg} & \textsc{pl}\tabularnewline \midrule 
 `stone' & \emph{ʁarʁa} & \emph{ʁarʁ-u-be}\tabularnewline
 `cheek' & \emph{laˤži} & \emph{laˤž-u-be}\tabularnewline
 `spike' & \emph{canzi} & \emph{canz-u-be}\tabularnewline
 `cradle' & \emph{kʷahni} & \emph{kʷahn-u-be}\tabularnewline
  \end{tabular}\tabularnewline
 \bottomrule
  \end{tabular}
\end{table}

Note that \emph{liga} `bone' also forms the plural stem by attaching
\emph{-u-} even though the first syllable does not contain [a].

Several nouns form their plural stems by changing the root vowel to
[u]. All of these words either have [e] in this syllable or
contain a labialized or labial consonant:

\begin{table}[h]
  % Table 15.
  \caption{Vowel change in the root}

  \begin{tabular}{@{}c@{\qquad}c@{}}
\toprule
\begin{tabular}{@{}lll@{}}
& \textsc{sg} & \textsc{pl}\tabularnewline \midrule
`melted butter' & \emph{nerχ} & \emph{nurχ-be}\tabularnewline
`cricket' & \emph{c'erc'} & \emph{c'urc'-be}\tabularnewline
`tear' & \emph{nerʁ} & \emph{nurʁ-be}\tabularnewline
`eyebrow' & \emph{ned} & \emph{nud-be}\tabularnewline
`boar' & \emph{t'oˤrʜ} & \emph{t'uˤrʜ-be}\tabularnewline
\end{tabular} & 
\begin{tabular}{@{}lll@{}}
& \textsc{sg} & \textsc{pl}\tabularnewline \midrule
 `armful' & \emph{kʷec'} & \emph{kuc'-be}\tabularnewline
 `lip' & \emph{k'ʷet'} & \emph{k'ut'-be}\tabularnewline
 `peach' & \emph{q'ʷarč} & \emph{q'urč-be}\tabularnewline
 `cattle-shed' & \emph{derqʷ} & \emph{durq-be}\tabularnewline
& & \tabularnewline
\end{tabular}\tabularnewline
    \bottomrule
\end{tabular}
\end{table}

\pagebreak

An assimilation occurs in stems ending with [n]: /n+be/ →
[mbe]:

\begin{table}[h]
  % Table 16.
  \caption{/n+be/ → [mbe]}
  
\begin{tabular}{@{}lll@{}}
\toprule
& \textsc{sg} & \textsc{pl}\tabularnewline\midrule
`stall' & \emph{t'eni} & \emph{t'um-be}\tabularnewline
`cooker' & \emph{wana} & \emph{wum-be}\tabularnewline
\bottomrule
\end{tabular}
\end{table}

If a stem ends in a labialized consonant, this consonant is
delabialized:

\begin{table}[h]
  % Table 17.
  \caption{Delabialization}
  
\begin{tabular}{@{}lll@{}}
\toprule
& Sg & Pl\tabularnewline\midrule
`cattle-shed' & \emph{derqʷ} & \emph{durq-be}\tabularnewline
\bottomrule
\end{tabular}
\end{table}

% 3.5.
\subsection{The plural suffixes \emph{-nube} and
  \emph{-urbe}}
\label{the-plural-suffixes--nube-and--urbe}

The suffix \emph{-nube} forms the plural of five lexemes. The
suffix \emph{-urbe} forms the plural of four lexemes. These suffixes are
similar to \emph{-tune} in that they may be analyzed as \emph{-ne} and
\emph{-re} followed by \emph{-be}. The \emph{-u-} of the suffixes
\emph{-nube} and \emph{-urbe} may be considered as resulting from the
final vowel change seen in \sectref{the-plural-suffix--ne} above.
Synchronically, \emph{-nube} and \emph{-urbe} are
monomorphemic suffixes with a very limited lexical distribution (\tabref{tab:3:18}):

\begin{table}
  % Table 18.
  \caption{The plural suffixes \emph{-nube} and \emph{-urbe}}\label{tab:3:18}

\begin{tabular}{@{}lll@{}}
\toprule
& \textsc{sg} & \textsc{pl}\tabularnewline \midrule
`thief' & \emph{curku} & \emph{curk-nube}\tabularnewline
`small stone' & \emph{ħarħa} & \emph{ħarħ-nube}\tabularnewline
`belt' & \emph{irʔi} & \emph{irʔ-nube}\tabularnewline
`onion' & \emph{šerši} & \emph{šerš-nube}\tabularnewline
`burnt clay' & \emph{t'arħa} & \emph{t'arħ-nube}\tabularnewline
`door' & \emph{unza} & \emph{unz-urbe}\tabularnewline
`swamp' & \emph{šinʔa} & \emph{šinʔ-urbe}\tabularnewline
`grapes' & \emph{t'ut'i} & \emph{t'ut'-urbe}\tabularnewline
`wheat' & \emph{anč'e} & \emph{anč'-urbe}\tabularnewline
\bottomrule
\end{tabular}
\end{table}

% 3.6.
\subsection{The plural suffix \emph{-me}}
\label{the-plural-suffix--me}

With the suffix \emph{-me}, the following rules apply:
\begin{enumerate}[topsep=\medskipamount,itemsep=0pt,partopsep=0pt,parsep=0pt,label={\arabic*})]
\item % 1)
  One-syllable words with CV structure usually attach the suffix
\emph{-me}.

\item % 2)
  If a stem consisting of two or more syllables ends in a vowel, this
vowel is dropped.

\item % 3)
  Some nouns attach \emph{-u-} after dropping the last vowel.
\end{enumerate}


\tabref{tab:3:19} illustrates Rule 1:

\begin{table}[H]
  % Table 19.
  \caption{Rule 1}\label{tab:3:19}
  
\begin{tabular}{@{}lll@{}}
\toprule
& \textsc{sg} & \textsc{pl}\tabularnewline \midrule
`fire' & \emph{c'a} & \emph{c'a-me}\tabularnewline
`nit' & \emph{q'i} & \emph{q'i-me}\tabularnewline
`horn' & \emph{qi} & \emph{qi-me}\tabularnewline
`village' & \emph{ši} & \emph{ši-me}\tabularnewline
`oath' & \emph{qʷe} & \emph{qʷe-me}\tabularnewline
`blood' & \emph{ħi} & \emph{ħi-me}\tabularnewline
`name' & \emph{ʔu} & \emph{ʔu-me}\tabularnewline
\bottomrule
\end{tabular}
\end{table}

\tabref{tab:3:20} illustrates Rule 2:

\begin{table}[H]
  % Table 20.
  \caption{Rule 2}\label{tab:3:20}
\begin{tabular}{@{}lll@{}}
\toprule
& \textsc{sg} & \textsc{pl}\tabularnewline \midrule
`turnip' & \emph{q'aħa} & \emph{q'aħ-me}\tabularnewline
`(female) goat' & \emph{q'aˤca} & \emph{q'aˤc-me}\tabularnewline
`bolter' & \emph{ʔula} & \emph{ʔul-me}\tabularnewline
`(male) sheep' & \emph{kʷiha} & \emph{kʷih-me}\tabularnewline
`light' & \emph{šala} & \emph{šal-me}\tabularnewline
`cliff' & \emph{šuri} & \emph{sur-me}\tabularnewline
`scythe' & \emph{čuri} & \emph{čur-me}\tabularnewline
`bottom of a dress' & \emph{suri} & \emph{sur-me}\tabularnewline
\bottomrule
\end{tabular}
\end{table}

Some nouns form plural stems by attaching \emph{-u-} after dropping the
last vowel. All of them contain an [u] or a labial/labialized
consonant. One may notice that in most cases, after the final vowel drop
has been applied, [u] is inserted to\pagebreak[3] avoid a phonologically
illegitimate consonant cluster.
There is, however, no such consonant cluster in \emph{uq'lah-u-me} (cf.\
\emph{kʷih-me} `sheep, PL'). The Russian loanword \emph{bidra} `bucket' also
belongs to this group. \tabref{tab:3:21} below illustrates this process.

\begin{table}[h]
  % Table 21.
  \caption{Plural stem formation by attaching \emph{-u-}}\label{tab:3:21}
\begin{tabular}{@{}lll@{}}
\toprule
& \textsc{sg} & \textsc{pl}\tabularnewline\midrule
`spoon' & \emph{q'usla} & \emph{q'usl-u-me}\tabularnewline
`bullet' & \emph{gulla} & \emph{gull-u-me}\tabularnewline
`bucket' & \emph{bidra} & \emph{bidr-u-me}\tabularnewline
`window' & \emph{uq'laha} & \emph{uq'lah-u-me}\tabularnewline
`shroud' & \emph{bišri} & \emph{bišr-u-me}\tabularnewline
`thought' & \emph{pikri} & \emph{pikr-u-me}\tabularnewline
`jewel' & \emph{laˤwlu} & \emph{laˤwl-u-me}\tabularnewline
`mind' & \emph{waq'lu} & \emph{waq'l-u-me}\tabularnewline
\bottomrule
\end{tabular}
\end{table}

The words \emph{laˤwlu} and \emph{waq'lu} are also analyzed as dropping
their last vowel and then attaching \emph{-u-}:

\medskip
\emph{laˤwlu + me → laˤwl + me → laˤwl + -u- + -me → laˤwl-u-me}

\medskip
Under this analysis, the [u] in the plural is not the same as the
[u] in the singular.

% 3.7.
\subsection{The plural suffix \emph{-lume}}
\label{the-plural-suffix--lume}

The following words form the plural with the suffix \emph{-lume}, which
historically seems to be the plural suffix \emph{-le} with a change of
the final vowel before the plural suffix \emph{-me} (\tabref{tab:3:22}):

\begin{table}[b]
  % Table 22.
  \caption{The plural suffix \emph{-lume}}\label{tab:3:22}
\begin{tabular}{@{}lll@{}}
\toprule
& \textsc{sg} & \textsc{pl}\tabularnewline\midrule
`garden' & \emph{baχča} & \emph{baχč-lume}\tabularnewline
`corner' & \emph{murʔa} & \emph{murʔ-lume}\tabularnewline
`shadow' & \emph{daˤχc'i} & \emph{daˤχc'-lume}\tabularnewline
`ceiling' & \emph{burχa} & \emph{burχ-lume}\tabularnewline
\bottomrule
\end{tabular}
\end{table}

\pagebreak

% 3.8.
\subsection{The plural suffix \emph{-e}}
\label{the-plural-suffix--e}

The suffix \emph{-e} attaches to one-syllable stems. It can attach
directly to CVC(C) stems. In some cases, the rules for plural stem
formation derive one-syllable stems from more-than-one syllable stems and
are as follows:
\begin{enumerate}[topsep=\medskipamount,itemsep=0pt,partopsep=0pt,parsep=0pt,label={\arabic*})]
\item % 1.
  If a stem ends in a vowel, the vowel is dropped.

\item % 2.
  If a stem consists of more than one syllable, all the vowels except
the first undergo syncope.
\end{enumerate}

\begin{table}[h]
  % Table 23.
  \caption{The plural suffix \emph{-e}}
\begin{tabular}{@{}lll@{}}
\toprule
& \textsc{sg} & \textsc{pl}\tabularnewline\midrule
`root' & \emph{maq'ʷ} & \emph{maq'ʷ-e}\tabularnewline
`nut' & \emph{χihʷ} & \emph{χihʷ-e}\tabularnewline
`finger' & \emph{t'ul} & \emph{t'ul-e}\tabularnewline
`bread' & \emph{t'ult'} & \emph{t'ult'-e}\tabularnewline
`bull' & \emph{unc} & \emph{unc-e}\tabularnewline
`gut' & \emph{rud} & \emph{rud-e}\tabularnewline
`khinkal' & \emph{χinč'} & \emph{χinč'-e}\tabularnewline
`hand' & \emph{naˤʁ} & \emph{noˤʁ-e}\footnotemark
\tabularnewline
\bottomrule
\end{tabular}
\end{table}

\footnotetext{The word \emph{naˤʁ} `hand' appears to undergo the \emph{/a/} → \emph{/u/} vowel alternation described in \sectref{the-plural-suffix--ne}. Since this alternation does not affect the word \emph{maq'ʷ} `root' that has a similar phonetic structure, it is possible to hypothesize that the suffix \emph{-e} has originates from several different suffixes that merged in the \emph{-e} form due to phonetic changes.}

\tabref{tab:3:24} illustrates Rule 1:

\begin{table}[H]
  % Table 24.
  \caption{Rule 1}\label{tab:3:24}
\begin{tabular}{@{}lll@{}}
\toprule
& \textsc{sg} & \textsc{pl}\tabularnewline \midrule
`horse' & \emph{urči} & \emph{urč-e}\tabularnewline
`bee' & \emph{mirqi} & \emph{mirq-e}\tabularnewline
`nettle' & \emph{nizbi} & \emph{nizb-e}\tabularnewline
`ear' & \emph{lugi} & \emph{lug-e}\tabularnewline
`sparkle' & \emph{purχi} & \emph{purχ-e}\tabularnewline
\bottomrule
\end{tabular}
\vspace{-\jot}
\end{table}


\pagebreak[4]

\tabref{tab:3:25} illustrates the vowel syncope described in Rule 2:

\begin{table}[H]
  % Table 25.
  \caption{Rule 2}\label{tab:3:25}
\begin{tabular}{@{}lll@{}}
\toprule
& \textsc{sg} & \textsc{pl}\tabularnewline\midrule
`worm' & \emph{muleʁ} & \emph{mulʁ-e}\tabularnewline
`helminth' & \emph{šulek} & \emph{šulk-e}\tabularnewline
`bull-calf' & \emph{k'umeš} & \emph{k'umš-e}\tabularnewline
`toe' & \emph{gubul} & \emph{gubl-e}\tabularnewline
`plank' & \emph{ulq'uli} & \emph{ulq'l-e}\tabularnewline
`white (of an egg)' & \emph{šuhari} & \emph{šuhr-e}\tabularnewline
`egg' & \emph{ǯigari} & \emph{ǯigr-e}\tabularnewline
\bottomrule
\end{tabular}
\end{table}

% 3.9.
\subsection{The plural suffix \emph{-re}}
\label{the-plural-suffix--re}

This suffix has a limited lexical distribution. The rules for plural
stem formation are similar to the rules for other C\emph{e}
suffixes\footnote{I do not have a satisfactory explanation for this
  parallel. It is possible that \emph{-e}, which was originally present
  in all plural suffixes including \emph{-t}, as confirmed by other
  Dargwa lects, at some point became associated with the expression of
  plurality, and the consonants came to be interpreted as parts of
  the plural stem of the noun.} (see also \sectref{the-plural-suffix--be}):
\begin{enumerate}[topsep=\medskipamount,itemsep=0pt,partopsep=0pt,parsep=0pt,label={\arabic*})]
\item % 1)
  If a stem ends in a vowel, the vowel is dropped.

\item % 2)
  One-syllable roots tend to form their plural stems by changing the
root vowel to [u]. Since, for this suffix, I do not have any
examples of words consisting of more than one syllable after dropping
the last vowel, I cannot say whether they do or do not undergo this
vowel change.
\end{enumerate}

The suffix \emph{-re} prefers one-syllable words and two-syllable stems
ending with~[i].

\tabref{tab:3:26} illustrates Rule 1:

\begin{table}[h]
  % Table 26.
  \caption{Rule 1}\label{tab:3:26}
\begin{tabular}{@{}lll@{}}
\toprule
& \textsc{sg} & \textsc{pl}\tabularnewline\midrule
`leaf' & \emph{k'ap'i} & \emph{k'ap'-re}\tabularnewline
`cross-beam' & \emph{duk'i} & \emph{duk'-re}\tabularnewline
`mouth' & \emph{dubi} & \emph{dub-re}\tabularnewline
`nipple' & \emph{ut'i} & \emph{ut'-re}\tabularnewline
\bottomrule
\end{tabular}
\end{table}

\tabref{tab:3:27} illustrates Rule 2:\vspace{-\jot}


\begin{table}[H]
  % Table 27.
  \caption{Rule 2}\label{tab:3:27}
\begin{tabular}{@{}lll@{}}
\toprule
& \textsc{sg} & \textsc{pl}\tabularnewline \midrule
`fly' & \emph{t'ant'} & \emph{t'unt'-re}\tabularnewline
`fish' & \emph{k'as} & \emph{k'us-re}\tabularnewline
`pocket' & \emph{č'ep} & \emph{č'up-re}\tabularnewline
`paw' & \emph{k'ʷac} & \emph{k'uc-re}\tabularnewline
\bottomrule
\end{tabular}
\end{table}

However, there are exceptions to Rule 2. Two roots contain [a] but
do not undergo vowel change (\tabref{tab:3:28}):

\begin{table}[H]
  % Table 28.
  \caption{Exceptions (Rule 2)}\label{tab:3:28}
\begin{tabular}{@{}lll@{}}
\toprule
& \textsc{sg} & \textsc{pl}\tabularnewline \midrule
`neck' & \emph{qaˤb} & \emph{qaˤb-re}\tabularnewline
`manure' & \emph{qʷa} & \emph{qʷa-re}\tabularnewline
\bottomrule
\end{tabular}
\end{table}

The [r] in the suffix \emph{-re} can, but need not, assimilate to the [l] of the
stem (\tabref{tab:3:29}):\vspace{-2\jot}

\begin{table}[h]
  % Table 29.
  \caption{Assimilation /r/ → /l/} \label{tab:3:29}
\begin{tabular}{@{}lll@{}}
\toprule
& \textsc{sg} & \textsc{pl}\tabularnewline \midrule
`house' & \emph{qali} & \emph{qul-le}/\emph{qul-re}\tabularnewline
\bottomrule
\end{tabular}
\vspace{-\jot}
\end{table}

% 3.10.
\subsection{The plural suffix \emph{-le}}
\label{the-plural-suffix--le}

The plural suffix \emph{-le} only occurs with four nouns. If the stem ends
in a vowel, the vowel is dropped. The vowel of the stem changes to~/u/
(\tabref{tab:3:30}):

\begin{table}[H]
  % Table 30.
  \caption{The plural suffix \emph{-le}}\label{tab:3:30}
\begin{tabular}{@{}lll@{}}
\toprule
& \textsc{sg} & \textsc{pl}\tabularnewline\midrule
`body' & \emph{čarχ} & \emph{čurχ-le}\tabularnewline
`handle' & \emph{arʔ} & \emph{urʔ-le}\tabularnewline
`worm' & \emph{serhʷ} & \emph{surhʷ-le}\tabularnewline
`rope' & \emph{ʁʷaˤrʁoˤ} & \emph{ʁʷoˤrʁ-le}\tabularnewline
\bottomrule
\end{tabular}
\end{table}

% 3.11.
\subsection{The plural suffixes \emph{-he} and \emph{-še}}
\label{the-plural-suffixes--he-and--ux161e}

The suffix \emph{-he} occurs with two nouns. Both have irregular plural
stems, so the plural formation may be considered to be weak suppletion (\tabref{tab:3:31}):

\begin{table}[H]
  % Table 31.
  \caption{The plural suffix \emph{-he}}\label{tab:3:31}
\begin{tabular}{@{}lll@{}}
\toprule
& \textsc{sg} & \textsc{pl}\tabularnewline \midrule
`woman' & \emph{xunul} & \emph{xu-he}\tabularnewline
`dog' & \emph{χʷe} & \emph{χur-he}\tabularnewline
\bottomrule
\end{tabular}
\end{table}

% \pagebreak[4]

The plural suffix \emph{-še} occurs with one noun, \emph{qu} `field' (\tabref{tab:3:32}):

\begin{table}[H]
  % Table 32.
  \caption{The plural suffix \emph{-še}}\label{tab:3:32}
\begin{tabular}{@{}lll@{}}
\toprule
& \textsc{sg} & \textsc{pl}\tabularnewline \midrule
`field' & \emph{qu} & \emph{qu-še}\tabularnewline
\bottomrule
\end{tabular}
\end{table}

% 3.12.
\subsection{The associative plural suffix \emph{-qale}}
\label{the-associative-plural-suffix--qale}

The plural suffix \emph{-qale} most probably results from
grammaticalization of the noun \emph{qali} `house'. In the case of
Mehweb, this suffix covers the so-called associative plural meaning
`X~and his or her family' (in spontaneous texts also `X and those with
him/her', `X and his/her group'). For Tanti Dargwa, \citet{lander2008}
observes that the suffix \emph{-qale} has developed a regular plural
meaning. This evolution has not been reported for standard Dargwa. In
Mehweb Dargwa, regular plural uses of \emph{-qale} are attested on nouns
for `mother' and `father'; for `grandmother' and probably `grandfather',
both regular and associative plural readings are attested. \tabref{tab:3:33}
illustrates the use of this suffix:

\begin{table}[b]
  \caption{The associative plural suffix \emph{-qale}}\label{tab:3:33}
  \begin{tabularx}{\textwidth}{@{}lp{5em}lX@{}}
\toprule
\textsc{sg} &  & \textsc{pl} &  
\tabularnewline \midrule
\emph{abaj} & `mom' & \emph{abaj-qale} & `moms'\tabularnewline
\emph{adaj} & `dad' & \emph{adaj-qale} & `dads'\tabularnewline
\emph{baba} & `grandma' & \emph{baba-qale} & `grandmas' \emph{or} 
`grandma and her family'\tabularnewline
\emph{Abakar} & `Abakar' (man's \rlap{name)} & \emph{Abakar-qale} & `Abakar and
his family / his group'\tabularnewline
\bottomrule
\end{tabularx}
\end{table}

\is{plural|)}
\pagebreak

% 4.
\section{Oblique stem}\label{oblique-stem}

The genitive case suffix attaches directly to the nominative stem (in
all nouns but not in all pronouns – cf.\ \emph{di-la} I.\textsc{obl}-\textsc{gen}
`my'). All other cases (including ergative) require an oblique stem. In
the plural, all case suffixes attach directly to the plural marker.

The oblique stem marker has three allomorphs: \emph{-li}, \emph{-j}, and
\emph{-i}. The marker \emph{-li} is the default way to form an oblique
stem and is applicable to almost any stem.

The marker \emph{-i-} may be considered prothetic (to resolve consonant
clusters) and is generally not separated or glossed in this book. The
use of the segmental marker \emph{-li-} is a lexical property. With some
nouns, the two strategies are in competition:

\ea
\gll {muħammad-li-ni muħammadi-šu}\\
Muhammad-\textsc{obl}-\textsc{erg} Muhammad-\textsc{ad(lat)}\\
\z

The oblique stem marker \emph{-li-} may (but does not have to) change to \emph{-j-}. 
\tabref{tab:3:34} shows contexts that license the change. The first column shows the vowel
preceding the last consonant. The second column shows the consonant and
the vowel that can follow it:

\begin{table}[h]
  % Table 34.
  \caption{Possible stem endings for the \emph{-li → -j} change} \label{tab:3:34}
\begin{tabular}{@{}cl@{}}
\toprule
{Second last syllable} & {Last syllable}\tabularnewline \midrule
a & l/li/la/n/ni\tabularnewline
i & l/li/la/n/ni\tabularnewline
oˤ & l/li/la\tabularnewline
u & l/n\tabularnewline
\bottomrule
\end{tabular}
\end{table}

Example (\ref{ex:3:2}) illustrates the process (see more in \citealt{moroz2019}):

\ea \label{ex:3:2}
\gll {rasul rasuj-ni}\\
Rasul Rasul.\textsc{obl}-\textsc{erg}\\
\z

% 5.
\section{Nominal inflection system}
\label{nominal-inflection-system}

% Nominal inflection system

The nominal inflection of Mehweb Dargwa consists of two parts
(sub-\hskip0pt paradigms): \emph{grammatical cases} and \emph{locative forms}. The
two types of inflectional forms differ in their morphology: grammatical
case forms contain one inflectional morpheme (\tabref{tab:3:35}); locative forms
contain two inflectional morphemes. The first morpheme of a locative
form designates the \emph{localization}: the spatial area defined\pagebreak[3] with
respect to a landmark (rows in \tabref{tab:3:36} below). The second 
designates the \emph{orientation} (columns in \tabref{tab:3:36} below): the
trajectory of the object with respect to the area designated by the
localization.



The core function of locative forms is to describe spatial relations 
between a figure and a ground \citep{rubin2001}.
Grammatical cases are primarily used to express grammatical relations
and abstract semantic roles. However, across East Caucasian, this is
only a typical division of labour, and both types of inflection can be
used in both functions \citep{kibrik2003}. In Mehweb, grammatical cases do
not have any spatial uses (except for the fact that the genitive suffix is
identical to the elative suffix) but spatial cases can have (nearly)
abstract functions.



In Mehweb, there are five localization morphemes and five orientation
morphemes. Each localization can take each of the orientations, forming
a system of 25 locative forms. The subsections below are named according to the
grammatical case labels and localization markers. One localization
morpheme can designate several distinct spatial areas. I thus use the
labels written in small-caps (e.g.\ \textsc{inter}) as a semantic label,
not as a gloss of a morphological category (as in the rest of the papers
in this collection).

I do not discuss the semantics of the orientation markers in separate
subsections. Their spatial functions are introduced in \tabref{tab:3:36} and are
independent from the semantics of the localization they combine with. In
their non-spatial uses, most locative forms cannot be described
compositionally by referring separately to the semantics of the
localization and orientation markers. I thus discuss these uses among
the functions of the individual localization markers in the relevant
subsections.



\begin{table}[b]
\vspace{-2\jot}
  % \renewcommand\baselinestretch{.95}
  % \renewcommand\arraystretch{.95}
  % Table 35.
\caption{Mehweb functional sub-paradigm}\label{tab:3:35}
\begin{tabular}{@{}lll@{}}
\toprule
{Case} & \textsc{sg} & \textsc{pl}\tabularnewline \midrule
{Nominative} & ø & (Plural form)\tabularnewline
{Ergative} & -\textsc{obl}-ø/\emph{ʔini}/\emph{ini}/\emph{ijni}/\emph{ni} &
-\textsc{pl}\emph{-ʔini}/\emph{ini}/\emph{ijni}/\emph{ni}\tabularnewline
{Genitive} & \emph{-la}/\emph{wa}/\emph{jja} &
-\textsc{pl}\emph{-la}\tabularnewline
{Dative} & -\textsc{obl}\emph{-s} & -\textsc{pl}\emph{-s}\tabularnewline
{Comitative} & -\textsc{obl}\emph{-ču} &
-\textsc{pl}\emph{-ču}\tabularnewline
{Causal} & -\textsc{obl}\emph{-čeble} &
-\textsc{pl}\emph{-čeble}\tabularnewline
{Substitutive} & -\textsc{obl}\emph{-čemadal} &
-\textsc{pl}\emph{-čemadal}\tabularnewline
{Replicative} & -\textsc{obl}\emph{-sum} &
-\textsc{pl}\emph{-sum}\tabularnewline
\bottomrule
\end{tabular}
\end{table}

The structure of the case system is shown in the two tables below.
\tabref{tab:3:35} shows grammatical cases. \tabref{tab:3:36} shows
locative forms, together with
their core meanings. The abbreviations for the morphemes in the
orientation slot are as follows: \textsc{lat} – lative, \textsc{ess}
– essive, \textsc{el} – elative, \textsc{trans} – translative,
\textsc{dir} – directive, \textsc{cl} – gender agreement marker.



\begin{table}[t]
  \renewcommand\baselinestretch{.95}  
  \renewcommand\arraystretch{.95}
% Table 36.
\advance\tabcolsep-2pt
  \caption{Mehweb locative sub-paradigm}\label{tab:3:36}
  \small
\begin{tabular}{@{}p{.15\textwidth}<{\raggedright}p{.12\textwidth}<{\raggedright}p{.1\textwidth}<{\raggedright}p{.2\textwidth}<{\raggedright}p{.15\textwidth}<{\raggedright}p{.17\textwidth}<{\raggedright}@{}}
\toprule
  Orientation  & \multicolumn{5}{c@{}}{Localization} \tabularnewline \cmidrule{2-6}
& \textsc{lat}\newline
\footnotesize `to the area denoted by the local- ization' & 
\textsc{ess}\newline
\footnotesize `no movement' & 
\textsc{el}\newline
\footnotesize `away from the area denoted by the localization' & 
\textsc{trans}\newline
\footnotesize `through the area denoted by the localization' & 
\textsc{dir}\newline
\footnotesize `in the direction of the area denoted by the localization'\tabularnewline \midrule
\textsc{super} {\footnotesize `on'},
\textsc{cont}\footnotemark{}
 & 
\emph{-če} & 
\emph{-če-}\textsc{cl} & 
\emph{-če-la}\newline
\emph{-če-}\textsc{cl}\emph{-ad}((\emph{-al})\emph{-a}) & 
\emph{-če-di} & 
\emph{-če-baˤʜ}\tabularnewline \midrule
\textsc{in}\newline
{\footnotesize `in a \rlap{container'}} & 
\emph{-ħe} / ø & 
\emph{-ħe-}\textsc{cl} / \newline
ø-\textsc{cl} & 
\emph{-ħe-la}\newline
\emph{-ħe-}\textsc{cl}\emph{-ad}((\emph{-al})\emph{-a})\newline
ø\emph{-la}\newline
ø-\textsc{cl}\emph{-ad}((\emph{-al})\emph{-a}) & 
\emph{-ħe-di} / ø\emph{-di} & 
\emph{-ħe-baˤʜ}\newline
ø\emph{-baˤʜ}\tabularnewline \midrule
\textsc{inter}\newline
{\footnotesize `in a subs\rlap{tance',}}\newline
\textsc{cont} & 
\emph{-ze} & 
\emph{-ze-}\textsc{cl} & 
\emph{-ze-la}\newline
\emph{-ze-}\textsc{cl}\emph{-ad}((\emph{-al})\emph{-a}) & 
\emph{-ze-di} & 
\emph{-ze-baˤʜ}\tabularnewline \midrule
\textsc{ad} {\footnotesize `near'} & 
\emph{-šu} & 
\emph{-šu-}\textsc{cl} & 
\emph{-šu-la}\newline
\emph{-šu-}\textsc{cl}\emph{-ad}((\emph{-al})\emph{-a}) & 
\emph{-šu-di} & 
\emph{-šu-baˤʜ}\tabularnewline\midrule
\textsc{apud}\newline
{\footnotesize `in the func\rlap{tional} area of a \rlap{landmark'}} & 
\emph{-ʡeˤ} & 
\emph{-ʡeˤ-}\textsc{cl} & 
\emph{-ʡeˤ}\emph{-la}\newline
\emph{-ʡeˤ-}\textsc{cl}\emph{-ad}((\emph{-al})\emph{-a}) & 
\emph{-ʡeˤ-di} & 
\emph{-ʡeˤ-baˤʜ}\tabularnewline
\bottomrule
\end{tabular}
% \vspace{-\baselineskip}
\end{table}


\footnotetext{\textsc{cont} is the functional label of a spatial
  configuration in which the object is located on the surface of a
  landmark and stays there because of the nature of the contact between the
  object and the landmark, or because it is a part thereof. Typical
  \textsc{cont} contexts are: `(a picture) on the wall', `(a ring) on a
  finger', `(wings) on the back', `(a birthmark) on the face'. Many
  East Caucasian languages have a separate localization marker for
  the \textsc{cont} configuration. In Mehweb, this configuration is
  divided between \emph{-če-} (labelled \textsc{super}, discussed in
  \sectref{the-locative-marker-ce}) and \emph{-ze-} (labelled \textsc{inter}, discussed in
  \sectref{the-locative-morpheme-ze}).}



Example (\ref{ex:3:3}) illustrates how the locative markers function:

\ea \label{ex:3:3}
\glll ʁarʁa ʁarʁa-li-če ʁarʁa-li-če-w ʁarʁa-li-ze-b\\
stone(\textsc{nom}) stone-\textsc{obl}-\textsc{super}(\textsc{lat}) stone-\textsc{obl}-\textsc{super}-\textsc{m}(\textsc{ess}) stone-\textsc{obl}-\textsc{inter}-\textsc{n}(\textsc{ess})\\
`(a)~stone' `onto~the~stone' `(he~is)~on~the~stone' `(it~is)~in~the~stone'\\

\z

The lative (\textsc{lat}) is expressed by the absence of an orientation
marker. The essive (\textsc{ess}) is expressed by the presence of the
gender agreement slot (shown as -\textsc{cl} in the table). The agreement
is controlled by the NP designating the trajector. Since the two markers
do not have their own dedicated exponency, their glosses are bracketed.



% 5.1. 
\subsection{Nominative}\label{nominative}

The nominative case marks the S of an intransitive verb and the P of a
transitive verb:

\ea
\gll {ʡali} w-ak'-ib.\\
Ali(\textsc{nom}) \textsc{m}-come:\textsc{pfv}-\textsc{aor}\\
\glt `Ali came'

\ex
\gll {adaj-ni mašinka-li-ni {muc'ur} b-erč-ur.}\\
father-\textsc{erg} hair.cutter-\textsc{obl}-\textsc{erg} beard(\textsc{nom}) \textsc{n}-cut.hair:\textsc{pfv}-\textsc{aor}\\
\glt `The father cut his beard with clippers.'
\z

The nominative case is also used when addressing someone:

\ea
\gll {baba} nab inc'ul uk-es ħa-d-ig-an.\\
granny I.\textsc{dat} more \textsc{m}.eat:\textsc{pfv}-\textsc{inf} \textsc{neg}-\textsc{npl}-want:\textsc{ipfv}-\textsc{hab}\\
\glt `Granny, I don't want to eat any more.'
\z

The nominative is also used in constructions like (6):

\ea
\gll χʷe-li-če-la {ažda} b-uh-ub.\\
dog-\textsc{obl}-\textsc{super}-\textsc{el} crocodile \textsc{n}-become:\textsc{pfv}-\textsc{aor}\\
\glt `The dog has become a crocodile.'
\z

% 5.2.
\subsection{Ergative}\label{ergative}

The ergative case marks the A of a transitive verb and the instrument:

\ea
\gll {adaj-ni} {mašinka-li-ni} muc'ur b-erč-ur.\\
father-\textsc{erg} hair.cutter-\textsc{obl}-\textsc{erg} beard(\textsc{nom}) \textsc{n}-cut.hair:\textsc{pfv}-\textsc{aor}\\
\glt `The father cut his beard with clippers.'
\z

The ergative case also marks periods of time. The semantics of such constructions can be formulated as `X did something for two hours', i.e. the result was not necessarily achieved:

\ea
\gll {k'ʷi-jal {saʡaˤt-li-ni} rasul ħule w-ilz-uwe le-w-re   ši-la surt-me-če.}\\
two-\textsc{card} hour-\textsc{obl}-\textsc{erg} Rasul(\textsc{nom}) look \textsc{m}-\textsc{lv}:\textsc{ipfv}-\textsc{cvb.ipfv} be-\textsc{m}-\textsc{pst} village-\textsc{gen} picture-\textsc{pl}-\textsc{super}(\textsc{lat})\\
\glt `Rasul has been looking at the photos of (his) village for two hours.'
\z

% 5.3.
\subsection{Genitive}
\label{genitive}

The genitive case marker is \emph{-la}. It can undergo the following
processes:
\begin{enumerate}[topsep=\medskipamount,itemsep=0pt,partopsep=0pt,parsep=0pt,label={\arabic*})]
\item % 1)
  when attached to words ending in [ul], the marker can change into
\emph{-wa}: e.g.\ \emph{rasul} `Rasul' – \emph{rasu-wa}
`Rasul-\textsc{gen}';

\item % 2)
  when attached to words ending in [Vl], the marker can change into
\emph{-jja}: \emph{rasul} `Rasul' – \emph{rasu-jja}
`Rasul-\textsc{gen}'. This is the only context in which [jj] occurs in
Mehweb.

\item % 3)
  when attached to words ending in [ala]\emph{,} the suffix
\emph{-la} can undergo haplology: the genitive form of \emph{č'imič'ala}
`eyelash' can be either \emph{č'imič'\textbf{ala}-\textbf{l}a} or
\emph{č'imič'\textbf{a}-\textbf{l}a.}
\end{enumerate}

{The genitive of place names is formed with \emph{-la} or \emph{-ja}
(probably derived from \emph{-n-la}; see below), while their \emph{-la}
form serves as the elative. Note that place names in Mehweb are a separate
part of speech possessing morphological and syntactic properties of both
nouns and locative adverbs. They lack an oblique stem and have an irregular
genitive form. They attach orientation markers directly, like spatial
adverbs. Their quotation form is also the essive form. Hence, the
\emph{-la} marker in \tabref{tab:3:37} is not only a genitive marker but also an elative
marker:}

\begin{table}[h]
  % Table 37.
  \caption{The Genitive of Place Names}\label{tab:3:37}
\begin{tabular}{@{}llll@{}}
\toprule
{Placename} &  & {Genitive} & {Elative}\tabularnewline \midrule
\emph{meħʷe} & `(in) Mehweb' & \emph{meħʷe-la, meħʷ-aja} &
\emph{meħʷe-la}\tabularnewline
\emph{surʁatli} & `(in) Sogratl' & \emph{surʁatli-la, surʁatl-aja} &
\emph{surʁatli-la}\tabularnewline
\emph{ʜaˤnnuqara} & `(in) Keger' & \emph{ʜaˤnnuqar-aja} &
\emph{ʜaˤnnuqara-la}\tabularnewline
\emph{žixatli} & `(in) Rugudzha' & \emph{žixatl-aja} &
\emph{žixatli-la}\tabularnewline
\bottomrule
\end{tabular}
\end{table}

The main function of the genitive case is to mark a noun that is
dependent on another noun (possessive construction):

\ea
\gll {rasuj-ni ar-d-uk-ib {muħammad-la} kʷihme.}\\
Rasul.\textsc{obl}-\textsc{erg} away-\textsc{npl}-lead:\textsc{pfv}-\textsc{aor} Muhammad-\textsc{gen} sheep.\textsc{pl}\\
\glt `Rasul took away Muhammad's sheep.'
\z

In possessive predication, the possessor genitive is ``free'' in that it
does not form a single constituent with the possessum.

\ea
\gll {{nuša-la} le-b ʁarʁ-u-be-la qali.}\\
we-\textsc{gen} be-\textsc{n} stone-\textsc{pl.stem}-\textsc{pl}-\textsc{gen} house\\
\glt `We have a stone house.'
\z

In the predicative possessive construction, Mehweb distinguishes two types
of possessors: locative possessor and genitive possessor. Locative
possession is only possible in predicative constructions, while genitive
possession can be either adnominal or predicative (free genitive). The
semantic difference between the two constructions is that the locative
possessor has an object with/on her, but this object does not
necessarily belong to her. The genitive possessor possesses an object,
i.e.\ it belongs to her:

\ea
\gll {{muħammad-la} kʷihme.}\\
Muhammad-\textsc{gen} sheep.\textsc{pl}\\
\glt `Muhammad's sheep (PL).'

\ex
\gll {{musa-la} le-b qali.}\\
Musa-\textsc{gen} be-\textsc{n} house\\
\glt `Musa has a house.'

\ex
\gll {{rasuj-ze-b} di-la dis le-b.}\\
Rasul.\textsc{obl}-\textsc{inter}-\textsc{n}(\textsc{ess}) I.\textsc{obl}-\textsc{gen} knife be-\textsc{n}\\
\glt `Rasul has got my knife', `My knife is with Rasul'.

\z

The difference does not apply to adnominal possessive constructions. It
is  not possible to use the localization  marker \emph{-ze} in an adnominal
possessive construction:

\ea
\gll {*rasuj-ze-b dis.}\\
Rasul.\textsc{obl}-\textsc{inter}-\textsc{n}(\textsc{ess}) knife\\
\glt `(someone else's) knife that Rasul has got.'
\z

% 5.4.
\subsection{Dative}\label{dative}

The dative case marker is \emph{-s}. It attaches to the oblique stem.
Its basic function is to mark the recipient in the `give' construction:

\ea
\gll  abaj-ni gi-b sadaq'ači-li-s t'ult'.\\
mother-\textsc{erg} give:\textsc{pfv}-\textsc{aor} pauper-\textsc{obl}-\textsc{dat} bread\\
\glt `Mother gave bread to a pauper.'
\z

The dative also marks the benefactive and several other related roles:

\ea
\gll {har duže rasuj-ni dursi-li-s χabar-t luč'-ib.}\\
every night Rasul.\textsc{obl}-\textsc{erg} girl-\textsc{obl}-\textsc{dat} story-\textsc{pl} read:\textsc{ipfv}-\textsc{ipft}\\
\glt `Every night Rasul read stories to his daughter.'

\ex \label{ex:3:18}
\gll {nuša-jni qali b-aq'-ib-i rasuj-s.}\\
we-\textsc{erg} house \textsc{n}-do:\textsc{pfv}-\textsc{aor}-\textsc{atr} Rasul.\textsc{obl}-\textsc{dat}\\
\glt `We built a house for Rasul.'
\z

The two types of predicative possession described in \sectref{ergative} are
paralleled by different strategies for encoding the recipient, as shown
in (\ref{ex:3:18}). The two types of transmission are encoded by the dative vs.
inter-lative form. If the rights of possession are transmitted together
with the object, the recipient is encoded with the dative case. If they
are not transmitted, as in (\ref{ex:3:19}), the recipient is marked with
\emph{-ze}:

\ea \label{ex:3:19}
\gll {rasuj-ni gi-b muħammadi-ze dis.}\\
Rasul.\textsc{obl}-\textsc{erg} give:\textsc{pfv}-\textsc{aor} Muhammad-\textsc{inter}(\textsc{lat}) knife\\
\glt `Rasul lent a knife to Muhammad.'
\z

The dative is also used for some experiencers. Experiential verbs have
one of the two case frames: [experiencer = \textsc{inter(lat)}, stimulus =
\textsc{nom}] and [experiencer = \textsc{dat}, stimulus = \textsc{nom}]. A dative
experiencer is only possible with the verb
\textsc{cl}\emph{-iges} `love/want' and complex predicates:

\ea
\gll {ħu nab eba uh-ub.}\\
you.sg I.\textsc{dat} boring \textsc{m}.become:\textsc{pfv}-\textsc{aor}\\
\glt `You bored me.'

\ex
\gll {jusupi-s d-ig-uwe le-r pat'imat.}\\
Jusup-\textsc{dat} \textsc{f1}-want:\textsc{ipfv}-\textsc{cvb.ipfv} be-\textsc{f} Patimat\\
\glt `Jusup loves Patimat.'
\z

% 5.5.
\subsection{Comitative}\label{comitative}

A co-participant is expressed by the comitative:

\ea
\gll {rasul urʁes w-ik-ib muħammadi-ču.}\\
Rasul fight:\textsc{ipfv}-\textsc{inf} \textsc{m}-\textsc{lv:pfv}-\textsc{aor}  Muhammad-\textsc{comit}\\
\glt `Rasul fought with Muhammad.'
\z

This case is also used for instruments, including consumables:

\ea
\gll {rasuj-ni ulq'uli rasdisi-ču b-elk-un.}\\
Rasul.\textsc{obl}-\textsc{erg} plank saw-\textsc{comit} \textsc{n}-cut:\textsc{pfv}-\textsc{aor}\\
\glt `Rasul sawed the plank with a saw.'

\ex
\gll {rasuj-ni ħi šin-ču d-urʔun d-aq'-ib.}\\
Rasul.\textsc{obl}-\textsc{erg} blood water-\textsc{comit} \textsc{npl}-clean \textsc{n}-do:\textsc{pfv}-\textsc{aor}\\
\glt `Rasul washed the blood off with water.'
\z

% 5.6.
\subsection{Causal}\label{causal}

According to \citet{magometov1982}, there is a case that marks the cause of a
situation. My consultants did not confirm Magometov's examples and
rejected the \emph{-čeble}/\mbox{\emph{-čible}} forms that I constructed. I assume
that the case no longer exists in Mehweb. Examples (\ref{ex:3:25}) and (\ref{ex:3:26}) are
cited from \citet[49]{magometov1982}:

\ea \label{ex:3:25}
\gll {\textsuperscript{?}se-li-čible ħu tusnaq' w-aq'-ib-i?}\\
what-\textsc{obl}-\textsc{causal} you.sg arrest \textsc{m}-do:\textsc{pfv}-\textsc{aor}-\textsc{atr}\\
\glt `Why did you get arrested?'

\ex \label{ex:3:26}
\gll {\textsuperscript{?}di-la χuligan-deši-čible nu tusnaq' w-aq'-ib.}\\
I.\textsc{obl}-\textsc{gen} hooligan-\textsc{nmlz}-\textsc{causal} I arrest \textsc{m}-do:\textsc{pfv}-\textsc{aor}\\
\glt `I got arrested because of my hooliganism.'

\z

% 5.7.
\subsection{Substitutive}\label{substitutive}

The morpheme \emph{-čemadal} has substitutive semantics, i.e.\ it
indicates that the actor performs an action instead of someone who was
supposed to perform it, the latter being coded by this case form:

\ea
\gll {nu adaj-čemadal tukaj-ħe w-aˤq'-un-na}\\
I father-\textsc{subst} shop.\textsc{obl}-\textsc{in}(\textsc{lat}) \textsc{m}-go:\textsc{pfv}-\textsc{aor}-\textsc{ego}\\
\glt `I went to the shop instead of father'
\z

Diachronically, this form can be analyzed as \emph{-če-m-ad-al}, in
which \emph{-če-} marks \textsc{super} localization, \emph{-m-} is an
unknown morpheme that occupies the localization slot and \emph{-adal}
is the elative marker (cf.\ \tabref{tab:3:36} above).

% 5.8.
\subsection{Replicative}\label{replicative}

The last non-spatial case suffix is \emph{-sum}. It conveys the
semantics of performing an action in the way similar to how someone or
something else performs it, or in the way it is usually done. The form
attaches to an irregular oblique stem:

\ea
\gll dilaj-sum b-aq'-a\\
I.\textsc{obl}-\textsc{repl} \textsc{n}-do:\textsc{pfv}-\textsc{imp}.\textsc{tr}\\
\glt `Do as I do'
\z

The following sections deal with spatial forms.

% 5.9.
\subsection{The locative marker \emph{-če-}}\label{the-locative-marker-ce}

The basic semantics of the locative marker \emph{-če-} is
\textsc{super}, i.e.\ by default this marker is used in contexts like
the following:

\ea
\gll {ustuj-če-b ʁadara le-b.}\\
table.\textsc{obl}-\textsc{super}-\textsc{n}(\textsc{ess}) plate be-\textsc{n}\\
\glt `A plate is on the table.'
\z

The locative marker \emph{-če-} is also used to mark the \textsc{cont}
configuration. It shares this function with the locative marker
\emph{-ze-}, whose basic semantics is \textsc{inter} (\sectref{the-locative-morpheme-ze}). The instances involving
\textsc{cont} semantics seem to be distributed over the two markers,
but the rules are difficult to formulate. Examples (\ref{ex:3:30}) and (\ref{ex:3:31}) show
that the two locative markers are not in free distribution in spatial
contexts:

\ea \label{ex:3:30}
\gll {surat aqi-le le-b baˤʜi-ze-b \textup/ *baˤʜi-če-b.}\\
picture up-\textsc{advz} be-\textsc{n} wall-\textsc{inter}-\textsc{n}(\textsc{ess}) / *wall-\textsc{super}-\textsc{n}(\textsc{ess})\\
\glt `A picture is hanging on the wall.'

\ex \label{ex:3:31}
\gll {iχija b-arš-ib-i t'uleka le-b t'uj-če-b \textup/ *t'uj-ze-b.}\\
this.\textsc{gen} \textsc{n}-become.beautiful:\textsc{pfv}-\textsc{aor}-\textsc{atr} ring be-\textsc{n} finger.\textsc{obl}-\textsc{super}-\textsc{n}(\textsc{ess}) / finger.\textsc{obl}-\textsc{inter}-\textsc{n}(\textsc{ess})\\
\glt 
`She has a beautiful ring on her finger.'
\z

The locative marker \emph{-če-} can be used in `support' contexts like
\emph{put against} (\emph{a~tree} etc.):

\ea \label{ex:3:32}
\gll {ʡali-ni mažar baˤʜi-če b-ix-ib.}\\
Ali-\textsc{erg} rifle wall-\textsc{super}(\textsc{lat}) \textsc{n}-put:\textsc{pfv}-\textsc{aor}\\
\glt `Ali put the rifle against the wall.'

\ex
\gll nu baˤʜi-če-la ʡaˤq ʡaˤr-aˤq'-un-na.\\
I wall-\textsc{super}-\textsc{el} far away-\textsc{m}.go:\textsc{pfv}-\textsc{aor}-\textsc{ego}\\
\glt `I stepped away from the wall.'
\z

In comparative constructions, the object of comparison is marked with
\emph{-če-}:

\ea
\gll rasul quwati le-w muħammadi-če-w.\\
Rasul strong be-\textsc{m} Muhammad-\textsc{super}-\textsc{m}(\textsc{ess})\\
\glt `Rasul is stronger than Muhammad.'
\z

The morpheme \emph{-če-} is used to mark the target of an oriented action,
e.g.\ with verbs such as `hit', `bark', `shout at', `be angry at',
`look at', `laugh at':

\ea
\gll {rasul laχu uk'-uwe le-w muħammadi-če.}\\
Rasul scream \textsc{m}.\textsc{lv}:\textsc{ipfv}-\textsc{cvb.ipfv} be-\textsc{m} Muhammad-\textsc{super}(\textsc{lat})\\
\glt `Rasul is shouting at Muhammad.'
\z

The \textsc{super}-\textsc{elative} \emph{-če-la} is used with verbs of
avoidance: `run away', `hide', `fear', etc.:

\ea
\gll {rasul w-aˤld-un muħammadi-če-la.}\\
Rasul \textsc{m}-hide:\textsc{pfv}-\textsc{aor} Muhammad-\textsc{super}-\textsc{el}\\
\glt `Rasul hid from Muhammad.'
\z

The marker \emph{-če-} is also used to mark periods of time.  The semantics of such constructions can be formulated as `X did something in two hours', i.e. the result was achieved:

\ea
\gll k'ʷi-jal saʡaˤti-če rasuj-ni kung b-elč-un.\\
two-\textsc{card} hour-\textsc{super}(\textsc{lat}) Rasul.\textsc{obl}-\textsc{erg} book \textsc{n}-read:\textsc{pfv}-\textsc{aor}\\
\glt `Rasul read the book in two hours.'
\z

% 5.10.
\subsection{The locative morpheme \emph{-ħe-}}
\label{the-locative-morpheme--ux127e-}

The locative morpheme \emph{-ħe-} expresses the configuration
\textsc{in} when one object is inside another one. The ground is, or is
conceptualized as, a container.

\ea \label{ex:3:38}
\gll {ħarši k'unk'ur-le-ħe-r le-r.}\\
soup pot-\textsc{obl}-\textsc{in}-\textsc{npl}(\textsc{ess}) be-\textsc{npl}\\
\glt `The soup is in the pot.'
\z

In (\ref{ex:3:38}), the morpheme \emph{-ħe-} causes vowel assimilation (i → e) in
the oblique stem marker. Between two vowels, [ħ] may be dropped, and
the vowels contract. In such cases, the only indication of \textsc{in}
semantics is the vowel change:

\ea
\gll {ħarši k'unk'ur-le-r le-r.}\\
soup pot-\textsc{obl}.\textsc{in}-\textsc{npl}(\textsc{ess}) be-\textsc{npl}\\
\glt `The soup is in the pot.'
\z

This localization does not have any non-locative uses in any of the
Dargwa dialects, including Mehweb.

% 5.11.
\subsection{The locative morpheme \emph{-ze-}}
\label{the-locative-morpheme-ze}

The morpheme \emph{-ze-} denotes the configuration when an object is
within the spatial area of the landmark and the landmark is either a
substance or a set of objects (e.g.\ `forest'). This configuration in
labelled \textsc{inter}:

\ea
\gll {k'as ħark'ʷi-ze-b le-b.}\\
fish river-\textsc{inter}-\textsc{n}(\textsc{ess}) be-\textsc{n}\\
\glt `The fish is in the river.'
\z

The morpheme \emph{-ze-} is also used in some \textsc{cont} contexts
(also see \sectref{the-locative-marker-ce}):

\ea
\gll {surat aqi-le le-b baˤʜi-ze-b.}\\
picture up-\textsc{advz} be-\textsc{n} wall-\textsc{inter}-\textsc{n}(\textsc{ess})\\
\glt `A picture is hanging on the wall.'
\z

Forms in \emph{-ze-la} (\textsc{inter}-\textsc{el}) express an \emph{involuntary
agent} – a participant who becomes the agent or cause of a situation
unintentionally. Only the inter-elative forms in \emph{-la} but not its
variants are used in this function:

\ea \label{ex:3:42}
\gll {di-ze-la \textup/ *di-ze-b-adala mašina b-oˤrʡ-oˤb.}\\
I.\textsc{obl}-\textsc{inter}-\textsc{el} / *I.\textsc{obl}-\textsc{inter}-\textsc{n}-\textsc{el} car \textsc{n}-break:\textsc{pfv}-\textsc{aor}\\
\glt `I accidentally broke the car.'
\z

The involuntary agent construction seems to combine only with
intransitive (labile in \ref{ex:3:42}) verbs and thus is a means of introducing an
agent-like participant rather than decreasing control on the part of a
true agent. The same locative form is also found in contexts of
participant-internal possibility:

\ea
\gll {rasuj-ze-la aq b-aq'-as b-uh-es ʁarʁa.}\\
Rasul.\textsc{obl}-\textsc{inter}-\textsc{el} up \textsc{n}-do:\textsc{pfv}-\textsc{inf} \textsc{n}-become:\textsc{pfv}-\textsc{fut} stone\\
\glt `Rasul will be able to lift the stone.'
\z

The morpheme \emph{-ze-} marks a temporary possessor (cf.\ \sectref{genitive}),
temporary recipient (cf.\ \sectref{dative}) and the addressee with verbs of speech:

\ea
\gll {rasuj-ni gi-b muħammadi-ze dis.}\\
Rasul.\textsc{obl}-\textsc{erg} give:\textsc{pfv}-\textsc{aor} Muhammad-\textsc{inter}(\textsc{lat}) knife\\
\glt `Rasul lent Muhammad a knife.'

\ex
\gll {rasuj-ni si-k'al ħa-ib muħammadi-ze.}\\
Rasul.\textsc{obl}-\textsc{erg} what-\textsc{ptcl} \textsc{neg}-say:\textsc{pfv}.\textsc{aor} Muhammad-\textsc{inter}(\textsc{lat})\\
\glt `Rasul said nothing to Muhammad'
\z

The functional range of \emph{-ze-} shows that its uses are not always
related to its spatial meaning, and that the spatial metaphor, when
present, may be weak.

% 5.12.
\subsection{The locative morpheme \emph{-šu-}}
\label{the-locative-morpheme--ux161u-}

The \textsc{ad} \emph{-šu-} localization is used to express the fact
that one object is located in close proximity to another object:

\ea
\gll nuša ustuj-šu-b ka-b-iʔ-i-ra.\\
we table.\textsc{obl}-\textsc{ad}-\textsc{hpl}(\textsc{ess}) \textsc{pv}-\textsc{hpl}-sit:\textsc{pfv}-\textsc{aor}-\textsc{ego}\\
\glt `We are sitting near the table.'
\z

It is also used as a personal locative:

\ea
\gll nu w-aˤq'-un-na aħmadi-šu.\\
I \textsc{m}-go:\textsc{pfv}-\textsc{aor}-\textsc{ego} Ahmad-\textsc{ad}(\textsc{lat})\\
\glt `I visited Ahmad.'
\z

% 5.13.
\subsection{The locative morpheme \emph{-ʡeˤ-}}
\label{the-locative-morpheme--ux2a1eux2c1-}

The \textsc{apud} marker \emph{-ʡeˤ-} denotes an area close to an
object, in which the figure must be located to interact with the object
(functional proximity). This suffix shows a very restricted
distribution. It is only compatible with words designating landmarks
that have an area associated with them in this way; e.g.\ \emph{ustul}
`table', \emph{iniz} `water source', \emph{qali} `house'. In different
languages, the same landmark may be conceptualized as having such an
area or not. In Mehweb the set of words to which this suffix can be attached
varies across speakers. The following examples illustrate the difference
between the \textsc{ad} \emph{-šu-} and \textsc{apud} \emph{-ʡeˤ-}
localizations:

\ea
\gll nuša ustuj-ʡeˤ-b ka-b-iʔ-i-ra.\\
we table.\textsc{obl}-\textsc{apud}-\textsc{hpl}(\textsc{ess}) \textsc{pv}-\textsc{hpl}-sit:\textsc{pfv}-\textsc{aor}-\textsc{ego}\\
\glt `We are sitting at the table.'

\ex
\gll nuša ustuj-šu-b ka-b-iʔ-i-ra.\\
we table.\textsc{obl}-\textsc{ad}-\textsc{hpl}(\textsc{ess}) \textsc{pv}-\textsc{hpl}-sit:\textsc{pfv}-\textsc{aor}-\textsc{ego}\\
\glt `We are sitting near the table.'

\ex
\gll lut'i-le-ʡeˤ-b\\
bottom-\textsc{obl}-\textsc{apud}-\textsc{n}(\textsc{ess})\\
\glt `on the bottom' (of a pond etc.)
\z

It also expresses the meaning of an exchange equivalent:

\nopagebreak

\ea
\gll rasuj-ni bars b-aq'-ib q'ʷaˤl šu-wal {kʷiha-le-ʡeˤ-b}.\\
Rasul.\textsc{obl}-\textsc{erg} exchange \textsc{n}-do:\textsc{pfv}-\textsc{aor} cow five-\textsc{card} sheep-\textsc{obl}-\textsc{apud}-\textsc{n}(\textsc{ess})\\
\glt `Rasul exchanged the cow for five sheep.'
\z

With appropriate grounds, the morpheme \emph{-ʡeˤ-} may be used to
designate the area not near to but bounded by the landmark:

\ea \label{ex:3:52}
\gll {škaf unza-le-ʡeˤ-di b-aˤq'-un.}\\
wardrobe door-\textsc{obl}-\textsc{apud}-\textsc{trans} \textsc{n}-go:\textsc{pfv}-\textsc{aor}\\
\glt `The wardrobe went through the door.'
\z

It thus becomes semantically similar to \emph{-ħe-}; in (\ref{ex:3:53}),
\emph{-ħe-} is used in the same context:

\ea \label{ex:3:53}
\gll {škaf unza-le-ħe-di b-aˤq'-un.}\\
wardrobe door-\textsc{obl}-\textsc{in}-\textsc{trans} \textsc{n}-go:\textsc{pfv}-\textsc{aor}\\
\glt `The wardrobe went through the door.'
\z

Like \emph{-ħe-}, \emph{-ʡeˤ-} causes vowel assimilation
\emph{i} → \emph{e} in the oblique stem marker (cf.\ \ref{ex:3:52} and \ref{ex:3:53}).



% 6.
\section{Irregular locatives}\label{irregular-locatives}

A limited number of nouns form locatives in an irregular way. Such
irregular locatives usually mark the default location associated with
the landmark. As with the locative forms discussed above, the presence
of a gender agreement slot conveys the meaning of stative location
(essive form), and the same form without the slot conveys the meaning of
direction towards (lative). \tabref{tab:3:38} shows the irregular locatives attested so
far.

\begin{table}[h]
  % Table 38.
  \caption{Irregular locatives}\label{tab:3:38}
  \begin{tabular}{@{}llp{.48\textwidth}@{}}
\toprule 
&  Nominative & Locative  \tabularnewline \midrule  
`forest' &  \emph{duz}  & \emph{duzani-}\textsc{cl} \tabularnewline
`grave' & \emph{χʷaˤb} (PL = \emph{χʷaˤrbe}) &  \emph{χʷaˤre-}\textsc{cl} `in a grave',
   cf. \emph{χʷaˤrbeze-}\textsc{cl} `at a graveyard'
    (lit. `between graves')\tabularnewline
`road' & \emph{huni} & \emph{hunħe-}\textsc{cl} \tabularnewline
`village'  &  \emph{ši}  &  \emph{ša-}\textsc{cl} \tabularnewline
`room', `house' &  \emph{qali} & \emph{quli-}\textsc{cl} \tabularnewline
`cattle-shed' &  \emph{derqʷ} & \emph{durqe-}\textsc{cl} \tabularnewline
`field'  & \emph{qu}  & \emph{qu-}\textsc{cl} \tabularnewline
`gorge', `street' & \emph{q'aq'a} & \emph{q'aq'a-}\textsc{cl} \tabularnewline
`hole' & \emph{tarqi} & \emph{turqe-}\textsc{cl}\tabularnewline
\bottomrule
  \end{tabular}
\vspace{-2\jot}
\end{table}




% 7.
\section{Place names}

Names of local villages form a separate morphological class close to
adverbs; they lack functional cases and attach orientation markers
directly to the stem. Their unmarked locative (i.e.\ lative) form also
serves as quotation form. They are nominalized by adding \emph{-n} (also
used in the nominalization of adjectives) and form plurals in \emph{-t} to
designate the inhabitants of the village. While the genitive in
\emph{-la} is produced by simple suffixation of the genitive marker, the
variant genitive in \emph{-ja} probably derives from the nominalized
form in \emph{-n} (\emph{-ja} \textless{} \emph{-n-la}, as discussed in
\citet{moroz2019}, thus meaning not `that of the village of Mehweb' but `that
of a Mehweb villager'.

{The inflection of local place names is given in \tabref{tab:3:39}. Declension
of \emph{anži} `Ma\-khach\-kala' and \emph{maskaw} `Moscow', which are not
local placenames and behave like regular nouns, is given for the sake of
comparison in the last lines of each column.}

\begin{table}[h]
  \vspace{-\jot}
  % Table 39.
  \caption{Place names}\label{tab:3:39}
  \advance\tabcolsep-3pt
\begin{tabular}{@{}llll@{}}
\toprule
& \textsc{quot} & \textsc{ess} & \textsc{el}\tabularnewline \midrule
`Mehweb' & \emph{meħʷe} & \emph{meħʷe-}\textsc{cl} &
\emph{meħʷe-}\textsc{cl}\emph{-adal}, \emph{meħʷe-la}\tabularnewline
`Sogratl' & \emph{surʁatli} & \emph{surʁatli-}\textsc{cl} &
\emph{surʁatli-}\textsc{cl}\emph{-adal},
\emph{surʁatli-la}\tabularnewline
`Obokh' & \emph{qʷaˤdulli} & \emph{qʷaˤdulli-}\textsc{cl} &
\emph{qʷaˤdulli-}\textsc{cl}\emph{-adal, qʷaˤdura-ja}\tabularnewline
`Gunib' & \emph{ʁuni} & \emph{ʁuni-}\textsc{cl} &
\emph{ʁuni-}\textsc{cl}\emph{-adal, ʁuni-la}\tabularnewline
`Keger' & \emph{ʜaˤnnuqara} & \emph{ʜaˤnnuqara-}\textsc{cl} &
\emph{ʜaˤnnuqara-}\textsc{cl}\emph{-adal},
\emph{ʜaˤnuqara-la}\tabularnewline
`Makhachkala' & \emph{anži} & \emph{anži-li-}\textsc{cl} &
\emph{anži-li-}\textsc{cl}\emph{-adal}, \emph{anži-la}\tabularnewline
`Moscow' & \emph{maskaw} & \emph{maskawi-ze-}\textsc{cl} &
\emph{maskawi-ze-la}\tabularnewline \midrule
& \textsc{lat} & \textsc{gen} & \textsc{pl}\tabularnewline \midrule
`Mehweb' & \emph{meħʷe} & \emph{meħʷ-aja} & \emph{meħʷ-an-t} (the Mehweb
people)\tabularnewline
`Sogratl' & \emph{surʁatli} & \emph{surʁatl-aja} & \emph{surʁatl-an-t}
(the Sogratl people)\tabularnewline
`Obokh' & \emph{qʷaˤdulli} & \emph{qʷaˤdur-aja} & \emph{qʷaˤdur-an-t}
(the Obokh people)\tabularnewline
`Gunib' & \emph{ʁuni} & \emph{ʁuni-}\textsc{cl}\emph{-adi}\emph{-ja} &
\emph{ʁuni-}\textsc{cl}\emph{-adil} (the Gunib people)\tabularnewline
`Keger' & \emph{ʜaˤnnuqara} & \emph{ʜaˤnnuqara-ja} &
\emph{ʜaˤnnuqara-n-t} (the Keger people)\tabularnewline
`Makhachkala' & \emph{anžili} & \emph{anži-la} &
\textsuperscript{?}*\emph{anžili-}\textsc{cl}\emph{-adil}\tabularnewline
`Moscow' & \emph{maskawi-ze} & \emph{maskaw-la} &
\textsuperscript{?}*\emph{maskawi-ze-}\textsc{cl}\emph{-adil}\tabularnewline
\bottomrule
\end{tabular}
  \vspace{-\jot}
\end{table}



\section*{Acknowledgements}

The author is grateful to the Mehweb people for being extremely
  generous in sharing their knowledge of the language, to his fellow
  fieldworkers, for their support, and to his teachers for their
  careful guidance and endless patience.

%   \clearpage

\section*{List of abbreviations}

\begin{longtable}[l]{@{}ll@{}}
\textsc{ad}	& spatial domain near the landmark \\
\textsc{advz}	& adverbializer \\
\textsc{dir}	& motion directed towards a spatial domain \\
\textsc{aor}	& aorist \\
\textsc{apud}	& spatial domain near the landmark \\
\textsc{atr}	& attributivizer \\
\textsc{card}	& cardinal numeral \\
\textsc{causal}	& causal (case form) \\
\textsc{cl}	& gender (class) agreement slot \\
\textsc{comit}	& comitative \\
\textsc{dat}	& dative \\
\textsc{ego}	& egophoric \\
\textsc{el}	& motion from a spatial domain \\
\textsc{erg}	& ergative \\
\textsc{ess}	& static location in a spatial domain \\
\textsc{f}	& feminine (gender agreement) \\
\textsc{f1}	& feminine (unmarried and young women gender prefix) \\
\textsc{fut}	& future \\
\textsc{gen}	& genitive \\
\textsc{hab}	& habitual (durative for verbs denoting states) \\
\textsc{hpl}	& human plural (gender agreement) \\
\textsc{imp}	& imperative \\
\textsc{in}	& spatial domain inside a (hollow) landmark \\
\textsc{inf}	& infinitive \\
\textsc{inter}	& spatial domain between multiple landmarks \\
\textsc{ipft}	& imperfect \\
\textsc{ipfv}	& imperfective (derivational base) \\
\textsc{lat}	& motion into a spatial domain \\
\textsc{lv}	& light verb \\
\textsc{m}	& masculine (gender agreement) \\
\textsc{n}	& neuter (gender agreement) \\
\textsc{neg}	& negation (verbal prefix) \\
\textsc{nmlz}	& nominalizer \\
\textsc{nom}	& nominative \\
\textsc{npl}	& non-human plural (gender agreement) \\
\textsc{obl}	& oblique (nominal stem suffix) \\
\textsc{pfv}	& perfective (derivational base) \\
\textsc{pl}	& plural \\
\textsc{pst}	& past \\
\textsc{ptcl}	& particle \\
\textsc{pv}	& preverb (verbal prefix) \\
\textsc{repl}	& replicative (nominal case) \\
\textsc{subst}	& substitutive (nominal case) \\
\textsc{super}	& spatial domain on the horizontal surface of the landmark \\
\textsc{tr}	& transitive \\
\textsc{trans}	& motion through a spatial domain \\
\end{longtable}

\printbibliography[heading=subbibliography,notkeyword=this]




\iffalse

\section*{References}


  
Daniel M. (2019). Mehweb verb morphology. In M. Daniel, D. Ganenkov, N. Dobrushina (Ed.), Mehweb: Selected essays on phonology, morphology and syntax. Berlin, Language Science Press.

Haspelmath, M., \& Sims, A. (2010). Understanding Morphology. Oxford University Press.

Kibrik A. E. (2003). Konstanty i peremennye jazyka. Sankt-Peterburg, Aleteja.

Rubin, E. (2001). Figure and Ground. In Yantis, S.(Ed.), Visual Perception. (pp. 225-229). Philadelphia, Psychology Press.

Magometov A. A. (1982). Megebskij dialekt darginskogo jazyka (Issledovanie i
teksty). Tbilisi, Mecniereba.

Moroz G. (2019). Phonology of Mehweb. In M. Daniel, D. Ganenkov, N. Dobrushina (Ed.), Mehweb: Selected essays on phonology, morphology and syntax. Berlin, Language Science Press.
  

\fi
\end{document}

%%% Local Variables:
%%% mode: latex
%%% TeX-master: "../main"
%%% End:
