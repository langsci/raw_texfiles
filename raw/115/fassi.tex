%!TEX root = ../main.tex

\author{Abdelkader Fassi Fehri\affiliation{Mohammed V University}}
\title{New roles for Gender: Evidence from Arabic, Semitic, Berber, and Romance}
% \epigram{Change epigram}
\abstract{Contrary to a widespread sex-based typology/theory of Gen(der), where
  it is essentially construed as (a) a \textit{nominal class marking} device,
  (b) semantically \textit{sex-based}, and (c) syntactically \textit{reflected
  in gender agreement} through sexed-animate controllers, I argue instead that
  Gen is (a) \textit{polysemous}, (b) \textit{multi-layeredly distributed} in
  the DP, CP, or SAP architecture, and (c) it exhibits a variety of distinct
  controllers and properties of agreement. Consequently, its grammar,
  semantics/pragmatics, and representation turn out to be radically different
  from what is standardly assumed. The analysis is implemented in a minimalist
  Distributed Morphology model.}
\ChapterDOI{10.5281/zenodo.1116769}

\maketitle

\begin{document}

\section{Introduction}\label{sec:fassi:intro}
 

% \textbf{Keywords}: gender polysemy, distributed gender, individuative,
% plurative, evaluative, performativity

Up until very recently, both typologists and theoretical linguists have
entertained a rather simplistic (and exclusive) view of Gender\is{gender} and its role in
the grammar, despite its well-acknowledged complexity. Hence back to (at least)
\citet{Grimm1822} for Indo-European, or \citet{Caspari1859} for \ili{Semitic}, a
wide-spread typology/theory sees Gen(der) as (a) essentially a nominal class
marking device, (b) semantically sex-based
\citep[e.g.][]{Corbett1991,KibortEtAl2008}, or animacy-based \citep{Dahl2000},
in addition to (c) being reflected in gender\is{gender} \isi{agreement} \citep{KibortEtAl2008}
with sexed controllers (or goals). But back to \citet{Brugmann1897} for Indo-European, or \citet{Brockelmann1910} for \ili{Semitic} (among other sources), Gen
(and typically the feminine) has been associated with diverse meanings
including \textit{individuation}, \textit{collectivity}, \textit{abstractness},
\textit{quantity}, \textit{size}, etc. Old or new grammarians have added even
more new meanings and structures, including qualitative \textit{evaluation}
(`depreciative', `affective', `endearing', etc.), \textit{perspectivization}
(of plurality, `attenuation', etc.), and \textit{speech act role modification}
or \textit{performativity} in expressive contexts (as I will show). This
polysemy and the differentiated multitude of structures are not expected if Gen
is confined to the \textit{n} (and `lexical') domain, construed as sex, and
gender\is{gender} \isi{agreement} limited to sexed configurations rather than appropriately
distributed over various layers of the DP structure, or even the more higher \isi{CP}
and Speech Act role cartography \citep[as in][]{SpeasEtAl2003,Hill2014}, with
productive non-sex interpretations and interrelations.

Overall, the contribution aims at providing a more \emph{integrative}
description of the \textit{gender\is{gender} polysemy} than the `orthodox' sex/animate
view can allow for. It is meant to be \textit{constructional}, and hence
providing room for more `unorthodox' syntax (such as that of \isi{CP}, or the higher
SAP). The various \textit{distributed} positions of Gen, and its plausibly
related orthodox and unorthodox meanings make Gen potentially and semantically
\textit{hyperonymic} (i.e. general enough to embrace more diverse and
structurally organized and related meanings found cross-linguistically), and
sex/animacy only a \textit{hyponym} (or special) case. Our polysemic treatment
and representation is inspired partly by \citet{Jurafsky1996} and
\citet{Grandi2015} analysis of evaluative meanings, and it receives further
support from work on neural correlates of \isi{semantic} ambiguity, offering
behavioral and neurophysiological support for a single-entry model of polysemy
(in contrast to homonymic separate entries), in line with
\citet{BerettaEtAl2005,PylkkanenEtAl2006}, or
\citet{Marantz2005}.
The article is organized as follows. In \sectref{sec:fassi:facets}, I present
various instances of the rich \isi{semantic} diversity of Gender\is{gender}, as illustrated by
Standard and Moroccan \ili{Arabic} varieties. In \sectref{sec:fassi:singulativity}, I
investigate the properties of two unorthodox gendered constructions: the
\textit{singulative} and the \textit{plurative}, and their forms of \isi{agreement}
alternations. In \sectref{sec:fassi:layers}, I motivate the identification of
\textit{five layers} of Gen architecture which produce essential
interpretations of Gen (including conceptual Gen, and `performative' Gen).
Multiple distinct valued features (including $\pm$ fem, $\pm$ indiv, $\pm$
group, $\pm$ small/big, $\pm$ bad/good, $\pm$ endearing, etc.) are made use of,
when interpretable. \sectref{sec:fassi:modification} is dedicated to
investigate size and performative evaluation. The latter interpretation is
implemented in a Speech Act \isi{Cartography} à la \citet{SpeasEtAl2003} and
\citet{Hill2014}. In \sectref{sec:fassi:extensions}, I turn to more cross-
linguistic motivation of the polysemic distributed view of Gen by identifying
and investigating some relevant gender\is{gender} patterns in \ili{Berber}, \ili{Hebrew}, and \ili{Romance}.
In \sectref{sec:fassi:semprag}, I discuss the issue of semantics-pragmatics and
morpho-syntax interfaces, and the representation of Gen polysemy.
\sectref{sec:fassi:conclusion} provides a conclusion. Throughout the paper, I
will be assuming a minimalist distributed-morphology model of grammar based
on \citet{Chomsky1995,HalleEtAl1993,Marantz1997,Harley2014}, among others.

\section{The many various facets and uses of Gen}\label{sec:fassi:facets}

\subsection{Sex-based and formal Gen}

`Natural' sex gender\is{gender} (interpretable as \textsc{female/male}) plays only a
partially productive role in the grammar of \ili{Arabic} `inflection' (the
\xv{-at} suffix often marking the feminine, a general property of \ili{Semitic}). In
\REF{ex:fassi:1}, the feminine suffix \xv{-at} is added to the `masculine' form
to derive the feminine:\footnote{Unless stated otherwise, the examples given
are from Standard \ili{Arabic}.}

\begin{exe}
  \ex\label{ex:fassi:1}
  \xv{kalb} dog \gl{he-dog} → \xv{kalb-at} dog-\textsc{fem} \gl{she-dog}
\end{exe}

But the feminine is also largely expressed as an (inherently) `lexical' gender\is{gender},
as in \REF{ex:fassi:2}:

\begin{exe}
  \ex\label{ex:fassi:2}
  \begin{xlist}
    \ex\label{ex:fassi:2a} \xv{qird} monkey \gl{he-monkey} → \xv{qi\v{s}\v{s}-at} monkey-\textsc{fem} \gl{she-monkey}
    \ex\label{ex:fassi:2b} \xv{\d{h}imaar} donkey \gl{he-donkey} → \xv{ʔa\b{t}aan} donkey-\textsc{fem} \gl{she-donkey}
  \end{xlist}
\end{exe}

Note, however, that the morphological feminine tends to replace the `lexical'
counterpart in modern standard usage, as exemplified in \REF{ex:fassi:3}. In
the colloquials, only the regular morphological formation tends to be used in
these cases, as exemplified by the Moroccan \ili{Arabic} pairs in \REF{ex:fassi:4}:

\begin{exe}
  \ex\label{ex:fassi:3} Standard \ili{Arabic}
  \begin{xlist}
    \ex\label{ex:fassi:3a} \xv{qird} monkey \gl{he-monkey} → \xv{qird-at} monkey-\textsc{fem} \gl{she-monkey}
    \ex\label{ex:fassi:3b} \xv{\d{h}imaar} donkey \gl{he-donkey} → \xv{\d{h}imaar-at} donkey-\textsc{fem} \gl{she-donkey}
  \end{xlist}
\end{exe}

\begin{exe}
  \ex\label{ex:fassi:4} Moroccan \ili{Arabic}
  \begin{xlist}
    \ex\label{ex:fassi:4a} \xv{qard} monkey \gl{he-monkey} → \xv{qard-a} monkey-\textsc{fem} \gl{she-monkey}
    \ex\label{ex:fassi:4b} \xv{\d{h}maar} donkey \gl{he-donkey} → \xv{\d{h}maar-a} donkey-\textsc{fem} \gl{she-donkey}
  \end{xlist}
\end{exe}

Formal `idiosyncratic' gender\is{gender} has been claimed to be a property of nouns like
the following:

\begin{exe}
  \ex\label{ex:fassi:5}
  \begin{xlist}
    \ex\label{ex:fassi:5a} \xv[sun]{\v{s}ams}, \textsc{fem} (compare with \ili{French} \xv{soleil}, \textsc{masc})
    \ex\label{ex:fassi:5b} \xv[moon]{qamar}, \textsc{masc} (cf. \ili{French} \xv{lune}, \textsc{fem})
    \ex\label{ex:fassi:5c} \xv[river]{nahr}, \textsc{masc} (cf. \ili{French} \xv{rivière}, \textsc{fem})
  \end{xlist}
\end{exe}

\subsection{Less `orthodox' meanings}

What is more important is the long list of `unorthodox' gender\is{gender} meanings. I will
exemplify only some instances here, with no pretention to be exhaustive.

\subsubsection{Singulative}

In singulative expressions (traditionally called \xv[nouns of unit]{ism waḥd-ah} 
by \ili{Arabic} traditional grammarians), a `feminine' suffix (\xv{-at}) forms a
singular nP denoting a discrete \textit{unit} from a kind base. It also
controls a feminine \isi{agreement} (although the controller is not a female):

\begin{exe}
  \ex\label{ex:fassi:6}
  \begin{xlist}
    \ex\label{ex:fassi:6a} \xv{na\d{h}l} bee \gl{bees} → \xv{na\d{h}l-at} bee-\textsc{unit} \gl{a bee}
    \ex\label{ex:fassi:6b} \gll \v{s}taray-tu samak-at-an kabiir-at-an\\
    bought-I fish-\textsc{unit-acc} big-\textsc{fem-acc}\\
    \glt `I bought a big fish.'
    \ex\label{ex:fassi:6c} \gll \v{s}taray-tu samak-an kabiir-an\\
    bought-I fish-\textsc{acc} big-\textsc{acc}\\
    \glt `I bought big fish.'
  \end{xlist}
\end{exe}

The suffix \xv{-at} here is known as `singulative' in the literature. It has
been qualified as playing essentially the same role as an individualizing
classifier (\citealt{Greenberg1972}, after the \ili{Arabic} tradition, back to \citealt{Sibawayhi1938,FassiFehri2004,FassiFehri2012,Mathieu2012,Zabbal2002},
among others). Typologically in fact, the singulative is closer to a noun Class
than to a Classifier, although it fulfils essentially the same
role.\footnote{The comparison has been made between Gender\is{gender}, Class, and
Classifier by \citet{Seifart2010}, as well as \citet{CrismaEtAl2011}, among
others, using distinctive criteria. They both conclude that the Chinese
classifier type is singled out as not implicating \isi{agreement}, in contrast to the
other two (in \ili{Romance} and \ili{Bantu}), which appear to be closer to Gen
manifestations.}

\subsubsection{Plurative}

In plurative expressions (in my terminology), the same gender\is{gender} morpheme \xv{-at}
forms a \textit{group} or a collection individual from a singular or a plural
of individuals \citep[see][]{FassiFehri1988,FassiFehri2012}:

\begin{exe}
  \ex\label{ex:fassi:7}
  \begin{xlist}
    \ex\label{ex:fassi:7a} \xv[inhabitant]{saakin} → \xv[inhabitants, population]{saakin-at}
    \ex\label{ex:fassi:7b} \xv{muʕtazil(-ii)} solitary \gl{a member of the (so-named) theologian thinker group} → \xv[the (so-named) theologian thinker group]{muʕtazil-at}
    \ex\label{ex:fassi:7c} \xv[unbeliever]{kaafir} → \xv[unbelievers]{kafar} → \xv[unbelievers (as a group)]{kafar-at}
  \end{xlist}
\end{exe}

In the relevant cases, the constructed nP denotes an \textit{integrated whole},
and the morpheme contributes to shape this whole. It can be thought of as a
sort of classifier (or a ``grouper''). I return later on to its exact
contribution. Note that the plurative, like the singulative, controllers
feminine singular \isi{agreement}, as illustrated by the following construction:

\begin{exe}
  \ex\label{ex:fassi:8}
  \gll s-saakin-at-u \d{h}tajj-at\\
  the-inhabitant-\textsc{fem-nom} protested-\textsc{fem}\\
  \glt `The inhabitants (as a group) protested.'\\
\end{exe}

\subsubsection{Gendered augmentative}

Augmentatives are internally formed first, then \xv{-at} can be affixed to
them. The affix then functions as intensive or evaluative:

\begin{exe}
  \ex\label{ex:fassi:9} \xv{raa\d{h}il} \gl{travelling, traveller} →
  \xv{ra\d{h}\d{h}aal} \gl{a big traveller}\\ →
  \xv{ra\d{h}\d{h}aal-at} \gl{traveller + augmentative + \textsc{fem}}
  \begin{xlist}
    \ex\label{ex:fassi:9a} intensive: `an extremely big traveller'
    \ex\label{ex:fassi:9b} evaluative: `an acknowledged big traveller'
  \end{xlist}
\end{exe}

\subsubsection{Gendered diminutive}

When a diminutive is internally formed, and the morpheme \xv{-at} is suffixed
to it, it expresses `intensive' decrease in size, affectivity, or eventually a
`unit reading', as is exemplified by the various meanings of \REF{ex:fassi:10}:

\begin{exe}
  \ex\label{ex:fassi:10} \xv{zayt} \gl{oil} → \xv{zuwayt} oil.\textsc{dim} \gl{small quantity of oil} → \xv{zuwayt-at} oil.\textsc{dim-fem}
  \begin{xlist}
    \ex\label{ex:fassi:10a} intensive: `an extremely small quantity of oil'
    \ex\label{ex:fassi:10b} evaluative: `a beloved small quantity of oil'
    \ex\label{ex:fassi:10c} unit reading: `a discrete small quantity of oil'
  \end{xlist}
\end{exe}

\subsubsection{Gendered event units}

An event nominal acting as a cognate object can express a \textit{kind event},
as in \REF{ex:fassi:11a}, where it denotes that one or more dances have been
performed, or a countable \textit{event unit} (or instance) as in
\REF{ex:fassi:11b}:

\begin{exe}
  \ex\label{ex:fassi:11}
  \begin{xlist}
    \ex\label{ex:fassi:11a}
    \gll raqa\d{s}a raq\d{s}-an\\
    danced dance-\textsc{acc}\\
    \glt `He danced some dancing.'\\
    \ex\label{ex:fassi:11b}
    \gll raqa\d{s}a raq\d{s}-at-an; raq\d{s}-at-ayn\\
    danced dance-\textsc{unit-acc} dance-\textsc{unit-dual}\\
    \glt `He danced a dance; two dances.'\\
  \end{xlist}
\end{exe}

The formation of event units here parallels that of concrete nouns formed in
\REF{ex:fassi:6}; see \textcite{FassiFehri2005,FassiFehri2012} for detail.

\subsubsection{Gendered abstract nouns}

Abstract nouns or concepts which name qualities, doctrines, sects, etc. also
behave syntactically like feminine nPs, and they are affixed with the feminine
marker:

\begin{exe}
  \ex\label{ex:fassi:12}
  \begin{xlist}
    \ex\label{ex:fassi:12a}
    \gll suhuul-at-un kabiir-at-un\\
    easy-\textsc{fem-nom} big-\textsc{fem-nom}\\
    \glt `A great easiness.'\\
    \ex\label{ex:fassi:12b} \xv[arabity]{ʕuruub-at}; \xv[negritude]{zunuuj-at}
  \end{xlist}
\end{exe}

In most cases, these nouns are formed from an adjectival base to denote the
name of the property or quality, or abstract concept. Nouns such as those are
often feminine in other languages as well, as in \ili{French} \xv[easy]{facile} → \xv
[easy-ness]{facilité}.

\subsection{A new picture}

In Indo-European studies, \citet{Brugmann1897} observed that the same marker is
employed for collectives, abstractions, and the feminine, which suggests
questioning the ``sexual content'' of the feminine, rather than ``feminizing''
collectives and abstractions. \citet{Leiss1994} reformulated Brugmann's insight
in terms of \textit{perspectivization}, in the sense that the function of
gender\is{gender} is to provide a ``different perspective to represent a multitude of
entities'' (203).\footnote{Perspective, construal, point of view, or
subjectivity have been used as terms to designate the speaker's perception of
the entity involved. According to \citet{Unterbeck2000}, quantity is the
feature that connects the two categories \isi{Num} and Gen: \isi{Num} expresses a
multitude, and Gen different perspectives of multitudes \citep[see
also][]{Hachimi2007}. I adopt the perspectivization view of Gen below, and
provide a representation of its place in the DP.}

In the \ili{Arabic} grammatical and philological tradition, regular descriptions of
Gen connect feminine, collectives, abstractions, plurals, intensives, etc. I
derive these connections through the architecture of quantity (\#, as in
\citealt{Borer2005}), sex ($\pm$ fem), and size ($\pm$ big / small). Evaluation
is especially included in the \ili{Arabic} tradition for the diminutive, and only
marginally for the augmentative.\footnote{Regarding Western sources, I refer to
\citet{Ibrahim1973} for an early synopsis of the traditions of thoughts,
\citet{Hachimi2007} for a good overview of the patterns and issues involved, in
addition to \citet{Fleisch1961,Roman1990}, and Wright (\citeyear{Wright1971};
originally written in \ili{German} by \citet{Caspari1859}, with many \ili{Arabic} sources
included).}
  

\section{Singulativity and plurativity}\label{sec:fassi:singulativity}

\subsection{Singulativity}

\subsubsection{Essential properties}

\citet{FassiFehri2016} provides a list of the most salient properties of the
singulative:

\begin{enumerate}
  \item It is a process by which a collective (and less frequently a mass noun) is turned into a single individual or unit.
  \item It is commonly marked via Gender\is{gender} (or the feminine) cross-linguistically (\ili{Arabic}, \ili{Berber}, \ili{Breton}, \ili{Welsh}, \ili{Somali}, \ili{Hebrew}, \ili{Russian}, etc.; \citealt[see e.g.][]{Mathieu2013}).
  \item It triggers feminine singular \isi{agreement} on its target.
  \item It has the interpretation of a singularity (not that of an `inclusive' or `week' plural, as in \REF{ex:fassi:14c} below).
  \item It can be dualized, pluralized, or counted by numerals.
\end{enumerate}

In \REF{ex:fassi:13}, the feminine appears to individualize a mass noun:

\begin{exe}
  \ex\label{ex:fassi:13}
  \begin{xlist}
    \ex\label{ex:fassi:13a} \xv{xa\v{s}ab} \gl{wood} (mass) → \xv[piece of wood]{xa\v{s}ab-at}
    \ex\label{ex:fassi:13b} \xv{\v{s}amʕ} \gl{wax} (mass) → \xv{\v{s}amʕ-at} wax-\textsc{unit} \gl{a candle}
  \end{xlist}
\end{exe}

In \REF{ex:fassi:14a}, the singulative is singular, in \REF{ex:fassi:14b}, it
is dual; but in \REF{ex:fassi:14c}, the general noun is rather interpreted as
`weak plural' (i.e. as singular or plural):

\begin{exe}
  \ex\label{ex:fassi:14}
  \begin{xlist}
    \ex\label{ex:fassi:14a}
    \gll ʔakal-tu tamr-at-an\\
    ate-I date-\textsc{unit-acc}\\
    \glt `I ate a date.'\\
    \ex\label{ex:fassi:14b}
    \gll ʔakal-tu tamr-at-ayn\\
    ate-I date-\textsc{unit-dual.acc}\\
    \glt `I ate two dates.'\\
    \ex\label{ex:fassi:14c}
    \gll ʔakal-tu tamr-an\\
    ate-I date-\textsc{acc}\\
    \glt `I ate (one or more) dates.'\\
  \end{xlist}
\end{exe}

By contrast, the plural of the singulative in \REF{ex:fassi:15} can only be
`strong' or `exclusive' (which means that only more than one date can be
involved):

\begin{exe}
  \ex\label{ex:fassi:15}
  \gll ʔakal-tu tamar-aat-in\\
  ate-I date-\textsc{unit.plural-acc}\\
  \glt `I ate (many) dates.'\\
\end{exe}

\subsubsection{Structure}

We can see from \REF{ex:fassi:14} and \REF{ex:fassi:15} that there is no
complementary distribution between the individualizer (Div or Cl) and \isi{Num} (\#),
the dual, or the multiplying plural. I postulate \REF{ex:fassi:16} as a
structure of \REF{ex:fassi:15}, in which the singulative (Cl) and the plural
(\isi{Num}) co-occur:\footnote{\citet{Ouwayda2014}, although arguing that \isi{Num} and Gen
are separate categories in this sound plural construction, maintains the view
that the plural here is a mere \isi{agreement} marker (with a hidden numeral). But
there is enough evidence to reject this complementarity view. See
\textcite{FassiFehri2012,FassiFehri2016} for detail.}
 


\ea\label{ex:fassi:16}
\begin{forest}
  [NumP (\#P)
  [\isi{Num}\slash\# [Pl [\textit{aa}]]]
  [Cl\slash DivP [Div [\textit{at}]] [n [\textit{tamr},name=tamr]]]
  ]
  \node[right=2em of tamr] {(\textit{tamar-aat})};
\end{forest}
\z

\subsection{The plurative}

Contrary to the singulative, the \textit{plurative} is only marginally
mentioned in the literature, identified, or investigated. Few rather informal
uses of this term are found in the Africanist literature (see e.g.
\citealt{Dimmendaal1983}, or \citealt{Mous2008}), basically seeing it as the
opposite process of the singulative. Discussing
\citeauthor{Hayward1984}'s~(\citeyear{Hayward1984}) observation that in the
Cushitic language Arbore, many nouns have a general form (which is non-
specified as to the singular/plural distinction), although they can be
pluralized, as in:

\begin{exe}
  \ex\label{ex:fassi:17} \xv[dog(s)]{kér} → \xv[dogs]{ker-ó}
\end{exe}

\citet[17, fn. 11]{Corbett2000} made the following comment: ``If one uses
`singulative' consistently for singular forms which correspond to a more basic
plural form, then it would be logical to use the term `plurative' for plural
forms which correspond to a more basic singular, as in \xv[dog]{kér}
\textasciitilde\xspace \xv[dogs]{ker-ó} above, as suggested by
\citet[224]{Dimmendaal1983}''.

Compared to the singulative, the plurative appears to be taking an opposite
path to be derived, as schematized in \REF{ex:fassi:18}:

\begin{exe}
  \ex\label{ex:fassi:18}
  \begin{xlist}
    \ex\label{ex:fassi:18a} `collective' → singulative
    \ex\label{ex:fassi:18b} plurative ← `collective'
  \end{xlist}
\end{exe}

In the Africanist literature, the plurative appears to be a process by which a
strong or distributive plural is derived from a base which is a general noun
\citep[see][]{Mous2008}. The exact \ili{Arabic} counterpart of such a process would
then be the plural of a collective, which is rather exclusive. The following
derivation illustrates such a process:

\begin{exe}
  \ex\label{ex:fassi:19}
  \begin{xlist}
    \ex\label{ex:fassi:19a} \xv{samak} \gl{fish} (collective) → \xv{ʔasmaak} \gl{many fish} (plurative)
    \ex\label{ex:fassi:19b}
    \gll \v{s}taray-tu ʔasmaak-an mulawwan-at-an\\
    bought-I fish.\textsc{pl-acc} coloured-\textsc{fem-acc}\\
    \glt `I bought (many) coloured fish.'\\
    \ex\label{ex:fassi:19c}
    \gll \v{s}taray-tu samak-an mulawwan-an\\
    bought-I fish-\textsc{acc} coloured-\textsc{acc}\\
    \glt `I bought (one or more) coloured fish.'\\
  \end{xlist}
\end{exe}

Compared to \REF{ex:fassi:19c}, which can be felicitous even if only one fish
is bought, \REF{ex:fassi:19b} cannot be so interpreted, and the number of fish
must be more than one, comparable to the interpretation of the strong
interpretation associated with the plural of the singulative in
\REF{ex:fassi:15} above. But because \REF{ex:fassi:19b} might be seen as
pluralizing a weak plural (the so-called general noun), it is often thought to
be a `double plural'; although the plural of the singulative cannot be so
conceived (see \citealt{FassiFehri2012} for detail).

According to \citeauthor{Mous2012}~(\citeyear{Mous2012}, p.c.), the most
important property of the Cushitic plurative is that it triggers a `third
gender\is{gender}' \isi{agreement}, which takes the form of a plural. But note that the \ili{Arabic}
plurative, as I construe it, is not the plural of the collective, as in
Cushitic, but rather the closest counterpart to the singulative. Both control a
`feminine' (singulative) \isi{agreement}, and the plurative is also forming a unit,
or a group. Like the singulative, the \ili{Arabic} plurative can be seen as closer to
noun Class and Gender\is{gender}, unlike the Cushitic plurative, which may be, if it is
really a `gender\is{gender}', as Mous put it, closer to the gender\is{gender} found with \ili{Arabic} non-
human plurals.\footnote{See \citet{FassiFehri2016} for examples of non-human
plurals controlling feminine singular \isi{agreement}. My proposal for the Cushitic
plurative is only speculative at this stage, as it is still very poorly
understood.}

\subsubsection{Essential properties}

The most salient properties of the plurative include the following:

\begin{enumerate}
  \item The plurative derivation is a process by which a collective, a singular, or a plural nP is turned into a group unit (or a collection unit).
  \item It is morphologically marked by the same feminine suffix, on the controller and/or the target.
  \item Syntactically, it takes part in feminine singular \isi{agreement}.
  \item When the plurative marked nP participates in (or controls) normal plural \isi{agreement}, it `looses' its group meaning.
  \item Semantically, it expresses a plurality, or more precisely a `perspective' on plurality. It controls reciprocity, or plural predication, etc.
  \item The plurative is potentially countable, and can undergo dualization or pluralization in relevant contexts (see \citealt{FassiFehri2016} for detail).
  \item The plurative is in complementary distribution with both \isi{Number} and other Gen (including the singulative).
\end{enumerate}
%
The group or collection unit is formed from various classes of nouns, only few
of which are exemplified here.

\subsubsection{Professional groups, corporations, property sharing, or collections units}

Standard \ili{Arabic} uses \xv{-at}, and Moroccan \ili{Arabic} \xv{–a} as exponents:

\begin{exe}
  \ex\label{ex:fassi:20} Standard \ili{Arabic}\\
  \xv[carpenter]{najjaar} → \xv[the corps of carpenters]{najjaar-at}
  \ex\label{ex:fassi:21} Moroccan \ili{Arabic}\\
  \xv[thief]{\v{s}effaar} → \xv[thieves (as a group)]{\v{s}effaar-a}
  \ex\label{ex:fassi:22} Moroccan \ili{Arabic}\\
  \xv{jebl-ii} mountain-sing \gl{an inhabitant of the mountain} → \xv[inhabitants of the mountain]{jbal-a}
\end{exe}
%
Groups based on property sharing are normally derived from adjectives or
participles:

\begin{exe}
  \ex\label{ex:fassi:23}
  \begin{xlist}
    \ex\label{ex:fassi:23a} \xv[unbeliever]{kaafir} → \xv[unbelievers (as a group)]{kafar-at}
    \ex\label{ex:fassi:23b} \xv[magician]{saa\d{h}ir} → \xv[magicians (as a group)]{sa\d{h}ar-at}
  \end{xlist}
\end{exe}
%
With feminine singular \isi{agreement}, pluratives behave more like `kind/collective'
nouns when the latter are read as collection units:

\begin{exe}
  \ex\label{ex:fassi:24}
  \begin{xlist}
    \ex\label{ex:fassi:24a}
    \gll al-fursu wa-r-rum-u \v{s}tarak-at-aa fii \d{h}arb-in \d{d}idda l-ʕarabi\\
    the-Persians and-the-Romans participated-\textsc{fem-dual} in war-\textsc{gen} against the-Arabs\\
    \glt `Persians and Romans participated together (as a group) in a war against Arabs.'\\
    \ex\label{ex:fassi:24b}
    \gll al-fursu wa-r-rumu \v{s}tarakuu fii \d{h}arb-in \d{d}idda l-ʕarabi\\
    the-Persians and-the-Romans participated-\textsc{pl.masc} in war-\textsc{gen} against the-Arabs\\
    \glt `Persians and Romans participated together in a war against Arabs.'\\
  \end{xlist}
\end{exe}
%
Likewise, pluratives can control a dual (or a plural) target:

\begin{exe}
  \ex\label{ex:fassi:25}
  \gll al-muʕtazil-at-u wa-l-ʔa\v{s}ʕariyy-at-u tawa\d{h}\d{h}ad-at-aa fii haa\b{d}aa\\
  the-Mutazilite-\textsc{fem-nom} and-the-Asharite-\textsc{fem-nom} unified-\textsc{fem-dual} in this\\
  \glt `Mutazilites and Asharites have unified (their view) on this.'\\
\end{exe}
%
The dualization of the plurative suggests that pluratives are potentially
countable.

Note that simple collective nouns, plurative nPs/DPs can either trigger a plurative \isi{agreement}, as in \REF{ex:fassi:8} above, or `normal' plural \isi{agreement} as in \REF{ex:fassi:26}:

\begin{exe}
  \ex\label{ex:fassi:26}
  \gll s-saakinat-u \d{h}tajj-uu\\
  the-inhabitant-\textsc{fem} protested-\textsc{pl.masc}\\
  \glt `The inhabitants protested.'\\
\end{exe}
%
This `hybridity' in \isi{agreement} points to a duality in behavior of the plurative
DP, being denoting either a group, as in \REF{ex:fassi:8}, or a sum, as in
\REF{ex:fassi:26}; see \citet{FassiFehri2012,FassiFehri2016} for detail.

\subsubsection{The ``hybrid'' plurative}

The plurative then appears to be neither a pure Gen, nor a pure \isi{Num} (as in the
Mous/Corbett dispute), but rather a sort of hybrid complex of both:

(a) It is not (a low) Gen, since it cannot be interpreted semantically on the
scale of sex;

(b) Unlike Gen in other contexts, the plurative Gen feature is not compatible
with variation in \isi{Num} values (being invariably in the form of the feminine
singular), as illustrated by the contrast in interpretation above.

Another important property is that the plurative is a \textit{syntactic
plurality}, rather than a singularity. For example, it controls syntactic
reciprocity:

\begin{exe}
  \ex\label{ex:fassi:27}
  \gll \v{s}-\v{s}iiʕ-at-u t-antaqidu baʕ\d{d}-a-haa baʕ\d{d}-an\\
  the-Shiite-\textsc{fem-nom} \textsc{fem}-criticize some-her some-\textsc{acc-her}\\
  \glt `The Shiites criticize each other.'\\
\end{exe}

It is used with plural predicates, unlike singulars:

\begin{exe}
  \ex\label{ex:fassi:28}
  \gll taka\b{t}\b{t}al-at \v{s}-\v{s}iiʕatu \d{d}idda daai\v{s}-a\\
  united-\textsc{fem} the-Shiites against Daesh-\textsc{acc}\\
  \glt `The Shiites made a coalition against ISIS.'\\
\end{exe}

But note also that the hybridity of the plurative comes from the fact that it
can be treated as a singular. For example, the dual used in the construction
\REF{ex:fassi:25} above counts the two groups.

Finally, with respect to its semantics, the hybridity of the plurative is
confirmed by the fact that it shares the semantics of groups (or ``collective''
nouns), as described e.g. by \citet{Barker1992}, typically their twofold
potential of being atoms/individuals or sums/sets, as reflected by \isi{agreement}
alternations. See also \citet{Pearson2011}. But its hybridity is even stronger
than normal group since it appears to be both a plurality (at some low layer)
and a singularity (at a higher layer), as reflected by its structure given
below in \REF{ex:fassi:29}; see \citet{FassiFehri2016} for more detail and
references.

\subsection{Structure of the ``perspectivizing'' Gen}

Various options for the structure of pluratives are explored there, but shown
to be inadequate. The following structure is motivated by various
considerations, taking into account the fact that pluratives are collection
units formed in syntax (or ``particulars'' in the perspective of the speaker),
rather than normal plurals (or simple atomic groups). For the sake of
illustration, I propose then that the structure of the DP in \REF{ex:fassi:8}
is as in \REF{ex:fassi:29}:

\begin{exe}
  \ex\label{ex:fassi:29}
  \begin{forest} baseline  
    [GroupP {(= GenP)}
      [Group[\xv{-at}]]
      [NumP
        [Pl[\xv{\sout{-at}}]]
        [nP[\xv{saakin-}]]
      ]
    ]
  \end{forest}
\end{exe}

The structure represents the view that a plurative is formed as a plural of a
specific sort first, then \textit{perspectivized} as a unit (or group) through
Gen, assuming that it is Gen which provides the perspectivization of plurality,
then Gen (or Group) is placed higher, to ``scope over'' Plural, or
\isi{Num}.\footnote{For concreteness sake, I assume that \xv{-at} is placed first in
the \isi{Num} position, and then moves higher to Group/Gen. N also moves there, and
then higher to D as in the usual N-to-D movement
\citep[see][]{Longobardi2001,FassiFehri1993}.}

\section{Gender layers and architecture}\label{sec:fassi:layers}

To account for the various meanings of the feminine (or Gender), I depart from
the view that Gen is confined to a dedicated syntactic position, be it GenP
\citep[as in][]{Picallo2008}, or nP (as in \citealt{Kihm2005},
\citealt{Lowenstamm2008}, or \citealt{Kramer2014Gender}, among others), and it is
interpreted as basically male/female \citep{Percus2011}. Gen is rather
distributed over the various layers of the nP/DP, in the spirit of
\citet{SteriopoloEtAl2010,Pesetsky2013}, or \citet{Ritter1993}, and
even higher in the \isi{CP}, or SAP\@. Gen and its meanings then turn out to be
essentially \textit{constructional}, contra lexicalist or natural views.
Furthermore, at least \textit{five distinct layers} (or sources) of Gen are
postulated and motivated in the grammatical nP/DP architecture: (a) conceptual
Gen; (b) n Gen; (c) Cl Gen; (d) \isi{Num} Gen; (e) D/C Gen, or even higher, SAP Gen.

\subsection{Conceptual and n Gender}

Consider first cases of nominalized abstract feminine nouns, compared to their
(gendered) bases:\is{gender}

\begin{exe}
  \ex\label{ex:fassi:30}
  \begin{xlist}
    \ex\label{ex:fassi:30a} \xv[father]{ʔab} → \xv[fatherhood]{ʔubuww-at}
    \ex\label{ex:fassi:30b} \xv[mother]{ʔumm} → \xv[motherhood]{ʔumuum-at}
    \ex\label{ex:fassi:30c} \xv[man]{rajul} → \xv[manliness]{rujuul-at}
  \end{xlist}
  \ex\label{ex:fassi:31}
  \begin{xlist}
    \ex\label{ex:fassi:31a} \xv[paternal uncle]{ʕamm} → \xv[paternal aunt]{ʕamm-at} → \xv[paternal auntness or uncleness]{ʕumuum-at}
    \ex\label{ex:fassi:31b} \xv[maternal uncle]{xaal} → \xv[maternal aunt]{xaal-at} → \xv[maternal auntness]{xuʔuul-at}
  \end{xlist}
\end{exe}

The gender\is{gender} complexity of these forms point to the existence of (at least) two
distinct layers of Gen, needed for interpretation: one is \textit{conceptually-based}
(i.e. a `father' is masculine, a `mother' is feminine, a `maternal uncle
or aunt' has two genders, and the same is true for a `paternal uncle or
aunt').\footnote{Note that \ili{Arabic} kinship terms are more specific than those of
Germanic or \ili{Romance}, in that there is no such a ``vague'' kinship relationship
like `cousin', `uncle', `aunt', etc. Rather, each of these relationships
in \ili{Arabic} must indicate whether it connects to the mother or the father (e.g.
cousin from the mother, or aunt from the father), as the examples and their
translations illustrate.} Call this ``lower'' gender\is{gender} \textit{conceptual Gen}. The
second grammatical upper gender\is{gender} (marked by \xv{-at}) forms an \textit{n}
(entity or concept) from a property. Call it \textit{n Gen}. The need for
conceptual Gen has been pointed out by e.g. \citet{Kopcke2010}, who have argued
that ``\dots{} much of the \ili{German} grammatical gender\is{gender} is \textit{conceptually}
motivated in that certain \isi{semantic} fields tend to be marked by some specific
gender\is{gender} [italics mine; FF]'', despite ``the widespread view among autonomist
grammarians that [\dots{}] gender\is{gender} in \ili{German} is most purely grammatical [totally
arbitrary] category, not motivated in any way by conceptual factors'' (172). 
Various other motivations have also been more recently brought in by
\citet{McConnellGinet2015} for the equivalent ``notional'' gender\is{gender}, or
\citet{Mithun2015} for ``cultural'' gender\is{gender}, among others.

\subsubsection{Various conceptual sources of female/male pairs}

Sources of gender\is{gender} may be conceptually or ``culturally'' different (even in the
same language), and derivations from these sources may lead to various results.
Consider the following pairs of feminization:

\begin{exe}
  \ex\label{ex:fassi:32} \xv[man]{rajul} → \xv[woman]{mraʔ-at}
  \ex\label{ex:fassi:33} \xv[he-cat]{qi\d{t}\d{t}} → \xv[she-cat]{qi\d{t}\d{t}-at}
  \ex\label{ex:fassi:34} \xv[man, male person]{mruʔ} → \xv[woman]{mraʔ-at}
  \ex\label{ex:fassi:35} \xv[man]{rajul} → \xv[a property of a strong woman]{rajul-at} (an adjective)
\end{exe}
%
The first pair in \REF{ex:fassi:32} is conceptually/semantically the minimal
pair to name the female/male human pair, although the members of the pair do
not share any common morpho-phonological base. In contrast, \xv{mraʔ-at} and
\xv{mruʔ} in \REF{ex:fassi:34} are grammatically and morpho-phonologically
related, although they are not the genuine counterparts of `man' and `woman' in
English; the first member means `male person' rather than `man'. As for the
\REF{ex:fassi:35} pair, it shows that although \xv{rajul} can be made feminine,
the only feminine it can form is a manner adjective, not a noun.

Note that contrary to what happens in the examples (\ref{ex:fassi:30a} \&
\ref{ex:fassi:30b}) above, where the feminine affix \xv{-at} can be taken as a
\textit{categorizer}, or part of the categorizing \textit{n} process, the
morpheme in the examples (\ref{ex:fassi:32}--\ref{ex:fassi:34}) can hardly be
taken as a nominalizer. First, the `masculine' base is already nominal or
adjectival (or coerced to be so) as the contrast between \REF{ex:fassi:34} and
\REF{ex:fassi:35} suggests. If this is so, then the base of the derivation may
be seen as providing a conceptual ground for forming a feminine (or masculine)
of an entity or a property. If gender\is{gender} is only taken as a feature of the
category \textit{n}, and no distinction is made between the contribution of the
conceptual (or root) gender\is{gender} and that of the functional gender\is{gender}, it is hard to
see how such contrasts can be accounted for.

\subsubsection{The placement of n Gen}

Let assume that the suffix \xv{-at} in \REF{ex:fassi:30} is a
\textit{categorizer} (n Gen), forming the abstract noun. Let us also take it to
be a \textit{head} feature of the category \textit{n}, by virtue of
contributing to its abstract (rather than concrete) nouniness, in addition to
is interpretation as naming a property (rather than an object). Such a
`category change' property is clearer in cases of (abstract) property nouns
deriving from adjectives, as has been seen in examples \REF{ex:fassi:12} above.
I assume that Gen there is interpretable, contributing to name an abstract
property.

As for Gen in cases like \REF{ex:fassi:33}, it may be in a different position.
It is not a head categorizer, since the derivation operates on what is already
a noun, and the affix does not operate any ``category change'' or  ``mutation''
here. It is rather a \textit{modifier} feature.

Other cases may be included in the categorizing case. Consider the following
pair:

\ea \label{ex:fassi:36}
\textit{maktab} ‘office’ → \textit{maktab-at} ‘library’
\z
%
Although a (formal) derivational relation can be established between the two
nouns, the semantics of the second member is in no way compositional (with
respect to the first member). We can account for these properties by
postulating that Gen is a categorizing head feature in this case, since its
contributes to shaping the content of the noun.

\subsection{Cl Gen and Num Gen}

The singulative/individuative Gen investigated above instantiates a
classifier/Class gender\is{gender}, as explained there. The plurative gender\is{gender}, on the other
hand, instantiates the case of \isi{Number} that is ``gendered'', or \isi{Num} Gen, as an
expression of perspectivization, as explained earlier.

\section{Size and evaluative modification}\label{sec:fassi:modification}

\subsection{Diminuitive Gen}

Diminutive and augmentative \ili{Arabic} morphemes behave mostly as modifiers,
denoting either decrease/increase in size, or expressive/evaluative meanings.
They occasionally behave as heads (and individualizers), with a portioning out
that produces countable units, as has been established for some languages, but
only when they are gendered in \ili{Arabic}.\footnote{See \citet{Wiltschko2008%
,deBelder2008%
,Mathieu2012%
,Steriopolo2013%
}, among others.}
It is then the feminine suffix that can be held responsible for this potential
meaning.

Three different meanings of the morpheme can then be distinguished, and
represented structurally: (a) ClP (or DivP in Borer's sense), (b) SizeP (DimP
or AugmentP, as in \citealt{Cinque2014}), and (c) EvalP for the evaluative
(endearing, pejorative, etc.). The following example from Moroccan \ili{Arabic}
instantiates the multiple role of diminutive Gen:

\begin{exe}
  \ex\label{ex:fassi:37} Moroccan \ili{Arabic}\\
  \xv[buttermilk]{lben} → \xv{lbeyy-in} buttermilk-\textsc{dim} \gl{a small quantity of buttermilk} → \xv{lbin-a} buttermilk.\textsc{dim-fem}
  \begin{xlist}
    \ex\label{ex:fassi:37a} intensive: `a very small quantity of buttermilk'
    \ex\label{ex:fassi:37b} evaluative: `an appreciated small quantity of buttermilk' 
    \ex\label{ex:fassi:37c} individuative: `a discrete small portion of buttermilk'
  \end{xlist}
\end{exe}

Two distinct structures can be proposed for the intensive (modifier) and the
individualizing (head) readings of \xv{lbin-a}, respectively:\footnote{A
reviewer wonders whether there are two morphemes involved here (\xv{-i} as
diminutive, and \xv{–a} as feminine), or just one `feminine' \xv{-a}, which can
be used as diminutive. The first option is motivated by the fact that the two
morphologies distribute separately, the diminutive being regularly internal to
the stem, whereas the evaluative is regularly external to the stem. The
realizations of the diminutive as \xv{-y-} or \xv{-i-} are morpho-phonologically conditioned, being a glide or a short vowel, depending on
whether the syllable is open or closed. Moreover, there is no independent
evidence that the two morphemes are fused.}

\begin{exe}
  \ex\label{ex:fassi:38}
  \begin{forest} baseline
    [nP\textsubscript{3}, name=np3
      [Intens[\xv{-a}]]
      [nP\textsubscript{2}
        [Dim,calign=first[\xv{lbiyn},name=lbiyn]]
        [nP\textsubscript{1},name=np1]
      ]
    ]
    \draw (lbiyn) -- (np1);
    \node [right=3em of np3] {(intensive modifier)};
  \end{forest}

  \ex\label{ex:fassi:39}
  \begin{forest} baseline
    [ClP,name=clp
      [Cl[\xv{-a}]]
      [nP\textsubscript{2}
        [Dim [\xv{lbiyn},name=lbiyn]]
        [nP\textsubscript{1},name=np1]
      ]
    ]
    \draw (lbiyn) -- (np1);
    \node [right=3em of clp] {(head individualizer)};
  \end{forest}
\end{exe}

\subsection{Augmentative Gen}

Augmentatives can get intensive and evaluative readings through augmentative
morphemes and Gender\is{gender}. I can think of no case where the augmentative is an
individualizing head. In \REF{ex:fassi:40}, a participle adjective undergoes
both augmentative and Gender\is{gender} affixation, to yield either an intensive reading
or an evaluative:

\begin{exe}
  \ex\label{ex:fassi:40} \xv[traveler]{raa\d{h}il} → \xv{ra\d{h}\d{h}aal} (traveler + augmentative) \gl{big traveler} → \xv{ra\d{h}\d{h}aal-at} traveler + augmentative + \textsc{fem} \gl{famous big traveler} % TODO (Alec): check with additional wording in the gloss
\end{exe}

\subsection{Evaluative Gen}

In the ``appreciative'' diminutive in \REF{ex:fassi:37}, I assume that Eval is
placed inside the DP (as a sort of degree phrase), and interpreted in DP:

\begin{exe} 
  \ex\label{ex:fassi:41}
  \begin{forest} baseline
    [DP,name=DP
      [D[\{\dots{}; \textsc{+end}\}]]
      [nP\textsubscript{4}
        [\dots]
        [nP\textsubscript{3}
          [Eval[\{\xv{a}; \textsc{-end}\}]]
          [nP\textsubscript{2}
            [Dim [\xv{lbiyn},name=lbiyn]]
            [nP\textsubscript{1},name=np1]
          ]
        ]
      ]
    ]
    \draw (lbiyn) -- (np1);
    \node[right=3em of DP] {(diminutive modifier)};
  \end{forest}\\
  \noindent(\textsc{end} = endearing; - for uninterpretable, + for interpretable)
\end{exe}

For the sake of simplicity, I leave aside the details of the granularity of
Eval, and the issue of whether more cartography needs to be involved
here.\footnote{\citet[8; Table 1]{Cinque2014} proposes a cartographic
hierarchization of expressives, as in (i):

  \begin{exe}
    \exi{(i)}\label{ex:fassi:fn11} augmentative > pejorative > diminuitive > endearment
  \end{exe}

With respect to such a hierarchization, \ili{Arabic} seems to go in inverse order,
given that EndP appears higher than both both DimP and AugP. I have no
explanation at this point for this reversal. Further research is needed to
clarify the nature of such variation.}

As for the augmentative evaluative in \REF{ex:fassi:40}, I assume that its Eval
here is similar to the diminutive Eval, and should be represented in a strictly
parallel way, inside DP:

\begin{exe}
  \ex\label{ex:fassi:42}
  \begin{forest}
    [DP, name=DP
      [D\\{[\dots; \textsc{eval}\textsubscript{i}]}]
      [nP\textsubscript{4}
        [\dots]
        [nP\textsubscript{3}
          [Eval\textsuperscript{u}[{[\xv{-at}]}]]
          [nP\textsubscript{2}
            [Aug [\xv{ra\d{h}\d{h}aal-},name=rahhal]]
            [nP\textsubscript{1},name=np1]
          ]
        ]
      ]
    ]
    \node[right=3em of DP] {(augmentative modifier)};
    \draw(rahhal) -- (np1);
  \end{forest}
\end{exe}

\subsection{`Performative' expressive Gen}

Previous evaluative Gen occurred in contexts where a quantitative size
modification can obtain, with an internal DP source. I turn here to cases where
Gen lacks both such a quantitative option, and internal DP interpretive source.
These cases are unique, in that they are devoted to qualitative evaluation or
expressivity, with specific external characteristics.

Consider e.g. the following constructions (\textsc{end} for
endearing):\footnote{Note that the third person pronoun \xv{–h} is used here
for the speaker (or `first' person), as is usually the case in some European
language styles.}

\begin{exe}
  \ex\label{ex:fassi:43}
  \gll yaa ʔab-at-i!\\
  oh father-\textsc{end}-mine\\
  \glt `Oh my beloved father!'\\
  \ex\label{ex:fassi:44} % TODO (Alec): Footnote suggests -h is 1P. Not in the original. What to do?
  \gll waa ʔumm-at-aa-h!\\
  oh mother-\textsc{end-exclam}-his\\
  \glt `Oh my beloved mother!'\\
  \ex\label{ex:fassi:45}
  \begin{xlist}
    \ex\label{ex:fassi:45a}
    \gll yaa wayl-at-i!\\
    oh misery-distress-mine\\
    \glt `Oh my terrible woe!'\\
    \ex\label{ex:fassi:45b} Moroccan \ili{Arabic}\\
    \gll waa saʕd-at-i!\\
    oh chance-\textsc{end}-mine\\
    \glt `Oh my great chance!'\\
  \end{xlist}
\end{exe}

In none of these expressions, can the `feminine' noun (or morpheme) be
associated with a female, a singulative, or an intensive interpretation. There
is obviously no `female father' interpretation in \REF{ex:fassi:43}, neither a
`female mother' in \REF{ex:fassi:44}; there is no `individuative' involved in
\REF{ex:fassi:45}, and no `intensive' anywhere. The only available ``meaning''
here is an expression of the speaker's emotional feelings (endearment,
distress, etc.). What is even more appealing is that these `feminine' forms
cannot be used outside these illocutionary marked contexts. It is also striking
that the existence of this rather original expression and meaning of gender\is{gender} has
hardly been acknowledged in the \ili{Arabic} or orientalist literature, and it did
not generate any preliminary account, as far as I can
tell.\footnote{\citet[II, 87--88]{Wright1971} did mention the constructions
in \REF{ex:fassi:43} and \REF{ex:fassi:44} in the context of expressives, but
he did not indicate what is the content of \xv{-at} there, describing them as
'peculiar forms'! Likewise, \citet[601]{HämeenAnttila2000} qualifies the case
of \REF{ex:fassi:43} as 'obscure'! In the early \ili{Arabic} grammatical
tradition, the morpheme \xv{-at} is seen as fulfilling a morpho-phonological
role, i.e. ``replacing'' the possessive mark (\xv{-y} `mine'), or ``compensating''
(\xv{taʕwii}) its absence. 
}

There is evidence that these evaluatives are clause-dependent, or interpreted
in the \isi{CP} (or some level higher), unlike those examined above (which are DP
dependent). First, contrary to the previous evaluatives, the constructions
under investigation do not occur as normal DPs in contexts where the sentence
force is not crucial for interpretation, as in e.g. declarative clauses:

\begin{exe}
  \ex\label{ex:fassi:46}
  \begin{xlist}
    \ex[]{\label{ex:fassi:46a}
      \gll najaa ʔab-ii mina l-\.{g}araq-i\\
      escaped father-mine from the-drowning-\textsc{gen}\\
      \glt `My father escaped from drowning.'\\
    }
    \ex[*]{\label{ex:fassi:46b}
      \gll najaa ʔab-at-i mina l-\.{g}araq-i\\
      escaped father-\textsc{end}-mine from the-drowning-\textsc{gen}\\
    }
    \ex[*]{\label{ex:fassi:46c}
      \gll najat ʔumm-at-aa-hu mina l-\.{g}araq-i\\
      escaped mother-\textsc{end-exclam}-his from the-drowning-\textsc{gen}\\
    }
  \end{xlist}
\end{exe}

The contrast between the ill-formed\-ness of (\ref{ex:fassi:46b} \&
\ref{ex:fassi:46c}) and the well-formed\-ness of \REF{ex:fassi:43} and
\REF{ex:fassi:44} point to a DP/\isi{CP} divide in the syntax/semantics of
evaluatives. In the latter case, evaluatives can only be interpreted outside
the DP, in a position higher in the \isi{CP}, or even higher and outside the \isi{CP}, in a
clearly performative context (the vocative here).

What are the bases and motivations of such a divide, and how are outer
evaluatives anchored in the \isi{CP}? For the sake of concreteness, let us assume
some cartographic representation of the \isi{CP} a la Cinque/Rizzi/Moro, enriched
with Speech Act role cartography (SAP) a la \citet{Hill2014}, among others. In
the expanded \isi{CP} cartography, vocatives tend to be associated with a high
functional projection located in the \isi{CP}, possibly above \isi{Force} \citep[as
in][]{Moro2003}. Hill proposed that they be associated with a SAP projected
above (and outside) the \isi{CP}, in line with \citet{SpeasEtAl2003}. Moreover, the
structure of vocatives is sensitive to the speaker/hearer
hierarchization.\footnote{Thus, \citet[207]{Hill2014} distinguishes among
speech acts between \textit{speaker-oriented clause types} like exclamations
(which convey the speaker's point of view about situations), and \textit
{hearer-oriented} ones like direct addresses (which convey the speaker's
manipulation of the interlocutor). Since the structural placement of the
speaker and the hearer is distinct, it is the lower segment of the SAP which is
dedicated to (the \isi{merger} of) the vocative. However, the existence of the upper
segment in the SAP of the vocative is not superfluous, because the speaker's
field may interact with the hearer's (direct address) field in speaker-oriented
vocatives and other vocative contexts. See \citet{Hill2014} for detail, and
relevant references cited there.}

There are reasons to take the gender\is{gender} in the vocative phrase examined to be
speaker-oriented, and interpreted in the speaker field. First, the evaluative
gender\is{gender} in \REF{ex:fassi:43} is exclusively interpreted as a modifier of (the
subjectivity of) the speaker. It cannot be associated with the hearer, as the
ungrammaticality of \REF{ex:fassi:47} indicates:\footnote{A reviewer wonders
what is the status of a parallel of \REF{ex:fassi:44} in this case, i.e. the
following construction:

\begin{exe}
  \exi{(i)}[*]{
    \gll yaa ʔumm-at-aa-k!\\
    oh mother-\textsc{end-exclam}-your\\
    \glt Intended: 'Oh your beloved mother!'\\
  }
\end{exe}

Its ungrammaticality indicates that the same observations can be extended to
`mother' as well (or, in fact, to any other relational noun).}

\begin{exe}
  \ex[*]{\label{ex:fassi:47}
    \gll yaa ʔab-at-aa-k!\\
    oh father-\textsc{end-exclam}-your\\
    \glt Intended: `Oh your beloved father!'\\ 
  }
\end{exe}

What the judgement indicates is that the gender\is{gender} of VocP can only probe for the
higher SA role, the \isi{Speaker} (which c-commands it), not the lower SA hearer.
Second, note that the gender\is{gender} on the imperative verb (\isi{agreeing} with the second
person hearer) is exclusively dedicated to the hearer in the lower segment
(which also c-commands it), as the following construction
illustrates:\footnote{In the embedded imperative inside the vocative, the verb
agrees in \isi{Num} and Gen with the (hidden) addressee, and only covertly in
2\textsuperscript{nd} Pers:

\begin{exe}
  \exi{(i)}
  \gll \d{t}maʔinn-ii!\\
  reassure-\textsc{fem}\\
  \glt `Be reassured!' (for a single female)\\
  \exi{(ii)}
  \gll \d{t}maʔinn-uu!\\
  reassure-\textsc{pl}\\
  \glt `Be reassured!' (for a plurality of males)\\
\end{exe}

  These patterns can be taken as forms of allocutary \isi{agreement} \citep[as in][]{Miyagawa2012}. See  \citet{FassiFehri2016} for other details.}

\begin{exe}
  \ex\label{ex:fassi:48}
  \gll yaa ʔumm-at-aa-hu \d{t}maʔinn-ii!\\
  oh mother-\textsc{end-exclam}-his reassure-\textsc{fem}\\
  \glt `Oh beloved mother, be reassured!'\\
\end{exe}

Two genders are involved here, the endearing evaluative \xv{-at} on the
vocative DP expression, and the feminine \xv{-ii} on the imperative verb. In
both cases, the gender\is{gender} realized can be assumed to be ``displaced'', or
uninterpretable in situ. The lower gender\is{gender} on the verb is interpretable higher,
its goal being the 2\textsuperscript{nd} \isi{Person} of the SA hearer. As for Gen on
the vocative DP, it is neither interpretable in the DP, as already established
through the \REF{ex:fassi:46} contrasts, nor by the lower SA hearer. It is only
interpretable higher in the SA cartography, in the speaker ``field'' (as part of
the speaker subjectivity). These contrasts give credence to the speaker vs.
hearer differentiation in SAPs, as postulated by \citet{Hill2014}, among
others. I tentatively represent the relevant part of the structure of
\REF{ex:fassi:43} as follows (s for speaker, h for hearer):

\begin{exe}
\small
  \ex\label{ex:fassi:49}
  \begin{forest} baseline
    [SAsP
      [\isi{Speaker}\\ {[1; end\textsubscript{i}]}]
      [SAs'
        [SAs]
        [SAhP
          [VocP
            [Voc [\xv{yaa}]]
            [DP
              [D,name=D]
              [PossP
                [{[}\textsubscript{D}{[}\textsubscript{Poss}{[}\textsubscript{n}{[}\textsubscript{n}{[}\textit{ʔab}-{]} \textsubscript{n} {[}-\textit{at}{]} \textsubscript{End}\textsuperscript{u}{]}\textsubscript{n} {[}-\textit{i}{]}\textsubscript{Poss}{]}\textsubscript{D}{]]]]},name=formula
            ]
          ]
        ]
      ]
    ]
        [\isi{CP}]
      ]
    ]
    \draw (formula) -- (D);
  \end{forest}
\end{exe}

I assume that the head noun \xv{ʔab} here has moved to D, after having
integrated the endearing `feminine', and the cliticized \isi{possessor}. If the
hidden \isi{Speaker} has an interpretable 1Pers feature, and an interpretable End
feature, then both are targeted in the probe-goal (or indexing) relationship
needed for interpretation.

As for \REF{ex:fassi:48}, its structure is as follows:

\ea\label{ex:fassi:50}
\small
\begin{forest}
  [SAsP
    [\isi{Speaker}\\{[}1; end\textsubscript{i}{]}]
    [SAs'
      [SAs]
      [SAhP
        [VocP
          [Voc [\xv{yaa}]]
          [DP
            [D, name=D]
            [PossP
              [{[}\textsubscript{D}{[}\textsubscript{Poss}{[}\textsubscript{n}{[}\textsubscript{n}{[}\xv{ʔumm}-{]}\textsubscript{n} {[}-\xv{at}{]}\textsubscript{End}\textsuperscript{u}{]}\textsubscript{n} {[}-\xv{aa}{]}\textsubscript{VOC}{]}\textsubscript{n} {[}-\xv{h}{]}\textsubscript{Poss}{]}\textsubscript{D}{]]]]},name=formula
        ]
      ]
    ]
  ]
        [\isi{CP}
          [{[[}\xv{ṭmaʔinn}{]} \textsubscript{V} {[}-\xv{ii}{]}\textsubscript{FEM}{]}\textsubscript{V}]
        ]
      ]
    ]
  ]
  \draw (formula) -- (D);
\end{forest}
\z

Note that the endearing \isi{agreement} involves only coindexation in person (for the
speaker or utterer). There is no formal gender\is{gender} \isi{agreement} here, compared to the
\isi{agreement} found with the singulative or the plurative (see
\citealt{FassiFehri2016} for detail).

\section{Cross-linguistic extensions}\label{sec:fassi:extensions}

This section does not intend to describe the vast number of gendered languages
that instantiate similar patterns and correlations, but only give some examples
for the sake of identification and comparison. The list includes \ili{Berber}
(Afroasiatic), \ili{Hebrew} (\ili{Semitic}), and \ili{Romance}.

\subsection{Berber}

\ili{Berber} has a two-gender\is{gender} opposition, expressing natural gender\is{gender}, abstracts,
units, size, expressive evaluation, and it interacts with ``enunciation''
\citep{Mettouchi1999}. The morpheme \xv{-t} (occurring as a reduplicating
discontinuous morpheme, or ``circumfix'') provides the formal means to express
these various meanings which compete for the same slot on the noun, without any
possibility of being added to each other (being in ``complementary
distribution''; \citealt{Kossmann2014}), while the augmentative is expressed via
a form of (uncommon) ``substractive'' morphology \citep{Grandi2015}. In the
descriptions provided, there are systematic relationships between gender\is{gender} forms
and meaning forms, e.g. between feminine and diminutive, or between masculine
and augmentative. There are also expressions of endearment, contempt,  ``in
relation to the speaker'', etc.

First, \xv{-t} expresses \emph{sex} for animates:

\begin{exe}
  \ex\label{ex:fassi:51} Kabyle \citep{Mettouchi1999}
  \begin{xlist}
    \ex\label{ex:fassi:51a} \xv[donkey]{agyul} → \xv[she-donkey]{t-agyul-t}
    \ex\label{ex:fassi:51b} \xv[veal]{aganduz} → \xv[heifer]{t-aganduz}
  \end{xlist}
  \ex\label{ex:fassi:52} Ayt Seghrouchen \citep{Kossmann2014}
  \begin{xlist}
    \ex\label{ex:fassi:52a} \xv[male child]{arba} → \xv[female child]{t-arba-t}
    \ex\label{ex:fassi:52b} \xv[boy]{afrux} → \xv[girl]{t-afrux-t}
    \ex\label{ex:fassi:52c} \xv[ox]{afunas} → \xv[cow]{t-afunas-t} % TODO (Alec): add citations
  \end{xlist}
\end{exe}
%
Second, \emph{unity} nouns are formed by the feminine:
\newpage

\begin{exe}
  \ex\label{ex:fassi:53}
  \begin{xlist}
    \ex\label{ex:fassi:53a} \xv[mosquitoes]{nnamus} → \xv[a single mosquito]{\b{t}anamus\b{t}}
    \ex\label{ex:fassi:53b} \xv[apricots]{l-ma\v{s}ma\v{s}} → \xv[a single apricot]{\b{t}am\v{s}ma\v{s}\b{t}}
  \end{xlist}
\end{exe}
%
Third, a \emph{quantitative diminutive} is expressed by the feminine:


\begin{exe}
  \ex\label{ex:fassi:54}
  \begin{xlist}
    \ex\label{ex:fassi:54a} \xv[hand]{afus} → \xv[little hand]{\b{t}-fus-tt}; variant: \xv{afus} → \xv{t-afus-t}
    \ex\label{ex:fassi:54b} \xv[small lizard]{t-aherdan-t} (also \gl{female lizard})
    \ex\label{ex:fassi:54c} \xv[small fish]{t-aslem-t} \citep{Kossmann2014,Grandi2015}
    \ex\label{ex:fassi:54d} \xv[chair]{lkursi} → \xv[little chair]{\b{t}akursitt}
    \ex\label{ex:fassi:54e} \xv[owl]{muka} → \xv[little owl]{\b{t}amukatt} \citep{Kossmann2014}
  \end{xlist}
\end{exe}
%
Fourth, \textit{abstract} nouns can be formed as feminine, expressing
qualities, professions, names of languages, etc.:

\begin{exe}
  \ex\label{ex:fassi:55}
  \begin{xlist}
    \ex\label{ex:fassi:55a} \xv{aryaz} (m) \gl{man} → \xv[manliness (courage)]{\b{t}aryaz\b{t}}
    \ex\label{ex:fassi:55b} \xv{aslma\b{t}i} (m) \gl{fisherman} → \xv{\b{t}aslma\b{t}i\b{t}} (f) \gl{profession of fisherman}
    \ex\label{ex:fassi:55c} \xv[Berber]{a\v{s}əl\d{h}i} → \xv[Berber language]{\b{t}a\v{s}əl\d{h}i\b{t}} \citep{Kossmann2014}
  \end{xlist}
\end{exe}
% 
As for \textit{augmentative}, it is said to be expressed by the `masculine':

\begin{exe}
  \ex\label{ex:fassi:56}
  \begin{xlist}
    \ex\label{ex:fassi:56a} \xv[garden]{t-a-bhir-t} → \xv[big garden]{a-bhir}
    \ex\label{ex:fassi:56b} \xv[thigh]{\b{t}am\d{s}a\d{t}\d{t}} → \xv[very big thigh]{am\d{s}a\d{d}} \citep{Kossmann2014}
    \ex\label{ex:fassi:56c} \xv[big owl]{amuka}
  \end{xlist}
\end{exe}

\citet{AbdelMassih1971} observes that ``certain feminine nouns give
augmentatives by a process that is the reverse of diminutive formation'', and
hence, only feminine nouns can be augmentativized (\xv{-t} if present is then
`deleted', in ``a typologically unusual instance of subtractive morphology'',
as \textcite[10]{Grandi2015}. As for masculine nouns, they can only be
diminutivized. A triplet of normal, singulative, and augmentative are given in
the following example:

\begin{exe}
  \ex\label{ex:fassi:57} \xv[chickpeas]{l\d{h}um\d{s}} → \xv[one chickpea]{\b{t}a\d{h}um\d{s}tt} → \xv[big individual chickpea]{a\d{h}um\d{s}}
\end{exe}

As for \textit{evaluative} endearment and contempt, \citet[219]{Mettouchi1999}
observes that ``both diminutives and augmentatives can be reinterpreted as
depreciative'', or else appreciative. Hence it is apparently possible to
depreciate/appreciate from the masculine to the feminine, or vice versa, as in
\REF{ex:fassi:58} and \REF{ex:fassi:59}, respectively:

\begin{exe}
  \ex\label{ex:fassi:58} \xv[man]{argaz} → \xv[mannish female]{t-argaz-t}
  \ex\label{ex:fassi:59} \xv[woman]{tamtut} → \xv[a wimp woman]{amtu}
\end{exe}

Endearment is also expressed via the diminutive feminine, as in
\REF{ex:fassi:60}:

\begin{exe}
  \ex\label{ex:fassi:60} \xv{baba} (m) \gl{my father} → \xv{\b{t}ababatt} (f) \gl{little father; endeared father} (\citealt{Kossmann2014}; second translation mine)
\end{exe}

As for the \textit{expressive performative} (in my terms), I have found what
appears to be one of instantiation of it in an example brought up by
\citet{Kossmann2014}, where the feminine establishes a relation (of low age),
in relation to the speaker:

\begin{exe}
  \ex\label{ex:fassi:61} \xv[my paternal uncle]{ʕəmm-i} → \xv[paternal uncle (younger than the speaker)]{\b{t}-aʕəmmi-tt}
\end{exe}

\subsection{Hebrew}


Early Semitic had a common feminine marker \xv{-at}, which was found
 before it split into East and West Semitic (\citealt{Hasselbach2014}, and references cited there).
 When compared to
\ili{Akkadian}, Classical \ili{Arabic}, and \ili{Géez}, \ili{Hebrew} appears to have a short list of
meanings. The feminine suffix \xv{-a} appears to be the most productive,
compared to other morphemes (including \xv{-t} or its variants \xv{-et},
\xv{-at}, \xv{ot}, etc.). Here are some patterns of \isi{semantic} diversity.

Female sex can be expressed by \xv{-a} or \xv{-it}:

\begin{exe}
  \ex\label{ex:fassi:62}
  \begin{xlist}
    \ex\label{ex:fassi:62a} \xv[teacher]{more} → \xv[female teacher]{more-a}
    \ex\label{ex:fassi:62b} \xv[dog]{kélev} → \xv[she-dog]{kalv-a}
  \end{xlist}
  \ex\label{ex:fassi:63} \xv[cook]{tanah} → \xv[female cook]{tanah-it}
\end{exe}
%
The feminine can mark abstracts:

\begin{exe}
  \ex\label{ex:fassi:64} \xv[vengeance]{neqam-a}
\end{exe}
%
It forms singulatives:

\begin{exe}
  \ex\label{ex:fassi:65} \xv[fleet]{oni} → \xv[a ship]{oniyy-a}
\end{exe}
%
The `collective' can be marked by the feminine, and the unit singular unmarked,
just as is the case in the \ili{Arabic} plurative:\footnote{See Hasselbach (ibid.),
among others, and relevant references cited there.}

\begin{exe}
  \ex\label{ex:fassi:66}
  \begin{xlist}
    \ex\label{ex:fassi:66a} \xv[a fish]{daag} → \xv[fish (as a collection)]{dagg-a}
    \ex\label{ex:fassi:66b} \xv[inhabitants as a group; population]{yoseb-et}
  \end{xlist}
\end{exe}

\subsection{Romance}

\citet{DeLaGrasserie1904} notes that gender\is{gender} as a sex appears only very late in
the historical grammatical hierarchical strata associated with gender\is{gender}, in fact
the last one. But languages like \ili{Bantu} has non-hierarchical multiple genders.
In a second stage from this state, there is development of a hierarchical
animate/inamimate opposition, rather than sex. In a third stage, sex is
allotted to nouns, even without reason, although construed by subjectivity, and
interlocution \citep[226--227]{DeLaGrasserie1904}. It is then 
`big/small',
`important/less important', 
`strong/weak' etc., 
or rather an opposition of
`wide, vague, or generic' (for the feminine) and
`specific, precise' for
the masculine. There is also a tendency to feminize nouns in languages that
have no neuter, ``which is in the middle''.

As an illustration, \citet[135]{KahaneEtAl1949} observe that ``\dots{} in the
\ili{Romance} languages the \emph{feminine} form of a noun may have an
\emph{augmentative} value in relation to the corresponding masculine'', e.g.
\xv[large sack]{sacca}, compared to \xv[sack]{saccu}. The augmentative use of
the feminine is further illustrated in a number of \ili{Italian} dialect
constructions, including the following examples \citet[138]{KahaneEtAl1949}:

\begin{exe}
  \ex\label{ex:fassi:67}
  \begin{xlist}
    \ex\label{ex:fassi:67a} \xv[big basket]{kavana} (\xv[basket]{kavan})
    \ex\label{ex:fassi:67b} \xv[large kitchen knife]{kortella} (\xv[knife]{kortello})
    \ex\label{ex:fassi:67c} \xv[large butterfly]{pavela} (\xv[small butterfly]{pavel})
  \end{xlist}
\end{exe}

By gender\is{gender} change, diminutive or intensive are also expressed \citep[139--141]{KahaneEtAl1949}:

\begin{exe}
  \ex\label{ex:fassi:68}
  \begin{xlist}
    \ex\label{ex:fassi:68a} \xv[small frying pan]{padellina} → \xv[very small frying pan]{padellino}
    \ex\label{ex:fassi:68b} \xv[small trumpet]{trombettina} → \xv[very small trumpet]{trombettino}
    \ex\label{ex:fassi:68c} \xv[small bark]{barchina} → \xv[tiny hunting boat]{barchino}
    \ex\label{ex:fassi:68d} \xv[drawer]{cassetta} → \xv[small drawer]{cassetto}
  \end{xlist}
\end{exe}

In a similar vein, \citet{Bergen1980} argues that there are various \isi{semantic}
uses of gender\is{gender} in (dialects of) Spanish, including natural sex, unitization,
small or large size, etc., built on the feminine suffix \xv{-a} \citep[49--50;
53; 56]{Bergen1980}:

\begin{exe}
  \ex\label{ex:fassi:69}
  \begin{xlist}
    \ex\label{ex:fassi:69a} \xv[cat]{gato} → \xv[female cat]{gat-a} (sex)
    \ex\label{ex:fassi:69b} \xv{Rafael} → \xv{Rafael-a} (female proper name)
  \end{xlist}
  \ex\label{ex:fassi:70} \xv[olive tree]{aceituno} → \xv[olive]{aceituna}
  \ex\label{ex:fassi:71} \xv{barco} → \xv[small ship]{barca} (diminuitive)
  \ex\label{ex:fassi:72} \xv{panero} → \xv[large basket]{panera} (augmentative)
\end{exe}

In sum, a gender\is{gender} polysemy can be established across languages, which
corroborates the \ili{Arabic} picture, and which supports the multi-layered approach
adopted here.\footnote{See \citet{FassiFehri2016} for more extensions to
\ili{German}, \ili{Dutch}, Spanish, and more relevant references.}

\section{Semantics-pragmatics, morpho-syntax, and representation}\label{sec:fassi:semprag}

Having established that the Gender\is{gender} functional affix is polysemous, and that its
morpho-syntax is distributed (rather than unique), I first discuss some
preliminary proposals made in the literature to account for regular polysemy
and sense extensions of similar morpho-syntax and semantics. I then postulate a
single representation of the various senses of the affix.

\subsection{Semantics, discourse, and interface with morpho-syntactic peculiarities}

\citet{Grandi2015}, building on previous work by Dressler and Jurafsky in
particular, argue for various \isi{semantic} and pragmatic interpretations formally
dependent on the peculiarities of language-specific evaluative word-formation
strategies (including affixation, gender\is{gender} shift, compounding\is{compound}, reduplication,
etc.). Cross-linguistically, evaluative constructions can express either (a)
descriptive/quantitative or (b) qualitative/expressive evaluation. In the case
of (a), the description relies on real/objective properties (of objects,
persons, actions, etc.), which are measured with respect to a standard/default
value, and seen as a deviation with respect to the norm (culturally or socially
determined). In the case of (b), the evaluative and subjective is concerned
with personal feelings or opinions. For example, \xv{cagnolino} in \ili{Italian} can
objectively describe a small dog, and \xv{cagnone} a big one, in relation to a
standardly sized one, using objective dimensional parameters. But if someone
calls his Great Dane \xv{cagnolino}, she/he would be expressing her/his
affection towards it, or feelings, which depend crucially on pragmatics or
discourse factors. The semantic-formal correlation is often unpredictable, but
there are numerous instances of regular morphological qualitative evaluation
(e.g. Slovak \xv[mother-\textsc{aug}]{mam-isko} expresses a pejorative, whereas \xv[mother-\textsc{dim}]{mam-ička} expresses an affectionate evaluative). See also \citet{Cinque2014}.

\citet{Wierzbicka1989} proposes to consider the evaluative functions as
instantiations of typological or \isi{universal} prototypes, based on \isi{semantic}
primitives: the quantitative \textsc{small/big}, and the qualitative
\textsc{good/bad}. \citet{Jurafsky1996} offers an in-depth view of the polysemy
of diminutives and their \isi{semantic} complexities via a ``radial model'' (inspired
by \citeauthor{Lakoff1987}'s~\citeyear{Lakoff1987} radial category). According
to him, the central (\isi{semantic}) category of the diminutive is \textsc{child}.
Other diminutive senses come about through a process of \textit{\isi{semantic}
change,} which uses various important mechanisms, including the
\textit{creation of metaphors}, \textit{bleaching}, and the
\textit{conventionalization of inference}. Finally, in
\citeauthor{Kortvelyessy2014}'s~\citeyear{Kortvelyessy2014} model of evaluative
formation, the \isi{semantic} pragmatic functions of quantitative and qualitative
evaluation are reflected in the form of two alternative paths of evaluative
formation. The semantics  of evaluation takes evaluative constructions as part
of a continuum of \textsc{quantity} (under or above) the default value, or a
`supercategory' including other categories.\footnote{According to her, the
categories subsumed include Plurality or Aktionsart, with concepts  of
multiplicity, iterativity, distributiveness, attenuation, etc., which are of
quantitative nature. See \citet{Kortvelyessy2014} for detail, and the relevant
references there.}

\subsection{A unique hierarchical representation of Gen polysemy}
\largerpage[2]

In a polysemic analysis of Gen, its multi-layered distributed architecture and
its distributed morphology model concur to provide an integrative view of
regularities, correlations, and patterns found in \ili{Arabic} varieties, and other
languages as well. The variety of meanings and morpho-syntactic features or
categories are interrelated and often regularly interfaced, rather than being
accidental. As regard meanings, it is possible to see Gen as a \isi{semantic}
`supercategory' or \textit{hyperonym} of \isi{Quantity} (or Quality), with a
hierarchization (or a tree geometry), in which a \textit{hyponym} Gen would be
sex, taking into account historical stages of gender\is{gender} evolutions, various gender\is{gender}
origins, as well as language-specific \isi{semantic} and formal gender\is{gender} uses. Providing
such a global and integrative model of Gen is far beyond the scope of this
work, although such a model is possible to construct, typically based on
empirical formal-\isi{semantic}/pragmatic regular correlations. By correlating a
unique (feminine) Gen morpheme to these various meanings and layers, we avoid
an unmotivated exclusion of numerous meanings and configurations in which Gen
is found.\footnote{The Distributed morphology model is precisely designed to
represent such complex and hierarchical \isi{semantic} and morpho-syntactic mappings.
Properties of traditional lexical terms are actually distributed across
separate lists in the model, each of which is relevant only to a subset of
functions of the traditional lexicon. Syntactic primitives (functional or
contentful) are ± interpretable feature bundles, and Vocabulary Items pronounce
terminal nodes in context only late in the derivation (given their ``Late
insertion'' property). See \citet{HalleEtAl1993,Harley2014}, among others, for
details.}

Given that Gen is neither mono-semic (but rather having the potential to
express many senses), nor mono-functional (not being limited e.g. to
`referential-tracking', but also expressing perspectivization of referents or
shifts, expressiveness, or illocutionary/speech act modification), an
associated semantics/pragmatics of Gender\is{gender} based on its alleged ``natural''
sex/animacy appears to be highly inappropriate. By contrast, our
minimalist/distributed treatment is designed to take into account both its
polysemy (with no homonymic alternative) and its polyfunctionality, in a
motivated constructional and integrative approach.

Building on various contributions in the literature to account for regular
polysemy, or sense extensions, and its representation or generation, I assume a
single geometric representation in which Gen can be (distributively)
\textit{hyperonymic}, embracing the diverse and structurally organized and
related meanings or functions found cross-linguistically, the sex (or animate)
meaning being only a \textit{hyponym}. This view builds on insightful relevant
work by \citet{DresslerEtAl1994,Jurafsky1996,Kortvelyessy2014,GrandiEtAl2015}
with regard to the \isi{semantic} treatment of evaluatives,
\citeauthor{Lakoff1987}'s~(\citeyear{Lakoff1987}) ``radial'' categorization, as
well as work on neural correlates of \isi{semantic} ambiguity, offering behavioral
and neurophysiological support for a single-entry model of polysemy, in line
with \citet{BerettaEtAl2005,Marantz2005,PylkkanenEtAl2006}.

\section{Conclusion}\label{sec:fassi:conclusion}

I have shown that Gender\is{gender} is more central and active in the nP/DP architecture,
as well as in the (upper and parallel) \isi{CP} structure or higher SAP than has been
thought so far. It is found in multiple layers of the grammar, and it employs
much more \isi{semantic} features. An integrative treatment of its polysemy and its
distributed syntax has been proposed. This multi-layered integrated account of
Gender\is{gender} has relevant and broad consequences for both the typology and the theory
of Gender\is{gender}, as well as other interrelated categories (namely \isi{Number}), and
processes such as Gender\is{gender} \isi{agreement} (which also turns out to be a cover for
various types, with different properties).

\section*{Acknowledgements}

Parts of this work have been presented at various academic events and places
during the academic year 2014--2015, including Paris \textsc{vii} University
ling-lunch on February 2015, Qatar University Linguistic Gulf 5 Conference
Keynote address in March 2015, the Syntax Workshop of \ili{Arabic} Varieties in
Geneva in August 2015, the SLE Conference in Leiden in September 2015, and the
Ottawa workshop on Gender\is{gender} in October 2015, the Linguistic Society of Morocco
Workshop in April 2014. I would like to thank the audiences there, and
acknowledge helpful discussions, remarks and comments by Bernard Fradin, Peter
Hallman, Anna Maria Di Sciullo, Noam Chomsky, Sylvain Bromberger, David
Pesetsky, Marten Mous, Frederic Hoyt, Ur Shlonsky, Ahmad Rizwan, Maathir Al-
Rawii, Margherita Pallottino, Pascal Amisli, Danièle Godard, Eric Mathieu,
Saleh al-Qahtani, Miryam Dali, and two anonymous reviewers of the volume.
Special thanks are due to Anders Holmberg for commenting on part of this work, and
providing insightful feedback. His original contributions have always been a source of inspiration for various topics in my research, as well as innovative work
in generative theory. The usual disclaimers apply.

{\sloppy
  \printbibliography[heading=subbibliography,notkeyword=this]
}
\end{document}
