\documentclass[output=paper]{LSP/langsci}
\author{Verner Egerland\affiliation{Lund University}
}
\title{First Person Readings of MAN: On semantic and pragmatic restrictions on an impersonal pronoun}
% \epigram{Change epigram}
 
\abstract{\citet{Cinque1988} notices that Italian impersonal \textit{si} can be interpreted so as to include the speaker and that such a reading is actually mandatory in certain contexts. A similar conclusion holds for impersonal \textit{man} in a language such as Swedish, with the difference that, in the relevant contexts, \textit{man} takes on the reading of 1st person singular, hence ‘I’ and not ‘we’.
In this paper, I argue that Cinque’s observation can only be understood in a theory explaining how impersonal readings (generic and existential) are restricted, rather than in a general theory of “inclusiveness”. 
The first part of paper is dedicated to showing how impersonal readings are restricted by the temporal and aspectual specification of the clause. This part summarizes some by now well-known facts concerning the interpretation of \textit{man}.
The second part of the paper discusses a further restriction on impersonal readings, stemming from focus and contrastiveness. The relevant effect is shown in cases of topicalization of SELF-anaphora in impersonal constructions in some Germanic languages. To my knowledge, these data have so far gone unobserved in the literature.}
 
\shorttitlerunninghead{First Person Readings of MAN}
\ChapterDOI{10.5281/zenodo.1116765}

\maketitle

\begin{document} 

\section{Introduction: “Inclusive” readings of impersonal pronouns}\label{sec:egerland:1}

The literature on impersonal pronouns has grown considerably in the last 20 years. Its findings suggest that “impersonal syntax” is a rather heterogeneous phenomenon which extensively correlates with different parts of grammar, semantics, and pragmatics.

In this paper, I intend to discuss two well-known empirical observations. First, in seminal work on impersonal pronouns, \citet{Cinque1988} notices that \ili{Italian} impersonal \textit{si} can be interpreted so as to include the speaker and that such a reading is actually mandatory in certain contexts. That is to say, while in \REF{ex:egerland:1}, \textit{si} can be interpreted as ‘we’, in \REF{ex:egerland:2} it has to be interpreted thus:\footnote{There is some regional and dialectal variation concerning the b-reading of \REF{ex:egerland:1} and the acceptability of \REF{ex:egerland:2}. My \ili{Italian} consultants are speakers of the Tuscan variety.}

\ea\label{ex:egerland:1}
\ili{Italian}\\
\gll Si  è  lavorato  per  due  mesi    per  risolvere    il    problema.\\
    \textsc{si}  is   worked  for    two  months  for    solve      the  problem\\
\glt a. ‘People have worked for two months to solve the problem.’\\
     b. ‘We have worked for two months to solve the problem.’
\z

\ea\label{ex:egerland:2}
\ili{Italian}\\
\gll Ieri      si  è  stati  licenziati.\\
     yesterday  \textsc{si}  is   been  fired\\
\glt ‘yesterday we were fired’
\z
Second, \citet{Kratzer1997,Kratzer2000} makes the observation the \ili{German} impersonal \textit{man} is understood to include the speaker in cases such as \REF{ex:egerland:3} and \REF{ex:egerland:4}:

\ea\label{ex:egerland:3}
\ili{German}\\
\gll Als   ich  klein  war,  wurde  man  nur  am  Freitag  gebadet. \\
     when  I     little  was   got     \textsc{man}  only  on    Friday  bathed\\
\glt ‘When I was little, we only had a bath on Fridays.’
\z

\ea\label{ex:egerland:4}
\ili{German}\\
\gll Wenn  ich  Kinder  hätte,  könnte  man  zusammen  Monopoly  spielen.\\
      If      I    kids    had  could    man  together    Monopoly  play.\\
\glt ‘If I had children, we could play Monopoly together.’
\z

In the following pages, I will refer to (\ref{ex:egerland:1}--\ref{ex:egerland:2}) as “Cinque’s observation”, and to (\ref{ex:egerland:3}--\ref{ex:egerland:4}) as “Kratzer’s observation”. The question arises as to whether the speaker-inclusion-effects observed in (\ref{ex:egerland:1}--\ref{ex:egerland:4}) have a common underlying source. In other words, should we try to formulate a general theory of “inclusiveness” that can account for all of (\ref{ex:egerland:1}--\ref{ex:egerland:4})? Some such suggestions have been advanced and discussed in the literature (different views are being expressed in e.g. \citealt{DAlessandroAlexiadou2003,DAlessandro2007,Malamud2006,Zobel2011}). In this paper, however, I argue that a unified account of (\ref{ex:egerland:1}--\ref{ex:egerland:4}) is implausible.

In fact, the two observations are essentially different in nature: An adequate account of Kratzer’s observation should explain why, in certain contexts, an impersonal pronoun \textit{must} be interpreted so as to include the speaker. An account of Cinque’s observation, on the other hand, should explain why, in certain contexts, an impersonal pronoun \textit{cannot} be interpreted either as generic or as existential.\footnote{In this paper, the readings of impersonal subjects will be defined as generic or existential (corresponding to generic and episodic time/\isi{aspect} \isi{reference}). For present purposes, I will avoid the term “arbitrary” which has frequently been used in the relevant literature.}

The paper is organized as follows: In \sectref{sec:egerland:2}, I list some arguments against a unified approach to (\ref{ex:egerland:1}--\ref{ex:egerland:4}), after which Kratzer’s observation is set aside: I assume that Kratzer’s claim is correct and, hence, that (\ref{ex:egerland:3}--\ref{ex:egerland:4}) can be successfully accounted for in a theory of logophoricity (as further developed in \citealt{Kratzer2009}). In \sectref{sec:egerland:3}, I claim that important restrictions on impersonal readings derive from the (interaction between) lexical and grammatical \isi{aspect}. In \sectref{sec:egerland:4}, I turn to the topicalization of the equivalents of \textit{self} in some Germanic languages. In \textit{self}-topicalization environments, a different restriction on impersonal readings emerges, deriving from the information structural notion of contrastiveness.


\section{Against a unified approach to “inclusiveness” phenomena}\label{sec:egerland:2}

There are several arguments against a unified account of (\ref{ex:egerland:1}--\ref{ex:egerland:4}). Four of them will be listed in \sectref{sec:egerland:2.1} -- \sectref{sec:egerland:2.4}.

\subsection{Inclusive readings vs. specific ones}\label{sec:egerland:2.1}
In \ili{Italian} (\ref{ex:egerland:1}-\ref{ex:egerland:2}) and \ili{German} (\ref{ex:egerland:3}-\ref{ex:egerland:4}) alike, impersonal pronouns receive a \textit{we}-reading, but there are languages in which the interpretation differs between the two cases. In \ili{Swedish} \REF{ex:egerland:5}, equivalent to Kratzer’s example \REF{ex:egerland:3}, \textit{man} is interpreted as ‘we’, quite as much as its \ili{German} counterpart. However, in \REF{ex:egerland:6}, the equivalent to \REF{ex:egerland:2}, the reading is 1\textsuperscript{st} singular, ‘I’:

\ea\label{ex:egerland:5}
\ili{Swedish}\\
\gll När  jag  var  liten  badade  man  bara  på  fredagar.\\
when  I    was  little  bathed  \textsc{man}  only  on  Fridays\\
\glt ‘When I was little, we only had a bath on Fridays.’
\z

\newpage 
\ea\label{ex:egerland:6}
\ili{Swedish}\\
\gll {I går}      blev  man  avskedad.\\
yesterday  was  \textsc{man}  fired\\
\glt ‘yesterday afternoon I was fired’
\z

Hence, the conclusion that impersonal pronouns in a context such as \REF{ex:egerland:2} include the speaker cannot be generalized to \ili{Swedish} \REF{ex:egerland:6}, in which the subject does not include, but is \textit{specifically} identified with the speaker.\footnote{\label{fn:egerland:3}Traditionally, the 1\textsuperscript{st} singular usage of \textit{man} has been considered substandard and not all speakers are inclined to accept it. Similar considerations hold true for specific readings of impersonal pronouns in several other languages, including the 1\textsuperscript{st} singular reading of \ili{Icelandic} \textit{maður} (to which I turn in \sectref{sec:egerland:4}), as well as the 1\textsuperscript{st} plural reading of \ili{French} \textit{on} and \ili{Italian} \textit{si}: Such interpretations are sometimes associated with dialectal\slash substandard registers and, therefore, are often stigmatized by prescriptive grammars.}

\subsection{The general availability of the 1st singular reading}\label{sec:egerland:2.2}

While Kratzer’s inclusiveness effect manifests itself in particular contexts, the 1\textsuperscript{st} singular reading of \ili{Swedish} \textit{man} is a generally available option. That is to say, \textit{man} can be interpreted as ‘I’ in virtually any context (although of course the scene setting can make such a reading far-fetched). Thus, an example such as \REF{ex:egerland:7} can have at least two interpretations: \textit{People in Spain are in the habit of having dinner late} or \textit{when I’m in Spain, I usually have dinner late}.

\ea\label{ex:egerland:7}
\ili{Swedish}\\
\gll I  Spanien  äter  man  middag  sent.\\
In  Spain    eats  \textsc{man}  dinner  late\\
\glt a. ‘In Spain people have dinner late’\\
     b. ‘In Spain I have dinner late’
\z

The same holds true for \ili{Italian} \textit{si} in the relevant varieties. The example \REF{ex:egerland:8} has two readings parallel to the \ili{Swedish} ones, but with the difference that the b-reading corresponds to 1\textsuperscript{st} plural: \textit{When we’re in Spain, we usually have dinner late}.

\ea\label{ex:egerland:8}
\ili{Italian}\\
\gll In  Spagna  si  mangia  tardi.\\
In  Spain    \textsc{si}  eats    late\\
\glt a.  ‘In Spain people have dinner late’\\
  b.  ‘In Spain we have dinner late’
\z

This state of affairs shows that both \ili{Swedish} \textit{man} in its 1\textsuperscript{st} singular reading, and \ili{Italian} \textit{si} in its 1\textsuperscript{st} plural reading, can be under the scope of a generic operator \citep{Chierchia1995}. This, in turn, suggests that such readings are lexicalized options. I will come back to this intuition shortly.

\subsection{The sensitivity to aspect}\label{sec:egerland:2.3}

The 1\textsuperscript{st} singular interpretation of \ili{Swedish} \textit{man} becomes mandatory as a result of the interaction of lexical and grammatical \isi{aspect} \citep{Egerland2003romance,Egerland2003swedish}. This effect manifests itself in a way which is perfectly parallel to \ili{Italian} as illustrated in (\ref{ex:egerland:1}--\ref{ex:egerland:2}).

First, consider the generic contexts of (\ref{ex:egerland:9}--\ref{ex:egerland:10}):

\ea\label{ex:egerland:9}
\ili{Swedish}\\
\gll Man  arbetat  för    mycket  nuförtiden.\\
\textsc{man}  works  too  much    nowadays\\
\glt ‘People/I have work too much nowadays.’
\z

\ea\label{ex:egerland:10}
\ili{Swedish}\\
\gll Man  blir  lätt    avskedad  nuförtiden.\\
\textsc{man}   is    easily    fired      nowadays\\
\glt ‘People / I get fired easily nowadays.’
\z

Let us concentrate on the impersonal reading, setting aside the 1\textsuperscript{st} singular one: In (\ref{ex:egerland:9}--\ref{ex:egerland:10}), the impersonal argument \textit{man} is interpreted generically. \textit{Man} can successfully be raised to subject position, say [Spec, T], regardless of whether it originates as an external argument, as in \REF{ex:egerland:9}, or as an internal argument, as in \REF{ex:egerland:10}. The derivations of generic \textit{man} can be illustrated with the structure in \REF{ex:egerland:11}:

\ea\label{ex:egerland:11}
… [\textsubscript{TP}\tikz[remember picture,baseline=-0.5ex] \node (egerlandman) {\textit{man}}; [\textsubscript{T’} T\textsubscript{GENERIC} [\textsubscript{VP} \tikz[remember picture,baseline=-0.5ex]\node (egerlandmanman) {\strut(\st{\textit{man}})}; [\textsubscript{V’} V \tikz[remember picture,baseline=-0.5ex] \node (egerlandmanmanman) {(\st{\textit{man}})};]]]]
\vspace*{2\baselineskip}
\begin{tikzpicture}[remember picture,overlay]
 \coordinate [below=5pt of egerlandmanman.base] (egermanman);
 \coordinate [below=5pt of egerlandmanmanman.base] (egermanmanman);
 \draw[-Stealth] (egermanman) -- ++(0,-0.5) -| (egerlandman.290);
 \draw[-Stealth] (egermanmanman) -- ++(0,-0.75) -| (egerlandman.250);
\end{tikzpicture}
\z

Then, consider the episodic contexts of the examples (\ref{ex:egerland:12}--\ref{ex:egerland:13}):

\ea\label{ex:egerland:12}
\ili{Swedish}\\
\gll Man  har  arbetat  i  två  månader  för    att  lösa  problemet.\\
\textsc{man}  has  worked  in  two  months  for    to  solve  problem.the\\
\glt ‘People/I have been working for two months to solve the problem’
\z

\ea\label{ex:egerland:13}
\ili{Swedish}\\
\gll {I går}      blev  man  avskedad.\\
yesterday  was  \textsc{man}  fired\\
\glt ‘yesterday *people were / I was fired.’
\z

In both of \REF{ex:egerland:12} and \REF{ex:egerland:13}, a generic reading of \textit{man} is excluded because of the perfective grammatical \isi{aspect}.\footnote{That the crucial notion is grammatical \isi{aspect} rather than specific time \isi{reference} was also pointed out by \citet{DAlessandroAlexiadou2003}.} However, in \REF{ex:egerland:12}, \textit{man} can be interpreted existentially, as ‘some (group of) people’, whereas in \REF{ex:egerland:13}, the existential reading too is barred. The derivation of existential \textit{man} can be illustrated with the structure in \REF{ex:egerland:14}:

\ea\label{ex:egerland:14}
… [\textsubscript{TP} \tikz[remember picture,baseline=-0.5ex] \node (eger14man) {\textit{man}}; [\textsubscript{T’} T\textsubscript{EPISODIC} [\textsubscript{VP} \tikz[remember picture,baseline=-0.5ex] \node (eger14manman) {(\st{\textit{man}})}; [\textsubscript{V’} V \tikz[remember picture,baseline=-0.5ex] \node (eger14manmanman) {(\st{\textit{man}})};]]]]
\vspace*{2\baselineskip}
\begin{tikzpicture}[remember picture,overlay]
 \coordinate [below=5pt of eger14manman.base] (egermanman14);
 \coordinate [below=5pt of eger14manmanman.base] (egermanmanman14);
 \draw[-Stealth] (egermanman14) -- ++(0,-0.5) -| (eger14man.290);
 \draw[-Stealth] (egermanmanman14) -- ++(0,-0.75) -| (eger14man.250) node [near start] {\textsf{\textbf{X}}} ;
\end{tikzpicture}
\z

In all of these examples, however, \textit{man} can be interpreted as 1\textsuperscript{st} singular, and this reading actually becomes mandatory in \REF{ex:egerland:13}. While the generic reading of both \REF{ex:egerland:12} and \REF{ex:egerland:13} is ruled out by the grammatical \isi{aspect}, it remains to be established what rules out the existential reading of \REF{ex:egerland:13}. I turn to this issue in \sectref{sec:egerland:3}.

On the contrary, inclusiveness in Kratzer’s theory does not obey any restriction concerning \isi{aspect}.\footnote{In fact, the examples offered by Kratzer are typically generic or habitual, as in (\ref{ex:egerland:3}--\ref{ex:egerland:4}), a fact which further underlines the difference between Kratzer’s observation and Cinque’s observation.}

\subsection{Cross-linguistic variation}\label{sec:egerland:2.4}

Cinque’s effect is subject to intricate cross-linguistic variation, also among closely related varieties. While \ili{Italian} \textit{si} is generally available with the \textit{we}-reading, no such reading is generally associated with Spanish \textit{se}. In Spanish \REF{ex:egerland:15}, the only available reading is that in which some people have been working for two months, whereas \REF{ex:egerland:16} is unacceptable:

\ea\label{ex:egerland:15}
Spanish\\
\gll Se    ha    trabajado  durante   dos  meses   para  resolver  el    problema.\\
\textsc{se}    has  worked     for     two  months  to    solve     the  problem\\
\glt ‘people have worked for two months...’
\z

\ea\label{ex:egerland:16}
Spanish\\
\gll * Ayer     se  fue   despedido.\\
{} yesterday  \textsc{se}  was   fired.\\
\z

The variation among Germanic languages is parallel to that between \ili{Italian} and Spanish. For instance, consider \ili{Norwegian} and \ili{German}. In contrast to \ili{Swedish}, the 1\textsuperscript{st} singular reading of impersonal \textit{man} is not generally available in either of \ili{Norwegian} or \ili{German}. That is to say, in \REF{ex:egerland:17} and \REF{ex:egerland:18}, \textit{man} is existentially interpreted as ‘(some) people’, while (\ref{ex:egerland:19}--\ref{ex:egerland:20}) are found unacceptable by my consultants.\footnote{This is not to say that 1\textsuperscript{st} singular or 1\textsuperscript{st} plural readings are all together excluded with \ili{Norwegian} and \ili{German} \textit{man}, nor with Spanish \textit{se}. In fact, impersonal readings in all of these languages can be contextually “manipulated” so as to refer to various discourse participants. However, in \ili{Norwegian}, \ili{German}, and Spanish, such readings are not generally available, unlike what we see in \ili{Swedish} and \ili{Italian}. Recall, however, that in all of these languages, such specific readings emerge as a matter of dialectal variation (see f.n. \ref{fn:egerland:3}). Therefore, this should not necessarily be understood as a comparison between national “standard” languages, but rather between different varieties of such languages. As for a discussion on the variation within Germanic, see e.g. \citet{Malamud2006,Hoekstra2010}.}

\ea\label{ex:egerland:17}
\ili{Norwegian}\\
\gll  Man  har  arbeidet  i  to    måneder    med  dette  problemet.\\
\textsc{man}  has  worked  in  two  months    with  this  problem.the\\
\z

\ea\label{ex:egerland:18}
\ili{German}\\
\gll Man  hat  zwei  Monate  lang  gearbeitet,  um  das  Problem  zu  lösen.\\
\textsc{man}  has  two  months  long  worked    for    the  problem  to  solve\\
\glt ‘(Some) people have been working for two months to solve the problem.’
\z

\ea\label{ex:egerland:19}
\ili{Norwegian}\\
\gll ?* {I går}    ble  man  oppsagt.\\
{} yesterday  was  \textsc{man}  fired\\
\z

\ea\label{ex:egerland:20}
\ili{German}\\
\gll ?* Gestern  wurde  man  gefeuert.\\
   {} yesterday  was    \textsc{man}  fired\\
\z

Kratzer’s observation, on the other hand, is not expected to be subject to such cross-linguistic variation. Rather, some basic properties of logophoric \isi{reference} are expected to be largely constant across languages.

For the purposes of this paper, I assume that Kratzer’s logophoricity account for cases of inclusiveness such as (\ref{ex:egerland:3}--\ref{ex:egerland:4}) is correct, and will not further discuss it here. In \sectref{sec:egerland:3}, I turn to the analysis of Cinque’s observation.

\section{The aspectual restrictions on impersonal readings}\label{sec:egerland:3}

As argued in Egerland \citeyear*{Egerland2003romance,Egerland2005}, Cinque’s observation, as well as some of the cross-linguistic variation, can be accounted for on the set of assumptions listed in \sectref{sec:egerland:3.1} -- \sectref{sec:egerland:3.3}

\subsection{The 1st person reading is lexical}\label{sec:egerland:3.1}

The \textit{man} pronoun in (the relevant variety of) \ili{Swedish} can be lexically associated with a 1\textsuperscript{st} singular reading. By this, I mean that 1\textsuperscript{st} singular \textit{man} is an independent lexeme acquired as such and, hence, a homonym to impersonal \textit{man}. I propose the same analysis of the 1\textsuperscript{st} plural reading of (the relevant variety of) \ili{Italian} \textit{si}. Therefore, such readings are not syntactically constrained but, essentially, always available. For instance, such lexicalized forms can be under the scope of a generic operator, as in (\ref{ex:egerland:7}b) and (\ref{ex:egerland:8}b).
%

\subsection{Impersonal pronouns are featurally deficient}\label{sec:egerland:3.2}

As we have seen, there are environments in which generic as well as existential readings of \textit{man} are excluded. Suppose that the ungrammaticality of Spanish \REF{ex:egerland:16}, \ili{Norwegian} \REF{ex:egerland:19}, and \ili{German} \REF{ex:egerland:20} arises as a result of the interaction between lexical and grammatical \isi{aspect}: While a generic reading is barred by perfective \isi{aspect}, the existential reading is barred by a “delimited” lexical \isi{aspect}, in the sense of \citet{Tenny1987}; i.e. the existential reading is excluded by the fact that the surface subject is the internal argument of a delimited event. The generalizations expressed in the structures \REF{ex:egerland:11} and \REF{ex:egerland:14} can be captured as in \REF{ex:egerland:21} (a reformulation of \citet[82]{Egerland2003swedish}:

\ea\label{ex:egerland:21}
\textit{Man} cannot be the impersonal existential subject of a delimited event, if \textit{man} itself corresponds to the argument that limits the event.
\z

There is a natural explanation to \REF{ex:egerland:21} on the assumption that, in order to establish whether an argument does or does not limit the event, the argument in question needs to have some inherent content or, informally speaking, a certain degree of referentiality. To be more precise, suppose that a feature corresponding to the Inner Aspect projects a phrase, say EventP \citep{Travis2000,Borer2005}:

\ea\label{ex:egerland:22}
... [\textsubscript{TP} T [\textsubscript{vP} DP v [\textsubscript{EventP} Event [\textsubscript{VP} V DP]]]]
\z

In \REF{ex:egerland:22}, the internal argument, but not the external one, needs to be matched against the Event. In order to enter into such a relation, the internal argument must carry some specification with regard to specificity and \isi{number}.\footnote{Recall that, for instance, the difference between the delimited reading of \textit{Dustin ate an apple} and the non-delimited reading of \textit{Dustin ate apples} depends on the \isi{number} specification of the object (\citealt{Carlson1977}, \citealt[113]{Tenny1987}).} As impersonal \textit{man} is underspecified for specificity and \isi{number}, it is unable to evaluate the Event. The generalization in \REF{ex:egerland:21} follows. Therefore, \ili{Swedish} \textit{man} is interpreted as 1\textsuperscript{st} singular, and \ili{Italian} \textit{si} as 1\textsuperscript{st} plural, because these are the only remaining options.\footnote{I assume that, in the case of generic \textit{man} as in the structure \REF{ex:egerland:11}, the \isi{semantic} content of \textit{man} is provided by the generic operator \citep{Chierchia1995}. Presumably, it is the presence of such an operator that makes it possible for generic \textit{man} to bind anaphors, while existential \textit{man} does not have this property, as pointed out by \citet{CabredoHofherr2010}.}

\subsection{The mandatory 1st person reading is a 'last resort'}\label{sec:egerland:3.3}

If an impersonal pronoun, in a given language, is not lexically associated with such specific readings, and if the context rules out generic and existential readings, the expression is not interpretable. This is what we observe with Spanish \textit{se} \REF{ex:egerland:16}, \ili{Norwegian} \textit{man} \REF{ex:egerland:19} and \ili{German} \textit{man} \REF{ex:egerland:20}.

The intuition behind such an account is that Cinque’s observation does not follow from an effect imposing inclusive readings on impersonal pronouns, but rather from independent restrictions on generic and existential readings of such pronouns.

The discussion of this section has taken into consideration restrictions that are aspectual in nature. Clearly, however, generic and existential readings can be restricted by other factors than \isi{aspect}. In the following section, I turn to a quite different set of data which I believe corroborate the approach outlined in \sectref{sec:egerland:3.1}--\sectref{sec:egerland:3.3}

\section{\textsc{self}-topicalization}\label{sec:egerland:4}

In this section, the hypothesis outlined in \sectref{sec:egerland:3} will be tested on a different set of data. The following discussion, which concerns information structure, will be limited to the comparison of four Germanic varieties, namely \ili{Swedish}, \ili{Icelandic}, \ili{Norwegian}, and \ili{German}.\footnote{The hypothesis cannot be tested on \ili{Romance} data, given that the equivalent elements (\ili{French} \textit{même}, \ili{Italian} \textit{stesso}, Spanish \textit{mismo}) cannot be topicalized in a way similar to what we observe in Germanic languages.}

\subsection{The topicalization of \textsc{self}-anaphora}\label{sec:egerland:4.1}

In all of these languages, \textsc{self} anaphora can appear in different positions of the clause. Given a setting such as the one stated as Context A, as in (\ref{ex:egerland:23}--\ref{ex:egerland:26}), \textsc{self} can appear in a sentence internal position, the exact nature of which is immaterial for the present discussion:\footnote{In all of the languages, \textsc{self} can appear in other possible positions as well which will not be considered here. For instance, it can follow the DP (\ili{Swedish} \textit{chefen själv} ‘the boss himself’) or even appear sentence-finally. This state of affairs can be taken as evidence that \textsc{self} anaphora such as those discussed in the text have “floating” properties \citep{Kayne1975,Sportiche1988}. On the other hand, an anonymous reviewer suggests that the two instances of \ili{German} \textit{selbst} in \REF{ex:egerland:26} and \REF{ex:egerland:30} could be separate lexemes though homonymous. For present purposes, this possibility can remain an open issue.}

\begin{description}
\item[Context A:] The coming week everyone in my office is taking a leave…
\ea\label{ex:egerland:23}
\ili{Swedish}\\
\gll Chefen  åker  själv  på  semester.\\
boss.the  goes  \textsc{self}  on  holiday\\
\z

\ea\label{ex:egerland:24}
\ili{Icelandic}\\
\gll Stjórinn  fer    sjálfur  í  frí.\\
boss.the  goes  \textsc{self}    on  holiday\\
\z

\ea\label{ex:egerland:25}
\ili{Norwegian}\\
\gll Sjefen  drar  sjøl  på ferie.\\
boss.the  goes  \textsc{self}  on  holiday\\
\z

\ea\label{ex:egerland:26}
\ili{German}\\
\gll Der  Chef  fährt  selbst    in  Urlaub.\\
the  boss  goes  \textsc{self}    on  holiday\\
\glt ‘...the boss himself / even the boss / the boss too is going on a holiday.’
\z
\end{description}

Furthermore, in all four languages, \textsc{self} can be topicalized, as in (\ref{ex:egerland:27}--\ref{ex:egerland:30}). This, however, is pragmatically appropriate in a different kind of setting, as for instance the one suggested in Context B:

\newpage 
\begin{description}
\item[Context B:] Everyone else in my office has to work over the weekend but …
\ea\label{ex:egerland:27}
\ili{Swedish}\\
\gll Själv  åker  chefen  på  semester.\\
\textsc{self}  goes boss.the  on  holiday\\
\z

\ea\label{ex:egerland:28}
\ili{Icelandic}\\
\gll Sjálfur  fer    stjórinn    í  frí.\\
\textsc{self}    goes  boss.the    on  holiday\\
\z

\ea\label{ex:egerland:29}
\ili{Norwegian}\\
\gll Sjøl  drar  sjefen  på  ferie.\\
\textsc{self}  goes  boss.the  on  holiday\\
\z

\ea\label{ex:egerland:30}
\ili{German}\\
\gll Selbst  fährt  der  Chef  in  Urlaub.\\
\textsc{self}    goes  the  boss  on  holiday\\
\glt ‘... but the boss, on the other hand, is leaving for a holiday.’
\z
\end{description}

Consider that, in the languages in question, \textsc{self} creates a contrastive reading. For concreteness, I chose to formulate the information structural notion of contrastiveness in terms of membership in a set, along the lines of e.g. \citet{VallduvíVilkuna1998}:\footnote{Contrastiveness, as in the definition in \REF{ex:egerland:31}, is presented as a “cover term for several operator-like interpretations of focus that one finds in the literature” \citep[83]{VallduvíVilkuna1998}. That is to say that the generalization we are interested in could be formulated in different terms, as for instance the identificational focus of \citet{Kiss1998}. For present purposes, \REF{ex:egerland:31} will suffice.}

\ea\label{ex:egerland:31}
\citep[83]{VallduvíVilkuna1998}\\
If an expression \textbf{a} is kontrastive, a membership set \textbf{M} = \{…, \textbf{a}, …\} is generated and becomes available to \isi{semantic} computation as some sort of quantificational domain …
\z

In all of (\ref{ex:egerland:23}--\ref{ex:egerland:30}), \textsc{self} generates a set reading and picks out one member of the set, the boss: In (\ref{ex:egerland:23}--\ref{ex:egerland:26}), the expression points out that the boss is (unexpectedly) part of the set (while he could have stayed at work, he is leaving together with the others). In (\ref{ex:egerland:27}--\ref{ex:egerland:30}), on the contrary, the boss is interpreted in contrast to the other members of the set (he is leaving while everyone else is staying at work). Now, let us turn to impersonal constructions.

\newpage 
\subsection{The relevance of \textsc{self}-topicalization for the interpretation of impersonal \textit{man}}\label{sec:egerland:4.2}

The reason why \ili{Icelandic} is taken into consideration at this point is that \ili{Icelandic} \textit{maður} shares with \ili{Swedish} \textit{man} the property of being interpretable as 1\textsuperscript{st} singular in a colloquial register. For \ili{Icelandic}, the effect was first discussed by \citet{Jónsson1992} who gives the example \REF{ex:egerland:32} (= his 43): \footnote{There are, however, independent differences between \ili{Swedish} and \ili{Icelandic}. In particular, unlike \ili{Swedish} \textit{man}, \ili{Icelandic} \textit{maður} is not compatible with the existential reading at all in episodic contexts \citep{Jónsson1992,Egerland2003romance,SigurðssonEgerland2009}. This difference need not concern us here.}

\ea\label{ex:egerland:32}
\ili{Icelandic}\\
\gll Eg   vona    að   maður  verði     ekki   of seinn.\\
I     hope    that   \textsc{maður}  will.be   not   too late\\
\glt ‘I hope I won’t be late.’
\z

Given Context A, when \textsc{self} appears in the sentence internal position, there are two possible readings as illustrated in \ili{Swedish} \REF{ex:egerland:33} and \ili{Icelandic} \REF{ex:egerland:34}:

\begin{description}
\item[Context A:] In hostels there is sometimes no room cleaning service, so…
\ea\label{ex:egerland:33}
\ili{Swedish}\\
\gll Man  måste    själv  städa    rummet.\\
\textsc{man}  must    \textsc{self}  clean    room.the\\
\z

\ea\label{ex:egerland:34}
\ili{Icelandic}\\
\gll Maður  verður  sjálfur  að  þrífa  herbergið.\\
\textsc{man}    must    \textsc{self}    to  clean  room.the\\
\glt a. ‘People have to clean their rooms themselves / on their own.’\\
     b. ‘I have to clean the room myself / on my own.’
\z
\end{description}


In the a-interpretation of (\ref{ex:egerland:33}--\ref{ex:egerland:34}), the impersonal is referring to people in general. In the b-interpretation (which is colloquial), \textit{man} and \textit{maður} specifically refer to 1\textsuperscript{st} singular. In other words, (\ref{ex:egerland:33}--\ref{ex:egerland:34}) can be taken to mean that whenever I stay in a hostel, I need to clean my room myself.

In \ili{Norwegian} and \ili{German}, the same sentence is acceptable in the same kind of context, however only with the generic reading:

\ea\label{ex:egerland:35}
\ili{Norwegian}\\
\gll Man  må  sjøl  rydde    rommet.\\
\textsc{man}  must  \textsc{self}  clean    room.the\\
\z

\ea\label{ex:egerland:36}
\ili{German}\\
\gll Man  muss  das  Zimmer  selbst    aufräumen.\\
\textsc{man}  must  the  room    \textsc{self}    clean\\
\glt ‘People have to clean their rooms themselves.’
\z

Thus, (\ref{ex:egerland:35}--\ref{ex:egerland:36}) confirm the earlier observation concerning \ili{Norwegian} and \ili{German}: \isi{Impersonal} \textit{man} is not associated with the 1\textsuperscript{st} singular reading. \footnote{But recall that it is always the case with generic readings that they encompass all the persons of the paradigm, hence also 1\textsuperscript{st} person.}

Furthermore, under particular circumstances, \textsc{self} can be topicalized in both \ili{Swedish} and \ili{Icelandic} impersonal sentences. Such a topicalization, however, requires a completely different kind of setting to be pragmatically appropriate. For instance, a child who is grounded while his/her companions are out playing could say something such as (\ref{ex:egerland:37}--\ref{ex:egerland:38}):

\begin{description}
\item[Context B:] All the other kids are out having fun, but ...
\ea\label{ex:egerland:37}
\ili{Swedish}\\
\gll Själv  måste    man  städa  rummet.\\
\textsc{self}  must    \textsc{man}  clean  room.the\\
\z

\ea\label{ex:egerland:38}
\ili{Icelandic}\\
\gll sjálfur  verður  maður  að  þrífa  herbergið.\\
\textsc{self}    must    \textsc{man}    to  clean  room.the\\
\glt ‘… but I have to stay at home and clean my room.’
\z
\end{description}

The utterance is only acceptable if the subject is identified with 1\textsuperscript{st} singular. I suggest this is so because of the contrastive reading associated with topicalization. Suppose that contrastiveness indeed generates a set reading, as stated in \REF{ex:egerland:31}. In \REF{ex:egerland:31}, “\textbf{M} is a set of objects matching \textbf{a} in \isi{semantic} type” \citep[84]{VallduvíVilkuna1998}. Arguably, then, contrastiveness can hold between specific individuals or groups of individuals. An impersonal pronoun radically lacks specificity and \isi{number} features. Hence, it cannot be put in contrast with another “object of the same \isi{semantic} type”, quite as much as it cannot delimit the event (see \sectref{sec:egerland:3}).\footnote{The radical featural deficiency of impersonal pronouns such as \textit{man} is also assumed in e.g.  \citet{CabredoHofherr2010}. In \citet{Egerland2003romance} this featural deficiency was taken to be directly linked to a certain variability in \isi{agreement} patterns attested in \ili{Swedish}. Admittedly, this conclusion may not extend to Germanic languages generally, as pointed out in \citet{Malamud2012}.}

I believe this restriction on impersonal readings may be illustrated with what is sometimes called generic nouns, such as English \textit{people} (and equivalent expressions in other languages), although an in depth analysis of such nouns goes far beyond the purposes of this study.\footnote{But in the theory of \citet{Hoekstra2010}, impersonal pronouns are taken to be the pronominal counterparts of such generic nouns.} Consider that \textit{people} cannot be contrasted with a single individual. I can say something like \textit{people around here usually come early to the office, but John doesn’t}, but I cannot express this meaning as a contrastive focus:

\ea\label{ex:egerland:39}
\textsuperscript{??}It is \textit{people} who come early to the office (not John).
\z

Under contrastive focus, namely, \textit{people} becomes a kind-denoting expression, as in (\ref{ex:egerland:40}--\ref{ex:egerland:41}) (cf. \citealt{Chierchia1998}):

\ea\label{ex:egerland:40}
It is \textit{people} who do bad things (not God).
\z

\ea\label{ex:egerland:41}
Around here, it is \textit{people} who do the work (not machines).
\z

However, unlike impersonal pronouns, \textit{people} is indeed a noun and thus compatible with a lexical restriction, such as a \isi{relative} clause. Not unexpectedly, a contrastive reading with a non kind-denoting \textit{people} becomes possible if \textit{people} is restricted so as to refer to a specific group of individuals:

\ea\label{ex:egerland:42}
It is \textit{people who come early to the office} who get things done (not John).
\z


\isi{Impersonal} subjects such as \textit{man} are weak pronominal elements: they cannot take restrictions such as the \isi{relative} clause in \REF{ex:egerland:42}, neither can they carry stress.\footnote{There are exceptions to this rule, such as West Frisian \textit{men} \citep{Hoekstra2010}.} Hence, the impersonal pronoun itself cannot be topicalized. However, the associate \textsc{self} is generally stressed and can indeed be topicalized.

For concreteness, then, assume that the complex [\textsc{man self}] originates as a phrase, and that \textsc{self} moves out of this phrase during the derivation. The details of such an analysis are not crucial for my line of reasoning, the important thing being that some interpretative dependency holds between the pronoun \textsc{man} and the anaphor \textsc{self}. The derivation of \REF{ex:egerland:37} is illustrated in the structure of \REF{ex:egerland:43}:

\ea\label{ex:egerland:43}
[\textsubscript{CP} \textsc{self}\textit{\textsubscript{i}} [\textsubscript{C’} \textit{måste} [\textsubscript{TP} [\st{\textsc{man}} \textsc{self]}\textit{\textsubscript{i}} … [\textsubscript{vP} [\st{\textsc{man}} \st{\textsc{self}}]\textit{\textsubscript{i}} VP]]]]
\z

The topicalization creates a reading in which the subject, [\textsc{man self}], is interpreted in contrast to some other participant of the discourse. As a deficient pronoun cannot be interpreted under contrastive focus, the lexicalized 1\textsuperscript{st} singular option is the only one remaining. Therefore, (\ref{ex:egerland:37}--\ref{ex:egerland:38}) can only be taken to refer to the 1\textsuperscript{st} singular.

Crucially, this line of reasoning gives rise to the prediction that the equivalent sentences are unacceptable in \ili{Norwegian} and \ili{German}, given that the 1\textsuperscript{st} singular alternative is not available in these languages. My consultants confirm this prediction:

\ea\label{ex:egerland:44}
\ili{Norwegian}\\
\gll * sjøl  må  man  rydde    rommet.\\
{} \textsc{self}  must  \textsc{man}  clean    room.the\\
\z

\ea\label{ex:egerland:45}
\ili{German}\\
\gll * selbst  muss  man  das  Zimmer  aufräumen.\\
{} \textsc{self}    must  \textsc{man}  the  room    clean\\
\z

What we observe in (\ref{ex:egerland:44}--\ref{ex:egerland:45}) is the same kind of effect as in the examples \REF{ex:egerland:19} and \REF{ex:egerland:20} in \sectref{sec:egerland:2.4}: When the impersonal readings are barred, \ili{Norwegian} and \ili{German} cannot recur to a lexicalized specific interpretation. \footnote{An anonymous reviewer points out the (s)he finds an example such as \REF{ex:egerland:44} acceptable in \ili{Norwegian}, quite unlike my consultants. My only suggestion as to why this could be the case, is that \textsc{self} in some \ili{Scandinavian} varieties can take on the meaning of ‘alone’. In fact, \ili{Swedish} \REF{ex:egerland:37} is also interpretable as ‘I have to clean the room alone’, a possibility which I have chosen to disregard. However, this meaning of \textsc{self} is usually taken to be more normal in \ili{Swedish} than in \ili{Norwegian}.}

\section{Conclusion}\label{sec:egerland:5}
While Kratzer’s observation presumably can be successfully analyzed within a theory explaining when a given impersonal must be interpreted as including the speaker, Cinque’s observation can only be understood in a theory explaining how impersonal readings are restricted. When they are, some languages can access lexicalized readings of impersonals, such as the 1\textsuperscript{st} singular reading of \ili{Swedish} \textit{man}, while other languages do not have any such alternative. I conclude from this that Kratzer’s observation and Cinque’s observation are fundamentally different in nature, despite the superficial similarities.

\section*{Abbreviations}
Abbreviations used in this article follow the Leipzig Glossing Rules’ instructions for word-by-word transcription, available at: \url{https://www.eva.mpg.de/lingua/pdf/Glossing-Rules.pdf}.
% % \section*{Acknowledgements}
% %
% %
{\sloppy
\printbibliography[heading=subbibliography,notkeyword=this]
} 
\end{document}
