\documentclass[output=paper]{langsci/langscibook}

 
% \usepackage{gb4e}
%\bibliography{makebelieve-blx}
% \usepackage{tipa}

%% Dj?\"arv et al preamble: 
%%
%\usepackage{biblatex}
%\usepackage{setspace}
%\usepackage{pifont}
%\usepackage{graphicx}
%%\bibpunct{(}{)}{,}{a}{}{,} %separator for multiple citations is comma, not semicolon
%\usepackage{lmodern}
%\usepackage{helvet}
%\usepackage{hyperref}
%\usepackage{amsfonts}
%\usepackage[normalem]{ulem}  %normalem option makes \emph give italics, not underline
%%\usepackage{ulem}
%\usepackage{gb4e}
%\usepackage{tipa}
%\usepackage{float}
%\usepackage[T1]{fontenc} 
%\usepackage[toc,page]{appendix}
%
%\usepackage{chngcntr}
%\usepackage{enumitem}
%
%\newcommand{\gltspace}{\hspace{10pt}}


\title{On a ``make-believe'' argument for Case Theory}
\author{Jonathan David Bobaljik       
\affiliation{University of Connecticut}
}

\abstract{I argue here that evidence from Icelandic challenges one argument for Case Theory given in Chomsky's seminal paper \textit{On Binding}. Chomsky suggested that a locality (adjacency) condition on structural case assignment explains the systematic absence of ditransitive ECM verbs. I argue here that Icelandic lacks this adjacency condition: structural Case in Icelandic is available to the second argument of a ditransitive in Icelandic. The Case-theoretic account would predict that Icelandic should therefore contrast with English and allow ditransitive ECM constructions. It does not. The absence of ditransitive ECM predicates is thus part of a broader generalization than Case Theory can explain.}

\ChapterDOI{10.5281/zenodo.1116773}


\begin{document}
\maketitle
\section{The \textit{make-believe} argument}
\citet[29]{Chomsky1980}, in the paper introducing GB \isi{Case} Theory, notes the absence of ditransitive ECM verbs, and suggests that \isi{Case} provides a straightforward account of this lexical gap. While there are \is{DOC}double object constructions like (\ref{bobaljik:doc}) and ECM (equivalently Raising-to-Object) predicates like (\ref{bobaljik:believe}), the two properties do not cooccur with a single predicate. There are no ditransitive ECM predicates, neither of the double object type (\ref{bobaljik:ecmconv}) nor with a matrix PP internal argument (\ref{bobaljik:ecmapp}). 

\largerpage
\ea[]{Leo gave Julia a book.} \label{bobaljik:doc}
\ex[]{Leo believes Julia\textsubscript{j} [ t\textsubscript{j} to have won ].} \label{bobaljik:believe}
\ex \label{bobaljik:convince}
\begin{xlista}
\ex[*]{Leo convinced Sarah Julia\textsubscript{j} [ t\textsubscript{j} to have won ].} \label{bobaljik:ecmconv}
\ex[*]{Leo persuaded Sarah Julia\textsubscript{j} [ t\textsubscript{j} to win ].} \label{bobaljik:persuade}
\ex[*]{Leo appealed to Sarah Julia\textsubscript{j} [ t\textsubscript{j} to be nominated ].} \label{bobaljik:ecmapp}
\end{xlista}
\end{exe}

\noindent Verbs that select an infinitive and two other arguments are systematically control predicates, or allow a \textit{for} complement: 

\begin{exe}
\ex \label{bobaljik:contr} \begin{xlista}
\ex[]{Leo convinced Sarah\textsubscript{i} [ PRO\textsubscript{i} to win ].} 
\ex[]{Leo appealed to Sarah\textsubscript{i} [ PRO\textsubscript{i} to (let him) win ].} 
\ex[]{Leo appealed to Sarah [ for Julia to be nominated  ].} 
\end{xlista}
\end{exe}

\noindent This is a curious gap, inasmuch as semantically, verbs like \textit{convince} and \textit{persuade} seem to mean roughly a kind of \isi{causative} of \textit{believe} (thus \ref{bobaljik:convleo} implies \ref{bobaljik:leobel}). There is no obvious reason why a verb meaning \textit{make-believe} should not be able to have the range of arguments available to \textit{believe}, plus a causer.

\begin{exe}
\ex[]{Sarah convinced/persuaded Leo [ that Julia won ].} 
\label{bobaljik:convleo}
\ex[]{Leo believes [ that Julia won ].} \label{bobaljik:leobel}
\end{exe}

\noindent Chomsky argues that \isi{Case} Theory accounts straightforwardly for this gap: structural case assignment is only possible to the adjacent complement of the verb, and the higher internal argument, whether an NP or PP, will invariably disrupt the adjacency between the verb and the infinitival subject required for structural case assignment.\footnote{This argument is revived in  \citet{Boehorn2005} with more modern technology: in place of adjacency, \citet{Boehorn2005} follow \citet{Boskovic2002} in claiming that structural case requires movement, and posit a structure under which movement across the higher NP in examples parallel to (\ref{bobaljik:ecmconv}) violates relativized minimality (they do not mention the PP cases). \citet{Boehorn2005} claim that the case on the theme in (\ref{bobaljik:doc}) is inherent and thus not subject to minimality/adjacency. This is implausible in \ili{Icelandic}, see note \ref{bobaljik:larsonq}.} 

\largerpage
The recent ascendance of Dependent \isi{Case} Theory {[DCT]} \citep{Marantz1991,Baker2015} as an alternative to (L)GB \isi{Case} Theory invites a reconsideration of established arguments for the latter. Under the strongest version of DCT, the syntactic distribution of NPs is not regulated by case (or \isi{Case}), rather, NPs are assigned a particular morphological case as a function of the grammatical structure in which they are found. As such, the explanation of the contrast in (\ref{bobaljik:believe}--\ref{bobaljik:convince}) originally sketched by Chomsky is unavailable under DCT, and thus constitutes a prima facie argument against a strong DCT. In this squib, I argue that Chomsky's argument that \isi{Case} is implicated does not withstand scrutiny. Specifically, the contrast in (\ref{bobaljik:believe}--\ref{bobaljik:convince}) is replicated in \ili{Icelandic}, although it can be shown that there is no intervention (or adjacency) effect on structural accusative case assignment in that language. This yields two conclusions: the absence of ditransitive ECM constructions is not a language-particular quirk of English, but at the same time, GB/MP-style \isi{Case} Theory is not a viable explanation of the gap. After presenting this argument, I will speculate that the absence of ditransitive ECM predicates is plausibly a special case of the oft-cited generalization that a single underived predicate may take no more than three obligatory arguments (see e.g., \citealt{Pesetsky1995}).

\section{Icelandic}
\largerpage[2]

\ili{Icelandic} has played a significant role in discussions of case across multiple generative frameworks, especially since the seminal article by \citet{Zmt1985}. A central finding is that \ili{Icelandic} (descriptively) lacks the adjacency or intervention condition on structural (accusative) case which plays the key role in Chomsky's account of why (\ref{bobaljik:ecmconv}) is excluded. The main observation comes from double-object constructions in \ili{Icelandic} of the \textit{give} type, illustrated in (\ref{bobaljik:ice1}):

\begin{exe}
\ex \label{bobaljik:ice1} \begin{xlista}
\ex \label{bobaljik:bokina}  \gll J\'on gaf \'Olafi b\'okina. \\ 
Jon.{\scshape nom} gave Olaf.{\scshape dat} book.the.{\scshape acc}\\
\glt `Jon gave Olaf the book.' \citep[187]{Holmplat1995}
\ex\label{bobaljik:bokin} \gll \'Olafi var gefin b\'okin. \\
	Olaf.{\scshape dat} was given book.the.{\scshape nom} \\
\glt `Olaf was given the book.' \citep{Falk1990}
\ex\label{bobaljik:tec} \gll {\TH}a{\dh} hafa einhverjum str\'ak veri{\dh} gefnar gjafir.\\
	{\scshape expl} have some.{\scshape dat} boy.{\scshape dat} been given.{\scshape pl} gifts.{\scshape nom}. \\
\glt `Some boy has been given presents.' {\citep[99]{Holmberg2002}}	
\end{xlista}
\end{exe}

\noindent Of the two internal arguments of ditransitive construction in \ili{Icelandic}, the higher one (the dative NP in \ref{bobaljik:bokina}) becomes the subject in the \isi{passive}, but the lower one in the configuration in (\ref{bobaljik:bokina}) undergoes the case alternation which is diagnostic of structural case: accusative in the active, but nominative in the \isi{passive}.\footnote{\label{bobaljik:larsonq}One might question whether the case alternation in \isi{passive} is sufficient evidence that the accusative on the theme is structural case. The literature at least since \citet{Andrews1982} has noted that \ili{Icelandic} has both inherent and structural accusative, and these are distinguished precisely by this diagnostic. For example, inherent accusative (as on the subject of  \textit{vanta} `lack'), unlike structural accusative, is preserved in the \isi{passive} of an ECM complement, as shown in the following:\\

\eafirst
 \gll Han telur mig vanta peninga. \\
he believes me.\textsc{acc} lack money \\  
\glt `He believes me to lack money.'
\z

\ea \gll Mig er tali{\dh} vanta peninga. \\
me.\textsc{acc} is believed lack money \\
\glt `I am believed to lack money.' \citep{Andrews1982}
\zlast
}

\noindent These examples have received extensive scrutiny in the literature since \cite{Zmt1985}, and it is very firmly established that the dative is the subject in (\ref{bobaljik:bokin}) (for example, it constitutes the associate in the transitive expletive construction \ref{bobaljik:tec}) and the nominative is an object.\footnote{As \citet{Holmberg1994} and \citet{Holmplat1995} discuss, an `inverted' order is also possible: the nominative theme may raise to subject position with this class of verbs, but this stems from an `inverted' order in the active, in which the theme precedes and c-commands the goal.} Whatever the analysis, these examples establish the baseline: in \ili{Icelandic}, structural case is available to the lower of two internal arguments in a ditransitive construction. If accusative is assigned by (a functional projection associated with) the verb, then (\ref{bobaljik:bokina}) and related examples show that this assignment is not subject to an adjacency or intervention condition.\footnote{\citet{Holmberg2002} argue that nominative case is subject to an intervention effect, accounting for the absence of \isi{impersonal passive}s of double-object constructions. In theory, one could maintain an intervention-like locality condition on all structural case in \ili{Icelandic}, but then posit an additional case-assigning head below the indirect object in examples like (\ref{bobaljik:bokina}); see \cite{svenonius06}. The source of structural accusative does not bear on the argument made in this squib; the important fact is that it is available to the lower NP in a ditransitive construction. As noted above, the accusative in (\ref{bobaljik:bokina}) patterns with structural, rather than inherent, case in \ili{Icelandic}, where the distinction is sharper than in English: inherent case in \ili{Icelandic}, unlike structural case, fails to alternate in the periphrastic \isi{passive}, and other contexts.} 

Like English, \ili{Icelandic} also has ECM verbs, like `believe':

\begin{exe}
\ex \gll \'Eg tel Harald hafa unni\dh. \\
I believe Harald.{\scshape acc} have.{\scshape win} won \\
\glt `I believe Harald to have won.' 
\end{exe}

\noindent And like English, the `convince' type verbs, taking an upstairs internal argument, may take a \isi{finite} or an infinitive (object control) complement, but disallow ECM:\footnote{The verb meaning `convince' in this context happens to be a particle verb, but this is not relevant to the generalization as just stated -- there are evidently no verbs with the frame in (\ref{bobaljik:iceecmconv}) with or without a particle.}

\begin{exe}
\ex \begin{xlista}
\ex[]{\gll \'Eg sannf{\ae}r{\dh}i {\th}\'a um [a{\dh} Harald-ur hef{\dh}i unni\dh]. \\
I convinced them P that Harald-{\scshape nom} had won \\ 
\glt `I convinced them that Harald had won.' }

\ex[]{\gll \'Eg sannf{\ae}r{\dh}i Harald um [a{\dh} PRO vinna]. \\
I convinced Harald.{\scshape acc} P to {} win.{\scshape inf} \\
\glt `I convinced Harald to win.' }

\ex[*]{\label{bobaljik:iceecmconv} \gll \'Eg sannf{\ae}r{\dh}i {\th}\'a um [Harald hafa unni\dh]. \\  
I convinced them P Harald.{\scshape acc} have.{\scshape inf} won \\
\glt `I convinced them Harald to have won.'} 
\end{xlista}
\end{exe}

\noindent Note finally, that \ili{Icelandic} has predicates like \textit{vir{\dh}ast} `seem' which (i) select an infinitive complement, (ii) treat the subject of that complement (\textit{Mar\'ia} in \ref{bobaljik:handm}) as a matrix object in an ECM-like fashion, and (iii) select a second internal NP argument, distinct from the embedded subject (\textit{Haraldi} in \ref{bobaljik:handm}). Crucially, though, all such verbs lack an external argument of the matrix predicate, and thus have a dative-nominative case array: the embedded subject behaves in the matrix clause as a nominative object (and not as a matrix subject). 

\begin{exe}
\ex \label{bobaljik:handm} \gll Harald-i vir{\dh}ist Mar\'ia vera {\th}reytt. \\
Harald-{\scshape dat} seems Maria.{\scshape nom} be.{\scshape inf} tired \\
\glt `Maria seems to Harald to be tired.' 
\end{exe}

\noindent \ili{Icelandic} has more options than can be seen in English, but in key respects, \ili{Icelandic} is like English, lacking ditransitive ECM predicates. However, since \ili{Icelandic} allows structural accusative to be assigned `across' an intervening NP or PP, the account given by Chomsky (and \citealt{Boehorn2005}) does not extend to \ili{Icelandic}. 

\section{Conclusion}

\citeauthor{Chomsky1980}'s intriguing observation that there are no ditransitive ECM verbs holds of \ili{Icelandic} as well, a language with an English-like ECM construction. This is in and of itself interesting, since it affirms \citeauthor{Chomsky1980}'s suggestion that this gap in the lexicon is systematic, and not accidental. At the same time, \ili{Icelandic} undermines the proposed analysis of this gap in terms of \isi{Case} Theory (and thus the corresponding argument for \isi{Case} Theory). Since \ili{Icelandic} evidently lacks the adjacency requirement that English (supposedly) has, that requirement cannot be the source of the absence of ditransitive ECM verbs across both languages. 

What direction might an alternative account take? I suggest that it is not implausible to see the absence of ditransitive ECM verbs as part of the broader generalization that there is an apparent upper bound on the number of arguments a non-derived predicate may take as part of its argument structure.\footnote{Derived predicates, such as \isi{causative}s, applicative, and other types of complex predicates, may take more.} Although there is some dissent, general opinion seems to place that limit at three.\footnote{Lisa Travis points me to \cite{Carter76} for the suggestion that the limit is four, on the basis of verbs like \textit{trade}: \textit{John traded his cobra to Mary for something}.} A ditransitive verb like \textit{give} or \textit{put} takes the maximum, with three arguments. So too do object control predicates \textit{convince} and \textit{appeal} likewise take three arguments apiece: an external NP, an internal NP or PP argument, and the infinitival complement. If the non-thematic position associated with raising predicates counts as one argument towards the maximum, then Chomsky's generalization is subsumed under this larger one: one argument of the raising verb is the infinitive complement (LFG's {\scshape xcomp}), and a second the athematic position that is the landing site of raising (whether to subject or object). This leaves only one `free' slot, which may be an external argument (as in \textit{believe}) or an internal one, as in \textit{seem} (with a PP experiencer). But crucially not both. I leave open here the explanation for the apparent limit to three arguments per predicate, noting, though, that as NPs, PPs, CPs and infinitival clauses (whether those are \isi{CP} or IP) all contribute towards the maximum, but only a subset of these bear \isi{Case}, any attempt to account for these effects in terms of \isi{Case} will necessarily cover only a subset of the generalization. 

\section*{Acknowledgments}

This squib arose out of a discussion with Susi Wurmbrand, \v{Z}eljko Bo\v{s}kovi\'c, and Mark Baker. I thank G\'isli R\'unar Har{\dh}arson, H\"oskuldur {\TH}r\'ainsson, and Jim Wood, for discussion of \ili{Icelandic} (and for the examples). For other feedback, I thank two anonymous reviewers, as well as Lisa Travis, Richard Larson and other audience members at McGill, EGG Brno, UConn, ReCoS Villa Salmi, and Stony Brook University.

\sloppy
\printbibliography[heading=subbibliography,notkeyword=this]
\end{document}