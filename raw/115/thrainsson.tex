\documentclass[output=paper]{LSP/langsci}
\author{Höskuldur Thráinsson\affiliation{University of Iceland}}

\title{On the softness of parameters: An experiment on Faroese}
% \epigram{Change epigram}
\abstract{%
This chapter evaluates the proposal, originally made by Anders Holmberg and Christer Platzack (e.g. \citeyear{Holmplat1995}), that several syntactic differences between \il{Scandinavian!Insular}Insular Scandinavian (ISc) on the one hand and \il{Scandinavian!Mainland}Mainland Scandinavian (MSc) on the other can be accounted for by postulating a single parameter that has one setting in ISc and another in MSc. While Faroese was originally supposed to belong to the ISc group, together with Icelandic, it has turned out that there is more variation in Faroese than in Icelandic with respect to the relevant syntactic phenomena. In this paper it is argued that it is exactly this variation within Faroese that makes it an interesting testing ground for hypotheses about parametric variation. It is then shown that while there is extensive intra-speaker variation in Faroese, there is some correlation between speakersʼ evaluation of sentences containing oblique subjects, Stylistic Fronting, null expletives\is{null expletive} and the transitive expletive construction, all supposedly typical ISc-phenomena. Although this correlation is not as strong as predicted by the standard parametric approach, it is intriguing and calls for an explanation. It is then suggested that a grammar competition account along the lines of \citet{Kroch1989} and \citet{Yang2002} provides a way of accounting for the observed data.
}
\ChapterDOI{10.5281/zenodo.1116753}

\maketitle

\begin{document}
% \href{mailto:hoski@hi.is}{\textit{hoski@hi.is}}


% \textbf{Keywords:} \ili{Insular Scandinavian}, \ili{Mainland Scandinavian}, \ili{Faroese}, parameters, \isi{grammar competition}, oblique subjects, \isi{Stylistic Fronting}, null expletives\is{null expletive}, transitive expletives

\section{Introduction}\label{sec:Thrainsson:1}
Comparative \ili{Scandinavian} syntax took a giant leap forwards in the late 1980s and early 1990s with the work of Christer Platzack and Anders Holmberg, joint and disjoint. The importance of their work on the nature and limits of syntactic variation in the \ili{Scandinavian} languages in the late 1980s and early 1990s (see \citealt{Holmplat1995} with references) can hardly be overestimated. The parameters they proposed guided research on \ili{Scandinavian} syntax for a long time and also had a more general effect on research into syntactic variation. Several researchers set out to test the predictions made by the proposed parameters and the general ideas behind them, or tried to refine them in different ways. As a result, various kinds of syntactic facts were discovered and syntacticians learned a lot about the nature of variation in general and in \ili{Scandinavian} syntax in particular.

Gradually, however, the whole parametric approach came under criticism, leading to a lively debate (see e.g. \citealt{Newmeyer2004,Newmeyer2005,Newmeyer2006}, \citealt{Haspelmath2008}, \citealt{Boeckx2011} vs. \citealt{Holmberg2010parameters}, \citealt{HolmbergEtAl2009}, \citealt{Roberts2005}; see also \citealt{BerwickEtAl2011} and \citealt{Sigurðsson2011Uniformity}). This particular debate mainly centered around the place and role (if any) of parameters in linguistic theory. The arguments were partly empirical (e.g. “Is there any evidence for the clustering of properties predicted by parameter A?”) and partly conceptual (e.g. “Is the concept of parameters compatible with the minimalist approach to language?”). Parallel to this debate, a different kind of discussion of the nature of parameters also emerged. In that discussion, one of the main issues is whether parameter values are acquired instantly (the \textsc{triggering approach}, cf. e.g. \citealt{Gibson1994}, \citealt{Lightfoot1999}) or gradually (the \textsc{variationist approach}, cf. e.g. \citealt{Yang2002,Yang2004,Yang2010}). Under the variationist approach to parametric setting, the child acquiring language will try out various possible grammars that are defined by the innate \isi{Universal Grammar} (UG) and these grammars will “compete” in the sense of \citet{Kroch1989,Kroch2001}. In the ideal situation, the target grammar will eliminate other possible grammars because these will only be compatible with some of the input but not all of it. This competition may take some time, depending on the amount and uniformity of relevant input, or as described by Yang:

\begin{quote}
[...] the rise of the target grammar is gradual, which offers a close fit with language development [...] non-target grammars stick around for a while before they are eliminated [...] the speed with which a parameter value rises to dominance is correlated with how incompatible its competitor is with the input \citep[454]{Yang2004}
\end{quote}

Although most of Yang’s work on parameters has revolved around the question of parameter settings by children during the acquisition period, his approach also has implications for the study of language variation, as he has pointed out:

\begin{quote}
In addition, the variational model allows the grammar and parameter probabilities to be values other than 0 and 1 should the input evidence be inconsistent; in other words, two opposite values of a parameter must coexist in a mature speaker. This straightforwardly renders Chomsky’s UG compatible with the Labovian studies of continuous variations at both individual and population levels [...] \citep[455]{Yang2004}
\end{quote}

It is tempting to relate this idea to Chomsky’s famous statement about the “ideal speaker-listener”:

\begin{quote}
Linguistic theory is concerned primarily with an ideal speaker-listener, in a completely homogenous speech-community, who knows its language perfectly ... \citep[3]{Chomsky1965}
\end{quote}

Under the standard assumption that linguistic parameters are binary,\footnote{Although this is the standard (and strongest) assumption, other values have also been proposed. But as \citet[541]{Roberts2005} state: “The only really substantive claim behind a binary formulation of parameters is that the values are discrete: there are no clines, squishes or continua.” This issue will be discussed in \sectref{sec:Thrainsson:5}.} we can then say that ideal speakers will have set all their parameter values to either + or – (1 or 0 if you will), but some speakers may not have fixed the setting for certain parameters. Instead they may be leaning towards either + or –, with different probabilities. In that sense their parameters can be said to be “soft”.\footnote{The formalization of this idea is a non-trivial issue. Saying that the relevant parameters are unspecified or have not yet been set is not a satisfactory description of the situation because the observed variation is not random, as we shall see. We will return to this issue in Sections~\ref{sec:Thrainsson:4} and \ref{sec:Thrainsson:5} below.}

It seems, however, that this approach to variation has been largely absent from studies of syntactic variation in \ili{Scandinavian} (but see \citealt{Thráinsson2013variation}, \citealt{Nowenstein2014}). Yet it would seem that comparative \ili{Scandinavian} syntax does in fact provide an ideal testing ground for ideas of this kind. One reason to believe so is the fact that inter- and intra-speaker variation seems much more prevalent in \ili{Scandinavian} syntax than previously assumed. This may be especially true of \ili{Faroese}, as will be discussed in the following sections.

The present paper reports on the results of a study of syntactic variation in \ili{Faroese}, referred to below as FarDiaSyn (for \ili{Faroese} Dialect Syntax). Because this study was much more extensive than any other research on \ili{Faroese}, both in terms of the number of speakers consulted and the number of constructions involved, it makes it possible to experiment with certain statistical methods to test parametric predictions. The study included the following phenomena among others: oblique subjects, \isi{Stylistic Fronting} (SF), null expletives\is{null expletive} and the Transitive Expletive Construction (TEC). All of these phenomena have been said to be related by Holmberg and Platzack’s \isi{Agr} parameter, as discussed below. As will be demonstrated, the results of FarDiaSyn are typically incompatible with the standard concept of strictly binary parameters because of the extensive intra-speaker variation observed. It will be argued that the variational approach suggested by Yang offers a more adequate account, to the extent that the results can be said to support any kind of parametric approach.

The paper is organized as follows: In \sectref{sec:Thrainsson:2}, Holmberg and Platzack’s Agr-para\-meter is reviewed, together with a selected set of facts that it is supposed to account for. In \sectref{sec:Thrainsson:3} I present data from \ili{Faroese} illustrating extensive inter- and intra-speaker variation with respect to evaluation of sentences involving oblique subjects, SF, null expletives\is{null expletive} and TEC. \sectref{sec:Thrainsson:4} then shows that despite the extensive variation, speaker judgments of these constructions correlate to some extent, although the correlations are not as general nor as strong as \citet{Holmplat1995} would have led us to expect. \sectref{sec:Thrainsson:5} is the conclusion.

\section{Holmberg and Platzack’s Agr-parameter revisited}\label{sec:Thrainsson:2}

As is well known, the Principles and Parameters (P{\&}P) approach to language variation goes back to Chomskyʼs \textit{Lectures on Government and Binding} (\citeyear{Chomsky1981}). The basic prediction of the P{\&}P approach is that “[i]nsofar as linguistic variation is due to variation with regard to parameters [...] we should find clusters of surface effects of these deep-lying parameters in the languages of the world” \citep[4]{Holmberg2010parameters}. If such a cluster consists of, say, four properties, every language should in principle either have all four of them or none of them, “all else being equal” \citep[5]{Holmberg2010parameters}.

Holmbergʼs paper just cited was partially a reaction to the claim advanced by several researchers, including \citet{Newmeyer2004,Newmeyer2005}, \citet{Haspelmath2008} and \citet{Boeckx2011}, that proposed parametrically conditioned clusters of surface effects “invariably fail to hold up when a wider range of languages are taken into account” \citep[12]{Holmberg2010parameters}. In an attempt to refute this claim, Holmberg sets out to reconsider the effects of the so-called Agr-parameter proposed in various works by himself and Christer Platzack in the late 1980s and early 1990s. This parameter was supposed to account for a number of syntactic differences between \il{Scandinavian!Insular}Insular Scandinavian (ISc) on the one hand and \il{Scandinavian!Mainland}\il{Mainland Scandinavian} (MSc) on the other. In earlier work by Holmberg and Platzack (henceforth H{\&}P) the parameter was believed to account for up to ten differences between ISc and MSc but \citet[13--14]{Holmberg2010parameters} reduces it to the following seven:

\ea%1
    \label{ex:Thrainsson:1}
    \begin{tabular}[t]{llll}
    \multicolumn{2}{l}{Holmbergʼs reduced list of Agr-related differences:} &        ISc   & MSc     \\
    1. & Rich subject-verb \isi{agreement}           &           +  &    –     \\
    2. & \isi{Oblique subjects}                      &        +     & –        \\
    3. & \isi{Stylistic Fronting}                    &          +   &     –    \\
    4. & Null expletives\is{null expletive}                       &       +      &   –      \\
    5. & Null generic subject pronoun          &            + &     –    \\
    6. & \isi{Transitive expletives}                 &         +    &  –       \\
    7. & Heavy subject postposing              &          +   &   –      \\
    \end{tabular}
    \stepcounter{xnumi}
\z

Although H{\&}P included Old \ili{Norse} and \ili{Faroese} in the ISc group together with \ili{Icelandic}, Holmberg only contrasts \ili{Icelandic} with MSc in this later paper (\citeyear{Holmberg2010parameters}) “to simplify the presentation”. It would obviously complicate the comparison to include a dead language like Old \ili{Norse}, although we now have more sophisticated tools to study that language than before (see e.g. \citealt{Rögnvaldsson2011,RögnvaldssonEtAl2011,Thráinsson2013oldnorse}). About the exclusion of \ili{Faroese} from the ISc vs. MSc comparison in the paper, Holmberg makes the following remark:

\begin{quote}
\ili{Faroese} is an interesting case in this connection, since it is undergoing changes that seem to crucially involve the parameter discussed in the text below. (\citealt{Holmberg2010parameters}:13n)
\end{quote}

If true, this indeed makes \ili{Faroese} especially interesting for the following reasons among others:

\begin{enumerate}[ref=\arabic*,label=(\arabic*),start=2] \item \label{ex:thrainsson:2}\begin{enumerate}[label=\arabic*.]
    \item If \ili{Faroese} is “undergoing changes that seem 
    to crucially involve the parameter” in question, this means that speakers acquiring \ili{Faroese}, growing up and living in the modern \ili{Faroese} society will be exposed to variable linguistic input.

    \item Under Yangʼs variationist approach to parametric setting (\citeyear{Yang2004}), this predicts that we should not only find extensive inter-speaker variation in \ili{Faroese} with respect to the relevant syntactic constructions but also considerable intra-speaker variation since the variationist model “allows the grammar and parameter probabilities to be values other than 0 and 1 should the input evidence be inconsistent” (cf. \citealt[455]{Yang2004}).

    \item Under the triggering approach to parametric setting described above (see e.g. \citealt{Gibson1994}, \citealt{Lightfoot1999} and later work), the observed variation in the \ili{Faroese} linguistic community should be the result of different parametric settings by speakers acquiring the language. Because the input is inconsistent, it will trigger the parametric value 1 for some speakers but 0 for others. Extensive intra-speaker variation in the relevant constructions is not predicted by the triggering approach.

    \item If the constructions under discussion are related by a single parameter, there should be a very strong correlation between judgments of all the relevant constructions under the triggering approach to parametric setting. Under the variationist approach we would also expect some correlation between the judgments, although not necessarily particularly strong because various grammar-external factors may influence the judgments when there is optionality.\footnote{Such “grammar-external factors” would include stylistic differences and issues having to do with pragmatics and discourse phenomena, which some speakers may be more sensitive to than others.} If the constructions under discussion are unrelated and governed by language-particular rules (e.g. in the sense of \citealt{Newmeyer2004,Newmeyer2005}), it is less clear what kind of correlations to expect, if any (more on this in Sections~\ref{sec:Thrainsson:4} and \ref{sec:Thrainsson:5} below).
\end{enumerate} \end{enumerate}

In the next section I will present some results from FarDiaSyn that can be used to test these predictions. This particular part of FarDiaSyn only included a subset of the constructions on Holmbergʼs reduced list of Agr-related differences in \REF{ex:Thrainsson:1} above, namely the following:

\ea%3
    \label{ex:Thrainsson:3}
    \begin{tabular}[t]{llll}
    \multicolumn{2}{l}{Agr-related differences tested in FarDiaSyn:} &        ISc   & MSc     \\
    1. & \isi{Oblique subjects}                      &        +     & –        \\
    2. & \isi{Stylistic Fronting}                    &          +   &     –    \\
    3. & Null expletives\is{null expletive}                       &       +      &   –      \\
    4. & \isi{Transitive expletives}                 &         +    &  –       \\
    \end{tabular}
\z
H{\&}P have illustrated the \ili{Icelandic} vs. MSc differences as follows (these examples are mainly taken from \citealt{Holmberg2010parameters} but (\ref{ex:Thrainsson:4}a,b) and (\ref{ex:Thrainsson:6}c,d) are taken from H{\&}Pʼs book \citeyear{Holmplat1995}: 11):

\ea%4
    \settowidth\jamwidth{(Ice)}
    \label{ex:Thrainsson:4}
%     \langinfo{lg}{fam}{src}\\
    \isi{Oblique subjects}\\
    \ea[]{
    \gll \textbf{Hana}   vantar   peninga.\\
	her\textsc{.acc}   lacks    money.\\ \jambox{(Ice)}
    \glt ʽShe needs money.ʼ}
    \ex[*]{
    \gll \textbf{Henne}   saknar  pengar.\\
          her      lacks    money\\ \jambox{(Sw)}}
    \ex[]{
    \gll \textbf{Mér} voru   gefnir   peningar.\\
    me\textsc{.dat}  were   given     money\\ \jambox{(Ice)}
    \glt ‘I was given money.’}
    \ex[*]{
    \gll \textbf{Mej} blev   givet/givna    pengar.\\
          me   was   given.\textsc{sg/pl}    money.\textsc{pl}\\ \jambox{(Sw)}}
    \z
\z


\ea%5
    \settowidth\jamwidth{(Ice)}
    \label{ex:Thrainsson:5}
    \isi{Stylistic Fronting} (SF)\\
    \ea[]{%
    \gll [Þeir sem \textbf{í} \textbf{Osló} hafa búið]     segja   að     það   sé     fínn   bær.\\
	those that in Oslo have lived   say   that   it     is     nice   town\\ \jambox{(Ice)}
    \glt ‘Those that have lived in Oslo say that itʼs a nice town.ʼ}
    \ex[*]{%
    \gll [De som \textbf{i} \textbf{Oslo} har bott]       säger   att     det   är     en fin stad.\\
	  those that in Oslo have lived    say  that  it    is    a   nice town\\\jambox{(Sw)}
    }
    \z
\z

\ea%6
    \settowidth\jamwidth{(Ice)}
    \label{ex:Thrainsson:6}
    Null expletives\is{null expletive}\\
    \ea
    \gll Nú   rignir     (*\textbf{það}).\\
	now   rains     it\\\jambox{(Ice)}
    \glt ‘Now it’s raining.’
    \ex
    \gll  Nu   regnar   *(\textbf{det}).\\
	   now  rains    it\\\jambox{(Sw)}
    \ex
    \gll {Í gær}       var   (*\textbf{það})   dansað   á skipinu  \\
         yesterday    was    there    danced  on the-ship\\\jambox{(Ice)}
    \ex
    \gll Igår       dansades       *(\textbf{det})    på  skeppet.     \\
         yesterday    was-danced    there    on the-ship\\\jambox{(Sw)}

    \z
\z

\ea%7
    \settowidth\jamwidth{(Ice)}
    \label{ex:Thrainsson:7}
    \isi{Transitive Expletive Construction} (TEC)\\
    \ea[]{%
    \gll \textbf{Það}   hefur einhver   köttur   étið     mýsnar.\\
	 there   has   some     cat     eaten     the-mice\\ \jambox{(Ice)}
    \glt}
    \ex[*]{%
    \gll \textbf{Det}   har   ein     katt     eti       mysene.\\
         there   has   a       cat     eaten     the-mice\\\jambox{(No)}}
    \z
\z

As can be seen, the MSc data come from \ili{Swedish} and \ili{Norwegian}, but \ili{Danish} data could just as well have been used.

\section{The Faroese experiment}\label{sec:Thrainsson:3}
\subsection{The elicitation methods of FarDiaSyn}

As mentioned above, recent studies of \ili{Faroese} indicate that there is considerable variation in \ili{Faroese} syntax. This means that in order to get reliable and statistically significant results about possible covariation of particular constructions, the study has to be quite extensive (see also the discussion in \citealt{Thrainsson2017}). Under Yang’s variationist approach, one would assume that probability of a given parameter setting for the relevant parameter for each speaker should predict how the speaker would judge sentences that are related by that particular parameter, “all else being equal”. But because other things are not always equal (e.g. because of lexical differences, different sensitivity to stylistic or pragmatic phenomena, etc.), these predictions are most reliably tested in studies that involve a reasonably large sample of the relevant sentences and a large number of speakers from different age groups and with a varying background.

In the study reported on here, 334 speakers of \ili{Faroese} were asked to evaluate selected sentences. The speakers came from different parts of the Faroes, they ranged in age from approximately 15--70 and there was an even split between male and female speakers (for a more detailed description of the population see \citealt{Thrainsson2017}). The evaluation method was typically one where the speakers were asked to check one of three possibilities on a written questionnaire as illustrated in \figref{ex:Thrainsson:8} (the instructions were given in \ili{Faroese}, of course, but here they have been translated into English).


%\ea
\begin{figure}
\caption{Questionnaire}\label{ex:Thrainsson:8}
  \begin{tabular}[t]{lll}
  \multicolumn{3}{l}{Put an X in the appropriate column:}\\
  yes  & = & A natural sentence. I could very well have said this.\\
  ?    & = & A doubtful sentence. I could hardly say this.\\
  no   &  =&  An unnatural or impossible sentence. I could not say this.\\
  \end{tabular}
  \begin{tabular}{|l|l|l|l|l|} \hhline{-----} & yes & ? & no & Comments\\\hhline{-----}
  \textit{Teir sjey dvørgarnir vóru í øðini.} & \multicolumn{1}{c}{\cellcolor{black!25}} & \multicolumn{1}{c}{\cellcolor{black!25}} & \multicolumn{1}{l|}{\cellcolor{black!25}} & \\
  The seven dwarfs were upset. & \multicolumn{1}{c}{\cellcolor{black!25}} & \multicolumn{1}{c}{\cellcolor{black!25}} & \multicolumn{1}{l|}{\cellcolor{black!25}} & \\ \hhline{~---~}
  Tað hevði onkur etið súreplið. & & & &  \\
  there had somebody eaten the-apple & & & & \\ \hhline{-----}
  \end{tabular}
%\z
\end{figure}
In addition, the subjects were also asked to choose between two (or sometimes three) alternatives in a setup like in \figref{ex:Thrainsson:9} (again, the instructions have been translated from \ili{Faroese}).
\begin{figure}\caption{Multiple choice test}
    \label{ex:Thrainsson:9}\framebox[\textwidth]{\parbox{.95\textwidth}{\vspace{.025\textwidth}%
    In the following examples you are asked to compare two possible alternatives in each sentence. Check the most natural one. Check both if you find them equally natural.\\
    \begin{exe}
    \sn{
    \gll Tað   regnar   ongantíð í Sahara.\\
    it       rains     never         in Sahara\\
	}\sn{
    \gll Í Havn $\def\arraystretch{.8}\begin{array}[c]{l}\\\\\square \text{regnar}\\\text{rains}\\[.5\baselineskip]\\\square \text{regnar tað}\\\text{rains it}\end{array}$  ofta.\\
    in Tórshavn {} often\\}
    \end{exe}
   }}
\end{figure}

Although the speakers were given the possibility to select both alternatives in this kind of task, they very rarely did so.

We now present the results for each of the constructions under consideration.

\subsection{Oblique subjects}\largerpage
Modern \ili{Icelandic} is famous for its oblique subjects, which can occur in the \isi{Accusative}, Dative and Genitive. \isi{Nominative} is obviously the default or structural subject case in \ili{Icelandic}, Genitive subjects are very rare, Acc subjects arguably irregular (quirky) in many instances but Dat subjects sometimes thematically related: Experiencer subjects often show up in the Dat in \ili{Icelandic} and some verbs previously taking Acc subjects now take Dat subjects in the language of many speakers (the (in)famous Dative Substitution or Dative Sickness, see e.g. \citealt{ZaenenEtAl1985}, \citealt[224]{Thráinsson2007}). Gen subjects have completely disappeared in \ili{Faroese} and Acc subjects have also virtually died out (see e.g. \citealt[252--251]{Thráinsson2012}, \citealt{Jónsson2005}, \citealt{Eythorsson2015}). A few verbs still take Dat subjects but in many instances there is variation between Dat and Nom.\footnote{Barnes claims (\citeyear[28]{Barnes1992}) that Nom is replacing Dat as a subject case in spoken \ili{Faroese}, especially among younger people. In our study younger speakers were somewhat less likely to accept Dat subjects in the examples we tested. Although the correlation between judgments and age was rather weak, it was statistically significant  for three of the four verbs listed in \REF{ex:Thrainsson:10} (it was not significant in the case of the loan verb \textit{mangla} ʽneed, lackʼ).}\linebreak Hence both variants were tested in FarDiaSyn as shown in the following examples:

\ea%10
    \label{ex:Thrainsson:10}
	\begin{xlist}[a2.]
	\exi{a1.}\label{ex:Thrainsson:10a1}
	\textit{Bilurin hjá Óla hevur verið til sýn.}\\
	\glt ‘Óli’s car has been inspected.’\\
	\gll \textbf{Honum}   \textbf{tørvar}   ikki   at   hugsa   meira   um   tað.\\
		him\textsc{.dat}  needs    not    to  think    more  about  that\\
	\glt ‘He doesn’t have to think more about that.’\\

	\exi{a2.}  \label{ex:Thrainsson:10a2}
	\textit{Hans veit ikki nógv um fiskiskap.}\\
	\glt ‘Hans doesn’t know much about fishing.’\\
	\gll \textbf{Hann} \textbf{tørvar}   ikki   at   hava   svar     til alt.\\
		he\textsc{.nom}  needs    not    to  have  answer  to  everything\\
	\glt ‘He doesn’t have to have answers to everything.’
	\exi{b1.}\label{ex:Thrainsson:10b1}
	\textit{Turið hevur sæð nógvar filmar}.
	\glt ‘Turið has seen many films.’\\
	\gll \textbf{Henni}   \textbf{dámar}   at   hyggja   í sjónvarp.\\
		  her\textsc{.dat}  likes    to  look    at  TV\\
	\glt ‘She likes to watch TV.’

	\exi{b2.}\label{ex:Thrainsson:10b2}
	\textit{Sára fer á konsertina í kvøld.}\\
	\glt ‘Sára going to the concert tonight.’\\
	\gll \textbf{Hon}     \textbf{dámar} at   lurta   eftir   tónleiki.\\
		she\textsc{.nom}  likes    to  listen  after  music\\
    \glt ‘She likes to listen to music.’

    \exi{c1.} \label{ex:Thrainsson:10c1}
    \textit{Kári hevur nógv at gera.}\\
    \glt  ‘Kári has a lot to do.’\\
    \gll  \textbf{Honum}   \textbf{manglar} at   gera   húsini     liðug.\\
           him\textsc{.dat}  needs    to  make  the-houses   ready\\
    \glt ‘He needs to finish the house.’

    \exi{c2.} \label{ex:Thrainsson:10c2}
    \textit{Anton reypar av at vera góður kokkur.}\\
    \glt  ‘Anton brags about beeing a good cook.’\\
    \gll \textbf{Hann}   \textbf{manglar} at   prógva   tað   í verki.\\
          he\textsc{.nom}  needs    to  prove    it    in  work\\
    \glt ‘He needs to prove it in action.’
\newpage
    \exi{d1.} \label{ex:Thrainsson:10d1}
    \textit{Stjórin hjá Súsannu ar altíð ov seinur til arbeiðis}.\\
    \glt  ‘Súsanna’s boss always comes too late to work.’\\
    \gll \textbf{Henni}   \textbf{nýtist}   ikki   at   hugsa   um klokkuna.\\
          her\textsc{.dat}  needs    not    to  think    about  the-clock\\
	\glt ‘She doesn’t have to think about the clock.’

	\exi{d2.} \label{ex:Thrainsson:10d2}
    \textit{Elin kennir øll tey ríku og kendu.}\\
	\glt ‘Elin knows all the rich and famous.’\\
    \gll \textbf{Hon}     \textbf{nýtist} ikki   at   standa   í bíðirøð.\\
	      she\textsc{.nom}  need  not    to  stand    in line\\
    \glt ‘She doesn’t have to stand in line.’
	\end{xlist}
\z


The evaluation of these examples is shown in \tabref{tab:Thrainsson:1} (percentages for the more positively evaluated variant highlighted by boldface):

\begin{table}
\begin{tabularx}{\textwidth}{lXrSrSrS}\lsptoprule &  & \multicolumn{2}{c}{{Yes}} & \multicolumn{2}{c}{{?}} & \multicolumn{2}{c}{{No}}\\\cmidrule(lr){3-4}\cmidrule(lr){5-6}\cmidrule(lr){7-8}
 {\#} & {Example} & \multicolumn{1}{c}{N} & \multicolumn{1}{c}{\%} & \multicolumn{1}{c}{N} & \multicolumn{1}{c}{\%} & \multicolumn{1}{c}{N} & \multicolumn{1}{c}{\%}\\\midrule
(\ref{ex:Thrainsson:10}a1) & \textbf{Honum tørvar} ikki at hugsa meira um tað. & 238 & \textbf{73.0} & 36 & 11.0 & 52 & 16.0\\
(\ref{ex:Thrainsson:10}a2) & \textbf{Hann t}\textbf{ørvar} ikki at hava svar til alt. & 89 & 27.6 & 89 & 27.6 & 145 & 44.9\\
(\ref{ex:Thrainsson:10}b1) & \textbf{Henni dámar} at hyggja í sjónvarp. & 287 & \textbf{86.7} & 24 & 7.3 & 20 & 6.0 \\
(\ref{ex:Thrainsson:10}b2) & \textbf{Hon dámar} at lurta eftir tónleiki. & 208 & 62.8 & 55 & 16.6 & 68 & 20.5\\
(\ref{ex:Thrainsson:10}c1) & \textbf{Honum manglar} at gera húsini liðug. & 196 & 60.1 & 62 & 19.0 & 68 & 20.9\\
(\ref{ex:Thrainsson:10}c2) & \textbf{Hann manglar} at prógva tað í verki. & 241 & \textbf{73.7} & 31 & 9.5 & 55 & 16.8\\
(\ref{ex:Thrainsson:10}d1) & \textbf{Henni nýtist} ikki at hugsa um klokkuna. & 246 & \textbf{75.0} & 36 & 11.0 & 46 & 14.0\\
(\ref{ex:Thrainsson:10}d2) & \textbf{Hon nýtist} ikki at standa í bíðirøð. & 210 & 64.4 & 49 & 15.0 & 67 & 20.6\\
\lspbottomrule
\end{tabularx}
\caption{Evaluation of Dat and Nom subjects with selected verbs in FarDiaSyn.}
\label{tab:Thrainsson:1}
\end{table}

\clearpage Interesting descriptive facts revealed by this table include the following:
\begin{enumerate}
 \item For three out of the four verbs, Dat is more generally accepted than Nom.
 \item There is clearly some intra-speaker variation in subject case assignment for at least three of these verbs (\textit{dáma}, \textit{mangla} and \textit{nýtast}) since the proportion of speakers accepting a Dat subject plus the proportion of speakers accepting a Nom subject is way over 100\% for these verbs. In other words, some speakers, but not all, accept both a Dat and a Nom subject for these verbs.
 \item The only verb where Nom is more generally accepted than Dat is the \ili{Danish} loanword \textit{mangla} ʽneed, lackʼ in (\ref{ex:Thrainsson:10}c). Since this verb is a (possibly rather recent) loan from \ili{Danish},\footnote{The \ili{Faroese}-\ili{Faroese} dictionary \textit{\citetitle{Føroysk1998}} (\citealt{Føroysk1998}) states that it is ``colloquial'' or belongs to the spoken language (Fa. \textit{talað mál}).} this is perhaps not so surprising. It is in fact more interesting that 60\% of the speakers accept it with a Dat subject since this shows that assignment of Dat to subjects is still alive in \ili{Faroese} (or was at the time when this verb was adopted into the language) and not just an old relic.
\end{enumerate}

This last point is consistent with the general belief that assignment of Dat case to subjects in \ili{Faroese} is not (or has not been) irregular or quirky.

While the facts summarized in \tabref{tab:Thrainsson:1} indicate considerable variation in the evaluation of Dat and Nom subjects, this method of presenting the data does not really show very clearly to what extent this is inter-speaker variation and to what extent the judgments of the same speaker may vary (intra-speaker variation). But \figref{fig:Thrainsson:1} shows that considerable intra-speaker variation is involved in the evaluation of Dat subjects. Here the answers to the questionnaire have been coded as follows (cf. the illustration in \figref{ex:Thrainsson:8} above): \textit{yes} = 3, ? = 2 and \textit{no} = 1. This means that if a speaker accepted all four Dat subject examples, (s)he would get the average score (or “grade)” of 3, if (s)he rejected all of them the score would be 1, etc.



\begin{figure}
\includegraphics[height=.4\textheight]{figures/ThrainssonFigure3.pdf}
\caption{Judgments of Dat subject sentences\label{fig:Thrainsson:1}.}
% \begin{tikzpicture}
% \begin{axis}[
%   ylabel={Frequency},
%   xlabel={Distribution of evaluations of Dat subjects},
%   ylabel near ticks,
%   xlabel near ticks,
%   nodes near coords,
%   xtick=data,
%   enlarge y limits={abs=20,upper},
%   legend pos=outer north east,
%   legend cell align=left,
%   legend style={draw=none},
%   ymin=0
% ]
% \addlegendimage{empty legend}\addlegendentry{%
% \frame{\begin{tabular}{l@{ }c@{ }S[table-format = 4.6,group-digits=false]}
% Mean & = & 2.5986\\
% Std. Dev & = & 0.48745\\
% N & = & 334\\
% \end{tabular}}}
% \addplot[ybar,fill=gray] plot coordinates {(1,4) (1.25,5) (1.5,12) (1.75,12) (2,24) (2.25,25) (2.5,60) (2.75,47) (3,145)};
% \end{axis}
% \end{tikzpicture}
\end{figure}

As shown here, 145 out of 334 speakers accepted all the Dat subject sentences and only four rejected all of them. But more than half accepted some and rejected others, or found the examples doubful. If acceptance of Dat subjects were governed by a strictly binary setting of a parameter, we would expect a more clear cut result than this.

\subsection{Stylistic Fronting}

As originally described by \citet{Maling1980}, \isi{Stylistic Fronting} (henceforth SF) fronts a constituent in a clause with a “subject gap”. There has been some controversy as to whether all fronting of constituents in such clauses should be considered SF or whether SF only fronts heads and fronting of a maximal projection (e.g. a PP) is a case of Topicalization, also when a subject gap is involved (for a review of the issues see \citealt[368--374]{Thráinsson2007}). As pointed out by H{\&}P and discussed by several linguists (e.g. \citealt{Barnes1992}, \citealt{Vikner1995}, \citealt{Thráinsson2012}, \citealt{Angantýsson2011}), SF also occurs in \ili{Faroese}, as it should if it is related to a positive setting of H{\&}Pʼs Agr-parameter and \ili{Faroese} is a true ISc language. In FarDiaSyn the following examples were used to test the speakersʼ acceptance of SF (fronted elements in boldface):

\ea%12
\label{ex:Thrainsson:12}
	\ea \small\let\eachwordone=\small\let\eachwordtwo=\small\textit{Studentarnir fingu summarfrí í gjár}.\\
	\glt ʽThe students got summer vacation yesterday.ʼ\\
	\gll Skúlastjórin     helt   talu     fyri   teimum,   sum   \textbf{liðug}   vóru   við   skúlan.\\
	the-principal   held   speech   for   those     that   done   were   with   the-school\\
	\glt ʽThe principal gave a speech for those who were graduating.ʼ
\newpage 
	\ex \normalsize\let\eachwordone=\normalsize\let\eachwordtwo=\normalsize\textit{Olga hevur ikki vaskað sær í fleiri dagar.}\\
	\glt ʽOlga hasnʼt washed for several days.ʼ\\
	\gll Hon   fer   ikki   í baðikarið,     um \textbf{har}         hava   verið   mýs.\\
	she  goes  not    in the-bathtub   if    in-that-place  have   been   mice\\
    \glt	ʽShe doesnʼt go into the bathtub if there have been mice there.ʼ

    \ex \small\let\eachwordone=\small\let\eachwordtwo=\small\textit{Fjórða barnið er á veg hjá Róa og Poulu.}\\
	\glt    ʽRói and Paula are expecting their fourth child.ʼ\\
    \gll    Tey   vilja   keypa   ein bil,   sum   \textbf{vælegnaður}   er   til   eina   barnafamilju.\\
            they  want  buy    a car    that  well-suited    is   for a     family-with-children\\
    \glt    ʽThey want to buy a car that is suitable for a family with children.ʼ

	\ex \normalsize\let\eachwordone=\normalsize\let\eachwordtwo=\normalsize\textit{Kokkurin hevði ikki gjørt nóg mikið av mati.}\\
	\glt ʽThe cook hadnʼt prepared enough food.ʼ\\
	\gll Øll,   sum   \textbf{einki}     høvdu   etið,   vóru   svong.\\
	all     that   nothing   had     eaten   were   hungry\\
	\glt ʽEverybody who hadnʼt eaten anything was hungry.ʼ

	\ex \textit{Kommunuval var í Føroyum í gjár.}\\
	\glt ʽMunicipal elections were held in the Faroes yesterday.ʼ\\
	\gll	Tillukku       til øll,   sum   \textbf{vald}     vórðu.\\
		congratulations   to all     that   elected   were \\
	\glt ʽCongratulations to all who were elected.ʼ

	\ex \textit{Samráðingar verða í annaðkvøld.}\\
	\glt	ʽThere will be negotiations tomorrow night.ʼ\\
	\gll	Lønarhækking  er   tað,   sum     \textbf{ovast}     er   á   breddanum.\\
			salary-raise    is  that  which   topmost   is   on the-page  \\
	\glt 	ʽSalary raise is at the top of the agenda.ʼ

	\ex \textit{Eg fari til Prag í Kekkia í næstu viku}.\\
	\glt	ʽIʼm going to Prague in the Czech Republic next week.ʼ\\
	\gll	Kennir   tú     onkran,   sum   \textbf{verið}  hevur   í   Kekkia?\\
			know    you  anybody   that   been   has   in  Czech-Republic  \\
	\glt	ʽDo you know anybody that has been to the Czech Republic?ʼ
	\z
\z

As can be seen from this list, the sentences contain fronted elements of different kinds, mostly in \isi{relative} clauses, but for the reasons described above we avoided examples with fronted constituents that would unambiguously be analyzed as maximal projections (these could arguably involve Topicalization rather than SF). The evaluation of these examples is illustrated in \tabref{tab:Thrainsson:2} (the highest percentages for each sentence in boldface):

\begin{table}[t]
\begin{tabularx}{\textwidth}{lXrSrSrS}\lsptoprule &  & \multicolumn{2}{c}{{Yes}} & \multicolumn{2}{c}{{?}} & \multicolumn{2}{c}{{No}}\\\cmidrule(lr){3-4}\cmidrule(lr){5-6}\cmidrule(lr){7-8}
 {\#} & {Example} & \multicolumn{1}{c}{N} & \multicolumn{1}{c}{\%} & \multicolumn{1}{c}{N} & \multicolumn{1}{c}{\%} & \multicolumn{1}{c}{N} & \multicolumn{1}{c}{\%}\\\midrule
(\ref{ex:Thrainsson:12}a) & Skúlastjórin helt talu fyri teimum, sum \textbf{liðug} vóru við skúlan. & 182 & \textbf{55.3} & 73 & 22.2 & 74 & 22.5\\
(\ref{ex:Thrainsson:12}b) & Hon fer ikki í baðikarið, um \textbf{har} hava verið mýs. & 155 & \textbf{47.3} & 65 & 19.8 & 108 & 32.9\\
(\ref{ex:Thrainsson:12}c) & Tey vilja keypa ein bil, sum \textbf{vælegnaður} er til eina barnafamilju. & 102 & 31.1 & 77 & 23.5 & 149 & \textbf{45.4}\\
(\ref{ex:Thrainsson:12}d) & Øll, sum \textbf{einki} høvdu etið, vóru svong & 231 & \textbf{70.4} & 48 & 14.6 & 49 & 14.9\\
(\ref{ex:Thrainsson:12}e) & Tillukku  til øll, sum \textbf{vald} vórðu. & 170 & \textbf{52.1} & 72 & 22.1 & 84 & 25.8\\
(\ref{ex:Thrainsson:12}f) & Lønarhækking er tað, sum \textbf{ovast} er á breddanum. & 170 & \textbf{52.5} & 67 & 20.7 & 87 & 26.9\\
(\ref{ex:Thrainsson:12}g) & Kennir tú onkran, sum \textbf{verið} hevur í Kekkia? & 128 & 39.0 & 52 & 15.9 & 148 & \textbf{45.1}\\
\lspbottomrule
\end{tabularx}
%%please move \begin{table} just above \begin{tabular .
\caption{Evaluation of Stylistic Fronting in FarDiaSyn.}
\label{tab:Thrainsson:2}
\end{table}

Again, we find considerable variation, but more speakers accept than reject most of the examples (examples \ref{ex:Thrainsson:12}c and \ref{ex:Thrainsson:12}g are an exception). The reason for this extensive variation could be that SF is probably stylistically marked, i.e. it may not belong to the colloquial style that the subjects were asked to have in mind when evaluating the examples.

As  before, we can check how the judgments spread, e.g. whether any of the speakers accept all of the SF-examples or reject all of them. This is shown on \figref{fig:Thrainsson:2}.



\begin{figure}
\includegraphics[height=.4\textheight]{figures/ThrainssonFigure4.pdf}
\caption{Judgments of Stylistic Fronting.
% \todo[inline]{DATA MISSING! Here, we have nine possible judgments, but only eight data points!}
\label{fig:Thrainsson:2}}
% \begin{tikzpicture}
% \begin{axis}[
%   ylabel={Frequency},
%   xlabel={Distribution of evaluations of SF},
%   ylabel near ticks,
%   xlabel near ticks,
%   nodes near coords,
%   xtick=data,
%   enlarge y limits={abs=10,upper},
%   legend pos=outer north east,
%   legend cell align=left,
%   legend style={draw=none},
%   ymin=0
% ]
% \addlegendimage{empty legend}\addlegendentry{%
% \frame{\begin{tabular}{l@{ }c@{ }S[table-format = 4.6,group-digits=false]}
% Mean & = & 2.1929\\
% Std. Dev & = & 0.46651\\
% N & = & 333\\
% \end{tabular}}}
% \addplot[ybar,fill=gray] plot coordinates {(1,8) (1.25,18) (1.5,38) (1.75,39) (2,61) (2.25,71) (2.5,69) (2.75,15) (3,0)};
% \end{axis}
% \end{tikzpicture}
\end{figure}
\largerpage[-2]
As shown here, very few subjects accept all of the SF-examples (only 15) and very few reject all of them (only 8). Most speakers accept some — typically more than half of them. This is somewhat unexpected if the acceptance of SF is governed by a binary parameter. But note that SF is an optional operation: In \isi{relative} clauses the subject gap can be left “empty” as it were and subject gaps can also be “filled” with an expletive, e.g. in examples like (\ref{ex:Thrainsson:12}b).\footnote{Holmberg has in fact argued (\citeyear{Holmberg2000}) that the element fronted in SF serves the same function as an expletive. One problem with his analysis is the fact that SF-elements and the expletive \textit{það} ʽthereʼ do not have the same distribution in \ili{Icelandic}: SF-elements can fill certain “subject gaps” that the expletive \textit{það} cannot (see e.g. \citealt[351]{Thráinsson2007}):
	
	\begin{xlist}
		\exi{(i)}
		\begin{xlista}
		\ex[]{\gll Þetta   er   mál   sem   {\uline{~~~~}}   hefur   verið   rætt.\\
				this  is  matter  that     {}       has    been  discussed\\}
		\ex[]{Þetta  er  mál  sem  \textbf{rætt}  hefur  verið  \uline{~~~~}\\}

		\ex[*]{\gll  Þetta   er  mál  sem  \textbf{það}  hefur  verið   rætt.\\
				   this  is  matter  that  there  has    been  discussed\\}
	    \end{xlista}
	\end{xlist}
Similar subject gaps can either be filled with an SF-element or an expletive in \ili{Faroese} so in that sense Holmbergʼs suggestion arguably works better for \ili{Faroese} than \ili{Icelandic}  (see e.g. \citealt[170]{Angantýsson2011}):

\begin{xlist}
	\exi{(ii)}
	\begin{xlista}
		\ex \gll Hetta  eru   mál    sum  \uline{~~~~}    hevur  verið  tosað  um.\\
		these  are   matters  that   {}   has    been  talked  about \\
		\ex   Hetta  eru   mál    sum  \textbf{tosað} hevur  verið  \uline{~~~~}    um.\\
		\ex Hetta  eru   mál    sum  \textbf{tað}    hevur  verið  tosað  um.\\
	\end{xlista}
\end{xlist}}
\largerpage
The choice between the alternatives is probably “stylistic” in nature to some extent (hence the name \isi{Stylistic Fronting}). Thus it is not given a priori that somebody will find a particular example of SF appropriate even if\largerpage SF is in principle possible in his or her grammar.

\subsection{Null expletives}

As discussed by many researchers, \ili{Icelandic} is famous for its null expletives\is{null expletive} (see e.g. \citealt[477--484]{Thráinsson1979}, \citealt[309--313]{Thráinsson2007}, \citealt[Chapter 6.3]{Sigurðsson1989}) and H{\&}P originally assumed that \ili{Faroese} works essentially the same way, as an ISc language should. Since linguists do not always mean the same thing when they talk about null expletives\is{null expletive}, the discussion here is limited to null expletives\is{null expletive} of the kind illustrated by H{\&}P with examples like those in \REF{ex:Thrainsson:6}, namely ones where some non-subject (or the \isi{finite} verb) is fronted in a main clause and an overt expletive would be obligatory in MSc but impossible in \ili{Icelandic}. Because it had been pointed out previously that there is some optionality in constructions of this sort in \ili{Faroese} (i.e. that the expletive can either be overt or non-overt, cf. e.g. \citealt{Vikner1995}:227, \citealt[285--288]{Thráinsson2012}), we tested both options, sometimes in pairs of sentences that differed only minimally. The relevant examples are shown in (\ref{ex:Thrainsson:13}--\ref{ex:Thrainsson:15}). The first set contains \isi{impersonal passive}s with and without an overt \isi{expletive}:

\ea%13
    \label{ex:Thrainsson:13}
    \ea \textit{Fyrr í tíðini vóru ongar teldur og einki sjónvarp}.\\
	\glt    ʽIn the old days there were no computers and no TV.ʼ\\
	\gll    Tá   varð   nógv   dansað   heima   við   hús.  \\
	        then  was  much  danced  home    with   house\\
    \glt    ʽThen there was a lot of dancing at home.ʼ
	\ex \textit{Fyrr sótu fólk í roykstovuni og arbeiddu}.\\
	\glt	ʽPreviously people would sit in the living room and work.ʼ\\
	\gll	Tá   varð   \textbf{tað}   tosað   saman   um     kvøldarnar.  \\
			then   was   there   talked together   during   the-evenings    \\
	\glt 	ʽThen people would talk during the evening.ʼ
	\ex	\textit{Stórt brúdleyp var í Nólsoy.}\\
	\glt ʽThere was a big wedding in Nólsoy.ʼ\\
	\gll	Í   fleiri   dagar varð   \textbf{tað} etið   og     drukkið.  \\
			in  many  days  was  there  eaten  and  drunk\\
	\glt	ʽPeople were eating and drinking for several days.ʼ
\z
\z

\newpage 
The second type is a weather expression which is a direct yes\slash no-question with a fronted verb and without an overt weather expletive: 

\ea%14
    \label{ex:Thrainsson:14}
	\textit{Abbin var blivin eitt sindur dølskur og spurdi:}\\
	\glt ʽGrandpa had become a bit slow and asked:ʼ\\
    \gll    Regnaði   {í gjár?}  \\
		    rained    yesterday\\
    \glt ʽDid it rain yesterday?ʼ
 \z

Then there were two examples where the subjects were asked to choose between a variant without the overt expletive and one with it. One of them was a weather expression and the other an Expletive Passive:

\ea%15
    \label{ex:Thrainsson:15}
	\ea
	\textit{Tað regnar ongantíð í Sahara}.\\
	\glt ʽIt never rains in Sahara.ʼ\\
    \gll Í Havn       regnar  /   regnar \textbf{tað} ofta.\\
		 in Tórshavn    rains  {}   rains   it    often\\
	\glt ʽIn Tórshavn it often rains.ʼ
	\ex
	\textit{Tað hendir nógv í Íslandi.}\\
	\glt	ʽMany things happen in Iceland.ʼ\\
	\gll	Fríggjadagin   bleiv   /   bleiv   \textbf{tað}  skotin ein hvítabjørn   har. \\
			the-Friday    was      {}  was  there  shot  a   {polar bear}  there\\
	\glt	ʽLast Friday a polar bear was shot there.ʼ
\z
\z

The results of the evaluation of the variants in (\ref{ex:Thrainsson:13}--\ref{ex:Thrainsson:14}) are shown in \tabref{tab:Thrainsson:3} (highest percentages for each example in boldface).

\begin{table}
\begin{tabularx}{\textwidth}{lXrSrSrS}
\lsptoprule &  & \multicolumn{2}{c}{{Yes}} & \multicolumn{2}{c}{{?}} & \multicolumn{2}{c}{{No}}\\\cmidrule(lr){3-4}\cmidrule(lr){5-6}\cmidrule(lr){7-8}
{\#} & {Example} & \multicolumn{1}{c}{N} & \multicolumn{1}{c}{\%} & \multicolumn{1}{c}{N} & \multicolumn{1}{c}{\%} & \multicolumn{1}{c}{N} & \multicolumn{1}{c}{\%}\\\midrule
(\ref{ex:Thrainsson:13}a) & Tá varð nógv dansað heima við hús. & 293 & \textbf{89.3} & 21 & 6.4 & 14 & 4.6\\
(\ref{ex:Thrainsson:13}b) & Tá varð \textbf{tað} tosað saman um kvøldarnar. & 229 & \textbf{69.4} & 47 & 14.2 & 54 & 16.4\\
(\ref{ex:Thrainsson:13}c) & Í fleiri dagar varð \textbf{tað} etið og drukkið. & 220 & \textbf{67.7} & 58 & 17.8 & 47 & 14.5\\
\REF{ex:Thrainsson:14}  & Regnaði í gjár? & 188 & \textbf{56.8} & 61 & 18.4 & 82 & 24.8\\
\lspbottomrule
\end{tabularx}
\caption{Evaluation of examples with and without an overt expletive.\label{tab:Thrainsson:3}}
\end{table}

In the first set of examples (the \isi{impersonal passive}s in \ref{ex:Thrainsson:13}) the variant without the overt expletive (the \textit{a-}example) gets a more positive evaluation than the ones with the overt expletive (examples \textit{b} and \textit{c}). The weather expression in \REF{ex:Thrainsson:14} does not have an overt expletive and it does not get as positive evaluation as (\ref{ex:Thrainsson:13}a), which also has a \isi{null expletive}, albeit of a different kind. This suggests that there might be a difference between “true” expletives (\textit{there}-expletives) and weather expletives (\textit{it-}expletives) in this respect. This would not be surprising since it has been argued that the weather expletive is more argument-like than the true expletive (Vikner even claims (\citeyear[228--229]{Vikner1995}) that weather expletives are true arguments). But the test sentences where the subjects were asked to choose between overt and non-overt expletives in a weather expression on the one hand and in an Expletive Passive on the other did not show a clear diffence between the two types, although a third of the speakers found that both variants are possible in the case of the weather expression but very few in the case of the Expletive Passive. This is shown in \tabref{tab:Thrainsson:4} (the most popular choice in boldface).

\begin{table}[b]
\begin{tabularx}{\textwidth}{lXrSrSrS}\lsptoprule
&  & \multicolumn{2}{c}{{without} {\textit{tað}}} & \multicolumn{2}{c}{{both variants}} & \multicolumn{2}{c}{with \textit{tað}}\\\cmidrule(lr){3-4}\cmidrule(lr){5-6}\cmidrule(lr){7-8}
{\#} & {Example} & \multicolumn{1}{c}{N} & \multicolumn{1}{c}{\%} & \multicolumn{1}{c}{N} & \multicolumn{1}{c}{\%} & \multicolumn{1}{c}{N} & \multicolumn{1}{c}{\%} \\\midrule
(\ref{ex:Thrainsson:15}a) & Í Havn regnar / regnar \textbf{tað} ofta. & 83 & 25.4 & 108 & 33.0 & 136 & \textbf{41.6}\\
(\ref{ex:Thrainsson:15}b) & Fríggjadagin bleiv / bleiv \textbf{tað} skotin ... & 111 & 34.9 & 28 & 8.8 & 179 & \textbf{56.3}\\
\lspbottomrule
\end{tabularx}
\caption{Selection between alternatives in expletive constructions.}\label{tab:Thrainsson:4}
\end{table}

Here we can also investigate how the the judgments spread, e.g. whether any of the speakers accept both instances of empty expletives or reject both of them (i.e. \ref{ex:Thrainsson:13}a and \ref{ex:Thrainsson:14} — we leave out the examples in \ref{ex:Thrainsson:15} because here the elicitation method was different). This is shown in \figref{fig:Thrainsson:3}, where the value 3 on the X-axis indicates that the relevant speakers found both of the examples with null expletives\is{null expletive} natural and the value 1 means that they rejected both of them.

\begin{figure}
\caption{Judgments of empty expletives.\label{fig:Thrainsson:3}}
\includegraphics[height=.4\textheight]{figures/ThrainssonFigure5.pdf}
% \begin{tikzpicture}
% \begin{axis}[
% ylabel={Frequency},
% xlabel={Distribution of evaluations of Transitive Expletives},
% ylabel near ticks,
% xlabel near ticks,
% nodes near coords,
% xtick=data,
% enlarge y limits={abs=10,upper},
% legend pos=outer north east,
% legend cell align=left,
% legend style={draw=none},
% ymin=0
% ]
% \addlegendimage{empty legend}\addlegendentry{%
% 	\frame{\begin{tabular}{l@{ }c@{ }S[table-format = 4.6,group-digits=false]}
% 	Mean & = & 0\\
% 	Std. Dev & = & 0\\
% 	N & = & 0\\
% 	\end{tabular}}}
% \addplot[ybar,fill=gray] plot coordinates {(1,0) (1.25,0) (1.5,0) (1.75,0) (2,0) (2.25,0) (2.5,0) (2.75,0) (3,0)};
% \end{axis}
% \end{tikzpicture}
\end{figure}

Here almost half of the speakers found both examples natural, very few (only 8) rejected both of them but a considerable number found them doubtful or liked one and not the other.

\subsection{Transitive expletives}
Let us finally look at the so-called Transitive Expletive Construction (TEC). Here the \ili{Icelandic} and MSc facts seem relatively clear cut: Speakers of \ili{Icelandic} find TECs fine whereas speakers of MSc typically reject them. But whereas \citet[189]{Vikner1995} maintained that TECs are not accepted in \ili{Faroese}, \citet[282]{Thráinsson2012} argued that they are accepted “by some speakers” and \citet[173]{Angantýsson2011} found that the majority of his subjects found TEC-examples to be natural. In several discussions of comparative \ili{Scandinavian}, TECs have played a major role (see e.g. \citealt{Bobaljik1998}, \citealt[333–340]{Thráinsson2007}, \citealt{Thrainsson2017}). The TEC-examples evaluated by participants in FarDiaSyn are shown in \REF{ex:Thrainsson:16}:

\ea%16
    \label{ex:Thrainsson:16}\label{ex:thrainsson:16}
	\ea
	\textit{Teir sjey dvørgarnir vóru í øðini.}\\
	\glt ‘The seven dwarfs were upset.’\\
	\gll       Tað hevði   onkur       etið   súreplið.\\
		      there  had  somebody  eaten  the-apple\\
	\glt       ‘Somebody had eaten the apple.’

\newpage
	\ex \textit{Fleiri hús á Signabø vóru til sølu.}\\
	\glt ‘Several houses in Signabo were for sale.’\\
	\gll     Tað   keypti   onkur     húsini     hjá Róa.\\
			 there  bought  somebody  the-houses  of  Rói\\
	\glt   ‘Somebody bought Rói’s house.’

	\ex \textit{Eg mátti ganga til hús.}\\
	\glt     ‘I had to walk home.’\\
	\gll      Tað   hevði onkur       tikið   súkkluna     hjá mær.\\
		     there  had  somebody  taken  the-cycle   of me\\
	\glt      ‘Somebody had taken my bike.’
 
	\ex  \textit{Hendan bókin er ógvuliga drúgv.}\\
	\glt    ‘This book is extremely long.’\\
	\gll      Tað   hevur helst     eingin   lisið   hana   til enda.\\
		      there  has  probably  nobody  read  her    to  end\\
	\glt      ‘Probably no-one has read it to the end.’
\z\z


An overview of the evaluations can be seen in \tabref{tab:Thrainsson:5} (highest percentages for each example in boldface as before).

\begin{table}
\begin{tabularx}{\textwidth}{lXrSrSrS}\lsptoprule &  & \multicolumn{2}{c}{{Yes}} & \multicolumn{2}{c}{{?}} & \multicolumn{2}{c}{{No}}\\\cmidrule(lr){3-4}\cmidrule(lr){5-6}\cmidrule(lr){7-8}
 {\#} & {Example} & \multicolumn{1}{c}{N} & \multicolumn{1}{c}{\%} & \multicolumn{1}{c}{N} & \multicolumn{1}{c}{\%} & \multicolumn{1}{c}{N} & \multicolumn{1}{c}{\%}\\\midrule
(\ref{ex:Thrainsson:16}a) & Tað hevði onkur  etið súreplið. & 80 & 24.4 & 58 & 17.7 & 190 & \textbf{57.9}\\
(\ref{ex:Thrainsson:16}b) & Tað keypti onkur húsini   hjá Róa. & 51 & 15.5 & 71 & 21.6 & 207 & \textbf{62.9}\\
(\ref{ex:Thrainsson:16}c) & Tað hevði onkur tikið súkkluna hjá mær. & 82 & 25.2 & 65 & 19.9 & 179 & \textbf{54.9}\\
(\ref{ex:Thrainsson:16}d) & Tað hevur helst eingin lisið hana til enda. & 148 & \textbf{45.4} & 62 & 19.0 & 116 & 35.6\\
\lspbottomrule
\end{tabularx}
\caption{Evaluation of transitive expletives in FarDiaSyn.}
\label{tab:Thrainsson:5}
\end{table}

\largerpage[-3]
More speakers reject than accept the first three examples but more speakers accept than reject the last one. Three of the examples contain an \isi{auxiliary} verb and the one where the \isi{finite} verb is a main verb (the \textit{b-}example) was less positively evaluated.\footnote{\citet[173]{Angantýsson2011} presents the evaluation results for two TEC-examples in \ili{Faroese}, one with an \isi{auxiliary} and one without. His subjects also found the one with the \isi{auxiliary} more acceptable. — It is also interesting to note that the acceptance rate of the TECs is considerably lower in the FarDiaSyn study reported on here than in Angantýssonʼs study. A likely reason for this difference is the fact that the logical subject in examples (\ref{ex:Thrainsson:16}a–c) is the simple indefinite pronoun \textit{onkur} ʽsomebodyʼ whereas corresponding examples in Angantýssonʼs study contained the more complex subject \textit{onkur útlendingur} ʽsome foreignerʼ, which might sound more natural in an expletive construction.\label{Fn:Thrainsson:7}}

Given what we have already seen, we would expect that the picture showing the spread of the judgments to look rather different from the pictures previously presented. This prediction is borne out, as shown on \figref{fig:Thrainsson:4}.

 
\begin{figure}

\includegraphics[height=.4\textheight]{figures/ThrainssonFigure6.pdf}
\caption{Judgments of Transitive Expletives\label{fig:Thrainsson:4}.}
% \begin{tikzpicture}
% \begin{axis}[
%   ylabel={Frequency},
%   xlabel={Distribution of evaluations of Transitive Expletives},
%   ylabel near ticks,
%   xlabel near ticks,
%   nodes near coords,
%   xtick=data,
%   enlarge y limits={abs=10,upper},
%   legend pos=outer north east,
%   legend cell align=left,
%   legend style={draw=none},
%   ymin=0
% ]
% \addlegendimage{empty legend}\addlegendentry{%
% \frame{\begin{tabular}{l@{ }c@{ }S[table-format = 4.6,group-digits=false]}
% Mean & = & 1.7535\\
% Std. Dev & = & 0.58473\\
% N & = & 333\\
% \end{tabular}}}
% \addplot[ybar,fill=gray] plot coordinates {(1,61) (1.25,37) (1.5,71) (1.75,28) (2,43) (2.25,29) (2.5,37) (2.75,13) (3,14)};
% \end{axis}
% \end{tikzpicture}
\end{figure}

As \figref{fig:Thrainsson:4} shows, very few speakers accept all the TEC-examples (only 14) and a considerable number of subjects reject all of them. As explained in the preceding footnote, the relatively low acceptance of TECs in this study compared to that of \citet{Angantýsson2011}, for instance, is probably due to an unfortunate choice of logical subject. But in any case, the judgments here indicate considerable intra-speaker variation similar to what we have seen before: Speakers typically accept some of the examples and not all of them.

\section{Comparison of the constructions}\label{sec:Thrainsson:4}
\subsection{Some correlations}

Having gone through the data concerning the individual constructions under discussion, we can now investigate whether there is any correlation between the judgments of the four different constructions. In the ideal world (or for ideal speakers) there should be a very strong correlation between these if the constructions are all related by a single parameter, such as H{\&}Pʼs Agr-parameter, “all else being equal”. But because of the extensive intra-speaker variation in the judgments observed in the preceding sections, it is not entirely clear a priori what to expect here. So let us look at \tabref{tab:Thrainsson:6} (the two strongest correlations highlighted by boldface).

\begin{table}
\resizebox{\textwidth}{!}{\begin{tabular}{l *3{S[table-format=<1.3,table-space-text-pre=p,input-comparators,table-comparator = true,detect-inline-weight = text]}}
\lsptoprule
& \multicolumn{1}{c}{Stylistic Fronting} & \multicolumn{1}{c}{Null expletives\is{null expletive}} & \multicolumn{1}{c}{Transitive Expletives}\\\midrule
\isi{Oblique subjects} 	& \bfseries r = 0.470 & r~ = 0.330 & r~ = 0.297\\
					& \textbf{p < 0.001} & p~ < 0.001 & p~ < 0.001\\
					& \multicolumn{1}{c}{N = 333} & \multicolumn{1}{c}{N = 333} &  \multicolumn{1}{c}{N = 333}\\\midrule
\isi{Stylistic Fronting}  &         & r~ = 0.354 & \textbf{r = 0.371}\\
					&  xxxx   & p~ < 0.001 & \textbf{p < 0.001}\\
					&         & \multicolumn{1}{c}{N = 333}  & \multicolumn{1}{c}{N = 333}\\\midrule
{Null expletives\is{null expletive}}   &         &                     & r~ = 0.168\\
					& xxxx 	  & xxxx			    &   p~ = 0.002\\
					&		  & 					&  \multicolumn{1}{c}{N = 333}\\
\lspbottomrule
\end{tabular}}
\caption{Correlation between the evaluations of the four constructions under investigation.}
\label{tab:Thrainsson:6}
\end{table}


As shown here, the correlations are typically only of medium strength.\footnote{The correlation coefficient \textit{r} can range from {\textminus}1.0 to +1.0, where {\textminus}1.0 is a perfect negative correlation, +1.0 a perfect positive correlation and 0.0 indicates no correlation at all. It is often said that if the correlation coefficient \textit{r} is around ±0.10, the correlation is weak, if it is around ±0.30 the correlation is of medium strength and it is strong if it reaches ±0.50 in studies of this kind.} The only one that could possibly be called strong is the correlation between judgments of examples involving oblique subjects and \isi{Stylistic Fronting} (\textit{r} = 0.470). Yet the correlations are all highly significant so it might seem tempting to say something like the following: “Look, there is a highly significant correlation between the evaluations of all the constructions -- \textit{p} is nowhere higher than 0.002, which in statistical terms should mean that there should be at most 2‰ chance that these correlations are an accident. So H{\&}P were right -- these constructions are all related by a single parameter.”

Unfortunately, things are not as simple  as this for several reasons, including the following:

\begin{enumerate}
	\item First of all, correlations can never be interpreted as a proof of a causal relationship.
	\item  Second, if all the constructions considered here were accepted by the majority of the speakers consulted, there should be some correlation between the speakersʼ evaluation of them: If a speaker is likely to accept construction A (s)he will also be likely to accept construction B because most speakers do, “all else being equal”. This need not mean that they are parametrically related.
	\item Since all the constructions investigated here were supposedly also found in Old \ili{Norse}, and thus in older stages of \ili{Faroese}, it is possible that the correlations observed are basically a reflection of some sort of conservatism in the language: If you are a conservative speaker of \ili{Faroese} you are likely to accept all these constructions even if they are not related by a single parameter.
\end{enumerate}

So let us look more closely at the data with these possibilities in mind.

As shown in Tables \ref{tab:Thrainsson:1}, \ref{tab:Thrainsson:2}, \ref{tab:Thrainsson:3} and \ref{tab:Thrainsson:4}, the acceptance of the example sentences varied considerably but we could “rank” their acceptability as shown in \tabref{tab:Thrainsson:7}.

\begin{table}
\begin{tabular}{lSS[table-format=1.2]}
\lsptoprule
{Construction} & \multicolumn{1}{p{4cm}}{\raggedright Speakers finding the examples ``natural'' (\%)} & \multicolumn{1}{l}{Mean ``grade''}\\\midrule
\isi{Oblique subjects} & 73.7 & 2.60 \\
Null expletives\is{null expletive} & 73.1 & 2.20 \\
\isi{Stylistic Fronting} (SF) & 49.7 & 2.19 \\
Transitive \isi{Expletives} (TEC) & 27.6 & 1.75 \\
\lspbottomrule
\end{tabular}
\caption{Acceptability ranking of the constructions under investigation.}
\label{tab:Thrainsson:7}
\end{table}

As shown in the middle column, an average of over 73\% of the speakers found the examples involving oblique subjects and null expletives\is{null expletive} natural whereas\linebreak about half of the speakers found the SF examples natural and only a little more than 27\% found the TEC examples natural. But since the speakers were using a three point scale (natural, doubtful, unnatural\slash ungrammatical) we can also assign a “mean grade” to each class of examples, where 3 would mean ``all subjects found all the examples natural'' and 1 would mean ``all subjects found all the examples unacceptable''. These grades are shown in the rightmost column. Here we see that the “acceptability ranking” of the constructions remains the same regardless of the ranking method (although there is virtually no difference between null expletives\is{null expletive} and \isi{Stylistic Fronting}).

Keeping this ranking (or popularity) of the constructions in mind, we might have expected the strongest correlations to hold between oblique subjects and null expletives\is{null expletive} since these were the two most “popular” constructions. But this is not what we find. Instead the strongest correlation (\textit{r} = 0.470) is between the evaluations of examples containing an oblique subject and examples containing SF. The next-highest correlation is between the judgments of the TEC and SF.

In order to determine whether the observed correlations are simply a reflection of some general conservatism, we can look for a clear innovation and see if or how it relates to the other constructions. FarDiaSyn included a study of the so-called New (\isi{Impersonal}) Passive (or New \isi{Impersonal} Construction), first made famous by Joan Maling and Sigríður Sigurjónsdóttir (cf. \citealt{Sigurjónsdóttir2001}, \citealt{Maling2002} and much later work). The New \isi{Impersonal} Passive (henceforth NIP) arguably comes in a couple of different guises as partly illustrated by the \ili{Icelandic} examples in (\ref{ex:Thrainsson:18}c) and (\ref{ex:Thrainsson:19}c):

\ea%18
\settowidth\jamwidth{(Canonical Passive)}
    \label{ex:Thrainsson:18}\label{ex:thrainsson:18}
    \ea
    \gll Einhver     lamdi     mig.\\
	somebody  hit      me\textsc{.acc}\\
    \ex
	\gll     Ég        var      laminn.  \\
     I\textsc{.nom}      was    hit.\textsc{m.sg}\\\jambox{(Canonical Passive)}
     \ex
     \gll Það      var      \textbf{lamið}    \textbf{mig}.\\
     there      was    hit.\textsc{n.sg}  me\\\jambox{(NIP)}
    \z
\z


\ea%19
\settowidth\jamwidth{(Canonical Passive)}
    \label{ex:Thrainsson:19}\label{ex:thrainsson:19}
	\ea
    \gll Einhver    lofaði      henni      tölvu.\\
	somebody  promised    her\textsc{.dat}    computer\textsc{.acc}\\
    \ex
    \gll  Henni      var    lofað          tölvu.\\
    her\textsc{.dat}    was  promised\textsc{.n.sg}    computer\textsc{.acc}\\\jambox{(Canonical Passive)}
    \ex
    \gll  Það      var    lofað           \textbf{henni}      tölvu. \\
    there      was  promised\textsc{.n.sg}    her\textsc{.dat}    computer\textsc{.acc}\\\jambox{(NIP)}
    \z
\z

The NIP in (\ref{ex:thrainsson:18}c) differs from the canonical \isi{passive} in (\ref{ex:thrainsson:18}b) in that the argument (the patient) shows up in the Acc instead of Nom and hence there is no \isi{agreement} with the participle. Besides, the argument can occur in an expletive construction of sorts although it is definite (an apparent violation of the Definiteness Constraint).\footnote{It is generally assumed that this argument is not a subject in the NIP. If so, then it is not to be expected that the Definiteness Effect plays any role.} The NIP in (\ref{ex:thrainsson:19}c) only differs from the canonical \isi{passive} in (\ref{ex:thrainsson:19}b) in that the definite Dat argument \textit{henni} occurs postverbally (i.e. in an object position). Definite subjects in the canonical \isi{passive} cannot occur in that position.

It is generally assumed that this NIP is a recent innovation in \ili{Icelandic} since it was first noticed by linguists towards the end of the last century (for a detailed discussion of the NIP, possible origin and review of the arguments see \citealt{Sigurðsson2012}). It does not seem to occur in MSc. But while the subjects in FarDiaSyn rejected the variant corresponding to (\ref{ex:thrainsson:18}c), a number of them accepted examples corresponding to (\ref{ex:thrainsson:19}c). These are listed in \REF{ex:Thrainsson:20}:

\ea%20
    \label{ex:Thrainsson:20}
    \ea
    \textit{Gentan hevði hjálpt beiggjanum alla vikuna}.\\
    \glt ʽThe girl had helped her brother the whole week.ʼ\\
    \gll Tað   \textbf{bleiv}   \textbf{lovað}     \textbf{henni}   eina teldu.\\
         there  was  promised her\textsc{.dat}   a computer\textsc{.acc}\\
    \ex \textit{  Hanus fekk onga læknaváttan.}\\
    \glt ʽHanus didnʼt get any doctorʼs certificate.ʼ\\
    \gll Tað   varð   \textbf{rátt}     \textbf{honum}   \textbf{frá}     at fara   við   skipinum.\\
          there  was  advised  him\textsc{.dat}  against  to go    with  the-ship\\
    \ex   \textit{Tvíburarnir fyltu 7 ár.}\\
    \glt   ʽThe twins turned 7 years old.ʼ\\
    \gll  Tað   bleiv   \textbf{givið}   \textbf{gentuni}     eina dukku.  \\
          there  was  given  the-girl\textsc{.dat}  a doll\textsc{.acc}\\
    \ex \textit{Drotningin kom at vitja tey eldru fólkini á ellisheiminum.}\\
	\glt ʽThe queen came to visit the people in the old peopleʼs home.ʼ\\
	\gll         Tað   bleiv   \textbf{vaskað}   \textbf{teimum} væl   um     hárið.\\
	         there  was  washed  them\textsc{.dat}   well   about     the-hair\\
	\ex \textit{Rógvarin Katrin Olsen stóð seg væl í Olympisku Leikunum.}\\
	\glt      ʽThe rower KO did well at the Olympics.ʼ\\
	\gll  Tað   bleiv   \textbf{róst}     \textbf{henni}   í bløðunum.\\
          there  was  praised  her\textsc{.dat}  in the-newspapers\\
	\ex \textit{Bókasavnið hevði framsýning}.\\
    \glt   ʽThe library had an exhibition.ʼ\\
    \gll Tað   bleiv   \textbf{víst}     \textbf{gestunum} nógv   tilfar     um Heinesen.\\
          there  was  shown  the-guests\textsc{.dat}  much  material  on Heinesen\\
    \z
\z

The subjectsʼ evaluation of these examples are shown in \tabref{tab:Thrainsson:8} (highest percentages for each example highlighted).

\begin{table}
\begin{tabularx}{\textwidth}{lXrSrSrS}\lsptoprule &  & \multicolumn{2}{c}{{Yes}} & \multicolumn{2}{c}{{?}} & \multicolumn{2}{c}{{No}}\\\cmidrule(lr){3-4}\cmidrule(lr){5-6}\cmidrule(lr){7-8}
 {\#} & {Example} & \multicolumn{1}{c}{N} & \multicolumn{1}{c}{\%} & \multicolumn{1}{c}{N} & \multicolumn{1}{c}{\%} & \multicolumn{1}{c}{N} & \multicolumn{1}{c}{\%}\\\midrule
(\ref{ex:Thrainsson:20}a) & Tað bleiv lovað henni eina teldu. & 167 & \textbf{50.6} & 70 & 21.2 & 93 & 28.2\\
(\ref{ex:Thrainsson:20}b) & Tað varð rátt honum frá at fara við skipinum. & 263 & \textbf{79.7} & 32 & 9.7 & 35 & 10.6\\
(\ref{ex:Thrainsson:20}c) & Tað bleiv givið gentuni eina dukku. & 65 & 19.9 & 65 & 19.9 & 197 & \textbf{60.2}\\
(\ref{ex:Thrainsson:20}d) & Tað bleiv vaskað teimum væl um hárið. & 87 & 26.4 & 65 & 19.8 & 177 & \textbf{53.8}\\
(\ref{ex:Thrainsson:20}e) & Tað bleiv róst henni í bløðunum. & 66 & 20.2 & 62 & 19.0 & 199 & \textbf{60.9}\\
(\ref{ex:Thrainsson:20}f) & Tað bleiv víst gestunum nógv tilfar um Heinesen. & 203 & \textbf{62.1} & 55 & 16.8 & 69 & 21.1\\
\lspbottomrule
\end{tabularx}
%%please move \begin{table} just above \begin{tabular .
\caption{Evaluation of New Impersonal Passive examples (w. Datives) in FarDiaSyn.}
\label{tab:Thrainsson:8}
\end{table}

Here we see considerable variation: Some of the examples are found to be natural by a majority of the subjects, others are rejected by a majority of the subjects. On the average only about 43\% of the subjects find the examples natural. Since this construction must be an innovation in \ili{Faroese}, it is of some interest to see how the judgments of it correlate with judgments of the constructions under discussion. The \textit{r-} and \textit{p-}values are shown in \tabref{tab:Thrainsson:9} (the one non-significant correlation highlighted).

\begin{table}
\begin{tabular}{l *4{S[table-format=<1.3,table-space-text-pre=p,input-comparators,table-comparator = true]}}
\lsptoprule & {Oblique subjects} & \multicolumn{1}{p{1.75cm}}{Stylistic Fronting} & \multicolumn{1}{c}{Null expletives\is{null expletive}} & \multicolumn{1}{p{1.75cm}}{Transitive Expletives}\\\midrule
NIP (Dat) & r~ = 0.482 & r~ = 0.464 & \textbf{r = 0.069}  & r~ = 0.426 \\
	  & p~ < 0.001 & p~ < 0.001 & \textbf{p = 0.209} & p~ < 0.001\\
	  & {N = 333} & {N = 333} & {N = 333} & {N = 333}\\
\lspbottomrule
\end{tabular}
\caption{Correlations between judgments of New Impersonal Passive examples and other constructions in FarDiaSyn.}
\label{tab:Thrainsson:9}
\end{table}

Interestingly, there is considerable correlation (almost “strong”) between the evaluations of the innovative NIP-examples (with a Dat argument) and the “old” constructions under investigation, except for null expletives\is{null expletive}. This kind of correlation can hardly be due to some general conservatism.

\subsection{Comparison of the variation}

Finally, let us return to the distribution of the variation shown in Figures~\ref{fig:Thrainsson:1}--\ref{fig:Thrainsson:4}, repeated here for convenience.
 
\begin{figure}
\resizebox{\textwidth}{!}{\includegraphics[width=.5\textwidth]{figures/ThrainssonFigure3.pdf}
	\includegraphics[width=.5\textwidth]{figures/ThrainssonFigure4.pdf}}
\resizebox{\textwidth}{!}{\includegraphics[width=.5\textwidth]{figures/ThrainssonFigure5.pdf}
\includegraphics[width=.5\textwidth]{figures/ThrainssonFigure6.pdf}}
 \caption{Judgments of examples of the four constructions investigated.}
\end{figure}


If the four constructions are related by a single parameter, we might have expected greater similarity between the evaluations than these figures reveal, even if we assume that the parameter settings can be “soft” (i.e., their probabilities ranging from 0 to 1). But maybe the figures are not as different as they seem. First, there is considerable similarity between the figures for Dat subjects and null expletives\is{null expletive}: Many speakers accept all the examples, very few speakers reject all of them and some speakers are in between. This would seem compatible with the concept of soft parameter settings. Second, we could argue that the figure for SF in fact reveals a similar situation: Very few speakers reject all the SF examples, most speakers find most of the examples natural and the reason why so few speakers find all the SF examples perfect might have to do with their stylistic value. But the figure for the TEC is clearly out of line since so many speakers find all the TEC examples unacceptable. This clearly calls for an explanation. A likely reason for this high rejection rate is the unfortunate choice of logical subjects in the TEC examples used (cf. fn. \ref{Fn:Thrainsson:7}), which seems to have had the effect that many more speakers rejected the TEC examples in FarDiaSyn than the TEC examples used in Angantýssonʼs study. The relatively high correlation between the judgments of the TEC and judgments of some of the other constructions investigated (cf. \tabref{tab:Thrainsson:6}) suggests that the TEC might in fact be related to the others in some fashion despite the different acceptability patterns revealed by the figures above.

\section{Conclusion and discussion}\label{sec:Thrainsson:5}
\subsection{Summary of the evidence}

The main points of this paper can be summarized as follows:

\begin{itemize}
\item As Holmberg pointed out (\citeyear[13n]{Holmberg2010parameters}), \ili{Faroese} offers an extremely interesting test case for the parametric approach to syntactic variation in general and in \ili{Scandinavian} in particular. The reason is the extensive inter- and intra-speaker variation found in \ili{Faroese} syntax in areas where it has been maintained that parameters play a role.
\item  Because FarDiaSyn was such an extensive study that included a number of supposedly related constructions and involved a large number of speakers, it offers a unique opportunity to test parametric predictions in a new fashion by applying statistical methods.
\item While this paper has shown that one has to be very careful in drawing conclusions about linguistic knowledge based on statistical data from syntactic performance (mostly evaluation of sentences in this case), the results from FarDiaSyn cannot be said to support the claim that the acquisition of oblique subjects, \isi{Stylistic Fronting}, null expletives\is{null expletive} and the Transitive Expletive Construction is simply governed by a single binary parameter, as originally suggested by H{\&}P.
\end{itemize}

One possible objection to the main conclusion above might be that the arguments in this paper are for the most part based on data elicited by having the subjects evaluate examples and pass acceptability judgments. The idea would then be that the extensive intra-speaker variation reported on here is a consequence of the methodology and not “real”. But several recent studies have found evidence for similar intra-speaker variation using a variety of elicitation techniques and comparing the results to production data (see e.g. \citealt[184–186]{Thráinsson2013variation}, \citealt{Nowenstein2014} and references cited by these authors; cf. also \citealt{Jónsson2005}). \isi{Intra-speaker variation} in syntax (and phonology) is much more pervasive than we have often assumed. It is difficult to reconcile this fact in principle with the concept of binary parameters fixed once and for all, ideally quite early in the acquisition period.

\subsection{The remaining options}
So what are we left with? The P{\&}P approach is a bold and interesting attempt to solve the so-called “logical problem of language acquisition”: How can most children come to know their native language very rapidly and in a fairly uniform fashion although the input (the “primary linguistic data”, PLD) is supposedly both limited and at times inconsistent and misleading (the standard “poverty of the stimulus” argument)? This is understandable if there is very little to learn, as maintained by the P{\&}P approach. The children ideally just have to set a few parameters and they only need very limited evidence to do so. This is presumably the main reason why so many linguists have embraced the P{\&}P approach. The data reviewed here suggest, however, that language acquisition may not always proceed as simply and quickly as the standard P{\&}P approach would predict if the relevant grammatical properties are parametrically related. So what are the options we are left with?

One alternative, of course, is that there are no parameters, just lan\-guage-par\-tic\-u\-lar  rules that speakers have to acquire. This is the account proposed by \citet{Newmeyer2004,Newmeyer2005,Newmeyer2006}. His main reason for doing so comes from typological evidence: He maintains that the clustering of properties predicted by the standard P{\&}P approach never holds when a large enough sample of languages is considered. Assuming (with H{\&}P) that ISc typically has oblique subjects, \isi{Stylistic Fronting}, null expletives\is{null expletive} and the TEC whereas MSc does not, one could then say that ISc has one set of rules accounting for the relevant properties whereas MSc has another. In their reply to Newmeyerʼs original article \citep{Newmeyer2004}, Roberts and Holmberg claim, however, that while such an account would be “observationally adequate”, it “makes no predictions whatsoever regarding the correlation of the properties” (\citeyear[551]{Roberts2005}). So if such a correlation holds for the properties under discussion, as they assume, the P{\&}P account proposed by H{\&}P is superior to Newmeyerʼs rule-based account, according to Roberts and Holmberg. To this Newmeyer replies in turn (\citeyear[7]{Newmeyer2006}) that “It has been known since the earliest days of transformational grammar that rules are both abstract and often shared by more than one language (just as parameter \ref{ex:thrainsson:2} [= \citeauthor{Holmplat1995}ʼs Agr-parameter 1995 or its equivalent] is probably best interpreted as a rule shared by the ISC languages)”. This statement suggests, however, that the difference between “rules” in Newmeyerʼs sense and typical P{\&}P parameters is smaller than we might have thought.

But now recall that H{\&}P were originally trying to account for cross-linguistic (or cross-dialectal) differences and similarities. In that sense they were concerned with \textsc{inter-speaker variation}, i.e. differences between speakers (or groups of speakers, rather). The same is true of the arguments presented in the debate between Newmeyer, Holmberg and Roberts. Thus Newmeyer states (\citeyear[183]{Newmeyer2004}) that “language-particular differences are captured by differences in language-particular rules” (and in \citeyear{Newmeyer2006} he also maintains that cross-linguistic similarities can be captured by assuming similar rules, as we have just seen), whereas \citet[538]{Roberts2005} state that they intend to defend the “principles-and-para\-meters model of crosslinguistic variation”. In the present paper we have argued, on the other hand, that \textsc{intra-speaker variation} is an important part of speakersʼ competence and that it is much more prevalent than typically assumed. This means that it has to be taken seriously and not just brushed aside as some sort of shallow and uninteresting performance phenomenon. But how can it be accounted for?

First, it is important to note that we do not seem to be dealing with variation that is syntactically free and simply conditioned by some non-linguistic factors like social situation. The data reported on here were elicited under the same social conditions and we also find variation in production by individual speakers, e.g. in the case marking of subjects, under the same circumstances and within seconds in spontaneous speech (see e.g. \citealt[236]{Jónsson2005}, \citealt[7]{Nowenstein2014}). Even more importantly, though, the \ili{Faroese} speakers reported on here typically show intra-speaker variation \textit{to a different extent}. Thus some of them are more likely to show ISc-like judgments than others, as shown by Figures~\ref{fig:Thrainsson:1}--\ref{fig:Thrainsson:4} above. This is something that needs to be accounted for.\footnote{As shown by \citet[182--184]{Thráinsson2013variation}, this kind of intra-speaker variation also has its parallels in phonological production. So it is clearly not an artifact of the methodology of FarDiaSyn.}

One proposal compatible with extensive intra-speaker variation is the \isi{grammar competition} approach advocated by \citet{Kroch1989,Kroch2001}. It is possible to think of \isi{grammar competition} in two ways. On the one hand we could say that during a period of linguistic change two “grammars” compete within a given linguistic community: An innovative construction (generated by the new grammar) then eventually (or ideally) drives out a conservative construction (generated by the old grammar). Their relative frequencies within the community shift, typically following an S-shaped curve. We could call this an E-language description of \isi{grammar competition} as it focuses on the relevant linguistic community as a whole. More interestingly for our purposes, we could also say that for a given individual exhibiting a intra-speaker variation there are two grammatical options within the same internal language. Grammar competition is then a part of the competence of individual speakers, a kind of bilingualism, and it is reflected in the speakers’ production or performance. We could call this an I-language description (if by I-language we mean the internalized language of individual speakers and not just the invariant \isi{universal} language faculty, as in some usages of the term (for relevant discussion see e.g. \citealt{Sigurðsson2011Uniformity})).

Yangʼs variational model (\citeyear{Yang2002} and later) is designed to account for this kind of situation and it can be thought of as an attempt to formalize Krochʼs \isi{grammar competition} approach. Assuming that the task of the child acquiring language is to select the grammar\footnote{Following Yang and others, I will mostly use the term ``grammar'' in the following discussion of competition and acquisition and return to the issue of parameters vs. rules at the end of the paper.} that best accounts for the data encountered by the child (the “primary linguistic data”, PLD), it is clear that when there is extensive variability of the relevant kind in the PLD, none of the grammars will account for all the data. Yang suggests that the child will then reinforce (or reward) a particular choice of grammar if the PLD (s)he encounters fit that grammar but otherwise (s)he will penalize it (make it less probable). Since the PLD encountered by different children will vary to some extent, the probability assigned to a given grammar by different children may vary. The variability in the PLD may also have the effect that it could take children a long time to settle on a particular “choice of grammar” and they may actually never rule one choice out although another option is favored to some extent. This will result in stable variation and give the appearance of “soft parameter settings”.\footnote{While one might want to propose that a possible way to express this “softness” would be to say that parametric settings could take on values between 0 and 1, e.g. 0.4 and 0.7 to indicate varying closeness to, say, typical MSc vs. ISc settings, this would not be allowed under the standard assumption that “the values [of parameter settings] are discrete: there are no clines, squishes or continua” (\citealt[541]{Roberts2005}).}

An approach to intra-speaker variation along these lines receives a general support from various acquisition studies: The more unambiguous evidence there is in the PLD, the easier it is for children to acquire the relevant grammatical property. Thus it has been reported, for instance, that there is a direct correlation between the length of the so-called root infinitive stage in Spanish, \ili{French} and English and the amount of unambiguous evidence that Spanish, \ili{French} and English children get for a “[+\isi{Tense}] grammar” (see \citealt{LegateEtAl2007}). The proportion of unambiguous evidence of this sort is highest in child-directed speech in Spanish and lowest in English and the root infinitive stage is shortest for children acquiring Spanish and longest for those acquiring English. In general, there is growing evidence for the claim that there is an interesting interaction between \isi{universal} principles of grammar and the statistical properties of the PLD in language acquisition (for a balanced overview see \citealt{Lidz2015}).

Finally, three comments are in order. First, Yang wants his model to account for various kinds of acquisition, both the acquisition of various kinds of rules (e.g. in morphology) and of parametric settings where appropriate, as can be seen from the quotes in the Introduction above. Hence his general approach could both be adopted by those who believe in rules and have given up on parameters and by those who believe that parameters still have a chance. Second, recall that despite the intra-speaker variation reported on in this paper, we have shown that there is an interesting correlation between the judgments by the speakers of the four constructions under consideration. While this correlation is not as strong as predicted in the ideal world of binary parameters that are set early and easily, it is still intriguing and calls for an explanation. \citet{Roberts2005} would obviously say that this correlation is incompatible with the language-particular rule approach advocated by Newmeyer (e.g. \citeyear{Newmeyer2004}), but this is not so clear if the relevant parameter can also be expressed as a rule, as maintained by \citet[7]{Newmeyer2006}. Newmeyer would point out in turn that the correlation is nowhere near as strong as the standard P{\&}P approach would predict.

\newpage 
The third and final comment is somewhat more complex. Recall that under Yangʼs approach the selection of a given grammar (or rule or parameter setting) is penalized if the PLD do not fit. Now assume that for a child acquiring \ili{Faroese} an ISc-type grammar and an MSc-type grammar are the options. The ISc-type grammar allows oblique subjects, null expletives\is{null expletive}, \isi{Stylistic Fronting} and TEC but the MSc-type grammar does not. Now assume that the child encounters data of the following kind (cf. the discussion around examples \ref{ex:Thrainsson:10}--\ref{ex:Thrainsson:16} above):

\ea%21
    \label{ex:Thrainsson:21}\label{ex:thrainsson:21}
    \ea
    \gll \textbf{Hon}    \textbf{dámar} at   lurta   eftir   tónleiki.\\
      she\textsc{.nom}  likes    to  listen  after  music\\
	\glt ‘She likes to listen to music.’
	\ex
	\gll Í   fleiri   dagar varð   \textbf{tað} etið   og     drukkið. \\
      in  many  days  was  there  eaten  and  drunk\\
	\glt ʽPeople were eating and drinking for several days.ʼ
	\ex
	\gll Kennir   tú     onkran,   sum   \textbf{hevur} \textbf{verið}  í   Kekkia?\\
	     know    you  anybody   that   has     been  in  Czech-Republic\\
	\glt ʽDo you know anybody that has been to the Czech Republic?ʼ
	\ex
	\gll \textbf{Onkur}     \textbf{hevði}   etið   súreplið.\\
      somebody  had    eaten  the-apple\\
	\glt      ‘Somebody had eaten the apple.’
	\z
\z

All of these examples are compatible with an MSc-type grammar: The verb \textit{dáma} ʽlikeʼ takes a Nom subject in (\ref{ex:Thrainsson:21}a) and not an oblique one, the expletive is overt in (\ref{ex:thrainsson:21}b) and not null, there is no \isi{Stylistic Fronting} in (\ref{ex:thrainsson:21}c) and there is no TEC in (\ref{ex:thrainsson:21}d). Interestingly, however, only (\ref{ex:thrainsson:21}a,b) are incompatible with an ISc-type grammar. For speakers of ISc-type languages, \isi{Stylistic Fronting} is optional. Thus the non-occurrence of \isi{Stylistic Fronting} in an environment where it \textit{could} occur (or could be applied, cf. \ref{ex:Thrainsson:12}g above) is perfectly compatible with such a language or grammar. Hence the counterpart of (\ref{ex:thrainsson:21}c) is fine in \ili{Icelandic} — and (\ref{ex:thrainsson:21}c) should be fine for all speakers of \ili{Faroese}, even those who have internalized the most ISc-like grammar. Similarly, TEC is always optional and hence (\ref{ex:thrainsson:21}d) is perfectly compatible with an ISc-type grammar although TEC could also occur there (cf. \ref{ex:thrainsson:16}a). Thus the counterpart of (\ref{ex:thrainsson:21}d) is fine in \ili{Icelandic}.

\largerpage
So why is this last comment important? It is because it demonstrates that if we assume Yangʼs variational acquisition account, ISc-type grammars will never be penalized for the non-occurrence of \isi{Stylistic Fronting} or TEC in contexts where they could occur. Yet some speakers of \ili{Faroese} do not seem to like \isi{Stylistic Fronting} or TEC. Under a parametric account where the availability vs. non-availability of \isi{Stylistic Fronting} and TEC follows from something else in the grammar, such as a particular parametric setting (or the likelihood of such a setting (in Yangʼs terms), or its equivalent in the form of an abstract rule, as suggested by \citealt{Newmeyer2006}) this is understandable. Otherwise it is a puzzle.

% \subsection*{Abbreviations}
\subsection*{Acknowledgements}
The research reported on in this paper was supported by the \ili{Icelandic} Research Fund to the project “Variation in \ili{Faroese} Syntax” (or “\ili{Faroese} Dialect Syntax”, henceforth FarDiaSyn for short), PI Höskuldur Thráinsson, co-applicants Jóhannes Gísli Jónsson and Thórhallur Eythórsson. This project was a part of the \ili{Scandinavian} research networks \ili{Scandinavian} Dialect Syntax (ScanDiaSyn) and Nordic Center of Excellence in Microcomparative Syntax (NORMS, for information see \url{http://norms.uit.no}). Many thanks to our \ili{Scandinavian} colleagues in these networks and in particular to our co-workers on the \ili{Faroese} project, who included \href{http://malvis.hi.is/asgrimur_angantysson}{Ásgrímur Angantýsson}{,} Einar Freyr Sigurðsson, Helena á Løgmansbø, Hlíf Árnadóttir, Lena Reinert, Per Jacobsen, Petra Eliasen, Rakul Napóleonsdóttir Joensen, \href{http://malvis.hi.is/tania_strahan}{Tania E. Strahan}{ and} Victoria Absalonsen. I would also like to thank the editors of this volume and two anonymous reviewers of this paper for very useful comments.

\largerpage
{\sloppy
\printbibliography[heading=subbibliography,notkeyword=this]
}
\end{document}
