\documentclass[output=paper]{LSP/langsci} 
\author{Jenneke van der Wal	\affiliation{Harvard University}
}
\title{Flexibility in symmetry: An implicational relation in Bantu double object constructions} 
\shorttitlerunninghead{Flexibility in symmetry}
\abstract{This paper presents new data from Bantu languages, from which a hitherto unnoticed typological pattern emerges: A) language-internally, causative, applicative and lexical (‘give’) ditransitives can differ with respect to symmetry; B) crosslinguistically, they are in an implicational relationship: if a language is symmetrical for one type of predicate, it is symmetrical for the predicate types to its right as well:
 
causative >  applicative > lexical ditransitive
 
This can be accounted for if symmetry is due to low functional heads being flexible to license an argument in either their complement or their specifier  \citep{HaddicanHolmberg2012,HaddicanHolmberg2015}. This flexibility is argued to be a sensitivity to topicality. The implicational relation can then be seen as a requirement for lower functional heads to have the same sensitivity: if Caus can license its specifier, then HAppl and LAppl should also be able to do so. 
}
\ChapterDOI{10.5281/zenodo.1116761}

\maketitle

\begin{document}
 

% \title{[Warning: Draw object ignored][Warning: Draw object ignored][Warning: Draw object ignored][Warning: Draw object ignored][Warning: Draw object ignored][Warning: Draw object ignored][Warning: Draw object ignored][Warning: Draw object ignored][Warning: Draw object ignored][Warning: Draw object ignored][Warning: Draw object ignored]Flexibility in symmetry: an implicational relation in \ili{Bantu} double object constructions}

  
\section{Introduction}\label{sec:vdw:1}\is{causative|(}\is{Appl|(}
\citet[54]{BakerEtAl2012} note that “for more than thirty years, symmetrical and asymmetrical object constructions have been a classic topic in the syntax of \ili{Bantu} languages and beyond”. \citet{BresnanMoshi1990} divided \ili{Bantu} languages into two classes -symmetrical and asymmetrical- based on the behaviour of objects in ditransitives: languages are taken to be symmetrical if both objects of a ditransitive verb behave alike with respect to passivisation and pronominalisation (see \citealt{Ngonyani1996,Buell2005} for further tests). In \ili{Zulu}, for example, either object can be object-marked on the verb \REF{ex:vanderwal:1}, and either object can be the subject of a \isi{passive} verb \REF{ex:vanderwal:2}.\\

\noindent \ili{Zulu} \citep[11]{Adams2010}\\
\ea\label{ex:vanderwal:1}%bkm:Ref278661534
\ea\label{ex:vanderwal:1a}
\gll U-mama  u-nik-e  aba-ntwana   in-cwadi.\\
    \oldstylenums{1}a-mama  \oldstylenums{1}\textsc{sm}{}-give-\textsc{pfv}  \oldstylenums{2}-children  \oldstylenums{9}-book\\
    \glt ‘Mama gave the children a book.’

 \ex\label{ex:vanderwal:1b}
\gll U-mama  u-\textbf{ba}{}-nik-e  in-cwadi  (aba-ntwana).\\
    \oldstylenums{1}a-mama  \oldstylenums{1}\textsc{sm}{}-\textbf{\oldstylenums{2}\textsc{om}}{}-give-\textsc{pfv}  \oldstylenums{9}-book  \oldstylenums{2}-children\\
    \glt ‘Mama gave them a book (the children).’

 \ex\label{ex:vanderwal:1c}
\gll U-mama  u-\textbf{yi}{}-nik-e  aba-ntwana   (in-cwadi).\\
    \oldstylenums{1}a-mama  \oldstylenums{1}\textsc{sm}{}-\textbf{\oldstylenums{9}\textsc{om}}{}-give-\textsc{pfv}  \oldstylenums{2}-children  \oldstylenums{9}-book\\
    \glt ‘Mama gave the children it (a book).’
\z
\z

\ea\label{ex:vanderwal:2}%bkm:Ref286386700 
     \ea\label{ex:vanderwal:2a}
\gll In-cwadi  y-a-fund-el-w-a  aba-ntwana.\\
    \oldstylenums{9}-book  \oldstylenums{9}\textsc{sm}{}-\textsc{rem}.\textsc{pst}{}-read-\textsc{appl}{}-\textsc{pass}{}-\textsc{fv}  \oldstylenums{2}-children\\
    \glt ‘The book was read (for) the children.’

 \ex\label{ex:vanderwal:2b}
\gll Aba-ntwana  b-a-fund-el-w-a  in-cwadi.\\
    \oldstylenums{2}-children  \oldstylenums{2}\textsc{sm}{}-\textsc{rem}.\textsc{pst}{}-read-\textsc{appl}{}-\textsc{pass}{}-\textsc{fv}  \oldstylenums{9}-book\\
    \glt ‘The children were read a book.’
\z
\z

However, it has become clear that the situation is not that black-and-white, with ‘symmetrical languages’ showing asymmetry in some part of the language (\citealt{Schadeberg1995}, cf. \citealt{Rugemalira1991,Thwala2006}). It is already known that this asymmetry can be found in a number of ways. First, languages can be symmetrical only for a subpart of the tests (e.g. for object marking but not word order; \citealt{Ngonyani1996,Moshi1998,Riedel2009}). Second, languages can vary in symmetry for different combinations of thematic roles (e.g. instruments versus benefactives; 
\citealt{Baker1988,Marantz1993,AlsinaMchombo1993,Simango1995,Ngonyani1996,Ngonyani1998,ZellerNgoboka2006,Jerro2015} and many others). Third, we are starting to see that combinations of syntactic operations (e.g. relativisation, passivisation, object marking) may also show asymmetry in otherwise symmetrical languages \citep{Adams2010,Zeller2014,HolmbergSheehanvanderWal2015}, see also \sectref{sec:vdw:4.2}.

  This paper presents new data from \ili{Bantu} languages, exhibiting a fourth way in which symmetrical languages can show asymmetry. From this, a hitherto unnoticed typological pattern emerges: A) language-internally, causative, applicative and lexical (‘give’) ditransitives can differ with respect to symmetry; B) crosslinguistically, they are in an implicational relationship: if a language is symmetrical for one type of predicate, it is symmetrical for the predicate types to its right in \REF{ex:vanderwal:3} as well.

\ea\label{ex:vanderwal:3}%bkm:Ref286393438
\gll causative >  applicative >  {lexical ditransitive} >  {(more restricted)}\\
{type 1} {} {type 2} {} {type 3} {}  {type 4}\\
\z

Having discovered this pattern, we want to understand and explain it, which is where Haddican \& Holmberg’s (\citeyear{HaddicanHolmberg2012,HaddicanHolmberg2015}) analysis of symmetry proves useful. In \sectref{sec:vdw:2}, I first show and illustrate the discovered pattern in different languages. In \sectref{sec:vdw:3} I propose a theoretical analysis for asymmetry and the implicational relation of symmetry, while \sectref{sec:vdw:4} presents potential trouble. Note that in the current paper I restrict myself to the thematic roles of Causee, Benefactive, Recipient and Theme; see the conclusion in \sectref{sec:vdw:5} for some discussion on other roles.

\section{Not all ditransitives are equal}\label{sec:vdw:2}

Apart from lexical ditransitive predicates such as ‘give’ or ‘teach’, \ili{Bantu} languages can productively create ditransitive predicates by increasing the valency of verbs with applicative and causative derivations (marked morphologically on the verb), as shown in \REF{ex:vanderwal:4} and \REF{ex:vanderwal:5}, respectively.\\


\noindent Makhuwa (\citealt[71]{VanderWal2009} and database)\\
\ea\label{ex:vanderwal:4}%bkm:Ref167588952
\ea\label{ex:vanderwal:4a}
\gll Amíná  o-n-rúwá  eshimá.\\
    \oldstylenums{1}.Amina  \oldstylenums{1}\textsc{sm}{}-\textsc{pres.cj}{}-stir  \oldstylenums{9}.shima\\
    \glt ‘Amina prepares shima.’
\ex\label{ex:vanderwal:4b}
\gll Amíná  o-n-aá-rúw-\textbf{él}’  éshimá  anámwáne.\\
    \oldstylenums{1}.Amina  \oldstylenums{1}\textsc{sm}{}-\textsc{pres.cj}{}-\oldstylenums{2}\textsc{om}{}-stir-\textsc{appl.fv}  \oldstylenums{9}.shima  \oldstylenums{2}.children\\
    \glt ‘Amina prepares shima for the children.’
\z
\z


\ea\label{ex:vanderwal:5}%bkm:Ref167588442
\ea\label{ex:vanderwal:5a}
\gll Ál’  átthw’  áálá  aa-wárá  eshaphéyu.\\
    \oldstylenums{2}.\textsc{dem}  \oldstylenums{2}.people  \oldstylenums{2}.\textsc{dem}  \oldstylenums{2}\textsc{sm.perf.dj}{}-wear  \oldstylenums{10}.hats\\
    \glt ‘These people wear hats.’
\ex\label{ex:vanderwal:5b}
 \gll O-\'{m}-wár-\textbf{íh}{}-á  mwalápw’  ááwé  ekúwó.\\
    \oldstylenums{1}\textsc{sm}.\textsc{perf}.\textsc{dj}{}-\oldstylenums{1}\textsc{om}{}-wear-\textsc{caus-fv}  \oldstylenums{1}.dog  \oldstylenums{1}.\textsc{poss}.\oldstylenums{1}  \oldstylenums{9}.cloth \\
    \glt ‘She dressed her dog in a cloth.’
\z
\z

Although the Benefactive (children) and the Causee (dog) fully belong to the argument structure of the verb, just like the Recipient and Theme in a lexical ditransitive such as ‘give’, not all languages treat the two objects in these three types of ditransitives in the same symmetrical or asymmetrical way. As mentioned, an implicational relationship appears between the symmetrical behaviour of double objects in causatives, applicatives and lexical ditransitives, as in \REF{ex:vanderwal:3} above. The types of symmetry patterns are illustrated for object marking in various languages below; passivisation is in the various languages confirmed or expected to follow the same pattern, but only object marking will be discussed in this paper.

\subsection{Type 1: fully symmetrical}\label{sec:vdw:2.1}

On one end of the continuum are languages that behave symmetrically for all three types of ditransitive constructions. \ili{Zulu} is one such language: both objects behave symmetrically, whether they belong to a lexical ditransitive verb or a derived applicative or causative. This is illustrated for object marking in (\ref{ex:vanderwal:6}--\ref{ex:vanderwal:8}) and yields the same results for passivisation. \ili{Zulu} is thus a language of type 1: symmetrical for all types of verbs.\\


\noindent \ili{Zulu} (\citealt{Zeller2011}, see also \citealt{Zeller2012})
\ea\label{ex:vanderwal:6}{lexical ditransitive}\\%bkm:Ref286393156
 \ea\label{ex:vanderwal:6a} 
\gll UJohn  u-nik-a  abantwana  imali.\\
    \oldstylenums{1}a.John  \oldstylenums{1}\textsc{sm}{}-give-\textsc{fv}  \oldstylenums{2}.children  \oldstylenums{9}.money\\
    \glt ‘John is giving the children money.’

 \ex\label{ex:vanderwal:6b}
\gll UJohn  u-\textbf{ba}{}-nik-a  imali  (abantwana). \\
    \oldstylenums{1}a.John  \oldstylenums{1}\textsc{sm}{}-\oldstylenums{2}\textsc{om}{}-give-\textsc{fv}  \oldstylenums{9}.money  \oldstylenums{2}.children\\
    \glt ‘John is giving them money (the children).’

 \ex\label{ex:vanderwal:6c}
\gll UJohn  u-\textbf{yi}{}-nik-a  abantwana   (imali). \\
    \oldstylenums{1}a.John  \oldstylenums{1}\textsc{sm-\oldstylenums{9}om}{}-give-\textsc{fv}  \oldstylenums{2}.children  \oldstylenums{9}.money \\
    \glt ‘John is giving it to the children (the money).’
\z
\z

\newpage 
 \ea\label{ex:vanderwal:7}
  {applicative}
  \ea\label{ex:vanderwal:7a}
\gll ULanga  u-phek-el-a  umama  inyama. \\
    \oldstylenums{1}a.Langa  \oldstylenums{1}\textsc{sm}{}-cook-\textsc{appl}{}-\textsc{fv}  \oldstylenums{1}a.mother  \oldstylenums{9}.meat\\
    \glt ‘Langa is cooking meat for mother.’

 \ex\label{ex:vanderwal:7b}
\gll ULanga  u-\textbf{m}{}-phek-el-a  inyama  (umama). \\
    \oldstylenums{1}a.Langa  \oldstylenums{1}\textsc{sm-\oldstylenums{1}om}-cook-\textsc{appl}{}-\textsc{fv}  \oldstylenums{9}.meat  \oldstylenums{1}a.mother \\
    \glt ‘Langa is cooking meat for her (mother).’

 \ex\label{ex:vanderwal:7c}
\gll ULanga  u-\textbf{yi}{}-phek-el-a  umama  (inyama). \\
    \oldstylenums{1}a.Langa  \oldstylenums{1}\textsc{sm-\oldstylenums{9}om}{}-cook-\textsc{appl}{}-\textsc{fv}  \oldstylenums{1}.mother  \oldstylenums{9}.meat \\
    \glt ‘Langa is cooking it for mother (the meat).’
\z
\z

 \ea\label{ex:vanderwal:8}
      {causative}
      \ea\label{ex:vanderwal:8a}
\gll ULanga  u-phek-is-a  umama  ukudla. \\
    \oldstylenums{1}a.Langa  \oldstylenums{1}\textsc{sm}{}-cook-\textsc{caus}{}-\textsc{fv}  \oldstylenums{1}a.mother  \oldstylenums{15}.food \\
    \glt ‘Langa helps/makes mother cook food.’


 \ex\label{ex:vanderwal:8b}
\gll ULanga  u-\textbf{m}{}-phek-is-a  ukudla  (umama). \\
    \oldstylenums{1}a.Langa  \oldstylenums{1}\textsc{sm-\oldstylenums{1}om}{}-cook-\textsc{caus-fv}  \oldstylenums{15}.food  \oldstylenums{1}a.mother \\
    \glt ‘Langa helps/makes her cook food (mother).’

 \ex\label{ex:vanderwal:8c}
\gll ULanga  u-\textbf{ku}{}-phek-is-a  umama  (ukudla). \\
    \oldstylenums{1}a.Langa  \oldstylenums{1}\textsc{sm-\oldstylenums{15}om}{}-cook-\textsc{caus-fv}  \oldstylenums{1}a.mother  \oldstylenums{15}.food \\
    \glt ‘Langa makes mother cook it (the food).’
\z
\z

The same full symmetry has been observed in \ili{Kimeru} \citep{Hodges1977}, 
\ili{Shona} (\citealt{Mugari2013,MathangwaneOsam2006}), 
\ili{Lubukusu} \citep{BakerEtAl2012}, 
Kinyarwanda (\citealt{ZellerNgoboka2014,Ngoboka2005}), 
\ili{Kîîtharaka} \citep{Muriungi2008}, 
and Kikuyu (Peter Githinji, personal communication).

\subsection{Type 2: only lexical and applicative symmetrical}\label{sec:vdw:2.2}

One step further down the cline are languages of type 2, where objects of applicatives and lexical ditransitives behave symmetrically, but objects of causatives do not. In \ili{Southern Sotho}, either object of lexical ditransitives and applicatives can be object-marked, as in \REF{ex:vanderwal:9} and \REF{ex:vanderwal:10},\footnote{But see the influence of animacy as pointed out for Sesotho by \citet{MorolongHyman1977} and comparatively discussed in \citet{HymanDuranti1982}.} whereas with a causative only the Causee can be marked, not the Theme \REF{ex:vanderwal:11}.\\

\noindent \ili{Southern Sotho} 
\ea\label{ex:vanderwal:9}%bkm:Ref286433837
 {lexical ditransitive (Thabo Ditsele, personal communication)}\\
 \ea\label{ex:vanderwal:9a}
\gll Ntate  o  fa  bana  lijo. \\
    \oldstylenums{1}.father  \oldstylenums{1}\textsc{sm}  give  \oldstylenums{2}.children  \oldstylenums{5}.food\\
    \glt ‘Father gives the children food.’

 \ex\label{ex:vanderwal:9b}
\gll Ntate  o  \textbf{ba}  fa  lijo.\\
    \oldstylenums{1}.father  \oldstylenums{1}\textsc{sm}  \oldstylenums{2}\textsc{om}  give  \oldstylenums{5}.food\\
    \glt ‘Father gives them food.’

 \ex\label{ex:vanderwal:9c}
\gll Ntate   o  \textbf{li}  fa  bana.\\
    \oldstylenums{1}.father  \oldstylenums{1}\textsc{sm}  \oldstylenums{5}\textsc{om}  give  \oldstylenums{2}.children\\
    \glt ‘Father gives it to the children.’
\z
\z

 \ea\label{ex:vanderwal:10} applicative \citep[24]{Machobane1989}\\
  \ea\label{ex:vanderwal:10a}
\gll Banana  ba-pheh-el-a  ´me  nama.\\
    \oldstylenums{2}.girls  \oldstylenums{2}\textsc{sm}{}-cook-\textsc{appl}{}-\textsc{fv}  \oldstylenums{1}.mother  \oldstylenums{9}.meat\\
    \glt ‘The girls are cooking meat for my mother.’

 \ex\label{ex:vanderwal:10b}
\gll Banana  ba-\textbf{mo}{}-pheh-el-a  nama.\\
    \oldstylenums{2}.girls  \oldstylenums{2}\textsc{sm}{}-cook-\textsc{appl}{}-\textsc{fv}  \oldstylenums{9}.meat\\
    \glt ‘The girls are cooking meat for her.’

 \ex\label{ex:vanderwal:10c}
\gll Banana  ba-\textbf{e}{}-pheh-el-a  ´me.\\
    \oldstylenums{2}.girls  \oldstylenums{2}\textsc{sm-\oldstylenums{9}om}{}-cook-\textsc{appl}{}-\textsc{fv}  \oldstylenums{1}.mother \\
    \glt ‘The girls are cooking it for my mother.’
\z
\z

\ea\label{ex:vanderwal:11}%bkm:Ref286433872

     {causative \citep[31]{Machobane1989}}\\
 \ea[]{\label{ex:vanderwal:11a}
\gll Ntate  o-bal-is-a  bana  buka.\\
    \oldstylenums{1}.father  \oldstylenums{1}\textsc{sm}{}-read-\textsc{caus}{}-\textsc{fv}  \oldstylenums{2}.children  \oldstylenums{9}.book\\
    \glt ‘My father makes the children read the book.’}

 \ex[]{\label{ex:vanderwal:11b}
\gll Ntate  o-\textbf{ba}{}-bal-is-a  buka.\\
    \oldstylenums{1}.father  \oldstylenums{1}\textsc{sm}{}-\oldstylenums{2}\textsc{om}{}-read-\textsc{caus}{}-\textsc{fv}  \oldstylenums{9}.book\\
    \glt ‘My father makes them read the book.’}

 \ex[*]{\label{ex:vanderwal:11c}
\gll Ntate  o-\textbf{e}{}-bal-is-a  bana.\\
      \oldstylenums{1}.father  \oldstylenums{1}\textsc{sm}{}-\oldstylenums{9}\textsc{om}{}-read-\textsc{caus}{}-\textsc{fv}  \oldstylenums{2}.children\\
 \glt     int. ‘My father makes the children read it.’}
\z
\z

The same pattern is found in \ili{Otjiherero}, as shown in (\ref{ex:vanderwal:12}--\ref{ex:vanderwal:14}):

\noindent \ili{Otjiherero}\\
\ea\label{ex:vanderwal:12}
  {lexical ditransitive (Jekura Kavari, personal communication)}\\
 \ea\label{ex:vanderwal:12a}
\gll Omukazendu  ma  pe  ovazandu  ovikurya.\\
    \oldstylenums{1}.woman  \textsc{pres}  \oldstylenums{1}\textsc{sm}.give  \oldstylenums{2}.boys  \oldstylenums{8}.food\\
    \glt ‘The woman gives the boys food.’

 \ex\label{ex:vanderwal:12b}
\gll Omukazendu  me  \textbf{ve}  pe  ovikurya.\\
    \oldstylenums{1}.woman  \textsc{pres.\oldstylenums{1}sm}  \oldstylenums{2}\textsc{om}  give  \oldstylenums{8}.food\\
    \glt ‘The woman gives them food.’

 \ex\label{ex:vanderwal:12c}
\gll Omukazendu  me  \textbf{vi}  pe  ovazandu.\\
    \oldstylenums{1}.woman  \textsc{pres.\oldstylenums{1}sm}  \oldstylenums{8}\textsc{om}  give  \oldstylenums{2}.boys \\
    \glt ‘The woman gives it to the children.’
\z
\z

\ea\label{ex:vanderwal:13}
{applicative \citep[247]{MartenKula2012}}
 \ea\label{ex:vanderwal:13a}
\gll Má-vé  \textbf{vè}  tjáng-ér-é  òm-bàpírà.\\
    \textsc{pres}{}-\oldstylenums{2}\textsc{sm}  \oldstylenums{2}\textsc{om}  write-\textsc{appl}{}-\textsc{fv}  \oldstylenums{9}-letter \\
    \glt ‘They are writing them a letter.’

 \ex\label{ex:vanderwal:13b}
\gll Má-vá  \textbf{ì}  tjáng-ér-é  òvà-nâtjé.\\
    pres-\oldstylenums{2}\textsc{sm}  \oldstylenums{9}\textsc{om}  write-\textsc{appl}{}-\textsc{fv}  \oldstylenums{2}-children\\
    \glt ‘They are writing the children it.’
\z
\z

\ea\label{ex:vanderwal:14}
 {causative (Jekura Kavari, personal communication)}
 \ea[]{\label{ex:vanderwal:14a}
\gll Ma-ve  \textbf{ve}  tjang-is-a  om-bapira. \\
    \textsc{pres}{}-\oldstylenums{2}\textsc{sm}  \oldstylenums{2}\textsc{om}  write-\textsc{caus}{}-\textsc{fv}  \oldstylenums{9}-letter\\
    \glt ‘They make them write a letter.’}

 \ex[*]{\label{ex:vanderwal:14b}
\gll Ma-ve  \textbf{i}  tjang-is-a  ova-natje.\\
      \textsc{pres}{}-\oldstylenums{2}\textsc{sm}  \oldstylenums{9}\textsc{om}  write-\textsc{caus}{}-\textsc{fv}  \oldstylenums{2}-children\\
    \glt ‘They make the children write it.’}
\z
\z

\subsection{Type 3: only lexical symmetrical}\label{sec:vdw:2.3}

Type 3 is yet another step down the hierarchy in \REF{ex:vanderwal:3}. In Ki\ili{Luguru}, double objects behave symmetrically only for lexical ditransitives \REF{ex:vanderwal:15}, but show asymmetries with both applicative and causative predicates (\ref{ex:vanderwal:16}--\ref{ex:vanderwal:17}).\\

\newpage 
\noindent Ki\ili{Luguru} \citep[266, 269]{MartenRamadhani2001}
\ea\label{ex:vanderwal:15}%bkm:Ref286433906
{lexical ditransitive}\\
 \ea\label{ex:vanderwal:15a}
\gll Chibua  ko-\textbf{w}{}-eng’-a  iwana  ipfitabu.\\
    \oldstylenums{1}.Chibua  \oldstylenums{1}\textsc{sm}{}-\oldstylenums{2}\textsc{om}{}-give-\textsc{fv}  \oldstylenums{2}.children  \oldstylenums{8}.books\\

 \ex\label{ex:vanderwal:15b}
\gll Chibua  ko-\textbf{pf}{}-eng’-a  iwana  ipfitabu.\\
    \oldstylenums{1}.Chibua  \oldstylenums{1}\textsc{sm}{}-\oldstylenums{8}\textsc{om}{}-give-\textsc{fv}  \oldstylenums{2}.children  \oldstylenums{8}.books\\
    \glt ‘Chibua is giving children books.’

\z
\z
  
  \ea\label{ex:vanderwal:16}{applicative}\footnotemark\\
 \ea[]{\label{ex:vanderwal:16a}
\gll Mayi  ko-\textbf{w}{}-ambik-il-a  iwana  ipfidyo.\\
    \oldstylenums{1}.mother  \oldstylenums{1}\textsc{sm}{}-\oldstylenums{2}\textsc{om}{}-cook-\textsc{appl}{}-\textsc{fv}  \oldstylenums{2}.children  \oldstylenums{7}.food\\
    \glt ‘Mother is cooking food for the children.’}

 \ex[*]{\label{ex:vanderwal:16b}
\gll Mayi  ko-\textbf{pf}{}-ambik-il-a   ipfidyo  iwana.\\
      \oldstylenums{1}.mother  \oldstylenums{1}\textsc{sm-\oldstylenums{7}om}{}-cook-\textsc{appl}{}-\textsc{fv}  \oldstylenums{7}.food  \oldstylenums{2}.children\\
\glt int. ‘Mother is cooking food for the children.’}
\z
\z
\footnotetext{\citet[266]{MartenRamadhani2001} note that “both orders of objects are fine, but only the benefactive object may be object marked (in general, the object marked object precedes the unmarked object, and it is the first object which is emphasized. In addition, applicatives without valency change can be used for predicate emphasis”.}
  
  \ea\label{ex:vanderwal:17}
 \ea[]{\label{ex:vanderwal:17a}
  {causative}\\
\gll Wanzehe  wa-\textbf{mw}{}-ambik-its-a  Chuma  ipfidyo.\\
    \oldstylenums{2}.elders  \oldstylenums{2}\textsc{sm}{}-\oldstylenums{1}\textsc{om}{}-cook-\textsc{caus}{}-\textsc{fv}  \oldstylenums{1}.Chuma  \oldstylenums{8}.food\\
    \glt ‘The elders made Chuma cook food.’}

 \ex[*]{\label{ex:vanderwal:17b}
\gll Wanzehe  wa-\textbf{pf-}ambik-its-a  ipfidyo   Chuma.\\
       \oldstylenums{2}.elders  \oldstylenums{2}\textsc{sm}{}-\oldstylenums{8}\textsc{om}{}-cook-\textsc{caus}{}-\textsc{fv}  \oldstylenums{8}.food  \oldstylenums{1}.Chuma\\
    \glt ‘The elders made Chuma cook food.’}
\z
\z


\subsection{Type 4: fully asymmetrical}\label{sec:vdw:2.4}

Finally, type 4 languages do not show any symmetrical properties in \is{DOC}double object constructions -- these have always been known as asymmetrical languages. In ditransitives, applicatives and causatives, only the \mbox{Recipient}\slash applied\slash Causee object can be object-marked, as shown in (\ref{ex:vanderwal:18}--\ref{ex:vanderwal:20}).\\

\newpage 
\noindent \ili{Swahili}
\ea \label{ex:vanderwal:18}
lexical ditransitive\\
 \ea[]{\label{ex:vanderwal:18a}
\gll A-li-\textbf{m}{}-pa  kitabu.\\
    \oldstylenums{1}\textsc{sm-past-\oldstylenums{1}om}{}-give  \oldstylenums{7}.book\\
    \glt ‘She gave him a book.’}

 \ex[*]{\label{ex:vanderwal:18b}
\gll A-li-\textbf{ki}{}-pa  Juma.\\
    \oldstylenums{1}\textsc{sm-past-\oldstylenums{7}om}{}-give  \oldstylenums{1}.Juma\\
    \glt ‘She gave it to Juma.’}
\z
\z

\ea\label{ex:vanderwal:19}
 {applicative}\\
 \ea[]{\label{ex:vanderwal:19a}
\gll A-li-\textbf{m-}nunul-i-a  kitabu.\\
    \oldstylenums{1}\textsc{sm-past-\oldstylenums{1}om}{}-buy-\textsc{appl}{}-\textsc{fv}  \oldstylenums{7}.book\\
    \glt ‘She bought him a book.’}

 \ex[*]{\label{ex:vanderwal:19b}
\gll A-li-\textbf{ki}{}-nunul-i-a  Juma. \\
    \oldstylenums{1}\textsc{sm-past-\oldstylenums{7}om}{}-buy-\textsc{appl}{}-\textsc{fv}  \oldstylenums{1}.Juma\\
    \glt ‘She bought it for Juma.’}
\z
\z

\ea\label{ex:vanderwal:20}
{causative}\\
 \ea[]{\label{ex:vanderwal:20a}
\gll A-li-\textbf{m}{}-kat-ish-a  kamba.\\
    \oldstylenums{1}\textsc{sm-past-\oldstylenums{1}om}{}-cut-\textsc{caus-fv}  \oldstylenums{9}.rope\\
    \glt ‘She made him cut the rope.’}

 \ex[*]{\label{ex:vanderwal:20b}
\gll A-li-\textbf{i}{}-kat-isha  Juma.\\
    \oldstylenums{1}\textsc{sm-past-\oldstylenums{9}om}{}-cut-\textsc{caus-fv}  \oldstylenums{1}.Juma\\
    \glt ‘She made Juma cut it.’}
\z
\z

\subsection{Summary of (a)symmetrical patterns}\label{sec:vdw:2.5}

The languages studied thus illustrate that ‘symmetry’ is not necessarily a property of a whole language, and they also show that (some of) the variation in symmetrical object marking is structured, as summarised in \tabref{tab:vanderwal:1}.

\begin{table}[H]
\caption{Symmetrical properties of double object constructions cross-Bantu}\label{tab:vanderwal:1}
\begin{tabularx}{\textwidth}{lcccX}
\lsptoprule 
& \textsc{caus} & \textsc{appl} & \textsc{ditrans} & languages\\
\midrule
type 1 & \checkmark & \checkmark & \checkmark & {Zulu},\il{Zulu} \ili{Shona}, \ili{Lubukusu}, \ili{Kîîtharaka}, \ili{Kimeru}\\
type 2 &  & \checkmark & \checkmark &  {Otjiherero},\il{Otjiherero} \ili{Southern Sotho}\\
type 3 &  &  & \checkmark & {Luguru}\il{Luguru}\\
type 4 &  &  &  & {Swahili}\il{Swahili} etc. (asymmetrical)\\
\lspbottomrule
\end{tabularx}
\end{table}

\section{Implications of the implicational hierarchy}\label{sec:vdw:3}

This implicational relation poses an empirical as well as a theoretical question. The empirical question is the following: If the implicational hierarchy in \REF{ex:vanderwal:3} holds crosslinguistically, are there indeed no languages with symmetrical double objects for applicatives and/or causatives but not ditransitives, and similarly are there no languages with symmetrical causatives but no symmetrical applicatives? This is a very clear empirical prediction that should be tested as more data become available for more languages.

Assuming that the pattern in \tabref{tab:vanderwal:1} is not accidental, the theoretical question is how this implicational relation can be accounted for in a model of syntax. In order to answer that question, we need to establish how symmetry is derived, which in turn requires a theory of the functional structure of the lower part of the clause and of object marking. I first present the structure of ditransitives in \sectref{sec:vdw:3.1} and the mechanics of object marking in \sectref{sec:vdw:3.2}, then I introduce Haddican and Holmberg’s (\citeyear{HaddicanHolmberg2012,HaddicanHolmberg2015}) analysis of symmetry in \sectref{sec:vdw:3.3}, and I add a motivation for it in \sectref{sec:vdw:3.4}. With all these ingredients in place (summary in \sectref{sec:vdw:3.5}), I return to the implicational relationship in \sectref{sec:vdw:3.6}.

\subsection{The structure of ditransitives}\label{sec:vdw:3.1}

Following \citet{Pylkkänen2008}, and considering the overt applicative and causative morphology in \ili{Bantu}, I take the Recipient in a lexical ditransitive to be introduced by a low applicative head (LApplP), under V \REF{ex:vanderwal:21a}. The Benefactive for an applied verb is introduced by a high applicative head (HApplP), between V and v \REF{ex:vanderwal:21b}. For causatives, I assume that the Causee is introduced by a causative head (CausP) between V and v \REF{ex:vanderwal:21c}, although one could equally well assume a double little v with Caus in between, forming a bi-eventive structure (see further \citealt{Pylkkänen2008} on different heights of causatives).

\ea%bkm:Ref301010565
\label{ex:vanderwal:21}\begin{multicols}{2}
\ea\label{ex:vanderwal:21a}\begin{forest} [vP [EA] [,shape=coordinate [v] [VP [V] [LApplP [R] [,shape=coordinate [LAppl] [TH] ] ] ] ] ]\end{forest}
\ex\label{ex:vanderwal:21b}\begin{forest} [vP [EA] [,shape=coordinate [v] [HApplP [BEN] [,shape=coordinate [HAppl] [VP [V] [TH] ] ] ] ] ] \end{forest}
\ex\label{ex:vanderwal:21c}\begin{forest} [vP [EA] [,shape=coordinate [v] [CausP [CAUS] [,shape=coordinate [Caus] [VP [V] [\isi{TP}] ] ] ] ] ]  \end{forest}
\z
\end{multicols}
\z

If these structures underlie the double object constructions\is{DOC} discussed, then they (and indeed the underlying conceptual considerations of generative grammar) suggest that asymmetry is basic, and symmetry is derived.\footnote{This may be different for locative or instrumental applicatives -- tests involving animacy could help to assess whether there is a ‘dative alternation’ as in English or a true double object construction\is{DOC}, see \citet{Oehrle1976}, among others.} This appears to be correct, since asymmetries keep cropping up in otherwise symmetrical languages but never the other way around, suggesting that asymmetry is always available and hence more basic. Furthermore, the asymmetry is always the same across \ili{Bantu}: the Benefactive, Causee, or applied (i.e. higher) argument displays object properties, where the Theme argument lacks them. This supports an analysis of symmetry in terms of a derived accessibility of the Theme, i.e. the Theme starts out low and becomes available for syntactic operations (by movement, different featural probing or annihilating the intervening argument). This is further discussed in \sectref{sec:vdw:3.3}.

\subsection{Object marking in ditransitives}\label{sec:vdw:3.2}

I assume that \ili{Bantu} object marking in ditransitives is the result of an \isi{Agree} relation between little v and one of the objects. Within the Probe-Goal system of \isi{Agree} \citep{Chomsky2001}, I assume that object markers are the spell-out of little v’s uninterpretable φ features \isi{agreeing} with the interpretable φ features of an object Goal \citep{Roberts2010}.\footnote{Under Roberts’ (\citeyear{Roberts2010}) approach, object marking is the spell-out of an \isi{Agree} relation with a defective Goal: if the features of the Goal are a subset of the features of the Probe, the \isi{Agree} relation is indistinguishable from a copy/movement chain, where normally only the highest copy is spelled out. The lower copy is not spelled out, due to chain-reduction \citep{Nunes2004}. This gives rise to incorporation of the Goal, being spelled out on the Probe. Whether the \isi{Agree} relation is spelled out morphologically is thus dependent on the structure of the Goal. See \citet{Iorio2014} for details on the approach as applied to the \ili{Bantu} language Bembe, and \citet{VanderWal2015objectclitics} for a comparative approach to \ili{Bantu} object marking.} I further assume that lower arguments need \isi{Case} licensing,\footnote{This is debatable for the \ili{Bantu} languages; see \citet{Diercks2012,VanderWal2015abstractcase} and \citet{SheehanVanderWal2016}. However, the debatable status mostly concerns nominative \isi{Case}.}  and that \isi{Case} licensing can be independent of φ \isi{agreement}, in the sense that a lower functional head can be Case-licensing but not carry uφ features (\citealt{Baker2012Onthe,Preminger2014,Bárány2015}). Lower functional heads can thus have a [uφ] and/or a [\isi{Case}] feature.

  In a monotransitive structure, the uninterpretable features on v simply probe, find the first and only object (the Theme) and agree with it. In a \is{DOC}double object construction, however, the Theme argument is always lower than the Recipient\slash Benefactive\slash Causee argument. Assuming that locality conditions hold (Minimal Link Condition),\footnote{But see \citet{BakerCollins2006} who propose parameterisation of the Minimal Link Condition.}     the Theme is not available for \isi{agreement} with the v or T head for object marking and passivisation, respectively. This is due to one of two reasons: either the higher argument will intervene between the Probe on v/T and the Theme, or the \isi{Appl}/Caus head will already have licensed the Theme, making it inactive for further \isi{Agree} relations. This is what results in asymmetry: the LAppl/HAppl/Caus head always licenses the Theme in its c-command domain, and v can only license the highest argument. Since only v has φ features, it follows that only the highest object can be spelled out as object marking (if the Goal is defective). This is represented in \REF{ex:vanderwal:22}.

\ea%bkm:Ref301009664
\label{ex:vanderwal:22}v agrees with BEN (and can \isi{spell out} as object-marker)\\
\begin{forest}
[vP [~~~~~] [,shape=coordinate [v{[}φ{]}, name=phi] [HApplP [BEN, name=ben] [,shape=coordinate [HAppl, name=happl] [VP [V] [TH, name=th] ] ] ] ] ]
\draw[dashed,-{Stealth[]}] (phi) -- (ben);
\path[dashed,-{Stealth[]}] (happl) edge [bend right,out=250,in=250] (th);
\end{forest}
\z

\subsection{Symmetry}\label{sec:vdw:3.3}

In “symmetrical languages” the Theme can also be object marked. The [uφ] features of v must thus have established an \isi{Agree} relation with the lower Theme, despite an intervening Benefactive.\footnote{I will illustrate the analysis with a high applicative, but the same holds for the low applicative and the causative.} Assuming locality conditions, if the Theme is agreed with, it must either have been higher than the Benefactive at the time of \isi{agreement} (the locality approach), or the Benefactive must have somehow been invisible for v’s Probe (the \isi{Case} approach).

The locality analysis is proposed by 
McGinnis (\citeyear*{McGinnis1998a,McGinnis2001});
\citet{Anagnostopoulou2003,Doggett2004,Pylkkänen2008,Jeong2007}. They propose that a high applicative between V and v supplies a landing place for the Theme object in a second specifier \REF{ex:vanderwal:23}, whether attracted by \isi{Appl} itself or moving to a phase edge (\isi{Appl} being argued to be a phase head). This results in the Theme being closer to v than the applied argument. 

\ea\label{ex:vanderwal:23}%bkm:Ref287081629
\begin{forest}
[\isi{TP} [] [,shape=coordinate  [T] [vP [v] [ApplP [TH,name=th] [ApplP [BEN] [,shape=coordinate [\isi{Appl}] [VP [V] [\st{TH},name=stth] ] ] ] ] ] ] ] 
\path[-{Stealth[]}] (stth.south) edge [bend left, in=150, out=90] (th.south);
\end{forest}
\z
\citet{Ura1996} and \citet{Anagnostopoulou2003} explicitly link this movement to object shift (cf. \citealt{Kramer2014,Harizanov2014,BakerKramer2016}). However, there is not always evidence for such movement, for example when a language is by and large symmetrical but has a very strict word order, as in Luganda. Luganda double objects display symmetrical behaviour for the two tests of pronominalisation \REF{ex:vanderwal:24} and passivisation \REF{ex:vanderwal:25}.\\

\noindent Luganda \citep[67, 72]{Ssekiryango2006}
\ea\label{ex:vanderwal:24}%bkm:Ref286389666
     
     \ea\label{ex:vanderwal:24a}
\gll Maama  a-wa-dde  taata  ssente.\\
    \oldstylenums{1}.mother  \oldstylenums{1}\textsc{sm}{}-give-\textsc{pfv}  \oldstylenums{1}.father  \oldstylenums{10}.money \\

    \glt ‘Mother has given father money.’

 \ex\label{ex:vanderwal:24b}
\gll Maama  a-mu-wa-dde  ssente. \\
    \oldstylenums{1}.mother  \oldstylenums{1}\textsc{sm}{}-\oldstylenums{1}\textsc{om}{}-give-\textsc{pfv}  \oldstylenums{10}.money. \\
    \glt ‘Mother has given him money.’

 \ex\label{ex:vanderwal:24c}
\gll Maama  a-zi-wa-dde  taata.\\
    \oldstylenums{1}.mother  \oldstylenums{1}\textsc{sm}{}-\oldstylenums{10}\textsc{om-}give-\textsc{pfv}  \oldstylenums{1}.father\\
    \glt ‘Mother has given it father.’

\z
\z

\ea\label{ex:vanderwal:25}
     \ea\label{ex:vanderwal:25a}
\gll Maama  a-were-ddw-a  ssente. \\
    \oldstylenums{1}.mother  \oldstylenums{1}\textsc{sm}{}-give-\textsc{pass}{}-\textsc{fv}  money \\
    \glt ‘Mother has been given money.’ 

 \ex\label{ex:vanderwal:25b}
\gll Ssente  zi-were-ddw-a  maama.\\
    \oldstylenums{10}.money  \oldstylenums{10}\textsc{sm}{}-give-\textsc{pass}{}-\textsc{fv}  \oldstylenums{1}.mother \\
    \glt ‘The money has been given to mother.’ 
\z
\z

Nevertheless, Luganda shows a strict order Recipient > Theme, as is clear from \REF{ex:vanderwal:26} as compared to \REF{ex:vanderwal:24a}.


\ea\label{ex:vanderwal:26}%bkm:Ref286390540
Luganda \citep[69]{Ssekiryango2006}\\
\gll      * Maama  a-wa-dde  ssente  taata. \\
    {} \oldstylenums{1}.mother  \oldstylenums{1}\textsc{sm}{}-give-\textsc{pfv}  \oldstylenums{10}.money  \oldstylenums{1}.father\\
    \glt int. ‘Mother gave father money.’
\z

Furthermore, \citet{HaddicanHolmberg2012,HaddicanHolmberg2015} show that the correlation between object shift and symmetry is not corroborated by their research on \ili{Norwegian} and \ili{Swedish}, and they find that it is insufficient to rely on \textit{just} locality to account for all the patterns found in Germanic languages.

Another problematic \isi{aspect} of the locality-based approach, at least for \citet{McGinnis2001}, is that it predicts low applicatives to never be symmetrical. McGinnis proposes that lower arguments can only move to the second specifier of a phase head, that is, it ‘leapfrogs’ to the escape hatch. This functions well with high applicatives but does not work for low applicatives because, under McGinnis’ analysis, this HAppl is a phase whereas LAppl is not. However, even if LAppl could be a phase, then it would still not allow the Theme to be moved to its specifier, since this would involve moving too locally, the same argument merging again with the same head. \citet{Abels2003} observes that because of antilocality, direct complements of phase heads are frozen: they cannot escape by moving to the specifier of the phase head. For \is{DOC}double object constructions, this means that the Theme in a low applicative can never move higher than the Recipient (unless there is a higher phase head it can move to), and therefore it will never be the first argument found by v. However, if lexical ditransitives involve a low applicative (as suggested by their semantics), such symmetrical low applicative structures do exist -- they are even the most frequent in comparison with other ditransitive predicates, as the data in \sectref{sec:vdw:2} show.

\citet{HaddicanHolmberg2012,HaddicanHolmberg2015} propose a different approach to symmetry in \is{DOC}double object constructions: symmetry can derive from locality, but can also derive from variation in whether the extra \isi{Case} associated with an applicative construction is assigned to the Theme or the Benefactive. This can be rephrased as variation in the ability of a functional head (applicative, causative) to assign \isi{Case} to either the Theme object in its complement or to the Benefactive object in its specifier, as represented in \REF{ex:vanderwal:27}. This means that v agrees with the remaining object, which can be either the Benefactive or the Theme, thereby deriving symmetry.

\ea\label{ex:vanderwal:27}%bkm:Ref301001758
\begin{forest} %baseline
[\isi{TP} [~~~~~] [,shape=coordinate [T] [vP [EA] [,shape=coordinate [v\\{[}uφ{]}\\{[}\isi{Case}{]},base=top, align=center] [ApplP [BEN\\{[}iφ{]}\\{[}uCase{]},base=top, align=center,name=ben] [,shape=coordinate,s sep=10mm [\isi{Appl},name=appl] [VP [V] [TH\\{[}iφ{]}\\{[}uCase{]},base=top, align=left,name=ucase] ] ] ] ] ] ] ]
\node[below=.75cm of appl.north] (belowC) {[\isi{Case}]};
\path [-{Stealth[]}] (belowC.250) edge [bend left=65] (ben.south);
\path [-{Stealth[]}] (belowC.260) edge [bend right] (ucase.200);
\end{forest}
\z

There are thus two possible derivations. If the applicative head agrees with the Theme, then v agrees with the highest argument (Benefactive); this is the same as in asymmetrical languages, see \REF{ex:vanderwal:22}.\footnote{Beyond \ili{Bantu} there is another type of asymmetrical language with a so-called “indirective alignment” of double objects, where the lower functional head always licenses its specifier (e.g. \ili{Italian}). This is an independent parameter (see \sectref{sec:vdw:3.6}).}  If in a symmetrical language the applicative head assigns \isi{Case} to its specifier, i.e. to the Benefactive that it introduces, then this argument becomes invisible to v (cf. \citealt{McGinnis1998b}).\footnote{Assuming no defective intervention clause-internally, which has been argued for by \citet{Anagnostopoulou2003} and \citet{Bobaljik2008}. See also \citet{Bruening2014} for an argument against defective intervention per se.}  The Theme object can thus be probed by v, which agrees with it in both \isi{Case} and φ, and potentially \isi{spell out} as an object marker, as represented in \REF{ex:vanderwal:28}.

\ea\label{ex:vanderwal:28}%bkm:Ref304795802
v agrees with TH (and can object-mark it)\\
\begin{forest}
[vP [~~~~~] [,shape=coordinate [v {[}φ{]},name=vphi] [HApplP [BEN,name=ben] [,shape=coordinate [HAppl,name=happl] [VP [V] [TH,name=TH] ] ] ] ] ]
\path [-{Stealth[]}] (vphi) edge [bend right=90] (TH);
\path [-{Stealth[]}] (happl.west) edge [bend left=90] (ben);
\end{forest}
\z

Note that the applicative head here only has a [\isi{Case}] feature and no [uφ] features. The presence of the \isi{Case} feature ensures that the second object is licensed (and invisible for v), whereas the absence of [uφ] features on \isi{Appl} means that the argument \isi{agreeing} with \isi{Appl} cannot be object-marked: only the argument \isi{agreeing} with v can \isi{spell out} as an object marker. The presence of [uφ] just on v also accounts for the fact that there is only one object marker. 

  In languages with multiple object markers, such as Kinyarwanda \REF{ex:vanderwal:29}, I speculate that lower functional heads introducing an argument also carry φ features and can therefore \isi{spell out} additional object markers.


\ea\label{ex:vanderwal:29}%bkm:Ref304797060
Kinyarwanda (JD61, \citealt[183]{BeaudoinLietzEtAl2004})\\    
\gll Umugoré  a-  ra- na- ha- ki-  zi-  ba-  ku- n- someesheesherereza.\\
     \oldstylenums{1}woman  \textsc{sm\oldstylenums{1}}- \textsc{dj}{}- \textsc{also}{}- \textsc{om\oldstylenums{16}}- \textsc{om\oldstylenums{7}}- \textsc{om\oldstylenums{10}}- \textsc{om\oldstylenums{2}}- \textsc{om\oldstylenums{2}sg}- \textsc{om\oldstylenums{1}sg}{}-         read.\textsc{caus}.\textsc{caus}.\textsc{appl}.\textsc{appl}\\
    \glt ‘The woman is also making us read it (book) with them (glasses) to you for me there (in the house).’
\z

The derivation of multiple object markers would be as follows. Following \citet{Julien2002} I take it that the \ili{Bantu} verb head moves in the lower part of the clause, picking up derivational suffixal morphology. The verb also gathers the φ features on the different functional heads that are spelled out as prefixes at the completion of the phase. Further prefixes such as \isi{negation}, the subject marker and TAM morphology are heads that are spelled out in their individual positions and phonologically merged to the stem. The different derivations for object marking prefixes and other prefixes are reflected in the status of the stem plus the object marker(s) as a separate domain for tone rules, known as the “macrostem”.

This analysis predicts that \isi{agreement} with the Theme is always possible in these languages, i.e. that languages with multiple object markers are always symmetrical. This is indeed borne out for Tswana, Kinyarwanda, Kirundi, Ha, Haya, Luganda, Tshiluba, Totela and Chaga, the \textit{only} exception so far being Sambaa. \citet{Riedel2009} shows that Sambaa only allows object marking of the Theme if the Benefactive is also object marked, hence an asymmetrical pattern. This suggests that the additional probe responsible for multiple object marking in Sambaa is located not on lower functional heads, but on a higher functional head; see \citet{VanderWalSubmitted}. 
%%This could form a separate projection (AgrO, as in \citealt{Riedel2009}), it could be an additional probe on v (\citealt{Adams2010}, perhaps multiple probes as in \citealt{Ura1996,Hiraiwa2001}) or it could be a ‘renewable’ probe in cyclic \isi{Agree} (\citealt{BejarRezac2009}). More research is needed to confirm these analyses for languages with multiple object markers. 
For the current paper I focus on languages with only one object marker.

\subsection{Flexibility vs. optionality}\label{sec:vdw:3.4}
A question for this approach to flexibility, which \citet{HaddicanHolmberg2012,HaddicanHolmberg2015} do not address, is what determines whether a low functional head licenses an argument in its specifier or its complement. In an explanatory analysis this should not be completely optional. The hypothesis I want to put forward is that the ‘direction’ of licensing by a flexible head is determined by relative topicality of the two arguments. 

Concretely, the applicative head will Case-license the less topical of the two objects (Theme and Benefactive). The applicative head can do so because it introduces one of the arguments while also being merged with a structure that contains an unlicensed argument, thus ‘seeing’ both arguments. This analysis has obvious parallels with \citegen{AdgerHarbour2007} proposal to account for restrictions in the cooccurrence of speech act participants (\isi{PCC} effects), where the applicative head can also see both arguments. A difference is that in their analysis the applicative head can only license the \isi{Person} values on the Theme that the Recipient does \textit{not} have, whereas in my analysis it can only value a subset of what it \textit{does} have. Where the current account can still be extended along the lines of  \citet{AdgerHarbour2007} is the sensitivity of \isi{Appl} to \isi{Person} as well, not only to account for \isi{PCC} effects but also for animacy effects as observed for Sotho \citep{MorolongHyman1977} and \ili{Zulu} \citep{Zeller2011}. Preliminary results show that sensitivity to \isi{Person} indeed accounts for the attested animacy patterns (\citealt{VanderWal2016}).

More technically, I propose that the applicative head has a [uTopic] probe which is restricted by the value of the Benefactive argument in its specifier: the head can only license arguments that are equal or lower in topicality than the argument it introduces. If the probed Theme is equal or lower in topicality than the Benefactive, then default \isi{Agree}/Case-licensing downwards takes place. If the probed Theme is higher in topicality, the head instead licenses the Benefactive in the specifier. This can also be captured in binary terms, where objects have a topic feature with a + value or an absence of value. 

When the Benefactive is specified as [topic: + ], the applicative head licenses any Theme, whether [topic: + ] or [topic: \_ ], as represented in \REF{ex:vanderwal:30}. 

\ea%bkm:Ref324581325
\label{ex:vanderwal:30}
\begin{forest}
[vP [] [,shape=coordinate [v{[φ]},name=vphi] [HApplP, s sep=10mm [BEN\\{[top: +]},name=ben,base=top,align=center] [,shape=coordinate, s sep=10mm [HAppl,name=happl] [VP [V] [TH,name=th] ] ] ] ] ]
\path [-{Stealth[]}] (vphi.west) edge [bend right=70] (ben.north west);
\node [below=.75cm of happl.north] (utop) {[utop]};
\path [-{Stealth[]}] (utop) edge [bend right,in=220] (th);
\node [below=.75cm of th.north] (topunderscore) {[top: \_ \slash +]};
\end{forest}
\z
    
The Theme’s absence of a value for topicality ([topic: \_ ]) is compatible with the positive value for topicality on the Benefactive and hence the applicative head licenses the Theme. This entails that little v will in this situation always agree with the more topical Benefactive.

When the Theme is specified [topic: + ], the values of head and Theme are compatible as well, and \isi{Appl} will by default license the Theme, leaving the Benefactive again to be Case-licensed (and agreed with) by v. In other words, when both objects are topical, only the higher will be object-marked. This is in fact borne out in \ili{Zulu}: when both DP objects are dislocated, only the higher can be object-marked. In \REF{ex:vanderwal:31} we know that both objects are dislocated because of the disjoint form of the verb and the accompanying prosodic phrases (not indicated here), see further \citet{Zeller2015}.


\ea\label{ex:vanderwal:31}%bkm:Ref287076945
\ili{Zulu} (\citealt{Adams2010} via \citealt[224, 225]{Zeller2012})
\ea[]{\label{ex:vanderwal:31a}
\gll Ngi-ya-m-theng-el-a  u-Sipho  u-bisi.\\
    \oldstylenums{1}\textsc{sg.sm}{}-\textsc{pres.dj-\oldstylenums{1}om}{}-buy-\textsc{appl}{}-\textsc{fv}  \oldstylenums{1}a-Sipho  \oldstylenums{11}-milk\\}
 \ex[]{\label{ex:vanderwal:31b}
\gll Ngi-ya-m-theng-el-a  u-bisi  u-Sipho. \\
    \oldstylenums{1}\textsc{sg.sm-pres.dj-\oldstylenums{1}om}{}-buy-\textsc{appl}{}-\textsc{fv}  \oldstylenums{11}-milk  \oldstylenums{1}a-Sipho \\
    \glt ‘I am buying milk for Sipho.’ }

 \ex[*]{\label{ex:vanderwal:31c}
\gll Ngi-ya-lu-theng-el-a  u-Sipho  u-bisi. \\
    \oldstylenums{1}\textsc{sg.sm-pres.dj-\oldstylenums{11}om}{}-buy-\textsc{appl}{}-\textsc{fv}  \oldstylenums{1}a-Sipho  \oldstylenums{11}-milk\\}

 \ex[*]{\label{ex:vanderwal:31d}
\gll Ngi-ya-lu-theng-el-a  u-bisi  u-Sipho. \\
    \oldstylenums{1}\textsc{sg.sm-pres.dj-\oldstylenums{11}om}{}-buy-\textsc{appl}{}-\textsc{fv}  \oldstylenums{11}-milk  \oldstylenums{1}a-Sipho \\
  \glt   int. ‘I am buying milk for Sipho.’}
\z
\z


When the Benefactive is [topic: \_ ], this is also the restriction on the probing applicative head. Hence, if the Theme is [topic: \_ ], this is perfectly compatible with the Benefactive (and hence the applicative head), and Case-licensing from the applicative head is by default downwards, leaving v to agree with and Case-license the Benefactive.\footnote{It is in fact not possible to ascertain that v agrees with the Benefactive when both are non-topical since the object marker will in such cases not be spelled out anyway (under the view that the object marker \isi{spells out} the features of a defective goal, i.e. φP, as in \citealt{Roberts2010}). The correct V DP DP order comes out whether \isi{Appl} licenses Theme or Benefactive, so at present this is irrelevant to the discussion.} However, if the Theme is [topic: +], this is not compatible with the absence of a topic value, and hence the applicative head will Case-license the Benefactive in its specifier, leaving the topical Theme to be agreed with and Case-licensed by v, as sketched in \REF{ex:vanderwal:32}.

\ea%bkm:Ref309482848
\label{ex:vanderwal:32}
\begin{forest}
 [vP [] [,shape=coordinate [ v{[φ]},name=vphi] [HApplP [BEN\\{[top: \_]},base=top,align=center] [,shape=coordinate [HAppl, name=happl] [VP [V] [TH\\{[top: +]},base=top,align=center,name=th] ] ] ] ] ]
 \path [-{Stealth[]}] (vphi) edge [bend right=84,looseness=1.5] (th);
 \node [below=.75cm of happl.north] (utop) {[utop]};
 \path [-{Stealth[]}] (utop) edge [bend left,out=90,in=120] (ben.south);
\end{forest}
\z


A consequence of this analysis is that it is the more topical of the two arguments that will be left available for \isi{agreement} with v. Indeed, object marking (= \isi{agreement} with v) is crosslinguistically typically with the more topical or given object, in differential object marking as well as pronominalisation (see e.g. \citealt{Adams2010,Zeller2014,Zeller2015} for \ili{Zulu}, \citealt{BaxDiercks2012} for Manyika). Moreover, in a \isi{passive} clause where v does not have either \isi{Case} or φ features, T agrees with the more topical argument. This is expected, since it is known that a functional motivation behind a \isi{passive} is the promotion of an erstwhile object not only to the syntactic function of subject, but also to the discourse function of topic (\citealt[9]{Givón1994}). This is especially true for the \ili{Bantu} languages where the preverbal domain favours or is restricted to topical elements (e.g. \citealt{Morimoto2006,Henderson2006,Zeller2008,Zerbian2006,VanderWal2009,Yoneda2011}).

\largerpage
The sensitivity of low functional heads to information structure is not a new proposal: \citet{Creissels2004,Marten2003,CannMabugu2007}  and \citet{deKindEtAl2012} also show that applicatives are more than simple argument-introducing heads; in various \ili{Bantu} languages they can be used with a non-canonical, information-structural, interpretation. To give just one example, \citet{Creissels2004} first shows the familiar function of introducing a Benefactive argument in Tswana \REF{ex:vanderwal:33a}, and the function of making a peripheral argument (the locative ‘in the pot’ in \ref{ex:vanderwal:33b}) into a proper argument of the predicate.


\ea\label{ex:vanderwal:33}%bkm:Ref247018679
Tswana (S31, \citealt[13]{Creissels2004}, adapted)\\
\ea\label{ex:vanderwal:33a}
\gll Lorato  o  tlaa  ape-el-a  bana  motogo.\\
    \oldstylenums{1}.Lorato  \oldstylenums{1}\textsc{sm}  \textsc{fut}  cook-\textsc{appl}{}-\textsc{fv}  \oldstylenums{2}.children  \oldstylenums{3}.porridge\\
    \glt ‘Lorato will cook the porridge for the children.’
\ex\label{ex:vanderwal:33b}
\gll Lorato  o  tlaa  ape-el-a  motogo  mo  pitse-ng.\\
    \oldstylenums{1}.Lorato  \oldstylenums{1}\textsc{sm}  \textsc{fut}  cook-\textsc{appl}{}-\textsc{fv}  \oldstylenums{3}.porridge  \textsc{prep}  \oldstylenums{9}.pot-\textsc{loc}\\
    \glt ‘Lorato will cook the porridge in the pot.’
\z
\z


Interestingly, Creissels then shows that applicatives in Tswana can also have a non-canonical function as triggering a focus reading of the locative \REF{ex:vanderwal:34}.


 \ea \label{ex:vanderwal:34}%bkm:Ref247019242
Tswana (S31, \citealt[15]{Creissels2004})\\
 \gll      Lorato  o  ape-el-a  mo  jarate-ng.\\
  \oldstylenums{1}.Lorato  \oldstylenums{1}\textsc{sm}  cook-\textsc{appl}{}-\textsc{fv}  \textsc{prep}  \oldstylenums{9}.yard-\textsc{loc}\\
    \glt ‘Lorato does the cooking \textit{in the yard}.’
\z

This can be taken as independent evidence for the sensitivity of the applicative head, and potentially other low functional heads, to discourse-related properties.

\subsection{Interim summary}\label{sec:vdw:3.5}
\largerpage

To summarise, assuming that \is{DOC}double object constructions always involve an additional low functional head such as a causative, or a low or high applicative, the default structure is asymmetrical with the Theme lower than the Recipient/Benefactive/Causee argument. We can account for symmetrical behaviour of objects by appealing to flexibility of such a functional head to Case-license either the Theme in its complement or the argument in its specifier. I suggest that this is determined by the relative topicality of the two arguments. With this analysis of symmetry in place, we can return to the question of how we can understand the implicational relation between causative, applicative and lexical ditransitive predicates and symmetry.

\subsection{Capturing the implicational relationship}\label{sec:vdw:3.6}

The partial symmetry discovered for different predicate types can now be understood as subsets of low functional heads being flexible in licensing their complement or specifier. Languages vary, then, in which heads have this flexibility, i.e. flexible licensing must be parameterised. The implicational relation between different predicates can thus be captured in the following parameter hierarchy \REF{ex:vanderwal:35}.

\ea \label{ex:vanderwal:35}
Parameter hierarchy for the degree of symmetry\\
\begin{forest}
[Can low functional heads license their specifier? [N\\4: asymmetry]  [Y\\Can all low functional heads do so? [Y\\1: \ili{Zulu} etc.] [N\\Can all applicative heads do so? [Y\\2: Sotho{,} Herero] [N\\3: \ili{Luguru}] ] ] ] 
\end{forest}
\z
  

Apart from capturing the implicational relation between the different types of ditransitives, this parameter hierarchy is motivated by conceptual reasons too. First, organising parameters in a dependency relation rather than postulating independent parameters drastically reduces the number of possible combinations of parameter settings, i.e. the number of possible grammars, as shown by \citet{RobertsHolmberg2010}, and \citet{Sheehan2014}. 

  Second, the parameter hierarchy can serve to model a path of acquisition that is shaped by general learning biases (the ‘third factor’ in language design, \citealt{Chomsky2005}). \citet{BiberauerRoberts2015} suggest that two general learning biases combine to form a ‘minimax search algorithm’:

\ea  \label{ex:vanderwal:36}
Feature Economy (FE): postulate as few features as possible to account for the input [generalised from \citealt{RobertsRoussou2003}]
\z


\ea  \label{ex:vanderwal:37}
Input Generalisation (IG): maximise available features \\{}
[generalised from \citealt{Roberts2007}]
\z

If both FE and IG are observed with respect to applicative and causative heads, no features will be postulated on these heads, which for the current analysis of double objects results in default downward licensing and hence an asymmetrical system. When the language gives evidence that the higher object is sometimes licensed by a lower functional head, then an upwards licensing property must be postulated for such heads. This violates FE, but by IG the property is now taken to be present on all heads, leading to a system that is completely symmetrical (type \ref{ex:vanderwal:1}). If the language then gives evidence that \textit{some} heads are asymmetrical, the parameter question is which subset of heads has the property, e.g. applicatives versus causatives.\footnote{It remains to be seen what precise feature specification singles out the set of applicative heads.} We thus derive a ‘none-all-some’ order of implicational parameters and of parameter acquisition. 

If topicality is indeed the motivation for flexible licensing, then the parameter can be rephrased as ‘Which heads are sensitive to topicality?’. In fact, this fits into a more general hierarchy of ditransitive alignment patterns \citep{Sheehan2013}, which captures two types of asymmetry. The first is secundative alignment, where the Recipient object behaves like the monotransitive object, i.e. ‘I gave him the cake’ but not *‘I gave my friend it’ (as in English). The second is indirective alignment, where the Theme behaves like the monotransitive object, i.e. ‘I gave my friend it’ but not *‘I gave him the cake’ (as in \ili{Italian}). See further the typological overviews in \citet{Malchukov2010,Malchukov2013}.

\ea\label{ex:vanderwal:38}%bkm:Ref310612787
Parameter hierarchy for (a)symmetry in ditransitive alignment\\
\begin{forest}
[Do low functional heads license their specifier? [N\\4: secundative] [Y\\Do all low functional\\heads do so? [Y\\indirective] [N\\Are low functional heads\\topic-sensitive? [N\\<..>\footnotemark] [Y\\Are all low funct\\ heads topic-sens? [Y: \ili{Zulu}] [N\\ Are all appl\\ heads topic-sens? [Y\\2: Sotho{,} Herero] [N\\3: \ili{Luguru}] ] ] ] ] ]
\end{forest}  
\z \footnotetext{This is a theoretical possibility that I have not encountered in the data, representing flexible licensing that is sensitive to other factors.}
\section{Potential trouble}\label{sec:vdw:4}

Even within the type 1 languages, which are fully symmetrical, patches of asymmetry emerge, particularly in combinations of derivations (\isi{passive}, applicative, causative). I discuss two here.

\subsection{Combinations of extensions}\label{sec:vdw:4.1}

In \ili{Zulu}, objects of doubly derived verbs with both a causative and an applicative still behave symmetrically. That is, the Causee \REF{ex:vanderwal:39b}, the Benefactive \REF{ex:vanderwal:39a} or the Theme \REF{ex:vanderwal:39c} can be object marked.


\ea\label{ex:vanderwal:39}%bkm:Ref310612685
\ili{Zulu} \citep{Zeller2011}\\
 {applicative + causative}
 \ea\label{ex:vanderwal:39a}
\gll Usipho  u-\textbf{m}{}-fund-is-el-a  abafundi  {Zulu}  (uLanga). \\
    1aSipho  \oldstylenums{1}\textsc{sm}{}-\oldstylenums{1}\textsc{om}{}-learn-\textsc{caus}{}-\textsc{appl}{}-\textsc{fv}  \oldstylenums{2}.student  \oldstylenums{7}.{Zulu}  \oldstylenums{1}a.Langa\\
    \glt ‘Sipho is teaching the students {Zulu} for him (Langa).’

 \ex\label{ex:vanderwal:39b}
\gll Usipho  u-\textbf{ba}{}-fund-is-el-a  uLanga  {Zulu}   (abafundi). \\
    1aSipho  \oldstylenums{1}\textsc{sm}{}-\oldstylenums{2}\textsc{om}{}-learn-\textsc{caus}{}-\textsc{appl}{}-\textsc{fv}   \oldstylenums{1}a.Langa  \oldstylenums{7}.{Zulu}  \oldstylenums{2}.student \\
    \glt ‘Sipho is teaching them {Zulu} for Langa (the students).’

 \ex\label{ex:vanderwal:39c}
\gll Usipho  u-\textbf{si}{}-fund-is-el-a  uLanga  abafundi  ({Zulu}).\\
    1aSipho  \oldstylenums{1}\textsc{sm}{}-\oldstylenums{7}\textsc{om}{}-learn-\textsc{caus}{}-\textsc{appl}{}-\textsc{fv}   \oldstylenums{1}a.Langa  \oldstylenums{2}.student  \oldstylenums{7}.{Zulu} \\

    \glt ‘Sipho is teaching it to the students for Langa ({Zulu}).’
\z
\z


This forms an interesting contrast with \ili{Kîîtharaka}. \ili{Kîîtharaka} is also a type 1 symmetrical language, like \ili{Zulu}: either object can be object-marked in applicatives \REF{ex:vanderwal:40} as well as causatives \REF{ex:vanderwal:41}.

\noindent \ili{Kîîtharaka} \citep[83, 84]{Muriungi2008}\\
\ea\label{ex:vanderwal:40}%bkm:Ref310612725
 {applicative}\\
 \ea\label{ex:vanderwal:40a}
\gll Maria  a-kû-\textbf{mî}{}-tûm-îr-a  John.\\
    \oldstylenums{1}.Maria  \oldstylenums{1}\textsc{sm-T-\oldstylenums{9}om}{}-send-\textsc{appl}{}-\textsc{fv}  \oldstylenums{1}.John \\
    \glt ‘Maria has sent it to John.’ (a letter)

 \ex\label{ex:vanderwal:40b}
\gll Maria  a-kû-\textbf{mû}{}-tûm-îr-a  barûa.\\
    \oldstylenums{1}.Maria  \oldstylenums{1}\textsc{sm-T-\oldstylenums{1}om}{}-send-\textsc{appl}{}-\textsc{fv}  \oldstylenums{9}.letter \\
    \glt ‘Maria has sent him/her a letter.’

    \z
\z


\ea\label{ex:vanderwal:41}%bkm:Ref310612740
 {causative}\\

 \ea\label{ex:vanderwal:41a}
\gll Mu-borisi  a-kû-\textbf{mî}{}-nyu-ithi-a  mû-ûragani.\\
    \oldstylenums{1}-police  \oldstylenums{1}\textsc{sm-t-\oldstylenums{9}om}{}-drink-\textsc{crc-fv}  \oldstylenums{1}-murderer\\
    \glt ‘The policeman has coerced the murderer to drink it.’ (the poison)

 \ex\label{ex:vanderwal:41b}
\gll Mu-borisi  a-kû-\textbf{mû}{}-nyu-ithi-a  cûmû.\\
    \oldstylenums{1}.-police  \oldstylenums{1}\textsc{sm-t-\oldstylenums{1}om}{}-drink-\textsc{crc-fv}  \oldstylenums{9}-poison\\
    \glt ‘The policeman has coerced him/her to take the poison.’
\z
\z


However, when a predicate has both a causative and an applicative derivation, the objects in \ili{Kîîtharaka} are no longer symmetrical: only the applied object can be object-marked \REF{ex:vanderwal:42a}, and object-marking the Causee or the Theme results in ungrammaticality (\ref{ex:vanderwal:42}b, c).

\ea\label{ex:vanderwal:42}%bkm:Ref310612756
{applicative + causative} \citep[83]{Muriungi2008}\\

 \ea[]{\label{ex:vanderwal:42a}
\gll I-ba-ra-\textbf{ka}{}-thamb-ith-î-îr-i-e  Maria  nyomba.\\
    \textsc{foc-\oldstylenums{2}sm-psty-\oldstylenums{12}om}{}-wash-\textsc{crc-appl-pfv-ic-fv}  \oldstylenums{1}.Maria  \oldstylenums{9}.house\\
    \glt ‘They coerced Maria to wash the house for it (e.g the cat).’}

 \ex[*]{\label{ex:vanderwal:42b}
\gll N-a-ra-\textbf{ba}{}-thamb-ith-î-îr-i-e  ka-baka  nyomba.\\
      \textsc{foc-\oldstylenums{1}sm-psty-\oldstylenums{2}om}{}-wash-\textsc{crc-appl-pfv-ic-fv}  \oldstylenums{12}-cat  \oldstylenums{9}.house\\
    \glt ‘He/she coerced them to wash the house for the cat.’}

 \ex[*]{\label{ex:vanderwal:42c}
\gll I-ba-ra-\textbf{mî}{}-thamb-ith-î-îr-i-e  Maria  ka-baka.\\
      \textsc{f-\oldstylenums{2}sm-psty-\oldstylenums{9}om}{}-wash-\textsc{crc-appl-pfv-ic-fv}  \oldstylenums{1}.Maria  \oldstylenums{12}-cat \\
    \glt ‘They coerced Maria to wash it for the cat.’}
\z
\z


My hypothesis is that this sudden asymmetry is due to \ili{Kîîtharaka} having a combination of the short and long causative \citep{Bastin1986}, glossed by Muriungi as ‘\textsc{crc’} (coerce causative) and ‘\textsc{ic’} (inner causative), which occur on either side of the applicative. It may thus be that the coerce causative is flexible, but the structurally higher inner causative is not. If this is true, the hierarchy in \REF{ex:vanderwal:38} should involve an extra layer asking about different types of causatives.\footnote{See also \citegen{NgonyaniGithinji2006} multiple applicatives in Kikuyu, which appear to behave asymmetrically despite the language’s otherwise fully symmetrical properties. It remains to be seen how animacy plays a role in these counterexamples, and also at which height the higher applicative is merged.}

\subsection{Symmetry in passives}\label{sec:vdw:4.2}

In \ili{Zulu}, \ili{Lubukusu}, Kinyarwanda and Luganda both object marking and passivisation are symmetrical: either object can be object-marked and either object can become the subject of a \isi{passive}. However, the languages differ in the combination of these operations.

In Kinyarwanda and Luganda, either object can be object-marked in the active as well as the \isi{passive}. That is, the Theme can be object-marked in a Benefactive \isi{passive} (\ref{ex:vanderwal:43b}, \ref{ex:vanderwal:44aa}), and the Benefactive can be object-marked in a Theme \isi{passive} (\ref{ex:vanderwal:43c}, \ref{ex:vanderwal:44bb}).


\ea\label{ex:vanderwal:43}%bkm:Ref310612819
Kinyarwanda (\citealt[88]{Ngoboka2005}, glosses adapted)\\
{symmetrical \isi{passive} OM}

 \ea\label{ex:vanderwal:43a}
\gll Umusore  y-a-hiing-i-ye  umugore  umurima.\\
    \oldstylenums{1}.young.man  \oldstylenums{1}\textsc{sm-pst}{}-plough-\textsc{appl-asp}  \oldstylenums{1}.woman  \oldstylenums{3}.field\\
    \glt ‘The young man ploughed the field for the woman.’

 \ex\label{ex:vanderwal:43b}
\gll Umugore  y-a-\textbf{wu}{}-hiing-i-w-e  n’  umusore.\\
    \oldstylenums{1}.woman  \oldstylenums{1}\textsc{sm-pst-\oldstylenums{3}om}{}-plough-\textsc{appl-pass-asp}  by  \oldstylenums{1}.young.man\\
    \glt lit. ‘The woman was it ploughed for by the young man.’

 \ex\label{ex:vanderwal:43c}
\gll Umurima  w-a-\textbf{mu}{}-hiing-i-w-e  n’  umusore.\\
    \oldstylenums{3}.field  \oldstylenums{3}\textsc{sm-pst-\oldstylenums{1}om-}plough-\textsc{appl-pass-asp}  by  \oldstylenums{1}.young.man\\
    \glt ‘The field was ploughed (for) her by the young man.’
\z
\z


\ea\label{ex:vanderwal:44}
Luganda \citep{Ranero2015}\\

 
 \ea\label{ex:vanderwal:44b}\label{ex:vanderwal:44aa}
\gll O-mw-ana  y-a-zi-w-ew-a  luli  e-ssente.\\
    \textsc{aug}{}-\oldstylenums{1}-child  \oldstylenums{1}\textsc{sm}{}-\textsc{pst}{}-\oldstylenums{9}a\textsc{om}{}-give-\textsc{pass} the.other.day  \textsc{aug}{}-\oldstylenums{9}a.money\\
    \glt ‘The child was given it the other day, the money.’
 
 \ex \label{ex:vanderwal:44a}\label{ex:vanderwal:44bb}
\gll E-ssente   za-a-mu-w-ew-a   luli   o-mw-ana.\\
    \textsc{aug}{}-\oldstylenums{9}a.money  \oldstylenums{9}a\textsc{sm}{}-\textsc{pst}{}-\oldstylenums{1}\textsc{om}{}-give-\textsc{pass} the.other.day  \textsc{aug-}\oldstylenums{1}-child\\
    \glt ‘The money was given to him/her the other day, the child.’
\z
\z


In \ili{Zulu} and \ili{Lubukusu}, on the other hand, the Benefactive/Recipient cannot be object-marked in a (otherwise perfectly acceptable) Theme \isi{passive}, as in \REF{ex:vanderwal:45b} and \REF{ex:vanderwal:46b}, whereas the opposite is still possible, as shown in \REF{ex:vanderwal:45a} and \REF{ex:vanderwal:46a}.

\newpage 

\ea\label{ex:vanderwal:45}%bkm:Ref310612850
\ili{Lubukusu} (Justine Sikuku p.c. July 2015)\\
\ea[]{\label{ex:vanderwal:45a}
{Recipient-\isi{passive} with Theme-OM}\\
\gll   Baa-sooreri  ba-a-\textbf{chi}{}-eeb-w-a  (chi-khaafu). \\
\oldstylenums{2}.boys  \oldstylenums{2}\textsc{sm-past-\oldstylenums{10}om}{}-give-\textsc{pass-fv}  \oldstylenums{10}-cows\\
\glt ‘The boys were given them (cows).’}

\ex[??]{\label{ex:vanderwal:45b}
{Theme-\isi{passive} with Recipient-OM}\\
\gll   Chi-kaafu  cha-a-\textbf{ba}{}-eeb-w-a  (baa-sooreri). \\
\oldstylenums{10}-cows  \oldstylenums{10}\textsc{sm-pst-\oldstylenums{2}om}{}-give-\textsc{pass-fv}  \oldstylenums{2}-boys\\
\glt ‘Cows were given to them (the boys).’}
\z
\z


\ea\label{ex:vanderwal:46}%bkm:Ref310612860
\ili{Zulu} \citep[26]{Adams2010}\\
\ea[]{\label{ex:vanderwal:46a} 
Recipient-\isi{passive} with Theme-OM\\
\gll Aba-ntwana  ba-ya-\textbf{yi}{}-fund-el-w-a  (in-cwadi).\\
\oldstylenums{2}-child  \oldstylenums{2}\textsc{sm}{}-\textsc{pres.dj-\oldstylenums{9}om}{}-read-\textsc{appl}{}-\textsc{pass}{}-\textsc{fv}  \oldstylenums{9}-book\\
\glt ‘The children are being read it (the book).’}

\ex[*]{\label{ex:vanderwal:46b}
Theme-\isi{passive} with Recipient-OM\\
   \gll  In-cwadi  i-ya-\textbf{ba}{}-fund-el-w-a  (aba-ntwana).\\
    \oldstylenums{9}-book  \oldstylenums{9}\textsc{sm-pres.dj-\oldstylenums{2}om}{}-read-\textsc{appl}{}-\textsc{pass}{}-\textsc{fv}  \oldstylenums{2}-children\\
    \glt int. ‘The book is being read to them (the children).’}
    \z
\z



The generalisation is thus that the Theme can be object-marked in a Benefactive \isi{passive}, but the Benefactive cannot be object-marked in a Theme \isi{passive}. The same asymmetry holds for extraction: the Theme can be extracted from a Benefactive \isi{passive}, but the Benefactive cannot be extracted from a Theme \isi{passive}. Interestingly, \ili{Norwegian} and North-Western English, which are otherwise symmetrical too, show the same restriction as \ili{Zulu} and \ili{Lubukusu}. Crucially, there are no languages in which the asymmetry is the other way around (i.e. banning Theme extraction in a Benefactive \isi{passive}). 

A promising analysis of this asymmetry in passives takes v to be a phase in the active, but \textit{not} to be a phase in the \isi{passive} (\citealt{Chomsky2008,Legate2012}). Instead, in the \isi{passive}, \isi{Appl} (or Caus) is a phase and bears φ features, since \isi{Appl} is now the highest head with full argument structure (see \citegen{Chomsky2008} definition of the lower phase). If object marking is indeed the spell-out of a (downward) \isi{Agree} relation, the exceptional presence of φ features on \isi{Appl} in \ili{Zulu} and \ili{Lubukusu} passives implies that only the Theme can be object-marked, since the Benefactive is higher than \isi{Appl} and upwards \isi{agreement} cannot be spelled out as an object marker (under Roberts’ \citeyear{Roberts2010} approach to clitics). Either object is thus still available for passivisation, but only the Theme can be object-marked in the \isi{passive}. For Kinyarwanda, I proposed at the end of \sectref{sec:vdw:3.3} that \isi{Appl} is endowed with φ features in the active too (accounting for the occurrence of multiple object markers) -- the presence of φ features is thus independent of \isi{phasehood} in this language, which could explain the consistent symmetry throughout the \isi{passive} in this language. The same goes for Luganda, which also allows multiple object markers.

This analysis for the combination of \isi{passive} and extraction is further pursued in joint work with Anders Holmberg and Michelle Sheehan, suggesting that movement of the Theme to the outer specifier of the \isi{Appl} phase head traps the Benefactive object for \isi{A-bar movement} to specCP (under PIC2).

\section{Summary and conclusion}\label{sec:vdw:5}

Upon closer examination, \ili{Bantu} languages that display symmetrical \is{DOC}double object constructions all show some asymmetry. A novel type of partial asymmetry presented in this paper is the variation between different types of ditransitive predicates, which appears to have an implicational pattern: if a language is symmetrical for causatives, it is also symmetrical for applicatives, and if it is symmetrical for applicatives, it is also symmetrical for lexical ditransitive predicates. Assuming that object marking \isi{spells out} \isi{agreement} on little v, and assuming that second objects are introduced by separate lower functional heads (Caus, HAppl and LAppl), symmetrical behaviour of multiple objects can be understood as the ability of such heads to Case-license either the argument they introduce in their specifier or the lower argument in their complement. Which argument it licenses depends on their relative topicality, with the low functional head licensing the least topical of the two. The remaining argument will be Case-licensed and agreed with by little v (active) or T (\isi{passive}), which thus explains object marking and passivisation of the most topical argument. The implicational relationship between the types of predicates can be captured in a parameter hierarchy, motivated by third-factor principles.

Further research should clearly take into account more \ili{Bantu} languages to test whether the appearing implicational pattern indeed holds true (especially since type 3 is now only confirmed for one language, \ili{Luguru}). A particularly interesting language to look at here is Kinande, which shows a linker between two objects. \citet{BakerCollins2006} propose an account in terms of Case-licensing, which however \citet{SchneiderZioga2014} shows to not account for constructions in which the linker appears between an argument and an adjunct.

The current paper only concerns \is{DOC}double object constructions with two DP arguments that have thematic roles as Causee, Benefactive, Recipient and Theme. Taking into account predicates with a DP and a PP argument (cf. \citealt{Bruening2010,Jeong2007,BakerKramer2016}) and other grammatical roles such as Locatives and Instrumentals is likely to change the picture (see e.g.
% \citealt{Baker1988,Marantz1993,AlsinaMchombo1993,Ngonyani1996,Ngonyani1998,Simango1995,Nakamura1997,Ngoboka2005,Ngoboka2016,ZellerNgoboka2006,Jerro2015}%
\citealt{Baker1988,GerdtsWhaley1991,GerdtsWhaley1992,Marantz1993,AlsinaMchombo1993,Ngonyani1996,Ngonyani1998,Simango1995,Nakamura1997,Ngoboka2005,Ngoboka2016,ZellerNgoboka2006,Jerro2015}%
), 
as well as \isi{possessor} raising constructions that take a similar shape \citep{Simango2007,MorolongHyman1977}. However, it should be established beforehand whether the base-generated structure of these (locative, instrumental) constructions are the same as for the \is{DOC}double object construction, considering that the so-called dative alternation is argued to actually be based on different underlying structures (\citealt{Pesetsky1995,Harley2002,Bruening2010}; see also footnote 3).

A final point is that the current paper considers primarily object marking, with an extension to \isi{A-movement} in the \isi{passive}, but not much is known about the symmetrical or asymmetrical behaviour of different (causative, applicate) predicates for A-bar operations such as relativisation \citep{Nakamura1997}, which the proposed analysis does not make any independent predictions for.\is{causative|)}\is{Appl|)}
 

\section*{Abbreviations and symbols}
Numbers refer to noun classes, or to persons when followed by \textsc{sg} or \textsc{pl}.

\begin{multicols}{2}
\begin{tabbing}
\textsc{recpast12} \= double object construction\kill
\textsc{appl} \>  applicative\\
\textsc{asp} \>  aspect\\
BEN  \>  Benefactive\\
\textsc{cj} \>  conjoint verb form\\
\textsc{caus} \>  causative\\
\textsc{crc} \>  coerce\\
\textsc{dem} \>  demonstrative\\
\textsc{dj} \>  disjoint verb form\\
\textsc{doc} \>  double object construction\\
\textsc{fv} \> final vowel\\
\textsc{ic} \>  inner causative\\
int \> intended meaning\\
\textsc{om} \>  object marker\\
\textsc{opt} \>  optative\\
\textsc{pass} \>  passive\\
\textsc{poss} \>  possessive\\
\textsc{past} \>  past tense\\
\textsc{prog} \>  progressive\\
R  \>  Recipient\\
\textsc{recpast} \>  recent past\\
\textsc{sm} \>  subject marker\\
T   \> tense\\
TH   \> theme\\
\end{tabbing} 
\end{multicols}


\section*{Acknowledgements}

This research is funded by the European Research Council Advanced Grant No. 269752 \textit{Rethinking Comparative Syntax}. I want to express my thanks to Michael Marlo, Michael Diercks, Rodrigo Ranero, Nancy Kula, Jochen Zeller, Jean Paul Ngoboka, Leston Buell, David Iorio, Jekura U. Kavari, Thabo Ditsele, Hannah Gibson, Lutz Marten, Justine Sikuku, Andrej Malchukov, Carolyn Harford, Claire Halpert, Nikki Adams, Patricia Schneider-Zioga, Peter Githinji, Chege Githiora, Paul Murrell, Joyce Mbepera, Judith Nakayiza, Saudah Namyalo and the ReCoS team (Ian Roberts, Michelle Sheehan, Timothy Bazalgette, Alison Biggs, Georg Höhn, Theresa Biberauer, Anders Holmberg, Sam Wolfe and András Bárány), for sharing and discussing thoughts and data with me. Thanks also to the audiences at CALL 2015 and LAGB 2015, and to two anonymous reviewers. Any errors and misrepresentations are mine only.


{\sloppy
\printbibliography[heading=subbibliography,notkeyword=this]
}
\end{document}