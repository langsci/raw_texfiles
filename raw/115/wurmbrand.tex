\documentclass[output=paper]{LSP/langsci}
\author{Susi Wurmbrand\affiliation{University of Connecticut}
}
\title{Icelandic  as a partial null subject language: {E}vidence from fake indexicals}

\abstract{
The distribution and licensing of null subjects has been a much debated topic in generative grammar. In many recent works, Anders Holmberg has proposed an enlightening typology that distinguishes between three types of null subject languages (see \citealt{Holmberg2005,Holmberg2010finnish,Holmberg2010Null,HolmbergSheehan2010}): consistent null subject languages such as Spanish, discourse \textit{pro}{}-drop languages such as Chinese, and partial null subject languages. Among the latter are Finnish, Brazilian Portuguese, Marathi, and Icelandic. In this short note, I provide some new data from binding, in particular fake indexicals in Icelandic, that support Holmberg’s view that Icelandic is a partial null subject language.

One of the core defining characteristics of partial null subject languages is that 3\textsuperscript{rd} person subjects can be unexpressed when they receive a non-referential generic interpretation or when they are a bound variable. Non-null subject languages such as German, in contrast, do not allow null subjects of any kind.
}

\ChapterDOI{10.5281/zenodo.1116785}

\maketitle

\begin{document} 


\ea\label{ex:wurmbrand:}
\ea
\langinfo{Icelandic}{}{\citealt[106, (27a)]{Holmberg2010Null}}\\
\gll Nú  má Ø   fara  að  dansa.\\
now  may  Ø  go  to  dance\\
\glt  ‘One may begin to dance now.’
\ex
\ili{German}\\
\gll Jetzt  kann \checkmark man/*Ø  tanzen  gehen.\\
now  can  \checkmark one/*Ø  dance  go\\
\glt ‘Now, one can go dancing.’
\z
\z

The account offered by Holmberg is that in partial \isi{null subject} languages, null third person pronouns are weak deficient pronouns which contain $\varphi $-features (hence displaying \isi{agreement} with the verb) but no referential D-feature. The only way such $\varphi $Ps can be interpreted is via binding by a higher DP, or as default generic pronouns.


Partial \isi{null subject} languages differ regarding whether 1\textsuperscript{st} and 2\textsuperscript{nd} person pronouns can be null when they are used as indexicals: \ili{Finnish} and \ili{Hebrew} allow null indexical subjects, Marathi only allows a 2\textsuperscript{nd} person indexical, and \ili{Brazilian Portuguese} and \ili{Icelandic} allow neither. According to Holmberg, indexical subjects (i.e., referential 1\textsuperscript{st} and 2\textsuperscript{nd} person pronouns) in partial \isi{null subject} languages are always full definite pronouns including a referential D head, and languages differ regarding whether these subjects can be non-pronounced at PF. In consistent \isi{null subject} languages, on the other hand, null indexicals are weak deficient pronouns lacking a DP, and the referential interpretation is contributed via a D-feature in I/T.


  A prediction this account makes is that in partial \isi{null subject} languages, even 1\textsuperscript{st} and 2\textsuperscript{nd} person pronouns should be allowed to be null (bare $\varphi $Ps) when they are not interpreted referentially—i.e., not as indexicals but as bound pronouns. As shown in \REF{ex:wurmbrand:2}, 1\textsuperscript{st} and 2\textsuperscript{nd} person pronouns in English, \ili{German}, and \ili{Icelandic} can be interpreted as bound variables.\footnote{ The tenses are varied in some of the examples to avoid syncretism. This has no influence on fake indexicals.} As indicated by the paraphrase, in these contexts, the person features of indexicals are not interpreted (e.g., the 1\textsuperscript{st} person pronoun \textit{my} is not interpreted as the speaker in the set of alternatives, but as a variable), hence the term fake indexicals.

\ea \label{ex:wurmbrand:2}
All: I/You did my/your best and no one else did their best.   
\ea
 English\\
Only I did my/*her best.
\ex
 \ili{German}\\
Only you did your/*his best.  
\ex
  \ili{German}\\
\gll Nur  ich  habe  mein/*ihr  Bestes  gegeben.\\
only  I  have.\textsc{1.sg}  my/*her  best  given\\
\ex
  \ili{German}\\
\gll Nur  du  hast  dein/*sein  Bestes  gegeben.\\
only  you  have.\textsc{2.sg}  your/*his  best  given\\
\newpage
\ex
\ili{Icelandic}\footnote{ All of the following \ili{Icelandic} examples are provided by Gísli Rúnar Harðarson.}\\
\gll  Aðeins ég  geri  mitt/*hennar  besta.\\
only  I  do.1.\textsc{sg}  my/*her  best\\
\ex
  \ili{Icelandic}\\
\gll Aðeins  þú  gerðir  þitt/*hans  besta.\\
only  you  did.2.\textsc{sg}  your/*his  best\\
\z
\z

Turning to fake indexicals in subject position, an interesting difference arises between \ili{Icelandic} and \ili{German}. Let us start with the \ili{Icelandic} examples in \REF{ex:wurmbrand:3}. In these cases, the 1\textsuperscript{st} and 2\textsuperscript{nd} person possessive pronouns are interpreted as bound variables (one cannot do someone else’s best). The embedded verbs obligatorily agree with the matrix subjects, and this \isi{agreement}, I  propose, is controlled by a \isi{null subject} in the embedded clause as indicated (an overt subject is not possible).

\ea  \label{ex:wurmbrand:3}
\ili{Icelandic}\\
\ea
\gll  Ég  er  sá  eini  sem Ø.1.\textsc{sg}  geri  mitt  besta.\\
I  am.1.\textsc{sg}  \textsc{dem}  one  that  Ø.1.\textsc{sg}  do.1.\textsc{sg}  my  best\\
\glt  ‘I am the only one who is doing my best.’
\ex
\gll  Þú  ert  sá  eini  sem Ø.2.\textsc{sg}  gerðir  þitt  besta.\\
you  are.2.\textsc{sg}  \textsc{dem}  one  that  Ø.2.\textsc{sg}  did.2.\textsc{sg}  your  best\\
\glt  ‘You (\textsc{sg}) are the only one who did your best.’
\z
\z

One may object that the null elements in \REF{ex:wurmbrand:3} are simply null \isi{relative} operators and not (true) \isi{null subjects}. While this is in part correct, the existence of a true \isi{null subject} in \REF{ex:wurmbrand:3} can nevertheless be motivated by two properties. First, as shown in (\ref{ex:wurmbrand:4}a), \ili{German} does not allow fake indexicals in contexts where the embedded verb agrees with the matrix subject. \ili{German} does, however, exhibit a special form of \isi{relative} pronoun ‘doubling’ where the \textit{d}{}- pronoun is paired with a regular personal pronoun (see \citealt{ItoMester2000}). For some speakers this is only possible in non-restrictive \isi{relative} clauses, but for others it is also possible in cases such as (\ref{ex:wurmbrand:4}b). When such a pronoun is added, the embedded verb must agree with the additional subject, and, crucially, a bound variable interpretation then becomes possible for the possessive pronoun.

\ea  \label{ex:wurmbrand:4}
\ea
\langinfo{German}{}{\citealt[206; (36a)]{Kratzer2009}}\\
\gll *Ich  bin  die einzige,  die  meinen Sohn  versorge.\\
I  am  the.\textsc{f.sg} only.one  who.\textsc{f.sg} my son  take.care.of.\textsc{1.sg}\\
\glt ‘I am the only one who is taking care of my son.’
\ex
\ili{German}\\
\gll \%Ich  bin  die einzige,  die  ich  meinen Sohn  versorge.\\
I  am  the.\textsc{f.sg} only.one  who.\textsc{f.sg}  I my son  take.care.of.\textsc{1.sg}\\
\glt  ‘I am the only one who is taking care of my son.’
\z
\z

Under Holmberg’s typology of \isi{null subjects}, the differences between \REF{ex:wurmbrand:3} and \REF{ex:wurmbrand:4} follow if it is assumed that the possessive pronoun requires a featurally identical antecedent in subject position, in order to be interpreted as a fake indexical (see \citealt{Wurmbrand2015} for a detailed account of fake indexicals along these lines). Since \ili{Icelandic} is a partial \isi{null subject} language, subjects can be unexpressed, but only if they are bound by a higher DP. This is the case in \REF{ex:wurmbrand:3}, illustrated in (\ref{ex:wurmbrand:5}a): the matrix (true) indexical pronoun binds the embedded \isi{null subject}, which in turn binds the possessive pronoun—thus both the embedded subject and the possessive pronoun are bound fake indexicals. In the \ili{German} varieties that allow \isi{relative} pronoun doubling in restrictive \isi{relative} clauses, the same configuration is possible, however, since \ili{German} is a non-\isi{null subject} language, the only option is to overtly realize the embedded subject.\footnote{This account has interesting consequences for the structure of \isi{relative} clauses and DPs in general. Since the \isi{relative} operator and the additional subject pronoun correspond to one argument, a DP structure is necessary that allows splitting, for instance, the D-part (the \isi{relative} operator/pronoun) and the $\varphi $-part (the additional pronoun).}

\ea\label{ex:wurmbrand:5}
\ea
  DP.1.\textsc{sg}  [\textsubscript{CP}  OP.3.\textsc{sg}  [\textsubscript{TP}  Ø.$\varphi $P.1.\textsc{sg  T.1.sg  ]]} \ili{Icelandic} 
\ex 
DP.1.\textsc{sg}  [\textsubscript{CP}  OP.3.\textsc{sg}  [\textsubscript{TP}  $\varphi $P1.\textsc{sg  T.1.sg  ]]} \ili{German}
\z
\z

The second piece of evidence for a \isi{null subject} in \REF{ex:wurmbrand:3} comes from constructions in which the embedded fake indexical subject cannot be bound. Note first that the examples in \REF{ex:wurmbrand:3} also have a counterpart in which the null operator corresponds to the head of the \isi{relative} clause, the 3\textsuperscript{rd} person DP \textit{the only one}. In these cases, the embedded verb shows 3\textsuperscript{rd} person \isi{agreement} and only the reflexive possessives are possible, as shown in \REF{ex:wurmbrand:6}.

\ea   \label{ex:wurmbrand:6}
\ili{Icelandic}\\
\ea 
\gll 
Ég  er  sá  eini  sem  gerir  sitt  besta.\\
I  am.1.\textsc{sg}  \textsc{dem}  one  that  do.3.\textsc{sg}  \textsc{refl}  best\\
\glt  ‘I am the only one who is doing her best.’
\ex
\gll  Þú  ert  sá  eini  sem  gerði  sitt  besta.\\
you  are.2.\textsc{sg}  dem  one  that  did.3.\textsc{sg}  refl  best\\
\glt ‘You (sg) are the only one who did her best.’
\z
\z

An important difference regarding binding arises in the inverted (specificational) sentences in \REF{ex:wurmbrand:7}. As shown in (\ref{ex:wurmbrand:7}a,b), the analogues of \REF{ex:wurmbrand:3} are impossible—fake indexical possessives, and as I suggest, fake indexical \isi{null subjects} are not licensed in these configurations. Crucially, as shown in (\ref{ex:wurmbrand:7}c,d), bound variable interpretations of the possessive are still possible, however, only when both the verb and the possessive show 3\textsuperscript{rd} person \isi{agreement}. If all that is involved in \REF{ex:wurmbrand:3} is a \isi{relative} operator, it would not be obvious why in cases such as \REF{ex:wurmbrand:3}/\REF{ex:wurmbrand:6} both 3\textsuperscript{rd} person bound pronouns and fake indexicals are possible, whereas in cases such as \REF{ex:wurmbrand:7} only the 3\textsuperscript{rd} person variant is available. An account based on the existence of \isi{null subjects}, which are only licensed in \ili{Icelandic} when bound by a higher DP, covers this difference very well. While the matrix DPs in \REF{ex:wurmbrand:3}/\REF{ex:wurmbrand:5} can bind and license an embedded \isi{null subject}, this is not possible in \REF{ex:wurmbrand:7} due to the lack of c-command in the inverted order.

\ea  \label{ex:wurmbrand:7}
\ili{Icelandic}\\
\ea
\gll  ?*Sá eini  sem  geri  mitt  besta  er  ég.\\
\textsc{dem} one  that  do.1.\textsc{sg}  my  best  am.1.\textsc{sg} I.\textsc{nom}\\
\glt ‘The only one who is doing my best is me.’
\ex
\gll  *Sá eini  sem  gerðir  þitt  besta  varst  Þú.\\
\textsc{dem} one  that  did.2.\textsc{sg}  your  best  was.2.\textsc{sg} you.\textsc{nom}\\
\glt  ‘The only one who did your best is you.’
\ex
\gll  Sá eini  sem  gerir  sitt  besta  er  ég.\\
\textsc{dem} one  that  do.3.\textsc{sg}  \textsc{refl}  best  am.1.\textsc{sg} I.\textsc{nom}\\
\glt  ‘The only one who is doing his/her best am I.’
\ex
\gll  Sá eini  sem  gerði  sitt  besta  varst  Þú.\\
\textsc{dem} one  that  did.3.\textsc{sg}  \textsc{refl}  best  was.2.\textsc{sg} you.\textsc{nom}\\
\glt  ‘The only one who did his/her best is you.’
\z
\z

Finally, the assumption that the additional subject in \ili{German} cases such as (\ref{ex:wurmbrand:4}b), like the \isi{null subject} in \ili{Icelandic}, is licensed by a higher c-commanding antecedent, predicts that this option should also disappear in inverted specificational sentences. The examples in \REF{ex:wurmbrand:8} show that this is correct—(\ref{ex:wurmbrand:8}a) is impossible for all speakers of \ili{German}, and only a 3\textsuperscript{rd} person possessive as in (\ref{ex:wurmbrand:8}b) is possible to express a bound variable interpretation.
\newpage
\ea  \label{ex:wurmbrand:8}
\ili{German} \\
\ea
\gll  *Die einzige  die  ich  mein Bestes  gegeben  habe  bin  ich.\\
the only.one  who  I  my best  given  have.1.\textsc{sg}  am.1.\textsc{sg}  I\\
\glt ‘The only one who did her (lit. my) best is me.’
\ex
\gll  Die einzige  die    ihr Bestes  gegeben  hat  bin  ich.\\
the only.one  who    her best  given  have.1.\textsc{sg}  am.1.\textsc{sg}  I\\
\glt  ‘The only one who did her best is me.’
\z
\z

While the behavior of fake indexicals in \isi{relative} clauses provides nice evidence for Holmberg’s \isi{null subject} typology, the conclusions have to also be taken with a grain of salt. As shown in \REF{ex:wurmbrand:9}, null fake indexicals are not possible in complement clauses. Even under the bound variable interpretation, the pronoun must be realized overtly.\footnote{A reviewer mentions that control contexts, under certain assumptions, may constitute another case of an obligatorily null bound variable subject. Since infinitival subjects in \ili{Icelandic} have \isi{Case} \citep{Sigurðsson1991} and $\varphi $-features (in particular in partial control contexts), the reviewer suggests that one could perhaps treat those subjects as \textit{pro} rather than PRO.}

\ea \label{ex:wurmbrand:9}
\ili{Icelandic}  \\
\ea
\gll  Aðeins  ég  held  að *(ég)  tali  íslensku.\\
only  í  think.1.\textsc{sg}  that  *(I)  talk.1.\textsc{sg.subj}  \ili{Icelandic} \\
\glt ‘Only I think that I can speak \ili{Icelandic}.’
\ex
\gll  Aðeins  þú  hélst  að  *(þú)  talaðir  íslensku.\\
only  you  thought.2.\textsc{sg}  that  *(you)  talked.2.\textsc{sg.subj}  \ili{Icelandic}\\
\glt ‘Only you thought that you could speak \ili{Icelandic}.’
\ex
\gll  Aðeins  hann  hélt  að  *(hann)  talaði  íslensku.\\
only  he  thought.3.\textsc{sg}  that  *(he)  talked.3.\textsc{sg.subj}  \ili{Icelandic}\\
\z
\z

Holmberg’s proposal which treats \ili{Icelandic} as a partial \isi{null subject} language makes surprising, but correct, predictions about subtle differences between \ili{Icelandic} and \ili{German} (and English) in the distribution of fake indexicals, yet leaves as still open the difference between \isi{relative} clauses and complement clauses. The question remains whether Holmberg will think that I am the only one who likes my extension of his analysis.

\section*{Acknowledgments}
Thanks to Jonathan Bobaljik, Gísli Rúnar Harðarson, and Ian Roberts for comments and feedback.
 

{\sloppy
\printbibliography[heading=subbibliography,notkeyword=this]
}
\end{document}