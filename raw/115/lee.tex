\documentclass[output=paper]{LSP/langsci}
\author{Patrick Chi-wai Lee\affiliation{Caritas Institute of Higher Education, Hong Kong}}
\title{Anaphoric object drop in Chinese}
% \epigram{Change epigram}
\abstract{This squib proposes a novel means of solving the problem of non-specific null object with an indefinite antecedent in Chinese whereas \citet{Huang1982,Huang1984,Huang1989} argued that a dropped object is bound by a topic which must be definite. This squib proposes a formal representation that develops \citegen{Holmberg2005} and \citegen{RobertsHolmberg2010} analysis of radical pro-drop as [uD] (unvalued determiner-feature). Null object arguments in Chinese are argued to have the same featural composition: [uD]. They can be valued from an antecedent, but it is with a referential index or a referential variable. It is hoped that this squib can make a valuable contribution to our understanding of anaphoric specific and non-specific object drop in Chinese, particularly in the simplicity of its theoretical machinery.
}
\ChapterDOI{10.5281/zenodo.1116783}

\maketitle

\begin{document}
 

% \textbf{Keywords}: antecedent, Chinese, NP-deletion, object drop, topic feature

\section{Introduction}

This squib aims to offer a concise description of the interpretation of null objects in Chinese, and further proposes a formal representation that develops \citegen{Holmberg2005} and \citegen{RobertsHolmberg2010} analysis of radical \isi{pro-drop} as [uD] (unvalued determiner-feature). It is hoped that the proposal can shed some light in the context of classical analyses of null objects, especially in the Chinese syntax literature, which early on argued that the null object is a variable bound by an empty topic \citep{Huang1982,Huang1984,Huang1989}. For this variable analysis, a significant problem is that the dropped object in Chinese can have an indefinite interpretation, even though a topic must be definite.\footnote{Besides indefinite object-drop, the second classic problem with Huang’s (\citeyear{Huang1982,Huang1984,Huang1989}) variable analysis of null object drop is the availability of null object arguments coindexed with an antecedent across an island boundary.  \citet[277]{LiWei2014} argue that “a missing object can occur within islands co-indexed with their antecedent across island boundaries”. They (\citeyear[282]{LiWei2014}) explain that “empty objects can be within islands bound by an A or A’-antecedent across island boundaries, unlike topicalization cases, which are subject to island constraints and only involve A’-antecedents”. It should be noted that this squib does not attempt to address the issues about the null object and island boundaries, but those issues are also well-noted (see \citealt{LiWei2014} and \citealt{Li2014}).} This squib proposes a novel means of solving this problem through its descriptive and analytic distinction between specific and non-specific object drop.

To start with, anaphoric object drop means an object is dropped when there is an antecedent, and anaphoric object drop is characteristic of Chinese. Consider \REF{ex:li:1} from \citet[533]{Huang1984},

\ea%1
    \label{ex:li:1}

\ea 
\gll  Zhangsan kanjian  Lisi le     ma?     \\
  Zhangsan see     Lisi \textsc{asp}   \textsc{q}  \\
\glt ‘Did Zhangsan see Lisi?’


\ex
\gll  Ta kanjian     e   le. \\
  he see          [Lisi/him] \textsc{asp}    \\
\glt ‘He saw (him).’    (\textsc{asp} = \isi{aspect} marker; \textit{e} = empty category; \textsc{q} = question particle)
\z
\z



(\ref{ex:li:1}b) shows that the empty category refers to \textit{Lisi}, that is, the specific null object is bound by the definite topic in the discourse. \citet{Huang1982,Huang1984,Huang1989} argued that an empty object is a variable bound by an empty topic, and topics can be null given that they can be identified with a topic in a topic chain. I now look at another example with a non-specific null object with an indefinite topic. Consider \REF{ex:li:2},


\ea%2
\label{ex:li:2}
\ili{Mandarin}\\
\gll Zhang  yao   yi  bu   che     Mali   ye   yao.      \\
  Zhang  want   one \textsc{cl}  car     Mali   also   want    \\
\glt `Zhang wants a car.  Mary also wants one.'  (\textsc{cl} = Classifier)
\z



In \REF{ex:li:2} the null object \textit{yi bu che} ‘one car’ is non-specific. It does not mean that there is a car and he or she wants it. \citet{Huang1984} argued that a dropped object is bound by a topic which must be definite; however, the antecedent in this case is indefinite. Hence, this squib attempts to propose a novel means of solving this problem.


\section{Types of anaphoric object drop}


I now begin by examining various types of anaphoric object drop. They are distinguished by types of antecedent and by types of object dropped.\footnote{It should be noted that the verb-types play a role in the thematic assignment to the arguments, and the \isi{semantic} properties of verbs are also significant when interpreting a missing object (see \citealt{HuangEtAl2009} and \citealt{LiWei2014}).} Briefly, a null object with specific reference has a definite antecedent, and a null object with specific \isi{reference} is allowed where the antecedent does not have to be definite. In addition, a null object with non-specific reference has an indefinite antecedent.


\subsection{Specific object drop}


In \REF{ex:li:3} a null object with specific \isi{reference} has a definite antecedent \textit{zhe zhi xiong} ‘this bear’, with a \isi{demonstrative} \textit{zhe} ‘this’.


\ea%3
    \label{ex:li:3}
\ili{Mandarin} \\
    \gll   Zhang  kanjian  zhe zhi xiong le   Mali   ye   kanjian.  \\
   Zhang   see       this \textsc{cl}  bear \textsc{asp}   Mali   also   see  	\\
    \glt `Zhang   saw this bear.  Mary also saw it.' (The context is that they are looking at the same bear.)
    \z


Chinese also allows specific object drop where the antecedent does not have to be definite, and in fact, does not have to be specific as in \REF{ex:li:4}.


\ea%4
    \label{ex:li:4}
    \ili{Mandarin}\\
    \gll	    Zhang  kanjian   yi zhi   xiong le   Mali   ye   kanjian.    \\
Zhang  see      one \textsc{cl}   bear \textsc{asp}   Mali   also   see  	\\
    \glt `Zhang saw a bear.  Mary also saw it.'   (the same bear)
    \z



Here it can be specific in \REF{ex:li:4}, so that it means ‘Zhang and Mary saw a specific bear’ (it’s the one in the zoo), but it can also have a non-specific reading (see 5).


\subsection{Non-specific object drop}\label{sec:li:2.2}


\subsubsection{Non-specific existential}


In \REF{ex:li:5} a null object with non-specific \isi{reference} has an indefinite antecedent \textit{yi zhi xiong} ‘one bear’.


\ea%5
    \label{ex:li:5}
\ili{Mandarin}\\
    \gll	     Zhang  kanjian yi zhi xiong le     Mali   ye   kanjian.    \\
	Zhang  see    one \textsc{cl} bear \textsc{asp}     Mali   also   see  \\
    \glt `Zhang saw a bear.  Mary also saw one.' (meaning ‘Mary saw a bear’. It can be a different bear.)
    \z


Here it can also have a non-specific existential reading in \REF{ex:li:5}: ‘There is a bear such that Mary saw it’, and a sloppy interpretation is available to a missing object in Chinese (see \REF{ex:li:4} and \REF{ex:li:5}).


\subsubsection{Non-specific generic}


In \REF{ex:li:6} a null object with non-specific \isi{reference} has a ‘generic reading’: Zhang likes anything which belongs to the \textit{kind} or \textit{species} ‘bear’.


\ea%6
    \label{ex:li:6}
\ili{Mandarin}\\
    \gll   Zhang  xihuan xiong   Mali   ye   xihuan.        \\
	Zhang  like    bear   Mali   also   like  \\
    \glt `Zhang likes bears.  Mary also likes them.'

    \z


\subsubsection{Non-specific attributive (‘attributive reading of NP’)}
Consider non-specific object drop in Chinese as in \REF{ex:li:2}, repeated here as \REF{ex:li:7}.


\ea%7
    \label{ex:li:7}
\ili{Mandarin}\\
    \gll     Zhang  yao   yi  bu   che   Mali   ye   yao.      \\
Zhang  want   one \textsc{cl}  car   Mali   also   want  	\\
    \glt `Zhang wants a car.  Mary also wants one.'
    \z


\noindent In \REF{ex:li:7} a null object with non-specific \isi{reference} is non-specific in a different sense, and I will call this the ‘attributive reading of NP’. It is non-existential; it might be called a non-referential reading, but in a sense it is still referential.


In summary, based on the above data, anaphoric object drop can be classified into two main types: \REF{ex:li:1} specific object drop and \REF{ex:li:2} non-specific object drop which is further divided into: (a) non-specific existential, (b) non-specific generic and (c) non-specific attributive.


\section{Argument ellipsis and the derivation of object drop}


There are many works on discussion of ellipsis. Among many others, \citet{Saito2007} suggests that radical \isi{pro-drop} is a kind of argument ellipsis. He (\citeyear[25]{Saito2007}) argues that “those languages that have argument ellipsis can use LF objects provided by the discourse in the derivation of a new sentence”. \citet[269]{Sigurðsson2011Conditions} proposes “a unified minimalist approach to referential null arguments, where all types of (overt and silent) definite arguments require C/edge linking”. \citet{Duguine2014} is in favour of a unitary approach, and she proposes to reduce both types of \isi{pro-drop} to ellipsis of full-fledged argument DPs. \citet{Li2014} also contributes her idea of True Empty Categories (TEC) on argument ellipsis. She (\citeyear[65]{Li2014}) explains that “a topic in the discourse not mentioned in the sentences containing the TEC can also be an antecedent (empty topic). It can also have a linguistic antecedent in the previous discourse by a different speaker or a preceding clause of a complex sentence by the same speaker”.


As for the derivation of object drop, I now turn to examine how specific and non-specific null objects are licensed. Following \citet{Holmberg2005,Holmberg2010Null}),\footnote{\citet{Holmberg2005} argues that in the context of a feature theory like the one in \citeauthor{Chomsky1995} (\citeyear[Ch.~4]{Chomsky1995}, \citeyear{Chomsky2001}) the phi-features of I (or T) are themselves uninterpretable (or unvalued), being assigned interpretation (or value) by \isi{agreement} with the subject, so they cannot specify the value of the subject. Instead, he argues, the \isi{null subject} pronoun has features just like an overt pronoun. “Following the Chomskyan\ia{Chomsky, Noam} approach to \isi{agreement}, the null pronoun has interpretable phi-features and assigns values to the inherently unvalued features of \isi{Agr}” \citep[548]{Holmberg2005}. Holmberg further discusses a difference between two types of \isi{null subject} languages (NSLs): consistent NSLs and partial NSLs. As for consistent NSLs like \ili{Italian}, they have referential \isi{agreement}, i.e. the phi-features in I/T include the feature [D(efinite)]. As for partial NSLs like \ili{Finnish}, they have \isi{agreement}, but it is not referential, i.e. there is no [D] feature in I/T. As for discourse \isi{pro-drop} languages like Chinese, they have no unvalued phi-features in I/T (no subject-verb \isi{agreement}) \citep[559]{Holmberg2005}.} I firstly assume that null object arguments in Chinese (discourse \isi{pro-drop} language) have the same featural composition: [uD, N]. The null arguments have an unvalued D-feature which needs to be assigned a value in the course of the derivation, and a nominal feature which means they can occur in all positions where nominal constituents are found. I explain that [uD] in Chinese can be valued from an antecedent, but it is with a referential index [D\textsubscript{i} N] or a referential variable [D\textsubscript{x} N]. The valuation can be depicted as in \REF{ex:li:8}, where DP needs to be in a local relation to the null pronoun.

\ea
\label{ex:li:8}
  {DP}{\textsubscript{i}} {... [uD, N]} {→}{ DP}{\textsubscript{i}}{ ... [}{D}{\textsubscript{i}}{, N]}
\z


Consider \REF{ex:li:9} and \REF{ex:li:10} as illustrations of both [Di N] and [Dx N].


\ea%9
    \label{ex:li:9}
Referential index (specific interpretation)\\
\ili{Mandarin}\\
    \gll	Zhang  kanjian yi zhi   xiong le   Mali ye   kanjian       e.   \\
Zhang  see    one \textsc{cl}  bear\textsubscript{i} \textsc{asp}   Mali also   see    [D\textsubscript{i} N]\textbf{} 	\\
    \glt `Zhang saw a bear. Mary also saw it.'
    \z


\ea%10
    \label{ex:li:10}
Referential variable (non-specific interpretation)\\
\ili{Mandarin}\\
    \gll  Zhang  kanjian   yi zhi   xiong le Mali   ye   kanjian      e. \\
	Zhang  see      one \textsc{cl}   bear \textsc{asp}   Mali   also   see        [D\textsubscript{x} N]\\
    \glt `Zhang saw a bear. Mary also saw one.'
    \z



\subsection{An Aboutness topic feature accounts for specific object drop in Chinese}
 \citet[78]{HolmbergNikanne2002} also point out that “a language is topic-prominent when the argument which is externalized need not be the subject, but can be any category capable of functioning as topic. English is generally taken as the perfect representative of subject-prominent languages, while representatives of topic-prominent languages include Chinese, Tagalog, and \ili{Hungarian}”. As for Chinese, declarative sentences have a feature in C which requires a topic specifier, and I will call this feature [Aboutness topic] (see \citealt{FrascarelliHinterhölzl2007} on the typology of topics; see \citealt{BadanDelGobbo2011} on types of topics in \ili{Mandarin}\footnote{\citet{BadanDelGobbo2011} discuss three different types of Topics in \ili{Mandarin}: Aboutness Topics, Hanging Topics (HT) and Left Dislocated (LD) ones. They state that those types are organized hierarchically and they precede the only \isi{Focus} projection that occurs above IP, the lian-\isi{Focus}: Aboutness \isi{Topic} 〉 HT 〉 LD 〉 lian-\isi{Focus} 〉 IP.}). According to \citet{Lambrecht1994}, aboutness topic represents what the sentence is about. An aboutness topic is an XP referring to the entity which the sentence is about. As such it is always referential, always definite, and often has the function of subject. This topic can be an overt phrase or a null pronoun. Typically this specifier will be the result of movement from IP, leaving a copy behind (a ‘trace’ in theories prior to \citealt{Chomsky1995}), where this copy is ‘deleted’, i.e. not pronounced. The specifier may be a null pronoun, with a null pronoun copy in IP. The null pronoun in spec, \isi{CP} needs to receive a referential index from a topic antecedent, and the copy in IP will share this index. There is also an ‘EPP-feature’ postulated with the \isi{Topic} feature in Chinese C, which is the formal trigger of the movement (see \citealt{Chomsky1995,Chomsky2001}). Chinese also has the option of base-generating a topic in spec, \isi{CP} with no copy in IP. The following is an example to illustrate a topic derived by base-generation.


\ea%11
    \label{ex:li:11}
\ili{Mandarin} \citep[202]{HuangEtAl2009}\\
    \gll  shuiguo  wo   zui   xihuan   xiangjiao\\
	  fruit,   I   most   like   banana\\
    \glt ‘(As for) fruits, I like bananas most.’      
    \z


Chinese has a topic feature in C (coupled with an EPP-feature). The interpretation of a null topic in terms of a topic chain follows from general, \isi{universal} properties of null topics: a null topic will pick up the index of a local, salient topic in the immediately preceding discourse context, if there is an immediately preceding linguistic context, non-linguistic otherwise (see \citealt{FrascarelliHinterhölzl2007}). This makes null definite object pronouns possible in Chinese. Chinese has movement of different types of topics to spec, \isi{CP} which can be null if it has an antecedent.


\subsection{NP-deletion with (null) determiner stranding accounts for non-specific object drop}
As discussed in \sectref{sec:li:2.2} {\textit{non-specific object drop}}{,} {t}{he indefinite case cannot be topic drop because}{ an}{ indefinite DP cannot be topic}{. Th}{erefore, the remaining question is how anaphoric non-specific object drop is to be licensed. First, \citet{Jackendoff1971} described a rule which he called N’-deletion, which strands a genitive phrase, but cannot strand an indefinite or definite article. In the more current framework} {of the DP-hypothesis \citep{Abney1987}, the rule can be redefined as NP-deletion, deleting the complement of D under certain conditions. \citet{Hoji1998}}\footnote{\citet{Hoji1998} further explains that a bare nominal in Japanese such as \textit{kuruma} ‘car’ can be translated as any of ‘a car’, ‘the car’, ‘cars’, or ‘the cars’, and argues that this is because a nominal projection whose sole content is its head N can be interpreted in various ways as just indicated. He \citeyear[142]{Hoji1998} proposes that “the content of the N head of the null argument is supplied by the context of discourse. If the N head that is supplied by the context is a Name, then it can participate in a coreference\is{reference} relation with another Name”. In addition, the supplied N head can be \textit{kuruma} ‘car’ and it can function on a par with an indefinite in English. He points out that the null argument in Japanese behaves either like a definite or an indefinite. \citet{Tomioka2003} agrees in part with Hoji’s approach to null arguments in Japanese. Tomioka argues that Japanese lacks obligatory marking of definiteness and plurality on NPs, and therefore bare NP arguments get a variety of interpretations. His main claim is that null pronouns in discourse \isi{pro-drop} languages like Japanese and Chinese are the result of NP-deletion with null determiner stranding.}{ and \citet{Tomioka2003} argue that discourse \isi{pro-drop} languages have bare, D-less NP arguments. If NP-ellipsis is applied in such a language, the result is a null argument}{. For Chinese, it is controversial whether overtly article-less arguments are bare NPs or DPs with a null article. In either case, if NP-ellipsis applies, the result will be a null argument. I}{n the case of} {\REF{ex:li:10},}{ the null object will be a deleted NP, where I assume that there is a null [uD]: [}{\textsubscript{DP}} {[}{\textsubscript{D’}} {uD [}{\textsubscript{NP}}{ Ø]],} {and} {a DP can have an index without a pronounced D (i.e. [uD] gets a value from an antecedent).}


As for non-specific and specific object drop, I further assume that [uD] in Chinese can be valued from an antecedent, but it is with a referential index [D\textsubscript{i} N] or a referential variable [D\textsubscript{x} N]. A specific interpretation is the result when [uD] is valued by a referential index, whereas a non-specific interpretation is the result when it is valued by a referential variable. In both cases \REF{ex:li:9} and \REF{ex:li:10} the N of null [uD, N] is recovered by virtue of the overt noun of the antecedent.


After the above discussion of NP-ellipsis, I will assume that Tomioka is right. \citet{Huang1984} argues that there is a null topic mediating between the antecedent and the null object, but that cannot be so in the indefinite cases (because an indefinite DP cannot be a topic). In the cases of non-specific object drop, they are derived by NP-ellipsis, stranding a null D. In the cases of specific object drop, they are derived by movement, as under Huang’s theory of topic drop.


\section{Conclusion}


This squib proposes a novel means of solving the problem of non-specific null object with a definite topic. Null object arguments in Chinese are argued to have the same featural composition: [uD]. They can be valued from an antecedent, but it is with a referential index [Di N] or a referential variable [Dx N]. In addition, two types of anaphoric object drop in Chinese were studied: specific and non-specific object drop, and they were analyzed to be due to the existential state of an antecedent. Lastly, it is hoped that this squib can make a valuable contribution to our understanding of anaphoric specific and non-specific object drop in Chinese, particularly in the simplicity of its theoretical machinery.

 
\printbibliography[heading=subbibliography,notkeyword=this]
\end{document}
