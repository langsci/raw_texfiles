\chapter{Texts}

The following six Pichi texts represent four types of genre: narrative, routine procedure, elicitation, and conversation. Each sentence is provided with its text codes (placed above the sentence it refers to). This allows comparison with the analysis of examples in the grammar section. In conversations, speakers can be identified by the two-letter speaker code at the beginning of the text code. All texts contain \ili{Spanish} material ranging from single words to whole sentences. An interlinear gloss of Spanish material is provided where it occurs in the same utterance along with Pichi material. For sentences entirely in Spanish, only a free translation is provided. There are only a few Bube elements in the text, all of which stem from speaker (ab). The presence of \ili{Bube} material is indicated in squared brackets. Bube was not transcribed due to the absence of a scientific grammar and comprehensive dictionary at the time of field research. This has been partly remedied by the publication of \citet{Bolekia2009}, but the description of Bube still leaves much to be desired.

\section{Narrative and conversation: Miguel falls sick}

\largerpage
The main narrator in the following text is Abuela ‘grandmother’ (ab). Other discourse participants are Francisca (fr) and myself (ko). The text begins with a conversation between (ab) and (fr) on the latter’s competence in the Bube language. From (023)–(038), the conversation gives way to a brief story by (ab), in which she relates the hardship she endured living as an adolescent away from her family with a \textit{mísis} ‘matron’. In (039)–(042), (ab) then draws a comparison between the style of upbringing back then and her grandson Miguel’s behaviour towards grown-ups today. 


This leads (ab) to the main narrative from (043)–(134), in which (ab) gives an account of how her grandson Miguel came down with malaria a few nights before the recording took place, and how he was brought to hospital. The protagonists of this personal narrative are (ab) herself, her grandson Miguel, and his mother Tokobé. The narrative is characterised by extensive codemixing between Pichi and Spanish, as well as Pichi and Bube. 

\setcounter{equation}{0}  % RJK the author numbers from 1 per text

\ea{ab03ab 008\\
\gll
Hɛ́  a,  yu  nó  de  tɔ́k  Bubɛ,  a  wɔ́nda  náw   \\
\textsc{intj}  \textsc{1sg.sbj}  \textsc{2sg}  \textsc{neg}  \textsc{ipfv}  talk  Bube  \textsc{1sg.sbj}  wonder  now    \\
\gll
lɛk  háw  e  dɔ́n  fɔgɛ́t  Bubɛ  wé  e  gó  Panyá.   \\
like  how  \textsc{3sg.sbj}  \textsc{prf}  forget  Bube  \textsc{sub}  \textsc{3sg.sbj}  go  Spain    \\
\glt
‘Hey I, you don’t speak Bube, I wonder now how she had forgotten Bube when she went to Spain.’
  }\ex{
ab03ab 009\\
\gll
\textit{Pero}  ɛf  e  dé    yá  wán  mún,  e    go   \\
but  if  \textsc{3sg.sbj}  \textsc{be.loc}  here  one  month  \textsc{3sg.sbj}  \textsc{pot}    \\
\gll
tɔ́k=an.   \\
talk=\textsc{3sg.obj}    \\
\glt
‘But if she were here for a month, she would speak it.’
  }\ex{
ab03ab 010\\
\gll
Dí  wán  dɔ́n  de  tɔ́k,  dí  wán  de  tɔ́k  fáyn  pás   \\
this  one  \textsc{prf}  \textsc{ipfv}  talk  this  one  \textsc{ipfv}  talk  fine  pass    \\
\gll
in  sísta.   \\
\textsc{3sg.poss}  sister    \\
\glt
‘This one (here) already speaks it, this one talks better than her sister.’
  }\ex{
fr03ab 011\\
\gll
Nó,  nóto  trú  \textit{abuela}.   \\
\textsc{neg}  \textsc{neg}.\textsc{foc}  be.true  grandmother    \\
\glt
‘No, that’s not true grandmother.’
  }\ex{
fr03ab 012\\
\gll
Lage  de  tɔ́k  Bubɛ  pás  mí.   \\
\textsc{name}  \textsc{ipfv}  talk  Bube  pass  \textsc{1sg.indp}    \\
\glt
‘Lage speaks Bube better than me.’
  }\ex{
ab03ab 013\\
\gll
E  de  tɔ́k  Bubɛ  pás  yú?   \\
\textsc{3sg.sbj}  \textsc{ipfv}  talk  Bube  pass  \textsc{2sg.indp}    \\
\glt
‘She speaks Bube better than you?’
  }\ex{
ab03ab 014\\
\gll
Dís  wán  sɛ́f,  yu  dɔ́n  de  tráy.   \\
this  one  \textsc{foc}  \textsc{2sg}  \textsc{prf}  \textsc{ipfv}  try    \\
\glt
‘Even this one [you], you’re making an effort.’
  }\ex{
ko03ab 015\\
\gll
Bɔt  yu  bin  de  tɔ́k  Bubɛ  bifó?   \\
but  \textsc{2sg}  \textsc{pst}  \textsc{ipfv}  talk  Bube  before    \\
\glt
‘But you were speaking Bube before?’
  }\ex{
ab03ab 016\\
\gll
E  bin  de  tɔ́k=an,  e  nó  bin  de  hía   \\
\textsc{3sg.sbj}  \textsc{pst}  \textsc{ipfv}  talk=\textsc{3sg.obj}  \textsc{3sg.sbj}  \textsc{neg}  \textsc{pst}  \textsc{ipfv}  hear    \\
\gll
ɔ́da  lángwej.   \\
other  language    \\
\glt
‘She was speaking it, she didn’t understand any other language.’
  }\ex{
fr03ab 017\\
\gll
Wé  a  bin  smɔ́l,  a  bin  de  tɔ́k  Bubɛ.   \\
\textsc{sub}  \textsc{1sg.sbj}  \textsc{pst}  be.small  \textsc{1sg.sbj}  \textsc{pst}  \textsc{ipfv}  talk  Bube    \\
\glt
‘When I was small, I was speaking Bube.’
  }
\newpage   
  \ex{
ab03ab 018\\
\gll
Wé  yu  kɔmɔ́t  sík  dán  sík  na  Panyá,  wé  yu   \\
\textsc{sub}  \textsc{2sg}  come.out  be.sick  that  be.sick  \textsc{loc}  Spain  \textsc{sub}  \textsc{2sg}    \\
\gll
bin  sík,  náw  yu  bigín  tɔ́k  Panyá.   \\
\textsc{pst}  be.sick  now  \textsc{2sg}  begin  talk  Spain    \\
\glt
‘When you had just been sick in Spain, when you were sick, then you began speaking Spanish.’
  }\ex{
ab03ab 019\\
\gll
\'{A}fta,  yu  dé    hía,  \textit{¿cuántos}  \textit{años}  \textit{estuviste}  \textit{aquí?}   \\
then  \textsc{2sg}  \textsc{be.loc}  here  how.many  year.\textsc{pl}  you.were  here    \\
\glt
‘Then, you were here, how many years were you here?’
  }\ex{
fr03ab 020\\
\gll
\textit{Medio}  \textit{año,}  \textit{seis}  \textit{meses}.   \\
half  year  six  month.\textsc{pl}    \\
\glt
‘Half a year, six months.’
  }\ex{
ab03ab 021\\
\gll
\'{A}fta  in  papá  sɛ́f  kán  ték=an.   \\
then  \textsc{3sg.poss}  father  self  come  take=\textsc{3sg.obj}    \\
\glt
‘Then her father himself came to take her [away from here].’
  }\ex{
ab03ab 022\\
\gll
\'{A}fta  \textit{es}  \textit{la}  \textit{respuesta}.   \\
then  it.is  \textsc{def}  answer    \\
\glt
‘Then that’s the answer.’
  }\ex{
ab03ab 023\\
\gll
Mí,  lɛk  háw  yu  de  sí  mí,  a  dɔ́n  sí   \\
\textsc{1sg.indp}  like  how  \textsc{2sg}  \textsc{ipfv}  see  \textsc{1sg.indp}  \textsc{1sg.sbj}  \textsc{prf}  see    \\
\gll
plɛ́nte  tín.   \\
be.plenty  thing    \\
\glt
‘As for me, as you see me, I’ve seen many things (in life).’
  }\ex{
ab03ab 024\\
\gll
A  nó  di  tɛ́n  wé  yu  de  smɛ́l  pamáyn,   \\
\textsc{1sg.sbj}  know  \textsc{def}  time  \textsc{sub}  \textsc{2sg}  \textsc{ipfv}  smell  oil    \\
\gll
swit-ɔ́yl.   \\
tasty.\textsc{cpd}{}-oil    \\
\glt
‘I know the time when you’d smell oil, sweet oil.’
  }\ex{
ab03ab 025\\
\gll
Yu  mísis  sɛ́n  yú  gó  na  shɔ́p,  sé  gó  báy   mí\\
\textsc{2sg}  matron  send  \textsc{2sg.indp}  go  \textsc{loc}  shop  \textsc{quot}  go  buy   \textsc{1sg.indp}\\

\gll
swit-ɔ́yl.   \\
tasty.\textsc{cpd}{}-oil    \\
\glt
‘The matron [head (\textsc{f}.) of the household that speaker (ab) was staying in] of the house would send you to the shop, saying “go buy some sweet oil for me”.’
  } 
\newpage  
\ex{
ab03ab 026\\
\gll
Yu  gó  yu  bríng  sɔn  ɔ́yl.   \\
\textsc{2sg}  go  \textsc{2sg}  bring  some  oil    \\
\glt
‘You would go (and) bring some oil.’
  }\ex{
ab03ab 027\\
\gll
Wé  e  lúk=an  só  \upshape{[exclamation]}.   \\
\textsc{sub}  \textsc{3sg.sbj}  look=\textsc{3sg.obj}  like.that      \\
\glt
‘And she’d look at it like this [exclamation in Bube].’
  }\ex{
ab03ab 028\\
\gll
Dís?   \\
this    \\
\glt
‘This?’
  }\ex{
ab03ab 029\\
\gll
Dís  nóto  Manolete.   \\
this  \textsc{neg}.\textsc{foc}  \textsc{name}    \\
\glt
‘This is not Manolete (oil).’
  }\ex{
ab03ab 030\\
\gll
Gó  lɛ́f=an,  gó  lɛ́f=an!   \\
go  leave=\textsc{3sg.obj}  go  leave=\textsc{3sg.obj}    \\
\glt
‘Go leave it, go leave it [bring it back]!’
  }\ex{
ab03ab 031\\
\gll
Di  trú  \textit{comedor}  dé    fɔ  \textit{soja},  Manolete,  Cordobés.   \\
\textsc{def}  be.true  dining-room  \textsc{be.loc}  \textsc{prep}  soya  \textsc{name}    \textsc{name}      \\
\glt
‘The real dining-room has soy bean oil, Manolete, Cordobes [vegetable oil brands], right?’
  }\ex{
ab03ab 032\\
\gll
Na  di  bɛ́tɛ               bɛ́tɛ  swít  ɔ́yl,  pyɔ́       pyɔ́  \textit{uvas}.   \\
\textsc{foc}  \textsc{def}  very.good  \textsc{rep}  tasty  oil  pure    \textsc{rep}  grapes    \\
\glt
‘That’s the very best sweet oil, (made from) purest grapes.’
  }\ex{
ab03ab 033\\
\gll
Ɛf  nóto  yu  báy,  dán    húman  go  bít    yú    sóté   \\
if  \textsc{neg}.\textsc{foc}  \textsc{2sg}  buy  that  woman  \textsc{pot}  beat  \textsc{2sg.indp}  until    \\
\gll
yu  go  gó  lɛ́f=an.   \\
\textsc{2sg}  \textsc{pot}  go  leave=\textsc{3sg.obj}    \\
\glt
‘If it weren’t the case that you had bought (the right oil), that woman would beat you until you would go leave it [bring it back].’
  }
\newpage  
\ex{
ab03ab 034\\
\gll
Wé  yu  dɔ́n  gó  lɛ́f=an,  yu  go  gí  di  mán   \\
\textsc{sub}  \textsc{2sg}  \textsc{prf}  go  leave=\textsc{3sg.obj}  \textsc{2sg}  \textsc{pot}  give  \textsc{def}  man    \\
\gll
wé  e  de  sɛ́l  di  \textit{funda},  e  nó  go   \\
\textsc{sub}  \textsc{3sg.sbj}  \textsc{ipfv}  sell  \textsc{def}  receptacle  \textsc{3sg.sbj}  \textsc{neg}  \textsc{pot}    \\
\gll
ték=an  fɔ  yú.   \\
take=\textsc{3sg.obj}  \textsc{prep}  \textsc{2sg.indp}    \\
\glt
‘When you’ve gone to leave it, you would give (the oil) to the man who is selling the receptacle (with the oil) and he wouldn’t take it (back) from you.’
  }\ex{
ab03ab 035\\
\gll
Yu  gɛ́fɔ    gó  fɔ  yu  fámbul.   \\
\textsc{2sg}  have.to  go  \textsc{prep}  \textsc{2sg}  family    \\
\glt
‘You would have to go to your (own) family.’
  }\ex{
ab03ab 036\\
\gll
Yú  gí  dɛ́n  dán  smɔ́l  pamáyn  mék  dɛn  gí   \\
\textsc{2sg.indp}  give  \textsc{3pl.indp}  that  small  oil  \textsc{sbjv}  \textsc{3pl}  give    \\
\gll
yú  mɔní  yu  go  báy  di  wán  wé  yu  mísis   \\
\textsc{2sg.indp}  money  \textsc{2sg}  \textsc{pot}  buy  \textsc{def}  one  \textsc{sub}  \textsc{2sg}  matron    \\
\gll
dé,  adɔnkɛ́  e  nó  sí  yú  wán  hól  dé,   \\
there  even.if  \textsc{3sg.sbj}  \textsc{neg}  see  \textsc{2sg.indp}  one  whole  day    \\
\gll
e  nó  bísin  wáns  yu  bríng  di  pamáyn.   \\
\textsc{3sg.sbj}  \textsc{neg}  care  once  \textsc{2sg}  bring  \textsc{def}  oil    \\
\glt
‘You would give them [your family] that little bit of oil so that they gave you money (so that) you would go buy the one [the correct oil] that your matron there, even if she didn’t see you for a whole day, she wouldn’t care once you brought the [correct] oil.’
  }\ex{
ab03ab 037\\
\gll
Mí  dɔ́n  sɔ́fa.   \\
\textsc{1sg.indp}  \textsc{prf}  suffer    \\
\glt
‘I have suffered.’
  }\ex{
ab03ab 038\\
\gll
A  dɔ́n  sí  bihɛ́n    \upshape{[continues  in  Bube}].   \\
\textsc{1sg.sbj}  \textsc{prf}  see  behind    \\
\glt
‘I have seen behind (...)’
  }\ex{
ab03ab 039\\
\gll
Pero  pikín  tidé,  náw  yu  sɛ́n=an,  dí  pikín,  wé   \\
but  child  today  now  \textsc{2sg}  send=\textsc{3sg.obj}  this  child  \textsc{sub}    \\
\gll
a  de  sɛ́n=an,  e  nó  de  gó  mɔ́.   \\
\textsc{1sg.sbj} \textsc{ipfv}  send=\textsc{3sg.obj}  \textsc{3sg.sbj}  \textsc{neg}  \textsc{ipfv}  go  more    \\
\glt
‘But a child today, (if) you send it (for something) now, this child, when I’m sending him, he doesn’t go anymore.’
  }\ex{
ab03ab 040\\
\textit{Ay,} \textit{todo} \textit{el} \textit{día} \textit{de} \textit{hoy} \textit{tú} \textit{me} \textit{vas} \textit{a} \textit{mandar,} \textit{¡vete} \textit{tú} \textit{misma!}   \\
\glt
‘[My grandson would say] “Oh, the whole day today you’re going to send me around, you go yourself!”’
  }\ex{
ab03ab 041\\
\gll
Náw  náw  mék  a  sɛ́n=an  na  gran-pá   \\
now  \textsc{rep}  \textsc{sbjv}  \textsc{1sg.sbj}  send=\textsc{3sg.obj}  \textsc{loc}  grand-pa    \\
\gll
in  rúm,  e  go  sé  e  de  fíɛ,  e  nó   \\
\textsc{3sg.poss}  room  \textsc{3sg.sbj}  \textsc{pot}  say  \textsc{3sg.sbj}  \textsc{ipfv}  fear  \textsc{3sg.sbj}  \textsc{neg}    \\
\gll
go  gí  mí  di  tín    wé  a  de    sɛ́n=an.   \\
\textsc{pot}  give  \textsc{1sg.indp}  \textsc{def}  thing    \textsc{sub}  \textsc{1sg.sbj}  \textsc{ipfv}  send=\textsc{3sg.obj}    \\
\glt
‘Right now, let me (try) send him to grandfather’s room, he [my grandson] would say that he’s afraid, (that) he wouldn’t give me the thing I’m sending him for.’
  }\ex{
ab03ab 042\\
\gll
Sé  ín  nó  wánt  in  \textit{abuelo}  skrách=an.   \\
\textsc{quot}  \textsc{3sg.indp}  \textsc{neg}  want  \textsc{3sg.poss}  grandfather  scratch=\textsc{3sg.obj}    \\
\glt
‘Because he [\textsc{emp}] doesn’t want his [deceased] grandfather to scratch him.’
  }\ex{
ab03ab 043\\
\gll
A  dɔ́n  tɛ́l  yú  wétin  pás  na  nɛ́t,  dán  nɛ́t.   \\
\textsc{1sg.sbj}  \textsc{prf}  tell  \textsc{2sg.indp}  what  happen  \textsc{loc}  night  that  night    \\
\glt
‘I’ve already told you what happened at night, that night.’
  }\ex{
ab03ab 044\\
\gll
Yɛ́stadé.   \\
yesterday    \\
\glt
‘Yesterday.’
  }\ex{
ab03ab 045\\
\gll
Mí  gó  na  mi  béd,  a  bigín  de  mɛ́mba  mi  yón   trɔ́bul.\\
\textsc{1sg.indp}  go  \textsc{loc}  \textsc{1sg.poss}  bed  \textsc{1sg.sbj}  begin  \textsc{ipfv}  remember  \textsc{1sg.poss}  own  trouble   \\
\glt
‘I [\textsc{emp}] went to bed, I began thinking about my own problems.’
  }\ex{
ab03ab 046\\
\gll
Dɛ́n  slíp  dɛn  de  \textit{ronca}.   \\
\textsc{3pl.indp}  sleep  \textsc{3pl}  \textsc{ipfv}  snore    \\
\glt
‘They [\textsc{emp}] [the others in the house] had lied down and were snoring.’
  }\ex{
ab03ab 047\\
\gll
\'{I}n  de  kakara,  kakara         kakara.   \\
\textsc{3sg.indp}  \textsc{ipfv}  \textsc{ideo}          \textsc{rep}  \textsc{rep}    \\
\glt
‘He [Miguel] was all fidgety.’
  }\ex{
ab03ab 048\\
\gll
E  de  \textit{costumbre}.   \\
\textsc{3sg.sbj}  \textsc{ipfv}  habit    \\
\glt
‘He’s was getting used to it.’
  }\ex{
ab03ab 049\\
\gll
Di  wé  in  áwa  nɔ́ba  rích  fɔ  slíp,  e  go   \\
\textsc{def}  way  \textsc{3sg.poss}  hour  \textsc{neg}.\textsc{prf}  arrive  \textsc{prep}  sleep  \textsc{3sg.sbj}  \textsc{pot}    \\
\gll
bigín  de  hala-hála  mí.   \\
begin  \textsc{ipfv}  \textsc{red}.\textsc{cpd}{}-shout  \textsc{1sg.indp}    \\
\glt
‘Since his time for sleeping hadn’t come yet, he was going to begin shouting for me.’
  }\ex{
ab03ab 050\\
\gll
Smɔ́ltɛn  slíp  kéch=an.   \\
shortly.after  sleep  catch=\textsc{3sg.obj}    \\
\glt
‘Shortly after, he became sleepy.’
  }\ex{
ab03ab 051\\
\gll
E  sé  “áy”  a  hía  di  hála.   \\
\textsc{3sg.sbj}  say  \textsc{intj}  \textsc{1sg.sbj}  hear  \textsc{def}  shout    \\
\glt
‘He said “ay”, I heard the shout.’
  }\ex{
ab03ab 052\\
\gll
In  mamá  sé  wétin  pás,  wétin  pás?   \\
\textsc{3sg.poss}  mother  say  what  happen  what  happen    \\
\glt
‘His mother said what happened, what happened?’
  }\ex{
ab03ab 053\\
\gll
E  sé  “mɔ́mi  mɔ́mi  yu  nó  de  sí  dán  mán  wé   \\
\textsc{3sg.sbj}  say  mum  mum  \textsc{2sg}  \textsc{neg}  \textsc{ipfv}  see  that  man  \textsc{sub}    \\
\gll
e  rɔ́n  gó  \textit{abuela}  in  rúm?”\\
\textsc{3sg.sbj}  run  go  grandmother  \textsc{3sg.poss}  room    \\
\glt
‘He said “mum, mum don’t you see that man who ran into grandmother’s room?”’
  }\ex{
ab03ab 054\\
\gll
“E  dɔ́n  pás,  e  dɔ́n  pás,  e  dɔ́n  pás.”\\
\textsc{3sg.sbj}  \textsc{prf}  pass  \textsc{3sg.sbj}  \textsc{prf}  pass  \textsc{3sg.sbj}  \textsc{prf}  pass    \\
\glt
‘“He has just passed by, he has just passed by, he has just passed by.”’
  }\ex{
ab03ab 055\\
\gll
E  bigín  de  trímbul.   \\
\textsc{3sg.sbj}  begin  \textsc{ipfv}  tremble    \\
\glt
‘He began to tremble.’
  }\ex{
ab03ab 056\\
\gll
Náw  e  sé/  in  mamá  tɛ́l=an  sé   \\
now  \textsc{3sg.sbj}  say  \textsc{3sg.poss}  mother  tell=\textsc{3sg.obj}  \textsc{quot}    \\
\gll
% missing glossing for:
nɔ́, \textit{abuela} \textit{fue} \textit{a} \textit{la} \textit{cocina} \textit{a}  \textit{beber} \textit{agua}.   \\
\textsc{neg}  grandmother  went  to  the  kitchen  to  drink  water\\
\glt
 ‘Now he said/ his mother told him that “no, grandmother went to the kitchen to drink water”.’
  }\ex{
ab03ab 057\\
\gll
E  sé  \textit{pero}  \textit{es}  \textit{un}  \textit{hombre.}   \\
\textsc{3sg.sbj}  say  but  it.is  \textsc{def}  man    \\
\glt
‘He said “but it’s a man”.’
  }\ex{
ab03ab 058\\
\gll
E  nó  kán  slíp  mɔ́  ó.   \\
\textsc{3sg.sbj}  \textsc{neg}  \textsc{pfv}  sleep  more  \textsc{sp}    \\
\glt
‘He actually didn’t sleep again.’
  }\ex{
ab03ab 059\\
\gll
Sɔn  káyn  fíba  kán  kéch=an,  Tokobé  nó  kán  sabí.   \\
some  kind  fever  \textsc{pfv}  catch=\textsc{3sg.obj}  \textsc{name}  \textsc{neg}  \textsc{pfv}  know    \\
\glt
‘He got a serious fever (and) Tokobé didn’t get to know (about it).’
  }\ex{
ab03ab 060\\
\gll
Mɔ́nin  tɛ́n  e  gráp  e  sé   \\
morning  time  \textsc{3sg.sbj}  get.up  \textsc{3sg.sbj}  say    \\
\gll
% missing glossing for:
\textit{Miguel} \textit{vete} \textit{a} \textit{hacer} \textit{pipí} \textit{y} \textit{vete} \textit{a} \textit{bañarte,} \textit{hay} \textit{clase.}   \\
Miguel  go  to  do  wee-wee  and  to  bathe  there.is  class\\
\glt
 ‘In the morning she got up (and) she said “Miguel go do a wee-wee and go take a bath, you have classes”.’
  }\ex{
ab03ab 061\\
\gll
E  kán,  e  sé  \textit{“abuela,}  \textit{llevame}  \textit{al}  \textit{hospital”.}   \\
\textsc{3sg.sbj}  come  \textsc{3sg.sbj}  \textsc{quot}  grandmother  bring.me  to  hospital    \\
\glt
‘He came, he said “grandmother take me to hospital”.’
  }\ex{
ab03ab 062\\
\gll
E  sé  \textit{“no}  \textit{puedo}  \textit{parar”.}   \\
\textsc{3sg.sbj}  \textsc{quot}  \textsc{neg}  I.can  stand    \\
\glt
‘He said “I can’t (even) stand”.’
  }\ex{
ab03ab 063\\
\gll
E  dé  na  grɔ́n.   \\
\textsc{3sg.sbj}  \textsc{be.loc}  \textsc{loc}  ground    \\
\glt
‘He was (lying) on the ground.’
  }\ex{
ab03ab 064\\
\gll
Na  só  e  de  swɛ́t.   \\
\textsc{foc}  so  \textsc{3sg.sbj}  \textsc{ipfv}  sweat    \\
\glt
‘He was sweating just like that.’
  }\ex{
ab03ab 065\\
\gll
Wé  a  kin  mék  só,  a  nó  de  fíl  hɔ́t.   \\
\textsc{sub}  \textsc{1sg.sbj}  \textsc{hab}  make so  \textsc{1sg.sbj}  \textsc{neg}  \textsc{ipfv}  feel  hot    \\
\glt
‘When I would do like this [places her hand on her forehead], I wasn’t feeling heat.’
  }\ex{
ab03ab 066\\
\gll
Pero  wé  a  kin  tɔ́ch  in  fút,  in  hán   \\
but  \textsc{sub}  \textsc{1sg.sbj}  \textsc{hab}  touch  \textsc{3sg.poss}  leg  \textsc{3sg.poss}  arm    \\
\gll
dé,  na  só  dɛn  kól  [ko::l].   \\
there  \textsc{foc}  so  \textsc{3pl}  be.cold    \\
\glt
‘But when I would touch his leg (and) his arm there, they were so incredibly cold.’
  }\ex{
ab03ab 067\\
\gll
A  ték=an  pút=an  pantáp  mi  bɛlɛ́.   \\
\textsc{1sg.sbj}  take=\textsc{3sg.obj}  put=\textsc{3sg.obj}  on  \textsc{1sg.poss}  belly    \\
\glt
‘I put him onto my stomach.’
  }\ex{
ab03ab 068\\
\gll
Na  só  a  de  wáyp=an,  a  de  \textit{sopla}  ín   \\
\textsc{foc}  so  \textsc{1sg.sbj}       \textsc{ipfv}  wipe=\textsc{3sg.obj}  \textsc{1sg.sbj}  \textsc{ipfv}  blow  \textsc{3sg.indp}    \\
\gll
fwífwífwí.   \\
\textsc{ideo}    \\
\glt
‘I was wiping him, I was fanning him just like that.’
  }\ex{
ab03ab 069\\
\gll
A  lúk=an.   \\
\textsc{1sg.sbj}  look=\textsc{3sg.obj}    \\
\glt
‘I looked at him.’
  }\ex{
ab03ab 070\\
\gll
Na  só  in  hát  mék  kutuku kutuku     kutuku.   \\
\textsc{foc}  so  \textsc{3sg.poss}  heart  make  \textsc{ideo}         \textsc{rep}  \textsc{rep}    \\
\glt
‘His heart was racing just like that.’
  }\ex{
ab03ab 071\\
\gll
A  kɔ́l  Tokobé  a  sé  “mɔ́mi”,  a  sé  “kán”.\\
\textsc{1sg.sbj}  call  \textsc{name}  \textsc{1sg.sbj}  \textsc{quot}  mum  \textsc{1sg.sbj}  \textsc{quot}  come    \\
\glt
‘I called Tokobé, I said “mother”, I said “come”.’
  }\ex{
ab03ab 072\\
\gll
A  bɛ́g,  lúk  dís  pikín,  dí  pikín  nó  dé  gúd.   \\
\textsc{1sg.sbj}  ask.for  look  this  child  this  child  \textsc{neg}  \textsc{be.loc}  good    \\
\glt
‘Please, look at this child, this child is not well.’
  }\ex{
ab03ab 073\\
\gll
Na  ín  e  de  kán  púl  mí  dán  torí.   \\
\textsc{foc}  \textsc{3sg.indp}  \textsc{3sg.sbj} \textsc{ipfv}  come  remove  \textsc{1sg.indp}  that  story    \\
\glt
‘That’s when she was coming to tell me that story.’
  }\ex{
ab03ab 074\\
\gll
E  sé  “na  nɛ́t”,  e  sé  “na  só  yu  bin  hía   \\
\textsc{3sg.sbj}  \textsc{quot}    \textsc{loc}  night  \textsc{3sg.sbj}      \textsc{quot}  \textsc{foc}  like.that  \textsc{2sg}  \textsc{pst}  hear    \\
\gll
ín  hála”.\\
\textsc{3sg.indp}  shout\\
\glt
‘She said “at night”, she said “that’s how you heard him shout”.’
  }\ex{
ab03ab 075\\
\gll
E  sé  frɔn  dán  hála  di  pikín  nó  slíp  mɔ́.   \\
\textsc{3sg.sbj}  \textsc{quot}  from  that  shout  \textsc{def}  child  \textsc{neg}  sleep  more    \\
\glt
‘She said “since that shout the child didn’t sleep again”.’
  }\ex{
ab03ab 076\\
\gll
E  kán  gɛ́t  fíba.   \\
\textsc{3sg.sbj}  \textsc{pfv}  get  fever    \\
\glt
‘He got a fever.’
  }\ex{
ab03ab 077\\  
{[A  sentence  in  Bube]}.   \\
  }\ex{
ab03ab 078\\
\textit{“Vete,} \textit{a} \textit{bañar.”}\\
\glt
‘“Off you go, go have a bath”.’
  }\ex{
ab03ab 079\\
\gll
E  púl=an  na  pantáp  di  béd.   \\
\textsc{3sg.sbj}  remove=\textsc{3sg.obj}  \textsc{loc}  on  \textsc{def}  bed    \\
\glt
‘She pulled him from the bed.’
  }\ex{
ab03ab 080\\
\gll
Na  só  e  de  swɛ́t.   \\
\textsc{foc}  so  \textsc{3sg.sbj}  \textsc{ipfv}  sweat    \\
\glt
‘He was sweating just like that.’
  }\ex{
ab03ab 081\\
\gll
A  púl  in  klós,  a  híb=an   \\
\textsc{1sg.sbj}  remove  \textsc{3sg.poss}  clothing  \textsc{1sg.sbj}  heave=\textsc{3sg.obj}    \\
\gll
pantáp  di  béd.   \\
on  \textsc{def}  bed    \\
\glt
‘I removed his clothes, I heaved him onto the bed.’
  }\ex{
ab03ab 082\\
\gll
Sé  “papá gɔ́d,  ús=káyn trɔ́bul     dís?”\\
\textsc{quot}     \textsc{intj}     God    \textsc{q}=kind  trouble  this    \\
\glt
‘(I) said (to myself) “oh God, what kind of trouble is this?”’
  }\ex{\largerpage
ab03ab 083\\
\gll
A  púl  in  \textit{camiseta,}  a  pút=an   \\
\textsc{1sg.sbj}  remove  \textsc{3sg.poss}  singlet  \textsc{1sg.sbj}  put=\textsc{3sg.obj}    \\
\gll
pantáp  béd  a  gó  a  púl  di  trɔsís  a   \\
on  bed  \textsc{1sg.sbj}       go  \textsc{1sg.sbj}  remove  \textsc{def}  trousers  \textsc{1sg.sbj}    \\
\gll
híb=an  ínsay  di  \textit{bañera}.   \\
heave=\textsc{3sg.obj}  inside  \textsc{def}  bathtub    \\
\glt
‘I removed his singlet, I put him on the bed (and) I removed the trousers (and) I heaved him inside the bathtub.’
  }\ex{
ab03ab 084\\
\gll
A  ték  wán  kɔ́p  watá,  a  ték=an  a   \\
\textsc{1sg.sbj}  take  one  cup  water  \textsc{1sg.sbj}  take=\textsc{3sg.obj}  \textsc{1sg.sbj}    \\
\gll
mék  bwa   bwa  bwa  bwa.   \\
make  \textsc{ideo}   \textsc{rep}      \textsc{rep}    \textsc{rep}    \\
\glt
‘I took a cup of water, I took it (and) splushed him all over with water.’
  }\ex{
ab03ab 085\\
\gll
A  sé  \upshape{[continues in Bube]}.\\
\textsc{1sg.sbj}  \textsc{quot}      \\
\glt
‘I said (...)’
  }\ex{
ab03ab 086\\
\gll
Smɔ́ltɛn  e  mék  \upshape{[imitates exhalation]}.   \\
shortly.after  \textsc{3sg.sbj}  make      \\
\glt
‘Shortly he made [imitates exhalation].’
  }\ex{
ab03ab 087\\
\gll
A  sé    \textit{“¿cómo}  \textit{sientes?”}\\
\textsc{1sg.sbj}  \textsc{quot}      how    you.feel    \\
\glt
‘I said “how do you feel?”’
  }\ex{
ab03ab 088\\
\gll
E  sé  \textit{“abuela}  \textit{ya}  \textit{siento}  \textit{bien.”}\\
\textsc{3sg.sbj}  \textsc{quot}  grandmother  already  I.feel  good\\
\glt
‘He said “grandmother, I already feel fine”.’
  }\ex{
ab03ab 089\\
\gll
E  kɔmɔ́t  na  \textit{bañera,}  ín  sɛ́f  kán  gó.   \\
\textsc{3sg.sbj}  come.out  \textsc{loc}  bathtub  \textsc{3sg.indp}  self  \textsc{pfv}  go    \\
\glt
‘He came out of the bathtub, he himself left (it).’
  }\ex{
ab03ab 090\\
\gll
A  gí=an  di  haf-táwɛl.   \\
\textsc{1sg.sbj}  give=\textsc{3sg.obj}  \textsc{def}  half.\textsc{cpd}{}-towel    \\
\glt
‘I gave him the [his] little towel.’
  }\ex{
ab03ab 091\\
\gll
A  kɛ́r=an  gó  na  \textit{comedor}.   \\
\textsc{1sg.sbj}  carry=\textsc{3sg.obj}  go  \textsc{loc}  dining-room    \\
\glt
‘I carried him to the dining-room.’
  }\ex{
ab03ab 092\\
\gll
Sé  Tokobé,  kɛ́r  di  pikín  na  ɔspítul.   \\
\textsc{quot}  \textsc{name}  carry  \textsc{def}  child  \textsc{loc}  hospital    \\
\glt
‘(I) said Tokobé, bring this child to hospital.’
  }\ex{
ab03ab 093\\
\gll
Mí  nó  sé  di  pikín  \upshape{[continues  in  Bube]}.   \\
\textsc{1sg.indp}  \textsc{neg}  \textsc{quot}  \textsc{def}  child    \\
\glt
‘I know that the child (...)’
  }\ex{
ab03ab 094\\
\gll
A  wánt  ték  solwatá  mék  a  gí=an,   \\
\textsc{1sg.sbj}  want  take  saltwater  \textsc{sbjv}  \textsc{1sg.sbj}  give=\textsc{3sg.obj}    \\
\gll
a  sé  “chip”  nɔ́.   \\
\textsc{1sg.sbj}  \textsc{quot}    \phantom{‘}\textsc{skt}  \textsc{neg}    \\
\glt
‘I wanted to take saltwater and give it to him, I said \textsc{[skt]} no.’
  }\ex{
ab03ab 095\\
\gll
E  sé  na  hángri.   \\
\textsc{3sg.sbj}  \textsc{quot}  \textsc{foc}  hunger    \\
\glt
‘He said “it’s hunger” [that’s worrying me].’
  }\ex{
ab03ab 096\\
\gll
Dán  banána,  a  gí=an  sɔn.   \\
that  banana  \textsc{1sg.sbj}  give=\textsc{3sg.obj}  some    \\
\glt
‘That banana [points to a stalk lying in the corner], I gave him one.’
  }\ex{
ab03ab 097\\
\gll
E  sé  \textit{“abuela},  e  nó  kɛ́r”.\\
\textsc{3sg.sbj}  \textsc{quot}  grandmother  \textsc{3sg.sbj}  \textsc{neg}  carry    \\
\glt
‘He said “grandmother, it wasn’t enough.”’
  }\ex{
ab03ab 098\\
\gll
Mí  sé  \upshape{[continues  in  Bube]}.   \\
\textsc{1sg.indp}  \textsc{quot}      \\
\glt
‘I [\textsc{emp}] \textsc{quot} (...)’
  }\ex{
ab03ab 099\\
\gll
A  sé  nó  gí=an  \textit{leche},  gí=an  wɔtá!   \\
\textsc{1sg.sbj}  \textsc{quot}  \textsc{neg}  give=\textsc{3sg.obj}  milk  give=\textsc{3sg.obj}  water    \\
\glt
‘I said “don’t give him milk, give him water!”’
  }\ex{
ab03ab 100\\
\gll
A  gí=an.   \\
\textsc{1sg.sbj}  give=\textsc{3sg.obj}    \\
\glt
‘I gave him (the water).’
  }\ex{
ab03ab 101\\
\gll
E  sé  e  nó  kɛ́r.   \\
\textsc{3sg.sbj}  \textsc{quot}  \textsc{3sg.sbj}  \textsc{neg}  carry    \\
\glt
‘He said it wasn’t enough.’
  }
\newpage   
\ex{
ab03ab 102\\
\gll
Lɛk  háw  Tokobé  púl  di/  e  nɔ́ba  púl  di   \\
like  how  \textsc{name}  remove  \textsc{def}  \textsc{3sg.sbj}  \textsc{neg}.\textsc{prf}  remove  \textsc{def}    \\
\gll
glás  e  wánt  mɔ́.   \\
glass  \textsc{3sg.sbj}  want  more    \\
\glt
‘As soon as Tokobé removed the/ she hadn’t yet removed the glass (and) he wanted more.’
  }\ex{
ab03ab 103\\
\gll
Mí  gó  dɔ́n.   \\
\textsc{1sg.indp}  go  down    \\
\glt
‘I went down(stairs).’
  }\ex{
ab03ab 104\\
\gll
Wé  a  kɔmɔ́t  dɔ́n,  a  gó  sidɔ́n  bifór=an,   \\
\textsc{sub}  \textsc{1sg.sbj}  come.out  down  \textsc{1sg.sbj}  go  sit       before=\textsc{3sg.obj}    \\
\gll
a  sé  “\textit{¿cómo}  \textit{sientes}?”\\
\textsc{1sg.sbj}  \textsc{quot}     \phantom{‘¿}how  feel.\textsc{2sg}    \\
\glt
‘When I came back from downstairs, I went to sit before him (and) I said “how do you feel?”’
  }\ex{
ab03ab 105\\
\gll
E  sé  \textit{“abuela,}  \textit{siento}  \textit{mal,}  \textit{quiero}  \textit{ir}  \textit{al}   \\
\textsc{3sg.sbj}  \textsc{quot}  grandmother  I.feel  bad  I.want  go  to    \\
\gll
\textit{hospital”.}\\
hospital    \\
\glt
‘He said “grandmother, I feel bad, I want to go to the hospital”.’
  }\ex{
ab03ab 106\\
\gll
E  sé  \textit{“cuando}  \textit{una}  \textit{persona}  \textit{está}  \textit{enferma}   \\
\textsc{3sg.sbj}  \textsc{quot}  when  a  person  is  sick    \\
\gll
\textit{los}  \textit{demás}  \textit{no}  \textit{deben}  \textit{estar}  \textit{con}  \textit{ella}  \textit{sentada”.}\\
the  others  not  must  be  with him  seated   \\
\glt
‘He said “when a person is sick, the others are not supposed to be sitting with him”.’
  }\ex{
ab03ab 107\\
\textit{Porque}  \textit{cuando} \textit{se}  \textit{va}  \textit{a}     \textit{vomitar,} \textit{se}   \textit{va} \textit{a} \textit{mojar}  \textit{con}  \textit{vómito}.   \\
\glt
‘Because when he vomits they will get wet with vomit.’
  }\ex{
ab03ab 108\\
\gll
A  sé  \textit{“has}    \textit{vomitado?”}\\
\textsc{1sg.sbj}  \textsc{quot}  have.you  vomited    \\
\glt
‘I said “did you vomit?”’
  }
\newpage   
  \ex{
ab03ab 109\\
\gll
E  sé  \textit{“sí,}  \textit{abuela}  \textit{yo}  \textit{siento}  \textit{a}  \textit{vomitar.”}   \\
\textsc{3sg.sbj}  \textsc{quot}  yes  grandmother  I  I.feel  to  vomit    \\
\glt
‘He said “yes, grandmother I feel like vomiting.”
  }\ex{
ab03ab 110\\
\gll
A  sé  “Tokobé  kán  ó!”\\
\textsc{1sg.sbj}  \textsc{quot}  \textsc{name}  come  \textsc{sp}    \\
\glt
‘I said “Tokobé come, please!”’
  }\ex{
ab03ab 111\\
\gll
Tokobé  dɔ́n  wɛ́r  klós  gbogbogbo    “nó  fɔ   \\
\textsc{name}  \textsc{prf}  wear  clothing  \textsc{ideo}      \textsc{neg}  \textsc{prep}    \\
\gll
fɛ́n  \textit{cuaderno}”.\\
look.for  exercise.book    \\
\glt
‘Tokobé had already worn her clothes in a rush, “no we have to look for the patient’s logbook”.’
  }\ex{
ab03ab 112\\
\gll
E  mít  wán  ól  ól  \textit{cuaderno},  di  tɛ́n  fɔ   \\
\textsc{3sg.sbj}  meet  one  old  \textsc{rep}  exercise.book  \textsc{def}  time  \textsc{prep}    \\
\gll
Niumbɛ,  na  ín  e  bin  rɔ́n  wet=an  ɔspítul.   \\
\textsc{name}  \textsc{foc}  \textsc{3sg.indp}  \textsc{3sg.sbj}  \textsc{pst}  run  with=\textsc{3sg.obj}  hospital    \\
\glt
‘She found a very old patient’s book, from the time of Niumbe, that’s when she ran off to the hospital with him.’
  }\ex{
ab03ab 113\\
\gll
Dɛn  gó  na  ɔspítul.   \\
\textsc{3pl}  go  \textsc{loc}  hospital    \\
\glt
‘They went to the hospital.’
  }\ex{
ab03ab 114\\
\gll
Sé  nɔ́  bifó  di  dɔ́kta  de  kán  wé  a  de  kán   \\
\textsc{quot}  \textsc{neg}  before  \textsc{def}  doctor  \textsc{ipfv}  come  \textsc{sub}  \textsc{1sg.sbj}  \textsc{ipfv}  come    \\
\gll
fɔ́s.   \\
first    \\
\glt
‘Then before the doctor was coming I was already coming first [had come to the hospital from home as well].’
  }\ex{
ab03ab 115\\
\gll
Dɛn  gó  sé  \textit{análisis.}   \\
\textsc{3pl}  go  \textsc{quot}  analysis    \\
\glt
‘They went for an analysis.’
  }\ex{
ab03ab 116\\
\gll
Dɛn  rɔ́n  gó  mék  \textit{análisis.}   \\
\textsc{3pl}  run  go  make  analysis    \\
\glt
‘The rushed off to make an analysis.’
  }
\newpage 
  \ex{
ab03ab 117\\
\gll
Lɛk  háw  e  de  bríng  di  \textit{análisis,}  wi  sí  di   \\
like  how  \textsc{3sg.sbj}  \textsc{ipfv}  bring  \textsc{def}  analysis  \textsc{1pl}  see  \textsc{def}    \\
\gll
dɔ́kta  dɔ́n  de  kán,  \textit{ya}  \textit{era}  \textit{la}  \textit{una}  \textit{y}  \textit{algo.}   \\
doctor  \textsc{prf}  \textsc{ipfv}  come  here  was  \textsc{def}  one  and  something    \\
\glt
‘As soon she [Tokobé] brought the analysis, we saw the doctor coming (when) it was already past one o’clock.’
  }\ex{
ab03ab 118\\
\gll
\textit{Paciente}  dɛn  dé  na  sala,  yú  dɔ́kta  \textit{“la}  \textit{una}  yu   \\
patient  \textsc{pl}  \textsc{be.loc}  \textsc{loc}  hall  \textsc{2sg.indp}  doctor  \textsc{def}  one  \textsc{2sg}    \\
\gll
de  kán?”\\
\textsc{ipfv}  come    \\
\glt
‘Patients are in the waiting room, (and) you doctor, “you’re [only] coming at one o’clock?”’
  }\ex{
ab03ab 119\\
\gll
Dɔ́kta  dɛn  nó  dé  na  dís  kɔ́ntri  na  mék  pípul  dɛn   \\
doctor  \textsc{pl}  \textsc{neg}  \textsc{cop} \textsc{loc}  this  country  \textsc{foc}  make  people  \textsc{pl}    \\
\gll
de  dáy  plɛ́nte.   \\
\textsc{ipfv}  die  plenty    \\
\glt
‘There are no doctors in this country, that’s what’s making people die a lot.’
  }\ex{
ab03ab 120\\
\gll
Wántɛn  wé  e  lúk  di  pikín,  e  lúk  di   \\
at.once  \textsc{sub}  \textsc{3sg.sbj}  look  \textsc{def}  child  \textsc{3sg.sbj}  look  \textsc{def}    \\
\gll
\textit{análisis,}  \textit{“tiene}  \textit{paludismo}  \textit{de}  \textit{una}  \textit{cruz}  wé  kin  kíl   \\
analysis  he.has  malaria  of  one  cross  \textsc{sub}  \textsc{hab}  kill    \\
\gll
pikín  sɛ́f.”\\
child  \textsc{foc}    \\
\glt
‘At once, when he looked at the child, he looked at the analysis, “he has malaria of one cross [degree of intensity] that can even kill a child”.’
  }\ex{
ab03ab 121\\
\gll
Yu  de  mɛ́mba  sé  e  de  slíp.   \\
\textsc{2sg}  \textsc{ipfv}  remember  \textsc{quot}  \textsc{3sg.sbj}  \textsc{ipfv}  sleep    \\
\glt
‘You would think that he [the boy] was sleeping.’
  }\ex{
ab03ab 122\\
\gll
Dɛn  gí=an  mɛ́rɛsin.   \\
\textsc{3pl}  give=\textsc{3sg.obj}  medicine    \\
\glt
‘He was given medicine.’
  }\ex{
ab03ab 123\\
\gll
Dɛn  rɔ́n  na  \textit{farmacia,}  \textit{receta}  \textit{de}  \textit{mɛ́rɛsin}.   \\
\textsc{3pl}  run  \textsc{loc}  pharmacy  prescription  of  medicine    \\
\glt
‘They rushed to the pharmacy [to get a] prescription.’
  }\ex{
ab03ab 124\\
\gll
Dɛn  bin  gí=an  di  \textit{receta}  fɔ  kán  báy=an.   \\
\textsc{3pl}  \textsc{pst}  give=\textsc{3sg.obj}  \textsc{def}  prescription  \textsc{prep}  come  buy=\textsc{3sg.obj}    \\
\glt
‘They had given her [Tokobé] the prescription in order to come buy it.’
  }\ex{
ab03ab 125\\
\gll
Sé  mɔ́mi,  e  sé  \textit{“siento}  \textit{hambre.”}\\
\textsc{quot}  mum  \textsc{3sg.sbj}  \textsc{quot}    I.feel  hunger    \\
\glt
‘(He) said, “mum, I feel hungry”.’
  }\ex{
ab03ab 126\\
\gll
Mɔ́mi,  gó  báy  tú  \textit{bocadillo!}   \\
mum  go  buy  two  bun    \\
\glt
[I told his mum] ‘Mum, go buy two buns!’
  }\ex{
ab03ab 127\\
\gll
Tú  brɛ́d.   \\
two  bread    \\
\glt
‘Two (loaves) of bread.’
  }\ex{
ab03ab 128\\
\gll
Yu  pikín  sidɔ́n  de  chɔ́p  dɛn  tú  brɛ́d.   \\
\textsc{2sg}  child  sit  \textsc{ipfv}  eat  \textsc{3pl}  two  bread    \\
\glt
‘Your child [directed at the listener [fr]] was sitting (there) eating those two (loaves of) bread.’
  }\ex{
ab03ab 129\\
\gll
E  sé  “a  wánt  Fanta”.\\
\textsc{3sg.sbj}  \textsc{quot}  \textsc{1sg.sbj}  want  \textsc{name}    \\
\glt
‘He said “I want Fanta”.’
  }\ex{
ab03ab 130\\
\gll
Dís  smɔ́l  bɔ́tul  dɛn  Fanta,  wé  e  gɛ́t  Coca-Cola,   \\
this  small  bottle  \textsc{pl}  \textsc{name}  \textsc{sub}  \textsc{3sg.sbj}  get  \textsc{name}    \\
\gll
e  gɛ́t  Fanta,  a  gɛ́t  \textit{limón,}  e  báy=an   \\
\textsc{3sg.sbj}  get  \textsc{name}  \textsc{3sg.sbj}  get  lemon  \textsc{3sg.sbj}  buy=\textsc{3sg.obj}    \\
\gll
wán.   \\
one    \\
\glt
‘These small bottles of Fanta, of which there is (also) Coca-Cola, there is Fanta, there is lemon, she bought one for him.’
  }\ex{
ab03ab 131\\
\gll
E  nák=an.   \\
\textsc{3sg.sbj}  hit=\textsc{3sg.obj}    \\
\glt
‘He gulped it down.’
  }\ex{
ab03ab 132\\
\gll
Náw  e  dɔ́n  wánt  bigín  de  fɛ́t  wet  di  chía,  di   \\
now  \textsc{3sg.sbj}  \textsc{prf}  want  begin  \textsc{ipfv}  fight  with  \textsc{def}  chair  \textsc{def}    \\
\gll
say  wé  dɛn  sidɔ́n.   \\
side  \textsc{sub}  \textsc{3pl}  sit    \\
\glt
‘Now he wanted to begin fighting with the chair, where they were sitting [due to his delirium].’
  }
\newpage
  \ex{
ab03ab 133\\
\gll
Sé  “nɔ́,  dɔ́kta  wi  dɔ́n  fít  gó?”\\
\textsc{quot}  \textsc{neg}  doctor  \textsc{1pl}  \textsc{prf}  can  go    \\
\glt
‘(We) said “doctor, can we go now?”’
  }\ex{
ab03ab 134\\
\gll
E  sé  “una  dɔ́n  fít  gó.”\\
\textsc{3sg.sbj}  \textsc{quot}  \textsc{2pl}  \textsc{prf}  can  go    \\
\glt
‘He said “you can already go now”.’
}\z

\section{Narrative and conversation: Annobón sorcery}

The following text begins with a conversation between Francisca (fr), Rubi (ru), and Djunais (dj) in which (fr) tries to persuade (ru) to give an account of how he was bewitched. Speaker (fr) manages to coax (ru) into telling the story by jokingly threatening to report to the police (015) and to bring the matter into the Equatoguinean reality TV show “Vivencias” (016)–(017). Speaker (ru) then relates in (018)–(044) how he was bewitched by a fling of his from the island of Annobón, which has caused him to fall sick with fever. The protagonists are (ru), (dj), and (ru)’s fling “the girl from Annobón”. In the remainder of the text (057ff.), (fr) tries to convince (ru) and (dj) of the importance of malaria prevention.

\setcounter{equation}{0}  % RJK the author numbers from 1 per text

\ea{ru03wt 001\\
\gll
Wán  Annobón  gɛ́l  wích  mí  mán.   \\
one  \textsc{place}  girl  bewitch  \textsc{1sg.indp}  \textsc{intj}    \\
\glt
‘A girl from Annobón bewitched me, man.’
  }\ex{
fr03wt 002\\
\gll
Na  wán  Annobón  gɛ́l  wích  yú?   \\
\textsc{foc}  one  \textsc{place}  girl  bewitch  \textsc{2sg.indp}    \\
\glt
‘It’s a girl from Annobón that bewitched you?’
  }\ex{
fr03wt 003\\
\gll
Na  fɔ  dán  tín  mék  yu  gó  dɔ́kta.   \\
\textsc{foc}  \textsc{prep}  that  thing  \textsc{sbjv}  \textsc{2sg}  go  doctor    \\
\glt
‘That’s why you should go to the doctor.’
  }\ex{
fr03wt 004\\
\gll
\'{U}dat  tɛ́l  yú  sé  e  wích  yú?   \\
who  tell  \textsc{2sg.indp}  \textsc{quot}  \textsc{3sg.sbj}  bewitch  \textsc{2sg.indp}    \\
\glt
‘Who told you that she bewitched you?’
  }
  \newpage
\ex{
fr03wt 005\\
\gll
Na  torí  a  de  hía  ó!   \\
\textsc{foc}  story  \textsc{1sg.sbj}  \textsc{ipfv}  hear  \textsc{sp}    \\
\glt
‘I’m hearing the story [come on let’s hear the story]!’
  }\ex{
fr03wt 006\\
\gll
Yu  sabí  ús=káyn  tín  na  wích  nɔ́?   \\
\textsc{2sg}  know  \textsc{q}=kind    thing  \textsc{foc}  bewitch  \textsc{neg}    \\
\glt
‘You know what sorcery is, right?
  }\ex{
fr03wt 007\\
\gll
Annobón?   \\
\textsc{place}    \\
\glt
‘(And) Annobón?’
  }\ex{
fr03wt 008\\
\gll
Yu  sabí  ús=tin      na  Annobón    sɛ́f.   \\
\textsc{2sg}  know  \textsc{q}=thing  \textsc{foc}  \textsc{place}          \textsc{foc}    \\
\glt
‘You even know what Annobón is.’
  }\ex{
ko03ft 009\\
\gll
Yu  fɔgɛ́t  sé  a  dɔ́n  gó  dé.   \\
\textsc{2sg}  forget  \textsc{quot}  \textsc{1sg.sbj}  \textsc{prf}  go  there    \\
\glt
‘You forgot that I had already gone there.’
  }\ex{
fr03wt 010\\
\gll
E  bin  dé  na  Annobón  yɛ́stadé.   \\
\textsc{3sg.sbj}  \textsc{pst}  there  \textsc{loc}  \textsc{place}  yesterday    \\
\glt
‘He was in Annobón yesterday.’
  }\ex{
fr03wt 011\\
\gll
Djunais,  na  ín  mék  sé  mék  dɛn  wích=an.   \\
\textsc{name}  \textsc{foc}  \textsc{3sg.indp}  make  \textsc{quot}  \textsc{sbjv}  \textsc{3pl}  bewitch=\textsc{3sg.obj}    \\
\glt
‘(It’s) Djunais, it’s him who made them bewitch him.’
  }\ex{
dj03wt 012\\
\gll
Nó  \textit{mete}  mí  ínsay  dí  tɔ́k  a  bɛ́g!   \\
\textsc{neg}  put  \textsc{1sg.indp}  inside  this  talk  \textsc{1sg.sbj}  ask.for    \\
\glt
‘Don’t involve me in this matter, please!’
  }\ex{
ru03wt 013\\
\gll
Na  yú  mék=an.   \\
\textsc{foc}  \textsc{2sg.indp}  make=\textsc{3sg.obj}    \\
\glt
‘It’s you who made it [laughter].’
  }
\newpage
\ex{
fr03wt 014\\
\gll
\'{U}s=káyn  tín  e  mék?   \\
\textsc{q}=kind    thing  \textsc{3sg.sbj}  make    \\
\glt
‘What did he do?’
  }\ex{
fr03wt 015\\
\gll
If  mí  kɛ́r  dís  plába  náw,  ɛ́n,  na  \textit{comisaría,}   \\
if  \textsc{1sg.indp}  carry  this  trouble  now  \textsc{intj}  \textsc{loc}  police.station    \\
\gll
una  sabí  sé  dɛn  nó  lɛ́k  dís  tin,  nó  nátin  fɔ   \\
\textsc{2pl}  know  \textsc{quot}  \textsc{3pl}  \textsc{neg}  like  this  thing  \textsc{neg}  nothing  \textsc{prep}    \\
\gll
wích,  dí  wán  go  tɔ́n  plába  \textit{serio}.   \\
bewitch  this  one  \textsc{pot}  turn  trouble  serious    \\
\glt
‘If I take this matter, right, to the police-station, you [\textsc{pl}] know that they don’t like this thing, nothing concerning sorcery, this would turn into serious trouble.’
  }\ex{
fr03wt 016\\
\gll
\'{A}fta  dɛn  go  kɛ́r  una,  na  Vivencias  fɔ,  ús=wán   \\
then  \textsc{3pl}  \textsc{pot}  carry  \textsc{2pl}  \textsc{loc}  \textsc{name}  \textsc{prep}  \textsc{q}=one    \\
\gll
na  in  ném?   \\
\textsc{foc}  \textsc{3sg.poss}  name    \\
\glt
‘Then they’d take you [\textsc{pl}] to Vivencias” to, what’s his name?’
  }\ex{
fr03wt 017\\
\gll
Fɔ  Olinga,  wé  e  go  gó  chám  in  Panyá  dé.   \\
\textsc{prep}  \textsc{name}  \textsc{sub}  \textsc{3sg.sbj}  \textsc{pot}  go  chew  \textsc{3sg.poss}  Spanish  \textsc{be.loc}    \\
\glt
‘To Olinga and he would go speak his bad Spanish there.’
  }\ex{
fr03wt 018\\
\gll
Una  púl  di  torí!   \\
\textsc{2pl}  remove  \textsc{def}  story    \\
\glt
‘Tell [\textsc{pl}] the story!’
  }\ex{
ru03wt 019\\
\gll
E  dé  sé  dán  gɛ́l  e  bin  de  kán  yá.   \\
\textsc{3sg.sbj}  \textsc{be.loc}  \textsc{quot}  that  girl  \textsc{3sg.sbj}  \textsc{pst}  \textsc{ipfv}  come  here    \\
\glt
‘It’s that that girl used to come here.’
  }\ex{
ru03wt 020\\
\gll
Mí  nó  bin  de  lúk=an  ó.   \\
\textsc{1sg.indp}  \textsc{neg}  \textsc{pst}  \textsc{ipfv}  look=\textsc{3sg.obj}  \textsc{sp}    \\
\glt
‘Mind you, I [\textsc{emp}] wasn’t looking at [paying attention to] her.’
  }\ex{
ru03wt 021\\
\gll
Djunais  tɔ́k  sé,  nɔ́  Rubi  dí  gɛ́l  lɛ́k  yú,  dí   \\
\textsc{name}  talk  \textsc{quot}  \textsc{neg}  \textsc{name}  this  girl  like  \textsc{2sg.indp}  this    \\
\gll
gɛ́l  lɛ́k  yú,  náw  bigín  mék=an  só.   \\
girl  like  \textsc{2sg.indp}  now  begin  make=\textsc{3sg.obj}  like.that    \\
\glt
‘Djunais said, no Rubi, this girl likes you, this girl likes you, now begin doing it like this.’
  }\ex{
ru03wt 022\\
\gll
Nó  láf!   \\
\textsc{neg}  laugh    \\
\glt
‘Don’t laugh!’
  }\ex{
fr03wt 023\\
\gll
Djunais,  nó  láf!   \\
\textsc{name}  \textsc{neg}  laugh    \\
\glt
‘Djunais, don’t laugh!’
  }\ex{
ru03wt 024\\
\gll
Dán  tɛ́n  a  dé  fáyn.   \\
that  time  \textsc{1sg.sbj}  \textsc{be.loc}  fine    \\
\glt
‘That time I was fine.’
  }\ex{
ru03wt 025\\
\gll
A  gó,  a  lúk  di  gɛ́l,  wi  bigín  tɔ́k,  wi  bigín   \\
\textsc{1sg.sbj}  go  \textsc{1sg.sbj}  look  \textsc{def}  girl  \textsc{1pl}  begin  talk  \textsc{1pl}  begin    \\
\gll
tɔ́k,  wi  bigín  tɔ́k  \textit{tal}  \textit{tal}.   \\
talk  \textsc{1pl}  begin  talk  so  so    \\
\glt
‘I went, I had a look at the girl, we began to talk and talk and talk, and so on.’
  }\ex{
ru03wt 026\\
\gll
Tumɔ́ro  di  gɛ́l  wánt  sé  mék  wi  slíp.   \\
tomorrow  \textsc{def}  girl  want  \textsc{quot}  \textsc{sbjv}  \textsc{1pl}  sleep    \\
\glt
‘The next day the girl wanted us to sleep (with each other).’
  }\ex{
ru03wt 027\\
\gll
E  \textit{insiste}  sóté  \upshape{[click]}.   \\
\textsc{3sg.sbj}  insist  until      \\
\glt
‘She insisted until [clicks with his fingers].’
  }\ex{
fr03wt 028\\
\gll
Una  slíp?   \\
\textsc{2pl}  sleep    \\
\glt
‘You slept (with each other)?’
  }\ex{
ru03wt 029\\
\gll
Yɛ,  a  kán  tɛ́l=an  sé  ‘\textit{chica,}  mí  nó  lɛ́k   \\
yes  \textsc{1sg.sbj}  \textsc{pfv}  tell=\textsc{3sg.obj}  \textsc{quot}    girl  \textsc{1sg.indp}  \textsc{neg}  like    \\
\gll
yú  bɔt  wi  fít  dé    lɛk  kɔ́mpin’.   \\
\textsc{2sg.indp}  but  \textsc{1pl}  can  \textsc{be.loc}  like  friend    \\
\glt
‘Yeah, I eventually told her “girl, I [\textsc{emp}] don’t love you but we can be like friends”.’
  }
\newpage
  \ex{
ru03wt 030\\
\gll
“A  wɔ́nt  mék  yu  dú  mí  sɔn  fébɔ,  mék  yu   \\
\phantom{“}\textsc{1sg.sbj}  want  \textsc{sbjv}  \textsc{2sg}  do  \textsc{1sg.indp}  some  favour  \textsc{sbjv}  \textsc{2sg}    \\
\gll
wás  mí  sɔn  klós  dɛn.”\\
wash  \textsc{1sg.indp}  some  clothing  \textsc{pl}    \\
\glt
‘“I want you to do me a favour and wash some clothes for me.”’
  }\ex{
fr03wt 031\\
\gll
Ɔ́l  dán  tɛ́n  Djunais  de  gív=an  di  \textit{acción,}   \\
all  that  time  \textsc{name}  \textsc{ipfv}  give=\textsc{3sg.obj}  \textsc{def}  action    \\
\gll
e  de  pút  \textit{calor.}   \\
\textsc{3sg.sbj}  \textsc{ipfv}  put  heat    \\
\glt
‘All that time Djunais was causing commotion, he was fanning the flames.’
  }\ex{
fr03wt 032\\
\gll
Djunais  yu  badhát  ɛ́n.   \\
\textsc{name}  \textsc{2sg}  be.mean  \textsc{intj}    \\
\glt
‘Djunais, you’re mean, you know.’
  }\ex{
ru03wt 033\\
\gll
E  gó,  e  wás  di  klós  dɛn.   \\
\textsc{3sg.sbj}  go  \textsc{3sg.sbj}  wash  \textsc{def}  clothing  \textsc{pl}    \\
\glt
‘She went (and) she washed the clothes.’
  }\ex{
ru03wt 034\\
\gll
E  wás  di  klós  dɛn,  e  dráy  dɛ́n,  nɔ́,  na   \\
\textsc{3sg.sbj}  wash  \textsc{def}  clothing     \textsc{pl}  \textsc{3sg.sbj}  dry  \textsc{3pl.indp}  \textsc{neg}  \textsc{foc}    \\
\gll
mí  dráy  dɛ́n.   \\
\textsc{1sg.indp}  dry  \textsc{3pl.indp}    \\
\glt
‘She washed the clothes, she dried them, no, it was me who dried them.’
  }\ex{
ru03wt 035\\
\gll
\textit{Pero}  di  klós  dɛn  slíp  na  dɔ́n  ó.   \\
but  \textsc{def}  clothing  \textsc{pl}  lie  \textsc{loc}  down  \textsc{sp}    \\
\glt
‘But the clothes came to lie down [on the ground].’
  }\ex{
ru03wt 036\\
\gll
Mɔ́nin  tɛ́n  wé  a  kán  lúk  a  de  sí  sɔn   \\
morning  time  \textsc{sub}  \textsc{1sg.sbj}  \textsc{pfv}  look  \textsc{1sg.sbj}  \textsc{ipfv}  see  some    \\
\gll
klós  dɛn,  a  nó  de  sí  mi  yón   dɛn.   \\
clothing  \textsc{pl}  \textsc{1sg.sbj}  \textsc{neg}  \textsc{ipfv}  see  \textsc{1sg.poss}  own  \textsc{pl}    \\
\glt
‘In the morning, when I came to look, I saw some clothes (but) I didn’t see mine.’
  }\ex{
ru03wt 037\\
\gll
\'{A}fta  a  de  mít=an  nía  di  klós  dɛn  di   \\
then  \textsc{1sg.sbj}  \textsc{ipfv}  meet=\textsc{3sg.obj}  near  \textsc{def}  clothing     \textsc{pl}  \textsc{def}    \\
\gll
mɔ́nin  mɔ́nin  tɛ́n.   \\
morning  \textsc{rep}  time    \\
\glt
‘Then I find her next to the clothes early in the morning.’
  }\ex{
ru03wt 038\\
\gll
A  áks=an  sé  “ús=say  di  klós  dɛn  dé?”\\
\textsc{1sg.sbj}  ask=\textsc{3sg.obj}  \textsc{quot}    \textsc{q}=side    \textsc{def}  clothing  \textsc{pl}  \textsc{be.loc}    \\
\glt
‘I asked her “where are the clothes?”’
  }\ex{
ru03wt 039\\
\gll
E  sé  “nó,  a  de  sí  lɛk  sé  dɛn  dɔ́n  tíf   \\
\textsc{3sg.sbj}  \textsc{quot}   \textsc{neg}  \textsc{1sg.sbj}  \textsc{ipfv}  see  like  \textsc{quot}  \textsc{3pl}  \textsc{prf}  steal    \\
\gll
sɔn”.\\
some    \\
\glt
‘She said “no, it seems to me like some have been stolen”.’
  }\ex{
ru03wt 040\\
\gll
\'{U}s=say  mi  klós  dɛn  dé,    di  ívin  tɛ́n,   \\
\textsc{q}=side    \textsc{1sg.poss}  clothing  \textsc{pl}  \textsc{be.loc}  \textsc{def}  evening  time    \\
\gll
\upshape{[click]}  fíba,  fíba  sóté  a  kɔ́l=an.   \\
{}  fever  fever  until  \textsc{1sg.sbj}  call=\textsc{3sg.obj}    \\
\glt
‘Where were my clothes, in the evening [clicks with his fingers], fever, fever until finally I called her.’
  }\ex{
ru03wt 041\\
\gll
E  dé,  e  nó  de  ánsa  mí  mɔ́,  e   \\
\textsc{3sg.sbj}  \textsc{be.loc}  \textsc{3sg.sbj}  \textsc{neg}  \textsc{ipfv}  answer  \textsc{1sg.indp}  more  \textsc{3sg.sbj}    \\
\gll
de  pás  só  lɛk  sé  e  nó  nó  mí  mɔ́.   \\
\textsc{ipfv}  pass  so  like  \textsc{quot}  \textsc{3sg.sbj}  \textsc{neg}  know  \textsc{1sg.indp}  more    \\
\glt
‘She was there and wasn’t responding to me anymore, she was passing by as if she didn’t know me anymore.’
  }\ex{
ru03wt 042\\
\gll
A  tɛ́l=an  sé  “\textit{chica,}  sóté  yu  de  kán  na   \\
\textsc{1sg.sbj}  tell=\textsc{3sg.obj}  \textsc{quot}  girl  until  \textsc{2sg}  \textsc{ipfv}  come  \textsc{loc}    \\
\gll
mi  drim  dɛn  ɛ́n,  na  só      só  tín  yu  mék   \\
\textsc{1sg.poss}  dream  \textsc{pl}  \textsc{intj}  \textsc{foc}  so  \textsc{rep}  thing  \textsc{2sg}  make    \\
\gll
mí,  tráy  \textit{reduce}  ín”.\\
\textsc{1sg.indp}  try  reduce  \textsc{3sg.indp}    \\
\glt
‘I told her “girl, you even come into my dreams, you know, it’s this and that you did to me, try to reduce that”.’
  }\ex{
ru03wt 043\\
\textit{“¿Tú} \textit{piensas}  \textit{eso}  \textit{de}  \textit{mí?”}\\
\glt
‘[She replied] “You think that of me?”’
  }\ex{
ru03wt 044\\
\gll
A  dɔ́n  \textit{explica}  Boyé  dɛn,  sé  na  só  mí  de   \\
\textsc{1sg.sbj}  \textsc{prf}  explain  \textsc{name}  \textsc{pl}  \textsc{quot}  \textsc{foc}  so  \textsc{1sg.indp}  \textsc{ipfv}    \\
\gll
mɛ́mba,  ɔ́l  tín.   \\
remember  all  thing    \\
\glt
‘I’ve already explained to Boyé and the others, that’s how I remember everything.’
  }
\newpage
\ex{
fr03wt 045\\
\gll
Yu  dɔ́n  gó  sí  yu  mamá?   \\
\textsc{2sg}  \textsc{prf}  go  see  \textsc{2sg}  mother    \\
\glt
‘Have already gone to see your mother?’
  }\ex{
ru03wt 046\\
\gll
Nɔ́.   \\
\textsc{neg}    \\
\glt
‘No.’
  }\ex{
fr03wt 047\\
\gll
Wétin  yu  de  wét?   \\
what  \textsc{2sg}  \textsc{ipfv}  wait    \\
\glt
‘What are you waiting (for)?’
  }\ex{
dj03wt 048\\
\gll
Sé  in  mamá  go  dráyb=an  fɔ́s.   \\
\textsc{quot}  \textsc{3sg.poss}  mother  \textsc{pot}  drive=\textsc{3sg.obj}  first    \\
\glt
‘Because his mother would chase him away first.’
  }\ex{
fr03wt 049\\
\gll
In  mamá  go  dráyb=an  fɔ́s  \textit{pero}  in   \\
\textsc{3sg.poss}  mother  \textsc{pot}  drive=\textsc{3sg.obj}  first  but  \textsc{3sg.poss}    \\
\gll
mamá  na  di  ónli  pɔ́sin  wé  e  fít  gó  wáka   \\
mother  \textsc{foc}  \textsc{def}  only  person  \textsc{sub}  \textsc{3sg.sbj}  can  go  walk    \\
\gll
wet=an,  mí  nó  sabí  wáka.   \\
with=\textsc{3sg.obj}  \textsc{1sg.indp}  \textsc{neg}  know  walk    \\
\glt
‘His mother could chase him away first but his mother is the only person that could go walk with him [i.e. take care of his spiritual protection], I don’t know how to walk.’
  }\ex{
ru03wt 050\\
\gll
Annobón  mɛ́rɛsin  nó  de  tɔ́n  mi  héd.   \\
\textsc{place}  sorcery  \textsc{neg}  \textsc{ipfv}  turn  \textsc{1sg.poss}  head    \\
\glt
‘Annobón sorcery doesn’t turn my head [have an effect on me].’
  }\ex{
dj03wt 051\\
\gll
Annobón  mɛ́rɛsin,  e  nó  de  gó  bihɛ́n.   \\
\textsc{place}  sorcery  \textsc{3sg.sbj} \textsc{neg}  \textsc{ipfv}  go  behind    \\
\glt
‘As for Annobón sorcery, it doesn’t go behind [have a profound effect].’
  }\ex{
fr03wt 052\\
\gll
\textit{No obstante},  a  bɛ́g  gó  sí  dɔ́kta  fɔ́s,  hía?   \\
nonetheless  \textsc{1sg.sbj}  ask.for  go  see  doctor  first  hear    \\
\glt
‘Nonetheless, please go see the doctor first, (you) hear?’
  }
\newpage
\ex{
fr03wt 053\\
\gll
Na  fɔ  dán  tín  yu  nó  de  gó  dɔ́kta  \textit{porque}  yu   \\
\textsc{foc}  \textsc{prep}  that thing  \textsc{2sg}  \textsc{neg}  \textsc{ipfv}  go  doctor  because  \textsc{2sg}    \\
\gll
de  chɛ́k  sé  na  wích?   \\
\textsc{ipfv}  think  \textsc{quot}  \textsc{foc}  bewitch    \\
\glt
‘Is that why you’re not going to the doctor because you think it’s witchcraft?’
  }\ex{
fr03wt 054\\
\gll
Gó  dɔ́kta  fɔ́s,  wé  di  dɔ́kta  go  gí  yu  sɔn  tín   \\
go  doctor  first  \textsc{sub}  \textsc{def}  doctor  \textsc{pot}  give  \textsc{2sg}  some  thing    \\
\gll
mék  yu  fíl  smɔ́l  fáyn,  yu  bigín  mék  di  ɔ́da  tín   \\
\textsc{sbjv}  \textsc{2sg}  feel  a.bit  fine  \textsc{2sg}  begin  make  \textsc{def}  other  thing    \\
\gll
dɛn.   \\
\textsc{pl}    \\
\glt
‘Go to the doctor first, when the doctor will give you something for you to feel a fine a bit, you begin to do the other things.’
  }\ex{
fr03wt 055\\
\gll
Yú  de  hía?   \\
\textsc{2sg.indp}  \textsc{ipfv}  hear    \\
\glt
‘Do you hear?’
  }\ex{
ru03wt 056\\
\gll
A  hía.   \\
\textsc{1sg.sbj}  hear    \\
\glt
‘I hear.’
  }\ex{
fr03wt 057\\
\gll
E  fít  bí  sé  na  \textit{paludismo}.   \\
\textsc{3sg.sbj}  can  \textsc{be}  \textsc{quot}  \textsc{foc}  malaria    \\
\glt
‘It could be that it’s malaria.’
  }\ex{
fr03wt 058\\
\gll
\'{U}s=tɛ́n     una  lás  \textit{impregna}     \textit{una}   \textit{mosquiteros}  dɛn?   \\
\textsc{q}=time  \textsc{2pl}  be.last  impregnate  \textsc{2pl}     mosquito.nets  \textsc{pl}    \\
\glt
‘When did you [\textsc{pl}] last impregnate your [\textsc{pl}] mosquito nets?’
  }\ex{
fr03wt 059\\
\gll
E  dɔ́n  sté,  a  tínk  sé  e  dɔ́n  sté  wé   \\
\textsc{3sg.sbj}  \textsc{prf}  last  \textsc{1sg.sbj}  think  \textsc{quot}  \textsc{3sg.sbj}  \textsc{prf}  last  \textsc{sub}    \\
\gll
una  bin  gɛ́t  \textit{insecticida}  yá.   \\
\textsc{2pl}  \textsc{pst}  get  insecticide  here    \\
\glt
‘It’s been a long time, I think that it’s been a long time that you had insecticide here.’
  }
\newpage
\ex{
fr03wt 060\\
\gll
Dán  bíg  bíg  \textit{mosquito}  dɛn  wé  dɛn  fíba  \textit{aviones}  dɛn.   \\
that  big  \textsc{rep}  mosquito  \textsc{pl}  \textsc{sub}  \textsc{3pl}  resemble  plane.\textsc{pl}  \textsc{pl}    \\
\glt
‘Those huge mosquitos that resemble airplanes.’
  }\ex{
fr03wt 061\\
\gll
\textit{Aunque}  nóto  \textit{paludismo},  if  dɛn  gív  yú   \\
even.if  \textsc{neg}.\textsc{foc}  malaria  if  \textsc{3pl}  give  \textsc{2sg.indp}    \\
\gll
\textit{tratamiento}  yu  nó  go  dáy.   \\
treatment  \textsc{2sg}  \textsc{neg}  \textsc{pot}  die    \\
\glt
‘Even if it’s not malaria, if they give you a treatment you won’t die.’
}\z

\section{Conversation: Dinner for four}

The text that follows is an extensive conversation involving four people: Boyé (ye), Djunais (dj), Francisca (fr), and sporadically myself (ko). The conversation was recorded during a dinner hosted by (fr). A relaxed and cheerful atmosphere reigns during the conversation and the discourse participants, who are members of the same extended family, joke and tease each other on numerous occasions (e.g. in (015)–(019), (091)–(94) and the entire section from (130)–(143)). The conversation also contains many instances of Pichi-Spanish codemixing (e.g. (001)–(008)).


The text features three themes between which the speakers switch to and fro. The main theme is the ongoing construction of a family house commissioned by (fr) and overseen by (ye). This discussion is contained in sections (001)–(038), (99)–(120), (154)–(164), and (173)–(178) and is chiefly concerned with problems in a cement delivery ordered from two protagonists named Buehu and Gabriel. The sections on the construction works are driven by (fr), who repeatedly brings the conversation topic back to this issue of great importance to her.



A second theme revolves around eating. In (080)–(097), (dj) and (ye) comment on each other’s cooking abilities, in (121)–(127), an exchange ensues about the effect of the pepper in the food, and in (132)–(143), (ye) teases (dj) because the latter has just drunk tap water (which is not without risk in Malabo). In (144)–(153) and (164)–(172), both (dj) and (ye) complain about the eating habits of Pancho (pa) who is not present at the table. Both (dj) and (ye) live in one place with (pa) and the account of (ye) in (173)–(178) shows that (pa) was also supposed to run an errand for (fr) as part of the building activities. A third theme is the interlude in (051)–(078) in which (fr) and (ye) scoff at Olinga, the TV presenter of “Vivencias”, a popular Equaotoguinean TV reality show.

\setcounter{equation}{0}  % RJK the author numbers from 1 per text

\ea{ye03cd 001\\
\gll
\textit{Pues}  \textit{hemos}  \textit{estado}  \textit{ahí,}  a  tínk  sé  wán  las   \\
so  we.have  been  there  \textsc{1sg.sbj}  think  \textsc{quot}  one  the.\textsc{pl}    \\
\gll
\textit{cuatro}  wé  di  chɛ́f  kɔmɔ́t  e  nó  \textit{aparece}  yet.   \\
four  \textsc{sub}  \textsc{def}  boss  go.out  \textsc{3sg.sbj}  \textsc{neg}  appear  yet    \\
\glt
‘So we were there, I think around four o’clock that the boss went out (and) he hadn’t appeared yet.’
  }\ex{
ye03cd 002\\
\gll
Di  ɔ́da  mán  tɛ́l  mí  sé  dɛn  dɔ́n  báy  \textit{veinte}   \\
\textsc{def}  other  man  tell  \textsc{1sg.indp}  \textsc{quot}  \textsc{3pl}  \textsc{prf}  buy  twenty    \\
\gll
\textit{sacos}.   \\
bags    \\
\glt
‘The other man told me that they had bought twenty bags.’
  }\ex{
ye03cd 003\\
\gll
E  lɛ́f  \textit{doce}.   \\
\textsc{3sg.sbj}  remain  twelve    \\
\glt
‘Twelve remain.’
  }\ex{
ye03cd 004\\
\gll
E  \textit{falta}  mɔní  fɔ  púl  \textit{saco}  dɛn  dé  fɔ   \\
\textsc{3sg.sbj}  lack  money  \textsc{prep}  remove  sack  \textsc{pl}  there  \textsc{prep}    \\
\gll
kɛ́r=an  na  hós.   \\
carry=\textsc{3sg.obj}  \textsc{loc}  house    \\
\glt
‘The money is lacking to remove the bags there in order to bring them to the house.’
  }\ex{
fr03cd 005\\
\gll
\textit{Me}  \textit{van}  \textit{a}  \textit{tocar}  \textit{los}       \textit{cojones}  \textit{porque}  mí  gí   \\
me  they.will  to  touch  the\textsc{.pl}   testicle.\textsc{pl}  because  \textsc{1sg.indp}  give    \\
\gll
dɛ́n  \textit{diez}  \textit{mil}  fɔ  \textit{transporte}.   \\
\textsc{3pl.indp}  ten  thousand  \textsc{prep}  transport    \\
\glt
‘They’re going to get me really annoyed because I gave them ten thousand for transport.’
  }\ex{
ye03cd 006\\
\gll
Na  só  ín  de  tɛ́l  mí.   \\
\textsc{foc}  so  \textsc{3sg.indp}  \textsc{ipfv}  tell  \textsc{1sg.indp}    \\
\glt
‘That’s what he [\textsc{emp}] told me.’
  }\ex{
fr03cd 007\\
\gll
Mék  dɛn  \textit{transporta}  di  \textit{cemento}  na  {Ela Nguema}   \\
\textsc{sbjv}  \textsc{3pl}  transport  \textsc{def}  cement  \textsc{loc}  \textsc{place}    \\
\gll
\textit{porque}  \textit{no}  \textit{estaba}  \textit{dicho}  \textit{que}  dɛn  go  gó  lɛ́f  di   \\
because  \textsc{neg}  was  said  that  \textsc{3pl}  \textsc{pot}  go  leave  \textsc{def}    \\
\gll
\textit{cemento}.   \\
cement    \\
\glt
‘Let them transport the cement to Ela Nguema because it hadn’t been agreed that they would go leave the cement [lying there].’
  }\ex{
fr03cd 008\\
\gll
Di  \textit{cemento,}  \textit{estaba}  \textit{dicho}  \textit{que}  na  fɔ  kɛ́r=an   \\
\textsc{def}  cement  was  said  that  \textsc{foc}  \textsc{prep}  carry=\textsc{3sg.obj}    \\
\gll
\textit{directamente}  na  {Ela  Nguema}.   \\
directly  \textsc{loc}  \textsc{place}    \\
\glt
‘The cement, it had been agreed that it is to be taken directly to Ela Nguema.’
  }
\newpage
\ex{
fr03cd 009\\
\gll
Dát  min  sé  Buehu  nó  kán  e  nó  gí  nó   \\
that  mean  \textsc{quot}  \textsc{name}  \textsc{neg}  come  \textsc{3sg.sbj}  \textsc{neg}  give  \textsc{neg}    \\
\gll
mɔní  nó  nátin.   \\
money  \textsc{neg}  nothing    \\
\glt
‘That means that Buehu didn’t come (and) he didn’t give (them) any money at all.’
  }\ex{
ye03cd 010\\
\gll
Nó  nátin.   \\
\textsc{neg}  nothing    \\
\glt
‘Nothing at all.’
  }\ex{
ye03cd 011\\
\gll
Tumɔ́ro  mɔ́nin  tɛ́n,  wán  \textit{las}    \textit{siete}    só    a    go  gó   \\
tomorrow  morning  time  one  the\textsc{.pl}  seven  like.that  \textsc{1sg.sbj}  \textsc{pot}  go    \\
\gll
dé.   \\
there    \\
\glt
‘Tomorrow morning, around seven o’clock or so I’ll go there.’
  }\ex{
ye03cd 012\\
\gll
Ɔ  bɔkú  mán  dɛn  bin  de  fɛ́n/\\
or  much  man  \textsc{pl}  \textsc{pst}  \textsc{ipfv}  look.for\\
\glt
‘Or many people were looking for/’
  }\ex{
ye03cd 013\\
\gll
\textit{¿Que}  \textit{vas}  \textit{escribiendo}  \textit{así?}   \\
what  you.go  writing  so    \\
\glt
‘What are you writing like that?’
  }\ex{
ye03cd 014\\
\gll
Ɔ́l  dí  \textit{compromiso}  dɛn  fɔ  \textit{escribiendo}  dán  bɔ́y   \\
all  this  agreement  \textsc{pl}  \textsc{prep}  writing  that  boy    \\
\gll
in  \textit{apellido},  wétin  mék  yu  ráyt  mi  ném?   \\
\textsc{3sg.poss}  surname  what  make  \textsc{2sg}  write  \textsc{1sg.poss}  name    \\
\glt
‘All these agreements writing that guy’s surname, how come you’ve written my name?’
  }\ex{
ye03cd 015\\
\gll
Mék  nó  mi  \textit{caligrafía}  gó  na  dán  pépa!   \\
\textsc{sbjv}  \textsc{neg}  \textsc{1sg.poss}  handwriting  go  \textsc{loc}  that  paper    \\
\glt
‘None of my handwriting should go on that paper!’
  }\ex{
fr03cd 016\\
\gll
Dát  min  sé  yu  nó  go  hɛ́p  mí?   \\
that  mean  \textsc{quot}  \textsc{2sg}  \textsc{neg}  \textsc{pot}  help  \textsc{1sg.indp}    \\
\glt
‘That means you’re not going to help me?’
  }\ex{
ye03cd 017\\
\gll
Na  fɔ  ús=káyn  tín,  \textit{explica}  mí!   \\
\textsc{foc}  \textsc{prep}  \textsc{q}=kind    thing  explain  \textsc{1sg.indp}    \\
\glt
‘It’s for what, explain to me!’
  }\ex{
fr03cd 018\\
\gll
A  níd  wán  \textit{lista}  \textit{de}  \textit{participantes.}   \\
\textsc{1sg.sbj}  need  one  list  of  participants    \\
\glt
‘I need a list of participants.’
  }\ex{
ye03cd 019\\
\gll
Na  \textit{compromiso}  dát  ó.   \\
\textsc{foc}  agreement  that  \textsc{sp}    \\
\glt
‘That’s actually an agreement.’
  }\ex{
ye03cd 020\\
\gll
Dán  \textit{ficción},  Bata-mán  go  tɔ́n=an  rɔn-say.   \\
that  fiction  \textsc{place}{}-man    \textsc{pot}  turn=\textsc{3sg.obj}  wrong.\textsc{cpd}{}-side    \\
\glt
‘That fiction [fictitious agreement], the Fang [the person delivering the cement] will turn it upside down.’
  }\ex{
ye03cd 021\\
\gll
E  sé  ɔ́l  tidé  e  bin  de  kɔ́l  yú  yu   \\
\textsc{3sg.sbj}  \textsc{quot}  all  today  \textsc{3sg.sbj}  \textsc{pst}  \textsc{ipfv}  call  \textsc{2sg.indp}  \textsc{2sg}    \\
\gll
nó  ték  \textit{teléfono.}   \\
\textsc{neg}  take  telephone    \\
\glt
‘He said the whole of today, he was calling you (and) you didn’t pick the phone.’
  }\ex{
fr03cd 022\\
\gll
“Ɔ́l  tidé  e  bin  de  kɔ́l  mí”,  e  kɔ́l   \\
\phantom{“}all  today  \textsc{3sg.sbj}  \textsc{pst}  \textsc{ipfv}  call  \textsc{1sg.indp}  \textsc{3sg.sbj}  call    \\
\gll
mí  wán  tɛ́n  dásɔl.   \\
\textsc{1sg.indp}  one  time  only    \\
\glt
‘“All of today he was calling me”, he called me only once.’
  }\ex{
fr03cd 023\\
\gll
\textit{Bueno,}  a  bɛ́g  tumɔ́ro,  gó  \textit{recupera}  di  mɔní   \\
alright  \textsc{1sg.sbj}  ask.for  tomorrow  go  recover  \textsc{def}  money    \\
\gll
wé  yu  lɛ́f.   \\
\textsc{sub}  \textsc{2sg}  leave    \\
\glt
‘Alright, please tomorrow, go recover the money that you left.’
  }\ex{
fr03cd 024\\
\gll
Náw  só  a  gɛ́fɔ    pé  mɔní  mɔ́  fɔ  gó  kɛ́r   \\
now  so  \textsc{1sg.sbj}  have.to  pay  money  more  \textsc{prep}  go  carry    \\
\gll
di  \textit{cemento}  na  hós.   \\
\textsc{def}  cement  \textsc{loc}  house    \\
\glt
‘Now I have to pay money again in order to bring the cement to the house.’
  }\ex{
fr03cd 025\\
\gll
Nɔ́,  yu  sabí  di  tín  wé  yu  go  tɛ́l=an?   \\
\textsc{neg}  \textsc{2sg}  know  \textsc{def}  thing   \textsc{sub}  \textsc{2sg}  \textsc{pot}  tell=\textsc{3sg.obj}    \\
\glt
‘No, you know what you’re going to tell him?’
  }
  \largerpage
  \ex{
fr03cd 026\\
\gll
Yu  go  tɛ́l=an  sé  wi  nó  de  ték  di   \\
\textsc{2sg}  \textsc{pot}  tell=\textsc{3sg.obj}  \textsc{quot}  \textsc{1pl}  \textsc{neg}  \textsc{ipfv}  take  \textsc{def}    \\
\gll
\textit{cemento,}  mék  e  bák  yú  di  mɔní  wé   \\
cement  \textsc{sbjv}  \textsc{3sg.sbj}  give.back  \textsc{2sg.indp}  \textsc{def}  money  \textsc{sub}    \\
\gll
e  gɛ́t,  ɛ́n?   \\
\textsc{3sg.sbj}  get  \textsc{intj}    \\
\glt
‘You’ll tell him that we’re not taking the cement (and) that he should give you back the money that he has, right?’
  }\ex{
fr03cd 027\\
\gll
Mék  e  bák  yú  di  mɔní  wé  e  gɛ́t.   \\
\textsc{sbjv}  \textsc{3sg.sbj}  give.back  \textsc{2sg.indp}  \textsc{def}  money  \textsc{sub}  \textsc{3sg.sbj}  get    \\
\glt
‘Let him give you back the money that he has.’
  }\ex{
ye03cd 028\\
\gll
Di  wán  wé  e  dɔ́n  \textit{sobra}  ín.   \\
\textsc{def}  one  \textsc{sub}  \textsc{3sg.sbj}  \textsc{prf}  remain  \textsc{3sg.indp}    \\
\glt
‘The one [amount] that has remained with him.’
  }\ex{
fr03cd 029\\
\gll
Di  wán  wé  e  dɔ́n  \textit{sobra}  ín.   \\
\textsc{def}  one  \textsc{sub}  \textsc{3sg.sbj}  \textsc{prf}  remain  \textsc{3sg.indp}    \\
\glt
‘The one that has remained with him.’
  }\ex{
fr03cd 030\\
\gll
\'{A}fta,  \textit{como}  ín  níd  \textit{cemento},  mék  e  gó   \\
then  because  \textsc{3sg.indp}  need  cement  \textsc{sbjv}  \textsc{3sg.sbj}  go    \\
\gll
gí  yú  di  mɔní,  nóto  tumɔ́ro  e  go  gí  di   \\
give \textsc{2sg.indp}  \textsc{def}  money  \textsc{neg}.\textsc{foc}  tomorrow  \textsc{3sg.sbj}  \textsc{pot}  give  \textsc{def}    \\
\gll
mɔní.   \\
money    \\
\glt
‘Then, since he [\textsc{emp}] needs cement [as well], let him go give you the money, it won’t be tomorrow that he’ll give (you) the money.’
  }\ex{
fr03cd 031\\
\gll
\textit{Pero}  di  tɛ́n  wé  ín  go  gɛ́fɔ    báy  \textit{cemento},   \\
but  \textsc{def}  time  \textsc{sub}  \textsc{3sg.indp}  \textsc{pot}  have.to  buy  cement    \\
\gll
mék  e  ték  dán  \textit{cemento}  dé.   \\
\textsc{sbjv}  \textsc{3sg.sbj}  take  that  cement  there    \\
\glt
‘But when he himself has to buy cement, let him take that cement (there).’
  }\ex{
fr03cd 032\\
\gll
\'{A}fta,  \textit{bueno,}  tɛ́l=an  sé  mék  e  bák   \\
then  alright  tell=\textsc{3sg.obj}  \textsc{quot}  \textsc{sbjv}  \textsc{3sg.sbj}  give.back    \\
\gll
yú  di  mɔní,  mí  go  tɔ́k  wet=an.   \\
\textsc{2sg.indp}  \textsc{def}  money  \textsc{1sg.indp}  \textsc{pot}  talk  with=\textsc{3sg.obj}    \\
\glt
‘Then, alright, tell him to give you back the money, I myself will talk to him.’
  }\ex{
fr03cd 033\\
\gll
Mék  e  ték  dán  \textit{cemento}.   \\
\textsc{sbjv}  \textsc{3sg.sbj}  take  that  cement    \\
\glt
‘Let him take that cement.’
  }\ex{
fr03cd 034\\
\gll
\'{A}fta,  e  go  ték  di  \textit{cemento}.   \\
then  \textsc{3sg.sbj}  \textsc{pot}  take  \textsc{def}  cement    \\
\glt
‘Then he’ll take the cement.’
  }\ex{
fr03cd 035\\
\gll
\'{A}fta,  \textit{como}  e  go  gɛ́fɔ    pé  dán  ɔ́da  mán   \\
then  because  \textsc{3sg.sbj}  \textsc{pot}  have.to  pay  that  other  man    \\
\gll
sɛ́f  \textit{transporte},  dán  tɛ́n  e  go  \textit{devuelve}  mí   \\
\textsc{foc}  transport  that  time  \textsc{3sg.sbj}  \textsc{pot}  give.back  \textsc{1sg.indp}    \\
\gll
di  mɔní  fɔ  \textit{transporte}.   \\
\textsc{def}  money  \textsc{prep}  transport    \\
\glt
‘Then, since he’ll have to pay transport for the other man, too, at that time he’ll give me back the money for transport.’
  }\ex{
fr03cd 036\\
\textit{Porque}  \textit{yo}  \textit{no}  \textit{estoy}  \textit{para}  \textit{esas}  \textit{cosas.}   \\
\glt
‘Because I don’t like these (kind of) things.’
  }\ex{
fr03cd 037\\
\textit{Qué}  \textit{barbaridad.}   \\
\glt
‘What nonsense.’
  }\ex{
ye03cd 038\\
\gll
A  go  \textit{firma}  wé  a  go  dɔ́n  chɔ́p.   \\
\textsc{1sg.sbj}  \textsc{pot}  sign  \textsc{sub}  \textsc{1sg.sbj}  \textsc{pot}  \textsc{prf}  eat    \\
\glt
‘I’ll sign when I’ll have eaten.’
  }\ex{
fr03cd 039\\
\textit{A}  \textit{firmar}  \textit{antes,}  \textit{que}  \textit{no}  \textit{firme,}  \textit{que}  \textit{no}  \textit{coma.}   \\
\glt
‘First sign, you don’t sign, you don’t eat [laughter].’
  }\ex{
ye03cd 040\\
\gll
A  nó  de  ɛ́nta/   \\
\textsc{1sg.sbj}  \textsc{neg}  \textsc{ipfv}  enter  \op...\cp{}\\
\glt
‘I don’t enter/’
  }\ex{
fr03cd 041\\
\textit{Así}  \textit{que}  \textit{no}  \textit{firme,}  \textit{que}  \textit{no}  \textit{coma.}   \\
\glt
‘So you don’t sign, you don’t eat.’
  }\ex{
fr03cd 042\\
\gll
A  bɛ́g  gí  mí  dán  pépa  yu  gí  mí   \\
\textsc{1sg.sbj}  ask.for  give  \textsc{1sg.indp}  that  paper  \textsc{2sg}  give  \textsc{1sg.indp}    \\
\gll
wán  bolí.   \\
one  pen    \\
\glt
‘Please, give me that paper (and) give me a pen.’
  }\ex{
ye03cd 043\\
\gll
A  go  firma,  wét  fɔ  mék  a  chɔ́p,  a   \\
\textsc{1sg.sbj}  \textsc{pot}  sign  wait  \textsc{prep}  \textsc{sbjv}  \textsc{1sg.sbj}  eat  \textsc{1sg.sbj}    \\
\gll
bɛ́g.   \\
ask.for    \\
\glt
‘I’ll sign, wait for me to eat, please.’
  }\ex{
fr03cd 044\\
\gll
Nó,  nó,  nó,  a  nó  dé  ínsay  dán  stáyl.   \\
\textsc{neg}  \textsc{neg}  \textsc{neg}  \textsc{1sg.sbj}  \textsc{neg}  there  inside  that  style    \\
\glt
‘No, no, no, I’m not into that (kind of) style.’
  }\ex{
ye03cd 045\\
\gll
Lɛ́f  mék  a/,  Djunais!   \\
leave  \textsc{sbjv}  \textsc{1sg.sbj}  \textsc{name}    \\
\glt
‘Let me/, Djunais!’
  }\ex{
fr03cd 046\\
\gll
Yu    chénch   \textit{caligrafía}. \\
\textsc{2sg}   change    handwriting \\
\glt
‘You changed (your) handwriting.’
  }\ex{
ye03cd 047\\
\gll
\'{U}dat?   \\
who    \\
\glt
‘Who?’
  }\ex{
ye03cd 048\\
\gll
\'{I}n  sénwe  de  tɔ́k=an  dé.   \\
\textsc{3sg.indp}  \textsc{foc}  \textsc{ipfv}  talk=\textsc{3sg.obj}  there    \\
\glt
‘He [\textsc{emp}] himself says it.’
  }\ex{
fr03cd 049\\
\gll
\'{U}dat  tíf,  tɛ́l  mí  di  ném!   \\
who  steal  tell  \textsc{1sg.indp}  \textsc{def}  name    \\
\glt
‘Who stole, tell me the name!’
  }\ex{
ye03cd 050\\
\gll
Fíba  go  sube  yú  mɔ́.   \\
fever  \textsc{pot}  go.up  \textsc{2sg.indp}  more    \\
\glt
‘Fever will rise on you again.’
  }\ex{
ye03cd 051\\
\gll
Dɛn  go  só=an  na  Vivencias,  na  di  tín  dɛn   \\
\textsc{3pl}  \textsc{pot}  show=\textsc{3sg.obj}  \textsc{loc}  \textsc{name}  \textsc{foc}  \textsc{def}  thing  \textsc{3pl}    \\
\gll
de  tɔ́k  dé.   \\
\textsc{ipfv}  talk  there    \\
\glt
‘They’ll show it on Vivencias [a TV show], that’s the (kind of) thing they talk (about) there.’
  }\ex{
ye03cd 052\\
\gll
Wán  mán  wé  e  nó  gɛ́t  mɔní,  e  dɔ́n  gɛ́t   \\
one  man  \textsc{sub}  \textsc{3sg.sbj}  \textsc{neg}  get  money  \textsc{3sg.sbj}  \textsc{prf}  get    \\
\gll
\textit{sesenta}  \textit{años.}   \\
sixty  year.\textsc{pl}    \\
\glt
‘[In Vivencias there was] a man who doesn’t have money, he’s already sixty years old.’
  }\ex{
ye03cd 053\\
\gll
E  nó  sabí  tɔ́k  ni  Panyá,  e  sé  e   \\
\textsc{3sg.sbj}  \textsc{neg}  know  talk  even  Spanish  \textsc{3sg.sbj}  \textsc{quot}  \textsc{3sg.sbj}    \\
\gll
wánt  \textit{muchachita}  \textit{dé}  \textit{diecisiete}  \textit{años.}   \\
want  young.girl  of  seventeen  years    \\
\glt
‘He didn’t even know how to speak Spanish, he said he wants a young girl of seventeen years.’
  }\ex{
ye03cd 054\\
\gll
E  sé  “\textit{yo}  \textit{quiero}  \textit{una}  \textit{muchachita}  \textit{de}  \textit{diecisiete}   \\
\textsc{3sg.sbj}  \textsc{quot}  I  I.want  one  young.girl  of  seventeen    \\
\gll
\textit{años}”.   \\
year.\textsc{pl}    \\
\glt
‘He said “I want a young girl of seventeen years”.’
  }\ex{
ye03cd 055\\
\gll
E  nó  wánt  \textit{ni}  \textit{treinta}  \textit{ni}  \textit{cuarenta},  mm   \\
\textsc{3sg.sbj}  \textsc{neg}  want  neither  thirty  neither  forty  \textsc{intj}    \\
\gll
mm,  \textit{diecisiete}  \textit{años.}   \\
\textsc{intj}  seventeen  year.\textsc{pl}    \\
\glt
‘He neither wanted thirty nor forty years, no no, seventeen years.’
  }\ex{
ye03cd 056\\
\gll
Na  Vivencias  dɛn  de  só  dán  tín.   \\
\textsc{loc}  \textsc{name}  \textsc{3pl}  \textsc{ipfv}  show  that  thing    \\
\glt
‘In Vivencias they show that (kind of) thing.’
  }\ex{
ye03cd 057\\
\gll
\'{U}s=tín     dɛn  kin  de  gó  fɛ́n    mán  dɛn  wé  nó   \\
\textsc{q}=thing  \textsc{3pl}  \textsc{hab}  \textsc{ipfv}  go  look.for  man  \textsc{pl}  \textsc{sub}  \textsc{neg}    \\
\gll
sabí  tɔ́k  Panyá?   \\
know  talk  Spanish    \\
\glt
‘Why do they always go look for people who don’t know how to talk Spanish?’
  }\ex{
ye03cd 058\\
\gll
Dí  Olinga.   \\
this  \textsc{name}    \\
\glt
‘This Olinga.’
  }\ex{
fr03cd 059\\
\gll
A  nó  sabí  lɛk  háw  e  dé    in  wók,    bɔt   \\
\textsc{1sg.sbj}  \textsc{neg}  know  like  how  \textsc{3sg.sbj}  \textsc{be.loc}  \textsc{3sg.poss}  work  but    \\
\gll
e  nó  sabí  tɔ́k  Panyá.   \\
\textsc{3sg.sbj}  \textsc{neg}  know  talk  Spanish    \\
\glt
‘I don’t know how he is at his work, but he doesn’t know how to speak Spanish.’
  }\ex{
ko03cd 060\\
\gll
\'{U}dat?   \\
who    \\
\glt
‘Who?’
  }\ex{
fr03cd 061\\
\gll
Olinga  na  wán  \textit{presentador}  fɔ  wán  \textit{programa}.   \\
\textsc{name}  \textsc{foc}  one  presenter  \textsc{prep}  one  programme    \\
\glt
‘Olinga is a presenter of a programme.’
  }\ex{
fr03cd 062\\
\gll
Na  \textit{reportero},  \textit{el}  \textit{programa}    \textit{más}   \textit{popular}  \textit{de}  \textit{este}   \\
\textsc{foc}  reporter  the  programme     most     popular   of  this    \\
\gll
\textit{país,}  in  ném  na  Vivencias.   \\
country  \textsc{3sg.poss}  name  \textsc{foc}  \textsc{name}    \\
\glt
‘He’s a reporter, the most popular programme of this country, its name is Vivencias.’
  }\ex{
fr03cd 063\\
\gll
Di  mán  e  nó  sabí  tɔ́k  Panyá.   \\
\textsc{def}  man  \textsc{3sg.sbj}  \textsc{neg}  know  talk  Spain    \\
\glt
‘The man doesn’t know how to speak Spanish.’
  }\ex{
fr03cd 064\\
\gll
E  de  chɛ́r  wán  káyn  chɛ́r  min  sé  e  de   \\
\textsc{3sg.sbj}  \textsc{ipfv}  tear  one  kind  tear  mean  \textsc{quot}  \textsc{3sg.sbj}  \textsc{ipfv}    \\
\gll
mék  \textit{fallos}  dɛn,  \textit{faltas}.   \\
make  error.\textsc{pl}  \textsc{pl}  mistake.\textsc{pl}    \\
\glt
‘He “tears one kind of Spanish” means that he makes errors, mistakes.’
  }\ex{
fr03cd 065\\
\gll
E  de  chɛ́r  wán  káyn  chɛ́r  wé  mí  yón    \\
\textsc{3sg.sbj}  \textsc{ipfv}  tear  one  kind  tear  \textsc{sub}  \textsc{1sg.indp}  own    \\
\gll
Panyá/\\
Spanish    \\
\glt
‘He makes such serious mistakes where my own Spanish/’
  }\ex{
ye03cd 066\\
\gll
E  tɔ́k  sé  ín  na  \textit{poeta}.   \\
\textsc{3sg.sbj}  talk  \textsc{quot}  \textsc{3sg.indp}  \textsc{foc}  poet    \\
\glt
‘He said he’s a poet.’
  }\ex{
ye03cd 067\\
\gll
E  kin  de  híb  sɔn  \textit{poesía}  dɛn,  \textit{chico}  nɔ́.   \\
\textsc{3sg.sbj}  \textsc{hab}  \textsc{ipfv}  throw  some  poetry  \textsc{pl}  \textsc{intj}  \textsc{neg}    \\
\glt
‘He kicks some poetry, man really.’
  }\ex{
ye03cd 068\\
\gll
Olinga  kɔmɔ́t  frɔn  bɔtɔ́n.   \\
\textsc{name}  come.out  from  bottom    \\
\glt
‘Olinga comes from the bottom [has very modest origins].’
  }\ex{
ko03cd 069\\
\gll
Bɔt  na  ín  wé  pípul  layk=an  nɔ́.   \\
but  \textsc{foc}  \textsc{3sg.indp}  \textsc{sub}  people  like=\textsc{3sg.obj}  \textsc{neg}    \\
\glt
‘But that’s why people like him, right.’
  }
\newpage
\ex{
fr03cd 070\\
\gll
\'{A}fta  \textit{primera}  \textit{dama}  báy=an  wán  motó,  wán   \\
then  first  lady  buy=\textsc{3sg.obj}  one  car  one    \\
\gll
\textit{todo terreno,}  wán  \textit{cuatro por cuatro,}  mék  e   \\
cross-country.vehicle  one  four-wheel.drive  \textsc{sbjv}  \textsc{3sg.sbj}    \\
\gll
fít  de  rích  ɔ́l  say  sóté  na  Riaba.   \\
can  \textsc{ipfv}  arrive  all  side  until  \textsc{loc}  \textsc{place}    \\
\glt
‘Then the first lady bought him a car, a cross-country vehicle, a four-wheel drive so that he could reach all places even up to Riaba.’
  }\ex{
fr03cd 071\\
\gll
Wán  dé  wán  pikín  bin  de  sík.   \\
one  day  one  child  \textsc{pst}  \textsc{ipfv}  be.sick    \\
\glt
‘One day a child was sick.’
  }\ex{
fr03cd 072\\
\gll
A  nó  sabí  ús=káyn   tín  bin  pás.   \\
\textsc{1sg.sbj}  \textsc{neg}  know  \textsc{q}=kind  thing  \textsc{pst}  happen    \\
\glt
‘I don’t know what had happened.’
  }\ex{
fr03cd 073\\
\gll
Nó,  na  wán  mán,  ɛhɛ́  wán  \textit{accidente}  fɔ  motó   \\
\textsc{neg}  \textsc{foc}  one  man  exactly  one  accident  \textsc{prep}  car    \\
\gll
bin  dé.   \\
\textsc{pst}  \textsc{be.loc}    \\
\glt
‘No, it was a man, oh yes, there had been a car accident.’
  }\ex{
fr03cd 074\\
\gll
A  nó  sabí  ús=tín     bin      kán  pás       áfta  e   gó   \\
\textsc{1sg.sbj}  \textsc{neg}  know  \textsc{q}=thing  \textsc{pst}       \textsc{pfv}  happen  then  \textsc{3sg.sbj}   go    \\
\gll
na  \textit{hospital}.   \\
\textsc{loc}  hospital    \\
\glt
‘I don’t know what had happened for him to arrive at the hospital.’
  }\ex{
fr03cd 075\\
\gll
Di  bɔ́y  dé    dé  e  dɔ́n  de    dáy.   \\
\textsc{def}  bɔ́y  \textsc{be.loc}  there  \textsc{3sg.sbj}  \textsc{prf}  \textsc{ipfv}  die    \\
\glt
‘The guy [a casuality] was already dying.’
  }\ex{
fr03cd 076\\
\gll
E  pút  \textit{micrófono}  \textit{así,}  e  sé  \textit{“los}  \textit{últimos}   \\
\textsc{3sg.sbj}  put  microfone  so  \textsc{3sg.sbj}  \textsc{quot}  the\textsc{.pl}  last.\textsc{pl}    \\
\gll
\textit{suspiros,}  \textit{de}  \textit{un}  \textit{momento}  \textit{al}  \textit{otro}  \textit{se}  \textit{va}  \textit{a}  \textit{morir}”.\\
sigh.\textsc{pl}  of  one  moment  to  other  \textsc{refl}  go  to  die    \\
\glt
‘He put the microfone like this, he said “the last sighs, from one moment to another he’ll die”.’
  }\ex{
fr03cd 077\\
\gll
\textit{Esto}  na  wán  ɔ́da  kɔ́ntri  dɛn  go  púl  yú   \\
this  \textsc{loc}  one  other  country  \textsc{3pl}  \textsc{pot}  remove  \textsc{2sg.indp}    \\
\gll
\textit{inmediatamente,}  dɛn  de  púl  yú  wók.   \\
immediately  \textsc{3pl}  \textsc{ipfv}  remove  \textsc{2sg.indp}  work    \\
\glt
‘This is in another country, they would remove you immediately, they would remove you from work.’
  }\ex{
ye03cd 078\\
\gll
Mán  dɔ́n  \textit{diaboliza.}   \\
man  \textsc{prf}  diabolise    \\
\glt
‘People have become devilish.’
  }\ex{
fr03cd 079\\
\gll
\textit{Chico,}  yu  nó  bríng  mí  glás?   \\
\textsc{intj}  \textsc{2sg}  \textsc{neg}  bring  \textsc{1sg.indp}  glass    \\
\glt
‘Man, you haven’t brought me a glass?’
  }\ex{
fr03cd 080\\
\gll
Lɛ́f=an,  a  go  chɔ́p,  áfta  a  go  dring.   \\
leave=\textsc{3sg.obj}  \textsc{1sg.sbj}  \textsc{pot}  eat  then  \textsc{1sg.sbj}  \textsc{pot}  drink    \\
\glt
‘Leave it, I’ll eat, then I’ll drink.’
  }\ex{
ye03cd 081\\
\gll
Yu  nó  sabí  na  mí  kúk?   \\
\textsc{2sg}  \textsc{neg}  know  \textsc{foc}  \textsc{1sg.indp}  cook    \\
\glt
‘You don’t know it’s me who cooked?’
  }\ex{
fr03cd 082\\
\gll
Boyé  sabí  kúk?   \\
\textsc{name}  know  cook    \\
\glt
‘Boyé knows how to cook?’
  }\ex{
fr03cd 083\\
\gll
\'{U}s=káyn   tín  e  kúk?   \\
\textsc{q}=kind  thing  \textsc{3sg.sbj}  cook    \\
\glt
‘What did he cook?’
  }\ex{
fr03cd 084\\
\gll
E  kin  tráy  náw?   \\
\textsc{3sg.sbj}  \textsc{hab}  try  now    \\
\glt
‘So he’s making an effort now?’
  }\ex{
fr03cd 085\\
\gll
\'{U}s=káyn  tín  e  kúk,  fray-rɛ́s?   \\
\textsc{q}=kind     thing  \textsc{3sg.sbj}  cook  fry.\textsc{cpd}{}-rice    \\
\glt
‘What did he cook, fried rice?’
  }
\newpage
\ex{
dj03cd 086\\
\gll
E  kin  kúk  súp.   \\
\textsc{3sg.sbj}  \textsc{hab}  cook  soup    \\
\glt
‘He cooks soups.’
  }\ex{
fr03cd 087\\
\gll
\'{U}s=káyn   súp?   \\
\textsc{q}=kind  soup    \\
\glt
‘Which kind of soups?’
  }\ex{
dj03cd 088\\
\gll
Maluka.   \\
maluka    \\
\glt
‘Maluka’
  }\ex{
fr03cd 089\\
\gll
Maluka  e  nó  bin  tú  drɔ́,  pantáp  \textit{diez,}   \\
maluka  \textsc{3sg.sbj}  \textsc{neg}  \textsc{pst}  too  draw  on  ten    \\
\gll
ús=káyn   \textit{nota}  yu  go  gí=an?   \\
\textsc{q}=kind  mark  \textsc{2sg}  \textsc{pot}  give=\textsc{3sg.obj}    \\
\glt
‘The maluka, wasn’t it too sticky, out of ten which mark would you give him?’
  }\ex{
dj03cd 090\\
\gll
\textit{Cuatro}  \textit{con}  \textit{cinco.}   \\
four  with  five    \\
\glt
‘Four out of five.’
  }\ex{
ye03cd 091\\
\gll
\'{I}n  tɛ́l  yú  sé  pantɔ́p  \textit{cinco}  ɔ  pantɔ́p  \textit{diez,}   \\
\textsc{3sg.indp}  tell  \textsc{2sg.indp}  \textsc{quot}  on  five  or  on  ten    \\
\gll
yu  de  gí  mí  \textit{cuatro}  \textit{con}  \textit{cinco.}   \\
\textsc{2sg}  \textsc{ipfv}  give  \textsc{1sg.indp}  four  with  five    \\
\glt
‘She told you out of five or out of ten, (and) you’re giving me four over five.’
  }\ex{
fr03cd 092\\
\gll
Ɛhɛ́,  dán  wán  min  sé  ús=wán    na  di  \textit{escala?}   \\
exactly  that  one  mean  \textsc{quot}  \textsc{q}=one  \textsc{foc}  \textsc{def}  scale    \\
\glt
‘Exactly, that means which one [of the two] is the scale?’
  }\ex{
ye03cd 093\\
\gll
Di  tɔ́p,  di  \textit{nota}  \textit{máxima}  na  \textit{diez},  \textit{entonces}  yu  de   \\
\textsc{def}  top  \textsc{def}  mark  highest  \textsc{foc}  ten  so  \textsc{2sg}  \textsc{ipfv}    \\
\gll
gí  mí  \textit{cuatro}  \textit{con}  \textit{cinco.}   \\
give  \textsc{1sg.indp}  four  with  five    \\
\glt
‘The top, the highest mark is ten, and you give me four over five.’
  }\ex{
ye03cd 094\\
\gll
Nó  wi  de  \textit{conversa,}  nó  vɛ́ks  Djunais.   \\
\textsc{neg}  \textsc{1pl}  \textsc{ipfv}  converse  \textsc{neg}  be.angry  \textsc{name}    \\
\glt
‘No, we’re conversing, don’t be angry Djunais.’
  }\ex{
dj03cd 095\\
\gll
A  nɔ́  de  vɛ́ks.   \\
\textsc{1sg.sbj}  \textsc{neg}  \textsc{ipfv}  be.angry    \\
\glt
‘I’m not angry.’
  }\ex{
ye03cd 096\\
\gll
Wi  de  \textit{conversa}  na  tébul.   \\
\textsc{1pl}  \textsc{ipfv}  converse  \textsc{loc}  table    \\
\glt
‘We’re conversing at the table.’
  }\ex{
ye03cd 097\\
\gll
A  nó  fít  kɛri  yú  \textit{restaurante}  bikɔs  sé  yu   \\
\textsc{1sg.sbj}  \textsc{neg}  can  carry  \textsc{2sg.indp}  restaurant  because  \textsc{quot}  \textsc{2sg}    \\
\gll
go  fɛ́t  wet  sɔn  pɔ́sin  dé.   \\
\textsc{pot}  fight  with  some  person  there    \\
\glt
‘I can’t take you to a restaurant because you would fight with somebody there.’
  }\ex{
fr03cd 098\\
\gll
Wétin  yu  de    tɔ́k/\\
what  \textsc{2sg}  \textsc{ipfv}  talk    \\
\glt
‘What are you talking/ [music from below drowns the recording for a few minutes]
  }\ex{
ye03cd 099\\
\gll
Gabriel  e  gɛ́t  jege.   \\
\textsc{name}  \textsc{3sg.sbj}  get  ?    \\
\glt
Gabriel has a “jege”.’
  }\ex{
fr03cd 100\\
\gll
\'{U}s=káyn   tín  na  jege?   \\
\textsc{q}=kind  thing  \textsc{foc}  ?    \\
\glt
‘What’s a “jege”?’
  }\ex{
ye03cd 101\\
\gll
E  gɛ́t  sɔn,  sɔn  smɔ́l  jege  ínsay  in  yáy  só.   \\
\textsc{3sg.sbj}  get  some  some  small  ?  inside  \textsc{3sg.poss}  eye  like.that    \\
\glt
‘He has a, a small “jege” inside his eye like that.’
  }\ex{
fr03cd 102\\
\gll
\'{U}s=káyn  tín  na  jege?   \\
\textsc{q}=kind  thing  \textsc{foc}  ?    \\
\glt
‘What’s “jege”?’
  }\ex{
dj03cd 103\\
\gll
Dán  wáyt  tín  wé  e  dé    na  in    yáy.   \\
that  white  thing  \textsc{sub}  \textsc{3sg.sbj}  \textsc{be.loc}  \textsc{loc}  \textsc{3sg.poss}  eye    \\
\glt
‘That white thing that’s in his eye.’
  }\ex{
ye03cd 104\\
\gll
Sɔn  tín  de  \textit{tapa}  in  \textit{retina.}   \\
some  thing  \textsc{ipfv}  cover  \textsc{3sg.poss}  retina    \\
\glt
‘Something covers his retina.’
  }\ex{
dj03cd 105\\
\gll
Nɔ́,  wán  \textit{accidente}  wé  e  bin  gɛ́t.   \\
\textsc{neg}  one  accident  \textsc{sub}  \textsc{3sg.sbj}  \textsc{pst}  get    \\
\glt
‘No, it’s an accident that he had.’
  }\ex{
ye03cd 106\\
\gll
Wán  yáy  dé    \textit{blanco},  e  nó  de  sí.   \\
one  eye  \textsc{be.loc}  white  \textsc{3sg.sbj}  \textsc{neg}  \textsc{ipfv}  see    \\
\glt
‘One eye is white, it doesn’t see.’
  }\ex{
fr03cd 107\\
\gll
Mí  nó  \textit{fija}  ín.   \\
\textsc{1sg.indp}  \textsc{neg}  notice  \textsc{3sg.indp}    \\
\glt
‘I didn’t notice it.’
  }\ex{
ye03cd 108\\
\gll
Yu  gɛ́fɔ    \textit{fija}  ín.   \\
\textsc{2sg}  have.to  notice  \textsc{3sg.indp}    \\
\glt
‘One has to notice it.’
  }\ex{
ye03cd 109\\
\gll
E  de  \textit{para}  na  dán  in  yáy  bɔkú  bád.   \\
\textsc{3sg.sbj}  \textsc{ipfv}  stand  \textsc{loc}  that  \textsc{3sg.poss}  eye  much  extremely    \\
\glt
‘It sits there in his eye real bad.’
  }\ex{
ye03cd 110\\
\gll
Ɛhɛ́,  ús=tín       wi  go  tɔ́k  fɔ  Gabriel?   \\
exactly  \textsc{q}=thing  \textsc{1pl}  \textsc{pot}  talk  \textsc{prep}  \textsc{name}    \\
\glt
‘So, what are we going to say to Gabriel?’
  }\ex{
fr03cd 111\\
\gll
Tumɔ́ro,  lɛk  háw  yu  tɔ́k  wet  Buehu,  yu  kɔ́l  mí,   \\
tomorrow  like  how  \textsc{2sg}  talk  with  \textsc{name}  \textsc{2sg}  call  \textsc{1sg.indp}    \\
\gll
if  yu  tɔ́k  wet=an  ɔ  yu  nó  tɔ́k  wet=an.   \\
if  \textsc{2sg}  talk  with=\textsc{3sg.obj}  or  \textsc{2sg}  \textsc{neg}  talk  with=\textsc{3sg.obj}    \\
\glt
Tomorrow, as soon as you’ve talked to Buehu, you call me, whether you talk to him or you don’t talk to him.’
  }\ex{
fr03cd 112\\
\gll
If  yu  nó  tɔ́k  wet=an,  yu  kɔ́l  mí,  dán  tɛ́n   \\
if  \textsc{2sg}  \textsc{neg}  talk  with=\textsc{3sg.obj}  \textsc{2sg}  call  \textsc{1sg.indp}  that  time    \\
\gll
yu  go  gɛ́fɔ    kán  na  wók,  mék  a  gí   \\
\textsc{2sg}  \textsc{pot}  have.to  come  \textsc{loc}  work  \textsc{sbjv}  \textsc{1sg.sbj}  give    \\
\gll
yú  wán  \textit{cheque,}  mék  yu  gó  ték  mɔní  fɔ  báy   \\
\textsc{2sg.indp}  one  cheque  \textsc{sbjv}  \textsc{2sg}  go  take  money  \textsc{prep}  buy    \\
\gll
di  \textit{bloque}s  dɛn  wé  dɛn  lɛ́f.   \\
\textsc{def}  brick.\textsc{pl}  \textsc{pl}  \textsc{sub}  \textsc{3pl}  remain    \\
\glt
‘If you don’t talk to him, you call me, then you’ll have to come to work, so that I give you a cheque, in order for you to go get money to buy the remaining bricks.’
  }\ex{
fr03cd 113\\
\gll
\'{A}fta  yu  fɔ  pé  dɛ́n.   \\
then  \textsc{2sg}  \textsc{prep}  pay  \textsc{3pl.indp}    \\
\glt
‘Then you have to pay them.’
  }\ex{
fr03cd 114\\
\gll
Sɔn  \textit{bloques}  dɛn  lɛ́f  wé  dɛn  gɛ́fɔ    \textit{monta}  nɔ́?   \\
some  brick.\textsc{pl}  \textsc{pl}  remain  \textsc{sub}  \textsc{3pl}  have.to  mount  \textsc{neg}    \\
\glt
‘Some bricks remain that they have to build up, right?’
  }\ex{
fr03cd 115\\
\gll
\textit{Pero}  e  bin  tɛ́l  mí  sé  mék  a  báy   \\
but  \textsc{3sg.sbj}  \textsc{pst}  tell  \textsc{1sg.indp}  \textsc{quot}  \textsc{sbjv}  \textsc{1sg.sbj}  buy    \\
\gll
\textit{cuarenta}  \textit{bloques.}   \\
forty  brick.\textsc{pl}    \\
\glt
‘But he told me to buy forty bricks.’
  }\ex{
ye03cd 116\\
\gll
Yɛ́s  \textit{cuarenta}.   \\
yes  forty    \\
\glt
‘Yes forty.’
  }\ex{
fr03cd 117\\
\gll
\textit{Entonces}  fɔ  \textit{monta}  ɔ́l  di  baf-rúm,              e   \\
so  \textsc{prep}  mount  all  \textsc{def}  bathe.\textsc{cpd}{}-room  \textsc{3sg.sbj}    \\
\gll
bin  tɛ́l  mí  sé  na  \textit{cuarenta}  \textit{mil}  fɔ  di  wók   \\
\textsc{pst}  tell  \textsc{1sg.indp}  \textsc{quot}  \textsc{foc}  forty  thousand  \textsc{prep}  \textsc{def}  work    \\
\gll
wet  di  \textit{bloques}  dɛn  wé  dɛn  lɛ́f.   \\
with  \textsc{def}  brick.\textsc{pl}  \textsc{pl}  \textsc{sub}  \textsc{3pl}  remain    \\
\glt
‘So in order to build the whole bathroom, he had told me that it’s forty thousand for the work with the remaining bricks.’
  }\ex{
fr03cd 118\\
\gll
Dán  tɛ́n  tumɔ́ro,  ɛf  yu  nó  kán  sí   dán  mán,  mék   \\
that  time  tomorrow  if  \textsc{2sg}  \textsc{neg}  \textsc{pfv}  see  that  man  \textsc{sbjv}    \\
\gll
a  kán  mék  a  gí  yú  di  mɔní.   \\
\textsc{1sg.sbj}  come  \textsc{sbjv}  \textsc{1sg.sbj}  give  \textsc{2sg.indp}    \textsc{def}  money    \\
\glt
‘By that time tomorrow, if you don’t see that man, let me come and give you the money.’
  }\ex{
ye03cd 119\\
\gll
Mék  a  gí  yú  di  \textit{cheque}  mék  yú  gó   \\
\textsc{sbjv}  \textsc{1sg.sbj}  give  \textsc{2sg.indp}  \textsc{def}  cheque  \textsc{sbjv}  \textsc{2sg.indp}  go    \\
\gll
na  \textit{banco}  yu  gó  \textit{cobra}.   \\
\textsc{loc}  bank  \textsc{2sg}  go  receive    \\
\glt
‘Let me give you the cheque so that you go to the bank and receive (the money).’
  }\ex{
ye03cd 120\\
\gll
\'{A}fta  una  báy  di  \textit{bloques}  dɛn  tumɔ́ro.   \\
then  \textsc{2pl}  buy  \textsc{def}  brick.\textsc{pl}  \textsc{pl}  tomorrow    \\
\glt
‘Then you [\textsc{pl}] buy the bricks tomorrow.’
  }\ex{
dj03cd 121\\
\gll
Na  in  fés,  na  in  héd,  sí=an!   \\
\textsc{loc}  \textsc{3sg.poss}  face  \textsc{loc}  \textsc{3sg.poss}  head  see=\textsc{3sg.obj}    \\
\glt
[Comments on the effects of the pepper in the food (ye) has just tried] ‘In his face, in his head, look at him!’
  }\ex{
ye03cd 122\\
\gll
Tɔ́k  bifó  di  [unintelligible]\\
talk  before  \textsc{def}    \\
\glt
‘Talk in front of/ (...)’
  }\ex{
ye03cd 123\\
\gll
\textit{¿Sí}  \textit{o}  \textit{no?}   \\
yes  or  no    \\
\glt
‘Yes or no?’
  }\ex{
ye03cd 124\\
\gll
Yu  nó  hía  wé  a  tɛ́l  Pancho  sé  \textit{quiero}  \textit{cocinar?}   \\
\textsc{2sg}  \textsc{neg}  hear  \textsc{sub}  \textsc{1sg.sbj}       tell  \textsc{name}  \textsc{quot}  I.want  cook    \\
\glt
‘Didn’t you hear when I told Pancho that I wanted to cook?’
  }\ex{
ye03cd 125\\
\gll
Djunais  tɔ́k  trú!   \\
\textsc{name}  talk  true    \\
\glt
‘Djunais tell the truth!’
  }\ex{
ye03cd 126\\
\gll
Sóté  a  tɛ́l  Djunais  sé  pút  mí  wet  Pancho,   \\
until  \textsc{1sg.sbj}  tell  \textsc{name}  \textsc{quot}  put  \textsc{1sg.indp}  with  \textsc{name}    \\
\gll
wi  go  chɔ́p  wán  say.   \\
\textsc{1pl}  \textsc{pot}  eat  one  side    \\
\glt
‘I even told Djunais to put [dish the food] for me and Pancho, we’ll eat in one place.’
  }
\newpage
\ex{
ye03cd 127\\
\gll
\textit{Porque}  a  chɛ́k  sé/\\
because  \textsc{1sg.sbj}  think  \textsc{quot}    \\
\glt
‘Because I thought that/’
  }\ex{
fr03cd 128\\
\gll
\textit{Porque}  ɛ́ni  tɛ́n  wé  mí  de  kɔ́l/  e  nó  gɛ́t   \\
because  every  time  \textsc{sub}  \textsc{1sg.indp}  \textsc{ipfv}  call  \textsc{3sg.sbj}  \textsc{neg}  get    \\
\gll
\textit{móvil}  mɔ́?   \\
mobile  more    \\
\glt
‘Because anytime that I call/ doesn’t he have a mobile-phone anymore?’
  }\ex{
ye03cd 129\\
\gll
\'{U}s=nɔ́mba  yu  gɛ́fɔ    dán  \textit{móvil?}   \\
\textsc{q}=number  \textsc{2sg}  have.to  that  mobile    \\
\glt
‘Which number do you have in that [your] mobile?’
  }\ex{
ye03cd 130\\
\gll
Yu  nó  sí  dán  gyál  de  chénch,  e  de   \\
\textsc{2sg}  \textsc{neg}  see  that  girl  \textsc{ipfv}  change  \textsc{3sg.sbj}  \textsc{ipfv}    \\
\gll
chénch-chénch  dán  nɔ́mba  dɛn  lɛk  \textit{terrorista}  wé  e   \\
\textsc{red}.\textsc{cpd}{}-change  that  number  \textsc{pl}  like  terrorist  \textsc{sub}  \textsc{3sg.sbj}    \\
\gll
nó  wánt  mék  dɛn  kéch=an.   \\
\textsc{neg}  want  \textsc{sbjv}  \textsc{3pl}  catch=\textsc{3sg.obj}    \\
\glt
‘Don’t you see that girl [referring to speaker (fr)] changes, she constantly changes those numbers like a terrorist who doesn’t want to be caught.’
  }\ex{
ye03cd 131\\
\gll
Wétin  yu  de  chench-chénch  nɔ́mba  dɛn  só?   \\
what  \textsc{2sg}  \textsc{ipfv}  \textsc{red}.\textsc{cpd}{}-change  number  \textsc{pl}  like.that    \\
\glt
‘Why are you constantly changing numbers like that?’
  }\ex{
ye03cd 132\\
\gll
Nó  drink  watá,  nó  drink  watá,  yu  go  \textit{siente}   \\
\textsc{neg}  drink  water  \textsc{neg}  drink  water  \textsc{2sg}  \textsc{pot}  feel    \\
\gll
ín  bád,  a  tɛ́l  yú.   \\
\textsc{3sg.indp}  extremely  \textsc{1sg.sbj}  tell  \textsc{2sg.indp}    \\
\glt
[Addresses speaker (dj) who is drinking tap water] ‘Don’t drink water, don’t drink water, you’ll feel it real bad, I tell you.’
  }\ex{
ye03cd 133\\
\gll
A  de  tɛ́l  yú,  yu  go  sí  náw  yu  nó  go   \\
\textsc{1sg.sbj}  \textsc{ipfv}  tell  \textsc{2sg.indp}  \textsc{2sg}  \textsc{pot}  see  now  \textsc{2sg}  \textsc{neg}  \textsc{pot}    \\
\gll
fínis  dán  watá.\\
finish  that  water\\
\glt
‘I’m telling you, you’ll see now you won’t finish that water.’
  }\ex{
ye03cd 134\\
\gll
A  bin  wánt  \textit{intenta}  dríng  watá.   \\
\textsc{1sg.sbj}  \textsc{pst}  want  intend  drink  water    \\
\glt
‘I had wanted to try to drink water.’
  }\ex{
dj03cd 135\\
\gll
Mí  nóto  yú.   \\
\textsc{1sg.indp}  \textsc{neg}.\textsc{foc}  \textsc{2sg.indp}    \\
\glt
‘I’m not you.’
  }\ex{
ye03cd 136\\
\gll
Sí,  sí,  e  fíba  vɔ́mit,  yu  de  sí?   \\
see  see  \textsc{3sg.sbj}  resemble  vomit  \textsc{2sg}  \textsc{ipfv}  see    \\
\glt
‘See, see, it [the water] seems like vomit, you see?’
  }\ex{
ye03cd 137\\
\gll
A  de  tɛ́l  yú,  e  fíba  vɔ́mit  ínsay   \\
\textsc{1sg.sbj}  \textsc{ipfv}  tell  \textsc{2sg.indp}  \textsc{3sg.sbj}  resemble  vomit  inside    \\
\gll
in  mɔ́t  náw.   \\
\textsc{3sg.poss}  mouth  now    \\
\glt
‘I’m telling you, it seems like vomit inside his mouth now.’
  }\ex{
dj03cd 138\\
\gll
Yu  dé    bád  ɛ́n.   \\
\textsc{2sg}  \textsc{be.loc}  bad  \textsc{intj}    \\
\glt
‘You’re mean, really,’
  }\ex{
ye03cd 139\\
\gll
E  dé    lɛkɛ  sé  yu  de  drink  \textit{ácido.}   \\
\textsc{3sg.sbj}  \textsc{be.loc}  like  \textsc{quot}  \textsc{2sg}  \textsc{ipfv}  drink  acid    \\
\glt
‘It’s as if you’re drinking acid.’
  }\ex{
ye03cd 140\\
\gll
Háw  yu  \textit{siente}  dán  watá?   \\
how  \textsc{2sg}  feel  that  water    \\
\glt
‘How does that water feel to you?’
  }\ex{
dj03cd 141\\
\gll
E  kin  táyt  mi  bɛlɛ́  náw  só,  chakrá  dán   \\
\textsc{3sg.sbj}  \textsc{hab}  be.tight  \textsc{1sg.poss}  belly  now  so  destroy  that    \\
\gll
\textit{pasta}  smɔ́l,  yu  nó  go  wín  mí.   \\
paste  a.bit  \textsc{2sg}  \textsc{neg}  \textsc{pot}  wín  \textsc{1sg.indp}    \\
\glt
‘It tightens my stomach like this, shakes up that pap [which is being served for dinner] a bit, you won’t defeat me.’
  }\ex{
dj03cd 142\\
\gll
Yu  nó  fít.   \\
\textsc{2sg}  \textsc{neg}  can    \\
\glt
‘You can’t.’
  }\ex{
ye03cd 143\\
\gll
Hó,  dán  mán  go  dú  vɔ́mit  tidé,  e  nó  go  slíp.   \\
\textsc{intj}  that man  \textsc{pot}  do  vomit  today  \textsc{3sg.sbj} \textsc{neg}  \textsc{pot}  sleep    \\
\glt
‘Ho, that man (dj) will vomit today, he won’t sleep.’
  }\ex{
ye03cd 144\\
\gll
A  go  tɛ́l  Pancho  sé  wi  de  mék  \textit{banquete}.   \\
\textsc{1sg.sbj}  \textsc{pot}  tell  \textsc{name}  \textsc{quot}  \textsc{1pl}  \textsc{ipfv}  make  banquet    \\
\glt
‘I’ll tell Pancho [who’s not present] that we were having a banquet.’
  }\ex{
ye03cd 145\\
\gll
Dán  káyn  tín,  yu  \textit{cuenta}  Pancho  dán  káyn  tín/.\\
that  kind  thing  \textsc{2sg}  narrate  \textsc{name}  that  kind  thing    \\
\glt
‘That kind of thing, if you tell Pancho that kind of thing/.’
  }\ex{
dj03cd 146\\
\gll
E  kin  vɛ́ks  bád.   \\
\textsc{3sg.sbj}  \textsc{hab}  be.angry  bad    \\
\glt
‘He gets really angry [for being left out of the dinner].’
  }\ex{
ye03cd 147\\
\gll
Tidé  e  kán  e  sé,  “a  tínk  sé  a  go   \\
today  \textsc{3sg.sbj}  come  \textsc{3sg.sbj}  \textsc{quot}  \textsc{1sg.sbj}  think  \textsc{quot}  \textsc{1sg.sbj}  \textsc{pot}    \\
\gll
fínis  ɔ́l  di  \textit{resto}”.\\
finish  all  \textsc{def}  rest    \\
\glt
‘Today he came and said “I think I’m going to finish all the rest [of the food]”.’
  }\ex{
dj03cd 148\\
\gll
E  de  fɔgɛ́t  sé  Rubi  nɔ́ba  chɔ́p.   \\
\textsc{3sg.sbj}  \textsc{ipfv}  forget  \textsc{quot}  \textsc{name}  \textsc{neg}.\textsc{prf}  eat    \\
\glt
‘He was forgetting that Rubi hadn’t eaten yet.’
  }\ex{
ye03cd 149\\
\gll
E  tɛ́l=an  sé  “papá  mí  nɛ́a  chɔ́p   \\
\textsc{3sg.sbj}  tell=\textsc{3sg.obj}  \textsc{quot}  father  \textsc{1sg.indp}  \textsc{neg}.\textsc{prf}  eat    \\
\gll
mí  sénwe”.\\
\textsc{1sg.indp}  \textsc{foc}    \\
\glt
‘He [Rubi] told him [Pancho] “please, I myself haven’t eaten yet”.’
  }\ex{
ye03cd 150\\
\gll
Sé  “\textit{chico,}  di  tín  nó  go  dú  mí”.\\
\textsc{quot}  \textsc{intj}  \textsc{def}  thing   \textsc{neg}  \textsc{pot}  do  \textsc{1sg.indp}    \\
\glt
‘[Pancho] said “man, this won’t do for me”.’
  }\ex{
ye03cd 151\\
\gll
A  tɛ́l  Pancho  sé  “yu  nó  lɛ́k  yu  sɛ́f”.\\
\textsc{1sg.sbj}  tell  \textsc{name}  \textsc{quot}  \textsc{2sg}  \textsc{neg}  like  \textsc{2sg}  self    \\
\glt
‘I said to Pancho “you don’t like yourself [should be ashamed of yourself]”.’
  }\ex{
ye03cd 152\\
\gll
Mí  wet  Rubi  wi  mék  jwɛn-jwɛ́n,     wi  báy   \\
\textsc{1sg.indp}  with  \textsc{name}  \textsc{1pl}  make  \textsc{red}.\textsc{cpd}{}-join   \textsc{1pl}  buy    \\
\gll
pía,  wi  báy  sadín,  wi  báy  \textit{tomates},  wi   \\
avocado  \textsc{1pl}  buy  sardine  \textsc{1pl}  buy  tomatoes  \textsc{1pl}    \\
\gll
\textit{desayuna}.   \\
breakfast    \\
\glt
‘Me and Rubi, we teamed up, we bought avocados, we bought sardine, we bought tomatoes (and) we had breakfast.’
  }\ex{
ye03cd 153\\
\gll
Pancho  de  lúk  mí  só.   \\
\textsc{name}  \textsc{ipfv}  look  \textsc{1sg.indp}  like.that    \\
\glt
‘Pancho was looking at me like this.’
  }\ex{
fr03cd 154\\
\gll
Mɔní  nó  dé    dɔ́n    mɔ́?   \\
money  \textsc{neg}  \textsc{be.loc}  down  more    \\
\glt
‘Is there no money left down (there) [for your daily expenses]?’
  }\ex{
ye03cd 155\\
\gll
E  nó  dé    mɔ́.   \\
\textsc{3sg.sbj}  \textsc{neg}  \textsc{be.loc}  more    \\
\glt
‘None is left.’
  }\ex{
fr03cd 156\\
\gll
\textit{Veinte}  \textit{mil}  wé  bin  dɔ́n  fínis?   \\
twenty  thousand  \textsc{sub}  \textsc{pst}  \textsc{prf}  finish    \\
\glt
‘Twenty thousand (that) have already finished?’
  }\ex{
ju03cd 157\\
\gll
Dɛn-ɔ́l    fínis.   \\
\textsc{3pl.cpd-}all  finish    \\
\glt
‘They’ve already finished.’
  }\ex{
fr03cd 158\\
\gll
\textit{Pero}  \textit{apenas}    \textit{dos}  \textit{semanas}  wé  yu  bin  tɛ́l  mí   \\
but  barely  two  week.\textsc{pl}  \textsc{sub}  \textsc{2sg}  \textsc{pst}  tell  \textsc{1sg.indp}    \\
\gll
sé  yu  níd  a  pút  veinte  mil  dɔ́n.   \\
\textsc{quot}  \textsc{2sg}  need  \textsc{1sg.sbj}  put  twenty  thousand  down    \\
\glt
‘But (it’s) barely two weeks (ago) that you told me that you needed me to put twenty thousand down for you.’
  }\ex{
ye03cd 159\\
\gll
Na  yú  bin  tɛ́l  mí.   \\
\textsc{foc}  \textsc{2sg.indp}  \textsc{pst}  tell  \textsc{1sg.indp}    \\
\glt
‘It was you who told me.’
  }\ex{
ye03cd 160\\
\gll
\textit{Bueno,}  una  bríng  mí  di  pépa.   \\
alright  \textsc{2pl}  bring  \textsc{1sg.indp}  \textsc{def}  paper    \\
\glt
‘Alright bring me the paper.’
  }\ex{
ye03cd 161\\
\gll
\textit{Porque}  \textit{en}  \textit{dos}  \textit{semanas}  mék  \textit{veinte}  \textit{mil}  fínis.   \\
because  in  two  weeks  make  twenty  thousand  finish    \\
\glt
‘Because to make twenty thousand finish in two weeks.’
  }\ex{
fr03cd 162\\
\gll
Háw  mɔ́ch  una  de  ték  \textit{por}  \textit{día?}   \\
how  much  \textsc{2pl}  \textsc{ipfv}  take  by  day    \\
\glt
‘How much do you take [spend] per day?’
  }\ex{
dj03cd 163\\
\gll
A  go  bríng  di  pépa.   \\
\textsc{1sg.sbj}  \textsc{pot}  bring  \textsc{def}  paper    \\
\glt
‘I’ll bring the paper.’
  }\ex{
ye03cd 164\\
\gll
“Mék  yu  tɛ́l  dɛ́n  sé  fɔ  mí,  ɛ́ni  dé  ɛf  yu   \\
\phantom{“}\textsc{sbjv}  \textsc{2sg}  tell  \textsc{3pl.indp}  \textsc{quot}  \textsc{prep}  \textsc{1sg.indp}  every  day  if  \textsc{2sg}    \\
\gll
de  ték  un  \textit{kilo,}  e  dú.”\\
\textsc{ipfv}  take  one  kilo  \textsc{3sg.sbj}  do    \\
\glt
[Continues quoting Pancho] ‘“Tell them that for me, every day, if you take one kilo, it’s enough.”’
  }\ex{
dj03cd 165\\
\gll
‘Di  dé  wé  yu  sí  bɛ́ta  chɔ́p  yu  de  chɔ́p  fáyn.’   \\
\textsc{def}  day  \textsc{sub}  \textsc{2sg}  see  very.good  food  \textsc{2sg}  \textsc{ipfv}  food  fine    \\
\glt
[Quotes his inner speech to Pancho] ‘The day [when] you find good food, you eat well.’
  }\ex{
dj03cd 166\\
\gll
Di  dé  wé  pɛ́pɛ  nó  dé  ínsay  pɔ́t  “a  nó  de   \\
\textsc{def}  day  \textsc{sub}  pepper  \textsc{neg}  \textsc{be.loc}  inside  \textsc{pot}  \textsc{1sg.sbj}  \textsc{neg}  \textsc{ipfv}    \\
\gll
chɔ́p  dí  \textit{porquería!}”\\
food  this  mess    \\
\glt
[Continues quoting his inner speech to Pancho] ‘The day [when] there is no pepper in the pot (you say) “I won’t eat this mess.”’
  }\ex{
ye03cd 167\\
\gll
A  tɛ́l  yú  sé  una  de  pík  pɛ́pɛ  aunque  na   \\
\textsc{1sg.sbj}  tell  \textsc{2sg.indp}  \textsc{quot}  \textsc{2pl}  \textsc{ipfv}  pick  pepper  even  \textsc{loc}    \\
\gll
bús,  e  go  chɔ́p  ɔ́l  káyn  tín  ɛf  e  gɛ́t  pɛ́pɛ.   \\
forest  \textsc{3sg.sbj}  \textsc{pot}  eat  all  kind  thing  if  \textsc{3sg.sbj}  get  pepper    \\
\glt
‘I tell you, you could pick pepper like in the forest, he would eat any kind of thing if it has pepper.’
  }
\newpage
\ex{
ye03cd 168\\
\gll
Yɛ́stadé  a  kúk  mí  sénwe,  \textit{al final}  a  gó  chɔ́p.   \\
yesterday  \textsc{1sg.sbj}  cook  \textsc{1sg.indp}  \textsc{foc}  finally  \textsc{1sg.sbj}  go  eat    \\
\glt
‘Yesterday I myself cooked (and) then I ate.’
  }\ex{
ye03cd 169\\
\gll
Na  Pancho  dɛn  bin  de  \textit{combate}  ín  dé   \\
\textsc{foc}  \textsc{name}  \textsc{pl}  \textsc{pst}  \textsc{ipfv}  fight  \textsc{3sg.indp}  there    \\
\gll
mɔ́nin  tɛ́n.   \\
morning  time    \\
\glt
‘It’s Pancho they were having an argument with there in the morning.’
  }\ex{
ye03cd 170\\
\gll
Sí,  na  só  mí  sɛ́f  kin  dé  wé  a  kin  kúk.   \\
see  \textsc{foc}  so  \textsc{1sg.indp}  \textsc{foc}  \textsc{hab}  \textsc{be.loc}  \textsc{sub}  \textsc{1sg.sbj}  \textsc{hab}  cook    \\
\glt
‘(You) see that’s how I am, too, when I cook.’
  }\ex{
ye03cd 171\\
\gll
Bɔt  wé  pɔ́sin  de  kúk  ín  sénwe  “chip”.\\
but  \textsc{sub}  person  \textsc{ipfv}  cook  \textsc{3sg.indp}  \textsc{foc}  \phantom{‘}\textsc{skt}    \\
\glt
‘But when somebody himself cooks, “chip”.’
  }\ex{
ye03cd 172\\
\gll
A  bigín  de  \textit{pica-píca},  wi  fráy  \textit{patata},  wi   \\
\textsc{1sg.sbj}  begin  \textsc{ipfv}  \textsc{red}.\textsc{cpd}{}-cut.up  \textsc{1pl}  fry  potato  \textsc{1pl}    \\
\gll
fráy  plantí.   \\
fry  plantain    \\
\glt
‘I began to cut up (the trimmings), we fried potatoes, we fried plantain.’
  }\ex{
fr03cd 173\\
\gll
Una  bin  tɔ́k  wet  Pancho?   \\
\textsc{2pl}  \textsc{pst}  talk  with  \textsc{name}    \\
\glt
‘Did you talk to Pancho?’
  }\ex{
ye03cd 174\\
\gll
Wi  dɔ́n  tɔ́k  wet=an.   \\
\textsc{1pl}  \textsc{prf}  talk  with=\textsc{3sg.obj}    \\
\glt
‘We’ve talked to him.’
  }\ex{
fr03cd 175\\
\gll
Bɔt  wétin  a  bin  gí  yú  dán  \textit{fax}?   \\
but  what  \textsc{1sg.sbj}  \textsc{pst}  give  \textsc{2sg.indp}  that  fax    \\
\glt
‘But (then) what did I give you that fax for?’
  }\ex{
ye03cd 176\\
\gll
Dán  dé  a  bít  Pancho,  a  bít=an  a   \\
that  day  \textsc{1sg.sbj}  beat  \textsc{name}  \textsc{1sg.sbj}  beat=\textsc{3sg.obj}  \textsc{1sg.sbj}    \\
\gll
tɛ́l=an  sé,  sóté  a  tɛ́l=an  sé  “ɛf  yu\\
tell=\textsc{3sg.obj}  \textsc{quot}  until  \textsc{1sg.sbj}  tell=\textsc{3sg.obj}  \textsc{quot}  if  \textsc{2sg}    \\
\gll
wánt,  a  de  \textit{alquila}   yú         \textit{taksí,}    yu   \\
want  \textsc{1sg.sbj}  \textsc{ipfv}  rent      \textsc{2sg.indp}  taxi     \textsc{2sg}    \\
\gll
\textit{sube}  ɔ́p,  e  sáful.”\\
go.up  up  \textsc{3sg.sbj}  be.slow    \\
\glt
‘That day I beat Pancho, I beat him and told him that, I even told him that “if you want I’ll rent you a taxi, you drive up, (and) it’s cool (like that)”.’
  }\ex{
ye03cd 177\\
\gll
“A  de  gí  yú  \textit{quinientos}”,  a  tɛ́l  di   \\
\phantom{“}\textsc{1sg.sbj}  \textsc{ipfv}  give  \textsc{2sg.indp}  fifteen  \textsc{1sg.sbj}  tell  \textsc{def}    \\
\gll
taksi-mán       sé  “\textit{arriba}      \textit{a}  \textit{mi}    \textit{casa}”.\\
taxi.\textsc{cpd}{}-man  \textsc{quot}     up        to  \textsc{1sg.poss}  house    \\
\glt
‘[I told Pancho] “I’ll give you five hundred”, I told the taxi driver “up to my house”.’
  }\ex{
ye03cd 178\\
\gll
Bɔkú  motó  dɛn  dé    yá  só,  a  nó  nó  sé   \\
much  car  \textsc{pl}  \textsc{be.loc}  here  so  \textsc{1sg.sbj}  \textsc{neg}  know  \textsc{quot}    \\
\gll
Pancho  mék  lɛkɛ  sé  e  de  \textit{sube}  bihɛ́n  wé   \\
\textsc{name}  make  like  \textsc{quot}  \textsc{3sg.sbj}  \textsc{ipfv}  go.up  behind  \textsc{sub}    \\
\gll
e  \textit{baja}  mɔ́.   \\
\textsc{3sg.sbj}  go.down  more    \\
\glt
‘(Because) there were many cars there, I didn’t know that Pancho pretended to go up behind and then went down again.’
}\z

\section{Conversation: On sun glasses}
\largerpage
The text below is the transcription of a brief conversation captured on video. It features the discourse participants Boyé (ye), Nenuko (ne), and Lage (ge). The style is informal and jovial. It involves peer-to-peer communication and is decidedly male in its orientation. The text opens with an anecdote by (ye) from his secondary school time (001)–(005). Having heard from a classmate that the President of Equatorial Guinea (Obiang Nguema) could supposedly see people naked through the pair of dark sunglasses that he wore in public (002), (ye) decides to ask his mother to get him such a pair on one of her trips abroad (003). 


In what follows, (ne) and (ye) carry the idea further. Of course, the implicit idea is that it would allow them to see the opposite sex naked in the streets. The ensuing conversation is of particular interest because it contains a number of linguistic forms that serve to express emphatic, emotionally involved speech in Pichi. It involves the generous use of emphatic prosodic features such as extra-high pitch, indicated by double acute accents in the text (\textit{bl\H{a}k} ‘really dark’ (001); \textit{sl\H{i}p} ‘sleep’ (010), \textit{p\H{e}n} ‘pain’ (015) and the entire sentence (012)), vowel lengthening (\textit{eyé} ‘\textsc{intj}’ (008), \textit{ɔ́l} ‘all’ (012)), and increased volume (sentences (009)–(010), (015), (017)–(018)). 



At the segmental level, we find additional defining elements of emphatic speech like interjections (\textit{por Dios} ‘by God’ (003), \textit{eyé} ‘good gracious’ (008), the term of address and interjection \textit{cuñado} ‘brother(-in-law)’ (010), the sentence particle \textit{ó} ‘\textsc{sp}’ (010)). Further, the conversation features two cognate objects (\textit{swít} ‘be tasty’ (006) and \textit{dáy} ‘die’ (016))\is{cognate objects}. The emphatic style of the text also transpires in the use of irrealis\is{irrealis modality} modality marking signalled by \textit{go} ‘\textsc{pot}’ in (009), (011) and (015); \textit{de} ‘\textsc{ipfv}’ in (010) and factative marking in (012) and (016)–(017). The hypothetical frame provides a backdrop to the boastful self-expression that characterises the conversation from (007) onwards. 


\largerpage
The video recording also reveals specific kinetic events that are characteristic for emphatic and self-expressive peer-to-peer communication in Pichi speech culture. For example, (ye) accompanies his interjection in (008) by a movement of the head and torso away from the speaker (ne). Equally, (ne) underlines his comment in (009) by getting up, walking briefly past (ye), and returning to sit on his stool, while laughing intensely. Both motion events are variations of what I assume to be an areal West African kinetic figure employed in certain genres of informal, interactional communication. In this figure, a person abruptly turns aways from the group during a communicative peak (i.e. after the punch line of a joke or an anecdote), describes a circular movement away from the group and joins it again after a brief moment, usually accompanied by laughing.

\setcounter{equation}{0}  % RJK the author numbers from 1 per text

\ea{ye07ga 001\\
\gll
A  sé,  wán  mi  kɔ́mpin  nɔ́,  e  bin  de   \\
\textsc{1sg.sbj}  \textsc{quot}  one  \textsc{1sg.poss}  friend  \textsc{neg}  \textsc{3sg.sbj}  \textsc{pst}  \textsc{ipfv}    \\
\gll
tɛ́l  mí  sé/  yu  sí  Obiang  Nguema,  dán  tɛ́n   \\
tell  \textsc{1sg.indp}  \textsc{quot}  \textsc{2sg}  see  \textsc{name}  that  time    \\
\gll
e  de  wɛ́r  sɔn  \textit{gafas}  dɛn  wé  dɛn  bl\H{a}k.   \\
\textsc{3sg.sbj}  \textsc{ipfv}  wear  some  glasses  \textsc{pl}  \textsc{sub}  \textsc{3pl}  be.black    \\
\glt
‘I say one of my friends, right, he was telling me that/ you see Obiang Nguema, that time he was wearing some glasses that were really dark.’
  }\ex{
ye07ga 002\\
\gll
E  sé  dɛn  bin  tɛ́l  mí  sé,  wé  e  kin  dé   \\
\textsc{3sg.sbj}  \textsc{quot}  \textsc{3pl}  \textsc{pst}  tell  \textsc{1sg.indp}  \textsc{quot}  \textsc{sub}  \textsc{3sg.sbj}  \textsc{hab}  \textsc{be.loc}    \\
\gll
na  \textit{estadio}  só,    yu  dé  na  \textit{estadio},  dɛn  de    mék   \\
\textsc{loc}  stadion  like.that  \textsc{2sg}  \textsc{be.loc}  \textsc{loc}  stadion  \textsc{3pl}  \textsc{ipfv}  make    \\
\gll
\textit{Copa de su Excelencia,}  dɛn  sé  dán  \textit{gafa},       e   \\
President’s.Cup             \textsc{3pl}  \textsc{quot}  that  glasses   \textsc{3sg.sbj}    \\
\gll
de  sí  ɔ́l  mán  nékɛd’.   \\
\textsc{ipfv}  see  all  man  be.naked    \\
\glt
‘He [my friend] said when he’s in the stadion like that, (when) you’re in the stadion (and) they’re doing the President’s Cup, they say (with) those glasses, he sees everybody naked.’
  }\ex{
ye07ga 003\\
\gll
Na  ín  wán  dé  a  bin  tɛ́l  wán  \textit{grand} \textit{frère}   \\
\textsc{foc}  \textsc{3sg.indp}  one  day  \textsc{1sg.sbj}  \textsc{pst}  tell  one  big    brother    \\
\gll
na,  na  mi  \textit{colegio}  dé,  a  tɛ́l=an  sé   \\
\textsc{loc}  \textsc{loc}  \textsc{1sg.poss}  college  there  \textsc{1sg.sbj}       tell=\textsc{3sg.obj}  \textsc{quot}    \\
\gll
“mi  mamá  de  \textit{viaja}  bɔkú,  a  go  tráy  mék   \\
\textsc{1sg.poss}  mother  \textsc{ipfv}  travel  much  \textsc{1sg.sbj}  \textsc{pot}  try  \textsc{sbjv}    \\
\gll
e  báy  mí  dán  káyn  \textit{gafas}    \textit{por}  \textit{Dios”.}\\
\textsc{3sg.sbj}  buy  \textsc{1sg.indp}  that  kind  glasses  by  God    \\
\glt
‘That’s why one day, I told one of my seniors [in French: ‘big brother’] in, in my secondary school there, I told him “my mother travels a lot, I’ll try to have her buy that kind of glasses for me by God”.’
  }\ex{
ye07ga 004\\
\gll
A  wánt  de  sí  ɔ́l  mán  nékɛd. \upshape{[laughter]}\\
\textsc{1sg.sbj}  want  \textsc{ipfv}  see  all  man  be.naked\\
\glt
‘I want to be seeing everybody naked.’
  }\ex{
ye07ga 005\\
\gll
A  wánt  dé    lɛk  Obiang  Nguema.   \\
\textsc{1sg.sbj}  want  \textsc{be.loc}  like    \textsc{name}    \\
\glt
‘I want to be like Obiang Nguema.’
  }\ex{
ye07ga 006\\
\gll
Dán  torí  bin  de  swít  mí  wán  swít.   \\
that  story  \textsc{pst}  \textsc{ipfv}  be.tasty  \textsc{1sg.indp}  one  be.tasty    \\
\glt
‘I was really enjoying that story.’
  }\ex{
ne07ga 007\\
\gll
A  fít  sé  if  yu  \textit{consigue}  \textit{gafas}  wé,  yu  go   \\
\textsc{1sg.sbj}  can  \textsc{quot}  if  \textsc{2sg}  obtain  glasses  \textsc{sub}  \textsc{2sg}  \textsc{pot}    \\
\gll
wɔ́k  na  ród.   \\
walk  \textsc{loc}  road    \\
\glt
‘I can tell you if you obtained glasses which, you would walk on the road.’
  }\ex{
ye07ga 008\\
\gll
Eyé  \upshape{[éjé::]}.   \\
\textsc{intj}    \\
\glt
‘Good gracious.’
  }\ex{
ne07ga 009\\
\gll
Dán  \textit{gafa}  yu  go  sl\H{i}p  wet=an.   \\
that  glasses  \textsc{2sg}  \textsc{pot}  sleep  with=\textsc{3sg.obj}    \\
\glt
‘Those glasses, you would sleep with them.’
  }\ex{
ye07ga 010\\
\gll
A  de  slíp  wet=an  \textit{cuñado}.   \upshape{[laughter]}\\
\textsc{1sg.sbj}  \textsc{ipfv}  sleep  with=\textsc{3sg.obj}  brother-in-law    \\
\glt
‘I would sleep with them brother.’
  }\ex{
ye07ga 011\\
\gll
A  go  púl=an  na  mi  yáy  sé  wétin?   \\
\textsc{1sg.sbj}  \textsc{pot}  remove=\textsc{3sg.obj}  \textsc{loc}  \textsc{1sg.poss}  eye  \textsc{quot}  what    \\
\glt
‘I would remove them from my eyes for what?’
  }\ex{
ye07ga 012\\
\gll
A  w\H{a}nt  dé  \textit{flipado}  ɔl  áwa,  ɔl~\upshape{[ɔ::l]}  áwa.   \\
\textsc{1sg.sbj}  want  \textsc{be.loc}  turned.on  all  hour  all      hour    \\
\glt
‘I would want to be turned on all the time, all the time.’
  }
\newpage
\ex{
ye07ga 013\\
\gll
Ɔ́l  áwa.   \\
all  hour    \\
\glt
‘All the time.’
  }\ex{
ne07ga 014\\
\gll
A  sé,  na  fɔ  tɔ́k  fɔ  dán (…)   \\
\textsc{1sg.sbj}  \textsc{quot}  \textsc{foc}  \textsc{prep}  talk  \textsc{prep}  that    \\
\glt
‘I say, one has to talk about that (…)
  }\ex{
ne07ga 015\\
\gll
Yu  go  lás  sí  sɔn  nékɛd  wé  na  ín  go   \\
\textsc{2sg}  \textsc{pot}  end.up  see some  be.naked     \textsc{sub}  \textsc{foc}  \textsc{3sg.indp}  \textsc{pot}    \\
\gll
mék  mék  yu  yáy  p\H{e}n  ó.   \\
make  \textsc{sbjv}  \textsc{2sg}  eye  pain  \textsc{sp}    \\
\glt
‘You’ll end up seeing some (kind of) nakedness that will really make your eyes pain.’
  }\ex{
ne07ga 016\\
\gll
Ey,  dán  káyn  spɛ́tikul  a  dáy  dáy.   \\
\textsc{intj}  that  kind  glasses  \textsc{1sg.sbj}  die  death    \\
\glt
‘That kind of glasses, I would really die.’
  }\ex{
ne07ga 017\\
\gll
Wé  yu  tɛ́l  húman,  “lúk  di  wán,  yu  wánt  tɔ́k  wet   \\
\textsc{sub}  \textsc{2sg}  tell  woman  look  \textsc{def}  one  \textsc{2sg}  want  talk  with    \\
\gll
mí?”\\
\textsc{1sg.indp}    \\
\glt
‘And you would say to women, “look at this one, you (actually) want to talk to me [now that I have seen all of you]?”’
  }\ex{
ye07ga 018\\
\gll
Yú,  yú?   \\
\textsc{2sg.indp}  \textsc{2sg.indp}    \\
\glt
‘You, you?’ [laughter]
  }\ex{
ne07ga 019\\
\gll
Kɔmɔ́t!   \\
go.out    \\
\glt
‘Get lost!’
  }\ex{
ye07ga 020\\
\gll
Aa,  kɔmɔ́t  dé!   \\
\textsc{intj}  go.out  \textsc{be.loc}    \\
\glt
‘Just get lost there!’
  }
\newpage
\ex{
ye07ga 021\\
\gll
\textit{¡Fuera!}   \\
outside    \\
\glt
‘Out!’
  }\ex{
la07ga 022\\
\gll
\textit{Te}  \textit{van}  \textit{a}  \textit{matar.}   \\
you  they.will  to  kill    \\
\glt
‘They [the women] will kill you.’
}\z
\section{Routine procedure: Preparing corn-porridge}

Below follows a procedural text in which Djunais (dj) explains to me (ko) and Lage (ge) how to prepare \textit{ógi} ‘corn porridge’. The text features the type of TMA marking characteristic for this narrative genre. Procedural texts may exhibit more than other genres the regular use of factative TMA marking (bare verbs) in order to describe routine procedures and when giving instructions (e.g. (001)–(005)). Likewise the text contains many instances of bare, non-initial verbs typical of clause chaining\is{clause chaining} (e.g. \textit{trowé=an} ‘pour=\textsc{3sg.obj}’ (040) \textit{bigín} (043) and \textit{pút=an} ‘put={}\textsc{3sg.obj}’ (051).


A second way of expressing (hypothetical) routines appears in (018)–(020). Here the potential mood marker \textit{go} ‘\textsc{pot}’ is used when (dj) briefly digresses to compare the preparation of \textit{ógi} with that of rice porridge. The text also contains a few instances of unexpressed subjects (\textit{sifta} ‘sift’ (007), \textit{fít} ‘can’ (008)) as well as a brief conversation (021)–(034) after which (dj) quickly turns back to describing the cooking:

\setcounter{equation}{0}  % RJK the author numbers from 1 per text

\ea{ko03do 001\\
\gll
Djunais  a  bɛ́g  \textit{explica}  mí.   \\
\textsc{name}  \textsc{1sg.sbj}  ask.for  explain  \textsc{1sg.indp}    \\
\glt
‘Djunais, please explain to me [how to prepare maize porridge].’
  }\ex{
dj03do 002\\
\gll
A  \textit{ralla}  di,  di,  di  \textit{maíz.}   \\
\textsc{1sg.sbj}  grate  \textsc{def}  \textsc{def}  \textsc{def}  corn    \\
\glt
‘I grate(d) the, the corn.
  }\ex{
ge03do 003\\
\gll
Yu  ték  di  \textit{maíz}  yu  hól=an.   \\
\textsc{2sg}  take  \textsc{def}  corn  \textsc{2sg}  hold=\textsc{3sg.obj}    \\
\glt
‘You take the corn and hold it.’
  }\ex{
dj03do 004\\
\gll
A  \textit{ralla}  in  wet  \textit{rallador}.   \\
\textsc{1sg.sbj}  grate  \textsc{3sg.indp}  with  grater    \\
\glt
‘I grate it with a a grater.’
  }\ex{
dj03do 005\\
\gll
Wé  a  \textit{ralla}  ín,  a  sifta  ín.   \\
\textsc{sub}  \textsc{1sg.sbj}  grate  \textsc{3sg.indp}  \textsc{1sg.sbj}  sift  \textsc{3sg.indp}    \\
\glt
‘When I have grated it, I sift it.’
  }\ex{
dj03do 006\\
\gll
Ɔ́l  dán  watá  dɛn  a  nó  de  pút  dɛ́n  ínsay.   \\
all  that  water  \textsc{pl}  \textsc{1sg.sbj}  \textsc{neg}  \textsc{ipfv}  put  \textsc{3pl.indp}  inside    \\
\glt
‘All that water, I don’t put it inside.’
  }\ex{
dj03do 007\\
\gll
Sifta,  wé  a  dɔ́n  sifta  ín,  e  de  lɛ́f   \\
sift  \textsc{sub}  \textsc{1sg.sbj}  \textsc{prf}  sift  \textsc{3sg.indp}  \textsc{3sg.sbj}  \textsc{ipfv}  remain    \\
\gll
wet  di  watá.   \\
with  \textsc{def}  water    \\
\glt
‘Sift (it), when I have sifted it, it remains with the water.’
  }\ex{
dj03do 008\\
\gll
Fít  sifta  ín  sóté  tú  tɛ́n  mék  mék  dán   \\
can  sift  \textsc{3sg.indp}  until  two  time  make  \textsc{sbjv}  that    \\
\gll
smɔ́l    smɔ́l  watá  dɛn  nó  lɛ́f.   \\
small   \textsc{rep}  water  \textsc{pl}  \textsc{neg}  remain    \\
\glt
‘(You) can sift it up to two times to make that little bit of water not remain.’
  }\ex{
dj03do 009\\
\gll
Sɔn  dé  yet  sɛ́f  wé  a  nó  mék,  \textit{entonces}  dán   \\
some  \textsc{be.loc}  yet  \textsc{foc}  \textsc{sub}  \textsc{1sg.sbj}  \textsc{neg}  make  so  that    \\
\gll
wán  wé  lɛ́f,  una  fít  kɛ́r=an  gó  \textit{aunque}  ínsay   \\
one  \textsc{sub}  remain  \textsc{2pl}  can  carry=\textsc{3sg.obj}  go  like  inside    \\
\gll
wán  bɔ́tul  fɔ  wán  \textit{mineral}  una  pút=an,  na   \\
one  bottle  \textsc{prep}  one  mineral  \textsc{2pl}  put=\textsc{3sg.obj}  \textsc{loc}    \\
\gll
\textit{congelador}.   \\
fridge    \\
\glt
‘Some still remains that I didn’t make, so that one that remains, you [\textsc{pl}] can put it inside a mineral (water) bottle and put it into the fridge.’
  }\ex{
dj03do 010\\
\gll
Wé  yu  de  mék=an  na  hós,  jɔ́s  ték=an   \\
\textsc{sub}  \textsc{2sg}  \textsc{ipfv}  make=\textsc{3sg.obj}  \textsc{loc}  house  just  take=\textsc{3sg.obj}    \\
\gll
pút=an  na  pɔ́t  \textit{aunque}  wán  \textit{tasa}  só.   \\
put=\textsc{3sg.obj}  \textsc{loc}  \textsc{pot}  like  one  cup  like.that    \\
\glt
‘When you make it at home, just take it and put it into a pot, approximately one cup or so.’
  }\ex{
dj03do 011\\
\gll
If  yu  de  mék=an  só    e  go  bɔkú      \textit{pero}   \\
if  \textsc{2sg}  \textsc{ipfv}  make=\textsc{3sg.obj}  like.that  \textsc{3sg.sbj}  \textsc{pot}  become.much  but    \\
\gll
na  só  e            gɛ́fɔ    dé.   \\
\textsc{foc}  like.that   \textsc{3sg.sbj}   have.to  \textsc{be.loc}    \\
\glt
‘If you do it like that it will be(come) much but that’s how it has to be.’
  }\ex{
dj03do 012\\
\gll
Wé  a  \textit{ralla}  ín,  a  mék=an,  pút  di  pɔ́t   \\
\textsc{sub}  \textsc{1sg.sbj}  grate  \textsc{3sg.indp}  \textsc{1sg.sbj}      make=\textsc{3sg.obj}  put  \textsc{def}  pot    \\
\gll
na  fáya  wet  smɔ́l  watá,  a  bigín  de  pút  dán   \\
\textsc{loc}  fire  with  small  water  \textsc{1sg.sbj}  begin  \textsc{ipfv}  put  that    \\
\gll
\textit{mezcla}  dé  sóté  e  dé    só      sénwe.   \\
mixture  there  until  \textsc{3sg.sbj}  \textsc{be.loc}  like.that    \textsc{foc}    \\
\glt
‘When I  grated it, I make it, (I) put the \textsc{pot} on the fire with a bit of water, I begin to put that mixture in there until it is just like this.’
  }\ex{
dj03do 013\\
\gll
\textit{Igual}  sɛ́f  wet  di  wán  fɔ  rɛ́s.   \\
equal  \textsc{foc}  with  \textsc{def}  one  \textsc{prep}  rice    \\
\glt
‘The same with the one (made) with rice.’
  }\ex{
ko03do 014\\
\gll
So  wán  dé  fɔ  rɛ́s  sɛ́f?   \\
so  one  \textsc{be.loc}  \textsc{prep}  rice  \textsc{foc}    \\
\glt
‘So there’s one made with rice, too?’
  }\ex{
ko03do 015\\
\gll
Na  di  sén  fásin  fɔ  dú=an?   \\
\textsc{foc}  \textsc{def}  same  manner  \textsc{prep}  do=\textsc{3sg.obj}    \\
\glt
‘Is it done the same way?’
  }\ex{
dj03do 016\\
\gll
Yu/  rɛ́s,  yu  de  bít=an.   \\
\textsc{2sg}  rice  \textsc{2sg}  \textsc{ipfv}  beat=\textsc{3sg.obj}    \\
\glt
‘You/ as for rice, you beat it.’
  }\ex{
dj03do 017\\
\gll
Bít=an  yu  mék=an  só      sɛ́f.   \\
beat=\textsc{3sg.obj}  \textsc{2sg}  make=\textsc{3sg.obj}  like.that  \textsc{foc}    \\
\glt
‘(You) beat it (and) make just like this.’
  }\ex{
ko03do 018\\
\gll
So  yu  go  bít  di  rɛ́s?   \\
so  \textsc{2sg}  \textsc{pot}  beat  \textsc{def}  rice    \\
\glt
‘So you beat the rice?’
  }
  \largerpage
  \ex{
dj03do 019\\
\gll
Yu  go  \textit{moja}  di  rɛ́s  na  watá,  fɔ  tidé,  tú  dé,   \\
\textsc{2sg}  \textsc{pot}  soak  \textsc{def}  rice  \textsc{loc}  water  \textsc{prep}  today  two  day    \\
\gll
lɛk  háw  yu  wánt  nɔ́,  di  dé  yu  de  \textit{calcula}  sé   \\
like  how  \textsc{2sg}  want  \textsc{neg}  \textsc{def}  day  \textsc{2sg}  \textsc{ipfv}  calculate  \textsc{quot}    \\
\gll
yu  wánt  chɔ́p=an.   \\
\textsc{2sg}  want  eat=\textsc{3sg.obj}    \\
\glt
‘You soak it in water, for today [one day], two days, as you want, right, the (number of) days you calculate that you want to eat it.’
  }\ex{
dj03do 020\\
\gll
Yu  wánt  chɔ́p=an  tú  dé  áfta,  yu  go   \\
\textsc{2sg}  want  eat=\textsc{3sg.obj}  two  day  then  \textsc{2sg}  \textsc{pot}    \\
\gll
mék=an  mék  e  dé  na  watá.   \\
make=\textsc{3sg.obj}  \textsc{sbjv}  \textsc{3sg.sbj}  \textsc{be.loc}  \textsc{loc}  water    \\
\glt
‘(If) you want to eat it two days afterwards, you make it be in the water [for that time].’
  }\ex{
ko03do 021\\
\gll
\'{U}s=say  yu  lán  fɔ  kúk?   \\
\textsc{q}=side    \textsc{2sg}  learn  \textsc{prep}  cook    \\
\glt
‘Where did you learn to cook?’
  }\ex{
dj03do 022\\
\gll
A  gó  skúl.   \\
\textsc{1sg.sbj}  go  school    \\
\glt
‘I went to school.’
  }\ex{
dj03do 023\\
\gll
A  gó  skúl  \textit{pero}  ɔ́l  di  smɔ́l  tín  dɛn   \\
\textsc{1sg.sbj}  go  school  but  all  \textsc{def}  small  thing  \textsc{pl}    \\
\gll
yá  só    na  tín  dɛn  wé  mí         de  mék=an  na   \\
here like.that  \textsc{foc}  thing  \textsc{pl}  \textsc{sub} \textsc{1sg.indp}  \textsc{ipfv}  make=\textsc{3sg.obj}  \textsc{loc}    \\
\gll
hós.   \\
house    \\
\glt
‘I went to school but all the small things here are things that I make at home.’
  }\ex{
dj03do 024\\
\gll
\textit{Pero},  Sita  bin  dé  nɔ́,  mamá.   \\
but  \textsc{name}  \textsc{pst}  \textsc{be.loc}  \textsc{neg}  mother    \\
\glt
‘But Sita was (still) around [alive], right, mother.’
  }\ex{
dj03do 025\\
\gll
\textit{Porque}  na  mí  mí  de  \textit{prepara}  ɔ́l  tín.   \\
because  \textsc{foc}  \textsc{1sg.indp}  \textsc{1sg.indp}  \textsc{ipfv}  prepare  all  thing    \\
\glt
‘Because it’s me, I [\textsc{emp}] prepare everything.’
  }\ex{
ko03do 026\\
\gll
Yu  húman  go  gládin.   \\
\textsc{2sg}  woman  \textsc{pot}  be.glad    \\
\glt
‘Your wife will be happy.’
  }\ex{
ko03do 027\\
\gll
Na  Djunais  go  kúk  fɔ  in  fámbul.   \\
\textsc{foc}  \textsc{name}  \textsc{pot}  cook \textsc{ass}  \textsc{3sg.poss}  family    \\
\glt
‘It’s Djunais who’ll cook for his family.’
  }
\newpage  
  \ex{
ko03do 028\\
\gll
Rubi  gó  Lubá?   \\
\textsc{name}  go  \textsc{place}    \\
\glt
‘Did Rubi go to Lubá?’
  }\ex{
dj03do 029\\
\gll
Yɛ́stadé.   \\
yesterday    \\
\glt
‘Yesterday.’
  }\ex{
ge03do 030\\
\gll
\'{U}dat,  Rubi?   \\
who  \textsc{name}    \\
\glt
‘Who, Rubi?’
  }\ex{
ko03do 031\\
\gll
\'{U}s=dé      e       go  tɔ́n  bák?   \\
\textsc{q}=day  \textsc{3sg.sbj}   \textsc{pot}  turn  back    \\
\glt
‘When will he return?’
  }\ex{
dj03do 032\\
\gll
E  fít  kán  tumára.   \\
\textsc{3sg.sbj}  can  come  tomorrow    \\
\glt
‘He might come tomorrow.’
  }\ex{
ge03do 033\\
\gll
E  gó  wet  in  mamá?   \\
\textsc{3sg.sbj}  go  with  \textsc{3sg.poss}  mother    \\
\glt
‘Did he go with his mother?’
  }\ex{
dj03do 034\\
\gll
Wet  in  smɔ́l  brɔ́da.   \\
with  \textsc{3sg.poss}  small  brother    \\
\glt
‘With his little brother.’
  }\ex{
dj03do 035\\
\gll
A  sé  dís  tín  yá  só,    ɛf  di  kɔ́n  bin  bɔkú   \\
\textsc{1sg.sbj}  \textsc{quot}  this  thing  here  like.that  if  \textsc{def}  corn  \textsc{pst}  be.much    \\
\gll
lɛk,  di  watá  náw  só,    di  watá/\\
like  \textsc{def}  water  now  like that  \textsc{def}  water\\
\glt
‘I say this thing right here, if the corn was a lot like, the water now, the water/’
  }\ex{
ge03do 036\\
\gll
Dán  tín  na  di  \textit{pasta}.   \\
that  thing  \textsc{foc}  \textsc{def}  paste    \\
\glt
‘That is the paste.’
  }\ex{
dj03do 037\\
\gll
Di  \textit{pasta}  yɛ́s.   \\
\textsc{def}  paste  yes    \\
\glt
‘The paste, yes.’
  }\ex{
dj03do 038\\
\gll
Na  di  tín,  na  ín  a  níd  fɔ  mék  di  \textit{pasta}   \\
\textsc{foc}  \textsc{def}  thing  \textsc{foc}  \textsc{3sg.indp}  \textsc{1sg.sbj}  need  \textsc{prep}  make  \textsc{def}  paste    \\
\gll
\textit{porque}  dɛn  de  sɛ́l=an  \textit{simple}  só.   \\
because  \textsc{3pl}  \textsc{ipfv}  sell=\textsc{3sg.obj}  simple  like.that    \\
\glt
‘That’s it, that’s what I need to make the paste because it [the flour] is sold simple like that.’
  }\ex{
dj03do 039\\
\gll
Yu  fɔ  trowé di  watá  yá    só,      na  háw  só      di  tín   \\
\textsc{2sg}  \textsc{prep}  pour   \textsc{def}  water  here  like.that  \textsc{foc}  how  like.that  \textsc{def}  thing    \\
\gll
bin  fɔ  lɛ́f  bɔtɔ́n.   \\
\textsc{pst}  \textsc{cond}  remain  bottom    \\
\glt
‘You have to pour this water here away, that’s how the thing should have remained at the bottom.’
  }\ex{
dj03do 040\\
\gll
Pero  e  bin  fɔ  lɛ́f  bɔkú  sé  \textit{de}  \textit{tal}  \textit{forma}   \\
but  \textsc{3sg.sbj}  \textsc{pst}  \textsc{cond}  remain  much  \textsc{quot}  of  so  form    \\
\gll
\textit{que}  \textit{sí,}  a  fít  ték  di  wɔtá  a  trowé=an,   \\
that  yes  \textsc{1sg.sbj}  can  take  \textsc{def}  water  \textsc{1sg.sbj}  pour=\textsc{3sg.obj}    \\
\gll
lɛ́f  di  pán  na  sán,  e  dráy  e  lɛ́f  lɛkɛ   \\
leave  \textsc{def}  pan  \textsc{loc}  sun  \textsc{3sg.sbj}  be.dry  \textsc{3sg.sbj}  remain  like    \\
\gll
garí  náw.   \\
gari  now    \\
\glt
‘But enough should have remained in such way that, yes, I can take the water and pour it away, leave the pan in the sun, (and then) it dries and remains like gari now.’
  }\ex{
ko03do 041\\
\gll
Na  só    a  sabí=an      sɛ́f.   \\
\textsc{foc}  like.that  \textsc{1sg.sbj}    know=\textsc{3sg.obj}  \textsc{foc}    \\
\glt
‘That’s how I know it, too.’
  }
\largerpage  
  \ex{
dj03do 042\\
\gll
Lɛk  háw  dɛn  de  mék=an  yu  de  sí  na   \\
like  how  \textsc{3pl}  \textsc{ipfv}  make=\textsc{3sg.obj}  \textsc{2sg}  \textsc{ipfv}  see  \textsc{foc}    \\
\gll
kɔsta  nɔ́,  wán  kɔsta,  sɔntɛ́n  na  só  dɛn  de  mék   \\
custard  \textsc{neg}  one  custard  perhaps  \textsc{foc}  like.that  \textsc{3pl}  \textsc{ipfv}  make    \\
\gll
ɔ  dɛn  de  pút  dán  \textit{colorante}  ínsay  wé  e  de   \\
or  \textsc{3pl}  \textsc{ipfv}  put  that  colourant  inside  \textsc{sub}  \textsc{3sg.sbj}  \textsc{ipfv}    \\
\gll
chénch.   \\
change    \\
\glt
‘The way it’s done, you see it’s a custard, a (kind of) custard, it may be done like that or that colourant that changes (the colour) is put inside.’
  }\ex{
dj03do 043\\
\gll
E  tɔ́n  \textit{arena},  dán  \textit{água}  dé  a  fít  ték  wán   \\
\textsc{3sg.sbj}  turn  sand  that  water  \textsc{be.loc}  \textsc{1sg.sbj}  can  take  one    \\
\gll
spún,  a  bigín  de  mék=an  \textit{normal.}   \\
spoon  \textsc{1sg.sbj}  begin  \textsc{ipfv}  make=\textsc{3sg.obj}  normal    \\
\glt
‘(When) it turns into sand [farina], that water over there, I can take a spoon (of it) and begin to make normally.’
  }\ex{
dj03do 044\\
\gll
\textit{Pero}  \textit{como}  di  arena  tú  lílí-lí,         kɔ́n  tú  smɔ́l   \\
but  because  \textsc{def}  sand  too  little-\textsc{rep}    corn  too   be.small    \\
\gll
náw,  a  \textit{mezcla}  ín  ɔ́l.   \\
now  \textsc{1sg.sbj}  mix  \textsc{3sg.indp}  all    \\
\glt
‘But since the sand [farina] is too little, the corn is too little now, I mixed all of it [in making the porridge].’
  }\ex{
dj03do 045\\
\gll
Wɛ́n  a  go  kliár=an,  sɔn,  bɔtɔ́n  mɔ́,  e   \\
\textsc{sub}  \textsc{1sg.sbj}  \textsc{pot}  clear=\textsc{3sg.obj}  some  bottom  more  \textsc{3sg.sbj}    \\
\gll
go  kán  gɛ́t  dí  tín.   \\
\textsc{pot}  come  get  this  thing    \\
\glt
‘When I clear it, some, at the bottom again, it will come to have this thing.’
  }\ex{
dj03do 046\\
\gll
\textit{Pero}  dí  watá,  una  nó  trowé=an  \textit{lo}  \textit{que}  \textit{sí,}  una   \\
but  this  water  \textsc{2pl}  \textsc{neg}  pour=\textsc{3sg.obj}    \textsc{def}  that  yes  \textsc{2pl}    \\
\gll
sí  sé  e  dɔ́n  slíp  e  dɔ́n  slíp  fáyn,  dí   \\
see  \textsc{quot}  \textsc{3sg.sbj}  \textsc{prf}  sleep  \textsc{3sg.sbj}  \textsc{prf}  lie  fine  this    \\
\gll
watá  dɔ́n  \textit{baja}.   \\
water  \textsc{prf}  go.down    \\
\glt
‘But the water, you [\textsc{pl}] don’t pour it away, rather, you [\textsc{pl}] see that it has settled, it has settled nicely, the water has gone down.’
  }\ex{
dj03do 047\\
\gll
Dí  tín  dɔ́n  \textit{baja},  wé  di  watá  una  dɔ́n  de   \\
this  thing  \textsc{prf}  go.down  \textsc{sub}  \textsc{def}  water  \textsc{2pl}  \textsc{prf}  \textsc{ipfv}    \\
\gll
sí=an  ɔ́p  lɛkɛ  sé  na  watá  \textit{normal}.   \\
see=\textsc{3sg.obj}  up  like  \textsc{quot}  \textsc{foc}  water  normal    \\
\glt
‘The thing [farina] has gone down, and as for the water, you [\textsc{pl}] see it above as if it were normal water.’
  }\ex{
dj03do 048\\
\gll
Ɛf  yu  ték  dán  watá  dé,  yu  trowé=an,  yu   \\
if  \textsc{2sg}  take  that  water  there  \textsc{2sg}  pour=\textsc{3sg.obj}    \textsc{2sg}    \\
\gll
trowé=an  \textit{pero}  mék  e,  yu  fít  ték  dán  watá   \\
pour=\textsc{3sg.obj}    but  \textsc{sbjv}  \textsc{3sg.sbj}  \textsc{2sg}  can  take  that  water    \\
\gll
yu  trowé=an  yu  pút  ɔ́da  nyú  wán  ínsay,  dán   \\
\textsc{2sg}  pour=\textsc{3sg.obj}    \textsc{2sg}  put  other  new  one  inside  that    \\
\gll
wán  sé  mék  e  nó  smɛ́l.   \\
one  \textsc{quot}  \textsc{sbjv}  \textsc{3sg.sbj}  \textsc{neg}  smell    \\
\glt
‘If you take that water, you pour it away, but let it, you can take that water and you pour it away and you put another new one [water] inside, that is in order for it not to smell.’
  }\ex{
dj03do 049\\
\gll
\textit{Porque}  e  de  \textit{sigue}    wán  bád  smɛ́l.   \\
because  \textsc{3sg.sbj}  \textsc{ipfv}  follow  one  bad  smell    \\
\glt
‘Because (otherwise) a bad smell follows.’
  }\ex{
ge03do 050\\
\gll
\'{A}fta  háw  fɔ  mék  di  ógi?   \\
then  how  \textsc{prep}  make  \textsc{def}  corn.porridge    \\
\glt
‘Then how do you make the corn porridge?’
  }\ex{
dj03do 051\\
\gll
Yu  fít  ték  náw,  wán,  wán  smɔ́l  kɔ́p  nɔ́,  yu   \\
\textsc{2sg}  can  take  now  one  one  small  cup  \textsc{neg}  \textsc{2sg}    \\
\gll
pút=an  na  fáya,  ínsay  di  pɔ́t.   \\
put=\textsc{3sg.obj}  \textsc{loc}  fire  inside  \textsc{def}  pot    \\
\glt
‘Now you can take, a, a small cup, right, you put it on the fire, inside the pot.’
  }\ex{
dj03do 052\\
\gll
Dásɔl,  wán  smɔ́l,  wán  glás,  yu  fúlɔp=an.   \\
only  one  small  one  glass  \textsc{2sg}  fill=\textsc{3sg.obj}    \\
\glt
‘Only, one small, one glass, you fill it up.’
  }\ex{
ge03do 053\\
\gll
Wán  glás  watá.   \\
one  glass  water    \\
\glt
‘A glass of water.’
  }\ex{
dj03do 054\\
\gll
Ɛhɛ́,  wán  glás  watá  \textit{aparte},    yu  pút=an  ínsay,   \\
exactly  one  glass  water  apart  \textsc{2sg}  put=\textsc{3sg.obj}  inside    \\
\gll
dán  wán  dé  yu  fít  ték  \textit{medio}  fɔ  dán  sén  glás,   \\
that  one  there  \textsc{2sg}  can  take  half  \textsc{prep}  that  same  glass    \\
\gll
fɔ  dí  tín  yá.   \\
\textsc{prep}  this  thing  here    \\
\glt
‘Exactly, a glass of water apart, you put it inside, as for that one you can take half in that same glass, in this thing here.’
  }\ex{
dj03do 055\\
\gll
Yu  de  tɔ́n=an,  yu  nó  fít,  yu  nó  \textit{para}  \textit{así,}   \\
\textsc{2sg}  \textsc{ipfv}  turn=\textsc{3sg.obj}  \textsc{2sg}  \textsc{neg}  can  \textsc{2sg}  \textsc{neg}  stop  like.this    \\
\gll
mék  yu  tɔ́n=an  \textit{porque}  bɔtɔ́n  go  rós.   \\
make  \textsc{2sg}  turn=\textsc{3sg.obj}  because  bottom  \textsc{pot}  burn    \\
\glt
‘You turn it, you can’t, you don’t stop like that, turn it because the bottom might burn.’
  }\ex{
dj03do 056\\
\gll
E  go  rós  e  go  lɛ́f  lɛkɛ  pan-kék.   \\
\textsc{3sg.sbj}  \textsc{pot}  burn  \textsc{3sg.sbj}  \textsc{pot}  remain  like  pan.\textsc{cpd}{}-cake    \\
\glt
‘It might burn and become like pancake.’
  }\ex{
dj03do 057\\
\gll
Yu  gɛ́fɔ    de  tɔ́n=an.   \\
\textsc{2sg}  have.to  \textsc{ipfv}  turn=\textsc{3sg.obj}    \\
\glt
‘You have to be turning it.’
  }\ex{
dj03do 058\\
\gll
Tɔ́n=an  tɔ́n=an,  mék  yu  nó  \textit{para}  sóté  mék   \\
turn=\textsc{3sg.obj}  turn=\textsc{3sg.obj}  \textsc{sbjv}  \textsc{2sg}  \textsc{neg}  stop  until  \textsc{sbjv}    \\
\gll
e  tík  lɛk  háw  e  bin  dé    só.   \\
\textsc{3sg.sbj}  be.thick  like  how  \textsc{3sg.sbj}  \textsc{pst}  \textsc{be.loc}  so    \\
\glt
‘Turn it, turn it, don’t stop until it is thick, just the way it was (here).’
  }\ex{
dj03do 059\\
\gll
\textit{Pero}  ɛf  di  tín  kán  bɔkú  mɔ́  pás  di  watá,   \\
but  if  \textsc{def}  thing  \textsc{pfv}  be.much  more  pass  \textsc{def}  water    \\
\gll
e  go  lɛ́f  wán  \textit{pasta},  e  go  lɛ́f  lɛkɛ,   \\
\textsc{3sg.sbj}  \textsc{pot}  remain  one  paste  \textsc{3sg.sbj}  \textsc{pot}  remain  like    \\
\gll
pan-kék  wán  tín  só,         e        go  tú  tík.   \\
pan.\textsc{cpd}{}-cake  one  thing  like.that   \textsc{3sg.sbj}   \textsc{pot}  too  become.thick    \\
\glt
‘But if the thing has become more than the water, a paste will remain, it will become like a kind of pancake, it will become too thick.’
}\z
\section{Elicitation: Caused positions}

The text below results from the elicitation of “caused positions” with the help of the corresponding set of video clips that form part of the “Manual for the field season 2001” of the Language and Cognition Group of the Max-Planck-Institute for Psycholinguistics in Nijmegen. Like most elicitations in the corpus, this one was conducted with two (or more) speakers – Lindo (li) and Djunais (dj) – simultaneously. The elicitation shows in an exemplary way the use of the intransitive/inchoative-stative vs. transitive/dynamic variants of Pichi locative verbs\is{locative verbs}. It features numerous other verbs with a spatial meaning component as well (e.g. \textit{pút} ‘put’ and \textit{dé} \textsc{‘be.loc’}).

\newpage 
\setcounter{equation}{0}  % RJK the author numbers from 1 per text

\ea{li07pe 001\\
\gll
E  pút  wán  písis  pantáp  tébul.   \\
\textsc{3sg.sbj}  put  one  piece.of.cloth  on  table    \\
\glt
‘She put a cloth on the table.’
  }\ex{
li07pe 002\\
\gll
Na  róp  dat.   \\
\textsc{foc}  rope  that    \\
\glt
‘That’s a rope.’
  }\ex{
li07pe 003\\
\gll
If  a  sé  dɛn  de  híb  sɔn  tín  na  dán  stík.   \\
if  \textsc{1sg.sbj}  \textsc{quot}  \textsc{3pl}  \textsc{ipfv}  throw  some  thing  \textsc{loc}  that  tree    \\
\glt
‘If I said they’re throwing something at that stick.’
  }\ex{
li07pe 004\\
\gll
Pero  údat  de  híb=an?   \\
but  who  \textsc{ipfv}  throw=\textsc{3sg.obj}    \\
\glt
‘But who is throwing it?’
  }\ex{
li07pe 005\\
\gll
\'{A}fta  di  róp  sɛ́f  wi  nó  sí  nó  mán  wé  e   \\
then  \textsc{def}  rope  \textsc{foc}  \textsc{1pl}  \textsc{neg}  see  \textsc{neg}  man  \textsc{sub}  \textsc{3sg.sbj}    \\
\gll
híb=an.   \\
throw=\textsc{3sg.obj}    \\
\glt
‘Then, even the rope, we didn’t see anybody who threw it.’
  }\ex{
li07pe 006\\
\gll
Wétin  e  hɛ́ng  dé?   \\
what  \textsc{3sg.sbj}  hang  there    \\
\glt
‘What’s hanging there?’
  }\ex{
li07pe 007\\
\gll
Na  brís  sék=an?   \\
\textsc{foc}  air  shake=\textsc{3sg.obj}    \\
\glt
‘Is it the air that shook it?’
  }\ex{
dj07pe 008\\
\gll
Sí  di  róp  ɔ́p  dé?   \\
see  \textsc{def}  rope  up  there    \\
\glt
‘(Do you) see the rope up there?’
  }\ex{
li07pe 009\\
\gll
Dɛn  jɔ́s  de  híb=an,  áfta  e  hɛ́ng.   \\
\textsc{3pl}  just  \textsc{ipfv}  throw=\textsc{3sg.obj}  then  \textsc{3sg.sbj}  hang    \\
\glt
‘It’s just being thrown, then it hangs.’
  }
\newpage 
  \ex{
li07pe 010\\
\gll
Nó  nátin  nó  dé  na  di  tébul.   \\
\textsc{neg}  nothing  \textsc{neg}  \textsc{be.loc}  \textsc{loc}  \textsc{def}  table    \\
\glt
‘Nothing is on the table.’
  }\ex{
li07pe 011\\
\gll
Nó  nátin  nó  dé          pantáp=an.   \\
\textsc{neg}  nothing  \textsc{neg}  \textsc{be.loc}  on=\textsc{3sg.obj}    \\
\glt
‘Nothing is on it.’
  }\ex{
li07pe 012\\
\gll
Náw  sɔn  tín  dɔ́n  dé    pan  di  tébul  wé  na  \textit{haricots}  dɛn.   \\
now  some  thing  \textsc{prf}  \textsc{be.loc}  on  \textsc{def}  table  \textsc{sub}  \textsc{foc}  beans  \textsc{pl}    \\
\glt
‘Now something is on the table that’s beans [<French ‘haricots’].’
  }\ex{
li07pe 013\\
\gll
Di  húman,  e  bríng  di  tú  bɔ́l  dɛn  pan  di  tébul.   \\
\textsc{def}  woman  \textsc{3sg.sbj}  bring  \textsc{def}  two  ball  \textsc{pl}  on  \textsc{def}  table    \\
\glt
‘The woman, she brought the two balls onto the table.’
  }\ex{
li07pe 014\\
\gll
E  kán  mék  di  sén  tín  nɔ́.   \\
\textsc{3sg.sbj}  come  make  \textsc{def}  same  thing  \textsc{neg}    \\
\glt
‘She did the same thing, right?’
  }\ex{
dj07pe 015\\
\gll
Fɔ́s  e  fíba  sé  dɛn  bin  dɔ́n  dɔ́n.   \\
first  \textsc{3sg.sbj}  resemble  \textsc{quot}  \textsc{3pl}  \textsc{pst}  \textsc{prf}  be.done    \\
\glt
‘First, it seemed that they [the beans] were done [cooked].’
  }\ex{
dj07pe 016\\
\gll
Náw  só  dɛn  nó  dɔ́n,  yu  sí?   \\
now  so  \textsc{3pl}  \textsc{neg}  be.done    \textsc{2sg}  see    \\
\glt
‘Right now they aren’t done, you see?’
  }\ex{
dj07pe 017\\
\gll
Náw  fɔ́s  \textit{haricots}  dɛn  bin  dɔ́n  kúk.   \\
now  first  beans  \textsc{pl}  \textsc{pst}  \textsc{prf}  cook    \\
\glt
‘Now first, the beans were cooked.’
  }\ex{
li07pe 018\\
\gll
A  nó  tínk.   \\
\textsc{1sg.sbj}  \textsc{neg}  think    \\
\glt
‘I don’t think (so).’
  }
\newpage 
\ex{
li07pe 019\\
\gll
Na  di  sén  tín.   \\
\textsc{foc}  \textsc{def}  same  thing    \\
\glt
‘It’s the same thing [in both video clips].’
  }\ex{
li07pe 020\\
\gll
E  bríng  \textit{haricots}  na  hán,  e  lɛ́f  dɛ́n  pan  di   \\
\textsc{3sg.sbj}  bring  beans  \textsc{loc}  hand  \textsc{3sg.sbj}  leave  \textsc{3pl.indp}  on  \textsc{def}    \\
\gll
tébul.   \\
table    \\
\glt
‘She brought beans in her hand (and) she left them on the table.’
  }\ex{
li07pe 021\\
\gll
Di  róp  dé  pantáp  di  tébul.   \\
\textsc{def}  rope  \textsc{be.loc}  on  \textsc{def}  table    \\
\glt
‘The rope is on the table.’
  }\ex{
li07pe 022\\
\gll
Di  róp  nó  fít  slíp.   \\
\textsc{def}  rope  \textsc{neg}  can  sleep    \\
\glt
‘The rope can’t lie.’
  }\ex{
li07pe 023\\
\gll
Na  pɔ́sin  de  slíp.   \\
\textsc{foc}  person  \textsc{ipfv}  sleep    \\
\glt
‘It’s a person that lies down.’
  }\ex{
ko07pe 024\\
\gll
E  lé  pantáp  di  tébul?   \\
\textsc{3sg.sbj}  lie  on  \textsc{def}  table    \\
\glt
‘[So can I say] it’s lying on the table?’
  }\ex{
li07pe 025\\
\gll
Nó,  e  dé    pantáp  di  tébul.   \\
\textsc{neg}  \textsc{3sg.sbj}  \textsc{be.loc}  on  \textsc{def}  table    \\
\glt
‘No, it’s on the table.’
  }\ex{
li07pe 026\\
\gll
Ɛf  e  lé  na  lɛk  sé  e  de  slíp.   \\
if  \textsc{3sg.sbj}  lie  \textsc{foc}  like  \textsc{quot}  \textsc{3sg.sbj}  \textsc{ipfv}  lie    \\
\glt
‘If it’s lying it’s like it’s lying.’
  }\ex{
li07pe 027\\
\gll
Na  pɔ́sin  de  lé.   \\
\textsc{foc}  person  \textsc{ipfv}  lie    \\
\glt
‘It’s a person that lies.’
  }
\newpage 
\ex{
li07pe 028\\
\gll
Na  kasára.   \\
\textsc{foc}  cassava    \\
\glt
‘That’s cassava.’
  }\ex{
li07pe 029\\
\gll
E  bríng  di  kasára  na  in  hán.   \\
\textsc{3sg.sbj}  bring  \textsc{def}  cassava  \textsc{loc}  \textsc{3sg.poss}  hand    \\
\glt
‘She brought the cassava in her hand.’
  }\ex{
li07pe 030\\
\gll
Di  \textit{cartón}  dé  pantáp  di  tébul.   \\
\textsc{def}  carton  \textsc{be.loc}  on  \textsc{def}  table    \\
\glt
‘The carton is on the table.’
  }\ex{
li07pe 031\\
\gll
E  pút  di  kasára  ínsay  di  \textit{cartón}  wé  dé  pantáp   \\
\textsc{3sg.sbj}  put  \textsc{def}  cassava  inside  \textsc{def}  carton  \textsc{sub}  \textsc{be.loc}  on    \\
\gll
di  tébul.   \\
\textsc{def}  table    \\
\glt
‘She put the cassava into the carton that is on the table.’
  }\ex{
li07pe 032\\
\gll
Yu  nó=an  ɛ́n?   \\
\textsc{2sg}  know=\textsc{3sg.obj}  \textsc{intj}    \\
\glt
‘You know her, right?’
  }\ex{
li07pe 033\\
\gll
Yu  nó  nó?   \\
\textsc{2sg}  \textsc{neg}  know    \\
\glt
‘You don’t know (her)?’
  }\ex{
li07pe 034\\
\gll
E  hɛ́ng=an  míndul  tú  stík  dɛn.   \\
\textsc{3sg.sbj}  hang=\textsc{3sg.obj}  middle  two  tree  \textsc{pl}    \\
\glt
‘He hung it up between two branches.’
  }\ex{
li07pe 035\\
\gll
Hɛ́ng=an  na  \textit{colgar}.   \\
hang=\textsc{3sg.obj}  \textsc{foc}  hang    \\
\glt
‘“Hɛ́ng=an” is “colgar” [in Spanish].’
  }\ex{
li07pe 036\\
\gll
Ɛf  e  kwís=an,  e  go  spwɛ́l.   \\
if  \textsc{3sg.sbj}  squeeze=\textsc{3sg.obj}  \textsc{3sg.sbj}  \textsc{pot}  spoil    \\
\glt
‘If he squeezes it, it will spoil.’
  }
\newpage 
\ex{
li07pe 037\\
\gll
Na  kandá  fɔ  kokonát.   \\
\textsc{foc}  skin  \textsc{prep}  coconut    \\
\glt
‘That’s the shell of a coconut.’
  }\ex{
dj07pe 038\\
\gll
Na  só      sénwe.   \\
\textsc{foc}  like.that  \textsc{foc}    \\
\glt
‘That’s exactly how it is.’
  }\ex{
li07pe 039\\
\gll
E  bríng  tú  bɔ́tul  ɛ́nti.   \\
\textsc{3sg.sbj}  bring  two  bottle  empty    \\
\glt
‘He brought two bottles empty.’
  }\ex{
li07pe 040\\
\gll
E  pút  dɛ́n  pan  di  tébul.   \\
\textsc{3sg.sbj}  put  \textsc{3pl.indp}  on  \textsc{def}  table    \\
\glt
‘He put them on the table.’
  }\ex{
dj07pe 041\\
\gll
Tú  dífrɛn  bɔ́tul  dɛn  fɔ  \textit{vino.}   \\
two  different  bottle  \textsc{pl}  \textsc{prep}  wine    \\
\glt
‘Two different bottles of wine.’
  }\ex{
dj07pe 042\\
\gll
Di  tú  bɔ́tul  dɛn  fít  slíp  pantáp  tébul  sɛ́f.   \\
\textsc{def}  two  bottle  \textsc{pl}  can  lie  on  table  \textsc{foc}    \\
\glt
‘The two bottles can (actually) even lie on the table.’
  }\ex{
li07pe 043\\
\gll
E  fínis  bɛ́n  di  písis  fáyn.   \\
\textsc{3sg.sbj}  finish  bend  \textsc{def}  piece.of.cloth  fine    \\
\glt
‘He has finished folding the piece of cloth nicely.’
  }\ex{
li07pe 044\\
\gll
E  pút  wán  smɔ́l  stík  nía  di  stík  wé  e  \textit{para.}   \\
\textsc{3sg.sbj}  put  one  small  tree  near  \textsc{def}  tree  \textsc{sub}  \textsc{3sg.sbj}  stand    \\
\glt
‘She put a small stick next to the tree that’s standing.’
  }\ex{
li07pe 045\\
\gll
E  \textit{apoya}  wán  háf  stík  fɔ  wán  stík.   \\
\textsc{3sg.sbj}  lean  one  half  tree  \textsc{prep}  one  tree    \\
\glt
‘She leaned a branch on a tree.’
  }
\newpage 
\ex{
dj07pe 046\\
\gll
\textit{Porque}  dí  wán  na  stík  wé  e  \textit{para.}   \\
because  this  one  \textsc{foc}  tree  \textsc{sub}  \textsc{3sg.sbj}  stand    \\
\glt
‘Because this one is a tree that’s standing.’
  }\ex{
dj07pe 047\\
\gll
Yu  fít  tɔ́k  sé  yu  líng  yu  sɛ́f  dé.   \\
\textsc{2sg}  can  talk  \textsc{quot}  \textsc{2sg}  lean  \textsc{2sg}  self  there    \\
\glt
‘You can say you’re abutting yourself there.’
  }\ex{
dj07pe 048\\
\gll
Yu  fít  tɔ́k  sé  \textit{chico,}  a  wánt  líng  mi  sɛ́f   \\
\textsc{2sg}  can  talk  \textsc{quot}  \textsc{intj}  \textsc{1sg.sbj}  want  lean  \textsc{1sg.poss}  self    \\
\gll
fɔ  dís  \textit{butaca.}   \\
\textsc{prep}  this  armchair    \\
\glt
‘You can say, man, I want to lounge in this armchair.’
  }\ex{
dj07pe 049\\
\gll
E  líng  wán  háf  stík  nía  wán  bíg      bíg  stík.   \\
\textsc{3sg.sbj}  lean  one  half  tree  near  one  big    \textsc{rep}  tree    \\
\glt
‘She leaned a branch against a tree.’
  }\ex{
li07pe 050\\
\gll
E  jám=an  nía  wán  stík  wé  e   \\
\textsc{3sg.sbj}  make.contact=\textsc{3sg.obj}  near  one  tree  \textsc{sub}  \textsc{3sg.sbj}    \\
\gll
\textit{tínap.}   \\
stand    \\
\glt
‘She placed it [the branch] in contact with the tree that’s standing.’
  }\ex{
li07pe 051\\
\gll
Yu  fít  ték  wán  stík  wé  e  kɔ́t  háf,  yu  \textit{apoya}   \\
\textsc{2sg}  can  take  one  tree  \textsc{sub}  \textsc{3sg.sbj}  cut  half  \textsc{2sg}  lean    \\
\gll
ín.   \\
\textsc{3sg.indp}    \\
\glt
‘You can take a branch that’s cut in half (and) abut it.’
  }\ex{
li07pe 052\\
\gll
Wán  stík  wé  dɛn  kɔ́t=an,  bíg    bíg  wán.   \\
one  tree  \textsc{sub}  \textsc{3pl}  cut=\textsc{3sg.obj}  big  \textsc{rep}  one    \\
\glt
‘A branch that’s been cut, a really big one.’
  }\ex{
li07pe 053\\
\gll
\textit{Uf,}  \textit{Pichi}  \textit{es}  \textit{una}  \textit{basura},  ɛ́n.   \\
\textsc{intj}  Pichi  it.is  a  rubbish  \textsc{intj}    \\
\glt
‘Phew, Pichi is real rubbish, right.’
  }\ex{
li07pe 054\\
\gll
E  líng  di  bɔ́tul  nía  di  stík.   \\
\textsc{3sg.sbj}  lean  \textsc{def}  bottle  near  \textsc{def}  tree    \\
\glt
‘He leaned the bottle against the tree.’
  }
\newpage 
\ex{
li07pe 055\\
\gll
E  de  kwís  di  bɔ́l  fɔ  mék  di  bɔ́l  fít  hɛ́ng  fáyn.   \\
\textsc{3sg.sbj}  \textsc{ipfv}  squeeze  \textsc{def}  ball  \textsc{prep}  \textsc{sbjv}  \textsc{def}  ball  can  hang  fine    \\
\glt
‘He’s squeezing the ball in order for the ball to be able to be suspended just right.’
  }\ex{
dj07pe 056\\
\gll
E  pút  di  bɔ́tul  pantáp  di  tébul  \textit{pero}  di  mɔ́t   \\
\textsc{3sg.sbj}  put  \textsc{def}  bottle  on  \textsc{def}  table  but  \textsc{def}  mouth    \\
\gll
dé  dɔ́n.   \\
\textsc{be.loc}  down    \\
\glt
‘He put the bottle on the table but with the mouth down.’
  }\ex{
li07pe 057\\
\gll
E  pút  di  bɔ́tul  pan  di  tébul  wet  di  mɔ́t  dɔ́n   \\
\textsc{3sg.sbj}  put  \textsc{def}  bottle  on  \textsc{def}  table  with  \textsc{def}  mouth  down    \\
\gll
ɔ  rɔn-sáy.   \\
or  wrong.\textsc{cpd}{}-side    \\
\glt
‘He put the bottle on the table with the mouth down or upside-down.’
  }\ex{
li07pe 058\\
\gll
Di  písis  hɛ́ng  na  di  stík,  bikɔs  nó  mán  nó   \\
\textsc{def}  piece.of.cloth  hang  \textsc{loc}  \textsc{def}  tree  because  \textsc{neg}  man  \textsc{neg}    \\
\gll
pút=an.\\
put=\textsc{3sg.obj}\\
\glt
‘The piece of cloth is hanging from the tree, because nobody has put it (there).’
  }\ex{
li07pe 059\\
\gll
Wí  de  sí  dásɔl  sé  di  písis  dɔ́n  hɛ́ng.   \\
\textsc{1pl.indp}  \textsc{ipfv}  see  only  \textsc{quot}  \textsc{def}  piece.of.cloth  \textsc{prf}  hang    \\
\glt
‘We only see that the piece of cloth is now hanging.’
  }\ex{
li07pe 060\\
\gll
E  dɔ́n  \textit{cuelga}  na  di  stík.   \\
\textsc{3sg.sbj}  \textsc{prf}  hang  \textsc{loc}  \textsc{def}  tree    \\
\glt
‘It’s hanging from the tree.’
  }\ex{
li07pe 061\\
\gll
Dís  wán  dé    sé  a  mít  wán  bɔ́tul  wé  e    dé   \\
this  one  \textsc{be.loc}  \textsc{quot}  \textsc{1sg.sbj}  meet  one  bottle  \textsc{sub}  \textsc{3sg.sbj}  \textsc{be.loc}    \\
\gll
míndul  tú  stík  dɛn.   \\
middle  two  tree  \textsc{pl}    \\
\glt
‘This one [still image] is like I’ve come across a bottle that’s between two trees.’
  }\ex{
li07pe 062\\
\gll
E  bríng  di  kasára  e  pút=an  nía  di  stík.   \\
\textsc{3sg.sbj}  bring  \textsc{def}  cassava  \textsc{3sg.sbj}  put=\textsc{3sg.obj}  near  \textsc{def}  tree    \\
\glt
‘She brought the cassava (and) she put it next to the tree.’
  }\ex{
li07pe 063\\
\gll
E  líng=an  dé.   \\
\textsc{3sg.sbj}  lean=\textsc{3sg.obj}  there    \\
\glt
‘She abutted it there.’
  }\ex{
li07pe 064\\
\gll
Dán  húman  lɔ́n  bad.   \\
that  woman  be.long  extremely    \\
\glt
‘That woman is really tall.’
  }\ex{
li07pe 065\\
\gll
\textit{Chico,}  \textit{Dios}  \textit{mío.}   \\
\textsc{intj}  my  God    \\
\glt
‘Wow, my God.’
  }\ex{
dj07pe 066\\
\gll
E  bríng  \textit{escalera,}  e  líng=an  nía  di  stík.   \\
\textsc{3sg.sbj}  bring  ladder  \textsc{3sg.sbj}  lean=\textsc{3sg.obj}  near  \textsc{def}  tree    \\
\glt
‘She brought a ladder, she leaned it against the tree.’
  }\ex{
dj07pe 067\\
\gll
E  bríng  trí  kasára,  e  lé  dɛ́n  pantáp  di    tébul.   \\
\textsc{3sg.sbj}  bring  three  cassava  \textsc{3sg.sbj}  lay  \textsc{3pl.indp}  on  \textsc{def}  table    \\
\glt
 ‘She brought three cassavas, she laid them on the table.’
  }\ex{
dj07pe 068\\
\gll
E  lé=an  pantáp  di  tébul.   \\
\textsc{3sg.sbj}  lay=\textsc{3sg.obj}  on  \textsc{def}  table    \\
\glt
‘She laid them on the table.’
  }\ex{
li07pe 069\\
\gll
E  fíks  dɛ́n  fáyn.   \\
\textsc{3sg.sbj}  fix  \textsc{3pl.indp}  fine    \\
\glt
‘She arranged them nicely.’
  }\ex{
li07pe 070\\
\gll
Dí  \textit{cartón,}  e  mít=an  yá?   \\
this  carton  \textsc{3sg.sbj}  meet=\textsc{3sg.obj}  here    \\
\glt
‘The carton, did she find it [lying] here?’
  }\ex{
li07pe 071\\
\gll
E  pút  di  róp  ínsay  di  \textit{cartón}  wé  e  dé   \\
\textsc{3sg.sbj}  put  \textsc{def}  rope  inside  \textsc{def}  carton  \textsc{sub}  \textsc{3sg.sbj}  \textsc{be.loc}    \\
\gll
pantáp  di  tébul.   \\
on  \textsc{def}  table    \\
\glt
‘She put the rope inside the carton that’s on the table.’
  }\ex{
li07pe 072\\
\gll
E  slíp  di  bɔ́tul  pantáp  di  tébul.   \\
\textsc{3sg.sbj}  lay  \textsc{def}  bottle  on  \textsc{def}  table    \\
\glt
‘She laid the bottle down on the table.’
  }
\newpage 
\ex{
li07pe 073\\
\gll
Di  bɔ́tul  lé  náw  pantáp  di  tébul.   \\
\textsc{def}  bottle  lie  now  on  \textsc{def}  table    \\
\glt
‘The bottle is now lying on the table.’
  }\ex{
li07pe 074\\
\gll
E  lé  di  bɔ́tul  pantáp  di  tébul,  e  slíp  di   \\
\textsc{3sg.sbj}  lay  \textsc{def}  bottle  on  \textsc{def}  table  \textsc{3sg.sbj}  lay  \textsc{def}    \\
\gll
bɔ́tul  pantáp  di  tébul.   \\
bottle  on  \textsc{def}  table    \\
\glt
‘She laid [\textit{le}] the bottle on the table, she laid [\textit{slíp}] the bottle on the table.’
  }\ex{
li07pe 075\\
\gll
Di  bɔ́tul  slíp  pantáp  di  tébul  bikɔs  di  bɔ́tul  lé  dé.   \\
\textsc{def}  bottle  lie  on  \textsc{def}  table  because  \textsc{def}  bottle  lie  there    \\
\glt
‘The bottle is lying [\textit{slíp}] on the table because the bottle is lying [\textit{lé}] there.’
  }\ex{
li07pe 076\\
\gll
Náw  e  ték  róp,  e  hɛ́ng  di  róp  na  di  stík   \\
now  \textsc{3sg.sbj}  take  rope  \textsc{3sg.sbj}  hang  \textsc{def}  rope  \textsc{loc}  \textsc{def}  tree    \\
\gll
wet  kasára.   \\
with  cassava    \\
\glt
‘Now he took a rope, he hung the rope from the tree with a cassava.’
  }\ex{
li07pe 077\\
\gll
Nóto  só,  a  tɔ́k=an  bád,  Djunais?   \\
\textsc{neg}.\textsc{foc}  so  \textsc{1sg.sbj}  talk=\textsc{3sg.obj}  bad  \textsc{name}    \\
\glt
‘Isn’t it so, did I say that wrong, Djunais?’
  }\ex{
dj07pe 078\\
\gll
E  táy  di  kasára  wet  róp  áfta  e  hɛ́ng=an.   \\
\textsc{3sg.sbj}  tie  \textsc{def}  cassava  with  rope  then  \textsc{3sg.sbj}  hang=\textsc{3sg.obj}    \\
\glt
‘He tied the cassava with a rope, then he hung it up.’
  }\ex{
li07pe 079\\
\gll
Dís  stík,  e  slíp  pan  di  tébul.   \\
this  tree  \textsc{3sg.sbj}  lie  on  \textsc{def}  table    \\
\glt
‘This stick, it’s lying on the table.’
  }\ex{
dj07pe 080\\
\gll
Di  kasára  lé  míndul  tú  stík.   \\
\textsc{def}  cassava  lie  middle  two  tree    \\
\glt
‘The cassava is lying between two trees.’
  }\ex{
li07pe 081\\
\gll
Di  kasára  tínap  míndul  tú  stík.   \\
\textsc{def}  cassava  stand  middle  two  tree    \\
\glt
‘The cassava is standing between two trees.’
  }
\newpage 
\ex{
li07pe 082\\
\gll
E  tínap  di  kasára  míndul  tú  stík.   \\
\textsc{3sg.sbj}  stand  \textsc{def}  cassava  middle  two  tree    \\
\glt
‘He stood up the cassava between two trees.’
  }\ex{
li07pe 083\\
\gll
E  tínap=an  [di  tú  kasára]  míndul  tú  stík.   \\
\textsc{3sg.sbj}  stand=\textsc{3sg.obj}  \phantom{[}\textsc{def}  two  cassava  middle  two  tree    \\
\glt
‘He stood up the cassavas between two sticks.’
  }\ex{
dj07pe 084\\
\gll
Gó  ték  mí  dán  \textit{teléfono}  wé  tánap  pantáp  di   \\
go  take  \textsc{1sg.indp}  that  telephone  \textsc{sub}  stand  on  \textsc{def}    \\
\gll
tébul.   \\
table    \\
\glt
‘Go take that telephone for me that’s standing on the table.’
  }\ex{
dj07pe 085\\
\gll
A  go  kán  a  go  lúk,  ɛf  na  dí  wán  dásɔl   \\
\textsc{1sg.sbj}  \textsc{pot}  \textsc{pfv}  \textsc{1sg.sbj}  \textsc{pot}  look  if  \textsc{foc}  this  one  only    \\
\gll
dé  a  go  tɔ́k  sé  a  nó  sí.   \\
\textsc{be.loc}  \textsc{1sg.sbj}  \textsc{pot}  talk  \textsc{quot}  \textsc{1sg.sbj}  \textsc{neg}  see    \\
\glt
‘I would come (and) I would look, if it’s only this one that’s there, I would say I didn’t find (it).’
  }\ex{
dj07pe 086\\
\gll
A  go  tɔ́k  sé  a  nó  sí  \textit{teléfono}  wé  e   \\
\textsc{1sg.sbj}  \textsc{pot}  talk  \textsc{quot}  \textsc{1sg.sbj}  \textsc{neg}  see  telephone  \textsc{sub}  \textsc{3sg.sbj}    \\
\gll
slíp  pantáp  di  tébul.   \\
lie  on  \textsc{def}  table    \\
\glt
‘I would say I haven’t seen a telephone that’s lying on the table.’
  }\ex{
li07pe 087\\
\gll
E  nó  kɔ́ba  ín.   \\
\textsc{3sg.sbj}  \textsc{neg}  cover  \textsc{3sg.indp}    \\
\glt
‘She hasn’t covered it [the pot].’
  }\ex{
li07pe 088\\
\gll
Di  pɔ́t  kán  sin  kɔ́ba.   \\
\textsc{def}  pot  come  without  cover    \\
\glt
‘The pot came without a cover.’
  }\ex{
li07pe 089\\
\gll
Dɛn  pút=an  mɔ́t  dɔ́n  fɔ  di  tébul.   \\
\textsc{3pl}  put=\textsc{3sg.obj}  mouth  down  \textsc{prep}  \textsc{def}  table    \\
\glt
‘It was put mouth down [upside-down] on the table.’
  }
\newpage 
\ex{
li07pe 090\\
\gll
E dè  kán  fɔdɔ́n  sóté  yá.   \\
\textsc{3sg.sbj}  \textsc{ipfv}  come  fall  until  here    \\
\glt
‘It’s coming and extending until here.’
  }\ex{
li07pe 091\\
\gll
\'{A}fta  dí  wán  wé  e  dé  yandá,  e  bíg.   \\
then  this  one  \textsc{sub}  \textsc{3sg.sbj}  \textsc{be.loc}  yonder  \textsc{3sg.sbj}  be.big    \\
\glt
‘Then, that one that’s over there, it’s big.’
  }\ex{
dj07pe 092\\
\gll
E  pín  di  stík  na  grɔ́n.   \\
\textsc{3sg.sbj}  stick  \textsc{def}  tree  \textsc{loc}  ground    \\
\glt
‘She stuck the stick into the ground.’
  }\ex{
dj07pe 093\\
\gll
Náw  e  tínap  na  grɔ́n.   \\
now  \textsc{3sg.sbj}  stand  \textsc{loc}  ground    \\
\glt
‘Now it’s standing (upright) in the ground.’
  }\ex{
dj07pe 094\\
\gll
Di  pɔ́t  náw  só  e  slíp  pan  di  tébul.   \\
\textsc{def}  pot  now  so  \textsc{3sg.sbj}  lie  on  \textsc{def}  table    \\
\glt
‘Right now, the pot is lying on the table.’
  }\ex{
dj07pe 095\\
\gll
E  slíp  di  \textit{escalera}  na  grɔ́n.   \\
\textsc{3sg.sbj}  lay  \textsc{def}  ladder  \textsc{loc}  ground    \\
\glt
‘She laid the ladder on the ground.’
}\z