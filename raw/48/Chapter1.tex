%\chapter{Preamble}


%\begin{quotation}
%An especially powerful form for theory is a body of underlying 
%mechanisms, whose interactions and compositions provide the answers to 
%all the questions we have. \citep{newell_unified_1990}
%\end{quotation}


This essay explores some conceptual foundations for understanding the causal mechanisms that determine why languages are the way they are. The motivation is outlined in Chapter \ref{causalunits}, below. At the core of the argument are three ideas. 

%\begin{enumerate}
%\item[] 
The first idea is that causal processes apply in multiple frames or \textquoteleft time scales’ simultaneously, and we need to understand and address each and all of these frames equally in our work on language. This is the topic of Chapter \ref{causaldynamics}. 

%\item[] 
This leads to the second idea: For culture (including language) to exist, its constituent parts must have successfully been diffused and kept in circulation in the social histories of human populations, a process that relies on convergent processes in multiple causal frames, and that depends most centrally on the micro-level social behavior of people in interaction. This is the topic of Chapter \ref{Transmission biases}. 

%\item[] 
The third idea, building on this, is that the socially-diffusing parts of language and culture are not just floating around, but are firmly integrated within larger systems, and so we must have a causal account of how mobile bits of knowledge and behavior form up into structured cultural systems such as languages. This is the topic of Chapter \ref{itemsystemproblem} (where the problem is articulated) and Chapter \ref{micromacrosolution} (where a solution is offered). 

%\end{enumerate}
In exploring these core ideas, this essay suggests a conceptual framework for explaining, in causal terms, what language is like and why it is like that. While methods of research on language keep changing, the underlying conceptual work---always independent from the methods being applied---must provide the foundation.
