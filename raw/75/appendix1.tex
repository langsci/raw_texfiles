%\documentclass[12pt]{article}
%\usepackage{harvard}
%\usepackage{fullpage}
%\usepackage{graphicx}
%\usepackage{times}
%\usepackage[safe]{tipa}
%\usepackage{tipa}
%\usepackage[section]{placeins}
%\bibliographystyle{kluwer}

%\renewcommand{\baselinestretch}{1.5} 

%\begin{document}
%\date{}
\chapter{Measures of familiarity}\label{app:socialgroups}
%\maketitle
\noindent This appendix provides the names of girls in each group.  Also provided is a number indicating how well I felt I knew each girl, where 5 is highly familiar and 1 is knew by name and sight only. 

Girls who took part in the perception experiment and whose speech was analysed for the production study are marked with two crosses (++).  Girls who only took part in the perception experiment are marked with a single cross (+).
%How I defined `how close with me'?  Scale of 1-5, where 1 is knew by sight and overhearing but little direct interaction, 2 is only talking with a few times, 3 is talk regularly but still feel like I don't know particularly well, 4 is talk regularly and know quite well, 5 = she seeks me out for company, texts me, and wants to or does hang out outside of school


\section{CR groups}

\subsection{The Sporty Girls}
\nopagebreak
\begin{table}[ht]
\caption{The Sporty Girls, by how central to group and how well I felt I knew them.}\label{append:Sporty}
	\centering
		\begin{tabular}{llr} \\
		\hline
		
			name & centrality to group & how close with me  \\
			\hline
Naomi   & main & 3 \\
Stella  & main & 3 \\
Rachael & main & 3 \\
Elise   & core & 2 \\
Candice & core & 2 \\
Patricia ++ & fringe & 3 \\
Ruby    & core &  2 \\
Betty  ++  & fringe & 4 \\
Kanani (previously of The Pasifika Group) ++  & fringe & 5 \\
   \hline
		\end{tabular}
\end{table}

\pagebreak

\subsection {The PCs}
\nopagebreak
\begin{table}[ht]
\caption{The PCs, by how central to group and how well I felt I knew them.}\label{PCnames}
	\centering
		\begin{tabular}{llr} \\
		\hline
		
			name & centrality to group & how close with me  \\
			\hline
			
			Joanna		&		main		&		2  \\
			June			&		core/main &  1 \\
			Tracy ++			&		core		&		4 \\
			Juliet ++	&   core		&		4 \\
			Emma ++		&		core		&		4 \\
			Kim				&		core		&		3 \\
			Pixie			&		core		&		3 \\
			Kendra		&		core		&		3 \\
			Daphne		&		core		&		3 \\
			Darby   	&		core		&		2 \\
			Marilyn 	&		core		&		2 \\
			Aurora		&		core		&		1 \\
			Zindri		&		core		&		1 \\
			Gabrielle	&		core		&		1 \\
			Minnie		&		core		&		1 \\
			Gina			&		fringe	&		2 \\
			Noelle		&		fringe	&		1 \\
			Amber   	&   fringe	&		1 \\
			Cleo			&		fringe	&		1 \\
			Katya			&		fringe	&		1 \\		
			
			   \hline
		\end{tabular}
\end{table}


\pagebreak

\subsection{Trendy Alternatives}
\nopagebreak
\begin{table}[ht]
\caption{The Trendy Alternatives, by how central to group and how well I felt I knew them.}	\label{append:Alternatives}
	\centering
		\begin{tabular}{llr} \\
		\hline
					name & centrality to group & how close with me  \\
			\hline
Justine ++ & main & 3 \\
Kelly & core & 4 \\
Clementine ++ & core & 3 \\
Jewel  & core & 3 \\
Carla  & core & 3 \\
Christina ++ & core & 3 \\
Felicity & core & 2 \\
Lily  & fringe & 5 \\
Pascal +  & fringe & 4 \\

   \hline
		\end{tabular}
\end{table}

\subsection{Rochelle's Group}
\nopagebreak
\begin{table}[ht]
\caption{Rochelle's Group, by how central to group and how well I felt I knew them.}	\label{append:Drama}
	\centering
		\begin{tabular}{llr} \\
		\hline
							name & centrality to group & how close with me  \\
			\hline
		Rochelle ++ & main & 5 \\
		Camden   & main/core & 5 \\
		Chantelle & core & 2 \\
		Mindy     & core & 2 \\
		Lorna (also friends with The Relaxed Group)    & fringe & 2 \\
		   \hline
				\end{tabular}
\end{table}


\pagebreak
\subsection {The BBs}
\nopagebreak
The original BBs in\-cluded Star, Ma\-di\-son, Ja\-clyn, Za\-ra, Gwen, and Pris\-cil\-la.  The other half of the merged group (originally referred to as Pam's group) included Pam, Odette, Glenda, Jane, Shannon, Annie, Brooke, Andrea, Natasha, Ursula, Denise, Laura, and Maya.  In Table \ref{BBnames}, names followed by an asterisk denote members of the subgroup usually referred to as the BBs.  The other girls were a part of what was originally referred to as Pam's Group.
\nopagebreak
\begin{table}[ht]
\caption{The BBs, by how central to group and how well I felt I knew them.}	\label{BBnames}
	\centering
		\begin{tabular}{llr} \\
		\hline
		
			name & centrality to group & how close with me  \\
			\hline
			
			Star * 		&		main	&		2 \\
			Madison *	&		main	&		3 \\
			Pam			&		main	&		3 \\
			Jaclyn *	&		core	&		3 \\
			Zara *	&		core	&		1 \\
			Priscilla * & core	&		1 \\
			Gwen *		&		fringe &	3 \\
			Glenda +	&		core	&		4 \\
			Jane ++		&		core	&	3 \\
			Andrea + & core & 5 \\
			Maya + (previously of The PCs) & core & 4 \\
			Annie & core & 3 \\
			Natasha & core & 3 \\
			Ursula (previously of The PCs) & core & 3 \\
			Brooke + & core & 2 \\
			Becky & core & 2 \\
			Shannon	&	core	& 1 \\
			Kristy + & core & 1 \\
			Laura + & fringe & 3 \\
			Odette	&		fringe	&	2 \\
			Tori  +  & fringe  & 1 \\
			Denise & fringe & 1 \\
			Alexis (also friends with Cecily's Group) & fringe & 1 \\
			Karen (also friends with Cecily's Group) & fringe & 1 \\
			   \hline
		\end{tabular}
\end{table}


\subsection{The Relaxed Group}
\nopagebreak
\begin{table}[ht]
\caption{The Relaxed Group, by how central to group and how well I felt I knew them.}	\label{append:Relaxed}
	\centering
		\begin{tabular}{llr} \\
		\hline
							name & centrality to group & how close with me  \\
			\hline
Rose ++ & main & 5 \\
Megan & main & 4 \\
Barbara ++ & core & 4 \\
Anita + & core & 4 \\
Katrina ++ & core/fringe & 4 \\
Lorna (also friends with Rochelle's Group) & fringe & 2 \\
   \hline
		\end{tabular}
\end{table}
\mbox{}

\pagebreak




\bigskip

\pagebreak
\section{NCR groups}
\nopagebreak
\subsection{The Pasifika Group}
\nopagebreak
\begin{table}[ht]
\caption{The Pasifika Group, by how central to group and how well I felt I knew them.}\label{append:Pasifika}
	\centering
		\begin{tabular}{llr} \\
		\hline
					name & centrality to group & how close with me  \\
			\hline
Masina & main & 3 \\
Marama ++ & core & 4 \\
Ariana & core & 3 \\
Angel  & core & 2 \\
Ripeka & core & 1 \\
   \hline
				\end{tabular}
\end{table}


\subsection{The Goths}
\nopagebreak
\begin{table}[ht]
\caption{The Goths, by how central to group and how well I felt I knew them.}\label{append:Goths}
	\centering
		\begin{tabular}{llr} \\
		\hline
					name & centrality to group & how close with me  \\
			\hline
Santra ++ & main & 4 \\
Vanessa ++ & core & 5 \\
Meredith ++ & core & 4 \\
Marissa ++ & core & 4 \\
Tania (previously of The Relaxed Group) ++ & core & 3 \\
Stevie & core & 1 \\
Melinda & core & 1 \\
Judith & core & 1 \\
Bianca (previously of The Geeks) ++ & fringe & 5 \\
   \hline
				\end{tabular}
\end{table}

\pagebreak
\subsection{The Real Teenagers}
\nopagebreak
\begin{table}[ht]
\caption{The Real Teenagers, by how central to group and how well I felt I knew them.}\label{append:RealTeens}
	\centering
		\begin{tabular}{llr} \\
		\hline
							name & centrality to group & how close with me  \\
			\hline
Onya & main & 5 \\
Claudia & main & 3 \\
Renee & main & 3 \\
Isabelle ++ & core & 5 \\
Sarah ++ & core & 4 \\
Alex (also friends with Cecily's Group) & fringe & 5 \\
Sally & fringe & 4 \\
Camelia & fringe & 1 \\
   \hline
				\end{tabular}
\end{table}


\subsection{The Christians}
\nopagebreak
\begin{table}[ht]
\caption{The Christians, by how central to group and how well I felt I knew them.}\label{append:Christians}
	\centering
		\begin{tabular}{llr} \\
		\hline
							name & centrality to group & how close with me  \\
			\hline
Esther ++ & main & 5 \\
Theresa ++ & main & 4 \\
\hline
				\end{tabular}
\end{table}

\subsection{Sonia's Group}
\nopagebreak
\begin{table}[ht]
\caption{Sonia's Group, by how central to group and how well I felt I knew them.  There were other girls in this group who I did not come to know.}\label{append:Sonia}
	\centering
		\begin{tabular}{llr} \\
		\hline
							name & centrality to group & how close with me  \\
			\hline
		Sonia & core & 1 \\
		Holly ++ & core & 1 \\
		   \hline
				\end{tabular}
\end{table}

\pagebreak
\subsection{The Geeks}
\nopagebreak
\begin{table}[ht]
\caption{The Geeks, by how central to group and how well I felt I knew them.}\label{append:Geeks}
	\centering
		\begin{tabular}{llr} \\
		\hline
							name & centrality to group & how close with me  \\
			\hline
Mariah ++ & main & 5 \\
Joy ++   & main & 4 \\
Kristen + & core & 3 \\
Nisha  &  core  &  3 \\
Jamie  & core & 2 \\
Aluna & core & 2 \\
Valentina & core & 2 \\
Aerial (previously of The Relaxed Group) & core & 2 \\
Bianca (also friends with The Goths) & fringe & 4 \\
   \hline
				\end{tabular}
\end{table}


\subsection{Cecily's Group}
\nopagebreak
\begin{table}[ht]
\caption{Cecily's Group, by how central to group and how well I felt I knew them.}\label{append:Cecily}
	\centering
		\begin{tabular}{llr} \\
		\hline
					name & centrality to group & how close with me  \\
			\hline
Cecily + & main & 3 \\	
Sally & core & 4 \\
Alex (also friends with The Real Teenagers) & core & 5 \\
Pania + & core & 2 \\
Keira + & core & 2 \\
Lindsey & core & 1 \\
Erin & core & 1 \\
Alexis (also friends with The BBs) & fringe & 1 \\
Karen (also friends with The BBs) & fringe & 1 \\
   \hline
				\end{tabular}
\end{table}

\pagebreak 

\subsection{Loners}
\nopagebreak
\begin{table}[ht]
\caption{Loners, by how well I felt I knew them.  They were not friends and did not form a group; they are only shown together in the table for convenience.}\label{append:loners}

	\centering
		\begin{tabular}{lr} \\
		\hline
					name & how close with me \\
		\hline
Charlie + & 2 \\
Polly     & 1 \\
   \hline
				\end{tabular}
\end{table}
