\addchap{Abkürzungen}
% \addchap{Abbreviations and symbols}
\begin{tabularx}{\textwidth}{lXlX}                                          
¨        &    Umlaut &                                             GEN        &    Genitiv \\
\hspace*{.5ex} ̑         &    Diphthongierung &                                  GEND        &    gender \\
ACC/AKK  &    accusative/Akkusativ &                               georg.        &    Georgisch \\
act        &    active/aktiv &                                     germ.        &    Germanisch \\
ADJ        &    Adjektiv &                                         hoch        &    Hochalemannisch \\
AGR        &    agreement &                                        h-st        &    Höchstalemannisch \\
Ahd        &    Ahd &                                              IC        &    Inflectional Class \\
ANIM        &    animated &                                        Ind        &    Indikativ \\
ART        &    Artikel &                                          Instr        &    Instrumental \\
ART.DEF        &    bestimmter Artikel &                           isol        &    isoliert \\
ART.INDEF        &    unbestimmter Artikel &                       K        &    Konsonant \\
bel        &    belebt &                                           L        &    Lexem \\
C        &    Class Index &                                        LFG        &    Lexical-Functional Grammar \\
CP        &    complementizer phrase &                             LOC        &    Lokativ \\
C-P        &    Content-Paradigm &                                 MASK/MASC        &    Maskulin \\
DAT        &    Dativ &                                            Mhd        &    Mittelhochdeutsch \\
DEF        &    definit(eness) &                                   M.ü.M.        &    Meter über Meer \\
Det        &    Determinierer &                                    N        &   Nomen\\
Determ        &    Determinant &                                   n        &    Blockindex \\
Dep        &    Dependent &                                        NEUT        &    Neutrum \\
diach        &    diachron, \mbox{diachrone Varietäten} &          Nhd        &    Neuhochdeutsch (=Standardsprache) \\
DP        &    Determiniererphrase &                               niederl.        &    Niederländisch \\
engl.        &    Englisch &                                       n-isol          &      nicht-isoliert \\
F        &    Feature &                                            NOM          &      Nominativ \\
FC        &    Form-Correspondent &                                norw.          &      Norwegisch \\
FEM        &    Feminin &                                          NP          &      Nominalphrase \\
FIN        &    finiteness &                                       NUM          &      Numerus \\
FK        &    Flexionsklasse &                                    oberr          &      Oberrheinalemannisch \\
F-P        &    Form-Paradigm &                                    OBJ          &      Objekt \\
FUT        &    future &                                           P          &      Präposition \\
\end{tabularx}\clearpage
% \begin{multicols}{2} 
\begin{tabular}{ll} 
pass          &      passive/passiv \\
PERFP          &      perfect participle \\
PERS          &      Person \\
PL          &      Plural \\
PP          &      Präpositionalphrase \\
POSS          &      Possessiv \\
PRED          &      predicate \\
PRES          &      present \\
PRON.DEM          &      Demonstrativpronomen \\
PRON.INTER          &      Interrogativpronomen \\
PRON.PERS          &      Personalpronomen \\
PRON.POSS          &      Possessivpronomen \\
r          &      root \\
RR          &      Realization Rule, Realisierungsregel \\
rum.        &    Rumänisch \\
schw.         &    Schwäbisch \\
SG          &    Singular \\
stand        &    deutsche Standardsprache \\
SUBJ        &    Subjekt \\
σ        &    Set an morphosyntaktischen Eigenschaften \\
τ        &    Property-Set-Index \\
unbel        &    unbelebt \\
ung.        &    Ungarisch \\
V        &    Verb \\
V        &    Vokal (nur in den phonologischen Regeln) \\
v        &    Value \\
VFORM        &    verb form \\
VP        &    Verbalphrase \\
XCOMP        &    complement \\
\end{tabular}
%
% \end{multicols} 