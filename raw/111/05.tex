\chapter{Discrepancies between meaning and marking}
\label{chapter5}
\setcounter{enums}{0}

\noindent \citet{bolinger:77} claims that the existence of one meaning
per one form and vice versa (i.e.\ an isomorphism between formal and
interpretive domains) is the most natural state of human language.
Natural human languages, however, provide many counterexamples to this
notion.  At the lexical level, homonymy and polysemy are two
widespread examples of a single ability to convey two or more
meanings.  Moreover, mismatches between meaning and form can sometimes
be caused by grammatical elements. For example, \ili{English} shows
discrepancies between form and meaning in counterfactuals,
constructions in which the speaker does not believe the given
proposition expressed in the antecedent is true. The most well known
factor which deeply contributes to the counterfactual meaning in many
languages is the past tense morpheme (e.g.\ `-ed' in English)
\citep{iatridou:00}.  The past tense morpheme in counterfactuals (also
known as fake past tense) does not denote an event that actually
happened in the past as exemplified in \myref{exe:iatridou}. Thus, the
mapping relationship between morphological forms and their meaning in
counterfactual sentences is not the same as that in non-counterfactual
sentences.


\myexe{\enumsentence{\toplabel{exe:iatridou}
\begin{tabular}[t]{ll}
a. & {If he were smart, he would be rich.} \\
   & {(conveying ``He isn't smart.'' and ``He isn't rich.'')} \\
b. & {I wish I had a car.} \\
   & {(conveying ``I don't have a car now.'') \citep[231--232]{iatridou:00}} \\
\end{tabular}}}

As with other grammatical phenomena, information structure also
exhibits discrepancies in form-meaning mapping. This chapter presents
several types of mismatches between the forms that express information
structure and the information structure meanings conveyed by those
forms.





\section{Ambivalent lexical markers}
\label{5:sec:lex}


In some languages, one lexical marker can correspond to meanings of
several components of information structure (i.e.\ no one-to-one
correspondence between form and meaning).  One such mismatch caused by
lexical markers is exhibited in \ili{Japanese} and \ili{Korean}. As is
well known, \wa in Japanese and \nun in Korean are regarded as lexical
markers to express the \isi{topic} of the sentence, but they can also
sometimes be used for conveying \isi{contrastive focus}.


\myexe{\eenumsentence{\toplabel{exe:kor:cf}
\item[Q:]\shortex{3}
{Kim-i	& onul	& o-ass-ni?}
{Kim-\textsc{nom} &	today & come-\textsc{pst}-\textsc{int}}
{`Did Kim come today?'}
\item[A:]\shortex{4}
{ani. & (Kim-un) & ecey-nun & o-ass-e.}
{No. &	Kim-\textsc{nun} & yesterday-\textsc{nun} & come-\textsc{pst}-\textsc{decl}}
{`No. Kim came yesterday.' [kor]}}}



\noindent The lexical marker \nun in Korean appears twice in
(\ref{exe:kor:cf}A); one occurrence is with the subject \textit{Kim},
and the other is combined with an adverb \textit{ecey}
`yesterday'. Although the same lexical marker is used,\is{lexical
  markers} they do not share the same properties of information
structure. It is clear that \isi{topic} is assigned to \textit{Kim-un} in
that the word is already given in the question and as indicated by the
parentheses, it is optional. By contrast, the \onun-marked
\textit{ecey} is newly and importantly mentioned by the replier, and
thereby it should be evaluated as containing a meaning of \isi{focus} rather
than topic. Moreover, if \textit{ecey-nun} disappears, the answer
sounds infelicitous within the context, which clearly implies it is
focused. Recall that I define focus as an information structure
component associated with an inomissible constituent.  Furthermore,
\myref{exe:kor:cf} passes the correction test to vet contrastive focus
\citep{gryllia:09}.\is{correction test} Since \textit{onul} `today' in
the question and \textit{ecey} `yesterday' in the reply constitute an
\isi{alternative set}, \textit{ecey} in (\ref{exe:kor:cf}A) has a
contrastive meaning.\is{contrast}  As a consequence, the information structure role
of \textit{ecey} in (\ref{exe:kor:cf}A) is \isi{contrastive focus}, even
though the so-called topic marker \nun is attached to it.



This \onun-marked constituent associated with \isi{contrastive focus}
is realized differently from the one associated with \isi{contrastive
  topic}. In (\ref{exe:kor:ct}A), the \onun-marked element in the first
position can be dropped as the parentheses imply. When \textit{ku
  chack-un} appears, the fronted constituent is associated with
contrast.  This finding echoes \citeauthor{choi:99}'s argument. She
claims that only elements with contrastive meaning can be scrambled in
\ili{Korean}, which means \textit{ku chack-un} `the book-\textsc{nun}'
in (\ref{exe:kor:ct}A) gives contrastive meaning.\is{contrast}

\myexe{\eenumsentence{\toplabel{exe:kor:ct}
\item[Q:]\shortex{4}
{nwuka & ku & chayk-ul & ilk-ess-ni?}
{who & the  & book-\textsc{acc} & read-\textsc{pst}-\textsc{int}}
{`Who read the book?'}
\item[A:]\shortex{4}
{(ku & chayk-un) & Kim-i & ilk-ess-e.}
{the &	book-\textsc{nun} & Kim-\textsc{nom} & read-\textsc{pst}-\textsc{decl}}
{`(As for the book,) Kim read it.' [kor]}}}

\noindent In fact, \citeauthor{choi:99} does not concede the existence
of \isi{contrastive topic} in \ili{Korean}, and the scrambled and
\onun-marked constituents are analyzed as only contrastive \isi{focus} in her
proposal. However, this notion is contradictory to the definition that
focus cannot be elided. Given that \textit{ku chayk-un} in
(\ref{exe:kor:ct}A) can felicitously disappear, we cannot say that it
is associated with focus. Since contrast should be realized as either
contrastive focus or contrastive topic, \textit{ku chayk-un} in
(\ref{exe:kor:ct}A) must be evaluated as a contrastive topic.\is{contrastive focus}


Therefore, \nun in \ili{Korean} can assign three meanings to an
adjoining NP: aboutness topic, contrastive \isi{topic}, and
contrastive \isi{focus}.  In other words, \nun provides constraints,
but only partial ones, which cause discrepancies between form and
meaning.  Because this marker can be combined with constituents that
are not topics, it is my position that `topic-marker' is not an
appropriate label.  The same goes for \wa in Japanese.  It should also
be noted that case markers in these languages (e.g.\ \ika and \ga for
nominatives) also convey an ambiguous interpretation, either focus or
\isi{background} (i.e.\ non-topic).



In some languages, a lexical marker known for marking \isi{topic}
coincides with cleft constructions which clearly carry a \isi{focus}
meaning.\is{clefting} \myref{exe:hil} in \ili{Ilonggo} (also known as
Hiligaynon, an Austronesian language spoken in the Philippines)
exemplifies such a mismatch \citep{schachter:73}. In Ilonggo, the
topic marker \textit{ang} is in complementary distribution with case
markers similarly to \wa in Japanese and \nun in \ili{Korean}. One
difference is that the case relation is marked by an affix attached to
the verb (e.g.\ the agentive marker \textit{nag-} in \ref{exe:hil}).


\myexe{\eenumsentence{\toplabel{exe:hil}
\item\shortex{6}
{nag- & dala & ang & babayi & sang & bata}
{\textsc{ag.top}- & bring & \textsc{top} & woman & \textsc{nontop} & child}
{`The woman brought a child.'}
\item\shortex{7}
{ang & babayi & ang & nag- & dala & sang & bata}
{\textsc{top} & woman & \textsc{top} & \textsc{ag.top}- & bring & \textsc{nontop} & child}
{`It was the woman who brought a child.' [hil] \citep[108]{croft:02}}}}

\noindent (\ref{exe:hil}a) is a topicalized construction in which the
\isi{topic} marker \textit{ang} is combined with \textit{babayi}
`woman'. (\ref{exe:hil}b) is a focused construction, in which the
topic marker \textit{ang} is still combined with the focused
constituent \textit{babayi}, and one more topic marker appears at the
beginning of the cleft clause \textit{nag- dala sang bata}, which
implies that the so-called topic marker does not necessarily express
topic meaning.


\section{Focus/Topic fronting}
\label{5:sec:fronting}
\largerpage[2]


My cross-linguistic survey of \isi{focus}/topic \isi{fronting} draws a
tentative conclusion: If focus and \isi{topic} compete for the
sentence-initial position, topic always wins.  To take an example, in
\ili{Ingush}, both topic and focus can precede the rest of the clause,
but a focused constituent must follow a constituent conveying topic,
as exemplified in \myref{exe:inh:focus-topic}.

\myexe{\enumsentence{\label{exe:inh:focus-topic}
\evnup{\begin{tabular}[h]{llllllllll}
Jurta & jistie \\
town.\textsc{gen} & nearby \\
(topic) \\
joaqqa & sag & ull & cymogazh & jolazh.\\
J.old & person & lie.\textsc{prs} & sick.\textsc{cv}sim & J.\textsc{prog}.\textsc{cv}sim\\
(focus) \\
\multicolumn{10}{l}{`In the next town an old woman is sick (is lying sick).'}\\
\end{tabular}}
\newline
\evnup{\begin{tabular}[h]{llllllllll}
Mista & xudar & myshta & duora?\\
sour & porridge & how & D.make.\textsc{impf}\\
(topic) & & (focus) \\
\multicolumn{10}{l}{`How did they make sour porridge? (How was sour porridge made?)'}
\\
\multicolumn{10}{r}{ [inh] \citep[683]{nichols:11}}\\
\end{tabular}}}}


\noindent This means that if \isi{topic} and (narrow) \isi{focus}
co-occur, topic should be followed by focus even in languages which
place focused constituents in the clause-initial
position.\is{clause-initial}\is{narrow focus} The same phenomenon can
be found in many other languages.  For example, in \ili{Nishnaabemwin}
(an Algic language spoken in the region surrounding the Great Lakes,
in Ontario, Minnesota, Wisconsin, and Michigan), if both the subject
and the object of a transitive verb appear preverbally, the first is
marked for topic and the second for focus \citep{valentine:01}. No
counterexamples to this generalization have been observed, at least
among the languages I have examined hitherto.\is{preverbal}




Yet, there are some cases in which it is unclear which role
(i.e.\ \isi{focus} or topic) the fronted constituent is assigned.  I
would like to label the constructions in which this kind of ambiguity
takes place `focus/\isi{topic} \isi{fronting}' (also known as
Topicalization).\is{topicalization} \citet{prince:84} provides two
types of OSV constructions in \ili{English}, and argues that the
change in word order is motivated by marking information status, such
as new and old information.


\myexe{\eenumsentence{\label{exe:prince:213:5}
\item{John saw Mary yesterday.}
\item{Mary, John saw yesterday.}
\item{Mary, John saw her yesterday. \citep[213]{prince:84}}}}




\is{topicalization} \is{left dislocation}
\noindent Both (\ref{exe:prince:213:5}b--c) relate to
(\ref{exe:prince:213:5}a), but (\ref{exe:prince:213:5}b) is devoid of
the resumptive pronoun in the main clause, whereas
(\ref{exe:prince:213:5}c) has \textit{her} referring to
\textit{Mary}. These are called Topicalization and Left-Dislocation by
\citeauthor{prince:84},\footnote{\citet{prince:84} argues that the
  choice of one over another is not random but is influenced by the
  information status of what the speaker is talking.  According to
  \citeauthor{prince:84}, Topicalization has two characteristics; one
  is that it is used to mark information status of the entity itself,
  and the other is that it involves an open proposition.  In short, in
  \citeauthor{prince:84}'s analysis, information status factors
  (e.g.\ new \vs given), have an effect on the composition in the OSV
  order, which removes the fronted NP referring to a discourse-new
  entity from a syntactic position that disfavours it.\is{syntactic
    positioning} As indicated in Chapter~\ref{chapter3}
  (Section \ref{3:sec:status}), since the present study is not concerned with
  information status, such a distinction based on new information \vs
  given information is not used in the present work.} but I use the
label \isi{focus}/\isi{topic} \isi{fronting} for the first type of
syntactic operation.




The \isi{focus}/topic \isi{fronting} constructions have two potential meanings, as
exemplified in \myref{exe:topicalization}. That is,
(\ref{exe:topicalization}a) can be paraphrased into either
(\ref{exe:topicalization}b) or (\ref{exe:topicalization}c), whose
information structures differ.


\myexe{\enumsentence{\label{exe:topicalization}
\begin{tabular}[t]{ll}
a. & {The book Kim read.}\\
b. & {It was the book that Kim read.}\\
c. & {As for the book, Kim read it.}\\
\end{tabular}}}


\noindent If the fronted NP is focused, its configuration is the same
as cleft constructions (\ref{exe:topicalization}b). If it behaves as
the topic within the context, the sentence can share the same
information structure as (\ref{exe:topicalization}c).  This means
(\ref{exe:topicalization}a) in itself would sound ambiguous, in the
absence of contextual information.  \citet{gundel:83}, in order to
distinguish the different structures, makes use of the two terms Focus
Topicalization and Topic Topicalization, suggesting that OSV
constructions like (\ref{exe:topicalization}a) are
ambiguous. \citet{gussenhoven:07} also takes notice of such an
ambiguity, and regards the constructions like
(\ref{exe:topicalization}a) as containing `reactivating focus'.




\myexe{\enumsentence{\label{exe:gussenhoven:reactivating:ch5}
\begin{tabular}[t]{ll}
Q: & {Does she know \textsc{John}?}\\
A: & {\textsc{John} she \textsc{dislikes}. \citep[96]{gussenhoven:07}}\\
\end{tabular}}}



\noindent Nevertheless, it is my position that the terms that
\citeauthor{gundel:83} and \citeauthor{gussenhoven:07} make use of
still lead to confusion.\footnote{From a different point of view, some
  \ili{English} native speakers say that (\ref{exe:topicalization}c)
  does not look like a proper paraphrasing of
  (\ref{exe:topicalization}a).\is{clefting} Intuitively, the fronted
  item \textit{The book} conveys only \isi{focus} meaning.  If this
  thought holds true, the focus/\isi{topic} \isi{fronting}
  constructions are actually equivalent to cleft constructions, like a
  pair of (\ref{exe:topicalization}a--b).  In fact, other native
  speakers who read focus/topic constructions in other languages have
  similar thoughts. For instance, one Cantonese informant says that
  \myref{exe:yue:topicalization} in \ili{Cantonese} can convey both
  meanings, but the first reading is predominant.  For now, I cannot
  draw a conclusion about which one is a sound interpretation, but
  what is important is that the name `Topicalization' is not
  appropriate in any cases.}


Other languages also have the \isi{focus}/topic \isi{fronting} constructions. In
the following \ili{Cantonese} example, the fronted constituent \textit{nei1
  bun2 syu1} `this book' can play the role of either focus or topic of
the sentence, and the choice between the two readings hinges on the
context.



\myexe{\enumsentence{\label{exe:yue:topicalization}
\shortex{5} 
{Nei1 & bun2 & syu1 & ngo5 & zung1ji3}
{\textsc{def} & \textsc{clf} & book & 1.\textsc{sg} & like}
{(a) `It is this book that I like.' or \\(b) `As for this book, I like it.' [yue] \citep[16]{man:07}}}}


\noindent The same phenomenon can also be observed in
\ili{Nishnaabemwin}.  Information structure in Nishnaabemwin, whose
basic word order is VOS, is also accomplished via syntactic means.  If
a verbal argument appears before the verb, then it is marked for
information structure.  Its meaning, just as in the previous examples
in \ili{English} and \ili{Cantonese}, becomes ambiguous only if one
argument is \isi{preverbal} \citep{valentine:01}.  In fact, this kind
of ambiguity frequently happens in languages in which \isi{focus}
shows up clause-initially (e.g.\ \ili{Ingush}).\is{clause-initial}




In brief, a single form has two different information structure
meanings; the construction often refered to as Topicalization
\citep{prince:84} sounds ambiguous unless the given context is
ascertained.  Regarding the selection of terminology, the present
study calls such a construction \isi{focus}/topic-fronting, because (i) this
explicitly displays the ambiguous meaning, and (ii) the previous
terminology (i.e.\ topicalization) confuses syntactic and pragmatic
notions.



\section{Competition between prosody and syntax}
\label{5:sec:competition}
 
There are potentially three subclasses of the connection between
prosody and syntax.  First, some languages have a system with very
weak or no interaction between prosody and syntax with respect to
\isi{focus}.\is{prosody} These include \ili{Catalan}
\citep{engdahl:vallduvi:96}, \ili{Akan} \citep{drubig:03}, and
\ili{Yucatec Maya} \citep{kugler:etal:07}. In those languages,
displacing constituents is the only way to identify focused
elements. The second subclass assigns focus to a particular
position. Constraints on this position necessarily correlate with
phonological marking in the second type of languages.  Hungarian
belongs to this type, in which the focused and accented item appears
immediately prior to the verb \citep{kiss:98,szendroi:01}. The third
type, which occasionally brings about a mismatch between form and
meaning, includes languages in which prosody and syntax compete in
expressing focus.\is{prosody} That is, in this type of language,
either prosodic or syntactic structure can be used to mark focus,
depending on the construction.


\citet{buring:10} calls the third type `Mixed Languages' and draws the
following generalization about them.

\myexe{\enumsentence{\label{def:mixed}
\textsc{Marked Word Order} \ensuremath{\rightarrow} \textsc{Unmarked Prosody}:
Marked constituent order may only be used for focusing X 
if the resulting prosodic structure is less marked than 
that necessary to \isi{focus} X in the unmarked
    constituent order. \citep[197]{buring:10}}}



\noindent \citeauthor{buring:10} argues that mixed languages include
\ili{Korean}, \ili{Japanese}, \ili{Finnish}, \ili{German}, European
\ili{Portuguese}, and most of the Slavic languages. According to my
survey, \ili{Russian} and \ili{Bosnian Croatian Serbian} (i.e.\ the
Slavic languages) clearly fall under this third mixed type: as they
can either (i) employ a specific accent to signal focus or (ii) assign
the focused constituent to the clause-final position.\is{clause-final}
For instance, the subject \textit{sobaka} `dog' in (\ref{exe:sobaka}a)
can have focus meaning if and only if it bears the accent for \isi{focus},
which means (\ref{exe:sobaka}a) is informatively ambiguous in the
absence of information about accent.  In contrast, (\ref{exe:sobaka}b)
where the subject is in the final position sounds unambiguous, and
\textit{sobaka} is evaluated as focused.


\myexe{\eenumsentence{\label{exe:sobaka}
\item\shortex{2} 
{Sobaka & laet.}
{dog & bark}
{`The dog bark.'}
\item\shortex{2} 
{Laet & sobaka.}
{bark & dog}
{`The \textsc{dog} bark.' [rus]}}}


 
\noindent The distinction between (\ref{exe:sobaka}a--b) is more
clearly shown with the \textit{wh}-test.\is{\textit{wh}-test} If the
question is \textit{Who barks?} as given in (\ref{exe:sobaka:wh}Q1),
both sentences can be used as the reply. If the reply is
(\ref{exe:sobaka}a) in the neutral word order, the verb \textit{laet}
bears an accent. In contrast, if the question is
(\ref{exe:sobaka:wh}Q2), which requires the predicate to be focused,
(\ref{exe:sobaka}b) cannot be an appropriate answer and also there
should be no sentential stress on the verb \textit{laet}.


\myexe{\eenumsentence{\toplabel{exe:sobaka:wh}
\item[Q1:]\shortex{2}
{Kto & laet?}
{who & barks}
{`Who barks?'}
\item[A1:]{Sobaka laet. / Laet sobaka.}

\newpage 
\item[Q2:]\shortex{3}
{\v{C}to & delaet & sobaka?}
{what & doing & dog}
{`What does the dog do?'}
\item[A2:]{Sobaka laet. / \#Laet sobaka. [rus]}}}


The same holds true for \ili{Bosnian Croatian Serbian}.  When the
question is given as (\ref{exe:pas:wh}Q2), the sentence in which the
subject is not in situ sounds infelicitous, and the verb
\textit{laje} is not allowed to bear a sentential stress.

\myexe{\eenumsentence{\toplabel{exe:pas:wh}
\item[Q1:]\shortex{2}
{Ko & laje?}
{who & barks}
{`Who barks?'}
\item[A1:]\shortex{5}
{Pas & laje. & / & Laje & pas.}
{dog & barks. & / & barks & dog.}
{`The dog barks.'}
\item[Q2:]\shortex{3}
{\v{S}ta(\v{S}to) & radi & pas?}
{what & doing & dog}
{`What does the dog do?'}
\item[A2:]{Pas laje. / \#Laje pas. [hbs]}}}


In summary, in the third type of language, prosody takes priority over
syntax in the neutral word order with respect to expressing \isi{focus}
(i.e.,\ the prosodic marking wins).\is{prosody} In contrast, when the sentence is
not in the default word order, syntactic structure wins.  Since
sentences in an unmarked word order are normally ambiguous along these
lines, focus position is not defined for sentences with unmarked word
order, only for those with other word orders.



\section{Multiple positions of focus}
\label{5:sec:multiple}


Even if a language employs a specific position for expressing
\isi{focus}, the focused constituent does not necessarily take that
position, as exemplified by \ili{Russian} in the previous
section. That is, focus can be assigned to multiple
positions.\is{clause-final} For instance, the focus in Russian may not
be clause-final (as presented in \ref{exe:sobaka}), if the accent
falls on another constituent. In this case, clause-final focus does
not seem to be the same as cleft constructions in
Russian,\is{clefting} and the accented constituent in situ is
not also necessarily equivalent to informational focus. A more complex
phenomenon with respect to syntactic operations on focus is
exemplified by \ili{Greek} \citep{gryllia:09}. In Greek, whose basic
word order is VSO or SVO, focus can be both \isi{preverbal} and
\isi{postverbal} and there is no informative difference between them.


\myexe{\eenumsentence{\label{exe:greek}
\item[Q:]\shortex{4}
{Thelis & kafe & i & tsai?}
{want.2\textsc{sg} & coffee.\textsc{acc} & or & tea.\textsc{acc}}
{`Would you like coffee or tea?'}
\item[A1:]\shortex{2}
{Thelo & [kafe]\mysub{C-Foc}.}
{want.1\textsc{sg} & coffee.\textsc{acc}}
{`I would like coffee.'}
\item[A2:]\shortex{2}
{[Kafe]\mysub{C-Foc} & thelo.}
{coffee.\textsc{acc} & want.1\textsc{sg}}
{`Coffee I would like.' [ell] \citep[44]{gryllia:09}}}}


\noindent The preverbal \isi{focus}, shown on \textit{kafe} in
(\ref{exe:greek}A2), is not in situ, because verbs precede
objects in the neutral word order in Greek.  Yet, there is no evidence
that preverbal focus plays the role of identification and this
sentential form is informatively the same as cleft constructions in
Greek.  \citeauthor{gryllia:09}, moreover, argues that focused
elements in both positions can receive the interpretation of
\isi{contrastive focus} as well as \isi{non-contrastive focus}.
That is, there are options for focus realization in Greek; (i) \isi{preverbal}
non-contrastive focus, (ii) preverbal contrastive focus,(iii)
postverbal non-contrastive focus, and (iv) \isi{postverbal}
contrastive focus. The multiple focus positions in Greek demonstrate
convincingly that forms which express information structure are not in
a one-to-one relation with information structure components and
thereby cannot unambiguously mark a specific information structure
meaning.









Another important phenomenon related to \isi{focus} positions can be found
in \ili{Hausa}.  According to \citet{hartmann:zimmermann:07}, Hausa
employs two strategies for marking focus. One is called ex
  situ focus, and the other is in situ focus. They are
exemplified in (\ref{exe:hau}A1-A2), respectively.

\myexe{\eenumsentence{\label{exe:hau}
\item[Q:]\shortex{3}
{M{\`e}e & suk{\`a} & kaam{\`a}a?}
{what & \textsc{3pl.rel.perf} & catch}
{`What did they catch?'}
\item[A1:]\shortex{4}
{\textbf{Kiifii} & (n{\`e}e) & suk{\`a} & kaam{\`a}a.}
{fish & \textsc{prt} & \textsc{3pl.rel.perf} & catch}
{`They caught \textsc{fish}.'}
\item[A2:]\shortex{3}
{Sun & kaam{\`a}a & \textbf{kiifii}.}
{\textsc{3pl.abs.perf} & catch & fish}
{`They caught \textsc{fish}.' [hau] \citep[242--243]{hartmann:zimmermann:07}}}}

\noindent \textit{In situ} \isi{focus} in \ili{Hausa} does not require any special
marking, whereas ex situ focus in the first position is
prosodically prominent. Moreover, Hausa employs two focus particles
\textit{n{\`e}e} and \textit{c{\`e}e}, but they can co-occur with only
ex situ focus as shown in (\ref{exe:hau}A1).\footnote{This is
  an intriguing phenomenon, because in other languages in
    situ foci in the unmarked word order normally require an
  additional constraint, such as pitch accents. In other words, as
  shown in the examples of the Slavic languages (presented in the
  previous section), it is common that focused constituents in the
  default position need to be accented if the language uses multiple
  strategies for marking focus or \isi{topic}.} For this reason,
\citet{buring:10} regards Hausa as a language without a specific
marking system for focus. This analysis of focus realization in Hausa
implies that some languages can assign focus to a constituent
in situ without the help of pitch accents.


The examples presented in this section motivate flexible
representation of information structure, particularly for sentences in
unmarked word order.  That is to say, in some circumstances, we cannot
exactly say where \isi{focus} is signaled.


\section{Summary}
\label{5:sec:summary}

Just as with other grammatical phenomena, there are discrepancies
between forms and meanings with respect to information structure.
This chapter has looked at several cases in which there are mismatches
in mapping between information structure markings and
meanings.\is{lexical markers} First, lexical markers that express
information structure occasionally cause such a mismatch. For example,
\wa and \nun in \ili{Japanese} and \ili{Korean} respectively are
\isi{topic} markers in these languages, but they can sometimes be used
for expressing \isi{contrastive focus}. Second, topic and focus appear
sentence-initially in quite a few languages, but there are some cases
in which we cannot decisively state whether the fronted item is
associated with topic or focus. Such a construction has often been
called `Topicalization' in previous literature, but I use different
terminology in order to be more accurate and I treat these
constructions as examples of focus/topic \isi{fronting}. Third, if
prosody and syntax compete for expressing information structure,
prosody takes priority in most cases. Finally, many languages place a
focused constituent in a specific position, but this placement is
optional in some languages.  The last two properties are related to
expressing focus in sentences in default word order. Information
structure in unmarked sentences is also addressed in
Section \ref{9:ssec:underspecification} \mypage{9:ssec:underspecification} in
terms of the implementation.

