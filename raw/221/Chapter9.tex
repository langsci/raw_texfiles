\chapter{Where do we go from here?} \label{final conclusion}

The present book set out to investigate the interface of morphology, {phonology and phonetics} by investigating \is{morphological gemination}{gemination} in English {affixation}. 
One main aim of this book was to clarify the role of \isi{boundary strength} on the \isi{acoustic realization} of complex words. 
While previous studies have found evidence for effects of \isi{boundary strength} on the phonetics of words, the specific nature of these effects is yet unclear (cf. Chapter \ref{Gemination}). 
 Clarifying the nature of these effects is, however, quite important, as a clarification is necessary to accurately model the morpho-phonological and the \isi{morpho-phonetic interface}. 

As shown in Chapter \ref{Theory}, different formal linguistic and psycholinguistic theories deviate in their conceptualization of  \isi{boundary strength}. While some theories assume a categorical difference between sets of affixes (e.g. \isi{Lexical Phonology}, \is{Stratal Optimality Theory}Stratal OT), others assume \isi{boundary strength} to be a gradient, probabilistic word-specific concept (e.g. the Decomposability Approach, the \is{informativeness}Morphological Informativeness Approach). 
While some approaches define \isi{boundary strength} by means of mainly lexical factors (e.g. \isi{Lexical Phonology}, \is{Stratal Optimality Theory}\is{Stratal Optimality Theory}Stratal OT), others mainly focus on prosodic aspects (e.g. \isi{Prosodic Phonology}), and others concentrate on semantics (e.g. \is{informativeness}Morphological Informativeness).  The differences in the conceptualization of \isi{boundary strength} mirror general differences in theoretical assumptions about the \isi{morpho-phonological interface}. 
 The different conceptualizations of \isi{boundary strength} also lead to different predictions for the \isi{phonetic realization} of complex words.
 Studies which test these predictions will, in turn, have important implications about the nature of the \is{morpho-phonetic interface}\is{morpho-phonological interface}{interface between morphology}, phonology and phonetics.
 
To further investigate possible effects of \isi{boundary strength}, and to investigate the nature of the morpho-phonological and the \isi{morpho-phonetic interface}, I conducted a corpus study and an experimental study on \isi{morphological gemination} in English.  As morphological {geminates} always occur across morphological boundaries, they provide the perfect test case for investigating possible effects of \isi{boundary strength}. 
 In the studies, I investigated the five English affixes \is{un-}\prefix{un}, \is{negative in-}negative \prefix{in}, \is{locative in-}locative \prefix{in}, \is{dis-}\prefix{dis} and \is{-ly}\suffix{ly}. I tested the predictions of various morpho-phonological and morpho-phonetic approaches. By finding out which approach can account best for the variation in \isi{gemination} with English affixes, important implications about the \is{morpho-phonetic interface}\is{morpho-phonological interface}{interface between morphology}, phonology and phonetics can be drawn.



 
The studies revealed that while the prefix \is{un-}\prefix{un} geminates in corpus and \isi{experimental speech}, the prefixes \is{locative in-}locative \prefix{in} and \is{negative in-}negative \prefix{in} show differences in their \isi{gemination} pattern depending on \isi{speech mode}. In the corpus data, in which one can assume a deeper \isi{semantic processing} than in the experimental data, the prefixes geminate to a similar degree as \is{un-}\prefix{un}. In the experimental data, they show smaller durational differences between doubles and singletons than \is{un-}\prefix{un}, and \isi{gemination} depends on the prosody of the derivatives. The prefix \is{dis-}\prefix{dis} geminates to a weaker degree than \is{un-}\prefix{un} in both studies, and the suffix  \is{-ly}\suffix{ly} never geminates.

These results falsify common assumptions about \isi{gemination} in English (cf. \sectref{Gemination in English}). They also falsify \is{stratal approach}{stratal approaches} of the \isi{morpho-phonological interface} (e.g. \citealt{Kiparsky.1982,Kiparsky.1985,Mohanan.1986,BermudezOtero.2012,Kiparsky.2015,BermudezOtero.2017}). 
The variation in \isi{gemination} can best be accounted for by the \is{informativeness}morphological informativeness and semantic \isi{segmentability} of the affixes. The more meaning an affix carries and the more informative it is, the stronger it geminates. 

The results, furthermore, indicate that, at least in some cases, prosodic structure affects \isi{gemination}. It is, however, yet to be investigated how the \isi{informativeness} and \isi{segmentability} of an affix relate to prosodic structure, and how this relation affects the \isi{acoustic realization} of complex words. 
While there are already approaches which assume a close relation between the prosodic structure of a word and its \isi{segmentability}, such as the Prosodic Word Approach proposed by \cite{Raffelsiefen.1999}, these approaches are currently not specified enough to account for the findings of this book.  More research on the interaction between the prosodic structure of a complex word, its \isi{segmentability} and its  \isi{informativeness} is needed to devise a more adequate model of the \isi{morpho-phonological interface}.

Apart from \isi{gemination}, the studies also revealed general effects of \isi{decomposability} on the \isi{acoustic realization} of complex words. These effects have important implications for models of \isi{morphological processing} and \is{speech production model}{models of speech production}. It was shown that both categorical affix-specific \isi{segmentability} and gradient {word-specific decomposability} affect the \isi{acoustic realization} of complex words. Theories of morpho-phonological processing must thus be devised to account for categorical and gradient effects of morphological structure on speech production.
Furthermore, the corpus study revealed that the two \is{homophone}{homophonous} prefixes \is{negative in-}negative \prefix{in} and \is{locative in-}locative \prefix{in} were realized with different nasal durations. This result calls for \is{speech production model}speech production models which allow morphological structure, or some correlate thereof, to affect \isi{phonetic detail}. 

To conclude, the present book shows that current approaches of the morpho-phonological and \isi{morpho-phonetic interface} are not able to account satisfactorily for the variation found in the \isi{phonetic realization} of complex words. 
 The studies conducted provide empirical evidence that there is a close connection between \is{informativeness}morphological informativeness, \isi{segmentability} and prosody, and that these three factors are crucial in the production of complex words. 
Further research is needed to accurately model their effects on the \isi{phonetic realization} of complex words.
Furthermore, the results of this book call for a revision of \is{speech production model}speech production models. These models must specify the \is{morpho-phonological interface}\is{morpho-phonetic interface}{morpho-phonological-phonetic interface} in such a way that it is able to account for categorical and gradient effects of morphological structure on the \isi{acoustic realization} of complex words.
