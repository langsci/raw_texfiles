\addchap{Preface}
\begin{refsection}
Languages change over time. The process of change is driven, to a large part, by our communicative needs for expressing development in the world around us. While many aspects of language can change, at the semantic level, words can acquire new senses or lose existing ones. They can even, depending on viewpoint, change the senses they represent. We refer to this process as diachronic or historical semantic change. 
There is rich empirical work on semantic change from historical linguistics, sociolinguistics, and cognitive linguistics. However, computational approaches to historical semantic change have only begun to take shape over the past two decades. It is the latter, computational approaches to semantic change, that are the focus of this edited volume.\largerpage[2]

The development of the computational field of semantic change has been motivated by a few primary aims.
Firstly, the study of semantic change itself, using large-scale digital data, that has been made possible by large-scale digitization efforts. These efforts, hand-in-hand with the rise of digital humanities and social sciences, have resulted in electronic longitudinal text at unprecedented scale. This has provided us with new opportunities for historical investigations of word meaning with the use of computational methods, thus enabling us to test existing hypotheses using data at a much larger scale. 

Recently, the inquiry into semantic change has been pursued not only on its own, but also as a basis for other diachronic textual investigation. These include lexicography, culturomics-style studies, temporal classification of unknown texts, and uncovering of document similarities over time. 

Next, semantic translations or accessability has been a driving force. With the rise of huge diachronic corpora that are easily accessible to anyone, one motivation  
has been to make these texts semantically understandable for non-historical linguistic experts. Here, semantic search and temporal information retrieval have been the driving forces.

Finally, semantic change has been used as an application area for modern computational methods. With new, fast, and efficient modeling tools -- both topic modeling as well as neural embeddings of different kinds -- many researchers have been interested in new problems, and data, to test the limits of computational methods.  The time-varying nature of lexical semantics, with many progressing data points, has been one motivation for the rise of interest in computational semantic change.

One of the main challenges for the computational semantic change community so far has been the lack of interaction and collaboration with traditional research and researchers of semantic change in fields like historical linguistics, semantics, typology, and so on. The 1st International Workshop on Computational Approaches to Historical Language Change (LChange'19), held in conjunction with ACL2019, was a first attempt to bring together the international research community around both traditional and computational semantic change, as well as application fields that benefit from semantic change research.\footnote{The scope of the workshop was wider and targeted all language change that could be found using textual corpora as a basis.} The understanding of how our languages behave over time should come from collaboration with, and draw on corresponding efforts within, traditional semantic change research. 

Our aim with LChange'19 was to facilitate better collaboration and understanding across fields. This book represents part of that effort, with the main focus on computational semantic change, its applications and open challenges. The scope of this book encompasses a survey of the field of computational semantic change (Chapter 1, \citealp{chapters/01}),  application fields that benefit, or directly use, semantic change in their research (Chapters 2--4,  \citealp{chapters/04,chapters/05,chapters/x}), methods for, and investigations into semantic change (Chapters 5--9, \citealp{chapters/03,chapters/06,chapters/07,chapters/08,chapters/09}). We provide an overview of existing systems and applications where  semantic change is incorporated (Chapter 10, \citealp{chapters/10}) and finally, an outlook into the future challenges (Chapter 11, \citealp{chapters/11}). 

Even after this book, there are many challenges that remain untackled, and many dimensions along which our field can develop. Bridging the gap between the needs of the widely different applications fields, and the possibilities of (unsupervised) modeling of large scale text, is an important dimension. Solid and shared evaluation frameworks, and evaluation data, is another. 

In particular, our field still lacks in-depth analysis of what semantic information each computational model captures, and whether this corresponds to the desired outcome. 
Because the \textit{optimal} result is highly context dependent, we need to consider the specific needs of the application field in which we are solving problems; for example, the semantic information needed for lexicography will be widely different from what is required in financial, medical, or historical domains. 
Most evaluation of current computational semantic change show that models capture change of some kind, often in high-dimensional vector spaces, and that this change coincides with certain known properties of our words.  However, few benefit from \textit{knowing of change} in high-dimensional space without \textit{knowing what} this change corresponds to, be it change in the set of senses associated to the word, or just a lack of interest in the word itself. 

We also need to know how much change the different models capture: do they predict change to 90\% of the vocabulary and are thus too broad? Do they handle short-term or long-term change? Do they model semantic, syntactic, contextual or cultural change? And do they capture change on different granularity, or only change to a word's main sense?

All of these questions represent opportunities for research, and offer us an exciting future to look ahead to. 

{\sloppy\printbibliography[heading=subbibliography]}
\end{refsection}
