\documentclass[output=paper]{langscibook}
\ChapterDOI{10.5281/zenodo.5761884}
\author{Yvette Bürki\orcid{0000-0002-3081-3622}\affiliation{Universität Bern}}
\title{Prefacio}
\IfFileExists{../localcommands.tex}{
  \addbibresource{localbibliography.bib}
  \usepackage{langsci-optional}
\usepackage{langsci-gb4e}
\usepackage{langsci-lgr}

\usepackage{listings}
\lstset{basicstyle=\ttfamily,tabsize=2,breaklines=true}

%added by author
% \usepackage{tipa}
\usepackage{multirow}
\graphicspath{{figures/}}
\usepackage{langsci-branding}

  
\newcommand{\sent}{\enumsentence}
\newcommand{\sents}{\eenumsentence}
\let\citeasnoun\citet

\renewcommand{\lsCoverTitleFont}[1]{\sffamily\addfontfeatures{Scale=MatchUppercase}\fontsize{44pt}{16mm}\selectfont #1}
  
  %% hyphenation points for line breaks
%% Normally, automatic hyphenation in LaTeX is very good
%% If a word is mis-hyphenated, add it to this file
%%
%% add information to TeX file before \begin{document} with:
%% %% hyphenation points for line breaks
%% Normally, automatic hyphenation in LaTeX is very good
%% If a word is mis-hyphenated, add it to this file
%%
%% add information to TeX file before \begin{document} with:
%% %% hyphenation points for line breaks
%% Normally, automatic hyphenation in LaTeX is very good
%% If a word is mis-hyphenated, add it to this file
%%
%% add information to TeX file before \begin{document} with:
%% \include{localhyphenation}
\hyphenation{
affri-ca-te
affri-ca-tes
an-no-tated
com-ple-ments
com-po-si-tio-na-li-ty
non-com-po-si-tio-na-li-ty
Gon-zá-lez
out-side
Ri-chárd
se-man-tics
STREU-SLE
Tie-de-mann
}
\hyphenation{
affri-ca-te
affri-ca-tes
an-no-tated
com-ple-ments
com-po-si-tio-na-li-ty
non-com-po-si-tio-na-li-ty
Gon-zá-lez
out-side
Ri-chárd
se-man-tics
STREU-SLE
Tie-de-mann
}
\hyphenation{
affri-ca-te
affri-ca-tes
an-no-tated
com-ple-ments
com-po-si-tio-na-li-ty
non-com-po-si-tio-na-li-ty
Gon-zá-lez
out-side
Ri-chárd
se-man-tics
STREU-SLE
Tie-de-mann
}
  \togglepaper[1]%%chapternumber
}{}
\abstract{\noabstract}
\begin{document}
\maketitle


Trabajos sobre contacto lingüístico entre el español y las lenguas amerindias no faltan en la actualidad. Todo lo contrario: muestra de su vitalidad y robustez es la prolífica publicación de monografías, revistas especializadas y volúmenes temáticos. Dentro de este rico panorama, el volumen preparado y editado por Santiago Sánchez Moreano (The Open University, Reino Unido) y Élodie Blestel (Université Sorbonne Nouvelle, Francia), \textit{Prácticas de lenguaje heterogéneas: nuevas perspectivas para el estudio del español en contacto con lenguas amerindias}, constituye un trabajo especialmente atractivo e innovador. Las diferentes contribuciones desvelan, en efecto, el desarrollo epistemológico y metodológico de los estudios de contacto que, con acierto y de manera crítica, describen Santiago Sánchez Moreano y Élodie Blestel en la introducción al volumen. Se trata de una andadura en la que los estudios de contacto se han ido independizando de posturas teóricas y de tradiciones lingüísticas heredadas y, en sintonía y sinergia con otras disciplinas lingüísticas como la antropología lingüística, la sociolingüística interaccional y la sociolingüística posestructuralista de corte crítico, han ido encontrando caminos propios en la búsqueda hacia el entendimiento y la explicación del contacto entre lenguas de manera más integral. 

Esta búsqueda de una explicación holística para abarcar los hechos del lenguaje implica superar el estudio de formas y estructuras que primaba en aquellos de cuño estructuralista y variacionista de la primera ola y girar la mirada hacia los hablantes quienes, en tanto que actores sociales, hacen uso del lenguaje —con sus manifestaciones semióticas y sus “códigos” y modos lingüísticos heterogéneos— para comunicar. El imperativo de entender y explicar el contacto del lenguaje (estas \textit{pratiques langagières} como se definen en el volumen), poniendo a los hablantes en el centro de los eventos comunicativos, por naturaleza siempre situados y dinámicos, trae consigo una serie de implicaciones teóricas y metodológicas que se aborda en la primera parte del volumen.  Las contribuciones de esta parte, que cumplen a carta cabal con lo que promete el título, ofrecen propuestas teóricas para el entendimiento del contacto incorporando conceptos surgidos fuera de la disciplina de la lingüística, como los de mercado, capital y hábito lingüísticos del sociólogo Pierre Bourdieu y que recuperan Nadiezdha Torres Sánchez y Alonso Guerrero Galván para el análisis del contacto. Si bien los conceptos arriba mencionados no son nuevos, pues en estos descansan las vertientes posestructuralistas actuales, sí resultan novedosos en los estudios hispánicos de contacto, y mucho más aún para aquellos centrados en el contacto entre las lenguas amerindias y el español.  También desde el punto de vista metodológico se buscan nuevos horizontes. Así, Isabelle Léglise, utilizando precisamente el término de \textit{pratiques langagères}, que desfocaliza la mirada logocéntrica para abarcar la amplia gama de manifestaciones semióticas con las que los hablantes construyen sus repertorios comunicativos, desarrolla una metodología poderosa para la anotación de corpus diseñados en el estudio de contacto contemplando justamente cuán difusas y porosas pueden ser las fronteras entre lenguas o variedades. También sobre el hecho de la permeabilidad lingüística llama la atención Élodie Blestel, quien apela a la reflexividad de las y los lingüistas para no caer en antiguos esquemas y formas de análisis que parten de ideas teóricas preconcebidas sobre el contacto. Carol Klee, por su parte, propone dialogar y navegar entre posturas y tradiciones epistemológicas muy distintas y solo aparentemente contradictorias para hacer el estudio del contacto del español con lenguas amerindias más productivo y cabal. 

\largerpage
La segunda parte cumple también con la apuesta del volumen, pues ofrece estudios de caso novedosos, proponiendo diferentes métodos y perspectivas como el diálogo constante entre lo micro y lo macro (Aldo Olate Vinet), incorporar como herramienta de análisis la multimodalidad, muy poco explotada en los estudios de contacto entre el español y las lenguas amerindias (Ignacio Satti y Mario Soto), o centrar el foco de investigación en la capacidad creativa de los hablantes quienes amplían y optimizan los recursos lingüísticos provenientes de diferentes sistemas lingüísticos (Azucena Palacios Alcaine). No olvida tampoco el volumen la importancia de los estudios de contacto debido a la migración, cuyas formas y dinámicas desde el “interior” a los centros capitalinos son de importancia insoslayable en los países latinoamericanos (Carola Mick). Y, finalmente, abre también ventanas hacia el conocimiento de variedades de contacto aún relativamente poco difundidas entre un público de lingüistas más amplio, como son por ejemplo los estudios entre el español y el yukuna en Leticia, Colombia (Aura Lemus Sarmiento y Magdalena Lemus Serrano). 

En definitiva, este volumen propuesto por Santiago Sánchez Moreano y Élodie Blestel constituye una valiosa contribución teórico-metodológica al campo de los estudios de contacto entre el español y las lenguas amerindias porque abre, explora y propone nuevas formas de entender y explicar un fenómeno tan antiguo y a la vez tan actual como el contacto en el lenguaje. 

\end{document} 
