% Chapter 1

\chapter{Introduction} % Write in your own chapter title
\label{Chapter1}
%\lhead{Chapter 1. \emph{Introduction}} % Write in your own chapter title to set the page header
Imagine Anna who studies Russian language and history. She reads a book \textit{Rossija, krov'ju umytaja} by Art\"{e}m Ves\"{e}lyj  and comes across the sentence \ref{ex:dovybirali}.

\exg.\label{ex:dovybirali}Okolo pravlenija, po predlo\v{z}eniju Banty\v{s}a, dovybirali \v{c}lena rady.\\
near administration.\glb{sg.gen} along proposal.\glb{sg.dat} Bantysh.\glb{gen} do.vy.take.imp.\glb{pst.pl} member rada.\glb{gen}\\
\vspace{0.3em}
`Near the administration building, following the proposal by Bantysh, a rada member was being elected.'

Anna looks in her Russian dictionary and does not find the verb \textit{dovybirat'} `to finish electing/to elect in addition' there.  What she can find is the verb \textit{vybrat'} `to select'  that has one prefix and one suffix less and is perfective. Anna knows that the semantic contribution of the prefix \Prefix{do-} is similar to `finish', but what she does not know is the aspect of the verb she encountered in \ref{ex:dovybirali}. 

Anna remembers from her Russian classes that one can form \isi{perfective verbs} by \isi{prefixation} and \isi{imperfective verbs} by attaching the \isi{imperfective suffix}. This case, however, is different, as the verb contains two prefixes and the \isi{imperfective suffix}. There are, thus, two possibilities for the order of affix attachment: first two prefixes and then the suffix, or one prefix, the suffix, and the other prefix. These two possibilities are, however, associated with different aspects of the derived verb. The questions ``What does this verb mean?" and ``What is its aspect?" remain unanswered. If \textit{dovybirali} is perfective, it must refer to a completed event of electing. If it is imperfective, it could refer to the process of finishing the elections (which it, in fact, does), or to a repeated event of electing. 

Surprisingly, neither Russian grammar and dictionaries, nor the linguistic literature provides a full answer to these questions. For example, the proposals by \citet{Svenonius:04b} and \citet{Tatevosov:07} predict different \isi{internal} structure and aspect of the verb \textit{dovybirat'}: according to \citet{Svenonius:04b}, the prefix \Prefix{do-} is attached last and the verb is perfective, and according to \citet{Tatevosov:07}, both steps of \isi{prefixation} precede the \isi{suffixation}, so the verb is imperfective.

As the predictions of the two proposals do not coincide, it seems an easy task to find out which one is wrong: one has to apply tests that are used to \isi{determine the aspect} of the verb and check which prediction is correct. These tests are based on the ability of \isi{imperfective verbs} to receive a \isi{progressive interpretation} in non-\isi{past} tense, a habitual interpretation in \isi{past} tense, and to be combined with the auxiliary verb \textit{budet} `will'. All these properties, however, allow to identify \isi{perfective verbs} only in terms of the absence of imperfective characteristics. The problem is the existence of \isi{biaspectual verbs}: verbs that, depending on the \isi{context}, can be used either as perfective or as imperfective. This means that standard tests in principle fail to identify \isi{biaspectual verbs}, as they pattern together with \isi{imperfective verbs}.

In Chapter~\ref{Chapter2} I develop a possible positive \isi{test for perfectivity} and show that in the case of verbs like \textit{dovybirat'} both \citet{Svenonius:04b} and \citet{Tatevosov:07} are to some extent right and wrong at the same time: both derivations (and thus aspects) are possible, but each theory fails to predict their coexistence. Learning from this, in Chapter~\ref{Chapter2} I not only present new data that is problematic for the existent analyses, but also develop a systematic approach that allows to collect and analyse data independently from the theoretical view on the structure of \isi{complex verbs} in Russian. I then show that, if this approach is adopted, it provides evidence for \isi{structural ambiguity} in some cases where no aspectual ambiguity is present, so the class of verbs that require reanalysis with respect to the established syntactic approaches to \isi{prefixation} is broadened.\footnote{Parts of Chapter~\ref{Chapter2} have been published as \citealt{ZinovaFilip:13} and \citealt{ZinovaOsswald:paper}.}

Another puzzling issue arises in situations where the predictions of different analyses (e.g., \citealt{Svenonius:04b} and \citealt{Tatevosov:07}) agree but depend on the interpretation of the prefix. This happens, for example, if the verb contains the \isi{imperfective suffix} and two prefixes, where the leftmost prefix is \Prefix{pere-}, as in the verb \textit{perevybirat'} `to be reelecting/to elect all of'. How can one find out which interpretations are available for the given verb? 

\hspace*{-0.64717pt}Traditional descriptive approaches, adopted in grammars and dictionaries such as \citet{Shvedova:82}, provide information about the range of interpretations a given prefix may receive, but do not indicate which interpretation applies in which situation, unless the derived verb is itself present in the dictionary. The most extensive and detailed analysis of prefix semantics in formal terms is proposed in the recent book by \citet{Kagan:book}. The goal of the study by \citet{Kagan:book} is to unify prefix representations on two levels: first, all prefixes receive scalar \isi{semantic analysis} and second, each prefix is assigned a common core meaning from which different interpretations can be derived. 

\citet{Kagan:book}, however, does not aim to distinguish between the situations where different submeanings arise, nor to explain \isi{prefix combinatorics} and interaction with the \isi{imperfective suffix}. This means that, despite the unified representation, one still cannot derive the exact meaning of the prefixed verb in a given sentence, as this would require more details about how the \isi{context} influences the interpretation of the verb.

In this work, I provide representations that allow to derive both the aspect and the semantics of a given verb. I also aim to predict which combinations of affixes are possible and to formulate the rules that govern complex verb formation in Russian. According to \citet{Shvedova:82}, there are 23 productive prefixes in Russian. They can stack and at some point of the derivation process the \isi{imperfective suffix} can be attached. So, in principle, for each verbal stem there can be more than 20 thousand derived verbs, not taking into account the \isi{polysemy} of individual prefixes. However, from the point of view of a native speaker, the number of possible derivations seems much more restricted. The primary means of explaining this restriction in the recent proposals is the division of all prefixes into \textit{lexical} and \textit{superlexical}. It originates from the proposal of \citet{Isachenko:60} and is advocated in such contemporary works on Russian \isi{prefixation} as \citet{Ramchand:04}, \citet{Svenonius:04b}, \citet{Romanova:06}, and \citet{Tatevosov:07, Tatevosov:09}.

The main idea of the division is to assign all verbal prefixes to either lexical or \isi{superlexical} class. Prefixes that belong to different classes are then associated with distinct structural positions. This allows to significantly limit the number of possible derived verbs. Surprisingly, various authors who agree that dividing prefixes into two classes is crucial for understanding Russian \isi{prefixation} system do not agree on how to perform this division, a fact already noted by \citet{Tatevosov:09}. It turns out that the assignment itself is controversial, because the criteria that are used to identify which class a given prefix belongs to are vague. In Chapter~\ref{Chapter4}, I discuss all of the properties that are typically assigned to verbs of either class and show that no pair among them is true of the same set of prefixes or prefix usages. Based on this, I argue that, despite the differences between the properties of certain prefixes, the view of a strict distinction is problematic and needs to be revised, probably in favour of a continuum between two extremes instead of a discrete classification. 

An implicit movement away from a bipartite distinction is, in fact, already present in papers that advocate the lexical/\isi{superlexical} split: \citet{Svenonius:04b} allows different structural positions for various prefixes of the \isi{superlexical} class, \citet{Tatevosov:07} argues for an additional class of \isi{intermediate prefixes}, and \citet{Tatevosov:09} introduces a \isi{three-way classification} among the \isi{superlexical} prefixes. However, explicit rejection of the bipartite distinction leads to a radical change as it forces us to abandon the hypothesis of distinct structural positions for different prefixes. This hypothesis, in turn, serves as a main limiting force in the \isi{syntactic accounts of} verbal \isi{prefixation} in Russian: it allows us to provide a structure of a given complex verb and predict which affix combinations are impossible.

Instead of the criticised syntactic explanation of \isi{prefix combinatorics}, I propose a formal semantic account that allows to make predictions and block derivations when semantic conflicts occur. In Chapter~\ref{Chapter5}, I prepare the ground for this formalisation: I discuss relevant properties of some of the usages of prefixes \textit{\mbox{za-}}, \Prefix{na-}, \Prefix{po-}, \Prefix{pere-}, and \Prefix{do-}. The analysis I develop is based on the \isi{scalar approach} to verbal \isi{prefixation}, proposed by \citet{Filip:08} and further elaborated by \citet{Kagan:12, Kagan:book}. In Chapter~\ref{Chapter5}, though, I mostly discuss data and provide generalisations based on it in order to do the formal modelling in Chapter~\ref{Chapter7}.

Working out the semantic contribution of prefixes makes it necessary to also account for pragmatic meaning components. The literature is inconclusive in this respect: \citet{Paducheva:96} and \citet{Romanova:06} claim that all \isi{perfective verbs} carry presuppositions, while \citet{Kagan:book} attributes this property only to the prefixes \Prefix{do-} and \Prefix{pere-}. In Chapter~\ref{Chapter6} I discuss these hypotheses. I apply standard tests for presuppositions and show that \isi{perfective verbs} in general are clearly not associated with a \isi{presupposition}, as has been already noticed by \citet{Gronn:04}.\footnote{This is joint work with Hana Filip and published as \citealt{ZinovaFilip:14}.}

Test results, however, do not provide a clear answer with respect to whether the prefixes \Prefix{do-} and \Prefix{pere-} carry presuppositions. To find out more, I collected data from native speakers of Russian using a special questionnaire. This questionnaire is based on the results of recent experimental work by \citet{Chemla:09}. After doing a statistical analysis of the results, I arrive at the conclusion that the idea of a presuppositional component carried by the prefixes has to be discarded. I then propose to model the observed inferences as entailments in positive contexts and (scalar) \isi{implicatures} in negative contexts.\footnote{This is joint work with Hana Filip and published as \citealt{ZinovaFilip:SALT}.}\is{scalar implicatures|see {implicatures}}

In the same chapter, I discuss another pragmatic issue: the competition of prefixed verbs derived from the same base. I show how, by using underspecified semantics and basic pragmatic principles, one can obtain distinct interpretations of the same prefix depending on the \isi{derivational base}. Such interpretation variability is traditionally described as \isi{polysemy}, and the problem of finding which submeaning applies in the particular case has been not accounted for earlier. This part, however, remains at the level of a preliminary proposal and I hope to return to implementing it in future work.

After the data analysis conducted in Chapter~\ref{Chapter5} and Chapter~\ref{Chapter6}, I propose formal semantic representations of the five Russian verbal prefixes in Chapter~\ref{Chapter7}. I show how they combine with the representations of the derivational bases and how the direct object contributes to the interpretation of the verbal phrase. I use a combination of frame semantics and Lexicalised Tree Adjoining Grammars as defined in \citealt{KallmeyerOsswald:13}. The choice of this formal framework is motivated by its flexibility as well as its potential to express \isi{semantic restrictions}. Another important factor of the framework selection is the possibility to provide an implementation of the analysis. 

The idea that drives frame semantics \citep{Loebner:2014} is that frames in the sense of \citet{Barsalou:92} constitute the universal format of representation of concepts. They are recursive attribute-value structures with functional attributes that can also be represented as directed graphs. Let me show the two graphs that emerge from my analysis for the verb \textit{dovybirat'} that Anna could not find in the dictionary.

The first graph, shown in \figref{graph:pf}, represents the semantics of the verb \textit{dovybirat'}$^{\PF}$ `to finish electing' derived from first suffixing the verb \textit{vybrat'}$^{\PF}$ `to elect' and then prefixing it with \Prefix{do-}. The central node of the frame is of the type \textit{bounded event} and is marked with a double circle. This event is a segment of the bounded event that is denoted by the verb \textit{vybrat'}$^{\PF}$ `to elect'. This is shown by a relation between the two nodes: a thicker arrow in the top part of the figure. These two events share the final stage (\FIN attribute) but have different initial stages (\INIT attribute). The final stage is at the same time the maximum of the event, and the initial point of the derived event does not have to be the minimum of the event. This is interpreted as `to finish electing'. The frame also contains information related to the arguments and manner of the verb \textit{vybrat'}$^{\PF}$ `to elect', that I have taken from the FrameNet project\footnote{\url{https://framenet.icsi.berkeley.edu/fndrupal/}}: manner \textit{choosing}, a set of possibilities, a cogniser, and a chosen that I represent as an attribute of the final stage of the event.

\begin{figure}[p]
\includegraphics[width=.95\textwidth,trim=0 0 0 30]{dovybiratpf.pdf}
\caption{Graph representation of the verb \textit{dovybirat'}$^{\PF}$ `to finish electing'\label{graph:pf}}
\end{figure}

\begin{figure}[p]
\includegraphics[width=.95\textwidth,trim=0 0 0 20]{dovybiratipf.pdf}
\caption{Graph representation of the verb \textit{dovybirat'}$^{\IPF}$ `to be finishing electing'\label{graph:ipf}}
\end{figure}

The second frame, shown in \figref{graph:ipf}, shares a lot with the first one. However, the crucial difference can be immediately seen: the central node (marked with the double border of the circle) is now of the type \textit{progression,} which provides an indication that the verb is imperfective. This is the case when the \isi{imperfective suffix} is attached in the last step of the derivation. The derived verb, thus, denotes a partial event of electing that is, in turn, a segment of the whole electing event that contains its final stage. The attributes of the core electing event remain the same.

The frame \isi{semantic analysis} of the Russian \isi{prefixation} system that I develop in Chapter~\ref{Chapter7} illustrates the power and flexibility of the formalism: with basic and easily readable semantics I manage to not only provide the exact interpretation of a given prefixed verb in \isi{context}, but also block unwanted derivations of \isi{complex verbs}, as well as prevent combinations of verbs with inappropriate direct objects and measure phrases.

I then implement the proposal using \isi{XMG} 2 \citep{Petitjean:16}. In Chapter~\ref{Chapter8}, I show parts of the implementation and discuss the technical details. Due to the current restrictions with respect to the tools available for parsing, I only implement a small fragment that consists of six prefix usages, one verbal base, the \isi{imperfective suffix}, and one noun that can serve as a direct object, supplying two different \isi{scales}. The output of the compiler consists of verb models that include various affixes. Each model is accompanied by a tree that shows its \isi{internal} structure, a set of syntactic properties (including aspect), and a frame that represents the  semantics of the verbal phrase. This allows to check the predictions of the account I propose without the risk of overlooking an unwanted derivation or of making a mistake during the derivation of the representation of a complex verb. This is extremely important if one wants to explore verbs that contain three or more derivational affixes.

In order to see how well my analysis does with respect to predicting the \mbox{(non-)}\linebreak[3]existence of certain affix combinations, I compare the output of my analysis with the proposal by \citet{Tatevosov:09}. For this, I implement the \isi{syntactic restrictions} for prefix attachment for the same grammar fragment. I then analyse all the models produced by the two implementations and calculate precision and recall. The \isi{comparison} shows that both approaches describe situations with one or two affixes rather accurately, but both precision and recall of the model built following the proposal of \citet{Tatevosov:09} get low values due to the incorrect predictions of the existence of more \isi{complex verbs}. As for the implementation of the approach I propose, it continues to deliver accurate predictions beyond two affix situations. In addition, the pragmatic reasoning I propose fine-tunes the system and allows to explain the non-existence of extra models produced by the implementation. From this it follows that, with the three component analysis of Russian \isi{prefixation} that I advocate in this thesis, one can achieve full precision and recall in predicting the existence of \isi{complex verbs} that are not listed in the dictionaries.\largerpage

In sum, in this thesis I develop a complex system that allows us to explain Russian \isi{prefixation} and predict the existence, aspectual properties, and semantics of \isi{complex verbs}. The crucial idea of the analysis is the interaction between syntax, semantics, and pragmatics. While all the components are kept simple, their combination allows to explain subtle distinctions and cases that seem exceptional when all the work is assigned to one linguistic module. An important property of the analysis is the possibility to implement it, which is partially performed in this work.
