\addchap{Contributors}

\textbf{Jennifer Ament} holds a PhD in Applied Linguistics specializing in Second Language Acquisition from Universitat Pompeu Fabra. She is an experienced {English} language teacher and has worked at all levels of education in the North American, European, and Asian settings. Her main research interests are in the field of {SLA}, bilingual education, {pragmatics}, and {language policy}. She is currently analysing the effects of {English}-medium instruction ({EMI}) on the {acquisition} of {pragmatic} markers focusing on individual learner differences. She works as a lecturer at the University of Barcelona, is head of production for The {Linguistics} Journal and reviewer for the International Journal for Higher Education. 

\textbf{Pilar Avello} has worked as a teaching and research assistant in the Department of Translation and Language Sciences at Universitat Pompeu Fabra. She completed her PhD within the Department’s doctorate program in Applied {Linguistics} and has also taught language courses within the UPF graduate Translation and Applied {Linguistics} program. Her research adopts a multidisciplinary perspective which combines theoretical linguistics with a strong applied linguistics approach in relation to second language {acquisition} processes and outcomes. She has analysed how {learning context} effects influence the {acquisition} of {English} as a second language, with a focus on the development of {phonological} {proficiency}.

\textbf{Júlia Barón} is a full-time lecturer in the Institute for Multilingualism at the Universitat Internacional de Catalunya and a lecturer in the {English} and {German} Department at the Universitat de Barcelona. She holds a PhD in Applied linguistics and is a member of the GRAL research group (Grup de recerca en adquisició de llengües). She is also the coordinator of the GIDAdEnA group (Grup d'Innovació Docent en Adquisició i Ensenyament de l'Anglès). Her main research interests lie within the {acquisition} of {English} as a Foreign Language, more specifically of the {acquisition} of the {target language} {pragmatics}, as well as how different teaching methodologies affect the instruction of {English} as a {foreign language}.

\newpage 
\textbf{Angélica Carlet} is a professor of {English} Phonetics and Phonology and head of {English} at the Faculty of Education at Universitat Internacional de Catalunya, in Barcelona, Spain. She holds a PhD in {English} philology and a Master’s Degree in second language {acquisition} from the Universitat Autònoma de Barcelona. The influence of one’s native language {phonology} in the {acquisition} of a second language constitutes the main area of her research interest along with the effect of {phonetic} training for non-native speakers of {English}. 

\textbf{Carmen Del Río} is currently a full-time TEFL teacher in Barcelona. She holds a PhD granted by the doctoral programme in Theoretical and Applied {Linguistics} Department of Translation and Language Sciences at Universitat Pompeu Fabra. Her research combines theoretical linguistics from within the field of phonetics, with a strong applied linguistics approach in relation to second language {acquisition} processes and outcomes in different learning contexts. She has focused on adolescent learners in mainstream education and study abroad. She is particularly interested in how {pronunciation} and phonetics penetrate the {foreign language classroom} in mainstream education and on teachers’ approaches to {pronunciation}. 

\textbf{Maria Juan-Garau} is currently Full Professor of Applied {Linguistics} at the Universitat de les Illes Balears. Her research interests centre on language {acquisition} in different learning contexts, whether naturalistic or formal, with special attention to study abroad and {CLIL}. Her work has appeared in various international journals and edited volumes. She has recently co-edited Content-Based Language Learning in Multilingual Educational Environments (Juan-Garau \& Salazar-Noguera, 2015, Springer) and Acquisition of {Romance} Languages (Gui\-ja\-rro-Fuentes, Juan-Garau \& Larrañaga, 2016, De Gruyter Mouton).

\textbf{Leah Geoghegan} is an {EFL} teacher at Caledonia {English} Centre. She holds an MA in Theoretical and Applied {Linguistics} from the Universitat Pompeu Fabra, specializing in language {acquisition} and learning, and is currently undertaking a second MA in Translation Studies at Portsmouth University. She was project manager and main writer for the website “Intclass” (\href{http://www.intclass.org/}{{www.intclass.org}}), a multimedia science tool for the promotion of three main {foreign language} contexts: study abroad, {immersion} and {formal instruction}. Her research interests include second language {acquisition}, {English} as a {lingua franca} and study abroad. She has contributed a chapter to the \textit{Multilingual Matters} volume on Study {Abroad}, Second Language Acquisition and Interculturality. 
 
\textbf{Sonia López-Serrano} is an {English} lecturer at the Department of Translation and Linguistic Sciences at Universitat Pompeu Fabra, where she is a member of the research group ALLENCAM (Language Acquisition from Multilingual Catalonia). She holds a Master’s degree in Second Language Acquisition from the Universidad de Murcia, Spain, and is following the Doctoral programme in {English} {Linguistics} at the same university. Her research interests focus on instructed second language {acquisition} and {L2} writing. In particular, she has conducted and published research on the language learning potential of writing tasks and the effects of different learning contexts on {foreign language} writing development. 

\textbf{Victoria Monje} is a teacher of French as a {foreign language} and holds a Master’s degree in Theoretical and Applied {Linguistics} from Universitat Pompeu Fabra. Her MA thesis (2016) was an innovative contribution to the larger Study {Abroad} and Language Acquisition (SALA) project, led by Professor Carmen Pé\-rez-Vidal. Specifically, her research focuses on the {acquisition} of {phonology} of {English} as an {L2}.

\textbf{Sofía Moratinos-Johnston} is currently in the final stages of her doctoral studies in the Department of {Spanish}, Modern and Classical Languages of the Universitat de les Illes Balears, where she is also a lecturer. Her doctoral thesis focuses on language {learning motivation} in different learning contexts, which include formal {classroom instruction}, {CLIL} and study abroad. Apart from her research at the University of the Balearic Islands, she has also carried out research in the Centre for Applied {Linguistics} (University of Warwick) and the School of Education (University of Birmingham) as a visiting scholar.

\textbf{Carmen Pérez-Vidal} is currently an accredited professor of {English} at the Department of Translation and Linguistic Sciences at the Universitat Pompeu Fabra. Her research interests lie within the field of {foreign language} learning, bilingualism, and the effects of different learning contexts on language {acquisition}, namely study abroad, {immersion} and instructed second language {acquisition}. On this topic, she has conducted extensive research, as the leading researcher of the Study {Abroad} and Language Acquisition (SALA) project and has published internationally. She has edited the volume \textit{Language Acquisition in Study {Abroad} and Formal Instruction Contexts} (Pérez Vidal, 2014), published by John Benjamins, and, more recently, contributed on the above topic to the Routledge Handbook of Instructed Second Language Acquisition, and to the Routledge Handbook on Study {Abroad}.  In 2009, she was the launching co-coordinator of the AILA Research Network (ReN) on Study abroad.

\textbf{Iryna Pogorelova} is currently an academic coordinator and an instructor at the Master's Degree in Teacher Training in Foreign Language Instruction at Universitat Pompeu Fabra and Universitat de Lleida in Spain. She has recently completed her PhD at the Department of Humanities, Universitat Pompeu Fabra. Her dissertation investigated {intercultural} adaptation and the development of {intercultural} sensitivity of {Catalan}/{Spanish} undergraduate students during study abroad. She is also a member of the research group GREILI-UPF (Research Group on Intercultural Spaces, Languages and Identities). Her research interests include academic mobility and the development of {intercultural} competence and language {acquisition} in study abroad contexts. 
 
\textbf{Joana Salazar-Noguera} is an associate professor at the Universitat de les Illes Balears, and since 2009 she coordinates the MA in Modern Languages and Literatures at the same university. She holds a PhD on {English} Philology. She has also taught at secondary education for eleven years and has published on the adequacy of {EFL} ({English} as a Foreign Language) methodologies in the {Spanish} educative context. Her most recent publications are the co-edited book Content-based learning in {multilingual} educational environments (Juan-Garau \& Salazar-Noguera, 2015), and the article A Case Study of Early Career Secondary Teachers’ Perceptions of their Preparedness for Teaching: Lessons from Australia and Spain (Salazar-Noguera \& McCluskey, 2017).

\textbf{Ariadna Sánchez-Hernández} is a postdoctoral scholar of Applied {Linguistics} at Leuphana University of Lüneburg. She is a member of LAELA research group ({Linguistics} Applied to the Teaching of {English} Language) at Universitat Jaume I. She holds a Ph.D. in Applied {Linguistics} from Universitat Jaume I, a Master of Education in Curriculum and Instruction from Shawnee State University, and she has been a lecturer of {Spanish} at Ohio University. Her research interests include {interlanguage} {pragmatics}, second language {acquisition}, {intercultural} competence, {acculturation}, and the study abroad context.

\textbf{Isabel Tejada-Sánchez} is currently an Assistant Professor in the Department of Languages and Culture at the University of Los Andes in Bogotá, Colombia, where she has been a faculty member since 2014. Isabel completed her Ph.D. on Language Sciences and Linguistic Communication in a joint program at Universities Paris 8 and Pompeu Fabra. Her research interests lie in the areas of instructed second language {acquisition}, language teacher education, and {language policy}. She has taken part in several Applied {Linguistics} international conferences. In particular, she was member of the organizing committee of the 39\textsuperscript{th} Language Testing Research Colloquium (LTRC). Dr. Tejada-Sánchez was also an ETS grantee between 2015 and 2017 within the {English}-language Researcher/Practi\-tion\-er Grant Program.

\textbf{Dakota J. Thomas-Wilhelm} is currently a full-time {English} as a Second Language instructor in {English} as a Second Language Programs at the University of Iowa. He holds a Master’s degree in Theoretical and Applied Linguistics specializing in Language Acquisition and Language Learning from Universitat Pompeu Fabra and is following the Doctoral program in {English} Studies with a focus in Second Language Acquisition at Universitat Autònoma de Barcelona. His research interests lie within the field of {SLA}, {English} grammar instruction, and the design and implementation of linguistically-informed ESL materials and curriculum. In particular, his current focus concerns the designing of linguistically-informed materials to teach difficult grammar concepts using syntax and semantics. 

\textbf{Mireia Trenchs-Parera} received her doctorate in Applied {Linguistics} from Teachers College-Columbia University, New York. She is an Associate Professor at the Department of Humanities, Universitat Pompeu Fabra, Barcelona, the lead researcher of group GREILI-UPF (\textit{Grup de Recerca en Espais Interculturals, Llengües i Identitats}) and a senior member of the research group ALLENCAM. Her current research interests include studies on language attitudes, ideologies and practices and teaching and learning languages in {multilingual} and study abroad contexts.  Her most recent research project deals with translingualism (The Translinguam Project). She has published her research in books from prestigious publishing houses (John Benjamins, Multilingual Matters, and Routledge, among others) and in recognized scientific journals (\textit{Language and Linguistic Compass, International Journal of Bilingual Education and Bilingualism}, \textit{Journal of Multilingual and Multicultural Development, Journal of Sociolinguistics, Canadian Modern Language Review,} and \textit{{English} for Specific Purposes}). 