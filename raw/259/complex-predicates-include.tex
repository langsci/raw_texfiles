\begin{document}
\maketitle
\label{chap-complex-predicates}


\section{Introduction}

Words such as verbs, nouns, adjectives or prepositions typically denote predicates that are
associated with arguments, and those arguments are typically syntactically realized as the subject,
complements or specifier of those words. For instance, a verb such as \textit{to eat} has two
arguments, realized as its subject and its object, and understood as agent (the eater) and patient
(what is eaten). Usually, arguments are associated with just one predicate (one word). However, in
constructions called \textit{complex predicates}, two or more predicates associated with words
behave as if they formed just one predicate, while keeping their status as different words in the
syntax.  For instance, tense auxiliaries in \ili{Romance} languages form a complex predicate with
the participle which follows, but they are different words, since they can be separated by an
adverb, as in \ili{French} \emph{Lucas a rapidement lu ce livre} `Lucas has quickly read this book';
see (\ref{GSexemple1}).  Several properties set apart complex predicates from ordinary predicates,
and those properties can differ from one language to another. In HPSG, complex predicates are
analyzed as constructions in which one predicate, the head, ``attracts'' the arguments of the other,
that is, the syntactic arguments of one word or predicate include the syntactic arguments of another
word or predicate. This chapter is devoted to the various analyses of complex predicates that have
been proposed within HPSG and some of the cross-linguistic variation in the behavior of complex
predicates, focusing on \ili{French}, \ili{German}, \ili{Korean} and \ili{Persian}.

\section{What are complex predicates?}


The term \textit{complex predicate} does not have a universally accepted definition. In this section, we explain
how it is used in HPSG to name a syntactic phenomenon where two (or more) words 
form what appears to be a single predicate because the head is attracting the (syntactic) arguments of its complement.
We then mention the work that has been done in different languages on this aspect of natural language grammars and the constructions in which it manifests itself. 
Finally, we contrast our use of the term \textit{complex predicates} with other uses of the term and
with related phenomena, in particular serial verb constructions (SVCs). 

%\inlinetodostefan{Stefan: To make the volume as such more uniform, please start with an introduction explaining what you are doing. A section should never start with a subsection right away.} % EP 25.07.2020: Taken care of.

\subsection{Definition}\label{GSsection1.1}

In the HPSG tradition, a complex predicate is composed of two or more words, each of which is itself a predicate. By predicate, we mean either a verb or a word of a different category (noun, adjective, preposition, particle) which is associated with an argument structure. A complex predicate is a construction in which the head attracts the arguments of the other predicate, which is its complement: the arguments selected by the complement predicate ``become'' the arguments of the head \citep{HN89b, HN94a, HN98a}. The phenomenon is called \emph{argument attraction}, \emph{argument composition}, \emph{argument inheritance} or \emph{argument sharing}.

To take an example, tense auxiliaries and the participle in \ili{Romance} languages are two different words, since they can be separated by adverbs, as in the \ili{French} examples in (\ref{GSexemple1}), but the two verbs belong to the same clause, and, more precisely, the syntactic arguments belong to one argument structure. We admit that the property of monoclausality can manifest itself differently in different languages \citep[57--59]{Butt2010a}. In the case of \ili{Romance} auxiliary constructions, the first verb (the auxiliary) hosts the clitics which pronominalize the arguments of the participle: corresponding to the NP complement \emph{son livre} `his book' in (\ref{GSexemple1a}), the pronominal clitic \emph{l(e)} is hosted by the auxiliary \emph{a} `has' in (\ref{GSexemple1b}) and (\ref{GSexemple1c}). This contrasts with the construction of a control verb such as \emph{vouloir} `to want', where the clitic corresponding to the argument of the infinitive is hosted by the infinitive, as in (\ref{GSexemple2}) \citep[from][406]{AG2002b-u}: 

\eal 
	\label{GSexemple1} 
	\ex[]{
	\gll Paul a   rapidement lu   son livre.\\
		 Paul has quickly    read his book\\\jambox*{(\ili{French})}
	\glt `Paul has quickly read his book.'}\label{GSexemple1a} 
		
	\ex[]{
	\gll Paul l'a    rapidement lu.\\
		 Paul {it has} quickly    read\\
	\glt `Paul has quickly read it.'}\label{GSexemple1b}  
		
	\ex[*]{
	\gll Paul a   rapidement le lu.\\
		 Paul has quickly    it read\\
	\glt Intended: `Paul has quickly read it.'}\label{GSexemple1c} 
\zl


\eal 
	\label{GSexemple2} 
	\ex[]{
	\gll Paul veut  lire son livre.\\
		 Paul wants read his book\\\jambox*{(\ili{French})}
	\glt `Paul wants to read his book.'}\label{GSexemple2a}
		
	\ex[]{
	\gll Paul veut  le lire.\\
		 Paul wants it read\\
	\glt `Paul wants to read it.'}\label{GSexemple2b}
		
		
	\ex[*]{
	\gll Paul le veut lire.{\footnotemark}\\
		 Paul it wants  read\\
	\glt Intended: `Paul wants to read it.' \footnotetext{Possible in an earlier stage of \ili{French}.}}\label{GSexemple2c}			
\zl


This approach to complex predicates goes back to Relational Grammar \citep{aissen1983clause}: although formalized in a different way, their analysis of causative constructions in \ili{Romance} languages relies on such argument attraction, under the name of \emph{clause union}. Similarly, in Lexical Functional Grammar, \citet{andrews1999complex} speak of complex predicates as building a domain of grammatical relations sharing. It is also present in Categorial Grammar \citep{Geach70a}, with complex categories whose definition takes into account the nature of the argument they combine with and the operation of function attraction. In particular, \citet[301]{kraak1998deductive} accommodates complex predicates by introducing a specific mode of combination called \emph{clause union mode}, where two verbs (two lexical heads) are combined. But, in this account, there is no argument attraction in general, the mechanism being specifically defined in order to account for clitic climbing.


There are other definitions of complex predicates. The term has been used to describe the complex
content of a word, when it can be decomposed. For instance, the verb \emph{dance} has been analyzed
as incorporating the noun \emph{dance} and considered a ``complex predicate'' \citep[31, 41]{HK97a-u}. In the sense adopted here, complex predicates involve at least two words, and are syntactic constructions. Closer to what we consider here to be complex predicates is the case of \ili{Japanese} passive or causative verbs, illustrated in (\ref{GSexemple3}).

\ea{
\label{GSexemple3}
\gll tabe-rare-sasete-i-ta.\\
     eat-\textsc{pass}-\textsc{caus}-\textsc{prog}-\textsc{pst}\\\jambox*{(\ili{Japanese})}
\glt `(Someone) was causing (something) to be eaten.'}
\z

\noindent
The causative morpheme adds a causer argument, and behaves as if it took the verb stem as its
complement (more precisely, the verb stem with the passive morpheme, in this case), whose expected
subject appears as the object of the causative verb. This operation is like argument
attraction. However, it happens in the lexicon rather than in syntax: the elements in
(\ref{GSexemple3}) are bound morphemes, and they form a word
\parencites{manning1999lexical}.\footnote{
  \citet{Gunji99a-u} proposes a dual representation of Japanese causatives, with a VP embedding structure
  as well as a monoclausal morphological and phonological structure.
% Gunji whole paper
} Thus, we do not consider
%Manning et al whole paper
causative verbs in \ili{Japanese} to constitute complex predicates.


Complex predicates are sometimes given a semantic definition: the two elements together describe one
situation. This may be appropriate for some complex predicates, such as light verb constructions (\emph{to have a rest}, \emph{to make a proposal}) \citep[71--74]{Butt2010a}. However, such a semantic definition does not coincide with the syntactic one. It is true that the head verb of a complex predicate tends to add tense, aspectual or modal information, while the other element describes a situation type. Thus, in (\ref{GSexemple1}), the two verbs jointly describe one situation, the auxiliary adding tense and aspect information. But the semantics of a complex predicate is not always different from that of ordinary verbal complements. Thus, there is no evident semantic distinction depending on whether the \ili{Italian} restructuring verb \emph{volere} `to want' is the head of a complex predicate (\ref{GSexemple4a}) or not (\ref{GSexemple4b}), and the two verbs do not seem to describe just one situation \citep[314]{Monachesi98a}.  

\eal 
	\label{GSexemple4} 
	\ex{ 
	\gll Anna lo vuole comprare.\\ 
	     Anna it wants buy\\\jambox*{(\ili{Italian})}
	\glt `Anna wants to buy it.'}\label{GSexemple4a}
		
	\ex{ 
	\glll Anna vuole comprarlo.\\
		   Anna vuole comprar-lo \\
	     Anna wants buy-it\\
    \glt `Anna wants to buy it.'}\label{GSexemple4b} 
\zl

The same point is made for \ili{Hindi} in \citew[289--297]{poornima2009hindi}. They show that there exist two structures combining an aspectual verb and a main verb; in one of them, the aspectual verb is the head of a complex predicate while, in the other one, it is a modifier of the main verb. In more general terms, complex predicates show that syntax and semantics are not always isomorphic in a language. Thus, although the semantic definition of complex predicates may be useful for some purposes, we will ignore it here.

The distinction between complex predicates and serial verb constructions, for example the one illustrated in (\ref{GSexemple5}) (from \citealt[294]{MH2016}), where both \emph{sàán} and \emph{rrá} are verbs, is not obvious (e.g.\ \citealt{andrews1999complex, MH2016}). The main reason is that the constructions which have been dubbed SVCs are different in different languages; we agree with \citet{andrews1999complex} that they do not share a grammatical mechanism, but they do share more superficial tendencies, such as their resemblance to paratactic constructions due to the absence of marking of complementation or coordination, and they also involve more semantic relations than are usually associated with complementation or coordination.

\ea{
	\label{GSexemple5}
	\gll \`Oz\'o s\`a\'an  rr\'a  \'ogb\`a.\\
	     Ozo     jump      cross  fence\\\jambox*{(\ili{Edo})}
	\glt `Ozo jumped over the fence.'} 
\z

Accordingly, SVCs are not within the purview of complex predicates, and will not be studied in this chapter (but see \citealt{lee14}).

\subsection{Constructions involving complex predicates}\label{GSsection1.2}

Complex predicates enter into a number of constructions across languages. They differ from ordinary constructions in different ways, depending on the construction, such as the position of pronominal clitics in \ili{Romance} languages (``clitic climbing''), word order or special semantic combinations. 

The following have been particularly studied in HPSG:

\begin{itemize}
	
	\item \ili{Romance} languages' tense auxiliaries, copulas and other verbs taking predicative complements, restructuring verbs headed by certain subject raising or control verbs, as well as certain causative and perception verbs \citep*{abeille1994complementation, abeille2000french, abeille2001deux, abeille2001varieties, AG2002b-u, AG2010, abeille1995doublestructure, AGMS98a, AGS1998, Monachesi98a, aguila20};
	
	\item certain constructions in \ili{German} and \ili{Dutch}, called coherent constructions, headed by tense auxiliaries, certain raising and control verbs, certain verbs with predicative complements, as well as the copula and particle verbs \citep{HN89b, HN94a, Rentier94, Kiss94, Kiss95a, BvN98a, HN98a, Kathol98b, Kathol2000a, Meurers2000b, DM2002, dKM2001a,  Mueller2002b, Mueller2003a, muller2018clause};
	
	\item \ili{Korean} auxiliaries, control verbs, \emph{ha} causative verb and light verb constructions \citep{ Sells1991, Ryu:93, Chung98a-u, lee2001argument, choi2001mixed, Yoo2003, Kim2016a-u};
	
	\item \ili{Hindi} aspectual predicates \citep{poornima2009hindi}; 
	
	\item \ili{Persian} light verb constructions (combinations of a semantically light verb with a predicate belonging to diverse categories; \citealt{bonami2010persian, MuellerPersian, bonami2015diversity});   
	
	\item causatives in various languages (among them \ili{German}, \ili{Italian}, \ili{Turkish}), including both analytical causatives (complex predicates in the sense adopted here) and synthetic causatives \citep{Webelhuth98a-u}. 
	
\end{itemize}

\noindent
In this chapter, we examine some of these constructions which illustrate the different ways in which complex predicates differ from ordinary verbs.


\section{The basic mechanism in HPSG: Argument attraction}\label{GSsection2}
\label{complex-predicates-sec-argument-attraction}

\largerpage
In HPSG, complex predicates are analyzed in the following way: one of the predicates is the head of
the construction, and it attracts the syntactic arguments of the other predicate, that is, its
complements and, possibly, its subject. We illustrate it with tense auxiliaries in \ili{French}
\citep{abeille1994complementation, AG2002b-u}.

In \ili{French}, auxiliary constructions consist of a tense auxiliary (\emph{avoir} `to have' or \emph{\^etre} `to be') followed by a past participle and its complements, as illustrated in (\ref{GSexemple1}) on p.\,\pageref{GSexemple1}. The auxiliary is the head.
It bears inflectional affixes (for tense and person) like any other verb, and if the sentence is declarative, it is in the indicative form as expected; for example, the auxiliary in (\ref{GSexemple1}) has the form of a present indicative third person. 
The auxiliary also hosts pronominal clitics, as verbal heads in general do, as shown in (\ref{GSexemple1b}) and (\ref{GSexemple1c}). Moreover, it can be gapped alone, as (\ref{GSexemple6a}) shows, while the participle can only be gapped with the auxiliary, as illustrated by (\ref{GSexemple6b}) and (\ref{GSexemple6c});\footnote{Note that (\ref{GSexemple6c}) is acceptable with the possession verb \emph{avoir}.} this is expected if the auxiliary is the head, since it behaves like \emph{pense} `think' in (\ref{GSexemple6d}), while the participle behaves like the infinitive in (\ref{GSexemple6e}) and (\ref{GSexemple6f}). 

\largerpage
\eal
	\label{GSexemple6}
	\ex[]{
        \longexampleandlanguage{
	\gll Lola a   achet\'e des   pommes, et  Alice (a)             cueilli des   p\^eches.\\
	     Lola has bought   some  apples  and Alice \spacebr{}has   picked  some  peaches\\}{French}
	\glt `Lola has bought apples, and Alice (has) picked peaches.'}\label{GSexemple6a}
		
	\ex[]{
	\gll Lola a   achet\'e des   pommes, et  Alice (a             achet\'e) des   p\^eches.\\
	     Lola has bought   some  apples  and Alice \spacebr{}has  bought    some  peaches\\
	\glt `Lola has bought apples, and Alice (has bought) peaches.'}\label{GSexemple6b}
		
	\ex[\#]{
	\gll Lola a   achet\'e des   pommes, et  Alice a   des   p\^eches.\\
		 Lola has bought   some  apples  and Alice has some  peaches\\
	\glt `Lola has bought apples, and Alice has peaches.'}\label{GSexemple6c}
				
	\ex[]{
	\gll Lola pense  acheter des   pommes, et  Alice (pense)            cueillir des  p\^eches.\\
	     Lola thinks buy     some  apples  and Alice \spacebr{}thinks   pick     some peaches\\
	\glt `Lola is thinking of buying apples, and Alice (is thinking of) picking peaches.'}\label{GSexemple6d}
		
	\ex[]{
	\gll Lola pense  acheter des  pommes, et  Alice (pense            acheter) des  p\^eches.\\
	     Lola thinks buy 	 some apples  and Alice \spacebr{}thinks  buy      some peaches\\
	\glt `Lola is thinking of buying apples, and Alice (is thinking of picking) peaches.'}\label{GSexemple6e}
		
	\ex[*]{
	\gll Lola pense  cueillir des  pommes et  Alice pense  des  p\^eches.\\
	     Lola thinks pick     some apples and Alice thinks some peaches\\
	\glt Intended: `Lola is thinking of picking apples and Alice is thinking of (picking) peaches.'}\label{GSexemple6f}
\zl

The auxiliary construction in \ili{French} is a complex predicate: the clitic corresponding to a complement of the participle is hosted by the auxiliary (it is said to ``climb'') as in (\ref{GSexemple1b}). Moreover, it occurs in bounded dependencies such as the infinitival complement of adjectives like \emph{facile} `easy' or \emph{impossible} `impossible', whose nominal complement is unexpressed, as in (\ref{GSexemple7a}); this unexpressed complement can be that of a participle (\ref{GSexemple7c}) but not that of an infinitive complement (\ref{GSexemple7b}). This follows if the unexpressed complement is in fact treated as the complement of the auxiliary.

\eal 
	\label{GSexemple7}
	\ex[]{
	\gll Cette technique est impossible \`a ma\^itriser en un  jour.\\
	     this  technique is  impossible to  master      in one day\\\jambox*{(\ili{French})}
	\glt `This technique is impossible to master in one day.'}\label{GSexemple7a}
		
	\ex[*]{
	\gll Cette technique est impossible \`a r\'eussir \`a ma\^itriser en un  jour.\\
		 this  technique is  impossible to  manage    to  master      in one day\\
	\glt Intended: `This technique is impossible to manage to master in one day.'}\label{GSexemple7b}
		
	\ex[]{
	\gll Cette technique est impossible \`a avoir ma\^itris\'e en un  jour.\\
	     this  technique is  impossible to  have  mastered     in one day\\
	\glt `This technique is impossible to have mastered in one day.'}\label{GSexemple7c}
\zl

\largerpage[2]
\noindent
These two properties (clitic climbing and occurrence in bounded dependencies) follow if the complements of the participle become those of \emph{avoir} `to have'.
In fact, both clitic climbing and the dependency found in `easy'/`impossible' constructions belong to the set of bounded dependencies.
In addition, the tense auxiliary \emph{avoir} `to have' is a subject raising verb (see \crossrefchapteralt{control-raising}): the subject is selected by the participle and shared by the auxiliary. For instance, \emph{Paul} is an agent in (\ref{GSexemple1a}) (\emph{Paul a lu son livre}, `Paul has read his book') because \emph{lire} `to read' requires an agent subject, and in e.g.\ \emph{Il a fait froid} (lit.\ It has made cold, `It [the weather] was cold'), the subject is the impersonal subject \emph{il}, because that is the subject of the participle \emph{fait froid}.
Thus, the auxiliary \emph{avoir} (like tense auxiliary \emph{\^etre} `to be') is, in fact, a
generalized raising verb: its whole argument structure is identified with that of the participle. A
simplified description of subject raising verbs and tense auxiliaries is given in (\ref{GSexemple8})
(for the feature [\light{}$\pm$], see Section~\ref{GSsection3}).\footnote{
  $\oplus$ stands for the relation \texttt{append}\isrel{append} and simply concatenates two lists. For
    example, \nliste{ a, b } $\oplus$ \nliste{ c, d } = \nliste{ a, b, c, d }.
}

\begin{exe}
	\ex 	\label{GSexemple8}
	\begin{xlist}
        \ex{Ordinary subject raising verb:\\
		\avm{[arg-st & \1 \+ <[subj & \1\\
                                       comps & <> ]> \+ list ]}
	\label{GSexemple8a}
	}
	
	    \ex{Tense auxiliary as head of a complex predicate:\\
		\avm{
			[arg-st & \1 \+	<[subj & \1\\
			         	  arg-st & \1 \+ \2\\
					  light & $+$ ]> \+ \2 ]
		}
	\label{GSexemple8b}
	}
	\end{xlist}
\end{exe}

\noindent
The subject raising verb takes a saturated complement, which is described as the second
element of the argument structure, expecting a subject \ibox{1} identified with the subject of the
raising verb. The notation \ibox{1} instead of \sliste{ \ibox{1} } indicates that this element may be
absent: it is meant to accommodate subjectless verbs. In addition, the raising verb may have its own
complements, noted here as \emph{list}. On the other hand, the auxiliary is not only a subject raising
verb, but takes as a complement a participle which has not combined with any complements.

The arguments of a word are made up of subject and complements. The relation between (expected)
arguments and realized subject and complements is as in (\ref{GSexemple9}) (see
\citealt[171]{GSag2000a-u}; \citealt[\page 12]{BMS2001a}). The arguments include the subject
and  the complements, but also a list of non-canonical elements (possibly empty; see below).\footnote{\citet[\page 170]{GSag2000a-u} state the
  following about $\ominus$: ``Here `$\ominus$' designates a relation of contained list difference. If
$\lambda_2$ is an ordering of a set $\sigma_2$ and $\lambda_1$ is a subordering of $\lambda_2$, then
$\lambda_2 \ominus \lambda_1$ designates the list that results from removing all members of
$\lambda_1$ from $\lambda_2$; if $\lambda_1$ is not a sublist of $\lambda_2$, then the contained
list difference is not defined.}

\largerpage%[2]
\ea
Argument Realization Principle\is{principle!Argument Realization (ARP)} (adapted from \citealt[171]{GSag2000a-u}):\\
\type{word} \impl
\avm{
[subj   & \1\\
% spr    & \2\\
 comps  & \2 \- \listOf{non-canonical}\\
 arg-st & \1 \+ \2]
}
\label{GSexemple9}
\z

In (\ref{GSexemple10a}), the participle \emph{lu} `read' selects the argument \emph{son livre} `her book', which is attracted by the auxiliary \emph{a} `has'. Accordingly, it is realized as the complement of the auxiliary \emph{a}. The structure of the VP in (\ref{GSexemple10a}) is given in Figure~\ref{GSfigure1}.

\eal
	\label{GSexemple10}
	\ex{
	\gll Marie a   lu   son livre.\\
	     Marie has read her book\\\jambox*{(\ili{French})}
	\glt `Marie has read her book.'}\label{GSexemple10a}
		
	\ex{
	\gll Marie l'a    lu.\\
	     Marie {it has} read\\
	\glt `Mary has read it.'}\label{GSexemple10b}
\zl

%%%%%%%%%%%%%%%%%%%%%%%%%%%%%%%%%%%%%%%%%%%%%%%%%%%%%%%%%%%%%%%%%

\begin{figure}
    {\centering
% \begin{forest}
%  [VP
%  [V [\avm{
%             [head &	[\type*{basic-verb}
% 					vform & indic ] \\
%             subj & < \1 > \\
%             comps & < \3, \2 > \\
%             arg-st & < \1, \3, \2 > ]
%             }[a\\has, align=center, base=bottom]]] 
%  [\ibox{3} V [\avm{
%             [head &	[\type*{basic-verb}\\
% 					vform & pst-ptcp ] \\
%             subj & < \1 > \\
%             comps & < \2 > \\
%             arg-st & < \1, \2 > ]
%             }[lu\\read, align=center, base=bottom, tier=word]]]
%  [\ibox{2} NP, before computing xy={s'+=30pt} 
%             [son livre\\her book, base=bottom, tier=word, roof]]]
% \end{forest}} 
\begin{forest}
 [VP
 [V [\avm{
            [head &	[\type*{basic-verb}
					vform & indic ] \\
            subj  & < \3 > \\
            comps & < \1, \2 > \\
            arg-st & < \3, \1, \2 > ]
            }[a\\has, align=center, base=bottom]]] 
 [\ibox{1} V [\avm{
            [head &	[\type*{basic-verb}\\
			 vform & pst-ptcp ] \\
            subj & < \3 > \\
            comps & < \2 > \\
            arg-st & < \3, \2 > ]
            }[lu\\read, align=center, base=bottom, tier=word]]]
 [\ibox{2} NP, before computing xy={s'+=30pt} 
            [son livre\\her book, base=bottom, tier=word, roof]]]
\end{forest}}
\caption{VP structure in French}
    \label{GSfigure1}
\end{figure}

%%%%%%%%%%%%%%%%%%%%%%%%%%%%%%%%%%%%%%%%%%%%%%%%%%%%%%%%%%%%%%%%%
\begin{figure}
\begin{forest}
type hierarchy
 [synsem
 [non-canon
    [aff]
    [gap]%,calign with current]
    [null-pro]]
 [canon]]
\end{forest}
\caption{Subtypes of \type{synsem}}\label{GSexemple11}
\end{figure}

Let us turn to pronominal clitics. Arguments are of type \emph{synsem}, which can have different subtypes (Figure~\ref{GSexemple11}). Usually, these subtypes are not specified on lexemes, but they are on
words occurring in sentences.

\ili{Romance} clitics, illustrated by \emph{l}(\emph{e}) in (\ref{GSexemple10b}), are analyzed as affixes (\emph{aff}) on verbs, which correspond to arguments of the verb \citep{MS97a-u}. They belong to the argument structure of the participle, and are attracted by the auxiliary, although they are not realized as complements. In (\ref{GSexemple10b}) and Figure~\ref{GSfigure2}, the arguments of the auxiliary are the subject \ibox{1}, the participle \ibox{3}, and \ibox{2}; \ibox{2} is typed as an affix, third person, masculine singular. It belongs to the argument structure, but not to the complement list of the auxiliary (see (\ref{GSexemple9})).


%%%%%%%%%%%%%%%%%%%%%%%%%%%%%%%%%%%%%%%%%%%%%%%%%%%%%%%%%%%%%%%%%

\begin{figure}
    {\centering
\begin{forest}
 [VP
 [V [\avm{%
            [head & reduced-verb \\
            subj & < \1 > \\
            comps & < \3 > \\
            arg-st & < \1, \3, \2 > ]}
        [l'a\\{it has}, align=center, base=bottom]]] 
 [\ibox{3} V [\avm{%
            [head & basic-verb \\
            subj & < \1 > \\
            comps & < \2 > \\
            arg-st & < \1, \2 [\type{aff}, \type{3rd}, \type{msg}] > ]}
            	[lu\\read, align=center, base=bottom, tier=word]]] ]
\end{forest}}\caption{Clitic climbing in French}
    \label{GSfigure2}
\end{figure}

%%%%%%%%%%%%%%%%%%%%%%%%%%%%%%%%%%%%%%%%%%%%%%%%%%%%%%%%%%%%%%%%%

We distinguish between \emph{basic verbs} and \emph{reduced verbs}, following \citet{AGS1998}. With basic verbs, the argument list is simply the concatenation of the subject and complements, while reduced verbs have at least one affix argument which belongs to the argument list, but not to the complement list. Such verbs are subject to a morphological rule which realizes this affixal argument as an affix, the so-called clitic pronoun \emph{l}(\emph{e}). Thus, in Figure~\ref{GSfigure1}, both the auxiliary \emph{a} `has' and the participle \emph{lu} `read' are basic verbs: the arguments tagged \ibox{3} and \ibox{2} are also complements. On the other hand, in Figure~\ref{GSfigure2}, the participle is a basic verb -- argument \ibox{2} is typed as an affix, but is also a complement -- while the auxiliary is a reduced verb: argument \ibox{2} is not a complement of the auxiliary, and the verb hosts the affix \emph{l}(\emph{e}). Note that the Argument Realization Principle\is{principle!Argument Realization (ARP)} (\ref{GSexemple9}) allows a verb to expect a complement typed as \emph{affix}: it allows arguments to be non-canonical (among which affixes), but it does not force complements to be canonical. If the complement is typed as affix, it has to be attracted by a different head, or it is realized as an affix. In the latter case, the verb must be a \emph{reduced verb}. This is not the case for the participle in Figure~\ref{GSfigure2}, which is a \emph{basic verb}.

In \ili{French}, past participles never host clitics, as we saw in (\ref{GSexemple1c}), which we assume to be a morphological property. But in \ili{Italian}, past participles may host clitics, although never when they combine
with the auxiliary. The specification that the participle complement of the auxiliary is a basic verb accounts for this property, because basic verbs are not the target of the morphological rule realizing the affixal argument as an affix. Although both verbs in Figure~\ref{GSfigure2} have an affixal argument, one is a basic verb (the participle), the affixal argument being also an expected complement, and the other is a reduced verb (the auxiliary), this affixal argument not being an expected complement.\footnote{It is worth noting that tense auxiliaries can take as complement a coordination of participles:
	
\eal
	\label{FootExample1}
	\ex{
	\gll Jean a   acheté et  lu   ce   livre.\\
	     Jean has bought and read this book\\\jambox*{(\ili{French})}
	\glt `Jean bought and read this book.'}\label{FootExample1a}
		
	\ex{
	\gll Jean l'a    acheté et  lu. \\
	     Jean {it has} bought and read\\
	\glt `Jean bought and read it'}\label{FootExample1b}
\zl
This may be seen as raising a difficulty for the analysis of their complement based on argument structure sharing, since argument structure characterizes words rather than phrases. However, coordinations of words are a special kind of phrases, since the conjuncts must share their argument structure. It is plausible that such coordinations inherit an argument structure from the conjuncts (for further discussion of coordination, see \crossrefchapteralt{coordination}).}

\section{Different structures for complex predicates:~ Restructur- ing verbs and the copula in Romance
  languages}\label{GSsection3}
\label{sec-copula-romance}

In addition to tense auxiliaries, \ili{Romance} languages have other cases of complex predicates that are headed by restructuring verbs, by the copula and other verbs taking predicative complements, and by certain causative and perception verbs. We focus here on restructuring
verbs and the copula. An analysis of causative and perception verbs is proposed in \citet{abeille1995doublestructure, AGMS98a, AG2010}. 

A comparison of the properties of constructions headed by restructuring verbs in different \ili{Romance} languages illustrates an important aspect of the phenome\-non: argument attraction is compatible with different syntactic structures. Restructuring verbs enter either a flat structure or a verbal complex \citep{Monachesi98a, abeille2001deux, AG2010}.\footnote{However, see recent work by \citet{aguila20}, who analyze French tense auxiliaries in terms of `periphrasis', with a VP complement.} As for the copula, it differs from tense auxiliaries and restructuring verbs in two respects: its complement always behaves like a phrase, although it can be fully saturated for its complements, partially saturated or not saturated at all \citep{abeille2001varieties, AG2002b-u}; and it has a uniform behavior and analysis across the \ili{Romance} languages.

\subsection{Romance restructuring verbs as head of complex predicates} \label{GSsection3.1}

Certain verbs in \ili{Romance} languages, called \emph{restructuring verbs}, exhibit two behaviors: either as ordinary verbs taking a VP complement or as heads of complex predicates \citep{rizzi1982issues, aissen1983clause}. Restructuring verbs are modal, aspectual or movement verbs (such as \emph{venire} `to come’, \emph{andare} `to go’, \emph{correre} `to run’, \emph{tornare} `to come back’ in \ili{Italian}). However, it must be kept in mind that this behavior is lexical: verbs which are close semantically may or may not be heads of complex predicates. 

Several properties show that such verbs can head complex predicates \citep[323--328]{Monachesi98a}. The first is clitic climbing, which is possible with restructuring verbs, though optional (while it is obligatory with tense auxiliaries). The examples in (\ref{GSexemple12}) all mean `John wants to eat them’ (examples from \citealt[113]{AG2010}). For each language, the first example illustrates the complex predicate, and the second one the VP complement construction, with the clitic downstairs.

\eal 
\settowidth\jamwidth{(\ili{Portuguese})}
\label{GSexemple12} 
\ex{
\gll Giovanni \emph{le} vuole mangiare.\\
     Giovanni them      wants eat\\\hfill{(\ili{Italian})}
\glt `Giovanni wants to eat them.'} \label{GSexemple 12a}  
		
\ex{
\glll Giovanni vuole mangiar\emph{le}.\\
		Giovanni vuole mangiar-le \\
     Giovanni wants eat-them\\ 
\glt `Giovanni wants to eat them.'} \label{GSexemple12b} 

\ex{
\gll Juan \emph{las} quiere comer.\\
     Juan them       wants  eat\\\hfill{(\ili{Spanish})}
\glt `Juan wants to eat them.'} \label{GSexemple12c} 
		
\ex{
\glll Juan quiere comer\emph{las}.	\\
	Juan quiere comer-las \\
     Juan wants  eat-them \\ 
\glt `Juan wants to eat them.'}\label{GSexemple12d} 
		
\ex{
\gll O            Jo\~ao quere-\emph{as} comer.	\\
     \textsc{det} Jo\~ao wants-them      eat \\\hfill{(\ili{Portuguese})} 
\glt `Jo\~ao wants to eat them.'}\label{GSexemple12e} 
	
\ex{
\gll O            Jo\~ao quer  comê-\emph{las}.	\\
     \textsc{det} Jo\~ao wants eat-them \\
\glt `Jo\~ao wants to eat them.'}\label{GSexemple12f}
		
\ex{
\gll En           Joan \emph{les} vol   menjar.	\\
     \textsc{det} Joan them       wants eat \\\hfill{(\ili{Catalan})}
\glt `Joan wants to eat them.'}\label{GSexemple12g} 
	
\ex{
\gll En           Joan vol   menjar-\emph{les}.\\
     \textsc{det} Joan wants eat-them\\
\glt `Joan wants to eat them.'}\label{GSexemple12h} 
\zl

The second property showing restructuring verbs' complex predicate status is the medio-passive or middle \emph{si} construction, where the verb hosts the reflexive clitic \emph{si} or \emph{se} (\ref{GSexemple13b}) (depending on the language), and the subject corresponds to the object of the active construction (\ref{GSexemple13a}), with an interpretation close to that of middles in \ili{English}. The construction is possible with restructuring verbs such as \emph{potere} `to be able to' (\ref{GSexemple13c}) and (\ref{GSexemple13d}) (see \citealt[333--336]{Monachesi98a}), but not with verbs only taking an infinitival VP complement such as \emph{parere} `to appear' (\ref{GSexemple13e}) (examples (\ref{GSexemple13d}) and (\ref{GSexemple13e}) from \citealt[122]{AG2010}).


\eal
 	\label{GSexemple13} 
    \ex[]{
	\gll Giovanni stira queste camicie facilmente.\\
	     Giovanni irons these  shirts  easily\\\jambox*{(\ili{Italian})}
	\glt `Giovanni irons these shirts easily.'}\label{GSexemple13a} 
		
	\ex[]{ 
	\gll Queste camicie si          stirano facilmente.\\ 
 	     these  shirts  \textsc{si} iron    easily\\
	\glt `These shirts iron easily.'}\label{GSexemple13b}
		
	\ex[]{
	\gll Giovanni pu\`o stirare queste camicie facilmente.\\
	     Giovanni can   iron    these  shirts  easily\\
	\glt `Giovanni can iron these shirts easily.'} \label{GSexemple13c} 
		
	\ex[]{ 
	\gll Queste camicie si          possono stirare facilmente.\\ 
	     these  shirts  \textsc{si} can     iron    easily\\
	\glt `These shirts can be ironed easily.'}\label{GSexemple13d}		
		
	\ex[*]{ 
	\gll Queste camicie si          paiono stirare facilmente.\\ 
	     these  shirts  \textsc{si} appear iron    easily\\
	\glt Intended: `These shirts appear to be ironed easily.'}\label{GSexemple13e}			
\zl

\largerpage
\noindent
The medio-passive verb alternates with a transitive verb: it is the result of a lexical rule, shown in (\ref{GSexemple14}), which takes a transitive verb like \emph{stirare} as in (\ref{GSexemple13a}) to give a verb whose subject corresponds to the expected object of the transitive verb and which acquires a reflexive clitic noted as \emph{a-aff} (realized as \emph{si} or \emph{se}) as in (\ref{GSexemple13b}) (\citealt[31]{AGS1998}; \citealt{Monachesi98a}). 

\begin{exe} 
\ex{Medio-Passive Lexical Rule\is{lexical rule!Medio-Passive}\label{GSexemple14}:\\	
\avm{
	[arg-st & < NP, NP![\type{acc}]!$_j$ > \+ \1 ]} $\mapsto$
\avm{
	[arg-st & < \NPj, ~![\type{a-aff, acc}]!$_j$ > \+ \1 ]
}
	}
\end{exe}

\noindent
What is crucial here is that the input is a verb taking an accusative NP complement. Hence, a verb
taking a VP complement like \ili{Italian} \emph{potere} `to be able to' or \emph{parere} `to appear'
cannot be the input, since it lacks an NP complement. On the other hand, the corresponding
restructuring verb \emph{potere} can be the input, since it inherits such a complement from the
infinitive: the verb \emph{potere} in (\ref{GSexemple13c}) inherits \emph{queste camicie} `these
shirts' from \emph{stirare} `to iron', allowing it to be the input to rule (\ref{GSexemple14}),
which gives the verb occurring in (\ref{GSexemple13d}).
%On the other hand, the verb \emph{parere} which is not a restructuring verb does not have an NP object and cannot be the input to this rule (\ref{GSexemple14}).

The third relevant property of restructuring verbs is their acceptability in bounded dependencies, as illustrated in (\ref{GSexemple7}) for tense auxiliaries and (\ref{GSexemple15}) for restructuring verbs. (\ref{GSexemple15b}) (from \citealt[341]{Monachesi98a}) relies on \emph{cominciare} `to begin' being a restructuring verb, while \emph{promettere} `to promise' is not (\ref{GSexemple15c}).  

\eal 
	\label{GSexemple15} 
    \ex[]{
	\gll Questa canzone \`e facile da apprendere.\\
		 this   song    is  easy   to learn\\\jambox*{(\ili{Italian})}
	\glt `This song is easy to learn.'}\label{GSexemple15a} 
		
	\ex[]{ 
	\gll Questa canzone \`e facile da cominciare a  apprendere.\\
		 this   song    is  easy   to begin      to learn\\
	\glt `This song is easy to begin to learn.'}\label{GSexemple15b}
		
	\ex[*]{
	\gll Questa canzone \`e facile da promettere di apprendere.\\
	     this   song    is  easy   to promise    to learn\\
	\glt Intended: `This song is easy to promise to learn.'}\label{GSexemple15c} 	
\zl

\largerpage[1.2]
\noindent
The complement of adjectives such as `easy' in \ili{Romance} languages is a bound\-ed dependency: they take an infinitival complement whose own expected complement (we analyze it as a null pronoun; see Figure~\ref{GSexemple11}) is coindexed with its subject (\citealt{AGS1998, Monachesi98a}).\footnote{Forms such as \emph{a}, \emph{da} and \emph{di}, which introduce infinitival complements in (\ref{GSexemple15}), are not analyzed as heads, but as markers, a part of speech which has the feature \textsc{marking} and whose value is specific to the form. Markers select the head with which they combine (for instance, \emph{da} selects an infinitival VP in (\ref{GSexemple15a})), and the feature is shared by the whole VP. Hence, the adjective \emph{facile} `easy' in \ili{Italian} takes as a complement an infinitival VP [\textsc{marking} \emph{da}].}

\ea
\avm{
[head   & adjective\\
 arg-st & < XP$_j$, VP[vform & infinitive\\ 
                        marking & da\\
                        comps   & < ![\type{null-pro}, \type{acc}]!$_j$ >  \+ list \\ ] > ]
}
\z

Complex predicates can occur in this construction because their head attracts the complement of their complement. Thus, in (\ref{GSexemple15b}), \emph{cominciare} `to begin' is expecting the same object as \emph{apprendere} `to learn', which is coindexed with the\pagebreak{} subject of the copular construction, in the same way as \emph{apprendere} is expecting an object in (\ref{GSexemple15a}). 

\largerpage
Fourth and finally, the possibility of preposing the verbal complement of a verb which can take a VP complement or be the head of a complex predicate disappears when there is evidence of a complex predicate. For the sake of simplicity, we now concentrate on \ili{Italian} and \ili{Spanish}. The data in (\ref{GSexemple17}), with a preposed VP, contrast with those in (\ref{GSexemple18}) (both examples from \citealt[132]{AG2010}), where the head verb bears a clitic corresponding to the expected complement of the infinitive. Preposing of the verbal complement is associated with pronominalization (\emph{lo}) in \ili{Italian} (\ref{GSexemple17a}) but not in \ili{Spanish} (\ref{GSexemple17b}), where it is more natural in contrastive contexts.

\begin{exe}
	\ex {[Context] Does he want to talk to Mary?}\label{GSexemple17} 
	\begin{xlist}
	\ex[]{
	\gll Parlare a  Maria, certamente lo vuole.\\
	     talk    to Maria  certainly  it wants\\\jambox*{(\ili{Italian})}
	\glt `Talk to Maria, certainly he wants to.'}\label{GSexemple17a}
		 
	\ex[]{ 
        \longexampleandlanguage{
	\glll Hablarle 	 a  Mar\'ia, seguramente quiere (pero          no  a  su  madre).\\
	Hablar-le 	 a  Mar\'ia  seguramente quiere \hphantom{(}pero          no  a  su  madre \\
	     talk-to.her to Mar\'ia  certainly   wants  \spacebr{}but  not to her mother\\}{Spanish}
	\glt `Talk to Maria, certainly he wants to (but not to her mother).'}\label{GSexemple17b}
	\end{xlist}
\end{exe}

\eal
	\label{GSexemple18} 
    \ex[*]{
	\glll Parlare, certamente glielo    vuole.\\
		Parlare certamente glie-lo    vuole \\
	     talk     certainly  to.him/her-it wants\\\jambox*{(\ili{Italian})}
	\glt Intended: `Talk to him, he certainly wants to.'}\label{GSexemple18a} 	
	
	\ex[*]{ 
	\gll Hablar, le         quiere (pero           no  mucho      tiempo).\\
	     talk    to.him/her wants  \spacebr{}but   not a.long time\\\jambox*{(\ili{Spanish})}
	\glt Intended: `Talk to him/her he wants to (but not for a long time).'}\label{GSexemple18b}
\zl

We assume that restructuring verbs have two possible descriptions: as ordinary verbs taking an infinitival VP complement, or as heads of complex predicates. They are related by the Argument Attraction Lexical Rules given in (\ref{GSexemple19}) (adapted from \citealt[331]{Monachesi98a}).\footnote{We leave aside the object control and object raising verbs (verbs of influence or perception verbs) which can also be the head of a complex predicate, and hence be the target of a similar lexical rule \citep{AGMS98a, AG2010}.}  

\begin{exe}
\ex {Argument attraction lexical rules\is{lexical rule!Argument Attraction} for \ili{Romance} restructuring verbs}:\label{GSexemple19} 
\begin{xlist}
\ex
\label{GSexemple19a}Subject control verbs:\\
\avm{
[head & verb\\
 arg-st & < XP$_i$, [head & [\type*{verb}
                             vform & inf ] \\ 
                     subj & < XP$_i$ > \\
                     comps & < > ] > \+ \1 ]}
		$\mapsto$ \\\smallskip
% for some reason \flushright does not work here, but the environment does
\begin{flushright}
\avm{[arg-st & < XP$_i$, V[\type*{basic-verb}
             comps & \2 \\
             light & $+$] > \+ \2 \+ \1 ]}
\end{flushright}
%\pagebreak
%\flushleft
\ex
\label{GSexemple19b}
%\flushleft
Subject raising verbs:\\
\avm{
[head & verb \\
 arg-st & \1 \+ < [head &	[\type*{verb}
                                 vform & inf ] \\ 
		   subj & \1 \\
                   comps & < > ] > \+ \2 ]}
		$\mapsto$ \\\smallskip
\flushright\avm{
[arg-st & \1 \+ < V[\type*{basic-verb}
                    comps & \3 \\
                    light & $+$] > \+ \3 \+ \2 ]}
\end{xlist}
\end{exe}

\noindent
In the input description, the verbal complement is saturated for its complements. The verb may have other complements in addition to the saturated infinitival VP, noted as list \ibox{1} in (\mex{0}a) and \ibox{2} in (\mex{0}b). We distinguish between subject control verbs and subject raising verbs to accommodate the case where the complement verb is subjectless, but with complements that can be attracted. In (\ref{GSexemple20a}), the verb \emph{sembra} `seems' is a raising verb, and the infinitive \emph{piacere} `to please' is an impersonal verb with no subject, but with a complement, realized by \emph{gli} on the head verb \emph{sembra} (there is another interpretation where \emph{gli} is the complement of \emph{sembra}, which is irrelevant).\footnote{Alternatively, in a grammar with null pronouns, impersonal and unaccusative\is{verb!unaccusative} verbs in \ili{Romance} languages could be analyzed as having a null pronoun subject, a representation which allows a common input for subject control and raising verbs in the Argument Attraction Lexical Rule (as in \citealt[331]{Monachesi98a}).} Note that there is inter-speaker variation: \emph{sembrare} `to seem' is not a restructuring verb for all \ili{Italian} speakers (hence \% on the examples).

The category of the subject is not specified: it can be an infinitival VP as well as an NP (or even a sentence); in the first case, the index is that of the situation (\ref{GSexemple20c}), in the second, it is the index of the nominal entity (\ref{GSexemple20b}). Again, the upstairs clitic \emph{gli} corresponds to the argument of \emph{piacere} `to please':

\eal
\judgewidth{\%}
	\label{GSexemple20}
    \ex[\%]{
	\gll Gli    sembra piacere molto.\\ 
	  	 to.him seems  please  a.lot\\\jambox*{(\ili{Italian})}
	\glt `It seems that he likes it a lot.'}\label{GSexemple20a}

	\ex[\%]{ 
	\gll [Questo           regalo] gli    sembra piacere.\\ 
		 \spacebr{}this    gift    to.him seems  please\\
	\glt `This gift seems to please him.'}\label{GSexemple20b}		
		
	\ex[\%]{ 
	\gll [Andare           in vacanza]  gli    sembra piacere\\
		 \spacebr{}go.away on vacation  to.him seems  please\\
		\glt `To go away on vacation seems to please him.'}\label{GSexemple20c}	
\zl

\subsection{The different structures of complex predicates with restructuring verbs} \label{GSsection3.2}

The point of this section is to show that argument attraction is compatible with different structures: complex predicate formation and structure are two different aspects of the grammar. In \ili{Romance} languages, restructuring verbs can take a VP complement, or be the head of a complex predicate. In the latter case, there are two possible structures: the restructuring verbs enter either a flat structure or a verbal complex. We speak of a flat structure when the complement verb as well as the complements that it subcategorizes for are all sisters of the head. We speak of a verbal complex when the head verb and the complement verb form a constituent by themselves, to the exclusion of their complements (see Figure~\ref{GSfigure3}).

%%%%%%%%%%%%%%%%%%%%%%%%%%%%%%%%%%%%%%%%%%%%%%%%%%%%%%%%%%%%%%%
\begin{figure}
\begin{subfigure}{.495\textwidth}
\begin{forest} 
for tree={%
    l sep=10pt}
[S
   [SN
      [Marco\\Marco\\Marco\\Marco, align=center, base=bottom, tier=word]]
   [VP
      [V[vuole\\wants\\quiere\\wants, align=center, base=bottom, tier=word]]
      [VP[lo-dare a Maria\\it-give to Maria\\darlo a María\\give.it to María, align=center, base=bottom, tier=word, roof]]
]]
\end{forest}
\caption{VP complement}
\label{GSfigure3a}
\end{subfigure}
\hfill
\begin{subfigure}{.495\textwidth}
\begin{forest} 
for tree={%
    l sep=19pt}
[S
   [NP
      [Marco\\Marco, align=center, base=bottom, tier=word]]
   [VP
      [V[lo-vuole\\it-wants, align=center, base=bottom, tier=word]]
      [V[dare\\give, align=center, base=bottom, tier=word]]
      [PP[a Maria\\to Maria, align=center, base=bottom, tier=word, roof]]
      ]]
\end{forest}
\caption{Flat structure}
\label{GSfigure3b}
\end{subfigure}
\\
\vspace{20pt}

\begin{subfigure}{.5\textwidth}
\centering
\begin{forest} 
sm edges
[S
   [NP
      [Marco;Marco]]
   [VP
      [V[V [lo-quiere;it-wants]] [V[dar;give]]]
      [PP[a María;to María, roof]]
]]]
\end{forest}
\caption{Verbal complex}
\label{GSfigure3c}
\end{subfigure}
\caption{Three constituent structures for Romance restructuring verbs}
\label{GSfigure3}
\end{figure}
%%%%%%%%%%%%%%%%%%%%%%%%%%%%%%%%%%%%%%%%%%%%%%%%%%%%%%%%%%%%%%%

We contrast \ili{Italian} and \ili{Spanish}.\footnote{In \ili{Portuguese}, restructuring verb
  constructions are also a flat structure, but with different ordering constraints than
  \ili{Italian}; the variety of \ili{Spanish} not described here is similar to
  \ili{Portuguese}. Except for the copula (see Section~\ref{GSsection3.4}), complex predicate
  constructions with head verbs entering only one structure also distribute between these two
  structures among \ili{Romance} languages: tense auxiliaries in \ili{French}, \ili{Italian} and
  \ili{Portuguese}, as well as \ili{Romanian} modal \emph{a putea} `can', are the head of a flat
  structure, while tense auxiliaries in the variety of \ili{Spanish} described here and in
  \ili{Romanian} enter a verbal complex \citep{AG2010}.} Note that in \ili{Spanish}, there is
variation among speakers: we describe here one usage of \ili{Spanish} complex predicates.  

%\largerpage[2]
The impossibility of preposing illustrated in (\ref{GSexemple18}) for both languages shows that the
sequence of the complement verb and its complements does not form a constituent (a VP) when there is
a complex predicate, a point made by \citet{rizzi1982issues} for \ili{Italian}, on the basis of a
series of constructions (pied-piping, clefting, Right Node Raising, Complex NP shift). However, the
two languages differ with respect to other properties. In what follows, the fact that there is a
complex predicate is indicated by the presence of a clitic on the head verb.
 
%\largerpage[2]
First, adverbs occur between the restructuring verb and the infinitive in \ili{Italian}
(\ref{GSexemple21a}), but not in \ili{Spanish} (\ref{GSexemple21b}) (though a few adverbs, such as
\emph{casi} `nearly', \emph{ya} `already' and \emph{apenas} `barely' are possible). In
\ili{Spanish}, an adverb may occur after the verb and before the infinitive if the complement is a
VP (\ref{GSexemple21c}) (examples in (\ref{GSexemple21}) from \citealt[139]{AG2010}).

\eal
	\label{GSexemple21} 
	\ex[]{
	\gll Giovanni \emph{lo} vuole spesso leggere.\\ 
	     Giovanni it        wants often  read\\\jambox*{(\ili{Italian})}
	\glt `Giovanni wants to read it often.'}\label{GSexemple21a}

	\ex[*]{ 
	\gll Juan \emph{lo} quiere {a menudo} leer.\\ 
	     Juan it        wants  often      read\\\jambox*{(\ili{Spanish})}
	\glt Intended: `Juan wants to read it often.'}\label{GSexemple21b}		
	
	\ex[]{ 
	\glll Juan quiere {a menudo} leer\emph{lo}.\\
		Juan quiere {a menudo} leer-lo \\
	     Juan wants  often      read.it\\
	\glt `Juan wants to read it often.'}\label{GSexemple21c}	
\zl

Second, an inverted subject NP can occur between the two verbs of a complex predicate in \ili{Italian} (\ref{GSexemple22a}), but not in \ili{Spanish} (\ref{GSexemple22b}). The subject can occur postverbally in interrogative sentences. In \ili{Italian}, it can occur between the two verbs with a special prosody, indicated by the small capitals in (\ref{GSexemple22a}), and with inter-speaker variation \citep{salvi1980ausiliari}. In \ili{Spanish}, this is not possible (except for the pronominal subject; \citealt{suner1982syntax}).

\eal
\judgewidth{\%}
	\label{GSexemple22} 
	\ex[\%]{
        \longexampleandlanguage{
	\gll Lo comincia \textsc{Maria} a  capire,     il problema, oppure no?\\ 
	     it begins   Maria          to understand  the problem  or     no\\}{Italian}
	\glt `Maria, she's beginning to understand it, the problem, yes or no?'}\label{GSexemple22a}

	\ex[*]{ 
	\gll ?`Lo         comienza Juan a  comprender?\\
		 \spacebr{}it begins   Juan to understand\\\jambox*{(\ili{Spanish})}
	\glt `Is Juan beginning to understand it?'}\label{GSexemple22b}		
	
	\ex[]{ 
	\glll ?`Comienza       Juan a  comprenderlo?\\
	?`Comienza       Juan a  comprender-lo? \\
		 \spacebr{}begins Juan to understand.it\\
	\glt `Is Juan beginning to understand it?'}\label{GSexemple22c}	
\zl

Finally, \ili{Italian} heads of complex predicates can have scope over the coordination of infinitives with their complements (\ref{GSexemple23a}), while this is not the case in \ili{Spanish} (\ref{GSexemple23b}). Again, the presence of a clitic on the head verb (\emph{lo vuole} lit.\ it wants, \emph{le volvi\'o} lit.\ to.him started.again) shows that this is a complex predicate construction (examples from \citealt[136--137]{AG2010}).

%\pagebreak
\eal
\judgewidth{\%}
\label{GSexemple23}%
\ex[\%]{
\longexampleandlanguage{
\gll Giovanni lo vuole comprare subito      e   dare a  Maria.\\ 
     Giovanni it wants buy      immediately and give to Maria\\}{Italian}
\glt `Giovanni wants to buy it immediately and give it to Maria.'}\label{GSexemple23a}

\ex[*]{ 
\longexampleandlanguage{
\gll Le         volvi\'o      a  pedir un aut\'ografo y   a  hacer proposiciones.\\
 to.him/her started.again to ask   an autograph   and to make  proposals\\}{Spanish}
\glt Intended: `He started again to ask him for an autograph and to make proposals to him/her.'}\label{GSexemple23b}	
\zl

\noindent
Constituency tests such as preposing, as in (\ref{GSexemple18}), show that the verbal complement is not a VP in either language. The verbal complex, in which the two verbs form a constituent without the complements, is well-suited to account for the absence of adverbs and of subject NPs, if such combinations exclude elements other than verbs (adverbs in particular). This constraint can be captured by the feature [\light{}$+$], which has been used in \ili{Romance} languages for other phenomena as well (\citealt{abeille2000french}; see Section~\ref{GSsection3.3}).%
    \footnote{The adverbs admissible in the \ili{Spanish} verbal complex are light.}
Hence, complex predicate constructions in \ili{Spanish} contain a verbal complex, while they form a flat structure in \ili{Italian} containing the complement verb and its complements. 

This is illustrated with examples in Figure~\ref{GSfigure3}, which all mean `Marco wants to give it to Maria'. The verb takes a VP complement in Figure~\ref{GSfigure3a} in both languages, it is the head of a flat VP in \ili{Italian} in Figure~\ref{GSfigure3b}, and it enters a verbal V-V complex in \ili{Spanish} in Figure~\ref{GSfigure3c} (from \citealt[146]{AG2010}).

%\largerpage
The possibility of the coordination in (\ref{GSexemple23a}) has been viewed as an argument in favor
of a complement VP, even when there is argument attraction \citep{andrews1999complex}. The data go
against such an analysis for \ili{Spanish}, since the coordination is not acceptable. For \ili{Italian},
although such sequences as (\ref{GSexemple23a}) can be analyzed as instances of coordinations of VP, they can
also be instances of Non-Constituent Coordinations (NCCs; an \ili{English} example would be \emph{John gives a book to Maria and discs to her brother}; see \crossrefchapteralt[Section~\ref{sec-non-constituent-coordination}]{coordination}). 
So, the question becomes: why is (\ref{GSexemple23b}) not an acceptable NCC in \ili{Spanish}?
\citet{AG2010} propose that NCCs are subject to a general constraint in \ili{Romance}
languages: the parallel elements of the coordination must be at the same syntactic level, otherwise
the acceptability is degraded. An example is the contrast between (\ref{GSexemple24a}) and
(\ref{GSexemple24b}) in \ili{Spanish}. The structure of (\ref{GSexemple23b}), repeated in
(\ref{GSexemple24c}), is similar to that of (\ref{GSexemple24b}), if it is a verbal complex ((\ref{GSexemple24}) from \citealt[137, 144]{AG2010}).

\eal
	\judgewidth{??}
	\label{GSexemple24} 
	\ex[]{
        \longexampleandlanguage{
	\gll Juan da    [el           libro de Proust] [a           Mar\'ia] y   [el          (libro)            de Camus] [a           Pablo].\\
		 Juan gives \spacebr{}the book  of Proust  \spacebr{}to Mar\'ia  and \spacebr{}the \spacebr{}book    of Camus  \spacebr{}to Pablo\\}{Spanish}
	\glt `Juan gives the book by Proust to Mar\'ia and the book by Camus to Pablo.'}\label{GSexemple24a}
	
	\ex[??]{ 
	\gll Juan da    [el           libro de Proust] [a           Mar\'ia] y   [de          Camus] [a Pablo].\\
		 Juan gives \spacebr{}the book  of Proust  \spacebr{}to Mar\'ia  and \spacebr{}of Camus  \spacebr{}to Pablo\\
	\glt Intended: `Juan gives the book  by Proust to Mar\'ia and the book by Camus to Pablo.'}\label{GSexemple24b}
	
	\ex[*]{ 
	\gll [Le                  volvi\'o      a  pedir] [un          aut\'ografo] y   [a           hacer]         [proposiciones].\\
    	 \spacebr{}to.him/her started.again to ask    \spacebr{}an autograph    and \spacebr{}to make \spacebr{}proposals\\
	\glt Intended: `He started again to ask him an autograph and to make proposals to him/her.'}\label{GSexemple24c}
\zl

In (\ref{GSexemple24a}), the NP \emph{el de Camus} `the one by Camus' is parallel to and at the same level as \emph{el libro de Proust} `the book by Proust', the PP \emph{a Pablo} `to Pablo' is parallel to and at the same level as \emph{a Mar\'ia} `to Mar\'ia', and the NP and the PP are both complements of \emph{da} `gives'. But, in (\ref{GSexemple24b}), \emph{de Camus} `by Camus' is parallel to \emph{de Proust} `by Proust', and not at the same level as \emph{el libro de Proust} or as \emph{a Pablo}: \emph{a Pablo} corresponds to the complement of \emph{da} `gives' while \emph{de Camus} corresponds to the complement of the noun \emph{libro} `book'. Thus, the acceptability is degraded. 

If the structure of a complex predicate is that of a verbal complex in \ili{Spanish}, the structure of (\ref{GSexemple24c}) is similar to that of (\ref{GSexemple24b}): \emph{a hacer} corresponds to a \emph{a pedir}, which is the complement V of \emph{volvi\'o} in a V-V constituent, and is not at the same level as \emph{proposiciones}, which corresponds to \emph{un aut\'ografo}, which is outside the V-V constituent.   

\subsection{Analysis of Romance restructuring verb constructions in HPSG} \label{GSsection3.3}
\label{sec-romance-complex-predicates}

It has been shown in Section~\ref{GSsection3.1} that the different \ili{Romance} languages all have
complex predicate constructions, and, in Section~\ref{GSsection3.2}, that, although they share some
properties (such as clitic climbing and occurrence in other bounded dependencies), they also show syntactic differences amongst themselves (separability of the head and the infinitive or participle in \ili{Italian}, but not in \ili{Spanish}, and the possibility of coordination of the complement verb with its complements in \ili{Italian}, but not in \ili{Spanish}). The flexibility of HPSG grammars allows us to describe both the commonalities and the differences. The common behavior follows from the fact that they share the mechanism of argument attraction, which characterizes certain classes of verbs; the differences follow from a different phrase structure: the restructuring verb enters a flat structure in \ili{Italian} (Figure~\ref{GSfigure3b}), while it enters a verbal complex in \ili{Spanish} (Figure~\ref{GSfigure3c}). This analysis contrasts with that of \citet{andrews1999complex} in LFG, who propose that complex predicates in \ili{Romance} languages arise when two verbs have a common domain of grammatical functions, but correspond to just one phrase structure, all these verbs taking a VP complement. It is not clear how they can account for the differences between the two languages.

Two ID schemata combining a head with its complements account for the distinction between the flat structure and the verbal complex: the usual head-complements phrase, and a different one, the head-cluster phrase, which is also used in \ili{German} (see Section~\ref{GSsection4.1.2}).

The \type{head-complements-phrase}\is{schema!Head-Complements} is defined as follows:

\vbox{
\ea
\label{GSexemple25}
% \type{head-complements-phrase} \impl \\	
% \avm{
% [synsem
% 		[loc|cat & [head  & \1 \\
% 		 comps & \3] \\
% 		 light $-$]\\
%  head-dtr|synsem|loc|cat [head  & \1 \\
% 		          comps & \2 \shuffle \3]\\
% 		  non-head-dtrs \rel{synsems2signs}(\2) \type{ne-list}\\
% 		]}
% \itdobl{This is not correct, since the identity of HEAD values is not a constraint on phrases of type \type{head-complements-phrase} but inherited from \type{headed-phrase}. Explain \type{ne-list}.}
\type{head-complements-phrase} \impl \\	
\avm{
[synsem [loc|cat|comps & \1 \\
	 light $-$]\\
 head-dtr|synsem|loc|cat|comps & \2 \shuffle \1\smallskip\\
 non-head-dtrs \rel{synsems2signs}(\2) \type{ne-list}]}
\z
}
\noindent
The \compsl is a list of \type{synsem} objects. It is converted into a list of signs by the
relational constraint \rel{synsems2signs}\isrel{synsems2signs} (see \citealt[\page 34]{GSag2000a-u} for a similar
proposal using \rel{synsems2signs}). \type{ne-list} stands for non-empty list and this
specification ensures that there is at least one element in the list of non-head daughters. The
phrase structure described in (\ref{GSexemple25}) is general: it allows for a flat structure as well
as binary structures, as in German (see Section~\ref{GSsection4.1.2}). The difference between the
two is that, in flat structures, the head daughter is specified as \mbox{[\light{}$+$]}, which is
not the case in binary structures.

%%%%%%%%%%%%%%%%%%%%%%%%%%%%%%%MAS LARG%%%%%%%%%%%%%%%%%%%%%%%%%%%%%%%%%%

\begin{figure}
\centerfit{%
\begin{forest}
sm edges
 [S
 [\ibox{1} \NPj
            [Marco;Marco]]
  [%
  \avm{
  	V[comps & <  > \\
  	  light & $-$ ]
  }, before computing xy={s'-=10pt}
    [%
    \avm{
    	V[comps & < \2, \3, \4 > \\
          arg-st & < \1, \2, \3, \4 > \\
    	  light & $+$ ]
	}, before computing xy={s'+=7pt}[vuole;wants]]
    [%
    \avm{
    	\2 V[vform & inf \\
             comps & < \3, \4 > \\
             arg-st & < \NPj, \3, \4 > \\
             light & $+$]
	}[dare;give]
	]
     [\ibox{3} NP
            [questo libro;this book, roof]]
     [\ibox{4} PP
            [a Maria;to Maria, roof]]]]
\end{forest}
}
\caption{Flat VP structure with an Italian restructuring verb}
    \label{GSfigure4}
\end{figure}
%
%%%%%%%%%%%%%%%%%%%%%%%%%%%%%%%MAS LARG%%%%%%%%%%%%%%%%%%%%%%%%%%%%%%%%%%

%\largerpage[-1]
%\enlargethispage{4pt}
In Romance languages, the \emph{head-complements-phrase} is usually saturated for the expected
complements, but not always: list \ibox{1} in(\ref{GSexemple25}) is usually empty, but does not have
to be (see the case of the copula in Section~\ref{GSsection3.4}). An example of the flat structure
with a restructuring verb is given in Figure~\ref{GSfigure4}.

\largerpage[2]
In the flat structure, the head verb takes as complements the infinitival verb and the canonical
complements expected by the infinitive, and combines with them. The VP, corresponding to the
\emph{head-complements-phrase}, is complement saturated. The presence of the \light feature
\citep{bonami2012phrase} renames the \textsc{weight} feature proposed in \citew{abeille2000french},
as well as the \lex feature used in \ili{German} (e.g.\ \citealt{HN89b, HN94a, Kiss95a,
  Meurers2000b, Mueller2002b, hohle2018spuren}). The \light feature has ordering as well as
structural consequences \citep{abeille2000french, AG2010}. It is appropriate both for words and
phrases. Words can be light or non-light; lexical verbs (finite verbs, participles or infinitives
without complements) are light. Most phrases are non-light; in particular, the VP, that is, the
phrase which combines with the subject in \ili{Romance} languages, is non-light.\footnote{Note that
  the head-only phrase is non-light. Hence, the VP which dominates a lexical verb only is
  non-light.} But some phrases can be light if they are composed of light constituents. Such is the
case for the \emph{head-cluster-phrase}.

The verbal complex corresponds to another kind of \emph{head-complements-phrase}, called the
\emph{head-cluster-phrase}, given in (\ref{GSexemple26}) (see \citealt[87]{Mueller2002b};
\citealt[39]{muller2018clause}).\footnote{This rule is also used in \ili{Romanian}. As in
  \ili{German}, we do not specify the category of the complement (which can be a noun in
  \ili{Spanish}, for instance). Note that Müller does not specify the \light value of the non-head
  daughter (see (\ref{ex-head-cluster-phrase-German})). This is not necessary since the auxiliaries select for the non-head daughter and hence
  they can determine the \light value. This is important since some auxiliaries do not require
  their arguments to be lexical. For example in so called auxiliary flip constructions, the verbal
  complex may contain non-verbal material. See \citew[Section~1.4]{HN94a}. The \light value of the
head daughter and the \light value of the mother is not specified either in grammars of German. } 

\ea
\is{schema!Head-Cluster}%
\type{head-cluster-phrase} \impl \\
\avm{
			[synsem &
				[loc|cat|comps & \1\\
				light $+$]\\
		  	head-dtr|synsem &
				[loc|cat &	[head & verb \\
		            					comps & \1 \+ < \2 >]\\
				 light $+$] \\
		  	non-head-dtrs & <[synsem & \2 [light & $+$] ]> ] }\label{GSexemple26}
\z

\noindent
This differs from the usual \emph{head-complements-phrase} on two accounts: there is only one daughter, and both constituents are [\light{}$+$]. 

The \emph{head-cluster-phrase} is illustrated in Figure~\ref{GSfigure5}: the phrase \emph{quiere dar} corresponds to the \emph{head-cluster-phrase} in (\ref{GSexemple26}), while the whole VP (\emph{quiere dar aquel libro a Mar\'ia} `wants to give that book to Mar\'ia') corresponds to the usual \emph{head-complements-phrase} in (\ref{GSexemple25}).

%%%%%%%%%%%%%%%%%%%%%%%%%%%%%%%MAS LARG%%%%%%%%%%%%%%%%%%%%%%%%%%%%%%%%%%

\begin{figure}
    \centering
\oneline{%
\begin{forest}
sm edges
 [S
 [{\ibox{1} \NPj}
            [Marco;Marco]]
  [%
  \avm{
  	VP[comps & < > \\
  	light & $-$ ]
  }
	[%
	\avm{
		V[comps & < \3, \4 > \\
		light & $+$ ]
	}, before computing xy={s'-=10pt} 
	    [%
	    \avm{
	    	V[comps & < \2, \3, \4 > \\
	    	arg-st & < \1, \2, \3, \4 > \\
	    	light & $+$ ]
	    }, before computing xy={s'+=4pt} [quiere;wants]]
		[%
		\avm{
			\2 V[comps & < \3, \4 > \\
			arg-st & < \NPj, \3, \4 > \\
			light & $+$ ]
		}, before computing xy={s'-=4pt} [dar;give]]]
	[\ibox{3} NP
		[aquel libro;that book, roof]]
	[\ibox{4} PP
		[a María; to María, roof]]]]
\end{forest}}
\caption{VP with a verbal complex with a Spanish restructuring verb}
    \label{GSfigure5}
%\itdopt{renumber}
\end{figure}

%%%%%%%%%%%%%%%%%%%%%%%%%%%%%%%MAS LARG%%%%%%%%%%%%%%%%%%%%%%%%%%%%%%%%%%

\largerpage[2]
Regarding the canonical complements in the verbal complex construction, the requirement is passed up
by the verbal complex, according to the description in (\ref{GSexemple26}) (the list \ibox{1} is
non-empty). The verbal complex itself combines with the canonical complements expected by the
infinitive (here, \ibox{3} and \ibox{4}).

More has to be said regarding the clitic \emph{lo} in the \ili{Italian} sentence \emph{Marco lo-vuole dare a Maria} `wants to give it to Maria' and \ili{Spanish} sentence \emph{Marco lo-quiere dar a Mar\'ia} `wants to give it to Mar\'ia') in Figure~\ref{GSfigure3}. The infinitive is a basic verb: there is no difference between the complements and the arguments (except for the subject); its complement list contains an affixal element (see Section~\ref{GSsection2}). Following the rule in (\ref{GSexemple19a}), this element is attracted to the argument list of the head verb, but it is not realized as a complement; the head verb is then a reduced verb (see Figure~\ref{GSfigure6}), which is the target of a morphological rule of cliticization, hence the clitic \emph{lo} `it' on the head verb \emph{vuole} or \emph{quiere} `wants'. 


%%%%%%%%%%%%%%%%%%%%%%%%%%%%%%%MAS LARG%%%%%%%%%%%%%%%%%%%%%%%%%%%%%%%%%%

\begin{figure}
\begin{subfigure}[b]{\textwidth}
\centering
\begin{forest}
sm edges
  [%
  \avm{VP[comps & < >]} 
    [%
    \avm{
    	V[\type*{reduced verb}
    	comps & < \2, \4 > \\
    	arg-st & < \NPj, \2, \3, \4 > \\
    	light & $+$ ]
    }[lo-vuole;it-wants]]
    [%
    \avm{
    	\2 V[\type*{basic verb} \\
    	comps & < \3, \4 > \\
    	arg-st & < \NPj, \3 \type{aff}, \4 > \\
    	light & $+$ ]
    }[dare;give]]
     [\ibox{4} PP
            [a Maria;to Maria, roof]]]
\end{forest}
\caption{Italian clitic climbing}
%\itdopt{\ibox{1} is missing}
\label{GSfigure6a}
\end{subfigure}
\\
\vspace{20pt}
\begin{subfigure}[b]{\textwidth}
\centering
\begin{forest}
sm edges
  [%
  \avm{VP[comps & < >]} 
  [%
  \avm{
  	V[comps & \4 \\
  	light & $+$ ]
  }
    [%
    \avm{
    	V[\type*{reduced verb} \\
    	comps & < \2, \4 > \\
    	arg-st & < \NPj, \2, \3, \4 > \\
    	light & $+$ ]
    }[lo-quiere;it-wants]]
    [%
    \avm{
    	\2 V[\type*{basic verb} \\
    	comps & < \3, \4 > \\
    	arg-st & < \NPj, \3 \type{aff}, \4 > \\
    	light & $+$ ]
    }[dar;give]]]
     [\ibox{4} PP, before computing xy={s'+=15pt}
            [a María;to María, roof]]]
\end{forest}
\caption{Spanish clitic climbing}
%\itdopt{\ibox{1} is missing}
\label{GSfigure6b}
\end{subfigure}
\caption{Clitic climbing with Italian and Spanish restructuring verbs}
\label{GSfigure6}
\end{figure}

It remains to ensure that \ili{Spanish} restructuring verbs are characterized by a verbal complex, and \ili{Italian} ones by a flat structure. In fact, nothing more has to be said for \ili{Italian}, since this language lacks the \emph{head-cluster-phrase}. We assume an additional constraint on phrases in \ili{Spanish}. According to (\ref{GSexemple27}), if the phrase is light, it follows that the non-head daughters are also light, and, conversely, if the phrase is non-light, the non-head daughters are non-light.

% \begin{exe}
%         \ex{	
%         \begin{avm}
% 		\[\normalfont{\emph{phrase}}\\
% 		light \ibox{1}\] \, \impl \[non-head-dtrs | light \ibox{1}\]
% 	\end{avm}\jambox*{(for \ili{Spanish})}\label{GSexemple27} }
% \end{exe}
\ea	
\label{GSexemple27}
\type{phrase} \impl\\
\avm{
[light \1\\
 non-head-dtrs \listOf{ [light \1] } \\ ] % todo avm
}\jambox*{(used in \ili{Spanish})}
\z
The structure of the flat VP does not obey this constraint: the infinitival verb which is a non-head
daughter is light, while the other complements are non-light (see Figure~\ref{GSfigure4}). When
constraint (\ref{GSexemple27}) applies, the head of a restructuring verb cannot enter a flat
structure.  


\largerpage[-1]
\ili{Romance} languages follow the general constraints on ordering in non-head-final
languages. According to constraint (\ref{GSexemple29}), the verb precedes the complements it
subcategorizes for. This is relevant not only for the head of the complex predicate, but also for
the participle complement of the tense auxiliary or the infinitive complement of a restructuring
verb. Although the latter do not combine with their expected complements, they still subcategorize
for them.\footnote{
  For more on the definition of such constraints, see \crossrefchapterw[Section~\ref{sec-id-lp}]{order}.
} 

\ea
V[\comps \sliste{ \ldots, \ibox{1}, \ldots } ] < [\synsem \ibox{1}]
	\jambox*{(head-initial languages)}\label{GSexemple29}
\z

\subsection{The complements of the copula in Romance languages}\label{GSsection3.4}\label{cp:sec-copula-romance}

It is an interesting fact that, while \ili{Romance} restructuring verbs enter two different structures (the flat structure and the verbal complex), the copula has the same complement structure across \ili{Romance} languages \citep{abeille2001varieties, AG2010}.\footnote{We concentrate on the predicative use of the copula.} Moreover, this complementation differs both from the flat structure and the verbal complex: the copula takes a non-light complement, which can be saturated or not. 

The complement of the copula is underspecified: it is predicative (encoded by [\prd +]), but it can be an adjective, a noun, a preposition or a passive participle (for the passive construction, see \citealt{AG2002b-u}). We illustrate clitic climbing with the same example in different \ili{Romance} languages (examples from \citealt[120]{AG2010}).

\eal
	\label{GSexemple30} 
	\ex{
	\gll Jean lui        \'etait fid\`ele.\\ 
		 Jean to.him/her was     faithful\\\jambox*{(\ili{French})}
	\glt `Jean was faithful to him/her.'}\label{GSexemple30a}

	\ex{ 
	\gll Giovanni le     era     fedele.\\
		 Giovanni to.her was     faithful\\\jambox*{(\ili{Italian})} 
	\glt `Giovanni was faithful to her.'}\label{GSexemple30b}
		
	\ex{ 
	\gll Juan le         era     fiel.\\
		 Juan to.him/her was     faithful\\\jambox*{(\ili{Spanish})}
	\glt `Juan was faithful to him/her.'}\label{GSexemple30c}
		
	\ex{ 
	\gll O Jo\~an era-lhe        fiel.\\ 
		 \textsc{det} Jo\~an was-to.him/her faithful\\\jambox*{(\ili{Portuguese})}
	\glt `Jo\~an was faithful to him/her.'}\label{GSexemple30d}
		
	\ex{ 
	\gll En Joan li      era fidel.\\ 
		 \textsc{det} Joan to.him/her was faithful\\\jambox*{(\ili{Catalan})}
	\glt `Joan was faithful to him/her.'}\label{GSexemple30e}
		
	\ex{ 
	\gll Ion \^ii       era credincios.\\
		 Ion to.him/her was faithful\\\jambox*{(\ili{Romanian})}
	\glt `Ion was faithful to him/her.'}\label{GSexemple30f}
\zl

\noindent
The properties of the construction differentiate it clearly from tense auxiliaries and restructuring verbs. For the sake of simplicity, we restrict the examples to \ili{French}, \ili{Italian} and \ili{Spanish}. The sequence of the head of the complement with its complements is a constituent, since, for instance, it can be dislocated and pronominalized (\ref{GSexemple31}) (examples in (\ref{GSexemple31}) and (\ref{GSexemple32}) from \citealt[133-134]{AG2010}).

\begin{exe}
	\ex{
[Context] Is John faithful to his friends?}\label{GSexemple31} 
	\begin{xlist}
        \ex[]{
        \longexampleandlanguage{
		\gll Fid\`ele \`a ses amis,   il l'est plus qu'\`a    ses convictions politiques.\\
		     faithful to  his friends he {it is} more {than to} his convictions political\\}{French}
		\glt `Faithful to his friends, he is, more than to his political ideas.'}\label{GSexemple31a} 
		
        \ex[?]{
		\gll {Fedele} ai     suoi amici,  (lo)         \`e pi\`u che  alle   sue idee  politiche.\\
		     faithful to.the his  friends \spacebr{}it is  more  than to.the his ideas political\\\jambox*{(\ili{Italian})}
		\glt `Faithful to his friends, he is, more than to his political ideas.'}\label{GSexemple31b} 
		
	\ex[]{
		\gll Fiel     a  sus amigos,  lo es m\'as que  a  sus convicciones pol\'iticas.\\
		     faithful to his friends  it is more  than to his convictions  political\\\jambox*{(\ili{Spanish})}
		\glt `Faithful to his friends, he is, more than to his political ideas.'}\label{GSexemple31c} 
	\end{xlist}
\end{exe}

Crucially, the construction differs from that of restructuring verbs in that the dislocated constituent can leave behind its complements (\ref{GSexemple32}).

\eal 
	\label{GSexemple32}
	\ex[]{
        \longexampleandlanguage{
	\gll Fid\`ele, il l'est plus \`a ses amis    qu'\`a  ses convictions politiques.\\
	     faithful  he {it is} more to  his friends than.to his convictions political\\}{French}
	\glt `As for being faithful, he is to his friends more than to his political convictions.’}\label{GSexemple32a} 
		
      \ex[]{
        \longexampleandlanguage{
	\gll Fedele,  lo \`e ai     sui amici   pi\`u che  alle   sue idee politiche.\\
	     faithful it is  to.the his friends more  than to.the his ideas political\\}{Italian}
	\glt `As for being faithful, he is to his friends more than to his political convictions.’}\label{GSexemple32b} 
		
	\ex[]{
        \longexampleandlanguage{
	\gll Fiel,    lo es m\'as a  sus amigos  que  a  sus convicciones pol\'iticas.\\
	     faithful it is more  to his friends than to his convictions  political\\}{Spanish}
	\glt `As for being faithful, he is to his friends more than to his political convictions.’}\label{GSexemple32c} 
\zl

Similarly, the predicative complement can be extracted with its complements or it can leave them
behind. Even if the complements are left behind, the predicate complement can be cliticized, as shown in (\ref{GSexemple33c}) (compare with examples (\ref{GSexemple17}) and (\ref{GSexemple18}) with restructuring verbs). In (\ref{GSexemple33}), the adjective is extracted (it corresponds to the predicative complement of \emph{\^etre} `to be') as part of a concessive adjunct (examples (\ref{GSexemple33}) and (\ref{GSexemple34}) from \citealt[146, 148]{AG2010}).


\begin{exe}
	\ex{
[Context] Is he really faithful to his friends?}\label{GSexemple33} 
	\begin{xlist}
    \ex[]{
        \longexampleandlanguage{
	\gll Aussi fid\`ele \`a ses amis    qu'il   soit, il ne          perd  pas de vue   ses int\'er\^ets.\\
		 as    faithful to  his friends {as he} is    he \textsc{ne} lose  not of sight his interests\\}{French}
	\glt `As faithful to his friends as he is, he does not lose sight of his interests.'}\label{GSexemple33a} 
		
    \ex[]{
	\gll Aussi fid\`ele qu'il   soit \`a ses amis,   il ne          perd  pas de vue   ses int\'er\^ets.\\
		 as    faithful {as he} is   to  his friends he \textsc{ne} lose  not of sight his interests\\
	\glt `As faithful as he is to his friends, he does not lose sight of his interests.'}\label{GSexemple33b} 
		
	\ex[]{
	\gll Aussi fid\`ele  qu'il  leur    soit, il ne          perd  pas de vue   ses int\'er\^ets.\\
		 as    faithful {as he} to.them is    he \textsc{ne} lose  not of sight his interests\\
	\glt `As faithful to them as he is, he does not lose sight of his interests.'}\label{GSexemple33c} 
	\end{xlist}
\end{exe}

Moreover, an adverb may intervene between the copula and the adjective, not only in \ili{French} or \ili{Italian}, where it is expected (it is possible with tense auxiliaries and restructuring verbs), but also in \ili{Spanish}, where it is not expected, if the structure is the same as with restructuring verbs. We illustrate this possibility with cliticization, in order to make the contrast with restructuring verbs clearer.

\eal
	\label{GSexemple34} 
	\ex[]{
	\gll Rom\'eo lui        sera    probablement fid\`ele.\\
	 	 Rom\'eo   to.him/her will.be probably     faithful\\\jambox*{(\ili{French})}
	\glt `Rom\'eo will probably be faithful to him/her.'}\label{GSexemple34a}

	\ex[]{ 
	\gll Romeo le     sar\`a  probabilmente fedele.\\
		 Romeo to.her will.be probably      faithful\\\jambox*{(\ili{Italian})}
	\glt `Romeo will probably be faithful to her.'}\label{GSexemple34b}
		
	\ex[]{ 
	\gll Romeo le 		  ser\'a  probablemente fiel.\\
		 Romeo to.him/her will.be probably      faithful\\\jambox*{(\ili{Spanish})}
	\glt `Romeo will probably be faithful to him/her.'}\label{GSexemple34c}
\zl

The data show that, contrary to restructuring verbs, the copula in \ili{Romance} languages has only
one complement structure. \citet{AG2002b-u,AG2010} propose that the copula takes a ``phrasal''
complement, which can be saturated or not. This analysis is implemented by saying that the
predicative complement is underspecified with respect to complement saturation or attraction, and
that it is non-light in all cases. If the predicative complement is a lexical item, a unary
branching phrase makes it [\light{}$–$] (see Figure~\ref{GSfigure8}).

\eas
Description of the copula in Romance languages: \\
\avm{
[arg-st & \1 \+ < [head & [prd +] \\
                 subj  & \1 \\
                 comps & \2 \\
                 light & $-$ ] > \+ \2 ]
        }\label{GSexemple35}
\zs

Like tense auxiliaries, the copula is a subject raising verb, hence the identical value \ibox{1} for
its subject and that of its predicative complement, which allows it to be empty. Its complement
differs from that of a tense auxiliary (\ref{GSexemple8b}) on several accounts: it is predicative,
which is not the case for tense auxiliaries, and it is non-light; in addition, it is not specified
for its category.\footnote{ The predicative complement in French can be a PP (\emph{Il est contre
    cette décision.} `He is against this decision.'). However, the complement of a preposition cannot
    be attracted or extracted, in a general way. Thus, a preposition alone can be a predicative
    complement only when its complement is unexpressed and interpreted anaphorically (\emph{Il est contre.}
    `He is against (it).').} 
Being non-light, it may have combined with its complements or some of
them, while the complement of the auxiliary is light, hence all its complements are attracted (see
Figures \ref{newGSfigure7}, \ref{GSfigure8}).\footnote{Note that the complements included in a
  predicative PP are not attracted by the copula. This is assured by a constraint on prepositions
  saying that \argst elements are of type \type{canonical}.}

\begin{figure}
    \centering
\begin{forest}
sm edges
  [%
  \avm{VP[comps & < >]}
	[%
	\avm{
		V[subj & < \1 > \\
		comps & < \2 > \\
		arg-st & < \1, \2 > ]
	}[sera;will.be]]
	[%
	\avm{
		\2 AP[head &	[prd & $+$]\\
		subj & < \1 > \\
		comps & < > \\
		light & $-$ \\]
	}[fid\`ele \`a ses amis;faithful to his friends, roof]]]
\end{forest}
    \caption{The Romance copula with a saturated complement}
    \label{newGSfigure7}
\end{figure}{}

Figure~\ref{GSfigure8} illustrates a case where the affix complement of the adjective is attracted
to the copula.
For cliticization and the notion of reduced verb, see Section~\ref{GSsection2}. 

\begin{figure}
    \centering
\begin{forest}
sm edges
  [%
  \avm{VP[comps & < >] }
	[%
	\avm{
		V[\type*{reduced-verb} \\
		subj & < \1 > \\
		comps & < \2 > \\
		arg-st & < \1, \2, \3 > ]
	}[leur-sera;to.them-will.be]]
	[%
	\avm{
        \2 AP[head &	\4  \\
	      subj & < \1 > \\
	      comps & < \3  > \\
	      light & $-$ ]}
           [\avm{
            A[head & \4 [prd & $+$]\\
	      subj & < \1 > \\
	      comps & < \3 \type{aff}\, > \\
	      light & $+$ ]}
            [fid\`ele;faithful]]]]
\end{forest}
    \caption{Clitic climbing with the Romance copula}
    \label{GSfigure8}
\end{figure}

Regarding the point made in Section~\ref{sec-copula-romance}, that argument attraction is compatible with different structures (a flat structure or a verbal complex), what the Romance copula shows is that still another structure is possible: the copula can inherit arguments from a phrasal complement.


\section{Complex predicates and word order}\label{GSsection4}

\largerpage[2]
In certain languages, a complex verb construction signals itself essentially by properties of word
order. This is the case for instance in \ili{German} \citep{HN89b, HN94a, Kiss94, Kiss95a, HN98a,
  Kathol98b, HN99d, Kathol2000a, Meurers2000b, DM2002, dKM2001a, Mueller2002b, Mueller2003a,
  MuellerCopula} and \ili{Dutch} \citep{Rentier94, BvN98a}, as well as \ili{Korean} \citep{
  Sells1991, Chung98a-u, Yoo2003, Kim2016a-u}. We concentrate on coherent constructions in
\ili{German}, and on \ili{Korean} auxiliaries.

\subsection{Verbal complexes in German}\label{GSsection4.1}

The contrast in \ili{German} between coherent and incoherent constructions is reinterpreted in terms
of complex predicate formation: coherent constructions constitute a complex predicate, as does the
copula with predicative adjectives. In coherent constructions, the two
predicates cannot be separated and form a predicate complex. 

\subsubsection{Coherent and incoherent constructions in German}\label{GSsection4.1.1}

\largerpage[2]
Among verbs with an infinitival complement, \ili{German} distinguishes between coherent and
incoherent constructions \citep{gunnar1955studien}. We speak of constructions rather than verbs,
because, although the constructions are triggered by lexical properties of verbs, many verbs can be
constructed either way. Verbs entering coherent constructions, obligatorily or optionally, belong to
different classes: they may be tense auxiliaries (where the verbal complement is an infinitive or a
participle), modals, subject and object raising verbs, subject and object control verbs, copulas,
predicative verbs, verbs entering resultative constructions, or particle verbs (see
\citealt[Chapters~2, 5 and~6]{Mueller2002b}).

Coherent and incoherent constructions differ with respect to several properties (separability of the
head verb and the infinitive, \isi{extraposition} of the infinitive with its complements, pied-piping in
relative clauses and scope of adjuncts). In incoherent constructions, an adverb such as \emph{nicht}
`not' may occur between the two verbs as in (\ref{GSexemple36a}) (from \citealt[42]{Mueller2002b}),
the infinitival phrase can be extraposed (compare (\ref{GSexemple36b}) and (\ref{GSexemple36c})),
and the infinitive may be pied-piped with its relative pronoun complement as in (\ref{GSexemple36d})
(examples from \citealt[117--118]{HN98a}).

\eal
	\label{GSexemple36} 
	\ex[]{
	\gll \ldots{} dass Karl zu schlafen nicht versucht\\ 
		 {}       that Karl to sleep    not tries \\%\jambox*{(\ili{German})}
	\glt `that Karl does not try to sleep'}\label{GSexemple36a}

    \ex[]{
	\gll \ldots{} dass Peter Maria das Auto zu kaufen überredet\\ 
	 	 {}       that Peter Maria the car  to buy    persuades \\
	\glt `that Peter persuades Maria to buy the car'}\label{GSexemple36b}

	\ex[]{ 
	\gll \ldots{} dass Peter Maria überredet, [das          Auto zu kaufen]\\
		 {}       that Peter Maria persuades    \spacebr{}the car  to buy\\
	\glt `that Peter persuades Maria to buy the car'}\label{GSexemple36c}	
		
	\ex[]{ 
	\gll Das  ist das Auto, [das   zu kaufen] er Peter  überreden wird\\
	     that is  the car   \spacebr{}which to buy    he Peter  persuade    will\\
	\glt `That is the car, which he will persuade Peter to buy.'}\label{GSexemple36d}
\zl

On the other hand, coherent constructions, of which the combination of the future auxiliary \emph{wird} `will' in (\ref{GSexemple37a}) or the raising verb \emph{scheinen} `to seem’ with an infinitival complement in (\ref{GSexemple37d}) are typical examples, do not allow for a non-verbal element between the two verbs, as shown in (\ref{GSexemple37b}), nor for extraposition of the infinitive with its complements, as shown in (\ref{GSexemple37c}) and (\ref{GSexemple37e}) (examples (\ref{GSexemple37a}), (\ref{GSexemple37c}), (\ref{GSexemple37d}) and (\ref{GSexemple37e}) from \citealt[43]{Mueller2002b}), nor for pied-piping of the infinitive in relative clauses (\ref{GSexemple37f}) and (\ref{GSexemple37g}) (examples adapted from \citealt[66]{HN99d}).\footnote{The head verb in coherent constructions is italicized.}    

\eal
	\label{GSexemple37} 
	\ex[]{
	\gll \ldots{} dass Karl das Buch lesen \emph{wird}\\ 
		 {}       that Karl the book read  will \\%\jambox*{(\ili{German})}
	\glt `that Karl will read the book'}\label{GSexemple37a}

	\ex[*]{
	\gll \ldots{} dass Karl das Buch lesen nicht wird\\ 
		 {}       that Karl the book read  not   will \\
	\glt Intended: `that Karl will not read the book'}\label{GSexemple37b}

	\ex[*]{ 
	\gll \ldots{} dass Karl wird das Buch lesen\\
		 {}       that Karl will the book read\\
	\glt Intended: `that Karl will read the book'}\label{GSexemple37c}	
		
	\ex[]{ 
	\gll \ldots{} weil Karl das Buch zu lesen \emph{scheint}\\
		 {}       because Karl the book to read seems\\
	\glt `because Karl seems to read the book'}\label{GSexemple37d}	
		
	\ex[*]{ 
	\gll \ldots{} weil Karl scheint das Buch zu lesen\\
		 {}       because Karl seems   the book to read\\
	\glt Intended: `because Karl seems to read the book'}\label{GSexemple37e}
		
	\ex[*]{ 
	\gll Das  ist das Buch das  lesen Karl wird.\\
		 this is  the book that read  Karl will\\
	\glt Intended: `This is the book that Karl will read.'}\label{GSexemple37f}
		
	\ex[*]{ 
	\gll Das  ist das Buch das  zu lesen Karl scheint.\\
		 this is  the book that to read  Karl seems\\
	\glt Intended: `This is the book that Karl seems to read.'}\label{GSexemple37g}
\zl

Scrambling of the complements of the two verbs, or of the subject of the head verb with the complements of the infinitival, is possible in a coherent construction. In (\ref{GSexemple38a}) the complements of \emph{sehen} `see' (\emph{Peter}) and of \emph{kaufen} `buy' (\emph{das Auto} `the car') are not interleaved. In (\ref{GSexemple38b}), \emph{Peter}, the complement of \emph{sehen}, occurs between \emph{das Auto}, which is the complement of \emph{kaufen}, and \emph{kaufen} (example (\ref{GSexemple38b}) from \citealt[117]{HN98a}).

\eal
	\label{GSexemple38} 
	\ex[]{
	\gll \ldots{} dass er Peter das Auto kaufen \emph{sehen} \emph{wird}\\ 
		 {}       that       he Peter the car  buy    see            will \\%\jambox*{(\ili{German})}
	\glt `that he will see Peter buy the car'}\label{GSexemple38a}

	\ex[]{ 
	\gll \ldots{} dass er das Auto Peter kaufen \emph{sehen}  \emph{wird}\\
		 {}       that       he the car  Peter buy    see             will\\
	\glt `that he will see Peter buy the car'}\label{GSexemple38b}
\zl

In the complex predicate approach of this chapter, these data point to the following analysis: incoherent constructions involve a saturated VP complement, while coherent constructions do not; rather, they involve a complex predicate, with a verb attracting the complements of its complement. We assume here a verbal complex for the complex predicate. Figure~\ref{GSfigure9a} represents example (\ref{GSexemple36b}), and Figure~\ref{GSfigure9b} represents example (\ref{GSexemple38b}).

\begin{figure}
\begin{subfigure}[b]{\textwidth}
\centering
\begin{forest}
sm edges
[S
   [NP [Peter;Peter]]
   [V\rlap{$'$}
      [NP [Maria;Maria]]
      [V\rlap{$'$} 
        [VP [das Auto zu kaufen;the car to buy, roof] ]
        [V [überredet;persuades]]]]]
\end{forest}
\caption{Incoherent construction (embedded clause)}
\label{GSfigure9a}
\end{subfigure}
\\
\vspace{20pt}
\begin{subfigure}[b]{\textwidth}
\centering
\begin{forest}
sm edges
[S
   [NP [er;he]]
   [V\rlap{$'$} 
     [NP [das Auto;the car, roof]]
     [V\rlap{$'$} 
       [NP [Peter;Peter] ]
       [V
         [V [V [kaufen;buy]]
            [V [sehen;see]]] 
         [V [wird;will]]]]]]
 \end{forest}
\caption{Coherent construction (embedded clause)}
\label{GSfigure9b}
\end{subfigure}
\caption{Incoherent and coherent constructions in German}
\label{GSfigure9}
\end{figure}

\subsubsection{Coherent constructions in HPSG}\label{GSsection4.1.2}

One might wonder whether it is possible to analyze the data in terms of word order instead of
structure: a verb governing a coherent construction would trigger a modification of the ordering
domain. More precisely, it would induce domain union of the two ordering domains associated with the
two verbal projections (see \crossrefchapteralt[Section~\ref{sec-domains}]{order} for a discussion of order domains). Usually, the domain in which constituents are ordered is identical with the phrase or the sentence which dominates them. In the linearization approach \citep{Reape94a}, dominance and ordering can be distinguished. In certain circumstances, the domain for ordering is larger than the domain of constituency, so that the elements belonging to different phrases can be reordered and interleaved, a phenomenon called domain union. Domain union could be responsible for the order in (\ref{GSexemple38b}): the structure would be the same as in incoherent constructions (see Figure~\ref{GSfigure9a}), but the ordering domain would be the whole sentence.

The existence of the remote (or long) passive goes against such an analysis \parencites[140--144]{HN94a}[Section~5.2]{Kathol98b}[94, 136--138, 154--157]{Mueller2002b}. A complex predicate construction can be passivized in such a way that the subject (in the nominative case) of the passive auxiliary corresponds to the object of the active infinitive complement. An (impersonal) passive construction like (\ref{GSexemple39a}) with an infinitival VP containing an accusative object (\emph{den Wagen} `the car') alternates with a coherent construction such as (\ref{GSexemple39b}), with a corresponding nominative (examples (\ref{GSexemple39a}) and (\ref{GSexemple39b}) from \citealt[137]{Mueller2002b}, (\ref{GSexemple39c}) and (\ref{GSexemple39d}) from \citealt[40]{Mueller2003a}). 

\eal
	\label{GSexemple39} 
	\ex[]{
	\gll \ldots{} weil       oft   versucht wurde, [den          Wagen zu reparieren]\\ 
		 {}       because    often tried    was    \spacebr{}the car   to repair\\
	\glt `because many attempts were made to repair the car'}\label{GSexemple39a}

	\ex[]{
	\gll \ldots{} weil       der Wagen oft   zu reparieren \emph{versucht} \emph{wurde}\\ 
		 {}       because    the car   often to repair     tried             was\\
	\glt `because many attempts were made to repair the car'}\label{GSexemple39b}

	\ex[]{ 
	\gll Karl darf       nicht versuchen zu schlafen.\\
		 Karl is.allowed not   try       to sleep\\
	\glt `Karl is not allowed to try to sleep.'\\
	\glt `Karl is allowed to not try to sleep.'}\label{GSexemple39c}	
		
	\ex[]{ 
	\gll Karl darf       versuchen, nicht zu schlafen.\\
		 Karl is.allowed try        not   to sleep\\
	\glt `Karl is allowed to try not to sleep.’}\label{GSexemple39d}
\zl

\noindent
In (\ref{GSexemple39a}), the infinitival VP is extraposed. In (\ref{GSexemple39b}), there is no
infinitival VP, as shown by the position of the adverb \emph{oft} `often', which occurs before
\emph{zu reparieren} `to repair', while modifying \emph{versucht} `tried'. In a coherence field, an
adverb can scope over any of the verbs that belong to it.\footnote{%
A \isi{coherence field} consists of all verbs entering a coherent construction and all arguments and
adjuncts depending on the involved verbs.
} In (\ref{GSexemple39c}), \emph{zu schlafen} `to sleep' is not part of the coherent construction,
because it is extraposed; \emph{nicht} `not' can have scope over \emph{darf} `is allowed' or
\emph{versuchen} `to try', not over \emph{schlafen} `to sleep'. In (\ref{GSexemple39d}),
\emph{nicht} belongs to the extraposed infinitival; accordingly, it can only scope over that. The
fact that \emph{oft} can scope over \emph{versucht} `tried' in (\ref{GSexemple39b}) shows that they belong
to the same coherence field. This means that \emph{zu reparieren} `to repair',
  \emph{versucht} `tried' and \emph{wurde} `was' form a verbal complex, in which the passive auxiliary \emph{wurde} combines with
\emph{zu reparieren versucht}. Since the passive participle \emph{versucht} `tried' attracts the
complement of \emph{reparieren} `to repair', \emph{zu reparieren versucht} behaves like a passivized transitive verb and together with the
passive auxiliary a verbal complex results that selects for a subject that corresponds to the
accusative object of \emph{zu reparieren}.

\ili{German} differs from \ili{Romance} languages in not distinguishing structurally between the
subject and the complements of finite verbs \citep{Pollard90a-Eng}: the subject of finite
  verbs is considered as a complement, and is introduced by the same rule. The structure of the sentence is usually represented as having binary branching daughters (see Figure~\ref{GSfigure9}). The constraint is as follows (\citealt[21]{muller2018clause}).\footnote{The description in (\ref{GSexemple40}) differs minimally from that in (\ref{GSexemple25}). Following (\ref{GSexemple40}), the complements are discharged one at a time from the complements list (binary structure), while (\ref{GSexemple25}) allows for several complements at the same level as well as a binary	structure. Thus (\ref{GSexemple40}) is a more constrained version of (\ref{GSexemple25}). Similarly, the description needed for representing flat VPs in Romance languages is a subtype of (\ref{GSexemple25}), specifying the head daughter as \mbox{[\light{}$+$]}.}

\begin{exe}
\ex{\emph{head-complement-phrase}\is{schema!Head-Complement} (\ili{German}) \impl \\
\avm{
      	[synsem &	[\punk{loc|cat|comps}{ \1 \+ \2} \\
      					light & $-$ ]\\
      head-dtr|synsem &	[loc|cat|comps \1 \+ < \3 > \+ \2 ] \\
      non-head-dtrs & <[synsem \3]> ]
  	  }\label{GSexemple40}
	}
\end{exe}

\noindent
Following constraint (\ref{GSexemple40}), the head combines with one complement at a time, noted as
\ibox{3}. The presentation of the list as composed of three parts, with the relevant one in any
position, allows for a free order. The phrase combining a head with a complement is [\light{}$-$].\footnote{The feature \light is the equivalent of \lex used in \ili{German}
  studies, although the properties of light elements may differ depending on the language. It does
  not belong to \textsc{local} features in (\ref{GSexemple40}), because an extracted constituent may
  differ from its trace as regards lightness (\citealt{Mueller96a,muller2018clause}; see \crossrefchaptert{udc} for discussion of extraction).} The structure of (\ref{GSexemple41}) is exemplified in Figure~\ref{GSfigure10} \citep[22]{muller2018clause}.

\ea[]{
	\label{GSexemple41}
	\gll \ldots{} weil    das Buch jeder     kennt\\
		 {}       because the book everybody knows\\%\hfill(\ili{German})
	\glt `because everybody knows the book'}

\z

%%%%%%%%%%%%%%%%%%%%%%%%%%%%%%%MAS LARG%%%%%%%%%%%%%%%%%%%%%%%%%%%%%%%%%%

\begin{figure}
    \centering
	\begin{forest}
	sm edges
 	[CP 
    [C [weil;because]]
    [{V[\,\type{fin}, \comps \eliste, \light{}$-$]}
        [\ibox{2} NP [das Buch;the book, roof]]    
        [{V[\,\type{fin}, \comps \sliste{ \ibox{2} }, \light{}$-$]} 
            [\ibox{1} NP [jeder;everybody]]
            [{V[\,\type{fin}, \comps \sliste{ \ibox{1}, \ibox{2} }, \light{}$+$]} [kennt;knows]]]]]
\end{forest}
    \caption{Clause structure in German}
    \label{GSfigure10}
\end{figure}


%%%%%%%%%%%%%%%%%%%%%%%%%%%%%%%MAS LARG%%%%%%%%%%%%%%%%%%%%%%%%%%%%%%%%%%

Turning to complex predicates, they form a verbal complex phrase: they cannot be separated by an adverb or an NP, as shown in (\ref{GSexemple37b}) and (\ref{GSexemple37c}). Given the structure of the \ili{German} sentence with binary branching, illustrated in Figure~\ref{GSfigure9}, this verbal complex only shows up structurally when there is a series of verbs attracting the complements of their complements, as in (\ref{GSexemple38}) (see Figure~\ref{GSfigure9b}).

The phrase structure constraint allowing complex predicates is as in (\ref{GSexemple42})
(\citealt{MuellerCopula}; \citealt[39]{muller2018clause}). It is called \type{head-cluster-phrase},
rather than \type{verbal"=complex"=phrase}, because it is not specialized for verbal heads (see also
(\ref{GSexemple26})).\footnote{Following \citet[\page 23]{HN94a} and \citet[\page 177]{dKM2001a}, but contrary to
  Müller (\citealt[\page 23]{Mueller2005c}; \citealt{muller2018clause}), we mention the lightness of the mother.}\footnote{The description of the \emph{head-cluster-phrase} in (\ref{GSexemple42}) is the same as that in (\ref{GSexemple26}), only more general, (\ref{GSexemple26}) being specified as having a verbal head.}  

\eas
\label{GSexemple42}\label{ex-head-cluster-phrase-German}%
\type{head-cluster-phrase}\is{schema!Head-Cluster} (\ili{German}) \impl \\
\avm{
[synsem & [\punk{loc|cat|comps}{ \1} \\
	   light & $+$] \\
 head-dtr|synsem & [\punk{loc|cat|comps}{ \1 \+ < \2 >} \\
                    light $+$ ] \\
 non-head-dtrs & <[synsem & \2 [light & $+$ ] ]> ]
}
\zs

We illustrate the analysis with sentence (\ref{GSexemple38b}) (\ldots{} \emph{dass er das Auto Peter
  kaufen sehen wird} `that he will see Peter buy the car’), elaborating on
Figure~\ref{GSfigure9b}. The description of \emph{werden} (the future auxiliary), a subject raising
verb and a verb constructing coherently, is as in (\ref{GSexemple43}) (from
\citealt[39]{muller2018clause}), and that of \emph{sehen} `to see', an object raising verb and an
obligatorily coherent verb, is as in (\ref{GSexemple44}) (adapted from
\citealt[102]{Mueller2002b}). The subject and other arguments are raised from the embedded verb. The
infinitive is analyzed as having the feature [\vform \textit{bse}], where \textit{bse} stands for
\emph{base}. What forces these verbs to be part of a \emph{head-cluster-phrase} is that their
infinitive complement is \mbox{[\light{}$+$]}.

\ea
\label{GSexemple43}
\emph{werden} (future auxiliary): \\
\avm{
[head   & verb\\
 arg-st & \1 \+ \2 \+ < V[\type{bse}, subj \1, comps \2, light$+$] > ] }
\z

\ea
\label{GSexemple44}
\emph{sehen} (obligatory coherent verb): \\
\avm{
[head   & verb\\
 arg-st & \sliste{ NP } \+ \1 \+ \2 \+ < V[\type{bse}, subj \1, comps \2, light$+$] > ]}
\z
As mentioned above, subjects of non-finite verbs are represented under \subj. Since the verbs above
attract all arguments, the \subjv and the \compsv are concatenated and represented on the \argstl of
the governing verb.
Hence, these lexical items are parallel to the ones given for the Romance languages (see
(\ref{GSexemple8b}) and (\ref{GSexemple19})) with the
  exception that the selected verb is the last argument in German (SVO vs.\ SOV) and that German
  always attracts the arguments from the \compsl rather than from \argst. The reason for attracting
  arguments from \comps is so called \isi{partial verb phrase fronting} \citep{Mueller96a}: verbs may be combined with a
  subset of their complements in fronted position and only the remaining complements are
  attracted. Since \argst contains the complete list of arguments, attraction has to take \comps as
  the source.

Sentence (\ref{GSexemple38b}) is represented in Figure~\ref{GSfigure11}. 


%%%%%%%%%%%%%%%%%%%%%%   LARG    &&&&&&&&&&&&&&&&&&&&&&&
%\inlinetodostefan{Stefan: I changed this figure. It is now CP and \vform is under head.}

\begin{figure}
%    \centering
\oneline{%
\begin{forest}
sm edges
%    fairly nice empty nodes,
[CP
  [C[dass;that]]
  [{V[\comps \eliste]}
     [\ibox{1} NP [er;he]]
     [{V[\comps \sliste{ \ibox{1} }]}
     	[\ibox{2} NP [das Auto;the car, roof]]
     	[{V[\comps  \sliste{ \ibox{1}, \ibox{2} } ]}
     	  [\ibox{3} NP [Peter;Peter]]
	  [V\feattab{\head \ibox{4},\\
                     \comps \sliste{ \ibox{1}, \ibox{3}, \ibox{2} }}
	     [\ibox{7} V\feattab{\head \ibox{5}, \\
                                 \subj \sliste{ \ibox{1} },\\
			     	 \comps \sliste{ \ibox{3}, \ibox{2} }}
               [\ibox{6} V\feattab{\vform bse, \\
                                   \subj \sliste{ \ibox{3} },\\
                		   \comps \sliste{ \ibox{2} } }
                					[kaufen;buy]]
               [V\feattab{\head \ibox{5} [\vform \type{bse}], \\
                          \subj \sliste{ \ibox{1} },\\
                	  \comps \sliste{ \ibox{3}, \ibox{2}, \ibox{6} } } 
                 [sehen;see]]]
	[ V\feattab{\head \ibox{4} [\vform \type{fin}], \\
     		    \comps \sliste{ \ibox{1}, \ibox{3}, \ibox{2}, \ibox{7} } }%,   before computing xy={s'-=35pt} 
     						[wird;will]]]]]]]
 \end{forest}}    
    \caption{Coherent construction with verbal complexes in German}
    \label{GSfigure11}
%\itdopt{I would project the \ibox{4} to the top node and renumber the whole figure.}
\end{figure}

%%%%%%%%%%%%%%%%%%%%%%   LARG    &&&&&&&&&&&&&&&&&&&&&&&


\subsubsection{The German copula}\label{GSsection4.1.3}\label{cp:sec-copula-German}

The copula in \ili{German}, with an adjectival argument, is also the head of a complex predicate.\footnote{As in \ili{Romance} languages, the \ili{German} copula accepts nominal and prepositional predicative complements. However, they are complement saturated.} The subject of the copula and the complements of the adjectives can be permuted (examples from \citealt[68]{Mueller2002b}; see (\ref{GSexemple38}) for coherent verbs):

\eal 
	\label{GSexemple45} 
    \ex[]{
	\gll \ldots{} dass die Sache                 dem Minister                ganz       klar  war\\ 
	     {}       that {the.\textsc{nom}} matter {the.\textsc{dat}} minister completely clear was\\
	\glt `that the matter was completely clear to the minister'}\label{GSexemple45a}

	\ex[]{ 
	\gll \ldots{} dass dem Minister                die Sache                 ganz       klar  war \\
		 {}       that {the.\textsc{dat}} minister {the.\textsc{nom}} matter completely clear was\\
	\glt `that the matter was completely clear to the minister'}\label{GSexemple45b}
\zl

Adverbs can have different scopings: in (\ref{GSexemple46}) (from \citealt[68]{Mueller2002b}),
\emph{immer} `always' can modify the modal or the adjective. This follows if there is just one
coherence field and both the modal and the copula are the head of a complex predicate (see
Section~\ref{GSsection4.1.2}, example (\ref{GSexemple39b}) for verbs constructing coherently).

\ea[]{
	\label{GSexemple46}
	\gll \ldots{} weil    der Mann             ihr              immer  treu     sein wollte\\
	     {}       because the.\textsc{nom} man her.\textsc{dat} always faithful be   wanted.to\\
	\glt `because the man always wanted to be faithful to her'\\
	\glt `because the man wanted to be faithful to her forever'}
\z

\citet{Mueller2002b} also shows that the copula does not take a saturated AP complement. Contrary to a construction with a verb constructing incoherently, this AP cannot be extraposed, as shown in (\ref{GSexemple47b}), or pied piped with a relative pronoun, as shown in (\ref{GSexemple47d}) (from \citealt[70]{Mueller2002b}; compare with (\ref{GSexemple36c}), (\ref{GSexemple36d})).   

\eal
	\label{GSexemple47} 
	\ex[]{
	\gll Karl ist auf seinen Sohn stolz gewesen.\\ 
		 Karl is  on  his    son  proud been\\
	\glt `Karl was proud of his son.'}\label{GSexemple47a}

	\ex[*]{ 
	\gll Karl ist gewesen auf seinen Sohn stolz.\\
		 Karl is  been    on  his    son  proud\\
	\glt Intended: `Karl was proud of his son.'}\label{GSexemple47b}
 
    \ex[]{
	\gll der Sohn, auf den  Karl stolz gewesen ist\\ 
		 the son   on  whom Karl proud been    is\\
	\glt `the son of whom Karl was proud'}\label{GSexemple47c}
		     
	\ex[*]{
	\gll der Sohn, auf den stolz Karl gewesen ist\\ 
		 the son on whom proud Karl been is\\
	\glt Intended: `the son of whom Karl was proud'}\label{GSexemple47d}
\zl

In addition, the \ili{German} copula, like the \ili{Romance} copula, is a subject raising verb: the semantic properties of the subject depend on the adjective (a human is proud or faithful, and a matter is clear, as shown also by the nominalizations, cf.\ the man's faithfulness, the clarity of the matter); moreover, the sentence can be subjectless (from \citealt[72]{Mueller2002b}): 

\ea[]{
	\label{GSexemple48}
	\gll Am     Montag ist schulfrei.\\
		 at.the Monday is  school.free\\
	\glt `There is no school on Monday.'}
\z

The description of the \ili{German} copula, restricted to its predicative use and to its syntactic
part, is as follows \citep[\page 226]{MuellerPredication}:\footnote{
  As \citet[\page 227]{MuellerPredication} notes, this copula also works for \ili{English} and
    other Germanic SVO languages. Since these languages do not have a Head-Cluster Schema, the
    copula has to be used in the Head-Complement Schema, which requires complements to be saturated,
    hence \ibox{2} is the empty list for English and other \ili{Germanic} SVO languages.
}
\ea
\label{GSexemple49}
\emph{sein} (copula): \\
\avm{
[head & verb\\
 arg-st & \1 \+ \2 \+ <[head  & [prd & $+$]\\
                        subj  & \1\\
                        comps & \2 ]> ]
}
\z

\noindent
It differs from the \ili{Romance} copula in not specifying the lightness of its predicative
complement. So, while German allows for the formation of a predicate complex (a head–cluster phrase) with predicative
  adjectives and normal head-complement structures with predicative NPs and PPs, the Romance copula only
  allows XP arguments, which can be complement saturated or not.
% \itdopt{What do you do about NPs and PPs in Romance languages if the predicative
%   complement has to be LIGHT+? PPs and NPs are not LIGHT+. How does the combination with the copula
%   work in Romance then?} 

\subsection{Argument attraction with Korean auxiliaries}\label{GSsection4.2}

Like \ili{German} complex predicates, \ili{Korean} auxiliary constructions allow the arguments of the auxiliary and its verb complement to be interleaved. Other properties (case marking, passivization) clearly show that the auxiliary forms a complex predicate with its verbal complement. Control verbs also allow for scrambling, but they do not exhibit the same behavior as auxiliaries, and we will not consider them as heads of complex predicates. As in \ili{German} again, the auxiliary and its verbal complement constitute a verbal complex.

\subsubsection{Properties of Korean auxiliaries}\label{GSsection4.2.1}
 
\largerpage 
\ili{Korean} resembles \ili{German} in that a complex predicate is associated with word order
properties (see \citealt{Sells1991, Chung98a-u, Yoo2003, Kim2016a-u}). We illustrate here the case
of auxiliaries.\footnote{\citet{Chung98a-u} also considers control verbs to be the head of complex
  predicates, and \citegen{Kim2016a-u} study, which excludes control verbs, includes serial verbs
  and light verb constructions.}

\ili{Korean} auxiliaries semantically resemble aspectual or modal verbs rather than tense auxiliaries:
they include such verbs as \emph{iss-} `to be in the process/state of', \emph{chiwu-} `to do
resolutely’, \emph{siph-} `to want', but also the verb of negation \emph{anh-} `not' (see also
\crossrefchapteralt[Section~\ref{sec-negative-auxiliary-verb}]{negation}). They bear the tense
marking for the sentence (\ref{GSexemple50a}), impose a certain ending to their verbal complement
(\emph{-e} in (\ref{GSexemple50a})), and, when they have a use as ordinary verbs
(\ref{GSexemple50b}), they have an argument structure which is absent in their auxiliary use
(examples from \citealt[85--86]{Kim2016a-u}). 

\eal
	\label{GSexemple50} 
    \ex[]{
	\gll Mia-ka             wul-e               pely-ess-ta.\\ 
		{Mia-\textsc{nom}} {cry-\textsc{conn}} {end.up-\textsc{pst}-\textsc{decl}}\\
	\glt `Mia ended up crying.'}\label{GSexemple50a}

    \ex[]{
	\gll Mimi-nun            congi-lul            hyucithong-ey            pely-ess-ta.\\ 
		{Mimi-\textsc{top}} {paper-\textsc{acc}} {trash.can-\textsc{loc}} {throw.away-\textsc{pst}-\textsc{decl}}\\
	\glt `Mimi threw away the paper in the trash can.'}\label{GSexemple50b}
\zl

In (\ref{GSexemple50b}), the verb has three arguments: agent subject, theme object, and location complement. This argument structure is absent in (\ref{GSexemple50a}).

Consider the sentences in (\ref{GSexemple51}). There is no evidence of scrambling in (\ref{GSexemple51a}): the subject \emph{Maryka} (`Mary' + nominative) starts the sentence, and the complement of the verb \emph{ilkko} `read' immediately precedes it. However, in (\ref{GSexemple51b}), the subject of the head verb \emph{issta} `be in the process of', namely \emph{Maryka}, occurs between the complement of \emph{ilkko}, namely \emph{ku chaykul} (`the book' + accusative), and the verb \emph{ilkko} itself.
\eal
	\label{GSexemple51}
	\ex[]{
	\gll Mary-ka           ku  chayk-ul          ilk-ko  	        iss-ta.\\ 
             Mary-\textsc{nom} the book-\textsc{acc} read-\textsc{conn} be.in.the.process.of-\textsc{decl}\\
	\glt `Mary is in the process of reading the book.'}\label{GSexemple51a}
		
	\ex[]{
	\gll Ku  chayk-ul          Mary-ka           ilk-ko  	        iss-ta.\\ 
	     the book-\textsc{acc} Mary-\textsc{nom} read-\textsc{conn} be.in.the.process.of-\textsc{decl}\\
	\glt `Mary is in the process of reading the book.'}\label{GSexemple51b}

\zl

A priori, these data could be explained in two ways: either the auxiliary always takes a VP
complement, and scrambling is due to linearization, in which case the domains of the two verbs are
unioned (see \citealt{Reape94a} and also \crossrefchapteralt[Section~\ref{sec-domains}]{order}); or
there is a complex predicate: the complement of the embedded verb (\emph{ku chaykul} `the book' +
accusative) is attracted by the auxiliary verb.

\largerpage[2]
There are several properties which show that auxiliaries attract their verbal complements'
arguments. First, the presence of the auxiliary allows for case alternation: the argument of a verb
like \emph{mek-} `to eat' is assigned accusative case, as shown in (\ref{GSexemple52a}); however,
when the verb is the complement of the auxiliary verb \emph{siph-} `to want' in
(\ref{GSexemple52b}), it can be either accusative or nominative (examples (\ref{GSexemple52}) from
\citealt[87]{Kim2016a-u}).

\eal
	\label{GSexemple52}
	\ex[]{
	\gll Mimi-ka			 sakwa-lul/*ka		      mek-ess-ta.\\ 
		 {Mimi-\textsc{nom}} {apple-\textsc{acc/nom}} {eat-\textsc{pst-decl}}\\
	\glt `Mimi ate an apple.'}\label{GSexemple52a}
		
	\ex[]{
	\gll Mimi-ka			 sakwa-lul/ka		      mek-ko		      siph-ess-ta.\\ 
		 {Mimi-\textsc{nom}} {apple-\textsc{acc/nom}} {eat-\textsc{conn}} {wish-\textsc{pst-decl}}\\
	\glt `Mimi would like to eat an apple.'}\label{GSexemple52b}
\zl

Given that case assignment is a local phenomenon, and a verb does not influence the case of the
complement of its complement, this indicates that \emph{sakwa-} `apple' becomes the complement of
the auxiliary (see also \citealt{Yoo2003}). Moreover, in \ili{Korean}, a negative polarity item such as
\emph{amwukesto} `anything' is licensed by a clause-mate negated element. (\mex{1}) provides examples. (\ref{GSexemple53a}) and (\ref{GSexemple53b}) show that the negative verb \emph{anh-} allows this negative polarity item as the argument of \emph{mek-} `to eat', the complement of the auxiliary \emph{siph-} `to want' (examples from \citealt[91]{Kim2016a-u}). On the other hand, this negative polarity item is not licensed when the negated verb is \emph{seltukha-} `to persuade', which is not an auxiliary (\ref{GSexemple53c}).

\eal
	\label{GSexemple53}
	\ex[]{
	\gll Mimi-nun			 amwukes-to	     mek-ci		         anh-ass-ta.\\ 
		 {Mimi-\textsc{top}} {anything-also} {eat-\textsc{conn}} {not-\textsc{pst-decl}}\\
	\glt `Mimi did not eat anything.'}\label{GSexemple53a}
		
	\ex[]{
	\gll Mimi-nun			 amwukes-to	     mek-ko		         siph-ci		      anh-ass-ta.\\ 
		 {Mimi-\textsc{top}} {anything-also} {eat-\textsc{conn}} {wish-\textsc{conn}} {not-\textsc{pst-decl}}\\
	\glt `Mimi did not feel like eating anything.'}\label{GSexemple53b}
	
	\ex[*]{
	\gll Mimi-lul			 amwukes-to	     mek-tolok		     seltukha-ci		      anh-ass-ta.\\ 
		 {Mimi-\textsc{acc}} {anything-also} {eat-\textsc{conn}} {persuade-\textsc{conn}} {not-\textsc{pst-decl}}\\
	\glt Intended: `(We) did not persuade Mimi to eat anything.'}\label{GSexemple53c}
	
\zl

\largerpage
Finally, the same argument can be levelled against an analysis which appeals to linearization, as
above in \ili{German} (Section~\ref{GSsection4.1.2}): so-called long passivization is possible with
certain auxiliaries like \emph{chiwu-} `to do resolutely', which cannot be accounted for by appeal
to linearization and domain union (examples from \citealt[164]{Chung98a-u}).\footnote{Such passives
  are judged unnatural by native speakers, hence the question mark.} (\ref{GSexemple54a})
exemplifies the active sentence, and (\ref{GSexemple54b}) the passive one. In (\ref{GSexemple54a}),
\emph{malssengmanhun solul} `the troublesome cow' is the complement of the complement verb
\emph{phal-} `to sell'. In (\ref{GSexemple54b}), \emph{malssengmanhun soka} is the subject of the
passivized verb \emph{chiwe ciessta}.

\eal
	\label{GSexemple54} 
	\ex[]{
	\gll Ku  nongpwu-ka            malssengmanhun so-lul             phal-a 		      chiw-ess-ta.\\ 
		 the {farmer-\textsc{nom}} troublesome    {cow-\textsc{acc}} {sell-\textsc{conn}} {do.resolutely-\textsc{pst}-\textsc{decl}}\\
	\glt `The farmer resolutely sold the troublesome cow.'}\label{GSexemple54a}
		
	\ex[?]{
	\gll Malssengmanhun so-ka              (ku           nongpwu-eyuyhay) phal-a               chiw-e                                  ci-ess-ta.\\ 
		 troublesome    {cow-\textsc{nom}} \spacebr{}the {farmer-by}      {sell-\textsc{conn}} {do.resolutely-\textsc{conn}} {\textsc{pass}-\textsc{pst}-\textsc{decl}}\\
	\glt `The troublesome cow was resolutely sold (by the farmer).'}\label{GSexemple54b}
\zl

\noindent
Since passivization only affects the complement of the verb which is itself passivized, it follows that \emph{malssengmanhun solul} `troublesome cow' is the complement of the auxiliary in (\ref{GSexemple54a}).

The scrambling data with control verbs, as in (\ref{GSexemple55}), are very similar to those with auxiliaries (examples from \citealt[189--190]{Chung98a-u}). There is no scrambling in (\ref{GSexemple55a}): the dative complement of the head verb is followed
by the other complement, a VP. However, in (\ref{GSexemple55b}), the subject of the head verb
(\emph{Maryka} `Mary' + nominative) occurs between the complement of the complement verb (\emph{ku
  chaykul} `the book' + accusative) and the dative complement of the head verb (\emph{Johnhanthey} `John' + dative).

\eal
	\label{GSexemple55} 
	\ex[]{
	\gll Mary-ka    	     John-hanthey        [ku           chayk-ul            ilk-ulako]       seltukha-yess-ta.\\ 
		 {Mary-\textsc{nom}} {John-\textsc{dat}} \spacebr{}the {book-\textsc{acc}} {read-\textsc{conn}} {persuade-\textsc{pst}-\textsc{decl}}\\
	\glt `Mary persuaded John to read the book.'}\label{GSexemple55a}
		
	\ex[]{
	\gll Ku  chayk-ul             Mary-ka    	     John-hanthey         ilk-ulako            seltukha-yess-ta.\\ 
		 the {book-\textsc{acc}} {Mary-\textsc{nom}} {John-\textsc{dat}}  {read-\textsc{conn}} {persuade-\textsc{pst-decl}}\\
	\glt `Mary persuaded John to read the book.'}\label{GSexemple55b}
\zl

However, we do not observe case alternation in this case, and control verbs fail to allow the negative polarity item \emph{amwukesto} `anything' as the complement of the verb complement (\citealt[91]{Kim2016a-u}).

\eal
	\label{GSexemple55added} 
	\ex[]{
	\gll Mimi-lul            amwukes-to    an mek-tolok           selkhuta-yess-ta.\\ 
		 {Mimi-\textsc{acc}} anything-also no {eat-\textsc{conn}} {persuade-\textsc{pst-decl}}\\
	\glt `(We) persuaded Mimi not to eat anything.'}\label{GSexemple55addeda}
	
	\ex[*]{
	\gll Mimi-lul            amwukes-to    mek-tolok           selkhuta-ci              anh-ass-ta.\\ 
		 {Mimi-\textsc{acc}} anything-also {eat-\textsc{conn}} {persuade-\textsc{conn}} {not-\textsc{pst-decl}}\\
	\glt Intended: `We did not persuade Mimi to eat anything.'}\label{GSexemple55addedb}
\zl

Accordingly, we follow \citet[93--94]{Kim2016a-u} in not analyzing control verbs as heads of complex predicates. They take VP complements, and scrambling in (\ref{GSexemple55}) must be due to a
different process (that is, domain union, as in \citealt{lee2001argument}; see \citealt{Reape94a}).


\subsubsection{Korean auxiliaries and the verbal complex}\label{GSsection4.2.2}

It has been shown in this chapter that different structures could be associated with argument attraction. \ili{Korean} auxiliaries are the head of a verbal complex (\citealt{Chung98a-u}; \citealt{Kim2016a-u}). The main fact is that nothing can intervene between the two verbs, for instance no parenthetical expression, such as \emph{hayekan} `anyway', as illustrated in (\ref{GSexemple56added}) (examples from \citealt[162]{Chung98a-u}). This contrasts with control verbs. In (\ref{GSexemple57added}), the adverb \emph{cengmal} `really' can occur before the embedded verb, or between the two verbs (example (\ref{GSexemple57added}) from \citealt[93]{Kim2016a-u}).


\eal
	\label{GSexemple56added} 
	\ex[]{
	\gll Mary-ka    	     hayekan sakwa-lul            mek-ko 	          iss-ta.\\ 
		 {Mary-\textsc{nom}} anyway  {apple-\textsc{acc}} {eat-\textsc{conn}} {be.in.the.process.of-\textsc{decl}}\\
	\glt `Anyway, Mary is eating an apple.'}\label{GSexemple56added-a}

	\ex[*]{
	\gll Mary-ka    	     sakwa-lul            mek-ko              hayekan iss-ta.\\ 
		 {Mary-\textsc{nom}} {apple-\textsc{acc}} {eat-\textsc{conn}} anyway  {be.in.the.process.of-\textsc{decl}}\\
	\glt Intended: `Anyway, Mary is eating an apple.'}\label{GSexemple56added-b}
\zl

\ea{
	\label{GSexemple57added} 	
	\gll Mimi-nun	         Haha-lul 	         (cengmal)           ttena-tolok	       (cengmal)	    seltukha-yess-ta.\\ 
		 {Mimi-\textsc{top}} {Haha-\textsc{acc}} \spacebr{}really    {leave-\textsc{conn}} \spacebr{}really {persuade-\textsc{pst-decl}}\\
	\glt `Mimi (really) persuaded Haha to (really) leave.'}
\z

In addition, there is evidence that the verb complement of an auxiliary and its complement do not form a constituent. While an NP may occur after the head verb in a so-called afterthought construction (\ref{GSexemple56a}), this is not possible for the embedded verb \emph{mek-} with its complement (\ref{GSexemple56b}) (from \citealt[162]{Chung98a-u}).

%As shown by \citet{Chung98a-u}, in \ili{Korean}, the structure of a complex predicate is different depending on whether the head verb is an auxiliary or a control verb. Auxiliaries are the head of a verbal complex (as in \ili{German} complex predicates), while control verbs can either take a saturated VP complement, or be the head of a flat structure (like restructuring verbs in \ili{Italian}). Thus, complex predicates in \ili{Korean} confirm the observation made for \ili{Romance} Languages, although the complex predicate manifests itself by different properties: argument attraction is not correlated with a specific structure. 

%A characteristic property of \ili{Korean} auxiliary constructions is that nothing can separate the two verbs: no parenthetical expression, such as \emph{hayekan}, can intervene (\ref{GSexemple55}). In addition, the embedded verb cannot occur by itself. Thus, while an NP may occur after the head verb in a construction called afterthought (\ref{GSexemple56a}), this is not possible for the embedded verb \emph{mekko} alone (\ref{GSexemple56b}) or with its complement (\ref{GSexemple56c}) (examples from \citealt[162]{Chung98a-u}). This behavior follows if the two verbs form a verbal complex (see Section~\ref{GSsection3.2}).

%\eal 
%	\label{GSexemple55} 
%	\ex[]{
%	\gll Mary-ka             hayekan sakwa-lul            mek-ko              iss-ta.\\ 
%		 {Mary-\textsc{nom}} anyway  {apple-\textsc{acc}} {eat-\textsc{part}} {be.in.the.process.of-\textsc{part}}\\
%	\glt `Anyway, Mary is eating an apple.'}\label{GSexemple55a}
%
 %   \ex[*]{
%	\gll Mary-ka             sakwa-lul            mek-ko              hayekan iss-ta.\\
%		 {Mary-\textsc{nom}} {apple-\textsc{acc}} {eat-\textsc{part}} anyway  {be.in.the.process.of-\textsc{part}}\\
%	\glt `Anyway, Mary is eating an apple.'}\label{GSexemple55b}
%\zl

\eal
	\label{GSexemple56} 
	\ex[]{
	\gll Mary-ka             mek-ko              iss-ta,                              sakwa-lul.\\ 
		 {Mary-\textsc{nom}} {eat-\textsc{conn}} {be.in.the.process.of-\textsc{decl}} {apple-\textsc{acc}}\\
	\glt `Mary is in the process of eating an apple.'}\label{GSexemple56a}

    \ex[*]{
	\gll Mary-ka            iss-ta,                              sakwa-lul            mek-ko.\\ 
		{Mary-\textsc{nom}} {be.in.the.process.of-\textsc{decl}} {apple-\textsc{acc}} {eat-\textsc{conn}}\\
	\glt Intended: `Mary is in the process of eating an apple.'}\label{GSexemple56b}
\zl

These data point to a verbal complex (see Section~\ref{GSsection3.2}). However, before coming to this conclusion, we must show that the two verbs do not form a compound word. \citet{no1991case} (summarized in \citealt{Chung98a-u}, \citealt{Kim2016a-u}) presents arguments to the effect that they combine in the syntax. The main one relies on the use of delimiters. A delimiter (such as \emph{-man} `only' or \emph{-to} `also') can combine with the embedded verb (\eg \emph{mekkoman issta} `to be only eating'). Delimiters are a syntactic phenomenon, not limited to verbal morphology. 
% Auxiliary constructions may involve several auxiliaries, and there is no limit to the number of auxiliaries which may combine, with the appropriate ending. It is not plausible to list all the possible combinations in the lexicon. (\ref{GSexemple57added-a}), which repeats (\ref{GSexemple51a}), exemplifies a sentence with one auxiliary, and (\ref{GSexemple57added-b}) a sequence of two auxiliaries (\citealt[172]{Chung98a-u}).
%
% \eal
% 	\label{GSexemple57added-ab}
% 	\ex[]{
% 	\gll Mary-ka            ku   chayk-ul            ilk-ko  	          iss-ta.\\ 
% 		{Mary-\textsc{nom}} the  {book-\textsc{acc}} {read-\textsc{conn}} {be.in.the.process.of-\textsc{decl}}\\
% 	\glt `Mary is in the process of reading the book.'}\label{GSexemple57added-a}
%	
% 	\ex[]{
% 	\gll Mary-ka            ku   chayk-ul           ilk-ke  	         po-ko               iss-ta.\\ 
% 		{Mary-\textsc{nom}} the {book-\textsc{acc}} {read-\textsc{conn}} {try-\textsc{conn}} {be.in.the.process.of-\textsc{decl}}\\
% 	\glt `Mary is in the process of giving the book a trial reading.'}\label{GSexemple57added-b}
% \zl
Thus, the head auxiliary and the complement verb form a verbal complex.

\subsubsection{Korean auxiliaries in HPSG}\label{GSsection4.2.3}

Given the free word order in \ili{Korean} (except for the verb), there are two ways of representing the sentence: either there is a flat structure (except for the verbal complex), where all the arguments, subject and complements, are sisters of each other (see, among others, \citealt{Chung98a-u} for \ili{Korean}), or there is a binary branching structure (see \citealt{Kim2016a-u} for \ili{Korean}). We adopt the flat structure here since the differences between the two approaches are irrelevant for the purpose of this chapter (but see \crossrefchapteralt[Section~\ref{sec-binary-flat}]{order} for binary branching).


The general schema for the sentence is given in (\ref{GSexemple60added}), adapted from
\citew[178]{Chung98a-u}.\footnote{This is an instance of a more general schema, needed independently
  for VSO languages, and subject inversion in English \parencites[\page 388]{ps2}[231--232]{GSag2000a-u}.}
% St. Mü. There is no / sign. I removed this sentence. 07.08.2020
%The sign `/' indicates that, by default, the value of \textsc{subj} and \textsc{comps} is the empty list.

\ea
\label{GSexemple60added}
\type{head-subject-complements-phrase}\is{schema!Head-Subject-Complements} (\ili{Korean}) \impl \\
\avm{
[synsem
		[loc|cat & [ subj  & < >\\
                    comps & < > ]\\
         light $-$ ]\\
 head-dtr|synsem 
 		[loc|cat  & [ head  & verb\\
% AUX- would not work here, since the verbal complex is AUX+. See figure. St.Mü. 07.08.2020
%[\type*{verb}\\ 
%                                      aux  & $-$]\\
                      subj  & \1\\
                      comps & \2]\\
         light $+$]\\
 non-head-dtrs \rel{synsems2signs}(\1 \+ \2) \type{ne-list} ]
    }
\z
This schema combines a head with its subject and its complements in one go. Since no LP constraints
are formulated, subjects and objects can be scrambled and permutations are accounted for. The \subjl and the
\compsl contain \type{synsem} elements. These lists are appended into one list, which is then
converted into a list of signs by the
relational constraint \rel{synsems2signs}. A further constraint -- not given in (\mex{0}) -- requires that the
non-head daughters must be [\light{}$-$].\footnote{%
  See \citet[\page 23]{Mueller2005c} and \citet[Section~2.2.4]{MuellerGS} for an explicit formulation of
  such a constraint in a grammar of \ili{German}.
} This ensures that arguments of auxiliaries cannot be realized
in flat structures licensed by (\ref{GSexemple60added}) since auxiliaries select for \light{}$+$
complements. 

The lexical item of the auxiliary \emph{issta} `be in the process of' in
(\ref{GSexemple57added-a}) is provided in (\ref{GSexemple62added}):\footnote{%
\citet[\page 94--95]{Kim2016a-u} argues that complex predicate formation in \ili{Korean} results from a Head-\textsc{lex}
construction that ensures that the \compsl of the mother is identical to the \compsl of the verb
daughter that is the complement to the auxiliary. For reasons of space and to make a comparison
between \ili{Korean} complex predicate formation and complex predicate formation in \ili{Romance}, \ili{German}, and
\ili{Persian} easier, we adopt a lexical analysis of complex predicate formation in \ili{Korean}, as proposed in
\citew{Chung98a-u}.
}
\ea
\gll Mary-ka           ku   chayk-ul          ilk-ko  	         iss-ta.\\ 
     Mary-\textsc{nom} the  book-\textsc{acc} read-\textsc{conn} be.in.the.process.of-\textsc{decl}\\
\glt `Mary is in the process of reading the book.'\label{GSexemple57added-a}
\z
\ea
\label{GSexemple62added}
Lexical description of \emph{issta} `be in the process of': \\
\avm{
[\form  < iss-ta >\\
 head & [\type*{verb}\\
           aux $+$]\\
 subj & \1\\
 comps& \2 \+ < V[vform & ko\\
            subj  & \1\\
            comps & \2\\
            light & $+$] > ]}
\z
Auxiliaries attract both the subject \iboxb{1} and the complements of their verbal complement (list \ibox{2}\,). The subject value is indicated as \ibox{1}, rather than \la\ibox{1}\ra, because the subject is not always realized in \ili{Korean}. 
To indicate which ending it imposes on its complement, we use the feature \textsc{vform}, thus
allowing for the selection of the appropriate ending by the auxiliary (\citealt{Chung98a-u},
\citealt{Kim2016a-u}). So, the verb \emph{issta} selects the ending \emph{-ko} for the verbal
complement, and \emph{ilkko} `read', whose \textsc{vform} value is \emph{ko}, is appropriate. 

% \begin{exe}
%     \ex {Description of \emph{issta}: \\
%     \begin{avm}
%       {\[form \<\normalfont{iss-ta}\>\\
%       head \[\normalfont{\emph{verb}}\\
%            aux +\]\\
%       arg-st \ibox{1} $\oplus$ \< v
%                     \[vform \normalfont{\emph{ko}}\\
%                     arg-st \ibox{1} $\oplus$ \ibox{2}\\
%                     light $+$\\\]\,\> \,\,$\oplus$ \ibox{2}\]}
%     \end{avm}}\label{GSexemple62added}
% \end{exe}



The verbal complex is headed by an auxiliary verb, which is [\aux~$+$], while other verbs are [\aux~$–$].  Thus only auxiliaries can enter this structure. The schema for the verbal complex is given in (\ref{GSexemple57}). The verbal complex is [\light{}$+$] and made up of two verbs, also [\light{}$+$] (see Section~\ref{GSsection3.3}). 


%%%%%%%%%%%%%%%%%%%%%%%%%%%%%%%%%%%%%%%%%%%%%%%%%%%%%%%%%%%%%%%%%%%%%%%%%%%%%%%%%%%%%%%%%%%%%%%%%%%%%%%%%%%%%%%%%%%%%%%%%%%%%%%%%%%%%%%%%%%%%%%%%%%%%%%%%%%%%%%%%%%%%%%%%%%%%%%%%%%%%%%%%%%%%%%%%%%%%%%%%%%%%%%%%%%%%%%%%%%%%%%%
%%%%%%%%%%%%%%%%%%%%%%%%%%%%%%%%%%%%%%%%%%%%%%%%%%%%%%%%%%%%%%%%%%%%%%%%%%%%%%%%%%%%%%%%%%%%%%%%%%%%%%%%%%%%%%%%%%%%%%%%%%%%%%%%
%\newpage
\eas 
\label{GSexemple57}%
\type{head-cluster-phrase}\is{schema!Head-Cluster} (\ili{Korean}) \impl \\
\avm{
[synsem|loc &
		[cat|comps & \1\\
         light & $+$]\\
 head-dtr|synsem &
 	    [loc|cat & 
        		[head  & [\type*{verb}\\
                          aux +]\\
%                                 subj  & \1\\
                 comps & \1 \+ < \2 >]\\
         light & $+$
]\\
 non-head-dtrs & < [synsem \2 [light & $+$]] %[head  & verb\\ superfluous St.Mü. 03.08.2020
                           % subj  & \1\\
                           % comps & \3\\
                           % light & $+$] 
> ]}
%\itdopt{The constraint on LIGHT+ in the non-head daughter is coming from the head.}
\zs


(\ref{GSexemple57}) is an instance of the more general description in (\ref{GSexemple26}), restricting the 	availability of the phrase to auxiliaries. The verbal complex schema saturates the last element of the \compsl of the head daughter. In this way it is parallel to the head-subject-complements phrase. The only difference is that the argument
that is combined with the auxiliary is [\light{}$+$] as is required by the auxiliary. The \subjl is not mentioned in the constraints on \type{head-cluster-phrase}. That the \subjv of the head daughter is identical to the \subjv of the mother follows from constraints on more general types that are inherited \crossrefchapterp[\page \pageref{prop:valence-principle}]{properties}. 


The structure of sentence (\ref{GSexemple57added-a}) is represented in Figure~\ref{GSfigure12}.

%%%%%%%%%%%%%%%%%%%%%%%%%%%%%%%MAS LARG%%%%%%%%%%%%%%%%%%%%%%%%%%%%%%%%%%
\begin{figure}
    \centering
\begin{forest}
sm edges
 [S [\ibox{1} NP
            [Mary-ka;Mary-\textsc{nom}]]
 [\ibox{2} NP
            [ku chayk-ul;the book-\textsc{acc}, roof]]
  [V\avm{
           [aux & $+$\\
            subj  & < \1 >\\
            comps & < \2 >\\
            light & $+$]}, before computing xy={s'+=15pt} 
    [\ibox{3} V\avm{
                [subj & < \1 >\\
                 comps & < \2 >\\
                 light & $+$]} [ilk-ko;read-\textsc{conn}]]
        [V\avm{[aux & $+$\\
                    subj & < \1 >\\ 
                    comps & < \2, \3 >\\
                 light & $+$] }
             [iss-ta;be.in.the.process.of-\textsc{decl}]]]] 
\end{forest} \caption{Clause structure with a verbal complex in Korean}
    \label{GSfigure12}
\end{figure}{}


The structure of (\ref{GSexemple57added-b}), with a series of two auxiliaries, is represented in
Figure~\ref{GSfigure13} (adapted from \citealt[171]{Chung98a-u}). 

\ea{
 	\gll Mary-ka            ku   chayk-ul           ilk-ke  	         po-ko               iss-ta.\\ 
 		{Mary-\textsc{nom}} the {book-\textsc{acc}} {read-\textsc{conn}} {try-\textsc{conn}} {be.in.the.process.of-\textsc{decl}}\\
 	\glt `Mary is in the process of giving the book a trial reading.'}\label{GSexemple57added-b}
\z

\noindent
The verb \emph{issta} `be in the process of' takes as its complement the verbal complex \emph{ilke
  poko} `try to read', whose head is \emph{poko} `try'. The verb \emph{poko}, being an auxiliary
like \emph{issta}, takes as its complement the verb \emph{ilke}, attracting its subject and
complements, which are transmitted to the verbal complex \emph{ilke poko}; \emph{ilke poko}
saturates the verbal complement expected by \emph{issta}, and transmits the subject and complements
to the head auxiliary (see (\ref{GSexemple62added})).

%\eal
%	\label{GSexemple58} 
%	\ex[]{
%	\gll Mary-ka            ku  chayk-ul            ilk-e                po-ko               iss-ta.\\ 
%		{Mary-\textsc{nom}} the {book-\textsc{acc}} {read-\textsc{part}} {try-\textsc{part}} {be.in.the.process.of-\textsc{decl}}\\
%	\glt `Mary is giving the book a trial reading.'}\label{GSexemple58a}
		
%	\ex[]{
%	\gll Ku  chayk-ul            Mary-ka             ilk-e                po-ko               iss-ta.\\ 
%		 the {book-\textsc{acc}} {Mary-\textsc{nom}} {read-\textsc{part}} {try-\textsc{part}} {be.in.the.process.of-\textsc{decl}}\\
%	\glt `Mary is giving the book a trial reading.'}\label{GSexemple58b}
%\zl

%%%%%%%%%%%%%%%%%%%%%%%%%%%%%%%MAS LARG%%%%%%%%%%%%%%%%%%%%%%%%%%%%%%%%%%

\begin{figure}
\oneline{
\begin{forest}
	sm edges
 [S [\ibox{1} NP [Mary-ka;Mary-\textsc{nom}]]
 [\ibox{2} NP [ku chayk-ul;the book-\textsc{acc}, roof]]
  [V \avm{[aux & $+$\\
            subj & < \1 >\\
            comps & < \2 >\\
            light & $+$]}, before computing xy={s'+=15pt}
    [\ibox{3} V\avm{
            [aux & $+$\\
             subj & < \1 >\\
             comps & < \2 >\\ 
             light & $+$]} [\ibox{4} V\avm{
            [subj & < \1 > \\
             comps & < \2 >\\
             light & $+$]} 
            [ilk-e;read-\textsc{conn}]]
            [V\avm{
            [aux & $+$\\
             subj & < \1 >\\
             comps & < \2, \4 >\\
             light & $+$]} 
            [po-ko;try-\textsc{conn}]]]
    [V \avm{
        [aux & $+$\\
         subj & < \1 >\\
         comps & < \2, \3 >\\
         light & $+$]}[iss-ta;be.in.the.process.of-\textsc{decl}]]]] \end{forest}}
    \caption{Clause structure with verbal complexes in Korean}
    \label{GSfigure13}
\end{figure}

%%%%%%%%%%%%%%%%%%%%%%%%%%%%%%%MAS LARG%%%%%%%%%%%%%%%%%%%%%%%%%%%%%%%%%%

%Contrary to auxiliaries, control verbs such as \emph{seltukha-} (\ref{GSexemple51a}), (\ref{GSexemple51b}) or \emph{cisi-} (\ref{GSexemple54}) can be separated from their verbal complement, for instance by an adverb as in (\ref{GSexemple59a}). They also allow for the infinitive and its complement to form a VP constituent as in (\ref{GSexemple59b}), where it is an afterthought. Thus, control verbs are analyzed in the same way as \ili{Italian} restructuring verbs: they either take a saturated VP complement (\ref{GSexemple59a}) (\ref{GSexemple59b}), or are the head of a complex predicate (\ref{GSexemple59c}) (\ref{GSexemple59d}). They contrast with \ili{Korean} Raising verbs which only take a VP complement (examples (\ref{GSexemple59a}), (\ref{GSexemple59b}) from \citealt[162]{Chung98a-u}, (\ref{GSexemple59c}) from \citealt[190]{Chung98a-u}, (\ref{GSexemple59d}) from \citealt[182]{Chung98a-u}).


%
%\eal
%	\label{GSexemple59} 
%	\ex[]{
%	\gll Mary-ka            ku  mwuncey-lul            phwullye-ko            (kkuncilkikey)		   sito-hayss-ta.\\ 
%		{Mary-\textsc{nom}} the {problem-\textsc{acc}} {solve-\textsc{part}} \spacebr{}ceaselessly {try-\textsc{pst}-\textsc{decl}}\\
%	\glt `Mary tried (ceaselessly) to solve the problem.'}\label{GSexemple59a}
%		
%    \ex[]{
%	\gll Mary-ka             sito-hayess-ta,          [ku           mwuncey-lul            phwullye-ko].\\ 
%		 {Mary-\textsc{nom}} {try-\textsc{part-decl}} \spacebr{}the {problem-\textsc{acc}} {solve-\textsc{part}}\\
%	\glt `Mary tried to solve the problem.'}\label{GSexemple59b}
%		
%	\ex[]{
%	\gll Ku  chayk-ul            Mary-ka             ilkulye-ko            sito-hayss-ta.\\ 
%		 the {book-\textsc{acc}} {Mary-\textsc{nom}} {read-\textsc{part}} {try-\textsc{pst}-\textsc{decl}}\\
%	\glt `Mary tried to read the book.'}\label{GSexemple59c}
%		
%    \ex[]{
%	\gll Ku  chayk-ul            Mary-ka             ilkulye-ko            John-hanthey        seltukha-yss-ta.\\ 
%		 the {book-\textsc{acc}} {Mary-\textsc{nom}} {read-\textsc{part}} {John-\textsc{dat}} {persuade-\textsc{pst}-\textsc{decl}}\\
%	\glt `Mary persuaded John to read the book.'}\label{GSexemple59d}
%\zl
%
%The scrambling data, together with the possibility of long passivization (\ref{GSexemple54}), show that there is a complex predicate. The subject of the head verb occurs between the complement of the infinitive and the infinitive in (\ref{GSexemple59c}), (\ref{GSexemple59d}). More precisely, there is no verbal complex in this case: the two verbs do not have to be contiguous, but can be separated, for instance by a complement as in (\ref{GSexemple59d}) \emph{(John-hanthey)}. Thus, they are the head of a flat structure as in Figure~\ref{GSfigure14} corresponding to (\ref{GSexemple59d}) (see \citealt[190]{Chung98a-u}). The head of verbal complexes in \ili{Korean} is [AUX +], see (\ref{GSexemple57}). Since control verbs are [AUX –], they cannot enter verbal complexes, and the structure for complex predicates whose head is a control verb is a flat structure, corresponding to the Head-subject-complements-phrase in (\ref{GSexemple52}). 
%
%
%
%%%%%%%%%%%%%%%%%%%%%%%%%%%%%%%%MAS LARG%%%%%%%%%%%%%%%%%%%%%%%%%%%%%%%%%%
%
%\begin{figure}
%    \centering
%    {\small
%\begin{forest}
%sm edges
% [S
% [\ibox{1} NP [Ku chayk-ul;the book-\textsc{acc}, roof]]
% [NP [Mary-ka;Mary-\textsc{nom}]]
%  [\ibox{3} V \ms{
%            light & $+$\\
%            comps & \liste{ \ibox{1} }} 
%    [ilk-ulako;read-\textsc{part}]] 
%  [\ibox{2} NP[John-hanthey;John-\textsc{dat}]]
%  [\ibox{3} V \ms{
%            light & $+$\\
%            comps & \liste{ \ibox{2}, \ibox{3}, \ibox{1} }} 
%    [seltukha-yss-ta;persuade-\textsc{pst}-\textsc{decl}]]] 
%    \end{forest}}
%    \caption{Clause structure with a control verb in Korean.}
%    \label{GSfigure14}
%\end{figure}{}
%
%%%%%%%%%%%%%%%%%%%%%%%%%%%%%%%%MAS LARG%%%%%%%%%%%%%%%%%%%%%%%%%%%%%%%%%%
%
%Control verbs which may be the head of a complex predicate are subject or object control verbs. They are the target of an argument attraction rule, in a way parallel to \ili{Italian} Restructuring verbs. 
%
%\begin{exe}
%	\ex {Argument Attraction Rules for \ili{Korean} control verbs}\label{GSexemple60} 
%	\begin{xlist}
%        \ex[]{Subject control verbs \\
%	{\small
%        \begin{avm}
%{\[head & \[{\normalfont{\emph{verb}}}\\
%		            aux-\]\\
%        subj & \<np{\normalfont{\emph{\textsubscript{i}}}}\>\\
%        comps & \<v \[subj & \<np{\normalfont{\emph{\textsubscript{i}}}}\>\\
%                    comps & \<\> \] \,\> \\
%                    light & $+$\]} \end{avm}
%          	$\Rightarrow$
%        \begin{avm}
%		{\[comps & \<v
%		            \[comps & \ibox{2}\\
%		            light & $+$\] \,\> \,$\oplus$ \ibox{2}\]}
%          	\end{avm}}}
%	
%	    \ex[]{Object control verbs \\
%    {\footnotesize
%        \begin{avm}
%		{\[head & \[{\normalfont{\emph{verb}}}\\
%		            aux-\]\\
%        subj & \<np{\normalfont{\emph{\textsubscript{i}}}}\>\\
%        comps & \<np{\normalfont{\emph{\textsubscript{j}}}}, v \[subj & \<np{\normalfont{\emph{\textsubscript{j}}}}\>\\
%                    comps & \<\> \] \,\>\\
%                    light & $+$\]}
%          	\end{avm}
%          	$\Rightarrow$
%        \begin{avm}
%		{\[comps & \<np{\normalfont{\emph{\textsubscript{j}}}}, v
%		            \[comps & \ibox{2}\\
%		            light & +\] \,\> \,$\oplus$ \ibox{2}\]}
%          	\end{avm}}} \end{xlist}
%\end{exe}
%
%An example with the verb \emph{seltukha-} is given in Figure~\ref{GSfigure14}, which represents (\ref{GSexemple59d}).
%


The head comes last in \ili{Korean}, except in the afterthought construction exemplified in (\ref{GSexemple56a}), which requires an additional mechanism. Constraint (\ref{GSexemple61}) mirrors constraint (\ref{GSexemple29}) for \ili{Romance} languages.
%\itdopt{(\ref{GSexemple29}) mentions V.}
\ea
\label{GSexemple61}
\avm{[synsem \1 ] !<! [comps <\ldots, \1, \ldots>]} \jambox*{(\ili{head-final languages})}
\z

\noindent
This constraint holds for the verbal complex, in which the head verb follows the complement verb.


%\citet{Chung98a-u} extends the possibility of argument attraction to adjuncts, as well as to constructions with a S complement, although in the latter case the data are somewhat marginal. The behavior of adjuncts is easily accounted for, if adjuncts can be treated as complements \citep{BMS2001a} by a verb. For the second case, Chung proposes a flat structure, in which the subject, the complements and the (lexical) verbs are all sisters. In this analysis, the definition of a complex predicate, which relies on a syntactic relation between words can then be maintained (but see \citealt{lee2001argument} for a different proposal based on linearization).

\section{Light verb constructions in Persian: Syntax and morphology, syntax and semantics}\label{GSsection5}

Light verb constructions constitute the third guise of complex predicates. They are characterized semantically: the verb and the second predicate constitute together a semantic predicate. For instance, the \ili{French} expression combining a semantically light verb and a noun \emph{faire une proposition} `to make a proposal’ is close to \emph{proposer} `to propose’. They have been studied in HPSG for \ili{Korean} \citep{Ryu:93, lee2001argument, choi2001mixed, Kim2016a-u}. We focus here on \ili{Persian} light verb constructions, which form a rich class and tend to replace simplex verbs.   
    
\subsection{What are complex predicates in Persian?}\label{GSsection5.1}

\ili{Persian} simplex verbs constitute a small closed class of about 250 members, only around 100 of which are commonly used. Speakers resort to complex predicates, sequences of a light verb and a preverbal element belonging to various categories (adjective, noun, particle, prepositional phrase). Following \citet{bonami2010persian} and \citet{pollet2012grammaire}, such sequences are ``multi-word expressions'', that is, they are made up of several words, which, together, form a lexeme. 

Several properties show that the elements are independent syntactic units \citep{Karimi-Doostan97a, Megerdoomian2002a, pollet2012grammaire}. We concentrate on noun + verb combinations, i.e.\ complex predicates in which the preverbal elements are nouns. In what follows, we simply refer to these nominal elements in the complex predicates as ``nouns''.
All inflection is prefixed or suffixed on the verb, as is the negation in (\ref{GSexemple62}), and never on the noun, i.e.\ the nominal part of the complex predicate.
\ea{
	\label{GSexemple62}
	\gll \emph{Dast} be gol-h\=a             na-zan.\\
		 hand          to {flower-\textsc{pl}} {\textsc{neg}-hit}\\
	\glt `Don't touch the flowers.'}
\z
The two elements can be separated by the future auxiliary, or even by clearly syntactic constituents, like the complement PP in (\ref{GSexemple62}). Both the noun and the verb can be coordinated, as shown in (\ref{GSexemple63}) and (\ref{GSexemple64}) respectively (from \citealt[3]{bonami2010persian}), where the coordinations are indicated by the brackets. 
\ea{
	\label{GSexemple63}
	\gll Mu-h\=a=y\=a\v s=r\=a    [\emph{boros} y\=a \emph{\v s\=ane}] zad.\\
		 hair-\textsc{pl=3sg=ra} \spacebr{}brush or            comb                hit\\
	\glt `He/she brushed or combed his/her hair.'}
\z

\ea{
	\label{GSexemple64}
	\gll Omid \emph{sili} [zad          va  xord].\\
		 Omid slap          \spacebr{}hit and stroke\\
	\glt `Omid gave and received slaps.'}
\z
The noun can be extracted, as in the topicalization in (\ref{GSexemple65}), where the sign -- indicates where the non-extracted noun would have occurred. 
\ea{
	\label{GSexemple65}
	\gll \emph{Dast} goft=am           [be          gol-h\=a             -- \hspace{.3em} na-zan].\\
	     hand        said=\textsc{1sg} \spacebr{}to {flower-\textsc{pl}} {} \hspace{.3em} {\textsc{neg}-hit}\\
	\glt `I told you not to touch the flowers.'}
\z
The fact that the noun is linked to a position belonging to a verbal complement (indicated by the brackets) shows that this is extraction, and not simply variation in order.
Complex predicates can also be passivized. In this case, the nominal element of the complex predicate (\emph{tohmat} `slander' in (\ref{GSexemple66a})) becomes the subject of the passive construction (\ref{GSexemple66b}), as does the object of a transitive construction (from \citealt[251]{pollet2012grammaire}). The nominal part of the complex predicate is italicized in the examples.





%\inlinetodostefan{There is a mistake in (\ref{GSexemple66b}) I do not know how to fix.}
\eal
	\label{GSexemple66} 
	\ex{
	\gll Maryam be Omid \emph{tohmat} zad.\\ 
		 Maryam to Omid slander         hit\\
	\glt `Maryam slandered Omid.'}\label{GSexemple66a}
		
    \ex{
	\gll Be Omid \emph{tohmat} zade \v                 sod.\\ 
		 to Omid  slander        hit.\textsc{pst.ptcp} become\\
	\glt `Omid was slandered.'}\label{GSexemple66b}
\zl

There is evidence that the verb and the nominal element in a complex predicate share one argument structure. In (\ref{GSexemple67a}), the verb \emph{d\=adan} `give' takes two complements, the noun \emph{\=ab} `water' and the PP \emph{be b\=aq\v ce} `to garden', while in (\ref{GSexemple67b}) the combination of \emph{d\=adan} and the noun \emph{\=ab} takes a direct object, which is marked with \emph{=r\=a}: in (\ref{GSexemple67b}), the noun \emph{\=ab} and the verb \emph{d\=ad} `gave' form a complex predicate.  

\eal 
	\label{GSexemple67} 
    \ex{
	\gll Maryam be b\=aq\v ce \emph{\=ab}  d\=ad.\\ 
	     Maryam to  garden      water gave\\
	\glt `Maryam watered the garden.'}\label{GSexemple67a}
		
	\ex{
	\gll Maryam b\=aq\v ce=r\=a    \emph{\=ab} d\=ad.\\ 
	     Maryam garden=\textsc{ra} water         gave\\
	\glt `Maryam watered the garden.'}\label{GSexemple67b}
\zl

Other properties show that the combination of the two elements, here a noun and a verb, behaves like a lexeme \citep{bonami2010persian}. Such combinations feed lexeme formation rules: for instance, the suffix \emph{-i} forms adjectives from verbs: \emph{xordan} `eat' > \emph{xordani} `edible', and the same is possible with complex predicates, as shown in (\ref{GSexemple68}); perfect participles can be converted into adjectives by adding the suffix \emph{-e}, and this also applies to complex predicates, as shown in (\ref{GSexemple69}) (see also Section~\ref{GSsection5.2}; from \citealt[5]{bonami2010persian}).

\eal 
	\label{GSexemple68} 
    \ex[]{
	\gll dust  d\=a\v stan  >  \hspace{.3em} dustd\=a\v stani\\ 
		friend {have `love'} {} \hspace{.3em} lovely\\
	}\label{GSexemple68a}
		
    \ex[]{
	\gll xat     xordan                  >  \hspace{.3em} xatxordani\\ 
		 scratch {strike `be scratched'} {} \hspace{.3em} scratchable\\
	}\label{GSexemple68b}
\zl

\eal
	\label{GSexemple69} 
	\ex[]{
	\gll dast xordan                   >  \hspace{.3em} dastxorde\\ 
		 hand {strike `be sullied'} {} \hspace{.3em} sullied\\
	}\label{GSexemple69a}
		
	\ex[]{
	\gll xat     xordan                   > \hspace{.3em} xatxorde\\ 
		 scratch {strike `be scratched'} {} \hspace{.3em} scratched\\
	}\label{GSexemple69b}
\zl

The meaning of the complex predicate is often a specialization of the predictable meaning of the combination: \emph{dast d\=adan} (lit.\ `hand give') means `shake hands', \emph{\v c\=aqu zadan} (lit.\ `knife hit') means `stab', \emph{\v s\=ane zadan} (lit.\ `comb hit') means `comb'. Each specialized meaning has to be learned in the same way as that of a lexeme. Analogy often plays a role in the creation of new lexemes, and this is also true of complex predicates. For instance, the family of complex predicates expressing manners of communication goes from \emph{telegr\=am zadan} `telegraph', where hitting \emph{(zadan)} is involved, to cases where hitting is not clearly involved: \emph{telefon zadan} `phone’, \emph{imeyl zadan} `email', \emph{esemes zadan} `text', etc.   

These complex predicates raise two problems: a morpho-syntactic one and a semantic one. To solve them, we rely crucially on the same property of HPSG as in the preceding syntactic cases, that is, the view of heads as sharing information with their expected complements. 

\subsection{Complex predicates and morphological processes}\label{GSsection5.2}

Although \ili{Persian} complex predicates are combinations of words, they may feed (some)
derivational rules; see Section~\ref{GSsection5.1}, examples (\ref{GSexemple68}) and
(\ref{GSexemple69}). We analyze here what appears to be a nominalization rule, studied in
\citew{MuellerPersian}.\footnote{Müller’s analysis adopts a slightly different approach to the
  issues discussed in the section.} More precisely, the combination of the light verb and its
predicative complement gives rise to a participle from which a noun or adjective can be derived (depending on the
lexeme). What is specially interesting, as pointed by \citet{MuellerPersian}, is that the participle
does not always exist independently of the complex predicate.

The suffix \suffix{ande} is added to the stem~1 of the verb, and it may be added to a complex predicate (\ref{GSexemple70}).

\eal 
	\label{GSexemple70} 
    \ex[]{
	\gll bordan, STEM1 bar \hspace{.3em} >  \hspace{.3em} barande\\ 
		{to win} {}    {}  \hspace{.3em} {} \hspace{.3em} winner\\
	}\label{GSexemple70a}
		
    \ex[]{
	\gll nevestan,  STEM1 nevis \hspace{.3em} >  \hspace{.3em} nevisande\\ 
		 {to write} {}    {}    \hspace{.3em} {} \hspace{.3em} writer\\
	}\label{GSexemple70b}

    \ex[]{
	\gll enteq\=am gereftan, STEM1 gir \hspace{.3em} >  \hspace{.3em} enteq\=am-girande\\ 
		 revenge   take      {}    {}  \hspace{.3em} {} \hspace{.3em} {avenger}\\
	}\label{GSexemple70c}	
\zl


In the case studied by Müller, \emph{b\=az-konande}, the participle corresponding to the light verb construction exists, although the simplex participle does not (\ref{GSexemple70bis}).

\eal 
	\label{GSexemple70bis} 
    \ex[]{
	\gll kardan, STEM1 kon \hspace{.3em} >  \hspace{.3em} *  konande\\ 
		 do      {}    {}  \hspace{.3em} {} \hspace{.3em} {} {Intended: doer}\\
	}\label{GSexemple70bis-a}
		
    \ex[]{
	\gll \v serkat     kardan               \hspace{.3em} >  \hspace{.3em} {\v serkat-konande}\\ 
		 participation {do `participate'} \hspace{.3em} {} \hspace{.3em} {participant (adjective or noun)}\\
	}\label{GSexemple70bis-b}

    \ex[]{
	\gll b\=az kon \hspace{.3em} >  \hspace{.3em} b\=az-konande\\ 
		 open  do  \hspace{.3em} {} \hspace{.3em} {opener}\\
	}\label{GSexemple70bis-c}	
\zl

Our analysis is as follows: the participle and its predicative complement may form a compound word, and it is to this compound that the suffix \suffix{ande} is added. We adopt the representation of compounds in (\ref{GSexemple71}) from \citep[178]{bonami2018lexeme}, here a compound noun, where the elements of the compound are the value of the feature \textsc{m-dtrs} (morphological daughters).\footnote{For more discussion on morphology in HPSG, see \crossrefchapterw{morphology}.}

\ea
\label{GSexemple71}%\scalebox{.95}{
\avm{
			[\type*{lexeme}
			phon & \1 \+ \2 \\ 
			\punk{synsem|loc|cat|head}{noun}\\
			m-dtrs &	<[\type*{lexeme} \\
						phon & \1 \\ 
						\punk{synsem|loc|cat|head}{noun} ],
						[\type*{lexeme}
						phon & \2 \\ 
						\punk{synsem|loc|cat|head}{noun} ]> ]
}
%}%scalebox
\z

Similarly, a compound is formed from the adjective and the verb in the case of \emph{b\=az-konande}. The verb \emph{kon} in the complex predicate \emph{b\=az kon} `to open' is described in (\ref{GSexemple72}). It expects a subject NP, the agent, and two complements, an adjective and an NP, which is attracted from the adjective. The content of the adjective is included in the content of the verb, as the nucleus of the caused \emph{soa} (state of affairs) `make something be adj' (see \citealt[642]{MuellerPersian}).

\ea
\label{GSexemple72}
\avm{
[cat & [ head & \type{verb}\\
         arg-st & < NP$_k$, \1 \NPj, A[ prd & $+$  \\
		                  arg-st & < \1 \NPj >\\
		                  cont   & \2 [\type*{open-relation} \\
		                               theme & j ] ] > ]\\
 cont  & [\type*{soa}\\
          nucleus & [\type*{cause-relation}\\
		     causer & k\\
		     soa-arg|nucleus \2 ]]]
}
\z
%
The lexeme \emph{b\=az-konande} is a noun, based on a compound with two m-daughters to which the suffix \suffix{ande} is appended.
These two daughters are similar to what they are in the complex predicate (\ref{GSexemple72}): the verbal element is expecting two complements, an adjective and an NP, and the semantics is as in (\ref{GSexemple72}). The verb denotes a cause relation taking as argument the adjective content, which is a relation taking the nominal complement as its argument (\textsc{ss} abbreviates \synsem). The description of this noun is given in (\ref{GSexemple73}).

\ea
\oneline{
\avm{
	  	[\type*{lexeme}
		\phon \1 \+ \2 \+ < ande > \\ 
		synsem &	[cat &	[head & noun \\
				         comps & < \3 NP > ] \\
      				cont &	[ind & k] ] \\
    	m-dtrs &	<[phon \1 \\ 
			  ss \4 [loc [cat  [ head & adj \\
                                              arg-st & < \3 \NPj > ]\\
				      cont \5 adj-rel!$(j)$!~~~~~~ ] ] ],
			  [phon \2 \\ 
                	   ss|loc [cat [ head & verb \\
                   	                 arg-st & < NP$_k$, \3, \4 > ] \\
				   cont|nucl	[\type*{cause-rel}
				                 causer & k \\
						 soa-arg|nucl \5 ] ] ] > ]
}}
\label{GSexemple73}
\z

\largerpage[2]
It is worth noticing that, as indicated in (\ref{GSexemple74}), the compound noun takes the NP
expected by the verb as its complement (indicated by the brackets in (\ref{GSexemple74})).

\ea
\label{GSexemple74}
\gll [dar-e botri] b\=az-konande\\
     \spacebr{}lid-\textsc{ez} bottle opener\\
\glt `a bottle opener'
\z

\noindent
This noun is accompanied by the appropriate changes: it denotes the causer, the
first argument of the verb m-daughter, and the suffix \emph{(-ande)} is appended to the sequence of
the two elements. Nothing in this description requires that the simplex participle
(\emph{*kon-ande}) exist independently of the compound. Hence, the intriguing data in
(\ref{GSexemple70bis}) are accounted for.

\subsection{The semantics of light verb constructions}\label{GSsection5.3}

In\is{semantics|(} complex predicates, the noun is not referential; rather, it participates in the meaning of the verbal combination. However, in general, these nouns may also be used as ordinary referential nouns. We assume that such nouns come in two guises: predicative, noted [PRD$+$], which occur in complex predicates, and as referential nouns, noted [PRD$–$].

These complex predicates do not have a homogeneous semantics. The general idea is that the verb serves to turn a noun into a verb \citep{bonami2010persian}, but there is a spectrum, going from a (relatively) semantically compositional combination, to idioms whose semantics is not predictable from the components. Complex predicates exploit different schemas, which can be extended to new nouns, describing new situations. We will exemplify certain common cases, drawing on the detailed study of \emph{zadan} `to hit' in \citew{pollet2012grammaire}. The uses of \emph{zadan} as a light verb are numerous and varied. We will not try to investigate them exhaustively; rather, we indicate different patterns for the combination of this verb with the noun. 

The semantics of a complex predicate is often a specialization of that of the simplex verb. For instance, \emph{lagad zadan} (lit.\ kick hit) means `kick', and \emph{sili zadan} (lit.\ slap hit) means `slap'. 

\ea[]{
\label{GSexemple75}
\gll Ol\=aq be Omid lagad zad.\\
     donkey to Omid kick  hit\\
\glt `The donkey kicked Omid.'}
\z
\largerpage[2]
For cases like (\mex{0}), the content of the complex predicate can be simply that of the noun, if they are of the
same semantic type. In the example, they both denote events (the nucleus is an event-relation). 

This is reminiscent of the way
\citet{Wechsler1995c} represents the import of a PP with a verb like \emph{talk}; the verb content
itself is represented as a \emph{soa} with one participant, the talker; the verb can take a number
of PP complements (headed by \emph{to}, \emph{about}, \ldots), which add semantic information
describing the situation. The result is a description of a \emph{soa} which combines partial
descriptions. Similarly here, the combination of the two contents is identical to the content of
\emph{lagad} `kick', as that latter content is more specialized than that of \emph{zadan}. The
complement of the complex predicate may be an NP or a PP headed by \emph{be} `to' (the preposition
is optional).


\ea
\label{GSexemple76}
\avm{
	  	[\type*{zadan1-lexeme}
		cat & [ head & verb \\
		        arg-st & < NP$_k$, (\type{be}) NP$_m$, N![prd$+$]!:\1 > ]\\
		cont & \1	[\type*{soa}
				nucleus & [\type*{kick-relation}
								actor & k \\
								undergoer & m ] ] ]
	  }
\z

Another case where the combination gives more information than the simplex verb is when this verb
takes as its predicative complement a noun which can also occur as a referential noun denoting an
instrument crucially involved in the situation \citep{bonami2010persian}. Examples are, in different
domains, \emph{\v c\=aqu zadan} (lit.\ knife hit) `stab', \emph{telefon zadan} (lit.\ phone hit)
`phone', \emph{pi\=ano zadan} (lit.\ piano hit) `play the piano'. We illustrate here with \emph{\v
  s\=ane zadan} (lit.\ comb hit) `comb' (example adapted from \citealt{bonami2010persian}).    

\ea[]{
	\label{GSexemple77}
	\gll Maryam {mu-h\=a=ya\v s=r\=a}     \v s\=ane zad.\\
		 Maryam {hair-\textsc{pl=3sg=ra}} comb      hit\\
	\glt `Maryam combed her hair.'}
\z

%\inlinetodostefan{Stefan: Avoid bold weherever possible}
\ea
\label{GSexemple78}
	  \avm{
	  	[\type*{zadan2-lexeme}
		synsem|loc &	[cat & [head & verb \\
					arg-st &	< NP$_k$, NP$_m$, N![prd$+$]!:\2 > ]\\
				 cont & \2	[\type*{soa}
						sit & \1 \\
						nucleus &	[\type*{comb-relation}
								actor & k \\
								undergoer & m ] ] ] \\
		background ~!\{\textsf{involves}(\,\1\textsf{,} $\exists$ \textsf{x} [ \textsf{comb}(\textsf{x}) $\land$ \textsf{use}(\,\1\textsf{, k, x}) ] ) \}! ]
	  }
\z

The condition in the background can be read as follows: the situation \ibox{1} involves that there
exists a comb and that the actor \emph{k} uses it in that situation. Although the complex predicate
includes the content of the predicative complement, the meaning of the complex predicate does not
reduce to that of its semantically more specialized member, as in the preceding case, but adds a
restriction on the background: the existence of an object and the fact that, in the situation, such
an object is used (see \citealt[10]{bonami2010persian}). The complex predicate formation relies on the same semantic
  process as a denominal verb, derived from an instrumental noun (\emph{to ski}, \emph{to iron}).
Further from a compositional or recoverable meaning is the use of \emph{zadan}, or more precisely
\emph{xod=r\=a zadan} (lit.\ self hit), with a series of nouns denoting illnesses, handicaps or
problematic states (like stupidity, ignorance, etc.): it means `to pretend, feign' the illness or
state in question (example (\ref{GSexemple79}) from \citealt[223]{pollet2012grammaire}).

\ea[]{
	\label{GSexemple79}
	\gll Maryam {xod=r\=a}         be div\=anegi zad.\\
		 Maryam {self=\textsc{ra}} to madness    hit\\
	\glt `Maryam feigned madness.'}
\z

\noindent
This use of \emph{zadan} may be seen as an extension of its use with nouns denoting some sort of
deceit, such as \emph{gul zadan} (lit.\ deceit hit) `to deceive’: as in (\ref{GSexemple76}), the
noun imposes its content on the combination, with a metaphorical use of the verb, retaining from the
physical violence meaning of \emph{zadan} `hit’ the idea of an action to the detriment of
someone. Nevertheless, nothing in the actual combination in (\ref{GSexemple79}) indicates
deception. Not all nouns for illnesses are acceptable, only those which cannot really be verified in
the situation: a state of fatigue, but not a heart attack. We group them as objects of type \emph{internal-problematic-state}. Here the combination of the verb and the noun is standard, in that the noun is a semantic argument of the verb, but the meaning of the verb is unpredictable.

\ea
\label{GSexemple80}%
\oneline{\avm{
[\type*{zadan3-lexeme}
cat & [head & verb \\
       arg-st & < NP$_k$, \upshape pro$_k$, PP[cat|head [pform & be \\
						  prd & $+$ ] \\
 					cont \1 [ nucleus [\type*{internal-problematic-state}
							   experiencer & k ] ] ] > ] \\
cont &	[\type*{soa}
		nucleus &	[\type*{pretend-relation}
          			actor & k \\
          			soa-arg & \1 ] ] ]
}}
\z

\noindent
Note that, contrary to \emph{zadan1-lexeme}, with which \emph{be} `to' is optional, the \emph{zadan3-lexeme} requires the predicative complement to be in fact a PP, headed by \emph{be}. We assume that the preposition \emph{be} (frequent in the complement of a complex predicate) is contentless and shares syntactic (the [\prd $\pm$] value) and semantic information with its complement, the predicative N ([\cont \ibox{1}]); this is indicated by treating \emph{be} as the value of the feature \textsc{p(reposition) form} \citep[Chapter~3]{ps}.  

Finally, we turn to an idiom: \emph{dast zadan} (lit.\ hand hit) meaning `to start'. The combination may mean, in a more recoverable way, `to touch' with PP complements denoting concrete objects (as in (\ref{GSexemple62})), or `to applaud' with a PP complement denoting a person (\ref{GSexemple81a}) (from \citealt[45]{pollet2012grammaire}). However, it means `to start' with a PP complement denoting an event as in (\ref{GSexemple81b}) (from \citealt[185]{pollet2012grammaire}).

\eal
	\label{GSexemple81} 
	\ex[]{
	\gll Bar\=a=ya\v s     xeyli dast zad-im.\\ 
		{for=\textsc{3sg}} a.lot hand {hit-\textsc{1pl}}\\
	\glt `We applauded him a lot.'}\label{GSexemple81a}
		
    \ex[]{
	\gll K\=argar-\=an       be e'tes\=ab dast zad-and.\\ 
		{worker-\textsc{pl}} to strike    hand {hit-\textsc{3pl}}\\
	\glt `The workers went on strike.'}\label{GSexemple81b}
\zl

To represent the idiom, we resort to the feature \textsc{lid} (lexical identifier) which is
associated with lexemes in the lexicon, contains semantic information and allows the
  verb to select a specific form \parencites[410--411]{Sag2007a}[127-133]{Sag2012a}. A noun or a
  verb can have a literal (\type{l-rel}) or an idiomatic content (\type{i-rel}); the verb of the
  idiom selects the second one. The noun \emph{dast} in the idiomatic complex predicate \emph{dast zadan}
  corresponds to the \type{i-dast-relation} and is selected by the idiomatic \emph{zadan}. The preposition
\emph{be}, which heads the other complement, is the same as in \emph{zadan-3}: it identifies its
content with that of its complement.

The description of \emph{zadan-4}, which occurs in the idiom \emph{dast zadan} `to start' is given in (\ref{GSexemple82}). The predicative noun complement being specified with the \textsc{lid} value \emph{dast}, it is only in combination with the noun \emph{dast} that \emph{zadan} acquires this meaning.


\ea
\label{GSexemple82}
\avm{
[\type*{zadan4-lexeme}
		cat & [ head & verb \\
		        arg-st & < NP$_k$, PP![\type{be}]!:\1, N![\textsc{prd}$+$, \textsc{lid} \type{i-dast-relation}]! > ]\\
		cont &	[\type*{soa}
				nucleus &	[\type*{i-start-relation}
							actor & k \\
							soa-arg & \1 [nucleus & event-relation ] ] ] ]
      }
\z

As usual with light verb constructions whose verb has a general meaning, the different instances of \emph{zadan} do not share a common core meaning. Rather, they are organized by similarities, analogies and metaphors, a configuration which Wittgenstein  called ``family resemblance'' (see the famous example of \emph{game} in \citealt[§66--67]{Wittgenstein2001a-u}). Nevertheless, the reader will have noticed that the four instances of \emph{zadan} discussed above have a lot in
common. The respective commonalities are factored out in a multiple inheritance hierarchy. The
details cannot be discussed here but see \crossrefchapterw[Section~\ref{prop:sec-lexicon}]{properties} and
\crossrefchapterw{lexicon} for more on the hierarchical lexicon in HPSG.
\is{semantics|)}

\section{Conclusion}\label{GSsection6}

Following the usual definition of complex predicates in HPSG as a series of (at least) two predicates, of which one is the head attracting the complements of the other, we have studied them in different languages: \ili{Romance} languages, \ili{German}, \ili{Korean} and \ili{Persian}. These languages illustrate three ways in which argument attraction (or composition) manifests itself: clitic climbing (and, more generally, bounded dependencies); flexible word order, mixing the arguments of the two predicates; and special semantic combinations, which build a lexeme out of the two predicates (particularly from the verb and the noun in light verb constructions). 

HPSG is well-equipped to model complex predicates. The feature structure associated with a predicate specifies which complements it is waiting for, and the feature structure associated with a phrase allows it to be non-saturated regarding its complements, a possibility exploited by a number of verbs which are or can be the head of a complex predicate: the phenomenon is lexically driven. Certain verbs have two entries, one which takes a saturated complement, one which is the head of a complex predicate; but a head can itself be flexible, accepting a complement which is saturated, partially saturated or not saturated at all: this is the case of the copula in \ili{Romance} languages.

Crucially, the mechanism of argument attraction is not tied to a specific syntactic structure; on
the contrary, it is compatible with different structures. We have shown that the properties of a
verbal complex (where the two predicates form a syntactic unit by themselves) differ from those of a
flat structure (where the two predicates form a unit with the complements). The structures can
characterize one language as opposed to another one (\ili{Spanish} restructuring verbs contrast with \ili{Italian}
ones), but they can also be present in the same language (as in \ili{Romanian}, for instance; see \citealt{Monachesi99b-u}). 

Similarly, the mechanism of argument attraction does not induce a specific semantic combination: it
is compatible with a compositional semantics (as in a verb + adjective combination in \ili{Persian}, or
modal verb + infinitive complement in \ili{Romance} languages), as well as a variety of senses specific to
the combination of the verb with a class of complements. 
The semantic description of complex predicates in HPSG can exploit different aspects of the formalism.
These include the hierarchical organization of the lexicon and the mechanism of conjunction of descriptions (informally referred to as unification, as with combinations specializing the meaning of the verb in \ili{Persian}); the informational richness of feature structures which include a \textsc{background} feature that a construction can impose restrictions on (as when the noun corresponds to an instrument implied in the action); and a \textsc{lid} feature which allows a particular complex predicate to point to a specific form (for representing idioms).    


\section*{Abbreviations}

%\inlinetodostefan{Stefan: Only add stuff that is not in Leipzig Glossing rules. The word particule is not \ili{English}, it seems. Add to the table below. Make sure it is complete. EZ is missing. RA as well. } % EP 25.07.2020: Taken care of.

\begin{tabularx}{.99\textwidth}{@{}lX}
\textsc{ez} & Persian suffix Ezafe \\
\textsc{conn} & connective\\
\textsc{ra} & Persian suffix \emph{rā}\\
\end{tabularx}




\section*{\acknowledgmentsUS}

We thank Anne Abeill\'e, Gabriela B\^ilb\^ie, Olivier Bonami, Caterina Donati, Han Joung-Youn, Jean-Pierre Koenig, Kil Soo Ko, Paola Monachesi, Stefan Müller, Tsuneko Nakazawa, Daniel Rojas Plata, and Stephen Wechsler. 

{\sloppy
	\printbibliography[heading=subbibliography,notkeyword=this]
}
\end{document}

%      <!-- Local IspellDict: en_US-w_accents -->
