\documentclass[output=paper,biblatex,babelshorthands,newtxmath,draftmode,colorlinks,citecolor=brown]{langscibook}
\ChapterDOI{10.5281/zenodo.5599852}

\IfFileExists{../localcommands.tex}{%hack to check whether this is being compiled as part of a collection or standalone
  \usepackage{../nomemoize}
  \usepackage{langsci-optional}
\usepackage{langsci-gb4e}
\usepackage{langsci-lgr}

\usepackage{listings}
\lstset{basicstyle=\ttfamily,tabsize=2,breaklines=true}

%added by author
% \usepackage{tipa}
\usepackage{multirow}
\graphicspath{{figures/}}
\usepackage{langsci-branding}

  
\newcommand{\sent}{\enumsentence}
\newcommand{\sents}{\eenumsentence}
\let\citeasnoun\citet

\renewcommand{\lsCoverTitleFont}[1]{\sffamily\addfontfeatures{Scale=MatchUppercase}\fontsize{44pt}{16mm}\selectfont #1}
  
  ../locallangscifixes.tex
  \togglepaper[14]
}{}

% you may switch off externalization of changed files here:
%\forestset{external/readonly}


\author{Jong-Bok Kim\affiliation{Kyung Hee University, Seoul}}
\title{Negation}


\abstract{Each language has a way to express (sentential) negation that reverses the truth value of a certain sentence, but employs language-particular expressions and grammatical strategies. There are four main types of negatives in expressing sentential negation: the adverbial negative, the morphological negative, the negative auxiliary verb, and the preverbal negative. This chapter discusses HPSG analyses for these four strategies in marking sentential negation.}


\begin{document}
\maketitle
\label{chap-negation}


\section{Modes of expressing negation}
\label{sec-modes-of-expressing-negation}

%In a typological study of sentential negation,  Dahl (1979) has
%identified three major ways of expressing negation in natural
%%languages as a morphological category on verbs, as an auxiliary
%verb, and as an adverb-like particle.

There\is{negation|(} are four main types of negative markers
in expressing negation in languages: the morphological negative,
the negative auxiliary verb, the adverbial negative, and the clitic-like
 preverbal negative (see \citealt{Dahl:79, Payne:85, Zanuttini:2001, Dryer:05}).\footnote{The term \textit{negator} or \textit{negative marker} is a cover term for any linguistic expression functioning as sentential negation.}
  %\addpages
 % these are papers (except Dryer) and hard to pin down the page numbers since the typology
 % is discussed across the papers.
Each of these types is illustrated in the following:

\eal
\ex\label{negation-1a}
\gll Ali  elmalar-i  ser-me-di-$\emptyset$. \\
Ali apples-\textsc{acc}  like-\textsc{neg}-\textsc{pst}-\textsc{3sg} \\ \hfill (\ili{Turkish})
\glt `Ali didn't like apples.'

\ex\label{negation-1b}
\gll sensayng-nim-i o-ci anh-usi-ess-ta. \\
teacher-\textsc{hon}-\textsc{nom} come-\textsc{conn} \textsc{neg}-\textsc{hon}-\textsc{pst}-\textsc{decl} \\  \hfill (\ili{Korean})
\glt `The teacher didn't come.'

\ex \label{negation-1c}
\gll Dominique (n') \'{e}crivait pas de lettre.\\
     Dominique \hphantom{(}\NEG{} wrote \textsc{neg} of letter \\ \hfill (\ili{French})
\glt `Dominique did not write a letter.'

\ex \label{negation-1d}
\gll Gianni non legge articoli di sintassi. \\
     Gianni \textsc{neg} reads articles of syntax \\ \hfill (\ili{Italian})
\glt `Gianni doesn't read syntax articles.'
\zl

\noindent
As shown in (\ref{negation-1a}), languages like \ili{Turkish}
have typical examples of morphological negatives where
negation is expressed by an inflectional category realized on the
verb by affixation. Meanwhile, languages like \ili{Korean}
 employ a negative auxiliary verb as in (\ref{negation-1b}).\footnote{\ili{Korean}
 is peculiar in that it has two ways to
 express sentential negation: a negative auxiliary (a long form
 negation)  and a morphological negative (a short form negation)
 for sentential negation. See \citet{Kim:00,Kim:16} and references therein for details.}
  The negative auxiliary
 verb here is marked with basic verbal categories such as agreement, tense, aspect, and mood, while the lexical, main verb remains in an invariant, participle form. The third major way of expressing negation is to use an adverbial
negative. This type of negation, forming an independent word, is found in
languages like \ili{English} and \ili{French}, as given in (\ref{negation-1c}). In these languages, negatives behave like adverbs in their ordering with respect to the verb.\footnote{In \ili{French}, the negator \emph{pas}
often accompanies the optional preverb clitic \emph{ne}. See \citet{Godard:2004} for detailed discussion on the uses of the clitic \emph{ne}.} The fourth
type is to introduce a preverbal negative. The negative marker in \ili{Italian} in (\ref{negation-1d}), preceding a finite verb like other types of clitics in the language,
belongs to this type.


In analyzing these four main types of sentential negation, there have been two main strands: derivational and non-derivational views. The derivational view has claimed that the positioning of all of the
four types of negatives is basically determined by the interaction of movement
operations, a rather large set of functional projections including NegP,
and their hierarchically fixed organization.
In particular, to account for the
fact that, unlike \ili{English}, only \ili{French} allows main or lexical verb inversion
as in (\ref{negation-1c}), \citet{Pollock:89,Pollock:94} and a number of subsequent researchers
have interpreted these contrasts as providing critical motivation for
the process of \isi{head movement} and the existence of functional
categories such as MoodP, TP, AgrP, and NegP (see \citealt{Belletti:90, Zanuttini:97,Chomsky:91,Chomsky:93,Lasnik:95, Haegeman:95,Haegeman:97, Vikner97a-u, Zanuttini:2001, Zeijlstra:15}).
Within the derivational view, it has thus been widely
accepted that the variation between \ili{French} and \ili{English} can be explained only in terms of the respective properties of verb movement and its interaction with a view of clause
structure organized around functional projections.

\largerpage[.8]
Departing from the derivational view, the non-derivational, lexicalist view
introduces no uniform syntactic category (e.g., Neg or NegP) for the different types of
negatives. This view allows negation to be realized in different grammatical categories, e.g., a
morphological suffix, an auxiliary verb, or an adverbial expression. For instance, the negative
\emph{not} in \ili{English} is taken to be an adverb like other negative expressions in
\ili{English} (e.g., \textit{never, barely, hardly}). This view has been suggested by
\citet[343--347]{Jackendoff:72}, \citet[\page 401]{Baker:91}, \citet{Ernst:92}, \citet[\page 91]{Kim:00},
and \citet[\page 181]{Warner2000a-u}. 
% Ernst whole paper.
In particular, \citet{KS:96}, \citet{AG:97}, \citet{Kim:00}, and \citet{KS:02} develop analyses of sentential negation in \ili{English}, \ili{French}, \ili{Korean}, and \ili{Italian} within the framework of HPSG, showing that the postulation of Neg and its projection NegP creates more empirical and theoretical problems than it solves (see \citealt{Newmeyer:2006} for this point).
%For the account of \ili{English} negation, \citet{Warner2000a-u},
%further developing the analyses of \citet{KS:96} and \citet{Kim:00},
%characterizes negation within the \ili{English} auxiliary system without the use of lexical rules, %explores inheritance hierarchies in interpreting the distributional possibilities of %negation in various environments. For instance, \citet{Warner2000a-u} classifies auxiliaries %into two subtypes with respect to negation and inversion, each of which is again %subclassified in terms of being negated and inverted.
In addition, there has been substantial work on negation in other languages within the HPSG framework, which
does not resort to the postulation of functional projections or movement operations to account for the various distributional possibilities
of negation (see \citealt{PK:99, BJ:00, Prz:00, Kupsc:02, Swart:02, Borsley:05, Crysmann:10, Bender:13}).

This chapter reviews the HPSG analyses of these four main types of negation,
focusing on the distributional possibilities of these four types of negatives in
relation to other main constituents of the sentence.\footnote{This chapter grew out of \citet{Kim:00,kim:18}.} When
necessary, the chapter also discusses implications for
the theory of grammar.
It starts with the HPSG analyses of adverbial negatives in \ili{English} and \ili{French}, which
have been most extensively studied in Transformational Grammars (Section~\ref{negation:sec-adverbial-negative}),
and then moves to the discussion of morphological
negatives (Section~\ref{negation:sec-morphological-negative}), negative auxiliary verbs
(Section~\ref{sec-negative-auxiliary-verb}), and preverbal negatives (Section~\ref{negation:sec-preverbal-negative}). The chapter
also reviews the HPSG analyses of phenomena like genitive of negation and
negative concord which are sensitive to the presence of negative expressions (Section~\ref{negation:sec-other-phenomena}). The
final section concludes this chapter.

\section{Adverbial negative}
\label{negation:sec-adverbial-negative}

\subsection{Two key factors}


\largerpage[.8]
The most extensively studied type of negation is the adverbial negative, which we find in
\ili{English} and \ili{French}.  There are two main factors that determine the position of an
adverbial negative: the finiteness of the verb and its intrinsic properties, namely whether it is an
auxiliary or a lexical verb (see \citealt[Chapter~3]{Kim:00},
\citealt{KS:02}).\footnote{\ili{German} also employs an adverbial negative \textit{nicht}, which
  behaves quite differently from the negative in \ili{English} and \ili{French}. See
  \citet[Section~11.7.1]{MuellerGT-Eng1} for a detailed review of the previous theoretical analyses
  of \ili{German} negation.}

First consider the finiteness of the lexical verb that affects
the position of adverbial
negatives in \ili{English} and \ili{French}.
\ili{English} shows us how the finiteness of a verb influences the
surface position of the adverbial negative \textit{not}:

\begin{exe}
\ex\label{negation-eng-fin-neg} \begin{xlist}
\ex[]{
Kim does not like Lee.
}
\ex[*]{
Kim not likes Lee.
}
\ex[*]{Kim likes not Lee.
}
\zl


\begin{exe}
\ex\label{negation-fr-fin-neg} \begin{xlist}
\ex[]{
Kim is believed [not [to like Mary]].
}
\ex[*]{
Kim is believed to [like not Mary].
}
\zl
%
\noindent 
As seen from the data above, the negation \textit{not} precedes an infinitive, but cannot follow
a finite lexical  verb (see \citealt[Chapter~15]{Baker:89}, \citealt{Baker:91,Ernst:92}).
\ili{French} is not different in this respect. Finiteness also affects the distributional
possibilities of the \ili{French} negative \emph{pas} (see \citealt{AG:97, KS:02, Zeijlstra:15}): 

\eal
\ex[]{
\gll Robin  (n')   aime   pas  Stacy. \\
     Robin  \hphantom{(}\NEG{} likes \NEG{} Stacy \\\hfill(French)
\glt`Robin does not like Stacy.'
}
\ex[*]{
\gll Robin ne     pas aime Stacy.\\
     Robin \NEG{} \NEG{} likes Stacy \\
}
\zl

\eal\ex[]{
\gll Ne  pas         parler    Fran\c{c}ais  est  un  grand d\'{e}savantage  en ce cas. \\
     \NEG{} \NEG{} to.speak  French  is  a great disadvantage  in this case \\
\glt `Not speaking \ili{French} is a great disadvantage in this case.'
}
\ex[*]{
\gll Ne     parler   pas    Fran\c{c}ais  est un  grand d\'{e}savantage en ce   cas.\\ 
     \NEG{} to.speak \NEG{} French        is  a   great disadvantage    in this case \\
}
\zl

\noindent
The data illustrate that the negator \textit{pas} cannot precede a finite verb,
but must follow it. But its placement with respect to
the non-finite verb is the reverse image. The negator \textit{pas}
should precede an infinitive.

\largerpage[-1]
The second important factor that determines the position of adverbial
negatives concerns the presence of an auxiliary or a lexical  verb.
Modern \ili{English} displays a clear example where this
intrinsic property of the verb influences the position of
the \ili{English} negator \textit{not}: the negator cannot follow
a finite lexical  verb, as in (\ref{negation-6a}), but when the finite verb is an auxiliary verb,
this ordering is possible, as in (\ref{negation-6b}) and (\ref{negation-6c}).

%\pagebreak
\eal
\ex[*]{
Kim left not the town.
} \label{negation-6a}
\ex[]{
Kim has not left the town.
} \label{negation-6b}
\ex[]{
Kim is not leaving the town.
} \label{negation-6c}
\zl

\noindent
The placement of \textit{pas} in \ili{French} infinitival
clauses is also affected by this intrinsic property of
the verb (\citealp[\page 355]{KS:02}):
% affects the position of the adverbial negative \textit{pas}:

\eal
\ex[]{
\gll Ne     pas          avoir de voiture dans cette ville rend la vie difficile. \\
     \NEG{} \NEG{} have a car      in   this city  make the life difficult\\
\glt `Not having a car in this city makes life difficult.'
}
\ex[] {
\gll N'     avoir pas          de voiture dans cette ville rend la  vie  difficile.\\
     \NEG{} have \NEG{} a  car     in   this  city  make the life difficult\\
\glt `Not having a car in this city makes life difficult.'
} \label{negation-28b}
\zl

\eal
\ex[]{
\gll Ne     pas \^{e}tre triste est une condition pour chanter des chansons. \\
     \NEG{} \NEG{}  be     sad     is   a condition for  singing  of songs\\
\glt `Not being sad is a condition for singing songs.'
}
\ex[]{
\gll N' \^{e}tre pas triste est   une condition pour chanter des chansons.\\
     \NEG{} be  \NEG{}   sad     is   a  condition for  singing  of songs\\
\glt `Not being sad is a condition for singing songs.'
} \label{negation-29b}
\zl

\noindent
The negator \textit{pas} can either follow or precede an infinitive
auxiliary verb, although the acceptability of the
ordering in (\ref{negation-28b}) and (\ref{negation-29b}) is restricted to certain conservative
varieties of French.

%\largerpage[-1]
In capturing the distributional behavior of such adverbial negatives in \ili{English} and
\ili{French}, as noted earlier, the derivational view (exemplified by \citealt{Pollock:89} and
\citealt{Chomsky:91}) has relied on the notion of verb movement and functional projections.  The
most appealing aspect of this view (initially at least) is that it can provide an analysis of the
systematic variation between \ili{English} and \ili{French}. By simply assuming that the two
languages have different scopes of verb movement -- in \ili{English} only auxiliary verbs move to a
higher functional projection, whereas all \ili{French} verbs undergo this process -- the
derivational view could explain why the \ili{French} negator \textit{pas} follows a finite verb,
unlike the \ili{English} negator \textit{not}.  In order for this system to succeed, nontrivial
complications are required in the basic components of the grammar, e.g., rather questionable
subtheories (see \citealt[Chapter~3]{Kim:00} and \citealt{KS:02} for detailed discussion).

Meanwhile, the non-derivational, lexicalist analyses of HPSG license all surface structures by the
system of phrase types and constraints.  That is, the\pagebreak{} position of adverbial negatives is taken to be
determined not by the respective properties of verb movement, but by their lexical properties, the
morphosyntactic (finiteness) features of the verbal head, and independently motivated Linear
Precedence (LP) constraints, as we will see in the following discussion.


\subsection{Constituent negation}

When \ili{English} \textit{not} negates an embedded constituent, it behaves
much like the negative adverb \textit{never}. The similarity between {\it
not} and \textit{never} is particularly clear in non-finite verbal
constructions (participle, infinitival, and bare verb phrases), as
illustrated in (\ref{negation-30}) and (\ref{negation-31}) (see \citealt{Klima:64,Kim:00},
\citealt[\page 199]{kimmichaelis:2020}):

\eal\label{negation-30}
\ex[]{
Kim regrets [never [having read the book]].
}
\ex[]{
We asked him [never [to try to read the book]].
}
\ex[]{
Duty made them [never [miss the weekly meeting]].
}
\zl

\eal\label{negation-31}
\ex[]{
Kim regrets [not [having read the book]].
}
\ex[]{
We asked him [not [to try to read the book]].
}
\ex[]{
Duty made them [not [miss the weekly meeting]].
}
\zl

\noindent
\ili{French} \textit{ne-pas} is no different in this regard.  \textit{Ne-pas} and
certain other adverbs precede an infinitival VP:

\eal
\ex[]{
\gll [Ne           pas  [repeindre      sa    maison]] est une n\'{e}gligence. \\
     \spacebr{}\NEG{}  \NEG{}  \spacebr{}paint one's house    is  a   negligence \\\hfill(French)
\glt `Not painting one's house is negligent.'
}
\ex[]{
\gll
[R\'{e}guli\`{e}rement   [repeindre    sa   maison]]   est  une  n\'{e}cessit\'{e}. \\
\spacebr{}regularly       \spacebr{}to.paint    one's   house    is    a   necessity \\
\glt `Regularly painting one's house is a necessity.'
}
\zl

\noindent
To capture such distributional possibilities, \citet{Kim:00} and \citet{KS:02} regard \textit{not} and \textit{ne-pas} as adverbs that modify
non-finite VPs, not as  heads of their own functional projection as in the derivational view. The
analyses view the lexical entries for \textit{ne-pas} and \textit{not} to include at
least the
information shown in (\ref{negation-c-neg}).\footnote{Here I assume that both languages
distinguish \textit{fin}(\textit{ite}) and \textit{nonfin}(\textit{ite}) verb forms, but that
certain differences exist regarding lower levels of organization. For example,
\textit{prp} (\textit{present participle}) is a subtype of \textit{fin} in \ili{French},
whereas it is a subtype of \textit{nonfin} in \ili{English}.}


\ea
\label{negation-c-neg}
% todo avm
\localvs of \emph{not} and \emph{ne-pas}:\\
\avm{
[ cat|head  [\type*{adv}
               mod  ~!\upshape VP[\type{nonfin}]!:\1\\
               pre-modifier  $+$~~~~~ ]\\
  cont      [ \type*{neg-rel}
               arg1 & \1]~~~~~~~~~~~~~~~~~~~~~~~~~~ ]
}
  % cont     & [ restr \{ [ pred & neg-rel\\
 %                        arg1 & \1 ] \} ] ]}
\z

\noindent %[JB: begins] CONT is added
The lexical information in (\ref{negation-c-neg}) specifies that
\textit{not} and \textit{ne-pas} modify a non-finite VP and that this
modified VP serves as the semantic argument of the negation.
%[JB: ends]
This simple lexical specification correctly describes the
distributional similarities between \ili{English} \textit{not} and \ili{French}
\textit{ne-pas}, as seen from the structure in Figure~\ref{negation-not-vp-mod}.

\begin{figure}
	\begin{forest}
		sm edges
		[VP
			[V\\
			\avm{
		          [head [\type*{adv}\\
                                 mod & \1  ]]
	                }
					[not/ne-pas]]
			[\ibox{1}\,VP\\
\avm{
[vform \type{nonfin}]}
					[\ldots]]]
	\end{forest}
\caption{Structure of constituent negation}\label{negation-not-vp-mod}
\end{figure}
%\fi
\noindent
The lexical specification as premodifier (\textsc{pre-modifier}+) together with an LP rule requiring
such adjuncts to precede the head they modify \crossrefchapterp[\page \pageref{lp-pre-modifier}]{order} ensures that
both \textit{ne-pas} and \textit{not} precede the VPs that they modify.
% \footnote{
% This also interacts with the LP (linear precedence) rule that specifies
% that a modifier precedes the head that it modifies.\itdopt{
% Stefan: This is not true for English. Modifiers can precede or follow the VP. You need a lexical
% specification in the modifier saying that this particular modifier is always pre-head. I discuss
% this in the chapter on order \crossrefchapterp{order}.}}  
Since the negator modifies a VP it follows that the negator does not separate an infinitival verb
from its complements, as observed from the following data (\citealp[\page 356]{KS:02}):
%\footnote{The exception to this
%generalization, namely cases where \textit{pas} follows an auxiliary
%infinitive (\textit{n'avoir pas d'argent}), is discussed in section
%5.2 below.}

\eal
\ex[] {
[Not [speaking \ili{English}]] is a disadvantage.
} \label{negation-35a}
\ex[*] {
[Speaking not \ili{English}] is a disadvantage.
} \label{negation-35b}
\zl

\eal
\ex[]{
\longexampleandlanguage{
\gll [Ne           pas  [parler  fran\c{c}ais]]  est  un grand d\'{e}savantage  en ce cas. \\
     \spacebr\NEG{}  \NEG{}  \spacebr{}to.speak French  is  a great disadvantage  in this case \\}{French}
} \label{negation-34a}
\ex[*]{
\gll [Ne  parler  pas  fran\c{c}ais]  est  un  grand d\'{e}savantage en ce cas.\\
\spacebr\NEG{}  to.speak \NEG{} French   is  a great disadvantage  in this case\\
} \label{negation-34b}
\zl

\noindent
Interacting with the LP constraints, the lexical specification in (\ref{negation-c-neg}) ensures
that the constituent negation precedes the VP it modifies. This predicts the grammaticality of
(\ref{negation-35a}) and (\ref{negation-34a}), where \textit{ne-pas} and \textit{not} are used as
VP[\textit{nonfin}] modifiers.  (\ref{negation-35b}) and (\ref{negation-34b}) are ungrammatical,
since the modifier fails to appear in the required position -- i.e., before all elements of the
non-finite VP.

The HPSG analyses sketched here have recognized the fact that finiteness plays a crucial role in
determining the distributional possibilities of negative adverbs. Its main explanatory capacity has
basically come from the proper lexical specification of these negative adverbs. The lexical
specification that \textit{pas} and \textit{not} both modify non-finite VPs has sufficed to predict
their occurrences in non-finite environments.



\subsection{Sentential negation}
\label{sec-sentential-negation}\label{negation:sec-sentential-negation}

%As just
%illustrated, the analysis of \textit{not} and \textit{ne-pas} as non-finite VP modifiers %provides a straightforward explanation for much of their distribution.
With respect to negation in finite clauses, there are important differences between \ili{English} and \ili{French}.
As I have noted earlier, it is a general fact of \ili{French} that \textit{pas} must follow a finite verb, in which case the verb optionally bears negative morphology (\textit{ne}-marking) (\citealp[\page 361]{KS:02}):

\eal
\ex[]{
\gll Dominique (n')   aime pas Alex.\\
     Dominique \hphantom{(}\NEG{} like  \NEG{} Alex\\\hfill(\ili{French})
\glt `Dominique does not like Alex.'
}
\ex[*]{
\gll Dominique pas aime Alex.\\
     Dominique \NEG{} like Alex\\
}
\zl
\noindent
In \ili{English}, \textit{not} must follow a finite
auxiliary verb, not a lexical (or main) verb:

\eal
\ex[]{
Dominique does not like Alex.
}
\ex[*]{ Dominique not does like Alex.
}
\ex[*]{ Dominique likes not Alex.
}
\zl

In contrast to its distribution in non-finite clauses, the distribution of \textit{not} in finite
clauses concerns sentential negation.  The need to distinguish between constituent and sentential
negation can be observed from many grammatical environments, including scope possibilities that one
can observe in an example like (\ref{negation-not-two})
\parencites(see)(){Klima:64}{Baker:91}{Warner2000a-u}[\page
200]{kimmichaelis:2020}.\footnote{\citet{Warner2000a-u} and \citet{BL:13} discuss scopal
  interactions of negation with auxiliaries (modals) and quantifiers within the system of Minimal
  Recursion Semantics (MRS). On MRS see also
  \crossrefchapterw[Section~\ref{semantics:sec-mrs}]{semantics}.}

\ea[]{\label{negation-not-two} The president could not approve the bill.
}
\z
%
Negation here could have the two different scope readings
paraphrased in the following:


\eal
\ex[]{
It would be possible for the president not to approve the bill.
}
\ex[]{
It would not be possible for the president to approve the bill.
}
\zl
%
The first interpretation is constituent negation; the second is
sentential negation.


The need for this distinction also comes from distributional possibilities.  The adverb
\textit{never} is a true diagnostic of a VP modifier, and I use these observed contrasts between
\textit{never} and \textit{not} to reason about what the properties of the negator \textit{not} must
be. As noted, the sentential negation cannot modify a finite VP, and is thus different from the
adverb \textit{never}:

\eal
\ex[]{
Lee never/*not left.\ \ \ \ (cf.\ Lee did not leave.)
}
\ex[]{
Lee will never/not leave.
}
\zl
%
The contrast in these two sentences shows one clear difference between \textit{never} and
\textit{not}: the negator \textit{not} cannot precede a finite VP, though it can freely occur as a
non-finite VP modifier, whereas \textit{never} can appear in both positions.

%, a
%property further illustrated by the following examples:
%
%\ees{\item John could [not [leave town]].
%
%\item John wants [not [to leave town]].}
%
%\ees{\item \bad John [not [left town]].
%
%\item \bad John [not [could
%leave town]].}

Another key difference between \textit{never} and \textit{not} can be found in
the VP ellipsis construction.  Observe the following
contrast (see \citealt{Warner2000a-u} and \citealt{KS:02}):\footnote{As
seen from an attested example like {\it I, being the size I am, could hide as one of them, whereas she could never}, in a limited context the adverb {\it never} is stranded
after a modal auxiliary, but not after a non-modal auxiliary verb like {\it be, have}
and {\it do}. Such a stranding seems to be possible when the adverb
expresses a contrastive focus meaning.}

\eal
\label{negation-vpe-not-ex}\ex[]{
Mary sang a song, but Lee never could \trace.
}
\ex[*]{
Mary sang a song, but Lee could never \trace.
}
\ex[]{
Mary sang a song, but Lee could not \trace.
}
\zl
%
\noindent The data here indicate that \textit{not} can appear
after the VP ellipsis auxiliary, but this is not possible with
\emph{never}.

%\enlargethispage{5pt}
\largerpage
We saw the lexical representation for constituent negation \textit{not} in (\ref{negation-c-neg})
above. Unlike the constituent negator, the sentential negator \textit{not} typically follows a
finite auxiliary verb. \textit{Too}, \textit{so}, and \textit{indeed} also behave like this: 

\eal
\ex[]{ Kim will not read it.  } 
\ex[]{ Kim will too/so/indeed read it.  } 
\zl

\noindent
These expressions are used to
reaffirm the truth of the sentence in question and
follow a finite auxiliary verb. This suggests that \emph{too}, \emph{so}, \emph{indeed} and the sentential \emph{not} belong to
a special class of adverbs (which I call Adv\sub{\textsc{i}}) that combine with a
preceding auxiliary verb (see \citealt[\page 94--95]{Kim:00}).
%The negator and these reaffirming expressions form
%a unit with the finite auxiliary, resulting in a lexical-level construction
 %, but shows
%different syntactic properties (while
% constituent negation need not follow an auxiliary
% as in \textit{Not eating gluten is dumb}).

Noting the properties of \emph{not} that were discussed so far,
the HPSG analyses of \citet{AG:97}, \citet[Section~3.4]{Kim:00}, and \citet{Warner2000a-u}
have taken this group of adverbs (Adv\sub{\textsc{i}}) including the sentential negation \emph{not}
to function as the complement of a finite auxiliary verb via the following lexical
rule:\footnote{The symbol $\oplus$ stands for the relation \texttt{append}\isrel{append}, i.e., a relation
  that concatenates two lists. The rule adds the adverb to the \compsl. More recent variants
    use the \argstl for valence representations. The rule can be adapted to the \argst format, but
    for the sake of readability, I stay with the \comps-based analysis.}

\ea
Adverb-Complement Lexical Rule:\\
\avm{
[\type*{fin-aux}
  synsem|loc|cat & [ head & [aux   & $+$\\
                             vform & \type{fin}]\\
                     comps & \1 ]] } $\mapsto$\\[2mm]
\hfill\avm{
[\type*{adv-comp-fin-aux}
  synsem|loc|cat|comps & < Adv\sub{I} > \+ \1 ] }
%\itdopt{This operates on \comps and the next LR operates on \argst.}
\z
%
This lexical rule specifies that when the input is a finite auxiliary verb,
the output is a finite auxiliary (\textit{fin-aux} $\mapsto$ \textit{adv-comp-fin-aux})
that selects Adv\jbsub{\textsc{i}} (including the sentential negator) as an additional
complement.\footnote{%
As discussed in the following, this type of lexical rule allows us to represent a key difference between \ili{English} and \ili{French}, namely that French has no restriction on the feature \AUX\ to introduce the negative adverb \emph{pas} as a finite verb’s complement.}  This would then
license a structure like in Figure~\ref{negation-fig:6}.

\begin{figure}
	\begin{forest}
		sm edges
		[VP\\
		\avm{
	          [\punk{head|vform}{\type{fin}}\\
		   subj  & < \1 NP >\\
		   comps & < > ]}, l sep*=1.5
			[V\\
			\avm{
			[\type*{adv-comp-fin-aux}
			head  & [ aux & $+$\\
                                  vform & \type{fin}]\\
		        subj  & < \1 >\\
		        comps & < \2, \3 > ]}
%				arg-st \; \< \@{1}\,NP{,} \@{2}\,$\textnormal{Adv}_\textnormal{I}${,} %\@{3}\,VP\>
				[could]]
			[\ibox{2} Adv\sub{I}
				[not]]
			[\ibox{3} VP
				[approve the bill,roof]]]
	\end{forest}
\caption{Structure of sentential negation}\label{negation-fig:6}
\end{figure}

\largerpage
As shown in Figure~\ref{negation-fig:6}, the finite auxiliary
verb \textit{could} combines with two complements, the negator
\textit{not} (Adv\jbsub{\textsc{i}}) and the VP \textit{approve the bill}.
This combination results in a well"=formed head"=complement phrase.
By treating \textit{not} as both a modifier (constituent negation)
and a lexical complement of a finite auxiliary (sentential negation), it is thus possible to
account for the scope differences in (\ref{negation-not-two}) with the
following two possible structures:

\eal
\label{negation-two-int}
\ex[]{
The president could [not [approve the bill]].
}
\ex[]{
The president [could] [not] [approve the bill].
}
\zl
%
In (\ref{negation-two-int}a), \textit{not} functions as a modifier to
the base VP, while  in (\ref{negation-two-int}b), whose partial structure is
given in Figure~\ref{negation-fig:6}, it is a sentential
negation serving as the complement of \emph{could}.

The present analysis allows us to have a simple account for other related phenomena,
including the VP ellipsis discussed in (\ref{negation-vpe-not-ex}). The key point
was that, unlike \textit{never}, the sentential negation can
host a VP ellipsis.  The VP ellipsis after \textit{not} is
possible, given that any VP complement of an auxiliary
verb can be unexpressed, as
specified by the following lexical rule (see \citealt[\page 99]{Kim:00} and \citealt[\page
209]{kimmichaelis:2020} for similar proposals):


\ea
\label{negation-vpe-cxt}
Predicate ellipsis lexical rule:\\
\avm{
[\type*{adv-comp-fin-aux}
 arg-st &  < \1 XP, \2 Adv\sub{I}, YP > ]}  $\mapsto$
\avm{
[\type*{aux-ellipsis-wd}
% comps  & < >\\
 arg-st & < \1, \2, YP![\type{pro}]! >]
}
%\itdobl{Stefan: 
%Type \type{neg-fin-aux} wrong?\\
%Why is COMPS empty? This does not work. It is incompatible with Figure 18.3.\\
% The specification of \type{pro} does not work, if the whole argument is shared via \ibox{3}. Either
% \type{pro} is compatible with \ibox{3}, then \ibox{3} could always be ommitted or it is not, then
% the lexical rule would fail.}
\z
%\itdgreen{JP: The rule must allow for more than 2 members of ARG-ST since, crucially, “not” must be %on it, c.f. Figure 3}
%
%
\largerpage
What the rule in (\ref{negation-vpe-cxt}) tells us is that an auxiliary verb selecting two arguments
can be projected into an elided auxiliary verb (\type{aux-ellipsis-wd}) whose third
argument is realized as a small \emph{pro}, which by definition
behaves like a slashed expression in not mapping into the syntactic grammatical
 function \comps (see \crossrefchaptert[Section~\ref{properties:lexemes-and-words}]{properties} and \crossrefchaptert[Section~\ref{argst-valence-sec}]{arg-st} for mappings from
 \argst to \comps). The YP without structure sharing is a shorthand for carrying over all
   information from the input of the lexical rule to the output with the exception of the type of
   the YP-AVM. The type at the input is \type{canonical} and the type at the output is \type{pro}.
This analysis would then license
the structure in Figure~\ref{negation-could-not}.
\begin{figure}
	\begin{forest}
		sm edges
		[VP
			[V\\
			\avm{
			[head|aux $+$\\
			 subj  < \1 >\\
		         comps < \2 >\\
			 arg-st < \1, \2, VP![\type{bse}, \type{pro}]! > ]}
					[could]]
			[\ibox{2} Adv\sub{I}\\
					[not]]]
	\end{forest}
\caption{A licensed VP ellipsis structure}\label{negation-could-not}
\end{figure}

As represented in Figure~\ref{negation-could-not}, the auxiliary verb \textit{could} forms a well-formed head"=complement phrase with \textit{not}, while its
VP[\textit{bse}] is unrealized (see \citealt{Kim:00, KS:08} for
detail). The sentential negator \textit{not} can ``survive'' VP ellipsis because it can be
licensed in the syntax as the complement of an auxiliary, independent
of the following VP.  However, an adverb like \textit{never} is only
licensed as a modifier of VP. Thus if the VP were elided, we would have the hypothetical
structure like the one in Figure~\ref{negation-fig-could-never}.
\begin{figure}
	\begin{forest}
		sm edges
		[VP
			[V{[\aux $+$]}
				[could]]
			[*VP
				[Adv{[\textsc{mod} VP]}
					[never]]]]
	\end{forest}
\caption{Ill-formed Head-Adjunct structure}\label{negation-fig-could-never}
\end{figure}
\largerpage
The adverb \textit{never} modifies a VP through the feature \textsc{mod},
which guarantees that the adverb requires the head VP that it
modifies. In an ellipsis structure, the absence of such a VP means
that there is no VP for the adverb to modify.  In other words, there
is no rule licensing such a combination -- predicting the
ungrammaticality of *\textit{has never}\is{adverb},  as opposed to \textit{has not}.


The HPSG analysis just sketched here can be easily extended to \ili{French} negation, whose
data is repeated here.

\eal
\ex[*]{
\gll Robin  ne pas aime  Stacy. \\
     Robin  \NEG{}  \NEG{}  likes  Stacy \\\hfill(\ili{French})
\glt `Robin does not like Stacy.'\label{negation-pas-good-a}
}
\ex[ ]{
\gll Robin  (n')                 aime  pas     Stacy. \\
     Robin  \hphantom{(}\NEG{} likes  \NEG{} Stacy \\
\glt `Robin does not like Stacy.'\label{negation-pas-good-b}}
\zl

\noindent
Unlike the \ili{English} negator \textit{not}, \textit{pas} must follow a finite verb. Such a
distributional contrast has motivated verb movement analyses, as mentioned above (see
\citealt{Pollock:89,Zanuttini:2001}).  By contrast, the present HPSG analysis is cast in terms of a
lexical rule that maps a finite verb into a verb with a certain adverb like \textit{pas} as an
additional complement.  The idea of converting modifiers in \ili{French} into complements has been
independently proposed by \citet{Miller92d-u} and \citet{AG:97} for \ili{French} adverbs including
\emph{pas}. Building upon this previous work, \citet{Kim:00} and \citet{AG:2002} allow the adverb
\emph{pas} to function as a syntactic complement of a finite verb in French.\footnote{Following
  \citet{AG:2002}, one could assume \textit{ne} to be an inflectional affix which can be optionally
  realized in the output of the lexical rule in Modern \ili{French}.}  This output verb
\textit{neg-fin-v} then allows the negator \textit{pas} to function as the complement of the verb
\textit{n'aime}, as represented in Figure~\ref{negation-pas-st}.

\begin{figure}
\begin{forest}
sm edges
[VP\\
 \avm{
 [\punk{head|vform}{\type{fin}}\\
  subj  & < \1 NP >\\
  comps & <> ]}, l sep*=1.5
  [V\\
   \avm{
   [\type*{neg-fin-v}\\
    \punk{head|vform}{\type{fin}}\\
    subj  & < \1 NP >  \\
    comps & < \2\textnormal{Adv}\sub{I}, \3 NP > ]}
				%arg-st \; \< \@{1}\,NP{,} \@{2}\,$\textnormal{Adv}_\textnormal{I}${,} %\@{3}\,NP\>
   [n' aime;\textsc{neg} likes]]
 [\ibox{2} Adv\sub{I}
	[pas;\textsc{neg}]]
 [\ibox{3} NP
	[Stacy;Stacy]]]
\end{forest}
\caption{Partial structure of (\ref{negation-pas-good-b})}\label{negation-pas-st}
\end{figure}

%\noindent
\largerpage[2]
The analysis also explains the position of \textit{pas} in
finite clauses. The placement of \textit{pas} before a finite verb
in (\ref{negation-pas-good-a})
 is unacceptable, since
\textit{pas} here is used not as a non-finite VP modifier, but as
a finite VP modifier. But in the present analysis which allows \textit{pas}-type negative adverbs
to serve as the complement of a finite verb,
\textit{pas} in (\ref{negation-pas-good-b}) can be the sister of the finite verb
\textit{n'aime}.
%\footnote{Of course, this
%word ordering
%conforms to the independent LP rule that a lexical head precedes
%all complements.}

Given that the imperative, subjunctive, and even present participle verb forms in \ili{French} are
finite, we can expect that \textit{pas} cannot precede any of these verb forms, which the following
examples confirm (\citealp[\page 142]{Kim:00}):

\eal
\ex[]{
\gll Si j'avais de l'argent, je n' ach\`{e}terais pas de voiture. \\
     if I.had   of money      I \NEG{} buy \NEG{} a car\\\hfill(\ili{French})
\glt `If I had money, I would not buy a car.'
}
\ex[*]{
\gll Si j'avais de l'argent, je ne pas   ach\`{e}terais de voiture.\\
     if I.had   of money      I \NEG{} \NEG{} buy     a car\\
}
\zl

\eal
\ex[]{
\gll Ne mange pas ta soupe.  \\
     \NEG{}  eat  \NEG{} your soup\\\hfill(\ili{French})
\glt `Don't eat your soup!'
}
\ex[*]{
\gll Ne pas mange ta soupe.\\
     \NEG{} \NEG{} eat your soup\\
%\glt
}
\zl

\eal
\ex[]{
\gll Il est important que vous ne r\'{e}pondiez pas. \\
     it  is important  that you \NEG{} answer \NEG{} \\\hfill(\ili{French})
\glt `It is important that you not answer.'
}
\ex[*]{
\gll Il est important que vous ne pas r\'{e}pondiez.\\
     it is   important that you \NEG{} \NEG{} answer \\
}
\zl

\eal
\ex[]{
\gll Ne parlant pas Fran\c{c}ais, Stacy avait des difficult\'{e}s. \\
     \NEG{} speaking  \NEG{} \ili{French}    Stacy   had  of  difficulties\\\hfill(\ili{French})
\glt `Not speaking \ili{French},  Stacy had difficulties.'
}
\ex[*]{
\gll Ne pas parlant Fran\c{c}ais, Stacy avait des difficult\'{e}s.\\
     \NEG{} \NEG{} speaking   \ili{French}    Stacy   had  of  difficulties\\
}
\zl

Note that this non-derivational analysis reduces the differences between
\ili{French} and \ili{English} negation to a matter of lexical properties.
The negators \textit{not} and \textit{pas} are identical in that they both are
VP[\textit{nonfin}]-modifying adverbs. But they are different with respect to
which verbs can select them as complements:  \textit{not} can be the
complement\is{adverb!as complement} of a finite auxiliary verb, whereas \textit{pas} can be the
complement of any finite verb.  So the only difference between \emph{not}
and \emph{pas} is the morphosyntactic value [\AUX\ $+$] of the verb they combine with, and this induces
the difference in the positions of the negators in \ili{English} and \ili{French}.



%
%In the non-derivational analysis sketched here, the required
%notion was the independently motivated morphosyntactic feature AUX
%(motivated from NICE constructions in \ili{English} and possibly from
%AUX-to-COMP and clitic climbing in old \ili{French}).
%Interacting with the notion of conversion, this elementary
%morphosyntactic feature has been able to capture the
%effects of the verb's intrinsic property in determining
%the positioning of the negative markers \textit{pas} and \textit{not}.



%The key fact is that the \ili{English} negative adverb \textit{not} leads a double life: one as a
%non-finite VP modifier, marking constituent negation, and the other
%as a complement of a finite auxiliary verb, marking sentential
%negation.\is{nonfinite}\is{negation} Constituent negation
%is the name for a construction where negation combines with some
%constituent to its right, and negates exactly that constituent (see Kim and Sag 2002, Kim and Sells %2008):

%
%The \ili{English} negative adverb \textit{not} leads a double life: one as a
%non-finite VP modifier, marking constituent negation, and the other
%as a complement of a finite auxiliary verb, marking sentential
%negation.\is{VP!nonfinite}\is{negation} Constituent negation
%is the name for a construction in which negation combines with some
%constituent to its right, and negates exactly that constituent.

\section{Morphological negative}
\label{negation:sec-morphological-negative}

As noted earlier, languages like \ili{Turkish} and \ili{Japanese} employ morphological negation
where the negative marker behaves like a suffix (\citealt[\page 171]{kelepir} for \ili{Turkish} and
\citealt{Kato:97,Kato:00} for \ili{Japanese}). Consider a \ili{Turkish} and a \ili{Japanese} example
respectively:

\eal
\ex
\label{negation-turkish-jap}
\gll Git-me-yece\~{g}-$\varnothing$-im \\
     go-\textsc{neg-fut-cop}-\textsc{1sg} \\\hfill(\ili{Turkish})
\glt `I will not come.'
\ex
\gll kare-wa kinoo kuruma-de ko-na-katta. \\
     he-\textsc{top} yesterday car-\textsc{inst} come-\NEG-\textsc{pst} \\\hfill(\ili{Japanese})
\glt `He did not come by car yesterday.'
\zl

\noindent
As shown by the examples, the sentential negation of \ili{Turkish} and \ili{Japanese} employ
morphological suffixes \textit{-me} and \textit{-na}, respectively.  It is possible to state the
ordering of these morphological negative markers in configurational terms by assigning an
independent syntactic status to them.  But it is too strong a claim to take the negative suffix
\textit{-me} or \textit{-na} to be an independent syntactic element, and to attribute its positional
possibilities to syntactic constraints such as verb movement and other configurational notions.  In
these languages, the negative affix acts just like other verbal inflections in numerous respects.
The morphological status of these negative markers is supported by their participation in
morphophonemic alternations.  For example, the vowel of the \ili{Turkish} negative suffix
\textit{-me} shifts from open to closed when followed by the future suffix, as in
\textit{gel-mi-yecke} `come-\NEG-\FUT'.  Their strictly fixed position also indicates their
morphological constituenthood. Though these languages allow a rather free permutation of syntactic
elements (scrambling), there exist strict ordering restrictions among verbal suffixes including the
negative suffix, as observed in the following:

\eal
\ex
\gll tabe-sase-na-i/*tabe-na-sase-i \\
     eat-\textsc{caus}-\NEG-\textsc{npst}/eat--\NEG-\textsc{caus}-\textsc{npst} \\\hfill(\ili{Japanese})

\ex
\gll tabe-rare-na-katta/*tabe-na-rare-katta \\
     eat-\textsc{pass}-\NEG-\textsc{pst}/eat-\NEG-\textsc{pass}-\textsc{pst} \\

\ex
\gll tabe-sase-rare-na-katta/*tabe-sase-na-rare-katta \\
     eat-\textsc{caus}-\textsc{pass}-\NEG-\textsc{pst}/eat-\textsc{caus}-\NEG-\textsc{pass}-\textsc{pst}\\
\zl

\noindent
The strict ordering of the negative affix here is a matter of morphology.
If it were a syntactic concern, then
the question would arise as to why
there is an obvious contrast in the ordering principles
of morphological and syntactic constituents, i.e., why the ordering
rules of morphology are distinct from the ordering rules of syntax. The
simplest explanation for this contrast is to accept
the view that morphological constituents including the negative marker
are formed in the lexical component and hence have no syntactic
status (see \citealt[Chapter~2]{Kim:00} for detailed discussion).

Given these observations, it is more reasonable to assume that the placement of a
negative affix is regulated by morphological principles, i.e., by
the properties of the morphological negative affix itself.
The process of adding a negative morpheme to a lexeme can be modeled
straightforwardly by the following lexical rule (for a similar treatment see \cites[36]{Kim:00}[111--112]{Crowgey:12}):
\ea
\label{lr-neg-word-formation}
Negative word formation lexical rule:\\
\avm{
[\type*{v-lxm}
 phon < \1 >\\
 synsem|loc|cont \2 &]} $\mapsto$
\avm{
[\type*{neg-v-lxm}
 phon < \normalfont !f$_{\mathit{neg}}$(\1)! >\\
 synsem|loc [ cat|head|pol \type{neg}\\
              cont [ \type*{neg-rel}
                     arg1 & \2 ]~~~~~]]
% todo avm
 %             cont|restr \{ [ pred & neg-rel\\
 %                             arg1 & \2 ] \} ] & ]
}
%\itdopt{Stefan: Is neg-v-lxm a subtype of v-lxm? If so, what prevents iterative application of the LR?}
\z
%
%
As shown here, any verb lexeme can be turned into a verb with the negative
morpheme attached. That is, the language-particular definition for
F\jbsub{\type{neg}} will ensure that an appropriate
negative morpheme is attached to the lexeme. For instance, the
suffix \suffix{ma} for \ili{Turkish} and \suffix{na} for \ili{Japanese} will be attached to the verb
lexeme, generating the verb forms in (\ref{negation-turkish-jap}).\footnote{In a similar
manner, \citet{PK:99} and \citet{Prz:00, Prz:01}
discuss aspects of \ili{Polish} negation, which is realized as the prefix
  \emph{nie} to a verbal expression.} See \crossrefchapterw{morphology} for details on how the
realization of inflectional features is modeled in HPSG.

This morphological analysis can be extended to the negation of languages
like Libyan Arabic\il{Arabic!Libyan}, as discussed in \citet{BK:12}. The language
has a bipartite realization of negation, the proclitic \emph{ma-} and the enclitic \emph{-\u{s}}:
%\itdgreen{JP: If ma- and -s are clitics, the appropriate boundary symbol is “=“. But given what you
%say below, I would simply call them prefix and suffix. Otherwise it’s confusing.}
% Stefan: Miller Sag treat clitics as affixes and this is also discussed in Borsley & Krer's work.

\ea
\gll la-wlaad ma-m\u{s}uu-\u{s} li-l-madrsa. \\
     the-boys \NEG-go.\pst.3.\pl-\NEG{} to-the-school\\\hfill(Libyan Arabic\il{Arabic!Libyan})
\glt `The boys didn't go to the school.'
\z
%
\largerpage[2]
Following \citet[\page 10]{BK:12}, one can treat these clitics as affixes and generate
a negative word. Given that the function f\sub{\type{neg}} in Libyan Arabic\il{Arabic!Libyan} allows
the attachment of the negative prefix \textit{ma-} and the suffix -\textit{\u{s}} to the verb
stem \emph{m\u{s}uu}, we would have the following output in accordance
with the lexical rule in (\ref{lr-neg-word-formation}):\footnote{%
   \citet{BK:12} note that the suffix -\textit{\u{s}} is not realized when a negative clause
   includes an n-word or an NPI (negative polarity item). See \citet{BK:12} for further details.}
%
%
%%
% The formulation given in
%\citet{BK:12} is slightly different from the one given here, but both
% have the same effects.}


\ea
%\avm{
%[\type*{v-lxm}
% phon <\normalfont{m\u{s}uu}>\\
% synsem|loc|cont \2 &]} $\mapsto$
\avm{
[\type*{neg-v-lxm}
 phon <\normalfont{ma-m\u{s}uu-\u{s}}>\\
 synsem|loc [ cat|head|pol \type{neg}\\
              cont \type{neg-rel}~~~~~~~~~ ] ]
% todo avm
}
\z

The lexicalist HPSG analyses sketched here have been built upon the
thesis that autonomous (i.e., non-syntactic) principles govern the
distribution of morphological elements \citep{BM:95}.
The position of the morphological negation is simply
defined in relation to
the verb stem it attaches to. There are no syntactic operations such
as head-movement or multiple functional projections in forming
a verb with the negative marker.
% \footnote{The lexical rule-based
% and construction-based approaches are alike in introducing unary rules
% from an input (daughter) to an output (mother). See, among others, 
% \citet{Koenig99a}, \citet{Sag:12}, \citet{Crysmann:Bonami:2016}, and  
% \citet{Hilpert:16}.
% \itdblue{JP: It is a little odd to contrast the lexical-rule based approach and constructional
%   approach to morphology in HPSG. For one thing, lexical rules are unary rules (constructions) in
%   HPSG since the early 2000s, for another Riehemann (1998) and Koenig (1999) are already
%   constructional approaches of morphology within HPSG and so is, more recently, IbM for inflection
%   (Cryssman and Bonami 2016). These are better developed and older ... I would reference this work,
%   but up to you.\newline Stefan: Yes, JP is right.}}



\section{Negative auxiliary verb}
\label{sec-negative-auxiliary-verb}

\largerpage[2]
Another way of expressing sentential negation, as noted earlier, is to employ
a negative auxiliary
verb. Some head-final languages like \ili{Korean} and \ili{Hindi} employ
negative auxiliary verbs. Consider a \ili{Korean} example:

\ea
\gll John-un ku chayk-ul ilk-ci anh-ass-ta. \\
     John-\textsc{top} that book-\textsc{acc} read-\textsc{conn} \textsc{neg}-\textsc{pst}-\textsc{decl}  \\\hfill(Korean)
\glt `John did not read the book.'
\z
%
%\ex
%\gll anil  kitaab\~{e}  nah\~{\i}\~{\i}  becegaa.\\
%     Anil-\textsc{nom} book-\textsc{pl}  not sell-\textsc{fu}
%\glt `Anil will not sell the books.'
%\zl
%
\noindent
The negative auxiliary in head-final languages like \ili{Korean}
typically appears clause"=finally, following the invariant form of the lexical verb.
In head"=initial SVO languages, however, the negative auxiliary
almost invariably occurs immediately before the lexical verb
\citep[212]{Payne:85}. \ili{Finnish} also exhibits this property \citep[\page 376]{Mitchell:91}:

\ea
\gll Min\"{a} e-n puhu-isi. \\
     I.\textsc{nom} \textsc{neg}-\textsc{1sg} speak-\textsc{cond} \\\hfill(Finish)
\glt `I would not speak.'
\z

\noindent
These negative auxiliaries have syntactic status: they can be inflected, above all. Like other
verbs, they can also be marked with verbal inflections such as agreement, tense, and mood.

In dealing with negative auxiliary constructions, most of the derivational approaches have followed
Pollock's and Chomsky's analyses in factoring out grammatical information (such as tense, agreement,
and mood) carried by lexical items into various different phrase-structure nodes (see, among others,
\citealt{Hagstrom:02}, \citealt{Han:07} for \ili{Korean}, and \citealt{Vasishth:00} for
\ili{Hindi}).  This derivational view has been appealing in that the configurational structure for
\ili{English}-type languages could be applied even for languages with different types of
negation. However, issues arise about how to address the grammatical properties of negative
auxiliaries, which are quite different from the other negative forms.

The \ili{Korean} negative auxiliary displays all the key properties of auxiliary verbs in the
language. For instance, both the canonical auxiliary verbs and the negative auxiliary alike require
the preceding lexical verb to be marked with a specific verb form (\vform), as illustrated in the
following:

\eal
\ex\label{negation-14a}
\gll ilk-ko/*-ci siph-ta. \\
     read-\textsc{conn}/\textsc{conn} would.like-\textsc{decl} \\\hfill(Korean)
\glt `(I) would like to read.'

\ex\label{negation-14b}
\gll ilk-ci anh-ass-ta. \\
     read-\textsc{conn} \NEG-\textsc{pst}-\textsc{decl} \\
\glt `(I) did not read.'
\zl
\noindent
The auxiliary verb \textit{siph-} in (\ref{negation-14a}) requires a
\textit{-ko}-marked lexical verb, while the negative auxiliary
 verb \textit{anh-} in (\ref{negation-14b}) asks for a \textit{-ci}-marked lexical
 verb. This shows that the negative is also an auxiliary verb in the language.

In terms of syntactic structure, there
are two possible analyses.  One is to assume that the negative auxiliary takes a VP complement and the other is to claim that it forms a verb complex with
an immediately preceding lexical verb, as represented in Figures~\ref{negation-fig:3a} and~\ref{negation-fig:3b}, respectively
\citep{Chung98a-u, Kim:16}.
%\footnote{Another possibility is to assume that the
%auxiliary verb are in the sisterhood relationship with the following
%lexical verb and its putative complement(s). For this option, see ???}
\begin{figure}
	\begin{subfigure}[b]{0.48\textwidth}
\centering
		\begin{forest}
%		sm edges
			[VP
				[VP
					[ \dots\ ]
					[V{[\textsc{vform} \textit{ci}]}
					]
					]
				[V{[\textsc{aux $+$}]}
					[anh-ta\\ \textsc{neg-decl}]
				]
			]	
		\end{forest}
	\caption{VP structure}\label{negation-fig:3a}
		\end{subfigure}	
\hfill
	\begin{subfigure}[b]{0.48\textwidth}
\centering
		\begin{forest}
%		sm edges
			[VP, s sep=1cm
				[ \dots\ ]
				[V
					[V{[\textsc{vform} \textit{ci}]}
						[\dots]]
					[V{[\textsc{aux $+$}]}
						[anh-ta\\ \textsc{neg-decl}]]]]
		\end{forest}
	\caption{Verb-complex structure}\label{negation-fig:3b}	
		\end{subfigure}
	\caption{Two possible structures for the negative auxiliary construction}
\end{figure}

%\noindent
The distributional properties of the negative auxiliary in the language support
 a complex predicate structure (cf.\ Figure~\ref{negation-fig:3b}) in which the negative auxiliary verb
forms a syntactic/semantic unit with the preceding lexical verb.
For instance, no adverbial expression, including
a parenthetical adverb, can intervene between
the main and the auxiliary verb, as illustrated by the
following:

\ea
\longexampleandlanguage{
\gll Mimi-nun          (yehathun)           tosi-lul          (yehathun)           ttena-ci            (*yehathun) anh-ass-ta. \\
     Mimi-\textsc{top} \hphantom{(}anyway city-\textsc{acc} \hphantom{(}anyway leave-\textsc{conn} \hphantom{(*}anyway \NEG-\textsc{pst}-\textsc{decl} \\}{Korean}
\glt `Anyway, Mimi didn't leave the city.'
\z
%
Further, in an elliptical construction, the elements of a verb complex
 always occur together. Neither the lexical  verb (\ref{negation-fragment}c) nor the
 auxiliary verb alone (\ref{negation-fragment}d) can serve
as a fragment answer to the corresponding polar question:
% The two verbs
%must occur together.

\eal
\label{negation-fragment}
\ex[]{
\gll Kim-i hakkyo-eyse pelsse tolawa-ss-ni? \\
     Kim-\textsc{nom} school-\textsc{src} already return-\textsc{pst}-\textsc{que} \\\hfill(Korean)
\glt `Did Kim return from school already?'
}
\ex[]{
\gll ka-ci-to anh-ass-e.\\
     go-\textsc{conn}-\textsc{del} \textsc{not}-\textsc{pst}-\textsc{decl} \\
\glt `(He) didn't even go.'
}
\ex[*]{
\gll ka-ci-to.\\
     go-\textsc{conn}-\textsc{del} \\
}
\ex[*]{
\gll anh-ass-e. \\
\NEG-\textsc{pst}-\textsc{decl}\\
}
\zl

\noindent
The lexical verb and the auxiliary must appear together as in (\ref{negation-fragment}b). These
constituenthood properties indicate that the negative auxiliary forms a syntactic unit with a
preceding lexical  verb in \ili{Korean}. 

To address these complex verb properties, one could assume that
an auxiliary verb forms a complex predicate, licensed by
the following schema (see \citealt[95]{Kim:16}):

\ea
\label{negation-hd-lex-cxt}
\head-\LIGHT Schema\is{schema!Head-Cluster}:\\*
\avm{
[\type*{head-light-phrase}
 comps & \1\\
 light & $+$\\
 head-dtr & \2\\
 dtrs     & < \3 [light & $+$], \2 [comps & \1 \+ < \3 >\\
                                    light & $+$ ] > ]}
\z
% \itdblue{JP: Since the syntactic schema already references the COMPS of the lexical complement (1),
%   I don’t see the point of adding (1) to the head. In fact, it would be more strictly
%   syntactic/constructional argument composition if (1) was not part of the COMPS value of the
%   head-daughter. What do you think?}
% \itdopt{Stefan: As it is now it cannot deal with the so-called remote passive. Since the lexical
% approach can do this and since this does not cause any problem for Korean, one should go for the
% lexical argument attraction approach. If I remember correctly, JB had the syntactic approach for the
% only reason that the LKB system does not have relational constraints and hence no append.}

\noindent
This construction schema means that a \LIGHT\ head expression combines with a \LIGHT\ complement,
yielding a light, quasi-lexical constituent \citep{BW:13}.  When this combination happens, there is
a kind of argument composition: the \comps value of this lexical complement is passed up to the
resulting mother.  The constructional constraint thus induces the effect of argument composition in
syntax, as illustrated by Figure~\ref{kor-v-complex}.
\begin{figure}
\begin{forest}
		[V\\
		\avm{
			[\type*{head-light-phrase}\\
			 head  & \1\\
			 comps & \2\\
			 light & $+$ ] }, l sep*=3
			[\ibox{3}\,V\\
			\avm{
				[\punk{head|vform}{ci}\\
				 comps  & \2 < NP >\\
                                 light  & $+$]}, edge label={node[midway,left,outer sep=1.5mm,]{Lexical arg.}}
				[ilk-ci\\read-\textsc{conn},tier=word]]
			[V\\
			\avm{
				[head  & \1\\
				 comps & \2 \+ < \3 > ]}, edge label={node[midway,right,outer sep=1.5mm,]{H}}
					[anh-ass-ta\\ \textsc{neg-pst-decl},tier=word]]]
\end{forest}
\caption{An example structure licensed by the \hdlight}\label{kor-v-complex}
\end{figure}
The auxiliary verb \textit{anh-ass-ta} `\NEG-\PST-\DECL' combines with the matrix verb \textit{ilk-ci} `read-\conn',
creating a well-formed \type{head-light-phrase}.
Note that the resulting construction inherits the
\COMPS\ value from that of the lexical complement \textit{ilk-ci} `read-\conn' in accordance with the structure-sharing
imposed by the \hdlight\
in (\ref{negation-hd-lex-cxt}). That is, the \hdlight\ licenses
the combination of an auxiliary verb with its lexical verb, while
inheriting the lexical verb's complement value through argument composition.
%\itdblue{JP: See my remark above.}
The present system thus allows argument composition at the syntax level, rather than in the lexicon.

\largerpage[1]
The HPSG analysis I have outlined has taken the negative auxiliary in \ili{Korean}
to select a lexical verb, the resulting combination forming a verbal complex. The present analysis
implies that there is no upper limit for the number of auxiliary verbs to occur in sequence, as long
as each combination observes the morphosyntactic constraint on the preceding auxiliary expression. Consider
the following:

\eal
\ex
\gll Sakwa-lul          [mek-ci anh-ta]. \\
     apple-\textsc{acc} \spacebr{}eat-\textsc{conn} \NEG-\textsc{decl} \\\hfill(Korean)
\glt `(I/he/she) do/does not eat the apple.'
%\itd{Stefan: Are theses full sentences? If so start with capital word and end with dot.}
%JBK: They are both full sentences and phrases.
\ex
\gll Sakwa-lul          [[mek-ko siph-ci] anh-ta]. \\
     apple-\textsc{acc} \hphantom{[[}eat-\textsc{conn} wish-\textsc{conn} \NEG-\textsc{decl} \\
\glt `(I/he/she) would not like to eat the apple.'
%
\ex \label{negation-20c}
\gll Sakwa-lul          [[[mek-ko siph-e] ha-ci] anh-ta]. \\
     apple-\textsc{acc} \hphantom{[[[}eat-\textsc{conn} wish-\textsc{conn} do-\textsc{conn} \NEG-\textsc{decl} \\
\glt `(I/he/she) do/does not like to eat the apple.'
%
\ex
\gll Sakwa-lul          [[[[mek-ko siph-e] ha-key] toy-ci] anh-ta]. \\
     apple-\textsc{acc} \hphantom{[[[[}eat-\textsc{conn} wish-\textsc{conn} do-\textsc{conn} become-\textsc{conn} \NEG-\textsc{decl} \\
\glt Literally: `(I/he/she) do/does not become to like to eat the apple.'
\zl
%
\largerpage[1]
As seen from the bracketed structures, it is possible to add one more auxiliary verb to
an existing \textsc{head-light} phrase with the final auxiliary bearing an appropriate
  connective marker. There is no upper limit to the possible number  of auxiliary
  verbs one can add (see \citealt[\page 88]{Kim:16} for detailed discussion).

The present analysis in which the  negative auxiliary forms a complex
predicate structure with a lexical verb can also be applied to languages
like \ili{Basque}, as suggested by \citet{CB:11}. They explore the interplay of sentential
negation and word order in \ili{Basque}. Consider their example (p.\,51):

\ea
\label{negation-basque-ex}
\gll ez-ditu irakurri liburuak \\
     \NEG-3\textsc{plo}.\textsc{prs}.\textsc{3sgs} read.\PRF{} book.\ABS.\pl\\\hfill(Basque)
\glt `has not read books'
\z
%
%
Unlike \ili{Korean}, the negative auxiliary \textit{ez-ditu} precedes
the main verb. Other than this ordering difference, just
like \ili{Korean}, the two form a verb complex structure, as represented in
Figure~\ref{negation-basque}.

\begin{figure}
\begin{forest}
sm edges without translation
[V\\
 \avm{
		[\type*{head-light-phrase}
		%head \; \3\\
		comps & \2\\
                light & $+$] }, l sep*=3
		  [V\\
		   \avm{
		   [\punk{head|aux}{$+$}\\
	            comps & < \1 > \+ \2\\
                    light & $+$]}
			[ez-ditu\\ \NEG-3\textsc{plo.pres.3sgs}]]
			[{\ibox{1}\,V\\
			  \avm{
			  [comps & \2 < NP > ]  }}
					[irakurri\\read.\textsc{prf}]]]
\end{forest}
\caption{Negation verb combination in Basque adapted from \citew[\page 51]{CB:11}}\label{negation-basque}
% \itdblue{JP: And still the same remark as above, the argument attraction, (2) here, does nothing
%   and, IMHO, should be dropped.\\
% Stefan: Argument attraction is important for remote passive. The attracted argument should be before
% the selected verb, since Korean is verb-final.}
\end{figure}
%
%
%
In the treatment of negative auxiliary verbs, HPSG analyses
have taken the negative auxiliary to be an independent lexical
verb whose grammatical (syntactic) information is not distributed
over different phrase structure nodes, but rather is incorporated into
its precise lexical specifications. In particular, the negative
auxiliary forms in many languages a verb complex structure whose
constituenthood is motivated by independent phenomena.


\section{Preverbal negative}
\label{negation:sec-preverbal-negative}

The final type of sentence negation is preverbal negatives, which
we can observe in languages like \ili{Italian} and \ili{Welsh}:

\eal
\ex \label{negation-position-1a}
\gll Gianni non telefona a nessuno.\\
     Gianni \NEG{} telephones to nobody\\ \hfill (\ili{Italian}, \citealt[\page 62]{Borsley:06})
\glt`Gianni does not call anyone.'
%\ex \label{negation-position-1b}
%\gll Jag har inte gett boken till henne. (\ili{Swedish})
%I have \NEG{} given the.book to her
%\glt `I hae not given the book to her.'
\ex \label{negation-position-1c}
\gll Dw i ddim wedi gweld neb.\\
     am I \NEG{} \textsc{prf} see nobody\\ \hfill  (\ili{Welsh}, \citealt[\page 108]{Borsley:05})
\glt `I haven't seen anybody.'
\zl
%
%
As seen here,
the \ili{Italian} preverbal negative \textit{non} -- also called negative particle or
clitic -- always precedes a lexical  verb, whether finite or
non-finite, as further attested by the following
examples (\citealp[Chapter~4]{Kim:00}):
%
%\ex[]{
%\gll
%Gianni non legge articoli di sintassi. \\
%Gianni   \NEG{}   reads  articles  of  syntax \\
%\glt`Gianni doesn't read syntax articles.'
%}
\eal
\ex[]{
\gll Gianni  vuole  che  io  non  legga  articoli  di  sintassi. \\
     Gianni  wants  that  I   \NEG{}   read  articles  of  syntax \\\hfill(\ili{Italian})
\glt `Gianni hopes that I do not read syntax articles.'
}
\ex[]{
\gll Non   leggere  articoli di sintassi   \`{e}  un vero peccato. \\
     \NEG{}  to.read  articles of syntax   is  a real shame \\
\glt `Not to read syntax articles is a real shame.'}
%
%
\ex[]{
\gll Non    leggendo  articoli di sintassi,  Gianni  trova  la linguistica  noiosa.\\
\NEG{}   reading articles of syntax  Gianni  finds  {} linguistics  boring\\
\glt `Not reading syntax articles, Gianni finds linguistics boring.'}
\zl
%
%
The derivational view again attributes the distribution of such a preverbal negative to the reflex
of verb movement and functional projections (see \citealt[Chapter~1]{Belletti:90}). This line of
analysis also appears to be persuasive in that the different scope of verb movement application
could explain the observed variations among typologically related languages. Such an analysis,
however, fails to capture unique properties of the preverbal negative in contrast to the
morphological negative, the negative auxiliary, and the adverbial negative.

\citet{Kim:00} offers an HPSG analysis of \ili{Italian} and \ili{Spanish} negation.
His analysis takes \textit{non}
to be an independent lexical head, even though it is a clitic.
This claim follows the  analyses sketched by \citet{Monachesi:93} and \citet{Monachesi:98},
which assume that there are two types of clitics: affix-like
clitics and word-like clitics. Pronominal clitics belong to the
former, whereas the clitic \textit{loro} `to them' belongs to the
latter. Kim's analysis suggests that \textit{non} also belongs
to the latter group.\footnote{One main difference between
\textit{non} and \textit{loro} is that \textit{non} is a head, whereas \textit{loro} is a complement XP. See
\citet{Monachesi:98} for further discussion of the
behavior of \textit{loro} and its treatment.}
%One key difference from
%pronominal clitics is thus that \textit{non} functions as an independent word.
Treating \textit{non} as a word-like element, as in the following, will allow us to capture its word-like
properties, such as the possibility of it bearing stress and
its separation from the first verbal element. However, it is not a
phrasal modifier, but an independent particle (or clitic) which combines with
the following lexical  verb (see \citealt{Kim:00} for
detailed discussion).

\ea
\label{negation-non}
Lexical specifications for \textit{non} in \ili{Italian}:\\
\avm{
[ phon \phonliste{ non }\\
  synsem|loc [ cat  & [ head \1\\
                        comps < V[ head & \1\\
                                   comps & \2\\
                                   cont  & \3] > \+ \2 ]\\
               cont & [ \type*{neg-rel}
                        arg1 & \3 ] ]]
}
%% \begin{avm}
%% \[form\ \q<\normalfont non\q>\\
%%   \SYN\ \; \[head\ \@1\\
%%          \COMPS\ \; \<V\[head\ \ \; \@1\\
%%                      \COMPS\ \ \; \textit{L}\]\> \ \; $\oplus$ \; \ \textit{L}
%%          \]\\
%%   \SEM\ \; \[\textsc{restr} \; \<\[\PRED\ \ \; \type{neg-rel}\]\>\]\]
%%   \end{avm}
\z
%
\noindent
This lexical entry roughly corresponds to the entry for
\ili{Italian} auxiliary verbs (and restructuring verbs with clitic climbing),
in that the negator \textit{non} selects a verbal complement and, further, that verb's
complement list. One key property of \textit{non}
is its \textsc{head} value: this value is in a sense undetermined, but structure-shared with the \textsc{head} value of its verbal complement.
The value is thus
determined by what it combines with. When \textit{non} combines with a finite
verb, it will be a finite verb, and when it combines with an infinitival verb, it will be a
non-finite verb.

In order to see how
this system works, let us consider an \ili{Italian} example where
the negator combines with a transitive verb as in (\ref{negation-1d}),
repeated here as (\ref{negation-read-it}):

\begin{exe}
\ex
\label{negation-read-it}
\gll Gianni non    legge articoli di sintassi.\\
     Gianni \NEG{} reads articles of syntax\\\hfill(\ili{Italian})
\glt `Gianni doesn't read syntax articles.'
\end{exe}

\noindent
When the negator \textit{non} combines with the finite verb \textit{legge} `reads' that
selects an NP object, the resulting combination will form
the verb complex structure given in Figure~\ref{negation-read-it-st}.

\begin{figure}
\begin{forest}
sm edges without translation
[V\\
 \avm{
  [\type*{head-light-phrase}\\
   head  & \1\\
   light & $+$\\
   comps & \2 ]}, l sep*=3
     [V\\
      \avm{
      [head  & \1\\
       light & $+$\\
       comps & < \3 > \+ \2  ]}
	 [non\\ \textsc{neg}]]
     [\ibox{3}\,V\\
      \avm{
      [head  & \1\\
       comps & \2 < NP > ]}
	[legge\\ reads]]]
\end{forest}
\caption{Verb complex structure of (\ref{negation-read-it})}\label{negation-read-it-st}
%\itdopt{Stefan: Remove material after \ibox{2} in \emph{non} and in mother node.}
\end{figure}

\citet{Borsley:06}, adopting \citegen{Kathol2000a} topological approach, provides a
linearization"=based HPSG approach to capturing the distributional possibilities of negation in
\ili{Italian} and \ili{Welsh}, which we have seen in (\ref{negation-position-1a}) and
(\ref{negation-position-1c}), respectively.  Different from \citegen{Borsley:05} selectional
approach where a negative expression selects its own complement, Borsley's linearization"=based
approach allows the negative expression to have a specified topological field.  For instance,
\citet[\page 79]{Borsley:06}, accepting the analysis of \citet{Kim:00} where \textit{non} is taken to be a
type of clitic-auxiliary, posits the following order domain:


\ea
\avm{
[ dom < [\type*{first}
         \phonliste{ Gianni } ],
        [\type*{second}
          neg $+$\\
          \phonliste{ non } ],
        [\type*{third}
         \phonliste{ telephona } ],
        [\type*{third}
         neg $+$\\
         \phonliste{ a nessuno } ] >]}
\z
%
With this ordering domain, \citet{Borsley:06} postulates
that the \ili{Italian} sentential negator \emph{non} bearing the positive \textsc{neg} feature is in the second field.\footnote{
\citet{Borsley:06} also notes that \ili{Italian} negative expressions like \emph{nessuno} `nobody' also bear the feature \textsc{neg}
but are required to be in the third field.}
The analysis then can attribute the distributional differences between \ili{Italian} and \ili{Welsh} negators
by referring to the difference in their domain value. That is,
in Borsley's analysis, the \ili{Welsh} \textsc{neg} expression \emph{ddim}, unlike \ili{Italian} \emph{non},
is required to be in the third field, as illustrated in the following domain for the sentence (\ref{negation-position-1c}) (from
\citealt[\page 76]{Borsley:06}):\footnote{Different from \citet{Borsley:06}, \citet{BJ:00} offer  a selectional analysis of \ili{Welsh} negation.
That is, the finite negative verb selects
two complements (e.g., subject and object) while
the non-finite negative verb selects a VP. See \citet{BJ:00} for details.}

\ea
\avm{
[ dom < [\type*{second}
         \phonliste{ dw } ],
        [\type*{third}
         \phonliste{ i } ],
        [\type*{third}
         neg $+$\\
         \phonliste{ ddim } ],
        [\type*{third}
         \phonliste{ wedi gweld neb }  ] > ]}
\z
As such,  with the assumption that constituents have an order domain to which ordering
constraints apply, the topological approach enables us to capture the complex distributional
behavior of the negators in \ili{Italian} and \ili{Welsh}.

\section{Other related phenomena}
\label{negation:sec-other-phenomena}

In addition to this work focusing on the distributional possibilities of negation, there has also
been HPSG work on genitive of negation and negative concord.

\citet{Prz:00} offers an HPSG analysis for the non-local genitive of negation in \ili{Polish}.  In
\ili{Polish}, negation is realized as the prefix \emph{nie} to a verbal expression (see
\citealt{PK:99, Prz:00, Prz:01}), and \ili{Polish} allows the object argument to be genitive-marked
when the negative marker is present, as in (\ref{negation-49b}).  The assignment of genitive case
to the object need not be local as shown in (\ref{negation-50b}) (data from \citealt[\page
120]{Prz:00}):

\eal
\ex  \label{negation-genitive}
\gll Lubi\c{e} Mari\c{e} \\
     like.1\textsc{sg} Mary.\textsc{acc}\\\hfill(\ili{Polish})
\glt `I like Mary.'
\ex \label{negation-49b}
\gll Nie lubi\c{e} Marii / * Mari\c{e} \\
     \NEG{} like.1\textsc{sg} Mary.\textsc{gen} {} {} Mary.\textsc{acc}\\
\glt `I don't like Mary.'
\zl

\eal
\ex \label{negation-genitive-1}
\gll  Janek wydawa\l{} si\c{e} lubi\'{c} Mari\c{e}.\\
      John seemed \textsc{rm}     like.\textsc{inf} Mary.\textsc{acc}\\\hfill(\ili{Polish})
\glt `John seemed to like Mary.'
\ex \label{negation-50b}
\gll  Janek nie wydawa\l{} si\c{e} lubi\'{c} Marii / Mari\c{e}.\\
      John \NEG{} seemed \textsc{rm} like.\textsc{inf}      Mary.\textsc{gen} {} Mary.\textsc{acc}\\
\glt `John did not seem to like Mary.'
\zl

\noindent
To account for this kind of phenomenon, \citet{Prz:00}
suggests that the combination of the
negative morpheme \emph{nie} with a verb stem introduces the
feature \textsc{neg}.
% \itd{Stefan: You talk about the morpheme \emph{nie} and the prefix \emph{nie}. It is written
%   separtely. If it was a prefix one would expect \prefix{nie}.}
With this lexical specification, his analysis introduces
 the following principle (adapted from \citealt[\page 143]{Prz:00}):

\ea
\label{negation-polish-gen-case}
Part of the Case Principle\is{principle!Case} for \ili{Polish}:\\
\avm{
[ head & [\type*{verb}
          neg & $+$ ]\\
  arg-st & \1 \type{nelist} \+ < ![case \type{str}]! > \+ \2  ]}  \impl
\avm{
[arg-st \1 \+ < ![case \type{sgen}]! > \+ \2  ]}
\z
The principle allows a \textsc{neg}+ verbal expression to assign
 the \textsc{case} value \textit{gen} to all
%\itdopt{Stefan: Is there just one? What if there are multiple arguments with structural case?} 
% I checked. Adam explains that it is all arguments.
non-initial arguments.
 % ensures that
 % the object NP in (\ref{negation-genitive}b) is \textsc{gen}-marked
%
This is why the negative word \emph{nie} triggers the object complement of
(\ref{negation-genitive}) to be \GEN-marked.
As for the non-local genitive in (\ref{negation-genitive-1}), \citet[\page 145]{Prz:00}
allows the verbal complement of a raising verb like \type{seem} to optionally undergo lexical
argument composition. This process yields the following output for the
matrix verb in (\ref{negation-50b}):

\ea
\label{negation-polish-case}
%\textsc{polish case assignment Rule:\\
Representation for \emph{nie wydawał siȩ} `did not seem' when combined with \emph{lubić} `like':\\
\avm{
[ phon & \phonliste{ nie wydawa\l{} si\c{e} }\\
  head & [\type*{verb}
          neg & $+$ ]\\
  arg-st & < NP, V[ comps \1 < NP![\type{str}]! > ] > \+ \1 ]
}
\z
%
This lexical specification allows the object NP of the embedded verb to be
\type{gen}-marked in accordance with the constraint in (\ref{negation-polish-gen-case}).
%\footnote{When
%there is no argument composition, the positive verb \type{lubi\'{c}}
%assigns \textsc{acc} to the object NP.}
In Przepiórkowski's analysis, the feature
\textsc{neg} thus tightly interacts with the mechanism of argument composition and
lexical construction-specific case assignment (or satisfaction).

Negation in languages like \ili{French}, \ili{Italian}, and \ili{Polish}, among others, also involves negative concord.
\Citet{Swart:02} investigate  negative concord in \ili{French}, where multiple occurrences
of negative constituents express either
double negation or single negation:

%\eal
\ea \label{negation-nc-ex}
\gll Personne (n')                 aime personne.\\
     no.one   \hphantom{(}\NEG{} likes no.one\\\hfill(\ili{French})
\glt `No one is such that they love no one.' \hfill (double negation)
\glt `No one likes anyone.' \hfill  (negative concord)
\z
%
%With the semantic assumption that the contribution of negation in a concord context is %semantically empty, they formulate an HPSG analysis for negative concord.
%
The double negation reading in (\ref{negation-nc-ex}) has two quantifiers, while the single
negation reading is an instance of negative concord, where the two
quantifiers merge into one. \Citet{Swart:02} assume that the information contributed by
each quantifier is stored in \textsc{qstore} and retrieved at the
lexical level in accordance with constraints on the verb's arguments and semantic
content. For instance, the verb \textit{n'aime} in (\ref{negation-nc-ex}) will have two different ways of retrieving the
\textsc{qstore} value, as given in the following:\footnote{The
\textsc{qstore} value contains information
roughly equivalent to first order logic expressions like \textit{NO}x[Person(x)]. See
\citet{Swart:02}.}

\eal
\ex
\label{negation-quant-two-retrieved}
\avm{
[phon   &  \phonliste{ n'aime }\\
 arg-st & < NP![\textsc{qstore} \{\ibox{1}\}], NP[\textsc{qstore} \{\ibox{2}\}]! >\\
 quants & < \ibox{1}, \ibox{2} > ]
}
\ex
\label{negation-quant-one-retrieved}
\avm{
[phon   & \phonliste{ n'aime } \\
 arg-st & < NP![\textsc{qstore} \{\ibox{1}\}], NP[\textsc{qstore} \{\ibox{2}\}]! >\\
 quants & < \1 > ]
}
\zl
%
%\; \; :
%\; \; :  $\neg\exists$x$\exists$y Love(x,y)
%
\noindent
In the AVM (\ref{negation-quant-two-retrieved}), the two quantifiers are retrieved, inducing double negation ($\neg\exists$x$\neg\exists$y[love(x,y)]) while in (\ref{negation-quant-one-retrieved}), the two have a resumptive interpretation in which the two are merged into one ($\neg\exists$x$\exists$y[love(x,y)]).\footnote{See \citet{Swart:02} for detailed formulation of the retrieval of stored value.} This analysis, coupled with the complement treatment of \textit{pas} as a lexically stored quantifier, can account
for why \emph{pas} does not induce a resumptive interpretation with a quantifier (from
\citealt[\page 376]{Swart:02}):


\ea
\gll Il ne     va   pas    nulle part, il va   à  son travail.\\
     he \NEG{} goes \NEG{} no    where he goes at his work\\\hfill(\ili{French})
\glt `He does not go nowhere, he goes to work.'
\z
%
In this standard \ili{French} example, \citet{Swart:02}, accepting
the analysis of \citet{Kim:00} of \textit{pas} as a complement,
specify the meaning of the adverbial complement \emph{pas} to be included as a negative quantifier in the \textsc{quants} value.
 This means there would be no resumptive
reading for standard \ili{French}, inducing double negation as in
(\ref{negation-qstore}):\footnote{See \citew{Swart:02}, \citew{RichterandSailer2001},
and \crossrefchapterw[Section~\ref{semantics:sec-basic-architecture}]{semantics} for cases where \textit{pas} induces negative concord.}

\ea
\label{negation-qstore}
\avm{
[phon   & \phonliste{ ne va }\\
 arg-st & < Adv\sub{i}![\textsc{qstore} \{\ibox{1}\}], NP[\textsc{qstore} \{\ibox{2}\}]! >\\
 quants & < \ibox{1}, \ibox{2} > ]}
\z

\citet{PK:99} and \citet{BJ:00} also  investigate negative concord in \ili{Polish} and \ili{Welsh}
and offer HPSG analyses. Consider a \ili{Welsh} example from \citet[\page 17]{BJ:00}:

\ea
\gll Nid         oes neb yn yr ystafell\\
\textsc{neg}    is no.one in the room\\\hfill(\ili{Welsh})
\glt `There is no one in the room.'
\z
\noindent \citet{BJ:00}, identifying n-words with the feature
\textsc{nc} (negative
concord),  takes the verb \emph{nid oes} `\textsc{not} is' to bear the positive \textsc{neg} value,
and specifies the subject \emph{neb} to carry the positive \textsc{nc} (negative
concord) feature. This selectional approach, interacting with
well-defined features, tries to capture how more than one
negative element can correspond to a single semantic negation (see
\citealt{BJ:00} for detailed discussion).



\section{Conclusion}

One of the most attractive consequences of the derivational perspective on negation has been that
one uniform category, given other syntactic operations and constraints, explains the derivational
properties of all types of negation in natural languages, and can further provide a surprisingly
close and parallel structure among languages, whether typologically related or not. However, this
line of thinking runs the risk of missing the particular properties of each type of negation. Each
individual language has its own way of expressing negation, and moreover has its own restrictions in
the surface realizations of negation which can hardly be reduced to one uniform category.


In the non-derivational HPSG analyses for the four main types of sentential negation that I have
reviewed in this chapter, there is no uniform syntactic element, though a certain universal aspect
of negation does exist, viz.\ its semantic contribution.  Languages appear to employ various
possible ways of negating a clause or sentence. Negation can be realized as different morphological
and syntactic categories.  By admitting morphological and syntactic categories, it was possible to capture their idiosyncratic properties in a simple and natural manner. Furthermore, this theory has
been built upon the \isi{Lexical Integrity} Principle, the thesis that the principles that govern the
composition of morphological constituents are fundamentally different from the principles that
govern sentence structures. The obvious advantage of this perspective is that it can capture the
distinct properties of morphological and syntactic negation, and also of their distribution, in a
much more complete and satisfactory way.
\is{negation|)}

\section*{Abbreviations}

\begin{tabularx}{.99\textwidth}{@{}lX}
\textsc{3sgs} & 3rd singular subject\\
\textsc{3plo} & 3rd plural object\\
\textsc{conn} & connective\\
\textsc{del}  & delimiter\\
\textsc{hon}  & honorific\\
\textsc{npst} & nonpast\\
\textsc{rm}   & reflexive marker\\
\end{tabularx}

\section*{\acknowledgmentsUS}

I thank the reviewers of this chapter for detailed comments and suggestions, which
helped improve the quality of this chapter a lot. I also thank
Anne Abeill\'{e}, Bob Borsley, Jean-Pierre Koenig, and Stefan Müller for constructive comments on
earlier versions of this chapter. My thanks also go to Okgi Kim, Rok Sim, and Jungsoo Kim for
helpful feedback.

%\fi
%
{\sloppy
\printbibliography[heading=subbibliography,notkeyword=this]
}

\end{document}



%      <!-- Local IspellDict: en_US-w_accents -->
