\documentclass[output=paper]{langsci/langscibook} 
\ChapterDOI{10.5281/zenodo.1300608}
\author{Carmen Pérez-Vidal\affiliation{Universitat Pompeu Fabra}\and  Sonia López-Serrano\affiliation{Universitat Pompeu Fabra; Universidad de Murcia}\and Jennifer Ament\affiliation{Universitat Pompeu Fabra}\lastand Dakota J. Thomas-Wilhelm\affiliation{University of Iowa; Universitat Autònoma de Barcelona}  }
\title{Context effects in second language acquisition: formal instruction, study abroad and immersion classrooms}
\shorttitlerunninghead{Context effects in second language acquisition}
 
\abstract{\noabstract}
\maketitle
\lehead{C. Pérez-Vidal, S. López-Serrano, J. Ament \& D. Thomas-Wilhelm}
\begin{document} 
 
This volume within the EuroSLA Studies Series has been motivated by two fundamental reasons. Firstly, the assumption that applied linguistics research should first and foremost deal with topics of great social relevance, and, secondly, that it should also deal with topics of scientific relevance. Both ideas have led us to choose the theme ‘contexts of language \isi{acquisition}’ as the topic around which the monograph would be constructed.

The aim of this introduction is to set the scene and present the three contexts on focus in the monograph and justify this choice of topic within second language \isi{acquisition} (\isi{SLA}) research, the perspective taken in this volume. Starting with the latter, in the past two decades the examination of the effects of different contexts of \isi{acquisition} has attracted the attention of researchers, based on the idea that “the study of \isi{SLA} within and across various contexts of learning forces a broadening of our perspective of the different variables that affect and impede \isi{acquisition} in general” (\citealt{CollentineFreed2004intro}: 157). The authors continue, “however, focusing on traditional metrics of \isi{acquisition} such as \isi{grammatical} development might not capture important gains by learners whose learning is not limited to the formal classroom (ibid: 158)”. With reference to the social relevance of the topic, European \isi{multilingual} policies in the past decades have been geared towards the objective of educating our young generations in order to meet the challenge of \isi{multilingualism} (\citealt{Coleman2015,Pérez-Vidal2015a}), ultimately as an effect of “globalization and the push for \isi{internationalization} [on campuses] across the globe”\citep[1]{Jackson2013}. Indeed, the majority of European member states have embraced the recommendations made by the Council of Europe, encapsulated in the well-known 1+2 formula, according to which European citizens should have democratic access to \isi{proficiency} in their own language(s) plus two other languages. In order to reach such a goal, a couple of decades ago the Council of Europe put forward a series of key recommendations to member states: i) an earlier start in \isi{foreign language} learning; ii) mobility (the European Action Scheme for the Mobility of University Students, ERASMUS, exchange programme was launched in 1987, and since then more than three million students have benefitted from it); and iii) bilingual education, whereby content subjects should be taught through a \isi{foreign language} (\citealt{CommissionoftheEuropeanCommunities1995,CommissionoftheEuropeanCommunities1995}). The latter recommendation has given rise to a number of \isi{immersion} programmes at primary, secondary and tertiary levels of education, in parallel to the existing elite international schools (see the Eurobarometer figures and \citealt{WächterMaiwörm2014}, respectively). Such programmes are mostly taught through \ili{English}, but also through French, \ili{German}, \ili{Catalan}, and other languages. Whether such learning contexts, which we have called ‘international classrooms’ and include classrooms at home and abroad (\citealt{Pérez-VidalLorenzoTrenchs2017}), are de facto conducive to language \isi{acquisition} is a matter which indeed needs to be investigated. 

Against such a backdrop, this research monograph deals with the effects of different learning contexts mainly on adult, but also on adolescent learners’ language \isi{acquisition}. More specifically, it aims at comparing the effects of three learning contexts by examining how they change language learners’ \isi{linguistic} performance, and non-\isi{linguistic} attributes, such as motivation, sense of identity and affective factors, as has been suggested not only by \citet{Collentine2004effects} mentioned above, but also by a number of other authors (to name but a few, \citealt{Pellegrino2005,Dewaele2007,Hernández2010,LasagabasterEtAl2014,TaguchiEtAl2016}). 

\newpage 
More specifically, the three contexts brought together in the monograph include i) a conventional instructed second language \isi{acquisition} (ISLA) context, in which learners receive \isi{formal instruction} (\isi{FI}) in \ili{English} as a Foreign Language (\isi{EFL}); ii) a study abroad (SA) context, which learners experience during mobility programmes, with the \isi{target language} no longer being a foreign but a second language, learnt in a naturalistic context; iii) the \isi{immersion} classroom, also known as an \isi{integrated content} and language (ICL) setting, in which learners are taught content subjects through the medium of the \isi{target language} - more often than not \ili{English}, hence the term  \ili{English}-Medium Instruction (\isi{EMI}), and possibly \ili{English} as a Lingua Franca (\isi{ELF}) (\citealt{Björkman2013,House2013}). One last point needs to be made, concerning the issue of internationalisation, as is clearly stated in the title of the monograph:at any rate, the three contexts of \isi{acquisition} on focus in this volume represent language/culture learning settings in which an \textit{international} \textit{stance} may be promoted in learners, as described below, in some cases also including the \isi{internationalization} of the curriculum \citep{Leask2015}. 

In the \isi{SLA} tradition in which the different chapters contained in the volume are framed, the comparison across contexts has been established under the assumption that contexts vary in the “type of input received by the learner (implicit vs. explicit), the type of interaction required of the learner (meaning-focus vs. form-focused)” (\citealt{LeonardShea2017}: 185), and, most importantly, the type of exposure to the \isi{target language}, with variations in the amount of “input, output and interaction opportunities available to them” (\citealt{Pérez-Vidal2014}: 23). As the focus is on three different learning contexts - SA, \isi{EMI}, and \isi{FI} - we suggest that they can be understood as situated on a continuum in which the most “interaction-based”, with more favorable quantity and quality of input, would occur during a SA period. Second in order would be a semi-\isi{immersion} context, as might take place \isi{EMI} programmes, and the most “classroom-based” being \isi{FI} in ISLA. Similarly, it is also along such a continuum, that these contexts make possible for learners to develop an attribute which \citet{UshiodaDörnyei2012} refer to as an \textit{international} \textit{stance}. That is to say, learners have the opportunity to incorporate a new view of the world that integrates languages and cultures other than their own, often through the use of \ili{English} as a \isi{lingua franca} as a means of communication.

Turning to the cognitive mechanisms made possible in different \isi{linguistic} environments or learning contexts, these have ultimately also been claimed to be different. \citet[213]{DeKeyser2007} draws on skill \isi{acquisition} theory, which distinguishes three stages -  declarative knowledge, procedural knowledge and automatization -  to suggest that, “a stay abroad should be most conducive to the third stage. It can – at least for some learners – provide the amount of practice necessary for the gradual reduction of reaction time, error rate, and interference with other tasks that characterize the automatization process”. Similar cognitive perspectives might be applied to the classroom \isi{immersion} context, on the assumption that it generates a ‘naturalistic’ academic context in which language is learnt through focusing on curricular content, one of the issues the monograph seeks to explore. 

  
As for the existing set of findings concerning how learners develop their \isi{target language} abilities in ISLA, research has reached considerable consensus around some of the main issues by now, although some remain controversial, some barely examined, and some entirely unexplored. Let us now turn to a brief presentation of current thinking.  

Instructed \isi{SLA} investigates \isi{L2} learning or \isi{acquisition} that occurs as a result of teaching \citep[2716]{Loewen2013}. This field of research theoretically and empirically aims to understand “how the systematic manipulation of the mechanisms for learning and/or the conditions under which they occur enable or facilitate development and \isi{acquisition} of a [second] language” \citep[2]{Loewen2015}. Formal instruction is a particular environment in instructed \isi{SLA} that has been extensively researched for many decades. 

In 1998, Michael Long reviewed eleven studies that examined the effect of \isi{FI} on the rate and success of \isi{L2} \isi{acquisition}. Of the studies that were reviewed, six of them showed that \isi{FI} helped, three indicated that the instruction was of no help, and two produced ambiguous results. \citet{Long1983} claimed that instruction is beneficial to children and adults, to intermediate and advanced students, as well as in acquisition-rich and acquisition-poor environments. His final conclusion was that \isi{FI} was more effective than “exposure-based” in \isi{L2} \isi{acquisition}. These findings led researchers to ask whether instruction (\isi{FI}) or exposure (SA, \isi{EMI}, etc.) produced more rapid or higher levels of learning. 

Since \citegen{Long1988} seminal review of the effects of \isi{FI}, there have been a number of studies of the effect of \isi{FI}. For example, \citet{NorrisOrtega2001} conducted a meta-analysis of the effects of \isi{L2} instruction. Their study used a systematic procedure for research synthesis and meta-analysis to summarize findings from experimental and quasi-experimental studies between 1980 and 1998 that investigated the effectiveness of \isi{L2} instruction. Through their meta-analysis, they found that the literature suggests that instructional treatments are quite effective. They went on to investigate how effective instruction was when compared to simple exposure and found that there was still a large effect observed in favor of instructed learning.

\newpage 
\citet{Trenchs-Parera2009} conducted a study on the effects of \isi{FI} and SA as it related to the \isi{acquisition} of oral \isi{fluency}. Her results found that although both contexts have different effects on oral \isi{fluency} and production, both of these contexts did have a positive effect. She went on to say that “the differences between these two contexts [\isi{FI} and SA] may not fulfill the popular expectation that SA makes learners produce more native-like speech than does \isi{FI} at all levels” (p. 382). While these results do indicate that \isi{FI} can have a positive effect on \isi{L2} \isi{acquisition}, they are unable to demonstrate that \isi{FI} has learning effects that are conclusively more positive than those of more naturalistic environments.

We now turn to the examination of the effects of SA, often contrasted with ISLA, and occasionally also with at-home \isi{immersion}. SA research has generated a wealth of studies,  monographs, and handbooks on both sides of the Atlantic, starting in 1995 with Barbara \citegen{Freed1995book} seminal publication, followed by, to name but a few, \citet{CollentineFreed2004intro,Pellegrino2005,DuFonChurchill2006}, \citet{DeKeyser2007,Collentine2009,Kinginger2009,Jackson2013,LlanesMuñoz2013,ReganEtAl2009,MitchellEtAl2015,Pérez-Vidal2014a,Pérez-Vidal2017},and \citet{SanzMorales2018}. Two periods can be distinguished in such research \citep{Collentine2009,Pérez-Vidal2014b}. The first one was initiated by Freed’s volume. In those years research mainly focused on the \isi{linguistic} gains, or lack thereof, accrued with SA, with some attention paid to the impact of learner profiles and previous SA experiences (see for example, \citealt{BrechtEtAl1995}). Following that, new themes, besides \isi{linguistic} impact, and new angles to approach them, have emerged throughout the second period. Following \citegen{Collentine2009} tripartite distinction, such new themes include: (i) cognitive, psycholinguistic approaches looking into cognitive processing mechanisms displayed while abroad; (ii) sociolinguistic approaches analyzing input and interaction from a macro- and a micro-perspective; and, most centrally, (iii) \isi{sociocultural} approaches derived from a paradigm shift from a language-centric (i.e. etic) approach to a learner-centric (i.e. emic) one \citep{Devlin2014studyabroad}. As established in \citet[341]{Pérez-Vidal2017}, indeed, within the latter paradigm, and in order to focus on the learner and his/her immediate circumstances, SA research has recently begun to investigate non-\isi{linguistic} individual differences which affect learning in such a context, “that is: (a) \isi{intercultural} sensitivity and identity changes; (b) affects, such as foreign \isi{language anxiety} (FLA) or willingness to communicate (WTC) and enjoyment; (c) social networks, particularly through the use of new technologies and social platforms, and their effect on \isi{linguistic} practice”. Now, as \citet[313]{DeKeyser2014} emphasizes, “a picture is beginning to emerge of what language development typically takes place [during SA] and what the main factors are that determine the large amount of variation found from one study to another”. 


Turning now to the positive effects of SA on learners’ \isi{linguistic} progress, in a nutshell, empirical studies paint a blurred picture. They seem to show that SA does not always result in greater success than \isi{FI} in ISLA -  some learners do manage to make significant \isi{linguistic} progress while abroad, while others do not (\citealt{DeKeyser2007,Collentine2009,Llanes2011,Pérez-Vidal2015b,Sanz2014}). In fact, what such results seem to prove, is the notorious variation in amount of progress made, which has often been attributed to the variation in learners’ ability to avail themselves of the opportunities for practice that a SA context offers. These differences in turn are explained by learners’ individual ability for self-regulation while abroad, as further discussed below \citet{UshiodaDörnyei2012}. 

Looking at progress in more detail, empirical research has repeatedly shown that \isi{oral production} seems to be the winner, with effects on \isi{fluency} being significantly positive after SA, (\citealt{TowellEtAl1996};  \citealt{FreedEtAl2004,LlanesMuñoz2009,Valls-FerrerMora2014}). One interesting related finding has been made concerning the nature of the programmes \citep{Beattie2014}: robust \isi{immersion} programmes organized at home and including a substantial number of hours of academic work on the part of the learners can be as beneficial as a similar length of time spent abroad (i.e.\citealt{FreedEtAl2004}). In contrast to the results for \isi{fluency} in \isi{oral production}, results for \isi{grammatical} accuracy and complexity have been mixed, with \citet{DeKeyser1991} not finding much improvement, whereas \citet{Howard2005} or \citet{Juan-GarauEtAl2014}, to name but a few, report that progress is made after a period spent abroad. The other main area of improvement is \isi{pragmatics}, in particular when associated with the use of \isi{formulaic} routines, and perception and production of speech acts (see for a summary \citealt{Pérez-VidalShivelyforthcoming}), and particularly when paired with \isi{pragmatics} instruction. This takes us back to the key question of how the nature of the exchange programme can affect \isi{linguistic} outcomes. More specifically, issues %
such as type of accommodation, length of the stay, or initial level, have been found to significantly determine \isi{linguistic} and cultural development while abroad. Concerning initial level, \citet{Collentine2009} stated that there should be a threshold level which learners must reach to benefit fully from the SA \isi{learning context}. Once that level has been reached, most studies report better results for their respective lower level groups, confirming that the kind of practice most common while abroad, that is interaction in daily communication, mostly benefits the less advanced learners, while academic work done outside the classroom may benefit the most advanced ones \citep{Kinginger2009}. As for type of accommodation, home-stays with families have proved most beneficial An alternative option is with the so called \textit{family} \textit{learning} \textit{housing}, where students reside with \isi{target language} speakers of their own age, having signed a language pledge not to use any other language but the \isi{target language} \citep{Kinginger2015}. Length of stay also seems to be associated with advanced level learners, who may require longer periods to automatize the larger number of structures they have learnt at home than the lower level learners (\citealt{DeKeyser2014}). However, interestingly, shorter periods abroad, of less than one month, may also significantly benefit \isi{EFL} learners’ \isi{fluency}, accuracy and listening abilities (\citealt{LlanesMuñoz2009}). Three month periods may be more beneficial than six months (\citealt{LaraEtAl2015}). Listening has in fact clearly been shown to undergo significant progress while abroad (\citealt{BeattieEtAl2014}), as has reading \citep{Dewey2004}. Writing and vocabulary have also been shown to significantly benefit from SA (\citealt{Sasaki2007,Sasaki2011,Barquin2012,ZaytsevaEtAl2018}).

Regarding learners’ individual differences, age seems to play a role, as SA has been shown to be more beneficial for children than for adults in relative terms (\citealt{LlanesMuñoz2013}). Regarding aptitude, a certain level of working memory (\citealt{SundermanKroll2009}), \isi{phonological} memory (\cite{O’BrienEtAl2007}) and processing speed \citep{Taguchi2008} seem to correlate with accurate \isi{L2} production, \isi{oral production} and reception of \isi{pragmatic} intentions, respectively. Finally, concerning the emotional variables underlying self-regulation during exchanges in the \isi{target language} country, the expectation is that motivation will have a positive role and that anxiety, paired with the capacity for enjoyment, will as well. \citet{DewaeleEtAl2015} have found that SA benefits emotional stability, self-confidence and resourcefulness.  While identity goes through a process of repositioning, this process is not exempt from difficulties, which often conditions degree of contact with \isi{target language} speakers while abroad. More willingness to communicate and less foreign \isi{language anxiety} seem to obtain during SA (\citealt{DewaeleWei2013}; \citealt{DewaeleEtAl2015}). 

Turning to the third type of context, although it is still in its infancy, \isi{immersion}, the integration of content and language as an educational approach in primary, secondary (\isi{CLIL}) and tertiary levels (ICL), has also given rise to a sizeable number of research studies (such as for example: \citealt{AdmiraalEtAl2006}; \citealt{Dalton-Puffer2008,Airey2012}; \citealt{CenozEtAl2014}). The integration of content and language in higher education (\isi{ICLHE}) came to be recognized in its own right in 2004, with the first conference examining this context, and has steadily grown to this day \citep{Wilkinson2004}. 

\newpage 
Findings from \isi{immersion} and \isi{CLIL} contexts, abundantly examined in the \isi{SLA} literature, report that \isi{CLIL} and \isi{immersion} learners demonstrate language gains superior to learners who participate in \isi{FI} alone, with equal or superior \isi{content learning} outcomes (\citealt{WescheSkehan2002,Genesee2004};  \citealt{JiménezCatalánEtAl2006}; \citealt{Seikkula-Leino2007}). Specifically, gains are reported in receptive skills, vocabulary, morphology, and \isi{fluency}, whereas fewer gains have been observed according to syntax, writing, \isi{pronunciation} and \isi{pragmatics} (\citealt{Dalton-Puffer2008}), although results may be mixed (\citealt{Pérez-VidalRoquet2014}). Research on non-\isi{linguistic} outcomes has found that \isi{CLIL} learners seem to be more motivated, or that \isi{CLIL} can maintain students’ interests and change attitudes towards \isi{multilingualism}. Moreover, students generally perceive \isi{CLIL} participation as a positive experience (\citealt{LasagabasterSierra2009}). 

Turning now to adult education, the main focus of this monograph, a large body of research has been generated within the frame of \isi{ICLHE} which is specifically interested in the widespread implementation of \ili{English}-taught programs at mainly post graduate levels. This has come to be known as \ili{English} medium instruction (\isi{EMI}) which is characterized as a setting where \ili{English} is used as a medium for instruction by, and for non-native \ili{English} speakers in non-\ili{English} speaking environments (\citealt{HellekjaerHellekjaer2015}). Researchers in this field have begun investigating the phenomenon from a wide variety of angles, for example by looking at the implementation and policy making end of the spectrum \citep{Tudor2007}. What has been found is that the implementation of \isi{EMI} must be carefully managed in order not to create tensions, considering the role of the first language, attitudes towards \ili{English}, and the widespread effects of \isi{internationalization}, not only affecting faculty and students, but also governing bodies and administration (\citealt{DoizEtAl2014policy}). Others report on beliefs, attitudes and challenges from both the student/learner perspective and the faculty/institution’s perspective. Findings show that stakeholders in \isi{EMI} relate \ili{English} instruction to \isi{internationalization} very clearly, with some believing that one cannot exist without the other (\citealt{HenryGoddard2015}). This belief also proves to be a strong motivator for students to enroll in \isi{EMI} courses \citep{MargićŽeželić2015}, although the experience does not always meet their expectations regarding \isi{language improvement} and more support is often desired \citep{Sert2008}. Finally, perhaps the least investigated aspect of \isi{EMI} involves the assessment of outcomes measured in \isi{linguistic} as well as non-\isi{linguistic} terms. 

On the one hand there are investigations looking at non-\isi{linguistic} effects from \isi{EMI} participation (\citealt{Gao2008}; \citealt{GonzálezArdeo2016}). Research shows that a gradual implementation supporting both faculty and students is the most effective for maintaining and creating positive attitudes and motivation (\citealt{ChenKraklow2015}). On the other hand, there are studies regarding the \isi{content learning} implications of learning through a \isi{foreign language} \citep{Dafouz2014}. It has been argued that upon completion of a degree program there is no difference in content knowledge (\citealt{DafouzCamacho-Miñano2016}). A few studies investigating language outcomes from such a context (\citealt{LeiHu2014};  \citealt{AmentPérez-Vidal2015,Ritcher2017}) show little evidence of \isi{language improvement} from \isi{EMI} participation. They also reveal that at this point there is simply not enough research to point to any clear conclusions. \isi{EMI} is growing rapidly around the world and its close relationship with \isi{internationalization} will ensure its continuance for time to come. What must be kept in mind is that, in order to properly implement, benefit from, and provide appropriate support to faculty and institutions offering \isi{EMI} instruction, and maintain quality education, more research on this context must be carried out, specifically considering both \isi{linguistic} and non-\isi{linguistic} effects, which is precisely what this monograph aims to bring to light. 

However, to our knowledge, no publication exists which places the three contexts along the continuum already mentioned, as suggested in \citet{Pérez-Vidal2011,Pérez-Vidal2014b} with SA as ‘the most naturalistic’ context on one extreme, ISLA on the other, and ICL somewhere in between. The present monograph seeks to make a first attempt at filling such a gap, by including a number of studies analysing the effects of \isi{EMI}, and another series of studies doing the same with SA, in contrast with ISLA. In such a comparison it is further assumed that \isi{EMI} programmes are often experienced at the home institution either as an ‘international experience at home’ (\isi{internationalization} at home), or as a preparation for the ‘real’ experience of an SA period spent in the \isi{target language} country, in which learners will most probably be expected to regularly attend academic courses. In such a circumstance, whatever the local language, quite probably some of the courses offered, if not all, will be \isi{EMI} courses for international students, that is, they will be what we call ‘international classrooms’ (\citealt{Coleman2013EMI,Leask2015}). 

The monograph will thus be organized around the two contexts, \isi{EMI} and SA, on the understanding that their effects will be contrasted with those obtained in ISLA, when appropriate. Both \isi{linguistic} and non-\isi{linguistic} phenomena will be investigated, employing quantitative but also qualitative methods, independently or combined. Regarding target countries in the \isi{immersion} programmes examined, they include data from Spain and Colombia. Of the SA programmes scrutinized, data include exchanges having the following destinations: England, Ireland, France, Germany and Spain, in Europe, but also Canada, the USA, China, Brazil and Australia. The \isi{EMI} chapters deal with tertiary level language learners, a section of the population which has received much less attention in research thus far, compared to secondary or primary learners, as mentioned above. Similarly, one SA chapter deals with adolescent learners, again a research population scarcely examined in such a context.

\largerpage
As for the internal organization of the volume, following the introduction by the editors, the first chapters will deal with \isi{EMI} contexts of \isi{acquisition}, and the remaining ones with SA contexts. 


More specifically, we open up the monograph with four chapters devoted to the \isi{immersion} context: three examine tertiary education data, and the last one primary and secondary. In Chapter 2, Dakota Thomas-Wilhelm and Carmen Pérez Vidal explore \isi{EMI} in Catalonia, Spain, in contrast with ISLA, focusing on a syntactic phenomenon and its cognitive correlates, namely \ili{English} \isi{countable} and \isi{uncountable} nouns. In Chapter 3, Jennifer Ament and Júlia Barón examine two \isi{EMI} programmes with different intensity, also in Catalonia, looking into \isi{pragmatics}, namely, the use of \ili{English} discourse markers and their \isi{acquisition} in the \isi{EMI} context. Chapter 4, by Sofia Moratinos-Johnston, Maria Juan-Garau and Joana Salazar-Noguera, analyses a non-\isi{linguistic} issue, that is, learners’ \isi{linguistic} self-confidence and perceived level of \ili{English} according to the number of \isi{EMI} subjects taken at university in the Balearic Islands, Spain. Chapter 5, by Isabel Tejada-Sánchez and Carmen Pérez-Vidal, closes the set of chapters devoted to \isi{immersion}, by investigating the complexity, accuracy and \isi{fluency} of written productions by young \isi{EFL} \isi{immersion} learners in Colombia.

\largerpage
Subsequently, the series of chapters on SA begins with Chapter 6, by Pilar Avello, which takes a fresh perspective and discusses the methodological intricacies associated with the measurement and analysis of \isi{pronunciation} gains obtained during a sojourn abroad in an \ili{English}-speaking country (England, Ireland, Canada, the USA, Australia). Chapter 7 by Victoria Monge and Angelica Carlet, contrasts ISLA and SA. These authors compare \isi{L2} \isi{phonological} development, following a three-month period in any of the above-mentioned \ili{English}-speaking countries, while controlling for \isi{proficiency} level, in an attempt to follow up on \citegen{Mora2008} seminal study with a reverse design. In Chapter 8 Carmen del Rio, Maria Juan-Garau and Carmen Pérez-Vidal contrast the impact of a three-month SA period and \isi{FI} at home, in the case of adolescent \isi{EFL} learners, an age band which has received comparatively less attention than others, focusing on the learners’ foreign \isi{accent} and \isi{comprehensibility}, as judged by a group of non-native listeners, with the objective of assessing progress, following \citet{TrofimovichIsaacs2012}. {Motivation, identity and international \isi{posture} is the focus of Chapter 9, in which Leah Geoghegan compares tertiary level students spending a SA in an \ili{English}-speaking country with those in Germany or France, using qualitative research tools in order to gain a more detailed picture of the role of \isi{ELF} in SA. After that, Chapter 10 by Iryna Pogorelova and Mireia Trenchs explore} \isi{intercultural} adaptation during the experience of a SA period in different countries in Europe, but also in Canada, the USA, China, Brazil, and Australia. Finally, in Chapter 11 Ariadna Sánchez-Hernández deals with \isi{acculturation} and \isi{pragmatic} learning by international students in the USA, to close the series of chapters dealing with SA. 


\section*{Acknowledgments}
This work was supported
by the Ministry of Economy and Competitiveness [FFI{\linebreak}2013-48640-C2-1-P];
by the AGENCIA UNIVERSITARIA DE RECERCA (AGAUR), in Catalonia, [2014 SGR 1568]; 
and by a EUROSLA workshop grant (2017).   The monograph follows the EUROSLA workshop on the same theme celebrated at the Universitat Pompeu Fabra in Barcelona, Spain, 23-24 May, 2016.
 
 
\sloppy
\printbibliography[heading=subbibliography,notkeyword=this] 
\end{document}