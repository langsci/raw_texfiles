


The Adampe\il{Adampe} system is in many respects different, so there may be doubts as to whether it indeed belongs together with Dangme\il{Dangme}. The Adampe evidence will be treated later in this chapter.

 

\begin{table}
\caption{\label{tab:3:65}Ga\il{Ga}-Dangme\il{Dangme} numerals}

\begin{tabularx}{\textwidth}{Xllr@{~}ll}
\lsptoprule
~ & Dangme\il{Dangme} & Ga\il{Ga} &  & Dangme\il{Dangme} & Ga\il{Ga}\\
\midrule
1 & kákē & é-kòmé & 7 & kpà-à-ɡō (6+1)? & kpà-wo (6+1?)\\
2 & é-ɲ{\`{\~ɔ}} & é-ɲ{\`{ɔ}} & 8 & kpà-a-ɲ{\={\~{ɔ}}} (6+2) & kpà-a-ɲ{\~{ɔ}} (6+2?)\\
3 & é-t{\={\~{ɛ}}} & é-t{\~{ɛ}} & 9 & n{\`{\~ɛ}}{\'{\~ɛ}} & n{\`{ɛ}}ɛh{\'ũ}\\
4 & é-yw{\`{ɛ}}/é-wì{\`{ɛ}} & é-ɟw{\`{ɛ}} & 10 & ɲ{\`{\~ɔ}}ŋm{\'ã} (PL: ɲ{\`ĩ}ŋm{\'ĩ}) & ɲ{\`{ɔ}}ŋmá\\
5 & é-n{\={\~{u}}}{\={\~{ɔ}}} & é-nùm{\~{ɔ}} & 20 & ɲ{\`ĩ}ŋm{\'ĩ} éɲ{\`{\~ɔ}} (10*2) & ɲ{\`{ɔ}}ŋmá -í éɲ{\`{ɔ}} (10*2)\\
6 & é-kpà & é-k͡pàa & 100 & làfá & ò-há, pl. -ì\\
~ &  &  & 1000 & à-kpé & à-kpé, pl.-ì\\
\lspbottomrule
\end{tabularx}
\end{table}


\subsection{Gbe}%3.2.2.
\il{Gbe}The reconstruction of the Proto-Gbe\il{Proto-Gbe} numeral system is straightforward, since alternative forms are few (\tabref{tab:3:66}). It is based on the available evidence from twelve of the Gbe\il{Gbe} dialects.

\begin{table}
\caption{\label{tab:3:66}Proto-Gbe\il{Proto-Gbe} numerals and patterns (*)}


\begin{tabularx}{.8\textwidth}{lXrl}
\lsptoprule

1 & è-ɖe/ɖe-kpo & 7 & ‘hand’+2, 5+2\\
2 & è-ve/e-wè & 8 & e-ɲí, ‘hand’+3\\
3 & è-t{\`{\~ɔ}} & 9 & 8+1, 5+4 \\
4 & è-n{\`{ɛ}} & 10 & e-wó, *bula\\
5 & à-t{\'{\~ɔ}}{\~{ɔ}} & 20 & 10*2, ko\\
6 & à-d{\'{\~ɛ}}/z{\'{\~ɛ}} & 40 & e-kà\\
100 & 40*2+20 & 1000 & à-kpé, kotok{\~{u}}\\
\lspbottomrule
\end{tabularx}
\end{table}

The Gbe\il{Gbe} term for ‘six’ is primary. Its form, however, differs significantly from the (also primary) one attested in the languages of the Ga\il{Ga}-Dangme\il{Dangme} group.

The term for ‘eight’ seems to be derived from ‘four’, whereas the term for ‘nine’ follows the pattern ‘8+1’.

The forms for ‘twenty’ follow the pattern ’X*2’ in Aja\il{Aja} (\textit{bulaa-ve}), Waci-Gbe\il{Waci-Gbe} (\textit{blá-ve}) and Ewe\il{Ewe} (\textit{blá-vè}), which suggests an alternative form for ‘ten’ (*\textit{bula}).

The etymological relationship between the term for ‘fifteen’ and a lexical root with the meaning ‘foot’ attested in two of the dialects is an apparent innovation: Maxi-Gbe\il{Maxi-Gbe} \textit{à-f{\`{ɔ}}-t{\'{\~ɔ}}} (‘foot’, ’3’) and Kotafon\il{Kotafon}-Gbe\il{Gbe} \textit{f{\'{ɔ}}-t{\`{\~ɔ}}} (‘foot’,’3’). This pattern is attested in a number of the NC languages (including Atlantic). 

A primary term for ‘forty’ is distinguishable (hence ‘50=40+10’, ‘60=40+20’, ‘70=40+30’, ‘80=40*2’, ‘90=40*2+10’).


\subsection{Ka-Togo}%3.2.3.
\largerpage
Ka-Togo is a quite diverse group of the Left Bank languages. The reconstructions for each of its three branches are provided in the table below (\tabref{tab:3:67}). Its rightmost column lists forms and patterns that are the most likely candidates for the Proto-Ka-Togo\il{Proto-Ka-Togo} reconstruction.

\begin{table}
\caption{\label{tab:3:67}Proto-Ka-Togo\il{Proto-Ka-Togo} numeral system (**)}  
\begin{tabularx}{\textwidth}{rp{24mm}p{17mm}Qp{23mm}}
\lsptoprule 
~ & *Avatime-\il{Avatime}Nyangbo\il{Nyangbo} & *Kebu-\il{Kebu}Animere\il{Animere} & *Ikposo-\il{Ikposo}Ahlo-\il{Ahlo}Bowili & \textbf{**Proto-Ka-Togo}\il{Proto-Ka-Togo}\\
\midrule
1 & o-le & ʈ{\'{ɛ}}-ì, bɛ-ɹi & {\`{ɛ}}-dɩ/{\`{ɛ}}-dɩ-gbo & \textbf{di}\\
2 & ɛ-bha & din/ji & {\`{ɛ}}-va/{\`{ɛ}}-fwa & \textbf{bha, din}\\
3 & ɛ-ta & tha & {\`{ɛ}}-ta/{\`{ɛ}}-la & \textbf{ta}\\
4 & ɛ-n{\'{ɛ}} & nie & {\`{ɛ}}-na & \textbf{na/nɛ}\\
5 & ɛ-tí, ɛ-cu & thu(ŋ) & {\`{ɛ}}-tɔ & \textbf{tu(N)}\\
6 & golo/holo & k{\`{ʊ}}r{\`ã}ŋ & {\`{ɛ}}-gɔlu/{\`{ɛ}}-wɔlu & \textbf{golo/ koro}\\
7 & 6+1 & 10--3 & 6+1, k{\`{ɔ}}n{\`{ɔ}}, ù-zòni & \textbf{6+1}\\
8 & 10--2? a-nsɛ & 4*2 & {\`{ɛ}}-lɛ?,<4 & \textbf{4*2, nsɛ}/\textbf{lɛ?} \\
9 & 10--1? zi+3? & 5+4? & 8+1, 10--1? & \textbf{8+1? 10--1} \\
10 & kɛ-fɔ & the & wa/wu, i-jo, *bula & \textbf{fo/wo, te, bula}\\
20 & 10*2 & 10*2? & bula-2, lye-2, ŋué-2, t{\'{ɛ}}{\'{ɛ}}yá? & \textbf{10*2}\\
100 & a-lafa (< Ewe)\il{Ewe} & tùùrù, sala & gbɔwa & \textbf{lafa?} \\
1000 & a-kpe (< Ewe?\il{Ewe}) & lààfā & a-kpe & \textbf{a-kpe}\\
\lspbottomrule
\end{tabularx}
\end{table}

  
It needs to be stressed that the forms marked with /**/ are only suggestive and should not be taken at face value. They are not reconstructions in the strict sense and only serve for comparative purposes, so the absence of a tonal marker in a reconstructed form should not be considered meaningful. It only shows that at this point the available evidence does not allow reconstructing a tone in the pertinent case.


\subsection{Na-Togo}%3.2.4.
An overview of numerical terms as attested in the branches of Na-Togo and some isolated languages is provided below (\tabref{tab:3:68}). A tentative reconstruction of the Na-Togo numeral system can be found in the rightmost column. 

\begin{table}
\caption{\label{tab:3:68}Proto-Na-Togo\il{Proto-Na-Togo} numeral system (**)}
\small
\begin{tabularx}{\textwidth}{r lQlQlQ}
\lsptoprule

~ & Adele\il{Adele} & Anii\il{Anii} & *Lelemi\il{Lelemi} & *Likpe-\il{Likpe}Santrokofi & Logba\il{Logba} & \textbf{**Proto-}\textbf{Na-Togo}\il{Proto-Na-Togo}\\
\midrule
1 & {\`{ɛ}}-kí & d{\={ɨ}}ŋ, *mi & ù-nwi/{\`{ɔ}}-w{\~{\^ɛ}} & n{\`{ʊ}}{\'{ɛ}}/nwíì (l{\`{ɛ}}w{\'{ɛ}}) & i-kpɛ & \textbf{i-wɛ}/\textbf{kpɛ?} , \textbf{di(N)?} \\
2 & {\`{ɛ}}-ny{\`{ɔ}}{\`{ɔ}}n & ī-ɲī{\={ʊ}} & í-ɲ{\'{ɔ}} & ɲ{\'{ɔ}}/nú{\`{ə}} & i-nyɔ & \textbf{i-nyɔ}\\
3 & à-sì & ī-rīū & {\`{ɛ}}-tɛ & ti{\'{ɛ}} & i-ta & \textbf{i-ta}\\
4 & {\`{ɛ}}-nàà & ī-nāŋ & í-na & na & i-na & \textbf{i-na}\\
5 & tòn & ī-n{\={ʊ}}ŋ & {\`{ɛ}}-lɔ & n{\'{ɔ}} & i-nú & \textbf{i-no(N)}\\
6 & kòòròn & ī-kōlōŋ & {\`{ɛ}}-ku & kua & i-gló & \textbf{golo}/\textbf{kolo, ku}\\
7 & 6 + 1 & kūlūmī (6+1?) & 4+3? & 6+1? & 6+1 & \textbf{6+1}\\
8 & nìy{\`{ɛ}} & 4PL & 4PL? & 4PL? & 4PL & \textbf{4PL}\\
9 & y{\`{ɛ}}-1 & tʃīīnī & 10--1 & nase & X-1 & \textbf{10--1}\\
10 & fò & t{\={ɘ}}b & vu/we & fo/wo? & u-ɖú & \textbf{fo, ɗu, təb}\\
20 &  &  & 10*2 & 10*2 &  & \textbf{10*2, ɔ-ɖɔ(n), ā-kōō, dìkpìlìn}\\
50 & 20*2+10 & 20-PL+10 & ti & 10*5 & 10*5 & \textbf{20*2+10}\\
100 & 20*5 & 20*5, ɡā-s{\={ɘ}}wā & 50*2, lafa & kò-lòfá & u-ga & \textbf{20*5, lofa, u-ga}\\
1000 & 200*5 & ū-f{\={ɘ}}l{\={ɘ}}, kōtōkū & pim, ka-kpi & k{\`{ɔ}}-kpí & a-kpi & \textbf{a-kpi, pim?} \\
\lspbottomrule
\end{tabularx}
\end{table}

  
The Lelemi\il{Lelemi} term for ‘fifty’ (\textit{lì-tì}) is peculiar because it is a likely source of ‘hundred’: \textit{è-tì} \textit{á-ɲ{\'{ɔ}}} (‘50*2’). 

 
\subsection{Nyo}%3.2.5.
\largerpage
The Nyo group, which is comprised of dozens of languages, is the most representative within the family. For this reason (even though the Nyo numeral systems are closely related to each other) they will be studied separately (by sub-group) and then compared to each other. 

\newpage   
\subsubsection{Agneby (Abbey, Abiji, Adioukru)}%3.2.5.1.
\il{Abbey}\il{Abiji}Alternative sources representative of these three languages are quoted below (\tabref{tab:3:69}). Significant variation of forms is sporadically attested.

\begin{table}
\caption{\label{tab:3:69}Proto-Agneby\il{Proto-Agneby} numeral system (*)}
\small
\begin{tabularx}{\textwidth}{rlllllQQ}
\lsptoprule

~ & Abbey1 & Abbey2 & Abiji1 & Abiji2 & Adioukru1 & Adioukru2 & \textbf{*Proto-Agneby}\il{Proto-Agneby}\\
\midrule
1 & {\`{ŋ}}kp{\={ɔ}} & {\`{ŋ}}kp{\={ɔ}} & {\'{n}} {\textprimstress}n{\'{ɔ}} & {\textsubdot{\'{n}}}n{\`{ɔ}} & ɲâm & ɲâm & \textbf{N-kpɔ, ɲ-âm, *a-ri}\\
2 & āɲ{\textsubtilde{\'{ʊ}}} & āɲ{\'{\~ʊ}} & áá {\textprimstress}n{\'{ʊ}} & áān{\={ʊ}} & yóɲ & ɲóɲ & \textbf{a-ɲ{\textsubbar{ʊ}}}/\textbf{n{\={ʊ}}}\\
3 & ār{\'{ɩ}} & āɾí & {\textsubtilde{\H{ɛ}}}{\textsubtilde{\H{ɛ}}} {\textprimstress}t{\textsubtilde{\H{ɩ}}} & {\'{\~ɛ}}{\={\~{ɛ}}}t{\={ɪ}} & ɲâh{\`{n}} & ɲâh{\`{n}} & \textbf{a-ti(N)/ ri}\\
4 & āl{\'{ɛ}} & àl{\'{ɛ}} & {\textsubtilde{\H{a}}}{\textsubtilde{\H{a}}} {\textprimstress}l{\H{a}} & {\'ã}{\={\~{a}}}lā & yâr & jâr & \textbf{a-n{\textsubtilde{í}}}/\textbf{la, jar}\\
5 & ōn{\textsubtilde{í}} & ōní & {\H{e}}{\H{e}} {\textprimstress}n{\H{e}} & éēnē & y{\^{e}}n & j{\^{e}}n & \textbf{o-ne,l{\`{ɔ}}h{\textsubtilde{\`{ʊ}}}, j{\^{e}}n}\\
6 & l{\`{ɔ}}h{\textsubtilde{\`{ʊ}}} & l{\`{ɔ}}h{\`{\~ɔ}} & náh{\`{ʊ}}{\textsubtilde{à}} & náh{\`{\~ʊ}}{\`ã} & n{\^{ɔ}}h{\`{n}} & n{\^{ɔ}}h{\`{n}} & \textbf{hu(n)}\\
7 & l{\`{ɔ}}h{\textsubtilde{\`{ʊ}}}-ár{\={ɩ}} & l{\`{ɔ}}h{\~{ʍ}}{\={\~{a}}}ɾí & n{\textsubtilde{\'{ɔ}}}b{\`{ʊ}} & n{\'{\~ɔ}}ᵐb{\`{ʊ}} & l{\'{ɔ}}b{\`{ŋ}} & l{\'{ɔ}}b{\`{ŋ}} & \textbf{6+1, bu(n)}\\
8 & èpyè & èpʲè & nówò & nówò & níw{\`{n}} & níw{\`{n}} & \textbf{è-pyè, wo(n)}\\
9 & ɲâkó & ɲāàkó & n{\textsubtilde{\H{ɛ}}} {\textprimstress}br{\H{ɛ}} & n{\'{\~ɛ}}ᵐbr{\`{ɛ}} & líbár{\`{m}} & líbár{\`{m}} & \textbf{bare(-n)}\\
10 & èn{\textsubtilde{\`{ə}}} & {\`{n}}n{\`{ɛ}} & {\'{n}}dí{\`{ɔ}} & {\textsubdot{\'{n}}}dí{\`{ɔ}} & l{\^{ɛ}}w & l{\^{ɛ}}w & \textbf{nɛ(n) (< 5\textsc{pl}}?), \textbf{diw/ liw}\\
20 & ēbrá-ɲ{\textsubtilde{\`{ʊ}}} & òbɾāɲ{\`{\~ʊ}} & àbrúá{\textsubtilde{\'{ɩ}}} & àbr{\'ũ}{\'ã}{\'{\~ɪ}} & lík{\`{ŋ}} & lík{\`{ŋ}} & \textbf{<‘}\textbf{hand’ *2?,li-kŋ}\\
100 & yā & jā & y{\v{a}} & j{\v{a}} & ék{\`{ŋ}}-yén & ékŋ j{\^{e}}n (20*5) & \textbf{ja, 20*5}\\
1000 & àkp{\={ɩ}} & àkpī &  & àkp{\v{ɪ}} &  & fándí (Engl.?) & \textbf{a-kpi}\\
\lspbottomrule
\end{tabularx}
\end{table}

The presence of the primary terms for ‘seven’, ‘eight’ and ‘nine’ is an important characteristic of this sub-group. 


 
\subsubsection{Attié}%3.2.5.2.
\il{Attié}Internal reconstruction of the Attié\il{Attié} numeral system yielded the following results (\tabref{tab:3:70}).
\newpage
\begin{table}
\caption{\label{tab:3:70}Attié\il{Attié} numeral system (*)}


\begin{tabularx}{\textwidth}{lXrl}
\lsptoprule

1 & kə(n) & 7 & nson\\
2 & mwə(n) & 8 & ma-4? 2 de 10?\\
3 & ha(n) & 9 & ŋgwan\\
4 & dʒí(n) < *kɥe? & 10 & kɛŋ\\
5 & bə(n) & 20 & ‘hand' (bwa?)*2?\\
6 & mu(n) & 100 & ja\\
~ &  & 1000 & a-kpi\\
\lspbottomrule
\end{tabularx}
\end{table}


  
\subsubsection{Awikam-Alladian}%3.2.5.3.
\il{Alladian}No numerical terms (except for ‘one’ and `nine') are reconstructable on the sub-group level. This raises doubts as to whether these languages should indeed be grouped together. A representation of the pertinent forms is presented in the table below (\tabref{tab:3:71}) and may serve as a starting point for further discussion.

\begin{table}
\caption{\label{tab:3:71}Avikam\il{Avikam}-Alladian\il{Alladian} numerals}
\small
\begin{tabularx}{\textwidth}{l llQ r llQ}
\lsptoprule

~ & Awikam & Alladian\il{Alladian} & \textbf{Awikam-Alladian}\il{Alladian} &  & Awikam & Alladian\il{Alladian} & \textbf{Awikam-Alladian}\il{Alladian}\\
\midrule
1 & {\'{ɛ}}t{\textsubtilde{\'{ɔ}}} & {\={ɛ}}t{\textsubtilde{ò}} & \textbf{ɛ-t{\textsubbar{o}}} & 7 & {\'{ɛ}}by{\textsubtilde{\'{ɔ}}} & {\={ɛ}}bw{\textsubtilde{è}} & \textbf{{\'{ɛ}}-by{\textsubtilde{\'{ɔ}}}, {\={ɛ}}-bw{\textsubtilde{è}}}\\
2 & áɲ{\textsubtilde{\'{ɔ}}} & {\textsubtilde{ā}}yr{\`{ɛ}} & \textbf{á-ɲ{\textsubbar{ɔ}}, {\textsubtilde{ā}}-yrɛ} & 8 & {\`{ɛ}}ty{\'{ɛ}} & ēɥrì & \textbf{{\`{ɛ}}-ty{\'{ɛ}}, ē-ɥrì}\\
3 & {\textsubtilde{á}}z{\textsubtilde{á}} & {\textsubtilde{ā}}{\textsubtilde{ò}} & \textbf{{\textsubtilde{á}}-z{\textsubtilde{á}}, {\textsubtilde{ā}}-{\textsubtilde{ò}}} & 9 & {\'{ɛ}}mr{\textsubtilde{\'{ɔ}}} & {\={ɛ}}mwr{\`{ɔ}} & \textbf{{\'{ɛ}}-mr{\textsubtilde{\'{ɔ}}}}\\
4 & àn{\textsubtilde{á}} & {\textsubtilde{ā}}z{\`{ɔ}} & \textbf{à-n{\textsubtilde{á}}, {\textsubtilde{ā}}-z{\`{ɔ}}} & 10 & èjú & {\={ɛ}}và & \textbf{è-jú, {\={ɛ}}-và}\\
5 & àɲú & {\={ɛ}}nrì & \textbf{à-ɲú, {\={ɛ}}-nrì} & 20 & {\`{ɛ}}v{\'{ɛ}} & {\={ɛ}}ɥá, *{\={ɛ}}kòɥì & \textbf{{\`{ɛ}}-v{\'{ɛ}}, {\={ɛ}}-ɥá}\\
6 & áwá & {\={ɛ}}wrè & \textbf{á-wá, {\={ɛ}}-wrè} & 100 & àkpá {\textprimstress}-ɲú & 20*5 & \textbf{20*5, àkpá {\textprimstress}-ɲú}\\
\lspbottomrule
\end{tabularx}
\end{table}

\subsubsection{Potou-Tano}%3.2.5.4.
\subsubsubsection{Potou}%3.2.5.4.1.
The following forms are distinguishable in the Potou sub-group (\tabref{tab:3:72}).

\begin{table}
\caption{\label{tab:3:72}Potou numerals}


\begin{tabularx}{\textwidth}{r llQ r lll}
\lsptoprule
\small
~ & Ebrie\il{Ebrie} & Mbato\il{Mbato} & \textbf{*Potou} &  & Ebrie\il{Ebrie} & Mbato\il{Mbato} & \textbf{*Potou}\\
\midrule
1 & b{\textsubtilde{\`{ɛ}}}/br{\`{ɛ}} & lóɓō & \textbf{b{\textsubtilde{\`{ɛ}}}/br{\`{ɛ}}, ló-ɓō; ce/se} & 7 & ákʰwácʰè & óɓīs{\textsubtilde{é}} & \textbf{6+1}\\
2 & m{\textsubtilde{\`{ɔ}}} & {\textsubtilde{ó}}n{\textsubbar{o}}{\textsubtilde{\'{ɔ}}} & \textbf{n{\textsubbar{o}}{\textsubtilde{\'{ɔ}}}} & 8 & áɓyá & ógɓī & \textbf{ɓyá/ gɓī}\\
3 & ɓwàɗyá & n{\textsubtilde{\'{ɛ}}}jē/n{\textsubtilde{ó}}jē & \textbf{ɗyá/je} & 9 & áɓr{\`{ɔ}} & ótr{\texthighriseu} & \textbf{ɓr{\`{ɔ}}, tr{\texthighriseu}}\\
4 & ɓwèɗí & n{\textsubtilde{\'{ɛ}}}ní/n{\textsubtilde{ó}}ní & \textbf{ɗi/ni} & 10 & áw{\'{ɔ}} & ówā & \textbf{wɔ}\\
5 & mw{\textsubtilde{à}}n{\textsubtilde{á}} & n{\textsubtilde{\'{ɛ}}}n{\textsubtilde{ā}} & \textbf{n{\textsubbar{a}}} & 20 & ápʰ{\textsubtilde{\`{ɛ}}} & óp{\textsubtilde{\={ɛ}}} & \textbf{p{\textsubbar{ɛ}}}\\
6 & ákʰwá & ókoā & \textbf{kwa} & 100 & àyà & y{\v{a}} & \textbf{ya}\\
\lspbottomrule
\end{tabularx}
\end{table}

  
\subsubsubsection{Tano}%3.2.5.4.2.
The Tano branch consists of nearly thirty languages. It seems reasonable to treat them by sub-groups.


\subparagraph{Western Tano}
~
\begin{table}[h]
\caption{\label{tab:3:73}Western Tano numerals}
\small
\begin{tabularx}{\textwidth}{r QQQl}
\lsptoprule
~ 	& Abure1 	& Abure2 			& Eotile\il{Eotile} 		& \textbf{Western Tano}\\			
\midrule                                                                                                                                        
1 	& okuè 		& ókúè 			& {\`{ɩ}}k{\`{ʊ}} 		& \textbf{o-kue} 					\\
2 	& aɲù 	& áɲ{\textsubtilde{û}} 	& àɲ{\textsubtilde{\'{ɔ}}} 	& \textbf{a-ɲu(n)} 					\\
3 	& nɳà 		& {\'{ŋ}}ŋâ 		& àh{\textsubtilde{á}} 		& \textbf{n-ha(n)} 					\\
4 	& nnàn 		& {\'{n}}n{\textsubtilde{â}}& {\textsubtilde{à}}n{\`{ɛ}} 	& \textbf{n-na(n)} 					\\
5 	& nnú 	& {\`{n}}n{\textsubtilde{ú}}& ànù 			& \textbf{n-nu(n)} 					\\
6 	& ncɪ{\`{ɛ}} 	& {\'{ɲ}}c{\'{ɩ}}{\`{ɛ}} 	&àh{\'{ɩ}}{\textsubtilde{\`{ɛ}}}& \textbf{n-c{\'{ɪ}}{\`{ɛ}}}/\textbf{hí{\`{ɛ}}} 	\\
7 	& nc{\`{ʋ}}n 	& {\'{ɲ}}c{\textsubtilde{\^{ʊ}}} 	& àfà 	& \textbf{n-c{\`{ʋ}}n, à-fà}\\
8 	& m{\`{ɔ}}k{\`{ʋ}}{\'{ɛ}} 	& m{\`{ɔ}}kúè 	& àn{\`{ɛ}}mr{\textsubtilde{\`{ɔ}}} 	&  \textbf{m{\`{ɔ}}-k{\`{ʋ}}{\'{ɛ}}, à-n{\`{ɛ}}mr{\textsubtilde{\`{ɔ}}}}\\
9 	& puál{\'{ɛ}}h{\`{ʋ}}n 	& p{\`{ʊ}}àl{\textsubtilde{\`{ʊ}}}h{\textsubtilde{\^{ʊ}}} 	& brúkú 	& \textbf{puál{\'{ɛ}}h{\`{ʋ}}n, brúkú}\\
10 	& óblún 	& òbùl{\textsubtilde{ú}} 	& èdí 	& \textbf{ò-bùl{\textsubtilde{ú}}, è-dí}\\
20 	& {\'{ɛ}}f{\'{ɪ}}n 	& {\'{ɛ}}f{\textsubtilde{\^{ɩ}}} 	& èfè 	&  \textbf{{\'{ɛ}}-fɪ(n)}\\
100 	& {\`{ɛ}}vá okuè 	& {\`{ɛ}}y{\v{a}} k{\H{u}}è 	& átá 	& \textbf{{\`{ɛ}}-vá/{\`{ɛ}}-y{\v{a}}, átá}\\
1000 	& akp{\'{ɪ}} okuè 	&  	&  	& \textbf{a-kpi}\\
\lspbottomrule
\end{tabularx}
\end{table}

\newpage  
\subparagraph{Central Tano}

\textbf{Akanic~(\tabref{tab:3:74})}\textbf{:}

\begin{table}
\caption{\label{tab:3:74}Akanic numerals}
\begin{tabularx}{\textwidth}{rlXXXl}
\lsptoprule
~ & Akan1 (Twi\il{Twi} dial.) & Akan2 & Abron1 & Abron2 & \textbf{*Akanic}\\
\midrule
1 & baakó{\textasciitilde} & baak{\'õ} & bak{\~{u}} & bìàk{\textsubtilde{\'{ʊ}}}ʔ & \textbf{ba-kó(n)}\\
2 & {\`{ə}}bìé-{\'{n}} & mmie-nú & mie-nu & m{\textsubtilde{ì}}èn{\textsubtilde{ú}}ʔ & \textbf{mie-nú}\\
3 & {\`{ə}}bìè-sá{\textasciitilde} & mmeɛ-ns{\'ã} & mie-nsá & m{\textsubtilde{ì}}{\textsubtilde{\`{ɛ}}}nz{\textsubtilde{á}}ʔ & \textbf{mie-nsá(n)}\\
4 & à-ná{\'{n}} & (ɛ)ná{\'{n}} & nain & {\'{n}}n{\textsubtilde{á}}{\textsubtilde{\'{ɩ}}} & \textbf{náín}\\
5 & {\`{ə}}-nú{\'{m}} & (e)nú{\'{m}} & num & {\`{n}}n{\textsubtilde{ú}}m & \textbf{nú{\'{m}}}\\
6 & {\`{ə}}-sìá{\textasciitilde} & (e)ns{\~{i}}{\'ã} & nsi{\~{a}} & {\`{n}}z{\textsubtilde{ì}}{\textsubtilde{á}} & \textbf{sìá(n)}\\
7 & {\`{ə}}-s{\'{ɔ}}{\'{n}} & (ɛ)nsó{\'{n}} & nsɔ & {\`{n}}z{\textsubtilde{\H{ʊ}}}{\textsubtilde{\H{ʊ}}} & \textbf{só(n)}\\
8 & à-wòtɕɥé/tw/& nwɔtwé & ŋɔt͡ʃwie & w{\`{ɔ}}cɥ{\'{ɩ}} & \textbf{twé/cué}\\
9 & à-kró{\'{n}} & (ɛ)nkró{\'{n}} & ŋkrɔŋ & {\`{ŋ}}g{\textsubtilde{\'{ɔ}}}n{\textsubtilde{\'{ɔ}}} & \textbf{n-kró{\'{n}}}\\
10 & dú & (e)dú & du & dúʔ & \textbf{dú}\\
20 & {\`{ə}}dùònú & aduonú & edu enu & àd{\`{ü}}òn{\textsubtilde{ù}} & \textbf{10*2}\\
100 & {\`{ɔ}}hà & ɔha & ɔha & hà & \textbf{ɔ-ha}\\
1000 & àpí{\'{m}} & apé{\'{m}} & apim &  & \textbf{a-pí{\'{m}}}\\
\lspbottomrule
\end{tabularx}
\end{table}
\subparagraph{Bia}

The numeral systems in these languages (Agni\il{Agni}, Baoule\il{Baoule}, Sefwi\il{Sefwi}, Nzema\il{Nzema}, Ahanta\il{Ahanta}, and Jwira\il{Jwira}-Pepesa) are virtually identical and can be described as follows (\tabref{tab:3:75}).

\begin{table}
\caption{\label{tab:3:75}Proto-Bia\il{Proto-Bia} numeral system (*)}
\begin{tabularx}{\textwidth}{lXrl}
\lsptoprule
1 & ko(n) & 7 & su(n)\\
2 & nu, ɲ{\`{ɔ}}(n) & 8 & cʊɛ/twɛ\\
3 & sa(n) & 9 & {\`{n}}ɡ{\`{\~ɔ}}l{\`ã}, nkró{\'{n}} \\
4 & na(n) & 10 & bulu\\
5 & nu(n)/nu(m) & 20 & 10*2\\
6 & sia(n) & 100 & ya\\
~ &  & 1000 & akpi\\
\lspbottomrule
\end{tabularx}
\end{table}

 
\newpage  
\subparagraph{Guang}\il{Guang}

This sub-group has two branches, Southern and Northe\textstyleTitreviCar{rn} Guang\il{Guang}, which consist of four and eleven languages, respectively). Despite, the Guang numeral systems do not differ significantly, hence quoting individual forms seems unreasonable. Our reconstructions for both branches, as well as the general Guang reconstruction, are given below (\tabref{tab:3:76}).

\begin{table}
\caption{\label{tab:3:76}Guang\il{Guang} numerals}
\begin{tabularx}{\textwidth}{rXXl}
\lsptoprule
~ & *Northern Guang\il{Guang} & *Southern Guang\il{Guang} & \textbf{**Guang}\il{Guang}\\
\midrule
1 & k{\'{ɔ}} & kɔ & \textbf{kɔ}\\
2 & ɲ{\'{ɔ}} & ɲ{\'{ɔ}} & \textbf{ɲ{\'{ɔ}}}\\
3 & sá & sa(n) & \textbf{sa(n)}\\
4 & ná & nɛ(n)/na & \textbf{na(n)}\\
5 & nú(n) & nu/ni & \textbf{nu(n)}\\
6 & síyé & siɛ(n) & \textbf{siɛ(n)}\\
7 & sún{\'{ɔ}} & sún{\H{ɔ}} & \textbf{súnɔ(n)}\\
8 & bùrùwá, kwé & twi/cwi & \textbf{bùrùwá, kwé}/\textbf{cwi}\\
9 & kpɔnɔ, sàng{\'{ɔ}}{\'{ɔ}}ʔ & kpunɔ & \textbf{kpunɔ, sàng{\'{ɔ}}{\'{ɔ}}ʔ}\\
10 & dú & du & \textbf{du}\\
20 & o-ko, 10*2 & 10*2 & \textbf{10*2, ko?}\\
100 & lafa (< Akan?\il{Akan}) & {\`{ɔ}}l{\`{ɔ}}f{\'{ɛ}}/lafa & \textbf{lafa}\\
1000 & kp{\'{ɪ}}ŋ, pim & a-kpe & \textbf{kpi(N), pim}\\
\lspbottomrule
\end{tabularx}
\end{table}

 
\subparagraph{Krobu; Basilia-Adele; Ega} \il{Krobu}\il{Basila}\il{Adele}\il{Ega}

To make our presentation complete, the evidence of these three isolated Tano languages is presented in the table below (\tabref{tab:3:77}).

\begin{table}
\caption{\label{tab:3:77}Numerals in Tano isolated languages}
\begin{tabularx}{\textwidth}{rXXl}
\lsptoprule

~ & Krobu\il{Krobu} & Basila-\il{Basila}Adele\il{Adele} & Ega\il{Ega}\\
\midrule
1 & k{\textsubtilde{\'{ɔ}}} & k{\textsubtilde{\^{ʊ}}}, li/diŋ & ì-lō-gɓó\\
2 & {\'{ɲ}}-ɲ{\textsubtilde{\'{ɔ}}} & ɲ{\textsubtilde{ú}}{\textsubtilde{à}} & {\`{ɩ}}-ɲ{\`{ɔ}}\\
3 & {\'{n}}-s{\textsubtilde{á}} & sa & {\`{ɩ}}-tà\\
4 & {\'{n}}-n{\textsubtilde{á}} & na & {\`{ɩ}}-l{\`{ɛ}}\\
5 & {\'{n}}-n{\textsubtilde{ù}} & ton, nun & ì-ŋwè\\
6 & {\'{n}}-s{\~{y}}{\textsubtilde{\={ɛ}}} & koron & 5+1\\
7 & {\'{n}}-s{\^{o}} & 6+1? & 5+2\\
8 & m{\`{ɔ}}-kw{\'{ɛ}} & 4--4, c{\'{ɥ}}{\'{ɛ}} & 5+3\\
9 & {\`{ŋ}}-gr{\`{ɔ}}{\textsubtilde{ā}} & -1, gwalan & 5+4\\
10 & brú & fo, teb, bulu & ì-zù\\
20 & à-brūā{\textsubtilde{\'{ɛ}}} (10*2?) & dikpilin, koo, bulV & ú-glū\\
100 & y{\v{a}} & 20*5 & 20*5\\
1000 &  & kpen? & \\
\lspbottomrule
\end{tabularx}
\end{table}

 
\subsection{Proto-Kwa}%3.2.6.
\il{Proto-Kwa}Intermediate reconstructions suggested above should be compared in order to reconstruct the forms of the Proto-Kwa\il{Proto-Kwa} numerals. It seems reasonable to group potentially related forms (or patterns) together. The rightmost column contains isolated forms attested in one particular group only.

\subsubsection{‘One’}%3.2.6.1.
\largerpage 
\begin{table}
\caption{\label{tab:3:78}Kwa\il{Kwa} stems for `1'}

\small
\begin{tabularx}{\textwidth}{Q llll}
\lsptoprule

     & 1 & 1 & 1 & 1\\
\midrule
{*Ga-}\il{Ga}\textbf{Dangme}\il{Dangme} & ká-kē, *go/wo &  &  & é-kòmé\\
{*Gbe}\il{Gbe} & ɖe-kpo & è-ɖe &  & \\
{*Ka-Togo} &  & di &  & \\
{*Na-Togo} & i-wɛ/kpɛ? & di(N)? &  & \\
\textit{*Nyo:}\\
~~~~{*Agneby} & N-kpɔ & *a-ri &  & ɲ-âm\\
~~~~{Attié}\il{Attié} & kə(n) &  &  & \\
~~~~{Awikam} &  &  & {\'{ɛ}}-t{\textsubtilde{\'{ɔ}}} & \\
~~~~{Alladian}\il{Alladian} &  &  & {\={ɛ}}-t{\textsubtilde{ò}} & \\
~~~~\textit{Potou-Tano}\\
~~~~~~~~{Potou} & *ce/se &  &  & b{\textsubtilde{\`{ɛ}}}/br{\`{ɛ}}, ló-ɓō\\
~~~~~~~~\textit{Tano}\\ 
~~~~~~~~~~~~{Western} & o-kue &  &  & \\
~~~~~~~~~~~~\textit{Central}\\
~~~~~~~~~~~~~~~{Akanic} & ba-kó(n) &  &  & \\
~~~~~~~~~~~~~~~{Bia} & ko(n) &  &  & \\
~~~~~~~~~~~~{Guang}\il{Guang} & kɔ &  &  & \\
~~~~~~~~~~~~{Krobu}\il{Krobu} & k{\textsubtilde{\'{ɔ}}} &  &  & \\
~~~~~~~~~~~~{Ega}\il{Ega} & ì-lō-gɓó & ì-lō-gɓó (< *li-kpo?) &  & \\
\lspbottomrule
\end{tabularx}
\end{table}

The Awikam-Alladian\il{Alladian} term for ‘one’ is definitely an innovation. 

The root *\textit{di} is attested in four branches out of five and thus is likely reconstructable at the Proto-Kwa\il{Proto-Kwa} level.

The forms given in the left column are more problematic. Each of them contains a velar consonant (the Potu form *\textit{ce} may have resulted from the palatalization of a velar before a front vowel, \textit{ce} \textit{<} \textit{*kue} – cf. Western Tano). 

Regular phonetic correspondences between these languages have not been established and therefore cannot be used for purposes of reconstruction. In any case, the following considerations might prove useful for the NC reconstruction. The inventory of forms attested in the eighty Kwa\il{Kwa} idioms may seem rather diverse. However, only two of them may be considered for the Proto-Kwa\il{Proto-Kwa} reconstruction, namely \textit{*di} and \textit{*k(p)o} (or the compound form \textit{*di-kpo} suggested by the Gbe\il{Gbe} (\textit{ɖe-kpo}) and Ega\il{Ega} (\textit{*li-gɓó}?) forms).

\subsubsection{‘Two’}%3.2.6.2.
\begin{table}
\caption{\label{tab:3:79}Kwa\il{Kwa} stems for `2'}


\begin{tabularx}{\textwidth}{lXXXl}
\lsptoprule

~ &   `2' & `2' & `2' & `2' \\
\midrule
{*Ga-}\il{Ga}{Dangme}\il{Dangme}   	& é-ɲ{\`{ɔ}}(n) &  &  & \\
{*Gbe}\il{Gbe}  			&  			&  			& è-ve/e-wè 			& \\
{*Ka-Togo}  				&  			& din 			&  			& bha\\
{*Na-Togo}  				& i-nyɔ 			&  			&  			& \\
\textit{*Nyo}\\
~~~~{*Agneby}				& a-ɲ{\textsubbar{ʊ}}/n{\={ʊ}} &  &  & {~}\\
~~~~{Attié}\il{Attié} 			&  			&  			& mwə(n) 			& {~}\\
~~~~{Awikam}   				& áɲ{\textsubtilde{\'{ɔ}}} &  &  & {~}\\
~~~~{Alladian}\il{Alladian}    		&  			& {\textsubtilde{ā}}yr{\`{ɛ}} &  & {~}\\
~~~~\textit{Potou-Tano}\\
~~~~~~~~{Potou}  			& n{\textsubbar{o}}{\textsubtilde{\'{ɔ}}} &  &  & {~}\\
~~~~~~~~\textit{Tano}\\
~~~~~~~~~~~~{Western} 			& a-ɲu(n) 			&  			&  			& {~}\\
~~~~~~~~~~~~\textit{Central}\\
~~~~~~~~~~~~~~~~{Akanic} 		& mie-nú 			&  &  & {~}\\
~~~~~~~~~~~~~~~~{Bia} 			& nu, ɲ{\`{ɔ}}(n) 			&  &  & {~}\\
~~~~~~~~~~~~{Guang}\il{Guang} 		& ɲ{\'{ɔ}} 			&  &  & {~}\\
~~~~~~~~~~~~{Krobu}\il{Krobu} 		& {\'{ɲ}}-ɲ{\textsubtilde{\'{ɔ}}} &  &  & {~}\\
~~~~~~~~~~~~{Ega}\il{Ega} 		& {\`{ɩ}}-ɲ{\`{ɔ}} &  &  & {~}\\
\lspbottomrule
\end{tabularx}
\end{table}
The only form reconstructable at the Proto-Kwa\il{Proto-Kwa} level is evidently \textit{*ɲɔ}.


 
\subsubsection{‘Three’ and ‘Four’}%3.2.6.3.
\largerpage
\begin{table}
\caption{\label{tab:3:80}Kwa\il{Kwa} stems for `3' and `4'}
 

\begin{tabularx}{\textwidth}{lQQQ}
\lsptoprule

 ~ & `3' & `4' & `4' \\
\midrule
{*Ga-}\il{Ga}{Dangme}\il{Dangme}   	& é-t{\~{ɛ}} &  & é-ɟw{\`{ɛ}}\\
{*Gbe}\il{Gbe}  			& è-t{\`{\~ɔ}} & è-n{\`{ɛ}} & \\
{*Ka-Togo}  				& ta & na/nɛ & \\
{*Na-Togo}  				& i-ta & i-na & \\
\textit{*Nyo}\\
~~~~{*Agneby}				& a-ti(N)/ri & a-n{\textsubtilde{í}}/la & jar\\
~~~~{Attié}\il{Attié} 			& ha(n) &  & dʒí(n) <* kɥe?\\
~~~~{Awikam}   				& {\textsubtilde{á}}z{\textsubtilde{á}} & àn{\textsubtilde{á}} & \\
~~~~{Alladian}\il{Alladian}    		& {\textsubtilde{ā}}{\textsubtilde{ò}} &  & {\textsubtilde{ā}}z{\`{ɔ}}\\
~~~~\textit{Potou-Tano}\\
~~~~~~~~{Potou}  			& ɗyá/je & ɗi/ni & \\
~~~~~~~~\textit{Tano}\\
~~~~~~~~~~~~{Western} 			& n-ha(n) & n-na(n) & \\
~~~~~~~~~~~~\textit{Central}\\
~~~~~~~~~~~~~~~~{Akanic} 		& mie-nsá(n) & náín & \\
~~~~~~~~~~~~~~~~{Bia} 			& sa(n) & na(n) & \\
~~~~~~~~~~~~{Guang}\il{Guang} 		& sa(n) & na(n) & \\
~~~~~~~~~~~~{Krobu}\il{Krobu} 		& {\'{n}}-s{\textsubtilde{á}} & {\'{n}}-n{\textsubtilde{á}} & \\
~~~~~~~~~~~~{Ega}\il{Ega} 		& {\`{ɩ}}-tà & {\`{ɩ}}-l{\`{ɛ}} & \\
\lspbottomrule
\end{tabularx}
\end{table}

Just as in the majority of the NC branches, the roots for ‘three’ and ‘four’ are the most persistent. Suggested Proto-Kwa\il{Proto-Kwa} reconstructions are *\textit{ta} and *\textit{na} respectively. 

\newpage  
\subsubsection{’Five’}%3.2.6.4.
\begin{table}
\caption{\label{tab:3:81}Kwa\il{Kwa} stems for `5'}


\begin{tabularx}{\textwidth}{lQrl}
\lsptoprule

& `5' & `5' & `5' \\
\midrule
{*Ga-}\il{Ga}{Dangme}\il{Dangme}   	&  & é-nù{\~{ɔ}} & \\
{*Gbe}\il{Gbe}  			& à-t{\'{\~ɔ}}{\~{ɔ}} &  & \\
{*Ka-Togo}  				& tu(N) &  & \\
{*Na-Togo}  				&  & i-no(N) & \\
\textit{*Nyo}\\
~~~~{*Agneby}				&  & o-ne & l{\`{ɔ}}h{\textsubtilde{\`{ʊ}}}, j{\^{e}}n\\
~~~~{Attié}\il{Attié} 			&  &  & bə(n)\\
~~~~{Awikam}   				&  & àɲú & \\
~~~~{Alladian}\il{Alladian}    		&  &  & {\={ɛ}}nrì\\
~~~~\textit{Potou-Tano}\\
~~~~~~~~{Potou}  			&  & n{\textsubbar{a}} & \\
~~~~~~~~\textit{Tano}\\
~~~~~~~~~~~~{Western} 			&  & n-nu(n) & \\
~~~~~~~~~~~~\textit{Central}\\
~~~~~~~~~~~~~~~~{Akanic} 		&  & nú{\'{m}} & \\
~~~~~~~~~~~~~~~~{Bia} 			&  & nu(n)/nu(m) & \\
~~~~~~~~~~~~{Guang}\il{Guang} 		&  & nu(n) & \\
~~~~~~~~~~~~{Krobu}\il{Krobu} 		&  & {\'{n}}-n{\textsubtilde{ù}} & \\
~~~~~~~~~~~~{Ega}\il{Ega} 		&  & ì-ŋwè & \\
\lspbottomrule
\end{tabularx}
\end{table}

The root *\textit{tan} (‘five’) is only traceable in the Left Bank languages. Another root, commonly attested in other languages (*\textit{nun}), is found in these languages as well. Both roots should be considered for the reconstruction (note that the former is comparable to the pertinent form reconstructed for Proto-Bantu\il{Proto-Bantu}). 

 \newpage 
\subsubsection{‘Six’}%3.2.6.5.
\begin{table}
\caption{\label{tab:3:82}Kwa\il{Kwa} stems for `6'}


\begin{tabularx}{\textwidth}{lXXXl}
\lsptoprule

& `6' & `6' & `6' & `6' \\
\midrule
{*Ga-}\il{Ga}{Dangme}\il{Dangme}   	&  & é-kpà &  & \\
{*Gbe}\il{Gbe}  			&  &  & à-d{\'{\~ɛ}}/z{\'{\~ɛ}} & \\
{*Ka-Togo}  				& golo/koro &  &  & \\
{*Na-Togo}  				& golo/kolo & ku &  & \\
\textit{*Nyo}\\
~~~~{*Agneby}				&  & hu(n) &  & \\
~~~~{Attié}\il{Attié} 			&  &  &  & mu(n)\\
~~~~{Awikam}   				&  &  &  & áwá\\
~~~~{Alladian}\il{Alladian}    		& {\={ɛ}}-wrè &  &  & \\
~~~~\textit{Potou-Tano}\\
~~~~~~~~{Potou}  			&  & kwa &  & \\
~~~~~~~~\textit{Tano}\\
~~~~~~~~~~~~{Western} 			&  &  & n-c{\'{ɪ}}{\`{ɛ}}/hí{\`{ɛ}} & \\
~~~~~~~~~~~~\textit{Central}\\
~~~~~~~~~~~~~~~~{Akanic} 		&  &  & sìá(n) & \\
~~~~~~~~~~~~~~~~{Bia} 			&  &  & sia(n) & \\
~~~~~~~~~~~~{Guang}\il{Guang} 		&  &  & siɛ(n) & \\
~~~~~~~~~~~~{Krobu}\il{Krobu} 		&  &  & {\'{n}}-s{\~{y}}{\textsubtilde{\={ɛ}}} & \\
~~~~~~~~~~~~{Ega}\il{Ega} 		&  &  &  & 5+1\\
\lspbottomrule
\end{tabularx}
\end{table}

The evidence presented in \tabref{tab:3:82} is inconclusive. At this stage our task is to process the complex Kwa\il{Kwa} data so that it can be compared to the evidence of other NC languages. In this respect, three provisional Kwa forms are noteworthy: \textit{*golo/kolo}, \textit{*kua,} and \textit{*ciɛ.} In any case, as the forms for ‘seven’ suggest, the Proto-Kwa\il{Proto-Kwa} term for ‘six’ was probably primary. 


 \newpage 
\subsubsection{‘Seven’} %3.2.6.6.
\begin{table}
\caption{\label{tab:3:83}Kwa\il{Kwa} stems and patterns for `7'}


\begin{tabularx}{\textwidth}{lXXXl}
\lsptoprule

& `7' & `7' & `7' & `7' \\
\midrule
{*Ga-}\il{Ga}{Dangme}\il{Dangme}   	& 6+1 &  &  & \\
{*Gbe}\il{Gbe}  			&  &  &  & 5+2, ‘hand’+2\\
{*Ka-Togo}  				& 6+1 &  &  & \\
{*Na-Togo}  				& 6+1 &  &  & \\
\textit{*Nyo}\\
~~~~{*Agneby}				& 6+1 &  & bu(n) & \\
~~~~{Attié}\il{Attié} 			&  & nson &  & \\
~~~~{Awikam}   				&  &  & {\'{ɛ}}by{\textsubtilde{\'{ɔ}}} & \\
~~~~{Alladian}\il{Alladian}    		&  &  & {\={ɛ}}bw{\textsubtilde{è}} & \\
~~~~\textit{Potou-Tano}\\
~~~~~~~~{Potou}  			& 6+1 &  &  & \\
~~~~~~~~\textit{Tano}\\
~~~~~~~~~~~~{Western} 			&  & n-c{\`{ʋ}}n &  & \\
~~~~~~~~~~~~\textit{Central}\\
~~~~~~~~~~~~~~~~{Akanic} 		&  & só(n) &  & \\
~~~~~~~~~~~~~~~~{Bia} 			&  & su(n) &  & \\
~~~~~~~~~~~~{Guang}\il{Guang} 		&  & súnɔ(n) &  & \\
~~~~~~~~~~~~{Krobu}\il{Krobu} 		&  & {\'{n}}-s{\^{o}} &  & \\
~~~~~~~~~~~~{Ega}\il{Ega} 		&  &  &  & 5+2\\
\lspbottomrule
\end{tabularx}
\end{table}

The forms presented in the table above point toward the pattern ‘6+1’ being used for the Proto-Kwa\il{Proto-Kwa} term for ‘seven’, whereas Proto-Nyo\il{Proto-Nyo} developed the primary term *\textit{sun}. 


\newpage 
\subsubsection{‘Eight’}%3.2.6.7.
\begin{table}
\caption{\label{tab:3:84}Kwa\il{Kwa} stems and patterns for `8'.} 


\begin{tabularx}{\textwidth}{lXXXXl}
\lsptoprule

& `8' & `8' & `8' & `8' & `8' \\
\midrule
{*Ga-}\il{Ga}{Dangme}\il{Dangme}   	&  &  &  &  & 6+2\\
{*Gbe}\il{Gbe}  			&  & e-ɲí & ‘hand’+3 &  & \\
{*Ka-Togo}  				& 4*2 & nsɛ/lɛ? &  &  & \\
{*Na-Togo}  				& 4PL &  &  &  & \\
\textit{*Nyo}\\
~~~~{*Agneby}				&  &  &  & è-pyè & wo(n)\\
~~~~{Attié}\il{Attié} 			& ma-4? &  &  &  & 10--2?\\
~~~~{Awikam}   				&  & {\`{ɛ}}ty{\'{ɛ}} &  &  & \\
~~~~{Alladian}\il{Alladian}    		&  & ēɥrì &  &  & \\
~~~~\textit{Potou-Tano}\\
~~~~~~~~{Potou}  			&  &  &  & ɓyá/gɓī & \\
~~~~~~~~\textit{Tano}\\
~~~~~~~~~~~~{Western} 			&  & m{\`{ɔ}}-k{\`{ʋ}}{\'{ɛ}} &  &  & à-n{\`{ɛ}}mr{\textsubtilde{\`{ɔ}}}\\
~~~~~~~~~~~~\textit{Central}\\
~~~~~~~~~~~~~~~~{Akanic} 		&  & twé/cué &  &  & \\
~~~~~~~~~~~~~~~~{Bia} 			&  & cʊɛ/twɛ &  &  & \\
~~~~~~~~~~~~{Guang}\il{Guang} 		&  & kwé/cwi &  &  & \\
~~~~~~~~~~~~{Krobu}\il{Krobu} 		&  & m{\`{ɔ}}-kw{\'{ɛ}} &  &  & \\
~~~~~~~~~~~~{Ega}\il{Ega} 		&  &  & 5+3 &  & \\
\lspbottomrule
\end{tabularx}
\end{table}

Based on the evidence attested in the table above, the Proto-Kwa\il{Proto-Kwa} term for ‘eight’ may be reconstructed as either primary (\textit{*kwe/} \textit{kye}) or derivative, in which case it must have been based on ‘four’ (*‘4PL’).


 
\newpage 
\subsubsection{‘Nine’}%3.2.6.8.
\begin{table}
\caption{\label{tab:3:85}Kwa\il{Kwa} stems and patterns for `9'}


\begin{tabularx}{\textwidth}{lXlXXXl}
\lsptoprule

& `9' & `9' & `9' & `9' & `9' & `9' \\
\midrule
{*Ga-}\il{Ga}{Dangme}\il{Dangme}   	&  &  &  &  &  & n{\`{\~ɛ}}{\'{\~ɛ}}(h{\'ũ}) \\
{*Gbe}\il{Gbe}  			& 8+1 &  & 5+4 &  &  & \\
{*Ka-Togo}  				& 8+1? &  & 10--1 &  &  & \\
{*Na-Togo}  				&  &  & 10--1 &  &  & \\
\textit{*Nyo}\\
~~~~{*Agneby}				&  & bare(-n) &  &  &  & \\
~~~~{Attié}\il{Attié} 			&  &  &  &  & ŋgwan & \\
~~~~{Awikam}   				&  & {\'{ɛ}}mr{\textsubtilde{\'{ɔ}}} &  &  &  & \\
~~~~{Alladian}\il{Alladian}    		&  & {\={ɛ}}mwr{\`{ɔ}} &  &  &  & \\
~~~~\textit{Potou-Tano}\\
~~~~~~~~{Potou}  			&  & ɓr{\`{ɔ}} &  &  &  & tr{\texthighriseu}\\
~~~~~~~~\textit{Tano}\\
~~~~~~~~~~~~{Western} 			&  & brúkú &  &  &  & puál{\'{ɛ}}h{\`{ʋ}}n\\
~~~~~~~~~~~~\textit{Central}\\
~~~~~~~~~~~~~~~~{Akanic} 		&  &  &  & n-kró{\'{n}} &  & \\
~~~~~~~~~~~~~~~~{Bia} 			&  &  &  & nkró{\'{n}} & {\`{n}}ɡ{\`{\~ɔ}}l{\`ã} & \\
~~~~~~~~~~~~{Guang}\il{Guang} 		&  &  &  &  &  & kpunɔ, sàng{\'{ɔ}}{\'{ɔ}}ʔ\\
~~~~~~~~~~~~{Krobu}\il{Krobu} 		&  &  &  &  & \mbox{{\`{ŋ}}-gr{\`{ɔ}}{\textsubtilde{ā}}} & \\
~~~~~~~~~~~~{Ega}\il{Ega} 		&  &  & 5+4 &  &  & \\
\lspbottomrule
\end{tabularx}
\end{table}

This is the hardest form to interpret. A rare pattern ‘8+1’ is attested in the Left Bank languages. In contrast to this, the Togo pattern is ‘10--1’, while the Nyo term (*\textit{brɔ}/\textit{mrɔ}) is ‘primary’. The latter is probably connected to the term for ‘ten’, although this connection does not necessarily imply a derivation (’10--1’) and could be explained by analogy.  All three forms/patterns are considered for reconstruction. 


\newpage 
 
\subsubsection{‘Ten’}%3.2.6.9.
\begin{table}
\caption{\label{tab:3:86}Kwa\il{Kwa} stems for `10'}


\begin{tabularx}{\textwidth}{lXXXXXl}
\lsptoprule

& `10' & `10' & `10' & `10' & `10' & `10' \\
\midrule
{*Ga-}\il{Ga}{Dangme}\il{Dangme}   	&  &  &  &  &  & ɲ{\`{ɔ}}ŋmá\\
{*Gbe}\il{Gbe}  			& e-wó & *bula &  &  &  & \\
{*Ka-Togo}  				& fo/wo & bula &  &  & te & \\
{*Na-Togo}  				& fo &  & ɗu &  & təb & \\
\textit{*Nyo}\\
~~~~{*Agneby}				&  &  &  & diw/liw &  & nɛ(n)<5PL?\\
~~~~{Attié}\il{Attié} 			&  &  &  &  &  & kɛŋ\\
~~~~{Awikam}   				&  &  & èjú &  &  & \\
~~~~{Alladian}\il{Alladian}    		& {\={ɛ}}-và &  &  &  &  & \\
~~~~\textit{Potou-Tano}\\
~~~~~~~~{Potou}  			& wɔ &  &  &  &  & \\
~~~~~~~~\textit{Tano}\\
~~~~~~~~~~~~{Western} 			&  & ò-bùl{\textsubtilde{ú}} &  & è-dí &  & \\
~~~~~~~~~~~~\textit{Central}\\
~~~~~~~~~~~~~~~~{Akanic} 		&  &  & dú &  &  & \\
~~~~~~~~~~~~~~~~{Bia} 			&  & bulu &  &  &  & \\
~~~~~~~~~~~~{Guang}\il{Guang} 		&  &  & du &  &  & \\
~~~~~~~~~~~~{Krobu}\il{Krobu} 		&  & brú &  &  &  & \\
~~~~~~~~~~~~{Ega}\il{Ega} 		&  &  & ì-zù &  &  & \\
\lspbottomrule
\end{tabularx}
\end{table}

Isolated forms are attested in Ga\il{Ga}-Dangme\il{Dangme} and Attié\il{Attié}. The root \textit{tə(b)} is traceable in the Ghana–Togo Mountain languages (Togo-remnant) and is not found elsewhere. Thus we are dealing with another isogloss suggesting that these languages belong to the same branch. The stem \textit{*du} supported by R. Blench could be proposed for Proto-Kwa\il{Proto-Kwa}. This stem is indeed attested in the majority of the groups that do not belong to the Left Bank languages (including Na-Togo).

The stem *\textit{bula} (Left Bank)/*\textit{bulu} (Tano) is distributed fairly evenly. 

Finally, a Niger-Congo root reflected in Kwa\il{Kwa} as \textit{*fo/wo} can be reconstructed in a number of languages.

\newpage 
 
\subsubsection{‘Twenty’}%3.2.6.10.
\begin{table}
\caption{\label{tab:3:87}Kwa\il{Kwa} stems and patterns for `20'}


\begin{tabularx}{\textwidth}{l Xl@{~}l@{~}l@{~}l@{~}l}
\lsptoprule

& `20' & `20' & `20' & `20' & `20' & `20' \\
\midrule
{*Ga-}\il{Ga}{Dangme}\il{Dangme}   	& 10*2 &  &  &  &  & \\
{*Gbe}\il{Gbe}  			& 10*2 & ko &  &  &  & \\
{*Ka-Togo}  				& 10*2 &  &  &  &  & \\
{*Na-Togo}  				& 10*2 & ā-kōō & dìkpìlìn &  &  & ɔ-ɖɔ(n) (<10?) \\
\textit{*Nyo}\\
~~~~{*Agneby}				& ‘hand’ (bra)*2? &  & li-kŋ &  &  & \\
~~~~{Attié}\il{Attié} 			& ‘hand' (bwa?)*2? &  &  &  &  & \\
~~~~{Awikam}   				&  &  &  & {\`{ɛ}}-v{\'{ɛ}} &  & \\
~~~~{Alladian}\il{Alladian}    		&  & *{\={ɛ}}kòɥì &  & {\={ɛ}}-ɥá &  & \\
~~~~\textit{Potou-Tano}\\
~~~~~~~~{Potou}  			&  &  &  &  & p{\textsubbar{ɛ}} & \\
~~~~~~~~\textit{Tano}\\
~~~~~~~~~~~~{Western} 			&  &  &  &  & {\'{ɛ}}-fɪ(n) & \\
~~~~~~~~~~~~\textit{Central}\\
~~~~~~~~~~~~~~~~{Akanic} 		& 10*2 &  &  &  &  & \\
~~~~~~~~~~~~~~~~{Bia} 			& 10*2 &  &  &  &  & \\
~~~~~~~~~~~~{Guang}\il{Guang} 		& 10*2 & ko? &  &  &  & \\
~~~~~~~~~~~~{Krobu}\il{Krobu} 		& \mbox{à-brūā{\textsubtilde{\'{ɛ}}}} (10*2?) &  &  &  &  & \\
~~~~~~~~~~~~{Ega}\il{Ega} 		&  &  &  &  &  & ú-glū\\
\lspbottomrule
\end{tabularx}
\end{table}

The pattern ‘10*2’ attested in the majority of the branches. The root *\textit{ko} is also to be taken.


  
 
 
\subsubsection{‘Hundred’ and ‘thousand’}%3.2.6.11.
\begin{table}
\caption{\label{tab:3:88}Kwa\il{Kwa} stems and patterns for `100' and `1000'}


\begin{tabularx}{\textwidth}{l XXllXl}
\lsptoprule

& `100' & `100' & `100' & `100' & `1000' & `1000' \\
\midrule
{*Ga-}\il{Ga}{Dangme}\il{Dangme}   	& làfá &  & ò-há &  & à-kpé & \\
{*Gbe}\il{Gbe}  			&  &  &  & 40*2+20 & à-kpé & \\
{*Ka-Togo}  				& lafa? &  &  &  & a-kpe & \\
{*Na-Togo}  				& lofa & 20*5 & u-ga &  & a-kpi & pim? \\
\textit{*Nyo}\\
~~~~{*Agneby}				&  & 20*5 & ja &  & a-kpi & \\
~~~~{Attié}\il{Attié} 			&  &  & ja &  & a-kpi & \\
~~~~{Awikam}   				&  &  &  & àkpá {\textprimstress}-2 &  & \\
~~~~{Alladian}\il{Alladian}    		&  & 20*5 &  &  &  & \\
~~~~\textit{Potou-Tano}\\
~~~~~~~~{Potou}  			&  &  & ya &  &  & \\
~~~~~~~~\textit{Tano}\\
~~~~~~~~~~~~{Western} 			&  &  & {\`{ɛ}}-vá/{\`{ɛ}}-y{\v{a}} & átá & a-kpi & \\
~~~~~~~~~~~~\textit{Central}\\
~~~~~~~~~~~~~~~~{Akanic} 		&  &  & ɔ-ha &  &  & a-pí{\'{m}}\\
~~~~~~~~~~~~~~~~{Bia} 			&  &  & ya &  & a-kpi & \\
~~~~~~~~~~~~{Guang}\il{Guang} 		& lafa &  &  &  & kpi(N) & pim\\
~~~~~~~~~~~~{Krobu}\il{Krobu} 		&  &  & y{\v{a}} &  &  & \\
~~~~~~~~~~~~{Ega}\il{Ega} 		&  & 20*5 &  &  &  & \\
\lspbottomrule
\end{tabularx}
\end{table}

In addition to the pattern ‘20*5’, the roots \textit{lafa}/\textit{lofa} and *\textit{ya}/\textit{ja} (Nyo) are reconstructable for ‘hundred’. The latter may be etymologically related to *\textit{ga}/\textit{ha}.

The term for ‘thousand’ is commonly attested as *\textit{a-kpi}. Its less common by-form is *\textit{pim}.

\newpage 
\largerpage
\tabref{tab:3:89} lists \textstylegtbafwordclickable{provisional} Proto-Kwa\il{Proto-Kwa} reconstructions based on the evidence discussed above.

\begin{table}[b!]
\caption{\label{tab:3:89}Proto-Kwa\il{Proto-Kwa} numeral system (*)}
\begin{tabularx}{\textwidth}{lXrl}
\lsptoprule
{1} & di-kpo & {7} & 6+1\\
{2} & ɲɔ, **di? & {8} & 4PL, kwe/kye\\
{3} & ta & {9} & 10--1?\\
{4} & na & {10} & fo/wo, bula, du\\
{5} & nu(n), ton & {20} & 10*2, ko\\
{6} & golo/kolo, kua, ciɛ & {100} & 20*5, lofa, ja/gya?\\
&  & {1000} & kpi, pim\\
\lspbottomrule
\end{tabularx}\\
\raggedright\footnotesize
The remaining roots and patterns are probably innovations that developed separately within a branch/language. They may help to adjust the internal classification of the Kwa\il{Kwa} languages. 
\end{table}

 
\clearpage  
\section{Ijo}%3.3

According to traditional classification, the Ijo family is comprised of the Ijaw\il{Ijaw} languages and the Defaka\il{Defaka} language. Some scholars express doubts as to whether the latter indeed belongs to this family. According to Roger Blench, ``The Ijo languages constitute a well-founded group, but the membership of Defaka (constituting Ijoid) remains problematic. Defaka has numerous external cognates and might be an isolate or independent branch of Niger-Congo which has come under Ijo influence''  \citep{Blench1993NigerCongo}. 

Ijaw\il{Ijaw} languages consist of the Eastern and the Western groups (the latter is sometimes called Central). 

The following reconstruction is based on the evidence of all three Ijo branches (\tabref{tab:3:90}).

\begin{table}
\caption{\label{tab:3:90}Proto-Ijo numeral system}


\begin{tabularx}{\textwidth}{XXXXl}
\lsptoprule

~ 		& {Defaka}\il{Defaka} 				& {*East} 		& {*West} 		& {**Ijo} 		\\
\midrule                                                                                                                                                                
{1} {(qualifying)} 		& ɡbérí 		& ɡbérí 		&? 		&? 							\\
{1} {(counting)} 			&? 			& {\`{ŋ}}g{\`{ɛ}}i 	& k{\`{ɛ}}nɪ 		& {*n-k{\`{ɛ}}ni} 			\\
{1} {in} {6} {(5+1)}& – 			& die/ie 		& die/zie 		& {*die} 				\\
{2} 					& mààmà 		& màmì 			& maamʊ 		& {*mamV} 				\\
{3} 					& táátó 		& tárú 		& t{\v{a}}rʊ 		& {*tató} 				\\
{4} 		 			& n{\'{ɛ}}ì 		& i-ne{\~{i}} 		& n{\'{ɛ}}ín/nóín 		& {*n{\'{ɛ}}ín} 			\\
{5} 					& túún{\`{ɔ}} & s{\'{ɔ}}n{\'{ɔ}} 	& s{\~{ɔ}}n{\~{ɔ}}-r{\~{ɔ}} 		& {*tún{\'{ɔ}}} 	\\
{6} 		& màànɡò 		& 5+1 		& 5+1 		& {*5+1}\\
{7} 		& 5+2 		& 5+2 		& 5+2 		& {*5+2}\\
{8} 		& 5+3 		& 4+4 		& 4+4 		& {*4+4}\\
{9} 		& 5+4 		& 5+4 		& 5+4? 		& {*5+4}\\
{10} 		& wóì 		& ójí/àtìé 		& ójí 		& {*(w)ójí}\\
{15} 		& 10+5 		& jìé 		& dié 		& {*dié}\\
{20} 		& síì 		& sí 		& síí 		& {*síí} \\
\lspbottomrule
\end{tabularx}
\end{table}

Both qualifying and counting terms for ‘one’ are attested in the Eastern Ijo languages (e.g. in Ibani\il{Ibani}). The Defaka\il{Defaka} form may be a borrowing. An unexplained allomorph for ‘one’ is attested as a part of the term for ‘six’ in Ijaw\il{Ijaw} (?). 

The root for ‘two’ (*\textit{mam}) is an Ijo innovation. It has no parallels outside this language family. Its phonetic similarity to several other forms is a mere coincidence, e.g. \textit{ma}- in the {Jaad}\il{Jaad}{ (Atlantic)} \textit{m}{\textit{aaɛ}}{ does not belong to the root and can be explained as a class prefix. The lexical meaning ‘twin, pair’ (as attested in Nembe}\il{Nembe}{ (East) according to \citep{Kaliai1964}) may underlie the Ijo term. However, no reliable parallels for this term with the meaning ‘twin, pair’ are establishable in NC.}

{The root for ‘three’ is apparently of NC origin, with its most archaic form attested in Defaka}\il{Defaka}{.}

{The term for ‘four’ is undoubtedly a reflex of the NC root.}

{The term for ‘five’ probably goes back to the NC root} {\textit{*tan(o)}}{. As in the case of ‘three’, its most archaic form is found in Defaka}\il{Defaka}{.}

{The terms for ‘six’, ‘seven’, and ‘nine’ follow the common patterns (‘5+1’, ‘5+2’, and ‘5+4’ respectively).} 

{The Ijaw}\il{Ijaw}{ term for ‘eight’ must have derived from ‘four’ by means of partial reduplication (}\textit{*ni-n{\'{ɛ}}ín}). This pattern is reconstructable on the Proto-NC\il{Proto-NC} level and will be discussed at length in the next chapter.

{A specific counting term for ‘ten’ is reconstructable in the Eastern Ijo languages (}{\textit{*}}\textit{àtìé}{). The Defaka}\il{Defaka}{ form is comparable to those found in the Ijaw}\il{Ijaw}{ languages.}

A special form for ‘fifteen’ is reconstructable in Ijaw\il{Ijaw} (\textit{*dié}), cf. e.g. the Nembe\il{Nembe} evidence: \textit{dì}\textit{é-}\textit{è}\textit{sí} `300’ (=‘15*20’). This form may go back to Ijaw \textit{*ɗ{\'{ɩ}}{\`{ɛ}}} ‘divide; separate into parts; split or break up into parts; share’, ‘distribute, donate’, cf. Nembe \textit{ɗ{\`{ɩ}}{\`{ɛ}}}, Ibani\il{Ibani} \citep{Koelle1963} \textit{dìè-,} \textit{dìé.}

{As in a number of other languages that belong to different families within NC, a special form is attested for the term for ‘twenty’ (}\textit{*síí}{). The term itself has several functions. It serves as a basis for a number of other terms for tens (also in Defaka}\il{Defaka}{), e.g. ‘40=20*2’, ... ‘100=20*5’. The Ijaw}\il{Ijaw}{ terms for 16--19 are based on it as well, e.g. ‘16=20–4’, etc.} 

\section{Kru}%3.4

Our analysis of the Kru numerals is based on nearly forty sources representative of five major groups and eleven major subgroups of the family. Preliminary reconstructions of the pertinent numerical terms (by sub-group) are represented in commented tables below. 


\subsection{‘One’, ‘Two’ and ‘Three’} %3.4.1.

As in the majority of the NC languages the term for ‘three’ is the most persistent: the root \textit{*taa(n)} can be reliably reconstructed for Proto-Kru\il{Proto-Kru}. 

\begin{table}
\caption{\label{tab:3:91}Kru stems for `1'-`3'}


\begin{tabularx}{\textwidth}{Xllllll}
\lsptoprule
& `1' & `1' & `1' & `2' & `2' & `3' \\
\midrule
Aizi\il{Aizi} &  & m{\textsubbar{u}}m{\textsubbar{ɔ}} & yre & i-ʃɩ &  & i-ta\\
\textit{Eastern}\\
~~~~Bakwe\il{Bakwe}/Wané\il{Wané} & ɗ{\^{o}} &  &  & s{\^{ɔ}} &  & ta\\
~~~~Bete\il{Bete}/Godié\il{Godié} &  & ɓlo/gbolo &  & sɔ &  & ta\\
~~~~Dida\il{Dida}/Neyo\il{Neyo} &  & bolo &  & s{\'{ɔ}} &  & ta\\
~~~~Kodia\il{Kodia} &  & ɡbɤlɤ/ɓɤlɤ &  & sɔː &  & taː\\
Kuwa\il{Kuwa} & dee &  &  & s{\~{ɔ}}r &  & t{\~{a}}{\`ã}\\
Seme\il{Seme} & dyu{\~{ɔ}} &  & by{\'{\~e}}{\~{e}} &  & n{\~{i}} & tyáār \\
\textit{Western}\\
~~~~Bassa\footnotemark{}\il{Bassa} & doo & (g)boo? &  & s{\'{\~ɔ}} &  & t{\~{a}}\\
~~~~Grebo\footnotemark{}\il{Grebo} & do(o) &  &  & s{\~{\v{ɔ}}} & hw{\~{ə}}/h{\~{ɔ}} & taa(n)\\
~~~~Klao\il{Klao}/Tajuasohn\il{Tajuasohn} & do &  &  & son &  & tan\\
~~~~Wee\footnotemark{} & due/too &  &  & sɔn &  & taan\\
\lspbottomrule
\end{tabularx}
\end{table}
\addtocounter{footnote}{-3}
\stepcounter{footnote}\footnotetext{ Bassa\il{Bassa}, Dewoin\il{Dewoin}, Gbii\il{Gbii}.}
\stepcounter{footnote}\footnotetext{ Grebo\il{Grebo}, Krumen\il{Krumen}, Glio-Oubi\il{Glio-Oubi}.}
\stepcounter{footnote}\footnotetext{ Wee is a Western Kru group  which includes (among other languages)  Sapo\il{Sapo}, Krahn\il{Krahn}, Nyabwa\il{Nyabwa}, Wobe\il{Wobe}.}


The same is applicable to the root for ‘two’ reconstructed as \textit{*so(n)} in Proto-Kru\il{Proto-Kru} (isolated forms are attested in the Seme\il{Seme} and Grebo\il{Grebo} sub-groups only). It should be noted that in general the Seme numeral system is peculiar in many respects. These peculiarities (e.g. Seme being the only language with a full set of primary terms covering the sequence from ‘one’ to ‘ten’) may be due to the isolated status of the language. In his recent article entitled “Le sèmè/siamou n’est pas kru” Vogler argues that Seme is not a Kru language (see \citealt{Vogler2015}). On the basis of a comparison between Kru, Gur and Mande (Samogo) morphology and lexicon he concludes that Seme is either remotely related to the Mande languages or represents a separate branch of Niger-Congo. As we hope to demonstrate below, Seme shows systematic correspondences with neither Kru nor Mande (including the contact Mande languages – Samogo and Jowulu\il{Jowulu}).

‘One’. It is likely that the root \textit{*do} should be reconstructed on the Proto-Kru\il{Proto-Kru} level. However, there is enough evidence for reconstructing the alternative root \textit{*(g)bolo.}

 
\subsection{‘Four’ and ‘Five’}%3.4.2.
\begin{table}
\caption{\label{tab:3:92}Kru stems for `4' and `5'}
\small
\begin{tabularx}{\textwidth}{l lll lQl}
\lsptoprule
              & `4' & `4' & `4' & `5' & `5' & `5' \\
\midrule  
 Aizi\il{Aizi} &  &  & yeɓi & yu-gbo &  & \\
\textit{Eastern}\\
~~~~Bakwe\il{Bakwe}/Wané\il{Wané} &  & hɪ{\~{ɛ}}⁴ & mɾ{\={ɔ}}ː & ɡ͡b{\`{ə}}{\={ə}}, ŋʷ{\~{u}} &  & \\
~~~~Bete\il{Bete}/Godié\il{Godié} &  &  & mʊ-wana & gbu/gbi &  & \\
~~~~Dida\il{Dida}/Neyo\il{Neyo} & na &  &  & gb{\'{ɪ}} &  & \\
~~~~Kodia\il{Kodia} & na &  &  & ⁿɡbɤ &  & \\
Kuwa\il{Kuwa} & ɲìj{\`{ɛ}}hɛ &  &  &  &  & wày{\`{ɔ}}ɔ\\
Seme\il{Seme} &  &  & yur &  &  & kw{\={\~{ɛ}}}l\\
\textit{Western}\\
~~~~Bassa\il{Bassa} & h{\`ĩ}-nyɛ(n) &  &  &  & h-mm & \\
~~~~Grebo\il{Grebo} &  & hɛn &  & gbə & mm & hun\\
~~~~Klao\il{Klao}/Tajuasohn\il{Tajuasohn} & nyì{\`{ɛ}} & hɛn &  &  & mù, hoom? (< m?) & \\
~~~~Wee & nyìɛ &  &  &  & mm & \\
\lspbottomrule
\end{tabularx}
\end{table}

The forms for ‘four’ in the left column apparently are the reflexes of the NC root that is preserved in its archaic form *\textit{na} in Eastern Kru, whereas in Western Kru it changes into \textit{nyì{\`{ɛ}}}.

Two major forms are observable for ‘five’, namely \textit{*gbə/} \textit{gbo} and *\textit{mm} (Western).


\subsection{‘Six’ to ‘Nine’} %3.4.3.
\begin{table}
\caption{\label{tab:3:93}Kru stems and patterns for `6'-'9'}
\small

\begin{tabularx}{\textwidth}{llllllll@{}ll@{}Q} 
\lsptoprule
  & `6' & `6' & `7' & `7' & `8' & `8' & `8' & `9' & `9' & `9' \\
\midrule 
Aizi\il{Aizi} &  & fɔ & fri+2 &  &  &  & patɛ &  &  & fi\\
\textit{Eastern}\\
Bakwe\il{Bakwe}/Wané\il{Wané} & 5+1 &  & 5+2 &  & 5+3 &  &  & 5+4 &  & \\
~~~~Bete\il{Bete}/Godié\il{Godié} & 5+1 &  & 5+2 &  & 5+3 &  &  & 5+4 &  & \\
~~~~Dida\il{Dida}/Neyo\il{Neyo} & 5+1 &  & 5+2 &  & 5+3 &  &  & 5+4 &  & \\
~~~~Kodia\il{Kodia} & 5+1 &  & 5+2 &  & 5+3 &  &  & 5+4 &  & \\

Kuwa\il{Kuwa} & 5+1 &  & 5+2 &  & 5+3 &  &  & 5+4 &  & \\
Seme\il{Seme} &  & kp{\={\~{a}}}â &  & k{\={\~{i}}}{\^{i}} &  &  & kpr{\={ɛ}}{\^{n}} &  &  & k{\={ɛ}}l/kal\\
\textit{Western} \\
~~~~Bassa\il{Bassa} & 5+1 &  & 5+2 &  & 5+3 &  &  & 5+4 &  & \\
~~~~Grebo\il{Grebo} & 5+1 &  & 5+2 &  & 5+3 &  &  & 5+4 &  & \\
~~~~Klao\il{Klao}/Tajuasohn\il{Tajuasohn} & 5+1 &  & 5+2 &  &  & 4PL &  &  & 10--1 & \\
~~~~Wee & 5+1 &  & 5+2 &  & 5+3 &  &  & 5+4 &  & \\
\lspbottomrule
\end{tabularx}
\end{table}

It is immediately apparent that these numerals already followed the pattern ‘5+X’ in Proto-Kru\il{Proto-Kru}. As noted above, the Seme\il{Seme} forms are innovations.

\newpage 

\subsection{‘Ten’ and ‘Twenty’} %3.4.4.

% THIS TABLE MOVED HERE OUT OF CHRONOLOGICAL ORDER
\begin{table}[b]
\caption{\label{tab:3:96}Proto-Kru\il{Proto-Kru} numeral system (*)}
\small
\begin{tabularx}{\textwidth}{lQrQ}
\lsptoprule
1 & do, (g)bolo & 7 & 5+2\\
2 & so(n) & 8 & 5+3\\
3 & taa(n) & 9 & 5+4\\
4 & na & 10 & pu, kʊgba?\\
5 & gbə/gbo, mm & 20 & golo\\
6 & 5+1 & 100 & 20*5\\
&  & 1000 & 400*2+200\\
\lspbottomrule
\end{tabularx}
\end{table}



\begin{table}
\caption{\label{tab:3:94}Kru stems for `10' and `20'}


\begin{tabularx}{\textwidth}{l QQQll} 
\lsptoprule
  & `10' & `10' & `20' & `20' & `20' \\
\midrule 
 Aizi\il{Aizi} & bɔ &  & gu &  & \\
\textit{Eastern} \\
~~~~Bakwe\il{Bakwe}/Wané\il{Wané} & p{\`{ʊ}}, bu? &  & ɡr{\`{ʊ}}, ɡᵓlɔ &  & \\
~~~~Bete\il{Bete}/Godié\il{Godié} &  & k{\'{ʊ}}gba & ɡwl{\'{ʊ}}/ɡɔlɔ &  & \\
~~~~Dida\il{Dida}/Neyo\il{Neyo} &  & k{\'{ʊ}}gba & ɡl{\'{ʊ}}/ɡóló &  & \\
~~~~Kodia\il{Kodia} &  & kʊgba & {\r{ɡ}}alo &  & \\
Kuwa\il{Kuwa} &  & kowaa &  & 10*2 & \\
Seme\il{Seme} & fu &  &  &  & kār \\
\textit{Western} \\
~~~~Bassa\il{Bassa} & ɓaɖa-bùè, puuɛ, vu &  &  & <10 & \\
~~~~Grebo\il{Grebo} & pu &  & ɡōrō/wl{\`{ʊ}} &  & \\
~~~~Klao\il{Klao}/Tajuasohn\il{Tajuasohn} & pue/punn &  & wlòh-2 &  & quilar-2 \\
~~~~Wee & pue/bue &  & ɡwlʊ-2 &  & kwela 2\\
\lspbottomrule
\end{tabularx}
\end{table}

The root \textit{kʊgba} is attested beside the common NC root for ‘ten’ (\textit{*pu/fu}) in Eastern and Kuwa\il{Kuwa}. The root for ‘twenty’ is attested as \textit{golo} in both Eastern and Western.


\subsection{‘Hundred’ and ‘Thousand’}%3.4.5.
\begin{table}
\caption{\label{tab:3:95}Kru stems and patterns for `100' and `1000'}


\begin{tabularx}{\textwidth}{lX lllll} 
\lsptoprule
& `100' & `100' & `1000' & `1000' & `1000' \\
\midrule 
Aizi\il{Aizi} &  & juyugbo &  &  & \\
\textit{Eastern} \\
~~~~Bakwe\il{Bakwe}/Wané\il{Wané} & 20*5 &  & 400*2+20*10 &  & \\
~~~~Bete\il{Bete}/Godié\il{Godié} & 20*5 &  & 400*2+200 &  & \\
~~~~Dida\il{Dida}/Neyo\il{Neyo} & 20*5 &  & 400*2+200 &  & \\
~~~~Kodia\il{Kodia} &  &  &  &  & \\
Kuwa\il{Kuwa} &  & k{\`{ɔ}}lɛh? &  & 100*10 & \\
Seme\il{Seme} & 20*5 &  &  &  & litː `goat one'\\
\textit{Western} \\
~~~~Bassa\il{Bassa} & 20*5 &  &  &  & borrowed\\
~~~~Grebo\il{Grebo} & 20*5 &  &  &  & borrowed\\
~~~~Klao\il{Klao}/Tajuasohn\il{Tajuasohn} & 20*5 &  &  &  & borrowed\\
~~~~Wee & 20*5 &  &  &  &? \\
\lspbottomrule
\end{tabularx}
\end{table}

All Kru sub-groups are characterized by the lack of a primary term for ‘hundred’. 

The form for ‘thousand’ in Western Kru was borrowed from the Mande languages. A primary term for ‘400’ (*\textit{dwi}) that developed in Eastern Kru served as the basis for a rare pattern for ‘thousand’ attested in these languages (‘400*2+200’).

The reconstruction of the Proto-Kru\il{Proto-Kru} numeral system is given in \tabref{tab:3:96}.


\newpage 
\section{Kordofanian}%3.5
\largerpage
The evidence of about twenty Kordofanian languages does not permit reconstructing the Proto-Kordo\-fa\-nian\il{Proto-Kordofanian} numeral system (assuming that Proto-Kordofanian existed). Comprehensive data for each of the four major groups is represented below (\tabref{tab:3:97}). Forms and patterns traceable in at least two groups are in bold. The forms are grouped within the lines in a more or less ad hoc manner, e.g. there is no special reason to believe that Talodi\il{Talodi} \textit{*lu(k)/} \textit{li(k)} ‘one’ corresponds to the forms with initial \textbf{t-/ʈ-} attested in other groups.


\begin{table}[b]
\caption{\label{tab:3:97}Kordofanian numerals 1--5}
\begin{tabularx}{\textwidth}{lQQ@{}QQQ}
\lsptoprule
\footnotesize
~ & {*Heiban}\il{Heiban} & {*Katla}\il{Katla} & {*Rashad} & {*Talodi}\il{Talodi} & {*Kordofanian}\\
\midrule
1 & kwɛ-(ʈ)ʈɛ(k) & ʈí-ʈʌk & -tta & lu(k)/li(k) & \textbf{ʈe(k)/lu(k)}\\
1 & ŋɔ-(ʈ)ʈɔ & ʌ-ʈeen/ʈɪɪn &  &  & \textbf{ʈɔ(n)}\\
1 & *-lel? &  &  & tleidi & \textbf{lel/led?}\\
2 &  & cik/heek & (k)ko(k) &  & \textbf{kok/kek/cik}\\
2 & -can~/-ɽan, rɔm &  &  & we-ɽʌk/-tta & (can/ɽan, rak, rɔm)\\
3 & tɔɽɔl/ʈeɽel & ʈʌʈ & tta & wa-ʈʈak & \textbf{tat/t{\`{ə}}ɽ/ʈak}\\
3 & -ɽɪcɪn/-ɡɪtʃɪn & i-hwʌy &  &  & (ɽitin/ɽicin, hwʌy)\\
4 & k(w)ɔ-ɽɔŋɔ/ma-ɽŋan/-rlon/-ɬɽʊ &  & ya-rem/wa-rʊm & -ɽandɔ & \textbf{-ɽɔŋ/-ɽandɔ/-ranto/-rʊm?} \\
4 &  & ʌ-ɡʌlʌm/i-hʌlʌm &  & kekka & (-ɡ{\'{ʌ}}l{\`{ʌ}}m, kekka)\\
5 & tʊ-dìní/-ðɛnɛ & i-duliin &  &  & \textbf{dinin/dulin?} \\
5 & ŋer-/ɲer- &  & *ɲer- &  & \textbf{ŋer-/ɲer-}\\
5 &  & ɟɔ-ɡbəlɪn & wʊ-ram, ma & ‘hand'-‘1', ki-liəgum & ('hand', ...)\\
\lspbottomrule
\end{tabularx}
\end{table}
\begin{table}[t]
\caption{\label{tab:3:97b}Kordofanian numerals >5}
\begin{tabularx}{\textwidth}{rp{15mm}QQQQ}
\lsptoprule
\footnotesize
~ & {*Heiban}\il{Heiban} & {*Katla}\il{Katla} & {*Rashad} & {*Talodi}\il{Talodi} & {*Kordofanian}\\
\midrule
6 & 5+1 & <5 & \mbox{ɲere(-r/-l/-y)} (< *5+1?) & 5+1 & \textbf{5+1}\\
6 & 3+3? 3\textsc{pl} &  &  &  & (3+3)\\
7 & 5+2 & 5+2 & 5+2 & 5+2 & \textbf< A{5+2}\\
7 & 4+3 & 3PL+1 &  &  & (4+3, 3PL+1)\\
8 & duuba(ŋ) &  & dubba/tuppa &  & \textbf{dubba}\\
8 & 5+3, 4 redupl.? &  &  & 5+3, 4 redupl. & \textbf{5+3,} 4 \textbf{redupl.}\\
8 & bɔ & {\textsubbridge{t}}{\'{ʌ}}ŋɡ{\`{ɪ}}l/{\textsubbridge{t}}ɪŋɛrɛy &  &  & (bɔ, ʈəŋi-)\\
9 & 10--1 & 10--1 & 10--1 &  & \textbf{10--1}\\
9 & 5+4 & ɟ{\'{ʌ}}lb{\`{ʌ}}ʈ{\'{ɪ}}n (<5?) &  & 5+4 & \textbf{5+4}\\
10 & di/ɗi/ri & *{\textsubbridge{t}}ʌʌ, ɔ-rɔ & kʊ-man (5PL) & ma-tu(l) &? \\
10 &  & rakpac, i-hedʌkun & fəŋən (fə-ŋən?) & tiəɽum, {\textsubbridge{n}}ipɽa, ɡurruŋ) &? \\
20 & 10*2 & 10*2 & 10+10 & 10*2 & 10*2\\
20 & tuɽí (‘grain'), \mbox{`big~figure'} &  &  & ‘body', (a-rial, a-(na)ttu) & ('body', …? )\\
100 & 20*5,\newline < Arabic\il{Arabic} & 10*10 & 10*10 & 10*10, 20*5 & 10*10, 20*5\\
1000 & Arabic,\newline 20*2*10 & absent & 10*10*10 & ɑ-ðɑɾ &? \\
\lspbottomrule
\end{tabularx}
\end{table}

 The systematic presence of the final velar -\textbf{k} in some of the terms can also be found in the Atlantic languages (especially in North Atlantic).

The term for ‘ten’ appears in numerous forms in the Kordofanian languages, which is rare. At the same time, no root for ‘ten’ is represented in at least two languages simultaneously. Moreover, nearly every language in a group has its own term for ‘ten’.

A primary term for ‘eight’ is distinguishable\footnote{I used data from the following Kordofanian languages and dialects: Aceron, Dagik\il{Dagik}, Heiban\il{Heiban}, Jomang\il{Jomang}, Katla\il{Katla}, Koalib\il{Koalib}, Lafofa\il{Lafofa}, Laro\il{Laro}, Logol\il{Logol}, Lumun\il{Lumun}, Moro\il{Moro}, Nding\il{Nding}, Orig\il{Orig}, Rere\il{Rere}, Shirumba\il{Shirumba}, Tagoi\il{Tagoi}, Talodi\il{Talodi}, Tegali\il{Tegali}, Tegem\il{Tegem}, Tima\il{Tima}, Tira\il{Tira}, Tocho\il{Tocho}, Utoro\il{Utoro}, Warnang\il{Warnang}.} in the Heiban\il{Heiban} and Rashad languages.

 
\section{Adamawa} %3.6

Adamawa is the most divergent of the NC families. The variety of numeral systems attested in the Adamawa  languages confirms this statement. This can be observed not only in cases of forms that belong to different groups, but often within groups and sub-groups as well, which makes the reconstruction of its numeral system quite problematic. In other words, it is not a rare case that small Adamawa branches consisting of only a pair of languages show incomparable forms. Some examples are in order here. 

Let us compare the terms from ‘one’ to ‘ten’ in the Kim\il{Kim} branch that is commonly attributed to the Mbum\il{Mbum}-(Day\il{Day}) group (Greenberg 14) (\tabref{tab:3:98}).

\begin{table}
\caption{\label{tab:3:98}Numerals in the Kim\il{Kim} branch}
\begin{tabularx}{\textwidth}{rXl} 
\lsptoprule
& {Besme}\il{Besme} & {Kim}\il{Kim}\\
\midrule
1 & mōndā/mbírāŋ & ɗú\\
2 & tʃírí & zí\\
3 & h{\textsubtilde{ā}}sī (h{\textsubtilde{ā}}-sī?) & tā\\
4 & ndày & ndà\\
5 & ndìyārá & nūw{\textsubtilde{ē}}y\\
6 & mānɡùl & mènènɡāl\\
7 & ɗīyārā & ɓēálā/ɓēálār\\
8 & ndā-sì (4+3?) & tīmāl/wá-zì-zí(10--2) \\
9 & nòmīnā & làmāɗō/wá-zì-ɗú (10--1) \\
10 & wàl & wòl\\
\lspbottomrule
\end{tabularx}
\end{table}

Only the terms for ‘four’, ‘six’, and ‘ten’ are comparable in these systems. 

The Longuda\il{Longuda} language constitutes a separate branch of Waja\il{Waja}-Jen (Greenberg 10). The table below gives an overview of the first ten numerical terms as attested in two dialects of Longuda (\tabref{tab:3:99}). The evidence for both dialects was collected by the same scholar (Ulrich Kleinewillinghöfer\footnote{\url{https://mpi-lingweb.shh.mpg.de/numeral/Niger-Congo-Adamawa.htm}}). Morphological analysis of the forms is given according to {{Longurama}}\il{Longurama}{{ of Koola (Longuda1) and Wala Lunguda (Longuda2).}}

\begin{table}
\caption{\label{tab:3:99}Longuda\il{Longuda} numerals}


\begin{tabularx}{\textwidth}{rXl} 
\lsptoprule
& {Longuda1} & {Longuda2}\\
\midrule
1 & laa-tw{\`{ɛ}} & naa-khal\\
2 & nàà-kw{\'{\~ɛ}} & naaa-shir\\
3 & nàà-ts{\'{ə}}r & naa-kwáí\\
4 & nèé-nnyìr & naa-nyìr\\
5 & nàà-ny{\'{ɔ}} & nàà-nyó\\
6 & tsààt{\`{ə}}n & na-khí-nà-kwáí (2*3)\\
7 & í-néé-nyìr i-nàà-ts{\'{ə}}r( 4+3) & nyi-na-kwáí (4+3)\\
8 & nyíí-tìn (<4?) & nyí-thìn (<4?)\\
9 & é-nàà-ny{\'{ɔ}} í-néé-nyìr( 5+4?) & nyi-na-nnyó (4+5)\\
10 & koo & n{\^{ɔ}}m\\
\lspbottomrule
\end{tabularx}
\end{table}

Although we are dealing with two dialects of the same language, the roots for ‘one’, ‘two’, ‘three’, ‘six’, and ‘ten’ attested in them are different. At the same time, the terms covering the sequence from ‘six’ to ‘nine’ follow patterns commonly attested elsewhere. Thus the differences between these dialects appear to be greater than those between the languages within Mande or Bantu families. This raises the question as to whether a Proto-Kim\il{Proto-Kim} or Proto-Longuda\il{Proto-Longuda} reconstruction is indeed relevant.

Moreover, the reconstruction is additionally hindered by the fact that numerical terms in the majority of the Adamawa languages are subject to the alignment by analogy more frequently than in other NC languages. General considerations regarding this problem can be found in \chapref{sec:2}. This is of special significance for the Adamawa languages since it affects etymological interpretations. The evidence from a number of languages belonging to the Duru\il{Duru} sub-group of Leko-Nimbari\il{Nimbari} (Greenberg 4) may serve as a case study (\tabref{tab:3:100}).

\begin{table}
\caption{\label{tab:3:100}Duru\il{Duru} numerals}


\begin{tabularx}{\textwidth}{l@{~}l@{~}lXllXl}
\lsptoprule

~ & Peere\il{Peere} & Doyayo\il{Doyayo} & Gimme\il{Gimme} & Gəunəm\il{Gəunəm} & Vɔmnəm\il{Vɔmnəm} & Momi\il{Momi} & Longto\il{Longto} \\
\midrule
1 & d{\'{ə}}ə & ɡbúnú & wɔɔna & mani & mà\textbf{n} & muzo\textbf{z} & w{\'{ə}}{\={ŋ}}ŋá\\
2 & i\textbf{ro} & éé\textbf{r{\'{ɛ}}} & idti\textbf{ɡè} & tɛ\textbf{k} & èt{\^{e}}\textbf{n} & {\`{ɪ}}tt{\'{ə}}\textbf{z} & sitt\textbf{ó}\\
3 & t{\~{a}}{\~{a}}\textbf{ro} & taa\textbf{rɛ} & taa\textbf{ɡè} & taarə\textbf{k} & tāá\textbf{n} & tàá\textbf{z} & t{\~{a}}{\~{a}}\textbf{bó}\\
4 & na\textbf{ro} & násɔ & náà\textbf{ɡè} & náár{\'{ə}}\textbf{k} & nānnò & ná\textbf{z} & nab\textbf{bó}\\
5 & núuno & noon{\'{ɛ}} & nɔɔn{\`{ɨ}}\textbf{ɡè} & nɔɔn{\`{ɔ}}\textbf{k}  & ɡbà náárò & ɡbanáá & n{\~{ɔ}}{\~{ɔ}}mó \\
6 & nón-d{\'{ə}}ə & n{\`{ɔ}}ɔn-ɡbúnú & nɔn\textbf{ɡè} & nɔɔ-waŋɡə & ɡbāā-s{\`{ə}} mâl & bámb{\'{ə}}\textbf{z} & sáámɛ\\
\lspbottomrule
\end{tabularx}
\end{table}

Matching final segments of the first few numerical terms in each of these languages are highlighted in red. I agree with Larry Hyman that “it might not be analogy, rather the use of a marker” (p.c.) but it should be noted that though these segments are different in each case (i.e. they do not match even within a pair of languages), they are present in each language under discussion.

In Mumuye\il{Mumuye}-Yandang, which is another branch of Leko-Nimbari\il{Nimbari} (Greenberg 5), an additional sub-morpheme (-t) is attested that is not present in Duru\il{Duru} (\tabref{tab:3:101}).

\begin{table}
\caption{\label{tab:3:101}Analogical alignments in Mumuye\il{Mumuye}-Yandang}


\begin{tabularx}{\textwidth}{XXXl}
\lsptoprule

~ & Mumuye\il{Mumuye} & Bali\il{Bali} & Yendang\il{Yendang} (dial.)\\
\midrule
2 & zi\textbf{ti} & i-ye & í-nī \\
3 & taː\textbf{ti} & taa\textbf{t} & tâː\textbf{t}\\
4 & d{\`{\~ɛ}}ː\textbf{tì} & naa\textbf{t} & nâː\textbf{t}\\
\lspbottomrule
\end{tabularx}
\end{table}

The following conclusions with regard to the Proto-Duru\il{Proto-Duru} numeral system can be reached upon the basis of this evidence. First, the final segments (whatever their phonetic difference) should not be viewed as a hinderance to the comparison of numerical terms. This means that Momi\il{Momi} \textit{tàáz} ‘three’ can (and should be) compared to Longto\il{Longto} \textit{t{\~{a}}{\~{a}}bó}. The question of whether their final segments should be analysed as morphemes or sub-morphemes is of secondary importance for our purposes. At the same time, the quality of the second consonant in Proto-Leko-Nimbari\il{Proto-Leko-Nimbari} is obscure, so we have to reconstruct the form as *\textit{taa}X, where X is an unknown consonant.    

As demonstrated above, numerical terms are exceptionally divergent within the family. In addition to this, systematic (diversified) alignment by analogy is often employed in the languages under study. Both factors make the reconstruction a challenging task, even though an attempt at reconstruction of the Adamawa numerals by a highly competent scholar is available (see \citealt{Boyd1989}). His results, however, are of limited relevance for our comparative purposes, as the following example shows. According to Boyd, the Proto-Adamawa\il{Proto-Adamawa} term for ‘one’ is to be reconstructed as \textit{*ku-di-n} (the root \textit{*di}) with *\textit{kwin} being its later development. His ideas on how this proto-form is reflected in particular branches of the Adamawa family are summarized in the table below (\tabref{tab:3:102}). Notations in the first column refer to Grinberg’s grouping of the Adamawa languages.

\begin{table}
\caption{\label{tab:3:102}*\textit{kwin-} reflexes in Adamawa according to Boyd}


\begin{tabularx}{\textwidth}{XXl}
\lsptoprule

~ & {~*Proto} & {~Reflexes}\\
\midrule
\textbf{G1} & kwin & kun\\
\textbf{G1} & kwin & kwaan\\
\textbf{G2} & kwin & gu-(a)s(a)\\
\textbf{G4} & kwin & gun, gbun, bin, wun-ga, guu\\
\textbf{G5} & kwi(t) & gbet, gorV\\
\textbf{G5} & kwin & in(d)i\\
\textbf{G6} & kwin-k & soŋ\\
\textbf{G7} & kwin & indi > fa-ndi\\
\textbf{G8} & kwin-kwin & bimbimi\\
\textbf{G8} & kwi(n) & gwi > ju\\
\textbf{G9} & kwin & tsuŋ/tsiŋ, cɔŋ\\
\textbf{G10} & kwi-t & > kwat > kal\\
\textbf{G13} & kwit & ɓuru, gulu\\
\textbf{G13} & kwit & > kwat > bara(k)\\
\textbf{G13} & kwin & ʈoŋ\\
\textbf{G14} & kwin & ɗu\\
\textbf{Day}\il{Day} & kwin-k & ngoŋ\\
\textbf{Day}\il{Day} & kwin & (k)wan > mɔn\\
\lspbottomrule
\end{tabularx}
\end{table}

\newpage 
Even if Boyd’s reconstruction of the Proto-Adamawa\il{Proto-Adamawa} form is correct, a diachronic interpretation that impies an etymological relationship between \textit{bimbimi}, \textit{cɔŋ}, \textit{ɗu} and \textit{gbet} does not fit the purpose of our integral comparative study of NC numerical terms because it can be used to justify nearly any etymological connection. In view of this, the Adamawa numerical terms will be treated in the same way as those from the preceding language families. First, the main forms of the numerical terms will be established, with no attempt at tracing them down to a provisional proto-form. Then the numeral systems of each of the Adamawa branches will be studied separately. Finally, an integral analysis of the available evidence pertaining to each of the terms will be offered. This approach will enable us to treat the Fali\il{Fali} languages and even Laal\il{Laal} together with the Adamawa languages, although their relationship to the latter is often questioned (in the case of Laal, doubts are raised as to whether it belongs to NC at all).  


\subsection{Fali-Yingilum (G11)}%3.6.1.

It should be noted that after a nasal, -\textit{r}- in the Fali\il{Fali} forms regularly corresponds to -\textit{N}- in those of Yingilum\il{Yingilum}, cf. ‘5’ Fali \textit{k{\textsubbar{ɛ}}rɛw} {\textasciitilde} Yingilum \textit{k{\'{ɛ}}ɲàu}, ‘7’ \textit{j{\textsubbar{ɔ}}r{\textsubbar{ɔ}}s} {\textasciitilde} Yingilum \textit{j{\'{ə}}n{\`{ə}}s}. An alignment by analogy is probably attested in the terms for ‘three’ and ‘four’ (\textit{*taaX} > \textit{taan} may have changed by analogy with \textit{*naan}).


\il{Fali}\il{Yingilum}
\begin{table}
\caption{\label{tab:3:103}Fali\il{Fali}-Yingilum\il{Yingilum} numerals}


\begin{tabularx}{\textwidth}{lXrl}
\lsptoprule

1 & kpolo/bʌlo (< *lo?) & 7 & j{\textsubbar{ɔ}}rɔs\\
2 & cuk, gbara & 8 & 4 redupl.\\
3 & taan (< taaX) & 9 & 10--1/ŋɡʌs kàm(kàn) k͡pòlò `rest hand one'\\
4 & naan & 10 & ra\\
5 & k{\~{ɛ}}rɛw & 20 & 10*2\\
6 & yira/yilo & 100 & < Fula\il{Fula}\\
&  & 1000 & < Fula\il{Fula}\\
\lspbottomrule
\end{tabularx}
\end{table}


\subsection{Kam (Nyimwom, G8)}%3.6.2.
\il{Kam}
\begin{table}
\caption{\label{tab:3:104}Kam\il{Kam} numerals}


\begin{tabularx}{\textwidth}{lXrl}
\lsptoprule

1 & b{\textsubbar{i}}{\textsubbar{i}} (Meek: bimbini) (< *b-ii?) & 7 & jùp yi-raak (6,2 - ‘second six’?)\\
2 & yi-raak (i-ra) & 8 & sâl\\
3 & càr & 9 & níízaa\\
4 & nár (< *naX) & 10 & bóò\\
5 & ŋwún & 20 & kpáímí ,*{\`{n}}kp{\textsubtilde{ó}}\\
6 & jù:p & 100 & 20*5\\
&  & 1000 &?\\
\lspbottomrule
\end{tabularx}
\end{table}

Within the NC context, a reversive alignment by analogy may be considered: \textit{*na}X ‘4’ > \textit{nar} by analogy with \textit{*car} ‘3’. As Boyd rightfully observes, in the case of ‘one’ it is often unclear whether the initial consonant is a part of the root, or a reflex of the noun class prefix. 

The term for ‘seven’ simulates the pattern ‘7=6+2’ (this phenomenon is not infrequent in NC). Sometimes (e.g. in some of the Mande languages) this impression is due to the fact that the term for ‘six’ originally derived from ‘5+’. Over time, an innovation replaced the original term for ‘five’, which was only preserved in the derived term for ‘six’. Alternatively, the term for ‘seven’ could be explained as ‘the other six’ (or ‘a big six’ is some languages), as perhaps in Kam\il{Kam}, assuming that \textit{jù:p} does not go back to the term for ‘five’.


\subsection{Leko-Duru-Mumuye (G4, G2, G5)} %3.6.3.
\il{Duru}\il{Mumuye} 
{ This group is often labeled Leko-Nimbari\il{Nimbari}. Here we follow Raimund Kastenholz and Ulrich Kleinewillinghöfer, who note that ``The term ‘Nimbari’ should not to be used as a classificatory term, nor should the scarce and surely in large parts erroneous data be given central significance in any comparative approach to Adamawa languages'' (\citealt{KastenholzKleinewillinghöfer2012}).} 
\subsubsection{Duru (G4)}%3.6.3.1.
\il{Duru}
\begin{table}
\caption{\label{tab:3:105}Duru\il{Duru} numerals}


\begin{tabularx}{\textwidth}{lQrQ}
\lsptoprule

1 & d{\'{ə}}ə, ɡbúnú, w{\'{ə}}-{\={ŋ}}ŋá/wɔɔna/dá(ŋ)ɡá/*nge, man(i)/*mal & 7 & 5+2, (ɡútambe, 6+'odd', d{\'{ə}}msàrà, 4+3)\\
2 & du/ru/to, te/re & 8 & 4PL/4+4, 5+3,( < Hausa)\il{Hausa}\\
3 & t{\~{a}}{\~{a}}tó/t{\~{a}}{\~{a}}ro & 9 & ` one finger is left `, n{\'{ɨ}}{\`{ŋ}}s{\'{ɨ}}nè, 5+4, 10--1\\
4 & nató/naró (< *naX) & 10 & bōʔ, kob/kop/fób\\
5 & núno/nɔɔn{\`{ɨ}}, ɡbà náárò/ɡbanáá, sáá & 20 & ɡbɛɡ/ɡbàhs{\'{ɨ}} (='staff'), *w{\'{ɔ}}{\'{ɔ}}ɡ ('head'), zul/zur (‘head'), (10*2, ráárò, jùɡúyɔ),\\
6 & ɡúú, 5+1 & 100 & tɛmere < Fula,\il{Fula} 20*5\\
&  & 1000 & uzinere < Fula,\il{Fula} (dukə)\\
\lspbottomrule
\end{tabularx}
\end{table}

This table provides an overview of forms and patterns attested in eleven sources for this sub-group. This degree of variety is not normally attested within a single sub-group, which raises doubts as to whether these languages should be grouped together.  

\subsubsection{Leko (G2)}%3.6.3.2.

Our study of this sub-group is based on the evidence of two languages. The summary table above is not descriptive of the language-specific mechanisms of the alignment by analogy. An overview of the numerical terms covering the sequence from ‘two’ to ‘five’ by language is provided in \tabref{tab:3:107}.

\begin{table}
\caption{\label{tab:3:106}Leko numerals}


\begin{tabularx}{\textwidth}{lQrQ}
\lsptoprule

1 & n{\'{ɨ}}ŋa/níiá (<ŋa?) & 7 & 5+2\\
2 & nnú, ra?, *-i? & 8 & 5+3, < Hausa\il{Hausa}\\
3 & toorà/toonú & 9 & 5+4,' one is left `\\
4 & naarà/nɛɛr-əb & 10 & kób/kóp\\
5 & núúnà/núnn-ub & 20 & nɛd níi ɡbɛd, laa-1\\
6 & n{\^{ɔ}}ŋɡ{\^{ɔ}}s/núŋɡ{\'{ɔ}}ɔs & 100 & 20*5, < Fula\il{Fula}\\
&  & 1000 & 20*10?,< Fula\il{Fula}\\
\lspbottomrule
\end{tabularx}
\end{table}

\begin{table}
\caption{\label{tab:3:107}Analogical alignments in two Leko languages}


\begin{tabularx}{\textwidth}{XXl} 
\lsptoprule
& Kolbila\il{Kolbila} (Zurá) & Samba Leko\il{Samba Leko}\\
\midrule
2 & in\textbf{nú} & ii\textbf{rà}\\
3 & too\textbf{nú} & too\textbf{rà}\\
4 & nɛɛr\textbf{əb} & naa\textbf{rà}\\
5 & núnn\textbf{ub} & núún\textbf{à}\\
\lspbottomrule
\end{tabularx}
\end{table}

Apparently, the terms from ‘three’ to ‘five’ in these two languages are related to each other. At the same time, two groups of terms (‘2--3’ and ‘3--4’) with an alignment by the ultima are observable in Kolbila\il{Kolbila}. This is applicable to a group of Samba Leko\il{Samba Leko} terms as well, namely ‘2--4’ (possibly also ‘5’; the fact that the Samba Leko terms are adjusted by both the vowel quality and the tone is noteworthy). This means that the seemingly unrelated roots for ‘two’ may have derived from a common etymon (still unknown to us) by means of alignment by analogy. The source form of ‘two’ remains obscure. Assuming that it was similar to the one reconstructed for the Duru\il{Duru} sub-group (e.g. *\textit{ru}), it is likely that the same form is to be reconstructed for Leko as well: \textit{*ru} > Kolbila \textit{nu} by analogy with \textit{toonu} ‘3’ ; \textit{*ru} > Samba Leko \textit{rà} by analogy with \textit{toorà} ‘3’. However, the evidence in favor of this reconstruction is inconclusive. Alternatively, the initial vowel of the term for ‘two’ (\textbf{*ii-/in-}) may reflect the source root, while the final segment is potentially explained via an alignment by analogy with `3'.

\subsubsection{Mumuye-Yandang (G5)}%3.6.3.3.
\il{Mumuye}
\begin{table}
\caption{\label{tab:3:108}Numerals in Mumuye\il{Mumuye}-Yandang}


\begin{tabularx}{\textwidth}{lQrQ}
\lsptoprule

1 & ɓīntī/ɓini (*< nti/ni?) , ɡbétè & 7 & 5+2\\
2 & ziti, ye, nī & 8 & 5+3\\
3 & taat & 9 & 5+4\\
4 & naat & 10 & kop/kob\\
5 & m{\v{a}}ːni, nɔng/ɡhìnān & 20 & mba-1, kar-1, mim-1\\
6 & 5+1 & 100 & 20*5\\
&  & 1000 & derived\\
\lspbottomrule
\end{tabularx}
\end{table}

This sub-group is represented by three languages that show different forms of ‘two’. The terms for ‘three’ and ‘four’ are adjusted by analogy. Studying them in a wider NC context reveals that the final consonant in ‘four’ was adjusted by analogy with ‘three’. The alignment itself must have occurred already at the Proto-Mumue-Yandang\il{Proto-Mumue-Yandang} level, which explains our provisional reconstructions suggested for this proto-language in the table above.

No evidence pertaining to the Nimbari\il{Nimbari} numerals is available to us. The forms of ‘one’ given by Boyd \citep{Boyd1989} are noteworthy (Nimbari \textit{(n)yeme/} \textit{geme/} \textit{(ʒeme?)}).

 
\subsection{Mbum-Day (G13, G14, G6, Day)}%3.6.4.
\il{Mbum}\il{Day}
\subsubsection{Bua (G13)} %3.6.4.1.
\il{Bua}This is very divergent branch that has been poorly documented. I'd like to thank Pascal Boyeldieu who has provided me with his personal data on Ɓa\il{Ba@Ɓa} (Bua\il{Bua}) and Lua\il{Lua} (Niellim\il{Niellim}),  as well as some other rare sources. The main forms and patterns are shown in \tabref{tab:3:109}.

\begin{table}
\caption{\label{tab:3:109}Bua\il{Bua} numerals}

\small
\begin{tabularx}{\textwidth}{r@{~}>{\raggedright}p{9mm}@{~}>{\raggedright}p{12mm}l@{~}>{\raggedright}p{11mm}@{~}p{14mm}l@{\,}Q@{~}Q@{}} 
\lsptoprule
& \textbf{Fanya}\il{Fanya} & \textbf{Niellim}\il{Niellim} & \textbf{Tunya}\il{Tunya} & \textbf{Bua}\il{Bua} & \textbf{Zan Gula}\il{Zan Gula} & \textbf{Kulaal}\il{Kulaal(Gula)}\il{Gula} & \textbf{Bolgo}\il{Bolgo} & \textbf{Koke}\il{Koke}\\
\midrule 
1 & do/lo 	& ɓúdū/ ɓúrū 	& sèlì 	& gúlu 	& sammā, saado 	& ʈóŋ 	& ba(k)ra, silla 	& barak\\
2 & i-ru/ li-ru & ndīdí/ ndīrí 	& à-rī 	& i-li/í-rīː 	& ɾisːi/lissi 	& r{\`{ɔ}}k 	& lēti, retè 	& lēdi\\
3 & taro 	& tērí 	& à-tā 	& í-tēr 	& toːɾi 	& tòòs 	& teri 	& tēri\\
4 & nagi/ naro 	& ni{\dropflata}ːní/ néni 	& à-nā 	& í-pāw/ paõ 	& naːsɪ 	& nòr{\textsubtilde{ò}} 	& har 	& hār\\
5 & lugni 	& lùní 	& à-lōnī 	& í-lwār 	& tɛ(r) 	& lúɲ 	& tisso 	& tisó\\
6 & kaba 	& táːr 	& nānò 	& t{\texthighdropa}ːr 	& 5+1 	& lú-én-ʈóŋ 	& tipsi 	& dípsil\\
7 & 5+2 	& longa 	& lúlú 	& l{\"ū}r 	& 5+2 	& lú-é-r{\`{ɔ}}k 	& 5+2 	& tiglén\\
8 & <4 		& 3+4,\newline <Bagirmi\il{Bagirmi} 	& k{\`{ɔ}}nt{\textsubtilde{ā}} 	& <*4 PL? 	& 5+3 	&  	& orhor\newline \mbox{(4 redupl.),} 5+3 	& 4 redupl.\\
9 & 10-X 	& <Bagirmi\il{Bagirmi} 	& à-tī 	& lór-lor 	& 5+4 	& sàk{\'{ɔ}}l{\'{ɩ}}nnòr{\textsubtilde{ò}} 	& diar, 6+3 	& jār\\
10 & teba 	& <Bagirmi\il{Bagirmi}, hul{\=ó}a 	& kùtù 	& húlil/ lor-poo 	& filoːle/ filori 	& yíppà 	& do(k) 	& dog\\
20 & 10*2 	& doksap 	& 10*2 	& <10PL 	& ʊ-faːlɛ 	&  	& a-rep, a-hun, tehu 	& \\
100 &  		& ro/ru 	& à-rū 	& a-ru 	& < Arabic\il{Arabic} 	& míà/míè 	&  	& ae léd\\
1000 &  	& dubu 	& dūbú 	& dubu 	& < Arabic\il{Arabic} 	& hálìf 	&  	& ae har\\
\lspbottomrule
\end{tabularx}
\end{table}

{\bfseries
\textmd{Numerals in the Bua}\il{Bua}\textmd{ group can be presented as follows (\tabref{tab:3:110})}}

\begin{table}
\caption{\label{tab:3:110}Bua\il{Bua} numerals (summarized)}


\begin{tabularx}{\textwidth}{llrQ}
\lsptoprule

1 & *do, *de?, bara(k), (ʈóŋ) & 7 & 5+2, 3+4, lúlú/lòŋɡ{\={ɔ}}/lur, (tiglen)\\
2 & *di, *ri?, *ru?, (r{\`{ɔ}}k), (rete) & 8 & 4 redupl., 5+3\\
3 & tar/tori/teri & 9 & ti, jar, 5+4, 10-X\\
4 & na/nagi/niani, har & 10 & do(k), (kùtù), (filoːle), (yíppà), (teba)\\
5 & luni/loni/*lu,tɛ(r), *kɔn?, (tiso) & 20 & 10*2, do-ksap, faːlɛ,  (a-rep), (a-hun)\\
6 & 5+1, táːr, (nānò), (kaba), tipsi & 100 & ro/ru\\
&  & 1000 & < Bagirmi\il{Bagirmi}\\
\lspbottomrule
\end{tabularx}
\end{table}

\clearpage 
\subsubsection{Kim (G14)}%3.6.4.2.
\il{Kim}The first ten terms of Besme\il{Besme} and Kim\il{Kim} are given in the table above (\tabref{tab:3:98}). The term for ‘twenty’ in these languages follows the pattern ‘10*2’, whereas the Kim term for ‘hundred’ is borrowed from Arabic\il{Arabic}. The Besme term for ‘hundred’ is borrowed from the French\il{French} \textit{sac} ‘sack’, whereas the term for ‘thousand’ is borrowed from Bagirmi\il{Bagirmi}.


 
\subsubsection{Mbum (G6)}%3.6.4.3.
\il{Mbum}
\begin{table}
\caption{\label{tab:3:111}Mbum\il{Mbum} numerals}


\begin{tabularx}{\textwidth}{lXrl}
\lsptoprule

1 & mbew/mbiew, b{\"{ɔ}}{\={ɔ}}ŋ/búónó/bóm/vaŋno & 7 & 10--3, rɪŋ, (r{\"{e}}nām, tàrn{\'ã}ɡà)\\
2 & seɗe/sere, ɡwa/ɓ{\`{ɔ}}-ɡ{\"{e}}, ɓà-tì & 8 & 10--2, nama/namma/nènmàʔ{\"{a}}\\
3 & say & 9 & 10--1, doraŋ\\
4 & nìŋ, nai & 10 & boo, dʒama/dʒémà, (dùɔ, hù-wàl{\"{e}} )\\
5 & ndiɓi/ndēɓē/dūwēe/dáp{\`{ɪ}} & 20 & 10*2, `2 hands', 10+10\\
6 & ze(y)/ye(a), (t{\'{ɔ}}t{\'{ɔ}}kl{\'{ɔ}}, bì-ɡírò) & 100 & s{\'{ɔ}}ɗ/sɔt, < Fula,\il{Fula} < Arabic\il{Arabic}\\
&  & 1000 & `sac', bag', < Fula,\il{Fula} < Bagirmi\il{Bagirmi}\\
\lspbottomrule
\end{tabularx}
\end{table}

This sub-group is represented by a dozen languages. Unlike Leko-Duru\il{Duru}-Mumue no alignment by analogy is attested. Some forms of ‘two’ are of unclear morphological structure. 

\subsubsection{Day}%3.6.4.4.
\il{Day}
\begin{table}
\caption{\label{tab:3:112}Day\il{Day} numerals}


\begin{tabularx}{\textwidth}{lXrl}
\lsptoprule

1 & nɡ{\={ɔ}}{\'{ŋ}}, *mon & 7 & 4+3\\
2 & dīí & 8 & 4 redupl.?\\
3 & tà & 9 & `lacking one'\\
4 & ndà, *bī-yām & 10 & m{\textsubtilde{ò}}\\
5 & s{\={ɛ}}rì & 20 & 10*2\\
6 & 5+1 & 100 & tù\\
&  & 1000 & < Bagirmi\il{Bagirmi}\\
\lspbottomrule
\end{tabularx}
\end{table}

This branch is comprised of an isolated language. Its attribution to Mbum\il{Mbum}-Day\il{Day} has been a subject of scholarly debate. The form *\textit{mon} `1' is postulated on the basis of \textit{s{\={ɛ}}rì} \textit{mòn} ‘six’, whereas the reconstruction of\textit{*bīyām} (\textit{*bī-yām}?) `4' is based on \textit{bīyām} \textit{tà} ‘seven’.


\subsection{Waja-Jen (G9, G10, G1, G7)}%3.6.5.
\il{Waja}\subsubsection{Jen (G9)}%3.6.5.1.
\begin{table}
\caption{\label{tab:3:113}Jen numerals}


\begin{tabularx}{\textwidth}{llrX}
\lsptoprule

1 & kwín/*ʃín/tsɨnɡ & 7 & 5+2\\
2 & ráb/*re, bwə-nɡ, bwa-yunɡ & 8 & 4PL, 5+3\\
3 & ɡbunuŋ, bwa-tə & 9 & 5+4\\
4 & net, bwa-nyə & 10 & ʃóób, bwa-hywə\\
5 & nóob/*na, bwa-hmə/*hw{\~{i}} & 20 & fa-1, nɡwu-1\\
6 & 5+1 & 100 & 20*5\\
&  & 1000 & ʃik-1, 20-fe\\
\lspbottomrule
\end{tabularx}
\end{table}

This branch is represented by two languages: Burak\il{Burak} and Jenjo\il{Jenjo} (Dza). The evidence from this group is among Boyd’s best arguments for the reconstruction of \textit{*kwin} (<\textit{*ku-di-n}) ‘one’. The primary term \textit{li} (\textit{bwa-li}) ’fifteen’ is attested in Jenjo. Accordingly, the term for ‘sixteen’ follows the pattern ‘15+1’ (\textit{bwali} \textit{ji} \textit{tsɨnɡ}). Interestingly, in Burak the term for ‘hundred’ is \textit{li} (\textit{li} \textit{kwín}).

The form \textit{*hw{\~{i}}} ‘five’ is traceable in Jenjo\il{Jenjo} compound terms covering the sequence from ‘six’ to ‘nine’ (\textit{hw{\~{i}}-tsɨnɡ} ‘six’, \textit{hw{\~{i}}-yunɡ}~‘seven’, etc.) as is the corresponding Burak\il{Burak} form \textit{*na} ‘five’ (\textit{naa-ʃín} ‘six’, \textit{náá-re} ‘seven’, \textit{ná-tát} ‘eight’). The form *\textit{re} ‘two’ is observable in \textit{náá-re} ‘seven’, whereas \textit{*ʃín} ‘one’ is traceable in \textit{naa-ʃín} ‘six’.

\subsubsection{Longuda (G10)}%3.6.5.2.
\il{Longuda}The evidence for the first ten numerals in two Longuda\il{Longuda} dialects can be found in the table above (\tabref{tab:3:99}). The term for ‘twenty’ in these languages follows the pattern ‘10*2’.  The forms of ‘hundred’ are \textit{pùlò(wé)/phulewe}.

\subsubsection{Waja (G1)}%3.6.5.3.
\il{Waja}
\begin{table}
\caption{\label{tab:3:114}Waja\il{Waja} numerals}


\begin{tabularx}{\textwidth}{llrX}
\lsptoprule

1 & w-in/d-in/kw-an/ɡ-ɛɛn/*k-un? & 7 & ni-bir/ni-ber/ni-bil/ni-bi(y)\\
2 & y{\'{ɔ}}-r{\'{ɔ}}b/rɔɔp/yob/yo, (su) & 8 & na-rib/na-lib/na-rub (4*2)\\
3 & taat, kunuŋ, (bwanbí) & 9 & 10--1, teer/teet/tɔɔrɔ\\
4 & naat, (ɡwár) & 10 & k{\'{ɔ}}b/kub/kwab/kpop/kwu\\
5 & nu(ŋ), (fwáːd) & 20 & 10*2, `2 hands'\\
6 & nu-kun (<5+1?) & 100 & <10?, wɔn, (bwa-tiɡɛ)\\
&  & 1000 & kʊʊl, nèe/kú-néŋ, 100*10, bi-kate, tedu\\
\lspbottomrule
\end{tabularx}
\end{table}

Some languages in this sub-group are characterized by a sub-morphological alignment of the terms for ‘three’ and ‘four’ well-attested in Adamawa: Dadiya\il{Dadiya} \textit{t}\textbf{\textit{al} }‘3’ {\textasciitilde} \textit{n}\textbf{\textit{al}} ‘4’, Bangunji\il{Bangunji} (dial.) 1 \textit{t}\textbf{\textit{áát}} ‘3’ {\textasciitilde} \textit{n}\textbf{\textit{áát}}‘4’, Bangunji (dial.) 2 \textit{t}\textbf{\textit{aar}} ‘3’ {\textasciitilde} \textit{n}\textbf{\textit{aar}} ‘4’, Tula\il{Tula} (Kɨtule) \textit{jí-tː}\textbf{\textit{à}} ‘3’ {\textasciitilde} \textit{jáː-n}\textbf{\textit{à}}‘4’. As a result, these terms are treated as minimal contrastive pairs in the paradigm.  Within the NC context, forms with the final -\textit{t} should be considered prototypical in the case of both terms. This means that \textit{*naa}X ‘four’ (final consonant unknown) may have evolved into *\textit{naat} by analogy with ‘three’ in Proto-Waja\il{Proto-Waja}. Later, an innovative form for ‘three’ developed in Awak\il{Awak} and Waja\il{Waja}: Awak \textit{kunúŋ}, Waja \textit{kunoŋ}. The Dijim\il{Dijim}-Bwilim \textit{bwanbí} is apparently an innovation. 

Interestingly, the froms for ‘six’ attested throuought the sub-group resemble the Awak\il{Awak} and Waja\il{Waja} forms for ‘three’. However, the forms for ‘six’ can be explained as ‘5+1’ (assuming that they include an allomorph of *\textit{kun} ‘one’).

\subsubsection{Yungur (G7)}%3.6.5.4.
\il{Yungur}
\begin{table}
\caption{\label{tab:3:115}Yungur\il{Yungur} numerals}


\begin{tabularx}{\textwidth}{llrX}
\lsptoprule

1 & fini/fandi/p{\'{ə}}nd{\'{ə}}ŋ (< *ndi?), wunú & 7 & nbutu\\
2 & raap, fətə/fiicì (< *tə/ci?) & 8 & 4 redupl.\\
3 & táák{\'{ə}}n/(tɑɑr{\'{ə}}n) & 9 & 5+4\\
4 & kurun & 10 & bú(u), (kutun)\\
5 & wonon/wonun & 20 & (10*2)\\
6 & mindike & 100 & (-ru)\\
&  & 1000 & (100*10)\\
\lspbottomrule
\end{tabularx}
\end{table}

The terms for ‘twenty’, ‘hundred’ and ‘thousand’ are attested in only one source (Kaan\il{Kaan} (Libo)) out of the eight sources available for this branch, hence they are quoted in brackets. Morphological analysis of the terms for ‘one’ and ‘two’ is unclear: \textit{*fV} may be a reflex of the original noun class prefix.


\subsection{Laal}%3.6.6.
\il{Laal}Finally, let us turn to the Laal\il{Laal} numeral system. Laal’s attribution to the Adamawa languages (as well as its attribution to NC) is debatable. Today it is assumed that it is an isolated case within Niger-Congo. Comparative study of its numerical terms may shed light on its genealogical relationship (\tabref{tab:3:116}). 

\begin{table}
\caption{\label{tab:3:116}Numerals in Laal\il{Laal}}


\begin{tabularx}{\textwidth}{lXrl}
\lsptoprule

1 & ɓ{\`{ɨ}}d{\'{ɨ}}l (ɓ{\`{ɨ}}-d{\'{ɨ}}l?) & 7 & 5+2\\
2 & ʔīsī (ʔī-sī?) & 8 & 4 redupl.\\
3 & māā & 9 & yàŋjáŋ~\\
4 & ɓīsān (ɓī-sān?) & 10 & tūū\\
5 & sāb, *swa- & 20 & 10*2\\
6 & cìcààn~ & 100 & 10-'big'\\
&  & 1000 & < Baguirmi < Hausa\il{Hausa}\\
\lspbottomrule
\end{tabularx}
\end{table}

As in many other NC languages, the major problem with Laal\il{Laal} numerals is the obscurity of their morphological structure. Pascal Boyeldieu established that traces of noun class suffixes are observable in Laal forms as their comparison to \textsc{sg} and \textsc{pl} forms show (see \citealt{Boyeldieu1982}). However, as I tried to demonstrate elsewhere \citep{Pozdniakov2010}, some traces of noun class prefixes had been preserved in this language as well.{} At this point, it seems reasonable to set the alternative variants aside for further comparison. 

What follows is an attempt to synthesize the Adamawa evidence. 


\subsection{Proto-Adamawa}%3.6.7.
\il{Proto-Adamawa}
\subsubsection{‘One’} %3.6.7.1.
The main forms are given in \tabref{tab:3:117}.


\begin{sidewaystable}
\caption{\label{tab:3:117}Adamawa stems for ‘1’} 

\footnotesize
\begin{tabularx}{\textwidth}{l lQlllQQl}
\lsptoprule
&  `1' & `1' & `1' & `1' & `1' & `1' & `1' & `1' \\
\midrule
Fali\il{Fali}&  &  & *-lo &  &  &  &  & \\
Kam\il{Kam}  & b-{\textsubbar{i}}{\textsubbar{i}} &  &  &  &  &  &  & \\
\textit{Leko}\\
~~~~Duru\il{Duru} &  & d{\'{ə}}ə &  & -(ŋ)ɡá/-na?/*nge & ɡbúnú & \mbox{man(i)/*mal} &  & \\
~~~~Leko &  &  &  & n{\'{ɨ}}ŋa/níiá (<ŋa?) &  &  &  & \\
~~~~Mumuye\il{Mumuye} &  & \mbox{ɓī-ntī/ɓi-ni}\newline\mbox{(*< nti/ni?)} &  &  &  &  &  & ɡbétè\\
\textit{Mbum}\il{Mbum}\\
~~~~Bua\il{Bua} &  & *dɛ & *do &  &  &  & bara(k) & ʈóŋ, *si?\\
~~~~Kim\il{Kim} &  &  & ɗú &  &  & mōndā & mbírāŋ & \\
~~~~Mbum\il{Mbum} &  &  &  &  & b{\"{ɔ}}{\={ɔ}}ŋ/búónó &  &  & mbew/mbiew\\
~~~~Day\il{Day} &  &  &  & nɡ{\={ɔ}}{\'{ŋ}} &  & *mon &  & \\
\textit{Waja}\il{Waja}\\
~~~~Jen & kw-ín/*ʃ-ín/ts-ɨnɡ (< *in) &  &  &  &  &  &  & \\
~~~~Longuda\il{Longuda} &  &  &  &  &  &  &  & khal, tw{\`{ɛ}}\\
~~~~Waja\il{Waja} & w-in/d-in/ɡ-ɛɛn/*k-un? &  &  &  &  &  &  & \\
~~~~Yungur\il{Yungur} &  & fi-ni/fa-ndi/p{\'{ə}}-nd{\'{ə}}ŋ (< *ndi?) &  &  & wunú &  &  & \\
Laal\il{Laal}  &  & ɓ{\`{ɨ}}d{\'{ɨ}}l (ɓ{\`{ɨ}}-d{\'{ɨ}}l?) &  &  &  &  &  & \\
\lspbottomrule
\end{tabularx}
\end{sidewaystable}


In accordance with Boyd’s hypotheses discussed above, the forms in the first two columns may be related in view of the reconstruction of the root *\textit{di} (possibly also *-\textit{in}), the noun class prefix *\textit{ku}- and the suffix *-\textit{n} (*\textit{ku-di-n} ’1’) 

The last column lists forms that are attested in one of the branches only. The roots that can be tentatively reconstructed as \textit{*do}, *\textit{nga/ngɔ}; \textit{*(g)bunuand}  and \textit{*mon} are noteworthy.

\subsubsection{‘Two’}%3.6.7.2.
The main forms of this root are quoted in \tabref{tab:3:118}. The grouping of forms is admittedly not substantiated enough. The variety of forms within this family is striking, even when unrestricted phonetic grouping is applied.

%MOVED THIS TABLE OUT OF CHRONOLOGICAL ORDER
\begin{table}[b]
\caption{\label{tab:3:119}Adamawa stems for ‘3’}
\small
\begin{tabularx}{\textwidth}{lXllll}
\lsptoprule

Fali-\il{Fali}Yingilum\il{Yingilum}   & taan (< taaX) &  &  &  & \\
Kam\il{Kam}  & càr &  &  &  & \\
\textit{Leko-Duru-Mumuye}\\
~~~~Duru\il{Duru} & t{\~{a}}{\~{a}}tó/t{\~{a}}{\~{a}}ro &  &  &  & \\
~~~~Leko & toorà/toonú &  &  &  & \\
~~~~Mumuye\il{Mumuye} & taat &  &  &  & \\
\textit{Mbum-Day}\\
~~~~Bua\il{Bua} & tar/tori/teri &  &  &  & \\
~~~~Kim\il{Kim} & tā &  &  & h{\textsubtilde{ā}}sī & \\
~~~~Mbum\il{Mbum} & say &  &  &  & \\
~~~~Day\il{Day} & tà &  &  &  & \\
\textit{Waja-Jen}\\
~~~~Jen & bwa-tə & ɡbunuŋ &  &  & \\
~~~~Longuda\il{Longuda} & ts{\'{ə}}r &  & kwáí &  & \\
~~~~Waja\il{Waja} & taat (bwanbí) & kunuŋ &  &  & \\
~~~~Yungur\il{Yungur} & táák{\'{ə}}n/(tɑɑr{\'{ə}}n) &  &  &  & \\
Laal\il{Laal}  &  &  &  &  & māā\\
\lspbottomrule
\end{tabularx}
\end{table}


\begin{sidewaystable}
\small 
\caption{\label{tab:3:118}Adamawa stems for ‘2’} 
\begin{tabularx}{\textwidth}{l p{18mm}llp{15mm}QQllp{15mm}} 
\lsptoprule
&  `2' & `2' & `2' & `2' & `2' & `2' & `2' & `2' & `2' \\
\midrule
Fali-\il{Fali}Yingilum\il{Yingilum} &  &  &  &  &  & gbara & cuk &  & \\
Kam\il{Kam}  & \mbox{yi-raak (i-ra)} &  &  &  &  &  &  &  & \\
\textit{Leko-Duru-Mumuye}\il{Mumuye}\il{Duru}\\
~~~~Duru\il{Duru} &  & du/ru, to &  & te/re &  &  &  &  & \\
~~~~Leko & ra? &  & ii-/in-? &  &  &  &  & nnú & \\
~~~~Mumuye\il{Mumuye} &  &  & ye &  & ziti &  &  & nī & \\
\textit{Mbum-Day}\il{Day}\il{Mbum} \\ 
~~~~Bua\il{Bua}   &  & *ru, (r{\`{ɔ}}k) & di/ri &  & (rete) &  &  &  & \\
~~~~Kim\il{Kim}   &  &  &  & zí & tʃírí &  &  &  & \\
~~~~Mbum\il{Mbum} &  &  &  & ɓà-tì & seɗe/sere & ɡwa/ɓ{\`{ɔ}}-ɡ{\"{e}} &  &  & \\
~~~~Day\il{Day}   &  &  & dīí &  &  &  &  &  & \\
\textit{Waja-Jen} \\
~~~~Jen & ráb/*re, &  &  &  &  &  &  &  & bwə-nɡ, bwa-yunɡ\\
~~~~ Longuda\il{Longuda} &  &  &  &  & shir & kw{\'{\~ɛ}} &  &  & \\
~~~~ Waja\il{Waja} & y{\'{ɔ}}-r{\'{ɔ}}b/rɔɔp/yob/yo &  &  &  &  &  & (su) &  & \\
~~~~ Yungur\il{Yungur} & raap &  &  & fətə/fiicì (< *tə/ci?) &  &  &  &  & \\
Laal\il{Laal}   &  &  &  & ʔīsī (ʔī-sī?) &  &  &  &  & \\
\lspbottomrule
\end{tabularx}
\end{sidewaystable}

\subsubsection{‘Three’}%3.6.7.3.


Comparative evidence for this root points to its reconstruction as *\textit{taat} (with further alignment by analogy within each of the branches). As in the other NC families, the root is exceptionally stable, in contrast to the roots for ‘one’ and ‘two’ that demonstrate a wide variety of forms. A shared innovation in Jen and Waja\il{Waja} (attested in Burak\il{Burak}, Awak\il{Awak} and Waja) is noteworthy.

 
\subsubsection{‘Four’}%3.6.7.4.
\begin{table}
\caption{\label{tab:3:120}Adamawa stems for ‘4’}


\begin{tabularx}{\textwidth}{l QlllQ}
\lsptoprule

Fali-\il{Fali}Yingilum\il{Yingilum}  & naan &  &  &  & \\
Kam\il{Kam}  & nár\newline (< *naX) &  &  &  & \\
\textit{Leko-Duru-Mumuye}\\
Duru\il{Duru} & nató/naró (< *naX) &  &  &  & \\
Leko & \mbox{naarà/nɛɛr-əb} &  &  &  & \\
Mumuye\il{Mumuye} & naat &  &  &  & \\
\textit{Mbum-Day}\\
~~~~Bua\il{Bua} & na/nagi/niani &  &  & har & \\
~~~~Kim\il{Kim} &  &  & ndà(y) &  & \\
~~~~Mbum\il{Mbum} & nai & nìŋ &  &  & \\
~~~~Day\il{Day} &  &  & ndà &  & *bī-yām\\
\textit{Waja-Jen}\\
~~~~Jen & net & bwa-nyə &  &  & \\
~~~~Longuda\il{Longuda} & nnyìr/nyìr &  &  &  & \\
~~~~Waja\il{Waja} & naat &  &  & ɡwár & \\
~~~~Yungur\il{Yungur} &  &  &  &  & kurun\\
Laal\il{Laal} &  &  &  &  & ɓīsān (ɓī-sān?)\\
\lspbottomrule
\end{tabularx}
\end{table}

The main NC form \textit{*na}X is predominant here, its second consonant being subject to alignment by analogy. The same root is likely to be reconstructed at the Proto-Adamawa\il{Proto-Adamawa} level as well. 

\newpage 
\subsubsection{‘Five’} %3.6.7.5.

The main root (\textit{nun}) may be the same as in the Gur languages and may be etymologically related to the term for ‘hand’. It is likely that the isolated forms quoted in the rightmost column go back to similar terms as well. The Jen root \textit{hmə} could be a borrowing from Chadian Arabic\il{Arabic}: \textit{xamsa} ‘5’. The Mbum\il{Mbum} forms \textit{ndēɓē/} \textit{dūwēe} may be influenced by Fula\il{Fula} (\textit{jowi} ‘five’).

\begin{table}
\caption{\label{tab:3:121}Adamawa stems for `5'}
\begin{tabularx}{\textwidth}{lQQQQ}
\lsptoprule
Fali-\il{Fali}Yingilum\il{Yingilum}  &  & k{\~{ɛ}}rɛw &  & \\
Kam\il{Kam}  & ŋwún &  &  & \\
\textit{Leko-Duru-Mumuye}\\
~~~~Duru\il{Duru} & núno/ nɔɔn{\`{ɨ}}, &  &  & ɡbà náárò/ ɡbanáá, sáá\\
~~~~Leko & núúnà/ núnn-ub &  &  & \\
~~~~Mumuye\il{Mumuye} & nɔng/ ɡhìnān &  &  & m{\v{a}}ːni\\
\textit{Mbum-Day}\\
~~~~Bua\il{Bua} &  &  &  & luni/loni/ *lu,tɛ(r), *kɔn?, (tiso)\\
~~~~Kim\il{Kim} & nūw{\textsubtilde{ē}}y &  & ndìyārá & \\
~~~~Mbum\il{Mbum} &  &  & ndiɓi/ dūwēe/ dáp{\`{ɪ}} & \\
~~~~Day\il{Day} &  &  &  & s{\={ɛ}}rì\\
\textit{Waja-Jen}\\
~~~~Jen & nóob/*na & -hmə/*hw{\~{i}} &  & \\
~~~~Longuda\il{Longuda} & ny{\'{ɔ}} &  &  & \\
~~~~Waja\il{Waja} & nu(ŋ) &  &  & fwáːd\\
~~~~Yungur\il{Yungur} & wo-non/wo-nun &  &  & \\
Laal\il{Laal}  &  &  &  & sāb, *swa-\\
\lspbottomrule
\end{tabularx}
\end{table}


\newpage 
\subsubsection{‘Six’} %3.6.7.6.
\begin{table}
\caption{\label{tab:3:122}Adamawa stems and patterns for `6'}

\small
\begin{tabularx}{\textwidth}{llllQ}
\lsptoprule

Fali-\il{Fali}Yingilum\il{Yingilum} &  &  &  & yira/yilo\\
Kam\il{Kam}  &  & jù:p &  & \\
\textit{Leko-Duru-Mumuye}\\
~~~~Duru\il{Duru} & 5+1 & ɡúú &  & \\
~~~~Leko &  &  &  & n{\^{ɔ}}ŋɡ{\^{ɔ}}s/núŋɡ{\'{ɔ}}ɔs\\
~~~~Mumuye\il{Mumuye} & 5+1 &  &  & \\
\textit{Mbum-Day}\\
~~~~Bua\il{Bua} & 5+1 &  &  & táːr, (nānò), (kaba), tipsi\\
~~~~Kim\il{Kim} &  &  &  & mānɡùl/mènènɡāl\\
~~~~Mbum\il{Mbum} &  &  &  & ze(y)/ye(a), t{\'{ɔ}}t{\'{ɔ}}kl{\'{ɔ}}, bì-ɡírò\\
~~~~Day\il{Day} & 5+1 &  &  & \\
\textit{Waja-Jen}\\
~~~~Jen & 5+1 &  &  & \\
~~~~Longuda\il{Longuda} &  &  & tsààt{\`{ə}}n & 2*3?\\
~~~~Waja\il{Waja} & nu-kun (<5+1?) &  &  & \\
~~~~Yungur\il{Yungur} &  &  &  & mindike\\
Laal\il{Laal} &  &  & cìcààn~ & \\
\lspbottomrule
\end{tabularx}
\end{table}

The most frequently attested pattern is ‘5+1’. However, there is a great variety of isolated forms (see the last column). The similarity between the Laal\il{Laal} and Longuda\il{Longuda} forms is noteworthy; both may go back to Chadian Arabic\il{Arabic} \textit{sitːe} ‘six’. The Kim\il{Kim} (and also Yungur\il{Yungur}?) form could be a borrowing from Bagirmi\il{Bagirmi} (\textit{mìká} ‘6’).

\newpage 
\subsubsection{‘Seven’}%3.6.7.7.
\begin{table}
\caption{\label{tab:3:123}Adamawa stems and patterns for `7'}


\begin{tabularx}{\textwidth}{llllQQ}
\lsptoprule

Fali-\il{Fali}Yingilum\il{Yingilum}  &  &  &  &  & j{\textsubbar{ɔ}}rɔs\\
Kam\il{Kam}  &  &  & ‘second six’ &  & \\
\textit{Leko-Duru-Mumuye}\\
~~~~Duru\il{Duru} & 5+2 & 4+3 & 6+'odd' &  & ɡútambe, d{\'{ə}}msàrà\\
~~~~Leko & 5+2 &  &  &  & \\
~~~~Mumuye\il{Mumuye} & 5+2 &  &  &  & \\
\textit{Mbum-Day}\\
~~~~Bua\il{Bua} & 5+2 & 3+4 &  &  & lúlú/lòŋɡ{\={ɔ}}/ lur, (tiglen)\\
~~~~Kim\il{Kim} &  &  &  & ɓēálā/ɓēálār & ɗīyārā\\
~~~~Mbum\il{Mbum} &  &  &  &  & 10--3, rɪŋ, r{\"{e}}nām, tàrn{\'ã}ɡà\\
~~~~Day\il{Day} &  & 4+3 &  &  & \\
\textit{Waja-Jen}\\
~~~~Jen & 5+2 &  &  &  & \\
~~~~Longuda\il{Longuda} &  & 4+3 &  &  & \\
~~~~Waja\il{Waja} &  &  &  & ni-bir/-bil/ -bi(y) & \\
~~~~Yungur\il{Yungur} &  &  &  &  & nbutu\\
Laal\il{Laal} & 5+2 &  &  &  & \\
\lspbottomrule
\end{tabularx}
\end{table}

As in the case of ‘six’, the predominant pattern (‘5+2’) for ‘seven’ is rather plain. It co-exists with a variety of isolated forms of uncertain etymology. 


\newpage 
\subsubsection{‘Eight’} %3.6.7.8.
\begin{table}
\caption{\label{tab:3:124}Adamawa stems and patterns for `8'} 
\begin{tabularx}{\textwidth}{llllX}
\lsptoprule

Fali-\il{Fali}Yingilum\il{Yingilum} & 4 redupl. &  &  & \\
Kam\il{Kam} &  &  &  & sâl\\
\textit{Leko-Duru-Mumuye}\\
~~~~Duru\il{Duru} & 4PL/4+4 & 5+3 &  & < Hausa\il{Hausa}\\
~~~~Leko &  & 5+3 &  & < Hausa\il{Hausa}\\
~~~~Mumuye\il{Mumuye} &  & 5+3 &  & \\
\textit{Mbum-Day}\\
~~~~Bua\il{Bua} & 4 redupl. & 5+3 &  & \\
~~~~Kim\il{Kim} & ndāsì (4PL?) &  & wázìzí (10–2) & tīmāl\\
~~~~Mbum\il{Mbum} &  &  & 10--2 & nam(m)a/ nènmàʔ{\"{a}}\\
~~~~Day\il{Day} & 4 redupl.? &  &  & \\
\textit{Waja-Jen}\\
~~~~Jen & 4PL & 5+3 &  & \\
~~~~Longuda\il{Longuda} &  &  &  & nyíthìn\\
~~~~Waja\il{Waja} & 4*2 &  &  & \\
~~~~Yungur\il{Yungur} & 4 redupl. &  &  & \\
Laal\il{Laal} & 4 redupl. &  &  & \\
\lspbottomrule
\end{tabularx}
\end{table}

The pattern ‘8=4 redupl.’ is to be reconstructed at the Proto-Adamawa\il{Proto-Adamawa} level.


 \newpage 
\subsubsection{‘Nine’}%3.6.7.9.
\begin{table}
\caption{\label{tab:3:125}Adamawa stems and patterns for `9'}
\begin{tabularx}{\textwidth}{llQl}
\lsptoprule

Fali-\il{Fali}Yingilum\il{Yingilum} &  & 10--1/ŋɡʌs kàm(kàn) kpòlò\newline `rest hand one' & \\
Kam\il{Kam} &  &  & níízaa\\
\textit{Leko-Duru-Mumuye}\\
~~~~Duru\il{Duru} &  & `one finger is left', n{\'{ɨ}}{\`{ŋ}}s{\'{ɨ}}nè, 5+4, 10--1 & \\
~~~~Leko & 5+4 & `one is left' & \\
~~~~Mumuye-\il{Mumuye}Yandang & 5+4 &  & \\
\textit{Mbum-Day}\\
~~~~Bua\il{Bua} & 5+4 & 10-X & ti, jar\\
~~~~Kim\il{Kim} &  & 10--1 & nòmīnā\\
~~~~Mbum\il{Mbum} &  & 10--1 & doraŋ\\
~~~~Day\il{Day} &  & `lacking one' & \\
\textit{Waja-Jen}\\
~~~~Jen & 5+4 &  & \\
~~~~Longuda\il{Longuda} & 5+4 &  & \\
~~~~Waja\il{Waja} &  & 10--1 & teer/teet\\
~~~~Yungur\il{Yungur} & 5+4 &  & \\
Laal\il{Laal} &  &  & yàŋjáŋ~\\
\lspbottomrule
\end{tabularx}
\end{table}

A primary term for ‘nine’ was apparently non-existent in Proto-Adamawa\il{Proto-Adamawa}. A comparison between Bua\il{Bua} \textit{diar} and Kanuri\il{Kanuri} \textit{ləɣár} may be suggestive if a borrowing is considered. The same applies to the terms for ‘nine’ in Waja\il{Waja} (\textit{tɔɔrɔ}) and Hausa\il{Hausa} (\textit{tara}).


\newpage    
\subsubsection{‘Ten’}%3.6.7.10.
 Two alternative roots for ‘ten’ (\tabref{tab:3:126}) are distinguishable (*\textit{boo} and *\textit{kob} attested in four and two groups respectively). The root \textit{d}(\textit{u})\textit{o} is observable in two Mbum\il{Mbum}-Day\il{Day} sub-groups. Finally, the root \textit{kutu}(\textit{n}) is found in two languages, namely in Tunya\il{Tunya} (Bua\il{Bua}) and Kaan\il{Kaan} (Yungur\il{Yungur}). Assuming that \textit{ku}- is a class prefix, this root may prove to be related to \textit{tūū} (Laal\il{Laal}). 

\begin{table}
\caption{\label{tab:3:126}Adamawa stems for ‘10’}
\begin{tabularx}{\textwidth}{llXllXX}
\lsptoprule

Fali-\il{Fali}Yingilum\il{Yingilum}  &  &  &  &  &  & ra\\
Kam\il{Kam}  & bóò &  &  &  &  & \\
\textit{Leko-Duru-Mumuye}\\
Duru\il{Duru} & bōʔ, & kob/kop/ fób &  &  &  & \\
Leko &  & kób/kóp &  &  &  & \\
Mumuye\il{Mumuye} &  & kop/kob &  &  &  & \\
\textit{Mbum-Day}\\
~~~~Bua\il{Bua} &  &  & do(k) & kùtù &  & (filoːle), (yíppà), (teba)\\
~~~~Kim\il{Kim} &  &  &  &  & wàl/ wòl/ wàr/ *wèy & \\
~~~~Mbum\il{Mbum} & boo &  & dùɔ &  & \mbox{hù-wàlë} & dʒama/ dʒémà\\
~~~~Day\il{Day} & m{\textsubtilde{ò}} &  &  &  &  & \\
\textit{Waja-Jen}\\
~~~~Jen &  & ʃóób &  &  &  & bwa-hywə\\
~~~~Longuda\il{Longuda} &  & koo/kù &  &  &  & n{\^{ɔ}}m\\
~~~~Waja\il{Waja} &  & k{\'{ɔ}}b/kub/ kwab/ kpop/ kwu &  &  &  & \\
~~~~Yungur\il{Yungur} & bú(u) &  &  & kutun &  & \\
Laal\il{Laal} &  &  &  & tūū &  & \\
\lspbottomrule
\end{tabularx}
\end{table}


 \newpage
\subsubsection{‘Twenty’}%3.6.7.11.
The term for ‘twenty’ (\tabref{tab:3:127}) in the Duru\il{Duru} languages either follows the pattern ‘20=10*2’ or goes back to the lexical roots for ‘head’ and ‘staff’. The Niellim\il{Niellim} term \textit{do-ksap} was likely borrowed from Bagirmi\il{Bagirmi} \textit{dùɡ} \textit{sap} ‘twenty’.

\begin{table}[h]
\caption{\label{tab:3:127}Adamawa stems and patterns for ‘20’}


\begin{tabularx}{\textwidth}{llllQQ}
\lsptoprule

Fali-\il{Fali}Yingilum\il{Yingilum}  & 10*2 &  &  &  & \\
Kam\il{Kam} &  &  &  &  & *{\`{n}}kp{\textsubtilde{ó}}, kpáímí\\
\textit{Leko-Duru-Mumuye}\\
~~~~Duru\il{Duru} & 10*2 &  &  & ɡbɛɡ/ ɡbàhs{\'{ɨ}} ('staff'), *w{\'{ɔ}}{\'{ɔ}}ɡ ('head'), zul/zur (‘head') & ráárò, jùɡúyɔ\\
~~~~Leko &  &  & laa-1 &  & nɛd níi ɡbɛd\\
~~~~Mumuye\il{Mumuye} &  &  &  &  & mba-1, kar-1, mim-1\\
\textit{Mbum-Day}\\
~~~~Bua\il{Bua} & 10*2 &  & faːlɛ &  & do-ksap, a-rep, a-hun\\
~~~~Kim\il{Kim} & 10*2 &  &  &  & \\
~~~~Mbum\il{Mbum} & 10*2 & `2 hands', 10+10 &  &  & \\
~~~~Day\il{Day} & 10*2 &  &  &  & \\
\textit{Waja-Jen}\\
~~~~Jen &  &  & fa-1 &  & nɡwu-1\\
~~~~Longuda\il{Longuda} & 10*2 &  &  &  & \\
~~~~Waja\il{Waja} & 10*2 & `2 hands' &  &  & \\
~~~~ Yungur\il{Yungur} & 10*2 &  &  &  & \\
Laal\il{Laal} & 10*2 &  &  &  & \\
\lspbottomrule
\end{tabularx}
\end{table}

\clearpage 
\subsubsection{‘Hundred’}%3.6.7.12.
\begin{table}
\caption{\label{tab:3:128}Adamawa stems and patterns for `100'}


\begin{tabularx}{\textwidth}{lllllQ}
\lsptoprule

Fali-\il{Fali}Yingilum\il{Yingilum} &  &  &  &  & < Fula\il{Fula}\\
Kam\il{Kam}  & 20*5 &  &  &  & \\
Leko-Duru-Mumuye\\
~~~~Duru\il{Duru} & 20*5 &  &  &  & < Fula\il{Fula}\\
~~~~Leko & 20*5 &  &  &  & < Fula\il{Fula}\\
~~~~Mumuye\il{Mumuye} & 20*5 &  &  &  & \\
Mbum-Day\\
~~~~Bua\il{Bua} &  &  &  & ro/ru & \\
~~~~Kim\il{Kim} &  &  &  &  & < Arabic\il{Arabic}\\
~~~~Mbum\il{Mbum} &  &  & s{\'{ɔ}}ɗ/sɔt &  & < Fula,\il{Fula} < Arabic\il{Arabic}\\
~~~~Day\il{Day} &  &  & tù &  & \\
Waja-Jen\\
~~~~ Jen & 20*5 &  &  &  & \\
~~~~Longuda\il{Longuda} &  &  &  &  & pùlò(wé)/phulewé\\
~~~~ Waja\il{Waja} &  & <10? &  &  & wɔn, bwa-tiɡɛ\\
~~~~ Yungur\il{Yungur} &  &  &  & (-ru) & \\
Laal\il{Laal} &  & 10-'big' &  &  & \\
\lspbottomrule
\end{tabularx}
\end{table}

The fact that this term was massively borrowed (most likely simultaneously) from Fula\il{Fula} and Arabic\il{Arabic} suggests that it was lacking in Proto-Adamawa\il{Proto-Adamawa}. It can be assumed that the root \textit{ru} attested in Bua\il{Bua} and Yungur\il{Yungur} is also a borrowing, this time from Bagirmi\il{Bagirmi} \textit{àrú} ‘hundred’.

 \newpage 
\subsubsection{‘Thousand’}%3.6.7.13.
\begin{table}
\caption{\label{tab:3:129}Adamawa stems and patterns for `1000'}


\begin{tabularx}{\textwidth}{lQl}
\lsptoprule

Fali-\il{Fali}Yingilum\il{Yingilum}   &  & < Fula\il{Fula}\\
Kam\il{Kam}   &? & \\
Leko-Duru-Mumuye\\
~~~~Duru\il{Duru} &  & < Fula,\il{Fula} < Hausa\il{Hausa}\\
~~~~Leko & 20*10? & < Fula\il{Fula}\\
~~~~Mumuye\il{Mumuye} &? & \\
Mbum-Day\\
~~~~Bua\il{Bua} &  & < Bagirmi\il{Bagirmi}\\
~~~~Kim\il{Kim} &  & < Bagirmi\il{Bagirmi}\\
~~~~Mbum\il{Mbum} & `sack', bag' & < Fula,\il{Fula} < Bagirmi\il{Bagirmi}\\
~~~~Day\il{Day} &  & < Bagirmi\il{Bagirmi}\\
Waja-Jen\\
~~~~Jen & ʃik-1, 20-fe & \\
~~~~Longuda\il{Longuda} &? & \\
~~~~ Waja\il{Waja} & kʊʊl, nèe/kú-néŋ, 100*10, bi-kate, tedu & \\
~~~~ Yungur\il{Yungur} & (100*10) & \\
Laal\il{Laal} &  & < Baguirmi, < Hausa\il{Hausa}\\
\lspbottomrule
\end{tabularx}
\end{table}

The term for ‘thousand’ was massively borrowed from Fula\il{Fula}, Bagirmi\il{Bagirmi} and Hausa\il{Hausa}, which points to its absence in the proto-language. 

\newpage  
\section{Ubangi}%3.7

What follows is a preliminary analysis of the evidence of five separate language groups including Ubangi-Banda\il{Banda}, Gbaya\il{Gbaya}-Manza-Ngbaka\il{Ngbaka}, Ngbandi\il{Ngbandi}, Sere\il{Sere}-Ngbaka-Mba\il{Mba} (A. Ngbaka-Mba, B.Sere), and Zande\il{Zande}.
 
 
\subsection{Banda}%3.7.1.
\il{Banda}


The form \textit{gba} ‘ten’ is traceable in the Mbanza\il{Mbanza} (Mabandja) terms for tens.
\begin{table}
\caption{\label{tab:3:130}Numerals in Banda\il{Banda}}


\begin{tabularx}{\textwidth}{ll@{}rl}
\lsptoprule

{1} & bàlē (bà-lē?) & {7} & 5+2\\
{2} & biʃi (bi-ʃi?) & {8} & 5+3, nɡebeɗeɗe\\
{3} & vɔ-tɑ & {9} & 5+4, 8+1\\
{4} & v{\`{ɑ}}-n{\={ɑ}} & {10} & mó-rófō, bu-fu, `two hands `,'all the fingers',*gba \\
{5} & mī-ndū & {20} & `one person', `the whole person', `body-person-all'\\
{6} & 5+1, ɡɑzɑlɑ & {100} & nɡàmb{\`{ɔ}}/nɡbànɡbò,'five persons' , < Sango\il{Sango} , < Lingala?\il{Lingala}\\
&  & {1000} & < French\il{French} `sack', < Lingala?\il{Lingala}\\
\lspbottomrule
\end{tabularx}
\end{table}



\subsection{Gbaya-Manza-Ngbaka}%3.7.2.
\il{Gbaya}\il{Ngbaka}
\begin{table}
\caption{\label{tab:3:131}Numerals in Gbaya\il{Gbaya}-Manza-Ngbaka\il{Ngbaka}}


\begin{tabularx}{\textwidth}{llrX}
\lsptoprule

{1} & *kp{\'{ɔ}}k/kpóm ;ndáŋ & {7} & *5+2\\
{2} & *bùà, *ɭíítò; bùwá (bù-wá?)/vàχ, -too & {8} & *5+3; 4PL\\
{3} & *tàr(à) & {9} & *5+4;kùsì\\
{4} & *nár(á) & {10} & *ɓú/ɓú-k{\textsubtilde{\'{ɔ}}}\\
{5} & *m{\`{ɔ}}{\`{ɔ}}r{\'{ɔ}}/m{\`{ɔ}}r-k{\textsubtilde{\'{ɔ}}} & {20} & *10*2\\
{6} & *5+1, (ɡàz{\`{ɛ}}l{\`{ɛ}}) & {100} & *g{\'{ɔ}}m-màá ; < Lingala\il{Lingala}\\
&  & {1000} & < French\il{French} `sack', < Lingala\il{Lingala}\\
\lspbottomrule
\end{tabularx}
\end{table}

Ives Moñino’s reconstructions \citep{Moñino1995} are quoted in the table under an asterisk. Selected noteworthy forms are also included.

In the diachronical perspective, the forms \textit{*ɭíítò} and \textit{*bùà} ‘two’ probably included noun class prefixes. They go back to \textit{*-too} and \textit{*-wa} respectively (cf.\textit{vàχ} ‘2’ in Gbaya\il{Gbaya} Mbodomo\il{Gbaya Mbodomo}).

In his discussion of \textit{*m{\`{ɔ}}{\`{ɔ}}r{\'{ɔ}}} Moñino states that “La variante \textit{*m{\`{ɔ}}{\`{ɔ}}r{\'{ɔ}}} semble être une contraction de \textit{*m{\`{ɔ}}r-k{\textsubtilde{\'{ɔ}}}}, dans laquelle on peut reconnaître l’élément \textit{k{\textsubtilde{\'{ɔ}}}} ‘main’ …~” \citep[655]{Moñino1995}. He also makes the folowing observation regarding the reconstruction of the term for ‘ten’: “\textit{*ɓú} ‘dix’ est en relation avec \textit{*ɓú} ‘façonner, faire un cercle, joindre les mains’; la série partielle \textit{ɓú-k{\textsubtilde{\'{ɔ}}}} est encore plus explicite, et décrit le geste qui accompagne l’énonciation du chiffre 10 chez tous les locuteurs” \citep[656]{Moñino1995}.\footnote{However, in some Gbaya\il{Gbaya} languages, these forms differ by tone: Gbaya (Roulon-Doko) \textstylefrm{ɓú ‘10’ {\textasciitilde} ɓu} \textstyletra{’to tap; to applaud, to roll’.}} This is an important point, especially in view of the relatively frequent occurrence of \textit{bu} in the NC languages and the possible etymological relationship between \textit{*ɓú} and phonetically similar forms attested in other branches. However, such a relationship would be doubtful within Moñino’s etymological hypothesis. 

The following etymology is suggested for ‘hundred’ by Thomas Elvis Guene\-ke\-an: “The word \textit{g{\`{\~ɔ}}m} means ‘cut’ or ‘gathered’ and \textit{n͡màː} means ‘things’.”\footnote{\url{https://mpi-lingweb.shh.mpg.de/numeral/Gbaya-Bossangoa.htm}} According to Moñino, the form literally means ‘frapper-l’une l’autre (les mains)’ \citep[657]{Moñino1995}.


\subsection{Ngbandi}%3.7.3.
\il{Ngbandi}The Ngbandi\il{Ngbandi} and Yakoma\il{Yakoma} evidence points toward the reconstruction outlined in the table below (\tabref{tab:3:132}).

\begin{table}
\caption{\label{tab:3:132}Numerals in Ngbandi\il{Ngbandi}}


\begin{tabularx}{\textwidth}{lXrX}
\lsptoprule

{1} & kɔ(i) & {7} & mbara-mbara\\
{2} & sɛ & {8} & miambe/my{\`{ɔ}}mbè\\
{3} & ta & {9} & ɡumbaya\\
{4} & siɔ/syɔ & {10} & sui, bàlé\\
{5} & k{\~{ɔ}}/k{\textsubtilde{ū}} & {20} & 10*2\\
{6} & mana, m{\`{ɛ}}r{\={ɛ}} & {100} & nɡbanɡbo\\
&  & {1000} & < Lingala,\il{Lingala} Arabic\il{Arabic}\\
\lspbottomrule
\end{tabularx}
\end{table}

\newpage  
\subsection{Sere-Ngbaka-Mba}%3.7.4.
\il{Sere}\il{Ngbaka}\il{Mba}Since the languages within this group are extremely divergent, it seems reasonable to treat the evidence from its two major sub-groups separately. 

Ngbaka\il{Ngbaka}-Mba\il{Mba} (\tabref{tab:3:133})

\begin{table}
\caption{\label{tab:3:133}Numerals in Ngbaka\il{Ngbaka}-Mba\il{Mba}}


\begin{tabularx}{\textwidth}{lXrQ}
\lsptoprule

{1} & kpó-/kpáà-, ɓa-wɨ,\newline  ɓī-nì/bì-rì, ú-ma & {7} & 5+2, (mā-nāníkà, l{\`{ɵ}}-rɵzi, zyálá, sáɓá), sílànā/sélènā/ʃíēnā (<4?)\\
{2} & bīʃ-ì/ɓī-sī, ɓi-né/bí-de, ɡbw{\`{ɔ}} & {8} & s{\'{ɛ}}nā (2*4?), ɡba-dzena/ mā-dʒ{\'{ɛ}}nà, (5+3, 10--2)\\
{3} & ba-ta/ba-la & {9} & 5+4, 10--1, (me-newá)\\
{4} & ba-na/ba-ɗa/ba-la & {10} & nzò kp{\textsubtilde{ā}}(‘head-hand')/ànɡbà,\newline  a-busa\\
{5} & bu-ruwe/bu-luve/θuwe, ʔeve/ve/vue & {20} & 10*2\\
{6} & ʃí-tà/si-ta (2*3),\newline  mā-ɗíà/ká-zyá, 5+1 & {100} & < Sango,\il{Sango} < Lingala,\il{Lingala} 20*5, (mya, k{\'{ʉ}}l{\'{ɔ}}, kpode, nɡūndānɡū)\\
&  & {1000} & gyu, kutu, < Arabic,\il{Arabic} < French\il{French} (‘sack’), 100*10\\
\lspbottomrule
\end{tabularx}
\end{table}

Sere\il{Sere} (\tabref{tab:3:134})

\begin{table}
\caption{\label{tab:3:134}Numerals in Sere\il{Sere}}


\begin{tabularx}{\textwidth}{lXrX}
\lsptoprule

{1} & nj{\~{e}}e & {7} & 5+2\\
{2} & so & {8} & 5+3\\
{3} & táʔò & {9} & 5+4\\
{4} & nàʔò & {10} & ɓ{\~{ï}}-k{\"{u}}r{\"{u}} , muʔɓì (‘on hands')\\
{5} & vo & {20} & `kill-person-one'\\
{6} & 5+1 & {100} & `kill-persons-five', < Arabic\il{Arabic}\\
&  &  {1000} & 100*10\\
\lspbottomrule
\end{tabularx}
\end{table}

\newpage 
Sere\il{Sere}-Ngbaka\il{Ngbaka}-Mba\il{Mba} (\tabref{tab:3:135})

\begin{table}
\caption{\label{tab:3:135}Sere\il{Sere}-Ngbaka\il{Ngbaka}-Mba\il{Mba} numeral system (*)}


\begin{tabularx}{\textwidth}{lXrX}
\lsptoprule

{1} & kí-lī, sa & {7} & 5+2\\
{2} & ī-jō/ī-yō/úé & {8} & 5+3\\
{3} & bíá-tá/ā-tā & {9} & 5+4\\
{4} & lu, bīà-nɡì {\textasciitilde} bīà-mà & {10} & ŋɡb{\={\~{ɔ}}}/bà-wē~\\
{5} & ì-sìbē/bī-sùè & {20} & `people one'\\
{6} & 5+1 & {100} & nd{\={ɔ}}ŋɡb{\'{ʉ}}, nɡbànɡbù< Sango\il{Sango}\\
&  & {1000} & sákì/sākè (< Sango\il{Sango} < French)\il{French}\\
\lspbottomrule
\end{tabularx}
\end{table}

  
\subsection{Proto-Ubangi}%3.7.5.
\il{Proto-Ubangi}The evidence pertaining to each of the numerical terms is summarized below.

\subsubsection{‘One’} %3.7.5.1.
\begin{table}
\caption{\label{tab:3:136}Ubangi stems for `1'}


\begin{tabularx}{\textwidth}{l@{} QQQll@{}ll@{}l}
\lsptoprule
Banda\il{Banda}   & bàlē (bà-lē?) &  &  &  &  &  & \\
Gbaya-\il{Gbaya}Manza-Ngbaka\il{Ngbaka}   &  & kp{\'{ɔ}}(k)/ (kpém) & ndáŋ &  &  &  & \\
Ngbandi\il{Ngbandi}  &  & kɔ(i) &  &  &  &  & \\
\textit{Sere-Ngbaka-Mba}\\
~~~~Ngbaka-\il{Ngbaka}Mba\il{Mba} & ɓī-nì/bì-rì & kpó-/ kpáà- &  &  & ɓa-wɨ & ú-ma & \\
~~~~Sere\il{Sere} &  &  &  & nj{\~{e}}e &  &  & \\
Zande\il{Zande}  & kí-lī &  &  &  &  &  & sa\\
\lspbottomrule
\end{tabularx}
\end{table}

Two competing roots (\textit{*le}/\textit{ne} and \textit{*k(p)o(k})) are distinguishable here. 

\newpage 
\subsubsection{‘Two’}%3.7.5.2.
\begin{table}
\caption{\label{tab:3:137}Ubangi stems for `2'}


\begin{tabularx}{\textwidth}{l lll@{}l}
\lsptoprule

Banda\il{Banda}& biʃi (bi-ʃi?) &  &  &  \\
Gbaya-\il{Gbaya}Manza-Ngbaka\il{Ngbaka} &  & bùwá (bù-wá?)/vàχ & -too &  \\
Ngbandi\il{Ngbandi}& sɛ &  &  &  \\
\textit{Sere-Ngbaka-Mba}\\
~~~~Ngbaka-\il{Ngbaka}Mba\il{Mba} & bī-ʃì/ɓī-sī & ɡbw{\`{ɔ}} &  & ɓi-né/bí-de\\
~~~~Sere\il{Sere} &  &  & so &  \\
Zande\il{Zande}&  &  &  & ī-jō/ī-yō/úé \\
\lspbottomrule
\end{tabularx}
\end{table}

The only root widely attested within this family is \textit{*si/ʃi}.

  
\subsubsection{‘Three’ and ‘four’}%3.7.5.3.
\begin{table}
\caption{\label{tab:3:138}Ubangi stems for `3' and `4'}


\begin{tabularx}{\textwidth}{lXl@{}l}
\lsptoprule

~ & `3' & `4' & `4' \\
\midrule
Banda\il{Banda}& vɔ-tɑ & v{\`{ɑ}}-n{\={ɑ}} & \\
Gbaya-\il{Gbaya}Manza-Ngbaka\il{Ngbaka} & tààr & náár & \\
Ngbandi\il{Ngbandi}& ta &  & siɔ/syɔ\\
\textit{Sere-Ngbaka-Mba}\\
~~~~Ngbaka-\il{Ngbaka}Mba\il{Mba} & ba-ta/ba-la & ba-na/ba-ɗa/ba-la & \\
~~~~Sere\il{Sere} & táʔò & nàʔò & \\
Zande\il{Zande}& bíá-tá/ā-tā &  & lu, bīà-nɡì {\textasciitilde} bīà-mà \\
\lspbottomrule
\end{tabularx}
\end{table}

The roots for ‘three’ and ‘four’ can be securely reconstructed as \textit{*taar} and \textit{*naar} respectively (with an alignment by analogy applied).

\newpage
\subsubsection{‘Five’}%3.7.5.4.
\begin{table}
\caption{\label{tab:3:139}Ubangi stems for `5'}


\begin{tabularx}{\textwidth}{l@{}l@{}l@{}l@{}l}
\lsptoprule

Banda\il{Banda}& mī-ndū &  &  & \\
Gbaya-\il{Gbaya}Manza-Ngbaka\il{Ngbaka} &  & m{\`{ɔ}}r-(k){\'{ɔ}} &  & \\
Ngbandi\il{Ngbandi}&  & k{\~{ɔ}}/k{\textsubtilde{ū}} &  & \\
\textit{Sere-Ngbaka-Mba}\\
~~~~Ngbaka-\il{Ngbaka}Mba\il{Mba} & bu-ruwe/-luve/θuwe &  & ʔeve {\textasciitilde} ve/vue & \\
~~~~Sere\il{Sere} &  &  & vo & \\
Zande\il{Zande}&  &  &  & ì-sìbē/bī-sùè\\
\lspbottomrule
\end{tabularx}
\end{table}

The Proto-Ubangi\il{Proto-Ubangi} form is unclear, since the term for ‘five’ is based on the lexical root meaning ‘hand’ (\textit{*kɔ}) in two groups out of five. The only root whose attestations are not limited to a single group is \textit{*du(w)/lu(w).}

    
\subsubsection{‘Six’}%3.7.5.5.
\begin{table}
\caption{\label{tab:3:140}Ubangi stems and patterns for `6'}


\begin{tabularx}{\textwidth}{lXlX}
\lsptoprule

Banda\il{Banda}& 5+1 & ɡɑ-zɑlɑ & \\
Gbaya-\il{Gbaya}Manza-Ngbaka\il{Ngbaka} & 5+1 & ɡà-z{\`{ɛ}}l{\`{ɛ}} & \\
Ngbandi\il{Ngbandi}&  &  & ma-na, m{\`{ɛ}}-r{\={ɛ}}\\
\textit{Sere-Ngbaka-Mba}\\
~~~~Ngbaka-\il{Ngbaka}Mba\il{Mba} & 5+1 & mā-ɗíà/ká-zyá & ʃí-tà/si-ta (2*3)\\
~~~~Sere\il{Sere} & 5+1 &  & \\
Zande\il{Zande}& 5+1 &  & \\
\lspbottomrule
\end{tabularx}
\end{table}

In addition to forms that follow the common pattern ‘6=5+1’, a number of other forms of uncertain etymology are attested in the first two groups (and possibly in Sere\il{Sere}-Ngbaka\il{Ngbaka}-Mba\il{Mba} as well, assuming that our morphological analysis of pertinent forms is correct).
\newpage  

\subsubsection{‘Seven’}%3.7.5.6.
\begin{table}
\caption{\label{tab:3:141}Ubangi stems and patterns for `7'}


\begin{tabularx}{\textwidth}{llQ}
\lsptoprule

Banda\il{Banda}& 5+2 & \\
Gbaya-\il{Gbaya}Manza-Ngbaka\il{Ngbaka} & 5+2 & \\
Ngbandi\il{Ngbandi}&  & mbara-mbara\\
\textit{Sere-Ngbaka-Mba}\\
~~~~Ngbaka-\il{Ngbaka}Mba\il{Mba} & 5+2 & mā-nāníkà, l{\`{ɵ}}-rɵzi, zyálá, sáɓá, sílànā/sélènā/ʃíēnā (<4?)\\
~~~~Sere\il{Sere} & 5+2 & \\
Zande\il{Zande}& 5+2 & \\
\lspbottomrule
\end{tabularx}
\end{table}

The variety of forms attested in Ngbaka\il{Ngbaka}-Mba\il{Mba} is noteworthy.

\subsubsection{‘Eight’} %3.7.5.7.
\begin{table}
\caption{\label{tab:3:142}Ubangi stems and patterns for `8'}


\begin{tabularx}{\textwidth}{lllX}
\lsptoprule

Banda\il{Banda}& 5+3 &  & nɡebeɗeɗe\\
Gbaya-\il{Gbaya}Manza-Ngbaka\il{Ngbaka} & 5+3 & 4PL & \\
Ngbandi\il{Ngbandi}&  &  & miambe/my{\`{ɔ}}mbè\\
\textit{Sere-Ngbaka-Mba}\\
~~~~Ngbaka-\il{Ngbaka}Mba\il{Mba} & 5+3 & s{\'{ɛ}}nā (2*4?) & ɡ͡ba-dzena/mā-dʒ{\'{ɛ}}nà, 10--2\\
~~~~Sere\il{Sere} & 5+3 &  & \\
Zande\il{Zande}& 5+3 &  & \\
\lspbottomrule
\end{tabularx}
\end{table}

\subsubsection{‘Nine’}%3.7.5.8.
\begin{table}
\caption{\label{tab:3:143}Ubangi stems and patterns for `9'}


\begin{tabularx}{\textwidth}{lXX}
\lsptoprule

Banda\il{Banda}& 5+4 & 8+1\\
Gbaya-\il{Gbaya}Manza-Ngbaka\il{Ngbaka} & 5+4 & kùsì\\
Ngbandi\il{Ngbandi}&  & ɡumbaya\\
\textit{Sere-Ngbaka-Mba}\\
~~~~Ngbaka-\il{Ngbaka}Mba\il{Mba} & 5+4 & 10--1, (me-newá)\\
~~~~Sere\il{Sere} & 5+4 & \\
Zande\il{Zande}& 5+4 & \\
\lspbottomrule
\end{tabularx}
\end{table}

Apparently, at the family level the common pattern ‘5+’ should be assumed for the terms from ‘six’ to ‘nine’. Isolated forms attested in groups and sub-groups are quoted here (as well as in the cases of other families) in order to collect exhaustive evidence for further etymological analysis. Moreover, a small chance that the Niger-Congo proto-form is traceable within only a single branch should not be ignored.

\subsubsection{‘Ten’} 
\begin{table}
\caption{\label{tab:3:144}Ubangi stems for `10'}


\begin{tabularx}{\textwidth}{lQQQ}
\lsptoprule

Banda\il{Banda}& bu-fu & *gba & mó-rófō, ` two hands', 'all the fingers'\\
Gbaya-\il{Gbaya}Manza-Ngbaka\il{Ngbaka} & ‘personne’\newline ('joindre les mains') &  & \\
Ngbandi\il{Ngbandi}&  &  & sui, bàlé\\
\textit{Sere-Ngbaka-Mba}\\
~~~~Ngbaka-\il{Ngbaka}Mba\il{Mba} &  & nzò-kp{\textsubtilde{ā}} 'head’-‘hand')/à-nɡbà & a-busa \\
~~~~Sere\il{Sere} &  &  & ɓ{\~{ï}}-k{\"{u}}r{\"{u}},\newline `on hands’\\
Zande\il{Zande}&  & ŋɡb{\={\~{ɔ}}}/bà-wē~ & \\
\lspbottomrule
\end{tabularx}
\end{table}

The reconstruction of the term for ‘ten’ is so problematic that it raises doubts as to whether it was present in Proto-Ubangi\il{Proto-Ubangi} at all. In view of the convincing internal etymology suggested by Ives Moñino, the root *\textit{bu} alternating with *\textit{pu} and *\textit{fu} in some of the NC families is an unlikely candidate. The reconstruction of \textit{*gba/} \textit{kpa} is worth considering. However, the root may not be primary.

\newpage 
\subsubsection{‘Twenty’} %3.7.5.10.
\begin{table}
\caption{\label{tab:3:145}Ubangi stems and patterns for `20'}


\begin{tabularx}{\textwidth}{l@{}l@{}l}
\lsptoprule

Banda\il{Banda}& ‘one person', ‘the whole person', `body-person-all' & \\
Gbaya-\il{Gbaya}Manza-Ngbaka\il{Ngbaka} &  & 10*2\\
Ngbandi\il{Ngbandi}&  & 10*2\\
\textit{Sere-Ngbaka-Mba}\\
~~~~Ngbaka-\il{Ngbaka}Mba\il{Mba} &  & 10*2\\
~~~~Sere\il{Sere} & ‘kill-person-one' & \\
Zande\il{Zande}& `people one' & \\
\lspbottomrule
\end{tabularx}
\end{table}

Two reconstruction possibilities are available here, i.e. the pattern ‘20=10*2’ commonly attested in NC, and a derivation from the lexical term meaning ‘person’.

 
\subsubsection{‘Hundred’}%3.7.5.11.
\begin{table}
\caption{\label{tab:3:146}Ubangi stems and patterns for `100'}


\begin{tabularx}{\textwidth}{l@{}l Q}
\lsptoprule

Banda\il{Banda}& nɡàmb{\`{ɔ}}/nɡbànɡbò & `five persons'< Sango\il{Sango}, < Bangala\il{Bangala} (< Lingala?\il{Lingala})\\
Gbaya-\il{Gbaya}Manza-Ngbaka\il{Ngbaka} &  & `cut/gathered'-‘things'? `clap hands'?, < Lingala\il{Lingala}\\
Ngbandi\il{Ngbandi}& nɡbanɡbo & \\
\textit{Sere-Ngbaka-Mba}\\
~~~~Ngbaka-\il{Ngbaka}Mba\il{Mba} &  & < Sango,\il{Sango} < Lingala,\il{Lingala} 20*5, (mya, k{\'{ʉ}}l{\'{ɔ}}, kpode, nɡūndānɡū)\\
~~~~Sere\il{Sere} &  & `kill-persons-five', < Arabic\il{Arabic}\\
Zande\il{Zande}& nɡbànɡbù < Sango\il{Sango} & `nd{\={ɔ}}ŋɡ͡b{\'{ʉ}}\\
\lspbottomrule
\end{tabularx}
\end{table}

Most of the forms are apparent borrowings which suggests that the term for ‘hundred’ was absent in Proto-Ubangi\il{Proto-Ubangi}.

\subsubsection{‘Thousand’}%3.7.5.12.

The absence of the term for ‘thousand’ in Proto-Ubangi\il{Proto-Ubangi} is even more evident than the absence of the term for ‘hundred.’
\begin{table}
\caption{\label{tab:3:147}Ubangi stems and patterns for `1000'}


\begin{tabularx}{\textwidth}{lQl}
\lsptoprule

Banda\il{Banda}& < French, \il{French}< Lingala?\il{Lingala} & \\
Gbaya-\il{Gbaya}Manza-Ngbaka\il{Ngbaka} & < French, \il{French}< Lingala, \il{Lingala}t{\'{ɔ}}ma{\`{y}} & \\
Ngbandi\il{Ngbandi}& < Lingala, \il{Lingala} < Arabic\il{Arabic} & \\
\textit{Sere-Ngbaka-Mba}\\
~~~~Ngbaka-\il{Ngbaka}Mba\il{Mba} & < Lingala, \il{Lingala} < Arabic, \il{Arabic} < French, \il{French} 100*10 & gyu\\
~~~~Sere\il{Sere} & 1000*10 & \\
Zande\il{Zande}& < Sango\il{Sango} < French\il{French} & \\
\lspbottomrule
\end{tabularx}
\end{table}


\section{Dogon and Bangime\il{Bangime}}%3.8

A step-by-step reconstruction of Dogon numerals does not seem reasonable because the family is relatively homogeneous. In addition, the formal differences between the numerical terms do not seem to correlate with the internal genealogical classification of the Dogon languages. The table below offers an overview of the pertinent data (\tabref{tab:3:148}) and is followed by a brief commentary. 

\begin{table}
\caption{\label{tab:3:148}Dogon numerals}


\begin{tabularx}{\textwidth}{lXrQ}
\lsptoprule

{1} & túrú/tumɔ, ti(i) & {7} & suli/soli/soye\\
{2} & l{\'{ɛ}}(y)/l{\'{ɔ}}(y)/n{\'{ɛ}}(y)/n{\'{ɔ}}(y) & {8} & gá(a)rà, sagi, sele (< Mande?)\\
{3} & taan & {9} & túw{\'{ɔ}}\\
{4} & nay(n), kɛɛso & {10} & p{\'{ɛ}}rú/p{\'{ɛ}}lú\\
{5} & nún{\'{ɛ}}{\'{ɛ}}(n)/n{\v{u}}ː(yn)/n{\^{u}}m & {20} & 10*2\\
{6} & kuro/kule & {100} & 80 (síìŋ/súŋ) +20, < Fula\il{Fula}\\
&  & {1000} & 800 (múɲú) +200\\
\lspbottomrule
\end{tabularx}
\end{table}

\paragraph*{‘One’:} Najamba\il{Najamba}-Kindige: \textit{kúndé} ‘1’, Mombo\il{Mombo} \textit{y{\`{ɛ}}ːtáːŋɡù} ‘1’.

\paragraph*{‘Two’:} The forms with the nasal \textbf{n}- attested in several dialects are variants of the basic form with \textbf{*l}-. It should be noted that the final palatal element is systematically attested in other numerical terms, e.g. in Ben Tey\il{Ben Tey} (\tabref{tab:3:149}).

\begin{table}
\caption{\label{tab:3:149}Final palatal in `2'}


\begin{tabularx}{\textwidth}{lXlQ}
\lsptoprule

2 & y{\v{e}}\textbf{y} & 6 & kúrò\textbf{y}\\
4 & n{\v{i}}ː\textbf{yⁿ} & 7 & súyⁿ{\`{ɔ}}\textbf{yⁿ}\\
5 & nùm{\v{u}}\textbf{yⁿ} & 8 & ɡáːrà\textbf{y}\\
\lspbottomrule
\end{tabularx}
\end{table}

Regardless of whether this element is a morpheme or not, we are certainly dealing with a phonetic alignment by the final segment. Thus the final -\textbf{y} should not be reconstructed even in those forms that show its presence in the majority of languages. 

\paragraph*{‘Three’:} This is a persistent form with only minor modifications applied to it (e.g. \textit{taandu}, \textit{taali}). 

\paragraph*{‘Four’:} This is the only term for which the final palatal (probably nasalized) is potentially reconstructable. If so, systematic alignments by analogy attested in final segments of other numerals are probably based on the form of ‘four’. The root \textit{kɛɛso/} \textit{k{\'{ɛ}}ːj{\'{ɔ}}/} \textit{k{\'{ɛ}}:j{\`{ɛ}}y/} \textit{c{\'{ɛ}}z{\`{ɔ}}/} \textit{yè-c{\'{ɛ}}z{\'{ɔ}}} is probably an innovation (see, however, Jeff Heath who argues for its archaic nature.\footnote{\url{http://dogonlanguages.org}}) The term may be etymologically connected to the term for ‘eighty’, cf. Najamba\il{Najamba}-Kindige \textit{s{\^{i}}m}, \textit{k{\`{ɛ}}ːs{\v{u}}m}, Tommo So\il{Tommo So} \textit{k{\`{ɛ}}{\`{ɛ}}súm} and a number of other related forms (Yorno So\il{Yorno So} \textit{d{\`{ɔ}}g{\`{ɔ}}-s{\v{u}}m}’80’, “Dogon hundred”, Valentin Vydrin, p.c., Perge Tegu\il{Perge Tegu} \textit{d{\`{ɔ}}g{\`{ɔ}}-s{\v{u}}ŋ} ‘80’, Yanda Dom\il{Yanda Dom} \textit{sìŋ} ‘80’ etc.).

\paragraph*{‘Five’:} The etymological connection of this term with the lexical root meaning ‘hand’ \textit{nùmà/} \textit{nùmó/} \textit{nùm{\'{ɔ}}/} \textit{n{\v{o}}{\~{y}}} is immediately apparent.

\paragraph*{‘Six’ and ‘seven’:} These are probably primary terms. 

\paragraph*{‘Eight’:} The root \textit{sagi} attested in Najamba\il{Najamba} and Yanda Dom\il{Yanda Dom} was probably borrowed from Mande. The forms \textit{sila,} \textit{seele} observable in a number of dialects may be related to it. The root  \textit{gá(a)rà} is commonly attested in the majority of languages of this group, sometimes with a partial reduplication (Donno So\il{Donno So}/Yorno So\il{Yorno So}/Toro So\il{Toro So} \textit{ga-gara/ga-gira}). Partial reduplication is a popular means of deriving ‘eight’ from ‘four’ commonly attested throughout NC. In view of the fact that the Dogon counting system is based on 8, this root should probably be compared to \textit{gàrá,} meaning ‘big, large, a large quantity, a lot, go beyond (limit), more, to a greater extent’. Tonal differences may be neglected in this case, especially since the derived forms tend to be formally marked, e.g. tonally.

\paragraph*{‘Hundred’:} The basic ‘large number’ in Dogon is ‘eighty’ rather than ‘hundred’, so this meaning should probably be reconstructed for \textit{siiŋ/suŋ}. In view of this, the fact that the term for ‘hundred’ was borrowed from Fula\il{Fula} in nearly all Dogon languages is not a coincidence.

\paragraph*{‘Thousand’:} Similarly, the root \textit{muɲu} (var. \textit{mùsú} \textit{/} \textit{mùdʒú}) ‘800’ incorporated into the pattern ‘1000=800+200’ is reconstructed in Dogon. 

The Bangime\il{Bangime} numeral system should also be considered here, since most of the numerical terms attested in this isolated language are comparable to those found in Dogon (\tabref{tab:3:150}).

\begin{table}
\caption{\label{tab:3:150}Bangime\il{Bangime} numerals}


\begin{tabularx}{\textwidth}{lXrX}
\lsptoprule

{1} & tòré/t{\v{i}}y{\'{ɛ}} (in counting) & {7} & k{\v{i}}jé\\
{2} & jíndò & {8} & sàáɡín (< Mande?)\\
{3} & táárù & {9} & t{\'{ɛ}}ɡò\\
{4} & nìj{\'{ɛ}} & {10} & kúr{\'{ɛ}}\\
{5} & n{\v{u}}ndí & {20} & tàá{\~{w}}á\\
{6} & k{\v{e}}ré & {100} & t{\`{ɛ}}{\`{ɛ}}m{\`{ɛ}}d{\'{ɛ}}r{\'{ɛ}} (< Fula\il{Fula} )\\
&  & {1000} & m{\v{u}}ʒú\\
\lspbottomrule
\end{tabularx}
\end{table}

As in Dogon, the terms covering the sequence from ‘six’ to ‘nine’ are primary. An isolated root for ‘forty’ (also represented in some of the Dogon languages) is attested in Bangime\il{Bangime}. Interestingly, the root is the same as the one found in some of the Mande languages, cf. Bangime \textit{d{\`{ɛ}}ʋ{\'{ɛ}}}, \textstyleStrong{\textmd{Dogulu Dom}}\il{Dogulu Dom}\textstyleStrong{\textmd{ (Dogon)} }\textit{d{\`{ɛ}}{\'{ɛ}}}, Mombo\il{Mombo} (Dogon) \textit{d{\^{ɛ}}ː}, Marka Dafing\il{Marka Dafing} \textit{dɛbɛ}, Bozo\il{Bozo} \textit{d{\`{ɛ}}b{\'{ɛ}}/} \textit{l{\'{ɛ}}w{\`{ɛ}}}, \textstyleStrong{\textmd{Bamana}}\il{Bamana}\textstyleStrong{\textmd{} }\textstyleStrong{\textmd{\textit{d{\`{ɛ}}b{\'{ɛ}}}}}\textstyleStrong{\textmd{.}}

The root for ‘ten’ does not correspond to the one attested in Dogon. The latter finds a direct parallel in Boko\il{Boko} (East Mande \textit{kuri} ‘ten’.

\section{Gur}%3.9

It should be noted that the Gur languages are extremely divergent in the majority of their numerical terms (including those that prove to be fairly persistent in other families). The approach we took for the evidence studied above (i.e. the establishing of the most common forms and their further comparison to the data from other branches) may not appear fruitful in the case of the Gur languages.

To deal with the problem, we are going to use the classification of the Gur languages found in \textit{Ethnologue}, namely A. Bariba\il{Bariba}, B. Central, C. Kulango\il{Kulango}, D. Lobi\il{Lobi}, E. Senufo, F. Teen\il{Teen}, G. Tiefo\il{Tiefo}, H. Tusia\il{Tusia}, I. Viemo\il{Viemo}, J. Wara\il{Wara}-Natioro\il{Natioro}.\footnote{This classification is accepted here with slight modifications based on recent studies. For instance, Dyan\il{Dyan} and Lobi\il{Lobi}  are treated as members of the same branch.} The Gur family comprises nearly a hundred languages. In terms of the classification outlined above, their distribution is uneven. Seven groups (Bariba, Kulango, Lobi, Teen, Tiefo, Tusia, Viemo) have an isolated language as their only member. Similarly, Wara-Natioro is represented by only three idioms. This means that the majority of the Gur languages are split between the two remaining groups, i.e. Senufo and Central. The former is comprised of about fifteen languages and is relatively homogenous. Its affiliation to Gur is often considered doubtful. Compared to Central, which embraces the majority of the Gur languages (nearly seventy), this group is relatively small. Two major sub-groups are identifiable within Central, i.e. Northern (38 languages) with Oti-Volta (33 languages) as the dominant branch and Southern (31 languages) with its dominant branch of Grusi (23 languages). In other words, 71 of the Gur languages (out of a total of 91) belong to either Oti-Volta, Grusi or Senufo. In addition to that, there are more than ten branches represented by a single isolated language each. No evidence points to their possible affiliation with the major branches or to their inter-relationship. The same can probably be said about several isolated languages affiliated (often uncritically) with the Central group (the Bwamu\il{Bwamu}, Kurumfe\il{Kurumfe}, Dogoso\il{Dogoso}-Khe\il{Khe}, Gan-Dogosé, and Kirma\il{Kirma}-Tyurama\il{Tyurama} branches). This already complex picture gets even more sophisticated in view of the following: 

\begin{enumerate}
 \item Branches represented by one or two languages (e.g. Buli\il{Buli}-Konni\il{Konni}, Notre\il{Notre}, Yom\il{Yom}-Nawdm\il{Nawdm}) are distinguishable even within the most reliably established bodies of genetically related languages of this family.
 \item  According to Ulrich Kleinewillinghöfer (p.c.), who is a renowned expert in both Gur and Adamawa comparative linguistics, a border between these two families is not clear at all. This means that some of the Gur branches may prove to be more closely related to Adamawa. 
\end{enumerate}
 

Our reconstruction of the Gur numeral system is based on nearly 120 sources that vary in regards to the evidence they offer (cf. our considerations above). By addressing one of the most problematic cases (i.e. the reconstruction of the Gur term for ‘one’) we hope to work out a general approach that will eventually allow further comparison of the Gur evidence to that of other NC families.

  
\subsection{‘One’}

The table below lists several forms of the term for ‘one’ in smaller Gur branches (\tabref{tab:3:151}).

\begin{table}
\caption{\label{tab:3:151}Diversity of stems for `1' in Gur}
 
\begin{tabularx}{\textwidth}{lQl}
\lsptoprule
Gurma\il{Gurma} & Grusi-Eastern & Grusi-Western\\
\midrule
Akaselem: \il{Akaselem} {\`{m}}-bá & Bago-Kusuntu: \il{Bago-Kusuntu} ŋʊrʊkpákpá & Chakali: \il{Chakali} d{\'{ɪ}}ɡ{\'{ɪ}}máná~\\
Bimoba: \il{Bimoba}yènn & Chala: \il{Chala} -re-, -d{\'{ʊ}}ndʊlʊŋ & Deg: \il{Deg} beŋ-kpaŋ/kpee\\
Miyobe: \il{Miyobe} n-ni (-sɛ) & Delo: \il{Delo} daale & Phuie: \il{Phuie} déò/dùdúmí\\
Nateni: \il{Nateni} -c{\={\~{ɔ}}}, dèn & Kabiye: \il{Kabiye} k{\'{ʊ}}-y{\'{ʊ}}m & Sisaala: \il{Sisaala} k{\`{ʊ}}-bàlá/d{\`{ɪ}}áŋ\\
Ngangam: \il{Ngangam} mi-kpìɛkm & Lama: \il{Lama} kó-ɖ{\'{ə}}m & Winyé: \il{Winyé} n-do\\
\lspbottomrule
\end{tabularx}
\end{table}

A brief study of these examples raises doubts as to whether the Gur numeral system is reconstructable at all (not to mention the Grusi-Northern system or those of the more isolated Gur branches). 

Even if we consider one syllable roots of the CV(C)-type only, the impression will remain that every concievable root for ‘one’ is attested in the Gur languages. However, none of these roots is traceable in at least half of the Gur groups. This situation is reflected in the matrix below (\tabref{tab:3:152}).

\begin{table}
\caption{\label{tab:3:152}Distribution of the CV(C)- forms for `1' in the Gur languages}
\small

\begin{tabularx}{\textwidth}{lCCC}
\lsptoprule

~ & { \textbf{I}} & { \textbf{A}} & { \textbf{U}}\\
\midrule
{\textbf{P (p/f)}} & {–} & {–} & –\\
{\textbf{B (b/w/m)}} & {3/5} & {1/4} & 1/1?\\
{\textbf{T (t)}} & {1/1} & {2/2} & –\\
{\textbf{D (d/l/r/n)}} & {3/16} & {–} & 3/13\\
{\textbf{C (c/s)}} & {–} & {–} & 1/1\\
{\textbf{J (j/y/ny)}} & {1/18} & {1/1} & 1/1\\
{\textbf{K (k/h/x)}} & {2/5} & {1/2} & 2/4\\
{\textbf{G (g/ŋ)}} & {1/5} & {1/1} & 1/1\\
\lspbottomrule
\end{tabularx}
\end{table}

The first figure refers to the number of groups where a form is attested (with a maximum of 10 groups), whereas the second one refers to the number of  languages. Thus, \textbf{B-I} denotes a form comprising a voiced labial consonant (b, w or m) and a front vowel that is attested in five languages within three groups (Central, Lobi\il{Lobi}-Dyan\il{Dyan} and Senufo) (\tabref{tab:3:153}).

\begin{table}
\caption{\label{tab:3:153}BI- forms for `1' in Gur (3 groups, 5 languages)}
\small

\begin{tabularx}{\textwidth}{lXllll}
\lsptoprule

\textbf{béé} & Ditammari\il{Ditammari} & B. Central & 1. Northern & C. Oti-Volta & ii. Eastern\\
\textbf{bì{\`{ɛ}}-} & Lobi\il{Lobi} & D. Lobi-\il{Lobi}Dyan\il{Dyan} &  &  & \\
\textbf{b{\`{\~ɛ}}ɡ} & Dyan\il{Dyan} & D. Lobi-\il{Lobi}Dyan\il{Dyan} &  &  & \\
\textbf{nì-bín} & Cebaara\il{Cebaara} & E. Senufo &  &  & \\
\textbf{nan-bin} & Shempire\il{Shempire} & E. Senufo &  &  & \\
\lspbottomrule
\end{tabularx}
\end{table}

 
The remaining forms are quoted  below as an illustration of their extreme divergency.

\newpage 
\ea
\ea \textbf{BA} (1/4) (\tabref{tab:3:154}).

\begin{table}
\caption{\label{tab:3:154}BA- forms for `1' in Gur (1 group, 4 languages)}
\small
\begin{tabularx}{.9\textwidth}{Xlllll}
\lsptoprule
\textbf{{\`{M}}-bá}  & Akaselem\il{Akaselem} & B. Central & 1. Northern & C. Oti-Volta & Gurma\il{Gurma}\\
\textbf{bàa} & Konkomba\il{Konkomba} & B. Central & 1. Northern & C. Oti-Volta & Gurma\il{Gurma}\\
\textbf{mi-ba} & Ngangam\il{Ngangam} & B. Central & 1. Northern & C. Oti-Volta & Gurma\il{Gurma}\\
\textbf{{\`{n}}.-bá/-b{\'{ɔ}}} & Ntcham\il{Ntcham} & B. Central & 1. Northern & C. Oti-Volta & Gurma\il{Gurma}\\
\lspbottomrule
\end{tabularx}
\end{table}

\ex \textbf{BU~}(1/1): only \textit{pú-wò} (possibly \textit{púw-ò}, \textbf{PU?}) in Wara\il{Wara} (J.Wara-Natioro\il{Natioro})

  
\ex \textbf{TI~}(1/1): only \textit{tía} in Baatonum\il{Baatonum} (A.Bariba\il{Bariba})

\largerpage
\ex \textbf{TA} (2/2) (\tabref{tab:3:155}).

\begin{table}
\caption{\label{tab:3:155}TA- forms for `1' in Gur}
\small
\begin{tabularx}{.9\textwidth}{XXl}
\lsptoprule
\textbf{ta, taà, t{\~{a}}{\`ã}} & Kulango\il{Kulango} (dial.) & C.Kulango\il{Kulango}\\
\textbf{tani} & Teen\il{Teen} (dial.) & F.Teen\il{Teen}\\
\lspbottomrule
\end{tabularx}
\end{table}
\clearpage 
  
\ex \textbf{DI} (3/15) (\tabref{tab:3:156}).

\begin{table}
\caption{\label{tab:3:156}DI- forms for `1' in Gur}
\small
\fittable{
\begin{tabular}{llllll}
\lsptoprule
\textbf{dè} & Bwamu\il{Bwamu} (Boore) & B. Central & 1. Northern & A. Bwamu\il{Bwamu} & \\
\textbf{nni} & Miyobe\il{Miyobe} & B. Central & 1. Northern & C. Oti-Volta & iii. Gurma\il{Gurma}\\
\textbf{dèn} & Nateni\il{Nateni} & B. Central & 1. Northern & C. Oti-Volta & iii. Gurma\il{Gurma}\\
\textbf{lé} & Khe\il{Khe} Southern & B. Central & 2. Southern & A. Dogoso-\il{Dogoso}Khe\il{Khe} & \\
\textbf{í-lèŋ} & Khisa\il{Khisa} & B. Central & 2. Southern & C. Gan-Dogose\il{Dogose} & \\
\textbf{re-} & Chala\il{Chala} & B. Central & 2. Southern & D. Grusi & i. Eastern\\
\textbf{dííŋ} & Paasaal\il{Paasaal} & B. Central & 2. Southern & D. Grusi & iii. Western\\
\textbf{déò} & Phuie\il{Phuie} & B. Central & 2. Southern & D. Grusi & iii. Western\\
\textbf{d{\`{ɪ}}áŋ} & Sisaala\il{Sisaala} (dial.) & B. Central & 2. Southern & D. Grusi & iii. Western\\
\textbf{d{\`{ɪ}}{\'{ɛ}}n} & Sisaala\il{Sisaala} (dial.) & B. Central & 2. Southern & D. Grusi & iii. Western\\
\textbf{diiɡɛ} & Tampulma\il{Tampulma} & B. Central & 2. Southern & D. Grusi & iii. Western\\
\textbf{déiŋ} & Kirma\il{Kirma} & B. Central & 2. Southern & E. Kirma-\il{Kirma}Tyurama\il{Tyurama} & \\
\textbf{d{\~{e}}{\~{e}}n-} & Turka\il{Turka} & B. Central & 2. Southern & E. Kirma-\il{Kirma}Tyurama\il{Tyurama} & \\
\textbf{n{\`{ɔ}}-ni} & Karaboro\il{Karaboro} (dial.) & E. Senufo &  &  & \\
\textbf{d{\`{\~ɛ}}}  & Tiefo\il{Tiefo} (dial.) & G. Tiefo\il{Tiefo} &  &  & \\
\lspbottomrule
\end{tabular}
}
\end{table}

  
\ex \textbf{DU} (3/13) (\tabref{tab:3:157})

\begin{table}
\caption{\label{tab:3:157}DU- forms for `1' in Gur}
\small
\begin{tabularx}{\textwidth}{lllll@{~}l}
\lsptoprule
\textbf{dò{\`ũ}} & Bwamu\il{Bwamu} & B. Central & 1. Northern & A. Bwamu\il{Bwamu} & \\
\textbf{dòòn} & Bwamu\il{Bwamu} & B. Central & 1. Northern & A. Bwamu\il{Bwamu} & \\
\textbf{dò} & Láá Láá\il{Láá Láá} & B. Central & 1. Northern & A. Bwamu\il{Bwamu} & \\
\textbf{rʊ} & Chala\il{Chala} & B. Central & 2. Southern & D. Grusi & i. Eastern\\
\textbf{kà-l{\`{ʊ}}} & Kasem\il{Kasem} (dial.)1 & B. Central & 2. Southern & D. Grusi & ii. Northern\\
\textbf{kà-lʊ} & Kasem\il{Kasem} (dial.)2 & B. Central & 2. Southern & D. Grusi & ii. Northern\\
\textbf{è-dù} & Lyele\il{Lyele} & B. Central & 2. Southern & D. Grusi & ii. Northern\\
\textbf{ù-dù} & Northern Nuni\il{Nuni} & B. Central & 2. Southern & D. Grusi & ii. Northern\\
\textbf{n{\`{ə}}-d{\`{ʊ}}} & Southern Nuni\il{Nuni} & B. Central & 2. Southern & D. Grusi & ii. Northern\\
\textbf{n-do} & Winyé\il{Winyé} & B. Central & 2. Southern & D. Grusi & iii. Western\\
\textbf{nú-nu} & Nafaanra\il{Nafaanra} & E. Senufo &  &  & \\
\textbf{d{\~{u}}de}  & Viemo\il{Viemo} & I.Viemo\il{Viemo} &  &  & \\
\lspbottomrule
\end{tabularx}
\end{table}

\ex \textbf{CU} (1/2): only \textit{mà-c{\'{\~ɔ}}} in Nateni\il{Nateni} (Central: 1. Northern: C.Oti-Volta: iii. Gurma\il{Gurma}
 
 \newpage 
\ex \textbf{JI} (1/19) (\tabref{tab:3:158})
\begin{table}
\caption{\label{tab:3:158}CI- forms for `1' in Gur}
\small
\begin{tabularx}{\textwidth}{l@{~}Q@{~~}l@{~~}l@{~~}l@{~~}l@{}}
\lsptoprule
\textbf{yéŋ/ wà-ɲī}  & Buli\il{Buli} & B. Central & 1. Northern & C. Oti-Volta & i. Buli-\il{Buli}Koma\\
\textbf{y{\~{ɛ}}n} & Mbelime\il{Mbelime} & B. Central & 1. Northern & C. Oti-Volta & ii. Eastern\\
\textbf{yènn} & Bimoba\il{Bimoba} & B. Central & 1. Northern & C. Oti-Volta & iii. Gurma\il{Gurma}\\
\textbf{yèn-} & Gurma\il{Gurma} & B. Central & 1. Northern & C. Oti-Volta & iii. Gurma\il{Gurma}\\
\textbf{jèn{\`{n}}}  & Moba\il{Moba} & B. Central & 1. Northern & C. Oti-Volta & iii. Gurma\il{Gurma}\\
\textbf{bõ-ƴén}  & Birifor\il{Birifor} (dial.) & B. Central & 1. Northern & C. Oti-Volta & iv. Western\\
\textbf{bo-yæn} & Birifor\il{Birifor} (dial.) & B. Central & 1. Northern & C. Oti-Volta & iv. Western\\
\textbf{bõ-yen} & Dagaara\il{Dagaara} (dial.) & B. Central & 1. Northern & C. Oti-Volta & iv. Western\\
 \textbf{yén-} & Dagaara\il{Dagaara} (dial.) & B. Central & 1. Northern & C. Oti-Volta & iv. Western\\
\textbf{yén} & Farefare\il{Farefare} & B. Central & 1. Northern & C. Oti-Volta & iv. Western\\
\textbf{yé} & Moore\il{Moore} & B. Central & 1. Northern & C. Oti-Volta & iv. Western\\
\textbf{b{\'{ʊ}}-ŋj{\`{ɪ}}ŋ}  & Wali\il{Wali} & B. Central & 1. Northern & C. Oti-Volta & iv. Western\\
\textbf{yín} & Dagbani\il{Dagbani} (Dagomba) & B. Central & 1. Northern & C. Oti-Volta & iv. Western\\
\textbf{yɪn-} & Hanga\il{Hanga} & B. Central & 1. Northern & C. Oti-Volta & iv. Western\\
\textbf{yín} & Kamara\il{Kamara} & B. Central & 1. Northern & C. Oti-Volta & iv. Western\\
\textbf{yén-} & Kantosi\il{Kantosi} & B. Central & 1. Northern & C. Oti-Volta & iv. Western\\
\textbf{y{\'{ɪ}}n} & Mampruli\il{Mampruli} & B. Central & 1. Northern & C. Oti-Volta & iv. Western\\
\textbf{ny{\v{ə}}ŋ} & Yom\il{Yom} (Pila) & B. Central & 1. Northern & C. Oti-Volta & v. Yom-\il{Yom}Nawdm\il{Nawdm}\\
\lspbottomrule
\end{tabularx}
\end{table}
  
\ex \textbf{JA} (1/1) – only \textit{à-yàʔ} in Safaliba\il{Safaliba} (B. Central: 1. Northern: C.Oti-Volta: iv. Western)

\ex \textbf{JU} (1/1) – only \textit{yòn} in Waama\il{Waama} (B. Central: 1. Northern: C.Oti-Volta: ii. Eastern)

  
\ex \textbf{KI} (2/5) (\tabref{tab:3:159})
\begin{table}
\small
\caption{\label{tab:3:159}KI- forms for `1' in Gur}
\begin{tabularx}{\textwidth}{lQllll}
\lsptoprule
\textbf{{\textsubdot{\`{m}}}-hén} & Nawdm\il{Nawdm} & B. Central & 1. Northern & C. Oti-Volta & v. Yom-\il{Yom}Nawdm\il{Nawdm}\\
\textbf{kpee} & Deg\il{Deg} & B. Central & 2. Southern & D. Grusi & iii. Western\\
\textbf{kpéé} & Vagla\il{Vagla} & B. Central & 2. Southern & D. Grusi & iii. Western\\
\textbf{nì-k{\`ĩ}} & Sìcìté\il{Sìcìté} Senufo & E. Senufo &  &  & \\
\textbf{nìŋ-kìn} & Supyire\il{Supyire} Senufo & E. Senufo &  &  & \\
\lspbottomrule
\end{tabularx}
\end{table}

\newpage 
\ex \textbf{KA} (1/2) (\tabref{tab:3:160})

\begin{table}
\caption{\label{tab:3:160}KA- forms for `1' in Gur}
\small
\begin{tabularx}{\textwidth}{lXlXlX}
\lsptoprule
\textbf{beŋ-kpaŋ} & Deg\il{Deg} & B. Central & 2. Southern & D. Grusi & iii. Western\\
\textbf{kpáŋ} & Vagla\il{Vagla} & B. Central & 2. Southern & D. Grusi & iii. Western\\
\lspbottomrule
\end{tabularx}
\end{table}

\ex \textbf{KU} (2/3) (\tabref{tab:3:161})

\begin{table}
\caption{\label{tab:3:161}KU- forms for `1' in Gur}
\begin{tabularx}{\textwidth}{lXXXl}
\lsptoprule
\textbf{kpò} & Khe\il{Khe} (dial.) & B. Central & 2. Southern & A. Dogoso-\il{Dogoso}Khe\il{Khe}\\
\textbf{tì-kpóʔ} & Dogose\il{Dogose} & B. Central & 2. Southern & C. Gan-Dogose\il{Dogose}\\
\textbf{tʰi-{\textsubdot{k}}po} & Kaansá\il{Kaansá} & B. Central & 2. Southern & C. Gan-Dogose\il{Dogose}\\
\textbf{nú-kú} & Toussian\il{Toussian} (dial.) & H. Tusia\il{Tusia} &  & \\
\lspbottomrule
\end{tabularx}
\end{table}
 
\ex \textbf{GI} (1/5) (\tabref{tab:3:162})

\begin{table}[h!]
\caption{\label{tab:3:162}GI- forms for `1' in Gur}
\small
\begin{tabularx}{\textwidth}{XXl}
\lsptoprule
\textbf{niŋ-ɡbe} & Palaka\il{Palaka} Senufo & E. Senufo\\
\textbf{nī-ɡbe} & Nyarafolo\il{Nyarafolo} Senufo & E. Senufo\\
\textbf{ni-ɡ{\`ĩ}}/\textbf{ni-ɡ{\~{i}}} & Mamara\il{Mamara} Senufo (Minyanka) & E. Senufo\\
\textbf{nin-ɡin} & Shempire\il{Shempire} Senufo & E. Senufo\\
\textbf{nu-ɡbe} & Tagwana\il{Tagwana} Senufo & E. Senufo\\
\lspbottomrule
\end{tabularx}
\end{table}
  

\ex \textbf{GA} (1/1) – only \textit{nuŋ-ɡ}\textit{ba} in Djimini\il{Djimini} Senufo (E. Senufo).

\ex \textbf{GU} (1/1) – only \textit{gbú} in Northern Khe\il{Khe} (B. Central: 2. Southern: A. Dogoso\il{Dogoso}-Khe).
\z
\z
  

The only lacuna in this presentation is due to the lack of forms with voiceless labial consonants (this, however, may not prove true in the case of Wara\il{Wara}-Natioro\il{Natioro}, as we hope to demonstrate below). It should be noted that the general distribution pattern is that a single form is attested in one branch out of ten, three forms are found in both two and three branches, and none of the forms is recorded in four or more branches. This makes an attempt at tracing them down to a source form (with its further comparison to the evidence of the other families) unreasonable. In view of the genetic classification of the Gur languages and the considerations presented above, the optimum solution to the problem probably lies within separate reconstructions of numerals in the following sixteen Gur branches that belong to ten major language groups of this family, assuming that each of them may shed some new light on the reconstruction of the Niger-Congo numeral system:

\begin{itemize} 
\item[1.]Bariba\il{Bariba}
\item[2.]Central: 1. Northern: A. Bwamu\il{Bwamu}
\item[2.]Central: 1. Northern: B. Kurumfe\il{Kurumfe}
\item[2.]Central: 1. Northern: C. Oti-Volta
\item[2.]Central: 2. Southern: A. Dogoso\il{Dogoso}-Khe\il{Khe}
\item[2.]Central: 2. Southern: C. Gan-Dogose\il{Dogose}
\item[2.]Central: 2. Southern: D. Grusi
\item[2.]Central: 2. Southern: E. Kirma\il{Kirma}-Tyurama\il{Tyurama}
\item[3.]Kulango\il{Kulango}
\item[4.]Lobi\il{Lobi}-Dyan\il{Dyan}
\item[5.]Senufo
\item[6.]Teen\il{Teen}
\item[7.]Tiefo\il{Tiefo}
\item[8.]Tusia\il{Tusia}
\item[9.]Viemo\il{Viemo}
\item[10.] Wara\il{Wara}-Natioro\il{Natioro}.
\end{itemize}

Numerical terms as attested in each of these branches will be examined below.

 
\subsection{Bariba}%3.9.1.
\il{Bariba}
\begin{table}
\caption{\label{tab:3:163}Bariba\il{Bariba} numerals}
\begin{tabularx}{\textwidth}{lXrl}
\lsptoprule
{1} & tiā & {7} & 5+2\\
{2} & ru & {8} & 5+3\\
{3} & i-ta & {9} & 5+4\\
{4} & {\`{n}}-nɛ & {10} & wɔ-kuru\\
{5} & n{\`{ɔ}}ɔbù & {20} & yɛndu\\
{6} & 5+1 & {100} & 20*5\\
&  & {1000} & f{\`{ɔ}}r{\`{ɔ}}tɔ? \\
\lspbottomrule
\end{tabularx}
\end{table}


\subsection{Central Gur}%3.9.2.
\subsubsection{Northern Central Gur}%3.9.2.1.
\subsubsubsection{Bwamu}%3.9.2.1.1.
\il{Bwamu}
\begin{table}
\caption{\label{tab:3:164}Bwamu\il{Bwamu} numerals}


\begin{tabularx}{\textwidth}{lXrl}
\lsptoprule

{1} & do & {7} & 5+2\\
{2} & ɲū & {8} & 5+3\\
{3} & t{\~{i}} & {9} & d{\`ĩ}iní/dènú\\
{4} & náa & {10} & pílú/píru/ˀɓúrúù\\
{5} & hò-nú & {20} & ɓóní/ɓénle/kēwēníì\\
{6} & 5+1 & {100} & kʰ{\~{i}}minù (< Mande keme )\\
&  & {1000} & 100*10, muaseé\\
\lspbottomrule
\end{tabularx}
\end{table}

 
\subsubsubsection{Kurumfe}%3.9.2.1.2.
\il{Kurumfe}
\begin{table}
\caption{\label{tab:3:165}Kurumfe\il{Kurumfe} numerals}


\begin{tabularx}{\textwidth}{lXrl}
\lsptoprule

{1} & dom & {7} & p{\~{ɛ}}{\~{ɛ}}\\
{2} & h{\~{i}}{\~{i}} & {8} & tɔɔ\\
{3} & t{\~{a}}{\~{a}} & {9} & fa\\
{4} & n{\~{a}}{\~{a}} & {10} & fɪ\\
{5} & nɔm & {20} & sofe (<10? )\\
{6} & hʊrʊ & {100} & bɛrʊ\\
&  & {1000} & tʊsrɪ < from Moore\\
\lspbottomrule
\end{tabularx}
\end{table}

 
\subsubsubsection{Oti-Volta~}%3.9.2.1.3.
\subparagraph{i. Buli\il{Buli}-Koma (\tabref{tab:3:166})}
~
\begin{table}
\caption{\label{tab:3:166}Buli\il{Buli}-Koma numerals}


\begin{tabularx}{\textwidth}{lXrl}
\lsptoprule

{1} & yéŋ (adj.), ní (count) & {7} & yòp{\={ɔ}}āī, p{\'{\~o}}{\`ĩ}\\
{2} & y{\`{ɛ}}, li & {8} & nāāniŋ/à-níì (<* 4 redupl., 4PL?)\\
{3} & tà & {9} & nèūk/{\`{ŋ}}w{\'{ɛ}}\\
{4} & nààsì/nísà & {10} & pī/bâŋ\\
{5} & nù & {20} & 10*2\\
{6} & yùèbì/óbìŋ & {100} & kòòk, kobɪɡa/bóɾà\\
&  & {1000} & < Engl.\\
\lspbottomrule
\end{tabularx}
\end{table}

  
\subparagraph{ii. Eastern (\tabref{tab:3:167})}
~
\begin{table}
\caption{\label{tab:3:167}Eastern Oti-Volta numerals}


\begin{tabularx}{\textwidth}{lQrQ}
\lsptoprule

{1} & c{\={ə}}r{\={ə}}, béé, dè{\`{n}}nì (counting), y{\~{ɛ}}nde/yòn, *de & {7} & pèléī/bérén, yīēkà/nyiekɛ, dood{\={ɛ}} (6+1)\\
{2} & dyā, d{\'{ɛ}}{\'{ɛ}}, diání/dɛɛni, yēdē/y{\'{ɛ}}ndí & {8} & nēī/n{\`{\~ɛ}}í/ni/niny{\={\~{ɛ}}}\\
{3} & tâati/tâadi/tāārī & {9} & wáī/wɛi/w{\={ɛ}}\\
{4} & naa(si) & {10} & pwíɡ{\={ə}}/pííkà/piíkɛ/piitɛ , *pi\\
{5} & num(mu)/nun & {20} & 10*2\\
{6} & kūà/kuɔ, dūo, h{\`ã}dwàm, kpàrùn & {100} & kòɣ{\={ə}}/kookɛ/k{\'{ɔ}}úkpà/k{\`{ɔ}}{\`{ɔ}}tà\\
&  & {1000} & túsírè\\
\lspbottomrule
\end{tabularx}
\end{table}

Please note the extreme divergency of languages within this branch: the variety of forms presented in the table above are attested in only four languages, i.e. Biali\il{Biali}, Ditammari\il{Ditammari}, Mbelime\il{Mbelime} and Waama\il{Waama}.


\subparagraph{iii. Gurma\il{Gurma} (\tabref{tab:3:168})}
~
\begin{table}
\caption{\label{tab:3:168}Gurma\il{Gurma} numerals}


\begin{tabularx}{\textwidth}{lXrl}
\lsptoprule
{1} & bá, yènn(do), den (isol.: ni, c{\={\~{ɔ}}}) & {7} & lòlé/lèlé (isol.: s{\'{ɛ}}{\'{ɛ}}i, yehì)\\
{2} & le/d{\'{ɛ}}/t{\'{ɛ}} & {8} & ni(n)\\
{3} & tà & {9} & w{\`{ɛ}}ʔ/w{\'{ɛ}}ɛ/w{\'{ɔ}}ì/wáī\\
{4} & nà(hì) & {10} & píík/pʷíʔ/fi/pita\\
{5} & mù/nù{\`{m}}/nu(p{\~{u}})/ŋùn & {20} & 10*2 (isol.: kòó, mù{\`{ŋ}}kú < mande?)\\
{6} & loòb/luu, k{\`{ɔ}}dì/kouul{\'ũ} & {100} & kúb (isol.: pílɛ, k{\`{ɔ}}ta)\\
&  & {1000} & < kùtùkú‘sack', borrowing\\
\lspbottomrule
\end{tabularx}
\end{table}

 
\subparagraph{iv. Western (\tabref{tab:3:169})}
~
\begin{table}
\caption{\label{tab:3:169}Western Oti-Volta numerals}
\begin{tabularx}{\textwidth}{lXrl}
\lsptoprule
{1} & yen/yin, dam?, (dàk{\'{\~o}}ʔ) & {7} & yopoi (< yo-poi?)\\
{2} & yi(ʔ) & {8} & nii(n)\\
{3} & ta & {9} & way/wey\\
{4} & naasi/naar/n{\~{a}}an & {10} & pia/pie\\
{5} & nú & {20} & 10*2\\
{6} & yobu & {100} & kob/kɔɔ\\
&  & {1000} & tur/tudi (borrowed?)\\
\lspbottomrule
\end{tabularx}
\end{table}

\subparagraph{v. Yom\il{Yom}-Nawdm\il{Nawdm} (\tabref{tab:3:170})}
~
\begin{table}
\caption{\label{tab:3:170}Yom\il{Yom}-Nawdm\il{Nawdm} numerals}
\begin{tabularx}{\textwidth}{lXrl}
\lsptoprule
{1} & hén, ny{\v{ə}}ŋ-/ny{\v{ə}}rɣə- & {7} & lèbléʔ (<6?), 5+2\\
{2} & li/ɾéʔ/*rɣa? & {8} & nìːndí; 10--2\\
{3} & ta/tâʔ & {9} & w{\'{ɛ}}ʔ, 10--1\\
{4} & naa/n{\`{ɛ}}{\`{ɛ}}s{\`{ə}} & {10} & ʔɾí?, fɛɣa\\
{5} & nu & {20} & 2PL\\
{6} & {\textsubdot{\`{m}}}ɾòːndí (X+1?), lèèw{\`{ə}}r & {100} & l{\'{ɛ}}mú, wʊr-\\
\lspbottomrule
\end{tabularx}
\end{table}

 
\subparagraph{Proto-Oti-Volta\il{Proto-Oti-Volta}}

The evidence of five Oti-Volta branches (isolated forms excluded) is summarized in \tabref{tab:3:171}.

\begin{table}
\caption{\label{tab:3:171}Numerals in Proto-Oti-Volta\il{Proto-Oti-Volta}}
\small
\begin{tabularx}{\textwidth}{rQQQQQQ}
\lsptoprule
~ & i. Buli-\il{Buli}Koma & ii.~Eastern & iii.~Gurma\il{Gurma} & \mbox{iv.~Western} & v. Yom-\il{Yom}Nawdm\il{Nawdm} & \textbf{*Proto-Oti-Volta}\il{Proto-Oti-Volta}\\
\midrule
1 & yéŋ,  ní & dè{\`{n}}nì, y{\~{ɛ}}nde/yòn,  *de & yènn(do),  den,  ni & yen/yin,  dam & hén,  ny{\v{ə}}ŋ & \textbf{den/yen,  ni,  de?} \\
2 & y{\`{ɛ}},  li & d{\'{ɛ}}{\'{ɛ}}(ni),  yēdē & le/d{\'{ɛ}} & yi(ʔ) & li/ɾéʔ/ *rɣa? & \textbf{li/yi}\\
3 & tà & tâati & tà & ta & ta & \textbf{ta(t)}\\
4 & nààsì & naa(si) & nà(hì) & naasi & naa/n{\`{ɛ}}{\`{ɛ}}s{\`{ə}} & \textbf{naa(si)}\\
5 & nù & nun & nù{\`{m}}/nu/ ŋùn & nú & nu & \textbf{nu}\\
6 & yùèbì/ óbìŋ & dūo & loòb/luu & yobu & lèèw-{\`{ə}}r & \textbf{lob/ yob}\\
7 & yòp{\={ɔ}}āī,  p{\'{\~o}}{\`ĩ} & dood{\={ɛ}} (6+1) & lòlé/lèlé & yopoi & lèbléʔ & \textbf{*lob-le (6+1)? poi(n)?} \\
8 & nāāniŋ/ à-níì & n{\`{\~ɛ}}í/ni/ niny{\={\~{ɛ}}} & ni(n) & nii(n) & nìːndí & \textbf{ni}\\
9 & nèūk/{\`{ŋ}}w{\'{ɛ}} & wáī/wɛi/ w{\={ɛ}} & w{\`{ɛ}}ʔ/w{\'{ɛ}}ɛ/ wáī & way/wey & w{\'{ɛ}}ʔ & \textbf{wey/ weʔ}\\
10 & pī & pwíɡ{\={ə}}/ pííkà/*pi & píík/pʷíʔ/ fi & pia/pie & fɛɣa & \textbf{pi(k)}\\
20 & 10*2 & 10*2 & 10*2 & 10*2 & 2PL & \textbf{10*2}\\
100 & kòòk,  kobɪɡa & kòɣ{\={ə}}/ kookɛ/ k{\'{ɔ}}úkpà & kúb & kob/kɔɔ & l{\'{ɛ}}mú,  wʊr- & \textbf{kob,  kook}\\
\lspbottomrule
\end{tabularx}
\end{table}

  
The reconstruction of the Oti-Volta numeral system is surprisingly unproblematic. In addition to the expectedly persistent reflexes of ‘three’ and ‘four’, homogeneous forms for ‘two’, ‘five’, and ‘ten’ are noteworthy. The term for ‘eight’ seems to be based on ‘four’ (either via the partial reduplication or according to the ‘4PL’ pattern). In addition to that, Oti-Volta is characterized by the presence of the primary (homogeneous) forms of ‘six’, ‘eight’, and ‘nine’. The forms of ‘seven’ are probably derived and follow the pattern ‘6+1’. It appears that the derivative form \textit{*lob-le} > \textit{lole} is already reconstructable at the Proto-Oti-Volta\il{Proto-Oti-Volta} level.

\subsubsection{Southern Central Gur}%3.9.2.2.
\subsubsubsection{Dogoso-Khe}%3.9.2.2.1.
\il{Dogoso}\il{Khe}
\begin{table}
\caption{\label{tab:3:172}Dogoso\il{Dogoso}-Khe\il{Khe} numerals}


\begin{tabularx}{\textwidth}{lXrX}
\lsptoprule

{1} & kpò, lé & {7} & 5+2\\
{2} & jɔ(n) & {8} & 5+3\\
{3} & thɔ & {9} & 5+4\\
{4} & dáa & {10} & kpélé\\
{5} & nɔ(n) & {20} & cúkúrì/g{\`{ʊ}}ʊsì\\
{6} & 5+1 & {100} & 20*5\\
&  & {1000} & kp{\'{ɛ}}\\
\lspbottomrule
\end{tabularx}
\end{table}

The forms pertaining to these languages that are not present in the main databases are quoted according to Kerstin Winkellmann in (\citealt{Winkelmann2007b}: 181--210). Although the numerals attested within the two languages of this group are quite persistent, Kerstin Winkellmann stresses their grammatical difference: \textit{“~…} while Dɔgɔ-sʊ uses noun suffixes, sʊ-Khe\il{Khe} is a prefixing language~” (Winkellmann 2007d: 209).

\subsubsubsection{Gan-Dogose}%3.9.2.2.2.
\il{Dogose}
\begin{table}
\caption{\label{tab:3:173}Gan-Dogose\il{Dogose} numerals}
\begin{tabularx}{\textwidth}{lXrl}
\lsptoprule
{1} & kpo/po, (lèŋ) & {7} & 5+2\\
{2} & y{\textsubtilde{\'{ɔ}}}/ɲ{\textsubbar{ɔ}}/d͡ʒ{\`{\~ɔ}}ŋ & {8} & 5+3\\
{3} & s{\textsubtilde{á}}{\textsubbar{a}}/tʰ{\`{ɔ}}ʔ & {9} & 5+4, 10--1\\
{4} & ɲee/ì-y{\textsubtilde{ì}}i̬, (á-dàa) & {10} & (kpooɡo, ɡbùnè, kpélé, sí-n{\~{ʊ}}y - 5PL)\\
{5} & mw{\~{a}}/w{\textsubtilde{à}}a, n{\`{\~ɔ}}n & {20} & ɡbeere, (tʃúkúrì)\\
{6} & 5+1 & {100} & 20*5\\
&  & {1000} & kpíɛ `a goat'\\
\lspbottomrule
\end{tabularx}
\end{table}

Three of the languages belonging to this branch show too many forms, suggesting that we are dealing with a heterogeneous branch. In view of its numerical terms, it is not immediately apparent why this branch has been singled out.

\subsubsubsection{Grusi}%3.9.2.2.3.
\subparagraph{i. *Eastern Grusi (\tabref{tab:3:174})}
~
\begin{table}
\caption{\label{tab:3:174}Eastern Grusi numerals (*)}
\begin{tabularx}{\textwidth}{lXrl}
\lsptoprule
{1} & ɖ{\'{ə}}m/l{\`{ʊ}}m/y{\'{ʊ}}m, re/{\'{ɔ}}ɖe & {7} & lʊbɛ, 6+1, 4+3, 10--3\\
{2} & la/l{\`{ɛ}} & {8} & 4 redupl., 4PL, 10--2, toozo, (k͡pèèrè)\\
{3} & tòòsó/tooro & {9} & 10--1, isolated forms\\
{4} & násá/naara & {10} & fu, (n{\'{ʊ}}á - 5PL, sàlá)\\
{5} & n{\'{ʊ}}/n{\'{ʊ}}ŋ, kpás{\`{ɪ}}/ɡb{\'ã}nzì & {20} & ko/kuo/koowu, (sao, nɛ{\'{ɛ}}l{\`{ɛ}}, 10*2)\\
{6} & loɖò/looro/lèèjò, (3PL) & {100} & 20*5, < Ewe,\il{Ewe} ('guinea fowl')\\
&  & {1000} & kòtòkó, kpoŋ\\
\lspbottomrule
\end{tabularx}
\end{table}


\largerpage[3]
\subparagraph{ii. *Northern Grusi (\tabref{tab:3:175})}
~
\begin{table}
\caption{\label{tab:3:175}Northern Grusi numerals (*)}
\begin{tabularx}{\textwidth}{lXrl}
\lsptoprule
{1} & du/lu, (ténɡí) & {7} & p{\`{ɛ}}, (4+3, 5+2)\\
{2} & le/l{\`{ə}}/(ɲìí) & {8} & nānā (4 redupl.), (lyɛlɛ, bàndá)\\
{3} & t{\`{ɔ}}/twà/c{\'{ɔ}}{\`{ɔ}} & {9} & n{\`{ʊ}}ɡʊ, nìbu, (10-X)\\
{4} & na/nīān/nàas & {10} & fúɡ{\'{ə}}, (fo)\\
{5} & nu & {20} & 10*2, (sāp{\={ʊ}}ā, 10+10, swéní)\\
{6} & d{\`{ʊ}}, (5+pi) & {100} & bi, (z{\v{ɔ}}m)\\
&  & {1000} & m{\`{ʊ}}r{\`{ʊ}}\\
\lspbottomrule
\end{tabularx}
\end{table}

\subparagraph{iii. *Western Grusi (\tabref{tab:3:176})}
~
\begin{table}[h]
\caption{\label{tab:3:176}Western Grusi numerals (*)}
\begin{tabularx}{\textwidth}{llrQ}
\lsptoprule
{1} & kpáŋ/kpee, bala, do/deo/dííŋ/digi & {7} & lʊp,p{\'{ɛ}}{\'{ɛ}}/piɛ~, 5+2\\
{2} & lɛ/nɛ/lìɛ & {8} & córí/kyórí, 5+3, (pɔɔ)\\
{3} & toro & {9} & n{\'{ɛ}}m{\'{ɛ}}/nìbí, 10--1, 5+4\\
{4} & naa/naasi/naare & {10} & fi\\
{5} & nue/nw{\'{\~ɔ}}/n{\`{ɔ}}ŋ & {20} & m{\'{ɛ}}r{\'{ɛ}}, mʊɡ{\'{ɔ}} (< Mande?), (mááɡí, toko, ma-cu?)\\
{6} & l{\`{ʊ}}r{\`{ʊ}}/*lug/d{\`{ʊ}}, 5+1, (ɡo) & {100} & k{\`{ɔ}}wá/k{\`{ɔ}}{\'{ɔ}}, z{\'{ɔ}}l{\'{ɔ}}, lafa\\
&  & {1000} & ɡboŋ/b{\'{ʊ}}í\\
\lspbottomrule
\end{tabularx}
\end{table}

  
The most probable *Proto-Grusi\il{Proto-Grusi} reconstructions based on the roots attested in at least two Grusi branches are summarized in the table below (\tabref{tab:3:177}).

\begin{table}
\caption{\label{tab:3:177}Proto-Grusi\il{Proto-Grusi} numeral system (*)}
\begin{tabularx}{\textwidth}{lXrl}
\lsptoprule
{1} & do/du/lu, de/re & {7} & pɛ/lʊ-pɛ/lʊ-bɛ, 5+2\\
{2} & lɛ/le/ne/ɲi & {8} & 4 redupl.\\
{3} & toro/toso/tɔ & {9} & 10--1, nibi/nibu (ni-bi/bu?)\\
{4} & naare/naasi/na & {10} & fu/fi\\
{5} & nu/nʊ & {20} & 10*2?\\
{6} & dʊ/lo-ɖo/lo-ro, 5+1 & {100} & 20*5? bi? kɔwa/kɔɔ?\\
&  & {1000} & kpoŋ/ɡboŋ\\
\lspbottomrule
\end{tabularx}
\end{table}

  
\subsubsubsection{Kirma-Tyurama}%3.9.2.2.4.
\il{Kirma}\il{Tyurama}
\begin{table}
\caption{\label{tab:3:178}Kirma\il{Kirma}-Tyurama\il{Tyurama} numerals}
\begin{tabularx}{\textwidth}{lXrl}
\lsptoprule
{1} & déiŋ/d{\~{e}}{\~{e}}ná & {7} & 5+2\\
{2} & h{\'ã}{\~{i}}/h{\~{a}}l & {8} & 5+3\\
{3} & síɛi/siɛl & {9} & 5+4, 10--1\\
{4} & na(a) & {10} & n{\'ũ}{\'{\~ɔ}}s{\`{\~ɔ}}/c{\'ĩ}ŋcíelùó\\
{5} & di & {20} & kómòrré/ɡu{\~{r}}\\
{6} & 5+1 & {100} & ɡundi, 20*5\\
&  & {1000} & 200*5, 800+200\\
\lspbottomrule
\end{tabularx}
\end{table}

 
\subsection{Kulango}%3.9.3.
\il{Kulango}
\begin{table}
\caption{\label{tab:3:179}Kulango\il{Kulango} numeral system}
\begin{tabularx}{\textwidth}{lXrl}
\lsptoprule
{1} & ta(a) < *t{\textsubbar{a}}{\textsubtilde{à}} & {7} & 5+2\\
{2} & bila( < Mande), nyʊ{\`{ʊ}} & {8} & 5+3\\
{3} & s{\~{a}}{\~{a}}be (< Mande) & {9} & 5+4\\
{4} & na & {10} & nuunu (< *5redupl.), *ji/yi\\
{5} & tɔ & {20} & yipì-/dʒipi-\\
{6} & 5+1 & {100} & kɛm{\`{ɛ}} (< Mande)\\
&  & {1000} & wulo (< Mande)\\
\lspbottomrule
\end{tabularx}
\end{table}

The source form of the term for ‘one’ with a nasalized vowel is reconstructed on the basis of the evidence presented by Stefan \citet[323]{Elders2007}. As we have seen, the Gur term for ‘five’ is reconstructed  as *\textit{nu} on the basis of the evidence provided by the groups discussed above. It should be noted that this form goes back to the lexical root meaning ‘hand’ (Kulango\il{Kulango} \textit{nu-gò}). The term for ‘ten’ in Kulango is a reduplicated *\textit{nu}, whereas a different root is attested for ‘five’. It is also noteworthy that the terms for ‘two’, ‘three’, ‘hundred’ and ‘thousand’ are borrowed from Mande.


\subsection{Lobi-Dyan}%3.9.4.
\il{Lobi}\il{Dyan}According to Anthony Naden’s classification \citep{Naden1989}, these languages belong to different groups of the Gur languages, so their evidence will be presented separately. 

“More recent classifications (Labouret and Manessy) regarded Lobi\il{Lobi} (Lobiri)\il{Lobi (Lobiri)} and J{\textsubbar{a}}{\textsubbar{a}}ne as closely related” \citep[212]{MieheTham2007} (\tabref{tab:3:180}).

\begin{table}
\caption{\label{tab:3:180}Lobi\il{Lobi}-Dyan\il{Dyan} numerals}
\begin{tabularx}{\textwidth}{rXll}
\lsptoprule
& Lobi\il{Lobi} & Dyan\il{Dyan} & *Lobi-\il{Lobi}Dyan\il{Dyan}\\
\midrule 
{1} & bì{\`{ɛ}}l, *do & b{\~{ɛ}}ɡ/ɓ{\textsubtilde{\`{ɛ}}}(ŋ)kù/bɪɛle, *d{\`{\textsubtilde{u}}} & bɪ{\`{ɛ}}l, *dò\\
{2} & ny{\`{ɔ}}/n{\`{ɔ}} & ny{\`{\~ɔ}} & ny{\`{ɔ}}(n)\\
{3} & tʰ{\v{e}}r & th{\`{\~ɛ}}s(i) & th{\`{\~ɛ}}s(i)/tʰ{\v{e}}r\\
{4} & n{\'ã} & n{\textsubtilde{à}}{\textsubtilde{à}} & n{\'ã}\\
{5} & m{\`{ɔ}}{\`{ɩ}}/*mà & dìèmà, *m{\`{ɔ}}l{\`{ɔ}} & m{\`{ɔ}}{\`{ɩ}}/*mà/*m{\`{ɔ}}l{\`{ɔ}}, dìèmà, \\
{6} & 5+1 & 5+1 & 5+1\\
{7} & 5+2 & 5+2 & 5+2\\
{8} & 5+3 & 5+3 & 5+3\\
{9} & 10--1 & 10--1 & 10--1\\
{10} & ny{\`{ʊ}}{\'{ɔ}}r & ni-kpo & ni-kpo, ny{\`{ʊ}}{\'{ɔ}}r\\
{20} & kpèle & ceeru & kpèle, ceeru\\
{100} & tàmâ & tàmúɡú & tàmâ\\
{1000} & ɡb{\`{ʊ}}lanɪ & 100*10 & ɡb{\`{ʊ}}lanɪ, 100*10\\
\lspbottomrule
\end{tabularx}
\end{table}

\subsection{Senufo}%3.9.5.
\begin{table}
\caption{\label{tab:3:181}Senufo numerals}
\begin{tabularx}{\textwidth}{lQrQ}
\lsptoprule
{1} & n{\`{ɔ}}n-, ni-ŋɡbe/nuŋɡba, nìk{\`ĩ}/ninɡin & {7} & 5+2, 6+1\\
{2} & sin/soin/sun/syen & {8} & 5+3, 6+2\\
{3} & t{\`ã}{\~{a}}/taàr & {9} & 5+4, 10--1, 6+3\\
{4} & tésyàr/sīc{\={ɛ}}r{\={ɛ}}/tityere & {10} & kɛ\\
{5} & bwa/bwɔ, guru/kuru (<`fist’), guno, (nɔ) & {20} & ɡbèɲ/ɡ͡bēy, fulo, toko/togo, nafa, isolated forms\\
{6} & kwa{\`{ɲ}}/kwāy, ɡbaara, ɡɔlɔŋ~, 5+1, (nõli) & {100} & 20*5, lafa (< Kwa)\il{Kwa}\\
&  & {1000} & 200*5, (gben-, bɔlɔ, pwoo, sakere)\\
\lspbottomrule
\end{tabularx}
\end{table}

Many of the forms are quoted in brackets, i.e. they are isolated forms attested within the Senufo group comprising about fifteen idioms. As in a number of other Gur branches, the last syllable/segment of a numerical term often represents a coordinating noun class suffix. Below is an excerpt from the table showing the inflection of numerals by class in Tenyer\il{Tenyer} (Syer\il{Syer} variety), as published by Klaudia Dombrowsky-Hahn in \citealt{Winkelmann2007f}:420, \tabref{tab:3:182}).

\begin{table}
\caption{\label{tab:3:182}Tenyer\il{Tenyer} numerals (a fragment)}
\begin{tabularx}{\textwidth}{lXXXl}
\lsptoprule
{Class} {SG} & {u} & {li} & {ke} & {te} {dim.}\\
\midrule
‘one’ & nun & nuni & nuŋ & nunge\\
\tablevspace
{Class} {PL} & {pi} & {ki} & {yi} & {te} {dim.}\\
\midrule
‘two’ & syob {\textasciitilde} syou & sy{\~{a}} & syii & syimbi\\
‘three’ & trab & tar & tar & tarbi\\
‘four’ & tikyireb & tihyɛr & tihyɛr & tihyɛrbi\\
\lspbottomrule
\end{tabularx}
\end{table}

  
This presentation illustrates how problematic defining the numerical roots can be.
% 

 \newpage 
\subsection{Teen}%3.9.6.
\il{Teen}
\begin{table}[h!]
\caption{\label{tab:3:183}Teen\il{Teen} numerals}


\begin{tabularx}{\textwidth}{lQrX}
\lsptoprule

1 & tani & 7 & 5+2\\
2 & nyor & 8 & 5+3\\
3 & sanr & 9 & 10--1\\
4 & nan & 10 & pɔrwɔ\\
5 & tɔ & 20 & toko\\
6 & 5+1 & 100 & 20*5\\
&  & 1000 & danyɛ\\
\lspbottomrule
\end{tabularx}
\end{table}

 
\subsection{Tiefo}%3.9.7.
\il{Tiefo}
\begin{table}
\caption{\label{tab:3:184}Tiefo\il{Tiefo} numerals}
\begin{tabularx}{\textwidth}{lQrX}
\lsptoprule
1 & d{\`{\~ɛ}} & 7 & 5+2\\
2 & j{\~{ɔ}} & 8 & 5+3\\
3 & s{\'ã} & 9 & 5+4\\
4 & ʔuʔ{\'{\~ɔ}}/ŋɔɔ & 10 & támú, k{\~{ɛ}}~\\
5 & k{\`ã} & 20 & kp{\~{a}}\\
6 & 5+1 & 100 & 20*5\\
&  & 1000 & waga (< Mande)\\
\lspbottomrule
\end{tabularx}
\end{table}

\largerpage[2]

\subsection{Tusia}%3.9.8.
\il{Tusia}
\begin{table}
\caption{\label{tab:3:185}Tusia\il{Tusia} numerals}
\begin{tabularx}{\textwidth}{lQrX}
\lsptoprule
1 & n{\'{ɔ}}nk{\`{ɩ}}, *n{\~{\^ə}}ŋ & 7 & 5+2\\
2 & nín{\'{ɔ}}, *n{\~{\^ɪ}}ŋ & 8 & 5+3\\
3 & t{\'{\~ɔ}}n{\'{ɔ}} & 9 & 5+4\\
4 & {\'{n}}y{\'ã}h/j{\~{\^a}} & 10 & ɡb{\~{a}}m/*ɡb{\~{ɔ}}/bw{\`{ɔ}}\\
5 & k(w)l{\'{ɔ}} & 20 & túkúrí, *tiki\\
6 & 5+1 & 100 & 20*5, kw{\v{ɛ}}\\
&  & 1000 & < píy `goat’, n{\'ã}ˤ'cow'\\
\lspbottomrule
\end{tabularx}
\end{table}

 \clearpage
\subsection{Viemo}%3.9.9.
\il{Viemo}
\begin{table}
\caption{\label{tab:3:186}Viemo\il{Viemo} numerals}


\begin{tabularx}{\textwidth}{lQrQ}
\lsptoprule

1 & d{\~{u}}de, *dun- & 7 & 5+2?\\
2 & niin{\~{i}} & 8 & 4*2, 5+3\\
3 & s{\~{a}}s{\~{i}} & 9 & 10--1\\
4 & jum{\~{i}} & 10 & kwɔm{\~{u}}\\
5 & kuɛɡe, *k{\textsubbar{ɔ}} & 20 & fɛrɛyɔ\\
6 & 5+1 & 100 & t{\~{a}}mõ\\
&  & 1000 & vie-?\\
\lspbottomrule
\end{tabularx}
\end{table}
 
\subsection{Wara-Natioro}%3.9.10.
\il{Wara}\il{Natioro}It should be noted that the most important evidence pertaining to this group is relatively recent. In his publication of the comparative lexical list Tasséré Sawadogo noted that Faniagara\il{Faniagara} is radically different from both Wara\il{Wara} and Natioro\il{Natioro} \citep{Sawadogo2002}. Its similarity index with the Natioro and Wara dialects is 12 and 30 percent respectively (the SIL list? idem., p. 15). Thus he had every reason to postulate the existence of an isolated language (Palɛn\il{Palɛn}) in the Wara-Natioro group. 

\sloppy
Since the data collected by Tasséré Sawadogo is absent from the major data\-bases that are now incorporated into the RefLex database by Guillaume Segerer, it seems reasonable to present it below for each Wara\il{Wara}-Natioro\il{Natioro}-Paleni idiom in order to suggest the reconstruction of numerical terms within each of the three sub-groups and within the group as a whole (\tabref{tab:3:187}).
\fussy 

\begin{table}
\caption{\label{tab:3:187}Wara\il{Wara}-Natioro\il{Natioro}-Paleni numerals}
\small
\begin{tabularx}{\textwidth}{QlQlllQ}
\lsptoprule

~ &  & `1' & `2' & `3' & `4' & `5' \\
\midrule
Natioro\il{Natioro} & Dinaoro\il{Dinaoro} & káːbà & ɲ{\'ĩ}nd{\'ĩ} & táe & ŋnáe & sùsú\\
Natioro\il{Natioro} & Timba\il{Timba} & káːbà & ɲ{\'ĩ}ndí & tá & n{\'ã} & sùsú\\
Natioro\il{Natioro} & Kawara\il{Kawara} & kābà & ɲ{\`ĩ}dí & tá & ná & sùsú\\
*Natioro\il{Natioro} &  & káːbà \mbox{(ka-ba?)\footnotemark{}} & ɲ{\'ĩ}ndí & tá(é) & ná (é) & sùsú\\
Wara?\il{Wara} & Sourani\il{Sourani} & p{\'{ɔ}} & b{\v{ɔ}} & t{\'ã} & nàsá & sùsú\\
Wara\il{Wara} & Negeni\il{Negeni} & kàpó & b{\v{o}} & t{\'ĩ}ː & n{\'ã}ːs{\'ũ} & sùsú\\
Wara\il{Wara} & Niansogoni\il{Niansogoni} & p{\'{ʊ}}ːwò & b{\v{o}} & t{\'{ɩ}}ː & náːsó & sùsú\\
*Wara\il{Wara} &  & p{\'{ɔ}} & b{\v{o}}, *n{\={\~{i}}}ntó & t{\'ã}(i) & naaso & sùsú, \\
Palɛn\il{Palɛn} & Faniagara\il{Faniagara} & káfā & bá & t{\'ã}ːré & náːré & sùsú\\
*Palɛn\il{Palɛn} & Faniagara\il{Faniagara} & ká-fā & bá, *n{\'ĩ}nté & t{\'ã}ːré & náːré & sùsú, *si/sɔ\\
\textbf{*Wara-}\il{Wara}\textbf{Natioro-}\il{Natioro}\textbf{Paleni} &  & \textbf{ba/fa, pɔ} & \textbf{n{\'ĩ}nté, b{\v{o}}} & \textbf{ta(r)i} & \textbf{na(r)i} & \textbf{sùsú, sV}\\
~ &  & `6' & `7' & `8' & `9' & `10' \\
Natioro\il{Natioro} & Dinaoro\il{Dinaoro} & ŋzàb{\'{ɔ}} & téːndé & n{\'ã}ŋgàn{\'ã}ŋgán{\`ĩ} & kâwó & pw{\`{ɔ}}ː\\
Natioro\il{Natioro} & Timba\il{Timba} & {\`{ŋ}}zàːb{\'{ɔ}} & déːndí & náŋgánáŋgánì & kāw{\`{ɔ}}m{\'ũ} & pw{\'{ɔ}}ː\\
Natioro\il{Natioro} & Kawara\il{Kawara} & nsàb{\'{ɔ}} & tèndí & nàŋgānàŋgádí & kàw{\={\~{u}}}mò & p{\'{ɔ}}\\
*Natioro\il{Natioro} &  & nsàb{\'{ɔ}} (sa-1?) & téndí & 4+4 & kawo & p(w){\'{ɔ}}\\
Wara?\il{Wara} & Sourani\il{Sourani} & sùrp{\'{ɔ}} & sūrùdó & s{\`ĩ}nt{\'ã} & s{\`ĩ}nːá & k{\`ã}nːsú\\
Wara\il{Wara} & Negeni\il{Negeni} & sírípò & s{\'ĩ}n{\={\~{i}}}ntó & s{\={\~{i}}}ntí & sīnːáːs{\'ũ} & k{\`ã}ːs{\'ã}\\
Wara\il{Wara} & Niansogoni\il{Niansogoni} & sírìpò & sùrùntó & s{\={ɩ}}ːnt{\'{ɩ}}ː & s{\'{ɩ}}nː{\'ã}ːs{\'ũ} & k{\`ã}ːs{\'ã}\\
*Wara\il{Wara} &  & si-1 & si-2 & si-3 & si-4 & k{\`ã}ːs{\'ã}\\
Palɛn\il{Palɛn} & Faniagara\il{Faniagara} & s{\'ĩ}n{\'ĩ}fà & s{\'ĩ}n{\'ĩ}nté & s{\={ɔ}}táːré & s{\={ɔ}}nːáːré & f{\'{ɔ}}\\
*Palɛn\il{Palɛn} & Faniagara\il{Faniagara} & si-1 & si-2 & s{\={ɔ}}-3 & s{\={ɔ}}-4 & f{\'{ɔ}}\\
\textbf{*Wara-}\il{Wara}\textbf{Natioro-}\il{Natioro}\textbf{Paleni} &  & \textbf{5+1} & \textbf{5+2, téndí?}  & \textbf{5+3, 4+4} & \textbf{5+4, kawo?}  & \textbf{p(w)ɔ/ fɔ, k{\`ã}ːs{\'ã}?} \\
\lspbottomrule
\end{tabularx}
\end{table}

\footnotetext{Regarding the Natioro\il{Natioro} forms for ‘one’ André Prost remarks: ‘\textit{puwolo} (après un substantif: \textit{kaaba}\textit{)’} \citep[78]{Prost1968}.  Thus, the opposition between the Wara\il{Wara} and Natioro forms of ‘one’ reflected in the table may be purely functional (for Wara Prost quotes the \textit{puwo} and \textit{kapo} forms).}
According to other sources, the forms \textit{w{\'ã}/} \textit{nwõ,} \textit{sɔ} are attested in Wara\il{Wara}-Natioro\il{Natioro} for ‘twenty’. The patterns ‘20*5’ and ‘400*2+200’ are attested for ‘hundred’ and ‘thousand’ respectively.

 
\subsection{Proto-Gur}%3.9.11.
\il{Proto-Gur}\subsubsection{‘One’}%3.9.11.1.
The main forms of ‘one’ reconstructable in sixteen branches of Gur are as follows (\tabref{tab:3:188}).

\begin{table}
\caption{\label{tab:3:188}Stems for `1' in Gur}
\small
\begin{tabularx}{\textwidth}{>{\raggedright}p{36mm}lQQlp{11mm}}
\lsptoprule

A. Bariba\il{Bariba} 				 		&  		&  		&  & tiā & \\
B. Central:\\~~~~1. Northern\\~~~~~~~~A. Bwamu\il{Bwamu} 			& do &  &  &  & \\
~~~~~~~~B. Kurumfe\il{Kurumfe} 					& dom 		&  &  &  & \\
~~~~~~~~C. *Proto-Oti-Volta\il{Proto-Oti-Volta} 		&  		& den/yen, de? &  &  & ni\\
~~~~Southern\\~~~~~~~~A. Dogoso-\il{Dogoso}Khe\il{Khe} 	&  & le & kpò &  & \\
~~~~~~~~C. Gan-Dogose\il{Dogose}		 		&  		& lèŋ & kpo/po &  & \\
~~~~~~~~D. *Proto-Grusi\il{Proto-Grusi}		 		& do/du/lu & de/re &  &  & \\
~~~~~~~~E. Kirma-\il{Kirma}Tyurama\il{Tyurama}  	&  & déiŋ/d{\~{e}}{\~{e}}ná &  &  & \\
C. Kulango\il{Kulango} 				 		&  		&  		&  & ta(a) < *t{\textsubbar{a}}{\textsubtilde{à}} & \\
D. Lobi-\il{Lobi}Dyan\il{Dyan}  		 		& *dò 		&  &  &  & \\
E. Senufo 					 		&  		&  		& ni-ŋɡbe/ nu-ŋɡba &  & nìk{\`ĩ}/ ninɡin\\
F. Teen\il{Teen}				   		&  		&  &  & tani & \\
G. Tiefo\il{Tiefo}  				 		&  		& d{\`{\~ɛ}} &  &  & \\
H. Tusia\il{Tusia} 				 		&  		&  		&  &  & n{\'{ɔ}}nk{\`{ɩ}}\\
I. Viemo\il{Viemo}   						& d{\~{u}}de, *dun- &  &  &  & \\
J. Wara-\il{Wara}Natioro-\il{Natioro}Paleni   			&  		&  		& pɔ &  & \\
\lspbottomrule
\end{tabularx}
\end{table}
An attempt to reconstruct a Proto-Gur\il{Proto-Gur} form is probably not reasonable at this point, since all the forms quoted above are important for comparative purposes.

\clearpage
\subsubsection{‘Two’}
\begin{table}
\caption{\label{tab:3:189}Stems for `2' in Gur}
\begin{tabularx}{\textwidth}{llllXX}
\lsptoprule
  & `2' & `2' & `2' & `2' & `2' \\
\midrule
A. Bariba\il{Bariba} 				 	 & ru &  &  &  & \\
B. Central:\\
~~~~1. Northern\\
~~~~~~~~A. Bwamu\il{Bwamu}& ɲū &  &  &  & \\
~~~~~~~~B. Kurumfe\il{Kurumfe} 				&  &  &  & h{\~{i}}{\~{i}} & \\
~~~~~~~~C. *Proto-Oti-Volta\il{Proto-Oti-Volta} 	&  & li/yi &  &  & \\
~~~~Southern\\
~~~~~~~~A. Dogoso-\il{Dogoso}Khe\il{Khe} 	& jɔ(n) &  &  &  & \\
~~~~~~~~C. Gan-Dogose\il{Dogose}		 	& y{\textsubtilde{\'{ɔ}}}/ɲ{\textsubbar{ɔ}}/dʒ{\`{\~ɔ}}ŋ &  &  &  & \\
~~~~~~~~D. *Proto-Grusi\il{Proto-Grusi}		 	&  & lɛ/le & ne/ɲi &  & \\
~~~~~~~~E. Kirma-\il{Kirma}Tyurama\il{Tyurama}  	&  &  &  & h{\'ã}{\~{i}}/h{\~{a}}l & \\
C. Kulango\il{Kulango} 				 	  & nyʊ{\`{ʊ}} &  &  &  & bila\newline  \mbox{(< Mande)}\\
D. Lobi-\il{Lobi}Dyan\il{Dyan}  		 	  & ny{\`{ɔ}}(n) &  &  &  & \\
E. Senufo 					 	  &  &  &  &  & sin/soin/ sun/syen\\
F. Teen\il{Teen}				   	  & nyor &  &  &  & \\
G. Tiefo\il{Tiefo}  				 	  & j{\~{ɔ}} &  &  &  & \\
H. Tusia\il{Tusia} 				 	  &  &  & nín{\'{ɔ}}, *n{\^{\~ɪ}}ŋ &  & \\
I. Viemo\il{Viemo}   					  &  &  & niin{\~{i}} &  & \\
J. Wara-\il{Wara}Natioro-\il{Natioro}Paleni   		  &  &  & n{\'ĩ}nté &  & b{\v{o}}\\
\lspbottomrule
\end{tabularx}
\end{table}

Apparent isolates and obvious borrowings are presented in the rightmost column.

\newpage 
\subsubsection{‘Three’ and ‘Four’}%3.9.11.3.
\begin{table}
\caption{\label{tab:3:190}Stems for `3' and `4' in Gur}


\begin{tabularx}{\textwidth}{llXXX}
\lsptoprule

  {~} &  {3} &  {3} &  {4} &  {4}\\
\midrule
A. Bariba\il{Bariba} 				 	&  {i-ta} &  {~} &  {{\`{n}}-nɛ} & \\
B. Central:\\~~~~1. Northern\\~~~~~~~~A. Bwamu\il{Bwamu}&  {t{\~{i}}} &  {~} &  {náa} & \\
~~~~~~~~B. Kurumfe\il{Kurumfe} 				&  {t{\~{a}}{\~{a}}} &  {~} &  {n{\~{a}}{\~{a}}} & \\
~~~~~~~~C. *Proto-Oti-Volta\il{Proto-Oti-Volta} 	&  {ta(t)} &  {~} &  {naa(si)} & \\
~~~~Southern\\~~~~~~~~A. Dogoso-\il{Dogoso}Khe\il{Khe} 	&  {thɔ} &  {~} &  {dáa} & \\
~~~~~~~~C. Gan-Dogose\il{Dogose}		 	&  {s{\textsubtilde{á}}{\textsubbar{a}}/tʰ{\`{ɔ}}ʔ} &  {~} &  {ɲee/ì-y{\textsubtilde{ì}}i̬, (á-dàa)} & \\
~~~~~~~~D. *Proto-Grusi\il{Proto-Grusi}		 	&  {toro/toso/tɔ} &  {~} &  {naare/naasi/na} & \\
~~~~~~~~E. Kirma-\il{Kirma}Tyurama\il{Tyurama}  	&  {síɛi/siɛl} &  {~} &  {na(a)} & \\
C. Kulango\il{Kulango} 				 	&  {} &  {s{\~{a}}{\~{a}}be\newline (< Mande)} &  {na} & \\
D. Lobi-\il{Lobi}Dyan\il{Dyan}  		 	&  {th{\`{\~ɛ}}s(i)/tʰ{\v{e}}r} &  {~} &  {n{\'ã}} & \\
E. Senufo 					 	&  {t{\`ã}{\~{a}}/taàr} &  {~} &  {~} & tésyàr/ sīc{\={ɛ}}r{\={ɛ}}/ tityere\\
F. Teen\il{Teen}				   	&  {sanr} &  {~} &  {nan} & \\
G. Tiefo\il{Tiefo}  				 	&  {s{\'ã}} &  {~} &  {~} & ʔuʔ{\'{\~ɔ}}/ŋɔɔ\\
H. Tusia\il{Tusia} 				 	&  {t{\'{\~ɔ}}n{\'{ɔ}}} &  {~} &  {{\'{n}}y{\'ã}h/j{\~{\^a}}} & \\
I. Viemo\il{Viemo}   					&  {s{\~{a}}s{\~{i}}} &  {~} &  {~} & jum{\~{i}}\\
J. Wara-\il{Wara}Natioro-\il{Natioro}Paleni   		&  {ta(r)i} &  {~} &  {na(r)i} & \\
\lspbottomrule
\end{tabularx}
\end{table}

The reflexes of the most persistent NC roots are observable in the majority of the branches.

\newpage 

\subsubsection{‘Five’}%3.9.11.4.
\begin{table}
\caption{\label{tab:3:191}Stems for `5' in Gur}


\begin{tabularx}{\textwidth}{l lXllX}
\lsptoprule

   & `5' & `5' & `5' & `5' & `5' \\
\midrule
A. Bariba\il{Bariba} 				 	& n{\`{ɔ}}ɔbù &  &  &  & \\
B. Central:\\~~~~1. Northern\\~~~~~~~~A. Bwamu\il{Bwamu}& hò-nú &  &  &  & \\
~~~~~~~~B. Kurumfe\il{Kurumfe} 				& nɔm &  &  &  & \\
~~~~~~~~C. *Proto-Oti-Volta\il{Proto-Oti-Volta} 	& nu &  &  &  & \\
~~~~Southern\\~~~~~~~~A. Dogoso-\il{Dogoso}Khe\il{Khe} 	& nɔ(n) &  &  &  & \\
~~~~~~~~C. Gan-Dogose\il{Dogose}		 	& n{\`{\~ɔ}}n & mw{\~{a}}/ w{\textsubtilde{à}}a &  &  & \\
~~~~~~~~D. *Proto-Grusi\il{Proto-Grusi}		 	& nu/nʊ &  &  &  & \\
~~~~~~~~E. Kirma-\il{Kirma}Tyurama\il{Tyurama}  	&  &  &  & di & \\
C. Kulango\il{Kulango} 				 	&  &  & tɔ &  & \\
D. Lobi-\il{Lobi}Dyan\il{Dyan}  		 	&  & m{\`{ɔ}}{\`{ɩ}}/ *mà/ *m{\`{ɔ}}l{\`{ɔ}} &  & dìèmà & \\
E. Senufo 					 	& guno, (nɔ) & bwa/ bwɔ &  &  & \\
F. Teen\il{Teen}				   	&  &  & tɔ &  & \\
G. Tiefo\il{Tiefo}  				 	&  &  &  &  & k{\`ã}\\
H. Tusia\il{Tusia} 				 	&  &  &  &  & k(w)l{\'{ɔ}}\\
I. Viemo\il{Viemo}   					&  &  &  &  & kuɛɡe, *k{\textsubbar{ɔ}}\\
J. Wara-\il{Wara}Natioro-\il{Natioro}Paleni   		&  &  & sùsú, sV &  & \\
\lspbottomrule
\end{tabularx}
\end{table}

The etymological relationship of *\textit{nu} ‘5’ and ‘hand’, is attested in Central Gur and possibly in Bariba\il{Bariba} and Senufo. Isolated bases may go back to this meaning as well. At the same time, the base preserved in Kulango\il{Kulango}, Teen\il{Teen} and possibly Wara\il{Wara}-Natioro\il{Natioro}-Paleni is comparable to *\textit{tan} found in BC and some other families.   


\newpage 
\subsubsection{‘Six’ and ‘Seven’} %3.9.11.5.
\begin{table}
\caption{\label{tab:3:192}Stems and patterns for `6' and `7' in Gur}


\begin{tabularx}{\textwidth}{l lXlXl}
\lsptoprule

  & `6' & `6' & `7' & `7' & `7' \\
\midrule
A. Bariba\il{Bariba} 				 	& 5+1 &  & 5+2 &  & \\
B. Central:\\~~~~1. Northern\\~~~~~~~~A. Bwamu\il{Bwamu}& 5+1 &  & 5+2 &  & \\
~~~~~~~~B. Kurumfe\il{Kurumfe} 				&  & hʊrʊ &  & p{\~{ɛ}}{\~{ɛ}} & \\
~~~~~~~~C. *Proto-Oti-Volta\il{Proto-Oti-Volta} 	&  & lob/yob &  & poi(n)? & *lob-le (6+1)?\\
~~~~Southern\\~~~~~~~~A. Dogoso-\il{Dogoso}Khe\il{Khe} 	& 5+1 &  & 5+2 &  & \\
~~~~~~~~C. Gan-Dogose\il{Dogose}		 	& 5+1 &  & 5+2 &  & \\
~~~~~~~~D. *Proto-Grusi\il{Proto-Grusi}		 	& 5+1 & dʊ/lo-ɖo/lo-ro & 5+2 & pɛ/lʊ-pɛ/lʊ-bɛ & \\
~~~~~~~~E. Kirma-\il{Kirma}Tyurama\il{Tyurama}  	& 5+1 &  & 5+2 &  & \\
C. Kulango\il{Kulango} 				 	& 5+1 &  & 5+2 &  & \\
D. Lobi-\il{Lobi}Dyan\il{Dyan}  		 	& 5+1 &  & 5+2 &  & \\
E. Senufo 					 	& 5+1, & kwa{\`{ɲ}}/ kwāy, ɡbaara, nõli & 5+2 &  & 6+1\\
F. Teen\il{Teen}				   	& 5+1 &  & 5+2 &  & \\
G. Tiefo\il{Tiefo}  				 	& 5+1 &  & 5+2 &  & \\
H. Tusia\il{Tusia} 				 	& 5+1 &  & 5+2 &  & \\
I. Viemo\il{Viemo}   					& 5+1 &  & 5+2? &  & \\
J. Wara-\il{Wara}Natioro-\il{Natioro}Paleni   		& 5+1 &  & 5+2 &  & téndí? \\
\lspbottomrule
\end{tabularx}
\end{table}

The patterns *’6=5+1’ and *’7=5+2’ can be safely reconstructed at the Proto-Gur\il{Proto-Gur} level. The exeptionally wide range of forms for ‘six’ attested in Senufo is noteworthy.

\newpage 
\subsubsection{‘Eight’ and ‘Nine’}%3.9.11.6.
\begin{table}
\caption{\label{tab:3:193}Stems and patterns for `8' and `9' in Gur}


\begin{tabularx}{\textwidth}{llllllQ}
\lsptoprule

   & `8' & `8' & `8' & `9' & `9' & `9' \\
\midrule
A. Bariba\il{Bariba} 				 	& 5+3 &  &  & 5+4 &  & \\
B. Central:\\~~~~1. Northern\\~~~~~~~~A. Bwamu\il{Bwamu}& 5+3 &  &  &  &  & d{\`ĩ}iní/dènú\\
~~~~~~~~B. Kurumfe\il{Kurumfe} 				&  &  & tɔɔ &  &  & fa\\
~~~~~~~~C. *Proto-Oti-Volta\il{Proto-Oti-Volta} 	&  &  & ni &  &  & wey/weʔ\\
~~~~Southern\\~~~~~~~~A. Dogoso-\il{Dogoso}Khe\il{Khe} 	& 5+3 &  &  & 5+4 &  & \\
~~~~~~~~C. Gan-Dogose\il{Dogose}		 	& 5+3 &  &  & 5+4 & 10--1 & \\
~~~~~~~~D. *Proto-Grusi\il{Proto-Grusi}		 	&  & 4 redupl. &  &  & 10--1 & nibi/nibu (ni-bi/bu?)\\
~~~~~~~~E. Kirma-\il{Kirma}Tyurama\il{Tyurama}  	& 5+3 &  &  & 5+4 & 10--1 & \\
C. Kulango\il{Kulango} 				 	& 5+3 &  &  & 5+4 &  & \\
D. Lobi-\il{Lobi}Dyan\il{Dyan}  		 	& 5+3 &  &  &  & 10--1 & \\
E. Senufo 					 	& 5+3 &  & 6+2 & 5+4 & 10--1 & 6+3\\
F. Teen\il{Teen}				   	& 5+3 &  &  &  & 10--1 & \\
G. Tiefo\il{Tiefo}  				 	& 5+3 &  &  & 5+4 &  & \\
H. Tusia\il{Tusia} 				 	& 5+3 &  &  & 5+4 &  & \\
I. Viemo\il{Viemo}   					& 5+3 & 4*2 &  &  & 10--1 & \\
J. Wara-\il{Wara}Natioro-\il{Natioro}Paleni   		& 5+3 & 4+4 &  & 5+4 &  & kawo? \\
\lspbottomrule
\end{tabularx}
\end{table}

In addition to the common patterns ‘8=5+3’ and ‘9=5+4’, alternative ones are attested for ‘eight’ and ‘nine’ (‘8=4 redupl.’ and ‘9=10--1’ respectively). 

\newpage 

\subsubsection{‘Ten’}%3.9.11.7.
\begin{table}
\caption{\label{tab:3:194}Stems for `10' in Gur}


\begin{tabularx}{\textwidth}{lQ@{}ll@{}lQ}
\lsptoprule

A. Bariba\il{Bariba} 				 	&  & wɔ-kuru &  &  & \\
B. Central:\\~~~~1. Northern\\~~~~~~~~A. Bwamu\il{Bwamu}& pílú/píru/ ˀɓúrúù &  &  &  & \\
~~~~~~~~B. Kurumfe\il{Kurumfe} 				& fɪ &  &  &  & \\
~~~~~~~~C. *Proto-Oti-Volta\il{Proto-Oti-Volta} 	& pi(k) &  &  &  & \\
~~~~Southern\\~~~~~~~~A. Dogoso-\il{Dogoso}Khe\il{Khe} 	& kpélé &  &  &  & \\
~~~~~~~~C. Gan-Dogose\il{Dogose}		 	&  & kpooɡo & n{\~{ʊ}}y - 5PL &  & ɡbùnè, kpélé, sí-\\
~~~~~~~~D. *Proto-Grusi\il{Proto-Grusi}		 	& fu/fi &  &  &  & \\
~~~~~~~~E. Kirma-\il{Kirma}Tyurama\il{Tyurama}  	&  &  & n{\'ũ}{\'{\~ɔ}}s{\`{\~ɔ}} &  & c{\'ĩ}ŋcíelùó\\
C. Kulango\il{Kulango} 				 	&  &  &  &  & nuunu\newline\mbox{(< *5 redupl.),} *ji/yi\\
D. Lobi-\il{Lobi}Dyan\il{Dyan}  		 	&  & ni-kpo & ny{\`{ʊ}}{\'{ɔ}}r &  & \\
E. Senufo 					 	&  &  &  & kɛ & \\
F. Teen\il{Teen}				   	& pɔrwɔ &  &  &  & \\
G. Tiefo\il{Tiefo}  				 	&  &  &  & k{\~{ɛ}}~ & támú\\
H. Tusia\il{Tusia} 				 	&  &  &  &  & ɡb{\~{a}}m/ *ɡb{\~{ɔ}}/ bw{\`{ɔ}}\\
I. Viemo\il{Viemo}   					&  & kwɔm{\~{u}} &  &  & \\
J. Wara-\il{Wara}Natioro-\il{Natioro}Paleni   		& p(w)ɔ/fɔ &  &  &  & k{\`ã}ːs{\'ã}?\\
\lspbottomrule
\end{tabularx}
\end{table}

This term exhibits a variety of isolated (and possibly non-primary) forms. The main form has a voiceless labial as its initial consonant. 


\newpage 
\subsubsection{‘Twenty’}%3.9.11.8.
\begin{table}
\caption{\label{tab:3:195}Stems and patterns for `20' in Gur}


\begin{tabularx}{\textwidth}{llXlXX}
\lsptoprule

   & `20' & `20' & `20' & `20' & `20' \\
\midrule 
A. Bariba\il{Bariba} 				 	 & &  &  &  & yɛndu\\
B. Central:\\~~~~1. Northern\\~~~~~~~~A. Bwamu\il{Bwamu} & & ɓóní/ ɓénle/ kēwēníì &  &  & \\
~~~~~~~~B. Kurumfe\il{Kurumfe} 				 &sofe (<10?) &  &  &  & \\
~~~~~~~~C. *Proto-Oti-Volta\il{Proto-Oti-Volta} 	 &10*2 &  &  &  & \\
~~~~Southern\\~~~~~~~~A. Dogoso-\il{Dogoso}Khe\il{Khe} 	 & & g{\`{ʊ}}ʊsì & cúkúrì &  & \\
~~~~~~~~C. Gan-Dogose\il{Dogose}		 	 & & ɡbeere & tʃúkúrì &  & \\
~~~~~~~~D. *Proto-Grusi\il{Proto-Grusi}		 	 &10*2? &  &  &  & \\
~~~~~~~~E. Kirma-\il{Kirma}Tyurama\il{Tyurama}  	 & & ɡu{\~{r}} &  &  & kómòrré\\
C. Kulango\il{Kulango} 				 	 & &  &  &  & yipì-/ dʒipi-\\
D. Lobi-\il{Lobi}Dyan\il{Dyan}  		 	 & & kpèle & ceeru &  & \\
E. Senufo 					 	 & & ɡbèɲ/ ɡbēy, &  & toko/ togo & fulo, nafa\\
F. Teen\il{Teen}				   	 & &  &  & toko & \\
G. Tiefo\il{Tiefo}  				 	 & &  &  &  & kp{\~{a}}\\
H. Tusia\il{Tusia} 				 	 & &  & túkúrí &  & *tiki\\
I. Viemo\il{Viemo}   					 & &  &  &  & fɛrɛyɔ\\
J. Wara-\il{Wara}Natioro-\il{Natioro}Paleni   		 & &  &  &  & w{\'ã}/nwõ, sɔ \\
\lspbottomrule
\end{tabularx}
\end{table}

In view of the great variety of forms and patterns attested for this term, the existence of the term for ‘twenty’ in Proto-Gur\il{Proto-Gur} is uncertain. 

\newpage 

\subsubsection{‘Hundred’}%3.9.11.9.
\begin{table}
\caption{\label{tab:3:196}Stems and patterns for `100' in Gur}


\begin{tabularx}{\textwidth}{lllXlQ}
\lsptoprule

A. Bariba\il{Bariba} 				 	&20*5 &  &  &  & \\
B. Central:\\~~~~1. Northern\\~~~~~~~~A. Bwamu\il{Bwamu}& &  &  &  & kʰ{\~{i}}minù\newline (< Mande keme)\\
~~~~~~~~B. Kurumfe\il{Kurumfe} 				& &  &  & bɛrʊ & \\
~~~~~~~~C. *Proto-Oti-Volta\il{Proto-Oti-Volta} 	& &  & kob, kook &  & \\
~~~~Southern\\~~~~~~~~A. Dogoso-\il{Dogoso}Khe\il{Khe} 	&20*5 &  &  &  & \\
~~~~~~~~C. Gan-Dogose\il{Dogose}		 	&20*5 &  &  &  & \\
~~~~~~~~D. *Proto-Grusi\il{Proto-Grusi}		 	&20*5? &  & kɔwa/kɔɔ? & bi? & \\
~~~~~~~~E. Kirma-\il{Kirma}Tyurama\il{Tyurama}  	&20*5 &  &  & ɡundi & \\
C. Kulango\il{Kulango} 				 	& &  &  &  & kɛm{\`{ɛ}}\newline (< Mande)\\
D. Lobi-\il{Lobi}Dyan\il{Dyan}  		 	& & tàmâ &  &  & \\
E. Senufo 					 	&20*5 &  &  &  & lafa\newline (< Kwa)\il{Kwa}\\
F. Teen\il{Teen}				   	&20*5 &  &  &  & \\
G. Tiefo\il{Tiefo}  				 	&20*5 &  &  &  & \\
H. Tusia\il{Tusia} 				 	&20*5 &  & kw{\v{ɛ}} &  & \\
I. Viemo\il{Viemo}   					& & t{\~{a}}mõ &  &  & \\
J. Wara-\il{Wara}Natioro-\il{Natioro}Paleni   		&20*5 &  &  &  & \\
\lspbottomrule
\end{tabularx}
\end{table}


  
\subsubsection{‘Thousand’}%3.9.11.10.
\begin{table}
\caption{\label{tab:3:197}Stems and patterns for `1000' in Gur}


\begin{tabularx}{\textwidth}{lQlQQ}
\lsptoprule

A. Bariba\il{Bariba} 				 	&  &  & f{\`{ɔ}}r{\`{ɔ}}tɔ? & \\
B. Central:\\~~~~1. Northern\\~~~~~~~~A. Bwamu\il{Bwamu}&  & 100*10 & muaseé & \\
~~~~~~~~B. Kurumfe\il{Kurumfe} 				&  &  &  & tʊsrɪ\newline \mbox{(< Moore)}\il{Moore}\\
~~~~~~~~C. *Proto-Oti-Volta\il{Proto-Oti-Volta} 	&  &  &  & \\
~~~~Southern\\~~~~~~~~A. Dogoso-\il{Dogoso}Khe\il{Khe} 	& kp{\'{ɛ}} &  &  & \\
~~~~~~~~C. Gan-Dogose\il{Dogose}		 	& kpíɛ\newline `a goat' &  &  & \\
~~~~~~~~D. *Proto-Grusi\il{Proto-Grusi}		 	&  &  & kpoŋ/ ɡboŋ & \\
~~~~~~~~E. Kirma-\il{Kirma}Tyurama\il{Tyurama}  	&  & 200*5, 800+200 &  & \\
C. Kulango\il{Kulango} 				 	&  &  &  & wulo\newline \mbox{(< Mande)}\\
D. Lobi-\il{Lobi}Dyan\il{Dyan}  		 	&  & 100*10 & ɡb{\`{ʊ}}lanɪ & \\
E. Senufo 					 	&  & 200*5 & gben-, bɔlɔ, pwoo, sakere & \\
F. Teen\il{Teen}				   	&  &  & danyɛ & \\
G. Tiefo\il{Tiefo}  				 	&  &  &  & waga\newline\mbox{(< Mande)}\\
H. Tusia\il{Tusia} 				 	& < píy `goat’, n{\'ã}ˤ\newline 'cow' &  &  & \\
I. Viemo\il{Viemo}   					& vie-? &  &  & \\
J. Wara-\il{Wara}Natioro-\il{Natioro}Paleni   		&  & 400*2+20 &  & \\
\lspbottomrule
\end{tabularx}
\end{table}

No evidence supports the reconstruction of the term for ‘thousand’ in this family.

\clearpage 
\section{Mande}%3.10.

 The intermediate step-by-step reconstructions available for the Mande languages in \citeauthor{Vydrinms}’s Mande Etymological Dictionary  and in \citealt{Vydrin2007}\footnote{I would like to thank V. Vydrin for his suggestions and comments on the preliminary draft of this chapter.}  has made treatment of the data easier.

 The genetic classification of Mande, outlined in the latter work, will serve as the basis for our analysis. This classification differs from the one suggested by Kastenholz and is accessible via \textit{Ethnologue} \citep{SimonsFenning2018}. According to V. Vydrin, 
 
 \begin{quote}
 {Its major innovations, in comparison with that of Kastenholz, are the following:} 

 \begin{itemize} 
\item {the Susu}\il{Susu}{–Jalonke}\il{Jalonke}{ group is put together with the Southwestern group, rather than with Kastenholz’s “Central Mande” (in fact, it is a return to the proposal of André \citealt{Prost1958});}

\item { Soninke}\il{Soninke}{–Bozo}\il{Bozo}{, Samogho and Bobo}\il{Bobo}{ are no longer considered as bran\-ches of the same genetic unit (Kastenholz’s “Northwestern Mande”), but rather as independent groups inside Western Mande;} 

\item { the Mokole group is put together with Vai}\il{Vai}{–Kono}\il{Kono}{, rather than with Manding;} 

\item { in the Southern Mande group, Mwan}\il{Mwan}{ is separated from Wan}\il{Wan}{ and put together with the Guro}\il{Guro}{–Yaure}\il{Yaure}{ subgroup;} 

\item {San}\il{San}{ (Samo}\il{Samo}{) is put together with Bisa}\il{Bisa}{, rather than with Busa}\il{Busa}{-Boko}\il{Boko}{.’} {(}\citealt{Vydrin2016}: 110).
 \end{itemize}
 
 \end{quote}


Let us note an important fact: the numeral system of Jowulu\il{Jowulu}  differs considerably in certain points both from other Samogho languages and from Mande languages in general. It is interesting to outline that in R. Kastenholz's classification (based on the method of shared innovations, rather than on lexicostatistics) Jowulu is given a special status, more precisely, the first split in his Northwestern Mande branch (Bozo\il{Bozo}-Soninke\il{Soninke} + Bobo\il{Bobo} + Samogo + Jowulu).

Our further analysis will be based on the evidence from twelve branches of Mande represented in \figref{schema:3:1}.

\begin{figure}
\begin{tikzpicture}
 
\node[minimum width=4cm, minimum height=.7cm, inner sep=0pt, draw, fill=lsLightBlue] (jowulu) {10. Jowulu};
\node[minimum width=4cm, minimum height=.7cm, inner sep=0pt, draw, fill=lsLightBlue] (samogo) [above=of jowulu,yshift=-1cm] {9. Samogo};
\node[minimum width=4cm, minimum height=.7cm, inner sep=0pt, draw, fill=lsLightBlue] (bobo) [above=of samogo,yshift=-1cm] {8. Bobo};
\node[minimum width=4cm, minimum height=.7cm, inner sep=0pt, draw, fill=lsLightBlue] (bozosoninke) [above=of bobo,yshift=-1cm] {7. Bozo-Soninke};
                     
                    
\node[minimum width=2cm, minimum height=.7cm, inner sep=0pt, draw, fill=lsYellow] (swm) [left=of jowulu,xshift=1cm]  {6. SWM};
\node[minimum width=2cm, minimum height=.7cm, inner sep=0pt, draw, fill=lsYellow] (susu) [above=of swm,yshift=-1cm] {5. Susu}; 
                                                                                
\node[minimum width=2.5cm, minimum height=.7cm, inner sep=0pt, draw, fill=lsSoftGreen] (vaikono) [left=of swm,xshift=1cm]  {4. Vai-Kono};
\node[minimum width=2.5cm, minimum height=.7cm, inner sep=0pt, draw, fill=lsSoftGreen] (mokole) [above=of vaikono,yshift=-1cm] {3. Mokole}; 
\node[minimum width=2.5cm, minimum height=.7cm, inner sep=0pt, draw, fill=lsSoftGreen] (jogojeri) [above=of mokole,yshift=-1cm] {2. Jogo-Jeri}; 
\node[minimum width=2.5cm, minimum height=.7cm, inner sep=0pt, draw, fill=lsSoftGreen] (manding) [above=of jogojeri,yshift=-1cm] {1. Manding}; 

\node[minimum width=3cm, minimum height=.7cm, inner sep=0pt, draw, fill=white!90!black] (southern) [right=of jowulu,xshift=-5mm]  {12. Southern};
\node[minimum width=3cm, minimum height=.7cm, inner sep=0pt, draw, fill=white!90!black] (eastern) [above=of southern,yshift=-1cm] {11. Eastern}; 

\end{tikzpicture}

% \begin{tabularx}{\textwidth}{lXXXXXX}
% \lsptoprule
% 
% {~} & {} & {\textbf{Western}} & {} &  & \textbf{South-Eastern} & \\
% \midrule
% {~} & {} & {} & {} &  &  & \\
% \hhline{~-~-~~~}
% ~ & 1. Manding &  & 7. Bozo-\il{Bozo}Soninke\il{Soninke} &  &  & \\
% \hhline{~-~-~~~}
% ~ & 2. Jogo-Jeri &  & 8. Bobo\il{Bobo} &  &  & \\
% \hhline{~---~-~}
% ~ & 3. Mokole & 5. Susu\il{Susu} & 9. Samogo & {} & {11. Eastern} & \\
% \hhline{~---~-~}
% ~ & 4. Vai-\il{Vai}Kono\il{Kono} & 6. SWM\il{SWM} & 10. Jowulu\il{Jowulu} & {} & {12. Southern} & \\
% \hhline{~---~-~}
% {~} & {} & {} & {} &  &  & \\
% \lspbottomrule
% \end{tabularx}
\caption{
% Scheme 3.1.
Mande languages}
\label{schema:3:1}
\end{figure}

\newpage 
\subsection{‘One’}%3.10.1.


\begin{table}
\caption{\label{tab:3:198}Mande stems for `1'}

\begin{tabularx}{\textwidth}{llQQQl}
\lsptoprule

Manding & *d{\'{ɔ}} & *kélen &  &  & \\
Jogo-Jeri & *do & *kɛlɛ (?) &  &  & díé(n)/dúlì\\
Mokole & *d{\'{ɔ}}nd{\`{ɔ}} & *kél{\textsubbar{e}} &  &  & \\
Vai-\il{Vai}Kono\il{Kono} & *d{\'{ɔ}}ndɔ & *N-kélen &  &  & \\
Susu\il{Susu} &  & *kédén & nde/{\`{n}}dá &  & \\
SWM\il{SWM} &  & *gìláaŋ & *tà &  & \\
Bozo-\il{Bozo}Soninke\il{Soninke} &  & kuɔn/ kɛnɛ/ k{\textsubbar{e}}/k{\textsubbar{o}}~ &  & s{\textsubbar{a}}n{\textsubbar{a}} & bane, fie\\
Bobo\il{Bobo} &  &  & tàlá/tèlé~ &  & \\
Dzuun\il{Dzuun} (Samogo) &  & *ké &  & *so/sɔʔi/ sw{\={\~{ɛ}}} & \\
Jowulu\il{Jowulu} &  &  & t{\~{e}}{\~{e}}na/ tenŋ &  & \\
SE-\il{SE}Eastern & *do & ɡ{\`{ɔ}}r{\'{ɔ}}/ ɡ{\^{o}}on? &  &  & \\
SE-\il{SE}Southern & *d{\d{ó}} &  &  &  & \\
\lspbottomrule
\end{tabularx}
\end{table}

Vydrin’s preliminary reconstructions, as well as isolated forms resulting from the analysis of the numerical terms, are marked with an asterisk [*].

The isoglosses for `one' suggest the existence of two alternative roots (\textit{*d\d{ó}} and \textit{*kelen}) attested in both major Mande groups. The latter root is distinguishable under the assumption that the forms with a voiced velar attested in the Eastern branch of the South-Eastern group (Matya Samo\il{Matya Samo} \textit{ɡ{\`{ɔ}}r{\'{ɔ}}}, Southern Samo\il{Samo} (Maka) \textit{ɡ{\^{o}}on}) are related to the \textbf{k}-forms found in Western Mande.

The next two roots, if related, may be suggestive with regard to the classification of Western Mande (otherwise, they probably represent similar unrelated forms). It should be noted that the root \textit{{\`{n}}dá} (Susu\il{Susu} \textit{nde} ‘one, certain’, \textit{ndende} ‘anybody, whoever; nobody’, Jalonke\il{Jalonke} \textit{{\`{n}}dá} ‘certain’) attested, according to Vydrin, in Susu-Jalonke may be related to \textit{*d\d{o}}. The determiner \textit{*d\d{ó}}, which can be reconstructed at the Proto-Mande\il{Proto-Mande} level, goes back to the root \textit{*do}. 

The rightmost column of the table embraces the isolated forms.


\subsection{‘Two’} %3.10.2.
\begin{table}
\caption{\label{tab:3:199}Mande stems for `2'}


\begin{tabularx}{\textwidth}{XX}
\lsptoprule

\textbf{Manding} & *fìlá\\
\textbf{Jogo-Jeri} & *fàlá\\
\textbf{Mokole} & *fìla\\
\textbf{Vai-}\il{Vai}\textbf{Kono}\il{Kono} & *fèLá\\
Susu\il{Susu} & *fìdí{\'{n}}\\
SWM\il{SWM} & *fèelé\\
Bozo-\il{Bozo}Soninke\il{Soninke} & p{\`{\~e}}ːndé, fíllò\\
Bobo\il{Bobo} & pálà\\
Dzuun\il{Dzuun} (Samogo) & fíː(kí)\\
Jowulu\il{Jowulu} & fúúli\\
SE-\il{SE}Eastern & *pela\\
SE-\il{SE}Southern & *pìì-lāŋ\\
\lspbottomrule
\end{tabularx}
\end{table}

A common root for ‘two’ that may be tentatively recorded as \textit{*pila} \textit{/} \textit{fila} is attested in all Mande branches. Its precise phonetic reconstruction is beyond the scope of our investigation. The reader can refer to the works of specialists in the historical phonetics of Mande. A reference designation that will enable us to compare this root to the evidence of the other NC families is sufficient for our reconstruction purposes.


\subsection{‘Three’}%3.10.3.
\begin{table}
\caption{\label{tab:3:200}Mande stems for `3'}


\begin{tabularx}{.66\textwidth}{lXl}
\lsptoprule

Manding & sàbá & \\
Jogo-Jeri & sèɡbá/siɡbù & \\
Mokole & sàwa/saba & \\
Vai-\il{Vai}Kono\il{Kono} & sàkpá/sagba/sáwa & \\
Susu\il{Susu} & sàxán/sàqáŋ/sawa & \\
SWM\il{SWM} & sàwá/sāaɓā & \\
Bozo-\il{Bozo}Soninke\il{Soninke} & síkkò, sike & \\
Bobo\il{Bobo} & sàà (?) & \\
Dzuun\il{Dzuun} (Samogo) & ʒiʔi/ʒìːɡī~{\'{ }}/ʃw{\`{ɛ}}/ɣei & \\
Jowulu\il{Jowulu} & bʒei < *jɔnŋ/i? & \\
SE-\il{SE}Eastern & sɔɔ/c{\'{ɔ}}w? & ʔààk{\~{ɔ}}\\
SE-\il{SE}Southern &  & *yààká\\
\lspbottomrule
\end{tabularx}
\end{table}

The common root \textit{*sakpa/} \textit{sagba/} \textit{sawa} is represented in all Western branches. The relationship between some of the forms attested in the Eastern group (Southern Samo\il{Samo} (Maka) \textit{s{\={ɔ}}{\={ɔ}}}, Matya Samo\il{Matya Samo} \textit{tjɔwɔ}) remains uncertain. The Jowulu\il{Jowulu} form is especially peculiar. It should be noted that the forms of some numerical terms differ significantly depending on the source. Our study is based on four Jowulu sources that provide the following evidence\footnote{\citet{Hochstetler1996,DjillaEtAl2004,Carlson1993,Prost1958}.} (\tabref{tab:3:201}).

\begin{table}[t]
\caption{\label{tab:3:201}Jowulu\il{Jowulu} numerals}

\fittable{
\begin{tabular}{llllll}
\lsptoprule

Source & `1' & `2' & `3' & `4' & `5' \\
\midrule
\citet{Hochstetler1996} 	& t{\~{e}}{\~{e}}na 				& fuuli 					& \textbf{b}ʒei, *dʒ{\~{ɔ}} 					& \textbf{p}ʃɪrɛ{\ᶦ} 					& t{\~{a}}{\~{a}} 							\\
\citet{DjillaEtAl2004} 	& tenŋ 						& fúúli 				& byàŋ, *j{\`{ɔ}}n 						& pyiiraŋ 						& táánŋ 								\\
\citet{Carlson1993} 	& t{\textsubtilde{è}}{\textsubtilde{è}}nì 	& fu{\textprimstress}u{\textprimstress}lī & byā\i᷇, *j{\textsubtilde{\={ɔ}}}{\textsubtilde{\={ɔ}}} 	& pi{\textprimstress}i{\textprimstress}rēī 	& t{\textsubbar{a}}{\textprimstress}{\textsubbar{a}}{\textprimstress} 	\\
\citet{Prost1958} 	& t{\^{e}}na 					& fole 						& dyue, *dy{\^{o}} 						& piœe 							& tâ 								\\
\tablevspace
Source                  & `6' & `7' & `8' & `9' & `10' \\
\midrule
\citet{Hochstetler1996} 	& t{\~{a}}m{\~{a}}nɪ 	& dʒ{\~{ɔ}}m-pʊn 	& ful-pʊn 	& t{\~{e}}m-pʊn 	& \textbf{b}ʒ{\~{i}}{\~{i}}\\
\citet{DjillaEtAl2004} 	& táán-mání 	& j{\`{ɔ}}n-pɔnni 	& fuuli-pɔnni 	& ten-pɔnni 	& byìnŋ\\
\citet{Carlson1993} 	& t{\textsubbar{a}}{\textprimstress}{\textsubbar{a}}{\textprimstress}-mānī 	& j{\textsubtilde{\={ɔ}}}{\textsubtilde{\={ɔ}}}-po{\textprimstress}nì 	& fu{\textprimstress}l-po{\textprimstress}nì 	& t{\textsubtilde{è}}{\textsubtilde{è}}-po{\textprimstress}nì 	& by{\textsubtilde{ì}} \\
\citet{Prost1958} 	& ton-te 	& dy{\^{o}}mp{\^{o}}n{\^{o}} 	& filep{\^{o}}n{\^{o}} 	& t{\^{e}}p{\^{o}}n{\^{o}} 	& b{\^{i}}\\

\lspbottomrule
\end{tabular}
}
\end{table}

The terms for ‘seven’, ‘eight’ and ‘nine’ follow the pattern ‘3,2,1+‘to lose’’ respectively (cf. their inaccurate interpretation in Hochstetler, see \sectref{sec:3.10.9}), hence the reconstruction of the term for ‘three’ with the initial palatal (*\textit{j{\`{ɔ}}n}). The forms quoted in Jowulu\il{Jowulu} for ‘three’, ‘four’, and ‘ten’ are uncommon. If we were dealing with a language with a noun class system, we would have to conclude that a noun class marker (\textsc{cl}19?) with two allomorphs (\textbf{p-} and \textbf{b-} before voiced and voiceless respectively) is traceable in the pertinent forms. However, we are dealing with a language that undoubtedly belongs to Mande, so no class-related morphemes can be involved. This leaves the presence of the initial labial in the term for ‘three’ unexplained. A borrowing from Gur or Kru cannot be assumed since these languages lack the comparable forms. The only plausible solution is the alignment of ‘three’ and ‘four’ by analogy with ‘ten’ where it must have been originally present.

A special term for ‘three’ appears in South-Eastern. In Eastern it can be reconstructed as \textit{*ʔààk{\~{ɔ}}} or possibly \textit{**ʔàà-}(\textit{k{\~{ɔ}}}), cf. Bisa\il{Bisa} \textit{kak{\'{ʊ}}}, Boko\il{Boko} \textit{ʔàà{\~{ɔ}}} (in \citealt{Koelle1963} \textit{ááɣ{\textsubbar{o}}}), Bokobaru\il{Bokobaru} (Zogb{\~{e}}) \textit{ʔààɡ{\~{ɔ}}}, Busa\il{Busa} \textit{ʔààk{\~{ɔ}}}, Maya Samo\il{Maya Samo} \textit{kàakú}, Kyanga\il{Kyanga} \textit{ˀāàː}, and Shanga\il{Shanga} \textit{ʔà}. The latter reconstruction is supported by the fact that the terms for ‘three’ and ‘four’ share the ultima, cf. the data are presented in \tabref{tab:3:202}.

\begin{table}
\caption{\label{tab:3:202}Final morphemes in the Boko\il{Boko}-Busa\il{Busa} numerals}


\begin{tabularx}{\textwidth}{XXlXl} 
\lsptoprule
& Boko\il{Boko} & Boko\il{Boko} \citep{Koelle1963} & Bokobaru\il{Bokobaru} & Busa\il{Busa}\\
\midrule
‘3’ & ʔàà-{\~{ɔ}} & áá-ɣ{\textsubbar{o}} & ʔàà-ɡ{\~{ɔ}} & ʔàà-k{\~{ɔ}}\\
‘4’ & síí-{\~{ɔ}} & síí-ɣ{\textsubbar{o}} & síí-ɡ{\~{ɔ}} & ʃíí-k{\~{ɔ}}\\
\lspbottomrule
\end{tabularx}
\end{table}

It should be noted that in these languages, the syllable in question is also present in the terms for ‘eight’ that are built according to the pattern ‘5+3’ (cf. e.g. Bobo\il{Bobo} Karu \textit{s{\'{ɔ}}r-ààɡ{\~{ɔ}}}). Here we may be dealing with alignment by analogy, possibly with an additional final morpheme of uncertain meaning. It should be stressed that the ultima in ‘three’ and ‘four’ is never the same in the Eastern sub-group of the South-Eastern languages, whereas the medial velar is only attested in ‘three’ but not in ‘four’. Assuming that the forms of the two Eastern branches are related, the term for ‘three’ can be reconstructed as \textit{*ʔààk{\~{ɔ}}/yààká}, whereas the term for ‘four’ may be interpreted as resulting from the alignment by analogy with the forms of ‘three’ attested in the Eastern branch of South-Eastern Mande. The evidence in favor of its etymological connection with \textit{*sakpa} is inconclusive.


\subsection{‘Four’} %3.10.4.
\begin{table}
\caption{\label{tab:3:203}Mande stems for `4'}


\begin{tabularx}{\textwidth}{XXl}
\lsptoprule

Manding & *náani & \\
Jogo-Jeri & náani & \\
Mokole & náani & \\
Vai-\il{Vai}Kono\il{Kono} & náánì & \\
Susu\il{Susu} & náání & \\
SWM\il{SWM} & *náánì & \\
Bozo-\il{Bozo}Soninke\il{Soninke} & na:na/nàt{\'ã}/nà:rá/naxat- & \\
Bobo\il{Bobo} & náà/nì{\={\~{a}}} & \\
Dzuun\il{Dzuun} (Samogo) & n{\~{ɑ}}{\~{ɑ}}i/naai/nàːl{\'{\~e}} & \\
Jowulu\il{Jowulu} &  & pʃɪrɛ{\ᶦ} <ʃɪrɛ{\ᶦ}?\\
SE-\il{SE}Eastern &  & s{\`{ɪ}}/síík{\~{ɔ}} \\
SE-\il{SE}Southern &  & *yìì-s{\`{\textsubtilde{i}}}{\`{\textsubtilde{i}}}y{\textsubtilde{\'{a}}}: z{\`ĩ}{\'{\~ɛ}}/y{\^{i}}{\^{i}}-sī{\"ē}\\
\lspbottomrule
\end{tabularx}
\end{table}

An easily recognizable NC form (\textit{*náání/} \textit{n{\~{ɑ}}{\~{ɑ}}i}) can be reconstructed in Western Mande, whereas in South-Eastern Mande it is replaced with an innovation (\textit{*s{\`{\textsubtilde{i}}}{\`{\textsubtilde{i}}}y{\textsubtilde{\'{a}}}}). This innovation may also be attested in Jowulu\il{Jowulu}.


\subsection{‘Five’} %3.10.5.
\largerpage
\begin{table}
\caption{\label{tab:3:204}Mande stems for `5'}


\begin{tabularx}{\textwidth}{XXXll}
\lsptoprule

Manding & dúuru/loolu & *wo (cf. ‘7’) &  & \\
Jogo-Jeri & sóólò/sóolo &  &  & \\
Mokole & l{\'{ɔ}}ɔlu & *wo (cf. ‘7’) &  & \\
Vai-\il{Vai}Kono\il{Kono} & dúʔu/sóó(ʔ)ú &  &  & \\
Susu\il{Susu} & suuli/sùlù & *fò (cf. ‘7’) &  & \\
SWM\il{SWM} & d{\'{ɔ}}{\'{ɔ}}lú/l{\'{ɔ}}{\'{ɔ}}lu & *wɔ/ng{\`{ɔ}} &  & \\
Bozo-\il{Bozo}Soninke\il{Soninke} &  & k{\'{ɔ}}l{\'{ɔ}}h{\`{ɔ}}/káráɡò &  & \\
Bobo\il{Bobo} &  & k{\={ʊ}}/kóò &  & \\
Dzuun\il{Dzuun} (Samogo) &  &  &  & n\`{\~{u}}\\
Jowulu\il{Jowulu} &  &  & t{\~{a}}{\~{a}} & \\
SE-\il{SE}Eastern & *sodu: s{\'{ɔ}}{\'{ɔ}}ro/s{\'{ɔ}}{\`{ɔ}} &  &  & \\
SE-\il{SE}Southern & s{\'{ɔ}}{\'{ɔ}}ɗú/sólú &  &  & \\
\lspbottomrule
\end{tabularx}
\end{table}

There is a correspondence between \textbf{d-/ l-/ s-} within Western Mande, hence the Eastern forms with the initial \textbf{s-} should not necessarily be treated separately. A discussion of the exact phonetic reconstruction is better left to specialists in the field. For our purposes, it is sufficient to record that the Proto-Mande\il{Proto-Mande} root for ‘five’ is reconstructed as \textit{dúuru/} \textit{s{\'{ɔ}}{\'{ɔ}}ru}.

However, the root(s) \textit{*wo, *ko} are traceable in the compound numerical terms attested in Western Mande. They may be etymologically related to the lexical root meaning ‘hand’ (Vydrin, p.c.; cf. Proto-South-Mande \textit{*k{\`{ɔ}}} ‘hand’). The latter may be a NC root, cf. e.g. the term for ‘hand’ in Proto-Gbaya\il{Proto-Gbaya} (\textit{k{\textsubtilde{\'{ɔ}}}}), Dida\il{Dida} (Kru) (\textit{k{\={ɔ}}}) and in other languages.

The Jowulu\il{Jowulu} and Samogo forms are peculiar. As we hope to demonstrate in the next chapter, two alternative roots for ‘five’ can be reconstructed for NC, namely \textit{*tan/} \textit{ton} and \textit{*nu(n)}. Both roots are directly attested in these marginal groups. Is this enough to reconstruct the terms for ‘five’ traceable in NC for the Mande languages? We will return to this question in the last chapter of the book.


\subsection{‘Six’}%3.10.6.
\begin{table}
\caption{\label{tab:3:205}Mande stems and patterns for `6'}


\begin{tabularx}{\textwidth}{XXl}
\lsptoprule

Manding & w{\'{ɔ}}rɔ (5+1) & \\
Jogo-Jeri & m{\`{ɔ}}{\`{ɔ}}dó (5+1?)/mìːlù & \\
Mokole & w{\'{ɔ}}ɔrɛ/wɔyɔ (5+1) & \\
Vai-\il{Vai}Kono\il{Kono} & w{\'{ɔ}}ɔlɔ/wɔɔrɔ (5+1) & \\
Susu\il{Susu} & sénní (5+1?) & \\
SWM\il{SWM} & *5+1 & \\
Bozo-\il{Bozo}Soninke\il{Soninke} & goro? (5+1?) & t{\'ũ}mù/t{\~{u}}mi\\
Bobo\il{Bobo} & 5+1 & \\
Dzuun\il{Dzuun} (Samogo) &  & t(s){\`ũ}m{\={\~{ɛ}}}~{\'{ }}/tsìì\\
Jowulu\il{Jowulu} & 5+1 & \\
SE-\il{SE}Eastern & 5+1~ & \\
SE-\il{SE}Southern & 5+1, wá{\'{ŋ}}? & \\
\lspbottomrule
\end{tabularx}
\end{table}

The reconstruction of the Mande term for ‘six’ is problematic. The root \textit{t(s)um} is worth considering, since it is attested in both Bozo\il{Bozo}-Soninke\il{Soninke} and Samogo (the root found in Susu\il{Susu} is probably isolated). Its reconstruction at the Proto-Mande\il{Proto-Mande} level is, however, unlikely. The common pattern ‘6=5+1’ is attested in both major branches. The root \textit{wɔrɔ} is non-primary and eventually goes back to the aforementioned pattern (or to the pattern ‘6’=‘hand’+1’ to be precise). This hypothesis is supported by the forms of ‘seven’ as well.


\subsection{‘Seven’}%3.10.7.
\begin{table}
\caption{\label{tab:3:206}Mande stems and patterns for `7'}


\begin{tabularx}{.66\textwidth}{XXl}
\lsptoprule

Manding & x+2 & \\
Jogo-Jeri & ma+2 & \\
Mokole & x+2 & \\
Vai-\il{Vai}Kono\il{Kono} & 5+2 & \\
Susu\il{Susu} & 5+2 & \\
SWM\il{SWM} & 5+2 & \\
Bozo-\il{Bozo}Soninke\il{Soninke} & ɲérù/jeeni & \\
Bobo\il{Bobo} & 5+2 & \\
Dzuun\il{Dzuun} (Samogo) & ɲ{\`{\~ɛ}}ːn{\'ũ} (<5?)/ɲ{\`{ɛ}}{\`{ɛ}} & \\
Jowulu\il{Jowulu} &  & 3+‘to lose’\\
SE-\il{SE}Eastern & 5+2 & \\
SE-\il{SE}Southern & 5+2 & \\
\lspbottomrule
\end{tabularx}
\end{table}

A few remarks are in order before we turn to the discussion of the term for ‘seven’. In the majority of the Mande branches, the term represents a compound. Its second element goes back to the term for ‘two’, cf. e.g. Jula\il{Jula} \textit{wólonfìlà} ‘7’, \textit{fìlà} ‘2’.

The relationship between the terms for ‘six’ and ‘seven’ is based on alignment by analogy. This bond sometimes results in unification of the terms, so that sources may explain ‘seven’ as ‘6+1’ (despite the fact that ‘two’, not ‘one’, is manifestly present in ‘seven’). This interpretation has become recurrent for the Mokole languages. According to Phillip Logan,\footnote{\url{https://mpi-lingweb.shh.mpg.de/numeral/Kuranko.htm}} the Kuranko\il{Kuranko} evidence is as follows: \textit{wɔrɔnfila} (‘\textbf{6+1'}) (?! –\textit{K.P}.), \textit{wɔrɔ} ‘6’, \textit{fila} ‘2’, \textit{kelen} ‘1’. The same idea is applied to Lele\il{Lele} (cf. Marc  Gebhard:\footnote{\url{https://mpi-lingweb.shh.mpg.de/numeral/Lele-Mande.htm}} \textit{wɔrɔŋ} \textit{kela (‘6+1'}),\footnote{According to \citet{Vydrine2009}, the Lele\il{Lele} term for ‘seven’ is \textit{w{\'{ɔ}}rɔncɛla} (or \textit{wɔyɛnkela} in the Southern dialect, \url{https://mpi-lingweb.shh.mpg.de/numeral/Jowulu.htm}) 
\textit{núú} \textit{ɡ͡bɔy{\'{ɔ}}nɡo} `20' ('person finished', \url{https://mpi-lingweb.shh.mpg.de/numeral/Mende.htm})
}  \textit{wɔɔrɔ} ‘6’, \textit{fela} ‘2’, \textit{kelɛŋ}~‘1’) and Kakabe\il{Kakabe} (cf. Daria Mishchenko:\footnote{\url{https://mpi-lingweb.shh.mpg.de/numeral/Kakabe.htm}} \textit{w{\'{ɔ}}rɔwila} (\textbf{‘6+1'}), \textit{w{\'{ɔ}}ɔrɔ} ‘6’, \textit{fìla} ‘2’, \textit{kélen} ‘1’). Other scholars are more reserved, stating that \textbf{‘}Kono\il{Kono} has a decimal system with special construction for 7’.\footnote{Raimund Kastenholz, \url{https://mpi-lingweb.shh.mpg.de/numeral/Kono.htm}} It is, however, quite evident that the forms in question follow the pattern ‘5+2’ (or at least ‘X+2’ with X being an unidentified component).

It is not a mere coincidence that the interpretation outlined above is recurrent in the Mokole languages, where the forms of ‘six’ and ‘seven’ have become partially unified. In a number of languages from other groups that have etymologically related terms for ‘six’ and ‘seven’, these terms differ in their second consonant, cf. Bamana\il{Bamana} (Manding): \textit{wólonwula} ‘7’, \textit{w{\'{ɔ}}ɔrɔ} ‘6’.

In both groups of South-Eastern Mande the patterns ‘5+1’ and ‘5+2’ for ‘six’ and ‘seven’ respectively are still clearly recognizable (\tabref{tab:3:207}).

\begin{table}
\caption{\label{tab:3:207}Stems for `6' and `7' in South-Eastern Mande}


\begin{tabularx}{\textwidth}{lXXXXX} 
\lsptoprule
& `5' & `1' & `6' & `2' & `7' \\
\midrule 
SE:\il{SE} Eastern: Busa\il{Busa} & s{\'{ɔ}}o & do & sóo-do & pia & soo-pia\\
SE:\il{SE} Southern: Beng\il{Beng} & s{\'{ɔ}}-ŋ & do & s{\'{ɔ}}-do & pla-ŋ & s{\'{ɔ}}-pla\\
\lspbottomrule
\end{tabularx}
\end{table}

Taking all of this into consideration, the most likely evolution scenario for ‘six’ and ‘seven’ is as follows: 

\begin{itemize}
\item At the most archaic Proto-Mande\il{Proto-Mande} level the terms for ‘six’, ‘seven’ (and also ‘eight’ as we hope to demonstrate below) followed the pattern ‘X+1,2,3’ respectively. The X-element in this pattern possibly represented an archaic root with the meaning ‘hand’ (?) \textit{*ko}(\textit{*N-ko} > \textit{*go/wo}?).
\item Proto-Mande\il{Proto-Mande} developed the root \textit{*dúuru/} \textit{s{\'{ɔ}}{\'{ɔ}}ru} ‘5’.
\item This new root served as the basis for the South-Eastern Mande terms for ‘six’, ‘seven’ and ‘eight’.
\item In Western Mande this process is only attested in single languages, e.g. in Vai\il{Vai} (\textit{sóóʔú} ‘5’, \textit{s{\^{ɔ}}ŋ} \textit{l{\`{ɔ}}nd{\'{ɔ}}}~‘6’ (\textit{l{\`{ɔ}}nd{\'{ɔ}}} ‘1’), \textit{s{\^{ɔ}}ŋ} \textit{f{\`{ɛ}}ʔá}~‘7’ (\textit{f{\`{ɛ}}ʔá} ‘2’)) and Looma\il{Looma}~(\textit{dooluo} ‘5’, \textit{dɔzita}~‘6’, \textit{dɔfela} ‘7’, \textit{d{\'{ɔ}}sáwà}~‘8’).
\item The majority of the Western Mande languages retained the inherent forms for ‘six’ and ‘seven’, but their derivational motivation became unapparent (at least in the case of the first component, cf. Bandi\il{Bandi} \textit{nd{\`{ɔ}}{\'{ɔ}}lú(ŋ)} ‘5’, but \textit{nɡ{\`{ɔ}}hítáŋ} ‘6’ (\textit{hítàŋ} ‘1’) and \textit{ŋɡ{\`{ɔ}}félàŋ~}‘7’ (\textit{feelé} ‘2’) in contrast to Looma\il{Looma}).
\item This factor conditioned the partial unification of the terms for ‘six’ and ‘seven’ (by analogy) in some of the Western Mande languages (Mokole in particular).
\end{itemize}


\subsection{‘Eight’} %3.10.8.
\begin{table}
\caption{\label{tab:3:208}Mande stems and patterns for `8'}


\begin{tabularx}{\textwidth}{lQQQ}
\lsptoprule

Manding & séegi/séki/séyi &  & \\
Jogo-Jeri &  & ma+3 & \\
Mokole & s{\'{ɛ}}ɛn/saɛn/seyi &  & \\
Vai-\il{Vai}Kono\il{Kono} & séi/séin & 5+3 & \\
Susu\il{Susu} &  & 5+3 & \\
SWM\il{SWM} &  & wá-yákpá/ wɔ-yaagba/ ng{\`{ɔ}}sákbá(n) (5+3) & \\
Bozo-\il{Bozo}Soninke\il{Soninke} & segi-/seegu &  & \\
Bobo\il{Bobo} & s{\'{ɛ}}kì/tʃèkí &  & \\
Dzuun\il{Dzuun} (Samogo) &  &  & kàà, 4pl\\
Jowulu\il{Jowulu} &  &  & 2+‘to lose’\\
SE-\il{SE}Eastern &  & *5+3 & síɲe, kíwísí (<4)\\
SE-\il{SE}Southern &  & s{\H{a}}ȁ-gā/sálààkā/ s{\`{ɔ}}làá/sé-y{\={\textsubtilde{a}}} (5+3?) & \\
\lspbottomrule
\end{tabularx}
\end{table}

The pattern ‘8=4*2’/‘4PL’ commonly found in the majority of the families discussed above is barely attested in Mande. Meanwhile, the phonetic similarity between \textit{naai} ‘4’ {\textasciitilde} \textit{ŋaai(n)} ‘8’ (attested in the majority of the Samogo dialects) is hardly an accident.

The etymology of \textit{kàà} (not found outside Seenku\il{Seenku}) is unknown. 

The pattern ‘5+3’ is inconclusive, because it often developss independently in various languages. The interpretation of the main Mande root (tentatively described as \textit{seki/} \textit{segi}) is uncertain. On the one hand, its current forms suggest that this root can be reconstructed not only for Proto-Western Mande\il{Proto-Western Mande}, but for Proto-Mande\il{Proto-Mande} as well (cf. South-Eastern forms, in particular \textit{s{\H{a}}ȁgā} ‘8’). On the other hand, such reconstruction is hindered by at least two issues.

Firstly, the second velar in the South-Eastern Mande forms does not belong to the root. It is part of a reduced segment that goes back to the term for ‘three’ (cf. Tura\il{Tura} \textit{yȁká} ‘3’), whereas the first segment goes back to the term for ‘five’ (cf. Tura \textit{s{\H{o}}l{\H{u}}}, \textit{s{\H{o}}{\H{o}}l{\H{u}}}, \textit{s{\H{ʋ}}l{\H{ʋ}}}). The comparative analysis of the forms of ‘eight’ attested in the South-Eastern Mande languages (not quoted here in detail) strongly suggests that the South-Eastern Mande pattern for ‘eight’ is ‘5+3’. 

Secondly, this reconstruction is problematic from a typological point of view. As has been demonstrated above, our evidence prevents us from reconstructing primary roots for ‘six’ and ‘seven’. In terms of typology, a primary root for ‘eight’ would look highly unusual in this context. Such a root could be expected in those few numeral systems where ‘eight’ is a basic numeral (just like ‘twelve’ is a basic numeral in some of the Benue-Congo numeral systems described above, hence ‘100=12*8+4’). However, ‘eight’ has never been a basic unit of counting in Mande systems. The existence of a primary term for ‘forty’ (assuming that ‘forty’ is ‘8*5’) in some of the Mande languages could be interpreted as a hint at a special status of ‘eight’. However, this is not supported by any real evidence. 

This raises a question about the etymology of the Western Mande term for ‘eight’ (\textit{seki/} \textit{segi}). Its resemblance to the term for ‘three’ (especially in Bozo\il{Bozo} and Soninke\il{Soninke}, cf. Jenaama Bozo \textit{sík{\`{\~ɛ}}{\~{u}}} ‘3’ {\textasciitilde} \textit{sèkːí} ‘8’) may be suggestive here. Is there enough evidence to reject the hypothesis that ‘eight’ in the Proto-Western Mande\il{Proto-Western Mande} was built according to the pattern ‘8=plus 3’ (this would assume a counting reference to ‘five’)?

Despite the doubts expressed above, these forms are worth comparing to other forms of ‘eight’ attested in other NC families.

 
\subsection{‘Nine’} \label{sec:3.10.9}
\begin{table}
\caption{\label{tab:3:209}Mande stems and patterns for `9'}


\begin{tabularx}{\textwidth}{lQll}
\lsptoprule

Manding &  & k{\`{ɔ}}nɔntɔ (10--1?) & \\
Jogo-Jeri & ma+4 &  & \\
Mokole &  & k{\`{ɔ}}nɔndɔn (10--1?) & \\
Vai-\il{Vai}Kono\il{Kono} & 5+4 & k{\`{ɔ}}n{\'{ɔ}}ntɔn & \\
Susu\il{Susu} & 5+4 &  & \\
SWM\il{SWM} & 5+4 & 10--1 & \\
Bozo-\il{Bozo}Soninke\il{Soninke} &  &  & kàpːí/káfì/kabi\\
Bobo\il{Bobo} &  & k{\`{ʊ}}r{\`{ʊ}}n{\^{ɔ}}ŋ~ & \\
Dzuun\il{Dzuun} (Samogo) &  &  & kjèːr{\'{\~o}}/kleːlo/kùòm{\`{ɛ}}~\\
Jowulu\il{Jowulu} &  & 1+‘lose'~ & \\
SE-\il{SE}Eastern & 5+4 & 10--1 & \\
SE-\il{SE}Southern & 5+4 &  & \\
\lspbottomrule
\end{tabularx}
\end{table}

Two competitive patterns are distinguishable here (‘9=5+4’  {and ‘9=10--1’). In some of the branches (e.g. SWM}\il{SWM}, Vai\il{Vai}-Kono\il{Kono}) they are attested side-by-side.

At the same time, these patterns cannot be postulated for some of the languages without additional support. The pattern ‘9=10--1’ seems to be apparent in South-Eastern Mande and some of the SWM\il{SWM} languages only, cf. Boko\il{Boko} ‘9’: \textit{k{\`{\~ɛ}}okwi} (litː ‘tear away 1 (from) 10'), \textit{kwi} ‘10’ ; in Busa\il{Busa} ‘9’: \textit{k{\'{\~ɛ}}ndo/k{\'ĩ}{\textsubdot{n}}dokwi} (litː ‘tear away 1 (from) 10'), \textit{kwi} ‘10’, \textit{do} ‘1’; in Bandi\il{Bandi} (SWM) \textit{taá-vu} ‘9’, \textit{ìtá(ŋ)} ‘1’, \textit{púu} ‘10’. According to Robert Carlson \citep[30]{Carlson1993}, the terms from ‘seven’ to ‘nine’ in Jowulu\il{Jowulu} follow the pattern ‘1--3’ + ‘lose’ (\textit{f{\'{ɔ}}nì}), i.e. \textit{j\~{ɔ}\~{ɔ}-p\'{ɔ}nì} ‘7’,  
\textit{fúl-p{\'{ɔ}}nì} ‘8’, and 
\textit{t{\`{\~e}}{\`{\~e}}-p{\'{ɔ}}nì} ‘9’ (note that these terms are misinterpreted as 3+4, 2*4, 5+4\footnote{\url{https://mpi-lingweb.shh.mpg.de/numeral/Jowulu.htm}} by Lee Hochstetler).

The root \textit{kònonto/k{\`{ɔ}}nɔndɔ(n)} attested in Manding and Mokole is unclear and deserves discussion by specialists. On the contrary, the forms interpreted as the combination of ‘5+4’ in the table below seem to be quite transparent (\tabref{tab:3:210}).

\begin{table}
\caption{\label{tab:3:210}‘9 = 5+4' in Mande}


\begin{tabularx}{\textwidth}{lXXX}
\lsptoprule

Language & `9' & `5' & `4' \\
\midrule
Kyanga\il{Kyanga} & sòòʃí & s{\'{ɔ}}{\'{ɔ}}rū & ʃíí\\
Tura\il{Tura} & s{\'{ɔ}}{\`{ɨ}}s{\={ɛ}} & sólú & j{\`{ɨ}}s{\={ɛ}}\\
Susu\il{Susu} & sólómánáání~ & súlí & náání\\
Vai\il{Vai} & s{\^{ɔ}}ŋ náánì & sóó(ʔ)ú & náánì\\
Bobo\il{Bobo} Madare & kórón{\~{\v{ɔ}}} & kóò & náà\\
\lspbottomrule
\end{tabularx}
\end{table}

This section, however, is not unproblematic. The Jogo-Jeri non-primary terms for ‘6--9’ are formed by two components. The second (i.e. the terms for ‘one’, ‘two’, ‘three’ and ‘four’ respectively) is easily recognizable, whereas the etymology of the first (\textbf{ma}-) is unclear.


\subsection{‘Ten’} %3.10.10.
\begin{table}
\caption{\label{tab:3:211}Mande stems for `10'}


\begin{tabularx}{\textwidth}{lXXX}
\lsptoprule

Manding & *tán & *b{\^{i}} & \\
Jogo-Jeri & táà(n), ta &  & \\
Mokole & tán & *bí & \\
Vai-\il{Vai}Kono\il{Kono} & tâŋ &  & \\
Susu\il{Susu} & *tònɡó & fùú & \\
SWM\il{SWM} &  & *puu & \\
Bozo-\il{Bozo}Soninke\il{Soninke} & tan/téeŋ/cɛmi &  & \\
Bobo\il{Bobo} &  & f{\`{\~ʊ}} & m̥{\textsubdot{\'{m}}}\\
Dzuun\il{Dzuun} (Samogo) & t(s)e\~u/ce{\~{u}} &  & \\
Jowulu\il{Jowulu} &  &  & bʒ{\~{i}}{\~{i}}/byìnŋ\\
SE-\il{SE}Eastern &  & *fu/*vu (<* pu) & kwi/kuri, wókòì\\
SE-\il{SE}Southern &  & *bù & ɡ{\'{ɔ}}{\^{ɔ}}(dō),kɔ̏ŋ s{\'{ɔ}}jɔlú, \\
\lspbottomrule
\end{tabularx}
\end{table}

This term is especially interesting in light of the fact that the distribution of the isoglosses of ‘ten’ served as the basis for Maurice Delafosse’s early classification of the Mande languages including the \textit{Mande-tan} and \textit{Mande-fu} groups. These two roots are indeed the main Mande roots with this meaning. However, their distribution does not correspond to the two major branches of Mande as they are distinguished today. The root *\textit{tan} is indeed found in all groups of the Western branch except for Bobo\il{Bobo} and SWM\il{SWM}. However, the attestations of the root *\textit{pu}/\textit{fu} are not limited to South-Eastern and extend to a number of the Western branches such as Bobo, SWM, Susu\il{Susu} (and possibly Manding-Mokole, assuming that its reflex denotes tens in compound numerals). Isolated forms attested in South-Eastern and in peripheral Western languages are noteworthy.

The reconstruction of *\textit{pu}/\textit{fu} for Proto-Mande\il{Proto-Mande} and the interpretation of *\textit{tan} as the Proto-Western Mande\il{Proto-Western Mande} innovation seem well-founded. 

The etymology of *\textit{tan} is obscure. Its similarity to the locally attested root *\textit{tan} (cf. Soninke\il{Soninke} \textit{tàán} ‘foot, leg’; ‘wheel’; `time’ (when counting), Bozo\il{Bozo} Tieyaxo \textit{tɔn} ‘foot, leg’; ‘time’ (when counting), Bozo Hainyaxo \textit{t{\v{a}}}, Bozo Tiemacewe \textit{tawa}, Bozo Sorogama \textit{taba}) is likely a coincidence. Lexical roots with the meaning ‘foot’ are attested in NC numeral systems, usually as a basis for the non-compound terms for ‘fifteen’. The logic behind this development is simple: ‘ten’ is ‘two hands’, ‘twenty’ means ‘man’, i.e. ‘two hands and two feet’, hence ‘fifteen’ is ‘foot’. This seems to be the case for Boko\il{Boko} and Busa\il{Busa}, where a non-compound term for ‘fifteen’ (\textit{ɡ{\`{\~ɛ}}o/} \textit{ɡ{\`{\~ɛ}}ro}) is attested (hence ‘16=15+1’ in these languages). This root is etymologically related to ‘foot, leg’ in Duungoma\il{Duungoma} (Samogo) \textit{g{\~{e}}}, Dan\il{Dan} \textit{g{\textsubtilde{\^{ɛ}}}} , Mano\il{Mano} \textit{g{\textsubtilde{à}}} (it should be noted that within Mande a non-compound root for ‘fifteen’ is also attested in Ligbi\il{Ligbi}, cf. \textit{tíɡán} \textit{/} \textit{tiɡa} ‘15’, \textit{tíɡá-ló} ’16).

In addition, a similarity to the term for ‘one’ as attested in some of the languages must be a coincidence.

A hypothesis assuming a semantic shift *NC \textit{*tan} ‘5’ > Proto-Western-Mande \textit{tan} ‘10’ in parallel with the development of the Mande innovation *\textit{dúuru/} \textit{s{\'{ɔ}}{\'{ɔ}}ru} ‘five’ seems to be a better explanation. 

It bears reminding that the Bokobaru\il{Bokobaru} root \textit{kuri} ‘ten’  has a direct parallel in the isolated Bangime\il{Bangime} language (\textit{kúr{\'{ɛ}}}. Cf. also Boko\il{Boko} \textit{kúúli} recorded by Koelle).


\subsection{‘Twenty’}%3.10.11.
\begin{table}
\caption{\label{tab:3:212}Mande stems and patterns for `20'}


\begin{tabularx}{\textwidth}{lXXX}
\lsptoprule

Manding & <‘human’? &  & \\
Jogo-Jeri &  &  & ɟ{\={ɑ}}l{\={ɑ}}m{\textsubbar{\`{ɑ}}}/kèl{\`{ɛ}}mó\\
Mokole & <‘human’? &  & \\
Vai-\il{Vai}Kono\il{Kono} & <‘human’ & 10*2 & \\
Susu\il{Susu} & <‘human’ &  & \\
SWM\il{SWM} & <‘human’? & 10*2 & \\
Bozo-\il{Bozo}Soninke\il{Soninke} &  & 10*2 & \\
Bobo\il{Bobo} &  &  & kpòró, c{\'{ɔ}}r{\`{ɔ}}\\
Dzuun\il{Dzuun} (Samogo) & <‘human’ &  & fw{\'{ɛ}}\\
Jowulu\il{Jowulu} &  &  & k{\~{ɔ}}ne/kɔnninŋ\\
SE-\il{SE}Eastern &  & 10*2 & kèè-/ka\\
SE-\il{SE}Southern & <‘human’\footnotemark{} & 10*2 & yɔ\\
\lspbottomrule
\end{tabularx}
\end{table}

\footnotetext{Mende\il{Mende} \textit{núú} \textit{ɡ͡bɔy{\'{ɔ}}nɡo} `20' ('person finished').  \url{https://mpi-lingweb.shh.mpg.de/numeral/Mende.htm}} 
There is every reason to believe that the term for ‘twenty’ was based on the lexical root(s) meaning ‘human person’ at the Proto-Mande\il{Proto-Mande} level. The etymology of some of the isolated forms presented in the table should be sought with this in mind.


\subsection{‘Hundred’}%3.10.12.
\begin{table}
\caption{\label{tab:3:213}Mande stems and patterns for `100'}


\begin{tabularx}{\textwidth}{lXX}
\lsptoprule

Manding & *k{\`{ɛ}}m{\'{ɛ}} & \\
Jogo-Jeri & {\v{c}}{\v{ɛ}}mé/tʃímí & 20*5\\
Mokole & k{\`{ɛ}}mɛ & \\
Vai-\il{Vai}Kono\il{Kono} & kɛmɛ & \\
Susu\il{Susu} & k{\`{ɛ}}m{\'{ɛ}} & \\
SWM\il{SWM} & kɛmɛ(ŋ) & Kpelle:\il{Kpelle} <‘head’ (ŋwú{\`{ŋ}})\\
Bozo-\il{Bozo}Soninke\il{Soninke} & kame/keme & ‘islam'-60\\
Bobo\il{Bobo} &  & ɟ{\={ɔ}}(l{\`{ɪ}})/z{\`{ɔ}}(l{\'{ʊ}})\\
Dzuun\il{Dzuun} (Samogo) &  & 20*5, 80+20\\
Jowulu\il{Jowulu} &  & `rope'*5\\
SE-\il{SE}Eastern &  & *20*5\\
SE-\il{SE}Southern & *k{\`{ɛ}}m{\'{ɛ}}? & k{\`ẽ}{\={ŋ}}/k{\`ã}{\'{\~ɨ}}, la/lú\\
\lspbottomrule
\end{tabularx}
\end{table} 

The root \textit{kɛmɛ}, widely attested throughout Western Africa, is noteworthy. Its original semantics deserve a separate study: it is well known that in some languages this root can be used for ‘sixty’ or ‘eighty’ and not for ‘hundred’ (the archaic Bamana\il{Bamana} counting system: \textit{mànink{\`{ɛ}}mɛ} ‘60’, \textit{bámanank{\`{ɛ}}mɛ} \textit{/} \textit{k{\`{ɛ}}mɛ} ‘80’, \textit{k{\`{ɛ}}mɛ} \textit{ní} \textit{mùgan} ‘100’ (80+20)) (\citealt{VydrinPerekhvalskaya2015}: 360).


\subsection{‘Thousand’}%3.10.13.
\begin{table}
\caption{\label{tab:3:214}Mande stems and patterns for `1000'}


\begin{tabularx}{\textwidth}{lQlQ}
\lsptoprule

Manding & wúlú/wúli & wáa/wá/wà/wága & bà\\
Jogo-Jeri & búlí, wúlú\newline (< manding) &  & \\
Mokole &  & wàa/wá/ waɡa & \\
Vai-\il{Vai}Kono\il{Kono} & wúl &  & \\
Susu\il{Susu} & wúlù/wúlì &  & \\
SWM\il{SWM} & wùlù & wála/wáá & \\
Bozo-\il{Bozo}Soninke\il{Soninke} & gulu & waxa & (`islam')-muso, wúdz{\`ũ}nè\\
Bobo\il{Bobo} &  &  & \\
Dzuun\il{Dzuun} (Samogo) &  & ɡbàˀà, baa & bi ‘goat’, 800+200, <juula\\
Jowulu\il{Jowulu} &  & wa{\textprimstress}a{\textprimstress} & 800+200\\
SE-\il{SE}Eastern &  & wàà `200' & 200*5,v{\^{u}}{\^{u}}, `dúú, pàdí, pə, boro\\
SE-\il{SE}Southern & wúlù/wl{\H{u}}/\newline gbl{\H{ɯ}} (?) & *wágá: wáá & kpi , kɛn\\
\lspbottomrule
\end{tabularx}
\end{table}

The roots for ‘thousand’ attested in the Mande languages were borrowed from by the Western African languages. The original meaning of the Mande root \textit{wáa/} \textit{wága} may be ‘a basket of cola nuts’ (Perekhvalskaja, \citealt{VydrinPerekhvalskaya2015}: 361), cf. Bamana\il{Bamana} \textit{wágá} ‘panier à colas', Bobo\il{Bobo} \textit{wágá} ‘panier qui sert à transporter les colas ou wòlōwágá.’ 

   
\tabref{tab:3:215} gives an overview of Mande forms and patterns that will be used for further comparison to the evidence of other families (\tabref{tab:3:207}).

\begin{table}
\caption{\label{tab:3:215}Numerals in Proto-Mande\il{Proto-Mande}}
\begin{tabularx}{\textwidth}{llrQ}
\lsptoprule
1 & do, kelen & 7 & wɔ-X-fila (‘hand’+2?)\\
2 & pila/fila & 8 & seki/segi (<‘plus’-3?)\\
3 & sakpa/sagba/sawa, ʔààk{\~{ɔ}}/yààká? & 9 & kònonto/k{\`{ɔ}}nɔndɔ(n) (10--1, 5+4)\\
4 & náání/n{\~{ɑ}}{\~{ɑ}}i & 10 & pu/fu, tan (< *‘5’?)\\
5 & dúuru/s{\'{ɔ}}{\'{ɔ}}ru, wo? ko? **tan? (> ‘10’?), n\~ù? & 20 & <‘human’\\
6 & wɔrɔ (wɔ-rɔ? ‘hand’+1?), t(s)um? & 100 & kɛmɛ, 20*5\\
&  & 1000 & wulu, wa(g)a\\
\lspbottomrule
\end{tabularx}
\end{table}

\clearpage  
\section{Mel}%3.11.

A narrow definition of the Mel family is preferred here (in accordance with the classification of the Atlantic languages suggested in (\citealt{PozdniakovSegerer2017}). This family comprises two compact language groups, namely Northern (Temne\il{Temne}, Landuma\il{Landuma}, and all Baga languages except for Baga Fore\il{Baga Fore} and Baga Mboteni\il{Baga Mboteni}, namely Baga Koba\il{Baga Koba}, Baga Maduri\il{Baga Maduri}, Baga Sitemu\il{Baga Sitemu} and others) and Southern (Kisi\il{Kisi}, Sherbro\il{Sherbro}, Mani\il{Mani}, and Krim\il{Krim}). Sua\il{Sua}, Limba\il{Limba} and Gola\il{Gola} are not included within the Mel family and are viewed as isolated NC languages. The numeral systems of the two Mel groups comprised of the distant languages are treated separately below.


\subsection{Southern Mel} %3.11.1.
\begin{table}
\caption{\label{tab:3:216}South Mel numerals}


\begin{tabularx}{\textwidth}{lQQQQQ} 
\lsptoprule
& Kisi\il{Kisi} & Sherbro\il{Sherbro} & Bullom\il{Bullom} & Mani\il{Mani} (Bullom\il{Bullom} So)\il{So} & Krim\il{Krim}\\
\midrule 
1 & pìl{\`{ɛ}}{\'{ɛ}}/pilɔ, *pum? & bul & (nim)-bul & nìm-búl & yì-m{\textsubbar{o}} \\
2 & díŋ/C-íŋ/ C-{\'{ɔ}}ŋ, danyõ & tɪŋ & (nin)-tsiŋ/ tiŋ & nìn-c{\'{ə}}ŋ & yì-ɣɪn/ yèèn, dím\\
3 & ŋɡ-àá/y-àá & ræ & (niin)-ra & nìn-rá & yì-ɣa/gàà\\
4 & hì{\'{ɔ}}{\'{ɔ}}lú & hy{\textsubbar{o}}l & (nii)-hiɔɔl & nìŋ-ny{\'{ɔ}}l/ -ny{\'{ɔ}}l & yì-h{\v{i}}{\textsubbar{o}}n\\
5 & ŋù{\`{ɛ}}{\'{ɛ}}nú & mɛn & (nii)-man & nìmán\newline < niN-wán? & yì-wɛn/ n-wén\\
6 & 5+1 & 5+1 & 5+1 & 5+1 & 5+1\\
7 & 5+2 & 5+2 & 5+2 & 5+2 & 5+2\\
8 & 5+3 & 5+3 & 5+3 & 5+3 & 5+3\\
9 & 5+4 & 5+4 & 5+4 & 5+4 & 5+4\\
10 & t{\'{ɔ}} & wāŋ & waan & wàm & wāŋ/wàn\\
20 & bídìí(ŋ)/ bélé & ‘finished it is man’ & u-tɔɔŋ & ù-t{\`{ɔ}}ŋ & <‘person’\\
100 & < Mande\il{Susu} & < English &  & pé, < Susu\il{Susu} & \\
1000 & < Mande\il{Susu} & < English &  & < Susu\il{Susu} & \\
\lspbottomrule
\end{tabularx}
\end{table}

Noun class markers are usually positioned as suffixes in Kisi\il{Kisi}. However, the first numerical terms in this language have noun class prefixes, which makes the forms look inconsistent, cf. \textit{mùúŋ/} \textit{mì{\'{ɔ}}{\'{ɔ}}ŋ} \textit{/} \textit{ŋì{\'{ɔ}}{\'{ɔ}}ŋ} \textit{/} \textit{dìíŋ,} \textit{tì{\'{ɔ}}{\'{ɔ}}ŋ/là-tì{\'{ɔ}}{\'{ɔ}}ŋ} ’two’.

The terms for ‘hundred’ and ‘thousand’ were probably absent in Proto-South-Mel\il{Proto-South-Mel}. The similarity between Kisi\il{Kisi} \textit{t{\'{ɔ}}} ‘ten’ and Bullom\il{Bullom}-Mani\il{Mani} \textit{t{\`{ɔ}}ŋ} ‘twenty’ is noteworthy. ‘Twenty’ may follow the pattern ‘20=10\textsc{pl}’. If so, the original \textit{t{\`{ɔ}}ŋ} ‘ten’ should be viewed as an early borrowing from Western Mande (\textit{*tan} ‘10’). In this case, \textit{*wan} ‘10’ is an innovation (probably based on \textit{*wan/wen} ‘five’) that developed in South Mel after Kisi had separated.  The numeral system of modern Kisi exhibits no significant changes from the forms described by Koelle. It includes the form \textit{ŋam-puum} ‘6’ (Tucker Childs: \textit{ŋ{\v{ɔ}}ŋpúm}) that may have retained an archaic allomorph of ‘one’ (\textit{*pum}). The forms that will be used for further comparison are summed up in the table below (\tabref{tab:3:217}).

\begin{table}
\caption{\label{tab:3:217}Proto-South Mel numeral system (*)}


\begin{tabularx}{\textwidth}{lQrl}
\lsptoprule

1 & pìl{\`{ɛ}}/pilɔ (< *lɛ/lɔ?), bul, mɔ & 7 & 5+2\\
2 & tsiŋ/tiŋ & 8 & 5+3\\
3 & ra & 9 & 5+4\\
4 & hiɔl & 10 & 5\textsc{pl}? , < *West Mande? \\
5 & wan/wen & 20 & ‘person’, 10\textsc{pl}? \\
6 & 5+1 &  {100,} 1000 & absent\\
\lspbottomrule
\end{tabularx}
\end{table}
  
 \largerpage
\subsection{Northern Mel}%3.11.2.

A higher degree of homogeneity observable in these languages allows an instant reconstruction of their numeral system at the Proto-Nothern Mel\il{Proto-Nothern Mel} (\tabref{tab:3:218})
\begin{table}
\caption{\label{tab:3:218}Proto-Northern Mel numeral system (*)}
\begin{tabularx}{\textwidth}{ll@{}rl}
\lsptoprule
1 & -in & 7 & 5+2\\
2 & -rəŋ & 8 & 5+3\\
3 & -sas & 9 & 5+4\\
4 & -ŋkɨlɛ/-nlɛ & 10 & tɔfʌt (< tɔ-f-ɔt?)/pu , wɨtʃɔ? \\
5 & kə-ʈamaʈ (< * kə- ʈa ‘hand’?) & 20 & 10*2, kə-ɡba (< *bay/bey ‘chief’?)\\
6 & 5+1 & 100, 1000 & absent\\
\lspbottomrule
\end{tabularx}
\end{table}


\subsection{Proto-Mel}%3.11.3.
\il{Proto-Mel}The table below gives an overview of South Mel and North Mel forms (\tabref{tab:3:219}).

\begin{table}
\caption{\label{tab:3:219}Proto-Mel\il{Proto-Mel} numeral system (*)}


\begin{tabularx}{\textwidth}{lQrl}
\lsptoprule

1 & -in,< *lɛ/lɔ? & 7 & 5+2\\
2 & díŋ/tsiŋ/tiŋ, -rəŋ & 8 & 5+3\\
3 & *tat (> sas, ra) & 9 & 5+4\\
4 & hiɔl, -ŋkɨlɛ/<-nlɛ? & 10 & *pu/fu, 5\textsc{pl}? \\
5 & wan/wen, <‘hand’ & 20 & ‘person’, 10\textsc{pl}? \\
6 & 5+1 &  {100,} 1000 & absent\\
\lspbottomrule
\end{tabularx}

\end{table}
\section{Atlantic}\label{sec:3.12}%3.12.

Our step-by-step reconstruction of numeral systems in the Atlantic languages will be based on their classification suggested in \citealt{PozdniakovSegerer2017} (forthcoming) that distinguishes two main groups within the Atlantic family, namely Northern and Bak.


\subsection{Northern}\label{sec:3.12.1}%3.12.1.
The numeral systems of Northern Atlantic are treated below by sub-group.

\subsubsection{Cangin}\label{sec:3.12.1.1} %3.12.1.1.
\begin{table}
\caption{\label{tab:3:220}Proto-Cangin\il{Proto-Cangin} numerals (*)}


\begin{tabularx}{\textwidth}{lQrl}
\lsptoprule

1 & no & 7 & 5+2\\
2 & nak & 8 & 5+3\\
3 & haj/ʔéeyə & 9 & 5+4\\
4 & nik-iɭ < *nak-iɭ? & 10 & sabbo (< Fula)\il{Fula}, daːŋkah\\
5 & jat (<`hand'), ʔiːp & 20 & 10*2\\
6 & 5+1 & 100, 1000 & < Wolof?\il{Wolof} Fula?\il{Fula}\\
\lspbottomrule
\end{tabularx}
\end{table}

Some of the reconstructions presented above are not immediately apparent and are in need of additional commentary. A detailed discussion of each of them would be impossible here, so we will take the reconstruction suggested for ‘four’ (\textit{nik-iɭ}) as a sample. 

At first glance, the forms of ‘four’ attested in the Cangin languages have nothing in common. Two of the five Cangin languages have \textit{kinil} ‘four’ (Ndut\il{Ndut}-Palor\il{Palor}), whereas in the remaining three (Laal\il{Laal}a\il{Laala}, Noon\il{Noon}, and Safin\il{Safin}) \textit{nikis} is used in this function. The easiest solution to the problem would be to postulate two alternative forms for this group. However, as the evidence of comparative-historical phonetics suggests, the final \textbf{-l} in Ndut-Palor regularly corresponds to the final \textbf{-s} in Laala-Ndut-Safin (\tabref{tab:3:221}).

\begin{table}
\caption{\label{tab:3:221}l {\textasciitilde} s regular correspondence in Cangin}
\begin{tabularx}{.8\textwidth}{lXXXl}
\lsptoprule
\textbf{*-ɭ~} & ‘eye’ & ‘black’ & ‘road’ & ‘four’\\
\midrule
\textbf{Ndut}\il{Ndut} & ʔi\textbf{l} & suu\textbf{l} & wa\textbf{l} & kini\textbf{l}\\
\textbf{Palor}\il{Palor} & ʔi\textbf{l} & suu\textbf{l} & waa\textbf{l} & kini\textbf{l}, eni\textbf{l}\\
\textbf{Laala}\il{Laala}\il{Laal} & kɔ\textbf{s} & *susu\textbf{s} & wa\textbf{s} & niki\textbf{s}\\
\textbf{Noon}\il{Noon} & kwa\textbf{s} & *su\textbf{j}u\textbf{s} & wa\textbf{z} & ni\textbf{g}i\textbf{s}\\
\textbf{Safin}\il{Safin} & xa\textbf{s} & *suzu\textbf{s} & wa\textbf{s} & niki\textbf{s}\\
\lspbottomrule
\end{tabularx}
\end{table}

This fact alone urges closer examination of the forms quoted above. Further analysis shows that a fossilized noun class prefix \textbf{kV-} is present in some of the Palor\il{Palor} numerals, cf. \textbf{\textit{ka}}\textit{-nak} ‘deux’, \textbf{\textit{ke}}\textit{-jek} ‘trois’, \textbf{\textit{ki}}\textit{-nil} ‘quatre’, \textbf{\textit{k}}\textit{ip} ‘cinq. At the same time, the suffix -\textbf{Vs} is observable in the Noon\il{Noon} numerals, cf. \textit{jet-}\textbf{\textit{us} }‘five’. This evidence combined suggests the following development of the forms for ‘four’ (\tabref{tab:3:222}).

\begin{table}
\caption{\label{tab:3:222}Development of *\textit{nik-Vɭ} `4' in Cangin}
\begin{tabularx}{.8\textwidth}{lQrl}
\lsptoprule
Proto-Cangin\il{Proto-Cangin} & \textbf{*nik-Vɭ}  &  & \\
Laala\il{Laala}\il{Laal}/Noon\il{Noon}/Safin\il{Safin} & *nik-\textbf{Vs} &  & \textbf{nikis}\\
Ndut\il{Ndut}/Palor\il{Palor} & *\textbf{ki}-nik- Vɭ & \textbf{ki}-nik-il & \textbf{kinil}\\
\lspbottomrule
\end{tabularx}
\end{table}

\subsubsection{Nyun-Buy}\label{sec:3.12.1.2}%3.12.1.2.
\il{Nyun}Numerical terms are highly divergent within this sub-group, so it seems reasonable to treat them by branch (\tabref{tab:3:223}).

\begin{table}
\caption{\label{tab:3:223}Nyun\il{Nyun}-Buy numerals}
\begin{tabularx}{.8\textwidth}{lXl} 
\lsptoprule
& \textbf{Nyun}\il{Nyun} & \textbf{Buy (Kobiana,}\il{Kobiana} \textbf{Kasanga)}\il{Kasanga}\\
\midrule
1 & duk & tee(na), -anɔʔ\\
2 & nak & naŋ\\
3 & lal & taar\\
4 & ren(d)-ek & sannaŋ\\
5 & ci-lax (<`hand'), -məkila & ju-roog (<‘hand’?)\\
6 & 5+1 & 5+1\\
7 & 5+2 & 5+2\\
8 & 5+3 & 4+4\\
9 & 5+4 & 5+4\\
10 & ha-lax (<`hands') & 5PL, ntaaj{\~{a}} \\
20 & <`king' & < Mande, 10*2\\
100 & < Mande & < Mande, < French\il{French}\\
1000 & < Mande & ŋ-kontu < Portuguese\footnotemark{}\\
\lspbottomrule
\end{tabularx}
\end{table}

\footnotetext{ Guillaume Segerer (p.c.).}
The pattern ‘5’=’hand’ {\textasciitilde} ‘10’=‘hands’ is immediately apparent in Nyun\il{Nyun}. In the case of Buy, it can be accepted only under the assumption that the derived term for ‘five’ became phonetically distant from its source form, cf. Kasanga\il{Kasanga} \textit{ji-rek}, Kobiana\il{Kobiana} \textit{ji-hak} ‘hand’ (these forms must be related to Nyun \textit{ci-lax} ‘hand’). In any case, the Kasanga term \textit{ŋaa-rooɡ} follows the pattern ‘5PL’ that uses the same plural noun class as the one attested in {\textit{ŋa-rek}}{ ‘hands’.} 

{The forms for ‘ten’ attested in Joola}\il{Joola}{ Ejamat}\il{Ejamat}{ (Atlantic Bak)} {\textit{si-ntaaja}}{ is important for the diachronic interpretation of the Kobiana}\il{Kobiana}{ form} {\textit{ntaaj{\~{a}}}}{. The evidence suggests that the latter was probably directly borrowed from Joola}\footnote{According to Guillaume Segerer (p.c.) it is possible that the Ejamat\il{Ejamat} and Kobiana\il{Kobiana} forms both come from Manjak\il{Manjak}.} { (as was} -\textit{anɔʔ} ‘one’{).}

\subsubsection{Jaad-Biafada}\label{sec:3.12.1.3}%3.12.1.3.
\il{Jaad}\il{Biafada}
\begin{table}
\caption{\label{tab:3:224}Jaad\il{Jaad}-Biafada\il{Biafada} numerals}
\begin{tabularx}{\textwidth}{lQrl}
\lsptoprule
1 & nnəmma, *ɲi/nɛ/-inɛ, -kk{\~{a}} & 7 & 5+2, 6+1 (< Manjak)\il{Manjak}\\
2 & ke, ma-ae & 8 & 5+3, wose/wase\\
3 & jo/tʃaw & 9 & 5+4, leberebo\\
4 & n(n)e/nnihi & 10 & (p)po\\
5 & bəda (<‘hand') & 20 & 10*2\\
6 & 5+1, paaji (< Manjak)\il{Manjak}, ŋka-? & 100, 1000 & < Fula\il{Fula}\\
\lspbottomrule
\end{tabularx}
\end{table}

The forms of ‘one’ (\textit{ɲi/} \textit{nɛ}) are distinguishable in the compound numerals, cf. Jaad\il{Jaad} \textit{ŋka-inɛ} ‘6’ (‘5+1’), Biafada\il{Biafada} \textit{mpaaji} \textit{nyi} ‘7’ (‘6+1’), etc. The term for ‘five’ goes back to the lexical root meaning ‘hand’ (Biafada \textit{gə-bəda}, Jaad \textit{ko-bəda}).

\subsubsection{Tenda}\label{sec:3.12.1.4} %3.12.1.4.
The reconstruction of the Proto-Tenda\il{Proto-Tenda} numerals \citep{Pozdniakovms} is based on a comparative analysis of five Tenda languages: Basari\il{Basari}, Tanda\il{Tanda}, Bedik\il{Bedik}, Bapen\il{Bapen}, Konyagi\il{Konyagi}.

\begin{table}
\caption{\label{tab:3:225}Tenda numerals (*)}
\begin{tabularx}{\textwidth}{lQrl}
\lsptoprule
1 & {ɓɑt, ndi/riye/diye/iye, mbɔ} & 7 & 5+2\\
2 & ki & 8 & 5+3\\
3 & taʈ & 9 & 5+4\\
4 & {n{\`{æ}}x} & 10 & poxw\\
5 & mbəɗ (<`hand'), cɔ/njɔ & 20 & 10*2\\
6 & 5+1 & 100, 1000 & < Fula,\il{Fula} < Mande\\
\lspbottomrule
\end{tabularx}
\end{table}

The etymology of the Konyagi\il{Konyagi} term for ‘five (\textit{mbəɗ}) is based on the Jaad\il{Jaad}-Biafada\il{Biafada} evidence (these languages belong to the same sub-group as Tenda). 

\subsubsection{Fula-Sereer}\label{sec:3.12.1.5}\il{Fula}\il{Sereer}%3.12.1.5.

The numerical terms are highly divergent within this sub-group, so it seems reasonable to treat them by language (\tabref{tab:3:226}).

\begin{table}
\caption{\label{tab:3:226}Fula\il{Fula}-Sereer\il{Sereer} numerals}
\begin{tabularx}{\textwidth}{rXX} 
\lsptoprule
& \textbf{Fula}\il{Fula} & \textbf{Sereer}\il{Sereer}\\
\midrule 
1 & ɡoʔo & leŋ\\
2 & ɗiɗi & ɗik\\
3 & tati & tadik\\
4 & na(y)i & nahik\\
5 & jo(w)i\footnotemark{} & ɓe-tVk\\
6 & 5+1 & 5+1\\
7 & 5+2 & 5+2\\
8 & 5+3 & 4+4\\
9 & 5+4 & 5+4\\
10 & sapp-o & xarɓ-\\
20 & noogas/noogay & 10*2\\
100 & teeme- & < Fula\il{Fula}\\
1000 & < Mande, < Hausa\il{Hausa} & < Wolof?\il{Wolof} \\
\lspbottomrule
\end{tabularx}
\end{table}

\footnotetext{ Reviewing my first version of the book, Guillaume Segerer has advanced a new interesting etymology for Fula\il{Fula}: \textit{jow-i} ‘5’ = \textit{jun-ngo} <\textit{jow-ngo} ‘hand’. His hypothesis is quite possible.}
The fact that the Seerer terms covering the sequence from ‘two’ to ‘five’ have the same final segment is noteworthy. This could potentially be interpreted as a special morpheme or as a sub-morpheme that resulted from alignment by analogy. This discussion will be resumed below. Here it can only be stated that the morphological analysis of the Sereer\il{Sereer} term for ‘five’ (\textit{ɓe-tVk}) suggested in the table below is not immediately apparent and is thus debatable. Within this approach the element \textbf{ɓe-} is interpreted as a noun class prefix despite the fact that such a class is lacking in Sereer. Complex issues pertaining to the reconstruction of the term for ‘five’ will not be treated here. We shall only note that the plural animate class is reconstructable as \textbf{ɓe-} (class 2) in Proto-Fula-Sereer\il{Proto-Fula-Sereer}. 

\subsubsection{Wolof}\label{sec:3.12.1.6}%3.12.1.6.
\il{Wolof}
\begin{table}
\caption{\label{tab:3:227}Wolof\il{Wolof} numerals}
\begin{tabularx}{\textwidth}{lQrl}
\lsptoprule
1 & CL-enn & 7 & 5+2\\
2 & ñaar (< *CL-(X)aar) & 8 & 5+3\\
3 & ñ-ett (< *CL-(X)ett) & 9 & 5+4\\
4 & ñ-ent (< *CL-(X)en(i)t) & 10 & fukk\\
5 & jurom & 20 & < `person', 10*2\\
6 & 5+1 & 100, 1000 & < Fula,\il{Fula} < Mande\\
\lspbottomrule
\end{tabularx}
\end{table}

\newpage 
The Wolof term for ‘one’ exhibits the agreement in noun class, cf. \textit{k-enn} \textit{nit} ‘one person’, \textit{g-enn} \textit{garab} ‘one tree’, \textit{f-enn} ‘somewhere’, \textit{l-enn} ‘something’, etc. The same can be applied to the terms covering the sequence from ‘two’ to ‘four’ as demonstrated in \citealt{Pozdniakov2015}: 82. Nothing is known about the original radical of the root (assuming there was one) since it was replaced by a noun class consonant.

Speaking of ‘twenty’, it should be said that the form \textit{nit(t)} (apparently related to the lexical root \textit{nit} ‘person’) is widely used alongside the common Wolof\il{Wolof} pattern ‘10*2’. 

\subsubsection{Nalu-Baga Fore-Baga Mboteni}\label{sec:3.12.1.7}%3.12.1.7.
\il{Nalu}\il{Baga Fore}This sub-group is the most problematic within Northern Atlantic. Admittedly, the evidence pertaining to their classification as Northern is inconclusive. Moreover, the sub-group itself is highly heterogeneous, which affects its numeral systems as well. The pertinent data for each of these languages is provided below (\tabref{tab:3:228}).

\begin{table}
\caption{\label{tab:3:228}Numerals in Nalu\il{Nalu}, Baga Fore\il{Baga Fore} and Baga Mboteni\il{Baga Mboteni}}


\begin{tabularx}{\textwidth}{lQQl} 
\lsptoprule
& \textbf{Nalu}\il{Nalu} & \textbf{Baga Fore}\il{Baga Fore} & \textbf{Baga Mboteni}\il{Baga Mboteni}\\
\midrule 
1 & deːndɪk & ki-ben & mb{\'{ɔ}}\\
2 & bi-lɛ & ci-di & sà-l{\'{ɛ}}\\
3 & p-aat & ci-tɛt & n-d{\'{ɛ}}r\\
4 & bii-naaŋ & ci-nɛŋ & í-nà\\
5 & teedoŋ (< t{\'{ɛ}} ‘hand’?) & su-sɑ(n) & {ì-rìβ{\v{ɛ}}, *ba(x)?} \\
6 & 5+1 & 5+1 & 5+1\\
7 & 5+2 & 5+2 & 5+2\\
8 & 5+3 & 5+3 & 5+3\\
9 & 5+4 & 5+4 & 5+4\\
10 & 5*2, *a-lafaŋ? & ɛ-tɛ-lɛ (<‘hands’+2) & {t{\`{ɛ}}n (< ‘*hand’? )}\\
20 & 10*2 & 10*2 & 10*2\\
100 & m-laak & bɔ-1 & < Mande\\
1000 & m-ɲaak (100pl?) < Susu\il{Susu} & tɛnɡbeŋ-1 &?\\
\lspbottomrule
\end{tabularx}
\end{table}

\subsubsection{Proto-Atlantic North}\label{sec:3.12.1.8}%3.12.1.8.
\il{Proto-Atlantic}The prospects for the reconstruction of the Proto-North Atlantic numerals are discussed below.

\subsubsubsection{‘One’ (\tabref{tab:3:229})}

\begin{table}
\caption{\label{tab:3:229}Numerals for `1' in Northern Atlantic}
\begin{tabularx}{\textwidth}{lXXXX}
\lsptoprule
\textbf{Cangin} &  & no &  & \\
\textbf{Nyun}\il{Nyun} &  &  &  & duk\\
\textbf{Buy} &  & nɔʔ &  & tee(na)\\
\textbf{Jaad-}\il{Jaad}\textbf{Biafada}\il{Biafada} & *ɲi/nɛ &  &  & nnəmma,pakk{\~{a}}\\
\textbf{Tenda} & di(ye) &  & mbɔ & bat\\
\textbf{Fula-}\il{Fula}\textbf{Sereer}\il{Sereer} & leŋ &  &  & ɡoʔo \\
\textbf{Wolof}\il{Wolof} & -enn &  &  & \\
\textbf{Nalu}\il{Nalu} & deendik &  & mb{\'{ɔ}} & ki-ben\\
\lspbottomrule
\end{tabularx}
\end{table}

Isolated forms are quoted in the rightmost column. Direct parallels to some other forms are attested in Cangin – Buy (\textit{nɔʔ}) and Konyagi\il{Konyagi} – Baga Mboteni\il{Baga Mboteni} (\textit{mbɔ}). The most common root is \textit{*di(n)/} \textit{li(n)/} \textit{ye(n)/} \textit{ne(n)} (assuming that these forms are related).

\subsubsubsection{‘Two’, ‘Three’ and ‘Four’ (\tabref{tab:3:230})}

\begin{table}
\caption{\label{tab:3:230}Numerals for `2'-'4' in Northern Atlantic}
\small
\begin{tabularx}{\textwidth}{l llllllQl}
\lsptoprule
~ & `2' & `2' & `2' & `2' & `3' & `3' & `4' & `4' \\
\midrule
{Cangin} & nak &  &  &  &  & haj & nik-iɭ < nak-iɭ? & \\
{Nyun}\il{Nyun} & nak &  &  &  & lal &  &  & ren(d)-ek\\
{Buy} & naŋ &  &  &  & taar &  &  & sannaŋ\\
{Jaad-}\il{Jaad}\textbf{Biafada}\il{Biafada} &  &  & ke &  &  & jo/caw & n(n)e(hi) & \\
{Tenda} &  &  & ki &  & taʈ &  & n{\`{æ}}x & \\
{Fula-}\il{Fula}\textbf{Sereer}\il{Sereer} &  & ɗik &  &  & tati(k) &  & na(y)i(k) & \\
{Wolof}\il{Wolof} &  &  &  & X-aar & X-ett &  & X-en(i)t & \\
{Nalu}\il{Nalu} &  & di/lɛ &  &  & tɛt/tat &  & naaŋ/nɛŋ/na & \\
\lspbottomrule
\end{tabularx}
\end{table}

\newpage 
The forms of ‘two’ in Tenda-Jaad\il{Jaad}-Biafada\il{Biafada} can be explained as a shared innovation, since these two branches belong to the same sub-group. The forms quoted in the two leftmost columns could be related, but the pertinent evidence is inconclusive. The roots \textit{*nak} and \textit{*di(k)} are reserved for further comparison.


As in the majority of other NC branches, the terms for ‘three’ and ‘four’ (tentatively recorded as \textit{*taʈ} ‘3’ and \textit{*nak} ‘4’) are fairly consistent in North Atlantic. Thus it appears that the terms for ‘two’ and ‘four’ are the same (or phonetically similar) across the languages of this branch. Cangin is the only language that does not comply with the additional distribution, because in the case of Cangin both terms are reconstructed as *\textit{nak}. Interestingly, the form of ‘four’ bears a suffix, hence it could potentially be explained as a derivative of ‘two’. At the same time, the root \textit{nak} ‘four’ is reminiscent of one of the most persistent NC roots with this meaning. 

In Jaad\il{Jaad}-Biafada\il{Biafada} we find the root \textit{*jow/caw} ‘3’. This is undoubtedly an innovation in the group which is represented by a remarkable isogloss. This is therefore an argument in favour of interpreting this group as part of the northern branch of the Atlantic family: Biafada -\textit{njo} \textit{/} \textit{bíí-co/}  \textit{bií-yo} ‘3’, Jaad \textit{ma-cao/}  \textit{ma-caw/}  \textit{má-cɔu}  ‘3’. It is possible that we are dealing with an ancient borrowing of Proto-Jaad-Biafada\il{Proto-Jaad-Biafada} from Mande (from \textit{saba} ‘three’).

In theory, it is possible that forms attested in the Cangin languages (\textit{ka-hay} \textit{/} \textit{*} \textit{ʔe-jɛʔ}), also originated from the Mande form (likely weakened to \textit{*habi} \textit{/} \textit{hawi}).

In this case, we find either reflexes of the Proto-NC\il{Proto-NC} form \textit{*tath} or borrowings (taking into account very ancient forms) – from the Mande languages in numerous Northern Atlantic languages.

\subsubsubsection{‘Four’}

The root \textit{*na}\textit{(h}\textit{)i}\textit{-}\textit{k} can be securely reconstructed for Proto-Northern Atlantic\il{Proto-Northern Atlantic}. As has been demonstrated above, the initial \textbf{ñ-} of the Wolof\il{Wolof} term is a reflex of a noun class prefix that replaced the initial radical of the root. The final -t in the Wolof term probably resulted from the alignment by analogy with the term for ‘three’ that ends in -t, cf.  *\textit{ñ-eenk}~? → \textit{ñ-eent} ‘4’ by analogy with \textit{ñ-ett} ‘3’.


\subsubsubsection{‘Five’ (\tabref{tab:3:231}) and the terms from ‘six’ to ‘nine’ }

\begin{table}
\caption{\label{tab:3:231}Numerals for `5' in Northern Atlantic}


\begin{tabularx}{\textwidth}{lllllQ}
\lsptoprule

{Cangin} & jat (<`hand') &  &  & ʔiːp & \\
{Nyun}\il{Nyun} & ci-lax (<`hand') &  &  &  & -məkila\\
{Buy}  &  &  & ju-roog &  & \\
{Jaad-}\il{Jaad}{Biafada}\il{Biafada} & bəda ('hand') &  &  &  & \\
{Tenda} & mbəɗ (<`hand'?) & co/njo &  &  & \\
{Fula-}\il{Fula}{Sereer}\il{Sereer} &  & jo(w)i & * ɓe-tVk &  & \\
{Wolof}\il{Wolof} &  & jurom &  &  & \\
{Nalu}\il{Nalu} & teedoŋ/*tee (‘hand'?) &  &  & ribə(l) & su-sa(n), *ba(x)? \\
\lspbottomrule
\end{tabularx}
\end{table}

The North Atlantic languages are characterized by the term for ‘five’ being systematically derived from the lexical root meaning ‘hand’. Interestingly, this development seems to post-date the replacement of the original root for ‘hand’ by an innovation in the majority of the branches. At least four independent formations of this kind are attested within eight branches (cf. the evidence quoted in the leftmost column of the table). Both Tenda and Jaad\il{Jaad}-Biafada\il{Biafada} terms for ‘five’ are of common ancestry: they seem to have developed from the root *\textit{ɓəda} at the Proto-Jaad-Biafada\il{Proto-Jaad-Biafada} level, since both languages belong to the same sub-group. This probably indicates that the pattern based on the term for ‘hand’ was used in the languages that belong to the Northern group at the proto-level (possibly as an alternative~to the inherent NC root for ‘five’). In view of this, the formal alterations of ‘five’ are easily explained as those automatically caused by the replacement of the inherent term for ‘hand’ by an innovation. As we hope to demonstrate in the next chapter, the derivational pattern ‘hand’ > ‘five’ is surprisingly rare in the NC languages. It is barely attested, for example, in Benue-Congo, thus being characteristic of the North Atlantic languages (and the Atlantic languages on the whole, see below). 

In view of this, the reflexes of the inherent NC root for ‘five’ could have been preserved in only a minority of North Atlantic branches. The roots \textit{*jo/} \textit{co}, \textit{*tVk/} \textit{rog} and \textit{*rib/} \textit{ʔiːp} unrelated to the term for ‘hand’ deserve special attention within this context.

The pattern ‘5+’ (‘hand’+) can be securely reconstructed for the terms covering the sequence from ‘six’ to ‘nine’. The uncommon pattern ‘7=6+1’ attested in Biafada\il{Biafada} was borrowed from one of the Manjak\il{Manjak} languages (Atlantic Bak), as was the derived term for ‘six’ (\textit{mpaaji}).

\newpage 
\subsubsubsection{‘Ten’ and ‘Twenty’ (\tabref{tab:3:232})}

\begin{table}
\caption{\label{tab:3:232}Numerals and patterns for `10' and `20' in Northern Atlantic}


\begin{tabularx}{\textwidth}{l llQllX}
\lsptoprule

~ & `10' & `10' & `10' & `20' & `20' & `20' \\
\midrule
{Cangin} &  &  & < Fula,\il{Fula} daːŋkah & 10*2 &  & \\
{Nyun}\il{Nyun} &  & <`hands' &  &  & <`king' & \\
{Buy} &  & 5PL & ntaaj{\~{a}} & 10*2 &  & < Mande\\
{Jaad-}\il{Jaad}{Biafada}\il{Biafada} & (p)po &  &  & 10*2 &  & \\
{Tenda} & pəxw &  &  & 10*2 &  & lapɛm\\
{Fula-}\il{Fula}{Sereer}\il{Sereer} &  &  & sapp-o, xarɓ- & 10*2 &  & noogas/ noogay\\
{Wolof}\il{Wolof} & fukk &  &  & 10*2 & `person' & \\
{Nalu}\il{Nalu} &  & 5*2 & *a-lafaŋ? & 10*2? &  & \\
\lspbottomrule
\end{tabularx}
\end{table}

With the evidence of the three branches, the reconstruction of the term for ‘ten’ (tentatively recorded as *\textit{pok}) seems secure. Its attestations are admittedly limited, apparently due to its replacement with derived terms based on ‘five’ (‘hand’). This reconstruction is also supported by the presence of the final velar: as we have seen, it is reconstructable in a number of other numerical terms at the proto-level. 

The pattern for ‘twenty’ is reconstructable as ‘20=10*2’. Particular derivates based on the typologically widely attested patterns (’20’ <‘person’, 20 <‘king’) seem to have formed independently.

\subsubsubsection{‘Hundred’ and ‘thousand’}

The evidence points to the absence of these terms in Proto-North Atlantic. Attested forms are borrowings from ‘influential’ languages such as Fula\il{Fula}, Wolof\il{Wolof}, Manding, Hausa\il{Hausa} (in the case of Niger Fulfulde\il{Fulfulde}). Interestingly, the terms in question are already borrowings in some of these source-languages.

\newpage 
\subsubsubsection{Proto-North Atlantic numeral system (\tabref{tab:3:233})}
\begin{table}
\caption{\label{tab:3:233}Proto-North Atlantic numeral system (*)}
\begin{tabularx}{\textwidth}{llSX}
\lsptoprule
1 & di(n)/li(n)/ye(n)/ne(n), mbɔ & 7 & 5+2\\
2 & di(k), nak & 8 & 5+3\\
3 & taʈ & 9 & 5+4\\
4 & nak & 10 & pok\\
5 & <‘hand’, jo, tVk/rog, rib/ʔiːp & 20 & 10*2\\
6 & 5+1 & 100, 1000 & absent\\
\lspbottomrule
\end{tabularx}
\end{table}

 
\subsection{Bak}\label{sec:3.12.2}%3.12.2.
\subsubsection{Joola languages}\label{sec:3.12.2.1}%3.12.2.1.
\il{Joola}Over a hundred sources covering the numeral systems of fifteen major Joola\il{Joola} dialects have been made available to us courtesy of Guillaume Segerer. His collection of evidence may be labeled a ‘dialect atlas’ of numerical terms. These terms often exhibit significant variations not only in their phonetics but in the inventory of lexical roots as well.\footnote{I wish to express my gratitude to G. Segerer for his assistance with regard to the dialectal attribution of sources.}  The name Joola pertains to a group of at least seven related languages (including Bayot\il{Bayot}). A study of their numeral systems may help set a clearer distinction between these languages. Moreover, it might shed some light on their (hitherto unclear) internal classification.

Numerical terms as attested in ten major Joola\il{Joola} languages are discussed below.

\subsubsubsection{‘One’ (\tabref{tab:3:234})}

\begin{table}
\caption{\label{tab:3:234}Joola\il{Joola} numerals for `1'}


\begin{tabularx}{\textwidth}{XXXXl}
\lsptoprule

Bliss\il{Bliss} & Kasa\il{Kasa} & Fogny\il{Fogny} & Keeraak\il{Keeraak} & Bayot\il{Bayot}\\
Banjal\il{Banjal} & Mlomp\il{Mlomp} & Karon\il{Karon} & Ejamat\il{Ejamat} & Kwaatay\il{Kwaatay}\\
\midrule 
-anɔʔ & -anor & -anor & -anor & \\
-anor & -anor & -anor & -anor & \\ 
~ & (akon) & əkon &  & (akon)\\ 
~ & (ta) &  &  & don\\
~ &  &  & yinka, (sia) & fɛnɛŋ\\
\lspbottomrule
\end{tabularx}
\end{table}

The main form is reconstructed as *-\textit{anor}, with the initial vowel forming a part of the root. The only languages where this root is not present are Bayot\il{Bayot} (\textit{don} ‘1’) and Kwaatay\il{Kwaatay} (\textit{fɛnɛŋ} ‘1’). The root \textit{əkon} with a vocalic opening (sporadically attested in Kasa\il{Kasa} and Bayot) is found in Fogny\il{Fogny} alongside *-\textit{anor}.

\subsubsubsection{‘Two’, ‘three’ and ‘four’ (\tabref{tab:3:235})}

\begin{table}
\caption{\label{tab:3:235}Joola\il{Joola} numerals for `2'-'4'}


\begin{tabularx}{\textwidth}{lQQQQ}
\lsptoprule

Bliss\il{Bliss} & Kasa\il{Kasa} & Fogny\il{Fogny} & Keeraak\il{Keeraak} & Bayot\il{Bayot}\\
Banjal\il{Banjal} & Mlomp\il{Mlomp} & Karon\il{Karon} & Ejamat\il{Ejamat} & Kwaatay\il{Kwaatay}\\
\midrule
{{‘2’}} & {} & {} & {} & {}\\
\midrule
si-lubəʔ & si-ɬubɤʔ & (liba) & sɩʼsubə & ʔi-ɾigəʔ/tɪɡɡa\\
si-rubə & sɩ-subəl & su-supək/ ɕi-ɕipəkʰ & si-lu:bɜʔ & sɩʼsubə\\
\tablevspace
{{‘2}{’}} & {} & {} & {} & {}\\
\midrule
~ & si-g{\"{ɑ}}b{\"{ɑ}}, (ku-mɛntɛn) & si-g{\"{ɑ}}b{\"{ɑ}}ʔ &  & \\
si-gabaʔ &  &  & si-g{\"{ɑ}}b{\"{ɑ}} & \\
\tablevspace
{{‘3}{’}} & {} & {} & {} & {}\\
\midrule
si-həəji & si-hɤ:ɟiʔ & si-feeɡiir/ si-fe:ɟiʔ & sɩ-hə:jɩ & i-fiigiʔ/ i-fəəʒi\\
gu-fɩ:gɩr/si-fɤɟiɾ & sɩ-hə:jɩl & si-həːciːl & si-həəji,\newline \mbox{(fu-fooateen)} & ki-hɤ:ɟiʔ\\
\tablevspace
{{‘4}{’}} & {} & {} & {} & {}\\
\midrule
si-b{\"{ɑ}}kir & si-b{\"{ɑ}}:kiɽ/si-b{\"{ɑ}}kiʔ & si-b{\"{ɑ}}kiɾ/ si-ba:ci:r & si-bacir & sɪ-bɐɣɪɾ \\
si-baagir & sɩ-bacɩl & ɕɪ-p{\"{ɑ}}kil/ si-ba:ci:l & si-b{\"{ɑ}}kir & ki-b{\"{ɑ}}kir\\
\lspbottomrule
\end{tabularx}
\end{table}

Two alternative roots for ‘two’ are attested in Joola\il{Joola}, namely \textit{*si-ɬubəʔ} and a relatively wide-spread \textit{*si-gabaʔ}.

The term for ‘three’ goes back to \textit{*si-feeɡir}, with its reflexes being attested in all dialects.

The term for ‘four’ is securely reconstructed as *\textit{si-bääkiɽ}.

\subsubsubsection{‘Five’ and ‘ten’ (\tabref{tab:3:236})}

\begin{table}
\caption{\label{tab:3:236}Joola\il{Joola} numerals for `5' and `10'}


\begin{tabularx}{\textwidth}{QQQQQ}
\lsptoprule
Bliss\il{Bliss} & Kasa\il{Kasa} & Fogny\il{Fogny} & Keeraak\il{Keeraak} & Bayot\il{Bayot}\\
Banjal\il{Banjal} & Mlomp\il{Mlomp} & Karon\il{Karon} & Ejamat\il{Ejamat} & Kwaatay\il{Kwaatay}\\
\midrule
{\textbf{‘5}\textbf{’}} & {} & {} & {} & {}\\
\midrule
hu-tɔk & hu-tɔkʰ & fu-tɔk/u-sɔk & hu-tɔk & o-to/ɔ-ɬɔ/ ɔ-rɔ\\
fu-tɔk &  & ɪ-ɕ{\"{ɑ}}kʰ/i-sak & fu-tɔk/ hu-ʂok & hu-tɔk\\
\tablevspace
{\textbf{‘5}\textbf{’}} & {} & {} & {} & {}\\
\midrule
~ & (naa-suan) &  &  & \\
~ & ŋaa-suwaŋ &  &  & \\
\tablevspace
{\textbf{‘5}\textbf{’}} & {} & {} & {} & {}\\
\midrule
~ &  & *fu-tam &  & \\
 *tən &  &  &  & \\
\tablevspace
{\textbf{‘10}\textbf{’}} & {} & {} & {} & {}\\
\midrule
ku-ŋɛn <`hands' & ku-ŋɛn <`hands' & ku-ŋɛn <`hands' & ku-ŋɛn <`hands' & \\
gu-ɲɛn <`hands' &  &  & ku-ŋɛn <`hands' & \\
\tablevspace
{\textbf{‘10}\textbf{’}} & {} & {} & {} & {}\\
\midrule
~ &  &  &  & gu-tie(pɔkɔ) `hands'\\
~ & sɛ-bɛɛs `hands' & ŋaa-suwan &  & su-moŋu/ su-ŋɔmu `hands'\\
\lspbottomrule
\end{tabularx}
\end{table}

The Banjal\il{Banjal} form *\textit{tən} (reconstructed on the basis of the compound numerical terms) and the (related?) Fogny\il{Fogny} form \textit{fu-tam} attested in a source dating to the seventeenth century \citep{dAvezac1845} are of special interest.

The Mlomp\il{Mlomp} form of ‘five’ (sporadically attested in Kasa\il{Kasa} as well) is identical to the Karon\il{Karon} form for ‘ten’ (\textit{ŋaa-suwan} in both cases). The etymology of these forms is unclear. At the same time, the majority of the forms for ‘ten’ (but not for ‘five’ as in the majority of the North Atlantic languages) go back to the lexical root meaning ‘hands’. To illustrate this point, the lexical stems for ‘hand’ in the Joola\il{Joola} languages are quoted in the table (\tabref{tab:3:237}).

\begin{table}
\caption{\label{tab:3:237}Joola\il{Joola} stems for `hand'}


\begin{tabularx}{\textwidth}{QlQQQ}
\lsptoprule

Bliss\il{Bliss} & Kasa\il{Kasa} & Fogny\il{Fogny} & Keeraak\il{Keeraak} & Bayot\il{Bayot}\\
Banjal\il{Banjal} & Mlomp\il{Mlomp} & Karon\il{Karon} & Ejamat\il{Ejamat} & Kwaatay\il{Kwaatay}\\
\midrule
{\textbf{‘hand'}} & {} & {} & {} & {}\\
ka-ŋɛn(ak) & ka-ŋɛn & ka-ɲen(ak)/ ka-ŋɛn & ka-ŋɛn & \\
ga-ɲɛn/ ka-ɲɛn(ak) &  & ka-ɲɛn & ka-ŋɛn(ak) & ka-ŋyɛn(ak)\\
{\textbf{‘‘hand'}} & {} & {} & {} & {}\\
~ & e-bɛɛs &  &  & \\
ɛ-pɛs & ɛ-bɛɛs & ɛ-pɛs/ɛ-bɛs &  & \\
{\textbf{‘hand'}} & {} & {} & {} & {}\\
~ &  &  &  & ɛ-mɔŋu/ ɛ-ŋɔmu\\
{\textbf{‘hand'}} & {} & {} & {} & {}\\
~ & ka-seʔ &  &  & ka-te/ga-te/ ʈe/kə-se\\
{\textbf{‘hand'}} & {} & {} & {} & {}\\
bu-lɛhɛj `hand' &  & ɛ-lɛcɛs\newline `upper arm' &  & \\
bi-lɛfɛj &  &  & bu-lɛfec\newline `inner hand' & \\
{\textbf{‘hand'}} & {} & {} & {} & {}\\
ka-ʂɛɲum(əku) &  &  & kə-ləɲum `hand' & \\
\lspbottomrule
\end{tabularx}
\end{table}

\newpage 
As can be deduced from the presentation above, at least four lexical roots for ‘hand’ that serve as a basis for the terms for ‘ten’ are distinguishable in Joola\il{Joola}. Interestingly, the source roots and the numerical terms that depend on them are not necessarily the same within a language. The main root is \textit{*ku-ŋɛn/} \textit{ku-ɲɛn} ‘10’ <‘hands’. At the same time, \textit{bɛɛs} ‘hand’ yields \textit{sɛ-bɛɛs} ‘ten’ in Mlomp\il{Mlomp}. This derivative is not attested in in Kasa\il{Kasa} and Karon\il{Karon} where \textit{bɛɛs} ‘hand’ alternates with \textit{ŋɛn/} \textit{ɲɛn} ‘hand’. The base \textit{*ka-ʈe} ‘hand’ attested in Bayot\il{Bayot} and Kasa yields \textit{gu-tie}- in Bayot. Finally, \textit{ɛ-mɔŋu} ‘hand’ > \textit{su-moŋu} ‘ten’ in Kwaatay\il{Kwaatay} (also \textit{ɛ-ŋɔmu} ‘hand’ > \textit{su-ŋɔmu} ‘ten’ with a metathesis).

As noted above, the root \textit{ɛ-ntaaja} attested in Keeraak\il{Keeraak} and Ejamat\il{Ejamat} was possibly incorporated into Kobiana\il{Kobiana} (North Atlantic). This root, admittedly very rare in the Joola\il{Joola} cluster, is the only primary one for ‘ten’ and as such it deserves special attention (especially in view of its later replacement with the derivatives based on ‘hand’). 

\subsubsubsection{{‘Twenty’,}  {‘hundred’,}  {and}  {‘thousand’}}

Two apparent derivational patterns are used for the term for ‘twenty’ in the Joola\il{Joola} languages:

\begin{exe}
\exi{<‘king’:} Bliss\il{Bliss} \textit{a-yɩɩy}, Banjal\il{Banjal} \textit{ə-vi/ə-vvi}, Kasa\il{Kasa} \textit{a-yi/} \textit{ɔ-ji}, Karon\il{Karon} \textit{əwi}, Bayot\il{Bayot} \textit{ə-y};
\exi{<‘person’:} Kasa\il{Kasa} \textit{an} \textit{/} \textit{bu-k-an}, Fogny\il{Fogny} \textit{ka-banan} ‘person finished’.
\end{exe}


In Kwaatay\il{Kwaatay} the term for ‘twenty’ is based on ‘mouth’ (\textit{bu-tum-an}).

The terms for ‘hundred’ and ‘thousand’ are borrowings from Mande or ‘influential’ Atlantic languages (often either Fula\il{Fula} or Wolof\il{Wolof}) in the majority of the dialects, cf. \textit{keme/teme} ‘100’, \textit{wuli,} \textit{juni} ‘1000’.

In conclusion it should be added that the Joola\il{Joola} terms covering the sequence from ‘six’ to ‘nine’ follow the common pattern ‘5+’.

\subsubsection{Manjak languages}\label{sec:3.12.2.2}%3.12.2.2.
\il{Manjak}This branch is represented by three closely related languages (Manjak\il{Manjak}, Mankanya\il{Mankanya}, Pepel\il{Pepel}). Numerical terms attested in them are presented in the table below (\tabref{tab:3:238}).

\begin{table}
\caption{\label{tab:3:238}Manjak\il{Manjak} numerals}


\begin{tabularx}{\textwidth}{lQrl}
\lsptoprule

1 & lɔɔl(e)/lɔŋ & 7 & 6+1, jand/{jaanʔ/} cand (Pepel)\il{Pepel}\\
2 & -təb/-təw, -puguʈ/pugus (Pepel)\il{Pepel} & 8 & 4PL, koas/ʊʌs\\
3 & wa-(y)anʈ/{wa-jenʈ/} {jens} & 9 & 10--1, (8+1)\\
4 & baakər/wakər & 10 & 5\textsc{pl} (‘hands’), (n)taaja/taaya, taim (Pepel)\il{Pepel}\\
5 & ɲɛɛn (‘hand’) & 20 & 10*2\\
6 & paagi/paaji & 100 & < French\il{French}\\
&  & 1000 & kʊnt\\
\lspbottomrule
\end{tabularx}
\end{table}

As can be gleaned from the table, the Manjak\il{Manjak} stems for numerals are very different from those attested in Joola\il{Joola}. At the same time, morphological and lexical evidence strongly suggests that these two branches are genetically the closest and belong to the same Bak sub-group. 

This implies that the numeral system of one of these branches must have undergone systematic innovations. We will reserve our conclusions until the evidence from the other Bak sub-groups, i.e. Balant\il{Balant} and Bijogo\il{Bijogo}, is reviewed.

\subsubsection{Balant}%3.12.2.3.
\il{Balant}Despite the fact that Balant\il{Balant} is usually treated as one language, we will present the evidence of Balant Ganja\il{Ganja} and Balant Kentohe\il{Kentohe} separately (\tabref{tab:3:239}), because the difference between these two idioms is of key importance to our study.

\begin{table}
\caption{\label{tab:3:239}Balant\il{Balant} numerals}


\begin{tabularx}{\textwidth}{r>{\raggedright}p{5cm}Q} 
\lsptoprule
&  {Balant}\il{Balant}  {Ganja}\il{Ganja} &  {Balant}\il{Balant}  {Kentohe}\il{Kentohe}\\
\midrule
1 & h{\'{ɔ}}dà/w{\'{ɔ}}dā/-ɔdaʔ, b{\'{ɔ}}{\'{ɔ}}d{\'{ɩ}}b{\'{ɔ}}/wɔdibɔ (counting) & -ɔɔdn/ho:dn/fóóda\\
2 & s{\`{ɩ}}b{\'{ɩ}}/-sebe & -sɩbm/-sebm/g-ʃííbn \citep{Koelle1963}\\
3 & {hàbí/yààbi}{\=ì} & -habm/káábn \citep{Koelle1963}\\
4 & tàllá/tàhàlā & -tasla/tahla/táʃiila \citep{Koelle1963}\\
5 & j{\`{ɩ}}{\'{ɩ}}f/jéèf & cɩf/`-cef/kiif {\textasciitilde} ciif \citep{Koelle1963}\\
6 & fááj/faac & mfaacɲ/faad \citep{Koelle1963}, 5+1\\
7 & 6+1 & 6+1, 5+2\\
8 & \mbox{táhtállà/tāntàhlā (4 redupl.), 6+2} & 5+3, 6+2 \citep{Koelle1963}\\
9 & jíntàllá/jīntàhlā (5+4) & 5+4, 6+3 \citep{Koelle1963}\\
10 & jímmín/jīnmīnn (<5?) & cɩfmɩɩn/f-cef meen (<5?), 6+4 \citep{Koelle1963}\\
20 & 10*2 & <‘person’\\
100 & g{\`{ɛ}}m{\'{ɛ}}/kɛmɛ (borrowed) & <‘5 persons’\\
1000 & wílí (borrowed), kont & f-koːnti\\
\lspbottomrule
\end{tabularx}
\end{table}

\largerpage
The opening sequence of the Ganja\il{Ganja} terms is quoted according to \citealt{CreisselsBiaye2015}. They form the most reliable part of the presentation. A few remarks pertaining to the differences in these Balant\il{Balant} dialects are in order. First of all, the Balant Kentohe\il{Kentohe} terms for ‘one’, ‘two’, ‘three’ and ‘six’ exhibit a final homorganic nasal of uncertain origin. The forms attested by Koelle in the 19\textsuperscript{th} century sources suggest that we are dealing with a morpheme  \textbf{-n} not assimilated to a preceeding consonant by point of articulation. Secondly, Koelle’s evidence speaks in favor of ‘six’ being a base for a larger group of numerical terms. According to him, not only ‘eight’ and ‘nine’ but also ‘ten’ followed the pattern ‘6+’. 

\subsubsection{Bijogo}\label{sec:3.12.2.4}%3.12.2.4.
\il{Bijogo}
\begin{table}
\caption{\label{tab:3:240}Bijogo\il{Bijogo} numerals}
\begin{tabularx}{\textwidth}{lQQ}
\lsptoprule
& \textbf{Bijogo}\il{Bijogo} \textbf{Kagbaga (Bubaque)} & \textbf{Bijogo}\il{Bijogo} \textbf{(other dialects)}\\
\midrule 
1 & n-ɔɔd (*-d) & \\
2 & n-somb (Segerer, p.c.), n-sombɛnʈ & sòòb{\'{ɛ}}/súngb/cuuwɛ, \newline ndank (Kamona)\\
3 & ɲ-ɲɔ-ɔkɔ (<‘fingers’) & \\
4 & ya-aɡɛnɛk & \\
5 & n-de-ɔkɔ (dɛ ‘to finish’, -ɔkɔ ‘hand’) & nu-duβ-ɔkɔ (Kamona)\\
6 & 5+1 & \\
7 & 5+2 & \\
8 & 5+3 & \\
9 & 5+4 & \\
10 & n-ruakɔ (ru ‘to rise’, -ɔkɔ ‘hand’) & \\
20 & o-joko (‘person’), -ansak-o-to (‘to finish’+‘somebody’) & ŋɔjɛt oto (Kamona), 

\citealt{Koelle1963}: rí{\textsubbar{a}}{\textsubbar{a}}k{\textsubtilde{\'{ɔ}}}{\textsubtilde{\'{ɔ}}}to/ŋórembaʃóóto\\
100 & 20*5 & \\
1000 & kuntu & \\
\lspbottomrule
\end{tabularx}
\end{table}

Let us examine an analysis of the Bijogo\il{Bijogo} numeral system found in \citep{Segerer2002}. According to him, the term for ‘one’ is \textit{nɔɔd} (``cette forme est retenue pour l’énu\-mé\-ration abstraite'', ibid.~171). His interpretation of *-\textbf{d} as the only true reflex of the etymon (with other segments ensuring the grammatical agreement) is immediately convincing, cf. the following examples quoted by him (ibid. 171):

\ea
\ea \textit{o-to} \textit{ɔ-nɔɔd} ‘a person’
\ex \textit{e-booʈi} \textit{ɛ-nɛɛd} ‘a dog’
\ex \textit{u-gbe} \textit{u-nɛɛd} ‘a road’
\ex \textit{ka-jɔkɔ} \textit{n-ka-d} ’a house’
\ex \textit{ŋɔ-katɔ} \textit{ŋ-ŋɔ-d} ’a fish’.
\z
\z

Segerer justly observes that \textit{‘}\textstyleCitationCar{\textup{La forme générale de l’élément ayant pour valeur ‘un (autre)’ est donc} }\textstyleCitationCar{\textbf{\textup{(V)-n-pC-d}}}\textstyleCitationCar{\textup{, où} }\textstyleCitationCar{\textbf{\textup{pC}}}\textstyleCitationCar{ \textup{est le préfixe de classe du nom déterminé}}\textit{’} (ibid. 171).

He also quotes the form \textit{dideeki} ‘seul’ (var. \textit{deeki} ‘tout seul’). A variant of this form probably appears as  \textit{èɖìg{\'{ɛ}}/} \textit{néédigɛ/} \textit{módiigɛ} ‘one’ in Wilson and Koelle.

As demonstrated by Segerer, the term for ‘three’ (\textit{ɲ-ɲɔɔkɔ}) is a Bijogo\il{Bijogo} innovation of a cultural origin, cf. \textsc{sg}  \textit{ɲɔ-ɔkɔ} - \textsc{pl} of \textit{nɔ-ɔkɔ} ‘finger’ (dim. <\textit{kɔ-ɔkɔ} ‘hand’): \textstyleCitationCar{\textup{‘Un roi bijogo ne se déplace jamais sans l’attribut symbolique de sa fonction, consitué par une sculpture de bois et de corne … Cet objet, nommé} }\textstyleCitationCar{u-ran kɔ-ɔkɔ}\textstyleCitationCar{\textup{, represente une main à trois doigts’~}}(ibid. 172). It should be noted that this root is attested in all Bijogo dialects and is already accounted for by Koelle (\textit{-ɲ{\'{ɔ}}{\'{ɔ}}gɔ}).

As established by Segerer, the same root is attested as \textit{ɔkɔ} in the terms for ‘five’ and ‘ten’. 

 
\subsubsection{Proto-Bak}%3.12.2.5.
\il{Proto-Bak}Now we will compare the Bak numerals.

\subsubsubsection{‘One’ (\tabref{tab:3:241})}

\begin{table}
\caption{\label{tab:3:241}Bak numerals for `1'}


\begin{tabularx}{\textwidth}{XXl}
\lsptoprule

{Joola}\il{Joola} & \textbf{don} & -anor, əkon, fɛnɛŋ, yinka, (sia), (ta)\\
{Manjak}\il{Manjak} & lɔɔl(e)/\textbf{lɔŋ} & \\
{Balant}\il{Balant} &  & -ɔdaʔ\\
{Bijogo}\il{Bijogo} & \textbf{*d} & -eɖìgɛ\\
\lspbottomrule
\end{tabularx}
\end{table}

A comparison of the terms quoted in the leftmost column yields the form that can be tentatively recorded as \textit{*don}. The rightmost column gives an overview of roots attested in only one out of four branches. 

\subsubsubsection{‘Two’ (\tabref{tab:3:242})}

\begin{table}
\caption{\label{tab:3:242}Bak numerals for `2'}


\begin{tabularx}{\textwidth}{XXX}
\lsptoprule

{Joola}\il{Joola} & si-ɬubəʔ & si-gabaʔ\\
{Manjak}\il{Manjak} &  & -təb/-təw, puguʈ/pugus\\
{Balant}\il{Balant} & sɩbɩ/-sebe & \\
{Bijogo}\il{Bijogo} & sòòb{\'{ɛ}}/súngb/cuuwɛ & \\
\lspbottomrule
\end{tabularx}
\end{table}

The leftmost column presents the root attested in three sub-groups. It is traceable to \textit{*ɬubəʔ.}

\subsubsubsection{‘Three’ and ‘four’ (\tabref{tab:3:243})}

\begin{table}
\caption{\label{tab:3:243}Bak numerals for `3' and `4'}


\begin{tabularx}{\textwidth}{XlXX}
\lsptoprule
~ & `3' & `4' & `4' \\
\midrule
{Joola}\il{Joola} & si-feeɡir & si-bääkiɽ & \\
{Manjak}\il{Manjak} & wa-(y)anʈ/wa-jenʈ/jens & baakər/wakər & \\
{Balant}\il{Balant} & habi/yabi &  & tasala/tahala\\
{Bijogo}\il{Bijogo} & ɲ-ɲɔ-ɔkɔ (<‘fingers’) &  & ya-aɡɛnɛk\\
\lspbottomrule
\end{tabularx}
\end{table}

For the first time in our step-by-step analysis of numeral systems in the numerous NC families we observe the existence of a separate root for ‘three’ in each of the branches of a language group.

The term for ‘four’ exhibits an isolated Joola\il{Joola}-Manjak\il{Manjak} innovation as well as isolated innovations in Balant\il{Balant} and Bijogo\il{Bijogo}.

\subsubsubsection{‘Five’ (\tabref{tab:3:244})}

\begin{table}
\caption{\label{tab:3:244}Bak numerals for `5'}
\begin{tabularx}{\textwidth}{lQl}
\lsptoprule
{Joola}\il{Joola} &  & fu-tɔk, tən?, ŋaa-suwaŋ? (cf. `10')\\
{Manjak}\il{Manjak} & ɲɛɛn (‘hand’) (cf. Joola\il{Joola} `10') & \\
{Balant}\il{Balant} &  & j{\`{ɩ}}{\'{ɩ}}f/jéèf\\
{Bijogo}\il{Bijogo} & n-de-ɔkɔ (dɛ ‘to finish’, -ɔkɔ ‘hand’) & \\
\lspbottomrule
\end{tabularx}
\end{table}

The pattern ‘hand’ > ‘5’ is traceable within two branches. However, the roots involved are different in each case. Numerous isolated forms are grouped together in the rightmost column.

\subsubsubsection{The terms from ‘six’ to ‘nine’ (\tabref{tab:3:245})}

\begin{table}
\caption{\label{tab:3:245}Bak numerals and patterns for `6'-'9'}
\begin{tabularx}{\textwidth}{llQQll}
\lsptoprule
~ & `6' & `6' & `7' & `8' & `9' \\
\midrule
{Joola}\il{Joola} & 5+1 &  & 5+2 & 5+3 & 5+4\\
{Manjak}\il{Manjak} &  & paagi/paaji & 6+1, jand/jaanʔ/cand & 4PL, koas/ʊʌs & 10--1, (8+1)\\
{Balant}\il{Balant} &  & fááj/faac & 6+1 & 4 redupl., 6+2 & 6+3, 5+4\\
{Bijogo}\il{Bijogo} & 5+1 &  & 5+2 & 5+3 & 5+4\\
\lspbottomrule
\end{tabularx}
\end{table}

The form \textit{*paag/paaj} ‘six’ is a common Manjak\il{Manjak}-Balant\il{Balant} isogloss.\footnote{Guillaume Segerer is right to note (p.c.) that the Manjak\il{Manjak}-Balant\il{Balant} form *\textit{paag-} ‘6’ may be ralated to Joola\il{Joola}  *-\textit{feeɡir}/-\textit{həəji} ‘3’}  It is not surprising that the primary term for ‘six’ attested in these languages served as the basis for the ‘7=6+1’ pattern. This pattern received further development in Balant where it was employed for terms up to ‘ten’ (i.e. ‘10=6+4’) according to the 19\textsuperscript{th} century sources. At the same time, the archaic pattern ‘8=4PL’/‘8=4 redupl.’ is attested in these languages alongside the pattern ‘8=6+2’.

\newpage 
\subsubsubsection{‘Ten’ (\tabref{tab:3:246})}

\begin{table}
\caption{\label{tab:3:246}Bak numerals for `10'}


\begin{tabularx}{\textwidth}{lQQQl}
\lsptoprule

{Joola}\il{Joola} & ɛ-ntaaja\footnotemark{} & ku-ŋɛn/ɲɛn `hands' & `hands'\newline (bɛɛs, moŋu/ŋɔmu, tie) & ŋaa-suwan\\
{Manjak}\il{Manjak} & (n)taaja/ taaya &  & 5\textsc{pl} (‘hands’) & taim\\
{Balant}\il{Balant} &  &  &  & jímmín, 6+4\\
{Bijogo}\il{Bijogo} &  &  & n-ruakɔ\newline (ru ‘to rise’, -ɔkɔ ‘hand’) & \\
\lspbottomrule
\end{tabularx}
\end{table}

\footnotetext{ The stem is attested only in Joola\il{Joola} Feloup\il{Feloup}, so, it seems to be borrowed from Manjak\il{Manjak}.}
In addition to the common pattern ‘10 = ‘hands’’, both branches share a common root (\textit{ntaaja}) that could be interpreted as a shared Proto-Joola\il{Proto-Joola}-Manjak\il{Manjak} innovation.

\subsubsubsection{‘Twenty’, ‘hundred’ and ‘thousand’}

The term for ‘twenty’ is based on the lexical root meaning ‘person’ in all of the branches (except for Manjak\il{Manjak}, where it was replaced with the pattern ‘20=10*2’). The same development is observable in Balant\il{Balant} Ganja\il{Ganja} as well.

The terms for ‘hundred’ and ‘thousand’ are most likely borrowings. However, the origin of \textit{kont}/\textit{kunt} ‘thousand’ attested in three of the Bak branches deserves special discussion (in North Atlantic this root (\textit{ŋ-kontu}) is found in both of the Buy languages).

\newpage 
\subsubsubsection{Overview of the Bak numerical terms (\tabref{tab:3:247})}

\begin{table}
\caption{\label{tab:3:247}Bak numerals}


\begin{tabularx}{\textwidth}{ll@{}rQ}
\lsptoprule

1 & don/lɔŋ, -anor, əkon & 7 & 6+1, 5+2, jand/{jaanʔ/} cand (Pepel)\il{Pepel}\\
2 & ɬubəʔ, -təb/-təw, -puguʈ/pugus & 8 & 4PL/4 redupl., {ʊʌs}\\
3 & feeɡir, yanʈ/{jenʈ,} habi/yabi & 9 & 5+4, 10--1, 6+3\\
4 & baakər/wakər, tasala/tahala & 10 & 5\textsc{pl} (‘hands’), (n)taaj, taim, -suwan\\
5 & ‘hand’, tɔk, tən? & 20 & ‘person’, 10*2\\
6 & paag/paaj, 5+1 & 100 & borrowed\\
&  & 1000 & kʊnt (borrowed?)\\
\lspbottomrule
\end{tabularx}
\end{table}

 
\subsection{North Atlantic and Bak Atlantic numerals in the comparative perspective}\label{sec:3.12.3}%3.12.3.
It should be stressed that the Atlantic family is among the most divergent within Niger-Congo. Some of the numerical terms in both of the Atlantic groups exhibit a variety of forms potentially explained as Proto-NC\il{Proto-NC} reflexes. Moreover, the comparative evidence presented in Tables \ref{tab:3:225} (Proto-North-Atlantic) and \ref{tab:3:239} (Proto-Bak\il{Proto-Bak}-Atlantic\il{Proto-Bak-Atlantic}) points to the near total absence of common roots present in both groups. The only exception to this is the root \textit{tɔk/} \textit{tVk} ‘five’.

In view of this, the only available solution would be the study of the Atlantic evidence within a wider NC context (i.e. in contrast to the reconstructions available for other NC families). A comparison of the intermediate reconstructions within the macro-family will be offered in the next chapter.

\section{Isolated languages vs. Atlantic and Mel}%3.13.

According to the traditional classification outlined in \citealt{Sapir1971}, Limba\il{Limba}, Sua\il{Sua} and Gola\il{Gola} belong to the Atlantic languages. However, as we tried to demonstrate in \citealt{PozdniakovSegerer2017} (forthcoming) this hypothesis is as ill-grounded today as it was half a century ago. 

An overview of the pertinent data for each language is presented in the tables below.

\newpage 
\subsection{Sua}%3.13.1.
\il{Sua}
~\vspace*{-\baselineskip}
\begin{table}[h]
\caption{\label{tab:3:248}Sua\il{Sua} numerals}
\begin{tabularx}{.66\textwidth}{lQrl}
\lsptoprule
1 & sɔn & 7 & 5+2\\
2 & cen & 8 & 5+3\\
3 & b-rar & 9 & 5+4\\
4 & b-nan & 10 & tɛŋi\\
5 & sɔŋɡun & 20 & 10*2\\
6 & 5+1 & 100 & kɛmɛ\\
&  & 1000 & uŋ-kɔntu\\
\lspbottomrule
\end{tabularx}
\end{table}


\subsection{Gola}%3.13.2.
\il{Gola}
\begin{table}
\caption{\label{tab:3:249}Gola\il{Gola} numerals}


\begin{tabularx}{.66\textwidth}{lQrl}
\lsptoprule
1 & ɡuùŋ & 7 & 5+2\\
2 & tì-yèe/tī-el/cel & 8 & 5+3\\
3 & taai/tāāl & 9 & 5+4\\
4 & tii-nàŋ & 10 & zììyà\\
5 & n{\`{ɔ}}{\`{ɔ}}n{\`{ɔ}}ŋ & 20 & kp{\`{ɛ}}(w)ùŋ\\
6 & 5+1 & 100 & 20*5\\
&  & 1000 & < English\\
\lspbottomrule
\end{tabularx}
\end{table}

\largerpage[3] 
\subsection{Limba}%3.13.3.
\il{Limba}
\begin{table}[h]
\caption{\label{tab:3:250}Limba\il{Limba} numerals}
\begin{tabularx}{.66\textwidth}{lQrl}
\lsptoprule
1 & ha-nthe & 7 & 5+2\\
2 & ka-le/kaa-ye & 8 & 5+3\\
3 & ka-tati & 9 & 5+4\\
4 & ka-naŋ & 10 & kɔhi\\
5 & ka-sɔhi & 20 & 10*2\\
6 & 5+1 & 100 & kɛmɛ, wuli (borrowed)\\
&  & 1000 & wulu (< Mande)\\
\lspbottomrule
\end{tabularx}
\end{table}


This chapter includes 250 tables presenting the evidence by group, branch or sometimes a dialect of a certain language. Among them are summary tables that provide an overview of the numerical terms in twelve major families of Niger-Congo and in a number of isolated languages. Our attempt at reconstructing the Proto-Niger-Congo\il{Proto-Niger-Congo} numeral system on the basis of this comprehensive evidence will be presented in \chapref{sec:4}.

