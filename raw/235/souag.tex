\documentclass[output=paper]{langsci/langscibook} 
\ChapterDOI{10.5281/zenodo.3744535}
\author{Lameen Souag\affiliation{CNRS, LACITO}}
\title{Berber}
\abstract{Arabic has influenced Berber at all levels – not just lexically, but phonologically, morphologically, and syntactically – to an extent varying from region to region. Arabic influence is especially prominent in smaller northern and eastern varieties, but is substantial even in the largest varieties; only in Tuareg has Arabic influence remained relatively limited.  This situation is the result of a long history of large-scale asymmetrical bilingualism often accompanied by language shift.}
\IfFileExists{../localcommands.tex}{
  \usepackage{langsci-optional}
\usepackage{langsci-gb4e}
\usepackage{langsci-lgr}

\usepackage{listings}
\lstset{basicstyle=\ttfamily,tabsize=2,breaklines=true}

%added by author
% \usepackage{tipa}
\usepackage{multirow}
\graphicspath{{figures/}}
\usepackage{langsci-branding}

  
\newcommand{\sent}{\enumsentence}
\newcommand{\sents}{\eenumsentence}
\let\citeasnoun\citet

\renewcommand{\lsCoverTitleFont}[1]{\sffamily\addfontfeatures{Scale=MatchUppercase}\fontsize{44pt}{16mm}\selectfont #1}
   
  %% hyphenation points for line breaks
%% Normally, automatic hyphenation in LaTeX is very good
%% If a word is mis-hyphenated, add it to this file
%%
%% add information to TeX file before \begin{document} with:
%% %% hyphenation points for line breaks
%% Normally, automatic hyphenation in LaTeX is very good
%% If a word is mis-hyphenated, add it to this file
%%
%% add information to TeX file before \begin{document} with:
%% %% hyphenation points for line breaks
%% Normally, automatic hyphenation in LaTeX is very good
%% If a word is mis-hyphenated, add it to this file
%%
%% add information to TeX file before \begin{document} with:
%% \include{localhyphenation}
\hyphenation{
affri-ca-te
affri-ca-tes
an-no-tated
com-ple-ments
com-po-si-tio-na-li-ty
non-com-po-si-tio-na-li-ty
Gon-zá-lez
out-side
Ri-chárd
se-man-tics
STREU-SLE
Tie-de-mann
}
\hyphenation{
affri-ca-te
affri-ca-tes
an-no-tated
com-ple-ments
com-po-si-tio-na-li-ty
non-com-po-si-tio-na-li-ty
Gon-zá-lez
out-side
Ri-chárd
se-man-tics
STREU-SLE
Tie-de-mann
}
\hyphenation{
affri-ca-te
affri-ca-tes
an-no-tated
com-ple-ments
com-po-si-tio-na-li-ty
non-com-po-si-tio-na-li-ty
Gon-zá-lez
out-side
Ri-chárd
se-man-tics
STREU-SLE
Tie-de-mann
} 
  \togglepaper[1]%%chapternumber
}{}

\begin{document}
\maketitle 
  


 \section{Current state and contexts of use}


 \subsection{Introduction}


Berber, or {Tamazight}, is the indigenous language family of northwestern Africa, distributed discontinuously across an area ranging from western Egypt to the Atlantic, and from the Mediterranean to the Sahel. Its range has been expanding in the Sahel within recent times, as \ili{Tuareg} speakers move southwards, but in the rest of this area, Berber has been present since before the classical period (\citealt{MúrciaSánchez2010}). Its current discontinuous distribution is largely the result of \isi{language shift} to \ili{Arabic} over the past millennium.

At present, the largest concentrations of Berber speakers are found in the highlands of Morocco (\ili{Tashelhiyt}, \ili{Tamazight}, \ili{Tarifiyt}) and northeastern Algeria (\ili{Kabyle}, \il{Shawiya}Chaoui). \ili{Tuareg}, in the central Sahara and Sahel, is more diffusely spread over a large but relatively sparsely populated zone. Across the rest of this vast area, Berber varieties constitute small islands – in several cases, single towns – in a sea of \ili{Arabic}.

This simplistic map, however, necessarily leaves out the effects of mobility – not limited to the traditional practice of nomadism in the Sahara and transhumance in parts of the Atlas mountains. The rapid urbanisation of North Africa over the past century has brought large numbers of Berber speakers into traditionally \ili{Arabic}-speaking towns, occasionally even changing the town's dominant language. The conquests of the early \isi{colonial} period created small Berber-speaking refugee communities in the Levant and Chad, while more recent emigration has led to the emergence of urban Berber communities in western Europe and even Quebec.


 
 \subsection{Sociolinguistic situation of Berber}


In North Africa proper, the key context for the maintenance of Berber is the village. Informal norms requiring the use of Berber with one's relatives and fellow villagers, or within the village council, encourage its maintenance not only there but in cities as well, depending on the strength of emigrants' (often multigenerational) ties to their hometowns. In some areas, such as \ili{Igli} in Algeria \citep{Mouili2013}, the introduction of mass education in \ili{Arabic} has disrupted these norms, encouraging parents to speak to their children in \ili{Arabic} to improve their educational chances; in others, such as \ili{Siwa} in Egypt \citep{Serreli2017}, it has had far less impact. Beyond the village, in wider rural contexts such as markets, communication is either in Berber or in \ili{Arabic}, depending on the region; where it is in \ili{Arabic}, it creates a strong incentive for \isi{bilingualism} independent of the state's influence.  For centuries, Berber-speaking villages in largely \ili{Arabic}-speaking areas have sporadically been shifting to \ili{Arabic}, as in the Blida region of Algeria (\citealt{ElArifi2014}); the opposite is also more rarely attested, as near Tizi-Ouzou in Algeria \citep[258]{Gautier1913}.

In urban contexts, on the other hand, norms enforcing Berber have no public presence – quite the contrary.  There one addresses a stranger in \ili{Arabic}, or sometimes \ili{French}, but rarely in Berber, except perhaps in a few Berber-majority cities such as Tizi{}-Ouzou \citep{Tigziri2008}.  Even within the family, \ili{Arabic} takes on increasing importance; in a study of \ili{Kabyle} Berbers living in Oran (Algeria), Ait Habbouche (\citeyear[79]{AitHabbouche2013}) found that 54\% said they mostly spoke \ili{Arabic} to their siblings, and 10\% even with their grandparents. In the Sahel, \ili{Arabic} is out of the picture, but there too family language choice is affected; 13\% of the Berber speakers interviewed by Jolivet (\citeyear[146]{Jolivet2008}) in Niamey (Niger) reported speaking no Tamasheq at all with their families, using \ili{Hausa} or, less frequently, \ili{Zarma} instead.

\hspace*{-1.09703pt}Bilingualism is widespread but strongly asymmetrical. Almost all Berber speakers learn dialectal \ili{Arabic} (as well as \ili{Standard} \ili{Arabic}, taught at school), whereas \ili{Arabic} speakers almost never learn Berber.  There are exceptions: in some contexts, \ili{Arabic}-speaking women who marry Berber-speaking men need to learn Berber to speak with their in-laws (the author has witnessed several \ili{Kabyle} examples), while \ili{Arabic} speakers who settle in a strongly Berber-speaking town – and their children – sometimes end up learning Berber, as in \ili{Siwa} (Egypt). Nevertheless, most \ili{Arabic} speakers place little value on the language, and some openly denigrate it; in Bechar (Algeria), anyone expressing interest in Berber can expect frequently to hear the contemptuous saying \textit{əš-šəlḥa} \textit{ma-hu} \textit{klam} \textit{wə-d-dhən} \textit{ma} \textit{hu} \textit{l-idam} `Shilha (Berber) is no more speech than vegetable oil is animal fat'. To further complicate the situation, \ili{French} remains an essential career skill (except in Libya and Egypt), since it is still the working language of many ministries and companies; in some middle-class families, it is the main home language spoken with children.

On paper, Berber ({Tamazight}) is now an official language of Morocco (since 2011) and Algeria (since 2016), while \ili{Tuareg} (Tamasheq/Tamajeq) is a recognised national language of {Mali} and Niger. In practice, “official language” remains a misleading term.  Official documents are rarely, if ever, provided in Berber, and there is no generalised right to communicate with the government in Berber.  However, Berber is taught as a school subject in selected Algerian, {Moroccan}, and (since 2012 or so) Libyan schools, while some Malian and Nigerien ones even use it as a medium of education. It is also used in broadcast media, including some TV and radio channels. Both Morocco and Algeria have established language planning bodies to promote neologisms and encourage publishing, with a view towards standardisation. The latter poses difficult problems, given that each country includes major varieties which are not inherently mutually intelligible.

Berber varieties have been written since before the second century BC \citep{Pichler2007} – although the language of the earliest inscriptions is substantially different from modern Berber and decipherable only to a limited extent – and southern Morocco has left a substantial corpus of pre-\isi{colonial} manuscripts \citep{Boogert1997}; many other examples could be cited from long before people such as \citet{Mammeri1976} attempted to make Berber a printed language. Nevertheless, writing seems to have had very little impact on the development of Berber as yet. Awareness of the existence of a Berber writing system – Tifinagh – is widespread, and often a matter of pride. However, most Berber speakers have never studied Berber, and do not habitually read or write in it in any script – with the increasingly important exception of social media and text messages, typically in \ili{Latin} or \ili{Arabic} script depending on the region. Efforts to create a standard literary Berber language have not so far been successful enough to exert a unifying influence on its dispersed varieties. In the North African context, this is often understood as implying that Berber is not a language at all – “language” (\ili{Arabic} \textit{luɣa}) being popularly understood in the region as “standardised written language”.
\largerpage
 
 \subsection{Demographic situation of Berber}


No reliable recent estimate of the number of Berber speakers exists; relevant data is both scarce and hotly contested. The estimates brought together by Kossmann (\citeyear[1]{Kossmann2011}; \citeyear[29--36]{Kossmann2013book}) suggest a range of 30–40\% for Morocco, 20–30\% for Algeria, 8\% for Niger, 7\% for {Mali}, about 5\% for Libya, and less than 1\% for Tunisia, Egypt, and Mauritania.  Selecting the midpoint of each range, and substituting in the mid-2017 populations of each of these countries (\citealt{CIA2017}) would yield a total speaker population of about 25 million, 22 million of them divided almost evenly between Morocco and Algeria.


 \section{Contact languages and historical development}


 \subsection{Across North Africa} \label{na}


Berber contact with \ili{Arabic} began in the seventh century with the Islamic conquests. For several centuries, \isi{language shift} seems to have been largely confined to major cities and their immediate surroundings, probably affecting \ili{Latin} speakers more than Berber speakers. The invasion of the \isi{Banū Hilāl} and \isi{Banū Sulaym} in the mid-eleventh century is generally identified as the key turning point: it made \ili{Arabic} a language of pastoralism, rapidly reshaping the linguistic landscape of Libya and southern Tunisia, then over the following centuries slowly transforming the High Plateau and the northern Sahara in general.  This rural expansion further reinforced the role of \ili{Arabic} as a lingua franca, while the recruitment of \ili{Arabic}-speaking soldiers from pastoralist tribes encouraged its spread further west to the \ili{Moroccan} Gharb.

The resulting linguistic divide between rural groups and towns remained a key theme of \ili{Maghrebi} sociolinguistics until the twentieth century. In several cases, a town spoke a different language than its hinterland; in much of the Sahara, Berber-speaking oasis towns such as Ouargla or \ili{Igli} formed linguistic islands in regions otherwise populated by \ili{Arabic} speakers, and in the north, towns such as Bejaia or Cherchell constituted small \ili{Arabic}-speaking communities surrounded by a sea of Berber-speaking villages. Even in larger cities such as Algiers or Marrakech, the dominance of \ili{Arabic} was counterbalanced by substantial regular immigration from Berber-speaking regions further afield.

Today all Berber communities are more or less multilingual, usually in \ili{Arabic} and often also in \ili{French}; outside of the most remote areas, \isi{monolingual} speakers are quite difficult to find. Even in the nineteenth century, however, \isi{monolingual} Berber speakers were considerably more numerous \citep[41]{Kossmann2013book}.

Alongside the coexistence of colloquial \ili{Maghrebi} \ili{Arabic} with Berber, \ili{Classical} \ili{Arabic} also had a role to play as the primary language of learning and in particular religious studies.  Major Berber-speaking areas such as Kabylie (northern Algeria) and the Souss (southern Morocco) developed extensive systems of religious education, whose curricula consisted primarily of \ili{Arabic} books (\citealt{Boogert1997}; \citealt{Mechehed2007}). The restriction of \ili{Classical} \ili{Arabic} to a limited range of contexts, and the relatively small proportion of the population pursuing higher education, gave it a comparatively small role in the contact situation; even in the lexicon, its influence is massively outweighed by that of colloquial \ili{Arabic}, and it appears to have had no structural influence at all.


 
 \subsection{In Siwa}


Examples of contact-induced change in this chapter are often drawn from Siwi, the Berber language of the oasis of \ili{Siwa} in western Egypt. Sporadic long-distance contact with \ili{Arabic} there presumably began in the seventh or eighth century with the Islamic conquests, and increased gradually as Cyrenaica and Lower Egypt became \ili{Arabic}-speaking and as the trade routes linking Egypt to West Africa were re-established.  During the eleventh century, the \isi{Banū Sulaym}, speaking a \ili{Bedouin} \ili{Arabic} dialect, established themselves throughout Cyrenaica.

In the twelfth century, al-Idrīsī\ia{al-Idrīsī, Muḥammad@al-Idrīsī, Muḥammad} reports Arab settlement within \ili{Siwa} itself, alongside the Berber population. Later geographers make no mention of an Arab community there, suggesting that these early immigrants were integrated into the Berber majority. Several core \ili{Arabic} loans in Siwi, such as the negative \isi{copula} \textit{qačči} < \textit{qaṭṭ} \textit{šayʔ} and the noon prayer \textit{luli} < \textit{al-ʔūlē}, are totally absent from surrounding \ili{Arabic} varieties today; such archaisms are likely to represent founder effects dating back to this period \citep{Souag2009}.  

The available data gives nothing close to an adequate picture of the linguistic environment of medieval \ili{Siwa}. We may assume that, throughout these centuries, most Siwis – or at least the dominant families – would have spoken Berber as their first language, and more mobile ones – especially traders – would have learned \ili{Arabic} (but whose \ili{Arabic}?) as a second language. Alongside these, however, we must envision a fluctuating population of \ili{Arabic}-speaking immigrants and West African slaves learning Berber as a second language. In such a situation, both Berber-dominant and \ili{Arabic}-dominant speakers should be expected to play a part in bringing \ili{Arabic} influences into Siwi.

The oasis was integrated into the Egyptian state by Muhammad Ali in 1820, but large-scale state intervention in the linguistic environment of the oasis only took effect in the twentieth century; the first government school was built in 1928, and television was introduced in the 1980s. An equally important development during this period was the rise of labour migration, taking off in the 1960s as Siwi landowners recruited Upper Egyptian labourers, and Siwi young men found jobs in Libya's booming oil economy. It has then grown further since the 1980s with the rise of tourism and the growth of tertiary education. The effects of this integration into a national economy include a conspicuous generation gap in local second-language \ili{Arabic}: older and less educated men speak a \ili{Bedouin}-like dialect with *q > \textit{g}, while younger and more educated ones speak a close approximation of \ili{Cairene} \ili{Arabic}.


 \section{Contact-induced changes in Berber}


 \subsection{Introduction}


As noted above, \isi{bilingualism} in North Africa has been asymmetrical for many centuries, with Berbers much more likely to learn \ili{Arabic} than vice versa. This suggests the plausible general assumption that the agents of contact-induced change were typically dominant in the (Berber) \isi{recipient language} rather than in \ili{Arabic}.  However, closer examination of individual cases often reveals a less clear-cut situation; as seen above in §\ref{na}, the history of Siwi suggests that Berber- and \ili{Arabic}-dominant speakers both had a role to play, and \textit{post} \textit{facto} analysis of the language's structure seems to confirm this assumption.  The loss of feminine plural \isi{agreement}, for example (§\ref{morph} below), can more easily be attributed to \ili{Arabic}-dominant speakers adopting Berber than to Berber-dominant speakers.  In the absence of clear documentary evidence, caution is therefore called for in the application of Van Coetsem's (\citeyear{VanCoetsem1988,VanCoetsem2000}) model to Berber.


 
 \subsection{Phonology}


The influence of \ili{Arabic} on Berber phonology is conspicuous; in general, every \isi{phoneme} used in a given region's dialectal \ili{Arabic} is found in nearby Berber varieties. Almost all \ili{Northern Berber} varieties have adopted from \ili{Arabic} at least the \isi{pharyngeals} /ʕ/ and /ḥ/, a series of voiceless emphatics: /ṣ/, /ḫ/, non-geminate /q/, and either /ḍ/ or /ṭ/. These phonemes presumably reached Berber through \isi{loanwords} from \ili{Arabic}, but have been extended to inherited vocabulary as well, through reinterpretation of \isi{emphatic} spread or through their use in “\isi{expressive formations}” \citep[199]{Kossmann2013book}, e.g. \ili{Kabyle} \textit{θi-ḥəðmər-θ} `breast of a small animal' < \textit{iðmar-ən} `breast'.

In Siwi (\citealt{Souag2013book}: 36–39; \citealt{SouagvanPutten2016}), at least nine phonemes were clearly introduced from \ili{Arabic}.  The pharyngealised coronals /ṣ/, /ḷ/, /\R/ and /ḍ/ have no regular source in Berber, and occur in inherited vocabulary almost exclusively as a result of secondary \isi{emphasis} spread (with the isolated exception of \textit{ḍəs} `to laugh').  The order of borrowing appears to be \textit{ḷ,} \textit{ṛ} > \textit{ṣ} > \textit{ḍ}; in a few older loans, \ili{Arabic} \textit{ṣ} is borrowed as \textit{ẓ} (e.g. \textit{ẓəffaṛ} `to whistle' < \textit{ṣaffar}), and in all but the most recent strata of loans, \ili{Arabic} \textit{ḍ/ð̣} is borrowed as \textit{ṭ} (e.g. \textit{a-ʕṛiṭ} `broad' < \textit{ʕarīḍ}).  The \isi{pharyngeals} /ḥ/ and /ʕ/ (e.g. \textit{ḥəbba} `a little' < \textit{ḥabba} `a grain', \textit{ʕammi} `paternal uncle' < \textit{ʕamm-ī} `uncle-\textsc{obl.1sg}') likewise have no regular source in Berber, although 1\textsc{sg} -\textit{ɣ}{}- has become \textit{{}-ʕ}{}- for some speakers (an \isi{irregular sound change} specific to this morpheme). \textit{ʕ} is lost in a number of older loans (e.g. \textit{annaš} `bier' < \textit{an-naʕš}), but \textit{ḥ} is always retained as such rather than being dropped or adapted (unlike \ili{Tuareg}, where it is typically adapted to \textit{ḫ}).  This suggests that Siwi continued to adapt \ili{Arabic} loans to its phonology by dropping \textit{ʕ} up to some stage well after the beginning of significant borrowing from \ili{Arabic}, but started accepting \ili{Arabic} loans with \textit{ḥ} too early for any adapted to survive, implying an order of borrowing \textit{ḥ} > \textit{ʕ}. Among the glottals, /h/ (e.g. \textit{ddhan} `oil' < \textit{dihān} `oils') appears in inherited vocabulary only in the distal \isi{demonstratives}, where comparison to Berber languages that do have \textit{h} suggests that it is excrescent, while /ʔ/ only rarely appears even in recent \isi{loanwords} (e.g. \textit{ʔəǧǧəṛ} `to rent' < \textit{ʔaǧǧar}). The mid vowel /o/ has been integrated into Siwi phonology as a result of borrowing from \ili{Arabic}; having been established as a \isi{phoneme}, however, it went on to emerge by irregular change from original {*u} in two inherited words (\textit{allon} `window', \textit{agṛoẓ} `palm heart'), and from irregular \isi{simplification} of {*aɣu} in some \isi{demonstratives} (e.g. \textit{wok} `this\textsc{.sg.m}' < *wa ɣuṛ-ək `this.\textsc{sg.m} at-\textsc{2sg.m')}. The interdentals /θ/ and /ð̣/ have a more marginal status, but are used by some speakers even in morphologically well-integrated loans, e.g. \textit{a-θqil} or \textit{a-tqil} `heavy' < \textit{θaqīl}. \ili{Arabic} influence may also be responsible for the treatment of [ʒ] and [dʒ] as free variants of the same \isi{phoneme} /ǧ/ \citep{Vycichl2005}, so that e.g. /taǧlaṣt/ `spider' is variously realised as [tʰæʒlˤɑsˤt] {\textasciitilde} [tʰædʒlˤɑsˤt] (\citealt{Naumann2012}: 152); other Berber languages with phonemic \textit{ž} normally have [dʒ] as a conditioned allophone (e.g. when geminated) or as a cluster.

\ili{Arabic} influence has also massively affected the \isi{frequency} of some phonemes. /q/ and /ḫ/ were marginal in Siwi before \ili{Arabic} influence, while *e had nearly disappeared due to regular sound changes, but all three are now quite frequent. Conversely, the influx of \ili{Arabic} loans has helped make labiovelarised phonemes such as \textit{gʷ} and \textit{qʷ} rare.


 
 \subsection{Morphology} \label{morph}


Berber offers numerous examples of the borrowing of \ili{Arabic} words together with their original \ili{Arabic} \isi{inflectional} morphology, a case of what \citet{Kossmann2010} calls Parallel System Borrowing. This phenomenon is most prominent for nominal number marking, but sometimes attested in other contexts too.

In Berber, most nouns are consistently preceded by a prefix marking \isi{gender} (masculine/feminine), number (singular/plural), and often case/state. Nouns borrowed from \ili{Arabic} normally either get assigned a Berber prefix, or fill the prefix slot with an invariant reflex of the \ili{Arabic} \isi{definite} \isi{article}: compare Figuig \textit{a-gʕud} vs. Siwi \textit{lə-gʕud} `young camel' (< \textit{qaʕūd}). The Berber plural marking system prior to \ili{Arabic} influence was already rather complex, combining several different types of affixal marking with internal ablaut strategies; many \ili{Arabic} loans are integrated into this system, e.g. \ili{Kabyle} \textit{a-bellar} `crystal' > pl. \textit{i-bellar-en} (< \textit{billawr}), Siwi \textit{a-kəddab} `liar' > pl. \textit{i-kəddab-ən} (< \textit{kaððāb}). However, in most Berber varieties, \ili{Arabic} loans have further complicated the system by frequently retaining their original plurals, e.g. \ili{Kabyle} \textit{l-kaɣeḍ} `paper' > \textit{le-kwaɣeḍ} (< \textit{kāɣid}), Siwi \textit{əl-gənfud} `hedgehog' > pl. \textit{lə-gnafid} (< \textit{qunfuð}). (The difference correlates fairly well with the choice in the singular between a Berber prefix and an \ili{Arabic} \isi{article}, but not perfectly; contrast e.g. Siwi \textit{a-fruḫ} `chick, bastard' < \textit{farḫ}, which takes the \ili{Arabic}-style plural \textit{lə-fraḫ}.)  Berber has no inherited system of dual marking, instead using analytic strategies. Nevertheless, for a limited number of measure words, duals too are borrowed, e.g. \ili{Kabyle} \textit{yum-ayen} `two days' < \textit{yawm-ayn} (although `day' remains \textit{ass}!), Siwi \textit{s-sən-t} `year' > \textit{sən-t-en} `two years' < \textit{san-at-ayn}.  \ili{Arabic} number morphology may sporadically spread to inherited terms as well, e.g. \ili{Kabyle} \textit{berdayen} `twice' < \textit{a-brid} `road, time', Siwi \textit{lə-gʷrazən} `dogs' < \textit{a-gʷərzni} `dog' \citep{Souag2013book}.

Whereas nouns are often borrowed together with their original \isi{inflectional} morphology, verbs almost never are. The only attested exception is \ili{Ghomara}, a heavily mixed variety of northern Morocco.  In \ili{Ghomara}, many (but not all) verbs borrowed from \ili{Arabic} are systematically conjugated in \ili{Arabic} in otherwise \isi{monolingual} utterances, a phenomenon which seems to have remained stable over at least a century: thus `I woke up' is consistently \textit{faq-aḫ}, but `I fished' is equally consistently \textit{ṣṣað-iθ} (\citealt{Mourigh2016}: 6, 137, 165). However, the borrowing of \ili{Arabic} participles to express progressive aspect is also attested in \ili{Zuwara}, if only for the two verbs of motion \textit{mašəy} `going' (pl. \textit{mašy-in}) and \textit{žay} `coming' (pl. \textit{žayy-in}), contrasting with inherited \textit{fəl} `go', \textit{asəd} `come' (\citealt{Kossmann2013book}: 284–285).

\pagebreak Prepositions are less frequently borrowed; in some cases where this does occur, however – including \ili{Igli} \textit{mənɣir-} `except', \ili{Ghomara} \textit{bin} `between' \citep[293]{Kossmann2013book} – they too occasionally retain \ili{Arabic} pronominal markers, e.g. Siwi \textit{msabb-ha} `for her' < \textit{min} \textit{sababi-hā}  `from reason.\textsc{obl}{}-\textsc{obl.3sg.f}' \citep[48]{Souag2013book}.  In \ili{Awjila}, more unusually, two inherited \isi{prepositions} somewhat variably take \ili{Arabic} pronominal markers, e.g. \textit{dit-ha} `in front of her' (\citealt{vanPutten2014}: 113).

A rarer but more spectacular example of morpheme borrowing is the borrowing of productive templates from \ili{Arabic}. Such cases include the \isi{elative} template əCCəC in Siwi, used to form the \isi{comparative} degree of triliteral adjectives irrespective of etymology – thus \textit{əmləl} `whiter' < \textit{a-məllal} alongside \textit{əṭwəl} `taller' < \textit{a-ṭwil} \textit{<} \ili{Arabic} \textit{ṭawīl} \citep{Souag2009} – and the \isi{diminutive} template CCiCəC in \ili{Ghomara} \citep{Mourigh2016}, e.g. \textit{aẓwiyyəṛ} `little \isi{root}' < \textit{aẓaṛ} alongside \textit{ləmwiyyəs} `little knife' < \textit{l-mus} < \ili{Arabic} \textit{al-mūsā} `razor' (gemination of \textit{y} is automatic in the environment i\_V). As the latter example illustrates, borrowed \isi{derivational} morphology sometimes becomes productive.

The effects of \ili{Arabic} on Berber morphology are by no means limited to the borrowing of morphemes. There is reason to suspect \ili{Arabic} influence of having played a role in processes of \isi{simplification} attested mainly in peripheral varieties, such as the loss of case marking in many areas. In Siwi, where \ili{Arabic} influence appears on independent grounds to be unusually high, the verbal system shows a number of apparent simplifications targeting categories absent in sedentary \ili{Arabic} varieties: the loss of distinct negative stems, the near-complete \isi{merger} of perfective with aorist, the fixed postverbal position of object clitics, and so on. It is tempting to explain such losses as arising from imperfect acquisition of Siwi by \ili{Arabic} speakers.

Structural \isi{calquing} in morphology is also sporadically attested. Siwi has lost distinct feminine plural \isi{agreement} on verbs, pronouns, and \isi{demonstratives}, extending the inherited masculine plural forms to cover plural \isi{agreement} irrespective of \isi{gender}. Within Berber, this is unprecedented; plural \isi{gender} \isi{agreement} is extremely well conserved across the family. However, it perfectly replicates the usual sedentary \ili{Arabic} system found in Egypt and far beyond.


 
 \subsection{Syntax}


Syntactic influence is often difficult to identify positively.  Nevertheless, Berber offers a number of examples, and \isi{relative} clause \isi{formation} is one of the clearest (\citealt{Souag2013book}: 151–156; \citealt{Kossmann2013book}: 369–407). Relative clauses in Berber are normally handled with a gap strategy combined with fronting of any stranded \isi{prepositions}, as in ‎(\ref{awjila}).

\ea
{\ili{Awjila} \citep[79]{Paradisi1961}}\\ \label{awjila}
\gll ərrafəqa-nnəs wi ižin-an-a nettin id-sin ksum\\
     friend\textsc{.pl}{}-\textsc{gen.}\textsc{3sg} \textsc{rel.pl.m} divide-\textsc{3pl.m-prf} 3\textsc{sg.m} with-\textsc{obl.}3\textsc{pl}.\textsc{m} meat\\
\glt `his friends with whom he divided the meat'
\z

In subject relativisation, a special form of the verb not agreeing in person (the so-called “\isi{participle}”) is used, as in (\ref{awj}); such a form is securely reconstructible for proto-Berber \citep{Kossmann2003}. 

\ea \label{awj}
{\ili{Awjila} \citep[162]{Paradisi1960}}\\
\gll amədən wa tarəv-ən nettin ʕayyan\\
     man \textsc{rel.sg.m} write.\textsc{ipfv-ptcp} \textsc{3sg.m} ill\\
\glt `The man who is writing is ill.'
\z

In several smaller easterly varieties apart from \ili{Awjila}, however, both of these traits have been lost. The strategy found in varieties such as Siwi – resumptive weak (affixal) pronouns throughout, and regular finite \isi{agreement} for subject relativisation – perfectly parallels \ili{Arabic}: 

\ea
\langinfo{Siwi}{}{\citealt{Souag2013book}: 151–152}\\
\gll tálti tən dəzz-ɣ{}-as ǧǧəwab\\
woman \textsc{rel.sg.f} send-\textsc{1sg-dat.3sg} letter\\
\glt `the woman to whom I sent the letter' \\
\ex 
{ Siwi (field data)}\\
\gll ággʷid wənn i-ʕəṃṃaṛ iməǧran\\
man \textsc{rel.sg.m} \textsc{3sg.m}{}-make.\textsc{ipfv} sickle.\textsc{pl}\\
\glt `the man who makes sickles' 
\z

In the case of verbal \isi{negation}, an originally syntactic \isi{calque} has often been morphologised in parallel in \ili{Arabic} and Berber. A number of varieties – especially the widespread \ili{Zenati} subgroup of Berber, ranging from eastern Morocco to northern Libya – have developed a postverbal negative \isi{clitic} \textit{-š(a)} from {*\'{k}ăra} `thing', apparently a \isi{calque} on \ili{Arabic} \textit{-š(i)} from \textit{šayʔ}; however, some instead use the direct borrowings \textit{ši} or \textit{šay} (\citealt{Lucas2007}; \citealt{Kossmann2013book}: 332–334).


 
 \subsection{Lexicon}


Lexical borrowing from \ili{Arabic} is pervasive in Berber. Out of 41 languages around the world compared in the Loanword Typology Project \citep{Tadmor2009}, \ili{Tarifiyt} Berber was second only to (Selice) \ili{Romani} in the percentage of \isi{loanwords} – more than half (51.7\%) of the concepts compared. More than 90\% of \isi{loanwords} examined in \ili{Tarifiyt} were from \ili{Arabic}, almost all from dialectal \ili{Maghrebi} \ili{Arabic}.  There is little reason to suppose that \ili{Tarifiyt} is exceptional in this respect among \ili{Northern Berber} languages; to the contrary, Kossmann (\citeyear[110]{Kossmann2013book}) finds its rate of basic vocabulary borrowing to be typical of \ili{Northern Berber}, whereas Siwi and \ili{Ghomara} go much higher. The rate of borrowing from \ili{Arabic}, however, is considerably lower further south and west; on a 200-word list of basic vocabulary, Chaker (\citeyear{Chaker1984}: 225–226) finds 38\% \ili{Arabic} loans in \ili{Kabyle} (north-central Algeria) vs. 25\% in \ili{Tashelhiyt} (southern Morocco) and only 5\% in Tahaggart \ili{Tuareg} (southern Algeria).

This borrowing is pervasive across the languages concerned, rather than being restricted to particular domains. Every semantic field examined for \ili{Tarifiyt}, including body parts, contained at least 20\% \isi{loanwords}, and verbs or adjectives were about as frequently borrowed as nouns were \citep{Kossmann2009}. Numerals stand out for particularly massive borrowing; most \ili{Northern Berber} varieties have borrowed all \isi{numerals} from \ili{Arabic} above a number ranging from `one' to `three' \citep{Souag2007}.

The effects of this borrowing on the structure of the lexicon remain insufficiently investigated, but appear prominent in such domains as \isi{kinship terminology}. Throughout \ili{Northern Berber}, a basic distinction between paternal kin and maternal kin is expressed primarily with \ili{Arabic} \isi{loanwords} (\textit{ʕammi} `paternal uncle' vs. \textit{ḫali} `maternal uncle' etc.), whereas in \ili{Tuareg} that distinction is not strongly lexicalised. Nevertheless, borrowing does not automatically entail lexical restructuring; \ili{Tashelhiyt}, for example, kept its vigesimal system even after borrowing the \ili{Arabic} word for `twenty' (\textit{ʕšrin}), cf. Ameur (\citeyear{Ameur2008}: 77).

The borrowing of analysable multi-word phrases – above all, \isi{numerals} followed by nouns – stands out as a rather common outcome of Berber contact with \ili{Arabic}. Usually this is limited to the borrowing of \isi{numerals} in combination with a limited set of measure words, such as `day'; thus in Siwi we find forms like \textit{sbaʕ-t iyyam} `seven days' rather than the expected regular \isi{formation} *\textit{səbʕa n nnhaṛ-at} \citep[114]{Souag2013book}. In \ili{Beni Snous} (western Algeria), the phenomenon seems to have gone rather further: Destaing (\citeyear[212]{Destaing1907}) reports that \isi{numerals} above `ten' systematically select for \ili{Arabic} nouns.  \citet{SouagKherbache2016}, however, explain this as a \isi{code-switching} effect, rather than a true case of one language's grammar requiring shifts into another.

\section{Conclusion}

The influence of \ili{Arabic} on Berber has come to be better understood over the past couple of decades, but much remains to be done.  Synchronically, Berber–\ili{Arabic} \isi{code-switching} remains virtually unresearched; rare exceptions include \citet{Hamza2007} and \citet{Kossmann2012}. Sociolinguistic methods could help us better understand the gradual integration of new \ili{Arabic} \isi{loanwords}; the early efforts of \citet{Brahimi2000} have hardly been followed up on. Diachronically, it remains necessary to move beyond the mere identification of \isi{loanwords} and contact effects towards a chronological ordering of different strata, an approach explored for some peripheral varieties by \citet{Souag2009} and \citet{vanPuttenBenkato2017}. While linguists are belatedly beginning to take advantage of earlier manuscript data to understand the history of Berber (\citealt{Boogert1997}; \citeyear{Boogert1998}; \citealt{Brugnatelli2011}; \citealt{Meouak2015}), this data has not yet been used in any systematic way to help date the effects of contact at different periods. For many smaller varieties, especially in the Sahara, basic documentation and description are still necessary before the influence of \ili{Arabic} can be explored. The unprecedented degree of \ili{Arabic} influence revealed in \ili{Ghomara} by recent work \citep{Mourigh2016}, extending to the borrowing of full verb paradigms, suggests that such descriptive work may yet yield dividends in the study of contact.

Despite all these gaps, the work done so far is more than sufficient to establish a general picture of \ili{Arabic} influence on Berber. Throughout \ili{Northern Berber}, \ili{Arabic} influence on the lexicon is substantial and pervasive, bringing with it significant effects on phonology and morphology. Structural effects of \ili{Arabic} on morphology, and \ili{Arabic} influence on Berber syntax, are less conspicuous but nevertheless important, especially in smaller varieties such as Siwi. Looking at these results through Van Coetsem's (\citeyear{VanCoetsem1988,VanCoetsem2000}) framework, this suggests that speakers dominant in the recipient languages have had an especially prominent role in \ili{Arabic}–Berber contact in larger varieties, whereas the role of speakers dominant in the \isi{source language} is more visible in smaller varieties.  However, this \textit{a} \textit{priori} conclusion should be tested against directly attested historical data wherever possible.

\section*{Further reading}

\begin{furtherreading}
\item The key reference for \ili{Arabic} influence on \ili{Northern Berber} is \citet{Kossmann2012}, frequently cited above; this covers all levels of influence including the lexicon, phonology, nominal and verbal morphology, borrowing of morphological categories, and syntax.
\item The most extensive in-depth study of \ili{Arabic} influence on a specific Berber variety is \citet{Souag2013book}, effectively a contact-focused grammatical sketch of Siwi Berber.
\item \citet{Mourigh2016} is a thorough synchronic description of by far the most strongly \ili{Arabic}-influenced Berber variety, \ili{Ghomara}, giving a uniquely clear picture of just how far the process can go without resulting in \isi{language shift}.
\end{furtherreading}

\section*{Acknowledgements}

The author thanks his consultants in \ili{Siwa}, especially the late Sherif Bougdoura, for their help with studying Siwi.

\section*{Abbreviations}
\begin{multicols}{2}
\begin{tabbing}
\textsc{ipfv} \hspace{1em} \= before common era\kill
\textsc{1, 2, 3} \> 1st, 2nd, 3rd person \\
\textsc{dat} \> dative \\
\textsc{f} \> feminine \\
\textsc{gen} \> genitive \\
\textsc{ipfv} \> imperfective  \\
\textsc{m} \> masculine \\
\textsc{obl} \> oblique \\
\textsc{pl}/pl. \> plural \\
\textsc{prf} \> perfect (suffix conjugation) \\
\textsc{ptcp} \> {participle} \\
\textsc{sg} \> singular
\end{tabbing}
\end{multicols}

\sloppy
\printbibliography[heading=subbibliography,notkeyword=this]
\end{document}
