\documentclass[output=paper]{langsci/langscibook}
\ChapterDOI{10.5281/zenodo.3744539}
\author{Dénes Gazsi}
\title{Iranian languages}
\abstract{Iranian languages, spoken from Turkey to Chinese Turkestan, have been in language contact with Arabic since pre-Islamic times. Arabic as a source language has provided phonological and morphological elements, as well as a plethora of lexical items, to numerous Iranian languages under recipient-language agentivity. \ili{New Persian}, the most significant member of this group, has been a prominent recipient of Arabic language elements. This study provides an overview of the historical development of this contact, before analyzing Arabic elements in \ili{New Persian} and other New Iranian languages. It also discusses how Arabic has influenced Modern Persian dialects, and how Persian vernaculars in the Persian Gulf region of Iran have incorporated Arabic lexemes from Gulf Arabic dialects.}


\begin{document}
\maketitle

\section{Current state and historical development}

\subsection{Iranian languages}

Iranian languages, along with \ili{Indo-Aryan} and \ili{Nuristani} languages, constitute the group of Indo-Iranian languages, which is a sizeable branch of the Indo-European language family. The term ``Iranian language'' has historically been applied to any language that descended from a proto-Iranian parent language spoken in  Asia in the late third to early second millennium BCE (\citealt{Skjærvø2012}).

Iranian languages are known from three chronological stages: Old, Middle, and New Iranian. Persian is the only language attested in all three historical stages. \ili{New Persian}, originally spoken in Fārs province, descended from Middle Persian, the language of the Sasanian Empire (third–seventh centuries CE), which is the progeny of Old Persian, the language of the Achaemenid Empire (sixth–fourth centuries BCE). \ili{New Persian} is divided into Early {Classical} (ninth–twelfth centuries CE), {Classical} (thirteenth–nineteenth centuries) and Modern Persian (from the nineteenth century onward), the latter considered to be based on the dialect of Tehran \citep[427]{Jeremiás2003}.

Today, Iranian languages are spoken from the Caucasus, Turkey, and Iraq in the west to Pakistan and Chinese Turkestan in the east, as well as in a large \isi{diaspora} in Europe and the Americas. New Iranian languages are divided into two main groups: {Western} and {Eastern} Iranian languages. The focus of this study is \ili{New Persian}, the most significant member among Iranian languages, but a brief overview of \ili{Arabic} influence on other New Iranian languages will also be provided. Below is a list of the most important members and their geographical distribution \citep[246]{Schmitt1989}.


\subsubsection{Western Iranian languages}



\subsubsubsection{Southwestern group}
Persian (\textit{Fārsī}) (spoken throughout Iran and adjacent areas), \ili{Tajik} (the variety of \ili{New Persian} in Central Asia), \ili{Darī} Persian (Afghanistan), \ili{Kumzārī} (Musandam Peninsula). Persian dialects in this group include \ili{Dizfūlī} (\ili{Khuzestan} province), \ili{Lurī} (ethnic group along the Zagros mountain range), \ili{Baḫtiārī} (nomadic tribe in the Zagros mountains), Fārs dialects (Fārs province), \ili{Lāristānī} dialects (Lāristān region of Fārs province), \ili{Bandarī} (dialects spoken around Bandar ʕAbbās in the Persian \ili{Gulf} region, to which \ili{Fīnī} also belongs).



\subsubsubsection{Northwestern group}
\ili{Kurdish}, \ili{Zazaki} (in eastern Turkey), \ili{Gurānī} (in eastern Iraq and western Iran), Balūčī (\ili{Balochi}, spoken chiefly in Iranian and Pakistani Baluchistan, and parts of Oman). Non-literary languages and dialects: \ili{Tātī}, \ili{Tālišī} and \ili{Gīlakī} (on the shores of the Caspian Sea), Central dialects (spoken in a vast area between Hamadān, Kāšān and Iṣfahān), \ili{Kirmānī} (south of the Dašt-i Kawīr).



\subsubsection{Eastern Iranian languages}

\subsubsubsection{Southeastern group}
Pashto (Afghanistan, Pakistan, eastern border region of Iran), \ili{Pamir} languages (\ili{Pamir} Mountains along the Pānj River).

\subsubsubsection{Northeastern group}
\ili{Yaɣnōbi} (Zarafšān region of Tajikistan), \ili{Ossetic} (central Caucasus).

\subsection{Historical development of Arabic–Persian language contact} %1.2. /

Language contact between \ili{Arabic} and Persian has been a \isi{reciprocal} process for the past 1500 years. During the pre-Islamic and early Islamic era (sixth–seventh centuries CE), Middle Persian, being embedded in the well-established and sophisticated Iranian culture, provided many \isi{loanwords} to pre-\ili{Classical} and \ili{Classical} \ili{Arabic} (\citealt{Gazsi2011}: 1015; see also van Putten, this volume)\ia{van Putten, Marijn@van Putten, Marijn} under \isi{RL} (\isi{recipient-language}) agentivity (\citealt{VanCoetsem1988,VanCoetsem2000}). With the collapse of the Sasanian Empire and expansion of Islam and the \ili{Arabic} language over vast territories outside Arabia, \ili{Classical} \ili{Arabic} began to exercise an unprecedented impact on the emerging \ili{New Persian} language. \ili{Arabic} never took \isi{root} in the everyday communication of the ethnically Persian population, although it gained some dominance as a written vehicle in the administrative, theological, literary and scientific domains in the eastern periphery of the Abbasid Caliphate. Instead, spoken Middle Persian (\textit{Darī}) flourished as a vernacular language. In the middle of the ninth century CE, it was in this part of Iran, specifically in Fārs province, that \textit{Darī} emerged in a new form as it repositioned itself in the culture and literature of the local populace. This new literary language, the revitalization of the Persian linguistic heritage, would be called \ili{New Persian}. Since its earliest phase, \ili{New Persian} has borrowed a staggering number of \isi{loanwords}. Initially, these \isi{loanwords} were borrowed from various northwestern and eastern Iranian languages, such as \ili{Parthian} and \ili{Sogdian}. Despite this relatively large group of loans, the most versatile lenders were the Arabs. Whereas in the pre-Islamic era \ili{Arabic} had almost exclusively taken lexical items from Middle Persian (in the fields of religion, botany, science and bureaucracy among others), \ili{New Persian} also incorporated \ili{Arabic} morphosyntactic elements.

The first \ili{Arabic} \isi{loanwords} began to permeate \ili{New Persian} in the ninth–tenth centuries CE (20–30\%). This process was not even diminished by the Iranian \textit{šuʕūbiyya} movement, the major output of which was all conducted in \ili{Arabic}. In subsequent centuries, Persian continued to absorb an ever-expanding set of \ili{Arabic} lexemes. By the turn of the twelfth century, the proportion of \ili{Arabic} loans increased to approximately 50\%. The majority of \ili{Arabic} loans had already been integrated into \ili{New Persian} by that time and have shown a remarkable steadiness until recently.

After the fall of Baghdad in 1258 CE, \ili{Arabic} lost its foothold in the eastern provinces of the Caliphate, thereby drawing the final boundary between the use of \ili{Arabic} and Persian \citep{Danner2000}. The Mongol Ilkhānids, who as non-\isi{Muslims} were not dependent on \ili{Arabic}, introduced Persian as the language of education and administration in Iran and \isi{Anatolia}. Despite the significant destruction the Mongols caused to northern Iran during their conquest, this period (thirteenth and fourteenth centuries CE) is considered to be the zenith of Persian literature. This is also the epoch when literary Persian is, in an excessive way, inundated with \ili{Arabic} language elements. This phenomenon is easily detectable in the works of one of the most significant personalities in \ili{Classical Persian} literature, and a pre-eminent poet of thirteenth-century Persia, Saʕdī of Shiraz. Following the norms of Persian prose writing and poetry of his time, Saʕdī flooded his writings with a bewildering array of \ili{Arabic} language elements. To illustrate this, here is a typical sentence from Saʕdī’s \textit{Gulistān} ‘Rose Garden’ (completed in 1258 CE), where words of \ili{Arabic} origin are highlighted in boldface (\citealt{Yūsifī2004}: 77).


\ea
\RL{\textarabic{
اگر راى عزيز فلان ، أحسن الله خلاصه ، به جانب ما التفات كند در رعايت خاطرش هرچه تمامترسعى كرده شود واعيان اين مملكت به ديدار او مفتقرند و جواب اين حروف را منتظر
}}\\
{\itshape agar \textbf{rāy}-i \textbf{ʕazīz}-i \textbf{fulān}, \textbf{aḥsana allāhu ḫalāṣahu}, ba \textbf{ǧānib}-i mā \textbf{iltifāt} kunad dar \textbf{riʕāyat}-i \textbf{ḫāṭir}aš har či \textbf{tamām}tar \textbf{saʕī}  karda  šawad wa \textbf{aʕyān}-i īn \textbf{mamlakat} ba dīdār-i ū \textbf{muftaqir}and wa \textbf{ǧawāb}-i īn \textbf{ḥurūf} rā \textbf{muntaẓir}.}\\
\glt ‘If the precious mind of that person, may God make the end of his affairs prosperous, were to look in our direction, the utmost efforts would be made to please him, because the nobles of this realm would consider it an honor to see him, and are waiting for a reply to this letter.’\footnote{Persian transcription in this chapter follows the \ili{Arabic} phonological conventions to avoid using two disparate systems.}
\z

It is easy to ascertain that, apart from verbs and adverbs, almost every lexical item in the sentence is of \ili{Arabic} origin. But writers of this era, such as Saʕdī, not only inundated their works with \ili{Arabic} elements, but even used \ili{Arabic} morphology and semantics freely by coining new and innovative meanings, e.g. \textit{ṣaʕqa} ‘lightning’ < \ili{MSA}/\ili{MSP} \textit{ṣāʕiqa} or \textit{baṭṭāl} ‘liar’ < \ili{MSA}/\ili{MSP} ‘inactive, unemployed person’,\footnote{In this chapter, references are made to Modern Standard \ili{Arabic} (\ili{MSA}) and Modern Standard Persian (\ili{MSP}) as a comparison to dialectal forms in both languages. This seemed more straightforward as it is not always feasible to determine at what point in time a lexeme was borrowed from \ili{Arabic} into Persian.} < \ili{MSA} \textit{mubṭil} ‘liar’. The Persian and \ili{Arabic} language use of Saʕdī and other literary figures in the \ili{Classical Persian} period came closest to what \citet{Lucas2015} calls \isi{convergence} under the language dominance principle. As reflected in the purely \ili{Arabic} and \ili{Arabic}-infused Persian segments of his oeuvre, Saʕdī was equally dominant in both \ili{Classical} \ili{Arabic} and \ili{Classical Persian} along with the dialect of Shiraz.

Modern Persian is still deeply rooted in \ili{Arabic}. \ili{Arabic} \isi{loanwords} constitute more than 50\% of its vocabulary, but in elevated styles (religious, scientific, literary) \ili{Arabic} loans may exceed 80\% \citep{Jeremiás2011}. Although the proportion of these \isi{loanwords} fluctuates according to age, genre, social context or idiolect, any style in Modern Persian deprived of \ili{Arabic} influence is almost impossible. An endeavor similar to \iai{Atatürk}’s to purge \ili{Turkish} of foreign language elements would be unrealistic in Modern Persian, even with recurring efforts by linguistic purists and the Academy of Persian Language and Literature (\textit{Farhangistān-i} \textit{zabān} \textit{wa} \textit{adab-i} \textit{fārsī}).\footnote{An example of their activity is the publication (by \citealt{Rāzī2004}) of a dictionary that lists “pure” Persian words.} It is noteworthy that when the need arose for new terminology to describe fledgling political concepts in Iran, for instance during the Constitutional Revolution in the early twentieth\textsuperscript{} century, as \citet{Elwell-Sutton2000} phrased it, “politicians and journalists instinctively turned to \ili{Arabic} rather than Persian”. Frequently, however, these “\ili{Arabic}” words were new coinages in the \isi{recipient language}, e.g. \textit{mašrūṭa} ‘constitution’, \textit{mawqiʕiyyat} ‘situation, position’. After the Islamic Revolution in 1979, another wave of \ili{Arabic} lexemes related to the new religious governing system surfaced, e.g. \textit{mustaẓʕifīn} ‘the needy, the enfeebled’ (< \ili{MSA} \textit{mustaḍʕafūna}\kern 0.5pt/\kern -1pt\textit{mustaḍʕafīna}).

Primary and secondary schools in contemporary \ili{Arabic}-speaking countries do not offer language education in Persian. In Iran, compulsory \ili{Classical} \ili{Arabic} instruction is part of the curriculum. However, the language is taught for religious purposes only, with no intention to utilize \ili{MSA} as a means of acquiring communication skills.

\section{Contact languages}

This section briefly describes the linguistic impact of \ili{Standard} \ili{Arabic} on several New Iranian languages. A more detailed analysis of contact-induced \isi{language change} in \ili{New Persian} will follow in §\ref{dial}.

\subsection{Arabic influence on New Iranian languages}

\subsubsection{Tajik (\textit{Tōǧīkī})}

\ili{Tajik}, written in a modified Cyrillic script, is the variety of \ili{New Persian} spoken throughout Central Asia, most notably in Tajikistan, \ili{Uzbekistan}, and northern Afghanistan. Similarly to all varieties of Persian, \ili{Arabic} borrowings constitute the earliest layer of foreign vocabulary in \ili{Tajik} \citep{Perry2009}. This lexicon was transferred under \isi{RL} agentivity. Although \ili{Arabic} lexical items have a firm hold in \ili{Tajik}, their pattern of distribution differs from that of \ili{New Persian}. For instance, \ili{Tajik} uses \textit{pēš} ‘before’ and \textit{pas} ‘after’ rather than \ili{MSA}/\ili{MSP} \textit{qabl} and \textit{baʕd}, but \textit{ōid} \textit{ba}{}-/\textit{ōid-i} (< \ili{MSA} \textit{ʕāʔid} ‘returning’) ‘concerning, relating to’ in lieu of \ili{MSP} \textit{rāǧiʕ} \textit{ba}{}- (< \ili{MSA} \textit{rāǧiʕ} ‘recurring’). Also, \textit{madaniyyat} ‘civilization’ (<~\ili{MSA} \textit{madaniyya} ‘civilization’; cf. \ili{MSA}/\ili{MSP} \textit{tamaddun} ‘civilization’), \textit{hōzir} ‘now’ (<~\ili{MSA} \textit{ḥāḍir} ‘present; ready’, \ili{MSP} \textit{ḥāẓir} ‘present’), \textit{ittifōq} ‘(labor) union’ (< \ili{MSA} \textit{ittifāq} ‘\isi{agreement}; contract’; cf.  \ili{MSP} \textit{ittiḥād} ‘[labor] union’).

\ili{Arabic} plural forms, both sound feminine plural and \isi{broken plural}, were lexicalized with \isi{collective} or singular meanings: \textit{hašarōt} ‘insect’, with regular plural ending \textit{hašarōthō} ‘insects’ (< \ili{MSA}/\ili{MSP} \textit{ḥašarāt} ‘insects’), \textit{talaba} ‘student’, pl. \textit{talabagōn} (< \ili{MSA}/\ili{MSP} \textit{ṭalaba} ‘students’), \textit{šarōit} ‘condition, stipulation’ (<~\ili{MSA}\slash\ili{MSP} \textit{šarāʔiṭ} ‘conditions’).


\subsubsection{Kurdish}

A characteristic feature of \ili{Kurdish}, the change of postvocalic /m/ > /v/ or /w/, also occurs frequently in words of \ili{Arabic} origin: \textit{silāv} ‘greeting’ (< \ili{MSA}/\ili{MSP} \textit{salām}; \citealt{Paul2008}).

\subsubsection{Gurānī}

The phonological system of \ili{Gurānī} dialects is similar to \ili{Kurdish} in the occurrence of \ili{Arabic} \isi{pharyngeal} and \isi{emphatic} sounds /ʕ/, /ḥ/, /ṣ/ \citep{MacKenzie2012}.

\subsubsection{Ossetic}
\ili{Ossetic} has incorporated terms related to Islam from \ili{Arabic} and Persian through neighboring Caucasian languages \citep{Thordarson2009}.

\subsection{Arabic-speaking communities in Iran}
\ili{Arabic}-speaking communities are known to be present within the boundaries of the Islamic Republic of Iran, but their exact number is not readily discernible from official statistics. It is estimated that 3\% of Iran’s 80 million citizens are Arabs, which would put the Arab population at approximately 2.5 million. The majority of Arabs live in the western parts of \ili{Khuzestan} province (see Leitner, this volume),\ia{Leitner, Bettina@Leitner, Bettina} but also along Iran’s Persian \ili{Gulf} coast and parts of Khorasan in eastern Iran \citep{Oberling2011}. Already during the Sasanian era, several Arab tribes, including the Bakr ibn Wāʔil and Banū Tamīm, settled in the area stretching from the Šaṭṭ al-ʕArab to the Zagros Mountains \citep{Daniel2011}. At the end of the sixteenth century, the Banū Kaʕb, originating from present-day Kuwait, settled in \ili{Khuzestan}. During subsequent centuries, more Arab tribes moved from southern Iraq to the province. As a result, \ili{Khuzestan}, which until 1925 was called ʕArabistān, became extensively Arabized. Members of these Arab tribes live on either side of the Iran–Iraq border. In the same way as \ili{Iraqi} \ili{Arabic} vernaculars, \ili{Khuzestan} \ili{Arabic} has been influenced by Persian. However, \ili{Khuzestan} \ili{Arabic} can most easily be distinguished from \ili{Iraqi} dialects by its wide-ranging \isi{transfer} of Persian lexemes (\citealt{Ingham1997}: 25; see also Leitner, this volume).\ia{Leitner, Bettina@Leitner, Bettina}

Arab presence has a well-documented history on the Iranian coastline of the Persian \ili{Gulf}, in what now constitutes Būšihr and Hurmuzgān provinces. According to travelogues from the eighteenth to the twentieth centuries CE, as well as British archival materials dating back to the British Residency of the Persian \ili{Gulf}, Arab tribes inhabited most fishing and pearling villages, as well as islands and coastal towns with strategic importance (e.g. Bandar ʕAbbās). The most significant tribes in this area were, and in some cases still are, the Qawāsim, Marāzīq, Āl Ḥaram, Āl ʕAlī, Āl Naṣūr, Banī Tamīm, Banī Ḥammād, Banī Bišr, among others. In contrast to most Persians and Khuzestani Arabs who are primarily Shiite, these tribes are Sunni \isi{Muslims}. A widespread exonym to designate Arabs on the Iranian coast, but shunned by the local population, is \textit{hōla} (variously referred to as \textit{hula}, \textit{huwala} or \textit{hawala}). Local tribes prefer the endonym ‘Arabs of the Coast’ (\textit{ʕarab} \textit{as-sāḥil}) \citep[110]{Gazsi2017}.

Most Khuzestani and Iranian Persian \ili{Gulf} Arabs are bilingual, speaking \ili{Arabic} as their mother tongue and Persian as a second language. Although \ili{Khuzestan} and the two Persian \ili{Gulf} provinces are geographically part of Iran, linguistically their Arab populations form a continuum with the southern \ili{Mesopotamian} Muslim \textit{gilit}{} dialects, and the dialects of the eastern coast of the Arabian Peninsula, respectively. In the spoken and written code, ‘Arabs of the Coast’ often engage in tetra-glossic switching between \ili{MSA}, \ili{Gulf} \ili{Arabic} (GA), \ili{MSP}, Colloquial Persian and one of its local dialects such as \ili{Bandarī}. In their speech, Persian phonological and lexical elements are borrowed into \ili{GA} under \isi{RL} agentivity.

\section{Contact-induced changes in \ili{New Persian} and modern Persian dialects} \label{dial}
Language contact between \ili{Arabic} and \ili{New Persian} is most evidently detectable in the \ili{New Persian} lexicon, and to a lesser extent in phonology and morphosyntax. This section summarizes the characteristics of this contact. In addition to standard \ili{New Persian}, and its contemporary variant \ili{MSP}, \ili{Arabic} has also influenced modern Persian dialects. This influence is slightly different, and in several ways more far-reaching, particularly in the realm of phonology and lexicon.

Persian dialects developed separately from and parallel to \ili{Classical Persian} and \ili{MSP}. Modern Persian dialects retain several Early {Classical} and \ili{Classical Persian} phonological and morphosyntactic features that are not present in \ili{MSP}. Additionally, they were in direct contact with the \ili{Arabic} language through Arab tribes that settled across Persia immediately after the Islamic conquest or in later centuries. Although most Arab tribes have long been integrated into the Persian-speaking population, the \ili{Arabic} language in the areas currently dominated by ethnic Arabs is still in contact with the surrounding Persian dialects. Unlike \ili{Arabic} influence on the standard version of \ili{New Persian}, \ili{Arabic} influence on modern Persian dialects is an understudied field that does not allow for providing an exhaustive list of contact-induced changes at this point. Instead, below is a preliminary description of salient examples of how \ili{Arabic} phonological and lexical elements were transferred to \ili{New Persian}, both its standard and dialectal variations.

\subsection{Phonology}


\subsubsection{\ili{New Persian}}

The initial step in the adoption of \ili{Arabic} lexemes was the adoption of the \ili{Arabic} script. \ili{New Persian} began to use a modified \ili{Arabic} script in the ninth century CE; it has 32 letters, 28 acquired from \ili{Arabic} and 4 new letters added to represent Persian phonemes (/p/, /č/, /ž/, /g/). \ili{Arabic} /θ/ and /ṣ/ collapse to Persian /s/, \ili{Arabic} /ð/, /ḍ/, /ð̣/ collapse to Persian /z/, and \ili{Arabic} /ṭ/ becomes Persian /t/. The phonemic inventory of Early \ili{Classical Persian} was augmented with the glottal stop, which originated in the two separate \ili{Arabic} phonemes /ʔ/ and /ʕ/.

\subsubsection{Modern Persian dialects}

This section highlights phonological features of modern Persian dialects that were the result of contact-induced \isi{language change} under \isi{RL} agentivity, either with \ili{Arabic} or with \ili{Classical Persian}, and subsequently \ili{MSP}.

\subsubsubsection{Adoption of Arabic pharyngeal sounds}

The two \ili{Arabic} \isi{pharyngeal} sounds undergo phonological integration in \ili{New Persian}: the voiceless \isi{pharyngeal} fricative /ḥ/ is pronounced as a voiceless glottal fricative /h/, and the voiced \isi{pharyngeal} fricative /ʕ/ as a glottal stop /ʔ/. The dialects of Dizfūl and Šūštar have acquired \isi{pharyngeal} sounds directly from \ili{Arabic}, which occur in \ili{Arabic} \isi{loanwords}: \textit{ʕaǧīb} ‘strange’, \textit{baʕd} ‘after’ \citep{MacKinnon2015}. The dialect of Jarkūya shares this feature: \textit{ḥasüd} ‘jealous’, \textit{ǧimʕa} ‘Friday’ \citep{Borjian2008}.

The dialect of Kulāb in Tajikistan also borrows \ili{Arabic} \isi{pharyngeal} sounds in words of \ili{Arabic} origin: \textit{ʕaib} ‘flaw’, \textit{daʕvō} ‘claim’, \textit{mıʕalim} ‘teacher’, \textit{ḥıkımat} ‘wisdom’, \textit{sōḥib} ‘owner’. \ili{Arabic} \isi{pharyngeal} sounds also occur in a few Persian/Tajiki words (\textit{ʕasp} ‘horse’, \textit{ḥamsōya} ‘neighbor’). Interestingly, the \isi{pharyngealized} form for ‘horse’ occurs far and wide within the Iranian linguistic domain, as \textit{ʕasb} in the \ili{Lurī} dialect of Šūštar, in Ḫānsāri\il{Hansari@Ḫānsāri} and Caucasian \ili{Tātī}. In the Arab \ili{Gulf} states, the \textit{ʕAǧam}, ethnic Persians holding Kuwaiti, Emirati and other \ili{Gulf} citizenship, pronounce \ili{Arabic} \isi{loanwords} in their Persian speech with \isi{pharyngeal} sounds.

\subsubsubsection{Dropping of Arabic pharyngeal sounds}

In several modern Persian dialects, the voiceless \isi{pharyngeal} fricative /ḥ/ is absent. The preceding vowel is lengthened or the subsequent vowel disappears too, e.g. \textit{mūtāǧ} ‘in need, destitute’\,<\,\ili{MSA}/\ili{MSP} \textit{muḥtāǧ} (\citealt{Īzadpanāh2001}: 190), \textit{ṣārā} ‘desert’\,<\,\ili{MSA}/\ili{MSP} \textit{ṣaḥrā} \citep[15]{Sarlak2002}, \textit{ṣāb} ‘owner’\,<\,\ili{MSA}/\ili{MSP} \textit{ṣāḥib} (\citealt{Ṣarrāfī1996}: 135), \textit{mulāẓa} ‘consideration, observation’\,<\,\ili{MSP} \textit{mulāḥiẓa}, cf. \ili{MSA} \textit{mulāḥað̣a} (\citealt{Ṣarrāfī1996}: 188), \textit{ṣul} ‘peace’\,<\,\ili{MSA}/\ili{MSP} \textit{ṣulḥ} \citep{Stilo2001}, \textit{ēsās} ‘feeling’\,<\,\ili{MSA}/\ili{MSP} \textit{iḥsās} (\citealt{Salāmī2004}: 160–161). In Kirmān, the sound change /uḥ/\,> /ā/ is attested, e.g. \textit{fāš} ‘insult’\,<\,\ili{MSA}/\ili{MSP} \textit{fuḥš} \citep{Borjian2017}.

The voiced \isi{pharyngeal} fricative /ʕ/, pronounced as a glottal stop in \ili{MSP}, can also be dropped. This may result in vowel lengthening: \textit{māṭal} ‘idle’\,<\,\ili{MSA}/\ili{MSP} \textit{muʕaṭṭal} (\citealt{Ṣarrāfī1996}: 184), \textit{māmila} ‘transaction’\,<\,\ili{NewP} \textit{muʕāmila}, cf. \ili{MSA}\linebreak \textit{muʕāmala} (\citealt{Ṣarrāfī1996}: 184; \citealt{Sarlak2002}: 15), \textit{rubbi} \textit{sāt} ‘quarter hour’\,<\,\ili{MSP} \textit{rubʕ} \textit{sāʕat}, cf. \ili{MSA} \textit{rubʕ} \textit{sāʕa} (\citealt{Ṣarrāfī1996}: 108), \textit{mānī} ‘meaning’\,<\,\ili{MSP} \textit{maʕnī}, cf. \ili{MSA} \textit{maʕnā} \citep[15]{Sarlak2002}, \textit{mōǧiza} ‘miracle’\,<\,\ili{MSA}/\ili{MSP} \textit{muʕǧiza} (\citealt{Īzadpanāh2001}: 190), \textit{tāǧub} \~{} \textit{tāǧuv} ‘surprise, wonder’\,<\,\ili{MSA}/\ili{MSP} \textit{taʕaǧǧub} (\citealt{Salāmī2004}: 162–163), \textit{rāyat} ‘regard’\,<\,\ili{MSP} \textit{riʕāyat}, cf. \ili{MSA} \textit{riʕāya} (\citealt{Ṣarrāfī1996}: 107).

\subsubsubsection{Dropping of the Arabic voiceless glottal fricative /h/}

The voiceless glottal fricative disappears in closed syllables in many Persian dialects, resulting in occasional vowel lengthening: \textit{ṭārat} ‘cleanliness’ < \ili{MSP} \textit{ṭahārat}, cf. \ili{MSA} \textit{ṭahāra} \citep[76]{Sarlak2002}, \textit{nāal} ‘impolite’ < \ili{MSP} \textit{nāahl} (\citealt{Īzadpanāh2001}: 192).

\subsubsubsection{Miscellaneous sound changes}

A range of additional consonant developments and shifts can be attested in Persian dialects. Some of these developments include:

\begin{altdescription}
\item[/ʕ/\,>\,/ḥ/:] In \ili{Lurī} and the dialect of Jarkūya, a shift occurs from the voiced to the voiceless \isi{pharyngeal}: \textit{ḥilāǧ} ‘cure’\,<\,\ili{MSA}/\ili{MSP} \textit{ʕilāǧ} (\citealt{Īzadpanāh2001}: 207), \textit{ṭaḥna} ‘sarcasm’\,<\,\ili{MSA}/\ili{MSP} \textit{ṭaʕna} \citep{Borjian2008}.

\item[/ḥ/\,>\,/ʔ/ occurring with occasional metathesis:] \textit{ṭaʔr} ‘plan’\,<\,\ili{MSA}/\ili{MSP} \textit{ṭarḥ} (\citealt{Ṣarrāfī1996}: 137), \textit{maʔla} ‘city quarter’\,<\,\ili{MSA}\slash\ili{MSP} \textit{maḥalla} (\citealt{Ṣarrāfī1996}: 188), \textit{maʔala} ‘city quarter’ (\citealt{NaǧībiFīni2002}: 133).

\item[/h/\,>\,/ʔ/:] \textit{muʔlat} ‘deadline, respite’\,<\,\ili{MSP} \textit{muhlat}, cf. \ili{MSA} \textit{muhla} (\citealt{Ṣarrāfī1996}: 190).

\item[/θ/\,>\,/t/:] This shift is also common in several \ili{Arabic} dialects, e.g. in Egypt and Morocco: \textit{mīrāt} ‘heritage’\,<\,\ili{MSP} \textit{mīrās}, cf. \ili{MSA} \textit{mīrāθ} (\citealt{Īzadpanāh2001}: 190).

\item[Word-final /b/ and /f/\,>\,/m/:] \textit{naǧīm} ‘noble’\,<\,\ili{MSA}/\ili{MSP} \textit{naǧīb} (\citealt{Īzadpanāh2001}: 193), \textit{niṣm} ‘half’\,<\,\ili{MSA}\slash\ili{MSP} \textit{niṣf} (\citealt{Īzadpanāh2001}: 195).

\item[/r/\,>\,/l/:] in Kirmān, \textit{zilar} \~{} \textit{zilal} ‘damage, loss’\,<\,\ili{MSP} \textit{ẓarar}, cf. \ili{MSA} \textit{ḍarar} (\citealt{Ṣarrāfī1996}: 136; \citealt{Dānišgar1995}: 163), \textit{ḥaṣīl} ‘straw mat’\,<\,\ili{MSA}/\ili{MSP} \textit{ḥaṣīr} (\citealt{Ṣarrāfī1996}: 85), \textit{qulfa} ‘small room for summer resting’\,<\,\ili{MSA} \textit{ɣurfa} ‘room’ (\citealt{Fāẓilī2004}: 151).

\item[\ili{Arabic} voiceless dental \isi{emphatic} /ṭ/\,>\,/d/:] \textit{mudbaḫ} \~{} \textit{madbaḫ} ‘kitchen’ < \ili{MSA} \textit{maṭbaḫ} (\citealt{Ṣarrāfī1996}: 186; not attested in \ili{MSP}), \textit{mudbaq} in \ili{Baḫtiārī} \citep[251]{Sarlak2002}.

\item[/b/\,>\,/f/:] \emph{muftilā} ‘afflicted’\,<\,\ili{MSP} \textit{mubtilā}, cf. \ili{MSA}  \textit{mubtalā} \citep{Borjian2017}.

\item[Medial and word-final /b/ > /v/:]\sloppy in \ili{Baḫtiārī}, \textit{ādāv} ‘customs’ < \ili{MSA}\slash\ili{MSP} \textit{ādāb} \citep[15]{Sarlak2002}, \textit{ʕajīv} ‘strange’ < \ili{MSA}\slash\ili{MSP} \textit{ʕajīb} \citep[25]{Sarlak2002}, \textit{qavīla} ‘tribe’~< \ili{MSA}/\ili{MSP} \textit{qabīla} \citep[199]{Sarlak2002}.

\item[Word-initial /ḫ/ > /h/:] in northern \ili{Lurī} and \ili{Baḫtiārī}, \textit{hāla} ‘aunt’ < \ili{MSA}\slash\ili{MSP} \textit{ḫāla} (\citealt{Īzadpanāh2001}: 204).

\item[/q/\,>\,/k/:] \textit{kabīla} ‘tribe’\,<\,\ili{MSA}/\ili{MSP} \textit{qabīla} (\citealt{NaǧībiFīni2002}: 21).

\item[/ɣ/\,>\,/q/:] \textit{šuql} ‘occupation’\,<\,\ili{MSP}/\ili{MSA} \textit{šuɣl} \citep{Stilo2001}.

\item[/ǧ/\,>\,/y/:] direct borrowing from \ili{Khuzestan} \ili{Arabic} dialects, \textit{mailis} ‘council’~< \ili{MSA}\slash\ili{MSP} \textit{maǧlis} (\citealt{Sarlak2002}: 260; \citealt{Fāẓilī2004}: 165).

\item[Metathesis:] \textit{qulf} ‘lock’\,<\,\ili{MSA}/\ili{MSP} \textit{qufl} (\citealt{Salāmī2004}: 84–85; \citealt{ImāmAhwāzī2000}: 146), \textit{ṣuḥb} ‘morning’\,<\,\ili{MSA}/\ili{MSP} \textit{ṣubḥ} (\citealt{Dānišgar1995}: 161; \citealt{NaǧībiFīni2002}: 23).
\end{altdescription}

The full /t/ of the \textit{tāʔ} \textit{marbūṭa} appears on words where it is absent in \ili{MSP}: \textit{ḥalmat} ‘attack’ < \ili{MSA}/\ili{MSP} \textit{ḥamla} (\citealt{Īzadpanāh2001}: 207), \textit{ḥaǧāmat} ‘cupping’ < \ili{MSA}/\ili{MSP} \textit{ḥaǧāma} (\citealt{Salāmī2004}: 92–93). This was a typical feature of \ili{Classical Persian} literature.

\subsection{Morphosyntax}

Several \ili{Arabic} morphosyntactic features were transferred to \ili{New Persian} in the realm of nominal morphology under \isi{RL} agentivity. These features encompass sound and \isi{broken plural} forms (\textit{musāfirīn} ‘passengers’, \textit{tablīɣāt} ‘propaganda’, \textit{dihāt} ‘villages’, \textit{ḥuqūq} ‘rights’), possessive constructions (\textit{fāriɣ} \textit{ut-taḥṣīl} ‘graduate’, \textit{wāǧib} \textit{ul-iǧrā} ‘peremptory’) and occasional \isi{gender} \isi{agreement} in lexicalized expressions (\textit{quwwa-yi} \textit{darrāka} ‘perceptive power’). Word \isi{formation} has been an active method of transferring \ili{Arabic} lexical elements into \ili{New Persian} from early on, either by way of \isi{derivation} (\textit{diḫālat} ‘interference’ < \ili{MSA} \textit{mudāḫala}, \textit{awlā-tar} ‘superior’ < \ili{MSA} \textit{awlā}, \textit{raqṣīdan} ‘to dance’ < \ili{MSA} \textit{raqṣ}, \textit{aks̱aran} ‘most, generally’ < \ili{MSA} \textit{akθar} ‘more, most’) or compounding. Compounding is a highly developed process of enlarging the \ili{New Persian} vocabulary. It is manifest in lexical \isi{compounds} (\textit{taɣẕia-šinās} ‘nutritionist’, \textit{ḫiānat-kārāna} ‘perfidiously’) and phrasal \isi{compounds} (\textit{iṭāʕat} \textit{kardan} ‘to obey’, \textit{ʕadam-i} \textit{wuǧūd} ‘non-existence’, \textit{ʕala} \textit{l-ḫuṣūṣ} ‘particularly’).

\subsection{Lexicon}

\subsubsection{Arabic lexicon in \ili{New Persian}}

Contact-induced \isi{language change} manifests itself most strikingly in the lexicon transferred from \ili{Arabic} to \ili{New Persian} under \isi{RL} agentivity. The earliest \isi{loanwords} entered \ili{New Persian} during the ninth–tenth centuries. This process occurred smoothly, as the phonological inventory of Early \ili{Classical Persian} was likely close to that of Middle Persian and also close to that of \ili{Classical} \ili{Arabic}.\footnote{In Early \ili{Classical Persian}, short vowels were likely pronounced as /u/ and /i/, and the \textit{alif} as /ā/. In \ili{MSP}, the pronunciation is /o/, /e/ and /ɒ/.} The influx of \ili{Arabic} \isi{loanwords} has unabatedly continued over the centuries until now. To showcase a recent example of \ili{Arabic} vocabulary in Modern Persian, below are titles of articles from \textit{Hamšahrī} ‘Fellow Citizen’, a major Iranian national newspaper, taken from its 29th January 2018 edition. \ili{Arabic} words are highlighted in boldface:

\ea
\ea
\gll \textbf{kulliyyāt}-i \textbf{lāyiḥa}-yi būdǧa-yi sāl-i 97-i \textbf{kull}-i kišwar \textbf{radd} šud\\
total.\textsc{pl-gen} bill-\textsc{gen} budget-\textsc{gen} year-\textsc{gen} 97-\textsc{gen} whole-\textsc{gen} country reject be.\textsc{pst.3sg}\\
    \glt ‘The total budget bill of the year 2018 for the whole country was rejected.’

\ex
\gll \textbf{daʕwat} {az} {tihrānīhā} barā-yi \textbf{ihdā}-yi ḫūn \textbf{asāmī}-yi \textbf{marākiz}-i \textbf{faʕʕāl}\\
call from Tehrani.\textsc{pl} for-\textsc{gen} donation-\textsc{gen} blood name.\textsc{pl-gen} center.\textsc{pl-gen} active\\
    \glt ‘Calling the residents of Tehran to donate blood. Names of active centers.’

\ex
\gll \textbf{iḥrāz}-i \textbf{huwiyyat} dar \textbf{muʕāmilāt}-i \textbf{milkī} {bā} {kārt-i} hūšmand-i \textbf{millī} {anǧām} {mī-šaw-ad}\\
authentication-\textsc{gen} \isi{identity} in transaction.\textsc{pl-gen} proprietary with card-\textsc{gen} smart-\textsc{gen} national complete \textsc{prs}-be-\textsc{3sg}\\
    \glt‘Personal authentication in real estate transactions is done with the national smart card.’
\z
\z


In the \ili{Arabic} lexicon of \ili{New Persian}, further characteristics can be observed, such as phonetic changes (\ili{NewP} \textit{maʔnī} ‘meaning’ < \ili{MSA} \textit{maʕnā}, \ili{NewP} \textit{madrisa} ‘school’ < \ili{MSA} \textit{madrasa}, \ili{NewP} \textit{šikl} ‘shape, form’ < SA \textit{šakl}), where in some cases the Persian pronunciation may follow \ili{Arabic} dialectal forms, semantic changes (\ili{NewP} \textit{kitābat} ‘writing’ and \textit{kitāba} ‘inscription’ < \ili{MSA} \textit{kitāba} ‘writing’, \ili{NewP} \textit{ṣuḥbat} ‘speech’ < \ili{MSA} \textit{ṣuḥba} ‘companionship’), and occasional \textit{imāla} in elevated or poetic style (\ili{NewP} \textit{ḥiǧīz} < \ili{MSA} \textit{ḥiǧāz}).

\subsubsection{Arabic lexicon in Persian dialects}

\ili{Arabic} \isi{loanwords} affect Persian dialects in two ways that differ from \ili{MSP}: i) semantic changes, where \ili{Arabic} lexemes assume new meanings unattested in both \ili{MSA} and \ili{MSP}: in Kirmān \textit{ðāt} ‘age’   (\citealt{Ṣarrāfī1996}: 106) < \ili{MSA}/\ili{MSP} ‘self, soul, essence, nature’, \textit{ðātī} ‘old’ < \ili{MSA}/\ili{MSP} ‘own, personal’; ii) lexemes and expressions directly borrowed from \ili{Arabic}, and not attested in \ili{MSP}: in Šūštar, \textit{ḥaya} ‘snake’ < \ili{MSA} \textit{ḥayya}, \ili{MSP} \textit{mār} (\citealt{Fāẓilī2004}: 140), \textit{ṭayyāra} ‘airplane’ < \ili{Arabic} dialects \textit{ṭayyāra}, \ili{MSA} \textit{ṭāʔira}, \ili{MSP} \textit{hawāpaimā} (\citealt{Fāẓilī2004}: 150), \textit{ṣaḥn} ‘bowl, dish’ < \ili{MSA} \textit{ṣaḥn}, \ili{MSP} \textit{bušqāb} (\citealt{Fāẓilī2004}: 150), \textit{ṭabaq} ‘plate, tray’ < \ili{MSA} \textit{ṭabaq}, \ili{MSP} \textit{sīnī} (\citealt{Fāẓilī2004}: 150), in Fīn, \textit{mismāl} ‘nail’ < \ili{MSA} \textit{mismār}, \ili{MSP} \textit{mīḫ} ‘nail’ (\citealt{NaǧībiFīni2002}: 133), in Kirmān, \textit{aḥad} \textit{un-nās} ‘nobody, somebody’ < \ili{MSA} \textit{aḥad} \textit{un-nās}, \ili{MSP} \textit{hīčkas} ‘nobody’, \textit{kasī} ‘somebody’ (\citealt{Ṣarrāfī1996}: 33).

On the Persian \ili{Gulf} coast of Iran, due to linguistic, economic and commercial connections with the Arabian Peninsula, Persian dialects have incorporated from \ili{Gulf} \ili{Arabic} a number of \ili{Arabic} technical terms relating to pearling, fishing and traditional shipbuilding: \textit{muḥār} ‘shellfish, oysters’ (cf. \ili{MSA} \textit{maḥār}), \textit{giyās} ‘measure, gauge’ (< \ili{GA} \textit{giyās}, cf. \ili{MSA} \textit{qiyās}), \textit{mīdāf} ‘helm (boat)’ (< \ili{GA} \textit{mīdāf}, cf. \ili{MSA} \textit{miǧdāf}), \textit{māčila} ‘meal (on a boat)’ (< \ili{GA} \textit{māčila}, cf. \ili{MSA} \textit{maʔkūl}). Two neighborhoods in the town of Bandar Linga (opposite Dubai, 180 km west of Bandar ʕAbbās) are called \textit{Maḥalla-yi} \textit{Baḥrainī} ‘\ili{Bahraini} Quarter’ and \textit{Maḥalla-yi} \textit{Sammāčī} ‘Fishers’ Quarter’ (< \ili{GA} \textit{sammāč}, cf. \ili{MSA} \textit{sammāk}) (\citealt{Baḫtiyārī1990}: 137–138).

\section{Conclusion}

Although \ili{Arabic}–Persian language contact has been a well-known phenomenon for centuries, academic research dedicated to this topic is far from abundant. Throughout the centuries, Persian writers and poets used \ili{Arabic} lexical elements in new meanings or coined non-standard Perso-\ili{Arabic} lexemes based on \ili{Arabic} \isi{derivational} patterns. Idiosyncratic features of individual Persian writers should be examined separately before compiling a comprehensive review of this contact-induced \isi{language change}. Substantial fieldwork needs to be conducted to describe the \isi{bilingualism} of ethnic Arab communities of Iran and ethnic Persians in \ili{Arabic}-speaking countries. Additionally, it is essential for linguists to look into \ili{Arabic} influence on Modern Persian dialects and Iranian languages other than \ili{New Persian}. This will help scholars understand the scale and depth of how \ili{Arabic} has shaped Iranian languages for the past thousand years.

Contact-induced \isi{language change} in New Iranian languages primarily transpired under \isi{RL} agentivity. It should be noted, however, that medieval Persian literati were so well-versed in \ili{Arabic} due to its \isi{prestige} and dominance, that their \isi{bilingualism} may have enabled \isi{convergence} in \ili{Arabic}–Persian language contact.

\section*{Further reading}
\begin{furtherreading}
\item \citet{Asbaghi1987} gathers eight hundred Persian words of {Arabic} origin in twenty-three groups and analyzes the semantic changes they underwent when transferred from {Arabic} to {New Persian}.
\item \citet{Gazsi2011} gives an overview of {Arabic}–Persian language contact from pre-Is\-lam\-ic times up to the modern era, also touching on {Arabic} dialects in Iran. A brief analysis of {Arabic} morphosyntactic features in {New Persian} is also provided.
\item \citet{Ṣādiqī2011} discusses a range of {Arabic} phonological, grammatical and semantic elements in {New Persian}.
\end{furtherreading}

\section*{Acknowledgements}

I would like to express my gratitude to Prof. Éva Jeremiás, Prof. Werner Arnold and Prof. Ali Ashraf Sadeqi for their support while I was working on {Arabic}–Persian language contact. I am thankful to members of the ‘Arabs of the Coast’ in the UAE, especially Sheikh Ibrahim, Sheikh Abdulrahman, Dr. Abdullah, Walid, Ahmed, and many others for providing language data in Gulf Arabic.

\section*{Abbreviations}
\begin{tabularx}{.5\textwidth}[t]{@{}lQ@{}}
BCE & before Common Era\\
CE & Common Era\\
GA & Gulf {Arabic}\\
\textsc{gen}   &  genitive\\
{MSA} &  Modern Standard {Arabic}\\
{MSP}  &  Modern Standard Persian\\
\end{tabularx}%
\begin{tabularx}{.5\textwidth}[t]{@{}lQ@{}}
NewP   &  {New Persian}\\
\textsc{pl}   &  plural\\
\textsc{prs}   &  present\\
\textsc{pst}   &  past\\
{RL} &  {recipient language} \\
\end{tabularx}%


{\sloppy\printbibliography[heading=subbibliography,notkeyword=this]}
\end{document}
