\documentclass[output=paper]{langsci/langscibook} 
\ChapterDOI{10.5281/zenodo.3744529}
\author{Andrei Avram\affiliation{University of Bucharest}}
\title{Arabic pidgins and creoles}
\abstract{The chapter is an overview of eight Arabic-lexifier pidgins and creoles: Turku, Bongor Arabic, Juba Arabic, Kinubi, Pidgin Madame, Jordanian Pidgin Arabic, Romanian Pidgin Arabic, and Gulf Pidgin Arabic. The examples illustrate a number of selected features of these varieties. The focus is on two types of transfer, imposition and borrowing, within the framework outlined by Van Coetsem (\citeyear{VanCoetsem1988,VanCoetsem2000,VanCoetsem2003}) and Winford (\citeyear{Winford2005,Winford2008}).}
\IfFileExists{../localcommands.tex}{
  \usepackage{langsci-optional}
\usepackage{langsci-gb4e}
\usepackage{langsci-lgr}

\usepackage{listings}
\lstset{basicstyle=\ttfamily,tabsize=2,breaklines=true}

%added by author
% \usepackage{tipa}
\usepackage{multirow}
\graphicspath{{figures/}}
\usepackage{langsci-branding}

  
\newcommand{\sent}{\enumsentence}
\newcommand{\sents}{\eenumsentence}
\let\citeasnoun\citet

\renewcommand{\lsCoverTitleFont}[1]{\sffamily\addfontfeatures{Scale=MatchUppercase}\fontsize{44pt}{16mm}\selectfont #1}
   
  %% hyphenation points for line breaks
%% Normally, automatic hyphenation in LaTeX is very good
%% If a word is mis-hyphenated, add it to this file
%%
%% add information to TeX file before \begin{document} with:
%% %% hyphenation points for line breaks
%% Normally, automatic hyphenation in LaTeX is very good
%% If a word is mis-hyphenated, add it to this file
%%
%% add information to TeX file before \begin{document} with:
%% %% hyphenation points for line breaks
%% Normally, automatic hyphenation in LaTeX is very good
%% If a word is mis-hyphenated, add it to this file
%%
%% add information to TeX file before \begin{document} with:
%% \include{localhyphenation}
\hyphenation{
affri-ca-te
affri-ca-tes
an-no-tated
com-ple-ments
com-po-si-tio-na-li-ty
non-com-po-si-tio-na-li-ty
Gon-zá-lez
out-side
Ri-chárd
se-man-tics
STREU-SLE
Tie-de-mann
}
\hyphenation{
affri-ca-te
affri-ca-tes
an-no-tated
com-ple-ments
com-po-si-tio-na-li-ty
non-com-po-si-tio-na-li-ty
Gon-zá-lez
out-side
Ri-chárd
se-man-tics
STREU-SLE
Tie-de-mann
}
\hyphenation{
affri-ca-te
affri-ca-tes
an-no-tated
com-ple-ments
com-po-si-tio-na-li-ty
non-com-po-si-tio-na-li-ty
Gon-zá-lez
out-side
Ri-chárd
se-man-tics
STREU-SLE
Tie-de-mann
} 
  \togglepaper[1]%%chapternumber
}{}

\begin{document}
\maketitle 


\section{Introduction}

This chapter aims to illustrate the emergence of \ili{Arabic}-{lexifier} pidgins and creoles for which the contact situation – i.e. socio-historical context, the agents of change, and the languages involved – is at least relatively well known.

  The varieties considered can be classified into two groups, in geographical, historical and developmental terms: the \ili{Sudanic pidgins} and creoles, and the immigrant pidgins in various Arab countries. Geographically, the {Sudanic}\il{Sudanic pidgins} varieties developed in Africa – in present-day South Sudan, Chad, Uganda, and Kenya. Historically, the\il{Sudanic}\il{Sudanic pidgins} varieties derive from a putative common ancestor, a {pidgin} that emerged in southern Sudan, in the first half of the nineteenth century. Various {Turkish}--Egyptian military expeditions between 1820 and 1840 opened southern Sudan for the slave trade. Permanent camps were set up soon after by slave traders in the White Nile Basin, Bahr el-Ghazal and Equatoria Province, inhabited by an \ili{Arabic}-speaking minority and a huge majority of slaves from various ethnic and linguistic backgrounds. After 1850, the slave traders’ settlements were turned into military camps in which a military {pidgin} emerged, which is traditionally referred to as “Common {Sudanic} Pidgin Creole \ili{Arabic}” (\citealt{ToscoManfredi2013}: 253).\il{Sudanic pidgins} Two subgroups of {Sudanic} varieties are recognized: the western branch, consisting of \ili{Turku} and \ili{Bongor} \ili{Arabic} (in Chad), and the eastern one, made up of \ili{Juba} \ili{Arabic} (in Sudan) and \ili{Kinubi} (spoken in Uganda and Kenya).\il{Sudanic pidgins} 

  Immigrant pidgins emerged in the eastern part of the Arab World, in Lebanon, Jordan, Iraq and the countries of the Arab Gulf. Historically, these do not go back more than 50 years. All these varieties are incipient pidgins.

  The contact situations illustrated presuppose: (i) a {source language} ({SL}) and a {recipient language} ({RL}); (ii) agents of contact-induced change, who may be either {SL} or {RL} speakers; (iii) a psycholinguistically dominant language, which is not necessarily a socially dominant language (\citealt{VanCoetsem1988,VanCoetsem1995,VanCoetsem2000,VanCoetsem2003,Winford2005,Winford2008}). A distinction is made between two types of {transfer}: {imposition} and borrowing (\citealt{VanCoetsem1988,VanCoetsem2000,VanCoetsem2003}). Imposition involves SL-dominant speakers as agents ({SL} agentivity), is typical of second-language (L2) acquisition, and induces changes mostly in phonology and syntax, although it may also include {transfer} of lexical items from the dominant {SL} into the non-dominant {RL} (\citealt[18]{VanCoetsem1995}; \citealt[376]{Winford2005}). Borrowing normally involves RL-dominant speakers as agents ({RL} agentivity), typically targets lexical items, but may also include {transfer} of morphological material from a non-dominant {SL} into the dominant {RL}.

  In light of their sociolinguistic history, the varieties considered all emerged under conditions of untutored, short-term L2 acquisition by adults dominant in their socially subordinate SLs. L2 acquisition, \textit{a} \textit{fortiori} with adults, triggers processes such as {imposition} via {SL} agentivity (i.e. {substrate} influence), {simplification} (\citealt{Trudgill2011}: 40, 101) – also known as restructuring \citep[529]{Lucas2015} – as well as language-internal (i.e. non-contact-induced) developments such as grammatical reanalysis \citep[415]{Winford2005}.

  As in \citet{Manfredi2018}, the focus of this chapter is on {imposition} and borrowing. It does not illustrate restructuring which does not involve any kind of {transfer}, but often involves a reduction in complexity \citep[529]{Lucas2015}. In the case of \ili{Arabic} pidgins and creoles, restructuring is manifest in the domain of morphology, in, for example, the loss of the \ili{Arabic} verbal affixes and of the nominal and verbal {derivation} strategies \citep{Miller1993}.

  The examples are illustrative only of selected contact-induced features of \ili{Arabic} pidgins and creoles and their number has been kept to a reasonable minimum. The examples from \ili{Arabic} and the pidgins and creoles considered appear in a uniform system of transliteration.

  The chapter is organized as follows. §\ref{sec:tur} and §\ref{sec:jub} are concerned with {Sudanic} pidgins and creoles. §\ref{sec:pid}, on the other hand, deals with \ili{Arabic} immigrant varieties. §\ref{sec:conc} summarizes the findings and introduces issues for further research.


 \section{Turku and Bongor Arabic}\label{sec:tur}


 \subsection{Current state and historical development}


\ili{Turku} is an extinct {pidgin}, formerly spoken in the Chari--\ili{Bagirmi} region in western Chad \citep{Muraz1926}. After the abolition of slavery by the {Turkish}--Egyptian government in 1879, the Nile \ili{Nubian} trader \iai{Rabeh} withdrew with his slave soldiers into Chad. From a sociolinguistic point of view, \ili{Turku} was initially a military {pidgin}. However, it later became one of three trade languages in what was then French Equatorial Africa, along with \ili{Sango} and Bangala (\citealt{ToscoOwens1993}: 183). \ili{Turku} was a stable {pidgin} which does not appear to have creolized (\citealt{ToscoOwens1993}).

\ili{Bongor} \ili{Arabic} is spoken in southwestern Chad, in and around the town of Bongor, the capital of the Mayo--Kebbi Est region, at the border with Cameroon \citep{Luffin2013}. Given the many structural features it shares with \ili{Turku}, it is plausible to assume that \ili{Bongor} \ili{Arabic} developed from the former. Sociolinguistically, \ili{Bongor} \ili{Arabic} is a trade {pidgin}, used by the local \ili{Masa} and \ili{Tupuri} populations with \ili{Arabic}-speaking traders. It is currently a stable {pidgin}, but it exhibits features indicative of depidginization under the influence of \ili{Chadian} \ili{Arabic} (ChA). No information about the number of speakers is available.


 
 \subsection{Contact languages}


The {lexifier} language of \ili{Turku} and \ili{Bongor} \ili{Arabic} is \ili{Western Sudanic} \ili{Arabic}. The substratal input was provided by languages of various genetic affiliations: \ili{Nilo-Saharan} – e.g. \ili{Bagirmi}, \ili{Mbay}, \ili{Ngambay}, \ili{Sar}, \ili{Sara} (Central \ili{Sudanic}), \ili{Kanuri} (Western Saharan); \ili{Afro-Asiatic} – \ili{Hausa} (West \ili{Chadic}); \ili{Niger-Congo} – \ili{Fulfulde}. In the case of \ili{Turku}, an additional contributor was the {creole} language \ili{Sango}. Both in \ili{Turku} and in \ili{Bongor} \ili{Arabic} there is also adstratal input from \ili{French}. The {adstrate} of \ili{Bongor} \ili{Arabic} additionally includes two languages: \ili{Masa} (\ili{Nilo-Saharan}, Western \ili{Chadic}) and \ili{Tupuri} (\ili{Niger-Congo}).


 
 \subsection{Contact-induced changes}
 \subsubsection{Phonology}

The {substrate} languages do not have /ḫ/, which is generally replaced by [k]: \ili{Turku} \textbf{\textit{k}}\textit{amsa} ‘five’ < \ili{ChA} \textbf{\textit{ḫ}}\textit{amsa}; \ili{Bongor} \ili{Arabic} \textbf{\textit{k}}\textit{ídma} ‘work’ < \ili{ChA} \textbf{\textit{ḫ}}\textit{idma}. Many of the {substrate} languages do not have /f/, which is substituted with [p] or perhaps [ɸ],\footnote{Transcribed as 〈pf〉 by Muraz (\citeyear[168]{Muraz1926}).} e.g. \ili{Turku} \textit{\textbf{pf}il} ‘elephant’ < \ili{ChA} \textit{\textbf{f}īl}. In \ili{French} {loanwords}, the reflexes of /v/ are either [b] or [w]: \ili{Bongor} \ili{Arabic} \textit{\textbf{b}oté} ‘to vote’ < \ili{French} \textit{\textbf{v}oter}, \textit{\textbf{w}otír} ‘car’ < \ili{French} \textit{\textbf{v}oiture}.  

  The consonants [ɲ] and [ŋ] occur only in {loanwords}: \ili{Bongor} \ili{Arabic} \textit{\textbf{ng}ambáy} ‘\ili{Ngambay}’ < \ili{Ngambay} \textit{\textbf{ng}àmbáy}; \ili{Turku} \textit{\textbf{ng}ari} ‘manioc’ < \ili{Mbay} \textit{\textbf{ng}àrì}, \textit{konpa}\textit{\textbf{n\kern -1pty}e} ‘company’ < \ili{French} \textit{compa}\textit{\textbf{gn}ie}; [v] and [ʒ] occur only in phonologically non-integrated words of \ili{French} origin: \ili{Turku} \textit{si}\textit{\textbf{v}il} ‘civilian’ < \ili{French} \textit{ci}\textit{\textbf{v}il}; \ili{Bongor} \ili{Arabic} \textit{\textbf{ž}urnalíst} `journalist' < \ili{French} \textit{\textbf{j}ournaliste}. 

   Variation affects several consonants. For instance, [f] occurs in variation with [b] or [p]: \ili{Turku} \textit{\textbf{f}išan} {\textasciitilde} \textit{\textbf{b}išan} ‘because’; \ili{Bongor} \ili{Arabic} \textit{má}\textit{\textbf{f}i} {\textasciitilde} \textit{má}\textit{\textbf{p}i} ‘\textsc{neg}’ < \ili{ChA} \textit{mā} \textit{\textbf{f}ī}, \textit{so}\textit{\textbf{f}ér} {\textasciitilde} \textit{so}\textit{\textbf{p}ér} ‘driver’ < \ili{French} \textit{chau}\textit{\textbf{ff}eur}. Most of the {substrate} languages do not have /š/, which accounts for [ʃ] {\textasciitilde} [s] variation, in words with either etymological /s/ or /ʃ/: \ili{Turku} \textit{ga\textbf{s}i} {\textasciitilde} \textit{ga\textbf{š}i} ‘expensive’ < \ili{ChA} \textit{gā\textbf{s}ī}, \textit{biri\textbf{š}} {\textasciitilde} \textit{biri\textbf{s}} ‘mat’ < \ili{ChA} \textit{birī\textbf{š}}; \ili{Bongor} \ili{Arabic} \textit{má\textbf{š}i} {\textasciitilde} \textit{má\textbf{s}i} ‘go’. The usual reflexes of \ili{French} /ʒ/, absent from the phonological inventories of the {substrate} languages, are [z], [ʤ] and [s] respectively: \ili{Turku} \textit{\textbf{ǧ}inenal} ‘general’ < \ili{French} \textit{\textbf{g}énéral}, \textit{\textbf{s}uska} ‘until < \ili{French} \textit{\textbf{j}usqu’à}; \ili{Bongor} \ili{Arabic} \textit{\textbf{z}úska} `when, during' < \ili{French} \textit{\textbf{j}usqu’à} `until'.

Finally, {vowel length} is not distinctive: \ili{Turku}, \ili{Bongor} \ili{Arabic} \textit{kal\textbf{a}m} ‘speech; speak’ < \ili{ChA} \textit{kal\textbf{ā}m} ‘speech’.


 \subsubsection{Morphology}

On current evidence (\citealt{Luffin2013}: 180–181), \ili{Bongor} \ili{Arabic} exhibits signs of depidginization under the influence of \ili{Chadian} \ili{Arabic}. The most striking instance of this is the use of pronominal suffixes, unique among \ili{Arabic}-{lexifier} pidgins and creoles:

\ea
{\ili{Bongor} \ili{Arabic} \citep[180]{Luffin2013}}\\
\gll    índi gáy árifu úsum-\textbf{i} \\
        2\textsc{sg} \textsc{impf} know name-\textsc{poss}.1\textsc{sg}\\
\glt    `You know my name.'
\z
 
\noindent Also, verbal affixes are sporadically used:


\ea
{\ili{Bongor} \ili{Arabic} \citep[181]{Luffin2013}}\\
\ea
\gll ána ma \textbf{n}-árfa \\
         1\textsc{sg} \textsc{neg} 1\textsc{sg}-know\\
\glt    `I don’t know.'
\ex
\gll  anína rikíb-\textbf{na} wotír da sáwa \\
         1\textsc{pl} ride.\textsc{prf}-1\textsc{pl} car \textsc{prox} together\\
\glt    `We took the car together.’
\z
\z
 
{These cases might be analyzed as borrowing under \textit{sui} \textit{generis} {RL} agentivity, whereby morphological material from a non-dominant {SL} is imported into a non-dominant (second) {RL}.}


 \subsubsection{Lexicon}

A part of the non-\ili{Arabic} vocabulary of \ili{Turku} can be traced back to its {substrate} languages (\citealt{Avram2019}). Most of the {loanwords} are from \ili{Sara}-\ili{Bagirmi} languages: \textit{adinbang} ‘eunuch’ < \ili{Bagirmi} \textit{ádim} \textit{mbàŋ} ‘servant of the sultan’; \textit{gao} ‘hunter’ < \ili{Sar} \textit{gáw}; \textit{ngari} ‘manioc’ < \ili{Mbay} \textit{ngàrì}. The second most significant important contributor is \ili{Sango}: \textit{kay} ‘paddle’ < \ili{Sango} \textit{kâî}, \textit{tipoy} ‘carrying hammock’ < \textit{típóí}. A few words can be traced to \ili{Fulfulde} and \ili{Kanuri}: \textit{kelkelbanǧi} ‘golden beads’ < \ili{Fulfulde} \textit{kelkel-banja}; \textit{wélik} ‘lightning’ < \ili{Kanuri} \textit{wulak} ‘flash of lightning’. In a number of cases, the exact {SL} cannot be established: \textit{koporo} ‘0.10 Francs’ < \ili{Fulfulde}, \ili{Sango}, \ili{Sara} \textit{koporo} ‘coin’; \textit{gurumba} ‘hat’ < \ili{Hausa} \textit{gurúmba}, \ili{Kanuri} \textit{gurumbá}. As for \ili{Bongor} \ili{Arabic}, its African {adstrate} languages have contributed only a few {loanwords}, such as \textit{bursdíya} ‘Monday’. There are also {loanwords} from \ili{French}. In \ili{Turku} most of these relate to the military (\citealt{ToscoOwens1993}: 262–263), e.g. \ili{Turku} \textit{itenan} ‘lieutenant’ < \ili{French} \textit{lieutenant}, \textit{permišon} ‘permission’ < \ili{French} \textit{permission}. In addition to nouns, \ili{French} {loanwords} include some verbs, such as \ili{Bongor} \ili{Arabic} \textit{komandé} ‘order’ < \ili{French} \textit{commander}, and at least one function word, \ili{Turku} \textit{suska}, \ili{Bongor} \ili{Arabic} \textit{zúska} ‘when, during’ < \ili{French} \textit{jusqu’à} ‘until’.

  The substratal influence on \ili{Turku} can also be seen in a number of compound calques (Avram \citeyear{Avram2019}; Manfredi, this volume).\ia{Manfredi, Stefano@Manfredi, Stefano} Some of these are modelled on \ili{Sara}-\ili{Bagirmi} languages: \textit{bahr} \textit{gum} ‘rising water’, cf. \ili{Ngambay} \textit{màn-kà}\textit{w}, lit. ‘river goes’; \textit{nugra} \textit{ana} \textit{asal} ‘beehive’, cf. \ili{Ngambay} \textit{bòlè-tǝnji}, lit. ‘hole (of) honey’. Other calques have equivalents in several SLs, such as \textit{nugra} \textit{haǧer} ‘cave’, lit. ‘hole mountain/stone’, cf. \ili{Kanuri} \textit{kûl} \textit{kau-be} lit. ‘cavity mountain-of’, \ili{Ngambay} \textit{bòlò-mbàl} lit. ‘hole mountain’, \ili{Sango} \textit{dûtênë} lit. ‘hole stone’.


 \section{Juba Arabic and Kinubi}\label{sec:jub}


 \subsection{Current state and historical development}


\ili{Juba} \ili{Arabic} is mainly spoken in South Sudan; there are also {diaspora} communities, mostly in Sudan and Egypt. Two main reasons make it difficult to estimate its number of speakers. Firstly, while \ili{Juba} \ili{Arabic} is spoken as a primary language by 47\% of the population of \ili{Juba}, the capital city of South Sudan, it is also used as a second or third language by the majority of the population of the country \citep[7]{Manfredi2017}. Secondly, the long coexistence of \ili{Juba} \ili{Arabic} with \ili{Sudanese} \ili{Arabic}, its main {lexifier} language, has led to the emergence of a continuum ranging from basilectal, through mesolectal, to acrolectal varieties; delimiting acrolectal \ili{Juba} \ili{Arabic} from \ili{Arabic} is no easy task, particularly in the case of the large {diaspora} communities in Khartoum and Cairo.

  \ili{Juba} \ili{Arabic} emerged as a military {pidgin}. Sociolinguistically, it is today an inclusive {identity} marker for the ethnically and linguistically diverse population of South Sudan (\citealt{ToscoManfredi2013}: 507). Developmentally, \ili{Juba} \ili{Arabic} is a pidgincreole.\footnote{A pidgincreole is “a former {pidgin} that has become the main language of a {speech community} and/or a mother tongue for some of its speakers” \citep[131]{Bakker2008}.}

The Mahdist revolt, which started in 1881, eventually brought about the end of {Turkish}--Egyptian control over Equatoria, in southern Sudan. Following an invasion by Mahdist rebels, the governor fled to Uganda, accompanied by slave soldiers loyal to the central government. These soldiers subsequently became the backbone of the British King’s African Rifles. While some of the troops remained in Uganda, others were moved to Kenya and Tanzania. This led to the dialectal division between Ugandan and \ili{Kenyan Kinubi}. Like \ili{Juba} \ili{Arabic}, therefore, \ili{Kinubi} started out as a military {pidgin}, then underwent stabilization and expansion. Today, however, \ili{Kinubi} is the only \ili{Arabic}-{lexifier} fully creolized variety, that is, a native language for its entire {speech community}.

\ili{Kinubi} is spoken in Uganda and in Kenya. The number of speakers of \ili{Kinubi} is a matter of debate. Ugandan \ili{Kinubi} was spoken by some 15,000 people, according to the 1991 census, and \ili{Kenyan Kinubi} by an estimated 10,000 in 2005. However, other estimates put the combined number of speakers at about 50,000. The largest communities of \ili{Kinubi} speakers are in Bombo (Uganda), Nairobi (the Kibera neighbourhood) and Mombasa (Kenya).


 
 \subsection{Contact languages}


The main {lexifier} language of \ili{Juba} \ili{Arabic} is \ili{Sudanese} \ili{Arabic} (SA), with some input from \ili{Egyptian Arabic} (EA) and \ili{Western Sudanic} dialects as well. The {substrate} is represented by a relatively large number of languages, belonging to super-phylums, \ili{Nilo-Saharan} and \ili{Niger-Congo}. The former includes {Eastern} \ili{Sudanic} languages, such as \ili{Bari}, \ili{Lotuho} ({Eastern} Nilotic), \ili{Acholi}, \ili{Belanda Bor}, \ili{Dinka}, \ili{Jur}, \ili{Nuer}, \ili{Päri}, \ili{Shilluk} (Western Nilotic), \ili{Didinga} (Surmic), and Central \ili{Sudanic} languages, such as \ili{Avokaya}, \ili{Baka}, \ili{Bongo}, \ili{Ma’di}, \ili{Moru}; the \ili{Niger-Congo} super-phylum is represented by, for example, \ili{Zande} and \ili{Mundu}. The main {substrate} language is considered to be \ili{Bari}, including its dialects \ili{Kakwa}, \ili{Kuku}, \ili{Pojulu}, and \ili{Mundari}.\footnote{Sometimes considered to be separate languages \citep[207]{Wellens2003}.}

Given its sociolinguistic history, \ili{Kinubi} shares much of its {substrate} with \ili{Juba} \ili{Arabic}. However, the {substrate} of Ugandan \ili{Kinubi} additionally includes {Eastern} \ili{Sudanic} languages, such as \ili{Alur}, \ili{Luo} (Western Nilotic), and Central \ili{Sudanic} languages such as \ili{Mamvu}, \ili{Lendu} and \ili{Lugbara} (\citealt{Owens1997}: 161; \citealt{Wellens2003}: 207), spoken in Uganda. Unlike \ili{Juba} \ili{Arabic}, \ili{Kinubi} also exhibits the effects of the adstratal influence exerted by two Bantu languages, \ili{Luganda} – particularly in Ugandan \ili{Kinubi} – and \ili{Swahili} – particularly in \ili{Kenyan Kinubi}. One other language that should be mentioned is \ili{English}, official both in Uganda and in Kenya.


 
 \subsection{Contact-induced changes}
 \subsubsection{Phonology}

A number of consonants found in \ili{Arabic}, but absent from the phonological inventories of the {substrate} languages, are either deleted or substituted. Consider the reflexes of {pharyngeals}: \textit{\textbf{h}áfla} ‘feast’ < \ili{SA} \textit{\textbf{ḥ}afla}; \textit{\textbf{á}rabi} ‘\ili{Arabic}’ < \ili{SA} \textit{\textbf{ʕ}arabī}. The {pharyngealized} consonants are replaced by their plain counterparts: \textit{\textbf{t}owíl} ‘long’ < \ili{SA} \textit{\textbf{ṭ}awīl}; \textit{\textbf{d}ul} ‘shadow’ < \ili{SA} \textit{\textbf{ḍ}ull}; \textit{\textbf{s}úlba} ‘hip’ < \ili{SA} \textit{\textbf{ṣ}ulba}; \textit{\textbf{z}úlum} ‘to anger’ < \ili{SA} \textit{\textbf{ẓ}\kern -1ptulum}. The velar fricatives of \ili{Arabic} are always replaced by velar stops: \textit{\textbf{k}ábara} ‘piece of news’ < \ili{SA} \textit{\textbf{ḫ}abar}; \textit{šó\textbf{k}ol} ‘work’ < \ili{SA} \textit{šo\textbf{ɣ}ol}, \textit{\textbf{g}árib} ‘west’ < \ili{SA} \textit{\textbf{ɣ}ar(i)b}. 

As in \ili{Juba} \ili{Arabic}, the {pharyngeals} of \ili{Arabic} are either replaced or lost in \ili{Kinubi} (\citealt{Owens1985}: 10; \citealt{Wellens2003}: 209–212). The earliest records of Ugandan \ili{Kinubi}\footnote{The main ones are: \citet{Cook1905}, \citet{Jenkins1909}, \citet{Meldon1913}, and \citet{OwenKeane1915}.} are replete with illustrative examples \citep{Avram2017talk}: \textit{\textbf{h}aǧa} ‘thing’ < \ili{SA} \textit{\textbf{ḥ}āǧa}, \textit{aram} ‘thief’ < \ili{SA} \textit{\textbf{ḥ}arāmi}, \textit{līb} < ‘to play’ < \ili{SA} \textit{li\textbf{ʕ}ib}. The {pharyngealized} consonants are replaced by their plain counterparts, as in these examples from early Ugandan \ili{Kinubi}: \textit{\textbf{t}owil} ‘long’ < \ili{SA} \textit{\textbf{ṭ}awīl}; \textit{\textbf{d}ulu} ‘shadow’ < \ili{SA} \textit{\textbf{ḍ}ull}, \textit{hi\textbf{s}iba} ‘measles’ < \ili{SA} \textit{ḥi\textbf{ṣ}ba}; \textit{\textbf{z}ulm} ‘to anger’ < \ili{SA} \textit{\textbf{ẓ}\kern -1ptulum}. Like \ili{Juba} \ili{Arabic}, \ili{Kinubi} substitutes velar stops for the \ili{Arabic} velar fricatives. Consider the following early Ugandan \ili{Kinubi} forms: \textit{\textbf{k}idima} ‘work’ < \ili{SA} \textit{\textbf{ḫ}idma}; \textit{šo\textbf{k}olo} ‘work’ < \ili{SA} \textit{šo\textbf{ɣ}ol}, \textit{bala\textbf{g}o} ‘commandment’ < \ili{SA} \textit{balā\textbf{ɣ}} ‘message’. Substratal influence also accounts for consonant {degemination}, given that the {substrate} languages “lack these in all but a few morphonologically determined contexts” \citep[162]{Owens1997}. 

Substratal influence can also be seen in the occurrence of certain consonants. Consider first /ɓ/ and /ɗ/: \ili{Juba} \ili{Arabic} \textit{\textbf{d'}éngele} ‘liver’ < \ili{Bari} \textit{denggele}; \ili{Juba} \ili{Arabic} \textit{\textbf{b'}ónǧo} ‘pumpkin’ < \ili{Bongo} \textit{\textbf{b'}onǧo}. The other consonants which occur only in {loanwords} from the {substrate} and/or {adstrate} languages are [p] [v], [ʧ], [ɲ], and [ŋ]: \ili{Kinubi} \textit{lí\textbf{p}a} ‘to pay’ < \ili{Swahili} \textit{-li\textbf{p}a}; \ili{Kinubi} \textit{cam\textbf{p}} ‘camp’ < \ili{English} \textit{cam\textbf{p}}; \ili{Kinubi} \textit{\textbf{v}íta} ‘war’ < \ili{Swahili} \textit{\textbf{v}ita}; \ili{Juba} \ili{Arabic} \textit{\textbf{č}am} ‘food’ < \ili{Acholi}, \ili{Belanda Bor}, \ili{Jur} \textit{\textbf{č}ama}, \ili{Juba} \ili{Arabic} \textit{\textbf{č}ayniz} < \ili{English} \textit{\textbf{Ch}inese}, \ili{Kinubi} \textit{\textbf{č}ay} ‘tea’ < \ili{Swahili} \textit{\textbf{ch}ai}; \ili{Juba} \ili{Arabic} \textit{\textbf{n\kern -1pty}ékem}, \ili{Kinubi} \textit{\textbf{n\kern -1pty}ékem} ‘chin’ < \ili{Bari} \textit{\textbf{n\kern -1pty}ékem}, \ili{Kinubi} \textit{\textbf{n\kern -1pty}á\textbf{n\kern -1pty}a} ‘tomato’ < \ili{Swahili} \textit{\textbf{n\kern -1pty}a\textbf{n\kern -1pty}a}; \ili{Juba} \ili{Arabic} \textit{\textbf{ŋ}un} ‘divinity’ < \ili{Bari} \textit{\textbf{ng}un}. The integration of these phonemes is thus a result of borrowing (under {RL} agentivity) rather than of {imposition}.

  The following instances of consonant variation are more common in \ili{Juba} \ili{Arabic} (\citealt{Manfredi2017}: 25–27). The most frequent is [ʃ] {\textasciitilde} [s]: \textit{geš} {\textasciitilde} \textit{ge\textbf{s}} ‘grass’. Further, [z] is in variation with [ʤ] before /o/ and /a/: \textit{\textbf{z}ówǧu} {\textasciitilde} \textit{\textbf{ǧ}ówǧu} ‘to marry’, \textit{\textbf{z}áman} {\textasciitilde} \textit{\textbf{ǧ}áman} ‘time; when’. There is also [p] {\textasciitilde} [f] variation in word-initial position, including in {loanwords}: \textit{\textbf{p}oǧúlu} {\textasciitilde} \textit{\textbf{f}oǧúlu} ‘\ili{Pojulu}’, \textit{\textbf{p}rótestan} {\textasciitilde} \textit{\textbf{f}rótestan} ‘protestant’. Finally, the {phoneme} /f/ may also be phonetically realized as [p]: \textit{nédi\textbf{f}u} {\textasciitilde} \textit{nédi\textbf{p}u} ‘to clean’. Of these cases of variation, the latter has been specifically attributed to substratal influence from \ili{Bari} (\citealt{Miller1989}; \citealt{Manfredi2017}). It might be argued, however, that all these instances of consonant variation reflect the influence of the {substrate} languages, regardless of their genetic affiliations. The following do not have /ʃ/: \ili{Acholi}, \ili{Avokaya}, \ili{Baka}, \ili{Bari}, \ili{Belanda Bor}, \ili{Bongo}, \ili{Dinka}, \ili{Jur}, \ili{Lotuho}, \ili{Ma’di}, \ili{Moru}, \ili{Mundu}, \ili{Nuer}, \ili{Päri}, \ili{Shilluk}, \ili{Zande}. Of these, \ili{Acholi}, \ili{Belanda Bor}, \ili{Bongo}, \ili{Dinka}, \ili{Jur}, \ili{Nuer}, \ili{Päri} and \ili{Shilluk} do not have /s/ either. A number of {substrate} languages do not have /z/: \ili{Acholi}, \ili{Bongo}, \ili{Belanda Bor}, \ili{Dinka}, \ili{Jur}, \ili{Lotuho}, \ili{Nuer}, \ili{Päri}, and \ili{Shilluk}. All of these, however, have /ʤ/. Finally, /f/ is not part of the phonological inventory of \ili{Acholi}, \ili{Bongo}, \ili{Dinka}, \ili{Jur}, \ili{Nuer}, \ili{Päri}, and \ili{Shilluk}, which do, however, have /p/. Given the intricacies of the distribution of /ʃ/, /s/, /z/, /ʤ/, /f/, and /p/ across the {substrate} languages, the types of variation illustrated are not surprising. 

  As in \ili{Juba} \ili{Arabic}, [ʃ] is in variation with [s] in \ili{Kinubi} (\citealt{Owens1985}: 237; \citealt{Owens1997}: 161; \citealt{Wellens2003}: 38; \citealt{Luffin2005}: 62; \citealt{Avram2017talk}): early Ugandan \ili{Kinubi} \textit{\textbf{š}abaka} {\textasciitilde} \textit{\textbf{s}abaka} ‘net’). Although it is etymological /š/ which is typically subject to variation, occasionally this also applies to etymological /s/: early Ugandan \ili{Kinubi} \textit{\textbf{s}ikin} {\textasciitilde} \textit{\textbf{š}ekin} ‘knife’ < \ili{SA} \textit{sikkīn} \citep{Avram2017talk} and modern \ili{Kenyan Kinubi} \textit{flu\textbf{š}} {\textasciitilde} \textit{flu\textbf{s}} ‘money’ < \ili{SA} \textit{fulūs} \citep[63]{Luffin2005}. Note that [š] {\textasciitilde} [s] variation also extends to {loanwords} from \ili{Swahili} (\citealt{Wellens2003}: 80; \citealt{Luffin2005}: 63; \citealt{Avram2017talk}): early Ugandan \ili{Kinubi} \textit{\textbf{š}amba} {\textasciitilde} \textit{\textbf{s}amba} ‘field’ < \ili{Swahili} \textit{\textbf{sh}amba}. Like \ili{Juba} \ili{Arabic}, \ili{Kinubi} exhibits [z] {\textasciitilde} [ʤ] variation (\citealt{Owens1985}: 235; \citealt{Owens1997}: 161; \citealt{Wellens2003}: 215; \citealt{Luffin2005}: 63; \citealt{Avram2017talk}): early Ugandan \ili{Kinubi} \textit{\textbf{ǧ}alan} {\textasciitilde} \textit{\textbf{z}alan} ‘angry’ < \ili{SA} \textit{\textbf{z}aʕlān}. However, unlike \ili{Juba} \ili{Arabic}, in \ili{Kinubi} the [z] {\textasciitilde} [ʤ] variation also occurs before the two front vowels /i/ and /e/: \textit{\textbf{z}e} {\textasciitilde} \textit{\textbf{ǧ}e} ‘as’, early Ugandan \ili{Kinubi} \textit{an\textbf{ǧ}il} {\textasciitilde} \textit{en\textbf{z}il} ‘descend’. According to \citet[161]{Owens1997}, this “is due perhaps to \ili{Bari} substratal influence, since \ili{Bari} has only \textit{j}, not \textit{z}.” In fact, as in the case of \ili{Juba} \ili{Arabic}, the same is true of several other {substrate} languages. Lastly, there are instances of [l] {\textasciitilde} [r] variation (\citealt{Wellens2003}: 214; \citealt{Luffin2005}: 65), affecting both etymological /l/ and etymological /r/ in \ili{Arabic}-derived words, e.g. \textit{tá\textbf{l}e} {\textasciitilde} \textit{tá\textbf{r}e} ‘go out’, \textit{ge\textbf{r}í} {\textasciitilde} \textit{ge\textbf{l}í} ‘near’, and in borrowings, e.g. Ugandan \ili{Kinubi} \textit{čá\textbf{l}o} {\textasciitilde} \textit{čá\textbf{r}o} ‘village’ < \ili{Luganda} \textit{e-kya\textbf{l}o}; \ili{Kenyan Kinubi} \textit{tumbí\textbf{l}i} {\textasciitilde} \textit{tumbí\textbf{r}i} ‘monkey’ < \ili{Swahili} \textit{tumbi\textbf{l}i}. This variation seems to reflect the influence of \mbox{Luganda} and \ili{Swahili}. In the former, [l] and [r] are in complementary distribution, with [r] occurring after the front vowels /i/ and /e/, and [l] elsewhere \citep[214]{Wellens2003}, while in the latter [l] and [r] are in free variation \citep[79]{Luffin2014}. 

As in the {substrate} languages, there is no distinction between short and long vowels: \ili{Juba} \ili{Arabic} \textit{sud\textbf{á}ni} ‘{Sudanese}’ < \ili{SA} \textit{sud\textbf{ā}nī}, \ili{Kinubi} \textit{kab\textbf{í}r} ‘big’ < \ili{SA} \textit{kab\textbf{ī}r}.


 \subsubsection{Morphology}

Apart from the \ili{Arabic}-derived plural suffixes -\textit{at} and -\textit{in}, \ili{Juba} \ili{Arabic} uses the plural marker of \ili{Bari} origin -\textit{ǧín} (\citealt{Nakao2012}: 131; \citealt{Manfredi2014plural}: 58), which is attached only to {loanwords} from local languages:


\ea
{\ili{Juba} \ili{Arabic} \citep[58]{Manfredi2014plural}}\\
\ea
\gll kɔrɔpɔ-ǧín (\textup{< Bari} \textit{kɔrɔpɔ})\\
    leaf-\textsc{pl}\\
\glt       `leaves' 

\ex
\gll beng-ǧín (< \ili{Dinka} \textit{beng}) \\
         chief-\textsc{pl}\\
\glt       `chiefs'


\ex
\gll  b'angiri-ǧín (< \textup{Zande} \textit{b'angiri}) \\
         cheek\textsc{-pl}\\
\glt       `cheeks' 
\z
\z

Another phenomenon worth mentioning is the occurrence in the speech of young urban speakers of hybrid forms, which consist of the \ili{Bari} {relativizer} \textit{lo-} and a noun either from \ili{Arabic} or from one of the {substrate}/{adstrate} languages \citep[131]{Nakao2012}. Note, however, that there is a functional overlap between \ili{Bari} \textit{lo}- and \ili{Sudanese} \ili{Arabic} \textit{abu}.

\ea\label{lo}
{\ili{Juba} \ili{Arabic} \citep[46]{Manfredi2017}}\\
\ea\gll lo-beléde (\textup{< Bari} \textit{lo-} \textup{+ \ili{SA}} \textit{beled})\\
\textsc{rel}-country\\
\glt       `peasant'

\ex
\gll lo-pómbe (< \ili{Bari} \textit{lo-} + \textup{Swahili} \textit{pombe})\\
 \textsc{rel}-alcohol\\
\glt    `drunkard'
\z
\z

Given that a relatively large number of \ili{Bari}-derived words contain \textit{lo-} (\citealt{Miller1989}; \citealt{Manfredi2017}: 46), the examples in \REF{lo} confirm the fact that morphological innovations are typically introduced through lexical borrowings via {RL} agentivity, and subsequently become productive in the {RL}. 

Note, finally, that most of the speakers who use the plural marker -\textit{ǧín} and the {relativizer} \textit{lo}{}- are dominant in \ili{Juba} \ili{Arabic}. These cases therefore confirm the fact that {RL} monolinguals can be agents of borrowing (\citealt{VanCoetsem1988}: 10).

A small number of \ili{Kinubi} nouns borrowed from \ili{Swahili} exhibit the Bantu nominal classifiers:

\ea

{\ili{Kinubi} \citep[57]{Wellens2003}}\\
 
\ea\textbf{mu}-zé               \textbf{wa}-zé\\
 
{\textsc{nc}1-old.man   \textsc{nc2}-old.man}\\
 {`old man, old men'}\\
 
\ex
\gll  \textbf{mu}-zukú \textbf{wa}-zukú\\
          \textsc{nc}1-grandchild   \textsc{nc}2-grandchild\\
\glt       `grandchild, grandchildren'

\ex
\gll \textbf{m}-zúngu \textbf{wa}-zúngu\\
         \textsc{nc}1-European   \textsc{nc}2-European\\
\glt       `European, Europeans'
\z
\z

 \subsubsection{Syntax}

Bureng Vincent (\citeyear[77]{BurengVincent1986}) first noted the similarity between the prototypical {passive} construction in \ili{Juba} \ili{Arabic} and its \ili{Bari} counterpart:


\ea
{\ili{Juba} \ili{Arabic} (\citealt{BurengVincent1986}: 77)}\\
\ea\gll  bágara áyinu \textbf{ma} Wáni\\
        cow see.\textsc{pst} with Wani\\
\glt      `The cow has been seen by Wani.'

\ex
{\ili{Bari} (\citealt{BurengVincent1986}: 77)}\\
\gll             kítɜŋ a mɛtà kɔ̀ Wànì\\
                 cow \textsc{pst} see with Wani\\
\glt     `The cow has been seen by Wani.'
\z
\z

As can be seen, in both \ili{Juba} \ili{Arabic} and \ili{Bari} the agent is introduced by the comitative {preposition} ‘with’. This is a case of lexico-syntactic {imposition} via identification of {SL} and {RL} lexemes \citep[415]{Manfredi2018}: the \ili{Juba} \ili{Arabic} lexical entry \textit{ma} is derived from \ili{Sudanese} \ili{Arabic} \textit{maʕ}, but its semantics reflects the influence of \ili{Bari} \textit{kɔ̀}. The same is true of \ili{Kenyan Kinubi}:

\ea
{       \ili{Kinubi} \citep[230]{Luffin2005}}\\
\gll yal-á al akulú \textbf{ma} nas tomsá\\
     child-\textsc{pl} \textsc{rel} eat.\textsc{pst.pass} with \textsc{pl} crocodile\\
\glt     `the children who were eaten by a crocodile'
\z

Consider next the syntax of {numerals} in \ili{Kinubi} (\citealt{Wellens2003}: 90; \citealt{Luffin2014}: 309). Their post-nominal placement is calqued on \ili{Swahili}:

\ea
{\ili{Kinubi} \citep[309]{Luffin2014}}\\
\gll wéle \textbf{kámsa} ma baná \textbf{árba}\\
     boy five with girl.\textsc{pl} four\\
\glt     `five boys and four girls'
\ex
{\ili{Swahili} \citep[309]{Luffin2014}}\\
\gll miti \textbf{mia} \textbf{tatu}\\
     tree hundred three\\
\glt     `three hundred trees'
\z

With cardinal {numerals}, the order is hundred + unit and thousand + unit respectively:

 

 \ea
{\ili{Kinubi} \citep[309]{Luffin2014}}\\

\gll   elf wáy\\
       thousand one\\
\glt      `one thousand'
\z

\ili{Kinubi} thus follows the \ili{Swahili} model:

\ea
{\ili{Swahili} \citep[309]{Luffin2014}}\\
\gll            elfu moja \\
                thousand one \\
\glt     `one thousand'
\z

Consider also a case of syntactic change induced by lexical {calquing}. \ili{Juba} \ili{Arabic} \textit{(fu)wata} ‘ground’ functions as an impersonal subject in weather expressions:

\ea
{\ili{Juba} \ili{Arabic} \citep[141]{Nakao2012}}\\
\gll   \textbf{(fu)watá} súkun\\
       ground hot\\
\glt     `It is hot.'
\z

Nakao (\citeyear[141]{Nakao2012}) shows that this is also the case in \ili{Acholi} and \ili{Ma’di}:

\ea
{\ili{Acholi} \citep[141]{Nakao2012}}\\
\gll \textbf{piiny} lyeet\\
     ground warm\\
\glt     `It is warm.'
\ex 
{\ili{Ma’di} \citep[141]{Nakao2012}}\\
\gll \textbf{vu} aci\\
     ground hot\\
\glt     `It is hot.'
\z

In fact, these types of sentences are widespread in Western Nilotic {substrate} languages, such as \ili{Dinka}, \ili{Jur}, \ili{Päri}, and \ili{Shilluk}:

\ea
{\ili{Dinka} \citep[202]{Nebel1979}}\\
\gll            \textbf{piny} a-tuc\\
                ground 3\textsc{sg}-warm\\
\glt     `It is warm.'
\z

  In both \ili{Juba} \ili{Arabic} and \ili{Kinubi} \textit{ras} ‘head’ also occurs in the complex {preposition} \textit{fi} \textit{ras} ‘on’:

\ea
\ea \ili{Juba} \ili{Arabic} \citep[141]{Nakao2012}\\
\gll     merísa fí fi \textbf{ras} terebéza\\
         beer \textsc{exs} on head table\\
\glt       `The beer is on the table.'

\ex
\ili{Kinubi} \citep[159]{Wellens2003}\\
\gll     fi \textbf{rá}\textbf{s} séder\\
         on head tree\\
\glt       `on top of the tree'
\z
\z

Nakao (\citeyear[141]{Nakao2012}) attributes this function of \textit{ras} to substratal influence from \ili{Acholi} and \ili{Ma’di}:

\ea
{\ili{Acholi} \citep[141]{Nakao2012}}\\
\gll            cib \textbf{wi}-meja\\
                put head-table\\
\glt     `Put it on the table.'
\z
 
However, other possible sources include Western Nilotic languages such as \ili{Belanda Bor}, \ili{Jur}, \ili{Päri} and \ili{Shilluk}:

\ea
{\ili{Jur}  (\citealt{PozzatiPanza1993}: 342)}\\
\gll     kedh ŋo \textbf{wi} tarabesa\\
         put 3\textsc{sg} head table\\
\glt     `Put it on the table.'
\z

Moreover, a {preposition} ‘on’ derived from the noun ‘head’ is also attested in \ili{Bongo} (Central \ili{Sudanic}) and \ili{Zande} (\ili{Niger-Congo}):

\ea
{\ili{Bongo} (\citealt{Moietal2014}: 39)}\\
\gll            ba \textbf{do} mbaa\\
                3\textsc{sg} on car\\
\glt     `He is on a car.'
\ex
{\ili{Zande} (\citealt{DeAngelis2002}: 288)}\\
\gll            mo mai he \textbf{ri} ngua\\
                2\textsc{sg} put 3\textsc{sg} on wood\\
\glt     `Put it on the wood.'
\z

The verb \textit{gal}/\textit{gale}/\textit{gali} ‘say’ is used in \ili{Juba} \ili{Arabic} and Ugandan Nubi as a {complementizer}, with \textit{verba} \textit{dicendi} and verbs of cognition:

\ea
\ea \ili{Juba} \ili{Arabic} \citep[469]{Miller2001}\\
\gll     úwo kélem \textbf{gal} úwo bi-ǧa\\
         3\textsc{sg} speak \textsc{comp} 3\textsc{sg} \textsc{irr}-come\\
\glt       `He said that he would come.'

\ex
Ugandan \ili{Kinubi} \citep[204]{Wellens2003}\\
\gll     úmon áruf \textbf{gal} fí difan-á al gi-ǧá\\
         3\textsc{pl} know \textsc{comp} \textsc{exs} guest-\textsc{pl} \textsc{rel} \textsc{prog}-come\\
\glt    `They know that there are are guests who are coming.' 
\z
\z

The use of a \textit{verbum} \textit{dicendi} as a {complementizer} resembles the situation in \ili{Bari},\footnote{Unsurprisingly, in \ili{Juba} \ili{Arabic} “the use of \textit{adi} as in \ili{Bari} [is] the most frequent […] in particular among speakers of \ili{Bari} origin” (\citealt[470]{Miller2001}; author's translation).} where \textit{adi} ‘say’ introduces direct speech (\citealt{Owens1997}: 163; \citealt{Miller2001}: 469): 

\ea

{\ili{Bari} \citep[469]{Miller2001}}\\
\gll    mukungu a-kulya \textbf{adi} nan d'ad'ar kakitak merya-mukanat\\
                sub-chief \textsc{pst}-say \textsc{comp} 1\textsc{sg} want worker fifty\\
\glt     `The sub-chief spoke saying: I want fifty workers.'
\z


 \subsubsection{Lexicon}

Since \ili{Bari} is the main {substrate} language of \ili{Juba} \ili{Arabic}, unsurprisingly it contributes most of its African-derived words: \textit{gúgu} ‘granary’ < \ili{Bari} \textit{gugu}; \textit{kení} ‘co-wife’ < \ili{Bari} \textit{köyini}; \textit{loɲumég} `hedgehog' < \ili{Bari} \textit{lónyumöng}; \textit{tóŋga} ‘pinch’ < \ili{Bari} \textit{toŋga}. In several cases, the \ili{Juba} \ili{Arabic} form can be traced to a specific dialect: \textit{d'oŋóŋ} ‘back of head’ < \ili{Pojulu} \textit{doŋoŋ}; \textit{láŋa} ‘wander’ < \ili{Mundari} \textit{laŋa} ‘travel’; \textit{nyéte} vs \textit{ŋéte} ‘black-eyed pea leaf’ < \ili{Bari} \textit{nyete} vs \ili{Kakwa}, \ili{Pojulu} \textit{ŋete}. Moreover, “more \ili{Bari} lexical items are being borrowed” in Youth \ili{Juba} \ili{Arabic} \citep[131]{Nakao2012}: \textit{kapaparát} ‘butterfly’ < \ili{Bari} \textit{kapoportat}; \textit{lukulúli} ‘bat’ < \ili{Bari} \textit{lukululi}. Several other {substrate} and {adstrate} languages have contributed to the lexicon of \ili{Juba} \ili{Arabic} (\citealt{Nakao2012,Nakao2015}): \textit{adúngú} ‘harp’ < \ili{Acholi} \textit{aduŋu}; \textit{b'ónǧ}\textit{o} ‘pumpkin’ < \ili{Bongo} \textit{b'onǧo}; \textit{báfura} ‘cassava’ < \ili{Dinka} \textit{bafora} ‘manioc, (sweet) cassava’; \textit{káwu} ‘cowpea’ < \ili{Ma’di} \textit{kau}; \textit{malangí} < bottle’ < Bangala/Lingala \textit{molangi}; \textit{kámba} ‘belt’ < \ili{Swahili} \textit{kamba}; \textit{imbíró} ‘palm tree’ < \ili{Zande} \textit{mbíró}. Some sixty lexical items found in the earliest records of Ugandan \ili{Kinubi} can be traced back to various {substrate} languages \citep{Avram2017talk}: \textit{lawoti} ‘neighbours’ < \ili{Acholi} \textit{lawoti} ‘fellow, friend’; \textit{korufu} ‘leaf’ < \ili{Bari} \textit{korofo} {\textasciitilde} \textit{kɔrɔ}\textit{pɔ} ‘leaves’; \textit{lwar} ‘abscess’ < \ili{Dinka} \textit{luär} ‘pain of a swelling’; \textit{seri} ‘fence’ < \ili{Lugbara} \textit{seri} ‘plant used for fencing’; \textit{mukuta} ‘key’ < \ili{Päri} \textit{mukuta}.

The influence of \ili{Luganda} and \ili{Swahili} as {adstrate} languages is already documented in early Ugandan \ili{Kinubi} \citep{Avram2017talk}: Ugandan \ili{Kinubi} \textit{kibra} {\textasciitilde} \textit{kibera} ‘forest’ < \ili{Luganda} \textit{e-kibira}, \textit{nyinveza} ‘fix’ < \ili{Luganda} \textit{nyweza} ‘make firm, hold firmly’; \textit{dirisa} ‘window’ < \ili{Swahili} \textit{dirisha}; \textit{kibanda} ‘shed’ < \ili{Swahili} \textit{kibanda} ‘small shed’. The lexicon of modern Ugandan \ili{Kinubi} is characterized by a large number of {loanwords} from \ili{Luganda} and \ili{Swahili} (\citealt{Wellens2003}; \citealt{Nakao2012}: 133–134), such as: \textit{mé(é}\textit{)mvu} ‘banana’ < \ili{Luganda} \textit{amaemvu} ‘bananas’; \textit{ntulége} ‘zebra’ < \ili{Luganda} \textit{e-ntulege}; \textit{karibísha} ‘welcome’ < \ili{Swahili} \textit{karibisha} ‘welcome’; \textit{sangá} {\textasciitilde}  \textit{šangá} ‘be surprised’ < \ili{Swahili} \textit{shangaa}. In some cases, these {loanwords} have replaced previously attested {compounds} consisting of \ili{Arabic}-derived elements:\footnote{See also Tosco \& Manfredi (\citeyear[509]{ToscoManfredi2013}).} early Ugandan \ili{Kinubi} \textit{mária} \textit{bitá} \textit{murhúm} ‘widow’, lit. ‘wife of the deceased’ vs. modern Ugandan \ili{Kinubi} \textit{mamwándu} ‘widow’ < \ili{Luganda} \textit{nnamuwandu}. As for the lexicon of modern \ili{Kenyan Kinubi}, it is strongly influenced by \ili{Swahili}. \citet{Luffin2004} lists some 170 {loanwords} from \ili{Swahili} (out of approximately 1,400 words recorded), from a wide range of domains, for example: \textit{barabára} ‘highway’ < \ili{Swahili} \textit{barabara}; \textit{serikáli} ‘government’ < \ili{Swahili} \textit{serikali}; \textit{tafaúti} ‘difference’ < \ili{Swahili} \textit{tafauti}; \textit{úza} ‘sell’ < \ili{Swahili} \textit{ku-uza}. \ili{Swahili} has also contributed several function words: \textit{badáye} ‘after’ < \ili{Swahili} \textit{baadaye} ‘afterwards’; \textit{íle} ‘these’ < \ili{Swahili} \textit{ile}; \textit{na} ‘and, with’ < \ili{Swahili} \textit{na}. \ili{Kenyan Kinubi} lexical items have occasionally undergone semantic shift or semantic {extension} under the influence of the meanings of their \ili{Swahili} counterparts \citep[315]{Luffin2014}: \textit{destúr} ‘tradition’, cf. \ili{Swahili} \textit{desturi} ‘tradition’; \textit{fáham} ‘to understand, to remember’, cf. \ili{Swahili} -\textit{fahamu} ‘to understand, to remember’. 

In some cases, the exact origin of {loanwords} found in \ili{Juba} \ili{Arabic} cannot be established: \textit{búra} ‘cat’ < \ili{Acholi}, \ili{Bongo}, \ili{Dinka}, \ili{Päri} \textit{bura}, \ili{Didinga} \textit{buura}; \textit{daŋá} ‘bow’ < \ili{Bari}, \ili{Jur} \textit{daŋ}, \ili{Didinga} \textit{d'anga}, \ili{Dinka} \textit{dhaŋ}; \textit{pondú} ‘cassava leaf’ < Bangala, \ili{Kakwa}, Lingala \textit{pondu}, \ili{Pojulu} \textit{pöndu}. The same holds for a number of {loanwords} attested in early Ugandan \ili{Kinubi} \citep{Avram2017talk}: \textit{bongo} ‘cloth’ < \ili{Acholi}, \ili{Lendu}, \ili{Lugbara}, \ili{Zande} \textit{bongo}, \ili{Bari} \textit{boŋgo}; \textit{godogodo} ‘thin from illness’ < \ili{Acholi}, \ili{Avokaya}, \ili{Bari}, \ili{Baka}, \ili{Lotuho}, \ili{Moru}, \ili{Zande} \textit{godogodo} ‘thin, sick(ly)’; \textit{mukungu} ‘headman’ < \ili{Acholi} \textit{mukuŋu}, \ili{Bari} \textit{mʊkʊ}\textit{ŋgʊ}, \ili{Luganda} \textit{o-mukungu}, \ili{Lugbara} \textit{mukungu} ‘(sub-) chief’. This is also true of several \ili{Kinubi} words attested in more recent sources (\citealt{Wellens2003}; \citealt{Nakao2012}: 133–134): \textit{júju} ‘shrew’ < \ili{Bari} \textit{juju}, \ili{Ma’di} \textit{juju}; \textit{kingílo} ‘rhinoceros’ < \ili{Avokaya} \textit{kiŋgili}, \ili{Moru} \textit{kingile}. In some cases, the occurrence of alternative forms is due to their different SLs: \textit{ban\textbf{ǧ}a} ‘debt’ < \ili{Bari} \textit{ban\textbf{j}a}, \ili{Lugbara} \textit{ban\textbf{j}a}, \ili{Luganda} \textit{e-bban\textbf{j}a} vs. \textit{ban\textbf{y}a} ‘debt’ < \ili{Acholi} \textit{ban\textbf{y}a}. 

Under the influence of the {substrate} and {adstrate} languages, some \ili{Arabic}-derived lexical items have undergone semantic {extension}, thereby becoming polysemous in \ili{Juba} \ili{Arabic} \citep[136]{Nakao2012}, e.g. \textit{gówi} ‘hard; difficult’, cf. \ili{Acholi} \textit{tek}, \ili{Bari} \textit{logo’}, \ili{Lotuho} \textit{gol}, \ili{Ma’di} \textit{okpo}, \ili{Swahili} \textit{kali}.

\ili{Juba} \ili{Arabic} “compensates its lexical gaps through the lexification of \ili{Arabic} morphosyntactic sequences”  (\citealt{ToscoManfredi2013}: 509). A case in point are \ili{Juba} \ili{Arabic} {compounds}, formed via juxtaposition or with their two members linked by the possessive particle \textit{ta} (\citealt{Manfredi2014relex}: 308–309). These include calques after several {substrate} languages \citep[136]{Nakao2012}, e.g. \textit{ída} \textit{ta} \textit{fil} ‘elephant trunk’, cf. \ili{Acholi} \textit{ciŋ} \textit{lyec}, \ili{Bari} \textit{könin} \textit{lo} \textit{tome}, \ili{Dinka} \textit{ciin} \textit{akɔɔn}, \ili{Jur} \textit{ciŋ} \textit{lyec}, \ili{Lotuho} \textit{naam} \textit{tome}, \ili{Shilluk} \textit{bate} \textit{lyec}, lit. ‘arm (of) elephant’. \ili{Kinubi} also exhibits a number of calques (\citealt{Nakao2012}; \citealt{Avram2017talk}; Manfredi, this volume).\ia{Manfredi, Stefano@Manfredi, Stefano} Some of these {compounds} and phrases can be traced to several SLs, as in the following early Ugandan \ili{Kinubi} examples \citep{Avram2017talk}: \textit{gata} \textit{kalam} ‘decide, judge’, cf. \ili{Acholi} \textit{ŋɔlɔ} \textit{kop} ‘decide, give judgment’, \ili{Bongo} \textit{ad'oci} \textit{kudo}, \ili{Jur} \textit{ŋɔl} \textit{lubo}, \ili{Päri} \textit{ŋondi} \textit{lubo}, lit. ‘cut word/speech’; \ili{Dinka} \textit{wèt} \textit{tèm} ‘decide, give the sentence’, lit. ‘word cut’; \textit{jua} \textit{bita} \textit{ter} ‘nest’, cf. \ili{Acholi} \textit{ot} \textit{winyo}, \ili{Bari} \textit{kadi-na-kwen}, \ili{Belanda Bor} \textit{kwɔt} \textit{winy}, \ili{Shilluk} \textit{wot} \textit{winyo}, \ili{Zande} \textit{dumô} \textit{zirê}, lit. ‘house (of) bird’. Other calques, presumably more recent ones, reflect the growing influence of \ili{Swahili} on \ili{Kenyan Kinubi} \citep[315]{Luffin2014}: \textit{bakán} \textit{wá}\textit{y} ‘together’, cf. \ili{Swahili} \textit{pamoja} ‘together’, lit. ‘place one’, \textit{mára} \textit{wá}\textit{y} \textit{wáy} ‘seldom’, cf. \ili{Swahili} \textit{mara} \textit{moja} \textit{moja} ‘seldom’, lit. ‘time one one’.

To conclude, {SL} agentivity accounts for the small number of {loanwords} and calques recorded in the earliest stage (i.e. pidginization) of \ili{Juba} \ili{Arabic} and \ili{Kinubi}. At a later stage (i.e. after nativization), the larger number of {loanwords} and calques is a result of borrowing under {RL} agentivity.


 \section{Arabic-lexifier pidgins in the Middle East}\label{sec:pid}


 \subsection{Current state and historical development}


Several \ili{Arabic}-{lexifier} pidgins have emerged in the Middle East. These include \ili{Romanian Pidgin} \ili{Arabic}, \ili{Pidgin Madame}, \ili{Jordanian Pidgin} Arabic, and \ili{Gulf Pidgin Arabic}. The first three can be classified as work force pidgins.\footnote{These are pidgins which “came into being in work situations” \citep[28]{Bakker1995}.} \ili{Gulf Pidgin Arabic} also started out as work force {pidgin} \citep[83]{Smart1990}, but it is now an interethnic contact language \citep[13]{Avram2014Pidgin}.\footnote{That is, one which is “used not just for trade, but also in a wide variety of other domains” \citep[28]{Bakker1995}.}

\ili{Romanian Pidgin} \ili{Arabic} \citep{Avram2010} was a short-lived {pidgin}, formerly used on \ili{Romanian} well sites in Iraq, in locations in the vicinity of Ammara, Basra, Kut, Nassiriya, Rashdiya and Rumaila. \ili{Romanian Pidgin} \ili{Arabic} emerged after 1974, when \ili{Romanian} well sites started operating in Iraq. Romanians typically made up two thirds of the oil crews, with Arabs making up the final third. The first Gulf War and the subsequent withdrawal of the {Romanian} oil rigs put an end to the use of \ili{Romanian Pidgin} \ili{Arabic}. 

Immigration of Sri Lankan women to \ili{Arabic}-speaking countries is reported to have started in 1976 \citep[16]{Bizri2010}, but the large influx into Lebanon came later, in the early 1990s. \ili{Pidgin Madame} is spoken in Lebanon by Sri Lankan female domestic workers and their Arab employers, mostly in the urban centres of the country. 

  \ili{Jordanian Pidgin} \ili{Arabic} (\citealt{Al-Salman2013}) is used in the city of Irbid, in the Ar-Ramtha district in the north of Jordan, in interactions between Jordanians and Southeast Asian migrant workers of various linguistic backgrounds. However, only \ili{Jordanian Pidgin} \ili{Arabic} as spoken by Bengalis is documented.  

\ili{Gulf Pidgin Arabic} is a blanket term designating the varieties of pidginized \ili{Arabic} used in {Saudi Arabia} and the countries on the western coast of the Arab Gulf, i.e. Kuwait, the United Arab Emirates, Oman, Bahrain, and Qatar. 


 
 \subsection{Contact languages}


The main languages involved in the emergence of \ili{Romanian Pidgin} \ili{Arabic} are \ili{Romanian}, \ili{Egyptian Arabic} (spoken by immigrant workers), and \ili{Iraqi} \ili{Arabic} (IA). A small minority of the participants in the language-contact situation had some knowledge of \ili{English}.

The other pidginized varieties of \ili{Arabic} in the Middle East share the characteristic of having various Asian languages as their {substrate}.\footnote{Bizri (\citeyear[385]{Bizri2014}) therefore suggests the cover term “Asian Migrant \ili{Arabic} pidgins”.} For \ili{Pidgin Madame}, the main contact languages are \ili{Lebanese} \ili{Arabic}, as the {lexifier} language, and \ili{Sinhalese}. Another language, with a much smaller contribution, is \ili{English}. In the case of \ili{Jordanian Pidgin} \ili{Arabic}, the contact languages are mainly \ili{Jordanian} \ili{Arabic} (JA) and \ili{Bengali}. The contribution of \ili{English} is very limited. As for \ili{Gulf Pidgin Arabic}, it emerged in a contact situation of striking complexity. On the one hand, \ili{Arabic}, the {lexifier} language, is represented by several dialects, which are not all subsumed under what is known as \ili{Gulf} \ili{Arabic} (GA), in spite of what the name of the {pidgin} suggests. On the other hand, the number of languages spoken by the immigrant workers is staggering: for instance, in the United Arab Emirates the 200 nationalities and 150 ethnic groups speak some 150 languages. Adding to the complexity of the language-contact situation is the fact that these languages are typologically diverse. Last but not least, \ili{English} also plays a role in interethnic communication, particularly in the service sector.


 
 \subsection{Contact-induced changes}
 \subsubsection{Phonology}

The phonology of all the pidginized varieties of \ili{Arabic} in the Middle East exhibits the outcomes of {SL} agentivity, which also accounts for the occurrence of considerable intra- and  inter-speaker variation (\citealt{Avram2010}: 21–22; \citealt{Bizri2014}: 393; \citealt{Avram2017article}: 133).

Consider first \ili{Romanian Pidgin} \ili{Arabic}. The following are features characteristic of speakers with \ili{Romanian} as their first language (L1). The phrayngeals are either replaced or deleted: \textit{\textbf{h}\kern -.5ptabib} ‘friend’ < \ili{IA}/\ili{EA} \textit{\textbf{ḥ}\kern .5ptabīb}; \textit{mufta} ‘key’ < \ili{IA}/\ili{EA} \textit{muftā\kern -.5pt\textbf{ḥ}}; \textit{saa} ‘hour’ < \ili{IA}/\ili{EA} \textit{sā\textbf{ʕ}a}. Plain consonants are substituted for {pharyngealized} ones: \textit{hala\textbf{s}} ‘ready’ < \ili{IA}/\ili{EA} \textit{ḫalā\textbf{ṣ}}. Both velar fricatives are replaced: \textit{\textbf{h}amsa} ‘five’ < \ili{IA}/\ili{EA} \textit{\textbf{ḫ}\kern .5ptamsa}; \textit{šo\textbf{g}ol} ‘work (\textsc{n})’ < \ili{IA} \textit{šu\textbf{ɣ}(u)l}. Geminate consonants are degeminated: \textit{si\textbf{t}a} ‘six’ < \ili{IA}/\ili{EA} \textit{si\textbf{tt}a}. There is no distinction between short and long vowels, either in lexical items of \ili{Arabic} origin or in those from \ili{English}: \textit{l\textbf{a}zim} ‘must’ < \ili{IA}/\ili{EA} \textit{l\textbf{ā}zim}; \textit{sl\textbf{i}p} ‘sleep’ < \ili{English} \textit{sleep}. A feature typical of speakers with \ili{Iraqi} or \ili{Egyptian Arabic} as L1 is the substitution of /b/ for \ili{Romanian} or \ili{English} /p/ and /v/: \textit{\textbf{b}i\textbf{b}ul} ‘people, men’ < \ili{English} \textit{\textbf{p}eo\textbf{p}le}; \textit{gi\textbf{b}} ‘give, bring’ < \ili{English} \textit{gi\textbf{v}e}.

Consider next several selected features, generally typical of \ili{Pidgin Madame}, \ili{Jordanian Pidgin} \ili{Arabic}, and \ili{Gulf Pidgin Arabic}.  Pharyngeals are either replaced: \ili{Pidgin Madame} \textit{\textbf{h}\kern -.5ptareb} ‘war’ < LA \textit{\textbf{ḥ}\kern .5ptareb}; \ili{Jordanian Pidgin} \ili{Arabic} \textit{bisalli\textbf{h}} ‘repair’ < \ili{JA} \textit{biṣalli\textbf{ḥ}} ‘repair.\textsc{impf}.\textsc{3sg.m}’; \ili{Gulf Pidgin Arabic} \textit{a\textbf{k}san} ‘best’ < \ili{GA} \textit{a\textbf{ḥ}san}, \textit{\textbf{h}ut} ‘put’ < \ili{GA} \textit{\textbf{ḥ}uṭṭ} ‘put.\textsc{imp.2sg.m}’; or deleted: \ili{Pidgin Madame} \textit{\textbf{ē}ki} ‘cry’ < LA \textit{ə\textbf{ḥ}ki} ‘cry.\textsc{imp}.2\textsc{sg}.\textsc{f}’; \ili{Jordanian Pidgin} \ili{Arabic} \textit{\textbf{a}rabi} ‘\ili{Arabic}’ < \ili{JA} \textit{\textbf{ʕ}arabi}; \ili{Gulf Pidgin Arabic} \textit{araf} ‘know’ < \ili{GA} \textit{\textbf{ʕ}araf}. The {pharyngealized} consonants are replaced by plain counterparts:  \ili{Pidgin Madame} \textit{\textbf{s}arep} ‘envelope’ < LA \textit{\textbf{ẓ}\kern -.75ptaref}; \ili{Jordanian Pidgin} \ili{Arabic} \textit{bandora} ‘tomato’ < \ili{JA} \textit{ban\textbf{ḍ}ōra}; \ili{Gulf Pidgin Arabic} \textit{hala\textbf{s}} `finish' < \ili{GA} \textit{ḫalā\textbf{ṣ}}; or they are realized as retroflex: \ili{Pidgin Madame} \textit{\textbf{ʈ}awīle} ‘long’ < LA \textit{\textbf{ṭ}awīle} ‘long.\textsc{f}.\textsc{sg}’. The velar fricatives are replaced by velar stops or, less frequently, by /h/: \ili{Pidgin Madame} \textit{so\textbf{k}on} ‘warm’ < LA \textit{su\textbf{ḫ}un} ‘warm’, \textit{so\textbf{g}ol} < LA \textit{šə\textbf{ɣ}əl} ‘work’; \ili{Jordanian Pidgin} \ili{Arabic} \textit{\textbf{k}amsa} ‘five’ < \ili{JA} \textit{\textbf{ḫ}\kern .5ptamsa}, \textit{su\textbf{k}ul} ‘work (\textsc{n})’ < \ili{JA} \textit{šu\textbf{ɣ}l}, \textit{za\textbf{g}īr} ‘small’ < \ili{JA} \textit{ṣa\textbf{ɣ}īr}; \ili{Gulf Pidgin Arabic} \textit{\textbf{k}ubus} ‘bread’ < \ili{GA} \textit{\textbf{ḫ}\kern .5ptubz}; \textit{\textbf{h}alas} ‘finish’ < \ili{GA} \textit{\textbf{ḫ}\kern .5ptalaṣ}; \textit{yisto\textbf{k}ol} ‘work’ < \ili{GA} \textit{yištu\textbf{ɣ}ul} ‘work.\textsc{impf.3sg.m}’, \textit{šu\textbf{g}l} ‘work’ < \ili{GA} \textit{šu\textbf{ɣ}l}. Geminate consonants generally undergo {degemination} (\citealt{Næss2008}: 36; \citealt{Avram2014Pidgin}: 15): \ili{Jordanian Pidgin} \ili{Arabic} \textit{si\textbf{t}in} ‘sixty’ < \ili{JA} \textit{si\textbf{tt}īn}; \ili{Gulf Pidgin Arabic} \textit{si\textbf{t}a} ‘six’ < \ili{GA} \textit{si\textbf{tt}a}.

  Moreover, consonants not found in the L1s of the users of \ili{Gulf Pidgin Arabic} may also be replaced. For instance, \ili{Indonesian}, \ili{Javanese}, \ili{Sinhalese} and \ili{Tagalog} speakers may substitute [p] for /f/: \ili{Pidgin Madame} \textit{\textbf{p}alē\textbf{p}il} ‘falafel’ < LA \textit{\textbf{f}alē\textbf{f}il}; \ili{Jordanian Pidgin} \ili{Arabic} \textit{\textbf{p}i} ‘in’ < \ili{JA} \textit{fī}; \ili{Gulf Pidgin Arabic} \textit{na\textbf{p}ar} ‘person’ < \ili{GA} \textit{na\textbf{f}ar}; \ili{Indonesian} and \ili{Sinhalese} speakers may realize /z/ as [s] or [ʤ]: \ili{Pidgin Madame} \textit{e\textbf{s}a} ‘if’ < LA \textit{iza}; \ili{Gulf Pidgin Arabic} \textit{\textbf{s}ēn} {\textasciitilde} \textit{\textbf{ʤ}ēn} ‘good’ < \ili{GA} \textit{\textbf{z}ēn} (\citealt{Bizri2014}: 393; \citealt{Avram2017article}: 133). \ili{Bengali} and \ili{Sinhalese} speakers may replace /š/ with [s]: \ili{Pidgin Madame} \textit{\textbf{s}ū} ‘what’ < LA \textit{\textbf{š}ū}; \ili{Jordanian Pidgin} \ili{Arabic} \textit{\textbf{s}u} ‘what’ < \ili{JA} \textit{\textbf{š}ū}.

Finally, although phonetically long vowels do occur, {vowel length} is not distinctive, as shown by the occurrence of variation, e.g. \ili{Gulf Pidgin Arabic} \textit{b\textbf{a}d\textbf{e}n} {\textasciitilde} \textit{b\textbf{a}d\textbf{ē}n} ‘then’ < \ili{GA} \textit{baʕd\textbf{ē}n}. 


 \subsubsection{Syntax} 

There is relatively little that can be attributed to {SL} agentivity in the syntax of the \ili{Arabic}-{lexifier} pidgins in the Middle East (\citealt{Almoaily2013}; \citealt{Al-Salman2013}; \citealt{Avram2014Pidgin}; \citealt{Bizri2014}; \citealt{Avram2017article}; \citealt{Bakir2017}).

Since the {substrate} of these varieties, with the exception of \ili{Romanian Pidgin} \ili{Arabic}, consists of many SOV languages, e.g. \ili{Bengali}, \ili{Hindi}/\ili{Urdu}, \ili{Malayalam}, \ili{Punjabi}, \ili{Persian}, \ili{Sinhalese}, \ili{Tamil}, this {word order} is occasionally attested (\citealt{Avram2017article}: 133–134; \citealt{Bizri2014}: 403). For instance, direct objects may occur in pre-verbal position:                                                              


\ea
\ea \ili{Pidgin Madame} \citep[227]{Bizri2010}\\
\gll     misʈer kilot sīli\\
         mister underwear take off\\
\glt       `Mister takes off his underwear.'     

\ex \ili{Gulf Pidgin Arabic} \citep[133]{Avram2017article}\\
\gll     ana čiko sūp\\
         1\textsc{sg} child see\\
\glt       `I will see my children.'                                        
\z                                             
\z

In attributive possession constructions the order of constituents is possessor–possessee:      

\ea
\ea \ili{Pidgin Madame} \citep[198]{Bizri2010} \\
\gll     kullu māmā benet \\
         all mother girl    \\
\glt `All mother’s girls.'

\ex \ili{Gulf Pidgin Arabic} (\citealt{Næss2008}: 87)\\
\gll     ana jawd bādēn ysīr Jakarta stokol\\
         1\textsc{sg} husband then go Jakarta work\\
\glt       `Then my husband went to work in Jakarta.' 
\z
\z

Adjectives generally precedes the nouns they modify:

\ea
{ \ili{Pidgin Madame} \citep[119]{Bizri2010}}\\
\gll   \textbf{bī}\textbf{r} bēt\\
       big house\\
\glt     `A big house.'
\z

Similarly, adverbs precede the adjectives they modify:

\ea
\ea \ili{Pidgin Madame} \citep[119]{Bizri2010}\\
\gll     \textbf{ʈīr} gūɖ\\
         very good\\
\glt       `very good'

\ex
\ili{Gulf Pidgin Arabic} \citep[25]{Avram2014Pidgin}\\
\gll     \textbf{sem}-\textbf{sem} kalām\\
         same speak \\
\glt       `They speak in the same way.'
 \z
 \z

  Occasional instances of postpositions are attested:
  
\ea
\ea \ili{Pidgin Madame} \citep[132]{Bizri2010}\\
\gll     mister \textbf{mayik} masārē\\
         mister with money\\
\glt    `Mister has the money.'

\ex
\ili{Gulf Pidgin Arabic} \citep[25]{Avram2014Pidgin}\\
\gll     zamal \textbf{fok} \\
         camel above\\
\glt       `Above the camel.'
\z
\z

Interestingly, \ili{Pidgin Madame} has a focalized negative {copula}, derived etymologically from \ili{English} \textit{no}:

\ea
{ \ili{Pidgin Madame} \citep[133]{Bizri2010}}\\
\gll   māmā bīrūt \textbf{no}\\
       mother Beirut \textsc{neg}.\textsc{foc}\\
\glt     `It’s not in Beirut that my mother is.'
\z

This resembles the \ili{Sinhalese} negator \textit{nemiyi}, which “is used only in focalized phrases” \citep[69]{Bizri2010}:

\ea
{ \ili{Pidgin Madame} \citep[69]{Bizri2010}}\\
\gll   bat kāve mama \textbf{nemeyi}\\
       rice ate 1\textsc{sg} \textsc{neg}.\textsc{foc}\\
\glt     `It is not I who ate the rice.'
\z

 \subsubsection{Lexicon}

Imposition under {SL} agentivity accounts for the fact that there are few instances of {transfer} of lexical items from the various SLs into the non-dominant {RL} (i.e. the {pidgin}). 

The lexicon of \ili{Romanian Pidgin} \ili{Arabic} includes words of \ili{Romanian} and \ili{English} origin \citep[32]{Avram2010}: \textit{mašina} ‘car’ < \ili{Romanian} \textit{maşină}, \textit{sonda} ‘oil rig’ < \ili{Romanian} \textit{sonda}; \textit{spik} ‘speak, say, tell’ < \ili{English} \textit{speak}, \textit{tumač} ‘much, many’ < \ili{English} \textit{too} \textit{much}. Occasionally, non-\ili{Arabic} words undergo semantic {extension} under the influence of phonetically similar \ili{Arabic} words \citep[32]{Avram2010}: \textit{gib} ‘give; bring’ < \ili{English} \textit{give}, cf. \ili{EA} \textit{gīb} ‘bring.\textsc{imp.2sg.m}’. 

  The lexicon of all the other pidginized varieties of \ili{Arabic} spoken in the Middle East includes {loanwords} from \ili{English}: \ili{Pidgin Madame} \textit{ambasi} `embassy' < \ili{English} \textit{embassy}; \textit{go} `go' < \ili{English} \textit{go}, \textit{kam} `come' < \ili{English} \textit{come}, \textit{no} \textit{gūɖ} ‘bad’ < \ili{English} \textit{no} \textit{good}, \textit{oké} `OK' < \ili{English} \textit{OK}; \ili{Jordanian Pidgin} \ili{Arabic} \textit{bēbi} ‘child’ < \ili{English} \textit{baby}, \textit{finiš} ‘finish’ < \ili{English} \textit{finish}, \textit{fisa} ‘visa’ < \ili{English} \textit{visa}; \ili{Gulf Pidgin Arabic} \textit{hazband} `husband' < \ili{English} \textit{husband}, \textit{pēšent} ‘patient’ < \ili{English} \textit{patient}. However, as noted by \citet[113]{Smart1990} concerning \ili{Gulf Pidgin Arabic}, “it is difficult to say […] whether they are a true part of the {pidgin}” or rather nonce borrowings.  

Given the extreme diversity of the {substrate}, it is not surprising that only a few words from the SLs have made it into the lexicon of \ili{Gulf Pidgin Arabic} (\citealt{Avram2017article}: 134–135): \textit{ača} ‘fine’ < \ili{Urdu} \textit{achā} ‘good, very well’, \textit{ǧaldi} {\textasciitilde}  \textit{ǧeldi} < \ili{Hindi}/\ili{Urdu} \textit{jaldī} ‘quick’. 

\ili{Jordanian Pidgin} \ili{Arabic} and \ili{Gulf Pidgin Arabic} exhibit light-verb constructions which may well be calques on \ili{Bengali} (noun/adjective + \textit{kara} ‘make’) and\slash or \ili{Hindi}/\ili{Urdu} (noun/adjective + \textit{karnā} ‘make’) and/or \ili{Persian} – noun/adjective + \textit{kardan} ‘make’): \ili{Jordanian Pidgin} \ili{Arabic} \textit{sawwi} \textit{zadīd} ‘renew’, lit. ‘make new’; \ili{Gulf Pidgin Arabic} \textit{sawwi} \textit{suāl} ‘ask’, lit. ‘make a question’, \textit{sawwi} \textit{zalān} ‘upset’, lit. ‘make angry’. 

\section{Conclusion}\label{sec:conc}

This chapter has shown that \ili{Arabic}-{lexifier} contact languages emerged primarily through {imposition} under {SL} agentivity, in line with the typology of contact languages (\citealt{Winford2005}: 396; 2008: 128). 

The effects of {imposition} are most obvious in the phonology, syntax and the syntax-semantics interface, and to a lesser extent in the morphology and the lexicon. In the phonology, {SL} agentivity induces the loss or replacement of certain phonemes not found in the SLs. However, there are also instances of {imposition} in the sense of {transfer} from the SLs. As seen, for example, in \ili{Turku} and \ili{Bongor} \ili{Arabic}, some consonants occur only in {loanwords} from the {substrate} languages. The occurrence of such {loanwords} confirms the fact that {imposition} under {SL} agentivity may include {transfer} of lexical items into the {RL}. Borrowing under {RL} agentivity has generally played a far less significant role in the development of \ili{Arabic} pidgins and creoles. As expected, it mostly involves {transfer} of lexical items; these may lead to the borrowing of certain consonant phonemes, as seen in, for example, \ili{Juba} \ili{Arabic} and \ili{Kinubi}. Finally, borrowing has been shown to include {transfer} of morphological material as well.

A notable difference between \ili{Juba} \ili{Arabic} and \ili{Kinubi} on the one hand, and the \ili{Arabic}-{lexifier} pidgins in the Middle East on the other hand, resides in the {relative} weight of {imposition} under {SL} agentivity and borrowing under {RL} agentivity. As we have seen, \ili{Juba} \ili{Arabic} and \ili{Kinubi} exhibit the effects of both {imposition} in their earliest stage (i.e. pidginization), and of borrowing in their latest stage (i.e. nativization). In contrast, {imposition} is pervasive in the \ili{Arabic}-{lexifier} pidgins in the Middle East, given that these varieties have not undergone nativization.

  There are still a number of issues awaiting resolution. For instance, the identification of the SLs is rendered difficult by their number and typological diversity. This difficulty is further compounded by the fact that some {substrate} languages are still under-researched. This is particularly the case of the {substrate} languages of \ili{Juba} \ili{Arabic} and \ili{Kinubi}. Also, the distinction between {substrate} and {adstrate} languages is blurred \citep[132]{Nakao2012}, particularly when varieties emerge and develop \textit{in} \textit{situ}, as, for example, with \ili{Juba} \ili{Arabic}. Further research also needs to consider the effects of the existence of a {creole} continuum in \ili{Juba} \ili{Arabic} as well as of bilingual and {monolingual} speakers of the language on the {relative} importance of restructuring, {imposition} and borrowing. The extent of restructuring and {imposition}, for instance, is presumably much greater in basilectal and L2 varieties, as opposed to acrolectal and {monolingual} varieties of the language. The same holds for \ili{Bongor} \ili{Arabic}, which, as shown, appears to be undergoing depidginization. Last but not least, further investigations are necessary to establish whether \ili{Gulf Pidgin Arabic} is evolving towards stabilization, possibly becoming closer to its {lexifier} via borrowing of morphological material, or is rather undergoing constant repidginization, essentially via {imposition}.  

\section*{Further reading}
\begin{furtherreading}
\item \citet{Miller1993}, \citet{Nakao2012}, and \citet{Luffin2014} illustrate in detail substratal and adstratal influence on \ili{Juba} \ili{Arabic} and \ili{Kinubi}.
\item \citet{Avram2019} analyzes the substratal input in the lexicon of \ili{Turku}.
\item \citet{Avram2017article} and \citet{Bakir2017} discuss the various sources of \ili{Gulf Pidgin Arabic}.
\end{furtherreading}

\section*{Abbreviations}
\setlength{\columnsep}{30pt}
\begin{multicols}{2}
\begin{tabbing}
\textsc{ipfv} \hspace{1em} \= before common era\kill
\textsc{1, 2, 3} \> 1st, 2nd, 3rd person \\
{ChA} \> {Chadian} {Arabic} \\
{EA} \> {Egyptian Arabic} \\
\textsc{exs} \> {existential} \\
\textsc{foc} \> focus \\
{GA} \> Gulf Arabic \\
{IA} \> {Iraqi} {Arabic} \\
\textsc{impf} \> imperfect (prefix conjugation) \\
{JA} \> {Jordanian} {Arabic} \\
L1 \> first language\\
L2 \> second language\\
\textsc{n} \> noun \\
\textsc{nc} \> noun class \\
\textsc{neg} \> negative \\
\textsc{pass} \> {passive} \\
\textsc{pst} \> past \\
\textsc{pl} \> plural \\
\textsc{poss} \> possessive \\
\textsc{prf} \> perfect (suffix conjugation) \\
\textsc{prox} \> proximal \\
\textsc{rel} \> {relative} \\
{RL} \> {recipient language} \\
SA \> {Sudanese} {Arabic} \\
{SL} \> {source language} \\
\textsc{sg} \> singular
\end{tabbing}
\end{multicols}


\sloppy\printbibliography[heading=subbibliography,notkeyword=this]\end{document}
