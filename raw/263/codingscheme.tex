\documentclass[output=paper]{langsci/langscibook}
\ChapterDOI{10.5281/zenodo.4018372}

\author{Simeon Floyd\affiliation{Department of Anthropology, Universidad San Francisco de Quito}\and Giovanni Rossi\affiliation{Department of Sociology, University of California, Los Angeles}\lastand N. J. Enfield\affiliation{Department of Linguistics, The University of Sydney}}

\title{A coding scheme for recruitment sequences in interaction}

\abstract{This chapter provides an annotated coding scheme for analyzing recruitment sequences in video-recorded social interaction. The scheme provided a basis for the research presented in the eight language-specific chapters of this book, and as such it gives necessary context for understanding the comparative project reported on here and in associated work. It is also intended to serve as a stepping stone for other researchers to use in the analysis of recruitment sequences in other languages. The scheme features guidelines for building collections and aggregating cases based on interactionally relevant similarities and differences among instances. The questions and categories featured in the scheme are motivated by inductive observations of interactional data, grounded in the framework outlined in the introduction to this volume.}
\begin{document}
\maketitle
\label{sec:coding}

\section{Introduction}

The present coding scheme provides a way to systematically analyze a core set of formal and interactional features of recruitment sequences (defined in \chapref{sec:intro}, \sectref{sec:intro:4}). The scheme is the product of the collaborative qualitative study of such sequences in different languages, based on the analysis of video recordings of social interaction and focusing on the details of language and other conduct surrounding recruitments. Such analysis allowed us to identify recurrent social-interactional dimensions and patterns of language usage, leading to the formulation of questions aimed at capturing these aspects in different languages. The coding scheme is therefore inductively derived from an iterative process of observation, analysis, and group discussion of naturally occurring data (see also \citealt{StiversEnfield2010,DingemanseEtAl2016}, among others).

The coding scheme (\sectref{sec:coding:6}) is preceded by a definition of the phenomenon (\sectref{sec:coding:2}), further specifications for inclusion/exclusion of a case from the data considered (\sectref{sec:coding:3}), instructions for sampling and collecting cases (\sectref{sec:coding:4}), and general guidelines for coding (\sectref{sec:coding:5}). The questions and entries in the body of the scheme are extensively annotated to aid in the understanding of distinctions and replicability of coding.\footnote{We include all of the original questions and entries. However, note that a few categories were not coded with sufficient reliability by the researchers involved in this project; such categories may require further training and calibration of coders, or else reformulation to make reliable coding possible.}

Besides documenting the analytical procedure of the project, the scheme is published here as a reference for future work, to foster comparable and cumulative research in the interactional domain of recruitment. The scheme can be applied to any type of face-to-face naturally occurring interaction featuring people getting others to do things for or with them.

\section{Definition and terminology}\label{sec:coding:2}

Recruitment sequences are defined as in \chapref{sec:intro}, \sectref{sec:intro:4}. A recruitment sequence is a sequence of two moves with the following characteristics:

\begin{description}
\item[Move A:] participant A says or does something to participant B, or that B can see or hear;
\item[Move B:] participant B does a practical action for or with participant A that is fitted to what A has said or done.
\end{description}

For the purpose of this coding scheme, the following components of a \textsc{recruitment sequence} are identified:

\begin{itemize}
\item two participants: \textsc{A} (the \textsc{recruiter}) producing the instigating action, and \textsc{B} (the \textsc{recruitee}) responding to it;
\item \textsc{move a}: the instigating or recruiting action;
\item \textsc{move b}: an action addressing the relevance of immediate cooperation as a result of the previous instigating action, including:

\begin{itemize}
\item \textsc{fulfillment}: a practical action involving physical work performed for or with A;
\item \textsc{rejection}: the conveyance of inability or unwillingness to fulfill the recruitment;
\item \textsc{deferment} of fulfillment;
\item \textsc{initiation of repair} (often leading to fulfillment after repair).
\end{itemize}

If B \textsc{ignores} Move A, then the sequence must include an explicit, on-record practice of initiating recruitment and/or pursuit of response in order to be included.

The sequence can be expanded by a \textsc{move c}, doing \textsc{acknowledgment}.
\end{itemize}

\section{Further specifications for the inclusion/exclusion of cases}\label{sec:coding:3}

In including and excluding cases for coding, there are a number of further specifications that can help to delimit the phenomenon of recruitment from other kinds of related sequences of interaction.

\begin{enumerate}
\item \textbf{Providing information vs. speech-based practical action.} While requests for information (e.g. \textit{What time is it?}) are {excluded} from the collection, requests for practical actions involving speech (e.g. \textit{Go tell him to come}) are {included} when they require similar kinds of physical work as other practical actions (e.g. \textit{Go get me that knife}).

\item \textbf{Perception directives} (e.g. \textit{Look! Listen!}). Cases in which the perception directive is used as a discourse marker (e.g. \textit{Look I don’t really know what to do}, \textit{Listen I have told you this many times}, \citealt{sidnell_look-prefaced_2007}) or primarily to remark on and share a perceptual experience (e.g. \textit{Look what a beautiful sunset}) should be excluded. On the other hand, cases in which the function is to draw B’s attention to something that is relevant for a practical purpose (e.g. \textit{Look!} when the boiling content of a pot is spilling over) should be included, as the (re)direction of B’s perception here is in the service of getting them to act (e.g. remove the pot from the stove).

\item \label{par:coding:offers} \textbf{Offers of assistance in response to “trouble”} (\citealt{Curl2006}). These should be included as recruitment sequences (see question B04 below) only when the assistance provided in Move B is instigated by a display of current or anticipatable trouble in Move A. Put another way, the assistance provided should demonstrably be an action \textit{required} to remedy a trouble that A is having rather than an \textit{optional} action volunteered out of B’s nicety. This distinction necessitates an understanding of the social context and knowledge of the practical circumstances of each particular case. Here are two example scenarios to aid in the judgment.
\renewcommand{\theenumii}{\roman{enumii}}
\begin{enumerate}
\item After the table is set, A looks around in search of her plate, which is missing; B notices this and walks to the kitchen to get it.
\item As A begins to eat her hamburger, B notices that the hamburger does not contain mayonnaise and passes the mayonnaise jar to A.
\end{enumerate}

The first case should be included insofar as having a plate is necessary for A to be able to eat, as displayed by A’s looking for it. The second case should be excluded insofar as the additional ingredient is not necessary for A to be able to have her hamburger, as shown by the fact that she is already eating it.

\item \textbf{Dependent vs. independent recruitment.} We identify as “dependent” or “responsive” recruitment a case in which a practical action is recruited as a direct result of the fulfillment of an earlier recruitment. For example:

\begin{tabular}{l l l l }
A: & \textit{Give me a knife} & & \\
B: & ((gets the knife)) & & \\
B: & \textit{Here you are} & << & \textsc{not} \textsc{a} \textsc{separate sequence} \\
A: & ((reaches out to grab knife)) & << \\
\end{tabular}

These should {not} be considered as separate recruitment sequences. At the same time, dependent or responsive recruitments should be distinguished from cases in which a practical action is independently recruited to deal with a contingency arising along the way toward the fulfillment of an earlier recruitment. For example:

\begin{tabular}{l l c c c}
A: & \textit{Hold this for me} ((brings pot toward B)) & & Move A\textsubscript{i} & (Case 1) \\
B: & \textit{Wait I need to put this down first} & << & Move A\textsubscript{ii} & (Case 2) \\
A: & ((stops)) & << & Move B\textsubscript{ii} &  (Case 2) \\
B: & \textit{Okay} ((reaches out for pot)) & & Move B\textsubscript{i} & (Case 1) \\
\end{tabular}

These should be analyzed as two overlapping or nested recruitment sequences.

\item \textbf{Repair initiation and solution.} When B responds to A’s first attempt at recruitment (e.g. \textit{I need a knife}) with repair initiation (e.g. \textit{A knife?}) and A then provides a repair solution (e.g. \textit{Yes}), this repair solution should be considered as Move A of a subsequent, separate recruitment case with its own unique identifier, even though the two cases belong to the same sequence (the sequential position of cases is coded in C01 below).

\begin{tabular}{l c l}
Case 1 &  Move A & \textit{I need a knife} \\
                &  Move B & \textit{A knife?} \\
Case 2 &  Move A & \textit{Yes} \\
                &  Move B & ((brings knife)) \\
\end{tabular}

\item \textbf{Stand-alone vocatives.} Vocatives and address terms (e.g. \textit{Hey}, \textit{Mary}, \textit{Mr. Smith}) are often used to secure the recipient’s attention in preparation for a further, as-yet unknown sequence of action \citep[chap. 4]{Schegloff2007}. While vocatives generally do not constitute a recruiting move alone (i.e. without additional elements), there are a few exceptions:
\renewcommand{\theenumii}{\roman{enumii}}
\begin{enumerate}
  \item   The vocative functions as a recruitment pursuit after one or more earlier attempts at recruitment have been made, e.g.

\begin{tabular}{l l }
Mom:  &  \textit{Stop pestering your sister John!} \\
John: &  ((no uptake)) \\
Mom: &   \textit{Stop!} \\
John: & ((no uptake)) \\
Mom: & \textit{John!} \\
\end{tabular}

  \item   The vocative accompanies a meaningful nonverbal component, e.g. \textit{John} ((while holding out a cup)).
  \item   A heavily specified context allows B to know what task they are supposed to carry out just by hearing the vocative.
\end{enumerate}
\end{enumerate}

\section{Guidelines for sampling interactions and collecting cases}\label{sec:coding:4}

Coding must be based on a systematic and coherent collection of recruitment sequences from a broad sample of face-to-face interactions. The sample should include a range of different activities, settings, and speakers; it should contain both dyadic and multi-person interactions; and it should span both interactions that are task-focused (e.g. playing a game, preparing food, doing work together) and others that are talk-focused (e.g. just gossiping or chatting). Below are some guidelines for systematically sampling interactions and collecting cases.

\begin{itemize}
\item Begin collecting either at an arbitrary point in the recording or from the beginning.
\item Wherever you begin, collect all the cases you find in a continuous stretch of interaction.
\item Collect cases liberally -- that is, if you are unsure whether something is an example, you should note it down anyway. It is easy to remove cases later on, but it is a lot of work to go back and look for cases that you failed to include.
\item Recruitment sequences may be infrequent in certain kinds of interactions, especially those that are talk-focused. For this reason, it is more effective to search through entire recordings rather than to take short segments of equal size. On the other hand, recruitment sequences can be extremely abundant in task-focused interactions. To avoid overrepresentation of these, the number of cases from a single interaction should be capped, for example at 15.
\end{itemize}

The goal of these guidelines is to construct a sample that is representative of the diversity of the corpus at hand. How well this diversity represents social interaction in the target language will depend on how the corpus as a whole has been built.


\section{General coding guidelines}\label{sec:coding:5}

The coding sheet should contain a transcript of the core interactional moves of each recruitment sequence, including a basic transcription of verbal elements and a concise description of nonverbal elements (see \sectref{sec:coding:6} A. Basic data, below). This transcript is intended to make the coding data intelligible to other analysts and as a reference for coding; however, it is often not enough to be able to accurately code certain features of the sequence, for example, the strengthening of the recruiting move by means of prosody (question C11) or the visibility of a target object (question E05). For this reason, coding should be based whenever possible on direct inspection of audio and video streams, possibly supported by a more detailed transcript of the larger interaction.

When in doubt, coders should choose the most conservative coding choice, or a choice that best reflects the potential equivocalness of the feature in question, or the “can’t tell” option, if available. For any coding decision, coders should be able to provide a reasoned argument and evidence to support it. Comments about particular coding choices should be entered in a notes field at the end of the sheet so as to be available to other analysts examining the coding and as a bridge between the complexity of human behavior and its reduced representation in coding data \citep{Stivers2015}.

In this project, we decided to compile a glossary of certain verbal practices that make up a language’s repertoire of resources for initiating and responding to recruitment, including modal constructions (C07), mitigators and strengtheners (C11), and benefactive markers (C12). The coding process provides an opportunity for easily compiling such a glossary by creating an entry in a dedicated tab of the coding sheet every time a recurrent practice is identified in connection with a coding question. Questions suitable for glossary entries are marked with a superscript “gl” (e.g. C07\textsuperscript{gl}).

\section{The coding scheme}\label{sec:coding:6}

\noindent \textbf{A.} \textbf{Basic data}

\medskip

\noindent This section records the basic data for every recruitment sequence, including Moves A and B. Each case has a unique identifier, which is used to locate it in a recording, refer to it in qualitative analysis, and for statistical purposes in quantitative analysis. Verbal elements are transcribed and translated and nonverbal elements are concisely described to make cases intelligible to other analysts.

\begin{description}
\item A01   \textbf{Language}
\item A02  \textbf{Unique identifier for the case.} Suggested format: recording\_timecode (e.g. Housemates\_3211246). In the rare event that two cases begin simultaneously, use an additional symbol to distinguish them\\(e.g. Housemates\_3211246a, Housemates\_3211246b).
\item
{A03   \textbf{Is a child }\textbf{involved as either recruiter or recruitee?}
\begin{enumerate}
\item {yes}
\item {no}
\end{enumerate}
{Make this choice according to your understanding of the duration of childhood in this particular culture.}\footnotemark{}}
\footnotetext{ For different projects, different sociodemographic categories (based on age, ethnicity, socioeconomic status, etc.) may be flagged to allow for sorting or comparison (see, e.g., \citealt{StiversMajid2007}). In this project, we decided to treat recruitments involving children separately to increase comparability among cultures and corpora.}
\item
{A04  \textbf{Move A verbal component.} Transcription in original language or, alternatively, a code to indicate that there is no (relevant) verbal component [none] or that the verbal component is inaudible [can’t tell].}
\item
A05  \textbf{Move A translation.} (If applicable).
\item
A06  \textbf{Move A nonverbal component.} Concise description or, alternatively, a code to indicate that the participant’s visible conduct is not related to the construction of the move [not relevant] or that it cannot be inspected because the participant is momentarily off camera or hidden [can’t tell].
\item
A07  \textbf{Move B verbal component.} Transcription in original language or, alternatively, a code to indicate that there is no (relevant) verbal component [none] or that the verbal component is inaudible [can’t tell].
\item
A08  \textbf{Move B translation.} (If applicable).
\item
A09  \textbf{Move B nonverbal component.} Concise description or, alternatively, a code to indicate that the participant’s visible conduct is not relevant or related to the construction of the move [not relevant] or that it cannot be inspected because the participant is momentarily off camera or hidden [can’t tell].

Descriptions of nonverbal behavior should be concise and pitched at an appropriate level of granularity. For example, ((gestures at the salt)) is too general and ((raises arm and extends index finger toward the salt)) is too elaborate. Different projects will have different requirements for the description of nonverbal behavior and cases differ in complexity, but for the example above, ((points at salt)) is the right level of granularity for most purposes. Moreover, descriptions of nonverbal behavior should stick to what is visible at that particular moment. For example, ((stands and walks toward spices shelf)) is more objective than ((goes to get salt)) because at that moment it is still not certain how B’s compliance with the recruitment will develop.

  In this project, we decided not to record gaze behavior in the transcription of Move A’s nonverbal component for reasons of economy, as there is a dedicated coding question about gaze (C16). However, we did record gaze in the transcript when it was used as a pointing gesture or “eye point” \citep{Wilkins2003} toward an object or location relevant to the recruitment, which made it easier to code this as a pointing gesture in question C02.
\end{description}

\noindent \textbf{B.} \textbf{Recruitment categories} 

\medskip

\noindent This section contains questions designed to identify the main interactional “problem” or “point” of a recruitment sequence with reference to four categories, which were inductively derived from qualitative observation and analysis of recruitment episodes in our languages. This categorization serves multiple purposes, including grouping cases that share similar interactional contingencies, such as the visibility and accessibility of a target object (see questions E05, E06), and grouping cases in which Move A has a similar function, to test whether this has an effect on its form across languages.

In keeping with our recruitments approach, cases are assigned to categories primarily on the basis of B’s response (see \chapref{sec:intro}, \sectref{sec:intro:4}). However, A’s instigating behavior, along with any subsequent pursuits, should also be considered, especially in cases where B ignores it; here, the answer will reflect an understanding of what A’s behavior is “going for” or working toward.

While it is possible to assign a case to more than one category, it is usually more useful to pick the most fitting or salient one. Also, while cases should be assigned to a category whenever possible, there will be cases that do not fit any of the categories but still fall within the broader definition of recruitment, for example, calling or motioning for a collective clinking of glasses, or initiating a joint recreational activity like playing chess.

Finally, a note on multiple cases belonging to the same recruitment sequence: when B responds to A’s first attempt at recruitment (e.g. \textit{I need a knife}) with repair initiation (e.g. \textit{A knife?}) and A’s subsequent action is constituted by a repair solution (e.g. \textit{Yes}) followed by B’s fulfillment (e.g. brings knife), questions B01-B04 should be answered in the same way across the two cases (the sequential position of cases is coded in C01).
Example:\\

\begin{tabular}{l c l l}
{Case 1} &  Move A & \textit{I need a knife} & \\
                &  Move B & \textit{A knife?} &  B01 = yes\\
{Case 2} &  Move A & \textit{Yes} & \\
                &  Move B & ((brings knife)) & B01 = yes\\
\end{tabular}

\begin{description}
\item
B01  \textbf{Does B give an object to A in response to some behavior by A? Or, alternatively, was this the point of a sequence that went unfulfilled?}

\begin{enumerate}
\item{yes}
\item {no}
\end{enumerate}

“Giving an object” specifically refers to the physical transfer of a moveable object from the control of one person to another, usually released and grasped by the hands. This does not include cases in which B moves out of the way or otherwise facilitates A’s taking possession of an object.

\item
B02  \textbf{Does B do a service for A in response to some behavior by A? Or, alternatively, was this the point of a sequence that went unfulfilled?}

\begin{enumerate}
\item{yes}
\item {no}
\end{enumerate}

A “service” is intended as a practical action involving some manipulation of the material environment (e.g. washing the dishes, feeding the chickens). While giving an object to someone can also be seen as a “service”, it is a particular kind of service that is worth considering separately (see, e.g., questions E05, E06), so if you have answered “yes” to question B01, you should normally answer “no” to B02.

\item
B03  \textbf{Does B alter the trajectory of their in-progress behavior in response to some behavior by A? Or, alternatively, was this the point of a sequence that went unfulfilled?}

\begin{enumerate}
\item{yes}
\item {no}
\end{enumerate}

“Altering” the trajectory of an ongoing behavior includes both adjusting or changing the behavior (doing the same thing in a different way) and ceasing the behavior altogether. These two kinds of alteration have different implications for how the recruitment sequence is fulfilled: doing something differently vs. not doing something anymore. Do not answer “yes” when the cessation of a behavior is incidental to performing a service or object transfer. Consider the following: if A says \textit{Stop playing with your food and eat}, the cessation of eating is an integral part of the recruitment; however, if A says \textit{Is there still some beer left?} and B stops watching the TV in order to go to the kitchen to get beer for A, the cessation of B’s ongoing behavior is incidental to the recruitment.

\item
B04  \textbf{Does B do a practical action to address some current or anticipatable trouble for A?}

\begin{enumerate}
\item{yes}
\item {no}
\end{enumerate}

This question is aimed at capturing cases in which B provides assistance without this being solicited or expected by A, but rather instigated by A’s display of current or anticipatable trouble (e.g. A arrives at a door with her hands full of heavy objects and B opens the door, or A grasps for the salt but is unable reach it so B pushes the salt closer).
\end{description}

\noindent \textbf{C.} \textbf{Move A: The recruiting move}

\begin{description}
\item
C01  \textbf{In the in-progress sequence, what is the position of Move A?}

\begin{enumerate}
\item {one and only}
\item {first of non-minimal}
\item {last of non-minimal}
\item {\textit{n}th}
\end{enumerate}

{Here we consider the sequential position of Move A, coding whether it is: the first and only attempt in a minimal sequence that is immediately completed (“one and only”), a first attempt in a longer sequence that is not completed in one go (“first of non-minimal”), a final attempt in a longer sequence (“last of non-minimal”), or a subsequent attempt that was neither the first nor the last (``\textit{n}th'').}
{  When considering the sequential position of Move A, it is important to remember that certain preliminary moves, also referred to as “pre-requests” (see \citealt[chap. 6]{Levinson1983}; \citealt{Rossi2015b}), can function as recruiting moves on their own and lead to immediate completion.}

\begin{tabular}{l l l }
Move A & \textit{Are you using that pen?} &  C01 = one and only\\
Move B & ((passes A the pen)) &  \\
\end{tabular}

{However, preliminary moves may also be responded to with a go-ahead leading to a subsequent, more explicit Move A (or to multiple subsequent attempts).}

\begin{tabular}{l c l l}
{Case 1} &  Move A & \textit{Are you using that pen?} &  C01 = first of non-minimal \\
                &  Move B & \textit{No} & \\
{Case 2} &  Move A &  \textit{Can I use it for a sec?} &  C01 = last of non-minimal \\
                &  Move B & ((passes A the pen)) & \\
\end{tabular}

Other preliminary moves that cannot mobilize the relevant practical action in next position should not be considered as recruiting moves, though they are part of the recruitment episode. These typically involve generic pre-sequences \citep[chap. 4]{Schegloff2007}:

\begin{tabular}{l l l }
Summons  & \textit{Hey Bob!} &  \textsc{not a recruiting move}\\
Answer & \textit{What?} &  \\
Move A &  \textit{Come here} & C01 = one and only \\
Move B & ((goes to A)) & \\
\end{tabular}

\smallskip

\item
\textit{Only answer C02 if A06 ${\neq}$ [not relevant] or [can’t tell]}

\item
C02   \textbf{Concerning the nonverbal behavior, what does it consist of?}

\begin{enumerate}
\item current or anticipatable trouble
\item pointing gesture
\item reach to receive object from B
\item holding out object for B to do something with
\item iconic gesture
\item other
\item can’t tell
\end{enumerate}
These types of nonverbal behavior were derived from qualitative observation and analysis across languages, and were identified as recurrent relevant behaviors that either accompany verbal elements of the recruiting move or initiate recruitment on their own.

If multiple types of nonverbal behavior co-occur in the same recruiting move, answer this question based on the most salient behavior.

Pointing gestures include not only manual points, but also head points, lip points, and “eye points” \citep{Wilkins2003}. As explained above in Section A, gaze should be considered here only when it is used to indicate an object or location relevant to the recruitment. Gaze used for recipient selection \citep{Lerner2003} is dealt with by question C16 and -- in this project -- is not be transcribed in A06.

\smallskip

\item
\textit{Only answer C03--C12 if A04 ${\neq}$ [none] or [can’t tell]}

\newpage
\item
C03  \textbf{Does the verbal behavior consist of a simple or complex construction?}

\begin{enumerate}
\item simple
\item complex
\end{enumerate}

There are two main types of cases where verbal behavior should be coded as “complex”:

\begin{enumerate}
\item self-repair (e.g. \textit{Pass me t- uh will you pass me the salt please?});
\item complex constructions packaged as a single unit including two or more predicates or target actions (e.g. \textit{Stop playing with that bucket and get me some water}).
\end{enumerate}

\item
C04  \textbf{Does the verbal behavior include a directly-involved nominal referent?}

\begin{enumerate}
\item yes, full noun phrase
\item yes, pronominal
\item no
\item can’t tell
\end{enumerate}

By “directly-involved nominal referent” we intend a referent that is the target object (e.g. \textit{\textbf{Water} please}) or that is otherwise implicated in the recruited action (e.g. \textit{Is the \textbf{window} open?} when the goal is to have B close the window). Such a referent may be encoded either with a full noun phrase (e.g. \textit{Pass me \textbf{the salt}}) or with a pronominal element (e.g. \textit{Pass me \textbf{that}}). Directly-involved nominal referents are easy to identify with most transitive predicates (e.g. \textit{Clean \textbf{the table}}, \textit{Light \textbf{my cigarette}}). With ditransitive (three-place) predicates, relevant referents will also include the “recipient” of the action (e.g. \textit{Give it to \textbf{Dad}}, \textit{Pass \textbf{him}} the lighter). Verbs such as \textit{get} and \textit{take} can also be treated as belonging to this group in that their semantics involves a location where an object is taken or gotten \textit{from} (e.g. \textit{Get it from \textbf{the trolley}}) \citep{Fillmore1977}. Semantics aside, nominals can be directly involved in recruited actions in different ways and no single rule can capture all eventualities. But here are some examples that were collectively discussed during the project with an explanation of the rationale for the coding decision.

\begin{itemize}
\item \textit{Sit on the \textbf{chair}}. Answer “yes” because A is telling B to sit specifically on the chair and not just anywhere (e.g. on the couch or floor); this referent is integral to the recruited action.
\item \textit{The stock cube is in the \textbf{cupboard}} (where the goal is to have B move the camera away from the cupboard). Answer “yes” because the problem is the specific location of the camera in front of the cupboard, and the recruited action involves moving the camera away from it.
\item \textit{Aren’t those \textbf{fish} going to die?} (where the goal is to put more water in the pot where the fish are). Answer “no” because although the fish benefit from the addition of water, the target action does not involve them, only the pot and the water.
\end{itemize}

“Full noun phrases” typically involve open-class items referring to people, things, locations, whereas “pronominal” elements are reduced, closed-class pro-forms such as demonstratives and other deictics. For languages with the possibility of zero anaphora, if there is no overt pronominal form, stick to the surface and answer “no”.

In the case of a complex verbal component (see question C03), consider it holistically; for example, a complex construction with multiple referents like \textit{Get a \textbf{pot} from that \textbf{cabinet} and put \textbf{it} on the \textbf{stove}} should be coded as “yes, full noun phrase”.

\item
C05  \textbf{What is the sentence type?}
\begin{enumerate}
\item
imperative
\item
interrogative
\item
declarative
\item
other
\item
there is no predicate
\item
can’t tell
\end{enumerate}

In most languages, it is possible to distinguish different formal types of sentences that prototypically encode asserting or informing (declaratives), asking or questioning (interrogatives), and directing or ordering (imperatives) \citep{Lyons1977,SadockZwicky1985,KönigSiemund2007}. These are formal, logico-semantic types that encode basic ways of dealing with propositional content. The criteria for assigning utterances to these three types may vary according to the internal organization of each language, but as a rule of thumb you can ask: how would this utterance be understood out of context? Some languages may have “other” major or minor sentence types (e.g. exclamatives, “insubordinated” \textit{if} constructions, etc.). Since sentence type is based on there being a predicate, when Move A does not include a predicate (e.g. \textit{Water please}), choose “there is no predicate”.

\smallskip

\item
\textit{Only answer C06--C07 if C05 ${\neq}$ “there is no predicate”}

\item
C06   \textbf{Is there a predicate that refers to the target action?}

\begin{enumerate}
\item yes
\item no
\item can’t tell
\end{enumerate}

This questions asks whether the action that is the projected outcome of the recruitment is explicitly referred to by a predicate in Move A. Examples:

\begin{tabular}{l l}
\textit{Give me the knife} & C06 = yes \\
\textit{Don’t do that!}    & C06 = yes \\
\textit{Do you have a lighter?} &  C06 = no \\
\end{tabular}

\smallskip

\item
\textit{Only answer C07 if C05 = “interrogative”, “declarative”, or “other”}

\item
{C07\textsuperscript{gl} \textbf{Does the predicate encode obligation, permission, ability, or volition to perform an act?}}

\begin{enumerate}
\item
{yes, obligation/necessity}
\item
{yes, permission/authorization}
\item
{yes, ability/possibility}
\item
{yes, volition/willingness}
\item
{yes, a combination of the above}
\item
{no}
\item
{can’t tell}
\end{enumerate}

{This question is about specific modal categories: obligation/necessity (e.g. \textit{You must finish your dinner}, \textit{The door is to be shut}), permission/authorization (e.g. \textit{May I have that last piece of cake?}), ability/possibility (e.g. \textit{Can you pass me the salt?}, \textit{You could start washing up}), volition/willingness (e.g. \textit{I would like some water}, \textit{Will you hand that to me please?}). These meanings must be semantically encoded. For example, a sentence like \textit{You’re standing on my foot}, although it may pragmatically oblige the recipient to step away, does not encode obligation. In English and other European languages, obligation, permission, ability, and volition are frequently encoded with modal verbs such as \textit{must}, \textit{have to}, \textit{need}, \textit{may}, \textit{can}, \textit{will}. Other languages may use affixes or dedicated constructions. Cha'palaa, for example, encodes necessity with an infinitive verb followed by a finite ‘be’ verb.}

\smallskip

\item
{\textit{Only answer C08 if C05 ${\neq}$ “imperative”}}

\item
{C08  \textbf{Is the main subject overtly marked for person?}}
\begin{enumerate}
\item {yes, first person}
\item {yes, second person}
\item {yes, third person}
\item {yes, other}
\item {no overt marking}
\end{enumerate}

“Overt grammatical person” refers to morphosyntactic and lexical categories in the language that encode the person of the subject-like argument of the verbal component, whether as a noun or noun phrase (e.g. \textit{\textbf{Grandma} needs a blanket}), pronoun (e.g. \textit{Can \textbf{you} pass the salt?}), clitic, verbal inflection (etc.), or a combination of these. This must be an overt, surface form. “Third person” refers to grammatical third person, regardless of whether the referent is a potential participant in the speech event (e.g. \textit{\textbf{Somebody} should close the door}) or not (e.g. \textit{\textbf{The door} should be closed}, \textit{\textbf{It}’s cold in here}).

Vocatives (e.g. \textit{John, water please!}) do not constitute a form of person marking, but stand-alone pronouns do (e.g. \textit{You, water!}). If you find overt grammatical subjects in your sample that bridge two or more of the categories listed above, code for the most specific person value that can be obtained from the construction, or choose “other” if you feel that none of the above choices apply.

\item
C09  \textbf{Does the move include additional elements beyond core grammatical constituents?}

\begin{enumerate}
\item{yes}
\item {no}
\item {can’t tell}
\end{enumerate}

“Core constituents” refers to a predicate with its core arguments, which will normally be up to one argument with intransitive verbs (e.g. \textit{You stop!}), up to two arguments with transitive verbs (e.g. \textit{Can you close the door?}), and, in some languages, up to three arguments with ditransitive verbs (e.g. \textit{You should give her a spoon}). Sometimes you will have to decide whether an element is a true ditransitive object (\textit{Can you pass \textbf{him} the salad?}) or is marking a non-core beneficiary (as in the Italian: \textit{Tieni\textbf{mi} questo}, ‘Hold \textbf{me} this’). In the former case, the answer to this question is “no”; in the latter, it is “yes”.

  Additional elements beyond core grammatical constituents typically belong to one of the following four categories:

\begin{itemize}
\item benefactives (e.g. \textit{Could you move that a little bit \textbf{for me} please?})
\item clausal explanations (e.g. \textit{Keep stirring the sauce \textbf{so it doesn’t get lumpy}}).
\item vocatives (e.g. \textit{Come here \textbf{John}})
\item mitigators and strengtheners (e.g. \textit{I need you to stop \textbf{immediately}})
\end{itemize}

\smallskip

\item
\textit{Only answer C10--C11 if C09 = “yes”}

\item
C10  \textbf{Does the move include a clausal explanation?}
\begin{enumerate}
\item{yes}
\item {no}
\item {can’t tell}
\end{enumerate}

A “clausal explanation” makes reference to a past, present, or future state of affairs or event that provides grounds for the recruitment or makes it more intelligible to B. This covers any kind of reason-giving, including accounts for untoward or imposing behavior (e.g. \textit{Stop talking so loudly, \textbf{I have a headache}}) as well as more general explanations that make the recruitment more understandable or clear (e.g. \textit{Keep stirring the sauce \textbf{so it doesn’t get lumpy}}).

  The clausal explanation must be built into Move A (single package). If the explanation is provided after a self-contained Move A has been produced (two packages) then there are two main possibilities.

\renewcommand{\theenumi}{\roman{enumi}}
\begin{enumerate}
\item Participant A gives a reason after no uptake comes from B or when it is not clear that B will comply. In this case the explanation effectively counts as a second attempt at recruitment and must be entered as a separate case. Example:

\begin{tabular}{l c l l}
{Case 1} &  Move A & \textit{Bring me a knife}  &   C10 = no \\
                & & (1.0) & \\
{Case 2} &  Move A &  \textit{{I need to cut these apples}} &  {C10 = no} \\
                &  Move B & \textit{Okay} & \\
\end{tabular}

\item
Participant A gives a reason for the recruitment after it has been already fulfilled by B, or after B has clearly shown that they are on their way to comply. We can define this as a “post-hoc” explanation, justifying the launch of the recruitment sequence after it has been complied with. Such explanations are not part of Move A. Example:

\begin{tabular}{l l l}
Move A & \textit{Can you give me some water?}  &    \\
Move B & \textit{Here} ((gives water to A)) & \\
 &  (0.5) & \\
 Post-hoc explanation &  \textit{{It’s hot today, I’m so thirsty}} &  C10 = no\\
\end{tabular}
\end{enumerate}

\item
C11\textsuperscript{gl}  \textbf{Does the move include a mitigating or strengthening element?}

\renewcommand{\theenumi}{\arabic{enumi}}
\begin{enumerate}
\item {yes, mitigating}
\item {yes, strengthening}
\item {no}
\item {can’t tell}
\end{enumerate}

This question asks about elements that mitigate or soften the recruiting move (e.g. \textit{Move the car \textbf{if it’s not too much trouble}}, \textit{Can I have a \textbf{little} water?}) or, alternatively, that strengthen or aggravate it (e.g. \textit{I would \textbf{really} like some water}, \textit{Get the key \textbf{right now}}). These elements may be clauses, phrases, adverbs, particles, affixes, or other forms. Do not consider clausal explanations (C10) when answering this question.

\item
C12\textsuperscript{gl}  \textbf{Is there formal benefactive marking?}

\begin{enumerate}
\item {yes, marking A}
\item {yes, marking other}
\item {no}
\item {can't tell}
\end{enumerate}

This question asks if the verbal component includes an explicit beneficiary of the recruited action, which may be A (e.g. \textit{Could you move that a little bit \textbf{for me} please?}, \textit{Read \textbf{me} a book!}) or, alternatively, another participant or combination of participants: B (e.g. \textit{Will you pass the cards so I can cut them \textbf{for you}}?, \textit{Grab \textbf{yourself} a beer!}), both A and B (e.g. \textit{Can you set the table \textbf{for everyone}}?), or a third party C (e.g. \textit{Get \textbf{him} a fork!}). Formal benefactive marking includes datives (e.g. \textit{Read \textbf{me} a book!}), prepositional phrases (e.g. \textit{Could you move that a little bit \textbf{for me} please?}) and other resources such as specific constructions (e.g. \textit{\textbf{Do me a favor} and}...). Constructions with verbs of need (e.g. \textit{I need a lighter}) do not qualify as including formal benefactive marking.

\item
C16  \textbf{Does A gaze at B during Move A?}

\begin{enumerate}
\item {yes}
\item {no}
\item {can't tell}
\end{enumerate}

The purpose of this question is to code gaze as a design feature of the recruiting move. What is relevant is whether A is looking at, or trying to establish eye contact with, B. Answer “yes” on the basis of A’s behavior regardless of whether B perceives being gazed at or not. If you have reasons to believe that B does not perceive being gazed at by A, it is recommended to flag this in the general notes field.

\end{description}

\noindent \textbf{D.} \textbf{Move B and Move C: Responding and acknowledging} 

\medskip

\noindent The questions in this section concern the responding move or Move B, and the potential expansion of the sequence with an acknowledgment or Move C.

\begin{description}
\item
D01  \textbf{What is the response doing relative to the recruitment?}

\begin{enumerate}
\item {quickly fulfills or provides assistance}
\item {plausibly starts fulfilling or providing assistance}
\item {rejects}
\item {initiates repair}
\item {other}
\item {ignores}
\item {can’t tell}
\end{enumerate}

\textbf{\textit{“Quickly fulfills” versus “Plausibly starts fulfilling”}}. It is useful to distinguish between two ways of positively responding to recruitment: doing the target action within a short time frame immediately after Move A and doing something that could plausibly be construed as the beginning of fulfillment (but still possibly equivocal), over a longer time span. To make this decision, put yourself in the position of participant A and ask what he or she would be aware of in the first few seconds after Move A. Note that some recruited activities inherently take more time than others (e.g. setting the table for lunch, getting an object that is far away) and should always be coded as “plausibly starts fulfilling”. Cases in which B commits to later fulfillment (e.g. \textit{Oh sorry I’m busy right now but I’ll do that in half an hour}) should also be coded as “plausibly starts fulfilling”.

\textbf{\textit{“Rejects”}}. All clearly negative responses such as refusing (e.g. \textit{No I won’t do that}) and/or giving an account for non-compliance (e.g. \textit{I’m too busy now}) should be coded as “rejects”. Different types of rejections are distinguished by subsequent questions (D02 and D03).

\textbf{\textit{“Initiates repair”, “ignores”, and “other”}}. Besides responding positively or negatively, B may respond to the recruiting move in other ways. One possibility is to initiate repair. Another is to ignore the recruiting move by not taking it up at all. This applies both to cases in which B would be in a position to hear/see the recruiting move but intentionally ignores it and to cases in which B might not have heard/seen the recruiting move (for example, because they are involved in a parallel activity, or too far away, etc.). Other cases in which the recruiting move is taken up but the response does not fit any of the above categories should be coded as “other”. Examples of “other” responses are:

\begin{itemize}
\item delegating to a third party (e.g. A asks B to pass the salt; in response, B turns to C and tells them to pass A the salt);
\item responding with information that A can use to do the action him/herself (A: \textit{I need a fork} B: \textit{In the drawer in the kitchen});
\item making a counter-proposal (A: \textit{Can you add oil and salt in the salad bowl?} B: \textit{Why don’t we leave the salad undressed instead?}).
\end{itemize}

\item
D02  \textbf{Does the response include a positive or negative polar element?}

\begin{enumerate}
\item {yes, positive}
\item {yes, negative}
\item {no}
\item {can’t tell}
\end{enumerate}

Positive polar elements include verbal/vocal elements such as \textit{yes}, \textit{okay}, \textit{sure}, \textit{mm hm} as well as nonverbal elements like a head nod. Negative polar elements include verbal/vocal elements such as \textit{no}, \textit{mh mh} as well as nonverbal elements like a head shake. Coding should take into account that linguistic systems differ. For example, in some languages like Mandarin and Cha’palaa, one way to do a polar response is by repeating the verb.

\smallskip

\item
\textit{Only answer D03 if A07 ${\neq}$ [none] or [can’t tell]}

\item
D03 \textbf{Does the response include a clausal explanation/account?}

\begin{enumerate}
\item {yes}
\item {no}
\item {can’t tell}
\end{enumerate}

See notes for question C10 on explanations and accounts.

\item
D04\textsuperscript{gl}  \textbf{Is there acknowledgment by A?}

\begin{enumerate}
\item {yes}
\item {no}
\end{enumerate}

“Acknowledgment” includes thanking (e.g. \textit{Gracias!}), other expressions of gratitude (e.g. \textit{Cheers, I appreciate it, Oh I’m so glad you can do this for me}), and more generally any positive conveyance of appreciation or satisfaction by the recruiter immediately after receiving a response indicating fulfillment. In some cases, fulfillment may be still ongoing or forthcoming at the time of the acknowledgment.

\smallskip

\item
\textit{Only answer D05--D06 if D04 = “yes”}

\item
D05  \textbf{Transcribe and translate the acknowledgment} (\textit{Muchas gracias} ‘Thanks a lot’). If the acknowledgment includes nonverbal behavior, briefly describe it, e.g. ((nods repeatedly)).

\item
D06\textsuperscript{gl}  \textbf{Is there a subsequent move by B responding to the acknowledgment?}

\largerpage
\begin{enumerate}
\item{yes} (e.g. \textit{You’re welcome}, \textit{Don't mention it})
\item  {no}
\end{enumerate}

Such a response may be conventionalized (e.g. \textit{You’re welcome}) or not (e.g. \textit{Oh well I owed you this one}).

\smallskip

\item
\textit{Only answer D07 if D06 = “yes”}

\item
D07  \textbf{Transcribe and translate the subsequent move by B responding to the acknowledgment.}
\end{description}

\noindent \textbf{E.} \textbf{Other elements of the recruitment sequence}

\medskip

\noindent The questions in this section code for other elements of the recruitment sequence beyond Moves A, B, C.
\begin{description}
\item E01  \textbf{Is there an evident local immediate beneficiary for the recruitment?}

\begin{enumerate}
\item {yes, A}
\item {yes, other}
\item {no}
\end{enumerate}

The answer to this question is in principle independent of, and possibly incongruous with, the answer given to question C12 (which deals with \textit{formal marking} of beneficiaries). E01 can be a tricky question, but try not to overthink the issue and choose the most straightforward answer. If in doubt, be conservative and answer “no”.

\item
E02  \textbf{Is the interaction dyadic?}

\begin{enumerate}
\item{yes}
\item {no}
\item {can't tell}
\end{enumerate}

For the answer to be “yes”,  there should normally be only two people in the video recording at the time at which the recruitment occurs; if there are three or more people, answer “no”. In some cases, a stretch of interaction may be considered dyadic even though other people are present in the immediate vicinity but are clearly not part of the interaction.

\item E03  \textbf{Is a vocative used?}

\begin{enumerate}
\item{yes}
\item {no}
\item {can't tell}
\end{enumerate}

Vocatives normally involve a proper name, kin term, title, or similar, and provide a way of explicitly addressing the recruiting move to a specific recipient or set of recipients. The vocative may be built into Move A (e.g. \textit{Can you pass me the knife \textbf{John}}?, \textit{\textbf{You two guys}}, \textit{come with me}) or be part of a summons-answer sequence that precedes the recruiting move (see also C01):

\begin{tabular}{l l l}
Summons & \textit{Hey \textbf{Bob}}! & E03 = yes   \\
Answer & \textit{What?}  & \\
Move A  & \textit{Come here} & \\
Move B &  ((goes to A)) & \\
\end{tabular}

\item E04  \textbf{Can A and B’s relationship be characterized as socially asymmetrical?}

\begin{enumerate}
\item {yes, A} > {B}
\item {yes, A < B}
\item {no, A = B}
\item {can’t tell}
\end{enumerate}

In this question we code for any salient social asymmetry between A and B, based on the researcher’s knowledge of the society. The question refers to enduring asymmetries between A and B that hold across contexts. Social asymmetries can be based on age (e.g. older-younger siblings) as well as other kinds of social status (e.g. authority roles such as husband-wife, parent-child). The answer should be based on prescriptive norms and general cultural expectations of the community, not on the instantiation of the relationship in the recruitment sequence, so you should not use the recruitment sequence as a basis for your judgment: evidence for the asymmetry must be independent of it. Social asymmetry is gradient, so judge whether a dyad is \textit{relatively} symmetrical or asymmetrical.

\item
\textit{Only answer E05--E06 if B01 = “yes”}

\item
E05  \textbf{Is the target object visible to A?}

\begin{enumerate}
\item{yes}
\item {no}
\item {can’t tell}
\end{enumerate}

\item
E06  \textbf{Does B have better access to the object in question than A?}

\begin{enumerate}
\item {yes, B is currently using the object}
\item {yes, B is in possession of the object but is not using it}
\item {yes, B is closer to the object than A}
\item {no}
\item {can’t tell}
\end{enumerate}

Typically, B is “using an object” when he or she is currently manipulating it. Cases in which B has been making use of the object all along and has only momentarily rested it somewhere when the recruitment is attempted should be coded as “yes, B is in possession of the object but is not using it”. Possession does not require that B be the legal or socially recognized owner of the object; it applies to all cases where B has the object “on them” (e.g. in their pocket) as well as to cases where the object is enclosed into another possession of B’s (e.g. their bag). For cases where relative closeness is relevant, try not to overthink the issue and answer “yes” only when there is a clear difference in distance (e.g. the object is within B’s reach and visibly far from A).

\smallskip

\item
\textit{Only answer E07 if B02 = “yes”}

\item
E07  \textbf{Is B in charge of, or especially responsible for, the service in question?}

\begin{enumerate}
\item{yes}
\item {no}
\end{enumerate}

Only choose “yes” if the answer is clear. The kind of responsibility implied cannot be just a matter of proximity or availability, but must be linked to an individual and his or her social role, or derived from a previous agreement to do the action (preferably documented in the recording). As an example of the former, in Chachi society young girls are expected to bring water from the river to the house, and are more responsible for this task than males and people of other ages. As an example of the latter, in one case in the Italian corpus a woman agreed to add stock cubes to a soup but was then distracted and did not do it; fifteen minutes later, she was told to do the task. As with other questions, if you are unsure, be conservative and answer “no”.
\end{description}

\section*{Acknowledgments}

This coding scheme has benefited from the sustained input of all project contributors -- Julija Baranova, Joe Blythe, Mark Dingemanse, N. J. Enfield, Simeon Floyd, Kobin H. Kendrick, Giovanni Rossi, Jörg Zinken -- and of other participants and conceptual collaborators in data analysis and discussion -- Tyko Dirksmeyer, Paul Drew, Rósa S. Gísladóttir, Steve Levinson, Elizabeth Manrique.

\sloppy
\printbibliography[heading=subbibliography,notkeyword=this]
\end{document}
