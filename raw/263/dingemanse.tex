\documentclass[output=paper]{langsci/langscibook}
\ChapterDOI{10.5281/zenodo.4018388}

\author{Mark Dingemanse\affiliation{Centre for Language Studies, Radboud University}}

\title{Recruiting assistance and collaboration: {A} West-African corpus study}

\shorttitlerunninghead{Recruiting assistance and collaboration: {A} West-African corpus study}

\abstract{Doing things for and with others is one of the foundations of human social life. This chapter studies a systematic collection of 207 recruitments of assistance and collaboration from a video corpus of everyday conversations in Siwu, a Kwa language of Ghana. A range of social action formats and semiotic resources reveals how language is adapted to the interactional challenges posed by recruitment. While many of the formats bear a language-specific signature, their sequential and interactional properties show important commonalities across languages. Two tentative findings are put forward for further cross-linguistic examination: a “rule of three” that may play a role in the organization of successive response pursuits, and a striking commonality in animal-oriented recruitments across languages that may be explained by convergent cultural evolution. The Siwu recruitment system emerges as one instance of a sophisticated machinery for organizing collaborative action that transcends language and culture.}
\maketitle
\label{sec:dingemanse}
\begin{document}

\section{Introduction}

%\todo{check epsilon and open o for missing accents}
Doing things for and with others is one of the foundations of human social life. The question of how we recruit assistance and collaboration has venerable roots in ethnography (\citealt{malinowski_problem_1923,frake_how_1964}) and in the philosophical study of speech acts (\citealt{austin_how_1962,searle_speech_1969}). Yet it has only recently become possible to address it more systematically using records of actual behavior in conversation \citep{drew2014requesting}. Here I study one of the most concrete forms of prosociality in everyday social interaction: recruitments, when someone gets another to carry out a practical action for or with them. Examining the interactional practices by which people come to do things for and with each other contributes to our understanding of the role of language in human sociality.

Much prior work on requesting in social interaction has focused on how requests are shaped by participants’ claims of entitlement \citep{heinemann_will_2006,curl_contingency_2008} or how formats are selected depending on the degree of imposition on a recipient \citep{brown_universals_1978,fukushima_request_1996}. To bring out differences clearly, such analyses often contrast a small number of formats under broad social or situational asymmetries. Complementing such approaches, this study presents a survey of the recruitment system of one language based on a systematic collection of 207 recruiting and responding moves from a corpus of informal conversation. By focusing on the recruitment of practical actions, we can observe a range of factors that shape how people get others to do things in everyday interaction.

One way of understanding the organization of verbal and nonverbal resources in recruitment sequences is as addressed to a set of interactional challenges. People have to reach a joint understanding of who will carry out the practical action and why; what exactly needs to be done and when; how to coordinate bodily behavior and manipulate the physical environment; how to relate the desired action to preceding, ongoing, and projected activities; and other contingencies that require some degree of implicit or explicit calibration \citep{clark_social_2006,goodwin_calibration_2013,enfield_human_2014}. The elements of recruitment sequences appear to be adapted to these challenges, which provides us with a roadmap to the interactional practices surveyed in this chapter (\tabref{tab:dingemanse:1}).

\begin{table}[t]
\small
\begin{tabularx}{\textwidth}{lQQQ}
\lsptoprule
& \textsc{interactional challenge} & \textsc{resources for participant A include} & \textsc{resources for participant B include}\\
\midrule
(i) & Establishing addresseeship & Gaze, address terms, summonses, interjections & Self-selecting, attending or ignoring\\
(ii) & Impinging on freedom of action & Invoking rights and duties by means of reasons and social roles; mitigating and strengthening; pursuing a response & Assenting or resisting (if the latter, providing reasons)\\
(iii) & Specifying desired action & Formulating a request or noticing; pointing and placing; providing reasons & Fulfilling; initiating repair; proposing another action\\
(iv) & Coordinating physical presence & Producing preparatory movements like holding out or reaching to receive & Fulfilling; accounting for delay or inability\\
(v) & Managing activity structure & Formulating relation of request to current involvement; specifying consecutive actions; sequence closing thirds & Verbally committing while finishing current activity\\
\lspbottomrule
\end{tabularx}
\caption{Interactional challenges to be negotiated in recruitment sequences, along with some of the interactional practices mobilized to address them.}
\label{tab:dingemanse:1}
\end{table}

Not all resources make their appearance in every recruitment sequence. When people are already in a dyadic interaction, close to each other, and involved in an activity with a projectable structure, a recruiting move and its response can be minimal, even nonverbal \citep{Rossi2014}. In other situations, interactional contingencies may need to be negotiated more explicitly, bringing a wider range of practices into play. This way, the recruitment system provides for a flexible organization of verbal and nonverbal resources adapted to the task of organizing assistance and collaboration.

\subsection{The Siwu language}

Siwu is a Kwa language spoken north of Hohoe in Ghana’s Volta Region. It has somewhere between 15.000 and 25.000 speakers depending on how the diaspora community is counted. This paper is based on Siwu as spoken in the village of Akpafu-Mempeasem. Siwu is a language in which grammatical relations are established primarily by word order (which canonically is SVO) along with an extensive system of noun classification and agreement. The earliest lexical records for the language date back to the late 19th century, and there are recent sketches of phonology, morphosyntax, and the repair system \citep{kropp_dakubu_central_1988,dingemanse_other-initiated_2015}.

Studies of informal social interaction in West African languages are rare, as linguists have traditionally privileged phonetics, phonology and morphosyntax over semantics, pragmatics and language use (but see \citealt{ameka_ewe_1991,obeng_conversational_1999,meyer_self_2010} for prior work on interactional routines in some West African languages). By describing practices for getting another's assistance or collaboration in Siwu, this paper contributes not only to the documentation of this language, but also to a larger program of understanding how language is shaped by and for social interaction. As we shall see, interactional practices in a basic domain such as getting assistance and collaboration combine universal structural properties with language-specific resources. So the practices and principles described here are of broad relevance to the cross-linguistic study of recruitments and of talk-in-interaction.

\subsection{Data collection and corpus}\label{sec:dingemanse:1.2}

This work is based on a video corpus of naturally occurring conversations in Siwu, collected from consenting participants over the period 2007--2013. The target behavior was maximally informal social interaction: the primary ecology of language in use and the most promising baseline for cross-cultural comparison \citep{dingemanse_conversation_2014}. All of the recordings were made outdoors, where most social interaction between family and friends happens. The recordings cover dyadic as well as multiparty conversations between family and friends. To achieve a diverse and representative collection of recruitment sequences, multiple 10-minute stretches from a total of 11 different interactions were exhaustively sampled, amounting to a total of almost 3 hours of conversation in everyday settings.

A first sweep through this corpus identified a total of 389 candidate recruitments, which amounts to over two recruitments for every minute sampled. This includes 173 cases involving small children as recruiter or recruitee, reflecting the fact that children engage in interactive prosocial behavior from a young age \citep{warneken_emergence_2013}. Such recruitment sequences stand out from other cases in a number of ways, most striking among them a higher number of noticeably absent responses and concomitant response pursuits (see \sectref{sec:dingemanse:5}).

To avoid skewing the sample and to maintain comparability with other languages, recruitment sequences involving small children were not included in the core collection for Siwu, leaving only sequences involving adults and children roughly from age eight onward (when they are clearly treated as having their own deontic authority, along with typical domestic rights and duties).\footnote{Any boundary drawn in order to achieve comparability is debatable. In \sectref{sec:dingemanse:5} I discuss excluded cases and offer some observations on notable differences.}  This leaves a core collection of 207 recruiting moves initiating 146 independent recruitment sequences. Even this conservative count finds roughly one recruiting move for every minute of conversation sampled, showing the fundamental importance of these interactional practices for social life.

\section{Basics of recruitment sequences}

There are many ways of conceptualizing assistance and collaboration in interaction, giving rise to a variety of terms and definitions in prior work. To achieve cross-linguistic comparability, the focus of this study is on sequences of interaction where one participant \textit{recruits} another to do something practical. The phenomenon of recruitment is defined as a sequence of two moves with the following characteristics (see \chapref{sec:intro}, \sectref{sec:intro:4}):

\begin{description}
\item[Move A:] participant A says or does something to participant B, or that B can see or hear;
\item[Move B:] participant B does a practical action for or with participant A that is fitted to what A has said or done.
\end{description}

This definition characterizes the phenomenon as a conversational sequence, implying that a variety of semiotic resources may be used to implement it. The sequential nature of the definition means that we can use the “natural”  or “sequential control method” (\citealt{dingemanse_conversation_2014}; cf. \citealt{zimmerman_horizontal_1999}) to locate comparable cases across settings and societies. The main focus is on practical actions in the here-and-now. Of course, people also recruit assistance or collaboration for matters that cannot be fulfilled immediately (e.g. building a house or borrowing a car). These cases are beyond our scope here, though they are likely to involve substantially similar resources.

\subsection{Minimal recruitment sequence}

Many recruiting moves are minimally formatted and straightforwardly complied with. In \REF{ex:dingemanse:1}, participants are checking some batches of rice (\figref{fig:dingemanse:1}). Eku asks Yawa to give her ‘the deep calabash one’ (line 1), referring to some rice in a deep calabash resting at Yawa’s feet. She reaches out to receive it (line 2) in anticipation of Yawa handing it over (line 4).  In the transcripts, ▶ and ▷ are used to mark the moves in focus, distinguishing initiating and responding moves where relevant. The individual frames within the figures are designated as \textit{a}, \textit{b}, \textit{c}, etc. from left to right.

\transheader{ex:dingemanse:1}{Maize1\_6539207}\vspace{-1mm}
%
\begin{mdframednoverticalspace}[style=firstfoc]
\begin{transbox}{1}{eku}
\begin{verbatim}
kà  su   kabubu        amɛ    ire [tã  mɛ lònyɔ.
ING take deep.calabash inside one  let me 1SG:look
\end{verbatim}
take the deep calabash one and let me see
\end{transbox}
\end{mdframednoverticalspace}
%
\begin{mdframednoverticalspace}[style=firstfoc]
\begin{transbox}{2}{~}
\begin{verbatim}
                                  [((reaches out for calabash, Fig. 1a))
\end{verbatim}
\end{transbox}
\end{mdframednoverticalspace}
%
\begin{transbox}{3}{yaw}
\begin{verbatim}
àrĩ  abùà      agbagba[rà       ló
rice it:exceed it:IDPH.be.large FP
\end{verbatim}
{this} rice is really large-grained
\end{transbox}
%
\begin{mdframednoverticalspace}[style=secondfoc]
\begin{transbox}{4}{~}
\begin{verbatim}
                      [((takes calabash and hands it to Eku, Fig. 1b))
\end{verbatim}
\end{transbox}
\end{mdframednoverticalspace}\vspace{-2mm}
%
\begin{transbox}{5}{eku}
\begin{verbatim}
àba     ɔrãrã  ànaà.
it:have weight too
\end{verbatim}
it’s heavy, too
\end{transbox}

\begin{figure}
\begin{tabularx}{\textwidth}{ll}
\centering
\includegraphics[width=.47\textwidth]{figures/siwu-img1.jpg} & \includegraphics[width=.47\textwidth]{figures/siwu-img2.jpg}
\end{tabularx}

\caption{(\textit{a}) recruiting move by Eku (sitting right, lines 1--2); (\textit{b}) responding move by Yawa (line 4).}
\label{fig:dingemanse:1}
\end{figure}

This recruitment is minimal in the sense that it consists of an initiating move -- Eku’s ‘take the deep calabash one and let me see’ (\figref{fig:dingemanse:1}\textit{a}) -- and a single response -- Yawa taking the calabash and handing it to Eku (\figref{fig:dingemanse:1}\textit{b}). About two thirds of all independent recruitment sequences in the corpus (102 out of 146) have this kind of simple two-part structure of initiating move and response.

\subsection{Non-minimal recruitment sequence}

The complex interactional challenges at play in everyday recruitments are easy to overlook in minimal sequences, where a pre-existing shared focus of attention, physical co-presence, and activity structure conspire to enable a simple request that is immediately fulfilled. About one third of independent recruitment sequences (44 out of 146) take more than one attempt to reach completion. In such non-minimal sequences, the levels of coordination are pulled apart a bit, similar to the way in which an exploded-view diagram can show the elements and order of assembly of a complex piece of machinery.

Two common ways in which non-minimal sequences happen are
(i) when a response to the recruiting move is noticeably absent or delayed, which often results in the recruiter pursuing a response, and (ii) when a recruitee claims a problem of hearing or understanding and initiates repair. \extref{ex:dingemanse:2} illustrates the first type (for the second type, see \sectref{sec:dingemanse:4.2} below). Beatrice is cleaning some pots and pans while Afua, her mother, is holding Beatrice’s infant. When the infant becomes increasingly restless, Afua asks Beatrice to wash her hands and take him over (line 1). Beatrice immediately provides an affirmative verbal response (line 2), but in the next 10 seconds she appears to continue her current involvement, even taking up another pot to clean. This leads to multiple response pursuits by Afua (lines 5--6, 8) until Beatrice carries out the requested action.

\transheader{ex:dingemanse:2}{Kitchen1\_1052883}\vspace{-1mm}
%
\begin{mdframednoverticalspace}[style=firstfoc]
\begin{transbox}{1}{afu}
\begin{verbatim}
Beatrice fore nrɔ̃   si  àba      àakɔ         ũ=
PSN      wash hands LNK 2SG:come 2SG:FUT:take him
\end{verbatim}
Beatrice wash your hands, so you can come and take him
\end{transbox}
\end{mdframednoverticalspace}
%
\begin{mdframednoverticalspace}[style=secondfoc]
\begin{transbox}{2}{bea}
\begin{verbatim}
=ao
yes
\end{verbatim}
yeah
\end{transbox}
\end{mdframednoverticalspace}
%
\begin{transbox}{3}{afu}
\begin{verbatim}
nɛ ɔ̃ũ    bùa    ɔsɛ
TP he.TP exceed 3SG:sit
\end{verbatim}
cause he’s done sitting
\end{transbox}
%
\emptytransbox{4}{(10.0) ((Beatrice takes up another pot and starts cleaning it))}
%
\begin{mdframednoverticalspace}[style=firstfoc]
\begin{transbox}{5}{afu}
\begin{verbatim}
Beatrice mɛ sɔ   fore nrɔ̃
PSN      me says wash hands
\end{verbatim}
Beatrice I said wash your hands
\end{transbox}
\end{mdframednoverticalspace}
%
\begin{mdframednoverticalspace}[style=firstfoc]
\begin{transbox}{6}{~}
\begin{verbatim}
si   àba      àa      kɔ   ũ   si  ɔnyũa    kàku ɔɔbiɛ ló.
LNK  2SG:come 2SG:FUT take him LNK 3SG:stop cry  crying FP
\end{verbatim}
so you can come and take him, so he’ll stop crying
\end{transbox}
\end{mdframednoverticalspace}
%
\begin{mdframednoverticalspace}[style=secondfoc]
\begin{transbox}{7}{bea}
\begin{verbatim}
aoo: ((speeds up and finishes cleaning, starts washing her hands))
yes
\end{verbatim}
ye:s
\end{transbox}
\end{mdframednoverticalspace}
%
\begin{transbox}{8}{afu}
\begin{verbatim}
nɛ ɔ̃ũ    bùa    ose
TP he.TP exceed 3SG:sit
\end{verbatim}
because he’s done sitting
\end{transbox}
%
\emptytransbox{9}{(37.0) ((Beatrice finds a towel, dries her hands, and walks towards Afua; baby cries))}
%
\begin{mdframednoverticalspace}[style=secondfoc]
\begin{transbox}{10}{bea}
\begin{verbatim}
ooo! ((picks up baby))
EXCL
\end{verbatim}
ooh!
\end{transbox}
\end{mdframednoverticalspace}
%
\emptytransbox{11}{(3.0) ((baby calms down))}\\

\normalsize
This case illustrates a range of practices commonly used to manage the interactional challenges posed by recruitment sequences. For Afua, this includes using a proper name to secure joint attention (line 1), providing a reason that orients to Beatrice’s current involvement, pursuing response by marking the follow-up as a resaying (lines 5--6), and invoking Beatrice’s responsibilities for the task at hand (lines 3, 6). For Beatrice, this includes using affirmative responses to signal willingness to comply (lines 2, 7), visibly speeding up and shifting tasks to signal imminent availability (line 7), and finally carrying out the requested action (line 10). All of the practices noted here are discussed in more detail below.

That there are non-minimal sequences means that not all 207 recruiting moves in the core collection are independent events: some are pursuits of response following problems in compliance or other-initiations of repair.\footnote{In \sectref{sec:dingemanse:5.1}, I discuss an apparent limit to the number of pursuits observed.} In what follows, where relevant, I make a distinction between \textit{initial} (or \textit{independent}) versus \textit{subsequent} recruiting moves, and I reserve the term \textit{recruitment sequence} for the full sequence -- minimal or non-minimal -- an initial recruiting move gives rise to.

\subsection{Subtypes of recruitment sequences}

The practical actions instigated by recruiting moves can be classified into types. Three common ones are (i) the transfer of an object from B to A, (ii) the provision of a service by B for A, and (iii) the alteration of a trajectory of action. We have seen an object transfer in  \REF{ex:dingemanse:1}, where a calabash changes hands, and the provision of a service in  \REF{ex:dingemanse:2}, where a mother is recruited to take care of her child. The notion of “service” is the broadest of the three and it is no surprise that this turns out to be the most frequent category in the corpus (\tabref{tab:dingemanse:2}).

\begin{table}
\begin{tabularx}{\textwidth}{XCr}
\lsptoprule
Recruitment type &  Count &  Examples\\
\midrule
Transferring an object & 16 &  \REF{ex:dingemanse:1},  \REF{ex:dingemanse:9},  \REF{ex:dingemanse:17},  \REF{ex:dingemanse:18},  \REF{ex:dingemanse:22},  \REF{ex:dingemanse:27}\\
Providing a service & 111 &  \REF{ex:dingemanse:2},  \REF{ex:dingemanse:5},  \REF{ex:dingemanse:6},  \REF{ex:dingemanse:8},  \REF{ex:dingemanse:15},  \REF{ex:dingemanse:16},\\ 
& & \REF{ex:dingemanse:19}, \REF{ex:dingemanse:20},  \REF{ex:dingemanse:21},  \REF{ex:dingemanse:25},  \REF{ex:dingemanse:28}\\
Altering a trajectory & 19 &  \REF{ex:dingemanse:3},  \REF{ex:dingemanse:13},  \REF{ex:dingemanse:14},  \REF{ex:dingemanse:26}\\
\lspbottomrule
\end{tabularx}

\caption{Types of recruitment sequence and their frequency in Siwu (counting only independent sequences).}
\label{tab:dingemanse:2}
\end{table}

\extref{ex:dingemanse:3} below illustrates the third type of recruitment, where one person asks another to alter an ongoing trajectory of behavior. Yao and Afua are producing palm oil when Lucy stops by their compound to ask something (line 1, 4). She happens to position herself right before the camera. Yao draws attention to this and asks her to move aside.

\newpage
\transheader{ex:dingemanse:3}{Palmoil1\_1118517}\vspace{2mm}
%
\begin{transbox}{1}{luc}
\begin{verbatim}
ǹdɔrɛ̃    kasorekɔ̃         misee? ((moves in front of camera))
firewood LOC:gather:place 2PL:go:Q
\end{verbatim}
are y’all going to the firewood place?
\end{transbox}
%
\begin{transbox}{2}{yao}
\begin{verbatim}
m[m
\end{verbatim}
\end{transbox}
%
\begin{transbox}{3}{afu}
\begin{verbatim}
 [mm
\end{verbatim}
\end{transbox}
%
\begin{transbox}{4}{luc}
\begin{verbatim}
mikɛlɛgu    ilɛ?
2PL:go.with place
\end{verbatim}
where are you going to bring \{it\}?
\end{transbox}
%
\begin{mdframednoverticalspace}[style=firstfoc]
\begin{transbox}{5}{yao}
\begin{verbatim}
nyɔ  àta      àbɔrɛ    gu   fɔ   ɛh
look 2SG:PROG 2SG:move with your HES
\end{verbatim}
look move away with your uh
\end{transbox}
\end{mdframednoverticalspace}
%
\begin{mdframednoverticalspace}[style=firstfoc]
\begin{transbox}{6}{~}
\begin{verbatim}
ɔpò m̀mɔ   nɛ həhəhəh ndza marɔ̃ 
tub there TP         how  3PL:call
\end{verbatim}
tub there həhəhəh what-d’you-call-it
\end{transbox}
\end{mdframednoverticalspace}
%
\begin{mdframednoverticalspace}[style=secondfoc]
\xtransbox{7}{luc}{((steps aside and takes a look at the camera))}
\end{mdframednoverticalspace}\vspace{-2mm}
%
\begin{transbox}{8}{~}
\begin{verbatim}
↑hɛ↑ (.)↑ah↑
\end{verbatim}
\end{transbox}\\

Yao’s request to ‘move away with your uh tub there’ (lines 5--6) is not a response to Lucy’s question. Instead it launches a new course of action, with the turn preface ‘look’ marking a departure from the current course of action \citep{sidnell_look-prefaced_2007} and helping to redirect Lucy’s attention to the camera, which is behind her. She turns around and takes a look at the camera (line 7), producing two high-pitched exclamations of surprise (line 8) which also claim unawareness of the situation and therefore serve to account for her prior actions.

The sequential definition of recruitments used here relies on the recognition of Move B as a practical action for or with participant A. This opens the door to a further possible distinction with regard to how Move B arises. Often, it is prompted by an explicit request in Move A, as we have seen in the examples so far. But it can also arise in anticipation of a current or imminent need. This is illustrated in \REF{ex:dingemanse:4}. 

Emma, a blind woman, is inside a room while some others are chatting and preparing food outside. One of them, Aku, is sitting in the doorway. When it becomes clear that Emma is going to go outside (line 1), Aku stands up from the doorway to make way for her (line 4).

\transheader{ex:dingemanse:4}{Compound4\_2054269}\vspace{-1mm}
%
\begin{mdframednoverticalspace}[style=firstfoc]
\xtransbox{1}{emm}{[((audibly takes some shuffling footsteps toward doorway, Fig. 2a))}
\end{mdframednoverticalspace}
%
\begin{transbox}{2}{kof}
\begin{verbatim}
[mmakosò
 kin.F.junior
\end{verbatim}
\hspace{0.07cm} aunty
\end{transbox}
%
\begin{transbox}{3}{emm}
\begin{verbatim}
°mmakosò     [ɔbi°
kin.F.junior  child
\end{verbatim}
°aunty’s child°
\end{transbox}
%
\begin{mdframednoverticalspace}[style=secondfoc]
\begin{transbox}{4}{aku}
\begin{verbatim}
             [((looks over her shoulder and stands up, 
              freeing doorway, Fig. 2b}))
\end{verbatim}
\end{transbox}
\end{mdframednoverticalspace}
%
\begin{transbox}{5}{kof}
\begin{verbatim}
yara  so
brace self
\end{verbatim}
be careful
\end{transbox}
%
\xtransbox{6}{emm}{((takes further steps, stands still in doorway))}
%
\begin{transbox}{7}{kof}
\begin{verbatim}
nɛ gɔ  ata      àba      nɛ, ɔɔ     ta    ɔ   nɛ-
so how 2SG:PROG 2SG:come TP, 3SG:PF stand 3SG TP
\end{verbatim}
so because you’re coming, she stood up-
\end{transbox}
%
\begin{transbox}{8}{emm}
\begin{verbatim}
mm
\end{verbatim}
\end{transbox}
%
\begin{transbox}{9}{kof}
\begin{verbatim}
ũ  ɔre  Akuvi   ɔta       i   kayogodɔ̃.
my wife PSN:DIM she:stand LOC doorway
\end{verbatim}
my dear Aku stood up from the doorway
\end{transbox}
%
\xtransbox{10}{emm}{((leans against portal and takes a careful step down))}

\begin{figure}
\begin{tabularx}{\textwidth}{ll}
\centering
\includegraphics[width=.45\textwidth]{figures/siwu-img3.jpg} & \includegraphics[width=.45\textwidth]{figures/siwu-img4.jpg}
\end{tabularx}
\caption{(\textit{a}) Aku sits in the doorway as Emma approaches from inside (line 1); (\textit{b}) Aku stands up and frees doorway (line 4). Kofi is not visible in the frame.}
\label{fig:dingemanse:2}
\end{figure}

\normalsize
Cases like \REF{ex:dingemanse:4}, in which someone responds to anticipated trouble, can be challenging to identify because the recruiting move itself is not on-record: Emma does not ask Aku to get up. However, in this case, another participant happens to provide a running commentary that supports an analysis of this event as a recruitment. Kofi, a distant relative hanging around and engaging in occasional chats with the others, first cautions Emma to be careful stepping out the door, then describes what happened in causal and sequential terms, stating how one behavior occassioned another: ‘so because you’re coming, she stood up’ (lines 7, 9). This comment glosses Aku’s assistance as relevant and potentially expected given the context.

\hspace*{-.5mm} Fully nonverbal recruitments like \REF{ex:dingemanse:4} are in the minority and straddle the boundary between offers of help and responses to requests \citep{Curl2006,couper-kuhlen_what_2014}. One reason they are interesting is that off-record cues may, over time, develop into conventionalized signals, and may come to be seen as part of an ordered paradigm of interactional practices \citep{manrique_suspending_2015}. For instance, on urban sidewalks, an audible footstep is often sufficient to “ask” others to make space, and appears to be preferred over an explicit request, a format that tends to be reserved for subsequent attempts. In the following sections, we will explore a range of formats that are more directly on-record as requests for assistance or collaboration.

\section{Formats in Move A: The recruiting move}

\subsection{Nonverbal behavior in recruiting moves}

Most recruiting moves are multimodal utterances composed of speech and bodily behavior. The semiotic resources work in concert to produce the recruiting move, with a division of labor appropriate to the affordances of each modality \citep{goodwin_action_2000,clark_wordless_2012}. Three common forms of nonverbal behavior found in recruiting moves are: (i) reaching to receive an object, illustrated in  \REF{ex:dingemanse:1} above; (ii) holding out an object; and (iii) pointing, illustrated in the following case.

Eku is preparing food. Her teenage daughter Kpɛi has just come back from school and is standing next to the water tank. Eku starts with an imperative \textit{su} ‘take’, then self-repairs to ask Kpɛi to check whether there is water in the tank. After receiving confirmation, she produces a complex request that involves taking a container, filling it with water, pouring that water somewhere, then putting it on the fire (lines 3--6). The under-specification of the verbal content is made up for by a series of pointing gestures, three of which are illustrated in \figref{fig:dingemanse:3}.

\newpage
\transheader{ex:dingemanse:5}{Maize3\_276559a}\vspace{2mm}
%
\begin{transbox}{1}{eku}
\begin{verbatim}
su   ɛ:. ndu   pia mmɔ: ((points in direction of water tank))
take HES water be  there:Q
\end{verbatim}
take uh:. is there water there?
\end{transbox}
%
\begin{transbox}{2}{kpɛ}
\begin{verbatim}
mm.
INTJ
\end{verbatim}
mm
\end{transbox}
% 
\begin{mdframednoverticalspace}[style=firstfoc]
\begin{transbox}{3}{eku}
\begin{verbatim}
su   fore  si  àsu      ɛh  gálɔn  gangbe ((points to gallon, Fig. 3a))
take pour  LNK 2SG:take HES gallon AGR:this
\end{verbatim}
take and pour- then take this gallon
\end{transbox}
\end{mdframednoverticalspace}
%
\begin{mdframednoverticalspace}[style=firstfoc]
\begin{transbox}{4}{~}
\begin{verbatim}
si  àfore    ndu ((points to water, Fig. 3b))
LNK 2SG:pour water
\end{verbatim}
then pour some water
\end{transbox}
\end{mdframednoverticalspace}
%
\emptytransbox{5}{(0.4)}\vspace{-3mm}
%
\begin{mdframednoverticalspace}[style=firstfoc]
\begin{transbox}{6}{eku}
\begin{verbatim}
si  àsu      àsɛ     aàsia       ɔtɔ. ((points to fireplace, Fig. 3c))
LNK 2SG:take 2SG:set 2SG:FUT:put fire
\end{verbatim}
then put it on the fire
\end{transbox}
\end{mdframednoverticalspace}

\begin{figure}
\begin{tabularx}{\textwidth}{lll}
\centering
\includegraphics[width=.305\textwidth]{figures/siwu-img5.jpg} & \includegraphics[width=.305\textwidth]{figures/siwu-img6.jpg} & \includegraphics[width=.305\textwidth]{figures/siwu-img7.jpg}
\end{tabularx}
\caption{Pointing gestures accompanying (\textit{a}) ‘take this gallon’ (line 3), (\textit{b}) ‘pour some water’ (line 4), (\textit{c}) ‘put it on the fire’ (line 6).}
\label{fig:dingemanse:3}
\end{figure}

\normalsize
Besides the three consecutive pointing gestures, this sequence reveals a range of verbal elements that enter into the design of recruiting moves, to which we now turn.

\subsection{Verbal elements: constructions for formulating recruiting moves}\label{sec:dingemanse:3.2}

Recruiting moves come in different \textit{formats}, conventionalized linguistic practices that deliver social actions \citep{thompson_clause_2005,fox_rethinking_2016}. For recruiting turns that include a predicate, it is possible to distinguish between a number of constructions and grammatical moods (\tabref{tab:dingemanse:3}). There is a small number of recruiting turns that do not feature a predicate (for instance, combining ‘hey’ with a pointing gesture to draw someone’s attention to an actionable matter). Also, in 11 mixed cases, formats are combined. The basic construction types reviewed here can be further enriched with a range of final particles and other elements, described in the next section.

\begin{table}
\begin{tabularx}{.85\textwidth}{Xrrrr}
\lsptoprule
Format & Initial & Subsequent & Total & Examples\\
\midrule
Imperative & 83 & 31 & 114 &  \REF{ex:dingemanse:3},  \REF{ex:dingemanse:6}\\
\textit{Si}-prefaced & 10 & 3 & 13 &  \REF{ex:dingemanse:5},  \REF{ex:dingemanse:14}\\
Declarative & 6 & 5 & 11 &  \REF{ex:dingemanse:7},  \REF{ex:dingemanse:8}\\
Interrogative & 7 & 1 & 8 &  \REF{ex:dingemanse:9},  \REF{ex:dingemanse:10}\\
Jussive & 6 & 2 & 8 &  \REF{ex:dingemanse:11},  \REF{ex:dingemanse:12}\\
No predicate & 4 & 3 & 7 &  \REF{ex:dingemanse:20},  \REF{ex:dingemanse:23}\\
Mixed & 6 & 5 & 11 &  \REF{ex:dingemanse:1},  \REF{ex:dingemanse:2}\\
\lspbottomrule
\end{tabularx}
\caption{Verbal formats of 172 recruiting moves (excluding 35 fully nonverbal cases).}
\label{tab:dingemanse:3}
\end{table}

As \tabref{tab:dingemanse:3} shows, all construction types occur in initial as well as subsequent position. However, there are some patterns that suggest an ordering of resources. For instance, 7 out of 8 interrogatives are found in initial position (the only subsequent case is a response pursuit that repeats an initial interrogative). So an interrogative is never selected as an upgrade of another format. But the opposite does occur, as when an initial interrogative is reformulated as a proposal in \REF{ex:dingemanse:10} below. Conversely, some non-predicative formats like \textit{anɔ:} ‘y’hear?’ in \REF{ex:dingemanse:20} occur only in subsequent position, as a result of the fact that one can pursue a response to a recruiting move by repeating only part of it -- in this case, the final tag.

Linguistic labels such as those in \tabref{tab:dingemanse:3} are employed here for ease of reference. However, the analysis of these formats below is focused more on understanding the interactional work done with these formats, each of which has its own affordances for social action. To briefly preview the interactional work done with the main constructions: imperatives allow people to direct each other’s actions; \textit{si}-prefaced recruiting moves present a requested action as a logical consequence; declaratives are noticings that present reasons for action; negative interrogatives mark deviations from expected courses of action; and jussives frame recruitments as suggestions for courses of action.

\subsubsection{Imperative}

The basic imperative in Siwu consists simply of the bare verb, usually morphologically unmarked for person and number, though occasionally the second person plural prefix \textit{mi-} can be found. Some imperatives feature just a verb phrase (e.g. \textit{sa mà} ‘chase them away’, \extref{ex:dingemanse:6}), others add a beneficiary (e.g. \textit{su tã mɛ} ‘gimme back’, lit. ‘take give me’, \extref{ex:dingemanse:22}) or a more elaborate specification of the desired action (e.g. \textit{ba fore mɛ ndu} ‘come pour me water’, \extref{ex:dingemanse:21}). Serial verb constructions, as in the latter two examples, are common.

Although a plural form of the imperative does exist, most recruiting moves are unmarked for person or number, even when the recruited action is taken up by multiple people. An example of this is given in  \REF{ex:dingemanse:6}, where one participant notices some goats getting too close to the food and issues a directive to ‘chase them away’. Her recruiting move is unaddressed and unmarked for person or number, and is taken up by two people who are closer to the goats than the recruiter is (lines 5, 6).

\transheader{ex:dingemanse:6}{Cooking1\_1545188}\vspace{2mm}
%
\emptytransbox{1}{((goats approach food))}
%
\begin{mdframednoverticalspace}[style=firstfoc]
\begin{transbox}{2}{afu}
\begin{verbatim}
sà    ma
chase them
\end{verbatim}
chase them \{away\}
\end{transbox}
\end{mdframednoverticalspace}
%
\emptytransbox{3}{(0.5)}
%
\begin{transbox}{4}{afu}
\begin{verbatim}
sà [ma
chase them
\end{verbatim}
chase them \{away\}
\end{transbox}
%
\begin{mdframednoverticalspace}[style=secondfoc]
\begin{transbox}{5}{taw}
\begin{verbatim}
   [kai (0.4) [↑kai ((waves arm))
    INTJ       INTJ
\end{verbatim}
\hspace{0.55cm} skai \hspace{1.05cm} kai
\end{transbox}
\end{mdframednoverticalspace}
%
\begin{mdframednoverticalspace}[style=secondfoc]
\begin{transbox}{6}{adz}
\begin{verbatim}
              [hî   hî,  hî,  hî   ↑híì↑ ((waves arm))
               INTJ INTJ INTJ INTJ INTJ
\end{verbatim}
\hspace{2.2cm} hî   hî,  hî,  hî   ↑híì↑
\end{transbox}
\end{mdframednoverticalspace}
%
\emptytransbox{7}{((goats flee the scene))}\\

\normalsize
Imperatives are by far the most common construction type in the Siwu data, accounting for 59\% of all recruiting moves and over 70\% of moves featuring speech. As we will see below, there are several ways of designing imperative recruiting moves to specify consecutive actions (\textit{si}-prefacing, \sectref{sec:dingemanse:3.3.2}) or to mark fine differences in stance or illocutionary force (final particles, \sectref{sec:dingemanse:3.3.3}).

\subsubsection{Declaratives and interrogatives}\label{sec:dingemanse:3.2.2}

Some recruiting moves in the collection come in the form of declaratives. All of them are noticings of some actionable event or matter that requires attention. In \REF{ex:dingemanse:7}, two women are chatting while preparing food. Vicky is in the process of telling a story when she sees a chicken coming up behind Tawiya. She interrupts her telling mid-turn (line 3) to tell Tawiya of the chicken, marking it as a piece of advice with the final particle \textit{ló} (line 4), which results in Tawiya shooing away the chicken using the animal-oriented interjection \textit{shuɛ} (line 5). Without missing a beat, Vicky then resumes the story by recycling material from the turn she abandoned (line 7).

\transheader{ex:dingemanse:7}{Compound4\_1600030}\vspace{2mm}
%
\begin{transbox}{1}{vic}
\begin{verbatim}
ma- masɛ   maàmala       ɔ̃   ara    ideye,
3PL 3PL:go 3PL:PST:store her things it:seems
\end{verbatim}
they they went and stored her things
\end{transbox}
%
\begin{transbox}{2}{~}
\begin{verbatim}
màamala       ɔ̃   ara    ideye,
3PL:PST:store her things it:seems
\end{verbatim}
they stored her things
\end{transbox}
%
\begin{transbox}{3}{~}
\begin{verbatim}
si  màanyɔ-
LNK 3PL:PST:see
\end{verbatim}
then they saw-
\end{transbox}
%
\begin{mdframednoverticalspace}[style=firstfoc]
\begin{transbox}{4}{~}
\begin{verbatim}
kɔkɔ  to   ɔki        ɔlɔ[ɔ           mmɔ   ló ((bends forward))
chick PROG 3SG:circle 3SG:hover:2SG.O there FP
\end{verbatim}
a chicken is hovering around you there \textit{ló}
\end{transbox}
\end{mdframednoverticalspace}
%
\begin{mdframednoverticalspace}[style=secondfoc]
\begin{transbox}{5}{taw}
\begin{verbatim}
                         [↑shuɛ:↑ ((moves to chase away chicken))
                           INTJ
\end{verbatim}
\hspace{3.95cm} shoo!
\end{transbox}
\end{mdframednoverticalspace}
%
\emptytransbox{6}{((chicken moves away))}\vspace{-2mm}
%
\begin{transbox}{7}{vic}
\begin{verbatim}
si  màanyɔ      Mercy ɔɔkpese        ɔkpa      ànaà.
LNK 3PL:PST:see PSN   3SG:PST:return 3SG:leave again
\end{verbatim}
then they saw Mercy had gone back and disappeared
\end{transbox}\bigskip

\normalsize
A similar case happens later in the same interaction, when Tawiya has put a pan on the fire next to her and Vicky sees it sliding from one of the firestones, at risk of toppling. Vicky notifies Tawiya by pointing out the trouble and Tawiya responds by righting the pan.

\transheader{ex:dingemanse:8}{Compound4\_1655650}\vspace{2mm}
%
\emptytransbox{1}{((pan slides off one of the firestones))}
%
\begin{mdframednoverticalspace}[style=firstfoc]
\begin{transbox}{2}{vic}
\begin{verbatim}
kãrã te      ìturu. ((points to trouble))
pan  it:PROG it:tilt
\end{verbatim}
the pan is tilting
\end{transbox}
\end{mdframednoverticalspace}
%
\begin{transbox}{3}{taw}
\begin{verbatim}
↑mm↑ ((turns to look, repositions the pan))
\end{verbatim}
\end{transbox}\\

\normalsize
In both cases, the declarative formatting is well suited to delivering a verbal “noticing” of some actionable trouble which the other may not have noticed yet and is in a good position to resolve (see also Kendrick, \chapref{sec:kendrick}, \sectref{sec:kendrick:4.2.3}; Rossi, \chapref{sec:rossi}, \sectref{sec:rossi:3.3.4}; Baranova, \chapref{sec:baranova}, \sectref{sec:baranova:3.3.3}).\footnote{A reviewer points out that the beneficiary of the target action here is not clearly the person producing the recruiting turn, making the recruitment akin to what \citet{couper-kuhlen_what_2014} describes as “suggestions”. However, such suggestions in Couper-Kuhlen’s English data are “likely to be resisted in everyday conversation” (p. 635) and often have the recipient as the primary beneficiary; here, no such resistance is in evidence and the beneficiary is neither self nor other alone, but both together.}



Question-formatted recruiting moves are rare in the Siwu data, and the most common type is a negative interrogative format. In \REF{ex:dingemanse:9}, Dora spots somebody walking off and asks ‘hey, aren’t you bringing me water?’ (line 1). The negative interrogative design gives the recruiting move a complaining quality \citep{heinemann_will_2006} and appears to orient to a decreased likelihood of immediate fulfillment. Indeed, Efi indicates that she will be going someplace else before coming back. Dora’s response provides further evidence of the complaint-like quality of the initial formulation: ‘it’s because of you this woman has not bathed yet’ (line 5).

\transheader{ex:dingemanse:9}{Maize1\_6136999}\vspace{-1mm}
%
\begin{mdframednoverticalspace}[style=firstfoc]
\begin{transbox}{1}{dor}
\begin{verbatim}
HÀƐ: AITÀ         BƆ    MƐ NDU:
INTJ 2SG:NEG:PROG bring me water:Q
\end{verbatim}
HE:Y AREN'T YOU BRINGING ME WATER?
\end{transbox}
\end{mdframednoverticalspace}
%
\emptytransbox{2}{(1.0)}\vspace{-2mm}
%
\begin{transbox}{3}{efi}
\begin{verbatim}
losɛ   kàto ngbe loba.
1SG:go top  here 1SG:come
\end{verbatim}
I'm going up, I’ll be back
\end{transbox}
%
\emptytransbox{4}{(1.3)}\vspace{-3mm}
%
\begin{transbox}{5}{dor}
\begin{verbatim}
ƆƆNYA       FƆ  ƆSO    ƆRƆ̃GO GƆǸGBE   ŨIPIE         NDU
2SG:PFV:see 2SG reason woman REL:here 3SG:NEG:bathe water
\end{verbatim}
YOU SEE IT'S BECAUSE OF YOU THIS WOMAN HAS NOT BATHED YET
\end{transbox}\bigskip

\normalsize
In \REF{ex:dingemanse:10}, mealtime is approaching and Afua calls out to her fellow clan member Eku asking ‘won’t you eat food?’, the plural \textit{mi}- signaling that Eku is in the company of others. When no response follows, she upgrades the recruiting move, shifting from an interrogative to a jussive format, discussed in the next section. The recruitment attempts are ignored and then abandoned as the conversation lapses.

\transheader{ex:dingemanse:10}{Cooking1\_1266243}\vspace{-1mm}
%
\begin{mdframednoverticalspace}[style=firstfoc]
\begin{transbox}{1}{afu}
\begin{verbatim}
Daa    Ɛku (.) mìite        mìde    ara:
sister PSN     2PL-NEG-PROG 2PL-eat thing
\end{verbatim}
Sister Eku (.) won’t you eat food?
\end{transbox}
\end{mdframednoverticalspace}
%
\emptytransbox{2}{(0.4)}
%
\begin{mdframednoverticalspace}[style=firstfoc]
\begin{transbox}{3}{afu}
\begin{verbatim}
mìba     mìade       adera.
2PL:come 2PL:FUT:eat food
\end{verbatim}
you should come and eat food
\end{transbox}
\end{mdframednoverticalspace}
%
\emptytransbox{4}{((interaction lapses))}

\normalsize
\subsubsection{Jussives}
Recruiting moves can be formulated as proposals using the verb forms \textit{ba} ‘come’ and \textit{tã} ‘give’, which can be structurally characterized as jussives. The first is often heard in the formulaic proposal ‘come let’s eat’ that is routinely addressed to passers-by when people are sharing a meal. In  \REF{ex:dingemanse:11}, Ruben invites Kodzo to share a meal, though Kodzo declines. This first person plural formulation is the most commonly encountered version of the \textit{ba} ‘let’s’ format. An instance of the second person plural version is found in \REF{ex:dingemanse:10} above.

\transheader{ex:dingemanse:11}{Compound4\_2048169}\vspace{2mm}
%
\begin{transbox}{1}{rub}
\begin{verbatim}
kà  ba   bòde    adera  ló
ING come 1PL:eat food   FP
\end{verbatim}
come let’s eat \textit{ló}
\end{transbox}
%
\begin{transbox}{2}{kod}
\begin{verbatim}
oò,  mìla     i   mmɔ   ló.
INTJ 2PL-hold LOC there FP
\end{verbatim}
oh, you just keep at it \textit{ló}
\end{transbox}\bigskip

\normalsize
Another jussive format frames the recruitment as a proposal with a beneficiary, owing to the semantics of the \textit{tã} auxiliary, derived from \textit{tã} ‘give’. We have seen one example in \REF{ex:dingemanse:1}, where the beneficiary is the recruiter herself (‘let me look’); in \REF{ex:dingemanse:12}, it is a third person (‘let him sit by your side’). Evidence of the auxiliary status of \textit{tã} comes from the occurrence of negative forms like ‘don’t let me get sore’ in \REF{ex:dingemanse:13}. If \textit{tã} were a bona fide verb here, it would require the benefactive to follow immediately after it (\textit{tã mɛ} ‘give me’); instead, it conveys a jussive meaning ‘let \{it\}’ and the main predicate is \textit{bɛbɛrɛ} ‘burn, feel sore’.

\transheader{ex:dingemanse:12}{Maize3\_673020 (see \extref{ex:dingemanse:20} for full sequence)}\vspace{2mm}
%
\begin{transbox}{4}{aku}
\begin{verbatim}
tã  ũ   ɔsɛ     i   fɔ  kɔrɛ.
let him NOM-sit LOC 2SG side
\end{verbatim}
let him sit by your side
\end{transbox}\medskip

\transheader{ex:dingemanse:13}{Compound5\_366774}\vspace{2mm}
%
\begin{transbox}{1}{aku}
\begin{verbatim}
daa tã   bɛbɛrɛ mɛ
NEG let  burn   me
\end{verbatim}
don’t let me get sore
\end{transbox}\bigskip

\normalsize
Interrogatives and jussives are typically classified as more indirect than imperatives, and prior work in cross-cultural pragmatics suggests “a strong preference for conventional indirectness” in languages like English and German \citep{ogiermann_politeness_2009}. In Siwu, by contrast, imperative constructions are the main workhorse for recruiting moves, and interrogatives and jussives play relatively minor roles. With imperatives, declaratives, interrogatives and jussives, we have exhausted the basic grammatical distinctions found in recruitment predicates in Siwu.

\subsection{Additional verbal elements}

\subsubsection{Vocatives}
One prerequisite for fulfilling a recruitment is establishing who will do it. In multiparty interaction, vocatives -- linguistic resources such as proper names and interjections used for addressing people -- provide one way to address recruiting moves to specific participants and to get their attention. We saw this in earlier examples where recruiting moves are prefaced by proper names: ‘Beatrice wash your hands [...]’ \REF{ex:dingemanse:2} and ‘Sister Eku, won’t you eat?’ \REF{ex:dingemanse:10}. In both cases, the recruitment happens in multiparty interaction, and the vocative helps to cut across established participation frameworks and activities to address a specific recipient.

Proper names and other terms of address can also show up in summons-answer sequences preceding the recruiting move. Though not an “additional element” in such cases (see \chapref{sec:coding}, \sectref{sec:coding:6}), I discuss them here because of their connection to vocatives. An example is given in \REF{ex:dingemanse:14} below. Bella calls her mother with the vocative \textit{mama} and, after getting an answer, asks her to get up and sit elsewhere while preparing the food. A summons-answer sequence serves the role of establishing an open channel for interaction \citep{Schegloff1968}. Other examples can be found in \REF{ex:dingemanse:15}, \REF{ex:dingemanse:31}, and \REF{ex:dingemanse:33}.

\transheader{ex:dingemanse:14}{Cooking1\_1188540}\vspace{2mm}
%
\begin{transbox}{1}{bel}
\begin{verbatim}
mama.
\end{verbatim}
\end{transbox}
%
\begin{transbox}{2}{mom}
\begin{verbatim}
↑m
\end{verbatim}
\end{transbox}
%
\begin{transbox}{3}{bel}
\begin{verbatim}
ta     si  àbara  nɛ   ngbe ((walks with a bench in direction of table))
get.up LNK 2SG:do this here
\end{verbatim}
get up and do it here
\end{transbox}
%
\xtransbox{4}{mom}{((finishes her task of peeling cassava, then gets up and repositions herself))}\bigskip

\normalsize
Vocative interjections like ‘hey’ can be used in the same two sequential environments: as a summons separate from the recruiting turn or as an element within the turn. We saw an example of the latter in \REF{ex:dingemanse:9}, where Dora addresses someone in the distance with ‘hey, aren’t you bringing me water?’.

\subsubsection{Marking consecutive actions and giving reasons}\label{sec:dingemanse:3.3.2}

Many recruiting moves in the collection consist of an imperative followed by the specification of a consecutive action that is introduced using the morpheme \textit{si}. An example of this can be found in \REF{ex:dingemanse:2}, ‘wash your hands \textit{si} you come take him’. For this morpheme, I adopt the term “linker” from Ameka's (\citeyear{ameka_aspect_2008}) analysis of Ewe \textit{né}, a form with a similar range of uses. In \REF{ex:dingemanse:15}, Eku asks her daughter Afua to take a broom and sweep the compound, introducing the second element of the action with \textit{si}.

\transheader{ex:dingemanse:15}{Neighbours\_4593390}\vspace{2mm}
%
\begin{transbox}{1}{eku}
\begin{verbatim}
Afua
PSN
\end{verbatim}
Afua
\end{transbox}
%
\emptytransbox{2}{(0.8)}
%
\begin{mdframednoverticalspace}[style=firstfoc]
\begin{transbox}{3}{eku}
\begin{verbatim}
su   ibubù  si  kà  afifiɛ         ngbe.
take broom  LNK IMM you:PLUR~sweep here
\end{verbatim}
take a broom and sweep here
\end{transbox}
\end{mdframednoverticalspace}
%
\xtransbox{4}{afu}{((gets up to take broom))}\\

\normalsize
In these and other examples, there is a complex recruiting turn specifying more than one action, where the first action (usually formatted as an imperative) appears to be a first step for later actions, and the later actions are introduced in a \textit{si}-prefaced subordinate clause. In this context, \textit{si} can often be translated as ‘then’, ‘so that’, or ‘in order to’ (\tabref{tab:dingemanse:4}). Sometimes the \textit{si} clause refers to a component of the recruited activity (e.g. ‘come take the child’, ‘sweep here’), while in other cases it need not be done by the recipient (e.g. ‘so that I can wash my hands’, ‘so that \{he\} be dressed’). What unites all cases is that \textit{si} marks a consecutive relation in which one action follows another, the first often addressing a precondition for the one introduced by \textit{si}.

\begin{table}
\small %\footnotesize
\begin{tabularx}{\textwidth}{XXr}
\lsptoprule
First step (imperative) & Consecutive action (\textit{Si-}prefaced) & Example\\
\midrule
‘wash your hands’ & ‘come take the child’ & \REF{ex:dingemanse:2}\\
‘get up’ & ‘\{continue to\} do it here’ & \REF{ex:dingemanse:14}\\
‘take a broom’ & ‘sweep here’ & \REF{ex:dingemanse:15}\\
‘come pour me water’ & ‘\{so\} I can wash my hands’ & \REF{ex:dingemanse:21}\\
\lspbottomrule
\end{tabularx}

\caption{First steps and consecutive actions in multi-part recruiting turns.}
\label{tab:dingemanse:4}
\end{table}

A case discussed earlier, reproduced in part as \REF{ex:dingemanse:16} below, provides a closer look at the relation between different stages of recruitment and the design of \textit{si}-prefaced recruiting formats. Eku first launches a bare imperative format, then self-repairs and turns it into a question about a necessary precondition: ‘take uh: is there water there?’. The self-repair reveals an orientation to the conditions necessary for fulfilling the recruitment. Once it is clear that this condition is met, she goes on to formulate a recruiting turn combining an imperative and a \textit{si}-prefaced target action (line 3). That turns out to be only the first in a series of actions requested of Kpɛi, all introduced by \textit{si}-prefaced clauses: 
‘\textit{si} you take this gallon’, ‘\textit{si} you pour some water’, ‘\textit{si} you put it on the fire’. This supports the analysis of \textit{si} as encoding consecutive actions. The consecutive action verbs all have irrealis mood and so can be described collectively as linked by a form of co-subordination, a situation similar to the linker \textit{né} in Ewe \citep{ameka_aspect_2008}.\footnote{Homophonous with \textit{si} ‘so that’ as a marker of consecutive action is a \textit{si} ‘if’ form that introduces conditional antecedents. It is possible that the two are related, which would render \textit{si} heterosemous and would make the \textit{si}-prefaced format akin to independent if-clauses \citep{ford_conditionals_1986}, which have been found in many languages to develop into a dedicated request format \citep{evans_insubordination_2007,lindstrom_interactional_2016}. However, many of the \textit{si}-prefaced recruiting turns do not lend themselves to a conditional reading; indeed, they tend to be closer to the consequent (‘then’) than to the antecedent of a conditional.}

\transheader{ex:dingemanse:16}{Maize3\_276559a (excerpted from \extref{ex:dingemanse:5})}\vspace{-1mm}
%
\begin{mdframednoverticalspace}[style=firstfoc]
\begin{transbox}{1}{eku}
\begin{verbatim}
su   ɛ:. ndu   pia mmɔ: ((points in direction of water tank))
take HES water be  there:Q
\end{verbatim}
take uh:. is there water there?
\end{transbox}
\end{mdframednoverticalspace}
%
\begin{transbox}{2}{kpɛ}
\begin{verbatim}
mm.
INTJ
\end{verbatim}
mm.
\end{transbox}
%
\begin{transbox}{3}{eku}
\begin{verbatim}
su   fore si  àsu      ɛh  gálɔn  gangbe ((points to gallon))
take pour LNK 2SG:take HES gallon AGR:this
\end{verbatim}
pour it \{and\} take uh this gallon
\end{transbox}\bigskip

\normalsize
The consequential or consecutive reading of \textit{si} opens up the possibility for \textit{si}-prefaced clauses to be used in providing reasons for recruitment. An example is given in \REF{ex:dingemanse:17}. Mom calls on Sesi, her teenage son, to bring her a ‘knife and uh tub’  (line 1). When, moments later, Sesi arrives with only a knife, she repeats the request for a tub, now adding a \textit{si}-prefaced reason: ‘so I \{can\} peel the cassava’ (line 4). Peeling the cassava is an activity for which one needs a knife and a container. By mentioning this activity and marking it as a consecutive action, Mom renews the relevance of getting the tub and adds weight to her repeated request.

\transheader{ex:dingemanse:17}{Neighbours\_662742}\vspace{2mm}
%
\begin{transbox}{1}{mom}
\begin{verbatim}
Sesi bɔ    mɛ ipɛmi ku  ɛɛ  kàpoi   anɔ:?
PSN  bring me knife and HES tub:DIM you:hear:Q
\end{verbatim}
Sesi bring me a knife and uh tub y’hear?
\end{transbox}
%
\emptytransbox{2}{(14.0)}
%
\xtransbox{3}{ses}{((arrives with knife))}
%
\begin{mdframednoverticalspace}[style=firstfoc]
\begin{transbox}{4}{mom}
\begin{verbatim}
hɛ   bɔ    mɛ kàpoi,  si   lòyɛrɛ  igbedi. ((receives knife))
INTJ bring me tub:DIM LNK 1sg:peel cassava
\end{verbatim}
hey bring me a tub so I \{can\} peel the cassava
\end{transbox}
\end{mdframednoverticalspace}
%
\emptytransbox{5}{(0.8)}
%
\begin{transbox}{6}{mom}
\begin{verbatim}
bɔ    mɛ kàpoi
bring me tub:DIM
\end{verbatim}
bring me a tub
\end{transbox}
%
\xtransbox{7}{ses}{((goes off to get tub))}
%
\emptytransbox{8}{(23.0)}\vspace{-1mm}
%
\xtransbox{9}{ses}{((arrives with tub))}\\

\normalsize
Because of their consecutive meaning, \textit{si}-prefaces can be used to present “in-order-to motives” \citep{schutz_common-sense_1962} in interaction. In such cases, the \textit{si}-prefaced clause is the motive for which the recruitment is a means, as here for Mom’s request to be brought a tub so she can peel the cassava.

Another type of reason that people may use in recruitment sequences refers to “because-motives” \citep{schutz_common-sense_1962}. These are not marked with \textit{si} but presented as declarative statements. We saw both types together in \REF{ex:dingemanse:2}, where Beatrice was told ‘wash your hands \textit{si} you come take him’ (an \textit{in-order-to} motive), ‘cause he’s done sitting’ (a \textit{because}-motive).

Reasons are provided in 26 out of 207 initial and subsequent recruiting moves. Most commonly, they occur in pursuits of response when there was a problem in uptake, as we saw in  \REF{ex:dingemanse:2} and  \REF{ex:dingemanse:17}. In the relatively rarer cases when they occur in first position, they may be designed to help disambiguate a request \REF{ex:dingemanse:21} or to anticipate a question about rights and duties that might otherwise come up. These functions of reason-giving, which can be summarized as rendering recruitments more intelligible and making fulfillment more likely, correspond closely to those found in a dedicated study of a collection of 56 recruitment sequences featuring reasons in Russian (\citealt{BaranovaDingemanse2016}; see also Baranova, \chapref{sec:baranova}, \sectref{sec:baranova:3.4.2}).

\subsubsection{Mitigating and strengthening recruiting moves}\label{sec:dingemanse:3.3.3}

In their seminal work on the structure of therapeutic interaction, \citet{LabovFanshel1977} noted that some linguistic devices appeared to soften requests (“mitigators”) while others may serve to strengthen them (“aggravators”). Conver\-sa\-tion-analytic work since then has showed that such devices can be understood with reference to the sequential structure of interaction \citep{heritage_garfinkel_1984,schegloff_sequence_2007}. We have already seen some of the strategies for upgrading the strength of subsequent versions of recruiting moves, for instance by adding a marker of resaying or by providing a reason.

Like many West-African languages, Siwu has a system of final particles, two of which are of particular interest with regard to the question of how people can modulate the force of recruiting moves. The final particle \textit{ló} conveys ‘I advise you’, implying no claim about prior knowledge. The form \textit{ní} conveys ‘you should have already understood’, implying a claim about prior knowledge and a complaint that this has not been acted upon. The two forms are never found together in the same utterance in the corpus, and seem to occur in complementary sequential positions.

We saw a case of \textit{ló} in \REF{ex:dingemanse:7}, where Vicky noticed a chicken behind Ella and told her about it so she could take action. One affordance of  \textit{ló} is its “no fault” quality \citep[271]{heritage_garfinkel_1984}. Its usage does not imply prior knowledge and so it does not blame the other for failing to know or notice something. This is why it can also serve as a gentle nudge that makes a recruiting move sound more affiliative. In terms of sequential position, it tends to occur in initial but not in subsequent versions of recruiting moves, as seen in the following case. 

Emilia is preparing porridge in the kitchen while Aku is sitting outside, a few meters away, with her back to Emilia. After a lapse in the conversation \citep[see][]{hoey_lapses:_2015}, Emilia calls on Aku to bring her bowl, with the implication that she can get some food. The recruiting move contains \textit{ló}, marking it as advice and perhaps orienting to the possibility that Aku, sitting outside, may not be aware that food is ready to be served. When Aku does not respond immediately, Emilia pursues a response by first calling her, then repeating the recruiting move, this time without \textit{ló} (line 6).

\transheader{ex:dingemanse:18}{Cooking1\_521410}\vspace{2mm}
%
\emptytransbox{1}{(7.0)}
%
\begin{mdframednoverticalspace}[style=firstfoc]
\begin{transbox}{2}{emi}
\begin{verbatim}
Aku bɔ    mɛ fɔ   irɔi ló
PSN bring me your bowl FP.advice
\end{verbatim}
Aku bring your bowl \textit{ló}
\end{transbox}
\end{mdframednoverticalspace}
%
\emptytransbox{3}{(1.4)}
%
\begin{transbox}{4}{emi}
\begin{verbatim}
Aku
PSN
\end{verbatim}
Aku
\end{transbox}\vspace{1mm}
%
\begin{transbox}{5}{aku}
\begin{verbatim}
mm.
CONT
\end{verbatim}
mm
\end{transbox}\vspace{1mm}
%
\begin{mdframednoverticalspace}[style=firstfoc]
\begin{transbox}{6}{emi}
\begin{verbatim}
bɔ    mɛ fɔ   irɔi
bring me your bowl
\end{verbatim}
bring me your bowl
\end{transbox}
\end{mdframednoverticalspace}
%
\begin{mdframednoverticalspace}[style=secondfoc]
\xtransbox{7}{aku}{((gets up))}
\end{mdframednoverticalspace}

\normalsize
The final particle \textit{ní} is almost a mirror image of \textit{ló}. It rarely occurs in the initiating turn of a recruitment sequence and instead appears in subsequent versions that pursue a response. In  \REF{ex:dingemanse:19}, Emma is shuffling across the compound heading towards an overturned bench which she cannot see (this happens moments after \extref{ex:dingemanse:4}, where Aku stood up from the doorway to let her through). Aku instructs Emma to ‘pass here’. When Emma does not appear to be listening and instead places her cane on the overturned bench, Aku pursues response by saying ‘pass here \textit{ní}’, the \textit{ní} particle marking the recruitment as something that should have been understood and acted on already.\footnote{Similar strengthening uses of \textit{ní} are found in a sequence analyzed in \REF{ex:dingemanse:28} and  \REF{ex:dingemanse:29} below, where a mother attempts to get her teenage son to run an errand.}

\transheader{ex:dingemanse:19}{Compound4\_2076833}\vspace{2mm}
%
\xtransbox{1}{emm}{((blind, walking with cane, is about to stumble over overturned bench))}
%
\begin{mdframednoverticalspace}[style=firstfoc]
\begin{transbox}{2}{aku}
\begin{verbatim}
ki   ngbe. ((pulls Emma’s arm))
pass here.
\end{verbatim}
pass here
\end{transbox}
\end{mdframednoverticalspace}
%
\xtransbox{3}{emm}{((places cane on turned-over bench))}
%
\begin{mdframednoverticalspace}[style=firstfoc]
\begin{transbox}{4}{aku}
\begin{verbatim}
ki   ngbe ní ((pulls Emma’s and leads her around the bench))
pass here FP
\end{verbatim}
pass here \textit{ní}
\end{transbox}
\end{mdframednoverticalspace}
%
\xtransbox{5}{emm}{((lets herself be led by Aku))}\\

\normalsize
In sum, the final particles \textit{ló} and \textit{ní} help to manage accountability by making claims about the recipient’s knowledge (or lack of knowledge) about what they should be doing. \textit{Ló} can be seen as a general dispensation, conveying ‘I advise you’ without implying a complaint; \textit{ní} conveys the reverse: ‘you should have known this and acted on it already’, and therefore holds the recruitee accountable for the failure to respond. These usages are in line with the use of the particles in non-recruitment contexts, where they have similar implications.

Another device that can be used to strengthen recruiting moves is \textit{anɔ:} ‘you hear’, illustrated in \REF{ex:dingemanse:20}, where a little boy is making tottering steps around three women: Aku, Charlotte, and Emma. Aku produces a request: ‘let him sit by your side’. Then Charlotte adds ‘his mother is winnowing rice’, accounting for the unavailability of the primary caregiver. Although neither the request nor the reason for it are clearly addressed, the fact that two of three participants present have jointly formulated a request plus reason makes a response relevant by the third participant, Emma. When no response follows, Aku upgrades her request by specifying the action and adding a strengthening particle \textit{anɔ:} ‘y’hear?’ (line 8).\footnote{In \REF{ex:dingemanse:30}, which continues this extract, the particle is repeated on its own in a further pursuit of response.}

\transheader{ex:dingemanse:20}{Maize3\_673020}\vspace{2mm} %\emptytransbox{~}{((a little boy is hanging around while two women are sitting down and a third, Aku, is about to go on an errand))}
%
\begin{transbox}{4}{aku}
\begin{verbatim}
tã  ũ   ɔsɛ     i   fɔ  kɔrɛ.
BEN him NOM:sit LOC 2SG side
\end{verbatim}
 let him sit by your side
\end{transbox}
%
\emptytransbox{5}{(0.5)}
%
\begin{transbox}{6}{cha}
\begin{verbatim}
ɔ̃        ɔnyĩ   tó   ɔ   fɛ     kàmɔ.
3SG.POSS mother PROG 3SG winnow rice
\end{verbatim}
his mother is winnowing rice
\end{transbox}
%
\emptytransbox{7}{(0.8)}
%
\begin{mdframednoverticalspace}[style=firstfoc]
\begin{transbox}{8}{aku}
\begin{verbatim}
puta ũ   (.) anɔ:?
lift him     2SG:hear:Q
\end{verbatim}
 pick him up (.) you hear?
\end{transbox}
\end{mdframednoverticalspace}
%
\emptytransbox{9}{(0.8)}\vspace{-1.5mm}
%
\emptytransbox{~}{((continues in \extref{ex:dingemanse:30} below))}\\

\normalsize
\textit{Anɔ:} ‘y’hear?’ is a tag question with affirmation as the preferred response.\footnote{Another instance can be found in \REF{ex:dingemanse:17}, line 1, above.} Adding it to a recruiting move has the effect of soliciting a commitment to fulfill the recruitment: after all, admitting to hearing a request makes it harder to escape the normative requirement to comply with it.

\subsection{Fully nonverbal recruiting moves}\label{sec:dingemanse:3.4}

So far we have reviewed a range of linguistic, verbal resources for building recruiting moves. Only 23 independent recruiting moves in the corpus are fully nonverbal. These can be arranged according to the degree to which they are presented and treated as on-record. An off-record nonverbal recruiting move was illustrated in \REF{ex:dingemanse:4} above, where some imminent trouble on the part of one participant provides a reason for another participant to help out. In such cases, the trouble does not make a response conditionally relevant \citep{Schegloff1968}: participant A cannot be said to have asked anything, and participant B cannot be held accountable for inaction. On-record nonverbal recruiting moves are rare in the Siwu data (3 independent sequences, 9 moves in total), and only seem to happen when the recruitment occurs as part of an already established activity sequence which can provide the context for their interpretation (\citealt{Rossi2014} and \chapref{sec:rossi}, \sectref{sec:rossi:3.1}; see also Kendrick, \chapref{sec:kendrick}, \sectref{sec:kendrick:4.1.3}; Zinken, \chapref{sec:zinken}, \sectref{sec:zinken:3.1} Baranova, \chapref{sec:baranova}, \sectref{sec:baranova:3.1}).

One situation where we find such nonverbal recruiting moves is when a prior request has made relevant the execution of a related subtask. In  \REF{ex:dingemanse:21}, an extended recruitment sequence is initiated when Atasi tells Eku to ‘get some water so I can wash my hands’. The \textit{si}-prefaced reason here (see \sectref{sec:dingemanse:3.3.2}) helps to disambiguate and specify the request: one might need water for any of a number of purposes, with consequences for the quantity desired and the container to be used -- in \REF{ex:dingemanse:5} above, for example, a gallon of water is needed for cooking, and in \REF{ex:dingemanse:9} an even larger quantity is needed for taking a bath. With the request and its reason made clear, Eku’s standing up (line 2) marks a commitment to provide this service, and her return with a calabash with water (some 20 seconds later) marks the start of compliance. Now a series of nonverbal actions ensues in which Atasi holds out her hands and Eku pours some water in response (lines 18--22), a process that is repeated five more times until the task is completed.

\transheader{ex:dingemanse:21}{Compound5\_846793}\vspace{2mm}
%
\begin{transbox}{1}{ata}
\begin{verbatim}
ba   fore mɛ ndu   sí  lòfore   kɔ̃rɔ̃
come pour me water LNK 1SG:pour hand
\end{verbatim}
come pour me water so I can wash my hands
\end{transbox}
%
\xtransbox{2}{eku}{((stands up to fetch water))}
%
\emptytransbox{~}{((20 seconds pass, during which an unrelated story is told by a third party, after which Eku returns with a calabash of water and Eku and Atasi stand together))}\vspace{2mm}
%
\begin{mdframednoverticalspace}[style=firstfoc]
\xtransbox{18}{ata}{((holds out hands and assumes 'washing hands' position))}
\end{mdframednoverticalspace}
%
\begin{mdframednoverticalspace}[style=secondfoc]
\xtransbox{19}{eku}{((pours water over A's hands))}
\end{mdframednoverticalspace}
%
\emptytransbox{20}{(2.3)}
%
\begin{mdframednoverticalspace}[style=firstfoc]
\xtransbox{21}{ata}{((opens hands palms up for more water))}
\end{mdframednoverticalspace}
%
\begin{mdframednoverticalspace}[style=secondfoc]
\xtransbox{22}{eku}{((pours more water))}
\end{mdframednoverticalspace}
%
\emptytransbox{~}{((actions in 21-22 repeated five times))}\vspace{-1mm}
%
\xtransbox{33}{ata}{((shakes water off her hands, walks back to seat))}\\

\normalsize
Cases like \REF{ex:dingemanse:21} show that recruitments can assume a fractal nature, where an initiation and its response can set up a context for a number of subsidiary sequences. To the extent that such subsidiary sequences occur in the context provided by the base sequence and are part of a default script associated with the base activity, they are often implemented nonverbally.

A recruitment episode with subsidiary sequences as in \REF{ex:dingemanse:21} raises the question of how we can distinguish between a series of recruitments versus a sequence of behaviors done in the service of one instance of recruitment (see also \chapref{sec:coding}, \sectref{sec:coding:3}). The most reductive approach would be to stipulate that only base sequences count as recruitments. So, in \REF{ex:dingemanse:21}, there would be a single Move A (‘come pour me water so I can wash my hands’, line 1) and its fulfillment would be the full sequence of moves implementing that complex action, starting when Eku stands up to get the water (line 2) and ending when Atasi shakes the water off her hands (line 33). However, this analysis would fail to capture the contingent nature of Atasi’s repeated nonverbal requests for more water (lines 21--22ff). The number of times water has to be poured is not preestablished and is under Atasi’s control, while for the pouring of the water she fully depends on Eku. Therefore, Atasi’s opening up her hands palm up is analyzed here as a Move A in its own right and Eku’s pouring more water as a corresponding Move B, and a series of such moves in quick succession expands the base adjacency pair.

Another example of a fully nonverbal recruiting move is given in \REF{ex:dingemanse:22}. Bella is holding Aku’s phone and taking a call Aku asked her to pick up. Speaking into the phone, she notes she is ‘not sister Aku’. When it becomes clear the caller wants Aku, Aku asks Bella to give the phone back (line 2). After a place in which a response would have been relevant (line 3), she asks again, now with an added gesture of reaching out to receive the phone (line 4). When Bella continues to speak on the phone, Aku produces one more response pursuit, this time fully nonverbal (line 6), after which she is handed back the phone.

\newpage
\transheader{ex:dingemanse:22}{Neighbours\_818304}\vspace{2mm}
%
\begin{transbox}{1}{bel}
\begin{verbatim}
mɛ  nyɛ sistà  Aku oo  ló.
NEG COP sister PSN NEG FP
\end{verbatim}
I’m not sister Aku \textit{ló}
\end{transbox}
%
\begin{transbox}{2}{aku}
\begin{verbatim}
su   tã   mɛ.
take give me
\end{verbatim}
give me back
\end{transbox}
%
\emptytransbox{3}{(0.8)}
%
\begin{transbox}{4}{aku}
\begin{verbatim}
su   tã   mɛ ((reaches out for phone))
take give me
\end{verbatim}
give me back
\end{transbox}
%
\begin{transbox}{5}{bel}
\begin{verbatim}
èvìa      ye. ((turns towards Aku))
child:DEF FOC
\end{verbatim}
her child
\end{transbox}
%
\begin{mdframednoverticalspace}[style=firstfoc]
\xtransbox{6}{aku}{((extends hand further and makes grasping gesture, Fig. 4))}
\end{mdframednoverticalspace}
%
\begin{mdframednoverticalspace}[style=secondfoc]
\xtransbox{7}{bel}{((hands over phone))}
\end{mdframednoverticalspace}
%
\begin{transbox}{8}{aku}
\begin{verbatim}
hɛlo  mɛka      ye?
hello person:CQ FOC
\end{verbatim}
hello, who is this?
\end{transbox}\bigskip

\begin{figure}
\centering
\includegraphics[width=\textwidth]{figures/siwu-img8.jpg}
\caption{Aku reaches to receive the phone in an upgraded response pursuit (line 6).}
\label{fig:dingemanse:4}
\end{figure}

\normalsize
Like the subsidiary recruitments in \REF{ex:dingemanse:21}, the response pursuit in \REF{ex:dingemanse:22} occurs in an environment where it is already abundantly clear what needs to be done and by whom. So both cases fit the generalization that fully nonverbal requests tend to occur only when the activity structure, participation framework and prior context render verbal specification unnecessary \citep{Rossi2014}.

\subsection{Animal-oriented recruitments}

Recruitments are defined in this study as interactional sequences with human participants, in line with a focus of the larger research project on human sociality. However, people also have interactional practices oriented towards animals \citep{bynon_domestic_1976,spottiswoode_reciprocal_2016}. Indeed, humans are not alone in producing communicative signals aimed at other species \citep{krebs_animal_1984}. Animal-oriented recruitments provide an interesting limiting case of how semiotic resources adapt to situations in which there are radical asymmetries in agen\-cy and linguistic capability between interactants.

In Siwu, as in many other languages, animal-oriented recruitments often involve a set of dedicated interjections \citep{ameka_interjections:_1992}. Two examples occurred in extracts discussed earlier, relevant portions of which are reproduced below. In \REF{ex:dingemanse:23}, Tawiya’s interjection \textit{kai} can be said to effectively recruit the goats to go away, and in  \REF{ex:dingemanse:24}, the interjection \textit{shuɛ} has a similar effect on the chicken.\footnote{Conversation analysis shies away from attributing intentions to participants in interaction, aiming instead to base analyses on publicly observable sequences of behavior \citep{heritage_intention_1990}. This methodological stance renders CA suitable for analyzing at least some forms of non-human animal communication \citep{rossano_sequence_2013}.}

\transheader{ex:dingemanse:23}{Cooking1\_1545188 (excerpted from \extref{ex:dingemanse:6} above)}\vspace{2mm}
%
\begin{transbox}{5}{taw}
\begin{verbatim}
kai, [↑kai ((waves arm))
\end{verbatim}
\end{transbox}
%
\begin{transbox}{6}{adz}
\begin{verbatim}
     [hm, hm, ↑hm, hm↑ ((waves arm))
\end{verbatim}
\end{transbox}
%
\emptytransbox{7}{((goats flee the scene))}

\transheader{ex:dingemanse:24}{Compound4\_1600030 (excerpted from \extref{ex:dingemanse:7} above)}\vspace{2mm}
%
\begin{transbox}{4}{taw}
\begin{verbatim}
↑shuɛ:↑ ((moves to chase away chicken))
INTJ
\end{verbatim}
shoo!
\end{transbox}
%
\emptytransbox{5}{((chicken walks away))}\\

\normalsize
The shape of at least some of these animal-oriented interjections appears not to be arbitrary but motivated. Take \textit{shuɛ} ‘shoo’, the interjection for chasing away domestic fowls. A survey of functional equivalents reported for other languages from around the world shows that shooing words seem to converge on sibilant sounds, variously transcribed as \textit{s}, \textit{ʃ}, \textit{š}, \textit{ç} (\tabref{tab:dingemanse:5}).\footnote{Most of the sources cited do not give phonetic renditions, so forms are presented here without adjustments. The table presents a sample of typologically diverse languages selected by searching grammars and dictionaries for forms translated as ‘shooing/chasing away chicken/fowl’.}

\begin{table}
\small
\begin{tabularx}{\textwidth}{llllr}
\lsptoprule
Language & Phylum & ‘shoo’ & ‘chicken’ & Source\\
\midrule
Chaha Gurage & Afro-Asiatic & \textit{(ə)ʃʃ} & \textit{kutara} & \citealt{leslau_etymological_1979}\\
Tamazight & Afro-Asiatic & \textit{hušš} & \textit{afulus} & \citealt{bynon_domestic_1976}\\
Semelai & Austroasiatic & \textit{cuh} & \textit{hayam} & \citealt{kruspe_grammar_2004}\\
Kambera & Austronesian & \textit{hua} & \textit{manu} & \citealt{klamer_grammar_1998}\\
Muna & Austronesian & \textit{sio} & \textit{manu} & \citealt{berg_grammar_1989}\\
West Coast Bajau & Austronesian & \textit{si’} & \textit{manuk} & \citealt{miller_grammar_2007}\\
English & Indo-European & \textit{shoo} & \textit{chicken} & Oxford Dictionary\\
Louisiana French & Indo-European & \textit{ʃuʃ} & \textit{poule} & \citealt{valdman_dictionary_2009}\\
Russian & Indo-European & \textit{kš-k} & \textit{kuritsa} & \citealt{liston_defining_1971}\\
Japanese & Japonic & \textit{shi} & \textit{niwatori} & \citealt{bolton_language_1897}\\
Siwu & Niger-Congo & \textit{shuɛ} & \textit{kɔkɔ} & current study\\
Ewe & Niger-Congo & \textit{suí} & \textit{koklo} & \citealt{ameka_ewe_1991}\\
Zargulla & Omotic & \textit{čúk} & \textit{kútto} & \citealt{amha_directives_2013}\\
Kashaya & Pomoan & \textit{ša} & \textit{kayi:na} & \citealt{oswalt_interjections_2002}\\
Atong & Sino-Tibetan & \textit{sa} & \textit{tawʔ} & \citealt{breugel_grammar_2014}\\
Lahu & Sino-Tibetan & \textit{š} & \textit{á-gâʔ} & \citealt{matisoff_dictionary_1988}\\
Lao & Tai-Kadai & \textit{sóò}, \textit{ʃ:} & \textit{kaj1} & \citealt{enfield_grammar_2007}\\
\lspbottomrule
\end{tabularx}
\caption{Interjections for ‘shoo’ and words for ‘chicken’ in 17 languages from 11 phyla around the world, showing strong convergence towards sibilant sounds in the interjections but not in the words for ‘chicken’.}
\label{tab:dingemanse:5}
\end{table}

Sibilant sounds show up in shooing words in a diverse sample of languages, many of which are not historically related. Some of the commonalities may be due to language contact. After all, the domestic fowl (\textit{Gallus g. domesticus}) has itself been culturally dispersed \citep{liu_multiple_2006} and some words may have traveled along. However, it is unlikely that the global similarities can be explained solely by cultural diffusion, as this would predict words for ‘chicken’ to show similar global commonalities, which they do not (\tabref{tab:dingemanse:5}). Nor can the global similarities be explained solely by inheritance from a common ancestor, as this would require a temporal stability that even basic vocabulary is not known for; and again, words for ‘chicken’ do not show such global similarities. A more parsimonious explanation is that some sounds are more effective than others for the goal of shooing birds, and come to function as cultural attractors biasing the transmission of shooing words -- a form of convergent cultural evolution.

Convergent cultural evolution has been put forward as an explanation for a range of cross-linguistic similarities \citep{caldwell_convergent_2008,dingemanse_is_2013,blythe_genesis_2018}. Animal-oriented interjections present a particularly illuminating view of the phenomenon, as the evolutionary landscape to which such words must adapt is strongly constrained by the perceptual and behavioral systems of the animals in question. The effectiveness of prolonged sibilants in shooing words for domestic fowls can be connected to the fact that continuous high frequency sounds are among the sound stimuli domestic fowls are most aversive to \citep{MacKenzieEtAlI}.

Owing to the narrow ranges of behavior they seek to elicit, animal-oriented signals may present one of the few areas of language that can be truthfully said to bring behavior under the control of some stimulus, as \cite{skinner_verbal_1957} envisioned. The principle of semiotic adaption to perceptual systems is likely to hold across a wide range of animal-oriented communicative signals across languages.\footnote{In an ethnological study of domestic fowls, \cite{fischer_sound_1972} shows that sound stimuli featuring repeated low-frequency sounds are most likely to induce following. This generates the prediction that, across languages, words for calling domestic fowls will feature more repetition and lower-frequency sounds than words for shooing them.}

\section{Formats in Move B: The responding move}

So far, we have considered the design of Move A, the move by which a recruitment is initiated or pursued. But a recruitment sequence is not complete without a Move B. In what follows, we consider the design of Move B and the further development of the sequence, from simple closure in the case of fulfillment to sequence expansion in the wake of resistance and rejection.

\subsection{Nonverbal and verbal elements of responses}\label{sec:dingemanse:4.1}

Since recruitments by definition involve getting another to perform a practical action, many relevant responses are nonverbal and simply consist of the doing of the target action. Examples of this are shown in Figures \ref{fig:dingemanse:1}\textit{b}, \ref{fig:dingemanse:2}\textit{b}, and \ref{fig:dingemanse:6}\textit{b}, and further examples are transcribed in Extracts \ref{ex:dingemanse:3},  \ref{ex:dingemanse:4},  \ref{ex:dingemanse:15},  \ref{ex:dingemanse:17},  \ref{ex:dingemanse:18},  \ref{ex:dingemanse:19},  \ref{ex:dingemanse:21}, and  \ref{ex:dingemanse:22}. About two thirds of responses to initial recruiting moves are fully nonverbal, and the great majority of these fulfill the target action or plausibly start doing so.

Although the focus here is on the composition of Move B, an important factor in its design is the format used in Move A, which initiates the recruitment sequence. Consider the relative frequency of fully nonverbal responses. \tabref{tab:dingemanse:X} gives the proportion of fully nonverbal Moves B relative to the format of Move A. It shows that nonverbal Moves A are followed by a fully nonverbal Move B in 77\% of cases; the remaining 23\% is either composite or verbal only. On the other hand, responses to interrogative recruiting moves are fully nonverbal in only 17\% of cases.\footnote{See Rossi, \chapref{sec:rossi}, \sectref{sec:rossi:4.1} for comparable distributions of fully nonverbal responses in Italian.}

\begin{table}
\caption{Proportion of nonverbal Moves B relative to the format of Move A in Siwu recruitment sequences}
\begin{tabular}{lll}
\lsptoprule
Move A & \multicolumn{2}{l}{What proportion of Moves B is nonverbal?} \\
\midrule
%texttt used to show the U+2588 Full Block in monospace font; \char"2000 used for 10 spaces on first row so that approx width of table corresponds to what would be 100%. Numbers of blocks proportional to proportions (32, 29, 21, 17, 7).
nonverbal     &
\nvbar{77}
% 77\%                              &
% \texttt{████████████████████████████████\char"2000\char"2000\char"2000\char"2000\char"2000\char"2000\char"2000\char"2000\char"2000\char"2000}  
\\
imperative    & 
\nvbar{69}
% 69\%                              & \texttt{█████████████████████████████}              
\\
declarative   &
\nvbar{50}
% 50\%                              & \texttt{█████████████████████}                       
\\
\textit{si}-prefaced   &
\nvbar{40}
% 40\%                              & \texttt{█████████████████}                  
\\
interrogative &
\nvbar{17}
% 17\%                              & \texttt{█████████}                                   
\\
\lspbottomrule
\end{tabular}
\label{tab:dingemanse:X}
\end{table}

% the command \barplot has 4 arguments:
% #1: Label X axis (Initiating move)
% #2: Label Y axis (\#)
% #3: names of the bars (Service, ...)
% #4: list of values for each bar. ((Service, 118), (), ()...)
% For better readability, the fourth argument is split over several lines

% \begin{figure}
% \caption{Proportion of nonverbal Moves B relative to the format of Move A in Siwu recruitment sequences.}
% \label{fig:dingemanse:5}
% \barplot{\textsc{move a format}}{\textsc{nonverbal move b (\%)}}{Nonverbal, Imperative, Declarative, Si-prefaced, Interrogative}{
%  (Nonverbal,77)
%   (Imperative,69)
%   (Declarative,50)
%   (Si-prefaced,40)
%   (Interrogative,17)
% }
% \end{figure}

Recruiting formats in Move A can be ranked on a cline from more to less coercive \citep{brown_universals_1978}. One way to explain this cline is in terms of the “response space” created by the formats \citep[see][]{VinkhuyzenSzymanski2005,RossiZinken2016}. As we saw above, nonverbal recruiting moves occur only in situations where the context makes abundantly clear what is requested, which places considerable constraints on the response space and makes relevant immediate (and nonverbal) fulfillment. Imperatives similarly push fairly directly for fulfillment and leave little room for other types of responses \citep[see, e.g.,][]{kent_compliance_2012,Rossi2012}. At the other end of the spectrum, interrogative recruiting turns in Siwu tend to be negative interrogatives like ‘why don’t you’, which formulate things either as complaints or proposals, both of which allow verbal or composite responses and push less directly for fulfillment.

One of the main uses of verbal material in the responding move is to signal commitment to fulfilling the recruitment. We see this in  \REF{ex:dingemanse:25}. Becca, seated on a low bench, is winnowing rice; Ama, who is trying on a new dress, comes standing with her back to Becca and says ‘fix me’ (\figref{fig:dingemanse:6}\textit{a}). Becca immediately responds ‘now, I’m coming’, takes a second to put down the rice winnower and stands up. Then she carries out the requested action, zipping up Ama’s dress (\figref{fig:dingemanse:6}\textit{b}).

\transheader{ex:dingemanse:25}{Tailor\_995460}\vspace{2mm}
%
\emptytransbox{1}{(3.0) ((Ama walks towards Becca))}
%
\begin{mdframednoverticalspace}[style=firstfoc]
\begin{transbox}{2}{ama}
\begin{verbatim}
di  mɛ ((comes standing with back to Becca, Fig. 6a))
fix me
\end{verbatim}
fix me
\end{transbox}
\end{mdframednoverticalspace}
%
\begin{mdframednoverticalspace}[style=secondfoc]
\begin{transbox}{3}{bec}
\begin{verbatim}
kɔ̃rɔ̃ nɛ, ũto     lò ba   ló
now  TP  1S:PROG 1S come FP.advice
\end{verbatim}
now, I’m coming \textit{ló}
\end{transbox}
\end{mdframednoverticalspace}
%
\emptytransbox{4}{(0.7) ((Becca puts down rice winnower, stands up))}\vspace{-1mm}
%
\begin{mdframednoverticalspace}[style=secondfoc]
\xtransbox{5}{bec}{((zips up Ama’s dress, Fig. 5b))}
\end{mdframednoverticalspace}

\begin{figure}
\begin{tabularx}{\textwidth}{ll}
\centering
\includegraphics[width=.46\textwidth]{figures/siwu-img9.jpg} & \includegraphics[width=.46\textwidth]{figures/siwu-img10.jpg}
\end{tabularx}
\caption{(\textit{a}) Ama stands with her back to Becca (line 2); (\textit{b}) Becca zips up Ama’s dress (line 5).}
\label{fig:dingemanse:6}
\end{figure}

\normalsize
So verbal responses can claim a commitment to fulfilling a recruitment when something stands in the way of immediate fulfillment. At the moment the are produced, these are, of course, claims rather than demonstrations. We saw this in \REF{ex:dingemanse:2}, where Beatrice said ‘yes’ to a request while finishing another activity. She was subsequently held accountable for not stopping the other activity soon enough. So recruiters may hold the recruitee accountable when verbal claims become incongruent with visible actions.

Sometimes verbal elements of responses can address aspects of the design of a recruiting turn. For instance, in \REF{ex:dingemanse:8}, Vicky notified Ella that a pan was sliding off a firestone. Ella responded by righting the pan and by uttering a high-pitched response token \textit{↑mm↑}, marking Vicky’s noticing as something counter to expectation. Another example where the nonverbal element of the response fulfills the recruitment while a verbal element responds to the formulation of the recruiting move is given in \REF{ex:dingemanse:26} below. Odo, carrying a small metal pan holding some food that is possibly hot, walks towards a bench to sit down but finds Bella standing in his way. He issues a crude request to Bella to get out of the way, which she does, but not without voicing her disapproval at his formulation with the response token \textit{woo:}.

\transheader{ex:dingemanse:26}{Neighbours\_880320}\vspace{-1mm}
%
\begin{mdframednoverticalspace}[style=firstfoc]
\begin{transbox}{1}{odo}
\begin{verbatim}
rùi    bie  kakɔiɔ      sɛ  wãrã.
uproot find place:INDEF sit rest
\end{verbatim}
get out of the way and find somewhere \{else\} to relax
\end{transbox}
\end{mdframednoverticalspace}
%
\begin{mdframednoverticalspace}[style=secondfoc]
\begin{transbox}{2}{bel}
\begin{verbatim}
woo: ((steps aside to make way))
INTJ
\end{verbatim}
woo:
\end{transbox}
\end{mdframednoverticalspace}
%
\xtransbox{3}{odo}{((sits down on bench))}\\

\normalsize
A number of features of turn design conspire to make Odo’s recruiting move akin to an extreme case formulation and give it complaint-like qualities \citep{pomerantz_extreme_1986}. The verb \textit{rùi} literally means ‘uproot’; the indefinite marker \textit{ɔ} attached to \textit{kakɔi} ‘place’ works to suggest that Bella should be anywhere but where she is; and the construal of her current action as ‘relaxing’ implies that Bella, perhaps unlike Odo, has nothing to do. Bella’s interjection of disapproval \textit{woo:} appears to be addressed to these features.

In sum, we have seen that the bulk of complying responses to recruitments are nonverbal. Verbal elements of responses may vary in relation to the format of the recruiting move and may be used to (i) claim commitment when something stands in the way of immediate fulfillment and (ii) respond to action affordances of the design of the recruiting move. But a further, major role for verbal elements of responses to recruitments is in the domain of resistance and rejection, to which we now turn.

\subsection{Repair, resistance, and rejection}\label{sec:dingemanse:4.2}

Sometimes recruitments are not immediately fulfilled, but questioned, resisted, or even rejected. Resistance and rejection rarely come in the form of explicit claims of unwillingness. Rather, participants have a variety of ways to avoid immediate compliance \citep{kent_compliance_2012}, though none of them comes for free: as we will see, resistance and rejection (and more generally, dispreferred responses) tend to lead to interactional turbulence.

“Repair” refers to the practices people use to deal with problems in speaking, hearing, and understanding \citep{schegloff_preference_1977}. In \REF{ex:dingemanse:27}, Mom and Dad are preparing food with Sesi and some other family members close by. Following a joke, Dad produces extended laughter, in overlap with which Mom asks Sesi to get something, the request infused with a laughter particle. In response, Sesi initiates repair using ‘what?’ and Mom redoes the recruiting move, providing a more explicit formulation, after which Sesi complies.

\transheader{ex:dingemanse:27}{Neighbours\_4875900 \citep[234]{dingemanse_other-initiated_2015}}\vspace{2mm}
%
\begin{transbox}{1}{dad}
\begin{verbatim}
həh hɛ hɛ hɛ HA [HA HA HA HA HA HA
\end{verbatim}
\end{transbox}
%
\begin{mdframednoverticalspace}[style=firstfoc]
\begin{transbox}{2}{mom}
\begin{verbatim}
                [Sesi su   ɛ(h)ɛh iraɔ̀        tã  mɛ
                 PSN  take HES    thing:INDEF DAT me
\end{verbatim}
\hspace{2.5cm} Sesi take uhuh: the thingy for me
\end{transbox}
\end{mdframednoverticalspace}
%
\begin{transbox}{3}{ses}
\begin{verbatim}
be:
what:Q
\end{verbatim}
what
\end{transbox}
%
\begin{transbox}{4}{mom}
\begin{verbatim}
su   kadadìsɛ̃ĩbi   bɔ    mɛ.
take small.pot.DIM bring me
\end{verbatim}
take the small pot and bring it to me
\end{transbox}
%
\begin{mdframednoverticalspace}[style=secondfoc]
\xtransbox{5}{ses}{((complies by bringing small pot))}
\end{mdframednoverticalspace}

\normalsize
An other-initiation of repair starts a side sequence \citep{jefferson_side_1972}, signaling some trouble that first needs to be resolved before the base sequence can be resumed. A side effect is that the position where a response would be relevant is pushed back at least until the embedded side sequence is closed (in \extref{ex:dingemanse:27}, until after line 4). This makes repair initiation a powerful tool that can also be used for secondary purposes \citep{Sacks1992,schegloff_relevance_1979}. Earlier we saw how affirmative verbal responses may claim alignment with the goal of a recruitment, but may also hold off actual fulfillment. In a similar way, repair initiations claim communicative trouble but at the same time can be a device for protracting a sequence and delaying fulfillment (see also Blythe, \chapref{sec:blythe}, \sectref{sec:blythe:4.2.3}).

Consider \REF{ex:dingemanse:28}, where Sesi is asked to fetch a bag to go get a load of plantain from a household in a neighboring hamlet. Although Mom’s formulation is sufficiently vague to allow Sesi to choose a fitting bag himself (‘from inside this thing here’, line 2, a reference to a shed nearby), he initiates repair by asking ‘what d’you mean bag?’(line 3).\footnote{The dismissive connotation of the indefinite marker \textit{ɔ} in \textit{bagɔ} is hard to capture in translation. Possible alternatives are ‘whatever bag?’, ‘which bag?’, ‘what bag?’.} The other people present are quick to respond: Aunty taunts ‘you’ll just go with your bare hands?’ and Dad suggests ‘your school bag’, a suggestion which, after laughs all around, is elaborated by Aunty to reveal the absurdity of Sesi’s question (line 8). After this barrage of non-serious responses, Mom’s seemingly serious follow-up question (line 9) remains unanswered by Sesi.

\newpage
\transheader{ex:dingemanse:28}{Neighbours\_1131171}\vspace{-1mm}
%
\begin{mdframednoverticalspace}[style=firstfoc]
%
\begin{transbox}{1}{mom}
\begin{verbatim}
ba   su   ira   ní, ba-  ba   fe   àdi      ɛɛ-
come take thing FP  come come pass 2SG:take HES
\end{verbatim}
come get \{the\} thing, come come pass \{so\} you take uh
\end{transbox}
\end{mdframednoverticalspace}
%
\begin{transbox}{2}{~}
\begin{verbatim}
ɛ   bagì i   iraɔ        amɛ    mmɔ   ní.
HES bag  LOC thing:INDEF inside there FP
\end{verbatim}
uh a bag from inside this thing here
\end{transbox}
%
\begin{mdframednoverticalspace}[style=secondfoc]
\begin{transbox}{3}{ses}
\begin{verbatim}
mmɛ   bágɔ:
which bag:INDEF:Q
\end{verbatim}
what d’you mean bag?
\end{transbox}
\end{mdframednoverticalspace}
%
\emptytransbox{4}{(0.9)}
%
\begin{transbox}{5}{aun}
\begin{verbatim}
nɛ   nrɔ̃-nrɔ̃ 	  aàsɛ[:
CONJ hand~DIST  2SG:FUT:go:Q
\end{verbatim}
so you’ll just go with your bare hands?
\end{transbox}
%
\begin{transbox}{6}{dad}
\begin{verbatim}
                    [fɔ      skúl   bagì.
                     3SGPOSS school bag
\end{verbatim}
\hspace{3.1cm} your school bag
\end{transbox}
%
\emptytransbox{7}{((all laugh together))}
%
\begin{transbox}{8}{aun}
\begin{verbatim}
kɛlɛ adi        sìko  sɛ si  àsu.
go   2SG:remove books ?  LNK 2SG:take
\end{verbatim}
throw your books out and take it
\end{transbox}
%
\begin{transbox}{9}{mom}
\begin{verbatim}
bagì na   i   ɛɛ  ngbe gɔ  fɔ   ɔse    sia áwu     sa   mmɔ:
bag  lack LOC HES here REL your father put clothes farm there:Q
\end{verbatim}
there’s no bag uh where your dad puts his farming clothes?
\end{transbox}
%
\begin{transbox}{10}{bel}
\begin{verbatim}
shuɛ: (.) màkɔkɔ  maũ     ta   madaa        kutsùɛ ní.
INTJ      chicken they.TP PROG they:disturb ear    FP
\end{verbatim}
shoo: the chickens are disturbing
\end{transbox}
%
\emptytransbox{11}{(3.0)}\\

\normalsize
So here we have a recruiting turn followed by a repair initiation that not all parties to the conversation take entirely seriously as an indication of trouble. What the repair initiation \textit{is} taken as becomes clear later in the interaction, when half a minute has passed and there is still no sign of Sesi fulfilling the request. As \REF{ex:dingemanse:29} shows, Mom pursues a response, upgraded with \textit{mlàmlà} ‘quickly’ and a final particle \textit{ní} (line 38), implying, as we have seen in \sectref{sec:dingemanse:3.3.3}, that the recruiting move should have been attended to before. In the continued absence of a response, Aunty observes that ‘kids are difficult’ and Mom adds ‘kids are extremely difficult’ in a second-position upgrade that allows her to agree yet also assert her own epistemic access to the matter \citep{heritage_terms_2005}. The extract starts 27 turns or 35 seconds after line 10 in \REF{ex:dingemanse:28}.

\transheader{ex:dingemanse:29}{Neighbours\_1131171 (continues from \extref{ex:dingemanse:28})}\vspace{2mm}
%
\emptytransbox{37}{(2.4)}
%
\begin{mdframednoverticalspace}[style=firstfoc]
%
\begin{transbox}{38}{mom}
\begin{verbatim}
bɔ:    mlàmlà       ní.
bring  IDPH.quickly FP
\end{verbatim}
bring it quickly now!
\end{transbox}
\end{mdframednoverticalspace}
%
\emptytransbox{39}{(1.0)}
%
\begin{transbox}{40}{aun}
\begin{verbatim}
màbi     bɔle.
children have:force.
\end{verbatim}
kids are difficult
\end{transbox}
%
\begin{transbox}{41}{mom}
\begin{verbatim}
màbi      ba   ɔle   pápápápápápa
children  have force IDPH.extremely
\end{verbatim}
kids are extremely difficult
\end{transbox}\bigskip

\normalsize
Mom and Aunty’s statements that kids are ‘difficult’ treat Sesi’s troubles in this sequence as related to his teenager status rather than as a true problem in hearing or understanding. In fact, they seem to take Sesi to be exploiting repair in order to delay or even avoid fulfilling a recruitment -- a possibility that also puts his behavior in \REF{ex:dingemanse:17} and \REF{ex:dingemanse:27} in a new light.

Repair is not the only way to resist recruitment. Several other ways are illustrated in \REF{ex:dingemanse:30}, which continues from \REF{ex:dingemanse:20} above. Three women are chatting together. Aku and Charlotte have asked Emma to watch over a little boy for a moment while his mother is occupied with a task in a neighboring compound. At line 7, Emma ignores the initial recruiting move. Following a response pursuit by Aku, Emma then objects ‘I don’t know who’s picking him up’ (line 10), a crafty formulation that enables her to imply that she is unwilling to fulfill the recruitment without going on record as saying so. Aku formulates a high-pitched response pursuit ‘↑you hear?↑’, reasserting the relevance of a response to the request. Following this second pursuit, Emma produces a well-positioned yawn, hearable as a claim of tiredness and by implication inability (line 13). In a final bid to secure compliance, Aku repeats the recruiting move, now adding ‘I myself \{will do it\} when I’m back’, thereby trying to overcome Emma’s unwillingness by proposing to share the task but also accounting for her own inability to do it immediately.

\transheader{ex:dingemanse:30}{Maize3\_673020 (continues from \extref{ex:dingemanse:20} above)}\vspace{2mm}
%
\emptytransbox{7}{(0.8)}
%
\begin{mdframednoverticalspace}[style=firstfoc]
%
\begin{transbox}{8}{aku}
\begin{verbatim}
puta ũ   (.) anɔ:
lift him     2SG:hear:Q
\end{verbatim}
 pick him up (.) you hear?
\end{transbox}
\end{mdframednoverticalspace}
%
\emptytransbox{9}{(0.8)}
%
\begin{transbox}{10}{emm}
\begin{verbatim}
lèiye        ngɔ     toòputa       ũ   ní
1SG:NEG:know REL:who PROG:SCR:lift him FP
\end{verbatim}
I don’t know who’s picking him up \textit{ní}
\end{transbox}
%
\emptytransbox{11}{(1.0)}
%
\begin{mdframednoverticalspace}[style=firstfoc]
\begin{transbox}{12}{aku}
\begin{verbatim}
↑anɔ:↑
2SG:hear:Q
\end{verbatim}
↑you hear?↑
\end{transbox}
\end{mdframednoverticalspace}
%
\emptytransbox{12}{(0.7)}
%
\begin{transbox}{13}{emm}
\begin{verbatim}
mmmhhh ((yawn))
\end{verbatim}
\end{transbox}
%
\emptytransbox{14}{(1.1)}
%
\begin{transbox}{15}{aku}
\begin{verbatim}
la    ũ    si   lò   ba   (.)  mmɛ nìtɔ  si  lò   ba.
hold  him  LNK  1SG  come      I   self  LNK 1SG  come
\end{verbatim}
hold him until I’m back (.) I myself \{will\} when I’m back
\end{transbox}\bigskip

\normalsize
So we see here that a recruiting move can be resisted by simply ignoring it (line 7), claiming a lack of knowledge as to who should fulfill the recruitment (line 10), or producing a yawn where a response would have been relevant (line 12). Of note is that, throughout, Emma avoids going on record as being unwilling, revealing the lengths to which participants will go to avoid directly rejecting a recruitment (see also Blythe, \chapref{sec:blythe}, \sectref{sec:blythe:4.2} and \sectref{sec:blythe:7}).

The yawn, a physical display functioning as a claim of unavailability, brings us into the territory of accounts \citep{heritage_explanations_1988}, that is, the explanations that often accompany dispreferred responses. Embodied accounts such as Emma’s yawn are relatively rare, and special in being off-record. More commonly, accounts are verbal and on-record, as in \REF{ex:dingemanse:13}, where Dora asked ‘aren’t you bringing me water?’ and Efi answered ‘I’m just going up here, I’ll be back’, accounting for her failure to fulfill the recruitment by noting a competing commitment.

Yet another way to resist recruitment is to propose another course of action, and by far the rarest way to reject a recruitment is to actually say ‘no’. Both of these occur in \REF{ex:dingemanse:31}, below. Odo is asked to hold Aku’s child for a moment. Other participants include Mercy, a 3-year-old child, Hope, Odo’s 9-year-old son, a hairdresser, and her client, both visibly occupied. Even though Aku has already walked up to Odo and is holding up the child before him, Odo declines. He does so using a complex turn format featuring a declination, a reason, and an alternative course of action: ‘no, I didn’t give birth to the child (.) I’m like (\hspace{0.3cm}), give it to uh’ (line 4). The features of this turn are all consistent with what we know about the design of dispreferred responses (\citealt[334--35]{Levinson1983}; \citealt[265–66]{heritage_garfinkel_1984}).

\begin{figure}
\begin{tabularx}{\textwidth}{ll}
\centering
\includegraphics[width=.47\textwidth]{figures/siwu-img11.jpg} & \includegraphics[width=.47\textwidth]{figures/siwu-img12.jpg}
\end{tabularx}
\caption{(\textit{a}) Aku (rightmost) approaches Odo (with hand on water drum) holding out her infant (line 3); (\textit{b}) after Odo’s refusal, Hope (foreground) is recruited to hold the infant (line 13).}
\label{fig:dingemanse:7}
\end{figure}

\transheader{ex:dingemanse:31}{Compound5\_737320}\vspace{2mm}
%
\begin{transbox}{1}{aku}
\begin{verbatim}
ee, Odoi!
voc PSN:DIM
\end{verbatim}
hey, little Odo
\end{transbox}
%
\emptytransbox{2}{(0.7)}
%
\begin{mdframednoverticalspace}[style=firstfoc]
\begin{transbox}{3}{aku}
\begin{verbatim}
mɔɛ  Victor la   mɛ ((walks towards Odo, holds up infant, Fig. 6a))
grab PSN    hold me
\end{verbatim}
hold Victor for me
\end{transbox}
\end{mdframednoverticalspace}
%
\begin{mdframednoverticalspace}[style=secondfoc]
\begin{transbox}{4}{odo}
\begin{verbatim}
aɔ, leiye              ɔbi (.) ite     ibra    mɛ (  ), su   tã   ɛ:
no  1SG:NEG:give.birth child   it:PROG it:make me       take give HES
\end{verbatim}
no, I didn’t give birth to the child (.) I’m like (\hspace{0.3cm}), give it to uh:
\end{transbox}
\end{mdframednoverticalspace}
%
\emptytransbox{5}{(0.3)}
%
\begin{transbox}{6}{aku}
\begin{verbatim}
Me- Hope ba   [mɔɛ  ɔbi]  la   mɛ ((moves towards Odo’s daughter Hope))
PSN PSN  come grab  child hold me
\end{verbatim}
Me- Hope come hold the child for me
\end{transbox}
%
\begin{transbox}{7}{odo}
\begin{verbatim}
              [ Mercy  ]
                PSN
\end{verbatim}
\hspace{2.3cm} Mercy
\end{transbox}
%
\emptytransbox{8}{(0.5)}
%
\begin{transbox}{9}{odo}
\begin{verbatim}
su   ũ   tã   mɛ [pɛ nɛ Hope kà  [ɔ̃ũ pie   ndu
take 3SG give me  PƐ TP PSN  IMM  he bathe water
\end{verbatim}
okay whatever hand him to me, Hope is going to bath
\end{transbox}
%
\begin{transbox}{10}{hop}
\begin{verbatim}
                 [((comes running to Aku, holding out arms))
\end{verbatim}
\end{transbox}
%
\begin{transbox}{11}{aku}
\begin{verbatim}
                                 [nɛ  abu       sɔ 
                                  and 2SG:think QT 
\end{verbatim}
\hspace{5cm} so you thought 
\end{transbox}
%
\begin{transbox}{12}{~}
\begin{verbatim}
Mercy iba      [wo      ũ   puta:?
PSN   NOM.have  be.able 3SG lift
\end{verbatim}
Mercy would be able to lift him up?
\end{transbox}
%
\begin{transbox}{13}{hop}
\begin{verbatim}
               [((takes over infant, Fig. 6b))
\end{verbatim}
\end{transbox}
%
\xtransbox{14}{aku}{((reties her dress))}\\

\normalsize
In response to Odo’s rejection, Aku starts to formulate a name ‘Me-’, then self-repairs to Odo’s son ‘Hope’, walking away from Odo and asking Hope to hold the child. Odo meanwhile finishes his word search and says ‘Mercy’ (line 7), likely the name that Aku abandoned. Odo then begrudgingly volunteers to take the child after all, since he had other plans for his son Hope (line 9), but Hope already comes running towards Aku and Odo. Aku takes issue with Odo’s suggestion (lines 11--12) while Hope takes over the child (line 13). The expansion of the sequence after Odo’s rejection is typical of what happens after dispreferred responses \citep{schegloff_sequence_2007}.

Summing up, how do people resist recruitment? Not without collateral damage to the conversational sequence. They may initiate repair, which has the effect of buying some extra time, but as the side sequence closes a response is still relevant and they are likely to provide it \REF{ex:dingemanse:27}, or be held accountable for failing to do so \REF{ex:dingemanse:28}. They may try to ignore the recruiting move, but are likely to be held accountable for failing to respond, as in \REF{ex:dingemanse:29} and \REF{ex:dingemanse:30}. They can provide a reason \REF{ex:dingemanse:12}, propose another course of action, say no outright, or any combination of these things \REF{ex:dingemanse:31}, but all of these tend to lead to post-expansion of the sequence \citep[chap. 7]{Schegloff2007}.

In short, it seems the deck is firmly stacked against resistance and rejection, and the organization of interactional resources point to fulfillment as the most expedient way to reach sequence closure. This reflects an observation made in some of the earliest work on the organization of preferred/dispreferred actions: such actions ‘are both inherently structured and actively used so as to maximize cooperation and affiliation and to minimize conflict in conversational activities’ \citep[55]{atkinson_structures_1984}.

\subsection{Acknowledgment in third position}

Sometimes, a two-part recruitment sequence is followed by an expression that has the interactional function of closing the sequence: a “sequence closing third” \citep{schegloff_sequence_2007}. An example is given in \REF{ex:dingemanse:32}, where Awusi tells Yawa to pour water in a pan with plantain to be put on the fire. While Yawa is pouring, Awusi says \textit{milɛɛ} ‘that’s good’ to indicate that there is now enough water in the pan. This expression is also used when one is poured a drink, to indicate ‘this is enough’.

\transheader{ex:dingemanse:32}{Maize3\_286780}\vspace{2mm}
%
\begin{transbox}{1}{awu}
\begin{verbatim}
fore ndu-   fore ndu   i  bɛrɛdzo  amɛ. ((points to pan with plantain))
pour water- pour water in plantain pan
\end{verbatim}
pour water- pour water in the plaintain \{pan\}
\end{transbox}
%
\xtransbox{2}{yaw}{((takes jerrycan, pours water))}
%
\begin{transbox}{3}{awu}
\begin{verbatim}
milɛ:
AGR.N:be.good
\end{verbatim}
tha:t’s good
\end{transbox}\bigskip

\normalsize
One type of sequence closing third that is not attested in the Siwu collection is an acknowledgment like ‘thank you’. The simple and immediate practical actions studied here never receive verbal expressions of gratitude in Siwu. Instead, such expressions appear to be reserved for more momentous occasions, for instance when people have spent a day assisting each other with manual labor on the farm or in town. The importance of expressing gratitude in such cases is enshrined in a Siwu greeting routine often heard in the morning: \textit{gu fɔ kɔmakade karabra} ‘for your work yesterday’, which is answered with \textit{(gu) fɔ kpɛ:} ‘and yours’.

The absence of acknowledgments like ‘thank you’ in everyday recruitments in Siwu stands in contrast with accounts of frequent thanking practices in some other societies \citep{aston_say_1995,becker_spontaneous_1986}. However, these studies tend to focus on service encounters, which are quite different from the kinds of recruitments studied here \citep{apte_thank_1974}. One crucial difference is that everyday recruitments are almost always \textit{repayable in kind}. The comparative results of the project reported on in this volume suggest that thanking and other ways of verbalizing gratitude are less necessary because of an implicit norm that, where possible, we hold ourselves available and are willing to help others in turn \citep{FloydEtAl2018}, a norm that underlies the web of interdependence and reciprocity in resource-sharing that is typical of human societies \citep{melis_one_2016}. In contrast, service encounters present an asymmetry: we obtain services or goods that we do not control or produce ourselves, so paying back in kind is harder, which makes it more important to verbally express gratitude.\footnote{Children, like adults in service encounters, are also frequently in the position of not being able to pay back in kind. So perhaps the fact that children are socialized (in some societies) to say ‘thank you’ and indeed to use more prolix forms in general is a reflection of this asymmetry in agency.}

\section{Sequential structure and social asymmetries}\label{sec:dingemanse:5}
\subsection{A “rule of three” in social interaction?}\label{sec:dingemanse:5.1}

Non-minimal sequences amount to a little less than a third of initial recruitments in the core collection (44 out of 146). Most of them are resolved after one pursuit (33 cases); the remaining ones take two pursuits (10 cases) except for one case with three pursuits.\footnote{The only cases involving more than three attempts are those involving small children. As we will see below, these cases are dissimilar in other ways as well, a key difference being that small children are not held accountable for misunderstandings and failures to response in the same way as other participants.} We see the same in other-initiated repair in Siwu, where non-minimal sequences amount to about a quarter of 153 independent sequences and resolving a single troublesome bit of talk tends to take just one, sometimes two, and rarely more than three other-initiations of repair \citep{dingemanse_other-initiated_2015}.\footnote{I am indebted to Nick Enfield for our discussion of this pattern in sequences of other-initiated repair. The general pattern seems to be confirmed even in conversations involving people with Parkinson’s disease, where one might expect more protracted sequences of other-initiated repair \citep{GriffithsEtAl2015}. In his discussion of self-repair, Schegloff notes that “[a]lthough not common, two successive repairs on a same repairable, yielding (together with the repairable) three tries at that bit of talk, are not rare” \citep[277]{schegloff_relevance_1979}.}  So recruitment and repair usually take only one attempt (as in a minimal sequence), sometimes two, and seldom three or more attempts (\tabref{tab:dingemanse:6}).

\begin{table}
\begin{tabularx}{\textwidth}{Xlllllr}
\lsptoprule
Sequence type / \textit{N} attempts & 1 & 2 & 3 & 4 & ≥5 & Total \\
\midrule
Recruitment & 102 & 33 & 10 & 1 & -- & 146\\
Other-initiated repair & 117 & 26 & 8 & 2 & -- & 153\\
\lspbottomrule
\end{tabularx}
\caption{Distribution of independent sequences of recruitment and other-initiated repair and number of attempts (adult interaction only).}
\label{tab:dingemanse:6}
\end{table}

If this pattern proves representative and robust, it may point to a “rule of three” (or a “three strikes” principle) in social interaction: a disruption of progressivity in pursuit of a fitting response rarely takes more than three successive attempts, with a preference for fewer attempts. Research is needed here, starting with the identification of deviant cases, which may reveal to what degree it is a consequence of the structure of complex social action, and to what extent participants orient to it as a socially normative phenomenon.\footnote{An indication that a “rule of three” may relate specifically to disruptions of progressivity (as opposed to being a general limit on repeated behavior) is that multiples of successful recruitment sequences in close succession do occur, as in \REF{ex:dingemanse:21}, which features at least six nonverbal requests and responses.} Perhaps the needs addressed in recruitment and repair can overwhelmingly be solved in one go, and the increasingly lower frequency of cases with more than one attempt is in line with an expected probabilistic distribution. Perhaps participants balance intersubjectivity and progressivity \citep{heritage_intersubjectivity_2007}, and three attempts mark a tipping point where pursuits become too disruptive to overall progressivity. This may also be a fruitful area for cross-species comparison (cf. \citealt{wilkinson_requesting_2012} on repeated requests for meat sharing among chimpanzees), linking to a more general theme of communicative persistence.

\subsection{Social asymmetries}\label{sec:dingemanse:5.2}

An interest in social asymmetries has long been a prominent feature of cross-linguistic studies of requests \citep{brown_universals_1978,blum-kulka_cross-cultural_1989}. On the basis of this literature, one might expect the organization of assistance in interaction to be influenced by social asymmetries, such that, for instance, the selection of one format over another, or the nature of responsive actions, would differ depending on the relative social status of participants.

There is one large set of recruitments where social asymmetries clearly play a role: those involving small children as recruitees (recall that these were collected separately from the 207 cases that make up the core collection of Siwu recruitments, \sectref{sec:dingemanse:1.2}). The following extract is from a multiparty conversation in which a mother asks her toddler, less than 2 years of age, to come to her. The sequence involves six pursuits until compliance in line 11.

\transheader{ex:dingemanse:33}{Cooking1\_93710}\vspace{-1mm}
%
\begin{mdframednoverticalspace}[style=firstfoc]
%
\begin{transbox}{1}{mom}
\begin{verbatim}
Sise (.) ba.
Sise     come.
\end{verbatim}
Sise (.) come
\end{transbox}
\end{mdframednoverticalspace}
%
\emptytransbox{2}{(0.8)}
%
\begin{transbox}{3}{mom}
\begin{verbatim}
ba.
come.
\end{verbatim}
come
\end{transbox}
%
\emptytransbox{4}{(0.4)}
%
\begin{transbox}{5}{esi}
\begin{verbatim}
mama   sɔ ba.
mother QT come
\end{verbatim}
Mom  says come
\end{transbox}
%
\begin{transbox}{6}{mom}
\begin{verbatim}
↑ba:↑
come
\end{verbatim}
↑co:me↑
\end{transbox}
%
\emptytransbox{7}{(0.6)}
%
\begin{transbox}{8}{esi}
\begin{verbatim}
↑ma↑ma sɔ   ↑ba↑
mother QT   come
\end{verbatim}
↑mom↑  says ↑come↑
\end{transbox}
%
\emptytransbox{9}{(1.2)}
%
\begin{transbox}{10}{esi}
\begin{verbatim}
↑MA↑MA  SƆ  BA
mother  QT  come
\end{verbatim}
MOM SAYS COME
\end{transbox}
\begin{mdframednoverticalspace}[style=secondfoc]
\xtransbox{11}{chi}{((turns and walks towards mother))}
\end{mdframednoverticalspace}
\begin{transbox}{12}{ama}
\begin{verbatim}
ɔ    nyɔ   nɛ  yaa.
3SG  watch it  IDPH.absently
\end{verbatim}
{he} was just staring \textit{yaa} ((absently))
\end{transbox}\vspace{1mm}
%
\xtransbox{13}{mom}{((holds up underpants))}\vspace{-1mm}
%
\xtransbox{14}{chi}{((steps into underpants))}\\

\normalsize
This sequence differs in several ways from those considered so far. The number of pursuits appears to flout the “rule of three” (though none of the participants individually puts in more than three attempts). The pursuits are all simple repetitions with few changes except in prosody, in stark contrast with other pursuits we saw earlier which involve reformulations and reason-giving. Despite many pursuits, the child does not provide any form of response until the nonverbal action in line 11, and there is no evidence here that the child has mastery of strategies like repair initiation or other practices that others use in non-minimal sequences. Whereas recruiting and responding moves usually tend to be taken as a matter between recruiter and recruitee, here two other participants join in pursuing a response (lines 5, 8, 10), and a third provides an account for the lack of response of the child (line 12), showing that its absence is seen as accountable while at the same time implying that the child cannot (yet) speak for itself.

Combined, these observations suggest that child recruitees are treated differently. They are treated as still having to learn how to respond to recruiting moves, and they are not held accountable for their interactional conduct and for possible troubles in understanding in the same way that other participants typically are. While it may be tempting to say the child is treated this way \textit{because} of a social asymmetry, it is at least as plausible to say that cases like this show how social asymmetries are socially constructed and reinforced. The sequence is a socialization routine as much as an attempt to get the child to do something.

Social asymmetries also surface in sequences other than those involving very young children. Particularly telling of the social construction of asymmetries are moments when participants orient to them. Recall some of the turbulent sequences involving Sesi -- a teenager -- and his parents and alloparents. When, in \REF{ex:dingemanse:29}, Sesi’s aunt and mom note that ‘kids are difficult’, they invoke the category of \textit{kids}, which forms a contrast set with \textit{adults}, to make a complaint about Sesi’s unwillingness. It may be a universal feature of teenage behavior to try and find ways to escape household chores. Likewise, it may be a universal feature of caregiver talk to complain about this. That is one way in which social asymmetries can become tangible in interaction.

Although I have focused so far on evidence for social asymmetries in the moment-by-moment unfolding of the interaction, such social asymmetries do not emerge out of nothing. Knowledge about social membership categories and kinship relations is usually available, or at least assumed to be available, to participants in interaction \citep{terkourafi_beyond_2005,enfield_relationship_2013}, and so these categories and relations may also influence social interactions without being explicitly oriented to in talk. The most relevant durable social asymmetries for Siwu speakers are grounded in a combination of age and kinship relations. Older age generally comes with higher social status, and kinship structure provides a framework for allocating rights and duties (such that parents and alloparents can exercise deontic authority over younger kin). Based on this, most recruitment sequences in the corpus can be classified as involving a dyad that is either: (i) symmetrical with A and B having approximately the same social status, or (ii) asymmetrical with A higher in status than B, or (iii) asymmetrical with A lower in status than B (\tabref{tab:dingemanse:7}).\footnote{For 18 cases, it was not possible to assess this with sufficient confidence.}

%\todo{how to include examples in graph?}
% \todo{make bars solid grey}
\begin{table}
\begin{tabularx}{\textwidth}{Qrrr}
\lsptoprule
Relation & Count & Proportion & Examples\\
\midrule
No asymmetry (A${\approx}$B) & 91 & 66\% & \REF{ex:dingemanse:1},  \REF{ex:dingemanse:3},  \REF{ex:dingemanse:18},  \REF{ex:dingemanse:21},  \REF{ex:dingemanse:25}\\
A higher than B (A>B) & 31 & 21\% &  \REF{ex:dingemanse:2},  \REF{ex:dingemanse:22},  \REF{ex:dingemanse:27},  \REF{ex:dingemanse:28}\\
A lower than B (A<B) & 6 & 4\% &  \REF{ex:dingemanse:14}\\
Unclear & 18 & 12\% & \REF{ex:dingemanse:9}\\
\lspbottomrule
\end{tabularx}
\caption{Social asymmetry of participants in 146 independent recruitment sequences.}
\label{tab:dingemanse:7}
\end{table}

For a large majority of participants in recruitment sequences, there is no evidence of a social asymmetry between them, reflecting the fact that a lot of everyday social interaction in the corpus is between peers. In about one fifth of cases, participant A can be considered higher in social status than participant B; most commonly, these are cases where parents or alloparents address younger people in the household. In contrast, there are only 6 cases where participant A is clearly lower in social status than participant B. The relative paucity of such cases suggests that people may be somewhat less likely to recruit the assistance or collaboration of others who are higher in social status -- possibly as a way to avoid resistance, rejection, or other types of interactional turbulence (\citealt{brown_universals_1978}; see also Floyd, \chapref{sec:floyd}, \sectref{sec:floyd:6}; Enfield, \chapref{sec:enfield}, \sectref{sec:enfield:6}; Baranova, \chapref{sec:baranova}, \sectref{sec:baranova:6}). So social asymmetries may influence how likely people are to recruit assistance or collaboration from others.

Do social asymmetries also influence matters of formulation or format selection? An analysis of the core collection of recruitment sequences provides little evidence that social asymmetry (as operationalized here) is a decisive factor in format selection or in the design of responsive actions.\footnote{The following elements of format design and selection did not seem to be affected by the absence, presence, or direction of social asymmetry: type of recruitment (object transfer versus service); verbal or nonverbal means for recruitment; construction types (imperative, interrogative, declarative, \textit{si}-prefaced); presence or absence of an account in the recruiting turn; use of mitigating or strengthening devices; relative frequency of fulfillment versus resistance or repair; presence or absence of an account in the response. For three variables, there are not enough cases in the collection to draw firm conclusions about a possible role for social asymmetries: the relative frequency of recruitments to alter an ongoing trajectory of behavior; the relative frequency of assistance prompted by current or anticipatable trouble; and the relative frequency of resistance and rejection.} Instead, as we have seen throughout this study, many matters of formulation and selection appear to be more directly affected by local factors such as establishment of joint attention, relation to ongoing activities, and sequential position as initial or subsequent. This fits a recurring theme in systematic comparative work on informal conversation: micro-scale local factors like attention, participation framework, and sequential position seem more directly consequential than macro-sociological factors like social status, power, or politeness.

\section{Conclusions}

The domain of recruitments provides a microcosm of how linguistic resources combine with bodily conduct and adapt to social interaction. Malinowski, observing everyday social interaction on the Trobiand Islands a century ago, noted that “the structure of all this linguistic material is inextricably mixed up with, and dependent upon, the course of the activity in which the utterances are embedded” \citep[311]{malinowski_problem_1923}. Recruitments provide a privileged locus for observing this intertwining of speech and action.

Some of the resources used in recruitment sequences bear a language-specific signature. For instance, Siwu makes available a \textit{si}-prefaced format to mark consecutive actions in larger projects, and final particles like \textit{ló} and \textit{ní} for mitigating and strengthening recruiting moves.  But beneath the language-specific resources, the recruitment system appears to be fundamentally cut from the same cloth across languages and cultures. Recruiting and responding moves are adapted to recurrent interactional challenges, from calibrating joint commitments to specifying practical actions and managing activity structure. The Siwu recruitment system appears to be one instantiation of a sophisticated machinery for organizing collaborative action that transcends language and culture.

\section*{Transcription conventions and abbreviations}

Conversational transcripts follow the conventions developed by Jefferson (\citeyear{jefferson_glossary_2004}). In addition, words in free translations with no direct equivalent in the original material are \{marked so\}. Interlinear glosses follow the Leipzig glossing rules \citep{comrie_leipzig_2020} with the following additions: \textsc{cont} continuer • \textsc{fp} final particle •
\textsc{hes} hesitation marker •
\textsc{ing} ingressive •
\textsc{lnk} linker •
\textsc{o} object marker •
\textsc{plur} pluractional reduplication •
\textsc{psn} person name •
\textsc{scr} subject cross-reference marker.
Conflicts between conversation analytic conventions and Leipzig glossing rules (e.g. marking of self-repair vs. morpheme breaks using dashes) are resolved in favor of the former.

\section*{Acknowledgments}
Thanks to Rev. Wurapa and Ruben and Ella Owiafe for welcoming me into the Siwu-speaking community, and to Ɔdimɛ Kanairo for assistance in transcription and translation. \textit{Mi ndo karabra ló!} Thanks to the other members of the Recruitments project, especially Simeon Floyd, Giovanni Rossi, and Nick Enfield for feedback and guidance; to Felix Ameka for helpful comments; to Steve Levinson for fostering a department at MPI Nijmegen where the project could blossom; and to Herb and Eve Clark for hosting me at Stanford and providing the best writing environment. Sharon Rose helped me clear up shooing words in Gurage. The fieldwork and data collection underlying the study was funded by ERC Starting grant 240853 to N. J. Enfield. The author is supported by NWO Talent grant 016.vidi.185.205.

% a Veni grant from the Netherlands Organization for Scientific Research and by the Max Planck Society for the Advancement of Science.

\sloppy
\printbibliography[heading=subbibliography,notkeyword=this]
\end{document}
