\documentclass[output=paper,modfonts]{langscibook}
\ChapterDOI{10.5281/zenodo.4018378}
\usepackage{babel}
\usepackage{graphicx}
\usepackage{fancybox,framed}


\title{The recruitment system in Italian}

\author{Giovanni Rossi\affiliation{Department of Sociology, University of California, Los Angeles}}

\abstract{This chapter describes the resources that speakers of Italian use when recruiting assistance and collaboration from others in everyday social interaction. The chapter draws on data from video recordings of informal conversation in Italian, and reports language-specific findings generated within a large-scale comparative project involving eight languages from five continents (see other chapters of this volume). The resources for recruitment described in this chapter include linguistic structures from across the levels of grammatical organization, as well as gestural and other visible and contextual resources of relevance to the interpretation of action in interaction. The presentation of categories of recruitment, and elements of recruitment sequences, follows the coding scheme used in the comparative project (see \chapref{sec:coding} of the volume). This chapter extends our knowledge of the structure and usage of Italian with detailed attention to the properties of sequential structure in conversational interaction. The chapter is a contribution to an emerging field of pragmatic typology. }

\begin{document}
\maketitle
\label{sec:rossi}

\section{Introduction}\label{sec:rossi:1}
Social life would not be called such if there were not a system for people to get one another's help. Whatever their language and culture, people need others to get by in the small and big practicalities of everyday life, be it passing food, moving a heavy object, or doing some other chore. This chapter documents the main practices that speakers of Italian use to recruit assistance and collaboration from others, as observed in video recordings of naturally occurring interaction, analyzed as part of the comparative project reported on in this volume. After a brief description of the Italian language (\sectref{sec:rossi:1.1}) and of the data used for the study (\sectref{sec:rossi:1.2}), I begin by illustrating the basic structure of recruitment sequences (\sectref{sec:rossi:2}). I then survey the nonverbal and verbal practices used to design recruiting moves (\sectref{sec:rossi:3}), including pointing gestures, imperatives, different types of interrogatives and declaratives, and additional verbal elements. I then examine responding moves, focusing on how their design is fitted to that of the recruiting move (\sectref{sec:rossi:4}). I also discuss the occurrence and import of acknowledgment in third position (\sectref{sec:rossi:5}) and the role of social asymmetries (\sectref{sec:rossi:6}). The conclusion situates the findings in light of the cross-linguistic perspective adopted in the volume.

\subsection{The Italian language}\label{sec:rossi:1.1}
Italian is a Romance language spoken by over 60 million people in Italy, Southern Switzerland, and by migrant communities in several other countries, the largest of which are found in the United States, France, and Canada \citep{LewisSimonsFennig2014}. While being characterized by a profusion of geographical variation, the Italian language has certain core features that are shared across regional varieties. Verbs inflect for person, number, tense, and mood. Nouns, pronouns, adjectives and articles inflect for gender and number. Subject pronouns are normally dropped, though they can be maintained for emphasis or contrast. Word order is flexible, but the basic order is Subject Verb Object (SVO).\footnote{For more comprehensive descriptions of the grammar and sound patterns of Italian, see Lepschy and Lepschy (\citeyear{LepschyLepschy1988}), Bertinetto and Loporcaro (\citeyear{BertinettoLoporcaro2005}), Maiden and Robustelli (\citeyear{MaidenRobustelli2007}).} 

Of particular interest for the purposes of this chapter is the distinction among the three main sentence types identified by linguists cross-linguistically: imperatives, declaratives, and interrogatives \citep{SadockZwicky1985,KönigSiemund2007,aikhenvald2010}. In Italian, imperatives are distinguished from other sentence types by morphology and syntax, the rules of which are explained in \sectref{sec:rossi:3.3.2}. At the same time, there are generally speaking no morphosyntactic means for distinguishing declaratives from polar (yes/no) interrogatives. While it is commonly held that intonation compensates for this \citep[e.g.][]{GiliFivelaEtAl2015}, interactional research urges caution in claiming straightforward mappings between intonation and polar questions \citep{Rossano2010}. That said, recent work also provides evidence for the association of distinct intonation contours with specific types of polar questions, particularly questions involved in other-initiation of repair and related actions \citep{Rossi2015c,Rossi2020b}. More evidence for the role of intonation in marking interrogative utterances in Italian comes from findings discussed later in this chapter (\sectref{sec:rossi:3.3.3}). 

There is a growing body of studies on the Italian language in social interaction, including studies of family life and socialization \citep[e.g.][]{Sterponi2003,Fatigante2007,ArcidiaconoPontecorvo2010,Pauletto2017}, storytelling \citep{MonzoniDrew2009}, medical interaction \citep[e.g.][]{PinoMortari2012,MortariPino2014}, and basic domains of social organization such as the question-answer system \citep{Rossano2010} and gaze behavior \citep{Rossano2012}. Recent research has explored the linguistic design of social actions such as invitations \citep{MarguttiGalatolo2018}. A study by \citet{GaleanoFasulo2009} has looked at request sequences between parents and children, including the use of address terms, preliminary questions, forms of requesting that are more or less coercive, the role of normative reasoning, and the structure of sequences of “concatenated” requests. These themes resonate with those explored in the present study. Informed by previous and ongoing work in this area \citep{Rossi2011,Rossi2012,Rossi2014,Rossi2015c,Rossi2015a,Rossi2017,Rossi2018,RossiZinken2016}, this chapter provides an overview of requesting behavior in informal interaction among adult speakers of Italian as part of the broader phenomenon of recruitment.

\subsection{Data collection and corpus}\label{sec:rossi:1.2}
The video corpus on which this research is based was constructed in accordance with a set of guidelines developed by and for the members of the comparative project reported on in the volume (see Chapters 1--2). The video recordings were made between 2009 and 2013 in several locations within the province of Trento and the urban area of Bologna, in northern Italy. The interactions recorded were all informal, among family and friends, and involved not only casual conversation but also everyday activities such as cooking, having meals, and playing games. Participants received no instruction other than to go about whatever activity they were engaged in. From this corpus, I sampled 15 interactions for a total of 3.5 hours, yielding 221 recruitment sequences. 

Conventions for transcription, glossing, and translation are explained in a dedicated section at the end of the chapter.

\section{Basics of recruitment sequences}\label{sec:rossi:2}
As defined in \chapref{sec:intro}, \sectref{sec:intro:4}, a recruitment is a basic cooperative phenomenon in social interaction consisting of a sequence of two moves with the following characteristics:

\begin{description}
\item[Move A:] participant A says or does something to participant B, or that B can see or hear;
\item[Move B:] participant B does a practical action for or with participant A that is fitted to what A has said or done.
\end{description}

This is the basic structure and development of a recruitment sequence, an example of which is given in \REF{ex:rossi:1} below. Other details of what can happen, including what participant B can do in Move B to fulfill or reject the recruitment, are illustrated in later sections. In the transcripts, ▶ and ▷ designate Move A and Move B, respectively. 

\subsection{Minimal recruitment sequence}\label{sec:rossi:2.1}

\extref{ex:rossi:1} exemplifies a typical recruitment sequence. Sergio and Plinio are washing the dishes. As Sergio finishes rinsing a baking pan, he turns to Plinio, who is wiping washed cutlery, and recruits his collaboration with an imperative request: \textit{´{}PLInio; a´{}sciUga anche \`{}QUEsta.} ‘Plinio wipe this one too’ (Move A). He then walks to Plinio and hands him the baking pan. In response, Plinio takes the baking pan and begins to wipe it (Move B).

\transheader{ex:rossi:1}{CampFamLava\_1518767}\vspace{2mm}
%
\emptytransbox{1}{(33.0)}
%
\begin{mdframednoverticalspace}[style=firstfoc]
\begin{transbox}{2}{ser}
\begin{verbatim}
´PLInio; a´sciUga    anche ↘QUEsta. ((shakes baking pan over sink))
NAME     dry-IMP.2SG also  this-F
\end{verbatim}
Plinio wipe this one too
\end{transbox}
\end{mdframednoverticalspace}\vspace{1mm}
%
\xtransbox{3}{~}{((walks with baking pan to Plinio))}
%
\begin{mdframednoverticalspace}[style=secondfoc]
\xtransbox{4}{pli}{((takes baking pan from Sergio))}
\end{mdframednoverticalspace}\vspace{-1mm}
%
\xtransbox{5}{~}{((sets baking pan on counter and begins to wipe it))}\\

In Move A, Sergio uses an imperative, a verbal form that is intimately connected to the process of recruitment by virtue of its semantics, which encodes the speaker's attempt to get another to do something (\citealt[746--748]{Lyons1977}; \citealt[170--171]{SadockZwicky1985}). One of the properties of this verbal form is that it anticipates only the fulfillment of the recruitment, which is what Plinio does in Move B. In cases like \REF{ex:rossi:1}, the recruitment sequence unfolds as an adjacency pair \citep{Schegloff1968,SchegloffSacks1973}, where the fulfillment of a practical action by participant B is normatively expected after what participant A says or does. In other cases, participant B's cooperation is not obliged by the recruiting move but rather occasioned by it, meaning that its absence may not be sanctionable or accountable in the same way. 

\subsection{Non-minimal recruitment sequence}\label{sec:rossi:2.2}
People are often successful in recruiting others at the first go. But there are times when a first attempt fails, either because the recruiting move is not heard, seen, or understood, or because it is ignored. In yet other cases, the recruitment may be rejected. \extref{ex:rossi:2} gives an example of this, illustrating how a first attempt at stopping someone from doing something is pursued with further attempts, generating a non-minimal sequence.

During a family dinner, Luca picks up a piece of the dessert before everyone has finished the main course (line 2). Olga, who is sitting across the table from him, notices this behavior and tries to rectify it by saying \textit{<{}<h,f>in´{}tAnto il ˆ{}dOlce si mangia mia a\`{}DESso>.} ‘by the way the dessert is really not to be eaten now’ (line 3), then adding \textit{\`{}DOpo.} ‘later’ (line 7). Instead of complying, Luca holds the pastry close to his mouth (line 8) and expresses his resistance with a “purse hand” gesture (line 10, \figref{fig:rossi:1}). This leads Olga to pursue the recruitment with further attempts.

\transheader{ex:rossi:2}{PranzoAlbertoni01\_1837927}\vspace{2mm}
%
\begin{transbox}{1}{~}
\begin{verbatim}
+(2.1)           +(0.9)
\end{verbatim}
\end{transbox}
%
\begin{transbox}{2}{\textit{luc}}
\begin{verbatim}
+picks up pastry +removes wrapper--> 
\end{verbatim}
\end{transbox}
%
\begin{mdframednoverticalspace}[style=firstfoc]
\begin{transbox}{3}{olg}
\begin{verbatim}
<<h,f>in´tAnto  il  ˆdOlce  si   mangia +mia a↘DESso.>
      meanwhile the dessert IMPS eat-3SG PTC now
\end{verbatim}
\hspace{0.8cm} by the way the dessert is really not to be eaten now
\end{transbox}
\end{mdframednoverticalspace}\vspace{1mm}
%
\begin{transbox}{4}{\textit{luc}}
\begin{verbatim}
--------------------------------------->+raises pastry to mouth-->
\end{verbatim}
\end{transbox}\vspace{-1mm}
%
\begin{transbox}{5}{~}
\begin{verbatim}
(0.1)*(0.1)
\end{verbatim}
\end{transbox}\vspace{-1mm}
%
\begin{transbox}{6}{~}
\begin{verbatim}
     *gazes up at Olga-->>
\end{verbatim}
\end{transbox}\vspace{-1mm}
%
\begin{mdframednoverticalspace}[style=firstfoc]
\begin{transbox}{7}{olg}
\begin{verbatim}
↘DO+po.
after
\end{verbatim}
later
\end{transbox}
\end{mdframednoverticalspace}\vspace{2mm}
%
\begin{mdframednoverticalspace}[style=secondfoc]
\begin{transbox}{8}{\textit{luc}}
\begin{verbatim}
-->+holds up pastry-->
\end{verbatim}
\end{transbox}
\end{mdframednoverticalspace}\vspace{-1mm}
%
\begin{transbox}{9}{~}
\begin{verbatim}
(1.8)Δ(0.2)#
\end{verbatim}
\end{transbox}
%
\begin{mdframednoverticalspace}[style=secondfoc]
\begin{transbox}{10}{~}
\begin{verbatim}
     Δmakes “purse hand” gesture-->
\end{verbatim}
\end{transbox}
\end{mdframednoverticalspace}% \vspace{0.5mm}
%
\begin{transbox}{11}{\fig}
\begin{verbatim}
           #Figure 1
\end{verbatim}
\end{transbox}\vspace{-0.75mm}
%
\begin{mdframednoverticalspace}[style=firstfoc]
\begin{transbox}{12}{olg}
\begin{verbatim}
puoi    ^METterlo       ↘lÀ Δ 'l  dolce. ((points at tray))
can.2SG put=INF=3SG.ACC there the dessert
\end{verbatim}
can you put the dessert \{back\} there
\end{transbox}
\end{mdframednoverticalspace}\vspace{1mm}
%
\begin{transbox}{13}{\textit{luc}}
\begin{verbatim}
--------------------------->Δ
\end{verbatim}
\end{transbox}
%
\emptytransbox{14}{(0.3)}
%
\begin{mdframednoverticalspace}[style=secondfoc]
\begin{transbox}{15}{luc}
\begin{verbatim}
perché.
\end{verbatim}
why
\end{transbox}
\end{mdframednoverticalspace}\vspace{1mm}
%
\begin{mdframednoverticalspace}[style=firstfoc]
\begin{transbox}{16}{olg}
\begin{verbatim}
perché  si   mangia  dopo;=te      dao      questo;=
because IMPS eat-3SG after 2SG.DAT give-1SG this
\end{verbatim}
because it is to be eaten later -- I'll give you this
\end{transbox}
\end{mdframednoverticalspace}
%
\emptytransbox{~}{((points at large cake))}
%
\begin{mdframednoverticalspace}[style=secondfoc]
\begin{transbox}{17}{luc}
\begin{verbatim}
=ma  io      non lo      mangio,
 but 1SG.NOM not 3SG.ACC eat-1SG
\end{verbatim}
\hspace{0.07cm} but I'm not going to eat it
\end{transbox}
\end{mdframednoverticalspace}\vspace{1mm}
%
\emptytransbox{18}{(0.6)}
%
\begin{transbox}{19}{luc}
\begin{verbatim}
anche se mi      piace      però_
even  if 1SG.DAT please-3SG but
\end{verbatim}
though I do like it but
\end{transbox}\vspace{1mm}
%
\begin{transbox}{20}{~}
\begin{verbatim}
(0.1)+(1.1)
\end{verbatim}
\end{transbox}
%
\begin{transbox}{21}{~}
\begin{verbatim}
---->+eats pastry-->>
\end{verbatim}
\end{transbox}\vspace{-2mm}
%
\begin{transbox}{22}{olg}
\begin{verbatim}
lo      sai      cos'è_ ((points at large cake))
3SG.ACC know-2SG what=be.3SG
\end{verbatim}
do you know what it is
\end{transbox}\medskip

\begin{figure}
\centering
\includegraphics[height=.35\textheight]{figures/Rossi_Picture1}
\caption{Frame from \extref{ex:rossi:2}, line 11. Luca challenges Olga's first attempt to stop him from having dessert with a “purse hand” gesture (≈ ‘what's the problem?!’)}
\label{fig:rossi:1}
\end{figure}
%\caption{{\small

Olga's first attempt to stop Luca from having dessert ahead of time is unsuccessful. Luca first shows non-compliance by bringing the pastry to his mouth (line 4), and then goes on to express overt resistance with a “purse hand” gesture \citep{Poggi1983,Kendon1995}. This emblematic gesture, where all the fingers are drawn together so as to be in contact with one another at the tips, may be roughly translated here as ‘what’s the problem?!’. Olga pursues the recruitment by changing strategy, using an interrogative form instead: \textit{puoi \^{}METterlo \`{}lÀ 'l dolce.} ‘can you put the dessert \{back\} there’ (see \sectref{sec:rossi:3.3.3} below). This second attempt is also unsuccessful. Luca continues his challenge by soliciting an account (\textit{perché.} ‘why’, line 15). Olga then restates the norm of behavior invoked a moment earlier (\textit{perché si mangia dopo;} ‘because it is to be eaten later’) and adds an enticement (\textit{=te dao questo;} ‘I'll give you this’), referring to a large cake (in the foreground of \figref{fig:rossi:1}) that will be the dessert's highlight. However, Luca continues to push back by saying that he is not going to eat from that large cake (line 17). A moment later, he goes ahead and eats the pastry he picked up a the beginning of the extract (line 21). 

The development of the sequence shows the sustained relevance of compliance with the recruitment initiated by Olga, which she pursues with multiple attempts. This leads to an expansion of the basic two-part structure illustrated in \sectref{sec:rossi:2.1} above.

\subsection{Subtypes of recruitment sequence}\label{sec:rossi:2.3}

The phenomenon investigated in this project encompasses a range of social-inter\-actional events that have in common the mobilization of someone's practical action. As discussed in \chapref{sec:coding}, \sectref{sec:coding:6}, most recruitment sequences fall into four broad subtypes. The two examples examined in the previous section illustrate two: the provision of a service \REF{ex:rossi:1}, where someone is recruited to perform a manual task, and an (attempted) alteration of trajectory \REF{ex:rossi:2}, where someone is recruited to stop or change an ongoing behavior. I now illustrate the two remaining subtypes, starting with object transfers.

Some time before \REF{ex:rossi:3}, Furio has offered Sara a piece of the banana he is eating. In line 2, she asks him for one more piece.

\transheader{ex:rossi:3}{BiscottiPome01\_2168783}\vspace{2mm}
%
\emptytransbox{1}{(1.3)}
%
\begin{mdframednoverticalspace}[style=firstfoc]
\begin{transbox}{2}{sar}
\begin{verbatim}
me      ne  ^DAi     un  altro ↘pEzzo.
1SG.DAT PTV give-2SG one other piece
\end{verbatim}
\{will\} you give me one more piece
\end{transbox}
\end{mdframednoverticalspace}\vspace{1.5mm}
%
\emptytransbox{3}{(4.4)}
%
\begin{mdframednoverticalspace}[style=firstfoc]
\begin{transbox}{4}{sar}
\begin{verbatim}
per fa↘VOre.
for favor
\end{verbatim}
please
\end{transbox}
\end{mdframednoverticalspace}\vspace{1.25mm}
%
\emptytransbox{5}{(1.4)}\vspace{-0.25mm}
%
\begin{mdframednoverticalspace}[style=secondfoc]
\begin{transbox}{6}{fur}
\begin{verbatim}
((gives Sara one more piece of banana))
\end{verbatim}
\end{transbox}
\end{mdframednoverticalspace}\vspace{0.5mm}
%
\begin{transbox}{~}{~}
\begin{verbatim}
((10 seconds not shown))
\end{verbatim}
\end{transbox}\vspace{-1.75mm}
%
\begin{transbox}{7}{sar}
\begin{verbatim}
grazie.
\end{verbatim}
thanks
\end{transbox}\bigskip

Sara's initial recruiting move is followed by a long silence (line 3). She then pursues the request with the formulaic \textit{per fa\`{}VOre.} ‘please’. After another, shorter silence, Furio eventually fulfills the recruitment by giving Sofia one more piece of banana.

The fourth subtype of recruitment sequence is trouble assistance, where participant B steps in to help in response to participant A's current trouble. In \REF{ex:rossi:4}, Sergio is styling Greta's hair. During the process, a strand of dye-soaked hair rolls down on Greta's face (line 3, \figref{fig:rossi:2}\textit{a}), causing her to gasp (line 5). As Sergio realizes what has happened (line 8), he promptly gathers the strand of hair and folds it back over Greta's head (lines 9--12, \figref{fig:rossi:2}\textit{b}).

\transheader{ex:rossi:4}{Tinta\_1445710}\vspace{2mm}
%
\begin{transbox}{1}{~}
\begin{verbatim}
+(2.3)Δ(0.4)#
\end{verbatim}
\end{transbox}
%
\begin{transbox}{2}{\textit{luc}}
\begin{verbatim}
+kneads hair-->
\end{verbatim}
\end{transbox}
%
\begin{transbox}{3}{~}
\begin{verbatim}
      Δstrand of hair rolls down on Greta's face-->>
\end{verbatim}
\end{transbox}
%
\begin{transbox}{4}{\fig}
\begin{verbatim}
            #Figure 2a
\end{verbatim}
\end{transbox}
%
\begin{mdframednoverticalspace}[style=firstfoc]
\begin{transbox}{5}{gre}
\begin{verbatim}
°HH*HH†H ((gasps))
\end{verbatim}
\end{transbox}
\end{mdframednoverticalspace}
%
\begin{transbox}{6}{~}
\begin{verbatim}
   *tilts head-->>
\end{verbatim}
\end{transbox}
%
\begin{transbox}{7}{~}
\begin{verbatim}
      †raises hand to face-->>
\end{verbatim}
\end{transbox}
%
\begin{transbox}{8}{ser}
\begin{verbatim}
<<f,h>↘Uu:+:.>
\end{verbatim}
\hspace{1.05cm}oo::
\end{transbox}\vspace{1mm}
%
\begin{mdframednoverticalspace}[style=secondfoc]
\begin{transbox}{9}{~}
\begin{verbatim}
          +gathers strand of hair and folds it back up-->
\end{verbatim}
\end{transbox}
\end{mdframednoverticalspace}\vspace{-1mm}
%
\begin{transbox}{10}{~}
\begin{verbatim}
(0.3)#(0.8)+
\end{verbatim}
\end{transbox}
%
\begin{transbox}{11}{\fig}
\begin{verbatim}
     #Figure 2b
\end{verbatim}
\end{transbox}\vspace{-0.75mm}
%
\begin{transbox}{12}{~}
\begin{verbatim}
---------->+
\end{verbatim}
\end{transbox}\vspace{-2mm}
%
\begin{transbox}{13}{ser}
\begin{verbatim}
scusa_
\end{verbatim}
sorry
\end{transbox}\bigskip

\begin{figure}
\caption{Frames from \extref{ex:rossi:4}, illustrating a case of trouble assistance.}
\label{fig:rossi:2}
\subfigure[A strand of dye-soaked hair rolls down \newline on Greta's face (line 4).]{
  \includegraphics[height=.35\textheight]{figures/Rossi_Picture2} % \label{fig:rossi:2a}
}
\subfigure[Sergio helps Greta by gathering the strand of hair to fold it back up (line 11).]{
  \includegraphics[height=.35\textheight]{figures/Rossi_Picture3} % \label{fig:rossi:2b}
}
\end{figure}

Greta's gasp is seemingly produced as an instinctive reaction to the sudden discomfort of dye-soaked hair rolling down onto her face and possibly into her eye; it is arguably not intended or designed to elicit Sergio's help. What this shares with other recruiting behaviors, however, is that it makes apparent Greta's need for assistance, instigating Sergio to step in. %Rather than being solicited, Sergio's help here is volunteered.

\section{Formats in Move A: The recruiting move}\label{sec:rossi:3}

It has long been noted that people use a wide range of strategies to get others to do things (see \chapref{sec:intro}, \sectref{sec:intro:2}). In the framework of this project, this means looking at the resources that are available to people to design Move A, the recruiting move. Most of the literature on this topic focuses on verbal formats. But in face-to-face interaction, recruiting moves often involve a composite of verbal and nonverbal elements, and may also be fully nonverbal. This section surveys the range of options available to Italian speakers.

\subsection{Fully nonverbal recruiting moves}\label{sec:rossi:3.1}
Fully nonverbal recruiting moves in Italian are much less frequent than ones involving language, making up only 10\% of the cases (\textit{n}=22/221). One reason for this is that fully nonverbal recruiting moves normally occur in relatively constrained contexts. \extref{ex:rossi:5} provides an example.

Four friends have just finished playing a card game. Flavia announces how many points her team has to ‘pay’ (lines 1 and 4), that is, subtract from the previous score. A moment later, Bianca initiates a recruitment to retrieve the notepad on which the scores are kept.

\transheader{ex:rossi:5}{Circolo01\_402024}\vspace{2mm}
%
\begin{transbox}{1}{fla}
\begin{verbatim}
e   ades te      pago    zinquantazin[que,
and now  2SG.DAT pay-1SG fifty_five
\end{verbatim}
and now I'll pay you fifty-five
\end{transbox}
%
\begin{transbox}{2}{bia}
\begin{verbatim}
                                     [sì.
\end{verbatim}
\hspace{5.6cm} yes
\end{transbox}\vspace{-1mm}
%
\emptytransbox{3}{(0.6)}\vspace{-1mm}
%
\begin{transbox}{4}{fla}
\begin{verbatim}
<<rall>cinquanta?> cinque,=
       fifty       five
\end{verbatim}
\hspace{0.95cm} fifty- \hspace{1.05cm} five
\end{transbox}\vspace{1mm}
%
\begin{transbox}{5}{sil}
\begin{verbatim}
=<<p>sie+te.>
\end{verbatim}
\hspace{0.7cm} seven
\end{transbox}\vspace{1mm}
%
\begin{transbox}{6}{\textit{bia}}
\begin{verbatim}
        +puts cards back on top of deck-->
\end{verbatim}
\end{transbox}
%
\begin{transbox}{7}{~}
\begin{verbatim}
(0.1)+(0.3)*(0.8)+(0.5)+Δ#(0.2)Δ(0.6)    +(0.6)+(0.2)Δ(0.7)
\end{verbatim}
\end{transbox}
%
\begin{transbox}{8}{\textit{bia}}
\begin{verbatim}
           *turns and gazes at notepad-------------------------------->
\end{verbatim}
\end{transbox}
%
\begin{mdframednoverticalspace}[style=firstfoc]
\begin{transbox}{9}{~}
\begin{verbatim}
---->+           +.....+points at notepad+,,,,,+           
\end{verbatim}
\end{transbox}
\end{mdframednoverticalspace}
%
\begin{mdframednoverticalspace}[style=secondfoc]
\begin{transbox}{10}{\textit{sil}}
\begin{verbatim}
                        Δturns Δreaches for notepad  Δpasses notepad-->
\end{verbatim}
\end{transbox}
\end{mdframednoverticalspace}
%
\begin{transbox}{11}{\fig}
\begin{verbatim}
                         #Figure 3
\end{verbatim}
\end{transbox}\vspace{-1mm}
%
\begin{transbox}{12}{~}
\begin{verbatim}
+(0.1)Δ
\end{verbatim}
\end{transbox}
%
\begin{transbox}{13}{\textit{bia}}
\begin{verbatim}
+puts glasses on-->>
\end{verbatim}
\end{transbox}\vspace{-1mm}
%
\begin{transbox}{14}{\textit{sil}}
\begin{verbatim}
----->Δ
\end{verbatim}
\end{transbox}\bigskip

\begin{figure}
\centering
\includegraphics[height=.35\textheight]{figures/Rossi_Picture4} %don't forget to include caption
\caption{Frame from \extref{ex:rossi:5}. Bianca points at the notepad; Silvia turns toward it (line 11).}
\label{fig:rossi:3}
\end{figure}

Shortly after approving Flavia's count (line 2), Bianca turns to her right and gazes over at the notepad across the table (line 8), which is out of her reach but within Silvia's (\figref{fig:rossi:3}). Bianca then points at the notepad (line 9); Silvia turns toward the notepad, reaches for it, and passes it to Bianca (line 10). Silvia can be expected to comply with Bianca's request in that it is made in direct contribution to a shared activity that Silvia is participating in (see \sectref{sec:rossi:3.3.2} below).

The action recruited here is embedded in the ordinary development of the ongoing activity \citep{Rossi2014}. At the end of every game, the points for each team are counted and the scores updated in the game's record, which is kept on the notepad. For all the previous games, Bianca has been responsible for updating the record. So when Flavia marks the end of the count by repeating her team's score (line 4), the projectable next action is Bianca writing it down. This is an environment in which Bianca gazing and pointing at the notepad is all that is needed for Silvia to understand that she is being recruited to pass it.

For a fully nonverbal recruiting move to be successful, the action being recruited needs to be projectable. A common source of projection in informal interaction is the structure of an activity, which sets up expectations about people's actions within the activity (see \citealt{Levinson1979,Robinson2013}, among others). The structure of an activity is a form of common ground \citep[93]{Clark1996} that can and should be relied upon by participants when recruiting one another's collaboration. When the passing of an object is an expectable contribution to a joint activity, as in \REF{ex:rossi:5}, participant A can minimize the recruiting move by simply making known the wanted object to participant B and preparing to receive it. Such minimization is motivated by fundamental principles of human communication \citep{Grice1975,Levinson2000}. These principles provide a common basis for the production of fully nonverbal recruiting moves across languages (see Kendrick \chapref{sec:kendrick}, \sectref{sec:kendrick:4.1.3}; Zinken, \chapref{sec:zinken}, \sectref{sec:zinken:3.1}; Baranova, \chapref{sec:baranova}, \sectref{sec:baranova:3.1}; Dingemanse \chapref{sec:dingemanse}, \sectref{sec:dingemanse:3.4}). 

\subsection{Nonverbal behavior in composite recruiting moves}\label{sec:rossi:3.2}

Nearly half of all recruiting moves in the Italian data involve a combination of verbal and nonverbal elements (47\%, \textit{n}=96/206).\footnote{Fifteen cases were excluded from the count as the recruiter was momentarily off camera or hidden by another participant at the time of the recruiting move.} The types of nonverbal behavior that co-occur with language in these composite recruiting moves are given in \tabref{tab:rossi:1}.

\begin{table}
\begin{tabularx}{.55\textwidth}{Xrr}
\lsptoprule
Type & Count & Proportion \\
\midrule
Pointing & 42 & 44\% \\
Holding out & 10 & 11\% \\
Iconic gesture & 9 & 9\% \\
Placing & 9 & 9\% \\
Reaching out & 5 & 5\% \\
Other & 21 & 22\% \\
\lspbottomrule
\end{tabularx}
\caption{Nonverbal behaviors in composite recruiting moves (\textit{n}=96).}
\label{tab:rossi:1}
\end{table}

The majority of nonverbal behaviors fall into three basic types identified in the comparative project (see \chapref{sec:coding}, \sectref{sec:coding:6}): pointing, holding out an object for someone to take and do something with, and reaching out to receive an object. But there are two other types that figure prominently in the Italian data. One is placing an object in a meaningful location for someone to do something with \citep[249-50]{Clark2003}; the other is iconic gestures that depict the shape of the target object or action. In what follows, I focus on the use of pointing and iconic gestures, leveraging previous research on co-speech gesture to shed light on its role in recruitments.

\citet{EnfieldKitaRuiter2007} have shown that, when used by speakers to refer to localities, pointing gestures can take two main forms: “big” and “small”. Big points are articulated with the whole arm, usually with head and gaze also oriented to the target. Small points are reduced in size and articulatory effort, with the head and gaze less frequently oriented to the target. \citeauthor{EnfieldKitaRuiter2007} argue that the two pointing forms are functionally distinct. Big points are used when the information in the gesture is the primary, foregrounded component of the message, while small points are used when the speech is informationally foregrounded and the gesture adds to it in the background.

The argument is that big points occur when the location of a referent is focal (see \citealt[chap. 5]{Lambrecht1994}). In these cases, the speech typically contains a deictic element (such as ‘here’, ‘there’, ‘this’, ‘that’) but it is the gesture that supplies the key information. So it needs to be maximally accurate. Small points, on the other hand, occur in a variety of contexts where a referent seems “likely but not certain to be recognizable” \citep[1730]{EnfieldKitaRuiter2007}. In these cases, the speech should be sufficient for reference to be secured, but it might not. Speakers therefore strike a balance between the risks of “under-telling” and “over-telling” by adding a bit of extra information in their gesture.\footnote{On under-telling and overt-telling, cf. \citet{Grice1975}, \citet{Levinson2000}, \citet[140]{Schegloff2007a}.}

When we look at pointing gestures in recruiting moves, the Italian data suggest that their form is sensitive to the distinctions proposed by \citet{EnfieldKitaRuiter2007}. Compare the following two cases, taken from the same interaction, where a group of friends are making cocktails. 

When \REF{ex:rossi:6} begins, Silvio has just stopped pouring soda in a carafe and is proceeding to add gin. However, Bino and Fabio alert him to the fact that the quantity of soda is not yet sufficient (lines 1--2). By so doing, they recruit Silvio to add more soda (line 5). As Silvio begins to do this, Fabio produces another recruiting move -- the one in focus here -- aimed at further adjusting the trajectory of Silvio's actions. He tells Silvio to pour the soda \textit{\`{}QUA;} ‘\{in\} here’, that is, into another container. The location of ‘here’ is supplied by a “big” pointing gesture (\figref{fig:rossi:4}\textit{a}). %aimed at altering the trajectory of Silvio's actions

\transheader{ex:rossi:6}{MasoPome\_2058866}\vspace{2mm}
%
\begin{transbox}{1}{bin}
\begin{verbatim}
non è      mezza [bozza  ancora.
not be.3SG half   bottle yet
\end{verbatim}
it's not yet half bottle
\end{transbox}\vspace{0.5mm}
%
\begin{transbox}{2}{fab}
\begin{verbatim}
                 [sì  non è      mezza bozza.
                  yes not be.3SG half  bottle
\end{verbatim}
\hspace{2.6cm} right it's not half bottle
\end{transbox}\vspace{-0.5mm}
%
\begin{transbox}{3}{~}
\begin{verbatim}
(0.2)
\end{verbatim}
\end{transbox}
%
\begin{transbox}{4}{~}
\begin{verbatim}
Δ(0.4)*(0.4)   Δ(0.5)              Δ(0.5)
\end{verbatim}
\end{transbox}
%
\begin{transbox}{5}{\textit{sil}}
\begin{verbatim}
Δputs gin down Δpicks soda back up Δpours soda into carafe-->
\end{verbatim}
\end{transbox}
%
\begin{transbox}{6}{\textit{fab}}
\begin{verbatim}
      *gazes at other container----------------------------->
\end{verbatim}
\end{transbox}
%
\begin{mdframednoverticalspace}[style=firstfoc]
\begin{transbox}{7}{fab}
\begin{verbatim}
↑´bU+ttalo             gi+ù ↘QU#A;
 throw-IMP.2SG=3SG.ACC down here
\end{verbatim}
\hspace{0.07cm} pour it \{in\} here
\end{transbox}
\end{mdframednoverticalspace}
%
\begin{transbox}{8}{~}
\begin{verbatim}
    +....................+points at other container-->
\end{verbatim}
\end{transbox}
%
\begin{transbox}{9}{\fig}
\begin{verbatim}
                               #Figure 4a
\end{verbatim}
\end{transbox}
%
\begin{transbox}{10}{~}
\begin{verbatim}
(0.2)+*(0.5)+(0.4)+(0.1)Δ(0.1)
\end{verbatim}
\end{transbox}
%
\begin{transbox}{11}{\textit{fab}}
\begin{verbatim}
---->+,,,,,,+     +....................-->
\end{verbatim}
\end{transbox}
%
\begin{transbox}{12}{~}
\begin{verbatim}
----->*
\end{verbatim}
\end{transbox}\vspace{-0.5mm}
%
\begin{transbox}{13}{\textit{sil}}
\begin{verbatim}
----------------------->Δputs soda down-->
\end{verbatim}
\end{transbox}
%
\begin{mdframednoverticalspace}[style=firstfoc]
\begin{transbox}{14}{fab}
\begin{verbatim}
´bUttalo             *giù +↘L#Ì;
throw-IMP.2SG=3SG.ACC down there
\end{verbatim}
pour it \{in\} there
\end{transbox}
\end{mdframednoverticalspace}\vspace{1mm}
%
\begin{transbox}{15}{~}
\begin{verbatim}
..........................+points again at other container-->
\end{verbatim}
\end{transbox}
%
\begin{transbox}{16}{~}
\begin{verbatim}
                     *gazes again at other container-------->
\end{verbatim}
\end{transbox}
%
\begin{transbox}{17}{\fig}
\begin{verbatim}
                             #Figure 4b
\end{verbatim}
\end{transbox}\vspace{-0.5mm}
%
\begin{mdframednoverticalspace}[style=secondfoc]
\begin{transbox}{18}{sil}
\begin{verbatim}
<<cresc>aspetta:Δ:+*:.>
        wait-IMP.2SG
\end{verbatim}
\hspace{1.1cm} wait
\end{transbox}
\end{mdframednoverticalspace}\vspace{2mm}
%
\begin{transbox}{19}{~}
\begin{verbatim}
--------------->Δ
\end{verbatim}
\end{transbox}
%
\begin{transbox}{20}{\textit{fab}}
\begin{verbatim}
----------------->+,,,-->
\end{verbatim}
\end{transbox}
%
\begin{transbox}{21}{~}
\begin{verbatim}
------------------>*
\end{verbatim}
\end{transbox}
%
\begin{transbox}{22}{~}
\begin{verbatim}
(0.1)+
\end{verbatim}
\end{transbox}\vspace{-1.5mm}
%
\begin{transbox}{23}{\textit{fab}}
\begin{verbatim}
,,,,,+
\end{verbatim}
\end{transbox}\bigskip

\begin{figure}
\caption{Frames from \extref{ex:rossi:6}, illustrating a “big” pointing gesture accompanying an imperative request.}
\label{fig:rossi:4}
\subfigure[Fabio uses a “big point” while saying \newline \textit{↑´{}bUttalo giù \`{}QUA;} ‘pour it \{in\} here’ \newline (lines 7--9).]{
  \includegraphics[height=.37\textheight]{figures/Rossi_Picture5a} % \label{fig:rossi:2a}
}
\subfigure[Fabio uses another “big point” while saying \textit{´{}bUttalo giù \`{}LÌ;} ‘pour it \{in\} there’ (lines 14--17).]{
  \includegraphics[height=.37\textheight]{figures/Rossi_Picture5b} % \label{fig:rossi:2b}
}
\end{figure}

While saying \textit{↑´{}bUttalo giù \`{}QUA;} ‘pour it \{in\} here’, Fabio produces a big point, with the arm stretched out and the finger fully extended to pick out the other container with precision, his gaze fixed on the referent (lines 6--9, \figref{fig:rossi:4}\textit{a}). When Silvio does not immediately comply, Fabio repeats the same composite utterance, changing only the deictic form (‘here’ → ‘there’) and using another big point (lines 11, 14--17, \figref{fig:rossi:4}\textit{b}).

In \citeauthor{EnfieldKitaRuiter2007}'s terms, Fabio's recruiting move here has a “location focus”, that is, it is about where Silvio should pour the soda and designed to direct him to another container which he has apparently not considered using for the current purpose. A big point here is fitted to locating and identifying the target container.

The second case involves Fabio initiating a recruitment with an analogous verbal form, an imperative, which is again coupled with a pointing gesture. This time, however, the point is “small”.

When \REF{ex:rossi:7} begins, the participants are debating over the qualities of vodka and gin, the two liquors they have on the table to make cocktails. Fabio and Silvio argue that gin is ‘disgusting’ and ‘tastes like shit’ (lines 3--4), which is reason for mixing it with a larger quantity of soft drink. Bino's subsequent repair initiation \textit{come fa cagare.} ‘what do you mean it tastes like shit’ (line 5) projects his disagreement with the assessment (see \citealt[279]{Rossi2015c}; \citealt{RaymondSidnell2019}). To settle the issue, Fabio initiates a recruitment for Bino to taste the gin.

\transheader{ex:rossi:7}{MasoPome\_1912588}\vspace{2mm}
%
\begin{transbox}{1}{bin}
\begin{verbatim}
ne  fa       due in più  questa del    gin.
PTV make-3SG two in more this   of-THE gin
\end{verbatim}
this contains two \{percent\} more \{alcohol\} than gin
\end{transbox}\vspace{1mm}
%
\emptytransbox{2}{(0.6)}\vspace{-0.5mm}
%
\begin{transbox}{3}{fab}
\begin{verbatim}
sì  vabè ma:[: il  gin fa       schifo.
yes PTC  but   the gin make-3SG disgust
\end{verbatim}
yes well but:: gin is disgusting
\end{transbox}\vspace{1mm}
%
\begin{transbox}{4}{sil}
\begin{verbatim}
            [sì  ma  il  gin fa       cagare   è      quello magari che:.
             yes but the gin make-3SG shit-INF be.3SG that   maybe  REL
\end{verbatim}
\hspace{1.85cm} yes but gin tastes like shit -- that's maybe what:
\end{transbox}\vspace{1mm}
%
\begin{transbox}{5}{bin}
\begin{verbatim}
come fa       cagare.
how  make-3SG shit-INF
\end{verbatim}
what do you mean it tastes like shit
\end{transbox}\vspace{1mm}
%
\begin{transbox}{6}{~}
\begin{verbatim}
(0.1)*(0.2)
\end{verbatim}
\end{transbox}
%
\begin{transbox}{7}{\textit{fab}}
\begin{verbatim}
     *turns and gazes at Bino-->
\end{verbatim}
\end{transbox}
%
\begin{mdframednoverticalspace}[style=firstfoc]
\begin{transbox}{8}{fab}
\begin{verbatim} 
↑ˇPRO*va     a +tastarne:    +en    #    go+↘ZA*T.+
 try-IMP.2SG to taste-INF=PTV one        drop
\end{verbatim}
\hspace{0.07cm} try taste some
\end{transbox}
\end{mdframednoverticalspace}\vspace{1mm}
%
\begin{transbox}{9}{~}
\begin{verbatim}
---->*gazes at gin---------------------------->*gazes back at Bino-->
\end{verbatim}
\end{transbox}
%
\begin{transbox}{10}{~}
\begin{verbatim}
               +.............+points at gin+,,,,,,+
\end{verbatim}
\end{transbox}\vspace{-0.25mm}
%
\begin{transbox}{11}{~}
\begin{verbatim}
                                    #Figure 5a
\end{verbatim}
\end{transbox}\vspace{-1mm}
%
\emptytransbox{12}{(0.2)}
%
\begin{mdframednoverticalspace}[style=firstfoc]
\begin{transbox}{13}{fab}
\begin{verbatim}
↘TA#s[ta,
taste-IMP.2SG
\end{verbatim}
taste
\end{transbox}
\end{mdframednoverticalspace}\vspace{2mm}
%
\begin{transbox}{14}{~}
\begin{verbatim}
   #Figure 5b                                
\end{verbatim}
\end{transbox}\vspace{-0.5mm}
%
\begin{mdframednoverticalspace}[style=secondfoc]
\begin{transbox}{15}{bin}
\begin{verbatim}
     [ma  no così*     liscio.
      but no like_this straight
\end{verbatim}
\hspace{0.8cm} well not straight like this
\end{transbox}
\end{mdframednoverticalspace}
%
\begin{transbox}{16}{\textit{fab}}
\begin{verbatim}
---------------->*
\end{verbatim}
\end{transbox}\bigskip

\begin{figure}
\caption{Frames from \extref{ex:rossi:7}, illustrating a “small” pointing gesture accompanying an imperative request.}
\label{fig:rossi:5}
\subfigure[Fabio uses a “small point” while say- \newline ing \textit{↑ˇPROva a tastarne: en go\`{}ZAT.} \newline ‘try taste some’ (lines 8--11).]{
  \includegraphics[height=.37\textheight]{figures/Rossi_Picture6a} % \label{fig:rossi:2a}
}
\subfigure[Fabio does not repeat the gesture as he says again \textit{\`{}TAsta,} ‘taste’ (lines 13--14).]{
  \includegraphics[height=.37\textheight]{figures/Rossi_Picture6b} % \label{fig:rossi:2b}
}
\end{figure}

The form of Fabio's pointing gesture here is quite different from what we have seen in \REF{ex:rossi:6}. It is articulated with the lower arm only and with the finger not fully extended (\figref{fig:rossi:5}\textit{a}). Also, instead of looking at the referent throughout, Fabio turns to Dino before speaking (line 7), then shifts his gaze to the gin, and then back to Bino before the end of the utterance (line 9). Fabio keeps looking at Bino also during the second imperative (\textit{\`{}TAsta,} ‘taste’, line 13), where notably he does \textit{not} redo the pointing gesture (\figref{fig:rossi:5}\textit{b}). 

All this contributes to characterizing Fabio's gesture here as a “small point”, conveying supplemental and possibly dispensable information. The recruitment here is not location focused; the goal is not to direct the recruitee to where he should put or do something, but rather to instigate action on a referent that has already been thematized (lines 3--5).

While there is only one bottle on the table containing straight gin (the glass bottle with the yellow label in \figref{fig:rossi:5}), a plausible alternative interpretation of Fabio's ‘try taste some’ is with reference to the gin mixed in the cocktail they have been making (contained in the green plastic bottles). In this context, a small point serves as an “informational safety net” \citep[1734]{EnfieldKitaRuiter2007}, available but inconspicuous, provided just in case the reference turned out to be ambiguous.

I conclude this section on composite recruiting moves by illustrating the use of iconic gesture. Although less frequent compared to pointing, iconic gestures are approximately as frequent as the other main types of nonverbal behavior in the Italian data (see \tabref{tab:rossi:1} above).\footnote{This finding is consistent with experimental research on the relatively high frequency of iconic gestures by Italian speakers in other contexts \citep{Campisi2014}.} \extref{ex:rossi:8} provides an example.

Before the extract begins, Rocco has unsuccessfully attempted to initiate a recruitment sequence with Loretta, who has been involved in a concurrent conversation. As the concurrent conversation ends with general laughter (line 1), Loretta finally answers Rocco's summons, clearing the way for Rocco's request.

\transheader{ex:rossi:8}{CampFamTavolo\_1803413}\vspace{2mm}
%
\begin{transbox}{1}{~}
\begin{verbatim}
(4.1) ((general laughter))
\end{verbatim}
\end{transbox}\vspace{1mm}
%
\begin{transbox}{2}{lor}
\begin{verbatim}
dimmi               scusa.
say-IMP.2SG=1SG.DAT excuse-IMP.2SG
\end{verbatim}
tell me -- sorry
\end{transbox}\vspace{1mm}
%
\emptytransbox{3}{(0.9) ((Rocco makes room for Romeo to sit on kitchen bench))}\vspace{-0.5mm}
%
\begin{mdframednoverticalspace}[style=firstfoc]
\begin{transbox}{4}{roc}
\begin{verbatim}
mi      passe´rEsti   +un  bicchier d'a+cqua:    #nor^MAΔle:+:;
1SG.DAT pass-COND-2SG  one glass    of=water      normal
\end{verbatim}
\{would\} you pass me a glass of plain water
\end{transbox}
\end{mdframednoverticalspace}\vspace{0.5mm}
%
\begin{transbox}{5}{~}
\begin{verbatim}
                      +................+makes iconic gesture+,,,-->
\end{verbatim}
\end{transbox}
%
\begin{transbox}{6}{\fig}
\begin{verbatim}
                                                 #Figure 6
\end{verbatim}
\end{transbox}\vspace{-0.75mm}
%
\begin{mdframednoverticalspace}[style=secondfoc]
\begin{transbox}{7}{\textit{lor}}
\begin{verbatim}
                                                        Δnods----->
\end{verbatim}
\end{transbox}
\end{mdframednoverticalspace}
%
\begin{transbox}{8}{~}
\begin{verbatim}
(0.1)Δ(0.2)+
\end{verbatim}
\end{transbox}
%
\begin{transbox}{9}{\textit{lor}}
\begin{verbatim}
---->Δ
\end{verbatim}
\end{transbox}
%
\begin{transbox}{10}{\textit{roc}}
\begin{verbatim}
---------->+
\end{verbatim}
\end{transbox}\vspace{0.5mm}
%
\begin{transbox}{11}{~}
\begin{verbatim}
((Loretta walks to sink to get water))
\end{verbatim}
\end{transbox}\vspace{0.75mm}
%
\begin{transbox}{~}{~}
\begin{verbatim}
((45 seconds not shown))
\end{verbatim}
\end{transbox}\vspace{0.5mm}
%
\begin{transbox}{12}{~}
\begin{verbatim}
((Loretta comes back with glass of water))
\end{verbatim}
\end{transbox}\vspace{-1.75mm}
%
\begin{transbox}{13}{roc}
\begin{verbatim}
grazie:,
\end{verbatim}
tha:nks
\end{transbox}\bigskip

\begin{figure}
\centering
\includegraphics[height=.35\textheight]{figures/Rossi_Picture7} %don't forget to include caption
\caption{Frame from \extref{ex:rossi:8}, lines 3--5. Rocco makes an iconic gesture while saying \textit{mi passe´{}rEsti un bicchier d'acqua nor\^{}MAle;} ‘\{would\} you pass me a glass of plain water’.}
\label{fig:rossi:6}
\end{figure}

Rocco's recruiting move includes an iconic gesture, with the thumb and index finger vertically aligned and kept at a distance (\figref{fig:rossi:6}). The gesture may represent the size or height of a drinking glass, or metaphorically refer to the amount of water contained in it.

\subsection{Verbal elements: construction types and subtypes}\label{sec:rossi:3.3}

In this section, I survey the range of options that Italian speakers have for designing the verbal component of recruiting moves. Italian speakers make use of all three main sentence types: imperatives, interrogatives, and declaratives. While imperatives are the most frequent, interrogatives and declaratives are also common (\tabref{tab:rossi:2}). The two cases of an “other” construction type feature the antecedent of a conditional sentence functioning as a main clause: (e.g. \textit{questo se me lo mettete dentro} ‘this one if you guys put it away for me’).\footnote{This format is more frequent in other languages examined in this volume (see Dingemanse, \chapref{sec:dingemanse}, \sectref{sec:dingemanse:3.2}).} Finally, Italian speakers also make use of utterances without a predicate, including noun phrases and single words. Since the use of such “minimal” utterances is sensitive to criteria related to those explained above for fully nonverbal forms (\sectref{sec:rossi:3.1}), I begin the analysis with these.

\begin{table}
\begin{tabularx}{.65\textwidth}{Xrr}
\lsptoprule
Construction type & Count & Proportion \\
\midrule
Imperative & 77 & 39\% \\
Interrogative & 50 & 25\% \\
Declarative & 49 & 25\% \\
No predicate & 21 & 11\% \\
Other & 2 & 1\% \\
\lspbottomrule
\end{tabularx}
\caption{Construction types in recruiting moves with a verbal component (\textit{n}=199).}
\label{tab:rossi:2}
\end{table}

\subsubsection{No predicate}\label{sec:rossi:3.3.1}
In \sectref{sec:rossi:3.1}, we saw that speakers do not use language when the action being recruited is projectable from the development of the ongoing activity \citep{Rossi2014}. However, projectability is not an all-or-nothing dimension. Besides \textit{fully} projectable actions, there are also \textit{partially} projectable actions, some element of which cannot be anticipated by the recruitee and therefore needs to be verbally specified \citep[54--57]{Rossi2015a}.

In \extref{ex:rossi:9}, the card players we met in \REF{ex:rossi:5} are starting a new game. Bianca begins to deal the cards, giving out two at a time (lines 1--4). As the players have discussed previously in the interaction, dealing two cards at a time increases the chance that cards will cluster in combinations from the prior game. Here, however, Bianca has apparently forgotten about this. 

\transheader{ex:rossi:9}{Circolo01\_1948857}\vspace{2mm}
%
\begin{transbox}{1}{~}
\begin{verbatim}
Δ(0.8)          Δ              
\end{verbatim}
\end{transbox}
%
\begin{transbox}{2}{\textit{bia}}
\begin{verbatim}
Δdeals two cardsΔ
\end{verbatim}
\end{transbox}
%
\begin{transbox}{3}{bia}
\begin{verbatim}
le  robe   se Δle      fa     senzaΔ: dove[rseΔle_]
the things RFL 3PL.SCL do-3PL without must-INF-3PL.ACC
\end{verbatim}
things should be done without having to
\end{transbox}\vspace{1mm}
%
\begin{transbox}{4}{~}
\begin{verbatim}
              Δdeals two more cardsΔ          Δdeals two more cards-->
\end{verbatim}
\end{transbox}
%
\begin{mdframednoverticalspace}[style=firstfoc]
\begin{transbox}{5}{fla}
\begin{verbatim}
                                          [ˆUnΔa. ]
\end{verbatim}
\hspace{6.4cm} one
\end{transbox}
\end{mdframednoverticalspace}\vspace{-0.5mm}
%
\begin{transbox}{6}{~}
\begin{verbatim}
(0.4)
\end{verbatim}
\end{transbox}
%
\begin{mdframednoverticalspace}[style=firstfoc]
\begin{transbox}{7}{fla}
\begin{verbatim}
ˆUna.Δ
\end{verbatim}
one
\end{transbox}
\end{mdframednoverticalspace}\vspace{1.75mm}
%
\begin{transbox}{8}{\textit{bia}}
\begin{verbatim}
---->Δ
\end{verbatim}
\end{transbox}
%
\begin{transbox}{9}{~}
\begin{verbatim}
(0.1)Δ(1.3)              Δ
\end{verbatim}
\end{transbox}\vspace{-1mm}
%
\begin{mdframednoverticalspace}[style=secondfoc]
\begin{transbox}{10}{\textit{bia}}
\begin{verbatim}
     Δtakes one card backΔ
\end{verbatim}
\end{transbox}
\end{mdframednoverticalspace}

Flavia initiates a recruitment for Bianca to alter the way she is dealing the cards by saying \textit{ˆUna.} ‘one’ (line 5). This first iteration of Flavia's “naming” is simultaneous with Bianca dealing a third pair of cards (line 4). As Bianca is still in the process of doing this, Flavia repeats the recruiting turn (line 7). Bianca then complies by taking back one of the two cards she has just dealt.

In this environment, most elements of the recruited action are projectable: the target object (cards) and the action to be done with it (dealing). What is not projectable in light of Bianca's ongoing conduct is the object's quantity, which is what gets named.

Other no-predicate cases include nominal references to the object to be passed or manipulated (e.g. \textit{coltello} ‘knife’) or to its location (e.g. \textit{quell'altro} ‘the other one’) or destination. A no-predicate recruiting format allows the recruiter to verbally specify only what is necessary, leaving out what is not (cf. \citealt{Mondada2011,Mondada2014b,SorjonenRaevaara2014}).

\subsubsection{Imperatives}\label{sec:rossi:3.3.2}

Imperatives are the most frequent construction type used by Italian speakers to get others to do things (see \tabref{tab:rossi:2}). The Italian language has both morphological and syntactic means to distinguish imperatives from interrogative and declarative sentence types. Imperative endings are available for the second person singular of verbs in the main conjugation class (e.g. \textit{parl-are} ‘to speak’, \textit{parl-i} ‘you speak’ vs. \textit{parl-a} ‘speak!’) and of certain irregular verbs. Another reliable cue, especially for morphologically ambiguous forms, is the position of clitic pronouns in the clause. In interrogatives and declaratives, clitic pronouns like \textit{mi} ‘to/for me’ precede the main verb (\textit{mi leggi un libro} ‘you read a book for me’); in imperatives, they follow it (\textit{leggimi un libro} ‘read me a book’).

\begin{figure}
\centering
\includegraphics[height=.36\textheight]{figures/Rossi_Picture8} %don't forget to include caption
\caption{Frame from \extref{ex:rossi:1}, line 2. Sergio says \textit{´{}PLInio; a´{}sciUga anche \`{}QUEsta.} ‘Plinio wipe this one too’ while Plinio is wiping washed cutlery.}
\label{fig:rossi:8}
\end{figure}

We already encountered examples of imperatively formatted recruiting moves in the previous sections. \extref{ex:rossi:1} exemplifies the typical environment for this construction type: before the recruitment sequence occurs, the participants have engaged in a joint activity or project (washing dishes) and the recruiting move is made made within this joint project to solicit an action that contributes to it (\textit{a´{}sciUga anche \`{}QUEsta.} ‘wipe this one too’) \citep{Rossi2012}. 

But previous examples also show that imperatives are not the only form occurring in such environments. Similar recruiting moves that further a joint project may also be formatted nonverbally \REF{ex:rossi:5} or without a predicate \REF{ex:rossi:9}. In order to be understood, nonverbal and no-predicate recruiting moves require the full or partial projectability of the target action. The use of an imperative, on the other hand, is sensitive to the action not being projectable. \extref{ex:rossi:1} again serves to illustrate this. When Sergio initiates a recruitment for Plinio to wipe the washed baking pan, Plinio is wiping cutlery (\figref{fig:rossi:8}). Wiping the baking pan is not a projectable next action at this point of the activity; it has to be “slotted into” what Plinio is currently doing \citep[318]{Rossi2014}. This is grounds for using a clausal form that fully specifies the target action.

Consider another case, which can be directly compared against the nonverbal and no-predicate recruiting moves in \REF{ex:rossi:5} and \REF{ex:rossi:9}. During the same card game, Flavia has just drawn a card that allows her to lay down a first combination (lines 1--2). Upon inspecting the cards played by Flavia, Bianca indicates a problem (line 4). She leans across the table and counts the cards while pointing at them (line 6). Then, after a brief pause, she tells Flavia \textit{´{}mEti zo ’n altro \`{}AMbo.} ‘put down another double’, which is needed to complete the combination. Moments later, Flavia fulfills the recruitment by laying down two sevens (line 11).

\transheader{ex:rossi:10}{Circolo01\_677062}\vspace{2mm}
%
\begin{transbox}{1}{fla}
\begin{verbatim}
[una due tre   quatro (che)  te      l'ho             pescada? (.) to':?
 one two three four   (CONN) 2SG.DAT 3SG.ACC=have-1SG draw-PCP     INTJ
\end{verbatim}
\hspace{0.07cm} one two three four -- I drew it (.) here we go
\end{transbox}\vspace{1mm}
%
\begin{transbox}{2}{~}
\begin{verbatim}
((lays cards down in a new combination))
\end{verbatim}
\end{transbox}
%
\begin{transbox}{3}{cla}
\begin{verbatim}
ah [per-?
oh because
\end{verbatim}
oh bec-
\end{transbox}\vspace{1mm}
%
\begin{transbox}{4}{bia}
\begin{verbatim}
   [<<f,h>´NO:_> ((leans forward across table))
\end{verbatim}
\hspace{1.5cm} no:
\end{transbox}\vspace{1mm}
%
\begin{transbox}{5}{sil}
\begin{verbatim}
por[ca miseria.
piggy misery
\end{verbatim}
holy cow
\end{transbox}\vspace{1mm}
%
\begin{transbox}{6}{bia}
\begin{verbatim}
   [due quatro::_ ((points at cards))
    two four
\end{verbatim}
\hspace{0.5cm} two four::
\end{transbox}\vspace{2mm}
%
\emptytransbox{7}{(1.2)}\vspace{-0.25mm}
%
\begin{mdframednoverticalspace}[style=firstfoc]
\begin{transbox}{8}{bia}
\begin{verbatim}
´mEti        zo   'n  altro ↘AMbo. ((keeps pointing at cards))
put-NPST-2SG down one other double
\end{verbatim}
put down another double
\end{transbox}
\end{mdframednoverticalspace}\vspace{1.5mm}
%
\emptytransbox{9}{(2.5) ((Flavia looks at cards in her hand))}
%
\begin{transbox}{10}{fla}
\begin{verbatim}
de sete  l'      g'ho;
of seven 3SG.ACC LOC=have-1SG
\end{verbatim}
I have one of sevens
\end{transbox}\vspace{3mm}
%
\emptytransbox{~}{((10 seconds not shown))}\vspace{-1mm}
%
\begin{mdframednoverticalspace}[style=secondfoc]
\begin{transbox}{11}{fla}
\begin{verbatim}
((lays down a double of sevens))
\end{verbatim}
\end{transbox}
\end{mdframednoverticalspace}

Bianca initiates the recruitment after Flavia has laid down an illegal combination of cards. The recruitment is aimed at solving a problem that has arisen during the game, but that was not projected by its structure. After Bianca first raises the problem (\textit{<{}<f,h>´{}NO:\_>} ‘no:’, line 4), Flavia's silence indicates her uncertainty as to how to proceed. Also, the fact that Bianca needs to count the cards before she can instruct Flavia (line 6) shows that the next relevant action is hard to anticipate. Here, Bianca's pointing to the incriminated cards would not be enough for Flavia to understand what to do next (cf. Extract \ref{ex:rossi:5}). The action being recruited needs to be fully articulated.

In sum, the imperative form is typically used to solicit actions that contribute to an already established joint project and that cannot be projected from its advancement \citep{Rossi2012,Rossi2014}. The imperative so used is usually bare and unmitigated \citep{Rossi2017}. Other less frequent uses of the imperative are more likely to be mitigated with additional elements (see (\sectref{sec:rossi:3.4}).

\subsubsection{Interrogatives}\label{sec:rossi:3.3.3}
Interrogatives are the second most frequent construction type after imperatives (see \tabref{tab:rossi:2}). As mentioned in \sectref{sec:rossi:1.1}, there are generally speaking no morphosyntactic means for distinguishing polar (yes/no) interrogatives from declaratives in Italian.\footnote{In many areas of Italy, speakers may alternate or mix the national language with a local Romance vernacular \citep{MaidenParry1997}. Regional Italian and vernacular are often inextricably interwoven in the speech of Italian speakers and both are integral parts of local Italian culture. In the province of Trento, where most my video recordings were made, the local Romance vernacular is the Trentino language. Unlike Italian, this language does have morphosyntactic means to distinguish between polar interrogatives and declaratives. This is due to the presence of subject clitics \citep{Lusini2013}, which are positioned before the main verb in declaratives (e.g. \textit{te gai} ‘you have’) and after the main verb in interrogatives (e.g. \textit{ga-t} ‘do you have’). See \REF{ex:rossi:18}  and \REF{ex:rossi:22} for examples.} At the same time, recent research by the author has documented the association of distinct intonation contours with specific types of polar questions, particularly questions involved in other-initiation of repair and related actions \citep{Rossi2015c,Rossi2020b}. This lends some support to the claim that intonation compensates, at least partly, for the lack of interrogative morphosyntax \citep[e.g.][]{GiliFivelaEtAl2015}, though this should not be taken to imply a straightforward mapping between intonation and polar questions as a whole \citep{Rossano2010}. % , which can be pragmatically realized using various interactional resources

The analysis of recruitments provides further evidence for the role of intonation in marking interrogative utterances in Italian. Recruiting turns that make relevant acceptance or confirmation in the form of a polar answer (see Extracts \ref{ex:rossi:8}, \ref{ex:rossi:11}, \ref{ex:rossi:12}) are normally produced with either a \textsc{rise-fall} or a \textsc{rise from low} intonation contour. In the main variety of Italian spoken in my corpus -- Trentino Italian -- these intonation contours are of the same type found on requests for confirmation and questioning repetitions \citep{Rossi2015c,Rossi2020b}, and are both distinct from the \textsc{hat-pattern} and \textsc{fall} contours that are instead found on imperative and declarative recruiting turns \citep{Rossi2011}. These intonation contours fulfill a criterion of formal distinguishability between construction types in that they “form a system of alternative choices that are mutually exclusive” \citep[278]{KönigSiemund2007}. On this account, I refer to recruiting turns systematically produced with intonation contours associated with polar questions as interrogatives.

There are three main subtypes of interrogative used in recruiting moves in Italian, and they are found in different interactional environments. The most frequent subtype is what I refer to as the simple interrogative \citep[chap. 3]{Rossi2015a}, which can be rendered in English with ‘will you x’. Unlike its English translation, however, the construction does not contain any modal verb but only an action verb inflected for second person, simply asking if the recipient is going to do something (lit. ‘you x?’). The action verb is typically preceded by a first person dative pronoun \textit{mi} ‘to/for me’ expressing that the action is directed to, or for the benefit of, the speaker. The use of this interrogative subtype is illustrated in the following example, where a group of friends are playing cards.

Before the extract begins, Franco has gotten himself a piece of paper towel from a cabinet next to him in order to blow his nose. As he finishes wiping his nose (line 1), he turns back to the table (line 3), reengaging in the game. This is the context in which Beata recruits him to get a piece of paper towel for her too.

\transheader{ex:rossi:11}{CampUniTaboo01\_172458}\vspace{2mm}
%
\xtransbox{1}{fra}{((finishes wiping nose, folds paper towel, puts it into pocket))}\vspace{-0.75mm}
%
\begin{transbox}{2}{san}
\begin{verbatim}
è      veramente comunque per[verso [(   )
be.3SG really    anyway   perverse
\end{verbatim}
anyway \{that thing\} is really perverse (\hspace{0.5cm})
\end{transbox}\vspace{1.5mm}
%
\begin{transbox}{3}{fra}
\begin{verbatim}
                             [((turns back to table))
\end{verbatim}
\end{transbox}\vspace{-0.5mm}
%
\begin{mdframednoverticalspace}[style=firstfoc]
\begin{transbox}{4}{bea}
\begin{verbatim}
                                    [mi      b- ˆDAi
                                     1SG.DAT b- give-2SG
\end{verbatim}
\hspace{5.4cm} \{will\} you b- give
\end{transbox}
\end{mdframednoverticalspace}\vspace{1.5mm}
%
\begin{transbox}{5}{~}
\begin{verbatim}
anche a ↓↓me      un pezzo di <<creaky>↘scOt[tex.>
also  to  1SG.ACC a  piece of          paper_towel  
\end{verbatim}
a piece of paper towel to me too
\end{transbox}\vspace{0.25mm}
%
\begin{mdframednoverticalspace}[style=secondfoc]
\begin{transbox}{6}{fra}
\begin{verbatim}
                                            [sì; ((turns around to get))
\end{verbatim}
\hspace{6.65cm} yes
\end{transbox}
\end{mdframednoverticalspace}\vspace{1mm}
%
\begin{transbox}{7}{san}
\begin{verbatim}
è      è      veramente per[verso il::_ il  trabicolo   lì.
be.3SG be.3SG really    perverse  the   the contraption there
\end{verbatim}
it's it's really perverse the:: the contraption there
\end{transbox}\vspace{1mm}
%
\begin{transbox}{8}{fra}
\begin{verbatim}
                           [((holds paper towels out across table))
\end{verbatim}
\end{transbox}
%
\begin{transbox}{9}{~}
\begin{verbatim}
(0.3) ((Beata tears off paper towel))
\end{verbatim}
\end{transbox}
%
\begin{transbox}{10}{san}
\begin{verbatim}
il tre[piedi.
\end{verbatim}
the tripod
\end{transbox}
%
\begin{transbox}{11}{bea}
\begin{verbatim}
      [sì:_
\end{verbatim}
\hspace{0.95cm} ye:s
\end{transbox}\bigskip

\label{par:rossi:unilateral} Unlike the imperative and no-predicate recruiting moves examined above, the request here is not part of a joint project. For one thing, it is unrelated to the ongoing game. Also, it is made at a point when Franco has just completed his own individual course of action with the paper towel and repositioned his body to reengage in the game with the other players. Turning back to get another piece of paper towel requires him to disengage from the game again. %[...] Departure. Discontinuity. 
% When a simple interrogative is used,
Such relation of discontinuity typically goes together with the fact that the action being recruited is in the interest of the requester as an individual. Rather than contributing to a shared goal, the action benefits the recruiter alone (\citealt{Rossi2012,Rossi2015a}: chap. 3).

\label{par:rossi:unwillingness} Another subtype of interrogative is \textit{puoi x} ‘can you x’, a modal construction asking about the ability of the recipient to do something. Much like in simple interrogative sequences, actions recruited using \textit{puoi x} ‘can you x’ typically involve a departure from what the recruitee is currently doing. At the same time, what distinguishes the usage of \textit{puoi x} ‘can you x’ is an anticipation of the recruitee's unwillingness to comply (\citealt[chap. 4]{Rossi2015a}; cf. Zinken, \chapref{sec:zinken}, \sectref{sec:zinken:3.3.4}). In \REF{ex:rossi:2}, for example, Olga asks Luca if he ‘can put the dessert’ back on the tray (line 12) after a first attempt to get him to do so, which he has resisted (lines 8--11). By using a \textit{puoi x} ‘can you x’ interrogative, Olga recognizes the problematic nature of the recruitment and attempts to overcome Luca's unwillingness by appealing to his cooperativeness, or put another way, by persuading him \citep[cf.][280--282]{ZinkenOgiermann2011}. This orientation is reflected also in Olga's subsequent use of an enticement: ‘I'll give you this’ (line 16).

The third subtype of interrogative is \textit{hai x} ‘do you have x’, a construction asking if the recipient is in possession of an object. \extref{ex:rossi:12} gives us an example. A group of people are hanging out in the living room. Snacks and drinks are on the table, including beer and juice, but not milk.

\transheader{ex:rossi:12}{DopoProve09-2\_293350}\vspace{2mm}
%
\begin{transbox}{1}{mag}
\begin{verbatim}
Ada,
NAME
\end{verbatim}
Ada
\end{transbox}\vspace{2mm}
%
\begin{transbox}{2}{ada}
\begin{verbatim}
((looks up))
\end{verbatim}
\end{transbox}
%
\begin{mdframednoverticalspace}[style=firstfoc]
\begin{transbox}{3}{mag}
\begin{verbatim}
^HAi     un  goccio di ^lAtte.
have-2SG one drop   of milk
\end{verbatim}
do you have a bit of milk
\end{transbox}
\end{mdframednoverticalspace}\vspace{1.5mm}
%
\emptytransbox{4}{(0.5)}\vspace{-0.5mm}
%
\begin{mdframednoverticalspace}[style=secondfoc]
\begin{transbox}{5}{ada}
\begin{verbatim}
mm hm[::? ((nods))
\end{verbatim}
mm hm::
\end{transbox}
\end{mdframednoverticalspace}\vspace{2mm}
%
\begin{transbox}{6}{min}
\begin{verbatim}
     [vuoi     il  succo? ((to Magda))
      want-2SG the juice
\end{verbatim}
\hspace{0.8cm} do you want juice
\end{transbox}\vspace{0.75mm}
%
\emptytransbox{7}{(0.5)}\vspace{-0.5mm}
%
\begin{transbox}{8}{mag}
\begin{verbatim}
[no grazie (   )
\end{verbatim}
\hspace{0.07cm} no thanks (\hspace{0.5cm})
\end{transbox}
%
\begin{mdframednoverticalspace}[style=secondfoc]
\begin{transbox}{9}{ada}
\begin{verbatim}
[((stands up and walks to kitchen))
\end{verbatim}
\end{transbox}
\end{mdframednoverticalspace}

In line 1, Magda addresses Ada -- the group's host -- and asks if she has milk, which is not among the beverages available on the table. Ada responds with a positive polar token (\textit{mm hm::?}, line 5), accompanied by nodding, and shortly after proceeds to fulfill the request (line 9).

The availability of an object is a precondition -- a material and practical prerequisite -- for the object to be passed or utilized by someone. In recruitment sequences, the function of a \textit{hai x} ‘do you have x’ interrogative is to check an object's availability when this is uncertain, for example because the object is not visible (see also Floyd, \chapref{sec:floyd}, \sectref{sec:floyd:3.3.3}; Enfield, \chapref{sec:enfield}, \sectref{sec:enfield:4.3.1}). This subtype of interrogative, in other words, works as a pre-request (see \citealt{Rossi2015b} and references therein). If the target object is available, the recruitee often responds by fulfilling the projected request immediately (see also \citealt{Fox2015}), as in \REF{ex:rossi:12}. Other response affordances of this form are illustrated in \sectref{sec:rossi:4.2.2} below.\\

\noindent % To conclude this section (\sectref{sec:rossi:3.3.3})
Regardless of subtype, interrogative recruiting moves make fulfillment of the recruitment contingent upon the recruitee's response. This distinguishes them from imperative recruiting moves, which instead assume compliance. One reason for a recruiter not to assume compliance is that the action being recruited is unrelated to what the recruitee is doing and, rather than contributing to a joint project, serves an individual goal of the recruiter. This is when a simple interrogative is normally used. If, in addition, the recruiter anticipates that the recruitee may be unwilling to comply, they can select a semantically and syntactically more complex interrogative -- \textit{puoi x} ‘can you x’ -- to recognize the problematic or delicate nature of the recruitment. Yet another reason for not assuming compliance is when a precondition for recruitment is uncertain. When the object to be passed or utilized may not be available, recruiters can use a \textit{hai x} ‘do you have x’ interrogative to check on this.

\subsubsection{Declaratives}\label{sec:rossi:3.3.4}
Declarative recruiting moves are as frequent as interrogative ones in Italian (see \tabref{tab:rossi:2}) and, like interrogatives, fall into three main subtypes. The first subtype is personal modal declaratives, which include constructions expressing a person's obligation or necessity to do something, such as \textit{devi x} ‘you have to x’ or ‘you must x’.

In \REF{ex:rossi:13}, Sofia and Furio are making cookies in Furio's kitchen. Before the extract begins, Sofia has left the table to weigh some of the ingredients on a scale. In line 1, she complains that she is having trouble turning the scale on.

\transheader{ex:rossi:13}{BiscottiMattina01\_3000055}\vspace{2mm}
%
\begin{transbox}{1}{sof}
\begin{verbatim}
non si  accende?    non so, ((fiddles with scale))
not RFL turn_on-3SG not know-1SG
\end{verbatim}
it doesn't turn on -- I don't know
\end{transbox}\vspace{2mm}
%
\begin{mdframednoverticalspace}[style=firstfoc]
\begin{transbox}{2}{fur}
\begin{verbatim}
devi     clic¯cAre:: [´plUrime ˇVOLte.
must-2SG click-INF    multiple times
\end{verbatim}
you have to press:: multiple times
\end{transbox}
\end{mdframednoverticalspace}\vspace{2mm}
%
\begin{mdframednoverticalspace}[style=secondfoc]
\begin{transbox}{3}{sof}
\begin{verbatim}
                     [mhm. ((presses button again))
\end{verbatim}
\hspace{3.2cm} mhm
\end{transbox}
\end{mdframednoverticalspace}
%
\begin{transbox}{4}{~}
\begin{verbatim}
fatto;
\end{verbatim}
done
\end{transbox}
%
\begin{transbox}{5}{fur}
\begin{verbatim}
devi     convincerla.
must-2SG convince-INF=3SG.ACC
\end{verbatim}
you have to persuade it
\end{transbox}\bigskip

Furio's recruiting move is responsive to Sofia's trouble. As she fiddles with the scale and signals a problem, Furio instructs her how to solve it. Sofia then complies and announces that she has succeeded. Note that, after the recruitment sequence is complete, Furio uses again the same \textit{devi x} ‘you have to x’ form to reiterate how Sofia should handle the scale (line 5). While still connected to what Sofia has just done, the instruction in this position no longer refers to a here-and-now action and acquires broader temporal scope or applicability. This follow-up by Furio sheds light on the social-interactional import of \textit{devi x} ‘you have to x’ relative to other recruiting formats. Similarly to the imperative, this form can be used to solicit a contribution to an undertaking that has already been committed to by the recruitee (see \sectref{sec:rossi:3.3.2}). However, while an imperative directs the recruitee to perform a here-and-now, one-off action, a \textit{devi x} ‘you have to x’ declarative imparts an instruction that transcends the local circumstances and is applicable in the future (see \citealt{Parry2013,Raevaara2017}; cf. \citealt[117--130]{Zinken2016}). Pressing the scale's button multiple times to turn it on is relevant not only for Sofia's current purpose but more generally every time she will have to operate the scale.

The second subtype of declarative is constituted by impersonal deontic constructions like \textit{bisogna x} ‘it is necessary to x’, which express the obligation or necessity to do an action without tying it to a particular individual (see also Floyd, \chapref{sec:floyd}, \sectref{sec:floyd:3.3.4}; Zinken, \chapref{sec:zinken}, \sectref{sec:zinken:3.3.2}; Baranova, \chapref{sec:baranova}, \sectref{sec:baranova:3.3.3}).\footnote{In Italian, this can be grammatically achieved by using an impersonal verb (e.g. \textit{bisogna tagliare il pane} ‘it is necessary to cut the bread’) or by intransitive constructions with a non-human subject (e.g. \textit{c’è il pane da tagliare} ‘the bread is to be cut’).} Impersonal deontic declaratives have a complex pragmatics that extends beyond recruitment (see \citealt{ZinkenOgiermann2011,RossiZinken2016}). That said, an important affordance of this declarative subtype as a recruiting format is its potential to make participation in the necessary action negotiable. This means that different individuals may have to sort out who will take on the action.

\extref{ex:rossi:14} is taken from the same interaction as \REF{ex:rossi:4}. Sergio, Greta and Dino are chatting while Sergio styles Greta's hair. Before the extract begins, Greta has asked Sergio to remove a ‘thingy’ from her forehead, which turns out to be a wisp of hair (line 1). When Sergio realizes that the hair has glued up on Greta's forehead because some dye has run down on it, he initiates a recruitment sequence using an impersonal deontic declarative.

\transheader{ex:rossi:14}{Tinta\_ 2051380}\vspace{2mm}
%
\begin{transbox}{1}{ser}
\begin{verbatim}
[questo_ ((holds wisp of hair))
 this
\end{verbatim}
\hspace{0.07cm} this
\end{transbox}\vspace{2mm}
%
\begin{transbox}{2}{gre}
\begin{verbatim}
[(eh  non lo      so)       c'ho         un coso;>
 (PTC not 3SG.ACC know-1SG) LOC=have-1SG a  thingy
\end{verbatim}
\hspace{0.07cm} (well dunno) I have a thingy
\end{transbox}\vspace{1mm}
%
\begin{mdframednoverticalspace}[style=firstfoc]
\begin{transbox}{3}{ser}
\begin{verbatim}
scusa *↘SÌ +bisogna   + * puΔ¯lI•re=
sorry  yes  necessitate-3SG clean-INF 
\end{verbatim}
sorry yes it is necessary also to wipe
\end{transbox}
\end{mdframednoverticalspace}\vspace{1mm}
%
\begin{transbox}{4}{~}
\begin{verbatim}
      *gazes at Dino--->*gazes back to Greta's head-->
\end{verbatim}
\end{transbox}\vspace{-0.5mm}
%
\begin{transbox}{5}{~}
\begin{verbatim}
           +moves hand+
\end{verbatim}
\end{transbox}\vspace{-1.5mm}
%
\begin{mdframednoverticalspace}[style=secondfoc]
\begin{transbox}{6}{\textit{din}}
\begin{verbatim}
                            Δturns to paper towel and reaches for it-->>
\end{verbatim}
\end{transbox}
%
\begin{transbox}{7}{\textit{gre}}
\begin{verbatim}
                                •reaches for paper towel-------------->>
\end{verbatim}
\end{transbox}
\end{mdframednoverticalspace}
%
\begin{transbox}{8}{ser}
\begin{verbatim}
=an[che *la cre]ma         *dalla <<creaky>↘FRONte.>
 also    the cream          from-the       forehead
\end{verbatim}
\hspace{0.07cm} the dye from the forehead
\end{transbox}\vspace{1mm}
%
\begin{transbox}{9}{~}
\begin{verbatim}
------->*gazes back at Dino*gazes back to Greta's head-->>
\end{verbatim}
\end{transbox}\vspace{-1.5mm}
%
\begin{transbox}{10}{din}
\begin{verbatim}
   [faccio io. ]
    do-1SG 1SG.NOM
\end{verbatim}
\hspace{0.5cm} I'll do it
\end{transbox}\bigskip

Wiping the dye away could in principle be taken on by any of the three participants, including the recruiter himself. Sergio is most immediately involved in the styling process and his apology \textit{scusa} ‘sorry’ indicates that he is responsible for having let the dye drip on Greta's forehead. While saying the word \textit{bisogna} ‘it is necessary to’, Sergio moves a hand (line 5), possibly in the direction of the paper towel, but then hesitates. At the same time, he gazes at Dino (line 4), inviting him to get involved (see \citealt[chap. 3]{StiversRossano2010,Rossano2012}). 

Dino is arguably in a better position to do the wiping, one reason being that Sergio is wearing gloves that are stained with dye. Also, Dino has already assisted Sergio earlier in the styling process, seeing to similar side tasks such as cleaning. Here, too, Dino steps in to help, turning toward the paper towel on the table and reaching for it (line 6). As he begins to reach, however, Greta does the same (line 7). In the midst of this, Dino verbalizes his intention to take on the task (\textit{faccio io.} ‘I'll do it’, line 10). It is not clear whether this verbal response is addressed primarily to Greta or Sergio; regardless, it reflects a negotiation over who should fulfill the recruitment.
% This happens while Sergio is still in the process of completing the impersonal deontic statement (\textit{anche la crema dalla <<creaky>`{}FRONte.>} ‘the dye from the forehead’, line 8)

This example shows that an impersonal deontic declarative such as \textit{bisogna x} ‘it is necessary to x’ does not constrain participation in the action being recruited and can make a response relevant for multiple people. Although the responsibility for the action in question sometimes falls on a specific person (see \citealt{RossiZinken2016}), an impersonal deontic declarative can generate a negotiation of who the doer is ultimately going to be.

The third declarative subtype is constituted by factual declaratives: non-modal constructions that present a description of a state of affairs. Although the format cannot be defined by a single lexicosyntactic formula, they often refer to the lack of something (e.g. \textit{manca sale} ‘there isn't enough salt’), the reaching of a stage in a process (e.g. \textit{bolle l'acqua} ‘the water is boiling’), a property or quality of an object (e.g. \textit{questo è un po’ unticcio} ‘this is a bit slimy’), or an untoward circumstance (e.g. \textit{i piatti stanno bloccando lo scarico} ‘the dishes are blocking the drain’). 

Like impersonal deontic declaratives, factual declaratives do not specify a recruitee. In addition, they also do not specify the action being recruited. When using a factual declarative, the recruiter relies on the recruitee's ability to infer the target action on the basis of a shared understanding of the practical circumstances (see also Kendrick, \chapref{sec:kendrick}, \sectref{sec:kendrick:4.2.3}; Enfield, \chapref{sec:enfield}, \sectref{sec:enfield:4.3.1}; Baranova, \chapref{sec:baranova}, \sectref{sec:baranova:3.3.3}; Dingemanse, \chapref{sec:dingemanse}, \sectref{sec:dingemanse:3.2.2}). 

Utterances such as \textit{it's cold in here} or \textit{the matches are all gone} have been traditionally referred to as “indirect requests” or “hints” that allow the speaker not to commit to a request intention, leaving interpretation up to the recipient, and thus affording the option not to get involved (\citealt[42]{ErvinTripp1976}; \citealt[69, 216]{BrownLevinson1987}; \citealt{Weizman1989}). But using a factual declarative is not simply a matter of indirectness. More important, it allows the speaker to do more than just getting another person to do something. In everyday informal interaction, a recurrent function of factual declaratives alongside initiating recruitment is to inform the recipient of something they do not know \citep{Rossi2018}.

In \REF{ex:rossi:15}, Mirko is working with others in the kitchen. At the beginning of the extract, Emma walks in, addresses Mirko, and tells him that ‘the feed drip has finished’, referring to the intravenous drip being administered to a family member in another room.

\transheader{ex:rossi:15}{Camillo\_ 2039498}\vspace{2mm}
%
\begin{transbox}{1}{emm}
\begin{verbatim}
Mirko.
NAME
\end{verbatim}
Mirko
\end{transbox}\vspace{1.5mm}
%
\begin{transbox}{2}{mir}
\begin{verbatim}
sì?
\end{verbatim}
yes
\end{transbox}\vspace{1mm}
%
\emptytransbox{3}{(0.5)}\vspace{-1mm}
%
\begin{mdframednoverticalspace}[style=firstfoc]
\begin{transbox}{4}{emm}
\begin{verbatim}
volevo        ↘DIRte          ↑¯chE è      finì       la ↓↘FLEbo.
want-IMPF-1SG say-INF=2SG.DAT  COMP be.3SG finish-PCP the feed_drip
\end{verbatim}
I wanted to tell you that the feed drip has finished
\end{transbox}
\end{mdframednoverticalspace}\vspace{1mm}
%
\emptytransbox{5}{(0.3)}\vspace{-1mm}
%
\begin{mdframednoverticalspace}[style=secondfoc]
\begin{transbox}{6}{mir}
\begin{verbatim}
a::h.
INTJ
\end{verbatim}
o::h
\end{transbox}
\end{mdframednoverticalspace}\vspace{1.5mm}
%
\emptytransbox{7}{(0.8)}\vspace{-1mm}
%
\begin{mdframednoverticalspace}[style=secondfoc]
\begin{transbox}{8}{mir}
\begin{verbatim}
buono_ possiamo liberare la  Milena allora.
good   can-1PL  free-INF the NAME   then
\end{verbatim}
good we can release Milena then
\end{transbox}
\end{mdframednoverticalspace}\vspace{-1mm}
%
\begin{transbox}{9}{emm}
\begin{verbatim}
eh.
INTJ
\end{verbatim}
right
\end{transbox}\bigskip

The focal content of Emma's turn (\textit{è finì la ↓\`{}FLEbo.} ‘the feed drip has finished’) is prefaced by a formulation of the turn as an informing (\textit{volevo \`{}DIRte ↑¯{}chE} ‘I wanted to tell you that’). This characterization of Emma's action is consonant with Mirko's first response in the form of a change-of-state token \textit{a::h} ‘o::h’ \citep{Heritage1984a}, which signals that his state of knowledge has changed and thus receipts the information reported by Emma as news (line 6). A moment later, Mirko expands his response with another unit, which includes an assessment of the news as ‘good’ and then a commitment to going and nursing Milena (‘we can release Milena then’), showing his understanding of Emma's action not only as an informing but also as a request.

So factual declaratives are often used to inform the recruitee of something they do not know, which functions as a vehicle for recruiting their assistance or collaboration \citep{Rossi2018}.

\subsection{Additional verbal elements}\label{sec:rossi:3.4}
This section looks at verbal elements in the recruiting turn beyond the basic linguistic frame created by the construction type and subtype being used. As discussed in \chapref{sec:coding}, \sectref{sec:coding:6}, additional verbal elements tend to fall into four main categories: vocatives (e.g. \textit{\textbf{´{}PLInio;} a´{}sciUga anche \`{}QUEsta.} ‘Plinio wipe this one too’, \extref{ex:rossi:1}), benefactives (e.g. \textit{tieni\textbf{mi} questi un attimo} ‘hold these for me one second’), explanations, and mitigators or strengtheners. The following subsections focus on the latter two categories and illustrate their usage in the context of imperative recruiting turns, to make comparison easier with cases examined in earlier sections.

\subsubsection{Explanations}\label{sec:rossi:3.4.1}
Explanations, accounts, and more generally reason-giving occur at various places in interaction (see \citealt{Goodwin1987,Antaki1994,Drew1998,Waring2007,Parry2009,BoldenRobinson2011}, among others). In recruitment sequences, explanations refer to circumstances that are grounds for the recruitment to be initiated or that make it more understandable or warranted \citep[see][]{Parry2013,BaranovaDingemanse2016,Rossi2017}.\footnote{To count as an additional element rather than as a stand-alone recruiting turn, the explanation must be produced as an appendage to a construction type among those surveyed in \sectref{sec:rossi:3.3}.} % , forming a single package with it (see \chapref{sec:coding})

\extref{ex:rossi:16} is taken from the same interaction as \REF{ex:rossi:1}. Plinio and Rocco are in charge of drying the dishes that others are washing. As they wait for the next round of washed dishes to dry, Plino picks up a dishwasher tray and asks if it is going to be used again (line 1). After Agnese responds with ‘no’, Plinio puts the tray away (line 4). Shortly after this, Rocco tells Plinio to put away another tray that is lying on the floor.

\transheader{ex:rossi:16}{CampFamLava\_591294}\vspace{2mm}
%
\begin{transbox}{1}{pli}
\begin{verbatim}
questo servirà       ancora; ((holds up white tray)) 
this   serve-FUT-3SG again/still
\end{verbatim}
is this going to be used again
\end{transbox}\vspace{1mm}
%
\emptytransbox{2}{(2.1)}\vspace{-1mm}
%
\begin{transbox}{3}{agn}
\begin{verbatim}
no.
\end{verbatim}
no
\end{transbox}\vspace{2mm}
%
\emptytransbox{4}{(5.0) ((Plinio puts white tray away))}
%
\emptytransbox{5}{(9.5) ((Plinio wanders between sink and dishwasher))}
%
\begin{mdframednoverticalspace}[style=firstfoc]
\begin{transbox}{6}{roc}
\begin{verbatim}
ˇMETti       via  anche quello lì    ↘GIALlo ((points at yellow tray))
put-NPST-2SG away also  that   there yellow
\end{verbatim}
put away that yellow one too
\end{transbox}\vspace{0.75mm}
\end{mdframednoverticalspace}
%
\begin{transbox}{7}{~}
\begin{verbatim}
che  se no gli     pestiam  ↘SOpra;
CONN if no 3SG.DAT step-1PL above
\end{verbatim}
otherwise we're going to step on it
\end{transbox}
%
\begin{mdframednoverticalspace}[style=secondfoc]
\begin{transbox}{8}{pli}
\begin{verbatim}
((picks yellow tray up and puts it away))
\end{verbatim}
\end{transbox}
\end{mdframednoverticalspace}

The recruitment is initiated within a joint project that recruiter and recruitee are involved in (see \sectref{sec:rossi:3.3.2}). The explanation appended to the imperative recruiting turn (\textit{che se no gli pestiam \`{}SOpra;} ‘otherwise we're going to step on it’) indicates that the recruitment is in the interest of both participants, with the goal of preventing an unwanted consequence, and thus articulates and specifies the contribution of the recruitment to their joint project. Of 17 explanations added to imperative recruiting turns in the Italian sample, 13 have an analogous function. For a more detailed account of the interactional processes involved in reason-giving for recruitments, see \cite{BaranovaDingemanse2016}.

\subsubsection{Mitigators}\label{sec:rossi:3.4.2}
Recruiting moves can include design features to mitigate or soften the imposition on the recruitee or, alternatively, to emphasize the urgency of the action being recruited (see \citealt{BrownLevinson1987,BlumKulkaHouseKasper1989}). The following case gives us an example of mitigation. 

Extract \ref{ex:rossi:17} is taken from the same card game as other examples examined above. Teammates Bianca and Flavia are consulting on their next move, while Clara and Silvia are waiting for their turn. During the wait, Silvia takes a piece of cake from a shared plate on the table (line 4). This occasions Clara's initiation of recruitment.

\transheader{ex:rossi:17}{Circolo01\_1270484}\vspace{2mm}
%
\begin{transbox}{1}{bia}
\begin{verbatim}
se te  ghe n'hai        doi?
if SCL LOC PTV=have-2SG two
\end{verbatim}
if you have two of them
\end{transbox}\vspace{1mm}
%
\begin{transbox}{2}{fla}
\begin{verbatim}
no nó  ghe n'ho         doi no.
no not LOC PTV=have-1SG two no
\end{verbatim}
no I don't have two of them
\end{transbox}\vspace{2mm}
%
\begin{transbox}{3}{~}
\begin{verbatim}
(0.3)Δ(0.6)Δ(0.9)                         Δ(0.2)+(0.3)
\end{verbatim}
\end{transbox}
%
\begin{transbox}{4}{\textit{sil}}
\begin{verbatim}
     Δ.....Δtakes piece of cake from plateΔ,,,,,,,,,,,-->
\end{verbatim}
\end{transbox}
%
\begin{transbox}{5}{\textit{cla}}
\begin{verbatim}
                                                +.....-->
\end{verbatim}
\end{transbox}\vspace{-0.5mm}
%
\begin{mdframednoverticalspace}[style=firstfoc]
\begin{transbox}{6}{cla}
\begin{verbatim}
´dAme                quel +Δmigolin ¯LÌ  Δva´lÀ per +pia↘ZER.
give-IMP.2SG=1SG.DAT that   crumble there PTC   for  favor
\end{verbatim}
give me that tiny piece there please \textit{valà} ((≈ will you))
\end{transbox}
\end{mdframednoverticalspace}\vspace{1.5mm}
%
\begin{transbox}{7}{~}
\begin{verbatim}
..........................+points at cake---------->+
\end{verbatim}
\end{transbox}
%
\begin{mdframednoverticalspace}[style=secondfoc]
\begin{transbox}{8}{\textit{sil}}
\begin{verbatim}
,,,,,,,,,,,,,,,,,,,,,,,,,,,Δ             Δ.............-->
\end{verbatim}
\end{transbox}
\end{mdframednoverticalspace}
%
\begin{transbox}{9}{~}
\begin{verbatim}
Δ(0.5)              
\end{verbatim}
\end{transbox}
%
\begin{transbox}{10}{~}
\begin{verbatim}
Δtakes another piece and passes it to Clara-->>
\end{verbatim}
\end{transbox}\vspace{-2mm}
%
\begin{transbox}{11}{cla}
\begin{verbatim}
grazie,
\end{verbatim}
thanks
\end{transbox}\bigskip

The recruiting turn includes two mitigators: \textit{per piazer} ‘please’ and \textit{valà}, a northern Italian particle which in this context can be rendered with the English tag ‘will you’ -- an appeal to the recipient's benevolence or goodwill. These two additional elements mark the imperative request as requiring some kind of redress \citep{BrownLevinson1987}. Such mitigators are normally not found in imperative requests of the kind illustrated in \REF{ex:rossi:1}, \REF{ex:rossi:10}, and \REF{ex:rossi:16}. Take \REF{ex:rossi:10}, for instance, which takes place during the same card game. In that sequence, the request contributes to the progress of the card game. In \REF{ex:rossi:17}, by contrast, the request is for a good to be consumed by the requester alone, not unlike requests designed with a simple interrogative \REF{ex:rossi:11}. 

There is no space here to discuss the conditions that support the use of an imperative in \REF{ex:rossi:17} (see \citealt{Rossi2017} for an account). What is important to note is that the imperative request here differs functionally and interactionally from those seen above (see \sectref{sec:rossi:3.3.2}), and that this difference is associated with the use of mitigators.\\

\noindent
To sum up this whole section on Move A (\sectref{sec:rossi:3}), I have surveyed a range of verbal and nonverbal resources that speakers of Italian have at their disposal for initiating recruitment. The use of fully nonverbal forms (e.g. simply pointing or reaching toward an object) is generally constrained to contexts that afford the projectability of the action being recruited. If language is needed to specify the action, Italian speakers calibrate the verbal component of the recruiting turn from phrasal or single-word formats to clausal ones. The use of alternative clausal types (imperative, interrogative, declarative) and subtypes is sensitive to a range of factors including the sequential and functional relation of the recruitment to what the recruitee is currently doing, the benefit brought by the action being recruited, the availability of objects, the anticipation of the recruitee's unwillingness, the negotiability of participation, and the performance of other actions (e.g. informing) as a vehicle for getting another to do something. I have also observed patterns in the use of pointing gestures and noted the frequency of iconic gestures in recruiting moves. Finally, the verbal component of a recruiting move can be enriched beyond the basic linguistic frame being used with additional elements. Focusing on imperative recruiting moves, we have seen that recruiters may add explanations to articulate the contribution of the recruitment to an ongoing joint project, or alternatively they may add mitigators to soften the use of an imperative format outside a joint project.

\section{Formats in Move B: The responding move}\label{sec:rossi:4}
Like Move A, Move B can include nonverbal and/or verbal behavior. However, since the goal of a recruitment sequence is to mobilize practical action, fulfillment naturally requires nonverbal, physical work. This is often all the recruitee does in the responding move. When we look quantitatively at the modality of complying responses, over half are fully nonverbal (52.8\%, \textit{n}=75/142).\footnote{Six cases were excluded from this count where it was not possible to ascertain whether the responding move did or did not include a verbal component.} 

In what follows, I consider the modality of the responding move with an eye to what it can tell us about the nature of the recruiting move. After examining particular kinds of verbal responses that may accompany nonverbal fulfillment, I look at exclusively verbal responses that indicate a problem with the recruitment, including different ways of rejecting it.

\subsection{Response modality}\label{sec:rossi:4.1}
As mentioned above, a fully nonverbal response is often all that is needed to fulfill a recruitment. However, fully nonverbal responses are not equally distributed across the dataset. \tabref{tab:rossi:3} shows the modality of complying responses by recruiting format. Nonverbal, no-predicate, and imperative recruiting moves are more frequently responded to nonverbally; interrogative and declarative recruiting moves, by contrast, are more frequently responded to verbally. 

\begin{table}
\begin{tabularx}{\textwidth}{lXXXX}
\lsptoprule
& \multicolumn{2}{c}{Fully nonverbal response} & \multicolumn{2}{c}{Composite/verbal response} \\
Recruiting format & \# & \% & \# & \% \\
\midrule
Nonverbal & 16 & 80\% & 4 & 20\% \\
No predicate & 8 & 67\% & 4 & 33\% \\
Imperative & 35	& 66\% & 18 & 34\% \\
Interrogative & 9 & 31\% & 20 & 69\% \\
Declarative & 5	& 19\% & 21 & 81\% \\
Other & 0 & 0\% & 2 & 100\% \\
\lspbottomrule
\end{tabularx}
\caption{Modality of complying responses relative to the format of the recruiting move (\textit{n}=142).}
\label{tab:rossi:3}
\end{table}

% \begin{table}
% \begin{tabularx}{\textwidth}{lrr}
% \lsptoprule
% Recruiting format & Fully nonverbal response & Composite or verbal response \\
%  \midrule
% Nonverbal & 80\% (\textit{n}=16) & 20\% (\textit{n}=4) \\
% No predicate & 67\% (\textit{n}=8) & 33\% (\textit{n}=4) \\
% Imperative & 66\% (\textit{n}=35) & 34\% (\textit{n}=18) \\
% Interrogative & 31\% (\textit{n}=9) & 69\% (\textit{n}=20) \\
% Declarative & 19\% (\textit{n}=5) & 81\% (\textit{n}=21) \\
% Other & 0\% (\textit{n}=0) & 100\% (\textit{n}=2) \\
% \lspbottomrule
% \end{tabularx}
% \caption{Modality of fulfilling responses relative to the format of the recruiting move (\textit{n}=142).}
% \label{tab:rossi:3}
% \end{table}

For imperative recruiting moves, the pattern is consistent with earlier research showing that the imperative format projects only the fulfillment of a request or directive (\citealt[chap. 3]{Wootton1997,Goodwin2006,CravenPotter2010,Kent2011,kent_compliance_2012,Rossi2012,Rossi2015a}). 

In \sectref{sec:rossi:3.1} and \sectref{sec:rossi:3.3.1}, we saw that nonverbal and no-predicate recruiting moves occur in similar environments as imperatives, namely within joint projects that support an expectation of compliance with recruitments serving the project's advancement (see Extracts \ref{ex:rossi:5} and \ref{ex:rossi:9}). This suggests that, while nonverbal and no-predicate formats do not have the semantics of an imperative clause, they may be similarly understood as making relevant only the fulfillment of the recruitment (see also Dingemanse, \chapref{sec:dingemanse}, \sectref{sec:dingemanse:4.1} on nonverbal recruiting formats receiving nonverbal responses).

The modality of responses to interrogative and declarative recruiting moves is also consistent with the findings of earlier research, discussed in the next section. As we will see, these recruiting formats make relevant more than one response option, with declaratives affording an open response space. We will also see that complying responses that include verbal elements involve more than the fulfillment of the recruitment.

\subsection{Verbal elements of responses}\label{sec:rossi:4.2}

\subsubsection{Accepting, confirming, and agreeing}\label{sec:rossi:4.2.1}

Earlier research has shown that interrogative recruiting formats, specifically polar interrogatives, are legitimately responded to with acceptance before fulfillment or with a negative answer (\citealt[chapp. 3--4]{Wootton1997,Raymond2003,CravenPotter2010,Kent2011,Rossi2012,Rossi2015a}). Unlike an imperative, a polar interrogative conveys that the recruitee's compliance is not being assumed (cf. \citealt[74]{Searle1975}; \citealt[60]{ErvinTripp1976}; \citealt[159]{Wierzbicka1991}). In Italian, as in other languages, recruiting moves designed as polar interrogatives are accepted with a positive polar token. In \REF{ex:rossi:11}, for example, Franco says \textit{sì} ‘yes’ before fulfilling Beata's simple interrogative request; in \REF{ex:rossi:8}, Loretta accepts a similar request from Rocco with a head nod. % , that is, with a word or gesture that approves or validates a proposition

Among the interrogative subtypes in Italian, the \textit{hai x} ‘do you have x’ format exhibits special properties that have consequences for how the recruiting turn can be responded to. Like simple and \textit{puoi x} ‘can you x’ interrogatives, a \textit{hai x} ‘do you have x’ recruiting turn makes fulfillment contingent on the recruitee's response. But it does so in a different way. In \sectref{sec:rossi:3.3.3}, we saw that this format functions as a pre-request checking a precondition for recruitment. This affords two types of response that support the accomplishment of the sequence: one is immediate fulfillment, optionally accompanied by a positive polar answer (see \extref{ex:rossi:12}); the other is a go-ahead response (\citealt[30]{Schegloff2007}), confirming that the precondition obtains. An example of this is given in the extract below.

\transheader{ex:rossi:18}{Circolo01\_2718316}\vspace{-1mm}
%
\begin{mdframednoverticalspace}[style=firstfoc]
\begin{transbox}{1}{sil}
\begin{verbatim}
ghe ´NAt? ((points at card combination))
LOC PTV=have-2SG=2SG.SCL
\end{verbatim}
do you have any
\end{transbox}
\end{mdframednoverticalspace}\vspace{0.75mm}
%
\begin{mdframednoverticalspace}[style=secondfoc]
\begin{transbox}{2}{cla}
\begin{verbatim}
una_
\end{verbatim}
one
\end{transbox}
\end{mdframednoverticalspace}\vspace{1.5mm}
%
\begin{transbox}{3}{sil}
\begin{verbatim}
↘DAmela?
give-IMP.2SG=1SG.DAT=3SG.ACC
\end{verbatim}
give it to me
\end{transbox}
%
\begin{transbox}{4}{cla}
\begin{verbatim}
((passes card))
\end{verbatim}
\end{transbox}\bigskip

Clara responds to Silvia's \textit{hai x} ‘do you have x’ interrogative by asserting that she has one unit of the target object. This is followed by Silvia producing another first pair part, this time in imperative form, which is responded to with fulfillment. For a more detailed account of expanded sequences like \REF{ex:rossi:18}, see \cite{Rossi2015b} and references therein. % (see also \citealt{Merritt1976}; \citealt[chap. 6]{Levinson1983}; \citealt[chap. 4]{Schegloff2007})

While interrogatives make relevant at least two alternative types of verbal response, declaratives have been shown to afford an even wider range of options (\citealt{VinkhuyzenSzymanski2005,RossiZinken2016}). In \REF{ex:rossi:15}, for instance, Mirko responds to Emma's factual declarative with two distinct responses that address two different actions accomplished by her recruiting turn. The change-of-state token \textit{a::h.} ‘o::h’ (line 6) treats what Emma has told Mirko as news, while his subsequent commitment to releasing Milena from the feed drip (line 8) orients to it as a request. In \REF{ex:rossi:14}, Dino responds to Sergio's \textit{bisogna x} ‘it is necessary to x’ declarative with \textit{faccio io.} ‘I'll do it’, volunteering to do the necessary action. These examples already exhibit a wider range of response types than any of the recruiting formats we have considered so far.

The next example illustrates yet another type of response afforded by declarative recruiting turns: agreement. Fabio, Rino, and other friends are making a booklet of short readings, the printouts of which are scattered on the table. It is now time to type up the excerpts on the computer. When the extract begins, Fabio has just offered to dictate the excerpts to Rino. His question ‘which one do we write up first’ (line 1) implies an understanding that all the excerpts they have considered will eventually be included in the booklet. In response, Rino rejects this understanding and recruits everyone to make a selection of the readings for inclusion.

\transheader{ex:rossi:19}{Precamp01\_831126}\vspace{2mm}
%
\begin{transbox}{1}{fab}
\begin{verbatim}
no qual  è      che  mettiam giù  prima;
no which be.3SG COMP put-1PL down before
\end{verbatim}
which one do we write up first
\end{transbox}\vspace{1mm}
%
\begin{mdframednoverticalspace}[style=firstfoc]
\begin{transbox}{2}{rin}
\begin{verbatim}
eh  ↘NO; bisogna         ˇSCEglierle;
PTC no   necessitate.3SG choose-INF=3PL.ACC
\end{verbatim}
well no it is necessary to make a selection
\end{transbox}
\end{mdframednoverticalspace}\vspace{1mm}
%
\begin{mdframednoverticalspace}[style=secondfoc]
\begin{transbox}{3}{fab}
\begin{verbatim}
eh  e↘SATto. (.) bisogna <<creaky>↘SCEglierle.>
PTC exactly      necessitate-3SG  choose-INF=3PL.ACC
\end{verbatim}
right exactly (.) it is necessary to make a selection
\end{transbox}
%
\emptytransbox{4}{((taps on one excerpt to propose it for selection))}
\end{mdframednoverticalspace}

Before complying with Rino's recruiting move nonverbally (line 4), Fabio says \textit{eh e\`{}SATto.} ‘right exactly’, by which he agrees with Rino's statement and the view of the world it presents. Fabio then strengthens his agreement with a near-verbatim repetition of the statement, a practice that is used to assert one's epistemic right over what someone else has just said (\citealt{Stivers2005}; cf. \citealt{Schegloff1996}). These types of responses are afforded only by declarative recruiting formats. In this particular case, the impersonal deontic construction used by Rino asserts the existence of a need or obligation, which may be agreed with or -- as we see in the next section -- disagreed with. 

\subsubsection{Rejecting, blocking, and disagreeing}\label{sec:rossi:4.2.2}

Another function of verbal elements in the responding move is to reject the recruitment. Rejection is a dispreferred response that thwarts the course of action initiated by the recruiter and poses a potential threat to social solidarity (see \citealt[265--80]{Heritage1984b}; \citealt{BrownLevinson1987}; \citealt[chap. 5]{Schegloff2007}). The dispreferred status of rejections is reflected in their design, as illustrated by the following examples.

\transheader{ex:rossi:20}{Capodanno02\_655722}\vspace{-1mm}
%
\begin{mdframednoverticalspace}[style=firstfoc]
\begin{transbox}{1}{eva}
\begin{verbatim}
ma  ˆmEteghe            'l  ˆCO:so ↘prIma
but put-IMP.2SG=3SG.DAT the thingy before
\end{verbatim}
but put the thi:ngy first
\end{transbox}
\end{mdframednoverticalspace}\vspace{1mm}
%
\begin{mdframednoverticalspace}[style=secondfoc]
\emptytransbox{2}{(0.7)}
\end{mdframednoverticalspace}\vspace{-0.5mm}
%
\begin{mdframednoverticalspace}[style=secondfoc]
\begin{transbox}{3}{ada}
\begin{verbatim}
<<breathy>ma: pensavo        ˆSOra.>
          but think-IMPF-1SG above
\end{verbatim}
\hspace{1.4cm} but: I was thinking \{to put it\} on top
\end{transbox}
\end{mdframednoverticalspace}\vspace{1mm}
%
\begin{transbox}{4}{eva}
\begin{verbatim}
<<pp>↑ah [vabem.>
      oh  PTC
\end{verbatim}
\hspace{0.8cm} oh okay
\end{transbox}
%
\begin{mdframednoverticalspace}[style=secondfoc]
\begin{transbox}{5}{ada}
\begin{verbatim}
         [↑↘SOra l'è        pu   gudu↘RIOso,
           above SCL=be.3SG more pleasurable
\end{verbatim}
\hspace{1.55cm} on top is more delicious
\end{transbox}
\end{mdframednoverticalspace}

\transheader{ex:rossi:21}{BiscottiPome01\_1884369}\vspace{-1mm}
%
\begin{mdframednoverticalspace}[style=firstfoc]
\begin{transbox}{1}{azi}
\begin{verbatim}
Furio mi      ´PREsti  le  chiavi del    ga¯RAge
NAME  1SG.DAT lend-2SG the keys   of-the garage
\end{verbatim}
Furio \{will\} you lend me \{your\} garage keys
\end{transbox}
\end{mdframednoverticalspace}\vspace{1mm}
%
\begin{transbox}{2}{~}
\begin{verbatim}
che  te      le      riporto    alle   ↘TRE;
CONN 2SG.DAT 3PL.ACC return-1SG at-the three
\end{verbatim}
which I'm going to return to you at three
\end{transbox}\vspace{1mm}
%
\begin{mdframednoverticalspace}[style=secondfoc]
\emptytransbox{3}{(4.7)}
\end{mdframednoverticalspace}\vspace{-1mm}
%
\begin{mdframednoverticalspace}[style=firstfoc]
\begin{transbox}{4}{azi}
\begin{verbatim}
↑non ce  le      ↘HO.
 not LOC 3PL.ACC have-1SG
\end{verbatim}
\hspace{0.07cm} I don't have \{mine\}
\end{transbox}
\end{mdframednoverticalspace}\vspace{1.25mm}
%
\emptytransbox{5}{(0.3)}\vspace{-2.5mm}
%
\begin{mdframednoverticalspace}[style=secondfoc]
\begin{transbox}{6}{fur}
\begin{verbatim}
eh  öh eh  sono   mi- ¯Anche le  mie chiavi di cat- di ˆCAsa.
PTC uh PTC be.3PL mi- also   the my  keys   of ho-  of house
\end{verbatim}
well uh well they're m- also my c- house keys
\end{transbox}
\end{mdframednoverticalspace}\bigskip

These two examples illustrate some of the typical features of dispreferred responses that have been extensively documented in the literature: delays, prefatory particles (\textit{ma} ‘but’, \textit{eh} ‘well’), hesitations (\textit{öh} ‘uh’), and the provision of reasons for not complying. These features are found in negative responses to variously formatted recruiting moves, including imperatives \REF{ex:rossi:20} and simple interrogatives \REF{ex:rossi:21}.\footnote{See also the responses in \REF{ex:rossi:2} above and in \REF{ex:rossi:23} below.} But now consider another case where the recruitment is initiated with a \textit{hai x} ‘do you have x’ interrogative (cf. Extracts \ref{ex:rossi:12} and \ref{ex:rossi:18}).

\transheader{ex:rossi:22}{Circolo01\_2718316}\vspace{2mm}
%
\begin{transbox}{1}{sil}
\begin{verbatim}
suo, ((points to Clara))
hers
\end{verbatim}
\{it's\} hers
\end{transbox}\vspace{1.5mm}
%
\emptytransbox{2}{(0.7)}\vspace{-1mm}
%
\begin{mdframednoverticalspace}[style=firstfoc]
\begin{transbox}{3}{fla}
\begin{verbatim}
öh ti      ghe ´NAt? ((to Bianca))
uh 2SG.NOM LOC PTV=have-2SG=2SG.SCL
\end{verbatim}
uh do you have any
\end{transbox}
\end{mdframednoverticalspace}\vspace{-0.5mm}
%
\begin{mdframednoverticalspace}[style=secondfoc]
\begin{transbox}{4}{bia}
\begin{verbatim}
no. ((shakes head))
\end{verbatim}
no
\end{transbox}
\end{mdframednoverticalspace}\bigskip

Like in \REF{ex:rossi:20} and \REF{ex:rossi:21}, Bianca's ‘no’ is structurally dispreferred in that it does not support the accomplishment of the course of action initiated by the recruiting move. Yet it lacks all the features seen earlier. The explanation for this lies in the nature of the particular action performed by a \textit{hai} x ‘do you have x’ interrogative. In \sectref{sec:rossi:3.3.3}, we saw that pre-requests check a precondition for a request to be made successfully. This means that a negative response to the pre-request is not a response to the projected request -- that is, it is not a rejection. Rather, it is a blocking response \citep[30]{Schegloff2007}. A blocking response like Bianca's ‘no’ in \REF{ex:rossi:22} indicates a state of affairs -- here, the unavailability of the target object -- that prevents the further development of the activity and that is normally beyond the control of the recruitee, rather than a matter of disposition or uncooperative behavior. For this reason, the negative response does not need to be mitigated in the same way a rejection does \citep{Rossi2015b}.

The last example in this section illustrates a particular form of rejection that is afforded by declarative recruiting formats; here in particular by a declarative of the \textit{bisogna x} ‘it is necessary to x’ subtype (cf. Extracts \ref{ex:rossi:14} and \ref{ex:rossi:19} above). Elena is sitting at the kitchen table, finishing her food. Across from her, Agata is loading the dishwasher and is doing so without pre-rinsing the dishes. When the extract begins, Elena points out the need to select a heavy wash cycle, on the grounds that the food they have eaten may otherwise not come off in the dishwasher. % (especially if the dishes are not pre-rinsed)

\transheader{ex:rossi:23}{Capodanno02\_21779}\vspace{-1mm}
%
\begin{mdframednoverticalspace}[style=firstfoc]
\begin{transbox}{1}{ele}
\begin{verbatim}
bisogna         ´dArghe           
necessitate-3SG give-INF=3SG.DAT
\end{verbatim}
it is necessary to select
\end{transbox}
\end{mdframednoverticalspace}\vspace{1mm}
%
\begin{transbox}{2}{~}
\begin{verbatim}
en programma molto ↘ALto ↘Agata [eh, per]ché=
a program    very  high  NAME    PTC because
\end{verbatim}
a very intense program Agata you know because
\end{transbox}\vspace{0.75mm}
%
\begin{mdframednoverticalspace}[style=secondfoc]
\begin{transbox}{3}{aga}
\begin{verbatim}
                                [↑↑mac↘CHÉ.]                                                    
                                   INTJ
\end{verbatim}
\hspace{5.1cm} not at all
\end{transbox}
\end{mdframednoverticalspace}\vspace{2mm}
%
\begin{transbox}{4}{ele}
\begin{verbatim}
=questo s' <<breathy>attacca    en d' en  ´M[Odo_>
 this   RFL          attach-3SG in of one manner
\end{verbatim}
\hspace{0.07cm} this sticks so much
\end{transbox}\vspace{0.5mm}
%
\begin{transbox}{5}{aga}
\begin{verbatim}
                                            [sì  ma  l'è        l- la
                                             yes but SCL=be.3SG 
\end{verbatim}
\hspace{6.6cm} yes but it's l-
\end{transbox}
%
\begin{transbox}{6}{~}
\begin{verbatim}
g'avem       giusto magnà.
LOC=have-1PL just   eat-PCP
\end{verbatim}
we've just eaten on it
\end{transbox}
%
\begin{transbox}{6}{~}
\begin{verbatim}
non è      arivà      neanche a  secarse,
not be.3SG arrive-PCP neither to dry_up-INF=RFL
\end{verbatim}
it hasn't even had the time to harden
\end{transbox}\bigskip

Agata's response begins with the interjection \textit{↑↑mac\`{}CHÉ.} ‘not at all’ (or ‘of course not’). With it, Agata confutes the veracity of the assertion expressed by Elena's declarative (\textit{bisogna ´{}dArghe en programma molto \`{}ALto} ‘it is necessary to select a very intense program’), in other words, she disagrees with it. 

Disagreement is not found in rejections to imperative and interrogative recruiting moves as these are not treated as statements committing to the truth of a proposition. A statement of need, on the other hand, makes a claim about the material and social world, and exposes it to the evaluation of others against their own understanding of that world \citep[see also][]{ZinkenOgiermann2011}. In \REF{ex:rossi:19}, we saw that Furio agrees with Rino's \textit{bisogna x} ‘it is necessary to x’ declarative with \textit{eh e\`{}SATto.} ‘right exactly’ and then strengthens his agreement by repeating the statement. In \REF{ex:rossi:23}, Agata does the opposite: after expressing her disagreement with \textit{↑↑mac\`{}CHÉ.} ‘not at all’, she goes on to dispute the grounds upon which the Eva's claim is based: although it may be true that the food sticks on plates, they have just finished eating so, according to Agata, the food has not yet had the time to cake on the plates. The implication is that, in her view, this makes the selection of a heavy wash cycle unnecessary.\\

\noindent
In sum, this whole section (\sectref{sec:rossi:4}) has shown that formats in Move B are closely patterned relative to formats in Move A. The grammar and the particular actions accomplished by various recruiting formats place different constraints on, and provide different affordances for, how exactly the recruitee can comply with or reject the recruitment. For complying responses, imperatives project only nonverbal fulfillment, while interrogatives allow the recruitee to accept before fulfilling, and declaratives provide an open space of options, including receipting information, agreeing with what has been said, and volunteering assistance. For rejections, negative responses to most recruiting formats are normally marked as dispreferreds, with declaratives allowing for disagreement. Negative responses to \textit{hai x} ‘do you have x’ interrogatives, however, are not designed as dispreferreds, as they do not constitute a rejection but a blocking response to a pre-request.

\section{Acknowledgment in third position}\label{sec:rossi:5}
Across the languages examined in this volume, acknowledging fulfillment of a recruitment with a third-position turn like ‘great’ or ‘thank you’ is rare \citep{FloydEtAl2018}. At the same time, Italian shows a relatively higher proportion of such turns than other languages (13.5\%, \textit{n}=20/148). This includes 8 cases of the dedicated expression \textit{grazie} ‘thanks’, two examples of which are found in \REF{ex:rossi:3} and \REF{ex:rossi:8} above. Other cases involve positive assessments (e.g. \textit{ottimo} ‘excellent’, \textit{bravo} ‘well done’) and interjections such as \textit{bom} ‘alright’ and \textit{eh} ‘right’; an example of the latter is found in \REF{ex:rossi:15} above. % We encountered an example of the latter in \REF{ex:rossi:15} where \textit{eh} ‘right’ is used to ratify the recipient's uptake of the request as the sought-after outcome. 

If acknowledgment in third position is generally infrequent in recruitment sequences, what do we make of the cases where acknowledgment does occur? In a recent study \citep{ZinkenRossiReddy}, we addressed this question with particular reference to thanking. What we found is that, in informal interaction, thanks are given to recognize another person's agency in providing assistance. In recruitment sequences that end with thanks, the recruiter treats fulfillment as not taken for granted and rather as the result of the recruitee's autonomous decision to help. This happens most obviously in offer sequences, where assistance is provided without having been requested. We also observe thanking in delicate request sequences, where there is an anticipation of actual or potential unwillingness on the part of the recruitee; if unwillingness is successfully overcome, this is grounds for acknowledging the recruitee's compliance. %  p. \pageref{par:rossi:unwillingness}

But thanking may also occur after compliance with unproblematic requests. Here, the recruiter treats compliance as not taken for granted even though, contextually, it is largely expectable. Thanking then functions reflexively to accentuate the recruitee's agency in providing assistance. 

Even apparently reflexive practices, however, can be sensitive to the interactional environment in which they are used. \extref{ex:rossi:24} serves to illustrate this.

During dinner at a family gathering, Plinio finds himself without a fork (\textit{a me manca la forchetta. hah hah hah} ‘I don't have a fork hah hah hah’, line 1). One of the diners sitting across from him hears the comment and directs him to a container with forks located on a service table (\textit{è lì:;} ‘it's there’, line 3). As Plinio looks over to where the forks are, Fabrizio walks in with a sponge cloth to wipe the service table, on which he accidentally spilled food moments earlier. Plinio calls out to Fabrizio and, after repeated attempts to get his attention (lines 6, 10), asks him to pass a fork (line 12). As Fabrizio turns to the forks container, he playfully rejects the request (\textit{no:\_} ‘no’, line 16) and then quickly fulfills it by passing a fork, which Plino acknowledges with \textit{grazie,} ‘thanks’ (line 19). 

\transheader{ex:rossi:24}{NataleSala02\_2007128}\vspace{2mm}
%
\begin{transbox}{1}{pli}
\begin{verbatim}
a  me      manca    la  forchetta. hah hah hah
to 1SG.ACC lack-3SG the fork
\end{verbatim}
I don't have a fork hah hah hah
\end{transbox}\vspace{2mm}
%
\emptytransbox{2}{(0.6)}\vspace{-1mm}
%
\begin{transbox}{3}{cle}
\begin{verbatim}
è     *lìΔ:; ((points))
be.3SG there
\end{verbatim}
it's there
\end{transbox}\vspace{2mm}
%
\begin{transbox}{4}{\textit{pli}}
\begin{verbatim}
      *looks over-->>
\end{verbatim}
\end{transbox}\vspace{-0.5mm}
%
\begin{transbox}{5}{\textit{fab}}
\begin{verbatim}
         Δapproaches table with sponge cloth-->
\end{verbatim}
\end{transbox}
%
\begin{transbox}{6}{pli}
\begin{verbatim}
öeh:: <<all>FabriΔzio Fabrizio Fabrizio_>>
            NAME      NAME     NAME
\end{verbatim}
uhm:: \hspace{0.9cm} Fabrizio Fabrizio Fabrizio
\end{transbox}\vspace{2.5mm}
%
\begin{transbox}{7}{\textit{fab}}
\begin{verbatim}
---------------->Δbegins to wipe table-->
\end{verbatim}
\end{transbox}\vspace{-0.5mm}
%
\begin{transbox}{8}{~}
\begin{verbatim}
(0.3)Δ(0.1)
\end{verbatim}
\end{transbox}
%
\begin{transbox}{9}{\textit{fab}}
\begin{verbatim}
---->Δturns around with upper body-->
\end{verbatim}
\end{transbox}\vspace{-0.5mm}
%
\begin{transbox}{10}{pli}
\begin{verbatim}
<<f>FabriΔzio;>
\end{verbatim}
\hspace{0.525cm} Fabrizio
\end{transbox}\vspace{2mm}
%
\begin{transbox}{11}{\textit{fab}}
\begin{verbatim}
-------->Δlooks at Plinio-->
\end{verbatim}
\end{transbox}\vspace{-0.25mm}
%
\begin{mdframednoverticalspace}[style=firstfoc]
\begin{transbox}{12}{pli}
\begin{verbatim}
mi      ´pAssi # una forˆCHETΔta. ((points to forks container))
1SG.DAT pass-2SG a   fork
\end{verbatim}
\{will\} you pass me a fork
\end{transbox}
\end{mdframednoverticalspace}\vspace{1.75mm}
%
\begin{transbox}{13}{\fig}
\begin{verbatim}
               #Figure 8a
\end{verbatim}
\end{transbox}\vspace{-0.55mm}
%
\begin{mdframednoverticalspace}[style=secondfoc]
\begin{transbox}{14}{\textit{fab}}
\begin{verbatim}
---------------------------->Δturns to forks container-->
\end{verbatim}
\end{transbox}
\end{mdframednoverticalspace}
%
\begin{transbox}{15}{~}
\begin{verbatim}
(0.3)
\end{verbatim}
\end{transbox}
%
\begin{transbox}{16}{fab}
\begin{verbatim}
no:_ ((shakes head slightly))
\end{verbatim}
no
\end{transbox}\vspace{2mm}
%
\begin{transbox}{17}{~}
\begin{verbatim}
Δ(0.8)    Δ(0.5)Δ(0.3)    
\end{verbatim}
\end{transbox}
%
\begin{transbox}{18}{\textit{pli}}
\begin{verbatim}
Δgets forkΔturnsΔholds fork out across table-->
\end{verbatim}
\end{transbox}
%
\begin{transbox}{19}{pli}
\begin{verbatim}
grazie,
\end{verbatim}
thanks
\end{transbox}\vspace{2mm}
%
\begin{transbox}{20}{~}
\begin{verbatim}
(0.2)#(0.1)+Δ(0.4)
\end{verbatim}
\end{transbox}
%
\begin{transbox}{21}{~}
\begin{verbatim}
     #Figure 8b
\end{verbatim}
\end{transbox}
%
\begin{transbox}{22}{\textit{pli}}
\begin{verbatim}
           +takes fork-->>
\end{verbatim}
\end{transbox}\vspace{-1.5mm}
%
\begin{transbox}{23}{\textit{fab}}
\begin{verbatim}
----------->Δturns back and resumes wiping-->>
\end{verbatim}
\end{transbox}\bigskip

\begin{figure}
\caption{Frames from \extref{ex:rossi:24}.} 
\label{fig:rossi:9}
\subfigure[Plinio points to the forks container while saying \textit{mi ´{}pAssi una forˆCHETta.} ‘\{will\} you pass me a fork’; Fabrizio has turned his neck and is looking at him while keeping a hand with the sponge cloth on the service table (line 13).]{
  \includegraphics[height=.37\textheight]{figures/Rossi_Picture9a} 
}~~
\subfigure[Fabrizio passes a fork to Plinio while still keeping a hand on the service table \newline (line 21).]{
  \includegraphics[height=.37\textheight]{figures/Rossi_Picture9b} 
}
\end{figure}

Plinio's simple interrogative request initiates a course of action that is in his interest as an individual (see \sectref{sec:rossi:3.3.3}). As we saw in \REF{ex:rossi:11}, a key aspect of these sequences is the lack of continuity between what is requested and what the recruitee is doing at the moment, which often requires departing from one's business in order to fulfill the request. In \REF{ex:rossi:11}, Franco has to briefly disengage from the game to get Beata a paper towel; in \REF{ex:rossi:24}, Fabrizio has to suspend his ongoing task of wiping the table to get Plinio a fork.

With this in mind, let us look more closely at some of the particulars of how Fabrizio's assistance is recruited and acknowledged in \REF{ex:rossi:24}. A first notable feature is the work that Plino does to establish mutual attention. In his initial summons, Plinio repeats Fabrizio's name three times as he approaches the table and begins to wipe it (lines 5--7). Plinio then produces yet another, louder vocative (line 10), before Fabrizio finally turns around (line 11). Note that Fabrizio rotates only the upper part of his body, mainly his neck (\figref{fig:rossi:9}\textit{a}); this body torque displays Fabrizio's commitment to his primary involvement in wiping the table \citep{Schegloff1998}. Note also that he keeps his hand with the sponge cloth on the table throughout the sequence (\figref{fig:rossi:9}\textit{b}). 

This configuration highlights Fabrizio's position as a participant with his own business to tend to, who is being recruited to assist in someone else's project. Fabrizio's agency in this episode is further underscored by his playful rejection of the request (\textit{no:\_} ‘no’, line 16). By teasing Plinio with rejection, he draws attention to the fact that he has a choice, which helps to see his subsequent granting as an autonomous decision.

Treating compliance with a request as an autonomous decision is often a matter of construal, and a locus of cultural diversity \citep{ZinkenRossiReddy}. However, some of the elements that seem particularly conducive to recognizing the recruitee's agency in \REF{ex:rossi:24} can be observed also in other cases where compliance is acknowledged. In \REF{ex:rossi:8}, for instance, Rocco's request for a glass of water comes after an earlier attempt to get Loretta's attention; her agency as a recruitee is reflected in how she makes herself available only after she is done with a concurrent conversation (\textit{dimmi scusa.} ‘tell me -- sorry’, line 2). In \REF{ex:rossi:3}, Sara pursues her request for one more piece of banana with \textit{per fa`{}VOre.} ‘please’ (line 4), which attributes Furio agency in deciding whether or not to share more of his food with her.

\section{Social asymmetries}\label{sec:rossi:6}

This study is based on video recordings of informal interaction around the home, in the family, and in people's proximate community of friends and neighbors. My sample of recruitment sequences included only adult participants. I did not identify noticeable social asymmetries between the individuals participating in these sequences. Possible sources of asymmetry such as gender, age, or socioeconomic status did not emerge in the analysis. 

Exploring the larger corpus, I identified one noticeably asymmetrical relation between a daughter-in-law and her mother-in-law. The social asymmetry in this dyad is reflected in the daughter-in-law's use of the second person formal pronoun \textit{Lei} and corresponding verb inflections, which is not reciprocated by the mother-in-law. The sample did not include any recruitment sequences between these two individuals.
% A goal for future research should be to compare these findings domestic and informal interaction with data from institutional settings, where social asymmetries are more likely to influence the recruitment of assistance and collaboration.

\section{Discussion}\label{sec:rossi:7}
This chapter has provided an overview of how speakers of Italian recruit one another's assistance and collaboration in everyday informal interaction. Following the common structure adopted in the contributions to this volume, I have examined a range of interactional resources for initiating recruitment (Move A) and for responding to it (Move B), paying particular attention to the fit between the two moves.

For Move A, Italian speakers use all three main construction types found cross-linguistically: imperatives, interrogatives, and declaratives, as well as utterances without a predicate (e.g. \textit{una} ‘one’). While imperatives are the most frequent construction type, interrogatives and declaratives are also common. %(cf. Kendrick, \chapref{sec:kendrick} and Zinken, \chapref{sec:zinken} of this volume) 
The verbal component of the recruiting move can be enriched with additional elements including explanations and mitigators, one function of which is to attune the basic format being used (e.g. an imperative) to the interactional circumstances. %(cf. Baranova, \chapref{sec:baranova}). 

The nonverbal component of the recruiting move often involves one of three basic behaviors found across languages: pointing, holding out an object for someone to take and do something with, and reaching out to receive an object. These behaviors can also function as recruiting formats on their own, without words. In speech-plus-gesture composites, functional distinctions can be observed between different forms of pointing (“big” and “small”) that have been documented in other languages and interactional contexts \citep{EnfieldKitaRuiter2007}. At the same time, Italian speakers use a relatively high proportion of iconic gestures in Move A, which is consistent with previous research on the frequency of iconic gestures by Italian speakers in other settings \citep{Campisi2014}.

In surveying the repertoire of strategies for initiating recruitment, I have also tried to account for the selection between alternative formats, examining a number of social-interactional factors that influence the design of the recruiting move. One is the projectability of the action being recruited, which is particularly important for the use of nonverbal and no-predicate formats. The criterion of projectability is grounded in fundamental principles of human communication, including informational calibration and the recognizability of action, which are likely to be shared across languages (\citealt{Grice1975,Clark1996,Levinson2000}). Another factor that plays a role in the design of recruiting moves -- particularly in the use of interrogative vs. imperative formats -- is whether the action being recruited contributes to an already established joint project between recruiter and recruitee or else it initiates a new course of action that is in the interest of the recruiter as an individual. While these social-interactional concerns may be more prone to cultural variation, there is also evidence for their relevance across languages (see Kendrick, \chapref{sec:kendrick}, \sectref{sec:kendrick:4.2.2}; Zinken, \chapref{sec:zinken}, \sectref{sec:zinken:3.3.1}). 

Yet another factor that influences the design of the recruiting move is the anticipation of the recruitee's actual or potential unwillingness to comply. In these cases, Italian recruiters use the interrogative format \textit{puoi x} ‘can you x’ to recognize and attempt to overcome unwillingness through persuasion. While the basic concern for dealing with delicate recruitments may be universal \citep{BrownLevinson1987}, the particular formats used to do this are more likely to differ. In contrast to this, a form-function mapping that may be cross-linguistically valid is the one between the need to check a precondition for recruitment and the use of an interrogative format (see Floyd, \chapref{sec:floyd}, \sectref{sec:floyd:3.3.3}; Enfield, \chapref{sec:enfield}, \sectref{sec:enfield:4.3.1}). In the Italian data, this often involves querying the recruitee as to the availability of a target object with \textit{hai x} ‘do you have x’. 

Factual declaratives that present a description of a state of affairs (e.g. \textit{è finì la flebo} ‘the feed drip has finished’) are yet another format that appears to work similarly across languages (see Kendrick, \chapref{sec:kendrick}, \sectref{sec:kendrick:4.2.3}; Baranova, \chapref{sec:baranova}, \sectref{sec:baranova:3.3.3}; Dingemanse, \chapref{sec:dingemanse}, \sectref{sec:dingemanse:3.2.2}). An important affordance of this format emerging from the Italian data is its capacity to get others to do things by means of informing them of something they do not know. As for impersonal deontic declaratives (e.g. \textit{bisogna x} ‘it is necessary to x’), while these are not available in the same way across languages, they have counterparts in at least some of those examined in this volume (see Floyd, \chapref{sec:floyd}, \sectref{sec:floyd:3.3.4}; Zinken, \chapref{sec:zinken}, \sectref{sec:zinken:3.3.2}; Baranova, \chapref{sec:baranova}, \sectref{sec:baranova:3.3.3}).

Coming now to Move B, I have surveyed a range of options that Italian speakers have for complying with or rejecting a recruitment, paying special attention to how the use of alternative responding formats is sensitive to the nature of Move A. 

Nonverbal fulfillment is the appropriate response to recruiting moves that project only compliance, such as those designed with an imperative. Verbal acceptance or confirmation (e.g. \textit{sì} ‘yes’) is a relevant response to recruiting moves that formally anticipate the possibility of rejection or failure of the recruitment, first and foremost polar interrogatives. Similar principles of responding apply in other languages as well (see Kendrick, \chapref{sec:kendrick}, \sectref{sec:kendrick:4.2.3}). % Preceding or accompanying fulfillment 

With polar interrogatives that function as pre-requests (e.g. \textit{hai x} ‘do you have x’), a positive polar answer counts as a go-ahead, confirming that the precondition for recruitment obtains, whereas a negative answer counts as a blocking response which, unlike a rejection, may not need to be justified or mitigated. This pattern is based in generic properties of action and sequential structure \citep{Schegloff2007,KendrickEtAl2020}. 

Finally, declarative recruiting formats afford an open response space. Factual declaratives that convey new information to the recruitee, for instance, can be taken up with a news receipt (e.g. \textit{ah} ‘oh’). Other declaratives that make a claim about the material and social world (e.g. \textit{bisogna x} ‘it is necessary to x’) can be agreed or disagreed with.

Looking beyond recruiting and responding moves, Italian speakers may acknowledge fulfillment of a recruitment by thanking (e.g. \textit{grazie} ‘thanks’), with positive assessments (e.g. \textit{ottimo} ‘excellent’), and with sequence-closing interjections (e.g. \textit{eh} ‘right’). Such acknowledgments in third position are uncommon in recruitment sequences around the world \citep{FloydEtAl2018}. At the same time, Italian is one of two languages in this project, together with English, where acknowledgment is relatively more frequent. The occurrence of acknowledgment, particularly in the form of thanking, reflects a preoccupation with recognizing individual agency in the provision of assistance \citep{ZinkenRossiReddy}. %Social asymmetries do not appear to play a major role in the recruitment of cooperation among family and friends. (cf. Baranova?)

In conclusion, the findings presented in this chapter show a tightly organized system of resources for recruiting assistance and collaboration. While the system is inflected according to the Italian language and culture, it shares many formal and functional elements with that of other languages, pointing to a common infrastructure for the management of cooperation in social life. % shaped by a set of recurrent social-interactional concerns

\section*{Transcription, glossing, and translation}

Transcripts follow basic conventions established in conversation analysis \citep{jefferson_glossary_2004,HepburnBolden2013}. Prosodic features are represented according to GAT 2 conventions \citep{couper-kuhlen_system_2011}, which include symbols for indicating pitch movement on accented syllables and at the end of the utterance (e.g. \textit{una ri\^{}STAMpa;}).\footnote{I have transcribed pitch movement on accented syllables only for focal turns. The grave accent character “\`{}” used in GAT 2 to represent falling pitch on an accented syllable was unavailable under the particular LaTeX setup used for transcripts in this volume, so I replaced it with the symbol “↘” (see, e.g., \extref{ex:rossi:1}, line 2).} Elements of visible behavior are generally noted as comments in double parentheses ((nonverbal behavior)). For some extracts, visible behavior is represented in greater detail using Mondada's (\citeyear{Mondada2019}) conventions. Interlinear glosses generally follow the Leipzig rules \citep{comrie_leipzig_2020}. I have added a few abbreviations and shortened others for economy:\medskip

\noindent
\begin{tabularx}{.45\textwidth}{>{\scshape}lQ}
conn & connective\\
dim & diminutive\\
form & formal\\
ger & Gerund\\
imps & impersonal\\
impf & Imperfect\\
intj & interjection\\
\end{tabularx}
\begin{tabularx}{.45\textwidth}{>{\scshape}lQ}
name & proper name\\
pcp & Participle\\
ptc & particle\\
ptv & partitive\\
rfl & reflexive\\
scl & subject clitic\\
\end{tabularx}\bigskip

\noindent
Free translations may include the following symbols:\medskip

\noindent
\begin{tabularx}{\textwidth}{lX}
-- & An en dash separates parts of the translation that may otherwise be ambiguous to parse, syntactically or pragmatically. \\
\{words\} &	Words in curly brackets are supplied to make the translation more understandable or idiomatic; these words have no direct counterpart in the original Italian. \\
\textit{word} ((≈ meaning)) & Words in italics cannot be translated; an approximate meaning is given in double parentheses preceded by an almost-equal-to sign. \\
\end{tabularx}

\section*{Acknowledgments}

Thank you to my fellow project members for their input and intellectual engagement over the years. I am also grateful to two reviewers for their helpful commentary. This research was supported by the European Research Council (grant no. 240853 to Nick Enfield) and by the Academy of Finland (grant no. 284595 to Marja-Leena Sorjonen).

\sloppy
\printbibliography[heading=subbibliography,notkeyword=this]
\end{document}

