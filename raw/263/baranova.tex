\documentclass[output=paper,modfonts,nonflat]{langsci/langscibook}
\ChapterDOI{10.5281/zenodo.4018386}
\author{Julija Baranova\affiliation{Department of Language and Communication, Radboud University}}

\title{Recruiting assistance and collaboration in Russian}

\abstract{This chapter describes the resources that speakers of Russian use when recruiting assistance and collaboration from others in everyday social interaction. The chapter draws on data from video recordings of informal conversation in Russian, and reports language-specific findings generated within a large-scale comparative project involving eight languages from five continents (see other chapters of this volume). The resources for recruitment described in this chapter include linguistic structures from across the levels of grammatical organization, as well as gestural and other visible and contextual resources of relevance to the interpretation of action in interaction. The presentation of categories of recruitment, and elements of recruitment sequences, follows the coding scheme used in the comparative project (see \chapref{sec:coding} of the volume). This chapter extends our knowledge of the structure and usage of Russian with detailed attention to the properties of sequential structure in conversational interaction. The chapter is a contribution to an emerging field of pragmatic typology.}

\IfFileExists{../localcommands.tex}{
  \usepackage{langsci-optional}
\usepackage{langsci-gb4e}
\usepackage{langsci-lgr}

\usepackage{listings}
\lstset{basicstyle=\ttfamily,tabsize=2,breaklines=true}

%added by author
% \usepackage{tipa}
\usepackage{multirow}
\graphicspath{{figures/}}
\usepackage{langsci-branding}

  
\newcommand{\sent}{\enumsentence}
\newcommand{\sents}{\eenumsentence}
\let\citeasnoun\citet

\renewcommand{\lsCoverTitleFont}[1]{\sffamily\addfontfeatures{Scale=MatchUppercase}\fontsize{44pt}{16mm}\selectfont #1}
  
  %% hyphenation points for line breaks
%% Normally, automatic hyphenation in LaTeX is very good
%% If a word is mis-hyphenated, add it to this file
%%
%% add information to TeX file before \begin{document} with:
%% %% hyphenation points for line breaks
%% Normally, automatic hyphenation in LaTeX is very good
%% If a word is mis-hyphenated, add it to this file
%%
%% add information to TeX file before \begin{document} with:
%% %% hyphenation points for line breaks
%% Normally, automatic hyphenation in LaTeX is very good
%% If a word is mis-hyphenated, add it to this file
%%
%% add information to TeX file before \begin{document} with:
%% \include{localhyphenation}
\hyphenation{
affri-ca-te
affri-ca-tes
an-no-tated
com-ple-ments
com-po-si-tio-na-li-ty
non-com-po-si-tio-na-li-ty
Gon-zá-lez
out-side
Ri-chárd
se-man-tics
STREU-SLE
Tie-de-mann
}
\hyphenation{
affri-ca-te
affri-ca-tes
an-no-tated
com-ple-ments
com-po-si-tio-na-li-ty
non-com-po-si-tio-na-li-ty
Gon-zá-lez
out-side
Ri-chárd
se-man-tics
STREU-SLE
Tie-de-mann
}
\hyphenation{
affri-ca-te
affri-ca-tes
an-no-tated
com-ple-ments
com-po-si-tio-na-li-ty
non-com-po-si-tio-na-li-ty
Gon-zá-lez
out-side
Ri-chárd
se-man-tics
STREU-SLE
Tie-de-mann
}
  \addbibresource{../localbibliography.bib}
  \togglepaper[1]%%chapternumber
}{}

\begin{document}
\maketitle
\label{sec:baranova}

\section{Introduction}

The work in this chapter was carried out as part of the comparative project on recruitment systems in eight languages presented in this volume. \chapref{sec:intro} defines recruitment as an interactional phenomenon; \chapref{sec:coding} outlines the coding scheme and explains the comparative categories used in the analysis. The present chapter offers an overview of the main practices used by speakers of Russian to recruit assistance and collaboration from their peers and family members in everyday activities such as talking, having dinner, or cooking together. The data come from a set of video recordings made by the author in Russia. The chapter begins with a brief introduction to the Russian language and a description of the corpus. I then present the basic structure of recruitment sequences. The recruiting move is analyzed in a separate section that discusses nonverbal elements of its composition as well as its linguistic formats (imperative, declarative, and interrogative). Attention is also given to additional elements such as reasons and diminutives that mitigate the recruiting move. Then the chapter discusses responding moves, broadly divided into complying and non-complying. Towards the end, I discuss the expression of gratitude and the role of social (a)symmetries in recruitment sequences. Finally, I summarize the findings and present some ideas for future research.

\subsection{The Russian language}

Russian is an East-Slavic language of the Indo-European language family. About 150 million people speak Russian as their first language. Russian is an official language of the Russian Federation, Belarus, Kazakhstan, Kyrgyzstan, and Tajikistan. The basic word order is SVO (\citealt{Hawkins1983,Tomlin1986}). Interrogatives are formed mainly through
intonation,
question words, and
interrogative particles.
Russian verbs come in aspectual pairs: perfective and imperfective. They inflect for tense, person, number, and on certain occasions for gender. Russian nouns are marked for gender (feminine, masculine, and neuter), number (singular and plural), and case (six cases). The grammar of Russian has been extensively studied, but we are only beginning to understand how Russian is used in everyday conversational interaction (e.g. \citealt{bolden_doing_2003,Bolden2004,Bolden2008,RobinsonBolden2010,Baranova2015,BaranovaDingemanse2016}). This paper offers a contribution to this line of work by focusing on the recruitment system in informal Russian.

\subsection{Data collection and corpus}

The corpus on which this work is based was constructed in accordance with guidelines developed by and for the members of the comparative project reported on in this volume (see Chapters 1--2). Russian data come from nineteen recordings made by the author during three field trips to Russia in 2011 and 2012. The recordings took place in several locations in the region of Chelyabinsk, at participants’ homes, and on two occasions at their work places. The interactions were all informal involving friends and family. The total sampled recording time was 3 hours and 20 minutes, resulting in 200 recruitment cases. The length of the sample per recording varied from 10 to 25 minutes.

Interlinear glosses in the data extracts generally follow the Leipzig glossing rules \citep{comrie_leipzig_2020}, albeit with some simplification in terms grammatical categories that are less relevant for the purposes of the chapter. The focus in the glosses is mainly on grammatical tense, aspect, case, and gender.

\section{Basics of recruitment sequences}

As defined in \chapref{sec:intro}, \sectref{sec:intro:4}, a recruitment is a basic cooperative phenomenon in social interaction consisting of a sequence of two moves with the following characteristics:

\begin{description}
\item[Move A:] participant A says or does something to participant B, or that B can see or hear;
\item[Move B:] participant B does a practical action for or with participant A that is fitted to what A has said or done.
\end{description}

This is the basic and canonical sequence, an example of which is given in the following section. Other details of what can happen, including what participant B can say or do in Move B to fulfill or reject the recruitment, are illustrated in later sections. In the transcripts, ▶ and ▷ designate Move A and Move B, respectively.

\subsection{Minimal recruitment sequence}

When a recipient responds to Move A with immediate fulfillment of the recruitment, the result is a minimal recruitment sequence. This is illustrated in \REF{ex:baranova:1} where several family members are gathered for dinner at Lida’s place. The extract starts with an offer sequence between Tanya and her young child.

%\label{bkm:Ref447705798}
\transheader{ex:baranova:1}{20120114\_family\_visit\_2\_164605}\vspace{2mm}
%
\begin{transbox}{1}{tan}
\begin{verbatim}
mozhet malaka?
maybe  milk.GEN
\end{verbatim}
maybe some milk?
\end{transbox}
%
\begin{transbox}{2}{chi}
\begin{verbatim}
((nods with his head))
\end{verbatim}
\end{transbox}
%
\begin{transbox}{3}{~}
\begin{verbatim}
(0.7) ((Tanya turns away from child and towards Lida))
\end{verbatim}
\end{transbox}
%
\begin{mdframednoverticalspace}[style=firstfoc]
\begin{transbox}{4}{tan}
\begin{verbatim}
malaka ((nods))
milk.GEN
\end{verbatim}
some milk % \hspace{1.2cm}
\end{transbox}
\end{mdframednoverticalspace}
%
\begin{mdframednoverticalspace}[style=secondfoc]
\xtransbox{5}{lid}{((takes milk from refrigerator, pours it into a cup, and places the cup on the table in front of the child))}
\end{mdframednoverticalspace}\\

\begin{figure}
\caption{Tanya initiates a recruitment for milk (\extref{ex:baranova:1}, line 4).}
\label{fig:baranova:1}
\includegraphics[height=.28\textheight]{figures/baranova-img001.jpg}
\end{figure}

\begin{figure}
\caption{Lida gets milk from the refrigerator (\extref{ex:baranova:1}, line 5).}
\label{fig:baranova:2}
\includegraphics[height=.28\textheight]{figures/baranova-img002.jpg}
\end{figure}

Tanya offers her child some milk (line 1) and he accepts the offer (line 2). However, Tanya is unable to get out from the table easily. She recruits Lida’s assistance using a no-predicate construction: \textit{malaka} ‘some milk’ (line 4, \figref{fig:baranova:1}). Lida starts complying immediately (line 5, \figref{fig:baranova:2}). This recruitment sequence is minimal as consists of a recruiting turn followed by fulfillment with no other actions in between, such as repair initiations or redoings of the recruiting move. Fulfillment is entirely nonverbal: Lida takes the milk out of the refrigerator, pours it in a cup, and puts it on the table in front of the child.

\subsection{Non-minimal recruitment sequence}

Non-minimal recruitment sequences are sequences where fulfillment does not immediately take place. Instead, the initial response is something other than fulfillment or compliance, such as a question or a rejection. Sometimes there is no immediate relevant response at all and the recruiting move is effectively ignored. In these cases, recruiters may pursue compliance, for example by offering a repair solution, answering a clarification question, offering a reason for the recruitment, or by simply redoing the recruiting move.

An example of a non-minimal recruitment sequence can be found in \REF{ex:baranova:2}. The scene features Maria and her adult daughters Katya and Olga. Maria is standing at the kitchen counter talking to Olga, who is in an adjacent room. At one point, Maria places a cup with hot water on the table for Katya who is about to make herself some instant coffee.

\transheader{ex:baranova:2}{20110827\_Family\_2\_820127}\vspace{2mm}
%
\begin{transbox}{1}{mar}
\begin{verbatim}
((puts a cup with hot water on the table in front of Katya))
\end{verbatim}
\end{transbox}
%
\begin{transbox}{2}{kat}%ya
\begin{verbatim}
[((opens up the bag of instant [coffee))
\end{verbatim}
\end{transbox}
%
\begin{transbox}{3}{mar}%ia
\begin{verbatim}
                               [nu  vot Ol’ka
                                PTC PTC Olia.DIM
\end{verbatim}
\hspace{4.7cm} so, Olia
\end{transbox}
%
\begin{transbox}{4}{~}
\begin{verbatim}
ja kartoshku-ta [padzha:rila,
I  potato.PTC    fried
\end{verbatim}
I fried the potatoes
\end{transbox}
%
\begin{transbox}{5}{kat}
\begin{verbatim}
                [((picks up the teaspoon from the table))
\end{verbatim}
\end{transbox}
%
\begin{transbox}{6}{mar}
\begin{verbatim}
shias nada,
now   need.MOD
\end{verbatim}
now \{I\} need
\end{transbox}
%
\begin{transbox}{7}{~}
\begin{verbatim}
ka[pu:staj    zaniatsa
cabbage.INSTR get_busy
\end{verbatim}
to get busy with the cabbage
\end{transbox}
%
\begin{mdframednoverticalspace}[style=firstfoc]
\begin{transbox}{8}{kat}
\begin{verbatim}
  [↑daj            lo:shku   dru[guju  pazhalu(sta)
   give.IMP.PFV.SG spoon.ACC other.ACC please
\end{verbatim}
\hspace{0.4cm} give \{me\} another spoon please
\end{transbox}
\end{mdframednoverticalspace}
%
\begin{mdframednoverticalspace}[style=secondfoc]
\begin{transbox}{9}{mar}
\begin{verbatim}
                                [↑lo:shku- (.) drug↑uju?
                                  spoon.ACC    other.ACC
\end{verbatim}
\hspace{5cm} a spoon? (.) another one?
\end{transbox}
\end{mdframednoverticalspace}
%
\begin{mdframednoverticalspace}[style=firstfoc]
\begin{transbox}{10}{kat}
\begin{verbatim}
uhu:m,
\end{verbatim}
uhu:m
\end{transbox}
\end{mdframednoverticalspace}
%
\begin{mdframednoverticalspace}[style=secondfoc]
\begin{transbox}{11}{mar}
\begin{verbatim}
[((opens the drawer))
\end{verbatim}
\end{transbox}
\end{mdframednoverticalspace}
%
\begin{mdframednoverticalspace}[style=firstfoc]
\begin{transbox}{12}{kat}
\begin{verbatim}
[ana v  malake: pa    xodu  dela     eta
 she in milk    along route business DEM.F
\end{verbatim}
\hspace{0.07cm} it looks like this one has been \{dipped\} in the milk
\end{transbox}
\end{mdframednoverticalspace}
%
\begin{transbox}{13}{mar}
\begin{verbatim}
ta    da:. v  malake:
DEM.F yes  in milk.LOC
\end{verbatim}
that one, yes, \{it’s been dipped\} in the milk
\end{transbox}
%
\begin{mdframednoverticalspace}[style=secondfoc]
\xtransbox{14}{~}{((gives a teaspoon to Katya))}
\end{mdframednoverticalspace}\vspace{-1mm}
%
\begin{transbox}{15}{kat}
\begin{verbatim}
spasiba ((putting coffee into her cup with the given spoon))
thanks
\end{verbatim}
thanks
\end{transbox}\bigskip

\begin{figure}
\includegraphics[height=.28\textheight]{figures/baranova-img003.jpg}
\caption{Katya is about to put instant coffee into the cup of hot water (\extref{ex:baranova:2}).}
\label{fig:baranova:3}
\end{figure}

At line 8, Katya initiates recruitment of Maria by using an imperative construction with rising-falling intonation. That is, she starts with a high pitch and end with a low one: \textit{↑daj lo:shku druguju pazhalusta} ‘↑give \{me\} another spoon please’. Instead of immediately complying, Maria initiates repair: ‘a spoon? (.) another one?’. With this repair initiation, she claims to have trouble hearing or understanding Katya's recruiting turn. Katya responds with the confirmation \textit{uhu:m} ‘uhu:m’ (line 10). It appears, however, that Maria’s ‘a spoon? (.) another one?’ is not a simple repair initiation; it also embodies a kind of challenge (see \citealt{Baranova2015}). Maria’s turn may be understood to be using a claim of trouble of hearing or understanding as a way to question the need for recruitment \citep[102--106]{Schegloff2007}. When Katya issues the request, she is holding a teaspoon (\figref{fig:baranova:3}), raising the obvious question as to why she cannot use the one she already has in her hand. At line 12, Katya expands on her initial recruiting move by orienting to just this question and supplying the reason: ‘it looks like this one has been \{dipped\} in the milk’. So the first part of Katya's response (line 10) targets the potential problem of hearing, while the second part offers a reason that defends the relevance and purpose of recruitment here (line 12).

Maria’s repair initiation (‘a spoon? (.) another one?’) delays compliance and expands the recruitment sequence into a non-minimal one, in which the recruiter supplies a repair solution and a reason to back up her original recruiting turn. Also, both lines 10 and 12 serve here as renewals of the original recruiting turn, making a response relevant \citep{Davidson1984,Pomerantz1984response}. Maria complies at line 14 by giving a clean teaspoon to Katya. The recruitment sequence is closed off with an acknowledgment \textit{spasiba} ‘thanks’ in line 15.

\subsection{Subtypes of recruitment sequence}\label{sec:baranova:2.3}

In the larger comparative project, we distinguish four main recruitment types based on the nature of the response by the recruitee (see \chapref{sec:coding}, \sectref{sec:coding:6}). As \tabref{tab:baranova:1} shows, the provision of a service is the most frequently encountered response type in the Russian sample. We saw an example of this in \REF{ex:baranova:1}, where the recipient Lida fulfilled the recruitment by pouring milk into a cup for the recruiter. Recruitments resulting in the passing or moving of an object fall into the category of object transfers as exemplified by \REF{ex:baranova:2}, in which a spoon was handed to the recruiter. The two remaining recruitment types involve alterations of trajectory of behavior (e.g. getting someone to desist from doing something) and assistance in response to visible or anticipatable trouble (e.g. open the door for someone when their hands are occupied). I discuss these in the following two extracts.

\begin{table}
\begin{tabularx}{.75\textwidth}{Xrr}
\lsptoprule
Recruitment sequence subtype & Count & Proportion\\
\midrule
Service provision & 121 & 61\%\\
Alteration of trajectory & 37 & 18\%\\
Object transfer & 29 & 15\%\\
Trouble assistance & 13 & 7\%\\
\lspbottomrule
\end{tabularx}
\caption{Relative frequencies of recruitment sequence subtypes (\textit{n}=200).}
\label{tab:baranova:1}
\end{table}

\hspace*{-1.2mm}Alterations of trajectory form the second largest group of recruitment sequence types in the current sample. In \REF{ex:baranova:3}, Marina is visiting her mother-in-law Anna. Both women are sitting at the kitchen table. Marina is holding her small dog on her lap and playing with it, while Anna is sitting next to her having dinner.

%\label{bkm:Ref327181049}
\newpage
\transheader{ex:baranova:3}{20110807\_Family\_evening\_1\_459097 }\vspace{2mm}
%
\begin{transbox}{1}{mar}
\begin{verbatim}
sla:tkaja maja: de- ((to dog))
sweet.F   my.F
\end{verbatim}
my sweet gi-
\end{transbox}
%
\begin{transbox}{2}{~}
\begin{verbatim}
[(devachka)
  girl
\end{verbatim}
\hspace{0.07cm} (girl)
\end{transbox}
%
\begin{mdframednoverticalspace}[style=firstfoc]
\begin{transbox}{3}{ann}
\begin{verbatim}
[nu  Marish,  [pusti:         ejo, ja pa- pae:m       spako:jna
 PTC Name.DIM  let_go.IMP.PFV her  I      eat.FUT.1SG quietly
\end{verbatim}
\hspace{0.07cm} sweet Marina, let her go \{so that\} I finish eating in peace
\end{transbox}
\end{mdframednoverticalspace}
%
\begin{transbox}{4}{~}
\begin{verbatim}
              [((waves with one hand from left to right))
\end{verbatim}
\end{transbox}
%
\emptytransbox{5}{(0.2)}
%
\begin{mdframednoverticalspace}[style=secondfoc]
\begin{transbox}{6}{mar}
\begin{verbatim}
ja sh  tibe       nichio, ni  eta.
I  PTC you.SG.DAT nothing NEG PTC
\end{verbatim}
but I nothing, well
\end{transbox}
\end{mdframednoverticalspace}
%
\begin{mdframednoverticalspace}[style=secondfoc]
\begin{transbox}{7}{~}
\begin{verbatim}
↑my sh  tibe       nich↑io ni  delaem,
 we PTC you.SG.DAT nothing NEG do.PL
\end{verbatim}
\hspace{0.07cm} but we aren’t doing anything to you
\end{transbox}
\end{mdframednoverticalspace}

\begin{figure}
\caption{Anna tells Marina to remove her dog from the table (\extref{ex:baranova:3}, line 3)}
\label{fig:baranova:4}
\includegraphics[height=.28\textheight]{figures/baranova-img004.png}
\end{figure}

Marina is playing with her dog at the table (lines 1--2). Assuming that there is a special relationship between dogs and their owners, Marina’s play with her dog might be seen as a private activity that does not include Anna. Nonetheless, Anna intervenes, which might be seen as a delicate matter. This may be why Anna’s recruiting move is accompanied by a reason: ‘\{so that\} I finish eating in peace’ (line 3). The request-reason combination implies that finishing eating the meal in peace is incompatible with the presence of the dog at the table. Marina orients to this negative implication by offering a counter-reason: ‘but we aren’t doing anything to you’ (line 7).

The examples discussed so far involve on-record verbal recruiting moves that make explicit the type of practical action being recruited. By contrast, recruitments of the trouble-assist type feature visible trouble but no on-record request to solve the trouble and no instruction as to how to do so. While there is no explicit initiation of recruitment, recruitees nevertheless provide assistance. This assistance may involve altering behavior, transferring an object, or performing a service. Why are such cases recruitments at all? They certainly share features with requests and other on-record recruitments. First, there is the issue of accountability. While a participant who merely sees that someone is in need is presumably less accountable for failing to assist than someone who is the addressee of an on-record recruitment, it can be argued that their failure will still be noticeable. Second, trouble-assist recruitments are hardly distinguishable from verbal recruiting moves that verbalize a trouble using a declarative statement. For instance, in \REF{ex:baranova:15} discussed later in this chapter, the recruiting turn consists of a declarative statement that makes the speaker's trouble clear: her toddler is chewing on a paper napkin. In response, the toddler’s grandmother takes the napkin away from her.

\extref{ex:baranova:4} illustrates how a participant can assist another person after observing the trouble that they are experiencing. This fragment is taken from a conversation between Inna and her adult niece Sasha. The women are in Inna's narrow kitchen when Sasha’s mobile phone starts ringing in the corridor. Sasha visibly struggles to stand up from the kitchen bench as the table blocks her movements.

\transheader{ex:baranova:4}{Niece\_1\_1517800}\vspace{2mm}
%
\emptytransbox{1}{((mobile phone rings))}
%
\begin{transbox}{2}{sas}
\begin{verbatim}
eta minia [kto-ta  patirial
DEM I.GEN somebody lost
\end{verbatim}
that's me somebody is looking for
\end{transbox}
%
\begin{mdframednoverticalspace}[style=firstfoc]
\begin{transbox}{3}{~}
\begin{verbatim}
          [((struggles standing up)) 
\end{verbatim}
\end{transbox}
\end{mdframednoverticalspace}
%
\begin{transbox}{4}{inn}
\begin{verbatim}
(tak) [(0.2) (   )
 so
\end{verbatim}
\hspace{0.07cm} so (0.2) (\hspace{0.5cm}) % \hspace{0.5cm}
\end{transbox}\vspace{1mm}
%
\begin{mdframednoverticalspace}[style=secondfoc]
\begin{transbox}{5}{~}
\begin{verbatim}
      [((pulls the table for Sasha to pass through))
\end{verbatim}
\end{transbox}
\end{mdframednoverticalspace}\vspace{-1mm}
%
\xtransbox{6}{sas}{((passes through the opening between the table and the kitchen cabinet))}

\begin{figure}
\caption{Inna pulls the table so that Sasha can leave (\extref{ex:baranova:4}, line 5)}
\label{fig:baranova:5}
\includegraphics[height=.28\textheight]{figures/baranova-img005.png}
\end{figure}

Sasha stands up from the kitchen bench with visible difficulty. She is squeezed between the table and the kitchen cabinets, unable to pass through. Inna is sitting just in front of Sasha. Inna pulls the table to make more space for Anna to leave (line 5, \figref{fig:baranova:5}), which Anna is then able to do.

In this example, Sasha does not explicitly recruit Inna’s assistance, but Inna gets it all the same. Important here is that Sasha might not be free to push the table forward as this would put Inna in an uncomfortable position. Also, Inna is the host here and bears some responsibility for the comfort of her guests. These features make it more likely that Inna will offer assistance without an on-record recruiting move being made.

To summarize, I have introduced four main recruitment types: performing services, transferring objects, altering behavior, and trouble assistance. While the first three types are straightforward and refer to the nature of the recruited action, the last type is different, but it should still be seen as belonging to the domain of recruitment in its broad definition (see Chapters 1--2). Trouble-assist recruitments do not involve an on-record initiating move. One of the participants assists another when a trouble manifests itself. This assistance can involve performing a service, transferring an object, or altering a behavior.

\section{Formats in Move A: The recruiting move}

While the previous section was mainly concerned with what kind of assistance or collaboration is being recruited, in this section the focus is on the format or formulation of the recruiting move. Numerous strategies are observed, the use of which is influenced by both the immediate situational context (\citealt{Rossi2015a}) and cultural preferences for (in)directness \citep{Ogiermann2009,bolden2017}. Initiating moves in recruitment sequences might be fully nonverbal, fully verbal, or a combination.

\subsection{Fully nonverbal recruiting moves}\label{sec:baranova:3.1}

In some situations, verbalizing a recruiting move appears unnecessary and a mere gesture might be clear enough to indicate what kind of assistance or collaboration is being called for. In \REF{ex:baranova:5}, Pavel is one of Anna's guests at a dinner gathering. The extract begins when Anna offers Pavel a drink.

% bkm:Ref447705622
\transheader{ex:baranova:5}{20120602\_family\_friends\_2\_1085520}\vspace{2mm}
%
\begin{transbox}{1}{ann}
\begin{verbatim}
Pavel ↑chaj kofe
name   tea  coffee
\end{verbatim}
Pavel, tea, coffee?
\end{transbox}
%
\emptytransbox{2}{(0.7)}
%
\begin{transbox}{3}{pav}
\begin{verbatim}
.hhhh chijku      esli tol'ka luchshe
      tea.DIM.GEN if   only   better
\end{verbatim}
.hhhh if \{possible\} better some tea % \hspace{0.9cm}
\end{transbox}
%
\begin{transbox}{4}{ann}
\begin{verbatim}
((takes a tea bag from [the box))
\end{verbatim}
\end{transbox}
%
\begin{transbox}{5}{pav}
\begin{verbatim}
                       [((lifts his cup and looks into it))
\end{verbatim}
\end{transbox}
%
\begin{transbox}{6}{~}
\begin{verbatim}
o:pa
INTJ
\end{verbatim}
oh
\end{transbox}
%
\begin{transbox}{7}{~}
\begin{verbatim}
u    minia eshio jest' An'
with I.GEN still is    name.VOC
\end{verbatim}
I still have some, Anna
\end{transbox}
%
\begin{transbox}{8}{ann}
\begin{verbatim}
[((turns to different speaker)) ↑Ir
                                 name.VOC
\end{verbatim}
\hspace{4.9cm} Ira?
\end{transbox}
%
\begin{transbox}{9}{pav}
\begin{verbatim}
[((finishes his tea))
\end{verbatim}
\end{transbox}
%
\emptytransbox{10}{(0.9)}
%
\begin{transbox}{11}{ira}
\begin{verbatim}
(ni  budu       [spasiba)
 NEG be.FUT.1SG  thanks
\end{verbatim}
\hspace{0.07cm} I won't, thank you
\end{transbox}
%
\begin{transbox}{12}{ann}
\begin{verbatim}
                [((puts tea bag on the [table ))
\end{verbatim}
\end{transbox}
%
\begin{mdframednoverticalspace}[style=firstfoc]
\begin{transbox}{13}{pav}
\begin{verbatim}
                                       [((holds out his cup for Anna))
\end{verbatim}
\end{transbox}
\end{mdframednoverticalspace}
%
\begin{mdframednoverticalspace}[style=secondfoc]
\xtransbox{14}{ann}{((puts tea bag into Pavel's cup))}
\end{mdframednoverticalspace}
%
\begin{mdframednoverticalspace}[style=secondfoc]
\xtransbox{15}{~}{((takes the cup, pours hot water into it, and gives it back to Pavel))}
\end{mdframednoverticalspace}

\begin{figure}
\includegraphics[height=.28\textheight]{figures/baranova-img006.png}
\caption{Pavel holds out his cup and Anna puts a tea bag into it (\extref{ex:baranova:5}, lines 13--14).}
\label{fig:baranova:6}
\end{figure}

Pavel accepts Anna's offer by specifying that he would like tea (line 3). Then he notices that there is still some tea left in his cup and this is what he tells Anna at line 7: ‘I still have some, Anna’. Anna treats this as a rejection of her offer because she immediately turns to Ira to offer tea to her. At line 13, Pavel holds out his cup towards Anna and she takes this gesture as a request for tea. With no questions asked she puts a teabag in Pavel’s cup and fills it with hot water. 

Such nonverbal recruiting moves can only be successful in environments that maximally disambiguate them (\citealt{Rossi2014} and \chapref{sec:rossi}, \sectref{sec:rossi:3.1}; see also Kendrick \chapref{sec:kendrick}, \sectref{sec:kendrick:4.1.3}; Zinken, \chapref{sec:zinken}, \sectref{sec:zinken:3.1}; Dingemanse, \chapref{sec:dingemanse}, \sectref{sec:dingemanse:3.4}). In our case, the meaning of Pavel’s gesture is clear in the context of the preceding offer sequence.

This class of nonverbal recruiting moves is different from those involved in trouble-assist type recruitments like \REF{ex:baranova:4}. Rather than simply making a problem visible in an off-record way, fully nonverbal recruiting moves like \REF{ex:baranova:5} involve on-record practices for soliciting a practical action by the recipient.

In my Russian recruitments corpus there are 31 fully nonverbal recruiting moves. This number is high compared to other languages examined in the comparative project (see other chapters in this volume). One reason for this is the relatively high number of cases in which speakers initiate clinking glasses with one another, thus getting the other to drink (see below).

\subsection{Nonverbal behavior in composite recruiting moves}

Recruiting moves are often composite utterances consisting of both verbal and nonverbal elements. In 87 cases in my corpus, recruiters combine nonverbal and verbal elements in Move A. Nonverbal elements observed in initiating moves are of four main types, as found across languages in this volume: pointing, holding out an object, reaching for an object, and iconic gestures (see \chapref{sec:coding}, \sectref{sec:coding:6}). In my Russian sample, two more specific subcategories can be identified: holding a glass/cup out for clinking and holding a glass/cup out to receive a drink (see \tabref{tab:baranova:2}). The emergence of these categories can be explained by a prevalence of celebratory gatherings in my sample of informal interactions. The category “other” in the table includes recruitments where a speaker places an object on the table for a recipient to take it or refill it.

\begin{table}
\begin{tabularx}{\textwidth}{Xrr}
\lsptoprule
Nonverbal practice & Count & Proportion\\
\midrule
Pointing & 21 & 24\%\\
Holding out an object to give & 19 & 22\%\\
Holding a glass/cup out for clinking & 16 & 18\%\\
Reaching out to receive an object & 16 & 18\%\\
Holding a glass/cup out for receiving a drink & 8 & 9\%\\
Iconic gesture & 2 & 2\%\\
Other & 5 & 6\%\\
\lspbottomrule
\end{tabularx}
\caption{Nonverbal practices in composite recruiting moves (\textit{n}=87).}
\label{tab:baranova:2}
\end{table}

In this section I illustrate some of the attested nonverbal practices. \extref{ex:baranova:6} gives us an example of reaching to receive an object. As Inna asks her husband Fyodor to pass her magnifying glass, she reaches out to receive it.

% \label{bkm:Ref327181072}
\transheader{ex:baranova:6}{20110816\_Sisters\_A\_1\_332247}\vspace{-1mm}
%
\begin{mdframednoverticalspace}[style=firstfoc]
%
\begin{transbox}{1}{inn}
\begin{verbatim}
dava[j            maju     lupu
give.IMP.IMPFV.SG my.ACC.F magnifying glass.ACC.F
\end{verbatim}
give \{me back\} my magnifying glass
\end{transbox}
\end{mdframednoverticalspace}
%
\begin{transbox}{2}{~}
\begin{verbatim}
    [((reaches out with her hand))
\end{verbatim}
\end{transbox}
%
\emptytransbox{3}{(0.3)}
%
\begin{transbox}{4}{fyo}
\begin{verbatim}
((puts his hand in the pocket of his [trousers))
\end{verbatim}
\end{transbox}
%
\begin{transbox}{5}{inn}
\begin{verbatim}
                                     [zabral u    minia =
                                      took.M from I.GEN
\end{verbatim}
\hspace{5.6cm} you took away my
\end{transbox}
%
\begin{transbox}{6}{~}
\begin{verbatim}
lupu
magnifying.glass.ACC.F
\end{verbatim}
magnifying glass
\end{transbox}
%
\emptytransbox{7}{(1.4)}
%
\begin{transbox}{8}{fyo}
\begin{verbatim}
jestestvena
naturally
\end{verbatim}
naturally
\end{transbox}
%
\emptytransbox{9}{(0.7)}
%
\xtransbox{10}{fyo}{((retrieves the magnifying glass from his pocket and hands it over to Inna))}\bigskip
%
\begin{figure}
\caption{Inna reaches with her hand towards Fyodor (\extref{ex:baranova:6}, line 2).}
\label{fig:baranova:7}
\includegraphics[height=.28\textheight]{figures/baranova-img007.jpg}
\end{figure}

\begin{figure}
\caption{Fyodor gives the magnifying glass to Inna (\extref{ex:baranova:6}, line 10).}
\label{fig:baranova:8}
\includegraphics[height=.28\textheight]{figures/baranova-img008.jpg}
\end{figure}

Inna uses the imperfective imperative verb \textit{davaj}. She complements the verbal component of her recruiting move with a gestural component: stretching out her hand in Fyodor's direction with her palm turned upwards (line 2, \figref{fig:baranova:7}). Inna holds this gesture until she receives her magnifying glass (line 10, \figref{fig:baranova:8}). Fyodor and Inna coordinate their moves: they bring their hands close to each other. An advantage of gestures such as reaching out is that they can persist through time in a way that a verbal utterance cannot. By holding the gesture after the verbal component of a recruiting move has been spoken, Inna may, for instance, emphasize the urgency of the recruitment and encourage prompt compliance. Another possible function of this gesture is to minimize Fyodor’s efforts, as he does not need to bring the magnifying glass all the way to Inna but only meet her hand halfway.

\extref{fig:baranova:7} illustrates the use of a pointing gesture in a recruiting move. Maria has just taken a seat on the kitchen bench with her back blocking the view of the video camera. Her daughter Katya alerts Maria to this problematic state of affairs. After Maria fails to respond, Katya makes an explicit request for Maria to change her position at the table and sit on the chair that she is pointing to with her index finger.

% \label{bkm:Ref447714269}
\transheader{ex:baranova:7}{20110827\_Family\_2\_437830}\vspace{2mm}
%
\begin{transbox}{1}{mar}
\begin{verbatim}
[Kir           padvin'sia ((to the cat))
 name:cat.VOC  move.over.IMP.PFV.SG
\end{verbatim}
\hspace{0.07cm} Kira, move over
\end{transbox}
%
\begin{transbox}{2}{~}
\begin{verbatim}
[((sits down on the kitchen bench next to the cat))
\end{verbatim}
\end{transbox}
%
\begin{transbox}{3}{kat}
\begin{verbatim}
[(   )
\end{verbatim}
\end{transbox}
%
\emptytransbox{4}{(0.4)}
%
\begin{transbox}{5}{kat}
\begin{verbatim}
ja patom k       kantsu u    nivo  zabrala,
I  later towards end    from him   took.away
\end{verbatim}
later, towards the end, I took \{it\} away from him
\end{transbox}
%
\emptytransbox{6}{(1.0)}
%
\begin{mdframednoverticalspace}[style=firstfoc]
\begin{transbox}{7}{kat}
\begin{verbatim}
e:ta
PTC
\end{verbatim}
well
\end{transbox}
\end{mdframednoverticalspace}
%
\emptytransbox{8}{(0.3)}
%
\begin{mdframednoverticalspace}[style=firstfoc]
\begin{transbox}{9}{kat}
\begin{verbatim}
ty     naverna  [sela  v't kak  ras
you.SG probably  sat.F  DEM just right
\end{verbatim}
you’ve probably sat down exactly
\end{transbox}
\end{mdframednoverticalspace}
%
\begin{transbox}{10}{~}
\begin{verbatim}
                [((points at camera))
\end{verbatim}
\end{transbox}
%
\emptytransbox{11}{(0.9)}
%
\begin{mdframednoverticalspace}[style=firstfoc]
\begin{transbox}{12}{kat}
\begin{verbatim}
zakrylasia     [na stul  tuda  sadis'
covered.REFL.F  on chair there sit
\end{verbatim}
\{it\} got obscured, sit on the chair there
\end{transbox}
\end{mdframednoverticalspace}
%
\begin{mdframednoverticalspace}[style=firstfoc]
\begin{transbox}{13}{~}
\begin{verbatim}
               [((points at chair with index finger))
\end{verbatim}
\end{transbox}
\end{mdframednoverticalspace}
%
\emptytransbox{14}{(0.6)}
%
\begin{mdframednoverticalspace}[style=secondfoc]
\begin{transbox}{15}{mar}
\begin{verbatim}
((shifts on the kitchen bench))
\end{verbatim}
\end{transbox}
\end{mdframednoverticalspace}

\begin{figure}
\caption{Katya points with her index finger (\extref{ex:baranova:7}, line 13).}
\label{fig:baranova:9}
\includegraphics[height=.28\textheight]{figures/baranova-img009.jpg}
\end{figure}

In the initiating move, Katya produces a statement: ‘well (0.3) you’ve probably sat down exactly’ (lines 7--9). Intonationally and informationally, this statement sounds unfinished. The accompanying point to the camera (line 10), however, completes the trouble statement and clarifies its import. When Maria fails to respond, Katya adds more information about the problem together with an explicit recruiting component: ‘\{it\} got obscured, sit on the chair there’ (line 12). This turn further explicates the problem and offers a solution for it. At the same time, Katya now points to the chair with her index finger (line 13, \figref{fig:baranova:9}). In response, Maria partially complies. Instead of taking a seat on the pointed-to chair, she shifts on the bench, partially uncovering the view of the camera.

In the next example, the recruiting move contains two relevant nonverbal elements: holding out an object and pointing (in this case, a head point). The example is from an interaction between Sasha, the host, and Ksenia, a friend. Earlier in the interaction, Sasha presented Ksenia with one of her own photographs as a gift. Here, Ksenia asks Sasha to put the photograph on top of her jacket.

\transheader{ex:baranova:8}{20110826\_Old\_friends\_B\_1\_550898}\vspace{-1mm}
%
\begin{mdframednoverticalspace}[style=firstfoc]
\begin{transbox}{1}{kse}
\begin{verbatim}
Sash     palazhi        mne   na ku:rtku a?
name.VOC put.IMP.PFV.SG I.DAT on jacket  PTC
\end{verbatim}
Sasha, put \{it\} on top of my jacket eh?
\end{transbox}
\end{mdframednoverticalspace}
%
\begin{transbox}{2}{~}
\begin{verbatim}
((holds out photograph))
\end{verbatim}
\end{transbox}
%
\begin{transbox}{3}{~}
\begin{verbatim}
[tam ((head point))
 there
\end{verbatim}
\hspace{0.07cm} there
\end{transbox}
%
\begin{mdframednoverticalspace}[style=secondfoc]
\begin{transbox}{4}{sas}
\begin{verbatim}
((takes the photograph and leaves))
\end{verbatim}
\end{transbox}
\end{mdframednoverticalspace}

\begin{figure}
\caption{Ksenia holds out the photograph; Sasha reaches out to take it (\extref{ex:baranova:8}).}
\label{fig:baranova:10}
\includegraphics[height=.28\textheight]{figures/baranova-img010.jpg}
\end{figure}

Ksenia instructs Sasha to put the photograph on top of Ksenia's jacket, which is in the corridor. During the production of the request, she holds the photograph and stretches her arm in Sasha's direction, gazing at her. Subsequently, she verbally refers to the place where the photograph should be put: \textit{tam} ‘there’. At the same time, she head-points in the direction of the corridor. Sasha takes the picture and leaves the room.

Ksenia and Sasha’s hand gestures in \REF{ex:baranova:8} are similar to those we saw in \REF{ex:baranova:6}. In that example, Inna’s reaching for the magnifying glass facilitated Fyodor’s action of handing it to her. Likewise, in \REF{ex:baranova:8}, Ksenia’s holding out the photograph makes it easier for Sasha to take it. By easier I mean that the distance Sasha’s hand has to travel to take the object is shorter. Another example of holding-out gesture can be found in \REF{ex:baranova:5}, where the speaker holds out his cup for the recipient to take it and fill it with tea.

In this section I have described some nonverbal elements observed in recruiting moves, including pointing gestures, holding out objects, and reaching for objects. Nonverbal elements not only complement recruiting turns with relevant information, they also pursue and facilitate compliance.

\subsection{Verbal elements: construction types and subtypes}\label{sec:baranova:3.3}

The Russian language provides its speakers with a multitude of linguistic formats to initiate recruitments. The relative frequency of imperatives, declaratives, interrogatives, and no-predicate constructions are shown in \tabref{tab:baranova:3}. \chapref{sec:coding}, \sectref{sec:coding:6} explains in more detail how recruiting moves were coded for verbal elements. In this section, I discuss the Russian imperative, declarative, and interrogative forms. For an example of a no-predicate construction I refer the reader to \REF{ex:baranova:1} above: ‘some milk’ is a minimally designed recruiting turn only containing a reference to the object being requested.

\begin{table}
\begin{tabularx}{.66\textwidth}{Xrr}
\lsptoprule
Sentence type & Count & Proportion\\
\midrule
Imperative & 100 & 63\%\\
No predicate & 25 & 16\%\\
Declarative & 18 & 11\%\\
Interrogative & 16 & 10\%\\
\lspbottomrule
\end{tabularx}
\caption{Sentence type of recruiting moves including spoken elements (\textit{n}=159). For non-minimal recruitment sequences, only the first attempt is counted in this table.}
\label{tab:baranova:3}
\end{table}

I have analyzed and coded the linguistic structure of recruiting turns based on their sentence type only, without reference to their intonational contour. Note that, in Russian, imperative utterances can have rising-falling intonation. I discuss this issue in the following section on imperatives.

\subsubsection{Imperatives}

Imperatives are the most frequent format of recruiting moves in my sample of informal Russian interactions. Imperatively formatted moves have recently been identified as a default strategy in another corpus of informal Russian interaction \citep{bolden2017}. While research on Italian and English has shown that imperatives are typically used in the context of ongoing joint projects in which the recruitee’s commitment has been secured (e.g. \citealt{Rossi2012,Rossi2015a,Zinken2016}), Russian imperatives appear to be used in a broader range of recruitment contexts. \citet{bolden2017} shows that Russian imperatives are used in contexts where Italian and English speakers would normally opt for interrogatives. These are contexts where the i) recipient is not yet committed to the project of which the recruitment is part, ii) contexts where compliance requires the recipient to abandon or alter his/her own ongoing activity, and iii) contexts where compliance involves a relatively elaborate physical activity.

Aspect is important in the Russian verb system. As mentioned in the introduction to this chapter, Russian verbs can be perfective or imperfective. It is not yet entirely clear which recruitment contexts prefer which aspectual type, but it appears that the imperfective is more often used when compliance can be expected (\citealt{Benacchio2002}; cf. Zinken, \chapref{sec:zinken}, \sectref{sec:zinken:3.3.1} on the usage of perfective and imperfective imperatives in Polish).

Imperatives can be singular or plural. Singular imperatives are used when there is just one addressee. Additionally, imperatives may be of the perfective aspectual type (see Extracts \ref{ex:baranova:2}, \ref{ex:baranova:3}, and \ref{ex:baranova:8}) or the imperfective aspectual type (see \extref{ex:baranova:6}). Perfective and imperfective imperatives have a plural form when there are multiple recipients or when the second person polite plural pronoun \textit{vy} is used. The latter situation is illustrated in the next example. The recruitment sequence in \REF{ex:baranova:9} comes from a conversation between school custodians who are having lunch in the staff room. Alifa is about to join her colleagues at the table who are already having soup.

\transheader{ex:baranova:9}{201220120\_colleagues\_casual\_2\_498040 }\vspace{2mm}
%
\begin{transbox}{1}{ali}
\begin{verbatim}
((takes a bowl from the cabinet and puts it on the table next to Anna))
\end{verbatim}
\end{transbox}
%
\begin{transbox}{2}{ann}
\begin{verbatim}
[(losh)ku (  )
 spoon-ACC
\end{verbatim}
\hspace{0.07cm} a spoon (\hspace{0.4cm})
\end{transbox}
%
\begin{mdframednoverticalspace}[style=firstfoc]
\begin{transbox}{3}{ali}
\begin{verbatim}
[Anna- Anna Batkiyevna,
 name  name patronymic
\end{verbatim}
\hspace{0.07cm} Anna- Anna the daughter of a father
\end{transbox}
\end{mdframednoverticalspace}
%
\begin{transbox}{4}{ann}
\begin{verbatim}
aye:
INTJ
\end{verbatim}
hey
\end{transbox}
%
\emptytransbox{5}{(0.3)}
%
\begin{mdframednoverticalspace}[style=firstfoc]
\begin{transbox}{6}{ali}
\begin{verbatim}
pazhalsta  nakla:dyvajte    mneh
please     put.IMP.IMPFV.PL I.DAT
\end{verbatim}
you may do \{some\} serving for me please
\end{transbox}
\end{mdframednoverticalspace}
%
\emptytransbox{7}{(0.4)}
%
\begin{transbox}{8}{ver}
\begin{verbatim}
khahahm[hmhm ((laughter))
\end{verbatim}
\end{transbox}
%
\begin{transbox}{9}{(mar)}
\begin{verbatim}
       [(Ret')kiyevna
         patronymic
\end{verbatim}
\hspace{1.2cm} daughter of the (radish)
\end{transbox}
%
\begin{mdframednoverticalspace}[style=secondfoc]
\begin{transbox}{10}{ann}
\begin{verbatim}
[((puts her loaf of bread on the table))
\end{verbatim}
\end{transbox}
\end{mdframednoverticalspace}
%
\begin{transbox}{11}{?}
\begin{verbatim}
[h.hehehe
\end{verbatim}
\end{transbox}
%
\emptytransbox{12}{(0.5)}
%
\begin{mdframednoverticalspace}[style=secondfoc]
\begin{transbox}{13}{ann}
\begin{verbatim}
((takes Alifa’s bowl from the table))
\end{verbatim}
\end{transbox}
\end{mdframednoverticalspace}
%
\begin{transbox}{14}{ali}%ifa
\begin{verbatim}
[ty    zhe  po:var  u    na[s
 you.SG PTC  chef   with us
\end{verbatim}
\hspace{0.07cm} you are our chef here
\end{transbox}
%
\begin{mdframednoverticalspace}[style=secondfoc]
\begin{transbox}{15}{ann}
\begin{verbatim}
[((stands up and starts serving the soup))
\end{verbatim}
\end{transbox}
\end{mdframednoverticalspace}\vspace{-1mm}
%
\begin{transbox}{16}{~}
\begin{verbatim}
                            [eta to:chna
                             it  exactly
\end{verbatim}
\hspace{4.3cm} exactly
\end{transbox}\bigskip

\begin{figure}
\caption{Alifa makes a request for Anna to serve her some soup (\extref{ex:baranova:9}).}
\label{fig:baranova:11}
\includegraphics[height=.28\textheight]{figures/baranova-img011.jpg}
\end{figure}

This recruitment sequence is rather complex. Here I focus only on the format of the imperative Alifa uses to recruit Anna’s assistance. Alifa asks Anna to serve her some soup starting in line 3, where she draws Anna's attention with a non-serious patronymic: ‘Anna- Anna the daughter of a father’. A patronymic is formed by adding a suffix to the father's first name. Ivan's daughter, for instance, would have \textit{Ivanovna} as her patronymic. The use of a patronymic goes hand in hand with the use of plural ‘you’ as a polite form of address. The formation of a patronymic using the generic ‘father’, as Alifa does here, is occasionally used as a joking patronymic when the real one is not known. Alifa continues with ‘you may do \{some\} serving for me please’. Her use of the imperfective imperative with the plural ending \textit{-te} makes it clear that she is addressing the recipient with the plural ‘you’.

Another example of an imperative with a plural ending is given in \REF{ex:baranova:10}. This time, the imperative is of the perfective type and is directed to multiple recipients. Pavel is about to go to the village brook with a family guest. The host of the gathering, Inna, is trying to persuade her son Dennis and grandson Kostia to join them.

\transheader{ex:baranova:10}{Family\_dinner\_Country\_A\_2\_1038680}\vspace{2mm}
%
\begin{transbox}{1}{lid}
\begin{verbatim}
na::: e:: ↑na ruchej    pajdiote?
to         to brook.ACC go.FUT.2PL
\end{verbatim}
to::: uhm:: are you guys going to the brook?
\end{transbox}
%
\begin{transbox}{2}{pav}
\begin{verbatim}
((head nod))
\end{verbatim}
\end{transbox}
%
\begin{mdframednoverticalspace}[style=firstfoc]
\begin{transbox}{3}{inn}
\begin{verbatim}
aha, eh [Dennis ajda-te     [sxadite       Kostia,
uhuh     name  PTC.IMP-PL   go.IMP.PFV.2PL name
\end{verbatim}
yeah, eh, Dennis come on you guys, go ((plural)) Kostia
\end{transbox}
\end{mdframednoverticalspace}
%
\begin{transbox}{4}{~}
\begin{verbatim}
        [((touches Dennis)) [((reaches for Kostia’s arm))
\end{verbatim}
\end{transbox}
%
\begin{transbox}{5}{kos}
\begin{verbatim}
                            [((turns his head and looks at Inna))
\end{verbatim}
\end{transbox}
%
\begin{mdframednoverticalspace}[style=secondfoc]
\begin{transbox}{6}{~}
\begin{verbatim}
pajdiom ((looks at Dennis))
go-FUT-1PL
\end{verbatim}
shall we go?
\end{transbox}
\end{mdframednoverticalspace}
%
\begin{mdframednoverticalspace}[style=secondfoc]
\begin{transbox}{7}{den}
\begin{verbatim}
((bows his head to the side))
\end{verbatim}
\end{transbox}
\end{mdframednoverticalspace}

\begin{figure}
\includegraphics[height=.28\textheight]{figures/baranova-img012.jpg}
\caption{Inna touches Dennis' arm as she recruits him and Kostia to go to the brook (\extref{ex:baranova:10}, lines 3--4).}
\label{fig:baranova:12}
\end{figure}

\begin{figure}
\includegraphics[height=.28\textheight]{figures/baranova-img013.jpg}
\caption{Inna reaches for Kostia's arm as she recruits him and Dennis to go to the brook (\extref{ex:baranova:10}, lines 3--4).}
\label{fig:baranova:13}
\end{figure}

In line 1, Lida is talking to her husband Pavel, who is standing outside the window. She is asking whether Pavel is indeed going to take their guest to the village brook. After Pavel’s confirmation at line 2, Inna initiates a recruitment involving Dennis and Kostia: \textit{aha, eh Dennis ajda-te sxadite Kostia} ‘yeah, eh, Dennis come on you guys, go ((plural)) Kostia’. As in \REF{ex:baranova:9}, the imperative \textit{sxadi}\textbf{\textit{te}} ‘go’ has the plural ending \textit{-te}. While Kostia indirectly agrees to go to the brook by inviting Dennis at line 6 (‘shall we go?’), Dennis seems reluctant (line 7). His lack of response is noticeable. In the interaction subsequent to the extract, Dennis provides several reasons why he cannot go, while Inna insists on his participation. Dennis and Kostia do end up joining Pavel and the family guest on their walk to the brook.

Occasionally, Russian imperatives are combined with particles. For instance, the imperative in \REF{ex:baranova:3} contains the sentence-initial particle \textit{nu} (\textit{nu Marish, pusti: ejo, ja pa- pae:m spako:jna} ‘sweet Marina, let her go \{so that\} I finish eating in peace’). This particle has multiple functions, one of which is to convey insistence on carrying out the requested action (\citealt{Bolden2016,bolden2017}). Imperatives can also sometimes be preceded by the particle \textit{na}. This particle is also found as a stand-alone directive. In spoken Russian it conveys the meaning of ‘take’ or ‘here you are’, as illustrated in \REF{ex:baranova:11}. In this example, Fyodor and Inna are visiting their daughter Nadya and Nadya's young daughter. Inna is entertaining the toddler.

\transheader{ex:baranova:11}{Granddaughter\_605308}\vspace{-1mm}
%
\begin{mdframednoverticalspace}[style=firstfoc]
\begin{transbox}{1}{nad}
\begin{verbatim}
na  daj  yej von vazachku ana budit   sidet'  s    nej ((to Inna))
PTC give her PTC vas.DIM  she will be sit.INF with her
\end{verbatim}
take \{this\}, give her this little basket, she'll be sitting with it
\end{transbox}
\end{mdframednoverticalspace}
%
\begin{transbox}{~}{~}
\begin{verbatim}
((puts the basket with sweets in toddler's hands))
\end{verbatim}
\end{transbox}
%
\emptytransbox{2}{(0.3)}
%
\begin{mdframednoverticalspace}[style=firstfoc]
\begin{transbox}{3}{nad}
\begin{verbatim}
na. ((to toddler))
PTC
\end{verbatim}
take / here you are
\end{transbox}
\end{mdframednoverticalspace}
%
\begin{mdframednoverticalspace}[style=firstfoc]
\begin{transbox}{4}{~}
\begin{verbatim}
idi ba:be           (.) ba:bu           ugashiaj        kanfetkami
go  grandmother.DAT     grandmother.ACC treat.IMP.IMPFV sweets.INSTR
\end{verbatim}
go, for grandmother (.) treat grandmother with sweets
\end{transbox}
\end{mdframednoverticalspace}
%
\begin{transbox}{5}{inn}
\begin{verbatim}
((returns the basket to the table))
\end{verbatim}
\end{transbox}\\

\begin{figure}
\caption{Nadya lifts the basket from the table to give it to the child (\extref{ex:baranova:11}, line 1).}
\label{fig:baranova:15}
\includegraphics[height=.28\textheight]{figures/baranova-img014.png}
\end{figure}

Nadya's first recruiting turn (line 1) is directed at Inna. The turn begins with the particle \textit{na}, which has the meaning of ‘take’ in this case. The particle is followed by the verb \textit{daj} ‘give’. This results in a complex recruiting expression that combines the actions of ‘taking’ and ‘giving’. Nadya's second recruiting turn (lines 3--4) is directed at the toddler and again begins with the particle \textit{na} and combines multiple actions: ‘take’, ‘go’, ‘treat’. She says: ‘take, go, for grandmother (.) treat grandmother with sweets’.

Another imperative type makes use of a double-verb construction, where the first verb has a frozen imperative form and the second denotes the required action (see \citealt{Zinken2013} and \chapref{sec:zinken}, \sectref{sec:zinken:3.3.2} on a comparable construction in Polish). The construction in Russian combines the verb ‘give’ with a relevant action verb. Its use is shown in \REF{ex:baranova:12}. Participants in this interaction are friends who have gathered at Ksenia’s apartment for dinner and drinks. Ksenia’s elderly mother enters the room, where people are seated, and addresses a request to Ksenia.

% \label{bkm:Ref328992720}
\transheader{ex:baranova:12}{20110813\_School\_Friends\_2\_618255a}\vspace{2mm}
%
\begin{transbox}{1}{mom}
\begin{verbatim}
Ksiush  [pirashki-
name.VOC pastries
\end{verbatim}
Ksenia, pastry
\end{transbox}
%
\begin{transbox}{2}{kse}
\begin{verbatim}
        [Vo:f
         name.VOC
\end{verbatim}
\hspace{1.2cm} Vova
\end{transbox}
%
\begin{mdframednoverticalspace}[style=firstfoc]
\begin{transbox}{3}{mom}
\begin{verbatim}
ty  davaj    vyzyvaj        etaj (.) gazafshi:tsu [nu-,
you give.IMP call.IMP.IMPFV DEM.F    gas.worker    PTC
\end{verbatim}
go ahead call the (.) gas worker
\end{transbox}
\end{mdframednoverticalspace}
%
\begin{mdframednoverticalspace}[style=secondfoc]
\begin{transbox}{4}{kse}
\begin{verbatim}
                                                  [nu xva:tit
                                                   PTC enough
\end{verbatim}
\hspace{7.4cm} enough \{already\}
\end{transbox}
\end{mdframednoverticalspace}
%
\begin{mdframednoverticalspace}[style=secondfoc]
\begin{transbox}{5}{~}
\begin{verbatim}
                                                 [((waves))
\end{verbatim}
\end{transbox}
\end{mdframednoverticalspace}
%
\begin{transbox}{6}{mom}
\begin{verbatim}
kak su:xa ta
how dry   PTC
\end{verbatim}
but how dry
\end{transbox}\bigskip

\begin{figure}
\includegraphics[height=.28\textheight]{figures/baranova-img015.png}
\caption{Mom requests that Ksenia make a call (\extref{ex:baranova:12}, line 3).}
\label{fig:baranova:14}
\end{figure}

Mom's request (\textit{ty davaj vyzyvaj etaj (.) gazafshi:tsu} ‘go ahead call the (.) gas worker’) consists of the frozen imperative \textit{davaj} combined with the imperative verb expressing the requested action, here \textit{vyzyvaj} ‘call’. In this context, where no object transfer is involved, \textit{davaj} loses its independent meaning of ‘giving’ and takes on the meaning of ‘come on’ or ‘go ahead’. Mom’s request to call gas services is met with resistance by her daughter Ksenia, who says ‘enough \{already\}’ and literally waves the recruitment away (lines 4--5). After this response, Mom supports her recruitment with a reason (‘but how dry’). As it becomes clear from the unfolding conversation, she is referring to the pastry that was made earlier in the day which turned out dry due to presumed problems with the gas (see also line 1).

For interrogatives, Russian mainly relies on intonation along with the use of in-situ question words and particles. An imperative construction can be given a question-like quality by means of intonation, which can be applied to any relevant turn-constructional unit. This leads to a hybrid recruiting format containing both imperative and interrogative features. Such imperatives with interrogative features can be found in \REF{ex:baranova:2} and \REF{ex:baranova:8}. The recruiting turn in \REF{ex:baranova:2} has rising-falling intonation on the word ‘give’: \textit{↑daj lo:shku druguju pazhalusta} ‘↑give me another spoon please’. In \REF{ex:baranova:8}, the interrogation is done with the final particle \textit{a?} uttered with rising intonation: \textit{Sash, palazhi mne na ku:rtku a?} ‘Sasha, put \{it\} on top of my jacket eh?’. These cases can be contrasted with imperatives containing no interrogative features in \REF{ex:baranova:3}, \REF{ex:baranova:6}, \REF{ex:baranova:7}, \REF{ex:baranova:11}, \REF{ex:baranova:12}.

So, even though imperatives in Russian are used in a wider range of contexts than, for instance, in Italian and English, the Russian system of imperatives shows greater diversity, involving aspectual pairs (imperfective and perfective), distinction in number (singular and plural), the use of interrogative features, and diminutives particles on the verb (see \sectref{sec:baranova:3.4.1}).

\subsubsection{Interrogatives}

Although imperatives can be considered a default way of recruiting assistance and collaboration in Russian, we also find recruiting turns that are interrogatively formatted. In the next example, Ksenia is visiting her friend Sasha. She asks whether Sasha will let out a guest who is already in the corridor and about to leave her place.

\transheader{ex:baranova:13}{20110826\_Old\_friends\_A\_2\_66555}\vspace{2mm}
%
\begin{transbox}{1}{kse}
\begin{verbatim}
((points towards the corridor))
\end{verbatim}
\end{transbox}
%
\emptytransbox{2}{(0.3)}
%
\begin{transbox}{3}{kse}
\begin{verbatim}
pra↑vodish=
let_out.FUT.2SG
\end{verbatim}
will you let \{her\} out?
\end{transbox}
%
\begin{transbox}{4}{sas}
\begin{verbatim}
=uhum,
\end{verbatim}
\hspace{0.07cm} uhum
\end{transbox}
%
\begin{transbox}{5}{~}
\begin{verbatim}
((leaves the room to let the guest out))
\end{verbatim}
\end{transbox}\bigskip

\begin{figure}
\includegraphics[height=.28\textheight]{figures/baranova-img016.jpg}
\caption{Ksenia points towards the corridor (\extref{ex:baranova:13}, line 1).}
\label{fig:baranova:16}
\end{figure}

First, Ksenia points to the corridor (line 1). When there is no response, she asks ‘will you let \{her\} out?’. This recruiting turn has an interrogative format. Similarly formatted recruiting turns in English (‘would/will you x?’) tend to occur when where are perceived contingencies or obstacles to compliance, and where the recruiter has a low degree of entitlement to make the request \citep{CurlDrew2008}. In this case, Ksenia’s entitlement is an issue. By asking whether Sasha will let her guest out implies that Sasha has failed in her responsibilities as a host. Ksenia, too, is a guest here and her entitlement to initiate such a recruitment is arguably low.

In another type of interrogative strategy, \textit{wh}-questions can also be used to recruit assistance. In \REF{ex:baranova:14}, several girlfriends are looking at Sasha's photographs. Ksenia is curious about the photographs that Sasha and Lida are talking about.

\transheader{ex:baranova:14}{Old\_friends\_B\_1\_302784}\vspace{-1mm}
%
\begin{mdframednoverticalspace}[style=firstfoc]
%
\begin{transbox}{1}{kse}
\begin{verbatim}
chio tam?
what there
\end{verbatim}
what's there?
\end{transbox}
\end{mdframednoverticalspace}
%
\begin{transbox}{2}{lid}
\begin{verbatim}
u    tibia   ↑dve   takix?
with you.GEN  two.F such
\end{verbatim}
do you have two of these?
\end{transbox}
%
\begin{transbox}{3}{sas}
\begin{verbatim}
nave:rna u    minia vot-
probably with I.GEN PTC
\end{verbatim}
probably, I have-
\end{transbox}
%
\begin{mdframednoverticalspace}[style=firstfoc]
\begin{transbox}{4}{kse}
\begin{verbatim}
kakie      paka[zhi,
which.Q.PL show.IMP.PFV.SG
\end{verbatim}
which ones, show \{me\}
\end{transbox}
\end{mdframednoverticalspace}
%
\begin{mdframednoverticalspace}[style=secondfoc]
\begin{transbox}{5}{lid}
\begin{verbatim}
               [((looks at Ksenia and turns the photograph so that it  
                faces Ksenia))
\end{verbatim}
\end{transbox}
\end{mdframednoverticalspace}\bigskip
%
%\begin{transbox}{6}{sas}
%\begin{verbatim}
%                [katorye vo:t ana (  ) tozhe eta na ploshidi
%\end{verbatim}
%\end{transbox}\bigskip

\begin{figure}
\includegraphics[height=.28\textheight]{figures/baranova-img017.jpg}
\caption{Lida shows the photograph she is holding to Ksenia (\extref{ex:baranova:14}, line 5).}
\label{fig:baranova:17}
\end{figure}
\largerpage
Sasha and Lida are talking about one of Sasha’s photographs, which Lida is holding in her hands. Already at line 1, Ksenia expresses her interest in the pictures by asking: ‘what’s there?’, which can be seen as the initiation of the recruitment sequence. Her interest becomes even clearer when Ksenia uses the interrogative ‘which one’ at line 4, together with the imperative ‘show \{me\}’. Ksenia is entering into someone else’s ongoing conversation. This justifies the use of the interrogative construction. The recruitment is immediately fulfilled as Lida turns the photograph around so Ksenia can see it.

\subsubsection{Declaratives}\label{sec:baranova:3.3.3}

We have seen that sometimes participants in interaction notice that someone is in need of help and they offer practical assistance without being explicitly asked to do so (see \extref{ex:baranova:4}). In other cases, the need for help might not be that apparent and the nature of the trouble needs to be verbalized. This is demonstrated by the following extract from an interaction between family members who gathered at Nadya's place. Inna is holding Nadya's toddler on her lap when Nadya initiates a recruitment.

\transheader{ex:baranova:15}{Family\_dinner\_B\_2\_649099}\vspace{-1mm}
%
\begin{mdframednoverticalspace}[style=firstfoc]
\begin{transbox}{1}{nad}
\begin{verbatim}
ma:m     ana salfetku   von   zhujot
mama.VOC she napkin.ACC there chew.3SG
\end{verbatim}
Mom, she's chewing on the napkin
\end{transbox}
\end{mdframednoverticalspace}
%
\begin{transbox}{2}{inn}
\begin{verbatim}
((leans her head towards the toddler))
\end{verbatim}
\end{transbox}
%
\begin{transbox}{3}{~}
\begin{verbatim}
e: ((to the toddler)) 
INTJ
\end{verbatim}
hey
\end{transbox}
%
\emptytransbox{4}{(0.4)}
%
\begin{transbox}{5}{inn}
\begin{verbatim}
e:
INTJ
\end{verbatim}
hey
\end{transbox}
%
\begin{transbox}{6}{~}
\begin{verbatim}
((removes the napkin from toddler’s hands))
\end{verbatim}
\end{transbox}\bigskip

\begin{figure}
\includegraphics[height=.28\textheight]{figures/baranova-img018.png}
\caption{Nadya tells Inna than her toddler is chewing on a napkin (\extref{ex:baranova:15}, line 1).}
\label{fig:baranova:18}
\end{figure}

Instead of instructing Inna to remove the napkin from her toddler's mouth, Nadya simply describes the problem that needs to be addressed (see also Kendrick, \chapref{sec:kendrick}, \sectref{sec:kendrick:4.2.3}; Rossi, \chapref{sec:rossi}, \sectref{sec:rossi:3.3.4}; Enfield, \chapref{sec:enfield}, \sectref{sec:enfield:4.3.1}; Dingemanse, \chapref{sec:dingemanse}, \sectref{sec:dingemanse:3.2.2}). Inna responds by leaning towards the child and removing the napkin. Nadya’s request is indirect as it does not explicitly ask for any assistance and does not specify the practical action required from the recipient. Inna, however, acts quickly and removes the napkin from the toddler. 

\hspace*{-.5mm}Recruitments of the trouble-assist kind (\sectref{sec:baranova:2.3}) and statements of the kind shown in \REF{ex:baranova:15} are similar in certain respects. Seeing or being alerted to a source of trouble and being able to act upon it seems sufficient for the recruitee to step in and solve the problem. Note that Nadya is facing the child and has better visual access to the toddler's behavior than Inna does \citep[see also][]{Rossi2018}. On the other hand, Nina is in a better position to solve the problem because she is the one closest to the toddler.

Another way of conveying that some action is required is to state that it ‘needs’ to happen (see also Floyd, \chapref{sec:floyd}, \sectref{sec:floyd:3.3.4}; Rossi, \chapref{sec:rossi}, \sectref{sec:rossi:3.3.4}; Zinken, \chapref{sec:zinken}, \sectref{sec:zinken:3.3.2}). At a memorial dinner, with the entire family present, Pavel’s daughter Lena asks whether it is necessary to eat the rice porridge. In what follows, Pavel tries to convince Lena to eat the porridge that is traditionally consumed at memorials.

% \label{bkm:Ref329010309}
\newpage
\transheader{ex:baranova:16}{memorial\_1\_424599}\vspace{2mm}
%
\begin{transbox}{1}{len}
\begin{verbatim}
a   chio mnoga kashi    nada     sjest’ -ta?
PTC what a.lot porridge need.MOD eat.INF-PTC
\end{verbatim}
does one need to eat a lot of porridge?
\end{transbox}
%
\begin{transbox}{2}{pav}
\begin{verbatim}
ne:t
\end{verbatim}
n:o
\end{transbox}
%
\begin{mdframednoverticalspace}[style=firstfoc]
\begin{transbox}{3}{~}
\begin{verbatim}
Le:n. (.) hm (.) nada     abiza:til'na=
name.VOC         need.MOD necessarily
\end{verbatim}
Lena (.) hm (.) one necessarily needs
\end{transbox}
\end{mdframednoverticalspace}
%
\begin{mdframednoverticalspace}[style=firstfoc]
\begin{transbox}{4}{~}
\begin{verbatim}
=etu vot kashu   sjest'
 DEM PTC porrige eat.INF.PFV
\end{verbatim}
\hspace{0.07cm} to eat this porridge
\end{transbox}
\end{mdframednoverticalspace}
%
\emptytransbox{5}{(0.7)}
%
\begin{mdframednoverticalspace}[style=firstfoc]
\begin{transbox}{6}{pav}
\begin{verbatim}
[lozhichku
 spoon.DIM.ACC
\end{verbatim}
\hspace{0.07cm} a little spoon
\end{transbox}
\end{mdframednoverticalspace}
%
\begin{transbox}{7}{~}
\begin{verbatim}
[((scoops some rice with a spoon))
\end{verbatim}
\end{transbox}
%
\begin{transbox}{8}{~}
\begin{verbatim}
((brings the spoon to Lena's [plate))
\end{verbatim}
\end{transbox}\vspace{-1mm}
%
\begin{transbox}{9}{~}
\begin{verbatim}
                             [lo:zhichku    fsio ravno nada
                              spoon.DIM.ACC anyway     need.MOD
\end{verbatim}
\hspace{4.5cm} a little spoon is still necessary
\end{transbox}\bigskip

\begin{figure}
\includegraphics[height=.28\textheight]{figures/baranova-img019.jpg}
\caption{Pavel serves his daughter Lena some porridge (\extref{ex:baranova:16}).}
\label{fig:baranova:19}
\end{figure}

\hspace*{-.5mm}Rice porridge is traditionally served at Russian memorial dinners. When Pavel’s daughter expresses her reluctance to eat it, he tries to persuade her to do so. He uses an impersonal declarative construction: \textit{nada} ‘one needs’ (line 3). This construction is similar to the Polish \textit{trzeba} \citep{ZinkenOgiermann2011}. Pavel combines the impersonal declarative with the address term \textit{Lena}. The use of this address term is pragmatically marked because it is already clear who the addressee is: Pavel is responding to Lena’s question. At the same time, the impersonal declarative expresses a general requirement to eat the porridge. Pavel even uses the adverb \textit{nada abizatil'na} ‘necessarily’, which strengthens the statement. He also takes the liberty of serving his daughter some porridge without securing her acceptance (lines 7--8). This is another piece of evidence that Pavel considers eating rice porridge to be an obligation in this context, regardless of a person’s own wishes. Pavel does, however, orient to the girl's reluctance to eat the porridge by using the diminutive \textit{lozhichku} ‘a little spoon’ (lines 6 and 9). This makes it clear that, although not eating the porridge is out of the question, it would be sufficient to eat only a little bit.\\

\noindent
To summarize this whole section (\sectref{sec:baranova:3.3}), we have seen that Russian recruiting turns come in four main linguistic formats. Imperatives are the most widely used format, followed by no-predicate constructions, declaratives, and interrogatives. Russian imperatives come in aspectual pairs: perfective and imperfective. They also have singular and plural forms. Declarative recruiting turns can be compared to recruitments of the trouble-assist type: declaratives often verbalize a trouble; trouble-assist recruitments do not involve language but feature trouble that is visible in the context.

\subsection{Additional verbal elements}\label{sec:baranova:3.4}

The core elements of a linguistically-formulated recruiting move (i.e. a predicate with its core arguments) can be complemented by additional verbal elements, among which are vocatives (see Extracts \ref{ex:baranova:3}, \ref{ex:baranova:8}, \ref{ex:baranova:9}, \ref{ex:baranova:10}, \ref{ex:baranova:15}, \ref{ex:baranova:16} for examples), benefactive markers (e.g. ‘you may do \{some\} serving for me please’, Extract \ref{ex:baranova:9}), reasons (see \extref{ex:baranova:3} above and \sectref{sec:baranova:3.4.2} below), and mitigators. In the next subsections, I focus on additional verbal elements that mitigate the recruiting move and elements that explain it.

\subsubsection{Mitigators}\label{sec:baranova:3.4.1}

Recruitments always involve some degree of imposition on recipients. Because of the potential threat to “face” and to the social relationships at hand, recruiters sometimes use strategies to mitigate the potential imposition of a recruitment (\citealt{BrownLevinson1987}). Here I illustrate several strategies.

The following extract features two types of mitigation: one is marked on the noun and the other on the imperative form of the verb. Vladimir and his wife Julia have their family over for food and drinks. At line 2 Vladimir produces a recruitment directed at Julia: \textit{daj-}\textbf{\textit{ka}} \textit{riumki nam} ‘give us \{some\} glasses’.

\transheader{ex:baranova:17}{cooking\_3\_226998}\vspace{2mm}
%
\emptytransbox{1}{(1.9)}
%
\begin{mdframednoverticalspace}[style=firstfoc]
\begin{transbox}{2}{vla}%dimir
\begin{verbatim}
daj -ka  [riumki      nam
give-PTC  glasses.ACC we.DAT
\end{verbatim}
give us \{some\} glasses
\end{transbox}
\end{mdframednoverticalspace}
%
\begin{transbox}{3}{jul}%ia
\begin{verbatim}
         [((looks at Vladimir))
\end{verbatim}
\end{transbox}
%
\emptytransbox{4}{(0.3)}
%
\begin{mdframednoverticalspace}[style=firstfoc]
\begin{transbox}{5}{vla}
\begin{verbatim}
[s    pamidorchikam
 with tomato
\end{verbatim}
\hspace{0.07cm} with a little tomato
\end{transbox}
\end{mdframednoverticalspace}
%
\begin{mdframednoverticalspace}[style=secondfoc]
\begin{transbox}{6}{jul}%ia
\begin{verbatim}
[((opens the kitchen cabinet and gets several glasses))
\end{verbatim}
\end{transbox}
\end{mdframednoverticalspace}

\begin{figure}
\includegraphics[height=.28\textheight]{figures/baranova-img020.png}
\caption{Vladimir recruits Julia (\extref{ex:baranova:17}).}
\label{fig:baranova:20}
\end{figure}

The imperative \textit{daj} ‘give’ is accompanied by the diminutive particle \textit{-ka} which makes the action sound more casual and low-cost. Possibly in response to the absence of an immediate response by Julia in line 4 -- the next transition-relevance place \citep{clayman_turn-constructional_2013} -- Vladimir increments his recruiting turn by adding \textit{s pamidorchikam} ‘with a little tomato’. The diminutive \textit{-chik-}, added to the basic form \textit{pamidoram}, formally attenuates Julia’s effort in serving the vegetable (\citealt{Ogiermann2009}).

We saw this strategy in \REF{ex:baranova:16} as well, where Pavel is persuading his daughter to eat rice porridge: \textit{lo:zh}\textbf{\textit{ichk}}\textit{u fsio ravno nada} ‘a little spoon is still necessary'. The diminutive \textit{-ich-}, added to the basic form \textit{lozhku}, minimizes the effort that his daughter would have to make in order to comply. % We saw this strategy in \ref{bkm:Ref329010309}

Recruiters can also use a diminutive address term as an expression of their affection for the recipient. This may serve as a way of downplaying the recruiter’s desire to impose. We saw this in \REF{ex:baranova:3}, where Marina is visiting her mother-in-law Anna. In Anna's recruiting turn \textit{nu Marish, pusti: ejo, ja pa- pae:m spako:jna} ‘sweet Marina, let her go \{so that\} I finish eating in peace’, the name Marina is rendered in the diminutive form \textit{Mari}\textbf{\textit{sh}}\textit{a}. In addition to the affectionate vocative, Anna provides a reason for the recruitment. This is another mitigating device, discussed in the next section. % We saw this in \extref{bkm:Ref327181049}

\subsubsection{Reasons}\label{sec:baranova:3.4.2}

Complementing a recruiting turn component with a clause that offers a reason for the recruitment goes beyond mere mitigation (e.g. \citealt{Waring2007,Parry2013}). The current sample includes 21 recruitment sequences in which the recruiter gives a reason in support of the recruitment. Reasons for informal recruitments in the Russian sample deal with i) informationally underspecified recruitments, ii) delicate recruitments, and iii) recruitments involving actions beyond requesting, for instance joking and complaining (\citealt{BaranovaDingemanse2016}). I now give examples of reasons supplied in the contexts of an underspecified recruitment and a delicate recruitment.

\extref{ex:baranova:18} shows a recruitment sequence where the reason provided by the recruiter adds information that is crucial for compliance. Several family members are having dinner together on the porch of a country house. One of them, Julija, has gone outside to take some photos. She is a guest visiting from abroad. Julija’s uncle, Pavel, was sleeping when Julija left the table. So, at the beginning of this extract, Pavel is unlikely to be aware of Julija’s whereabouts.

\transheader{ex:baranova:18}{Family\_dinner\_Country\_A2\_876874}\vspace{2mm}
%
\begin{transbox}{1}{~}
\begin{verbatim}
((Pavel joins the others at the table after being outside))
\end{verbatim}
\end{transbox}
%
\begin{transbox}{2}{pav}
\begin{verbatim}
dozhdik  zamarasil       [u    vas
rain.DIM drizzle.PST.PFV  with you.PL
\end{verbatim}
it has started drizzling in your area
\end{transbox}
%
\begin{mdframednoverticalspace}[style=firstfoc]
\begin{transbox}{3}{lid}%a
\begin{verbatim}
                         [pasmatri,
                          look.IMP.PFV.2SG
\end{verbatim}
\hspace{3.8cm} take a look
\end{transbox}
\end{mdframednoverticalspace}
%
\begin{transbox}{4}{~}
\begin{verbatim}
vyjdi      iz-za: ako:li-
go_out.IMP from   fen-
\end{verbatim}
go out behind the fen-
\end{transbox}
%
\begin{transbox}{5}{~}
\begin{verbatim} 
eh eta samae Julia pashla     (pa-moemu)        snimat’, 
   PTC PTC   name  go.PST.3SG (according to me) record.INF
\end{verbatim}
uhm Julija went to take pictures I think
\end{transbox}
%
\emptytransbox{6}{(0.3)}
%
\begin{mdframednoverticalspace}[style=secondfoc]
\begin{transbox}{7}{pav}%el
\begin{verbatim}
shias (pajdu)
now    go.FUT.1SG
\end{verbatim}
in a bit (I’ll go)
\end{transbox}
\end{mdframednoverticalspace}
%
\emptytransbox{8}{(0.8)}
%
\begin{transbox}{9}{lid}%a
\begin{verbatim}
[pajdiosh?
 go.FUT.2SG
\end{verbatim}
\hspace{0.07cm} are you going?
\end{transbox}
%
\begin{transbox}{10}{pav}%el
\begin{verbatim}
[((goes into the house))
\end{verbatim}
\end{transbox}
%
\emptytransbox{11}{(19.8)}
%
\begin{transbox}{12}{pav}%el
\begin{verbatim}
((returns to the porch with his jacket on))
\end{verbatim}
\end{transbox}\vspace{-1mm}
%
\begin{transbox}{13}{~}
\begin{verbatim}
((goes outside to find Julija))
\end{verbatim}
\end{transbox}\bigskip

\begin{figure}
\includegraphics[height=.28\textheight]{figures/baranova-img021.jpg}
\caption{Pavel has just put his jacket on to go outside (\extref{ex:baranova:18}).}
\label{fig:baranova:21}
\end{figure}

\largerpage
As Pavel returns (line 1) and while he is commenting on the weather (line 2), Lida addresses a recruiting turn to him: ‘take a look go out behind the fen-’, instructing Pavel to go outside, which is in conflict with his observation that it is raining. Up until this point, Lida’s recruiting turn is also lacking information on what Pavel should do once he is outside. However, Lida goes on to provide a reason that deals with these issues: ‘uhm Julija went to take pictures I think’. This reason refers to Julija who is out in the rain. She is a family member who is visiting from abroad. The family does not see her often in person, which makes her an honored guest. Pavel agrees to comply by saying \textit{shias (pajdu)} ‘in a bit (I’ll go)’ (line 7). After getting his jacket, he leaves the porch to find Julija outside.

The previously discussed \extref{ex:baranova:3} also contains a reason: \textit{‘sweet Marina, let her go \{so that\} I finish eating in peace’}, which deals with the delicacy of the requested action. The reason implies that finishing dinner in peace is incompatible with the presence of the dog at the table. Marina orients to this negative implication with a counter-reason ‘but we aren’t doing anything to you’.\\

\noindent
This section (\sectref{sec:baranova:3.4}) has discussed additional verbal elements used to mitigate and explain recruitments in the Russian sample. Russian vocatives, imperatives, and nouns can be formatted with diminutives and minimizers. Recruiters can also increase the chances of compliance by providing a reason. Reasons supply information necessary for fulfillment, or explain a recruitment that is otherwise unclear, delicate, or imposing.

\section{Formats in Move B: The responding move}
\largerpage
After having discussed recruiting moves, I now analyze the types of response they receive. \tabref{tab:baranova:4} shows the distribution of the main types of response relative to the format of the recruiting turn.\footnote{Nonverbal recruiting moves and recruitments of the trouble-assist type are not included in this table. Response categorized as “other” do not fit any of the main types or are unclear.} 

\begin{table}
\begin{tabularx}{\textwidth}{Xrrrr}
\lsptoprule
& \multicolumn{4}{c}{\textsc{format of recruiting turn}} \\
\textsc{type of response} & Declarative & Imperative & No predicate & Interrogative\\
\midrule
Fulfillment & 11 & 56 & 17 & 6\\
Ignoring & 4 & 20 & 5 & 4\\
Rejection & 0 & 14 & 1 & 3\\
Repair initiation & 0 & 5 & 0 & 0\\
Other & 3 & 5 & 2 & 3\\
\lspbottomrule
\end{tabularx}
\caption{Types of response by format of the recruiting turn (\textit{n}=159).}
\label{tab:baranova:4}
\end{table}

Most recruitments in the Russian sample are fulfilled. Since fulfillment involves some practical action, most responses are nonverbal: 145 recruiting moves received a fully nonverbal response.\footnote{In 39 cases in which the response was analyzed as fully nonverbal, the recruitee does say something while fulfilling the recruitment nonverbally but what they say is unrelated to the recruitment sequence.} In three cases, the recipient’s response was not visible or hearable. In the remaining 52 recruitments, the response involved a relevant verbal element. Such verbal responses can co-occur with fulfillment; at other times, they are indicative of rejection or delay in compliance. In what follows, I illustrate several response types in recruitment sequences.

\subsection{Compliance}

Compliance with a recruitment is usually evident from recipients’ nonverbal behavior as they proceed to give an object to the recruiter, perform a service, or cease/alter their ongoing behavior (see Extracts \ref{ex:baranova:1}, \ref{ex:baranova:4}, \ref{ex:baranova:6}, \ref{ex:baranova:8}, and \ref{ex:baranova:9}, and \ref{ex:baranova:13}). Occasionally, verbal elements complement such nonverbal responses.

Sometimes the verbal element expresses the recipient's commitment to fulfill the recruitment prior to the actual fulfillment. Consider again \REF{ex:baranova:13}, where Ksenia and Sasha were in the kitchen and Lena asked Sasha to let a guest out of her place. Sasha first responded to the recruiting move with the confirming expression \textit{uhum} and then displayed behavior consistent with this: she left the kitchen and went to the corridor to let the guest out. In this case, leaving the kitchen may not in itself be a clear indication of compliance since her behavior is not visible once she leaves the kitchen. So Sasha’s confirmation helps to convey to the recruiter Ksenia that she will indeed comply.

We saw a similar response to Lida's request ‘take a look go out behind the fen-’ in \REF{ex:baranova:18}. Here the recruitee Pavel displays his commitment to fulfilling the recruitment with \textit{shias (pajdu)}, literally ‘now (I'll go)’ (line 7). The Russian word \textit{shias} is used to indicate unstable and still changeable time in the present and, in combination with a future/imperfective inflection of the verb, it can also convey the meaning of ‘in a bit’. Basically, Pavel indicates that he will comply in the very near future. Instead of immediately going outside, he goes into the house to get his jacket. The verbal element \textit{sejchas (pajdu)} was necessary to make it clear that he would carry out the recruitment when his actual behavior could have been interpreted as non-compliance.

\subsection{Non-compliance}

In the Russian sample, rejecting a recruitment was never done with a simple ‘no’ response. Rejection was usually achieved by counter-proposals and the giving of reasons why compliance would not take place.

Consider again \REF{ex:baranova:3}, where Anna asks Marina to take the dog away from the table so that Anna can ‘finish eating in peace’. Marina does not comply but instead gives a counter-reason, i.e. a reason not to comply: ‘but we're not doing anything to you’. Marina keeps holding the dog on her lap for another five minutes. 

Another, even more indirect non-compliance strategy is to ignore the recruiting move, to give no response to it either way (see also Blythe, \chapref{sec:blythe}, \sectref{sec:blythe:4.2.4}). This is illustrated in the next example, where several school custodians gathered for lunch in their staff room. Vera makes a request of Anna, while Lena and Marina are involved in an unrelated conversation.

% Hi Nick. Please stop editing for a sec. OK why
% I have some local changes here to sync, and your editing prevents them from getting accepted. Should be 1 min.
% done

\transheader{ex:baranova:19}{20120120\_colleagues\_casual\_2\_339070}\vspace{2mm}
%
\begin{transbox}{1}{~}
\begin{verbatim}
((Anna is standing in front of the open cabinet))
\end{verbatim}
\end{transbox}
%
\begin{mdframednoverticalspace}[style=firstfoc]
\begin{transbox}{2}{ver}
\begin{verbatim}
[grenki      tam   eshio [dastan'
 breadsticks there also   take_out.IMP.PFV.SG
\end{verbatim}
\hspace{0.07cm} also get the bread sticks out
\end{transbox}
\end{mdframednoverticalspace}
%
\begin{transbox}{3}{len}%a
\begin{verbatim}
                         [a?
                          INTJ
\end{verbatim}
\hspace{3.8cm} huh?
\end{transbox}\vspace{1mm}
%
\begin{transbox}{4}{mar}%ina
\begin{verbatim}
ni  uexala              eshio?
NEG go_away.PST.PFV.3SG yet
\end{verbatim}
hasn’t she left yet?
\end{transbox}
%
\begin{transbox}{5}{len}%a
\begin{verbatim}
kto?
who
\end{verbatim}
who?
\end{transbox}
%
\emptytransbox{6}{(0.7)}
%
\begin{transbox}{7}{ann}%a
\begin{verbatim}
((opens up the cabinet))
\end{verbatim}
\end{transbox}
%
\begin{transbox}{8}{mar}%ina
\begin{verbatim}
((finger [points towards the wall behind her))
\end{verbatim}
\end{transbox}
%
\begin{mdframednoverticalspace}[style=firstfoc]
\begin{transbox}{9}{ver}%a
\begin{verbatim}
         [Sasha padi     tozhe pajest=
          name  probably also  eat.FUT.3SG
\end{verbatim}
\hspace{1.4cm} Sasha will probably also have \{some\}
\end{transbox}
\end{mdframednoverticalspace}
%
\begin{transbox}{10}{len}%a
\begin{verbatim}
=dir↑ektar
 director
\end{verbatim}
\hspace{0.07cm} the director?
\end{transbox}
%
\emptytransbox{11}{(0.4)}
%
\begin{transbox}{12}{len}
\begin{verbatim}
[.hhhh ana- ushla       v-    ushli        ani  s    Lugaevaj
       she  went_away.F in/to went_away.PL they with surname
\end{verbatim}
\hspace{0.07cm} .hhh she left to-, she left with Lugaeva
\end{transbox}
%
\begin{transbox}{13}{ann}%a
\begin{verbatim}
[((looks into the cabinet and takes out two bowls))
\end{verbatim}
\end{transbox}\bigskip

\begin{figure}
\includegraphics[height=.28\textheight]{figures/baranova-img022.jpg}
\caption{Vera makes a request for Anna to get breadsticks out of the cabinet (\extref{ex:baranova:19}).}
\label{fig:baranova:22}
\end{figure}

At line 2 Vera makes a request of Anna: ‘also get the bread sticks out’. Anna is standing in front of the cabinet, where the breadsticks are, ready to take something out of it. So she is the most suitable person to get the breadsticks. However, she does not respond to Vera's request and seems to be looking in Marina's direction instead. Vera pursues her request at line 9 by offering a reason for it: ‘Sasha will probably also have \{some\}’. After this, Anna still shows no signs of compliance, but Vera does not pursue the request further. It may be that Anna has simply failed to notice or register the ongoing recruitment. But in this case, the recruiter does not treat the absence of a response as a problem of hearing or registering the recruitment, as she would do by repeating the request, for instance. Instead, Vera offers a reason. There is still no response from Anna and the recruitment remains unfulfilled. Instead of getting the breadsticks, Anna takes two bowls out of the cabinet (line 13).

A recipient may initiate repair in response to a recruiting move, as in \REF{ex:baranova:2}, where Katya asked Maria to give her another spoon. Maria’s initial response was a repair initiation: ‘a spoon? (.) another one?’. It seems that the repair initiation is used here not only to indicate a problem of hearing, but also as a challenge to Katya’s request, questioning whether the recruitment is necessary. Katya’s response -- explaining that the spoon she has is dirty -- targets both potential readings of Maria’s repair initiation.

In sum, most recruitments in the Russian sample were fulfilled. Often, fulfillment involved nothing more than performing the relevant practical action. Sometimes the responding move featured a verbal component, for example conveying the recipient’s commitment to comply, before fulfillment actually occurred. In non-compliance, verbal responses included giving reasons not to comply and initiating repair.

\section{Acknowledgment in third position}

A recruitment sequence is a paired sequence consisting of a recruiting move and a responding move. This sequence can potentially be extended with a “sequence-closing third” (\citealt{Schegloff2007}) in the form of an acknowledgment by the recruiter of the assistance provided, such as ‘thank you’. In only three out of 176 fulfilled recruitments (2\%) did the recruiter produce some kind of acknowledgment of the recruitee’s efforts. On one occasion, the acknowledgment was the confirmative \textit{uhum} and the other two involved the conventional expression of gratitude \textit{spasibo} `thanks' (see \extref{ex:baranova:2}). In informal interaction, it seems that expressing gratitude for another's assistance or collaboration is a universally rare practice, with a minority of languages (including English and Italian) showing slightly greater frequency of occurrence \citep{FloydEtAl2018}.

\section{Social asymmetries}\label{sec:baranova:6}

The phenomenon of recruitment is sensitive to social asymmetries between recruiters and recruitees (\citealt{BrownLevinson1987}). Such asymmetries may be referenced through the format of the recruiting move. So recruiting assistance or collaboration in the workplace may be done differently than in an informal setting (\citealt{Garvey1975,Corsaro1977,Dixon2015,TakadaEndo2015}). Also, recruitments involving children are known to have different features from the ones that only involve adults (\citealt{DrewHeritage1992}). Child-directed recruitments were not included in this study.

Relative social status is difficult to operationalize. In this study, I took indicators of relative social status to be participants’ ages, the kind of relationship between them, and the way they address each other in the recording. In Russian culture, older people tend to be accorded a higher status. This is often expressed in the way they are addressed. Normally, a combination of their first name and their patronymic is used. Also, the use of the plural ‘you’ \textit{vy} is preferred over the singular \textit{ty}. However, if the older person is a close family relation, he or she can be addressed with the singular \textit{ty} and the corresponding kin term, such as ‘grandmother’, sometimes in combination with the person’s first name, e.g. ‘aunt Olga’. When there was no apparent age difference between the recruiter and recruitee, I looked at the relationship between them. Friends, spouses, siblings, and in-laws were considered to be of equal status when the participants were of an approximately same age. Such relationships are characterized by the use of the singular ‘you’ \textit{ty} and only first names when referring to or addressing each other.

Based on these criteria, the relationship in each recruiter-recruitee dyad was categorized according to the three types defined in the project's coding scheme (see \chapref{sec:coding}, \sectref{sec:coding:6}), namely, i) participants of equal status (A=B); ii) recruiter has higher status than recruitee (A>B); and iii) recruiter has lower status (A<B). As \tabref{tab:baranova:5} shows, the majority of recruitment sequences in the Russian sample (61\%) involved participants that were in a socially symmetrical relationship. In 35\% of cases the relationship between the participants was considered to be asymmetrical based on the above criteria.

\begin{table}
\begin{tabularx}{.5\textwidth}{Xrr}
\lsptoprule
(A)symmetry & Count & Proportion\\
\midrule
A=B & 122 & 61\%\\
A>B & 49 & 25\%\\
A<B & 21 & 11\%\\
unclear & 8 & 4\%\\
\lspbottomrule
\end{tabularx}
\caption{Relative frequencies of dyads by type of social (a)symmetry (\textit{n}=200).}
\label{tab:baranova:5}
\end{table}

Social status did not have a straightforward effect on the type of response that recipients produced. Fulfillment, rejection, and non-response rates were relatively equally distributed across the (a)symmetry types. The main finding here is that, if there are asymmetries, recruitment is more likely to be initiated in a downward direction. Recruitments initiated from a lower-status position were relatively rare. This is in line with Brown and Levinson’s (\citeyear{BrownLevinson1987}) theory of politeness: if the imposition is in an upward direction (A<B), then there is a greater “threat to face” than in other kinds of case (A>B, A=B). When the threat to face is higher, potential recruiters would be more likely not to carry out the face-threatening act at all and instead to perform the desired action themselves, or to mobilize someone of a lower or equal status for the task (see also Floyd, \chapref{sec:floyd}, \sectref{sec:floyd:6}; Enfield, \chapref{sec:enfield}, \sectref{sec:enfield:6}; Dingemanse, \chapref{sec:dingemanse}, \sectref{sec:dingemanse:5.2}).

\section{Conclusion}

This chapter has presented an overview of recruitment practices in conversational Russian. Imperatives are the most frequent linguistic format of recruiting moves. This is in line with Bolden’s (\citeyear{bolden2017}) conclusion that Russian imperatives are used in a wider range of recruitment contexts compared to languages such as English and Italian. Moreover, Russian imperatives form a diverse set. First, they feature an aspectual distinction between perfective and imperfective. Second, imperatively formatted recruiting turns can be produced with interrogative features and diminutive particles.

Similar to Italian and Polish, declarative recruiting turns in the Russian sample may be designed with an impersonal predicate \textit{nada}, which can be translated into English as ‘one needs to’ or ‘it is necessary to’. In this manner, the speaker can frame the recruitment in terms of shared responsibilities that hold for the recipient, but also for the speaker and perhaps the entire community (see also \citealt{ZinkenOgiermann2011}; \citealt[chap. 6]{Zinken2016}).

Russian has a rich diminutive morphology. Diminutive vocatives were observed in recruiting turns, expressing speakers’ affection for recipients, and may orient to the relationship between recruiter and recruitee. Diminutive nouns and particles can also be used to minimize the perceived imposition on recipients (see also \citealt{bolden2017}).

In responding moves, overt rejections are dispreferred and nonverbal fulfillment is the most frequent response. Rejection is usually done by means of reasons and counter-proposals. Overt refusals to comply were not found in the sample. It is not clear from our sample whether social asymmetry affects the way a recruiting move is formatted. Future research on this question may offer further insights into how recruitees are selected and how recruiting turns are designed.

\sloppy\printbibliography[heading=subbibliography,notkeyword=this]
\end{document}
