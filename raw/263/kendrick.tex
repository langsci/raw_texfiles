\documentclass[output=paper,nonflat,modfont,draft]{langsci/langscibook}
\ChapterDOI{10.5281/zenodo.4018376}
\author{Kobin H. Kendrick\affiliation{Department of Language and Linguistic Science, University of York}}
\title{Recruitment in English: A quantitative study}
\abstract{This chapter describes the resources that speakers of English use when recruiting assistance from others in everyday social interaction. The chapter draws on data from video recordings of informal conversation in English, and reports language-specific findings generated within a large-scale comparative project involving eight languages from five continents (see other chapters of this volume). The resources for recruitment described in this chapter include linguistic structures from across the levels of grammatical organization, as well as gestural and other visible and contextual resources of relevance to the interpretation of action in interaction. The presentation of categories of recruitment, and elements of recruitment sequences, follows the coding scheme used in the comparative project (see \chapref{sec:coding} of the volume). This chapter extends our knowledge of the structure and usage of English with detailed attention to the properties of sequential structure in conversational interaction. }

\IfFileExists{../localcommands.tex}{
  \usepackage{langsci-optional}
\usepackage{langsci-gb4e}
\usepackage{langsci-lgr}

\usepackage{listings}
\lstset{basicstyle=\ttfamily,tabsize=2,breaklines=true}

%added by author
% \usepackage{tipa}
\usepackage{multirow}
\graphicspath{{figures/}}
\usepackage{langsci-branding}

  
\newcommand{\sent}{\enumsentence}
\newcommand{\sents}{\eenumsentence}
\let\citeasnoun\citet

\renewcommand{\lsCoverTitleFont}[1]{\sffamily\addfontfeatures{Scale=MatchUppercase}\fontsize{44pt}{16mm}\selectfont #1}
  
  %% hyphenation points for line breaks
%% Normally, automatic hyphenation in LaTeX is very good
%% If a word is mis-hyphenated, add it to this file
%%
%% add information to TeX file before \begin{document} with:
%% %% hyphenation points for line breaks
%% Normally, automatic hyphenation in LaTeX is very good
%% If a word is mis-hyphenated, add it to this file
%%
%% add information to TeX file before \begin{document} with:
%% %% hyphenation points for line breaks
%% Normally, automatic hyphenation in LaTeX is very good
%% If a word is mis-hyphenated, add it to this file
%%
%% add information to TeX file before \begin{document} with:
%% \include{localhyphenation}
\hyphenation{
affri-ca-te
affri-ca-tes
an-no-tated
com-ple-ments
com-po-si-tio-na-li-ty
non-com-po-si-tio-na-li-ty
Gon-zá-lez
out-side
Ri-chárd
se-man-tics
STREU-SLE
Tie-de-mann
}
\hyphenation{
affri-ca-te
affri-ca-tes
an-no-tated
com-ple-ments
com-po-si-tio-na-li-ty
non-com-po-si-tio-na-li-ty
Gon-zá-lez
out-side
Ri-chárd
se-man-tics
STREU-SLE
Tie-de-mann
}
\hyphenation{
affri-ca-te
affri-ca-tes
an-no-tated
com-ple-ments
com-po-si-tio-na-li-ty
non-com-po-si-tio-na-li-ty
Gon-zá-lez
out-side
Ri-chárd
se-man-tics
STREU-SLE
Tie-de-mann
}
  \addbibresource{../localbibliography.bib}
  \togglepaper[4]%%chapternumber
}{}

\begin{document}
\maketitle
\label{sec:kendrick}

\section{Introduction}
The recruitment of assistance is a basic social organizational problem for which participants in interaction have practiced solutions (\citealt{KendrickDrew2016}; \chapref{sec:intro} of this volume). In our daily lives we carry out countless mundane courses of action: we may reach out and pick up a pen from a table, connect a power supply to a computer, turn the page of a book, put a dirty dish in the sink. For the most part, we execute these courses of action individually, whether alone or in the company of family, friends, or colleagues. If a trouble emerges -- the pen is too far to reach, the power supply is nowhere to be found -- we resolve the trouble on our own as well \citep{Kendrick2017}. But in the presence of others, a trouble in the realization of a course of action is a public event, and its resolution may become interactional achievement, not an individual one. Someone may see us visibly searching the environment \citep{DrewKendrick2018} or hear our imprecations as signs of trouble and therefore offer their assistance \citep{KendrickDrew2016}. We need not, however, wait for those around us to take notice and volunteer to help. Using the resources of language and the body, we can agentively solicit solutions from others to practical problems that emerge in the course of our activities. We may use a gesture to point to a box of biscuits so that someone will hand it to us, ask someone to locate a bag that we cannot find, or direct someone to move over to make room for us on the couch. However someone comes to perform assisting actions such as these, whether occasioned by a trouble or solicited by a request, we will have in effect \textit{recruited} them to give or offer assistance. This chapter presents a quantitative study of some such recruitment phenomena in English, focusing primarily on requests, as observed in a corpus of video recordings of everyday social interaction made in the US and UK.

\subsection{Recruitment: initial specimens}
We will begin with a set of cases that outline, in broad strokes, the general domain of recruitment. The first is a case in which we observe an opportunity for recruitment, but in which no recruitment occurs. The extract comes from an interaction between a group of friends as they prepare a barbecue in a public park. Just prior to the extract, Alison, the woman in the white shirt in \figref{fig:kendrick:1}, has been playing with a dog on the grass behind the picnic table. We then see her walk towards the table, stop, direct her gaze toward it, furrow her brow -- a facial gesture obscured in \figref{fig:kendrick:1} to protect her identity -- and place her hand on her chin, in an elaborated form of a “thinking face” \citep{GoodwinGoodwin1986}. She holds this complex of gestures, virtually motionless, for approximately 1.4 seconds as the other participants talk about a local concert series and prepare the meal. She then turns her head slowly to the right, a movement which takes approximately 0.6 seconds.

\transheader{ex:kendrick:1}{BBQ 08:25}

\begin{transbox}{1}{kim}
\begin{verbatim}
I used to work concerts in the park, in fuckin’
\end{verbatim}
\end{transbox}

\begin{transbox}{2}{~}
\begin{verbatim}
(   ).
\end{verbatim}
\end{transbox}

\emptytransbox{3}{(0.2)}

\begin{transbox}{4}{kim}
\begin{verbatim}
beer garden. +[awesome.
\end{verbatim}
\end{transbox}

\begin{transbox}{5}{don}
\begin{verbatim}
             +[ah:
\end{verbatim}
\end{transbox}

\begin{transbox}{6}{\textit{ali}}
\begin{verbatim}
             +brows together, hand on chin-->       -----
\end{verbatim}
\end{transbox}

\begin{transbox}{7}{don}
\begin{verbatim}
it’s so fun though.=like I miss#ed everyone this      |
\end{verbatim}
\end{transbox}

\begin{transbox}{8}{\fig}
\begin{verbatim}
                               #fig.1a              (2.0)
\end{verbatim}
\end{transbox}

\begin{transbox}{9}{~}
\begin{verbatim}
sum+m[er.                                             |
\end{verbatim}
\end{transbox}

\begin{transbox}{10}{\textit{ali}}
\begin{verbatim}
   +moves toward table-->                           -----
\end{verbatim}
\end{transbox}

\begin{transbox}{11}{ali}
\begin{verbatim}
     [here it i[s.
\end{verbatim}
\end{transbox}

\begin{transbox}{12}{kim}
\begin{verbatim}
               [we should +work it next su#mm+er.<oh
\end{verbatim}
\end{transbox}

\begin{transbox}{13}{\textit{ali}}
\begin{verbatim}
                          +reaches out----#--+picks up-->
\end{verbatim}
\end{transbox}

\begin{transbox}{14}{\fig}
\begin{verbatim}
                                          #fig.1b
\end{verbatim}
\end{transbox}

\begin{transbox}{15}{kim}
\begin{verbatim}
wait never mind you’re going +(traveling).
\end{verbatim}
\end{transbox}

\begin{transbox}{16}{\textit{ali}}
\begin{verbatim}
                             +walks around table-->
\end{verbatim}
\end{transbox}

\emptytransbox{17}{(0.5)}

\begin{transbox}{18}{don}
\begin{verbatim}
I’m not gonna be here. it’ll be much better. (   ).
\end{verbatim}
\end{transbox}

\emptytransbox{19}{(0.4)}

\begin{transbox}{20}{kim}
\begin{verbatim}
yeah.=I +[heard- (0.2) +#I don’t know if this would ever
\end{verbatim}
\end{transbox}

\begin{transbox}{21}{ali}
\begin{verbatim}
        +[okay.
\end{verbatim}
\end{transbox}

\begin{transbox}{22}{~}
\begin{verbatim}
        +,,,,,,,,,,,,,,+#sets down
\end{verbatim}
\end{transbox}

\begin{transbox}{23}{\fig}
\begin{verbatim}
                        #fig.1c
\end{verbatim}
\end{transbox}\bigskip

\begin{figure}
\caption{Alison (white shirt) walks towards the table, stops, furrows her brow and places her hand on her chin, making a “thinking face”. She then reaches out, picks up a lighter, walks to the front of the table, and sets it down. }
\includegraphics[width=\textwidth]{figures/KendrickFigure1.jpg}
\label{fig:kendrick:1}
\end{figure}

Alison’s visible bodily actions can be seen as a display of puzzlement, though the crux of the puzzle is initially obscure. As she turns her head, we come to see her actions as a visual search of the environment \citep{DrewKendrick2018} -- she’s evidently looking for something. The visible bodily resources that she uses to conduct the search, her facial and manual gestures, her head movement, allow it to be recognizable as such -- she’s not only looking for something; she’s \textit{doing} looking for something \citep{Sacks1984}. After a search of approximately 2.0 seconds, she apparently spots the sought after object, a lighter, and announces the end of her search with \textit{here it is} (line 11). This announcement, like her gestures, orients to the accountability of her actions \citep{Garfinkel1967}, even though, as we can see in \figref{fig:kendrick:1}\textit{a}, only the eye of the camera is on her as she conducts her search.\footnote{The individual frames within the figures in the chapter are designated as \textit{a}, \textit{b}, \textit{c}, etc. from left to right.} She then reaches out and picks up the lighter (see \figref{fig:kendrick:1}\textit{b}), walks around to the front of the table, and sets it down (see \figref{fig:kendrick:1}\textit{c}) as she marks the completion of the course of action with \textit{okay} (line 21). The lighter is later used to light the coals in the grill.

In this case we can see an opportunity for recruitment emerge, and then pass. Alison begins a course of action, encounters a trouble that disrupts its progressive realization, and makes this publicly available through her visible bodily actions. As we will see, explicit displays of trouble such as this can recruit the assistance of co-participants. But the opportunity here is twofold: just as Alison’s visible bodily actions could have recruited a co-participant, so too could Alison have used resources of language to do so explicitly. Consider the following case, in which Kimmy searches for a paper bag and then asks her co-participants for its location. Paper torn from the bag had been used by Kimmy as kindling and will be used again to light the coals in the grill.

\transheader{ex:kendrick:2}{BBQ 14:27}

\emptytransbox{1}{
(3.7) ((Kimmy visibly searches as she walks
}

\emptytransbox{2}{
around the table, see fig.2a-b))
}

\begin{mdframedkendrick}[style=firstfoc]
\begin{transbox}{3}{kim}
\begin{verbatim}
where the fuck is my little #fire starting bag.
\end{verbatim}
\end{transbox}
\end{mdframedkendrick}\vspace{-5mm}

\begin{transbox}{4}{\fig}
\begin{verbatim}
                            #fig.2b
\end{verbatim}
\end{transbox}

\begin{transbox}{5}{ca?}
\begin{verbatim}
(°°    °°)                      -----
\end{verbatim}
\end{transbox}

\begin{transbox}{6}{ali}
\begin{verbatim}
(          )                    (4.3)
\end{verbatim}
\end{transbox}

\begin{transbox}{7}{ca?}
\begin{verbatim}
(°°     +  °°)                  -----
\end{verbatim}
\end{transbox}

\begin{transbox}{8}{\textit{ali}}
\begin{verbatim}
        +gazes down-->
\end{verbatim}
\end{transbox}

\begin{transbox}{9}{kim}
\begin{verbatim}
+fire starting# bag.=is that i*t?
\end{verbatim}
\end{transbox}

\begin{mdframedkendrick}[style=secondfoc]
\begin{transbox}{10}{\textit{ali}}
\begin{verbatim}
+gestures with arm-->
\end{verbatim}
\end{transbox}
\end{mdframedkendrick}\vspace{-5mm}

\begin{transbox}{11}{\fig}
  \begin{verbatim}
              #fig.2c
  \end{verbatim}
\end{transbox}

\begin{transbox}{12}{\textit{kim}}
\begin{verbatim}
                              *reaches out-->>
\end{verbatim}
\end{transbox}
\clearpage

\begin{figure}
  \caption{Kimmy (red pants) visibly searches around the table for 3.7 seconds (\textit{a}) before she asks for assistance. After 4.3 seconds in which Kimmy continues her search, Alison gazes down into a bag to her right and gestures toward it with her arm. Kimmy reaches into the bag and retrieves the sought after object.   }
  \includegraphics[width=\textwidth]{figures/KendrickFigure2.jpg}
  \label{fig:kendrick:2}
\end{figure}

After she visibly searches near the grill and around the table for approximately 3.7 seconds (see \figref{fig:kendrick:2}\textit{a,b}), she asks her co-participants to locate the bag for her.\footnote{The insertion of \textit{the fuck} into the construction of the turn formulates this not merely as an inquiry, but also as a complaint; someone has moved \textit{her} bag.} As Kimmy rounds the table, Alison, who has been involved in a quiet conversation with another participant, looks down at a paper bag to her right (line 8), gestures towards it with her hand (while holding a bunch of scallions, see \figref{fig:kendrick:2}\textit{c}). Kimmy then reaches out and takes the bag (see \figref{fig:kendrick:2}\textit{d}).

In contrast to \REF{ex:kendrick:1}, in which Alison encountered a trouble in the realization of a course of action and resolved the trouble on her own, in this case Kimmy encounters trouble, performs a remedial action that does not resolve it, and then recruits a co-participant to assist her, using linguistic resources to do so. The recruiting action, as we will call it, is an interrogative question about the location of an object. It explicitly and accountably asks the co-participants to locate the object and thereby to assist Kimmy in her search.

The diversity of linguistic and embodied practices that participants use to explicitly and accountably recruit one another to facilitate practical courses of action will be a major theme throughout this chapter (see \sectref{sec:kendrick:4}). But the boundaries of recruitment are not so narrowly defined. Subtle visible bodily actions, through which a trouble becomes publicly recognizable, can recruit others to assist even when these actions are not, in the first instance, accountable as requests for assistance or other forms of solicitation. The following case, which comes from an interaction among a group of students in a common area of a university building, demonstrates this. Here Mark, the man in the patterned shirt in \figref{fig:kendrick:3}, can be seen to encounter some difficulty as he looks across the table at a picture in a book held by Rachael. Rachael then holds the book up for him to see.

\transheader{ex:kendrick:3}{RCE22a 23:15}

\begin{transbox}{1}{rac}
\begin{verbatim}
god that looks rude. ((about a picture in a book))
\end{verbatim}
\end{transbox}

\emptytransbox{2}{
(1.3)‡\#(0.5)
}

\begin{transbox}{3}{\textit{con}}
\begin{verbatim}
     ‡leans over and gazes at book-->
\end{verbatim}
\end{transbox}

\begin{transbox}{4}{\fig}
\begin{verbatim}
      #fig.3a
\end{verbatim}
\end{transbox}

\begin{transbox}{5}{con}
\begin{verbatim}
oh wow. .h heh
\end{verbatim}
\end{transbox}

\emptytransbox{6}{
+(0.8)}

\begin{transbox}{7}{\textit{mar}}
\begin{verbatim}
+leans forward and gazes at book-->
\end{verbatim}
\end{transbox}

\begin{transbox}{8}{con}
\begin{verbatim}
that really do(h)es(hh)
\end{verbatim}
\end{transbox}

\emptytransbox{9}{
(0.4)+(0.6)*\#
}

\begin{mdframedkendrick}[style=firstfoc]
\begin{transbox}{10}{\textit{mar}}
\begin{verbatim}
  -->+tilts head to side-->
\end{verbatim}
\end{transbox}
\end{mdframedkendrick}\vspace{-5mm}

\begin{transbox}{11}{\textit{rac}}
\begin{verbatim}
           *gazes at Ben-->
\end{verbatim}
\end{transbox}

\begin{transbox}{12}{\fig}
\begin{verbatim}
            #fig.3b
\end{verbatim}
\end{transbox}

\emptytransbox{13}{
(0.4)*(0.8)
}

\begin{mdframedkendrick}[style=secondfoc]
\begin{transbox}{14}{\textit{rac}}
\begin{verbatim}
  -->*.....holds book up-->>
\end{verbatim}
\end{transbox}
\end{mdframedkendrick}\vspace{-5mm}

\begin{transbox}{15}{mar}
\begin{verbatim}
what exactly is happening+# [in this.+
\end{verbatim}
\end{transbox}

\begin{transbox}{16}{\textit{mar}}
\begin{verbatim}
                         +untilts head+
\end{verbatim}
\end{transbox}

\begin{transbox}{17}{\fig}
\begin{verbatim}
                          #fig.3c
\end{verbatim}
\end{transbox}

\begin{transbox}{18}{rac}
\begin{verbatim}
                            [↑I don’t know.
\end{verbatim}
\end{transbox}\bigskip

\begin{figure}
\caption{ Mark leans forward and looks at Rachael’s book (\textit{a}). Rachael gazes up at Mark after he tilts his head to the side (\textit{b}). She then holds the book up for him to see (\textit{c}).}
\includegraphics[width=\textwidth]{figures/KendrickFigure3.jpg}
\label{fig:kendrick:3}
\end{figure}

An assessment of the picture begins the sequence (line 1) and draws the attention of Connor, seated to Rachael’s right (\figref{fig:kendrick:3}\textit{a}). Connor reacts to the picture with surprise (line 5), drawing the attention of Mark, who then leans forward and gazes at the picture from across the table (line 7). Mark holds this position for approximately 0.5 seconds and then torques his head back and to the side (cf. \citealt{Schegloff1998}), such that the orientation of his head comes to approximate the orientation of the book (see \figref{fig:kendrick:3}\textit{b}). The torque of Mark’s head makes public a minor trouble, namely that from his perspective, seated on the other side of the table, the picture appears upside-down and would therefore be difficult to see. Mark’s head movement attracts Rachael’s gaze (line 11), at which point she would be able to see his head in an unstable position and his gaze directed to the picture. Shortly thereafter she lifts the book and holds it up for Mark to see (line 14, see \figref{fig:kendrick:3}\textit{c}) and thereby resolves the trouble.

In contrast to \REF{ex:kendrick:2}, in which Kimmy employed a linguistic practice to recruit a co-participant to assist her explicitly and accountably, in this case Mark encounters a trouble in the realization of a course of action, performs a remedial action to resolve the trouble on his own, and thereby recruits Rachael’s assistance. His visible bodily action exposes the trouble, making it public, and thereby provides an occasion for Rachael to assist him, voluntarily. The action in effect recruits Rachael, even though in the first instance it is recognizable and accountable as an action taken by Mark to resolve the trouble independently, without assistance.

\subsection{The anatomy of recruitment}
With these cases in mind, we can now characterize recruitment and the interactional environment in which it occurs in more general terms. In each case, a course of action performed by an individual is impeded or disrupted, for example, by the lack of a necessary object (Extracts \ref{ex:kendrick:1} and \ref{ex:kendrick:2}) or constraints on the interactional space (\extref{ex:kendrick:3}). A set of methods exists with which participants can resolve such troubles, either individually, via self-remediation (\extref{ex:kendrick:1}), or interactionally, via recruitment (Extracts \ref{ex:kendrick:2} and \ref{ex:kendrick:3}). The nature of these methods and their organization is a central concern of research on recruitment. The methods are organized along a continuum and include requests for assistance; reports of troubles, difficulties, or needs; trouble alerts; embodied displays of trouble; and the projection and anticipation of troubles before they occur \citep{KendrickDrew2016}.

A basic distinction can be made between methods that create a normative \textit{obligation} for assistance by Other (e.g. the request in \extref{ex:kendrick:2}) and those that create a systematic \textit{opportunity} for such assistance to be given or offered voluntarily (e.g. the embodied display of trouble in \extref{ex:kendrick:3}) (\citealt{KendrickDrew2014, KendrickDrew2016}). This distinction also concerns who generates the possible solution to the trouble. With a request for assistance, Self generates a solution for Other to implement (e.g. Self identifies the object that will resolve her own trouble in \extref{ex:kendrick:2}). In contrast, with forms of voluntary assistance, it is Other who generates the solution and either implements it directly or offers to do so (e.g. holding up the book in \extref{ex:kendrick:3}). Recruitment thus encompasses the initiation of assistance by Self and Other as alternatives methods for the resolution of troubles \citep{KendrickDrew2016}.

The methods for recruitment include not only those implemented through language (e.g. the verbal request in \extref{ex:kendrick:2}) but also those implemented through visible bodily action (e.g. the visible searches of the environment in Extracts \ref{ex:kendrick:1} and \ref{ex:kendrick:2}, and the torque of the head in \extref{ex:kendrick:3}). Visible bodily actions that display difficulty, discomfort, or exertion, for example, create systematic opportunities for Others to give or offer assistance and thus constitute methods of recruitment \citep{KendrickDrew2016}. Such visible bodily actions are commonly, though not exclusively, forms of self-remediation, that is, actions produced by Self to resolve troubles independently (e.g. the visible search in \extref{ex:kendrick:1}). Remedial actions by Self commonly precede other methods of recruitment (e.g. a visible search precedes the request in \extref{ex:kendrick:2}), which together with other evidence suggests that self-remediation is a preferred alternative in the organization of assistance in interaction \citep{Kendrick2017}.

\subsection{The present study}
This chapter reports on a quantitative study of some recruitment phenomena in English, as observed in a corpus of video recordings of everyday social interaction in the US and UK. As a contribution to a cross-linguistic comparison \citep{FloydEtAl2014b}, the study employs an operational definition of recruitment and examines cases along specific dimensions set out by a coding scheme (see \chapref{sec:coding}). The study therefore focuses primarily, though not exclusively, on requests as “moves” -- a term used in this volume, after \citet{Goffman1969}, for social actions -- that recruit others to assist. It does not consider the full continuum of methods for recruitment identified by \citet{KendrickDrew2016}. The quantitative analyses presented in this chapter are descriptive in nature, reporting the relative frequencies and proportions of various coding categories. Inferential statistics are reported in the cross-linguistic comparative studies (e.g. \citealt{FloydEtAl2018}).

The chapter is organized as follows. After a discussion of the corpus and collection (\sectref{sec:kendrick:2}) and the basic structure of recruitment sequences (\sectref{sec:kendrick:3}), the analysis considers the visible bodily actions and grammatical formats that participants use to construct recruiting moves (\sectref{sec:kendrick:4}) and then turns to the ways in which participants respond (\sectref{sec:kendrick:5}). The chapter concludes with a discussion of the operational definition of recruitment employed in the present study and the concept articulated by \citet{KendrickDrew2016}.

\section{The corpus and collection}
\label{sec:kendrick:2}
The data for the study came from a corpus of 21 video recordings of social interactions between speakers of English in the US and UK with a total duration of 11 hours and 53 minutes. The interactions involved various activities, such as preparing a barbecue in a public park, eating a meal with friends, and playing a board game, as well as ordinary conversation. Interactions between children and caregivers were not included in the study. The video recordings came from a number of sources: (i) a set of recordings made by Giovanni Rossi in 2011; (ii) the Language and Social Interaction Archive (2014) by Leah Wingard; and (iii) a recording of a game of Monopoly by Heidi Kevoe-Feldman. Informed consent was obtained from all participants. In addition, two video recordings widely used in conversation-analytic research, “Chicken Dinner” and “Virginia”, were also included. A total of 211 recruitment sequences were identified, using the criteria described in \chapref{sec:coding}. The majority of recruiting moves in the resulting dataset were produced by speakers of a North American variety of English, whether recorded in the US or UK (\textit{n}=149), with the remainder produced by speakers of British varieties (\textit{n}=59) or non-native speakers (\textit{n}=3). The transcripts employ conventions developed by \citet{jefferson_glossary_2004} for the transcription of talk and those developed by \citet{Mondada2014c} for the transcription of visible bodily actions. A description of the multimodal transcriptions conventions can be found at the end of this chapter.

\section{The structure of recruitment sequences}
\label{sec:kendrick:3}
\subsection{Minimal sequences}
A minimal recruitment sequence includes two actions, referred to in the comparative study as \textit{moves}, a recruiting move and a responding move. In the transcripts, ▶ and ▷ designate the recruiting and responding moves, respectively. In the following extract, for example, as Vivian and Shane sit together on a couch, Vivian tells Shane to move over.

\newpage
\transheader{ex:kendrick:4}{Chicken Dinner 00:05 (simplified)}

\begin{mdframedkendrick}[style=firstfoc]
\begin{transbox}{1}{viv}
\begin{verbatim}
move over *a li:ttle,* can you?
\end{verbatim}
\end{transbox}
\end{mdframedkendrick}\vspace{-5mm}

\begin{mdframedkendrick}[style=secondfoc]
\begin{transbox}{2}{\textit{sha}}
\begin{verbatim}
          *..........*moves over-->
\end{verbatim}
\end{transbox}
\end{mdframedkendrick}\vspace{-5mm}

\emptytransbox{3}{
(.)*
}

\begin{transbox}{4}{~}
\begin{verbatim}
-->*
\end{verbatim}
\end{transbox}

\begin{transbox}{5}{sha}
\begin{verbatim}
yep.
\end{verbatim}
\end{transbox}

\begin{transbox}{6}{viv}
\begin{verbatim}
thanks.
\end{verbatim}
\end{transbox}\bigskip

Even before Vivian’s turn has come to possible completion, Shane begins to comply with the request, as indicated by the preparation phase of his movement in the transcript. Upon the completion of his movement, Shane responds verbally with an answer to Vivian’s tag question, at which point Vivian closes the sequence with a non-obligatory third move, a display of gratitude. The majority of recruitment sequences in the dataset were minimal, including only a single recruiting move (65.9\%, \textit{n}=139).

Most recruitment sequences in the dataset are organized as adjacency pairs \citep{SchegloffSacks1973, Sacks1992, schegloff_sequence_2007}, in which the recruiting move creates a normative obligation for a response. This includes imperative requests like \textit{move over a little}, which make embodied compliance the conditionally relevant response \citep{Goodwin2006,kent_compliance_2012}, as well as interrogative requests like \textit{where the fuck is my little fire starting bag} in \REF{ex:kendrick:2}. But conditional relevance, understood as a normative obligation to produce a specifiable next action, was not a criterion for the identification of recruitment sequences. Indeed, visible bodily actions as subtle as the tilting of one’s head -- a move that does not accountably mandate a response -- can effectively recruit another’s assistance \citep[see][8]{KendrickDrew2016}.

\subsection{Non-minimal sequences}
In a minority of cases, the sequence included more than two recruiting moves. One recurrent basis for this was the absence of a response to an initial move, as in the following extract. Here, after no one responds to her request for a fork, Donna pursues a response \citep{Pomerantz1984response, BoldenMandelbaumWilkinson2012}.

\transheader{ex:kendrick:5}{BBQ 52:19}

\begin{transbox}{1}{ali}
\begin{verbatim}
>I’ll’ve so:me.<
\end{verbatim}
\end{transbox}

\emptytransbox{2}{(0.3)}

\begin{transbox}{3}{car}
\begin{verbatim}
heh
\end{verbatim}
\end{transbox}

\emptytransbox{4}{(0.4)}

\begin{transbox}{5}{jam}
\begin{verbatim}
it almost sounds like you’re speaking an Asian
\end{verbatim}
\end{transbox}

\emptytransbox{6}{language.}

\begin{transbox}{7}{ali}
\begin{verbatim}
hHA HAH HAH hah °hah°
\end{verbatim}
\end{transbox}

\begin{mdframedkendrick}[style=firstfoc]
\begin{transbox}{8}{don}
\begin{verbatim}
is [there a fork] over the[re  k i : d s ? ]
\end{verbatim}
\end{transbox}
\end{mdframedkendrick}\vspace{-5mm}

\begin{transbox}{9}{jam}
\begin{verbatim}
   [I’ll: so:me.]
\end{verbatim}
\end{transbox}

\begin{transbox}{10}{ali}
\begin{verbatim}
                          [>I’ll’ve so:me.<]
\end{verbatim}
\end{transbox}

\emptytransbox{11}{
(0.2)+(0.8)+
}

\begin{transbox}{12}{\textit{car}}
\begin{verbatim}
     +sets sausage on Ali’s plate+
\end{verbatim}
\end{transbox}

\begin{transbox}{13}{ali}
\begin{verbatim}
yay:::
\end{verbatim}
\end{transbox}

\emptytransbox{14}{(0.2)}

\begin{transbox}{15}{don}
\begin{verbatim}
no?
\end{verbatim}
\end{transbox}

\begin{transbox}{16}{jam}
\begin{verbatim}
what?
\end{verbatim}
\end{transbox}

\begin{mdframedkendrick}[style=firstfoc]
\begin{transbox}{17}{don}
\begin{verbatim}
fork.
\end{verbatim}
\end{transbox}
\end{mdframedkendrick}\vspace{-5mm}

\emptytransbox{18}{(0.5)}

\begin{transbox}{19}{jam}
\begin{verbatim}
oh.
\end{verbatim}
\end{transbox}

\emptytransbox{20}{
*(2.7) ((background talk omitted))
}

\begin{mdframedkendrick}[style=secondfoc]
\begin{transbox}{21}{\textit{jam}}
\begin{verbatim}
*picks up fork and hands it to Donna-->
\end{verbatim}
\end{transbox}
\end{mdframedkendrick}\vspace{-5mm}

\begin{transbox}{22}{don}
\begin{verbatim}
thank *you.
\end{verbatim}
\end{transbox}

\begin{transbox}{23}{\textit{jam}}
\begin{verbatim}
----->*
\end{verbatim}
\end{transbox}\bigskip

Although the request occurs at the possible completion of a sequence by the other participants (lines 5--7), the sequence is contingently expanded (lines 9--10), resulting in overlap that obscures the request (lines 8--10). After no one responds or attends to the request, Donna produces a candidate answer to her question, itself designed as a question, and thereby pursues a response (line 15). This attracts the attention of James, who turns to look at Donna and initiates repair (see \citealt{Kendrick2015} for a review), providing an opportunity for Donna to reissue her request to a now available recipient. In this context, after a pursuit, the subsequent request takes a minimal form, simply the name of the requested object (line 17, see \sectref{sec:kendrick:4.2.4}). This successfully initiates an object transfer and completes the sequence. Overall, 79.1 percent of recruiting moves (\textit{n}=167) were sequence initial, whereas 20.9 percent (\textit{n}=44) were subsequent attempts (e.g. pursuits or repair solutions).

\subsection{Recruitment types}
Participants recruit each other to manage a variety of practical contingencies. To provide a general sense of the distribution for the comparative study recruitments were classified into four types. The most frequent type was the provision of a service, that is, the performance of a practical action for the recruiter (49.8\%, \textit{n}=105). Transfers of objects were also especially frequent (38.8\%, \textit{n}=82). Less frequent were sequences in which one participant stopped or altered the trajectory of another’s actions (e.g. \textit{leave it alone}, 7.6\%, \textit{n}=16) and those in which a visible trouble elicited a direct provision of assistance (3.8\%, \textit{n}=8).

\section{Recruiting moves}
\label{sec:kendrick:4}
To recruit others to act on their behalf, participants in social interaction draw on an arsenal of resources, both linguistic and embodied. In this section we will review the most frequent forms of language and visible bodily action observed in the first move of recruitment sequences. We begin first with the body and examine the forms of visible bodily action that either constitute or accompany recruiting moves, and then turn our attention to language and consider the grammatical formats and linguistic components that participants use to recruit others through talk.

\subsection{Visible bodily actions}
Language is not necessary for recruitment. Even a subtle movement of the body, as one maneuvers to inspect a picture from across a table or searches the local environment, can elicit a helpful action from a co-participant \citep[see][]{KendrickDrew2016, DrewKendrick2018}. Such exclusively embodied recruiting moves are striking specimens, but they are rare (see also Extracts \ref{ex:kendrick:8} and \ref{ex:kendrick:9}). Only 7.6 percent of recruiting moves in the dataset were exclusively visual (\textit{n}=15). However, this number does not include visual recruiting moves that elicited offers of assistance which were not included in the operational definition of recruitment. More common were complex multimodal actions, in which the move to recruit had both linguistic and embodied components, such as asking for something and reaching for it concurrently (41.4\%, \textit{n}=82). But despite the abundance of visible bodily actions, a narrow majority of recruiting moves were exclusively linguistic, with no relevant visual components (51\%, \textit{n}=101).

When participants do use visible bodily actions, what do they look like? \tabref{tab:kendrick:1} presents the types, frequencies, and proportions of relevant visible bodily actions observed in the dataset.

\begin{table}
\begin{tabularx}{0.66\textwidth}{Xrr}
\lsptoprule
Visible bodily action & Frequency & Proportion\\
\midrule
Pointing & 30 & 30.9\%\\
Reaching out & 12 & 12.4\%\\
Holding out & 12 & 12.4\%\\
Visible trouble	& 12 & 12.4\%\\
Other gesture & 9 & 9.3\%\\
Instrumental & 8 & 8.2\%\\
Searching & 6 & 6.2\%\\
Other & 8 & 8.2\%\\
\lspbottomrule
\end{tabularx}

\caption{Frequencies and proportions of visible bodily actions in recruiting moves (\textit{n}=97).}
\label{tab:kendrick:1}
\end{table}

The set of body behaviors identified as relevant is diverse, including those whose function is accountably communicative (e.g. pointing at an object) as well as those whose function may, in the first instance, be instrumental (e.g. visibly searching for an object, on which \citealt [see] [] {DrewKendrick2017}). But within this diversity, one body-behavioral resource emerged as dominant: the hands. Over two thirds of all visible bodily actions in recruiting moves involved manual gestures or manual actions (64.9\%, \textit{n}=63). Within the dataset as a whole, a third of all recruiting moves included relevant manual movements, in one form or another.

With the exception of visible displays of trouble, an example of which was given in \REF{ex:kendrick:3}, and visible searches, which can be seen in \REF{ex:kendrick:2}, the remainder of this section illustrates the forms of visible bodily actions observed in the first move of recruitment sequences. As the analysis of these cases will show, the different forms of visible bodily action differ in how and to what extent they specify what the recipient should do in response.

\subsubsection{Pointing}
By far the most common form of visible bodily action used for recruitment was a pointing gesture. Points occurred not only in recruiting objects (\textit{n}=11), where they index the object in demand, but also in recruiting services (\textit{n}=15), where they designate a location for the action to be done, among other possibilities. In the following extract, a pointing gesture is used to recruit a co-participant to pass an object \citep[see also][22--23]{DrewCouper-Kuhlen2014a}. At the possible completion of a sequence, John raises his arm and points to a box of biscuits on the table (see \figref{fig:kendrick:4}).

\newpage
\transheader{ex:kendrick:6}{RCE14\_109822}

\begin{transbox}{1}{ann}
\begin{verbatim}
I'm gonna have to actually A write the paper and
\end{verbatim}
\end{transbox}

\begin{transbox}{2}{~}
\begin{verbatim}
then B get round to sorting it out.
\end{verbatim}
\end{transbox}

\begin{transbox}{3}{joh}
\begin{verbatim}
.hh ye(h)ah hh in the way these things do hhh[hh
\end{verbatim}
\end{transbox}

\begin{transbox}{4}{ann}
\begin{verbatim}
                                             [phh hh
\end{verbatim}
\end{transbox}

\begin{transbox}{5}{~}
\begin{verbatim}
+yeah.#
\end{verbatim}
\end{transbox}

\begin{mdframedkendrick}[style=firstfoc]
\begin{transbox}{6}{\textit{joh}}
\begin{verbatim}
+points-->
\end{verbatim}
\end{transbox}
\end{mdframedkendrick}\vspace{-5mm}

\begin{transbox}{7}{\fig}
\begin{verbatim}
      #fig.4
\end{verbatim}
\end{transbox}

\emptytransbox{8}{(0.3)}

\begin{transbox}{9}{ann}
\begin{verbatim}
which is a *(   )(0.1)+(0.2)*
\end{verbatim}
\end{transbox}

\begin{transbox}{10}{~}
\begin{verbatim}
           *sets down pen---*
\end{verbatim}
\end{transbox}

\begin{transbox}{11}{\textit{joh}}
\begin{verbatim}
                   -->+,,,-->
\end{verbatim}
\end{transbox}

\emptytransbox{12}{~}

\begin{transbox}{13}{joh}
\begin{verbatim}
*°°c[ookie°°+
\end{verbatim}
\end{transbox}

\begin{transbox}{14}{~}
\begin{verbatim}
         -->+
\end{verbatim}
\end{transbox}

\begin{mdframedkendrick}[style=secondfoc]
\begin{transbox}{15}{\textit{ann}}
\begin{verbatim}
*reaches for box-->
\end{verbatim}
\end{transbox}
\end{mdframedkendrick}\vspace{-5mm}

\begin{transbox}{16}{ann}
\begin{verbatim}
   [biscuit? *biscui[t biscuit biscuit
\end{verbatim}
\end{transbox}

\begin{transbox}{17}{~}
\begin{verbatim}
          -->*sets box in front of John-->
\end{verbatim}
\end{transbox}

\begin{transbox}{18}{joh}
\begin{verbatim}
                    [°biscuit biscuit°
\end{verbatim}
\end{transbox}

\begin{transbox}{19}{ann}
\begin{verbatim}
°yeah°*
\end{verbatim}
\end{transbox}

\begin{transbox}{20}{~}
\begin{verbatim}
   -->*
\end{verbatim}
\end{transbox}

\emptytransbox{21}{(0.3)}

\begin{transbox}{22}{ann}
\begin{verbatim}
shall I show you what I've-
\end{verbatim}
\end{transbox}

\begin{transbox}{23}{joh}
\begin{verbatim}
yea[h
\end{verbatim}
\end{transbox}

\begin{transbox}{24}{ann}
\begin{verbatim}
   [pictures I've picked up
\end{verbatim}
\end{transbox}\bigskip

\begin{figure}
\caption{John points to a box of biscuits as Anne looks down at the pen in her hands.}
\includegraphics[width=\textwidth]{figures/KendrickFigure4.png}
\label{fig:kendrick:4}
\end{figure}

The pointing gesture by John is recognizable as a move to recruit Anne to act on the pointed-at object. But unlike linguistic requests, which formulate an action for the recipient to perform (e.g. \textit{can you pass me the biscuits?}), a point does not specify a next action to be done. It instructs the recipient to redirect her attention to the object and invites her to search for its current relevance to the situation. In this case, the relevance of the biscuits is transparent. At the moment John’s gestures reaches its apex, Anne’s gaze is directed downward to a pen in her hands. John holds the gesture for approximately 700 ms until Anne quickly sets the pen down on the table, an action that displays her (late) recognition of the move to recruit her (lines 6--11). As he retracts his gesture and she reaches for the box, John softly names the object (line 13), a linguistic action that occurs after the recipient has begun to comply but before the recruitment has been fulfilled, a position in which linguistic recruiting moves serve to “expedite” the completion of the transaction \citep{KentKendrick2016}.

\subsubsection{Reaching out}
The shape and orientation of a manual action can not only index an object but also specify a relevant next action. Extending one’s hand towards an object, with an open, vertical orientation, is recognizable as a move to recruit the recipient of the gesture to transfer the object. The recognizability of this as a recruiting move comes from the specific hand shape, which visibly anticipates taking the object \citep[see][47 on prehensile postures]{Streeck2009}. In the following extract, as Mark produces a request for candy from a bag held by Rachael, he simultaneously reaches out towards the bag (see \figref{fig:kendrick:5}).

\transheader{ex:kendrick:7}{RCE22a\_690761}

\begin{mdframedkendrick}[style=firstfoc]
\begin{transbox}{1}{mar}
\begin{verbatim}
+ohh can I+ have# s*ome.
\end{verbatim}
\end{transbox}
\end{mdframedkendrick}\vspace{-5mm}

\begin{transbox}{2}{~}
\begin{verbatim}
+.........+reaches out-->
\end{verbatim}
\end{transbox}

\begin{transbox}{3}{\fig}
\begin{verbatim}
                #fig.5a
\end{verbatim}
\end{transbox}

\begin{mdframedkendrick}[style=secondfoc]
\begin{transbox}{4}{\textit{rac}}
\begin{verbatim}
                   *holds bag out-->
\end{verbatim}
\end{transbox}
\end{mdframedkendrick}\vspace{-5mm}

\emptytransbox{5}{
(0.8)+\#(1.3)
}

\begin{transbox}{6}{\textit{mar}}
\begin{verbatim}
  -->+puts fingers into bag-->
\end{verbatim}
\end{transbox}

\begin{transbox}{7}{\fig}
\begin{verbatim}
      #fig.5b
\end{verbatim}
\end{transbox}

\emptytransbox{8}{
(2.3)*
}

\begin{transbox}{9}{\textit{rac}}
\begin{verbatim}
  -->*moves bag closer to Mark-->
\end{verbatim}
\end{transbox}

\emptytransbox{10}{
(0.6)+(0.2)*
}

\begin{transbox}{11}{\textit{mar}}
\begin{verbatim}
  -->+lifts bag-->>
\end{verbatim}
\end{transbox}

\begin{transbox}{12}{\textit{rac}}
\begin{verbatim}
        -->*retracts-->>
\end{verbatim}
\end{transbox}

\begin{transbox}{13}{mar}
\begin{verbatim}
sorry. heh heh huh
\end{verbatim}
\end{transbox}\bigskip

\begin{figure}
\caption{Mark reaches out as he asks Rachel for some candy. Rachel holds out the bag as Mark puts his fingers into it.}
\includegraphics[width=\textwidth]{figures/KendrickFigure5.jpg}
\label{fig:kendrick:5}
\end{figure}

Even before Mark’s verbal turn is complete, Rachael begins to fulfill the request, holding the bag out towards Mark (lines 1--4). Although the shape of Mark’s hand anticipates receiving the bag, Rachael tilts the bag towards him, such that he can reach inside. This precipitates some difficulty as he inserts his fingers into the bag and fumbles as he tries to extract the candies (lines 6--11). Unlike pointing gestures, which occurred with recruitments of all types, reaching actions such as this occurred exclusively in recruiting objects.

\subsubsection{Holding out}\label{sec:kendrick:4.1.3}
As we have seen, transferring objects is a common contingency that participants recruit each other to manage. Just as an object can move from B to A, from the recruit to the recruiter, as it were, so too can it travel in the opposite direction, from A to B. Holding out an object towards a recipient initiates a transaction in which the recipient should take the object and act on it. Similar to a pointing gesture, which directs the visual attention of its recipient to an object but does not specify what he or she should do with it, holding out an object presents the recipient with a puzzle: what should be done? It is no surprise, then, that participants use this form of recruiting move in specific contexts and for specific objects that radically constrain the possibility space of relevant next actions. In the following extract, after Ellen finishes a bowl of cheesecake, she picks up the empty bowl and holds it out toward Daniel (see \figref{fig:kendrick:6}).

\transheader{ex:kendrick:8}{RCE26b\_560620}

\begin{transbox}{1}{dan}
\begin{verbatim}
well but you could (.) have it with something savory.
\end{verbatim}
\end{transbox}

\begin{transbox}{2}{~}
\begin{verbatim}
+like some beef or someth+ing.
\end{verbatim}
\end{transbox}

\begin{transbox}{3}{\textit{ell}}
\begin{verbatim}
+........................+picks up bowl-->
\end{verbatim}
\end{transbox}

\emptytransbox{4}{
(0.7)+(0.8)\#
}

\begin{mdframedkendrick}[style=firstfoc]
\begin{transbox}{5}{\textit{ell}}
\begin{verbatim}
  -->+holds out to Dan-->
\end{verbatim}
\end{transbox}
\end{mdframedkendrick}\vspace{-5mm}

\begin{transbox}{6}{\fig}
\begin{verbatim}
           #fig.6
\end{verbatim}
\end{transbox}

\emptytransbox{7}{
*(1.2)
}

\begin{mdframedkendrick}[style=secondfoc]
\begin{transbox}{8}{\textit{dan}}
\begin{verbatim}
*steps forward, reaches out, grasps bowl-->
\end{verbatim}
\end{transbox}
\end{mdframedkendrick}\vspace{-5mm}

\emptytransbox{9}{
(0.2)+*
}

\begin{transbox}{10}{\textit{ell}}
\begin{verbatim}
  -->+lets go of bowl, retracts-->
\end{verbatim}
\end{transbox}

\begin{transbox}{11}{\textit{dan}}
\begin{verbatim}
   -->*takes bowl, sets off camera-->
\end{verbatim}
\end{transbox}

\emptytransbox{12}{(0.2)}

\begin{transbox}{13}{dan}
\begin{verbatim}
I’m not sure meringue+ beef would be the best=
\end{verbatim}
\end{transbox}

\begin{transbox}{14}{\textit{ell}}
\begin{verbatim}
                  -->+
\end{verbatim}
\end{transbox}

\begin{transbox}{15}{dan}
\begin{verbatim}
=combination but*
\end{verbatim}
\end{transbox}

\begin{transbox}{16}{~}
\begin{verbatim}
             -->*-->>
\end{verbatim}
\end{transbox}\bigskip

\begin{figure}
\caption{Ellen holds her bowl out towards Daniel}
\includegraphics[width=\textwidth]{figures/KendrickFigure6.jpg}
\label{fig:kendrick:6}
\end{figure}

This action recognizably recruits Daniel to take the bowl and perform some action with it. As Ellen’s arm reaches maximum extension, Daniel steps forward from his position against the kitchen counter, reaches out to take the bowl, and then sets it in the sink off camera. But how does Daniel recognize that some action from him is due and select an appropriate response? The deictic aspect of Ellen’s arm extension “points” to Daniel and thereby addresses the action specifically to him.\footnote{Note that this can also be done with gaze direction, but here Ellen averts her gaze as she holds out the bowl.} Holding out an object towards a recipient, with one’s arm at maximum extension, not only addresses the action but also makes accountably relevant an embodied response in which the recipient takes the object. The visible form of the action does not, however, specify that the recipient should then put the object in the sink, as Daniel does. The solution to this puzzle, one which Daniel himself may have used, lies in the routine organization of the current activity \citep[see][]{Rossi2014} and the local ecology of the room. Clearing dishes is a routine (and hence anticipatable) course of action after one has finished a meal, and standing near the sink, Daniel is in a position to place the bowl in the sink on Ellen’s behalf. Embodied actions such as this are analytically distinct from requests (e.g. \textit{would you put this in the sink for me?}) on the grounds that, unlike such requests, they do not specify the action the recipient is to perform in next position.

\subsubsection{Other gestures}
The most frequent forms of visible bodily action in recruiting moves, as we have seen, involve transferring objects from one participant to another, hand to hand. But manual gestures also accompany and constitute recruiting moves for practical actions, not only objects. In the following extract, after Rachael begins to turn the page of a book, revealing a picture on the next page, Mark leans forward and produces a verbal display of disgust (lines 1--3). Rachel, presumably unaware that Mark had seen the picture on the next page, abandons turning the page, allowing Mark to view the current one.\footnote{This, too, is a case of recruitment, but not the focus of the present analysis.} At this moment Mark produces two quick finger movements that iconically depict turning the page (see \figref{fig:kendrick:7}).

\newpage
\transheader{ex:kendrick:9}{RCE22a 35:46}

\emptytransbox{1}{+(0.6)}

\begin{transbox}{2}{\textit{mar}}
\begin{verbatim}
+leans forward-->
\end{verbatim}
\end{transbox}

\begin{transbox}{3}{mar}
\begin{verbatim}
euww:::
\end{verbatim}
\end{transbox}

\emptytransbox{4}{
(0.6)+\#(0.8)+
}

\begin{mdframedkendrick}[style=firstfoc]
\begin{transbox}{5}{\textit{mar}}
\begin{verbatim}
  -->+flicks finger twice+
\end{verbatim}
\end{transbox}
\end{mdframedkendrick}\vspace{-5mm}

\begin{transbox}{6}{\fig}
\begin{verbatim}
      #fig.7a
\end{verbatim}
\end{transbox}

\emptytransbox{7}{
*(0.3)\#(0.5)*
}

\begin{mdframedkendrick}[style=secondfoc]
\begin{transbox}{8}{\textit{rac}}
\begin{verbatim}
*turns page *
\end{verbatim}
\end{transbox}
\end{mdframedkendrick}\vspace{-5mm}

\begin{transbox}{9}{\fig}
\begin{verbatim}
      #fig.7b
\end{verbatim}
\end{transbox}

\emptytransbox{8}{(0.6)}

\begin{transbox}{9}{mar}
\begin{verbatim}
okay never mind.
\end{verbatim}
\end{transbox}

\emptytransbox{10}{(0.3)}

\begin{transbox}{11}{mar}
\begin{verbatim}
I thought that was a mouth open.
\end{verbatim}
\end{transbox}\bigskip

\begin{figure}
\caption{Mark flicks his finger up and down and then Rachel turns the page of her book.}
\includegraphics[width=\textwidth]{figures/KendrickFigure7.jpg}
\label{fig:kendrick:7}
\end{figure}

The iconic gesture visually depicts the action that Rachel should perform and thereby specifies the relevant response. After Rachel fulfills the recruitment, Mark accounts for his interest in the picture and brings the exchange to a close.

\subsubsection{Instrumental actions}
The forms of conduct we have seen thus far are used by participants to initiate transactions in which some practical action from a co-participant is accountably due. But not all forms of visible bodily actions observed in recruiting moves do this. In some cases, a participant who produces a recruiting move through language also performs an embodied action that facilitates the recruitment. In the following extract, for example, as Fabian directs Kate to put her coat on, he reaches out and pulls it up, facilitating the fulfillment of the sequence (see \figref{fig:kendrick:8}).

\transheader{ex:kendrick:10}{RCE02 04:00}

\begin{transbox}{1}{fab}
\begin{verbatim}
uhm
\end{verbatim}
\end{transbox}

\emptytransbox{2}{(1.8)}

\begin{transbox}{3}{fab}
\begin{verbatim}
but yeah.# are you co:ld.
\end{verbatim}
\end{transbox}

\begin{transbox}{4}{\fig}
\begin{verbatim}
         #fig.8a
\end{verbatim}
\end{transbox}

\emptytransbox{5}{(0.5)}

\begin{transbox}{6}{kat}
\begin{verbatim}
mm hm.
\end{verbatim}
\end{transbox}

\emptytransbox{7}{
+(0.9)
}

\begin{transbox}{8}{\textit{fab}}
\begin{verbatim}
+.....-->
\end{verbatim}
\end{transbox}

\begin{mdframedkendrick}[style=firstfoc]
\begin{transbox}{9}{fab}
\begin{verbatim}
then cover yourself# up+ pro*perly.=
\end{verbatim}
\end{transbox}
\end{mdframedkendrick}\vspace{-5mm}

\begin{transbox}{10}{~}
\begin{verbatim}
                    -->+pulls coat-->
\end{verbatim}
\end{transbox}

\begin{transbox}{11}{\fig}
\begin{verbatim}
                   #fig.8b
\end{verbatim}
\end{transbox}

\begin{transbox}{12}{\textit{kat}}
\begin{verbatim}
                            *.....-->
\end{verbatim}
\end{transbox}

\begin{transbox}{13}{kat}
\begin{verbatim}
=well yeah but (0.5) oh
\end{verbatim}
\end{transbox}

\emptytransbox{14}{
(0.6)*(0.7)
}

\begin{mdframedkendrick}[style=secondfoc]
\begin{transbox}{15}{\textit{kat}}
\begin{verbatim}
  -->*lifts herself off ground-->
\end{verbatim}
\end{transbox}
\end{mdframedkendrick}\vspace{-5mm}

\begin{transbox}{16}{kat}
\begin{verbatim}
there's *dirt* all+ round the back of my
\end{verbatim}
\end{transbox}

\begin{transbox}{17}{~}
\begin{verbatim}
     -->*sits*
\end{verbatim}
\end{transbox}

\begin{transbox}{18}{\textit{fab}}
\begin{verbatim}
               -->+
\end{verbatim}
\end{transbox}

\emptytransbox{19}{(0.6)}

\begin{transbox}{20}{kat}
\begin{verbatim}
[thing.
\end{verbatim}
\end{transbox}

\begin{transbox}{21}{fab}
\begin{verbatim}
[yeah.
\end{verbatim}
\end{transbox}

\emptytransbox{22}{(0.6)}

\begin{transbox}{23}{fab}
\begin{verbatim}
and whose fault is that.
\end{verbatim}
\end{transbox}

\emptytransbox{24}{(0.7)}

\begin{transbox}{25}{kat}
\begin{verbatim}
.tsk
\end{verbatim}
\end{transbox}\bigskip

\begin{figure}
\caption{Fabian gazes at Kate before he asks whether she is cold. After she confirms that she is, he reaches out and begins to pull her coat up.}
\includegraphics[width=\textwidth]{figures/KendrickFigure8.jpg}
\label{fig:kendrick:8}
\end{figure}

The embodied action in this case illustrates a distinct mechanism for recruitment. In the majority of cases in the dataset, a speaker produces a recruiting move through language, using a grammatical format, such as an interrogative or imperative, that normally encodes an obligation to respond. The recruiting move is thus recognizable and accountable as a social action that combines turn-constructional and sequence-organizational practices into a mechanism for recruitment. Visible bodily actions such as points and iconic gestures are analogous in that they recognizably initiate transactions in which a responding move is due (even if they do not fully specify what form it should take). A distinct mechanism, the one illustrated by this example, is for a participant to begin a course of action that \textit{necessarily} involves co-participation. The recognition of the incipient course of action and the one’s participation in its completion is a mechanism for recruitment in its own right. Here, as Fabian pulls Kate’s coat up, he begins a course of action that requires participation from her and thereby recruits her to carry out the action with him. Kate evidently recognizes this and lifts herself off the ground to allow Fabian to pull her coat up.

\subsection{Grammatical formats}\label{sec:kendrick:4.2}
The grammar of a language includes a multitude of forms and formats that speakers use to construct turns at talk and produce recognizable social actions. For the recruitment of co-participants to act on one’s behalf -- surely one of the most basic of all social actions -- the forms of grammar that speakers of English use most frequently come from only three basic types. Over 90 percent of linguistic recruiting moves in the dataset have an interrogative, imperative, or declarative grammatical format (see \tabref{tab:kendrick:2}).

\begin{table}
\begin{tabularx}{0.66\textwidth}{Xrr}
\lsptoprule
Grammatical format 	& Frequency	& Proportion \\
\midrule
Interrogative	& 78	& 39.8\%         \\
Imperative	& 73	& 37.2\%            \\
Declarative	& 33	& 16.8\%           \\
Non-clausal	& 11	& 5.6\%            \\
Other	& 1	& 0.5\%                   \\
\lspbottomrule
\end{tabularx}

\caption{Frequency and proportion of grammatical formats in recruiting moves (\textit{n}=196).}
\label{tab:kendrick:2}
\end{table}

\newpage
Referred to as sentence types by linguists \citep{SadockZwicky1985, Palmer2001, KönigSiemund2007}, these grammatical formats institutionalize basic social relations (e.g. epistemic and deontic authority) and recurrent interactional contingencies (e.g. the redistribution of knowledge and the performance of practical actions) that all participants in social interaction must have ways to manage. The intricate relations between grammatical formats and social actions form a complex web, with no simple one-to-one correspondences \citep{Schegloff1984, Levinson2013}. Imperatives, for example, can and frequently do direct the actions of others (e.g. \textit{drink that}), but so too can they offer (e.g. \textit{have the last one}), admonish (e.g. \textit{just watch it, okay?}), initiate repair (e.g. \textit{pardon me}), or grant a request (e.g. \textit{go for it}), among other possibilities \citep{KentKendrick2016}. But even within such a complex web of relations, order emerges, as particular forms are tied to general domains of action \citep{Couperkuhlen2015}.

Given the large number of recruiting moves in the dataset that employ these basic formats, a complete enumeration of all types and subtypes is not possible within the confines of this chapter. Instead, for each format, we will examine a small set of cases in order to address a specific question or to bring a novel phenomenon into view. And for those recruitments without a predicate, which therefore do not belong to one of the three basic types, a simple discussion of their rather restricted context of use will suffice.

\subsubsection{Interrogatives}\label{sec:kendrick:4.2.1}
To recruit a co-participant to perform a practical action one can simply ask him or her to do so. In the dataset, the most frequent grammatical format for linguistic recruiting moves is the interrogative. Speakers generally use interrogatives to query the abilities or desires of recipients to perform an action (e.g. \textit{can you pass me the butter, will you hand me that}) or to ask about the availability or location of objects (e.g. \textit{do you have a cup, where’s the bottle opener}). Such questions exploit an asymmetry between an unknowing speaker and a knowing recipient, indexed by the epistemic stance of interrogative grammar \citep{heritage_terms_2005, heritage2012}, as a generic mechanism for recruitment. Traditionally referred to as indirect speech acts \citep{Searle1975, BrownLevinson1987} because they ostensibly concern the practical and personal contingencies of performing an action and not its performance per se, interrogatives such as these are better thought of as social action formats \citep{Fox2007} for recruitment, each with its own quirks and specifiable domains of use.

\tabref{tab:kendrick:3} presents the frequencies and proportions of interrogative recruiting moves identified in the dataset. Those with fewer than two attestations appear under other.

\begin{table}
\begin{tabularx}{0.66\textwidth}{Xrr}
\lsptoprule
Format	& Frequency	& Proportion  \\
\midrule
\textit{can I/we} 	& 14	& 17.9\%          \\
\textit{can/could you} 	& 14	& 17.9\%     \\
\textit{do you have} 	& 12	& 15.4\%       \\
\textit{will/would you} 	& 9	& 11.5\%     \\
\textit{where is} 	& 7	& 9.0\%            \\
\textit{do you want} 	& 6	& 7.7\%         \\
\textit{how about} 	& 2	& 2.6\%           \\
\textit{is there} 	& 2	& 2.6\%            \\
\textit{are we} 	& 2	& 2.6\%              \\
other	& 10	& 12.8\%              \\
\lspbottomrule
\end{tabularx}
\caption{\label{tab:kendrick:3} Frequency and proportion of interrogative formats in recruiting moves (\textit{n}=78).}
\end{table}

Although in principle one could investigate each of these forms to arrive at a description of the specific socio-interactional conditions under which they occur (see \citealt{Rossi2015a,Zinken2015,fox_rethinking_2016,FoxHeinemann2017}), we will here restrict our discussion to  a comparison of just two of these forms.

It has been suggested that the distinction between \textit{can/could you} and \textit{will/would you}  is “of relatively minor significance” \citep[150, fn. 1]{curlcontingency2008}. An examination of the distribution of these formats in the present dataset, however, suggests at least two possible differences. The first concerns the grantability of the request. The selection of \textit{will/would you} over \textit{can/could you} appears to orient to possible contingencies that may affect the grantability of the request, in line the observation by Curl and Drew that \textit{can/could you} displays relatively little orientation to such matters. Consider, for example, the request in the following extract, which comes from the early moments of a family mealtime. After Britney hands her mother a plate, and as her mother begins to take her seat, she asks for the butter, using the form \textit{can you}.

\transheader{ex:kendrick:11}{SLF 24:53}

\begin{transbox}{1}{~}
\begin{verbatim}
+(2.4)+*(0.2)*
\end{verbatim}
\end{transbox}

\begin{transbox}{2}{\textit{bri}}
\begin{verbatim}
+picks up plate and holds out it to Mom+
\end{verbatim}
\end{transbox}

\begin{transbox}{3}{\textit{mom}}
\begin{verbatim}
       *takes plate*
\end{verbatim}
\end{transbox}

\begin{transbox}{4}{mom} %\begin{transbox}{4}{MOM\hspace*{-2mm}}
\begin{verbatim}
*tha:nk yo:u.
\end{verbatim}
\end{transbox}

\begin{transbox}{5}{~}
\begin{verbatim}
*sets it on table-->
\end{verbatim}
\end{transbox}

\emptytransbox{6}{
(1.1)*
}

\begin{transbox}{7}{~}
\begin{verbatim}
  -->*
\end{verbatim}
\end{transbox}

\begin{mdframedkendrick}[style=firstfoc]
\begin{transbox}{8}{bri}
\begin{verbatim}
c’n #you pass me the butter:.
\end{verbatim}
\end{transbox}
\end{mdframedkendrick}\vspace{-5mm}

\begin{transbox}{9}{\fig}
\begin{verbatim}
    #fig.9
\end{verbatim}
\end{transbox}

\emptytransbox{10}{
(0.4)*(0.4)
}

\begin{mdframedkendrick}[style=secondfoc]
\begin{transbox}{11}{\textit{mom}}
\begin{verbatim}
     *picks up butter-->
\end{verbatim}
\end{transbox}
\end{mdframedkendrick}\vspace{-5mm}

\begin{transbox}{12}{mo?}
\begin{verbatim}
mm:
\end{verbatim}
\end{transbox}

\emptytransbox{13}{(0.8)}

\begin{transbox}{14}{mom} %\begin{transbox}{14}{MOM\hspace*{-2mm}}
\begin{verbatim}
*d’we (.) go through enough butter* and bread last
\end{verbatim}
\end{transbox}

\begin{transbox}{15}{~}
\begin{verbatim}
*holds it out to Britney----------*
\end{verbatim}
\end{transbox}%\bigskip

\emptytransbox{16}{
night?
}

\begin{transbox}{17}{bri}
\begin{verbatim}
oh my go:sh.
\end{verbatim}
\end{transbox}\bigskip

\begin{figure}
\caption{Immediately after the mother (in black) sets down her plate, Britney points to the butter and asks her to pass it.}
\includegraphics[width=\textwidth]{figures/KendrickFigure9.jpg}
\end{figure}

There are few, if any, contingencies that could affect the grantability of this mundane request. It occurs in a setting where such requests (and their granting) are common. It is produced at a very precise moment -- immediately after the mother has set her plate on the table, but before she has had an opportunity to begin a next course of action (lines 5--7) -- thereby obviating one possible source of contingency. And it requests an object, the butter dish, that is directly next to the mother’s plate, readily within her reach (cf. \citealt{KeisanenRauniomaa2012}).

The request in the following extract, in contrast, occurs under less opportune circumstances. As Kimmy recounts a problem with a client at her work to Carrie (lines 1--6), Donna, who has not been involved in the telling, looks around the table, picks up a bottle of beer, inspects it, and sets it back down. She then points to a bottle of beer on the other side of the table and asks Carrie, the recipient of Kimmy’s telling, to hand it to her, using the form \textit{will you}.

\transheader{ex:kendrick:12}{BBQ 27:27}

\begin{transbox}{1}{kim}
\begin{verbatim}
if a seventy eight year old man I can teach to
\end{verbatim}
\end{transbox}

\emptytransbox{2}{swim, and I can teach like a five year old kid}

\emptytransbox{3}{to swim, you should be able to swim lady.=if}

\emptytransbox{4}{you can’t something’s fuckin wrong with you.}

\emptytransbox{5}{(0.4)}

\begin{transbox}{6}{kim}
\begin{verbatim}
it’s not me.
\end{verbatim}
\end{transbox}

\begin{mdframedkendrick}[style=firstfoc]
\begin{transbox}{7}{car}
\begin{verbatim}
+(honey) will yo+u #hand me [that.
\end{verbatim}
\end{transbox}
\end{mdframedkendrick}\vspace{-5mm}

\begin{transbox}{8}{~}
\begin{verbatim}
+...............+points at beer across table-->
\end{verbatim}
\end{transbox}

\begin{transbox}{9}{\fig}
\begin{verbatim}
                   #fig.10a
\end{verbatim}
\end{transbox}

\begin{transbox}{10}{kim}
\begin{verbatim}
                            [it’s you.
\end{verbatim}
\end{transbox}

\begin{transbox}{11}{don}
\begin{verbatim}
that’s so wei:rd.
\end{verbatim}
\end{transbox}

\begin{transbox}{12}{kim}
\begin{verbatim}
*yea:h.
\end{verbatim}
\end{transbox}

\begin{transbox}{13}{\textit{don}}
\begin{verbatim}
*leans forward and looks around on table-->
\end{verbatim}
\end{transbox}

\emptytransbox{14}{(0.3)}

\begin{transbox}{15}{don}
\begin{verbatim}
what’d you* need?
\end{verbatim}
\end{transbox}

\begin{transbox}{16}{~}
\begin{verbatim}
       -->*
\end{verbatim}
\end{transbox}

\emptytransbox{17}{(0.2)}

\begin{transbox}{18}{car}
\begin{verbatim}
   +the beer.+
\end{verbatim}
\end{transbox}

\begin{transbox}{19}{~}
\begin{verbatim}
-->+,,,,,,,,,+
\end{verbatim}
\end{transbox}

\emptytransbox{20}{        
*(0.7)
}

\begin{mdframedkendrick}[style=secondfoc]
\begin{transbox}{21}{\textit{don}}
\begin{verbatim}
*picks up beer and holds it out to Carrie-->
\end{verbatim}
\end{transbox}
\end{mdframedkendrick}\vspace{-5mm}

\begin{transbox}{22}{do?}
\begin{verbatim}
(       )
\end{verbatim}
\end{transbox}

\begin{transbox}{23}{kim}
\begin{verbatim}
and I’d be totally*# nicer about it but (.) she
\end{verbatim}
\end{transbox}

\begin{transbox}{24}{\textit{don}}
\begin{verbatim}
               -->*
\end{verbatim}
\end{transbox}

\begin{transbox}{25}{\fig}
\begin{verbatim}
                   #fig.10b
\end{verbatim}
\end{transbox}

\begin{transbox}{26}{~}
\begin{verbatim}
was a bitch.
\end{verbatim}
\end{transbox}\bigskip

\begin{figure}
\caption{Donna points to a bottle of beer as she asks Carrie to hand it to her.}
\includegraphics[width=\textwidth]{figures/KendrickFigure10.jpg}
\end{figure}

Although Donna’s request is produced at the possible completion of a turn-constructional unit (lines 6--7), suggesting that she has timed it so as not to interrupt, it occurs at a position in which the telling sequence is not yet possibly complete.\footnote{Note that Carrie has not yet responded to the telling, which she does at line 10 before she fulfills the request.} The request is thus interjected into an on-going activity and addressed to a participant whose status as the recipient of the telling renders her less than fully available to grant the request at that moment. To do so immediately would require that she suspend or postpone one action (i.e. responding to the telling) in order to produce the other. The selection of \textit{will you} over \textit{can you} in this case appears to orient to local contingencies such as these that influence the grantability of the request. Note that the request indeed runs into trouble as Carrie must initiate repair before she can fulfill it (lines 15--18). In the dataset, interrogative requests for which no discernable contingencies exist occur as \textit{can/could you}, whereas those that involve subtly more complex circumstances or actions appear as \textit{will/would you}.


A second distinction between \textit{can/could you} and \textit{will/would you} concerns a specific type of request that occurs in one form but not the other. Requests that find fault in the actions -- or more specifically the \textit{inactions} -- of the recipient occur as \textit{can/could you} but not as \textit{will/would you} (cf. \citealt{KentKendrick2016}). The following extracts illustrate such “fault-finding” requests.

\transheader{ex:kendrick:13}{RCE06 15:48}

\begin{mdframedkendrick}[style=firstfoc]
\begin{transbox}{1}{jes}
\begin{verbatim}
can you move up cause I’m like *really [long=
\end{verbatim}
\end{transbox}
\end{mdframedkendrick}\vspace{-5mm}

\begin{transbox}{2}{\textit{sar}}
\begin{verbatim}
                               *..........-->
\end{verbatim}
\end{transbox}

\begin{transbox}{3}{sar}
\begin{verbatim}
                                       [ye:ah.
\end{verbatim}
\end{transbox}

\begin{transbox}{4}{jes}
\begin{verbatim}
=and you’re just hogging the whole thing.*
\end{verbatim}
\end{transbox}

\begin{mdframedkendrick}[style=secondfoc]
\begin{transbox}{5}{\textit{sar}}
\begin{verbatim}
 ........prepares to move over........-->*moves-->
\end{verbatim}
\end{transbox}
\end{mdframedkendrick}\vspace{-5mm}

\emptytransbox{6}{
(0.4)*
}

\begin{transbox}{7}{~}
\begin{verbatim}
  -->*
\end{verbatim}
\end{transbox}

\begin{transbox}{8}{sar}
\begin{verbatim}
↑why’d you say that.↑
\end{verbatim}
\end{transbox}\smallskip

\transheader{ex:kendrick:14}{RCE08 04:05}

\begin{mdframedkendrick}[style=firstfoc]
\begin{transbox}{1}{ben}
\begin{verbatim}
can you get the milk off your chin cause you're
\end{verbatim}
\end{transbox}
\end{mdframedkendrick}\vspace{-5mm}

\emptytransbox{2}{
being filmed and the milk on your chin is not a
}

\emptytransbox{3}{
good im*pression.
}

\begin{transbox}{4}{\textit{ker}}
\begin{verbatim}
       *......-->
\end{verbatim}
\end{transbox}

\emptytransbox{5}{
*(0.6)
}

\begin{mdframedkendrick}[style=secondfoc]
\begin{transbox}{6}{~}
\begin{verbatim}
*wipes chin-->>
\end{verbatim}
\end{transbox}
\end{mdframedkendrick}\vspace{-5mm}

\begin{transbox}{7}{ben}
\begin{verbatim}
well done.
\end{verbatim}
\end{transbox}\bigskip

In the first case, as Jessica and Sarah sit on a blanket on the lawn of a university campus, Jessica asks Sarah to move over to make room for her. The request includes an account that finds fault with Sarah’s inaction, blaming the trouble (i.e. that Jessica does not have enough space on the blanket) on her. Note that after Sarah complies with the request, she immediately challenges the account, orienting to its fault-finding character. In the second case, after Ben evidently notices that Kerry, who has been eating a bowl of cereal, has milk on her chin, he asks her to remove it, and like the first case he also includes an account that (teasingly) finds fault with her inaction. In each of these, the speaker produces a multi-unit turn in a \textsc{request} + \textsc{account} format with no prosodic boundary between the two units, and the complex action that results both asks the recipient to perform an action and holds her to account for not having already done so. The motivation for the selection of \textit{can/could you} over \textit{will/would you} for such requests supports the conclusion that\textit{ will/would you} indexes greater contingency. For the speaker to find fault with the recipient’s inaction, there should be no local contingencies that would have impeded the performance of the action and could therefore provide an account for the recipient’s inaction.

In comparison to the differences that \citet{curl_contingency_2008} observed between \textit{can/could you} and \textit{I wonder if} -- a form of request that does not occur in the present dataset -- the distinction between \textit{can/could you} and \textit{will/would you} is indeed subtle, and additional data must be brought to bear on this issue before a final verdict can be reached. But the data at hand suggest the two forms are not equivalent and the differences between them, while perhaps minor, are interactionally significant.

\subsubsection{Imperatives}\label{sec:kendrick:4.2.2}
Under what circumstances does one participant \textit{tell} another to perform an action rather than \textit{ask} him or her to do so? In general, speakers use imperatives in interactional contexts in which the sequential contingency of an interrogative request, which orients to a recipient’s right to refuse, is unnecessary, such as after participants have agreed explicitly or tacitly to a collaborative activity \citep{Wootton1997, Rossi2012, ZinkenOgiermann2013}, or is otherwise inexpedient, such as when the situation calls for immediate action. In contrast to the epistemic stance of interrogatives, imperatives typically encode a deontic stance in which the speaker claims the authority \citep{StevanovicPerakyla2012, StevanovicPerakyla2014} or the entitlement \citep{CravenPotter2010} to direct the actions of the recipient. Although imperatives implement a diversity of social actions (see \sectref{sec:kendrick:4.2} for examples), the recruitment of a co-participant to perform a practical action is among the most common. Referred to as directives by some \citep{Goodwin2006, Kent2011, kent_compliance_2012} and  requests by others \citep{Rossi2012, couper-kuhlen_what_2014}, such imperatives name a practical action and thereby make relevant the performance of that action as a preferred response (see \citealt{Kent2011}, on the preference organization of directive sequences).\footnote{In comparison to interrogatives, which have a number of highly frequent forms, imperatives are relatively homogeneous. The vast majority of imperatives in the dataset (83.6\%, \textit{n}=61) were simple verbal predicates; six (8.2\%) were prohibitives formed with \textit{don’t} (e.g. \textit{don’t get it out} in \extref{ex:kendrick:3}); and six (8.2\%) included either let me or let’s (e.g. \textit{let me have that}).}

A basic distinction that runs through the set of imperatives in the dataset involves the complex relationship between (i) the imperative, (ii) the practical action it makes relevant, and (iii) the course of action in which they both occur \citep[see][]{KentKendrick2016}. As \citet{Wootton1997} and \citet{Rossi2012} have observed, a common home for imperatives is in the midst of a collaborative activity. While speakers can use imperatives to initiate courses of action (e.g. the passing of plates at the dinner table), it is more common that they use them to manage courses of action that have already been set in motion.

In \REF{ex:kendrick:10} above, for example, before Fabian directs Kate to cover herself up properly with an imperative, he initiates the course of action with an interrogative (i.e. \textit{are you cold?}). Similarly, in the following extract, after Hailey offers Britney pickles, an action that initiates a course of action, she uses an imperative to direct her to ask their father, an action that progresses the course of action.

\transheader{ex:kendrick:15}{Sunday Lunch}

\begin{transbox}{1}{hal}
\begin{verbatim}
did you want pickles Britney?
\end{verbatim}
\end{transbox}

\begin{transbox}{2}{bri}
\begin{verbatim}
un-uh.
\end{verbatim}
\end{transbox}

\begin{transbox}{3}{hal}
\begin{verbatim}
ask dad.
\end{verbatim}
\end{transbox}

\emptytransbox{4}{(0.5)}

\begin{transbox}{5}{bri}
\begin{verbatim}
DA:::D.
\end{verbatim}
\end{transbox}\bigskip

The courses of action in which imperatives occur can have recognizable structures, such that the relevance of a specific next action is projectable on the basis of prior actions or events. In the example above, while the relevance of a subsequent offer to another member of the family may be projectable (e.g. on the basis of etiquette or other social norms), the delegation of this task to Hailey surely is not. It is a contingent and opportunistic development of the course of action (i.e. Britney’s declination), not one that is anticipatable on the basis of the initial offer. As a point of contrast, consider the course of action in the following extract. Here an imperative directs a recipient to perform an action whose relevance \textit{precedes} the imperative itself. During a break in a game of Monopoly, after Luke opens a can of beer, he notices that the beer is partially frozen (line 4). The beer then begins to overflow from the can (the result of a chemical reaction as the sudden decrease in pressure in the can lowers the freezing point of the beer, causing it to freeze and expand). As a solution to this emergent problem, Luke sips the frozen beer intermittently as it comes out.

\transheader{ex:kendrick:16}{Monopoly Boys 37:41}

\begin{transbox}{1}{~}
\begin{verbatim}
*(0.2)    *
\end{verbatim}
\end{transbox}

\begin{transbox}{2}{\textit{luk}}
\begin{verbatim}
*opens can*
\end{verbatim}
\end{transbox}

\emptytransbox{3}{(0.3)}

\begin{transbox}{4}{luk}
\begin{verbatim}
o:h, [it's slushy.
\end{verbatim}
\end{transbox}

\begin{transbox}{5}{ric}
\begin{verbatim}
     [aw:.
\end{verbatim}
\end{transbox}

\begin{transbox}{6}{~}
\begin{verbatim}
*(0.7)     *(0.8)    *(0.6)     *(0.8)#
\end{verbatim}
\end{transbox}

\begin{transbox}{7}{\textit{luk}}
\begin{verbatim}
*raises can*sips beer*lowers can*holds-->
\end{verbatim}
\end{transbox}

\begin{transbox}{8}{\fig}
\begin{verbatim}
                                      #fig.11
\end{verbatim}
\end{transbox}

\begin{transbox}{9}{ric}
\begin{verbatim}
what the hell.
\end{verbatim}
\end{transbox}

\emptytransbox{10}{(0.5)}

\begin{transbox}{11}{luk}
\begin{verbatim}
do you have a *cup?
\end{verbatim}
\end{transbox}

\begin{transbox}{12}{~}
\begin{verbatim}
           -->*raises can-->
\end{verbatim}
\end{transbox}

\emptytransbox{13}{
(0.6)*
}

\begin{transbox}{14}{~}
\begin{verbatim}
  -->*sips beer-->
\end{verbatim}
\end{transbox}

\begin{transbox}{15}{ric}
\begin{verbatim}
yeah but what's going on *with that. I've never seen
\end{verbatim}
\end{transbox}

\begin{transbox}{16}{\textit{luk}}
\begin{verbatim}
                      -->*lowers can-->
\end{verbatim}
\end{transbox}

\emptytransbox{17}{
that before.
}

\emptytransbox{18}{
(.)*
}

\begin{transbox}{19}{\textit{luk}}
\begin{verbatim}
-->*
\end{verbatim}
\end{transbox}

\begin{mdframedkendrick}[style=firstfoc]
\begin{transbox}{20}{ric}
\begin{verbatim}
hey. (.) drink that.=↑quick.↑=↑↑it's coming ba:ck.↑↑
\end{verbatim}
\end{transbox}
\end{mdframedkendrick}\vspace{-5mm}

\emptytransbox{21}{(0.2)}

\begin{transbox}{22}{luk}
\begin{verbatim}
I- (.) ahh
\end{verbatim}
\end{transbox}

\begin{transbox}{23}{ric}
\begin{verbatim}
COME ON.=IT’S COMING OUT.=IT’S GONNA GET ON MY
\end{verbatim}
\end{transbox}

\begin{transbox}{24}{~}
\begin{verbatim}
my-* (.) ta:bl:e.*
\end{verbatim}
\end{transbox}

\begin{transbox}{25}{\textit{luk}}
\begin{verbatim}
   *raises can-->*sips-->>
\end{verbatim}
\end{transbox}\bigskip

\begin{figure}
\caption{Luke (white shirt) holds his gaze on the beer can for 0.8 seconds as frozen beer begins to emerge.}
\includegraphics[width=\textwidth]{figures/KendrickFigure11.png}
\end{figure}

The course of action that develops as Luke manages the problem has a projectable structure. At least two bases for this can be identified. Firstly, the very recognition of the problem allows for the projection of a set of possible solutions, such as sipping the frozen beer (lines 7 and 14) or pouring it into a cup (line 11). Thus once a problem has been recognized and publicly registered, as Rick does with \textit{what the \uline{he}ll} in line 9, the provision of a possible solution becomes relevant. Secondly, after Luke has twice sipped the beer after it began to overflow, that he could or should do so again becomes anticipatable as a possible solution to the problem. In this way, the local structure of the sequence provides a basis for the projection of possible next actions.

\largerpage
With this in mind, we can now see that the imperative that Rick produces at line 20 (\textit{\uline{dri}nk that}) occurs in a position at which the action it directs Luke to perform is already relevant as a possible solution to the problem. Furthermore, before Rick directs him to do so, Luke has had an opportunity to perform the relevant action. Note that Rick produces the particle \textit{hey}, a minimal form that alerts Luke of the reemergence of the trouble, and then pauses briefly before he issues the imperative (line 20). This prompts action from Luke and creates an opportunity for him to act, one that he does not take. Thus both the relevance of the requested action and an opportunity to perform it \textit{precede} the imperative. This stands in clear contrast to the imperative in \REF{ex:kendrick:15} (\textit{ask dad}), in which the relevance of the directed action and the opportunity to perform it both \textit{follow} the imperative.


What is the consequence of this difference? The position in which an imperative request occurs “colors” its action, such that imperatives that follow the relevance of the requested action and an opportunity to perform it not only request a recipient to perform an action but also “admonish” him or her for not having already done so \citep{KentKendrick2016}. Within the present dataset, imperatives frequently occur in this po fsition and frequently do more than just recruit a co-participant to act (cf. \citealt{Mandelbaum2014}). The data therefore suggest that speakers use imperatives not only for the management of practical courses of action -- to recruit others to do things per se -- but also for the \textit{ex post facto} enforcement of social norms that regulate practical courses of action.

\subsubsection{Declaratives}\label{sec:kendrick:4.2.3}
Just as one can ostensibly inquire into the abilities, desires, and future actions of co-participants to recruit them, so too can one inform co-participants of one’s own desires, needs, or future actions to do so \citep[see][]{Stevanovic2011, Childs2012}. The majority of declaratives in the dataset (\textit{n}=19, 58\%) include linguistic forms that index the obligations, volitions, or abilities of either the speaker or recipient. The most frequent forms (i.e. those with multiple attestations) are \textit{you should} (\textit{n}=4, 12\%), \textit{I want} (\textit{n}=4, 12\%), \textit{I need} (\textit{n}=2, 6\%), \textit{I’ll have} (\textit{n}=2, 6\%), and \textit{we need} (\textit{n}=2, 6\%).

Typically declarative grammar encodes an epistemic stance that is the inverse of interrogatives: the speaker has knowledge that the recipient lacks \citep{heritage_terms_2005, Raymond2010, heritage2012}. But as recruiting moves, declaratives nonetheless frequently exploit an epistemic asymmetry in which the recipient is in a K+ position. A speaker who informs a recipient of a need, as in the following extract, claims to know what should be done but not how to do it.

\transheader{ex:kendrick:17}{BBQ 23:09}

\begin{transbox}{1}{jam}
\begin{verbatim}
I was expecting like [a deposit in my accou(h)nt.
\end{verbatim}
\end{transbox}

\begin{transbox}{2}{ali}
\begin{verbatim}
                     [oh James.
\end{verbatim}
\end{transbox}

\begin{transbox}{3}{jam}
\begin{verbatim}
heh heh heh
\end{verbatim}
\end{transbox}

\begin{transbox}{4}{ali}
\begin{verbatim}
James.
\end{verbatim}
\end{transbox}

\emptytransbox{5}{(0.2)}

\begin{transbox}{6}{ali}
\begin{verbatim}
really quick?=
\end{verbatim}
\end{transbox}

\begin{transbox}{7}{jam}
\begin{verbatim}
wha[:t.
\end{verbatim}
\end{transbox}

\begin{transbox}{8}{ali}
\begin{verbatim}
   [can I get he:lp for something.
\end{verbatim}
\end{transbox}

\begin{transbox}{9}{jam}
\begin{verbatim}
what’s up.
\end{verbatim}
\end{transbox}

\begin{transbox}{10}{ali}
\begin{verbatim}
can you put that down for just a minute.
\end{verbatim}
\end{transbox}

\begin{transbox}{11}{jam}
\begin{verbatim}
hold on lemme just get done with (this last piece)
\end{verbatim}
\end{transbox}

\emptytransbox{12}{((four lines omitted))}

\begin{transbox}{13}{jam}
\begin{verbatim}
wh[at’s up babe.
\end{verbatim}
\end{transbox}

\begin{mdframedkendrick}[style=firstfoc]
\begin{transbox}{14}{ali}
\begin{verbatim}
  [I need to check- I need to check th*e uhm (0.2) the
\end{verbatim}
\end{transbox}
\end{mdframedkendrick}\vspace{-5mm}

\begin{mdframedkendrick}[style=secondfoc]
\begin{transbox}{15}{\textit{jam}}
\begin{verbatim}
                                      *stands up-->
\end{verbatim}
\end{transbox}
\end{mdframedkendrick}\vspace{-5mm}

\emptytransbox{16}{        
microphone quality?
}

\emptytransbox{17}{(0.5)}

\begin{transbox}{18}{jam}
\begin{verbatim}
do you have earphones?*
\end{verbatim}
\end{transbox}

\begin{transbox}{19}{~}
\begin{verbatim}
                   -->*holds-->
\end{verbatim}
\end{transbox}

\emptytransbox{20}{(0.3)}

\begin{transbox}{21}{ali}
\begin{verbatim}
yeah I do but I don’t know *where the earphone plugin is.
\end{verbatim}
\end{transbox}

\begin{transbox}{22}{\textit{jam}}
\begin{verbatim}
                           *walks behind camera-->>
\end{verbatim}
\end{transbox}

\begin{transbox}{23}{jam}
\begin{verbatim}
°okay.°
\end{verbatim}
\end{transbox}\bigskip
% \todo{° is used in verbal part. What is its meaning?}

The mechanism of recruitment in such cases is analogous to that of a ‘my side’ telling \citep{pomerantz1980}, in which a speaker reports his or her limited access to an event and thereby elicits additional information from a recipient who has greater access \citep{Childs2012}. In recruiting moves, however, a speaker reports his or her knowledge of a situation that requires action (i.e. a practical problem) and thereby elicits action from a recipient (i.e. a practical solution) who has a greater ability, availability, or obligation to perform the action. In the example above, the basis for James’s ability to resolve the problem is made explicit earlier in the conversation when Alison reports that he used to own the same model of video camera. Recruitments such as this reveal a complex relationship between the epistemic status of the recipient (i.e. James knows how to operate the camera) and the recipient’s obligation to provide assistance.

The selection of a declarative over other formats does not necessarily depend on the epistemic and deontic status of the speaker, however. The grammatical format of a recruiting move also affects the opportunity for response in important ways. Whereas interrogative requests constrain the response such that non-granting responses are dispreferred actions, declarative requests leave the response relatively open, allowing for a larger set of possible next actions \citep{VinkhuyzenSzymanski2005,rossi_grammar_2016}. In the present dataset, the influence of the grammatical format can be seen indirectly in the quantitative distribution of responses. While over two thirds of interrogatives received a verbal response (69.2\%, \textit{n}=54), only half of the declaratives did (51.5\%, \textit{n}=17), with the remainder receiving an embodied response or no response at all. This shows that participants orient to the two formats in different ways. The preference for polar interrogatives to receive polar responses \citep{Raymond2003} may have contributed to this difference. Polar tokens, such as \textit{yeah}, \textit{okay}, \textit{no}, and \textit{nah}, occurred in response to 40.9 percent of polar interrogatives (\textit{n}=27), in contrast to only 15.2 percent for declaratives (\textit{n}=5). The quantitative distribution of responses in the dataset supports previous observations that declarative requests place fewer, or at least different, constraints on the response space than interrogatives do (cf. Rossi, \chapref{sec:rossi}, \sectref{sec:rossi:4.2}). The selection of a declarative format for recruitment thus affords greater agency to the recipient, who has an opportunity to select a response from a larger set of alternatives, including sequence initiating actions \citep[cf.][111]{KendrickDrew2014}.

Although the majority of declaratives in the dataset are modal statements that index the desires, abilities, or obligations of the participants, many are not. One type that calls co-participants to action without reference to such personal states is the announcement of the completion of a task (e.g. the familiar \textit{dinner’s ready}) \citep[cf.][384]{Rossi2018}. Announcements of task completion exploit the normative organization of complex activities, which can involve transformations of the participation framework, as a mechanism for recruitment. For this to work, the course of action that comes to completion (e.g. the preparation of food) must belong to a complex activity (e.g. the meal as a whole) that includes a subsequent course of action (e.g. serving and eating the food). The completion of one course of action makes the initiation of a next conditionally relevant \citep[cf.][213--215]{schegloff_sequence_2007}. For some activities, the participation framework also changes; a course of action that involves few participants (e.g. those who prepared the food) can transform into one that involves many (e.g. those who will eat the food). The relevance of the announcement thus derives from the need to solicit participation in the next phase of the activity. In the following extract, which again comes from the interaction between friends as they prepare a barbecue, an announcement by Kimmy that the coals are ready recruits Carrie to begin the relevant next course of action.

\transheader{ex:kendrick:18}{BBQ 36:16}

\begin{mdframedkendrick}[style=firstfoc]
\begin{transbox}{1}{kim}
\begin{verbatim}
I think those coals# are ready for your sausages.
\end{verbatim}
\end{transbox}
\end{mdframedkendrick}\vspace{-5mm}

\begin{transbox}{2}{\fig}
\begin{verbatim}
                   #fig.12a
\end{verbatim}
\end{transbox}

\begin{transbox}{3}{ali}
\begin{verbatim}
yeah man.
\end{verbatim}
\end{transbox}

\begin{transbox}{4}{jam}
\begin{verbatim}
I think you *might wanna# pull up +the ra:ck? a little bit.
\end{verbatim}
\end{transbox}

\begin{transbox}{5}{\fig}
\begin{verbatim}
                        #fig.12b
\end{verbatim}
\end{transbox}

\begin{mdframedkendrick}[style=secondfoc]
\begin{transbox}{6}{\textit{car}}
\begin{verbatim}
            *reaches for sausages-->
\end{verbatim}
\end{transbox}
\end{mdframedkendrick}\vspace{-5mm}

\begin{transbox}{7}{\textit{kim}}
\begin{verbatim}
                                  +walks over to grill-->
\end{verbatim}
\end{transbox}

\emptytransbox{8}{(0.2)}

\begin{transbox}{9}{kim}
\begin{verbatim}
*yep.
\end{verbatim}
\end{transbox}

\begin{transbox}{10}{\textit{car}}
\begin{verbatim}
*picks up sausages-->
\end{verbatim}
\end{transbox}

\emptytransbox{11}{
(0.5)*(1.6)
}

\begin{transbox}{12}{\textit{car}}
\begin{verbatim}
     *moves towards grill-->>
\end{verbatim}
\end{transbox}

\begin{transbox}{13}{jam}
\begin{verbatim}
(it’s)+ pretty hot.
\end{verbatim}
\end{transbox}

\begin{transbox}{14}{\textit{kim}}
\begin{verbatim}
   -->+
\end{verbatim}
\end{transbox}\bigskip

\begin{figure}
\caption{Kimmy (red pants) announces that the grill is ready for Carrie’s sausages. Carrie (black hat) then reaches for the sausages before she moves towards the grill. }
\includegraphics[width=\textwidth]{figures/KendrickFigure12.jpg}
\end{figure}

The announcement also marks a transformation of the participation framework in the activity as a whole. Earlier in the interaction Kimmy had been recruited to light the coals (cf. \extref{ex:kendrick:2}) and Carrie had revealed that she had brought sausages to cook on the grill. With the announcement, Kimmy informs Carrie that her task is complete and that the relevant next course of action can commence. Note that while Kimmy does return to the grill (see lines 7 and 14), she does not do so to assist Carrie with the preparation of the food, but rather to resolve a practical problem in the course of action that she had just announced as complete (see line 3, which is also a recruiting move). As this example shows, announcements of task completion can be used by participants to manage a transformation of participation across successive courses of action within a complex activity and thereby recruit others to assist.

\subsubsection{Non-clausal}\label{sec:kendrick:4.2.4}
As we have seen, the three most frequent grammatical formats for recruiting moves -- interrogatives, imperatives, and declaratives -- differ in important ways. But behind such differences lies a common linguistic structure: the clause. Defined as a predicate (e.g. a verb or a verb complex) and its associated arguments (e.g. noun phrases), a clause is a linguistic structure that, in the words of \citet{thompson_clause_2005}, “can be thought of as a crystallization of solutions to the interactional problem of signaling and recognizing social actions” (p. 484). The observation that 93.8 percent of verbal recruiting moves in the dataset have clausal formats (see \tabref{tab:kendrick:3}) clearly supports the dominance of clausal formats in this domain of social action (or perhaps sequence-initiation more generally).

But what about the linguistic moves that do not have a clausal format? There are three recurrent types of phrasal recruiting moves in the dataset. First, a phrasal format can occur as a response to an other-initiation of repair that locates a clausal recruitment as a trouble source (e.g. A: \textit{is there a fork over there ki:ds?} B: \textit{what?} A: \textit{fork} in \extref{ex:kendrick:4}). Second, a phrasal format can occur as a pursuit of a response to a clausal format (e.g. A: \textit{l'mme have that butter when yer thr\uline{ou}gh there.} A: \textit{\uline{bu}tter please.}). Third, a phrasal format can occur as an address term that either pursues a response (e.g. A: \textit{Britney do you wanna help me set the table?} A: \textit{\uline{Br}it.}) or admonishes a recipient’s actions (e.g. \textit{shhh. \uline{Ow}:en.} after an inappropriate remark). As this list indicates, aside from the use of address terms as admonishments, phrasal formats tend to occur in sequence-subsequent positions (see also Floyd, \chapref{sec:floyd}, \sectref{sec:floyd:3.3.1}) after an initiation of a sequence by a clausal format. An exception to this generalization is the phrasal °°\textit{cookie}°° in \REF{ex:kendrick:5}, which occurs after an embodied recruiting move, a pointing gesture to a box of cookies, not a verbal one. This case nonetheless shows that phrasal formats tend to occur in sequence-subsequent positions after a more explicit move to recruit a co-participant, even if the initiation of the sequence is done without language.

\subsection{Additional turn components}
In addition to the grammatical format of the recruiting move, participants also optionally select among a set of turn-constructional components that adjust or modify the action in various ways. In this section we review three such components that were frequently observed in the dataset: address terms, mitigations, and accounts.

\subsubsection{Address terms}\label{sec:kendrick:4.3.1}
Address terms, such as names (e.g. \textit{Haley can you get the salt and peppy}), terms of endearment (e.g. \textit{honey will you hand me that}), and person reference categories (e.g. \textit{is there a f\uline{or}k over there ki:ds?} in \extref{ex:kendrick:5}), frequently occurred with recruiting moves in the dataset. Nearly 20 percent (\textit{n}=39) of linguistic recruiting moves included an address term, either as a turn-constructional component in turn-initial or turn-final positions or as a generic pre-expansion of the sequence (see, e.g., \extref{ex:kendrick:17}, lines 2 and 4). In comparison, a study of 328 questions in English conversation -- most of which were requests for information, confirmation, or repair -- found that only 2.1 percent (\textit{n}=7) occurred with an address term \citep[2777]{Stivers2010}. This suggests that address terms occur more frequently with recruiting moves than in other domains of social action in English.

Two possible explanations for the conspicuously high proportion of address terms in recruitments present themselves. The first concerns a generic problem of social interaction, one that is especially acute when more than two participants are involved: the selection of who should act or speak next.\footnote{Note that 86.3\% of sequences in the dataset, \textit{n}=182, came from multiperson interactions.} A participant can use an address term to designate a particular co-participant as the addressed recipient of the recruiting move and thereby select him or her as the one who should respond \citep[cf.][]{SacksetAl1974, Lerner2003}. Similarly, an address term can be used as a generic pre-expansion to secure the attention a particular co-participant before a base sequence \citep[cf.][]{Schegloff1968, schegloff_sequence_2007}. Because recruiting moves so frequently initiate courses of action -- in contrast to requests for information, confirmation, or repair, which also frequently occur in non-initial positions (e.g. as insert expansions, post-expansions, and various “follow-up” questions) -- the socio-interactional problem for which address terms are a solution arises with greater frequency. Some indirect evidence for this comes from the relatively low proportion of address terms with imperatives, which tend to occur in non-initial positions within a course of action (see \sectref{sec:kendrick:4.2.2}): only 8.2 percent (\textit{n}=6) of imperatives include an address term, whereas 30.8 percent (\textit{n}=24) of interrogatives do (\citealt[see also][271, fn. 7]{ZinkenOgiermann2013}; \citealt[96]{Rossi2015a}).

The second explanation for the high proportion of address terms concerns the nature of the action itself. A special relationship between requests, broadly understood, and address terms was first noted by \citet{brown_universals_1978, BrownLevinson1987} who argued that such terms signal in-group membership between speaker and recipient and thereby mitigate the request’s threat to the recipient’s negative face \citep[cf.][1207]{RendleShort2010}. In an analysis of sequence-initiating actions in general, \citet{Lerner2003} distinguishes between turn-initial and turn-final address terms and argues that the latter can “demonstrate a particular stance toward or relationship with a recipient under circumstances where that demonstration is particularly relevant” (\citealt[185]{Lerner2003}; see also \citealt{Clayman2010,ButlerDanbyEmmison}). Although the present analysis does not distinguish between turn-initial and turn-final positions, the use of address terms to affirm one’s relationship with a recipient as one solicits his or her assistance -- a circumstance in which the relationship is especially relevant -- is also a plausible explanation for the high proportion of address terms observed with recruitment (see also Zinken, \chapref{sec:zinken}, \sectref{sec:zinken:6}; Baranova, \chapref{sec:baranova}, \sectref{sec:baranova:3.4}).

\subsubsection{Mitigators}
The linguistic construction of a recruiting move may also include a variety of practices that, in various ways, mitigate the action (e.g. \textit{can I have a little bit}). The observation that speakers use linguistic devices to mitigate particular social actions can be traced back to linguistic research by \citet{Lakoff1973} and \citet{BrownLevinson1987} on “hedges” (see \citealt{Schneider2010} for a review). From a conver\-sa\-tion-analytic perspective, the use of practices to mitigate a recruiting move orients to the preference for agreement in conversation \citep{Sacks1987}, as well as a general principle of social solidarity \citep{heritage_garfinkel_1984}, in that such practices minimize what the recipient must do to comply and thereby maximize the opportunity for an affiliative response. Consider, for example, the following extract. As James and his housemates prepare their respective dinners in a communal kitchen, he asks for an onion.

\transheader{ex:kendrick:19}{RCE09 11:43}

\begin{mdframedkendrick}[style=firstfoc]
\begin{transbox}{1}{jam}
\begin{verbatim}
has anyone got a spare onion I can borrow.
\end{verbatim}
\end{transbox}
\end{mdframedkendrick}\vspace{-5mm}

\emptytransbox{2}{(.)}

\emptytransbox{3}{
half an onion.
}

\begin{mdframedkendrick}[style=secondfoc]
\begin{transbox}{4}{ben}
\begin{verbatim}
ye:ah, go in there.
\end{verbatim}
\end{transbox}
\end{mdframedkendrick}

The construction of the request includes at least three practices that mitigate the action. First, the selection of \textit{a spare onion} over \textit{an onion} anticipates a possible basis for rejection, namely that one of the housemates may have an onion but may need it for his or her own meal. This provides a recipient who wishes to reject the request with a means to do so, thereby minimizing the potential for discord. Second, the selection of the word \textit{borrow} over \textit{have} implies only a temporary transfer of possession. Even though one is not expected to return consumables such as onions, the selection of \textit{borrow} nonetheless orients to the preference for agreement in that it ostensibly makes the request easier to grant. Third, after a short pause (just under 200 ms) in which no one responds, James issues a self-repair that minimizes the request in an even more transparent manner, cutting it in half. Transition space self-repairs such as this are common in requests and orient to the possibility of rejection that the absence of a response can indicate \citep{Davidson1984}.

\subsubsection{Accounts}
An account is an action, defined as a clausal turn-constructional unit in the present study, that provides a reason or explanation for a participant’s action or inaction (see \citealt{Robinson2016} for a review). Early conversation-analytic research on accounts by \citet{AtkinsonDrew1979} and \citet{heritage_garfinkel_1984} showed, among others,  that participants commonly provide accounts for responding actions that fail to align with normative expectations set by initiating actions (i.e. for dispreferred responses). Subsequent research has shown that accounts also accompany some initiating actions, most notably requests and directives \citep{Goodwin1990, HoutkoopSteenstra1990, Schegloff2007sequence, Raevaara2011, Parry2013, BaranovaDingemanse2016}. A possible explanation for the occurrence of accounts with requests comes from \citet{Schegloff2007sequence}, who argues that requests are dispreferred actions and that the construction of requests reveals a preference for offers (see \citealt{KendrickDrew2014} for counterarguments). One piece of evidence that Schegloff cites to support the dispreferred status of requests is the “regular” provision of accounts with them (p. 83). In light of the previous research on the relationship between requests and accounts, one would expect that accounts should be frequent within the present dataset. But in fact they are rare. Only 13.3 percent of linguistic recruiting moves (\textit{n}=26) include an account (see Extracts \ref{ex:kendrick:13}, \ref{ex:kendrick:14}, and \ref{ex:kendrick:16}).

Why are accounts for recruiting moves so rare? To answer this question we must first take a step back and review the organization of recruitment \citep{KendrickDrew2016}. In general terms, a recruiting move solicits or occasions a practical action from a co-participant in order to resolve a trouble encountered by its speaker in a practical course of action. The nature of such troubles varies: one may not have enough space on a couch (\extref{ex:kendrick:4}); one may need a utensil in order to eat a meal (\extref{ex:kendrick:5}); one may not be able to see a picture in a book from across a table (\extref{ex:kendrick:3}); and so on. The recruiting move frequently, but not invariably, formulates a possible solution to the trouble for the co-participant to perform. An imperative such as \textit{move over} (\extref{ex:kendrick:4}) formulates an action that constitutes a possible solution to a practical problem, and an interrogative such as \textit{is there a fork over there} (\extref{ex:kendrick:5}) similarly refers to an object that could resolve the participant’s problem. In contrast to the explicitness of the solutions, often embodied in the form of the recruiting move, the nature of the trouble is frequently left implicit, as participants treat the problems as transparently recognizable. In some cases, a problem has emerged in the interaction (e.g. the overflow of beer in \extref{ex:kendrick:16}) and thus need not be stated. In others, the problem is recognizable by reference to the norms of an activity such as a meal (e.g. the lack of a fork). The explicit articulation of such troubles is the most common function that accounts for recruiting moves serve. Accounts are thus rare in the dataset because the “here and now” troubles that arise in face-to-face interaction are routine, often manifest explicitly in the situation, and are transparently recognizable to co-participants \citep[see also][647]{BaranovaDingemanse2016}. Given this, the question becomes not why accounts are so rare, but why they should occur with requests at all. The fault-finding character of those requests that do include accounts (e.g. \extref{ex:kendrick:13}, \ref{ex:kendrick:14}, and \ref{ex:kendrick:16}) points to one possible answer and an avenue for future research.

\section{Responding moves}
\label{sec:kendrick:5}
The recipient of a recruiting move finds him- or herself in a position to respond. In this section we will consider this second move in a recruitment sequence. In contrast to the technical definition of a response as the second pair part of an adjacency pair \citep{SchegloffSacks1973, Schegloff2007sequence}, the term is used more freely in the comparative study to refer to any relevant action in a next position to the recruiting move (see \chapref{sec:coding}). This includes not only actions that complete the sequence, such as a move to fulfill a request, but also actions that, in one way or another, orient to the recruitment but leave the sequence open (e.g. a repair initiation). This includes cases in which participants recognize a trouble and volunteer assistance, thereby \textit{initiating} a course of action from a second position (cf. \citealt{Schegloff2007sequence} on “retro-sequences”). We will first consider the distribution of some general types of responding moves, so defined, that were observed in the dataset before we turn our attention to two specific types -- deferring the recruitment and recruiting the recruiter -- that require more detailed discussion.

\subsection{Response types}
At the most general level, the recipient of a recruiting move can either fulfill the sequence, that is, carry out a relevant practical action, or opt not to do so. As \tabref{tab:kendrick:4} shows, in the majority of cases (61.6\%, \textit{n}=130) the sequence is fulfilled. Furthermore, if one considers only those recruiting moves that occur in a sequence-final position, including minimal two-move sequences and terminal moves in expanded sequences (see \sectref{sec:kendrick:2}), the proportion of fulfillment increases to over 70 percent. This is clear evidence for the preference for agreement \citep{Sacks1987} and the principle of social solidarity \citep{heritage1984garfinkel}, two specifications of the general bias towards cooperation in social interaction (see also \citealt{FloydEtAl2018}).

\begin{table}
\begin{tabularx}{\textwidth}{Xrrrr}
\lsptoprule
	& \multicolumn{2}{c}{All cases}& \multicolumn{2}{c}{Sequence-final cases}\\
Response	& Frequency& Proportion	& Frequency& Proportion\\
 \midrule
Fulfillment	& 130	&   61.6\%	& 118	& 71.1\%\\
Other	        & 28	&  13.3\%	& 20	& 12.1\%\\
No response	& 27	&  12.8\%	& 14	& 8.4\%\\
Rejection	& 18	&  8.5\%	& 11	& 6.6\%\\
Repair initiation& 8	&  3.8\%	& 3	& 1.8\%\\
\lspbottomrule
\end{tabularx}

\caption{Frequency and proportion of response types for all cases (\textit{n}=211) and for sequence-final cases (\textit{n}=166)}
\label{tab:kendrick:4}
\end{table}

Consistent with this, rejection was rare in the dataset. Less than one in ten moves to recruit co-participants were rejected. This included explicit rejections, such as \textit{no} and \textit{fuck you}, as well as various accounts that accomplish rejection \citep{Drew1984}, such as \textit{I don’t know how to} and \textit{I’ll leave it to you}. More common were recruiting moves that received no response whatsoever. In such cases, the recipient produced neither a verbal response to the move nor a practical action that recognizably fulfilled (or began to fulfill) the sequence. This can be understood as an alternative to rejection, in that the recipient opts not to provide assistance but also not to reject the recruiting move explicitly. It can also be the result of a failure in addressing the move to a particular co-participant (see, e.g., \extref{ex:kendrick:5}). And in a small number of cases, the recipient of a recruiting move initiates repair (again, see \extref{ex:kendrick:5}). Responding moves that did not correspond to the types described thus far were analyzed as “other” for the comparative study. Three of these will be examined in detail in subsequent sections.

As for the modality of the responding move, nearly three quarters included relevant visible bodily actions (72.6\%, \textit{n}=143). Given that recruitment involves the performance of a practical action (and not, for instance, the provision of information), this comes as little surprise. Fully nonverbal responses were not the norm, however, as these accounted for just over a third of all cases (35\%, \textit{n}=69), whereas multimodal responses with both verbal and nonverbal components were slightly more frequent (37.6\%, \textit{n}=74). Fully verbal responses without relevant nonverbal behavior were relatively infrequent (18.8\%, \textit{n}=37).

\subsection{Deferring the recruitment}
Recruitment, by definition, involves the relevance of practical action in the here and now of the interaction and not proposals for some other time and place. The immediacy of recruitment poses a practical problem for participants who are already engaged in a course of action when a move to recruit them comes. The practical problem of multiple involvements \citep{ToerienKitzinger2007, RaymondLerner2014} or multiactivity \citep{Mondada2011, haddington2014multiactivity} is one for which participants have practiced solutions. In this and the next section, we will review two practices that recipients of recruiting moves use to manage the emergence of multiple involvements.

The first is deferring the recruitment. Rather than abandon or suspend the course of action the recipient is engaged in, the recipient can continue the course of action, using resources of the body, and defer the recruitment verbally, as in the following cases.

\transheader{ex:kendrick:20}{BBQ 23:09}

\begin{transbox}{4}{ali}
\begin{verbatim}
James.
\end{verbatim}
\end{transbox}

\emptytransbox{5}{(0.2)}

\begin{transbox}{6}{ali}
\begin{verbatim}
really quick?=
\end{verbatim}
\end{transbox}

\begin{transbox}{7}{jam}
\begin{verbatim}
wha[:t.
\end{verbatim}
\end{transbox}

\begin{transbox}{8}{ali}
\begin{verbatim}
   [can I get he:lp for something.
\end{verbatim}
\end{transbox}

\begin{transbox}{9}{jam}
\begin{verbatim}
what’s up.
\end{verbatim}
\end{transbox}

\begin{mdframedkendrick}[style=firstfoc]
\begin{transbox}{10}{ali}
\begin{verbatim}
can you put that down for just a minute.
\end{verbatim}
\end{transbox}
\end{mdframedkendrick}\vspace{-5mm}

\begin{mdframedkendrick}[style=secondfoc]
\begin{transbox}{11}{jam}
\begin{verbatim}
hold on lemme just get done with (this last piece)
\end{verbatim}
\end{transbox}
\end{mdframedkendrick}\vspace{-4mm}

\transheader{ex:kendrick:21}{RCE22b 08:23} % \vspace{-3mm}

\begin{mdframedkendrick}[style=firstfoc] % \noindent
\begin{transbox}{1}{lis}
\begin{verbatim}
Megan, do you know how to projector::ize::
\end{verbatim}
\end{transbox}
\end{mdframedkendrick}\vspace{-5mm}

\emptytransbox{2}{
[↑th:i:[ng:s:?
}

\begin{transbox}{3}{mar}
\begin{verbatim}
[yeah?
\end{verbatim}
\end{transbox}

\begin{mdframedkendrick}[style=secondfoc]
\begin{transbox}{4}{meg}
\begin{verbatim}
       [yep. #can you give me like two:: se:cs:.
\end{verbatim}
\end{transbox}
\end{mdframedkendrick}\vspace{-5mm}

\begin{transbox}{5}{\fig}
\begin{verbatim}
             #fig.13
\end{verbatim}
\end{transbox}\bigskip

\begin{figure}
\caption{Lisa (far right) asks Megan, who is using a computer, whether she knows how to use the projector.}
\includegraphics[width=\textwidth]{figures/KendrickFigure13.jpg}
\end{figure}

In \REF{ex:kendrick:20}, repeated in part from \REF{ex:kendrick:17}, after Alison asks James to stop the course of action he is engaged in -- peeling carrots -- to help her, he responds with a request to defer his assistance until the completion of his task. In this way, James tacitly commits to assisting Alison, but he prioritizes the progressive realization of his current task over that of the recruitment. Likewise, in \REF{ex:kendrick:21}, while Megan is engaged with her computer, Lisa asks her whether she knows how to use a video projector, which can be understood in the situation as a request for Megan to assist her. Megan first responds to the format of the recruiting move, confirming that she does know how, and then responds to its action implication, deferring her assistance to a later point in time. In sequence-structural terms, deferments initiate pre-second insert sequences that displace the relevant next action (i.e. the assisting action) from next position in the sequence \citep{Schegloff2007sequence} and thereby constitute one method that recipients have to manage the emergence of multiple involvements.

\subsection{Recruiting the recruiter}\label{sec:kendrick:5.3}
A second practice that participants have to manage the emergence of multiple involvements is for a recipient of the recruiting move to invert the recruitment sequence, that is, to recruit the recruiter to perform the action him- or herself. In the following extract, repeated with additional detail from \REF{ex:kendrick:19}, a recipient of a request, Ben, recruits its speaker, James, to fulfill the request himself.

\transheader{ex:kendrick:22}{RCE09 11:43}

\begin{transbox}{1}{jam}
\begin{verbatim}
has anyone got a spare onion I can borr#ow.
\end{verbatim}
\end{transbox}

\begin{transbox}{2}{\fig}
\begin{verbatim}
                                       #fig.14a
\end{verbatim}
\end{transbox}

\emptytransbox{3}{
*(.)
}

\begin{transbox}{4}{\textit{ben}}
\begin{verbatim}
*turns to left-->
\end{verbatim}
\end{transbox}

\begin{transbox}{5}{jam}
\begin{verbatim}
half an onion.
\end{verbatim}
\end{transbox}

\begin{transbox}{6}{ben}
\begin{verbatim}
*ye:ah,* #go in there.
\end{verbatim}
\end{transbox}

\begin{transbox}{7}{~}
\begin{verbatim}
*......*points----->>
\end{verbatim}
\end{transbox}

\begin{transbox}{8}{\fig}
\begin{verbatim}
         #fig.14b
\end{verbatim}
\end{transbox}\bigskip

\begin{figure}
\caption{Ben stands (white shirt) in front of the microwave as he waits for his food to be heated. In response to the recruiting move by James, Ben turns and points into the adjacent room as he recruits James to fulfill the recruitment on his own.}
\includegraphics[width=\textwidth]{figures/KendrickFigure14.jpg}
\label{fig:kendrick:14}
\end{figure}

The request comes as Ben waits for his food to be heated in the microwave in front of him (see \figref{fig:kendrick:14}\textit{a}). In response to the request, Ben rotates his body and gazes and points into an adjacent room (see \figref{fig:kendrick:14}\textit{b}). He confirms verbally that he has an onion for James and then directs him to go into the adjacent room to retrieve it. In principle, Ben could have walked into the adjacent room, retrieved the onion, and given it to James. That is, one solution to the problem of multiple involvements is for a recipient of the recruiting move to abandon a course of action. By recruiting the recruiter, however, Ben neither abandons the course of action he is engaged in nor defers the recruitment until the course of action is complete; he suspends one course of action (the preparation of the meal) in order to recruit James to pursue the second course of action in his stead (cf. Blythe, \chapref{sec:blythe}, \sectref{sec:blythe:4.2.2}).

\section{Discussion}
This chapter has presented a quantitative study of recruitment phenomena identified in a sample of video recordings of everyday interactions among speakers of English. Using an operational definition of recruitment designed for a cross-linguistic comparison (see Chapters 1--2), the present study has documented diverse forms of linguistic and embodied action that participants use to recruit the assistance of others. As we have seen, visible bodily actions were common;  about half of all recruiting moves in the dataset included a relevant visual component. Recruiting assistance through visible bodily action alone, however, was relatively rare. The most frequent visible bodily actions observed in recruiting moves involved manual actions, such as pointing, reaching out, or holding out an object to be taken, which reflects the high proportion of hand-to-hand object transfers in the dataset. Of the grammatical formats observed in recruiting moves, interrogatives were the most frequent and exhibited the greatest variation, with multiple recurrent subformats \citep[see][]{fox_rethinking_2016, foxheinemann2017}. In contrast, imperatives, the second most frequent format, showed relatively little structural variation. Regardless of format, recruiting moves overwhelmingly elicited cooperative responses, a statistical trend that reflects the general preference for agreement in interaction \citep{Sacks1987} and testifies to the fundamentally prosocial nature of human behavior -- one important finding to emerge from the comparative study \citep{FloydEtAl2018}.

The operational definition of recruitment developed for the cross-linguistic comparison, to which the present study is but one contribution, differs in important respects from the articulation of the concept by Kendrick and Drew (\citeyear{KendrickDrew2016}; see also \citealt{DrewKendrick2017}). Our articulation was rooted in the observation that offering and requesting, which had been understood as distinct forms of action, each with its own grammatical formats and sequential environments \citep[see, e.g.,][]{Curl2006, curlcontingency2008}, in fact have an organizationally symbiotic relationship \citep{KendrickDrew2014}. This observation, which itself has antecedents in previous research (\citealt{Schegloff1995}; see also \citealt{Heritage2016}), prompted us to collect and analyze not only requests in all forms, as indeed the present study has done, but also all \textit{offers} and all actions that systematically occasion them, whether delivered through talk or embodied conduct. It became clear that the organizational symbiosis between offering and requesting centers on the recognition of troubles in the realization of practical courses of action and the alternative methods available to participants to resolve them \citep{KendrickDrew2016}. Offers and requests, we observed, differ in two principal ways: who initiates the recruitment of assistance and who generates a possible solution to the trouble. With a request, the one who experiences the trouble, Self, formulates a possible solution for an Other to perform and thereby initiates the assistance (e.g. \textit{move over a little, can you?} in \extref{ex:kendrick:4}). With an offer, it is the Other who formulates a possible solution and initiates the assistance (e.g. \textit{you want \uline{th}at}, \citealt[7]{KendrickDrew2016}). The two actions thus involve an inversion of social relations (e.g. who initiates and who responds) and interactional contingencies (e.g. who generates the solution), yet both constitute methods for the recruitment of assistance. The cross-linguistic comparison, however, has \textit{its} roots an investigation of requesting across cultures, one vestige of which is a somewhat equivocal treatment of offers (\chapref{sec:coding}).

Another difference in the two articulations concerns the use of the term “recruitment” itself. Having conceptualized offers and requests as alternative methods by Self and Other for the resolution of troubles, we came to recognize them as parts in a complex social organization of action, which we termed \textit{the organization of assistance} \citep{KendrickDrew2016}. Recruitment -- that is, one’s having been recruited -- is central to the organization of assistance. But it should not be understood as a category of action, one that somehow subsumes offering and requesting within it. Indeed, it was the very tendency towards the analytic conflation of interactionally distinct actions into categories that we meant to disrupt with the concept of recruitment. As we see it, recruitment does not refer to a type of social action, but rather to an outcome or effect that participants have alternative methods to achieve \citep[2]{KendrickDrew2016}. In the language of speech act theory, recruitment is more akin to a perlocutionary effect \citep{austin_how_1962} than a class of illocutionary force \citep{Searle1976}.

To illustrate the distinction and its ramifications, consider a recurrent source of trouble that routinely impedes practical courses of action and two alternative methods to resolve it. If a course of action requires an object for its completion (e.g. the lighter in \extref{ex:kendrick:1} or the bag in \extref{ex:kendrick:2}), the absence of the object will necessarily disrupt the progressivity of the course of action. Among the set of methods that participants have to resolve such troubles interactionally are requests (e.g. \extref{ex:kendrick:2}) and visible searches of the environment, which may occasion offers of assistance \citep[see][]{DrewKendrick2017}. In both cases, we would say that the Other provides assistance, whether solicited or volunteered, and has therefore been recruited. We would not say, however, that requests for objects and visible searches of the environment are themselves “recruitments”. To do so would conflate analytically distinct forms of action into one conceptual category and thereby obscure the systematic, interactionally-relevant differences between them.

At risk of belaboring the point, consider the ways in which requests for objects differ from visible searches of the environment \citep[cf.][10--11]{KendrickDrew2016}. First, the resources with which the participants construct the actions differ completely. A request specifies a solution to a trouble through linguistic forms or communicative gestures. In contrast, a visible search is in the first instance recognizable as an instrumental visible bodily action. Second, who initiates the sequence that comes to resolve the trouble differs. With a request Self initiates the sequence whereas with a visible search it is Other who does so. Third, the two actions differ in who generates the solution to the trouble. A request for an object formulates a solution for Other to perform whereas a visible search requires that Other recognize the trouble and generate a solution independently. Fourth, how the trouble manifests and hence becomes recognizable differs. With a visible search Self’s actions embody the trouble whereas with a request the nature of the trouble is left implicit. Fifth and finally, the actions also differ in whether they establish a normative obligation for assistance as a response. A request for an object initiates an adjacency pair sequence in which assistance is a conditionally relevant response whereas a visible search for the environment does not.

\hspace*{-.5mm}On what basis, then, could one say that requests for objects and visible searches of the environment are instances of the same action or the same type of action? What unites them is not a similarity of \textit{action}. A participant who searches for an object is not performing the same action, or even the same kind of action, as one who asks a co-participant for it. What unites them is the outcome they may achieve: the recruitment of assistance and the resolution of a trouble. If a category of “recruitments” exists, it is by virtue of this common interactional outcome, not a similarity in the methods participant use to arrive at it. The same argument holds equally for other methods of recruitment. Reports of troubles, for example, are analytically and interactionally distinct from requests \citep{KendrickDrew2016}, and neither should be conflated into a single category of action, though each has its place in the organization of assistance.

Terminology aside, research on recruitment marks a shift of analytic focus away from singular actions (e.g. requests) and theoretical categories of action (e.g. directives). Rather than begin with an action and examine its implementation, research on recruitment begins with a social organizational problem -- how do participants in interaction recognize and resolve troubles that emerge in practical, embodied courses of action? -- and investigates its various solutions, the recruitment of assistance being one. This mode of analysis, which has its roots in classic conversation analytic research on the organization of interaction, is generic and widely applicable to the study of action.

\section{Conventions for multimodal transcription}
Embodied actions are transcribed according to the following conventions developed by \citet{Mondada2014c}.\medskip

\noindent
\begin{tabularx}{\textwidth}{lX}
% {*}{~}{~}{~}{*} & Gestures and descriptions of embodied actions are delimited \\
\verb|*   *| & Gestures and descriptions of embodied actions are delimited \\
% {+}{~}{~}{~}{+} & between ++ two identical symbols (one symbol per participant) and are synchronized with correspondent stretches of talk. \\
\verb|+   +| & between two identical symbols (one symbol per participant) and are synchronized with correspondent stretches of talk. \\
{*}{-}{-}{-}{>} & The action described continues across subsequent lines until the \\                 
{-}{-}{-}{>}{*} & same symbol is reached. \\
{>}{>} & The action described begins before the excerpt’s beginning. \\
{-}{-}{-}{>}{>} & The action described continues after the excerpt’s end.  \\
{.}{.}{.}{.}{.} & Action’s preparation. \\
{,}{,}{,}{,}{,} & Action’s retraction. \\
\end{tabularx}

\begin{tabularx}{\textwidth}{lX}
\textsc{ali} & Participant doing the embodied action is identified when (s)he is not the speaker. \\
{\fig} \#  & The exact moment at which a screen shot has been taken is indicated with a specific sign showing its position within turn at talk. \\
\end{tabularx}
% \todo{add (°°    °°)   to list of abbreviations}

\sloppy
\printbibliography[heading=subbibliography,notkeyword=this]
\end{document}
