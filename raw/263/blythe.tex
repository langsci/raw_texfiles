\documentclass[output=paper,nonflat,colorlinks,citecolor=brown]{langsci/langscibook}
\ChapterDOI{10.5281/zenodo.4018382}
\author{Joe Blythe\affiliation{Department of Linguistics, Macquarie University}}
\title{Recruitments in Murrinhpatha and the preference organization of their possible responses}
\shorttitlerunninghead{Recruitments in Murrinhpatha and the preference organization of responses}
\abstract{This chapter describes the resources that speakers of Murrinhpatha use when recruiting assistance and collaboration from others in everyday social interaction. The chapter draws on data from video recordings of informal conversation in Murrinpatha, and reports language-specific findings generated within a large-scale comparative project involving eight languages from five continents (see other chapters of this volume). The resources for recruitment described in this chapter include linguistic structures from across the levels of grammatical organization, as well as gestural and other visible and contextual resources of relevance to the interpretation of action in interaction. The presentation of categories of recruitment, and elements of recruitment sequences, follows the coding scheme used in the comparative project (see \chapref{sec:coding} of the volume). This chapter extends our knowledge of the structure and usage of Murrinhpatha with detailed attention to the properties of sequential structure in conversational interaction. The chapter is a contribution to an emerging field of pragmatic typology. }
\begin{document}
\maketitle
\label{sec:blythe}

\section{Introduction}\label{sec:blythe:1}

This chapter presents a first survey of recruiting moves and their responses in informal face-to-face conversation conducted in the Australian Aboriginal language Murrinhpatha. I begin by introducing the language and its speakers, and by discussing the corpus that informs this collection. In \sectref{sec:blythe:2} I then illustrate some basic recruitment sequences and present the recruitment subtypes that we consider in the larger comparative project. In \sectref{sec:blythe:3}, I present the formats used as recruiting moves, while in \sectref{sec:blythe:4} I present the formats used as responses. The survey reveals a hierarchically governed array of responses, including structurally preferred compliant responses, as well as a range of dispreferred refusal formats, which either overtly or implicitly reject the recruitment. In \sectref{sec:blythe:6} I discuss the possible effects of social asymmetry on recruitments in Murrinhpatha before concluding the chapter in \sectref{sec:blythe:7}.

\subsection{The Murrinhpatha language}\label{sec:blythe:1.1}

Murrinhpatha is an indigenous regional lingua franca spoken by approximately 2700 people in Wadeye, Nganmarriyanga and in various smaller communities within the Fitzmaurice and Moyle Rivers region of Australia’s Northern Territory (see \figref{fig:blythe:1}). It is spoken by people affiliated to the Murrinhpatha, Marri Ngarr, Marri Tjevin, Marri Amu, Magati Ke, Ngan’gityemerri and Jaminjung languages who, prior to the 1940s and 50s, would have been multilingual hunter-gatherers. Today all Aboriginal people in this region speak Murrinhpatha natively on a daily basis. It is one of only 18 traditional Australian languages still being acquired by children \citep[3]{aiatsis_national_2005}. Until they encounter English at school, most children in Wadeye grow up as monolingual Murrinhpatha speakers (\citealt{kelly_indigenous_2010}; \citealt{Forshaw-et-al2017}).

\begin{figure}
\includegraphics[width=\textwidth]{figures/murrinhpatha-img1.png}
\caption{
The Fitzmaurice and Moyle Rivers region of Australia's Northern Territory.
\label{fig:blythe:1}
}\end{figure}

Murrinhpatha is a polysynthetic, headmarking language with grammaticalized kinship inflections. Its verbal morphology is templatic \citep{Nordlinger2010b}. Complex predicates are comprised of bipartite stems, often consisting of discontinuous morphs. Nominal entities are classifiable in terms of ten semantically transparent noun classes, which do not form the basis for verbal agreement.

Previous research has described the language’s genetic status \citep{Green2003}, its complex polysynthetic verbal morphosyntax (\citealt{%
Walsh1976,Walsh1996,Walsh1987,
Street1980,Street1987,
Blythe2009,Blythe2010a,Blythe2013,
Nordlinger2010a,Nordlinger2010b,
Mansfield2014b,
Forshaw2016,Forshaw-et-al2017}), the system of nominal classification (\citealt{Walsh1993,Walsh1997}), syntax (\citealt{Nordlinger2011,Mujkic2013}), the marking of tense, aspect and mood categories (\citealt{Nordlinger2012}), and the kinship system \citep{Blythe2018}. Interactional research has investigated person reference (\citealt{Blythe2009,Blythe2010b,Blythe2013}), spatial reference \citep{Blythe2016}, teasing \citep{Blythe2012}, and other-initiated repair \citep{Blythe2015}.

\subsection{Data collection and corpus}\label{sec:blythe:1.2}

Of the seventeen Murrinhpatha interactions sampled in this study, thirteen were collected by the author between 2007 and 2012 and four were collected in 2012 by John Mansfield. The recordings were made either in the communities of Wadeye, Nganmarriyanga, or on the estates of one of the local clan groups. From 3.5 hours of transcribed Murrinhpatha conversation 145 recruitments were sampled.


Most of the recordings were made on picnics in the bush, away from the noisy community of Wadeye. For this reason many of the recruitments under examination relate to procurement of cigarettes or tobacco, or to the production of billy tea. They are generally low cost, low contingency requests for imminent action. In accordance with the guidelines of the project (see \chapref{sec:intro}, \sectref{sec:intro:4}) higher contingency requests for more distant future action were excluded from the collection.

\section{Basics of recruitment sequences}\label{sec:blythe:2}

As defined in \chapref{sec:intro}, \sectref{sec:intro:4}, a recruitment is a basic cooperative phenomenon in social interaction consisting of a sequence of two moves with the following characteristics:

\begin{description}
\item[Move A:] participant A says or does something to participant B, or that B can see or hear;
\item[Move B:] participant B does a practical action for or with participant A that is fitted to what A has said or done.
\end{description}

Such sequences encompass requests for objects or other services as well as directives to move or modify behavior. They also include actions that occasion assistance or collaboration without necessarily having been produced with the intention to elicit that effect.


The basic minimal sequence will be illustrated below in \sectref{sec:blythe:2.1} while non-minimal sequences will be discussed in \sectref{sec:blythe:2.2}. The subtypes will be elaborated in \sectref{sec:blythe:2.3}. In the transcripts, ▶ and ▷ designate Move A and Move B respectively.

\subsection{Minimal recruitment sequence}\label{sec:blythe:2.1}

\extref{ex:blythe:1} exemplifies a minimal recruitment sequence. The initial move by Mary is multimodally packaged as a composite utterance (\citealt{Kendon2004,Enfield2009}). The second person singular imperative verb \textit{nangamutkathu} in line 2 is accompanied by eye-gaze toward Lily, directing her to ‘give \{something\} here to me’. The vaguely expressed entity of the vegetable \textit{mi}{}-class is minimally specified by the accompanying gesture. Mary’s outstretched hand is open, ready to receive an item small enough to be passed by hand. This is inferable as either tobacco or a tobacco product. When ready, Lily passes Mary a \textit{larrwa}, a conical tobacco pipe, packed with tobacco (\textit{mi beka,} line 4).

\transheader{ex:blythe:1}{Da Ngarne 20091121JBvid03\_906530\_915256}\vspace{-1mm}
%
\begin{mdframednoverticalspace}[style=firstfoc]
\begin{transbox}{1}{mar}
\begin{verbatim}
[°ya mi nangamutkathuya;°                             ]
 ya  mi     na                -nga   -mut -gathu =ya
 HES NC:VEG 2SG.S.poke(19).FUT-1SG.IO-give-hither=CL
\end{verbatim}
\hspace{0.07cm} ah, give me some vegetable class stuff
\end{transbox}
\end{mdframednoverticalspace}
%
\begin{mdframednoverticalspace}[style=firstfoc]
\begin{transbox}{2}{~}
\begin{verbatim}
[     ((reaches out to Lily with an open hand))       ]
\end{verbatim}
\end{transbox}
\end{mdframednoverticalspace}
%
\emptytransbox{3}{(4.7)}\vspace{-2mm}
%
\begin{mdframednoverticalspace}[style=secondfoc]
\begin{transbox}{4}{lil}
\begin{verbatim}
((passes conical smoking pipe to Mary))
\end{verbatim}
\end{transbox}
\end{mdframednoverticalspace}

This canonical minimal sequence consists of an initiating move (Move A, M-A) by participant A and a responsive move (Move B, M-B) by participant B. These canonical minimal sequences form the building blocks for non-minimal sequences.

\subsection{Non-minimal recruitment sequence}\label{sec:blythe:2.2}

Interactionally, non-minimal sequences are less straightforward than the minimal sequences. Usually their non-minimality is brought about because the initiating move is problematic, because the expected responding move is not easily complied with, or because the recruitee is either unable to fulfill or is reticent about fulfilling the recruitment.

The non-minimal sequences are numerous and varied in type. In some sequences, the responsive move (M-B) becomes an initiating move for a subsequent sequence, as a counter or deflected sequence (see \sectref{sec:blythe:4.2.2}). This sequence might also be non-minimal. In other non-minimal sequences, the expected responsive move (M-B) does not eventuate and participant A pursues a response by reissuing, modifying, or elaborating upon the prior move (M-A\textsubscript{2}). Alternatively (or additionally), there may be contingencies to be attended to by participant B before the responsive move can be produced. Thus, before committing to comply with a request, the recruitee might need convincing that s/he is capable of performing the requested action. This is exemplified in \REF{ex:blythe:2}.

The three young men in \REF{ex:blythe:2} speak very little English and have few dealings with white people. Because they have no tobacco, one of them, Dave, tries to encourage Dom to procure some from a white man standing nearby.

\transheader{ex:blythe:2}{Ngandimeli 20120715\_JB\_video\_GYHM100\_02\_51660\_68736}\vspace{2mm}
%
\begin{transbox}{1}{dav}
\begin{verbatim}
ngawu!
\end{verbatim}
hey!
\end{transbox}
%
\begin{transbox}{2}{~}
\begin{verbatim}
(1.2)
\end{verbatim}
\end{transbox}
%
\begin{mdframednoverticalspace}[style=firstfoc]
\begin{transbox}{3}{dav}
\begin{verbatim}
tjewirndurt thanadharrpu [mi beka ngarra ku karrim pangu warri ]
tje                  -wirndurt tha               -rna     -dharrpu
2SG.S.poke.RR(21).FUT-arise    2SG.S.poke(19).FUT-3SG.M.IO-ask
mi     beka    ngarra ku     karrim               pangu warri
NC:VEG tobacco LOC    NC:ANM 3SG.S.stand(3).EXIST DIST  Fa/So
\end{verbatim}
son get up and ask the white bloke standing there for tobacco
\end{transbox}
\end{mdframednoverticalspace}
%
\begin{mdframednoverticalspace}[style=firstfoc]
\begin{transbox}{4}{~}
\begin{verbatim}
                         [((turns head, lip-points)) Figure 2  ]
\end{verbatim}
\end{transbox}
\end{mdframednoverticalspace}
%\figref{fig:blythe:2}
\begin{transbox}{5}{dom}
\begin{verbatim}
                         [((turns head to follow Dave's gaze)) ]
\end{verbatim}
\end{transbox}
%
\begin{transbox}{6}{~}
\begin{verbatim}
(0.4)
\end{verbatim}
\end{transbox}
%
\begin{mdframednoverticalspace}[style=firstfoc]
\begin{transbox}{7}{dav}
\begin{verbatim}
narnawu:; (0.6) manitjpirr charge up ngamanu
na              -rna      mani-dhatjpirr charge_up ngama            -nu
2SG.S.say(8).FUT-3SG.M.IO like-INTS      recharge  1SG.S.say(34).FUT-FUT
\end{verbatim}
tell him something like “I'll become more lively
\end{transbox}
\end{mdframednoverticalspace}
%
\begin{mdframednoverticalspace}[style=firstfoc]
\begin{transbox}{8}{~}
\begin{verbatim}
mi ngurduwinungi kardamatha (.) mangini pirditjme ngengerrennimenu.
mi     ngurdu             -wi   -nu =ngi               kardamatha
NC:VEG 1SG.S.shove(29).FUT-smoke-FUT=1SG.S.sit(1).FUT  right_here
mangini pirditjme  nge                -ngerren    -neme     -nu
similar long_time  1DC.EX.S.sit(1).FUT-be_speaking-PC.M.NSIB-FUT
\end{verbatim}
as I sit here smoking (.) and thus we'll be able to sit and talk for ages”
\end{transbox}
\end{mdframednoverticalspace}
%
\begin{transbox}{9}{~}
\begin{verbatim}
(0.4)
\end{verbatim}
\end{transbox}
%
\begin{mdframednoverticalspace}[style=firstfoc]
\begin{transbox}{10}{dav}
\begin{verbatim}
[kardu pa↑tha::;      ]
 kardu    patha-wa
 NC:HUMAN good –EMPH
\end{verbatim}
% \hspace{0.07}
\{he's a\} good bloke!
\end{transbox}
\end{mdframednoverticalspace}
%
\begin{mdframednoverticalspace}[style=firstfoc]
\begin{transbox}{11}{~}
\begin{verbatim}
[((points with thumb))]
\end{verbatim}
\end{transbox}
\end{mdframednoverticalspace}
%
\begin{transbox}{12}{~}
\begin{verbatim}
(0.2)
\end{verbatim}
\end{transbox}
%
\begin{transbox}{13}{dom}
\begin{verbatim}
I want- (0.2) give me smoke (.) I you:: (0.4) [(fix one and)] story.
\end{verbatim}
\end{transbox}
%
\begin{transbox}{14}{dav}
\begin{verbatim}
                                              [(           )]
\end{verbatim}
\end{transbox}
%
\begin{transbox}{15}{dav}
\begin{verbatim}
[yu.
\end{verbatim}
\hspace{0.07cm} yeah.
\end{transbox}
%
\begin{mdframednoverticalspace}[style=secondfoc]
\begin{transbox}{16}{dom}
\begin{verbatim}
[((gets up to go))
\end{verbatim}
\end{transbox}
\end{mdframednoverticalspace}

\begin{figure}
\includegraphics[width=\textwidth]{figures/murrinhpatha-img2.jpg}
\caption{Dom lip-points toward the white man off-screen who has tobacco (\extref{ex:blythe:2}, line 4).
\label{fig:blythe:2}}\end{figure}

Having secured Dom’s recipiency with a summons (line 1), Dave lip-points toward the white man nearby (line 4) and tells Dom to get up and ask him for tobacco (line 3, M-A\textsubscript{1}). When Dom does not move after 0.4 seconds, he adds that he should provide the following rationale for providing tobacco: namely, that the boys will be revitalized and able to sit and talk for much longer (lines 7 and 8, M-A\textsubscript{2}). When Dom (although smiling) still does not move after 0.4 seconds, Dave reassures him in line 10 that the white man is a good bloke (\textit{kardu pathawa}, M-A\textsubscript{3})\footnote{The nominal \textit{kardu} class ordinarily pertains to Aboriginal people who can be related to as actual or classificatory kin. Non-Aboriginal people (social outsiders, effectively) are ordinarily  grouped with animates in the nominal \textit{ku}{}-class (\citealt{Walsh1997}; \citealt{Blythe2015}). Dave’s initial reference to the white man at line 3 is with the \textit{ku} classifier (\textit{ku karrim pangu}, approximately ‘the non-Aboriginal standing over there’). In the subsequent reference at line 10 Dave refers to him as \textit{kardu patha}, literally ‘good Aboriginal person’. The shift in classifier signals a pragmatic construal of the erstwhile alien as, for all intents and purposes, \textit{kardu darrikardu} ‘a fellow countryman’, and thus as someone who can effectively be coerced into providing tobacco.\\
}.  Before complying with the request, at line 13 Dom rehearses what he will say to the white man in English. As Dave ratifies this rehearsal as adequate (line 15), Dom fulfills the request (line 16, M-B) by getting up to go and ask. Here the contingencies -- what exactly to say to the white man in a language he seldom uses -- are dealt with before the responsive move is enacted.

A handful of sequences can be considered non-minimal because they consist of an initial move by participant A (M-A) followed by two responsive moves by participant B. The first of these responsive moves (M-B\textsubscript{1}) expresses B’s commitment to fulfill the recruitment, whereas the second (M-B\textsubscript{2}) constitutes the actual fulfillment. We will encounter two of these three-move sequences below in \REF{ex:blythe:14}.

\subsection{Subtypes of recruitment sequence}\label{sec:blythe:2.3}

As outlined in \chapref{sec:coding}, \sectref{sec:coding:6}, recruitments mobilize a range of cooperative actions, which can be broadly categorized as being of four subtypes: i) service provision, i.e. performing a practical task for or with someone, ii) object transfer, i.e. giving someone an object, iii) alteration of trajectory, i.e. changing or stopping one’s behavior, and iv) trouble assistance, i.e. stepping in to help someone in response to current or anticipatable trouble. \tabref{tab:blythe:1} shows the relative proportions of the various recruitment subtypes within the Murrinhpatha corpus.

\begin{table}
\begin{tabularx}{.75\textwidth}{Xrr}
\lsptoprule
Recruitment subtype & Count & Proportion\\
\midrule
Service provision & 110 & 76\%\\
Object transfer & 41 & 28\%\\
Alteration of trajectory & 21 & 14\%\\
Troubles assistance & 0 & 0\%\\
\lspbottomrule
\end{tabularx}
\caption{Relative proportions of recruitment subtypes (\textit{n}=145).}
\label{tab:blythe:1}
\end{table}

Service provision and object transfer sequences have already been exemplified in \REF{ex:blythe:2} and \REF{ex:blythe:1} respectively. The Murrinhpatha corpus contained no offers of assistance for evident trouble, possibly because all of the recordings were made outdoors in the open, rather than confined indoors where people may need, for example, to make way for each other (cf. Enfield, \chapref{sec:enfield}, \sectref{sec:enfield:3.3}). \extref{ex:blythe:3} illustrates an alteration of trajectory recruitment. In this case the recruitee is exhorted to not cease an activity she was already engaged in.

\transheader{ex:blythe:3}{Dingalngu 20110730\_JB\_video\_GYHM100\_04\_253128}\vspace{2mm}
%
\begin{transbox}{1}{lau}
\begin{verbatim}
((stares behind Maggie's ear))
\end{verbatim}
\end{transbox}
%
\begin{transbox}{2}{~}
\begin{verbatim}
((reaches into Maggie's hair))       1.9 sec
\end{verbatim}
\end{transbox}
%
\begin{transbox}{3}{~}
\begin{verbatim}
((stops and scratches her own head)) 2.7 sec
\end{verbatim}
\end{transbox}
%
\begin{mdframednoverticalspace}[style=firstfoc]
\begin{transbox}{4}{mag}
\begin{verbatim}
awu kuka mere nawey- (.)
awu kuka       mere nawey
no  NC:ANM-TOP NEG  STRI
\end{verbatim}
no! don't sto-
\end{transbox}
\end{mdframednoverticalspace}
%
\begin{mdframednoverticalspace}[style=firstfoc]
\begin{transbox}{5}{~}
\begin{verbatim}
[nangiwewaytji  [kuka tjirrangiwertirt weyida.
 na               -ngi  -weway       =tji
 2SG.S.GRAB(9).FUT-1S.DO-examine_hair=2SG.S.SIT(1).FUT
 ku    -ka  tjirra             -ngi  -wertirt-weyida
 NC:ANM-TOP 2SG.S.WATCH(28).FUT-1S.DO-delouse-continue
\end{verbatim}
\hspace{0.07cm} keep on looking in my hair for lice
\end{transbox}
\end{mdframednoverticalspace}\vspace{1mm}
%
\begin{transbox}{6}{ali}
\begin{verbatim}
[hm  hm  hm  hm [ha  ha  ha  ha
\end{verbatim}
\end{transbox}
%
\begin{mdframednoverticalspace}[style=secondfoc]
\begin{transbox}{7}{lau}
\begin{verbatim}
                [((resumes searching for lice))
\end{verbatim}
\end{transbox}
\end{mdframednoverticalspace}
%
\begin{transbox}{8}{kar}
\begin{verbatim}
yu
\end{verbatim}
yeah
\end{transbox}
%
\begin{transbox}{9}{~}
\begin{verbatim}
(0.5)
\end{verbatim}
\end{transbox}\\
%8 in original chapter

In \REF{ex:blythe:3} Laura (at line 1) appears to notice something behind Maggie’s ear (presumably, a louse) so reaches into her hair (line 2) to search for it. At line 3 she stops reaching and scratches her own head.  At line 4 Maggie begins a negatively framed recruiting turn that is truncated midway through the verb. The negative morphosyntactic framing is replaced in self-repair by a positively framed recruiting component which exhorts Laura to continue searching for the louse.\footnote{The repairable is not easy to translate. In all likelihood, the animate \textit{ku} classifier is used to evoke the louse. The negatively framed repairable appears to have been shooting for something like ‘don’t stop searching for the critter’.} Laura resumes the search (line 7) before Maggie has even finished articulating her recruiting turn.

\section{Formats in Move A: The recruiting move}\label{sec:blythe:3}

In multiparty interaction, two key dimensions of recruitments are the question of who is being recruited, and how that person comes to recognize what they are being recruited for. The successful recruiting move must address both the person-selection dimension \citep{Lerner2003} and the action ascription dimension \citep{Levinson2013}. These dimensions can be separately handled through the visuo-corporal modality, through the audio-vocal modality, or jointly handled through both as a composite, multimodal utterance. The \textit{move} is the fundamental unit of social action within interaction (\citealt{Enfield2009}; \citealt{Goffman1981}). This semiotically rich unit is more often than not multimodal, that is, comprises verbal and kinesic components (see also \citealt{Kendon2004}). In this paper, both kinesic behavior and spoken behavior are represented in the transcripts. I will be considering both person-selection and action ascription dimensions of the recruitment, as well as functional distinctions between the various forms of the recruiting moves.

\subsection{Nonverbal behavior in recruiting moves}\label{sec:blythe:3.1}

Of the 145 recruiting moves in the collection, 92 (63\%) have a seemingly relevant kinesic component. These nonverbal components include pointing, reaching out a hand to receive an object, holding out an object for a recipient to take, as well as iconic and conventionalized gestures.

Eye gaze and/or body torque toward the targeted recruitee can be critical in achieving the person-selection dimension of recruitment.  Thus in \REF{ex:blythe:1} Mary manages the person selection issue by gazing toward Lily and reaching her arm out in her direction.\footnote{The same arm also manages aspects of the action-ascription dimension. The open hand is ready to receive a small passable object.  \\
} Other examples where person selection is successfully managed through eye gaze and physical embodiment include \REF{ex:blythe:4}, \REF{ex:blythe:13}, \REF{ex:blythe:14}, \REF{ex:blythe:16} and \REF{ex:blythe:17}.

\subsection{Fully nonverbal recruiting moves}\label{sec:blythe:3.2}

Six of the 92 recruiting moves incorporating kinesic components were delivered entirely without speech. \extref{ex:blythe:4} exemplifies this phenomenon.

\transheader{ex:blythe:4}{Thuykem2011 0901\_JB\_video\_GYHM100\_02}\vspace{2mm}
%
\begin{transbox}{1}{dav}
\begin{verbatim}
[kigay matha purrunimenu marnanu. kigay damatha purrunimenu.]
 kigay     matha purru    -nime     -nu  ma           -rna     -nu
 young_men INTS  1NS.INC.S-PC.M.NSIB-FUT 1SG.S.SAY(34)-3SG.M.IO-FUT
 kigay     damatha purru    -nime     -nu
 young_men INTS    1NS.INC.S-PC.M.NSIB-FUT
\end{verbatim}
\hspace{0.07cm} “we boys will go”, I'll tell him, “we'll go”
\end{transbox}
%
\begin{transbox}{2}{bru}
\begin{verbatim}
[            ((pours tea into his own cup))                 ]
\end{verbatim}
\end{transbox}
%
\begin{mdframednoverticalspace}[style=firstfoc]
\begin{transbox}{3}{bru}
\begin{verbatim}
((rubs fingers together, Figure 3))
\end{verbatim}
\end{transbox}
\end{mdframednoverticalspace}
%
\begin{mdframednoverticalspace}[style=secondfoc]
\begin{transbox}{4}{dav}
\begin{verbatim}
((passes Bruce the spoon))
\end{verbatim}
\end{transbox}
\end{mdframednoverticalspace}\vspace{-2mm}
%
\begin{transbox}{5}{~}
\begin{verbatim}
nakurlu kardu::: (0.9) femili ngamanu pigarrkatngime.
nakurl-nu  kardu    femili ngama            -nu
later –FUT NC:HUMAN family 1SG.S.SAY(34).FUT-FUT
pi               -garrkat-ngime
1INC.S.sit(1).FUT-??     -PC.F.NSIB
\end{verbatim}
later they::: (0.9) I'll tell the family “we'll go”
\end{transbox}\\
%
\begin{figure}
\includegraphics[width=\textwidth]{figures/murrinhpatha-img3.jpg}
\caption{Bruce, gazing at the spoon in Dave’s cup, rubs his index- and middle-finger against his thumb (\extref{ex:blythe:4}, line 3).
\label{fig:blythe:3}
}\end{figure}

In \REF{ex:blythe:4} Dave’s left hand holds a cup of tea that is sitting on the ground. The cup has a spoon in it. At line 1 Dave is announcing his intention to convey a message in the future to somebody who is not present. As he does this, Bruce fills his cup with tea (line 2). When he finishes this, he turns to face Dave and rubs his index- and middle-finger against his thumb (line 3, \figref{fig:blythe:3}). This abstract (and perhaps conventionalized) gesture is at least partly indexical in that it is oriented toward the spoon in Dave’s cup – as is Bruce’s eye gaze.  Dave pauses as he passes the spoon to Bruce (line 4), and then resumes his announcement (line 5). The momentarily suspended lexico-syntactic channel belies no evidence for there even being a recruitment, as this sequence takes place entirely within the visuo-corporal modality.\footnote{However, the recruitment sequence is at least partly evidenced by prosodic lengthening of the ‘human’ classifier \textit{kardu}::: followed by the 0.9s of  silence in line 5. This combination does suggest possible nonverbal activity.\\
} Bruce manages the person-selection dimension of the recruitment by twisting his body and gaze toward Dave (and the spoon in his possession) and away from co-present Phil.


\subsection{Verbal elements: construction types and subtypes}\label{sec:blythe:3.3}

In this section we consider the various grammatical structures that best characterize the verbal components of recruiting moves. In addition to the three basic sentence types, imperative, declarative and interrogative, we also consider those that lack a predicate altogether. The relative proportions of these construction types are given in \tabref{tab:blythe:2}.

\begin{table}
\begin{tabularx}{.66\textwidth}{Xrr}
\lsptoprule
Construction type & Count & Proportion\\
\midrule
Imperative & 67 & 48\%\\
No predicate & 46 & 33\%\\
Declarative & 13 & 9\%\\
Interrogative & 4 & 3\%\\
\lspbottomrule
\end{tabularx}
\caption{Proportions of construction types in the sample that include a verbal component (\textit{n}=139).}
\label{tab:blythe:2}
\end{table}

\subsubsection{Imperatives}\label{sec:blythe:3.3.1}

Imperative constructions are the most frequent of the verbal components of recruiting moves. Because they explicitly name the action to be performed and because the grammatical form of the predicate indexes the elicitation of action (\citealt[774–78]{Lyons1977}; \citealt[170–71]{SadockZwicky1985}), in this collection they are the most overt, on-record method for recruiting action. Those that have second singular subjects (the majority) are used to single out the person being recruited. The imperative mood is morphologically distinguished from other moods. Both future and past indicative, as well as future and past irrealis moods are double-marked within the template of the polysynthetic verb; firstly in the initial portmanteau classifier stems, and secondly in a morphological slot dedicated to marking TAM distinctions. This is not the case, however, with imperatives. In the imperative mood, the dedicated TAM slot remains unfilled (\citealt{Nordlinger2012}).

We have already encountered second person singular imperatives in \REF{ex:blythe:1} (\textit{nangamutkathuya}, line 1) and \REF{ex:blythe:2} (\textit{tjewirndurt} and \textit{thanadharrpu}, line 3, and \textit{narna}, line 7). \tabref{tab:blythe:3} compares the imperatives \textit{tjewirndurt} and \textit{thanadharrpu} to their future indicative counterparts. The imperatives lack the future tense morpheme \textit{-nu} that otherwise appears within future indicatives.

\begin{table}
\let\eachwordone=\footnotesize
\let\eachwordtwo=\footnotesize
\let\eachwordthree=\footnotesize
\begin{tabularx}{\textwidth}{l@{~}l}
\lsptoprule
Imperative & Future Indicative\\
\midrule
\parbox{5cm}{
\glll tje-                                                       wirndurt\\
      2\textsc{sg}.\textsc{s}.poke.\textsc{rr}(21).\textsc{fut}- arise\\
    {}[CS]-                                                     [LS]\\
\glt\footnotesize `stand up'
}
&
\parbox{6cm}{
		    \glll tje                  {}-wirndurt -nu\\
			  2\textsc{sg}.\textsc{s}.poke.\textsc{rr}(21).\textsc{fut} -arise   -\textsc{fut}\\
	           	  {}[CS]                                                             -[LS]    -[TAM]\\
                    \glt\footnotesize `you will stand up'
}\\
\midrule
\parbox{5.5cm}{
\glll tha                                     -rna                                 -dharrpu\\
2\textsc{sg}.\textsc{s}.poke(19).\textsc{fut} -3\textsc{sg}.\textsc{m}.\textsc{io} -ask\\
[CS]              {}-[Obj]   {}-[LS] \\
\glt\footnotesize `ask him'
}
&
\parbox{8cm}{
			\glll  tha                                           -rna                                 -dharrpu -nu\\
  			       2\textsc{sg}.\textsc{s}.poke(19).\textsc{fut} -3\textsc{sg}.\textsc{m}.\textsc{io} -ask     -\textsc{fut}\\
			       {}[CS]                                        -[Obj]                               -[LS]    -[TAM]\\
			\glt\footnotesize `you will ask him’
}\\
\lspbottomrule
\end{tabularx}
\caption{Imperative forms compared with their future indicative counterparts; future indicative forms are doubly marked for future tense.}
\label{tab:blythe:3}
\end{table}


While the participation framework evoked by an imperative predicate with a second person singular subject will convey that specific addressing is being performed, recipients’ identification of the intended target, within a multiparty setting, hinges on the particular person selection devices which accompany the predicate.\footnote{\cite[182]{Lerner2003} suggests that the second person pronoun \textit{you} is a “recipient indicator” but not a “recipient designator”. “[S]peakers can indicate that they are addressing a specific participant in a manner that does not itself reveal who that individual is" (ibid: 183). In multiparty interaction, who specifically is being addressed through the use of the pronoun is managed through eye gaze or some other device, or inferentially when epistemic or deontic advantage is skewed toward a particular individual. However, as discussed in fn. \ref{fn:blythe:7}, the inferences to be drawn from the English pronoun \textit{you} differ from its Murrinhpatha counterparts.} In \REF{ex:blythe:1} Mary’s eye gaze and outstretched arm toward Lily serves to select Lily as the proper recipient for the 2SG.S inflected predicate \textit{nangamutkathuya} ‘give it here to me’. In \REF{ex:blythe:2} the kinterm \textit{warri} ‘father’/‘son’ (line 1)\footnote{The “kinterm” \textit{warri}, a recent innovation, is mainly used by young men. In Australian kinship systems it is very unusual for reciprocal kinterms (e.g. terms like \textit{cousin} which apply equally in both directions, unlike \textit{father} and \textit{son}) to be used for persons separated by a single generation – although exceedingly common for two generations of separation. \textit{Warri} may be a reanalyzed borrowing from the interjection \textit{warriwarri}, which in the East Kimberley region of Western Australia and the Victoria River district of the Northern Territory (in the Jaru, Gija and Gurindji languages, amongst others), is produced as a sympathetic response by recipients who hear mention of a certain kinsman. In these languages, the term is used for fathers, sons, and other kin besides. Under similar circumstances, contrasting interjections are used for different classes of kin (\citealt[99]{Mcconvell1982}; Blythe, Gija \& Jaru fieldnotes 2016).} serves to select Dom and not co-present Bruce (Dave’s classificatory brother) as the intended target for the recruitment, and as the addressee for the 2SG.S inflected predicates \textit{tjewirndurt} ‘get up’ and \textit{thanadharrpu} ‘ask him’ in line 3. However, when recruiters are unconcerned about who specifically should fulfill the recruitment, the second person imperative predicate will have a non-singular subject.

\transheader{ex:blythe:5}{Thuykem20110901\_JB\_video\_GYHM100\_01\_381373}\vspace{-1mm}
%
\begin{mdframednoverticalspace}[style=firstfoc]
\begin{transbox}{1}{dav}
\begin{verbatim}
[puy nangkarnuwardangu kura tiyu.
 puy   na                 -ngkarnu-warda-wangu    kura     ti -yu
 go_on 2DU.SIB.S.go(6).FUT-mix_up –TEMP -thither  NC:WATER tea-CL
\end{verbatim}
\hspace{0.07cm} go on, you two brothers, mix up some tea
\end{transbox}
\end{mdframednoverticalspace}
%
\begin{mdframednoverticalspace}[style=firstfoc]
\begin{transbox}{2}{~}
\begin{verbatim}
[((points at billycan))
\end{verbatim}
\end{transbox}
\end{mdframednoverticalspace}
%
\begin{transbox}{3}{phi}
\begin{verbatim}
[((removes his cap))
\end{verbatim}
\end{transbox}\vspace{-1mm}
%
\begin{mdframednoverticalspace}[style=secondfoc]
\begin{transbox}{4}{~}
\begin{verbatim}
((uses cap to take the hot billycan off the fire))
\end{verbatim}
\end{transbox}
\end{mdframednoverticalspace}

The three boys in \REF{ex:blythe:5} are classificatory brothers. In line 1 Dave exhorts his brothers to make some tea. The imperative verb \textit{nangkarnuwardangu} is inflected as second person dual sibling: ‘you two siblings go on and mix it up’. The non-specific second person dual sibling subject is not accompanied by a vocative. As Dave initiates the recruitment he points at the billycan on the fire (line 1). He does not gaze at either of his two brothers. Thus, specifically which brother should concern himself with making the tea is left up to them to decide upon.\footnote{In Murrinhpatha, second person predicates are marked for number (SG/DU/PC/PL), and (when DU or PC) gender (M/F), as well as siblinghood (siblings/non-siblings). In English however, the pronoun \textit{you} is unmarked for number, or any other contrasts. This gives the languages different inferential affordances within in multiparty interaction. Upon hearing a Murrinhpatha inflected predicate with a 2SG subject, recipients can infer that the speaker \textit{definitely} has as specific addressee in mind; whereas the English pronoun \textit{you} conveys that the speaker \textit{perhaps} has a specific individual in mind – except when the participation frame is dyadic. The converse is also true for Murrinhpatha. When a second person predicate is \textit{not} singular, then the inference to be drawn is that the speaker is \textit{not} singling out any specific individual from the group of addressed recipients. Dave's gaze at the billycan, rather than at either of the two brothers, accords with the inference of \textit{non-specificity} to be drawn from dual inflected subject.\label{fn:blythe:7}} Actually, while Dave is speaking at line 1, Phil has apparently already taken it upon himself to make the tea. At line 3 he removes his cap, which at line 4 he uses to insulate his hand as he removes the hot billycan from the fire. He then goes on to make the tea.


\subsubsection{No predicate}

Because, as the name suggests, the “no predicate” recruiting moves lack a predicate that expresses the action being elicited from the recruitee, they constitute a grab-bag mixture of structural possibilities. This category includes examples in which the sole lexical content is either an interjection or a vocative devoted to managing the person-selection dimension of the recruitment, leaving the action-ascription dimension to be handled gesturally or through inference. More often however, with object-transfer requests, the object required is explicitly mentioned, as in \REF{ex:blythe:6}.

\transheader{ex:blythe:6}{Nanthak2011 0828\_JB\_video\_GYHM100\_03\_472600\_479711}\vspace{2mm}
%
\begin{transbox}{1}{kar}
\begin{verbatim}
ay kuraka djiwa karrimbuk[tharr.
ay kura    -ka  djiwa karrim            -buktharr
oh NC:WATER-TOP that  3SG.S.stand(3)NFUT-be_red
\end{verbatim}
ay that tea is too strong ((too red))
\end{transbox}
%
\begin{mdframednoverticalspace}[style=firstfoc]
\begin{transbox}{2}{ali}
\begin{verbatim}
                         [yawu munak [kura path- pathayu]=
                          yawu munak  kura     STRI patha=yu
                          hey  sister NC:WATER STRI good =CL
\end{verbatim}
\hspace{11.9em} hey sis, fresh water,
\end{transbox}
%
\begin{transbox}{3}{~}
\begin{verbatim}
                                     [ ((points to car))]
\end{verbatim}
\end{transbox}
%
\begin{transbox}{4}{~}
\begin{verbatim}
=murruwurlnyingka
 murruwurl-nyi   -ngka
 beautiful-2SG.DO-eye/face
\end{verbatim}
\hspace{0.07cm} beautiful face
\end{transbox}
\end{mdframednoverticalspace}\vspace{1mm}
%
\begin{transbox}{5}{~}
\begin{verbatim}
(0.7)
\end{verbatim}
\end{transbox}\vspace{-1mm}
%
\begin{mdframednoverticalspace}[style=secondfoc]
\begin{transbox}{6}{kar}
\begin{verbatim}
ma Rita ma nyinyirda tjewirndurttharra
ma  Rita  ma  nyinyirda tje                  -wirndurt-tharra
but ♀name but ANAPH     2SG.S.POKE.RR(21).FUT-arise   -ahead
\end{verbatim}
hey Rita, you get up for it
\end{transbox}
\end{mdframednoverticalspace}\bigskip

In \REF{ex:blythe:6} when Karen complains that the tea she is making is too strong (line 1), Alice, addressing her as \textit{munak} ‘sister’, points to the car nearby and names the item required to solve the problem (\textit{kura patha}, ‘fresh water’) and mitigating the illocutionary force of the directive with the compliment \textit{murruwurlnyingka} ‘you are beautiful’ (line 4). Karen rejects the recruitment by deflecting it toward a somewhat younger woman (line 6).

Thirty percent of the no-predicate verbal recruiting moves (\textit{n}=14/46) we can call “nominal-hither” constructions. These are exclusively used for object transfer recruitments. In these expressions an overt noun phrase is used to refer to the item being requested. The first element in the majority of Murrinhpatha noun phrases is the nominal classifier applicable to the relevant noun class. The nominal classifier may be followed by a noun, an adjective, a demonstrative and/or a numeral. However most Murrinhpatha noun phrases are under-elaborated: as bare nouns, as bare nominal classifiers, or as the nominal classifier plus a noun/adjective/demonstrative or numeral. If an item is being requested, eye gaze toward the desired item makes the targeted referent reasonably clear. \extref{ex:blythe:7} exemplifies.

\transheader{ex:blythe:7}{Nanthak2011 0828\_JB\_video\_GYHM100\_03\_879400}\vspace{2mm}
%
\begin{transbox}{1}{lil}
\begin{verbatim}
kapkathu [tepala;
kap       -gathu  tepala
receptacle-hither deaf
\end{verbatim}
the billycan here, deaf one
\end{transbox}
%
\begin{transbox}{2}{ali}
\begin{verbatim}
         [((passes billycan to Lily))
\end{verbatim}
\end{transbox}\\

In line 1 Lily leans toward her classificatory sister Alice and addresses her as \textit{tepala} ‘deaf one’.\footnote{In face-to-face conversation, sisters and female cousins tend to address each as \textit{tepala} ‘deaf one’, rather than address each other by name. This mild form of personal name avoidance does not extend to third person reference.} The recruiting move consists of the noun \textit{kap}, used to refer to the item being requested (‘receptacle’ < \textit{cup}), here encliticized with the directional adverbial \textit{-gathu} ‘hither’. As she says this Lily gazes toward the billycan of tea which Alice then passes to Lily.

\transheader{ex:blythe:8}{Ngantimeli20120715\_JB\_video\_GYHM100\_02\_389636}\vspace{2mm}
%
\begin{transbox}{1}{dom}
\begin{verbatim}
[warri (0.3) kurathu;
 warri kura    -gathu
 Fa/So NC:WATER-hither
\end{verbatim}
\hspace{0.07cm} Dad, a drink here!
\end{transbox}\vspace{1mm}
%
\begin{transbox}{2}{~}
\begin{verbatim}
[((points at billycan, Figure 4))
\end{verbatim}
\end{transbox}\vspace{-1mm}
%
\begin{transbox}{3}{dav}
\begin{verbatim}
((passes billycan to Dom))
\end{verbatim}
\end{transbox}\\

\begin{figure}
\includegraphics[width=\textwidth]{figures/murrinhpatha-img4.pdf}
\caption{Whilst holding an empty cup, Dom points to the billycan (\extref{ex:blythe:8}).}
\label{fig:blythe:4}
\end{figure}

\extref{ex:blythe:8} is almost identical to \REF{ex:blythe:7}, except that rather than use a noun to specify the requested item, Dave, while pointing to the nearby billycan (line 2, \figref{fig:blythe:4}), uses the bare water-class classifier \textit{kura} in conjunction with the ‘hither’ adverbial -\textit{gathu} (line 1). In the absence of an explicit predicate, the deictic adverbial -\textit{gathu} implies an object transfer recruitment by indicating the direction that the requested object ought to be transferred. The vast majority of these recruitments (92\%, \textit{n}= 13/14) are accompanied by eye gaze toward the object of desire.

\subsubsection{Interrogatives}\label{sec:blythe:3.3.3}

In some languages like English and Italian interrogatives are a major sentence type utilized in recruiting moves (see Kendrick, \chapref{sec:kendrick}, \sectref{sec:kendrick:4.2}; Rossi, \chapref{sec:rossi}, \sectref{sec:rossi:3.3}), while in other languages like Cha’palaa, Lao, Polish, Russian, and Siwu interrogatives are much less frequent (see Floyd, \chapref{sec:floyd}, \sectref{sec:floyd:3.3}; Enfield, \chapref{sec:enfield}, \sectref{sec:enfield:4.3.1}; Zinken, \chapref{sec:zinken}, \sectref{sec:zinken:3.3}; Baranova, \chapref{sec:baranova}, \sectref{sec:baranova:3.3}; Dingemanse, \chapref{sec:dingemanse}, \sectref{sec:dingemanse:3.2}). In the Murrinhpatha dataset, there are only 4 recruiting moves that are built using interrogative structures. Three of these are built around the interrogative word \textit{ngarra} ‘what’/‘where’, as the next examples illustrates. In \REF{ex:blythe:9} a multiparty conversation has undergone a schism. To facilitate legibility, extraneous overlapping talk has been removed from the transcript.

\transheader{ex:blythe:9}{Dingalngu20110730\_JB\_video\_GYHM100\_04\_231240 (transcript simplified)}\vspace{2mm}
%
\begin{transbox}{1}{ali}
\begin{verbatim}
bere memnginthawarrk (0.2) ng(h)arra
bere       mem            -ngintha  -warrk        ngarra
completion 3SG.S.10RR.NFUT-DU.F.NSIB-lose_oneself LOC
\end{verbatim}
the two of them got lost going
\end{transbox}
%
\begin{transbox}{2}{~}
\begin{verbatim}
(k(h)unungumng(h)intha) (0.4) ngarra Yilimu (1.0) ah ha
kunungam       -ngintha   ngarra yilimu
3SG.S.7go.EXIST-DU.F.NSIB LOC    ♀name
\end{verbatim}
to where Yilimu is ((laughing))
\end{transbox}
%
\begin{transbox}{4}{~}
\begin{verbatim}
(.)
\end{verbatim}
\end{transbox}
%
\begin{mdframednoverticalspace}[style=firstfoc]
\begin{transbox}{5}{mag}
\begin{verbatim}
ngarra mi thawuy:.
ngarra     mi     thawuy
where/what NC:VEG chewing_tobacco
\end{verbatim}
where \{is\} some chewing tobacco?
\end{transbox}
\end{mdframednoverticalspace}
%
\begin{transbox}{6}{~}
\begin{verbatim}
(1.0)
\end{verbatim}
\end{transbox}
%
\begin{mdframednoverticalspace}[style=secondfoc]
\begin{transbox}{8}{car}
\begin{verbatim}
mi thawuy:ka::: tjimngemardamardaka Yilimu damatha;=
mi     thawuy         -ka  tjim           -nge     -mardamarda-ka
NC:VEG chewing_tobacco-TOP 2SG.S.1sit.NFUT-3SG.F.IO-wait_for  -TOP
yilimu damatha
♀name  INTS
\end{verbatim}
as for chewing tobacco, you \{should\} really wait for Yilimu
\end{transbox}
\end{mdframednoverticalspace}
%
\begin{transbox}{9}{~}
\begin{verbatim}
=mi wunku mi thawuy yulirn kandjinkadhukwurran.
 mi     wanku mi     thawuy          yulirn
 NC:VEG also  NC:VEG chewing_tobacco ashes
 kandjin                -kadhuk=wurran
 3SG.S.22bring/take.NFUT-EXIST =3SG.S.6go.NFUT
\end{verbatim}
\hspace{0.07cm} she has both chewing tobacco and ashes
\end{transbox}
%
\begin{transbox}{10}{~}
\begin{verbatim}
(0.7)
\end{verbatim}
\end{transbox}\\

At line 5 Maggie requests chewing tobacco from anyone who might be able to provide it. She does so with the ‘where/what’ interrogative \textit{ngarra}. At lines 8 and 9 Carol informs her that no-one present is able to fulfill her request and that, if she wants chewing tobacco, she will have to wait for another woman to return from fishing.

The remaining, solitary example of a polar interrogative recruiting move is not fulfilled, possibly because the polar question is produced in overlap. In Murrinhpatha, polar questions are not distinguished morphosyntactically from declaratives and, like declarative assertions, generally have falling intonation contours. Given that the linguistic cues to interrogativity are relatively thin, they may be poorly disposed toward recruiting assistance from others.

\subsubsection{Declaratives}

Declaratives are less direct recruiting moves than interrogatives. They do not make explicit the action being elicited. Furthermore, because they mostly have third singular subjects, they do not specify a particular target for the recruitment. As such, they generally highlight a problem. One of those present must take it upon themselves to remedy the issue, if they see fit to do so. % (see also Kendrick, \chapref{sec:kendrick}; Rossi, \chapref{sec:rossi}; Baranova, \chapref{sec:baranova}; Dingemanse, \chapref{sec:dingemanse} of this volume)

\extref{ex:blythe:10} contains a non-minimal sequence. Initially Mary tries to get Edna to fill her cup with water (lines 1 and 2). At line 3 Edna implicitly rejects the recruitment, accounting for her non-compliance by both exclaiming the bottle to be empty (line 3) as well as demonstrating it to be empty by holding it up for Mary to see (\figref{fig:blythe:5}).

\begin{figure}
\includegraphics[height=0.3\textheight]{figures/murrinhpatha-img5.jpg}
\caption{Edna holds up bottle and says \textit{makura karrim} ‘there’s no water’ (\extref{ex:blythe:10}, lines 3--4).}
\label{fig:blythe:5}
\end{figure}

\transheader{ex:blythe:10}{20070728JBvid01c\_10378\_16778}\vspace{-1mm}
%
\begin{mdframednoverticalspace}[style=firstfoc]
\begin{transbox}{1}{mar}
\begin{verbatim}
[kurathu         (1.3) (            )]
 kura    -gathu
 NC:Water-hither
\end{verbatim}
\hspace{0.07cm} water here!
\end{transbox}
\end{mdframednoverticalspace}\vspace{1mm}
%
\begin{mdframednoverticalspace}[style=firstfoc]
\begin{transbox}{2}{~}
\begin{verbatim}
[((holds empty cup out towards Edna))]
\end{verbatim}
\end{transbox}
\end{mdframednoverticalspace}
%
\begin{mdframednoverticalspace}[style=thirdfoc]
\begin{transbox}{3}{edn}
\begin{verbatim}
[makuraya karrim.                         ]
 ma –kura    =ya karrim
 NEG-NC:WATER=CL 3SG.S.stand(3).EXIST
\end{verbatim}
\hspace{0.07cm} there’s no water.
\end{transbox}
\end{mdframednoverticalspace}\vspace{1mm}
%
\begin{mdframednoverticalspace}[style=thirdfoc]
\begin{transbox}{4}{~}
\begin{verbatim}
[((holds up empty water bottle, Figure 5))]
\end{verbatim}
\end{transbox}
\end{mdframednoverticalspace}\vspace{-2mm}
%
\begin{mdframednoverticalspace}[style=secondfoc]
\begin{transbox}{5}{mar}
\begin{verbatim}
((gets up to get water))
\end{verbatim}
\end{transbox}
\end{mdframednoverticalspace}

As well as being an account for Edna’s non-compliance, the syntactically declarative \textit{makuraya karrim} also initiates a counter-recruitment. It does not specify what needs doing, nor who specifically should do it. Mary instantly gets up (line 5) and takes it upon herself to get some water.

As was the case in the previous extract, the third singular declarative predicate in \REF{ex:blythe:11} also does not specify a target for the recruitment. Feasibly, it might not even have been intentionally produced as a recruiting move.

\transheader{ex:blythe:11}{Thuykem 20110901\_JB\_video\_GYHM100\_01\_810540}\vspace{2mm}
%
\begin{transbox}{1}{phi}
\begin{verbatim}
milkka ngarra¿
milk-ka  ngarra
milk-TOP where
\end{verbatim}
where's the milk?
\end{transbox}
%
\begin{transbox}{2}{~}
\begin{verbatim}
(1.0)
\end{verbatim}
\end{transbox}
%
\begin{transbox}{3}{phi}
\begin{verbatim}
wurda damatha ma↑nandji marndarri.
wurda damatha ma -nandji mam             -rdarri
NEG   INTS    NEG-NC:RES 3SG.S.do(8).NFUT-BACK
\end{verbatim}
there isn't any, it \{must be\} behind \{in Wadeye\}
\end{transbox}
%
\begin{transbox}{4}{~}
\begin{verbatim}
(0.3)
\end{verbatim}
\end{transbox}
%
\begin{mdframednoverticalspace}[style=firstfoc]
\begin{transbox}{5}{dav}
\begin{verbatim}
[awu milk karrimwa:.               ]
 awu milk karrim              -wa
 no  milk 3SG.S.stand(3).EXIST-EMPH
\end{verbatim}
\hspace{0.07cm} no, there \textit{is} milk!
\end{transbox}\vspace{1mm}
%
\begin{transbox}{6}{~}
\begin{verbatim}
[((turns and gazes at camping box))]
\end{verbatim}
\end{transbox}
\end{mdframednoverticalspace}
%
\begin{mdframednoverticalspace}[style=secondfoc]
\begin{transbox}{7}{bru}
\begin{verbatim}
((stands up))
\end{verbatim}
\end{transbox}
\end{mdframednoverticalspace}
%
\begin{transbox}{8}{dav}
\begin{verbatim}
[na:; manganart nawa:;    ]
 na  mangan            -art      na -wa
 TAG 3SG.S.grab(9).NFUT-get/take TAG-EMPH
\end{verbatim}
\hspace{0.07cm} hey, he brought it, eh?
\end{transbox}
%
\begin{transbox}{9}{bru}
\begin{verbatim}
[((goes to look for milk))]
\end{verbatim}
\end{transbox}\\

When Phillip’s inquiry about where the milk for the tea might be (line 1) yields no response after one second, he complains  at line 3 that it must have been left behind in Wadeye. However, whilst turning to gaze toward the camping box where the milk ought be, Dave contradicts him, ‘no’, and reassures him that ‘there \textit{is} milk’ (\textit{awu karrimwa}, line 5), then further asserting that ‘he’ (the ethnographer) did in fact bring the milk (line 8). Upon hearing this, Bruce gets up (line 7) and takes it upon himself to retrieve it (line 9), fulfilling the recruitment at line 5 that may not have been intentionally initiated for him specifically to act upon. Feasibly, the recruitment is perhaps an incidental outcome of Dave’s correcting Phillip’s misunderstanding (and perhaps also incidental on Dave, like Phil, wanting milk in his tea).

\subsection{Additional verbal elements}

In this section we examine additional elements within the recruiting move that are not core grammatical constituents. These might include vocative expressions like names and kinterms, interjections, benefactives, strengtheners and mitigators, and explanations.

\subsubsection{Names, kinterms and interjections}

Personal names and kinterms used as vocatives address the person selection dimension of recruitments by picking out the intended recipient. We see personal names functioning as “recipient designators” (\citealt[182]{Lerner2003}) in \REF{ex:blythe:20} and \REF{ex:blythe:22}, and similarly functioning kinterms in \REF{ex:blythe:1}, \REF{ex:blythe:6}, \REF{ex:blythe:8}, \REF{ex:blythe:15} and \REF{ex:blythe:21}. In \REF{ex:blythe:7} and \REF{ex:blythe:23} we see similar use of \textit{tepala} ‘deaf one’ as a characteristic form of address between women who are actual or classificatory sisters.

The interjection \textit{yawu} ‘hey’, when used turn initially, can also be used as a recipient designator to elicit mutual eye gaze between recruiter and would-be recruitee. In line 5 of \REF{ex:blythe:18} the recruiter (Karen) does this before redirecting the recruitee’s attention, with a point, to someone else.

\subsubsection{Benefactives, strengtheners and mitigators}

Benefactive marking in recruitments makes explicit an alleged beneficiary. These are usually marked within the verbal template by bound indirect object pronouns; such as the first person singular -\textit{nga} in \textit{mi na}\textbf{\textit{nga}}\textit{mardakutkathungadha} ‘take a bit out \textbf{for} \textbf{me}’ in \REF{ex:blythe:16}, and the the first person non-singular inclusive -\textit{nye} in \textit{na}\textbf{\textit{nye}}\textit{ngkarnu} ‘mix in some fresh water \textbf{for} \textbf{us}’ in \REF{ex:blythe:15}. Recruiters can use first person non-singular inclusive pronouns strategically by including the addressee as a potential beneficiary of the solicited action, thus downplaying the perception of the benefit being for the recruiter alone.


\hspace*{-1mm}Murrinhpatha deontic adverbials occur both as free-standing words or as bound morphs, some being incorporated into dedicated slots within the polysynthetic verbal template. Those that strengthen are more semantically transparent than those that mitigate. The strengtheners include the emphatic suffix \textit{{}-wa} and the intensfiers \textit{dhatjpirr} and \textit{damatha} (as in \textit{kura burrburr damatha} ‘\{put in\} cold water!’). The mitigating adverbials like -\textit{ngadha}, often translated as ‘for a while’, are difficult to gloss and are less well understood.\footnote{Some of these adverbials are translated, at least sometimes, with temporal semantics.} Other mitigators include ad-hoc compliments being paid to the recruitee (such as \textit{murruwurlnyingka} ‘you are beautiful’, in line 4 of \extref{ex:blythe:6}).

\subsubsection{Explanations}

Explanations or accounts for a recruitment may be added after the recruiting move, as in line 9 of \REF{ex:blythe:23}, or they may precede it, as in \REF{ex:blythe:12}.

\transheader{ex:blythe:12}{Nanthak 20110828\_JB\_video\_GYHM100\_03\_537800\_546223}\vspace{2mm}
%
\begin{transbox}{1}{kar}
\begin{verbatim}
[ngawu (1.0) thagilkilktharra
 ngawu tha               -gilkilk-tharra
 hey!  2SG.S.POKE(19).FUT-hang   -ahead
\end{verbatim}
\hspace{0.07cm} hey! (1.0) poke \{this\} through the handle and carry it
\end{transbox}
%
\begin{transbox}{2}{~}
\begin{verbatim}
[((picks up billycan with a stick, passing it toward Rita))
\end{verbatim}
\end{transbox}
%
\begin{transbox}{3}{~}
\begin{verbatim}
(0.5)
\end{verbatim}
\end{transbox}
%
\begin{transbox}{4}{kar}
\begin{verbatim}
[karduka tjinengirdarribangnukun.]=panguwangu nabatjtharra.
 kardu -ka  tjina              -ngi  -rdarri-bang -nukun
 NC:HUM-TOP 2SG.S.HEAT(27).FIRR-1S.IO-back  -scald-FIRR
 pangu-wangu  na               -batj    -tharra
 DIST-thither 2SG.S.GRAB(9).FUT-get/take-ahead
\end{verbatim}
\hspace{0.07cm} you might scald me on the back, take it over that way
\end{transbox}
%
\begin{transbox}{5}{~}
\begin{verbatim}
[  ((hands the stick to Rita))   ]
\end{verbatim}
\end{transbox}\vspace{-2mm}
%
\begin{transbox}{6}{rit}
\begin{verbatim}
((takes the hot billycan away to fill with cold water))
\end{verbatim}
\end{transbox}\\

In \REF{ex:blythe:12} Rita is standing up on the beach ready to take a very hot billycan to where there is water with which to cool down the tea. At line 2 Karen picks up the billycan with a stick, placing the billy on the ground near Rita, meanwhile telling her to poke the stick through its handle in order to carry it (line 1).  As she passes the stick to Rita, Karen explains in line 4 that Rita might scald her with the hot tea (\textit{karduka tjinengirdarribangnukun}) then instructs her to ‘take it over that way’ (\textit{panguwangu nabatjtharra}), through the gap where no one is sitting.

In the next section we consider the range of possible ways that would-be recruitees respond to a recruiting move, or not as the case may be.

\section{Formats in Move B: The responding move}\label{sec:blythe:4}

A substantial body of research in conversation analysis investigates how the design of turns delivering initiating actions (\citealt{Wootton1997,VinkhuyzenSzymanski2005,Curl2006,curl_contingency_2008,CravenPotter2010,EnfieldEtAl2010,StiversRossano2010,DeRuiter2012,Rossi2012,KendrickDrew2016}) impose constraints upon the sorts of responses they receive (\citealt{Raymond2003,SchegloffLerner2009,FoxThompson2010,Lee2013,ThompsonFoxCouperKuhlen2015}). In this Murrinhpatha dataset, only 46\% of recruitments were either fulfilled promptly (24\%, \textit{n}=35) or indications were provided suggesting possible imminent fulfillment (22\%, \textit{n}=32). Counts on response types to particular recruiting formats do \textit{not}, at this stage, suggest that any particular format (e.g. imperative, declarative, interrogative, etc.) is more or less likely to successfully elicit the desired response than any other format.

Just as the formats used in recruiting moves range between the overt, on-record strategies, through to more covert, off-record strategies, so too do the range of possible responses. Overt on-record responses include both immediate compliance and relatively prompt rejection of the recruitment, while physical movements suggestive of possible compliance are more covert and less on-record. In this corpus overt on-record rejections are considerably less frequent than implicit rejections or non-fulfillments; such as counter-recruitments, deflected recruitments, and generally just ignoring the recruitment. Non-responses evade overt refusal or rejection of the recruitment. We will see evidence below that by neither complying nor committing to complying, ignoring a recruitment can usually be taken as an implicit refusal to comply.

\subsection{Prompt or imminent compliance}

We have already encountered many recruitment sequences in which the response is physical compliance delivered relatively promptly without an accompanying verbal component (Extracts \ref{ex:blythe:1}, \ref{ex:blythe:2}, \ref{ex:blythe:3}, \ref{ex:blythe:4}, \ref{ex:blythe:5}, \ref{ex:blythe:7} and \ref{ex:blythe:8}). We have also seen in \REF{ex:blythe:5} how removing a hat and then removing a billycan from the fire suggest possible imminent compliance, which is ultimately followed by actual compliance. Possible imminent compliance can also be verbally hinted at without giving a commitment to actually comply, as \REF{ex:blythe:13} demonstrates.

\transheader{ex:blythe:13}{Nanthak 20110828\_JB\_video\_GYHM100\_03\_427300}\vspace{-1mm}
%
\begin{mdframednoverticalspace}[style=firstfoc]
\begin{transbox}{1}{lil}
\begin{verbatim}
ngarra kurayu.
ngarra     kura    =yu
what/where NC:WATER=CL
\end{verbatim}
where's the water/tea?
\end{transbox}
\end{mdframednoverticalspace}
%
\begin{transbox}{2}{~}
\begin{verbatim}
((gazes at Alice, Figure 6))
\end{verbatim}
\end{transbox}
%
\begin{transbox}{3}{~}
\begin{verbatim}
(0.5)
\end{verbatim}
\end{transbox}
%
\begin{mdframednoverticalspace}[style=secondfoc]
\begin{transbox}{4}{ali}
\begin{verbatim}
kuguk marrawangu.
kuguk marra  -wangu
wait  new/now-thither
\end{verbatim}
wait it's coming
\end{transbox}
\end{mdframednoverticalspace}
%
\begin{figure}
\includegraphics[width=\textwidth]{figures/murrinhpatha-img6.jpg}
\caption{Lily gazes at Alice (\extref{ex:blythe:13}, line 2).}
\label{fig:blythe:6}
\end{figure}

\extref{ex:blythe:13} occurs near the beginning of a protracted episode of multiple recruitments, all of which deal, in some fashion, with the procurement of cold water for a very hot billy of tea.\footnote{The four, mostly elderly, women in this conversation are tired and feeling lethargic. The water required to cool the hot billycan is nearby on the beach, in a heavy 20-liter container. The women each display justifiable resistance to getting up and retrieving the water.} Lily’s question at line 1 (\textit{ngarra kurayu}) is built around the ‘where’/‘what’ interrogative \textit{ngarra} and the bare \textit{water}-classifier (‘where is the tea/water?’). Whilst certainly a request, it can also be heard as a possible complaint. Although Lily’s eye gaze is directed on Alice who is seated near the billycan, it is Karen, rather than Alice, who is preparing the tea. While Alice’s reply \textit{kuguk marrawangu} ‘wait it’s coming’ does address the substance of the possible complaint (being slow in arriving), it does not commit to future compliance and is agnostic as to who will be responsible for ultimately fulfilling the request.  The question of who will get the water remains unresolved for quite some time.

\extref{ex:blythe:14} consists of two interlocking non-minimal sequences commencing with nonverbal recruiting moves. Each non-minimal sequence is of the three-move variety previously mentioned in \sectref{sec:blythe:2.2}, where participant B firstly commits to complying (M-B\textsubscript{1}) with the recruitment, then actually complies with it soon after (M-B\textsubscript{2}).

At line 2 of \REF{ex:blythe:14} Dom (who has a cigarette in his mouth) leans forward. Whether leaning forward was intended as an offer is unclear, but either way it seems to occasion a recruiting move from Dave at line 3, where he holds his hand out to receive the cigarette. Dom’s response to this recruiting move is semiotically and sequentially complex. The sweeping point from Dave to Bruce (see  \figref{fig:blythe:7}) is an iconic depiction of the trajectory Dom intends the cigarette to travel along when Dave finishes with it. The drawing in the air conveys graphically that the recruitment is of the object transfer variety.

\transheader{ex:blythe:14}{Ngantimeli 20120715\_JB\_video\_GYHM100\_02\_196571}\vspace{2mm}
%
\begin{transbox}{1}{dom}
\begin{verbatim}
[mhm
\end{verbatim}
\end{transbox}
%
\begin{transbox}{2}{~}
\begin{verbatim}
[((leans toward Dave with cigarette in mouth))
\end{verbatim}
\end{transbox}
%
\begin{mdframednoverticalspace}[style=firstfoc]
\begin{transbox}{3}{dav}
\begin{verbatim}
((holds out hand to receive cigarette))
\end{verbatim}
\end{transbox}
\end{mdframednoverticalspace}
%
\begin{mdframednoverticalspace}[style=thirdfoc]
\begin{transbox}{4}{dom}
\begin{verbatim}
((sweeping point from Dave toward Bruce, Figure 7))
\end{verbatim}
\end{transbox}
\end{mdframednoverticalspace}\vspace{-1mm}
%
\begin{mdframednoverticalspace}[style=secondfoc]
\begin{transbox}{5}{dav}
\begin{verbatim}
[nakurl ngaliwe nganamutnu.               ]
 nakurl ngaliwe nga               -rna     -mut -nu
 later  short   1SG.S.poke(19).FUT-3SG.M.IO-give-FUT
\end{verbatim}
\hspace{0.07cm} I'll give a bit to him after
\end{transbox}
\end{mdframednoverticalspace}
% M-B1\textsubscript{S2}\\
\begin{mdframednoverticalspace}[style=thirdfoc]
\begin{transbox}{6}{dom}
\begin{verbatim}
[((sweeping point from Dave toward Bruce))]
\end{verbatim}
\end{transbox}
\end{mdframednoverticalspace}
% M-B1\textsubscript{S2 (repeat)} \\
\begin{transbox}{7}{~}
\begin{verbatim}
((takes a drag on the cigarette))
\end{verbatim}
\end{transbox}
%
\begin{mdframednoverticalspace}[style=secondfoc]
\begin{transbox}{8}{~}
\begin{verbatim}
((passes cigarette to Dave))
\end{verbatim}
\end{transbox}
\end{mdframednoverticalspace}
%
\begin{transbox}{9}{~}
\begin{verbatim}
((30 seconds of talk deleted, Dave smokes cigarette))
\end{verbatim}
\end{transbox}
%
\begin{mdframednoverticalspace}[style=secondfoc]
\begin{transbox}{10}{dav}
\begin{verbatim}
((dave passes cigarette to Bruce))
\end{verbatim}
\end{transbox}
\end{mdframednoverticalspace}
%M-B2\textsubscript{S1}

\begin{figure}
\includegraphics[width=\textwidth]{figures/murrinhpatha-img7.jpg}
\caption{
Dom points from Dave to Bruce (\extref{ex:blythe:14}, line 4). This sweeping point is both an explicit-object transfer request and an implicit commitment to imminently comply with Dave’s request for the cigarette.
\label{fig:blythe:7}
}\end{figure}

This depictive point is repeated at line 6.\footnote{The repetition of the point is instantaneous and fluidly produced (and is hence more akin to reduplication than actual repetition), as if the invisible line in the air is being heavily bolded.} The gesture recruits Dave to pass the cigarette to Bruce when he is finished with it. In overlap with the repeat of the point, at line 5 Dave gives a verbal undertaking to comply with this recruitment (\textit{nakurl ngaliwe nganamutnu} ‘I’ll give him the stub later’). %(M-B1\textsubscript{S2}).
This is the only vocal move in either of the two sequences.

Dom’s sweeping points (lines 4 and 6) do more than merely recruit. By virtue of the fact that the cigarette is retained in Dom’s mouth, they also can be seen as him giving an implicit undertaking to imminently comply with Dave’s recruiting move at line 3. %(M-B1\textsubscript{S1}).
Dom’s passing of the cigarette at line 8 is the eventual fulfillment implicitly promised at lines 4 and 6. %(M-B2\textsubscript{S1})
Likewise, when Dave passes the cigarette to Bruce at line 10, %(M-B2\textsubscript{S2}),
this can be seen as the fulfillment of the recruitment initiated by Dom that Dave had committed to fulfilling at lines 4 and 6.

Imminent possible compliance, or incipient compliance (\citealt{Schegloff1989,kent_compliance_2012}), can be projected visibly (as Dom does in lines 4 and 6 of \extref{ex:blythe:14}) or verbally (as Dave does in line 3 of \extref{ex:blythe:14} and Alice does in line 4 of \extref{ex:blythe:13}). In the next section we will encounter a mixed-message example, where the physical responsive behavior contradicts the verbally delivered response.

\subsection{Rejection and non-compliance}\label{sec:blythe:4.2}

The preferred response to a request, or any sort of recruiting move, is to comply with or fulfill the recruitment, or at least display that probable compliance is forthcoming. Anything less is dispreferred. The range of dispreferred alternatives is scalar. The most dispreferred alternatives are the overt refusals or rejections, which are vanishingly rare in this collection (\textit{n}=3 from 145 recruitments, 2\%). Only two refusals include the rejection token \textit{awu} ‘no’.

Of the 145 recruitment sequences in the Murrinhpatha collection, 54\% (\textit{n}=78) were not promptly complied with, nor was possible compliance projected as imminent. This may be because a request is problematic, unreasonable, or that other matters must be attended to before the recruitment can be fulfilled. The various possible alternatives to the preferred response include both explicit and implicit refusal. Delaying dispreferred responses can project that an imminent refusal is forthcoming (perhaps to be delivered with an overt rejection token), or that non-compliance is to be inferred from the silence that ensues. Other-initiated repair has the effect (whether intentional or otherwise) of delaying the expected compliance or refusal, such that potentially problematic requests become vulnerable to sequential deletion.

\subsubsection{Overt rejections}

Overt refusals or rejections are socially dispreferred responses. As such, dispreferred responses tend to be delayed, mitigated, and accounted for (\citealt[265–80]{heritage1984garfinkel}; \citealt{Pomerantz1984agree}; \citealt[58–96]{Schegloff2007sequence}; \citealt{PomerantzHeritage2013}).

\hspace*{-.8mm}Just prior to \REF{ex:blythe:15}, the ethnographer poured himself a cup of hot tea from the billy and, before walking away from the scene, remarked that he likes hot unsweetened black tea. This is anathema to the four women in this extract, as they normally drink sweet white lukewarm tea from metal pannikins, which heat up when hot liquid is poured into them.

\transheader{ex:blythe:15}{Nanthak 20110828\_JB\_video\_GYHM100\_03\_453900\_460860}\vspace{-1mm}
%
\begin{mdframednoverticalspace}[style=firstfoc]
\begin{transbox}{1}{ali}
\begin{verbatim}
[munak kura pathadhatjpirr nanyengkarnu.
 munak  kura     patha-dhatjpirr na             -nye       -ngkarnu
 sister NC:WATER good -INTS      2SG.S.do(8).FUT-1NS.INC.IO-mix_into
\end{verbatim}
\hspace{0.07cm} sister, mix in some fresh water for us
\end{transbox}
%
\xtransbox{2}{~}{[((points to water-bottle/vehicle))}
\end{mdframednoverticalspace}
%
\emptytransbox{3}{(0.3)}
%
\begin{mdframednoverticalspace}[style=secondfoc]
\begin{transbox}{4}{kar}
\begin{verbatim}
[ya beremanangatha dendurr pigurdugurduk.
 ya  beremanangatha  dendurr pi                -gurdu-gurduk
 HES never_mind.INTS hot     1NS.INC.sit(1).FUT-RDP-drink
\end{verbatim}
\hspace{0.07cm} um, it really doesn't matter, we’ll drink it hot
\end{transbox}
%
\xtransbox{5}{~}{[((points into billycan))}
\end{mdframednoverticalspace}
%
\begin{transbox}{6}{~}
\begin{verbatim}
(.)
\end{verbatim}
\end{transbox}
%
\begin{transbox}{7}{ali}
\begin{verbatim}
[awu ku(h)rdu]nyidham(h)arrarrnukun[:;
 awu kurdu               -nyi       -dhamarrarr -nukun
 no  3SG.S.shove(29).FIRR-1NS.INC.DO-burn_throat-FIRR
\end{verbatim}
\hspace{0.07cm} no! i(h)t might b(h)urn our throats!
\end{transbox}
%
\begin{transbox}{8}{lil}
\begin{verbatim}
[ (h)a:wu;!  ]                     [↓karraya;↓
     awu                             karraya
     no                              goodness!!
\end{verbatim}
\hspace{0.65cm} n(h)o! \hspace{3.85cm} good grief!!
\end{transbox}\bigskip

At line 1 Alice, addressing Karen with the kinterm \textit{munak} ‘sister’, tells her to mix cold water into the hot tea. Karen refuses the request at line 4. Her tongue-in-cheek refusal echoes the ethnographer’s earlier remark by insisting (sarcastically) that they will drink their tea hot. Despite the proposal being non-serious, the refusal is genuine. The dispreferred nature of the response is evident in the delay provided by the hesitation marker \textit{ya}, approximately ‘um’/‘ah’, and the adverbial interjection \textit{beremanangatha} ‘it doesn’t really matter’. The refusal to comply is implicit in the reason (albeit, a preposterous one) for not complying (we’ll drink it hot!). The refusal elicits both disagreement and complaint from both Alice and Lily, whose responses at lines 7 and 8, respectively, are infused with laughter particles.

The overt refusal in \REF{ex:blythe:15} is verbally delivered. Furthermore Karen’s physical behavior does not suggest any likelihood of her possibly complying in the future. Her physical behavior accords with her verbal behavior. However, in \REF{ex:blythe:16} the overt, vocally delivered dispreferred refusal is somewhat contradicted by the refuser’s physical actions, which instead suggest possible imminent compliance.

\transheader{ex:blythe:16}{Nanthak 2011 0828\_JB\_video\_GYHM100\_03\_760030\_770043}\vspace{2mm}
%
\begin{transbox}{1}{kar}
\begin{verbatim}
(nga mi nanyemawathawarra.) ba berenguny berenguny
 nga mi     na                -nye       -ma  -watha-warra
 hey NC:VEG 2SG.S.hands(8).FUT-1NS.INC.IO-hand-make -ahead
 ba   berenguny berenguny
 STRI OK        OK
\end{verbatim}
\hspace{0.07cm} (hey roll us a cigarette), oh it's alright, it's alright
\end{transbox}
%
\begin{mdframednoverticalspace}[style=firstfoc]
\begin{transbox}{2}{ali}
\begin{verbatim}
[aa mi numigathungadha aa mi nangamardakutkathungadha.
 aa mi     numi-gathu -ngadha aa mi
 ah NC:VEG one -hither-while  Ah NC:VEG
 na                -nga   -mardakut-gathu -ngadha
 2SG.S.hands(8).FUT-1SG.IO-take_out-hither-while
\end{verbatim}
\hspace{0.07cm} ah, give me one, take a bit out for me
\end{transbox}
\end{mdframednoverticalspace}
%M-A\textsubscript{S1}\\
\begin{mdframednoverticalspace}[style=firstfoc]
\begin{transbox}{3}{~}
\begin{verbatim}
[((holds out hand to receive)) -->
\end{verbatim}
\end{transbox}
\end{mdframednoverticalspace}
%
\emptytransbox{4}{(0.5)}
%
\begin{mdframednoverticalspace}[style=secondfoc]
\begin{transbox}{5}{ali}
\begin{verbatim}
[awu; mi nukunudha nginarr puleyu.
 awu mi     nukunu-dha nginarr       pule    =yu
 no  NC:VEG 3SG.M-PIMP poison_cousin esteemed=CL
\end{verbatim}
\hspace{0.07cm} no it's from him (your) poison cousin ((FMBS))
\end{transbox}
\end{mdframednoverticalspace}
% M-B\textsubscript{S1}
\emptytransbox{6}{[(Karen gets out tobacco, Alice holds out hand, Figure 8))}
%
\begin{mdframednoverticalspace}[style=firstfoc]
\begin{transbox}{7}{ali}
\begin{verbatim}
mi mamawatha;
mi     ma               -ma  -watha
NC:VEG 1SG.S.hand(8).FUT-hand-make
\end{verbatim}
I want to roll some
\end{transbox}
\end{mdframednoverticalspace}
%M-A\textsubscript{S2}\
\begin{mdframednoverticalspace}[style=firstfoc]
\begin{transbox}{8}{~}
\begin{verbatim}
--> ((holds out hand to receive)) -->
\end{verbatim}
\end{transbox}
\end{mdframednoverticalspace}
%
\begin{transbox}{9}{~}
\begin{verbatim}
(0.7)
\end{verbatim}
\end{transbox}
%
\begin{mdframednoverticalspace}[style=secondfoc]
\begin{transbox}{10}{kar}
\begin{verbatim}
[thaninapartwardaya,
 thani          -rna     -part –warda=ya
 2SG.S.be(4).FUT-3SG.M.IO-leave-TEMP =CL
\end{verbatim}
\hspace{0.07cm} leave it for him
\end{transbox}
\end{mdframednoverticalspace}
%M-B\textsubscript{S2}\\
\emptytransbox{11}{[((Karen looks into tobacco tin, Alice holds out hand))}\\

\begin{figure}
\includegraphics[width=\textwidth]{figures/murrinhpatha-img8.jpg}
\caption{
While taking her tobacco out of her pocket, Karen says, \textit{awu mi nukunudha nginarr puleyu} ‘no it’s from your poison cousin’ (\extref{ex:blythe:16}, line 5).
\label{fig:blythe:8}}\end{figure}

The group of conversationalists in \REF{ex:blythe:16} have been sitting on the beach for a while, drinking tea and smoking. None of them have much tobacco left. At line 1 Karen seems to request something, but then backs down, canceling the request.\footnote{The translation alleged for the utterance \textit{nga mi nanyemawathawarra} is ‘hey, roll us a cigarette’. Why Karen would say this is unclear, as she already has an unlit cigarette in her mouth! That said, her motives for canceling the request are perhaps clearer.} At line 2 Alice combines a nominal-hither construction (\textit{mi numigathungadha} ‘one portion of/more tobacco over here’) with an imperative verb (\textit{nangamardakutkathungadha} ‘take some out for me over here’) to request tobacco from Karen’s tin, meanwhile holding her hand out to receive it. At line 5 Karen refuses the request (\textit{awu} ‘no’), accounting for the refusal by claiming that it was provided by (or that it belongs to) her husband. However, rather than referring to him by name, or with a self-anchored kinterm as ‘my husband’ (\textit{nangkun ngay}) \citep{Blythe2010b}, she instead uses the alternative recognitional \citep{Stivers2007} \textit{nginarr puleyu} ‘\{your\} poison cousin’, implicitly anchored to her addressee, Alice. The kinterm \textit{nginarr} -- here, ‘father’s mother’s brother’s son’ -- connotes extreme avoidance; the implication being that the tobacco, like the kinsman, ought best be avoided. Despite this rationale being provided, Karen gets out the tobacco tin from her pocket, hinting that the provision of some tobacco is not out of the question (see \figref{fig:blythe:8}).  Unfazed, Alice, still holding her hand out, pursues the request with \textit{mi mamawathangadhaya} ‘I’d like to roll some’ (line 7).\footnote{When Karen mentions \textit{nginarr puleyu} ‘\{your\} poison cousin’, Alice waggles the hand she is holding out (see \figref{fig:blythe:8}) and then continues to hold it there; thereby demonstrating that either, if the kinship relation is a genuine cause for concern, she is prepared to wear the consequences, or that Karen’s excuse is fanciful and won’t wash with her.} At line 9 Karen again declines the request verbally (\textit{thaninapartwardaya} ‘leave it for him/on account of him’) whilst inspecting the tobacco tin’s contents (line 10), again hinting that possible compliance might be forthcoming. Despite the overt, verbally delivered refusals, Alice ultimately receives skerricks of tobacco from both Karen and co-present Lily, sufficient to roll herself a cigarette.

The dispreferred nature of the refusals are evident in the silence preceding the replies (0.5s at line 4 and 0.7s at line 9) and in the reason provided at line 5.\footnote{While strictly speaking the gaps are not necessarily longer than various others which precede certain \textit{preferred} second pair parts (cf. \citealt{Gardner2015} for Garrwa conversation), they nevertheless reveal diminishing prospects for prompt compliance.} The hard line of the vocally delivered refusal is mitigated somewhat by the visual behavior that projects an alternative reality to that being projected verbally.


\subsubsection{Implicit refusals: Counters, deflections and accounts.}\label{sec:blythe:4.2.2}

In the absence of an overt rejection token, with implicit refusals, rejection of the recruitment is inferable from the design of the responding move. Implicit refusal may be delivered solely as an account for not complying (as in line 6 of \extref{ex:blythe:17}). Two further varieties are counters and deflections. Both can have the effect of derailing recruitments. This is because the opportunities for compliance to be fitted sequentially, as responses to initiating actions, tend to rapidly evaporate. \extref{ex:blythe:17} illustrates this with a counter-recruitment (cf. Kendrick, \chapref{sec:kendrick}, \sectref{sec:kendrick:5.3}).

\transheader{ex:blythe:17}{Thuykem 2011 0824\_JB\_video\_GYHM100\_02\_1214705}\vspace{-1mm}
%
\begin{mdframednoverticalspace}[style=firstfoc]
\begin{transbox}{1}{gre}
\begin{verbatim}
[dadhawibuwathu.
 da                -dhawibu         -gathu
 2SG.S.BASH(14).FUT-ignite_cigarette-hither
\end{verbatim}
\hspace{0.07cm} light this cigarette.
\end{transbox}
\end{mdframednoverticalspace}
%M-A1
\begin{mdframednoverticalspace}[style=firstfoc]
\begin{transbox}{2}{~}
\begin{verbatim}
[((holds out an unlit cigarette for Mike to take))
\end{verbatim}
\end{transbox}
\end{mdframednoverticalspace}
%
\emptytransbox{3}{(0.7)}
%
\begin{mdframednoverticalspace}[style=thirdfoc]
\begin{transbox}{4}{mik}
\begin{verbatim}
dadhawibu.
da                -dhawibu
2SG.S.BASH(14).FUT-ignite_cigarette
\end{verbatim}
light the cigarette
\end{transbox}
\end{mdframednoverticalspace}
% M-B1, M-A2\\
\emptytransbox{5}{(1.0)}
%
\begin{mdframednoverticalspace}[style=secondfoc]
\begin{transbox}{6}{gre}
\begin{verbatim}
ngay merengadha ngiku.
ngay mere-ngadha ngi             -ku
1SG  NEG –still  1SG.S.sit(1).FUT-get_going
\end{verbatim}
I can't move
\end{transbox}
\end{mdframednoverticalspace}\bigskip

At line 2 of \REF{ex:blythe:17} Greg holds out an unlit cigarette toward Mike who is seated in front of him. In the absence of a lighter, at line 1 he produces an imperatively formatted recruiting move \textit{dadhawibuwathu} ‘light the cigarette’. After 0.7s delay, Mike counters by firing back more-or-less the same recruiting move, \textit{dadhawibu}, effectively ‘light the cigarette \{yourself\}’ (line 4). Greg refuses the counter recruitment by literally providing a “lame” excuse: \textit{ngay merengadha ngiku} ‘I can’t move’ (line 6); the account here serves as an implicit rejection. Greg’s recruiting move remains unfulfilled.\footnote{After further unsuccessful attempts by Greg at recruiting someone to light it (see \extref{ex:blythe:20}), Mike eventually offers to light it. Offers, however are initiating moves rather than responsive moves.} In the next section below, we will see a further dramatic rejection delivered as a counter (at line 7 of \extref{ex:blythe:23}).

In \REF{ex:blythe:18} we see an implicit refusal via a deflected recruitment. Karen and Alice are both speaking to Maggie, a woman of about 90 years of age, who is hard of hearing. Just prior to this extract Maggie had been requesting chewing tobacco, but none was available (see \extref{ex:blythe:9}). Karen has just lit a cigarette, which she is holding in her hand. At line 3, Alice whispers to Karen that Maggie wants to smoke. Thus, she attempts to recruit Karen into passing Maggie her cigarette. Karen’s dispreferred response is delayed initially by 0.7 seconds (line 4) and further delayed by the interjection \textit{yawu} ‘hey!’ (line 5). The interjection initially draws Maggie’s eye gaze toward her (\figref{fig:blythe:9}\textit{a}), and then subsequently in the direction of her classificatory brother standing off-screen (\figref{fig:blythe:9}\textit{b}). Karen then directs Maggie to ask the man off-screen (for permission to be granted the request).\footnote{Karen’s classificatory brother (\textit{Kembutj}) has brought Maggie out bush, from the frail-aged hostel in Wadeye. By evoking him as a responsible person (given that he has taken responsibility for her wellbeing, at least for the day), she thereby abdicates any responsibility she might have, as provider of cigarettes, for the potentially detrimental effects smoking could have for an old woman.}

\transheader{ex:blythe:18}{Dingalngu 2011 0730\_JB\_video\_GYHM100\_04\_845780\_855106}\vspace{2mm}
%
\begin{transbox}{1}{kar}
\begin{verbatim}
nyiniwa kangkurl nyinyiyu kumban nyiniyu.
nyini-wa   kangkurl nyinyi=yu kumban          nyini=yu
ANAPH-EMPH wBSC     2SG   =CL 3PL.S.6go.EXIST ANAPH=CL
\end{verbatim}
they're your grandsons, all of them
\end{transbox}
%
\emptytransbox{2}{(0.3)}
%
\begin{mdframednoverticalspace}[style=firstfoc]
\begin{transbox}{3}{ali}
\begin{verbatim}
°°purdiwinuwarda°°
  purdi       -wi   -nu –warda
  3SG.S.30.FUT-swell-FUT-TEMP
\end{verbatim}
\hspace{0.14cm} she wants to smoke
\end{transbox}
\end{mdframednoverticalspace}
%M-A\\
\emptytransbox{4}{(0.7)}
%
\begin{mdframednoverticalspace}[style=secondfoc]
\begin{transbox}{5}{kar}
\begin{verbatim}
[yawu! (.)    ] thadharrpu ngawu. (0.4) [kardu ngaynukun;]
 yawu     tha             -dharrpu ngawu kardu  ngay-nukun
 hey!     2SG.S.19Poke.FUT-ask     hey!  NC:HUM 1SG -FIRR
\end{verbatim}
\hspace{0.07cm} hey, you ask hey! the \{brother\} of mine
\end{transbox}
\end{mdframednoverticalspace}
%
\begin{mdframednoverticalspace}[style=secondfoc]
\begin{transbox}{6}{\fig}
\begin{verbatim}
[  Figure 9a  ]                         [    Figure 9b   ]
\end{verbatim}
\end{transbox}
\end{mdframednoverticalspace}
%
\emptytransbox{7}{(0.2)}
%
\begin{transbox}{8}{ali}
\begin{verbatim}
kembutj [thadharrpu.
kembutj    tha             -dharrpu
man's_name 2SG.S.19Poke.FUT-ask
\end{verbatim}
ask Kembutj ((for permission))
\end{transbox}
%
\begin{transbox}{9}{kar}
\begin{verbatim}
        [mama thadharrpu; (0.7) ngathan narna.
         mama   tha             -dharrpu ngathan 
         mother 2SG.S.19Poke.FUT-ask     brother
         na              -rna
         2SG.S.say(8).FUT-3SG.M.IO
\end{verbatim}
\hspace{1.25cm} ask him mum. (0.7) ask \{my\} brother
\end{transbox}

\begin{figure}
\includegraphics[width=\textwidth]{figures/murrinhpatha-img9.jpg}
\caption{
(\textit{a}) \textit{yawu} ‘hey!’; (\textit{b}) \textit{kardu ngaynukun} ‘to my \{brother\}’ (\extref{ex:blythe:18}, line 6).
\label{fig:blythe:9}}\end{figure}

The classificatory brother subsequently becomes drawn into the conversation (not shown in the extract). Maggie does not ask him for permission, and she does not receive a smoke. Her desire to to smoke remains unaddressed. Alice’s recruitment initiation is derailed without the need for an overt refusal. Deflected recruitments reallocate responsibility for complying to a third party, such that the likelihood of the desired outcome arising is diminished.

\subsubsection{Other-initiations of repair}\label{sec:blythe:4.2.3}

As responsive moves that neither comply nor project compliance to recruitments, nor outrightly reject recruitments, other-initiations of repair (OIR) produced by the target of a recruitment are dispreferred responses. Not being of the category type projected by the recruiting turn (\citealt{Raymond2003,HeritageRaymond2012}), other-initiations of repair results in delay of the expected category type response. This characteristic feature of dispreference can forecast imminent refusal of the recruitment (\citealt[380]{schegloff_preference_1977}).

In \REF{ex:blythe:19} Karen, Alice, Lily and Maggie are conversing in a group as they sit on one side of a 4WD which has a trailer behind it. On the other side of the trailer, another group of women are also seated on the ground, and also being recorded on video as they converse. The car and the trailer creates a visual barrier between the groups that obscures their lines of sight. At line 1 of \REF{ex:blythe:19} Karen summons one of the women in the other group (Lily, apparently) to come and explain something to Alice. She does this with two interjections \textit{yawu} ‘hey!’ and \textit{kagawu} ‘come here!’ and with the second person singular imperative verb \textit{thurduriyitjmani} ‘try and explain it’. As she yells this summons she tries to look underneath the trailer to get a visual on her target. When this summons yields no result after 1.5s (line 2), Karen reissues the summons with another second person singular imperative verb \textit{thurrumaniyethu} ‘come here will you’ (line 3). After further delay (0.6s, line 4), Lily initiates repair on the second person singular subjects of these verbs with the person-specific content question \textit{nangkal} ‘who’. At line 7 Karen specifies the previous speaker, Lily, as the target of the intended summons (\textit{nyinyi nyinyi} ‘you, you’), which is echoed by Alice at line 8. At line 10 Lily refuses the summons, invoking the video camera in accounting for the refusal.

\transheader{ex:blythe:19}{Dingalngu 2011 0730\_JB\_video\_GYHM100\_04\_341515\_350670}\vspace{-1mm}
%
\begin{mdframednoverticalspace}[style=firstfoc]
\begin{transbox}{1}{kar}
\begin{verbatim}
↑YAWU kardu thurduriyitjmani kagawu!↑
 yawu kardu    thurdu      -riyitj –mani   kagaw
 hey! NC:HUMAN 2SG.S.29.FUT-explain-try_to come_here
\end{verbatim}
\hspace{0.07cm} HEY! try come here and explain \{to her\}
\end{transbox}
\end{mdframednoverticalspace}
%
\begin{transbox}{2}{~}
\begin{verbatim}
(1.5)
\end{verbatim}
\end{transbox}
%
\begin{mdframednoverticalspace}[style=firstfoc]
\begin{transbox}{3}{kar}
\begin{verbatim}
thurrumaniyethu
thurru         -mani   -gathu
2SG.S.go(6).FUT-be_able-HITHER
\end{verbatim}
come here will you
\end{transbox}
\end{mdframednoverticalspace}
%
\emptytransbox{4}{(0.6)}
%
\begin{mdframednoverticalspace}[style=secondfoc]
\begin{transbox}{5}{lil}
\begin{verbatim}
nangka:l;
who
\end{verbatim}
who?
\end{transbox}
\end{mdframednoverticalspace} %M-B, OIR\\
%
\begin{transbox}{6}{~}
\begin{verbatim}
(0.2)
\end{verbatim}
\end{transbox}
%
\begin{transbox}{7}{kar}
\begin{verbatim}
nyinyi [nyinyi.
2SG     2SG
\end{verbatim}
you, you!
\end{transbox}
%
\begin{transbox}{8}{ali}
\begin{verbatim}
       [nyinyi.
        2SG
\end{verbatim}
\hspace{1.1cm} you!
\end{transbox}
%
\begin{transbox}{9}{~}
\begin{verbatim}
(.)
\end{verbatim}
\end{transbox}\vspace{-2mm}
%
\begin{transbox}{10}{lil}
\begin{verbatim}
ya nandji kanyinu nga ngay ngurdamyitjnganam.
ya  nandji kanyi-nu  nga ngay
HES NC:RES PROX -DAT Hey 1SG
ngurdam                 -yitj=nganam
1SG.SB.SHOVE.RR(30).NFUT-tell=1SG.SB.BE(4).NFUT
\end{verbatim}
I'm telling stories into this thing ((a video camera))
\end{transbox}\bigskip

The delay induced by an other-initiation of repair can also have the effect that the necessity for the recruitee to comply, or account for not complying, disappears through the unrolling of interactional events (see also Dingemanse, \chapref{sec:dingemanse}, \sectref{sec:dingemanse:4.2}). Thus in \REF{ex:blythe:20} Greg continues attempting to enlist someone to light the cigarette. Turning to his right, he addresses Dom by name and instructs him with an imperatively formatted predicate (\textit{dadhawibu}, line 1) to light the cigarette previously mentioned in \REF{ex:blythe:17}. Dom does not have a clear view of Greg because Ray is sitting between them (see \figref{fig:blythe:10}). After two seconds delay, Dom initiates repair with the “open” interrogative \textit{thangku} \citep{Blythe2015}. Greg does not bother repairing the problematic recruiting move because by this stage, Mike (the target of the request in \extref{ex:blythe:17}), offers to light the cigarette by wiggling the fingers of the hand he is reaching out toward Greg (line 5). Greg then passes him the cigarette (line 6) and Mike lights it on a coal from the fire.

\transheader{ex:blythe:20}{Thuykem 2011 0824\_JB\_video\_GYHM100\_02\_1222731\_1242143}\vspace{-1mm}
%
\begin{mdframednoverticalspace}[style=firstfoc]
\begin{transbox}{1}{gre}
\begin{verbatim}
Dom dadhawibu.
Dom  da                 -dhawibu
♂name 2SG.S.BASH(14).FUT-ignite_cigarette
\end{verbatim}
Dom light the cigarette
\end{transbox}
\end{mdframednoverticalspace}
%
\begin{transbox}{2}{~}
\begin{verbatim}
(2.0)
\end{verbatim}
\end{transbox}
%
\begin{mdframednoverticalspace}[style=secondfoc]
\begin{transbox}{3}{dom}
\begin{verbatim}
[thang[ku.]
\end{verbatim}
\hspace{0.07cm} what?
\end{transbox}
\end{mdframednoverticalspace}
%
\begin{mdframednoverticalspace}[style=secondfoc]
\xtransbox{4}{\fig}{[Figure 10]}
\end{mdframednoverticalspace}
%
\begin{transbox}{5}{mik}
\begin{verbatim}
      [((wiggles fingers))
\end{verbatim}
\end{transbox}
%
\xtransbox{6}{gre}{((passes cigarette to Mike))}
%
\xtransbox{7}{mik}{((lights cigarette from a coal))}\\

\begin{figure}
\includegraphics[width=\textwidth]{figures/murrinhpatha-img10.jpg}
\caption{
As Dom initiates repair (\textit{thangku} ‘what?’), his view of Greg is obscured by Ray (\extref{ex:blythe:20}, line 3).
\label{fig:blythe:10}}
\end{figure}

\subsubsection{Ignoring}\label{sec:blythe:4.2.4}

Of the 78 recruitments that were not promptly complied with, or for which possible compliance was projected as imminent, more than half were not noticeably responded to at all, and thus apparently ignored. That the lack of response should be taken as off-record implicit refusals cannot always be evidenced interactionally, as \REF{ex:blythe:21} demonstrates.

\transheader{ex:blythe:21}{Dingalngu 2011 0730\_JB\_video\_GYHM100\_04\_384070\_389631}\vspace{2mm}
%
\begin{transbox}{1}{kar}
\begin{verbatim}
purrimanukun na panawayu;
purrima    -nukun na  pana-wa  =yu
wHuZi/wBrWi-DAT   TAG RECN-EMPH=CL
\end{verbatim}
those belong to \{your\} purrima ((BrZiWi)).
\end{transbox}
%
\begin{transbox}{2}{~}
\begin{verbatim}
(0.4)
\end{verbatim}
\end{transbox}
%
\begin{mdframednoverticalspace}[style=firstfoc]
\begin{transbox}{3}{kar}
\begin{verbatim}
.hh >nginarr kura ti yawu.<
nginarr   kura     ti  yawu
MBDD/FZDD NC:WATER tea hey!
\end{verbatim}
.hh hey \{daughter\}-in-law, \{more\} tea.
\end{transbox}
\end{mdframednoverticalspace}
%
\begin{transbox}{4}{~}
\begin{verbatim}
(1.5)
\end{verbatim}
\end{transbox}\vspace{-2mm}
%
\begin{transbox}{5}{ali}
\begin{verbatim}
yu ngatin kaya kanyi; (0.4) ↑Aa kanyika ku nyinyiwa;↑
yu ngatin kaya kanyi  aa  kanyi-ka  ku     nyinyi-wa
yeah raw  DEM  PROX   Ah! PROX -TOP NC:ANM 2SG   -EMPH
\end{verbatim}
yeah these are raw, oh! are these yours?
\end{transbox}\bigskip

In \REF{ex:blythe:21} Karen and Alice have been talking about some shellfish they have been eating. At line 2 Karen looks up to see Laura walking in front of her, rejoining the group. Gazing at Laura, Karen addresses her with the kinterm \textit{nginarr} (MBDD, line 3) and requests that she make more tea. Laura continues walking slowly and then sits down where she had been previously been sitting. She does not make any tea, Karen does not pursue a response and tea is not mentioned again for some time. Although we cannot be entirely sure that Laura heard Karen’s recruiting move, the recording reveals clear articulation from Karen and Laura was standing in front of her, right where her voice is being projected. There is no reason therefore to think Laura did not hear it. She appears instead to ignore the request completely.

For other examples, such as \REF{ex:blythe:22}, we can be quite convinced that would-be recruitees refuse to acknowledge the recruiting move, by ignoring the recruiter altogether. At line 1 Dom picks up an empty billycan and peers into it. At line 2 he then targets co-present Mike (by name) and requests water from him with the nominal-hither construction (\textit{kura pathathu kura patha}). Whether Mike actually hears Dom’s request, or merely ignores him, is unclear.\footnote{Dom himself is unclear. His utterances at lines 2, 7 and 12 are all mumbled.\\
} Mike has been engaged in discussion with Bill, an ethnographer, about how much they will be paid for being recorded on camera, a discussion that Bill concludes at line 6, as he walks away from the scene.

\transheader{ex:blythe:22}{20110824\_JB\_video\_GYHM100\_02\_1886635}\vspace{2mm}
%
\xtransbox{1}{dom}{((picks up empty billycan, peers into it, replaces it))}
%
\begin{transbox}{2}{~}
\begin{verbatim}
Mike kura pathathu kura pa↓tha.
Mike  kura     patha-gathu  kura patha
♂name NC:WATER good –hither NC:WATER good
\end{verbatim}
Mike, some fresh water here
\end{transbox}
%
\begin{transbox}{3}{~}
\begin{verbatim}
(0.4)
\end{verbatim}
\end{transbox}
%
\xtransbox{4}{mik}{we'll get two hour Bill.}
%
\begin{transbox}{5}{~}
\begin{verbatim}
(0.2)
\end{verbatim}
\end{transbox}
%
\begin{transbox}{6}{bill}
\begin{verbatim}
ok, (.) [(0.4)       [puyya.
OK                    puy   =ya
OK                    onward=CL
\end{verbatim}
alright \hspace{2.25cm} carry on!
\end{transbox}
%
\begin{transbox}{7}{dom}
\begin{verbatim}
        [kura pathath[u.
         kura     patha-gathu
         NC:WATER good -hither
\end{verbatim}
\hspace{1.25cm} water here
\end{transbox}
%
\begin{transbox}{8}{~}
\begin{verbatim}
                     [((gazes to his left))
\end{verbatim}
\end{transbox}\vspace{-1mm}
%
\begin{transbox}{9}{~}
\begin{verbatim}
(0.3)
\end{verbatim}
\end{transbox}
%
\begin{transbox}{10}{mik}
\begin{verbatim}
alright
\end{verbatim}
\end{transbox}
%
\begin{transbox}{11}{~}
\begin{verbatim}
(0.3)
\end{verbatim}
\end{transbox}
%
\begin{transbox}{12}{dom}
\begin{verbatim}
[mi biskitkathu.
 mi     biskit -gathu
 NC:VEG biscuit-hither
\end{verbatim}
\hspace{0.07cm} give me a biscuit/the biscuits
\end{transbox}
%
\xtransbox{13}{~}{[((touches Ray's leg twice, Figure 11))} %\figref{fig:blythe:11}
%
\xtransbox{14}{ray}{((no response))}

\begin{figure}
\includegraphics[width=\textwidth]{figures/murrinhpatha-img11.jpg}
\caption{
Dom touches Ray on the leg twice (\extref{ex:blythe:22}, line 13).
\label{fig:blythe:11}}\end{figure}

At line 7 Dom redoes the recruiting move with a repetition of the same construction \textit{kura pathathu} ‘water-hither’. He says this whilst gazing at various items located on the ground between Greg and Mike. Thus, in the absence of an explicit vocative, no particular recruitee is being targeted; and there is no uptake from the other young men. At line 13 Dom touches Ray twice on the leg (see \figref{fig:blythe:11}) and, with another nominal-hither construction (\textit{mi biskitkathu}, line 12), requests biscuits from Ray. Ray does not respond and does not move. In order to avoid Dom’s gaze, Ray turns his head slightly to his left, away from Dom who is seated slightly to Ray’s right, but very much within Ray’s “transactional segment” \citep{Kendon1990}.\footnote{An individual’s transactional segment is “the space into which he looks and speaks, into which he reaches to handle objects” (\citealt[211]{Kendon1990}). It encompasses the arc projected 30° either side of the sagittal plane, as radiating out from individual’s lower body (ibid: 212). When Ray tilts his head to the left, he torques his head to the left edge of his transactional segment, relegating Dom to the right periphery of his field of view. Thus, not looking at Dom requires active gaze avoidance on Ray’s behalf.} He is thus actively ignoring Dom. We know nothing of the reason for the non-fulfillment. Evidently, however, this is an utter, albeit implicit, refusal to comply with the request, and a refusal to even acknowledge the requestor’s presence.

In \REF{ex:blythe:23} the rejection implicit in the silence that follows an ignored recruiting move is made explicit when recruitment is then pursued. The extract continues on from where \REF{ex:blythe:15} left off.

\transheader{ex:blythe:23}{Nanthak 2011 0828\_JB\_video\_GYHM100\_03\_453900\_460860}\vspace{2mm}
%
\begin{transbox}{1}{ali}
\begin{verbatim}
[awu ku(h)rdu]nyidham(h)arrarrnukun[:;
 awu kurdu               -nyi       -dhamarrarr -nukun
 no  3SG.S.shove(29).FIRR-1NS.INC.DO-burn_throat-FIRR
\end{verbatim}
\hspace{0.07cm} no! i(h)t might b(h)urn our throats!
\end{transbox}
%
\begin{transbox}{2}{lil}
\begin{verbatim}
[ (h)a:wu;!  ]                     [↓karraya;↓
     awu                             karraya
     no                              goodness!!
\end{verbatim}
\hspace{0.65cm} n(h)o! \hspace{3.85cm} good grief!!
\end{transbox}
%
\begin{transbox}{3}{~}
\begin{verbatim}
(0.7)
\end{verbatim}
\end{transbox}\vspace{-1mm}
%
\begin{mdframednoverticalspace}[style=firstfoc]
\begin{transbox}{4}{lil} %\resizebox{3mm}{!}{\textrightarrow}
\begin{verbatim}
cupwangu nanyekut yawu. (.) haphapnu.
kap       -wangu   na               -nye       -kut
receptacle-thither 2SG.S.grab(9).FUT-1NS.INC.IO-gather
yawu hap-hap -nu
hey! RDP-half-DAT
\end{verbatim}
hey! put it evenly into our cups
\end{transbox}
\end{mdframednoverticalspace}
%
\begin{transbox}{5}{~}
\begin{verbatim}
(1.3) ((Karen pours milk into billycan, Alice ignores Lily))
\end{verbatim}
\end{transbox}
%
\begin{mdframednoverticalspace}[style=firstfoc]
\begin{transbox}{6}{lil} %\resizebox{3mm}{!}{\textrightarrow}
\begin{verbatim}
yawu tepala (0.4) kap!
yawu tepala       kap
hey  deaf         receptacle
\end{verbatim}
hey deaf one! (0.4) cup!
\end{transbox}
\end{mdframednoverticalspace}\vspace{1mm}
%
\begin{mdframednoverticalspace}[style=thirdfoc]
\begin{transbox}{7}{ali} %\resizebox{3mm}{!}{\textrightarrow}
\begin{verbatim}
>KURA PATHAWARRA NGAY YAWU!< (0.3)
kura     patha-warra ngay yawu
NC:WATER good -ahead 1SG  hey
\end{verbatim}
HEY! \{BRING\} ME / I \{WANT\} WATER FIRST
\end{transbox}
\end{mdframednoverticalspace}
%
\begin{transbox}{8}{~}
\begin{verbatim}
(0.8)
\end{verbatim}
\end{transbox}
%
\begin{transbox}{9}{ali}
\begin{verbatim}
[PURDUNYIDHAMA]rrarr↓nu!
 purdu              -nyi       -dhamarrarr -nu
 3SG.S.shove(29).FUT-1NS.INC.DO-burn_throat-FUT
\end{verbatim}
\hspace{0.07cm} it will burn your throat!
\end{transbox}
%
\begin{transbox}{10}{kar}
\begin{verbatim}
[(           )]
\end{verbatim}
\end{transbox}
%
\begin{transbox}{11}{~}
\begin{verbatim}
(0.4)
\end{verbatim}
\end{transbox}\vspace{-1mm}
%
\begin{transbox}{12}{ali}
\begin{verbatim}
dendurr.
dendurr
hot
\end{verbatim}
it’s hot!
\end{transbox}\bigskip

At line 4 Lily instructs Alice (presumably, it is Alice she is gazing at) to ‘put the tea half-and-half into the cups’. Alice does not return Lily’s gaze, nor, while Karen pours milk into the billycan at line 5, does she concern herself with either tea or cups. When Lily at line 6 pursues a response with the interjection \textit{yawu} ‘hey!’ and by addressing Alice as \textit{tepala} ‘deaf one’, she elicits a fiery response from Alice in the form of a shouted counter-recruiting move: ‘HEY! \{BRING\} ME / I \{WANT\} WATER FIRST’, followed by a reason (line 9) for not serving out the tea prematurely (‘it will burn your throat!’). The bald counter-recruiting move (which, incidentally, is also ignored) is neither delayed nor mitigated. In overlap with Alice, Karen at line 8 also shouts something that cannot be discerned. Karen, who at line 5 had been pouring milk into the billycan, like Alice, displays the irritation she had previously suppressed.

That such a substantial number of recruiting moves elicited no response, and are seemingly ignored, is alarming. Although this collection of sequences clearly deserves expanded investigation, it is already evident that “no-response” is to be considered a valid response. In some cases the initiating move is clearly problematic or perhaps difficult to comply with, but in other cases, we can evidently infer that the would-be recruitee considers the substance of the recruitment to not even merit an overt refusal.

\section{Acknowledgment in third position}\label{sec:blythe:5}

Of the languages surveyed in this cross-linguistic project, only Italian and English showed at least some degree of acknowledgment of the recruitments’ fulfillment; most languages had only a few if any \citep{FloydEtAl2018}. There were only three in the Murrinhpatha collection, one being a simple nod, the others being seemingly ad-hoc acknowledgments which I will not elaborate on here.

\section{Social asymmetries}\label{sec:blythe:6}

Most Australian Aboriginal societies are generally held to be egalitarian and non-hierarchical (e.g. \citealt{Flanagan1989,Boehm1993,Peterson1993,}). Social asymmetries are generally not reflected within grammatical contrasts, nor in the choice of lexical items used for address. In this dataset there are only a few occasions that we notice social asymmetry being born out within the interaction. One asymmetry that is brought into play is age, and the seniority that comes with greater experience. Elders are held in great esteem and may be referred to as \textit{pule} ‘respected’/‘boss’. Age related seniority may lie behind Ray’s refusal in \REF{ex:blythe:22} to even acknowledge his pesky younger brother’s existence. Ray is the eldest of a group of brothers and cousins who name themselves after a particular heavy metal band (\citealt{Mansfield2013,mansfield_listening_2014a}). Ray is said to be ‘boss’ for that group.

The by now familiar episode on the beach in which the four women resist fetching water for the hot tea is ultimately resolved when the three eldest women assert their seniority over Rita. In \REF{ex:blythe:24} particularly, Karen launches a sarcastic, melodramatic tirade at Rita.

\transheader{ex:blythe:24}{Nanthak 2011 0828\_JB\_video\_GYHM100\_03\_509480 (simplified transcript)}\vspace{2mm}
%
\begin{transbox}{1}{kar}
\begin{verbatim}
↑ya [KARDU MARDINYBUYKA] panguwardathu kem-
 ya  kardu  mardinybuy-ka  pangu-warda-gathu  kem-
 HES NC:HUM young_girl-TOP DIST –TEMP -hither STRI
\end{verbatim}
\hspace{0.07cm} ah there’s a young girl over there sit-
\end{transbox}
%
\begin{transbox}{2}{~}
\begin{verbatim}
    [((points at Rita))]
\end{verbatim}
\end{transbox}
%
\begin{transbox}{3}{~}
\begin{verbatim}
panguwardathu kemnyekekngime pangu↓wathu.
pangu-warda-gathu  kem               -nye      -kek       -ngime
DIST –TEMP –hither 3SG.S.SIT(1).EXIST-NS.INC.IO-be_rainbow-P.CF.NSIB
pangu-gathu
DIST –hither
\end{verbatim}
sitting over there gleaming at us like a rainbow
\end{transbox}
%
\begin{transbox}{4}{~}
\begin{verbatim}
(0.8)
\end{verbatim}
\end{transbox}
%
\begin{transbox}{5}{kar}
\begin{verbatim}
kardu nekingimedangu (0.4) kardu mani pubernungkardunungime.
kardu  nekingime   -wangu   kardu  mani
NC:HUM 1PC.INC.NSIB-thither NC:HUM similar
pube                     -nu-ngkardu –nu -ngime
1NS.INC.S.BASH.RR(15).FUT-RR-see/look-FUT-PC.F.NSIB
\end{verbatim}
\{facing\} towards us (0.4) like we'll see ourselves \{in the video\}
\end{transbox}
%
\begin{transbox}{6}{~}
\begin{verbatim}
(1.8)
\end{verbatim}
\end{transbox}
%
\begin{transbox}{7}{kar}
\begin{verbatim}
kardu nginipuny mani pubernungkardungime; (0.5)
kardu  nginipuny mani
NC:HUM similar   similar
pube                     -nu-ngkardu –nu -ngime
1NS.INC.S.BASH.RR(15).FUT-RR-see/look-FUT-PC.F.NSIB
\end{verbatim}
it’s like we'll see ourselves \{in the video\}
\end{transbox}
%
\begin{transbox}{8}{~}
\begin{verbatim}
kardu [damnyiwebawaywardangime.
kardu  dam                -nyi       -we  -baway   -warda-ngime
NC:HUM 3SG.S.POKE(19).NFUT-1NS.INC.IO-hair-be_white-TEMP -PC.F.NSIB
\end{verbatim}
with our white hair on our heads
\end{transbox}
%
\begin{transbox}{9}{rit}
\begin{verbatim}
      [((stands up))
\end{verbatim}
\end{transbox}
%
\begin{transbox}{10}{~}
\begin{verbatim}
(1.6)
\end{verbatim}
\end{transbox}\vspace{-1mm}
%
\begin{transbox}{11}{kar}
\begin{verbatim}
ku wakay warda manda warda
ku     wakay  warda manda warda
NC:ANM finish TEMP  near  TEMP
\end{verbatim}
for whom death is near
\end{transbox}\bigskip

Karen contrasts Rita, as young (\textit{kardu mardinybuy} ‘a young girl’, line 1) and radiant (literally, a ‘rainbow’, \textit{kemnyekekngime}, line 3) with the other white-haired women (\textit{damnyiwebawaywardangime}, line 8) with one foot in the grave (\textit{ku wakay warda manda warda} ‘for whom death is near’, line 11). This fanciful comparison breaks the deadlock because Rita gets up (line 9) in order to take the billycan to get some cool water (see \extref{ex:blythe:12}). She has drawn the short straw here as she herself is a grandmother and is Karen’s junior by merely two years!

\section{Discussion}\label{sec:blythe:7}

In most conversation-analytic research on preference structure, dispreferred second pair parts are analyzed in terms of their dispreference features as delayed, hedged, accounted for, etc. An implicit criterion for this approach is detection of the dispreferred second pair part for analysis of these features. An empirical question then is: when an expected response is absent, can its notable absence be legitimately read as a dispreferred response?

When conversation analysis was in its infancy, telephone recording technologies were adopted more widely by conversation analysts than was video. Most of the seminal works on preference organization were conducted on phone call data. Because participants speaking on the phone are not co-located in space, when requests are made, seldom can the substance of the request be fulfilled immediately. Thus phone call requests are normally higher contingency, future actions, for which arrangements need to be made in advance. The substance of the request may well be the actual reason for the call (\citealt{Sacks1992,SchegloffSacks1973,Couper-kuhlen2001}). Usually, the possible imposition on the requestee is foregrounded, becoming the substance of deferential behavior and politeness considerations. Preliminaries need to be dealt with through backgrounding and pre-sequences (\citealt{Schegloff1980,Schegloff2002,Schegloff2007sequence}). However, like each dataset in our comparative project, the Murrinhpatha corpus consists entirely of casual face-to-face conversation amongst friends and family. All of the recruitments call for similarly immediate action to be performed within the general vicinity. A likely outcome of this is that, at least in the Murrinhpatha corpus, there are no pre-recruitment sequences (but see Floyd, \chapref{sec:floyd}, \sectref{sec:floyd:3.3.3}; Rossi, \chapref{sec:rossi}, \sectref{sec:rossi:3.3.3}).

This chapter has presented the Murrinhpatha system of language use pertaining to recruitments. As per the other chapters, it has surveyed the range of possible recruiting formats followed by the array of possible actions and formats in the responding move.

Here I concentrate the discussion on response types and their relative proportions. The payoff in considering response options paradigmatically, as a set of alternatives, is immediately evident (see also \citealt{ThompsonFoxCouperKuhlen2015}). From among the range of possible responses, “no-response” (ignoring) substantially emerges as a legitimate option existing intermediately between overt compliance and overt rejection (see \figref{fig:blythe:12}).\footnote{The denominator has been reduced here from 145 to 139 due to the untypable responses: those where the vocal component of the move is insufficiently audible to be adequately categorized, and/or when the respondent is obscured from view or off-screen.} Extracts \ref{ex:blythe:21}--\ref{ex:blythe:23} show that, at least for Murrinhpatha speakers, silence plus a lack of physical action following a recruiting move can be understood not as a harbinger of imminent refusal, but as actual implicit refusal. There is reason, however, to think that this state of affairs is not culturally specific to Murrinhpatha speakers.

\begin{figure}
% \includegraphics[width=\textwidth]{figures/murrinhpatha-img12.png}
\barplot[xticklabel style={text width = 2cm,
                          align=center},
        every axis x label/.style={
                    at={(0.5,-0.5)
                    }
        },
        ]{← preferred {\rule[.6ex]{2em}{0.6pt}} \textsc{response types} {\rule[.6ex]{2em}{0.6pt}} dispreferred →}{Proportion (\%) of total responses}{%
                                                                                                Prompt compliance,
                                                                                                Projected possible compliance,
                                                                                                OIR,
                                                                                                Implicit rejection,
                                                                                                No-response (ignoring),
                                                                                                Overt rejection%
    }{
  (Prompt compliance, 24)
  (Projected possible compliance, 23)
  (OIR, 4)
  (Implicit rejection, 14)
  (No-response (ignoring), 32)
  (Overt rejection,2)
}
% \todo[inline]{check numbers}

\caption{Relative proportions of response types. Projected possible compliance includes visible behavior that hints at fulfilling the recruitment, as well as explicit commitments to future compliance. Alongside prompt compliance these are the preferred responses. Implicit rejections included counters and deflections, as well as rejections delivered as accounts for non-compliance (see \sectref{sec:blythe:4.2.2}). Pragmatically, ignoring is a “morphologically unrealized” subtype of implicit rejection.}
\label{fig:blythe:12}
\end{figure}

Discussing an example reproduced below as \REF{ex:blythe:23}, Levinson (\citeyear[320-321]{Levinson1983}) demonstrates how a two-second silence following a pre-request is taken to be a negative response to the pre-request. The pre-request deals with the call-taker’s availability, a prerequisite condition for arranging a future meeting.\footnote{Levinson suggests that the two-second silence at line 3, following the caller’s pre-request, is allocated by the turn-taking system to the call-taker, as the next-selected speaker. As such, the call-taker owns the silence. The caller hears the silence as a projecting a dispreferred negative response to the pre-request, which would effectively block the caller’s projected request. Pre-empting the blocking response, the caller answers his/her own question, wrongly as it seems. Having then established the call-taker’s availability, the request eventuates at line 8. “Note here the remarkable power of the turn-taking system to assign the absence of any verbal activity to some particular participant as his turn: such a mechanism can then quite literally make something out of nothing, assigning to a silence or pause, itself devoid of interesting properties, the property of being A’s, or B’s, or neither A’s nor B’s” (\citealt[321]{Levinson1983}).}  The caller’s reading of the silence as conveying unavailability ultimately proved to be unfounded (presumably, the call-taker was actually checking his/her schedule during the silence). Irrespective of the caller drawing the wrong conclusion, the extract illustrates how silence following a specifically allocated first-pair part mobilizes the inferential machinery such that a sub-optimal outcome is imagined.

\transheader{ex:blythe:25}{\citep[320-21]{Levinson1983}}

\begin{transbox}{1}{cal}
\begin{verbatim}
So I was wondering would you be in your office on
\end{verbatim}
\end{transbox}

\begin{transbox}{2}{~}
\begin{verbatim}
Monday (.) by any chance?
\end{verbatim}
\end{transbox}

\begin{transbox}{3}{~}
\begin{verbatim}
(2.0)
\end{verbatim}
\end{transbox}

\begin{transbox}{4}{cal}
\begin{verbatim}
Probably not
\end{verbatim}
\end{transbox}

\begin{transbox}{5}{tak}
\begin{verbatim}
Hmm yes=
\end{verbatim}
\end{transbox}

\begin{transbox}{6}{cal}
\begin{verbatim}
=You would?
\end{verbatim}
\end{transbox}

\begin{transbox}{7}{tak}
\begin{verbatim}
Ya
\end{verbatim}
\end{transbox}

\begin{transbox}{8}{cal}
\begin{verbatim}
So if we came by could you give us ten minutes of your time?
\end{verbatim}
\end{transbox}\\

%\begin{tabbing}
%1 \hspace{0.5em} \=\sc{caller}\hspace{2em} \= { So I was wondering would you be in your office on}\\
%2 \> ~ \> { Monday (.) by any chance?}\\
%3 \> ~ \> { (2.0)}\\
%4 \>\sc{caller} \> { Probably not}\\
%5 \>\sc{call-taker} \> { Hmm yes= }\\
%6 \>\sc{caller} \> { =You would?}\\
%7 \>\sc{call-taker} \> { Ya}\\
%8 \>\sc{caller} \> { So if we came by could you give us ten minutes of your time?}\\
%\end{tabbing}

In the absence of pre-sequences, a no-response following a conditionally relevant first-pair part is hearable \textit{not} as projecting an impending block of a yet-to-emerge base-sequence, but rather as non-fulfillment of, or non-compliance with, the first-pair part of the base sequence. Ignoring is the “zero-morph” of responses to recruitment. No-response is a meaningful declining response that stands in paradigmatic opposition to fulfillment, as one “format” within a range of dispreferred alternative formats that explicitly reject the substance of the recruitment (overt refusals), implicitly reject it (ignoring, counters, deflections, accounts as rejections), or defer the expected base second pair part (OIR).\footnote{In the protracted episode with the hot billycan on the beach, all participants but especially Rita use the full range of these refusal formats to doggedly resist recruitment after recruitment. In this battle of wits, twenty-seven (!) recruiting moves were produced before possible imminent compliance was projected.} The utility of no-response lies in conveying rejection without leaving any on-record token of rejection.

While the rate of non-compliance in Murrinhpatha is high, the rate of no-response is strikingly high. However, we should be careful to interpret these high rates as reflecting a cultural difference, as they might at least partly influenced by the nature of the interactions and people represented in the sample used for this study. Many cases come from interactions among old and relatively infirm participants who are recruited to do things such as lifting heavy water bottles, which requires a high level of physical exertion. Other cases involve demands that are silly or unreasonable, such as Karen’s instructions to Maggie in \REF{ex:blythe:18} that she ask her brother for permission to smoke \citep{BlytheUnpublished}. Nevertheless, the high no-response rate still raises interesting questions, especially for politeness theorists and intercultural communication researchers. If making requests is inherently face-threatening for the requestor, why would Murrinhpatha speaking recruiters risk threats to their positive face when the likelihood of refusal is so substantial? Do cultural expectations based on demand sharing \citep{Peterson1993} diminish potential threats to the recruiter’s positive face such that the chance of refusal merits the risk? Might ignoring recruitments actually be the politest method for declining them? Is ignoring a mechanism for coping with \textit{humbug}?\footnote{\textit{Humbug} is a colloquial Aboriginal English term for the annoying pressure placed on an individual with the intention of eliciting material goods or future deeds. When a person \textit{humbugs} someone, they make persistent demands and requests for such things as food, money, tobacco, and lifts in vehicles; perhaps even performed with a degree of with menace or intimidation (\citealt{Gerrard1989}; \citealt[40–42]{Blythe2001}).} Is the reason many Europeans working in Aboriginal communities feel excessively overburdened by \textit{humbug} \citep{Gerrard1989} because they do not imagine ignoring to be an acceptable option for refusing requests? I will not attempt to answer any of these questions here. However, the fact that they emerge from these results underscores the immense value in taking an emic perspective on social interaction: taking video recordings of informal conversation conducted within a single social group as baseline interactional data; allowing researchers to ground their understanding of cultural expectations upon members’ normative responses to recurrent social actions. Having then compared practices from other social groups, using analogous datasets (as per the approach of pragmatic typology), intercultural communication researchers can draw on these data to better understand communication between participants from different cultural and linguistic backgrounds.

\section*{Abbreviations}

\begin{tabularx}{.49\textwidth}{>{\scshape}lQ}
anaph& anaphoric demonstrative               \\
brziwi& brother’s sister’s wife              \\
cl& clitic                                   \\
cs& classifier stem                          \\
dist& distal demonstrative                   \\
emph& emphatic                               \\
f& feminine                                  \\
fut& future                                  \\
firr& future irrealis                        \\
hes& hesitation                              \\
inc& inclusive of the addressee              \\
ints& intensifier                            \\
loc& locative                                \\
ls& lexical stem                             \\
nc:anm& “animate” noun class                 \\
nc:human& “human” noun class                 \\
nc:pl/t& “place/time” noun class             \\
\end{tabularx}
\begin{tabularx}{.49\textwidth}{>{\scshape}lQ}
nc:res& “residue” noun class                 \\
nc:speech& “speech” noun class               \\
nfut& non-future                             \\
nsib& non-sibling                            \\
ns& non-singular                             \\
pimp& past imperfective                      \\
pc& paucal                                   \\
recn& recognitional demonstrative            \\
s& subject                                   \\
sg& singular                                 \\
sib& sibling                                 \\
stri& same turn initiation of repair         \\
tag& tag particle                            \\
tam& tense/aspect/mood                       \\
temp& temporal adverbial                     \\
top& topic                                   \\
\\
\end{tabularx}

\section*{Acknowledgments}

The corpus was prepared with the consultative assistance of Kinngirri Carmelita Mardigan, Mawurt Ernest Perdjert, Lucy Tcherna, Phyllis Bunduck, Gertrude Nemarlak, Desmond Pupuli and Jeremiah Tunmuck. Thanks also to Mark Crocombe at the Kanamkek Yile Ngala Museum and Languages Centre in Wadeye for logistical support. I am grateful to John Mansfield for making the conversations he recorded available for this collection. The paper has benefited from the comments and advice of Giovanni Rossi, Simeon Floyd, Scott Barnes and two anonymous reviewers. I’m grateful for the lively discussions with fellow collaborators in the HSSLU project – Julija Baranova, Paul Drew, Mark Dingmanse, Tyko Dirksmeyer, Nick Enfield, Simeon Floyd, Rósa Gísladóttir, Kobin Kendrick, Steve Levinson, Ely Manrique and Giovanni Rossi. This research was funded by the European Research Council (StG 240853) and the Australian Research council (DP110100961, DE130100399).

\sloppy
\printbibliography[heading=subbibliography,notkeyword=this]
\end{document}

% \section{Introduction}
%
%
% \transheader{ex:blythe:23}{Nanthak 20110828\_JB\_video\_GYHM100\_03\_453900\_460860}
% \begin{transbox}{1}{Ali}
% \begin{lstlisting}
% [Awu ku(h)rdu]nyidham(h)arrarrnukun[:;
%  Awu kurdu               -nyi       -dhamarrarr -nukun
%  No  3SG.S.shove(29).FIRR-1NS.INC.DO-burn\_throat-FIRR
% \end{lstlisting}\vspace*{-1mm}
%  No! I(h)t might b(h)urn our throats!
% \end{transbox}
%
% \begin{transbox}{2}{Lil}
% \begin{lstlisting}
% [ (h)A:wu;!  ]                     [Karraya;
%     Awu                              karraya
%     no                               goodness!!
% \end{lstlisting}\vspace*{-1mm}
%     N(h)o!                           Good grief!!
% \end{transbox}
%
% \emptytransbox{3}{(0.7)}
%
% \begin{transbox}{4}{Lil}
% \begin{lstlisting}
% Cupwangu nanyekut yawu. (.) haphapnu.
% kap       -wangu   na               -nye       -kut
% receptacle-thither 2SG.S.grab(9).FUT-1NS.INC.IO-gather
% yawu hap-hap -nu
% hey! RDP-half-DAT
% \end{lstlisting}\vspace*{-1mm}
% Hey! Put it evenly into our cups.
% \end{transbox}
%
% \emptytransbox{5}{(1.3) ((Karen pours milk into billycan, Alice ignores Lily)) }
%
%
%
% \begin{mdframednoverticalspace}[style=firstfoc]
% \begin{transbox}{6}{Lil}
% \begin{lstlisting}
% Yawu tepala (0.4) Kap!
% yawu tepala kap
% Hey  deaf   receptacle
% \end{lstlisting}\vspace*{-1mm}
% Hey deaf one! (0.4) Cup!
% \end{transbox}
% \end{mdframednoverticalspace}
%
%
%
% \begin{mdframednoverticalspace}[style=secondfoc]
% \begin{transbox}{7}{Ali}
% \begin{lstlisting}
% >KURA PATHAWARRA NGAY YAWU!< (0.3)
% kura     patha-warra ngay yawu
% NC:WATER good -ahead 1SG  hey
% \end{lstlisting}\vspace*{-1mm}
% HEY! \{Bring] me/I \{want\} water first!!!
% \end{transbox}
% \end{mdframednoverticalspace}
%
%
% \emptytransbox{8}{(0.8)}
%
% \begin{transbox}{9}{Ali}
% \begin{lstlisting}
% [PURDUNYIDHAMA]rrarr↓nu!
% purdu              -nyi       -dhamarrarr -nu
% 3SG.S.shove(29).FUT-1NS.INC.DO-burn_throat-FUT
% \end{lstlisting}\vspace*{-1mm}
% It will burn your throat!
% \end{transbox}
%
%
% \xtransbox{8}{Kar}{[XXXXXXXXXXXXX]}
%
% \emptytransbox{8}{(0.4)}
%
% \begin{transbox}{9}{Ali}
% \begin{lstlisting}
% Dendurr.
% dendurr
% hot
% \end{lstlisting}\vspace*{-1mm}
% It’s hot!
% \end{transbox}

