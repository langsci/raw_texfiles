\documentclass[output=paper]{langsci/langscibook} 
\ChapterDOI{10.5281/zenodo.4450083}

\author{Madiha Kassawat\affiliation{Université Sorbonne Nouvelle (ESIT)}} 
\title[The internationalized text and its localized variations]{The internationalized text and its localized variations: A parallel analysis of blurbs localized from English into Arabic and French} 

\abstract{In an increasingly globalized world, accessibility to digital content has become indispensable for people around the world. This accessibility would not be possible without translation which plays an important role in linguistic and cultural mediation, as well as in marketing. As the majority of products is promoted for and sold on the internet, their web pages are often localized based on the market, including the language and the culture. The required speed in this type of work, its tools and process play a remarkable role which influences the quality of the localized texts. Therefore, it is necessary to analyze these texts, explore the different interpretations of a text in several languages and cultures, and the adaptation level which should convince the consumer to purchase the product. This pilot study is an attempt to compare the product descriptions provided in English and localized into Arabic and several French versions. The results show the relationship between the international text and the localized texts on the linguistic and cultural levels.}
\glottocodes{stan1293,stan1290,arab1395}
\begin{document}
\maketitle

\section{Introduction}

We are living in a globalized world where translation is a daily-lived practice \parencite{ladmiral14}. The majority of translated texts nowadays are not literary but utilitarian \parencite{ledisez04}. At the same time, the digital environment is considered a decisive channel of marketing. It absorbs increasingly between 20 to 30 percent of advertising expenses \parencite{steboulaj14}. From this perspective, localization plays an important role in enriching the digital content, particularly websites which are used for different purposes. This role is manifested by the various linguistic versions of the same product whose target audience is large and diversified. Such a big audience makes the localized product more visible to the public and more accessible on the internet than the printed products \parencite{jimenez13}. This, in turn, makes the product (the website in this case) a cultural object which conveys cultural markers \parencite{remon05}.

Localization started in the 1990s when software localization was most common, including the software content and its printed or online help text \parencite{esselink03}. The change and use of websites have resulted in the idea of content localization which focuses on the linguistic skills more than on the technical ones \parencite{esselink03}. In addition, the year 1995 witnessed the first official advertisements on commercial websites of corporates such as Amazon, eBay and Yahoo! \parencite{steboulaj14}. Digital marketing has become a popular tendency that the majority of international corporations have adopted since. It is even considered an indispensable strategy for product distribution at the global level.

First, this paper will attempt to discuss the \textit{locale} as an elastic term which can represent a country or a territory, or a cluster of countries or territories. This should explain the necessity of cultural considerations in the target market as specified by the client. Second, the process of website localization will be briefly explained, while highlighting the internationalization phase and its goals regarding the adaptation of the localized text. Third, a brief discussion will take place on the applied theories of \enquote{traditional} advertising translation. This topic will be necessary to reframe these theories in a relatively recent industry such as localization. The paper will attempt to answer the question of whether and how the localized text is adapted for the different cultures. It examines adaptation through comparing how the linguistic and cultural elements have been treated in the Arabic and French versions, taking the international English text as a reference.

\section{Localization: A locale and a process}

\subsection{The amalgamation of the locale}

The \textit{locale} is a term used in the industry. It combines a language variety and cultural norms using the market criteria to resolve contradictions between sociolinguistic levels \parencite{pym05}. These criteria can include language, currency, and the consumers’ education level or their revenues, based on the communication nature \parencite{pym11}. By looking at the international corporate website addresses, each one has an identifier that is usually a combination of the country/region and the language. For example, the code of a website whose target country is France can include fr-fr in its URL, and an Arabic site for Egypt can include eg-ar, and en-uk for a site targeting the UK, etc.

However, this strategy is not always as specific or organized as it appears. Not all countries have the same opportunity of being offered a culturally adapted website if they do not represent significant markets. This is where amalgamation of the different geographic denominations happens instead of using country codes. For example, \citet{jimenez10} found that the main target hispanic markets are Spain (42\%), Mexico (32\%), the United States (27\%) and Argentina (27\%). This tendency of generalizing the communication language of different cultural communities highlights the marketing approach which focuses on the \textcquote{pym00}{languages of consumption} instead of languages belonging to specific cultures. In addition, the geography that is associated to language is unspecific \parencite[26--27]{guidere00}. This geography is supposed to specify linguistic, cultural and economic peculiarities \parencite[29]{guidere00}.

\subsection{From internationalization to localization}

Localization is a necessary process for migrating information to other sites, where languages other than the original language of the content are used \parencite[28--29]{cronin06}. A typical localization project passes through three phases: the project preparation, translation and quality assurance \parencite[114]{quah06}. However, \textit{localization} may often be used as a general term without mentioning the general life cycle of the product. Localization is one of the GILT phases (Globalization, Internationalization, Localization and Translation) \parencite{munday08}. Globalization is the encompassing cycle of the product, where internationalization includes planning and preparing the product, while localization is the actual adaptation of the product for its target market \parencite{anasch10}. As for internationalization, it is about adapting the products to facilitate their localization in the international markets \parencite{esselink03}. The central aspect of internationalization, as pointed out by \textcite{esselink00}, is displaying the characters according to the local standards of the target locale. For example, double-byte character compatibility should be provided prior to the product translation \parencite[3]{esselink00}. In other words, the product should be \enquote{enabled} in order to be usable in certain countries and regions \parencite[2]{esselink98}. More generally, this phase necessitates removing any cultural, linguistic, technical, religious, philosophical, value-related peculiarities from the product and its complements \parencite{gouadec03}.

This extraction of the language- and culture-dependent elements during the internationalization phase has also been highlighted by \textcite{schaeler07}. Hence, internationalization is not only applied to the technical aspects of the product but also to its textual content. The internationalized text should facilitate the transfer to the maximum number of languages without producing any complications \parencite[26]{jimenez13}. This phase includes the source text pre-editing as well. It is used as a form of quality assurance and limiting the cost to meet the need for translating into several languages. In his discussion of the notion of the internationalized text, Pym uses the term \enquote{one-to-many geometry} versus \enquote{language-into-language situations} which is adopted in translation to refer to the source text and the target text \parencite*{pym06}. This strategy is adopted particularly with the product launch simultaneously or successively, in several languages and multiple countries, and within a very short time around \parencite[45]{quah06}.

After text internationalization, the localization phase consists of the linguistic and cultural adaptation of the text in order to distribute the digital products and services independently of the characteristics of the original country \parencite{schaeler07}. The Localization Industry Standards Association, which was deprecated in 2011, provided a definition of localization that \enquote{involves taking a product and making it linguistically and culturally appropriate to the target locale (country/region and language) where it will be used and sold} \parencites[LISA 2003: 13, in][]{jimenez13}[cf.][]{yunker03}. The Globalization and Localization Association (GALA) explains that \textcquote{gala19}{the aim of localization is to give a product the look and feel of having been created specifically for a target market, no matter their language, culture, or location}.

Localization is sometimes viewed as a practice that goes beyond translation due to the adaptation to the culture of the target text \parencite{anasch10}, and to the fact that it includes technical aspects in addition to the \enquote{traditional translation} tasks \parencite{austermuehl06}. However, adaptation is required in translation \parencite{nord05}. This is underlined in another definition of localization which shows that it is a type of functionalist translation whose goal is the communicative purpose: \blockcquote[18]{jimenez13}{Localization is therefore conceptualized as a target-oriented translation type and, in line with the functionalist notion of adequacy, emphasizes users’ expectations and achieving the communicative purpose for which the localization was commissioned, rather than equivalence relationships to source texts (STs).}

Despite the importance of the technical aspects in localization, whether on the language agents' or engineers' side \parencite[concerning localization agents, cf.][]{canim17}, this paper focuses on the linguistic and cultural adaptation in the localized text and how its function is treated in the frame of the aforementioned definitions of localization.

\subsection{From an international to a local text: contradictory phases}

It is important to point out the contradiction between the product globalization phases. The process starts with internationalization and filtering the cultural references of a product in order to make it look locally made. This area has been highlighted by Jiménez-Crespo who hints to this contradiction in the localization industry discourse. On the one hand, localization aims to make websites give the impression that they have been created in the target country. On the other hand, internationalization neutralizes the products in terms of language and culture \parencite{jimenez10}. He finds that internationalizing a communication has direct consequences on the languages and the translation process itself \parencite[10]{jimenez13}. Therefore, it is intriguing to analyze the potential internationalization impact on the localized product, on the adaptation level in particular.

\section{Advertising translation: Which function in localization?}

The localization definitions discussed earlier show that the text function is essential. It has a direct relationship with the communicative purpose of the products and the inter-linguistic and intercultural approach of marketing. In looking at the text as a marketing tool for a product, the appellative intention should be the keyword to persuade the recipient to adopt a certain opinion or perform a certain activity \parencites{nord05}[cf.][]{tatilon90}{boivineau72}. In the case of advertising translation for example, the persuasive function of the message is pivotal and the distinction between the source text and the translated one becomes difficult \parencite{cruz18} due to its reformulation. This intention should be well explained in the \enquote{brief} as it will be received and read before reading the source text \parencite{nord05}. Moreover, it is the client’s desired effect which determines the translation strategy. As put by \textcite[76]{ladmiral14}, the \enquote{sourciers} are those who translate based on the source text, while the \enquote{ciblistes} concentrate on the message and the effect to be translated. For the latter, they make use of all the available tools and ways of the target language. Such an effect is only achieved through extracting all the elements which can shock the consumer regarding his/her beliefs, feelings, traditions, attitudes, customs and anything related to his/her cultural package \parencite{tatilon90}. The advert effect has been studied by \textcite{gully96} who analyzed Egyptian Arabic advertisements of TV, radio and magazines. He found different strategies used for persuasion such as the use of metaphors, rhetorical expressions and the local dialect. Having said that, persuasion can be achieved through the use of references from the target language and culture.

\section{The translator-localizer and the audience\dots\ cultural packages}

Translation strategies differ based on the guidelines, and can vary more depending on the translator’s preferences and knowledge as a first reader of the text \parencite{plassard07}. According to \textcite{munday08}, the interpretative theory in translation identifies three phases of the process: comprehension, de-verbalization and re-expression. The result is an association of the linguistic and non-linguistic sets of knowledge. Therefore, this association can be understood as a cube: it can generate multiple possibilities of translations of the same text. This can be complemented by the fact that language is related to culture. In the Onion Model \parencite{hofetal10}, culture consists of two layers. The first represents practices (symbols, heroes and rituals). The second represents values. Although words exist at the surface of culture, within symbols, they are considered the vehicles of cultural transfer \parencite{hofetal10}. Hence, the translator’s cultural package has an unavoidable impact on the translation, which is the final result that encompasses the linguistic and cultural knowledge.

The language-culture combination is done by the translator-localizer in this case. At the same time, readers (users of the localized products) \textcquote[96]{nord05}{who want to \enquote{understand}, have to connect or associate the new information given by the text with the knowledge of the world already stored in their memories}. Here comes the importance of adapting the information which can be \enquote{trivial} for the source text recipients, depending on their own cultural package, but can be unknown to the target text audience \parencite[107]{nord05}. In other words, there is a need to fill in the gaps that exist in the recipient’s knowledge \parencite{baker11}.

This adaptation necessitates a transformation, a mediation and a change \parencite[159]{maitland17} as \textcquote[07]{maitland17}{the articulation of another’s experience in one’s own words requires the importation of other ideas, other viewpoints, other worldviews}. Moreover, and although a person belongs to a specific culture, they do not know all the aspects of that culture \parencite[42]{gudykunst04}. Their point of view will definitely differ from that of another. Such differences in viewing and understanding the world lead to multiple variations and possibilities in the translation. 

\section{The loyalty to the translation and the purpose of the product}

Coming back to localization, adaptation is sometimes considered an additional element introduced by localization, as opposed to the literal nature of translation: \blockcquote[15]{jimenez13}{Adaptations are seen as the additional component that localization provides, as opposed to the textual or wordly nature of \enquote{translation}. The term \textit{adaptation} is typically used to indicate the performative action of the localization process}.

Nonetheless, adaptation is just a modification procedure, besides transposition, which tends towards the literality, and re-writing \parencite{guidere00}. As for adaptation, it can be formal, where it affects the structure of the original statement, or idea-based to meet the cultural expectations of the target recipients \parencite[124]{guidere00}. On the contrary, re-writing tends to provide a different expressive orientation to the message initial idea \parencite[129]{guidere00}. It should be indicated here that the term \textit{re-writing} reminds us with a more commonly used term nowadays particularly by translation agencies but which has taken its position in Translation Studies as well: transcreation \parencites[cf.][]{pedersen14}{katan16}. In discussing the \textit{skopos} of multilingual communication, \textcite[17]{guidere08} sets two main rules: the coherence rule and the loyalty rule, which indicates the need to keep a sufficient relationship between the target text and the source message in order to not consider the translation as too literal. 

Such a perspective seems to limit the \textit{skopos} which is based on the text function. For example, in the case of the localized text, the user, and even the client, would not necessarily be interested in the loyalty \textit{per se}. It is the purpose of the product and its usage that determine the strategy. The translation strategy can require liberty in translating in a way to make the product suitable for the target culture and the message effect similar on the audience in question. This need for liberty becomes necessary when the target language and culture do not have the words which express certain concepts, or when these concepts are absent in the life of the other nation \parencite[54]{ranzato16}. Having a distance from the source text is also important to avoid providing a target text which sounds like a translation and has a heavy style and is difficult to read \parencite{boivineau72}.

\section{A trilingual localization pilot study}

In order to shed light on the variations which result from the different translation strategies, two variables will be taken into account: the language combination and the culture based on the country. The selected corpus of this pilot study is multilingual \parencite{olohan04}. It was selected from three international corporate web pages that are localized in several languages. Three websites are included in the corpus, one per industry. The texts are informative and commercial, and describe cosmetic, technology and furniture products. A text consists of a tagline, a subtitle and\slash or a short description. The structure of the text is similar across the analyzed websites, although its length can vary slightly.

Given the lack of access to the \textcquote{pym04}{internal knowledge}, including the source language, the international English version is taken as a reference only. The term \textit{reference text} will therefore be used instead of \textit{source text}. The target locales in this study are Arabic (Saudi Arabia, as a representative target market on several websites) and French (Canada, France, Switzerland and Morocco). Analyzing texts in Arabic and French, besides country-based variations, should help in exploring the different cultural interpretations compared to the international text and the possible adopted approaches in the translation. This method has been also used in a study which analyzes the \enquote{uniformization} level through internationalizing the linguistic content or the affirmation of the cultural differences \parencite{boucai06}.

On the one hand, the analysis focuses on the adaptation level, the difference or similarity between the translation and the international version. On the other hand, it distinguishes the cultural points of view in each localized version and how the same message was interpreted. The analysis takes random samples from the selected websites and excludes the reasoning of the strategy, whether it is stylistic, cultural, intuitive or personal. It focuses on meeting the function of the text rather than accuracy, particularly that online versions might not be updated simultaneously, which can create discrepancies in meaning.

\section{Results and discussion}

The analysis of the selected websites shows a variety in the used strategies and procedures from a version to another, in relation to the reference text. The differences take place at the level of the linguistic and cultural elements. The localized versions sometimes meet the expectations of the studied genre and the required results in localization, such as the use of adaptation, locale-specific expressions and stylistic choices. These cases reflect the understanding and interpretation of the translator (or the agent dealing with the text) as a reader of the text \parencite{plassard07}. They also correspond to the knowledge of the world that the target readers have \parencite[96]{nord05}, and attempt to fill in the gaps that exist in their knowledge \parencite{baker11}.

Moreover, it has been noted that when the international text does not achieve the \enquote{\itshape zéro spécifique} (`the specific zero') \parencite{gouadec03}, but rather introduces cul\-ture-specific structures, the other versions introduce their own. However, when the international version is neutralized by extracting language- and culture-spe\-cif\-ic elements \parencite{schaeler07}, it often leads to similar translations that can be considered comprehensible in the target locale but miss the local voice and the dynamism which is encouraged in this kind of translations \parencite{tatilon90}. This lack of dynamism can reduce the possibilities of convincing the consumer due to the gap between the localized text and the common advertising text features in the consumer’s language and culture. Furthermore, the appellative aspect \parencite{nord05} is not always present in the translations, although it is a necessary one for achieving the desired effect \parencite{ladmiral14}.

Having said that, the use of linguistic and non-linguistic sets of knowledge has not been always observed. Sometimes, the translation did not go beyond the Practices layer in Hofstede's Onion Model. It rather stayed limited to the Symbols layer (which includes words), conveying a wording similar to the international and other versions. From the localization industry perspective, adaptation and the offer of the \enquote{look and feel} of the target country are the \enquote{additional} characteristics in localization \parencite[15]{jimenez13}. However, the translations seem to be influenced by several factors, including the localization process, the degree of internationalization, or whether this phase was applied or not. This interpretation corresponds to what Jiménez-Crespo pointed out with regard to the consequences of internationalization on the language and the translation process \parencite*[10]{jimenez13}.

Regarding the locale amalgamation discussed earlier, creative variations have been observed among the French versions. This does not necessarily suggest that the variations are specifically done for the target country/culture. Many expressions used are common in different French-speaking countries. Their use depends on the intuitive of the translator-localizer and his/her interpretation associated to his/her cultural package. As for the Arabic translations, the same copies were used for several Arab countries regardless of how many similarities and differences they may have, which reflects the notion of \textcquote{pym00}{languages of consumption}. That also shows how the geographic location and language association remains unspecific \parencite[26--27]{guidere00} and subject to the marketing strategy. The analyzed translations can be generally considered neutral and suitable without introducing shocking linguistic or cultural aspects \parencite{tatilon90}. Even so, there are cases where more suitable and target-oriented adaptations could have been applied, and more culture-related aspects could have been included, as illustrated in the following paragraphs.

\subsection{From untranslatability to creativity}

The first example [T1] is a description of a mobile phone camera. The example includes the Arabic version of the Saudi Arabia page and the French version of Morocco: 

International English version:

\begin{center}
  Super Slow-mo

  The camera that slows down time,
  
  making everyday moments epic.
\end{center}

Arabic version – Saudi Arabia

\begin{center}
    
    \arabtext{ميزة الحركة البطيئة جدا}

    \arabtext{تتباطأ اللقطات،}

    \arabtext{لتعيش اللحظات.}
\end{center}

\begin{center}
  ‘myzat alḥaraka albaṭy’a ǧiddan

  tatabaṭa’ allaqaṭāt,

  litaʿyš allaḥaẓāt’
\end{center}

French version – Morocco:

\begin{center}
  Super Slow-mo

  La caméra qui ralentit le temps,
  
  rendre les moments quotidiens épiques.
\end{center}

The Arabic translation does not have similar words to the English text: \enquote{the function of the very slow motion {\textbar} the footage slows down, to let the moments live}. On the contrary, the French version (Morocco) is more similar to the international one; even the function name \enquote{Super slow-mo} is in English. Having said that, the similarity is not viewed as a bad aspect but the translation lacks cultural references which can enrich the translation. The Arabic translation of this example seems to be creative in terms of the use of rhymes (footage/moments): \textit{laqaṭāt} and \textit{laḥaẓāt}. This stylistic choice is often used in Arabic audiovisual advertising as well \parencite{gully96}.

In the above example, creativity is demonstrated by the use of stylistic elements and cultural references from Arabic; while similarity to the international version was noticed in the French version. It is important to point out that the Moroccan version is available only in French on the website of this example.

The second example [T2] is a description of a mobile phone. Although culture- and language-specific references were used in both French versions, each one employed a different metaphor. 

International English version:

\begin{center}
  It doesn't just stand out. It stands apart.

  Completely redesigned to remove interruptions.

  No notch, no distractions. Precise laser cutting and a Dynamic AMOLED screen that's easy on the eyes make the Infinity Display our most innovative yet.
\end{center}

French version – France:

\begin{center}
  Il atteint de nouveaux sommets

  Vous pensiez savoir à quoi ressemble un smartphone ? Écran Infinity nouvelle génération, lecteur d'empreinte sous l'écran, technologie Dynamic AMOLED : l'écran du Galaxy S10 est une fenêtre vers le futur.
\end{center}

French version – Switzerland:

\begin{center}
  Le téléphone qui sort résolument du lot

  Un design entièrement repensé pour que rien ne vienne perturber votre vue. Pas d'encoche, pas de distractions visuelles. Grâce à la découpe laser précise, au dispositif de sécurité par reconnaissance digitale sous l'écran et à la technologie Dynamic AMOLED qui est un régal pour les yeux, l'Infinity Display est l'écran Galaxy le plus innovant jamais conçu.
\end{center}

In the French version of France, the image used for distinguishing the product from others is associated to the height and the progress achieved by the product. As for the Swiss version, the distinction is represented with regard to a group of similar objects. This choice is linguistic metaphoric, stylistic and cultural. The metaphors and the expressions address the consumers directly through using references from their cultures, i.e. elements that exist in their knowledge and memory \parencite[96]{nord05}. Although both choices seem common in the French and Swiss cultures, the variations enrich both cultures and make the product closer to its consumers. It is noticed that the descriptions differ from the international version as well. There are additions in both versions as well, probably due to updating the content or the differences in the guidelines.

\subsection{Simplicity: is it easier to translate?}

The translations of the example below [T3] are semantically and syntactically close to the international text, besides using a similar style and cultural neutrality. The text is a short and simple description of a piece of furniture: 

International English version:

\begin{center}
  SHOE STORAGE, COAT AND HAT RACKS

  Coat, hat, shoes and go!

  How do you get the hallway to be that stumble-free, get-ready-in-the-morning-without-thinking part of your everyday? Our different styles of shoe storage and coat and hat racks help make your outdoor things easy to get at without using up too much of your space.
\end{center}

Arabic version – Saudi Arabia:

\begin{center}
  
  \arabtext{خزائن الأحذية ورفوف المعاطف والقبعات}

  \arabtext{معطف وقبعة وحذاء وانطلق!}

  \arabtext{كيف تحصل على مدخل خالي من الفوضى وتكون جاهزًا في الصباح دون أن تشغل بالك في جزء منه كل يوم؟ لدينا أشكال}

  \arabtext{مختلفة من خزائن الأحذية ورفوف المعاطف والقبعات تُساعد في ترتيب أشيائك الخارجية لسرعة الوصول إليها دون شغل}

  \arabtext{مساحة كبيرة من المكان.}
\end{center}

\begin{center}
  ‘ẖaza’en al’aḥḏya wa rufūf almaʿātef w alqubbaʿāt

  miʿṭaf wa qubbʿa wa ḥiḏāʾ w ānṭaleq!

  kayfa taḥṣal ʿala madẖal ẖāly min alfawḍa wa takwn ǧāhizan fy ṣṣabāḥ dwna an tašġal bālak fy

  ǧuzʾn minhu kulla yawm? ladynā aškālun muẖtalifa min ẖazāin ālaḥḏya wa rufūf almʿāṭif w

  ālqubbʿāt tusāʿdu fy tartyb ašyā'ka alẖāriǧyya lisurʿat ālwuṣūl ilayhā dūna šaġli masāḥa kabyra

  min almakān.’
\end{center}

French version – Canada: 

\begin{center}
  Étagères pour manteaux, chaussures et chapeaux

  Chaussures, manteau, chapeau, c'est parti!

  Une entrée ordonnée et dégagée relève du fantasme chez vous? Vous rêvez de partir le matin sans perdre du temps à chercher? Nos range-chaussures et portemanteaux de styles variés gardent vos vêtements d'extérieur bien rangés sans occuper beaucoup d'espace.
\end{center}

The international text does not have visible cultural references and should be suitable for all the cultures, assuming that shoes, coats and hats are widely used. Moreover, the second phrase shows the function of the product as a suitable object for a hectic life style. This was translated similarly in both versions and was supposed to be sufficient. However, the elements used in the example do not necessarily suit all the cultures. Although the majority of the Arab countries have similar dressing habits, wearing hats and coats for example is not a traditional custom for either men or women in Saudi Arabia and other Gulf countries.

Therefore, adopting a neutral and simple approach might not be sound for these societies as the cultural references are not associated with their culture and habits. These elements could have been adapted for the target culture using the Saudi coat \textit{bichte}, the \textit{shemagh}, the \textit{agale}, etc. It is important to shed light on the fact that the website provides the same version for all the Arab countries. Furthermore, \textit{get-ready-in-the-morning-without-thinking part of your everyday} was translated in an incomprehensible way in Arabic, compared to the French version of Canada. In Arabic, the translation says: \enquote{being ready in the morning without thinking about a part of it} where \textit{it} can relate to \textit{the entrance} or \textit{the morning}, which is not a clear structure.

On the contrary, the French version presents a creative structure that is shorter, more persuasive and readable through introducing a question \textit{Une entrée ordonnée et dégagée relève du fantasme chez vous?} which means \enquote{Is an organized and tidy entrance a fantasy in your home?}. The Arabic version can be explained by the influence of its source, whether it was the English text or another version, although there are other factors which are not the subject of this article. Finally, \textit{coat and hat racks} were translated as \enquote{coat and hat shelves} in Arabic, using words that do not make much sense as coats and hats are normally not stored on shelves. 

\subsection{The problematic untranslatable}

The last example [T4] is a description of a cosmetic product. The product name is associated to the American culture, which complicates the localization further. This case requires a detailed \enquote{brief} on the treatment of this kind of issues. Nonetheless, this article focuses on the final localized product at the adaptation level.

The international English version:

\begin{center}
  GALifornia powder blush

  sunny golden pink blush

  GALifornia dreamin!

  Benefit’s NEW GALifornia golden pink blush is part sun, pure fun! It blends bright pink with shimmering gold, for a sunkissed glow that complements all skintones. The soft, blendable formula captures the warmth of California sunshine, while the signature scent features notes of pink grapefruit \& vanilla.
\end{center}

The Arabic version – Saudi Arabia:

\begin{center}

  GALifornia \arabtext{مستحضر}

  \arabtext{أحمر خدود ذهبي مسم}

  \arabtext{من منّا لا تحبّ إطلالة الفتاة الكاليفورنية!}

  \arabtext{الجديد من بنفت روح المرح والإشراقة المشمسة! إنه يجمع بين اللّون} GALifornia \arabtext{يعطي أحمر الخدود الوردي الذهبي}

  GALifornia \arabtext{الوردي المشرق والذهبي المتلألئ ليحتضن توهج شمس كاليفورنيا الدافئة في علبة. تتميز رائحة مستحضر}

  \arabtext{بنفحات فاكهة الجريب فروت الوردية والفانيليا. يأتي هذا المستحضر مع فرشاة خاصة مستديرة الشكل لتطبيق ناعم ومتناسق.}
\end{center}

\begin{center}
  ‘mustaḥḍar GALifornia

  aḥmar ẖudūd ḏahaby musmr

  man minnā lā tuḥib iṭlālaẗ alfatāẗ alkalyfurnyya!

  yuʿṭy aḥmar ālẖudūd alwardy alḏahby GALifornia alǧadyd min benefit rūḥ almaraḥ wal išrāqa

  almušmisa! innahu yaǧmaʿ bayn allawn alwardy almušriq wa ḏḏahaby almutal'le' lyaḥtaḍina

  tawhhuǧa šams kālyfūrnyā aldāfe'a fy ʿlba. tatamayyaz rā'eḥat mustaḥḍar Galifornia binafaḥāt

  fākihaẗ alǧryb frūt alwardyya w alfanylyā. ya'ty haḏā almustaḥḍar mʿ furšā ẖāṣa mustadyraẗ

  alškl litaṭbyqin nāʿem wa mutanāsiq.’
\end{center}

The French version – France:

\begin{center}
  GALifornia {\textbar} blush poudre soleil rose doré

  Le soleil californien dans un bel écrin.

  Les GALifornia girls, tout le monde les adore !

  Le NOUVEAU blush rose doré GALifornia de Benefit, c'est une dose d'éclat ensoleillé, adoptez-le ! Il mélange le rose vif et l'or chatoyant, capturant la lumière du soleil californien dans un poudrier. Le parfum envoûtant de GALifornia contient des notes de pamplemousse rose et de vanille. Inclut un pinceau blush à bout arrondi sur mesure pour une application diffuse et tout en douceur.
\end{center}

In this example, the product name is associated to California and has a play on words, where the first letters of \textit{girl} and \textit{California} are merged. It is obvious that the name choice has marketing purposes as it is used in English in all the analyzed versions, but the description should have provided more clarification. The French translation had the liberty to transform the part \textit{GALifornia dreamin!} It adds specific references to \textit{California}: the sun. The Arabic translation though seems to be a re-writing as it provides a different expressive orientation to the message \parencite[129]{guidere00}. Nonetheless, it remains as ambiguous as the reference text: \enquote{Who among us doesn’t like the style of the Californian girl!} What the Californian girl refers to in terms of beauty is not clear for Arab women. Although foreign names are used for marketing purposes, California does not represent a particular reference in the Arab culture. The name would need more explanation or another name should have been used to suit the target locale.

In addition, the product type \textit{powder blush} was replaced by the word \textit{product} simply in the Arabic tagline and description, but was included in the subtitle. The pink color was adapted to brown to suit the image associated with the sun in desert-like environments, which is also the Arab women’s skin color in general. As for the description, the part \textit{a sunkissed glow} was adapted in both versions for different reasons. In Arabic, \textit{kissing} was replaced by \textit{hugging} to avoid sexual connotations. In the French version, the expression was adapted into \enquote{catching the sunlight}, perhaps due to the lack of a similar expression in the French language.

\section{Conclusion}

It can be noticed from the examples discussed above that several translation strategies were used, including different interpretations generated from the same strategy. Moreover, the treatment of the texts does not always take the cultural peculiarity of the audience into account. A website version is typically influenced by the general expectations of the consumers. The examples illustrated translation and creative variations which represent the fruit of cultural diversity, whether among the translators-localizers or the recipients who have their different cultural packages even when they share the same language. However, this creativity was not always present.

The localization process and that of internationalization should have an effect on the adaptation. As for the effect of internationalization, there was a noticeable simplicity and neutrality at the language and culture levels in several localized versions, where the English and localized versions had semantic and syntactic similarities. In addition, the international text is not always culturally filtered. Having said that, it might not be as helpful as required even for an international diverse audience. This is clear particularly with the product names and metaphors. Such an approach is sometimes necessary for marketing, but it introduces obstacles during the translation and adaptation of the text.

Moreover, the translator-localizer has the liberty to transform the text, adapt it and make it comprehensible for the consumers. Leaving a leeway of change to the translator should put internationalization in question with regard to its goal of helping in the product localization, i.e. the internationalized text seems to have two contradictory functions: reducing the adaptation time and encouraging cultural adaptation. This contradiction can also be associated to the international-to-local approach of the process \parencite{jimenez10}. While this study does not look into the internationalized function per se, it is important to explore its effect on the quality of the translated text and to pay more attention to this area in research.

With regard to the employed methodology in this pilot study, it provided a general idea of the existing adaptation practices. However, a more detailed analysis is needed, particularly for each locale and industry. A detailed study can reveal the strengths and pain points in a localized version more accurately. The use of the notion of the \enquote{reference text} was necessary as the researcher cannot know the source language used to localize a website. Knowing the source language or version would have helped in providing a deeper interpretation of the spotted practices. Although this obstacle limited the analysis to a certain extent, it helped the researcher avoid the comparison with and the influence of a source text. Moreover, the study focused on the adaptation aspect, which is target-related in the first place.

\hspace*{-1mm}To conclude, this paper attempted to explore how localized versions are adapted by comparing the variations at the country and culture levels, and highlighting the cultural richness these variations can bring if adaptation is used in localizing the textual content. The method used has provided both a multi-lingual and multi-cultural approach which goes beyond words and encompasses the product personalization in order to represent the local culture. This pilot study can contribute in more culture- and language-oriented research in website localization, which often focuses on technical aspects.

\subsection*{Analyzed texts}

[T1, T2] Available on \url{http://www.samsung.com}, [accessed on 22 April 2019]\\{}
[T3] Available on \url{https://www.ikea.com}, [accessed on 18 April 2019]\\{}
[T4] Available on \url{https://www.benefitcosmetics.com}, [accessed on 19 April 2019]
  
{\sloppy\printbibliography[heading=subbibliography,notkeyword=this]}

\end{document}
