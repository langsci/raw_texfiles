\chapter{Morphophonological alternations of verbs}

Pre-notes:

\begin{enumerate}[label=(\alph*.)]
\item The following lists correspond to the stem classes (Stems No. 1-17 and irregular verbal stems) and affix classes (Types A-D) discussed in \sectref{sec:key:8.2};
\item The affixes shown below exclude the Group II inflectional affixes discussed in \tabref{tab:key:55} in \sectref{sec:key:8.1} since they do not directly follow verbal roots and the morphophonological alternation caused by them are very transparent;
\item All of the non-italic verbal forms express the surface forms;
\item Examples other than those marked by the asterisk (*) were actually pronounced (in the elicitation or the natural discourse) by the speaker TM;
\item The examples marked by the asterisk (*) are created by the present author using synchronic (morpho)phonological rules in Yuwan;
\item The question mark (?) means that the speaker TM never uttered the form and that the present author could not find the natural context where the form can be used in elicitation;
\item Infinitives (simple forms and lengthened forms) are shown in the final page.
\end{enumerate}

\section{Stem No. 1 (ending with //V\textsubscript{non-back}r//): \textit{hingir-} ‘escape’}

\ea Type-A affixes\\
\glll {\textit{-an} (NEG)}  {\textit{-ar(ɨr)} (PASS) } {\textit{-ar(ɨr)} (CAP)}  {\textit{-as} (CAUS)}  {\textit{-azɨi} (NEG.PLQ)}  {\textit{-ɨ} (IMP)}  {\textit{-ɨba} (SUGS)}  {\textit{-oo}(INT)}\\
hingir-an  hingir-at-ta  hingir-arɨk=kai  {N/A\footnotemark}  hingir-azɨi  hingir-ɨ  hingir-ɨba  hingir-oo\\
escape-NEG  escape-PASS-PST  escape-CAP=DUB    escape-NEG.PLQ  escape-IMP  escape-SUGS  escape-INT\\
\z\footnotetext{A different root is used with \textit{-as} (CAUS), i.e. /hing{}-jas-ju-i/ \textit{hing{}-as-jur-i} (escape-CAUS-UMRK-NPST).}

\ea Type-B affixes\\
\glll \textit{-tar} (PST)  \textit{-tuk} (PRPR)  \textit{-tur} (PROG)  \textit{-təər} (RSL)  \textit{-tɨ} (SEQ)  \textit{-tai} (LST)  \textit{-təəra} ‘after’\\
hingi-tat=too  hingi-tuk-ɨ  hingi-tu-i  hingi-təəp-pa  *hingi-tɨ  *hingi-tai  *hingi-təəra\\
escape-PST=ASS  escape-PRPR-IMP  escape-PROG-NPST  escape-RSL-CSL  escape-SEQ  escape-LST  escape-after\\
\ex Type-C affixes\\
\glll \textit{-jawur} (POL)  \textit{-jaa} ‘person’  \textit{-jur} (UMRK)  \textit{-jagacinaa} (SIM)\\
*hingi-jawu-i  hingi-jaa  hingi-jus-sa  hingi-jagacinaa\\
escape-POL-NPST  escape-person  escape-UMRK-POL  escape-SIM\\
\ex Type-D affixes and clitic\\
\glll \textit{-ba} (CSL)  \textit{-boo} (CND)  \textit{-gadɨ} ‘until’  \textit{-na} (PROH)  \textit{kai} (DUB)\\
hingip-pa  *hingip-poo  *hingik-kadɨ  hingin-na  *hingik=kai\\
escape-CSL  escape-CND  escape-until  escape-PROH  escape=DUB\\
\z

\section{Stem No. 1 (ending with //V\textsubscript{non-back}r//): \textit{abɨr-} ‘call’}

\ea Type-A affixes\\
\glll \textit{-an} (NEG)  \textit{-ar(ɨr)} (PASS)  \textit{-ar(ɨr)} (CAP)  \textit{-as} (CAUS)  \textit{-azɨi} (NEG.PLQ)  \textit{-ɨ} (IMP)  \textit{-ɨba} (SUGS)  \textit{-oo}(INT)\\
abɨr-an  abɨr-at-tɨ  abɨr-arɨ-n=nja  abɨr-as-ɨ  abɨr-azɨi  abɨr-ɨ  abɨr-ɨba  abɨr-oo\\
call-NEG  call-PASS-SEQ  call-CAP-NPST=PLQ  call-CAUS-IMP  call-NEG.PLQ  call-IMP  call-SUGS  call-INT\\


\ex Type-B affixes\\
\glll \textit{-tar} (PST)  \textit{-tuk} (PRPR)  \textit{-tur} (PROG)  \textit{-təər} (RSL)  \textit{-tɨ} (SEQ)  \textit{-tai} (LST)  \textit{-təəra} ‘after’\\
abɨ-ta  abɨ-tuk-ɨ  abɨ-tu-i  abɨ-təət=too  *abɨ-tɨ  *abɨ-tai  *abɨ-təəra\\
call-PST  call-PRPR-IMP  call-PROG-NPST  call-RSL=ASS  call-SEQ  call-LST  call-after\\


\ex Type-C affixes\\
\glll \textit{-jawur} (POL)  \textit{-jaa} ‘person’  \textit{-jur} (UMRK)  \textit{-jagacinaa} (SIM)\\
*abɨ-jawu-i  abɨ-jaa  abɨ-ju-i  abɨ-jagacinaa\\
call-POL-NPST  call-person  call-UMRK-NPST  call-SIM\\


\ex Type-D affixes and clitic\\
\glll \textit{-ba} (CSL)  \textit{-boo} (CND)  \textit{-gadɨ} ‘until’  \textit{-na} (PROH)  \textit{kai} (DUB)\\
abɨp-pa  abɨp-poo  *abɨk-kadɨ  abɨn-na  *abɨk=kai\\
call-CSL  call-CND  call-until  call-PROH  call=DUB\\
\z

\section{Stem No. 1 (ending with //V\textsubscript{non-back}r//): \textit{kəər-} ‘exchange’}


\ea Type-A affixes\\
\glll \textit{-an} (NEG)  \textit{-ar(ɨr)} (PASS)  \textit{-ar(ɨr)} (CAP)  \textit{-as} (CAUS)  \textit{-azɨi} (NEG.PLQ)  \textit{-ɨ} (IMP)  \textit{-ɨba} (SUGS)  \textit{-oo}(INT)\\
kəər-an  kəər-at-tup-pa  kəər-ar-an  kəər-as-oo  kəər-azɨi  kəər-ɨ  kəər-ɨba  kəər-oo\\
exchange-NEG  exchange-PASS-PROG-CSL  exchange-CAP-NEG  exchange-CAUS-INT  exchange-NEG.PLQ  exchange-IMP  exchange-SUGS  exchange-INT\\


\ex Type-B affixes\\
\glll \textit{-tar} (PST)  \textit{-tuk} (PRPR)  \textit{-tur} (PROG)  \textit{-təər} (RSL)  \textit{-tɨ} (SEQ)  \textit{-tai} (LST)  \textit{-təəra} ‘after’\\
kəə-tat=too  kəə-tuk-ɨ  kəə-tu-i  kəə-təəp-pa  *kəə-tɨ  *kəə-tai  *kəə-təəra\\
exchange-PST=ASS  exchange-PRPR-IMP  exchange-PROG-NPST  exchange-RSL-CSL  exchange-SEQ  exchange-LST  exchange-after\\


\ex Type-C affixes\\
\glll \textit{-jawur} (POL)  \textit{-jaa} ‘person’  \textit{-jur} (UMRK)  \textit{-jagacinaa} (SIM)\\
*kəə-jawu-i  kəə-jaa  kəə-ju-i  kəə-jagacinaa\\
exchange-POL-NPST  exchange-person  exchange-UMRK-NPST  exchange-SIM\\


\ex Type-D affixes and clitic\\
\glll \textit{-ba} (CSL)  \textit{-boo} (CND)  \textit{-gadɨ} ‘until’  \textit{-na} (PROH)  \textit{kai} (DUB)\\
kəəp-pa  kəəp-poo  *kəək-kadɨ  kəən-na  *kəək=kai\\
exchange-CSL  exchange-CND  exchange-until  exchange-PROH  exchange=DUB\\
\z

\section{Stem No. 2 (ending with //V\textsubscript{back}r//): \textit{kˀuur-} ‘close’}

\ea Type-A affixes\\
\glll \textit{-an} (NEG)  \textit{-ar(ɨr)} (PASS)  \textit{-ar(ɨr)} (CAP)  \textit{-as} (CAUS)  \textit{-azɨi} (NEG.PLQ)  \textit{-ɨ} (IMP)  \textit{-ɨba} (SUGS)  \textit{-oo}(INT)\\
kˀuur-an  kˀuur-at-tat-tu  kˀuur-arɨ-n=nja  kˀuur-as-oo  kˀuur-azɨi  kˀuur-ɨ  kˀuur-ɨba  kˀuur-oo\\
close-NEG  close-PASS-PST-CSL  close-CAP-NPST=PLQ  close-CAUS-INT  close-NEG.PLQ  close-IMP  close-SUGS  close-INT\\


\ex Type-B affixes\\
\glll \textit{-tar} (PST)  \textit{-tuk} (PRPR)  \textit{-tur} (PROG)  \textit{-təər} (RSL)  \textit{-tɨ} (SEQ)  \textit{-tai} (LST)  \textit{-təəra} ‘after’\\
kˀuu-tat=too  kˀuu-tuk-ɨ  kˀuu-tut=too  kˀuu-təə-tat-tu  kˀuu-tɨ  *kˀuu-tai  *kˀuu-təəra\\
close-PST=ASS  close-PRPR-IMP  close-PROG=ASS  close-RSL-PST-CSL  close-SEQ  close-LST  close-after\\


\ex Type-C affixes\\
\glll \textit{-jawur} (POL)  \textit{-jaa} ‘person’  \textit{-jur} (UMRK)  \textit{-jagacinaa} (SIM)\\
*kˀuu-jawu-i  ?  kˀuu-ju-i  kˀuu-jagacinaa\\
close-POL-NPST    close-UMRK-NPST  close-SIM\\


\ex Type-D affixes and clitic\\
\glll \textit{-ba} (CSL)  \textit{-boo} (CND)  \textit{-gadɨ} ‘until’  \textit{-na} (PROH)  \textit{kai} (DUB)\\
kˀuup-pa  *kˀuup-poo  *kˀuuk-kadɨ  kˀuun-na  *kˀuuk=kai\\
close-CSL  close-CND  close-until  close-PROH  close=DUB\\
\z

\section{Stem No. 2 (ending with //V\textsubscript{back}r//): \textit{koor-} ‘buy’}

\ea Type-A affixes\\
\glll \textit{-an} (NEG)  \textit{-ar(ɨr)} (PASS)  \textit{-ar(ɨr)} (CAP)  \textit{-as} (CAUS)  \textit{-azɨi} (NEG.PLQ)  \textit{-ɨ} (IMP)  \textit{-ɨba} (SUGS)  \textit{-oo}(INT)\\
koor-an-ta  koor-at-ta  koor-arɨk=kai  koor-as-oo  koor-azɨi  koor-ɨ  koor-ɨba  koor-oo\\
buy-NEG-PST  buy-PASS-PST  buy-CAP=DUB  buy-CAUS-INT  buy-NEG.PLQ  buy-IMP  buy-SUGS  buy-INT\\


\ex Type-B affixes\\
\glll \textit{-tar} (PST)  \textit{-tuk} (PRPR)  \textit{-tur} (PROG)  \textit{-təər} (RSL)  \textit{-tɨ} (SEQ)  \textit{-tai} (LST)  \textit{-təəra} ‘after’\\
koo-ta-n  koo-tuk-an-boo  koo-tut=too  koo-tə-n  koo-tɨ  *koo-tai  *koo-təəra\\
buy-PST-PTCP  buy-PRPR-NEG-CND  buy-PROG=ASS  buy-RSL-PTCP  buy-SEQ  buy-LST  buy-after\\


\ex Type-C affixes\\
\glll \textit{-jawur} (POL)  \textit{-jaa} ‘person’  \textit{-jur} (UMRK)  \textit{-jagacinaa} (SIM)\\
*koo-jawu-i  koo-jaa  koo-ju-n  koo-jagacinaa\\
buy-POL-NPST  buy-person  buy-UMRK-PTCP  buy-SIM\\


\ex Type-D affixes and clitic\\
\glll \textit{-ba} (CSL)  \textit{-boo} (CND)  \textit{-gadɨ} ‘until’  \textit{-na} (PROH)  \textit{kai} (DUB)\\
koop-pa  *koop-poo  *kook-kadɨ  koon-na  *kook=kai\\
buy-CSL  buy-CND  buy-until  buy-PROH  buy=DUB\\
\z

\section{Stem No. 2 (ending with //V\textsubscript{back}r//): \textit{tur-} ‘take’}

\ea Type-A affixes\\
\glll \textit{-an} (NEG)  \textit{-ar(ɨr)} (PASS)  \textit{-ar(ɨr)} (CAP)  \textit{-as} (CAUS)  \textit{-azɨi} (NEG.PLQ)  \textit{-ɨ} (IMP)  \textit{-ɨba} (SUGS)  \textit{-oo}(INT)\\
tur-an  tur-arɨ-∅  tur-ar-an  tur-as-an-tat-tu  tur-azɨi  tur-ɨ  tur-ɨba  tur-oo\\
take-NEG  take-PASS-INF  take-CAP-NEG  take-CAUS-PST-CSL  take-NEG.PLQ  take-IMP  take-SUGS  take-INT\\


\ex Type-B affixes\\
\glll \textit{-tar} (PST)  \textit{-tuk} (PRPR)  \textit{-tur} (PROG)  \textit{-təər} (RSL)  \textit{-tɨ} (SEQ)  \textit{-tai} (LST)  \textit{-təəra} ‘after’\\
tu-ta  tu-tuk-ii  tu-tu-ta  tu-təəp-pa  tu-tɨ  *tu-tai  *tu-təəra\\
take-PST  take-PRPR-INF  take-PROG-PST  take-RSL-CSL  take-SEQ  take-LST  take-after\\


\ex Type-C affixes\\
\glll \textit{-jawur} (POL)  \textit{-jaa} ‘person’  \textit{-jur} (UMRK)  \textit{-jagacinaa} (SIM)\\
*tu-jawu-i  tu-jaa  tu-ju-n  tu-jagacinaa\\
take-POL-NPST  take-person  take-UMRK-PTCP  take-SIM\\


\ex Type-D affixes and clitic\\
\glll \textit{-ba} (CSL)  \textit{-boo} (CND)  \textit{-gadɨ} ‘until’  \textit{-na} (PROH)  \textit{kai} (DUB)\\
tup-pa  tup-poo  *tuk-kadɨ  tun-na  *tuk=kai\\
take-CSL  take-CND  take-until  take-PROH  take=DUB\\
\z

\section{Stem No. 3 (ending with //pp//): \textit{app-} ‘play’}

\ea Type-A affixes\\
\glll \textit{-an} (NEG)  \textit{-ar(ɨr)} (PASS)  \textit{-ar(ɨr)} (CAP)  \textit{-as} (CAUS)  \textit{-azɨi} (NEG.PLQ)  \textit{-ɨ} (IMP)  \textit{-ɨba} (SUGS)  \textit{-oo}(INT)\\
app-an  app-at-tat-tu  app-arɨ-n=nja  app-as-an  app-azɨi  app-ɨ  app-ɨba  app-oo\\
play-NEG  play-PASS-PST-CSL  play-CAP-NPST=PLQ  play-CAUS-NEG  play-NEG.PLQ  play-IMP  play-SUGS  play-INT\\


\ex Type-B affixes\\
\glll \textit{-tar} (PST)  \textit{-tuk} (PRPR)  \textit{-app} (PROG)  \textit{-təər} (RSL)  \textit{-tɨ} (SEQ)  \textit{-tai} (LST)  \textit{-təəra} ‘after’\\
at-ta  ?  at-tur-ɨ  at-tə-i  at-tɨ  *at-tai  *at-təəra\\
play-PST    play-PROG-IMP  play-RSL-NPST  play-SEQ  play-LST  play-after\\


\ex Type-C affixes\\
\glll \textit{-jawur} (POL)  \textit{-jaa} ‘person’  \textit{-jur} (UMRK)  \textit{-jagacinaa} (SIM)\\
*app-jawu-i  ?  app-jur-u  app-jagacinaa\\
play-POL-NPST    play-UMRK-PFC  play-SIM\\


\ex Type-D affixes and clitic\\
\glll \textit{-ba} (CSL)  \textit{-boo} (CND)  \textit{-gadɨ} ‘until’  \textit{-na} (PROH)  \textit{kai} (DUB)\\
app-uba  app-uboo  *app-ugadɨ  app-una  *app=ukai\\
play-CSL  play-CND  play-until  play-PROH  play=DUB\\
\z

\section{Stem No. 4 (ending with //b//): \textit{narab-} ‘line up’}

\ea Type-A affixes\\
\glll \textit{-an} (NEG)  \textit{-ar(ɨr)} (PASS)  \textit{-ar(ɨr)} (CAP)  \textit{-as} (CAUS)  \textit{-azɨi} (NEG.PLQ)  \textit{-ɨ} (IMP)  \textit{-ɨba} (SUGS)  \textit{-oo}(INT)\\
narab-an  narab-at-ta  narab-arɨk=kai  narab-as-oo  narab-azɨi  narab-ɨ  narab-ɨba  narab-oo\\
line.up-NEG  line.up-PASS-PST  line.up-CAP=DUB  line.up-CAUS-INT  line.up-NEG.PLQ  line.up-IMP  line.up-SUGS  line.up-INT\\


\ex Type-B affixes\\
\glll \textit{-tar} (PST)  \textit{-tuk} (PRPR)  \textit{-tur} (PROG)  \textit{-təər} (RSL)  \textit{-tɨ} (SEQ)  \textit{-tai} (LST)  \textit{-təəra} ‘after’\\
nara-da  nara-duk-ɨ  nara-du-i  nara-dəəp-pa  *nara-dɨ  *nara-dai  *nara-dəəra\\
line.up-PST  line.up-PRPR-IMP  line.up-PROG-NPST  line.up-RSL-CSL  line.up-SEQ  line.up-LST  line.up-after\\


\ex Type-C affixes\\
\glll \textit{-jawur} (POL)  \textit{-jaa} ‘person’  \textit{-jur} (UMRK)  \textit{-jagacinaa} (SIM)\\
*narab-jawu-i  narab-jaa  narab-ju-i  narab-jagacinaa\\
line.up-POL-NPST  line.up-person  line.up-UMRK-NPST  line.up-SIM\\


\ex Type-D affixes and clitic\\
\glll \textit{-ba} (CSL)  \textit{-boo} (CND)  \textit{-gadɨ} ‘until’  \textit{-na} (PROH)  \textit{kai} (DUB)\\
narab-uba  narab-uboo  *narab-ugadɨ  narab-una  *narab=ukai\\
line.up-CSL  line.up-CND  line.up-until  line.up-PROH  line.up=DUB\\
\z

\section{Stem No. 5 (ending with //Vm//): \textit{jum-} ‘read’}

\ea Type-A affixes\\
\glll \textit{-an} (NEG)  \textit{-ar(ɨr)} (PASS)  \textit{-ar(ɨr)} (CAP)  \textit{-as} (CAUS)  \textit{-azɨi} (NEG.PLQ)  \textit{-ɨ} (IMP)  \textit{-ɨba} (SUGS)  \textit{-oo}(INT)\\
jum-an  jum-at-ta  jum-arɨ-i  jum-as-oo  jum-azɨi  jum-ɨ  jum-ba  jum-oo\\
read-NEG  read-PASS-PST  read-CAP-NPST  read-CAUS-INT  read-NEG.PLQ  read-IMP  read-SUGS  read-INT\\


\ex Type-B affixes\\
\glll \textit{-tar} (PST)  \textit{-tuk} (PRPR)  \textit{-tur} (PROG)  \textit{-təər} (RSL)  \textit{-tɨ} (SEQ)  \textit{-tai} (LST)  \textit{-təəra} ‘after’\\
ju-da  ju-duk-ɨba  ju-dur-ɨba  ju-dəəp-pa  *ju-dɨ  *ju-dai  *ju-dəəra\\
read-PST  read-PRPR-SUGS  read-PROG-SUGS  read-RSL-CSL  read-SEQ  read-LST  read-after\\


\ex Type-C affixes\\
\glll \textit{-jawur} (POL)  \textit{-jaa} ‘person’  \textit{-jur} (UMRK)  \textit{-jagacinaa} (SIM)\\
*jum-jawu-i  jum-jaa  jum-ju-n  jum-jagacinaa\\
read-POL-NPST  read-person  read-UMRK-PTCP  read-SIM\\


\ex Type-D affixes and clitic\\
\glll \textit{-ba} (CSL)  \textit{-boo} (CND)  \textit{-gadɨ} ‘until’  \textit{-na} (PROH)  \textit{kai} (DUB)\\
jum-ba  jum-boo  jum-gadɨ  jum-na  *jum=kai\\
read-CSL  read-CND  read-until  read-PROH  read=DUB\\
\z

\section{Stem No. 6 (ending with //nm//): \textit{tanm-} ‘ask’}

\ea Type-A affixes\\
\glll \textit{-an} (NEG)  \textit{-ar(ɨr)} (PASS)  \textit{-ar(ɨr)} (CAP)  \textit{-as} (CAUS)  \textit{-azɨi} (NEG.PLQ)  \textit{-ɨ} (IMP)  \textit{-ɨba} (SUGS)  \textit{-oo}(INT)\\
tanm-an  tanm-ar-ɨ  tanm-ar-an  tanm-as-oo  tanm-azɨi  tanm-ɨ  tanm-ɨba  tanm-oo\\
ask-NEG  ask-PASS-IMP  ask-CAP-NEG  ask-CAUS-INT  ask-NEG.PLQ  ask-IMP  ask-SUGS  ask-INT\\


\ex Type-B affixes\\
\glll \textit{-tar} (PST)  \textit{-tuk} (PRPR)  \textit{-tur} (PROG)  \textit{-təər} (RSL)  \textit{-tɨ} (SEQ)  \textit{-tai} (LST)  \textit{-təəra} ‘after’\\
tan-da  tan-duk-ɨ  tan-du-tɨ  tan-də-i  *tan-dɨ  *tan-dai  *tan-dəəra\\
ask-PST  ask-PRPR-IMP  ask-PROG-SEQ  ask-RSL-NPST  ask-SEQ  ask-LST  ask-after\\


\ex Type-C affixes\\
\glll \textit{-jawur} (POL)  \textit{-jaa} ‘person’  \textit{-jur} (UMRK)  \textit{-jagacinaa} (SIM)\\
*tanm-jawu-i  ?  tanm-jut=too  tanm-jagacinaa\\
ask-POL-NPST    ask-UMRK=ASS  ask-SIM\\


\ex Type-D affixes and clitic\\
\glll \textit{-ba} (CSL)  \textit{-boo} (CND)  \textit{-gadɨ} ‘until’  \textit{-na} (PROH)  \textit{kai} (DUB)\\
tanm-uba  *tanm-uboo  *tanm-ugadɨ  tanm-una  *tanm=ukai\\
ask-CSL  ask-CND  ask-until  ask-PROH  ask=DUB\\
\z

\section{Stem No. 7 (ending with //V\textsubscript{non-}\textit{\textsubscript{i} }k//): \textit{kak-} ‘write’}

\ea Type-A affixes\\
\glll \textit{-an} (NEG)  \textit{-ar(ɨr)} (PASS)  \textit{-ar(ɨr)} (CAP)  \textit{-as} (CAUS)  \textit{-azɨi} (NEG.PLQ)  \textit{-ɨ} (IMP)  \textit{-ɨba} (SUGS)  \textit{-oo}(INT)\\
kak-an-ta  kak-at-ta  kak-arɨk=kai  kak-as-i+gjaa  kak-azɨi  kak-ɨ  kak-ɨba  kak-oo\\
write-NEG-PST  write-PASS-PST  write-CAP=DUB  write-CAUS-INF+PURP  write-NEG.PLQ  write-IMP  write-SUGS  write-INT\\


\ex Type-B affixes\\
\glll \textit{-tar} (PST)  \textit{-tuk} (PRPR)  \textit{-tur} (PROG)  \textit{-təər} (RSL)  \textit{-tɨ} (SEQ)  \textit{-tai} (LST)  \textit{-təəra} ‘after’\\
ka-cja  ka-cjuk-ɨ  ka-cjur-an-ta  ka-cjə-i  ka-cjɨ  *ka-cjai  *ka-cjəəra\\
write-PST  write-PRPR-IMP  write-PROG-NEG-PST  write-RSL-NPST  write-SEQ  write-LST  write-after\\


\ex Type-C affixes\\
\glll \textit{-jawur} (POL)  \textit{-jaa} ‘person’  \textit{-jur} (UMRK)  \textit{-jagacinaa} (SIM)\\
*kak-jawu-i  kak-jaa  kak-ju-mɨ  kak-jagacinaa\\
write-POL-NPST  write-person  write-UMRK-PLQ  write-SIM\\


\ex Type-D affixes and clitic\\
\glll \textit{-ba} (CSL)  \textit{-boo} (CND)  \textit{-gadɨ} ‘until’  \textit{-na} (PROH)  \textit{kai} (DUB)\\
kak-uba  kak-uboo  kak-ugadɨ  kak-una  kak=ukai\\
write-CSL  write-CND  write-until  write-PROH  write=DUB\\
\z

\section{Stem No. 8 (ending with //V\textsubscript{non-}\textit{\textsubscript{i} }kk//): \textit{sukk-} ‘pull’}

\ea Type-A affixes\\
\glll \textit{-an} (NEG)  \textit{-ar(ɨr)} (PASS)  \textit{-ar(ɨr)} (CAP)  \textit{-as} (CAUS)  \textit{-azɨi} (NEG.PLQ)  \textit{-ɨ} (IMP)  \textit{-ɨba} (SUGS)  \textit{-oo}(INT)\\
sukk-an  *sukk-arɨ-i  *sukk-arɨ-i  *sukk-as-oo  *sukk-azɨi  *sukk-ɨ  *sukk-ɨba  *sukk-oo\\
pull-NEG  pull-PASS-NPST  pull-CAP-NPST  pull-CAUS-INT  pull-NEG.PLQ  pull-IMP  pull-SUGS  pull-INT\\


\ex Type-B affixes\\
\glll \textit{-tar} (PST)  \textit{-tuk} (PRPR)  \textit{-sukk} (PROG)  \textit{-təər} (RSL)  \textit{-tɨ} (SEQ)  \textit{-tai} (LST)  \textit{-təəra} ‘after’\\
*suc-cja  *suc-cjuk-ɨ  *suc-cju-i  *suc-cjə-i  suc-cjɨ  *suc-cjai  *suc-cjəəra\\
pull-PST  pull-PRPR-IMP  pull-PROG-NPST  pull-RSL-NPST  pull-SEQ  pull-LST  pull-after\\


\ex Type-C affixes\\
\glll \textit{-jawur} (POL)  \textit{-jaa} ‘person’  \textit{-jur} (UMRK)  \textit{-jagacinaa} (SIM)\\
*sukk-jawu-i  *sukk-jaa  sukk-ju-i  *sukk-jagacinaa\\
pull-POL-NPST  pull-person  pull-UMRK-NPST  pull-SIM\\


\ex Type-D affixes and clitic\\
\glll \textit{-ba} (CSL)  \textit{-boo} (CND)  \textit{-gadɨ} ‘until’  \textit{-na} (PROH)  \textit{kai} (DUB)\\
*sukk-uba  *sukk-uboo  *sukk-ugadɨ  sukk-una  *sukk=ukai\\
pull-CSL  pull-CND  pull-until  pull-PROH  pull=DUB\\
\z

\section{Stem No. 9 (ending with //Vs//): \textit{us-} ‘push’}

\ea Type-A affixes\\
\glll \textit{-an} (NEG)  \textit{-ar(ɨr)} (PASS)  \textit{-ar(ɨr)} (CAP)  \textit{-as} (CAUS)  \textit{-azɨi} (NEG.PLQ)  \textit{-ɨ} (IMP)  \textit{-ɨba} (SUGS)  \textit{-oo}(INT)\\
us-an-boo  us-at-ta  us-arɨk=kai  us-as-oo  us-azɨi  us-ɨ  us-ɨba  us-oo\\
push-NEG-CND  push-PASS-PST  push-CAP=DUB  push-CAUS-INT  push-NEG.PLQ  push-IMP  push-SUGS  push-INT\\


\ex Type-B affixes\\
\glll \textit{-tar} (PST)  \textit{-tuk} (PRPR)  \textit{-tur} (PROG)  \textit{-təər} (RSL)  \textit{-tɨ} (SEQ)  \textit{-tai} (LST)  \textit{-təəra} ‘after’\\
u-cja  u-cjuk-ɨ  u-cjut=too  u-cjəəp-pa  *u-cjɨ  *u-cjai  *u-cjəəra\\
push-PST  push-PRPR-IMP  push-PROG=ASS  push-RSL-CSL  push-SEQ  push-LST  push-after\\


\ex Type-C affixes\\
\glll \textit{-jawur} (POL)  \textit{-jaa} ‘person’  \textit{-jur} (UMRK)  \textit{-jagacinaa} (SIM)\\
*us-jawu-i  us-jaa  us-jut=too  us-jagacinaa\\
push-POL-NPST  push-person  push-UMRK=ASS  push-SIM\\


\ex Type-D affixes and clitic\\
\glll \textit{-ba} (CSL)  \textit{-boo} (CND)  \textit{-gadɨ} ‘until’  \textit{-na} (PROH)  \textit{kai} (DUB)\\
us-ɨba  us-ɨboo  *us-ɨgadɨ  us-ɨna  *us=ɨkai\\
push-CSL  push-CND  push-until  push-PROH  push=DUB\\
\z

\section{Stem No. 10 (ending with //ss//): \textit{kuss-} ‘kill’}

\ea Type-A affixes\\
\glll \textit{-an} (NEG)  \textit{-ar(ɨr)} (PASS)  \textit{-ar(ɨr)} (CAP)  \textit{-as} (CAUS)  \textit{-azɨi} (NEG.PLQ)  \textit{-ɨ} (IMP)  \textit{-ɨba} (SUGS)  \textit{-oo}(INT)\\
kuss-an  kuss-at-ta  kuss-ar-an  kuss-as-oo  kuss-azɨi  kuss-ɨ  kuss-ɨba  kuss-oo\\
kill-NEG  kill-PASS-PST  kill-CAP-NEG  kill-CAUS-INT  kill-NEG.PLQ  kill-IMP  kill-SUGS  kill-INT\\


\ex Type-B affixes\\
\glll \textit{-tar} (PST)  \textit{-tuk} (PRPR)  \textit{-tur} (PROG)  \textit{-təər} (RSL)  \textit{-tɨ} (SEQ)  \textit{-tai} (LST)  \textit{-təəra} ‘after’\\
kuc-cja  kuc-cjuk-ɨ  kuc-cju-i  kuc-cjə-i  *kuc-cjɨ  *kuc-cjai  *kuc-cjəəra\\
kill-PST  kill-PRPR-IMP  kill-PROG-NPST  kill-RSL-NPST  kill-SEQ  kill-LST  kill-after\\


\ex Type-C affixes\\
\glll \textit{-jawur} (POL)  \textit{-jaa} ‘person’  \textit{-jur} (UMRK)  \textit{-jagacinaa} (SIM)\\
*kuss-jawu-i  kuss-jaa  kuss-jur-oo  kuss-jagacinaa\\
kill-POL-NPST  kill-person  kill-UMRK-SUPP  kill-SIM\\


\ex Type-D affixes and clitic\\
\glll \textit{-ba} (CSL)  \textit{-boo} (CND)  \textit{-gadɨ} ‘until’  \textit{-na} (PROH)  \textit{kai} (DUB)\\
kuss-ɨba  *kuss-ɨboo  *kuss-ɨgadɨ  kuss-ɨna  *kuss=ɨkai\\
kill-CSL  kill-CND  kill-until  kill-PROH  kill=DUB\\
\z

\section{Stem No. 11 (ending with //t//): \textit{ut-} ‘hit’}

\ea Type-A affixes\\
\glll \textit{-an} (NEG)  \textit{-ar(ɨr)} (PASS)  \textit{-ar(ɨr)} (CAP)  \textit{-as} (CAUS)  \textit{-azɨi} (NEG.PLQ)  \textit{-ɨ} (IMP)  \textit{-ɨba} (SUGS)  \textit{-oo}(INT)\\
ut-an  ut-at-tɨ  ut-arɨk=kai  ut-as-oo  ut-azɨi  ut-ɨ  ut-ɨba  ut-oo\\
hit-NEG  hit-PASS-SEQ  hit-CAP=DUB  hit-CAUS-INT  hit-NEG.PLQ  hit-IMP  hit-SUGS  hit-INT\\


\ex Type-B affixes\\
\glll \textit{-tar} (PST)  \textit{-tuk} (PRPR)  \textit{-tur} (PROG)  \textit{-təər} (RSL)  \textit{-tɨ} (SEQ)  \textit{-tai} (LST)  \textit{-təəra} ‘after’\\
uc-cja  uc-cjuk-ɨ  uc-cju-tɨ  uc-cjəəp-pa  uc-cjɨ  *uc-cjai  *uc-cjəəra\\
hit-PST  hit-PRPR-IMP  hit-PROG-SEQ  hit-RSL-CSL  hit-SEQ  hit-LST  hit-after\\


\ex Type-C affixes\\
\glll \textit{-jawur} (POL)  \textit{-jaa} ‘person’  \textit{-jur} (UMRK)  \textit{-jagacinaa} (SIM)\\
*uc-jawu-i  uc-jaa  uc-ju-i  uc-jagacinaa\\
hit-POL-NPST  hit-person  hit-UMRK-NPST  hit-SIM\\


\ex Type-D affixes and clitic\\
\glll \textit{-ba} (CSL)  \textit{-boo} (CND)  \textit{-gadɨ} ‘until’  \textit{-na} (PROH)  \textit{kai} (DUB)\\
uc-ɨba  uc-ɨboo  *uc-ɨgadɨ  uc-ɨna  *uc=ɨkai\\
hit-CSL  hit-CND  hit-until  hit-PROH  hit=DUB\\
\z

\section{Stem No. 12 (ending with //\$C(G)//): \textit{jˀ-} ‘say’}

\ea Type-A affixes\\
\glll \textit{-an} (NEG)  \textit{-ar(ɨr)} (PASS)  \textit{-ar(ɨr)} (CAP)  \textit{-as} (CAUS)  \textit{-azɨi} (NEG.PLQ)  \textit{-ɨ} (IMP)  \textit{-ɨba} (SUGS)  \textit{-oo}(INT)\\
jˀ-an-tɨ  jˀ-at-tɨ  jˀ-arɨɨr-u  jˀ-as-oo  jˀ-azɨi  jˀ-ɨ  jˀ-ɨba  jˀ-oo\\
say-NEG-SEQ  say-PASS-SEQ  say-CAP-PFC  say-CAUS-INT  say-NEG.PLQ  say-IMP  say-SUGS  say-INT\\


\ex Type-B affixes\\
\glll \textit{-tar} (PST)  \textit{-tuk} (PRPR)  \textit{-tur} (PROG)  \textit{-təər} (RSL)  \textit{-tɨ} (SEQ)  \textit{-tai} (LST)  \textit{-təəra} ‘after’\\
jˀi-cja  jˀi-cjuk-ɨ  jˀi-cju-tɨ  jˀi-cjə-n  jˀi-cjɨ  *jˀi-cjai  *jˀi-cjəəra\\
say-PST  say-PRPR-IMP  say-PROG-SEQ  say-RSL-PTCP  say-SEQ  say-LST  say-after\\


\ex Type-C affixes\\
\glll \textit{-jawur} (POL)  \textit{-jaa} ‘person’  \textit{-jur} (UMRK)  \textit{-jagacinaa} (SIM)\\
*jˀ-awu-i  jˀ-aa  jˀ-ur-u  jˀ-aagacinaa\\
say-POL-NPST  say-person  say-UMRK-PFC  say-SIM\\


\ex Type-D affixes and clitic\\
\glll \textit{-ba} (CSL)  \textit{-boo} (CND)  \textit{-gadɨ} ‘until’  \textit{-na} (PROH)  \textit{kai} (DUB)\\
jˀ-uuba  jˀ-uuboo  *jˀ-uugadɨ  jˀ-uuna  *jˀ=uukai\\
say-CSL  say-CND  say-until  say-PROH  say=DUB\\
\z

\section{Stem No. 12 (ending with //\$C(G)//): \textit{mj-} ‘see’}

\ea Type-A affixes\\
\glll \textit{-an} (NEG)  \textit{-ar(ɨr)} (PASS)  \textit{-ar(ɨr)} (CAP)  \textit{-as} (CAUS)  \textit{-azɨi} (NEG.PLQ)  \textit{-ɨ} (IMP)  \textit{-ɨba} (SUGS)  \textit{-oo}(INT)\\
mj-an  mj-at-ta  mj-ar-an-ba  mj-as-oo  mj-azɨi  mj-ɨ  mj-ɨba  mj-oo\\
see-NEG  see-PASS-PST  see-CAP-NEG-CSL  see-CAUS-INT  see-NEG.PLQ  see-IMP  see-SUGS  see-INT\\


\ex Type-B affixes\\
\glll \textit{-tar} (PST)  \textit{-tuk} (PRPR)  \textit{-tur} (PROG)  \textit{-təər} (RSL)  \textit{-tɨ} (SEQ)  \textit{-tai} (LST)  \textit{-təəra} ‘after’\\
mji-cja  mji-cjuk-ɨ  mji-cju-tɨ  mji-cjəəp-pa  mji-cjɨ  *mji-cjai  *mji-cjəəra\\
see-PST  see-PRPR-IMP  see-PROG-SEQ  see-RSL-CSL  see-SEQ  see-LST  see-after\\


\ex Type-C affixes\\
\glll \textit{-jawur} (POL)  \textit{-jaa} ‘person’  \textit{-jur} (UMRK)  \textit{-jagacinaa} (SIM)\\
*mj-awu-i  ?  mj-u-i  mj-aagacinaa\\
see-POL-NPST    see-UMRK-NPST  see-SIM\\


\ex Type-D affixes and clitic\\
\glll \textit{-ba} (CSL)  \textit{-boo} (CND)  \textit{-gadɨ} ‘until’  \textit{-na} (PROH)  \textit{kai} (DUB)\\
mj-uuba  mj-uuboo  mj-uugadɨ / mjik-kadɨ  mj-uuna  mj=uukai / mjik=kai\\
see-CSL  see-CND  see-until  see-PROH  see=DUB\\
\z

\section{Stem No. 13 (ending with //ij//): \textit{kij-} ‘cut’}

\ea Type-A affixes\\
\glll \textit{-an} (NEG)  \textit{-ar(ɨr)} (PASS)  \textit{-ar(ɨr)} (CAP)  \textit{-as} (CAUS)  \textit{-azɨi} (NEG.PLQ)  \textit{-ɨ} (IMP)  \textit{-ɨba} (SUGS)  \textit{-oo}(INT)\\
kij-an  kij-at-tɨ  kij-ar-an  kij-as-oo  kij-azɨi  kij-ɨ  kij-ɨba  kij-oo\\
cut-NEG  cut-PASS-SEQ  cut-CAP-NEG  cut-CAUS-INT  cut-NEG.PLQ  cut-IMP  cut-SUGS  cut-INT\\


\ex Type-B affixes\\
\glll \textit{-tar} (PST)  \textit{-tuk} (PRPR)  \textit{-tur} (PROG)  \textit{-təər} (RSL)  \textit{-tɨ} (SEQ)  \textit{-tai} (LST)  \textit{-təəra} ‘after’\\
ki-cja  ki-cjuk-ɨ  ki-cjut=too  ki-cjəəp-pa  ki-cjɨ  *ki-cjai  *ki-cjəəra\\
cut-PST  cut-PRPR-IMP  cut-PROG=ASS  cut-RSL-CSL  cut-SEQ  cut-LST  cut-after\\


\ex Type-C affixes\\
\glll \textit{-jawur} (POL)  \textit{-jaa} ‘person’  \textit{-jur} (UMRK)  \textit{-jagacinaa} (SIM)\\
*ki-jawu-i  ki-jaa  ki-ju-mɨ  ki-jagacinaa\\
cut-POL-NPST  cut-person  cut-UMRK-PLQ  cut-SIM\\


\ex Type-D affixes and clitic\\
\glll \textit{-ba} (CSL)  \textit{-boo} (CND)  \textit{-gadɨ} ‘until’  \textit{-na} (PROH)  \textit{kai} (DUB)\\
kip-pa  kip-poo  kig-gadɨ  kin-na  *kik=kai\\
cut-CSL  cut-CND  cut-until  cut-PROH  cut=DUB\\
\z

\section{Stem No. 14 (ending with //V\textsubscript{non-}\textit{\textsubscript{i}} g//): \textit{tug-} ‘whet’}

\ea Type-A affixes\\
\glll \textit{-an} (NEG)  \textit{-ar(ɨr)} (PASS)  \textit{-ar(ɨr)} (CAP)  \textit{-as} (CAUS)  \textit{-azɨi} (NEG.PLQ)  \textit{-ɨ} (IMP)  \textit{-ɨba} (SUGS)  \textit{-oo}(INT)\\
tug-an  tug-at-ta  tug-arɨk=kai  tug-as-oo  tug-azɨi  tug-ɨ  tug-ɨba  tug-oo\\
whet-NEG  whet-PASS-PST  whet-CAP=DUB  whet-CAUS-INT  whet-NEG.PLQ  whet-IMP  whet-SUGS  whet-INT\\


\ex Type-B affixes\\
\glll \textit{-tar} (PST)  \textit{-tuk} (PRPR)  \textit{-tur} (PROG)  \textit{-təər} (RSL)  \textit{-tɨ} (SEQ)  \textit{-tai} (LST)  \textit{-təəra} ‘after’\\
tu-zja  tu-zjuk-ɨ  tu-zjut=too  tu-zjəəp-pa  *tu-zjɨ  *tu-zjai  *tu-zjəəra\\
whet-PST  whet-PRPR-IMP  whet-PROG=ASS  whet-RSL-CSL  whet-SEQ  whet-LST  whet-after\\


\ex Type-C affixes\\
\glll \textit{-jawur} (POL)  \textit{-jaa} ‘person’  \textit{-jur} (UMRK)  \textit{-jagacinaa} (SIM)\\
*tug-jawu-i  tug-jaa  tug-ju-mɨ  tug-jagacinaa\\
whet-POL-NPST  whet-person  whet-UMRK-PLQ  whet-SIM\\


\ex Type-D affixes and clitic\\
\glll \textit{-ba} (CSL)  \textit{-boo} (CND)  \textit{-gadɨ} ‘until’  \textit{-na} (PROH)  \textit{kai} (DUB)\\
tug-uba  tug-uboo  *tug-ugadɨ  tug-una  *tug=ukai\\
whet-CSL  whet-CND  whet-until  whet-PROH  whet=DUB\\
\z

\section{Stem No. 15 (ending with //ik//): \textit{kik-} ‘hear’}

\ea Type-A affixes\\
\glll \textit{-an} (NEG)  \textit{-ar(ɨr)} (PASS)  \textit{-ar(ɨr)} (CAP)  \textit{-as} (CAUS)  \textit{-azɨi} (NEG.PLQ)  \textit{-ɨ} (IMP)  \textit{-ɨba} (SUGS)  \textit{-oo}(INT)\\
kik-jan  kik-jar-an  kik-jarɨ-i  kik-jas-i  kik-jazɨi  kik-jɨ  kik-jɨba  kik-joo\\
hear-NEG  hear-PASS-NEG  hear-CAP-NPST  hear-CAUS-INF  hear-NEG.PLQ  hear-IMP  hear-SUGS  hear-INT\\


\ex Type-B affixes\\
\glll \textit{-tar} (PST)  \textit{-tuk} (PRPR)  \textit{-tur} (PROG)  \textit{-təər} (RSL)  \textit{-tɨ} (SEQ)  \textit{-tai} (LST)  \textit{-təəra} ‘after’\\
ki-cja  ki-cjuk-ɨ  ki-cju-tɨ  ki-cjəəp-pa  *ki-cjɨ  *ki-cjai  *ki-cjəəra\\
hear-PST  hear-PRPR-IMP  hear-PROG-SEQ  hear-RSL-CSL  hear-SEQ  hear-LST  hear-after\\


\ex Type-C affixes\\
\glll \textit{-jawur} (POL)  \textit{-jaa} ‘person’  \textit{-jur} (UMRK)  \textit{-jagacinaa} (SIM)\\
*kik-jawu-i  kik-jaa  kik-ju-n  kik-jagacinaa\\
hear-POL-NPST  hear-person  hear-UMRK-PTCP  hear-SIM\\


\ex Type-D affixes and clitic\\
\glll \textit{-ba} (CSL)  \textit{-boo} (CND)  \textit{-gadɨ} ‘until’  \textit{-na} (PROH)  \textit{kai} (DUB)\\
kik-uba  *kik-uboo  *kik-ugadɨ  kik-una  *kik=ukai\\
hear-CSL  hear-CND  hear-until  hear-PROH  hear=DUB\\
\z

\section{Stem No. 16 (ending with //i(n)g//): \textit{uig-} ‘swim’}

\ea Type-A affixes\\
\glll \textit{-an} (NEG)  \textit{-ar(ɨr)} (PASS)  \textit{-ar(ɨr)} (CAP)  \textit{-as} (CAUS)  \textit{-azɨi} (NEG.PLQ)  \textit{-ɨ} (IMP)  \textit{-ɨba} (SUGS)  \textit{-oo}(INT)\\
uig-jan  uig-jat-ta  uig-jarɨk=kai  uig-jas-oo  uig-jazɨi  uig-jɨ  uig-iba  uig-joo\\
swim-NEG  swim-PASS-PST  swim-CAP=DUB  swim-CAUS-INT  swim-NEG.PLQ  swim-IMP  swim-SUGS  swim-INT\\


\ex Type-B affixes\\
\glll \textit{-tar} (PST)  \textit{-tuk} (PRPR)  \textit{-tur} (PROG)  \textit{-təər} (RSL)  \textit{-tɨ} (SEQ)  \textit{-tai} (LST)  \textit{-təəra} ‘after’\\
ui-zja  ui-zjuk-ɨ  ui-zju-i  ui-zjəəp-pa  *ui-zjɨ  *ui-zjai  *ui-zjəəra\\
swim-PST  swim-PRPR-IMP  swim-PROG-NPST  swim-RSL-CSL  swim-SEQ  swim-LST  swim-after\\


\ex Type-C affixes\\
\glll \textit{-jawur} (POL)  \textit{-jaa} ‘person’  \textit{-jur} (UMRK)  \textit{-jagacinaa} (SIM)\\
*uig-jawu-i  uig-jaa  uig-ju-n  uig-jagacinaa\\
swim-POL-NPST  swim-person  swim-UMRK-PTCP  swim-SIM\\


\ex Type-D affixes and clitic\\
\glll \textit{-ba} (CSL)  \textit{-boo} (CND)  \textit{-gadɨ} ‘until’  \textit{-na} (PROH)  \textit{kai} (DUB)\\
uig-uba  uig-uboo  uig-ugadɨ  uig-una  *uig=ukai\\
swim-CSL  swim-CND  swim-until  swim-PROH  swim=DUB\\
\z

\section{Stem No. 16 (ending with //i(n)g//): \textit{ming-} ‘grab’}

\ea Type-A affixes\\
\glll \textit{-an} (NEG)  \textit{-ar(ɨr)} (PASS)  \textit{-ar(ɨr)} (CAP)  \textit{-as} (CAUS)  \textit{-azɨi} (NEG.PLQ)  \textit{-ɨ} (IMP)  \textit{-ɨba} (SUGS)  \textit{-oo}(INT)\\
ming-jan  ming-jat-ta  ming-jar-an  ming-jas-oo  ming-jazɨi  ming-jɨ  ming-jɨba / ming-iba  ming-joo\\
grab-NEG  grab-PASS-PST  grab-CAP-NEG  grab-CAUS-INT  grab-NEG.PLQ  grab-IMP  grab-SUGS  grab-INT\\


\ex Type-B affixes\\
\glll \textit{-tar} (PST)  \textit{-tuk} (PRPR)  \textit{-tur} (PROG)  \textit{-təər} (RSL)  \textit{-tɨ} (SEQ)  \textit{-tai} (LST)  \textit{-təəra} ‘after’\\
min-zjat=too  min-zjuk-ɨ  min-zjur-ɨ  min-zjəəp-pa  *min-zjɨ  *min-zjai  *min-zjəəra\\
grab-PST=ASS  grab-PRPR-IMP  grab-PROG-IMP  grab-RSL-CSL  grab-SEQ  grab-LST  grab-after\\


\ex Type-C affixes\\
\glll \textit{-jawur} (POL)  \textit{-jaa} ‘person’  \textit{-jur} (UMRK)  \textit{-jagacinaa} (SIM)\\
*ming-jawu-i  ?  ming-ju-i  ming-jagacinaa\\
grab-POL-NPST    grab-UMRK-NPST  grab-SIM\\


\ex Type-D affixes and clitic\\
\glll \textit{-ba} (CSL)  \textit{-boo} (CND)  \textit{-gadɨ} ‘until’  \textit{-na} (PROH)  \textit{kai} (DUB)\\
ming-uba  *ming-uboo  *ming-ugadɨ  ming-una  *ming=ukai\\
grab-CSL  grab-CND  grab-until  grab-PROH  grab=DUB\\
\z

\section{Stem No. 17 (ending with //in//): \textit{sin-} ‘die’}

\ea Type-A affixes\\
\glll \textit{-an} (NEG)  \textit{-ar(ɨr)} (PASS)  \textit{-ar(ɨr)} (CAP)  \textit{-as} (CAUS)  \textit{-azɨi} (NEG.PLQ)  \textit{-ɨ} (IMP)  \textit{-ɨba} (SUGS)  \textit{-oo}(INT)\\
sin-jan  sin-jat-tɨ  sin-jarɨp-poo  sin-ja-cja-n  sin-jazɨi  sin-jɨ  sin-ba  sin-joo\\
dile-NEG  dile-PASS-SEQ  dile-CAP-CND  dile-CAUS-PST-PTCP  dile-NEG.PLQ  dile-IMP  dile-SUGS  dile-INT\\


\ex Type-B affixes\\
\glll \textit{-tar} (PST)  \textit{-tuk} (PRPR)  \textit{-tur} (PROG)  \textit{-təər} (RSL)  \textit{-tɨ} (SEQ)  \textit{-tai} (LST)  \textit{-təəra} ‘after’\\
si-zja  ?  si-zjup-pa  si-zjəəp-pa  si-zjɨ  *si-zjai  *si-zjəəra\\
dile-PST    dile-PROG-CSL  dile-RSL-CSL  dile-SEQ  dile-LST  dile-after\\


\ex Type-C affixes\\
\glll \textit{-jawur} (POL)  \textit{-jaa} ‘person’  \textit{-jur} (UMRK)  \textit{-jagacinaa} (SIM)\\
*sin-jawu-i  ?  sin-juk=kai  ?\\
dile-POL-NPST    dile-UMRK=DUB  \\


\ex Type-D affixes and clitic\\
\glll \textit{-ba} (CSL)  \textit{-boo} (CND)  \textit{-gadɨ} ‘until’  \textit{-na} (PROH)  \textit{kai} (DUB)\\
sin-ba  *sin-boo  *sin-gadɨ  sin-na  sin=kai\\
dile-CSL  dile-CND  dile-until  dile-PROH  dile=DUB\\
\z

\section{Irregular type verbal stems (a): \textit{sɨr-} ‘do’}

\ea Type-A affixes\\
\glll \textit{-an} (NEG)  \textit{-ar(ɨr)} (PASS)  \textit{-ar(ɨr)} (CAP)  \textit{-as} (CAUS)  \textit{-azɨi} (NEG.PLQ)  \textit{-ɨ} (IMP)  \textit{-ɨba} (SUGS)  \textit{-oo}(INT)\\
sɨr-an  sɨr-at-ta  sɨr-arɨ-i  sɨr-as-oo  sɨr-azɨi  sɨr-ɨ  sɨr-ɨba  sɨr-oo\\
do-NEG  do-PASS-PST  do-CAP-NPST  do-CAUS-INT  do-NEG.PLQ  do-IMP  do-SUGS  do-INT\\


\ex Type-B affixes\\
\glll \textit{-tar} (PST)  \textit{-tuk} (PRPR)  \textit{-tur} (PROG)  \textit{-təər} (RSL)  \textit{-tɨ} (SEQ)  \textit{-tai} (LST)  \textit{-təəra} ‘after’\\
sjat=too  sjuk-uba  sju-i  sjəə=sɨ  sjɨ  sjai  *sjəəra\\
do.PST=ASS  do.PRPR-CSL  do.PROG-NPST  do.RSL=FN  do.SEQ  do.LST  do.after\\


\ex Type-C affixes\\
\glll \textit{-jawur} (POL)  \textit{-jaa} ‘person’  \textit{-jur} (UMRK)  \textit{-jagacinaa} (SIM)\\
*s-jawu-i  s-jaa  s-ju-i  s-jaagacinaa\\
do-POL-NPST  do-person  do-UMRK-NPST  do-SIM\\


\ex Type-D affixes and clitic\\
\glll \textit{-ba} (CSL)  \textit{-boo} (CND)  \textit{-gadɨ} ‘until’  \textit{-na} (PROH)  \textit{kai} (DUB)\\
sɨp-pa  sɨp-poo  sɨk-kadɨ  sɨn-na  sɨk=kai\\
do-CSL  do-CND  do-until  do-PROH  do=DUB\\
\z

Notes: \textit{sɨr-} ‘do’ and \textit{moosɨr-} (die.HON) behave like the verbal stem No. 1 (ending with //V\textsubscript{non-back}r//) except for the following cases.

(i)  Type-B affixes are fused with the preceding verbal root, e.g. \textit{sɨr-tar} (do-PST) > /sja/ (not /sɨ-ta/);

(ii)  Before the type-C affixes, \textit{sɨr-} ‘do’ becomes /s/, and \textit{moosɨr-} (die.HON) becomes /moos/;

(iii)  Before the infinitival affix, \textit{sɨr-} ‘do’ becomes /s/, and \textit{moosɨr-} (die.HON) becomes /moos/ (see also the final page of the appendix).

\section{Irregular type verbal stems (b): \textit{k-} ‘come’}

\ea Type-A affixes\\
\glll \textit{-on} (NEG)  \textit{-oor(ɨr)} (PASS)  \textit{-oor(ɨr)} (CAP)  \textit{-oos} (CAUS)  \textit{-oozɨi} (NEG.PLQ)  \textit{-oo} (IMP)  \textit{-ooba} (SUGS)  \textit{-oo}(INT)\\
k-on  k-oorɨp-poo  k-oorɨ-n=nja  k-oos-an  k-oozɨi  k-oo  k-ooba  k-oo\\
come-NEG  come-PASS-CND  come-CAP-NPST=PLQ  come-CAUS-NEG  come-NEG.PLQ  come-IMP  come-SUGS  come-INT\\


\ex Type-B affixes\\
\glll \textit{-tar} (PST)  \textit{-tuk} (PRPR)  \textit{-tur} (PROG)  \textit{-təər} (RSL)  \textit{-tɨ} (SEQ)  \textit{-tai} (LST)  \textit{-təəra} ‘after’\\
cˀja  ?  cˀjup-pa  cˀjən  cˀjɨ  cˀjai  *cˀjəəra\\
come.PST    come.PROG-CSL  come.RSL-PTCP  come.SEQ  come.LST  come.after\\


\ex Type-C affixes\\
\glll \textit{-jawur} (POL)  \textit{-jaa} ‘person’  \textit{-jur} (UMRK)  \textit{-jagacinaa} (SIM)\\
*k-jawu-i  ?  k-ju-i  k-jaagacinaa\\
come-POL-NPST    come-UMRK-NPST  come-SIM\\


\ex Type-D affixes and clitic\\
\glll \textit{-ba} (CSL)  \textit{-boo} (CND)  \textit{-gadɨ} ‘until’  \textit{-na} (PROH)  \textit{kai} (DUB)\\
kˀ-uuba  kˀ-uuboo  kˀ-uugadɨ  kˀ-uuna  *kˀ=uukai\\
come-CSL  come-CND  come-until  come-PROH  come=DUB\\
\z

Notes: \textit{k-} ‘come’ and \textit{tɨkk-} ‘bring’ behave like the verbal stem No. 7 (ending with //V\textsubscript{non-}\textit{\textsubscript{i}} k//) except for the following cases.

(i)  The initial vowel of the type-A affixes is //oo// (or //o//);

(ii)  Type-B affixes are fused with the preceding verbal root \textit{k-} ‘come,’ e.g. \textit{k-tar} (do-PST) > /cˀja/;

(iii)  Before the type-D affixes, \textit{k-} ‘come’ becomes /kˀ/.

\section{Irregular type verbal stems (c): \textit{ik-} ‘go’}

\ea Type-A affixes\\
\glll \textit{-an} (NEG)  \textit{-ar(ɨr)} (PASS)  \textit{-ar(ɨr)} (CAP)  \textit{-as} (CAUS)  \textit{-azɨi} (NEG.PLQ)  \textit{-ɨ} (IMP)  \textit{-ɨba} (SUGS)  \textit{-oo}(INT)\\
ik-jan  ik-jat-tɨ  ik-jarɨ-n=nja  ik-jas-ju-i  ik-jazɨi  ik-jɨ  ik-jɨba  ik-joo\\
go-NEG  go-PASS-SEQ  go-CAP-NPST=PLQ  go-CAUS-UMRK-NPST  go-NEG.PLQ  go-IMP  go-SUGS  go-INT\\


\ex Type-B affixes\\
\glll \textit{-tar} (PST)  \textit{-tuk} (PRPR)  \textit{-tur} (PROG)  \textit{-təər} (RSL)  \textit{-tɨ} (SEQ)  \textit{-tai} (LST)  \textit{-təəra} ‘after’\\
i-zja  ?  i-zjur-ɨ  i-zjəəp-pa  i-zjɨ  i-zjai  *i-zjəəra\\
go-PST    go-PROG-IMP  go-RSL-CSL  go-SEQ  go-LST  go-after\\


\ex Type-C affixes\\
\glll \textit{-jawur} (POL)  \textit{-jaa} ‘person’  \textit{-jur} (UMRK)  \textit{-jagacinaa} (SIM)\\
*ik-jawu-i  *ik-jaa  ik-ju-i  ik-jagacinaa\\
go-POL-NPST  go-person  go-UMRK-NPST  go-SIM\\


\ex Type-D affixes and clitic\\
\glll \textit{-ba} (CSL)  \textit{-boo} (CND)  \textit{-gadɨ} ‘until’  \textit{-na} (PROH)  \textit{kai} (DUB)\\
ik-uba  ik-uboo  ik-ugadɨ  ik-una  *ik=ukai\\
go-CSL  go-CND  go-until  go-PROH  go=DUB\\
\z

Note: \textit{ik-} ‘go’ behaves like the verbal stem No. 15 (ending with //ik//) except for the following case.

(i)  The initial consonant of the type-B affixes becomes /zj/ (not /cj/) after \textit{ik-} ‘go.’

\section{Irregular type verbal stems (d): \textit{umoor-} (move.HON)}

\ea Type-A affixes\\
\glll \textit{-an} (NEG)  \textit{-ar(ɨr)} (PASS)  \textit{-ar(ɨr)} (CAP)  \textit{-as} (CAUS)  \textit{-azɨi} (NEG.PLQ)  \textit{-ɨ} (IMP)  \textit{-ɨba} (SUGS)  \textit{-oo}(INT)\\
umoor-an  umoor-at-tat-tu  umoor-arɨ-n=nja  umoor-as-an-boo  umoor-azɨi  umoor-ɨ  umoor-ɨba  umoor-oo\\
move.HON-NEG  move.HON-PASS-PST-CSL  move.HON-CAP-NPST=PLQ  move.HON-CAUS-NEG-CND  move.HON-NEG.PLQ  move.HON-IMP  move.HON-SUGS  move.HON-INT\\


\ex Type-B affixes\\
\glll \textit{-tar} (PST)  \textit{-tuk} (PRPR)  \textit{-tur} (PROG)  \textit{-təər} (RSL)  \textit{-tɨ} (SEQ)  \textit{-tai} (LST)  \textit{-təəra} ‘after’\\
umoo-cja  ?  umoo-cjuk=ka  umoo-cjə-i  umoo-cjɨ  *umoo-cjai  *umoo-cjəəra\\
move.HON-PST    move.HON-PROG=DUB  move.HON-RSL-NPST  move.HON-SEQ  move.HON-LST  move.HON-after\\


\ex Type-C affixes\\
\glll \textit{-jawur} (POL)  \textit{-jaa} ‘person’  \textit{-jur} (UMRK)  \textit{-jagacinaa} (SIM)\\
?  ?  umoo-ju-i  umoo-jagacinaa\\
    move.HON-UMRK-NPST  move.HON-SIM\\


\ex Type-D affixes and clitic\\
\glll \textit{-ba} (CSL)  \textit{-boo} (CND)  \textit{-gadɨ} ‘until’  \textit{-na} (PROH)  \textit{kai} (DUB)\\
umoop-pa  *umoop-poo  *umook-kadɨ  umoon-na  *umook=kai\\
move.HON-CSL  move.HON-CND  move.HON-until  move.HON-PROH  move.HON=DUB\\
\z

Note: The honorific verbs such as \textit{umoor-} (move.HON) behaves like the verbal stem No. 2 (ending with //V\textsubscript{back}r//) except for the following case.

(i)  The initial consonant of the type-B affixes become /cj/ (not /t/) after honorific verbs (although \textit{moosɨr-} (die.HON) behaves like \textit{sɨr-} ‘do’).

\section{Irregular type verbal stems (e): \textit{hijaw-} ‘pick up’}

\ea Type-A affixes\\
\glll \textit{-an} (NEG)  \textit{-ar(ɨr)} (PASS)  \textit{-ar(ɨr)} (CAP)  \textit{-as} (CAUS)  \textit{-azɨi} (NEG.PLQ)  \textit{-ɨ} (IMP)  \textit{-ɨba} (SUGS)  \textit{-oo}(INT)\\
hijaw-an  hijoo-t-tat-tu  hijoo-r-an-ta  hijoo-s-oo  hijaw-azɨi  hijaw-ɨ  hijaw-ɨba  hijaw-oo\\
pick.up-NEG  pick.up-PASS-PST-CSL  pick.up-CAP-NEG-PST  pick.up-CAUS-INT  pick.up-NEG.PLQ  pick.up-IMP  pick.up-SUGS  pick.up-INT\\


\ex Type-B affixes\\
\glll \textit{-tar} (PST)  \textit{-tuk} (PRPR)  \textit{-tur} (PROG)  \textit{-təər} (RSL)  \textit{-tɨ} (SEQ)  \textit{-tai} (LST)  \textit{-təəra} ‘after’\\
hija-ta  hija-tuk-ɨ  hija-tut=too  hija-təəp-pa  hija-tɨ  *hija-tai  *hija-təəra\\
pick.up-PST  pick.up-PRPR-IMP  pick.up-PROG=ASS  pick.up-RSL-CSL  pick.up-SEQ  pick.up-LST  pick.up-after\\


\ex Type-C affixes\\
\glll \textit{-jawur} (POL)  \textit{-jaa} ‘person’  \textit{-jur} (UMRK)  \textit{-jagacinaa} (SIM)\\
*hija-jawu-i  hija-jaa  hija-ju=sə=ə  hijəə-jagacinaa\\
pick.up-POL-NPST  pick.up-person  pick.up-UMRK=FN=TOP  pick.up-SIM\\


\ex Type-D affixes and clitic\\
\glll \textit{-ba} (CSL)  \textit{-boo} (CND)  \textit{-gadɨ} ‘until’  \textit{-na} (PROH)  \textit{kai} (DUB)\\
hijəə-ba  *hijəə-boo  hijəə-gadɨ  hijəə-na  *hijəə=kai\\
pick.up-CSL  pick.up-CND  pick.up-until  pick.up-PROH  pick.up=DUB\\
\z

Notes: The verbal stems that end with //aw// behave like the verbal stem No. 2 (ending with //V\textsubscript{back}w//) except for the following cases.

(i)  The stem-final //aw// becomes /oo/ before \textit{-ar(ɨr)} (PASS), \textit{-ar(ɨr)} (CAP) or \textit{-as} (CAUS), and also these affixes delete their initial vowels;

(ii)  The stem-final //aw// becomes /əə/ before \textit{-jagacinaa} (SIM), the type-D affixes and clitic, or the infinitival affix (see aslo the final page of the appendix).

\section{Irregular type verbal stems (f): \textit{sij-} ‘know’}

\ea Type-A affixes\\
\glll \textit{-an} (NEG)  \textit{-ar(ɨr)} (PASS)  \textit{-ar(ɨr)} (CAP)  \textit{-as} (CAUS)  \textit{-azɨi} (NEG.PLQ)  \textit{-ɨ} (IMP)  \textit{-ɨba} (SUGS)  \textit{-oo}(INT)\\
sij-an  sij-at-təəp-pa  ?  sij-as-oo  sij-azɨi  ?  ?  sij-oo\\
know-NEG  know-PASS-RSL-CSL    know-CAUS-INT  know-NEG.PLQ      know-INT\\


\ex Type-B affixes\\
\glll \textit{-tar} (PST)  \textit{-tuk} (PRPR)  \textit{-tur} (PROG)  \textit{-təər} (RSL)  \textit{-tɨ} (SEQ)  \textit{-tai} (LST)  \textit{-təəra} ‘after’\\
sic-cjat=too  ?  sic-cju-i  sic-cjə-n  *sic-cjɨ  *sic-cjai  *sic-cjəəra\\
know-PST=ASS    know-PROG-NPST  know-RSL-PTCP  know-SEQ  know-LST  know-after\\


\ex Type-C affixes\\
\gll \textit{-jawur} (POL)  \textit{-jaa} ‘person’  \textit{-jur} (UMRK)  \textit{-jagacinaa} (SIM)\\
?  ?  ?  ?\\


\ex Type-D affixes and clitic\\
\gll \textit{-ba} (CSL)  \textit{-boo} (CND)  \textit{-gadɨ} ‘until’  \textit{-na} (PROH)  \textit{kai} (DUB)\\
?  ?  ?  ?  ?\\
\z

Notes: \textit{sij-} ‘know’ behaves like the verbal stem No. 13 (ending with //ij//) except for the following case.

(i)  The stem-final consonant //j// of \textit{sij-} ‘know’ becomes /c/ before the type-B affixes, e.g. \textit{sij-tar} (know-PST) > /sic-cja/ (not /si-cja/).

\section{Irregular type verbal stems (g): \textit{jurukub-} ‘happy’}

\ea Type-A affixes\\
\glll \textit{-an} (NEG)  \textit{-ar(ɨr)} (PASS)  \textit{-ar(ɨr)} (CAP)  \textit{-as} (CAUS)  \textit{-azɨi} (NEG.PLQ)  \textit{-ɨ} (IMP)  \textit{-ɨba} (SUGS)  \textit{-oo}(INT)\\
jurukub-an  jurukub-at-ta  jurukub-ar-an  jurukub-as-oo  jurukub-azɨi  jurukub-ɨ  ?  jurukub-oo\\
happy-NEG  happy-PASS-PST  happy-CAP-NEG  happy-CAUS-INT  happy-NEG.PLQ  happy-IMP    happy-INT\\


\ex Type-B affixes\\
\glll \textit{-tar} (PST)  \textit{-tuk} (PRPR)  \textit{-tur} (PROG)  \textit{-təər} (RSL)  \textit{-tɨ} (SEQ)  \textit{-tai} (LST)  \textit{-təəra} ‘after’\\
juruku-da  ?  juruku-dup-pa  juruku-də-i  juruku-dɨ  *juruku-dai  *juruku-dəəra\\
happy-PST    happy-PROG-CSL  happy-RSL-NPST  happy-SEQ  happy-LST  happy-after\\


\ex Type-C affixes\\
\glll \textit{-jawur} (POL)  \textit{-jaa} ‘person’  \textit{-jur} (UMRK)  \textit{-jagacinaa} (SIM)\\
*jurukub-jawu-i  ?  ?  jurukub-jagacinaa\\
happy-POL-NPST      happy-SIM\\


\ex Type-D affixes and clitic\\
\glll \textit{-ba} (CSL)  \textit{-boo} (CND)  \textit{-gadɨ} ‘until’  \textit{-na} (PROH)  \textit{kai} (DUB)\\
jurukun-ba  jurukun-boo  *jurukun-gadɨ  jurukun-na  *jurukun=kai\\
happy-CSL  happy-CND  happy-until  happy-PROH  happy=DUB\\
\z

Notes: \textit{jurukub-} ‘happy’ behaves like the verbal stem No. 4 (ending with //b//) except for the following case.

(i)  The stem-final consonant //b// of \textit{jurukub-} ‘happy’ becomes /n/ (strictly speaking, the archiphoneme /N/) before the type-D affixes and clitic, e.g. \textit{jurukub-ba} (happy-CSL) > /jurukun-ba/ (not /jurukub-uba/).

\section{Irregular type verbal stems (h): \textit{hənk-} ‘enter’}

\ea Type-A affixes\\
\glll \textit{-an} (NEG)  \textit{-ar(ɨr)} (PASS)  \textit{-ar(ɨr)} (CAP)  \textit{-as} (CAUS)  \textit{-azɨi} (NEG.PLQ)  \textit{-ɨ} (IMP)  \textit{-ɨba} (SUGS)  \textit{-oo}(INT)\\
hənk-jan  hənk-jat-ta  hənk-jarɨk=kai  hənk-jas-oo  hənk-jazɨi  hənk-jɨ  hənk-jɨba  hənk-joo\\
enter-NEG  enter-PASS-PST  enter-CAP=DUB  enter-CAUS-INT  enter-NEG.PLQ  enter-IMP  enter-SUGS  enter-INT\\


\ex Type-B affixes\\
\glll \textit{-tar} (PST)  \textit{-tuk} (PRPR)  \textit{-tur} (PROG)  \textit{-təər} (RSL)  \textit{-tɨ} (SEQ)  \textit{-tai} (LST)  \textit{-təəra} ‘after’\\
hən-cja  hən-cjuk-ɨ  hən-cjut=too  hən-cjəəp-pa  *hən-cjɨ  *hən-cjai  *hən-cjəəra\\
enter-PST  enter-PRPR-IMP  enter-PROG=ASS  enter-RSL-CSL  enter-SEQ  enter-LST  enter-after\\


\ex Type-C affixes\\
\glll \textit{-jawur} (POL)  \textit{-jaa} ‘person’  \textit{-jur} (UMRK)  \textit{-jagacinaa} (SIM)\\
*hənk-jawu-i  ?  hənk-ju-n  hənk-jagacinaa\\
enter-POL-NPST    enter-UMRK-PTCP  enter-SIM\\


\ex Type-D affixes and clitic\\
\glll \textit{-ba} (CSL)  \textit{-boo} (CND)  \textit{-gadɨ} ‘until’  \textit{-na} (PROH)  \textit{kai} (DUB)\\
hənk-uba  *hənk-uboo  hənk-ugadɨ  hənk-una  *hənk=ukai\\
enter-CSL  enter-CND  enter-until  enter-PROH  enter=DUB\\
\z

Notes: \textit{hənk-} ‘enter’ behaves like the verbal stem No. 7 (ending with //V\textsubscript{non-}\textit{\textsubscript{i}} k//) except for the following case.

(i)  /j/ is inserted between \textit{hənk-} ‘enter’ and the type-A affixes, e.g. \textit{hənk-an} (enter-NEG) > /hənk-jan/. In other words, \textit{hənk-} ‘enter’ behaves like the verbal stem No. 15 (ending with //ik//) although it does not include //i// in the stem-final syllable.

\section{Infinitives (simple forms and lengthened forms)}
 {\footnotesize
\begin{longtable}{ *{7}{l} }
\lsptoprule\endfirsthead\midrule\endhead\endfoot\lspbottomrule\endlastfoot
Stem No.   & \multicolumn{3}{l}{1. V\textsubscript{non-back}r} & \multicolumn{3}{l}{2. V\textsubscript{back}r, V\textsubscript{back}w}\\
ex.        & \textit{hingir-} & \textit{abɨr-} & \textit{kəər-} & \textit{ˀkuur-} & \textit{nugoor-} & \textit{koow-}\\
           & ‘escape’         & ‘call’         & ‘exchange’     & ‘close’         & ‘don’t do’       &  ‘buy’\\
Simple     &  hingi           & abɨ            &  kəə           & ˀkuu-i          & nugoo-i          &  koo-i\slash ko-i\\
Lengthened &  hingii          & abɨɨ           &  kəə           & ˀkuu-ii         & nugoo-ii         &  koo-ii\\
\midrule
Stem No.   & 2. V\textsubscript{back}r & 3. pp         & 4. b            & 5. Vm            & 6. nm          & 7. V\textsubscript{non-\textit{i}}k\\
ex.        & \textit{tur-}             & \textit{app-} & \textit{narab-} & \textit{jum-}    & \textit{tanm-} & \textit{kak-}\\
           & ‘take’                    & ‘play’        & ‘line up’       & ‘read’           & ‘ask’          & ‘write’\\
Simple     & tu-i                      & app-i         & narab-i         & jum\slash jum-i  & tanm-i         &  kak-i\\
Lengthened & tu-ii                     & app-ii        & narab-ii        & jum\slash jum-ii & tanm-ii        & kak-ii\\
\midrule

Stem No.   & 8. V\textsubscript{non-\textit{i}}kk & 9. Vs        & 10. ss         & 11. t        & \multicolumn{2}{l}{12. Only C(G)}\\
ex.        & \textit{sukk-}                       & \textit{us-} & \textit{kuss-} & \textit{ut-} & \textit{jˀ-} & \textit{mj-}\\
           & ‘pull’                               & ‘push’       & ‘kill’         & ‘hit’        & ‘say’        & ‘see’\\
Simple     & sukk-i                               & us-i         & kuss-i         & uc-i         & jˀ-ii        & m-ii\\
Lengthened & sukk-ii                              & us-ii        & kuss-ii        & uc-ii        & jˀ-ii        & m-ii\\
\midrule

Stem No.   & 13. ij        & 14. V\textsubscript{non-\textit{i}} g & 15. ik        & 16. i(n)g     &                & 17. in\\
ex.        & \textit{kij-} & \textit{tug-}                         & \textit{kik-} & \textit{uig-} & \textit{ming-} & \textit{sin-}\\
           & ‘cut’         & ‘whet’                                & ‘hear’        & ‘swim’        & ‘grab’         & ‘die’\\
Simple     & ki-i          & tug-i                                 & kik-i         & uig-i         & ming-i         & sin\slash sin-i\\
Lengthened & ki-i          & tug-ii                                & kik-ii        & uig-ii        & ming-ii        & N/A\\
\midrule

Irregular      & a.            & b.          & c.           & d.               & e.               & f.\\
ex.             & \textit{sɨr-} & \textit{k-} & \textit{ik-} & \textit{umoor-}  & \textit{hijaw-}  & \textit{sij-}\\
                & ‘do’          & ‘come’      & ‘go’         & (move.HON)       & ‘pick up’        & ‘know’\\
Simple          & s-ii          & k-ii        & ik-i         & umoo-i           & hijəə-∅          & si-i\\
Lengthened      & s-ii          & k-ii        & ik-ii        & umoo-ii          & hijəə-∅          & ?\\
\midrule

Irregular       &  g.               & h.\\
ex.             & \textit{jurkub-}  & \textit{hənk-}\\
                & ‘happy’           & ‘enter’\\
Simple          & jurukub-i         &  hənk-i\\      
Lengthened      & jurukub-ii        &  hənk-ii\\
\end{longtable}    }
