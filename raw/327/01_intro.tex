%%% -*- Mode: LaTeX -*-

\chapter{Introduction}
\label{ch.intro}

\noindent

This book is primarily concerned with the history of linguistics, but
it is not simply \emph{about} the history of linguistics.  For one
thing, positions are taken below on issues which (while they arise in
a historical context) are discussed for their own sake, such as the
motivation for assuming a significant level of phonetic representation
in linguistics.  Further, while the text traces the development of
phonological theory in the twentieth century, the goal of this
exercise is not only to contribute to the study of the history of
linguistics \emph{per se}.  Our intent is to study this history in
relation to a particular issue: the balance between \textit{Rules}
and \textit{Representations} as components of a theory of
language, and more particularly, as components of a theory of sound
structure. It is our thesis here that current views on this issue can
only be understood and appreciated in the context of the historical
development of the field, and this leads to a presentation of the
issue through the study of the forms it has taken in the work of
various major figures over approximately the past 100 years.

In the course of doing this, it will be necessary to present various
conclusions and opinions concerning the history of linguistics. Some
of these may be novel or controversial; often they will be incomplete
(especially with respect to the position on other than phonological
issues of the scholars discussed).  The reader with a comprehensive
interest in the history of the field should bear the more specific
purpose of this study in mind when assessing the adequacy of its
broader conclusions.  I can hope, however, that the picture presented
here of the development of twentieth century linguistics does not
seriously misrepresent it.

\section*{Why study the history of 20th century phonology?}

If we take as our subject matter the development of twentieth century
phonology, it is reasonable to ask just how coherent an object of
study that is. Does it make sense, that is, to limit our attention to
a) the twentieth century, and b) phonology?  Of course, any study
limited to part of a field runs the risk of losing touch with other
parts of the same field which developed at the same time and in a
relation of mutual influence.  There is also a risk of artificially
isolating the work of a given period from that which preceded it (and
ultimately led to it).  On both of these counts, though, the proposed
limits of our scope can be defended.

The {Linguistic Society of America} was founded in 1924 on the basis of
a Call sent out to a range of students of language, convening an
organizational meeting. In the first pages of the journal of the new
society, \name{Leonard}{Bloomfield} argued for its existence:
\begin{quotation}
  The layman --- natural scientist, philologian, or man in the street
  --- does not know that there is a science of language.  Such a
  science, however, exists; its aims are so well defined, its methods
  so well developed, and its past results so copious, that students of
  language feel as much need for a professional society as do
  adherents of any other science. \\
  \citep[1]{bloomfield1925:why.lsa}
\end{quotation}

The establishment of the LSA did not at all represent the commencement
of such study, but rather served to bring into focus an approach to
the study of human language that had already become established.  The
present choice of time span (roughly, the 1880s through the early
2000s) is motivated by the fact that linguistics as practiced in the
twentieth century evidently made a fairly sharp break with its
(immediate) past.  \name{Ferdinand}{de Saussure} and Jan {\DeCourtenay}
in Europe, and \name{Franz}{Boas} in America, began to articulate views on the
nature of language which obviously developed in many ways out of the
issues discussed by previous generations of linguists, but which were
also rather clearly at odds with the two dominant traditions of the
time: on the one hand, rationalist traditional grammar as inherited
from medieval philosophers and grammarians, and on the other, the
newer developments in comparative linguistics which were the
particular achievement of nineteenth century linguistics.

Around the turn of the century (liberally construed), several distinct
figures contributed to the establishment of what was to become the new
tradition in the study of language.  Most of this innovative work was
concerned with the development of a `structuralist' view of language,
though the early figures in its formation did not necessarily see
their contributions in that light.  In some cases, their influence was
quite limited until somewhat later, and occasionally the
interpretation of their work as `structuralist' seems largely to have
been imposed in retrospect. Nonetheless, `Structural Linguistics' can
be said to begin more or less simultaneously with the twentieth
century, and to form a reasonably coherent and organic object of
study.

Accepting the proposition that twentieth century views on language are
sufficient distinct from their predecessors to warrant a separate
study, we may then ask what justifies the focus of the present work on
phonology. Again, coincidentally, the year 1924 figures in clarifying
the matter. The role of sound in language had certainly been a central
interest in earlier scholarship, but terminology, and with it a clear
understanding of the ways in which sound could be studied, was less
clear. It is in the writings of \name{Otto}{Jespersen} that we first find a
proposal for a clear delineation of the sort we assume today:
\begin{quotation}
  It would, perhaps, be advisable to restrict the word ``phonetics''
  to universal or general phonetics and to use the word
  \emph{phonology} of the phenomena peculiar to a particular language
  (e.g. ``{English} Phonology'').\\
  \citep[35]{jespersen24:philosophy-of-grammar}
\end{quotation}

A focus on phonology certainly does not represent a judgment that
other aspects of the study of language are without interest; but it
does recognize that (however general its goals in principle),
structural linguistics in fact concentrated its theoretical attention
on an understanding of the language particular nature of sound
structure.  Insofar as there can be said to have been a structuralist
\emph{theory} of morphology or of syntax, it was mostly a matter of
importing into those domains the results that were felt to have been
achieved by work in phonology.  Though the post-structuralist
development of the field can fairly be said to have given much more
independent interest to syntax, it is probably also reasonable to
suggest that even here, many of our current notions of linguistic
structure derive from structuralist views on the character of
phonology.  It is surely also fair to say for most of the linguists
under consideration (at least up until the 1960s) that they would have
wanted to be judged by the validity of their work in phonology, for it
was here that the majority of them felt something had been achieved
which constituted a fundamental advance in our understanding of the
nature of language.

Granting, then, that a study of the development of phonology over the
past century is potentially a coherent topic from the point of view of
the history of linguistics, we can then go on to ask more generally
about the methods and motives for such historical study.  The history
of linguistics is a field which has attracted considerable attention
from a variety of perspectives, and we can identify a number of
distinct motivations behind past research. 

One possible reason that students of any field may look into its
history is to find support for their current preoccupations and
predispositions. We often find that our predecessors (preferably those
with general reputations as savants, but in a pinch nearly anyone in
the sufficiently distant past) were concerned with something we can
interpret as the same thing we are working on, or that they made
proposals that were similar in content, or at least in form, to our
own. The parading of such precedents is sometimes seen as lending a
kind of legitimacy to our concerns, or even an \emph{imprimatur} to
our views.

This ‘roots' variety of history can be found from time to time in many
fields, but it is arguably the case that it was particularly prominent
in views expressed by generative linguists in the 1950s and early
1960s on the origins of their discipline. Specific examples would not
be particularly illuminating at this point, though they will appear
below; to the extent this attitude can be documented, however, it is
somewhat ironic in light of the common criticism in early
anti-generativist writings that \isi{generative grammar} was fundamentally
iconoclastic and hostile to its past. It should be clear in principle
that where we can identify historical discussions of this
self-justifying type, we should be rather suspicious of their
conclusions. Such inverted ‘guilt by association' hardly serves as a
valid substitute for argument. Its only possible validity is in
countering the opposite assertion, that some current view is
hopelessly outré and unworthy of serious consideration.

An alternative (and intellectually somewhat more respectable)
motivation for studying the history of one's field is the search for
genuine insights and enlightenment. Few researchers feel that they
have direct access to all of the truth that is worth
seeking. Naturally, one looks to one's contemporaries for help, but
unless we hold with particular rigidity to the view that historical
development is a matter of monotonically nondecreasing progress, with
the present always \emph{ipso facto} more enlightened than the past,
there is no reason not to treat our intellectual ancestors with
similar respect. Many genuine insights arrived at in the past have
gone unassimilated by the field at large, perhaps for want of an
appropriate theoretical framework within which to place an observation
or conclusion, or on account of the unfashionable character of some of
their assumptions; or perhaps because of an unfortunate mode of
expression or simply the obscurity of a particular investigator. In
returning to the works of those who have studied our subject before
us, we can always hope to find pearls that subsequent research has
overlooked without warrant.

In adopting either of the motivations for historical research just
considered, however, we run a serious risk of falsifying the thought
of earlier scholars by attributing our concerns to them, or by putting
their work into the terms of a present-day framework quite foreign to
their point of view. Of course, that need not diminish the value of
what we learn from such study about our own present projects, but it
certainly makes for bad history \emph{qua} history.

An example of this sort is furnished by \posscitet{postal64:boas}
remarks on {\Boas}' phonological theory.  {\Postal} was responding to the
remarks of \citet{voegelins63:history}, who had characterized {\Boas} as
a “monolevel” structuralist in the sense that he believed in only one
structurally significant level of (phonological)
representation. Basing his comments on a single paper by {\Boas} dealing
with Iroquoian, which he admits is isolated and possibly not
representative of all of {\Boas}'s work, {\Postal} cites a number of
instances in which {\Boas}'s locutions imply a conversion of one
representation into another by phonological principles particular to
the grammar of a language. From this he concludes that {\Boas} must have
maintained at least two {levels} of phonologically significant
structure. Since one of these {levels} would clearly have to have
properties that could not be discovered directly from the surface
phonetic form by procedures of \isi{segmentation} and classification, {\Postal}
suggests that {\Boas} actually held a view on which abstract phonological
structure is converted into surface phonetic form in a way quite
similar to a generative phonology.

It is of course possible that \posscitet{postal64:boas} view of {\Boas}
is correct (though I will suggest below in chapter~\ref{ch.boas} that
it is not), but the methodology used to establish it is less than
satisfying. In particular, it is only when we put {\Boas}'s words into a
contemporary mouth that they have the implications {\Postal} found in
them. Avoiding anachronism, we see that {\Boas} was simply using the
rhetoric of traditional grammar for describing alternations. In those
terms, to say “A \emph{becomes} B (under some conditions)” is not to
assert the antecedent existence of an A, which is later converted into
a B if the conditions in question obtain. It means merely that where
we might \emph{expect} (perhaps on the basis of other, related forms)
to find an A, but where the relevant conditions are satisfied, we find
B instead. Such a mode of expression in describing alternations is
quite general in nineteenth-century and earlier descriptions, and
perfectly compatible with a view on which the \isi{sound structure} of a
form has only one significant level of representation. The development
of \isi{generative grammar} led to a climate in which ‘A becomes B under
condition C' involves a relation between a more abstract
representation (in which A appears) and a more concrete one (in which
B appears instead); but this is not the only (or even the most direct)
interpretation of such a way of talking about alternations. There is
certainly no reason to attribute it to {\Boas}, as {\Postal} does.

Another, perhaps even more drastic, example of a similar sort is
furnished by \posscitet{lightner71:swadesh.and.voegelin} criticism of
\posscitet{swadesh.voegelin39:alternation} classic paper on
morphophonemic theory. {\Lightner} discusses, in particular, {\Swadesh} and
{\Voegelin}'s analysis of the \isi{alternation} found in \ili{English}
\emph{lea\textbf{f}}/\emph{lea\textbf{v}es} and other pairs in which a
final voiceless fricative corresponds to a voiced segment in the
plural. {\Swadesh} and {\Voegelin} represent the stem of \ili{English} \emph{leaf}
as |liF| (as opposed to
|b{ə}lif| \emph{belief}, with nonchanging
[f]). {\Lightner} raises the question of just what the symbol
|F| in {\Swadesh} and {\Voegelin}'s \isi{transcription}
signifies. He concludes (as they do) that it cannot be simple [f],
since other forms ending in [f] (e.g., \emph{belief}) do not show the
\isi{alternation} in question.  Similarly, |F| cannot
represent [v], since there are words ending in non-alternating [v]
(such as \emph{leave} `furlough').

From these considerations, {\Lightner} surmises that {\Swadesh} and {\Voegelin}
must have intended their |F| to represent some
other segment --- say {[f]} with an additional feature, such as
pharyngealization, glottalization, etc.  Obviously, however, the
phonetic value feature of whatever is chosen for this purpose will
never be realized on the surface, since |F| is
always produced either as {[f]} or as {[v]} without attendant
pharyngealization, glottalization, etc. to distinguish it from other
cases of {[f]}, {[v]}.  Therefore, {\Lightner} concludes, {\Swadesh} and
{\Voegelin} must have used whatever feature they intend to distinguish
|F| from |f| as a diacritic.
The choice of a particular feature (e.g., {[pharyngealized]}) is totally
arbitrary (so long as the feature chosen is not otherwise used in the
language), and their analysis is ultimately incoherent in consequence.

The weakness of this attack on {\Swadesh} and {\Voegelin} should be
evident. In fact, those authors give no reason to suspect that they
intend the difference between |F| and
|f| to be interpretable in phonetic terms at all.
They are quite explicit about the fact that a symbol like
|F| is simply an abbreviation for a morphophonemic
formula (|F| = `/f/ in the singular, /v/ in the
plural form').  In the spirit of the 1930s, such a formula is simply
a statement of distribution (conditioned, in this instance, by
morphological rather than phonemic factors). As such, it is
phonetically (or phonemically) interpretable only by reference to its
environment: in some environments, its value is /f/, while in others
it is /v/.  Nowhere, crucially, is it something in between (or somehow
distinct from both).

At the time of {\Lightner}'s discussion, however, most generative
phonologists believed that every symbol in an underlying (or
`systematic phonemic') form should be interpretable in terms of a
uniform set of phonetic/phonological features.  Thus, for {\Lightner}, if
|F| was not to be either /f/ or /v/) it had to be
distinct from them in terms of some one of the (phonetically
motivated) \isi{distinctive features} provided by the theory. Whatever
appeal this view may have today as a potentially restrictive theory of
\isi{sound structure}, there is no reason whatsoever to attribute such a
position to {\Swadesh} and {\Voegelin}, who took some care to point out that
their morphophonemic symbols were not to be interpreted directly as
phonetic or phonemic.  To do so and then present this as an argument
against the coherence of their position is simply not a historically
accurate way to approach their work.

Somewhat more defensible as a basis for doing history, it would seem,
is a third possible motivation: one may simply wish to understand the
past in order to know how the present came to be.  So stated, this is
a virtually empty formulation of the importance of history, but it
conceals a more specific point.  Many of the ideas, problems, research
strategies and emphases of a field, taken for granted by its
practitioners and passed on to their students, have been derived in
similar fashion from \emph{their} predecessors.  What is important
is the fact that this process of transmission typically takes place without
necessarily entailing that the notions passed on are rethought and
justified anew at each step.  As a result, we may find ourselves
centrally concerned with problems which are really those of a previous
generation, and which would have no particular claim on our attention
if we were to redesign our field entirely on the basis of our current
understanding of its object.

Often, that is, certain fundamental or guiding ideas of a field
{change}, but without this having the result that all of the rest of its
content is thought out again from basic principles so as to form a
unified whole. It is of course true that many of the problems, etc.,
of a discipline remain the same through substantial {change} in
theories, and the need to rethink \emph{everything} constantly to
maintain complete consistency with innovations might well be thought
prohibitive.  The price of assuming a high degree of basic continuity,
though, may well be the importation into the field of a certain amount
of poorly digested conceptual residue: notions that were central for
previous theories, but that have little or no relevance in light of
current understanding. We may thus wish to study the history of our
field, in part, just so as to identify such anomalous situations.

The course of development of \isi{generative grammar} suggests that this
last motivation for studying history may be particularly apposite in
this context.  This is because \isi{generative grammar} almost from the
outset involved a major shift in the conception held by most
researchers of the nature of the object of study in linguistics.
Especially in the American tradition, linguists assumed that their
concern was with the study of \textit{languages}, taken as
(potentially unlimited) sets of possible sentences (or utterances,
etc.) forming unitary and coherent systems. Gradually, however, the
emphasis in research shifted from the properties of languages to the
properties of \textit{grammars}, in the sense of systems of {rules}
which specify the properties of the (well-formed) sentences in such a
system.

The {change} involved is a somewhat subtle one, in that previous
researchers who were concerned explicitly to characterize a
language naturally presented their description in the form of a
grammar; while present work (equally naturally) tends to identify the
grammar under study by the sentences it specifies as well-formed.
Thus, both approaches involve simultaneous attention to objects of
both sorts. However, where previous generations of linguists saw the
presentation of a grammar as a way of satisfying the basic requirement
of specifying a particular language, current interest is more on the
evidence provided by individual grammars for the specification of the
\emph{general form of grammars}.  There is a definite shift of the
object of inquiry in linguistics involved, though it is difficult to
formulate the issue with complete clarity.

Nonetheless, the {change} is highly significant for the conceptual
structure of the field: the shift from studying sets of sentences to
studying systems of {rules} results, at a minimum, in a {change} in the
areas in which we might expect to find empirical contact between
theoretical constructs and objects or structures in the physical
(neurological, psychological, etc.)
world. \citet{chomsky81:rules.and.representations} for example
suggested that the empirical properties of `languages' (in the sense
of sets of sentences forming the system of communication within a
given speech community) may be such as to render the notion an
ill-defined one, or at least an inappropriate one for systematic
study.  Since this notion had formed the basis of essentially all
previous theories of language, the consequences of abandoning it are
potentially rather far-reaching.  There is no particular reason to
believe \emph{a priori} that theories of languages and theories of
grammars are even commensurate with one another, and certainly not
that the assumptions and results of a theory of one type carry over
directly to theories of the other type.

\section*{Motivations for the present book}

It is with regard to this issue that the present study should be
situated.  I suggest that the notion of linguistic
\textit{representations} is one that arises as the central object of
study in a theory concerned with languages (construed as sets of
sentences, words, utterances, etc.); while the notion of
\textit{rules} is one that arises particularly in connection with the
study of grammars.  This is hardly an iron-clad, binary distinction
between theories: many of the central concerns of the field at present
are fundamentally questions about the basic properties of
\isi{representations}; but if this is a notion that pertains especially to
theories of \emph{languages}, and we accept the proposal that the
appropriate object of study in linguistics is actually
\emph{grammars}, then it follows that concerns about properties of
\isi{representations} must at minimum be raised anew, and justified in terms
of the logical structure we otherwise attribute to the field.

Let us make this issue more concrete within the domain of sound
structure. As a result of the spoken character of Language, the
organizational foundation of the sound pattern of a natural language
lies in the way it establishes \isi{systematic relations} between
(physically) distinct acts of speaking.  To characterize this sound
pattern in a particular language is essentially to describe the range
of {variation} that is permissible in such events if they are still to
count as linguistically `the same.' This problem evidently arises even
for consecutive repetitions of the `same' utterance, since it is in
general possible for measuring instruments of sufficient
sophistication to identify some differences, no matter how minute,
between any two physical events.  This is true even though physically
distinct repetitions of a particular utterance may be completely
indistinguishable from the point of view of a speaker of a given
language.

From the point of view of the linguist, this sense in which
differences always exist between distinct events may seem a mere
technicality (however important it might be to a physicist or a
philosopher). But a similar problem arises at rather grosser {levels},
in ways that only a linguist can address seriously. For example, in
\ili{English} the vowels of stressed syllables show considerable {variation}
in duration as a function of the properties of a following
consonant. \emph{Ceteris paribus}, the ‘same' vowel will be longest
before a following /z/, shortest before a following /p/ or /k/, and
intermediate in duration before voiced obstruents, nasals,
etc. (cf. \citealt{lehiste70:suprasegmentals} for a review of the
relevant facts). Nonetheless, we have considerable difficulty in
saying that the \emph{a}'s of, for example, \emph{jazz} and
\emph{Jap} are (at least in \ili{English}) anything other than ‘the same';
and indeed, a nonlinguist native speaker may be quite reluctant to
believe that they are not. The sound pattern of any language involves
a large number of such {regularities} of pronunciation, which establish
the more or less precise limits on the range and attendant conditions
of \isi{variation} in the ‘same' sound.

It would be possible to present a set of examples illustrating a
completely gradual transition between cases such as the length of
\ili{English} stressed vowels (as a function of following \isi{consonants}) and
others in which the differences involved are much more obvious. That
is, related sounds may be physically very different indeed (and,
hence, not the same at any level of phonetic analysis), while still
counting as \emph{linguistically} the same because they correspond to
one another in different variants of the same larger linguistic unit
(formative or `morpheme'). Thus, the [s] of \emph{cats}, the [z] of
\emph{dogs}, and the [{ə}z] of \emph{horses} all represent
essentially the same component of the marking of plurality in
\ili{English}. Similarly, the [k] of \emph{fanatic} is in some sense the
‘same' as the [s] of \emph{fanaticism}, though the two are physically
quite different. At a yet higher level, we might say that there is a
sense in which both \emph{am} and \emph{is} contain the same verb
\{\emph{be}\}; less radically, we may find that the same higher unit
is represented by the same segmental content differently organized in
different forms, as in \ili{Georgian} \emph{mo}-\textbf{\d{k}lav}-\emph{s}
‘kills' vs. \emph{mo}-\textbf{\d{k}vl}-\emph{a} ‘killing'.

Whenever we study any of these sorts of \isi{systematic relations} between
sound forms, we seek to determine the range and conditions of
\isi{variation} in what counts as ‘the same' linguistic element. To
determine this, we study the {rules} that make up the sound system of
the language: \isi{systematic relations} such as (in \ili{English}) ‘vowels are
longest before /z/, next longest before /n/, etc.'; ‘word-final
obstruents have the same \isi{voicing} as a preceding \isi{obstruent}'; ‘/k/ in
many forms corresponds to /s/ if an ending beginning with a non-low
front vowel follows'; (in \ili{Georgian}) ‘the sequence obstruent plus
liquid (or nasal) plus /v/ does not occur; where we might expect it,
we find the same \isi{consonants} but with the liquid or nasal following the
/v/'; etc. Statements of this sort, whatever formalism they may be
couched in, are fundamentally expressions that establish
correspondences between related forms: the form \emph{x} under
conditions \emph{A} corresponds to the systematically related form
\emph{y} under conditions \emph{B}. From the point of view of the
identity of higher-level units, that is, the difference between
\emph{x} (under conditions \emph{A}) and \emph{y} (under conditions
\emph{B}) is insignificant: they count as ‘the same thing'.

But whenever we say that (by virtue of rule R), \emph{x} and \emph{y}
count as linguistically ‘the same thing', the temptation is virtually
irresistible to ask what that ‘thing' is. When we propose an answer to
this sort of question, we are no longer describing the rule relating
\emph{x} and \emph{y}, but are rather proposing a
\textit{Representation} of what they have in common.

In linguistics (or at least in the study of the linguistic role of
\isi{sound structure}), a basic insight was gained by distinguishing between
\emph{phonetics} and \emph{phonology}, first expressed in these words
by {\Jespersen} as noted above.  This was based on the realization that
language-particular descriptions had to give some sort of account of
each language's distinctive {rules}, in the sense above of systematic
relations between phonetically distinguishable sounds that are
identified under specified conditions by the language in question.  It
is arguably the case that the earliest clearly `phonological' views
within the Western tradition were primarily or exclusively theories of
Rules, and that the question of Representations only arose somewhat
later.  As discussed in later chapters, Saussure deals with the
problem of identity vs.  diversity at the level of individual
segments, as do {\DeCourtenay} and {\Kruszewski} at the level of
morphological units; but the issue of uniform \isi{representations} for the
sets of elements so related only came up years later.

Quite soon in the ensuing development of {\Saussure}'s, {\Baudouin}'s and
{\Kruszewski}'s ideas, though, attention came to be focused on the
character of the presumed \emph{invariant} \isi{representations} which are
apparently implied as underlying the \emph{variants} whose relation is
systematically governed by the {rules} of a language.  As a result, most
of the history of twentieth century phonology is the history of
theories of Representations, devoted to questions such as ``what is
the nature of the \isi{phoneme}, morphophoneme, morpheme, etc.?''

From a strictly nominalist position, or even less extremely, we might
argue that this issue rests on a logical error. The fact that \emph{x}
and \emph{y} count as `the same thing' doesn't need to imply the
availability for study of such a `thing' which they both are (a point
made in a different but related context by
\citealt{linell79:psych.reality}).  The fact that we can construct an
expression `the set of all sets which are not members of themselves'
need not imply that such a paradoxical item has a claim to existence;
and in general we must not be misled in our ontology by the
possibilities provided by our meta-language.  But even a rather less
radical critique, not denying the significance of underlying
invariants in linguistic structure, might still argue that an
exclusive focus on questions of \isi{representations} leaves a great deal of
the rule-governed regularity which characterizes linguistic \isi{variation}
unaccounted for.  This is especially true if one accepts the shift of
focus noted above from \emph{languages} to \emph{grammars} as the
objects of linguistic inquiry.

In this work, the history of the balance between the study of {rules}
and the study of \isi{representations} (in the above sense) will be of
primary importance.  I hope to trace the sources of influence that
have led to a concentration at different times on one or the other;
and also to explore the ways in which facts that seem to fall clearly
in one domain can be accommodated within a theory of the other. In
this regard, it should be emphasized that it is not our intention to
argue that one sort of consideration is \emph{right} and the other
\emph{wrong} in a linguistic theory.  In fact, theories of {rules} and
theories of \isi{representations} deal with intimately inter-related and
indissoluble aspects of the same linguistic structure. In order to
understand that structure, however, both aspects must be appreciated,
and this has certainly not always been the basis on which inquiry into
\isi{sound structure} has proceeded.

\section*{The historical origins of modern views: a concrete example}

The motivation for a historical approach to the issues just raised is
clear.  When we seek to understand the conceptual bases of our own
theories, we can only do so in light of the recognition that they are
in part the residue of views held by others (our teachers, and their
predecessors). In order really to appreciate the logical content of
our \emph{own} views, then, it may be necessary (somewhat
paradoxically) to approach this task through a prior appreciation of
their historical antecedents --- \emph{taken on their own terms.} We
must `get inside' the position within which some problem originally
arose in order to understand its motivations and logical
underpinnings.

I  consider one example of such intellectual inheritance here.  It
is often taken as self-evident in phonological studies that underlying
(`phonemic' or `phonological') \isi{representations} should contain only
\textit{distinctive} or non-redundant material.  That is, in
arriving at the phonological representation of a form, one of the
steps involved is the elimination of all predictable properties, and
the reduction of the form to the minimum of specification from which
all of its other properties can be deduced by general rule. For many,
indeed, such a step establishes the fundamental difference between the
`phonological' and the `phonetic' representation of a given form.

Sometimes, however, this elimination of \isi{redundancy} turns out to have
undesired consequences. Occasionally, two or more properties, each of
which is predictable in terms of its environment, are inter-related in
such a way that both cannot be simultaneously eliminated from
\isi{phonological representations} without reducing the generality of the
resulting description.  In such a case, we must conclude that a
minimally redundant representation is not really to be desired.

In \ili{Russian}, for example, it has often been noted that the difference
between front [i] and back (or central) {[ɨ]} is not
distinctive: {[ɨ]} appears after `hard' (i.e.
non-palatalized) \isi{consonants}, while [i] appears elsewhere.  On this
basis, writers such as \citet{trubetzkoy39:grundzuge} concluded that the
phonological unit /i/ (represented by {[ɨ]} after `hard'
\isi{consonants} and by [i] elsewhere) is opposed to /u/ only in rounding,
and to /e/ in height.  A minimally redundant representation of this
vowel then would not contain any value for the feature [$\pm$Back],
since this is uniformly predictable.

Many \isi{consonants} in \ili{Russian} belong to pairs of corresponding `hard' and
`soft' \isi{consonants}, but not all.  Among those that are not
contrastively paired in this way are the velar obstruents /k/, /g/,
and /x/. Each of these appears in a phonetically `hard' variant before
back vowels ([u], [o], [a]) and in a phonetically `soft' variant
(phonetically, palatal) before front vowels ([i], [e]).  Since the
difference between the `hard' (velar) and `soft' (palatal) variants of
/k/, /g/, and /x/ is thus perfectly predictable, these segments are
presumably not to be specified for this property in a redundancy-free
description.

But now it should be apparent that there is a problem.  The backness
of the vowel /i/ is predictable from the presence vs.  absence of a
preceding \isi{hard consonant}; but the `hardness' of a prevocalic /k/, for
example, is predictable from the frontness of the following vowel.  In
fact, the sequence /ki/ is always pronounced with a `soft' [k,] and
front [i] (as in [puš'k,in] `Pushkin'); but if neither /k/ nor /i/
is specified for backness, it is not clear how to describe these
facts.

Of course, if /i/ is specified (redundantly) as `basically' front,
there is no problem: we need only say that a) velars become palatals
before front vowels, and b) /i/ becomes [+Back] after `hard'
\isi{consonants}.  Assuming that the difference between `hard' and `soft'
\isi{consonants} in \ili{Russian} is a matter of the same feature ([$\pm$Back]) as
the difference between back vowels and front vowels, this set of {rules}
expresses the assimilatory nature of the mutual accommodation between
vowel and consonant in a quite appropriate way.  If phonological
elements are only specified for their non-redundant properties,
however, the rule of velar \isi{palatalization} cannot make reference to the
frontness of a following /i/, and must be formulated as `velars become
palatals before a following non-low, non-round vowel.' The fact that
such vowels will in all cases be phonetically front (by virtue of the
rule which makes /i/ front after `soft' \isi{consonants}) is thus treated as
in principle quite independent, and the assimilatory nature of the
{change} is obscured.

In such a case, an apparently redundant property must evidently be
specified in \isi{phonological representations} if the generality of the
description is to be maintained.  Of course, it might be claimed that
this example is isolated and atypical of the structure of natural
languages.  In reality, however, the phenomenon of reciprocally
dependent properties is quite frequent in language, although its
consequences are not always recognized to be problematic.

The simplest case of this type, in fact, occurs so frequently that it
is not generally even noticed.  Suppose that two properties are
completely predictable from one another (at least in some
environment): e.g., given a cluster of nasal plus stop in many
languages, it is possible to predict the point of \isi{articulation} of
either from that of the other.  Typically, we specify one of the
properties (e.g., the \isi{articulation} of the stop) phonologically, and
include a rule to introduce the other (the \isi{articulation} of the nasal,
in this example).  We must realize however, that, from the point of
view of eliminating \isi{redundancy}, the decision to eliminate one of two
such inter-dependent properties rather than the other is either
completely arbitrary or at least based on ancillary principles of a
somewhat ad hoc sort which are seldom made explicit or precise.  In
the worst case, we may be forced to make choices that cannot be
defended on principled grounds just in order to meet the requirement
of eliminating \isi{redundancy}.  Among twentieth century phonologists, only
those of the British Prosodic school (cf.  chapter~\ref{ch.firth} below) have
been willing to take this point seriously enough to reconsider the
basis of the role played by \isi{redundancy} in linguistic descriptions.

I do not mean to defend here the opposite position, to wit, that all
predictable properties should be (systematically) included in
phonological form rather than being eliminated and then re-introduced
by rule. The point is rather to argue that there is at least an issue
to be addressed, and that particular answers to the problem of how
much information to include in \isi{phonological representations} have other
consequences which require them to be justified on principled
grounds. In particular, the position that such \isi{representations} should
be redundancy-free is not self-evidently correct. It is interesting to
note, indeed, that some speech scientists make exactly the opposite
assumption: that the only linguistically significant representation of
linguistic forms which speakers manipulate is one which is maximally
specified down to very low {levels} of phonetic detail (\name{Dennis}{Klatt},
personal communication).

In fact (as I will have occasion to argue below), it is perfectly
possible to develop a view of phonological forms which is consistent
with the fundamental function of these \isi{representations} in a grammar,
but in which (at least some) predictable detail is present. Again, it
is not our purpose here to argue for the correctness of such a view,
but only for the logical coherence of holding it.  For many linguists,
however, such a notion seems totally incompatible with the fundamental
nature of the difference between phonological and phonetic form.  It
is worth asking why this should be the case: what \emph{is} the relation
between `phonological' status and \isi{predictability}, and how did the
position arise according to which it is (all and) only
\emph{un}predictable properties that appear in phonological
\isi{representations}?

If indeed there are reasonable alternatives to such a view, and
positive arguments in favor of them, it is likely that the answer to
such a question will come only from a study of the history of the
relevant notions: \isi{phonological representations}, and predictable (or
redundant) properties. Consideration of these suggests two distinct
sources for the strongly held conviction that predictable properties
must necessarily be absent from \isi{phonological representations}.

A possible motivation for this position is found in one interpretation
of {\Saussure}'s notion of \isi{sound structure} (though this is not, I will
argue below in chapter~\ref{ch.saussure_sound}, the only one, or even
the one {\Saussure} himself appears to have held), on which the units in
phonological structures are \emph{identified} with sets of properties
setting them apart from other such units.  The doctrine that ``dans la
\isi{langue}, il n'y a que les différences'' has often been interpreted as
equating the phonological character of a sound with exactly those
properties that distinguish it from others --- no less, but no more.
Thus, there would be no room in such a representation for properties
that were not distinctive.

Secondly, and for completely independent reasons, the development of
the field of \isi{information theory} during the 1940s and 1950s stressed
the elimination of \isi{redundancy} as a necessary step in identifying the
information content of a message.  Those who (like {\Jakobson}, for
example: see chapter~\ref{ch.jakobson}) identified the phonological
form of an utterance with its potential information content were thus
led to require the elimination of predictable information from
phonological forms for this reason as well.

Either (or both) of these lines of reasoning may well be taken as
quite persuasive, and lead us to require that all redundant properties
be eliminated from phonological forms.  However, we should recognize
that current views on this issue are often not the product of
independent thought about the question itself, but rather are
inherited from previous researchers who reached them on the basis of
considerations such as those just adduced, and provided them to us
with the status of definitions.  As a result, if we want to assess
their value, we have to be able to reconstruct the arguments that led
to them --- and this implies reconstructing the logic of those who
developed them.

To do this, we cannot simply look for our own concerns to be reflected
in earlier work.  We must rather try to understand how \emph{our}
work reflects \emph{earlier} concerns.  To see our assumptions and
methodologies in the light of antecedent conceptual frameworks which
gave rise to them, we must ask what earlier workers thought they were
doing, and why, and how the results of their reflection were
transmitted to subsequent generations, including ourselves.

In the present instance, for example, we can note that the
interpretation given to {\Saussure}'s ideas by many of his immediate
successors arose out of their own conception of `\isi{structuralism},'
rather than out of any logical necessity inherent in Saussure's
position, and this weakens the force of their line of argument.
Similarly, the constructs of \isi{information theory} which seemed quite
persuasive to {\Jakobson} in the early 1950s would probably appear much
less relevant to the study of natural language today, given our
current understanding of the sheer bulk and internal \isi{redundancy} of the
mental storage of information.  Since the claim that ``{\Saussure} said
this,'' and so it must be true, and the notion that \isi{information theory}
dictates such a view --- two of the underpinnings of the \isi{redundancy}
free notion of phonological form --- can thus be argued to be less
than persuasive in present-day terms, we might well want to
re-evaluate our assumptions in this area.

Until I justify (in subsequent chapters) some of the assertions just
made about the history of phonological ideas, the argument just
outlined cannot by itself carry much conviction.  Nonetheless, it
should serve to illustrate the general point.  Despite what sometimes
appear to be dramatic changes in scientific `para\-digms' (in roughly
the sense of \citealt{kuhn62:revolutions}), it is often true that our
agenda of issues was set for us (at least in part) by our
predecessors.  Similarly, the range of possible solutions to any given
problem may well have been delimited by a previous generation in a way
we would not find adequate today, but which we retain as a part of the
cumulative conceptual structure of the field.  In order to understand
these issues, and to rethink them where necessary, we have to
understand the considerations that originally led to them.  That may
often require a considerable effort, where the basic work in an area
is remote from us in time and underlying assumptions.

\section*{The organization of the present book}

As a result of these considerations, any historically based study of
an important conceptual issue tends to involve a great deal of
more or less circumstantial biographical detail about influential
figures. This book is no exception: it might be proposed, for example,
to organize the subsequent chapters by issues to be addressed; but in
fact our discussion centers in a thoroughly traditional way around
individual figures and groups of figures, arranged in two parallel and
more or less chronological sequences.  Such a `great man' approach to
history may sometimes be inadequate to reveal the texture and
motivation of events, but it can be argued that when we study for
essentially internal reasons the rather limited domain of the history
of an individual intellectual discipline (such as linguistics), the
nature of the problem as posed makes it essential.

Furthermore, I would argue that the character of the field up until
around the 1950s makes the limitations of an approach centering on
individuals rather than issues comparatively innocuous.  During most
of the period studied here linguists worked in much greater isolation
from one another than is presently common, and the development of many
issues can be identified with the work of particular scholars to a
greater extent than would be possible in light of the  size and
professional coherence of the present community of linguists.

I start, then, by tracing in chapters~\ref{ch.saussure_life}
and~\ref{ch.saussure_sound} the development of phonology in the
beginnings of `\isi{structuralism}' in Europe.  This is based on a
consideration of the work of Saussure, and especially of the views on
sound-structure which we can reconstruct from the \textsl{Cours de
  linguistique g\'en\'erale}.  Although the views attributed to
{\Saussure} only became influential somewhat after their initial
expression, they nonetheless had a fundamental determinative influence
(at least in the interpretation given to them by later linguists) on
the development of the basic concepts of the field.  Furthermore, the
very nebulous character of {\Saussure}'s actual proposals in the area of
`phonology' (as this is presently construed) makes this somewhat
Delphic work an ideal source on the basis of which to introduce a
variety of issues that play a role in the subsequent discussion. In
this connection, I introduce in chapter~\ref{ch.saussure_sound} a
range of ways in which Saussure's basic insights about \isi{sound structure}
could potentially be worked out.  While the point of this exercise is
to develop a \isi{typology} of phonological theories which will prove useful
in later chapters, I also suggest a somewhat non-traditional
interpretation of {\Saussure}'s own theoretical position in this area.

I then move on in chapter~\ref{ch.kazan} to another of the initial
developers of the field, Jan {\DeCourtenay}, and his
collaborator \name{Mikołaj}{Kruszewski}.  Though rather less well known than
{\Saussure} (in substance, at least), {\DeCourtenay} and
{\Kruszewski} also had a important formative influence on the field,
especially on {Russian} and Eastern European linguists.  I follow this
influence in chapter~\ref{ch.prague} through the \isi{Moscow school} and the
early \isi{Prague School}, and eventually to the work of {\Trubetzkoy} and
{\Jakobson}.  {\Jakobson}'s own later development of this work is discussed
apart in chapter~\ref{ch.jakobson}.

Though they do not fit into the linear development implied by these
chapters, it is impossible to ignore the fact that there have been
other forms of `European \isi{structuralism}' than that represented by the
Moscow-Prague-Jakobson tradition. In chapter~\ref{ch.hjelmslev} I
sketch the phonological side of \isi{glossematic theory} (identified notably
with the work of Louis {\Hjelmslev}), which is of interest both because
its details are comparatively less known than others in current
discussion, and because it represents what is arguably the most
abstract version of the doctrines of \isi{structuralism}.  In
chapter~\ref{ch.martinet} I provide a brief account of the ``Functional
Phonology''of \name{André}{Martinet}, a view that grows out of the positions
of the \isi{Prague School} and that he developed in part in contact with
{\Hjelmslev}.

A more radical break with a single developmental sequence is forced in
order to consider work in the British tradition of Prosodic Analysis.
The close conceptual relations between this point of view and later
generative proposals about Autosegmental and Metrical structure in
phonology require that I give in chapter~\ref{ch.firth} at least an
abbreviated characterization of this theory, whose proud independence
from all of the forms of structuralist (phonemic) phonology is well
known --- indeed, sometimes overstated.

I then return in chapter~\ref{ch.boas} to the beginning of the
century, to trace the development of linguistics in North America.
While the earliest linguists of importance on this continent (e.g.
{\Whitney}) may still be said to fall into the European tradition, this
cannot be said of \name{Franz}{Boas} and his students, who originated a
genuinely independent approach to linguistic problems.  From {\Boas} we
continue in chapter~\ref{ch.sapir} to {\Sapir} and in
chapter~\ref{ch.bloomfield} {\Bloomfield}, and then treat in
chapter~\ref{ch.structuralists} the way in which the views of these
figures were (or were not) reflected in the influential theories of
(post-Bloomfieldian) American \isi{structuralism}.  The particular problem
of the status of \isi{morphophonemics} in this theory leads us to an
evaluation of the beginnings of generative phonology; in the unluckily
numbered chapter~\ref{ch.genphon} I attempt to sum up the relation
(both real and imagined) of generative phonology to the two major
strands (European and American) which constitute its past. In
chapter~\ref{ch.spe} I discuss the program of {\Chomsky} \& {\Halle}'s
landmark work \textsl{The Sound Pattern of English}, and assess some
of the immediate reactions to perceived deficiencies in the program of
that work. 

FInally, in chapter~\ref{ch.otlabphon}, I consider ways in which
phonological thinking evolved in the last decades of the
century. These began with questions of representational structure,
including the emergence of autosegmental and metrical formalisms and
the emergence of ``\isi{feature geometry}'' as a research issue. Soon,
however, these comparatively modest deviations from the \textsl{\isi{SPE}}
were replaced by a much more radical departure: \isi{Optimality Theory}, a
view that denied outright the existence of language-particular
\isi{phonological rules}, in favor of a set of (in theory, universal)
\isi{constraints} on \isi{representations} applying simultaneously on the basis of
a language-particular ranking to produce a direct mapping between
underlying and surface \isi{representations}. Although this view rapidly
captured most thinking in the field, the early years of the new
century would see new problems and some disillusion with ``OT'' and
the development of quite different approaches, but an assessment of
those developments must await another book. The chapter concludes with
some discussion of another radical attempt to re-found the study of
phonological structure, the \isi{Laboratory Phonology} movement and the
related theory of Articulatory Phonology.

The present organization into two broad streams of development,
European and American, contributes to the contuity of development
which can be found in each, but unfortunately tends to obscure the
inter-relations and contacts that existed between the two. A useful
perspective on exactly this issue is provided by
\citet{newmeyer21:structuralism}, who focuses exactly on the
connections between these two streams of theoretical thought.

This is hardly the first book to be written about the history of
linguistics, or the first to deal with phonology in the twentieth
century.  In preparing the first edition, such works as
\citet{robins67:history}, and especially
\citet{fischer-jorgensen75:trends} provided a wealth of invaluable
information about the general history of the field, without which the
present study would hardly have been possible.  Specialized studies,
including a number of the articles in \citet{jakobson71:sel.wr.2} and
\citet{hymes74:traditions}; \citet{kilbury76:morphophonemics};
\citet{stankiewicz72:baudouin.anthology};
\citet{langendoen68:london.school}; and
\citet{hymes.fought81:structuralism} among many others were also of
great use.  \citet{newmeyer:ltia2} was also quite valuable, though his
work concentrates its attention on the history of syntactic
studies. In the years since the appearance of the first edition of the
current book, a number of specialized studies of the work of
individual linguists have appeared, and these will be cited where
appropriate in the chapters below. A vaulable comprehensive survey by
a range of scholars dealing with topics in the history of phonology,
\citet{dresher.vanderhulst21:oxford.handbook} is currently in
progress, and I have made use of some of the chapters of this work in
pre-publication form.

The primary concern of much of the existing literature dealing with
the development of phonological theory, however, has been to establish
the external history of the figures and events involved in the
development of the field, to clarify their influences on one another,
and to present their views in responsible and coherent form. These are
anything but negligible accomplishments; they do not, however, obviate
studies of particular central issues and their origins such as the
present one. If much of the substantive content of the chapters below
can be found in the published literature, there is still a point in
bringing it together in a different way, and in applying it to the
problem which forms our focus.

While there is much that is well-known and accepted about what I have
to say below concerning various historical figures, there are also
some places in which the interpretations to be presented diverge from
commonly held views.  This is particularly true with regard to
{\Saussure}, who made the task of subsequent historians immeasurably more
difficult by not himself presenting any account whatsoever of sound
structure (as a part of general linguistics), at least in published
form.  We are thus left to infer his position from rather sparse
notes, and from the codification his views received at the hands of
his students and colleagues.

Substantiating a particular reading of {\Saussure} raises another general
issue. If the present work were intended exclusively as a contribution
to the history of linguistics \emph{per se}, it would be incumbent
on us to establish this interpretation with extensive citations from
the literature of Saussureana.  Such an effort of scholarship would,
however, take us rather far from our central concerns.  I content
myself here with suggesting what I feel to be a plausible view,
and sketching its relation to the ways in which {\Saussure}'s work has
generally been interpreted in the tradition.  It is to be hoped that
our view will be found sufficiently useful to warrant subsequent
scholarship which may determine its justification as an interpretation
of {\Saussure}'s own picture of \isi{sound structure}.  Similar (but perhaps
less important) considerations could be raised in regard to our
presentation of other historical figures as well.

Most of the work of establishing the facts concerning the history of
phonology in the twentieth century has either been done by others, or
would divert us from the purpose of examining the issue of how \isi{rules}
and \isi{representations} relate to one another.  Nonetheless, it is to be
hoped that enough of a picture has been presented below to allow even
those not familiar with the specialized literature to form a coherent
and generally accurate view of the development of the field.

%%% Local Variables: 
%%% mode: latex
%%% TeX-master: "/Users/sra/Dropbox/Docs/Books/P20C_2/LSP/main.tex"
%%% End: 
