\documentclass[output=paper]{LSP/langsci} 
\author{Martina Wiltschko  \affiliation{University of British Columbia}
}
\title{Response particles beyond answering} 
% \epigram{Change epigram}
\abstract{In recent years, response particles (\textit{yes/no}) have received some attention in the formal syntactic and semantic literature. 
Most analyses focus on the use of response particles as answers to polar questions as well as (to a lesser extent) as responses to affirmations. In this paper I extend the empirical domain to explore the use of response markers as responses to other clause types including wh-questions, imperatives, and exclamatives. It is established that response particles can be used as (dis)agreement markers. Moreover it is shown that in German, response particles can also be used to mark the following utterance as a response.
A unified analysis is developed according to which the difference in function of response markers is syntactically conditioned.
Following recent work on the syntax of speech acts, an articulated speech act layer is utilized to derive these functions.  The case is made for a more fine-grained typology of response markers than previously assumed.} 

\ChapterDOI{10.5281/zenodo.1117726}
\maketitle

\begin{document}
\newcommand{\qab}[2][.88]{\parbox[t]{.8\textwidth}{\vspace*{-#1\baselineskip}#2}}
\section{Introduction}\label{sec:wiltschko:1}

In his recent monograph \citep{holmberg:15}, Anders Holmberg extends the empirical domain for generative syntacticians by exploring the syntax of \textit{yes} and \textit{no} (henceforth \textit{response particles}, \textit{ResPrt}) as in \REF{ex:wiltschko:1} (see also \citealt{Holmberg2001yesnofinnish,Holmberg2002,Holmberg2007,Holmberg2013,Holmberg2014finnishquestion}).

\ea
\label{ex:wiltschko:1}
   \begin{xlist}\exi{Q:} {Did you feed the dog?}
   \exi{A:}\begin{xlista}
      \ex
	  {Yes.}   (= {I fed the dog.})
      \ex
	    {No.}  (= {I didn’t feed the dog}.)  
      \end{xlista}
\end{xlist}\z

While \textit{ResPrt}s have been explored within other subfields of linguistics (e.g., conversation analysis) they have not been part of the core body of data generativists have typically taken into account (with the early exception of \citealt{Pope1976}, and more recent studies such as \citealt{FarkasBruce2009,KramerRawlins2009}, and \citealt{Krifka2013}.) The absence of \textit{ResPrt} from the syntactician’s empirical domain may have to do with two factors. First, \textit{ResPrts} are only found in conversations, while syntactic theory is typically concerned with sentences in isolation. Secondly, \textit{ResPrt}s – as the term \textit{particle} suggests – are frequently morphologically simplex. That is, in many languages, neither positive nor negative \textit{ResPrts} display any surface complexity: they are mono-morphemic.
If we consider syntax to be concerned with understanding the ways complex structures are derived, then \textit{ResPrt}s are not obviously an interesting object of exploration. However, modern syntactic theory is not only concerned with understanding word- or morpheme-order restrictions but it is a way to explore the relation between form and meaning. And in this respect, \textit{ResPrts} are in fact interesting. Despite their \isi{morphological} simplicity, they are able to convey a full fledged positive or negative proposition. So, the first \isi{question} that is of interest to syntactic theory concerns the relation between the form of the \textit{ResPrt} and its interpretation: how can we model the fact that a seemingly simplex form can convey a full proposition? And how is the content of this proposition determined? In \sectref{sec:wiltschko:2}, I review two current approaches to this \isi{question}: Holmberg’s ellipsis-based account and \citegen{Krifka2013} pronominalization account. I then move on to the core empirical contribution of this paper. In particular, I explore other uses of \textit{ResPrts} (\sectref{sec:wiltschko:3}) and whether they can be accounted for under current analyses. \textit{ResPrts} serve as answers if they are used to respond to polar questions; but this is not their only function. Rather, I show that \textit{ResPrts} can be used as responses to clause-types other than polar questions, in which case they function as agreement or disagreement markers, respectively. In \sectref{sec:wiltschko:4}, I propose an analysis for the (dis-)agreement function of \textit{ResPrts:} they establish how the trigger of the response relates to the responder’s set of beliefs. Furthermore, in \sectref{sec:wiltschko:5}, I introduce another use of \textit{ResPrts:} in \ili{German} \textit{ResPrts} can be used to mark the \isi{utterance} they precede as a response. 
In \sectref{sec:wiltschko:6}, I conclude.

For the purpose of this paper, I adopt the following terminological and representational conventions. It will be useful to distinguish between what the \textit{ResPrt} \textit{responds to} and what it \textit{responds with}. I refer to the former as the \textsc{trigger (}of response) and to the latter as the \textsc{content (}of response\textsc{).} This is exemplified in (\ref{ex:wiltschko:2}′) for the example in \REF{ex:wiltschko:1}. Here the \textsc{trigger} of the response is the polar \isi{question} (\textit{Did you feed the dog}?), which (by virtue of containing an unvalued \isi{polarity} variable) introduces a proposition and its \isi{negation} (p (\textit{B fed the dog)} $\lor ¬$p (\textit{B didn’t feed the dog})). If the answer given is \textit{yes}, the \textsc{content} of the response is the affirmation of the positive proposition (p: \textit{B fed the dog.}). If the answer given is \textit{no}, the \textsc{content} of the response is the \isi{negation} of the proposition ($¬$p: \textit{B didn’t feed the dog)}.\footnote{For a discussion of answers to negative questions see \sectref{sec:wiltschko:2} below.}  

\begin{exe}
\exp{ex:wiltschko:1}\label{ex:wiltschko:2}\settowidth\jamwidth{\textsc{trigger:} polar \isi{question} (p $\lor ¬$p)}
\begin{xlist}
\exi{A:} Did you feed the dog?   \jambox{\textsc{trigger:} polar \isi{question} (p $\lor ¬$p)}
\exi{B:}\begin{xlist}
         \exi{} Yes. (= I fed the dog.) \jambox{\textsc{content:} affirming p (= p)}
         \exi{} No.  (= I didn’t feed the dog.)  \jambox{\textsc{content:} negating p (= $¬$p)}
\end{xlist}
\end{xlist}
\end{exe}

\noindent Furthermore, I will use the term \textit{responder} to refer to the speech-act participant who is responding; and I will use the term \textit{respondee} to refer to the speech-act participant who the responder is responding to (i.e., the {person} who uttered the \textsc{trigger} of the response).

\section{Holmberg’s syntax of answers}\label{sec:wiltschko:2}

\citet{holmberg:15} (following previous work of his) argues that \textit{ResPrts} that are used to answer polar questions are best analyzed as combining with a full propositional structure the content of which depends on the preceding \isi{question}. Their apparent simplicity stems from the fact that the propositional structure can be elided (i.e., remain unpronounced) as shown in \REF{ex:wiltschko:3}, where strike-through indicates the elided constituent. 

\ea\label{ex:wiltschko:3}
   Q: {Did you feed the dog?}

   A: {Yes} [\sout{I fed the dog}].
\z

This much accounts for the distributional properties of \textit{ResPrts} – as we shall see – but what about their interpretation? How can they serve as answers to polar questions? Holmberg argues that polar questions introduce a \isi{polarity} variable [±pol] inside the propositional structure (henceforth p-structure). In particular, as illustrated in \REF{ex:wiltschko:4}, the \isi{polarity} variable is analyzed as the head of a \isi{polarity} phrase between CP\is{complementizer} and \isi{TP} (though the position of PolP is assumed to be subject to cross-linguistic variation in \citet{holmberg:15}; cf. 
also \citet{Laka1990} for an early version of this idea ).

\ea\label{ex:wiltschko:4}
    {Did you feed the dog?}\\
    \begin{forest}
     [CP\is{complementizer} [C\\Aux\is{Auxiliary},base=top,align=center] [PolP [Subj] [Pol\is{polarity} [Pol\is{polarity}\\\relax\textbf{{[}± pol{]}}] [\isi{TP} [~~~~~~~~~,roof]] ] ] ]
    \end{forest}
\vspace*{-\baselineskip}
\z
 
Thus, according to \citet[4]{holmberg:15} the interpretation of a polar \isi{question} is something like: \textit{What is the value of} [±pol] \textit{such that ‘you fed} [±pol] \textit{the dog’ is true?}. The contribution of the \textit{ResPrt} is to bind the \isi{polarity} variable in the embedded p-structure. 
It does so from the specifier position of a focus phrase (FocP). If the answer is \textit{yes}, the \isi{polarity} variable is valued as [+pol], yielding the answer [\textit{you} [+pol] \textit{fed the dog}] as in \REF{ex:wiltschko:5a}. In contrast, if the answer is \textit{no}, the \isi{polarity} variable is valued as [-pol] yielding the answer [\textit{you} [-pol] \textit{fed the dog}]\textit{}, as in \REF{ex:wiltschko:5b}, which translates as ‘\textit{You didn’t feed the dog’.} 


\noindent\parbox{\textwidth}{\ea\label{ex:wiltschko:5}
The \textit{ResPrt} binds the \isi{polarity} variable
\begin{multicols}{2}
\begin{xlista}
 \ex\label{ex:wiltschko:5a} the contribution of \textit{yes}\\
\begin{forest} nice empty nodes
[FocP [\textit{\bfseries yes},name=yes] [Foc\is{Focus} [Foc\is{Focus}] [PolP [] [Pol\is{polarity} [Pol\is{polarity}\\\relax\textbf{{[}+ pol{]}},name=pol] [TP [~~~~~~~~~,roof]] ] ] ] ] 
\path [-{Stealth[]}] (yes) edge [bend right] (pol);
\end{forest}
\ex\label{ex:wiltschko:5b} the contribution of \textit{no}\\
\begin{forest} nice empty nodes
[FocP [\textit{\bfseries no},name=no] [Foc\is{Focus} [Foc\is{Focus}] [PolP [] [Pol\is{polarity} [Pol\is{polarity}\\\relax\textbf{{[}- pol{]}},name=pol] [TP [~~~~~~~~~,roof]] ] ] ] ]  
\path [-{Stealth[]}] (no) edge [bend right] (pol);
\end{forest}
\end{xlista}
\end{multicols}
\z}

%%\includegraphics[width=\textwidth]{a13Wiltschko-img15}

The reason that the constituent following the \textit{ResPrt} (i.e., PolP) can be elided is that it is essentially identical to the propositional clause in the \isi{question} it answers, i.e., it has an antecedent. 

Since anaphoricity can be signalled via ellipsis or via pronominalization, it is not surprising that \textit{ResPrts} have also been analyzed in terms of pronominalization. For example, \citealt{Krifka2013} argues that \textit{ResPrt}s can be viewed as \textit{propositional anaphors}. As such, they are assumed to replace the entire p-structure, as illustrated in \REF{ex:wiltschko:6}.\footnote{Replacing p-structure is however not the only possibility for response markers in \citegen{Krifka2013} model. In particular, he assumes that p-structure is dominated by a speech act Structure (ActP) which, in turn, can also serve as the antecedent for a propositional anaphor. Depending on which layer of the clausal spine the propositional anaphor picks out, their interpretation differs. As we shall see the proposal developed here builds on this insight, but introduces a more fine-grained speech act structure.}  

\ea\label{ex:wiltschko:6}
   \textit{ResPrt} as propositional anaphors\\[1\baselineskip]
   \begin{forest}
      [p-structure,draw [\textit{yes\slash no},roof] ]
   \end{forest}
\z

%%\includegraphics[width=\textwidth]{a13Wiltschko-img16}

One empirical fact that speaks in favor of the ellipsis approach of the type developed by Holmberg (see also  \citealt{KramerRawlins2009} and  \citealt{HaegemanWeir2015}) is the fact that the proposition that serves as the antecedent for the \textit{ResPrt} can be pronounced, as shown in \REF{ex:wiltschko:7}.

\ea\label{ex:wiltschko:7}
   \begin{xlist}\exi{A:}  Did you feed the dog?
   \exi{B:}
   \begin{xlista}
    \ex Yes, I fed the dog.
    \ex No, I didn’t feed the dog.
   \end{xlista}
\end{xlist}\z

The well-formedness of the complex answers in \REF{ex:wiltschko:7} is immediately predicted by the ellipsis analysis: the p-structure need not be elided since ellipsis is generally not obligatory. In contrast, the pronominal analysis, according to which \textit{ResPrts} are propositional anaphors, will have to be augmented to accommodate the facts in \REF{ex:wiltschko:7}. 

\citet[2--6]{holmberg:15} discusses two more pieces of evidence for the syntactic complexity of \textit{ResPrts}: one pertaining to their form and the other to their meaning. 

Consider first variation in the \textit{form} of polar responses. Not all languages make use of \textit{ResPrt}s to answer polar questions. Another cross-linguistically common strategy to answer polar questions is to repeat (echo) the verb (or auxiliary\is{Auxiliary}) of the \isi{question} with the remainder of the proposition elided.
This is exemplified in \REF{ex:wiltschko:8} on the basis of \ili{Finnish}.  

\ea\label{ex:wiltschko:8}
\langinfo{Finnish}{}{\citealt[3 (6)]{holmberg:15}}\\
\begin{xlist}\exi{Q:}
  \gll  Tul-i-vat-ko lapset kotiin?\\
  come-\textsc{pst}-\textsc{\oldstylenums{3}pl}-\textsc{q} children home\\
   \glt ‘Did the children come home?’ 
\exi{A:}
\gll  Tul-i-vat.\\
      come-\textsc{pst}-\textsc{\oldstylenums{3}pl}\\
 \glt ‘Yes.’ 
\end{xlist}\z

This cross-linguistic pattern lends support to the ellipsis analysis of \textit{ResPrts} as it allows for a unified analysis of polar responses. 

The other piece of evidence Holmberg considers pertains to differences in the distribution and \textit{interpretation} of \textit{ResPrts}. There are essentially two types of patterns languages display. The two patterns are distinguishable based on responses to negative polar \isi{question}. The first strategy is the so called \textit{agree/disagree} system (cf.
\citealt{Kuno1973,Pope1976}, and  \citealt{SadockZwicky1985}) also known as the \textit{truth-based} system \citep{Jones1999}. This system is characterized by the fact that a positive response to a negative polar \isi{question} indicates agreement with the respondee: both the respondee and the responder believe in the negative proposition. Hence, a positive answer is used to assert a [-pol] value for p. This is exemplified by the \ili{Cantonese} data in \REF{ex:wiltschko:9}. 

\ea\label{ex:wiltschko:9}
\langinfo{Cantonese}{}{\citealt[4 (9)]{holmberg:15}}\\
\begin{xlist}\exi{Q:}
\gll John m jam gaafe?\\
John not drink coffee\\
\glt ‘Does John not drink coffee?’ 

\exi{A:} 
\gll Hai.\\
yes\\
\glt (‘John does not drink coffee.’)
\end{xlist}\z

The second strategy is the \textit{positive/negative system} also known as the \textit{polarity-based} system. This system is characterized by the fact that a negative response to a negative polar \isi{question} indicates that the \isi{polarity} of the proposition is valued as [-pol]. Hence, unlike in the agree/disagree system, a negative answer is used to assert a [-pol] value for p. This is exemplified by the \ili{Swedish} data in \REF{ex:wiltschko:10}.

\ea\label{ex:wiltschko:10}
\ili{Swedish} (\citealt[4 (10)]{holmberg:15})\\
\begin{xlist}\exi{Q:}
\gll Dricker Johan inte kaffe?\\
drinks Johan not coffee \\
\glt ‘Does Johan not drink coffee?’
\exi{A:} 
\gll Nej. \\
no \\
\glt (‘He doesn’t drink coffee.’)
\end{xlist}\z

In sum, in a truth-based system, the use of a positive \textit{ResPrt} results in an interpretation according to which the negative proposition is asserted to be true.
In contrast, in a \isi{polarity} based system the same effect is achieved by means of the negative \textit{ResPrt}. According to Holmberg, the difference between the two systems reduces to a syntactic difference in \isi{negation}. That is, \citet{Ladd1981} observes that a negative polar \isi{question} like \REF{ex:wiltschko:11} can have two readings (see also \citealt{BüringGunlogson2000,RomeroHan2004,AsherReese2007}). The first reading \REF{ex:wiltschko:11i} introduces a negative bias and is characterized by \textit{low scope} of \isi{negation}; hence this is known as the “inside \isi{negation} reading”. The second reading \REF{ex:wiltschko:11ii} introduces a positive bias and is characterized by high scope of \isi{negation}; hence it is known as the “outside \isi{negation} reading”. 

\ea\label{ex:wiltschko:11}
\begin{xlist}\exi{Q:}   Doesn’t John drink coffee?

\exi{A:}
\begin{xlisti}
  \ex\label{ex:wiltschko:11i}\relax   {Is it true that John does} \textbf{{not}} {drink coffee?}  [low \textsc{neg}]
  \ex\label{ex:wiltschko:11ii}\relax   {Is it} \textbf{{not}} {the case that John drinks coffee?}  [high \textsc{neg}]
\end{xlisti}\end{xlist}
\z

To distinguish the two readings we can add the negative \isi{polarity} item \textit{either}, which forces the low \isi{negation} reading (\ref{ex:wiltschko:12i}). Alternatively, we can add the positive \isi{polarity} item \textit{too}, which forces the high \isi{negation} reading (\ref{ex:wiltschko:12ii}).  

\ea\label{ex:wiltschko:12}
\begin{xlisti}
\ex\label{ex:wiltschko:12i}\relax {Doesn’t he drink coffee either?}        [low \textsc{neg}]\\
= Is it also the case that \textbf{he does not drink coffee}?

\ex\label{ex:wiltschko:12ii}  {Doesn’t he drink coffee too?}         [high \textsc{neg}]\\
      = Is it \textbf{not} also the case that \textbf{he drinks coffee}?
\end{xlisti}
\z

\noindent The difference between high and low \isi{negation} affects the syntax of \textit{ResPrts}: if the elided proposition contains \isi{negation} (as is the case with low \isi{negation}), then a positive \textit{ResPrt} is used to mean ‘\textit{Yes it is the case that not p}’\textit{;} if the elided proposition does not contain \isi{negation} (as is the case with high \isi{negation}), then the negative \textit{ResPrt} has to be used to achieve the same result because the positive \textit{ResPrt} would have to be interpreted as ‘\textit{Yes, it is not the case that p}’, which is not a well-formed answer. In other words, \textit{yes}, has to agree in \isi{polarity} with the \isi{assertion} rather than with the proposition. 

\ea\label{ex:wiltschko:13}
\begin{xlist}\exi{Q:}   {Doesn’t John drink coffee}
\exi{A:}
\begin{xlisti}
      \ex\relax \begin{tabular}[t]{lll} {Yes}. & (={He does drink coffee.})    & \textsc{content:} p\\
                                               & (={He doesn’t drink coffee.}) & \textsc{content:} $¬$p\\ 
                \end{tabular}
      \ex       \begin{tabular}[t]{lll} No.    & (={He doesn’t drink coffee.}) & \textsc{content:} $¬$p\\
                                               & (= {He does drink coffee.})   & \textsc{content:} p  
                \end{tabular}
\end{xlisti}
\end{xlist}
\z

\noindent To obtain a positive response in such contexts, some languages make use of a dedicated \textit{ResPrt}, namely a \isi{polarity} reversing particle.
This is exemplified in \REF{ex:wiltschko:14} by \ili{German} \textit{doch} (\citealt{holmberg:15}: ch. 6; \citealt{Krifka2013}).\footnote{The Old English \textit{ResPrt} system used to distinguish between two forms of positive \textit{ResPrts:} \textit{gae} was used to answer positive utterances while \textit{gyse} was used to answer negative ones, mirroring the difference between \ili{German} \textit{ja} and \textit{doch} \citep{WallageVanderWurff2013}.}  

\ea\label{ex:wiltschko:14}
\ili{German}\\
\begin{xlist}\exi{Q:}
   \gll Trinkt Hans nicht Kaffee?     \\
   drinks Hans  not   coffee \\
\glt ‘Does Hans not drink coffee?’
\exi{A:}
  \gll \textbf{Doch} (er trinkt Kaffee).\\ 
  yes\\
  \glt (‘He does drink coffee.’)
\end{xlist}
\z

\noindent In sum, what Holmberg’s study establishes is that \textit{ResPrts} are syntactically complex: they are sensitive to categories that are syntactically defined, namely the distinction between low and high \isi{negation}.

In addition, the syntactic treatment of \textit{ResPrts} has another advantage: it makes it possible to explore the cross-linguistic differences in a systematic way. And there are good reasons to explore this variation. The form and function of \textit{ResPrts} is under-documented: existing grammars of individual languages do not often contain information about the strategies used to answer polar questions. Hence, exploring this \isi{question} from a cross-linguistic point of view will contribute to our knowledge base, which in turn will inform the formal analyses of \textit{ResPrts}. 

The present paper contributes to the \isi{question} regarding the range of variation. In particular, I explore other uses of \textit{ResPrts}, hence extending the typological space within which to investigate them. That is, in addition to Holmberg’s two questions 
(i) does a language make us of the \textit{ResPrt} strategy and 
(ii) how do \textit{ResPrts} pattern as answers to negative questions, we can also ask questions about the other functions of \textit{ResPrts}. In particular, in what follows, I show that \textit{ResPrts} can be used as markers of (dis)agreement (Sections \ref{sec:wiltschko:3}--\ref{sec:wiltschko:4}) and as generalized response markers (\sectref{sec:wiltschko:5}). 

\section{\textit{Yes} and \textit{no} as markers of (dis)agreement}\label{sec:wiltschko:3}

The bulk of \citegen{holmberg:15} treatment of \textit{ResPrts} is dedicated to their use as answers to polar questions (henceforth the \textbf{\textit{answering}} \textit{function}). This \textit{answering} function of \textit{ResPrts} comes about when the \textsc{trigger} of the response is a polar \isi{question} and the \textsc{content} is either affirmation or \isi{negation}, as summarized in \REF{ex:wiltschko:15}.

\ea\label{ex:wiltschko:15}
Conditions for the answering function of \textit{ResPrts}
\begin{itemize}
 \item[] \textsc{trigger:} polar \isi{question} (p $\lor ¬$p)
 \item[] \textsc{content} of response:
	 \begin{xlisti}
	  \ex \textit{yes:} affirming p (= p),
          \ex \textit{no:}   negating p (= $¬$p)
	 \end{xlisti}
\end{itemize}
\z

However, \textit{ResPrts} can be used in a variety of other contexts that go beyond the answering function. 

\subsection{\textsc{triggers} across clause-types}

In this section, I explore the use of \textit{ResPrts} following \textsc{triggers} other than polar questions. To make a systematic exploration possible, it is useful to make explicit some assumptions about the relation between \isi{utterance} form (\textit{clause type}) and \isi{utterance} function (\textit{speech act type}). I assume a (simplified) mapping between clause-type and speech act-type.
In particular, I assume that declaratives map onto assertions; interrogatives map onto questions; imperatives map onto commands or requests; and exclamatives map onto exclamations. Thus, for the purpose of this paper, I abstract away from indirect speech acts and other forms of modifying speech acts. The mapping is summarized in 1. 

\begin{table}
\begin{tabularx}{.5\textwidth}{XX}
\lsptoprule
{Utterance form} & {Utterance function}\\
\midrule
Declarative & \isi{Assertion}\is{assertion/assertive}\\
Interrogative & \isi{Question}\is{question}\\
Imperative & Command\slash request\\
Exclamative & Exclamation\\
\lspbottomrule
\end{tabularx}
\caption{Mapping between utterance form and utterance function}

\end{table}

In what follows I explore the possibility of responding with a \textit{ResPrt} to each of these \isi{utterance} forms.

\subsubsection{Responding to assertions}
As discussed in \citet{holmberg:15}, \textit{ResPrts} can be used to respond to assertions (cf. 
also \citealt{FarkasBruce2009,Krifka2013}). In this use, they are sometimes referred to as \textit{rejoinders} (\citealt{HallidayHasan1976}) but I will refer to them as (dis)agreement markers. Consider the examples in (\ref{ex:wiltschko:16}--\ref{ex:wiltschko:17}). Assertions are encoded with declarative syntax and falling intonation (indicated by {\textbackslash}). Note that (dis)agreement markers, too, are associated with falling intonation. 

\settowidth\jamwidth{\textsc{content:} disagreement w/p}
\ea\label{ex:wiltschko:16}
\begin{xlist}
 \exi{A:} John speaks \ili{French} really well {\textbackslash}.   \jambox{\textsc{trigger:} \isi{assertion} (p)}
 \exi{B:} \begin{xlisti}
          \ex\relax {\textit{Yes} {\textbackslash}}. (= p)  \jambox{\textsc{content:} agreement w/p}
          \ex\relax {\textit{No} {\textbackslash}}. (=$¬$p)\footnotemark \jambox{\textsc{content:} disagreement w/p}
          \end{xlisti}
\end{xlist}          
(adapted from \citealt[211 (4)]{holmberg:15})
\z 
\footnotetext{According to \citet{holmberg:15}, \textit{no} cannot be used as a disagreement marker without adding more content to the response. According to my consultants, however, the short answer is well-formed though it comes across as confrontational. One might therefore reframe Holmberg’s generalization as follows: a negative response is ill-formed only in polite conversations. Note also that there appears to be a special intonation associated with it. I tentatively identify this as the contradiction contour (\citealt{LibermanSag1974}).} 

\ea\label{ex:wiltschko:17}
\begin{xlist}
\exi{A:} You stole the cookie {\textbackslash}.     \jambox{\textsc{trigger:} \isi{assertion} (p)}
\exi{B:}
  \begin{xlisti}
  \ex \textit{Yes {\textbackslash}}. (= p)        \jambox{\textsc{content:} agreement w/p}
  \ex \textit{No {\textbackslash}}.  (=$¬$p)      \jambox{\textsc{content:} disagreement w/p}
  \end{xlisti}
 \end{xlist}
(adapted from \citealt[2 (2a)]{Krifka2013})
\z

Despite the difference in the \textsc{trigger}\textsc{}, \textit{ResPrts} still express the same \textsc{content} as in their answering function\textsc{:} affirmation or \isi{negation}\textsc{.} Nevertheless, the effect of the \textit{ResPrt} is different. With a positive response to an \isi{assertion}, the responder \textit{agrees} with the previous \isi{utterance} and conversely, with a negative response, the responder \textit{disagrees} with the previous \isi{utterance} (cf.
\citealt{FarkasBruce2009}). 

This contrasts with \textit{ResPrts} when used as answers to polar questions. In this case, there is nothing to agree with, because no statement is being made with which the responder could agree or disagree.
Polar questions are used to shift the commitment to p from the speaker (S) to the addressee (A) \citep{Gunlogson2003}, thereby requesting an answer from A. If the respondee is committed to the content of her \isi{utterance} (as is the case with an \isi{assertion}), it follows that the response will be interpreted as (dis)agreement. In contrast, if the respondee is not committed to the content of her \isi{utterance} (as is the case with polar questions), it follows that the response is not interpreted as agreement or disagreement, but as an answer. 

Within the syntactic analysis developed in \citet{holmberg:15}, the difference between the answering function and the (dis)agreement function is as follows. As we saw above, \textit{ResPrts} used as answers are analyzed as occupying SpecFocP c-commanding an embedded p-structure, which contains an unvalued \isi{polarity} variable (the head of PolP). \textit{Yes} values this variable as [+pol] while \textit{no} values it as [-pol]. 

As for their (dis)agreement function, \citet[81]{holmberg:15} suggests that it does “not assign a value to a \isi{polarity} variable, because there is no \isi{polarity} variable in the preceding statement.”

\citet{holmberg:15} doesn’t offer an explicit syntactic analysis for the (dis)agreement function of \textit{ResPrts}, but given his description of this phenomenon, we may conclude that the structure is something like in \REF{ex:wiltschko:18}.\footnote{I have left the label for the structure dominating the \textit{ResPrt} vague (X). This is because \citet{holmberg:15} suggests two possible analyses: one according to which the (dis)agreement function is instantiated by a different type of \textit{yes/no}, one that is more akin to predicates like \textit{true} or \textit{false} which can take a valued proposition as their complement. The other option is that the (dis)agreement function is instantiated by the same lexical element as the answering function: it still is associated with a focus \isi{projection} but it doesn’t bind the \isi{polarity} variable associated with PolP.} 

\ea\label{ex:wiltschko:18}
\ea
{yes}\\\vspace*{-\baselineskip}
\begin{forest} 
 [XP, s sep=2cm [\textit{yes}] [Pol\is{polarity},name=Pol [Pol\is{polarity}\\\relax\textbf{{[}+ pol{]}},name=pol] [TP,name=TP [~~~~~~~~~~~~~~~~~~,roof] ] ] ]
 \draw ($ (pol.south west) +(-.5cm,-.25cm) $) .. controls ($ (Pol) +(0,2cm) $) .. ($ (TP) +(1.5cm,-1cm) $) coordinate[pos=0.75,above] (elidedstart);
 \node[right=2em of elidedstart, baseline=elidedstart] (elidedtext) {elided};
 \draw[-{Stealth[]}] (elidedstart) -- (elidedtext);
\end{forest}
\ex
no\\\vspace*{-\baselineskip}
\begin{forest} 
 [XP, s sep=2cm [\textit{no}] [Pol\is{polarity},name=Pol [Pol\is{polarity}\\\relax\textbf{{[}- pol{]}},name=pol] [TP,name=TP [~~~~~~~~~~~~~~~~~~,roof] ] ] ]
 \draw ($ (pol.south west) +(-.5cm,-.25cm) $) .. controls ($ (Pol) +(0,2cm) $) .. ($(TP) +(1.5cm,-1cm) $) coordinate[pos=0.75,above] (elidedstart);
 \node[right=2em of elidedstart, baseline=elidedstart] (elidedtext) {elided};
 \draw[-{Stealth[]}] (elidedstart) -- (elidedtext);
\end{forest}
\vspace*{-\baselineskip}
\z
\z


Like in their answering function, the (dis)agreement \textit{ResPrts} combine with an elided p-structure, the \textsc{content} of which (including its \isi{polarity} value) is determined by the \textsc{trigger} of the response.
This assumption is consistent with the fact that the \textsc{content} of the response can be overtly spelled out. 


\ea\label{ex:wiltschko:19}
\begin{xlist}
 \exi{A:} John speaks \ili{French} really well {\textbackslash}.
 \exi{B:}
      \begin{xlisti}
       \ex\relax {Yes{\textbackslash}}. {He \{does{\textbackslash}, speaks \ili{French} really well{\textbackslash}}\}.    
       \ex\relax {No{\textbackslash}}.  {He \{doesn’t{\textbackslash}, speak \ili{French} really well{\textbackslash}}\}. 
       \end{xlisti}
 \end{xlist}
\z

\ea\label{ex:wiltschko:20}
\begin{xlist}
\exi{A:}  You stole the cookie {\textbackslash}.
\exi{B:} \begin{xlisti}
          \ex\relax {Yes{\textbackslash}}. \{{I did{\textbackslash},   I stole the cookie{\textbackslash}}\}.
          \ex\relax {No{\textbackslash}}.  \{{I didn’t{\textbackslash}, I didn’t steal the cookie{\textbackslash}}\}.  
         \end{xlisti}
\end{xlist}
\z

This analysis raises the \isi{question} as to what the contribution of the \textit{ResPrt} is in this configuration. That is, if it doesn’t serve to value the \isi{polarity} variable, how does the positive \textit{ResPrt} contribute to agreement and the negative \textit{ResPrt} to disagreement with the \textsc{trigger}? This is a particularly pressing problem with the negative answer \textit{(no}), because there is no negative proposition available to serve as the antecedent for the embedded p-structure.
 
Holmberg’s analysis correctly predicts that the answering function differs from the (dis)agreement function. Empirical support for this difference stems from the fact that other expressions of agreement (\textit{true, right, that’s right)} and disagreement (\textit{false, wrong, that’s wrong}) can be used as responses (\ref{ex:wiltschko:21}--\ref{ex:wiltschko:22}) but unlike \textit{ResPrt}s, they cannot be used as answers, as shown in the examples in (\ref{ex:wiltschko:23}--\ref{ex:wiltschko:24}) (adapted from \citealt[211 (5)]{holmberg:15}).\footnote{The difference between \textit{ResPrt} and other expressions of (dis)agreement has to be explored in more detail. An informal survey suggests that matters are complicated.
While \textit{true/false} can be used in response to assertions, they are less well-formed (though not fully ruled out) in response to rising declaratives (i) or tag questions (ii). 

\begin{exe}
 \ex 
 \begin{xlist}
 \exi{Q:} You fed the dog?
 \exi{A:} Yes./?True./?Correct.
 \end{xlist}
 \ex 
 \begin{xlist}
 \exi{Q:} You fed the dog, didn’t you?
 \exi{A:} Yes./?True./?Correct.
 \end{xlist}
 \ex 
 \begin{xlist}
 \exi{Q:} Did you feed the dog?
 \exi{A:} Yes./*True./*Correct.
 \end{xlist}
\end{exe}

\noindent Before we can develop an analysis that captures these differences, it is necessary to properly establish the empirical facts. I will have to leave this as an avenue for future research however.}\largerpage[2]



\ea\label{ex:wiltschko:21}
\settowidth\jamwidth{\textsc{content:} agreement}
\begin{xlist}
\exi{A:}  {John speaks \ili{French} really well.}   \jambox{\textsc{trigger:} \isi{assertion} (p)}
\exi{B:} 
\begin{xlisti}
\ex\relax {Yes.}            \jambox{\textsc{content:} agreement w/p}
\ex\relax  {True.}          \jambox{\textsc{content:} agreement w/p}
\ex\relax   {Right.}        \jambox{\textsc{content:} agreement w/p}
\ex\relax  {That’s right.}  \jambox{\textsc{content:} agreement w/p}
\end{xlisti}
\end{xlist}
(adapted from \citealt[211 (4)]{holmberg:15})
\z


\ea\label{ex:wiltschko:22}
\settowidth\jamwidth{\textsc{content:} disagreement w\slash p}
\begin{xlist}
\exi{A:} {John speaks \ili{French} really well.} \jambox{\textsc{trigger:} \isi{assertion} (p)}
\exi{B:} 
\begin{xlisti}
\ex\relax {No.}            \jambox{\textsc{content:} disagreement w/p}
\ex\relax  {False.}        \jambox{\textsc{content:} disagreement w/p}
\ex\relax   {Wrong.}       \jambox{\textsc{content:} disagreement w/p}
\ex\relax  {That’s wrong.} \jambox{\textsc{content:} disagreement w/p}
\end{xlisti}
\end{xlist}
\z


\ea\label{ex:wiltschko:23}
\settowidth\jamwidth{\textsc{trigger:} polar \isi{question} (p $\lor ¬$p)}
\begin{xlist}
\exi{A:}  Does John speak \ili{French}?\jambox{\textsc{trigger:} polar \isi{question} (p $\lor ¬$p)}
\exi{B:}
\begin{xlisti}
\ex[]{Yes. \jambox{\textsc{content:} affirming p}}
\ex[*]{True.}
\ex[*]{Right.}
\ex[*]{That’s right.}
\end{xlisti}
\end{xlist}

\z

\ea\label{ex:wiltschko:24}
\settowidth\jamwidth{\textsc{trigger:} polar \isi{question} (p $\lor ¬$p)}
\begin{xlist}
\exi{A:}  Does John speak \ili{French}? \jambox{\textsc{trigger:} polar \isi{question} (p $\lor ¬$p)}
\exi{B:}
\begin{xlisti}
\ex[]{No.\jambox{\textsc{content:} negating p}}
\ex[*]{False.}
\ex[*]{Wrong.}
\ex[*]{That’s wrong.}
\end{xlisti}
\end{xlist}
\z

Thus, \textit{ResPrt}s have a wider distribution than other forms of agreement. This confirms \citegen{Pope1976} insight that English is simultaneously an agreement-based system and a polarity-based system. When the \textsc{trigger} is a polar \isi{question}, \textit{yes} shows up in its \isi{polarity} guise: it values the \isi{polarity} value.
When the \textsc{trigger} is an \isi{assertion} it shows up in its agreement guise.
This still leaves us with the \isi{question} as to what \textit{yes} and \textit{no} contribute when they function as (dis)agreement markers. How is this function derived? 

Suppose that as an agreement marker, \textit{yes} asserts the truth of the preceding proposition while as a disagreement marker, \textit{no} asserts that the preceding proposition is false, thereby establishing agreement or disagreement with the interlocutor, respectively.\linebreak However, this potential analysis cannot be right, given what we find with negative assertions. First consider positive answers. Just as with negative questions, \textit{yes} is ambiguous: it can be used to agree with the negated proposition or else it can be used to assert the truth of the proposition and hence reject the \isi{negation} of the proposition \REF{ex:wiltschko:25}. In this way, \textit{yes} differs from the other predicates of agreement and hence cannot simply be analyzed as a predicate of agreement (like \textit{true} or right). 


\ea\label{ex:wiltschko:25}
\settowidth\jamwidth{\textsc{trigger:} negative declarative $¬$p}
\begin{xlist}
\exi{A:}  {John doesn’t speak \ili{French} well.} \jambox{\textsc{trigger:} negative declarative $¬$p}
\exi{B:}
\begin{xlisti}
\ex\label{ex:wiltschko:25i}\relax  
Yes. \jambox{\textsc{content:} agreement w/$¬$p}
\exi{} \jambox{\textsc{content:} disagreement w/$¬$p}
\ex\relax   True.		     \jambox{\textsc{content:} agreement w/$¬$p}
\ex\relax   Right.		     \jambox{\textsc{content:} agreement w/$¬$p}
\ex\relax   That’s right. 	     \jambox{\textsc{content:} agreement w/$¬$p}
\end{xlisti}
\end{xlist}
\z

Next consider the negative answers. Here \textit{no} – unlike the other predicates of rejection – is ambiguous. It can be used to reject the negated proposition or else it can be used to agree with it. The other predicates of rejection, in contrast, can only be used to disagree with the negated proposition. 


\ea\label{ex:wiltschko:26}
\settowidth\jamwidth{\textsc{trigger:} negative declarative $¬$p}
\begin{xlist}
\exi{A:} {John doesn’t speak \ili{French} well.} \jambox{\textsc{trigger:} negative declarative $¬$p}
\exi{B:} \begin{xlisti}
\ex\label{ex:wiltschko:26i}  No. \jambox{\textsc{content:} disagreement w/$¬$p}
\exi{} \jambox{\textsc{content:} agreement w/$¬$p}
\ex  False.        \jambox{\textsc{content:} disagreement w/$¬$p}
\ex  Wrong.       \jambox{\textsc{content:} disagreement w/$¬$p}
\ex  That’s wrong. \jambox{\textsc{content:} disagreement w/$¬$p}
\end{xlisti}
\end{xlist}
\z

This establishes that the contribution of \textit{ResPrt}s cannot simply be asserting or negating the truth of p. So we are still left with the \isi{question} about the contribution of \textit{ResPrt}s when they function as (dis)agreement markers. Moreover, the data in \REF{ex:wiltschko:25i} and \REF{ex:wiltschko:26i} raise the additional \isi{question} as to how interlocutors determine the contribution of the \textit{ResPrt}s, if both are ambiguous. Of course, this is the signature of a system that is simultaneously an agree-based system and a \isi{polarity} based system. \citet{GoodhueWagner2015}, and \citet{GoodhueEtAl2013} show that the ambiguity of the \textit{ResPrt}s is resolved by means of intonation contours: speakers most frequently use the Contradiction Contour (\citealt{LibermanSag1974}) when reversing, and they use declarative intonation when confirming, regardless of the particular \textit{ResPrt} used.

We have also established that the agreement vs. \isi{polarity} function of \textit{ResPrt}s does not correlate with the difference between binding the \isi{polarity} value of the embedded proposition or not, because both functions are possible with answers to polar questions (where \textit{ResPrt}s bind the \isi{polarity} value) and with responses to assertions (where there is no open \isi{polarity} variable to be bound).

In the remainder of this section, I show that \textit{ResPrts} have an even wider distribution than typically discussed.
That is, they are not restricted to serve as responses to polar questions or assertions. Instead they can be used to respond to all kinds of speech acts – a fact that makes the \isi{question} as to what their contribution is even more pressing.  

\subsubsection{{Responding to wh-questions}}

Wh-questions differ from polar questions in that they require an answer to the open variable denoted by the wh-word in the \isi{question}.\footnote{Thus the meaning of a wh-\isi{question} is not a proposition with a valued \isi{polarity} variable.
According to Hamblin’s (\citeyear{Hamblin1958,Hamblin1973}) influential work, wh-questions denote sets of propositions (as indicated by \{p\textsubscript{1}, p\textsubscript{2}, p\textsubscript{3}…\} in \REF{ex:wiltschko:33}).}  

\ea\label{ex:wiltschko:27}
\begin{xlist}
\exi{A:}  When did you feed the dog?
\exi{B:}
\begin{xlisti}
 \ex[]{\{At around eight{\textbackslash}, After I had breakfast{\textbackslash},…\}}
 \ex[*]{Yes{\textbackslash}!}
 \ex[*]{No{\textbackslash}!} 
 \end{xlisti}
  \end{xlist}
\z

\ea\label{ex:wiltschko:28}
\begin{xlist}
\exi{A:}  Why did you feed the dog?
\exi{B:} 
\begin{xlisti}
\ex[]{\{Because he was hungry{\textbackslash}, Because you told me to{\textbackslash}, …\}}
\ex[*]{Yes{\textbackslash}!}
\ex[*]{No{\textbackslash}!}
     \end{xlisti}
      \end{xlist}
\z

The temporal wh-word in \REF{ex:wiltschko:27} requires the answer to give an indication of the time of feeding whereas the causal wh-word (\textit{why}) in \REF{ex:wiltschko:28} requires the answer to give an indication of the reason for feeding, etc.
Unsurprisingly, in these contexts, simplex \textit{ResPrt}s are ill-formed.

However, there are contexts where \textit{ResPrt}s are possible as a response to a wh-\isi{question}. Consider the examples in \REF{ex:wiltschko:29}{}--\REF{ex:wiltschko:31} from the corpus of American soap operas (SOAP; \url{http://corpus.byu.edu/soap/}).\footnote{SOAP was chosen over other available corpora of spoken language for several reasons.  While soap operas are in part scripted, they are not necessarily scripted in full detail (many discourse markers may not be found in the script; \citealt{Thoma2016}). Moreover, the current exploration is ultimately one of competence.
I assume that both the script writers as well as the actors will create conversations that do not violate their conversational competence.
Finally, according to \citegen{JonesHorak2014} study, the spoken language used in a British Soap Opera (\textit{EastEnders}) is similar to unscripted conversational language in other spoken language corpora. Our quantitative study is based on the episodes aired in 2012 which consists of 2.2 million words.} 


\ea\label{ex:wiltschko:29}
Katie:    {Why would he do something like that?} \\
Brooke:   \textbf{{Yes}}{, I know. That is the \isi{question}.} \\
BB-2012-05-23\footnote{Abbreviations underneath the SOAP examples are as follows: BB (Bold and Beautiful), (DAYS) Days of Our Lives, (GH) General Hospital, YR (Young and Restless). The 8-digit number following the abbreviation represents the release date for the episode from which the example is selected.} 
\z



\ea\label{ex:wiltschko:30}
Brady:     {Why is joining Basic Black so important to me?} \\
Madison:   \textbf{{Yes}}{, please tell me, Brady, because I really want to know.}\\
DAYS-2012-01-06
\z


\ea\label{ex:wiltschko:31}
Avery:   {How did that happen?} \\
Lauren:   (Chuckles) \textbf{{yes}}. \\
Michael:   {It happened because your amazing nephew convinced Daisy to move out of the building.}\\
YR-2012-05-17\\[\baselineskip]

Bill:     {What do you want to bet?} \\
Liam:     \textbf{No}, I am not playing this game with you.\\
BB-2012-03-27\\[\baselineskip]

Sami:     {Rafe, what are you doing here?} \\
Rafe:     \textbf{{No}}{, I'm sorry to drop by so late.}\\
DAYS-2012-02-10
\z

\noindent These responses do not answer the wh-\isi{question} \textsc{triggers} but they are still well-formed.
With the use of the positive \textit{ResPrt}s the responders indicate that they have the same \isi{question} as the respondee.
In other words, the responder indicates agreement with the respondee in their evaluation of the situation as triggering a particular \isi{question}. This is confirmed by the content of the statements following the \textit{ResPrt}s. Note that these statements are more or less obligatory in these contexts. They all suggest that the responder has no real answer to the preceding \isi{question} precisely because s/he has the same \isi{question}. Hence, we can conclude that \textit{ResPrt}s can be used to respond to wh-questions despite the fact that they do not serve as answers. 

The \isi{question} still remains, however, as to what exactly the \textit{ResPrt}s contributes and how. Ideally, an analysis of \textit{ResPrt}s should be able to account for all uses of \textit{ResPrt}s. The ellipsis-based analysis developed in \citet{holmberg:15} cannot straightforwardly account for \textit{ResPrt}s when used to respond to wh-questions. This is because the proposed structure has an embedded p-structure containing a valued \isi{polarity} variable as in \REF{ex:wiltschko:32} repeated from \REF{ex:wiltschko:18} above.
However, if the \textsc{trigger} of the response is a wh-\isi{question}, then the elided structure cannot be a p-structure with a valued \isi{polarity} variable.

\ea\label{ex:wiltschko:32}
\ea
{agreement w/assertion}\\\vspace*{-\baselineskip}
\begin{forest} 
 [XP, s sep=2cm [\textit{yes}] [Pol\is{polarity},name=Pol [Pol\is{polarity}\\\relax\textbf{{[}+ pol{]}},name=pol] [\isi{TP},name=TP [~~~~~~~~~~~~~~~~~~,roof] ] ] ]
 \draw ($ (pol.south west) +(-.5cm,-.25cm) $) .. controls ($ (Pol) +(0,2cm) $) .. ($ (TP) +(1.5cm,-1cm) $) coordinate[pos=0.75,above] (elidedstart);
 \node[right=2em of elidedstart, baseline=elidedstart] (elidedtext) {elided};
 \draw[-{Stealth[]}] (elidedstart) -- (elidedtext);
\end{forest}

\ex
disagreement w/\isi{assertion}\\\vspace*{-\baselineskip}
\begin{forest} 
 [XP, s sep=2cm [\textit{no}] [Pol\is{polarity},name=Pol [Pol\is{polarity}\\\relax\textbf{{[}- pol{]}},name=pol] [\isi{TP},name=TP [~~~~~~~~~~~~~~~~~~,roof] ] ] ]
 \draw ($ (pol.south west) +(-.5cm,-.25cm) $) .. controls ($ (Pol) +(0,2cm) $) .. ($ (TP) +(1.5cm,-1cm) $) coordinate[pos=0.75,above] (elidedstart);
 \node[right=2em of elidedstart, baseline=elidedstart] (elidedtext) {elided};
 \draw[-{Stealth[]}] (elidedstart) -- (elidedtext);
\end{forest}
\vspace*{-\baselineskip}
\z
\z

So the \isi{question} remains as to the contribution of the \textit{ResPrt}s. Descriptively, the contribution of the positive \textit{ResPrt} is to agree that the respondee’s \isi{question} is a valid \isi{question} and the contribution of the negative \textit{ResPrt} is to disagree that the respondee’s \isi{question} is a valid \isi{question}, at least not from the responder’s point of view. This is summarized in \REF{ex:wiltschko:33}. 

\ea\label{ex:wiltschko:33}
\settowidth\jamwidth{\textsc{trigger:} wh-\isi{question} \{p\textsubscript{1}, p\textsubscript{2}, p\textsubscript{3}…\}}
\begin{xlist}
\exi{A:} \relax [{Wh …?}]   \jambox{\textsc{trigger:} wh-\isi{question} \{p\textsubscript{1}, p\textsubscript{2}, p\textsubscript{3}…\}}
\exi{B:}
\begin{xlisti}
\ex Yes …   \jambox{\textsc{content:} agreement with wh-question}
\ex  No …   \jambox{\textsc{content:} disagreement with wh-question}
\end{xlisti}
  \end{xlist}
\z

\noindent But how does this (dis)agreement function come about?  

\subsubsection{{Responding to imperatives}}

We now turn to imperatives, a clause-type that is used to express requests and commands. Unlike questions, imperatives do not explicitly solicit a response in the form of an answer from the addressee.
However, we have already seen that \textit{ResPrt}s are not restricted to answering contexts. They can serve as more general response markers. Hence, we might expect that they can also be used to respond to imperatives. This is indeed the case, as exemplified by the data in \REF{ex:wiltschko:34}{}--\REF{ex:wiltschko:38}, which are all from SOAP. 

\ea\label{ex:wiltschko:34}
Alison:   {So go back to the farmhouse and wait for us}. \\
Deacon:   \textbf{{Yes}}{, Ma'am.} \\
BB-2012-06-20
\z

\ea\label{ex:wiltschko:35}
Steffy:     {Treat me like one of your patients..} \\
Taylor:   \textbf{{Yes}}{, I will.} \\
BB-2012-06-29
\z

\ea\label{ex:wiltschko:36}
Michael:   {Breathe}{!}\\
Starr:     {{Yes}}. \\
GH-2012-03-29
\z

\ea\label{ex:wiltschko:37}
Tracy:     {Give it to me!} \\
Maxie:   \textbf{{No}}{!}\\ 
GH-2012-01-20
\z

\ea\label{ex:wiltschko:38}
Billy:    {Hey, open the door! Let me in!} \\
Chloe:     \textbf{{No}}{, I am not letting you in. Forget about it!}\\
YR-2009-03-16
\z

The well-formedness of these examples indicate that \textit{ResPrt}s can be used to respond to imperatives. In this context they can roughly be paraphrased as \textit{Yes, I will do what you requested of me} vs. \textit{No, I won’t do what you requested of me}. 

Again, existing analyses of \textit{ResPrt}s cannot account for this use.
This is because, like wh-questions, imperatives do not denote propositions, and hence do not make available a proposition to agree with nor a proposition whose \isi{polarity} value has to be valued.
Instead, an imperative is often analyzed as denoting a property that can only be true of the addressee \citep{Portner2004}. So again, the {question} arises as to what the contribution of the \textit{ResPrt} is when it is used to respond to an imperative.
Descriptively, the contribution of the positive \textit{ResPrt} is to agree with the respondee’s evaluation of the situation that a command is in order (and hence the responder indicates that s/he will comply with it). In contrast, the contribution of the negative \textit{ResPrt}s is to disagree with the validity of the command in this situation (and hence the responder indicates that they refuse to comply with it). This is summarized in \REF{ex:wiltschko:33}. 


\ea\label{ex:wiltschko:39}
\settowidth\jamwidth{\textsc{content:} disagreement with command}
\begin{xlist}
\exi{A:}\relax  [Imperative!]   \jambox{\textsc{trigger}\textsc{:} Imperative P}
\exi{B:}
\begin{xlisti}
    \ex Yes …    \jambox{\textsc{content:} agreement with command}
    \ex  No …    \jambox{\textsc{content:} disagreemen with command}
   \end{xlisti}
    \end{xlist}
\z

\subsubsection{{Responding to exclamatives}}

Finally, we consider exclamatives. While some languages have dedicated exclamative clause-types, it is also the case that all kinds of utterances can be interpreted as exclamations, provided they have the right intonation and occur in the right context. What is crucial for our purpose is that responders can respond to commands with a \textit{ResPrt}. This is exemplified by the data in (\ref{ex:wiltschko:40})--(\ref{ex:wiltschko:45}). Note that none of the examples from the corpus are exclamations that are based on the dedicated exclamative clause-type.
Nevertheless they still are instances of exclamations. Furthermore, the constructed example in \REF{ex:wiltschko:42} shows that the use of \textit{ResPrt}s as a response to dedicated exclamative clause-types is also well-formed.

\ea\label{ex:wiltschko:40}
Steffy: {Whoo-hoo.} \\
Liam: \textbf{{Yes}}{!}\\  
BB-2012-05-03
\z

\ea\label{ex:wiltschko:41}
Brooke:   {Steffy is leaving town}. \\
Hope:     {No way!} \\
Brooke:   (Squeals) \textbf{{Yes}}{! I shouldn't say “good” because she is Ridge's daughter, and I really shouldn't celebrate, but I am}. \\
BB-2012-03-19
\z

\ea\label{ex:wiltschko:42}
A: {What a beautiful sunset}.\\
B:  \textbf{{Yes}}{, I know. Isn’t it gorgeous}. 
\z

\ea\label{ex:wiltschko:43}
Anita:    {She found it at Victor's.} \\
Chelsea:   {Oh, my God!} \\
Anita:     \textbf{{No}}{, relax. It's Victor's problem.}\\
{YR-2012-02-17}
\z

\ea\label{ex:wiltschko:44}
Will:     {What a perfect time to lay low.} \\
Gabi:     \textbf{{No}}{, Will, look, I'm trying to find an agent.}\\
{DAYS-2012-05-15}
\z

\ea\label{ex:wiltschko:45}
Michael:   {What a lovely family tradition to hand on to your own} {niece.} \\
Avery:   \textbf{{No,}} {I got to know Daisy through all this.}\\
{YR-2012-02-24}
\z

In this context \textit{ResPrt}s can roughly be paraphrased as follows. The positive \textit{ResPrt} indicates that the responder agrees with the evaluation of the situation by the respondee (\ref{ex:wiltschko:46}i); the negative \textit{ResPrt} indicates that the responder does not agree with the evaluation of the situation by the respondee (\ref{ex:wiltschko:46}ii).

\ea\label{ex:wiltschko:46}
\begin{xlist}
\exi{A:}\relax  [Exclamative{!}]    \jambox{\textsc{trigger:} Exclamative \{p\textsubscript{1}, p\textsubscript{2}, p\textsubscript{3},…\}}
\exi{B:}
\begin{xlisti}
\ex\label{ex:wiltschko:46i} Yes …   \jambox{\textsc{content:} agreement w/exclamation}
\ex\label{ex:wiltschko:46ii} No …   \jambox{\textsc{content:} disagreement w/exclamation}
\end{xlisti}
\end{xlist}
\z

Again, existing analyses of \textit{ResPrt}s cannot account for this use.
This is because, like wh-questions and imperatives, exclamatives do not denote propositions, and hence do not make available a proposition to agree with nor a proposition whose \isi{polarity} value has to be valued.
Instead, as indicated in \REF{ex:wiltschko:46}, an exclamative can be analyzed as denoting a set of alternative propositions \citep{ZanuttiniPortner2003}. So again, the \isi{question} arises as to how the (dis)agreement function of \textit{ResPrt} is derived when they are used to respond to an exclamative.


\subsection{The analytical challenge} 
We have now explored \textit{ResPrt} as responses to all major clause-types and we have seen that they are not only used as answers to polar questions. \textsc{I}n fact, a survey of 1013 tokens of positive \textit{ResPrt} in SOAP reveals that the vast majority of instances of \textit{yes} is used to respond to preceding assertions (\textit{n} = 654), followed by responses to yes/no questions (\textit{n} = 279). The other functions of \textit{yes} are much less frequent, but nevertheless occur: response to exclamatives (\textit{n} = 44); response to imperatives (\textit{n} = 36); and response to wh-questions (\textit{n} = 9). This is summarized in \figref{fig:wiltschko:yes}.\footnote{In this study, we looked at 1469 tokens of \textit{yes} and 3093 tokens of \textit{no.}  Not all tokens are included in the quantitative analysis above.
In particular, not included in the chart above are those tokens that respond to tag questions and rising declaratives, as well as echo-questions, addresses, and backchannels.}  

\begin{figure}
\caption{Distribution of \textit{yes} across different \textsc{triggers}}
\label{fig:wiltschko:yes}
  \barplot{clause type}{instances}{YNQ,assertion,imperative,exclamative,wh-question}{
	      (YNQ,279) 
	      (assertion,654) 
	      (imperative,36)
	      (exclamative,44)
	      (wh-question,9)
  } 
  \label{fig:barplot}
\end{figure}

As shown in \figref{fig:wiltschko:no}, the numbers are similar for \textit{no}. The vast majority is used to respond to preceding assertions (\textit{n} = 1387), followed by responses to yes/no questions (\textit{n} = 711). The other functions of \textit{no} are again much less frequent, but nevertheless occur: response to exclamatives (\textit{n} = 16), response to imperative (\textit{n} = 172), and response to wh-questions (\textit{n}=58).


\begin{figure}
\caption{Distribution of \textit{no} across different \textsc{triggers}}
\label{fig:wiltschko:no}
  \barplot{clause type}{instances}{YNQ,assertion,imperative,exclamative,wh-question}{
	      (YNQ,711) 
	      (assertion,1387) 
	      (imperative,172)
	      (exclamative,16)
	      (wh-question,58)
  } 
\end{figure}

We have established above that the function of the \textit{ResPrt} differs depending on the clause type of the \textsc{trigger}, as summarized in \tabref{tab:wil:ResPRTs}.

\begin{table}
\begin{tabularx}{\textwidth}{lQcQcQ}
\lsptoprule
\multicolumn{2}{l}{\textsc{Trigger} of response} & \textit{yes} & \textsc{function} & \textit{no} & \textsc{function}\\
\midrule
Polar \isi{question} & positive polar questions & ✔ & Answer: affirmative & ✔ & Answer: negative\\
& negative polar questions & ✔ & Answer:

i) polarity-based

ii) agreement-based & ✔ & Answer

i) polarity-based

ii) agreement-based\\
\tablevspace
\multicolumn{2}{l}{Declarative} & ✔ & agreement w/\isi{assertion} & ✔ & disagreement w/\isi{assertion}\\
\tablevspace
\multicolumn{2}{l}{Interrogative} & ✔ & agreement w/\isi{question} & ✔ & disagreement w/\isi{question}\\
\tablevspace
\multicolumn{2}{l}{Imperative} & ✔ & agreement w/command & ✔ & disagreement w/command\\
\tablevspace
\multicolumn{2}{l}{Exclamative} & ✔ & agreement w/exclamation & ✔ & disagreement w/exclamation\\
\lspbottomrule
\end{tabularx}
\caption{Distribution and function of ResPRTs\label{tab:wil:ResPRTs}}

\end{table}

Note that \textit{ResPrts} function as answers only if they serve to answer polar questions. In all other contexts they serve to express agreement or disagreement with the speech act of their \textsc{trigger.} At first sight, the fact that \textit{ResPrts} can be used to express agreement might not be surprising within a language that makes use of an agree/disagree-based system (referred to as \textit{truth-}based in \citealt{holmberg:15}). But unfortunately, this is not sufficient to understand this pattern. First, we do not have a good understanding of what the contribution of \textit{yes} and \textit{no} is when they are used to mark agreement and disagreement. \textsc{W}e have seen throughout the discussion that it is not immediately clear how to extend Holmberg’s analysis to cover the full range of functions of \textit{ResPrt.} Second, if the multi-functional profile of the type identified for English \textit{ResPrt} is dependent on English having a \isi{polarity} based system AND an agree/disagree based system in the sense of \citet{Kuno1973}, then we would expect that languages where answers are polarity-based will have a different profile, and that \textit{ResPrts} could not be used as (dis)agreement markers following \textsc{triggers} other than polar questions. However, this prediction is not borne out, as I now show.

According to \citet[Section~4.2]{holmberg:15}, \ili{German} does not have an agree/disagree-based system. Nevertheless, \ili{German} \textit{ResPrts} can be used with all of the \textsc{triggers} discussed for English \textit{ResPrts} and with the same functions. This is shown below with examples from the Upper Austrian variety of \ili{German} (henceforth UAG).\footnote{There are several reasons to use dialectal data for this discussion. First, conversations like those reported on here are a spoken language phenomenon and the status of Standard \ili{German} as a spoken language is questionable (\citealt{Weiß2004,Auer2004}). In addition, in ongoing work on the form and function of response particles we find a staggering range of variation even among dialects of the same language (i.e., \ili{German}).}  

Relative to the parameters explored in \citet{holmberg:15}, UAG \textit{ResPrts} have the following profile.
The first thing to note is that \textit{ResPrt} exist in this language.
That is, in answering a polar \isi{question}, UAG employs dedicated particles \textit{jo} (‘yes’) and \textit{na} (‘no’). As shown in \REF{ex:wiltschko:47}, both can be used in isolation or be followed by the \textsc{content} of the response (i.e., the proposition introduced in the \textsc{trigger} of the response). 

\ea\label{ex:wiltschko:47}
\settowidth\jamwidth{\textsc{trigger:} p $\lor ¬$p}
\ili{Upper Austrian German}\\
\begin{xlist}
\exi{Q:} 
  \gll Host  du   an   Hund  gfuattat?    \\
  Have you \textsc{det} dog     fed\\\jambox{\textsc{trigger:} p $\lor ¬$p}
  \glt ‘Did you feed the dog?’

\exi{A:}
\begin{xlista}
 \ex
  \gll Jo.   (I hob   an    Hund gfuattat.)  \\
	Yes. I  have \textsc{det}  dog    fed\\\jambox{\textsc{content:} p}

  \ex
  \gll Na. (I hob   an   Hund  net   gfuattat.)  \\
    No. I  have \textsc{det}  dog    \textsc{neg} fed\\\jambox{\textsc{content:} $¬$p}
  \end{xlista}
\end{xlist}
\z

Moreover, according to the criteria \citet{holmberg:15} adopts, UAG has a polarity-based system: negative questions cannot be answered with the positive \textit{ResPrt} (\ref{ex:wiltschko:48}i). Like Standard \ili{German}, UAG has a dedicated \isi{polarity} reversing strategy: the positive \textit{ResPrt} is prefixed with \textit{oh} (\ref{ex:wiltschko:48}iii). With this strategy the \textsc{content} expressed by the response is p by virtue of reversing the \isi{negation} of p ($¬$($¬$p)).


\ea\label{ex:wiltschko:48}
\ili{Upper Austrian German}\\
\begin{xlist}
\exi{Q:} 
  \gll Trinkt  da    Hons  net   an      Kaffee?   \\
  drinks \textsc{det}  Hans   \textsc{neg}  \textsc{det}   coffee \\\jambox{\textsc{trigger:} $¬$p $\lor ¬$($¬$p)}
  \glt ‘Does Hans not drink coffee?’

\exi{A:}
\begin{xlisti}
\ex[*]{Jo.  (=He does drink coffee.)      \jambox{*\textsc{content:} p}}
\ex[]{Na.   (= He doesn’t drink coffee.)  \jambox{\textsc{content:} $¬$p}}
\ex[]{Oh jo.   (= He does drink coffee.)  \jambox{\textsc{content:} p =($¬$($¬$p))}}
\end{xlisti}
\end{xlist}
\z

Now, if the possibility for \textit{ResPrt} to be used as (dis)agreement markers were contingent on the answering system of the language being an agree/disagree based system, then we would predict that \textit{ResPrt} in UAG cannot be used in this way. However, this prediction is not borne out. The same \textit{ResPrts} that can be used as answers to polar questions can also be used to respond to assertions \REF{ex:wiltschko:49}, wh-questions \REF{ex:wiltschko:50}, commands \REF{ex:wiltschko:51}, and exclamations \REF{ex:wiltschko:52}. 


\ea\label{ex:wiltschko:49}\settowidth\jamwidth{\textsc{trigger:} \isi{assertion} (p)}
\ili{Upper Austrian German}\\
\begin{xlist}
\exi{A:} 
      \gll Da  Hons  red-t           guat  Französisch {\textbackslash}.\\
      \textsc{det} Hons speak-\textsc{\oldstylenums{3}sg}  well  \ili{French}\\\jambox{\textsc{trigger:} \isi{assertion} (p)}
      \glt ‘Hans speaks \ili{French} well.’
\exi{B:}
\begin{xlisti}
    \ex \textit{Jo{\textbackslash}}. (= p)    \jambox{\textsc{content:} affirming p}
    \ex \textit{Na{\textbackslash}}.  (=$¬$p) \jambox{\textsc{content:} negating p}
\end{xlisti}
\end{xlist}
\z


\ea\label{ex:wiltschko:50}\settowidth\jamwidth{\textsc{trigger:} wh-question}
\ili{Upper Austrian German}\\
\begin{xlist}
\exi{A:}
\gll Wonn foast         denn du jetzt eigentlich?   \\
When leave-\oldstylenums{2}\textsc{sg} \textsc{prt} you now \textsc{prt}\\\jambox{\textsc{trigger:} wh-question}
\glt ‘When are you finally leaving?’
\exi{B:}
\begin{xlisti}
    \ex 
	\gll  \textbf{{Jo,}}   des   is a         guate   frog.\\
	Yes, \textsc{dem} is \textsc{indf} good   \isi{question}\\\jambox{\textsc{content:} agree w/Q}
	\glt ‘Yes, that’s a good \isi{question}.’ 
   \ex
	  \gll \textbf{Na}, des  deafst      me ned frogn.\\
	  No, \textsc{dem} may-\oldstylenums{2}\textsc{sg} me \textsc{not} ask\\\jambox{\textsc{content:}disagree w/Q}
	  \glt ‘No. You can’t ask me that.’
\end{xlisti}
\end{xlist}
\z

\ea\label{ex:wiltschko:51}\settowidth\jamwidth{\textsc{trigger:} command}
\ili{Upper Austrian German}\\
\begin{xlist}
\exi{A:}
\gll Jetzt geh endlich ins Bett.\\
Now go finally into.the bed\\\jambox{\textsc{trigger:} command}
\glt ‘Go to bed now!’
\exi{B:}
\begin{xlisti}
\ex	\gll \textbf{{Jo}}  i geh jo    eh    scho. \\
	yes I go   \textsc{prt} \textsc{prt} \textsc{prt}\\\jambox{\textsc{content:} agree w/command}
	\glt ‘but I’m going already.’  
\ex	\gll \textbf{{Na}} wirkli ned. \\
	  No really not \\\jambox{\textsc{content:} disagree w/command}
      \glt ‘No way.’
\end{xlisti}
\end{xlist}
\z

\ea\label{ex:wiltschko:52}\settowidth\jamwidth{\textsc{trigger:} exclamation}
\ili{Upper Austrian German}\\
\begin{xlist}
\exi{A:} 
\gll Ma   is des a liaba Hund. \\
 \textsc{Prt}  is  \textsc{dem} \textsc{indf} cute dog\\\jambox{\textsc{trigger:} exclamation}
\glt ‘What a cute dog that is!’
\exi{B:}
\begin{xlisti}
\ex 	\gll \textbf{{Jo}} wirkli woa, geu? \\
	    yes really true, \textsc{tag}?\\\jambox{\textsc{content:} agree w/excl.}
	    \glt ‘Yeah, that’s true, isn’t it?’
 
\ex      \gll \textbf{{Na}} owa wirkli ned. \\
      no  \textsc{prt}   really \textsc{neg}\\\jambox{\textsc{content:} disagree w/excl.}
      \glt ‘No, that’s really not true.’
 
\end{xlisti}
\end{xlist}
\z


This establishes that the possibility for using \textit{ResPrt} as responses to speech acts other than assertions is not contingent on the answering system being an agree/disagree based one.
And, as indicated in the above examples, the general function of \textit{ResPrt} in contexts where the \textsc{trigger} is not an \isi{assertion} is still agreement or disagreement with the \textsc{triggering} speech act. Thus we can conclude that the ability of \textit{ResPrts} to express agreement or disagreement is not restricted to agree/disagree based answer systems. 

But this still leaves us with the \isi{question} as to how to analyse the (dis)agreement function of \textit{ResPrt}. 

\section{The syntax of (dis-)agreement} \label{sec:wiltschko:4}

To understand the difference between the answering function and the (dis)agreement function of \textit{ResPrt} it is useful to compare their contribution with two \textsc{triggers:} polar questions vs. wh-questions. With polar questions, the \textit{ResPrts} are used to affirm or negate the \textit{proposition} \REF{ex:wiltschko:53} embedded in the \isi{question} while with wh-questions, they are used to agree with or reject the \textit{question} \REF{ex:wiltschko:54}. 

\ea\label{ex:wiltschko:53}\settowidth\jamwidth{\textsc{trigger:} polar \isi{question} (p $\lor ¬$p)}
\begin{xlist}
\exi{A:}\relax [{y/n...?}]    \jambox{\textsc{trigger:} polar \isi{question} (p $\lor ¬$p)}
\exi{B:}
\begin{xlisti}
\ex Yes …     \jambox{\textsc{content:} affirming p (= p)}
\ex No …      \jambox{\textsc{content:} negating p (= $¬$p)}
\end{xlisti}
\end{xlist}
\z

\ea\label{ex:wiltschko:54}\settowidth\jamwidth{\textsc{trigger:} wh-\isi{question} \{p\textsubscript{1}, p\textsubscript{2}, p\textsubscript{3}…\}}
\begin{xlist}
\exi{A:}\relax [{Wh …..?}]    \jambox{\textsc{trigger:} wh-\isi{question} \{p\textsubscript{1}, p\textsubscript{2}, p\textsubscript{3}…\}}
\exi{B:}
\begin{xlisti}
\ex Yes …   \jambox{\textsc{content:} agreement w/wh-question}
\ex No …    \jambox{\textsc{content:} disagreement w/wh-question}
\end{xlisti}
\end{xlist}
\z

Thus when responding to a wh-\isi{question}, the \textsc{content} of the response is the same as the \textsc{trigger}, namely the speech act of questioning itself. 
This contrasts with the answering function of \textit{ResPrt}s in response to polar questions. Here the \textsc{content} of the response is the proposition embedded in the polar \isi{question}, and not the polar \isi{question} itself.
To account for this difference, let us begin by assuming that the analysis for \textit{ResPrts} in their answering function is essentially as in \citet{holmberg:15}: the \textit{ResPrt} values the \isi{polarity} value associated with the p-structure, as in \REF{ex:wiltschko:55}, repeated from \REF{ex:wiltschko:5} above.

\noindent\parbox{\textwidth}{\ea\label{ex:wiltschko:55}
\textit{ResPrts} bind the \isi{polarity} variable
%%\includegraphics[width=\textwidth]{a13Wiltschko-img20}
\begin{multicols}{2}
\begin{xlista}
  \ex the contribution of \textit{yes} \\
  \begin{forest} nice empty nodes
   [FocP [\textit{\bfseries yes},name=yes] [Foc\is{Focus} [Foc\is{Focus}] [PolP [] [Pol\is{polarity} [Pol\is{polarity}\\\textbf{{[}+pol{]}},name=pol, align=center,base=top] [TP [~~~~~~~~~~~~~~,roof]]]]]]
   \path[-{Stealth[]}] (yes) edge [bend right] (pol);
  \end{forest}

  \ex
  the contribution of \textit{no}\\
  \begin{forest} nice empty nodes
   [FocP [\textit{\bfseries no},name=yes] [Foc\is{Focus} [Foc\is{Focus}] [PolP [] [Pol\is{polarity} [Pol\is{polarity}\\\textbf{{[}-pol{]}},name=pol, align=center,base=top] [TP [~~~~~~~~~~~~~~,roof]]]]]]
   \path[-{Stealth[]}] (yes) edge [bend right] (pol);
  \end{forest}
\end{xlista}
\end{multicols}
\z}

However, as we have seen, \textit{ResPrts} are not restricted to indicating the \isi{polarity} value of a proposition. Hence this cannot be their intrinsic content. In fact, the association with \isi{polarity} is, on this analysis, syntactically conditioned.
\textit{ResPrts} value an open \isi{polarity} variable, but they do not themselves establish \isi{polarity} \textit{per se}. 

So suppose that the core content of the \textit{ResPrts} is to value an unvalued clausal feature as either positive (\textit{yes}) or negative (\textit{no}). Positive and negative values are themselves not restricted to propositional \isi{polarity}. Instead, all types of features have been assumed to be bi-valent such that one value is positive and the other negative \citep{Jakobson1932,Trubetzkoy1939}. I propose that -- when used to establish (dis)agreement – the contribution of \textit{ResPrts} is to value an unvalued feature in the speech act structure.
In particular, following \citet{WiltschkoInPress,WiltschkoHeimInPress,Thoma2016}, I assume that speech act structure contains a \textit{grounding} layer, which is responsible for encoding the commitment of S towards p. The label \textit{Ground}P is meant to evoke \citegen{ClarkBrennan1991} mechanism of \textit{grounding} as well as the notion of the common ground (cf.
\citet{HeimEtAl2014,Thoma2016} and \citet{WiltschkoHeimInPress} for discussion). In particular, GroundP takes the CP\is{complementizer} (typed p-structure) as a complement and an abstract argument referring to the S’s ground (Ground-S) in its specifier as in \REF{ex:wiltschko:56}.\footnote{For evidence that the speaker’s ground (Ground-S) and the addressee’s ground (Ground-A) are associated with two distinct layers in the structure, see \citet{Lam2014,HeimEtAl2014,Thoma2016}. Since Ground-A plays no role in the analysis of the \textit{ResPrts} discussed here, I will not discuss it here.}  


\noindent\parbox{\textwidth}{\ea\label{ex:wiltschko:56}
Speech act structure\\
\begin{forest}
 [GroundP [\textit{\bfseries Ground-S}] [\isi{Ground} [\isi{Ground}\\\textbf{{[}\textit{u}coin{]}},base=top,align=center] [CP\is{complementizer} [~~~~~~~~~~~~~~,roof]]]]
\end{forest}
\z}

This structure follows the basic template for functional categories assumed in \citet{Wiltschko2014}: they are transitive heads which establish a relation between their complement and an abstract argument in their specifier. The relation is established via the unvalued coincidence feature [\textit{u}coin] which is universally associated with all clausal heads. This feature establishes whether or not the two arguments coincide and is independent of the dimension relative to which they coincide.
That is, coincidence may be in time, place, participancy or belief states, among other things. That coincidence is a central \isi{universal} characteristic of a variety of grammatical categories was first observed in \citet{Hale1986} (see \citealt{Wiltschko2014} for detailed discussion).

On this analysis then, the contribution of \textit{ResPrts} is to value the unvalued coincidence feature associated with \isi{Ground}.
So when the \textsc{trigger} is a wh-\isi{question}, the structure associated with the \textit{ResPrt} is as in \REF{ex:wiltschko:57}. The \textit{ResPrt} attaches to GroundP, which in turn takes a CP\is{complementizer} as its complement. This CP\is{complementizer} corresponds to the \textsc{trigger} and is typically elided but can also be spelled out, as shown in \REF{ex:wiltschko:58}.\footnote{In \REF{ex:wiltschko:57} the \textit{ResPrt} is represented as attaching to GroundP in the same fashion as \textit{ResPrts} are analysed in \citet{holmberg:15}. It may be the case, however, that \textit{ResPrts} are better analysed as heads associating directly with the \isi{Ground} head.
For the purpose of this discussion, the \isi{question} whether \textit{ResPrts} function as heads or phrases can be put aside.}   
\enlargethispage{2\baselineskip}

\ea\label{ex:wiltschko:57}
\textit{ResPrt} values [\textit{u}coin] in \isi{Ground}\\
\begin{xlista}
\ex the contribution of \textit{yes}\\\vspace*{-.33\baselineskip}
\begin{forest}
 [ [\textbf{\textit{yes}},name=yes] [GroundP [\textbf{\textit{Ground-S}}] [\isi{Ground} [\isi{Ground}\\\textbf{{[}+coin{]}},base=top,align=center,name=ground] [CP-interrog [~~~~~~~~~~~~~~,roof] ]]]]
 \path[-{Stealth[]}] (yes) edge [bend right=90] (ground);
\end{forest}\vspace*{-\baselineskip}
\ex the contribution of \textit{no}\\
\begin{forest}
 [ [\textbf{\textit{no}},name=yes] [GroundP [\textbf{\textit{Ground-S}}] [\isi{Ground} [\isi{Ground}\\\textbf{{[}-coin{]}},base=top,align=center,name=ground] [CP-interrog [~~~~~~~~~~~~~~,roof] ]]]]
\end{forest}
\end{xlista}
\z

\ea\label{ex:wiltschko:58}
\begin{xlist}
\exi{A:} When are you leaving? 
\exi{B:} 
\begin{xlisti}
 \ex Yes. (When am I leaving?) That’s the \isi{question}.
 \ex No! (When am I leaving?) You can’t ask me that.
\end{xlisti}
  \end{xlist}
\z

\noindent According to this analysis, \textit{yes} values [\textit{u}coin] associated with \isi{Ground} as [+coin], thereby asserting that the wh-\isi{question} is in the speaker’s ground; in contrast, \textit{no} values [\textit{u}coin] as [-coin] thereby asserting that the \isi{question} which serves as the \textsc{trigger} is not in the speaker’s ground.
The assumption that questions can be part of someone’s ground (in addition to propositions and discourse referents) has been independently established in \citet{Ginzburg1995I,Ginzburg1995II} and \citet{Roberts1996}. They argue that the discourse component associated with wh-questions is a \isi{Question}\is{question} Set (a set of propositions). Evidence that this is so comes from the fact that a \isi{question} may serve as a discourse referent, just like propositions do. Hence they can be anaphorically referenced, as in \REF{ex:wiltschko:58} by \textit{that}. 

According to this analysis, the multi-functionality of \textit{ResPrts} derives from the fact that they can associate with the clausal spine in two different positions: 
i) immediately above p-structure and 
ii) above the speech act structure.
In the former case, which is the one that Holmberg discusses, \textit{ResPrts} serve to value an open \isi{polarity} variable associated with the proposition. This derives their answering function because they provide the value for the open variable.
Since by hypothesis, there is no \isi{polarity} variable associated with wh-questions, this function is not available if the \textsc{trigger} is a wh-\isi{question}. The felicity of \textit{ResPrts} in this context derives from the fact that the \textit{ResPrts} can also associate with the spine above the speech-act structure.
In this context, they serve to value the unvalued coincidence feature.
This derives the (dis)agreement function of \textit{ResPrts}. In particular, if the responder asserts that the \textsc{trigger} \isi{question} is in their ground it follows that they agree with the responder.
By virtue of asking the \isi{question} in the first place, the respondee makes it clear that this \isi{question} is in their ground.
If the same \isi{question} is also in the respondee’s ground, it follows that they agree on the felicity of the speech act. In this way, the proposed analysis can derive the fact that \textit{ResPrts} can be used to respond to all clause-types. As just discussed, with assertions, the discourse component is a proposition; with wh-questions the discourse component is a \isi{Question}\is{question} Set. And following \citet{Portner2004}, we can assume that with imperatives, the discourse component is a \textit{to do list}. Finally, for expository reasons I assume that with exclamatives, the discourse component is list of \textit{exclaimables.} 

Hence the agreement function is communicated without a dedicated agreement mark\-er. The essence of this analysis is summarized in \REF{ex:wiltschko:59}. 

\ea\label{ex:wiltschko:59}
The agreement vs. answering function of \textit{ResPrt}\\\vspace*{\baselineskip}
\begin{tikzpicture}[baseline]
\node at (0,0) (GroundP) {GroundP};
\node[below left=\baselineskip and 1em of GroundP,baseline=GroundP] (yes) {\textit{\bfseries\strut yes}}; \node[below=\baselineskip of GroundP.south east,baseline=GroundP] (CP) {\strut CP};
\draw [black!50] let \p1=($ (yes.north east) !.5! (CP.north west) $), \p2=(GroundP.south) in  (\x1,\y2) -- (yes.north east) -- (CP.north west) -- cycle;
\draw [black!50] (CP.south) -- +(-.5cm,-.5cm) -- +(.5cm,-.5cm) -- cycle;
\draw [-{Triangle[open]}] (yes) -- (CP);
\draw [-{Triangle[open]},black] (yes.north) -- (GroundP.190);
\node [right=2em of GroundP,draw,baseline=GroundP] (agreement) {\bfseries\itshape\strut agreement};
\node [below=2\baselineskip of agreement.base west,anchor=base west,draw,baseline=GroundP] (answer) {\bfseries\itshape\strut affirmative answer};
\draw [-{Triangle[open]}] (GroundP) -- (agreement);
\draw [-{Triangle[open]}] (CP) -- (answer);
\end{tikzpicture}\vspace{\baselineskip}
\begin{tikzpicture}[baseline]
\node at (0,0) (GroundP) {GroundP};
\node[below left=\baselineskip and 1em of GroundP,baseline=GroundP] (yes) {\textit{\bfseries\strut no}}; \node[below=\baselineskip of GroundP.south east,baseline=GroundP] (CP) {\strut CP};
\draw [black!50] let \p1=($ (yes.north east) !.5! (CP.north west) $), \p2=(GroundP.south) in  (\x1,\y2) -- (yes.north east) -- (CP.north west) -- cycle;
\draw [black!50] (CP.south) -- +(-.5cm,-.5cm) -- +(.5cm,-.5cm) -- cycle;
\draw [-{Triangle[open]}] (yes) -- (CP);
\draw [-{Triangle[open]},black] (yes.north) -- (GroundP.190);
\node [right=2em of GroundP,draw,baseline=GroundP] (agreement) {\bfseries\itshape\strut rejection};
\node [below=2\baselineskip of agreement.base west,anchor=base west,draw,baseline=GroundP] (answer) {\bfseries\itshape\strut negative answer};
\draw [-{Triangle[open]}] (GroundP) -- (agreement);
\draw [-{Triangle[open]}] (CP) -- (answer);
\end{tikzpicture} 
\z

In the remainder of this section, I discuss two predictions of this analysis. First, I show that the (dis)agreement function does not interact with \isi{negation}. This follows, because the agreement function arises by associating \textit{ResPrt} with GroundP, and hence is too high to interact with \isi{negation} within the propositional structure.
Second, I show that the (dis)agreement function is also available with polar questions. 

Evidence that the agreement function derives from a high position of the \textit{ResPrts} comes from the fact that in this position they do not interact with \isi{negation} in the same way as they do when they serve the answering function. Recall that in English, answers to negative questions are ambiguous between the \isi{polarity} and the truth-based reading because the \textit{ResPrts} may or may not take \isi{negation} in their scope.
The relevant data exemplifying this pattern are repeated below for convenience.



\ea\label{ex:wiltschko:60}\settowidth\jamwidth{{\textsc{content:} $¬$p}}
\begin{xlist}
 \exi{Q:} Doesn’t John drink coffee?
 \exi{A:}
 \begin{xlisti}
  \ex Yes. (=He does drink coffee.)\jambox{\textsc{content:} p}
  \exi{} \hspaceThis{Yes.} (= He doesn’t drink coffee.)\jambox{\textsc{content:} $¬$p}
  \ex No. (=He doesn’t drink coffee.)\jambox{\textsc{content:} $¬$p}
  \exi{} \hspaceThis{No.} (= He does drink coffee.)\jambox{\textsc{content:} p}
 \end{xlisti}\end{xlist}
\z

If \textit{ResPrts} in their (dis)agreement function associate with the spine above the speech act phrase we predict that they cannot interact with \isi{negation} in the same way. This prediction is borne out. When a wh-\isi{question} contains \isi{negation}, the positive \textit{ResPrt} agrees with the negated \isi{question} (\ref{ex:wiltschko:61}i/ii) while the negative \textit{ResPrt} has to disagree with the negated \isi{question} (\ref{ex:wiltschko:61}iii/iv). Hence no ambiguity arises with \textit{ResPrts} in this context and negated wh-questions behave just like their positive counterparts. 


\ea\label{ex:wiltschko:61}
\begin{xlist}
 \exi{A:} Why wouldn’t he do something like that?
 \exi{B:}
 \begin{xlisti}
  \ex[] {\textbf{Yes}. That is the \isi{question}.}
  \ex[*]{\textbf{Yes}. That is not the \isi{question}.}
  \ex[*]{\textbf{No}. That’s the \isi{question}.}
  \ex[] {\textbf{No}. That’s not the \isi{question}.} 
 \end{xlisti}
 \end{xlist}
\z
    
Next we turn to another \isi{question} that the analysis raises: Why does the function of the \textit{ResPrt} correlate with the speech act of the \textsc{trigger}? That is, up until now we have seen that as responses to polar questions \textit{ResPrts} function as answers while as responses to wh-questions as well as other speech acts, they function as (dis)agreement markers. Everything else being equal, we might expect that \textit{ResPrts} could be associated with the answering function and the agreement function with any speech act. However, everything else is not equal. First, answering requires there to be an open variable inside the p-structure of the \textsc{trigger.} This is the case in polar questions, but not in other speech act types such as assertions, content questions, commands, and exclamations. However, there are other ways to ask questions: rising declaratives and tag questions. And indeed, \textit{ResPrts} can serve the answering function when these questions are the \textsc{triggers} for the response.


\ea\label{ex:wiltschko:62}
\begin{xlist}
\exi{A:}  You fed the dog/?
\exi{B:}
\begin{xlisti}
 \ex \textbf{Yes}. I fed the dog.
 \ex \textbf{No}. I didn’t feed the dog.
\end{xlisti}
\end{xlist}
\z


\ea\label{ex:wiltschko:63}
\begin{xlist}
  \exi{A:}  You fed the dog, didn’t you? 
  \exi{B:} 
   \begin{xlisti}
    \ex   \textbf{Yes}. I fed the dog.
    \ex   \textbf{No}. I didn’t feed the dog.
\end{xlisti}    
\end{xlist}
\z

But what about the (dis)agreement function? The analysis predicts that the agreement function should also be available when the \textsc{trigger} is a polar \isi{question}. This prediction is indeed borne out. \textit{ResPrts} can be used to (dis)agree with polar questions as well. That is, they can serve not only to answer the polar \isi{question} but also to agree with or disagree with its felicity. Note however, that this use of the \textit{ResPrt} is much more marked.
It seems to improve with an initial \textit{hmmm}, which, I assume, marks the responder’s evaluation of the \isi{question}. 


\ea\label{ex:wiltschko:64}
\begin{xlist}
 \exi{A:}  Did you feed the dog?
 \exi{B:}
    \begin{xlisti}
    \ex (Hmm) Yes. Did I feed the dog? That’s a good \isi{question}.
    \ex (Hmm) No.  Did I feed the dog? That’s an unfair \isi{question}.
    \end{xlisti}
\end{xlist}
\z

\noindent In sum, I have argued that the two different functions of \textit{ResPrts} we have identified are syntactically conditioned.
The answering function arises if the \textit{ResPrt} associates just above p-structure and values the open \isi{polarity} value; the (dis)agreement function arises if the \textit{ResPrt} associates above the speech-act structure and values [\textit{u}coin] to assert whether or not the embedded speech act is in the responder’s ground.

Note that simple \textit{ResPrts} cannot be felicitously used with all types of \isi{assertion} \textsc{triggers.} Specifically, agreement is only possible if the \textsc{content} of the response is already in the responder’s ground at the time of the exchange. However, if the respondee reports on something that is new to the responder (as indicated by the initial phrase \textit{guess what}), then a simple \textit{ResPrt} is infelicitous; rather, the \textit{ResPrt} has to be modified.
In such cases, as shown in \REF{ex:wiltschko:65}, in English the positive \textit{ResPrt} is preceded by \textit{oh}, which marks the newness of the \textsc{trigger} while at the same time, \textit{yes} indicates that there are no contradictory beliefs in the responder’s ground.
Hence, this modified \textit{ResPrt} serves to indicate acceptance.
Note also that there is a rising intonation on \textit{yes}, which indicates that the responder is requesting confirmation that this proposition is really true.
Thus, with the rising intonation the responder indicates that s/he accepts the interlocutor as the authority over the truth of the proposition. As shown in \REF{ex:wiltschko:66}, UAG has a dedicated particle that serves the acceptance function: it simultaneously indicates the newness of the proposition in the responder’s ground and its acceptance.
Like in English, this particle is realized with a rising intonation. 


\ea\label{ex:wiltschko:65}
\begin{xlist}
\exi{A:}  Guess what. {My sister just gave birth to a baby}{\textbackslash}.
\exi{B:} \begin{xlisti}
          \ex[*]{Yes{\textbackslash}.} 
          \ex[]{Oh, yes/?}
 \end{xlisti}
\end{xlist}
\z

\ea\label{ex:wiltschko:66}
\ili{Upper Austrian German}\\
\begin{xlist}
 \exi{A:}
 \gll Stoe  da  voa.  Mei Schwesta  hot   grod  a      Kind  kriagt{\textbackslash}. \\
     Put  \textsc{\oldstylenums{2}sg}  \textsc{prt}. my  sister  \textsc{aux} just  \textsc{det} child  got.\\
 \glt ‘Imagine that. My sister just had a baby.’
\exi{B:}
\begin{xlisti}
  \ex[*]{Jo\textbackslash.}
  \ex[]{Aso/?\footnotemark} 
\end{xlisti}
\end{xlist}
\z\footnotetext{The standard \ili{German} version of this particle is \textit{ach so.}}


This much establishes that languages can have special means to mark the status of a particular proposition relative to the responder’s ground: in English and in \ili{German}, special markers are available to mark the newness of the proposition in the common ground.
This is akin to the marking of the novelty or familiarity of a given discourse referent (i.e., definiteness marking). Given that definiteness is not marked across all languages, we may expect that the marking of novel propositions too is also not universally available.
Hence this is another potential source of cross-linguistic variation that should be tracked when developing a typology of \textit{yes} and \textit{no.}   

\section{Marking response}\label{sec:wiltschko:5}

We have now seen that there are at least two different functions available for \textit{ResPrts.} They can be used as answers to polar questions and they can be used as markers of (dis)agreement with the speech act. I have argued that the difference between these two functions is syntactically conditioned: associating a \textit{ResPrt} with the spine just above the p-structure results in the answering function, while associating it above the speech act structure (GroundP) results in the (dis)agreement function. The assumption that \textit{ResPrts} can associate with different positions in the spine and that they can thereby acquire different functions raises the {question} as to whether there are any other positions that \textit{ResPrts} can associate with and that would derive other functions for \textit{ResPrts.} In this section I show that this is indeed the case.


In \citet{WiltschkoInPress}, it is argued that the speech act structure consists not only of the GroundP but also contains an articulated response layer above GroundP. That is, many speech acts can be characterized not only by the commitment the speaker displays towards the proposition (encoded in GroundP) but also by a request for the addressee as to how to respond to the \isi{utterance}.
This is known as the \textit{Call on Addressee} (henceforth CoA; \citealt{BeyssadeMarandin2006}). In English, CoA can be encoded by the intonational contour associated with a given \isi{utterance}.
For example, rising intonation can be analysed as encoding a request to respond (\citealt{BeyssadeMarandin2006}), and according to \citet{WiltschkoInPress} is associated with another layer in the speech act structure, namely RespP (see also \citealt{HeimEtAl2014}). This is schematized in \REF{ex:wiltschko:67}. 

\ea\label{ex:wiltschko:67}
A fully articulated speech act structure\\
\begin{forest} nice empty nodes
[RespP,name=resp [,name=emptyOne] [GroundP,name=groundp [,name=emptyTwo] [CP\is{complementizer}, name=cp [~~~~~~~~~~,roof]]]] 
\draw (emptyOne.north) -- (groundp.north);\draw (emptyTwo.north) -- (cp.north);
\draw[-{Triangle[open]}] (resp) -- ++(2cm,0cm) node [right,draw] {\itshape\bfseries CoA}; 
\draw[-{Triangle[open]}] (groundp) -- ++(2cm,0cm) node [right,draw] {\itshape\bfseries Commitment}; 
\end{forest}
%%\includegraphics[width=\textwidth]{a13Wiltschko-img24}
\z

Given the structure in \REF{ex:wiltschko:67}, we might expect that \textit{ResPrts} can also associate with RespP. There is indeed a use of \textit{ResPrts} in UAG which is amenable to such an analysis. In particular, \textit{ResPrts} can be used to mark the following \isi{utterance} as a response.
In this case the \textsc{trigger} of the response can be an immediately preceding \isi{utterance} as in \REF{ex:wiltschko:68} but also an immediately preceding (non-linguistic) situation as in \REF{ex:wiltschko:69}). 

\ea\label{ex:wiltschko:68}
\ili{Upper Austrian German}\\
\textbf{Context.} A and B work in the same cubicle.
A usually leaves work at 4, but sometimes his schedule is a bit off. B wants to know if A is indeed planning to leave at 4 today.
\begin{xlist}
 \exi{B:} \gll Gehst    du   heit    um 4 ham?\\
      Go-\oldstylenums{2}\textsc{sg} you today at   4  home.\\
      \glt ‘Are you going home at 4 today?’
 \exi{A:} 
 \begin{xlisti}
  \ex \gll \textbf{{Jo/Na}} des   was-st      doch   eh.   I geh imma um 4 ham.\\
    \textsc{yes/no} that   know-\oldstylenums{2}\textsc{sg} \textsc{prt}  \textsc{prt}  \textsc{i} go    always at 4 home.\\
  \glt ‘But you know that. I always go home at 4.’
  \ex \gll \textbf{{Jo}}/\textbf{{Na}}    des was-st       leicht  net?\\
      \textsc{yes/no}   that  know-\oldstylenums{2}\textsc{sg}  \textsc{prt}      \textsc{neg}\\
      \glt ‘So you don’t know that?’
  \end{xlisti}
\end{xlist}
\z

\ea\label{ex:wiltschko:69}
\ili{Upper Austrian German}\\
\textbf{Context.} A and B are co-workers. Their working hours are fixed and they always go home at 4.30. Typically, they get ready to leave at 4.25 so they can be out the door by 4.30. Today B is not showing any signs of getting ready even at 4.25. A comments:
\begin{xlista}
\ex
\gll  \textbf{{Jo/Na}} wonn  gehst   denn   du   heit   ham?\\
    \textsc{yes/no} when go-\textsc{\oldstylenums{2}sg}  \textsc{prt} you   today   home\\
  \glt ‘So when are you leaving today?’

\ex
 \textbf{Jo}/\textbf{Na} geh-st   du     heit    ned    ham?\\
    \textsc{yes/no} go-\oldstylenums{2}\textsc{sg}  you  today  \textsc{neg} home\\
    \glt ‘So aren’t you going home today?’
\end{xlista}
\z

\noindent In this use of the \textit{ResPrt}, the \textsc{content} of the response is not established by the response marker itself, but instead by the following \isi{utterance}.
This has a number of consequences for the distribution of the \textit{ResPrt} when used in this function. First, the following \isi{utterance} cannot be elided.
And second, the \textsc{content} of the response does not differ. depending on whether the positive or the negative \textit{ResPrt} is used.\footnote{An anonymous reviewer points out that the interchangeability of the positive and negative \textit{ResPrt} might indicate that at least in certain cases they might effectively be used expletively. To support this idea, the reviewer points out that in South African English there are certain uses of \textit{no} that do not seem to mean \textit{no} at all, as for example in i).

\ea\upshape
A:  How are you today? \\
B: \textbf{No}, I’m doing really well. 
\z 
It is not clear that \textit{no} is in fact meaningless here.

In particular, \textit{ResPrts} do not only respond to propositional content and speech acts, but they may also respond to the mere fact that the \textsc{trigger} expresses a belief on behalf of the speaker. So for example in ii) \textit{yeah} and \textit{no} co-occur without introducing a contradiction. In particular, \textit{yeah} expresses that B accepts that A beliefs p, but \textit{no} indicates that B does not agree (see \citealt{Guntly2016yeahno,Guntly2016Wiltschkoktunaxa} for further discussion).

\ea\upshape
A:  Yeah you don't know which is you don't know which is worse. \\
B: \textbf{{Yeah no}} i know which is worse. (Switchboard Corpus 02078A)
\z In light of the data in ii), I hesitate to conclude that \textit{no} in i) is really \isi{expletive}. But to determine its function will have to await further research.}
Finally, given that the \textsc{trigger} can be a non-linguistic situation, we may expect there to be no restrictions on the type of linguistic \textsc{triggers. T}his is indeed the case.
All types of speech acts can serve as \textsc{triggers} for this use of the \textit{ResPrt}. In \REF{ex:wiltschko:68}, the trigger is a polar \isi{question}, and in the data below we observe all other speech act types serving as \textsc{triggers: wh-}questions \REF{ex:wiltschko:70}, assertions \REF{ex:wiltschko:71}, commands \REF{ex:wiltschko:72}, and exclamations \REF{ex:wiltschko:73}.


\ea\label{ex:wiltschko:70}
\ili{Upper Austrian German}\\
\begin{xlist}
 \exi{A:}\gll Wonn gehst    denn du    heit    ham? \\
      When go-\oldstylenums{2}\textsc{sg} \textsc{prt} you  today home.\\
      \glt ‘When are you going home today?’
  \exi{B:} \gll \textbf{{Jo/Na}} des   was-st    doch   eh.   I geh imma    um 4 ham.\\
    \textsc{ja/no} that  know-\oldstylenums{2}\textsc{sg}   \textsc{prt}  \textsc{prt}  \textsc{i} go    always at  4 home.\\
    \glt ‘You know that already. I always go home at 4.’
\end{xlist}
\z


\ea\label{ex:wiltschko:71}
\ili{Upper Austrian German}\\
\begin{xlist}
\exi{A:} 
\gll I boag     ma gschwind dei    Auto aus. \\
      I {borrow} me  quickly    your car   \textsc{prt}\\
     \glt ‘I’m going to quickly {borrow} your car.’
\exi{B:}
\gll  \textbf{{Jo/Na}}     des  geht  owa  ned.\\
      \textsc{yes/no}   that goes but    \textsc{neg} \\
      \glt ‘But that’s not okay.’
\end{xlist}
\z

\ea\label{ex:wiltschko:72}
\ili{Upper Austrian German}\\
\begin{xlist}\exi{A:} 
\gll Jetzt geh endlich ins         Bett. \\
Now go   finally  into.the bed\\
\glt ‘Go to bed now!’
\exi{B:}
\gll \textbf{Jo/Na}  i  geh  jo   eh   scho.\\
      \textsc{yes/no} I go   \textsc{prt} \textsc{prt}  \textsc{prt}\\
 \glt ‘But I’m going already.’
\end{xlist}
\z

\ea\label{ex:wiltschko:73}
\ili{Upper Austrian German}\\
\begin{xlist}
\exi{A:} 
\gll Ma a so a grossa Hund. \\
\textsc{prt} a so a big      dog\\
\glt ‘Gee, what a big dog!’
\exi{B:}  
\gll \textbf{{Jo/Na}} ho-st         den no   ned  gsegn?\\
      \textsc{yes/no} have-\oldstylenums{2}\textsc{sg} \textsc{dem} \textsc{prt} \textsc{neg} seen\\
      \glt ‘Haven’t you seen him before?’
\end{xlist}
\z

I assume that the head of the response phrase (RespP) is associated with an unvalued coincidence feature [\textit{u}coin], just as any other clausal \isi{projection}. It relates the \isi{utterance} to the interlocutor’s response set. With the use of the positive \textit{ResPrt}, [\textit{u}coin] receives a positive value [+coin] and thus asserts that the \isi{utterance} coincides with the responder’s response set thereby marking it as a response, as in \REF{ex:wiltschko:74}.


\ea\label{ex:wiltschko:74}
Valuing [\textit{u}coin] in Resp\\
\begin{forest} %nice empty nodes
 [ [\textbf{\textit{yes}},name=yes] [RespP [\textit{\textbf{Response Set}}] [Resp [Resp\\\textbf{{[}+coin{]}},name=resp,base=top,align=center] [\isi{Ground} [~~~~~~~~~~~~~~,roof] ] ] ] ] 
 \path[-{Stealth[]}] (yes) edge[bend right=90] (resp);
\end{forest}
\z

However, this raises the {question} as to why the negative \textit{ResPrt} can also be used in this context. Everything else being equal, we expect it to value [\textit{u}coin] as [-coin] as in \REF{ex:wiltschko:75}.


\ea\label{ex:wiltschko:75}
Valuing [\textit{u}coin] in Resp\\
\begin{forest} %nice empty nodes
 [ [\textbf{\textit{no}},name=yes] [RespP [\textit{\textbf{Response Set}}] [Resp [Resp\\\textbf{{[}-coin{]}},name=resp,base=top,align=center] [\isi{Ground} [~~~~~~~~~~~~~~,roof] ] ] ] ] 
 \path[-{Stealth[]}] (yes) edge[bend right=90m,looseness=2] (resp);
\end{forest}
\z

So why is it possible to express the same thing by valuing [\textit{u}coin] as either positive or negative? I tentatively suggest that this may have to do with the timing of when the \textsc{content} of the response entered into the responder’s response set. Specifically, with the positive \textit{ResPrt} the responder indicates that the content \textit{is} in the response set \textit{now} ([+coin])\textit{;} in contrast, with the negative \textit{ResPrt} the responder indicates that the content \textit{was} \textit{not} in the response set prior to the time of \isi{utterance} ([-coin]).\footnote{This is reminiscent of the difference between languages with and without definiteness marking as analysed in \citet{Wiltschko2014}.{}}  This is compatible with its being in the response set at the time of \isi{utterance}.
Hence, both the positive and the negative \textit{ResPrt} can express the same content, with a difference in perspective.
If the negative \textit{ResPrt} is used, then the fact the response is now in the response set contrasts with the \isi{assertion} that it wasn’t in the response set prior to the time of \isi{utterance}.
Hence, the use of the negative \textit{ResPrt} focusses on the surprising nature of the response.


Note that the possibility for a \textit{ResPrt} to mark the \isi{utterance} as a response is subject to cross-linguistic variation. While in UAG this function is possible for \textit{ResPrt}, it is not in English: neither the positive nor the negative \textit{ResPrt} are well-formed in this context \REF{ex:wiltschko:76a}; instead, the particle \textit{so} is used \REF{ex:wiltschko:76b}.


\ea\label{ex:wiltschko:76}
 
\textbf{Context.} A and B are co-workers. Their working hours are fixed and they always go home at 4.30. Typically, they get ready to leave at 4.25 so they can be out the door by 4.30. Today B is not showing any signs of getting ready even at 4.25. A comments:\\
    \ea[*]{\label{ex:wiltschko:76a}\textbf{{Yes/*no}}, when are you leaving today?}
    \ex[]{\label{ex:wiltschko:76b}\textbf{{So}}, when are you leaving today?}
    \z
\z

It may be noted, though that UAG is not the only language where \textit{ResPrts} can be used in this way. While relevant information about \textit{ResPrts} is not easy to come by in grammars, I have found two candidates for \textit{ResPrts} that serve to mark the \isi{utterance} as a response, one in Macushi (Cariban) and the other in Cambodian. I briefly describe the relevant data in turn. 

Consider first Macushi. Here the positive \textit{ResPrt} (\textit{inna}) can be used to answer polar questions as in \REF{ex:wiltschko:77} but it can also be used after questions of other types (i.e., wh-questions) as in \REF{ex:wiltschko:78}, in which case it seems to express “\textit{Yes I’m answering you.”} \citep[46--49]{Abbott1991}.


\ea\label{ex:wiltschko:77}
\langinfo{Macushi}{}{\citealt[46]{Abbott1991}}\\
\begin{xlist}
 \exi{A:} \gll attî   pra   nan? \\
    \oldstylenums{2}.go   \textsc{neg}   \oldstylenums{2}.be.\textsc{q}\\
    \glt ‘Didn’t you go?’
\exi{B:}
    \gll inna,   uutî   pra   wai.   Aminke man.\\
    Yes  \oldstylenums{1}.go  \textsc{neg} \oldstylenums{1}.be  far   \oldstylenums{3}.be\\
    \glt ‘Yes, I didn’t go. It was far.’
\end{xlist}    
\z

\ea\label{ex:wiltschko:78}
\langinfo{Macushi}{}{\citealt[49]{Abbott1991}}\\
\begin{xlist}
\exi{A:}
\gll {î’ warapo}  i-tî-pai-nîkon    nai? \\
how:many  \textsc{adv-}go-\textsc{desid-coll}  \textsc{\oldstylenums{3}.be.q}\\
\glt ‘How many are wanting to go?’
\exi{B:} 
\gll inna,   tamî’nawîrî  anna  wîtî  e’-pai    man.\\
    yes,  all    \oldstylenums{1}:\textsc{excl} go  be-\textsc{desid}  3:be\\
\glt ‘Yes, we are all wanting to go.’
\end{xlist}
\z

Thus, in Macushi, \textit{ResPrt} can be used for answering as well as for marking the following \isi{utterance} as a response.

A similar pattern is also found in Khmer (Cambodian), where affirmative responses to yes/no questions may consist of repeating the main verb in the \isi{question}, or a full repetition of the \isi{question} in affirmative form. Crucially, in polite speech, the echoed verb is usually preceded by a form of the response particle \textit{baat} for men \REF{ex:wiltschko:79} and \textit{caah} for women. In the examples below, the optional ‘full’ responses are shown in brackets.


\ea\label{ex:wiltschko:79}
\langinfo{Khmer}{}{\citealt[24]{Huffman19701991}}\\
\begin{xlist}
\exi{A:}
\gll Look     sok-səpbaay   ciə   tee? \\
you(polite)  well-well  well  \textsc{prt}\\
\glt ‘Are you well?’
\exi{B:} 
\gll Baat (kñom)  sok-səpbaay   (ciə   tee).\\
Yes  (I)  well-well  well  \textsc{prt}  \\
\glt ‘Yes, I’m quite well.’
\end{xlist}
\z

Interestingly, negative responses to yes-no questions may consist solely of the negative particle \textit{tee} which is often followed by the negative form of the main verb.
Relevant for our purpose is the fact that in polite speech, \textit{tee} may be preceded by the appropriate form of the positive \textit{ResPrt} in which case it is followed by the full negative answer to the \isi{question}. This is shown in \REF{ex:wiltschko:80}.

\ea\label{ex:wiltschko:80}
\langinfo{Khmer}{}{\citealt[24]{Huffman19701991}}\\
\begin{xlist}
\exi{A:}
\gll look     sdap  baan  tee? \\
You(polite)  listen  can  \textsc{q}\\
\glt ‘Can you understand?’
\exi{B:}
\gll (baat) tee,   (kñom) sdap  min   baan tee.\\
  Yes    \textsc{q}     I      listen \textsc{neg}  can   \textsc{q}\\
  \glt ‘(Resp) no (I) don’t understand.’
\end{xlist}
\z

Given the profile of the \textit{ResPrt} in Macushi and Khmer, I conclude that in these languages, \textit{ResPrts} can be used to mark the host \isi{utterance} as a response, just like in UAG, though a more thorough investigation will have to confirm that this analysis is indeed on the right track. 

The use of \textit{ResPrts} as markers of response is yet another source of cross-linguistic variation that will have to be tracked when developing a typology of \textit{yes} and \textit{no.}   

\section{Conclusion}\label{sec:wiltschko:6}

In this paper I have shown that \textit{ResPrts} are multi-functional: they can be used as answers to polar questions, as markers of (dis)agreement with preceding utterances no matter what their speech act type; and finally they can also be used to mark the \isi{utterance} they precede as a response to some situation (linguistic or non-linguistic).\footnote{There are still other uses of \textit{ResPrt} that I haven’t discussed here.
These include backchannels (in the sense of \citealt{Yngve1970}) and discourse particles.}  We have seen that there is considerable cross-linguistic variation. For example, in UAG simple positive \textit{ResPrts} cannot be used to answer a negative polar \isi{question}. On the other hand, in English, \textit{ResPrts} cannot be used to mark a following \isi{utterance} as a response.
This is summarized in \tabref{tab:wiltschko:3}.

\begin{table}
\begin{tabular}{llcccc}
\lsptoprule
& & \multicolumn{2}{c}{English} & \multicolumn{2}{c}{UAG}\\\cmidrule(lr){3-4}\cmidrule(lr){5-6}
& & \textit{yes} & \textit{no} & \textit{jo} & \textit{na}\\
\midrule
\multirow{2}{*}{Answering function} & Positive \isi{question} & ✔ & ✔ & ✔ & ✔\\
& Negative \isi{question} & ✔ & ✔ & ✘ & ✔\\
\multicolumn{2}{l}{Marker of (Dis)agreement}  & ✔ & ✔ & ✔ & ✔\\
\multicolumn{2}{l}{Marker of Response}  & ✘ & ✘ & ✔ & ✔\\
\lspbottomrule
\end{tabular}
\caption{Three functions of \textit{ResPrts}\label{tab:wiltschko:3}}
\end{table}

In the analysis I have developed here, I have assumed (following \citealt{Wiltschko2014}) that multi-functionality can be syntactically conditioned. A given unit of language may acquire different functions depending on its place of association with the syntactic spine.
In addition, I have assumed an updated version of Ross’ \citeyear{Ross1970} performative hypothesis according to which speech-act structure is part of the syntactic computation. With these assumptions we were able to develop a unified analysis for the three different functions of \textit{ResPrts} we have discussed.

In this context, it is interesting to note that \textit{ResPrt}s can also grammaticalize.\footnote{I am grateful to an anonymous reviewer to draw my attention to this fact.}  In particular, \textit{no}{}-elements are a common source of negative reinforcers and/or \isi{presupposition} \isi{negation} markers \citep{Zanuttini1997,Poletto2008negativedoubling,Poletto2008focusnegation,deVosVanderAuwera2013}  while \textit{yes}{}-type elements can grammaticalise as sentence-internal discourse particles in \ili{German} (\textit{Er hat} \textbf{\textit{ja}} \textit{gesagt, dass …}). It will be interesting to explore whether there are any correlations between the types of responses \textit{ResPrts} can be used for and their grammaticalization paths. 

These findings highlight the importance of Holmberg’s insight that 
i) \textit{ResPrts} have a syntax, and 
ii) that the cross-linguistic patterns of \textit{ResPrts} should be carefully studied.
In fact, give the recent interest in the syntacticization of speech acts (\citealt{SpeasTenny2003,Sigurdsson2004,Giorgi2010,Giorgi2015,Haegeman2013,HaegemanHill2013} a.o.) it seems that \textit{ResPrts} will provide valuable insights into the articulation of speech act structure.

\section*{Acknowledgments}

Research on this paper was financially supported by a SSHRC Insight grant (‘\textit{Towards a typology of confirmationals’)} awarded to the author.

I wish to thank Strang Burton, Lisa Matthewson, Michael Rochemont and members of the \textit{eh-}lab at UBC (\url{http://syntaxofspeechacts.linguistics.ubc.ca}) for useful discussion and help with data collection. In particular, Yifang Yang is responsible for the SOAP corpus study and Jordan Chark for his help with the search for information on \textit{ResPrts} in grammars. Furthermore, I thank the students in the seminar on the grammar of discourse held at UBC, audiences of the workshop on \textit{Linguistic Variation in the Interaction between Internal and External Syntax} (Utrecht, February 2014) and the students in Leslie Saxon’s seminar at the University of Victoria.

This paper is dedicated to Anders Holmberg, who - once again - has been a pioneer in extending the empirical base for generative syntacticians by studying the syntax of response particles.
 
{\sloppy\printbibliography[heading=subbibliography,notkeyword=this]}
\end{document}