\documentclass[output=paper]{LSP/langsci}
\ChapterDOI{10.5281/zenodo.1228271} 
\author{Tuomas Huumo\affiliation{University of Turku, University of Tartu}}
\title{The partitive A: On uses of the Finnish partitive subject in transitive clauses}
%\epigram{Change epigram in chapters/03.tex or remove it there }

\abstract{Finnish existential clauses are known for the case marking of their S arguments, which alternates between the nominative and the partitive. Existential S arguments introduce a discourse-new referent, and, if headed by a mass noun or a plural form, are marked with the partitive case that indicates non-exhaustive quantification (as in ‘There is some coffee in the cup’).  In the literature it has often been observed that the partitive is occasionally used even in transitive clauses to mark the A argument. In this work I analyze a hand-picked set of examples to explore this \textit{partitive A}. I argue that the partitive A phrase often has an animate referent; that it is most felicitous in low-transitivity expressions where the O argument is likewise in the partitive (to indicate non-culminating aspect); that a partitive A phrase typically follows the verb, is in the plural and is typically modified by a quantifier (‘many’, ‘a lot of’). I then argue that the pervasiveness of quantifying expressions in partitive A phrases reflects a structural analogy with (pseudo)partitive constructions where a nominative head is followed by a partitive modifier (e.g. ‘a group of students’). Such analogies may be relevant in permitting the A function to be fulfilled by many kinds of quantifier + partitive NPs.}

%\keywords{argument marking, Finnish, case, partitive, transitivity, quantifiers}

\maketitle

\begin{document}

\section{Introduction}
\label{15-hu-sec:1}

The \ili{Finnish} argument marking system is known for its extensive case alternation that concerns the marking of S (single arguments of \isi{intransitive} predications) and O (object) arguments, as well as predicate adjectives. In this system, the \isi{partitive case} plays a central role: it alternates with the accusative in the marking of O (see \eg \citealt{Heinamaki1984Aspect,Heinamaki1994Aspect,Kiparsky1998Partitive,Huumo2010Nominal,Huumo2013incompleteness}, and \sectref{15-hu-sec:3} of this paper), and with the nominative in the marking of both S (in existential clauses; see \eg \citealt{Huumo2003Incremental}) and predicate adjectives (see \citealt{Huumo2009Fictive}). By contrast, A arguments are, in principle, always in the nominative in Standard \ili{Finnish}. However, since the late 19th century, scholars have pointed out that the partitive is occasionally used even in the marking of A arguments, in spite of the fact that until the 1990’s, the \ili{Finnish} language planning authorities condemned such uses as ungrammatical.

In this paper, I study such \textit{partitive A} arguments with data manually gathered from the Internet. I will argue that the partitive A combines features of canonical (nominative) A arguments (\isi{animate} and agentive referents are typical) and existential S arguments (the referent is typically discourse-new and indicates non-exhaustive quantity).  I will also argue that the most typical context for the partitive A are low-\isi{transitivity} expressions. This is the main reason why not only A but also O is in the \isi{partitive case} in most instances: when marking the O (of an affirmative clause) the partitive indicates a non-culminating event or a quantitatively non-exhaustive reference. If the accusative O is used with a partitive A, then the reading of A is distributive: each of its referents participates in the event individually. 

In the marking of S arguments, the \ili{Finnish} \isi{partitive subject}\footnote{The subjecthood of this element is under dispute; see \citet{Huumoetal2015Subject} and the literature mentioned there. However, the term \textit{subject} is conventionally used for it, and I follow this practice for convenience.} is best known for its use in existential clauses, where the \isi{partitive case} marks subject NPs headed by mass nouns or plurals. The referent of the existential partitive S is typically discourse-new and consists of a non-exhaustive quantity of a substance (mass nouns) or of a multiplicity (plurals); a characteristic feature of the partitive is non-exhaustive reference (an \isi{indefinite}, open or unbounded quantity in different terminologies), whereas its counterparts, the nominative and the accusative, typically indicate exhaustive reference (\isi{definite}, closed, or bounded quantity). Consider \REF{15-hu-ex:1} and \REF{15-hu-ex:2}, which are canonical existential clauses with a clause-final S; for the uses of the \isi{partitive subject} in old literary \ili{Finnish}, see \citet{DeSmit2016Fluid}.
In existential clauses, only S arguments headed by a singular count noun take the \isi{nominative case} \REF{15-hu-ex:3}.

\ea\label{15-hu-ex:1}
\gll Kupi-ssa on kahvi-a.\\
cup-\textsc{ine} be.\textsc{prs}.\textsc{3sg} coffee-\textsc{par}\\
\glt ‘There is coffee in the cup.’
\z

\ea%2
\label{15-hu-ex:2}
\gll Leikkikentä-llä juokse-e laps-i-a.\\
playground-\textsc{ade} run-\textsc{prs}.\textsc{3sg} child-\textsc{pl}-\textsc{par}\\
\glt ‘There are children running in the playground.’
\z

\ea%3
\label{15-hu-ex:3}
\gll Pöydä-llä on kirja.\\
table-\textsc{ade} be.\textsc{prs}.\textsc{3sg} book.\textsc{nom}\\
\glt ‘There is a book on the table.’
\z

As can be seen from \REF{15-hu-ex:2}, the partitive S does not trigger verb agreement: the verb is in a 3rd person singular form in spite of the plural partitive. The typical position of the existential S arguments is after the verb, but since the \ili{Finnish} \isi{word order} is discourse-pragmatically conditioned (for details, see \citealt[35–62]{Vilkuna1989Free}), existential S arguments may also have a clause-initial position. On the other hand, \isi{indefinite} or focused non-existential S arguments are likewise often placed towards the end of the clause (see \citealt[187–191]{Vilkuna1989Free}). This means that \isi{word order} is not a reliable criterion in distinguishing existential and \isi{intransitive} \REF{15-hu-ex:4} constructions from each other (\cf also \citealt{Huumoetal2014Partitives}, for a comparison of \ili{Finnish} and \ili{Estonian}). 

\ea\label{15-hu-ex:4}
\gll Kirja on pöydä-llä.\\
book.\textsc{nom} be.\textsc{prs}.\textsc{3sg} table-\textsc{ade}\\
\glt ‘The book is on the table.’
\z

In \ili{Finnish} linguistics, a lively debate on existential clauses and the functions of the partitive S has been going on since the 1950’s.  It has been pointed out that though actual usage concentrates around certain semantically pale existential verbs (\eg \textit{olla} ‘be’ and \textit{tulla} ‘come’), the range of (\isi{intransitive}) verbs available for the existential construction is actually wide and includes even highly agentive verbs such as \textit{juosta} ‘run’, \textit{opiskella} ‘study’, \textit{tapella} ‘fight’ and \textit{tanssia} ‘dance’. In earlier works, scholars attempted to build exhaustive lists of “existential verbs”, but in the 1970’s this attempt was more or less given up. More recent analyses (\eg \citealt{Huumo2003Incremental},  \citealt{Huumoetal2015Subject}), with \citet{Schlachter1958Partitiv} and \citet{Siro1974Eksistentiaalilauseen} as their early predecessors,  have emphasized the construction-level meaning of the existential clause, arguing that the construction backgrounds the activity indicated by the verb and foregrounds the locational relationship that prevails between the typically clause-initial locative adverbial and the existential S, the referent of which is introduced as a discourse-new entity within the location. 

In many works on \ili{Finnish} existential clauses, it has been pointed out that the \isi{partitive subject} is occasionally used even in transitive clauses, especially if the verb and the object form an idiomatic phrase \REF{15-hu-ex:5} but sometimes in other low-\isi{transitivity} predications as well \REF{15-hu-ex:6}--\REF{15-hu-ex:7}  (\eg \citealt[77]{Siro1964Suomen}; \citealt{Ikola1972Partitiivi}; \citealt{Saarimaa1967Kielenopas,}; \citealt{Penttil1963Suomen}; \citealt[156–157]{Yli-Vakkuri1979Partitiivisubjektin}; \citealt[167–168]{Hakulinenetal1979Nykysuomen}; \citealt{Sandsetal2001Non-canonical}; a generative approach is \citealt{Nikanne1994Notes}). 

\ea\label{15-hu-ex:5}
\gll Use-i-ta sotila-i-ta sa-i surma-nsa taistelu-ssa.\\
several-\textsc{pl}-\textsc{par} soldier-\textsc{pl}-\textsc{par} get-\textsc{pst}.\textsc{3sg} death-\textsc{acc}.\textsc{3poss} battle-\textsc{ine}\\
\glt ‘Several soldiers got killed (literally: ‘got their death’) in the battle.’
\z

\ea\label{15-hu-ex:6}
\gll Mon-i-a ihmis-i-ä odott-i satee-ssa bussi-a.\\
many-\textsc{pl}-\textsc{par} person-\textsc{pl}-\textsc{par} wait-\textsc{pst}.\textsc{3sg} rain-\textsc{ine} bus-\textsc{par}\\
\glt ‘Many people were waiting for the bus in the rain.’
\z

\ea\label{15-hu-ex:7}
\gll Keitto-a seuras-i erilais-i-a liha-, kala- ja vihannes-ruok-i-a.\\
soup-\textsc{par} follow-\textsc{pst}.\textsc{3sg} differentd-\textsc{pl}-\textsc{par} meat fish and vegetable-dish-\textsc{pl}-\textsc{par}\\
\glt The soup was followed by different dishes of meat, fish, and vegetables’. \citep{Ikola1972Partitiivi}
\z

Especially during the 20th century, such uses attained the attention of language planning authorities (\eg \citealt{Saarimaa1967Kielenopas}; \citealt{Ikola1972Partitiivi}; \citeyear[139]{Ikola1986Nykysuomen}) who considered them errors and recommended the use of the nominative instead (\eg \textit{usea-t sotilaa-t sai-vat}… [many-\textsc{pl}.\textsc{nom} soldier-\textsc{pl}.\textsc{nom} get-\textsc{pst}.\textsc{3pl}] in \REF{15-hu-ex:5}). A more tolerant approach was adopted by Itkonen (\eg 1988 and the more recent \citealt{Itkonenetal2007Uusi}) who accepted the partitive A in transitive constructions that are semantically close to (and can be rephrased by) existential clauses proper, such as \REF{15-hu-ex:5} above, but condemned the wider use as “going against language intuition”. 

In an insightful paper, \citet{Yli-Vakkuri1979Partitiivisubjektin} analyzed the partitive A\footnote{Yli-Vakkuri did not use this term.} with data consisting of examples from the earlier linguistic literature, as well as a hand-picked set of 39 examples she collected from literary fiction, newspapers and spoken discourse. In her data, the following transitive verbs and verb–object combinations are used with the partitive A: 1) \textit{seurata} ‘follow’, \textit{kohdata} ‘meet’, \textit{odottaa} ‘wait’; 2) expressions where the object elaborates the activity designated by the verb rather than introduces a referent, as in ‘play cards’ or ‘sing hymns’, 3) perception verbs such as \textit{kuunnella} ‘listen’ and \textit{katsella} ‘watch’.\footnote{It may be worth pointing out that the verbs in groups 1 and 3 assign the (aspectually motivated) \isi{partitive case} to their object, while the verbs in group 2 also allow the accusative object, if the culmination of the event is indicated (‘to sign a hymn from the beginning to the end’). However, in their uses with the partitive A, group 2 verbs take the \isi{partitive object}, which then indicates progressive aspect. For object marking in \ili{Finnish} in general, see \sectref{15-hu-sec:3}.} \citeauthor{Yli-Vakkuri1979Partitiivisubjektin} also observed that in most instances the subject NP includes a quantifying expression, as in examples \REF{15-hu-ex:5} and \REF{15-hu-ex:6} (‘several’, ‘many’). Unquantified subject phrases that consist of a partitive-marked noun alone occurred only three times in her 39 examples. In addition, there were three instances where the NP included an adjectival modifier. The rest of Yli-Vakkuri’s examples, 33 instances, include a quantifying element. The role of the quantifying element thus seems to be central and will be discussed below (\sectref{15-hu-sec:6}) in more detail.

The partitive A appears to be quite rare in text corpora. In the Syntax Archives corpus at the University of Turku there is only one occurrence of a partitive NP used as a transitive subject in written\footnote{In the spoken dialect materials of the Syntax Archives, there are some occurrences where the pronoun \textit{ke-tä} [who-\textsc{par}] is used as a transitive subject. However, such occurrences do not count as instances of the partitive A proper, because this partitive form of the pronoun \textit{kuka} ‘who’ is productively used in the function of the nominative in the southwestern dialects of \ili{Finnish}; see also the discussion of the quantifier \textit{monta} (which is morphologically a partitive form but behaves like a nominative syntactically) in (\sectref{15-hu-sec:6-2}).} text materials \REF{15-hu-ex:8}.

\ea%8
\label{15-hu-ex:8}

\gll tulirokko-a seura-a usein jälkitaute-j-a\\
scarlet.fever-\textsc{par} follow-\textsc{prs}.\textsc{3sg} often complication-\textsc{pl}-\textsc{par}\\
\glt ‘Scarlet fever is often followed by complications.’
\z


Example \REF{15-hu-ex:8} has a partitive A in the clause-final position. It is a plural form introducing a discourse-new, quantitatively non-exhaustive multiplicity (‘[some] complications’) and thus resembles the canonical existential S in \REF{15-hu-ex:1}--\REF{15-hu-ex:3}. As is typical in existentials, the verb in example \REF{15-hu-ex:8} does not agree in number with the plural partitive A – in general, verbs never show agreement with a partitive S in number or person in \ili{Finnish}, and such a use would be blatantly ungrammatical for the native speaker’s ear.\footnote{\citet[395]{Serzant2015Independent} points out that in North \ili{Russian}, as well as in Veps, which is closely related to \ili{Finnish}, the verb sometimes shows number and person agreement with a plural \isi{partitive subject} (see also \citealt{Koptjevskaja-Tammetal2001Circum}).} 

The Syntax Archives corpora thus suggest that the partitive A is indeed a rare phenomenon. However, its rarity in edited written texts may be caused by language planning authorities, who have considered usages like \REF{15-hu-ex:8} as errors. Being aware of this, the authors and editors of the texts in the corpora may have avoided using it. Unedited Internet texts turn out to be a more fruitful source, where it is relatively easy to find occurrences, if only one knows what to look for.  

In this work, I use hand-picked (or Google-picked) data to discuss the partitive A. As my starting point, I have the set of examples from \citegen{Yli-Vakkuri1979Partitiivisubjektin} work, as well as another hand-picked set of 20 examples (courtesy of Jaakko Leino), including examples such as \REF{15-hu-ex:9} and \REF{15-hu-ex:10}:

\ea\label{15-hu-ex:9}
\gll Jo-ta-in unkarilais-i-a esitt-i siellä kansa-n-tansse-j-a.\\
some-\textsc{par}-\textsc{clit} Hungarian-\textsc{pl}-\textsc{par} perform-\textsc{pst}.\textsc{3sg} there folk-\textsc{gen}-dance-\textsc{pl}-\textsc{par}\\
\glt `Some Hungarians were performing folk dances there.’ (from a private conversation; example courtesy of Jaakko Leino)
\z

\ea%10
\label{15-hu-ex:10}
 (Miten on mahdollista, että)\\
\gll  korti-n on saa-nut henkilö-i-tä,\\
 card-\textsc{acc} have.\textsc{prs}.\textsc{3sg} get-\textsc{ptcp} person-\textsc{pl}-\textsc{par}\\
\glt (joilla ei ole mitään yhteyttä keskustaan?)\\
\glt  `(How is it possible that) the card has been given to persons (who have no connection with the Centre [a political party])?’ (Verkko-Ilta-Sanomat 28.5.2010, example courtesy of Jaakko Leino)
\z

In example \REF{15-hu-ex:9}, the partitive A is clause-initial, \isi{animate} and agentive. It differs clearly from the partitive A in example \REF{15-hu-ex:8}, which is more similar to existential S arguments by being clause-final and \isi{inanimate}. In both \REF{15-hu-ex:8} and \REF{15-hu-ex:9}, the partitive A is an \isi{indefinite} plural form. In example \REF{15-hu-ex:10}, the partitive A is clause-final and \isi{animate} but not agentive. Another special feature of \REF{15-hu-ex:10} is that the object \textit{kortin} is in the \isi{accusative case} – note that in the earlier examples, \REF{15-hu-ex:8} and \REF{15-hu-ex:9}, not only the partitive A but also the object NP (O) has been in the partitive. Indeed, it seems that the partitive A favors contexts where O is likewise in the partitive. This is a striking feature, because it results in two partitive-marked NPs being arguments of the same verb. 

In collecting data for this study with Google, I have used search strings with a specific verb form that is either preceded or followed by a partitive form of a semantically schematic noun, or by a quantifying expression that typically combines with a partitive-marked noun to form an NP. Such data are of course extremely biased and give no ground for a statistical analysis. Nevertheless, as a result I have a set of attested 117 examples from actual language use, and it would be easy to expand this set by further searches. This means that my study can give a picture of contexts and constructions where the partitive A \textit{at least} can be used in unedited written \ili{Finnish}. I have also used my own and my colleagues’ native speaker intuitions for grammaticality judgments of the examples. Against the background that language planning authorities have considered the partitive A an error, it may be surprising that practically no example in my data set feels blatantly ungrammatical to the native speaker’s ear. 

More specifically, the following points of view will be brought up in this work; points 1–3 concern the synchronic distribution of the partitive A while point 4 also has diachronic connotations.


\begin{enumerate}
\item 
What is the partitive A like in terms of its own grammatical structure and lexical semantics (\eg \isi{animacy})? What is its \isi{semantic role} in the clause? What kinds of verbs does it occur with? 

\item 
What is the role of the quantifiers that seem to be typical in NPs with the function of a partitive A?

\item 
Why is the object NP also in the partitive in most instances? When is the accusative object used?

\item 
What is the motivation for using the \isi{partitive case} in a transitive subject NP? Are there grammatical systems in the language that pave the road for the \isi{partitive subject} to spread into transitive clauses?

\end{enumerate}

I will discuss the grammatical and semantic features of the partitive A and the range of verbs in my data in \sectref{15-hu-sec:2}. \sectref{15-hu-sec:3} concentrates on the object NP and its \isi{case marking}, and \sectref{15-hu-sec:4} on agreement and \isi{word order}. \sectref{15-hu-sec:5} discusses transitive infinitival constructions used as adverbial modifiers in matrix clauses that have an \isi{intransitive} verb and a partitive S, arguing that such constructions probably give analogical support to the partitive A. \sectref{15-hu-sec:6} discusses the role of the quantifiers that are common in NPs with the function of the partitive A, from the point of view of the argument put forward by \citet{Yli-Vakkuri1979Partitiivisubjektin} that the quantitative function of the partitive NP with a quantifier is fundamentally different from that of a bare partitive form. \sectref{15-hu-sec:7} sums up the results of the study. In the following sections, all examples are from the Internet, unless followed by the symbol (C) which marks examples coined by the author as a native speaker of \ili{Finnish}.

\section{Nouns and verbs in typical clauses with a partitive A}
\label{15-hu-sec:2}

In general, the lexical range of nouns that can head the partitive A phrase (which is an NP) resembles that available for the partitive S: there are occurrences extending from \isi{inanimate} nouns, as in example \REF{15-hu-ex:8}, to \isi{animate} ones, as in example \REF{15-hu-ex:9}. However, even though the data used here do not permit statistical conclusions, it may be relevant that it is quite easy to find instances of the partitive A with an agentive human referent, which of course is a feature typical of (nominative) A arguments but less so for partitive S arguments . In \citegen{Yli-Vakkuri1979Partitiivisubjektin} set of 39 examples, \isi{animate} referents dominate likewise: there are only 10 examples with an \isi{inanimate} referent, while in the majority of \citeauthor{Yli-Vakkuri1979Partitiivisubjektin}’s examples, the referent of the partitive A is human. This suggests a difference between the typical partitive S (referring to an \isi{inanimate} \textsc{theme}) and the typical partitive A (referring to a human \textsc{agent}). With respect to \isi{animacy}, the partitive A thus resembles the nominative A which in the spoken-language data of \citet[92]{Helasvuo2001Syntax} has a human referent in 91.4\% of the cases. 

In the Google searches I performed, the verb \textit{seurata} ‘follow’ turned out to be an especially fruitful candidate to search for. It is common in \citegen{Yli-Vakkuri1979Partitiivisubjektin} data as well. In my searches, \textit{seurata} produced numerous hits with a partitive A, with both \isi{animate} and \isi{inanimate} referents. The presence of a quantifying element in the partitive A phrase also appears to be typical: in my data the partitive-marked noun is often preceded by a quantifier even if the quantifier was not part of the search string. For instance, the string \textit{ihmis}-\textit{i}-\textit{ä} \textit{seuras}-\textit{i} [person-\textsc{pl}-\textsc{par} follow-\textsc{pst.3sg}] produced numerous hits like \REF{15-hu-ex:11}, where a numeral quantifier (here \textit{tuhans}-\textit{i}-\textit{a} [thousand-\textsc{pl}-\textsc{par}] ‘thousands [of]’, likewise in the partitive), precedes the partitive form \textit{ihmisiä}. However, it also produced (fewer) hits where the partitive noun constitutes the subject NP alone \REF{15-hu-ex:12}. 

\ea\label{15-hu-ex:11}
\gll Tuhans-i-a ihmis-i-ä seuras-i tapahtum-i-a ääneti.\\
thousand-\textsc{pl}-\textsc{par} person-\textsc{pl}-\textsc{par} follow-\textsc{pst}.\textsc{3sg} event-\textsc{pl}-\textsc{par} silently\\
\glt `Thousands of people were following / followed the events silently.’ %(\url{http://nine.tweettunnel.com/reverse2.php?textfield=TuomelaTiina})
\z

\ea (Esityspaikka oli täynnä, niin että)\label{15-hu-ex:12}\\
\gll ihmis-i-ä seuras-i puhe-tta esityspaika-n ulkopuole-lla-kin.\\
person-\textsc{pl}-\textsc{par} follow-\textsc{pst}.\textsc{3sg} speech-\textsc{par} venue-\textsc{gen} outside-\textsc{ade}-also\\
\glt (The venue was full, so that) there were people following the speech even outside the venue.’ %(\url{http://juhanitikkanen.blogspot.fi/2008/10/lassi-nummi-turun-kirjamessuilla.html})
\z

The more specific string \textit{si}-\textit{tä} \textit{seuras}-\textit{i} \textit{use}-\textit{i}-\textit{ta} [it-\textsc{par} follow-\textsc{prs}.\textsc{3sg} several-\textsc{pl}-\textsc{par}] ‘it was followed by several…’ which specifies both the pre-verbal object NP (the pronoun ‘it’ in the \isi{partitive case}) and the post-verbal partitive-marked quantifier, produced \isi{inanimate} hits only. In those instances, the meaning of \textit{seurata} ‘follow’ is typically that of temporal succession, as in \REF{15-hu-ex:13}. 

\ea\label{15-hu-ex:13}
(Ensin kuultiin pienempi räjähdys ja)\\
\gll si-tä seuras-i use-i-ta voimakka-i-ta räjähdyks-i-ä luol-i-ssa.\\
it-\textsc{par} follow-\textsc{pst}.\textsc{3sg} several-\textsc{pl}-\textsc{par} powerful-\textsc{pl}-\textsc{par} explosion-\textsc{pl}-\textsc{par} cave-\textsc{pl}-\textsc{ine}\\
\glt (First a minor explosion was heard and) it was followed by several powerful explosions in the caves.’ %(\url{http://www.geocaching.com/geocache/GC43WDC_uudenkylan-asevarikon-rajahdys-1965?guid=2d54ae10-a7c2-4349-932f-bad23711fa8b})
\z

As I mentioned above, the verb \textit{seurata} ‘follow’ appears to be particularly common with the partitive A. This is comprehensible, because the verb is polysemous and also has \isi{intransitive} uses which make it available in existential clauses proper; consider the existential \REF{15-hu-ex:14} where the initial elative (‘from’) phrase indicates a reason for the problems that arise. 

\ea%14
\label{15-hu-ex:14}

\gll Teoria-sta-si seura-a ongelm-i-a.\\
theory-\textsc{ela}-\textsc{2sg}.\textsc{poss} follow-\textsc{prs}.\textsc{3sg} problem-\textsc{pl}-\textsc{par}\\
\glt `Problems will follow from your theory.’ 
\z


Since \textit{seurata} ‘follow’ appears to be a verb rather productively used with the partitive A, I used it as a test case to gather statistical information about the phenomenon from the Finnish Internet Parsebank.\footnote{My cordial thanks are due to Veronika Laippala, Filip Ginter and Jenna Kanerva for the Parsebank data (for the Parsebank, see http://bionlp.utu.fi/finnish-internet-parsebank.html).} In a dataset of approximately 4 million sentences, the program found altogether 7875 transitive uses of \textit{seurata} ‘follow’, of which 13, \ie  0.17 \%, have a partitive A. In all 13 instances the \isi{word order} is OVS (overall, there are 6121 SVO and 1190 OVS sentences), and the NP with the function of the partitive A includes quantifying elements in 9 instances. Among the four non-quantified instances, there is one with a genitive modifier, one with an adjectival modifier, and two that consist of the partitive-marked noun alone. This demonstrates that the partitive A is indeed quite rare in actual usage.\footnote{A fully automatic search that would be able to recognize partitive A constructions in the internet must be left for future research.} 

Other transitive verbs that produced hits in my Google searches include, \eg \textit{esittää} ‘perform’ (example \REF{15-hu-ex:9} above), \textit{palvella} ‘serve’ (\REF{15-hu-ex:15} below), \textit{jättää} ‘leave’ \REF{15-hu-ex:16}, \textit{hakea} ‘fetch; apply [for]’ \REF{15-hu-ex:17}, \textit{viettää} ‘spend’ \REF{15-hu-ex:18}, \textit{tarkkailla} ‘observe’, \textit{lukea} ‘read’, \textit{tuijottaa} ‘stare’, \textit{nähdä} ‘see’, \textit{tanssia} ‘dance’, \textit{odottaa} ‘wait’, \textit{tehdä} ‘do’, \textit{kuunnella} ‘listen’, and locative transitives such as \textit{ympäröidä} ‘surround’ or \textit{reunustaa} ‘rim’ (as in \textit{The park is rimmed by beaches}), among others. As the choice of the verbs that were searched for were based on the two small hand-picked sets of examples I had, together with my intuition and “educated guessing”, the list is of course not exhaustive. 

\ea%15
\label{15-hu-ex:15}
\gll Tuhans-i-a kelkko-j-a palvele-e matkailu-a.\\
thousand-\textsc{pl}-\textsc{par} sleigh-\textsc{pl}-\textsc{par} serve-\textsc{prs}.\textsc{3sg} tourism-\textsc{par}\\
\glt ‘Thousands of sleighs serve tourism.’ (Newspaper headline, Helsingin Sanomat 9.12.2000, example courtesy of Jaakko Leino)
\z

\ea\label{15-hu-ex:16}
\gll Sato-j-a-tuhans-i-a ihmis-i-ä jätt-i Suome-n.\\
hundred-\textsc{pl}-\textsc{par}-thousand-\textsc{pl}-\textsc{par} person-\textsc{pl}-\textsc{par} leave-\textsc{pst}.\textsc{3sg} Finland-\textsc{acc}\\
\glt ‘Hundreds of thousands of people left Finland.’% %(\url{http://blogit.iltalehti.fi/maria-guzenina- richardson/2010/03/12/maahanmuutto/comment-page-1/})
\z

\ea\label{15-hu-ex:17}
\gll Viera-i-ta ihmis-i-ä hak-i tavaro-i-ta piene-stä vaaleanpunaise-sta huonee-sta.\\
strange-\textsc{pl}-\textsc{par} person-\textsc{pl}-\textsc{par} fetch-\textsc{pst}.\textsc{3sg} thing-\textsc{pl}-\textsc{par} small-\textsc{ela} pink-\textsc{ela} room-\textsc{ela}\\
\glt Strange people were fetching things from the small pink room.’ %(\url{http://www.peda.net/verkkolehti/tampere/tyk?m=content&a_id=9812})
\z

\ea\label{15-hu-ex:18}
\gll Vasta.ranna-n möki-llä ihmis-i-ä viett-i ilta-a ja soittel-i kitara-a.\\
opposite.shore-\textsc{gen} cabin-\textsc{ade} person-\textsc{pl}-\textsc{par} spend-\textsc{pst}.\textsc{3sg} night-\textsc{par} and play-\textsc{pst}.\textsc{3sg} guitar-\textsc{par}\\
\glt At the cabin on the opposite shore there were people spending the night and playing the guitar.’ 
\z

It is worth noting that the partitive A seems to occur almost exclusively in the plural. It is very difficult to find hits where the partitive A is a singular form of a mass noun, as in \citegen[261]{Vilkuna1989Free} example \REF{15-hu-ex:19}. 

\ea\label{15-hu-ex:19}
\gll että tietty-j-ä historiallis-i-a muutoks-i-a seura-a välttämättä jo-ta-kin muu-ta\\
that certain-\textsc{pl}-\textsc{par} historical-\textsc{pl}-\textsc{par} change-\textsc{pl}-\textsc{par} follow-\textsc{prs}.\textsc{3sg} necessarily something<\textsc{par}> else-\textsc{par}\\
\glt 'that certain historical changes are necessarily followed by something else’
\z

In sum, the partitive A seems to favor \isi{animate}, often human referents, though \isi{inanimate} referents can also be found. This is comprehensible, because in most instances both A and O are in the partitive (for reasons that will be discussed in \sectref{15-hu-sec:3}), and \isi{animacy} of A is then one factor that keeps them apart. The \isi{semantic role} of the \isi{animate} referent is often agentive, as in \REF{15-hu-ex:16}--\REF{15-hu-ex:18}, while \isi{inanimate} referents are more typical in clauses that express a transitive locative (\eg ‘surround’) or a temporal (\eg ‘follow’) relationship. In spite of the agentive role of the \isi{animate} NPs, the data show a low level of \isi{transitivity}, as will be seen in the following section. 

\section{Case marking and the aspectual function of the object NP}\label{15-hu-sec:3}

As the examples discussed so far show, many verbs that occur with the partitive A are agentive, or at least such that they allow an \isi{animate} subject. It is also noteworthy that in most instances not only the subject but also the object NP is in the partitive and not in the accusative, which would be the other option and which might be expected for morphosyntactic reasons (\ie to differentiate between A and O by \isi{case marking}). In \citegen{Yli-Vakkuri1979Partitiivisubjektin} set of 39 examples, there are 10 instances with an accusative object and 27 with a \isi{partitive object}; in two examples the construction is elliptical and lacks an overt object . 

In general, the accusative\footnote{In this paper, following the convention of traditional Finnish  grammars, I use the term \textit{accusative object} as a syntactic cover term for all objects that are not in the \isi{partitive case}. In morphological terms, the category of accusative objects comprises a) singular objects (of personal constructions) with the historical accusative ending \textit{-n}, b) nominative singular objects in imperative and passive constructions, c) the special accusative form of personal pronouns with the ending \textit{-t}, and d) plural nominative objects. This morphologically heterogeneous category (as a whole) constitutes the counterpart of the partitive in the object-marking alternation based on the oppositions of quantification, aspect, and polarity.} vs. partitive opposition in \ili{Finnish} object marking reflects three features (see \eg \citealt{Heinamaki1984Aspect,Heinamaki1994Aspect,Kiparsky1998Partitive,Huumo2010Nominal,Huumo2013incompleteness}): 
1) exhaustive [\textsc{acc}] vs. non-exhaustive [\textsc{par}] quantity, 
2) culminating [\textsc{acc}] vs. non-culmi\-nating [\textsc{par}] aspect, and 
3) positive [\textsc{acc}] vs. negative [\textsc{par}] polarity. Condition 3) dominates in the sense that the partitive is used in all object NPs under negation, regardless of the two other conditions. In affirmative clauses, condition 2) dominates over condition 1) (as has been argued \eg by \citealt{Vilkuna1996Suomen}), in the sense that non-culmi\-nating aspect triggers the partitive irrespective of whether the quantity is exhaustive or non-exhaustive. It is only in instances where the aspect is culminating (\eg in achievements such as ‘I found some mushrooms’) that the partitive can indicate non-exhaustive quantity (for a more detailed hierarchy of the functions, see \citealt{Huumo2013incompleteness}). Thus the partitive signals non-culminating aspect in \REF{15-hu-ex:20} and non-exhaustive quantity in \REF{15-hu-ex:21}. The accusative object is used if the clause is affirmative, designates the culmination of the event and has an object NP that designates a \isi{definite} quantity; \cf the accusative versions of \REF{15-hu-ex:20} and \REF{15-hu-ex:21}. Note that the singular partitive would be ungrammatical in \REF{15-hu-ex:21} where the verb indicates a punctual achievement and thus the reading with non-culminating aspect (\ie progressive) is excluded; likewise the quantity of the referent (book) cannot be understood as non-exhaustive, which would trigger the partitive. In \REF{15-hu-ex:22} the verb is \isi{atelic} and the aspect thus non-culminating; therefore only the \isi{partitive object} is possible.

\ea
\label{15-hu-ex:20}

\gll Rakens-i-n talo-n {\textasciitilde} talo-a.\\
 build-\textsc{pst}-\textsc{1sg} house-\textsc{acc} {\textasciitilde} \textsc{par}\\
\glt ‘I built [and completed] a/the house.’ [\textsc{acc}] / ‘I was building a/the house {\textasciitilde} built a/the house a bit {\textasciitilde} did some house-building.’ [\textsc{par}] (C)
\z

\ea
\label{15-hu-ex:21}

\gll Löys-i-n kirja-n {\textasciitilde} kirja-t {\textasciitilde} kirjo-j-a {\textasciitilde} *kirja-a\\
 find-\textsc{pst}-\textsc{1sg} book-\textsc{acc} {\textasciitilde} \textsc{pl}.\textsc{nom} {\textasciitilde} \textsc{pl}-\textsc{par} {\textasciitilde} *\textsc{sg}.\textsc{par}\\
\glt  ‘I found [a/the] book.’ [\textsc{sg}.\textsc{acc}]\\
 ‘I found the books.’ [\textsc{pl}.\textsc{nom}]\\
 ‘I found some books.’ [\textsc{pl}-\textsc{par}] (C)\\
\z

\ea
\label{15-hu-ex:22}

\gll Ihaile-n Kallu-a{\textasciitilde}*Kallu-n.\\
 admire-\textsc{prs}-\textsc{1sg} Kallu-\textsc{par}{\textasciitilde}*\textsc{acc}\\
\glt  ‘I admire Kallu.’ (C)
\z

Both in the data I collected for this study and in the data analyzed by \citet{Yli-Vakkuri1979Partitiivisubjektin}, partitive objects are more common than accusative objects. Though this may be surprising in morphosyntactic terms, it is semantically reasonable when one considers the verbs that appear to be typical with the partitive A, \ie  verbs with meanings such as ‘serve’, ‘follow’, ‘observe’, ‘stare’, ‘dance’, ‘wait’ and ‘listen’. Most of these are low-\isi{transitivity} verbs indicating activities that do not culminate. The partitive marking of the object then reflects this aspectual feature. Even in instances where the partitive A is used with an accomplishment verb (\eg ‘perform’, ‘read’, ‘do’), the object is usually in the plural partitive that signals the non-exhaustive quantity of its referent(s). This means that the overall event consists of iterated accomplishments, the number of which is unknown, and therefore the aspect may be non-culminating in two ways: either by indicating a progressive meaning (‘Some Hungarians were performing dances’ in example \REF{15-hu-ex:9} above) or by indicating a higher-level \isi{atelic} event (‘Some Hungarians performed some dances’), in which case the partitive marking of the object NP (‘dances’) in example \REF{15-hu-ex:9} means that the quantity of the dances performed by ‘some Hungarians’ was non-exhaustive. While performing one dance counts as an accomplishment, performing several dances in a row is an activity. It is also noteworthy that a singular accusative object \textit{tanssi-n} [dance-\textsc{acc}] would make example \REF{15-hu-ex:9} less acceptable with its partitive A. This suggests the generalization that the partitive A is most acceptable in low-\isi{transitivity} clauses that are aspectually non-culminating.

However, there are also instances where the partitive A is used with an accusative object, both in my data and the in data of \citet{Yli-Vakkuri1979Partitiivisubjektin}. An interesting feature of such sentences is that the event is not collective but distributive: each referent of the partitive A (which is in the plural) achieves or accomplishes something individually. Consider example \REF{15-hu-ex:23}, which is from the webpage of a newspaper. 

\ea\label{15-hu-ex:23}

\emph{(Se on vanha palkinto, joka annetaan nuorille kirjailijoille,)}

\gll ja se-n on saa-nut tosi hieno-j-a kirjailijo-i-ta.\\
and it-\textsc{acc} have.\textsc{prs}.\textsc{3sg} get-\textsc{ptcp} really fine-\textsc{pl}-\textsc{par} writer-\textsc{pl}-\textsc{par}\\
\glt (It is an old prize given to young writers,) and some really fine writers have won it.’ %(\url{http://www.ts.fi/kulttuuri/566482/Leena+Parkkinen+sai+Kalevi+Jantin+palkinnon})
\z

In \REF{15-hu-ex:23}, the pronominal object \textit{se-n} [it-\textsc{acc]} refers to a prize that each winning author has won once. Because ‘winning a prize’ is an achievement (\ie an instance of culminating aspect), the accusative object is used. It seems that the distributive reading is the factor that makes the accusative object in \REF{15-hu-ex:23} acceptable. The importance of the distributive meaning of the partitive A was also pointed out by \citet[77]{Siro1964Suomen}, who argued that a possible motivation for the use of the partitive in transitive subjects may be the avoidance of a collective interpretation which the \isi{nominative case} might evoke. All examples with an accusative object in \citegen{Yli-Vakkuri1979Partitiivisubjektin} data are likewise distributive, and attempts to form examples with a collective meaning result in ungrammaticality. Consider the coined example \REF{15-hu-ex:24} where the (ungrammatical) accusative object would indicate the culmination of a collective accomplishment. The \isi{partitive object} is better,\footnote{The question mark indicates the fact that such examples are considered ungrammatical in Standard \ili{Finnish} and may not be acceptable for all speakers.} as it indicates non-culminating (in this case, progressive) aspect. Example \REF{15-hu-ex:25} is an attested occurrence with a \isi{partitive object}. 

\ea\label{15-hu-ex:24}
\gll (?)Kirkko-a / *kirkon rakens-i kymmen-i-ä sukupolvi-a.\\
church-\textsc{par} {} *\textsc{acc} build-\textsc{pst}.\textsc{3sg} ten-\textsc{pl}-\textsc{par} generation-\textsc{pl}-\textsc{par}\\
\glt ‘Tens of generations built [participated in the building of] the church.’ (C)
\z

\ea\label{15-hu-ex:25}
\gll Kymmen-i-ä-tuhans-i-a ihmis-i-ä rakens-i tä-tä linja-a\\
ten-\textsc{pl}-\textsc{par}-thousand-\textsc{pl}-\textsc{par} person-\textsc{pl}-\textsc{par} build-\textsc{pst}.\textsc{3sg} this-\textsc{par} line-\textsc{par}\\
(ja me kävimme siis yhdessä bunkkerissa.)\\
\glt `There were tens of thousands of people building [participated in the building of] this [defense] line (and so we visited one bunker).’ %(\url{http://heavenbounding.blogspot.fi/2014/10/seurakuntamatka-pohjois-ja-etela.html})
\z

In the same vein, example \REF{15-hu-ex:25} would be odd with the object in the accusative (\textit{tämän linjan}), to indicate that the people collectively built and completed the defense line. What example \REF{15-hu-ex:25} (like the partitive version of the coined \REF{15-hu-ex:24}) means is that the quantity of people referred to by the partitive A took part in the building. In spite of the transitive structure, an existential kind of meaning is involved (‘There were tens of thousands of people who participated in the building of the defense line’). Note that in spite of its \isi{partitive object}, which often indicates a progressive meaning, example \REF{15-hu-ex:25} is not progressive in the sense of indicating a “cross-section” of an ongoing event where a non-exhaustive quantity of people are simultaneously participating. The participation of the people need not be simultaneous; the example rather means that there have been people involved in the building of the defense line at different times during its construction. In this respect, the partitive A resembles the canonical partitive S (of existentials), the reference of which may change as the event unfolds (see \citealt{Huumo2003Incremental} for details). In general, the aspectual meaning of the examples with a partitive A relates to non-culminating aspect: the events are \isi{atelic} processes, or if telic (as the examples with ‘build’), not understood as reaching their culmination. 

\section{A note on word order }
\label{15-hu-sec:4}

As far as \isi{word order} is concerned, the examples discussed so far show that the constructions with the partitive A may have an AVO (\REF{15-hu-ex:15}–\REF{15-hu-ex:18}) as well as OVA (\REF{15-hu-ex:8}, \REF{15-hu-ex:10}, \REF{15-hu-ex:13}) \isi{word order}. In lack of systematic corpus data it is impossible to say which order is more common in actual usage, or whether the \isi{word order} variants pattern around different verbs. However, it is easy to see a motivation for both patterns: \ili{Finnish} is an AVO language but has a discourse-pragmatically conditioned \isi{word order} (see \citealt[35–62]{Vilkuna1989Free}) which allows \isi{indefinite} subjects to occur in a postverbal position, not only in existential clauses but also in non-existential constructions, including transitive clauses. In actual usage, the postverbal position is typical of \isi{indefinite}, structurally heavy subject NPs that introduce a discourse-new referent (for written language, see \citealt{Huumo1995Position}; for spoken language, \citealt[75–81]{Helasvuo2001Syntax}). One can thus see two competing motivations for the \isi{word order} in transitive clauses with the partitive A: the AVO order that is typical of \ili{Finnish} transitive clauses, and the XVS order of existentials, combined with the tendency for \isi{indefinite} subjects to occur towards the end of the clause. 

Because object NPs are also commonly in the partitive in the data, ambiguity may be expected to arise: which partitive NP is the subject and which one the object? It seems, though, that real ambiguity is rare in actual usage, because in many cases the lexical meaning of the partitive A shows that it is the subject. For instance, the partitive A (but not O) is often \isi{animate} in cases where the verb selects for an \isi{animate} subject. Furthermore, if the partitive marking of the object NP unambiguously reflects non-culminating aspect, not quantity, then that NP cannot be understood as the partitive A, which follows the rules of existential S marking in that the partitive indicates non-exhaustive quantity, not aspect. In spite of these facts, there are some ambiguous instances in my data. In \REF{15-hu-ex:26} both A and O are plural partitive NPs with a human referent, and thus the example as such is ambiguous between the AVO and the OVA readings. 

\ea\label{15-hu-ex:26}
\gll Sotila-i-ta seuras-i aina huolto.joukko-j-a ja kauppia-i-ta huolto.varmuude-n ylläpitämise-ksi.\\
soldier-\textsc{pl}-\textsc{par} follow-\textsc{pst}.\textsc{3sg} always maintenance.troop-\textsc{pl}-\textsc{par} and vendor-\textsc{pl}-\textsc{par} maintenance.certainty-\textsc{gen} securing-\textsc{tra}\\
\glt `Soldiers were always followed by maintenance troops and vendors to secure the maintenance.’% %(\url{http://esahakala.blogspot.fi/2013/08/petr-sinebrjuhov-toi-perheensa.html})
\z

Even in this case, however, the context reveals that it is the maintenance troops and vendors who follow the soldiers (into conquered territories), not vice versa. The example is thus OVA. In purely grammatical terms, though, nothing would prevent the AVO reading, and in the coined, context-less example \REF{15-hu-ex:27} the AVO and OVA interpretations are equal.

\ea\label{15-hu-ex:27}
\gll Tyttö-j-ä seuras-i poik-i-a.\\
girl-\textsc{pl}-\textsc{par} follow-\textsc{pst}.\textsc{3sg} boy-\textsc{pl}-\textsc{par}\\
\glt `[Some] girls followed [some/the] boys’ / ‘[Some/the] girls were followed by [some] boys.’
\z

As the English translation of \REF{15-hu-ex:27} shows, the partitive A is always \isi{indefinite} but the partitive O may be either \isi{definite} or \isi{indefinite}. If O is understood as \isi{definite} (‘the girls’, ‘the boys’), then its partitive marking reflects the non-culminating aspect only. This is also the reason why example \REF{15-hu-ex:28} below can only be an AVO instance where \textit{häntä} ‘him/her’ is the grammatical object and not a partitive A: its \isi{partitive case} is not motivated by a non-exhaustive quantity but by non-culminating aspect.

\ea\label{15-hu-ex:28}
\gll Kymmen-i-ä, ell-ei sato-j-a sotila-i-ta seuras-i hän-tä.\\
ten-\textsc{pl}-\textsc{par} if-\textsc{neg} hundred-\textsc{pl}-\textsc{par} soldier-\textsc{pl}-\textsc{par} follow-\textsc{pst}.\textsc{3sg} \textsc{3sg}-\textsc{par}\\
\glt ‘Tens if not hundreds of soldiers followed him/her.’
\z

Another grammatical feature that relates to \isi{word order} is the lack of subject–verb plural agreement in colloquial spoken \ili{Finnish} (see \eg \citealt[67]{Helasvuo2001Syntax}) but also in nonstandard written varieties, such as Internet texts. In such varieties, the singular 3rd person verb form is used even with plural nominative subjects. In Standard \ili{Finnish}, this is considered an error – however, there is clearly a pressure from the colloquial varieties against plural agreement, and this pressure seems to be strongest in clauses where an \isi{indefinite} plural nominative subject follows the verb. According to my observations, even university students of \ili{Finnish} (who are educated to be specialists in the language) have difficulties in marking plural agreement if the nominative plural subject is \isi{indefinite} and follows the verb. Keeping in mind that the partitive S does not trigger verb agreement, it is possible (as also suggested by \citealt{DeSmit2016Fluid}) that the decay of agreement, which is clearly manifest in spoken and nonstandard written \ili{Finnish}, is another feature paving the road for the partitive marking to spread into \isi{indefinite} plural subjects even in transitive clauses. When there is no agreement even with a (post-verbal) nominative subject, then constructions with a nominative vs. a \isi{partitive subject} resemble each other in all respects except the \isi{case marking} of the subject – in other words, there is no agreement to prevent the use of the partitive. 


\section{Semi-transitive infinitival constructions}\label{15-hu-sec:5}

If looked at in isolation, transitive clauses with a partitive A may appear striking, but there are in fact a few infinitival constructions, also acceptable in Standard \ili{Finnish}, that bring the partitive S and an object NP close to being arguments of the same \isi{complex predicate}. In the (coined) example \REF{15-hu-ex:29}, the predicate verb is \isi{intransitive} and has a partitive S but also an infinitival modifier, traditionally parsed as an adverbial, consisting of a transitive verb which has its own object NP. 

\ea\label{15-hu-ex:29}
\gll Turiste-j-a saapu-u ihastele-ma-an rakennus-ta.\\
tourist-\textsc{pl}-\textsc{par} arrive-\textsc{prs}.\textsc{3sg} admire-\textsc{inf}-\textsc{ill} building-\textsc{par}\\
\glt ‘Tourists arrive to admire the building.’ (C)
\z

Example \REF{15-hu-ex:29} has an \isi{intransitive} motion verb (‘arrive’) which is quite typical in existential clauses. Therefore the partitive S is grammatical. The example also includes an infinitival form of the transitive verb ‘admire’, which in turn has a grammatical object but no subject argument of its own – the infinitive is controlled by the matrix verb in the sense that the A argument of the matrix verb is understood as the agent of the infinitive as well. In traditional grammars of \ili{Finnish}, such infinitival forms are analyzed as adverbials of the finite verbs, and since the object is part of the infinitival construction, it is not considered to be an object at the level of the matrix clause. If the relationship between the finite verb and the infinitive is relatively tight (\ie if they are understood as forming a \isi{complex predicate} where the function of the matrix verb resembles that of an auxiliary), then “almost-transitive” clauses arise where the partitive A and the O can be understood as arguments of the same \isi{complex predicate} (not of different verb forms); consider \REF{15-hu-ex:30}.

\protectedex{
\ea\label{15-hu-ex:30}
\gll Mon-i-a lahjakka-i-ta ihmis-i-ä on teke-mä-ssä ulkopolitiikka-a.\\
many-\textsc{pl}-\textsc{par} talented-\textsc{pl}-\textsc{par} person-\textsc{pl}-\textsc{par} be.\textsc{prs}.\textsc{3sg} do-\textsc{inf}-\textsc{ine} foreign.policy-\textsc{par}\\

\emph{(mutta sitä tehdään omissa lokeroissa eivätkä eri osa-alueet kohtaa.)}\\
 
\glt `There are many talented people carrying out [our] foreign policy (but they do it in their individual lockers and the different areas do not meet).’ %(\url{http://www.formin.fi/public/default.aspx?contentid=291002&contentlan=1&culture=en-US})
\z
}

In \REF{15-hu-ex:30}, the finite verb is \textit{olla} ‘be’, which, on the one hand, is the most typical existential verb, but, on the other hand, has functions as an auxiliary when it is combined with infinitival forms to form \isi{complex predicate} constructions. The infinitival form in \REF{15-hu-ex:30} is \textit{teke-mä-ssä}, the so-called 3rd infinitive inessive form of the verb meaning ‘do’ (roughly translatable as ‘in doing’). This infinitive often combines with the verb ‘be’ to form a progressive construction; \cf \REF{15-hu-ex:31}.

\ea\label{15-hu-ex:31}
\gll Ole-n luke-ma-ssa tä-tä raportti-a-si.\\
be-\textsc{prs}.\textsc{1sg} read-\textsc{inf}-\textsc{ine} this-\textsc{par} report-\textsc{par}-\textsc{2sg}.\textsc{poss}\\
\glt `I am reading this report of yours.’  (C)
\z

Though the \ili{Finnish} \textit{olla} (‘be, exist’) + the 3rd infinitive inessive (‘in-the-activity-of’) construction is not a fully grammaticalized progressive but maintains a locative-absen\-tive meaning (by implying that the agent is absent from the location of the speech event, at another location where the activity takes place; \cf \citealt{Markkanen1979Tense,Tommola2000Progressive,Onikki-Rantajsk2005Elv}), it is nevertheless a more grammaticalized combination of an existential finite verb and its transitive infinitival “adverbial” modifier than the constructions in example \REF{15-hu-ex:29}. In constructions like \REF{15-hu-ex:30}, the partitive S and the O are close to being arguments of the same predicate. \citet[165]{Yli-Vakkuri1979Partitiivisubjektin} also points out that in her data of the partitive A, many instances could alternatively be expressed by using the progressive construction, as they indicate an ongoing event. 

Note that the analogy of expressions such as \REF{15-hu-ex:30} may also for its part  explain why the \isi{partitive object} is more natural than the accusative in transitive clauses with a partitive A. The partitive O can reflect different types of non-culminating aspect, among which the progressive meaning is a typical one. Thus if progressive constructions such as \REF{15-hu-ex:30} give analogical support to the partitive A, then it is reasonable that the progressive meaning is also typical in transitive clauses with the partitive A. However, at a more general level it can be pointed out that both the partitive S and the partitive O associate with low \isi{transitivity}\footnote{However, as pointed out to me by an anonymous reviewer, it seems to be the case that not all low-\isi{transitivity} constructions accept the partitive A.} (in aspectual terms, \isi{atelic}, progressive or cessative aspect as opposed to telic predicates such as accomplishments, \cf \citealt{Huumo2010Nominal}). This may also motivate the dominance of partitive objects in clauses with the partitive A. 

\section{The role of quantifiers}\label{15-hu-sec:6}

In this section, I will take a closer look at the quantifier expressions that are typical in NPs with the function of the partitive A. Subsection \sectref{15-hu-sec:6-1} introduces and discusses different types of mass (‘a lot of’, ‘much’) and plurality (‘several’, ‘a few’) quantifiers that are common in this function, while subsection \sectref{15-hu-sec:6-2} concentrates on the singular quantifier \textit{moni} ‘many’ (+singular), and its partitive form \textit{mon-ta}, which has been reanalyzed as a nominative in many contexts and, as a consequence, given rise to the pleonastic double partitive \textit{mon-ta-a} that explicitly indicates the function of a partitive. The form \textit{monta} alternates between the functions of a nominative and a partitive and is typical in (partitive) A phrases as well. 

\subsection{Quantifiers in the partitive A phrase}\label{15-hu-sec:6-1}

A characteristic feature of phrases with the function of the partitive A is the presence of quantifying elements such as ‘several’, ‘a lot of’, as well as \isi{indefinite} numerals that are themselves in the \isi{partitive case} (‘hundreds / thousands of’). These quantifiers can be roughly divided into two groups depending on whether they are able to quantify both mass nouns and plurals (as the English \textit{a lot of coffee {\textasciitilde} a lot of cars}) or plurals only (*\textit{several coffee {\textasciitilde} several cars}). I will refer to these two groups as \textit{mass quantifiers} and \textit{plurality quantifiers}, respectively (detailed analyses [in Finnish] include \citealt{Hakulinenetal1979Nykysuomen,Huumo2017Objektia,Huumo2017S}). \ili{Finnish} plurality quantifiers, like adjectival modifiers in general, agree with their head (the quantified noun) in number and case \REF{15-hu-ex:32}, while mass quantifiers are fossilized forms not inflected in number and case \REF{15-hu-ex:33}. Both kinds of quantifiers are used in NPs with the function of a \isi{partitive subject} (S or A).

\ea%32
\label{15-hu-ex:32}

\gll Use-i-ta auto-j-a seiso-o piha-lla.\\
several-\textsc{pl}-\textsc{par} car-\textsc{pl}-\textsc{par} stand-\textsc{prs}.\textsc{3sg} yard-\textsc{ade}\\
\glt ‘There are several cars standing in the yard.’  (C)
\z

\ea%33
\label{15-hu-ex:33}

\gll Paljon auto-j-a seiso-o piha-lla.\\
a.lot.of car-\textsc{pl}-\textsc{par} stand-\textsc{prs}.\textsc{3sg} yard-\textsc{ade}\\
\glt ‘There are a lot of cars standing in the yard.’   (C)
\z

The tendency for partitive A phrases to include quantifiers was also observed by \citet{Yli-Vakkuri1979Partitiivisubjektin}. In her data of 39 examples collected from actual usage, 33 examples have a quantifying element preceding the partitive noun. Yli-Vakkuri also made a query to 103 native-speaker informants regarding the acceptability of different subtypes of clauses with a partitive A. She found out that the clear majority of the informants considered versions with a quantifier more acceptable than those with a bare (unquantified) partitive noun form. She also asked the informants to correct the sentences they considered ungrammatical. The result was, remarkably, that many informants added a quantifier but maintained the partitive marking of the quantified NP instead of changing it into the nominative \citep[175]{Yli-Vakkuri1979Partitiivisubjektin}. This raises the question about the central role of the quantifier in the partitive A phrases. 

In my data gathered with Google, quantifying elements are also common, even if they were not searched for. For example, in the hits produced by the search string “\textit{ihmisiä} \textit{seurasi}” (‘people[\textsc{par}] followed’; see the examples in \sectref{15-hu-sec:2}), most hits where \textit{ihmisiä} was a part of a partitive A phrase had some kind of a quantifying element preceding the form \textit{ihmisiä}. The search also produced hits (not targeted for) where the partitive form \textit{ihmisiä} is a post-modifier of a nominative head with a collective meaning, such as ‘group’ or ‘team’, \ie  a collective that consists of a number of persons, as in \REF{15-hu-ex:34} and \REF{15-hu-ex:35} (which of course are not instances of the partitive A).

\ea%34
\label{15-hu-ex:34}
\gll Täysi torillinen ihmis-i-ä seuras-i Valoviikko-jen avajais-i-a Tamperee-lla.\\
full market.place.full person-\textsc{pl}-\textsc{par} follow-\textsc{pst}.\textsc{3sg} ligh.week-\textsc{pl}.\textsc{gen} opening-\textsc{pl}-\textsc{par} Tampere-\textsc{ade}\\
\glt ‘A full market-place-full of people was following the openings of the Illuminations in Tampere.’ %(\url{https://www.facebook.com/aamulehti/posts/739109549436923?stream_ref=5})
\z

\ea\label{15-hu-ex:35}
\gll Suuri joukko ihmis-i-ä seuras-i Schwarzeneggeri-n ja olympiatule-n yhteis-tä matka-a.\\
big crowd person-\textsc{pl}-\textsc{par} follow-\textsc{pst}.\textsc{3sg} Schwarzenegger-\textsc{gen} and olympic.fire-\textsc{gen} joint-\textsc{par} journey-\textsc{par}\\
\glt ‘A big crowd of people followed the journey of Schwarzenegger and the Olympic Flame.’ %(\url{http://www.iltalehti.fi/kisaextra/2010021211107910_ki.shtml})
\z

In \REF{15-hu-ex:34} the head of the subject NP is the nominative form \textit{torillinen} ‘market-place-full’, derived from the noun \textit{tori} ‘market place’ to designate something that fulfills the whole market place. The partitive \textit{ihmisiä} is a post-modifier of this noun. In example \REF{15-hu-ex:35} the head noun of the subject NP, \textit{joukko} ‘crowd’, is in the nominative, and it is followed by the partitive modifier \textit{ihmisiä} ‘people’. These examples are thus not instances of the partitive A but illustrate a “legitimate” construction (from the point of view of language planning authorities) where the subject NP that contains a partitive form has the function of A. In the light of these examples, now consider \REF{15-hu-ex:36}--\REF{15-hu-ex:38}. 

\ea \label{15-hu-ex:36}
\gll Runsaa-sti ihmis-i-ä seuras-i vappupuhe-i-ta aurinkoise-lla mutta tuulise.lla kauppatori-lla.\\
abundant-\textsc{adv} person-\textsc{pl}-\textsc{par} follow-\textsc{pst}.\textsc{3sg} 1st.of.May.speech-\textsc{pl}-\textsc{par} sunny-\textsc{ade} but windy-\textsc{ade} market.square-\textsc{ade}\\
\glt ‘A lot of [lit. abundantly] people were following the 1st  of May speeches on the sunny but windy market square.’ %(\url{http://www.ilkka.fi/uutiset/maakunta/ihmisia-tulvi-seinajoen-vapputorille-katso-lisaa-kuvia-1.1187380})
\z

\protectedex{
\ea\label{15-hu-ex:37}
\emph{(Elvis Presleyn kuolema vuonna 1977 toi välittömästi yli 100 000 surijaa Gracelandin porteille, ja)}\\
\gll sama-n verra-n ihmis-i-ä seuras-i paika-n pää-llä häne-n hautajais-i-a-an.\\
same-\textsc{acc} amount-\textsc{acc} person-\textsc{pl}-\textsc{par} follow-\textsc{pst}.\textsc{3sg} spot-\textsc{gen} on-\textsc{ade} \textsc{3sg}-\textsc{gen} funeral-\textsc{pl}-\textsc{par}-\textsc{3poss}\\
\glt (Elvis Presley’s death in 1977 immediately brought over 100~000 mourners to the gates of Graceland, and) the same amount of people followed his funeral on the spot.’ %(\url{https://www.jyu.fi/kulttuurintutkimus/kolumni3_11.html})
\z
}
\ea\label{15-hu-ex:38}

\emph{(Missä hän menikin, niin)}

\gll paljon ihmis-i-ä seuras-i hän-tä.\\
a.lot.of person-\textsc{pl}-\textsc{par} follow-\textsc{pst}.\textsc{3sg} \textsc{3sg}-\textsc{par}\\
\glt (Wherever He [Christ] went), a lot of people followed Him.’ %(\url{http://www.vapis.fi/jeesus-on-silta-jumalan-luokse} (Missä hän menikin, niin))
\z

In \REF{15-hu-ex:36}--\REF{15-hu-ex:38}, the partitive form \textit{ihmisiä} is preceded by a mass quantifier which is more abstract than the collective nouns of examples \REF{15-hu-ex:34}--\REF{15-hu-ex:35}. It is not always clear whether the head of the subject phrase is the quantifier or the partitive. For example, the influential \ili{Finnish} syntax book by  \citet[147]{Hakulinenetal1979Nykysuomen} mentions both possibilities for the analysis of such phrases, as either NPs or “quantifier phrases”. However, unlike the collective nouns in \REF{15-hu-ex:34}--\REF{15-hu-ex:36}, the quantifiers in \REF{15-hu-ex:36}--\REF{15-hu-ex:38} are not referential: they do not designate a group or other kind of a collective that would be understood as the actual referent of the phrase. For instance, in \REF{15-hu-ex:36} the adverb \textit{runsaasti} ‘abundantly’ used as a mass quantifier does not refer to a group but specifies the quantity indicated by the partitive \textit{ihmisiä} ‘people’. This means that, in semantic terms at least, there are good reasons to consider the partitive-marked noun the head of the phrase. 

Morphologically, \textit{runsaasti} is derived from the adjective \textit{runsas} ‘abundant’ by adding the adverb-forming affix –\textit{sti}, in the same way as the English \textit{abundant-ly}, which is semantically close to it. The quantifier \textit{paljon} \REF{15-hu-ex:38}, in turn, is historically the accusative form of the quantifier \textit{paljo} ‘multitude’ (\cf \citealt{Tuomikoski1978Objektinsijaisista}), which has grammaticalized into an opaque quantifier and only used in its accusative form in present-day \ili{Finnish} (see \citealt{Karttunen1975Syntax} for the grammar of \textit{paljon}). Though \citet{Karttunen1975Syntax}, following \citet{Penttil1963Suomen} considers \textit{paljon} the head of the phrases such as that in \REF{15-hu-ex:38}, this element resembles \textit{runsaasti} of example \REF{15-hu-ex:36} in being a quantifier, not a noun, and there are equally good reasons to argue that the partitive form is actually the head and the phrase is an NP. The more recent comprehensive grammar \citep[§657]{Hakulinenetal2004Iso} states that such quantifiers occur “next to the NP” they quantify, hinting that the quantifiers might be external to the NP. The expression \textit{verran} in \REF{15-hu-ex:37} apparently has a similar background as \textit{paljon}: it is a grammaticalized accusative form of the noun \textit{verta} meaning ‘worth’ or ‘match’ (as in \textit{He is no match to me}). In any case, it is not referential in \REF{15-hu-ex:37}.

In sum, all subject phrases in \REF{15-hu-ex:36}--\REF{15-hu-ex:38} include mass quantifiers that are not inflected and, for instance, cannot be pluralized, unlike the collective heads proper in \REF{15-hu-ex:34} and \REF{15-hu-ex:35}, yielding \textit{jouko-t ihmis-i-ä} ‘groups of people’ (which in a subject position triggers plural agreement in the verb in Standard \ili{Finnish}). The collective nouns can also be case inflected, as in \textit{torillise-lle} [\textsc{allative}]\textit{ ihmis-i-ä} ‘to a/the market-square-full of people’, where, irrespective of the \isi{case marking} of the collective noun, the partitive postmodifier keeps its partitive in all contexts – this is another feature demonstrating that the collective noun is indeed the head. The quantifying expressions in \REF{15-hu-ex:36}--\REF{15-hu-ex:38}, in contrast, are not inflected and show no behavior of a head of a subject NP (\ie  do not trigger verb agreement). 

In terms of prescriptive grammar, transitive clauses such as \REF{15-hu-ex:34}--\REF{15-hu-ex:35} are acceptable, because the collective noun is the head of the subject NP and it is in the nominative. In contrast, examples \REF{15-hu-ex:36}--\REF{15-hu-ex:38}  have been considered ungrammatical by some language planning authorities, because they bring the \isi{partitive subject} into a transitive clause (in an analysis where the partitive is the head). However, it is easy to see a similarity between the two constructions, and it is very likely that expressions such as \REF{15-hu-ex:34} and \REF{15-hu-ex:35} serve as an analogy for the use of the partitive A with a quantifier as in \REF{15-hu-ex:36}--\REF{15-hu-ex:38}. Note, furthermore, that verb agreement does not help to distinguish the head in examples like \REF{15-hu-ex:36}--\REF{15-hu-ex:38}  in the way of the English alternation between \textit{A flock of geese is {\textasciitilde} are in the yard}, where the verb form shows whether \textit{flock} or \textit{geese} is understood as the head of the subject NP (see \citealt[53]{Langacker2009Investigations}). This is because the quantifiers in examples \REF{15-hu-ex:36}--\REF{15-hu-ex:38}  cannot be morphologically pluralized (to trigger plural agreement in the verbs; note that they do not trigger semantic plural agreement either).

On the other hand, plurality quantifiers agree with the quantified noun in number and case; see \REF{15-hu-ex:39} and \REF{15-hu-ex:40} below. 

\ea \label{15-hu-ex:39}

\emph{(Sitten huomasin, että)}

\gll minu-a tuijott-i use-i-ta silmä.pare-j-a varjo-i-sta.\\
\textsc{1sg}-\textsc{par} stare-\textsc{pst}.\textsc{3sg} several-\textsc{pl}-\textsc{par} eye.pair-\textsc{pl}-\textsc{par} shadow-\textsc{pl}-\textsc{ela}\\
\glt (Then I noticed that) I was stared at by several pairs of eyes from the shadows.’ %(\url{http://z6.invisionfree.com/Bioklaani/index.php?showtopic=1287&st=20})
\z

\protectedex{
\ea\label{15-hu-ex:40}
\emph{(vaikka näkisikin että)}

\gll sato-j-a ihmis-i-ä on luke-nut viesti-si\\
hundred-\textsc{pl}-\textsc{par} person-\textsc{pl}-\textsc{par} have.\textsc{prs}.\textsc{3sg} read-\textsc{ptcp} message-\textsc{acc}.\textsc{2sg}.\textsc{poss}\\
\\

\emph{(niin harva kuitenkaan vaivautuu vastaamaan)}

\glt `(Even though you see that) hundreds of people have read your message, (only few bother to answer you).’ %(\url{http://keskustelu.suomi24.fi/node/2139997}\\

\z
}

Like examples \REF{15-hu-ex:36}--\REF{15-hu-ex:38}, examples \REF{15-hu-ex:39} and \REF{15-hu-ex:40} include a quantifying element that precedes the partitive noun. The difference is that in \REF{15-hu-ex:39} and \REF{15-hu-ex:40} the quantifying element is a plurality quantifier and therefore agrees with the partitive-marked noun. Such NPs thus seem to be partitive subjects indisputably. However, \citet{Yli-Vakkuri1979Partitiivisubjektin} argues that in spite of the partitive of the quantifier, such phrases differ from unquantified partitive subjects which indicate a non-exhaustive quantity. The quantity indicated by phrases such as those in \REF{15-hu-ex:39} and \REF{15-hu-ex:40} are, in \citeauthor{Yli-Vakkuri1979Partitiivisubjektin}’s terms, quantitatively marked. This can be seen best by analyzing uses where such phrases have the function of a grammatical object; recall that the partitive marking of the object NP may reflect non-culminating aspect or non-exhaustive quantity in affirmative clauses. \citet{Yli-Vakkuri1979Partitiivisubjektin} demonstrates that the quantity expressed by phrases including a partitive quantifier (such as the subject NPs in \REF{15-hu-ex:39} and \REF{15-hu-ex:40}) behaves like (in the current terminology) an exhaustive quantity in certain contexts. For instance, if the phrase \textit{use-i-ta ihmis-i-ä} [several-\textsc{pl}-\textsc{par} person-\textsc{pl}-\textsc{par}] has the function of a grammatical object, it behaves, in terms of quantification, like a plural accusative object (which is morphologically in the \isi{nominative case} and indicates an exhaustive quantification), not like an unquantified partitive NP. This can be seen by considering the behavior of the durative modifiers \textit{tunni-n} [hour-\textsc{acc}] ‘for an hour’ vs. \textit{tunni-ssa} [hour-\textsc{ine}] ‘in an hour’, which, like their English counterparts, are a good test indicator for non-culminating vs. culminating aspect, respectively. Consider the following examples. 

\ea\label{15-hu-ex:41}
\gll Poim-i-n sien-i-ä tunni-n (*tunni-ssa).\\
pick-\textsc{pst}.\textsc{1sg} mushroom-\textsc{pl}-\textsc{par} hour-\textsc{acc} (*\textsc{ine})\\
\glt ‘I picked mushrooms for (*in) an hour.’  (C)
\z

\ea\label{15-hu-ex:42}
\gll Poim-i-n siene-t tunni-ssa (*tunni-n).\\
pick-\textsc{pst}.\textsc{1sg} mushroom-\textsc{pl}.\textsc{nom} hour-\textsc{ine} (*\textsc{acc})\\
\glt ‘I picked the mushrooms in (*for) an hour.’   (C)
\z

\ea\label{15-hu-ex:43}
\gll Poim-i-n use-i-ta sien-i-ä tunni-ssa (*tunni-n).\\
pick-\textsc{pst}.\textsc{1sg} several-\textsc{pl}-\textsc{par} mushroom-\textsc{pl}.\textsc{nom} hour-\textsc{ine} (*\textsc{acc})\\
\glt ‘I picked several mushrooms in (*for) an hour.’  (C)
\z

These examples all designate an iterative event of picking mushrooms, with the duration of an hour. Because the unquantified \isi{partitive object} in \REF{15-hu-ex:41} designates a non-exhaustive quantity of mushrooms, the number of the sub-events (of picking one mushroom at a time) is likewise non-exhaustive (unbounded), and the accusative-marked durative adverbial \textit{tunnin} ‘for an hour’ must be used to indicate the temporal boundaries of the event. In \REF{15-hu-ex:42} the plural accusative (syntactically accusative, morphologically nominative) object indicates an exhaustive quantity of mushrooms, which yields a bounded number of the sub-events; hence the inessive \textit{tunnissa} ‘in an hour’ must be used. Remarkably, even though both the quantifier \textit{useita} and the head \textit{sieniä} ‘mushrooms’ in \REF{15-hu-ex:43} are in the partitive, the example aligns with the accusative object in \REF{15-hu-ex:42}, not with the bare partitive in \REF{15-hu-ex:41}, by selecting the inessive durative element.\footnote{However, it deserves to be pointed out that if the partitive marking of the object NP is triggered by non-culminating aspect alone, not by non-culminating aspect based on non-exhaustive quantity, then the phrases with a partitive quantifier align with partitive objects: \textit{Heikki rakast-i \{nais-ta /*naise-n / nais-i-a / *naise-t / use-i-ta nais-i-a\}} [Heikki love-\textsc{pst}.\textsc{3sg} woman-\textsc{par} / *woman-\textsc{acc} /woman-\textsc{pl}-\textsc{par} / *woman-\textsc{pl}.\textsc{nom} / several-\textsc{pl}-\textsc{par} woman-\textsc{pl}-\textsc{par}] ‘Heikki loved [a/the] woman /[ø/the] women / several women’. Because the verb ‘love’ is \isi{atelic}, the accusative object is ungrammatical, but both the unquantified partitive (singular or plural) and the plural partitive quantified by \textit{useita} are fine. A more detailed analysis of the grammatical functions of phrases with plural partitive quantifiers must be left for future research.}As \citet{Yli-Vakkuri1979Partitiivisubjektin} points out, the (syntactic) accusative object with the plural NP \textit{usea}-\textit{t} \textit{siene}-\textit{t} [several-\textsc{pl}.\textsc{nom} mushroom-\textsc{pl}.\textsc{nom}] would indicate a more specialized meaning, \ie ‘several sets of mushrooms’, \eg for different mushroom dishes. Therefore, she argues, the case distribution (\textsc{nom}/\textsc{acc} vs. \textsc{par}) of quantified NPs differs from that of unquantified NPs. 

This is strong evidence for \citegen{Yli-Vakkuri1979Partitiivisubjektin} point that the quantity indicated by an NP with a plurality quantifier is fundamentally different from the quantity indicated by an unquantified NP. The same can be said of examples such as \REF{15-hu-ex:40}, with a plural partitive of a numeral, which can only be formed of numerals divisible by ten (‘tens of’, ‘hundreds of’, ‘thousands of’, but not for instance ‘*eights of’). In \ili{Finnish}, such expressions, when used in the function of a subject, alternate between the nominative (\eg \textit{kymmene-t ihmise-t} [ten-\textsc{pl}.\textsc{nom} person-\textsc{pl}.\textsc{nom}]) and the partitive (\eg \textit{kymmen}-\textit{i}-\textit{ä} \textit{ihmis}-\textit{i}-\textit{ä} [ten-\textsc{pl}-\textsc{par} person-\textsc{pl}-\textsc{par}]), both of which can be translated into English as \textit{tens of people}. The nominative version can mean either ‘ten sets of people' [\eg ten work teams] or, more vaguely, ‘several sets of (ten) people’, in which case the opposition between the partitive and the nominative is neutralized, as both expressions are vague as to how many such sets they refer to. 

When such a phrase is used as the subject of a transitive clause in Standard \ili{Finnish}, it would be expected to be in the nominative. However, as the data of \citet{Yli-Vakkuri1979Partitiivisubjektin} and this study suggest, in unedited texts at least, the plural partitive numeral is quite common and acceptable. According to \citet{Yli-Vakkuri1979Partitiivisubjektin}, one motivation for the expansion of the partitive in this construction is the fact that the nominative might imply a too specific interpretation for the quantified partitive noun (\eg ‘tens of \textit{the} people’, or (specifically) ‘ten \textit{sets} of people’), which is not intended. Thus the partitive quantifier may be gaining ground in uses where the nominative would indicate too specific meanings. As in example \REF{15-hu-ex:43}, the partitive plural numeral also indicates a \isi{definite} quantity when used as the object in iterative expressions; consider \REF{15-hu-ex:44}. 

\ea%44
\label{15-hu-ex:44}

\gll Poim-i-n kymmen-i-ä sien-i-ä tunni-ssa (*tunni-n).\\
pick-\textsc{pst}-\textsc{1sg} ten-\textsc{pl}-\textsc{par} mushroom-\textsc{pl}-\textsc{par} hour-\textsc{ine} (*\textsc{acc})\\
\glt ‘I picked tens of mushrooms in (*for) an hour.’  (C)
\z



In semantic terms, the grammatical behavior of the phrases with partitive-marked quantifiers thus suggests that they designate a \isi{definite} quantity. Like uninflected, fossilized mass quantifiers such as \textit{paljon} ‘a lot of’ \REF{15-hu-ex:38} or \textit{runsaasti} ‘abundantly’ \REF{15-hu-ex:36}, plural partitive quantifiers suffice to quantify the partitive-marked noun. For instance, in \REF{15-hu-ex:44} this means that there are an \isi{indefinite} number of higher-order quantities that consist of ten mushrooms each. This, perhaps surprisingly, yields a bounded quantity of the mushrooms, even though the plural partitive \textit{kymmeniä} ‘tens (of)’ would suggest that the number of such quantities (with ten mushrooms in each) is unbounded. One might in fact say the same of the English translation of \REF{15-hu-ex:44}: the expression \textit{tens of mushrooms} literally indicates an \isi{indefinite} number of quantities of ten mushrooms. Likewise in English, though, the durative modifier must be of the type \textit{in an hour}, not \textit{for an hour}. In sum, there are good reasons to concur with \citegen{Yli-Vakkuri1979Partitiivisubjektin} argument that the overwhelmingly most common kind of phrase used as a partitive A, that is, an NP with a quantifying element preceding the partitive-marked noun, is fundamentally different from a bare partitive form in terms of quantification. 

It is also worth pointing out that if such a quantifier is added to one of the partitive NPs in the ambiguous example \REF{15-hu-ex:29} ‘girls\textsc{\textsubscript{[par]}} followed boys\textsc{\textsubscript{[par]}}’, then a strong inclination arises to understand the quantified phrase as the subject, even though in principle it could still be the object as well. Consider the following examples.

\ea%45
\label{15-hu-ex:45}

\gll Tyttö-j-ä seuras-i use-i-ta poik-i-a.\\
girl-\textsc{pl}-\textsc{par} follow-\textsc{pst}.\textsc{3sg} several-\textsc{pl}-\textsc{par} boy-\textsc{pl}-\textsc{par}\\
\glt The girls were followed by several boys’ / ?? ’[Some] girls followed several boys.’  (C)
\z

\ea%46
\label{15-hu-ex:46}

\gll Kymmen-i-ä tyttö-j-ä seuras-i poik-ia.\\
ten-\textsc{pl}-\textsc{par} girl-\textsc{pl}-\textsc{par} follow-\textsc{pst}.\textsc{3sg} boy-\textsc{pl}-\textsc{par}\\
\glt ‘Tens of girls followed the boys’. / ??’Tens of girls were followed by boys.’  (C)
\z

Furthermore, the quantifier \textit{paljon} ‘a lot’ in fact makes this test sentence unambiguous, because it cannot quantify the object of an \isi{atelic} verb (see \citealt{Karttunen1975Syntax}), and thus the \textit{paljon} phrase must be the subject in \REF{15-hu-ex:47}.

\ea%47
\label{15-hu-ex:47}

\gll Tyttö-j-ä seuras-i paljon poik-i-a\\
girl-\textsc{pl}-\textsc{par} follow-\textsc{pst}.\textsc{3sg} a.lot.of boy-\textsc{pl}-\textsc{par}\\
\glt ‘The girls were followed by a lot of boys.’  (C)
\z

Such effects disappear and the ambiguity returns if both phrases include a quantifier:

\ea%48
\label{15-hu-ex:48}

\gll Kymmen-i-ä tyttö-j-ä seuras-i use-i-ta poik-i-a.\\
ten-\textsc{pl}-\textsc{par} girl-\textsc{pl}-\textsc{par} follow-\textsc{pst}.\textsc{3sg} several-\textsc{pl}-\textsc{par} boy-\textsc{pl}-\textsc{par}\\
\glt Tens of girls were followed by several boys.’ / ’Tens of girls followed several boys.’  (C)
\z

However, such combinations seem to be extremely rare in actual usage. In \citeauthor{Yli-Vakkuri1979Partitiivisubjektin} data, there is not a single instance of the type illustrated by \REF{15-hu-ex:48}, and I have not been able to find such hits with my searches either. As most partitive A phrases include quantifiers, and most object phrases do not, this suggests that the system nevertheless rather successfully keeps the A and O grammatically apart in the majority of cases. 

\subsection{The problematic \textit{monta} ‘many[\textsc{par}?]’}
\label{15-hu-sec:6-2}

Among the quantifying expressions commonly used in partitive A phrases, the form \textit{mon}-\textit{ta} [many-\textsc{par}] ‘many’ has an especially interesting role (see also \citealt{Huumo2017Moni}). First of all, it is (historically) a singular partitive form of the quantifier \textit{moni} ‘many’, and the element it quantifies is likewise in the singular partitive, not in the plural like most partitive A phrases. The nominative form \textit{moni} modifies a singular nominative head, but it has a more specific (‘many of the’) type of meaning, \eg \textit{moni} \textit{mies} [many.\textsc{sg}.\textsc{nom} man.\textsc{sg}.\textsc{nom}], \cf the English \textit{many} \textit{a} \textit{man}. For this quantifier, the form \textit{mon}-\textit{ta}, in spite of its \isi{partitive case}, has been generalized to many uses where it has a function similar to the nominative form of cardinal numerals. In spite of this, the earlier literature on partitive A (until \citealt{Branch2001Montaa}) has treated \textit{monta} expressions as partitive phrases, without paying attention to their special nature. 

To grasp the idiosyncratic nature of \textit{monta} phrases, consider first the use of cardinal numerals in \ili{Finnish}. \ili{Finnish} cardinal numerals in the nominative combine with a singular partitive noun that indicates the quantified entity type, \eg \textit{viisi} \textit{mies}-\textit{tä} [five.\textsc{nom} man-\textsc{sg}.\textsc{par}] ‘five men’. In other case forms, however, the quantified noun and the numeral carry the same case. The numeral can also occur in the partitive if used for instance in the function of a \isi{partitive object};\footnote{To indicate non-culminating aspect or negative polarity – note that the quantity indicated by the numeral phrase is of the exhaustive type, which is why the partitive marking cannot be motivated by non-exhaustive quantity.} consider example \REF{15-hu-ex:49}.

\ea%49
\label{15-hu-ex:49}

\gll Heikki rakasta-a kolme-a nais-ta.\\
Heikki love-\textsc{prs}.\textsc{3sg} three-\textsc{par} woman-\textsc{sg}.\textsc{par}\\
\glt ‘Heikki loves three women.’  (C)
\z

If the numeral is in the nominative, it is analyzed as the head by grammars, and the quantified partitive form as a post-modifier \REF{15-hu-ex:50}. However, in other case forms the numeral agrees with the quantified noun (like an adjectival modifier), which is why the quantified noun is then considered the head; \cf example \REF{15-hu-ex:51} where the possessor NP is marked with the adessive.

\ea%50
\label{15-hu-ex:50}

\gll Viisi mies-tä saapu-i.\\
five.\textsc{nom} man-\textsc{sg}.\textsc{par} arrive-\textsc{pst}.\textsc{3sg}\\
\glt `Five men arrived.’  (C)
\z

\ea%51
\label{15-hu-ex:51}

\gll Viide-llä miehe-llä on flunssa.\\
five-\textsc{ade} man-\textsc{sg}.\textsc{ade} is.\textsc{prs}.\textsc{3sg} flu.\textsc{nom}\\
\glt ‘Five men have the flu.’  (C)
\z

As I pointed out above, the form \textit{monta}, though morphologically a partitive, behaves in many contexts like the nominative (not partitive) form of a numeral \citep{Branch2001Montaa}; consider \REF{15-hu-ex:52} and \REF{15-hu-ex:53}. 

\ea%52
\label{15-hu-ex:52}

\gll Mon-ta mies-tä saapu-i.\\
many-\textsc{par} man-\textsc{sg}.\textsc{par} arrive-\textsc{pst}.\textsc{3sg}\\
\glt `Many men arrived.’  (C)
\z

\ea%53
\label{15-hu-ex:53}

\gll Viisi (*viit-tä) mies-tä saapu-i.\\
five.\textsc{nom} (*\textsc{par}) man-\textsc{sg}.\textsc{par} arrive-\textsc{pst}.\textsc{3sg}\\
\glt `Five men arrived.’  (C)
\z

It is in examples like \REF{15-hu-ex:52} that the partitive form \textit{monta} behaves like the nominative form of a numeral \REF{15-hu-ex:53}. In principle, the nominative \textit{moni} \textit{mies} [many.\textsc{nom} man.\textsc{nom]} would be expected, but as \citet{Yli-Vakkuri1979Partitiivisubjektin} and \citet{Branch2001Montaa} point out, it would easily be understood as meaning ‘many of the men’ [\ie  some members of a \isi{definite} set] or the idiomatic ‘many a man’. Note that the subject NP in \REF{15-hu-ex:52} is not functionally similar to a \isi{partitive subject} proper, as singular count nouns cannot be used in this function (see examples \REF{15-hu-ex:1}--\REF{15-hu-ex:3}). Example \REF{15-hu-ex:53} shows that numerals must take the nominative in such a context. 

Since \textit{mon}-\textit{ta}, in spite of its partitive ending, is functionally similar to the nominative of numerals, the pleonastic “double partitive” form \textit{mon}-\textit{ta}-\textit{a} [many-\textsc{par}-\textsc{par}] has arisen to explicitly indicate the partitive meaning. Like the partitive of the numeral ‘five’ in \REF{15-hu-ex:53}, the form \textit{montaa} would be ungrammatical in \REF{15-hu-ex:52}. \textit{Montaa} is used in contexts where numerals are likewise in the partitive, e.g., in the functions of aspectually partitive-marked or negative-polarity partitive objects. It is in a grammatical opposition with the “nominativized” \textit{monta} in contexts where aspect can alternatively be understood as culminating or not culminating; consider \REF{15-hu-ex:54} (with a nominative numeral or \textit{monta}) vs. \REF{15-hu-ex:55} (with a partitive numeral or \textit{montaa}).

\ea%54
\label{15-hu-ex:54}

\gll Ole-n luke-nut mon-ta {\textasciitilde} kaksi kirja-a.\\
have-\textsc{prs}.\textsc{1sg} read-\textsc{ptcp} many-\textsc{par} {\textasciitilde} two.\textsc{nom} book-\textsc{sg}.\textsc{par}\\
\glt ‘I have read many {\textasciitilde} two books [completely].’  (C)
\z

\ea%55
\label{15-hu-ex:55}

\gll Ole-n luke-nut mon-ta-a {\textasciitilde} kah-ta kirja-a.\\
have-\textsc{prs}.\textsc{1sg} read-\textsc{prtc} many-\textsc{par}-\textsc{par} {\textasciitilde} two-\textsc{par} book-\textsc{sg}.\textsc{par}\\
\glt ‘I have read many {\textasciitilde} two books [not completely]’; ‘I have been reading many {\textasciitilde} two books.’   (C)
\z



In example \REF{15-hu-ex:54}, the form \textit{monta}, like the nominative numeral \textit{kaksi} ‘two’, indicates a culminating aspect: the books have been read completely. Functionally they thus resemble the accusative object. In \REF{15-hu-ex:55}, on the other hand, the form \textit{montaa}, as well as the partitive \textit{kahta}, indicate that the reading is either ongoing or that it has not (yet) concerned the whole books. 

Until the mid-1990’s, the pleonastic \textit{montaa} was considered an error by language planning authorities, but in 1995 it was accepted in contexts such as \REF{15-hu-ex:55}, where the partitivity needs to be explicitly indicated \citep{Lansimaki1995Montaa,Nyman2000Naekymaetoen,Branch2001Montaa}. However, if the aspect is unambiguously of the non-culminating type, then even \textit{monta} can still have the function similar to that of a partitive numeral \REF{15-hu-ex:56};\footnotetext{In a Google search (13.11.2014), the string \textit{rakastaa} \textit{mon}-\textit{ta}-\textit{a} [love.\textsc{prs}.\textsc{3sg} many-\textsc{par}-\textsc{par}] produced over 5000 hits, while \textit{rakastaa} \textit{mon}-\textit{ta} [many-\textsc{par}] produced slightly more than 1000 hits. Though such numbers must be taken with great caution, this might suggest that in Internet language, the double partitive is more common (as expected), but both forms are nevertheless used in the function of the \isi{partitive object} of the \isi{atelic} verb \textit{rakastaa} ‘love’ (which does not take an accusative object outside some resultative constructions such as ‘She loved him crazy’).} \cf \REF{15-hu-ex:57} with a numeral proper.

\ea\label{15-hu-ex:56}
\gll Eemeli rakasta-a mon-ta(-a) nais-ta.\\
 Eemeli love-\textsc{prs}.\textsc{3sg} many-\textsc{par}(-\textsc{par}) woman-\textsc{sg}.\textsc{par}\\
\glt ‘Eemeli loves many women.’
\z

\ea\label{15-hu-ex:57}
\gll Eemeli rakasta-a kah-ta (*kaksi) nais-ta.\\
Eemeli love-\textsc{prs}.\textsc{3sg} two-\textsc{par} (*\textsc{nom}) woman-\textsc{sg}.\textsc{par}\\
\glt ‘Eemeli loves two women.’
\z


In \REF{15-hu-ex:56}, both \textit{monta} and \textit{montaa} are fine in the function of the \isi{partitive object} of the \isi{atelic} verb \textit{rakastaa}. This shows that \textit{monta} has not completely lost its ability to be a functional partitive, if the context unambiguously assigns such a function to it. Example \REF{15-hu-ex:57} shows that the nominative form of the numeral \textit{kaksi} ‘two’ is not possible in this context. 

What relates this lengthy discussion of \textit{monta} with the partitive A is the fact that \textit{monta} phrases quite frequently occur as transitive subjects, as in examples \REF{15-hu-ex:58} and \REF{15-hu-ex:59} below. 

\ea\label{15-hu-ex:58}
\gll Minu-a katsel-i mon-ta utelias-ta silmä-ä.\\
\textsc{1sg}-\textsc{par} watch-\textsc{pst}.\textsc{3sg} many-\textsc{par} curious-\textsc{sg}.\textsc{par} eye-\textsc{sg}.\textsc{par}\\
\glt ‘I was watched by many curious eyes.’ %(\url{http://rovaniemennuva.blogspot.fi/2011/04/istumista-kerrakseen.html})
\z

\ea\label{15-hu-ex:59}
\gll Mon-ta sukupolve-a rakens-i kirkko-a\\
many-\textsc{par} generation-\textsc{sg}.\textsc{par} build-\textsc{pst}.\textsc{3sg} church-\textsc{par}\\

\emph{(näkemättä sitä valmiina)}\\
\glt `Many generations were (= participated in) building the church (without seeing it finished).’ %(\url{http://www.uusielama.net/images/ue7_08.pdf})
\z

\citet{Branch2001Montaa} reports that such uses of \textit{monta} phrases in the function of A were already discussed by linguists at the end of the 19th century, which shows that its reanalysis as a nominative may have been going on for a relatively long time. Such a quantifier which is formally a partitive but functionally a nominative is probably another factor paving the road for quantified partitive phrases to spread into the function of A. Because \textit{monta} is functionally a nominative, I do not consider examples such as \REF{15-hu-ex:58} and \REF{15-hu-ex:59} as instances of the partitive A proper. However, their existence must be taken into account as a factor supporting the partitive A. 

The constraint discussed in \sectref{15-hu-sec:3}, stating that the clause with a partitive A cannot denote a collective accomplishment, seems to hold for \textit{monta} subjects as well. Thus \REF{15-hu-ex:60}, with its accusative object, is understood in the distributive sense where \textit{monta} ‘many’ has a wide scope over the \isi{indefinite} object ‘house’, \ie  that each person has built their own house, whereas \REF{15-hu-ex:61}, with the nominative numeral \textit{sata} ‘hundred’ has both a collective and a distributive interpretation. 

\ea\label{15-hu-ex:60}
\gll Mon-ta ihmis-tä on rakenta-nut talo-n.\\
many-\textsc{par} person-\textsc{sg}.\textsc{par} have.\textsc{prs}.\textsc{3sg} build-\textsc{prtc} house-\textsc{acc}\\
\glt ‘Many people have built a house [each their own].’ 
\z

\ea\label{15-hu-ex:61}
\gll Sata ihmis-tä on rakenta-nut talo-n.\\
hundred person-\textsc{sg}.\textsc{par} have.\textsc{prs}.\textsc{3sg} build-\textsc{prtc} house-\textsc{acc}\\
\glt ‘A hundred people have built a/the house [together or each their own].’  (C)
\z

The pleonastic partitive \textit{montaa}, like partitive forms of (singular) numerals, cannot occur in the function of the partitive A. Because it is a singular partitive form, its use in existentials is restricted to contexts where it quantifies a mass noun, which must then be understood in a special sense (‘many kinds of a substance’); \cf \REF{15-hu-ex:62}. In contrast, the forms with \textit{monta}, as well as nominative numerals, are quite typical in existential S argument NPs \REF{15-hu-ex:63}. 

\ea\label{15-hu-ex:62}
\gll Tä-ssä on mon-ta-a {\textasciitilde} viit-tä kahvi-a.\\
here-\textsc{ine} be.\textsc{prs}.\textsc{3sg} many-\textsc{par}-\textsc{par} {\textasciitilde} five-\textsc{par} coffee-\textsc{sg}.\textsc{par}\\
\glt ‘Here is coffee of many {\textasciitilde} five kinds.’  (C)
\z

\ea%63
\label{15-hu-ex:63}

\gll Tä-ssä on mon-ta {\textasciitilde} viisi kahvi-a.\\
here-\textsc{ine} be.\textsc{prs}.\textsc{3sg} many-\textsc{par}  {\textasciitilde} five.\textsc{nom} coffee-\textsc{sg}.\textsc{par}\\
\glt ‘Here are many {\textasciitilde} five [portions of] coffee.’ (C)
\z

Summing up, in addition to the infinitival constructions discussed in \sectref{15-hu-sec:5}, different quantifier phrases “pave the road” for the partitive-marked NP to spread into transitive clauses. A special case of this is the quantifier \textit{monta} ‘many’, which is formally a singular partitive but has the function of a nominative numeral. However, other quantifying expressions in the plural likewise serve as an analogy to the transitive constructions with the partitive A.

\subsection{Quantifiers: interim summary}
\label{15-hu-sec:6-3}

Quantifying expressions turned out to be common in the occurrences of the partitive A I collected for this study, which suggests that they may play an important role in the spread of partitive NPs into the function of A. The study has demonstrated that both mass (‘a lot of’) and plurality (‘several’) types of quantifiers are in use. In more general terms, \ili{Finnish} partitive NPs with quantifiers seem to have an intermediate status between nominative phrases indicating exhaustive quantification and (unquantified) partitive phrases indicating non-exhaustive quantification. 
This is clearest if we consider the use of such phrases as grammatical objects (\cf \sectref{15-hu-sec:6-1}): partitive NPs with quantifiers behave like accusative (not partitive) objects with respect to the modification of duration by selecting durative modifiers of the type ‘in an hour’ (inessive-marked in \ili{Finnish}). 
On the other hand, the nominative forms of many plurality quantifiers have acquired more specific quantificational meanings (\eg ‘many of the’ or ‘several sets of’) which clearly restrict their use and make the partitive the unmarked option in many contexts. 

The partitive-marked quantifier that has developed furthest in this direction is \textit{monta} (‘many’), which functionally behaves like a nominative of a cardinal numeral. However, other partitive-marked (plurality) quantifiers may be following this path by replacing the nominative in some contexts. When taking on these functions typical of nominative (or accusative, in object marking) NPs, the quantified partitive phrases themselves undergo a functional transition and become more similar to nominative/accusative than unquantified partitive NPs. 

\section{Conclusions}
\label{15-hu-sec:7}

As has become evident in this study, it is difficult to obtain data of the partitive A, which seems to be a rare phenomenon in general, and occurs most typically in registers of unedited written language. Though considered an error by language planning authorities, the partitive A is used at least occasionally, and the examples I have collected, as well as those analyzed by \citet{Yli-Vakkuri1979Partitiivisubjektin}, do not sound ungrammatical to the native speaker’s ear. It seems that the uses of the partitive A concentrate around \isi{atelic} expressions of low \isi{transitivity}. This semantic feature partially explains why the object NP is also in the partitive in most cases. Accusative objects seem to be in minority, and if used, they are understood in a distributive sense where each referent of the partitive A (which is practically always in the plural) performs the activity individually. The partitive A seems to be clearly ungrammatical with the accusative object indicating a collective accomplishment.

I have also proposed that there are some grammatical subsystems and constructions that, figuratively speaking, pave the road for the partitive marking to spread into the subject of transitive clauses: 1) decay of verb agreement in clauses with an \isi{indefinite}, clause-final plural subject (\cf also \citealt{DeSmit2016Fluid}); 2) constructions that combine an \isi{intransitive} finite verb with a transitive infinitive “adverbial”, such as the progressive ‘be doing’ construction, and 3) the system of quantifying expressions where even partitive-marked quantifiers such as \textit{use}-\textit{i}-\textit{ta} [several-\textsc{pl}-\textsc{par}] ‘several’ or \textit{sato}-\textit{j}-\textit{a} [hundred-\textsc{pl}-\textsc{par}] ‘hundreds of’ indicate a \isi{definite} quantity. This supports \citegen{Yli-Vakkuri1979Partitiivisubjektin} argument that a typical partitive A is not quantitatively non-exhaustive in the way a bare \isi{partitive subject} is. Furthermore, the nominative forms of these quantifying expressions, which have been recommended by language planning authorities to be used instead of the partitive, have gained narrower \isi{definite} meanings and thus might evoke implications the speaker does not wish to convey. If such semantic oppositions conventionalize, then the partitive form of such quantifiers may be developing into an unmarked indicator of an \isi{indefinite} subject.

In sum, the observations suggest that there is a pressure to mark \isi{indefinite} plural subjects with the partitive not only in existential clauses (which are \isi{intransitive}) but also in some transitive clauses, \ie those that indicate an aspectually non-culminating, low-\isi{transitivity} event. If existential clauses are considered a subtype of \isi{intransitive} clauses,\footnote{In the \ili{Finnish} tradition, existentials are usually treated apart from both \isi{intransitive} and transitive clauses, which share many features such as the nominative subject, SV/AV \isi{word order}, and subject–verb agreement.} then it can be generalized that among \isi{intransitive} clauses the partitive marking concerns S arguments that are \isi{indefinite} and indicate non-exhaustive quantification of a discourse-new referent (a substance or a multiplicity). Such an option has been missing from the marking of the A argument in Standard \ili{Finnish}, even though A arguments can likewise indicate discourse-new multiplicities (as the English \isi{indefinite} plural in \textit{Several bystanders witnessed the accident}). This may result in an analogical motivation for a similar system of case oppositions to arise in the marking of A arguments (\cf \citealt[336–338]{Serzant2013Rise}).

The \ili{Finnish} partitive A fulfills the definition of differential argument (subject) marking presented by \citetv{Witzlacketal2017Differential}. Their broad definition (\cf also \citealt{Woolford2008Differential}) states that DAM is “any kind of situation where an argument of a predicate bearing the same generalized \isi{semantic role} (or macrorole) may be coded in different ways, depending on factors other than the argument role itself”. The narrow definition they provide states that DAM is “any kind of situation where an argument of a predicate bearing the same generalized \isi{semantic role} (or macrorole) may be coded in different ways, depending on factors other than the argument role itself and/or the clausal properties of the predicate such as polarity, TAM, embeddedness, etc.” The \ili{Finnish} partitive A (and obviously also partitive S) seems to fit both definitions. The partitive marking of the S argument, and (as the data discussed in the present paper show) sometimes even the A argument, typically concerns plural forms that are \isi{indefinite} in two ways (as already argued by \citealt{Siro1957Suomen}): 1) in the notional sense (= they have a discourse-new referent) and 2) in the quantitative sense (= they indicate a non-exhaustive quantity). However, since the presence of a quantifier, which is often partitive-marked itself, seems to be common in NPs with the function of the partitive A, feature 2 seems to concern only a minority of the instances. Considering the potential motivations for a DAM system listed by \citet{Witzlacketal2017Differential}, the \ili{Finnish} Partitive A includes features of both argument-triggered DAM (it concerns \isi{indefinite} discourse-new plurals) and predicate-triggered DAM (it concerns certain low-\isi{transitivity} verbs, especially verbs of perception as well as verbs that indicate a locative arrangement such as ‘follow’ or ‘surround’). 

Occasional uses of the partitive as a marker of the transitive subject have been pointed out in the literature for over a hundred years. In lack of statistical data and a comparable set of unedited written language from an earlier era, it is difficult to say whether this indicates an ongoing change in the marking of the transitive subject. However, as \citegen{DeSmit2016Fluid} analysis demonstrates, the nominative has been in use in Old \ili{Finnish} as the case of plural existential S arguments which would take the partitive in present-day \ili{Finnish}. This suggests that the partitive has been expanding as a marker of the existential S in \isi{intransitive} clauses during the last few centuries, and there may thus be a tendency to continue its expansion into transitive clauses to mark plural \isi{indefinite} subjects as well. 

\section*{Acknowledgements}
This research was funded by the Academy of Finland (Project 285739) and the Finnish Cultural Foundation (Grant 00152335). I thank Ilja Seržant for first suggesting this topic to me, and for encouragement and feedback at  different stages of writing this paper. 

\section*{Abbreviations}
\begin{tabularx}{.45\textwidth}{lQ}
\textsc{1} & first person\\
\textsc{2} & second person\\
\textsc{3} & third person\\
\textsc{acc} & accusative\\
\textsc{ade} & adessive\\
\textsc{all} & allative\\
\textsc{dat} & dative\\
\textsc{dom} & differential object marking\\
\textsc{ela} & elative\\
\textsc{gen} & genitive\\ 
\textsc{ill} & illative\\
\end{tabularx}
\begin{tabularx}{.45\textwidth}{lQ}
\textsc{ine} & inessive\\
\textsc{inf} & infinitive\\
\textsc{neg} & negation, negative\\
\textsc{nom} & nominative\\
\textsc{par} & partitive \\
\textsc{pl} & plural\\
\textsc{poss} & possessive\\
\textsc{prs} & present\\
\textsc{ptcp} & participle \\
\textsc{pst} & past\\
\textsc{sg} & singular\\
\end{tabularx}

{\sloppy
\printbibliography[heading=subbibliography,notkeyword=this] }
\end{document}
