\documentclass[output=paper]{LSP/langsci}
\ChapterDOI{10.5281/zenodo.1228257} 
\author{Marco García García\affiliation{University of Cologne}}
\title{Nominal and verbal parameters in the diachrony of differential object marking in Spanish}
\shorttitlerunninghead{Nominal and verbal parameters in the diachrony of DOM in Spanish}
\abstract{This paper deals with the influence that nominal and verbal parameters have on DOM in the diachrony of Spanish. Comparing selected corpus studies, I will focus first on the different nominal parameters that build up the animacy and referentiality scales, in particular on animacy and definiteness. In order to clarify how far DOM has diachronically evolved, special attention will be paid to inanimate objects, which can be viewed as the alleged endpoint in the development of DOM in Spanish. Secondly, I will provide a systematic overview of the relevant verbal parameters, which include aspect, affectedness and agentivity. The study will show a complex interaction of nominal and verbal parameters, revealing some unexpected correlations: Obligatory object marking is not only found with human and strongly affected objects involved in a telic event, but also with inanimate, non-affected and agentive objects embedded in a stative event. In other words, in Spanish DOM patterns with both extremely high and extremely low transitivity. These findings sharply contrast with traditional accounts concerning the development as well as the explanation of DOM. 

% \keywords{Differential Object Marking - diachronic change - animacy - definiteness- agentivity - affectedness - aspect - telicity - perfectivity}
}
\maketitle

\begin{document}
\section{Introduction}
\label{08-sec:1}

Differential object marking (DOM) is a well-attested phenomenon within Romance languages (for an overview see \citealt[218--230]{Bossong1998Marquage}). While in some Romance languages, such as \ili{Catalan} or Modern \ili{Portuguese}, DOM is confined to a reduced number of contexts, in others, such as \ili{Sardinian} or \ili{Spanish}, it is found in many more contexts. This paper will focus on \ili{Spanish}, where DOM seems to have reached a greater stage of development than in any other Romance language.

As in most of the other Romance languages, DOM in \ili{Spanish} is signaled by \textit{a}, which goes back to the \ili{Latin} preposition \textit{ad} ‘to’. From its beginnings as a preposition with an exclusively locative-directional meaning, this preposition was firstly grammaticalized into a marker for indirect objects, \ie datives. However, even in early Hispano-Romance, the \textit{a}-marker was already regularly used not only with indirect objects, but also with certain direct objects, in particular with those showing typical dative properties such as strong personal pronouns referring to humans (\cf \citealt[184--185]{Pensado1995Creacion} and \citealt[205]{Company2002Grammaticalization}). Since then, DOM is reported to have evolved gradually along both the \isi{definiteness} and the \isi{animacy} scales (\cf \eg \citealt[470--471]{Aissen2003Differential}). DOM in \ili{Spanish} is said to depend not only on nominal parameters such as \isi{animacy} and \isi{definiteness}, but also on certain verbal parameters such as \isi{telicity} and \isi{affectedness} (\cf \citealt[1784--1791]{Torrego1999Gramatica} among others). Despite the vast literature, which mainly focuses on nominal parameters, there are still several core questions that remain open. To begin with, it is not clear which of the verbal parameters are the most important for the (diachronic) distribution of DOM in \ili{Spanish}. Moreover, it is not obvious how verbal parameters such as \isi{telicity} interact with nominal parameters such as \isi{animacy}. Lastly, there are quite different views about how far DOM in \ili{Spanish} has evolved.

The main purpose of this paper is to give an overview of the current state of research dealing with these questions. Firstly, I will critically review and compare several corpus studies in order to clarify how far DOM in \ili{Spanish} has actually developed. To this end, I will concentrate on nominal parameters such as \isi{animacy} and \isi{definiteness}. Particular attention will be paid to \isi{inanimate} objects. These can be seen as the alleged endpoint in the evolution of DOM in \ili{Spanish}. According to \citet[147]{Company2002Avance}, at least Mexican \ili{Spanish} is strongly heading towards a complete generalization of object marking not only for animates, but also for \isi{inanimate} objects. This raises the question of whether \ili{Spanish} is typologically shifting from a language with DOM to a language without DOM, \ie to a grammatical system with a sort of a regular accusative \isi{case marking}. Secondly, I will provide a systematic overview of the less well-studied verbal parameters associated with DOM, which include aspect (\isi{telicity} and perfectivity), \isi{affectedness} and \isi{agentivity}. As far as \isi{agentivity} is concerned, I will build on my previous analyses for Modern \ili{Spanish} \citep{GarciaGarcia2014Objektmarkierung} and extend them to a diachronic perspective providing a test corpus study for the reversible verbs \textit{seguir} ‘to follow’ and \textit{preceder} ‘to precede’ (\cf \sectref{08-ga-sec:4.3.2}).

The paper is organized as follows: \sectref{08-ga-sec:2} introduces the main conditions determining DOM in Modern \ili{Spanish} as well as its description by means of the \isi{animacy scale} and the \isi{definiteness} scale. \sectref{08-ga-sec:3} explores the \isi{diachrony} of DOM in \ili{Spanish} along these scales on the basis of \citet{Laca2006Objeto} and a number of other corpus studies. \sectref{08-ga-sec:4} focuses on the aforementioned verbal parameters (aspect, \isi{affectedness}, \isi{agentivity}) and elaborates on their complex interaction with nominal parameters. \sectref{08-ga-sec:5} summarizes and discusses the main findings.

\section{Prominence scales and diachronic DOM} \label{08-ga-sec:2}

DOM in \ili{Spanish} is reported to depend first of all on what \citet{Laca2002Gramaticalizacion,Laca2006Objeto} calls \textit{local factors}, \ie \isi{animacy}, \isi{definiteness} and \isi{referentiality}. Besides this, the distribution of \textit{a}-marking also seems to be influenced by what \citet[429--432; 454--462]{Laca2006Objeto} labels \textit{global factors}, \ie different kinds of contextual conditions, such as \isi{topicality} and certain verbal parameters.\footnote{Note that \citet{Laca2006Objeto} does not use the terms local and global in the typological sense of \citet{Silverstein1976Hierarchy}, also followed by \citetv{Witzlacketal2017Differential}. Thus, her notions are not associated with the distinction between languages where differential object marking is local in the sense that it only depends on the semantic properties of the object (\eg \isi{animacy}), and languages where the marking is rather global, \ie where it also depends on the properties of another co-argument such as the \isi{animacy} of the subject. The question of whether DOM in \ili{Spanish} is local or rather global in the sense of \citet{Silverstein1976Hierarchy} is not explicitly addressed in this paper. See, however, \sectref{08-ga-sec:4.3} focussing on the relative \isi{agentivity} of the subject with respect to the object, as well as \citet[40--43, 76--81]{GarciaGarcia2014Objektmarkierung}, which deals with the relative \isi{animacy} of subject and object.} However, these are usually seen as additional, \ie less important conditions, at least from a synchronic perspective. As for Standard Modern \ili{Spanish}, it is generally assumed that DOM is confined to human or at least \isi{animate} (non-human) referents (\cf \eg \citealt[1782]{Torrego1999Gramatica}). For \isi{definite} human objects, \textit{a}-marking is more or less obligatory (\cf~\REF{08-ga-ex:1a}), while for \isi{indefinite} human objects there is more variation. Generally, \textit{a}-marking is required for \isi{indefinite} human objects that are specific (\cf~\REF{08-ga-ex:1b}). Note, though, that \textit{a}-marked direct objects need not be specific. This is shown in \REF{08-ga-ex:1c}, where the subjunctive mood of the verb in the relative clause signals that the \isi{direct object}, \ie \textit{una actriz} ‘an actress’, is non-specific, regardless of whether it is marked by \textit{a} or not (\cf \citealt[82--86]{Leonetti2004Specificity} for discussion). 

\begin{exe}
\ex%\1
\ea \label{08-ga-ex:1a}
\gll Pepe ve \textbf{*ø/a} la actriz.\\
 Pepe see[\textsc{3sg}] ø/to the actress\\
\glt ‘Pepe sees the actress.’

\ex \label{08-ga-ex:1b}
\gll Pepe busc-a \textbf{*ø/a} una actriz que \textbf{habl-a} arameo.\\
 Pepe look\_for-3\textsc{sg} ø/to an actress who speak-3\textsc{sg} Aramaic\\
\glt ‘Pepe is looking for an actress who speaks Aramaic.’

\ex \label{08-ga-ex:1c}
\gll Pepe busc-a \textbf{ø/a} una actriz que \textbf{habl-e} arameo.\\
 Pepe look\_for-3\textsc{sg} ø/to an actress who speak-3\textsc{sg.sbjv} Aramaic\\
\glt ‘Pepe is looking for an actress who speaks Aramaic.’ (non-specific reading)\\
\z
\end{exe}

As for \isi{animate} non-human objects, \textit{a}-marking is optional, even if the object is \isi{definite} as in \REF{08-ga-ex:2}. With \isi{inanimate} (\isi{definite}) objects, \textit{a}-marking is generally ungrammatical (\cf~\REF{08-ga-ex:3}).

\ea \label{08-ga-ex:2}
\gll Pepe ve \textbf{ø/a} la vaca.\\
Pepe see[\textsc{3sg]} ø/to the cow\\
\glt ‘Pepe sees the cow.’
\z

\ea \label{08-ga-ex:3}
\gll Pepe ve \textbf{ø/*a} la película.\\
Pepe see[\textsc{3sg]} ø/to the film\\
\glt ‘Pepe sees the film.’
\z

Fitting these overall generalizations, DOM in \ili{Spanish} is usually described by means of the \isi{animacy scale}~\REF{08-ga-ex:4}, the \isi{definiteness} scale~\REF{08-ga-ex:5} or a combination of these prominence scales (\cf \citealt[417--418]{Aissen2003Differential}, \citealt[436]{Laca2006Objeto}).

\ea \label{08-ga-ex:4}
Animacy scale:\\

 human > \isi{animate} > \isi{inanimate}

\z

\ea \label{08-ga-ex:5}
Definiteness scale:

personal pronoun (pron.) > proper name (PN) > \isi{definite} NP (def. NP) > \isi{indefinite} specific NP (spec. NP) > non-specific NP (non-spec. NP)
\z

As is well known, these scales provide a rough means to capture not only language-specific generalizations, but also cross-linguistic tendencies about DOM and related phenomena (for a critical discussion see \citealt{Bickeletal2015Typological,Haspelmath2014Descriptive,Sinnemki2014Typological}, and \citealt{Witzlacketal2017Differential} [this volume]). Typically, the scales are conceived of as implicational hierarchies. Among other things, they make the implicational prediction that if object marking is required for \isi{definite} NPs in a given language, it will also be used for all higher ranging categories of the \isi{definiteness} scale, \ie proper names and personal pronouns. Conversely, it is implicated that if object marking is ungrammatical for \isi{definite} NPs, it is also ruled out for all the lower ranging categories, \ie \isi{indefinite} specific and \isi{indefinite} non-specific NPs. 

Languages with DOM differ in at least two respects. Firstly, object marking may be sensitive to either one of the mentioned scales or to both of them (\cf \citealt[202]{Bossong1998Marquage} among many others, for a different view see \citealt{Sinnemki2014Typological}). For example, in \ili{Hebrew} or \ili{Turkish}, DOM seems to depend only on the \isi{definiteness} scale whereas in \ili{Spanish} or \ili{Romanian} DOM hinges on both the \isi{definiteness} and the \isi{animacy scale}. 
Secondly, languages contrast with respect to the transition point, \ie the right-most category within the relevant scale(s) that requires obligatory object marking. In \ili{Hebrew}, for instance, object marking is obligatory for all \isi{definite} NPs but not for \isi{indefinite} NPs. As an implication, object marking is also compulsory for all the higher ranging categories in the \isi{definiteness} scale, namely proper names and personal pronouns. DOM in \ili{Turkish} shows a very similar distribution. In contrast to \ili{Hebrew}, however, \ili{Turkish} also requires DOM for \isi{indefinite} specific NPs (\cf \citealt[453--454]{Aissen2003Differential} and references cited therein).

Since in \ili{Spanish} DOM depends on both the \isi{animacy} and the \isi{definiteness} scale, the interaction of these scales has to be taken into account. A very elegant way to represent this interaction has been proposed by \citet[40]{vonHeusingeretal2005Evolution}, who use a cross-classification (\cf \tabref{08-ga-tab:1}). This representation provides a clear though still simplified picture of the conditions under which the \textit{a}-marking of the \isi{direct object} in Modern Standard \ili{Spanish} is obligatory (+), optional (±) and ungrammatical (--).

\begin{table}
\begin{tabularx}{\textwidth}{XXXXXX}
\lsptoprule
Definiteness ${\rightarrow}$ & pron. > & PN > & def. NP > & spec. NP > & non-spec. NP\\
\midrule
Animacy ${\downarrow}$ & & & & &\\
human & + & + & + & + & ±\\
\isi{animate} & + & + & ± & ± &--\\
\isi{inanimate} & Ø & ± &-- &-- &--\\
\lspbottomrule
\end{tabularx}
\caption{DOM in Standard Spanish (\cf \citealt[40]{vonHeusingeretal2005Evolution})
}\label{08-ga-tab:1}
\end{table}

The \isi{animacy} and \isi{definiteness} scales are taken to be relevant not only for the synchronic distribution of DOM, but also for its \isi{diachronic development}. The diachronic expansion is claimed to proceed from the more prominent categories on the left/top of the scales to the less prominent ones to the right/bottom of these scales. The opposite holds true for the retraction of DOM in that it is supposed to affect the less prominent categories before the more prominent ones. This less well attested case seems to be evidenced by the \isi{diachronic development} of DOM in \ili{Catalan} (\cf \citealt[212]{Dalrympleetal2011Objects}) and \ili{Portuguese} (\cf \citealt{Delille1970Entwicklung}). 

\newpage 
Thus, at an initial stage, object marking may be restricted to human pronouns. At a further stage, it may become regular also for the less prominent categories of one or both scales, \ie \isi{animate} pronouns, human proper names, \isi{animate} \isi{definite} NPs and so forth. As is sometimes suggested in the literature, this may ultimately lead to a full \isi{grammaticalization} of the differential \isi{object marker} into a regular \isi{accusative case} marker (\cf \citealt[255]{Aissen2003Differential}). In this respect, \citet[191--196]{Villar1983Ergatividad} has argued that Proto-\ili{Indo-European} had a differential \isi{object marker} which, in the historic \ili{Indo-European} languages, developed into an obligatory object case marker (for discussion see \citealt{Bossong1984Review}). As has already been noted in the introduction and will be discussed with more detail in the next section, a similar development has also been claimed regarding \ili{Spanish}. 


\section{Nominal parameters and diachronic DOM in Spanish}\label{08-ga-sec:3}

\subsection{Diachronic corpus studies}\label{08-ga-sec:3.1}

The historic development of DOM in \ili{Spanish} has been analyzed in a number of studies focusing on the impact of different factors such as \isi{animacy} and \isi{definiteness} (\cf \eg \citeauthor{Company2002Grammaticalization} \citeyear*{Company2002Grammaticalization}, \citealt{Laca2002Gramaticalizacion,Aissen2003Differential}), \isi{topicality} (\cf \citealt{Melis1995Objetodirecto}) or \isi{affectedness}, \ie the influence of certain verb classes (\cf \citealt{vonHeusinger2008Verbal,vonHeusingeretal2011Affectedness}). Recently, not only monotransitive but also \isi{ditransitive} constructions have been systematically taken into account (\cf \citealt{Ortiz2005Objetos,Ortiz2011Construcciones}, \citealt{vonHeusinger2017Diachronic} [this volume]). 
While most of the empirical studies are confined to human and \isi{animate} objects, some of them deal exclusively with \isi{inanimate} objects (\cf \citeauthor{Company2002Avance} \citeyear*{Company2002Avance}, \citealt{Barraza2003Evolucion,Barraza2008Marcacion}). The most detailed empirical investigation is provided by \citet{Laca2006Objeto}, whose corpus findings will serve as a reference point in the following sections. Laca’s corpus analysis comprises data from the 12th to the 19th century. The data are taken from nine texts, \ie between one and three text samples per century.\footnote{The corpus is composed of samples from the following texts: \textit{Poema de mio Cid} (12th cent.); \textit{El Conde Lucanor} (14th cent.); \textit{La Celestina} (15th cent.); \textit{Lazarillo de Tormes}, \textit{Documentos lingüísticos de la Nueva España} (16th century); \textit{Don Quijote} (17th cent.); \textit{La comedia nueva}, \textit{El sí de las niñas}, \textit{Documentos lingüísticos de la Nueva España} (18th cent.); \textit{El Periquillo sarniento}, \textit{Pepita Jiménez} (19th cent.).} It goes without saying that, given this rather restricted empirical basis, one has to act with caution when interpreting the data. Whenever possible, her data will be complemented and compared with the findings from other empirical studies. In order to give a critical overview of what is known about the diachronic expansion of DOM in \ili{Spanish}, I will first concentrate on the impact of nominal parameters, \ie \isi{animacy} and \isi{definiteness}. To this end, I will focus on human objects (\sectref{08-ga-sec:3.2}), \isi{animate} (non-human) objects (\sectref{08-ga-sec:3.3}) and \isi{inanimate} objects (\sectref{08-ga-sec:3.4}). In a further step, I will discuss the role of verbal parameters, \ie aspect, \isi{affectedness} and \isi{agentivity} (\sectref{08-ga-sec:4.1}--\sectref{08-ga-sec:4.4}).


\subsection{Human objects}\label{08-ga-sec:3.2}

Following \citet[436--438]{Laca2006Objeto}, I will use the \isi{animacy scale} in \REF{08-ga-ex:4} as well as the somewhat simplified \isi{definiteness} scale given in \REF{08-ga-ex:6}.\footnote{In contrast to the more fine-grained distinctions proposed by \citet[439--443]{Laca2006Objeto}, the scale in \REF{08-ga-ex:6} neither includes the differentiation between NPs with and without lexical heads, nor the distinction between definite-like NPs with universal quantifiers (\eg \textit{cada} ‘each’) and indefinite-like NPs with existential quantifiers (\eg \textit{algo} ‘some’). Consequently, these categories have not been taken into account in \tabref{08-ga-tab:2} and \figref{08-ga-fig:1}. For a discussion of these categories \cf \citet[437--439]{Laca2006Objeto} and \citet[82--87]{GarciaGarcia2014Objektmarkierung}.} 

\ea \label{08-ga-ex:6}
personal pronoun > proper name > \isi{definite} NP > \isi{indefinite} NP > bare noun
\z

The latter scale differs slightly from the hierarchy given in \REF{08-ga-ex:5}. Most importantly, it does not include the category of specificity but that of bare nouns. Whereas \isi{indefinite} NPs may be specific or non-specific, bare nouns are always non-specific. As a consequence, \REF{08-ga-ex:6} will not allow for systematic observations concerning correlations between specificity and DOM.

On the basis of \citegen{Laca2006Objeto} corpus results and the simplified \isi{definiteness} scale in \REF{08-ga-ex:6}, \tabref{08-ga-tab:2} and \figref{08-ga-fig:1} show the \isi{diachrony} for DOM with human objects. It is to be noted that neither in the figure nor in the table have personal pronouns been considered since with these categories object marking was already obligatory in Old \ili{Spanish}.

\begin{table}
\begin{tabularx}{\textwidth}{lXXXXXXX}
\lsptoprule
& XII & XIV & XV & XVI & XVII & XVIII & XIX\\
\midrule
Proper name & 96\% & 100\% & 100\% & 95\% & 100\% & 86\% & 89\%\\
 & (25/26) & (8/8) & (35/35) & (42/44) & (65/65) & (24/28) & (24/27)\\

Definite NP & 36\% & 55\% & 58\% & 70\% & 86\% & 83\% & 96\% \\
 & (13/36) & (36/66) & (38/65) & (85/122) & (117/136) & (44/53) & (73/76)\\

In\isi{definite} NP & 0\% & 6\% & 0\% & 12\% & 40\% & 63\%& 41\% \\
 & (0/6) & (2/31) & (0/11) & (7/59) & (21/53) & (20/32) & (12/29)\\

Bare noun & 0\% & 0\% & 17\% & 5\% & 3\% & 9\% & 6\% \\
 & (0/12) & (0/7) & (2/12) & (2/40) & (1/39) & (2/22) & (1/17)\\
\lspbottomrule
\end{tabularx}
\caption{ Diachrony of DOM with human objects (adapted from \citealt[442--443]{Laca2006Objeto}).}\label{08-ga-tab:2} 
\end{table}

\begin{figure}
\includegraphics[width=\textwidth]{figures/08-ga-fig1.pdf}
\caption{Diachrony of DOM with human objects (based on \citealt[442--443]{Laca2006Objeto})}\label{08-ga-fig:1}
\end{figure}
 
 
\tabref{08-ga-tab:2} and \figref{08-ga-fig:1} allow for a number of observations: Firstly, the expansion of DOM is roughly confined to \isi{definite} and \isi{indefinite} NPs. With \isi{definite} NPs, the frequency of \textit{a}-marked objects increases significantly and more or less continuously. Starting with 36\% of \textit{a}-marked objects with \isi{definite} NPs in the 12th century, we already find 58\% in the 15th century, 86\% in the 17th century and, finally, 96\% in the 19th century. Thus, from being an optional marker for \isi{definite} human objects in Old \ili{Spanish}, essentially restricted to dislocated, \ie topicalized NPs (\cf~\REF{08-ga-ex:7} vs.~\REF{08-ga-ex:8}), \textit{a}-marking has become an almost obligatory requirement for any kind of \isi{definite} human object in Modern \ili{Spanish}, including non-topicalized NPs (\cf~\REF{08-ga-ex:9}).

\ea \label{08-ga-ex:7}
\gll En braço-s tened-es mi-s fija-s tan blanc-a-s como el sol.\\
in arm-\textsc{pl} hold-2\textsc{pl} \textsc{1sg.poss}-\textsc{pl} daughter-\textsc{pl} so white-\textsc{f-pl} as the sun\\
\glt ‘In your arms you hold my daughters as white as the sun.’ (\textit{Cid} 2333, \textit{apud} \citealt[455]{Laca2006Objeto})
\z

\ea \label{08-ga-ex:8}
\gll \textbf{a} las su-s fija-s en-braço las prend-ia\\
to the 3\textsc{poss-pl} daughter-\textsc{pl} in-arm them take-\textsc{ipfv[3sg}]\\
\glt ‘He took his daughters in his arms.’ (\textit{Cid} 275, apud \citealt[428]{Laca2006Objeto})
\z

\ea \label{08-ga-ex:9}
\gll En brazo-s ten-éis \textbf{a} mi-s hija-s tan blanc-a-s como el sol.\\
in arm-\textsc{pl} hold-2\textsc{pl} to \textsc{1sg.poss}-\textsc{pl} daughter-\textsc{pl} so white-\textsc{f-pl} as the sun\\
\glt ‘In your arms you hold my daughters as white as the sun.’
\z

As illustrated by these examples, one of the driving forces for the spread of DOM seems to be \isi{topicality}. However, since topics are typically human and necessarily referential, it is not clear whether \isi{topicality} is also relevant for the spread of DOM concerning other subsets of direct objects, such as those expressed by human \isi{indefinite} NPs. For a discussion on the impact of \isi{topicality} on (diachronic) DOM in \ili{Spanish}, see 
\citeauthor{Laca1995Acusativo} (\citeyear[85--89]{Laca1995Acusativo};  \citeyear[455--456]{Laca2006Objeto}); 
\citet[134, 161]{Melis1995Objetodirecto}; 
\citet[196--225]{Pensado1995Creacion};
\citet[85]{Delbecque2002Construction}; 
\citet[86--107]{Leonetti2004Specificity}; 
\citet[41--45]{vonHeusingeretal2005Evolution}, and
\citet[ch. 8.5.2]{IemmoloBookDifferential}. % TODO replace with a published reference, the author has been contacted

As already mentioned above, \tabref{08-ga-tab:2} and \figref{08-ga-fig:1} also show a remarkable evolution with respect to human objects expressed by \isi{indefinite} NPs. Contrary to \isi{definite} NPs, however, we do not observe a continuous but rather a discontinuous development with \isi{indefinite} NPs. From the 12th to the 16th century, \textit{a}-marking of indefinites is attested very scarcely, showing no relevant tokens in the 12th and 15th century and merely 6\% and 12\% of \textit{a}-marked NPs in the 14th and 16th century, respectively. In the 17th century, there is an abrupt rise of \textit{a}-marked NPs up to 40\% followed by a peak of 63\% case-marked indefinites in the 18th century. Interestingly, \isi{case marking} in this century is clearly more frequent than in the 19th century, where it is attested in merely 41\% of the transitive constructions, \ie just as often as 200 years before. As noted by \citet[460]{Laca2006Objeto}, the relatively high percentage of \textit{a}-marking in the 18th century seems to be due to a verbal factor, namely to the disproportionately high number of causative constructions that are attested in the corresponding text samples. I will comment on this observation in \sectref{08-ga-sec:4.4}. 

Comparing the development of human \isi{definite} and \isi{indefinite} objects, \tabref{08-ga-tab:2} and \figref{08-ga-fig:1} allow for a second general observation: During the whole period, the frequency of marked \isi{definite} objects is clearly and constantly higher than that of \isi{indefinite} objects. This distribution is completely in line with the expected development based on the prominence scales. 

A further observation that follows from \tabref{08-ga-tab:2} and \figref{08-ga-fig:1} is that with both proper names and bare nouns, there is no attested evolution: similarly to strong personal pronouns, proper names already required object marking in the 12th century (\cf~\REF{08-ga-ex:10} as well as the findings from \citealt[207]{Company2002Grammaticalization} given in \tabref{08-ga-tab:5}). Although \figref{08-ga-fig:1} shows a slight retraction in the 18th and 19th century, it is still the strongly preferred option today. 

\ea \label{08-ga-ex:10}
\gll Mat-astes \textbf{a} Bucar e arranc-amos el canpo.\\
kill-\textsc{2sg.pst} to Bucar and take-\textsc{1pl.pst} the field\\
\glt ‘You killed Bucar and we have won the battle.’
 (\textit{Cid} 2458, \textit{apud} \citealt[447]{Laca2006Objeto})
\z

With bare nouns, object marking is hardly ever attested across the centuries. Note that the absolute numbers are extremely low with respect to this category showing only two or fewer tokens with \textit{a}-marked objects per century. This is also the case for the 15th century, where the relatively high frequency of 17\% of DOM corresponds to only 2 out of 12 relevant instances. Even in Modern \ili{Spanish}, DOM of bare nouns is generally blocked. It is only found under certain conditions: (i) with bare plural objects governed by some verbs such as \textit{golpear} ‘to beat’ (\cf example~\REF{08-ga-ex:16} in \sectref{08-ga-sec:4.2}); (ii) with bare plural objects that are modified by an attribute as in \REF{08-ga-ex:11}; and (iii) with bare plurals expressing a \isi{contrastive focus} as in \REF{08-ga-ex:12}.

\ea \label{08-ga-ex:11}
\ea
\gll \textup{\textsuperscript{??}}Detuv-ieron \textbf{a} hincha-s.\\
 arrest.\textsc{pst-3pl} to supporter-\textsc{pl}\\
 \glt ‘They arrested some supporters.’
 
\ex 
\gll Detuv-ieron \textbf{a} hincha-s peligros-o-s del Atlético.\\
 arrest.\textsc{pst-3pl} to supporter-\textsc{pl} dangerous-\textsc{m-pl} of.the Atlético\\%changed the gloss, it was of.ART
\glt ‘They arrested some dangerous Atlético supporters.’
\citep[87]{Leonetti2004Specificity}
\z
\z

\ea \label{08-ga-ex:12}
\ea
\gll \textup{\textsuperscript{??}}En el poblado vi \textbf{a} pescador-es.\\
 in the village see.\textsc{pst.1sg} to fisher-\textsc{pl}\\
\glt ‘In the village I saw some fishers.’

\ex
\gll En el poblado vi \textbf{a} PESCADOR-ES, no \textbf{a} turista-s.\\
 in the village see.\textsc{pst.1sg} to fisher-\textsc{pl} \textsc{neg} to tourist\textsc{-pl}\\
\glt ‘I saw fishers in the village, not tourists.’
\citep[88]{Leonetti2004Specificity}
\z
\z

By way of summary, it is important to stress a fact that has not received the necessary attention in the literature: The expansion of DOM within the domain of humans only applies to \isi{definite} and \isi{indefinite} object NPs. For the other NP types, there is no observable evolution. DOM was either already required in Old \ili{Spanish}, as is the case with proper names, or it was and still is blocked today, as is evidenced by bare nouns.

\subsection{Animate non-human objects}\label{08-ga-sec:3.3}

Let us turn to \isi{animate} objects that do not refer to human individuals such as animals. \tabref{08-ga-tab:3} summarizes the corresponding corpus results from \citet{Laca2006Objeto}. Due to the many gaps and the very low numbers of relevant tokens across all categories, no clear picture emerges from these findings. 

\begin{table}
\begin{tabularx}{\textwidth}{lXXXXXXX}
\lsptoprule
 & XII & XIV & XV & XVI & XVII & XVIII & XIX\\
 \midrule
Proper name & 100\%	&--		&-- 		&-- & 100\%&--&--\\
 & (1/1) 	& (0/0) 	& (0/0)	&(0/0) &(10/10) & (0/0) & (0/0)\\

Definite NP & 0\% & 10\% & 20\% & 0\% & 41\% & 6\% & 36\%\\
	& (0/2) & (2/20) & (1/5) & (0/10) & (16/39) & (1/18) & (4/11)\\

In\isi{definite} NP &-- & 0\% &-- & 0\% & 7\% & 4\% & 0\% \\
		&(0/0) & (0/10) & (0/0) & (0/4) & (1/15) & (1/27) & (0/5)\\

Bare noun &-- & 0\% &-- & 0\% & 0\% & 0\% & 0\% \\
		 & (0/0) & (0/5) & (0/0) & (0/11) & (0/5) & (0/6) & (0/5)\\

\lspbottomrule
\end{tabularx}
\caption{Diachrony of DOM with animate non-human objects (adapted from \citealt[442--443]{Laca2006Objeto})\label{08-ga-tab:3}}

\end{table}

With regard to proper names, \isi{indefinite} NPs and bare nouns, no conclusions whatsoever can be drawn on the basis of these numbers. The results are slightly better for \isi{definite} NPs. Here, one may assume a certain increase of DOM: Whereas in the 14th century only 10\% of the \isi{definite} NPs occur with \textit{a}-marking, we find 41\% of marked objects in the 17th century and 36\% in the 19th century. Note, however, that there are no cases of DOM in the 16th century and that there is a remarkable retraction in the 18th century, where, in contrast to the preceding centuries, only 6\% of \textit{a}-marked objects are attested. 

The results from another \isi{diachronic corpus} analysis, namely that by \citet{Company2002Avance,Company2002Grammaticalization}, suggest a much clearer picture. However, the overall distribution of \textit{a}-marked \isi{animate} objects is considerably lower, showing 3\% of \textit{a}-marked \isi{animate} objects in the 13th and 14th century, 6\% in the 15th century, and 7\% in the 16th century (\cf \tabref{08-ga-tab:5} in \sectref{08-ga-sec:3.4}, below). These percentages may indicate a slight and constant increase of DOM, but one has to be cautious. Firstly, because the category of animates has not been differentiated with respect to \isi{definiteness} in the aforementioned corpus study. This means that the frequencies within the same study may not be comparable. While the attested cases of DOM in the 13th century may contain \isi{animate} \isi{indefinite} NPs, the corresponding data of the 16th century may be confined to \isi{animate} \isi{definite} NPs or proper names. 
Secondly, \citeauthor{Company2002Grammaticalization}'s (2002b) 
%TODO@LSP \citeyear{Company2002Grammaticalization} produced 2002 not 2002b
study does not provide information about the distribution of DOM with animates beyond the 16th century. Thus, in contrast to the development of \textit{a}-marking with human objects, the \isi{diachrony} of DOM with animates is far from clear. 

On the basis of the corpus studies carried out so far, we cannot assess whether there really has been an evolution of DOM with \isi{animate} non-human objects. We clearly need further analyses grounded on much broader empirical bases. Moreover, there are some additional parameters that must be taken into account with respect to \isi{animate} non-human objects, especially with regard to the category of animals. Beyond \isi{definiteness} and other related semanto-pragmatic criteria such as specificity and \isi{topicality}, DOM with animals additionally seems to depend on the species of the animal denoted by the lexical noun as well as on the affective relation between the speaker and the animal referent in question (\cf \citealt[159]{Bossong1991Differential}; \citealt[457]{Aissen2003Differential}; \citealt[2635]{RealAcademia2010Espanola}). Furthermore, \textit{a}-marking also hinges on the \isi{agentivity} of the animal referent in the given event: Based on data from \textit{Don Quijote} (17th century), \citet[42]{Garcia1993Syntactic} observes that \textit{a}-marking of \isi{definite} animal objects is more likely in contexts where the animals are moving and acting on their own than in contexts where no movement of the animals is asserted. These parameters may be responsible for a great amount of both synchronic and diachronic variation.

Summing up the results presented so far, it can be concluded that there has been a clear evolution of DOM along the \isi{definiteness} scale. However, the evolution only concerns human referents, specifically human objects expressed by full \isi{definite} and \isi{indefinite} NPs. While in Old \ili{Spanish} the \textit{a}-marking was optional (±) for human \isi{definite} objects and was not attested for human \isi{indefinite} NPs (--), in Modern \ili{Spanish} we find near obligatory \textit{a}-marking of the former (+) and at least optional \textit{a}-marking (±) of the latter category (\cf \tabref{08-ga-tab:4}). 

\begin{table}
\begin{tabularx}{\textwidth}{XXXX}
\lsptoprule
{}[+human] & Old \ili{Spanish} & Modern \ili{Spanish} & evolution\\
& (12th century) & (19th century) & \\
\midrule
Personal pronoun & + & + & no\\
Proper name & + & + & no\\
Definite NP & ± (36\%) & + (96\%) & yes\\
In\isi{definite} NP &-- (0\%) & ± (41\%) & yes\\
Bare noun &-- &-- & no\\
\lspbottomrule
\end{tabularx}
\caption{ Evolution of DOM with human objects along the definiteness scale}\label{08-ga-tab:4}
\end{table}

\subsection{Inanimate objects} \label{08-ga-sec:3.4}

Let us consider the \isi{diachrony} of DOM with \isi{inanimate} objects. Interestingly, \textit{a}-marking with \isi{inanimate} objects is already found in Old \ili{Spanish}, though it is only attested very scarcely (\cf \sectref{08-ga-sec:4.3.2} for some examples). \citet{Laca2006Objeto} does not give any numbers concerning the development of DOM with \isi{inanimate} objects. However, her conclusion with respect to this lexical subset of object NPs is fairly clear: “On the basis of the analyzed corpus, one cannot assume an increase of the frequency of occurrences of object marking with inanimates, the use of the \isi{object marker} is always marginal in these cases” (\citealt[450]{Laca2006Objeto}, my translation).\footnote{“Partiendo del corpus examinado, no puede hablarse de un aumento de las ocurrencias ante inanimados, antes bien, la marca en estos casos es siempre marginal” \citep[450]{Laca2006Objeto}.}

In contrast, \citet{Company2002Avance,Company2002Grammaticalization} comes to a different conclusion. Her corpus study considers DOM with humans, animates and inanimates from the 13th--20th century. The data from the 20th century are exclusively from Mexican \ili{Spanish}. Based on this corpus study, the author observes that \textit{a}-marking has not only become more frequent for \isi{animate} objects, in particular for humans, but also for \isi{inanimate} objects (\cf \tabref{08-ga-tab:5}).

\begin{table}
{\small \begin{tabularx}{\textwidth}{LLLLLL} 
\lsptoprule
& XIII & XIV & XV & XVI & XX\\
\midrule
Pronoun & 100\% & 100\% & 99\% & 99\% & 100\% \\
& (53/53) & (46/46) & (67/68) & (182/183) & (55/55)\\

PN & 99\% & 99\% & 96\% & 88\% & 100\% \\
	 & (124/125) & (170/172) & (129/134) & (124/147) & (32/32)\\

Human & 42\% & 35\% & 35\% & 50\% & 57\% \\
	 & (243/574) & (224/631) & (181/518) & (541/1086) & (81/141)\\

Animate & 3\% & 3\% & 6\% & 7\% &--\\
	 & (4/155) & (2/64) & (2/34) & (11/168) &--\\

In\isi{animate} & 1\% & 0\% & 3\% & 8\% & 17\% \\
	 & (2/300) & (1/300) & (8/300) & (54/641) & (64/373)\\
\lspbottomrule
\end{tabularx}
}\caption{The diachrony of DOM in Spanish according to \citet[207]{Company2002Grammaticalization}} \label{08-ga-tab:5}
\end{table}

As for the 20th century, the data shows 17\% (64/363) of \isi{inanimate} objects with \textit{a}-marking. Although Company Company does not differentiate between \isi{definite} and \isi{indefinite} NPs, it is very likely that the \textit{a}-marked \isi{inanimate} objects are mostly \isi{definite} (\cf \citealt[28, 108]{Barraza2003Evolucion}, \citealt[38--39, 81--87]{GarciaGarcia2014Objektmarkierung}). According to \citet{Company2002Avance,Company2002Grammaticalization}, the corpus results clearly indicate that (Mexican) \ili{Spanish} is heading towards a full \isi{grammaticalization} of the differential \isi{object marker} into a proper \isi{accusative case} marker:

\begin{quote}
Nowadays, the last stage of the \isi{grammaticalization} is going on; an interesting slow invasion of the \textit{a} case-marker into the prototype \isi{inanimate} zone is taking place, it is no more a classifier ‘personal \textit{a}’, it is becoming a true case-marker, generalizing its meaning and syntactic distribution. \citep[208]{Company2002Grammaticalization}
\end{quote}

However, (Mexican) \ili{Spanish} actually seems to be rather far from entering this last stage of \isi{grammaticalization}. In addition to the above-mentioned findings from \citet[450]{Laca2006Objeto}, this is shown by a number of further empirical analyses (\cf \citealt{Buyse1998Accusative,Barraza2003Evolucion,Tippets2011Differential, GarciaGarcia2014Objektmarkierung}). In what follows, I will briefly comment on these studies.

\citet{Barraza2003Evolucion} is a detailed \isi{diachronic corpus} analysis confined to \isi{inanimate} objects. The data are based on different text types (literary texts, newspapers, academic texts) from the 16th, 18th and 20th centuries. One half of the texts stem from Spain, the other half from Mexico. Similar to \citet{Company2002Avance,Company2002Grammaticalization}, the findings from Barraza Carbajal also suggest an increase of \textit{a}-marking with \isi{inanimate} objects. However, the increase is much lower, showing 2\% (12/547) of \textit{a}-marked instances in the 16th century, 3\% (15/546) in the 18th century and only 5\% (49/962) in the 20th century. 

Similar results for the 20th century are provided by \citet{Tippets2011Differential}, a contrastive analysis of DOM based on exclusively oral material from Buenos Aires, Madrid and Mexico City. At least as far as \isi{inanimate} objects are concerned, the distribution of \textit{a}-marking is notably higher in Buenos Aires but still comparably low in all three cities: \citet[113]{Tippets2011Differential} found 8\% (26/339) of \textit{a}-marked instances in Buenos Aires, 5\% (18/345) in Madrid, and 5\% (13/283) in Mexico City. Particularly the percentages for Madrid and Mexico resemble the above-mentioned results from \citet{Barraza2003Evolucion}. Altogether, the distribution of \textit{a}-marking with \isi{inanimate} objects across the three varieties considered by \citet{Tippets2011Differential} is 5.9\% (57/967).

\citegen{Buyse1998Accusative} study is a synchronic corpus analysis that uses mainly written texts from 20th century European \ili{Spanish}. Regarding \isi{inanimate} objects, his corpus shows only 3.2\% (65/1,936) of marked instances. According to my own empirical research \citep[71]{GarciaGarcia2014Objektmarkierung}, the frequency of \textit{a}-marked \isi{inanimate} objects in the 20th century is even lower, namely 1.2\% (573/48,231). My corpus analysis is based on the \textit{Base de Datos de Verbos,} \textit{Alternancias de Diátesis y Esquemas Sintáctico-Semánticos del Español} (ADESSE), an open source data base of 1.5 million words that pertain to written and oral texts stemming from Spain (80\%) and \ili{Latin} America (20\%).\footnote{For details see \url{http://adesse.uvigo.es/index.php/}.} \figref{08-ga-fig:2} summarizes the results of DOM with \isi{inanimate} objects obtained in the previously mentioned corpus studies. (\textit{DO} refers to morphologically non-marked direct objects and \textit{a~DO} to \textit{a}-marked direct objects). 

\begin{figure}
\includegraphics[width=\textwidth]{figures/08-ga-fig2.pdf}
\caption{Percentages of DOM with inanimate objects in different corpora (20th century)}\label{08-ga-fig:2}
\end{figure}
 
As can be observed in this figure, the percentages of \isi{inanimate} objects with \textit{a}-marking found in the cited studies range from 1.2\% to 17.2\%. Interestingly, the reasons for the differing results seem to be neither connected to the origin of the data (Spain, Mexico etc.), nor to the type of the data (oral vs. written), but rather to the notion of \isi{animacy}. This category is usually taken for granted and not defined explicitly. Particularly important in this regard is the categorization of objects denoting collectives such as \textit{equipo} ‘team’ or \textit{empresa} ‘company’, which are more likely to occur with \textit{a}-marking. Crucially, in some corpus studies such as in \citet{Barraza2003Evolucion}, collectives are classified as inanimates, whereas in others, such as my own \citep{GarciaGarcia2014Objektmarkierung}, they are subsumed under the category of animates. This may be one of the causes for the differing results (\cf \citealt[72--75]{GarciaGarcia2014Objektmarkierung}). In order not to blur the distinction between animates and inanimates, the most adequate treatment would be to put collectives in a separate class, or, as Ilja A. Seržant (p.c.) has suggested, to simply exclude them from the analysis of DOM. This would do justice to the problem that the \isi{animacy} association of these nouns is context-dependent and not uniform.

To summarize this section, it can be concluded that there is no clear support for an evolution of DOM with \isi{inanimate} NPs. Although \textit{a}-marking of \isi{inanimate} objects seems to be attested already in Old \ili{Spanish}, it is still very rare today. Thus, there is no evidence for the hypothesis that the differential \isi{object marker} is becoming a non-differential \isi{accusative case} marker. On the contrary, the empirical findings discussed in this section suggest that the evolution of \textit{a}-marking from Old to Modern \ili{Spanish} is basically restricted to human \isi{definite} and human \isi{indefinite} objects. This may lead to the conclusion that the \textit{a}-marker is basically “a marker of \isi{animate} direct objects” (\citealt[132]{deSwart2007Cross-linguistic}), or human direct objects, to be more precise. However, this is a somewhat problematic simplification since, in combination with certain verbs, \textit{a}-marking may also be required for \isi{inanimate} objects (\cf \sectref{08-ga-sec:4.3} below).

\section{Verbal parameters and diachronic DOM in Spanish}\label{08-ga-sec:4}

In this section, I will consider different verbal parameters, elaborating on their interaction with nominal parameters and their influence on synchronic and diachronic DOM. I will first look at aspect, focusing on \isi{telicity} (\sectref{08-ga-sec:4.1}), then take into account the role of \isi{affectedness} (\sectref{08-ga-sec:4.2}), and, finally, point to the relevance of \isi{agentivity} (\sectref{08-ga-sec:4.3}--\sectref{08-ga-sec:4.4}).

\subsection{Aspect}\label{08-ga-sec:4.1}

According to \citet[1787--1790]{Torrego1999Gramatica}, aspect has a clear and systematic influence on DOM in Modern \ili{Spanish}. She states that direct objects governed by telic verbs, \ie by \citet{Vendler1957Verbs} \textsc{achievement} and \textsc{accomplishment} verbs such as \textit{insultar} ‘to insult’ and \textit{curar} ‘to treat’, take the \textit{a}-marker obligatorily, at least if the object referents are human. This is illustrated in \REF{08-ga-ex:13}.

\ea \label{08-ga-ex:13}
\gll Insult-aron *\textbf{ø/a} un estudiante.\\
insult-\textsc{3pl}.\textsc{pst} ø/to a student\\
\glt ‘They insulted a student.’
\z

Even though the \isi{direct object} in \REF{08-ga-ex:13} is \isi{indefinite}, \textit{a}-marking is not optional but categorical. Note, however, that the verbs considered by Torrego Salcedo are not only characterized by being telic, but also by two further non-aspectual properties: firstly, verbs such as \textit{insultar} ‘to insult’, \textit{sobornar} ‘to bribe’, \textit{curar} ‘to treat’ and \textit{emborrachar} ‘to make drunk’ involve an affected object (\cf \sectref{08-ga-sec:4.2}). Secondly and more importantly, these verbs only accept object arguments that are human. Thus, the alleged lexicalization of the \textit{a}-marker assumed for these verbs might not be tied to \isi{telicity} but rather to their strong preference for human objects (\cf also \citealt[28--29]{vonHeusinger2008Verbal}). Further evidence for this view is provided by the fact that direct objects governed by typical telic predicates with a strong preference for \isi{inanimate} objects such as the \textsc{achievement} verbs \textit{abrir} ‘to open’ or \textit{cerrar} ‘to close’ are systematically blocked for DOM.\footnote{Note also that there are some verbs such as \textit{preceder} ‘to precede’ and \textit{suceder} ‘to follow’ that require \textit{a}-marking even when the object is \isi{inanimate}. Clearly, these verbs denote \isi{atelic} rather than telic events (\cf \sectref{08-ga-sec:4.3}).} 

\ea \label{08-ga-ex:14}
\gll Pepe abr-e \textbf{ø/*a} la puerta.\\
Pepe open-\textsc{3sg}.\textsc{pst} ø/to the door\\
\glt ‘Pepe opens the door.’
\z

\citeauthor{Torrego1999Gramatica} (\citeyear{Torrego1999Gramatica}) also considers \isi{atelic} verbs, \ie \citegen{Vendler1957Verbs} \textsc{activities} (\eg \textit{besar} ‘to kiss’) and \textsc{states} (\eg \textit{conocer} ‘to know’). They seem to differ with respect to the transition point of DOM, \ie the right-most category within the relevant scales requiring object marking. Contrary to the above-mentioned telic predicates, with verbs denoting \textsc{activities} and \textsc{states,} \textit{a}-marking of \isi{indefinite} human objects is not obligatory but rather optional. According to \citet[1788--1789]{Torrego1999Gramatica}, object marking with \textsc{activity} verbs may lead to a shift from an \isi{atelic} to a telic interpretation. However, this is controversial. As convincingly argued by \citet[95--97]{Delbecque2002Construction}, the telic reading does not depend on DOM. This is shown in \REF{08-ga-ex:15}, which clearly denotes a telic event, regardless of whether the object is \textit{a}-marked or not.

\ea \label{08-ga-ex:15}
\gll Bes-aron \textbf{ø/a} varios ciclista-s en una hora.\\
kiss-3\textsc{pl}.\textsc{pst} ø/to several cyclist-\textsc{pl} in one hour\\
\glt ‘They kissed several cyclists in one hour.’
\z

From a diachronic perspective, the influence of aspect on DOM has been studied by \citet{Barraza2008Marcacion}. This study is confined to \isi{inanimate} objects. Therefore, it allows for an animacy-independent evaluation of the impact of aspect. Besides \isi{telicity}, her study also considers perfectivity, \ie the proper aspectual parameter related to the viewpoint of an event (perfective vs. imperfective). As far as \isi{telicity} is concerned, the results of \citet[343--346]{Barraza2008Marcacion} show that \textit{a}-marking through time does not correlate with telic verbs such as \textit{comprar} ‘to buy’, but rather with \isi{atelic} verbs such as \textit{conocer} ‘to know’ (\cf \tabref{08-ga-tab:6}).

\begin{table}
\begin{tabularx}{\textwidth}{XXXXX} 
\lsptoprule
& \multicolumn{2}{c}{DO { }{ }{ }{ }{ }{ }{ }{ }{ }} & \multicolumn{2}{c}{a DO { }{ }{ }{ }{ }{ }{ }{ }{ }{ }{ }{ }}\\
& {\itshape atelic} & {\itshape telic} & {\itshape atelic} & {\itshape telic}\\
\midrule
XV-XVI & 61\% & 39\%& 75\% & 25\% \\
 & (326/535) & (209/535) & (18/24) & 25\%\\

XVIII & 76\% & 24\% & 93\% & 7\% \\
& (404/531) & (127/531) & (67/72) & (5/72)\\

XX & 70\% & 30\% & 72\% & 28\% \\
& (639/913) & (274/913) & (133/185) & (52/185)\\
\lspbottomrule
\end{tabularx}
\caption{Telicity and diachronic DOM with inanimate objects \citep[345]{Barraza2008Marcacion}}\label{08-ga-tab:6}
\end{table}

In each of the considered time periods in \tabref{08-ga-tab:6}, the percentages of \textit{a}-marked objects are clearly higher with \isi{atelic} than with telic verbs. This is particularly evident for the 18th century, where 93\% of the \textit{a}-marked objects are governed by \isi{atelic} verbs. Note that, in all centuries, there is also a clear correlation between \isi{atelic} verbs and the absence of \textit{a}-marking. For example, in the 15th--16th century we find that not only 75\% of the cases with DOM are attested with \isi{atelic} verbs, but also that 61\% of the instances without DOM combine with \isi{atelic} predicates. Though in all of the time periods the percentages of \isi{atelic} verbs are always higher for objects with \textit{a}-marking than for those without \textit{a}-marking, it is striking that, in the 20th century, the difference is only minimal (72\% vs. 70\%). This suggests that, diachronically, the influence of \isi{atelic} verbs has decreased. Nowadays, the frequency of \isi{atelic} verbs with \textit{a}-marked objects roughly corresponds to the frequency of these verbs with objects without \textit{a}-marking. The same applies for telic verbs (28\% vs. 30\%). Consequently, \isi{telicity} itself does not seem to be a relevant factor for DOM in Modern \ili{Spanish}, at least as far as \isi{inanimate} objects are concerned (\cf \citealt[345]{Barraza2008Marcacion}).

The results for perfectivity, that is, the criterion related to the viewpoint aspect, resemble those for \isi{telicity}. \citeauthor{Barraza2008Marcacion}'s (\citeyear[346--348]{Barraza2008Marcacion}) data show that there is a slight diachronic preference for DOM in imperfective rather than in perfective events. For the 20th century, the corpus findings show that 79\% (146/185) of the \textit{a}-marked objects co-occur with an imperfective verb form while only 21\% (39/185) are attested with a perfective verb form. Similar to what is the case with \isi{telicity}, the percentages for the constructions without DOM are comparable: While 74\% (676/913) of the sentences without \textit{a}-marking denote an imperfective event, 26\% (237/913) express a perfective event.

To sum up, our brief discussion of aspect points to the following conclusions: Firstly, the alleged lexicalization of the \textit{a}-marker found with certain telic verbs such as \textit{insultar} ‘to insult’ may not be due to \isi{telicity} but rather to the verb’s restriction for human objects. Secondly, \citegen{Barraza2008Marcacion} analysis of \isi{inanimate} objects suggests that aspect in itself has only a minor influence on DOM in \ili{Spanish}. Thirdly, it seems that this influence decreases through time. Finally, it is remarkable that (diachronic) DOM does not correlate with telic and perfective but with \isi{atelic} and imperfective events, \ie with verbal parameters indicating a low rather than a high degree of \isi{transitivity}. This correlation seems to contradict the findings concerning the second important verbal parameter related to DOM, namely \isi{affectedness}.


\subsection{Affectedness}\label{08-ga-sec:4.2}
\largerpage
The relevance of \isi{affectedness} for DOM in \ili{Spanish} has been pointed to by \citet{Spitzer1928Rum}, \citet{Pottier1968Emploi} and \citet{Torrego1999Gramatica}, among others. Similarly to \isi{telicity}, \citet[1791]{Torrego1999Gramatica} notes that, in Modern \ili{Spanish}, objects governed by verbs selecting an affected object such as \textit{golpear} ‘to beat’ require \textit{a}-marking even for human objects that are \isi{indefinite} and non-specific. As~\REF{08-ga-ex:16} shows, even bare nouns require the \textit{a}-marker, at least with the verb \textit{golpear} ‘to beat’.

\ea \label{08-ga-ex:16}
\gll Siempre golpe-an *\textbf{ø/a} turistas.\\
always beat-\textsc{3pl} ø/to tourists\\
\glt ‘They always beat tourists.’
\z

According to the literature, some of the verbs selecting an affected object such as \textit{castigar} ‘to punish’, \textit{sobornar} ‘to bribe’ or \textit{odiar} ‘to hate’ seem to have lexicalized the \isi{object marker} for all human objects (\cf \citealt[84]{Leonetti2004Specificity} among others). However, it is not clear whether this alleged lexicalization is really due to \isi{affectedness}. Again, most of these verbs only accept human objects. Verbs that also allow for \isi{inanimate} objects such as \textit{odiar} ‘to hate’ only require \textit{a}-marking when the object is human. As stated by \citet[9]{vonHeusinger2008Verbal}: “It rather seems that it is just the condition of being human that triggers (obligatory) DOM.” Moreover, the assumption that verbs such as \textit{odiar} ‘to hate’ select an affected object is not without problems. Usually, such predicates are analyzed as psychological verbs having an \textsc{experiencer} and a \textsc{stimulus} as their arguments, whereby neither the former nor the latter represents a properly affected participant.

The diachronic impact of \isi{affectedness} on DOM in \ili{Spanish} has been systematically analyzed by \citet{vonHeusinger2008Verbal} and \citet{vonHeusingeretal2011Affectedness}. In the latter study, \isi{affectedness} is defined as the “persistent change of an event participant” (\citealt[593]{vonHeusingeretal2011Affectedness}). Moreover, \isi{affectedness} is taken as a gradual notion that is specified by means of \citeauthor{Tsunoda1985Remarks}’s (\citeyear{Tsunoda1985Remarks}: 388) \isi{transitivity} or \isi{affectedness scale}, where different verb classes are ordered with respect to the degree of \isi{affectedness} of the \isi{patient argument} (\cf \tabref{08-ga-tab:7}). 

\begin{table}
{\small \begin{tabularx}{\textwidth}{L L L L L L L}
\lsptoprule
\multicolumn{2}{c}{ 1} & \multicolumn{2}{c}{ 2} & 3 & 4 & 5\\
\multicolumn{2}{c}{Direct effect on patient} & \multicolumn{2}{c}{ Perception} & Pursuit & Knowledge & Feeling\\
\multicolumn{2}{c}{(=effective action)} & & & & &\\
1a & 1b & 2a & 2b & & &\\
+result & --result & +attained & --attained & & &\\
\midrule 
\textit{matar} ‘kill’, \textit{herir}, ‘violate’ & \textit{golpear} ‘hit’, \textit{tirar} ‘shoot’ & \textit{ver} ‘see’, \textit{oír} ‘hear’ & \textit{escuchar} ‘listen’, \textit{mirar} ‘look at’ & \textit{buscar} ‘search for’, \textit{esperar} ‘wait for’ & \textit{conocer} ‘know’, \textit{entender} ‘understand’ & \textit{querer} ‘like’, \textit{temer} ‘fear’\\
\lspbottomrule
\end{tabularx}}
\caption{Affectedness scale of \citet[388, first 5 classes]{Tsunoda1985Remarks} with Spanish verbs (\citealt[609]{vonHeusingeretal2011Affectedness})}\label{08-ga-tab:7}
\end{table}

The left-most class, \ie \textsc{Effective Action}, comprises prototypical transitive verbs such as \textit{kill} or \textit{hit}. This class can further be subdivided into two subclasses (1a and 1b), depending on whether the event denoted by the predicate has a direct result on the patient or not. Verbs from the \textsc{Effective Action} class 1a such as \textit{kill} are supposed to impose the highest degree of \isi{affectedness} on the corresponding patient. The verb classes to the right imply a respectively lower degree of \isi{affectedness}. 

Focusing on the five verb classes given in the \isi{affectedness scale} in \tabref{08-ga-tab:7}, \citet{vonHeusingeretal2011Affectedness} carried out a \isi{diachronic corpus} analysis considering 12 verbs, \ie 2 verbs per class, including the subclasses of the \textsc{Effective Action} type. Their study comprises 2,000 sentences from the 15th, 17th and 19th centuries extracted from the \textit{Corpus del Español} and CORDE. While they only considered human NPs, they carefully differentiated between \isi{definite} and \isi{indefinite} NPs. They found clear significant correlations between verb classes and diachronic DOM with both \isi{definite} and \isi{indefinite} objects. Here, I will only consider the latter NP subtype, \ie human \isi{indefinite} objects, since the impact of verb classes on DOM is more obvious with these objects. The results are presented in \tabref{08-ga-tab:8} and \figref{08-ga-fig:3}.

\begin{table}
\begin{tabularx}{\textwidth}{Llll}
\lsptoprule
& 15th cent. & 17th cent. & 19th cent.\\
\midrule 
1a + 1b \textsc{Effective Action}: \textit{matar}, \textit{herir}, & 18\% & 40\% & 79\%\\
\textit{golpear}, \textit{tirar}& (9/51) & (21/53) & (46/58) \\

2a +2b \textsc{Perception}: \textit{oír ver}, \textit{escuchar}, \textit{mirar} & 17\% & 71\% & 93\% \\
				& (1/6) & (22/31) & (27/29)\\

3 \textsc{Pursuit}: \textit{buscar}, \textit{esperar} & 11\% & 23\% & 41\%\\
				 & (1/9) & (8/35) & (17/41)\\

4 \textsc{Knowledge}: \textit{conocer}, \textit{entender} &-- & 31\% & 67\% \\
					 & (0/0) & (5/16) & (14/21)\\

5 \textsc{Feeling}: \textit{querer}, \textit{temer} &-- & 52\% & 75\% \\
			& (0/0) & (11/21) & (15/20)\\
\lspbottomrule
\end{tabularx}
\caption{Percentages of \textit{a}-marking of human indefinite direct objects for five verb classes \citep[611]{vonHeusingeretal2011Affectedness}}\label{08-ga-tab:8}
\end{table}

Von Heusinger \& Kaiser’s (\citeyear{vonHeusingeretal2011Affectedness}) findings show a great influence of verb classes on DOM through time. Furthermore, they suggest at least a partial correlation between diachronic DOM and \isi{affectedness}. For example, there are clearly higher percentages of \textit{a}-marked instances in each of the centuries for direct objects governed by verbs of the \textsc{Effective Action} class (\eg \textit{matar} ‘to kill’, \textit{golpear} ‘to hit’) than for direct objects combining with the \textsc{Pursuit} class (\eg \textit{buscar} ‘to search for’, \textit{esperar} ‘to wait for’). 

\begin{figure}
\includegraphics[width=\textwidth]{figures/08-ga-fig3.jpg}
\caption{Percentages of \textit{a}-marking of human indefinite depending on verb classes and time \citep[611]{vonHeusingeretal2011Affectedness}}\label{08-ga-fig:3}
\end{figure}


However, as noted by \citet{vonHeusingeretal2011Affectedness}, the corpus results do not fully mirror the expectations based on \citegen{Tsunoda1985Remarks} \isi{affectedness scale}. There are some interesting mismatches concerning the correlation between diachronic DOM and \isi{affectedness}. The most striking mismatch concerns the class of \textsc{Feeling}, which represents the lowest ranking class in the proposed \isi{affectedness scale} (\cf \tabref{08-ga-tab:7}). Contrary to expectation, this class showed a much greater affinity for object marking than the \textsc{Pursuit} or the \textsc{Knowledge} class. Taking a closer look at the \textsc{Feeling} class, \citet{vonHeusingeretal2011Affectedness} found that the two selected verbs, \ie \textit{querer} ‘to like’ and \textit{temer} ‘to fear’, behave very differently. While the first shows the expected lower preference for object marking, the latter demonstrates an unexpected strong preference for \textit{a}-marking. The authors explain the unpredicted behavior of DOM with \textit{temer} ‘to fear’ as follows: 

\protectedex{
\begin{quote}
[T]he \isi{direct object} of ‘fear’ has more typical properties of a subject than a prototypical object of ‘like’ (see \citealt{Kirsneretal1976Inference}). This might be the cause of \textit{temer}’s high scores. This behaviour, however, has nothing to do with \isi{affectedness}, but rather with the competition between the \isi{agentivity} of the participants involved in the event. (\citealt[613]{vonHeusingeretal2011Affectedness})
\end{quote}
}

A similar contrast as the one between \textit{querer} ‘to like’ and \textit{temer} ‘to fear’ is found within the \textsc{Perception} class. Here, the verbs of auditory perception, \ie \textit{escuchar} ‘to listen’ and \textit{oír} ‘to hear’ show a notably stronger preference for diachronic DOM than the visual perception verbs \textit{mirar} ‘to look at’ and \textit{ver} ‘to see’ (\cf \citealt[614]{vonHeusingeretal2011Affectedness}). The different behavior of these verbs can be explained along the same lines as the contrast between \textit{querer} ‘to like’ and \textit{temer} ‘to fear’. While the verbs of auditory perception presuppose a noise-producing source as their object argument, \ie a physically \isi{active} and thus agent-like participant, the object argument of visual perception verbs need not be an agentive participant (\cf also \citealt[244--273]{Enghels2007Modalites}). 

Summing up, on the one hand there seems to be a clear diachronic correlation between \isi{affectedness} and the spread of DOM. On the other hand, however, the unexpected strong preference for diachronic DOM found with the \textsc{Feeling} verb \textit{temer} ‘to fear’, as well as with the verbs of auditory \textsc{perception} \textit{escuchar} ‘to listen’ and \textit{oír} ‘to hear’, suggest a rather contrary correlation, namely that DOM is not favored by a higher degree of the object’s \isi{affectedness} but by a higher degree of the object’s \isi{agentivity}. As we will see in the next section, \isi{agentivity} is also the key notion for understanding the rare and seemingly exceptional cases of DOM with \isi{inanimate} objects.

\subsection{Agentivity and DOM with inanimate objects}\label{08-ga-sec:4.3}

\subsubsection{DOM-sensitive verb classes in Modern Spanish}\label{08-ga-sec:4.3.1}

As shown in \sectref{08-ga-sec:3.4}, \textit{a}-marking of \isi{inanimate} direct objects is generally ungrammatical in Modern \ili{Spanish}; \cf \REF{08-ga-ex:3} repeated in \REF{08-ga-ex:3repeat} for convenience:

\ea%3
\label{08-ga-ex:3repeat}
\gll Pepe ve \textbf{{ø/*a}} la película.\\
Pepe see[\textsc{3sg]} ø/to the film\\
\glt ‘Pepe sees the film.’
\z

However, in some cases, such as those given in \REF{08-ga-ex:17}, \textit{a}-marking of \isi{inanimate} objects is obligatory or at least the strongly preferred option.

\ea\label{08-ga-ex:17}
\ea \label{08-ga-ex:17a}
\gll Un artículo preced-e \textbf{*ø/a} un sustantivo.\\
 a article precede-3\textsc{sg} ø/to a noun\\
\glt ‘An article precedes a noun.’

\ex \label{08-ga-ex:17b}
\gll En este cóctel el vodka pued-e sustitu-ir *\textbf{ø/a} la ginebra.\\
 in this cocktail the vodka can-3\textsc{sg} substitute-\textsc{inf} ø/to the gin\\
\glt ‘In this cocktail, vodka can be substituted by gin.’

\ex \label{08-ga-ex:17c}
\gll La euforia caracteriz-a \textbf{\textsuperscript{??}}\textbf{ø/a} la situación.\\
 the euphoria characterize-3\textsc{sg} ø/to the situation\\
\glt ‘Euphoria characterizes the situation.’

\ex \label{08-ga-ex:17d}
\gll La mujer venc-ió \textbf{\textsuperscript{??}}\textbf{el/al} destino.\\
 the woman beat-3\textsc{sg.pst} the/to.the destiny\\
\glt ‘The woman beat destiny.’

\ex \label{08-ga-ex:17e}
\gll No llam-an conflicto \textbf{*ø/a} una pelea.\\
\textsc{neg} call-3\textsc{pl} conflict ø/to a fight\\
\glt ‘They do not call a fight a conflict.’
\z
\z

Note that these examples challenge many of the standard assumptions about DOM. Firstly, they call into question the implicational predictions associated with prominence scales mentioned in \sectref{08-ga-sec:2}: The observation based on~\REF{08-ga-ex:17}, that (\isi{definite} and \isi{indefinite}) \isi{inanimate} objects must take the \textit{a}-marker, would lead to the wrong prediction that \textit{a}-marking is also obligatory for \isi{animate} non-human objects.\footnote{Though it is more usual to find \isi{definite} rather than \isi{indefinite} NPs among \isi{inanimate} objects with \textit{a}-marking (in particular with those that are not modified by an attribute), \isi{definiteness} is not a necessary condition for \textit{a}-marking (\cf \REF{08-ga-ex:17a} and \REF{08-ga-ex:17e}).} Obviously, this is not the case. In most contexts, \textit{a}-marking of \isi{animate} non-human objects is rather optional than categorical (\cf \tabref{08-ga-tab:1}). As noted by \citet[1788]{Torrego1999Gramatica}, among others, \textit{a}-marking in sentences such as those in \REF{08-ga-ex:17} is not determined by nominal but by verbal factors, more specifically by lexical verbs such as \textit{preceder} ‘to precede’. 

This conclusion is certainly true, but it involves a second problem. It contests the traditionally assumed hierarchy of DOM conditions in \ili{Spanish}, according to which object marking depends first and foremost on nominal parameters (\isi{animacy} and \isi{definiteness}) rather than on verbal parameters. 

The very impact of verbal parameters involves yet a third puzzle for the standard assumptions about DOM (in \ili{Spanish}). The main verbal factors that are taken to be relevant for DOM in \ili{Spanish} are \isi{telicity} and \isi{affectedness} (\cf \sectref{08-ga-sec:4.1} and \sectref{08-ga-sec:4.2}). However, in \REF{08-ga-ex:17} neither the former nor the latter factors are at play. Apart from~\REF{08-ga-ex:17d}, the sentences given in \REF{08-ga-ex:17} do not denote a telic, but a stative situation. Furthermore, they involve a non-affected rather than an affected object. 

Following \citet{Weissenrieder1985Exceptional,Weissenrieder1991Functional} and \citet{Delbecque2002Construction}, I have argued elsewhere (\cf \citealt[65--66]{Garcia2007Inanimate}; \citeyear[147--189]{GarciaGarcia2014Objektmarkierung}) that DOM with \isi{inanimate} objects occurs mainly with a small number of verb classes, namely with those given in \REF{08-ga-ex:18}.

\begin{exe}
\ex DOM-sensitive verb classes \label{08-ga-ex:18}

\ea Verbs of sequencing (\eg \textit{preceder} ‘to precede’, \textit{suceder} ‘to succeed’). \label{08-ga-ex:18a} 
\ex Verbs of replacement (\eg \textit{sustituir} ‘to substitute’, \textit{reemplazar} ‘to replace’) \label{08-ga-ex:18b} 
\ex Verbs of competition (\eg \textit{vencer} ‘to win’, \textit{derrotar} ‘to defeat’) \label{08-ga-ex:18c} 
\ex Verbs of attribution (\eg \textit{caracterizar} ‘to characterize’, \textit{definir} ‘to define’)\label{08-ga-ex:18d} 
\ex Verbs of naming (\eg \textit{considerar} ‘to consider’, \textit{llamar} ‘to call’) \label{08-ga-ex:18e} 
\z
\end{exe}

The unexpected affinity for DOM with \isi{inanimate} objects found with these verbs seems to be triggered by their specific role semantics, at least as far as the classes (18a--d) are concerned.\footnote{DOM with verbs of naming is mostly found in double object constructions, in particular when the object argument and the predicative nominal are adjacent, as in \REF{08-ga-ex:17e}. Thus, with this verb class DOM is rather due to syntactic factors (\cf \citealt[102--104]{GarciaGarcia2014Objektmarkierung}).} According to the generalization of \textit{thematic distinctness} proposed in \citeauthor{Garcia2007Inanimate} (\citeyear[71]{Garcia2007Inanimate}, \citeyear[145]{GarciaGarcia2014Objektmarkierung}); \textit{a}-marking of \isi{inanimate} direct objects is required when the subject does not outrank the object in terms of \isi{agentivity}. Before illustrating this generalization, I will briefly specify my notion of \isi{agentivity}, which is based on \citeauthor{Primus1999Cases}’ (\citeyear{Primus1999Cases};  \citeyear{Primus2006Hierarchy}) Proto-Role model, a refined version of that by \citet{Dowty1991Thematic}. 

\citeauthor{Primus1999Cases} (\citeyear*{Primus1999Cases,Primus1999Rektionsprinzipien,Primus2006Hierarchy}) distinguishes two types of thematic information that define Proto-Roles: \textit{involvement} and \textit{dependency}. Involvement is characterized by the number and content of Proto-properties, which roughly correspond to those mentioned by \citet[573]{Dowty1991Thematic}, that is, control, (autonomous) movement, experience and possession. The second type of thematic information, viz. dependency, describes the causal relation between the involved co-arguments. According to \citeauthor{Primus1999Cases} (\citeyear[52]{Primus1999Cases}; \citeyear[56]{Primus2006Hierarchy}), the \textsc{Proto-Patient} always depends on the \textsc{Proto-Agent} (co-argument dependency). Crucially, the co-argument dependency relation is taken as the central criterion that distinguishes the \textsc{Proto-Agent} from the \textsc{Proto-Patient}. Whereas the \textsc{Proto-Patient} is defined by its causal dependency on the \textsc{Proto-Agent,} the \textsc{Proto-Agent} is conceived of as a \textit{causally independent co-argument}, \ie as an argument whose existence and involvement in a given event do not depend on any other argument. 

Following \citet{Primus2006Hierarchy}, not just participants accumulating many or all of the Proto-Agent involvement properties (control, experience etc.), such as the first argument of \textit{Uma kills Bill}, will count as \textsc{Proto-Agents}. Participants showing a minimal number or even none of the corresponding involvement properties, such as the subject in \textit{Uma is brave}, are also considered as \textsc{Proto-Agents}, though as logically weaker ones.\footnote{\textsc{Proto-Agents} having many or all of the corresponding involvement properties are specified as A\textsuperscript{max}, whereas \textsc{Proto-Agents} with only a minimal or even none of the relevant involvement properties are referred to as A\textsuperscript{min} (\cf \citealt[61]{Primus2006Hierarchy}).} This is due to the fact that, in both situations, \textit{Uma} functions as a causally independent co-argument.

On the basis of Primus’ notion of \isi{agentivity}, let me now illustrate the above-mentioned generalization of thematic distinctness. I will focus on the verbs of sequencing~\REF{08-ga-ex:18a} and the verbs of replacement~\REF{08-ga-ex:18b}, which can be subsumed under the more abstract class of reversible predicates since they both point to a reversible relation between their co-arguments. Consider~\REF{08-ga-ex:17a}, where the verb \textit{preceder} ‘to precede’ denotes a merely temporal ordering of the core arguments \textit{artículo} ‘article’ and \textit{sustantivo} ‘noun’. According to \citet[56]{Primus2006Hierarchy}, both arguments can be categorized as \textsc{Proto-Agents}. This follows from the fact that, in the sequencing event denoted by \textit{preceder} ‘to precede’, none of the co-arguments depends on the other. Note that the same (truth-functional) meaning as in \REF{08-ga-ex:17a} can be expressed by means of the verb \textit{suceder} ‘to succeed/come after’, which is the converse counterpart of \textit{preceder} ‘to precede’: 

\ea \label{08-ga-ex:19} 
\gll Un sustantivo suced-e \textbf{*ø/a} un artículo.\\
 a noun succeed-3\textsc{sg} ø/to a article\\
\glt ‘A noun comes after an article.’
\z

As predicted by the generalization of thematic distinctness, \textit{a}-marking is required in \REF{08-ga-ex:17a}, as well as in \REF{08-ga-ex:19}. Note that the \textit{a}-marked NPs in \REF{08-ga-ex:17a} and~\REF{08-ga-ex:19} are not indirect but direct objects. Though from a semantic point of view neither \textit{preceder} ‘to precede’ nor \textit{suceder} ‘to succeed’ are typically transitive predicates, morphosyntactically they behave as canonical transitive verbs. This is evidenced by the fact that these verbs fulfill the standard morphosyntactic criteria for \isi{transitivity} in \ili{Spanish}. They allow for both pronominalization of the object by means of an accusative clitic and transformation into a passive (\cf \citealt[55--56]{GarciaGarcia2014Objektmarkierung}). 

The obligatory object marking in \REF{08-ga-ex:17b} can also be accounted for by thematic distinctness. Similar to~\REF{08-ga-ex:17a},~\REF{08-ga-ex:17b} also denotes a reversible relation between the corresponding co-arguments. Obviously,~\REF{08-ga-ex:17b} does not encode an asymmetric substitution event, with \textit{vodka} and\textit{ gin} functioning as the respective \textsc{Proto-Agent} and \textsc{Proto-Patient} arguments. Rather, \textit{vodka} and\textit{ gin} are conceived of as replaceable ingredients. This means that~\REF{08-ga-ex:17b} neither entails a proper causation on the part of the subject, nor a proper affection on the part of the object argument. Again, both arguments can be analyzed as \textsc{Proto-Agents} since none of the participants depends on the other. To put it differently, in the referred situation \textit{vodka} and\textit{ gin} serve the same role-semantic function: They can both be used to cause a specific change of state concerning the taste, the alcoholic content or some other characteristic property of the cocktail in question (\cf \citealt[80]{Garcia2007Inanimate};  \citeyear[137--138]{GarciaGarcia2014Objektmarkierung}, and \citealt[78]{Primus2012Animacy}).

Although reversible verbs generally show a very strong preference for \textit{a}-marked direct objects, there are some conspicuous differences among the lexical predicates that form this class. As I have shown in detail elsewhere \citep[162--167]{GarciaGarcia2014Objektmarkierung}, this is particularly obvious with respect to the sequencing verbs \textit{preceder} ‘to precede’, \textit{suceder} ‘to succeed’ and \textit{seguir} ‘to follow’. In the corpus data base ADESSE (20th century), \isi{inanimate} direct objects of \textit{preceder} and \textit{suceder} are exclusively attested with \textit{a}-marking. This suggests that these verbs have lexicalized the \textit{a}-marker. However, in combination with \textit{seguir} \textit{a}-marking is only found in 7.5\% (12/160) of the cases. The different behavior of \textit{preceder} ‘to precede’ and \textit{suceder} ‘to succeed’, on the one hand, and \textit{seguir} ‘to follow’, on the other, is connected to the fact that the latter predicate is a polysemous verb. \textit{Seguir} can be used not only with a reversible meaning in the sense of ‘x comes after y’~\REF{08-ga-ex:20a}, but also with different non-reversible meanings such as ‘to follow (with the eyes), ‘to observe’~\REF{08-ga-ex:20b} or ‘to continue’~\REF{08-ga-ex:20c}. As illustrated in \REF{08-ga-ex:20}, \textit{a}-marking is only found when \textit{seguir} is used with the reversible meaning.

\begin{exe}
\ex \label{08-ga-ex:20} %\20
\ea \label{08-ga-ex:20a} 
\gll la-s pausa-s que sig-uen {\ob}…{\cb} \textbf{a} su-s tarea-s de copista\\
 the-\textsc{pl} pause-\textsc{pl} that follow-3\textsc{pl} ~ to \textsc{3sg.poss}-\textsc{pl} task-\textsc{pl} of copyist\\
\glt ‘the pauses that come after his tasks as a copyist’ 
 (ADESSE, PAI: 086, 02)
 
\ex \label{08-ga-ex:20b} 
\gll el animal-it-o {\ob}…{\cb} segu-í-a cada movimiento de su-s mano-s\\
 the animal-\textsc{dim-masc} ~ follow-\textsc{ipfv-3sg} each movement of his-\textsc{pl} hand-\textsc{pl}\\
\glt ‘the little animal followed/observed every movement of his hands’ (ADESSE, TER: 074, 16)

\ex \label{08-ga-ex:20c} 
\gll te quitaban la chuleta y seg-uí-as el examen\\
 2\textsc{sg.acc} remove the crib and follow-\textsc{ipfv-2sg} the exam\\
\glt ‘they took the crib away from you and you continued the exam’ (ADESSE, MAD: 417, 05)
\z
\end{exe}

Whereas~\REF{08-ga-ex:20a} denotes a situation similar to the ones expressed in \REF{08-ga-ex:17a} and~\REF{08-ga-ex:19}, \ie a merely temporal relation in which the object is as agentive as the subject argument, both the event referred to in \REF{08-ga-ex:20b} and in \REF{08-ga-ex:20c} involve an object that is clearly less agentive than the respective subject participant. This correlates with the absence of \textit{a}-marking. 

In sum, the observations on reversible predicates show that the relative \isi{agentivity} of the \isi{direct object} is a crucial factor for DOM, at least as far as \isi{inanimate} objects in Modern \ili{Spanish} are concerned (for further evidence, including the other DOM-sensitive verb classes mentioned in \REF{08-ga-ex:18}, see \citealt[Ch. 6]{GarciaGarcia2014Objektmarkierung}). Building on these synchronic insights, let us now examine whether \isi{agentivity} is also a diachronically relevant factor for DOM in \ili{Spanish}.

\subsubsection{DOM-sensitive verb classes from a diachronic perspective}\label{08-ga-sec:4.3.2}

It is noteworthy that, despite its rareness, DOM with \isi{inanimate} objects is already attested in older stages of \ili{Spanish}, at least with \isi{definite} NPs (\cf \tabref{08-ga-tab:5}, \tabref{08-ga-tab:6} and~\REF{08-ga-ex:21}). As noted by \citet[451]{Laca2006Objeto}, it typically occurs with certain verbal lexemes such as those given in the examples from Fernando de Rojas’ \textit{Celestina} (1499) and Miguel de Cervantes’ \textit{Don Quijote} (1605, 1615) in \REF{08-ga-ex:21}. 

\begin{exe}
\ex \label{08-ga-ex:21} %\21 
\ea \label{08-ga-ex:21a} 
\gll que preced-e \textbf{a} lo corporal\\
 that precede-3\textsc{sg} to the physical\\
\glt ‘that it precedes the physical things’ (\textit{Celestina}, VI. 178, apud \citealt[451]{Laca2006Objeto})

\ex \label{08-ga-ex:21b}
\gll \textbf{a} los {\ob}…{\cb} clar-o-s sol-es, nublad-o-s scur-o-s {\ob}…{\cb} ve-mos suced-er\\
 to the ~ bright-\textsc{m-pl} sun-\textsc{pl}, cloudy-\textsc{m-pl} dark-\textsc{m-pl} {} see-\textsc{1pl} follow-\textsc{inf}\\
\glt ‘we see that bright sunlight is followed by dark clouds’ (\textit{Celestina}, VIII. 215, apud \citealt[451]{Laca2006Objeto})

\ex \label{08-ga-ex:21c}
\gll La noche que sigu-ió \textbf{al} día del rencuentro de la Muerte.\\
 the night that follow-\textsc{3sg.pst} to.the day of.the reunion of the death\\
\glt ‘The night that followed the day with the reunion with death.’ (\textit{Quijote}, 752, apud \citealt[451]{Laca2006Objeto})

\ex \label{08-ga-ex:21d}
\gll Y \textbf{a} ést-a-s llam-as señales de salud.\\
 and to this-\textsc{f}-\textsc{pl} call-\textsc{2sg} signs of health\\
 \glt ‘And you call those signs of health.’ (\textit{Celestina}, VI. 178, apud \citealt[451]{Laca2006Objeto})

\ex \label{08-ga-ex:21e}
\gll la voluntad \textbf{a} la razón no obedece\\
 the will to the reason \textsc{neg} obey-\textsc{2sg}\\
\glt ‘will does not obey reason’ (\textit{Celestina}, I. 9, apud \citealt[452]{Laca2006Objeto})
\z
\end{exe}

Interestingly, most of these verbs correspond to the same verb classes that are also relevant for Modern \ili{Spanish}: While the examples in \REF{08-ga-ex:21a}--\REF{08-ga-ex:21c} contain the sequencing verbs \textit{preceder} ‘to precede’, \textit{suceder} ‘to succeed’ and \textit{seguir} ‘to follow’,~\REF{08-ga-ex:21d} shows a double object construction with the verb of naming \textit{llamar} ‘to call’. Besides, verbs having a strong preference for (agent-like) human objects such as \textit{obedecer} ‘to obey’~\REF{08-ga-ex:21e} also seem to allow for object marking with inanimates. In order to evaluate the diachronic influence of these verb classes and the impact of \isi{agentivity} on DOM more thoroughly, further research is needed. 

As a first step towards this research task, I carried out a test corpus analysis for the sequencing verbs \textit{preceder} ‘to precede’ and \textit{seguir} ‘to follow’. On the basis of the \textit{Corpus del Español}, I have checked data from the 13th to the 20th century. For each century, I have analyzed the first 100 tokens with \textit{preceder} and \textit{seguir}, respectively. Data containing \isi{animate} objects as well as cliticized objects were excluded. As a consequence, only about 20 relevant tokens per verb and century could be evaluated. The results of the corpus analysis are shown in \tabref{08-ga-tab:9} and the simplified representation in \figref{08-ga-fig:4}.\footnote{In contrast to \tabref{08-ga-tab:9}, \figref{08-ga-fig:4} does not include the findings for the 14th century. In this century, only data with \textit{seguir} ‘to follow’ but no relevant tokens with the verb \textit{preceder} ‘to precede’ were found.} 


\begin{table}
\begin{tabularx}{\textwidth}{XXXXXXXXX} 
\lsptoprule
& XIII & XIV & XV & XVI & XVII & XVIII & XIX & XX\\
\midrule
{\itshape preceder} & 100\% (1/1) &------ & 85\% (11/13) & 77\% (20/26) & 88\% (7/8) & 92\% (22/24) & 94\% (29/31) & 98\% (39/40)\\
{\itshape seguir} & 29\% (6/21) & 6\% (1/17) & 5\% (1/22) & 10\% (3/30) & 6\% (2/34) & 19\% (6/32) & 22\% (4/18) & 13\% (3/23)\\

\lspbottomrule
\end{tabularx}
\caption{ Distribution of DOM with inanimate objects depending on \textit{preceder} ‘precede’ and \textit{seguir} ‘follow’ (Corpus del Español)}\label{08-ga-tab:9}
\end{table}


\begin{figure}
\includegraphics[width=0.95\textwidth]{figures/08-ga-fig4.pdf}
\caption{Percentages of \textit{a}-marking with inanimate objects depending on \textit{preceder} ‘to precede’ and \textit{seguir} ‘to follow’ (Corpus del Español)}\label{08-ga-fig:4}
\end{figure}


\tabref{08-ga-tab:9} and \figref{08-ga-fig:4} allow for the following observations: Firstly, in combination with the sequencing verbs \textit{preceder} ‘to precede’ and \textit{seguir} ‘to follow’, \textit{a}-marking of \isi{inanimate} objects is already attested in the 13th century. Since then, the frequency of DOM with these verbs has remained quite stable. Note that al\-though \textit{a}-marking shows a minimal increase from the 18th century onwards, the highest percentages of DOM with both verbs are documented in the 13th century. This suggests that there have not been any significant changes, neither for DOM in combination with \textit{preceder} ‘to precede’ nor with \textit{seguir} ‘to follow’. Secondly, the verbs obviously have a very different affinity for DOM over time. While \textit{a}-marking with \textit{preceder} is highly frequent, ranging between 77\% and 100\%, with \textit{seguir} it is rather rare. With this verb, the percentages of \isi{inanimate} objects with \textit{a}-marking only range between 5\% and 29\%.

A closer look at the data reveals that the different diachronic behavior of these verbs is due to the same role-semantic reasons as in Modern \ili{Spanish}. The verb \textit{preceder} is nearly exclusively documented with a reversible meaning in the sense of ‘x comes before y’, as in \REF{08-ga-ex:22a}. Only twice is it found within a non-reversible predication, as in \REF{08-ga-ex:22b}. Here, it is not restricted to the denotation of a mere sequencing event, but rather used in the sense of ‘to guide’ or ‘to determine’, thus expressing a causation between the subject and the object participant (\cf \citealt[92--93]{Delbecque2002Construction} for similar meaning variations of \textit{preceder} in Modern \ili{Spanish}).

\begin{exe}
\ex \label{08-ga-ex:22}%\22
\ea \label{08-ga-ex:22a}
\gll El matrimonio {\ob}...{\cb} preced-e \textbf{a}los otr-o-s sacramento-s.\\
 the marriage ~ precede-3\textsc{sg} to.the other-\textsc{m-pl} sacrament-\textsc{pl}\\
\glt ‘Marriage precedes the other sacraments.’
 (13th century, Alf. X., \textit{Siete partidas})

\ex \label{08-ga-ex:22b}
\gll la certeza y seguiridad {\ob}…{\cb} deb-e preced-er su ejercicio\\
 the certainty and confidence ~ must-3\textsc{sg} precede-\textsc{inf} 3\textsc{sg.poss} practice\\
\glt ‘certainty and confidence must guide his practice’
 (16th century, Solórzano Pereira, \textit{Política indiana})
\z
\end{exe}

Contrary to \textit{preceder}, the verb \textit{seguir} is only rarely attested with a reversible predication in the sense of ‘x comes after y’, as in \REF{08-ga-ex:23a}. It is used much more frequently with a non-reversible meaning such as ‘to continue’, illustrated in \REF{08-ga-ex:23b}.

\begin{exe}
\ex \label{08-ga-ex:23}%\23
\ea \label{08-ga-ex:23a}
\gll sigu-e \textbf{a}la primer-a faz de Aries\\
follow-3\textsc{sg} to.the first-\textsc{f} phase of Aries\\
\glt ‘it follows/comes after the first phase of Aries’
(13th century, Alf. X., \textit{Judizios de las estrellas})

\ex \label{08-ga-ex:23b}
\gll non quis-o ssegu-ir el pleito\\
 \textsc{neg} want.\textsc{pst-3sg} follow-\textsc{inf} the lawsuit\\
\glt ‘he did not want to continue the lawsuit’
 (13th century, Alf. X., \textit{Espéculo})
\z
\end{exe}

As shown in \REF{08-ga-ex:22} and~\REF{08-ga-ex:23}, \isi{inanimate} objects of reversible relations are regularly marked with \textit{a}, both in combination with \textit{preceder} and \textit{seguir} while those found in non-reversible predications, which are much more common with \textit{seguir}, lack \textit{a}-marking. These observations suggest that it is not the verb \textit{per se} that triggers DOM through time but rather the \isi{agentivity} of the \isi{direct object} that follows from the more or less frequently attested reversible meanings of the investigated verbs. This claim is supported by the synchronic distribution of DOM found with most of the other DOM-sensitive verb classes mentioned in \REF{08-ga-ex:18}. A case in point are the verbs of replacement \textit{sustituir} ‘to substitute’ and \textit{reemplazar} ‘to replace’: Similar to \textit{seguir} ‘to follow’, both \textit{sustituir} and \textit{reemplazar} have a reversible meaning (‘x takes the place of y’) and a non-reversible meaning (‘x substitutes/replaces y (with z)’), whereby the reversible variant patterns systematically with DOM and the non-reversible patterns with the absence of object marking (\cf \citealt[395--396]{Weissenrieder1985Exceptional}; \citealt[149--154]{GarciaGarcia2014Objektmarkierung}). However, so far, these verbs have only been examined in Modern \ili{Spanish}. 

In order to obtain a more detailed picture of the diachronic impact of \isi{agentivity} on DOM, the diachronic test corpus study undertaken for \textit{preceder} ‘to precede’ and \textit{seguir} ‘to follow’ must be complemented by empirical analyses considering all the other DOM-sensitive verb classes mentioned in \REF{08-ga-ex:18}, in particular by verbs of replacement (\eg \textit{reemplazar} ‘to replace’), verbs of attribution (\eg \textit{caracterizar} ‘to characterize’) and verbs of competition (\eg \textit{vencer} ‘to win’).

\subsection{\textit{Accusativus-cum-infinitivo}-constructions (AcI)}\label{08-ga-sec:4.4}

This section deals with AcI-constructions with causative and perception verbs. Thus, it does not consider a proper verbal but a constructional parameter. As we will see, AcI-constructions also seem to underpin the (diachronic) influence of \isi{agentivity} on DOM. Let us reconsider the \isi{diachronic development} of DOM with human \isi{indefinite} objects reported in \sectref{08-ga-sec:3.2}. As illustrated in \tabref{08-ga-tab:2} and \figref{08-ga-fig:1}, the expansion of \textit{a}-marking with this subset of objects shows a striking irregularity. While there are 40\% (21/53) of \textit{a}-marked objects in the 17th century and 41\% (12/29) in the 19th century, the greatest percentage of \textit{a}-marking with \isi{indefinite} human objects is found in the 18th century, showing a remarkable peak of 63\% (20/32). As noted by \citet[460]{Laca2006Objeto}, the relatively high percentage of \textit{a}-marked objects found in this century is due to the disproportionately high number of causative constructions attested in one of the corresponding text samples, namely the \textit{Documentos lingüísticos de la Nueva España}. In this text sample, 9 out of 12 of the \textit{a}-marked \isi{indefinite} human objects contain a causative construction such as the one given in \REF{08-ga-ex:24}.

\ea \label{08-ga-ex:24}
\gll hiz-o parec-er ante sí \textbf{a} un yndio que {\ob}…{\cb} dij-o llamarse Pedro Martín\\
 make\textsc{.pst-}3\textsc{sg} appear-\textsc{inf} before \textsc{refl} to an Indian who ~ say.\textsc{pst-}3\textsc{sg} to.be.called Pedro Martín\\
\glt ‘He summoned to him an Indian who said that he was called Pedro Martín.’
(\textit{DLNE}, 1733, 189.487, apud \citealt[460]{Laca2006Objeto})
\z

The affinity of AcI-constructions for DOM is not only evidenced by constructions with causative verbs, but also by those with perception verbs. Although DOM is probably less frequent with the latter type of AcI-construction than with the causative type (\cf \citealt[316--317]{Roegiest2003Argument}), it is still very common to also use the \isi{object marker} in AcI-constructions with perception verbs, at least in Modern \ili{Spanish}: 

\ea \label{08-ga-ex:25}
\gll Se o-yó maull-ar \textbf{a} un gato.\\
\textsc{REFL} hear-3\textsc{sg.pst} meow-\textsc{inf} to a cat\\
\glt ‘We heard the meowing of a cat.’ (Corrales Egea, apud \citealt[50]{Roegiest1979Accusatif})

\z

The question here is why AcI-constructions show such a striking preference for DOM. One can assume that this is due to \isi{agentivity}, \ie to the semi-agentive status of the object participant. As argued by \citet[50]{Roegiest1979Accusatif}, the \isi{direct object} of the matrix verb is concurrently the “subject” of the infinitival verb, whereby the latter relation involves an “activation” of the object, that is, an agentive interpretation of the corresponding participant. Within the Proto-Role model, it can be specified that the second participant of an AcI-construction shows both proto-agent and proto-patient properties (\cf \citealt[161--162]{Primus1999Rektionsprinzipien}). This is particularly obvious with respect to~\REF{08-ga-ex:25}. Whereas the first argument of the perception event denoted by \textit{oír} ‘to hear’ has the Proto-Agent property \textit{experience}, the second argument, \ie the \isi{indefinite} non-human NP \textit{un gato} ‘a cat’, is not only characterized by the converse Proto-Patient property of \textit{being experienced}, \ie of being perceived, but also by the Proto-Agent property \textit{move}, entailed by the infinitival verb \textit{maullar} ‘to meow’. Note that the Proto-Agent property \textit{move} is associated with any form of autonomous physical activity (\cf \citealt[55]{Primus2006Hierarchy}). 

The close connection between the \isi{direct object}’s \isi{agentivity} and DOM is also corroborated by \citegen[241--273]{Enghels2007Modalites} fine-grained study on AcI-constructions with perception verbs in Modern \ili{Spanish} (\cf also \citealt[1792]{Torrego1999Gramatica}). Enghels differentiates between different factors that determine the \isi{agentivity} degree of the \isi{direct object}, \ie of the second argument of an AcI-construction, such as (i) the modality of the perception verb (visual vs. auditory), (ii) the \isi{animacy} of the second argument (human, \isi{animate}, \isi{inanimate} etc.) and (iii) the semantics of the infinitival verb (transitive, unergative, unaccusative). With respect to the latter factor, it is assumed that AcI-constructions embedding predicates that are transitive, such as \textit{matar} ‘to kill’, presuppose a high \isi{agentivity} degree of the second argument, while AcI-constructions embedding unergative verbs such as \textit{reír} ‘to laugh’ and those having unaccusative verbs such as \textit{morir} ‘to die’ imply a respectively lower \isi{agentivity} degree of the second argument. \citeauthor{Enghels2007Modalites}’ (2007: 241--273) findings reveal that the more the mentioned factors indicate an agentive interpretation of the \isi{direct object} argument, the greater the probability for \textit{a}-marking. Though the modality of the perception verb (visual vs. auditory) and the \isi{animacy} of the second argument are the most relevant factors, there is also a clear and independent effect with respect to the semantics of the infinitival verb (\cf \tabref{08-ga-tab:10}).


\begin{table}
\begin{tabularx}{\textwidth}{Xllll}
\lsptoprule
infinitival predicate &DO&  &a DO& \\
\midrule
transitive & 1.4\% & (5/369) & 98.6\% & (364/369)\\
unergative & 4.5\% & (17/308) & 94.5\% & (291/308\\
unaccusative & 28.9\% & (123/425) & 71.1\% & (302/425)\\
\lspbottomrule
\end{tabularx}
\caption{Distribution of DOM with human objects in AcI-constructions depending on the semantics of the infinitival predicate (adapted from \citealt[268]{Enghels2007Modalites})}\label{08-ga-tab:10}
\end{table}

\tabref{08-ga-tab:10} represents the influence of the embedded infinitival predicate on DOM in AcI-constructions with human direct objects. As can be observed, \textit{a}-marking is noticeably more frequent with transitive verbs (98.6\%) than with \isi{intransitive} verbs, especially in comparison with unaccusative verbs (71.1\%), that is, with those predicates presupposing the lowest \isi{agentivity} degree of the \isi{direct object}.

\section{Conclusion}\label{08-ga-sec:5}

In \ili{Spanish}, DOM is diachronically triggered not only by nominal, but also verbal parameters. The general picture that emerges from the current research on nominal parameters (\isi{animacy} and \isi{definiteness}) is that DOM is a remarkably stable system. Although there has clearly been an evolution of DOM from Old to Modern \ili{Spanish}, this development is basically restricted to human \isi{definite} and \isi{indefinite} objects (\cf \tabref{08-ga-tab:4}). Other NP types do not seem to have undergone any remarkable changes. This applies in particular to the category of inanimates: The \textit{a}-marking of \isi{inanimate} direct objects was and still is a scarcely attested phenomenon (\cf \figref{08-ga-fig:2}). Thus, there is no clear support for the hypothesis that the \textit{a}-marker is grammaticalizing into a proper \isi{accusative case} marker and, consequently, that \ili{Spanish} is changing from a language with DOM to a language without DOM. Nevertheless, it would be wrong to conclude that DOM in \ili{Spanish} is essentially driven by humanness.

The discussion of verbal parameters has revealed that the occurrence of DOM through time is also influenced by \isi{agentivity}, \isi{affectedness} and, in some rather inconsistent way, also by aspect. As for \isi{agentivity}, the test corpus analysis of \textit{preceder} ‘to precede’ and \textit{seguir} ‘to follow’ (13th--20th century) has shown that agentive objects require \textit{a}-marking even when the referent is \isi{inanimate}. Thus, in both Modern and Old \ili{Spanish}, \isi{agentivity} overrides the strong DOM condition of humanness. Further evidence for the relevance of \isi{agentivity} is provided by the unexpected preference for DOM with verbs such as \textit{temer} ‘to fear’ (\cf \citealt[613]{vonHeusingeretal2011Affectedness}), as well as by AcI-constructions, which also show a clear preference for DOM, at least from the 18th century on. In these constructions the \isi{direct object} not only functions as a patient, but also as an agent argument. 

Note that the conclusion that DOM is diachronically conditioned by both the object’s humanness and the object’s \isi{agentivity} is no contradiction. On the contrary, humanness can be taken as an inherent nominal feature that encodes a very typical, though not necessary, property of an agent. As pointed out by \citet[398]{Delbecque1998Frames} and \citet[78--79]{Primus2012Animacy}, among others, human direct objects can be conceived of as potential agents.

The interaction of nominal and verbal parameters, though, remains challenging. As has been shown, diachronic DOM also depends on \isi{affectedness} and, to some extent, on \isi{telicity}. However, these factors only seem to be relevant with respect to human objects. While there are some telic predicates involving a highly affected object that have lexicalized the \textit{a}-marker, such as \textit{matar} ‘to kill’ and \textit{insultar} ‘to insult’, it must be emphasized that these verbs only accept human or at least \isi{animate} objects. If we only consider \isi{inanimate} objects, \isi{telicity} has a rather negative influence on diachronic DOM (\cf \tabref{08-ga-tab:6}). Besides, we also find \isi{atelic} verbs selecting a non-affected object such as \textit{preceder} ‘to precede’ and \textit{suceder} ‘to succeed’ that seem to have lexicalized DOM, too. This leads to the puzzling conclusion that, in terms of \citet{Hopperetal1980Transitivity}, DOM in \ili{Spanish} is driven by both extremely high and extremely low \isi{transitivity} (\cf also \citealt[67]{Fabregas2013Differential}). Obligatory \textit{a}-marking is not only found with human, strongly affected objects involved in a telic event, but also with \isi{inanimate}, non-affected and agentive objects embedded in a stative event. 

In order to understand these contrary facts, more research on the interaction of nominal and verbal parameters is needed. In particular, systematic analyses of \isi{agentivity}, \isi{affectedness} and \isi{telicity} that are independent of \isi{animacy} are necessary. 

\section*{Acknowledgments}
I would like to thank Ilja A. Seržant, Javier Caro Reina and Klaus von Heusinger for their very useful comments on the entire manuscript.

 
{\sloppy
\printbibliography[heading=subbibliography,notkeyword=this] }
\end{document}
