\chapter{Theories of fragments} \label{sec:chapter-theories}

Since \citet{morgan1973} introduced the notion of \textit{fragment} and first described the phenomenon, there has been considerable debate and disagreement on the syntax of these expressions. In the first part of this book (Chapters \ref{sec:chapter-theories} and \ref{sec:chapter-experiments-syntax}), I therefore discuss and experimentally investigate aspects of the syntax of fragments on which competing theories disagree and which will allow us to test the validity of the theories' predictions. Besides contributing to our theoretical understanding of fragments, the experiments lay the ground for the experiments on their usage in the second part of this work.

This chapter summarizes some representative versions of the most influential generative\is{Generative grammar} theories of fragments. Among these, two families of syntactic accounts are to be distinguished. On the one hand, \textit{nonsentential accounts}\is{Nonsentential account} (Section \ref{sec:theories-nonsentential}) treat fragments as truly nonsentential expressions that lack any sort of unarticulated structure. This requires some modification of syntactic theory in order to allow for well-formed subsentential output \citep[see e.g.][]{barton.progovac2005, fortin2007}. On the other hand, \textit{sentential accounts} (Section \ref{sec:theories-sentential}) claim that fragments are derived by ellipsis from linguistically complete sentences. There are two versions of sentential accounts: the \textit{in situ deletion} account \citep{reich2007}, which derives fragments from regular sentences, and the \textit{movement and deletion}\is{Movement and deletion account} account \citep{merchant2004}, which states that the future fragment has to occupy a left-peripheral position in the full sentence before ellipsis applies. Finally, in Section \ref{sec:theories-ungrammatical} I discuss the claim by \citet{bergen.goodman2015}\is{Ungrammaticality of fragments} that fragments are actually \textit{ungrammatical}, but that speakers can still use them if they manage to get their message across. The experiments presented in this book do not explicitly address the predictions of theories of fragments in other syntactic frameworks, like HPSG\is{HPSG} \citep{ginzburg.sag2000, fernandez.ginzburg2002, schlangen2003}. Since these accounts assume relatively complex structures for individual types of fragments, which model connectivity effects\is{Case connectivity} and other properties, it is difficult to falsify them empirically and compare their predictions to generative\is{Generative grammar} accounts that derive fragments by more abstract and general principles. Nonetheless, in the discussion of the results I address issues that are relevant to the empirical predictions of HPSG\is{HPSG} accounts.

This chapter is structured as follows: In Sections \ref{sec:theories-nonsentential}--\ref{sec:theories-ungrammatical}, I present the central ideas of these theories and avoid controversial or conflicting evidence as much as possible. Section \ref{sec:theories-predictions} discusses a series of phenomena that have been argued by the respective authors to constitute evidence for or against specific accounts. As will become clear, most of the theories explain most of the data, but there are some aspects on which they disagree and which will serve as a testing ground for the competing theories in experiments \ref{exp:case}--\ref{exp:mvb}.

With the exception of \citet{bergen.goodman2015}\is{Ungrammaticality of fragments}, all of the accounts presented here have been developed by authors working in a Chomskyan generative\is{Generative grammar} framework. Therefore, they focus on explaining why we observe specific restrictions on the form of fragments, but neglect their processing and psychological reality. This might raise the question of whether modeling the syntactic derivation is relevant at all to the processing and interpretation of fragments by the hearer. For instance, from the hearer's perspective, it might seem irrelevant whether the speaker had a linguistic structure in mind, which is only partially articulated, or there was nothing but a fragment to begin with: The fragment she%
%
\footnote{Throughout this book I use arbitrary gender pronouns in order to refer to abstract hearers and speakers. Sometimes the speaker will be female and the hearer male and vice versa, but I use the same pronoun for the same imaginary person in a situation.}\afterfn%
% 
has to interpret is identical in both cases. However, there are at least two good reasons to take the derivations proposed by the different theories of fragments seriously. First, if fragments are generated by grammatical mechanisms, knowledge about these will guide the hearer in retrieving the intended message. Second, if such grammatical mechanisms restrict the form of possible fragments, they will restrict the set of alternative encodings of a proposition to those fragments which are a well-formed output of grammar.

\section{Fragments as nonsententials}
\is{Nonsentential account|(}
\label{sec:theories-nonsentential}
According to nonsentential accounts\is{Nonsentential account}, fragments do not contain any sort of unarti\-culated structure. As \citet{stainton2006} points out, the assumption that fragments are \textit{genuinely} nonsentential presupposes that there is neither silent material in fragments nor that any parts of the utterance are deleted in course of the derivation. This requires some modification to standard syntax in order to allow for subsentential expressions to be a well-formed output of syntax.

\citet{barton.progovac2005} sketch a theory of fragments that is based upon this idea, which is grounded in the minimalist Program\is{Minimalist program} \citep{chomsky1995}. They propose two adjustments of the theory in order to allow for syntactically well-formed subsentential objects. First, any maximal projection XP can be a well-formed output of grammar, and second, case checking\is{Case feature} requirements are relaxed in fragments. Besides bare XP fragments, their theory is designed to explain other omissions found in the ETP\is{Tense phrase} corpus\is{Corpus} \citep{libben.tesak1994}, a corpus\is{Corpus} of elicited `telegraphese' data. As discussed by \citet{barton1998}, this register is characterized by frequent omissions of functional elements like articles\is{Article omission}, first person subject pronouns and auxiliary verbs.

The first modification to standard syntax that \citet{barton.progovac2005} propose is that the derivation can stop at any maximal projection, as long as it is internally well-formed and there are no lexical items left in the numeration. The fragment in \Next is consequently analyzed as a bare VP\is{Verb phrase} that does not contain a TP\is{Tense phrase}. \citet{barton.progovac2005} argue that there is no evidence for a T head in the derivation, because the verb \textit{play} is not inflected for person or tense.

\ex. What does John do all summer? \hfill \citep[81]{barton.progovac2005}\\
Play baseball. 

The second modification that \citeauthor{barton.progovac2005} propose is the Case Feature Corollary (CFC) \Next. In minimalism\is{Minimalist program}, case-marked DPs\is{Determiner phrase} are assumed to have uninterpretable case feature\is{Case feature}s, which must be checked in a specific syntactic configuration by a head carrying the same feature \NNext. The CFC loosens this requirement for fragments \Next, but not for full sentences.

\ex. Case Feature Corollary (CFC)\hfill \citep[78]{barton.progovac2005}\\Nonsententials differ from sententials in one property: they are not required to check Case feature\is{Case feature}s.

\ex. \Tree [.VP Spec [.V' V\textsuperscript{Acc} DP\textsuperscript{\textit{u}Acc} ] ]

\citet{barton.progovac2005} motivate the CFC with the observation of differing case-marking preferences between fragments and sentences. For instance, English\il{English} pronominal short answers\is{Fragment, short answer} \Next[a,b] seem to be more acceptable in accusative\is{Accusative case} case than in nominative\is{Nominative case}. In full sentences \Next[c,d] the pattern is inverted, even though the pronoun has the same grammatical function in both cases.\largerpage[-1]

\ex. Who can eat another piece of cake? \hfill \citep[77]{barton.progovac2005} \label{ex:barton.cake}
\a. ?*I/?*We/?*He/?*She.
\b. Me/Us/Him/Her.
\c. I/We/He/She can.
\d. *Me/*Us/*Him/*Her can.

\citeauthor{barton.progovac2005} argue that this follows from the default case\is{Default case} status of the English\il{English} accusative\is{Accusative case}, while nominative\is{Nominative case} is considered structural case\is{Structural case}. In this line of reasoning, structural case\is{Structural case} is assigned in specific syntactic configurations and does not contribute to semantics, unlike inherent\is{Inherent case} case, which encodes specific a \texttheta-role. In the case of English\il{English}, nominative\is{Nominative case} is checked by a T head. \citet{barton.progovac2005} interpret the data in \Last as evidence nominative\is{Nominative case} in fragments is ungrammatical because of the absence of a covert T head, which would check nominative\is{Nominative case} case feature\is{Case feature}s. In contrast, accusative\is{Accusative case} is acceptable in fragments according to \citet[78]{barton.progovac2005}, because it is the default case\is{Default case} in English\il{English}, i.e. the most unmarked form. They argue that the use of accusative\is{Accusative case} in predicative DPs\is{Determiner phrase} \Next evidences this, because nominative\is{Nominative case} is assigned only to the specifier of TP\is{Tense phrase}.

\ex. \a. This is me/him/us.\hfill \citep[79]{barton.progovac2005}
    \b. ?This is I/he/we.

As the predictions of \citeauthor{barton.progovac2005}'s account crucially rely on the concept of default case\is{Default case}, this notion requires some further attention. First of all, it is controversial whether default case\is{Default case} exists at all. In minimalism\is{Minimalist program}, case is modeled by the assumption of specific features and it has been assumed since \citepos{chomsky1981} \textit{case filter} that derivations converge only if all DPs\is{Determiner phrase} are case-marked. However, even \citet{merchant2004a}, who argues against default case\is{Default case}, assumes that resumptive pronouns like \textit{who} in his example \Next are base-generated in a left-peripheral position and that they cannot undergo the regular case checking\is{Case feature} mechanisms. 

\ex. Who\textsubscript{i} do you think that if the voters elect him\textsubscript{i}, the country will go to ruin.

\citet{schutze2001} argues that default case\is{Default case} \textit{can} be integrated into a minimalist framework if it is defined as a residual category of case-marking that is assigned only to those DPs\is{Determiner phrase} which are not marked with a more specific case.%
%
\footnote{\citeauthor{schutze2001} adopts concepts from Distributed Morphology\is{Distributed morphology}  \citep{halle.marantz1993}, in particular, the idea of \textit{late insertion} of lexical items into the derivation in a postsyntactic spell-out module. \citeauthor{schutze2001} proposes that only arguments receive uninterpretable case feature\is{Case feature}s before entering the numeration, which are different from the optional morphological case feature\is{Case feature}s that determine which case marking a DP\is{Determiner phrase} receives when it is selected during late insertion \citep[230--231]{schutze2001}. Non-arguments do not require syntactic case marking at all.}\afterfn%
%
Default case\is{Default case} simply appears whenever no other case marking is available. According to \citeauthor{schutze2001}, this becomes evident when the DP\is{Determiner phrase} appears in a position where no syntactic relation to expressions that check case can be established, such as hanging topics\is{Topic, hanging} \Next[a] or predicative constructions \Next[b]. 

\ex. \label{ex:theory-schuetze-default}
\a. Me/*I, I like beans. \hfill \citep[210]{schutze2001}
\b. The real me/*I is finally emerging. \hfill \citep[215]{schutze2001}

Since default case\is{Default case} has the status of a residual category in \citeauthor{schutze2001}'s theory, he predicts considerable crosslinguistic variation both with respect to the contexts where it occurs and to the form that default case\is{Default case} takes. First, languages can differ in whether a case checking\is{Case feature} relation is established in a specific syntactic position, so that the distribution of default case\is{Default case} can differ crosslinguistically. With respect to the form, crosslinguistic equivalents of \ref{ex:theory-schuetze-default} suggest that in some languages, such as English\il{English}, Irish\il{Irish} and Norwegian\il{Norwegian}, accusative\is{Accusative case} is default case\is{Default case}, whereas it is nominative\is{Nominative case} in German\il{German}, Russian\il{Russian} or Dutch\il{Dutch} \citep[229]{schutze2001}. \citet[51]{progovac2006} argues that it is also nominative\is{Nominative case} in Serbian\il{Bosnian/Croatian/Serbian} \Next.

\exg. Ona/*Nju predsednik kluba?! (Vi se šalite.)\\
she.\textsc{nom}/her president club.\textsc{gen} ~you \textsc{refl} kid\\ 
\trans{Her president of the club?! (You must be kidding!)}\hspace{-3em}\exsourceraised{\citep[51]{progovac2006}}

Therefore, the nonsentential account\is{Nonsentential account} makes different predictions on the case marking of DP\is{Determiner phrase} fragments in examples such as \ref{ex:barton.cake} depending on which case is the default case\is{Default case} in a language. The German\il{German} version of the question-answer pair \ref{ex:barton.cake}, which is given in \Next, is in line with this prediction.%
%
\footnote{It shall be noted that \citet[221]{schutze2001} also notes that subject DP\is{Determiner phrase} fragments like \ref{ex:barton.cake} receive accusative\is{Accusative case} case marking in English\il{English}, but nominative\is{Nominative case} in German\il{German}. However, unlike \citet{barton.progovac2005} and \citet{progovac2006}, \citeauthor{schutze2001} does not simply explain this by the different default case\is{Default case} in both languages but argues that DP\is{Determiner phrase} fragments are only a ``possible default-case environment'' \citep[229]{schutze2001}: He argues that it is an ``actual'' one in English\il{English}, but not in German\il{German}, which uses the ``strategy'' of always matching case in question and answer.}\afterfn%
%
Note that this example does not contradict sentential accounts, which interpret \Next as evidence \textit{for} an unarticulated T head checking nominative\is{Nominative case} in fragments. I return to this issue in Section \ref{sec:theories-predictions-case}.

\exg. Wer kann noch ein Stück Kuchen essen?\\
who can more one piece cake eat\\
\trans{Who can eat another piece of cake?}
\a. Ich/Wir/Er.\hfill Nominative\is{Nominative case}
\b. *Mich/*Uns/*Ihn. \hfill Accusative\is{Accusative case}

The discussion on case checking\is{Case feature} in fragments in \citet{barton.progovac2005} focuses on the distinction between structural\is{Structural case} and default case\is{Default case}, but does not explicitly discuss fragments appearing in inherent\is{Inherent case} case, such as dative\is{Dative case} or genitive\is{Genitive case}. \citet[338--341]{progovac.etal2006} argue that in Serbian\il{Bosnian/Croatian/Serbian} non-nominative\is{Nominative case} case is inherent\is{Inherent case} case, because it is associated with a specific \texttheta-role. For instance, ``dative\is{Dative case} objects are typically associated with the theta-role of goal/recipient'' \citep[339]{progovac.etal2006}, so that dative\is{Dative case} has interpretable case features\is{Case feature} which do not need to be checked at all. Consequently, the nonsentential account\is{Nonsentential account} predicts inherent\is{Inherent case} case-marked fragments to be acceptable in an appropriate context and restricts anticonnectivity effects\is{Case connectivity} as in \ref{ex:barton.cake} to instances where DPs\is{Determiner phrase} receive structural\is{Structural case} case marking in complete sentences.

As a syntactic theory, \citepos{barton.progovac2005} nonsentential account\is{Nonsentential account} is primarily concerned with deriving the fragments that are grammatical in a language. The theory does not explain how fragments are licensed or how they are interpreted. As for licensing, \citet[89]{barton.progovac2005} suggest that recoverable expressions can be omitted, be it from linguistic\is{Context, linguistic} or extralinguistic context\is{Context, extralinguistic}. With respect to their interpretation, nonsentential account\is{Nonsentential account}s assume that this requires pragmatic enrichment. \citet{stainton2006} sketches a mechanism for this, which assumes that a salient nonlinguistic conceptual object%
%
\footnote{\citet[186--189]{stainton2006} terms it ``logical form'', but explicitly delimits his use of this term from that referring to the semantic representation of an utterance. \citeauthor{stainton2006} refers to some kind of conceptual nonlinguistic representation instead.}\afterfn%
%
is used to enrich the fragment to a complete proposition. The crucial difference between sentential and nonsentential theories of fragments is therefore whether the contextually salient objects licensing fragments are linguistic or only conceptual. Reflexes of linguistic structure, like structural case\is{Structural case} marking or movement restrictions\is{Movement restriction}, which are not contained in nonlinguistic representations, will be crucial for differentiating between both families of theories.
\is{Nonsentential account|)}

\section{Fragments as elliptical sentences}
\label{sec:theories-sentential}

Sentential accounts are motivated by the observation that fragments can be used for the same communicative purposes as full sentences despite their reduced form. For instance, the fragments in \Next appear to be a bare PP\is{Preposition phrase} \Next[a] or DP\is{Determiner phrase} \Next[b], but in both cases they are used to perform speech acts, just like their fully sentential counterparts in \NNext. If sentence mood is encoded in the left periphery \citep[see e.g.][]{rizzi1997}, this possibility of performing speech acts with fragments seems surprising, as there is no direct evidence for a left periphery in these utterances.

\ex. \a. [Passenger to taxi driver:] To the university, please!
    \b.  [Friends sharing a pizza:] Another slice?
    
\ex. \a. Take me to the university, please!
     \b. Would you like another slice?

Since \citet{morgan1973}, one explanation for this apparent mismatch between form and function has been that fragments do not really lack these projections, but that they are actually full sentences, parts of which are deleted by ellipsis.%
%
\footnote{This ``deletion'' is assumed to occur only on the phonological form (PF) that determines the acoustic realization of the sentences, but not on the logical form (LF) that determines their meaning in the terminology of \citet{chomsky1981}.}\afterfn%
%
This analysis has the advantage that, beyond mechanisms for licensing ellipsis (which are needed anyway in order to explain other instances of ellipsis), no amendments to syntactic theory are required in order to derive fragments. Their semantics can also be calculated compositionally as in regular sentences. Besides those theoretical advantages of sentential accounts, they are empirically supported by \textit{connectivity effects}\is{Case connectivity}, morphosyntactically and semantically identical behavior of a constituent as a fragment and within a full sentence. Such effects concern, for example, case marking (see Section \ref{sec:theories-predictions-case}) and binding \citep{merchant2004}.

Advocates of a sentential account do not agree on what exactly the underlying structure looks like, specifically on whether fragments involve an obligatory movement step\is{Movement and deletion account}, as suggested by \citet{merchant2004} or not, as \citet{reich2007}\is{In situ deletion account} argues. In what follows, I present the central ideas of each of these two approaches.

\subsection{In situ deletion} \label{sec:theories-insitu}
\is{In situ deletion account|(}

The most straightforward version of a sentential account derives fragments from regular sentences using those ellipsis mechanisms that are needed anyway to account for other types of ellipsis, such as gapping\is{Gapping} or sluicing\is{Sluicing}. \citet{reich2007} presents such an account.%
%
\footnote{\citeauthor{reich2007}'s theory is specifically motivated by a set of similarities between short answers\is{Fragment, short answer} and gapping\is{Gapping}. I restrict the presentation of this account to fragments.}\afterfn%
%
In a nutshell, he argues that all those parts of the utterance which are not focused are elided, and that the distribution of focus is determined by the relevant Question under Discussion (QuD, \citenob{roberts1996})\is{Question under Discussion}, which can be either implicit or explicit.

The restriction of ellipsis to non-focused expressions follows from \textit{ques\-tion-answer congruence}, the licensing condition that \citet{reich2007} imposes on ellipsis. \citet{reich2007} assumes a question-based discourse structure \citep[following][]{roberts1996}, so that the information structure\is{Information structure} of a sentence is determined by the immediately preceding QuD\is{Question under Discussion}. The QuD\is{Question under Discussion} can be either explicit or implicit. In \citeauthor{reich2007}'s examples of fragments it is explicit, because he discusses short answers\is{Fragment, short answer} and not discourse-initial fragments\is{Fragment, discourse-initial}. However, his theory can account for discourse-initial fragments\is{Fragment, discourse-initial} in the same way, since one of his main goals is a uniform analysis of fragments and gapping\is{Gapping}, where ellipsis is licensed by an implicit QuD\is{Question under Discussion}.

\citet{reich2007} resorts to \citepos{rooth1992} theory of question-answer congruence in order to formally define the relationship between question and answer. Following \citet{rooth1992}, Reich assumes that the meaning of a question is equivalent to the set of its potential answers, which can be obtained by replacing the \textit{wh}-phrase by an existentially bound variable. For an answer to be well-formed, it must obey the two constraints in \Next: First, \textsc{C-Answer} \Next[a] determines that the answer A must be included in the denotation of Q \citep[472]{reich2007}. Second, \textsc{F-Answer} determines that the answer's focus value, which, following \citet{rooth1992}, is calculated by replacing focused expressions with existentially bound variables, must be a superset of the denotation of the question \citep[472]{reich2007}.

\ex. \a. \textsc{C-Answer}: [[A]] $\in$ [[Q]].\label{ex:reich-qa-congruence-constraints}
\b. \textsc{F-Answer}: [[Q]] $\subseteq$ [[A]]\textsubscript{F} (and |[[Q]] $\cap$ [[A]]\textsubscript{F} | $\geq$ 2)

\citeauthor{reich2007} shows how these constraints explain in interaction why \NNext[a], but not \NNext[b] or \NNext[c] are information-structural\is{Structural case}\is{Information structure}ly well-formed answers to \Next. 

\ex. \a. Which student did John invite \textit{t}?\label{ex:reich-qa-example-q} \hfill \citep[472]{reich2007} 
\b. [[\ref{ex:reich-qa-example-q}]] = \{p; $\exists$x[x a student \& p $=$ that John invited x]\}

\ex. \a. John invited [\textit{Sue}]\textsubscript{F}.\label{ex:reich-theory-goodfoc} \hfill\citep[472]{reich2007}
\b. *[\textit{Sue}]\textsubscript{F} invited John.\label{ex:reich-theory-badfoc}
\c. \#John invited [\textit{Noam Chomsky}]\textsubscript{F}.\label{ex:reich-theory-badfoc-full}

\citet[472]{reich2007} defines the focus values for \Last[a] and \Last[b] as \Next. The focus value of \Last[a] in \Next[a] entails the denotation of the question \LLast[b] and thus conforms to \textsc{F-Answer}. Since the answer is included in the denotation of the question (provided that Sue is a student), \textsc{C-Answer} is also respected. In the case of \Last[b], its focus value in \Next[b] does not entail \LLast[b], therefore the answer is not congruent. The focus value of \Last[c] does entail \LLast[b], but \textsc{C-Answer} is violated, because \textit{Noam Chomsky} is not contained in the set of students so that \Last[c] is not included in the set of possible answers.

\ex. \a. [[\ref{ex:reich-theory-goodfoc}]]\textsubscript{F} = \{p; $\exists$x[x $\in$ D\textsubscript{e} \& p $=$ that John invited x]\}
\b. [[\ref{ex:reich-theory-badfoc}]]\textsubscript{F} = \{p; $\exists$x[x $\in$ D\textsubscript{e} \& p $=$ that x invited John]\}

Syntactically, \citet[472]{reich2007} links the question to the answer by assuming a squiggle operator $\sim$, which adjoins\is{Adjunct} to the highest node of the syntax tree of the answer, the CP\is{Complementizer phrase}. The operator introduces a variable $\Gamma$, which is coindexed with the question \Next. The operator presupposes that the answer is congruent to the question with respect to the two constraints in \ref{ex:reich-qa-congruence-constraints} discussed above. This notion of question-answer congruence is the licensing condition for ellipsis.%
%
\footnote{See \citet[474--477]{reich2007} for a comparison to \citepos{merchant2001} notion of e-givenness\is{E-givenness}.}\afterfn%

\ex. \a. [Which professor did John invite \textit{t} ?]\textsubscript{1} \hfill \citep[472]{reich2007}
\b. [John invited [Noam Chomsky]\textsubscript{F}]  $\sim\Gamma$\textsubscript{1}.

\citet{reich2007} defines ellipsis as PF-deletion, which can target only non-focused constituents, because the F-mark\is{F-marking} on focused ones requires them to receive a pitch accent \citep{selkirk1984}.%
%
\footnote{\citet{ott.struckmeier2016} sketch a very similar account but argue that it is the background of the utterance that can be deleted rather than the focus that cannot. They argue that this accounts better for the ability of German\il{German} modal particles (MPs) to survive ellipsis \Next, because MPs do not encode propositional meaning but the attitude of the speaker. According to \citeauthor{ott.struckmeier2016}, MPs neither belong to the focus nor to the background, so that the PF-deletion rules in \citet{reich2007} predict them to be omitted, while their own account does not.\label{fn-ott-struckmeier}

\ex. Who did Peter invite? \hfill\citep[227--228]{ott.struckmeier2016}
\ag. Er hat wohl seine \textit{Freunde} eingeladen.\\
    he.\textsc{nom} has \textsc{prt} his friends invited\\
    \trans{Presumably he has invited his friends.}
\bg. Wohl seine \textit{Freunde}.\\
    \textsc{prt} his friends\\
    \trans{Presumably his friends.}

}\afterfn%
%
Defining ellipsis as a post-spellout phenomenon, which applies to PF only, explains why it has no effects on LF. Technically, \citeauthor{reich2007} proposes that PF-deletion proceeds top-down starting at the sister node of \~$\Gamma$ (CP\is{Complementizer phrase}, the root node of the answer) according to the rules in \Next.

\ex. PF-deletion \hfill \citep[473]{reich2007}
\a. F-markers\is{F-marking} are upper bounds to PF-deletion.
\b. Maximize PF-deletion. \hfill (short answers\is{Fragment, short answer} and gapping\is{Gapping})

Taking the sentential answer \LLast[b] as a starting point, the application of these PF-deletion rules yields the fragment in \Next[a] as the only acceptable outcome of the operation. Preserving larger parts of the structure, e.g. \Next[b], is ruled out by the need to maximize PF-deletion spelled out in \Last[b]. \citeauthor{reich2007} suggests that this second clause of the rule is specific to short answers\is{Fragment, short answer} and gapping\is{Gapping}, whereas \Last[a] applies to all types of ellipses.

\ex. \a. [Noam Chomsky]\textsubscript{F}.
     \b. *Invited [Noam Chomsky]\textsubscript{F}.

The theory by \citet{reich2007} makes a series of testable predictions on the form of fragments. First, just like other sentential accounts, it predicts fragments to exhibit connectivity effects\is{Case connectivity} due to the unarticulated syntactic structure which they contain. Second, linguistic context\is{Context, linguistic}, and specifically the relevant QuD\is{Question under Discussion}, should have a strong effect on the form of fragments, because ellipsis is licensed only if the answer is congruent to the question. This is specifically expected when the QuD\is{Question under Discussion} is explicit, as it is in question-answer sequences or other adjacency pairs. Implicit QuD\is{Question under Discussion}s must be inferred by the hearer, who will try to accommodate a QuD\is{Question under Discussion} that is congruent to the fragment. If the speaker is cooperative, such a QuD\is{Question under Discussion} will always be accessible, because otherwise the speaker would prefer to utter a full sentence. Third, the form of fragments will also be constrained by focus projection rules, because only F-marked\is{F-marking} constituents survive ellipsis and the background is PF-deleted. Language-specific differences with respect to these rules will be reflected in different possible forms of fragments. Finally, \citet{reich2007} allows for discontinuous non-constituent fragments. This contrasts with most of the other accounts of fragments discussed in this section, which require fragments to be a single constituent. If multiple independent constituents are F-marked\is{F-marking} in a specific context, e.g. in case of multiple \textit{wh}-questions \Next, all of them must survive ellipsis.

\ex. [Waiter serving a couple their food:] Who ordered what?\\
Customer: She \sout{ordered} the pizza. \label{ex:theories-discontinuous-en-pizza}
\is{In situ deletion account|)}


\subsection{Movement and deletion} \label{sec:theories-movement}
\is{Movement and deletion account|(}

While \citet{reich2007} develops a unified account of fragments and gapping\is{Gapping}, \citet{merchant2004} observes a set of similarities between fragments and sluicing\is{Sluicing}. This motivates the extension of his theory of sluicing\is{Sluicing} \citep{merchant2001}, which derives sluices by regular \textit{wh}-movement followed by ellipsis of the remnant, to fragments. The central claim of the account is that all fragments undergo movement to a left-peripheral position before ellipsis applies to the remnant.

According to \citet{merchant2004}, ellipsis is triggered by a specific syntactic item, the E feature\is{E feature}. \citeauthor{merchant2004} argues that there are different varieties of E\is{E feature}, each of which is related to a specific type of ellipsis, such as sluicing \citep{merchant2001}, fragments \citep{merchant2004} and VP ellipsis\is{Verb phrase ellipsis} \citep{merchant2013}. Each variety of E\is{E feature} has its own lexicon entry, which encodes its syntactic, phonologic and semantic properties. To illustrate the idea, the derivation that \citeauthor{merchant2004} assumes for the sluice in \Next is given in Figure \ref{fig:merchant-sluicing}.%
%
\footnote{\citet[671]{merchant2004} notes that the assumption of independent lexical entries for the specific varieties of E\is{E feature} also accounts for crosslinguistic variation. For instance, he argues that German\il{German} has no VP ellipsis\is{Verb phrase ellipsis} because this language lacks the corresponding variety of E, while it shares with English\il{English} the varieties found in fragments and sluicing\is{Sluicing}.}\afterfn%
%

\ex. Abby was reading something, but I don’t know what $\langle$Abby was reading~\textit{t}$\rangle$.\label{ex:merchant-sluicing} \mbox{}\hfill \citep[670]{merchant2004}

\begin{figure}
 \begin{tikzpicture}[baseline]
   \tikzset{}
  \Tree[.CP \node(w){what\textsubscript{[wh]}}; [.C' C\textsuperscript{\textbf{[E]}}\textsubscript{[wh,Q]} [.TP \edge[roof]; {Abby was reading \textit{t}} ] ] ];
\draw[semithick, -stealth] (4.4,-3.3) to [bend left=60] (w);
\end{tikzpicture}

\caption{Derivation of the sluice in \ref{ex:merchant-sluicing} according to \citet[670]{merchant2004}.\label{fig:merchant-sluicing}}

\end{figure}
%
E\is{E feature} is always located on the head of a functional projection, like CP\is{Complementizer phrase} in Figure \ref{fig:merchant-sluicing}. The syntactic properties of E, which consist of a set of uninterpretable features, determine which head can host the feature. For instance, E\textsubscript{S}\is{E feature}, the E\is{E feature} feature found in sluicing, has the features [\textit{u}wh*, \textit{u}Q*] \citep[670]{merchant2004}. This ensures that it can be hosted only by heads that are [wh,Q] and that therefore can check these features, such as C in interrogatives. The variants of E\is{E feature} found in other types of ellipsis may have different feature specifications and are thereby restricted to other functional heads. \citet[671]{merchant2004} suggests that the varieties of E\is{E feature} are identical with respect to their phonology and semantics and differ only in these syntactic specifications. The phonological effect of the E feature\is{E feature} is that the complement of the head it is located on  remains unarticulated at PF. In \Last, this concerns the complete TP\is{Tense phrase} of the second conjunct in \Last. Both sentential accounts discussed so far, \citet{merchant2004} and \citet{reich2007}, agree that no syntactic structure is deleted during the derivation. Even though parts of it are unarticulated at PF, the unarticulated words are still present on LF. In \Last, this results in the wh-phrase being the only articulated word in the sluice, because it leaves the ellipsis site through \textit{wh}-movement to [Spec, CP\is{Complementizer phrase}].

According to \citet{merchant2004}, the licensing condition on omissions in fragments is \textit{e-givenness}\is{E-givenness}, which is included in the semantics of the E feature\is{E feature} \Next: E\is{E feature} requires the complement of the head hosting E\is{E feature} to be e-given\is{E-givenness}. E-givenness\is{E-givenness} is the identity condition licensing ellipsis in \citeauthor{merchant2001}'s theory and consists basically in a bidirectional givenness relation in the sense of \citet{schwarzschild1999}. An expression E counts as e-given\is{E-givenness} when it has a salient antecedent A which entails the existential closure of the focus value of A and vice versa. 

\ex. [[E]] = $\lambda$p: e-given (p) [p] \hfill \citep[672]{merchant2004}

The requirement for the complement of the head hosting E to be e-given\is{E-givenness} ensures that ellipsis is licensed only if there is a structural\is{Structural case}ly parallel antecedent available in context, and that it is blocked if there remains a constituent within the complement that is not e-given\is{E-givenness}. \Next exemplifies the mechanism for the sluicing\is{Sluicing} example in \LLast: The antecedent has the focus structure in \Next[a], whose existential closure \Next[b] is entailed by the sluice \Next[c]. As the existential closure of \Next[c] is identical to the one of the antecedent in \Next[b], the opposite relation also holds, so that the ellipsis in \LLast is licensed by e-given\is{E-givenness}ness.

\ex. \a. Abby was reading [something]\textsubscript{F}.
     \b. $\exists$x. Abby was reading x
     \c. Abby was reading [what]\textsubscript{F}.
     
\begin{figure}
\begin{tikzpicture}[baseline]
   \tikzset{}
  \Tree[.FP \node(w){[\textsubscript{DP} John]\textsubscript{2}}; [.F' F [.CP \node(i){[\textit{t}\textsubscript{2}]}; [.C' C\textsuperscript{\textbf{[E]}} [.TP \edge[roof]; {she saw \textit{t}\textsubscript{2}} ] ] ] ] ];
\draw[semithick, -stealth] (i) to [bend left=60] (w);
 \draw[semithick, -stealth] (4,-5.5) to [bend left=60] (i);
\end{tikzpicture}

\caption{Derivation of the fragment answer in \ref{ex:merchant-shortanswer} according to \citet{merchant2004}.\label{ex:merchant.structure-full}}
\end{figure}
%
\citet{merchant2004} extends this analysis to fragments. His theory accounts for discourse-initial fragments\is{Fragment, discourse-initial} (see below for details), but he focuses mostly on short answer fragments\is{Fragment, short answer} like \Next, for which he assumes the structure in Figure \ref{ex:merchant.structure-full}. Again, the E feature\is{E feature} is hosted by C in the left periphery, while the fragment is moved to the specifier of a functional projection FP immediately above CP\is{Complementizer phrase}. This movement operation proceeds cyclically through [Spec, CP\is{Complementizer phrase}].

\ex. \a. Who did she see? \hfill \citep[673]{merchant2004}
\b. John.\label{ex:merchant-shortanswer}

The major difference between sluicing\is{Sluicing} and fragments is that E\textsubscript{F}, the variety of E\is{E feature} found in fragments, and E\textsubscript{S} have different syntactic features, which are [\textit{u}C*, \textit{u}F] for E\textsubscript{F} and [\textit{u}wh*, \textit{u}Q*] for E\textsubscript{S}. The strong \textit{u}C* feature ensures that E\is{E feature} is located on a C head, while the weak \textit{u}F feature can be checked under Agree \citep[707]{merchant2004}, because weak features don't need to be checked locally according to the theory. Otherwise, the derivation is identical to sluicing\is{Sluicing}: After the fragment has been moved, ellipsis applies to the TP\is{Tense phrase}.

With respect to the landing site of the fragment in [Spec, FP], \citeauthor{merchant2004} avoids committing himself to an analysis of what kind of projection FP is. However, \citet[675]{merchant2004} tentatively suggests that it is a focus projection in the sense of \citet{rizzi1997}.%
%
\footnote{Elsewhere \citep[703]{merchant2004} he relates the movement operation that results in fragments to Clitic Left Dislocation\is{Clitic left dislocation} (CLLD\is{Clitic left dislocation}, \citenob{cinque1990}) rather than to focus. See Section \ref{sec:theories-predictions-focus} for a discussion.
}\footnote{The idea that FP is a focus projection is further developed by \citet{gengel2007}, who argues explicitly that movement in fragments\is{Movement and deletion account} occurs to check a [$+$contrastive] feature in [Spec, FP]. This conclusion might be too strong, since in languages like German\il{German} or English\il{English} fronting foci is possible yet marked. Specifically, as \citet{weir2014} notes and I discuss in greater detail below, object DP\is{Determiner phrase} fragments are acceptable in situations where fronting objects is definitely not.}\afterfn
%
Whether or not FP is a focus projection is highly relevant to the theory, because this would provide an explanation for why movement in fragments\is{Movement and deletion account} would occur at all. Since \citeauthor{merchant2004}'s theory is embedded in a minimalist framework \citep{chomsky1995}, movement cannot be optional, but is a \textit{last resort} operation that is mostly driven by the need to check strong features in a local (specifier-head) configuration. In \citepos{merchant2001} account of sluicing\is{Sluicing}, the \textit{wh}-phrase reaches [Spec, CP\is{Complementizer phrase}] through \textit{wh}-movement, which is driven by uninterpretable features of the \textit{wh}-phrase. Similarly, movement in fragments\is{Movement and deletion account} requires a trigger which the E feature\is{E feature} cannot provide: Its syntax, as defined above, contains only uninterpretable features that determine on which head it can appear. If FP was related to an information-structural\is{Structural case}\is{Information structure} concept such as focus or topic, an uninterpretable feature related to this notion could trigger movement in fragments\is{Movement and deletion account} independently from E, just like \citet{merchant2001} argues for sluicing\is{Sluicing}.

From an empirical perspective, \citet{merchant2004} requires evidence that fragments have actually moved. Since he analyzes movement in fragments\is{Movement and deletion account} as regular A'-movement, his theory predicts that the derivation of fragments is subject to movement restrictions\is{Movement restriction} that are observed in full sentences: Only those constituents that can be moved to [Spec, FP] and appear in a left-peripheral position in full sentences are predicted to be possible fragments. \citet{merchant2004} presents introspective data from different phenomena and languages in support of this prediction, some of which will provide the testing ground for his theory in my experiments.%
%
\footnote{See Section \ref{sec:theories-predictions-movement} for details.}\afterfn%
%

However, \citet{weir2014} shows that the assumption that structures presumably underlying movement and deletion\is{Movement and deletion account} are acceptable across the board is falsified even by simple examples such as \Next. The short answer fragment\is{Fragment, short answer} in \Next[a] is fine despite the ungrammaticality of the presumably underlying fronting structure \Next[b]. The acceptability of left dislocation in a sentence seems not to be necessarily related to the acceptability of the corresponding fragment.
 
\ex. What did you eat?  \hfill \citep[168]{weir2014} \label{ex:weir-english-fronting}
\a. Chips.
\b. *Chips, I ate \textit{t}.

In order to account for such data while maintaining the idea of movement and deletion\is{Movement and deletion account}, \citet{weir2014} claims that movement in fragments\is{Movement and deletion account} is a special type of movement\is{Exceptional movement account} which is restricted to elliptical utterances and which differs from movement in narrow syntax, i.e. before spell out. According to \citeauthor{weir2014}, this \textit{exceptional} movement\is{Exceptional movement account} is triggered by a clash between the prosodic properties of focused expressions, which are marked with a pitch accent, and the ellipsis site, which the E feature\is{E feature} requires to be silent. As \citet{weir2014} assumes a similar underlying structure as \citet{merchant2004} does (see Figure \ref{ex:merchant.structure-full}), that is, a regular sentence whose C head hosts the E feature\is{E feature}, the TP\is{Tense phrase} is marked for PF-deletion, but still contains the focused DP\is{Determiner phrase} \textit{John}. This conflict is solved by moving the focused expression(s) out of the ellipsis site and adjoin\is{Adjunct}ing them to CP\is{Complementizer phrase}.

Exceptional movement\is{Exceptional movement account} differs from narrow syntactic movement. First, it is not driven by feature checking; in fact, \citet[195]{weir2014} denies that there is a focus feature in English\il{English}.%
%
\footnote{Focus fronting is still acceptable in English\il{English} if the focus is contrastive in the sense of \citet{krifka2007}, i.e., when alternatives to the focused expression are given in context \Next.

\ex. \textit{Him} I invited, not \textit{her}. \hfill

This could still be accounted for by a more specific feature that appears only in contrastive contexts. However, even in cases as \Last, focus fronting does not seem to be obligatory, therefore the English\il{English} data require closer investigation, specifically if movement is to be assumed as non-optional (provided the relevant features are present).}\afterfn%
%
According to \citeauthor{weir2014}, exceptional movement\is{Exceptional movement account} is nevertheless a last resort operation, because there is no other way of saving the derivation from crashing due to the clash between focus and ellipsis at PF. Second, exceptional movement\is{Exceptional movement account} has no effect on the semantics of the utterance. This is in line with the observation that, unlike \citet{gengel2007} suggests, fragments are not necessarily contrastive. \citet[183]{weir2014} attributes the absence of semantic effects of exceptional movement\is{Exceptional movement account} to its application after spell-out and at PF only. He argues that this also explains why it is restricted to elliptical utterances: The only purpose of exceptional movement\is{Exceptional movement account} is to evacuate focused constituents from the ellipsis site, and because focused constituents can remain in situ in full sentences, exceptional movement\is{Exceptional movement account} is ruled out by economy considerations.

As the discussion in Section \ref{sec:theories-predictions-movement} will show, the assumption of exceptional movement\is{Exceptional movement account} notably complicates the empirical evaluation of the movement and deletion\is{Movement and deletion account} account, because the strong correlation between the acceptability of fronting and fragments is no longer predicted. Therefore, the experiments presented below test \citepos{merchant2004} version of the theory\is{Movement and deletion account} in the first place, but I also discuss the relevance for the exceptional movement\is{Exceptional movement account} theory whenever its predictions differ from \citet{merchant2004}.
\is{Movement and deletion account|)}


\subsection{Discourse-initial fragments under sentential accounts} \label{sec:theories-initial}
\is{Fragment, discourse-initial|(}

Up to this point, the theoretical discussion has focused mostly on short answer fragments\is{Fragment, short answer}, although I argued in the introduction that the most uncontroversial instances of fragments are discourse-initial\is{Fragment, discourse-initial} ones. Discourse-initial fragments challenge any sentential account of fragments: Given the licensing conditions of ellipsis discussed so far, ellipsis requires an antecedent, and in examples such as \Next no such antecedent seems to be available. Nonsentential accounts\is{Nonsentential account} do not face this problem, as they derive the propositional meaning of fragments by pragmatic inference. Since some of the experiments presented in this work rely on discourse-initial\is{Fragment, discourse-initial} fragments, in what follows I discuss how sentential accounts can account for these utterances and some of their properties. In particular, I argue that in those situations where discourse-initial\is{Fragment, discourse-initial} fragments are used, an antecedent that licenses ellipsis can be retrieved from extralinguistic context\is{Context, extralinguistic}. This facilitates a unified sentential account of fragments with and without overt antecedents. 

\ex. \label{ex:fragments-dilang}
\a. [Passenger to taxi driver:] To the university, please! \label{ex:fragments-dilang-uni}
\b. [Customer to barista:] A coffee, please! \label{ex:fragments-dilang-coffee}
\c. [Taking a postcard out of the post box:] From John!\label{ex:fragments-dilang-letter}

Sentential accounts that assume a QuD-based\is{Question under Discussion} model of context \citep{reich2007, weir2014} can explain the utterances in \Last by assuming implicit QuDs\is{Question under Discussion}, which are evoked by the extralinguistic context\is{Context, extralinguistic} and which are appropriate antecedents \Next. For instance, a pedestrian approaching a taxi is likely to ask for a ride and a guest in the coffee shop is very likely to order a drink or some food. 

\ex.  
\a. $\langle$Where shall I take you?$\rangle$\label{ex:uni-taxi-qud}\\
	   \sout{Take me} to the university, please!
     \b. $\langle$What would you like to have?$\rangle$\\
	   \sout{I'd like to have} a coffee, please!
	   
\citet[1852]{reich2011} notes that such discourse-initial fragments\is{Fragment, discourse-initial} are \textit{indeterminate}, that is, there are several possible paraphrases of the missing material. For instance, the answer in \LLast[a] could be understood as \textit{Take me to the university!}, \textit{I'd like to go to the university.} or \textit{Drive to the university!}. \citeauthor{reich2011} takes this to be a defining feature of what he calls situation-based ellipsis (\textit{s-ellipsis})\is{Situation-based ellipsis}, for the resolution of which the hearer must resort to extralinguistic context\is{Context, extralinguistic}. In contrast, antecedent-based \textit{a-ellipses}\is{Antecedent-based ellipsis}, which have a linguistic antecedent\is{Context, linguistic} (e.g. gapping\is{Gapping}, right node raising\is{Right node raising} and VPE\is{Verb phrase ellipsis}), can be unambiguously resolved \Next.

\ex. John goes to the university and Mary \sout{goes} to the pub.

According to \citet[1852]{reich2011}, indeterminacy suggests that, unlike short answers\is{Fragment, short answer}, discourse-initial fragments\is{Fragment, discourse-initial} are syntactically genuine nonsententials. This implies a non-uniform analysis to fragments: If they have a linguistic antecedent\is{Context, linguistic}, like an explicit QuD\is{Question under Discussion} in the case of short answers\is{Fragment, short answer}, they are elliptical, and if they do not, they are nonsentential. However, in \citet{reich2007}, he notes that also in antecedent-based ellipsis\is{Antecedent-based ellipsis} like gapping\is{Gapping}, the focus structure of the second conjunct can vary. If there is wide focus on the first conjunct, different focus structures \Next and henceforth different omission patterns \NNext are possible in the second conjunct depending on which implicit QuD\is{Question under Discussion} is assumed.\largerpage[2]

\ex. \a. [John gave a book to \textit{Sue}]\textsubscript{F}, and John gave [a \textit{baseball}]\textsubscript{F} [to \textit{Bill}]\textsubscript{F}. 
    \b. [John gave a book to \textit{Sue}]\textsubscript{F}, and [\textit{Peter}]\textsubscript{F} gave a book [to \textit{Ann}]\textsubscript{F}.\\ \mbox{}\hfill\hfill\citep[478]{reich2007}

\ex. \a. [John gave a book to \textit{Sue}]\textsubscript{F}, and [a \textit{baseball}]\textsubscript{F} [to \textit{Bill}]\textsubscript{F}. 
    \b. [John gave a book to \textit{Sue}]\textsubscript{F}, and [\textit{Peter}]\textsubscript{F} [to \textit{Ann}]\textsubscript{F}.\\ \mbox{}\hfill\hfill\citep[478]{reich2007}
    
\citet[477]{reich2007} argues that in such cases, ``a complete set of possible QuDs\is{Question under Discussion} [\dots] is reconstructed, from which the speaker chooses exactly one as the most salient.'' The hearer then has to figure out which QuD\is{Question under Discussion} out of this set is the one that the speaker had in mind. Besides extralinguistic context\is{Context, extralinguistic}, a strong cue toward the QuD\is{Question under Discussion} intended by the speaker is the form of the utterance: If only focused expressions survive gapping\is{Gapping}, \Last[a] will accommodate a QuD\is{Question under Discussion} as \textit{What did John give to whom?}  and \Last[b] \textit{Who gave a book to whom?}. This reasoning applies equally to fragments: If the hearer must infer which QuD\is{Question under Discussion} the speaker had in mind from context and the form of the elliptical utterance in gapping\is{Gapping}, there is no reason to assume that she is not able to infer the QuD\is{Question under Discussion} in case of fragments. The set of potential QuDs\is{Question under Discussion} might often be more restricted in case of gapping\is{Gapping} than in discourse-initial fragments\is{Fragment, discourse-initial} by the first conjunct, so that there might be a quantitative difference between the size of the set of possible QuDs\is{Question under Discussion} reconstructed in gapping\is{Gapping} and fragments. However, there is not necessarily a categorical difference between both constructions.

Furthermore, from a psycholinguistic perspective, the theories of fragments discussed so far are production accounts. What matters primarily to them is the speaker's perspective: Ellipsis is licensed if there is a QuD\is{Question under Discussion} in context which the speaker believes to be sufficiently salient. Since the hearer is aware of this, she knows that there must be such a contextually salient QuD\is{Question under Discussion} as soon as she realizes that the speaker's utterance is elliptical. If the speaker is cooperative or at least has the intention to get his message across \citep{grice1975, sperber.wilson1995}, he will only use a fragment when he believes that the QuD\is{Question under Discussion} is relatively easy to retrieve. For instance, if there is a high risk of being misunderstood due to several equally likely competing QuDs\is{Question under Discussion} that differ in meaning, the full sentence will be preferred. In fact, this is supported by the experiments on script knowledge\is{Script knowledge} in Chapter \ref{sec:chapter-infotheory-experiments} of this book. Consequently, indeterminacy of the meaning of a QuD\is{Question under Discussion} does not impede communication if fragments are used only when the QuD\is{Question under Discussion} is relatively predictable. Even if the meaning of the QuD\is{Question under Discussion} is retrievable, its lexicalization is not necessarily, as the set of semantically similar QuDs\is{Question under Discussion} listed above for the taxi example showed. However, communication can succeed even if the hearer fails to recover exactly the lexicalization of the QuD\is{Question under Discussion} that the speaker had in mind, as long as the recovered QuD\is{Question under Discussion} causes the hearer to perform the intended action. No matter which of the paraphrases the driver chooses in order to enrich the fragment in \ref{ex:uni-taxi-qud}, she will still carry the passenger to the university.

The difference between implicit and explicit QuDs\is{Question under Discussion} is furthermore specifically relevant to the hearer's perspective. If the speaker chooses to produce a fragment, he must have a particular QuD\is{Question under Discussion} in mind, be it implicit or explicit. Consequently, from his perspective there is no categorical difference between explicit QuDs\is{Question under Discussion} (in case of short answers\is{Fragment, short answer}) and implicit ones (in case of discourse-initial fragments\is{Fragment, discourse-initial}). Obviously, this changes from the perspective of the hearer, who has to reconstruct the missing material, because this task is  facilitated by a QuD\is{Question under Discussion}. Still though, as I noted above, a speaker who wants to get her message across will only choose to use the fragment if the effort\is{Processing effort} required for the hearer to infer the intended QuD\is{Question under Discussion} is reasonably small. This is confirmed by my experiments \ref{exp:scripts-rating} and \ref{exp:scripts-production}, which show that in particular predictable, that is, easily recoverable, words are omitted.

The assumption that the resolution of ellipsis might require some degree of inference is further supported by research on relatively uncontroversial instances of antecedent-based ellipsis\is{Antecedent-based ellipsis}. For instance, both gapping\is{Gapping} and VPE\is{Verb phrase ellipsis} allow for mismatches between the antecedent and target of ellipsis. The VPE\is{Verb phrase ellipsis} example in \Next[a] requires the hearer to reconstruct an active VP\is{Verb phrase} \textit{look into this problem} given a passive antecedent. Similarly, in gapping\is{Gapping}, hearers can reconstruct a plural verb given a singular antecedent \Next[b]. Therefore, rather than being a copy-and-paste process \citep{frazier.clifton2001}, the reconstruction of elided material seems to be a task that requires retrieving the omitted material from available contextual evidence and that becomes more effortful the more antecedent and target differ \citep{arregui.etal2006}.

\ex. \a. This problem was to have been looked into, but obviously nobody did $\langle$look into this problem$\rangle$.\hfill \citep[548]{kehler2002}
     \b. They are going to Chicago, and I $\langle$am going$\rangle$ to San Francisco.\\ \mbox{}\hfill \citep[755]{pullum.zwicky1986}

Taken together, a uniform analysis of discourse-initial\is{Fragment, discourse-initial} and short answer fragments\is{Fragment, short answer} as elliptical sentences is possible. Just like fragments, some antecedent-based ellipses\is{Antecedent-based ellipsis} require inferential reasoning about the omitted material, so that the difference between short answers and dis\-course-initial fragments is a gradual one: In the case of short answers, the missing material is explicitly given, so that its retrieval will be easier than in discourse-initial\is{Fragment, discourse-initial} fragments (on average). If such a uniform account explains the data equally well, it is simpler than a mixed account\is{Mixed account} that distinguishes between discourse-initial\is{Fragment, discourse-initial} fragments and short answers and consequently to be adopted unless there is evidence against it.

Unlike \citet{reich2007} and \citet{weir2014}, \citet{merchant2004} does not rely on the concept of QuD\is{Question under Discussion}, but his theory can also account for discourse-initial fragments\is{Fragment, discourse-initial}. Since e-givenness\is{E-givenness}, which licenses fragments in his theory, operates on semantic representations, fragments in principle require a salient linguistic antecedent. \citeauthor{merchant2004}'s account distinguishes between fragments that occur in highly conventionalized contexts like the taxi example \ref{ex:fragments-dilang-uni} and those which do not, like \Next[a]. In the case of highly predictive contexts, he notes that speakers have strong expectations about what is likely to happen and to be said in such contexts. \citet[730--731]{merchant2004} argues that this \textit{script knowledge}\is{Script knowledge} in the sense of \citet{schank.abelson1977}%
%
\footnote{See Section \ref{sec:infotheory-scripts} for details on the concept of script and a discussion of its psychological effects.}\afterfn%
% 
can make specific linguistic expressions manifest in the sense of relevance theory\is{Relevance theory}, that is, ``capable [\dots] of representing it mentally and accepting its representation as true or probably true'' \citep[39]{sperber.wilson1986}. Such manifest linguistic objects can then serve as antecedents and license omission if they result in parts of the utterance being e-given\is{E-givenness}\is{Context, extralinguistic}. For non-conventionalized cases, \citet[722--727]{merchant2004} argues that context can still make entities and concepts like the letter and its origin in \Next[a] manifest. This licenses the ellipsis of very basic deictic expressions, as the predicate \textit{do it} for actions and pronouns for entities in the structure in \Next[b] which he assumes for the fragment in \Last. 

\ex. \a. [Bill walks into the living room waving a postcard. He says:]
	   \sout{The postcard is} from John!
\b. From John \sout{it is}.

The deictic expressions are then resolved from context\is{Context, extralinguistic}, just like they are in regular sentences. \citet[722]{merchant2004} argues that the assumption that the unarticulated structure in such fragments consists in the minimally required and semantically less specified expressions also accounts for the indeterminacy of discourse-initial fragments\is{Fragment, discourse-initial}. Pronouns can be often paraphrased with various complex DPs\is{Determiner phrase}, so that this apparent property of fragments turns out to be just a more general property of deictic expressions. 

Taken together, the assumption that ellipsis can be licensed by extralinguistic context\is{Context, extralinguistic} provides an empirically necessary extension of the sentential account of fragments to (apparently) discourse-initial fragments\is{Fragment, discourse-initial}. Nonsentential accounts\is{Nonsentential account}, which rely on pragmatic inference in any case, do not face this problem, but might resort to the same mechanisms (e.g. script knowledge\is{Script knowledge} and implicit QuDs\is{Question under Discussion}) in order to explain how fragments are interpreted.

\is{Fragment, discourse-initial|)}

\section{Fragments as ungrammatical utterances}  \label{sec:theories-ungrammatical}
\is{Ungrammaticality of fragments|(}

All of the theories discussed so far agree on the assumption that fragments are grammatical objects, be it by postulating mechanisms that license and trigger ellipsis or by modifying the theory in order to allow for nonsententials to be a well-formed output of syntax. This view is challenged by \citet{bergen.goodman2015}\is{Ungrammaticality of fragments}, who sketch a game-theoretic\is{Game theory} account of fragment usage and argue that fragments are actually ungrammatical, but that under some circumstances they might still be the preferred means of communicating a message. In simplified terms, \citeauthor{bergen.goodman2015} argue that hearers use a repair mechanism that helps to figure out the omitted parts of the utterance in order to infer the intended meaning from fragments. Such a mechanism is needed independently of fragments in order to deal with utterances that are corrupted by e.g. acoustic noise\is{Noisy channel}. Hearer and speaker are mutually aware of this possibility, so that the usage of a fragment and the subsequent inference about the intended meaning is more economic than that of a full sentence if the missing parts of the utterance are relatively easy to retrieve. Since this theory is highly usage-oriented and closely related to information theory\is{Information theory}, I discuss it in greater detail in Section \ref{sec:infotheory-uid}. For the time being, what matters is the presumed ungrammaticality of fragments.

Such an account predicts that there are no restrictions on the form of fragments, provided the chosen utterance is the most suitable for communicating a message in a specific context. In contrast, syntactic accounts categorically distinguish grammatical from ungrammatical fragments. Even though the acceptability of a grammatical fragment will also be determined to a large extent by context, there are grammatical principles that cannot be overridden. For instance, the movement restrictions\is{Movement restriction} that \citet{merchant2004} presents as evidence for his theory, or the requirement for fragments to be maximal XPs according to \citet{barton.progovac2005}, impose restrictions on the form of fragments that are context-independent.

It is hardly possible to empirically confirm the claim that fragments are ungrammatical objects, but it can be falsified by evidence for context-independent constraints on the form of fragments. The empirical picture is mixed: On the one hand, as my experiment \ref{exp:pstranding-defaultcase} will show, gradual differences in acceptability among ungrammatical fragments are in line with \citepos{bergen.goodman2015} account. On the other hand, there are grammatical constraints that override constraints that otherwise license omissions. For instance, \citet{lemke.etal2017} find that the omission of articles\is{Article omission} in German\il{German} newspaper headlines is subject to processing considerations that are related to those in \citet{bergen.goodman2015}\is{Game theory}. In standard German\il{German} however, including newspaper text, article omission\is{Article omission} is restricted only to very specific contexts (e.g. predicative and plural nouns). If grammaticality did not matter at all, the same amount of omissions would be expected in similar contexts across text types which are less constrained by normative pressures, like colloquial speech.

This objection concerns only the presumed ungrammaticality of fragments, but not the mechanism that \citet{bergen.goodman2015}\is{Ungrammaticality of fragments} propose in order to account for the speaker's choice between a fragment and a sentence. Such a mechanism is needed in any case,%
%
\footnote{See Section \ref{sec:fragments-game} for a more detailed sketch of such an account.}\afterfn%
%
because also when fragments are assumed to be grammatical the speaker must somehow decide on the form of the utterance. An account of this choice process is beyond the scope of generative\is{Generative grammar} syntactic theory, which does not attempt to model production preferences. Taken together, the main difference between the account in \citet{bergen.goodman2015}\is{Ungrammaticality of fragments} and syntactic theories of fragments concerns the question of whether the set of utterances that are possible in a situation is somehow constrained by syntax or derived by arbitrary omissions.%
%
\footnote{Note that \citet{bergen.goodman2015} do not discuss how this set of possible utterances is derived and present only a very simple example. Specifically, it is unclear whether they assume that fragments are genuine nonsententials or whether they take grammatical utterances as point of departure and generate the alternative by applying arbitrary ellipses to them.}\afterfn%
%
\is{Ungrammaticality of fragments|)}


\section{Testable predictions of theories of fragments}\largerpage
\label{sec:theories-predictions}

The accounts discussed in the preceding section assume very different derivations of fragments and syntactic structures underlying them. However, all of them are designed to account in principle for the same acceptability data reported in the literature. Therefore, in order to evaluate their predictions empirically, it is necessary to isolate phenomena with respect to which these predictions differ. In this section, I present four such phenomena, which might provide a testing ground for the theories: case connectivity effects\is{Case connectivity}, discontinuous fragments, information-structural\is{Structural case}\is{Information structure} restrictions, and movement restrictions\is{Movement restriction}. As will turn out, only case connectivity and movement evidence will be useful to distinguish between the theories. With respect to constituency and focus marking, the theories either coincide for independent reasons or they do not make precise empirical predictions with respect to them at all.

\subsection{(Anti)connectivity effects: Case marking} \label{sec:theories-predictions-case}

\is{Case connectivity|(}
Both those authors who defend a sentential analysis of fragments and those supporting a nonsentential account\is{Nonsentential account} consider case connectivity effects\is{Case connectivity} to be an important diagnostic for the syntax of fragments. \citet[676--679]{merchant2004} presents evidence for connectivity effects\is{Case connectivity} from a diverse sample of languages. His general observation is that DP\is{Determiner phrase} short answers\is{Fragment, short answer} like the German\il{German} example \Next receive the same case morphology as their counterparts in complete sentences: Just like the \textit{wh}-phrase in the question, the DP\is{Determiner phrase} in the short answer has to receive accusative\is{Accusative case} case marking \Next[a], whereas the dative\is{Dative case} \Next[b] is ungrammatical. From the sententialist perspective, case connectivity provides indirect evidence for unarticulated structure: In minimalism\is{Minimalist program}, at least non-default case\is{Default case} must be checked in a local syntactic configuration between the DP\is{Determiner phrase} and another head, so the grammaticality of such a case marking indicates the presence of an unarticulated licensor. For instance, in \Next[a] accusative\is{Accusative case} must be checked by a verb, so that the acceptability of case marking in the absence of an overt verb suggests that the structure contains that verb, but that it has been PF-deleted in the course of the derivation. This pattern is reversed in \NNext, where the verb in the question requires dative\is{Dative case}. In this case, a dative\is{Dative case}, but not accusative\is{Accusative case}, DP\is{Determiner phrase} fragment is acceptable.

\exg. Wen sucht Hans?\label{ex:merchant.caseconn-acc}\\
who.\textsc{acc} seeks Hans\\
\trans{Who is Hans looking for?} \hfill \exsourceraised{\citep[677]{merchant2004}}
\ag. Den Lehrer.\label{ex:merchant.caseconn-acc-subex}\\ 
the.\textsc{acc} teacher \mbox{}\\
\bg. *Dem Lehrer.\\
the.\textsc{dat} teacher\\

\exg. Wem folgt Hans?\label{ex:merchant.caseconn-dat}\\
who.\textsc{dat} follows Hans\\
\trans{Who does Hans follow?} \hfill \exsourceraised{\citep[677]{merchant2004}}
\ag. Dem Lehrer. \label{ex:merchant.caseconn-dat-subex}\\ 
the.\textsc{dat} teacher\\
\bg. *Den Lehrer.\\
the.\textsc{acc} teacher\\

Case connectivity effects\is{Case connectivity} thus support unarticulated structure in fragments. In contrast, as I already noted above, anticonnectivity effects\is{Case connectivity}, that is, mismatching case morphology between fragments and full sentences, support nonsentential accounts\is{Nonsentential account}. An example for anticonnectivity is the ungrammaticality of nominative\is{Nominative case} DP\is{Determiner phrase} fragments in the English\il{English} example \Next, although this case is required in the corresponding full sentence. Recall that \citet{barton.progovac2005} explained these data with the inability of fragments to exhibit structural case\is{Structural case} morphology under the assumption that nominative\is{Nominative case} is structural case\is{Structural case} in English\il{English}.

\ex. Who can eat another piece of cake? \hfill \citep[77]{barton.progovac2005} \label{ex:barton.case-rep}
\a. ?*I/?*We/?*He/?*She.
\b. Me/Us/Him/Her.

The introspective data reported in the literature is contradictory. \citeauthor{barton.progovac2005}'s example \Last exhibits anticonnectivity\is{Case connectivity}, but \citeauthor{merchant2004}'s German\il{German} examples in \ref{ex:merchant.caseconn-acc} and \ref{ex:merchant.caseconn-dat} seem to evidence case connectivity. Both sententialists and nonsententialists present explanations for at least part of the data that seems to contradict their respective theories. In what follows, I first review how the nonsentential account\is{Nonsentential account} could account for examples like \ref{ex:merchant.caseconn-acc-subex} and conclude that only the sentential account explains such connectivity effects. This rests on the assumption that accusative\is{Accusative case} is structural case\is{Structural case} in German\il{German}. This is supported by tests that \citet{progovac2006} use to show that Serbian\il{Bosnian/Croatian/Serbian} accusative\is{Accusative case} is inherent\is{Inherent case} case, but which yield the opposite result when they are applied to German. I then discuss how sentential accounts can deal with anticonnectivity\is{Case connectivity} effects as in \LLast. 

\subsubsection{Nonsentential accounts and connectivity}

As discussed above, \citet{progovac.etal2006} distinguish between default\is{Default case}, structural\is{Structural case} and inherent\is{Inherent case} case. In contrast to structural\is{Structural case} case feature\is{Case feature}s, which are uninterpretable, inherent\is{Inherent case} case feature\is{Case feature}s can and must be interpreted by semantics, because inherent\is{Inherent case} case is related to a specific \texttheta-role. The relatively uncontroversial claim that dative\is{Dative case} is inherent\is{Inherent case} case by \citet[339]{progovac.etal2006} explains the acceptability of dative\is{Dative case} DP\is{Determiner phrase} fragments as \ref{ex:merchant.caseconn-dat-subex}. In contrast, fragments that appear in structural\is{Structural case} case must be ungrammatical according to the nonsentential account\is{Nonsentential account}, because structural\is{Structural case} case feature\is{Case feature}s must be checked and structural\is{Structural case} case-marked DP\is{Determiner phrase} fragments lack an appropriate verbal head under a nonsentential analysis. This prediction is challenged by the acceptability of accusative\is{Accusative case} DP\is{Determiner phrase} fragments in German\il{German} \ref{ex:merchant.caseconn-acc-subex}, under the assumption that accusative\is{Accusative case} is structural\is{Structural case} case in German\il{German}.

\citet{progovac.etal2006} however argue that accusative\is{Accusative case} is not necessarily structural case\is{Structural case} in languages where nominative\is{Nominative case} is the default case\is{Default case}. They exemplify this idea for Serbian\il{Bosnian/Croatian/Serbian} and present three arguments that attempt to show that Serbian\il{Bosnian/Croatian/Serbian} accusative\is{Accusative case} is also inherent\is{Inherent case}, interpretable case. First, they note that an accusative\is{Accusative case} DP\is{Determiner phrase} ``is typically in fact a theme/patient'' \citep[339]{progovac.etal2006} and exclude a set of other \texttheta-roles for accusative\is{Accusative case}s (agent, goal/recipient, instrument, locative). As discussed above, the association with a specific \texttheta-role is the central criterion for categorizing a specific case as inherent\is{Inherent case} case. Second, they argue that in Serbian\il{Bosnian/Croatian/Serbian} accusative\is{Accusative case} contributes to semantics and present the contrast in \Next as evidence that accusative\is{Accusative case} has a universal quantificational meaning, while genitive\is{Genitive case} quantifies existentially. Finally, they state that, in contrast to English\il{English}, Serbian\il{Bosnian/Croatian/Serbian} accusative\is{Accusative case} objects do not always appear adjacent to the verb. This observation is not discussed in greater detail but is probably intended to suggest that no specific syntactic configuration is required for accusative\is{Accusative case} to be licensed.

\ex. \label{ex:serbian-case-quantification}
\ag. Dodaj vodu.\\
add water.\textsc{acc}\\
\trans{Add (all) the water.} \hfill \exsourceraised{\citep[340]{progovac.etal2006}}
\bg. Dodaj vode.\\
add water.\textsc{gen}\\
\trans{Add (some) water.}

Whether these arguments are empirically correct for Serbian\il{Bosnian/Croatian/Serbian} is beyond the scope of this book. However, if they applied to German\il{German} as well, this could derive grammatical accusative\is{Accusative case} DP\is{Determiner phrase} fragments like \ref{ex:merchant.caseconn-acc-subex} under a nonsentential account\is{Nonsentential account} and consequently undermine the status of case connectivity effects\is{Case connectivity} as evidence for unarticulated structure in fragments. 

Nevertheless, the diagnostics used by \citet{progovac.etal2006} to analyze the Serbian\il{Bosnian/Croatian/Serbian} accusative\is{Accusative case} as structural case\is{Structural case} do not yield the same result for the German\il{German} accusative\is{Accusative case}. The \texttheta-role argument that accusative\is{Accusative case} DP\is{Determiner phrase}s are typically themes or patients probably holds in German\il{German} too. For instance, it seems reasonable to assume that a random accusative\is{Accusative case} DP\is{Determiner phrase} drawn from a corpus\is{Corpus} is more likely to be a patient than a random nominative\is{Nominative case} DP\is{Determiner phrase} is. However, it is unclear how the gradual likelihood of being a patient can be aligned with the categorical distinction between structural\is{Structural case}, inherent\is{Inherent case} and default case\is{Default case} upon which \citepos{barton.progovac2005} theory relies. This probably concerns the Serbian data as well. The remaining two arguments for analyzing Serbian\il{Bosnian/Croatian/Serbian} accusative\is{Accusative case} as inherent\is{Inherent case} case do not hold in German\il{German}. Quantification relies on the presence of the definite article in German%
%
\footnote{This is evidenced by the German\il{German} counterparts of \ref{ex:serbian-case-quantification} in \Next. The adjective has accusative\is{Accusative case} case morphology in both examples but the meaning depends on the presence of the definite article.

\ex. \ag. Geben Sie den frischen Zitronensaft hinzu.\\
	  give.\textsc{imp} you the.\textsc{acc} fresh.\textsc{acc} lemon.juice to\\
	  \trans{Add (all of) the fresh lemon juice.}
    \bg. Geben Sie frischen Zitronensaft hinzu.\\
	  give.\textsc{imp} you fresh.\textsc{acc} lemon.juice to\\
	  \trans{Add (some of the) fresh lemon juice.}
	  
}\afterfn%
%
and accusative\is{Accusative case} DPs\is{Determiner phrase} still appear adjacent to the verb in the unmarked word order\is{Word order} in German\il{German}. This suggests that at least for German\il{German} the analysis of accusative\is{Accusative case} as structural case\is{Structural case} is correct. Consequently, if empirically confirmed, the acceptability of the accusative\is{Accusative case} DP fragments in German\il{German} challenges \citepos{barton.progovac2005} theory because it does indeed predict anticonnectivity effects\is{Case connectivity} in that case.%
%
\footnote{Having defined Serbian\il{Bosnian/Croatian/Serbian} accusative\is{Accusative case} as inherent\is{Inherent case} case, \citet[340]{progovac.etal2006} argue that nominative\is{Nominative case} DP\is{Determiner phrase} fragments are degraded as answers to questions asking for an accusative\is{Accusative case} \Next for pragmatic reasons. They claim that the accusative\is{Accusative case} short answer\is{Fragment, short answer} is preferred over the nominative\is{Nominative case} one because it is more informative. This cannot explain the German\il{German} data: If accusative\is{Accusative case} is structural case\is{Structural case}, the only grammatical short answer\is{Fragment, short answer} is the nominative\is{Nominative case} one. The glosses and the grammaticality judgement in \Next[b] were added by  R.L. based on the text accompanying the example in \citet{progovac.etal2006}.\label{fn:defaultcase-acc-pragmatics}

\ex. \label{ex:accusative-inherent-progovac} \ag. Koga je Ana posetila?\\
who.\textsc{acc} is Ana visited? \\
\trans{Who did Ana visit?}\hfill \exsourceraised{\citep[340]{progovac.etal2006}}
\bg. \#Vera!\\
Vera.\textsc{nom}\\

}\afterfn%
%
Experiment \ref{exp:case} disconfirms this prediction.

\subsubsection{Sentential accounts and anticonnectivity}
Anticonnectivity effects\is{Case connectivity} have been presented as evidence against sentential accounts, but at least some of the data can be explained by these theories. With respect to English\il{English} pronominal short answers\is{Fragment, short answer} discussed above, \citet[700--704]{merchant2004} reports similar data for Greek\il{Greek}, Dutch\il{Dutch}, German\il{German} and French\il{French}, which, aside from cross-linguistic differences, are characterized by formal differences between the pronouns found in fragments and those in full sentences. For instance, the French\il{French} data in \Next show that in fragments only the strong, tonic pronoun \textit{moi} is acceptable \Next[a], whereas the clitic \textit{me} must be used in regular sentences \Next[b].

However, \citeauthor{merchant2004} notes that in the left periphery only those pronouns that appear in fragments are acceptable. For instance, in Clitic Left Dislocation\is{Clitic left dislocation} (CLLD\is{Clitic left dislocation}) in French\il{French} \NNext[a] or hanging topic\is{Topic, hanging} left dislocation (HTLD)\is{Topic, hanging} in English\il{English} \NNext[b] the form of the pronoun matches the one found in short answers\is{Fragment, short answer}.

\exg. Il voulait qui?\\
he wanted who\\
\trans{Who did he want?}\hfill \exsourceraised{\citep[701]{merchant2004}}
\ag. Moi/*Me.\\
me.strong/me.weak\\
\bg. Il me/*moi voulait.\\  
he me.weak/strong wanted\\
\trans{He wanted me.}

\ex. \ag. Moi/*Me, il me voulait.\\
me.strong/weak he me wanted\\
\trans{Me, he wanted.}\hfill \exsourceraised{\citep[702]{merchant2004}}
\b. Me/*I, I watered the plants. \hfill \citep[703]{merchant2004}

\citeauthor{merchant2004} argues that this is empirically in line with his theory, which assumes structures such as \Last to be the source of fragments. However, he is reluctant to assume HTLD\is{Topic, hanging} as the structure underlying fragments, because, unlike fragments, this construction is insensitive to island\is{Island}s. Instead, he suggests that the formal restrictions on pronominal fragments are due to the fact that weak pronouns cannot be focused. \citet{merchant2004} takes this as evidence for the movement component of his theory, but the observation that only pronouns which can be focused are possible fragments is perfectly in line with \citepos{reich2007} in situ ellipsis account\is{In situ deletion account}: \Next[a] shows that the English\il{English} pronoun \textit{me} can receive a pitch accent marking narrow or contrastive focus\is{Focus, contrastive}, whereas in French\il{French}, the tonic pronoun \Next[b] can but the clitic \Next[c] cannot. The deletion of all non-focused constituents under a semantic identity condition as proposed by \citet{reich2007} consequently yields the same pattern as movement and deletion\is{Movement and deletion account}.\largerpage%
%
\footnote{Some authors account for mismatches between sentences and fragments with respect to preposition omission\is{Preposition omission} by deriving fragments from clefts\is{Cleft} (see e.g. \citet{rodrigues.etal2009} for Spanish\il{Spanish} and Brazilian Portuguese\il{Portuguese} and \citet{vancraenenbroeck2010} for English\il{English}). Under such an analysis, according to which accusative\is{Accusative case} in fragments evidences a connectivity effect\is{Case connectivity} than an anticonnectivity effect\is{Case connectivity}, \ref{ex:barton.cake} would be derived from a structure like \Next.

\ex. A: Who can eat another piece of cake?\\
     B: \sout{It's} me \sout{who can eat another piece of cake}.

}\footnote{The French data in \TextNext are also compatible with the nonsentential account\is{Nonsentential account}. Clitics like the French \textit{me} appear always adjacent to an inflected verb, which the nonsentential account argues to be absent in fragments. I thank Sebastian Nordhoff for pointing this out.}\afterfn%

\ex. A: Did they elect Ann?
\a. B: No, they elected \textit{me}.
\bg.  B: Non, ils ont élu \textit{moi}.\\
~~ no they have elected I.strong\\
\cg. B: *Non, ils \textit{me} ont élu.\\ 
~~ no they I.weak have elected\\

\begin{sloppypar}\noindent
Taken together, the anticonnectivity effects\is{Case connectivity} observed by \citet{barton.progovac2005} for pronominal DP\is{Determiner phrase} short answers\is{Fragment, short answer} are explained not only by \citepos{barton.progovac2005} default case\is{Default case} hypothesis, but also by both types of sentential accounts. Case connectivity effects concerning inherent\is{Inherent case} case are also expected under both sentential and nonsentential accounts\is{Nonsentential account}.\end{sloppypar}

What remains disputed is the possibility of case marking on full DP\is{Determiner phrase} fragments, which will serve as a diagnostic for unarticulated structure in my experiments. While sentential accounts predict case connectivity\is{Case connectivity} also for structural case\is{Structural case}, if the nonsentential account\is{Nonsentential account} is right, DP\is{Determiner phrase}s that receive structural case\is{Structural case} marking in full sentences appear in default case\is{Default case} in fragments. As noted by \citet{progovac.etal2006}, due to crosslinguistic variation, it is necessary to carefully determine whether a specific case is actually a structural case\is{Structural case} in a given language. Finally, if fragments are ungrammatical, as suggested by \citet{bergen.goodman2015}\is{Ungrammaticality of fragments}, no grammatical restrictions on case marking in fragments are expected. Still though, case in fragments should tend toward matching that in the corresponding sentence, because it is a strong cue toward the meaning intended by the speaker.
\is{Case connectivity|)}

\subsection{Constituency}\label{sec:theories-predictions-constituents}
The theories presented in the previous section disagree on whether fragments must be a single constituent: The nonsentential account\is{Nonsentential account} by \citet{barton.progovac2005} as well as movement and deletion\is{Movement and deletion account} predict this, but in situ deletion\is{In situ deletion account} and \citepos{bergen.goodman2015} approach allow for non-constituent fragments. 

\subsubsection{Constituency under the nonsentential account}
\citet{barton.progovac2005} argue that any maximal projection XP can be a well-formed output of syntax, therefore a fragment must always consist of a single constituent of arbitrary size and internal complexity. Sequences of constituents that do not have a unifying node cannot be derived. The examples discussed in \citet{barton.progovac2005} clearly conform to this restriction. \citet{progovac2006} discusses more controversial examples like \Next, which are harder to analyze as a single constituent, because the predicate \textit{a bargain} does not have a verbal projection, whose specifier could host the subject.

\ex. This a bargain!? \hfill \citep[28]{progovac2006} \label{ex:progovac.sc-bargain-root}

In order to account for such apparent DP\is{Determiner phrase}-DP\is{Determiner phrase} sequences too, which in fact are relatively frequent \citep{reich2017}, \citet{progovac2006} resorts to a small clause\is{Small clause} analysis. Small clauses\is{Small clause} \citep{williams1975, stowell1981} are expressions that contain a subject and a predicate, but no verbal material. They can appear as root small clauses \Last or be embedded as verbal complement inside a matrix clause \Next.\largerpage

\ex. I consider this a bargain. \hfill \citep[41]{progovac2006} \label{ex:progovac.sc-bargain-embedded}

\begin{figure}
\begin{minipage}{.3\textwidth}
\begin{center}
   \Tree [.DP This [.D' [.D a ] [.N bargain ] ] ]
   
\end{center}
\end{minipage}
\begin{minipage}{.02\textwidth}
~~
\end{minipage}
\begin{minipage}{.3\textwidth}
\begin{center}
  \Tree [.PP Class [.P' [.P in ] [.N session ] ] ]
  
\end{center} 
\end{minipage}
\caption{DP and PP small clauses according to \citet[39]{progovac2006}.\label{p06-smallclause}}
\end{figure}
%
\citet[61]{progovac2006} claims that ``every sentence/clause is underlyingly a small clause.'' The subject is merged into the specifier of a projection whose category is determined by the predicate, as the examples in Figure \ref{p06-smallclause} illustrate.%
%
\footnote{This analysis can be traced back to \citet{stowell1981}, see \citet{citko2011} for an overview of alternative structural\is{Structural case} analyses of small clauses.}\afterfn%
%
If the small clause\is{Small clause} analysis is correct, it enables the nonsentential account\is{Nonsentential account} to derive XP-YP sequences as small clauses\is{Small clause}, as long as they do not contain uninterpretable features. \citet[51]{progovac2006} argues that the latter condition is met, because the subjects of root and embedded small clauses in English\il{English} appear in accusative\is{Accusative case} default case\is{Default case} \Next[a], which does not need to be checked according to the nonsentential account\is{Nonsentential account}. If the small clause\is{Small clause} serves as input to a regular inflected clause \Next[b] instead, the subject is selected from the lexicon with nominative\is{Nominative case} case features\is{Case feature} that are checked as usual in course of the derivation by T. In the embedded \Next[c], case is checked through ECM by the verb \citep[46]{progovac2006}.\largerpage 

\ex. \a. Her give up?! (Never!) \hfill \citep[41]{progovac2006} \label{ex:progovac.sc-give-up-root}
     \b. She gives up?!
     \c. I never saw her give up.

This analysis works for the (mostly) English\il{English} examples discussed by \citet{progovac2006}, but there are instances of fragments for which the small clause\is{Small clause} analysis fails. For instance, the German\il{German} fragment in \Next[a] seems to be acceptable just like its English\il{English} counterpart \ref{ex:progovac.sc-bargain-root} is, but, unlike in English\il{English} \ref{ex:progovac.sc-bargain-embedded}, it cannot be embedded under a verb in a matrix clause \Next[b]. Instead, the complement must be a clause headed by a copula \Next[c]. \citet{reich2017} presents further arguments against a small clause\is{Small clause} analysis of German\il{German} XP-YP fragments, which are based on corpus\is{Corpus} data from newspaper headlines. He concludes that such cases involve a null copula and consequently analyzes \Next[a] as underlyingly sentential.

\ex. \ag. Das ein Angebot?!\\ 
	  this a bargain\\
     \bg. *Ich finde das ein Angebot.\\
	 I find this a bargain\\
	 \trans{I consider this a bargain.}
    \cg. Ich finde, das ist ein Angebot.\\
	I find this is a bargain\\

Furthermore, in question-answer pairs like \ref{ex:theories-discontinuous-en-pizza}, repeated here as \Next, fragments can consist of two non-predicative DPs\is{Determiner phrase} with an omitted main verb. A small clause\is{Small clause} analysis is ruled out in this case, because the missing verbal element is not the copula. It is unclear how \citet{barton.progovac2005} would account for such fragments without assuming an unarticulated verbal node, ellipsis, or a discontinuous fragment, which their account cannot derive.
 
\ex. [Waiter to customers in the restaurant:] Who ordered what?\\
Customer: She \sout{ordered} the deep dish pizza. 

The nonsentential account\is{Nonsentential account} hence predicts that only expressions consisting of a single constituent are possible fragments. Otherwise, the grammar assumed by \citet{barton.progovac2005} would not be able to generate them. \citepos{progovac2006} small clause\is{Small clause} analysis accounts for some apparent non-constituent fragments, but presupposes that small clauses\is{Small clause} are grammatical in the a language. The comparison between German\il{German} and English\il{English} suggests that similar fragments are acceptable in both languages even though the small clause\is{Small clause} analysis does not account for the relevant data in German\il{German} \citep{reich2017}. Furthermore, \Last suggests that even in English\il{English} not all fragments can be traced back to a small clause\is{Small clause}.

\subsubsection{Constituency under sentential accounts}\largerpage
\citepos{merchant2004} movement and deletion\is{Movement and deletion account} account also predicts that fragments are always single constituents. As their derivation involves movement to [Spec, FP] for feature checking purposes, only one constituent can be moved at a time. The reason for this is that there is only one landing site for the fragment: the specifier of a head carrying the E feature\is{E feature}.%
%
\footnote{If multiple specifiers are assumed, as the minimalist program permits \citep[285]{chomsky1995} this restriction obviously does not hold anymore. However, \citet{merchant2004} does not seem to resort to multiple specifiers in order to explain apparent non-constituent fragments.}\afterfn
%
Of course, there are no restrictions on the internal complexity of this constituent, so that \citet{merchant2004} can account for small clause\is{Small clause} \ref{ex:progovac.sc-give-up-root} or VP\is{Verb phrase} fragments \Next:

\ex. What would you like to do tonight?\\ \label{ex:vp-fronting}
Go to the cinema\sout{, I'd like to}.

As \citet{merchant2004} allows for unarticulated structure in his elliptical sentences, the set of apparent non-constituent fragments that his account can explain is larger than that covered by the nonsentential account\is{Nonsentential account}. For instance, a sequence of a temporal and a locative adverbial\is{Adverbial} does not form a small clause\is{Small clause} but seems to be felicitous as a fragment \Next. For the movement and deletion\is{Movement and deletion account} account to derive such fragments, they need to form a single constituent at some point of the derivation. The assumption that this is possible in case of such adverbial\is{Adverbial}s is supported by German\il{German} data from \citet{haider2000}. \citeauthor{haider2000} observes that such \textit{clusters} consisting of event-related adverbial\is{Adverbial}s, may appear together in the German\il{German} sentence-initial prefield\is{Prefield} \NNext, which is generally considered to host only one constituent (see also the discussion of experiment \ref{exp:mvb} in Section \ref{sec:mvb-background}).%
% 
\footnote{\citet[104--105]{haider2000} also rules out an alternative derivation that fronts the complete VP\is{Verb phrase} after extracting the verb for independent reasons, but cf. \citet{muller2004} for such a proposal.}\afterfn%
%

\ex. Q: When did you last meet him?\\
    A: Last night in the pub.

\exg. Gestern im Hörsaal als der Vortrag begann hustete er wie verrückt.\\
yesterday in.the lecture.room when the lecture started coughed he like mad\\
\trans{Yesterday in the lecture room as the lecture started he coughed like mad.}\\\mbox{}\hfill\raisebox{3\baselineskip}[0pt][0pt]{\citep[104]{haider2000}}

\vspace{-11pt}
A similar point is made by \citet{muller2002} in his analysis of (apparent) multiple prefield\is{Prefield} constituents in German\il{German}. As pointed out above, German\il{German} declarative matrix clauses are verb-second (in what follows, V2), so that only one constituent may precede the inflected verb. In spite of this, \citet{muller2002, muller2003, muller2005} reports a diverse sample of corpus\is{Corpus} data which seem to violate this constraint. For instance, in \Next, taken from \citet[38]{muller2005}, a locative and a temporal PP\is{Preposition phrase}, which do not straightforwardly form a single constituent, can appear together in the prefield\is{Prefield}.%
%
\footnote{Note that, unlike \ref{ex:vp-fronting}, this example does not involve fronting of the complete VP\is{Verb phrase}.}\afterfn%
%
\citet{muller2002} develops a HPSG\is{HPSG} account of such data, whose central assumption is that the prefield\is{Prefield} consists of a single constituent which is headed by an unarticulated verb.\largerpage

\exg. [Vor drei  Wochen] [in Memphis] hatte Stich noch in drei Sätzen gegen Connors verloren.\\ 
ago three weeks    in Memphis  had   Stich  still   in three sets   against Connors lost\\
\trans{Three weeks ago in Memphis Stich had still lost in three sets against Connors.}

The exact predictions of the movement and deletion\is{Movement and deletion account} account thus depend to a large extent on how many and which movement operations and covert elements are assumed in general, therefore it is difficult to derive testable predictions unless there is consensus on these preliminary assumptions. The picture becomes even more complicated if a fine-grained left periphery is assumed. Even in the original sketch of the CP\is{Complementizer phrase} layer in \citet{rizzi1997} there is a further topic projection above FocP, and \citet{beninca.poletto2004} distinguish about half a dozen highly specified functional projections, most of which are located above those related to focus. Depending on where \citet{merchant2004} would locate the E feature\is{E feature} in a language, the material in these projections could survive ellipsis, so that (if these projections exist in a language) the movement and deletion\is{Movement and deletion account} account predicts the possibility of genuine non-constituent fragments in highly specific information-structural\is{Structural case}\is{Information structure} configurations. Finally, the version of movement and deletion\is{Movement and deletion account} by \citet{weir2014} also accounts for non-constituent fragments. \citeauthor{weir2014} simply adjoins\is{Adjunct} the moved constituents to CP\is{Complementizer phrase} and, since adjunction is recursive, there is no upper bound limit on the number of constituents extracted with this mechanism.

Taken together, the existence of (apparently) discontinuous fragments as evidence for or against specific accounts of fragments does not seem promising, because theories that in principle predict constituency to be a determining property of fragments can account for some instances of what superficially must be classified as non-constituents. Specifically, movement and deletion\is{Movement and deletion account} requires an extensive set of preliminary assumptions about which syntactic operations and functional projections are available in a given language in order to make clear predictions on non-constituent fragments. Therefore, I do not use constituency directly as a diagnostic for particular theories.%
%
\footnote{An exception is experiment \ref{exp:mvb}, which compares the acceptability of apparent multiple prefield\is{Prefield} configurations as fragments and full sentences. However, the experiment does not depend on a specific syntactic analysis of the constructions tested, but on the parallelism between fragments and left dislocations predicted by the movement and deletion\is{Movement and deletion account} account.}\afterfn%
%

\subsection{Information structure and focus}\largerpage
\is{Information structure|(}
\label{sec:theories-predictions-focus}
All sentential accounts of fragments discussed so far \citep{merchant2004, reich2007, weir2014} assume that information structure\is{Information structure}, in particular the focus-background structure of an utterance, determines which words can be omitted in fragments. Since the nonsentential account of fragments does not impose information-structural licensing conditions on fragments, focus-sensitivity could appear to be a promising testing ground to differentiate between sentential and nonsentential accounts\is{Nonsentential account}. \citet{reich2007} and \citet{weir2014} make the strongest claim on the issue by assuming that fragments are necessarily focused. \citeauthor{reich2007} argues that only F-marked\is{F-marking} expressions survive ellipsis, whereas \citeauthor{weir2014} considers focus to trigger exceptional movement\is{Exceptional movement account}.%
%
\footnote{If there is more than one focus, their predictions might differ with respect to ordering. \citet{reich2007} predicts the same ordering as in regular sentences, but for \citeauthor{weir2014}'s theory it matters whether several exceptional movement\is{Exceptional movement account} operations proceed from the top of the syntax tree to its bottom or vice-versa. E\is{E feature} is located on the C head dominating the whole TP\is{Tense phrase}, so that if the constituents closest to E\is{E feature} were evacuated first, both accounts would predict differing orderings.}\afterfn%
%

As for \citet{merchant2004}, this is less clear: On the one hand, he tentatively suggests identifying the landing site for fragments as a FocP \citep[675]{merchant2004}, on the other hand he emphasizes similarities between fragments and CLLD\is{Clitic left dislocation} \citep[703]{merchant2004}. However, according to the literature on CLLD\is{Clitic left dislocation}, this construction does not involve focus movement but targets a topic position.%
%
\footnote{\citet[53]{beninca.poletto2004} distinguish between topic and focus positions in the left periphery by noting that the former ``are connected with a clitic or a \textit{pro} in the sentence'', while the latter leave a variable which is bound by the moved phrase. Based on mostly Spanish\il{Spanish} data, \citet{arregi2003} argues that CLLD\is{Clitic left dislocation}ed constituents are contrastive topics\is{Topic, contrastive}. Contrastive topics differ from foci both in their syntactic properties and in their prosody \citep[see e.g.][]{buring1997, krifka2007}.}\afterfn%
%
Further confusion on the status of the presumed left dislocation of fragments comes from the German\il{German} data in \Next. \citet[702]{merchant2004} argues that the form of pronouns in the preverbal position (the \textit{prefield}\is{Prefield}) equals that in the fragments in \NNext (judgments are \citeauthor{merchant2004}'s).\largerpage[1.75] 

\exg. Was wolltest du?\\
     what wanted you\\
     \trans{What did you want?} \hfill \exsourceraised{\citep[701]{merchant2004}}
\ag. {Das/*Es} wollte ich.\\
    this/it wanted I\\
    \trans{This/It I wanted.} \hfill \exsourceraised{\citep[702]{merchant2004}}
\b.  Das/*Es.

\ex. *Das\textsubscript{i}/*Es\textsubscript{i} wollte \sout{ich \textit{t}\textsubscript{i}}.

The structure in \LLast[a] is derived neither by CLLD\is{Clitic left dislocation} nor by focus movement, because it is a garden-variety verb-second declarative matrix clause. As I discuss in greater detail in Section \ref{sec:mvb-background}, the mainstream analysis of German\il{German} V2 consists in moving the inflected verb to C, whose specifier must be filled by any other constituent \citep{denbesten1989}. Crucially, there are almost no restrictions on the category or information-structural\is{Information structure} status of the constituent in [Spec, CP]\is{Complementizer phrase}, which can be an aboutness or contrastive topic\is{Topic, contrastive} as well as a focus \citep{frey2005}. If \LLast[a] is a standard verb-second clause and the mainstream analysis of V2 is correct,%
%
\footnote{\citet{muller2004} proposes an analysis of V2 that consists in VP\is{Verb phrase} fronting after those constituents that appear in post-verbal positions have been extracted. As he assumes that the VP\is{Verb phrase} contains only the finite verb and the prefield\is{Prefield} when it is fronted, it makes the same predictions as the mainstream account of V2, that is, the verb must survive ellipsis in fragments.}\afterfn%
%
the fragment in \Last[b] cannot be derived from \Last[a] by assuming an E feature\is{E feature} on C: E\is{E feature} triggers only PF-deletion of the complement of C, hence the account incorrectly predicts the verb to survive ellipsis:%
%
\footnote{A possible explanation would be that T-to-C movement of the verb occurs only in order to satisfy some PF constraint and is therefore not required (which is equivalent to not being allowed in minimalism\is{Minimalist program}) in elliptical sentences. I am not aware of any proposal in this direction.}\afterfn%
%
Furthermore, the structure in \LLast[a] is not an instance of CLLD\is{Clitic left dislocation}. Even though German\il{German} declarative matrix clauses are verb-second, a further constituent can be placed left of the prefield\is{Prefield} \Next. DPs\is{Determiner phrase} appearing there sometimes exhibit no case connectivity\is{Case connectivity} at all \Next[a] and must therefore be analyzed as a hanging topic\is{Topic, hanging} \citep{rodman1974, vat1981}, but in \Next[b] the DP \textit{den Peter} resembles CLLD\is{Clitic left dislocation} in being case-marked. Unlike \LLast[a], both of these structures require doubling of the left-peripheral constituent by a pronoun.\largerpage[3]

\ex. \ag. Der Peter, den habe ich gestern erst getroffen.\\
	  the.\textsc{nom} Peter, him have I yesterday just met.\\
	  \trans{(As for) Peter, I just met him yesterday.}
     \bg. Den Peter, den habe ich gestern erst getroffen.\\
     	  the.\textsc{acc} Peter, him have I yesterday just met.\\
	  \trans{Peter, I just met him yesterday.}

However, the structure in \Last[b] cannot be the source of fragments. If E\is{E feature} is located on C, again, both the verb and the pronoun would survive ellipsis. The derivation in Figure \ref{fig:merchant-spec-cp} shows that this would yield the ungrammatical \textit{*Den Peter, den habe}. Taken together, E\is{E feature} cannot be located on C in German\il{German}.

\begin{figure}
\begin{tikzpicture}[baseline,sibling distance=-10pt]
   \tikzset{}
  \Tree[.FP \node(w){[\textsubscript{DP} {Den Peter}]\textsubscript{2}}; [.F' F [.CP \node(i){[den]\textsubscript{j}}; [.C' {C\textsuperscript{\textbf{[E]}} + habe\textsubscript{i}} [.TP \edge[roof]; {\sout{ich gestern erst \textit{t}\textsubscript{j} getroffen} \textit{t}\textsubscript{i}} ] ] ] ] ];
 \draw[semithick, -stealth] (5.2,-5.5) to [bend left=60] (i);
\end{tikzpicture}

\caption{The derivation shows that if the E feature was located on [Spec, CP], German finite words are incorrectly predicted to survive ellipsis in fragments.\label{fig:merchant-spec-cp}}
\end{figure}
%
\citet{merchant2004} specifies the syntactic properties of different varieties of E\is{E feature} in their lexical entries, so there is no principled reason to assume that the German\il{German} E\textsubscript{F}\is{E feature} must also be [\textit{u}C*]. If it had an [\textit{u}F*] feature and were thus located on F, as \citet{merchant2004} initially suggested for English\il{English}, the verb would always be PF-deleted and the existence of DP\is{Determiner phrase} fragments straightforwardly explained. The problem with this assumption is that \citet{merchant2004} rejected it in order to account for the island\is{Island} sensitivity of fragments based on the PF-deletion of illegal traces\is{Trace (movement)} (see Section \ref{sec:theories-predictions-movement} for discussion). Therefore, if the E feature\is{E feature} was located on a F head above CP\is{Complementizer phrase} in German\il{German}, German\il{German} fragments should not be island-sensitive\is{Island}, but \Next shows that they are.%
%
{\interfootnotelinepenalty=10000\footnote{\citet{merchant2004} judges the English counterpart of \ref{ex:merchant-island-ger-shortanswer} as ungrammatical when it is interpreted as \Next[a]. If \ref{ex:merchant-island-ger-shortanswer} was interpreted as \Next[b], he predicts the fragment to be grammatical, because fronting \textit{Charlie} in the matrix clause does not require the extraction out of an island.

\ex. \ag. Nein, Abby spricht die gleiche Balkansprache, die Charlie spricht.\\
	 no Abby speaks the same balkan.language that Charlie speaks\\
	 \trans{No, Abby speaks the same Balkan language that Charlie speaks.}
     \bg. Nein, Charlie spricht die gleiche Balkansprache, die Ben spricht.\\
     	 no Charlie speaks the same balkan.language that Ben speaks\\
       \trans{No, Charlie speaks the same Balkan language that Ben speaks.}
       
}}
\afterfn%
%

\ex. \ag. Spricht Abby die gleiche Balkansprache, die \textit{Ben} spricht?\\
	  speaks Abby the same balkan.language that Ben speaks\\
	  \trans{Does Abby speak the same Balkan language that \textit{Ben} speaks?}
\bg. *Nein, \textit{Charlie}.\label{ex:merchant-island-ger-shortanswer}\\
      no Charlie\\
      \trans{No, Charlie.}\exsourceraised{\citep[translated from][708]{merchant2004}}

Taken together, the predictions of \citet{merchant2004} with respect to focus marking\is{F-marking} are vague, because it is not totally clear whether he assumes fragments to target the focus position [Spec, FP] or whether their landing site is [Spec, CP]\is{Complementizer phrase}. Neither of these versions can account for the full range of the data discussed in this section. The exceptional movement\is{Exceptional movement account} version of the theory \citep{weir2014} does not require information structure\is{Information structure}-related projections as FocP but simply adjoins\is{Adjunct} fragments to CP\is{Complementizer phrase}, so that it is not affected by these issues.\largerpage

Nonsentential accounts\is{Nonsentential account} do not make reference to focus, but the conditions on fragment use that they require are related to information-structural\is{Information structure} notions. \citet{barton1990} proposes to ``delete up to recoverability'' and \citet{stainton2006} requires that a salient nonlinguistic LF which allows to enrich the fragment to a proposition must be present in context. As foci tend toward being new and consequently not recoverable, both of these ideas make similar predictions with respect to the acceptability of fragments as a focus-based account.

Focus sensitivity therefore does not offer a promising testing ground to distinguish the predictions of the theories of fragments that I investigate. Besides that, the effect of focus is relatively difficult to investigate experimentally. In German\il{German} and English\il{English}, focus is often marked prosodically with a H* pitch accent \citep{gussenhoven1983, pierrehumbert.hirschberg1990} and the prominence of prosodic focus marking varies gradually as a function of the size of the focus domain \citet{baumann.etal2006, baumann.etal2007}. This is hard to manipulate experimentally: As \citet{baumann.etal2007} report, speakers make use of different strategies to modulate the prosodic prominence of foci, so that items might not be understood as intended. Furthermore, while work in experimental pragmatics has provided evidence for an effect of pitch accents on interpretation of complete sentences \citep[see e.g.][]{chevallier.etal2008, zondervan2010}, it is difficult to apply this to fragments. For instance, in DP\is{Determiner phrase} fragments consisting only of a noun and an article, the most prominent accent always falls on the noun, hence it is not possible to vary the pitch accents on fragments in order to elicit different focus structures.
\is{Information structure|)}


\subsection{Evidence for movement}
\label{sec:theories-predictions-movement}
\is{Movement restriction|(}

In contrast to any of the other accounts, \citepos{merchant2004} theory requires fragments to undergo obligatory movement to a left-peripheral position. From a na\"{i}ve perspective, this predicts that only expressions that can occur in a left-peripheral position, more specifically, to the left of the head hosting the E feature\is{E feature}, are possible fragments. Consequently, whatever might restrict movement to the left periphery in full sentences will also constrain the form of fragments. In the next chapter, I empirically investigate two of the movement restrictions\is{Movement restriction} discussed by \citet{merchant2004}. These restrictions, whose effects on the form of fragments have been first tested in \citet{merchant.etal2013}, concern complement clause\is{Complement clause} topicalization and preposition stranding\is{Preposition stranding}. The reason for choosing these phenomena is that \citet{merchant.etal2013} present the first experimental evidence on them, what suggests that they consider them a valid testing ground for the movement and deletion account\is{Movement and deletion account}. Preposition stranding restrictions have the additional advantage that they cannot be overridden by exceptional movement\is{Exceptional movement account} according to \citet{weir2015} and hence also allow us to test \citepos{weir2014} version of the theory.

The idea that movement restrictions\is{Movement restriction} constrain the form of fragments is exemplified for complement clause\is{Complement clause} topicalization in \Next and \NNext, from \citet[690]{merchant2004}: As has been repeatedly claimed in the theoretical literature \citep[see e.g.][]{morgan1973, chomsky1981, stowell1981, webelhuth1992}, the complementizer in English\il{English} non-factive\is{Factivity} complement clause\is{Complement clause}s is optional when the complement clause\is{Complement clause} appears in its base position \Next[a], but becomes obligatory when the complement clause\is{Complement clause} appears in the left-periphery \Next[b]. \citet{merchant2004} observes that the same holds for fragments \NNext. He concludes that this is unexpected under the in situ deletion\is{In situ deletion account} account, because the complementizer would be optional in fragments too if their underlying structure was \Next[a] instead of \Next[b]. The other movement restrictions\is{Movement restriction} discussed by \citet{merchant2004} behave similarly, that is, expressions which cannot appear in the left periphery seem to be unacceptable as fragments. 

\ex. \label{ex:merchant.teacher-sent}
\a. No one believes (that) I’m taller than I really am. 
\b. *(That) I’m taller than I really am, no one believes.

\ex. \label{ex:merchant.teacher-frag}
What does no one believe? \hfill \citep[690]{merchant2004}\\
\mbox{}\hspace{-.45em}\#(That) I’m taller than I really am.

\citet{merchant2004}\is{Movement and deletion account} interprets such data as evidence in favor of his account, but in order for this to constitute valid evidence for movement in fragments\is{Movement and deletion account}, it is necessary to rule out alternative explanations for the observed pattern, which do not require movement. Throughout this book, some of these data will turn out to be relatively straightforwardly captured by the nonsentential or the in situ deletion\is{In situ deletion account} accounts. For instance, a construction might have properties that block both movement in full sentences and ellipsis in fragments without having to assume that the latter necessarily undergo movement.

Besides this need for caution when interpreting coincidences between fragments and left dislocation as evidence for movement, acceptable fragments that cannot be left-dislocated potentially constitute counterevidence to \citepos{merchant2004} theory. If ungrammatical left dislocations in full sentences always yielded unacceptable fragments, movement and deletion\is{Movement and deletion account} would be falsified even by the most basic examples, such as the unavailability of fronting of a DP\is{Determiner phrase} which is not contrastive in English\il{English} \citep{weir2014}, or the island\is{Island}-sensitivity of fragments described by \citet{merchant2004}. This requires them to assume \textit{repair effects} \citep{merchant2004} or exceptional movement\is{Exceptional movement account} \citep{weir2014} in order to conceal the theory with conflicting data.

Repair effects are widely acknowledged in the literature on ellipsis \citep[see e.g.][]{fox.lasnik2003, merchant2008, muller2011, lasnik2015}. The general observation is that in some cases ellipses are acceptable even though the presumably underlying nonelliptical structure is not. The idea is that ellipsis can remedy ill-formed structures by deleting those expressions that cause the problem. Since ellipsis is a PF phenomenon, this concerns only degraded PFs, but not derivations which are ungrammatical in the narrow syntax. A prototypical instance of such repair effects is the island\is{Island}-insensitivity of sluicing\is{Sluicing}. Recall that \citet{merchant2001} derives sluicing\is{Sluicing} by regular \textit{wh}-movement followed by deletion of the TP\is{Tense phrase} in the sluice. \Next[a] shows that sluicing\is{Sluicing} is fine although the derivation assumed by \citeauthor{merchant2004} involves an ungrammatical island\is{Island} violation: The \textit{wh}-phrase must be extracted across the boundary of the embedded relative clause introduced by \textit{who speaks} \Next[b].

\ex. \a. They want to hire someone who speaks a Balkan language, but I don't remember which. \hfill\citep[705]{merchant2004}
 \b. *They want to hire someone who speaks a Balkan language, but I don't remember [which Balkan language]\textsubscript{i} they want to hire someone who speaks \textit{t}\textsubscript{i}.
 
 \begin{figure}
(No,) \begin{tikzpicture}[baseline]
   \tikzset{}
  \Tree[.FP \node(w){[\textsubscript{DP} Charlie]\textsubscript{i}}; [.F' F [.CP \node(i){[*\textit{t}\textsubscript{i}]}; [.C' C\textsuperscript{\textbf{[E]}} [.TP \edge[roof]; {\sout{\dots speaks \dots *\textit{t}\textsubscript{i}}} ] ] ] ] ];
\draw[semithick, -stealth] (i) to [bend left=60] (w);
 \draw[semithick, -stealth] (4.4,-5.45) to [bend left=60] (i);
\end{tikzpicture}

\caption{Derivation of a fragment answer to \ref{ex:merchant-abby-q} according to \citet[708]{merchant2004}.\label{fig:merchant-abby-a}}
\end{figure}
%
\citet[706]{merchant2004} accounts for this by assuming that traces\is{Trace (movement)} of movement across island\is{Island} boundaries have a feature *, which renders PF representations that contain such features uninterpretable. As \citeauthor{merchant2004} assumes that ellipsis is PF-deletion, it can delete such traces\is{Trace (movement)} at PF and thus ``repair'' the defective structure. \citeauthor{merchant2004} notes that his hypothesis can also account for the observation that sluicing\is{Sluicing} is not sensitive to island\is{Island}s but other ellipses, like VPE\is{Verb phrase ellipsis} and fragments, are. In sluicing\is{Sluicing}, the E feature\is{E feature} is located on C, so that it deletes all intermediate *\textit{t} traces\is{Trace (movement)} within the TP\is{Tense phrase} at PF. In contrast, movement in fragments\is{Movement and deletion account} targets [Spec, FP] and, by proceeding cyclically, it leaves a *\textit{t} in [Spec, CP\is{Complementizer phrase}], as the derivation of fragment answer to \Next in Figure \ref{fig:merchant-abby-a} shows. The trace in the specifier survives ellipsis and causes the derivation to crash, because E\is{E feature} is placed on C in fragments. 

\ex. Does Abby speak the same Balkan language that Ben speaks?\label{ex:merchant-abby-q}

In fact, the need to account for the island\is{Island} sensitivity of fragments is what motivates \citeauthor{merchant2004} to reject his initial assumption that E\is{E feature} is placed on F in fragments, which is illustrated in Figure \ref{ex:merchant.structure-reduced}. If this derivation was correct, the PF-deletion of the defective trace\is{Trace (movement)} would render fragments insensitive to island\is{Island}s, just like \citet{merchant2001} argues for sluicing\is{Sluicing}.

\begin{figure}
\begin{tikzpicture}[baseline]
   \tikzset{}
  \Tree[.FP \node(w){[\textsubscript{DP} John]\textsubscript{2}}; [.F' F\textsuperscript{\textbf{[E]}} [.TP \edge[roof]; {\sout{she saw \textit{t}\textsubscript{2}}} ] ] ];
\draw[semithick, -stealth] (2,-3.3) to [bend left=60] (w);
\end{tikzpicture}
\caption{Derivation of fragments without an intermediate CP according to \citet[675]{merchant2004}.\label{ex:merchant.structure-reduced}} 
\end{figure}

Repair effects complicate the use of parallelisms between fragments and sentences as a diagnostic for movement. As \citet[711]{merchant2004} puts it, ``[\dots] the general argument is that parallelisms support a movement and ellipsis analysis, while non-parallelisms reveal repair effects.'' Nevertheless, repair effects cannot just be stipulated but require an analysis, like the one involving the * feature on traces\is{Trace (movement)} that results from island\is{Island} violation.

\citepos{weir2014} exceptional movement\is{Exceptional movement account} account predicts an even larger set of mismatches between fragments and sentences than \citet{merchant2004} because exceptional movement\is{Exceptional movement account} occurs only in fragments and is therefore relatively independent from movement in sentences. \citeauthor{weir2014} only restricts exceptional movement\is{Exceptional movement account} to those movement operations that are ``in principle'' available in a language.  The derivation of empirically testable predictions for his account would require criteria that define which types of movement are available in principle and which are not, but \citet[10]{weir2015} offers no such criteria and instead proposes that it ``is most easily shown by example'' how a movement operation is to be classified. He exemplifies this reasoning with the impossibility of fronting NPIs \Next[a] and bare quantifiers \NNext[a] in English\il{English} (the judgments are Weir's) and argues that left dislocation of such expressions is not blocked due to a syntactic restriction because argument DPs\is{Determiner phrase} can be fronted in English\il{English} \ref{ex:weir-fronting}. Consequently, he attributes the ungrammaticality of \Next[a] and \NNext[a] to some kind of semantic ill-formedness and takes this to illustrate the PF-only character of exceptional movement\is{Exceptional movement account}.

\ex.What didn't John buy? \hfill \citep[adapted from][3]{weir2015}
\a. ??Any wine, John didn't buy \textit{t}.
\b. Any wine.

\ex. Who will you talk to? \hfill \citep[adapted from][3]{weir2015}
\a. *Someone, I will talk to \textit{t}.
\b. Someone.

\ex. That guy, I saw \textit{t}.\label{ex:weir-fronting}  \hfill \citep[10]{weir2015} 

The absence of criteria for which types of movement are in principle available makes it almost impossible to empirically evaluate the exceptional movement\is{Exceptional movement account} account. However, what matters from an empirical perspective is that \citet[11]{weir2015} explicitly excludes P-stranding\is{Preposition stranding} \citep{pullum.huddleston2002}, and left-branch extraction \citep{ross1967, boskovic2005} in languages which do not allow for either of these operations from the set of movement operations that are available in principle. Therefore, languages which do not lack these phenomena but which will allow for the generation of the corresponding fragments would also provide evidence against exceptional movement.

Repair effects, and in particular exceptional movement\is{Exceptional movement account}, notably complicate the predictions of the movement and deletion\is{Movement and deletion account} account on the correlation between the acceptability of fragments and left dislocation. Specifically, some of the mismatches in acceptability between acceptable elliptical and nonelliptical structures would not provide evidence against movement and deletion\is{Movement and deletion account} if they can be explained by some kind of repair effect. Nevertheless, repair effects increase the complexity of the theory and therefore require independently motivated accounts that justify them. The assumption of defective traces\is{Trace (movement)} by \citet{merchant2004} discussed above might provide such an account, but it still requires independent empirical evidence, for instance, one showing that similar observations can be made for related phenomena. However, and in the first place, empirical perspective however, the movement and deletion\is{Movement and deletion account} account requires evidence \textit{for} movement in fragments\is{Movement and deletion account} that cannot be explained by other theories, like nonsentential approaches or the derivationally simpler in situ deletion\is{In situ deletion account} account.\is{Movement restriction|)}

\subsection{Summary}
In this section I have discussed different potential testing grounds which might allow for the empirical evaluation of the competing accounts of fragments. Their respective predictions are summarized in Table \ref{tab:predictions}. 

\begin{table}
  \begin{tabular}{p{2.2cm}p{1.5cm}p{1.3cm}p{2.2cm}p{2.5cm}}
\lsptoprule
 ~~ & \citet{merchant2004} & \citet{reich2007} & \citet{barton.progovac2005} & \citet{bergen.goodman2015}\\
\midrule
Structural case & ~\ding{51} & \ding{51} & \ding{55} & \ding{51} \\
 Non-constituents  & (\ding{55}) &  \ding{51} & \ding{55} & \ding{51} \\
  Focus\linebreak sensitivity & ~\ding{51} & \ding{51} & \ding{55} & \ding{55} \\
  Movement\linebreak restrictions & (\ding{51})  & \ding{55} & \ding{55} & \ding{55} \\
\lspbottomrule
 \end{tabular}
\caption{Overview of the empirical predictions the accounts of fragments make on the phenomena discussed in this section. The predictions of \citet{merchant2004} with respect to non-constituent fragments and effects of movement restrictions depend on further theory-internal assumptions. Non-constituent fragment is predicted ot be acceptable when multiple movement to the left periphery is possible, e.g. in Italian \citep{cinque1990, rizzi1997}. Under the exceptional movement version of the theory \citep{weir2014}, movement restriction effects are only expected for movement which is unavailable \textit{in principle} in a language.\label{tab:predictions}}
 \end{table}
%
With respect to the question of whether fragments involve unarticulated structure, case connectivity effects\is{Case connectivity}, specifically with respect to structural case\is{Structural case} marking in fragments, turn out to be the most promising testing ground. Sentential accounts predict strict case connectivity, whereas the nonsentential account\is{Nonsentential account} predicts that fragments cannot appear in structural case\is{Structural case} and will exhibit default case\is{Default case} morphology instead. As argued above, whether a specific case is structural case\is{Structural case} in a language or whether it is not might be controversial, but the alternative, focus sensitivity, is even more difficult to test. The nonsentential account\is{Nonsentential account} does not rely on the concept of focus, but makes similar predictions due to the necessity to retrieve deleted expressions from context. Furthermore, it is difficult to empirically elicit specific focus structures, because focus is mostly encoded prosodically in English and German \citep[see e.g.][1658--1660]{zimmermann.onea2011}.

Testing whether fragments are derived by movement and deletion\is{Movement and deletion account}, as claimed by \citet{merchant2004}, obviously requires the investigation of whether movement restrictions\is{Movement restriction} constrain the form of fragments: Restrictions on the form of fragments and left-dislocated expressions would provide evidence for movement and deletion\is{Movement and deletion account} and the corresponding mismatches evidence against it. As for the former, it must also be shown that the nonsentential or the in situ deletion\is{In situ deletion account} accounts cannot account for the data, while the possibility of repair effects must be considered in case of apparent non-parallelisms. My experiments investigating movement restrictions\is{Movement restriction} take the studies by \citet{merchant.etal2013} on preposition omission\is{Preposition omission} and complementizer omission\is{Complementizer omission} as a starting point. The first of these phenomena will also allow for conclusions on \citepos{weir2014} exceptional movement\is{Exceptional movement account} theory, which seems to have a greater empirical coverage than the original version of movement and deletion\is{Movement and deletion account}, but it is also harder to test because of the lack of clear criteria that would distinguish movement operations that can occur in fragments from those that cannot.

As for the assumption that fragments are inherently ungrammatical, I noted above that it is impossible to verify but that it still can be falsified by evidence for linguistic constraints on the form of fragments that cannot be reduced to being the result of game-theoretic\is{Game theory} reasoning, as \citet{bergen.goodman2015} suggest.

In the next chapter, I present a series of experiments that test some of the predictions of the competing theories of fragments which I discussed in this section. Currently there is no consensus on the appropriate syntactic analysis of fragments, and there has been no systematic and empirical investigation of the partially contrary predictions of the theories. My experiments will to some extent fill this gap. The experiments address two main questions that allow us to differentiate between the theories presented in this chapter: First, I test whether fragments are underlyingly sentential. Since the results support a sentential account, I  address the question of whether movement restrictions\is{Movement restriction} constrain the form of fragments at the case of preposition omission\is{Preposition omission}, complementizer omission\is{Complementizer omission} and multiple prefield\is{Prefield} constituents in German\il{German}. Furthermore, the results on the syntax of fragments inform my experiments on their usage in Chapter \ref{sec:chapter-infotheory-experiments}.
