\chapter{The verbal complex}
\label{sec:TAM}

\section{Introduction}
\label{sec:TAMIntro}


In this chapter, I describe the verbal complex and its encoding of the grammatical categories of tense, aspect, mood, and negation. Gyeli has two main verbal construction types: (i) those with a single verb, which I call simple predicates, and (ii) those with two or three verbs, which I call complex predicates. There are two subcategories of complex predicates. One is formed with a single \textsc{stamp} marker (\sectref{sec:SCOP}), an auxiliary verb, and one or two non-finite verbs. The other involves the \textsc{stamp} marker and a finite form of {\itshape bɛ̀} `be', which is followed by another \textsc{stamp} marker and a finite verb form. I present simple predicates in \sectref{sec:SimpPred} and complex predicates in \sectref{sec:CompPred}. 

Simple predicates occur significantly more frequently than complex predicates, as shown in \figref{Fig:FeatCP} for the 214 simple verbal clauses (\sectref{sec:SC}) in the corpus.  Complex predicates can be subdivided into those that occur with a single \textsc{stamp} marker and those that have a double \textsc{stamp} marker. The complex predicates with a single \textsc{stamp} marker take an auxiliary and either one or two non-finite main verbs (\sectref{sec:ComplAUX} and \sectref{sec:ComplSemi}). The constructions with only one main verb constitute roughly three quarters of complex predicate constructions in the corpus. Complex predicates with a double \textsc{stamp} marker are formed by two constituents: (i) a \textsc{stamp} marker followed by an inflected form of {\itshape bɛ̀} `be' and (ii)  a second \textsc{stamp} marker that is identical in its reference to the first one and followed by another inflected verb form (\sectref{sec:Compbe}).

\iffalse
\begin{table}
%\scalebox{0.9}{
\begin{tabular}{p{1cm}l|ll}
 \lsptoprule
\multicolumn{2}{l|}{Feature} & \multicolumn{2}{l}{Frequency}  \\
 \midrule
\multicolumn{2}{l|}{Simple predicates}  & 158 & 73.8\% \\
\multicolumn{2}{l|}{Complex predicates} & 56 & 26.2\% \\
 & single \textsc{stamp} auxiliary constructions & 55 & 25.7\% \\
 & \hspace{.5cm} one non-finite verb & 	\hspace{.5cm} 42 & \hspace{.5cm} (76.4\%) \\
& \hspace{.5cm} two non-finite verbs & \hspace{.5cm}	13 & \hspace{.5cm} (23.6\%) \\
 & double \textsc{stamp} auxiliary constructions & 1 & .5\% \\
%juxtaposed		& 35 & 63.6\% \\
%separated		& 20  & 36.4\% \\
 \midrule
Total 		& 	& 214 & \\
 \lspbottomrule
\end{tabular}
\caption{Distribution of predicate types in simple verbal clauses}
\label{Tab:FeatCP}
\end{table}
\fi

\begin{figure}
\begin{forest}    forked edges
  [total (214)
    [simple predicates (158)]
    [complex predicates (56)
        [{single \textsc{stamp}\\ auxiliary constructions (55)}, text width=5cm
            [one non-finite verb (42)]
            [two non-finite verbs (13)]
        ]
        [{double \textsc{stamp}\\ auxiliary construction (1)}, text width=4cm]
    ]
  ]
\end{forest}
\caption{Distribution of predicate types in simple verbal clauses}
\label{Fig:FeatCP}
\end{figure}

The expression of grammatical categories such as tense, aspect, mood, and negation is achieved through multiple strategies for both  simple and complex predicates, such as tonal patterns, morphological marking, and periphrastic structures including auxiliaries.  Marking of tense and mood is more interdependent than aspect or negation marking: tense and mood categories form an interlocking system, as they are conjointly marked by tonal patterns. I therefore refer to them as ``tense-mood (TM) categories''.   
The different verbal predicate structure types do not straightforwardly map onto specific grammatical categories. Instead, simple and complex predicates both encode a range of tense, mood, aspect, and negation categories. There are, however, certain tendencies in the distribution of grammatical categories across predicate types. For instance,  tense-mood categories are mainly encoded through simple predicates, whereas aspect and negation categories are primarily found in complex predicates. 

The discussion in this section is organized according to  verbal predicate type, as opposed to  semantic category. Before proceeding with that analysis, I define the terminology I use for broad grammatical categories such as tense, mood, negation, aspect, and negation and provide a general discussion of their encoding in Gyeli. 


\subsection*{Tense}
Grammatical tense, and its relation to aspect, has been extensively discussed in the literature.
\citet[9]{comrie85}, for instance, defines tense as ``grammaticalised expression of location in time''.  \citet[25]{dahl85} notes more precisely that ``[non-relative] tenses are typically deictic categories, in that they relate time points to the moment of speech. Aspects, on the other hand, are non-deictic categories''.  As \citet[5]{comrie76} explains, ``[a]spect is not concerned with relating the time of the situation to any other time-point, but rather with the internal temporal constituency of the one situation''. Or, as \citet[315]{timberlake2007} puts it: ``aspect locates events (and measures their progress or change or results or liminality) in relation to an internal time''. 

Gyeli is  a ``tense language'', since tense (and mood) marking is in several respects more prominent than aspect marking. First, aspect marking is not obligatory, whereas tense and mood are obligatorily marked. Second, no aspect category is present in  every tense. Instead, most aspect categories are restricted to a specific tense-mood category in which they can occur. And third, aspect markers do not occur in negative polarity, whereas tense markers do. Negation marking depends on different tense-mood distinctions. For example, the \textsc{present} category has a specific negation marking strategy while the \textsc{future} and the \textsc{past} each use different negation lexemes. These are, however, determined by the tense-mood categories and not by aspectual categories. Tense categories are discussed in detail in \sectref{sec:GramTM}.


\subsection*{Mood and modality}  The term ``grammatical mood'', as discussed by \citet{nuyts2016}, has come to refer to a heterogeneous set of distinctions: (i)  grammatical coding of modal meanings through the verb, (ii) the distinction between  basic sentence types and their related illocutionary categories, and (iii) the distinction between indicative and subjunctive or between realis and irrealis.

The challenge of adopting the term {\itshape mood} is assuaged by the form-based approach taken in this grammar, since it is not necessary to specify how Gyeli encodes the general (and unclear) category of mood, but rather to examine forms and their interpretations, wherein mood simply designates a class of related types of interpretations.
 


Mood and modality in Gyeli are expressed through various construction types, differing in their structural complexity.  The distinctions among sentences associated with different illocutionary categories are encoded by different basic tonal patterns for indicative vs.\ imperative or subjunctive.  The distinction between realis and irrealis is encoded through additional syntactic tone patterns. Finally, grammatical coding of fine-grained modal meanings is achieved with auxiliaries and/or combinations of tense categories (future) or other mood distinctions (subjunctive).

 

%The most primary distinction among these drawn in Gyeli is the encoding of a basic mood distinction between ``realis'' and ``irrealis''. 

%The literature discusses the term ``mood'' generally in conjunction with ``modality''.   \citet[326]{timberlake2007} views grammatical systems of mood as ``modality crystallized as morphology''. Other mood categories discussed in the literature on other languages include, for instance, `indicative' versus `non-indicative', and also `imperative'. mood as comprising illocution (i.e. types of speech acts) and  modality (i.e., expressions which can be characterized in terms of possibility and necessity (cf. van der Auwera \& Plungian 1998).modality is concerned with the status of the proposition that describes the event (Palmer 1986: 1) modality ``is consideration of alternative realities mediated by an authority'' \citet[315]{timberlake2007}




\iffalse

\begin{table}

\begin{tabular}{l|lll}
 \lsptoprule
\multirow{2}{*}{simple predicates} & basic tone patterns & $\rightarrow$ & indicative/imperative/subjunctive\\
 \midrule
					& syntactic tone patterns & $\rightarrow$ & realis/irrealis\\
complex predicates & auxiliaries	& $\rightarrow$ & modality\\
 \midrule
complement clauses &?? & $\rightarrow$ & subjunctive\\
 \lspbottomrule
\end{tabular}
\caption{Constructions expressing mood and modality}
\label{Tab:MoMod}
\end{table}

\fi

I will refer to {\itshape mood} throughout this chapter as pertaining only to grammatical tense-mood categories whereas the term {\itshape modality}  will pertain to the more specific semantic categories, such as possibility or ability. \tabref{Tab:Modality} gives an overview of the expression of different types of modality.

\begin{table}

\begin{tabularx}{\textwidth}{X ll}
\lsptoprule
Type & Mood category &  \\
 \midrule
Ability/dynamic (can) & expressed by realis & $\rightarrow$ realis  H tone\\
Deontic (must) & expressed by realis & $\rightarrow$ realis  H tone\\
Possibility  & expressed by irrealis (\textsc{fut}) & $\rightarrow$ no realis  H tone\\
Bouletic & expressed by irrealis (\textsc{sbjv})  & $\rightarrow$ no realis  H tone\\
\lspbottomrule
\end{tabularx}
\caption{Modality expression and mood}
\label{Tab:Modality}
\end{table}

The mood distinction between realis and irrealis is presented in \sectref{sec:GramTM}, while modality categories are described in \sectref{sec:ComplSemi}.

\subsection*{Aspect}
Tense and aspect are often referred to as an interlocking system. It sometimes can be hard to distinguish whether a given form expresses tense or aspect since, in many languages, forms may express both at the same time. For this reason, some authors \citep{dahl85, bybee94} prefer to investigate so-called ``gram-types'', i.e.\ categories such as ``future'', ``past'', ``perfective'', and ``imperfective'', without attempting to group these grams into higher categories, such as tense or aspect.
In my account of Gyeli tense-mood-aspect categories, I will also consider gram-type-like categories, based on their formal commonalities. I represent these categories with small capitals, for instance \textsc{progressive} or \textsc{habitual}.

Tense-mood and aspect marking are for the most part differentiated formally. While tense-mood is mainly expressed tonally (and obligatorily) on the \textsc{stamp} marker and verb, aspect marking is achieved through (optional) segmental material, mainly auxiliary verbs.
%REPETITIVE While tense-mood marking in Gyeli is obligatory since every \textsc{stamp} marker and verb has to surface with a certain tonal pattern that characterizes the single categories, aspect marking is optional. 
Aspect marking is also significantly less frequent in the corpus (122 occurrences), compared to utterances with tense-mood marking only (369 occurrences).  


Gyeli has eight aspect markers, which are presented in \tabref{Tab:Aspect}.
The table contains information on the morphosyntactic status of each aspect marker,  the tonal pattern of its \textsc{stamp} marker, its form,  its tense-mood restriction, and its function that is used in glossing examples and texts.

\begin{table}
\fittable{%
\begin{tabular}{l llll}
 \lsptoprule &   \textsc{stamp} &  Auxiliary form     & Restrictions           & Function \\
 \midrule
\multirow{6}*{True auxiliary} & yà &  {\bfseries nzíí} & special pattern 1 & \textsc{prog.pres} \\
&  yà  & {\bfseries nzɛ́ɛ́} & special pattern 1 & \textsc{prog.sub}  \\
 & yà, yáà &  {\bfseries nzí} & \textsc{pst1}, \textsc{pst2} & \textsc{prog} \\
&  yá &  {\bfseries lɔ́}   & \textsc{prs} & \textsc{retro} \\
&  mɛ̀, yá &  {\bfseries múà} `be'  &  special pattern 2 & \textsc{prosp}   \\
 & yà, yáà &  {\bfseries bwàá} `have' & \textsc{pst1}, \textsc{pst2} & \textsc{prf}\\ 
  \midrule
Stem reduplication & yá &  STEM-copy & \textsc{prs} & \textsc{hab}\\ 
 \midrule
Postverbal particle & yà &  {\bfseries mɔ̀/-Ṽ́Ṽ̀} & \textsc{pst1} & \textsc{compl}\\ 
 \lspbottomrule
\end{tabular}}
\caption{Distribution of aspect markers across morphological and syntactic constructions}
\label{Tab:Aspect}
\end{table}

\tabref{Tab:Aspect} reveals that aspect marking is structurally diverse. While most aspect categories are encoded by a true auxiliary (\sectref{sec:AUX}) in a complex predicate construction (\sectref{sec:ComplAUX}), other aspect marking strategies are achieved through expanded simple predicates, which are morphologically complex, but not syntactically complex (\sectref{sec:MorphSimp}).

Only grammaticalized markers are counted here as grammatical aspect markers (\sectref{sec:AUX} and \sectref{sec:ComplAUX}). There are, however,  also non-grammaticalized semi-aux\-il\-ia\-ries which can carry aspectual meaning, such as {\itshape kɛ̀} `go', which can have an altrilocal meaning (i.e.\ the event takes place at a different location than the utterance) or {\itshape sílɛ} `finish', which can lend itself to a non-complete accomplishment reading (\sectref{sec:ComplSemi}).
Aspect categories are discussed both in simple predicates (\sectref{sec:MorphSimp}) and complex predicates (\sectref{sec:ComplAUX}). 







\subsection*{Negation}

  

Gyeli uses different negation markers and strategies for different grammatical categories and clausal constructions, as summarized in \tabref{Tab:NEG}. The table also shows the frequency of each negation marker in the corpus.

\begin{table}

\begin{tabularx}{\textwidth}{X @{}ll r@{~}r}
 \lsptoprule
Negation marker & Status & Distribution & \multicolumn{2}{l}{Frequency}\\
 \midrule
\multicolumn{5}{l}{Standard negation} \\
\midrule
{\bfseries -lɛ} &  negation suffix & Present & 23 & 59.0\% \\
{\bfseries sàlɛ́/pálɛ́} &  true auxiliary & Past tenses & 4 & 10.3\%  \\
{\bfseries kálɛ̀} &  true auxiliary & Future & 3 & 7.7\%  \\
 \midrule
\multicolumn{5}{l}{Non-standard clausal negation} \\
\midrule
{\bfseries dúù} `must not' & modal  semi-auxiliary & Subjunctive, present & 2 & 5.1\% \\
{\bfseries tí} &  true auxiliary & Imperative, infinitive,  & 7 & 17.9\% \\
 &  & present (PCF focus) & &  \\
 \midrule
Total & & & 39 &  \\
 \lspbottomrule
\end{tabularx}
\caption{Negation markers}
\label{Tab:NEG}
\end{table}

I distinguish standard from non-standard negation, following \citet[1]{miestamo2005} in his definition of standard negation being ``the basic way(s) a language has for negating declarative verbal main clauses''.  In Gyeli, standard negation differs not only in the form of negation markers across tense categories, but also in the negation markers' morphosyntactic status. While negation in the \textsc{past} tenses and the \textsc{future} is syntactically marked by true auxiliaries,  \textsc{present} negation is achieved morphologically through a suffix that attaches to the finite main verb. Non-standard clausal negation comprises two negation markers, a modal semi-auxiliary, and a true auxiliary, all of which are used in different tense-mood categories, sentence types, and information structure constructions, as outlined in detail in \sectref{sec:MorphSimp} and \sectref{sec:ComplAUX}.











\section{Simple verbal predicates}
\label{sec:SimpPred}

Simple verbal predicates consist of the \textsc{stamp} marker (as discussed in \sectref{sec:SCOP}) and a finite main verb:\footnote{The finite verb can take an inflectional suffix or postverbal clitic in accordance with its properties as a finite verb. This finite verb inflection, however, does not change the overall structure of simple verbal predicates.}

\begin{center}
\textsc{stamp} - Verb\textsubscript{finite}
\end{center}

\noindent The combined tonal patterns of the \textsc{stamp} marker and the verb instantiate tense-mood categories, as further discussed in \sectref{sec:GramTM}. \REF{simppred} shows that simple predicates can encode further grammatical information: subpattern I pertains to a verb-final H tone that attaches to the verb in certain tense-mood categories if the verb is in non-phrase-final position. The presence or absence of the grammatical H tone correlates with a realis/irrealis mood distinction.


\eabox{\label{simppred}
\begin{tabular}{l|lll}
Simple predicates:   &  \textsc{stamp} Verb & $\rightarrow$ & Tense-Mood\\
\hspace{.5cm} Subpattern I: &  \textsc{stamp} Verb(-H) &  $\rightarrow$ & Realis/Irrealis\\
\hspace{.5cm} (Subpattern II: &  \textsc{stamp} Verb-Suffix/Clitic & $\rightarrow$ & Aspect, Negation) \\
\end{tabular}
}

\noindent Subpattern II includes expanded simple predicates that are morphologically complex in that they involve a verbal suffix (\sectref{sec:NEGSuff}) or verbal clitic (\sectref{sec:mo}) encoding certain aspect and negation categories on the finite verb. 
Valency changing derivational suffixes, as described in \sectref{sec:VDeriv}, do not fall into this category as they are not inflectional, i.e.\ their occurrence is not restricted to finite verbs. \REF{VSuff1} shows that both the negation and the derivational reciprocal suffix attach to the finite verb of the sentence.

\ea\label{VSuff1}
\ea  \label{VSuff1a}
  \glll  bá dyúlɛ́ \\
          ba-H dyû-lɛ \\
            2-\textsc{prs} kill-{\NEG} \\
    \trans `They do not kill.'
\ex\label{VSuff1b}
 \glll  bá dyúwàlà \\
         ba-H dyû(w)-ala\\
	2-\textsc{prs} kill-{\RECIP} \\
    \trans `They kill each other.'
\z
\z

\noindent In complex predicates with true auxiliaries, however, the negation suffix cannot attach to the main verb, whereas derivational suffixes can, as shown in \REF{VSuff2}.\footnote{The tonal pattern on the \textsc{stamp} marker changes with true auxiliaries, as discussed in \sectref{sec:ComplAUX}. This is accounted for in the example: the ungrammaticality does not derive from the tonal pattern but from the morphology.}

\ea\label{VSuff2}
\ea[*]{\label{VSuff2a}
  \glll  ba nzí dyúlɛ̀ \\
         ba \textsc{prs}.{\PROG} dyû-lɛ \\
        2 kill-{\NEG} \\
    \trans `They are not killing.'}
\ex\label{VSuff2b}
 \glll  bà nzí dyúwàlà \\
         ba nzí dyû(w)-ala\\
	2-\textsc{prs} kill-{\RECIP} \\
    \trans `They are killing each other.'
\z
\z


Another argument for verb derivational suffixes and inflectional morphology belonging to different categories comes from their distribution: aspect and negation markers are in complementary distribution and cannot co-occur, as shown in \REF{VSuff3}. Although \REF{VSuff3a} and \REF{VSuff3b} would be ungrammatical no matter what since they have a conflict in their tense categories (-{\itshape lɛ} negates the present and {\itshape mɔ̀} occurs only in past tenses), \REF{VSuff3c} illustrates that the co-occurrence of aspect and negation in a complex predicate is ungrammatical even within the same tense category.

\ea\label{VSuff3}
\ea[*]{\label{VSuff3a}
  \glll  bá dyúlɛ́ mɔ̀  \\
         ba-H dyû-lɛ mɔ̀ \\
        2-\textsc{prs} kill-{\NEG} {\COMPL} \\
    \trans `They have not killed.'}
\ex[*]{\label{VSuff3b}
 \glll bá dyú mɔ́lɛ́ \\
         ba-H dyû-H mɔ̀-lɛ \\
	2-\textsc{prs} kill-{\R} {\COMPL}-{\NEG} \\
    \trans `They have not killed.'}
\ex[*]{\label{VSuff3c}
 \glll bà sàlɛ́ dyû mɔ̀ \\
         ba sàlɛ́ dyû mɔ̀ \\
	2 {\NEG}.{\PST} kill {\COMPL} \\
    \trans `They have not killed.'}
\z
\z

\noindent In contrast,  derivational suffixes can combine with negation marking on finite verbs across different tenses, as illustrated in \REF{VSuff4a}, with the derivational suffix preceding the negation suffix. If the lexical verb is not the finite verb, as in \REF{VSuff4b}, then negation is encoded by the finite verb auxiliary, while the derivational suffix still attaches to the non-finite lexical verb. 

\ea\label{VSuff4}
\ea  \label{VSuff4a}
  \glll  bá dyúwálálɛ́  \\
          ba-H dyû(w)-ala-lɛ \\
            2-\textsc{prs} kill-{\RECIP}-{\NEG} \\
    \trans `They do not kill each other.' \\
\ex\label{VSuff4b}
  \glll  bà sàlɛ́ dyúwàlà \\
          ba sàlɛ́ dyú(w)-ala\\
            2 {\NEG}.{\PST} kill-{\RECIP} \\
    \trans `They did not kill each other.' \\
\z
\z

The remainder of this section is organized as follows: I first present the most basic simple predicates, which consist only of the \textsc{stamp} marker and the finite verb (\sectref{sec:GramTM}). I then outline simple predicate subpattern I,  which involves the presence or absence of a realis-marking H tone (\sectref{sec:SynH}) before I turn to discussing subpattern II, involving morphologically expanded simple predicates (\sectref{sec:MorphSimp}). 






\subsection{Basic simple predicates}
\label{sec:GramTM}

A remarkable feature of Gyeli is that tense-mood distinctions are entirely expressed through tone, lacking any segmental material (except for vowel lengthening in some tense-mood categories).\footnote{Although tone also plays a central role in TAM marking in other northwestern Bantu languages, there is usually some segmental marking in those languages as well. Compare, for instance, \citet{makasso2012} for Basaa (A43) and \citet{beavon91} for Kɔɔzime (A842).}
Consider the surface forms of the minimal pair in \REF{Tmin}.

\ea\label{Tmin}
\ea  \label{Tmin1}
  \glll  mɛ́ dè \\
  mɛ-H dè \\
            1\textsc{sg}-{\PRS} eat\\
    \trans `I eat.'
\ex\label{Tmin2}
 \glll   mɛ̀ dé \\
 mɛ dé \\
         1\textsc{sg}.{\PST}1 ate\\
    \trans `I ate.'
\z
\z

\noindent In the \textsc{present} in \REF{Tmin1}, the \textsc{stamp} marker has an H tone while the tone on the verb stem is L. In contrast, in \REF{Tmin2}, the \textsc{past} form is characterized by an L tone on the \textsc{stamp} marker and an H tone on the verb.
Form patterns thus arise from the tonal combinations of the \textsc{stamp} marker and the simple finite predicate.\footnote{Tonal patterns of the \textsc{stamp} marker are different in some categories of complex predicates that use a true auxiliary, as described in \sectref{sec:ComplAUX}.} 

Gyeli exploits all tonal possibilities of the language in tense-mood encoding, including three different tonal patterns on verb stems and four on \textsc{stamp} markers, as shown in \REF{PredTone}. These patterns surface when the predicate is in phrase-final position.\footnote{The verb tone pattern changes in some tense-mood categories that take a grammatical H tone when the verb is not in phrase-final position. This is discussed in \sectref{sec:SynH}.}

\ea \label{PredTone}
\ea  Verb tones: L, H, HL
\ex \textsc{stamp} tones: L, H, HL, LH
\z
\z

\noindent The combination of the verb and \textsc{stamp} marker tonal patterns instantiates seven categories that mainly encode tense and mood, to varying degrees (the \textsc{inchoative} category  also carries some aspectual meaning). While mood encoding is most obvious for the tenseless \textsc{imperative} and \textsc{subjunctive} categories,  the other categories also inherently belong to the realis or irrealis category, as explained in \sectref{sec:SynH}.  

As \tabref{Tab:TM-Tone} shows, the verb tone patterns express  basic meaning distinctions: an L verb tone indicates \textsc{non-past} tense-mood categories, an H tone indicates \textsc{past} tense-mood categories, and an HL pattern on the verb encodes tenseless categories. Tonal patterns on the \textsc{stamp} marker then reflect more fine-grained subcategories.\footnote{The \textsc{stamp} marker of the \textsc{imperative} category is marked in parentheses in \tabref{Tab:TM-Tone}, since the first person plural is the only agreement class in which the \textsc{stamp} marker appears, as described in \sectref{sec:imp}.} While tonal patterns in a specific category are the same across persons, there is an exception in the \textsc{future}, which generally is characterized by an HL tone on a long \textsc{stamp} marker vowel. For the first and second person singular and the \textsc{stamp} marker of agreement class 1, however, the long vowel has an LL tone pattern. There are further exceptions regarding the \textsc{stamp} marker tone in some grammatical categories: the \textsc{stamp} marker is different in the morphologically marked \textsc{present} negation with -{\itshape lɛ} (\sectref{sec:NEGPRES}) and in complex predicates that contain the \textsc{progressive} markers {\itshape nzíí} or {\itshape nzɛ́ɛ́} (\sectref{sec:PROG}), the \textsc{prospective} auxiliary {\itshape múà} `be almost' (\sectref{sec:PROSP}), or the negation marker {\itshape tí} when it is used in \textsc{present} main clauses (\sectref{sec:NEGti}).


\begin{table}
\small
\begin{tabularx}{\textwidth}{ll@{} llX l}
 \lsptoprule
Basic  & TM  & \textsc{stamp} & Verb& Verb & Gloss\\
 distinction &  category &  & Stem & Tone &   \\
 \midrule
 & \textsc{prs} & {\bfseries yá} &  {\bfseries dè} & \multirow{3}{*}{L} &  `we eat' \\
\textsc{non-pst} & \textsc{inch} & {\bfseries yàá} & {\bfseries dè} & &  `we are at the beginning of eating' \\
 & \textsc{fut} & {\bfseries yáà/mɛ̀ɛ̀} &  {\bfseries dè}  & &  `we/I will eat' \\
 \midrule
\multirow{2}{*}{\textsc{pst}} &  \textsc{pst1} & {\bfseries yà} &  {\bfseries dé} & \multirow{2}{*}{H} &  `we ate (recently)' \\
 & \textsc{pst2} & {\bfseries yáà} & {\bfseries dé} & &  `we ate (a long time ago)' \\
 \midrule
\multirow{2}{*}{tenseless} & \textsc{imp} &  {\bfseries (yá)}  & {\bfseries dê}    & \multirow{2}{*}{HL} &  `let's eat!' \\
 & \textsc{sbjv} & {\bfseries yá}  & {\bfseries déè} & &  `may we eat' \\
 \lspbottomrule
\end{tabularx}
\caption{Tonal patterns of tense-mood categories}
\label{Tab:TM-Tone}
\end{table}

The tenseless categories \textsc{imperative} and \textsc{subjunctive}\footnote{These categories are form-identical to monosyllabic HL stems and monosyllabic HL stems with a long vowel, respectively. For instance, {\itshape nyɛ̂} `see' encodes both the non-finite form and the imperative form, and {\itshape ntã́ã̀} `climb over' encodes both the non-finite and the subjunctive form.} differ from one another not only in their final vowel length, but also in the underlying tonal process which pertains to the presence or absence of High Tone Spreading (HTS) in trisyllabic verb forms. While no {\HTS} occurs in \textsc{imperatives} where the penultimate syllable in trisyllabic verbs surfaces as L, {\HTS} occurs in \textsc{subjunctives} in the same phonological environment. Thus, the penultimate syllable in trisyllabic verbs surfaces as H, as shown in \tabref{Tab:TM-verbs}. In contrast to the \textsc{imperative}, the \textsc{subjunctive} further shows phonetic variation of the final long vowel. This vowel may occur with a glottal stop, as indicated by the apostrophe in, for instance, {\itshape á dé'è} `may he eat', or as a pharyngealized vowel. All these forms occur in free variation. In fast speech, there is a tendency for the vowel to be lengthened, but not pharyngealized or glottalized.


As described in \sectref{sec:Tinventory}, verb stems have one, two, or three syllables, of which only the first syllable is specified for tone. In contrast, second and third syllables are underlyingly toneless. The verb {\itshape dè} `eat' used as an example in \tabref{Tab:TM-Tone} thus only represents one tonal-phonological set of verbs, namely the monosyllabic ones specified with an L tone.  The tonal rules that apply to the other tonal-phonological verb sets are described in \sectref{sec:HTSl}. \tabref{Tab:TM-verbs} further provides an overview of the tonal patterns for different phonological verb types in the different tense-mood categories.\footnote{Monosyllabic HL verb stems with a long vowel, such as \ {\itshape láà} `tell', are form-identical in their non-finite, \textsc{imperative}, and \textsc{subjunctive} forms.}



\begin{table}
 \fittable{%
\begin{tabular}{l l@{~}l @{\qquad} l@{~}l @{\qquad} l@{~}l}
 \lsptoprule
TM          & L verb         &  HL verb           & L $\emptyset$ verb & H $\emptyset$ verb  & L $\emptyset$ $\emptyset$ verb & H $\emptyset$ $\emptyset$ verb\\
category          & {\itshape kɛ̀} `go' & {\itshape nyɛ̂} `see'   & {\itshape gyàga} `buy'       & {\itshape gyíbɔ} `call' & {\itshape vìdɛga} `turn'  & {\itshape lúmɛlɛ} `send' \\
 \midrule
  \textsc{prs} &  kɛ̀ & nyɛ̂ & gyàgà & gyíbɔ̀ & vìdɛ̀gà & lúmɛ̀lɛ̀ \\
 \textsc{inch} & kɛ̀ & nyɛ̂ & gyàgà & gyíbɔ̀  & vìdɛ̀gà & lúmɛ̀lɛ̀ \\
  \textsc{fut} &   kɛ̀  & nyɛ̂ & gyàgà  & gyíbɔ̀  & vìdɛ̀gà & lúmɛ̀lɛ̀\\
 \midrule
 \textsc{pst1} &  kɛ́ & nyɛ́ & gyàgá & gyíbɔ́  & vìdɛ́gá & lúmɛ́lɛ́\\
  \textsc{pst2}  & kɛ́ & nyɛ́ & gyàgá & gyíbɔ́ & vìdɛ́gá & lúmɛ́lɛ́\\
 \midrule
 \textsc{imp}   & kɛ̂    & nyɛ̂ & gyàgâ & gyíbɔ̂  & vìdɛ̀gâ & lúmɛ̀lɛ̂ \\
  \textsc{sbjv}  & kɛ́ɛ̀ & nyɛ́ɛ̀  & gyàgáà & gyíbɔ́ɔ̀ & vìdɛ́gáà & lúmɛ́lɛ́ɛ̀ \\
 \lspbottomrule
\end{tabular}}
\caption{Verb tone patterns in different TM categories by phonological verb set}
\label{Tab:TM-verbs}
\end{table} 

Looking at the occurrence of the different tense-mood categories in the Gyeli corpus, it becomes clear that the categories are not evenly distributed.  \tabref{Tab:TMFreq} shows the frequency of each tense-mood category expressed through simple predicates in the corpus. It also specifies the mood category to which each tense-mood category belongs (\sectref{sec:SynH}). 

\begin{table}
\begin{tabular}{lllrr}
 \lsptoprule
Basic distinction & TM category & Mood & \multicolumn{2}{c}{Frequency}\\\midrule
 & \textsc{prs} & realis &  217 & 58.8\% \\
\textsc{non-pst} & \textsc{inch} & realis & 5 & 1.4\% \\
 & \textsc{fut} & irrealis &  40 & 10.8\% \\ \midrule
\multirow{2}{*}{\textsc{pst}} &  \textsc{pst1} & realis &  69 & 18.7\% \\
 & \textsc{pst2} & realis & 8 & 2.2\% \\ \midrule
\multirow{2}{*}{other} & \textsc{imp} & irrealis &  13 & 3.5\% \\
 & \textsc{sbjv} & irrealis  & 17 & 4.6\% \\ \midrule
Total & & & 369 & \\
 \lspbottomrule
\end{tabular}
\caption{Frequency of tense-mood categories in the corpus}
\label{Tab:TMFreq}
\end{table}

\noindent  There are 369 instances of simple predicates in the corpus. The vast majority (58.8\%) are encoded for the \textsc{present} category. While \textsc{past1} and \textsc{future} are still relatively frequent, the other tense-mood categories occur rarely. In the following sections, I discuss each tense-mood category with respect to its meaning and usage.








\subsubsection{\textsc{Present}}
\label{sec:pres}

The \textsc{present} is the most frequent tense-mood category in the corpus for all text genres and can be viewed as the default tense-mood category in narratives. For example, in the autobiographical narrative presented in \appref{sec:Antelope}, the narrator switches to the \textsc{present} in the tenth intonation phrase, despite having started out in the \textsc{past 1}.

In out-of-the-blue contexts, the \textsc{present} primarily relates to a time that is identical to speech time. Thus, the sentence in \REF{PRES1} is set at the time of utterance.


\ea\label{PRES1}
  \glll  mɛ́ gyámbɔ́ bédéwɔ̀ \\
         mɛ-H gyámbɔ-H H-be-déwɔ̀ \\
            1\textsc{sg}-\textsc{prs} cook-{\R} {\OBJ}.{\LINK}-be8-food\\
    \trans `I cook food.'
\z

\noindent Within a specific context requiring common ground for the speech act participants, however, the sentence in \REF{PRES1} can alternatively relate to a time that follows speech time. The \textsc{present} can thus be used to refer to future events as well as present ones. It is hard to delimit how far into the future the \textsc{present} may refer, and does not seem to be categorically bounded by, for instance, time of day or even periods of multiple days. Especially when temporal adverbs or other means of time reference are used, as in \REF{presfut},\footnote{The speaker was not in Ngolo when he uttered this sentence. The verb {\itshape kɛ̀} `go' has an  altrilocal meaning, as described in \sectref{sec:ComplSemi}, and is not a grammatical means of marking future tense.} the grammatical \textsc{present} form can extend into at least several days in the future.

\ea\label{presfut}
  \glll  {\bfseries mɛ́} kɛ́ jì ɛ́ Ngòló sɔ́ndɔ̀ nɔ́nɛ́gá \\
         mɛ-H kɛ̀-H jì ɛ́ Ngòló sɔ́ndɔ̀ n-ɔ́nɛ́gá \\
          1\textsc{sg}-\textsc{prs} go-{\R} stay {\LOC} $\emptyset$7.{\PN} $\emptyset$1.week 1-other\\
    \trans `I will stay in Ngolo next week.'
\z


The \textsc{present} tense form can also be used for imperative meanings, as in \REF{presimp}.
Formally, the \textsc{present} in \REF{presimp1} is clearly distinct from the \textsc{imperative} pattern in \REF{presimp2} in terms of the presence or absence of the \textsc{stamp} marker, the tonal pattern on the verb, and the realis-marking H tone in the \textsc{present} (see \sectref{sec:SynH}), which is absent in the \textsc{imperative}. 

\ea\label{presimp}
\ea \label{presimp1}
  \glll  bwáá láá bɔ̂  \\
         bwáa-H láà-H b-ɔ̂ \\
            2\textsc{pl}-\textsc{prs} tell-{\R} 2-{\OBJ}     \\
    \trans `You tell them!'
\ex\label{presimp2}
  \glll  láà ngá bɔ̂ \\
         láà nga-H b-ɔ̂ \\
           tell.{\IMP} {\PL}-{\OBJ}.{\LINK} 2-{\OBJ}     \\
    \trans `Tell them!'
\z
\z



The \textsc{present} is further used in generic contexts or for states that persist, as in \REF{presgen}. Here, the speaker talks about a general problem that applies to the time of utterance but also extends to an unbounded time both before and after.

\ea\label{presgen}
  \glll     y{\bfseries á} tfúg{\bfseries á} nà ngùndyá mpángì \\
            ya-H tfúga-H nà ngùndyá mpángì \\
              1\textsc{pl}-\textsc{prs} suffer-{\R} {\COM} $\emptyset$9.raffia $\emptyset$7.bamboo\\
    \trans `We suffer from the straw, the bamboo [that is used for thatched roofs].'
\z

While the \textsc{present} tense-mood category seems to be easily applied to the time at and after speech time, it extends less easily to time before the utterance. Thus, the sentence in \REF{PRES1} cannot be interpreted, under any circumstances, as having happened already. This correlates with the macro-distinction between \textsc{non-past} and \textsc{past} tense-mood categories.





\subsubsection{\textsc{Inchoative}} 
\label{sec:inch}


The \textsc{inchoative} form refers to the entry into a state or to the beginning of an event.
In the literature,  the inchoative is generally assumed to be an aspectual category, which may differ in flavor depending on the language:
the inchoative has been observed as part of the viewpoint aspectual system{\textemdash}\textsc{aspect}\textsubscript{1} in \posscitet{sasse2002} terms{\textemdash}for example by \citet{melchert80} and \citet{wichaya2013}, who gives an example for Fengshun Hakka in \REF{Hakka}.


\ea\label{Hakka} Fengshun Hakka; Sinitic \citep[50]{wichaya2013}\\
  \gll Nai\textsuperscript{11}  min\textsuperscript{11} phak\textsuperscript{55} liau\textsuperscript{42} \\
      1\textsc{sg} understand \textsc{inch} \\
\trans `I have understood.'
\z

\noindent The \textsc{inchoative} has also been related to the Aktionsart of a verb (Sasse's \textsc{aspect}\textsubscript{2}) by, for instance, \citet{botne83}, \citet{klein95}, and \citet{talmy2007}. An example is given for Russian in \REF{Russian}.

\ea\label{Russian} Russian; Slavic \citep[226]{braginsky2008}\\
  \gll zvezda za-sverkala\textsuperscript{PRF} na nebe \\
      star  \textsc{inch}-twinkled on sky\\
\trans `The star started twinkling in the sky.'
\z

The Gyeli inchoative both shifts the viewpoint to the beginning of a situation and locates the situation temporally at speech time (or narration time in the case of story-telling). This is clearly the case when opposing the \textsc{inchoative} with other aspectual categories (see \sectref{sec:ComplAUX}) in elicitation, as in \REF{INCH11}.

\ea\label{INCH11}
\ea \label{INCH11a}
  \glll  mɛ̀ɛ́ dè  \\
         mɛ̀ɛ́ dè\\
           1\textsc{sg}.{\INCH} eat\\
    \trans `I'm beginning to eat.'
\ex\label{INCH11b}
  \glll  mɛ̀ nzíí dè  \\
         mɛ nzíí dè\\
           1\textsc{sg} {\PROG}.\textsc{prs} eat\\
    \trans `I'm eating.'
\ex\label{INCH11c}
  \glll  mɛ̀ múà dè  \\
         mɛ múà dè\\
           1\textsc{sg} {\PROSP} eat\\
    \trans `I'm about to eat.'
\z
\z

\noindent Speakers describe that, in \REF{INCH11a}, the focus is on the starting point of the action: the person is just taking the first few bites of her meal. In contrast, \REF{INCH11b} emphasizes the ongoing character of the eating event, without specifying the exact point within the action (beginning, middle, or end). Also the \textsc{prospective} aspect, shown in \REF{INCH11c}, differs in that the person is about to take the first bites, but has not actually started eating yet.

The example in \REF{INCH1} is taken from natural text and can be similarly interpreted. It is at the moment when the woman arrives at the river bank that she breaks out in tears, and the activity of crying is (theoretically) unbounded.

\ea\label{INCH1}
  \glll  ndɛ̀náà pámò lébũ̂ {\bfseries àá} gyì \\
         ndɛ̀náà pámo H-le-bũ̂ àá gyì \\
        like.this arrive {\OBJ}.{\LINK}-le5-river.bank 1.{\INCH} cry\\
    \trans `Having arrived like this [without the child] at the river bank, she starts to cry.'
\z

Activities{\textemdash}in terms of \posscitet{vendler67} classification of Aktionsarten{\textemdash}can also be accompanied by temporal adjuncts specifying the duration of the event, as shown in \REF{inchbound}.

\ea\label{inchbound}
  \glll {\bfseries àá} bámál{\bfseries á} tɔ́bá mpfùmɔ̀ nà pámò mɛ́nɔ́\\
       àá bámala-H tɔ́bá mpfùmɔ̀ nà pámo mɛ́nɔ́ \\
       1.{\INCH} scold-{\R} since  $\emptyset$3.midnight {\CONJ} arrive $\emptyset$7.morning\\
    \trans `He is starting to scold [now] at midnight and [it] will continue until the morning.'
\z

The inchoative is also compatible with a perfective reading and can be used with punctual events, as shown in \REF{inchfut}.

\ea\label{inchfut}
  \glll  pílɔ̀ {\bfseries àá} pándɛ̀ àà kfùmàlà bédéwɔ̀ bè sílɛ̃́ɛ̃̀ \\
         pílɔ̀ àá pándɛ àà kfùmala bédéwɔ̀ be sílɛ̃́ɛ̃̀ \\
           when 1.{\INCH} arrive 1.{\FUT} find {\OBJ}.{\LINK}-be8-food 8 finish.{\COMPL}  \\
    \trans `When he has arrived, he will find that the food is finished.'
\z





\subsubsection{\textsc{Future}}
\label{sec:fut}

The \textsc{future} category primarily relates to a time some point after speech time. Often, it is accompanied by temporal adverbials, as in \REF{FUT1}, where Nzambi tells the mice that they will eat the bones of the burned bodies the next day.

\ea\label{FUT1}
  \glll àà nàmɛ́nɔ́ bw{\bfseries áà} dè, nàmɛ́nɔ́ \\
        àà nàmɛ́nɔ́ bwáà dè nàmɛ́nɔ́ \\
       {\EXCL} tomorrow 2\textsc{pl}.{\FUT} eat tomorrow\\
    \trans `Ah, tomorrow you will eat, tomorrow.'
\z

\noindent The \textsc{future} category can also relate to intended acts, as in \REF{futintention}.


\ea\label{futintention}
  \glll  pílɔ̀ {\bfseries mɛ̀ɛ̀} {\bfseries bɛ̀} nyá mùdì {\bfseries mɛ̀ɛ̀} {\bfseries tɛ̀lɛ̀} mùdà ndáwɔ̀ \\
         pílɔ̀ mɛ̀ɛ̀ bɛ̀ nyá m-ùdì mɛ̀ɛ̀ tɛ̀lɛ mùdà ndáwɔ̀ \\
           when 1\textsc{sg}.{\FUT} be big \textsc{n}1-person 1\textsc{sg}.{\FUT} place great $\emptyset$9.house\\
    \trans `When I grow up, I will build a great house.'
\z

\noindent The \textsc{future} may also be used for promises, as in \REF{futpromise}.

\ea\label{futpromise}
  \glll  mɛ́ kàgɛ́ wɛ̂ nâ {\bfseries mɛ̀ɛ̀} {\bfseries njì} nàmɛ́nɔ́ \\
         mɛ-H kàgɛ-H wɛ̂ nâ mɛ̀ɛ̀ njì nàmɛ́nɔ́ \\
           1\textsc{sg}-\textsc{prs} promise-{\R} 2\textsc{sg}.{\OBJ} {\COMP} 1\textsc{sg}.{\FUT} come tomorrow\\
    \trans `I promise you that I will come tomorrow.'
\z


Apart from temporal reference, the \textsc{future} also expresses modal possibility, as in \REF{futposs}. The sentence in this example has two readings. In the first, the speaker is convinced that the bag will break; thus, a more temporal reading is implied. In the alternative reading, the speaker is understood to be expressing uncertainty, merely presenting the possibility that the bag might break.

\ea\label{futposs}
  \glll  ká wɛ́ kíyá lékɔ́'ɔ̀ kwámɔ́ dè kwámɔ́ {\bfseries nyíì} búlɛ̀ \\
         ká wɛ-H kíya-H H-le-kɔ́'ɔ̀ kwámɔ́ dè kwámɔ́ nyíì búlɛ \\
           if 2\textsc{sg}-\textsc{prs} put-{\R} {\OBJ}.{\LINK}-le5-stone $\emptyset$9.bag {\LOC} $\emptyset$9.bag 9.{\FUT} break\\
    \trans `If you put the stone into the bag, the bag will/might break.'
\z

\noindent Another example is given in \REF{Possibility}, in which a possibility reading (with a universal time reference) is intended.

\ea\label{Possibility}
  \glll ndí wɛ́ lèmbó nâ mbvúndá {\bfseries nyíì} bvúdà nà mbvúndá \\
        ndí wɛ-H lèmbo-H nâ mbvúndá nyíì bvúda nà mbvúndá \\
         but 2\textsc{sg}-\textsc{prs} know-{\R} {\COMP} $\emptyset$9.trouble 9.{\FUT} fight {\COM} $\emptyset$9.trouble\\
    \trans `But you know that violence will create more violence.'
\z





\subsubsection{\textsc{Recent past ({\PST}1)}}
\label{sec:pst1}

Gyeli distinguishes two \textsc{past} tense forms: the \textsc{recent past} ({\PST}1) and the \textsc{remote past} ({\PST}2).  The choice in using either one of the two \textsc{past} categories may depend more on subjective, attitudinal factors than on an objective deictic time reference. The \textsc{recent past} is the default past category.
It refers to situations that happened before speech time, as in \REF{pst1a}, where a more precise time is further specified by a temporal adverb.

\ea\label{pst1a}
  \glll  mɛ̀ gyámbɔ́ bédéwɔ̀ nàkùgúù \\
         mɛ gyámbɔ-H H-be-déwɔ̀ nàkùgúù\\
            1\textsc{sg}.{\PST}1 cook-{\R} {\OBJ}.{\LINK}-be8-food yesterday\\
    \trans `I cooked food yesterday.'
\z

According to \citet[22]{nurse08}, many Bantu languages distinguish past tense categories  such as hesternal and hodiernal past based on objective time intervals, namely days.  This, however, is not the case in Gyeli. Thus, when a phrase is lacking further time specification, as in \REF{pst1b}, it is not inferrable at what time precisely the event has transpired. The event in this sentence (visiting the Ngumba) could, based on context, be understood to have occurred earlier the same day, the day before, the week before, or even a year before speech time.

\ea\label{pst1b}
  \glll   mɛ̀ bɛ́ ngyɛ̃̂ Ngvùmbɔ̀ \\
          mɛ bɛ̀-H n-gyɛ̃̂ Ngvùmbɔ̀ \\
           1\textsc{sg}.{\PST}1 be-{\R} \textsc{n}1-guest $\emptyset$7.{\PN}  \\
    \trans `I was a guest of the Ngumba.'
\z

Temporal proximity is not based on objectively measurable parameters, but rather relates to the speaker's attitude towards the situation and, potentially, its impact on speech time. Thus, different situations that have the same temporal distance may be judged differently and therefore coded variously as \textsc{recent past} or \textsc{remote past}. For instance, speakers may use the \textsc{recent past} when reporting that they ate out with good friends the day before. In contrast, they may use the \textsc{remote past} to refer to their last meal at the same temporal distance (the day before) if they have not eaten anything since then, because not eating in 24 hours would be considered a long time.

The \textsc{recent past} is also used in story-telling to set the scene, as in \REF{pst1gen}. Even though this autobiographical event took place many years before the telling of the story (\appref{sec:Antelope}), the temporal distance is not important to the speaker at this point. Therefore, he uses the default \textsc{past} category.

\ea\label{pst1gen}
  \glll   yɔ́ɔ̀ ngã̀ nû à bɛ́ ngã̀   \\
          yɔ́ɔ̀ ngã̀ nû a bɛ̀-H ngã̂ \\
         so $\emptyset$1.healer 1.{\DEM}.{\PROX} 1.{\PST}1 be-{\R} $\emptyset$1.healer\\
    \trans `So, this healer was a healer.'
\z



\subsubsection{\textsc{Remote past ({\PST}2)}}
\label{sec:pst2}

The \textsc{remote past} category is the more marked past form, and it occurs significantly less frequently in the corpus. It refers to events that have happened relatively distantly in the past, where this notion of distance is based on the speaker's attitude rather than on objective deictic parameters. It can also have a pluperfect interpretation, although a following event at a later point in time need not be explicitly expressed.  A hint for a pluperfect reading of the \textsc{remote past} comes from translations into French, whereby a phrase such as {\itshape mɛ́ɛ̀ dé} `I ate (a long time ago)' is generally translated by speakers with the French pluperfect {\itshape j'avais mangé} `I had eaten'. 

The sentence in \REF{PST2a} illustrates both the subjective time distance to the event and the pluperfect interpretation. In this example, the chief of Ngolo talks about the dangers of the Bagyeli's lifestyle and points to a scar on his face that he got from a machete. By using the \textsc{remote past}, he expresses his attitude towards the injuring event as being temporally far away, but also implies that, in the meantime, things have changed again.

\ea\label{PST2a}
  \glll    mɛ́ bvú nâ nkwálá w{\bfseries úù} tfùndɛ́ mɛ̂ vâ \\
           mɛ-H bvû-H nâ nkwálá wúù tfùndɛ-H mɛ̂ vâ \\
              1\textsc{sg}-\textsc{prs} think-{\R} {\COMP} $\emptyset$3.machete 3.{\PST}2 miss-{\R} 1\textsc{sg}.{\OBJ} here\\
    \trans `I think that the machete had missed [injured] me here [and, since then, the wound has completely healed and only left a scar].'
\z

The same is true for his statement in \REF{PST2b}. Here, he talks about the former settlement before the current village of Ngolo was built. Again, it is not objectively inferrable whether the speaker had settled in the former village when he was a child or a young man or even only two years ago. Using the \textsc{remote past}, however, shows that in terms of relevance to the present situation, settling in the old village is rather remote.

\ea\label{PST2b}
  \glll   ɛ́ pɛ̀ m{\bfseries ɛ́ɛ̀} tɛ́ \\
          ɛ́ pɛ̀ mɛ́ɛ̀ tɛ̂-H \\
           {\LOC} over.there 1\textsc{sg}.{\PST}2 found-{\PST} \\
    \trans `I had originally settled over there [and since then I moved to the new place].'
\z

Presumably, the \textsc{remote past} is used in \REF{PST2125} rather than the \textsc{recent past} in order to stress the fact that the speaker in this folktale is too late to save his child, since it has already been devoured.

\ea\label{PST2125}
  \glll w{\bfseries ɛ́ɛ̀} dé mwánɔ̀ nɔ́ɔ̀ \\
       wɛ́ɛ̀ dè-H m-wánɔ̀ nɔ́ɔ̀ \\
      2.{\PST}2 eat-{\R} \textsc{n}1-child no\\
    \trans `You have eaten the child, haven't you?'
\z

The tense generally used in narratives is the \textsc{present}.
The \textsc{remote past} is, however, also found in narrations, such as the Nzambi folktale, when the narrator occasionally switches from \textsc{present} to past, as seen in \REF{pst2story}, where the three sentences appear in the same order in the story. \REF{pst2story1} starts out in the \textsc{present}, \REF{pst2story2} shows a temporal rupture using the remote past, and in \REF{pst2story3}, the speaker switches back to the general \textsc{present}.


\ea\label{pst2story}
\ea \label{pst2story1}
  \glll yɔ́ɔ̀ Nzàmbí wà núú nìyɛ̀ \\
        yɔ́ɔ̀ Nzàmbí wà núú nìyɛ \\
         so $\emptyset$1.{\PN} 1:{\ATT} 1.{\DEM}.{\DIST} return\\
    \trans `So that Nzambi returns [home].'
\ex\label{pst2story2}
  \glll ɛ́kɛ̀! Nzàmbí wà nú {\bfseries áà} sàlɛ́ bɛ̀ nà bã̂ líná-á pámò \\
      ɛ́kɛ̀! Nzàmbí wà nú áà sàlɛ́ bɛ̀ nà bã̂ líná a-H pámo\\
        {\EXCL} $\emptyset$1.{\PN} 1:{\ATT} 1.{\DEM}.{\DIST} 1.{\PST}2 {\NEG}.{\PST} be {\COM} $\emptyset$7.word when 1-\textsc{prs} arrive\\
    \trans `Oh! That Nzambi had no words when he arrived.'
\ex\label{pst2story3}
\glll nyɛ̀ nâ álè \\
       nyɛ nâ álè \\
       1 {\COMP} allez[French]  \\
    \trans `He [says]: ``{\itshape Allez!} [Ok]''.'
\z
\z

It seems that the use of the \textsc{remote past} is intended to sporadically relocate the story in time and emphasize that this (fictional) story happened a very long time ago. At the same time, the narrator can use the \textsc{remote past} as a means to distance himself from the story and comment about it. While the general chain of events is told in the \textsc{present}, the narrator's comments about the state of the character are realized in a different tense-mood category, the \textsc{remote past} in this case. 









\subsubsection{\textsc{Imperative}}
\label{sec:imp}


The category of  \textsc{imperative} is characterized by an HL tonal pattern on its ultimate syllable.
For semantic/pragmatic reasons, the \textsc{imperative} category is restricted with respect to the grammatical persons with which it can combine, yielding three subgroups: (i) singular forms that have no \textsc{stamp} marker, but only the bare imperative verb form, (ii) plural forms which have no \textsc{stamp} marker either, but a plural particle following the imperative verb form, and (iii) what I label as ``cohortative''  forms, which are almost identical to plural imperatives, with the exception that a  first person plural \textsc{stamp} marker with an H tone precedes the verb form. These are schematized  in \REF{IMPstruc}. As they all have the same verb tone pattern as well as the same negation strategy with {\itshape tí} (see \sectref{sec:NEGti}), they are unified under a single category.

\ea\label{IMPstruc}
\ea  2\textsc{sg}: [$\emptyset$  Verb.{\IMP}]
\ex 2\textsc{pl}: [$\emptyset$  Verb.{\IMP} plural]
\ex 1\textsc{pl}: [\textsc{stamp}  Verb.{\IMP} plural]
\z
\z

%\noindent Only the cohortative surfaces with the \textsc{stamp} marker, as shown in \tabref{Tab:TM-Tone}. The second person \textsc{imperative} forms surface without the \textsc{stamp} marker and are distinguished by the absence or presence of a verbal plural marker {\itshape (n)ga} (see chap\ref{sec:VParticle}) that is also part of the cohortative.

\noindent In the following, I provide examples of each subcategory.







For second persons, the \textsc{imperative} expresses requests, demands, and orders. 
 \REF{IMPSG} provides examples of singular imperative forms, translated with an exclamation mark. The examples cover all syllable lengths and tonal patterns found for verbs.

\ea\label{IMPSG}
\ea  dê `eat (sg.)!'
\ex nyɛ̂ `see (sg.)!'
\ex gyàgâ `buy (sg.)!'
\ex gyámbɔ̂ `cook (sg.)!' 
\ex vìdɛ́gâ `turn (sg.)!' 
\ex lúmɛ́lɛ̂  `send (sg.)!' 
\z
\z

In the corpus, \textsc{imperative} occurrences are rare, as they are pragmatically restricted to direct communicative interactions between speech act participants, as in \REF{impconv}.


\ea\label{impconv}
  \glll bímbú lɛ́ mámbòngò mâ wɛ̀ mɛ́dɛ́ díg{\bfseries ɛ̂} mɛ́dɛ́\\
         bímbú lɛ́ ma-mbòngò mâ wɛ mɛ́dɛ́ dígɛ̂ mɛ́dɛ́ \\
       $\emptyset$5.amount 5:{\ATT} ma6-plant 6.{\DEM}.{\PROX} 2\textsc{sg} self look.{\IMP} self\\
    \trans `The number of these plants, take a look yourself.'
\z


\noindent In narratives, they occur in the form of reported direct speech, as in \REF{impdrd}, where the \textsc{imperative} form is, in fact, the indicator of reported discourse through a switch of the deictic perspective.

\ea\label{impdrd}
  \glll  bàmbɛ́ k{\bfseries ɛ̂} jíì mbúmbù mwánɔ̀ sá yí dè  \\
        bàmbɛ́ kɛ̂ jíì mbúmbù m-wánɔ̀ sá yí dè \\
           sorry  go.{\IMP} ask $\emptyset$1.namesake \textsc{n}1-child $\emptyset$7.thing 7:{\ATT} eat\\
    \trans `Excuse me, go and ask [my] namesake [the other Nzambi] for a little to eat.'
\z



 

If the addressee of an order consists of more than one person, the plural particle {\itshape ga}, or its variant {\itshape nga}, is used, following the \textsc{imperative} verb form, as in \REF{IMPPL}.

\ea\label{IMPPL}
\ea
  \glll dê gà \\
         dê ga\\
         eat.{\IMP} {\PL}\\
    \trans `Eat (pl.)!' 
\ex 
  \glll gyàgâ gà \\
         gyàgâ ga\\
         buy.{\IMP} {\PL}\\
    \trans `Buy (pl.)!' 
\ex 
  \glll vìdɛ̀gâ gà \\
         vìdɛ̀gâ ga\\
         turn.{\IMP} {\PL}\\
    \trans `Turn (pl.)!'
\z
\z



Plural \textsc{imperatives} are less frequent than their singular counterparts in the corpus. Examples are given in \REF{IMPa} and \REF{IMPb}.

\ea\label{IMPa}
  \glll nyáà {\bfseries ngà} sílɛ́ nyî ndáwɔ̀ dé tù \\
       nyáà ngà sílɛ́-H nyî ndáwɔ̀ dé tù \\
       shit.{\IMP} {\PL} finish-{\R} enter $\emptyset$9.house {\LOC} inside\\
    \trans `Piss off, everybody go into the house!'
\z

\ea\label{IMPb}
  \glll sílɛ̂ {\bfseries ngà} nyî vâ \\
       sílɛ̂ ngà nyî vâ \\
        finish.{\IMP} {\PL} enter here\\
    \trans `Enter all here.'
\z




The cohortative describes a wish or invitation directed towards the first person plural and can be translated with English `let's'. Examples are given in \REF{IMP1PL}.

\ea\label{IMP1PL}
\ea
  \glll yá dê gà \\
         ya-H dè-HL ga\\
         1\textsc{pl}-\textsc{prs} eat-{\IMP} {\PL}\\
    \trans `Let's eat!' 
\ex 
  \glll yá gyàgâ gà \\
         ya-H gyàga-HL ga\\
         1\textsc{pl}-\textsc{prs} buy-{\IMP} {\PL}\\
    \trans `Let's buy [sth.]!' 
\ex 
  \glll yá vìdɛ́gâ gà \\
         ya-H vìdɛga-HL ga\\
         1\textsc{pl}-\textsc{prs} turn-{\IMP} {\PL}\\
    \trans `Let's turn around!'
\z
\z











\subsubsection{\textsc{Subjunctive}}
\label{sec:opt}

Examples of the \textsc{subjunctive} category in Gyeli are given in \REF{SBJVforms}, in this case with the agreement class 1 \textsc{stamp} marker. As outlined in \sectref{sec:GramTM}, the final long vowel may also be glottalized or pharyngealized, as in \REF{SBJV}.


\ea\label{SBJVforms}
\ea  á déè   `May he eat!'
\ex á nyɛ́ɛ̀ `May he see!'
\ex á gyàgáà `May he buy!' 
\ex á gyámbɔ́ɔ̀ `May he cook!'
\ex á vìdɛ́gáà `May he turn!'
\ex á gyíkɛ́sɛ́ɛ̀ `May he teach!'
\z
\z

The \textsc{subjunctive} in Gyeli is often (but not exclusively) used in subordinate clauses to express (i) wishes or advice \REF{SBJV1}, (ii) obligations \REF{SBJV2}, or (iii) prohibitions \REF{SBJV3}.

\ea\label{SBJV}
\ea \label{SBJV1}
  \glll   á lã́ã́ mɛ̂ nâ mɛ́ v{\bfseries ɛ́'ɛ̀} bwánɔ̀ bèfùmbí \\
         a-H lã́ã̀-H mɛ̂ nâ mɛ-H vɛ́'ɛ̀ b-wánɔ̀ be-fùmbí \\
             1-\textsc{prs} tell-{\R} 1\textsc{sg}.{\OBJ} {\COMP} 1\textsc{sg}-\textsc{prs} give.{\SBJV} ba2-child be8-orange\\
    \trans `He tells me that I should give the children oranges.'
\ex \label{SBJV2}
  \glll  yíì mpìnàgà nâ wɛ́ k{\bfseries ɛ́'ɛ̀} sùkúlì \\
         yíì mpìnàgà nâ wɛ-H kɛ́'ɛ̀ sùkúlì\\
              7 $\emptyset$3.obligation {\COMP} 2\textsc{sg}-\textsc{prs} go.{\SBJV} $\emptyset$7.school\\
    \trans `It's an obligation that you go to school.'
\ex \label{SBJV3}
  \glll  yíì mpìndá nâ wɛ́ jíw{\bfseries ɔ́'ɔ̀} bésâ \\
           yíì mpìndá nâ wɛ-H jíwɔ́'ɔ̀ H-be-sâ\\
              7 $\emptyset$9.prohibition {\COMP} 2\textsc{sg}-\textsc{prs} steal.{\SBJV} {\OBJ}.{\LINK}-be8-thing\\
    \trans `It's forbidden that you steal things.'
\z
\z

The \textsc{subjunctive} is also used to express a goal, as in \REF{SBJVb}, where the verb {\itshape dyùù} `kill', which is marked for the subjunctive, is part of a purpose clause.

\ea\label{SBJVb}
  \glll   á lèmbó nâ bùdì báà bà múà búɛ̀lɛ̀ nâ bá {\bfseries dyúù} nyɛ̂  \\
a-H lèmbo-H nâ b-ùdì báà ba múà búɛlɛ̀ nâ ba-H dyùù nyɛ̀  \\
1-\textsc{prs} know-{\R} {\COMP} ba2-person 2.{\DEM}.{\PROX} 2 {\PROSP} fish {\COMP} 2-\textsc{prs} kill.{\SBJV} 1.{\OBJ}   \\
    \trans `He knows that these people are about to fish [look for him] in order to kill him.'
\z

The \textsc{subjunctive} can further be used in a consecutive context, as in \REF{SBJVa}, which lacks an animate entity that could have wishes or intentions. When translating these phrases, speakers consistently assign the French verb {\itshape vouloir} `want' to the inanimate entity. %The example further shows that the \textsc{stamp} marker usually preceding the \textsc{subjunctive} form can be elided.

\ea\label{SBJVa}
  \glll    ká yí nyí mɛ̂ mbɔ̀ mpángì yí kùgá nâ ny{\bfseries íì} wɛ̀ mbɔ̀\\
           ká yi-H nyî-H mɛ̂ m-bɔ̀ mpángì yi-H kùga-H nâ nyíì wɛ̀ m-bɔ̀ \\
             when 7-\textsc{prs} enter-{\R} 1\textsc{sg} \textsc{n}3-arm $\emptyset$7.bamboo 7-\textsc{prs} can-{\R} {\COMP} enter.{\SBJV} 2\textsc{sg} \textsc{n}3-arm\\
    \trans `When it goes into my arm . . . the bamboo can sting your arm.'
\z

The \textsc{subjunctive} expresses bouletic modality, as in \REF{Bouletic}, which concerns the speaker's desire in relation to what is necessary or possible. Other types of modality, e.g.\ deontic or dynamic, are encoded by semi-auxiliaries in complex predicates \sectref{sec:ComplSemi}.

\ea\label{Bouletic}
  \glll  mɛ́ làwɔ́ náà màndáwɔ̀ má zì má kùg{\bfseries áà} mɛ̂ vâ\\
         mɛ-H làwɔ-H nâ ma-ndáwɔ̀ má zì ma-H kùgáà mɛ̀ vâ \\
            1\textsc{sg}-\textsc{prs} say-{\R} {\COMP} ma6-house 6:{\ATT} $\emptyset$7.tin 6-\textsc{prs} be.enough.{\SBJV} 1\textsc{sg}.{\OBJ} here\\
    \trans `I say that there ought to be enough tin (roofed) houses here for me.'
\z


\noindent While most \textsc{subjunctive} forms occur in a subordinate complement clause involving the complementizer {\itshape nâ} (\sectref{sec:CompC}), \textsc{subjunctive} forms can also occur  in subordinate clauses without the complementizer {\itshape nâ}, as in \REF{SBJVc}.

\ea\label{SBJVc}
  \glll     yɔ́ɔ̀ mɛ́ wúmbɛ́ mándáwɔ̀ má zì má tɛ́w{\bfseries ɔ́'ɔ̀} mɛ̂ vâ ndá zì \\
            yɔ́ɔ̀ mɛ-H wúmbɛ-H H-ma-ndáwɔ̀ má zì ma-H tɛ́wɔ̀ɔ̀ mɛ̂ vâ ndá zì \\
              so 1\textsc{sg}-\textsc{prs} want-{\R} {\OBJ}.{\LINK}-ma6-house 6:{\ATT} $\emptyset$7.tin 6-\textsc{prs} put.{\SBJV} 1\textsc{sg}.{\OBJ} here {\ATT}[Bulu] $\emptyset$7.tin[Bulu]\\
    \trans `So I want tin (roofed) houses be put here for me, of tin.'
\z

There are a few examples where the \textsc{subjunctive} is not restricted to a subordinate clause, but can occur in the main clause, as in \REF{SBJVd}. This construction marks a politely phrased order or invitation.

\ea\label{SBJVd}
  \glll  bɛ̀yá {\bfseries njíì} bíyɛ̀ kfùmàlà \\
         bɛ̀ya-H njì bíyɛ̀ kfùmala\\
         2\textsc{pl}-\textsc{prs}  come.{\SBJV} 1\textsc{pl}.{\OBJ} find\\
    \trans `You (pl.) may come to meet us.'
\z

The \textsc{subjunctive} has its own negator  {\itshape dúù} (\sectref{sec:NEGduu}).









\subsection{The realis-marking H tone}
\label{sec:SynH}

The basic simple predicate structure carries further grammatical information through the presence or absence of a grammatical H tone that attaches to the right of verb stems in certain tense-mood categories when the finite verb is not in phrase-final position (see subpattern I \textsc{stamp} - V(-H) in \sectref{sec:SimpPred}). It is inherent to each tense-mood category if the H tone will attach to the finite verb or not. The presence of the H tone correlates with realis categories, while its absence indicates irrealis categories, as shown in \tabref{Tab:RIRRaxis}. The \textsc{present} tense is split between its affirmative constructions, which belong to the realis category, while their negative counterparts cluster with the irrealis mood. 


\begin{table}
\begin{tabularx}{.8\textwidth}{Xl}
\lsptoprule
H tone presence & H tone absence\\
$\rightarrow$ Realis & $\rightarrow$ Irrealis\\
 \midrule
\textsc{present} & \textsc{future} \\
\textsc{inchoative} & \textsc{imperative}  \\
\textsc{recent past}  & \textsc{subjunctive} \\
\textsc{remote past} &  \textsc{present negation}\\
\lspbottomrule
\end{tabularx}
\caption{Distribution of realis and irrealis categories}
\label{Tab:RIRRaxis}
\end{table}

\REF{M} provides examples for all realis-marking tense-mood categories, where the grammatical H tone is marked in bold. The H tone that appears on the following noun is a distinct syntactic tone rather than a phonologically conditioned surface form (\sectref{sec:HLinker}).\footnote{Grammatical verb-final H tones seem to be recurrent in Bantu languages, but have not yet found a unitary and transparent explanation. The term ``metatony'' is frequently used in the context of verb-final H tone phenomena \citep{dimmendaal95, angenot71, hyman2012, schadeberg95, hadermann2005,costa2008, makasso2012, nurse08}. The origins and functions assigned to metatonic H tones in the literature differ, however, considerably across diverse Bantu languages.}


\ea\label{M}
\ea  \label{M1}
  \glll  mɛ́ wúmb{\bfseries ɛ́} békwàndɔ̀ \\
          mɛ-H wúmbɛ-{\bfseries H} H-be-kwàndɔ̀\\
         1\textsc{sg}-\textsc{prs} want-{\R} {\OBJ}.{\LINK}-be8-plantain\\
    \trans `I want plantains.'
\ex\label{M2}
  \glll   mɛ̀ɛ́ wúmb{\bfseries ɛ́} békwàndɔ̀ \\
        mɛ̀ɛ́ wúmbɛ-{\bfseries H} H-be-kwàndɔ̀ \\
          1\textsc{sg}.{\INCH} want-{\R} {\OBJ}.{\LINK}-be8-plantain\\
    \trans `I'm beginning to want plantains.'
\ex\label{M3}
  \glll   mɛ̀ wúmb{\bfseries ɛ́} békwàndɔ̀ \\
         mɛ wúmbɛ-{\bfseries H} H-be-kwàndɔ̀ \\
          1\textsc{sg}.{\PST}1 want-{\R} {\OBJ}.{\LINK}-be8-plantain\\
    \trans `I wanted plantains (recently).'
\ex\label{M4}
  \glll   mɛ́ɛ̀ wúmb{\bfseries ɛ́} békwàndɔ̀ \\
          mɛ́ɛ̀ wúmbɛ-{\bfseries H} H-be-kwàndɔ̀ \\
          1\textsc{sg}.{\PST}2 want-{\R} {\OBJ}.{\LINK}-be8-plantain\\
    \trans `I wanted plantains (a long time ago).'
\z
\z

While the tonal change from a phrase-final L to a non-phrase final H tone is obvious in the \textsc{non-past} categories \textsc{present} and \textsc{inchoative}, such a change is less clear in the two \textsc{past} categories, recent and remote. These categories are specified for a final H tone  in verb-final positions, thereby collapsing both tense and mood marking in non-phrase final position.  In terms of glossing examples, I mark phrase-final H tones on \textsc{past} verbs as `{\PST}', as in \REF{metapst1}. In  non-phrase final position, however, H tones in \textsc{past} categories are marked as `{\R}', as in \REF{metapst2}, emphasizing the mood distinction.

\ea\label{metapst}
\ea  \label{metapst1}
  \glll  mɛ̀ gyámbɔ́  \\
          mɛ gyámbɔ-H \\
         1\textsc{sg}.{\PST}1 cook-{\PST}  \\
    \trans `I cooked.'
\ex\label{metapst2}
  \glll   mɛ̀ gyámbɔ́ békwàndɔ̀ \\
        mɛ gyámbɔ-H H-be-kwàndɔ̀ \\
          1\textsc{sg}.{\PST}1 cook-{\R} {\OBJ}.{\LINK}-be8-plantain\\
    \trans `I cooked plantains.'
\z
\z

Examples of the irrealis tense-mood categories are given in \REF{noM}. The finite verbs in these sentences do not take the grammatical H tone; they are only inflected for their tense-mood category as basic simple predicates \REF{sec:GramTM}.\footnote{The second person plural and the cohortative in the \textsc{imperative} category have the same tonal pattern on the verb as \REF{noM2}, but the tonal structure of the object noun is different due to the postverbal plural particle. As this concerns, however, the syntactic H tone rather than the realis-marking grammatical H tone, this phenomenon is discussed in \sectref{sec:HLinker}.}

\ea\label{noM}
\ea  \label{noM1}
  \glll  mɛ̀ɛ̀ gyámb{\bfseries ɔ̀} békwàndɔ̀ \\
          mɛ̀ɛ̀ gyámbɔ H-be-kwàndɔ̀ \\
         1\textsc{sg}.{\FUT} cook {\OBJ}.{\LINK}-be8-plantain\\
    \trans `I will/might cook plantains.'
\ex\label{noM2}
  \glll   gyámb{\bfseries ɔ̂} békwàndɔ̀ \\
        gyámbô H-be-kwàndɔ̀ \\
          cook.{\IMP} {\OBJ}.{\LINK}-be8-plantain\\
    \trans `Cook (sg.) plantains!'
\ex\label{noM3}
  \glll   mɛ́ wúmbɛ́ nâ wɛ́ gyámb{\bfseries ɔ́ɔ̀} békwàndɔ̀\\
         mɛ-H wúmbɛ-H nâ wɛ-H gyámbɔ́ɔ̀ H-be-kwàndɔ̀ \\
          1\textsc{sg}-\textsc{prs} want-{\R} {\COMP} 2\textsc{sg}-\textsc{prs} cook.{\SBJV} {\OBJ}.{\LINK}-be8-plantain\\
    \trans `I want you to cook plantains.'
\z
\z




In the realis categories that do take the grammatical H tone, all parts of speech that follow the verb trigger the appearance of the H tone, as \REF{MetaPOS} shows. Thus, the decisive criterion is not a restriction to certain parts of speech, but rather a prohibition of the verb being intonation phrase final.

\eabox{\label{MetaPOS}
%{\small 
\begin{tabular}{llll}
a. & mɛ́ gyámbɔ̀ &  `I cook.' & \\
b. &  mɛ́ gyámbɔ́ bé-kwàndɔ̀&  `I cook plantains.' &  $\underline{\quad}$\textsc{n} \\
c. &   mɛ́ gyámbɔ́ byɔ̂ &  `I cook it.' & $\underline{\quad}${\PRO} \\
d. &  mɛ́ gyámbɔ́ ndáà &   `I cook today.' &  $\underline{\quad}${\ADV} \\
e. &   mɛ́ gyámbɔ́ ɛ́ kìsíní dé tù &   `I cook in the kitchen.' &  $\underline{\quad}${\PREP} \\
f. & mɛ́ gyámbɔ́ nà wɔ́mbɛ̀lɛ̀ &  `I cook and sweep.' &  $\underline{\quad}${\CONJ} \\
\end{tabular}
}

\noindent As shown in \REF{MetaPOS}, the phrase-final verb {\itshape gyámbɔ} `cook' surfaces with an L tone. If it is, however, followed by a noun, pronoun, adverb, preposition, or conjunction, the verb takes a final H tone. The same is true for complex predicates, as illustrated in \REF{MetaPOS1}. Again, if the verb {\itshape wúmbɛ} `want' occurs phrase finally, it surfaces with an L tone. If it is followed by another element, however, in this case the non-finite main verb {\itshape gyámbɔ} `cook', it takes a final H tone.

\eabox{\label{MetaPOS1}
\begin{tabular}{llll}
a. & bá wúmbɛ̀ &  `They want [something].'  & \\
b. &  bá wúmb{\bfseries ɛ́} gyámbɔ̀ &  `They want to cook.' &  $\underline{\quad}$\textsc{v} \\
\end{tabular}
}

It is, however, only the finite verb that undergoes tonal change. If a second, non-finite verb is not intonation phrase-final, it keeps its default tones, as shown in \REF{MetaPOS2}. In this example, the modal verb {\itshape wúmbɛ} `want' takes the grammatical H tone that indicates the realis category. The final tone on {\itshape gyámbɔ} `cook' surfaces as L.


\ea\label{MetaPOS2}
  \glll     bá wúmb{\bfseries ɛ́} gyámb{\bfseries ɔ̀} békwàndɔ̀\\
             ba-H wúmbɛ-H gyámbɔ H-be-kwàndɔ̀ \\
              2-\textsc{prs} want-{\R} cook {\OBJ}.{\LINK}-be8-plantain\\
    \trans `They want to cook plantains.'
\z


















\subsection{Expanded simple predicates}
\label{sec:MorphSimp}

Simple predicates can be expanded, making them morphologically more complex through the addition of inflectional verbal suffixes or particles, as described in \sectref{sec:SimpPred} under subpattern II. This morphological expansion includes the negation suffix -{\itshape lɛ} (\sectref{sec:NEGPRES}), stem reduplication expressing \textsc{habitual} (\sectref{sec:HAB}), and the postverbal particle {\itshape mɔ} encoding \textsc{completive}  (\sectref{sec:COMPL}).\footnote{There are other verbal suffixes used in verbal derivation (\sectref{sec:VDeriv}) that bring about a  valency change. These are, however, not treated here as morphologically complex predicates--although they are considered as such by, for instance, \citet[51]{butt2010} on morphological causativization{\textemdash}due to their differing morphosyntactic behavior in Gyeli (\sectref{sec:SimpPred}.)}









\subsubsection{Negation with -{\itshape lɛ} in the \textsc{present}}
\label{sec:NEGPRES}

In the \textsc{present} tense-mood category, the verbal suffix -{\itshape lɛ} is used in negation. I consider this suffix to be toneless since its surface tones depend on the verb stem's tonal specification. Negation with -{\itshape lɛ} shows structural and paradigmatic asymmetry in the sense of \posscitet{miestamo2007}: the verb stem takes it own tonal pattern under negation, the \textsc{stamp} marker differs from its positive counterpart in some person categories, and the realis-marking H tone is absent, which marks \textsc{present} negation as an irrealis category. I first discuss the tonal pattern of the negated verb, then I describe the patterns of the {\STAMP} marker, and finally the relation between present negation and the mood category. 


The tonally specified first TBU of a verb stem (\sectref{sec:Tinventory}) determines the tonal pattern of a verb negated with the suffix -{\itshape lɛ}. In monosyllabic verb stems, the stem always changes to an H tone, which then also spreads to the negation suffix. \REF{leL} gives examples of verb stems with underlying L tones and \REF{leHL} gives examples of monosyllabic verb stems whose tones surface as HL in their uninflected form.


\ea\label{leL} L $\rightarrow$ H
\ea  dè `eat' > dé-lɛ́
\ex  kɛ̀ `go' > kɛ́-lɛ́ 
\z
\z

\ea\label{leHL} HL $\rightarrow$ H
\ea  nyɛ̂ `see' > nyɛ́-lɛ́
\ex pɛ̂ `choose' > pɛ́-lɛ́
\z
\z

For disyllabic verbs, the determining factor for the negated surface form is the first syllable's tonal specification. If the tonal pattern of a disyllabic verb is H {\O}, the H tone spreads to the second, underlyingly toneless mora of the verb and also to the negation suffix, as in \REF{leHLbi}.

\ea\label{leHLbi} H {\O} $\rightarrow$ H H
\ea  síndya `change' > síndyá-lɛ́
\ex símɛ  `respect' > símɛ́-lɛ́
\ex dzímbɛ `get lost'  > dzímbɛ́-lɛ́ 
\ex  ngwáwɔ `bend' > ngwáwɔ́-lɛ́
\z
\z

\noindent The same is true for trisyllabic verbs where the first mora is specified H and the two following morphemes are toneless. \REF{leHLL} shows that, again, the H tone from the first mora spreads to the right, all the way to the negation suffix.

\ea\label{leHLL} H {\O} {\O} $\rightarrow$ H H H
\ea  gyíkɛsɛ `teach' > gyíkɛ́sɛ́-lɛ́
\ex  líyɛlɛ  `show' > líyɛ́lɛ́-lɛ́
\ex lúmɛlɛ `send' >  lúmɛ́lɛ́-lɛ́
\ex  súmɛlɛ `greet' > súmɛ́lɛ́-lɛ́
\z
\z

The process changes if the first mora of a bi- or trisyllabic verb is specified with an L tone. In these cases, the tone on the first mora undergoes a featural change from L to H. This, however, does not affect the following toneless extension and negation suffix morphemes. These all surface as L, as shown in \REF{leLL} for disyllabic verbs and in \REF{leLLL} for trisyllabic verbs.

\ea\label{leLL} L {\O} $\rightarrow$ H L
\ea  gy{\bfseries à}ga `buy' > gy{\bfseries á}gà-lɛ̀
\ex  v{\bfseries ɔ̀}wa  `wake up' > v{\bfseries ɔ́}wà-lɛ̀
\ex l{\bfseries ù}nga `grow'  > l{\bfseries ú}ngà-lɛ̀
\ex ts{\bfseries ì}lɔ `write' >  ts{\bfseries í}lɔ̀-lɛ̀
\z
\z


\ea\label{leLLL} L {\O} {\O} $\rightarrow$ H L L
\ea  kf{\bfseries ù}bala `move' > kf{\bfseries ú}bàlà-lɛ̀
\ex  v{\bfseries ì}dɛga  `turn' > v{\bfseries í}dɛ̀gà-lɛ̀
\ex k{\bfseries à}mbala `defend' > k{\bfseries á}mbàlà-lɛ̀
\ex j{\bfseries ì}nɛsɛ `make sth. sink' >  j{\bfseries í}nèsɛ̀-lɛ̀
\z
\z

\REF{le} illustrates the verb tone asymmetries between a basic \textsc{present} form and its negative counterpart with an L tone verb in \REF{lea} that changes to an H on the first TBU in the stem in \REF{leb}, while the following verbal TBUs stay L.

\ea\label{le}
\ea  \label{lea} \glll  bá gìyɔ̀. \\
          ba-H gìyɔ \\
           2-\textsc{prs} cry\\
    \trans `They cry.'
\ex \label{leb} 
\glll  bá gíyɔ̀lɛ̀. \\
          ba-H gìyɔ-lɛ \\
           2-\textsc{prs} cry-{\NEG}   \\
    \trans `They do not cry.'
\ex \label{lec} \glll  bá límbɛ̀. \\
          ba-H límbɛ \\
           2-\textsc{prs} pull\\
    \trans `They pull.'
\ex  \label{led}
\glll  bá límbɛ́lɛ́. \\
          ba-H límbɛ-lɛ \\
           2-\textsc{prs} pull-{\NEG}   \\
    \trans `They do not pull.'
\z
\z

In contrast, verb stems that are lexically specified with an H tone on the first TBU, as in \REF{lec}, stay H and spread that H tone across the following TBUs, including the negation suffix, as in \REF{led}. This pattern also constitutes a structural asymmetry, since the basic simple predicate in the positive \textsc{present} surfaces as L.
 



As a default, the \textsc{stamp} marker under \textsc{present} negation has the same pattern as the non-negated form, as shown for the agreement class 2 \textsc{stamp} marker in \REF{le}. As with \textsc{future} non-negated \textsc{stamp} markers, however, there are a few exceptions in certain grammatical person categories. The \textsc{stamp} markers for first and second person singular as well as for class 1 take a special shape with a long vowel and rising LH pattern, as shown in \REF{le1} for the first person singular and the agreement class 1 \textsc{stamp} marker.

\ea\label{le1}
\ea  \label{le1a} \glll  mɛ́ gìyɔ̀ \\
          mɛ-H gìyɔ \\
           1\textsc{sg}-\textsc{prs} cry\\
    \trans `I cry.'
\ex \label{le1b} 
\glll  mɛ̀ɛ́ gíyɔ̀lɛ̀\\
          mɛ̀ɛ́ gìyɔ-lɛ \\
           1\textsc{sg}.{\NEG}.\textsc{prs} cry-{\NEG}   \\
    \trans `I do not cry.'
\ex \label{le1c} \glll  á límbɛ̀\\
          a-H límbɛ \\
           1-\textsc{prs} pull\\
    \trans `S/he pulls.'
\ex  \label{le1d}
\glll  àá límbɛ́lɛ́\\
          àá límbɛ-lɛ \\
           1.{\NEG}.\textsc{prs} pull-{\NEG}   \\
    \trans `S/he does not pull.'
\z
\z

\noindent Other examples of \textsc{present} negation  from natural texts are provided in \REF{leA} and \REF{leB}.


\ea\label{leA}
  \glll  {\bfseries má} {\bfseries dvúmɔ́lɛ́} mbvú mbì mbvû \\
        ma-H dvúmɔ́-lɛ́ mbvú mbì mbvû \\
           6-\textsc{prs} produce-{\NEG}  $\emptyset$3.year like[Kwasio] $\emptyset$3.year\\
    \trans `They [the palm trees] don't produce [fruit] every year.'
\z

\ea\label{leB}
  \glll {\bfseries mɛ̀ɛ́} {\bfseries jílɛ́} wɛ̂ bvúbvû \\
       mɛ̀ɛ́ jí-lɛ́ wɛ̂ bvúbvû \\
         1\textsc{sg}.\textsc{prs}.{\NEG} ask-{\NEG} 2\textsc{sg}.{\OBJ} much\\
    \trans `I don't ask you for much.'
\z

Negation of non-verbal existential constructions (\sectref{sec:nonverbalC}) is achieved through verbal \textsc{present} negation, using the verb {\itshape bɛ̀} `be', as in \REF{leOM}.

\ea\label{leOM}
\ea \label{leOMa}
  \glll  bùdì bá bɛ́lɛ́   \\
        b-ùdì ba-H bɛ̀-lɛ \\
         ba2-person 2-\textsc{prs} be-{\NEG}        \\
    \trans `There are no people.' / `The people are not there.'
\ex\label{leOMb}
  \glll  mùdì {\bfseries nú} bɛ́lɛ́  \\
        m-ùdì nu-H bɛ̀-lɛ \\
         \textsc{n}1-person 1-\textsc{prs} be-{\NEG}        \\
    \trans `Nobody is there.' / `The person is not there.'
\z
\z

As outlined in \sectref{sec:SCOP}, agreement class 1 has alternate \textsc{stamp} forms. Although their distribution is not exactly understood, it seems that there is a preference to use the form {\itshape nú} in the negation of existential clauses, as in \REF{leOMb}. Unlike the agreement class 1 negation \textsc{stamp} marker {\itshape àá}, however, {\itshape nú} clusters with the default \textsc{stamp} forms, carrying an H tone.




Although the \textsc{present} category is a realis mood characterized by a grammatical H tone on the verb in non-phrase final position, the realis-marking H tone is absent in \textsc{present} negation. Even if the negated verb appears in non-phrase final position, its tonal pattern does not change from the pattern outlined above for negated forms, as shown in \REF{leOBJ}.

\ea\label{leOBJ}

\ea  \label{leOBJa}
  \glll  á gyág{\bfseries á} békáládɛ̀ \\
         a-H gyàga-H H-be-káládɛ̀ \\
         1-\textsc{prs} buy-{\R} {\OBJ}.{\LINK}-be8-book\\
    \trans `He buys books.'
\ex \label{leOBJb}
  \glll  àá gyág{\bfseries àlɛ̀} békáládɛ̀\\
         àá gyàga-lɛ H-be-káládɛ̀ \\
         1.\textsc{prs}.{\NEG} buy-{\NEG} {\OBJ}.{\LINK}-be8-book\\
    \trans `He does not buy books.'
\ex\label{leOBJc}
  \glll  á {\bfseries dé} mántúà\\
          a-H dè-H H-ma-ntúà\\
           1-\textsc{prs} eat-{\R} {\OBJ}.{\LINK}-ma6-mango\\
    \trans `He eats mangoes.'
\ex \label{leOBJd} 
  \glll  àá {\bfseries délɛ́} mántúà\\
          àá dè-lɛ H-ma-ntúà\\
           1.\textsc{prs}.{\NEG} eat-{\NEG} {\OBJ}.{\LINK}-ma6-mango\\
    \trans `He does not eat mangoes.'
\z

\z


\noindent Since the negated verb in \REF{leOBJd} surfaces with an H tone,  one could assume that the H tone has merged with the realis-marking H tone. Since verbs of the pattern in \REF{leOBJb} do not take a verb-final H tone, however, I treat all negated verb forms in the \textsc{present} as having their own, fixed tonal pattern that lacks the grammatical H tone. The negated \textsc{present} thus belongs to the irrealis mood, which constitutes a paradigmatic asymmetry in comparison to the positive \textsc{present}.









\subsubsection{\textsc{Habitual} aspect by verb reduplication}
\label{sec:HAB}

Another expanded simple predicate construction involves verb stem reduplication, expressing \textsc{habitual} aspect, as in \REF{habitual}. The \textsc{habitual} relates to events that occur regularly or usually.

\ea\label{habitual}
  \glll  mɛ́ nyùl{\bfseries ɛ̀}nyùlɛ̀ \\
            mɛ-H nyùlɛ-nyulɛ \\
             1\textsc{sg}-\textsc{prs} drink-drink\\
    \trans `I often drink.'
\z

The reduplicated stem follows the original stem in the form of a suffix as opposed to constituting an independent word. Evidence for this comes from the duplicate's tonal pattern.  First, the duplicate is underlyingly toneless, while the original stem is specified for its first TBU. \REF{habitual1} shows that {\itshape pándɛ} `arrive' carries its lexical H tone on the first TBU in the stem, but this lexical H tone does not appear on the toneless duplicate, which might even lose more features of the stem, such as vowel length and nasalization, as shown in \REF{HAB2}.

\ea\label{habitual1}
  \glll  mɛ́ pándɛ̀p{\bfseries à}ndɛ̀ \\
            mɛ-H pándɛ-pandɛ  \\
             1\textsc{sg}-\textsc{prs} arrive-arrive\\
    \trans `I often arrive.'
\z

Second, if a grammatical (or syntactic) H tone attaches to the right of the verb, it spreads across all toneless TBUs, just like in verbal extension suffixes (\sectref{sec:VDeriv}), including the second and third syllables of the original stem, as shown in \REF{habituala} and \REF{HAB1}.

\ea \label{habituala}
  \glll  mɛ́ dílɛ́sɛ́dílɛ́sɛ́ bwánɔ̀ \\
            mɛ-H dílɛsɛ-dilɛsɛ-H b-wánɔ̀ \\
             1\textsc{sg}-\textsc{prs} feed-feed-{\R} ba2-child\\
    \trans `I often give food to the children.'
\z

\ea\label{HAB1}
  \glll     mɛ́ {\bfseries gyámbɔ́gyámbɔ́} bédéwɔ̀ \\
            mɛ-H gyámbɔ-gyambɔ-H H-be-déwɔ̀ \\
              1\textsc{sg}-\textsc{prs} prepare-prepare-{\R} {\OBJ}.{\LINK}-8-food\\
    \trans `I regularly prepare food.'
\z


Although the \textsc{habitual} aspect appeared to me to be very frequent in the conversations that I observed, it is barely found in the corpus. From elicitation, however, it is clear that the \textsc{habitual} is restricted to the \textsc{present} and \textsc{subjunctive} categories. An example of a \textsc{subjunctive} occurrence is given in \REF{HAB2} with {\itshape tã́ã̀-ta} `tell often'.


\ea\label{HAB2}
  \glll bàmpámbó bá {\bfseries líyɛ̀lìyɛ̀} nâ yá {\bfseries tã́ã̀tà} békàndá bé tè \\
       ba-mpámbó ba-H líyɛ-liyɛ nâ ya-H tã́ã̀-ta H-be-kàndá bé tè \\
         ba2-ancestor 2-\textsc{prs} leave-leave {\COMP} 1\textsc{pl}-\textsc{prs} tell.{\SBJV}-tell {\OBJ}.{\LINK}-be8-proverbs 8:{\ATT} there\\
    \trans `The ancestors leave [the proverbs to us], so that we tell the proverbs there.'
\z

\noindent The tonal marking of the subjunctive is on the original stem, while the duplicate is underlyingly toneless. The duplicate further loses vowel length and nasalization.









\subsubsection{\textsc{Absolute completive} aspect {\itshape mɔ̀}}
\label{sec:COMPL}

The verbal particle {\itshape mɔ̀} (\sectref{sec:VParticle}) expresses \textsc{absolute completive} aspect.\footnote{This category might be similar to what has been called a ``iamitive'' by \citet{olsson2013} for Southeast Asian languages. \citet{dahl2016} suggest that the iamitive category differs from the prefect in that it allows combination with statives, which is also the case in Gyeli. They note that iamitive forms are often grammaticalized from expressions for `already'. This is different in Gyeli, where the grammaticalization path more likely involves a verbal source (`finish').} Historically, it probably stems from a serial verb construction, which \citet[67]{nurse08} views as a Niger-Congo derivative from {\itshape -mala > -ma} `finish' and which is found in many northwestern Bantu languages, e.g., Maande (A46), Himba (B30), Yanzi (B85), and Nyanga (D43) \citep[100]{nurse08}. {\itshape mɔ̀} has an assimilated variant that merges with the preceding verb vowel, while adding length, nasality, and an HL tone pattern to it, as in \REF{mo1b}.

\ea\label{mo1}
\ea\label{mo1a}
  \glll    mɛ̀ lùngá mɔ̀ \\
           mɛ̀ lùngá mɔ̀  \\
             1\textsc{sg}  grow {\COMPL}  \\
    \trans `I have (already) grown up.'
\ex\label{mo1b}
  \glll    mɛ̀ lùngã́ã̀ \\
          mɛ̀ lùngã́ã̀ \\
             1\textsc{sg} grow.{\COMPL}    \\
    \trans `I have (already) grown up.'
\z
\z

The \textsc{absolute completive} is restricted to the recent \textsc{past}.\footnote{Unlike other aspectual categories, such as the \textsc{past} \textsc{progressive} form {\itshape nzí} or the \textsc{perfect} form {\itshape bwàà}, which allow both \textsc{past} tense-mood categories, the use of \textsc{pst2} is prohibited for the \textsc{absolute completive}.}
In the corpus, 17 occurrences of the \textsc{absolute completive} have the uncontracted form and twelve have the contracted form. In sum, the \textsc{absolute completive} is the most frequent aspect marker with 23.8\% of all aspect markers in the corpus.

The \textsc{absolute completive} mostly occurs with eventive verbs, as illustrated in \REF{mo3} through \REF{mo5}.

\ea\label{mo3}
  \glll   mínɔ̀ má bùdì mà k{\bfseries ɛ̃́ɛ̃̀} máà vé \\
          m-ínɔ̀ má b-ùdì ma kɛ̃́ɛ̃̀ máà vé \\
            ma6-name 6:{\ATT} ba2-person 6.{\PST}1 go.{\COMPL} 6.ID where\\
    \trans `The people's names have gone, where are they? [strangers come once, but do not return again]'
\z

\ea\label{mo4}
  \glll bon mpɔ̀ngɔ̀ síl{\bfseries ɛ̃́ɛ̃̀}\\
        bon mpɔ̀ngɔ̀ sílɛ̃́ɛ̃̀ \\
      OK[French] $\emptyset$7.generation finish.{\COMPL} \\
    \trans `OK, the generation has been wiped out.'
\z

\ea\label{mo5}
  \glll wɛ̀ dyúwɔ́ {\bfseries mɔ̀}\\
       wɛ dyúwɔ-H mɔ̀ \\
      2\textsc{sg}.{\PST}1 hear-{\R} {\COMPL}   \\
    \trans `Have you understood?'
\z

Although stative verbs rarely take this aspect marker, it is still possible, as \REF{mo6} shows.\footnote{Another explanation for this particular occurrence of {\itshape mɔ̀} with {\itshape lèmbɔ} `know' could be that this verb rather has an eventive character, along the lines of `coming to understand'. The restricted corpus, however, does not clarify this.}

\ea\label{mo6}
  \glll wɛ̀ lèmb{\bfseries ṍõ̀} sâ bányá màmbò nâ ká mɛ́ lúmɔ́ wɛ̂ nláà nâ \\
       wɛ lèmbṍõ̀ sâ H-ba-nyá m-àmbò nâ ká mɛ-H lúmɔ-H wɛ̂ nláà nâ \\
        2\textsc{sg}.{\PST}1 know.{\COMPL}  do {\OBJ}.{\LINK}-ba2-important ma6-thing {\COMP} if 1\textsc{sg}-\textsc{prs} send-{\R} 2\textsc{sg}.{\OBJ} $\emptyset$3.message {\COMP} \\
    \trans `You know to do the right thing so that, if I send you the message [ask you for help] that. . .'
\z

All of these examples have in common that the aspect marker conveys a meaning of completeness. They are usually translated into French as {\itshape déjà} `already' by speakers.  In \REF{mo3}, the people have completely left, in \REF{mo4}, the generation has completely been wiped out, and in \REF{mo5}, the process of understanding has to be complete in order to count as understanding. \REF{MPL} shows how the \textsc{absolute completive} compares to other aspect categories that relate to notions of completeness or perfectiveness, such as the \textsc{perfect} {\itshape bwàà} (\sectref{sec:PSTPRF}) and the semi-auxiliary {\itshape sílɛ} `finish', which has a non-complete accomplishment reading (\sectref{sec:ComplSemi}). 

\ea\label{MPL}
\ea \label{MPL1}
  \glll     mɛ̀ lá {\bfseries mɔ̀} kálàdɛ̀ yíndɛ̀ \\
          mɛ lâ-H mɔ̀ kálàdɛ̀ yí-ndɛ̀\\
           1\textsc{sg}.{\PST}1 read-{\R} {\COMPL} $\emptyset$7.book 7-{\ANA}  \\
    \trans `I have read this book [entirely, all of it].'
\ex\label{MPL2}
  \glll     mɛ̀ {\bfseries sílɛ́} lâ kálàdɛ̀ yíndɛ̀ \\
          mɛ sílɛ-H lâ kálàdɛ̀ yí-ndɛ̀ \\
           1\textsc{sg}.{\PST}1 finish-{\R} read $\emptyset$7.book 7-{\ANA}  \\
    \trans `I'm done reading this book. [but not necessarily the whole book]'
\ex\label{MPL3}
  \glll     mɛ̀ {\bfseries bwàá} lâ kálàdɛ̀ yí-ndɛ̀ \\
            mɛ bwàà-H lâ kálàdɛ̀ yí-ndɛ̀\\
          1\textsc{sg} have-{\R} read $\emptyset$7.book 7-{\ANA}  \\
    \trans `I have read this book [more general/experiential].'
\z
\z

\noindent The example compares different aspect meanings in the situation of reading a book. If {\itshape mɔ̀} is used, the interpretation is that the book has been read entirely. Therefore, I call this aspect category \textsc{absolute completive}. In comparison, the semi-auxiliary {\itshape sílɛ} `finish',   also carries a completive meaning in that the person has finished reading the book. The use of {\itshape sílɛ}, however, does not entail that the book has been read in its entirety, just that the subject has stopped reading (parts of) it. Therefore, I label this aspect  \textsc{non-complete accomplishment}. The \textsc{perfect} use in \REF{MPL3} suggests a more general and maybe experiential reading.
In this way, the \textsc{perfect} has some semantic overlap with the \textsc{absolute completive}, since  experiential meaning is also expressed by {\itshape mɔ̀}, as shown in \REF{mo7}.

\ea\label{mo7}
  \glll     wɛ̀  làdɔ́ mɔ̀ nà káliyâ \\
          wɛ  làdtɔ-H mɔ̀ nà káliyâ \\
           2\textsc{sg}.{\PST}1  meet-{\R} {\COMPL} {\COM} $\emptyset$1.sister:1\textsc{sg}.{\POSS}  \\
    \trans `Have you (already, ever) met my sister?'
\z

Finally, the \textsc{absolute completive} is used in more figurative and idiomatic ways. In \REF{mo8}, for instance, Nzambi's wife states that she is starving, using the \textsc{absolute completive} for {\itshape wɛ̀} `die', even though, obviously, she is still alive.

\ea\label{mo8}
  \glll  nyɛ̀ náà mùdì wã́ã̀ mɛ̀ w{\bfseries ɛ̃́ɛ̃̀} nà nzà \\
         nyɛ náà m-ùdì w-ã́ã̀ mɛ wɛ̃́ɛ̃̀ nà nzà\\
           1 {\COMP} \textsc{n}1-person 1-{\POSS}.1\textsc{sg} 1\textsc{sg} die.{\COMPL} {\COM} $\emptyset$9.hunger\\
    \trans `She [said]: ``My person, I'm dead hungry''.' 
\z

\noindent In the same way, speakers use the \textsc{absolute completive} in situations of announcing their departure, as in \REF{mo9}, although, literally, they have not left yet.

\ea\label{mo9}
  \glll yɔ́ɔ̀ Nzàmbí kí nâ bon mɛ̀ nìyɛ́ {\bfseries mɔ̀} \\
        yɔ́ɔ̀ Nzàmbí kì-H nâ bon mɛ nìyɛ-H mɔ̀ \\
        so $\emptyset$1.{\PN} say-{\R} {\COMP} good[French] 1\textsc{sg}.{\PST}1 return-H {\COMPL} \\
    \trans `So Nzambi says: ``Good, I am returning home''.'
\z



I consider the \textsc{absolute completive} to indicate the realis mood, since the finite verb always surfaces with a final H tone, which is characteristic of this mood category (\sectref{sec:GramTM} and \sectref{sec:SynH}). In comparison to other simple predicate constructions, the verb in the \textsc{absolute completive} never appears phrase finally, since the \textsc{absolute completive} marker {\itshape mɔ̀} behaves as a post-verbal element.  In \REF{PROGtone1A}, the grammatical H tone thus appears on the final vowel of {\itshape gyámbɔ} `cook'.

\ea\label{PROGtoneA}
\ea \label{PROGtone1A}
  \glll     mɛ̀ gyámb{\bfseries ɔ́} mɔ̀ bédéwɔ̀ \\
           mɛ gyámbɔ-H mɔ̀ H-be-déwɔ̀ \\
            1\textsc{sg} cook-{\R} {\COMPL} {\OBJ}.{\LINK}-be8-food\\
    \trans `I have cooked the food.'
\ex \label{PROGtone2A}
  \glll  mɛ̀ gyámb{\bfseries ɔ̃́}ɔ̃̀ bédéwɔ̀ \\
            mɛ gyámbɔ̃́ɔ̃̀ H-be-déwɔ̀ \\
             1\textsc{sg} cook:R:PRF {\OBJ}.{\LINK}-be8-food\\
    \trans `I have cooked the food.'
\z
\z

\noindent The more grammaticalized variant in \REF{PROGtone2A} also carries the H tone. Here, the verb and the \textsc{completive} marker {\itshape mɔ̀} have fused, resulting in a  long final vowel that is nasalized and that reflects the tonal pattern of the {\itshape mɔ̀} variant: first the grammatical H tone and then the L tone of the postverbal aspect marker, surfacing as a long HL vowel.
















\section{Complex verbal predicates}
\label{sec:CompPred}

According to \citet[50]{butt2010}, ``the term {\itshape complex predicate} refers to any construction in which two or more predicational elements each contribute to a joint predication''. 
In Gyeli, there are two types of complex predicates. I refer to the first type as complex predicates with a single \textsc{stamp} marker, which include the \textsc{stamp} marker, a finite auxiliary verb, and at least one non-finite lexical verb, as the template in \REF{CompTemp1} shows. Maximally, two non-finite verbs can occur in a complex predicate, as discussed in \sectref{sec:ComplMulti}. The adverb and pronominal object that appear in square brackets in the template are not part of the verbal predicate, but they can occur between the finite and the main verb.  I consider the second type to be a complex predicate construction with a double \textsc{stamp} marker, which has a template as in \REF{CompTemp2}.


\ea\label{CompTemp}  {\bfseries Complex predicate types}
\ea  \label{CompTemp1} {\bfseries Complex predicates with a single \textsc{stamp}}  \\
\textsc{stamp} -- Auxiliary verb -- [Adverb/pronominal object] -- Verb -- (Verb)
\ex\label{CompTemp2} {\bfseries Complex predicates with a double \textsc{stamp}} \\
\textsc{stamp}\textsubscript{i} -- (Auxiliary) -- {\itshape bɛ̀} `be' -- \textsc{stamp}\textsubscript{i} -- Auxiliary/Verb\textsubscript{finite} -- (Verb)
\z
\z

Single \textsc{stamp} predicates can further be subdivided into those that take only one non-finite verb and those that take two. \REF{Compmin} gives an example of a minimal single \textsc{stamp} predicate with the verbal predicate in brackets.

\ea\label{Compmin}
  \glll  mɛ̀gà [mɛ́ lígɛ́ dè] mwánɔ̀ wɔ́ɔ̀ \\
        mɛ-gà {\db}mɛ-H lígɛ-H dè m-wánɔ̀ w-ɔ́ɔ̀ \\
          1-{\CONTR} {\db}1\textsc{sg}-\textsc{prs} stay-{\R} eat ma1-child 1-{\POSS}.2\textsc{sg}  \\
    \trans `As of me, I stay and eat your child.'
\z

\noindent An example of a single \textsc{stamp} predicate with the maximal number of non-finite verbs is provided in \REF{Compmax}.

\ea\label{Compmax}
  \glll  áh gyí [wɛ́ lɔ́ njì gyɛ́sɔ̀] \\
        áh gyí {\db}wɛ-H lɔ́ njì gyɛ́sɔ \\
           {\EXCL} what {\db}2\textsc{sg}-\textsc{prs} {\RETRO} come look.for\\
    \trans `Ah, what have you just come to look for?'
\z

Elements that are external to the single \textsc{stamp} predicate, but which occur between the finite and the non-finite verb, such as adverbs, sentential modifiers, and object pronouns, always directly follow  the finite verb form, as in \REF{AUXadv1}.

\ea\label{AUXadv1}
  \glll  [wɛ́ yànɛ́ {\bfseries ná} gyàgà] ndísì \\
     {\db}wɛ-H yànɛ-H ná gyàga ndísì \\
        {\db}2\textsc{sg}-\textsc{prs} must-H again buy $\emptyset$3.rice\\
    \trans `You must again buy rice.'
\z

If a sentential modifier is used in a three-verb single \textsc{stamp} predicate, as in the combination of modal and aspectual auxiliaries in \REF{AUXadv2}, the modifier will still appear after the first, inflected auxiliary. It has not been observed to appear after the second auxiliary.

\ea\label{AUXadv2}
  \glll  bí bɔ́gà [yá wúmbɛ́ {\bfseries ndáà} pã̂ nyɛ̂] sâ bá gyíbɔ́ ngyùlɛ̀ wá kùrã̂ \\
         bí bɔ́-gà {\db}ya-H wúmbɛ-H ndáà pã̂ nyɛ̂ sâ ba-H gyíbɔ-H ngyùlɛ̀ wá kùrã̂ \\
          1\textsc{pl}.{\SBJ} 2-other {\db}1\textsc{pl}-\textsc{prs} want-{\R} also {\PRIOR} see $\emptyset$7.thing 2-\textsc{prs} call-{\R} $\emptyset$3.light 3:{\ATT} $\emptyset$7.electricity\\
    \trans `We others, we also want to first see the thing they call the light of electricity.'
\z

The same is true for fronted object pronouns (\sectref{sec:OBJfront}): the object pronoun will always appear after the first auxiliary, as in \REF{temponj1}, which contains a two-verb construction, and in \REF{tempobj2}, which contains a three-verb construction.

\ea\label{temponj1}
  \glll bùdì [bà sílɛ̃́ɛ̃̀ {\bfseries mɛ̂} wɛ̀] ndáwɔ̀ tù vâ\\
        b-ùdì {\db}ba sílɛ̃́ɛ̃̀ mɛ̂ wɛ̀ ndáwɔ̀ tù vâ \\
       ba2-person {\db}2.{\PST}1 finish.{\COMPL} 1\textsc{sg}.{\OBJ} die $\emptyset$9.house inside here\\
    \trans `The people have all died here inside the house.'
\z

\ea\label{tempobj2}
  \glll [báà sílɛ̀ {\bfseries bî} kúmbà lwɔ̃̂] mándáwɔ̀\\
        {\db}báà sílɛ bî kúmba lwɔ̃̂ H-ma-ndáwɔ̀ \\
        {\db}2.{\FUT} finish 1\textsc{pl}.{\OBJ} arrange build {\OBJ}.{\LINK}-ma6-house\\
    \trans `They will arrange for us to build houses.'
\z


These examples show that complex predicates in Gyeli are auxiliary-headed. \citet[9]{anderson2011b} explains that, in auxiliary-headed languages, the auxiliary verb serves as the head, while the lexical verb is its dependent,  appearing in its non-finite form. This is illustrated in, for instance, \REF{Compmin}, where the auxiliary {\itshape lígɛ́} `stay' carries the realis-marking H tone, while {\itshape dè} `eat' appears in its non-finite form.
The auxiliary occupies ``the position in the verb phrase that the lexical verb would occupy if it appeared alone in an inflected form'' \citep[10]{anderson2011b}. In Gyeli, this position is directly following the \textsc{stamp} marker and preceding the lexical verb. This pattern matches \posscitet{dryer2007b} observation that the auxiliary (generally) precedes the main verb in VO languages.

Double \textsc{stamp} predicates involve two \textsc{stamp} markers that share a referent. Each of the \textsc{stamp} markers is followed by a finite verb form. The first verb form always includes a form of the auxiliary {\itshape bɛ̀} `be', either finite as in \REF{DobPred1} or non-finite as in \REF{DobPred2}, while the second involves another simple predicate or complex single \textsc{stamp} predicate. The square brackets indicate the double \textsc{stamp} construction.

\ea\label{DobPred}
\ea \label{DobPred1}
  \glll   [mɛ́ɛ̀ {\bfseries bɛ́} mɛ́ {\bfseries gyámbɔ̀gyàmbɔ̀}] \\
          {\db}mɛ́ɛ̀ bɛ̀-H mɛ-H gyámbɔ-gyambɔ \\
              {\db}1\textsc{sg}.{\PST}2 be 1\textsc{sg}-\textsc{prs} cook-cook\\
    \trans `I used to cook (a long time ago).'
\ex\label{DobPred2}
 \glll   [mɛ̀ {\bfseries nzí} {\bfseries bɛ́} mɛ̀ {\bfseries nzí} {\bfseries gyámbɔ̀gyàmbɔ̀}] à nzí gyímbɔ̀  \\
 {\db}mɛ nzí bɛ̀-H mɛ nzí gyámbɔ-gyambɔ a nzí gyímbɔ  \\
              {\db}1\textsc{sg} {\PROG}.{\PST} be-{\R} 1\textsc{sg} {\PROG}.{\PST}1 prepare-prepare 1 {\PROG}.{\PST} dance\\
    \trans `While I was preparing [food], he was dancing.'
\z
\z

Double \textsc{stamp} predicates can be thought of as a combination of a single predicate (or complex predicate with single \textsc{stamp} marker)  with another single predicate (or complex predicate with single \textsc{stamp} marker). The two finite verbs usually differ in their tense-mood encoding, thereby shifting the viewpoint in temporal reference as well as enabling combinations of tense, mood, aspect, and negation that are excluded in single \textsc{stamp} constructions.

In the remainder of this chapter, I first discuss single \textsc{stamp} predicates. As outlined in \sectref{sec:AUX}, auxiliaries in Gyeli differ in their degree of grammaticalization. True auxiliaries are highly grammaticalized and have no synchronic lexical meaning. They are discussed in detail in \sectref{sec:ComplAUX}. In contrast, semi-auxiliaries do have a  lexical meaning, as well as a different distribution from that of true auxiliaries, as described in \sectref{sec:ComplSemi}. \sectref{sec:ComplMulti} presents different levels of complexity in single \textsc{stamp} predicates, namely those that are morphologically and syntactically complex and those that involve two non-finite verbs. \sectref{sec:Compbe} describes double \textsc{stamp} predicates.


















\subsection{Single \textsc{stamp} predicates with true auxiliaries}
\label{sec:ComplAUX}

Complex predicates with a single \textsc{stamp} construction that use true auxiliaries (\sectref{sec:AUX}) involve grammaticalized auxiliaries that, unlike semi-auxiliaries, are restricted to certain tense-mood categories.  This predicate type differs internally with respect to the degree of grammaticalization: highly grammaticalized true auxiliaries have synchronically no lexical meaning, whereas less grammaticalized true auxiliaries also maintain a lexical meaning. This distinction is indicated by an English gloss for the ones with a lexical meaning and a lack thereof for the ones without lexical meaning. \tabref{Tab:TAUX} lists all true auxiliaries that are used in complex predicates with single \textsc{stamp} constructions. Functionally, these auxiliaries encompass those that encode aspect and those that encode negation.

\begin{table}

\begin{tabularx}{\textwidth}{l llll}
 \lsptoprule 
 &   \textsc{stamp} example &  True  auxiliary  & Restrictions           & Function\\
 \midrule
\multirow{6}*{Aspect} & yà &  {\bfseries nzíí} & special pattern 1 & \textsc{prog.pres} \\
&  yà  & {\bfseries nzɛ́ɛ́} & special pattern 1 & \textsc{prog.sub}  \\
 & yà, yáà &  {\bfseries nzí} & \textsc{pst1}, \textsc{pst2} & \textsc{prog} \\
&  yá &  {\bfseries lɔ́}   & \textsc{prs} & \textsc{retro} \\
&  mɛ̀, yá &  {\bfseries múà} `be'  &  special pattern 2 & \textsc{prosp}   \\
 & yà, yáà &  {\bfseries bwàá} `have' & \textsc{pst1}, \textsc{pst2} & \textsc{prf}\\ 
  \midrule
\multirow{4}*{Negation} &  yà/yáà & {\bfseries sàlɛ́/pálɛ́} & \textsc{pst1}, \textsc{pst2}     & \textsc{neg} \\
&  yáà/mɛ̀ɛ̀ & {\bfseries kálɛ̀} &\textsc{fut} & \textsc{neg} \\
& yá/$\emptyset$/ & {\bfseries tí} & \textsc{imp}, \textsc{inf},  & \textsc{neg}  \\
& yà & {\bfseries tí} & special pattern 1  & \textsc{neg} \\
&  yá   & {\bfseries dúù} `must not'  & \textsc{prs}, \textsc{sbjv}     & \textsc{neg} \\
 \lspbottomrule 
\end{tabularx}
\caption{\textsc{stamp} markers for different aspect markers}
\label{Tab:TAUX}
\end{table}

\tabref{Tab:TAUX} further indicates the auxiliaries' restriction to certain tense-mood categories or special constructions (e.g.\ subordinate clauses, infinitives). While most true auxiliaries occur within a tense-mood category that is identical to those discussed under simple predicates (\sectref{sec:GramTM}), there are a four auxiliaries that take a special pattern.

Special pattern 1 includes the \textsc{present progressive} with {\itshape nzíí}, the \textsc{subordinate progressive} with {\itshape nzɛ́ɛ́}, and the \textsc{present} tense use with {\itshape tí}. This pattern is characterized by a \textsc{stamp} marker that surfaces with an L tone and a verb with an H tone. On the surface, this looks identical to the \textsc{recent past} pattern of simple predicates. Since the auxiliary, however, can never occur phrase finally, as it always requires a non-finite verb, it is not clear what underlying tone pattern the auxiliary verb has and thus whether it is indeed identical to the \textsc{recent past}. Given that this (on-the-surface) identical tone pattern occurs in different predicate construction types and has different functions, while the underlying tone pattern of the verb is not discernible, I consider the special pattern 1 as distinct from the \textsc{recent past}.  All categories that take the special pattern 1 occur in \textsc{present} tense ({\itshape nzíí} and {\itshape tí}) or tenseless ({\itshape nzɛ́ɛ́}) contexts. I suggest that, with these highly grammaticalized auxiliaries, the \textsc{stamp} marker is deprived of the H tone that surfaces on the \textsc{stamp} markers in simple predicate \textsc{present}. Tense information in these complex constructions is thus encoded lexically in the auxiliary, as in \REF{nzii1}.

\ea\label{nzii1}
  \glll  mɛ̀ nzí{\bfseries í} gyámbɔ̀ bédéwɔ̀ \\
         mɛ nzíí gyámbɔ H-be-déwɔ̀ \\
            1\textsc{sg} {\PROG}.\textsc{prs}.{\R} cook {\OBJ}.{\LINK}-be8-food\\
    \trans `I am cooking food.'
\z

To mark the difference between the \textsc{recent past} L tone of the \textsc{stamp} marker, as in \REF{nzii1a}, and the absence of the H tone for special pattern 1 in complex predicates, I only gloss the \textsc{stamp} marker in the latter for person. In contrast, the \textsc{recent past} \textsc{stamp} marker is additionally glossed for the tense information it encodes.

\ea\label{nzii1a}
  \glll  mɛ̀ gyámbɔ́ bédéwɔ̀ \\
         mɛ gyámbɔ-H H-be-déwɔ̀ \\
            1\textsc{sg}.{\PST}1 cook-{\R} {\OBJ}.{\LINK}-be8-food\\
    \trans `I cooked food.'
\z



Special pattern 2 is only found with the \textsc{prospective} aspect {\itshape múà}. Here, the tonal pattern of the \textsc{stamp} marker is comparable to that of the \textsc{future}, where some person categories have an exceptional tonal pattern. The first and second person singular as well as the agreement class 1 \textsc{stamp} marker are different from the other agreement classes. The actual shape, however, differs between \textsc{prospective} and \textsc{future} \textsc{stamp} markers. The \textsc{prospective} \textsc{stamp} markers have all short vowels with an L tone for the exceptional (1\textsc{sg}, 2\textsc{sg}, and 1) person categories and H tones for the others. In contrast, the \textsc{future} \textsc{stamp} markers have a long vowel, which has an L tone in the exceptional cases (1\textsc{sg}, 2\textsc{sg}, and 1) and an HL tone in the others.


Each aspect and negation category also cross-cuts with a mood category. Although there is no way to prove that a realis-marking H tone attaches to the auxiliary verb, since the auxiliary never occurs phrase finally and therefore its underlying tone pattern cannot be known, I classify the auxiliaries with a final H tone as realis mood and those with a final L tone as irrealis mood. This analysis is based on an assumed parallel behavior between semi-auxiliaries (\sectref{sec:ComplSemi}) and highly grammaticalized true auxiliaries, which are thought of as mirroring the mood category of their simple predicate counterparts.  As \tabref{Tab:AM} shows, this is true for {\itshape dúù} `must not', which belongs to the realis category when it occurs in the \textsc{present}, but to the irrealis category when it occurs in a \textsc{subjunctive} construction.


\begin{table}
\begin{tabular}{llll}
 \lsptoprule
Mood               & True auxiliary & TM  restriction         & Function\\\midrule
\textsc{realis}  &  {\bfseries nzíí} & special pattern 1 & \textsc{prog.pres} \\
                       & {\bfseries nzí} & \textsc{pst1, pst2} & \textsc{prog.pst} \\
                       & {\bfseries nzɛ́ɛ́} & special pattern 1 & \textsc{prog.sub} \\
                       &   {\bfseries lɔ́} & \textsc{prs} & \textsc{retro} \\
                       & {\bfseries bwàá} `have' & \textsc{pst} & \textsc{prf}\\ 
                       & {\bfseries sàlɛ́/pálɛ́} `have' & \textsc{pst1, pst2} & \textsc{neg.pst}\\ 
                       & {\bfseries dúù} `must not' & \textsc{prs} & \textsc{neg}\\ 
                       & {\bfseries tí}  & special pattern 1 & \textsc{imp, inf, pres}\\ 
  \midrule
\textsc{irrealis} & {\bfseries múà} `be almost'  &  special pattern 2 & \textsc{prosp}   \\
                       & {\bfseries kálɛ̀} & \textsc{fut} & \textsc{neg.fut}\\ 
                       & {\bfseries dúù} `have' & \textsc{sbjv} & \textsc{neg}\\ 
 \lspbottomrule
\end{tabular}
\caption{Mood categories of aspect markers}
\label{Tab:AM}
\end{table} 

While most auxiliaries belong to the realis mood, there are a few irrealis auxiliaries characterized by their final L tone: \textsc{prospective} {\itshape múà}, \textsc{future negative} {\itshape kálɛ̀}, and \textsc{subjunctive} {\itshape dúù}. Almost all auxiliaries match their simple predicate counterparts in their mood category.\footnote{I consider {\itshape múà} `be almost' is considered to belong to the \textsc{future} category based on its formal and semantic proximity.} The only exception is {\itshape tí}, which is the negation form of the \textsc{imperative}, infinitive constructions, and certain cases of the \textsc{present}. While {\itshape tí} clusters with the realis mood, both the \textsc{imperative} and the \textsc{present} negation with -{\itshape lɛ} (\sectref{sec:NEGPRES}) belong to the irrealis category. 
In the remainder of this section, I present each true auxiliary and the grammatical category it encodes.





%\subsubsection{Frequency of aspect markers in corpus}
%As mentioned, aspect markers are significantly less frequent in the corpus than constructions that use tense-mood marking only. They present a total of 122 occurrences, as shown in \tabref{Tab:AspectFreq}, while tense-mood marking only is represented 369 times in the corpus.


%\begin{table}
%\centering
%\scalebox{0.9}{
%\begin{tabular}{p{3cm}lll|ll}
% \midrule
%Status & Aspect  &  TM         & Function & \multicolumn{2}{l}{Frequency}  \\
%         & marker   &   restriction &              &                    &  \\
% \midrule
%Grammaticalized  & {\bfseries nzíí} &  \textsc{prs} & \textsc{prog} & 17 & 13.9\% \\
% verbs & {\bfseries nzí} &  \textsc{pst} & \textsc{prog} & 10 & 8.2\% \\
% & {\bfseries nzɛ́ɛ́}  & subordinate & \textsc{prog} & 0 & 0\% \\
 %& {\bfseries pã́}  &  none & \textsc{prior} & 11 & 9\% \\
%  \midrule
%Transparent & {\bfseries lɔ́} `come'  &  \textsc{prs} & \textsc{retro} & 17 & 13.9\%\\
%verbs & {\bfseries bwàá} `have' &  \textsc{pst} & \textsc{prf} & 3 & 2.5\% \\
%& {\bfseries múà} `be' &  \textsc{fut} & \textsc{prosp} & 14 or 10???& 11.5\%   17 but not all aspectual\\
%  & {\bfseries sílɛ̀} `finish' &  none & \textsc{nca} & 20 & 16.4\% \\
% \midrule
%Reduplication & {\bfseries STEM-STEM} &  \textsc{prs} & \textsc{hab} & 1 & 0.8\% \\
% \midrule
%Postverbal & {\bfseries mɔ̀/-Ṽ}  &  \textsc{pst1} & \textsc{compl} & 29 & 23.8\% \\
% \midrule 
%Total       &                        &                         &                     & 122 &    \\
%\end{tabular}}
%\caption{Frequency of aspect markers in corpus}
%\label{Tab:AspectFreq}
%\end{table}


%\begin{exe} 
%\ex\label{combmua}
%  \gll mɛ̀ nzíí múà dè  \\
%         1\textsc{sg} {\PROG} {\PROSP} eat\\
%    \trans `I'm being about to eat.'
%\z















\subsubsection{\textsc{Progressive} aspect {\itshape nzíí, nzí, and nzɛ́ɛ́}}
\label{sec:PROG}

The \textsc{progressive} aspect category has three suppletive forms for different tense related categories: {\itshape nzíí} for \textsc{present}, {\itshape nzí} for the general \textsc{past}, i.e.\ both recent and remote, and {\itshape nzɛ́ɛ́} as a tenseless dependent form.\footnote{The \textsc{stamp} markers of {\itshape nzíí} and {\itshape nzɛ́ɛ́ } take a special tone pattern that does not match the tense-mood categories of simple predicates, as outlined in \sectref{sec:ComplAUX}.}
The \textsc{progressive} forms for the \textsc{present} and both \textsc{past} tenses are used in main clauses, as shown in \REF{proga} with a temporal adverb in each example, and in most subordinate clauses, as in \REF{nzee4} and \REF{nzee5}.

\ea\label{proga}
\ea \label{proga1}
  \glll     mɛ̀ {\bfseries nzíí} gyámbɔ̀ tɛ́ɛ̀ \\
           mɛ nzíí gyámbɔ tɛ́ɛ̀ \\
              1\textsc{sg} {\PROG}.\textsc{prs}.{\R} cook now\\
    \trans `I'm cooking now.'
\ex\label{proga2}
  \glll   mɛ̀ {\bfseries nzí} gyámbɔ̀ nàkùgúù \\
          mɛ nzí gyámbɔ nàkùgúù \\
              1\textsc{sg}.{\PST}1 {\PROG}.{\PST}.{\R} cook yesterday\\
    \trans `I was cooking yesterday.'
\ex\label{proga3}
  \glll   mɛ́ɛ̀ {\bfseries nzí} gyámbɔ̀ mbvũ̂ lã̀ \\
          mɛ́ɛ̀ nzí gyámbɔ mbvũ̂ lã̀ \\
              1\textsc{sg}.{\PST}2 {\PROG}.{\PST}.{\R} cook $\emptyset$3.year pass\\
    \trans `I was cooking last year.'
\z
\z

In contrast, the tenseless \textsc{progressive} auxiliary {\itshape nzɛ́ɛ́} is a dependent form that occurs in three environments: (i) in the second constituent of a complex predicate construction with a double \textsc{stamp} marker (\sectref{sec:Compbe}), (ii) in a subordinate clause where {\itshape nzɛ́ɛ́} is the only marker of subordination (\sectref{sec:SUBnzee}), and (iii) in a complement clauses with {\itshape nâ} (\sectref{sec:CompC}).  \REF{nzee2} provides an instance of a complex predicate with a double \textsc{stamp} marker, where the referent of the \textsc{stamp} marker is identical for both constituents. As {\itshape nzɛ́ɛ́} is generally not specified for tense, tense-mood information is encoded in the first constituent, which involves {\itshape bɛ̀} `be'. Although the first constituent anchors the event in the \textsc{future}, which belongs to the irrealis mood, {\itshape nzɛ́ɛ́} always occurs with a realis-marking H tone, irrespective of the tense-mood category of the first constituent in a complex predicate (or the matrix clause).

\ea\label{nzee2}
  \glll  [mɛ̀ɛ̀ bɛ̀ [mɛ̀ {\bfseries nzɛ́ɛ́} kɛ̀]] \\
         {\db}mɛ̀ɛ̀ bɛ̀ {\db}mɛ nzɛ́ɛ́ kɛ̀ \\
            {\db}1\textsc{sg}.{\FUT} be {\db}1\textsc{sg} {\PROG}.{\SUB}.{\R} go\\
    \trans `I will be going.'
\z

In contrast to \REF{nzee2}, the structure in \REF{nzee} is not a complex predicate, but a case of ``linkless'' subordination. Although, on the surface, both examples look similar, \REF{nzee} is not an instance of joint predication, since the two \textsc{stamp} markers refer to different entities: the second person singular in the first  constituent and the first person singular in the second constituent. Another difference from \REF{nzee2} is that the finite verb in the first constituent is not the auxiliary {\itshape bɛ̀} `be'. Nevertheless, the tenseless \textsc{progressive} auxiliary {\itshape nzɛ́ɛ́} is used in this context, since both predicates share the same tense specification, anchoring the second constituent temporally at the time of the first.

\ea\label{nzee}
  \glll  ká wɛ́ pámó màwùlà lɔ̀mbì [wɛ́ kfùmàlà [mɛ̀ {\bfseries nzɛ́ɛ́} gyámbɔ̀]] \\
         ká wɛ-H pámo-H ma-wùlà lɔ̀mbì {\db}wɛ-H kfùmàlà {\db}mɛ nzɛ́ɛ́ gyámbɔ \\
           if 2\textsc{sg}-\textsc{prs} arrive-{\R} ma6-hour eight {\db}2\textsc{sg}-\textsc{prs} find {\db}1\textsc{sg}.{\SBJ} {\PROG}.{\SUB}.{\R} cook\\
    \trans `If you arrive at eight o'clock, you will find me cooking.'
\z

{\itshape nzɛ́ɛ́} also occurs in complement clauses with {\itshape nâ}, as in \REF{nzee3}, in places where the \textsc{subjunctive} would be used instead if the construction were a simple predicate.

\ea\label{nzee3}
  \glll  mɛ́ sìsɔ́ nâ wɛ̀ {\bfseries nzɛ́ɛ́} gyìmbɔ̀ \\
         mɛ-H sìsɔ-H nâ wɛ nzɛ́ɛ́ gyìmbɔ \\
         1\textsc{sg}-\textsc{prs} be.happy-{\R} {\COMP} 2\textsc{sg} {\PROG}.{\SUB}.{\R} dance\\
    \trans `I'm happy that you are dancing.'
\z

{\itshape nzɛ́ɛ́} does not, however, occur in every type of subordinate clause. In relative clauses (\sectref{sec:Relativeclauses}), for instance, a tensed form of the \textsc{progressive} auxiliary is used instead, as in \REF{nzee4}.

\ea\label{nzee4}
  \glll bá dyúwɔ́ lɛ́kɛ́lɛ̀ [{\bfseries lé} wɛ̀ {\bfseries nzíí} làwɔ̀]\textsubscript{\REL} \\
        ba-H dyúwɔ-H H-lɛ-kɛ́lɛ̀ {\db}lé wɛ nzíí làwɔ \\
        2-\textsc{prs} understand {\OBJ}.{\LINK}-le5-language {\db}5:{\ATT} 2\textsc{sg} {\PROG}.\textsc{prs}.{\R} speak\\
    \trans `They understand the language that you are speaking.'
\z


\noindent The same is true for conditional clauses (\sectref{sec:Cond}), as in \REF{nzee5}. The reason for this is most likely that these types of dependent clauses do not necessarily anchor the time of the subordinate clause at the same time of the matrix clause, even though these times can be identical, as in \REF{nzee5}. Therefore, the tenseless auxiliary {\itshape nzɛ́ɛ́} is prohibited.

\ea\label{nzee5}
  \glll [ká kɛ̃́ɛ̃́sɔ́ yì {\bfseries nzíí} wɛ̂ dyɔ̀dɛ̀]\textsubscript{\COND} wɛ́ yánɛ́ kílɔ̀wɔ̀ \\
        {\db}ká kɛ̃́ɛ̃́sɔ́ yi nzíí wɛ̂ dyɔ̀dɛ wɛ-H yánɛ kílɔwɔ. \\
         {\db}if $\emptyset$7.peer 7 {\PROG}.\textsc{prs} 2\textsc{sg}.{\OBJ} deceive 2\textsc{sg}-\textsc{prs} must be.vigilant\\
    \trans `If somebody is deceiving you, you must be vigilant.'
\z


The \textsc{progressive} emphasizes that an event is ongoing, as shown in \REF{PRG2}. In contrast, the unmarked tense-mood form in \REF{PRG1} does not give any information about the internal constituency of the event. 

\ea\label{PRG}
\ea \label{PRG1}
  \glll     mɛ́ dè \\
           mɛ-H dè \\
              1\textsc{sg}-\textsc{prs} eat\\
    \trans `I eat.'
\ex\label{PRG2}
  \glll   mɛ̀ {\bfseries nzíí} dè  \\
          mɛ nzíí dè  \\
              1\textsc{sg} {\PROG}.\textsc{prs}.{\R} eat\\
    \trans `I'm eating.'
\z
\z


The \textsc{progressive} in Gyeli  is commonly found in questions, as in \REF{Progquest}. While the unmarked, bare tense-mood form is also grammatically correct in questions, the \textsc{progressive} form is definitely preferred and much more frequent.\footnote{For more information on questions, see \sectref{sec:Questions}.}

\ea\label{Progquest}
  \glll nzá {\bfseries nzíí} mɛ̂ nyɛ̂ \\
    nzá nzíí mɛ̂ nyɛ̂ \\
         who {\PROG}.\textsc{prs} 1\textsc{sg}.{\OBJ} see\\
    \trans `Who is seeing me?'
\z

Gyeli \textsc{progressive} aspect does not seem to be restricted to any particular verb classes. Whereas English, for instance, disprefers \textsc{progressives} with verbs expressing states, in Gyeli all kinds of verbs can occur with the \textsc{progressive}. This is illustrated in \REF{Progstative} for a stative verb and in \REF{Progmodal} for a (desiderative) modal verb.

\ea\label{Progstative}
  \glll kó mbúmbù nyɛ̀ {\bfseries nzí} {\bfseries lèmbò} dyùù bɔ̂ fàmíì bá bùdì ná \\
       kó mbúmbù nyɛ nzí lèmbo dyùù b-ɔ̂ fàmíì bá b-ùdì ná \\
       {\EXCL} $\emptyset$1.namesake 1.{\PST}1 {\PROG} know kill 2-{\OBJ} $\emptyset$1.family 2:{\ATT} ba2-person how\\
    \trans `Oh namesake, how did you know how to kill them, the family of people?'
\z

\ea\label{Progmodal}
  \glll mɛ̀ {\bfseries nzí} {\bfseries wúmbɛ̀} nâ bwánɔ̀ bã̂ bá bwámóò ɛ́ mpù mìntángánɛ́ békúdɛ́ bé mpâ\\
        mɛ nzí wúmbɛ nâ b-wánɔ̀ b-ã̂ ba-H bwámóò ɛ́ mpù mi-ntángánɛ́ H-be-kúdɛ́ bé mpâ \\
        1\textsc{sg}.{\PST}1 {\PROG} want {\COMP} ba2-child 2-{\POSS}.1\textsc{sg} 2-\textsc{prs} become.{\SBJV} {\LOC} like.this mi4-white.person {\OBJ}.{\LINK}-be8-skin 8:{\ATT} good\\
    \trans `I wanted my children to get good skin like white people.'
\z

In addition to describing a situation as ongoing and unbounded, the \textsc{progressive} is also used for backgrounding information, as shown in \REF{progback}, which presents three chronological utterances by a speaker talking about his mother. The phrase in \REF{progback1} includes the main information, namely that the speaker's mother is in another village (and not in Ngolo). He then explains as backgrounding information in \REF{progback2} that she went there for his brother's funeral. In \REF{progback3}, this is supplemented with further background information, namely that the brother had died there.

\ea\label{progback}
\ea \label{progback1}
  \glll  nyã́ã̀ wã̂ núú Ntàbɛ̀tɛ́ndá pɛ̀\\
         nyã́ã̀ w-ã̂ núú Ntàbɛ̀tɛ́ndá pɛ̀ \\
          $\emptyset$1.mother 1-{\POSS}.1\textsc{sg} 1.{\DEM}.{\DIST} $\emptyset$7.{\PN} there\\
    \trans `My mother is over there in Ntabetenda [name of village].'
\ex\label{progback2} 
  \glll à {\bfseries nzí} kɛ̀ lètsíndɔ́ lé ntùmbà wã̂\\
        a nzí kɛ̀ le-tsíndɔ́ lé n-tùmbà w-ã̂ \\
         1 {\PROG}.{\PST}1 go le5-funeral.ceremony 5:{\ATT} \textsc{n}1-older.brother 1-{\POSS}.1\textsc{sg}   \\
    \trans `She was going to my older brother's funeral ceremony.'
\ex\label{progback3} 
  \glll nɔ́gá à {\bfseries nzí} wɛ̀ wû \\
        nɔ́-gá a nzí wɛ̀ wû \\
         1-{\CONTR} 1 {\PROG}.{\PST}1 die there\\
    \trans `As for him, he died over there.'
\z
\z

\noindent The phrase in \REF{progback3} is a particularly good illustration of the fact that, in these instances, the \textsc{progressive} form is most likely not concerned with an ongoing event, since the verb {\itshape wɛ̀} `die' is typically punctual rather than ongoing.



\subsubsection{\textsc{Retrospective} aspect {\itshape lɔ́}}
\label{sec:RETROaspect}


The \textsc{retrospective} auxiliary is the counterpart to the \textsc{prospective} (\sectref{sec:PROSP}) on the time line, looking back at the endpoint of an event that has just taken place. It is likely a loan construction from French {\itshape venir de faire quelque chose} `just having done something [lit. come from doing something]', while the lexeme {\itshape lɔ́} is a loanword from Basaa (A42), with the meaning `come' in Basaa. Although speakers are aware of the Basaa meaning, {\itshape lɔ́} does not have any lexical meaning in Gyeli nor does it occur outside of the \textsc{retrospective} context. I therefore gloss {\itshape lɔ́} only with its grammatical category instead of a lexical meaning. The \textsc{retrospective} auxiliary has only been observed to occur with eventive verbs and animate subjects in the corpus. It is restricted to the \textsc{present} (unlike French, where it can also be used in other tenses). Accordingly, \textsc{stamp} markers carry the \textsc{present} H tone, as shown in \REF{lo1}, while the verb {\itshape lɔ́} always occurs with a realis-marking H tone.\footnote{Since {\itshape lɔ́} never occurs phrase finally in Gyeli, there is no proof of any underlying tone. I therefore gloss {\itshape lɔ́} with an H tone also in the underlying form, which inherently carries the realis-marking grammatical H tone.} Unlike in the \textsc{prospective}, all \textsc{stamp} markers carry the same tone in this aspect category, as \REF{lo1a} and \REF{lo1b} show.


\ea\label{lo1}
\ea\label{lo1a}
  \glll    {\bfseries á} lɔ́ dè \\
           a-H lɔ́ dè \\
            1-\textsc{prs} {\RETRO}.{\R} eat\\
    \trans `He has just eaten [{\itshape Il vient de manger}].'
\ex\label{lo1b}
  \glll   {\bfseries bá} lɔ́ dè \\
         ba-H lɔ́ dè \\
             2-\textsc{prs} {\RETRO}.{\R} eat\\
    \trans `They have just eaten.'
\z
\z



The distance between speech time and the time of the event is typically short. In \REF{lo2}, for instance, the speech time follows the event of `coming to look for' immediately, while the event has ongoing affects during speech time. The addressee of the question is still present and is still looking for something.

\ea\label{lo2}
  \glll  áh gyí wɛ́ {\bfseries lɔ́} njì gyɛ́sɔ̀ \\
          áh gyí wɛ-H lɔ́ njì gyɛ́sɔ \\
           {\EXCL} what 2\textsc{sg}-\textsc{prs} {\RETRO}.{\R} come look.for\\
    \trans `Ah, what have you just come to look for?'
\z

Likewise, in \REF{lo3}, the event that is retrospectively looked at precedes the utterance time by about a few seconds. 

\ea\label{lo3}
  \glll     yá {\bfseries lɔ́} fwálà nà mɛ́ {\bfseries lɔ́} láwɔ̀ \\
            ya-H lɔ́ fwála nà mɛ-H lɔ́ láwɔ \\
              1\textsc{pl}-\textsc{prs} {\RETRO}.{\R} end {\COM} 1\textsc{sg}-\textsc{prs} {\RETRO}.{\R} speak\\
    \trans `We have just finished and I have just spoken.'
\z

There are, however, also instances in the corpus where more time has elapsed between the situation and the utterance. In \REF{lo4}, Nzambi's wife comes home after having lost her child and now explains the situation to her husband, namely that the husband's friend has taken the child in return for food. She reports that the friend had said that they don't work hard enough to earn their food. Between the situation where the friend said this (the retrospect situation) and the time of utterance, the wife has left the friend's home, walked all the way back to her own home, had cried, and had gotten picked up by her husband. Thus, in this case, situation and speech time are not at all proximate.

\ea\label{lo4}
  \glll yɔ́ɔ̀ á {\bfseries lɔ́} kì náà ɛ́ mpù wɛ̀ɛ́ gyángyálɛ́ bédéwɔ̀  \\
        yɔ́ɔ̀ a-H lɔ́ kì náà ɛ́ mpù wɛ̀ɛ́ gyángya-lɛ́ H-be-déwɔ̀  \\
        so 1-\textsc{prs} {\RETRO} say {\COMP} {\LOC} like.this 2\textsc{sg}.\textsc{prs}.{\NEG} work-{\NEG} {\OBJ}.{\LINK}-be8-food\\
    \trans `So he just said: ``Like this, you don't work for your food''.'
\z


The \textsc{retrospective} aspect is often viewed as \textsc{perfect} in the literature, and the example in \REF{lo4} could be taken as such. As \citet[64]{comrie76} states, the perfect is retrospective in that it establishes ``a relation between a state at one time and a situation at an earlier time''. As shown in this section, the Gyeli \textsc{retrospective} is different from Comrie's retrospectivity of the perfect. The Gyeli \textsc{perfect} has a distinct form, as I show in \sectref{sec:PSTPRF}.











\subsubsection{\textsc{Prospective} aspect {\itshape múà}}
\label{sec:PROSP}

The \textsc{prospective} marker {\itshape múà} `be almost' is the only aspect category that belongs to the irrealis mood, in Gyeli which is characterized by the absence of a realis-marking grammatical H tone on the auxiliary verb, as shown in \REF{mua}. It is  similar to the \textsc{future} irrealis category also in that the \textsc{stamp} markers of the first and second person singular as well as the class 1 \textsc{stamp} marker show a different tonal pattern from the other agreement classes, as contrasted between \REF{mua1} and \ref{mua2}.\footnote{See \sectref{sec:ComplAUX} for more information on tonal patterns of the \textsc{stamp} marker in complex predicates with true auxiliaries.}

\ea\label{mua}
\ea\label{mua1}
  \glll    {\bfseries à} {\bfseries múà} dè \\
             a múà dè  \\
             1  be.almost eat\\
    \trans `S/he is about to eat.'
\ex\label{mua2}
  \glll  {\bfseries bá} {\bfseries múà} dè \\
         ba-H múà dè \\
             2-\textsc{prs} be.almost eat\\
    \trans `They are about to eat.'
\z
\z

Since the \textsc{prospective} marker {\itshape múà} has a lexical meaning, `be almost', I gloss {\itshape múà} with its meaning rather than the grammatical category that it encodes. This is consistent with cases where {\itshape múà} `be almost' occurs in a simple predicate without another finite verb, as in \REF{Rmua1}.

\ea\label{Rmua1}
  \glll mɛ̀ {\bfseries múà} tísɔ̀nì  \\
       mɛ múà tísɔ̀nì  \\
         1\textsc{sg} be.almost $\emptyset$7.town\\
    \trans `I'm almost in town.'
\z


\noindent Due to its inflectional restrictions (\sectref{sec:ComplAUX}), however, I view {\itshape múà} as marking a grammatical category instead of being a non-grammaticalized semi-auxiliary (\sectref{sec:ComplSemi}).

\citet[64]{comrie76} describes the \textsc{prospective}  as an aspect ``where a state is related to some subsequent situation, for instance where someone is in the state of being about to do something''.   Speakers usually translate the use of this aspect marker in \REF{mua1} into Cameroonian French as {\itshape Je veux/vais déjà manger} `I want/will already eat'. In a detailed description of the situation in \REF{mua1}, speakers explain that a person would already be sitting  at a table with a plate of food,  being in the state of just being about to start eating.

The French modals used in translation also reflect the future orientation of the Gyeli \textsc{prospective}, similarly to what \citet{matthewson2012} describes for Gitksan (Tsimshianic; British Columbia, Canada) modals. This future orientation explains the affiliation to the irrealis mood. Even though in terms of alternative realities, it is highly probable that the person in \REF{mua1} will indeed engage in the described event, this is probably not the case for \REF{muaa}.

\ea\label{muaa}
  \glll  mɛ̀ {\bfseries múà} wɛ̀ nà nzà \\
        mɛ múà wɛ̀ nà nzà \\
          1\textsc{sg} be.almost die {\COM} $\emptyset$9.hunger\\
    \trans `I'm about to die from hunger.'
\z

\noindent This example shows that the prospected event is not inevitable and at the point of utterance, it is not certain that it will really happen. The same is true for \REF{Rmua2}, where the beating is probable, but not certain.

\ea\label{Rmua2}
  \glll  nyɛ̀ náà à {\bfseries múà} wɛ̂ bíyɔ̀ dẽ́ \\
        nyɛ nâ a múà wɛ̂ bíyɔ dẽ́\\
           1 {\COMP} 1 be.almost 2\textsc{sg}.{\OBJ} hit today\\
    \trans `He [says] that he is about to beat you today.'
\z


The \textsc{prospective} does not seem to be restricted to any particular verb classes: it can occur with both eventive and stative verbs. Further, its subjects can be both animate and inanimate. The latter is exemplified in \REF{muab}, where the speaker is talking about the port that is about to also affect the village of Ngolo.

\ea\label{muab}
  \glll à {\bfseries múà} njì lã̀ báà bù mpàgó \\
       a múà njì lã̀ báà bù mpàgó \\
        1 be.almost come pass 2.{\FUT} break $\emptyset$3.road\\
    \trans `It [the port] is about to come and pass [by here], they will build the road.'
\z










\subsubsection{\textsc{Perfect} aspect {\itshape bwàà} `have'}
\label{sec:PSTPRF}

The \textsc{perfect} in Gyeli is expressed by the auxiliary verb {\itshape bwàà} `have'. This aspect category is restricted to the past tense-mood categories and can occur in both \textsc{recent} and \textsc{remote past}, as shown in \REF{bwaa1}.

\ea\label{bwaa1}
\ea\label{bwaa1a}
  \glll    {\bfseries mɛ̀} {\bfseries bwàá} dè \\
            mɛ bwàà-H dè \\
             1\textsc{sg}.{\PST}1  have-{\R} eat\\
    \trans `I have eaten (recently).'
\ex\label{bwaa1b}
  \glll    {\bfseries mɛ́ɛ̀} {\bfseries bwàá} dè \\
            mɛ́ɛ̀ bwàà-H dè \\
             1\textsc{sg}.{\PST}2 have-{\R} eat\\
    \trans `I have eaten (long ago).'
\z
\z

Just like the \textsc{prospective} verb {\itshape múà}, {\itshape bwàà} can occur in simple predicates without another non-finite verb, namely when expressing identity relations, as in \REF{Pbwaa1}.


\ea\label{Pbwaa1}
  \glll  yɔ́ɔ̀ bàNzàmbí bá tè bà {\bfseries bwàá} sɔ́ \\
         yɔ́ɔ̀ ba-Nzàmbí bá tè ba bwàà-H sɔ́\\
            so ba2-{\PN} 2:{\ATT} there 2.{\PST}1 have-{\R} $\emptyset$1.friend\\
    \trans `So, the Nzambis there had been friends.'
\z


The \textsc{perfect} auxiliary verb {\itshape bwàà} is rather rare, both in the corpus and in the data gathered based on \posscitet{dahl85} TMA Questionnaire. It is thus challenging to delimit a core meaning for this category. At the same time, the \textsc{perfect} seems to be similar to other aspects, such as the \textsc{retrospective} and \textsc{absolute completive},  in the sense that the situation has been completed by speech time. In comparison to the \textsc{retrospective}, however, the emphasis of the \textsc{perfect} is on a relatively long period of time between the situation and speech time. The Gyeli \textsc{perfect} is usually translated into Cameroonian French with a perfect construction and the adverb {\itshape depuis} `since',  which gives the meaning of `a long time ago'. Thus, the phrase in \REF{bwaa2} is consistently translated as {\itshape Il est depuis allé rester comme ça} `He has since gone and stayed like that'.\footnote{Despite this translation and a possible implication of anteriority, I do not label {\itshape bwàà} as past perfect, since this would require an anteriority  relation to another thematically connected event in the past \citep{lee2017}. This other event in the past, however, is not given either in \REF{bwaa2} or in \REF{bwaa3a}.}

\ea\label{bwaa2}
  \glll à {\bfseries bwàá} yɛ́ɛ́ kɛ́ jì mpù \\
       a bwàà-H yɛ́ɛ́ kɛ̀-H jì mpù \\
        1 have-{\R} then go-{\R} stay like.this\\
    \trans `He [the other Nzambi] has gone and stood like this.'
\z


Also data from the ``EUROTYP Perfect Questionnaire'' \citep{dahl2000} support the claim that {\itshape bwàà} is used when the situation is temporally distant from speech time. \REF{bwaa3} shows two possible responses to the command `Don't speak so loud, you will wake up the baby', in which, in both cases, the person replies that the baby is already awake. For \REF{bwaa3a}, in which {\itshape bwàà} is used, speakers explain that the baby has already woken up a while ago. In contrast, the use of the \textsc{absolute completive} in \REF{bwaa3b} hints at the fact that he has only woken up recently.

\ea\label{bwaa3}
\ea\label{bwaa3a}
  \glll    à {\bfseries bwàá} vòwà \\
            a bwàà-H vòwa\\
             1.{\PST}1  have-{\R} wake\\
    \trans `He has woken up already (a while ago).'
\ex\label{bwaa3b}
  \glll    à vòwá {\bfseries mɔ̀} \\
            a  vòwa-H mɔ̀\\
             1.{\PST}1 wake-{\R} {\COMPL}    \\
    \trans `He has woken up already (recently).'
\z
\z

Given that the \textsc{perfect} can occur in both \textsc{past 1} and \textsc{past 2} tense-mood categories, i.e.\ temporal distance between situation and speech time can be manipulated, a relatively long temporal distance cannot be the only information that the \textsc{perfect} encodes. Also, there are examples such as \REF{bwaa4}, where speech time and the situation are more proximate.

\ea\label{bwaa4}
  \glll  yɔ́ɔ̀ Nzàmbí kí náà mɛ̀ {\bfseries bwàá} wɛ̂ tsíyɛ̀ lèkɛ́lɛ̀ dẽ́ nâ mɛ́ lígɛ́ dè mwánɔ̀ wɔ́ɔ̀ \\
       yɔ́ɔ̀ Nzàmbí kì-H náà mɛ bwàà-H wɛ̂ tsíyɛ le-kɛ́lɛ̀ dẽ́ nâ mɛ-H lígɛ-H dè m-wánɔ̀ w-ɔ́ɔ̀ \\
         so $\emptyset$1.{\PN} say-{\R} {\COMP} 1\textsc{sg}.{\PST}1 have-{\R} 2\textsc{sg}.{\OBJ} cut le5-speech today {\COMP} 1\textsc{sg}-\textsc{prs} stay-{\R} eat \textsc{n}1-child 1-{\POSS}.2\textsc{sg}\\
    \trans `So Nzambi says, ``I have cut your word today [I'm not listening to you]; I stay and eat your child''.'
\z

\noindent In fact, it seems that the narrator could have instead chosen to use the \textsc{retrospective} form here, or the \textsc{absolute completive} (\sectref{sec:COMPL}). The reason for this preference of {\itshape bwàà} over other aspect forms in this context is not clear. 









\subsubsection{Negation with {\itshape sàlɛ́/pálɛ́} in the \textsc{past}}
\label{sec:NEGPST}

As outlined in \sectref{sec:TAMIntro}, negation in Gyeli involves different negation markers and strategies across different tense-mood categories. For both the \textsc{recent past} and the \textsc{remote past} categories,  the negation auxiliary verbs {\itshape sàlɛ́} and {\itshape pálɛ́} are used. These forms seem to be freely interchangeable. Speakers state that they can both be used in the same context, and, due to a low frequency of both forms in the corpus, no difference in usage can be seen.  In \REF{sale}, for instance, {\itshape sàlɛ́} occurs with the \textsc{remote} \textsc{past} is used.


\ea\label{sale}
  \glll ɛ́kɛ̀ Nzàmbí wà nú áà {\bfseries sàlɛ́} bɛ̀ nà bã̂ líná-á pámò \\
      ɛ́kɛ̀ Nzàmbí wà nú áà sàlɛ́ bɛ̀ nà bã̂ líná a-H pámo\\
        {\EXCL} $\emptyset$1.{\PN} 1:{\ATT} 1.{\DEM}.{\DIST} 1.{\PST}2 {\NEG}.{\PST} be {\COM} $\emptyset$7.word when 1-\textsc{prs} arrive\\
    \trans `Oh! That Nzambi had no words as soon as he arrived.'
\z

\noindent In \REF{pale1} and \REF{pale2}, the negation verb occurs with a \textsc{recent past} \textsc{stamp} marker, which surfaces with an L tone. The \textsc{stamp} markers for both \textsc{past} categories exhibit the same pattern under negation as in non-negated forms (\sectref{sec:GramTM}).

\ea\label{pale1}
\ea \label{pale1a}
  \glll  yà {\bfseries pálɛ́} bɛ̀ nà bùdã̂\\
      ya pálɛ́ bɛ̀ nà b-ùdã̂ \\
        1\textsc{pl}.{\PST}1 {\NEG}.{\PST}.{\R} be {\COM} ba2-woman\\
    \trans `We did not have any women.'
\ex\label{pale1b}
  \glll  yà bɛ́ nà bùdã̂ \\
      ya  bɛ̀-H nà b-ùdã̂ \\
        1\textsc{pl}.{\PST}1 be-{\R} {\COM} ba2-woman\\
    \trans `We did not have any women.'
\z
\z

In \REF{pale2a}, the sentential modifier {\itshape lìí}  `not yet' (\sectref{sec:SentMod}) is used, which can only occur in negated clauses. In the positive counterpart in \REF{pale2b}, this sentential modifier cannot occur. Instead, the positive is expressed by the \textsc{absolute completive} aspect particle {\itshape mɔ̀} (\sectref{sec:COMPL}).

\ea\label{pale2}
\ea \label{pale2a}
  \glll  à {\bfseries pálɛ́} lìí bâ \\
      a pálɛ́ lìí bâ \\
          1.{\PST}1 {\NEG}.{\PST}.{\R} not.yet marry\\
    \trans `He is not yet married.'
\ex\label{pale2b}
  \glll  à bá mɔ̀ \\
      a bâ-H mɔ̀ \\
          1.{\PST}1 marry-{\R} {\COMPL}  \\
    \trans `He is already married.'
\z
\z


Both {\itshape sàlɛ́} and {\itshape pálɛ́} end in-{\itshape lɛ}, the negation suffix used also in  \textsc{present} negation. Since the meaning of {\itshape sà}- and {\itshape pá}- is unknown synchronically, however, I do not gloss -{\itshape lɛ} separately as a negation suffix, but treat the whole verb as a negation auxiliary.

Also, it seems that these negation auxiliaries are more grammaticalized than \textsc{present} negation suffix -{\itshape lɛ} in terms of their tonal behavior. Unlike the \textsc{present} negation suffix, which involves special tonal patterns    (\sectref{sec:NEGPRES}), the \textsc{past} negation auxiliaries both surface with a final realis-marking H tone, as seen in \REF{pale1} through \REF{pale2}.

Negation with {\itshape sàlɛ́/pálɛ́} is asymmetric with regards to its positive counterpart in several respects. First, there is a constructional asymmetry in terms of the predicate structure. The positive clause in \REF{pale3a} is a simple predicate construction in which the lexical verb is tonally inflected for the realis mood. In contrast, the negated counterpart with the auxiliary {\itshape sàlɛ́} in \REF{pale3b} is a complex predicate in which finiteness marking occurs on the auxiliary and not on the lexical verb.

\ea\label{pale3}
\ea \label{pale3a}
  \glll     mɛ̀ gyám{\bfseries bɔ́} bélɔ̀lɔ̀  \\
          mɛ gyámbɔ-H H-be-lɔ̀lɔ \\
              1\textsc{sg}.{\PST} cook-{\R} {\OBJ}.{\LINK}-be8-duck\\
    \trans `I cooked ducks.'
\ex\label{pale3b}
  \glll     mɛ̀ sà{\bfseries lɛ́}  gyám{\bfseries bɔ̀} bélɔ̀lɔ̀  \\
            mɛ sàlɛ́  gyámbɔ H-be-lɔ̀lɔ \\
              1\textsc{sg}.{\PST} {\NEG}.{\PST} cook {\OBJ}.{\LINK}-be8-duck\\
    \trans `I did not cook ducks.'
\z
\z

Second, there is a paradigmatic asymmetry: all aspect categories, such as the \textsc{progressive} in \REF{pale4a}, are lost under negation, as shown in \REF{pale4b}.

\ea\label{pale4}
\ea \label{pale4a}
  \glll     yà nzí  dè mántúà  \\
          ya nzí dè H-ma-ntúà \\
              1\textsc{pl}.{\PST} {\PROG}.{\PST} eat {\OBJ}.{\LINK}-ma6-mango\\
    \trans `We were eating mangoes.'
\ex\label{pale4b}
  \glll     yà sàlɛ́/pálɛ́ dè mántúà \\
            ya.{\PST} sàlɛ́/pálɛ́ dè H-ma-ntúà \\
              1\textsc{pl}.{\PST} {\NEG}.{\PST} eat {\OBJ}.{\LINK}-ma6-mango\\
    \trans `We did not eat mangoes.'
\z
\z

It is impossible to combine negation and aspect markers in a complex predicate with a single \textsc{stamp} marker. It is also impossible to combine two true auxiliaries, as in \REF{pale5a}, nor can the \textsc{progressive past} auxiliary {\itshape nzí} take the \textsc{present} negation suffix -{\itshape lɛ}, as in \REF{pale5b}.

\ea\label{pale5}
\ea[*]{\label{pale5a}
  \glll     yà sàlɛ́/pálɛ́ nzí/ì dè mántúà \\
            ya.{\PST} sàlɛ́/pálɛ́ nzí/ì dè H-ma-ntúà \\
            1\textsc{pl}.{\PST} {\NEG}.{\PST} {\PROG}.{\PST} eat {\OBJ}.{\LINK}-ma6-mango\\
    \trans `We were not eating mangoes.'}
\ex[*]{\label{pale5b}
  \glll  yà nzílɛ́ dè mántúà  \\
        ya.{\PST} nzí-lɛ dè H-ma-ntúà \\
        1\textsc{pl}.{\PST} {\PROG}.{\PST}-{\NEG} eat {\OBJ}.{\LINK}-ma6-mango\\
    \trans `We were not eating mangoes.'}
\z
\z

\noindent Aspect and negation can only be combined through complex predicates with a double \textsc{stamp} construction (\sectref{sec:Compbe}).









\subsubsection{Negation with {\itshape kálɛ̀} in the \textsc{future}}
\label{sec:NEGFUT}

Negation in the \textsc{future} is achieved through the auxiliary {\itshape kálɛ̀}. The \textsc{stamp} marker patterns are identical in the positive and negative \textsc{future}. For the first and second person singular and agreement class 1, the \textsc{stamp} marker has a long vowel with an L tone pattern, as in \REF{kale1}, while all other agreements classes have a long vowel with an HL pattern, as exemplified in \REF{kale2}.\footnote{Square brackets indicate the verbal predicate.}

\ea\label{kale1}
\ea \label{kale1a}
  \glll  [{\bfseries mɛ̀ɛ̀} kálɛ̀ ná bɛ̀ nà] jí ɛ́ vâ \\
        {\db}mɛ̀ɛ̀ kálɛ̀ ná bɛ̀ nà jí ɛ́ vâ \\
           {\db}1\textsc{sg}.{\FUT} {\NEG}.{\FUT} still be {\COM} $\emptyset$7.place {\LOC} here\\
    \trans `I won't have a place here anymore.'
\ex\label{kale1b}
  \glll  [{\bfseries mɛ̀ɛ̀} bɛ̀ ná nà] jí ɛ́ vâ \\
        {\db}mɛ̀ɛ̀ bɛ̀ ná nà jí ɛ́ vâ \\
           {\db}1\textsc{sg}.{\FUT} be still {\COM} $\emptyset$7.place {\LOC} here\\
    \trans `I will still have a place here.'
\z
\z

\textsc{Future} negation with {\itshape kálɛ̀} is asymmetric in the same ways that are described for negation with \textsc{past} {\itshape sàlɛ́/pálɛ́}. There is a constructional asymmetry between simple predicates in positive and complex predicates in negative \textsc{future}. In contrast to the \textsc{past} tenses, however, the \textsc{future} belongs to the irrealis mood, which lacks the realis-marking H tone on the finite verb. Despite the absence of the grammatical tone, it is clear from the position of the adverb {\itshape ná} `still' that {\itshape kálɛ̀} in \REF{kale1a} is the finite verb, while {\itshape bɛ̀ nà} in \REF{kale1b} is finite. The adverb always occurs after the finite verb (\sectref{sec:CompPred}).

\ea\label{kale2}
\ea \label{kale2a}
  \glll  ká wɛ́ kíyá lékɔ́'ɔ̀ ɛ́ kwámɔ́ kwámɔ́ [{\bfseries nyíì} kálɛ̀ búlɛ̀]\\
        ká wɛ-H kíya-H H-le-kɔ́'ɔ̀ ɛ́ kwámɔ́ kwámɔ́ {\db}nyíì kálɛ̀ búlɛ \\
           if 2\textsc{sg}-\textsc{prs} put-{\R} {\OBJ}.{\LINK}-le5-stone {\LOC} $\emptyset$9.bag $\emptyset$9.bag {\db}9.{\FUT} {\NEG}.{\FUT} break\\
    \trans `If you put the stone in the bag, the bag will not break.'
\ex\label{kale2b}
  \glll  ká wɛ́ kíyá lékɔ́'ɔ̀ ɛ́ kwámɔ́ kwámɔ́ [{\bfseries nyíì} búlɛ̀] \\
        ká wɛ-H kíya-H H-le-kɔ́'ɔ̀ ɛ́ kwámɔ́ kwámɔ́ {\db}nyíì búlɛ \\
           if 2\textsc{sg}-\textsc{prs} put-{\R} {\OBJ}.{\LINK}-le5-stone {\LOC} $\emptyset$9.bag $\emptyset$9.bag {\db}9.{\FUT} break\\
    \trans `If you put the stone in the bag, the bag will break.'
\z
\z

The paradigmatic asymmetry regarding the loss of aspect distinctions under negation as discussed for \textsc{past} negation in \sectref{sec:NEGPST} also applies with {\itshape kálɛ̀}.

%{\itshape kálɛ̀} has also been observed to negate cleft sentences, as in \REF{kale3}.

%\begin{exe} 
%\ex\label{kale3}
%  \glll {\bfseries kálɛ̀} mɛ̀ báà kì nâ bá dúù bɛ̀ bédéwɔ̀. \\
 %        kálɛ̀ mɛ̀ báà kì nâ ba-H dúù bɛ̀ H-be-déwɔ̀ \\
%       {\NEG} 1\textsc{sg}  2.{\FUT} say {\COMP} 2-\textsc{prs} must.not.{\SBJV} grow {\OBJ}.{\LINK}-be8-food\\
 %   \trans `It's not me, they [who] will say that they must not grow food.'
%\z







\subsubsection{Negation with {\itshape tí}}
\label{sec:NEGti}


%noun phrases:

%'without' {\itshape tí (bɛ̀ nà) N}

%tɔ̀sâ 'nothing/no'

%kàlɛ́ 'not + N'

There are three subtypes of the negation auxiliary {\itshape tí} with respect to the shape of the \textsc{stamp} marker:
(i) the H tone \textsc{stamp} marker {\itshape yá} precedes {\itshape tí}  for the first person plural imperative (cohortative), (ii) the \textsc{stamp} marker is absent when {\itshape tí} is used for negation with second person imperatives as well as for negation in infinitival adverbial subordinate clauses (\sectref{sec:InfSub}), and (iii) the \textsc{stamp} marker takes special pattern 1, as described in \sectref{sec:ComplAUX} for other auxiliaries as well, when {\itshape tí} is used as a negator of a \textsc{present} main clause. Since {\itshape tí} occurs in various tense-mood forms and construction types, unlike other negation auxiliaries, I gloss {\itshape tí} as {\NEG}.\footnote{Although the \textsc{present} suffix -{\itshape lɛ} is similarly glossed -{\NEG}, the difference between -{\itshape lɛ} and {\itshape tí} is obvious in glossing through their different morpheme status. -{\itshape lɛ} is glossed as a suffix, whereas {\itshape tí} is glossed as a free morpheme.}




When {\itshape tí} is used with the first person plural imperative, the \textsc{stamp} marker {\itshape yá} precedes the negation auxiliary {\itshape tí} with the H tone of the \textsc{present} category, as in \REF{ti1a}, which has the identical \textsc{stamp} marker tone pattern as in the affirmative \textsc{imperative} in \REF{ti1b}. In contrast to other tense-mood categories, the \textsc{imperative} requires a verbal plural marker {\itshape nga} (\sectref{sec:VParticle}) that occurs immediately after the finite verb form.
 
\ea\label{ti1}
\ea \label{ti1a}
  \glll  yá tí ngá dè \\
        ya-H tí nga dè \\
           1\textsc{pl}-\textsc{prs} {\NEG}.{\R} {\PL}  eat\\
    \trans `Let's not eat!'
\ex\label{ti1b}
  \glll  yá dê ngà \\
        ya-H dê nga\\
        1\textsc{pl}-\textsc{prs} eat.{\IMP} {\PL}     \\
    \trans `Let's eat!'
\z
\z

In that respect, {\itshape tí} cohortative negation is constructionally asymmetric to its positive counterpart: in the complex predicate in \REF{ti1a}, the auxiliary is the finite verb, whereas in the positive simple predicate counterpart, the lexical verb {\itshape dê} `eat' is the finite verb with an \textsc{imperative} tonal pattern on the verb.

Another asymmetry concerns the tonal pattern of the verbal plural marker {\itshape nga}, which surfaces as H under negation in \REF{ti1a}, but as L in the affirmative in \REF{ti1b}, a difference which can be explained by the presence or absence of high tone spreading from the preceding verb. The H tones on {\itshape nga} in \REF{ti2}  have different origins in the negative and the affirmative, as explained in \sectref{sec:HLinker}.

\ea\label{ti2}
\ea \label{ti2a}
  \glll  yá tí ngá gyàgà mántúà \\
        ya-H tí nga gyàga H-ma-ntúà \\
           1\textsc{pl}-\textsc{prs} {\NEG}.{\R} {\PL} buy {\OBJ}.{\LINK}-ma6-mango\\
    \trans `Let's not buy mangoes!'
\ex\label{ti2b}
  \glll  yá gyàgâ ngá màntúà \\
        yá gyàgâ nga-H mántúà \\
           1\textsc{pl}-\textsc{prs} buy.{\IMP} {\PL}-{\OBJ}.{\LINK} ma6-mango\\
    \trans `Let's buy mangoes!'
\z
\z








Negative imperatives addressed to second persons are expressed by the negation auxiliary {\itshape tí}, but lack the \textsc{stamp} marker.  An example for the second person singular with its affirmative counterpart is given in \REF{ti3}.

\ea\label{ti3}
\ea \label{ti3a}
  \glll   tí dè mántúà \\
          tí dè H-ma-ntúà  \\
         {\NEG}.{\R} eat {\OBJ}.{\LINK}-ma6-mango\\
    \trans `Don't (sg.) eat mangoes!'
\ex\label{ti3b}
  \glll   dê mántúà \\
          dê H-ma-ntúà  \\
         eat.{\IMP} {\OBJ}.{\LINK}-ma6-mango\\
    \trans `Eat (sg.) mangoes!'
\z
\z

Other lexical examples of the second person singular negation that follow the structure of \REF{ti3a} are given in \REF{NEGIMPSG}, without an object, and in \REF{NEGIMPSGOBJ}, with a following object.

\ea\label{NEGIMPSG}
\ea  tí dè `Don't (sg.) eat!'
\ex tí gyàgà `Don't (sg.) buy!'
\ex tí nyúlɛ̀ `Don't (sg.) drink!' 
\ex tí vìdɛ̀gà `Don't (sg.) turn!' 
\z
\z

\ea\label{NEGIMPSGOBJ}
\ea  tí dè mántúà!  `Don't (sg.) eat mangoes'
\ex tí gyàgà mántúà!  `Don't (sg.) buy mangoes!'
\ex tí nyúlɛ̀ májíwɔ́!  `Don't (sg.) drink water!'
\ex tí vìdɛ̀gà wámíyɛ̀! ̀ `Don't (sg.) turn fast!' 
\z
\z

An example for the second person plural with its affirmative counterpart is given in \REF{ti4}.

\ea\label{ti4}
\ea \label{ti4a}
  \glll   tí ngá dè mántúà \\
          tí nga dè H-ma-ntúà  \\
         {\NEG}.{\R} {\PL} eat {\OBJ}.{\LINK}-ma6-mango\\
    \trans `Don't (pl.) eat mangoes!'
\ex\label{ti4b}
  \glll   dê ngá màntúà \\
          dê nga-H ma-ntúà  \\
         eat.{\IMP} {\PL}-{\OBJ}.{\LINK} ma6-mango\\
    \trans `Eat (pl.) mangoes!'
\z
\z

Other lexical examples of the second person plural negation that follow the structure of \REF{ti4a} are given in \REF{NEGIMPPL}, without an object, and in \REF{NEGIMPPLOBJ}, with a following object.


\ea\label{NEGIMPPL}
\ea  tí ngá dè! `Don't (pl.) eat!'
\ex tí ngá gyàgà! `Don't (pl.) buy!'
\ex tí ngá nyúlɛ̀! `Don't (pl.) drink!' 
\ex tí ngá vìdègà! `Don't (pl.) turn!' 
\z
\z


\ea\label{NEGIMPPLOBJ}
\ea  tí ngá dè mántúà!  `Don't (pl.) eat mangoes'
\ex tí ngá gyàgà mántúà!  `Don't (pl.) buy mangoes!'
\ex tí ngá nyúlɛ̀ májíwɔ́!  `Don't (pl.) drink water!'
\ex tí ngá vìdɛ̀gà wámíyɛ̀!  `Don't (pl.) turn fast!' 
\z
\z









A common use of the negation auxiliary {\itshape tí} concerns the negation of infinitives. It is characteristic of these constructions that the negated lexical verb appears in its non-finite form, i.e.\ without tense-mood or realis H tone marking. Furthermore, the auxiliary {\itshape tí} is not preceded by a \textsc{stamp} marker in these constructions, as \REF{NEGinf1} and \REF{NEGinf2} show.

\ea\label{NEGinf1}
  \glll  gbĩ́-gbĩ̀-gbĩ́-gbĩ̀-gbĩ́  à múà nà bábɛ̀ {\bfseries tí} wúmbɛ̀ wɛ̀\\
            gbĩ́-gbĩ̀-gbĩ́-gbĩ̀-gbĩ́  a múà nà bábɛ̀ tí wúmbɛ wɛ̀\\
         {\IDEO}:roaming 1 {\PROSP} {\COM} $\emptyset$7.illness {\NEG} want-{\R} die\\
    \trans `[depiction of disease roaming in his body] He was about to be sick, not wanting to die.'
\z

\ea\label{NEGinf2}
  \glll    nà kɛ́ jìí dé tù nà ndzǐ pámò dẽ̂ {\bfseries tí} nyɛ̂ nyɛ̂ \\
          nà kɛ̀-H jìí dé tù nà ndzǐ pámò dẽ  tí nyɛ̂ nyɛ̂ \\
         {\COM} go-{\R} $\emptyset$7.forest {\LOC} inside {\COM} $\emptyset$9.path arrive today {\NEG} see 1.{\OBJ}\\
    \trans `And [he] goes in the forest on the path till today, without seeing him [without being seen].'
\z

\noindent The auxiliary verb {\itshape tí} and the infinitive together function as an infinitival subordinate clause (\sectref{sec:InfSub}), where the subject is supplied from the main clause. 

This negative infinitival construction with {\itshape bɛ̀ nà} `be with'  is likely the source of the prepositional use of {\itshape tí} (\sectref{sec:PREP}). As \REF{NEGinf1x} shows, {\itshape bɛ̀ nà} `be with' can also be elided, only leaving {\itshape tí} as the preposition `without'.

\ea\label{NEGinf1x}
  \glll  mɛ́ nyúlɛ́ kɔ̀fí {\bfseries tí} ({\bfseries bɛ̀} {\bfseries nà}) ngùɔ́ \\
            mɛ-H nyúlɛ-H kɔ̀fí tí bɛ̀ nà ngùɔ́ \\
         1\textsc{sg}-\textsc{prs} drink-{\R} $\emptyset$7.coffee {\NEG} be {\COM} $\emptyset$7.sugar\\
    \trans `I drink coffee without (having) sugar.'
\z






{\itshape tí} can also be used for negation in a \textsc{present} main clause, as shown in \REF{tilea}. This contrasts with the general \textsc{present} negation with the suffix -{\itshape lɛ} in \REF{tileb} (\sectref{sec:NEGPRES}).
The choice between standard -{\itshape lɛ} negation and {\itshape tí} negation in \textsc{present} tense main clauses relates to information structure principles and an immediate-after-verb focus position (\sectref{sec:IS}).

\ea\label{tile}
\ea \label{tilea}
  \glll  mɛ̀ {\bfseries tí} dè\\
         mɛ tí dè \\
           1\textsc{sg} {\NEG}  eat\\
    \trans `I don't EAT.'
\ex \label{tileb}
  \glll  mɛ̀ɛ́ dé{\bfseries lɛ́} \\
        mɛ̀ɛ́ dé-lɛ́ \\
         1\textsc{sg}.\textsc{prs}.{\NEG} eat-{\NEG}    \\
    \trans `I DON'T eat.'
\z
\z

\noindent In negation with {\itshape tí}, the lexical verb following the auxiliary is in focus position. In contrast, standard \textsc{present} negation with -{\itshape lɛ} focuses the negation.

Impressionistically, it seems that {\itshape tí} in main clauses is often used in conjunction with the adverb {\itshape ná} `still', giving a reading of `anymore' under negation. This might be the case because adverbs modify lexical verbs and the lexical verb is focused in \REF{tile2a}. When negation is focused, as in \REF{tile2b}, however, the use of adverbs such as {\itshape ná} `still' is also grammatical.

\ea\label{tile2}
\ea \label{tile2a}
  \glll  mɛ̀ {\bfseries tí} ná dè \\
         mɛ tí ná dè \\
           1\textsc{sg} {\NEG} still eat\\
    \trans `I don't EAT anymore.'
\ex \label{tile2b}
  \glll  mɛ̀ɛ́ dé{\bfseries lɛ́} ná \\
        mɛ̀ɛ́ dé-lɛ́ ná \\
         1\textsc{sg}.\textsc{prs}.{\NEG} eat-{\NEG} still\\
    \trans `I DON'T eat anymore.'
\z
\z


{\itshape tí} is the only Gyeli negation marker that frequently undergoes code-switching with Kwasio in the corpus, as in \REF{ti5}. In Kwasio, the regular correspondence to Gyeli {\itshape tí} is {\itshape kí} or {\itshape kílɛ̀} in \REF{ti6}.

\ea\label{ti5}
  \glll mɛ̀ {\bfseries kí} bɛ̀ nà tsídí \\
       mɛ kí bɛ̀ nà tsídí \\
       1\textsc{sg}.{\PST}1 {\NEG}[Kwasio] be {\COM} $\emptyset$1.meat\\
    \trans `I didn't have any meat.'
\z

\noindent The difference between {\itshape kí} and {\itshape kílɛ̀} in Kwasio might relate to different tense categories, as in \REF{ti5}, in which {\itshape kí} is located in the past, whereas {\itshape kílɛ̀} in \REF{ti6} encodes the present. If this is the case,\footnote{There is very little information on Kwasio, and \posscitet{woungly71} description of negation in Ngumba does not give a concise account of the different functions of {\itshape ki} or {\itshape kile}, but it seems that, as in Gyeli, both negation markers are found in different tense categories.} the Kwasio negation auxiliaries might encode different tense categories than Gyeli {\itshape tí}: if {\itshape kí} only substituted the form {\itshape tí} in \REF{ti5}, the tense reading should be present. Speakers are very clear, however, that the sentence encodes the past. Whether the Gyeli use of Kwasio negation markers is identical to their use in Kwasio in terms of tense encoding is a question that cannot be answered here.

\ea\label{ti6}
  \glll bá lã́ pámò vâ tɛ́ɛ̀ bà kwɛ̀lɔ̃́ɔ̃̀ yɔ̂ {\bfseries kílɛ̀} dyúwɔ̀  tsíyà \\
      ba-H lã̀-H pámo vâ tɛ́ɛ̀ ba kwɛ̀lɔ̃́ɔ̃̀ y-ɔ̂ kílɛ̀ dyúwɔ̀  tsíyà \\
       2\textsc{sg}-\textsc{prs} pass-{\R} arrive here now 2\textsc{sg}.{\PST}1 cut.{\COMPL} 7-{\OBJ} {\NEG}[Kwasio] hear $\emptyset$1.question\\
    \trans `They pass and arrive here now, they cut it already without asking [lit. without hearing a question].'
\z



\subsubsection{Negation with {\itshape dúù}}
\label{sec:NEGduu}

The auxiliary {\itshape dúù} `should/must not', although having a lexical meaning, is classified as a true auxiliary, since it is restricted to the \textsc{present} and \textsc{subjunctive} categories. In the \textsc{present}, {\itshape dúù} `should/must not' takes a realis-marking H tone, as in \REF{duu1a}, just as its positive counterpart {\itshape yánɛ} `must' in \REF{duu1b}.\footnote{{\itshape yánɛ} `must' is classified as a modal semi-auxiliary and discussed in \sectref{sec:ComplSemi}, since it does not seem to have any tense-mood restrictions, unlike {\itshape dúù} `must not'.}


\ea \label{duu1}
\ea \label{duu1a}
  \glll bé dúú vũ̀ũ̀\\
      be-H dúù-H vũ̀ũ̀ \\
        2\textsc{pl}-\textsc{prs} must.not-{\R} worry\\
    \trans `You (pl.) should/must not worry.'
\ex\label{duu1b}
  \glll bé yánɛ́ vũ̀ũ̀ \\
      be-H yánɛ-H vũ̀ũ̀ \\
        2\textsc{pl}-\textsc{prs} must-{\R} worry\\
    \trans `You (pl.) should/must worry.'
\z
\z

{\itshape dúù} is also used in its  \textsc{subjunctive} form in main clauses, as in \REF{duu2a}. The difference from the \textsc{present} forms in \REF{duu1} is that {\itshape dúù} `should/must not' lacks the realis-marking H tone. Its positive counterpart is a subjunctive construction in \REF{duu2b} instead of the modal semi-auxiliary construction in \REF{duu1b}.

\ea \label{duu2}
\ea \label{duu2a}
  \glll bé dúù kɛ̀ tísɔ̀nì \\
      be-H dúù kɛ̀ tísɔ̀nì \\
        2\textsc{pl}-\textsc{prs} must.not.{\SBJV} go $\emptyset$7.town\\
    \trans `You (pl.) may/should not go to town.'
\ex\label{duu2b}
  \glll bé kɛ́ɛ̀ tísɔ̀nì \\
      be-H kɛ́ɛ̀ tísɔ̀nì \\
        2\textsc{pl}-\textsc{prs} go.{\SBJV} $\emptyset$7.town\\
    \trans `You (pl.) may/should go to town.'
\z
\z

Like the positive \textsc{subjunctive} forms, the lexically negative \textsc{subjunctive} form of {\itshape dúù} `should/must not' is found in complement clauses, as in \REF{duu3a}. The affirmative counterpart is given in \REF{duu3b}.

\ea\label{duu3}
\ea \label{duu3a}
  \glll bùdì bà wúmbɛ́ nâ bá {\bfseries dúù} dyùù nyɛ̂\\
         b-ùdì ba wúmbɛ-H nâ ba-H dúù dyùù nyɛ̂ \\
       ba2-person 2.{\PST}1 want-{\R} {\COMP} 2-\textsc{prs} must.not.{\SBJV} kill  1.{\OBJ} \\
    \trans `The people wanted that he not be killed.'
\ex\label{duu3b}
  \glll bùdì bà wúmbɛ́ nâ bá dyúù nyɛ̂\\
         b-ùdì ba wúmbɛ-H nâ ba-H dyùù.{\SBJV} nyɛ̂ \\
       ba2-person 2.{\PST}1 want-{\R} {\COMP} 2-\textsc{prs} kill.{\SBJV} 1.{\OBJ} \\
    \trans `The people wanted that he be killed.'
\ex\label{duu3c}
  \glll bùdì bà sàlɛ́ wúmbɛ̀ nâ bá dyúù nyɛ̂\\
         b-ùdì ba sàlɛ́ wúmbɛ nâ ba-H dyùù nyɛ̂ \\
       ba2-person 2.{\PST}1 want-{\R} {\COMP} 2-\textsc{prs} must.not.{\SBJV} kill 1.{\OBJ} \\
    \trans `The people did not want that he be killed.'
\z
\z

\noindent Rather than the negative \textsc{subjunctive} {\itshape dúù} `should/must not', however, negation of the matrix clause is generally preferred, as in \REF{duu3c}.










\subsection{Single \textsc{stamp} predicates with semi-auxiliaries}
\label{sec:ComplSemi}

 The formal difference between true auxiliaries and semi-auxiliaries in Gyeli is discussed in \sectref{sec:AUX}. Semi-auxiliary verbs in Gyeli belong to three different semantic verb classes:
\begin{enumerate}
\itshapeem Aspectual verbs ({\itshape sílɛ} `finish', {\itshape pã̂} `do first', {\itshape táalɛ} `begin', {\itshape bàga nà} `stop')
\itshapeem Deictic motion/posture verbs ({\itshape kɛ̀} `go', {\itshape njì} `come', {\itshape lígɛ} `stay', {\itshape lã̀} `pass')
\itshapeem Modal verbs ({\itshape lèmbɔ} `can/know', {\itshape kwàlɛ} `like', {\itshape wúmbɛ} `want', {\itshape yánɛ} `must')
\end{enumerate}
I provide examples of each in the remainder of this section.



\subsubsection*{{\itshape sílɛ̀} `finish'}
The semi-auxiliary {\itshape sílɛ} `finish' is used aspectually in complex predicates with a \textsc{non-complete accomplishment} ({\NCA}) reading.\footnote{Special thanks to Hana Filip for her advice on aspect category meaning and terminology.} As explained in \sectref{sec:COMPL}, {\itshape sílɛ} `finish' implies that somebody has ceased to do an activity, without entailing that the activity has been carried out to completion (unlike the \textsc{absolute completive} {\itshape mɔ̀}). Thus, the question in \REF{silea} is interpreted as concerning whether the addressee is done sweeping, but not whether he or she has swept everything (the whole house or yard).

\ea\label{silea}
  \glll     nà wɛ̀ {\bfseries sílɛ́} wɔ̀mbɛ̀lɛ̀\\
           nà wɛ sílɛ-H wɔ̀mbɛlɛ \\
           Q 2\textsc{sg}.\textsc{pst}1 finish-{\R} sweep\\
    \trans `Have you finished sweeping?'
\z

Besides this non-complete accomplishment implication, one of  the core functions of {\itshape sílɛ̀} is to express distributivity of an event or kind. The description of palm wine in \REF{sile1},\footnote{The occurrence of semi-auxiliaries as finite or non-finite verbs in complex predicates is addressed in \sectref{sec:ComplMulti}.} for example, involves many episodes of `drinking a palm tree', namely coming back every day and harvesting the wine. This does not mean that there is not a drop of sap left in the palm trees at the end, but that the speaker will keep harvesting palm wine from the trees until he is done with these multiple actions. The same is true for \REF{silea}, where the event of sweeping is composed of many episodes of moving the broom over the ground.
% for publication, be more precise what distributivity exactly means here

\ea\label{sile1}
  \glll   mɛ̀ nzíí kɛ̀ nà vúlɛ́ lévúdũ̂ nà lèvúdũ̂ mɛ́ táálɛ́ {\bfseries sílɛ̀} nyùlɛ̀ \\
          mɛ nzíí kɛ̀ nà vúlɛ-H H-le-vúdũ̂ nà le-vúdũ̂ mɛ-H táálɛ-H sílɛ nyùlɛ \\
           1\textsc{sg} {\PROG}.\textsc{prs} go {\COM} take.away-{\R} {\OBJ}.{\LINK}-le5-one {\COM} le5-one 1\textsc{sg}-\textsc{prs} begin-{\R} finish drink\\
    \trans `I'm taking down [palm trees] one by one, I start to drink [them] up [make palm wine out of them].'
\z

Under this distributive function, {\itshape sílɛ} `finish' can only be used with plural subjects and only in certain contexts. For example, \REF{Nsile11}, where the event distributes over the different participants is grammatical, whereas \REF{Nsile12}, which has a singular subject, is ungrammatical.


\ea\label{Nsile1}
\ea  \label{Nsile11}
  \glll  bà sílɛ́ kɛ̀ \\
          ba sílɛ-H kɛ̀ \\
         2.{\PST}1 finish-{\R} go\\
    \trans `They have all gone.'
\ex[*]{\label{Nsile12}
  \glll   à sílɛ́ kɛ̀ \\
          a sílɛ-H kɛ̀\\
          1.{\PST}1 finish-{\R} go\\
    \trans `*He has all gone.'}
\z
\z

In this respect, {\itshape sílɛ} `finish' differs from other semi-auxiliaries, which do not have a distributive function, such as {\itshape táalɛ} `start' in \REF{taale}, which allows both plural and singular participants.

\ea\label{taale}
\ea  \label{taale1}
  \glll  bà táálɛ́ kɛ̀ \\
          ba táalɛ-H kɛ̀ \\
         2.{\PST}1 begin-{\R} go\\
    \trans `They began to walk.'
\ex\label{taale2}
  \glll   à táálɛ́ kɛ̀\\
           a táalɛ-H kɛ̀\\
          1.{\PST}1 finish-{\R} go\\
    \trans `He began to walk.'
\z
\z


\noindent A singular participant is, however, grammatical even with {\itshape sílɛ} `finish'  if there are several events over which the aspect marker is distributing. \REF{sile3} shows a coordinated clause where the first constituent is almost identical to the ungrammatical phrase in \REF{Nsile12}. The second constituent adds another event, however, over which {\itshape sílɛ} can distribute, thereby making \REF{sile3} acceptable.

\ea\label{sile3}
  \glll áà {\bfseries sílɛ́} kɛ̀ nà dvùwɔ́ dyúwɔ̀\\
     áà sílɛ-H kɛ̀ nà dvùwɔ-H dyúwɔ \\
        1.{\PST}2 finish-{\R} go {\CONJ} stuff-{\R} $\emptyset$7.top\\
    \trans `He has gone and stuffed the top [with straw],'
\z

Other examples of {\itshape sílɛ} as distributing over individuals are given in \REF{sile4} and \REF{sile5}. In \REF{sile4}, Nzambi of the story in \appref{sec:Nzambi} forces his friend's entire family  to enter a house. {\itshape sílɛ} `finish' refers to the individual people who have to enter one after the other.

\ea\label{sile4}
  \glll nyáà ngà {\bfseries sílɛ́} nyî ndáwɔ̀ dé tù \\
       nyáà ngà sílɛ́-H nyî ndáwɔ̀ dé tù \\
       shit.{\IMP} {\PL} finish-{\R} enter $\emptyset$9.house {\LOC} inside\\
    \trans `Piss off, everybody go into the house!'
\z

\noindent In \REF{sile5}, the chief of Ngolo talks about his fruit trees that will be destroyed once the road for the port passes through their village. Again, {\itshape sílɛ} does not necessarily imply that not a single tree will be left at the end, but rather points to the distributivity of destroying one tree after the other.

\ea\label{sile5}
  \glll   byɛ́sɛ̀ béè {\bfseries sílɛ̀} ntàmànɛ̀\\
      by-ɛ́sɛ̀ béè sílɛ ntàmanɛ \\
           8-all 8.{\FUT} finish ruin\\
    \trans `They will all be ruined.'
\z







\subsubsection*{{\itshape pã̂} `first'}
Although {\itshape pã̂} is consistently translated into French as {\itshape d'abord} `first', I gloss it as `do first', as it is clearly a semi-auxiliary verb (\sectref{sec:AUX}). {\itshape pã̂} `do first' has a priorative aspectual meaning. It has no tense-mood restrictions, however, in the corpus, {\itshape pã̂} never occurs in \textsc{past} tenses.  This may have semantic/pragmatic reasons. Examples for {\itshape pã̂} in the \textsc{present} are given in \REF{pa1} and \REF{pan1}.

\ea\label{pa1}
  \glll yíì pẽ̀'ẽ̀ nyà mwánɔ̀ mùdũ̂ mɛ́ {\bfseries pã́ã́} ná nyɔ̂ vɛ̀\\
       yíì pẽ̀'ẽ̀ nyà m-wánɔ̀ m-ùdũ̂ mɛ-H pã́ã̀-H ná ny-ɔ̂ vɛ̀ \\
      7.ID $\emptyset$9.wisdom 9:{\ATT} \textsc{n}1-child \textsc{n}1-male  1\textsc{sg}-\textsc{prs} do.first-H again 9-{\OBJ} give\\
    \trans `This is the wisdom of a boy [every child knows this], I will take revenge on him.'
\ex\label{pan1}
  \glll   wɛ̀ mɛ́dɛ́ p{\bfseries ã́} lígɛ̀ yá nà nyɛ̀ yá kɛ́ mánkɛ̃̂  \\
         wɛ mɛ́dɛ́ pã̂-H lígɛ ya-H nà nyɛ ya-H kɛ̀-H H-ma-nkɛ̃̂  \\
           2\textsc{sg} self do.first-{\R} stay 1\textsc{pl}-\textsc{prs} {\COM} 1  1\textsc{pl}-\textsc{prs} go-{\R} {\OBJ}.{\LINK}-6-field\\
    \trans `You stay first, we and her, we go to the field.'
\z

\noindent In \REF{pa2}, {\itshape pã̂} `do first' occurs in the \textsc{future} and therefore lacks the realis-marking H tone.

\ea\label{pa2}
  \glll bwáà {\bfseries pã́ã̀} ngâ dyà nà pówàlà wû\\
        bwáà pã́ã̀ ngâ dyà nà pówàlà wû \\
        2\textsc{pl}.{\FUT} do.first {\PL} sleep {\COM} $\emptyset$7.calm there\\
    \trans `You (pl.) will first sleep quietly there.'
\z

\noindent {\itshape pã̂} has also been observed to occur in the \textsc{imperative} form, as in \REF{pa3}.

\ea\label{pa3}
  \glll {\bfseries pã̂} bígɛ̀  \\
        pã̂ bígɛ̀.  \\
         do.first.{\IMP}  develop\\
    \trans `Go on [speak] first!'
\z




Other semi-auxiliaries that express the start or end point of an event are {\itshape táalɛ} `start' and {\itshape bàga nà} `stop doing sth.', as exemplified in \REF{start1} and \REF{stop1}, respectively.

\ea\label{start1}
  \glll  dɔ̃̀ bí yá {\bfseries táálɛ́} bê yàlànɛ̀ àà \\
        dɔ̃̀ bí ya-H táálɛ-H bê yàlanɛ àà \\
       so[French] 1\textsc{pl}.{\SBJ}  1\textsc{pl}-\textsc{prs} begin-{\R} 2\textsc{pl} respond[Bulu] {\EXCL}   \\
    \trans `So we start to respond to you, mhm.'
\ex\label{stop1}
  \glll  Tsímbɔ̀ à {\bfseries bàgá} {\bfseries nà} bâ básìgá \\
        Tsímbɔ̀ a bàga-H nà bâ H-ba-sìgá  \\
       $\emptyset$1.{\PN} 1.{\PST}1 stop-{\R} {\COM} smoke {\OBJ}.{\LINK}-ba6-cigarette\\
    \trans `Tsimbo stopped smoking.'
\z



\subsubsection*{Deictic motion and location verbs}
Deictic motion and location verbs can serve as semi-auxiliaries, as shown in \REF{AUXde1} through \REF{AUXde4}. The most pervasive motion verbs are {\itshape kɛ̀} `go' and {\itshape njì} `come'. {\itshape kɛ̀} `go', as in \REF{AUXde1}, always has an altrilocal meaning, i.e.\ the event expressed in the main verb takes place at a location different from where the speaker is at the point of utterance.


\ea\label{AUXde1}
  \glll    ngùndyá mɛ́ {\bfseries kɛ́} sɔ́lɛ̀gà ngùndyá dyúwɔ̀ \\
          ngùndyá mɛ-H kɛ̀-H sɔ́lɛga ngùndyá dyúwɔ̀ \\
              $\emptyset$9.raffia 1\textsc{sg}-\textsc{prs} go-{\R} chop $\emptyset$9.raffia on.top\\
    \trans `The raffia, I go to chop the raffia on top.'
\z

{\itshape njì} `come' naturally constitutes the counterpart to this altrilocal function. Thus, it expresses that the event of the lexical verb takes place at or towards the location of the speaker, as shown in \REF{AUXde2}.

\ea\label{AUXde2}
  \glll ɛ́ tè wɛ̀gà wɛ́ {\bfseries njí} sâ mbvúndá ɛ́ ndzǐ vâ \\
        ɛ́ tè wɛ̀-gà wɛ-H njì-H sâ mbvúndá ɛ́ ndzǐ vâ \\
        {\LOC} there 2\textsc{sg}-{\CONTR} 2\textsc{sg}-\textsc{prs} come-{\R} do $\emptyset$9.trouble {\LOC} $\emptyset$9.path here\\
    \trans `There you, you come to make trouble on the way here.'
\z

\noindent  {\itshape lígɛ} `stay' also expresses information about the location of an event, namely that it is the same as the location of the utterance, as in \REF{AUXde3}.
 

\ea\label{AUXde3}
  \glll  mɛ̀gà mɛ́ {\bfseries lígɛ́} dè mwánɔ̀ wɔ́ɔ̀ \\
        mɛ-gà mɛ-H lígɛ-H dè m-wánɔ̀ w-ɔ́ɔ̀ \\
          1-{\CONTR} 1\textsc{sg}-\textsc{prs} stay-{\R} eat \textsc{n}1-child 1-{\POSS}.2\textsc{sg}  \\
    \trans `As for me, I stay and eat your child.'
\z


\noindent Finally,  {\itshape lã̀} `pass' has also been observed to serve as a semi-auxiliary, as in \REF{AUXde4}.

\ea\label{AUXde4}
  \glll bá {\bfseries lã́} pámò vâ tɛ́ɛ̀ bà kwɛ̀lɔ̃́ɔ̃̀ yɔ̂ kílɛ̀ dyúwɔ̀  tsíyà \\
      ba-H lã̀-H pámo vâ tɛ́ɛ̀ ba kwɛ̀lɔ̃́ɔ̃̀ y-ɔ̂ kílɛ̀ dyúwɔ̀  tsíyà \\
       2\textsc{sg}-\textsc{prs} pass-{\R} arrive here now 2\textsc{sg}.{\PST}1 cut.{\COMPL} 7-{\OBJ} {\NEG}[Kwasio] hear $\emptyset$1.question\\
    \trans `They pass and arrive here now, they cut it already without asking [lit. without hearing a question].'
\z






\subsubsection*{Modal verbs}
Modal verbs constitute a third semantic class of semi-auxiliaries in Gyeli. \REF{lembo1} through \REF{AUXmo3} provide examples of various modal verbs.

\ea\label{lembo1}
  \glll wɛ̀ {\bfseries lèmbṍõ̀} sâ bányá màmbò \\
       wɛ lèmbṍõ̀ sâ H-ba-nyá m-àmbò\\
        2\textsc{sg}.{\PST}1 know.{\COMPL}  do {\OBJ}.{\LINK}-ba2-important ma6-thing\\
    \trans `You can/know to do the important things.'
\ex\label{kwale1}
  \glll á {\bfseries kwàlɛ́} ná gyìmbɔ̀ mánzã̀ mɛ́sɛ̀ \\
       a-H kwàlɛ-H ná gyìmbɔ H-ma-nzã̀ m-ɛ́sɛ̀ \\
        1-\textsc{prs} like-{\R} still dance {\OBJ}.{\LINK}-ma6-dance 6-all\\
    \trans `He still likes to dance all dances.'
\ex\label{want1}
  \glll     [mɛ́ {\bfseries wúmbɛ́} lɛ́ɛ̀] nà bɔ̂\\
           {\db}mɛ-H wúmbɛ-H lɛ́ɛ̀ nà bɔ̂ \\
              {\db}1\textsc{sg}-\textsc{prs} want-{\R} talk[Kwasio] {\COM} 2.{\OBJ}   \\
    \trans `I want to talk with them.'
\ex\label{want2}
  \glll  bí bɔ́gà [yá {\bfseries wúmbɛ́} ndáà pã̂ nyɛ̂] sâ bá gyíbɔ́ ngyùlɛ̀ wá kùrã̂ \\
         bí bɔ́-gà {\db}ya-H wúmbɛ-H ndáà pã̂ nyɛ̂ sâ ba-H gyíbɔ-H ngyùlɛ̀ wá kùrã̂ \\
          1\textsc{pl}.{\SBJ} 2-other {\db}1\textsc{pl}-\textsc{prs} want-{\R} also do.first see $\emptyset$7.thing 2-\textsc{prs} call-{\R} $\emptyset$3.light 3:{\ATT} $\emptyset$7.electricity[French]  \\
    \trans `We others, we also want to first see the thing they call the light of electricity.'
\ex\label{AUXmo3}
  \glll  dɔ̃̀ wɛ̀ bùdɛ́ ná bàfû wɛ́ {\bfseries yànɛ́} gyàgà bɔ̂\\
       dɔ̃̀ wɛ bùdɛ-H ná ba-fû wɛ-H yànɛ-H gyàga b-ɔ̂ \\
        so[French] 2\textsc{sg} be-{\R} again ba2-fish 2\textsc{sg}-\textsc{prs} must-{\R} buy 2-{\OBJ}  \\
    \trans `So, you have fish again, you have to buy them.'
\z



\noindent Many of the modal semi-auxiliaries are also used in the matrix clause of subordination through the complementizer {\itshape nâ} (\sectref{sec:CompC}).



\subsection{Types of complexity in single \textsc{stamp} predicates}
\label{sec:ComplMulti}

Complex predicates with a single \textsc{stamp} construction can be complex in different ways. First, they can include morphological complexity through the \textsc{absolute completive} marker {\itshape mɔ̀} (\sectref{sec:COMPL}). Second, they can differ in the number of finite verbs they contain (either one or two).  I will discuss both cases in turn, describing which grammatical categories can combine in complex predicates with a single \textsc{stamp} marker and which cannot.

The \textsc{absolute completive} marker {\itshape mɔ̀} occurs not only in simple predicates but also  in complex predicates. Unsurprisingly, {\itshape mɔ̀} (or its nasal vowel variant at the end of the verb) occurs on the finite verb, as in \REF{AUXas2}.

\ea\label{AUXas2}
  \glll kɛ́ mbúmbù bwánɔ̀ bà {\bfseries sílɛ̃́ɛ̃̀} {\bfseries kɛ̀} vɛ́ \\
       kɛ́ mbúmbù b-wánɔ̀ ba sílɛ̃́ɛ̃̀ kɛ̀ vɛ́ \\
        {\EXCL} $\emptyset$1.namesake ba2-child 2.{\PST}1 finish.{\COMPL} go where\\
    \trans `Ay namesake, where have all the children gone to?'
\z

\noindent What is more remarkable is that {\itshape mɔ̀} can also occur on the first non-finite verb, as in \REF{nzicompl}. This is the case when the finite verb is the true auxiliary {\itshape nzí}, which marks \textsc{progressive}. Other true auxiliary combinations with {\itshape mɔ̀} are ungrammatical. This includes any combination with negation auxiliaries, since aspect marking is lost under negation in single \textsc{stamp} constructions.

\ea\label{nzicompl}
  \glll nkɛ̀ nyì {\bfseries nzí} síl{\bfseries ɛ̃́ɛ̃̀} bédéwò \\
          nkɛ̀ nyi nzí sílɛ̃́ɛ̃̀ H-be-déwò. \\
          $\emptyset$9.field 9 {\PROG}.{\PST} finish.{\COMPL} {\OBJ}.{\LINK}-be8-food\\
    \trans `This field was already running out of food.'
\z

Complex predicates can also vary in their syntactic complexities.
 Having presented multiple examples of two-verb complex predicates in \sectref{sec:ComplAUX} and \sectref{sec:ComplSemi}, I show constructions with three verbs  in the following.
Regardless of whether a complex predicate has one or two non-finite verbs, true auxiliaries can only appear as the finite verb. 
An example of a true auxiliary with two non-finite verbs is given in \REF{lo7}.

\ea\label{lo7}
  \glll bɔ́nɛ́gá [bá {\bfseries lɔ́} sílɛ̀ làwɔ̀] nâ bvúlɛ̀ bá ntɛ́gɛ́lɛ́ bágyɛ̀lì \\
      bɔ́-nɛ́gá {\db}ba-H lɔ́ sílɛ làwɔ nâ bvúlɛ̀ ba-H ntɛ́gɛlɛ-H H-ba-gyɛ̀lì \\
        2-other {\db}2-\textsc{prs} {\RETRO}  finish speak {\COMP} ba2.Bulu 2-\textsc{prs} bother-{\R} {\OBJ}.{\LINK}-ba2-Gyeli\\
    \trans `The others have just said that the Bulu bother the Bagyeli.'
\z

\noindent The same construction is possible with a negation auxiliary, as in \REF{lo8}.

\ea\label{lo8}
  \glll bɔ́nɛ́gá [bà {\bfseries pálɛ́} sílɛ̀ làwɔ̀] \\
      bɔ́-nɛ́gá {\db}ba pálɛ́ sílɛ làwɔ \\
        2-other {\db}2.{\PST}1 {\NEG}.{\PST}.{\R}  finish speak\\
    \trans `The others have not finished speaking.'
\z

Since semi-auxiliaries have a lexical meaning and are less grammaticalized (\sectref{sec:AUX}), they can occur as either the finite or the non-finite verb in a complex predicate. In \REF{3AUX5}, {\itshape kɛ̀} `go' is the finite first verb, while in \REF{ke6}, it is the non-finite second verb.

\ea\label{3AUX5}
  \glll  bwánɔ̀ bá kálɛ́ bã̂ bɔ̀ [bá {\bfseries kɛ́} sílɛ̀ pándɛ̀] \\
          b-wánɔ̀ bá kálɛ́ b-ã̂ bɔ̂ {\db}ba-H kɛ̀-H sílɛ pándɛ \\
         ba2-child 2:{\ATT} $\emptyset$1.older.sister 2-{\POSS}.1\textsc{sg} 2.{\OBJ} {\db}2-\textsc{prs} go-{\R} finish arrive\\
    \trans `The children of my older sister, they all arrive.'
\ex\label{ke6}
  \glll [mɛ́ pã́ ná {\bfseries kɛ̀} dígɛ̀] mùdì wà nû ɛ́ pɛ́ɛ́ \\
        {\db}mɛ-H pã̂-H ná kɛ̀ dígɛ m-ùdì wà nû ɛ́ pɛ́-ɛ́ \\
        {\db}1\textsc{sg}-\textsc{prs} do.first-H again go see \textsc{n}1-person 1:{\ATT} 1.{\DEM}.{\PROX} {\LOC} over.there.{\DIST} \\
    \trans `I go first again to see this person over there.'
\z

\noindent The same distribution applies, for instance, to the semi-auxiliary {\itshape sílɛ} `finish' in \REF{3AUX3} and \REF{silex3}.

\ea\label{3AUX3}
  \glll  ɛ́ vâ mɛ̀ dyùwɔ́ nâ ɛ́ vâ [yíì {\bfseries sílɛ̀} njì búlɛ̀] \\
        ɛ́ vâ mɛ dyùwɔ-H nâ ɛ́ vâ {\db}yíì sílɛ njì búlɛ \\
         {\LOC} here 1\textsc{sg}.{\PST}1 hear-{\R} {\COMP} {\LOC} here {\db}7.{\FUT} finish come destroy\\
    \trans `Here, I heard that it will all become destroyed here.'
\ex\label{silex3}
  \glll   mɛ̀ nzíí kɛ̀ nà vúlɛ̀ lévúdũ̂ nà lèvúdũ̂ [mɛ́ táálɛ́ {\bfseries sílɛ̀} nyùlɛ̀] \\
          mɛ nzíí kɛ̀ nà vúlɛ H-le-vúdũ̂ nà le-vúdũ̂ {\db}mɛ-H táálɛ-H sílɛ nyùlɛ \\
           1\textsc{sg} {\PROG}.\textsc{prs} go {\COM} take.away {\OBJ}.{\LINK}-le5-one {\COM} le5-one {\db}1\textsc{sg}-\textsc{prs} begin-{\R} finish drink\\
    \trans `I'm taking down [palm trees] one by one, I start to drink [them] (= make palm wine out of them).'
\z

\noindent Lexical verbs that cannot serve as semi-auxiliaries, such as {\itshape nyùlɛ} `drink' in \REF{silex3}, can only ever occur as the final non-finite verb in a complex predicate. In contrast, verbs that serve otherwise as semi-auxiliaries, can also appear for their lexical meaning in the final non-finite verb position of a complex predicate, as in \REF{silex33}.

\ea\label{silex33}
  \glll   [bà nzí kɛ̀ {\bfseries sílɛ̀}] bédéwɔ̀ \\
          {\db}ba nzí kɛ̀ sílɛ H-be-déwɔ̀ \\
        {\db}2.{\PST}1 {\PROG}.{\PST} go finish {\OBJ}.{\LINK}-be8-food\\
    \trans `They were coming to finish the food.'
\z

















\subsection{Double \textsc{stamp} predicates with {\itshape bɛ̀} `be'}
\label{sec:Compbe}

The second type of complex predicate comprises those that involve two \textsc{stamp} markers that refer to the same entity and that both precede a finite verb form: 
\begin{center}[\textsc{stamp}\textsubscript{i} -- {\itshape bɛ̀} `be']\textsubscript{1} -- [\textsc{stamp}\textsubscript{i} -- V]\textsubscript{2}
\end{center}
 
\noindent The first constituent, which I also call the {\itshape bɛ̀} constituent, always involves the verb {\itshape bɛ̀} `be'. It expresses basic tense-mood and polarity distinctions, while the second constituent is specified for tense-mood and/or aspect marking. This complex predicate type thus allows  the combination of tense-mood, aspect, and negation categories that cannot all be combined in simple predicates or in single \textsc{stamp} complex constructions. In the following, I will show the different combinatory possibilities, which include the main combinations of (i) tense-mood with a different tense-mood category, (ii) tense-mood with aspect, and (iii) negation with aspect. These double \textsc{stamp} constructions are rare in the corpus, but they are more pervasive in questionnaires such as the ``EUROTYP''    future and perfect questionnaires \citep{dahl2000}, as well as in elicitations.

\subsubsection*{Combinations of two tense-mood categories}
Double \textsc{stamp} constructions can combine different tense-mood categories, shifting the temporal perspective on events. The different temporal perspective (relative to speech time) is expressed through the tense-mood category of the verb {\itshape bɛ̀} `be' in the first constituent.  The time of the  second constituent, indicated by square brackets, is then relative to the time anchor of the first constituent. In \REF{embed1}, for instance, the time perspective is moved to the \textsc{future} in the {\itshape bɛ̀} constituent. From this perspective, the \textsc{present} tense of the second constituent expresses temporal identity to the \textsc{present} in the {\itshape bɛ̀} constituent.

\ea{\label{embed1}
  \glll mɛ̀ɛ̀ bɛ̀ [mɛ́ gyámbɔ́ bédéwɔ̀]\textsubscript{\textsc{pres}}\\
        mɛ̀ɛ̀ bɛ̀  {\db}mɛ-H gyámbɔ-H H-be-déwɔ̀ \\
        1\textsc{sg}.{\FUT} be {\db}1\textsc{sg}-\textsc{prs} cook-{\R} {\OBJ}.{\LINK}-be8-food\\}\jambox*{({\FUT} - \textsc{prs})}
    \trans `I will be cooking food.'
\z


\REF{embed2} shows that a change of the tense-mood category in the second constituent entails a change in the relation between the newly adopted time perspective and the situation. While the {\itshape bɛ̀} constituent still anchors the time perspective in the \textsc{future}, the situation of cooking will have been completed in the \textsc{remote past}.

\ea{\label{embed2}
  \glll mɛ̀ɛ̀ bɛ̀ [mɛ́ɛ̀ gyámbɔ́ bédéwɔ̀]\textsubscript{{\PST}2}  \\
        mɛ̀ɛ̀ bɛ̀ {\db}mɛ́ɛ̀ gyámbɔ-H H-be-déwɔ̀ \\
        1\textsc{sg}.{\FUT} be {\db}1\textsc{sg}.{\PST}2 cook-{\R} {\OBJ}.{\LINK}-be8-food\\} \jambox*{({\FUT} - {\PST}2)}
    \trans `I will have cooked food.'
\z

In contrast, changing the tense-mood category in the {\itshape bɛ̀} constituent simply anchors speech time at that particular reference time. In \REF{embed3}, the second constituent contains \textsc{inchoative} marking. The tense-mood category of the {\itshape bɛ̀} constituent changes, however. In \REF{embed3a}, it is encoded for \textsc{future}, whereas it is encoded for  \textsc{recent} \textsc{past} in \REF{embed3b}.

\ea\label{embed3}
\ea {\label{embed3a}
  \glll  àà bɛ̀ [àá gyì]\textsubscript{{\INCH}} nàmɛ́nɔ́ \\
          àà bɛ̀ {\db}àá gyì nàmɛ́nɔ́\\
           1.{\FUT} be-{\PST} {\db}1.{\INCH} cry tomorrow\\}\jambox*{({\FUT} - {\INCH})}
    \trans `She will be starting to cry tomorrow.'
\ex\label{embed3b}
  {\glll  à bɛ́ [àá gyì]\textsubscript{{\INCH}} nàkùgúù \\
          a bɛ̀-H {\db}àá gyì nàkùgúù \\
           1.{\PST}1 be-{\PST} {\db}1.{\INCH} cry yesterday\\}\jambox*{ ({\PST}1 - {\INCH})}
    \trans `She was starting to cry yesterday.'
\z
\z

 Impressionistically, it seems that any two tense-mood categories can be combined. 
\REF{embed4}, taken from the corpus, shows that even the two \textsc{past} categories can be combined in double \textsc{stamp} constructions, a combination that might appear semantically or contextually unlikely.\footnote{Speakers translate this construction into Cameroonian French as {\itshape Il était étant couché. . .} `he was being lying'.}
Here, the {\itshape bɛ̀} constituent is encoded for the \textsc{remote past}, while the second constituent appears in the  \textsc{recent past}. The new time perspective relative to speech time is thus anchored in the  \textsc{remote past}, while the situation happens in the  \textsc{recent past}, relative to the new time anchor.

\ea\label{embed4}
  \glll áà bɛ́ [à bó nà màbádò nyúlɛ̀]\textsubscript{{\PST}1} \\
        áà bɛ̀-H {\db}a bô-H nà ma-bádò nyúlɛ̀ \\
        1.{\PST}2 be-{\R} {\db}1.{\PST}1 lie-{\R} {\COM} ma6-open.wound $\emptyset$9.body\\
    \trans `He was being lying with open wounds on the body.'
\z



\subsubsection*{Combinations of tense-mood and aspect}

Whereas true auxiliaries encoding aspect categories are restricted to certain tense-mood categories in single \textsc{stamp} constructions (\sectref{sec:ComplAUX}), aspect marking can be achieved for any tense-mood category in double \textsc{stamp} complex predicates. Anchoring speech time at a certain reference point is done in the {\itshape bɛ̀} constituent while aspect marking of the described situation is bound to the second constituent.
\REF{sub} illustrates this for the \textsc{progressive} aspect, which is anchored in the \textsc{future} in \REF{sub1} and in the \textsc{inchoative} in \REF{sub2}.\footnote{The \textsc{progressive} aspect is the only aspect auxiliary that has a suppletive form {\itshape nzɛ́ɛ́} for dependent constituents (\sectref{sec:PROG}), which has to be used in the second constituent instead of {\itshape nzíí} for the \textsc{present} or {\itshape nzí} for the \textsc{past} categories.}

\ea\label{sub}
\ea {\label{sub1}
  \glll    mɛ̀ɛ̀ bɛ̀ [mɛ̀ nzɛ́ɛ́ dè]\textsubscript{{\PROG}}\\
 mɛ̀ɛ̀ bɛ̀ {\db}mɛ nzɛ́ɛ́ dè \\
1\textsc{sg}.{\FUT} be {\db}1\textsc{sg} {\PROG}.{\SUB} eat\\}\jambox*{({\FUT} - {\PROG})}
    \trans `I will be eating.'
\ex{\label{sub2}
  \glll   mɛ̀ɛ́ bɛ̀ [mɛ̀ nzɛ́ɛ́ dè]\textsubscript{{\PROG}}\\
          mɛ̀ɛ́ bɛ̀ {\db}mɛ nzɛ́ɛ́ dè \\
              1\textsc{sg}.{\INCH} be {\db}1\textsc{sg} {\PROG}.{\SUB} eat\\}\jambox*{({\INCH} - {\PROG})}
    \trans `I start to be eating.'
\z
\z

\noindent Another example of the \textsc{progressive} in a double \textsc{stamp} construction is given in \REF{frame5}, showing a combination with the \textsc{remote past}.

\ea{\label{frame5}
  \glll  áà kɛ́ [à nzɛ́ɛ́ kɛ̀ nà gyìyɔ̀]\textsubscript{{\PROG}}   \\
          áà kɛ̀-H {\db}a nzɛ́ɛ́ kɛ̀ nà gyìyɔ \\
       1.{\PST}2 go-{\PST} {\db}1 {\PROG}.{\SUB} go {\COM} cry\\}\jambox*{({\PST}2 - {\PROG})}
    \trans `She left crying.'
\z

Other aspect markers, both particles and auxiliary verbs, also occur in the second constituent of a double \textsc{stamp} predicate, such as the \textsc{absolute completive} particle {\itshape mɔ̀ in \REF{suba1}} and  the \textsc{prospective} auxiliary {\itshape múà} in \REF{suba2}.

\ea\label{suba}
\ea {\label{suba1}
  \glll    mɛ̀ɛ̀ bɛ̀ [mɛ̀ lùngá mɔ̀]\textsubscript{{\PROG}}\\
            mɛ̀ɛ̀ bɛ̀ {\db}mɛ lùnga-H mɔ̀ \\
             1\textsc{sg}.{\FUT} be {\db}1\textsc{sg} grow-{\R} {\COMPL}\\}\jambox*{({\FUT} - {\COMPL})}
    \trans `I will have grown up.'
\ex{\label{suba2}
  \glll   mɛ́ɛ̀ bɛ́ [mɛ̀ múà dè]\textsubscript{{\PROG}} \\
          mɛ́ɛ̀ bɛ̀-H {\db}mɛ múà dè \\
              1\textsc{sg}.{\PST}2 be {\db}1\textsc{sg} {\PROSP} eat\\}\jambox*{({\PST}2 - {\PROSP})}
    \trans `I'm at the beginning of being eating.'
\z
\z



\subsubsection*{Combinations of negation and aspect}
Complex predicates with a double \textsc{stamp} marker also combine negation and aspect. Negation marking always appears in the {\itshape bɛ̀} constituent, which, at the same time,  specifies the reference time, as in \REF{eneg}. Aspect is encoded in the second constituent.

\ea\label{eneg}
\ea {\label{eneg1}
  \glll    mɛ̀ɛ́ bɛ́lɛ́ [mɛ̀ nzɛ́ɛ́ dè]\textsubscript{{\PROG}}\\
     mɛ̀ɛ́ bɛ́-lɛ {\db}mɛ nzɛ́ɛ́ dè \\
1\textsc{sg}.\textsc{prs}.{\NEG} be-{\NEG} {\db}1\textsc{sg} {\PROG}.{\SUB} eat\\}\jambox*{(\textsc{prs} - {\PROG})}
    \trans `I am not eating.'
\ex{\label{eneg2}
  \glll   mɛ̀ sàlɛ́ bɛ̀ [mɛ̀ nzɛ́ɛ́ dè]\textsubscript{{\PROG}}\\
          mɛ sàlɛ́ bɛ̀ {\db}mɛ nzɛ́ɛ́ dè \\
              1\textsc{sg}.{\PST}1 {\NEG}.{\PST} be {\db}1\textsc{sg} {\PROG}.{\SUB} eat\\}\jambox*{({\PST}1 - {\PROG})}
    \trans `I was not eating.'
\ex{\label{eneg3}
  \glll   mɛɛ̀̀ kálɛ̀ bɛ̀ [mɛ̀ nzɛ́ɛ́ dè]\textsubscript{{\PROG}}\\
          mɛ̀ɛ̀ kálɛ̀ bɛ̀ {\db}mɛ nzɛ́ɛ́ dè \\
              1\textsc{sg}.{\FUT} {\NEG}.{\FUT} be {\db}1\textsc{sg} {\PROG}.{\SUB} eat\\}\jambox*{({\FUT} - {\PROG})}
    \trans `I will not be eating.'
\z
\z

\noindent Future research is needed to explore the range of possible combinations and check whether all negation forms can combine with each aspect marker.



%\begin{exe} 
%\ex\label{frame2}
 % \glll  à múà [á kɛ́ jìí dé tù.] \\
 %         a múà a-H kɛ̀-H jìí dé tù  \\
 %      1.{\PST}1 be.almost 1-\textsc{prs} go-{\R} $\emptyset$7.forest {\LOC} inside\\
  %  \trans `He was about to go into the forest.'
%\z






