\addchap{Abbreviations}

For notation conventions, I use the Leipzig Glossing Rules. These may differ from abbreviations typically used in the lexicon. Abbreviations in the lexicon are generally in small characters ending in a dot while most abbreviations in glosses (except for noun class labels)  are represented in small capital letters. An exception form phonological abbreviations, which occur in capital letters.\\\largerpage[2]

\noindent\begin{tabularx}{.5\textwidth}[t]{@{}lQ}
 * & ungrammatical form  \\
%.  \\
%: \\
°& reconstructed form  \\
( )&  element in brackets is optional   \\
{}[] & phonetic transcription (\chapref{sec:Phon})\\
{}[Language]   & source language in code-switching (\appref{sec:AppII}) \\
- & morpheme boundary \\
<  & derived from \\
\textsuperscript{D} & proper name with a counterpart name in the other gender (\sectref{sec:properN}) \\
$\emptyset$ & prefixless noun class\\
1-9 &  agreement class 1-9 (\sectref{sec:AGR})  \\
%1nc  & noun class 1    \\
1\textsc{pl}  & first person plural    \\
2\textsc{pl}  & second person plural    \\
1\textsc{sg}  & first person singular    \\
2\textsc{sg}  & second person singular    \\
{\ADJ} & adjective (\sectref{sec:QUAL}) \\
{\ADV} & adverbial clause (\sectref{sec:ADVC}) \\
adv.  & adverb (\sectref{sec:ADV})   \\
{\AGR}  & agreement (\sectref{sec:AGR})   \\
{\ANA} & anaphoric marker (\sectref{sec:ANAfree}) \\
\end{tabularx}\begin{tabularx}{.5\textwidth}[t]{lQ@{}}
{\AP}  & associative plural (\sectref{sec:AP})  \\
autoc. & autocausative (\sectref{sec:AutoCaus}) \\
{\AUX} & auxiliary (\sectref{sec:AUX}) \\
{\APPL}/appl. & applicative (\sectref{sec:APP}, \appref{sec:AppendixIII}) \\
{\ATT}  & attributive marker (\sectref{sec:ATT})   \\
ba  & {\itshape ba}-noun class (\sectref{sec:NC}) \\
be & {\itshape be}-noun class (\sectref{sec:NC}) \\
{\BEN} & benefactive (\sectref{sec:CAU}) \\
C & consonant  (\sectref{sec:Consonants}, \sectref{sec:SyllIntStr})  \\
{\CAUS}/caus. & causative (\sectref{sec:CAU}/\appref{sec:AppendixIII}) \\
\textsc{cf} & citation form  (\sectref{sec:ToneLower})  \\
cl. & agreement class (\sectref{sec:AGR})    \\
{\COM} & comitative marker (\sectref{sec:COM}) \\
{\COMP} & complement clause (\sectref{sec:Compna}) \\
{\COMPL} & absolute completive (\sectref{sec:COMPL}) \\
{\COND} & conditional clause (\sectref{sec:Cond}) \\
{\CONJ} & conjunction (\sectref{sec:CONJ}, \sectref{sec:Coord}) \\
{\CONTR} & contrastive marker (\sectref{sec:CONTRS})\\
\end{tabularx}

\noindent\begin{tabularx}{.5\textwidth}[t]{@{}lQ}
{\COP} & \textsc{stamp} copula (\sectref{sec:COP}) \\
%CON  connective morpheme    \\
{\DEM} &  demonstrative (\sectref{sec:DEM})    \\
{\DIST} &  distal  (\sectref{sec:DEM})  \\
\textsc{\do} &  direct object (\sectref{sec:HLinker})   \\
{\EXCL} & exclamation (\sectref{sec:EXCL}) \\
{\EXP} & verb expansion (\sectref{sec:DiaEx}) \\
{\EXT} & verb extension (\sectref{sec:EXtp})  \\
\textsubscript{F} & female name (\sectref{sec:properN})\\
{\FOC} &  focus (\sectref{sec:IS})   \\
fric.  &  fricatives (\sectref{sec:Realization})   \\
{\FUT} &  future (\sectref{sec:fut})   \\
{\GEN} &  genitive marker (\sectref{sec:GEN})   \\
H  & high tone (\sectref{sec:Tinventory})   \\
{\HAB} & habitual (\sectref{sec:HAB}) \\
HL  & falling contour tone (\sectref{sec:Tinventory})    \\
\textsc{hort} & cohortative (\sectref{sec:imp}) \\
\HTS & high tone spreading (\sectref{sec:Trules}) \\
{\ID} & identificational marker (\sectref{sec:ID}) \\
{\IDEO} & ideophone (\sectref{sec:IDEO}) \\
{\IMP} &   imperative (\sectref{sec:imp})   \\
{\INCH} &  inchoative  (\sectref{sec:inch})  \\
{\INF} & infinitival clause (\sectref{sec:InfSub}) \\
{\INSTR} & instrumental (\sectref{sec:CAU})\\
{\IRR} & irrealis (\sectref{sec:SynH})\\
\textsc{interr} & interrogative (\sectref{sec:INTERRPRO}, \sectref{sec:ConstituentQ}) \\
inv. & invariable (\appref{sec:AppendixIII}) \\
\textsc{io} &  indirect object  (\sectref{sec:HLinker})  \\
L &  low tone  (\sectref{sec:Tinventory})  \\
lab. &  labialized (\sectref{sec:Realization})   \\
lat.\ approx. &  lateral approximants (\sectref{sec:Realization})   \\
le & {\itshape le}-noun class  (\sectref{sec:NC})\\
-\textsc{length} & pragmatic lengthening (\appref{sec:AppendixII})\\ 
\end{tabularx}\begin{tabularx}{.5\textwidth}[t]{lQ@{}}
LH &  raising contour tone (\sectref{sec:Tinventory})   \\
{\LOC} &  locative  (\sectref{sec:LOCe})  \\
\textsubscript{m} & male name (\sectref{sec:properN})\\
ma & {\itshape ma}-noun class (\sectref{sec:NC}) \\
mi & {\itshape mi}-noun class (\sectref{sec:NC}) \\
{\MOD} & nominal modifier (\sectref{sec:MODAgrPre}) \\
N  & nasal; {\itshape N}-noun class (\sectref{sec:NC}) \\
n.  & noun (\sectref{sec:N})  \\
NC & nasal + consonant (\sectref{sec:Prenasa}) \\
{\NCA} & non-complete accomplishment (\sectref{sec:ComplSemi}) \\
n.cl.  & noun class  (\sectref{sec:NC}) \\
{\NEG}  & negation (\sectref{sec:NEGPRES}, \sectref{sec:NEGPST}, \sectref{sec:NEGFUT}, \sectref{sec:NEGti}, \sectref{sec:NEGduu})    \\
{\NOM}  & nominalization (\sectref{sec:MorphType})   \\
NP  & noun phrase (\sectref{sec:NP})  \\
npp.  & nominalized past participle (\sectref{sec:NOMPart})  \\
{\NUM}/num. & numeral (\sectref{sec:NAdjuncts})/(\appref{sec:AppendixIII})\\
O  & onset (\sectref{sec:Phonotact})  \\
{\OBJ} & object (pronoun) (\sectref{sec:OBJPRO}, \sectref{sec:HLinker}) \\
{\OBJ}.{\LINK} & object linking H tone (\sectref{sec:HLinker}) \\
{\OBL} & oblique (\sectref{sec:OBL}) \\
obstr. & obstruents (\sectref{sec:Realization}) \\
\textsc{ord} & ordinal numeral (\sectref{sec:Ord}) \\
pal. & palatalized (\sectref{sec:Realization}) \\
pass. & passive (\sectref{sec:PASS}) \\
\textsc{pcf} & predicate focus (\sectref{sec:IS}) \\
{\PL}  & plural marker  (\sectref{sec:PRO}, \sectref{sec:imp})  \\
pl. & plural (\appref{sec:AppendixIII}) \\
plos. & plosives (\sectref{sec:Realization}) \\
{\PN}  & proper name  (\sectref{sec:properN})  \\
\end{tabularx}

\noindent\begin{tabularx}{.5\textwidth}[t]{@{}lQ}
{\pOS} & part of speech (\sectref{sec:POS}) \\
posit. & positional (\sectref{sec:PosVerbs}) \\
{\POSS} & possessor pronoun (\sectref{sec:POSS})    \\
\textsc{pred} & predicate (\chapref{sec:TAM}, \sectref{sec:0COP}) \\
pren. & prenasalized (\sectref{sec:Realization}) \\
{\PREP} & preposition (\sectref{sec:PREP}) \\
{\PRF} & perfect (\sectref{sec:PSTPRF}) \\
{\PRIOR} & priorative (\sectref{sec:ComplSemi})  \\
{\PRO} & pronoun  (\sectref{sec:PRO})  \\
{\PROG} & progressive  (\sectref{sec:PROG})  \\
{\PROSP} & prospective (\sectref{sec:PROSP}) \\
{\PROX} & proximal  (\sectref{sec:DEM})  \\
\textsc{prs} & present  (\sectref{sec:pres})  \\
{\PST}1  & recent past (\sectref{sec:pst1})  \\
{\PST}2  & remote past (\sectref{sec:pst2})   \\
{\Q}  & question marker  (\sectref{sec:PolarQ})  \\
{\QI} & quotative index (\sectref{sec:RD}) \\
{\Q}(tag) & question tag (\sectref{sec:LeadingQ}) \\
{\QUANT} & quantifier (\sectref{sec:NAdjuncts}) \\
{\R} & realis mood (\sectref{sec:SynH}) \\
{\RD} & reported discourse (\sectref{sec:RD}) \\
recip. & reciprocal (\sectref{sec:REC}) \\
{\REL} & relative clause (\sectref{sec:Relativeclauses}) \\
\end{tabularx}\begin{tabularx}{.5\textwidth}[t]{lQ@{}}\relax
{\RETRO} & retrospective  (\sectref{sec:RETROaspect})\\
\textsc{sg}  & singular    \\
{\SBJ} & subject (pronoun) (\sectref{sec:SBJPRO}, \sectref{sec:SBJ}) \\
{\SBJV}  & subjunctive  (\sectref{sec:opt})  \\
{\SEQU} & sequential marker \\
\textsc{sg} & singular (\sectref{sec:PRO}) \\
sg. & singular (\appref{sec:AppendixIII}) \\
{\SIM} & similative (\sectref{sec:SIMword}) \\
\textsc{stamp} & subject-tense-aspect-mood-polarity clitic (\sectref{sec:SCOP}) \\
stat. & stative (\appref{sec:AppendixIII}) \\
{\SUB} & subordinate (\sectref{sec:Sub}, \sectref{sec:PROG}) \\
\textsc{tbu}  & tone bearing unit (\sectref{sec:Tonology})   \\
{\TM} & tense-mood (\sectref{sec:SimpPred})\\
{\TOP} & topic (\sectref{sec:IS}) \\
\textsc{trans} & transnumeral (\sectref{sec:Gender}) \\
V & vowel (\sectref{sec:Vowels})  \\
v. & verb (\sectref{sec:V})  \\
v.i. & verb, intransitive (\sectref{sec:CPhon})  \\
{\VOC} & vocative (\sectref{sec:VOCSuff}) \\
v.t. & verb, transitive  (\sectref{sec:CPhon}) \\
%VOT & voice onset time \\
X & oblique (\sectref{sec:OBL}) \\
\end{tabularx}
%       1 AB
%       1 ABOUT
%       1 ADVERB
%       1 ANALYSIS
%       1 AS
%       1 AT
%       1 ATE
%       1 AUTOCAUS
%       1 BEAT
%       1 BEN
%       1 BORROWING
%       1 BUT
%       1 CATEGORY
%       1 CF
%       1 CFA
%       1 CHAPTER
%       1 CLAUSE
%       1 COMPLEMENT
%       1 CON
%       1 CONSTRUCTIONS
%       1 DEIC
%       1 DEPENDENT
%       1 DRC
%       1 EXPLICIT
%       1 EXTENDED
%       1 FCFA
%       1 FOC
%       1 FRAMING
%       1 GIVE
%       1 GL
%       1 GO
%       1 GYELI
%       1 HAB
%       1 HAVE
%       1 HEAD
%       1 HORT
%       1 IAV
%       1 INCHOM
%       1 INFO
%       1 INHERENT
%       1 LANGUAGE
%       1 LL
%       1 MAARTEN
%       1 MAMBI
%       1 MANGOES
%       1 MB
%       1 ND
%       1 NEEDS
%       1 NG
%       1 NGO
%       1 NPG
%       1 OF
%       1 ONE
%       1 OR
%       1 ORANGES
%       1 PART
%       1 PHONOLOGY
%       1 PREP
%       1 PRES
%       1 PRG
%       1 PROX
%       1 QUAL
%       1 QUOT
%       1 RELATOR
%       1 REPETITIVE
%       1 RICE
%       1 SENSIBLE
%       1 SEPARATE
%       1 SITUATION
%       1 STAMP
%       1 STATE
%       1 SUGGESTION
%       1 SVO
%       1 THIS
%       1 TMA
%       1 TOP
%       1 TS
%       1 UN
%       1 VC
%       1 VERB
%       1 VERSION
%       1 VI
%       1 VN
%       1 WH
%       1 WHICH
%       1 WHO
%       1 WHOM
%       1 WHY
%       1 WITH
%       1 XX
%       1 XXXXX
%       2 ADD
%       2 AL
%       2 AN
%       2 APPL
%       2 BE
%       2 BL
%       2 CAT
%       2 CONM
%       2 DEN
%       2 DON
%       2 ECXL
%       2 EXTEND
%       2 FILLED
%       2 GCS
%       2 HTSM
%       2 HTSN
%       2 HTSV
%       2 IDRS
%       2 III
%       2 IMPPL
%       2 IMPSG
%       2 INAL
%       2 INTERR
%       2 IRS
%       2 IS
%       2 LD
%       2 LMN
%       2 LV
%       2 MID
%       2 NEGIMPPL
%       2 NEGIMPPLOBJ
%       2 NEGIMPSG
%       2 NEGIMPSGOBJ
%       2 NPP
%       2 PCF
%       2 PHO
%       2 PUT
%       2 QCOM
%       2 QDO
%       2 QIO
%       2 QLOC
%       2 RC
%       2 SBJSCOP
%       2 SIL
%       2 STEM
%       2 TAM
%       2 THEM
%       2 TON
%       2 VL
%       2 WALS
%       2 WHEN
%       2 WHERE
