\chapter{The Stative/Non-stative distinction and
  change as a lexico-semantic concept}\label{sec:4}\label{ch:4}\is{Stative/Non-stative distinction|(}

\section{Introduction}\label{sec:4.0}

In this chapter I will discuss Change as the abstract semantic concept
associated with the notions of \isi{Stativity} [\textminus Change] and Non-stativity [+Change].  Even though this concept of Change is not generally
formalised in the literature on \isi{Aspect}, discussions more often than
not are pervaded by this concept as a way to distinguish between two
main classes of verbs – \isi{Stative}\slash Non-stative.  \citet{Verkuyl1993,Verkuyl1999},
may be accredited with positing [+/\textminus Change] as a feature
distinguishing verbs based on the \isi{Stative}\slash Non-stative opposition.
However, many authors including \citet{Vendler1967, Comrie1976, Mourelatos1981, Jackendoff1996} and
\citet{Krifka1998} have made reference to this notion of Change as a
basic concept associated with situations.  In general, the discussion
has not surrounded whether or not there are verbs that express Change
and those which do not but rather the complexity of Change, especially
in the context of the compositionality of \isi{Aspect}.  Thus, for example
authors such as \textcite{Verkuyl1993, Verkuyl1999, Tenny1994, Jackendoff1996} 
and \citet{Krifka1998,Krifka2001} while not principally
focused on the verb itself, are concerned with how Change in the verb
interacts with other elements to impact \isi{Telicity}.

In my opinion, the case of dual aspectual items in CECs and the
discussion of them that has developed places focus on the verb itself
and the applicability of what may be seen as a distinction that
captures a basic intuition -- the \isi{Stative}\slash Non-stative distinction.  Here, I will attempt to address the question of the unique aspectual
contribution of verbs\footnote{The items under discussion function
  dually as adjectives and verbs, thus using the label verbs here is
  not intended to imply a different categorisation.}  to \isi{Aspect}
despite items which appear to express dual aspects.  As indicated
(\chapref{ch:1}), the idea of an aspectual value associated with verbs
must be taken within the context of the compositionality of \isi{Aspect},
and in the approach that I take both concepts are seen as compatible.
Essentially, if the verb is accepted as a part of this composition, it
seems reasonable to assume that there is a basic value associated with
this element even if it may be impacted and modified through
interaction with other elements in composition.

In the approach that I take here, the notion of \isi{inherent aspect} is
reduced to the concept of Change, with its origin in the
lexico-semantic representation of verbs.  Despite what seems to be a
simplification of \citegen{Vendler1967} four classes of verbs (i.e.,
\isi{Activity}, \isi{Accomplishment}, \isi{Achievement} and \isi{State}), the concept of
Change itself is shown below to be quite complex.  Change is taken to
be composed of different (combinations of) primitives consistent with
the contrasts observed in the behaviour of different verbs in
transitivity alternations as highlighted in the work of
\citet{Levin1993}.  In the exploration of the semantic concept of
Change that I undertake here, I identify primitives such as \CAUSE and
\DO, and also \BECOME (see \citealt{McCawley1968,Carter1976, Dowty1979,})
as those relevant in the discussion of \isi{dual aspectual forms} in CECs.

As I will outline in my analysis of \isi{dual aspectual forms} in \chapref{ch:5}, 
the presence of a primitive such as \CAUSE (in conjunction with
\BECOME) is responsible for the presence of a Cause as seen in the
transitive (Non-stative) expression of JC\il{Jamaican Creole} items such as \textit{raip}
`make ripe',  \textit{sik} `make sick', \textit{redi} `make ready', etc.
I argue, however, that it is the introduction of either \CAUSE + \BECOME
or \BECOME in the Event Structure of items such as \textit{blak}
`black' that accounts for the transitive and inchoative realisation of
such items.  In a similar way, \DO accounts for the introduction of
Agency in the Non-stative use of items such as \textit{jelas} `act
jealously' or \textit{bad} `misbehave'.  Seminal works concerned with
verb meanings and primitive structures will be a point of focus in the
discussion that ensues.  Thus, while I assume, like contemporary
authors such as \citet{Verkuyl1999,Tenny1994,TennyPustejovsky2000,Rothstein2004,MacDonald2008,} 
etc., that \isi{Aspect} is compositional (see \chapref{ch:1} for discussion),
the earlier works of \citet{McCawley1968,Carter1976,Dowty1979,Pustejovsky1988,Pustejovsky1991} 
and \citet{Grimshaw1990} will be the ones surveyed here.

Significantly, these earlier works focused on the nature of verb
meaning,\linebreak whereas the more contemporary authors assumed a basic
contribution of the verb in the composition of \isi{Aspect} without focusing
on the nature of this element per se.  A bias towards these
earlier works is in line with my focus on the verb and interest in
accounting for the different uses in which property items appear.  The
more contemporary among these earlier works
(\citealt{Pustejovsky1988,Pustejovsky1991} and \citealt{Grimshaw1990}) will
provide a model of Event Structure that accounts for the ability of
(inherently) \isi{dual aspectual forms} to appear in the uses that they do
and with their associated meanings.  The earlier decompositions of
\citet{McCawley1968,Carter1976} and \citet{Dowty1979}
elucidate the significance of primitives and thus supply a background
for the identification of the specific type of primitive meanings that
are applicable in the Non-stative contrasts among property items.
This is particularly relevant in accounting for the semantic
difference between inherent (Non-stative) dual aspectual items and
those whose \isi{dual aspectual behaviour} results from morphological
derivation.

The chapter is organised as follows: In \sectref{sec:4.1} I will
outline the basic notion of an inherent aspectual contribution of the
verb to \isi{Aspect} in the form of Comrie’s \isi{State}\slash Non-state distinction.
In \sectref{sec:4.2}, I will explore lexico-semantic structures
representing Change and the significance of this notion by looking at
the different types and structures associated with events
(\sectref{sec:4.2.1}), the primary primitives of Change
(\sectref{sec:4.2.2}) and transitivity alternations
(\sectref{sec:4.2.3}).  The discussion coming out of this chapter will
form the basis for the analysis of \isi{dual aspectual forms} in CECs in
\chapref{sec:5}.

\section{The stative/non-stative distinction and the notion of
  Change}\label{sec:4.1} %4.1

\citet{Comrie1976,} in his discussion of \isi{Aspect}, contrasts the verb
\textit{know} with \textit{run} to highlight the Static\slash Dynamic (also
\isi{State}\slash Non-state) distinction. He points to a verb such as \textit{know} as not involving Change, in contrast to \textit{run} which involves ``necessary
Change" (p.~49).  Regarding the \isi{State}\slash Non-state distinction, Comrie
comments that it is:

\begin{quote}
one that seems reasonably clear intuitively, and in practice one finds
a large measure of agreement between individuals who are asked to
classify situations as static or dynamic. (p.~48)
\end{quote}


A key term in his description of this opposition is the term ``phase"
which allows one to look into the situation as it relates to the
notion of Change.  Essentially, in the case of a verb such as \textit{know}
(\isi{State}), he points out that

\begin{quote}
  all phases of the situation \textit{John knows where I live} are
  identical; whichever point of time we choose to cut in on the
  situation of John’s knowledge, we shall find exactly the same
  situation. (p.~49)
\end{quote}

In the case of the verb \textit{run} (dynamic) however, he points out that

\begin{quote}
this is not so: if we say \textit{John is running,} then different
phases of the situation will be very different: at one moment John
will have one foot on the ground, at another moment neither foot will
be on the ground and so on. Thus \textit{know} on the one hand
involves no Change whereas \textit{run} involves necessarily Change. (p.~49)
\end{quote}

While there are situations captured in this opposition that may not be
so straightforward, the key element here is the notion of ``necessary
Change." As Comrie points out, while there may be \isi{Stative} situations
that \textit{may} involve\linebreak Change, dynamic situations involve
``necessary Change" (p.~49). I take this to refer to the inherent
meaning components of the verb.  It is along these lines that I now
explore the notion of Change as an abstract semantic concept in these
sections.

The intuition regarding the \isi{State}\slash Non-state opposition as viewed by
Comrie is that this distinction captures the inherent meaning of the
verb rather than that of the VP (the verb and \isi{internal argument}).
Also, differences in the semantic content of the \isi{internal argument}
(see discussion in \chapref{ch:1}) will not\slash cannot trigger a change in
the nature of the verb itself from indicating Change to not doing so.
I provide the examples below showing the verbs \textit{know} and \textit{run} as
consistently expressing [\textminus Change] and [+Change]\footnote{This feature
  [+/\textminus Change] is also used by \citet{Verkuyl1999} to capture the
  contribution of the verb to \isi{Aspect} in his compositional model. This
  is noteworthy given the fact that Verkuyl argues against a lexical
  division of verbs.  What this shows is that the idea of a unique
  aspectual contribution of the verb to \isi{Aspect} is not opposed to
  \isi{Aspect} as compositional; it is however a matter of identifying the
  unique contribution of each element and of identifying how the
  different elements interact.} respectively (by default), in spite of
the influence of an \isi{internal argument}:\largerpage

\ea\label{ex:4:1} (personal examples)\footnote{What I posit here are
  default interpretations. I am fully aware of the effect that factors
  such as context and adverbials may have on such default
  interpretations especially in the case of the verb \textit{know} (cf.:
  \textit{John knew immediately!).} Since my intention is to capture
  as far as possible a default (inherent) meaning associated with the
  verb, the possible influence of other factors in these examples is
  not considered.}

\ea John \textbf{knows} (\isi{Stative}) \ex John \textbf{knows} the answer.
(\isi{Stative}) \ex John \textbf{knew} the answer. (\isi{Stative}) \z \z

\ea%2
\label{ex:4:2}
\ea John \textbf{runs} (Non-stative) \ex John \textbf{runs} a mile.
(Non-stative) \ex John \textbf{ran} a mile. (Non-stative) \z \z

The examples in \REF{ex:4:1} show the verb \textit{know} appearing in the
Present \isi{Tense} without an \isi{internal argument} (\ref{ex:4:1}a), with a
specified \isi{internal argument} (\ref{ex:4:1}b) and in \isi{Past} \isi{Tense} with a
specified \isi{internal argument}.  In all these cases, the default
interpretation of the verb \textit{know} may be said to not indicate any kind
of Change.  In contrast, in the case of \textit{run} in \REF{ex:4:2}, the
default interpretation associated with this verb is one that includes
Change specifically in terms of motion.

One difference that we may note among the Non-statives here is that
while (\ref{ex:4:2}a) may be used to refer to a situation with no
specified end time (\isi{Habitual} or Generic in this case \isi{Atelic}), 
both (\ref{ex:4:2}b) and (\ref{ex:4:2}c) refer to situations having a fixed end point and as such are
both \isi{Telic}.  In this sense, using \citegen{Vendler1967} terminology
widely adopted in the literature, (\ref{ex:4:2}a) may be called an
\isi{Activity} while (\ref{ex:4:2}b) and (\ref{ex:4:2}c) are
Accomplishments.  In the usage that I employ here the notions of
\isi{Activity} and \isi{Accomplishment} refer to the aspect established at VP
rather than the denotation of the verb itself.  In this way, the basic
contribution of the verb as indicating Change or not is separated from
any additional influence brought about by the \isi{internal argument},
\isi{grammatical aspect} or other elements.

The notion of Change applied in the sense above may be seen as an
abstract semantic concept in that, while it may not be either
morphologically or syntactically expressed, it has semantic force.
Thus for example, in \REF{ex:4:3} below as opposed to \REF{ex:4:4}
some (physical) Change in terms of the situation conveyed by the verb
must be conceived:

\ea%3
\label{ex:4:3}

(personal examples)\\
\ea John \textbf{runs} that race every year.
\ex John \textbf{eats} mom’s dinner every evening. 
\ex John \textbf{blinks} his left eye every ten minutes .\z \z


\ea%4
\label{ex:4:4}
\ea John \textbf{knows} everyone at school. 
\ex John \textbf{has} a better mentor. \z \z


Consistent with Comrie’s association of ``necessary change" in the
conceptualisation of Non-stative as opposed to \isi{Stative} verbs, we note
here that the meanings associated with the examples in \REF{ex:4:3}
involve ``necessary change" as opposed to those in \REF{ex:4:4} where
no ``necessary change" is expressed.

In order for (\ref{ex:4:3}a) to be true, movement\slash motion must be
accepted to take place upon each occurrence of the event of \textit{running}.
In this case, John is both the \isi{Agent} and the entity that is affected
(\isi{Agent} and Theme as \citet[28]{Pustejovsky1988} notes).  Through Change
instantiated by motion, John is translated from one point to the next.
The race is simply the domain in which the event takes place and it
provides a measure for the event.  In the case of (\ref{ex:4:3}b),
there is a Change that affects the \isi{internal argument} (in this case
\textit{mom’s dinner}) whereby it is changed from its original complete state
by being consumed.  In (\ref{ex:4:3}c) the action of \textit{blinking}
indicates that there is a (momentary) Change of state\is{State!Change of} and a return to
the original \isi{State}.  Thus each situation expressed in \REF{ex:4:3}
involves Change.  In the case of \REF{ex:4:4} however, there is no
``necessary change" for these to be true.  One may be able to imagine a
time before when the statement \textit{John knows everyone at school} or
\textit{John has a better mentor} was not true but the verbs in question
\textit{know} and \textit{have} do not capture a Change from a previous state to the one in question; only the state of affairs as it exists for the period
in question.

So far in this discussion, I have presented the notion of Change as an
abstract semantic notion.  In the section below I will look at the
primitives that are associated with the expression of Change.

\section{Event structures and primitives of change}\label{sec:4.2}

The difference between \isi{Stative} and Non-stative predicates has been
captured in models of lexico-semantic representations through the use
of primitives
(\citealt{McCawley1968,Carter1976,Dowty1979,Jackendoff1996}; etc.) and
Event Structure representations (\citealt{Pustejovsky1988,Pustejovsky1991,Grimshaw1990}).  With regard to the latter, \citet{Pustejovsky1988,Pustejovsky1991} and \citet{Grimshaw1990} identify three event types and a separation between \isi{State} on one hand and Non-state-\isi{Transition} and \isi{Process} on the other.

Overall, States are characterized by the absence of primitives
associated with Change while Non-states are defined by different
(combinations of) primitives associated with Change.  The primary
primitives associated with Change are \BECOME and \CAUSE which signal
inchoative Change and Change through an external Cause respectively.
Other primitives such as \DO (see \citealt{Dowty1979}) and \textsc{go}
\citep{Jackendoff1996} have been discussed in reference to verbs of
Agency and motion (both \isi{Process} verbs). The discussion of transitivity
alternations (\citet{Levin1993} also highlights primitives such as
\MOTION, \CONTACT, and \CHANGEOFSTATE as relevant in accounting for the
behaviour of different types of verbs.

In the sections which follow, I will look at three main areas as it
relates to the discussion of Change as a semantic concept: In
\sectref{sec:4.2.1}, I will look at the three types of structures that
characterise events.  In \sectref{sec:4.2.2}, I will look at the
primary primitives associated with Change and taken to be relevant in
the case of \isi{dual aspectual forms} in CECs.  These, are primarily \CAUSE,
\BECOME and \DO.  I use these concepts in \chapref{ch:5} to account in
particular for the derived Non-stative use of items such as JC\il{Jamaican Creole}
\textit{blak} `black' and \textit{red} `red' in inchoative and
transitive use.  Specifically the introduction of a primitive such as
\BECOME accounts for the inchoative use of these item while \BECOME +
\CAUSE account for their transitive use.  \DO is the primitive that is
relevant in the case of the Processual expression of \textit{jelas}
`jealous' and \textit{bad} `bad',  I summarise this discussion of
Change as a semantic concept by looking at the relevance of Change as
indicated by transitivity alternations in \sectref{sec:4.2.3}.


\subsection{Event types and structures}\label{sec:4.2.1}

Event Structures capture the differences between predicates expressing
Change and those which do not.  Representations such as those of
\citet{Pustejovsky1988,Pustejovsky1991} and \citet{Grimshaw1990} show
a difference between States and Non-states in the structure of
sub-events (e\textsubscript{1} and e\textsubscript{2}).  Non-states
are identified as associated with a first sub-event associated with
Causation and Change while pure States are captured in representation
not associated with Change.  Based on the representations of these
authors, there are three types of Event Structures: \isi{State} on one hand
and \isi{Process}\slash \isi{Activity} and \isi{Transition} on the other.  According to \citet{Pustejovsky1988},

\begin{quote}
the grammar specifies three primitive event-types: \textit{state,
  process, transition.}  A verb is identified as having one of these
event-types associated with it lexically.  Furthermore, all
eventuality-denoting sentences in the language must conform to one of
these templates. (p.~22)
\end{quote}

Seen from this perspective, not only is the notion of \isi{event type}
associated with verbs but also with sentences.  In this regard,
\citet{Pustejovsky1991} notes ``[b]ecause an \isi{event structure} is
recursively defined in the syntax, ``event-type” is also a property of
phrases and sentences.” (p.~55).  This essentially means that Event
Structure is redefined within the context of the interaction of other
aspectual elements within the syntactic domain.  In this way, it takes
into consideration the lexical contribution of the verb as well as the
compositionality of \isi{Aspect}.  Below I will look at the different event
types in turn; these will later be applied to an analysis of CEC dual
aspectual forms in \chapref{ch:5}.

\subsection{State}\label{sec:4.2.2}

A \isi{State} according to \citet{Pustejovsky1991} is a ``single event which
is evaluated relative to no other event” (p.~56).  This is shown in
\figref{ex:4:5}.

% % \ea%5
\begin{figure}
\caption{\isi{State}: A single event evaluated relative to no other event \citep[56]{Pustejovsky1991}\label{ex:4:5}}
\fbox{\parbox{3cm}{\centering
\begin{forest} [S [e]] \end{forest}
}} 
\end{figure}
% % \z

The Event Structure in \figref{ex:4:5} is associated with lexical
items such as \textit{be sick},  \textit{love}, \textit{know} etc. based on
\citet{Pustejovsky1991}.\footnote{I do not necessarily assume the same
  categorisation for similar lexical items in CECs. For example as
  will be seen in \sectref{sec:5.2} the categorisation of an item such
  as \textit{sik} `sick\slash become sick\slash sicken' based on its behaviour is
  that of a \isi{Transition} rather than a \isi{State}.}  As the representation
shows, a \isi{State} (S) is an event (e) that exists as inherently unrelated
to any other event.  This representation in \figref{ex:4:5} may be
compared with \citet{Jackendoff1996} which presents a \isi{State} as a
situation which ``just sits there with no dependence on time -- only a
location in time” (p.~327).  Jackendoff’s representation of a
canonical \isi{State} is thus of a situation in time that is not spatially
bound to time on an axis.  What this means essentially is that there
is no dependence between the time constituent and the structure of the
situation.  His representation is shown in \figref{ex:4:6}.

% % \ea%6
\begin{figure}\caption{\label{ex:4:6}Canonical \isi{State} \citep[327]{Jackendoff1996}}
[\textsubscript{Sit} F(X,Y); [\textsubscript{Time} T] ]

(where ‘Sit’ = Situation) 
\end{figure}
% % \z

Using the conceptual idea of a rotating axis, Jackendoff presents the movement of a situation
in time as related to the Changes in the situation for Non-stative situation (Events).  This
representation shows that the continuation of this situation (\isi{State})
in time is not affected by time as there is no axis binding the two.

Although the representations in Figures~\ref{ex:4:5} and~\ref{ex:4:6} are
schematically different, they capture the basic intuition that I apply
in relation to CEC property items and \isi{dual aspectual forms}, which is
that a \isi{State} is not associated with Change.  In the analysis of CEC
dual aspectual items the concept of \isi{State} will be important in
accounting for items that do not allow for Non-stative use.  It also
extends items which do appear in Non-stative use but do so, as I will
argue, through the introduction of primitives associated with Change
rather than due to their inherent lexical composition.

 
\subsection{Transition}\label{sec:4.2.3}

A \isi{Transition} is one of the two event types which are characterised by
Change.  The Event Structure representing it shows an event that is
evaluated relative to another event, entailing a \isi{State} and a \isi{Process}
(Change of state\is{State!Change of}).  This is shown in \figref{ex:4:7}.

% % \ea%7
\begin{figure}
\caption{Event Structure of \isi{Transition}\label{ex:4:7}}
\fbox{\parbox{5cm}{\centering
\begin{forest}
[{e$_0$\\{\footnotesize ~~~[transition]}} 
      [{e$_1$\\{[\isi{Process}]}}]
      [{e$_2$\\{[\isi{State}]}}]        
]
\end{forest}
}}\end{figure}
% % \z

A \isi{Transition} as shown here is an event (e\textsubscript{0})
constituting two events (e\textsubscript{1}, e\textsubscript{2}) which
are evaluated relative to each other.  The first event
(e\textsubscript{1}) represents a Change of state\is{State!Change of} (\isi{Process}) and the
second event (e\textsubscript{2}) represents the result of this Change
of state (\isi{State}).  A lexical item associated with this Event Structure
based on \citet{Pustejovsky1991} is \textit{close}.

The difference between the \isi{State} that forms a part of a \isi{Transition}
Event Structure and the pure \isi{State} shown in \figref{ex:4:6} is that the
former is linked to a \isi{Process} as shown in \figref{ex:4:7}, whilst a pure
\isi{State} does not encode such a relation.  It is important to note as
well that the \isi{Process} involved in a \isi{Transition} Event Structure is
distinct from that of a pure \isi{Process} in that it has a logical
opposition with which it is inherently linked.  According to
\citet{Pustejovsky1991}, a \isi{Transition} is ``an event identifying a
semantic expression, which is evaluated relative to its opposition”
(p.~56).  The concept of a \isi{Transition} Event
Structure will be used to account for CEC property items which appear
in Non-stative use expressing an opposition between contrarieties.
This notion of opposition will be crucial in differentiating between
derived and inherent Transitions.  I elaborate this in
\sectref{sec:5.2}.

\subsection{Process}\label{sec:4.2.4}

The Event Structure of \isi{Process} represents what may be seen as an
ongoing event characterised by Change.  Unlike a \isi{Transition}, however,
it is not evaluated relative to another event (i.e.,  a logical
result).  According to \citet{Pustejovsky1991} it is identified as ``a
sequence of events identifying the same semantic expression” (p.~56).
The structure representing this is shown in \figref{ex:4:8}:


% % \ea%8
\begin{figure}
\caption{\isi{Process} Event Structure \citep[56]{Pustejovsky1991}\label{ex:4:8}}
\fbox{\parbox{5cm}{\centering
\begin{forest}
[P
  [e$_1$,no edge]
  [~~~...~~~,roof]
  [e$_n$, no edge]
]
\end{forest}
}}\end{figure}
% % \z

Some verbs that are noted as associated with this Event Structure are
\textit{run}, \textit{push}, \textit{drag}, etc. \citep[56]{Pustejovsky1991}.  
Such verbs bear out what \citet{Tenny1994} refers to as the Measuring Out
Constraint (MOC) on direct internal arguments.  The MOC is a
constraint that addresses the interaction between the ``direct internal
argument” and a ``simple verb”. Within this interaction, according to
\citet{Tenny1994},

\begin{quote}
[t]he direct \isi{internal argument} [...] is constrained so that it
undergoes no necessary internal motion or change, unless it is motion
or change which ‘measures out the event’ over time. (p.~11)
\end{quote}

Necessary motion or change as indicated by \citet{Tenny1994} means
that this is ``required by the verb’s meaning” (ibid).  Thus she points
out for the sentence \textit{John ate the apple up} that it


\begin{quote}
describes an event in which the apple is necessarily changed by being
consumed.  John might also be changed by becoming full, but that is
not required in an interpretation of the sentence. John may or may not
become full, but the apple must be consumed. (p. 11--12)
\end{quote}

This constraint which may be associated with a \isi{Process} distinguishes
this Event Structure from that of a \isi{Transition} in that the structure
shows a continuous event but no logical result except that associated
with the progression of the event itself.  Thus, it may be said that
in the process of eating, something is necessarily consumed as the
event progresses but this event is not inherently associated with a
(resultant) \isi{State} outside of this context.  The \isi{event type} of \isi{Process} will be useful in accounting for the derived (Non-stative) use of
items such as \textit{jelas, bad} and \textit{ruud} as I discuss in
\sectref{sec:5.2.3}.

\citet{Jackendoff1996} criticises \citet{Tenny1994} and
\citet{Pustejovsky1991} among others for the ``snapshot"
conceptualisation indicated by the Event Structure representation in
\figref{ex:4:8} and the explanations associated with the notion of
\isi{Process}.  He observes that these authors present a \isi{Process} and in
particular an event of motion as ``a series of snapshots, each which
depicts the object of motion in a different location”
(p. 315\textbf{–}316).  He rejects this view, primarily on the grounds
that it ``misrepresents the essential continuity of events of motion”
(p.~316).  In his approach, he presents a conceptualisation where
``instead of treating motion as a finite sequence of states” it is
presented as ``continuous change over time” (p.~317).  Based on this,
he represents a \isi{Process} event as projected onto axes where


\begin{quote}
there are three axes to consider at once: the point situation is
projected onto a durative event [...] the point in space is projected
onto a path; and the point in time is projected onto a time interval
(p.~321)
\end{quote}

\noindent His representation is shown in \figref{ex:4:9}.

% % \ea%9

\begin{figure}
\caption{{\citegen[322]{Jackendoff1996} Process}\label{ex:4:9}}
% \begin{center}
\begin{tikzpicture}
\node (be)  {\textsubscript{Sit\vphantom{j}} BE};
\node (thing) [right=of be,xshift=-7mm] {([\textsubscript{Thing} X] ,};
\node (0d2) [right=of thing,xshift=-12mm] {[\textsubscript{Space}0d]);};
\node (time) [right=of 0d2,xshift=-7mm] {[\textsubscript{Time\vphantom{j}}0d]};
\node (crosssection) [right=of time] {[cross section]};

\node (0d1) [above = of be,yshift=-12mm] {0d~~~};

\node (1d1) [above = of be] {[1d]$^\alpha$};
\node (1d2) [above = of 0d2] {[1d]$^\alpha$};
\node (1d3) [above = of time] {[1d]$^\alpha$};
\node (spboundaxis) [above = of crosssection] {[sp-bound axis]\footnote{Refers to the structure-preserving (sp-)binding relationship between Situation, Path and Time axes \citep[322]{Jackendoff1996}}}; 


\draw (0d1) -- (1d1);
\draw (0d2) -- (1d2);
\draw (time) -- (1d3);

\draw ([xshift=.5mm]0d1.north)  -- ([xshift=.5mm]1d1.south);
\draw ([xshift=.5mm]0d2.north)  -- ([xshift=.5mm]1d2.south);
\draw ([xshift=.5mm]time.north) -- ([xshift=.5mm]1d3.south);

\path (be.south)    edge   [out=150,in=210,transform canvas={xshift=-5mm}]   (1d1.north);
\path (time.south)  edge   [out=30,in=330,transform canvas={xshift=5mm}]  (1d3.north);
\end{tikzpicture}
% % \end{center}
% \z
\end{figure}

As shown here, a \isi{Process} is presented as three axes (the Situation,
represented by \textsc{be}, Space, and Time) which are joined to each other in
such a way that any progression associated with one, effects
progression in the others.  In such a representation, Jackendoff
points out that ``measuring out is a consequence of the sp-binding of
the path event and time axes” (p.~323).

Jackendoff’s representation serves to capture the intuitions of
authors such as \citet{Pustejovsky1991,Krifka1998},
\citet{Tenny1994,Verkuyl1993} etc., especially as it relates
to \isi{Telicity} effects.  Nevertheless, in my discussion of CEC property
items in \chapref{ch:5}, I will make use of the basic Event Structure
representing a \isi{Process} employed by \citet{Pustejovsky1991} to account
for CEC property items which express Non-stativity consistent with the
meaning ``behave in accordance with X quality".  As I argue in
\sectref{sec:5.2.2}, such items are inherently associated with a \isi{State}
Event Structure, however, they are derived to express a \isi{Process}.  In
this manner, they are distinct from items inherently associated with
the Event Structure of \isi{Process}, which do not also appear in \isi{Stative}
use.

In the sections below, I will look at the different primitives of
Change that may be said to be active in and relevant to the Event
Structures of \isi{Transition} and \isi{Process}.  Note that States are
distinguished by the absence of any primitive associated with Change.

\section{Primitives of Change}\label{sec:4.3}

The discussion in this section will indicate that the systematic
difference in aspectual meaning between the \isi{Stative} and Non-stative
use of \isi{dual aspectual forms} may be linked to the presence or
introduction of particular semantic primitives associated with the
aspectual feature, Change.  This analysis is consistent with the view
that ``words are not unanalyzed atoms but can be decomposed into a set
of recurrent conceptual features or traits” \citep[350]{ChierchiaMcConnell-Ginet1992}.  
This view provides an understanding of the uses of JC\il{Jamaican Creole} \textit{raip}.  
For example, although there is no overt morphological difference between \textit{raip}
`ripe' that expresses a \isi{State} and \textit{raip} `ripe' that denotes a
\isi{Process}, in terms of aspect they denote [\textminus Change] in one instance and
[+Change] in another.  Thus, in the case of (\ref{ex:4:10}a), there is
only an indication of the \isi{State} of ripeness consistent with the
expression of the feature [\textminus Change].  However, in (\ref{ex:4:10}b),
there is an indication of this \isi{State} coming about [+Change].  This is
made explicit through the presence of the \isi{Progressive} \isi{aspect marker}.
Note also that (\ref{ex:4:10}c) shows the presence of a Cause or
\isi{Agent}:

\ea%10
\label{ex:4:10}
\ea
\gll di planten \textbf{raip.}\\
\textsc{art} plantain ripe\\
\glt `The plantain is ripe.'

\ex
\gll di planten \textbf{a} \textbf{raip.} \\
\textsc{art} plantain \textsc{asp} ripe\\
\glt  `The plantain is getting ripe.'\\


\ex
\gll dem \textbf{raip} di planten.\\
\textsc{3pl} ripe \textsc{art} plantain\\
\glt  `They ripen the plantain.' \\
\z \z

What I identify in these instances are different realizations of the
same lexical item based on the elements of meanings that are present
as a part of its conceptual structure at the lexico-semantic level.

Hence, an analysis of an item such as \textit{raip} `ripe' is expected
to reveal elements of meaning consistent with the different
interpretations of this item at the surface level.  A composite of
semantic primitives associated with a particular item provides a basis
for its different expressions.  This, in essence, is consistent with the
view that the behaviour of an item is determined by its meaning
\citep[1]{Levin1993}.  In the sections below, I will look
specifically at the primitives \BECOME, \CAUSE and \DO.  These are the
primitives which I use to account for the expression of Non-stativity
among property items in \chapref{ch:5}.

\subsection{\BECOME and \CAUSE}\label{sec:4.3.1}

The primitives \BECOME and \CAUSE have been used in the literature to
capture an internal Change of state\is{State!Change of} (\BECOME) or a Change of state\is{State!Change of}
brought about by a Cause (\CAUSE + \BECOME).  Both primitives appear in
\citegen{McCawley1968} decomposed representation of the lexical item
\textit{kill}. \BECOME is used to show the relation between the opposition
\ALIVE and \DEAD and the \isi{Transition} between these results in the
meaning associated with the English lexical item \textit{to die}.  The
introduction of the primitive \CAUSE in the same configuration accounts
for the difference in meaning between `die' and `kill' which is one of
Causation.  This is shown below:

\ea
\label{ex:4:11}
`Kill' (adapted from \citealt[73]{McCawley1968})\\
\begin{forest}
[S
  [\CAUSE] [x,before computing xy={s/.average={s}{siblings}}] [S
	    [\BECOME] [S
			[not] [S
				[alive] [x]
				]
			]
	  ]
]
\end{forest}
\z

Based on this representation, the meaning of the verb \textit{kill} is made
to contain the \isi{Stative} meanings `alive' and `dead\slash not alive' as well
as the Non-stative meanings `die', `kill' or `\CAUSE to \BECOME not
alive',  McCawley’s approach to decomposition is rooted in the
\isi{syntax-semantics interface} where it is believed that semantic
regularities whether or not they coincide with actual words in the
lexicon may be encoded in the syntax in terms of grammatical relations
in the expression of certain meanings.  According to
\citet{McCawley1973}, the primitives in his representation are not
features of the sentence but are ``relations between items of content
that figure in the sentence” (p.~344).  In other words, while the
primitives themselves will not appear in the sentence, they are
related to the appearance of items that appear in the sentence.

For example, although \textit{kill} itself would appear as a single word in a 
sentence, its decompositional content shows a relationship with its
argument structure and accounts for the fact that this item appears in
a transitive structure including a Patient as \isi{internal argument} and a
Cause\slash \isi{Agent} in the \isi{external argument} position.  The appearance of this
Cause\slash \isi{Agent} is licensed by the \CAUSE primitive in the conceptual
structure of the lexical item.  This primitive distinguishes the
transitive and Non-stative verb \textit{kill} from words with otherwise
similar meanings associated with `dead' (i.e., not \ALIVE), and the
inchoative `die' (i.e., \BECOME not \ALIVE) which McCawley took to be
part of the composite of the verb \textit{kill}.  Unlike \textit{kill} which has a
\isi{Causative} element of meaning, their lexico-semantic structures are
presented as licensing only a patient argument and this is what we see
reflected in the syntactic domain.

The logical relation between the lexical items corresponding to the
meanings `\CAUSE to \BECOME not \ALIVE' (i.e.: kill), `\BECOME not \ALIVE'
(i.e.: die) and `not \ALIVE' (i.e.: dead) where \textit{kill} logically
entails both the meanings of `die' and `dead', may be shown through
examples such as \REF{ex:4:12}:

\ea%12
\label{ex:4:12}
(personal examples)\\
\ea Mary killed the plants.
\ex The plants are dead [not \textsc{alive}] (because Mary killed them).
\ex The plants died [\BECOME not \textsc{alive}] (because of Mary) 
\z \z

In \REF{ex:4:12}, the example in (\ref{ex:4:12}a) entails both the meanings of (\ref{ex:4:12}b) and
(\ref{ex:4:12}c). Essentially, if (\ref{ex:4:12}a) holds true then (\ref{ex:4:12}b)
and (\ref{ex:4:12}c) are also true.  The configuration in
\REF{ex:4:11} captures this intuition and in this way may be said to
appeal to a sense of semantic logic.

McCawley has, however, been criticised for this kind of decomposition
as later authors point out that the representation in \REF{ex:4:11}
does not coincide with the treatment of ‘kill’ in any of the world’s
languages.  In this regard, \citet{Travis2000} for example, who
accepts the presence of a syntactic head where there is evidence for
this in only one language, points out regarding \textit{kill} that:

\begin{quote}
Since no language […] encodes \textit{kill} with morphological bits
meaning \textsc{cause become not alive}, I believe that syntax has no right
encoding all of these concepts (p.~182)
\end{quote}

From this perspective, McCawley’s representation may be analysed as
misguided.  However, what we can abstract from McCawley is the basic
idea that word meaning may be broken down to reflect generalities and
a connection between forms that are semantically related.  Note also
that current (Minimalist) Syntactic Theory assumes the presence of a
``light verb" equivalent to \CAUSE which licenses the \isi{external argument}
for any verb which takes an \isi{Agent} or Cause as \isi{external argument}.  From
this perspective the basic intuition captured by McCawley may be said
to be vindicated.

McCawley’s use of the primitives \BECOME and \CAUSE to indicate a Change
of state and Causation respectively is also reflected in the work of
\citet{Dowty1979} and \citet{Carter1976}.  As shown below,
\citet{Dowty1979} presents \textit{He sweeps the floor clean} as a
logical relationship between two propositions; [\textit{He sweeps the floor}]
and [\textit{the floor is clean}].  These are joined by the primitives
\CAUSE and \BECOME:

\ea%13
\label{ex:4:13}
\citep[93]{Dowty1979}\\
He sweeps the floor clean.\\\relax
[ [He sweeps the floor] \CAUSE [\BECOME [\textit{the floor is
  clean}] ] ]
\z
\citet{Dowty1979} is similar to \citet{McCawley1968} in his
employment of \CAUSE and \BECOME as primitive notions associated with
Change.  Note however a conceptual separation of Cause from Agency in
this representation which separates the action (\isi{Activity}) [He sweeps
the floor] from both the \CAUSE and the resulting Change of state\is{State!Change of}
[\textit{the floor is clean}].  In \sectref{sec:4.2.2} below, I will
look at his \DO primitive which overtly captures this separation
between Cause and Agency.

\citet{Carter1976} also employs \CAUSE as a primitive in his
representation of an item such as \DARKEN.  Although he does not
overtly present \BECOME in his representation, this may be said to be
captured in his use of \CHANGE as a primitive.  As shown below,
\citet{Carter1976} presents items such as \DARK and \DARKEN as
morphologically related through a \isi{Causative} primitive.  Based on his
representation the English form \textit{darken} is a relation between \CAUSE
and the state \BEDARK where the interaction points to the initiation
of a Change of state\is{State!Change of}:

\ea%14
\label{ex:4:14}
\citegen[6]{Carter1976} representation of \DARKEN\\
\DARKEN: x \CAUSE ( (y \BEDARK) \CHANGE) 
\z

Note here that \CHANGE may be interpreted as a way of capturing the
meaning of \BECOME as the bracketing suggests as well that \CHANGE is
introduced before \CAUSE.  Carter’s representation, different from
McCawley’s, includes the use of \textsc{be} as a primitive associated with a
\isi{State} which is embedded under the Non-stative meanings.  Based on the
apparent differences in these representations, it may be useful to
note a separation of notions associated with Change where \CAUSE and
\BECOME may be seen as subtypes of the articulation of Change.

Note that dark\,>\,darken is a \isi{morphological operation} which changes the
Event Structure -- something which will turn out to be relevant to the
analysis of CEC \isi{dual aspectual forms} in \chapref{ch:5}. I will posit
there that CEC forms indicating Colour may be distinguished among
property items by being open to a \isi{morphological process} similar to the
one which Carter captures here.

\subsection{\DO}\label{sec:4.3.2}

\DO appears in the literature as a primitive which denotes Agency.  I
discuss it here as a primitive relevant to the case of those property
items which, in derived Non-stative use, express a simple \isi{Process} or
\isi{Activity}, as will be discussed in \sectref{sec:5.2}. \citet{Dowty1979}
points to this primitive in his overall discussion to account for the
notion of volition (Agency) that distinguishes ``actives” such as
\textit{listen to} and \textit{watch} from the cognitives
\textit{hear} and \textit{see}. Regarding the meaning associated with
\DO he points out that ``a semantic factor which \DO contributes is
roughly the notion of volition (and\slash or intention), contemporaneous
with the act on the part of the subject”\footnote{The idea associated
  with this notion of ``act on the part of the subject'' is what I
  believe sets this primitive apart from one such as \textsc{go}
  (\citealt{Jackendoff1972, Jackendoff1996}) which also expresses Agency but
  also includes motion.} (p.~114).  This definition is important in
my decision to associate this primitive with the meaning seen in
derived Processes in CECs.

In positing \DO, \citet{Dowty1979} separates it from \CAUSE and \BECOME
on the basis of the lexical productivity of the latter primitives. In
this regard, he states that,

\begin{quote}
the evidence for \DO is less persuasive than that arguing for \CAUSE and
\BECOME, and the role played by \DO in the aspect calculus is less
significant than that played by \CAUSE and \BECOME. There is no
productive word formation process “adding” a \DO to a verb in English
(much less in other languages I know of) as there is the case of \CAUSE
and \BECOME in a large number of languages. (p.~119)
\end{quote}

However, the case of CECs \isi{dual aspectual forms} may provide some
evidence for \DO as a primitive associated with a productive process in
the lexicon. This is based on the behaviour of items such as JC\il{Jamaican Creole}
\textit{jelas} `jealous' and \textit{bad} `bad' which in Non-stative
use express a \isi{Process} (\isi{Activity}) not resulting in a Change of state\is{State!Change of} as
opposed to the causative or inchoative interpretations associated with
the Non-stative interpretations of other items.  In \chapref{ch:5}, I
discuss these in Non-stative use as possibly indicative of the
introduction of the element of meaning \DO. This is elaborated in my
analysis in \sectref{sec:5.2}.

\section{More on Change: Transitivity alternations}\label{sec:4.4}

Transitivity alternations have been discussed in the literature in
relation to the different behaviours displayed by groups of verbs.
The work of \citet{Levin1993} is perhaps the most extensive so far and
it will serve as my point of reference in this section.  My focus here
will be on how the presence or absence of elements of meaning
associated with Change affect whether or not a verb may participate in
a particular alternation.  This I believe provides some tangible
evidence of the syntactic relevance of Change which I have discussed
so far as a semantic concept.  As we will see, transitivity
alternations do not express the \isi{Stative}\slash Non-stative distinction as
these alternations pertain only to Non-stative verbs.  However, they
may be said to point to the syntactic relevance of Change and also to
the complexity of this abstract semantic notion in the way that the
presence of particular elements of meanings allows for the
participation of some verbs as opposed to others in these
alternations.

As I outline in this section, transitivity alternations are determined
based on the composite of primitives that are associated with some
verbs as opposed to others. This is in line with \citegen{Levin1993}
observation that

\begin{quote}
the behaviour of a verb, particularly with respect to the expression
and interpretation of its argument is to a large extent determined by
its meaning. (p.~1)
\end{quote}

\citegen{Jackendoff1975} observation that the first level of adequacy
in language description ``consists in providing each lexical item with
sufficient information to describe its behavior in the language”
(p. 639, cf. \citealt{Chomsky1965}) is pertinent in this regard.

As we will see below, in the discussion of transitivity alternations,
there is a separation among verbs based on the type of Change that
they express.  In the case of the middle and causative\slash inchoative
alternations, the relevant primitives determining the participation of
a verb are \BECOME (and \CAUSE) (discussed in \sectref{sec:4.3.1}).  The
\isi{body-part possessor ascension alternation} highlights the relevance of
the notion of \CONTACT, while it is the combined notions of \MOTION and
\CONTACT that are pertinent in the \isi{conative alternation}. I will discuss
these alternations in turn.


\subsection{\CAUSE and \BECOME in the middle and causative/inchoative
  alternations}\label{sec:4.4.1}

The \isi{middle alternation} allows for a generalisation to be made over the
behaviour of an \isi{internal argument} without the inclusion of a Cause or
\isi{Agent} in the expression.  As shown below in \REF{ex:4:15} the verbs
\textit{cut} and \textit{break} are distinct from others such as \textit{touch} and \textit{hit}
in their ability to appear in the \isi{middle alternation}.  Compare (\ref{ex:4:15}):

\ea\label{ex:4:15} Restrictions on the \isi{middle alternation} \citep[6, example 13]{Levin1993}\\
\ea[]{The bread \textbf{cuts} easily.}
\ex[]{Crystal vases \textbf{break} easily.}
\ex[*]{Cats \textbf{touch} easily.}
\ex[*]{Door frames \textbf{hit} easily.}
\z \z

As seen here, a verb such as \textit{break} or \textit{cut} may appear in the middle
alternation where this is not possible for a verb such as \textit{touch} or
\textit{hit}.  This is due to the presence of the element of meaning \BECOME
which points to a Change of state\is{State!Change of} in the \isi{internal argument} for verbs
such as \textit{cut} and \textit{break} but not \textit{touch} and \textit{hit}.

The causative\slash inchoative alternation, also highlights the syntactic
relevance of \CAUSE and \BECOME as primitives associated with Change.
In this alternation, there is a separation between verbs which denote
a Change of state\is{State!Change of} in an \isi{internal argument} without any implication of
an external Cause or \isi{Agent} and all others.  As \citet{Levin1993}
points out, this variation applies to

\begin{quote}
a pure change of state verb [...] denoting an entity undergoing a change
of state [...] the two argument form of the verb found in the causative
variant is derived by the addition of a notion of a cause. (p. 9--10)
\end{quote}

Thus for example, there is evidence of a separation between verbs such
as \textit{roll}, \textit{close} and \textit{break} as opposed to \textit{cut} in English which
does not permit this alternation.  Compare \REF{ex:4:16}:\largerpage

\ea%16
\label{ex:4:16}
Restrictions on the causative\slash inchoative alternation (my examples)\\
\ea[*]{The cloth \textbf{cut.}\footnote{Though this is not a possibility
  in English, similar structures are salient in varieties of CECs
  e.g.: JC\il{Jamaican Creole} \textit{Di klaat kot.} `The cloth is\slash has been cut'.  What
  this suggests is a difference in the semantic conceptualisation of
  such a form in CECs as opposed to their English lexifier or simply
  the availability of an unmarked passive (subject to constraints
  which are, as yet, unclear; see \citealt{Allsopp1983}).}}
  \ex[]{Mark \textbf{cut} the cloth.} 
  \ex[]{The ball \textbf{rolled}.} 
  \ex[]{The boy \textbf{rolled} the ball.} 
  \ex[]{The door \textbf{closed}.} 
  \ex[]{The boy\slash the wind \textbf{closed} the door.}
  \ex[]{The window \textbf{broke}.} 
  \ex[]{The boy\slash the wind \textbf{broke} the window.} 
\z \z

As noted here, items such as \textit{roll}, \textit{close},  and \textit{break} allow for the
causative\slash \isi{inchoative variation}.  These constitute the class of
unaccusative verbs which may be used Non-statively with no implication
of a Cause or \isi{Agent}.  The semantics of these verbs include the notion
of \BECOME (consistent with a Change of state\is{State!Change of}) without the inherent
involvement of a Cause or \isi{Agent}.  The introduction of a Cause or \isi{Agent}
accounts for the causative variations of these verbs.  The
causative\slash \isi{inchoative variation} is thus restricted to verbs which allow
for a separation between a Cause and a Change of state\is{State!Change of}.  This is not
the case for a verb like \textit{cut} in English, as a Change of state\is{State!Change of} in the
\isi{internal argument} seems to be linked inherently to the action of a
Cause or \isi{Agent}.


\subsection{\CONTACT in the body-part possessor ascension
  alternation}\label{sec:4.4.2}

The \isi{body-part possessor ascension alternation} distinguishes verbs
expressing \CONTACT from all others.  While I do not discuss \CONTACT as
a primitive relevant to the case of CEC property items its effect in
the case of the body-part possessor alternation highlights the basic
intuition underlying the concept of Change -- i.e., the fact that this
concept is one that is quite complex and expressed through a range of
primitive notions and combinations of such primitives.  Observe in the
examples in \REF{ex:4:17} a separation between verbs such as \textit{cut},
\textit{touch} and \textit{hit} as opposed to \textit{break}:

\ea\label{ex:4:17}Restrictions on the body-part possessor ascension
alternation \citep[7]{Levin1993}

\ea[]{Margaret \textbf{cut} Bill’s arm.} 
\ex[]{Margaret \textbf{cut} Bill on the arm.} 
\ex[]{Janet \textbf{broke} Bill’s finger.} 
\ex[*]{Janet broke Bill on the finger.} 
\ex[]{Terry \textbf{touched} Bill’s shoulder.} 
\ex[]{Terry \textbf{touched} Bill on the shoulder.} 
\ex[]{Carla \textbf{hit} Bill’s back.}
\ex[]{Carla \textbf{hit} Bill on the back.}
% \setcounter{enumerate}{7}
\z \z

Note that items such as \textit{cut}, \textit{touch} and \textit{hit} allow for a
relationship of contact between an \isi{Agent} (subject) and a body part to
be expressed through use of the preposition \textit{on}.  The resultant
alternation is not permitted in the case of a verb such as \textit{break}.
The difference observed between these items is the presence of the
notion of \CONTACT which is present in all verbs shown in \REF{ex:4:17}
except \textit{break}.  Thus the notion of \CONTACT appears to be one that is
relevant in the context of the expression of Change.

In the section below I will look at the combination of \MOTION +
\CONTACT in the \isi{conative alternation}.  I do this in the same spirit
that I have looked at \CONTACT here.  Note that my intention is not to
exhaustively decompose word meaning.  Rather as it relates to CEC
property items, the aim is to focus on what may be called primary
(aspectual) primitives that point to the presence of Change in
lexico-semantic representation.  Along these lines, both \CONTACT and
\MOTION as primitives will not be discussed beyond their involvement in
transitivity alternations.

\subsection{\MOTION + \CONTACT in the conative alternation}\label{sec:4.4.3}
The \isi{conative alternation} is one where a verb may be used in
conjunction with the preposition \textit{at} to express an attempted but not
(necessarily) achieved action.  This alternation highlights a
distinction between verbs such as \textit{cut} and \textit{hit} as opposed to
\textit{break} and \textit{touch}.  The difference between these verbs lies in the
fact that the former contain the combined meanings \MOTION and \CONTACT
while the latter do not.

\ea\label{ex:4:18} Restrictions on the \isi{conative alternation} \citep[6, example 14]{Levin1993}
\ea[]{Margaret \textbf{cut} at the bread.} 
\ex[]{Carla \textbf{hit} at the door.}
\ex[*]{Janet \textbf{broke} at the bread.} 
\ex[*]{Terry \textbf{touched} at the cat.} 
\z \z

As shown in these examples, the verbs \textit{cut} and \textit{hit} appear in the
\isi{conative alternation} where they are used to express an (attempted)
action without an actual result.  Essentially one part of the meaning
composite (\MOTION) is articulated but the other (\CONTACT) is not
established.  This type of alternation is only available to lexical
items which contain both elements of meanings (\MOTION and
\CONTACT). Thus items such as \textit{break} (\BECOME\slash\CAUSE + \BECOME) and
\textit{touch} (\CONTACT) do not permit this alternation since they do not
contain both these elements of meanings.

\section{Observations}\label{sec:4.5}

The complexity of Change is apparent in the way in which the presence
or absence of particular elements of meaning determine the extent to
which particular Non-stative verbs may be subject to transitivity
alternations.  This indicates that the notion of Change is
decomposable.  This is an important point that must be considered in
relation to CEC property items expressing \isi{dual aspectual behaviour} as
the contrast between the \isi{Stative} and Non-stative realisation of such
items indicates the presence or introduction of specific primitives.

In relation to the general meaning components which serve to express
different types of Change, it is useful to point out that while these
have aspectual ramifications in that they define Change, they are not
strictly speaking aspectual, whereas Change is.  Thus, for example, we
note that any expression of the feature Change, regardless of its
exact composition, predisposes a predicate to a \isi{Telic} (endpoint)
interpretation, provided that the necessary semantic information is
supplied in the \isi{internal argument} or other constituents.  In contrast,
a [\textminus Change] verb leaves a predicate \isi{Atelic} regardless of the nature of
the \isi{internal argument}.  Recall examples (\ref{ex:4:1}--\ref{ex:4:2}), repeated here for
convenience:

\ea%19
\label{ex:4:19}
\ea John \textbf{knows.} (\isi{Stative}) (\isi{Atelic}) \ex John \textbf{knows} the
answer. (\isi{Stative}) (\isi{Atelic}) \ex John \textbf{knew} the answer. (\isi{Stative})
(\isi{Atelic}) \z \z

\ea%20
\label{ex:4:20}
\ea John \textbf{runs.} (Non-stative) (\isi{Atelic}) 
\ex John \textbf{runs} a mile. (Non-stative) (\isi{Telic}) 
\ex John \textbf{ran} a mile (Non-stative) (\isi{Telic}) 
\z \z

All instances of the verb \textit{know} provide an \isi{Atelic} interpretation
while the interpretation associated with the Non-stative \textit{run} varies
between \isi{Telic} and \isi{Atelic} dependent on the nature of the internal
argument.  Such contrasts are consistently seen between \isi{Stative} and
Non-stative verbs.  We note additionally that all the examples in
\REF{ex:4:19} are \isi{Atelic} although (\ref{ex:4:19}b) and (\ref{ex:4:19}c) contain internal arguments that may be regarded as finite\footnote{\citet{Verkuyl1993} refers to
the feature contributed by the \isi{internal argument} as based on
finiteness. This is similar to the feature ``Specified Quantity of A"
[SQA] that has been used in reference to the contribution of the
verbal arguments to \isi{Telicity} (see  \citealt{Tenny1994,MacDonald2008}).} 
-- fitting the aspectual feature requirement of the \isi{internal argument} for \isi{Telicity}.  This is due to the fact that the Non-stative verb \textit{run} in \REF{ex:4:19}
contains the feature [+Change] while the \isi{Stative} \textit{know} in
\REF{ex:4:20} is [\textminus Change].  The pertinent aspectual difference
between the verbs in these sentences is not the type of Change that
they indicate but whether or not they indicate Change.

Nevertheless, it is a look at the different types of Change and the
decomposability of this concept that has provided insights into the
primitives that are involved.  In particular, seminal works in the
area of verb meaning reveal two types of Event Structures which
express Change and one which does not.  These are \isi{Transition} and
\isi{Process} on one hand and \isi{State} on the other.  In the case of a
\isi{Transition} Event Structure, we have seen that this is characterised by
primitives of Change such as \CAUSE and \BECOME consistent with its
expression of a Change of state\is{State!Change of}. While in the case of a \isi{Process} Event
Structure, the relevant element of meaning seems to be consistent with
\DO which conceptually separates Agency from Cause.  A \isi{State} Event
Structure may be identified based on the absence of any element of
meaning associated with Change.

In \chapref{ch:5}, I will provide an analysis of \isi{dual aspectual forms} in
CECs that makes use of the concept of Change, its decomposability and
its presence at the level of Event Structure.  We will see that
primitives such as \CAUSE, \BECOME and \DO may be called upon to account
for the systematic \isi{Stativity}\slash Non-stativity contrast that is observed
in property items.  Recall also, though, that these contrasts are not
the same for all property items; hence, different classes will be
distinguished and accounted for by appealing to differences in Event
Structures and the morphological operations that may be performed on
them.

It will become apparent that the existence of items displaying dual
aspectual behaviour in CECs does not disprove the basic intuition that
underlies the \isi{Stative}\slash Non-stative distinction and \isi{inherent aspect}
despite arguments that have been made in Creole studies to this effect
(cf. \citealt{Jaganauth1987}).
\is{Stative/Non-stative distinction|)}
