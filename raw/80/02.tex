\chapter{Aspect in Caribbean English Creoles: An overview of
  works}\label{sec:2}\label{ch:2}
\section{Background}\label{sec:2.0}

\isi{Aspect} in Caribbean English Creoles (CECs) has been discussed in the
context of descriptive studies focusing mainly on \isi{grammatical aspect}
markers and the interaction between these and different types of
verbs.  In this chapter, I will take a look at some of the work that
has been undertaken on \isi{Aspect} in CECs.  Due to the plethora of work in
the general area of Tense-\isi{Aspect} in Creoles, it is hardly possible to
mention all the works that might be relevant.  However, the ones that
I highlight in this discussion are those that may be credited with
somehow having advanced the study of \isi{Aspect} in CECs along the logical
line of progression that I outline here.

The general progression of works may be said to have moved from being
almost strictly concerned with \isi{grammatical aspect} (see
\citealt{Alleyne1980,Voorhoeve1957}) to the significance of inherent
aspect (see \citealt{Bickerton1975,Bickerton1981}).  Later on, questions about
the validity of Bickerton’s \isi{Stative}\slash Non-stative distinction and its
relevance to Creole studies (see
\citealt{Jaganauth1987}\footnote{There are many authors who have found
  fault with Bickerton’s approach but have not focused so clearly on
  refuting the stative\slash non-stative distinction.}) make way for more
contemporary trends in analyses which are sensitive to the impact of
various elements including \isi{inherent aspect}, \isi{grammatical aspect}, and
also the issue of context (see \citealt{Gooden2008,Sidnell2002,Winford1993,
Winford1997,Winford2000}, etc.).

I focus on these works in this chapter to create a general background
and context for the specific discussion that will be the primary
concern of this work.  As indicated in \chapref{ch:1}, this is the case
of lexical items which display aspectual multi-functionality (dual
aspectual forms) in that they may be used to express either \isi{Stativity}
or Non-stativity.  The issues surrounding these items in my estimation
are best reflected in the works of \citet{Alleyne1980},
\citet{Jaganauth1987}, and \citet{Winford1993} which address the
\isi{aspectual status} of such items and also \citet{Sebba1986},
\citet{Seuren1986} and \citet{Kouwenberg1996} which focus on the
\isi{categorial status} of these items.  I focus specifically on the
discussion in these works in \chapref{ch:3}.  However, since these
issues and the case of \isi{dual aspectual forms} fall within the larger
context of the discussion of \isi{Aspect} in CECs, I will spend some time
here looking at some of the principal works that have been undertaken
in \isi{Aspect} as a general area in CECs.

A review of the works indicated here will point to the general problem
that I tackle in this work, i.e.\ the issue of the \isi{Stative}\slash Non-stative
distinction and its application in Creole studies.  As we will see,
based on current discussions in the field, it seems that most authors
are willing to accept, to varying degrees, the impact of \isi{Stativity} on
temporal interpretation (see
\citealt{Gooden2008,Sidnell2002,Winford1993,Winford2000}).  Nevertheless,
the case of items that appear in both \isi{Stative} and Non-stative use (see
discussion of \citealt{Jaganauth1987} below) makes it difficult to commit to
whether or not the feature \isi{Stativity} is to be applied to the entire
predicate or only to the head of the predicate, the predicator.
Incidentally, this is true even of \citet{Bickerton1975}, who
unexpectedly concludes that \isi{Stativity} is to be applied to propositions
after seemingly arguing for the distinction at the level of the verb
(p.~30).  Conversely, \citet{Gooden2008} states that the feature
\isi{Stativity} is to be applied to the entire predicate, however, her tests for
\isi{Stativity} are applied to the verbs themselves effectively testing only
\isi{inherent aspect}.\footnote{Although \citet{Gooden2008} proposes the
  treatment of the \isi{Stative}\slash Non-stative distinction as ``a feature of
  the \isi{lexical aspect} of the verb” (p.~315) her discussion effectively
  conflates this with \isi{inner aspect} which includes the internal
  argument of the verb (cf. \citealt{Verkuyl1999}).}

In the sections below I will highlight the main concerns of relevant
works starting with \citet{Voorhoeve1957}.

\section{Some contributions to the study of Aspect in
  CECs}\label{sec:2.1}
\subsection{\citet{Voorhoeve1957}}\label{sec:2.1.1}

\citet{Voorhoeve1957} is one of the earliest works on \isi{Tense} and \isi{Aspect}
in Creole studies.  Its main concern is typical of many later works in
the field in its focus on the semantic content of certain
“prefixes”\footnote{``Prefixes'' here refers to grammatical Tense-\isi{Aspect}
  markers, generally referred to in later works as preverbal
  particles} occurring with verbal forms in \ili{Sranan} (SR\il{Sranan}).  He departs
from what he calls the ``translation” of SR\il{Sranan} into western \isi{Tense}
categories (p.~374) and was perhaps the first to note that forms such
as \textit{nati} `to be wet' \textit{hebi} `to be heavy',
\textit{siki}, `to be sick', \textit{kba} `to be ready', \textit{nen}
`to be named', etc., are verbs (rather than adjectives) in SR\il{Sranan}.  This he
observes based on their ability to ``combine with prefixes and to link
\dots\ with other words” (p.~377).  Highlighting this, he points out:

\begin{quote}
Many of these verbs are translated into European languages by means of
adjectives and this is the reason why in the existing grammars of
\ili{Sranan} they are wrongly considered as adjectives. (p. 376--377).
\end{quote}

He points out further, regarding the behaviour of a form such as
\textit{siki}, that this can be ``a noun (meaning: sickness), an
adjective (meaning: sick), an intransitive verb (meaning: to be sick),
and a transitive verb (meaning: to make ill)” (p.~377).  Voorhoeve’s
observation of this phenomenon was perhaps the base for later
discussions on the status of such items.  Regarding these, I will note
in \chapref{ch:3} that that discussion has taken two forms, namely the
question of the status of these as either verbs or adjectives, and the
question of (inherent) \isi{Stativity}.  In this work, I address both
questions pointing out that whereas a number of authors have focused
on the \isi{categorial status} of such items, a discussion in terms of
\isi{aspectual status} provides an informed basis on which to resolve the
issues, and should therefore be the starting-point.

In terms of \isi{grammatical aspect}, Voorhoeve identifies the prefix
\textit{e-} as a marker of aspect in SR\il{Sranan} and determines this as ``the
indicator of the \textit{non-completive aspect}” (p.~378).  This
marker is juxtaposed against the ``\isi{unprefixed form}” or what is known as
the bare or \isi{unmarked verb} which, according to him ``indicates
\textit{the completive aspect”} (p.~378).  The data he provides shows
forms occurring both in the bare (unprefixed) form and prefixed by the
aspectual marker \textit{e-} (\tabref{tab:srananverbs}).


\begin{table}[t]
  \caption{Verbs in bare form and prefixed by \textit{e-} in Sranan
    \citep[377--378]{Voorhoeve1957}}
  \label{tab:srananverbs}
  \begin{tabularx}{\textwidth}{QQ}
    \lsptoprule

    Bare verb & Prefixed by preverbal \textit{e-}\\
    \midrule \textit{a dede}\newline he is-dead & \textit{a}
    \textbf{\textit{e- dede}}\newline
    he dies \\

    \tablevspace \textit{a santi} \textbf{\textit{kba}}\newline the
    sand is-ready & \textit{a santi} \textbf{\textit{e-kba}}\newline
    the sand gets-ready\newline
    it has been nearly cleared away (from the         truck)\\

    \tablevspace \textit{a watra} \textbf{\textit{trubu}}\newline the
    water is-troubled & \textit{a watra}
    \textbf{\textit{e-trubu}}\newline
    the water becomes-troubled\\

    \tablevspace \textit{a} \textbf{\textit{dip}}\textit{i
      tumsi}\newline it is-deep too-much & \textit{ a}
    \textbf{\textit{e- dipi}} \textit{tumsi}\newline
    it gets-depth too-much\\

    \tablevspace \textit{mi} \textbf{\textit{wan waka}}\newline I want
    to walk & \textit{m} \textbf{\textit{e-wan}} \textit{waka}\newline
    I want-generally to walk\\

    \tablevspace \textit{i m’a} \textbf{\textit{kan}}
    \textbf{\textit{was}} \textit{a kros dj i} \newline your mother
    can wash the clothes for you & \textit{i m’a}
    \textbf{\textit{e-kan}} \textbf{\textit{e -was} }\textit{a kros dj
      i}\newline
    your mother can-all-the-time continue-to-wash the clothes for you\\

    \tablevspace \textit{i no} \textbf{\textit{sab}} \textit{j a mon
      e-du kon}\newline you not know how the money actually comes &
    \textit{i n} \textbf{\textit{e-sab}} \textit{j a mon e-du
      kon}\newline
    you not know-all-the time how the money actually    comes\\

    \tablevspace \textit{a} \textbf{\textit{sabi}} \textit{pasi
      kba}\newline he knows the road already & \textit{a}
    \textbf{\textit{e-sabi}} \textit{pasi kba}\newline
    he begins–to-know the road already \\

    \tablevspace \textit{pe Srnaman} \textbf{\textit{de}} \textit{j a
      moro prisiri}\newline where Surinam-people are you have more fun
    & \textit{pe Srnaman} \textbf{\textit{e-de}} \textit{j a moro
      prisiri}\newline
    where Surinam-people are–ever you have more pleasure (fun)\\
    \lspbottomrule
  \end{tabularx}
\end{table}

Having identified the marker \textit{e-} as generally expressing
Non-completive aspect, Voorhoeve observes further, based on data such
as shown here, that this form is an indicator of ``an imperfective, an
iterative, a durative, a progressive and an inchoative aspect.”
(p.~378).

For Voorhoeve, however, ``Non-completive'' serves as an umbrella
term for \isi{Imperfective}, Durative, \isi{Progressive} and \isi{Inchoative}.  He
points out that: ``All these mutually different values have in common
that the action is considered independent of the result.” (p.~378).
In the case of unmarked verbs, he points out that ``it is the
\isi{unprefixed form} which reveals the result of the action.”  Hence, the
notion ``Completive'' (p.~378).  Note here that Voorhoeve does
not make allowance for different types of verbs in his approach, but
focuses on the overall grammatical outlook that is indicated through
the interaction between the verb and marker.  Thus for him, the marker
\textit{e-} is associated with different aspectual values based on the
meaning that is denoted in its varying occurrences.

It may be possible, however,  to label the marker \textit{e-} as simply a
marker of Imperfectivity meaning that it focuses on different phases
of the situation \citep{Comrie1976} while the \isi{unmarked verb} indicates
Perfectivity (i.e. focuses on the situation as a whole).  Returning
to the data above, we observe some consistencies in the interpretation
of the marker \textit{e-} with the different verbs\textit{:} We note
that \textit{e-} is consistent with the meaning  `become’ or  `get’
in the case of \textit{dede} `to die', \textit{kba} `ready',
\textit{trubu} `troubled', \textit{dipi} `deep' signaling the
initiation of a Change of state\is{State!Change of}, i.e. an inchoative.  Elsewhere, with
verbs like \textit{wan} ‘want’, \textit{sabi} `know', \textit{de} `to
be' and modals such as \textit{kan} `can' we see a meaning akin to
`generally', `always' or `all the time', consistent with the \isi{Habitual}.
Further, with a verb like \textit{was} `wash', we see the meaning of
`to continue' which is consistent with the \isi{Progressive}.  In spite of
the fact that all these meanings fall under the abstract category of
Non-completive or, after \citet{Comrie1976}, \isi{Imperfective}, a question
that logically arises is whether there is something in the meanings of
these different types of verbs that allows for the differing aspectual
interpretations in each case.

This kind of data indicates that there are at least three different
types of verbs, consistent with the different types of \isi{Imperfective}
meanings that arise.  In particular, items of the type \textit{dede
} `dead', \textit{kba `ready'}, \textit{trubu} `troubled',
\textit{dipi,} `deep' etc. seem to be what I call Change of state\is{State!Change of}
predicates which may be used to express \isi{Stativity} as in the (a)
examples and Non-stativity as in the (b) examples.  In the presence of
Im\isi{perfective aspect} marking a Change of state\is{State!Change of} is overtly indicated
consistent with the meaning \BECOME, a primitive concept associated
with Change in lexico-semantic representations.\footnote{This is
  further elaborated in Chapters~\ref{ch:4} and~\ref{ch:5}.}  I will discuss items
of this type in \chapref{ch:5} but at this stage it suffices to say that
such items seem to behave differently from others, as it relates to
Im\isi{perfective aspect} marking and the meaning that they denote in this
respect.  As we see here in relation to \ili{Sranan}, the meaning indicated
by items such as \textit{wan} `want', \textit{sabi} `know',
\textit{de} `to be', and others such as the modal \textit{kan} `can'
and \textit{was} `wash', is consistent with a \isi{Habitual} and a
\isi{Progressive}, respectively.

In terms of a classification of verbs, Voorhoeve’s approach may be
said to be aimed at unifying the different classes of verbs in that he
does not attempt any clear classification based on aspectual
behaviours.  As it regards verbs such as \textit{nati} `to be wet' as
opposed to \textit{waka} `to walk', however, he notes that the
un-prefixed form of verbs such as \textit{waka} agrees with what he
calls the ``occidental perfect” while \textit{nati} corresponds with
the ``occidental present”.  Regarding such a difference, he points out
that:\largerpage

\begin{quote}
It is understandable that it is the verbs like \textit{nati} (to
be-wet), \textit{nen} (to be-called), \textit{abi} (to have)
\textit{de} (to be), etc., where the \isi{unprefixed form} agrees with an
occidental present, whilst the \isi{unprefixed form} of verbs like
\textit{waka} (to walk) agrees with an occidental perfect.  This is
because the former in our language only indicate a state of being and
possess a completive meaning. (p.~378--379)
\end{quote}

Though rudiments of a distinction between Statives and Non-statives
can be discerned here, Voorhoeve makes no overt reference to universal
inherent aspectual properties.  Instead, he treats the differences in
\isi{Tense} interpretations between a bare verb like \textit{nati} and
\textit{waka} as ``understandable” or something that is natural based
on the language-specific meanings of these items.

Overall, Voorhoeve’s work brought into focus that it is not sufficient
to analyse Creole languages based on English glosses.  In particular
the fact that words such as \textit{nati} `wet', \textit{hebi}
`heavy', \textit{siki} `sick' may be interpereted as shown; constitutes a significant difference in
the way that a Creole such as SR\il{Sranan} treats these items as opposed to
their treatment in European languages.  The aspectual categorization
and status of similar items will be the topic of \chapref{ch:5}.

\subsection{\citet{Alleyne1980}}\label{sec:2.1.2}

Alleyne may be said to adopt a similar approach to
\citet{Voorhoeve1957} in that both strive to unify the different types
of verbs and focus primarily on \isi{grammatical aspect}.
\citet{Alleyne1980} examines data from \ili{Guyanese Creole} (GC\il{Guyanese Creole}), Jamaican
Creole (JC\il{Jamaican Creole}), \ili{Krio}, \ili{Saramaccan} (SM\il{Saramaccan}), \ili{Sranan} (SR\il{Sranan}) and \ili{Gullah} (GU\il{Gullah}) for
peculiarities of particular languages within this group.  With regard
to the expression of \isi{Tense} Mood \isi{Aspect} (TMA) grammatical markers, he
notes that ``[t]he basic structure of the verb phrase is remarkably
uniform across the languages and dialects\footnote{Note that Alleyne
  utilises the term \textit{dialects} in reference to Caribbean Creole
  languages -- a practice which is no longer followed within the field.}
under consideration” (p.~77).  In this regard he points out that:

\begin{quote}
Verb phrases characteristically have particles preposed to the
predicate and by their occurrence, absence or combination express
aspect, tense and mood (imperative and conditional). (p.~80)
\end{quote}

With specific regard to \isi{Aspect}, Alleyne’s work attempts a general
description of \isi{Aspect} through evidence from \isi{grammatical aspect}.  His
basic opposition is between \isi{Perfective} and Non-perfective, where
\isi{Habitual} and \isi{Progressive} are subcategories of Non-perfective.
According to him,

\begin{quote}
aspect is part of the basic structure of the verb phrase in all but
imperative sentences. All dialects have two aspects: perfective and
nonperfective. (p.~82).
\end{quote}

\largerpage
He notes that the \isi{Perfective} is unmarked in all dialects, while the
Non-per\-fec\-tive takes different forms depending on the dialect and its
ancestry (p.~82).  The data he provides includes verbs like
\textit{waak} `walk', \textit{go} `go' and \textit{si} `see' as shown
below in \REF{ex:2:2}:

\ea\label{ex:2:2} {Unmarked verbs as \isi{Perfective} in Creoles
  \citep[82]{Alleyne1980}} \ea
JC\il{Jamaican Creole} \\
\gll Mi waak.\\
\textsc{1sg} walk   \\
\glt `I have walked.'\footnote{Also included: I (always, sometimes)
  walk’ (\isi{Habitual}) ditto for (2b and c) as well.}  \ex
GC\il{Guyanese Creole} \\
\gll Mi waak.\\
\textsc{1sg} walk\\
\glt `I have walked.'

\ex
GU\il{Gullah}\\
\gll Mi si ǝm.\\
\textsc{1sg} see it \\
\glt `I have seen it.'

\ex SR\il{Sranan}\\
\gll  Mi waka.\\
\textsc{1sg} walk     \\
\glt `I have walked.' 
\z \z

Here, Alleyne provides examples from different Creole languages to
show the unmarked form of the verb indicating \isi{Perfective} aspect.  When
it comes to verbs such as `love', `want', `know' etc. he points out
that there is nothing special about these and that the unmarked form
indicates \isi{Perfective} irrespective of what may be suggested by means of
an English gloss.  In this regard, he points out that:

\begin{quote}
a group of verbs, the same in all dialects and languages concerned,
have their \isi{perfective aspect} form glossed in English by a ``present
tense”. Thus \textit{mi sabi} (zero marker and therefore perfective)
is glossed in English as `I know' […] other verbs belonging to this
group are \textit{memba} `remember' , \textit{wan} `want' and
\textit{lob(i)} `love' There is nothing “irregular” about these
verbs. The forms cited above have \isi{perfective meaning} in
Afro-American,\footnote{What Alleyne refers to as Afro-American
  dialects includes the group of languages here referred to as CECs.}
irrespective of their English gloss. (p.~83)
\end{quote}

Alleyne does not provide examples for verbs like `love', `want' or
`know' in the \isi{Perfective} but points to examples of these with the
Im\isi{perfective aspect} marker as shown below:

% \setcounter{enumerate}{2}
\ea\label{ex:2:3} \citep[83]{Alleyne1980}
\ea SR\il{Sranan}\\

\gll Mi e sabi.\\
		\textsc{1sg} \textsc{asp} know\\
\glt `I (always) know' or `I begin to know.'

\ex SR\il{Sranan}\\
\gll Mi e wan en.\\
	\textsc{1sg} \textsc{asp} want it                \\
\glt `I (usually) want it.'

\ex
JC\il{Jamaican Creole}\\
\gll  A nuŋ mi a nuo.\\
\textsc{foc} now \textsc{1sg} \textsc{asp} know\\
\glt `It’s now that I am finding out.'
\z
\z

An \isi{Imperfective} marker is shown occurring with verbs such as
\textit{sabi, wan} and \textit{nuo} here translated as the English
`know' and `want'.  These are usually considered as \isi{Stative} verbs in
English and perhaps included in the discussion to show that they can
and do co-occur with \isi{Imperfective} aspect marking.\footnote{It is
  important to point out here that the co-occurrence of \isi{Imperfective}
  aspectual markers (\isi{Progressive} in particular) with \isi{Stative} verbs
  \citep{Vendler1967} has not been observationally adequate even for
  English. Essentially, \isi{Imperfective} aspect can interact with various
  verbs; where it does interact with \isi{Stative} verbs, the interpretation
  may be different from that of Non-stative verbs (note in
  \tabref{tab:srananverbs} above that in one meaning it is the onset of the
  situation that is indicated. The interaction between grammatical
  aspect and \isi{inherent aspect} is discussed in \chapref{ch:5}. See also
  discussion of \citet{Sidnell2002} below.} .

Alleyne also makes reference to preverbal \textit{done},\footnote{This
  form is derived from the English `done' in most if not all CECs.}
which he labels as a reinforcer of \isi{Perfective} aspect.  He points out
that while \isi{Perfective} aspect is unmarked everywhere, it ``can in all
dialects be recognized by its being optionally conjoined with a verb
meaning `finish',  which acts as a kind of reinforcer of the perfective
aspect.” (p.~82).  The status of \textit{done} as a marker of
\isi{Perfective} aspect has been questioned, however, based on its focus on
the completion of an event where \isi{Perfective} focuses on the event as “a
complete whole” \citep{Comrie1976}.  In this regard, \textit{done} may
be argued to be an indicator of perfect \isi{Tense} which is compatible with
completion.  According to \citet{Comrie1976}:

\begin{quote}
the perfect looks at a situation in terms of its consequences and
while it is possible for an incomplete situation to have consequences,
it is much more likely that consequences will be consequences of a
situation that has been brought to completion. (p.~64)
\end{quote}

Alleyne’s analysis of \textit{done} and similar forms in other Creole
languages brings into focus the ongoing discussion on the semantic
content of grammatical markers.  As seen here, Alleyne holds the
\isi{unmarked verb} as the indicator of perfectivity while \textit{done} is
seen as a reinforcer of this.  So in other words, it is an additional but
overt marker that reinforces what is indicated by the \isi{null marker}.  I
will review other analyses of the \isi{unmarked verb} and \textit{done} in
the discussion of \citet{Youssef2003} later on.

Overall, Alleyne’s focus seems to be on establishing grammatical
aspectual viewpoint as independent of other factors such as the nature
of the verb. Thus he focuses on the fact that a meaning such as
Perfectivity may be applied to a verb regardless of any categorisation
that may be associated with the verb itself.  In his view, any
\isi{unmarked verb} is consistent with the aspectual category \isi{Perfective}.
This is a significant observation in Creole studies as it relates to
\isi{Aspect}.  An analysis of the same type of data from the perspective of
\isi{Tense} would have revealed contrasts in the meanings of different verbs
as it relates to the indication of \isi{Past} and Present (compare
\citealt{Bickerton1975} below).  However, since \isi{grammatical aspect} must be
taken as a particular view of a situation, this view may be associated
with a situation regardless of the inherent nature of the
situation. This is actually the type of approach that I take in
establishing terminology for different levels of \isi{Aspect}.  However, I
am careful to note the interaction between the different levels.  In
the case of Alleyne, we may note that his discussion, being focused on
\isi{grammatical aspect}, simply treats the categories \isi{Perfective} and
\isi{Imperfective} as semantic aspectual categories without taking into
consideration other layers of \isi{Aspect} and the interaction between them.

Such an approach must be recognised as limited in terms of a treatment
of \isi{Aspect} as a semantically complex area.  However, like
\citet{Voorhoeve1957} Alleyne may be credited with a focus on \isi{Aspect}
without any distraction by \isi{Tense} as we have seen elsewhere in the
field (compare \citealt{Bickerton1975}).  In this treatment, Alleyne
advances the observation that ``aspect, which is always marked, is of
more importance than tense in the verbal systems of these dialects.”
(p.~85).  This is an important observation, since \isi{Aspect} has traditionally
fallen under the wider umbrella of TMA in the study of Creole
languages with \isi{Tense} being the main focus.  Studies with a primary
focus on \isi{Aspect} rather than \isi{Tense} may yield different and perhaps more
insightful results.

\subsection{\citet{Bickerton1975}}\label{sec:2.1.3}

\citet{Bickerton1975} stands out as the work which first identifies
the \isi{Stative}\slash Non-stative distinction as important in the interpretation
of the bare form (\isi{unmarked verb}) in Creoles.  In addressing the claim
that Creole languages ``use an invariant form of the verb in all
contexts” (p.~27), he claims that although the \isi{stem form} (i.e. the
\isi{unmarked verb}, without preverbal material) is very frequent in Creoles
it “has several different and quite distinct functions” (p.~28).  In
relation to GC\il{Guyanese Creole} he points out clearly that:

\begin{quote}
the functions of the \isi{stem form} in the Guyanese system depend on the
sta\-tive--non-stative distinction … with non-statives, it signifies
`unmarked\linebreak past' – that is a (usually) single action that happened at a
moment in the past that may or may not be specified but should not
predate any action simultaneously under discussion (p.~28).
\end{quote}

We note here that Bickerton’s analysis is from the perspective of
\isi{Tense} and the \isi{aspectual status} of different verbs is used to account
for (default) \isi{Tense} interpretation.  He points out that the \isi{stem form}
of Statives signifies `non-past'.  In contrast, Non-statives have to
be marked by what he calls the continuative-iterative (also
non-punctual\footnote{We see a trend here where a marker is labelled
  based on the different aspectual meanings that may be associated
  with it in interaction with different kinds of verbs. This marker is
  also discussed in other works as the Non-completive or
  Non-\isi{perfective aspect} marker (see \citet{Voorhoeve1957} and
  \citet{Alleyne1980}, for example.)}) marker to be interpreted in a
similar way (p.~28).  The examples below
(\ref{ex:2:4}--\ref{ex:2:6}) are used to illustrate this:

% \setcounter{enumerate}{2}
\ea\label{ex:2:4}Bare Non-statives denote past single action \citep[29]{Bickerton1975}

\ea
GC\il{Guyanese Creole}\\
L\_ \textbf{run} out\\
\glt `L\_ ran out.'

\ex
\gll Me  \textbf{run} out.\\
\textsc{1sg} run out\\
\glt `I ran out.'

\ex All of    them \textbf{hold} \textbf{on} pon me.\\
    all of    \textsc{3pl}  hold on on me\\
\glt `Everyone held on to me.' \z \z

\ea\label{ex:2:5}
{Bare \isi{Stative} denotes present state (p.~29)}\\
GC\il{Guyanese Creole}  \\
\ea
\gll Mi na \textbf{no} wai dem \textbf{a} \textbf{du} dis ting.\\
        \textsc{1sg}  \textsc{neg} know why \textsc{3pl} \textsc{asp} do this thing\\
\glt `I don’t know why they are doing this.'

\ex
\gll Di       rais wok \textbf{get} mo     iizia      fi du bika           tracta  \textbf{a}        \textbf{plau}  am.\\
\textsc{art} rice work get more easier to do because tractor \textsc{asp}   plough   it\\
\glt `Rice farming becomes easier to do because tractors do the ploughing.'
\z
\z

\ea\label{ex:2:6} ``Continuative-iterative'' \textit{a} marks
Non-stative as ``non-past'' {(p.~29)}\\
GC\il{Guyanese Creole}\\
\gll  Wi                      \textbf{a}      \textbf{pak} am op hai     laik  haus     an      wi  kaal di   plees karyaan.\\
\textsc{3pl} \textsc{asp} pack    it   up high like house and \textsc{3pl} call \textsc{art} place X\\
\glt `We pile it up as high as a house and we call the place the threshing-floor.' \z

As we see from the examples here, \isi{Stative} and Non-stative verbs, when
evaluated from the perspective of \isi{Tense}, establish a contrast in
interpretation.  In particular, the unmarked Non-stative verbs `run'
and `hold on' in \REF{ex:2:4} are shown to indicate \isi{Past} while the
unmarked \isi{Stative} \textit{no} `know' in (\ref{ex:2:5}a) is present.
Also as Bickerton shows in (\ref{ex:2:5}b) and (\ref{ex:2:6}), the
Present \isi{Tense} interpretation in the case of Non-stative verbs arises
in the presence of overt (\isi{Imperfective}) grammatical \isi{aspectual marking}
using what he calls the ``continuative-iterative”marker \textit{a.}

An analysis from the perspective of \isi{grammatical aspect} would no doubt
have yielded different results.  In particular, similar to what was
observed above for both \citet{Voorhoeve1957} and \citet{Alleyne1980},
all the unmarked verbs in the examples above \xxref{ex:2:4}{ex:2:6} may be analysed as
\isi{Perfective} independent of whether or not they are \isi{Stative} or
Non-stative.  The contrast from this perspective comes from the
appearance of the \isi{Imperfective} \isi{aspect marker} which establishes some
form of Imperfectivity.  However, we must be reminded that Bickerton’s
preoccupation was with \isi{Tense} interpretation and the role of the verb
in this, as opposed to that of Voorhoeve and Alleyne which had to do
with (grammatical) \isi{aspectual outlook}.  In this regard, while
\isi{grammatical aspect} may in a sense be established regardless of the
type of verb, it appears in effect that a focus on \isi{Tense} raises
different concerns.  In particular, as we observe from the examples in
\xxref{ex:4:4}{ex:4:6} above, there is a difference in the (default) \isi{Tense}
interpretation of the \isi{unmarked verb}, dependent on whether it is
\isi{Stative} or not.  However, there is no such contrast in the aspectual
outlook indicated in these Items.

Bickerton’s focus on \isi{Tense}, as opposed to \isi{Aspect}, ironically
highlights a level of aspectual interpretation below that of
\isi{grammatical aspect}.  This level, although not seemingly impactful in
the case of the \isi{unmarked verb}, assumes relevance in the case of
\isi{Imperfective} \isi{aspectual marking}.  In this regard, Bickerton observes a
contrast in the behaviour of \isi{Stative} and Non-stative verbs in the face
of “continuative” (also \isi{Progressive} or more generally \isi{Imperfective})
marking.  He points out that: ``the occurrence of statives and
continuative markers is as unacceptable as it is in
English.”\footnote{Compare with \citegen{Vendler1967} observation of
  a restriction on the occurrence of \isi{Stative} verbs with \isi{Progressive}
  aspect. Recall also that this restriction has been challenged. I
  return to this in \chapref{ch:5}} (p.~30).  The examples below show
this restriction:

\ea%7
\label{ex:2:7}
\citet[30]{Bickerton1975} \\
\ea[*]{
\gll     mi          a \textbf{nuo} da    \\
\textsc{1sg} \textsc{asp}      know that\\
\glt (`I am knowing that.')
}

\ex[*]{
\gll    dem           a \textbf{gat} wan kyar \\
\textsc{3pl} \textsc{asp}      have   one car \\
\glt (`They are having [sc. possessing] a car.')}
\z 
\z

As shown here, the \isi{Stative} verbs \textit{nuo} `know' and \textit{gat}
`possess' are shown to yield ungrammatical utterances in their
occurrence with the \isi{Imperfective} \isi{aspect marker} \textit{a}.

Note though that a number of authors have since challenged the
\isi{observational adequacy} of this, showing the combination of \isi{Stative}
verbs and \isi{Imperfective} aspect marking as well attested in CECs (see
\citealt{Jaganauth1987}; and \citealt{Sidnell2002} discussed below).  Of
interest, however, is the type of aspectual interpretation that may
arise in the face of this combination.  \citet{Sidnell2002} for
example, shows that the combination of \isi{Stative} verb and \isi{Imperfective}
aspect marking has predictable constraints on meaning.  In particular,
with \isi{Stative} verbs the interpretation is \isi{Habitual} while \isi{Imperfective}
aspect marking with Non-stative verbs yields an interpretation that is
\isi{Progressive} (cf. data above in \citet{Voorhoeve1957}, where similar
interpretations are shown for verbs like `want' as opposed to `wash').
This difference in aspectual interpretation suggests a need to further
understand the impact of the verb in aspectual interpretation; or what
exactly is responsible for such a difference in interpretation. This
is one of the motivators behind the overall study that develops in
this work.

Bickerton, even in his very definitive stance on the
\isi{Stative}\slash Non-stative distinction as crucial in Creoles, was not very
clear on its theoretical application.  Although his general discussion
indicates that he applies the \isi{Stative}\slash Non-stative distinction to
verbs, in addressing items such as those in \REF{ex:2:8} which
seemingly appear in both \isi{Stative} and Non-stative use, he indicates an
application of the distinction to propositions rather than to verbs.
Compare:

\ea%8
\label{ex:2:8}
\citet[30]{Bickerton1975}\\
\ea
\gll Tu an tu \textbf{mek} fo.\\
     two and two make four \\
\glt `Two and two make four.'

\ex
\gll Dem \textbf{mek} i stap.\\
     \textsc{3pl} make it stop\\
\glt `They made him stop.'\z \z

As observed here, the form \textit{mek} appears in both \isi{Stative} and
Non-stative use being expressed as the equivalent of the English
\isi{Stative} verb \textit{to be equal to} in (\ref{ex:2:8}a) and the causative
Non-stative \textit{cause} in (\ref{ex:2:8}b).  In following through with his
observation of default \isi{Tense} interpretations for \isi{Stative} and
Non-stative verbs, Bickerton observes that \textit{mek} in
(\ref{ex:2:8}a)

\begin{quote}
follows the rule for stative verbs (stem-only for non-past)” [while
\ref{ex:2:8}b] ``has a non-stative meaning and in it, \textit{mek}
must therefore follow the non-stative rule (stem-only for simple past)
(p.~30).
\end{quote}

Based on examples such as \REF{ex:2:8}, Bickerton states explicitly
(perhaps rather unexpectedly) that

\begin{quote}
the stative-non-stative distinction in \ili{Guyanese Creole} is a semantic
one entirely: That is to say, it is not the case that specific lexical
items are marked unambiguously [+stative] or [\textminus stative], rather that
these categories apply to propositions irrespective of their lexical
content. (p.~30)
\end{quote}

We note here that Bickerton applies the \isi{Stative}\slash Non-stative
distinction to\linebreak propositions rather than to lexical items.  However,
notice even here that it is the verb that is then associated with the
feature [+/\textminus \isi{Stative}] on the basis of the proposition in which it
appears.  Thus, while he notes a \isi{Stative} \textit{meaning} for the
proposition in (\ref{ex:2:8}a), he still applies this to the verb
itself, as he points out consequently that ``\textit{mek} must
therefore follow the non-stative rule (stem-only for simple past)”
(p.~30).  So the question still is whether or not the
\isi{Stative}\slash Non-stative distinction for him is applied to propositions or
verbs.

His statement that the application is to propositions rather than
lexical items seems contrary to his actual approach where it is
lexical items that are called on for observation and application of
the distinction.  However, what we note here is a genuine difficulty
in associating a form with \isi{lexical specification} and actual use.  Thus
in this case, the differences in meaning suggest that we are dealing
with two different forms altogether rather than a single form
associated with different realisations and consequently different
meanings.\footnote{Compare to items that I treat as dual aspectual
  forms in \chapref{ch:5}. These are different from the case of
  \textit{mek} in that they feature single lexical items in different
  uses and the uses are linked to a single abstract Event Structure
  which accounts for both uses.}  \textit{Mek} in (\ref{ex:2:8}a)
functions as a main verb with a meaning that points to a generic
(logical) result with no indication of Change (i.e.: the equivalent of
the English form with the meaning `to be equal to') while \textit{mek}
in (\ref{ex:2:8}b) is an overt \isi{causative verb} indicating and
introducing an initiator \CAUSE in a Change of state\is{State!Change of}.

While this difference in what appears as a single lexical item is
evident from a perspective which distinguishes forms based on
meanings, it is not so evident if the focus is principally on the
form.  Thus, from Bickerton’s perspective, based on the differences
observed in the instantiations of \textit{mek,}

\begin{quote}
one must either arbitrarily list \textit{mek}\textsubscript{1} [+stative] and \textit{mek}\textsubscript{2} [\textminus stative] in the
lexicon, or one must admit that the syntactic component can somehow
``read'' semantic information, i.e. that semantics is generative rather
than interpretive. (p.~30)
\end{quote}

Bickerton here appears to find the prospect of an ``arbitrary'' listing
in the lexicon of two different \textit{mek} as uneconomical.  For
him, in the case of \textit{mek} we are dealing with one lexical item
distinguished only by the feature \isi{Stativity}.  However, indications
based on the meaning components associated with the form in each
instance, are that we are indeed dealing with different semantic
forms.  The case of \textit{mek}\textsubscript{1} being a verb with
the associated meaning `to be equal to' and
\textit{mek}\textsubscript{2} is the \isi{causative verb} indicating an
initiator in a Change of state\is{State!Change of}.

While the case of \textit{mek} appears here as a case of homophony, a
real challenge to Bickerton’s \isi{Stative}\slash Non-stative distinction is the
case of those items which appear indeed as single lexical items in
both \isi{Stative} and Non-stative uses.  If we interpret Bickerton
benevolently, it would be fair to say that his approach to the
\isi{Stative}\slash Non-stative distinction has been made somewhat unclear with
his indication that it is to be applied to propositions; pointing to
more than just the verbs but treating verbs based on the propositions
that they can appear in.  In the approach that I follow in this work,
the notion of \isi{Stativity} is reduced to the semantic concept of Change
and applied directly to lexical items even in the context of dual
aspectual forms.

Bickerton’s work, starting with this (\citeyear{Bickerton1975}) work may be credited with the glut of work on TMA in Creole languages due to his insistence on
the importance of the \isi{Stative}\slash Non-stative distinction to understanding
\isi{Tense} in Creoles.  This impact is noted by authors outside of the
field of Creole studies, who also credit Bickerton with the fact that
Creole studies on TMA seem to exist in a world that is separate from
the general theoretical genre.  In this regard, \citet{Dahl1993} for
example notes that due to the impact of Bickerton’s work,

\begin{quote}
the study of Creole TMA systems has become an autonomous tradition,
with its own terminology and conceptual apparatus with an ensuing
relatively restricted influence on non-creolist TMA studies. (p.~251)
\end{quote}

\subsection{\citet{Bickerton1981}}\label{sec:2.1.3.1}

In his \citeyear{Bickerton1981} work, Bickerton again focuses on the \isi{Stative}\slash Non-stative
distinction which he now also refers to as the State-\isi{Process}
distinction (SPD), this time in relation to language acquisition.
Regarding this distinction, he claims that it is ``directly involved in
the acquisition of the English \isi{Progressive} marker -\textit{ing.}”
(p.~138).  He states that ``just as there are verbs that do not take
-\textit{ed,} there are verbs that do not take -\textit{ing…} such as
\textit{like}, \textit{want,} \textit{know,} \textit{see},
etc.” (p.~138).  He points out that:

\begin{quote}
These verbs are quite common in children’s speech, probably as common
as many of the irregular verbs to which children incorrectly attach
-\textit{ed.} Yet apparently, children never attach -\textit{ing} to
stative verbs (p.~138).
\end{quote}

Based on this, he presumes the SPD to be innate ``not because of its
universality […] but because it plays a crucial role in Creole
grammars.” (p.~142).  Here again he points to differences in the
behaviour and default interpretations of Statives as opposed to
Non-stative verbs stating that,

\begin{quote}
present-reference statives and present-reference nonstatives cannot be\linebreak
marked in the same way, and the same applies to past reference
statives and non-statives (p.~142).
\end{quote}

For this he provides the following table, which shows restrictions
based on the SPD in GC\il{Guyanese Creole} (\tabref{extab:9}).

\begin{table}
  \caption{Bickerton’s (\citeyear[142]{Bickerton1981}) model of GC Tense-Aspect}
  \label{extab:9}
  \begin{tabular}{lcc}
    \lsptoprule 
     & \isi{Stative} & Non-stative\\
    \midrule
    Present Reference & ∅ & a\\
    \isi{Past} Reference & bin & ∅  \\
    \lspbottomrule
  \end{tabular}
\end{table}


Similar to his observation in his \citeyear{Bickerton1975} work Bickerton shows here
that an unmarked \isi{Stative} verb by default is interpreted as present
while the unmarked Non-stative is interpreted as past.  To be
interpreted with present reference, the Non-stative verb must be
preceded by the \isi{Imperfective} \isi{aspect marker} (\textit{a} in this case);
the \isi{Stative} verb must be preceded by the overt \isi{Past} \isi{Tense} marker
\textit{bin} in order to be interpreted as \isi{Past}.\footnote{This same
  marker is allowed with Non-stative verbs but the interpretation in
  such cases is that of Anterior or ``\isi{Past} before \isi{Past}'' as opposed to a
  simple ``\isi{Past}''.}  This analysis by Bickerton, like his proposal of
a restriction on the occurrence of \isi{Stative} verbs with \isi{Imperfective}
aspect marking, has been challenged by authors such as
\citet{Jaganauth1987}, \citet{Winford1993}, and \citet{Gooden2008}, who
point to the involvement of other factors that may affect the
interpretation arising from these combinations.

Both \citegen{Bickerton1975} and (\citeyear{Bickerton1981}) works managed to provoke much
debate with regard to \isi{Tense} and \isi{Aspect} expression but, as I pointed
out, were not intended primarily to answer questions related to
\isi{Aspect}.  What \citet{Bickerton1975} did successfully, however, was to
show that Creoles do indeed have a \isi{Tense} system and that there is a
systematic way in which this is expressed, depending on the type of
verb and its \isi{inherent aspect}.

\subsection{\citet{Jaganauth1987}}\label{sec:jaganauth}\label{sec:Jaganauth}

\citet{Jaganauth1987} responds to Bickerton’s claims regarding the
\isi{Stative}\slash Non-sta\-tive distinction and claims this as ``empirically
invalid” both in terms of Bickerton’s predicted restrictions and
default interpretations (p.~21). Along these lines, she provides data
from GC\il{Guyanese Creole} that goes contrary to Bickerton’s observations.  In
particular, she shows the occurrence of ``so called” \isi{Stative} verbs with
the \isi{Imperfective} \isi{aspect marker}, as well as \isi{Stative} verbs appearing
with what she calls a dynamic verb interpretation, and also instances
where it is unclear whether the interpretation of a particular verb is
\isi{Stative} or Dynamic.  The data below is set out along these lines:

\ea%10
\label{ex:2:10}
\isi{Stative} verbs in Non-stative use (GC\il{Guyanese Creole})\\

\ea
\gll Somtaim           mi  \textbf{a} \textbf{de} \textbf{gud},  somtaim          mi           a       sik.\\      
  sometimes  \textsc{1sg}  \textsc{asp}       be          fine, sometimes   \textsc{1sg}  \textsc{asp}     ill\\ 

\glt `Sometimes I would be enjoying good health, other times I would not.' (p.~23)

\ex
\gll     A   now        mi \textbf{a}      \textbf{no} da.\\
\textsc{foc} now    \textsc{1sg}   \textsc{asp} know    that\\
\glt `I am now discovering that.' (\DYNAMIC) (p.~31)

\ex
\gll     I           na     bin   \textbf{a} \textbf{get}        non    pikni.\\
\textsc{3sg} \textsc{neg} {\TNS} \textsc{asp}        get \textsc{neg}   child\\
\glt `He wasn’t succeeding in fathering (obtaining) any children.'
(\DYNAMIC) (p.~30)

\ex
    \gll I na \textbf{bin}     \textbf{get} non   pikni.\\
\textsc{3sg} \textsc{neg} {\TNS}   get  \textsc{neg} child\\
\glt i. `He didn’t have (possess) any children.' (\STATIVE) 
\glt ii. `He didn’t father any children.' (\DYNAMIC) (p.~30) \z \z

(\ref{ex:2:10}a--c) shows the GC\il{Guyanese Creole} \isi{Stative} verbs
\textit{de} `be’, \textit{get} `obtain' and \textit{no} `know'
co-oc\-cur\-ring with the \isi{Imperfective} \isi{aspect marker} \textit{a.}
(\ref{ex:2:10}d) shows what is indicated as by Jaganauth as the
ambiguous interpretation of the item \textit{get.}  This data as
indicated, serves partly to disprove \citegen{Bickerton1975}
observation regarding the \isi{Stative}\slash Non-stative distinction and a
restriction on the occurrence of the \isi{Imperfective} marker with \isi{Stative}
verbs.  With regards to the appearance of \isi{Imperfective} aspect marking
with \isi{Stative} verbs, as I pointed out earlier, what is important in such
cases is not just the fact that these elements co-occur but more so
the aspectual meaning that arises.  In particular, if we examine the
data in \REF{ex:2:10}, we note in the case of the verb \textit{no}
`know' in (\ref{ex:2:10}b) for example that what is indicated is the
onset of the situation as opposed to its progression - an
interpretation associated with certain Non-stative verbs in
interaction with \isi{Imperfective} aspect marking.  I discuss this
interaction between \isi{Imperfective} aspect marking and different types of
verbs in \sectref{sec:5.1}.

Closer examination of (\ref{ex:2:10}a, c and d) also raises the
question of what it means to be ``\isi{Stative}'' and the inherent aspectual
status of the items \textit{de} and \textit{get} as used here.  In the
case of \textit{get,} the meaning indicated in (\ref{ex:2:10}c) is one
which includes Change\footnote{In \chapref{ch:4} I elaborate the notion
  of Change as an abstract semantic notion. This is contained in the
  semantic representation of the verb and may be represented through
  primitives indicating a Change of state\is{State!Change of} such as \BECOME, and
  causation through the primitive {\CAUSE} -- all associated with
  Non-stativity.}  (obtain) and can be contrasted with \textit{get}
which indicates possession (\ref{ex:2:10}d i.); an inherent \isi{State} based on the
definition I posit in \chapref{ch:4}.  Although, `get' appears as a
single lexical item on the surface, it seems that is expresses two
meanings.  The first a Change of state\is{State!Change of} (obtain) and the second a \isi{State}
(possess) -- possibly the result of the Change of state\is{State!Change of} meaning, though
not necessarily so.  Such behaviour may be associated with the Event
Structure of \isi{Transition} (\citealt{Pustejovsky1988,Pustejovsky1991}) 
as I discuss it in \chapref{ch:4}.  This would account for the 
ambiguity in meaning that Jaganauth indicates for (\ref{ex:2:10}d) above.
However, the use of the form in (\ref{ex:2:10}c) is clearly a Non-stative instantiation (`to father') and thus not the case of a \isi{Stative} 
verb occurring with Im\isi{perfective aspect} marking.

In the case of \textit{de} the meaning content here would need to be
further investigated.  Jaganauth seems to analyse \textit{de} here
under the condition that it is the \isi{Stative} (locative) form `to be'.
However, there appears to be a similar form \textit{de} which rather
than a locative (be), serves to introduce a stage level meaning in
contrast to the individual level meaning that may be associated with
the \isi{Stative} (locative) \textit{de.} Compare:

\ea%11
\label{ex:2:11}
(personal CEC data)\\

\ea
  \gll  Ai   \textbf{de} ier.\\
\textsc{1sg} \textsc{cop} here\\
\glt `I am here.' (I am just existing)

    \ex  Ai  gud.\\
\textsc{1sg}  good\\
\glt `I am well/ok.' (generally)

\ex
\gll  Ai \textbf{de} gud.\\
      Is \textsc{cop} good\\
\glt `I am well.' (at the moment)
\z \z

As we see from these examples, \textit{de} may express a location or a
psychological state of mind as in (\ref{ex:2:11}a) but in
(\ref{ex:2:11}c) it seems to introduce the meaning of Transitory
\isi{State}.  We see this when we compare (\ref{ex:2:11}b) to
(\ref{ex:2:11}c).  In the case of (\ref{ex:2:11}c) as opposed to
(\ref{ex:2:11}a), the predicate seems to be \textit{de gud} which
expresses a temporary \isi{State} as opposed to \textit{de} which expresses
a more general \isi{State}. When we further compare (\ref{ex:2:11}c) with
(\ref{ex:2:10}a) above which shows the presence of the \isi{Imperfective}
\isi{aspect marker}, the meaning is one of progressivity -- characteristic of
a Non-stative verb in interaction with \isi{Imperfective} marking.\footnote{I
  am not here suggesting that \textit{de} has the function of a
  Non-stative verb but rather that the combination \textit{de gud}
  introduces a meaning that is consistent with a more temporary
  \isi{State}. The interaction between this kind of predicate and the
  \isi{Imperfective} aspect marking provides a more \isi{Progressive} type
  interpretation as opposed to the \isi{Habitual} or inchoative that
  typically arises with Statives. See discussion of the interaction
  between the \isi{Progressive} and different types of verbs in \chapref{ch:5}
  (\sectref{sec:5.1}).}

In addition to the cases discussed in \REF{ex:2:10}, Jaganauth also
points to data in the form of items such as \textit{ded} `to die,
dead' and \textit{fraikn} `frighten, afraid' to highlight the
ambiguity that may be associated with lexical items. Also the case of
items such as \textit{sik} `ill, make ill', \textit{weeri} `weary,
make weary', \textit{redi} `ready, make ready' etc. which also appear
in both stative and Non-stative use. These are shown in \REF{ex:2:12}
and \REF{ex:2:13} respectively:

\ea%12
\label{ex:2:12}

{\isi{Stative} verb occurs with dynamic or ambiguous \isi{Stative}\slash Dynamic verb interpretation (GC\il{Guyanese Creole})} \citep[31--32]{Jaganauth1987}\\

\ea
  \gll  I \textbf{ded} siks a~klak     dis maanin.\\
\textsc{3sg} dead    six o’clock   this morning\\
\glt `He died at six this morning.' (\DYNAMIC) 

\ex
\gll Now          i \textbf{ded}     dem    kom    bak.\\
     now \textsc{3sg}       dead    they    come back \\
\glt `Now that he is dead/has died they have returned.'
(\STATIVE/\DYNAMIC) 

\ex
   \gll Mi \textbf{fraikn}  a daag.\\
\textsc{1sg}       frighten  a dog\\
\glt `I have scared the dog'. (\DYNAMIC) 

\ex
    \gll Mi \textbf{fraikn} a daag.\\
\textsc{1sg}        frighten of dog\\
\glt `I fear/am afraid of the dog'. (\STATIVE) \z \z

\ea%13
\label{ex:2:13}
GC\il{Guyanese Creole} \citep[31]{Jaganauth1987}\\
\ea
\gll Da tablit \textbf{sik} mi stomik.\\
    that tablet        sick \textsc{1sg} stomach\\
\glt `That pill has made me ill.'

\ex
\gll Dis baskit \textbf{weeri} mi.\\
     this basket        weary 1s\\
\glt `This basket has made me tired'.

\ex
     \gll I \textbf{redi} shi.\\
\textsc{3sg} ready \textsc{3sg}\\
\glt `He has gotten her ready'.  
\z\z

As shown here, lexical items such as \textit{ded} and \textit{fraikn}
appear in different instances associated with either the \isi{Stative}
meanings `dead' and `fear' or the Non-stative meaning `to die' and ‘to
frighten' respectively.  For the items in \REF{ex:2:13} we see these
appearing in Non-stative use indicating a Change from one \isi{State} to
another with an obvious Cause or \isi{Agent}.  These may be compared to
cases where these items appear in \isi{Stative} use.  Compare:

\ea%14
\label{ex:2:14}
(Personal JC\il{Jamaican Creole} data)\\
\ea
    \gll Mi stomik sik.\\
\textsc{1sg} stomach sick\\
\glt  `I am ill.'

\ex
  \gll  Mi weeri.\\
\textsc{1sg} weary \\
\glt `I am weary/tired.'

\ex
   \gll Shi redi.\\
\textsc{3sg} ready\\
\glt `She is ready.'\z \z

Based on cases such as these Jaganauth claims that:

\begin{quote}
there are instances […] where it would be impossible to determine with any
certainty whether the situation referred to is stative or
dynamic.\footnote{Note here that Jaganauth applies the
  \isi{Stative}\slash Non-stative distinction to propositions hence the
  difficulty in determining whether a sentence is \isi{Stative} or
  Non-stative as all the elements in the sentence contribute to
  this. If the \isi{Stative}\slash Non-stative distinction is applied to the verbs
  themselves, we will see that close examination of the meanings
  indicated by these verbs actually coincide with the feature
  Jaganauth applies to each of these sentences.}  (p.~35).
\end{quote}

Jaganauth’s argument for these is that the presence of an affected
object rather than the fact that the verb is \isi{Stative} or Non-stative
accounts for the difference in aspectual interpretations.  According
to her

\begin{quote}
it is the presence or absence of other elements in the proposition or
the non-linguistic context which serves to focus on one or the other
perspective of the situation (p.~35).
\end{quote}

While I do agree with Jaganauth’s observation of the involvement of
other factors (outside of \isi{inherent aspect}) in \isi{aspectual outlook}, the
concept of a basic primitive Event Structure associated with lexical
items provide a way of capturing an inherent contribution of the verb
to \isi{Aspect}.  Essentially, what Jaganauth calls the ``the presence or
absence of an affected object” is linked to a predicate’s ability to
express Change and thus related to (Non)-\isi{Stativity}.  In my treatment
of items expressing similar behaviours, I link the behaviour of such
items to a basic Event Structure template following \citet{Pustejovsky1988,Pustejovsky1991}.  Thus for example items such as \textit{ded} `dead' and
\textit{fraikn} `frighten' based on the behaviour that they display
are linked to an Event Structure of \isi{Transition} that includes Change
and are thus Non-stative.  Items of the type \textit{ded} `dead' which
fall within the larger group of property items in CECs will be the
focus of the discussion in Chapters~\ref{ch:4} and~\ref{ch:5} where I articulate a
treatment.

In the case of \textit{get} while I do not focus on forms of this
type, it appears to represent the case of homonyms; a single form
associated with distinct (though perhaps related) meanings.  From this
perspective as well, such forms may be associated with a status as
\isi{Stative} or dynamic based on the usage in which it appears.  Thus where
\textit{get} appears with a meaning equivalent to `possess' it would
be an instantiation that is \isi{Stative} and likewise, where the
interpretation is `obtain' it would be Non-stative.  In this way, the
\isi{aspectual status} of a form is linked directly to the meaning
associated with such a form and whether or not it expresses Change.
This treatment is further elaborated in \chapref{ch:5}.

An analysis of the type which I articulate in this work, which, takes
into consideration lexico-semantic representations and primitive
meanings in the treatment of these lexical items may have provided Jaganauth
with the tool to merge the idea of a unique aspectual contribution of
the verb with the variable behaviour of items in her data.  This is
something that I attempt to do in this work.

Jaganauth’s study, while noting the deficiencies in other accounts,
cannot be credited with presenting a workable alternative.  In
summarizing, Jaganauth takes into consideration the involvement of
several elements in aspectuality but fails to admit the contribution
of the verb to aspectuality.  The data that she provides in her study
points to the necessity of a treatment of \isi{Aspect} in CECs that
acknowledges and treats the case of forms which express ``dual
aspectual'' behaviour.

\subsection{\citet{Winford1993}}\label{sec:2.1.5}

\citet{Winford1993},\footnote{Later works by \citet{Winford1997,Winford2000}
generally reflect sentiments similar to those highlighted
  in this section. These works, though not discussed here, are alluded
  to later in \chapref{ch:3} where I deal with the problem of dual
  aspectual forms in the literature.}  like \citet{Jaganauth1987}, acknowledges a
certain superficiality in the nature of studies on TMA.\footnote{These
  sentiments are repeated in his later works (\citeyear{Winford1997}) and (\citeyear{Winford2000}).}
  In this regard, Winford states that while the general facts about the CEC
verb system are well known, they have been presented in a ``rather
piecemeal fashion and for the most part in informal terms” (p.~26).
His work attempts to put the pieces together.  Winford’s approach,
unlike that of Jaganauth, ``preserves the simple intuition that certain
predicators involve change or process while others do not” (p.~26). In
other words he may be said to intuitively accept the
\isi{Stative}\slash Non-stative distinction.  Like \citet{Bickerton1975,Bickerton1981} he
points to a clear contrast in the interpretation of bare Non-stative
verbs as opposed to Statives. This is shown below in \REF{ex:2:15} and
\REF{ex:2:16} \citep[33]{Winford1993}:

\ea%15
\label{ex:2:15}
CEC (Non-statives) \\
\ea
\gll Mieri \textbf{rait} wan leta.\\
		Mary write one letter\\
\glt `Mary wrote/has written a letter.'

\ex
\gll Jan \textbf{iit} di mango.\\
		John eat \textsc{art} mango\\
\glt `John ate/has eaten the mango.' \z \z

\ea%16
\label{ex:2:16}
CEC (\isi{Stative})\\
\ea
    \gll Di pikni \textbf{waant} waata.\\
\textsc{art} child        want water\\
\glt `The child wants water.'

\ex
\gll Sam \textbf{lov} di uman fi truu.\\
		Sam love \textsc{art} woman for true\\
\glt `Sam truly loves the woman.'  \z \z

Regarding these, Winford points out that, ``[a]s the translations
suggest, the default interpretation of base non-stative verbs is
``\isi{Past}”, while that of ``statives” is ``Present” (p.~34).

In his application of the \isi{Stative}\slash Non-stative distinction, Winford
rejects \citegen{Bickerton1975} view that the \isi{Stative}\slash Non-stative
distinction applies to propositions rather than to specific lexical
items.  In treating the flexibility of interpretation that we observe
in certain forms, however, he acknowledges difficulties in the
application of the \isi{Stative}\slash Non-stative distinction and, like
\citet{Jaganauth1987}, points to the influence of other factors.
According to him,

\begin{quote}
[t]he stative\slash non-stative distinction is sometimes quite difficult to
apply in practice for two reasons.  First, there is the problem of the
interaction between contextual influences and the ``inherent meanings”
of the verb. In many cases, predictions about the possible behaviour
of, say, a ``stative” verb turn out to be falsifiable, given the
appropriate discourse context.  Secondly … there is a great deal of
flexibility in how individual verbs may be used (p.~29).
\end{quote}

This statement shows sensitivity to the involvement of other factors
within the context of the application of the \isi{Stative}\slash Non-\isi{Stative}
distinction, namely the factor of discourse and the flexibility in the
usage of verbs.  The examples in \REF{ex:2:17} below, for example, show
how the presence of adverbials conditions the interpretation of
unmarked Statives:

\ea%17
\label{ex:2:17}
{Trinidadian Creole \citep[34--35]{Winford1993}}\\
\ea
     \gll Ai \textbf{noo}         hi     wen         hi     juuzta    liv   bai         hi   faada.\\  
\textsc{1sg}         know  \textsc{3sg}  when \textsc{3sg}  {used to}   live   by \textsc{3sg} father \\
\glt `I knew him when he used to live at his father’s.'

\ex
\gll    Ai \textbf{noo}          hi moda    in dem   deez.\\  
\textsc{1sg}       know \textsc{3sg} mother in them  days\\
\glt 'I knew his mother in those days.'

\ex
\gll    Ai  \textbf{noo} hi                    moda      fu    a  lang taim.\\
\textsc{1sg}        know \textsc{3sg} mother for \textsc{art} long time\\
\glt `I’ve known his mother for a long time.'

\ex
\gll Ai  \textbf{noo} hi            moda        sins    mi                   smaal.\\
		\textsc{1sg} know \textsc{3sg}  mother     since   \textsc{1sg} small\\
\glt `I’ve known his mother since I was small.' 
\z \z

In comparison to the \isi{Stative} verbs in \REF{ex:2:16} above, which are
interpreted as present, we note that the interpretation of the \isi{Stative}
verb \textit{noo} `know' here is \isi{Past} or \isi{Tense} neutral.  Winford
attributes this to context, explaining in the case of (\ref{ex:2:17}a)
that

\begin{quote}
the context clearly establishes that the \textit{hi} `he' in question
no longer lives at his father’s, and that \textit{noo} `know' refers
to a past acquaintance (p.~35).
\end{quote}

With regards to (\ref{ex:2:17}b--d), he points to the influence of
adverbials stating that such examples

\begin{quote}
reinforce the view that ∅-marked statives can be neutral with respect
to time reference, for which they rely on adverbial and other
specification. (p.~35)
\end{quote}

However, without such specification, he maintains that “the default
reading of unmarked statives […] is ``Present”” (p.~35).

By accounting for such cases through the influence of factors such as
context and the presence of adverbials, Winford establishes a
treatment of \isi{Aspect} that is sensitive to both primary and secondary
interpretations.  In essence, there is a default interpretation but
context may allow for the possibility of another (secondary)
interpretation.  This is one of the most appealing features of
Winford’s approach.  From this perspective it may be appropriate to
say that he transcends his predecessors in not depending on any single
element in his account.  However, while his approach points to the
involvement of different factors in \isi{Aspect} in CECs, he does not
provide a model that indicates how these factors interact to yield a
particular \isi{aspectual outlook}.  In addition, it is not clear how the
\isi{Stative}\slash Non-stative distinction applies at the \isi{lexical level} 
and how this works in relation to items which display \isi{aspectual flexibility}.

\subsubsection{\citet{Winford2001}}\label{sec:2.1.5.1}

\citet{Winford2001} is more focused on a typological classification of
CECs but he reinforces some of the points made in his \citeyear{Winford1993} work.  In particular, he maintains that the \isi{Stative}\slash Non-stative distinction is
applicable in the aspectual systems of CECs and in this regard, he
generalizes that ``[t]he stative\slash non-stative distinction is crucial to
the interpretation of temporal and aspectual meaning.” (p.~5).  He
uses this distinction as a basis for the interpretation of \isi{Tense},
indicating that unmarked Statives in the varieties under focus (SR\il{Sranan},
GC\il{Guyanese Creole}, JC\il{Jamaican Creole} and \ili{Belizean Creole} (BC\il{Belizean Creole})) all signal ``simple present” while
unmarked Non-statives all signal ``absolute past” (p.~5).

Regarding a typological classification of CECs he focuses on
\isi{grammatical aspect} and how CEC languages organise their systems around
the \isi{Perfective}\slash Imper\-fective distinction or the
\isi{Progressive}\slash Non-progressive distinction.  According to him,

\begin{quote}
[i]t is difficult to generalize about the typology of CEC aspectual
systems. There seems to be a broad distinction between CEC’s which
organize their aspectual systems around the \isi{Perfective}\slash \isi{Imperfective} distinction (e.g., \ili{Sranan}, rural Guyanese), 
and those that organise them around the \isi{Progressive} vs Non-progressive distinction (e.g., Jamaican and Belizean) (p. 6--7)
\end{quote}

Based on this, he characterises \ili{Sranan} and rural Guyanese as ``closure”
languages while Jamaican and Belizean are ``dynamicity” languages
(p.~7). As can be seen here, the characterization of these languages
is based on how the aspectual systems treat \isi{viewpoint aspect} or
\isi{grammatical aspect} in overall \isi{aspectual outlook}.

The main contribution of Winford’s work may be said to be his effort
to employ general theoretical models of \isi{Aspect} in his description of
CEC aspectual systems.  His work overall has to be credited with
finding a way to include \isi{inherent aspect} in an analysis of \isi{Aspect}
within the context of the influence of other elements.\footnote{This
  sensitivity to the variety of elements in involved in \isi{Aspect},
  allowed him to note in his earlier work \citep[389]{Winford2000} the
  compatibility of `non-punctual’ aspect with Statives.} In this
way, his work is one of the few which creates a basis for a holistic
treatment of \isi{Aspect}.

\subsection{\citet{Andersen1990}}\label{sec:2.1.6}

\citegen{Andersen1990} work on \ili{Papiamentu} is worthy of note here as
it effectively separates \isi{Aspect} from \isi{Tense} as an area of study in
Creole languages establishing the marker \textit{ta} that was
previously analysed as \isi{Tense} marker as a \isi{grammatical aspect}
marker.\footnote{The interpretation of \textit{ta} as an \isi{Imperfective}
  \isi{aspect marker} has been challenged by  \citet{KouwenbergLefebvre2007}}
  He also points to the treatment of both ``inherent semantic aspect” and
  ``grammatically imposed aspect” as ``useful as a framework to interpret the functional properties of the \ili{Papiamentu} tense-aspect morphological system.” (p.~66).  His approach stands out as one that acknowledges the compositionality and
complexity of \isi{Aspect}.

Regarding \isi{inherent aspect}, Andersen identifies the feature \isi{Stative} as
``very relevant to \ili{Papiamentu}” (p.~66).  In this regard, providing some
support for \citet{Bickerton1975}, he points out that ``most (but not
all) \isi{Stative} verbs in \ili{Papiamentu} cannot be preceded by either the
``present” \isi{Imperfective} marker \textit{ta} or the past \isi{Imperfective}
marker \textit{a.}  He also points to another group of \isi{Stative} verbs
that do allow for the \isi{Imperfective} \isi{aspect marker}.  It is not clear
what semantic restrictions allow for some \isi{Stative} verbs to be preceded
by the \isi{Imperfective} \isi{aspect marker} while restricting others.  However,
what this does suggest is that the picture is more complex than a mere
restriction on the occurrence of \isi{Stative} verbs with \isi{Imperfective}
\isi{grammatical aspect}, and that the group of \isi{Stative} forms may be more
diverse than previously thought.  Andersen’s groups of \isi{Stative} verbs
are shown below in \REF{ex:2:18} and \REF{ex:2:19}; groupings are
based on their ability to be preceded by \isi{Imperfective} grammatical
aspect marking.

\ea\label{ex:2:18} Andersen’s \isi{Stative} verbs that do not allow
  grammatical marking by \textit{ta} or \textit{a}
(p.~71) 
\ea \textit{ta} `be' \ex \textit{tin} `have, exist'
\ex \textit{por} `can, may' 
\ex \textit{sa} `know (something)'
\ex \textit{konosé} `know (someone)' 
\ex \textit{ke} `want' 
\ex \textit{mester} `have to, must, should'
\ex \textit{yama} `be called'
\z \z

\ea\label{ex:2:19} \isi{Stative} verbs that allow for \isi{grammatical aspect}
marking (p. 71--72)
\ea \textit{debe} `owe'
\ex \textit{gusta} `like'
\ex \textit{kosta} `cost’ 
\ex \textit{bal} `be worth, cost' 
\ex \textit{stima} `love' 
\ex \textit{meresé} `deserve'
\ex \textit{parse} `seem, look like'
\ex \textit{nifiká} `mean'
\z \z

He points out, regarding those in \REF{ex:2:18} that, ``[t]he only way
to mark these verbs explicitly for past time reference is with
\textit{tabata.”} – the past \isi{Imperfective} \isi{aspect marker} (p.~71).  In
this group in \REF{ex:2:18} he makes special reference to
\textit{konosé} `know (someone)’ which can be used with the
Im\isi{perfective aspect} marker \textit{a.} In such a case, he points out
that ``it then loses its stative meaning, and means entry into a
state”- the equivalent of the English verb \textit{met}.'' (p. 93, fn 6).
We see here again sensitivity to \isi{inherent aspect} and the interaction
between this and \isi{grammatical aspect}.  The case of the difference in
meaning which results when \textit{konosé} `know (someone)' appears
with the \isi{Imperfective} \isi{aspect marker} reinforces my position that rather
than focusing on a clear restriction between \isi{Imperfective} aspect
marking and \isi{Stative} verbs, the interest should be in the
interpretation that arises.

Andersen’s indication that there may be at least two groups of \isi{Stative}
verbs is consistent with my observations for CECs.  In Chapters~\ref{ch:4} and~\ref{ch:5}, 
I point to three subgroups within the group of property items in
CECs which may be labeled \isi{Stative}: Those that do not allow for
Non-stative expression and are thus incompatible with \isi{Progressive}
aspect and do not appear in transitive use; those which allow for
Non-stative use and express a Change of state\is{State!Change of}; and those which in
Non-stative use express a \isi{Process}.  The suggestion based on the
observation of the behaviour of such items is that the Class of
\isi{Stative} verbs may be as complex as that of Non-stative verbs.

\subsection{\citet{Sidnell2002}} %2.1.7

This work merits some mention here as it directly contributes to the
body of work in CECs with a direct focus on \isi{Aspect}.  Sidnell’s focus
is \isi{grammatical aspect} in GC\il{Guyanese Creole} and in particular the expression of
\isi{Habitual} and \isi{Imperfective} marking in GC\il{Guyanese Creole}.  In looking at this area he points to the relevance of \isi{inherent aspect} to \isi{grammatical aspect}
marking, stating that the ``[c]hoice of preverbal marker is shown to be
strongly conditioned by the stativity of the predicate (in the case of
habituals)” (p.~151).  In contrast to \citet{Bickerton1975} Sidnell
observes that

\begin{quote}
[t]he co-occurrence of \isi{Imperfective} \textit{a} with stative predicates
is in fact rather well attested in CGC\il{Conservative Guyanese Creole}\footnote{Conservative Guyanese
  Creole, also Rural \ili{Guyanese Creole}.} … and in general there is nothing
particularly odd about the marking of statives with \textit{a.}
(p.~166)
\end{quote}

This observation of Sidnell, post Bickerton’s citing of a restriction
on this kind of occurrence, concurs with the position of
\citet{Jaganauth1987} and is more observationally adequate based on
data that we have seen. Crucially, it brings us to a level where our
interest is not just in the fact of such a co-occurrence but rather
the interpretation of it.

In this regard, Sidnell examines data such as \xxref{ex:2:20}{ex:2:21}  which shows
preverbal (d)\textit{a} occurring with both \isi{Stative} and Non-stative
predicates:\largerpage[2]

\ea%20
\label{ex:2:20}
\citet[165]{Sidnell2002}\\
\ea
  \gll Dem       \textbf{a} \textbf{gat}   aal         mi  kotlas          de.\\
\textsc{3pl}    \textsc{asp}        have   all \textsc{1sg} cutlass \textsc{loc} \\
\glt `they always have all my cutlasses there.' (\isi{Habitual})

% (20b)
\ex \citet[154]{Sidnell2002}\\
\gll    Di  man dem \textbf{a} \textbf{de}  moor in sosaiiti. (\isi{Habitual})\\
\textsc{art} man \textsc{pl} \textsc{asp} \textsc{cop} more in society\\

\glt `The men tend to be more in the public spheres.' \z \z


\ea%21
\label{ex:2:21}
\citet[154]{Sidnell2002}\\

\ea
\gll Shi     \textbf{a} \textbf{pik} plom, wen  mi  aks  shi yestodee. (\isi{Progressive})\\
\textsc{3sg} \textsc{asp}       pick plum when \textsc{1sg} ask  \textsc{3sg} yesterday.\\
\glt `She was picking plums when I asked her yesterday.'

\ex
  \gll Dem   \textbf{a} \textbf{kot} pikni soo? (\isi{Habitual})\\
\textsc{3pl} \textsc{asp}      cut   child so\\
\glt `Do they do autopsies on such small children?'

\ex
\gll Hiir, Linda \textbf{a} \textbf{kaal}        yu. (\isi{Progressive})\\
	hear Linda \textsc{asp}         call \textsc{2sg}\\
\glt `Hear, Linda is calling you.' \z \z

We note here a variation between \isi{Habitual} and \isi{Progressive} in the
aspectual interpretation of the marker \textit{a} as it interacts with
different predicates.  In observing examples of this type Sidnell
concludes that

\begin{quote}
there do seem to be highly predictable constraints on the combination
[i.e.: \textit{a} + stative]. The use of \textit{a} with stative
predicates almost categorically results in \isi{habitual meaning}\footnote{I
  suggest that the meaning may be even narrower with Statives; the
  meaning is more generic than it is \isi{Habitual} since no iteration is
  indicated as is the case with Non-statives.} […] Thus the interpretation
of \textit{a} with stative predicates is considerably more narrow than
it is with non-statives where it expresses either progressive or
\isi{habitual meaning}. (p.~166)
\end{quote}

This observation is supported by the fact that as we observe in the
data above, the interpretation associated with the \isi{Stative} predicates
in \REF{ex:2:20} is strictly \isi{Habitual} in comparison to the interpretation of
Non-statives, which may be either \isi{Habitual} or \isi{Progressive}.  Based on
the different interpretations that may be associated with Non-stative
predicates, however, Sidnell suggests that the \isi{Stative}\slash Non-stative
distinction may not be sufficient to account for what we see in \isi{Tense}
and \isi{Aspect} marking and interpretation in CECs (p.~166).

In this regard, he points to a sub-categorization of both \isi{Stative} and
Non-stative predicates.  Within the class of Non-statives, he points to
verbs of motion (\textit{gu} `go' \textit{kom} `come'), inceptives
(\textit{staat} `start'), and a residual class including verbs of
speaking (\textit{taak} `talk') and activity verbs (\textit{waak}
`walk') (p.~166).  Among Statives he also points to a complex group
comprising modals (\textit{kyan} `can'), locatives (\textit{de} `be'),
predicate adjectives (\textit{nais} `nice'), finite copula
(\textit{bii} `be') and other \isi{Stative} verbs (\textit{ga/ge} `get'
(possessive), \textit{noo} `know').  This kind of sub-categorisation
of both Statives and Non-Statives provides further insights into the
notion of \mbox{(Non-)}\isi{Stativity} and is at least in part consistent with my
observation in \sectref{sec:4.3} that there are additional concepts
including \MOTION, \CONTACT etc. that further help to define the basic
concept of Change.\footnote{These notions are evident in the
  transitivity alternations \citep{Levin1993}.  However as I point out
  in \chapref{ch:4}, while they indicate different types of Change, they
  do not show an opposition between [+Change] and [\textminus Change] which is
  the interest of this work.}  Additionally, with regards to his
observation of a complex group of \isi{Stative} verbs, his observation
supports my treatment of ``\isi{dual aspectual forms}'' as characterised by a
unique aspectual structure among property items.

Another point of interest raised by this work is the variation in the
\isi{Stativity} of a verb (predicate) based on its use.  Like Jaganauth,
Sidnell points to the verb \textit{get} `have' in CGC\il{Conservative Guyanese Creole} to illustrate
variation in stativity:

\ea%22
\label{ex:2:22}
Variation in stativity of \textit{get}
according to use \citep[167]{Sidnell2002}  
\ea
\gll Wan wan taim  doz \textbf{get} mi ignorant.\\
		one one time does get \textsc{1sg} ignorant\\
\glt `Once in a while (they) make me abusive.'

\ex
    \gll Di children          dem \textbf{a}      \textbf{get} nalej.\\
\textsc{art} children \textsc{pl} \textsc{asp}            get knowledge\\
\glt `The children get the knowledge.'

\ex
   \gll Mii aloon doz \textbf{get} a lak-op    de.\\
\textsc{1sg} alone does         get a lock-up there \\
\glt `I am the only one who has a room which is often locked.' \z \z

Regarding these, Sidnell points out that in the case of
(\ref{ex:2:22}a) and (\ref{ex:2:22}b), these instantiations of
\textit{get} in (\ref{ex:2:22}a--b) are Non-stative ``roughly equivalent
to ``make” and ``acquire””, respectively (p.~167).  In (\ref{ex:2:22}c)
however ``the verb is equivalent to the English possessive ``have” and
is stative” (p.~167). Based on this, Sidnell points out that:

\begin{quote}
[i]t is thus necessary to categorize many verbs according to their
uses rather than according to some abstract \isi{lexical specification}
(which is perhaps what Bickerton was trying to get at when he said
that stativity applied to sentences rather than the verbs themselves.
(p.~167)
\end{quote}

What I suggest here is that what Sidnell calls the abstract
representation of a verb and its actual use are related, since what we see
here are different meanings associated with two homophonous and
clearly related lexical items.  As I pointed out previously, in the
discussion of Jaganauth’s data, these are semantically distinct items
(although represented by identical forms).  A speaker of the language
would thus select one or the other based on the meaning that is
intended. In other words, in my treatment, Sidnell’s ``abstract lexical
specification” is not distinct from a verb’s use.  This is made clear
in my discussion of Event Structures and primitives of Change in
Chapters~\ref{ch:4} and~\ref{ch:5}.

Nevertheless, what we encounter in Sidnell’s work is a picture of
\isi{Aspect} in CECs that is more complex than many that we have seen so far
in the field.  This points to a need to further understand the
semantics associated with particular predicates in order to account
for the variation that we see in \isi{Aspect} expression.  This is
consistent with my own findings; however, the question is at what
level is this relevant?  I maintain that the \isi{Stative}\slash Non-stative
distinction is relevant at the syntactic level as it relates to
\isi{Telicity} and even \isi{grammatical aspect} marking.  However, it is
important to have an understanding of the \isi{lexico-semantic domain} in
order to account for semantic and syntactic differences that may arise
in the interpretation of particular items.  This is particularly
relevant in the case of lexically identical forms associated with
different meanings and forms expressing seemingly dual aspectual
behaviours, which I treat in this work.

\subsection{\citet{Gooden2008}}\label{sec:2.1.8}

Gooden’s focus is \isi{Tense} but it is worthy of mention here as it renews
a call for attention to be paid to \isi{Aspect} as a basic factor in
understanding \isi{Tense}.\footnote{Cf. \citet{Alleyne1980} for the suggestion 
that \isi{Aspect} was perhaps more basic than \isi{Tense} in the Creole
  Tense-\isi{Aspect} system.} Gooden argues that:

\begin{quote}
an analysis of both the aktionsarten of the verbs and discourse
factors are critical to developing an understanding of the range of
meanings and functions of both the relative past marker and the
\isi{unmarked verb} (p.~306)
\end{quote}

Here we notice that she speaks from the perspective of \isi{Tense} and
points to the need for an understanding of \isi{inherent aspect} in order to
deal with \isi{Tense} interpretation.  Regarding \isi{Aspect}, she points out that
while there have been several analyses of Tense-\isi{Aspect} in CECs ``with
some discussion of the influence of narrative structure […] discussed
comparatively less is the influence of \isi{lexical aspect} […] on Creole
verbal morphology, though there are numerous discussions on the
influence of stative and non-stative predicates.” (p.~307).  This
suggests a distinction between \textit{lexical aspect} and the
\isi{Stative}\slash Non-stative distinction indicating that in contrast to the
approach of this work, Gooden does not apply this distinction at the
\isi{lexical level}.

Regarding ``\isi{lexical aspect}'', however, she proposes that:

\begin{quote}
  a more fine-grained analysis of \isi{lexical aspect} is needed, since an
  analysis simply in terms of the stative\slash non-stative distinctions
  does not account for all the facts. (p.~307)
\end{quote}

This, as we see here is similar to Sidnell’s observations and my
perspective that further understanding of the \isi{lexico-semantic domain}
is necessary for a clearer understanding of \isi{Aspect}.

Gooden’s application of the \isi{Stative}\slash Non-stative distinction is unlike
that of \citet{Winford1993} and \citet{Sidnell2002} in that these
notions are not applied to the verb but rather to the entire predicate
or proposition.  In Gooden’s approach, \isi{Stativity} is treated as a
``feature of the \isi{lexical aspect} (aktionsart) of the verb” (p.~315).
However, although she refers to Aktionsart and \isi{lexical aspect}, these
terms do not apply at all to the verb for her.  According to her:

\begin{quote}
[s]ince aktionsart is a set of properties of predicates, i.e., verbs
together with their objects and adjuncts, not just bare verbs, we must
also examine the properties residing in the situation as a whole, not
relying on identification on the basis of lexical form
only. (p. 315--316)
\end{quote}

As indicated here, Gooden does not refer to \isi{inherent aspect} in the
sense of that which is contained in the verb but rather the entire
predicate including adjuncts.  This approach takes \isi{Aspect} to be
compositional. consistent with the approach articulated in this work.
However, her focus is on the whole rather than on parts comprising the
whole.  With this approach we are still left with the question of the
precise contribution of the various elements, including the verb
itself.

Gooden follows authors such as \citet{Dowty1979} in applying several
classes of situation types to events (\isi{Accomplishment}, \isi{Achievement},
\isi{Activity}, \isi{State} etc.) and applies the features Static, Durative \isi{Telic}
to these.  The result is that although the discussion is seemingly
about ``\isi{lexical aspect}” (suggesting the verb) this does not apply to
the verb at all in any direct way but rather the entire predicate.
The examples below are provided to illustrate her distinctions:

\ea%23
\label{ex:2:23}
\citet[316]{Gooden2008}\\
\ea Mary baked the turkey in five hours. (accomplishment) \ex Mary won
the race. (achievement) \ex Mary drove the car. (activity) \ex Mary
loves John. (state) \ex Mary coughed. (semelfactive) \z \z

As shown here, the distinctions employed are applied to the entire
predicate.  Thus we are still left with the question regarding the
unique contribution of the verb.  This issue is discussed in my first
chapter where I address issues associated with terminology and the
need for a model which, though compositional, takes into account the
unique contribution of the different but related elements in the
composition of \isi{Aspect}.

In contrast to her application of the \isi{Stative}\slash Non-stative distinction
to the entire predicate, Gooden’s tests for \isi{Stativity} appear to apply
to verbs themselves.  Speaking of the possible differences in lexical
aspect between (morphologically) similar English and \ili{Belizean Creole} (BC\il{Belizean Creole}) verbs she
points out that:

\begin{quote}
Although the bulk of BC\il{Belizean Creole} verbs are derived from English, there is no
reason to expect their lexical aspects to be identical and indeed they
are not.  (p.~316).
\end{quote}

As indicated here she seems to be talking about the verbs themselves
and the relevant tests are applied to the verbs in question.
Essentially in her test for \isi{Stativity}, a verb is deemed Non-stative if
it occurs under the following conditions:

\ea%24
\label{ex:2:24}
Gooden’s (p.~317) tests for Non-stativity \\
\ea in the progressive \ex as complements of non-volitional verbs
e.g.: \textit{force, persuade} \ex with certain adverbials, e.g.:
\textit{deliberately, carefully} (etc.); \ex permit verb-phrase
anaphoric forms e.g. \textit{do so} \citep{Mufwene1984} \z \z

Based on these tests she points out that verbs such as
\textit{biliiv} `believe', \textit{tink} `think', \textit{nuo}
`know' in BC\il{Belizean Creole} “are rendered as stative since they do
not occur in the specified contexts.” (p.  318).

Gooden raises some important issues in the discussion of \isi{Aspect} and
recognises the complexity of \isi{Aspect} outside of just grammatical
marking.  Her main contribution includes a call to pay more attention
to \isi{lexical aspect} and the fact that this may be more than the
\isi{Stative}\slash Non-stative distinction.  I join with her in the call for a
more in depth study of the \isi{lexical aspect} of the verb.  However, where
for Gooden \isi{lexical aspect} is not purely the verb but includes
information that may be contained in the predicate, I aim to identify
the specific semantic contribution of the verb to \isi{Aspect}.

\subsection{\citet{Youssef2003}}\label{sec:2.1.9}

This work successfully pinpoints one main issue that has mitigated
against the overall effectiveness of Tense-\isi{Aspect} studies in CECs.
The main issue it deals with is terminology.  This is significant as
while authors have pointed to this as an issue \citep{Winford1993,Winford2001},
we have not seen a work dedicated to addressing this for
consistencies in general works in the field and directly calling for
the use of ``a consensual use of labels.” (p.~81).  In this regard,
Youssef notes that:

\begin{quote}
It is necessary for writers in Creole linguistics to make specific
recourse to the work of language typologists when using the
terminology of the field; otherwise they run the risk of using terms
in narrow and particular ways which obscure the field rather than
clarify it. (p.~81)\end{quote}

Similarly \citet{Winford1993,Winford1997,Winford2001} observes that it
is perhaps due to this that we have not been able to significantly
impact the general field of linguistics.  In discussing terminological
issues, Youssef explores works such as that of \citet{Bickerton1975},
\citet{Holm1988}, \citet{Solomon1993} and \citet{Winford1993} where
she notes differences in the analysis of the null \isi{grammatical aspect}
marker and aspectual \textit{done} and the references to its function.
\tabref{tab:2.25} is a summary of her observations.

We see here varied analyses for both the \isi{null marker} and aspectual
\textit{done} by the different authors.  As previously indicated, for
\citet{Bickerton1975}, the \isi{null marker} is analysed from the
perspective of \isi{Tense} where it indicates either \isi{Past} or Non-past
depending on the \isi{Stativity} of the verb.  Youssef points out that for
\citet{Holm1988}, the \isi{null marker} indicates focus time (from a
discourse perspective).  \citet{Solomon1993} and \citet{Winford1993}
provide an analysis from the perspective of \isi{Aspect} but these also
differ.  Youssef separates \citet{Solomon1993} from the other authors
by pointing out that he ``makes the clearest case so far for \NULL as
marking perfectivity in Creoles regardless of inherent meaning.”
(p.~92).  According to him:

\begin{quote}
A ``state'' is nothing but the result of an ``action'', that is a
completed ``action''.  One is dead because one has died; one is married
because one has got married; one loves because the process of falling
in love is complete. We can group all the unmarked predicators as
having the meaning `completed' or perfective. (\citealt[96]{Solomon1993}
as cited in \citealt[92]{Youssef2003})
\end{quote}

\begin{table}[h]
  \caption{Interpretations associated with the null marker and
    \textit{done} in CECs. \citep[96]{Youssef2003}\label{tab:2.25}}
  \label{extab:2:25}
  \small
  \begin{tabularx}{\textwidth}{l QlQp{3cm}}
    \lsptoprule
    & Bickerton & Holm & Solomon & {\citealt{Winford1993,Winford2000}}\\
    \midrule \NULL & \mbox{\isi{Past}/non-past} & Focus time & \isi{Perfective} \newline
    \isi{Completive} \newline Completion \newline Relevance & \isi{Perfective}
    \newline \isi{Perfect} (1993) \newline
    Unmarked (2000)\\
    \tablevspace Done & \isi{Completive} \newline \isi{Resultative} \isi{Perfect} &
    \isi{Completive}/\isi{Perfect} & \isi{Perfective} \newline \isi{Completive} \newline
    Completion \newline Relevance & Completion \newline \isi{Resultative}
    (1993) \newline \isi{Completive} \newline
    \isi{Perfect} (2000)\\
    \lspbottomrule
  \end{tabularx}
\end{table}

Solomon seems to be in line with both \citet{Voorhoeve1957} and
\citet{Alleyne1980} who considered bare verbs representing \isi{Completive}
and \isi{Perfective} respectively with no reference to inherent \isi{Aspect}.
Apparently, for Solomon, the \isi{Perfective}\slash Non-perfective captures
completion in an event so ``\isi{Perfective} means that the event expressed
by the predicator is to be regarded as completed: non-perfective means
that it is not to be so regarded.” (\citealt[96]{Solomon1993} in \citealt[91]{Youssef2003}).  However,
based on Solomon’s approach it seems that \isi{Aspect} would be based almost
solely on \isi{grammatical aspect} and not much else.

\citet{Winford1993}, as indicated in \sectref{sec:2.1.5}, analyses the
\isi{null marker} as a marker of Perfectivity.  However, for him;
Perfectivity is not necessarily associated with completion as it is
made to cover Habituals and Generics as well.  Youssef comments on
this suggesting that there could ``in fact [be] two \textsc{nulls} in the system
described, one perfective and the other \isi{Imperfective}” (p.  93).  She
argued for these in the context of the Creole continuum and the
``neutrality'' established by unmarkedness.  This, she argues, could
allow for the form to have ``markedly different functions in different
parts of the system.”  (\citealt{Youssef1995} as cited in \citealt[93]{Youssef2003}.)
This brings into focus the difference between form and
meaning.  As we have seen before, the same form can have different
functions depending on its appearance and use in terms of context.  In
this sense, as \citet{Winford1993} indicates, a form has both a
primary (default) and a secondary meaning and function.

Youssef adjusts her view that perhaps there are two nulls in the
system by reflecting

\begin{quote}
[m]ore recently, however, I have recognized the potential for an
overall compatibility in the broad categorical labelling applied: It
ultimately becomes clear that virtually any kind of time span can be
viewed either Perfectively or Imperfectively. (p.~93).
\end{quote}

This later perspective captures what is the essence of grammatical
marking as it is meant to provide a viewpoint of a situation and does
not necessarily concern itself with the nature of the situation. Thus,
the situation may be completed or not, \isi{Habitual} or only once, and the
speaker can choose a viewpoint not restricted by the nature of the
situation.

As far as aspectual \textit{done} is concerned Youssef points to a
level of consensus: All authors seem to associate this in some way
with the meaning \isi{Completive}, while at least three of the four under
review associate this with the \isi{Perfect} (Bickerton, Holm and Winford).
Solomon, like \citet{Alleyne1980} sees \textit{done} as a reinforcer
of \isi{Perfective} meaning.  As we can see from this, ascribing a
particular meaning to a marker is not necessarily an easy matter, in
fact it appears to be quite difficult especially given the fact that
discourse does affect the meaning that may arise from the use of a
particular form.  In this regard, Youssef points out that

\begin{quote}
In ascribing tense to \NULL, we recognise that writers are generally
reacting to context and the inherent meaning of verbs rather than
merely grammaticalized meaning; (p.~96)
\end{quote}

She also points out that

\begin{quote}
  It is clear that \textit{use} is a particular issue in
  distinguishing among forms which otherwise appear to share the same
  \textit{meanings.} We need to specify the detail of contextual use
  of forms more precisely when we are specifying the functions of the
  markers we are describing. (p.~97)
\end{quote}

As indicated here, there are a number of elements that we need to take
into account in ascribing meaning to \isi{grammatical aspect} markers.  It
seems that for many authors, interpretation of a particular marker may
differ according to the type of verb with which it interacts and, as
Youssef points out here as well, the context.  Based on what we have
seen so far, however, and in particular in the work of
\citet{Sidnell2002}, the different interpretations that may be
associated with a form are predictable.  In other words, it may not be
necessary to associate a particular grammatical form with all the
meanings that may be indicated through its use.  A more fruitful
approach may be to indicate the base or default function or meaning of
such a form.  In these regards, Youssef points out that

\begin{quote}
It is essential then that in every analysis we specify the exact way
in which we are defining the use of these terms as well as considering
how they have been used by other writers we discuss. The more abstract
conceptualization of the categories perfective, imperfective and
perfect are ultimately most useful in providing a consistent overall
categorization schema (p.~102)
\end{quote}

This is in line with the approach which I have taken in this work.  As
explained in \chapref{ch:1}, terminology within \isi{Aspect} has to be defined
based on the different levels and elements involved in aspectuality.
From this perspective, meanings arising from the interaction between
different areas of \isi{Aspect} may not be associated with the unique
contribution of a particular element.  Thus, for example an
Im\isi{perfective aspect} marker may be interpreted as \isi{Habitual} or
\isi{Progressive} in particular contexts, but this is not necessarily the
meaning of the marker itself but rather an instantiation of different
types of \isi{Imperfective} meanings due to the interaction between factors
such as the form itself, other elements in the utterance and also
context.

Youssef’s main contribution to the discussion of Tense-\isi{Aspect} may be
seen as a call to pay attention to and address the issues associated
with terminology in field.  This marks a step towards consolidating
the different contributions that have been made so far in the
discussion and also pointing us in the direction where our studies can
be merged with more general studies in the field of general
linguistics theory.

In \sectref{sec:2.2}, I will summarise the major issues arising out of
the studies that have been discussed above.


\section{Observations}\label{sec:2.2}

One crucial issue that comes out of the discussion of works concerns
the notion of \isi{Stativity} and how this affects Tense-\isi{Aspect}
interpretation in Creole languages.  While earlier writers such as
\citet{Voorhoeve1957} and later \citet{Alleyne1980} may be said to
avoid any overt reference to this notion, later authors starting with
\citet{Bickerton1975} address this as an area of importance and
concern in Creole studies.  Issues associated with the concept of
\isi{Stativity} stem from its application as evidenced in the works of not
only \citet{Bickerton1975} but also \citet{Jaganauth1987},
\citet{Gooden2008} and \citet{Sidnell2002}.  In the case of
\citet{Bickerton1975} we see that although his discussion points to a
classification of verbs, he indicates that the distinction must apply
to propositions rather than to verbs in order to account for the case
of forms which are identical but display different uses consistent
with expressions of \isi{Stativity} and Non-stativity.

Faced with data signaling the same phenomenon, Jaganauth
uncompromisingly rejects the relevance of the \isi{Stative}\slash Non-stative
distinction.  \citet{Gooden2008} points to a treatment at the level of
the entire sentence but her tests for \isi{Stativity} target the verbs
themselves.  \citet{Sidnell2002} is very perceptive in his suggestion
that the groups of Statives and Non-statives are complex and defined
based on additional concepts which account for their interpretations.
However, his overall treatment does not allow him to make a connection
between \isi{lexical specification} and the semantic use of forms in dealing
with seemingly \isi{dual aspectual forms}.  In this regard, he, like
Bickerton points to a classification of forms according to their uses
rather than abstract \isi{lexical specification}.  Although authors such as
Winford accept the \isi{Stative}\slash Non-stative distinction as applied to the
verb, without an addressing items which appear in both \isi{Stative} and
Non-stative use as exemplified in Jaganauth’s examples above, the
concept of verbs classified as \isi{Stative} or Non-stative remains a mere
intuition rather than an applied and explained concept.

In the chapters which follow, I will first present the specific case
of property items and the problem presented by these as it relates to
the \isi{Stative}\slash Non-stative distinction (\chapref{ch:3}).  In \chapref{ch:4},
I will elaborate the \isi{Stative}\slash Non-stative distinction and the feature
Change and will present a categorisation for this general group of
items in \chapref{sec:5}.
