\addchap{Acknowledgements}



\begin{quote}

\begin{hebrew}
יא כסי, סי אכזי אימי סלעא דילא כסאד ונפלתא ונאשא לא גמאינך אבה – אאיא זון! דלאבד בד נפלא בקימה וגרנא, ובד מרמא רישא לאזם אי סלעא! (פשט ויהי בשלח)
\end{hebrew}

\textit{O my kinsman, go and see which merchandise is little in demand and whose value is gone down and no one looks at it – that you buy! For its value will certainly go up and increase; that merchandise will surely rise in value!} (Pəšaṭ Wayəhî Bəšallaḥ 5:16, edition and translation by \cite{SabarNerwa})

\end{quote}



Many people inspired, aided, and supported me in the long road leading to the writing of this dissertation, and the few paragraphs ahead are dedicated to them. %Among the most prominent are my academic supervisors:

First and foremost, I would like to express my immense gratitude to my doctoral supervisor, \textit{Doktormutter} in the true sense of the word, \name{Eleanor}{Coghill}. I believe that few doctoral students have the luck I had of enjoying such a wonderful supervisor. Eleanor turned out to be much more than an academic supervisor, but also a true friend: On numerous occasions she fed me, provided me with drink and shelter, and lent her ear upon necessity. All this, it goes without saying, while keeping the highest academic standards, and giving me thorough and frequent guidance and critique. It is fair to say that most of the good ideas in this dissertation are hers, while most of the not-so-good ones were included despite her better advice. 

\name{Eran}{Cohen} served as an external supervisor throughout my work on the thesis. Much of the methodology of this research, as well as many of the ideas, are based on his work, and at the same time, he served as an excellent opponent for refining my own ideas. Yet his influence on my research goes much further back, as in fact he is responsible for attracting me to the field of Neo-Aramaic: While I was just a freshman of the linguistics department at the Hebrew University of Jerusalem, he advised me to take the course on Neo-Aramaic, which incidentally he was teaching. Little did we know the dire consequences of this...\footnote{In my third year of studies, moreover, I took his course about the \il{NENA!Nerwa texts}Jewish Nerwa texts, where I got acquainted with the above citation. If I understood it as a recommendation to further my studies in Aramaic, I bear sole responsibility for this...}

Upon my arrival at the University of Konstanz, \name{Frans}{Plank} kindly agreed to serve as my second doctoral supervisor.\footnote{A small anecdote is here apposite: When I was about to finish my studies at the Hebrew University, I asked the late professor \name{Gideon}{Goldenberg}, who introduced me to typology with the course \enquote{\ili{Semitic} Languages: Historical and Typological Perspectives}, to give me some advice as to where I could continue to study typology abroad. He recommended me to approach \name{Frans}{Plank}, the Editor-in-Chief of the Association of Linguistic Typology. Yet my route went elsewhere and I forgot all about it. Thus, it was a curious coincidence that I recalled this recommendation only after Frans agreed to act as my doctoral supervisor.} As such, I had the pleasure of attending some of his seminars and benefited from his sharp questions and critique. Indeed, the typological underpinnings of this research are, in a large part, based on his work. Moreover, the decision to concentrate on the topic of adnominal modification and \isi{construct state} in \ili{NENA} crystallised during the DEM GENITIV workshop, held in Konstanz in October 2012, which was organised by Frans.\footnote{See \url{http://typo.uni-konstanz.de/ocs/index.php/dem-genitiv/}.}

A fourth person deserving special thanks is \name{Pollet}{Samvelian}. Before moving to Konstanz, I started the project as a doctoral student at the Sorbonne Nouvelle University (Paris III) under the supervision of Pollet Samvelian. Indeed, Pollet encouraged me to take on this research project and, as is evident from the results, many of the research questions were inspired by her guidance.

On the inspirational level, moreover, my interest in contact linguistics and  realization of the vast research possibilities related to Neo-Aramaic in this domain were spurred after attending \ia{Stilo, Donald@Stilo, Donald}Don Stilo's course \enquote{Typological Features and the Areal Dimension
in the Languages of the Southern Caucasus, Northern Iran-Iraq and Eastern Turkey} given at the Leipzig Spring School on Linguistic Diversity (March-April 2008).\footnote{See \url{http://www.eva.mpg.de/lingua/conference/08_springschool/}.} This was a true eye-opener and Don's meticulous level of research served as a constant (unreachable) horizon for me. 

The choice, however, of continuing my graduate research in the domain of Neo-Aramaic was not always clear. In this respect, a person whom I don't know personally played a crucial role. \ia{Sabar, Ariel}Ariel Sabar's book \enquote{My Father's Paradise: A Son's Search for His Jewish Past in Kurdish Iraq} (Algonquin Books, 2008), in which he tells the fascinating story of his father, Prof.\ \name{Yona}{Sabar}, a native speaker and renowned researcher of Neo-Aramaic, convinced me to pursue research in the domain of Neo-Aramaic, not only because of my fascination for the language, but also for its speakers. In this respect I am grateful as well to my friend \name{Ashley}{Kagan} (née Burdick) who sent me a copy of the book from LA.

As the dissertation is in fact a culmination of my entire linguistics curriculum, I would like to thank all my linguistic teachers (in the broadest sense) during the last 12 years (and some even before). Besides my aforementioned supervisors, these include
\name{Alain}{Desreumaux},
\name{Alain}{Lemaréchal},
\name{Anbessa}{Teferra}, 
\name{Anne}{Christophe},
\name{Ari}{Rappoport},
\name{Benoît}{Crabbé},
\name{Bruno}{Poizat},
\name{Dana}{Taube},
\name{David}{Gil},
\name{Dominique}{Sportiche},
\name{Eitan}{Grossman},
\name{Florence}{Villoing},
\name{Gideon}{Goldenberg},
\name{Gilles}{Authier},
Jeroen van de Weijer\ia{Weijer, Jeroen van de},
\name{Kim}{Gerdes},
\name{Maarten}{Mous},
\name{Marian}{Klamer},
\name{Martine}{Mazaudon},
\name{Mori}{Rimon},
\name{Sérgio}{Meira},
\name{Sharon}{Peperkamp},
\name{Simon}{Hopkins},
\name{Uzzi}{Ornan},
\name{Vincent}{Homer},
Wido van Peursen\ia{van Peursen, Willem Th.@\MakeCapital {van} Peursen, Willem Th.},
\name{Willem}{Adelaar},
and \name{Yishai}{Peled}, among others. 


I am thankful to \name{Geoffrey}{Khan}, Lidia Napiórkowska\ia{Napiorkowska, Lidia@Napiorkowska, Lidia} and \name{Jasmin}{Sinha} for answering my questions regarding their material. I am grateful as well to \name{Miriam}{Butt}, who during my doctoral colloquium provided me with some comments and criticism which permitted me to sharpen my discussion of clitics.  

Thanks are due to my friend \name{Ivri}{Bunis} for reading an earlier draft of the manuscript and giving some useful comments and references. Thanks go also to \name{Adam}{Pospíšil}, who told me about the term \enquote{\isi{pertensive}}, which I would otherwise have overlooked. 
Special thanks go to my colleague \name{Doris}{Penka} who kindly translated the \ili{English} abstract into \ili{German}.\footnote{[The English and German abstracts can be found in the original thesis manuscript \citep{GutmanThesis}.]}

During the research project (and before it) I conducted many fieldwork sessions with speakers of Neo-Aramaic. In this respect I am grateful to \name{Hezy}{Mutzafi}, who initiated me into \ili{NENA} fieldwork, mentored me in this domain, put me in contact with speakers and was always happy to answer my questions.

While only a small part of the fieldwork data found its way into the dissertation (mainly due to time constraints related to the transcription) I would like to thank my many NENA consultants for their time, help and hospitality. These include the \iai{Audisho family} in Sarcelles, \name{Elie}{Avrahami}, \name{Hadassa}{Yeshurun} and her father Rabbi \name{Ḥaim}{Yeshurun}, \name{Isa}{Hamdo} and his family, 
\name{Ḥabib \& Sara}{Nourani}, \name{Kara}{Hermez} and her father Pinkhas\ia{Hermez, Pinkhas}, \name{Lea \& Nissim}{Sharoni}, \name{Nissan}{Bishana} and his family, \name{Oz}{Aloni} and his mother Batia\ia{Aloni, Batia}, \name{Ya'aqov}{Mordechai}, the \iai{Yaramis family} in Sarcelles, \name{Yoel}{Sabari}, \name{Ziad}{Mooshi}, and \name{Zvi}{Avraham}. % complete this list

Special thanks are due to \name{Joseph}{Alichoran}: first for teaching me (together with \name{Bruno}{Poizat}) conversational Neo-Aramaic at the INALCO, and secondly for helping me on numerous occasions with transcription of NENA data and answering various language questions. Thanks are also due to his lovely family for their hospitality.

At the institutional level, I would like to thank the \textit{Zukunftskolleg} at the University of Konstanz for hosting me professionally during the research. As a Ph.D. student I was spoiled with a spacious office and excellent working conditions, as well as enjoying the various interdisciplinary workshops and lectures. In this respect, I would like to thank especially the head of the Zukunftskolleg, \name{Giovanni}{Galizia}, as well as \name{Martina}{Böttcher} and \name{Anda}{Lohan} from the administration for their kindness and helpfulness. Thanks go also to my postdoctoral office-mates, \name{Yaron}{McNabb} and \name{Sven}{Lauer}, for enduring my presence and providing me with advice and coffee pads on occasion.

Adjacent to the Zukunftskolleg sits the Research Centre for Aramaean Studies. I enjoyed passing many afternoons there drinking coffee and chatting about Aramaeans and Aramaic. Special thanks are due to \name{Zeki}{Bilgic} and \name{Ralph}{Barczok} for their help with Modern Aramaic and Syriac. 

At the University of Konstanz, I am especially thankful for two people for facilitating my move to Konstanz and dealing with the local bureaucracy: These are \name{Johannes}{Dingler} from the Welcome Center, as well as the wonderful \name{Elisabeth}{Grübel} from the HR department, who became a true friend.   

The first year of my doctoral project, 2011--12, I spent at the Sorbonne Nouvelle University (Paris 3) and it was financed by a doctoral position granted by the French government through the \textit{École Normale Supèrieure}. At that time I also benefited from the \textit{entourage} of the research project \enquote{Langues, dialects et isoglosses dans l'aire Ouest-Asie}, led by \name{Pollet}{Samvelian} and \name{Anaïd}{Donabedian}, which is part of the LabEx Empirical Foundations of Language. I am grateful for this initial funding and the company.

For the subsequent years, 2012--16, my position at the University of Konstanz and research needs were financed by the German Research Foundation project \enquote{Neo-Aramaic morphosyntax in its areal-linguistic context} led by \name{Eleanor}{Coghill}. Here again, I am thankful for this invaluable support.

I am grateful to \name{Riki}{Manetsch} for her warm hospitality in Zurich  during the very final (and intensive!) phase of corrections of the manuscript. 

To my parents, who inspired me to pursue the academic path, Hélène and Per-Olof\ia{Gutman, Hélène \& Per-Olof}, I offer many thanks and love. 

Last but certainly not least, I would like to thank my beloved wife Solange\ia{Pawou Molu, Solange} for the passionate discussions about linguistics we held together and for the enduring moral support she gave me during the work on this thesis.

\begin{flushright}
\begin{tabular}{l}
\textit{Ariel Gutman} \\
\textit{Konstanz, September 2016} \\
\end{tabular}
\end{flushright}





